\documentclass[aps,pre,superscriptaddress,groupedaddress]{revtex4-2}
\usepackage{amsmath}
\usepackage{amssymb}
\usepackage{amsfonts}
\usepackage{graphicx}
\usepackage{dcolumn}
\usepackage{bm}
\usepackage{sidecap}
\usepackage{subfloat}
\usepackage{parskip}
\usepackage{grffile}
\usepackage{color}
\usepackage{hyperref}
\usepackage{hhline}
\usepackage{mathtools}
\usepackage{graphics}
\usepackage{multirow}
\usepackage{verbatim}
\usepackage{longtable}
\usepackage{rotating}
\usepackage{setspace}
\usepackage{epsfig}
\usepackage{subfigure}
\usepackage{epstopdf}
\usepackage{gensymb}
\usepackage[normalem]{ulem}


\setlength{\textfloatsep}{5pt}
\renewcommand{\figurename}{FIG.}
\renewcommand{\thefigure}{S-\arabic{figure}}
\renewcommand{\theequation}{S-\arabic{equation}}




\begin{document}


%------------------------------
\title{Supplemental Material: Prolonged hysteresis in Kuramoto oscillators with inertia having higher-order interactions}


\author{Narayan G Sabhahit$^{1}$, Akanksha S. Khurd$^{2}$, Sarika Jalan$^{3}$}\email{sarika@iiti.ac.in}


\affiliation{1. Department of Physical Sciences, Indian Institute of Science Education and Research, Kolkata}
\affiliation{2. Department of Physics, Indian Institute of Science Education and Research, Tirupati}
\affiliation{3. Complex Systems Lab, Department of Physics, Indian Institute of Technology Indore, Khandwa Road, Simrol, Indore-453552, India}


\maketitle


\subsection{Derivation for $m = 0$}
This section elaborates on the self-consistency analysis and the derivation for the forward and backward transition points for $m = 0$. In this case, a change from a smooth (second-order) transition to synchronization to an abrupt (first-order) one, along with hysteresis, is observed with an increase in the $K_2$ value. The occurrence of hysteresis can be accounted for by the shift of the backward transition point to lower $K_1$ values with an increase in $K_2$. The results for the first-order Kuramoto model with higher-order interactions were reported in \cite{skardal2020higher} using the Ott-Antonsen dimensionality reduction method. Here we provide the analytical predictions for the steady state behavior using self-consistency analysis. 

The dynamical equation for our model takes the following form.
\begin{equation}\label{mean field equation}
        \dot{\theta}_{i} =  \omega_i + \frac{K_1}{N}\sum_{j=1}^{N}\sin(\theta_j-\theta_{i}) + \frac{K_2}{N^2}\sum_{j=1}^{N} \sum_{k=1}^{N} \sin(2\theta_j-\theta_k -\theta_{i})
\end{equation}
Following the same procedure as in the manuscript, the mean-field equation in the rotating frame is obtained as $\dot{\theta_i} = \omega_i - q\sin(\theta_i)$, where $q = r_{1}(K_{1} +K_{2}r_{2})$. All the oscillators with intrinsic frequency $|\omega_i|\leq q$ go to the fixed point state, while others are in a drift state. Unlike in the finite-inertia case, here the limit of the integrals for forward and backward cases become the same ($\theta_{f} = \theta_{b} = \frac{\pi}{2}$), enabling us to study the system in full generality. In the thermodynamic limit, the drifting oscillators form a stationary distribution on the unit circle, thereby making a negligible contribution to the overall coherence. Hence, taking into account only the locked oscillator contribution, the self-consistent equations come out to be of the following form,
\begin{equation}\label{locked term}
    \begin{split}
    r_{p}^{l} = q\int_{-\theta_{f,b}}^{\theta_{f,b}}\cos(\theta)\cos(p\theta)g(q\sin(\theta))d\theta
    \end{split}
\end{equation}
On integrating Eq.~\ref{locked term} for $p = \{1,2\}$ and rearranging the terms, we end up with the following system of self-consistency conditions,
\begin{subequations}\label{total self consistent}
     \begin{equation} \label{massless r1 self consistent}
     \begin{split}
     r_{2} = \frac{2 - K_{1}(1 - r_{1}^{2})}{K_{2}(1 - r_{1}^{2})}
     \end{split}
     \end{equation}    \begin{equation}\label{massless r2 self consistent}
     \begin{split}
      r_{2} = \frac{2}{\pi q^{2}}\left[(q^{2} + 2)\tan^{-1}(q) -2q\right]
     \end{split}
     \end{equation}
\end{subequations}
To obtain the forward transition point ($K_{1}^{f}$), we put $r_{1} = r_{2} = 0$ in Eq.~\ref{massless r1 self consistent} which yields $K_{1}^{f} = 2$, which is the same as in the Kuramoto model. Therefore, we infer that the forward transition point is independent of the strength of higher-order interactions. Using the expression for $r_{2}$ from Eq.~\ref{massless r1 self consistent}, we have $q = r_{1}(K_{1}+ r_{2}K_{2}) = r_{1}\left(K_{1} + \frac{2 - K_{1}(1-r_{1}^{2})}{1 - r_{1}^{2}}\right) = \frac{2r_{1}}{1 - r_{1}^{2}}$. Substituting this expression for $q$, $r_{2}$ from Eq.~\ref{massless r1 self consistent} into Eq.~\ref{massless r2 self consistent} and rearranging the terms we have,
\\
\\
\begin{equation}\label{final version of the self consistency}
    K_{1} = \frac{2}{1-r_{1}^{2}} - \frac{K_{2}(1 - r_{1}^{2})^{2}}{\pi r_{1}^{2}}\left[\left(\frac{2r_{1}^{2}}{(1 - r_{1}^{2})^{2}}\right)\tan^{-1}\left(\frac{2r_{1}}{1 - r_{1}^{2}}\right) - \frac{2r_{1}}{(1-r_{1}^{2})}\right]
\end{equation}
\\
\\
\begin{figure}[t]
    \centering
    \includegraphics[scale=0.7]{SMm0.eps}
    \caption{Synchronization profiles for Eq.~\ref{Gen_KM_quar}, for and $K_2 = 0$ (red), $3 $ (green), and $6$ (blue). The filled and open circles represent simulation results for the forward and backward cases, while the scatter plots represent the analytically predicted values.}
    \label{fig: SM-m0}
\end{figure}

From Fig.~\ref{fig: SM-m0},  we notice that the backward de-synchronization point is fairly well approximated by the minima ($\frac{dK_{1}}{dr_{1}} = 0$) of the self-consistency curve. To that end, from Eq.~\ref{final version of the self consistency}, we obtain the value of $\frac{dK_{1}}{dr_{1}}$ as,
\begin{equation}\label{derivative of the self consistency}
    \frac{dK_{1}}{dr_{1}} = \frac{-2K_{2}(r_{1}^{2} - 1)^{3}(r_{1}^{2}+1)^{2}\tan^{-1}(\frac{2r_{1}}{1 - r_{1}^{2}}) + 4K_{2}r_{1}(-r_{1}^{8} + r_{1}^{6} + r_{1}^{2}-1) + 4\pi r_{1}^{4}(r_{1}^{2} + 1)}{\pi r_{1}^{3}(r^{2} - 1)^{2}(r^{2} + 1)} 
\end{equation}
As can be seen from Eq.~\ref{derivative of the self consistency}, finding a clean analytical expression between $r_{1}^{b}$ and $K_{2}$ is not possible. Hence, for a fixed value of $K_2$, we numerically seek the solution to the equation,
\begin{equation}
    f(K_{2},r_{1}) = -2K_{2}(r_{1}^{2} - 1)^{3}(r_{1}^{2}+1)^{2}\tan^{-1}(\frac{2r_{1}}{1 - r_{1}^{2}}) + 4K_{2}r_{1}(-r_{1}^{8} + r_{1}^{6} + r_{1}^{2}-1) + 4\pi r_{1}^{4}(r_{1}^{2} + 1) = 0. 
\end{equation}
Upon numerically obtaining the value of $r_{1}^{b}$ (for a particular $K_{2}$), we substitute it back in Eq.~\ref{final version of the self consistency} to obtain the value of backward de-synchronization point ($K_{1}^{b}$). This is used to plot the $K_{1}^{b}$ vs. $K_{2}$ in the manuscript. Fig.~\ref{fig: SM-m0} presents analytical and simulation results for multiple values of triadic coupling strength $K_{2}$. As predicted by our analysis, the forward synchronization transition happens at $K_1^f = 2$, while the backward transition point shifts to lower values of $K_1$ as $K_2$ increases. 


\subsection{Extension to quartic interactions}
This section explains how the self-consistency analysis presented in the manuscript can be easily extended to include quartic interactions.  We add a quartic interaction term to our model as proposed in \cite{skardal2020higher}. In our study, we find a good agreement between the analytical predictions of our self-consistency analysis with the simulation results. The detailed analysis of the impact of quartic interactions on the dynamics is out of the scope of this discussion. The dynamical equation for our model is given as follows:
\begin{equation}\label{Gen_KM_quar}
    \begin{split}
    m\ddot{\theta_i} =  -\dot{\theta_i} + \omega_i + \frac{K_{1}}{N}\sum_{j = 1}^{N}\sin(\theta_{j} - \theta_{i}) +\frac{K_{2}}{N^{2}} \sum_{j = 1}^{N}\sum_{k = 1}^{N}\sin(2\theta_{j} - \theta_{k}- \theta_{i}) + \frac{K_{3}}{N^{3}}\sum_{j = 1}^{N}\sum_{k = 1}^{N}\sum_{l = 1}^{N}\sin(\theta_{j} + \theta_{k}-\theta_{l} - \theta_{i})
    \end{split}
\end{equation}
Upon using the definition of the generalized order parameter as defined in the manuscript, we can convert the above equation into a mean-field form  as $ m\ddot{\theta}_{i} =  -\dot{\theta}_{i} + \omega_i + K_{1}r_{1}\sin(\psi_{1} - \theta_{i}) +K_{2}r_{1}r_{2}\sin(\psi_{2} - \psi_{1}- \theta_{i}) + K_{3}r_{1}^{3}\sin(\psi_{1} - \theta_{i})$. By moving to the rotating frame, we set $\psi_1$ and $\psi_2$ as zero, which gives $m\ddot{\theta}_{i} =  -\dot{\theta}_{i} + \omega_i - q\sin(\theta_{i})$, where $q = r_{1}(K_{1} +K_{2}r_{2} + K_{3}r_{1}^{2})$ is the overall coupling constant. We can proceed in an exact manner, as in the Letter, to arrive at the self-consistency equations defining the steady-state behavior of the model,
\begin{subequations}\label{total self consistent}
     \begin{equation} \label{r1 self consistent final}
     \begin{split}
     r_{1} = 2q\int_{0}^{\theta_{f,b}}\cos^{2}(\theta)g(q\sin(\theta))d\theta + 2\int_{\omega_{f,b}}^{\infty}    \omega\sqrt{\frac{m}{q}}\left[\sqrt{\omega^{2}\frac{m}{q} - \frac{qm}{1+\omega^{2}m^{2}}} - \omega\sqrt{ \frac{m}{q}}\right]g(\omega)d\omega
     \end{split}
     \end{equation}
     \begin{equation}\label{r2 self consistent final}
         \begin{split}
              r_{2} &= 2q\int_{0}^{\theta_{f,b}}\cos(\theta)\cos(2\theta)g(q\sin(\theta))d\theta + \\ 
              & 2\int_{\omega_{f,b}}^{\infty}    \left[\frac{\omega^{2} m^{2}-1}{\omega^{2} m^{2}+1}\right]\left[\frac{2\omega}{q}\left(\frac{\omega^{2}m^{1}+1}{\sqrt{qm}}\right)\left(\sqrt{\omega^{2}\frac{m}{q} - \frac{qm}{1+\omega^{2}m^{2}}} - \omega\sqrt{ \frac{m}{q}}\right) - 1\right]g(\omega)d\omega
         \end{split}
     \end{equation}
\end{subequations}

\begin{figure}[t]
    \centering
    \includegraphics[scale=0.7]{SMquar.eps}
    \caption{Synchronization profiles for the Kuramoto model with inertia involving quartic interactions with $m=1$ and different combinations of $K_2$ and $K_3$ values. The red circles and blue squares represent the simulation results for the forward and backward cases of Eq.~\ref{Gen_KM_quar} for $N=1000$ nodes, respectively. Whereas, the dashed and continuous curves are the analytical predictions from \ref{total self consistent} for the forward and backward cases. }
    \label{fig: quartic}
\end{figure}

\subsection{Derivation for the $\sin(\theta_j + \theta_k - 2\theta_i)$ model}
In this section, we present the self-consistency analysis for the Kuramoto model with inertia involving purely triadic interactions via the $\sin(\theta_j + \theta_k - 2\theta_i)$ coupling function. Research on the dynamics of Kuramoto oscillators coupled via the said sinusoidal function ($\dot\theta_i = \omega_i + \frac{K_2}{N^2} \sum_{j=1}^{N} \sum_{k=1}^{N} \sin(\theta_j + \theta_k - 2\theta_i )$) \cite{skardal2019abrupt} has shown the presence of cluster formation and a continuum of abrupt de-synchronization transitions based on initial conditions. Crucially, no synchronization transition has been reported for this model. Surprisingly, our studies report that inertia has no effect on the synchronization profile of this system. We infer that considering purely triadic interactions removes the distinction between inertia-less and finite-inertia cases. 

The dynamical equations for this model are given as:
\begin{equation}\label{MM-HO-M1OG}
    m\ddot{\theta_i} = \omega_i - \dot{\theta_i} + \frac{K_2}{N^2} \sum_{j=1}^{N} \sum_{k=1}^{N} \sin(\theta_j + \theta_k - 2\theta_i)
\end{equation}
Using the definition of the generalized order parameter, we write Eq.~\ref{MM-HO-M1OG} in the mean-field format as,
\begin{equation}\label{MM-Ho-M1MF}
m\ddot{\theta_i} = \omega_i - \dot{\theta_i} + K_2r_1^2\sin(2\psi_1 - 2\theta_i)
\end{equation}
Upon simulating the dynamics of this equation, we observe that analogous to the studies reported in \cite{skardal2019abrupt}, there is no forward synchronization in the system, but a sequence of de-synchronization transitions is observed, based on the asymmetry in the initial conditions. Thus, we will be focusing our analytical derivation only on the de-synchronization profiles.

Dropping the subscript $i$, by going to a suitable rotating frame, we set $\psi = 0$, which gives us $m\ddot{\theta} = \omega - \dot{\theta} - K_2r_1^2\sin(2\theta)$, where the probability distribution g($\omega$) is unimodal and symmetric about the mean $0$. To study the steady-state behavior, we invoke a time transformation as $\tau = \sqrt{\frac{K_2r_1^2}{m}} t$, which gives:
\begin{equation}\label{MM-HO-M1PS}
    \ddot\theta = \beta - \alpha\dot\theta - \sin(2\theta)
\end{equation}
where $\beta = \frac{\omega}{K_2r_1^2}$ and $\alpha = \frac{1}{\sqrt{K_2r_1^2m}}$. The parameter space of Eq.~\ref{MM-HO-M1PS} is qualitatively similar to that of the model considered in the manuscript. The quantitative differences between the two are as follows: (i) For each $\beta$ such that $|\beta| \le 1$, we now have two stable fixed points given by $\theta_1^* = \frac{1}{2}\sin^{-1}{\beta}, \,\,\, \text{and} \,\,\, \theta_2^* = \frac{1}{2}\sin^{-1}{\beta} + \pi$ separated by two saddles. (ii) The separatrix equation \cite{tabor1989chaos} is given by $\theta(t) =  \sin^{-1}{\tanh(\sqrt{2}t)}, \,\,\, \dot{\theta}(t) = \sqrt{2} \frac{1}{\cosh(\sqrt{2}t)}$. Using this equation of separatrix, and implementing Melnikov's method \cite{guckenheimer2013nonlinear}, the equation for homoclinic bifurcation can be obtained as $\beta = \frac{2\sqrt{2}}{\pi}\alpha$. Thus, as seen in Fig.~\ref{fig: SM-paramspace}

\begin{figure}[h]
\centering
\includegraphics[scale=0.08]{SMparamspacefinal.eps}
\caption{The parameter space for Eq.~\ref{MM-HO-M1PS}. The violet, yellow, and red regions represent the limit-cycle, tristable, and fixed-point regimes respectively. The line at $\beta = 1$ represents the saddle-node/infinite-period bifurcation in the system. On the boundary of the tri-stable and fixed-point regimes, the scatter plot indicates the actual homoclinic bifurcation curve while the continuous line is the approximation obtained using Melnikov's method ($\beta = \frac{2\sqrt{2}}{\pi}\alpha$).}
\label{fig: SM-paramspace}
\end{figure}

 \begin{enumerate}
    \item if $|\beta| > 1$, the system goes to a limit cycle; 
    \item if $|\beta| \le \frac{2\sqrt{2}}{\pi}\alpha$, the system goes to a stable fixed point state, where the choice of the fixed point depends upon the initial conditions. \item Finally, if $\frac{2\sqrt{2}}{\pi}\alpha<|\beta|<1$, we now have a tri-stable state with the simultaneous presence of one stable limit cycle and two stable fixed points. 
\end{enumerate}

In terms of $\omega$, $K_2$, and $r_1$, it means that for a particular $K_2$ and corresponding steady-state value of $r_1$, the oscillators with $|\omega| > K_2r_1^2$ become drifting oscillators. The oscillators with $|\omega| \le \frac{2\sqrt{2}}{\pi} \sqrt{\frac{K_2r_1^2}{m}}$ contribute to the formation of two diametrically opposite clusters of locked oscillators. Finally the oscillators with $\frac{2\sqrt{2}}{\pi} \sqrt{\frac{K_2r_1^2}{m}} < |\omega| \le 1$ are in the tri-stable stable region. However, as we will deal only with the systems' de-synchronization profile, the oscillators in the tri-stable region will also go to their respective fixed points and contribute to cluster formation.

The next step is calculating the locked and drifting oscillator contribution to the order parameter. To account for the two clusters of locked oscillators, we introduce a variable $\eta(\omega) \in [0,1]$. The values of $\eta(\omega)$ and $1-\eta(\omega)$ are the probabilities that an oscillator with intrinsic frequency $\omega$ will go to the fixed points $\theta_1^* = \frac{1}{2}\sin^{-1}{\beta}$ and $\theta_2^* = \frac{1}{2}\sin^{-1}{\beta} + \pi$, respectively. For simplicity, we consider only symmetric cases of the function $\eta(\omega)$, such that $\eta(\omega) = \eta(-\omega)$. The contribution of locked oscillators is then given by $r_1^l = \int_{-K_2r_1^2}^{K_2r_1^2} [(1-\eta(\omega)) e^{i\theta(\omega) + \pi} + \eta(\omega)e^{i\theta(\omega)}] g(\omega)d\omega$. As $e^{i\theta(\omega) + \pi} = - e^{i\theta(\omega)}$ and $\sin(-x) = -\sin(x)$, we have:
\begin{equation}\label{MM-HO-M1rl}
    r_1^l = \int_{-K_2r_1^2}^{K_2r_1^2} (2\eta(\omega) - 1) \cos \theta(\omega) g(\omega) d\omega
\end{equation}

\begin{figure}[h]
\centering
\includegraphics[width=\columnwidth]{SM2theta.eps}
\caption{ The top panel displays the $r_1$ versus $K_2$ behaviour for Eq.~\ref{MM-Ho-M1MF} for a) $m = 0$, b) $m = 1$, and c) $m = 3$  respectively. The violet, yellow and green curves represent the de-synchronization profiles for $\eta$ equal to 1, 0.9, and 0.8, respectively. For each case, the continuous curve represents the analytically obtained values. The bottom panel displays the simulation-obtained values for the Daido order parameter ($r_2$) for the corresponding cases (d) $m=0$, e) $m=1$, and f) $m=3$).}
\label{fig: Model1}
\end{figure}

Let $\rho_d(\theta,\omega)$ be the density of drifting oscillators which satisfies $\int_{-\pi}^{\pi} \rho_d(\theta,\omega) d\theta = 1$. The continuity equation for the conservation of the number of oscillators gives $\rho_d(\theta,\omega) = c/\dot\theta(\theta,\omega)$. An expression for $\dot\theta$ can be obtained by following an analogous method as in \cite{gao2018self} by considering $\dot\theta = A_0 + A_1 \cos(2\theta) + B_1 \sin(2\theta)$. By substituting this in Eq.~\ref{MM-HO-M1PS}, the expression for $\dot\theta$ can be obtained as:
\begin{equation}\label{MM-Ho-M1v}
\dot\theta(\omega,\theta) = \frac{\beta}{\alpha} + \frac{2 \beta \alpha}{\alpha^4+4\beta^2} \cos(2\theta) - \frac{\beta^3}{\alpha^4 + 4\beta^2}\sin(2\theta).
\end{equation}
Eq.~\ref{MM-Ho-M1v} can be simplified by considering $h_0 = \beta/ \alpha$ and $\frac{e^{ih_2}}{h_1} = 2h_0 + i\alpha$, which gives $\dot\theta = h_0 + h_1\cos(2\theta + h_2)$. By integrating over the normalization condition of $\rho_d$, we get $c = \sqrt{h_0^2 - h_1^2}/2\pi$, which gives the following equation for the density of drifting oscillators:

\begin{equation}\label{MM-HO-M1rho}
    \rho_d(\theta,\omega) = \left| \frac{1}{2\pi} \sqrt{\frac{\omega^2m}{K_2r_1^2} - \frac{K_2r_1^2m}{1+4m^2\omega^2}} \times \frac{1}{\frac{\omega\sqrt{m}}{\sqrt{K_2r_1^2}} + \sqrt{\frac{K_2r_1^2m}{1+4m^2\omega^2}}\cos(2\theta(\omega) + \arctan(\frac{1}{2m\omega}))} \right | 
\end{equation}


From Eq.~\ref{MM-HO-M1rho}, we notice that $\rho_d(\theta+\pi,\omega) = \rho_d(\theta,\omega)$. Thus, the contribution of drifting oscillators to order parameter $r_1^d = \int_{|\omega|>K_2r_1^2} \int_{-\pi}^{\pi} e^{i\theta(\omega)}\rho_d(\theta,\omega) g(\omega) d\theta d\omega$ can be simplified as:
\begin{equation}
r_1^d = \int_{|\omega|>K_2r_1^2}\int_{-\pi}^{0} [ e^{i\theta(\omega)} \rho_d(\theta,\omega) + e^{i(\theta(\omega) + \pi)} \rho_d(\theta+\pi,\omega)] g(\omega) d\theta d\omega.
\end{equation}
As $e^{i(\theta(\omega) + \pi)} = -e^{i\theta(\omega)}$, we get $r_1^d = 0$. Therefore, from Eq.~\ref{MM-HO-M1rl}, we get
\begin{equation}\label{MM-m1-rf}
    r_1 = \int_{-K_2r_1^2}^{K_2r_1^2} (2\eta(\omega) - 1) \cos \theta(\omega) g(\omega) d\omega
\end{equation}

We now compare the results obtained by simulating Eq.~\ref{MM-Ho-M1MF} with the analytical values predicted by the Eq.~\ref{MM-m1-rf} in Fig.~\ref{fig: Model1}. For simplicity, we consider $\eta(\omega)$ to be a constant function with respect to $\omega$. While simulating for a given value of $\eta$, we consider the initial phase of the oscillators as $0$ with probability $\eta$ and $\pi$ with probability $(1-\eta)$. With these initial conditions, we start from $K_2=K_{max} = 16$, and then adiabatically decrease $K_2$ till $K_{min}=0$. At each $K_2$, we integrate Eq.~\ref{MM-Ho-M1MF} using the RK4 algorithm and calculate the value of $r_1$ and $r_2$ by averaging over all the iterations after removing the transients.

We note that due to the vanishing of the drift term, Eq.~\ref{MM-m1-rf} is identical to the one previously obtained in \cite{xu2021spectrum}, for the $m=0$ case. Fig.~\ref{fig: Model1} also shows that the $r_1$ and $r_2$ graphs are almost identical for the finite mass and the mass-less case, with identical backward transition points. Hence, considering purely triadic interactions eliminates the distinction between the finite inertia and inertia-less case. For $\eta = 0.9$ (Yellow) and $\eta = 0.8$ (Green), we see that the $r_2$ values are higher than the $r_1$ values indicating  that a two-cluster state is present in the system, which is the expected outcome. 

A more generalized approach to further this study would be to incorporate pair-wise interactions along with the triadic ones to give a model like:
\begin{equation}\label{MM-HO-M1ext}
    m\ddot{\theta_i} = \omega_i - \dot{\theta_i} + \frac{K_1}{N} \sum_{j=1}^{N} \sin(\theta_j - \theta_i) + \frac{K_2}{N^2} \sum_{j=1}^{N} \sum_{k=1}^{N} \sin(\theta_j + \theta_k - 2\theta_i)
\end{equation} 
which in the mean-field format and rotating frame, with $\psi_1$ set to zero, reads as $m\ddot{\theta_i} = \omega_i - \dot{\theta_i} - K_1 r_1 \sin(\theta_i) - K_2 r_1^2 \sin(2\theta_i)$. However, the simultaneous presence of the first and second-order harmonics of the sinusoidal term makes it complicated to analytically study the parameter space of Eq.~\ref{MM-HO-M1ext}.

\begin{thebibliography}{99}

    \bibitem{skardal2020higher} Skardal, P. S., \& Arenas, A. Higher order interactions in complex networks of phase oscillators promote abrupt synchronization switching. {\em Communications Physics}, {\bf 3(1)}, 218 (2020).

    \bibitem{skardal2019abrupt}Skardal, P. S., \& Arenas, A. Abrupt desynchronization and extensive multistability in globally coupled oscillator simplexes. {\em Physical Review Letters}, {\bf 122(24)}, 248301 (2019).

    \bibitem{tabor1989chaos}Tabor, M. Chaos and integrability in nonlinear dynamics: an introduction. Wiley-Interscience (1989).

    \bibitem{guckenheimer2013nonlinear} Guckenheimer, J., \& Holmes, P. Nonlinear oscillations, dynamical systems, and bifurcations of vector fields. (Springer Science \& Business Media, 2013) Vol. 42.

    \bibitem{gao2018self} Gao, J., \& Efstathiou, K. Self-consistent method and steady states of second-order oscillators. {\em Physical Review E}, {\bf 98} (4), 042201 (2018).

    \bibitem{xu2021spectrum} Xu, C., \& Skardal, P. S. Spectrum of extensive multiclusters in the Kuramoto model with higher-order interactions. {\em Physical Review Research}, {\bf 3(1)}, 013013 (2021).
    

\end{thebibliography}


%The difference between the analytical and numerical backward de-synchronization points can be attributed to finite-size effects ($N=1000$ in Fig.~\ref{fig: SM-m0}). A higher number of nodes ($N\ge10000$) and higher $K_1^{max}$ value can give a better alignment between the results.

\end{document}