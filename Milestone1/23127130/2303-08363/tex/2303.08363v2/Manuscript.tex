\documentclass[aps,pre,twocolumn,showpacs,superscriptaddress,groupedaddress]{revtex4-2}
\usepackage{amsmath}
\usepackage{amsfonts}
\usepackage{amssymb}
\usepackage{graphicx}
\usepackage{dcolumn}
\usepackage{sidecap}
\usepackage{parskip}
%\usepackage{grffile}
\usepackage{xcolor}
\usepackage{hyperref}
\usepackage{hhline}
\usepackage{mathtools}
\usepackage{multirow}
\usepackage{verbatim}
\usepackage{rotating}
\usepackage{setspace}
\usepackage{epsfig}
\usepackage{epstopdf}
%\usepackage{float}
%\usepackage{subfloat}
\usepackage{subfigure}
%\usepackage{booktabs}
\usepackage[normalem]{ulem}

\usepackage{bm}
%\usepackage{comment}
%\usepackage{cleveref}

%\usepackage[ backend=biber,
%    style=numeric, sorting = none
%]{biblatex}

%\addbibresource{reference.bib}

\newcommand{\n}{\noindent}
\newcommand{\A}{\textbf{A}}
\newcommand{\I}{\textbf{I}}
\newcommand{\M}{\cal{M}}
\newcommand{\mk}{\langle k \rangle}
\newcommand{\iu}{\mathrm{i}\mkern1mu}

\def \hfillx {\hspace*{-\textwidth} \hfill}

\makeatletter
\def\@eqnnum{{\normalsize \normalcolor (\theequation)}}
 \makeatother

\hyphenation{ALPGEN}
\hyphenation{EVTGEN}
\hyphenation{PYTHIA}

\newcommand\filledcirc{{\color{black}\bullet}\mathllap{\circ}}


\begin{document}
%------------------------------
\title{Prolonged hysteresis in Kuramoto oscillators with inertia having higher-order interactions}

\author{Narayan G Sabhahit$^{1}$}
\email{narayan.g.sabhahit@gmail.com}
\author{Akanksha S. Khurd$^{2}$}
\author{Sarika Jalan$^{3}$}
\email{sarika@iiti.ac.in}

\affiliation{1. Department of Physical Sciences, Indian Institute of Science Education and Research, Kolkata}
\affiliation{2. Department of Physics, Indian Institute of Science Education and Research, Tirupati}
\affiliation{3.Complex Systems Lab, Department of Physics, Indian Institute of Technology Indore, Khandwa Road, Simrol, Indore-453552, India}

\date{\today}

\begin{abstract}
%The inclusion of inertia in the celebrated Kuramoto model has been reported to change the nature of phase transition to a coherence state in networks of interacting dynamical units. Recently, higher-order interactions have been increasingly realized as crucial for the functioning of such network systems ranging from brain to disease spreading. 




%The impact 
The inclusion of inertia in the Kuramoto model has been long reported to change the nature of phase transition, providing a fertile ground to model the dynamical behaviors of interacting units. More recently, higher-order interactions have been realized as essential for the functioning of real-world complex systems ranging from brain to disease spreading. Yet, analytical insights to decipher the role of inertia  with higher-order interactions remain challenging. To that end, this Letter studies Kuramoto oscillators with inertia on simplicial complexes, merging two research domains. We develop an analytical framework in a mean-field setting using self-consistent equations to describe the steady-state behavior, which reveals a prolonged hysteresis in the synchronization profile. 
%Further, an interplay of inertia and higher-order interactions control the forward and backward transition points.
Inertia and triadic interaction strength exhibit isolated influence on system dynamics by predominantly governing, respectively, the forward and backward critical points. This work sets a paradigm to deepen our understanding of real-world complex systems such as power grids modeled as the coupled Kuramoto model with inertia.

%This work sets a paradigm to deepen our understanding of real-world complex systems such as power grids modeled as the coupled Kuramoto model with inertia and to design a more robust synchronized profile.

 
\end{abstract}

\maketitle
The emergence of collective behavior in complex real-world systems has been a long-standing research interest \cite{boccaletti2006complex}. It was initially in the landmark paper \cite{kuramoto1975self} that Kuramoto modeled the phenomenon of synchronization using a system of network-coupled oscillators in an analytically tractable setting, illustrating that the system underwent a second-order phase transition from incoherent to a coherent state. Since then, numerous works on various extensions of the Kuramoto Model have been done, revealing several phenomena \cite{childs2008stability,sethia2008clustered,omel2012nonuniversal,olmi2015chimera,rodrigues2016the}. Of particular interest to us is the Kuramoto Model with inertia (also known as the second-order Kuramoto model). Inspired by the modeling of synchronized flashing in \textit{Pteroptix malaccae} by \textit{Ermentrout} \cite{ermentrout1991adaptive}, a second-order extension of the Kuramoto model was first proposed by \textit{Tanaka et al.} \cite{tanaka1997first,tanaka1997self}. They showed that the system underwent a first-order phase transition upon the introduction of inertia rather than the smooth second-order phase transition observed in the Kuramoto model. They put forth a self-consistent method akin to the one proposed by Kuramoto to study the steady-state behavior of the coupled oscillator system. Since then, the second-order Kuramoto model has been extensively explored in diluted networks \cite{ji2013cluster} and various real-world complex systems like Josephson junctions \cite{trees2005sync} and power grids \cite{ji2014basin, grzybowski2016on, rohden2012self, dorfler2013synchronization, witt2022col}. In \cite{filatrella2008analysis}, \textit{Filatrella et al.} explained how the second-order Kuramoto model originates in power grids by simply accounting for power conservation at each node of the grid, rendering it more than just a mathematical convenience.  

However, all these results were obtained by focusing on the interactions to be purely dyadic in nature. Recent research highlights that such a reductionistic view might not reveal the complete picture of the underlying mechanism of exotic phenomena observed in some real-world complex systems where the interactions between agents are inherently higher-order in nature \cite{benson2018simplicial,majhi2022dynamics,battiston2020networks}. We focus on the results presented by \textit{Skardal \& Arenas} in \cite{skardal2020higher} where higher-order interactions were incorporated into the Kuramoto model, inducing abrupt (de)synchronization transition. It was naturally inquisitive to observe that adding inertia to the Kuramoto model or incorporating higher-order interactions independently led to a first-order phase transition in the system. Unsurprisingly we were interested in understanding how the interplay of inertia and higher-order interactions manifests itself in the system and affect the synchronization profile, which to the best of our knowledge, has not been explored before. 
%we study the Kuramoto model with inertia on simplicial complexes. We develop a self-consistent system of equations to characterize the steady-state behavior of the system. We find that inertia and higher-order interactions independently influence the phenomenon of synchronization, due to which there is a prolonged hysteresis in the system. The forward transition point is found to be exclusively dependent on the mass, shifting to the right with an increase in the mass of the oscillators, whereas the backward transition point depends solely upon the strength of higher-order interaction and brings about a shift to the left with an increase in the magnitude of higher order coupling. Hence a prolonged hysteresis is observed where the macroscopic system exists in two stable states depending on the initial condition when both the mass and the strength of the higher-order coupling are increased simultaneously. Numerical simulations are in good agreement with analytical predictions. Such prolonged existence of the coherent branch can prove beneficial for the dynamics of real-world systems like power grids which are explicitly modeled using the Kuramoto model with inertia.

To that end, in this Letter, we unify these two disparate fields by providing a generalized analytical framework motivated by \cite{tanaka1997first} to study the steady-state behavior of coupled oscillator systems with inertia interacting via higher-order interactions. %We extend the model proposed in \cite{skardal2020higher} for globally coupled networks by adding inertia to the system and considering the simultaneous presence of  dyadic and triadic interactions. 
We study the inertial effects in the model proposed by \cite{skardal2020higher} for globally coupled networks considering the simultaneous presence of  dyadic and triadic interactions. 
Phases of $N$-coupled oscillators, each with mass $m$, evolve based on the following coupled nonlinear equations.
\begin{equation}\label{Gen_KM_hoi}
    \begin{split}
    m\ddot{\theta_i} =  -\dot{\theta_i} + \omega_i + \frac{K_{1}}{N}\sum_{j = 1}^{N}\sin(\theta_{j} - \theta_{i}) \\
    +\frac{K_{2}}{N^{2}} \sum_{j = 1}^{N}\sum_{k = 1}^{N}\sin(2\theta_{j} - \theta_{k}- \theta_{i})
    \end{split}
\end{equation}
In Eq.~\ref{Gen_KM_hoi}, $\theta_{i}$ and $\dot\theta_{i}$ refer to the  instantaneous phase and angular velocity of  the $i^{th}$ oscillator, respectively. $\omega_{i}$ is the intrinsic frequency of the $i^{th}$ oscillator derived from a unimodal symmetric probability distribution $g(\omega)$ with mean $\Omega$.  The coupling constants $K_1 \geq 0$ and $K_2$ are the dyadic and triadic coupling strengths, respectively. %Before diving into the analytical treatment for Eq.~\ref{Gen_KM_hoi}, we direct the reader to Fig.~\ref{}

We decouple the differential equations in Eq.~\ref{Gen_KM_hoi} and write them in terms of mean-field quantities by introducing the following general order parameter for $p \in \{1,2\}$. 
\begin{equation}\label{Order Parameter}
     r_{p}e^{i\psi_{p}} = \frac{1}{N}\sum_{j = 1}^{N}e^{ip\theta_{j}}
\end{equation}
From the above definition, $r_{1}$ measures the global phase coherence and can be interpreted as the centroid of phases of oscillators on a unit circle in the complex plane, and $\psi_{1}$ measures the average phase of the oscillators. $r_{2}$, referred to as the Daido order parameter \cite{daido1996multi} captures cluster synchronization. As we are interested in the steady state behavior of the system, we omit the time dependence in the definition of the general order parameter. In the incoherent state, the phases of the oscillators are scattered uniformly on the unit circle and hence $r_{1} \approx r_{2} \approx 0$. Meanwhile, in the coherent state, a single group of oscillators is formed locked to the mean phase $\psi_{1}$ rotating uniformly at angular velocity $\Omega$, hence $r_{1} \approx r_{2} \approx 1$. Using Eq.~\ref{Order Parameter}, Eq.~\ref{Gen_KM_hoi} can be written in terms of mean-field quantities as,
\begin{equation}\label{mean field equation}
    \begin{split}
        m\ddot{\theta}_{i} =  -\dot{\theta}_{i} + \omega_i + K_{1}r_{1}\sin(\psi_{1} - \theta_{i}) \\  +K_{2}r_{1}r_{2}\sin(\psi_{2} - \psi_{1}- \theta_{i})
    \end{split}
\end{equation}
Because of the rotational symmetry in the model, the mean of the $g(\omega)$ distribution can be set to zero by moving into the rotating frame at the frequency $\Omega$. This can be facilitated by making the transformation $\theta_{i} \rightarrow \theta_{i} + \Omega t$ in Eq.~\ref{Gen_KM_hoi}. Once in the rotating frame, by choosing appropriate initial conditions,  $\psi_{1}$ and $\psi_{2}$ can be set to zero without loss of generality. Eq.~\ref{mean field equation}, now takes the following form,
\begin{equation}\label{mean field equation 2}
    m\ddot{\theta}_{i} =  -\dot{\theta}_{i} + \omega_i - q\sin(\theta_{i})
\end{equation}
Where, for the ease of notation, $q= r_{1}(K_{1} +K_{2}r_{2})$. Note that for a fixed $K_{2}$, Eq.~\ref{mean field equation 2} has three variables $K_{1}$, $r_{1}$ and $r_{2}$. Hence, to chalk out the steady state behavior of Eq.~\ref{mean field equation 2}, we develop a system of self-consistent equations and seek the values of $(K_{1},r_{1},r_{2})$ which simultaneously satisfy them. We start by taking the thermodynamic limit ($N\rightarrow\infty$); the coupled oscillator system in the steady state is then described by a probability density $\rho(\theta,\omega)$ where for a given intrinsic frequency $\omega$, $\rho(\theta,\omega)d\theta$ represents the fraction of oscillators with their phase between $\theta$ and $\theta + d\theta$. The general order parameter in Eq~\ref{Order Parameter} takes the  following form in the continuum limit,
\begin{equation}\label{Order Parameter Thermodynamic}
     r_{p}e^{i\psi_{p}} = \int_{-\pi}^{\pi}\int_{-\infty}^{\infty} e^{ip\theta}\rho(\theta,\omega)g(\omega)d\omega d\theta
\end{equation}

\begin{figure}[t]
    \centering 
    \includegraphics[width= \columnwidth]{Finalfig.eps}
    \caption{(Color online) Prolonged Hysteresis. a) Schematic depiction of emerging collective behavior in  the Kuramoto Model (KM). ($a^{\prime}$), ($c^\prime$), and ($d^\prime$) plot the usual behavior of KM \cite{kuramoto1975self} in the sole impression of higher-order \cite{skardal2020higher} or inertia \cite{tanaka1997first}, whereas 
    %Introducing $K_{2}$ in KM shifts the backward transition point leftwards. 
    ($b^{\prime}$) illustrates a simultaneous forward and backward shift in the transition point upon introduction of $m$ and $K_{2}$ in KM (Eq.~\ref{Gen_KM_hoi}), revealing a prolonged hysteresis. 
    The green arrow indicates the direction of the shift in the transition point.     %($c^{\prime}$) The second order phase transition to synchronization in KM at the critical coupling value 2. ($d^{\prime}$) Introducing $m$ in KM shifts the forward transition point rightwards. 
    b) $r_1$ versus $K_1$ plot for $K_2 = 1$ and $m = 1$ (blue-circles) and $K_2 = 7$ and $m = 3$ (red-squares). Filled circles and squares represent the simulation results for the forward, and hollow circles and squares represent the backward processes. The dashed and continuous curves represent the forward and backward analytical predictions, respectively.}
    \label{fig: figure1}
\end{figure}

In the steady state, the oscillator population splits up into two groups depending on their intrinsic frequency. One group of oscillators is locked to the mean phase; meanwhile, the other oscillators drift over the locked oscillators. Hence the overall phase coherence ($r_{p}$) can be split into contributions from the locked ($r_{p}^{l}$) and drifting ($r_{p}^{d}$) oscillators, i.e, $r_{p} = r_{p}^{l} + r_{p}^{d}$.
Before calculating $r_p^l$ and $r_p^d$, we point out that systems whose motion is governed by Eq.~\ref{mean field equation 2} are known to depict hysteresis and have been well studied in \cite{tanaka1997first,tanaka1997self,strogatz2018nonlinear}. For the sake of completeness, we briefly summarise the reason for the hysteretic behavior here. Dropping the subscript $i$ and by introducing a new timescale $\tau = \sqrt{\frac{q}{m}}t$, Eq.~\ref{mean field equation 2} is transformed to a second order differential equation with just two parameters as %\begin{equation}\label{dimensionless mean field equation}
    $\ddot{\theta} =  -\alpha \dot{\theta} + \beta -\sin(\theta)$,
%\end{equation}
where $\alpha = \frac{1}{\sqrt{qm}}$ is the damping term and $\beta = \frac{\omega}{q}$. This equation %Eq.~\ref{dimensionless mean field equation} 
has two fixed points, a saddle, and a sink for $\beta<1$, obtained by setting $\dot\theta = 0$ and $\ddot\theta = 0$. The sink is a stable fixed point if $\alpha$ is large enough or if $\beta$ is close to one; otherwise, it is a stable spiral. At $\beta = 1$, the system undergoes a saddle-node bifurcation annihilating the two fixed point solutions and admitting a unique stable limit cycle solution for all $\beta >1$ \cite{levi1978dynamics}. However, it so happens that as we decrease the value of $\beta$ to be less than one, the limit cycle persists for some small values of $\alpha$. Hence, bistability exists in the system, where a stable limit cycle and a sink coexist. A further decrease in $\beta$ will result in the disintegration of the limit cycle via a homoclinic bifurcation. Fig.~\ref{fig : figure2}a displays these three dynamical regimes in the $\alpha-\beta$ parameter space. For small values of the damping term $\alpha$, ensured by keeping finite inertia, the homoclinic bifurcation curve is seen to be approximated by a straight line Fig.~\ref{fig : figure2}a. Upon implementing Melnikov's method, \cite{strogatz2018nonlinear,guckenheimer2013nonlinear} the equation of the straight line comes out to be $\beta = \frac{4}{\pi} \alpha$. In conclusion, we see the presence of three different dynamical regimes %for Eq.~\ref{dimensionless mean field equation},
namely a fixed point ($\beta < \frac{4}{\pi} \alpha$), bi-stable region ($\frac{4}{\pi} \alpha < \beta < 1$), and a limit-cycle ($\beta > 1$) \cite{strogatz2018nonlinear}.



The bi-stable region turns out to be responsible for hysteresis in systems governed by equations like Eq.~\ref{mean field equation 2}. Hence, following \cite{tanaka1997first}, instead of studying the system in its full generality, we break down the self-consistency analysis for our model into forward $\left(f\right)$ and backward $\left(b\right)$ processes. In the forward process, we start from a small $K_1$ value, and therefore the system is in an incoherent state $\left(r_{1}\approx 0\right)$. This leads to high $\alpha$ and $\beta$ values, indicating that the oscillators are in the limit cycle regime. As we adiabatically increase $K_1$, the oscillators stay in the basin of attraction of the stable limit-cycle even after crossing $\beta = 1 (\omega = q)$ and fall into the locked cluster only after $\beta = \frac{4}{\pi}$ $\alpha (\omega = \frac{4}{\pi} \sqrt{\frac{q}{m}})$, below which the limit cycle vanishes. For the backward process, we start from a high $K_1$ value and hence the oscillators exist in the fixed-point state, i.e., the oscillators are locked in a cluster $\left(0<< r_1 < 1\right)$. As we adiabatically decrease $K_1$ from this state, the oscillators remain in the basin of attraction of the sink until  $\beta = 1$, when the fixed points vanish via a saddle node bifurcation. Thus, in the backward process, oscillators having $|\omega| \le q = \omega_{b}$ contribute to the locked oscillators, while in the forward process, only those with $|\omega| \le \frac{4}{\pi} \sqrt{\frac{q}{m}} = \omega_{f}$ are in a locked state and all the oscillators with $\omega > \omega_{f,b}$ drift around the locked cluster. We point out that $K_{2}$ is concealed in $q$ and hence directly affects the fraction of oscillators that are in a locked or drifting state. The contribution of the locked oscillator($r_{p}^{l}$) to overall coherence for the forward/backward process can now be calculated as $r_{p}^{l} = \int_{-\omega_{f,b}}^{\omega_{f,b}}e^{ip\sin^{-1}(\frac{\omega}{q})}g(\omega)d\omega $. The imaginary part of $r_{p}^{l}$ goes to zero as $g(-\omega) = g(\omega)$. Hence taking only the real part and noting that  $\theta_{f,b} = \sin^{-1}(\omega_{f,b}/q)$, we arrive at the expression for $r_{p}^{l}$ as follows,
\begin{equation}\label{locked term}
    \begin{split}
    r_{p}^{l} = q\int_{-\theta_{f,b}}^{\theta_{f,b}}\cos(\theta)\cos(p\theta)g(q\sin(\theta))d\theta
    \end{split}
\end{equation}
The contribution to overall coherence from the drifting oscillators can be accounted for by calculating  $r_{p}^{d} = \int_{|\omega|>\omega_{f,b}}\int_{-\pi}^{\pi} e^{ip\theta}\rho_{d}(\theta,\omega)g(\omega)d\omega d\theta$ where $\rho_{d}(\theta,\omega)$ is the density of drifting oscillator which satisfies $\rho_{d}(\theta,\omega) \propto 1/\dot\theta$ \cite{tanaka1997first}. The normalization condition for $\rho_{d}(\theta,\omega)$ gives, $ \int_{-\pi}^{\pi}\rho_{d}(\theta,\omega) d\theta= \int_{0}^{T}\rho_{d}(\theta,\omega)\dot{\theta}dt = 1$ (for a given $\omega$), where $T$ is the time period of the whirling limit cycle solution. Hence we end up with the relation $\rho_{d}(\theta,\omega) = \frac{1}{\dot{\theta}T}$, which when plugged into the form of $r_{p}^{d}$ gives us,
\begin{equation}\label{drift1}
    \begin{split}
     r_{p}^{d}  = \int_{|\omega|> \omega_{f,b}} \left[\frac{1}{T}\int_{0}^{T}e^{ip\theta}dt\right]g(\omega)d\omega
     \end{split}
\end{equation}
To calculate $r_p^d$, we first need to obtain an approximate analytic expression for the whirling limit cycle solution. We follow the method specified in \cite{gao2018self} of writing $\dot{\theta}$ as a Fourier series in  $\theta$ by only considering the first harmonics $(\dot{\theta} = A_{0} + A_{1}\cos(\theta)+B_{1}\sin(\theta))$. On substituting this in  $\ddot{\theta} =  -\alpha \dot{\theta} + \beta -\sin(\theta)$, we find the expression of the coefficients in terms of $\alpha$($=\frac{1}{\sqrt{qm}}$) and $\beta$($=\frac{\omega}{q}$) such that the first harmonic vanishes. This gives us $\dot{\theta} = \frac{\beta}{\alpha} + \frac{\alpha^{2}}{\alpha^{4} + \beta^{2}}\left[\frac{\beta}{\alpha}\cos(\theta) - \alpha \sin(\theta)\right]$ and upon integrating $\dot\theta$ with time, and choosing the constant of integration such that $\theta(0) = 0$, we end up with $\theta = \frac{\beta t}{\alpha} + \frac{\alpha^2}{\alpha^4 + \beta^2}\left[\frac{\alpha^2}{\beta} (\cos(\frac{\beta t}{\alpha}) -1) + \sin(\frac{\beta t}{\alpha})\right]$\cite{gao2018self}. Notice that as $\theta(t,-\omega) = -\theta(t,\omega) $ and $g(-\omega) = g(\omega)$, the imaginary part in Eq.~\ref{drift1} goes to zero. Thus,
\begin{equation}\label{drift term}
     r_{p}^{d} = \int_{|\omega|> \omega_{f,b}}\left<\cos(p\theta)\right>g(\omega)d\omega 
\end{equation}
 The expression for $\left<\cos(p\theta)\right>$ (for $p \in \{1,2\}$) can now be readily calculated as $\left<\cos(p\theta)\right> = \frac{1}{T}\int_{0}^{T}\cos(p\theta)dt = \int_{0}^{2\pi}\frac{\cos(p\theta)}{\dot{\theta}}d\theta \text{\LARGE $/$} \int_{0}^{2\pi}\frac{1}{\dot{\theta}}d\theta$ to obtain
%\begin{subequations}\label{average cos}
 %   \begin{equation} 
  $  \left<\cos(\theta)\right> = \frac{\beta}{\alpha} \left[ \sqrt{\frac{\beta^2}{\alpha^2} - \frac{\alpha^2}{\beta^2 + \alpha^4}} - \frac{\beta}{\alpha} \right]
   $ %\end{equation} 
    %\begin{equation} 
    %\begin{split}
   and $\left<\cos(2\theta)\right> = \left[\frac{\beta^2 - \alpha^4}{\beta^2 + \alpha^4}\right] \times
    \left[ \frac{2\beta(\beta^2 + \alpha^4)}{\alpha^3} \left(\frac{\beta}{\alpha} - \sqrt{\frac{\beta^2}{\alpha^2} - \frac{\alpha^2}{\beta^2 + \alpha^4}} \right) - 1\right] $. 
    %\end{split}
%\end{equation}
%\end{subequations}
We are now finally ready to write down the set of self-consistent equations that lets us describe the steady state of the coupled oscillator system governed by Eq.~\ref{Gen_KM_hoi}. For the remainder of the work, we consider the intrinsic frequency to be derived from Lorentz distribution, $g(\omega) = \frac{1}{\pi}\frac{1}{1 + \omega^{2}}$ with mean zero. Noting that the integrands in Eqs.~\ref{locked term} and \ref{drift term} for $p \in \{1,2\}$ are even functions, we arrive at,
%\begin{subequations}\label{total self consistent}
     \begin{equation} \label{r1 self consistent final}
     \begin{split}
     r_{p}=2q\int_{0}^{\theta_{f,b}}\cos(\theta)\cos(p\theta)g(q\sin(\theta))d\theta  \\
      + 2\int_{\omega_{f,b}}^{\infty}\left<\cos(p\theta)\right>g(\omega)d\omega
     \end{split}
     \end{equation}
     \begin{comment}
     \begin{equation}\label{r2 self consistent final}
     \begin{split}
      r_{2} = 2q\int_{0}^{\theta_{f,b}}\cos(\theta)\cos(2\theta)g(q\sin(\theta))d\theta \\
      + 2\int_{\omega_{f,b}}^{\infty}\left<\cos(2\theta)\right>g(\omega)d\omega
     \end{split}
     \end{equation}
    \end{comment}
%\end{subequations}


These two equations together describe the steady-state behavior of the system. To find the nontrivial branch (for both forward and backward processes), we numerically solve the above set of self-consistent equations. Fig.~\ref{fig: figure1}a provides a schematic representation of the synchronization profiles of our result in comparison to previously explored models \cite{kuramoto1975self,tanaka1997first,skardal2020higher}. Fig.~\ref{fig: figure1}b presents analytical and simulation results for the $r_{1}$ vs. $K_{1}$ curves for $(m, K_{2}) = (1,1)$ and $(m, K_{2}) = (3,7)$. As for the simulation protocol, we simulate Eq.~\ref{mean field equation} on a network of $N = 10^{4}$ nodes by splitting it into a pair of first-order differential equations and integrating them using the Runge-Kutta 4 algorithm (time-step 0.1). For a chosen value of $m$ and $K_2$, we start with random initial conditions for $\theta (\in [0,2\pi))$ and $\dot\theta (\in [-1,1])$ and $K_1 = 0$. We adiabatically increase $K_1$ in steps of $\Delta K_1$ ($=0.1$, unless specified otherwise) till $K_{1} = 12$ is reached (forward), followed by an adiabatic decrease till $K_1 = 0$ (backward). By adiabatic increase/decrease, we imply that for every $K_1$ except the first ($K_1 = 0$), the initial conditions are taken as the final state obtained for the previous $K_1$ value. At all coupling strengths $K_1$, the order parameter values are calculated after discarding transients by averaging over the steady state. 

\begin{figure}[t]
    \centering
    \includegraphics[width=\columnwidth]{SecondResult.eps}
    \caption{(Color online) a) $\alpha(=1/\sqrt{qm})$-$\beta(=\omega/q)$ parameter space. Different dynamical regimes present in the $\dot{\theta}$ vs. $\theta$ phase space of $\ddot{\theta} =  -\alpha \dot{\theta} + \beta -\sin(\theta)$. b)  Synchronization profile $r_1$ versus $K_1$ for $m = 1$ and different values of $K_2 = $ 0 (orange-squares), 3 (blue-triangles), and 5 (red-circles). c) Synchronization profile for a fixed value of $K_2 = 2$ and different values of $m = $ 3 (orange-squares), 1 (blue-triangles), and 0 (red-circles). In both b) and c), the hollow and filled symbols indicate the simulation results for the forward and backward cases, respectively. The dashed and continuous curves represent the analytically calculated values for the forward and backward processes, respectively. d) Backward transition points. The dashed curve represents the analytical predictions of $K_1^b$ for $m=0$ and different $K_2$.  The scatter plots are the $K_1^b$ vs. $K_2$ for different $m$ = 0, 1, 5, 10 obtained via numerical simulation. Inset: The dashed curve is the analytical prediction for $m=0$ till $K_2 = 30$. The solid line is the linear fit for the dashed curve (from $K_2=  5$ to $30$) having  slope $-0.65$ and intercept $5.17$.}
    \label{fig : figure2}
\end{figure}

Fig.~\ref{fig: figure1}b displays a good agreement between the simulation and analytical results. For the forward process, as $K_{1}$ is increased from zero, the system undergoes a first-order phase transition from incoherent to coherent state at a finite critical coupling value($K_{1}^{f}$). However, for the backward process, the system undergoes abrupt desynchronization at a value($K_{1}^{b}$), which is less than $K_{1}^{f}$. Hence hysteresis is observed where the system stays in two different states depending on the initial condition. The derived self-consistency equations can also be used with other extended-tailed distributions like the Gaussian distribution. Also, a better fit in Fig.~\ref{fig: figure1}b between analytical and numerical values for the backward process in the strongly synchronized regime can be obtained by increasing the maximum value of $K_{1}$ in the simulation protocol.  We point out that when $m$ and $K_{2}$ values are both increased, $K_{1}^{f}$ shifts to the right while $K_{1}^{b}$ shifts to the left, revealing a prolonged hysteresis region as seen in Fig.~\ref{fig: figure1}b. 

%\begin{figure}[t]
    %\centering
    %\includegraphics[width=0.90\columnwidth]{fig3.eps}
    %    \caption{(Color online) }
    %\label{synchronisation branches}
%\end{figure}

 A natural question would then be to address the dependency of the forward and backward transition points on  $m$ and $K_{2}$. To analytically obtain the expression for the forward transition point($K_{1}^{f}$), we evaluate Eq.~\ref{r1 self consistent final} in the limit $r_{1}\rightarrow 0^{+}$ ($q\rightarrow 0^{+}$). As we take this limit, we see that $\beta/\alpha$($=\omega\sqrt{\frac{m}{q}}$) tends to  very high value as compared to $\alpha^2/(\beta^2 + \alpha^4)$($= \frac{qm}{1+\omega^{2}m^{2}}$). This allows us to perform a Taylor series expansion of $\left<\cos(\theta)\right>$ for $\epsilon = \alpha^2/(\beta^2 + \alpha^4) << 1$ which gives, $\left<\cos(\theta)\right> = \frac{-\alpha^2}{2(\beta^2 + \alpha^4)} + \mathcal{O}(\epsilon^4) \approx \frac{-mq}{2(1+m^2\omega^2)}$. However, in the limit $r_{1}\rightarrow 0^{+}$, $r_{2}\rightarrow 0^{+}$  and the parameter $\alpha \rightarrow \infty$ implying that the limit of the integrals for the forward and backward processes become the same as there exists no bistability region in the parameter space. Taking $\theta_{f,b} = \frac{\pi}{2}$, dividing both sides of Eq.~\ref{r1 self consistent final} by $q$, and evaluating the limit (at which the two equations in Eq.~\ref{r1 self consistent final} decouple) we have, $\frac{1}{K_{1}^{f}}= \frac{\pi}{2}g(0) - m\int_{0}^{\infty}\frac{1}{1+m^{2}\omega^{2}}g(\omega)d\omega$. After evaluating the integral and rearranging the terms, we end up with $K_{1}^{f} = 2(m+1)$. We see that the forward transition point is independent of $K_2$ and purely depends on $m$ and hence, matches the previously derived value of the forward transition point in \cite{gupta2014non,rodrigues2016the}. Fig.~\ref{fig : figure2}b illustrates the effect of varying $K_2 (0.0,3.0,5.0)$ for the case of fixed $m (= 1)$. As expected, the forward critical coupling ($K_1^f$) remains the same for all three cases validating our analytical result. At this $K_1^f (= 4)$, the magnitude of the first-order jump for fixed $m$ is seen to increase with the value of $K_2$. In Fig.~\ref{fig : figure2}c, we study the effect of varying mass ($0.0, 1.0, 3.0$) for the case of fixed $K_2( = 2.0)$. As inertia increases, $K_1^f$ shifts to higher values as predicted analytically. However, we note that the analytically calculated values of $K_1^f$ do not match exactly with numerical simulations owing to the finite size effects. A detailed study on the same has been done in \cite{sb2014}. 

%\begin{figure}[t]
 %   \centering
  %  \includegraphics[width=0.7\columnwidth]{fig4.eps}
   % \caption{(Color online) }
    %\label{Backward desync}
%\end{figure}

A fairly good analytical approximation for $K_{1}^{b}$, as also pointed out in \cite{sb2014}, would be to obtain the minimum value of $K_{1}$ along the non-trivial branch of the backward self-consistent curve. The simulation results in  Fig.~\ref{fig: figure1}b and Fig.~\ref{fig : figure2}b and \ref{fig : figure2}c are seen to back up this observation for our model. However, obtaining a clean analytical expression for the same by calculating $\frac{dK_{1}}{dr_{1}} = 0$ is not possible because of the complexity of the integrand of the drift oscillator contribution in $r_2$. Alternatively, we resort to simulation results to decipher the dependency of $K_{1}^{b}$ on $m$ and $K_{2}$. From Fig.~\ref{fig : figure2}b, it can be seen that for the backward process, the coherent branch persists till increasingly smaller values of $K_1$ with an increase in the $K_2$ value, after which the system undergoes an abrupt transition to asynchrony. Hence it is clear that an increase in $K_{2}$ leads to a decrease in $K_{1}^{b}$. To study the effect of mass on $K_{1}^{b}$, we fix $K_{2}$ and vary $m$ as in Fig.~\ref{fig : figure2}c. It is observed that the backward branches for different $m$ values merge in the highly synchronized regime for fixed $K_{2}$(=2) and get separated in the weakly synchronized regime. There is an influence of $m$ on the nature of the curve in the weakly synchronized regime, which indicates the possibility of dependency of $K_{1}^{b}$ on $m$. 


It was shown in \cite{sb2014} that for the pure dyadic case ($K_{2} = 0$), $K_{1}^{b}$ decreases with an increase in $m$, until it reaches a plateau for high $m$ values. In Fig.~\ref{fig : figure2}d we address how this changes with the introduction of finite $K_{2}$. The $K_1^b$ obtained via simulation (performed for $N = 10^{3}$ number of nodes) for values of $K_2$ ranging from 0 to 10 and different values of $m$(0,1,5,10) are plotted. We see that for small values of $K_{2}$ and finite inertia, an increase in the values in $m$ leads to a decrease in $K_{1}^{b}$. However, we point out that for higher values of $K_{2}$, the effect of $m$ on $K_{1}^{b}$ becomes less pronounced, and desynchronization happens at the same value irrespective of mass. An analytical prediction of $K_{1}^{b}$ now becomes possible following this observation by considering the $m=0$ case. We derive self-consistent equations for this case \cite{SM} and obtain $K_{1}^{b}$ values corresponding to a particular $K_2$ by finding the minimum value of $K_{1}$ in the self-consistency curve. These analytically calculated $K_1^b$ values for the $m=0$ case are represented by the dashed line in Fig.~\ref{fig : figure2}d. The detailed derivations are given in \cite{SM}. It can be clearly observed that for higher values of $K_2$, the analytical predictions of $K_1^b$ match closely with the ones obtained via simulation for different masses. In the inset of Fig.~\ref{fig : figure2}d we plot the analytical predictions for $K_{1}^{b}$ for $m = 0$ till $K_{2} = 30$. For a high value of $K_{2}$($>5$), the curve is seen to be very well approximated by a straight line. We obtained the slope to be $-0.65$ and intercept $5.17$ after performing a linear fit for the predicted $K_{1}^{b}$ curve for high values of $K_{2}$(5-30). In conclusion, for $K_{2}>5$, irrespective of the mass of the oscillators, the backward desynchronization point can be fairly well estimated by $K_{1}^{b} \approx -0.65K_{2}+5.17$. 
%From the expression of forward and backward transition points, the width of the hysteresis region($\Delta W \approx K_{1}^{f} - K_{1}^{b} = 2m+0.65K_{2}-3.17$) is seen to increase linearly with increase in $m$ and $K_{2}$ for high values of $K_{2}$.

In this Letter, we have put forward a generalized analytical framework to study the steady-state behavior of coupled oscillator systems with inertia interacting via higher-order interactions. The analytical predictions, which are backed up by numerical simulation, show a prolonged hysteretic first-order phase transition to a (de)synchronized state. We show that the forward transition point increases linearly with $m$ and is independent of $K_2$. Meanwhile, the backward transition point is seen to decrease linearly with $K_2$ for high $K_2$ values. We have presented the results for triadic interactions; however, it is easy to extend our analysis to other powers of higher-order interactions, as long as the sinusoidal coupling function contains $\theta_i$ term only. As an example, the detailed analysis involving quartic interactions is presented in \cite{SM}. Further note that developing the self-consistent method for other choices of higher-order coupling functions, such as $\sin(\theta_j + \theta_k - 2\theta_i)$ \cite{tanakat2011multistable, skardal2019abrupt} along with pairwise coupling proves to be complicated because of the existence of higher order harmonics in the mean-field equation. However, we have analyzed a model using the self-consistency method for the pure triadic interaction case for such a coupling function in \cite{SM}. An immediate future direction of our work would be to extend our analysis to diluted simplicial complexes, which can provide fundamental insights into the dynamics of various real-world complex systems such as power grids.

%Our analytical formulation presents an avalanche of future directions, can be extended for diluted simplicial complexes, which can provide fundamental insights into the dynamics of various real-world complex systems such as power grids \cite{}, can incorporate adaption in the coupling strength \cite{} to name a few.
\begin{acknowledgments}
SJ gratefully acknowledges SERB Power grant SPF/2021/000136. The work is supported by the computational facility received from the Department of Science and Technology (DST), Government of India under FIST scheme (Grant No. SR/FST/PSI-225/2016). SJ is thankful to  Mehrnaz Anvari and Baruch Barzel for their comments on the manuscript.
\end{acknowledgments}

\begin{thebibliography}{99}

    \bibitem{boccaletti2006complex} Boccaletti, S., Latora, V., Moreno, Y., Chavez, M., \& Hwang, D. U. {Complex networks: Structure and dynamics.} {\em Physics reports} {\bf 424} (2006).

    \bibitem{kuramoto1975self} Kuramoto, Y. Self-entrainment of a population of coupled non-linear oscillators.( Springer. Berlin Heidelberg. 1975). {\em International Symposium on Mathematical Problems in Theoretical Physics: January 23–29, 1975, Kyoto University (pp. 420-422)}

    \bibitem{sethia2008clustered} Sethia, Gautam C., Abhijit Sen, and Fatihcan M. Atay. {Clustered chimera states in delay-coupled oscillator systems.} {\em Physical review letters} {\bf 100}, 144102 (2008).

    \bibitem{childs2008stability} Childs, L. M., \& Strogatz, S. H. {Stability diagram for the forced Kuramoto model}. {\em Chaos: An Interdisciplinary Journal of Nonlinear Science}, {\bf 18} (4), 043128 (2008).

    \bibitem{omel2012nonuniversal} Omel’Chenko, E., \& Wolfrum, M. Nonuniversal transitions to synchrony in the Sakaguchi-Kuramoto model. {\em Physical review letters}, {\bf 109} (16), 164101 (2012).

    \bibitem{olmi2015chimera} Olmi, S. {Chimera states in coupled Kuramoto oscillators with inertia}. {\em Chaos: An Interdisciplinary Journal of Nonlinear Science}, {\bf 25} (12), 123125 (2015).

    \bibitem{rodrigues2016the} Rodrigues, F. A., Peron, T. K. D., Ji, P., \& Kurths, J. (2016). The Kuramoto model in complex networks. {\em Physics Reports}, {\bf 610}, 1-98.

    \bibitem{ermentrout1991adaptive} Ermentrout, B. An adaptive model for synchrony in the firefly Pteroptyx malaccae. {\em Journal of Mathematical Biology}, {\bf 29} (6) (1991).

    \bibitem{tanaka1997first} Tanaka, H. A., Lichtenberg, A. J., \& Oishi, S. I. {First order phase transition resulting from finite inertia in coupled oscillator systems}. {\em Physical review letters}, {\bf 78} (11), 2104 (1997). 

    \bibitem{tanaka1997self} Tanaka, H. A., Lichtenberg, A. J., \& Oishi, S. I. Self-synchronization of coupled oscillators with hysteretic responses. {\em Physica D: Nonlinear Phenomena}, {\bf 100} (3-4) (1997).


    \bibitem{ji2013cluster} Ji, P., Peron, T. K. D., Menck, P. J., Rodrigues, F. A., \& Kurths, J. Cluster explosive synchronization in complex networks. {\em Physical review letters}, {\bf 110} (21), 218701 (2013).

     \bibitem{trees2005sync} Trees, B. R., Saranathan, V., \& Stroud, D. Synchronization in disordered Josephson junction arrays: Small-world connections and the Kuramoto model. {\em Physical Review E}, {\bf 71(1)}, 016215 (2005).

    \bibitem{ji2014basin} Ji, P., \& Kurths, J. Basin stability of the Kuramoto-like model in small networks. {\em The European Physical Journal Special Topics}, {\bf 223(12)} (2014).

    \bibitem{grzybowski2016on} Grzybowski, J. M. V., Macau, E. E. N., \& Yoneyama, T. On synchronization in power-grids modelled as networks of second-order Kuramoto oscillators. {\em Chaos: An Interdisciplinary Journal of Nonlinear Science}, {\bf 26(11)}, 113113 (2016).
    
    %\bibitem{watanabe1994constants} Watanabe, S., \& Strogatz, S. H. Constants of motion for superconducting Josephson arrays. {\em Physica D: Nonlinear Phenomena}, {\bf 74} (1994).

    \bibitem{rohden2012self} Rohden, M., Sorge, A., Timme, M., \& Witthaut, D. Self-organized synchronization in decentralized power grids. {\em Physical review letters}, {\bf 109} (6), 064101 (2012).

    \bibitem{dorfler2013synchronization} Dörfler, F., Chertkov, M., \& Bullo, F. Synchronization in complex oscillator networks and smart grids. {\em Proceedings of the National Academy of Sciences}, {\bf 110} (6) (2013).

    \bibitem{witt2022col} Witthaut, D., Hellmann, F., Kurths, J., Kettemann, S., Meyer-Ortmanns, H., \& Timme, M. Collective nonlinear dynamics and self-organization in decentralized power grids.{\em Reviews of modern physics}, {\bf 94(1)}, 015005 (2022).


    \bibitem{filatrella2008analysis} Filatrella, G., Nielsen, A. H., \& Pedersen, N. F. Analysis of a power grid using a Kuramoto-like model. {\em The European Physical Journal B}, {\bf 61} (2008).

    \bibitem{benson2018simplicial} Benson, A. R., Abebe, R., Schaub, M. T., Jadbabaie, A., \& Kleinberg, J. Simplicial closure and higher-order link prediction. {\em Proceedings of the National Academy of Sciences}, {\bf 115} (48) (2018).

    \bibitem{majhi2022dynamics} Majhi, S., Perc, M., \& Ghosh, D. Dynamics on higher-order networks: A review. {\em Journal of the Royal Society Interface}, {\bf 19} (188), 20220043 (2022).

    \bibitem{battiston2020networks} Battiston, F., Cencetti, G., Iacopini, I., Latora, V., Lucas, M., Patania, A., \& Petri, G. Networks beyond pairwise interactions: structure and dynamics. {\em Physics Reports}, {\bf 874}, (2020).

    \bibitem{skardal2020higher} Skardal, P. S., \& Arenas, A. Higher order interactions in complex networks of phase oscillators promote abrupt synchronization switching. {\em Communications Physics}, {\bf 3} (1), 218 (2020).

    \bibitem{daido1996multi} Daido, H. Multibranch entrainment and scaling in large populations of coupled oscillators. {\em Physical review letters}, {\bf 77(7)}, 1406 (1996).

    \bibitem{strogatz2018nonlinear} Strogatz, S. H. Nonlinear dynamics and chaos with student solutions manual: With applications to physics, biology, chemistry, and engineering.(CRC press. 2018).

    \bibitem{levi1978dynamics} Levi, M., Hoppensteadt, F. C., \& Miranker, W. L. Dynamics of the Josephson junction. {\em  Quarterly of Applied Mathematics}, {\bf 36} (2) (1978).

    \bibitem{guckenheimer2013nonlinear} Guckenheimer, J., \& Holmes, P. Nonlinear oscillations, dynamical systems, and bifurcations of vector fields. (Springer Science \& Business Media, 2013) Vol. 42.

    \bibitem{gao2018self} Gao, J., \& Efstathiou, K. Self-consistent method and steady states of second-order oscillators. {\em Physical Review E}, {\bf 98} (4), 042201 (2018).

     \bibitem{sb2014} Olmi, S., Navas, A., Boccaletti, S., \& Torcini, A. Hysteretic transitions in the Kuramoto model with inertia. {\em Physical Review E}, {\bf 90} (4), 042905 (2014

    \bibitem{gupta2014non} Gupta, S., Campa, A.,\& Ruffo, S. Nonequilibrium first-order phase transition in coupled oscillator systems with inertia and noise. {\em Physical Review E}, {\bf 89(2)}, 022123 (2014).

    \bibitem{SM} Supplementary Material (SM) contains self-consistency derivation for $m=0$ case, as well as derivation to predict the backward desynchronization points. It also contains analytical and simulation results for the model with quartic coupling. It further explores a different higher-order coupling function ($\sin(\theta_j + \theta_k - 2\theta_i)$) and describes analytical and numerical predictions of the dynamical behavior.

    \bibitem{tanakat2011multistable}Tanaka, T., \& Aoyagi, T. Multistable attractors in a network of phase oscillators with three-body interactions. {\em Physical Review Letters}, {\bf 106(22)}, 224101 (2011).
    
    \bibitem{skardal2019abrupt} Skardal, P. S., \& Arenas, A. Abrupt desynchronization and extensive multistability in globally coupled oscillator simplexes. {\em Physical review letters}, { bf 122(24)}, 248301 
    (2019

\end{thebibliography}

%\printbibliography

\end{document}
