\documentclass[11pt]{amsart}
\usepackage[english]{babel}
\usepackage[T1]{fontenc}
%\textwidth=150mm
\parindent=0in

\usepackage{geometry}

\geometry{height=21.5cm} 
\geometry{width=14.5cm} 
\usepackage{fullpage}
\usepackage[dvips]{graphicx}
\usepackage{color}

\usepackage{amsmath,amscd,amsthm,amsfonts,latexsym,epsfig,multirow, amssymb}

\theoremstyle{plain}
\usepackage{amssymb}
\usepackage{newlfont}
\usepackage{amsfonts}
\usepackage{graphicx}
\usepackage{xfrac}
\usepackage{enumerate}
\usepackage{tikz}
%\usepackage{calligra}
\usepackage[all]{xy}
\usepackage{subfigure}

%%%%%%%%%%%%%%%%%%%%%%%%%%%%%%%%%%%

%\theoremstyle{theorem}
\newtheorem{theorem}{Theorem}
\newtheorem{proposition}{Proposition}
\newtheorem{lemma}{Lemma}
\newtheorem{corollary}{Corollary}

\theoremstyle{definition}
\newtheorem{definition}{Definition}
\newtheorem{remark}{Remark}
\newtheorem{question}{Question}

\newtheorem{thmy}{Theorem}
\renewcommand{\thethmy}{\Alph{thmy}} 


\def \r{\mbox{${\mathbb R}$}}
\renewcommand{\Re}{\mathcal Re}
\flushbottom

\begin{document}

\title{A family of higher genus complete minimal surfaces that includes the Costa-Hoffman-Meeks one}

\author{Irene I. Onnis}
\address{Universit\`a degli Studi di Cagliari\\
Dipartimento di Matematica e Informatica\\
Via Ospedale 72\\
09124 Cagliari, Italy.}
\email{irene.onnis@unica.it}

\author{B\'arbara C. Val\'erio}
\address{Instituto de Matem\'atica e Estat\'istica\\
              Universidade de S\~ao Paulo\\
              05508-090 S\~ao Paulo, SP, Brazil. }
\email{barbarav@ime.usp.br}

\author{Jos\'{e} Antonio M. Vilhena}
\address{Universidade Federal do Par\'a\\ 
Instituto de Ci\^encias Exatas e Naturais\\ Faculdade de Matem\'atica\\ Rua Augusto Corr\^ea 01, 66075-110 Bel\'{e}m, PA,  Brazil.}
\email{vilhena@ufpa.br}

\date{March 2023}
\subjclass{53A10; 53C42}
\keywords{Examples of minimal surfaces. Elliptic functions. Riemann surface. Planar-type end. Catenoid-type end. Enneper-type end. Limits of minimal surfaces.}
\thanks{The first author was supported by a grant of Fondazione di Sardegna (Project GoAct) and by the Thematic Project: Topologia Álgebrica,  Geométrica e Diferencial, Fapesp process number 2016/24707-4. }

\begin{abstract}
In this paper,  we construct a one-parameter family of minimal surfaces in the Euclidean $3$-space of arbitrarily high genus and with three ends.   Each member of this family is immersed, complete and with finite total curvature.   Another interesting property is that the symmetry group of the genus $k$ surfaces $\Sigma_{k,x}$ is the dihedral group with $4(k+1)$ elements.  Moreover, in particular,  for $|x|=1$ we find the family of the Costa-Hoffman-Meeks embedded minimal surfaces,  which have two catenoidal ends and a middle flat end.
\end{abstract}
\maketitle


\section{Introduction}

 The Enneper surface is, after the plane, the simplest complete minimal surface in the three-dimensional Euclidean space.  It has genus zero and total  Gauss curvature equal to $-4\pi$.  In \cite{Chen.1982} C.C.  Chen and F.   Gackstatter constructed the first examples of complete minimal surfaces of genus one and two,  with one Enneper-type end of winding order three,  the same symmetries as the Enneper surface and total curvature $-8\pi$ and $-12\pi$, respectively.   In 1994 N. Do Espirito-Santo (\cite{Nedir.1994}) showed the existence of a complete minimal surface immersed in $\r^3$ of genus three,  having one Enneper-type end and total curvature $-16\pi$.  In \cite{Karcher.1989} H.  Karcher generalized the genus one Chen-Gackstatter surface by increasing the genus and the winding order of the end.  
 A similar generalization of the genus two Chen-Gackstatter surface was obtained by C.E.  Thayer in \cite{Thayer.1995},  solving numerically the associated period problem for genus as high as $35$.  Other generalizations of these Enneper-type minimal surfaces are given in \cite{Fang.1990} and \cite{Kang.2003}.\\
 
For what concerns minimal surfaces of genus one,  in 1982 C.  Costa has constructed a complete minimal surface of genus one in $\mathbb{R}^3$ with three embedded ends (see \cite{Costa.1982,  Costa.1984}).  Later,   D.  Hoffman and W.H.  Meeks in  \cite{Hoffman.1985}  have proved its embeddedness and, also,  in  \cite{Hoffman.1990} they have generalized this example for higher genus proving the existence of an infinite family  of embedded minimal surfaces $M_k$ of genus $k \geq 1$, with two catenoidal ends and one planar middle end (see the Main Theorem,  p. ~$1$). The total curvature of $M_k$ is $-4\pi(k+2)$ and its symmetry group is the dihedral group $\mathcal{D}(2k+2)$ with $4(k+1)$ elements generated by the orthogonal transformations
\begin{equation}\label{matrix}
	K=\left[
	\begin{array}{ccc}
	1&0&0\\
	0&-1&0\\
	0&0&1
	\end{array}
	\right]\qquad\text{and}\qquad
L_{\theta} = \left[
	\begin{array}{ccc}
	\mathcal{R}_{\theta}& &0\\
	 & &0\\
	0&0&-1
	\end{array}
	\right],
\end{equation}
where $\mathcal{R}_{\theta}$ represents the matrix of a rotation by $\theta=\pi/(k+1)$ radians in the $(x_1,x_2)$-plane.  The surface $M_k$ is known as ``Costa-Hoffman-Meeks surface'' of genus $k$.  \\

Inspired in previous results, in the first part of the paper  (see Section \ref{sec3}) we will use the theory of elliptic functions and the Enneper-Weierstrass Representation Theorem to construct in Theorem~\ref{teo3} a one-parameter family of minimal surfaces immersed in $\mathbb{R}^3$ of genus one,  starting from the Costa surface and replacing the catenoidal ends by  Enneper type ends.  Furthermore, we will prove that these surfaces have the same symmetries as the Costa surface.  We may restate some of Theorem~\ref{teo3} as follows:

\begin{thmy}\label{TeoA}
\textit{There exists a one-parameter family of complete, genus one,   
  minimal surfaces which are immersed in $\mathbb{R}^3$,  with  three ends and finite total curvature.  The family depends on a parameter $x$ with $|x| < \sqrt{\sqrt{8}-1}\, e_1$,  where $e_1=\wp(1/2)$ and $\wp$ is the Weierstrass function.  Moreover,  if $|x|=e_1$ one gets the Costa surface and if $|x|\neq e_1$,  the corresponding surfaces have total curvature equal to $-20 \pi$, two Enneper-type ends and one middle planar end.}
  \end{thmy}

In the Figure~\ref{Fig-sup} we present the images of four elements of this family of minimal surfaces with eight symmetries,  that will be denoted by $S_x^1$ .
	
\begin{figure}[h!]\label{Fig-sup}
\subfigure[\label{Figure_a}]{
\includegraphics[totalheight=6.2cm]{Sup_Sym1.png}}
\subfigure[\label{Figure_b}]{
\includegraphics[totalheight=6.2cm]{Sup_Sym2.png}}
\subfigure[\label{Figure_c}]{
\includegraphics[totalheight=6.2cm]{Sup_Costa_Sym.png}}
\subfigure[\label{Figure_d}]{
\includegraphics[totalheight=6.2cm]{Sup_Sym3.png}}
\caption{Computer graphics of the genus one minimal surfaces $S_x^1$ for: (a) $x=0$,  (b) $x=-\sfrac{1}{2}+e_1$,  (c) $x=e_1$ (Costa surface) and (d) $x=\sfrac{1}{2}+e_1.$}
\end{figure}


In the second part of the paper,  we were able to describe  a rich family of immersed, complete,  minimal surfaces $\Sigma_{k,x}$ that generalizes the family $S_x^1$ by increasing  the genus of the elements from $1$ to $k$.  These surfaces are also deformations of the Costa-Hoffman-Meeks genus $k$ surfaces,  obtained replacing the two catenoidal ends by Enneper type ends,  but preserving the symmetries.  Furthermore,  in this notation:

\begin{itemize}
\item $\Sigma_{1,1}$ is the Costa surface;
\item  $\Sigma_{1,x}$ are the genus one minimal surfaces $S_x^1$ of Theorem~\ref{TeoA};
\item the sub-family $\Sigma_{k,x}$ for $|x|=1$,  reduces to the family of Costa-Hoffman-Meeks  embedded minimal surfaces $M_k$.  
\end{itemize}

More especifically,  the main goal of the paper  is the following theorem 
\begin{thmy}
\textit{There exists a one-parameter family  $\Sigma_{k,x}$ of complete minimal surfaces in $\mathbb{R}^3$ of genus $k$,  with finite total curvature and three ends,  containing as sub-family $\Sigma_{k,\pm 1}$ the Costa-Hoffman-Meeks family  of embedded  surfaces $M_k$. Moreover,  if $|x| \neq 1$ and $|x| < \sqrt{2\sqrt{k+1}-1},$ then the immersed minimal surfaces $\Sigma_{k,x}$ have total curvature $ -4\pi(3k+2)$,  two Enneper-type ends and one middle planar end.}
\end{thmy}
We emphasize that the surfaces  $\Sigma_{k,x}$ have the same symmetry group as the Costa-Hoffman-Meeks surfaces and, then, they may be decomposed into $4(k+1)$ congruent pieces.  In the Figure~\ref{Fig-g-2} are given four genus two minimal surfaces of the family $\Sigma_{2,x}$,  constructed starting from a fondamental piece under the action of the twelve elements of the symmetry group of the surface.  In the pictures of the entire surface in Figure~\ref{Fig-g-2} the corresponding $12$ pieces are detached using different colors.
\begin{figure}[h!]
\subfigure[\label{Figure_a2}]{
\includegraphics[totalheight=7.2cm]{Sup_Sym0_g2.png}}
\subfigure[\label{Figure_b2}]{
\includegraphics[totalheight=6.7cm]{Sup_g2_0.8b.png}}
\subfigure[\label{Figure_c2}]{
\includegraphics[totalheight=5.8cm]{Sup_Costa_Hoffman_Meeks_g2.png}}
\subfigure[\label{Figure_d2}]{
\includegraphics[totalheight=5.2cm]{Sup_Sym1.2a.png}}
\caption{Computer graphics of the genus two minimal surfaces $\Sigma_{2,x}$ obtained for: (a) $x=0$, (b) $x=0.8$, (c) $x=1$ (Costa-Hoffman-Meeks surface) and (d) $x=1.2.$}
\label{Fig-g-2}
\end{figure}


\section{Basic theory and Enneper-Weierstrass representation}
This section is devoted to state some important results about elliptic functions and minimal surfaces,  largely used throughout this work. The reader can found the details in \cite {Chand.1985},  \cite{Costa.1984},  \cite{Fang.1990},  \cite{Osserman.2013} and \cite{vilhena.2021}.   We start reviewing one of the principal tools used to construct minimal surfaces in the Euclidean $3$-dimensional space,  in the formulation due to Osserman \cite{Osserman.2013}:

\begin{theorem}[Enneper-Weierstrass respresentation]\label{W} 
Let $\overline{M}$ be a compact Riemann surface and $M =\overline{M}-\{p_1, \cdots, p_n\}.$ Suppose that $\overline{g}: \overline{M} \rightarrow \mathbb{C}\cup \{\infty\} $ is a meromorphic function and $\eta$ is a meromorphic $1$-form such that whenever $g=\overline{g}|_M$ has a pole of order $k$, then $\eta$ has a zero of order $2k$ and $\eta$ has no other zeros on $M$. Let
\begin{equation}\label{WR}
\phi_1=\left(1-g^2 \right)\eta, \quad  \phi_2=i\left(1+g^2 \right)\eta, \quad \phi_3=2g\, \eta. 
\end{equation}
If for any closed curve $\alpha$ in $M$,
\begin{equation}\label{PP1}
\Re\int_{\alpha} \phi_j=0, \qquad  j =1,2,3,
\end{equation}
and every divergent curve $\gamma$ in $M$ has infinity length, i.e.,
\begin{equation}\label{PS0}
\int_{\gamma} (1+|g|^2)\,|\eta|=\infty,
\end{equation}
then the surface $S$, defined by $X: M \rightarrow \mathbb{R}^3$, is a complete regular minimal surface, where
\begin{equation}\label{PS}
X(z)=\Re\left(\int_{z_0}^z \phi_1, \int_{z_0}^z \phi_2, \int_{z_0}^z \phi_3\right).
\end{equation}
Here, $z_0$ is a fixed point of $M$. Moreover, 
the total curvature of $S$ is
\begin{equation}\label{CT}
C_T(S) = -4\pi\, \mathrm{deg}(\overline{g}).
\end{equation}
\end{theorem}
\begin{definition}
Given the lattice  
$\mathcal{L} = [1,i]$ in the complex plane $\mathbb{C}$,  the Weierstrass $\wp$ function (relative to $\mathcal{L} $) is the doubly periodic meromorphic function defined by 
\begin{equation}\label{PW}
\wp(z)=\frac{1}{z^2}+\sum_{\substack{\omega \in \mathcal{L}, \,\omega \neq 0}}\bigg\{\frac{1}{(z-\omega)^2}-\frac{1}{\omega^2} \bigg\}.
\end{equation}
The Weierstrass $\zeta$ function is defined by
\begin{equation}\label{ZW}
  \zeta(z)=\frac{1}{z}+\sum_{\substack{\omega \in \mathcal{L}, \,\omega \neq 0}}
  \left(\frac{1}{z-\omega}+\frac{1}{\omega}+\frac{z}{\omega^2}\right) 
\end{equation}
and 
\begin{equation*}\label{ZP}
\zeta'(z)=-\wp(z).
\end{equation*}
\end{definition}
Indicating with $F=\{ z \in \mathbb{C} \ | \ 0 \leqslant \Re(z)  < 1, \   0 \leqslant {\mathcal Im}(z) < 1\}$  the fundamental domain  it results that in $F$ the elliptic function $\wp(z)$ satisfies  the differential equation
\begin{equation}\label{EDPW}
	(\wp')^2= 4 \wp \left( \wp - e_1 \right) \left( \wp +e_1 \right),  
\end{equation}
where $e_1:=\wp(1/2)$.  Also,  by futher differentiation of \eqref{EDPW}, we obtain
\begin{equation}\label{EP2}
\wp^2=\frac{\wp''}{6}+\frac{e_1^2}{3}.
\end{equation}
From the Addition Theorem (see, for example,  \cite{Chand.1985}) it's easy to prove the following
\begin{proposition}\label{prop2}
Let $\mathcal{L}$ be a lattice and $z \in F$. Then, we have:
\begin{equation}\label{pe1}
\frac{2e_1^2}{\wp-e_1}=\wp(z-1/2)-e_1,
\qquad
\frac{2e_1^2}{\wp+e_1}=\wp(z-i/2)+e_1.
\end{equation}
Therefore, from \eqref{EP2},  we find that
\begin{equation}\label{pe1bis}
\frac{24 e_1^4}{(\wp-e_1)^2}=\wp''(z-1/2)-12e_1\wp(z-1/2)+8e_1^2,
\end{equation}
\begin{equation}\label{pe3bis}
\frac{24 e_1^4}{(\wp+e_1)^2}=\wp''(z-i/2)+12e_1\wp(z-i/2)+8e_1^2.
\end{equation}
\end{proposition}
Using the Legendre's relation we have 
\begin{proposition}[\cite{Costa.1984}]\label{prop6}
Let $\alpha_i: [0,1] \rightarrow \mathbb{C}$, $i=1,2$, be the paths
\begin{equation*}
\alpha_1(t) =\frac{i}{3}+t, \qquad \alpha_2(t)=\frac{1}{3}+it
\end{equation*}
of the homology basis of the torus $\displaystyle{T^2=\mathbb{C}/\mathcal{L}}.$ Then,
\begin{equation}\label{intA}
\int_{\alpha_1} \wp(z) \, dz = -\pi, \qquad \int_{\alpha_2} \wp(z) \, dz = i\pi.
\end{equation}
\end{proposition}

The symmetries of the minimal surfaces of genus $1$ that we will give in Section~\ref{sec3}  are  a consequence of the symmetries of the Weierstrass $\wp$ function in the fundamental domain $F$ (see Proposition~\ref{Prop3}) and of Proposition~\ref{Karcher.1989}.

\begin{proposition}[\cite{Hoffman.1985}]\label{Prop3}
Let $\wp(z)$ be the Weierstrass $\wp$-function for the unit-square lattice $\mathcal{L}$ and $w_2=(1+i)/2$,  then 
\begin{enumerate}
\item $\wp(\rho(w_2+z))=-\wp(w_2+z), \quad \rho(w_2+z)=w_2+iz,$
\item $\wp(\beta(w_2+z))=\overline{\wp(w_2+z)}, \quad \beta(w_2+z)=w_2+\overline{z},$
\item $\wp(\rho\circ\beta(w_2+z))=-\overline{\wp(w_2+z)},$
\item $\wp(\rho^2\circ\beta(w_2+z))=\overline{\wp(w_2+z)},$
\item $\wp(\mu(w_2+z))=-\overline{\wp(w_2+z)}, \ \ \mu(w_2+z)=w_2-i\overline{z}. $
\end{enumerate}
\end{proposition}

\begin{remark}
We point out that $\rho$, $\beta$, $\rho^2 \circ \beta$, $\rho \circ \beta$ and $\mu$ are,  respectively, a rotation by $\pi/2$ about $w_2$, a reflection about the horizontal line, a reflection about the vertical line, a reflection about the positive diagonal and a reflection about the negative diagonal through $w_2$.
\end{remark}

\begin{proposition}[\cite{Karcher.1989},  \cite{Wohlgemuth.1991}]\label{Karcher.1989}
Let $\zeta$ be a curve in $M$ such that $g \circ \zeta$ is contained either in a meridian or in the equator of $\mathbb{S}^2$, and $(g\eta)(\zeta')$ is real or imaginary. Then
$\zeta$ is a geodesic on $M$.
Moreover,  if $\zeta$ is a geodesic on $M$,  we have that
\begin{itemize}
\item [i)] $\zeta$ is a planar curve of symmetry provided that  $(dg \cdot\eta)(\zeta')\in\r$; 
\item [ii)] $\zeta$ is a straight line provided that  $(dg \cdot\eta)(\zeta')\in i\,\r$.
\end{itemize}
\end{proposition}

The following is another crucial result in the classical minimal surface theory.
\begin{theorem}[Schwarz Reflection Principle (see, for instance,  \cite{Fujimoto.2013, Karcher.1989})]
\noindent \begin{itemize}
\item [i)] If a minimal surface contains a segment of straight line $L$, then it is symmetric under
rotation by $\pi$ about $L$.
\item [ii)] If a nonplanar minimal surface contains a principal geodesic,  which is necessarily
a planar curve, then it is symmetric under reflection in the plane of that
curve.
\end{itemize}
\end{theorem}

\section{One-parameter family of minimal surfaces with genus one and three ends}\label{sec3}

In this section,  making use of the theory of elliptic functions,  we will provide a proof of the existence of a family of complete minimal surfaces immersed in $\mathbb{R}^3$ of finite total curvature and with three ends. We are going to prove the following result:
\begin{theorem}\label{teo3}
There exists a one-parameter family  $S_x$ of complete minimal surfaces immersed in $\mathbb{R}^3$ of genus one, finite total curvature and with three ends. Moreover, the family contains two sub-families with the following properties:
	\begin{enumerate}
	\item  The surfaces $S^1_x=S_{(x,y(x))}$, where
	$$
	y(x)=x, \qquad  |x| < \sqrt{\sqrt{8}-1}\,e_1,
	$$
which have as symmetry group the dihedral group $\mathcal{D}(4)$ of order $8$ generated by reflections in the $(x_1,x_3)$-plane and in the $(x_2,x_3)$-plane,  and by rotations of $\pi$ radians about the lines $x_1\pm x_2=x_3=0$.\\
Specifically,
\begin{itemize}
    \item [i)] If $|x|\neq e_1$,  then $S^1_x$ has total curvature  $-20 \pi$, two Enneper-type ends and one middle planar end;
    \item [ii)] If $|x| = e_1$,  then $S^1_x$ is precisely the Costa surface.
  \end{itemize}

   \item The surfaces $S^2_x=S_{(x,y(x))}$,  where
	$$
 y(x)=-5e_1^2/x, \qquad x\neq 0,
	$$
which have all symmetry group generated by two orthogonal vertical planes of reflectional symmetry.\\
Particularly,
\begin{itemize}
    \item [i)] If $|x|\neq e_1$ and $|x|\neq 5e_1$,  then $S^2_x$ has total curvature  $-20 \pi$, two Enneper-type ends and one middle planar end;
    \item [ii)] If $|x| = e_1$ or $|x| = 5e_1$,  then $S^2_x$ has total curvature  $-16 \pi$, one catenoid-type end, one Enneper-type end and one middle planar end.
  \end{itemize}
	\end{enumerate}

 \end{theorem}

\begin{proof}
Let $T^2=\mathbb{C}/\mathcal{L}$ be the torus with complex structure induced by the canonical projection $\pi: \mathbb{C} \rightarrow T^2.$ 
We consider $M=T^2 -\{p_1, p_2,p_3\}$,  where
\begin{equation}
p_1=\pi(1/2),\quad \ p_2=\pi(0), \quad \ p_3=\pi(i/2).
\end{equation}
The Weierstrass data  $(g, \eta)$ is given by

\begin{equation}\label{WR-WW}
\left\{
\begin{aligned}
	 g &= c \, \frac{\wp'}{\wp\, (\wp +x)\,(\wp-y)},  \qquad c >0,\\
	\mathbf \eta&=\frac{\wp\, (\wp +x)^2\,(\wp-y)^2}{(\wp-e_1)^2\,(\wp+e_1)^2} \, dz.
	\end{aligned}
	\right.
\end{equation}

The Figure~\ref{Fig1} below,  shows the zeros and poles of $g$, $\eta$ and $dh=g \eta$ on $F$ in the case $|x|\neq e_1$ and $|y|\neq e_1$.

\begin{figure}[h]\label{Fig1}
\subfigure[Zeros and poles of $g$.]{
\centering
\includegraphics[totalheight=3.5cm]{Fig_ZPg.eps}}
\subfigure[Zeros and poles of $\eta$.]{
\includegraphics[totalheight=3.5cm]{Fig_ZPeta.eps}}
\subfigure[Zeros and poles of $dh$.]{
\includegraphics[totalheight=3.5cm]{Fig_ZPdh.eps}}
\caption{}
\end{figure}

The degree of the Gauss map $g$ can be equal to $3$ (if $y=x=\pm e_1$), $4$ (if $|x|=e_1$ or  $|y| = e_1$) and 
$5$ (if $|x|\neq e_1$ and $|y|\neq e_1$).  It results that the total curvature
\begin{equation}\label{CT2}
C_T(S_x):=\int_M K \, dA = -4 \pi \,\text{deg}(g),
\end{equation}
can be equal to $-12\pi$, $-16\pi$ and $-20\pi$, respectively.\\

By using \eqref{WR}, \eqref{EDPW} and \eqref{WR-WW},  we obtain
$$
\left\{
\begin{aligned}
\phi_1&=\frac{\wp\, (\wp +x)^2\,(\wp-y)^2}{(\wp-e_1)^2\,(\wp+e_1)^2} \, dz - \frac{4c^2}{(\wp-e_1)(\wp+e_1)}\,dz,\\
\phi_2&=\frac{i\wp\, (\wp +x)^2\,(\wp-y)^2}{(\wp-e_1)^2\,(\wp+e_1)^2} \, dz +  \frac{4 i \,c^2}{(\wp-e_1)(\wp+e_1)}\,dz,\\ 
      \phi_3&=2c\,\frac{\wp'\, (\wp +x)\,(\wp-y)}{(\wp-e_1)^2\,(\wp+e_1)^2} \, dz.
\end{aligned}
\right.
$$
Therefore,  decomposing in partial fractions,  we can write
$$
\left\{
\begin{aligned}
\phi_1&=\Big[\wp+ 2(x-y)+\frac{(e_1-x)(e_1+y)(2e_1+y-x)}{2e_1 (\wp+e_1)}+\frac{(e_1+x)(e_1-y)(2e_1-y+x)}{2e_1 (\wp-e_1)}\\& -\frac{(e_1-x)^2(e_1+y)^2}{4e_1 (\wp+e_1)^2}+\frac{(e_1+x)^2(e_1-y)^2}{4e_1 (\wp-e_1)^2}\Big]\, dz - \frac{2c^2}{e_1}\Big(\frac{1}{\wp-e_1}-\frac{1}{\wp+e_1}\Big) \,dz,\\  
\phi_2&=i\Big[\wp+ 2(x-y)+\frac{(e_1-x)(e_1+y)(2e_1+y-x)}{2e_1 (\wp+e_1)}+\frac{(e_1+x)(e_1-y)(2e_1-y+x)}{2e_1 (\wp-e_1)}\\& -\frac{(e_1-x)^2(e_1+y)^2}{4e_1 (\wp+e_1)^2}+\frac{(e_1+x)^2(e_1-y)^2}{4e_1 (\wp-e_1)^2}\Big]\, dz + \frac{2i\,c^2}{e_1}\Big(\frac{1}{\wp-e_1}-\frac{1}{\wp+e_1}\Big) \,dz,\\  
\phi_3&=\frac{c}{2e_1^3}\Big[(e_1^2+xy)\, \Big(\frac{\wp'}{\wp-e_1}-\frac{\wp'}{\wp+e_1}\Big)
+\frac{e_1\,(e_1+x)(e_1-y)\wp'}{(\wp-e_1)^2}+\frac{e_1\,(e_1-x)(e_1+y)\wp'}{(\wp+e_1)^2}\Big]\, dz.
\end{aligned}
\right.
$$
Now,  using the formulas of Proposition~\ref{prop2}, we get
\begin{equation*}
\begin{aligned}
\int \phi_1&=-\zeta(z)+ 2(x-y)z+\frac{(e_1-x)(e_1+y)(2e_1+y-x)\,[e_1 z-\zeta(z-i/2)]}{4e_1^3}\\&-\frac{(e_1+x)(e_1-y)(2e_1-y+x)[e_1 z+\zeta(z+1/2)]}{4e_1^3}\\& -\frac{(e_1-x)^2(e_1+y)^2}{4e_1}\,\Big[\frac{\wp'(z-i/2)}{24e_1^4}-\frac{\zeta(z-i/2)}{2e_1^3}+\frac{z}{3e_1^2}\Big]\\& +\frac{(e_1+x)^2(e_1-y)^2}{4e_1}\,\Big[\frac{\wp'(z-1/2)}{24e_1^4}+\frac{\zeta(z-1/2)}{2e_1^3}+\frac{z}{3e_1^2}\Big] \\&-c^2 \,\frac{\zeta(z-i/2)-\zeta(z-1/2)-2e_1z}{e_1^3},\\
\int \phi_2&=i\,\Big[-\zeta(z)+ 2(x-y)z+\frac{(e_1-x)(e_1+y)(2e_1+y-x)\,[e_1 z-\zeta(z-i/2)]}{4e_1^3}\\&-\frac{(e_1+x)(e_1-y)(2e_1-y+x)[e_1 z+\zeta(z+1/2)]}{4e_1^3}\\& -\frac{(e_1-x)^2(e_1+y)^2}{4e_1}\,\Big(\frac{\wp'(z-i/2)}{24e_1^4}-\frac{\zeta(z-i/2)}{2e_1^3}+\frac{z}{3e_1^2}\Big)\\& +\frac{(e_1+x)^2(e_1-y)^2}{4e_1}\,\Big(\frac{\wp'(z-1/2)}{24e_1^4}+\frac{\zeta(z-1/2)}{2e_1^3}+\frac{z}{3e_1^2}\Big)\Big] \\&+i\, c^2\, \frac{\zeta(z-i/2)-\zeta(z-1/2)-2e_1z}{e_1^3},\\
\int \phi_3&=\frac{c}{2e_1^3}\Big[(e_1^2+xy)\ln \Big(\frac{\wp-e_1}{\wp+e_1}\Big)-\frac{e_1(e_1+x)(e_1-y)}{\wp-e_1}-\frac{e_1(e_1-x)(e_1+y)}{\wp+e_1}\Big].
\end{aligned}
\end{equation*}
As $\wp$ is periodic,  $\int\phi_3$ is also periodic and then 
$$\int_{\alpha_i} \phi_3=0, \qquad i=1,2.$$
From Proposition~\ref{prop6}, it follows that
\begin{equation*}
\Re\int_{\alpha_2} \phi_1= \Re\int_{\alpha_1} \phi_2 =0
\end{equation*}
and
\begin{equation}\label{P12}
\int_{\alpha_1} \phi_1=F_1+\frac{2c^2}{e_1^2},\qquad
\int_{\alpha_2} \phi_2=F_2+\frac{2c^2}{e_1^2},
\end{equation}
where 
$$\left\{
\begin{aligned}
F_1&=\frac{10\, e_1^4\,(x-y)-21\pi e_1^4-3\pi e_1^2\, (x^2+y^2)+2e_1^2\, xy\,(x-y)+12\pi e_1^2\, xy+3\pi x^2y^2}{12\, e_1^4},\\
F_2&=\frac{10\, e_1^4\,(y-x)-21\pi e_1^4-3\pi e_1^2\, (x^2+y^2)+2e_1^2\, xy\,(y-x)+12\pi e_1^2\, xy+3\pi x^2y^2}{12\, e_1^4}.
\end{aligned}
\right.$$
Therefore \eqref{PP1} holds if, and only if $F_1+2c^2/e_1^2=0=F_2+2c^2/e_1^2$, that is
\begin{equation*}
(y-x)\,(xy+5e_1^2)=0.
\end{equation*}
Also,  we have that
$$c(x,y)=\frac{e_1}{2}\sqrt{-(F_1+F_2)}=\frac{\sqrt{2\pi}}{4e_1}\sqrt{7e_1^4+(x^2+y^2-4xy)e_1^2-x^2y^2}.$$
Consequently, 
\begin{enumerate}
\item if $y=x$,  we get $$c(x)=\frac{\sqrt{2\pi}}{4e_1}\sqrt{8e_1^4-(x^2+e_1^2)^2}, \qquad x\in \Big(-\sqrt{\sqrt{8}-1}\,e_1,\sqrt{\sqrt{8}-1}\,e_1\Big);$$
\item if $y=-5e_1^2/x$,  we get $$c(x)=\frac{\sqrt{2\pi}}{4\,|x|}\sqrt{(x^2+e_1^2)^2+24e_1^4}, \qquad x\neq 0.$$
\end{enumerate}

The functions $\phi_1$, $\phi_2$ and $\phi_3$ have poles of order at least two at $p_1=\pi(1/2)$, $p_2=\pi(0)$ and $p_3=\pi(i/2)$. Hence, these functions do not have residues at $p_1, p_2$ and $p_3$, ensuring that the surfaces $S_x$ have no real periods around them. Therefore we obtain a one-parameter family of complete minimal surfaces $S_x$ containing two sub-families that will be indicated with $S^{n}_x$, $n=1,2$.  In the Figures~\ref{Fig-sup} and \ref{Sup_Cat_P_Enn} we present some pictures of minimal surfaces of the sub-families $S_x^1$ and $S_x^2$, respectively. 
\\

Finally,  in order to study the symmetries of the surfaces $S_x$,  we consider on $F$ the following curves:

\begin{equation}\label{eqcurva}
\begin{array}{lll}
&\zeta_1(u)=u, \ 0<u<\sfrac{1}{2}, \qquad &\zeta_2(u)=u, \ \sfrac{1}{2}<u<1,\\
&\zeta_3(u)=\frac{i}{2}+u, \ 0<u<1, \qquad &\zeta_4(u)=iu, \ 0<u<\sfrac{1}{2},\\
&\zeta_5(u)=iu, \ \sfrac{1}{2}<u<1, \qquad &\zeta_6(u)=\frac{1}{2}+iu, \ 0<u<1,\\
&\zeta_7(u)=u+i(1-u), \ 0<u<1, \qquad &\zeta_8(u)=u+iu, \ 0<u<1.
\end{array}
\end{equation}

Now one easily writes the expressions of the differential $dh=g \eta$ and $(dg\cdot \eta)$ as
$$\begin{aligned}
dh&=4c\,\sqrt{\frac{\wp}{(\wp^2-e_1^2)^3}}\, (\wp+x)(\wp-y)\, dz,\\
dg\cdot \eta&=\frac{2c\,[(x-y)\,(3\wp\,e_1^2-\wp^3)+\wp^2\,(5e_1^2-3\wp^2)-xy\, (\wp^2+e_1^2)]}{(\wp^2-e_1^2)^2}\,dz^2.
\end{aligned}$$
In the Table~\ref{table1} we sumarize the behaviour of $g$, $dh$ and   $dg\cdot \eta$ along the path $\zeta_j$, $j=1,\dots, 8$.
\begin{table}[h!]\caption{}
\renewcommand*{\arraystretch}{1.8}
\centering
\begin{tabular}{|c|c|c|c|c|}
 \hline
 \quad \text{Sub-family} $S^{n}_x$\quad  & \text{Path}\; $\zeta_j$ & $g \circ \zeta_j$  & $dh(\zeta_j')$ & $(dg\cdot \eta)(\zeta_j')$\\
 \hline
$n=1,2$ & $ j=1,2,3$  & $\r$    & $\r$ & $\r$ \\
 \hline
$n=1,2$ & $j=4,5,6$  & $\r$    & $i\r$& $\r$  \\
 \hline
 $n=1$ &  $j=7$ & $e^{\pm i\frac{\pi}{4}}\,\r$ & $i\r$ & $i\r$  \\ 
 \hline
$n=1$ & $j=8$ &$e^{\pm i\frac{\pi}{4}}\,\r$ &$\r$ &$i\r$  \\
 \hline
\end{tabular}
\label{table1}
\end{table}

From Proposition~\ref{Karcher.1989} and Table~\ref{table1}, we have that the curves
$\Gamma_j:=X\circ \zeta_j$, $j=1,\ldots,6$, are planar geodesics of $S_x$. We can easily show that $\Gamma_j$, $j=1,2,3,$ are contained in a plane 
parallel to the $(x_1,x_3)$-plane,  $\Gamma_j$, $j=4,5,6,$ are contained in a plane parallel to the $(x_2,x_3)$-plane and $\Gamma_j, \ j=7,8,$ are contained in the lines $x_1\pm x_2=x_3=0$ of $S^1_x$. The Schwarz Reflection Principle for minimal surfaces implies that the surfaces $S^1_x$ have the $(x_1, x_3)$-plane and the $(x_2, x_3)$-plane as reflective planes of symmetry and, also, they are invariant under rotations by $\pi$ about the lines $x_1\pm x_2=x_3=0$.  Furthermore, the surfaces $S^2_x$ have only the $(x_1, x_3)$-plane and the $(x_2, x_3)$-plane as reflective planes of symmetry. \\



Finally,  to determine the symmetry group of the surfaces $S^1_x$,  we introduce the dihedral group $\mathcal{D}(4)$ of order eight which is generated by $\beta$ and $\rho$. As the generators $\beta$ and $\rho$ can be identified with the orthogonal motions given by
$$
K=\left[
	\begin{array}{ccc}
	1&0&0\\
	0&-1&0\\
	0&0&1
	\end{array}
	\right], \qquad
	L_{\pi/2}=\left[
	\begin{array}{ccc}
	0&-1&0\\
	1&0&0\\
	0&0&-1
	\end{array}
	\right],
$$
	it results that the group $\mathcal{D}(4)=\langle L_{\pi/2}^j \, K^\ell \rangle$, $ j=0, \ldots, 3$,  $\ell=0,1$, acts on $\r^3$.
Now,  from \eqref{PS}, \eqref{WR-WW} and Proposition~\ref{Prop3},  we have on $S^1_x$ the following properties:
$$\begin{aligned}
X(\beta(w_2+z))& =( X_1, X_2,X_3)(\beta(w_2+z))=(X_1,-X_2,X_3)(w_2+z),\\
X(\rho(w_2+z))&=( X_1, X_2,X_3)(\rho(w_2+z))=( -X_2, X_1,-X_3)(w_2+z).
\end{aligned}
$$
Therefore,  as $X \circ \beta = K\,  X^t$ and $X \circ \rho =  L_{\pi/2} \, X^t$, we conclude that the symmetry group of $S^1_x$ is $\mathcal{D}(4)$ and this completes the proof.
\end{proof}

\begin{figure}[h!]
\subfigure[\label{Fig_a}]{
\includegraphics[totalheight=6cm]{S2xA_Fim_Plano_Cat_Enn.png}}
\subfigure[\label{Fig_b}]{
\includegraphics[totalheight=5.8cm]{S2xB_Fim_Plano_Cat_Enn.png}}
\caption{Computer graphics of the genus one minimal surfaces $S_x^2$ obtained for: (a)  $x=-e_1$ and (b)  $x=-e_1-0.5$.}\label{Sup_Cat_P_Enn}
\end{figure}


\begin{remark}
We observe that:
\begin{enumerate}
	\item The surfaces $S_x^1$ obtained for $y=x$,  with $|x|=\sqrt{\sqrt{8}-1}\, e_1$,  are doubly periodic minimal surfaces.
	\item The surfaces $S_x^2$ obtained by  $(x,y) \rightarrow (+\infty,0^-)$, $(x,y) \rightarrow (-\infty,0^+)$, $(x,y) \rightarrow (0^+,-\infty)$ or $(x,y) \rightarrow (0^-,+\infty)$ asymptotic to $y=-5e_1^2/x$ are given by $$(g,\eta)=\left(c \frac{\wp'}{\wp^2},\frac{\wp^3}{(\wp^2-e_1^2)^2}\,dz\right),\qquad c=\sqrt{2\pi}/4,$$ and are singular at zero.
\end{enumerate}
\end{remark}

\section{A family of high genus minimal surfaces with three ends}\label{Sec4}

The aim of this section is to find a one-parameter family  $\Sigma_{k,x}$ of complete minimal surfaces in $\mathbb{R}^3$ of genus $k$,  with finite total curvature and three ends, being one planar and two of Enneper-type, or one planar and two of catenoid-type.  Such family contains the one-parameter family of surfaces described in the Theorem~\ref{TeoA} and the Costa-Hoffman-Meeks embedded minimal surfaces $M_k$, of genus $k\geq 1$,  constructed in the Main Theorem of \cite{Hoffman.1990}.  The existence of this family is the main result of this paper,  summarized in the following theorem.


\begin{theorem}\label{Teo4}
For every integer $k \geq 1$, there exists a one-parameter family  $\Sigma_{k,x}$ of complete  
  minimal surfaces in $\mathbb{R}^3$ of genus $k$,  with finite total curvature and three ends.  Moreover such surfaces have the following properties:
  \begin{enumerate}
	\item If $|x| = 1$, then the minimal surfaces $\Sigma_{k,\pm 1}$ are precisely the Costa-Hoffman-Meeks embedded minimal surfaces  $M_k$.
	\item  If $|x| \neq 1$ and $|x| < \sqrt{2\sqrt{k+1}-1},$ then the immersed minimal surfaces $\Sigma_{k,x}$ have total curvature $C_T=-4\pi(3k+2)$,  two Enneper-type ends and one middle planar end.
	\item All the minimal surfaces $\Sigma_{k,x}$, $|x| < \sqrt{2\sqrt{k+1}-1},$ are symmetric by reflection about one of the $(k+1)$ vertical planes of the pencil through the $x_3$-axis and by rotation by $\pi$ radians about one of the $(k+1)$ straight lines on the surfaces in the $(x_1,x_2)$-plane.
	\item The symmetry group of $\Sigma_{k,x}$ is  the dihedral group $\mathcal{D}(2k+2)=\langle L_\theta^j \, K^\ell \rangle, \ j=0, \ldots , (2k+1)$,  $\ell=0,1$,  which has $4(k + 1)$ elements.
	\end{enumerate}
 \end{theorem}
\begin{proof}
For every integer $k \geq 1$,  we consider the closed Riemann surface of genus $k$ given by
\begin{equation}\label{RS}
 \overline{M}_k = \{(z,w) \in \mathbb{C}_\infty^2: w^{k+1} = z^k \, (z^2 - 1)\} .
\end{equation}
Let 
\begin{equation}
   \mathfrak{p}_0 = (0,0), \qquad \mathfrak{p}_{-1}=(-1,0), \qquad \mathfrak{p_1}=(1,0), \qquad \mathfrak{p}_{\infty}=(\infty, \infty)
\end{equation}
and we define $M_k=\overline{M}_k-\{\mathfrak{p}_{-1},\mathfrak{p}_1, \mathfrak{p}_{\infty}\}.$
 According to Enneper-Weierstrass Representation Theorem~\ref{W}, the Weierstrass data $(g,\eta)$ which we will use to construct the one-parameter family of minimal surfaces $\Sigma_{k,x}=X(M_k)$  is 
\begin{equation}\label{22}
\left\{
\begin{aligned}
	 g &=\Big( \frac{z^2-1}{z^2-x^2} \Big)\frac{c}{w},\\
	\mathbf \eta&=\frac{(z^2 - x^2)^2}{(z^2-1)^2} \left(\frac{z}{w}\right)^k dz,
	\end{aligned}
	\right.
\end{equation}
where $c\in\r$ is the unique positive constant for which we will prove that the immersion $X: M_k\to \r^3$ is well-defined on $M_k$.\\
 
In the Table~\ref{table2} we sumarize the behaviour of zeros and poles of $g$, $\eta$ and $dh$ at the saddles $\mathfrak{p}_{-x}$,  $\mathfrak{p}_{0}$,  $\mathfrak{p}_{x}$, $x \neq \pm 1$, and at the ends $\mathfrak{p}_{-1}, \mathfrak{p}_1$  and $\mathfrak{p}_{\infty}$.
\begin{table}[h!]\caption{}
\renewcommand*{\arraystretch}{1.3}
\centering
\renewcommand{\tabcolsep}{3.6 pt}
\begin{tabular}{|c|c|c|c|c|c|c|c|}
 \hline
 \quad  $(z,w)$ & $\mathfrak{p}_{-1}$ & $\mathfrak{p}_{-x}$  &  $\mathfrak{p}_{0}$ & $\mathfrak{p}_{x}$ & $\mathfrak{p}_{1}$ & $\mathfrak{p}_{\infty}$\\
 \hline
 $g$ & $0^k$ & $\infty^{k+1}$  & $\infty^k$ & $\infty^{k+1}$ & $0^k$ & $0^{k+2}$\\
 \hline
 $\eta$ & $\infty^{2k+2}$ & $0^{2k+2}$  & $0^{2k}$ & $0^{2k+2}$ & $\infty^{2k+2}$ & $\infty^{2}$\\
 \hline
  $dh$ & $\infty^{k+2}$ & $0^{k+1}$  & $0^k$ & $0^{k+1}$ & $\infty^{k+2}$ & $0^{k}$\\
 \hline
 Ends/saddle & Enneper end  &  regular point  & regular point & regular point  &  Enneper end & planar end \\
 \hline
\end{tabular}
\label{table2}
\end{table} 

In this case the degree de $g$ on $\overline{M}_k$ is $\mathrm{deg}(\overline{g})=3k+2$ and $C_T(S) = -4\pi\,(3k+2)$. From the Gackstatter \cite{Gackstatter.1976}, Jorge-Meeks \cite{Jorge.1983} formula given by
\begin{equation}\label{FJMG}
C_T(S)=2\pi \Bigg(2-2\mathbf{g}-N-\sum_{\nu=1}^N k_\nu\Bigg), 
\end{equation}
where $k_{\nu}$ is the order of the end,  it results that the order of the Enneper ends is $2k+1$ and the order of the flat end is $1$. Thus all the minimal surfaces $\Sigma_{k,x}$, $x \neq \pm 1$,  are not embedded. \\

If $x=\pm 1,$ then we recover the Costa-Hoffman-Meeks Weierstrass data
\begin{equation}\label{WR-HM}
\left\{
\begin{aligned}
	 g &= \frac{c}{w},\\
	\mathbf \eta&=\left(\frac{z}{w}\right)^k dz
	\end{aligned}
	\right.
\end{equation}
and from Table~\ref{table2} we also find the behaviour of zeros and poles of $g$, $\eta$ and $dh$ at $\mathfrak{p}_{-1}$, $\mathfrak{p}_{0}$, $\mathfrak{p}_1$  and $\mathfrak{p}_{\infty}$,  see Table~\ref{table3} below. 

\begin{table}[h!]\caption{}
\renewcommand*{\arraystretch}{1.3}
\centering
\begin{tabular}{|c|c|c|c|c|}
 \hline
 \quad  $(z,w)$ & $\mathfrak{p}_{-1}$  &  $\mathfrak{p}_{0}$ & $\mathfrak{p}_{1}$ & $\mathfrak{p}_{\infty}$\\
 \hline
 $g$ & $\infty$ & $\infty^{k}$  & $\infty$ & $0^{k+2}$\\
 \hline
 $\eta$ & * & $0^{2k}$  & * & $\infty^{2}$\\
 \hline
  $dh$ & $\infty$ & $0^{k}$ & $\infty$ & $0^{k}$\\
 \hline
 Ends/saddle & catenoidal end  & regular point & catenoidal end  & planar end \\
 \hline
\end{tabular}
\label{table3}
\end{table} 

Making use of \eqref{22} we may write
$$\left\{\begin{aligned}
\phi_1&=\eta-c^2\,\eta_2,\\
\phi_2&=i\,(\eta+c^2\,\eta_2), \\
\phi_3&=\frac{c}{2}\left(\frac{1-x^2}{(z-1)^2}+ \frac{1-x^2}{(z+1)^2}-\frac{1+x^2}{z+1}+\frac{1+x^2}{z-1} \right)\, dz,
\end{aligned}
\right.$$
where 
$$\eta=\frac{(z^2-x^2)^2}{(z^2-1)^3}w\, dz\qquad \text{and}\qquad \eta_2=\frac{dz}{w\, (z^2-1)}.$$
If we integrate $\phi_3$ and we get the real part,  we obtain
$$
\Re\int_{z_0}^z \phi_3=\frac{c (x^2+1)}{2} \ln\left|\frac{z-1}{z+1} \right|+c (x^2-1)\,\Re\left(\frac{z}{z^2-1}\right). 
$$
Thus,  we do not have period on $x_3$-axis.
Consequently, the period problem  $\Re\int_{\gamma}  (\phi_1,\phi_2,\phi_3)=0$ reduces to 
\begin{equation}\label{una}
\int_{\tilde{\gamma}_1^j} \eta=c^2\overline{\int_{\tilde{\gamma}_1^j} \eta_2}, \qquad
\int_{\tilde{\gamma}_2^j} \eta=c^2\overline{\int_{\tilde{\gamma}_2^j} \eta_2},
\end{equation}
where $\{\tilde{\gamma}_1^j,\tilde{\gamma}_2^j\},\ j=1,2, \ldots, k,$ is a homology basis $\{\tilde{\gamma}_1^j,\tilde{\gamma}_2^j\}$ of  $\overline{M}_k$.  This basis may be constructed as follows.  Let $\gamma_1^j$, $\gamma_2^j, \ j=1,2, \ldots, k,$ be  the oriented simple closed curves in the $z$-plane as in the Figure~\ref{BH}. 
The change from dotted line to continuous line indicates that the curve changes sheets on 
$\overline{M}_k$. Let $\tilde{\gamma}_1^j$ and $\tilde{\gamma}_2^j,\ j=1,2, \ldots, k,$ be  the unique lifts of $\gamma_1^j$ and $\gamma_2^j$ to $\overline{M}_k$, respectively. 
\begin{figure}[h!]
\includegraphics[totalheight=3cm]{z-plano.eps}
\caption{$\mathcal{F}$-sheets on $\overline{M}_k.$}\label{BH}
\end{figure} 



The curves $\gamma_1^j$ and $\gamma_2^j$ are homotopic to 
$\Gamma_1^j=\{z:|z|=\varepsilon\} \cup \overrightarrow{[0,1]} \cup \{z:|z-1|=\varepsilon\} \cup \overleftarrow{[0,1]}$ and 
$\Gamma_2^j=\{z:|z+1|=\varepsilon\} \cup \overrightarrow{[-1,0]} \cup \{z:|z|=\varepsilon\} \cup \overleftarrow{[-1,0]}$,  for $\varepsilon >0$ small, respectively.   Therefore, the computation  of the integrals in \eqref{una} on $\tilde{\gamma}_1^j$ gives
\begin{equation}\label{int_G1Et}
\int_{\Gamma_1^j} \eta= \int_{|z|=\varepsilon} \eta + \int_{\overrightarrow{[0,1]}} \eta +\int_{|z-1|=\varepsilon} \eta +
\int_{\overleftarrow{[0,1]}} \eta
\end{equation}
and
\begin{equation}\label{int_G1Et2}
\int_{\Gamma_1^j} \eta_2= \int_{|z|=\varepsilon} \eta_2 + \int_{\overrightarrow{[0,1]}} \eta_2 +\int_{|z-1|=\varepsilon} \eta_2 +
\int_{\overleftarrow{[0,1]}} \eta_2.
\end{equation} 
We note that the calculus of  the integrals in \eqref{una} on $\tilde{\gamma}_2^j$ follows similarly.  To avoid winding up with a divergent integral, Hoffman and Meeks added an exact one-form to $\eta_2$ (see p. 17 of \cite{Hoffman.1990}),  that is,
\begin{equation}\label{eta2}
\eta_2=-\frac{dz}{2w}-\frac{k+1}{2}\,\mathrm{d}\bigg(\frac{z}{w}\bigg).
\end{equation}
We also need add an exact one-form to $\eta$ to avoid winding up with a divergent integral. As $$\mathrm{d} w=\frac{w}{k+1}\bigg(\frac{k}{z}+\frac{2z}{z^2-1}\bigg)\,dz,$$ we obtain that
\begin{equation}\label{due}
\mathrm{d}\bigg(\frac{z w}{z^2-1}\bigg)=-\frac{2k}{(k+1)}\frac{z^2w\, dz}{(z^2-1)^2}+\frac{(2k+1)}{k+1}\frac{w\, dz}{(z^2-1)}
\end{equation}
and
\begin{equation}\label{F_eta3}
\mathrm{d}\bigg(\frac{z w}{(z^2-1)^2}\bigg)=-\frac{2 (2k+1)}{k+1}\eta_3-\frac{(2k+1)}{k+1}\frac{w\, dz}{(z^2-1)^2},
\end{equation}
where $$\eta_3:=\frac{w\, dz}{(z^2-1)^3}.$$
Also, we observe that \eqref{F_eta3} is equivalent to the following identity
\begin{equation}\label{F_eta3bis}\frac{w\, dz}{(z^2-1)^2}=-2\eta_3-\frac{k+1}{(2k+1)}\,\mathrm{d}\bigg(\frac{z w}{(z^2-1)^2}\bigg).
\end{equation}
Consequently,  from \eqref{due} and \eqref{F_eta3}, we get
$$\mathrm{d}\bigg(\frac{z w}{(z^2-1)^2}\bigg)-\frac{2k+1}{2k}\,\mathrm{d}\bigg(\frac{z w}{z^2-1}\bigg)=-\frac{2 (2k+1)}{k+1}\eta_3-\frac{(2k+1)}{2k (k+1)}\frac{w\, dz}{(z^2-1)}$$
and, thus,  we have the following expression 
\begin{equation}\label{eta3}
\eta_3=\frac{k+1}{4k}\,\mathrm{d}\bigg(\frac{z w}{z^2-1}\bigg)-\frac{w\, dz}{4k (z^2-1)}-\frac{k+1}{2 (2k+1)}\,\mathrm{d}\bigg(\frac{z w}{(z^2-1)^2}\bigg).
\end{equation}
Finally,  decomposing $\eta$ in partial fractions,  using \eqref{eta3} and \eqref{F_eta3bis} it results that
$$
\begin{aligned}
\eta&=\frac{w\, dz}{z^2-1}+2 (1-x^2)\frac{w\, dz}{(z^2-1)^2}+(1-x^2)^2\,\eta_3\\
&=\frac{w\, dz}{z^2-1}-\frac{2(k+1)(1-x^2)}{(2k+1)}\,\mathrm{d}\bigg(\frac{z w}{(z^2-1)^2}\bigg)-(1-x^2)(3+x^2)\,\eta_3
\end{aligned}
$$
and so
\begin{equation}\label{Eta_eta1}
\begin{aligned}
\eta&=\frac{(4k+3-2x^2-x^4)}{4k}\frac{w\, dz}{z^2-1}-\frac{(k+1)(1-x^2)^2}{2(2k+1)}\,\mathrm{d}\bigg(\frac{z w}{(z^2-1)^2}\bigg)+\\&-\frac{(k+1)(1-x^2)(3+x^2)}{4k}\,\mathrm{d}\bigg(\frac{z w}{z^2-1}\bigg).
\end{aligned}
\end{equation}
Define $\omega$ to be the branch of $w = \left[z^k \, (z^2 - 1)\right]^{1/(k+1)}$ defined on $\mathbb{C} - \{(-\infty,-1]\cup [0,1] \}$ that satisfies 
$$\lim_{\varepsilon \rightarrow 0^+} \arg\Big(\omega(\frac{1}{2}-i\,\varepsilon )\Big)=-\frac{\pi i}{k+1}.$$ Hence, from \eqref{Eta_eta1} and using the computations on p.  17 of \cite{Hoffman.1990},  we conclude that the integral in the left-hand side of \eqref{una} is equal to
$$\int_{\Gamma_1^j} \eta=\frac{(4k+3-2x^2-x^4)}{4k}\int_{\Gamma_1^j} \frac{w\, dz}{z^2-1}=c_1\, A\,\frac{(4k+3-2x^2-x^4)}{4k},$$
where $$c_1=e^{-\pi i/(k+1)}-e^{\pi i/(k+1)}\qquad \text{and}\qquad A=\int_0^1\frac{[t^k\, (1-t^2)]^{1/(k+1)}}{t^2-1}\,dt<0.$$
Moreover,  as
$$\int_{\Gamma_1^j} \eta_2=\overline{c_1}\, B,\qquad B=-\frac{1}{2}\int_0^1[t^k\, (1-t^2)]^{-1/(k+1)}\,dt<0,$$
substituting in \eqref{una},  we conclude that we can choose 
\begin{equation}\label{const_c}
c(k,x)=\sqrt{\frac{4k+3-2x^2-x^4}{4k}}\, \sqrt{\frac{A}{B}}, \qquad |x|<\sqrt{2\sqrt{k+1}-1}.
\end{equation}

To describe the simmetries of $\Sigma_{k,x}$, we start determining the expressions of the differential $dh=g\cdot\eta$ and of $(dg\cdot \eta)$ as
$$\begin{aligned}
dh&=c\, \frac{(z^2 - x^2)}{(z^2-1)^2}\, dz,\\
dg\cdot \eta&=c\,\Bigg[\frac{2(1-x^2)\, z}{(z^2-1)^3}-\Big(\frac{k}{k+1}\Big)\, \frac{(z^2-x^2)}{z \,(z^2-1)^2}-\Big(\frac{2}{k+1}\Big)\, \frac{(z^2-x^2)\,z}{(z^2-1)^3}\Bigg]\,dz^2.
\end{aligned}$$ 
In the Table~\ref{table4} we sumarize the behaviour of $g$, $dh$ and  $dg\cdot \eta$ along the path $\sigma_j$, $j=1,2,3$.
\begin{table}[h!]\caption{}
\renewcommand*{\arraystretch}{1.7}
\centering
\begin{tabular}{|c|c|c|c|}
 \hline
 \quad \text{Path}\; $\sigma_j$ & \quad $g \circ \sigma_j$  \quad & \quad $dh(\sigma_j')$\quad  & \quad $(dg\cdot \eta)(\sigma_j')$ \quad \\
\hline
 $\sigma_1(t)=t, \ 1< t <\infty$  & $ \r$ & $\r$ & $\r$  \\ 
\hline 
 $\sigma_2(t)=t, \ 0< t <1$  & $e^{-\frac{\pi i}{(k+1)}}\,\r$ & $\r$ & $\r$  \\  
 \hline 
$\sigma_3(t)=it, \ 0< t <\infty$ & $e^{-\frac{(k+2)\pi i}{2 (k+1)}}\,\r$ & $i \r$ & $i\r$  \\ 
 \hline
\end{tabular}
\label{table4}
\end{table}

Now we will study the real part of the period vector of non-trivial closed curves around each puncture of $M_k$, i.e, non-trivial closed curves around $\mathfrak{p}_{-1}$,  $\mathfrak{p}_1$ and $\mathfrak{p}_{\infty}.$ According to the Table~\ref{table4} and 
Proposition~\ref{Karcher.1989},  we have that the curves $X \circ \sigma_j , j = 1,2$,  are planar geodesics  and  $X \circ \sigma_3$ is a straight line on the fundamental piece of $\Sigma_{k,x}.$ Thus,  we have that the planes $P_1:= (x_1, x_3)$-plane and $P_2= -L_{\theta} \cdot P_1$ are reflective planes of symmetry containing the curves $X \circ \sigma_j , j = 1,2$,  respectively.  

Therefore,  a small curve $\sigma$ on $M_k$ around $\mathfrak{p}_{-1}$ is, after homology, invariant under reflections in $\sigma_j , j = 1,2$.  By virtue of this,  the period vector 
$\Re \int_{\sigma}(\phi_1, \phi_2, \phi_3)$ is perpendicular to the planes $P_1$ and $P_2$. But these planes are not parallel, because $P_1$ and $P_2$ make an angle of $\theta=\pi/(k+1)$.  Since the period vector $\Re \int_{\sigma} (\phi_1, \phi_2, \phi_3)$ must be perpendicular to both planes (see Lemma~$2$ in \cite{Wohlgemuth.1997}), we conclude that it is zero. We can apply the same argument as above to conclude that the period vector will be also zero in the other cases. \\

Let $\kappa(z,w)=(\overline{z},\overline{w})$ and $\lambda(z,w)=\left(-z, e^{i\,k\,\theta} w\right)$,  with $\theta = \pi/(k+1)$,  be the conformal mappings of $\mathbb{C}_\infty^2$, as defined in \cite{Hoffman.1990} (see p. 13).  Due to the symmetries of $M_k$ (see Corollary 3.2 in \cite{Hoffman.1990}) the group generated by $\kappa$ and $\lambda$ is the dihedral group $\mathcal{D}(2k+2)$ with $4(k+1)$ elements. By a straightforward calculation, we deduce 

$$
\left\{
\begin{aligned}
\lambda^{*}\phi_1&=\cos \theta \, \phi_1 - \sin \theta \, \phi_2,   \\
\lambda^{*}\phi_2&=\sin \theta \, \phi_1 + \cos \theta \, \phi_2,  \\
\lambda^{*}\phi_3&= - \phi_3,
\end{aligned}
\right.
\qquad \qquad 
\left\{
\begin{aligned}
\kappa^{*}\phi_1&= \overline{\phi_1}, \\
\kappa^{*}\phi_2&= - \overline{\phi_2},  \\
\kappa^{*}\phi_3&=  \overline{\phi_3}.
\end{aligned}
\right.
$$

Thus, $X \circ \lambda = L_\theta\,  X^t $ and $X \circ \kappa = K\,  X^t,$ where $ L_\theta$ and $K$ are the real orthogonal matrices defined in \eqref{matrix}.
Therefore, the group $\mathcal{D}(2k+2)=\langle L_\theta^j\, K^\ell \rangle, \ j=0, \ldots , (2k+1)$,  $\ell=0,1$,  acts by isometries on $\Sigma_{k,x}$,  which can be decomposed in $4(k+1)$ congruent pieces; see $X(\Omega)$ in the Figure~\ref{FR}.  

\begin{figure}[h!]
\includegraphics[totalheight=6.7cm]{Fund_Region.png}
\caption{Fundamental piece.}
\label{FR}
\end{figure}

Then,  we may conclude that the vertical planes, containing   the $x_3$-axis,  given by $P_{j+1}= -L_\theta\cdot P_j$,  $j=1,\dots, k$,  are planes of symmetry for the surfaces $\Sigma_{k,x}$ which have, also, $(k+1)$ straight lines of symmetry.  The regularity and the completeness of $\Sigma_{k,x}$ are easy to see.  This completes the proof of the theorem.
\end{proof}

\begin{remark}
The surfaces $\Sigma_{1,x}$ are the same minimal surfaces $S_x^1$ given in Theorem~\ref{teo3} and,  in particular,  $\Sigma_{1,1}$ is the Costa surface.
\end{remark}

\begin{corollary}\label{Cor_Lim}
Let $\Sigma_{k,x}$ be the family of minimal surfaces exhibited in Theorem~\ref{Teo4}.  
\begin{enumerate}
	\item If $k \rightarrow \infty$ and $|x| = 1$, then $\Sigma_{k,x}$ converges smoothly to Scherk's Fifth surface.
   \item If $k \rightarrow \infty$ and $|x| \neq  1$, then $\Sigma_{k,x}$ converges smoothly to Scherk-Enneper  surface.
\end{enumerate}
\end{corollary}
\begin{proof}
It is a straighforward computation, making use of  \eqref{22}  and  \eqref{const_c},  to show that
$$c(k,x)=2^{\frac{1}{k+1}}\sqrt{\frac{4k+3 - 2x^2-x^4}{4k}}\;\frac{\Gamma\left(\frac{k + 2}{2k + 2}\right)}{\Gamma\left(\frac{2k+3}{2k + 2}\right)}\,\sqrt{\frac{1}{2k+2}\cot\Big(\frac{\pi}{2k+2}\Big)},$$
where $\Gamma$ is the function Gamma,
$$\lim_{k \rightarrow \infty } w(z) =z, \qquad \lim_{k \rightarrow \infty } c(k,x) =1, \qquad \lim_{k \rightarrow \infty } g(z) = \frac{z^2-1}{z(z^2-x^2)}, \qquad \lim_{k \rightarrow \infty } \eta(z) = \frac{z(z^2-x^2)^2}{(z^2-1)^3}.$$


Thus, if $|x|=1$ then we have the  Weierstrass data of the Scherk's Fifth surface (see Figure~\ref{Figure_b1})
$$(g,\eta) = \left(\frac{1}{z},\frac{z}{z^2-1}\, dz \right).$$
Now, if $|x|\neq 1$, then we obtain the family of the  Scherk-Enneper surfaces (see Figure~\ref{Figure_a1}).  
\end{proof}

\begin{figure}[h]
\subfigure[\label{Figure_a1}]{
\includegraphics[totalheight=4.4cm]{Enneper_Scherk1.png}}
\subfigure[\label{Figure_b1}]{
\includegraphics[totalheight=3cm]{Scherk_Fifth.png}}
\caption{Computer graphics of $\Sigma_{\infty,x}$ for: (a) $x=0.8$ (Scherk-Enneper surface) and (b) $x=1$ (Scherk's Fifth surface).}
\label{Fig_Sup_Lim}
\end{figure}


\begin{remark}
We point out that:
\begin{enumerate}
\item A calculation shows that  $$(g,\eta) = \Big(\frac{1}{z},\frac{z}{z^2-1}\, dz \Big)$$ produces the same minimal
surface as $\cos y+\sinh x\,  \sinh z=0$.
\item In \cite{Hoffman_Meeks.1990},  D.  Hoffman and W.H. Meeks did a thorough study of limits to $\Sigma_{k,1}=M_k$. In this paper they proved the first item of 
Corollary~\ref{Cor_Lim} and also that $M_k$ converge to the union of the plane and the catenoid.
\end{enumerate}
\end{remark}


\bibliographystyle{amsplain} 
\bibliography{Bib_Minimal_Surfaces}


\end{document}
