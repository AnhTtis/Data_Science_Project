\subsubsection*{Destroying stars and diamonds}
Let $\mathcal{G}$ be the class of $\{K_{1,t}, \text{diamond}\}$-free graphs, for any fixed $t\geq 3$. We give a polynomial-time algorithm for \SCG.
The concept of $(p, q)$-split graphs was introduced by Gy\'{a}rf\'{a}s~\cite{DBLP:journals/jct/Gyarfas98}.
For $p\geq 1$, and  $q\geq 1$, if the vertices of a graph $G$ can be partitioned into two sets
$P$ and $Q$ in such a way that
the clique number of $G[P]$ and the independence number of $G[Q]$
are at most $p$ and $q$ respectively (i.e., $G[P]$ is $K_{p+1}$-free and $G[Q]$ is $(q+1)K_1$-free), 
then $G$ is called a $(p,q)$-split graph and $(P,Q)$ is a $(p,q)$-split partition of $G$.

\begin{proposition}[\cite{DBLP:conf/fsttcs/KolayP15, kolay2015parameterized, DBLP:journals/algorithmica/AntonyGPSSS22}]
\label{pro:split-part}
 For any fixed constants $p\geq 1$ and $q\geq 1$, 
 recognizing a $(p,q)$-split graph and obtaining all $(p,q)$-split partitions of a $(p,q)$-split graph 
 can be done in polynomial-time.
\end{proposition}

\begin{mdframed}
\textbf{Algorithm for \SCG}, where $\mathcal{G}$ is $\{K_{1,t},diamond\}$-free graphs, for any constant $t\geq 3$.\\
Input: A graph $G$.\\
Output: If $G$ is a yes-instance of \SCG, 
then returns \YES; otherwise returns \NO.
\begin{description}

\item [Step 1]: Let $S$ be the set of all degree-2 vertices of all the induced diamonds in $G$. If $G\oplus S\in \mathcal{G}$, then return \YES. 
\item [Step 2]: Let $r$ be the center of any induced 
$K_{1,t}$ in $G$ and let $I$ be the set of isolated
vertices in the subgraph induced by $N(r)$ in $G$.
For every subset $S\subseteq I$ such that $|S|\geq |I|-t+2$, if $G\oplus S\in \mathcal{G}$, then return \YES.
\item[Step 3]: For every edge $uv$ in $G$, do the following:
\begin{enumerate}
    \item If $N(u)\setminus N[v]$ or $N(v)\setminus N[u]$ does not induce a 
    $(t-1,t-1)$-split graph, then continue with Step 3.
    \item Compute $L(u\overline{v})$, the list of all $(t-1,t-1)$-split partitions of the graph induced by $N(u)\setminus N[v]$.
    \item Compute $L(\overline{u}v)$, the list of all $(t-1,t-1)$-split partitions of the graph induced by $N(v)\setminus N[u]$.
    \item Compute $L(uv)$, the list of all partitions of 
    the graph induced by $N(u)\cap N(v)$ into an independent set of size at most $t-1$ and the rest.
    \item For every $(S_1,T_1)\in L(u\overline{v})$, 
    for every $(S_2,T_2)\in L(\overline{u}v)$, for 
    every $(S_3,T_3)\in L(uv)$, do the following:
    \begin{enumerate}
        \item Let $S=S_1\cup S_2\cup S_3\cup \{u,v\}$.
        If $G\oplus S\in \mathcal{G}$, return \YES.
        \item For every vertex $w\in \overline{N[u]}\cap \overline{N[v]}$,
        let $S= S_1\cup S_2\cup S_3\cup \{u,v,w\}$. If $G\oplus S\in \mathcal{G}$, return \YES.
        \item For every edge $xy$ in the graph induced 
        by $\overline{N[u]}\cap \overline{N[v]}$, if the graph induced by
        $J=N[x]\cap N[y]\cap \overline{N[u]}\cap \overline{N[v]}$ is not a split graph then 
        continue with the current step. Otherwise,
        for every split partition $(S_4,T_4)$ of the graph
        induced by $J$, let $S= S_1\cup S_2\cup S_3\cup S_4\cup \{u,v\}$. If $G\oplus S\in \mathcal{G}$, then return \YES. 
    \end{enumerate}
\end{enumerate}
\item[Step 4]: Return \NO.
\end{description}
\end{mdframed}
\vspace{0.1cm}
Lemma~\ref{lem:is-diamond} and \ref{lem:is-k1t} deals with the case when $G$ is a yes-instance having a solution which is an independent set, the case handled in Step 1 and 2 of the algorithm.
\begin{lemma}
    \label{lem:is-diamond}
    Assume that $G$ is not diamond-free.
    Let $S\subseteq V(G)$ such that $G\oplus S\in \mathcal{G}$ and $S$ is an independent set. Then $S$
    is the set of all degree-2 vertices of all the induced diamonds in $G$.
\end{lemma}
\begin{proof}
    Since $S$ is an independent set and $G\oplus S\in \mathcal{G}$, both the degree-2 vertices of every induced diamond in $G$ must be in $S$. Assume 
    for a contradiction that $S$ has a vertex $v$
    which is not a degree-2 vertex of any of the induced diamonds in $G$. Let $D=\{d_1,d_2,d_3,d_4\}$
    induces a diamond in $G$, where $d_1$ and $d_2$
    are the degree-2 vertices of the diamond. 
    Clearly, $S\cap D=\{d_1,d_2\}$. We know that $v\neq d_1$ and $v\neq d_2$. If $v$ is not adjacent to $d_3$ in $G$, then $\{v,d_1,d_2,d_3\}$ induces 
    a diamond in $G\oplus S$, which is a contradiction.
    Therefore, $v$ is adjacent to $d_3$. 
    Similarly, $v$ is adjacent to $d_4$. 
   Then $\{v,d_1,d_3,d_4\}$ induced a diamond in $G$,
    where $v$ and $d_1$ are the degree-2 vertices, which is a contradiction.
\end{proof}

\begin{lemma}
    \label{lem:is-k1t}
    Assume that $G$ has no induced diamond but has 
    at least one induced $K_{1,t}$. Let $S\subseteq V(G)$ such that $G\oplus S\in \mathcal{G}$
    and $S$ is an independent set. Let $r$
    be the center of any induced $K_{1,t}$ in $G$.
    Let $I$ be the set of isolated vertices in the
    subgraph induced by $N(r)$ in $G$. Then $S\subseteq I$ and $|S|\geq |I|-t+2$.
\end{lemma}
\begin{proof}
    If $r\in S$, then none of the vertices in $N(r)$
    is in $S$ - recall that $S$ is an independent set.
    But then, none of the induced $K_{1,t}$ centered at
    $r$ is destroyed in $G\oplus S$.
    Therefore, $r\notin S$.
    Since $G$ is diamond-free, $N(r)$
    induces a cluster (graph with no induced path of length 3) $J$ in $G$.
    Since $r$ is the center 
    of an induced $K_{1,t}$ in $G$, there are at 
    least $t$ cliques in $J$.
    Since $G\oplus S$ is $K_{1,t}$-free,
    $S$ must contain all vertices of at least two
    cliques in $J$. Since $S$ is an independent set,
    $S$ contains at least two isolated vertices, say $s_1$ and $s_2$, in $J$. First we prove that
    $S\subseteq N(r)$. 
    For a contradiction, assume that there is a vertex
    $v\in S$ such that $v$ is not adjacent to $r$. 
    Then $\{v,s_1,s_2,r\}$ induces a diamond
    in $G\oplus S$, which is a contradiction.
    Therefore, $S\subseteq N(r)$. Next we prove that
    $S\subseteq I$. For a contradiction, assume that
    there is a vertex $v\in S\setminus I$. Then $v$
    is part of a clique $J'$ of size at least 2 in $J$.
    Let $v'$ be any other vertex in $J'$. Since 
    $S$ is an independent set, $v'\notin S$. 
    Then $\{v,v',s_1,r\}$ induces a diamond in $G\oplus S$, which is a contradiction. Therefore, $S\subseteq I$. If $|S|<|I|-t+2$, then there is 
    a $K_{1,t}$ centered at $r$ in $G\oplus S$, which 
    is a contradiction.
\end{proof}

Let $G$ be a yes-instance of \SCG.
Let $S\subseteq V(G)$ be such that $|S|\geq 2$, $G\oplus S\in \mathcal{G}$, and $S$ be not an independent set.
Let $u$, and $v$ be two  adjacent vertices in $S$. 
 Then with respect to $S, u, v$, 
 we can partition the vertices in $V(G)\setminus \{u,v\}$ into eight sets as given below, 
 and shown in Figure~\ref{disect}.
\begin{multicols}{2}
\begin{enumerate}[label=(\roman*)]%[(i)]
	\item $N_{S}(uv)=S\cap N(u) \cap N(v)$
	\item $N_{S}{(\bar{u}\bar{v})} = S\cap \overline {N[u]} \cap \overline{N[v]}$ 
	\item $N_S{({u}\bar{v})}=S\cap (N(u)\setminus N[v])$ 
	\item $N_S(\bar{u}{v})=S\cap (N(v)\setminus N[u])$
	\item $N_T(uv)=(N(u)\cap N(v))\setminus S$
	\item $N_T(\bar{u}\bar{v})=(\overline{N[u]}\cap\overline{N[v]})\setminus S$
	\item $N_T({u}\bar{v})=(N(u)\setminus N[v])\setminus S$
	\item $N_T(\bar{u}{v})=(N(v)\setminus N[u])\setminus S$
\end{enumerate}
\end{multicols}
  We notice that $S=N_{S}(uv)\cup N_{S}{(\bar{u}\bar{v})}\cup N_S(u\bar{v})\cup N_S(\bar{u}v)\cup \{u,v\}$.
 

\begin{figure}[ht]
  \centering
    \centering
    \begin{tikzpicture}[myv/.style={ellipse, draw, inner xsep=60pt,inner ysep=25pt}, myv1/.style={ellipse, draw, inner xsep=90pt,inner ysep=40pt},myv2/.style={circle,color=white, draw,inner sep=0.5pt}, myv3/.style={ellipse, draw, inner sep=1.5pt},myv4/.style={circle, draw, inner sep=1.5pt}, myv5/.style={ellipse, draw, inner sep=1.5pt}]


  
\node[myv2] (a)[label= right:{$N_{S}{(\bar{u}\bar{v})}$}] at (0.3,-0.8) {}; 
\node[myv2] (b)[label= right:{$N_{S}(uv)$}] at (0.3,0.9) {}; 
\node[myv2] (c)[label= below:{$N_S{({u}\bar{v})}$}] at (-1,0) {};
\node[myv2] (d)[label= below:{$N_S(\bar{u}{v})$}] at (2.9,0) {};
\node[myv2] (e)[label= right:{$N_T({u}\bar{v})$}] at (-3.6,0) {};
\node[myv] (f) [label= right:{$N_T(\bar{u}{v})$}] at (1,0) {};

\node[myv2] (h)[label= right:{$N_T(uv)$}] at (0.3,1.6) {};
\node[myv2] (j)[label= right:{$N_T(\bar{u}\bar{v})$}] at (0.3,-1.6) {};

   \node[myv1] (G)[label= right:{}] at (1,0) {};
  \node[myv4] (u) at (0.25,0) {$u$};
  \node[myv4] (v) at (1.75,0) {$v$};
 \node[myv5][fit= (u) (v),  inner xsep=0.25ex, inner ysep=0.25ex] {}; ] {}; 


 

  %[line width=0.5mm]
 
% \draw  (-0.2,1.9) -- (0.7,0.35);
 % \draw  (2.4,1.85) -- (1.5,0.35);
  %\draw (0.8,-0.4) -- (0,-1.95);
 % \draw (1.5,-0.35) -- (2.4,-1.9);
 
   \draw (0.45,0) -- (1.55,0);
  \draw (-0.7,1.85) -- (0.7,0.35);
  \draw (2.7,1.85) -- (1.3,0.35);
  \draw (0.7,-0.35) -- (-0.7,-1.85);
  \draw (1.3,-0.35) -- (2.7,-1.85);
  \draw (u)[line width=0.5mm] -- (0.95,0.6);
  \draw (u) [line width=0.5mm]-- (-2,1);
  \draw (u) [line width=0.5mm]-- (-0.5,0.8);
  \draw (u) [line width=0.5mm]-- (1,1.5);
  %\draw (u) --  (e);
  \draw (v)[line width=0.5mm] -- (0.95,0.6);
  \draw (v) [line width=0.5mm]-- (4,1);
  \draw (v) [line width=0.5mm]-- (2.5,0.8);
   \draw (v) [line width=0.5mm]-- (1,1.5);
 
  

\end{tikzpicture}


    \caption{Partitioning of vertices of $G$  based on $S$ and two adjacent vertices $u,v \in S$. 
    The bold lines represent the adjacency of  vertices $u$ and $v$~\cite{DBLP:journals/algorithmica/AntonyGPSSS22}.} 
    \label{disect}
\end{figure}

\begin{observation}
\label{ob: not an IS}
Then the following statements are true.
\begin{enumerate}[label=(\roman*)]
    \item $N(u)\setminus N[v]$ induces a $(t-1, t-1)$-split graph 
    with a $(t-1,t-1)$-split partition of ($N_{S}(u\overline{v}), N_{T}(u\overline{v})$).

    \item $N(v)\setminus N[u]$ induces a $(t-1, t-1)$-split graph 
    with a $(t-1,t-1)$-split partition of ($N_{S}(v\overline{u}), N_{T}(v\overline{u})$).

    \item $N_{T}(uv)$ induces an independent set with at most $(t-1)$ vertices.

    \item $N_{S}(\bar{u} \bar{v})$ induces a clique. 
    If $xy$ is an edge of the clique, 
    then $N[x]\cap N[y]$ in $\overline{N[u]} \cap \overline{N[v]}$
    induces a split graph with one split partition being $(N_S(\bar{u}\bar{v}), (N[x]\cap N[y] \cap  \overline{N[u]}\cap \overline{N[v]})\setminus (N_S(\bar{u}\bar{v})))$. 
    
\end{enumerate}
\end{observation}
\begin{proof}
    If $N_S(u\overline{v})$ has a $K_t$, then 
    $v$ along with the vertices of the $K_t$ induce a 
    $K_{1,t}$ in $G\oplus S$. 
    If $N_T(u\overline{v})$ has an independent set of 
    size $t$, then $u$ along with the vertices of the 
    independent set induce a $K_{1,t}$ in $G\oplus S$.
    Therefore, (i) holds true.
    Similarly we can prove the correctness of (ii).
    If there are two adjacent vertices $x$ and $y$ in
    $N_T(uv)$, then $\{x,y,u,v\}$ induces a diamond in
    $G\oplus S$. Therefore, $N_T(uv)$
    is an independent set. If it has at least $t$
    vertices then there is an induced $K_{1,t}$
    formed by those vertices and $u$ in $G\oplus S$.
    Therefore, (iii) holds true.
    If there are two nonadjacent vertices $x$ and $y$
    in $N_S(\bar{u}\bar{v})$,
    then there is a diamond induced by $\{x,y,u,v\}$
    in $G\oplus S$. Therefore, $N_S(\bar{u}\bar{v})$ is a clique. 
    Assume that $x,y\in N_S(\bar{u}\bar{v})$. 
    If $x$ and $y$ have two adjacent common neighbors $x'$ and $y'$ in $N_T(\bar{u}\bar{v})$, then 
    $\{x,y,x',y'\}$ induces a diamond in $G\oplus S$.
    Therefore, $N[x]\cap N[y]\cap \overline{N[u]}\cap \overline{N[v]}$ is a split graph with one split
    partition being $(N_S(\bar{u}\bar{v}), (N[x]\cap N[y] \cap  \overline{N[u]}\cap \overline{N[v]})\setminus (N_S(\bar{u}\bar{v})))$.
    \end{proof}

\begin{lemma}
\label{lem:diamondk1t}
$G$ is a yes-instance of \SCG\ if and only if 
the algorithm returns \YES.
\end{lemma}
\begin{proof}
    Since the algorithm returns \YES\ only when a solution is found, the backward direction of the statement is true. For the forward direction,
    let $G$ be a yes-instance. Assume that 
    there exists a solution $S$ which is an independent set. Further, assume that $G$ has an induced diamond. Then by Lemma~\ref{lem:is-diamond},
    $S$ is the set of all degree-2 vertices of the induced diamonds in $G$. Then Step 1 returns \YES.
    Assume that $G$ is diamond-free. Then by Lemma~\ref{lem:is-k1t}, $S\subseteq I$, where 
    $I$ is the set of isolated vertices in the graph
    induced by the neighbors of $r$, for a center $r$ of an induced 
    $K_{1,t}$ in $G$. Further $|S|\geq |I|-t+2$.
    Then Step 2 returns \YES. 
    Let $S$ be a solution which is not an independent set. Let $uv$ be an edge in the graph induced by $S$.
    The algorithm will discover $uv$ in one iteration of Step 3. 
    By Observation~\ref{ob: not an IS}, we know that 
    the graph induced by
    $N(u)\setminus N[v]$ is a $(t-1,t-1)$-split graph
    with a $(t-1,t-1)$-split partition $(N_S(u\overline{v}), N_T(u\overline{v}))$.
    Similarly, the graph induced by $N(v)\setminus N[v]$
    is a $(t-1,t-1)$-split graph with a 
    $(t-1,t-1)$-split partition $(N_S(\overline{u}v), N_T(\overline{u}v))$. Further, $N_T(uv)$ is an
    independent set of size at most $t-1$. Therefore, in one iteration of Step 3.5, we obtain $S_1=N_S(u\overline{v}), S_2=N_S(\overline{u}v)$, and $S_3=N_S(uv)$. If $N_S(\bar{u}\bar{v})$ is empty,
    then Step 3.5(a) returns \YES. If $N_S(\bar{u}\bar{v})$ is a singleton set,
    then Step 3.5(b) returns \YES. Assume that
    $|N_S(\bar{u}\bar{v})|\geq 2$.
    By Observation~\ref{ob: not an IS}, $N_S(\bar{u}\bar{v})$ is a clique and for every edge $xy$ in it, the common neighborhood of $x$ and $y$ in $\overline{N[u]}\cap \overline{N[v]}$ is a split graph with a partition being $N_S(\bar{u}\bar{v})$ and the rest.
    The algorithm will discover such an edge $xy$
    in one of the iterations of Step 3.5(c) and $N_S(\bar{u}\bar{v})$ will be discovered as $S_4$. Then \YES\ is returned at Step 3.5(c).
\end{proof}

By Proposition~\ref{pro:split-part}, $(t-1,t-1)$-split graphs can be recognized in polynomial-time and all $(t-1,t-1)$-split partitions of a $(t-1,t-1)$-split graph can be found in polynomial-time. Therefore,
each step in the algorithm runs in polynomial-time.
Then we obtain Theorem~\ref{thm:k1tdiamond} from Lemma~\ref{lem:diamondk1t}.
\begin{theorem}
\label{thm:k1tdiamond}
Let $\mathcal{G}$ be the class of $\{K_{1,t},diamond\}$-free graphs for any constant $t\geq 3$. Then \SCG\ can be solved in polynomial-time.
\end{theorem}

It remains open whether the problem is polynomial-time solvable when $\mathcal{G}$ is $H$-free for an $H\in \{K_{1,3},K_{1,4},\text{diamond}\}$.







