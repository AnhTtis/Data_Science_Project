\documentclass[english,aip,letter,superscriptaddress,floatfix,jcp,nourl,nofootinbib]{revtex4}
%nofootinbib is for footnotes not in the bibliography
\usepackage[utf8]{inputenc}
\usepackage[T1]{fontenc}
\usepackage{upgreek}
\usepackage{graphicx}
\usepackage{ae,aecompl}
\usepackage{color}
\usepackage{pstricks,pst-node}
\usepackage{psfrag}
\usepackage{verbatim}
\usepackage{bm}
%\usepackage{lscape}
\usepackage{pdflscape}
\usepackage{rotating}
\usepackage{stmaryrd}
\usepackage{amsfonts}
\usepackage{amsmath}
\usepackage{amssymb}
\usepackage{bbold}
\usepackage{endnotes}
\usepackage{ifthen}
\usepackage{bm}
\usepackage{setspace}
\usepackage{endnotes}
\usepackage{fancyhdr}
\usepackage{fancybox}
\usepackage{makeidx}
%\usepackage{caption}
%\usepackage{subfigure}
\usepackage{framed}
\usepackage{listings}
\usepackage[inline]{showlabels}
\usepackage{alphalph}
\usepackage{refcount}
%
\definecolor{shadecolor}{rgb}{0,.7,0}
\definecolor{shadecolor}{gray}{0.7}
\definecolor{shadecolor}{gray}{0.95}
\definecolor{ogreen}{rgb}{0,0.8,0}
\definecolor{magenta}{rgb}{1,0,1}
\definecolor{brown}{rgb}{0.7,0.4,0.2}
\definecolor{shadecolor}{gray}{0.9}
\newcommand{\ogreen}{\color{ogreen}}
\newcommand{\brown}{\color{brown}}
\newcommand{\shadecolor}{\color{shadecolor}}

% Used to mark changes
\newcommand{\del}[1]{{\color{red}#1}}
\newcommand{\new}[1]{{\color{blue}#1}}
\newcommand{\newest}[1]{{\color{gray}#1}}
\newcommand{\woobling}[1]{{\color{brown}#1}}
\newcommand{\rigid}[1]{{\color{brown}#1}}
\newcommand{\Pep}[1]{{\bf{\color{magenta}[#1]}}}
\newcommand{\Note}[1]{{\bf \color{red}#1}}
\newcommand{\Beware}[1]{{\color{magenta}#1}}

%\GK {\dot{\rho}_\mu}{\dot{\rho}_\nu}
\newcommand{\GK}[2]{\langle#1 || #2\rangle}

\newcommand{\define}{\triangleq}
\newcommand{\bra}[1]{\left< #1 \right|}
\newcommand{\ket}[1]{\left| #1 \right>}
\newcommand{\braket}[2]{\left. \left< #1 \right| #2 \right>}
\newcommand{\norm}{|\!|}
\newcommand{\llangle}{\left\langle}
\newcommand{\rrangle}{\right\rangle}
\newcommand{\esc}{\!\cdot\!}

\RequirePackage{dsfont}


\usepackage{xr-hyper}
\usepackage{hyperref}


\graphicspath{{./Figures/}}
\begin{document}

\title{Supplemental Material for ``The role of thermal fluctuations in the motion of a free body''}

\author{Pep Espa\~nol}
\affiliation{ Dept.   F\'{\i}sica Fundamental, Universidad Nacional
  de Educaci\'on a Distancia, Madrid, Spain}
\author{Mark Thachuk}
\affiliation{ Department of Chemistry, University of British Columbia, Vancouver, Canada}
\author{J.A. de la Torre}
\affiliation{ Dept.   F\'{\i}sica Fundamental, Universidad Nacional
  de Educaci\'on a Distancia, Madrid, Spain}

\date{\today}
\begin{abstract}
We present the explicit calculations required in the formulation of the stochastic differential equations governing the orientation and central moments of a free body, discussed in the main text.
\end{abstract}
\maketitle

  \color{black}
\section{A guide to the Supplemental Material}
In the  main text we have  presented the different building  blocks in
the SDE governing the dynamics of the CG variables, while the explicit
calculation of  these results has  been deferred to  this Supplemental
Material. The Sections  can be divided into groups.   The first group,
Secs.  \ref{App:Entropy}-\ref{App:cross}, presents  the calculation of
the building blocks entering the dynamic equations.  The second group,
Secs.   \ref{App:DerRot}-\ref{App:Sym}  presents   properties  of  the
rotation matrix, the angular velocity and its connection with the time
derivative  of the  orientation.  The  third group  collected in  Sec.
\ref{App:Mic} gives  different phase  functions expressed in  terms of
the coordinates  and velocities  of the  particles. In  particular, we
obtain the explicit microscopic form of the angular velocity, dilation
momentum,  dilational  force, as  well  as  a  number of  other  phase
functions related to  the derivatives of CG variables  with respect to
particles  coordinates.   The  explicit  form of  these  variables  is
required when their averages and correlations are to be measured in MD
simulations.   The  last  section \ref{App:Submanifold}  presents  the
calculation  of  special  Gaussian integrals  containing  Dirac  delta
functions of  conserved quantities that are  instrumental in computing
momentum integrals  in the evaluation  of the entropy  and conditional
expectations.  All the results  presented in the Supplemental Material
are used at some point in the calculations of the building blocks.

\setcounter{section}{0}
\renewcommand{\appendixname}{\large{Appendix}}
\renewcommand{\theequation}{\arabic{equation}}
\newcounter{my_equation_counter}
\renewcommand{\thesubsubsection}{\thesubsection.\alph{subsubsection}}
\renewcommand{\thesection}{\large{\Alph{section}}}
\renewcommand{\thesubsection}{\thesection.\arabic{subsection}}
\section{Calculation of the entropy}
\setcounter{equation}{\getrefnumber{finaleq}}
\label{App:Entropy}
In this section we evaluate the entropy function
at  the present  level of  description, which is given in terms of the integral
in phase space
\begin{align}
  \label{SM:1}
S_B({\bf    R},\boldsymbol{\Lambda},{{\bf M}},{\bf
    P},{\bf S},\boldsymbol{\Pi},E)
&  =k_B\ln \int dz
  \delta\left(\hat{\bf R}(z)-{\bf R} \right)
    \delta\left(\hat{\boldsymbol{\Lambda}}(z)-\boldsymbol{\Lambda}\right)
    \delta\left(\hat{{\bf M}}(z)-{{\bf M}} \right)
\nonumber\\
\times&\delta\left(\hat{\bf P}(z)-{\bf P} \right)
          \delta\left(\hat{\bf S}(z)-{\bf S} \right)
          \delta\left(\hat{\boldsymbol{\Pi}}(z)-\boldsymbol{\Pi}\right)
              \delta\left(\hat{H}(z)-E \right)
\end{align}
The evaluation  is possible  because the
momentum integrals involve the  intersection of a hypersphere, defined
by the  Dirac delta  on the  kinetic energy  in the  Hamiltonian, with
three   hyperplanes,   defined   by   the  three   Dirac   deltas   on
${\bf  P},{\bf   S},\boldsymbol{\Pi}$.   Motivated  by   the  Gaussian
integrals  discussed in Sec. \ref{App:Submanifold},  consider the
following   change  of   variables  from   $z$  to   \textit{co-moving
  coordinates} $z'(z)$
\begin{align}
{\bf r}'_{i}&={\bf r}_{i}-{\bf R}
\nonumber\\  
 {\bf p}_{i}'&
={\bf p}_{i} - m_{i}{\bf V} +  m_{i} ({\bf r}_{i}-{\bf R})\times\hat{\boldsymbol{\Omega}}
               -\frac{\partial \hat{{\bf M}}}{\partial {\bf r}_{i}}\esc\hat{\boldsymbol{\nu}}
\label{W:BodyFrame}
\end{align}
where we have introduced the linear, spin, and dilational velocities as
\begin{align}
  \label{eq:85}
    {\bf V}
  &=\frac{{\bf P}}{{\rm M}}
\nonumber\\
  \hat{\boldsymbol{\Omega}}
  &=\hat{\bf I}^{-1}\esc{\bf S}
    \nonumber\\
  \hat{\boldsymbol{\nu}}
  &=\hat{\bf T}^{-1}\esc\boldsymbol{\Pi}
\end{align}
Here,                            the                            values
${\bf R}, {\bf  P}, {\bf S}$, and $ \boldsymbol{\Pi}$ are
those appearing  in the Dirac  delta function of
(\ref{SM:1}).   They are  numbers,  not  phase functions.  The
total  mass of  the  body is  ${\rm  M}$ and  we  have introduced  the
dilation  inertia matrix  (with the  same physical  dimensions as  the
inertia tensor)
\begin{align}
  \label{eq:42b}
  \hat{\bf T}
  &\equiv\sum_i\frac{1}{m_i}\frac{\partial \hat{{\bf M}}}{\partial {\bf r}_i}
    \esc \frac{\partial \hat{{\bf M}}}{\partial {\bf r}_i}
\end{align}
where  the  dot  product  means   summation  over  the  components  of
${\bf r}_i$.  Even though we may  recognize in the change of variables
(\ref{W:BodyFrame}) a translation and rotation of the coordinates to a
rotating frame  in which the  linear and  spin momentum are  zero, the
last term  involving the dilational  velocity $\hat{\boldsymbol{\nu}}$
may  appear  intriguing.   As  we   show  below,  this  last  term  in
(\ref{W:BodyFrame})  ensures  \textit{all} momenta  (linear,  angular,
\textit{and}    dilational)    simultaneously    vanish    in    these
coordinates. Furthermore,  the Jacobian determinant of  this change of
variables   is    unity   because    $\hat{\boldsymbol{\Omega}}$   and
$\hat{\boldsymbol{\nu}}$  in  (\ref{eq:85})  do   not  depend  on  the
velocities ${\bf v}_i$.  The inverse transformation $z(z')$ is
\begin{align}
  {\bf r}_{i}
  &={\bf r}'_{i}+{\bf R} 
\nonumber\\  
  {\bf p}_{i}
  &={\bf p}'_{i} + m_{i}{\bf V} -  m_{i} {\bf r}'_{i}\times\hat{\boldsymbol{\Omega}}'(z')
    +\frac{\partial \hat{{\bf M}}(z')}{\partial {\bf r}'_{i}}\esc\hat{\boldsymbol{\nu}}(z')
    \label{W:BodyFrameInv}
\end{align}
where
\begin{align}
  \label{eq:217}
  \hat{\boldsymbol{\Omega}}'(z')
  &=  \hat{\boldsymbol{\Omega}}(z(z'))
    =\hat{\bf I}^{-1}(z(z'))\esc{\bf S}=
    [\hat{\bf I}']^{-1}(z')\esc{\bf S}
  \nonumber\\
  \hat{\boldsymbol{\nu}}'(z')
  &=\hat{\boldsymbol{\nu}}(z'(z))
    =\hat{\bf T}^{-1}(z(z'))\esc\boldsymbol{\Pi}
    =[\hat{\bf T}']^{-1}(z')\esc\boldsymbol{\Pi}
\end{align}
The inertia tensor in the new variables is
\begin{align}
  \label{eq:218}
\hat{\bf I}'(z')&\equiv\sum_im_i[{\bf r}_i']_\times^T \esc  [{\bf r}_i']_\times
\end{align}
and the tensor ${\bf T}'(z')$ is
\begin{align}
  \label{eq:228}
  \hat{\bf T}'(z')&\equiv  \hat{\bf T}(z(z'))
\end{align}
Define the operator ${\cal G}$
whose action on an arbitrary phase function is
\begin{align}
  \label{eq:35}
  {\cal G}\hat{F}(z')=\hat{F}(z(z'))
\end{align}
According to  the general  form for a change  of variables  (with unit
Jacobian)  (\ref{SM:1}) can be written as
\begin{align}
S_B({\bf    R},\boldsymbol{\Lambda},{{\bf M}},{\bf
  P},{\bf S},\boldsymbol{\Pi},E)
  &=k_B\ln \int dz'
\delta\left({\cal G}\hat{\bf R}(z')-{\bf R} \right)
    \delta\left({\cal G}\hat{\boldsymbol{\Lambda}}(z')-\boldsymbol{\Lambda}\right)
    \delta\left({\cal G}\hat{{\bf M}}(z')-{{\bf M}} \right)
\nonumber\\
&\quad\times\delta\left({\cal G}\hat{\bf P}(z')-{\bf P} \right)
              \delta\left({\cal G}\hat{\bf S}(z')-{\bf S} \right)
                  \delta\left({\cal G}\hat{\boldsymbol{\Pi}}(z')-\boldsymbol{\Pi}\right)
              \delta\left({\cal G}\hat{H}(z')-E \right)
\label{W:EntropyBody}
\end{align}
So we  only have to use (\ref{W:BodyFrameInv}) to determine the effect of the operator ${\cal G}$ on the CG  variables.
  The center of mass position 
transforms as
\begin{align}
{\cal G}\hat{\bf R}(z')&=  \hat{\bf R}(z(z'))=
\sum_{i}\frac{m_{i}}{{\rm M}}{\bf r}'_{i}+{\bf R} 
\equiv \hat{\bf R}(z') +{\bf R}   
\end{align}
The inertia tensor transforms as
\begin{align}
  \label{W:eq:208}
{\cal G}  \hat {\bf I}(z')
&=\sum_{i}m_{i}[{\bf r}'_{i}]_\times^T\esc[{\bf r}'_{i}]_\times 
\equiv \hat {\bf I}(z')+{\rm M}[\hat{\bf R}(z')]_\times\esc[\hat{\bf R}(z')]_\times
\end{align}
and the  gyration tensor as
\begin{align}
  \label{eq:287}
{\cal G}    \hat {\bf G}(z')
&=\frac{1}{4}\sum_{i}m_{i}{{\bf r}'}_{i}{{\bf r}'_{i}}^T
\equiv\hat{\bf G}(z')+\frac{{\rm M}}{4}\hat{\bf R}(z')\hat{\bf R}^T(z')
\end{align}
Because  the  inertia  tensor  only   depends  on  positions  and  the
transformation (\ref{W:BodyFrame}) is just  a translation,
the orientation and principal  moments in
the new configuration $r'$ are the same as in $r$, that is
\begin{align}
  \label{W:eq:209}
{\cal G}  \hat{\boldsymbol{\Lambda}}(z')
  &=\hat{\boldsymbol{\Lambda}}(z')
  \nonumber\\
{\cal G}  \hat{{\bf M}}(z')
  &=\hat{{\bf M}}(z')
\end{align}


The momentum transforms as 
\begin{align}
{\cal G}\hat{\bf P}(z') &=
\sum_{i}\left({\bf p}'_{i} -  m_{i} {\bf r}'_{i}\times{\hat{\boldsymbol{\Omega}}'}(z')+ m_{i}{\bf V} +\frac{\partial \hat{{\bf M}}(z')}{\partial {\bf r}'_{i}}\esc\hat{\boldsymbol{\nu}}'(z')\right)
\equiv     \hat{\bf P}(z') - {\rm M}\hat{\bf R}(z')\times{\hat{\boldsymbol{\Omega}}'(z')}+{\bf P}
\end{align}
where we have used  the translation invariance property (\ref{trans2}).
The  spin   transforms as
\begin{align}
{\cal G}  \hat{\bf S}(z')
  &=\sum_{i}{\bf r}'_{i}\times
    \left({\bf p}'_{i} -  m_{i} {\bf r}'_{i}\times{\hat{\boldsymbol{\Omega}}}(z')
    + m_{i}{\bf V} +\frac{\partial \hat{{\bf M}}(z')}{\partial {\bf r}'_{i}}\esc\hat{\boldsymbol{\nu}}(z')\right)
    \nonumber\\
  &\overset{(\ref{dsym})}{=}\sum_{i} {\bf r}'_{i}\times
    {\bf p}'_{i} 
    -\sum_{i} {\bf r}'_{i}\times
    \left(  m_{i} {\bf r}'_{i}\times{\hat{\boldsymbol{\Omega}}}(z') \right)
    +
    \hat{\bf R}(r')\times {\bf P} 
\nonumber\\
  &= \hat{\bf S}(z')+\hat{\bf I}(z')\esc {\hat{\boldsymbol{\Omega}}}(z')
    +\hat{\bf R}(z')\times{\bf P}
\end{align}
where the inertia tensor transforms as (\ref{W:eq:208}),
and (\ref{eq:85}) shows the second term on the right hand side is the constant vector ${\bf S}$.
The dilation momentum  transforms as
\begin{align}
  \label{eq:40}
{\cal G}  \hat{\boldsymbol{\Pi}}(z')
  &=\sum_i\frac{\partial \hat{{\bf M}}(z')}{\partial {\bf r}'_i}\esc
 \left({\bf v}'_{i} -   {\bf r}'_{i}\times{\hat{\boldsymbol{\Omega}}}(z')
    +{\bf V} +\frac{1}{m_i}\frac{\partial \hat{{\bf M}}(z')}{\partial {\bf r}'_{i}}\esc\hat{\boldsymbol{\nu}}(z')\right)
=
    \hat{\boldsymbol{\Pi}}(z')+\hat{\bf T}(z')\esc\hat{\boldsymbol{\nu}}(z') =
      \hat{\boldsymbol{\Pi}}(z')+\boldsymbol{\Pi}
\end{align}
where  we  have  used  the  translation  (\ref{trans2})  and  rotation
(\ref{dsym}) invariance  of the central moments,  and the definition
of the  dilational velocity  (\ref{eq:85}).  Finally,  the Hamiltonian
 transforms as
\begin{align}
{\cal G}\hat{H}(z') 
&= \sum_{i}\frac{m_{i}}{2}\left({\bf v}'_{i}+{\bf V}
                  -{\bf r}'_{i}\times\hat{\boldsymbol{\Omega}}(z')
                  +\frac{1}{m_i}\frac{\partial \hat{{\bf M}}(z')}
                  {\partial {\bf r}'_{i}}\esc\hat{\boldsymbol{\nu}}(z')
                  \right)^2+\hat{\Phi}(z')
                    \nonumber\\
    &= \hat{H}(z')
      + \hat{\bf P}(z')\esc{\bf V}
      + \hat{\bf S}(z')\esc\hat{\boldsymbol{\Omega}}(z')+\hat{\boldsymbol{\Pi}}(z')\esc\hat{\boldsymbol{\nu}}(z')
           -M (\hat{\bf R}(z')\times\hat{\boldsymbol{\Omega}}(z'))\esc{\bf V}
      \nonumber\\
  &+ \frac{{\rm M}{\bf V}^2}{2}
    + \frac{1}{2}\hat{\boldsymbol{\Omega}}^T(z')\esc\hat{\bf I}(z')\esc\hat{\boldsymbol{\Omega}}(z')
       + \frac{1}{2}\hat{\boldsymbol{\nu}}^T(z')\esc\hat{\bf T}(z')\esc\hat{\boldsymbol{\nu}}(z')
      + \frac{1}{2}{\rm M}(\hat{\bf R}(z')\times\hat{\boldsymbol{\Omega}}(z'))^2
\end{align}
where  we   have  used   the  fact   the  intramolecular  potential
$\hat{\Phi}(z)$ is translationally invariant, and (\ref{dsym}), (\ref{trans2}).


Substitution   of  the   above   transformed   phase  functions   into
(\ref{W:EntropyBody}) leads to
\begin{align}
  \label{W:eq:122}
S_B({\bf    R},\boldsymbol{\Lambda},{{\bf M}},{\bf
  P},{\bf S},\boldsymbol{\Pi},E)=
k_B\ln
\int dz'
  &\delta\left(\hat{\bf R}(z')\right)
    \delta\left(\hat{\boldsymbol{\Lambda}}(z')-\boldsymbol{\Lambda}\right)
    \delta\left( \hat{{\bf M}}(z')-{{\bf M}} \right)
  \delta\left(\hat{\bf S}(z')\right)
    \delta\left(\hat{\bf P}(z')\right)
    \delta\left(\hat{\boldsymbol{\Pi}}(z')\right)
  \nonumber\\
  &\times
    \delta\left(\hat{H}(z')-\left(E-\frac{{\bf P}^2}{2{\rm M}}
    -\frac{1}{2}{\bf S}^T\esc\hat{\bf I}^{-1}(z')\esc{\bf S}
        -\frac{1}{2}\boldsymbol{\Pi}^T\esc\hat{\bf T}^{-1}(z')\esc\boldsymbol{\Pi}
    \right)\right)
\end{align}
in which  the condition $\hat{\bf  R}(z')=0$, as imposed by  the Dirac
delta  functions in  (\ref{W:eq:122}), has  already been  applied.  We
show  in (\ref{TG})  that $\hat{\bf  T}(z)=\hat{\mathbb{G}}(z)$, where
$\hat{\mathbb{G}}(z)$  is  the  diagonalized gyration matrix, and  is a function  of the microstate $z$  only through
$\hat{\bf M}(z)$.  Similarly,  the $\hat{\bf I}$ is a  function of the
microstate  $z$   only  through   $\hat{\boldsymbol{\Lambda}}(z)$  and
$\hat{\bf  M}(z)$. Because  of  the  presence of  the  Dirac delta  on
$\hat{\bf    M}(z')$     and    $\hat{\boldsymbol{\Lambda}}(z')$    in
(\ref{W:eq:122}) we  may simply substitute these  phase functions with
the   numerical  values   ${{\bf   M}}$  and   $\boldsymbol{\Lambda}$,
respectively.  Therefore, we  may introduce the thermal  energy as the
total  energy  minus  the translational,  rotational,  and  dilational
kinetic energies, that is
\begin{align}
  \label{Thermal-App}
    {\cal E}&=E-K^{\rm trans}-    K^{\rm rot}-K^{\rm dil}
\end{align}
where the kinetic energies are
\begin{align}
  \label{W:eq:286}
  K^{\rm trans}
  &\equiv  \frac{{\bf P}^2}{2{\rm M}}
    \nonumber\\
  K^{\rm rot}
  &\equiv\frac{1}{2}{\bf S}^T\esc{\bf I}^{-1}\esc{\bf S}
    \nonumber\\
  K^{\rm dil}
  &\equiv  \frac{1}{2}{\boldsymbol{\Pi}}^T\esc\mathbb{G}^{-1}\esc{\boldsymbol{\Pi}}=
  \sum_\alpha \frac{{\Pi}_\alpha^2}{2{ M}_\alpha}
\end{align}
With these definitions we may write the entropy in (\ref{W:eq:122}) in the form
\begin{align}
  \label{W:eq:282}
  S_B({\bf R},\boldsymbol{\Lambda},{{\bf M}},{\bf P},{\bf S},\boldsymbol{\Pi},E)
  &= S^{\rm rest}(\boldsymbol{\Lambda},{{\bf M}}, {\cal E})
\end{align}
where the ``entropy at rest'' $S^{\rm rest}(\boldsymbol{\Lambda},{{\bf M}}, {\cal E})$
is the  entropy of a non-spinning and non-dilating body  of energy $ {\cal E}$ with its center of mass  fixed at the origin, that is
\begin{align}
  \label{W:eq:280}
  S^{\rm rest}(\boldsymbol{\Lambda},{{\bf M}},{\cal E})
  &\equiv  k_B\ln\Omega^{\rm rest}(\boldsymbol{\Lambda},{{\bf M}},{\cal E})
    \nonumber\\
  \Omega^{\rm rest}(\boldsymbol{\Lambda},{{\bf M}},{\cal E})
  &\equiv\int dz
    \delta\left(\hat{\bf R}(z)\right)
    \delta\left(\hat{\boldsymbol{\Lambda}}(z)-\boldsymbol{\Lambda}\right)
    \delta\left(\hat{{\bf M}}(z)-{{\bf M}}\right)
    \delta\left(\hat{\bf P}(z) \right)
    \delta\left(\hat{\bf S}(z) \right)
    \delta\left(\hat{\boldsymbol{\Pi}}(z)\right)
    \delta\left(\hat{H}(z)-{\cal E}  \right)
\end{align}
We   see  now   the  justification   for  the   change  of   variables
(\ref{W:BodyFrame}) as  it sets to zero  all the momenta in  the Dirac
delta  functions  that  define  $  S^{\rm  rest}$,  and  combines  the
arguments in terms of a  physically sensible thermal energy ${\cal E}$
defined in (\ref{Thermal-App}).

An  appealing  expression   is  obtained  for  the   rest  entropy  by
multiplying   and  dividing   the   argument  of   the  logarithm   in
(\ref{W:eq:280}) with the following measure
\begin{align}
  \label{W:eq:290}
    \Omega^{\rm MT}({\cal E})&= \int dz
             \delta\left(\hat{\bf R}(z) \right)
             \delta\left(\hat{\bf P}(z) \right)
             \delta\left(\hat{\bf S}(z) \right)
             \delta\left(\hat{H}(z)-{\cal E} \right)
\end{align}
In this way, we can express the rest entropy in the form
\begin{align}
  \label{W:eq:287}
  S^{\rm rest}(\boldsymbol{\Lambda},{{\bf M}},{\cal E})
  &=S_B^{\rm MT}({{\cal E}})+k_B\ln P^{\rm rest}_{{\cal E}}(\boldsymbol{\Lambda},{{\bf M}},{\bf 0})
\end{align}
where  the   entropy  at  the  macroscopic   thermodynamics  level  of
description  (with proper  account of  all dynamic  invariants of  the
system) is given by Boltzmann's original definition
\begin{align}
  \label{W:eq:291}
    S_B^{\rm MT}({\cal E})  &=k_B \ln \Omega^{\rm MT}({\cal E})
\end{align}
This  is the  entropy  of  a thermodynamic  system  when  it does  not
translate nor rotate.  The  probability introduced in (\ref{W:eq:287})
is defined as
\begin{align}
  \label{W:eq:288}
  P^{\rm rest}_{{\cal E}}(\boldsymbol{\Lambda},{{\bf M}},\boldsymbol{\Pi})
  &\equiv\int dz \rho^{\rm rest}_{{\cal E}}(z) 
    \delta\left(\hat{\boldsymbol{\Lambda}}(z)-\boldsymbol{\Lambda}\right)
    \delta\left(\hat{{\bf M}}(z)-{{\bf M}}\right)
    \delta(\hat{\boldsymbol{\Pi}}(z)-\boldsymbol{\Pi})
\end{align}
where the rest microcanonical ensemble is
\begin{align}
  \label{W:eq:289}
    \rho^{\rm rest}_{{\cal E}}(z)  &\equiv\frac{1}{\Omega^{\rm MT}({\cal E})}
                  \delta\left(\hat{\bf R}(z) \right)
                    \delta\left(\hat{\bf P}(z) \right)
                    \delta\left(\hat{\bf S}(z) \right)
                    \delta\left(\hat{H}(z)-{\cal E} \right),
\end{align}
We   assume    the  Hamiltonian   dynamics  is   of  the   mixing
type\footnote{This   is   not   guaranteed   for   bodies   that   are
  quasi-harmonic,    as    the   Fermi-Pasta-Ulam-Tsingou    numerical
  experiments revealed.}.  Therefore,  a molecular dynamics simulation
with  initial conditions  such that  the  system has  zero linear  and
angular momentum, i.e.  it is macroscopically ``at rest'', and with energy
${\cal E}$, will  sample $\rho^{\rm rest}_{{\cal E}}(z)$.
The  probability  $P_{{\cal E}}(\boldsymbol{\Lambda},{\bf  M},{\bf  0})$  in
(\ref{W:eq:287})  is  the  equilibrium   probability  of  finding  the
orientation $\boldsymbol{\Lambda}$,  central moments ${\bf  M}$, and
zero value of the dilational momentum $\boldsymbol{\Pi}$, when the
body is ``at rest''.

\subsection{Calculation of conditional expectations}
Consider the change of variables (\ref{W:BodyFrameInv}) when computing
conditional expectations, $\llangle\hat{F}\rrangle^a$, of an arbitrary
phase function $\hat{F}(z)$  at the
present level of description, that is
\begin{align}
  \llangle\hat{F}\rrangle^a
  \equiv\frac{1}{ \Omega(a)}
  \int dz\hat{F}(z)
  &\delta\left(\hat{\bf R}(z)-{\bf R}\right)
    \delta\left(\hat{\boldsymbol{\Lambda}}(z)-\boldsymbol{\Lambda}\right)
    \delta\left(\hat{{\bf M}}(z)-{{\bf M}}\right)
    \nonumber\\
\times  &
    \delta\left(\hat{\bf P}(z)-{\bf P}\right)
    \delta\left(\hat{\bf S}(z)-{\bf S}\right)
        \delta\left(\hat{\boldsymbol{\Pi}}(z)-\boldsymbol{\Pi}\right)
    \delta\left(\hat{H}(z)-E\right)
    \label{W:Fav}
\end{align}
Using the operator $\cal G$ to change variables $z\to z' $ gives
\begin{align}
  \label{eq:43}
  \llangle \hat{F}\rrangle^a
  &=\llangle {\cal G}\hat{F}\rrangle^{\boldsymbol{\Lambda}{\bf M}{\cal E}  }_{\rm rest}
\end{align}
where          the          rest          conditional          average
$\llangle  \cdots\rrangle^{\boldsymbol{\Lambda}{\bf M}{\cal  E} }_{\rm
  rest}$ is defined as
\begin{align}
  \label{W:Fav2}
  \llangle\hat{F}\rrangle^{\boldsymbol{\Lambda}{{\bf M}}{\cal E}}_{\rm rest}
  \equiv \frac{1}{ \Omega^{\rm rest}(\boldsymbol{\Lambda},{{\bf M}},{\cal E})}
  \int dz\hat{F}(z)
  &\delta\left(\hat{\bf R}(z)\right)
    \delta\left(\hat{\boldsymbol{\Lambda}}(z)-\boldsymbol{\Lambda}\right)
    \delta\left(\hat{{\bf M}}(z)-{{\bf M}}\right)
    \delta\left(\hat{\bf P}(z)\right)
    \delta\left(\hat{\bf S}(z)\right)
    \delta\left(\hat{\boldsymbol{\Pi}}(z)\right)
    \delta\left(\hat{H}(z)-{\cal E}\right)
\end{align}
with  $\Omega^{\rm   rest}(\boldsymbol{\Lambda},{{\bf  M}},{\cal  E})$
defined  in (\ref{W:eq:280}).   The rest  conditional ensemble  is the
conditional  average $\llangle\cdots\rrangle^a$  in (\ref{W:Fav})  for
the                particular                 values                of
$a=\{{\bf R}={\bf  0},\boldsymbol{\Lambda},{\bf M},{\bf  P}={\bf 0},{\bf
  S}={\bf 0},\boldsymbol{\Pi}={\bf  0}, E={\cal  E}\}$, and can  also be
expressed  in   the  following  convenient   form  in  terms   of  the
microcanonical ensemble (\ref{W:eq:289}) as
\begin{align}
  \label{eq:176}
    \llangle\hat{F}\rrangle^{\boldsymbol{\Lambda}{{\bf M}}{\cal E}}_{\rm rest}
&= \frac{1}{ P_{\cal E}^{\rm rest}(\boldsymbol{\Lambda},{{\bf M}},{\bf 0})}
  \int dz\hat{F}(z)\rho^{\rm rest}_{\cal E}(z)
    \delta\left(\hat{\boldsymbol{\Lambda}}(z)-\boldsymbol{\Lambda}\right)
    \delta\left(\hat{{\bf M}}(z)-{{\bf M}}\right)
    \delta\left(\hat{\boldsymbol{\Pi}}(z)\right)
\end{align}


\setcounter{my_equation_counter}{\value{equation}}
\section{Consequences of rotational symmetry}
\setcounter{equation}{\value{my_equation_counter}}
\label{App:W:RotInv}
In this section,  we examine the effect  of rotational symmetry on
the   functional  forms  for  the  entropy,   the  rest   probability
distribution, and the conditional expectations of tensorial quantities.

\subsection{Rotational symmetry of the entropy}
Because,  as  shown in  (\ref{W:eq:282}),  the  entropy can  be  fully
expressed in  terms of  the rest entropy,  we consider  the rotational
invariance of the latter.  Consider  the following change of variables
$z=Qz'$
\begin{align}
  \label{W:eq:182}
  {\bf r}_i
  &=\boldsymbol{\cal Q}\esc{\bf r}_i'=e^{[{\bf A}]_\times}\esc{\bf r}_i'
  \nonumber\\
  {\bf p}_i
  &=\boldsymbol{\cal Q}\esc{\bf p}_i'=e^{[{\bf A}]_\times}\esc{\bf p}_i'
\end{align}
where  $\boldsymbol{\cal  Q}=e^{[{\bf  A}]_\times}$  is  an  arbitrary
rotation  with orientation  ${\bf A}$. This change
of variables in the phase  space integral (\ref{W:eq:280}) gives (with
a renaming of $z'$ with $z$ for easier notation)
\begin{align}
  \label{W:eq:210}
&  S^{\rm rest}(\boldsymbol{\Lambda},{{\bf M}},{\cal E})
                     \nonumber\\
  &=  k_B\ln
    \int dz
    \delta\left(\hat{\bf R}(Qz)\right)
    \delta\left(\hat{\boldsymbol{\Lambda}}(Qz)-\boldsymbol{\Lambda}\right)
    \delta\left(\hat{{\bf M}}(Qz)-{{\bf M}}\right)
    \delta\left(\hat{\bf P}(Qz) \right)
    \delta\left(\hat{\bf S}(Qz) \right)
    \delta\left(\hat{\boldsymbol{\Pi}}(Qz)\right)
    \delta\left(\hat{H}(Qz)-{\cal E}  \right)
    \nonumber\\
  &=  k_B\ln
    \int dz
    \delta\left(\boldsymbol{\cal Q}\esc\hat{\bf R}(z)\right)
    \delta\left(\hat{\boldsymbol{\Lambda}}(Qz)-\boldsymbol{\Lambda}\right)
    \delta\left(\hat{{\bf M}}(z)-{{\bf M}}\right)
    \delta\left(\boldsymbol{\cal Q}\esc\hat{\bf P}(z) \right)
    \delta\left(\boldsymbol{\cal Q}\esc\hat{\bf S}(z) \right)
    \delta\left(\hat{\boldsymbol{\Pi}}(z)\right)
    \delta\left(\hat{H}(z)-{\cal E}  \right)
    \nonumber\\
  &=  k_B\ln
    \int dz
    \delta\left(\hat{\bf R}(z)\right)
    \delta\left(\hat{\boldsymbol{\Lambda}}(Qz)-\boldsymbol{\Lambda}\right)
    \delta\left(\hat{{\bf M}}(z)-{{\bf M}}\right)
    \delta\left(\hat{\bf P}(z) \right)
    \delta\left(\hat{\bf S}(z) \right)
    \delta\left(\hat{\boldsymbol{\Pi}}(z)\right)
    \delta\left(\hat{H}(z)-{\cal E}  \right)
\end{align}
where    in   the    second    line   we    have    used   the    fact
$\hat{\bf  R},\hat{\bf  P},\hat{\bf  S}$   transform  as  vectors  and
$\hat{{\bf  M}},\hat{\boldsymbol{\Pi}},\hat{H}$  are  invariant  under
rotations (as discussed in  Sec.  \ref{App:Orientation}). In the third
line we used the Fourier representation of the Dirac delta function to
show that
  \begin{align}
    \label{eq:202}
    \delta\left(\boldsymbol{\cal Q}\esc\hat{\bf P}(z) \right)
    &=\frac{1}{(2\pi)^3}\int d{\bf k}^T\exp\{i{\bf k}^T\esc\boldsymbol{\cal Q}\esc\hat{\bf P}(z) \}
      =\frac{1}{(2\pi)^3}\int d{{\bf k}^T}'\exp\{{i{\bf k}^T}'\esc\hat{\bf P}(z) \}=
      \delta\left(\hat{\bf P}(z) \right)
  \end{align}
  where the change of variables  ${\bf k}'=\boldsymbol{\cal Q}^T\esc{\bf k}$ with unit
  Jacobian determinant has been used.
Note  the  left  hand  side  of  (\ref{W:eq:210})  is  independent  of
${\bf  A}$ so  taking the  derivative of  both sides  with respect  to
${\bf A}$ and evaluating at ${\bf A}=0$ gives in component notation
\begin{align}
\label{W:eq:259}
  0&=\left . \frac{\partial}{\partial{\bf A}^\gamma} \int dz
     \delta\left(\hat{\bf R}(z)\right)
     \delta\left(\hat{\boldsymbol{\Lambda}}(Qz)-\boldsymbol{\Lambda}\right)
     \delta\left(\hat{{\bf M}}(z)-{{\bf M}}\right)
     \delta\left(\hat{\bf P}(z) \right)
     \delta\left(\hat{\bf S}(z) \right)
     \delta\left(\hat{\boldsymbol{\Pi}}(z)\right)
     \delta\left(\hat{H}(z)-{\cal E} \right)\right |_{{\bf A}=0}\cr
  &
=\int dz \delta\left(\hat{\bf R}(z)\right) \left .\left (
     \frac{\partial}{\partial {\Lambda}_{\gamma'}(Qz)}
     \delta\left(\hat{\boldsymbol{\Lambda}}(Qz)-\boldsymbol{\Lambda}\right)
    \frac{\partial}{\partial{\bf A}^\gamma}\hat{\Lambda}_{\gamma'}(Qz)\right )\right |_{{\bf A}=0}
    \nonumber\\
  &\times
    \delta\left(\hat{{\bf M}}(z)-{{\bf M}}\right)
    \delta\left(\hat{\bf P}(z) \right)
    \delta\left(\hat{\bf S}(z) \right)
    \delta\left(\hat{\boldsymbol{\Pi}}(z)\right)
    \delta\left(\hat{H}(z)-{\cal E} \right)\cr
  &=-\frac{\partial}{\partial {\Lambda}_{\gamma'}}
    \left .  \int dz \delta\left(\hat{\bf R}(z)\right)
  \frac{\partial }{\partial{\bf A}^\gamma}\hat{\Lambda}_\alpha(Qr)\right |_{{\bf A}=0}
     \delta\left(\hat{\boldsymbol{\Lambda}}(z)-\boldsymbol{\Lambda}\right)
    \delta\left(\hat{{\bf M}}(z)-{{\bf M}}\right)
    \delta\left(\hat{\bf P}(z) \right)
    \delta\left(\hat{\bf S}(z) \right)
    \delta\left(\hat{\boldsymbol{\Pi}}(z)\right)
    \delta\left(\hat{H}(z)-{\cal E} \right)
   \end{align}
   where we have used  the chain rule and noted in  the last line that
   $\hat{\boldsymbol{\Lambda}}$ is  a function  of position  only. Now
   make use of the following identity
\begin{align}
  \label{eq:24}
  \left.  \frac{\partial }{\partial{\bf A}^\gamma}\hat{\Lambda}_{\gamma'}(Qr)\right |_{{\bf A}=0}
  &\overset{\mbox{chain rule}}{=} \left.\sum_i \frac{\partial \hat{\Lambda}_{\gamma'}}
    {\partial{\bf r}_i^{\alpha}}(Qr)\frac{\partial \boldsymbol{\cal Q}^{\alpha\beta}{\bf r}_i^\beta}{\partial{\bf A}^\gamma}\right |_{{\bf A}=0}
    \overset{(\ref{eq:711})}{=}
    \sum_i        \frac{\partial\hat{\Lambda}_{\gamma'}}{\partial{\bf r}^{\alpha}_i}
    \epsilon_{\alpha\gamma\beta} {\bf r}^\beta_i
    =  -\sum_i \frac{\partial\hat{\Lambda}_{\gamma'}}{\partial{\bf r}^{\alpha'}}
    [{\bf r}_i]_\times^{\alpha\gamma}
    \overset{(\ref{eq:255})}{=}{B}_{\gamma'\gamma}(\hat{\boldsymbol{\Lambda}}(r))
\end{align}
Because ${B}_{\gamma'\gamma}(\hat{\boldsymbol{\Lambda}}(r))$
depends on the configuration $r$ only through the orientation, the
Dirac delta function on the orientation in (\ref{W:eq:259}) allows us to move
this term outside the integral giving
\begin{align}
  \label{eq:37}
0   = \frac{\partial}{\partial {\Lambda}_{\gamma'}}
    {B}_{\gamma'\gamma}(\boldsymbol{\Lambda}) \Omega^{\rm rest}(\boldsymbol{\Lambda},{{\bf M}},{\cal E})
    \end{align}
Performing this
derivative, using (\ref{W:eq:280}), and simplifying gives
\begin{align}
  \label{W:eq:266}
  \frac{\partial S^{\rm rest}}{\partial \boldsymbol{\Lambda}}(\boldsymbol{\Lambda},{{\bf M}},{\cal E})
  \esc  {\bf B}(\boldsymbol{\Lambda})
  + k_B \frac{\partial}{\partial \boldsymbol{\Lambda}}\esc{\bf B}(\boldsymbol{\Lambda})
  =& 0
\end{align}
This identity reflects the rotational invariance of the entropy.

\subsection{Rotational      symmetry      of      the      probability
  $ P^{\rm rest}_{\cal E}(\boldsymbol{\Lambda},{{\bf M}},\boldsymbol{\Pi})$}
\label{W:App:RotProb}

The rest  microcanonical ensemble (\ref{W:eq:289}) is  invariant under
the  rotation (\ref{W:eq:182}),  as are the  two phase  functions
$\hat{{\bf M}}(z),\hat{\boldsymbol{\Pi}}(z)$, according to (\ref{eq:47}),
  (\ref{eq:51}).   Therefore,   performing    the   change   of   variables
  (\ref{W:eq:182}) gives
  \begin{align}
    \label{eq:327}
      P^{\rm rest}_{\cal E}(\boldsymbol{\Lambda},{{\bf M}},\boldsymbol{\Pi})
  &=\int dz \rho^{\rm rest}_{\cal E}(z) \delta\left(\hat{\boldsymbol{\Lambda}}(Qz)-\boldsymbol{\Lambda}\right)
\delta\left(\hat{{\bf M}}(z)-{{\bf M}}\right)                                       \delta(\hat{\boldsymbol{\Pi}}(z)-\boldsymbol{\Pi}) 
\end{align}
 Analogous  to the procedure used in obtaining
(\ref{W:eq:266}),   by   taking   the   derivative   with   respect   to
${\bf A}^\gamma$ and evaluating the result at ${\bf A}=0$ on both sides we obtain
\begin{align}
  \label{W:eq:450}
  0
  &=\left .\int dz \rho_{\cal E}(z)  \frac{\partial}{\partial {\bf A}^\gamma}
    \delta\left(\hat{\boldsymbol{\Lambda}}(Qz)-\boldsymbol{\Lambda}\right)
    \right |_{{ {\bf A}=0 }}
    \delta\left(\hat{{\bf M}}(z)-{{\bf M}}\right)
    \delta(\hat{\boldsymbol{\Pi}}(z)-\boldsymbol{\Pi}) 
= -\frac{\partial}{\partial {\Lambda}_{\gamma'}}
    {B}_{\gamma'\gamma}(\boldsymbol{\Lambda})P^{\rm rest}_E(\boldsymbol{\Lambda},{{\bf M}},\boldsymbol{\Pi})
    \end{align}
Using the chain rule and some rearranging then gives
\begin{align}
  \label{W:eq:113}
  \left(\frac{\partial}{\partial {\Lambda}_{\gamma'}}
{B}_{\gamma'\gamma}(\boldsymbol{\Lambda})\right)
{B}^{-1}_{\gamma\alpha} +
  \frac{\partial}{\partial {\Lambda}_{{\alpha}}}
\ln P^{\rm rest}_E(\boldsymbol{\Lambda},{{\bf M}},\boldsymbol{\Pi})=0
\end{align}
Further use of (\ref{eq:173}) gives
\begin{align}
  \label{W:eq:128}
-\frac{\partial }{\partial {\Lambda}_\alpha}
\ln\left(\frac{1-\cos\Lambda}{\Lambda^2}\right) +
  \frac{\partial}{\partial {\Lambda}_{{\alpha}}}\ln
  P^{\rm rest}_E(\boldsymbol{\Lambda},{{\bf M}},\boldsymbol{\Pi})=0
\end{align}
implying
\begin{align}
  \label{eq:57}
  \frac{\partial}{\partial {\Lambda}_{{\alpha}}}\ln
\frac{  P^{\rm rest}_E(\boldsymbol{\Lambda},{{\bf M}},\boldsymbol{\Pi})}{\left(\frac{1-\cos\Lambda}{\Lambda^2}\right)}=0
\end{align}
which shows  the  argument of  the logarithm  is a  constant with
respect to $\boldsymbol{\Lambda}$ and therefore, must be a function of
${{\bf    M}},\boldsymbol{\Pi},E$   only,    that    we   denote    as
$P^{\rm rest}_E({\bf M},\boldsymbol{\Pi})$. We conclude
\begin{align}
  \label{eq:119}
% \label{W:eq:134}
  P^{\rm rest}_E(\boldsymbol{\Lambda},{{\bf M}},\boldsymbol{\Pi})=\frac{1-\cos\Lambda}{4\pi^2\Lambda^2}P^{\rm rest}_E({{\bf M}},\boldsymbol{\Pi})
 =P^{\rm Haar}(\boldsymbol{\Lambda})P^{\rm rest}_E({{\bf M}},\boldsymbol{\Pi})
\end{align}
where we have  introduced the probability density $P^{\rm Haar}(\boldsymbol{\Lambda})$   in the orientation
space
\begin{align}
  \label{eq:150}
P^{\rm Haar}(\boldsymbol{\Lambda})&\equiv  \frac{1-\cos\Lambda}{4\pi^2\Lambda^2}
\end{align}
We  have discussed  in  the main  text the  probability
density   of   the   Haar   measure  in   the   parameter   space   of
$\boldsymbol{\Lambda}$.    Because   this   probability   density   is
normalized  on   the  sphere  of   radius  $\pi$  in  the   space  of
orientations, the  factor $P^{\rm  rest}_E({\bf M},\boldsymbol{\Pi})$
 in    (\ref{eq:119})   is   just   the    marginal   of
$  P^{\rm  rest}_E(\boldsymbol{\Lambda},{{\bf  M}},\boldsymbol{\Pi})$,
that is
\begin{align}
  \label{eq:151}
  P^{\rm rest}_{\cal E}({{\bf M}},\boldsymbol{\Pi})
  &=\int dz \rho^{\rm rest}_{\cal E}(z)
    \delta\left(\hat{{\bf M}}(z)-{{\bf M}}\right)
    \delta(\hat{\boldsymbol{\Pi}}(z)-\boldsymbol{\Pi}) 
\end{align}
The rotational invariance expressed  in (\ref{eq:119}) implies, therefore,
that   the  orientation   $\boldsymbol{\Lambda}$  is   distributed  at
equilibrium according  to the  Haar measure,  and  is statistically
independent  of  the values  of  the  central moments  and  dilation
momentum.
{The Haar density probability is defined microscopically as
  \begin{align}
    \label{eq:33}
    P^{\rm Haar}(\boldsymbol{\Lambda})&\equiv \int dz \rho^{\rm rest}_E(z)\delta(\hat{\boldsymbol{\Lambda}}(z)- \boldsymbol{\Lambda})
  \end{align}
  Quite remarkably,  by recourse  to rotational invariance,  in this
  section  we   have  computed   explicitly  the  integral   in  this
  probability  density, with  the result  (\ref{eq:150}), without  the
  need of knowing  the explicit functional form of  the phase function
  $\hat{\boldsymbol{\Lambda}}(z)$.}

\subsection{Rotational symmetry of conditional expectations}
\label{App:RotInv}
In   this    section   we   derive   the    explicit   dependence   on
$\boldsymbol{\Lambda}$ for  the conditional expectations  of tensorial
quantities. Consider  the transformation  of three different  types of
phase functions:  scalar, vectorial,  or tensorial under  an arbitrary
rotation $\boldsymbol{\cal Q}$, respectively
\begin{align}
  \label{W:eq:52}
  \hat{F}\left(Qz\right)
  &=  \hat{F}\left(z\right)
    \nonumber\\
    \hat{\bf V}\left(Qz\right)
  &=   \boldsymbol{\cal Q}\esc\hat{\bf V}\left(z\right)
    \nonumber\\
    \hat{\bf W}\left(Qz\right)
  &=   \boldsymbol{\cal Q}\esc\hat{\bf W}\left(z\right)\esc\boldsymbol{\cal Q}^T
\end{align}
We  focus  our  discussion  on the  ``rest''  conditional  expectation
(\ref{W:Fav2})  as  we  will  consider almost  always  the  change  of
variables (\ref{W:BodyFrame}). The explicit result we will demonstrate in
this section is
\begin{align}
  \label{W:eq:469}
  \llangle \hat{F}\rrangle^{\boldsymbol{\Lambda}{{\bf M}}{\cal E}}_{\rm rest}
  &=F_0({\bf M},{\cal E})
    \nonumber\\
    \llangle \hat{\bf V}\rrangle^{\boldsymbol{\Lambda}{{\bf M}}{\cal E}}_{\rm rest}
  &= e^{[\boldsymbol{\Lambda}]_\times}\esc
    {\bf V}_0({{\bf M}},{\cal E})
    \nonumber\\
    \llangle \hat{\bf W}\rrangle^{\boldsymbol{\Lambda}{{\bf M}}{\cal E}}_{\rm rest}
  &= e^{[\boldsymbol{\Lambda}]_\times}\esc
    {\bf W}_0({{\bf M}},{\cal E})\esc e^{-[\boldsymbol{\Lambda}]_\times}
\end{align}
where $F_0({\bf  M},{\cal E})$,  ${\bf V}_0({{\bf M}},{\cal  E})$, and
${\bf W}_0({{\bf  M}},{\cal E})$  are independent of  the orientation.
Performing  the change  of variables  (\ref{W:eq:182}), and  noting the
Dirac                delta                 functions                of
${\bf R},{{\bf M}},{\bf P},{\bf S},\boldsymbol{\Pi}$, and $E$ are left
invariant  under this  change, the  conditional expectations  of these
three types of functions become
\begin{align}
  \label{W:eq:279}
\llangle\hat{F}\rrangle^{\boldsymbol{\Lambda}{{\bf M}}{\cal E}}_{\rm rest}
  =\frac{1}{ \Omega^{\rm rest}}
  \int dz\hat{F}(z)
  &\delta\left(\hat{\bf R}(z)\right)
    \delta\left(\hat{\boldsymbol{\Lambda}}(Qz)-\boldsymbol{\Lambda}\right)
    \delta\left(\hat{{\bf M}}(z)-{{\bf M}}\right)
    \delta\left(\hat{\bf P}(z)\right)
    \delta\left(\hat{\bf S}(z)\right)
    \delta\left(\hat{\boldsymbol{\Pi}}(z)\right)
    \delta\left(\hat{H}(z)-{\cal E}\right)
    \nonumber\\
\llangle\hat{\bf V}\rrangle^{\boldsymbol{\Lambda}{\bf M}{\cal E}}_{\rm rest}
  =\boldsymbol{\cal Q} \esc \frac{1}{ \Omega^{\rm rest}}
  \int dz\hat{\bf V}(z)
  &\delta\left(\hat{\bf R}(z)\right)
    \delta\left(\hat{\boldsymbol{\Lambda}}(Qz)-\boldsymbol{\Lambda}\right)
    \delta\left(\hat{{\bf M}}(z)-{{\bf M}}\right)
    \delta\left(\hat{\bf P}(z)\right)
    \delta\left(\hat{\bf S}(z)\right)
    \delta\left(\hat{\boldsymbol{\Pi}}(z)\right)
    \delta\left(\hat{H}(z)-{\cal E}\right)
    \nonumber\\
\llangle\hat{\bf W}\rrangle^{\boldsymbol{\Lambda}{\bf M}{\cal E}}_{\rm rest}
  =
\boldsymbol{\cal Q} \esc \frac{1}{ \Omega^{\rm rest}} \int dz\hat{\bf W}(z)
  &\delta\left(\hat{\bf R}(z)\right)
    \delta\left(\hat{\boldsymbol{\Lambda}}(Qz)-\boldsymbol{\Lambda}\right)
    \delta\left(\hat{{\bf M}}(z)-{{\bf M}}\right)
    \delta\left(\hat{\bf P}(z)\right)
    \delta\left(\hat{\bf S}(z)\right)
    \delta\left(\hat{\boldsymbol{\Pi}}(z)\right)
    \delta\left(\hat{H}(z)-{\cal E}\right)\esc\boldsymbol{\cal Q}^T
\end{align}
Now  take  the derivative  of  these  equations  with respect  to  the
arbitrary orientation vector  ${\bf A}$, noting the left  hand side is
independent of ${\bf A}$, to give component-wise
\begin{align}
  \label{W:eq:443}
0
  &=\frac{1}{ \Omega^{\rm rest}}
  \int dz\hat{F}(z)
  \delta\left(\hat{\bf R}(z)\right)\frac{\partial}{\partial {\bf A}^\gamma}
    \delta\left(\hat{\boldsymbol{\Lambda}}(Qz)-\boldsymbol{\Lambda}\right)
    \delta\left(\hat{{\bf M}}(z)-{{\bf M}}\right)
    \delta\left(\hat{\bf P}(z)\right)
    \delta\left(\hat{\bf S}(z)\right)
    \delta\left(\hat{\boldsymbol{\Pi}}(z)\right)
    \delta\left(\hat{H}(z)-{\cal E}\right)
    \nonumber\\
0
  &= \frac{1}{ \Omega^{\rm rest}}
\frac{\partial}{\partial {\bf A}^\gamma} \boldsymbol{\cal Q}^{\alpha\alpha'}
  \int dz\hat{\bf V}^{\alpha'}(z)
  \delta\left(\hat{\bf R}(z)\right)
    \delta\left(\hat{\boldsymbol{\Lambda}}(Qz)-\boldsymbol{\Lambda}\right)
    \delta\left(\hat{{\bf M}}(z)-{{\bf M}}\right)
    \delta\left(\hat{\bf P}(z)\right)
    \delta\left(\hat{\bf S}(z)\right)
    \delta\left(\hat{\boldsymbol{\Pi}}(z)\right)
    \delta\left(\hat{H}(z)-{\cal E}\right)
    \nonumber\\
0
  &=\frac{1}{ \Omega^{\rm rest}}\frac{\partial}{\partial {\bf A}^\gamma}
    {\boldsymbol{\cal Q}}^{\alpha\alpha'} {\boldsymbol{\cal Q}}^{\beta\beta'}
    \nonumber\\
  &\times\int dz\hat{\bf W}^{\alpha'\beta'}(z)
    \delta\left(\hat{\bf R}(z)\right)
    \delta\left(\hat{\boldsymbol{\Lambda}}(Qz)-\boldsymbol{\Lambda}\right)
    \delta\left(\hat{{\bf M}}(z)-{{\bf M}}\right)
    \delta\left(\hat{\bf P}(z)\right)
    \delta\left(\hat{\bf S}(z)\right)
    \delta\left(\hat{\boldsymbol{\Pi}}(z)\right)
    \delta\left(\hat{H}(z)-{\cal E}\right)
\end{align}
For the scalar function, by evaluating the derivative at ${\bf A}=0$, and using (\ref{eq:24}), and (\ref{eq:37}) we have
\begin{align}
  \label{W:eq:444}
  0  &=-\frac{1}{\Omega^{\rm rest}(\boldsymbol{\Lambda},{\bf M},{\cal E})}
    \frac{\partial}{\partial {\Lambda}_{\gamma'}}  {B}_{\gamma'\gamma}(\boldsymbol{\Lambda})
    \Omega^{\rm rest}(\boldsymbol{\Lambda},{{\bf M}},{\cal E})\llangle \hat{F}\rrangle^{\boldsymbol{\Lambda}{{\bf M}}{\cal E}}_{\rm rest}
       =-{B}_{\gamma'\gamma}(\boldsymbol{\Lambda})    \frac{\partial}{\partial {\Lambda}_{\gamma'}}  
  \llangle \hat{F}\rrangle^{\boldsymbol{\Lambda}{{\bf M}}{\cal E}}_{\rm rest}
\end{align}
Because ${\bf  B}(\boldsymbol{\Lambda})$ is non-singular it must be that
\begin{align}
  \label{W:eq:446}
    \frac{\partial}{\partial {\Lambda}_{\gamma'}}  
  \llangle \hat{F}\rrangle^{\boldsymbol{\Lambda}{{\bf M}}{\cal E}}_{\rm rest}
  &=0
\end{align}
which  means  the  conditional  average does  not  depend on  the
orientation, so that
\begin{align}
  \label{eq:321}
    \llangle \hat{F}\rrangle^{\boldsymbol{\Lambda}{{\bf M}}{\cal E}}_{\rm rest}&=F_0({\bf M},{\cal E})
\end{align}
For vectorial functions, evaluating the  derivative at ${\bf A}=0$ for
the second equation of (\ref{W:eq:443})  and using the first relation in
(\ref{eq:457}) gives
\begin{align}
  \label{W:eq:448}
  0
  &=\epsilon_{\alpha\gamma\alpha'}
\llangle \hat{\bf V}^{\alpha'}\rrangle^{\boldsymbol{\Lambda}{{\bf M}}{\cal E}}_{\rm rest}
-    
\frac{1}{ \Omega^{\rm rest}(\boldsymbol{\Lambda},{{\bf M}},{\cal E})}
    \frac{\partial{B}_{\gamma'\gamma}}{\partial {\Lambda}_{\gamma'}}(\boldsymbol{\Lambda})
    \Omega^{\rm rest}(\boldsymbol{\Lambda},{{\bf M}},{\cal E})\llangle \hat{\bf V}^{\alpha}\rrangle^{\boldsymbol{\Lambda}{{\bf M}}{\cal E}}_{\rm rest}
\end{align}
which with the use of (\ref{eq:37}) gives
\begin{align}
  \label{W:eq:449}
  0 &= -\llangle [\hat{\bf V}]_\times^{\alpha\gamma}
    \rrangle^{\boldsymbol{\Lambda}{{\bf M}}{\cal E}}_{\rm rest} -
    {B}_{\gamma'\gamma}(\boldsymbol{\Lambda})
    \frac{\partial}{\partial {\Lambda}_{\gamma'}}
    \llangle \hat{\bf V}^{\alpha}\rrangle^{\boldsymbol{\Lambda}{{\bf M}}{\cal E}}_{\rm rest}
\end{align}
Let us define the vector ${\bf V}_0(\boldsymbol{\Lambda},{{\bf M}},{\cal E})$ through
\begin{align}
  \label{W:eq:466}
  \llangle \hat{\bf V}\rrangle^{\boldsymbol{\Lambda}{{\bf M}}{\cal E}}_{\rm rest}
  &=   e^{[\boldsymbol{\Lambda}]_\times}\esc{\bf V}_0(\boldsymbol{\Lambda},{{\bf M}},{\cal E})
\end{align}
Using this definition in (\ref{W:eq:449}) along with 
(\ref{eq:99}) gives
\begin{align}
0&=-\llangle [\hat{\bf V}]_\times^{\alpha\gamma}
\rrangle^{\boldsymbol{\Lambda}{{\bf M}}{\cal E}}_{\rm rest} -
{B}_{\gamma'\gamma}(\boldsymbol{\Lambda})
\left ( \frac{\partial}{\partial {\Lambda}_{\gamma'}} \left[
e^{[\boldsymbol{\Lambda}]_\times}\right]_{\alpha\alpha'}\right )
{\bf V}^{\alpha'}_0(\boldsymbol{\Lambda},{{\bf M}},{\cal E}) -
{B}_{\gamma'\gamma}(\boldsymbol{\Lambda}) \left[
e^{[\boldsymbol{\Lambda}]_\times}\right]_{\alpha\alpha'}
\frac{\partial}{\partial {\Lambda}_{\gamma'}}
{\bf V}^{\alpha'}_0(\boldsymbol{\Lambda},{{\bf M}},{\cal E})\cr
0&=-\llangle [\hat{\bf V}]_\times^{\alpha\gamma}
\rrangle^{\boldsymbol{\Lambda}{{\bf M}}{\cal E}}_{\rm rest} -
{B}_{\gamma'\gamma}(\boldsymbol{\Lambda}) \epsilon_{\alpha\gamma''\beta'}
\left[e^{[\hat{\boldsymbol{\Lambda}}]_\times}\right]_{\beta'\alpha'}
{B}^{-1}_{\gamma''\gamma'}
{\bf V}^{\alpha'}_0(\boldsymbol{\Lambda},{{\bf M}},{\cal E}) -
{B}_{\gamma'\gamma}(\boldsymbol{\Lambda}) \left[
e^{[\boldsymbol{\Lambda}]_\times}\right]_{\alpha\alpha'}
\frac{\partial}{\partial {\Lambda}_{\gamma'}}
{\bf V}^{\alpha'}_0(\boldsymbol{\Lambda},{{\bf M}},{\cal E})\cr
0&=-\llangle [\hat{\bf V}]_\times^{\alpha\gamma}
\rrangle^{\boldsymbol{\Lambda}{{\bf M}}{\cal E}}_{\rm rest} -
\epsilon_{\alpha\gamma\beta'}
\llangle {\bf V}^{\beta'}\rrangle^{\boldsymbol{\Lambda}{{\bf M}}{\cal E}}_{\rm rest} -
{B}_{\gamma'\gamma}(\boldsymbol{\Lambda}) \left[
e^{[\boldsymbol{\Lambda}]_\times}\right]_{\alpha\alpha'}
\frac{\partial}{\partial {\Lambda}_{\gamma'}}
{\bf V}^{\alpha'}_0(\boldsymbol{\Lambda},{{\bf M}},{\cal E})\cr
0&=-{B}_{\gamma'\gamma}(\boldsymbol{\Lambda}) \left[
e^{[\boldsymbol{\Lambda}]_\times}\right]_{\alpha\alpha'}
\frac{\partial}{\partial {\Lambda}_{\gamma'}}
{\bf V}^{\alpha'}_0(\boldsymbol{\Lambda},{{\bf M}},{\cal E})
\end{align}
from which we conclude the vector ${\bf  V}_0$ is independent of the orientation. 


Proceeding in the same manner for the average of the tensor valued phase
function in (\ref{W:eq:279}) using (\ref{eq:457}) and noting
${\boldsymbol{\cal Q}}^{\alpha\alpha'}|_{{\bf A}=0}=\delta_{\alpha\alpha'}$ gives 
\begin{align}
  \label{W:eq:454}
0
  &=\frac{1}{ \Omega^{\rm rest}}\frac{\partial}{\partial {\bf A}^\gamma}
{\boldsymbol{\cal Q}}^{\alpha\alpha'}{\boldsymbol{\cal Q}}^{\beta\beta'}  \left . \int dz\hat{\bf W}^{\alpha'\beta'}(z)
  \delta\left(\hat{\bf R}(z)\right)
    \delta\left(\hat{\boldsymbol{\Lambda}}(Qz)-\boldsymbol{\Lambda}\right)
    \delta\left(\hat{{\bf M}}(z)-{{\bf M}}\right)
    \delta\left(\hat{\bf P}(z)\right)
    \delta\left(\hat{\bf S}(z)\right)
    \delta\left(\hat{\boldsymbol{\Pi}}(z)\right)
    \delta\left(\hat{H}(z)-{\cal E}\right)\right |_{{\bf A}=0}
    \nonumber\\
  &=\left .\left(\frac{\partial}{\partial {\bf A}^\gamma}
     {\boldsymbol{\cal Q}}^{\alpha\alpha'}{\boldsymbol{\cal Q}}^{\beta\beta'}\right )\right |_{{\bf A}=0}
    \llangle \hat{\bf W}^{\alpha'\beta'}\rrangle^{\boldsymbol{\Lambda}{{\bf M}}{\cal E}}_{\rm rest}
 -
    {B}_{\gamma'\gamma}(\boldsymbol{\Lambda})
    \frac{\partial}{\partial {\Lambda}_{\gamma'}}
\llangle \hat{\bf W}^{\alpha\beta}\rrangle^{\boldsymbol{\Lambda}{{\bf M}}{\cal E}}_{\rm rest}
    \nonumber\\
  &=\left(\epsilon_{\alpha\gamma\alpha'}\delta_{\beta\beta'}
    +\delta_{\alpha\alpha'}\epsilon_{\beta\gamma\beta'}\right)
    \llangle \hat{\bf W}^{\alpha'\beta'}\rrangle^{\boldsymbol{\Lambda}{{\bf M}}{\cal E}}_{\rm rest}
  -
    {B}_{\gamma'\gamma}(\boldsymbol{\Lambda})
    \frac{\partial}{\partial {\Lambda}_{\gamma'}}
\llangle \hat{\bf W}^{\alpha\beta}\rrangle^{\boldsymbol{\Lambda}{{\bf M}}{\cal E}}_{\rm rest}
    \nonumber\\
  &=\epsilon_{\alpha\gamma\alpha'}
    \llangle \hat{\bf W}^{\alpha'\beta}\rrangle^{\boldsymbol{\Lambda}{{\bf M}}{\cal E}}_{\rm rest}
    +\epsilon_{\beta\gamma\beta'}
    \llangle \hat{\bf W}^{\alpha\beta'}\rrangle^{\boldsymbol{\Lambda}{{\bf M}}{\cal E}}_{\rm rest}
-
    {B}_{\gamma'\gamma}(\boldsymbol{\Lambda})
    \frac{\partial}{\partial {\Lambda}_{\gamma'}}
\llangle \hat{\bf W}^{\alpha\beta}\rrangle^{\boldsymbol{\Lambda}{{\bf M}}{\cal E}}_{\rm rest}
\end{align}
Let us introduce the tensor  ${\bf W}_0(\boldsymbol{\Lambda},{{\bf M}},{\cal E})$ through
\begin{align}
  \label{W:eq:455}
  \llangle \hat{\bf W}\rrangle^{\boldsymbol{\Lambda}{{\bf M}}{\cal E}}_{\rm rest}
  &\equiv e^{[\boldsymbol{\Lambda}]_\times}\esc
    {\bf W}_0(\boldsymbol{\Lambda},{{\bf M}},{\cal E})\esc e^{-[\boldsymbol{\Lambda}]_\times}
\end{align}



The last term of (\ref{W:eq:454}) along with
relations (\ref{eq:99}) and (\ref{eq:183}) gives
\begin{align}
\label{W:eq:456}
&{B}_{\gamma'\gamma}(\boldsymbol{\Lambda})
 \frac{\partial}{\partial {\Lambda}_{\gamma'}}
\llangle \hat{\bf W}^{\alpha\beta}\rrangle^{\boldsymbol{\Lambda}{{\bf M}}{\cal E}}_{\rm rest}\nonumber\\
&=\left ( \frac{\partial}{\partial {\Lambda}_{\gamma'}}
\left[ e^{[\boldsymbol{\Lambda}]_\times}\right]_{\alpha\alpha''}\right )
{B}_{\gamma'\gamma}(\boldsymbol{\Lambda}) {\bf W}^{\alpha''\beta''}_0
\left[ e^{-[\boldsymbol{\Lambda}]_\times}\right ]^{\beta''\beta}
+ \left[ e^{[\boldsymbol{\Lambda}]_\times}\right]_{\alpha\alpha''}
{\bf W}^{\alpha''\beta''}_0
\left ( \frac{\partial}{\partial {\Lambda}_{\gamma'}}
\left[e^{-[\boldsymbol{\Lambda}]_\times}\right]_{\beta''\beta}\right )
{B}_{\gamma'\gamma}(\boldsymbol{\Lambda})\nonumber\\
&\quad+{B}_{\gamma'\gamma}(\boldsymbol{\Lambda})
\left [ e^{[\boldsymbol{\Lambda}]_\times}\right ]^{\alpha\alpha''}
    \left(\frac{\partial}{\partial {\Lambda}_{\gamma'}}
    {\bf W}^{\alpha''\beta''}_0\right)
\left[e^{-[\boldsymbol{\Lambda}]_\times}\right]_{\beta''\beta}\nonumber\\
&=\epsilon_{\alpha\gamma''\alpha'}
\left[e^{[\hat{\boldsymbol{\Lambda}}]_\times}\right]_{\alpha'\alpha''}
\left[{B}^{-1}(\boldsymbol{\Lambda})\right]_{\gamma''\gamma'}
{B}_{\gamma'\gamma}(\boldsymbol{\Lambda}) {\bf W}^{\alpha''\beta''}_0
\left[ e^{-[\boldsymbol{\Lambda}]_\times}\right ]^{\beta''\beta}\nonumber\\
&\quad+ \left[ e^{[\boldsymbol{\Lambda}]_\times}\right]_{\alpha\alpha''}
{\bf W}^{\alpha''\beta''}_0
\epsilon_{\beta\gamma''\beta'}
\left[e^{-[\hat{\boldsymbol{\Lambda}}]_\times}\right]_{\beta''\beta'}
\left[{B}^{-1}(\boldsymbol{\Lambda})\right]_{\gamma''\gamma'}
{B}_{\gamma'\gamma}(\boldsymbol{\Lambda})\nonumber\\
&\quad+{B}_{\gamma'\gamma}(\boldsymbol{\Lambda})
\left [ e^{[\boldsymbol{\Lambda}]_\times}\right ]^{\alpha\alpha''}
    \left(\frac{\partial}{\partial {\Lambda}_{\gamma'}}
    {\bf W}^{\alpha''\beta''}_0\right)
\left[e^{-[\boldsymbol{\Lambda}]_\times}\right]_{\beta''\beta}\nonumber\\
&=\epsilon_{\alpha\gamma\alpha'}
  \llangle{\bf W}^{\alpha'\beta}\rrangle^{\boldsymbol{\Lambda}{{\bf M}}{\cal E}}_{\rm rest}
+ \epsilon_{\beta\gamma\beta'}
\llangle {\bf W}^{\alpha\beta'}\rrangle^{\boldsymbol{\Lambda}{{\bf M}}{\cal E}}_{\rm rest}
 +{B}_{\gamma'\gamma}(\boldsymbol{\Lambda})
  \left[ e^{[\boldsymbol{\Lambda}]_\times}\right]_{\alpha\alpha''}
    \left(\frac{\partial}{\partial {\Lambda}_{\gamma'}}
    {\bf W}^{\alpha''\beta''}_0\right)
  \left[e^{-[\boldsymbol{\Lambda}]_\times}\right]_{\beta''\beta}
\end{align}
Inserting this result into (\ref{W:eq:454}) then gives
\begin{align}
  \label{W:eq:462}
  {B}_{\gamma'\gamma}(\boldsymbol{\Lambda})\left[ e^{[\boldsymbol{\Lambda}]_\times}\right]_{\alpha\alpha''}
    \left(\frac{\partial}{\partial {\Lambda}_{\gamma'}}
  {\bf W}^{\alpha''\beta''}_0\right)\left[e^{-[\boldsymbol{\Lambda}]_\times}\right]_{\beta''\beta}
  &=0
\end{align}
Because ${\bf B}$, $e^{[\boldsymbol{\Lambda}]_\times}$, and
$e^{-[\boldsymbol{\Lambda}]_\times}$ are non-singular matrices, this is
only possible if
\begin{align}
  \label{W:eq:465}
\frac{\partial}{\partial {\Lambda}_{\gamma'}}
    {\bf W}^{\alpha''\beta''}_0 &=0
\end{align}
which means the tensor ${\bf W}_0$ is independent of the orientation.

In conclusion,
the conditional expectation of scalar, vector, and tensor valued
phase functions is of the form (\ref{W:eq:469}).
Expressions for the quantities $F_0({\bf M},{\cal E})$, ${\bf  V}_0({{\bf M}},{\cal E})$,
and ${\bf W}_0({{\bf M}},{\cal  E})$ can be obtained by writing (\ref{W:eq:469}) as
\begin{align}
  \label{W:eq:470}
F_0({\bf M},{\cal E})
  &=   \llangle \hat{F}\rrangle^{\boldsymbol{\Lambda}{{\bf M}}{\cal E}}_{\rm rest}
    \nonumber\\
    {\bf V}_0({{\bf M}},{\cal E})
  &= 
    \llangle e^{-[\hat{\boldsymbol{\Lambda}}]_\times}\esc \hat{\bf V}\rrangle^{\boldsymbol{\Lambda}{{\bf M}}{\cal E}}_{\rm rest}
    \nonumber\\
    {\bf W}_0({{\bf M}},{\cal E})
  &= 
    \llangle e^{-[\hat{\boldsymbol{\Lambda}}]_\times}\esc\hat{\bf W}
\esc e^{[\hat{\boldsymbol{\Lambda}}]_\times}\rrangle^{\boldsymbol{\Lambda}{{\bf M}}{\cal E}}_{\rm rest}
\end{align}
where we  have introduced the  rotation matrix inside  the conditional
expectations by virtue of the Dirac delta function on the orientation.
The  main conclusion  of this  section is,  therefore, that  the rest
conditional  expectation of  a  tensor (of  any  order),  when
expressed  in  the   principal  axis  frame, is   independent  of  the
orientation.


%\newpage
\setcounter{my_equation_counter}{\value{equation}}
\section{Calculation       of      the      probability
  $P_{\cal E}^{\rm   rest}({\bf M},\boldsymbol{\Pi})$   in   the   Gaussian
  approximation}
\setcounter{equation}{\value{my_equation_counter}}
\label{App:PGauss}
Even  though  the  momentum   integrals  in  (\ref{eq:151})  with  the
microcanonical ensemble  (\ref{W:eq:289}) can be  explicitly computed,
simpler expressions are  obtained if we use a  canonical ensemble.  In
fact,   for  large   bodies   we  may   approximate  the   probability
(\ref{eq:151}) with
\begin{align}
  \label{eq:22}
  P^{\rm rest}_{\beta}({{\bf M}},\boldsymbol{\Pi})
  &=\int dz \rho^{\rm rest}_{\beta}(z)
    \delta\left(\hat{{\bf M}}(z)-{{\bf M}}\right)
    \delta(\hat{\boldsymbol{\Pi}}(z)-\boldsymbol{\Pi}) 
\end{align}
where the rest canonical ensemble is
\begin{align}
  \label{eq:16}
  \rho^{\rm rest}_{\beta}(z)
  &=\frac{1}{Z(\beta)}
    \delta\left(\hat{\bf R}(z) \right)
    \delta\left(\hat{\bf P}(z) \right)
    \delta\left(\hat{\bf S}(z) \right)
    e^{-\beta\hat{H}(z)}
\end{align}
To appreciate  the rigorous  connection between the  two probabilities
(\ref{eq:151}) and (\ref{eq:22}), insert      the      identity
\begin{align}
  \label{eq:450}
  \int d{\cal E}\;\delta(\hat{H}(z)-{\cal E})&=1
\end{align}
 inside   (\ref{eq:22}) to obtain
\begin{align}
  \label{eq:178}
  P^{\rm rest}_{\beta}({{\bf M}},\boldsymbol{\Pi})
  &=\int d{\cal E}\; P_\beta({\cal E})\;  P^{\rm rest}_{\cal {\cal E}}({{\bf M}},\boldsymbol{\Pi})
\end{align}
where the normalized probability is 
\begin{align}
  \label{eq:179}
  P_\beta({\cal E})&= \frac{e^{-\beta {\cal E}}\Omega^{\rm MT}({\cal E}) }{Z(\beta)}=  \frac{e^{S_B^{\rm MT}({\cal E})/k_B-\beta {\cal E}}}{Z(\beta)}  
\end{align}
This is the probability a system in the canonical ensemble has a value
of the energy given by ${\cal E}$, i.e.
\begin{align}
  \label{eq:50}
    P_\beta({\cal E})&=\int dz \rho^{\rm res}_\beta(z)\delta(\hat{H}(z)-{\cal E})
\end{align}
This probability is expected to be  highly peaked because both ${\cal E}$ and
$S^{\rm  MT}({\cal E})$ are  extensive quantities  expected to  scale as  the
number  $N$  of particles  in  the  body.  Setting the  derivative  of
(\ref{eq:179}) to  zero, the  maximum of  $P_\beta({\cal E})$ occurs  at ${\cal E}^*$
which is the solution of
\begin{align}
  \label{eq:180}
\beta&= \frac{1}{k_B} \frac{\partial S_B^{\rm MT}}{\partial {\cal E}}({\cal E}^*)= \frac{1}{k_BT^{\rm MT}({\cal E}^*)}
\end{align}
In the thermodynamic limit, $P_\beta({\cal E})\simeq\delta({\cal E}-{\cal E}^*)$ and
\begin{align}
  \label{eq:181}
  P^{\rm rest}_{\cal {\cal E}}({{\bf M}},\boldsymbol{\Pi})
  &\simeq P^{\rm rest}_{\beta({\cal E})}({{\bf M}},\boldsymbol{\Pi})
\end{align}
where $\beta({\cal E})$ is given by (\ref{eq:180}).
Now return to  (\ref{eq:22}), where we may separate position and momentum integrals as
\begin{align}
      \label{eq:161b}
      P^{\rm rest}_{\beta}({{\bf M}},\boldsymbol{\Pi})
      &=\frac{1}{Z(\beta)}\int dr     e^{-\beta\hat{\Phi}(z)}\delta\left(\hat{{\bf M}}(z)-{{\bf M}}\right)
        \delta\left(\hat{\bf R}(z) \right)
    \int dp e^{-\beta\sum_i\frac{{\bf p}^2_i}{2m_i}}    \delta\left(\hat{\bf P}(z) \right)
    \delta\left(\hat{\bf S}(z) \right)
\delta(\hat{\boldsymbol{\Pi}}(z)-\boldsymbol{\Pi}) 
\end{align}
Using (\ref{G0}), the momentum integral is $G^0({\bf P}=0,{\bf S}=0,\boldsymbol{\Pi},\beta)$ whose explicit expression in (\ref{eq:244}) gives
\begin{align}
  \label{eq:64b}
          P^{\rm rest}_{\beta}({{\bf M}},\boldsymbol{\Pi})
  &=\frac{1}{Z(\beta)}\int dr     e^{-\beta\hat{\Phi}(r)}\delta\left(\hat{{\bf M}}(r)-{{\bf M}}\right)
        \delta\left(\hat{\bf R}(r) \right)
    \nonumber\\
  &\times\left(\frac{\beta}{2\pi}\right)^{9/2}\prod_i^{N}\left(\frac{2\pi m_i}{\beta}\right)^{3/2}\frac{1}{{\rm M}^{3/2}
    \left(\det\hat{\bf I}(r)\right)^{1/2}
    \left(\det\hat{\bf T}(r)\right)^{1/2}}
    \exp\left\{-\frac{\beta}{2}\boldsymbol{\Pi}^T\esc\hat{\bf T}^{-1}(r)\esc\boldsymbol{\Pi}
    \right\}
\end{align}
The                                                        determinant
$\det\hat{\bf  I}=\hat{I}_1\hat{I}_2\hat{I}_3$  and,  as
shown     in     (\ref{TG}),     the     principal     mass     matrix
$\hat{\bf T}(z)=\hat{\mathbb{G}}(z)$  are functions of  the microstate
only through the  central moments.  The presence of  the Dirac delta
function  on  $\hat{{\bf  M}}$  allows one  to  substitute  the  phase
functions   $\hat{\bf   M}$   with    the   numerical   values   ${\bf
  M}$. Therefore, we have
\begin{align}
  \label{eq:70}
            P^{\rm rest}_{\beta}({{\bf M}},\boldsymbol{\Pi})
  &=\frac{1}{Z(\beta)}\left(\frac{\beta}{2\pi}\right)^{9/2}\prod_i^{N}\left(\frac{2\pi m_i}{\beta}\right)^{3/2}
    \frac{1}{{\rm M}^{3/2}    \left(I_1 I_2 I_3\right)^{1/2}}
    \frac{1}{  \left(M_1 M_2 M_3\right)^{1/2}}
\exp\left\{-\beta
    \sum_\alpha\frac{\Pi_\alpha^2}{2{ M}_\alpha}\right\}
    \nonumber\\
  &\times\int dr     e^{-\beta\hat{\Phi}(r)}\delta\left(\hat{{\bf M}}(r)-{{\bf M}}\right)
        \delta\left(\hat{\bf R}(r) \right)
\end{align}
Of  course, we  could do  the momentum  integrals in $Z(\beta)$ and some cancellations  of factors would
occur, but this is not necessary. Note that 
\begin{align}
  \label{eq:92}
  P^{\rm rest}_{\beta}({{\bf M}},\boldsymbol{\Pi})&=
  P^{\rm rest}_{\beta}({{\bf M}})  G^{\rm rest}_{\beta}({{\bf M}},\boldsymbol{\Pi})
\end{align}
where  the  normalized  Gaussian   probability  of  dilation  momentum
(conditional on ${{\bf M}}$) is
\begin{align}
  \label{eq:414}
    G^{\rm rest}_{\beta}({{\bf M}},\boldsymbol{\Pi})
  &=\frac{\beta^{3/2}}{(2\pi)^{3/2}(M_1 M_2 M_3)^{1/2}}
    \exp\left\{-\beta
    \sum_\alpha\frac{\Pi_\alpha^2}{2{M}_\alpha}\right\}
\end{align}

The physical meaning of $ P^{\rm rest}_{\beta}({{\bf M}})$ is simply the
marginal  equilibrium probability  of  finding  the principal  moments
${{\bf M}}$.  In fact,  because this probability  is normalized,  we may
integrate (\ref{eq:92}) with respect to $\boldsymbol{\Pi}$ and obtain
\begin{align}
  \label{eq:88}
  P^{\rm rest}_{\beta}({{\bf M}})&=\int d\boldsymbol{\Pi}
              P^{\rm rest}_{\beta}({{\bf M}},\boldsymbol{\Pi})
=\int dz \rho^{\rm rest}_{\beta}(z)
\delta\left(\hat{{\bf M}}(z)-{{\bf M}}\right)                                   
\end{align}
This probability could be measured,  in principle, from MD simulations
but  we  will consider  here  a  simple  approximation in  which  this
probability  has  a Gaussian  form.  For  example,  if we  assume  the
temperature  of the  body is  low  and particles  are executing  small
oscillations  around a  minimum, we  may expand  the potential  energy
around such a  minimum resulting in a Gaussian of  positions.  In this
way, the  resulting integral  is a  ``sectioned Gaussian''  similar to
(\ref{eq:134}).   The situation  is  somewhat  more delicate,  because
rotational and translational invariance  of the potential energy leads
to a  non-invertible matrix for  which (\ref{eq:134}) does  not exist.
Nevertheless, the calculation  can be performed and the  end result is
still  that  the  probability  $P^{\rm rest}_\beta({{\bf  M}})$  is  a
Gaussian of the form
\begin{align}
  \label{eq:110}
  P^{\rm rest}_\beta({\bf M})
  &=
    \frac{\beta^{3/2}}{\det(2\pi {\boldsymbol{\Sigma}})^{1/2}}
    \exp\left\{-\frac{\beta}{2}({{\bf M}}-{\bf M}^{\rm rest})^T\esc{\boldsymbol{\Sigma}}^{-1}\esc({{\bf M}}-{\bf M}^{\rm rest})\right\}
\end{align}
where  ${\bf M}^{\rm  rest}$ is  the rest  equilibrium average  of the
central moments  and $\boldsymbol{\Sigma}$ is  given by  the covariance  of the
central moments
\begin{align}
  \label{Cov}
\boldsymbol{\Sigma}&=\frac{1}{k_BT^{\rm MT}}\llangle  ({{\bf M}}-{\bf M}^{\rm rest})({{\bf M}}-{\bf M}^{\rm rest})^T\rrangle^{\rm eq}
\end{align}
\setcounter{my_equation_counter}{\value{equation}}
\section{Calculation of thermodynamic forces}
\setcounter{equation}{\value{my_equation_counter}}
\label{App:ThermForces}
In this section we show that
\begin{align}
  \label{App:7}
  {T}  \frac{\partial S_B}{\partial{\boldsymbol{\Lambda}}}
  &=-{\bf B}^{-T}\esc(\boldsymbol{\Omega}\times{\bf S}) +k_BT
    \frac{\partial}{\partial \boldsymbol{\Lambda}}\ln P^{\rm Haar}(\boldsymbol{\Lambda})
\\
  \label{App:380}
  T    \frac{\partial S_B}{\partial M_\alpha}
  &=\frac{1}{2}\nu_\alpha^2
    +2\left(    \boldsymbol{\Omega}_p^T\esc{\boldsymbol{\Omega}_p}-( {\Omega_p^{\alpha}})^2 \right)
  +    k_BT\frac{\partial}{\partial {
    M}_\alpha}\ln P^{\rm rest}_{\cal E}({{\bf M}},{\bf 0})
\end{align}
 The dependence of
the entropy (\ref{W:eq:282}) on $\boldsymbol{\Lambda}$ and ${\bf M}$ appears
explicitly through the probability and implicitly through the thermal energy
(\ref{Thermal-App}).  Using the chain rule then gives 
\begin{align}
  \label{eq:187}
  \frac{\partial S_B}{\partial{\boldsymbol{\Lambda}}}
  &= \frac{1}{T}
    \frac{\partial {\cal E}}{\partial{\boldsymbol{\Lambda}}} +
    k_B\frac{\partial}{\partial{\boldsymbol{\Lambda}}}\ln P^{\rm Haar}(\boldsymbol{\Lambda})
{\overset{(\ref{eq:150})}{=}} -\frac{1}{T}\frac{\partial K^{\rm rot}}{\partial\boldsymbol{\Lambda}}
+    k_B\frac{\partial}{\partial{\boldsymbol{\Lambda}}}\ln P^{\rm Haar}(\boldsymbol{\Lambda})
    \nonumber\\
  &{\overset{(\ref{eq:192})}{=}}-\frac{1}{T}{\bf B}^{-T}\esc(\boldsymbol{\Omega}\times{\bf S})
  + k_B\frac{\partial}{\partial{\boldsymbol{\Lambda}}}\ln P^{\rm Haar}(\boldsymbol{\Lambda})
    \\
  \label{eq:187d}  \frac{\partial S_B}{\partial{{\bf M}}}
  &= \frac{1}{T}
    \frac{\partial {\cal E}}{\partial{{\bf M}}} +
    k_B\frac{\partial}{\partial{{\bf M}}}
    \ln P^{\rm rest}_{\cal E} ({\boldsymbol{\Lambda}},{\bf M},{\bf 0})
\overset{(\ref{Thermal-App}) (\ref{eq:119})}{  =}-\frac{1}{T}  \frac{\partial }{\partial{{\bf M}}}
    \left(K^{\rm rot}+K^{\rm dil}\right) +
    k_B\frac{\partial}{\partial{{\bf M}}}
    \ln P^{\rm rest}_{\cal E} ({\bf M},{\bf 0})
\end{align}
where  ${\bf B}^{-T}$  denotes the  transpose  of the  inverse of  the
matrix ${\bf B}$. The derivative of  the logarithm of the Haar measure
is
\begin{align}
  \label{eq:365}
\frac{\partial}{\partial{\boldsymbol{\Lambda}}}\ln P^{\rm Haar}(\boldsymbol{\Lambda})
  &=-2f(\Lambda)\boldsymbol{\Lambda}
\end{align}
where $f(\Lambda)$
\begin{align}
  \label{eq:262}
  f(\Lambda) = \frac{1}{\Lambda^2}
\left ( 1 - \frac{\Lambda}{2}\cot\frac{\Lambda}{2}\right )
\end{align}

The derivatives of the kinetic energies with respect central moments are
\begin{align}
  \label{eq:375}
  \frac{\partial K^{\rm rot}}{\partial{{\bf M}}}  
  &=\frac{1}{2}
    \left(e^{-[\boldsymbol{\Lambda}]_\times}\esc{\bf S}\right)^T\esc
    \frac{\partial {\mathbb{I}}^{-1}}{\partial{{\bf M}}}\esc
    \left(e^{-[\boldsymbol{\Lambda}]_\times}\esc{\bf S}\right)
    \nonumber\\
    \frac{\partial K^{\rm dil}}{\partial {M}_\alpha}  &=-
\frac{1}{2}{\nu}_\alpha^2
\end{align}
Let us find a more explicit form of the derivative of the rotational kinetic energy by
looking at its components
\begin{align}
  \label{eq:376}
    \frac{\partial K^{\rm rot}}{\partial {M}_\alpha}  
  &=-\frac{1}{2}\left[e^{-[\boldsymbol{\Lambda}]_\times}\esc{\bf S}\right]^\mu
    \left[\mathbb{I}^{-1}\right]_{\mu\mu'}
    \frac{\partial \left[\mathbb{I}\right]_{\mu'\nu'}}{\partial {M}_\alpha}
    \left[\mathbb{I}^{-1}\right]_{\nu'\nu}
    \left[e^{-[\boldsymbol{\Lambda}]_\times}\esc{\bf S}\right]^\nu
\end{align}
By using that the intertia tensor can be expressed  in terms of
the gyration tensor $\hat{\mathbb{I}}= 4\left({\rm Tr}[\hat{\mathbb{G}}]\mathbb{1}-\hat{\mathbb{G}}\right)$,
we may evaluate the tensor
\begin{align}
  \label{eq:408}
\sum_{\mu'\nu'}      \left[\mathbb{I}^{-1}\right]_{\mu\mu'}
    \frac{\partial \left[\mathbb{I}\right]_{\mu'\nu'}}{\partial {M}_\alpha}
    \left[\mathbb{I}^{-1}\right]_{\nu'\nu}
  &=
\sum_{\mu'\nu'}      \left[\mathbb{I}^{-1}\right]_{\mu\mu'}
{\black  4}\left(\delta_{\mu'\nu'}   -\delta_{\mu'\nu'\alpha}\right)
    \left[\mathbb{I}^{-1}\right]_{\nu'\nu}
    \nonumber\\
&    ={\black  4}
\sum_{\mu'}      \left[\mathbb{I}^{-1}\right]_{\mu\mu'}
    \left[\mathbb{I}^{-1}\right]_{\mu'\nu}
-     \left[\mathbb{I}^{-1}\right]_{\mu\underline{\alpha}}
    \left[\mathbb{I}^{-1}\right]_{\underline{\alpha}\nu}
                  ={\black  4}\left(\frac{1}{I_{\underline{\mu}}^2}
\delta_{\underline{\mu}\nu}-\frac{1}
{I_{\underline{\alpha}}^2}
                  \delta_{\underline{\alpha}\mu}\delta_{\underline{\alpha}\nu}\right)
\end{align}
where the underlines indicate indices are not summed over.  Using
(\ref{eq:408}) in (\ref{eq:376}) gives
\begin{align}
  \label{eq:409}
      \frac{\partial K^{\rm rot}}{\partial {M}_\alpha} 
  &=-\sum_{\mu\nu}
    \frac{1}{2}\left[e^{-[\boldsymbol{\Lambda}]_\times}\esc{\bf S}\right]^\mu
    {\black  4}\left(
\frac{1}{I_{\mu}^2}\delta_{{\mu}\nu}-
\frac{1}{I_{\underline{\alpha}}^2}
                  \delta_{{\mu}\underline{\alpha}}\delta_{{\nu}\underline{\alpha}}
    \right)
    \left[e^{-[\boldsymbol{\Lambda}]_\times}\esc{\bf S}\right]^\nu
    \nonumber\\
  &=-{\black  2}\sum_\mu\left[e^{-[\boldsymbol{\Lambda}]_\times}\esc{\bf S}\right]^\mu
    \frac{1}{I_{\mu}^2}
    \left[e^{-[\boldsymbol{\Lambda}]_\times}\esc{\bf S}\right]^\mu
    +\frac{2}{I_{\underline{\alpha}}^2}
    \left[e^{-[\boldsymbol{\Lambda}]_\times}\esc{\bf S}\right]_{\underline{\alpha}}
    \left[e^{-[\boldsymbol{\Lambda}]_\times}\esc{\bf S}\right]_{\underline{\alpha}}
    \nonumber\\
   &=2\left( ( \boldsymbol{\Omega}_p^{\alpha})^2 - {\boldsymbol{\Omega}_p}^T\esc{\boldsymbol{\Omega}_p}\right)
\end{align}
where %using (\ref{eq:200}) gives
\begin{align}
  \label{eq:372}
  {\boldsymbol{\Omega}_p}\equiv
  % \boldsymbol{\omega}_0 + \boldsymbol{\Omega}_0 = {\mathbb{I}}^{-1}\esc e^{-[\boldsymbol{\Lambda}]_\times}\esc{\bf S} =
  e^{-[\boldsymbol{\Lambda}]_\times}\esc\boldsymbol{\Omega}
 \end{align}

Note that $\boldsymbol{\Omega}^T\esc\boldsymbol{\Omega}=\boldsymbol{\Omega}_p^T \esc\boldsymbol{\Omega}_p$.
Collecting terms together then gives
\begin{align}
  \label{eq:358}
T    \frac{\partial S_B}{\partial {M}_\alpha}
  &=2\left(    \boldsymbol{\Omega}_p^T\esc \boldsymbol{\Omega}_p
    -( \boldsymbol{\Omega}_p^\alpha )^2 \right)
+\frac{1}{2}{\nu}_\alpha^2
+    k_BT\frac{\partial\ln P^{\rm rest}_{\cal E}}{\partial{{M}_\alpha}}({{\bf M}},{\bf 0})
\end{align}


\setcounter{my_equation_counter}{\value{equation}}
\section{Calculation of the reversible terms }
\setcounter{equation}{\value{my_equation_counter}}
\label{App:ReviLLambda}

Consider first  the reversible  term corresponding to  the orientation
equation.   The   transform  of   conditional  expectations   to  rest
conditional    expectations    (\ref{eq:43}),    (\ref{eq:35}),    and
(\ref{W:BodyFrameInv})   gives   the    conditional   expectation   of
$i{\cal L}\hat{\boldsymbol{\Lambda}}$ as
\begin{align}
    \llangle  i{\cal L}\hat{\boldsymbol{\Lambda}}\rrangle^a
  &=
  \llangle {\cal G}
  \sum_{i}\frac{\partial \hat{\boldsymbol{\Lambda}}}{\partial {\bf r}_{i}}\esc{\bf v}_{i}\rrangle^{\boldsymbol{\Lambda}{\bf M}{\cal E}}_{\rm rest}
=\llangle
    \sum_{i}\frac{\partial \hat{\boldsymbol{\Lambda}}}{\partial {\bf r}'_{i}}\esc
        \left[{\bf v}'_{i}+{\bf V} - [{\bf r}'_{i}]_\times\esc\hat{\boldsymbol{\Omega}}
    +\frac{1}{m_i}\frac{\partial \hat{{\bf M}}}{\partial {\bf r}'_{i}}\esc\hat{\boldsymbol{\nu}}
    \right]
\rrangle^{\boldsymbol{\Lambda}{\bf M}{\cal E}}_{\rm rest}
\end{align}
The first  term, which is  linear in  ${\bf v}'_i$ gives  a conditional
average  where the  integrand  is  odd under  the  change of  variable
${\bf v}'_i\to  -{\bf v}'_i$, so  vanishes by symmetry. The  second term
with ${\bf V}$ vanishes because  of the translation invariant property
(\ref{trans}). The  fourth term  involving the dilation  velocity also
vanishes once (\ref{eq:H})  and (\ref{eq:449}) are used.  In the third
term with the spin velocity we use (\ref{eq:255}) to get
\begin{align}
  \label{eq:264}
  \llangle  i{\cal L}\hat{\boldsymbol{\Lambda}}\rrangle^a
  &=
    \llangle{\bf B}(\hat{\boldsymbol{\Lambda}}(z'))\esc\hat{\boldsymbol{\Omega}(z')}
    \rrangle^{\boldsymbol{\Lambda}{\bf M}{\cal E}}_{\rm rest}
=
  {\bf B}({\boldsymbol{\Lambda}})\esc{\boldsymbol{\Omega}}
\end{align}
where we have used $\langle f(\hat{A})\rangle^{a}=f(a)$.

Next, let us  consider the reversible term corresponding to the dilational
momentum.  The dilational force is $\hat{\boldsymbol{\cal K}}=i{\cal L}\hat{\boldsymbol{\Pi}}=i{\cal L}^2{\bf M}$, this is
\begin{align}
 \label{eq:91}
\hat{\boldsymbol{\cal K}}
  &=\sum_i\left ( \frac{\partial\hat{\boldsymbol{\Pi}}}{\partial{\bf r}_i}\esc {\bf v}_i
+\frac{1}{m_i}\frac{\partial\hat{\boldsymbol{\Pi}}}{\partial{\bf v}_i}\esc \hat{\bf F}_i
\right )
  =\sum_{i j}\frac{\partial^2 \hat{{\bf M}}}{\partial {\bf r}_{i}\partial {\bf r}_{j}}:{\bf v}_{i}{\bf v}_{j}
   +\sum_{i}\frac{1}{m_i}\frac{\partial \hat{{\bf M}}}{\partial {\bf r}_{i}}\esc\hat{\bf F}_{i }
\end{align}
in which $\hat{\bf F}_i$ is the force on particle $i$.  Applying $\cal G$ to
produce the change of variables (\ref{W:BodyFrameInv}) gives
\begin{align}
  \label{eq:94}
  {\cal G}\hat{\boldsymbol{\cal K}}
  =&-\sum_{i j}\left[
({\bf r}'_{i}\times\hat{\boldsymbol{\Omega}})\esc
\frac{\partial^2 \hat{{{\bf M}}}}{\partial {\bf r}'_{i}\partial {\bf r}'_{j}}
\esc\frac{1}{m_j}\frac{\partial \hat{{\bf M}}}{\partial {\bf r}'_{j}}
\esc\hat{\boldsymbol{\nu}}
+\frac{1}{m_i}\frac{\partial \hat{{\bf M}}}{\partial {\bf r}'_{i}}
\esc\hat{\boldsymbol{\nu}}\esc
\frac{\partial^2 \hat{{{\bf M}}}}{\partial {\bf r}'_{i}\partial {\bf r}'_{j}}
\esc ({\bf r}'_{j}\times\hat{\boldsymbol{\Omega}})\right]\nonumber\\
& +\sum_{i j}\left[({\bf r}'_{i}\times\hat{\boldsymbol{\Omega}})\esc
\frac{\partial^2 \hat{{{\bf M}}}}{\partial {\bf r}'_{i}\partial {\bf r}'_{j}}
\esc ({\bf r}'_{j}\times\hat{\boldsymbol{\Omega}})
+\hat{\boldsymbol{\nu}}\esc\frac{1}{m_i}\frac{\partial \hat{{\bf M}}}{\partial {\bf r}'_{i}}
\esc
\frac{\partial^2 \hat{{{\bf M}}}}{\partial {\bf r}'_{i}\partial {\bf r}'_{j}}
\esc\frac{1}{m_j}\frac{\partial \hat{{\bf M}}}{\partial {\bf r}'_{j}}
\esc\hat{\boldsymbol{\nu}}\right ]
\end{align}
where we  have already eliminated all  terms with $\bf V$  that vanish
due to  (\ref{trans2}) and all terms linear in the
  velocity  ${\bf v}_i'$, for which  the  corresponding conditional  average
  vanish by symmetry.
The first term is
\begin{align}
{\hat{\boldsymbol{\Omega}}}^\beta\sum_{i j}[{\bf r}'_i]_\times^{\gamma\beta}
\frac{\partial^2 \hat{M}_\alpha}
{\partial {{\bf r}'_i}^\gamma\partial {{\bf r}'_j}^{\gamma'}}
\frac{1}{m_j}\frac{\partial \hat{M}_{\alpha'}}
{\partial {{\bf r}'_j}^{\gamma'}} \hat{\nu}_{\alpha'}
        \overset{(\ref{eq:267})}{=}
{\hat{\boldsymbol{\Omega}}}^\beta              \sum_{ j}  \epsilon_{\gamma'\gamma\beta}  \frac{\partial\hat{M}_\alpha}{\partial {\bf r}^{\gamma'}_j}
      \frac{1}{m_j}\frac{\partial \hat{M}_{\alpha'}}
{\partial {{\bf r}'_j}^{\gamma}} \hat{\nu}_{\alpha'}
\end{align} while the second term is identical to this one, but with a minus sign. Therefore, the first two terms of (\ref{eq:94}) cancel each other. 
The remaining terms are
%
\begin{align}
\sum_{i j}[{\bf r}'_i]_\times^{\gamma\beta}{\hat{\boldsymbol{\Omega}}}^\beta
\frac{\partial^2 \hat{M}_\alpha}
{\partial {{\bf r}'_i}^\gamma\partial {{\bf r}'_j}^{\gamma'}}
[{\bf r}'_j]_\times^{\gamma'\beta'}{\hat{\boldsymbol{\Omega}}}^{\beta'}
&\overset{(\ref{eq:267})}{=}\sum_i
{\hat{\boldsymbol{\Omega}}}^\beta {\hat{\boldsymbol{\Omega}}}^{\beta'}
[{\bf r}'_i]_\times^{\gamma\beta}\epsilon_{\gamma\gamma'\beta'}
\frac{\partial\hat{M}_\alpha}{\partial {{\bf r}'_i}^{\gamma'}}
= \sum_i {\hat{\boldsymbol{\Omega}}}^\beta {\hat{\boldsymbol{\Omega}}}^{\beta'}
[{\bf r}'_i]_\times^{\alpha'}\epsilon_{\gamma\alpha'\beta}\epsilon_{\gamma\gamma'\beta'}
\frac{\partial\hat{M}_\alpha}{\partial {{\bf r}'_i}^{\gamma'}}\nonumber\\
&\overset{(\ref{eq:604})}{=} \sum_i {\hat{\boldsymbol{\Omega}}}^\beta {\hat{\boldsymbol{\Omega}}}^{\beta'}
{{\bf r}'_i}^{\alpha'}
\left [ \delta_{\alpha'\gamma'}\delta_{\beta\beta'}-
\delta_{\alpha'\beta'}\delta_{\beta\gamma'}\right ]
\frac{\partial\hat{M}_\alpha}{\partial {{\bf r}'_i}^{\gamma'}}\nonumber\\
&=\sum_i\left [
{\hat{\boldsymbol{\Omega}}}^\beta {\hat{\boldsymbol{\Omega}}}^\beta
{{\bf r}'_i}^{\gamma'}
\frac{\partial\hat{M}_\alpha}{\partial {{\bf r}'_i}^{\gamma'}}
-
{\hat{\boldsymbol{\Omega}}}^\beta {\hat{\boldsymbol{\Omega}}}^{\beta'}
{{\bf r}'_i}^{\beta'}
\frac{\partial\hat{M}_\alpha}{\partial {{\bf r}'_i}^\beta}
\right] \nonumber\\
&\overset{(\ref{eq:256})}{=}2
{\hat{\boldsymbol{\Omega}}}^\beta {\hat{\boldsymbol{\Omega}}}^\beta
\left[e^{-[\hat{\boldsymbol{\Lambda}}]_\times}\right]_{\underline{\alpha}\beta'}
\hat{\bf G}_{\beta'\gamma'}
\left[e^{-[\hat{\boldsymbol{\Lambda}}]_\times}\right]_{\underline{\alpha}\gamma'}
-2 {\hat{\boldsymbol{\Omega}}}^\beta {\hat{\boldsymbol{\Omega}}}^{\beta'}
\left[e^{-[\hat{\boldsymbol{\Lambda}}]_\times}\right]_{\underline{\alpha}\beta}
\left[e^{-[\hat{\boldsymbol{\Lambda}}]_\times}\right]_{\underline{\alpha}\gamma'}
\hat{\bf G}_{\gamma'\beta'}\nonumber\\
&=2 \hat{\boldsymbol{\Omega}}^{{T}}\esc \hat{\boldsymbol{\Omega}}
\hat{M}_\alpha - 2
\left[e^{-[\hat{\boldsymbol{\Lambda}}]_\times}\right]_{\underline{\alpha}\beta}
{\hat{\boldsymbol{\Omega}}}^\beta
\left[e^{-[\hat{\boldsymbol{\Lambda}}]_\times}\right]_{\underline{\alpha}\beta'}
{\hat{\boldsymbol{\Omega}}}^{\beta'}
\hat{M}_{\underline{\alpha}}\nonumber\\
&\overset{(\ref{eq:372})}{=}
2\left (\hat{\boldsymbol{\Omega}}_p^{{T}}\esc \hat{\boldsymbol{\Omega}}_p-
{\hat{\boldsymbol{\Omega}}_p}^{\underline{\alpha}}
{\hat{\boldsymbol{\Omega}}_p}^{\underline{\alpha}}\right)
\hat{M}_{\underline{\alpha}}(z')
\end{align}
and
\begin{align}
 \hat{\nu}_{\beta'}\sum_{i j} \frac{1}{m_i}\frac{\partial \hat{M}_{\beta'}}
{\partial {{\bf r}'_i}^\gamma}
\frac{\partial^2 \hat{M}_\alpha}
{\partial {{\bf r}'_i}^\gamma\partial {{\bf r}'_j}^{\gamma'}}
\frac{1}{m_j}\frac{\partial \hat{M}_{\alpha'}}
{\partial {{\bf r}'_j}^{\gamma'}} \hat{\nu}_{\alpha'}
&\overset{(\ref{eq:42b})}{=} \frac{1}{2}\sum_i \frac{1}{m_i}\frac{\partial \hat{M}_{\beta'}}
{\partial {{\bf r}'_i}^\gamma} \hat{\nu}_{\beta'}
\frac{\partial \hat{\bf T}^{\alpha\alpha'}}{\partial {{\bf r}'_i}^\gamma}
\hat{\nu}_{\alpha'}
\overset{(\ref{TG})}{=}\frac{1}{2} \hat{\nu}_{\beta'}
\hat{\nu}_{\underline{\alpha}}
\sum_i \frac{1}{m_i}\frac{\partial \hat{M}_{\beta'}}
{\partial {{\bf r}'_i}^\gamma}
\frac{\partial \hat{M}_{\underline{\alpha}}}{\partial {{\bf r}'_i}^\gamma}
\nonumber\\
&\overset{(\ref{eq:42b})}{=}\frac{1}{2} \hat{\nu}_{\beta'}
\hat{\nu}_{\underline{\alpha}}
\hat{\bf T}^{\beta'\underline{\alpha}}(z')
\overset{(\ref{TG})}{=}\frac{1}{2} \hat{\nu}_{\underline{\alpha}}
\hat{\nu}_{\underline{\alpha}}
\hat{M}_{\underline{\alpha}}(z')
\overset{(\ref{eq:85})}{=}\frac{1}{2}\hat{{\Pi}}_{\underline{\alpha}}
\hat{\nu}_{\underline{\alpha}}
\end{align}
Collecting all these results together then gives
%
\begin{align}
 \label{eq:194}
  {\cal G}\hat{\boldsymbol{\cal K}}
&=\hat{\boldsymbol{\cal K}}(z') -2\sum_{i j}{\bf v}'_{i}\esc
\frac{\partial^2 \hat{{{\bf M}}}}{\partial {\bf r}'_{i}\partial {\bf r}'_{j}}
\esc({\bf r}'_{j}\times\hat{\boldsymbol{\Omega}}) +
\frac{1}{2}\hat{{\Pi}}_{\underline{\alpha}}(z')
\hat{\nu}_{\underline{\alpha}}
+2\left (\hat{\boldsymbol{\Omega}}_p^{{T}}\esc \hat{\boldsymbol{\Omega}}_p-
{\hat{\boldsymbol{\Omega}}_p}^{\underline{\alpha}}
{\hat{\boldsymbol{\Omega}}_p}^{\underline{\alpha}}\right)
\hat{M}_{\underline{\alpha}}(z')
+\frac{1}{2} \hat{\nu}_{\underline{\alpha}}
\hat{\nu}_{\underline{\alpha}}
\hat{M}_{\underline{\alpha}}(z')
\end{align}
which when used in (\ref{eq:43}) gives the conditional expectation of the
dilational force to be
%
\begin{align}
  \label{eq:107}
  \llangle  \hat{\cal K}_\alpha\rrangle^a
  &= \llangle {\cal G}\hat{\cal K}_\alpha\rrangle^{\boldsymbol{\Lambda}{\bf M}{\cal E}}_{\rm rest}
  = \llangle \hat{\cal K}_\alpha\rrangle^{\boldsymbol{\Lambda}{\bf M}{\cal E}}_{\rm rest}  +
   2\left(\boldsymbol{\Omega}_p^{{T}}\esc\boldsymbol{\Omega}_p
  -\boldsymbol{\Omega}_p^{\underline{\alpha}}
    \boldsymbol{\Omega}_p^{\underline{\alpha}}
    \right){M}_{\underline{\alpha}}
    +  \frac{1}{2} \nu^2_{\underline{\alpha}}{M}_{\underline{\alpha}}
\end{align}
%
where the term linear in ${\bf v}'_i$ vanishes due to symmetry and the other
terms depend on the microstate only through the CG variables.
The    rest   conditional    average   of    the   dilational    force
$\llangle\hat{\boldsymbol{\cal K}}\rrangle^{\boldsymbol{\Lambda}{\bf M}{\cal E}}_{\rm  rest}$
involves  the  pair  forces between
particles, and an explicit expression  is difficult to obtain by brute
force, except  for very simple  linear force laws, i.e.   the harmonic
solid.


\setcounter{my_equation_counter}{\value{equation}}
\section{The reversibility condition}
\setcounter{equation}{\value{my_equation_counter}}
\label{App:RevCond}
 We collect  the reversible drift
in the form
\begin{align}
  \label{eq:62}
V(a)&\to  \left(
      \begin{array}{c}
        \llangle i{\cal L}\hat{\bf R}\rrangle^a
        \\
        \\
        \llangle i{\cal L}\hat{\boldsymbol{\Lambda}}\rrangle^a
        \\
        \\
        \llangle i{\cal L}\hat{{\bf M}}\rrangle^a
        \\
        \\
        \llangle i{\cal L}\hat{\bf P}\rrangle^a
        \\
        \\
        \llangle i{\cal L}\hat{\bf S}\rrangle^a
        \\
        \\
        \llangle i{\cal L}\hat{\boldsymbol{\Pi}}\rrangle^a
        \\
        \\
        \llangle i{\cal L}\hat{H}\rrangle^a
      \end{array}
  \right)
  =\left(
      \begin{array}{c}
       {\bf V}
        \\
        \\
{\bf B}\esc{\boldsymbol{\Omega}}
        \\
        \\
        \boldsymbol{\Pi}
        \\
        \\
        0
        \\
        \\
        0
        \\
        \\
        {\boldsymbol{\cal K}}
        \\
        \\
        0
      \end{array}
  \right)
\end{align}
The reversibility  condition
\begin{align}
  V^T(a)\esc\frac{\partial S_B}{\partial a}(a)+  k_B\frac{\partial \esc V}{\partial a}(a)=0
\end{align}
in this case takes the explicit form
\begin{align}
  \label{eq:394}
&  \llangle i{\cal L}\hat{\bf R}\rrangle^a\esc
    \frac{\partial S_B}{\partial{\bf R}}
    +   \llangle i{\cal L}\hat{\boldsymbol{\Lambda}}\rrangle^a\esc
    \frac{\partial S_B}{\partial{\boldsymbol{\Lambda}}}
    +        \llangle i{\cal L}\hat{{\bf M}}\rrangle^a\esc
    \frac{\partial S_B}{\partial{{\bf M}}}
    +   \llangle i{\cal L}\hat{\bf P}\rrangle^a\esc
     \frac{\partial S_B}{\partial{\bf P}}
     +        \llangle i{\cal L}\hat{\bf S}\rrangle^a\esc
     \frac{\partial S_B}{\partial{\bf S}}
     +        \llangle i{\cal L}\hat{\boldsymbol{\Pi}}\rrangle^a\esc
     \frac{\partial S_B}{\partial{\boldsymbol{\Pi}}}
+     \llangle i{\cal L}\hat{H}\rrangle^a\esc
     \frac{\partial S_B}{\partial E}
                                             \nonumber\\
  +&k_B        \frac{\partial }{\partial{\bf R}}\esc
     \llangle i{\cal L}\hat{\bf R}\rrangle^a
     +k_B        \frac{\partial }{\partial{\boldsymbol{\Lambda}}}\esc
     \llangle i{\cal L}\hat{\boldsymbol{\Lambda}}\rrangle^a
     +k_B        \frac{\partial }{\partial{{\bf M}}}\esc
     \llangle i{\cal L}\hat{{\bf M}}\rrangle^a
  +k_B        \frac{\partial }{\partial{\bf P}}\esc
     \llangle i{\cal L}\hat{\bf P}\rrangle^a
     +k_B        \frac{\partial }{\partial{\bf S}}\esc
     \llangle i{\cal L}\hat{\bf S}\rrangle^a
     +k_B     
     \frac{\partial }{\partial{\boldsymbol{\Pi}}}\esc
     \llangle i{\cal L}\hat{\boldsymbol{\Pi}}\rrangle^a
     \nonumber\\
  +& k_B             \frac{\partial }{\partial E}
     \llangle i{\cal L}\hat{H}\rrangle^a=0
\end{align}
which using (\ref{eq:163}) and (\ref{eq:62}) simplifies to
\begin{align}
  \label{eq:2}
  0  &=
       \frac{\partial S_B}{\partial{\boldsymbol{\Lambda}}}\esc{\bf B}\esc{\boldsymbol{\Omega}}
       +  \boldsymbol{\Pi} \esc         \frac{\partial S_B}{\partial{{\bf M}}}
       -      {\boldsymbol{\cal K}}\esc\frac{\boldsymbol{\nu}}{T}
       + k_B\frac{\partial}{\partial\boldsymbol{\Lambda}}
       \esc\left(
       {\bf B}({\boldsymbol{\Lambda}})\esc{\boldsymbol{\Omega}}\right)
  +   k_B\frac{\partial}{\partial\boldsymbol{\Pi}}\esc      {\boldsymbol{\cal K}}
\end{align}
By  using (\ref{W:eq:282}),  (\ref{W:eq:266}), and  (\ref{eq:25}), the
reversibility condition further simplifies to
\begin{align}
  \label{eq:236}
    0  &=
         \boldsymbol{\Pi} \esc         \frac{\partial S_B}{\partial{{\bf M}}}
       -      \frac{\boldsymbol{\nu}}{T}\esc{\boldsymbol{\cal K}}
              +   k_B\frac{\partial}{\partial\boldsymbol{\Pi}}\esc      {\boldsymbol{\cal K}}
\end{align}
For the last term  we  use   (\ref{eq:107})
\begin{align}
  \label{eq:112}
\frac{\partial}{\partial{\Pi}_\alpha}    \llangle  \hat{\cal K}_\alpha\rrangle^a
  &=\frac{\partial{\cal E}}{\partial{\Pi}_\alpha}\frac{\partial}{\partial{\cal E}}\llangle \hat{\cal K}_\alpha\rrangle^{\boldsymbol{\Lambda}{\bf M}{\cal E}}_{\rm rest}
    +   \sum_\alpha  {\nu}_{\alpha}
\overset{(\ref{Thermal-App})}{=}
    -{\nu}_{\alpha}\frac{\partial}{\partial{\cal E}}\llangle \hat{\cal K}_\alpha\rrangle^{\boldsymbol{\Lambda}{\bf M}{\cal E}}_{\rm rest}
    + \sum_\alpha  {\nu}_{\alpha}
\end{align}
which can be written in a compact form as
\begin{align}
  \label{eq:114}
  \frac{\partial}{\partial\boldsymbol{\Pi}}\esc    \llangle  \hat{\boldsymbol{\cal K}}\rrangle^a
  &=    \boldsymbol{\nu}\esc\left(-\frac{\partial}{\partial{\cal E}}\llangle \hat{\boldsymbol{\cal K}}\rrangle^{\boldsymbol{\Lambda}{\bf M}{\cal E}}_{\rm rest}+{\bf c}\right)
\end{align}
where ${\bf c}=(1,1,1)$. Using this in (\ref{eq:236}) along with (\ref{eq:85}) gives
\begin{align}
  \label{eq:75}
        0  &=
         \boldsymbol{\nu} \esc \left(T{\bf T}\esc        \frac{\partial S_B}{\partial{{\bf M}}}
       -      {\boldsymbol{\cal K}}
+ k_BT\left(-\frac{\partial}{\partial{\cal E}}\llangle \hat{\boldsymbol{\cal K}}\rrangle^{\boldsymbol{\Lambda}{\bf M}{\cal E}}_{\rm rest}+{\bf c}\right)\right)
\end{align}
This identity should be true for arbitrary values of $\boldsymbol{\nu}$ and we conclude
\begin{align}
  \label{eq:120}
{\boldsymbol{\cal K}}  &=
-k_BT\frac{\partial}{\partial{\cal E}}\llangle \hat{\boldsymbol{\cal K}}\rrangle^{\boldsymbol{\Lambda}{\bf M}{\cal E}}_{\rm rest}+k_BT{\bf c}+  {\bf T}\esc    T    \frac{\partial S_B}{\partial{{\bf M}}}
\end{align}
By using the form of the  mean dilational force (\ref{eq:107}) and the
derivative   of  the   entropy  (\ref{eq:358}),   a  quite   fortunate
cancellation of ``convective'' terms occur and (\ref{eq:120}) becomes
\begin{align}
  \label{eq:129}
  \frac{\partial}{\partial{\cal E}}\llangle \hat{\boldsymbol{\cal K}}\rrangle^{\boldsymbol{\Lambda}{\bf M}{\cal E}}_{\rm rest}
    &=-\frac{1}{k_BT} \llangle \hat{\boldsymbol{\cal K}}\rrangle^{\boldsymbol{\Lambda}{\bf M}{\cal E}}_{\rm rest}  
   +\left( {\bf T}\esc\frac{\partial}{\partial{{\bf M}}}\ln P^{\rm rest}_{\cal E}({{\bf M}},{\bf 0})+{\bf c}\right)
\end{align}
This  can  be understood  as  an  ordinary differential  equation  for
$\llangle  \hat{\boldsymbol{\cal K}}\rrangle^{\boldsymbol{\Lambda}{\bf
    M}{\cal E}}_{\rm rest} $ with the structure
\begin{align}
  \label{eq:221}
  x'(t)=-f(t)x(t)+g(t)
\end{align}
and with general solution
\begin{align}
  \label{eq:243}
  x(t)=e^{-\int_{t_0}^tf(t')dt'}\left(x(0)+\int_{t_0}^tdt''e^{\int_0^{t''}f(t')dt'}g(t'')\right)
\end{align}
% as can be checked straightforwardly
% \begin{align}
%   \label{eq:291}
%   x'(t)&=f(t) e^{\int_{t_0}^tf(t')dt'}\left(x(0)+\int_{t_0}^tdt''e^{-\int_0^{t''}f(t')dt'}g(t'')\right)
%          +e^{\int_{t_0}^tf(t')dt'}e^{-\int_0^{t}f(t')dt'}g(t)
% \end{align}
Therefore, the solution of (\ref{eq:129}) is
\begin{align}
  \label{eq:292}
  \llangle \hat{\boldsymbol{\cal K}}\rrangle^{\boldsymbol{\Lambda}{\bf M}{\cal E}}_{\rm rest}
  &=e^{-\int_0^{\cal E}d{\cal E}'\frac{1}{k_BT({\cal E}')}}
    \left(\llangle \hat{\boldsymbol{\cal K}}\rrangle^{\boldsymbol{\Lambda}{\bf M}0}_{\rm rest}
    +    \int_0^{\cal E}d{\cal E}''e^{\int_0^{{\cal E}''}d{\cal E}'\frac{1}{k_BT({\cal E}')}}
\left( {\bf T}\esc\frac{\partial}{\partial{{\bf M}}}\ln P^{\rm rest}_{{\cal E}''}({{\bf M}},{\bf 0})+{\bf c}\right)    \right)
\end{align}
The definition of temperature as the derivative of the entropy with respect to the energy  implies
\begin{align}
  \label{eq:307}
  \int_0^{\cal E}d{\cal E}'\frac{1}{k_BT({\cal E}')}=\frac{S_B({\cal E})-S_B(0)}{k_B}
\end{align}
where we leave implicit the rest of the variable dependencies. Also, the value of the entropy and of any conditional expectation at the zero energy should vanish, so that
\begin{align}
  \label{eq:326}
  S_B(0)&=0
          \nonumber\\
    \llangle \hat{\boldsymbol{\cal K}}\rrangle^{\boldsymbol{\Lambda}{\bf M}0}_{\rm rest}&=0
\end{align}
Therefore, (\ref{eq:292}) takes the form
\begin{align}
  \label{eq:330}
    \llangle \hat{\boldsymbol{\cal K}}\rrangle^{\boldsymbol{\Lambda}{\bf M}{\cal E}}_{\rm rest}
  &=    \int_0^{\cal E}d{\cal E}''e^{\frac{S_B({\cal E}'')-S_B({\cal E})}{k_B}}
\left( {\bf T}\esc\frac{\partial}{\partial{{\bf M}}}\ln P^{\rm rest}_{{\cal E}''}({{\bf M}},{\bf 0})+{\bf c}\right)
\end{align}
The prefactor containing the entropies is, by using the exact result (\ref{W:eq:287})
\begin{align}
  \label{eq:341}
 e^{\frac{S_B({\cal E}'')-S_B({\cal E})}{k_B}}
  & =  e^{ \frac{S^{\rm MT}({\cal E}'')- S^{\rm MT}({\cal E})}{k_B}}
\frac{ P^{\rm rest}_{{\cal E}''}({{\bf M}},{\bf 0})    }{ P^{\rm rest}_{{\cal E}}({{\bf M}},{\bf 0})    }
\end{align}
so that
\begin{align}
  \label{eq:342}
      \llangle \hat{\boldsymbol{\cal K}}\rrangle^{\boldsymbol{\Lambda}{\bf M}{\cal E}}_{\rm rest}
  &=    \int_0^{\cal E}d{\cal E}'' e^{ \frac{S^{\rm MT}({\cal E}'')- S^{\rm MT}({\cal E})}{k_B}}
\frac{ P^{\rm rest}_{{\cal E}''}({{\bf M}},{\bf 0})    }{ P^{\rm rest}_{{\cal E}}({{\bf M}},{\bf 0})    }
\left( {\bf T}\esc\frac{\partial}{\partial{{\bf M}}}\ln P^{\rm rest}_{{\cal E}''}({{\bf M}},{\bf 0})+{\bf c}\right)
\end{align}
No approximations have  been taken in order to arrive  at this result,
which is rigorous and exact. Observe that the mean dilational force at
rest is independent of the orientation.  It is a simple calculation to
show    (\ref{eq:342})  satisfies  the   reversibility  condition
(\ref{eq:129}).

Of  course,  (\ref{eq:342})  is  still  formal,  because  it  requires
knowledge  of  the functional  forms  of  $S^{\rm MT}({\cal  E})$  and
$P^{\rm  rest}_{{\cal  E}}({\bf  M},0)$.   Once they  are  known,  the
additional non-trivial integral over the  energy needs to be performed
explicitly. This is not easy to  do in general, and some approximation
is required  in order to have  a more explicit result.   We assume the
macroscopic entropy  $S^{\rm MT}({\cal E})$ is  an increasing function
of the thermal energy in such  a way that the temperature $T^{\rm MT}$
is always positive.  We also assume the entropy scales  as the size of
the body and, therefore, for a relatively large body (composed of more
than,       say,        ten       particles)        the       function
$e^{\frac{S^{\rm  MT}({\cal  E}'')-S^{\rm   MT}({\cal  E})}{k_B}}$  is
highly  peaked  around ${\cal  E}''={\cal  E}$.   For example,  for  a
harmonic  solid for  which  the macroscopic  thermodynamic entropy  is
$  S^{\rm MT}(E)=3Nk_B\ln E+{\rm ctn}$ this factor is
\begin{align}
  \label{eq:344}
  e^{\frac{S^{\rm MT}({\cal E}'')-S^{\rm MT}({\cal E})}{k_B}}&=\left(\frac{{\cal E}''}{{\cal E}}\right)^{3N}
\end{align}
As we  increase $N$ for  ${\cal E}''<{\cal  E}$, the region  for which
this function  is significantly different from  zero gets concentrated
around ${\cal E}''\simeq {\cal E}$.  Therefore, we may approximate the
integral in (\ref{eq:342}) as
\begin{align}
  \label{eq:331}
      \llangle \hat{\boldsymbol{\cal K}}\rrangle^{\boldsymbol{\Lambda}{\bf M}{\cal E}}_{\rm rest}
  &\simeq\left(        \int_0^{\cal E}d{\cal E}''
e^{\frac{S^{\rm MT}({\cal E}'')-S^{\rm MT}({\cal E})}{k_B}}\right)
\left( {\bf T}\esc\frac{\partial}{\partial{{\bf M}}}\ln P^{\rm rest}_{\cal E}({{\bf M}},{\bf 0})+{\bf c}\right)
\end{align}
For the harmonic solid this integral is
\begin{align}
  \label{eq:346}
    \int_0^{\cal E}d{\cal E}''e^{\frac{S^{\rm MT}({\cal E}'')-S^{\rm MT}({\cal E})}{k_B}}
  &=\int_0^{\cal E}d{\cal E}''\left(\frac{{\cal E}''}{{\cal E}}\right)^{3N}=
{\cal E}    \int_0^1dx x^{3N}=\frac{{\cal E}  }{3N-1}\simeq k_BT^{\rm MT}
\end{align}
For more general models of the entropy we may expand the entropy to first order and write the prefactor as
\begin{align}
  \label{eq:347}
  \int_0^{\cal E}d{\cal E}''e^{\frac{S^{\rm MT}({\cal E}'')-S^{\rm MT}({\cal E})}{k_B}}\simeq
      \int_0^{\cal E}d{\cal E}''e^{-\beta({\cal E}-{\cal E}'')}=\frac{1}{\beta}\left(1-e^{-\beta{\cal E}}\right)\simeq k_BT^{\rm MT}
\end{align}
where we have taken ${\cal E}/k_BT^{\rm MT}\gg1$ for a macroscopic
body.   Therefore,  approximating  the   integral  of  the exponential  of
entropies with the temperature seems to be a good approximation. In this way,
(\ref{eq:331}) becomes
\begin{align}
  \label{eq:38}
      \llangle \hat{\boldsymbol{\cal K}}\rrangle^{\boldsymbol{\Lambda}{\bf M}{\cal E}}_{\rm rest}
  &\simeq k_BT^{\rm MT}\left( {\bf T}\esc\frac{\partial}{\partial{{\bf M}}}\ln P^{\rm rest}_{\cal E}({{\bf M}},{\bf 0})+{\bf c}\right)
\end{align}
This is a remarkable result  relating the rest conditional expectation
of  the dilational  force with  the derivative  of the  probability of
central  moments.  Using  (\ref{TG})  for the  matrix  ${\bf T}$,  the
$\alpha$  component of  the  mean dilational  force in  (\ref{eq:107})
becomes
\begin{align}
  \label{eq:46}
{\cal K}_\alpha
  &={M}_{\underline{\alpha}}\left( \frac{1}{2}{\nu}_{\underline{\alpha}}^2
    +
2\left(  \boldsymbol{\Omega}_p^T\esc\boldsymbol{\Omega}_p-({\boldsymbol{\Omega}_p^{\underline{\alpha}}})^2\right)\right)
    +k_BT^{\rm MT}\left({M}_{\underline{\alpha}}     \frac{\partial}{\partial{M}_{\underline{\alpha}}}\ln
    P^{\rm rest}_{\cal E}({{\bf M}},{\bf 0})+1\right)
\end{align}
where underlined repeated indices  do not follow Einstein's convention
and are  not summed over.  In  particular, for the Gaussian  model for
$P^{\rm  rest}_{\cal  E}({{\bf  M}},{\bf  0})$  we  obtain  the  fully
explicit result
\begin{align}
  \label{eq:121b}
  {\cal K}_\alpha
  &={M}_{\underline{\alpha}}\left(\frac{1}{2}{\nu}_{\underline{\alpha}}^2
+ 2\left(  {\boldsymbol{\Omega}_p^T\esc\boldsymbol{\Omega}_p-(\boldsymbol{\Omega}_p^{\underline{\alpha}})^2}\right)
+\frac{k_BT^{\rm MT}}{2{M}_{\underline{\alpha}}}\right)
    -{M}_{\underline{\alpha}}[{\boldsymbol{\Sigma}}^{-1}]^{\underline{\alpha}\beta}({M}_\beta-{M}_\beta^{\rm rest})
\end{align}

\newpage\color{black}
\setcounter{my_equation_counter}{\value{equation}}
\section{Calculation of the dissipative matrix}
\setcounter{equation}{\value{my_equation_counter}}
\label{App:Gamma}

\subsection{The element $\boldsymbol{\Gamma}_{_\Lambda}$ of the dissipative matrix}
We compute first the element $\boldsymbol{\Gamma}_{_\Lambda}$ of the 
dissipative  matrix  defined through the Green-Kubo formula
\begin{align}
  \label{eq:196}
    {\boldsymbol{\Gamma}}_{_\Lambda}(a)
  &=  \frac{1}{k_BT}  \int_0^\infty dt
\llangle {\cal Q}i{\cal L}\hat{\boldsymbol{\Lambda}}(0)
{\cal Q}i{\cal L}\hat{\boldsymbol{\Lambda}}^T(t)\rrangle^{a}
\end{align}
where $ {\cal Q}i{\cal L}\hat{\boldsymbol{\Lambda}}(t)$ stands for
\begin{align}
  \label{eq:329}
\exp\{i{\cal Q}{\cal L}t\}{\cal Q}i{\cal L}\hat{\boldsymbol{\Lambda}}(z)\simeq \exp\{i{\cal L}t\}{\cal Q}i{\cal L}\hat{\boldsymbol{\Lambda}}(z)
    ={\cal Q}i{\cal L}\hat{\boldsymbol{\Lambda}}(z_t)
\end{align}
where we have taken the usual approximation of replacing the projected dynamics with the real dynamics,
and $z_t$ is  the  solution of  Hamilton's  equations with  initial
condition $z$.  Therefore, the conditional correlation  in (\ref{eq:196}) is an average
over initial conditions that correspond  to the specific values $a$ of
the  CG variables  $\hat{A}(z)$. Performing the  change of  variables
(\ref{W:BodyFrameInv})  
\begin{align}
\label{eq:308}
i{\cal L}\hat{\boldsymbol{\Lambda}}= \sum_{i}
\frac{\partial \hat{\boldsymbol{\Lambda}}}{\partial{\bf r}_{i}}\esc{\bf v}_{i}
=\sum_{i}\left[\frac{\partial \hat{\boldsymbol{\Lambda}}}{\partial {\bf r}_{i}'}
\esc{\bf v}_{i}'
+ \frac{\partial \hat{\boldsymbol{\Lambda}}}{\partial {\bf r}_{i}'}\esc{\bf V} 
- \frac{\partial \hat{\boldsymbol{\Lambda}}}{\partial {\bf r}_{i}'}
  \esc ({\bf r}_{i}'\times\hat{\boldsymbol{\Omega}})
  +{ \frac{\partial \hat{\boldsymbol{\Lambda}}}{\partial {\bf r}_{i}'}}\frac{\partial \hat{{\bf M}}}{\partial {\bf r}'_{i}}\esc\hat{\boldsymbol{\nu}}
  \right]
&=\sum_{i} \frac{\partial \hat{\boldsymbol{\Lambda}}}{\partial {\bf r}_{i}'}
\esc{\bf v}_{i}' +
{\bf B}(\hat{\boldsymbol{\Lambda}})\esc\hat{\boldsymbol{\Omega}}
  \nonumber\\
  &=\sum_{i} \frac{\partial \hat{\boldsymbol{\Lambda}}}{\partial {\bf r}_{i}'}
\esc{\bf v}_{i}' +
\llangle i{\cal L}{\hat{\boldsymbol{\Lambda}}}\rrangle^{\hat{A}}
\end{align}
where (\ref{trans}) was used to eliminate the term with ${\bf V}$, (\ref{eq:H}), (\ref{eq:449}) to
eliminate the term with $\hat{\boldsymbol{\nu}}$, and (\ref{eq:255}) and
(\ref{eq:264})  to rewrite the last term.
Rearranging this equation gives the projected current for the orientation
\begin{align}
\label{eq:309}
{\cal Q}i{\cal L}\hat{\boldsymbol{\Lambda}} =
i{\cal L}{\hat{\boldsymbol{\Lambda}}}-
\llangle i{\cal L}{\hat{\boldsymbol{\Lambda}}}\rrangle^{\hat{A}}  =
\sum_{i}\frac{\partial \hat{{\boldsymbol{\Lambda}}}}{\partial {\bf r}_{i}'}
\esc{\bf v}_{i}'\overset{(\ref{eq:35})}{=}{\cal G}i{\cal L}{\hat{\boldsymbol{\Lambda}}}
\end{align}
Using this relation in (\ref{eq:43}) and noting ${\cal G}^2=1$ then gives
\begin{align}
  \label{eq:7-S}
  \llangle {\cal Q}i{\cal L}{\hat{\boldsymbol{\Lambda}}}(0){\cal Q}i{\cal L}
  {\hat{\boldsymbol{\Lambda}}}^T(t) \rrangle^a
  &=  \llangle i{\cal L}{\hat{\boldsymbol{\Lambda}}}(0)i{\cal L}
    {\hat{\boldsymbol{\Lambda}}}^T(t) \rrangle^{\boldsymbol{\Lambda}{{\bf M}}{\cal E}}_{\rm rest}    
\end{align}
In words,  this equation  states that  the conditional  correlation of
projected  currents  ${\cal Q}i{\cal  L}{\hat{\boldsymbol{\Lambda}}}$,
conditional                                                         on
$a={\bf       R},\boldsymbol{\Lambda},{\bf      M},{\bf       P},{\bf
  S},\boldsymbol{\Pi},E$, is  the same  as the rest  frame conditional
correlation          of          the          time          derivative
$i{\cal   L}\hat{\boldsymbol{\Lambda}}$,  which   is  conditional   on
$a'=\left\{{\bf    R}=0,\boldsymbol{\Lambda},{\bf    M},    {\bf    P}=0,{\bf
  S}=0,\boldsymbol{\Pi}={\bf 0},{\cal  E}\right\}$.  Using (\ref{eq:283}) to write
$i{\cal                              L}\hat{\boldsymbol{\Lambda}}=\dot{\hat{\boldsymbol{\Lambda}}}={\bf
  B}(\hat{\boldsymbol{\Lambda}})   \hat{\boldsymbol{\omega}}$, the 
dissipative matrix   $\boldsymbol{\Gamma}_{_\Lambda}(a)$ in (\ref{eq:196}) becomes
\begin{align}
  \label{eq:193}
      \boldsymbol{\Gamma}_{_\Lambda}(a)
  &=
    {\bf B}({\boldsymbol{\Lambda}})\esc
    \boldsymbol{\cal D}({\boldsymbol{\Lambda}},{{\bf M}}, {\cal E})
    \esc{\bf B}^T({\boldsymbol{\Lambda}}) 
\end{align}
  where we have introduced the angular diffusion tensor
\begin{align}
  \label{eq:430}
    \boldsymbol{\cal D}(\boldsymbol{\Lambda},{{\bf M}}, {\cal E})
  &    \equiv \frac{1}{k_BT}\int_0^\infty dt
    \llangle\hat{\boldsymbol{\omega}}(0)\hat{\boldsymbol{\omega}}^{T}(t)
    \rrangle^{\boldsymbol{\Lambda}{{\bf M}}{\cal E}}_{\rm rest} 
\end{align}
Because the angular velocity $\hat{\boldsymbol{\omega}}$ transforms as
a                  vector,                 the                  dyadic
$\hat{\boldsymbol{\omega}}(0)\hat{\boldsymbol{\omega}}^{T}(t)$
transforms as  a tensor.   This is  true even  for the  time dependent
angular    velocity   $\hat{\boldsymbol{\omega}}(t)$,    because   the
Hamiltonian   is    invariant   under   rotations.    As    shown   by
(\ref{W:eq:469})  and (\ref{W:eq:470})  in Sec.  \ref{App:RotInv},
this tensor takes the form
\begin{align}
  \label{eq:355}
      \boldsymbol{\cal D}(\boldsymbol{\Lambda},{\bf M},{\cal E})
  &=e^{[{{\boldsymbol{\Lambda}}}]_\times}\esc
    \boldsymbol{\cal D}_0({{\bf M}},{\cal E})\esc e^{-[{\boldsymbol{\Lambda}}]_\times}
\end{align}
where the ``rest'' angular diffusion tensor is
\begin{align}
      \label{W:eq:356}
  \boldsymbol{\cal D}_0({{\bf M}},{\cal E})
  &\equiv \frac{1}{k_BT}\int_0^\infty dt
    \llangle\hat{\boldsymbol{\omega}}_{0}(0)\hat{\boldsymbol{\omega}}_{0}^{T}(t)
    \rrangle^{\boldsymbol{\Lambda}{{\bf M}}{\cal E}}_{\rm rest}
\end{align}
where $\hat{\boldsymbol{\omega}}_0=e^{-[\hat{\boldsymbol{\Lambda}}]_\times}\hat{\boldsymbol{\omega}}$.  Inserting these expressions in (\ref{eq:193}) then gives
\begin{align}
  \label{eq:130}
      \boldsymbol{\Gamma}_{_\Lambda}(a)
  &=
    {\bf B}({\boldsymbol{\Lambda}})\esc
e^{[{{\boldsymbol{\Lambda}}}]_\times}\esc
    \boldsymbol{\cal D}_0({{\bf M}},{\cal E})\esc e^{-[{\boldsymbol{\Lambda}}]_\times}
    \esc{\bf B}^T({\boldsymbol{\Lambda}}) 
\end{align}
The interest in expression  (\ref{eq:130}) lies with the  fact that
all  the dependence  on the  orientation is  now explicit  because, as
explained   in   Sec.  \ref{App:RotInv}  the   rest   conditional
expectation of a tensor in the  principal frame does not depend on the
orientation.  However, (\ref{eq:130}) is  still formal  because the  actual
functional form of $\boldsymbol{\cal D}_0({{\bf  M}}, {\cal E})$ is  not known.
In general the  conditional correlation  in (\ref{W:eq:356})  cannot be
  computed analytically nor through simulations. For this reason, we need
  to approximate or model this quantity. 
  One simple possibility is to approximate the conditional expectations with
    ordinary equilibrium expectations
    \begin{align}
  \label{}
    \boldsymbol{\cal D}_0({{\bf M}},{\cal E})
  \simeq          \boldsymbol{\cal D}_0^{\rm eq}({\cal E})
\end{align}
where
  \begin{align}
  \label{}
    \boldsymbol{\cal D}_0^{\rm eq}({\cal E})
  &\equiv \frac{1}{k_BT}\int_0^\infty dt
    \llangle\hat{\boldsymbol{\omega}}_{0}(0)\hat{\boldsymbol{\omega}}_{0}^{T}(t)
    \rrangle^{{\cal E}}_{\rm rest}
\end{align}

\subsection{The element $\boldsymbol{\Gamma}_{_\Pi}$ of the dissipative matrix}
Let    us   consider    now   the    calculation   of    the   element
$\boldsymbol{\Gamma}_{_\Pi}$ of  the dissipative matrix, as  defined by
the Green-Kubo  expression
\begin{align}
      {\boldsymbol{\Gamma}}_{_\Pi}(a)
  &=  \frac{1}{k_BT}  \int_0^\infty dt
\llangle \left({\cal Q}i{\cal L}\hat{\boldsymbol{\Pi}}\right)
\left(\exp\{i{\cal Q}{\cal L}t\}{\cal Q}i{\cal L}\hat{\boldsymbol{\Pi}}\right)^T\rrangle^{a}
\end{align}
 According to  the definition
of the dilation force we have
\begin{align}
  \label{eq:106}
    \llangle {\cal Q}i{\cal L}{\hat{\boldsymbol{\Pi}}}(0){\cal Q}i{\cal L}
  {\hat{\boldsymbol{\Pi}}}^T(t) \rrangle^a
  &=    \llangle \left(\hat{\boldsymbol{\cal K}}(0)-\llangle \hat{\boldsymbol{\cal K}}\rrangle^a\right)
 \left(\hat{\boldsymbol{\cal K}}^T(t)-\llangle \hat{\boldsymbol{\cal K}}^T\rrangle^a\right)\rrangle^a
\end{align}
which when transformed into the rest conditional
expectations using (\ref{eq:43}) gives
\begin{align}
  \label{eq:332}
    \llangle \left(\hat{\boldsymbol{\cal K}}(0)-\llangle \hat{\boldsymbol{\cal K}}\rrangle^a\right)
  \left(\hat{\boldsymbol{\cal K}}^T(t)-\llangle \hat{\boldsymbol{\cal K}}^T\rrangle^a\right)\rrangle^a
  &=  \llangle \left({\cal G}\hat{\boldsymbol{\cal K}}(0)-\llangle {\cal G}\hat{\boldsymbol{\cal K}}\rrangle^{\boldsymbol{\Lambda}{\bf M}{\cal E}}\right)
 \left( {\cal G}\hat{\boldsymbol{\cal K}}^T(t)-\llangle  {\cal G}\hat{\boldsymbol{\cal K}}^T\rrangle^{\boldsymbol{\Lambda}{\bf M}{\cal E}}\right)\rrangle^{\boldsymbol{\Lambda}{\bf M}{\cal E}}_{\rm rest}
\end{align}
Using (\ref{eq:194}) and (\ref{eq:107}) shows    that      the      conditional      expectations
$\llangle\cdots\rrangle^a$  can be  substituted  with the  conditional
expectations
$\llangle\cdots\rrangle^{\boldsymbol{\Lambda}{{\bf  M}}{\cal  E}}_{\rm
  rest}$, so that analogously to (\ref{eq:7-S}),
\begin{align}
  \label{eq:182}
  \llangle {\cal Q}i{\cal L}{\hat{\boldsymbol{\Pi}}}(0){\cal Q}i{\cal L}
  {\hat{\boldsymbol{\Pi}}}^T(t) \rrangle^a
  &=  \llangle
    \left(\hat{\boldsymbol{\cal K}}(0)-\llangle \hat{\boldsymbol{\cal K}}\rrangle^{\boldsymbol{\Lambda}{{\bf M}}{\cal E}}_{\rm rest}\right)
    \left(\hat{\boldsymbol{\cal K}}(t)-\llangle \hat{\boldsymbol{\cal K}}\rrangle^{\boldsymbol{\Lambda}{{\bf M}}{\cal E}}_{\rm rest}\right)^T\rrangle^{\boldsymbol{\Lambda}{{\bf M}}{\cal E}}_{\rm rest}    
\end{align}
The corresponding element in the dissipative matrix is
\begin{align}
  \label{eq:72}
  \boldsymbol{\Gamma}_{_\Pi}(\boldsymbol{\Lambda},{{\bf M}},{\cal E})=\frac{1}{k_BT}\int_0^\infty dt  \llangle
  \left(\hat{\boldsymbol{\cal K}}(0)-\llangle \hat{\boldsymbol{\cal K}}\rrangle^{\boldsymbol{\Lambda}{{\bf M}}{\cal E}}_{\rm rest}\right)
  \left(\hat{\boldsymbol{\cal K}}(t)-\llangle \hat{\boldsymbol{\cal K}}\rrangle^{\boldsymbol{\Lambda}{{\bf M}}{\cal E}}_{\rm rest}\right)^T\rrangle^{\boldsymbol{\Lambda}{{\bf M}}{\cal E}}_{\rm rest}    
\end{align}
and   is   a    function   of   the   CG    variables   only   through
$\boldsymbol{\Lambda},{\bf M}$ and the  thermal energy ${\cal E}$. The
element (\ref{eq:72}) cannot be directly computed using equilibrium MD
simulations as it involves a conditional expectation.  We will
approximate the conditional expectation with an ordinary equilibrium expectation
\begin{align}
  \label{eq:429}
\boldsymbol{\Gamma}_{_\Pi}(\boldsymbol{\Lambda},{{\bf M}},{\cal E})
  \simeq        \boldsymbol{\Gamma}_{_\Pi}^{\rm eq}({\cal E})
\end{align}
where
  \begin{align}
  \label{eq:428}
    \boldsymbol{\Gamma}_{_\Pi}^{\rm eq}&\equiv\frac{1}{k_BT}\int_0^\infty dt  \llangle
  \left(\hat{\boldsymbol{\cal K}}(0)-\llangle \hat{\boldsymbol{\cal K}}\rrangle^{{\cal E}}\right)
  \left(\hat{\boldsymbol{\cal K}}(t)-\llangle \hat{\boldsymbol{\cal K}}\rrangle^{{\cal E}}\right)^T\rrangle^{{\cal E}}
\end{align}

\section{The stochastic drift}
\setcounter{equation}{\value{my_equation_counter}}
  \label{App:StochDrift}
For the  present level  of
  description,  the  stochastic  drift term $    V_\mu^{\rm sto}(a)=k_B\frac{\partial M_{\mu\nu}}{\partial  a_\nu}$
reduces to two       components:
  $\boldsymbol{V}^{\rm sto}_{_\Lambda}$,  affecting  the orientation
  dynamics,  and $\boldsymbol{V}^{\rm  sto}_\Pi$, affecting the
  dilational  momentum dynamics
\begin{align}
  \label{eq:269-2}
\boldsymbol{V}^{{\rm sto }}_{_\Lambda}
  &=k_B\frac{\partial}{\partial\boldsymbol{\Lambda}}\esc
    T\boldsymbol{\Gamma}_{_\Lambda}
    \nonumber\\
\boldsymbol{V}^{\rm sto}_{_\Pi}
  &=k_B\frac{\partial}{\partial\boldsymbol{\Pi}}\esc T\boldsymbol{\Gamma}_{_\Pi}
\end{align}
The second component $  \boldsymbol{V}^{\rm sto}_{_\Pi}$  is easy to compute
under the equilibrium approximation (\ref{eq:429}) giving
\begin{align}
  \label{eq:253}
  \boldsymbol{V}^{\rm sto}_{_\Pi}
    \overset{(\ref{eq:429})}{=}
    k_B\frac{\partial}{\partial\boldsymbol{\Pi}}\esc T \boldsymbol{\Gamma}_{_\Pi}^{\rm eq}
    =
    \frac{k_B}{C^{\rm MT}}\left(\frac{\partial}{\partial T}T \boldsymbol{\Gamma}_{_\Pi}^{\rm eq}\right)
    \esc \frac{\partial {\cal E}}{\partial \boldsymbol{\Pi}}
    \overset{(\ref{Thermal-App})}{=}
    -\frac{k_B}{C^{\rm MT}}\frac{\partial T \boldsymbol{\Gamma}_{_\Pi}^{\rm eq}}{\partial T}\esc
    \boldsymbol{\nu}
\end{align}
where $C^{\rm MT}$ is the heat capacity of the body.
We now compute  $\boldsymbol{V}^{{\rm sto }}_{_\Lambda}$ using (\ref{eq:130})
\begin{align}
        \boldsymbol{\Gamma}_{_\Lambda}(a)
  &=
    {\bf B}\esc
e^{[{{\boldsymbol{\Lambda}}}]_\times}\esc
    \boldsymbol{\cal D}_0\esc e^{-[{\boldsymbol{\Lambda}}]_\times}
    \esc{\bf B}^T
\overset{(\ref{BeL})}{=}
    {\bf B}^T\esc
   \boldsymbol{\cal D}_0\esc{\bf B}
    \label{eq:103b}
\end{align}
and observing the angular diffusion tensor
$  \boldsymbol{\cal  D}_0=\boldsymbol{\cal D}_0({{\bf  M}},{\cal  E})$
depends upon the  orientation  through the  thermal energy.  Therefore,
$ \boldsymbol{\Gamma}_{_\Lambda}(a)$ has an explicit dependence on the
orientation through ${\bf  B}({\boldsymbol{\Lambda}})$, and an implicit
dependence through the thermal energy in the angular diffusion tensor.
We will separate these
dependencies by writing the derivative in (\ref{eq:269-2}) as two terms with
the first carrying the implicit dependence (for which the chain rule will be
used) and the second carrying the explicit dependence, that is
\begin{align}
  \label{eq:279}
  \boldsymbol{V}^{\rm sto}_{_\Lambda}
=
    \frac{k_B}{C^{\rm MT}}\frac{\partial T\boldsymbol{\Gamma}_{_\Lambda}}{\partial T}\esc 
    \frac{\partial {\cal E}}{\partial\boldsymbol{\Lambda}}
        +    k_BT\frac{\partial\esc \boldsymbol{\Gamma}_{_\Lambda}} {\partial\boldsymbol{\Lambda}}
\overset{(\ref{Thermal-App})}{=}
  = -\frac{k_B}{C^{\rm MT}}\frac{\partial T\boldsymbol{\Gamma}_{_\Lambda}}{\partial T}\esc
    {\bf B}^{-T}\esc(\boldsymbol{\Omega}\times{\bf S})+ k_BT\frac{\partial\esc \boldsymbol{\Gamma}_{_\Lambda}} {\partial\boldsymbol{\Lambda}}
\end{align}
where the first and second terms carry the implicit and explicit
dependencies, respectively. Evaluating the explicit derivative dependence in (\ref{eq:279}) gives
\begin{align}
  \label{eq:48}
\frac{\partial\boldsymbol{\Gamma}^{\alpha\beta}_{_\Lambda}}{\partial{\Lambda}_\beta}
  &\overset{(\ref{eq:103b})}{=}
\frac{\partial}{\partial{\Lambda}_\beta}
\left(    {B}_{\alpha'\alpha}
   \boldsymbol{\cal D}^{\alpha'\beta'}_0{B}_{\beta'\beta}\right)
\overset{\mbox{\small chain rule}}{=}
  \left(\frac{\partial{B}_{\alpha'\alpha}}{\partial{\Lambda}_\beta}\right)
    \boldsymbol{\cal D}^{\alpha'\beta'}_0{B}_{\beta'\beta}
    +   {B}_{\alpha'\alpha}
    \boldsymbol{\cal D}^{\alpha'\beta'}_0
    \left(  \frac{\partial{B}_{\beta'\beta}}{\partial{\Lambda}_\beta}\right)
\end{align}
The second term can be computed as follows
\begin{align}
  \label{eq:363}
   {B}_{\alpha'\alpha}
    \boldsymbol{\cal D}^{\alpha'\beta'}_0
  \left(  \frac{\partial{B}_{\beta'\beta}}{\partial{\Lambda}_\beta}\right)
  &    \overset{(\ref{eq:606})}{=} {B}_{\alpha'\alpha}
     \boldsymbol{\cal D}^{\alpha'\beta'}_02f(\Lambda) {\Lambda}_{\beta'}
     \overset{(\ref{eq:224})}{=}
      {B}_{\alpha'\alpha}
     \boldsymbol{\cal D}^{\alpha'\beta'}_0{B}_{\beta'\beta}2f(\Lambda) {\Lambda}_{\beta}
     \overset{(\ref{eq:103b})}{=}
      \boldsymbol{\Gamma}_{_\Lambda}^{\alpha\beta}2f(\Lambda) {\Lambda}_{\beta}
                                                                                             \nonumber\\
&  \overset{(\ref{eq:365})}{=}
  -   \boldsymbol{\Gamma}_{_\Lambda}^{\alpha\beta}\frac{\partial}{\partial{{\Lambda}_\beta}}\ln P^{\rm Haar}(\boldsymbol{\Lambda})
\end{align}
The first term on the right hand side of (\ref{eq:48}) is computed in Sec.~\ref{App:ThermDrift} giving (\ref{Fin-Fth}).

\newpage
\setcounter{my_equation_counter}{\value{equation}}
\section{The thermal drift}
\setcounter{equation}{\value{my_equation_counter}}
\label{App:ThermDrift}
In this section, we compute the thermal drift
\begin{align}
  \label{Tdrift}
     {\bf F}^{\rm th}
    &\equiv  \frac{\partial\esc   \boldsymbol{\Gamma}_{_\Lambda}}{\partial\boldsymbol{\Lambda}}
+  \boldsymbol{\Gamma}_{_\Lambda}\esc   \frac{\partial}{\partial \boldsymbol{\Lambda}}\ln P^{\rm Haar}(\boldsymbol{\Lambda})
\end{align}
using (\ref{eq:363}) in (\ref{eq:48}) to give
\begin{align}
  \label{eq:366}
  {\bf F}^{\rm th}_\alpha
  & =\frac{\partial\boldsymbol{\Gamma}^{\alpha\beta}_{_\Lambda}}{\partial{\Lambda}_\beta}
                           +  \boldsymbol{\Gamma}_{_\Lambda}^{\alpha\beta}\frac{\partial}{\partial{{\Lambda}_\beta}}
                           \ln P^{\rm Haar}(\boldsymbol{\Lambda})
                           =    \left(\frac{\partial{B}_{\alpha'\alpha}}{\partial{\Lambda}_\beta}\right)
                           \boldsymbol{\cal D}^{\alpha'\beta'}_0{B}_{\beta'\beta}
\equiv\boldsymbol{\cal D}^{\alpha'\beta'}_0{\Xi}_{\beta'\alpha'\alpha}
\end{align}
which coincides with   the  first  term   of
(\ref{eq:48}), and where we have introduced the third order tensor
\begin{align}
  \label{eq:3}
  {\Xi}_{\beta'\alpha'\alpha}&\equiv
                                          {B}_{\beta'\beta}\frac{\partial{B}_{\alpha'\alpha}}{\partial{\Lambda}_\beta}
\end{align}
To compute ${\Xi}_{\beta'\alpha'\alpha}$, we insert (\ref{eq:155}) and (\ref{eq:156}) in (\ref{eq:3}) giving
\begin{align}
  \label{eq:26}
  {\Xi}_{\beta'\alpha'\alpha}
  &=\left[
    \delta_{\beta'\beta}-\frac{\Lambda}{2}[{\bf n}]_\times^{\beta'\beta}
    +f(\Lambda)\Lambda^2\left[[{\bf n}]_\times\esc[{\bf n}]_\times\right]_{\beta'\beta}
    \right]\nonumber\\
    &\quad\times\left[
    -\frac{1}{2}\epsilon_{\alpha'\beta\alpha} +
    f'(\Lambda)\Lambda^2
    \left [[{{\bf n}}]_\times\esc [{{\bf n}}]_\times
    \right ]^{\alpha'\alpha}{\bf n}_\beta
    + f(\Lambda) \Lambda \left(\delta_{\alpha'\beta}{\bf n}_{\alpha}
    +    \delta_{\alpha\beta}{\bf n}_{\alpha'} -2\delta_{\alpha'\alpha}{\bf n}_\beta \right)
    \right]
    \nonumber\\
  &= -\frac{1}{2}\epsilon_{\alpha'\beta'\alpha} +
    f'(\Lambda)\Lambda^2
    \left [[{{\bf n}}]_\times\esc [{{\bf n}}]_\times
    \right ]^{\alpha'\alpha}{\bf n}_{\beta'}
    + f(\Lambda)\Lambda \left(\delta_{\alpha'\beta'}{\bf n}_{\alpha}
    +    \delta_{\alpha\beta'}{\bf n}_{\alpha'} -2\delta_{\alpha'\alpha}{\bf n}_{\beta'} \right)
    \nonumber\\
  &\quad +\frac{\Lambda}{4}[{\bf n}]_\times^{\beta'\beta}\epsilon_{\alpha'\beta\alpha} 
    -\frac{f(\Lambda)\Lambda^2}{2} \left([{\bf n}]_\times^{\beta'\alpha'}{\bf n}_{\alpha}
    +  [{\bf n}]_\times^{\beta'\alpha}{\bf n}_{\alpha'} \right)
    \nonumber\\
  &\quad-\frac{f(\Lambda)\Lambda^2}{2}\left[[{\bf n}]_\times\esc[{\bf n}]_\times\right]_{\beta'\beta}
    \epsilon_{\alpha'\beta\alpha} 
    + f^2(\Lambda)\Lambda^3 \left(\left[[{\bf n}]_\times\esc[{\bf n}]_\times\right]_{\beta'\alpha'}
    {\bf n}_{\alpha}
    +\left[[{\bf n}]_\times\esc[{\bf n}]_\times\right]_{\beta'\alpha}
    {\bf n}_{\alpha'} \right)
\end{align}
Use the properties of the Levi-Civita symbol to get
\begin{align}
  \label{eq:56}
  [{\bf n}]_\times^{\beta'\beta}\epsilon_{\alpha'\beta\alpha}
  &  =  {\bf n}_\gamma\epsilon_{\beta'\gamma \beta}\epsilon_{\alpha'\beta\alpha}
    = - {\bf n}_\gamma\left(\delta_{\beta'\alpha'}\delta_{\gamma\alpha}
      -\delta_{\gamma\alpha'}\delta_{\beta'\alpha}\right)
    =\delta_{\beta'\alpha}{\bf n}_{\alpha'}
    - \delta_{\beta'\alpha'} {\bf n}_\alpha
    \nonumber\\
  \left[[{\bf n}]_\times\esc[{\bf n}]_\times\right]_{\beta'\beta}\epsilon_{\alpha'\beta\alpha}
  &=  [{\bf n}]_\times^{\beta'\sigma}[{\bf n}]_\times^{\sigma \beta}\epsilon_{\alpha'\beta\alpha}
    = [{\bf n}]_\times^{\beta'\sigma}\left(
    \delta_{\sigma\alpha}{\bf n}_{\alpha'}
    -\delta_{\sigma\alpha'}{\bf n}_\alpha\right)
    = 
    [{\bf n}]_\times^{\beta'\alpha}{\bf n}_{\alpha'}
    - [{\bf n}]_\times^{\beta'\alpha'}{\bf n}_\alpha
\end{align}
and insert in (\ref{eq:26}) to give
\begin{align}
  \label{eq:77}
  {\Xi}_{\beta'\alpha'\alpha}
  &= -\frac{1}{2}\epsilon_{\alpha'\beta'\alpha} +
    f'(\Lambda)\Lambda^2
    \left [[{{\bf n}}]_\times\esc [{{\bf n}}]_\times
    \right ]^{\alpha'\alpha}{\bf n}_{\beta'}
    + f(\Lambda) \Lambda\left(\delta_{\alpha'\beta'}{\bf n}_{\alpha}
    +    \delta_{\alpha\beta'}{\bf n}_{\alpha'} -2\delta_{\alpha'\alpha}{\bf n}_{\beta'} \right)
    \nonumber\\
  &\quad +\frac{\Lambda}{4}\left(\delta_{\beta'\alpha}{\bf n}_{\alpha'}
    - \delta_{\beta'\alpha'}{\bf n}_\alpha\right)
    -f(\Lambda)\Lambda^2 [{\bf n}]_\times^{\beta'\alpha}{\bf n}_{\alpha'}
    + f^2(\Lambda) \Lambda^3\left(\left[[{\bf n}]_\times\esc[{\bf n}]_\times\right]_{\beta'\alpha'}
    {\bf n}_{\alpha}
    +\left[[{\bf n}]_\times\esc[{\bf n}]_\times\right]_{\beta'\alpha}
    {\bf n}_{\alpha'} \right)
\end{align}


Observe that  there are bits  in this tensor that  are antisymmetric
under the interchange of indices $\alpha', \beta'$.  The contraction
of      these      bits      with     the      symmetric      tensor
$\boldsymbol{\cal D}_0^{\alpha'\beta'}$ will vanish.  Therefore, the
thermal drift (\ref{eq:366}) becomes
\begin{align}
  \label{eq:93}
  {\bf F}^{\rm th}_\alpha
  &=f'(\Lambda)\Lambda^2\boldsymbol{\cal D}^{\alpha'\beta'}_0    
    \left [[{{\bf n}}]_\times\esc [{{\bf n}}]_\times \right]_{\alpha'\alpha}{\bf n}_{\beta'}
    + f(\Lambda)\Lambda \left(
    \boldsymbol{\cal D}^{\beta'\beta'}_0 {\bf n}_{\alpha} +
    \boldsymbol{\cal D}^{\alpha'\alpha}_0 {\bf n}_{\alpha'} -
   2\boldsymbol{\cal D}^{\alpha\beta'}_0 {\bf n}_{\beta'} \right)
    \nonumber\\
  &\quad
    +\frac{\Lambda}{4}\left(
      \boldsymbol{\cal D}^{\alpha'\alpha}_0 {\bf n}_{\alpha'} -
      \boldsymbol{\cal D}^{\beta'\beta'}_0  {\bf n}_\alpha \right)
    -    f(\Lambda)\Lambda^2 [{\bf n}]_\times^{\beta'\alpha}\boldsymbol{\cal D}^{\alpha'\beta'}_0{\bf n}_{\alpha'}
    \nonumber\\
  &\quad +f^2(\Lambda)\Lambda^3 \left (
    {\rm Tr}\left[ \boldsymbol{\cal D}_0\esc[{\bf n}]_\times\esc[{\bf n}]_\times\right]
    {\bf n}_{\alpha}
    +\boldsymbol{\cal D}^{\alpha'\beta'}_0 \left[[{\bf n}]_\times\esc[{\bf n}]_\times\right]_{\beta'\alpha}
    {\bf n}_{\alpha'} \right )
\end{align}
Collecting and simplifying terms, using (\ref{eq:603}) and writing in vector
notation then gives
\begin{align}
  \label{eq:241}
  {\bf F}^{\rm th}
  &=\left(\frac{\Lambda}{4}-f(\Lambda)\Lambda\right)
\left(\boldsymbol{\cal D}_0\esc {\bf n} -{\rm Tr}[\boldsymbol{\cal D}_0]{\bf n}\right)
    + f(\Lambda)\Lambda^2[{{\bf n}}]_\times\esc    \boldsymbol{\cal D}_0 \esc{\bf n}
    +\left(f'(\Lambda)\Lambda^2+f^2(\Lambda)\Lambda^3\right)
    [{{\bf n}}]_\times\esc [{{\bf n}}]_\times\esc
    \boldsymbol{\cal D}_0 \esc{\bf n}
\nonumber\\
&\quad +f^2(\Lambda)\Lambda^3 {\rm Tr}\left[  \boldsymbol{\cal D}_0\esc[{\bf n}]_\times\esc[{\bf n}]_\times\right]
    {{\bf n}}
\end{align}
We seek a more compact form for the thermal drift.  We use (\ref{eq:600}) in the form
\begin{align}
  \label{eq:235}
  [{\bf n}]_\times\esc[{\bf n}]_\times={\bf n}{\bf n}^T-\mathbb{1}
\end{align}
to write the last two terms of (\ref{eq:241}) as
\begin{align}
\left(f'(\Lambda)\Lambda^2+f^2(\Lambda)\Lambda^3\right)
({\bf n}{\bf n}^T-\mathbb{1})\esc\boldsymbol{\cal D}_0\esc{{\bf n}}& +
f^2(\Lambda) \Lambda^3{\rm Tr}\left[
    ({\bf n}{\bf n}^T-\mathbb{1})\esc\boldsymbol{\cal D}_0\right] {{\bf n}}  
\nonumber\\
&= \left(f'(\Lambda)\Lambda^2+f^2(\Lambda)\Lambda^3\right)
    \left(({\bf n}^T\esc\boldsymbol{\cal D}_0\esc{\bf n}){{\bf n}}
    -\boldsymbol{\cal D}_0    \esc{{\bf n}}\right)
    +f^2(\Lambda)\Lambda^3 \left(({\bf n}^T\esc\boldsymbol{\cal D}_0\esc{\bf n})-{\rm Tr}[\boldsymbol{\cal D}_0]
    \right){{\bf n}}  
\end{align}
which when inserted in (\ref{eq:241}) gives
\begin{align}
  \label{eq:209}
  {\bf F}^{\rm th}
  &=
    F_1{\rm Tr}[\boldsymbol{\cal D}_0]{{\bf n}}
    +    F_2 \boldsymbol{\cal D}_0\esc{{\bf n}}
    +F_3({\bf n}^T\esc\boldsymbol{\cal D}_0\esc{\bf n}){{\bf n}}
    + f(\Lambda)\Lambda^2[{{\bf n}}]_\times\esc\boldsymbol{\cal D}_0\esc{\bf n}
\end{align}
where
  \begin{align}
    \label{eq:247}
F_1&\equiv        - \frac{\Lambda}{4}+f(\Lambda)\Lambda    -f^2(\Lambda)\Lambda^3
     \nonumber\\
F_2&\equiv    \frac{\Lambda}{4}-f(\Lambda)\Lambda-f'(\Lambda)\Lambda^2-f^2(\Lambda)\Lambda^3 
     \nonumber\\
    F_3&\equiv     f'(\Lambda)\Lambda^2+2f^2(\Lambda)\Lambda^3=-F_1-F_2
\end{align}
Using (\ref{eq:262}) for $f(\Lambda)$, and with the help of Mathematica, we find
\begin{align}
  \label{eq:258}
  F_1&=\frac{1}{2}\frac{\sin\Lambda -\Lambda}{1-\cos\Lambda}
              \nonumber\\
F_2&=\cot\left(\frac{\Lambda}{2}\right)\left(1-\frac{\Lambda}{2}\cot\left(\frac{\Lambda}{2}\right)\right)
\end{align}
Using the last identity (\ref{eq:247}) we may write (\ref{eq:209}) as
\begin{align}
    \label{Fin-Fth}
   {\bf F}^{\rm th}
    &=
      F_1\left[{\rm Tr}[\boldsymbol{\cal D}_0]{{\bf n}}-({\bf n}^T\esc\boldsymbol{\cal D}_0\esc{\bf n}){{\bf n}}\right]
            +F_2 \left[\boldsymbol{\cal D}_0\esc{{\bf n}}-({\bf n}^T\esc\boldsymbol{\cal D}_0\esc{\bf n}){{\bf n}}\right]
      +      f(\Lambda)\Lambda^2[{{\bf n}}]_\times\esc    \boldsymbol{\cal D}_0\esc{{\bf n}}
      \nonumber\\
    &=F_1{\rm Tr}\left[\left[\mathbb{1}-{\bf n}{\bf n}^T\right]\esc\boldsymbol{\cal D}_0\right]{\bf n}
+      F_2 \left[\mathbb{1}-{\bf n}{\bf n}^T\right]\esc\boldsymbol{\cal D}_0\esc{{\bf n}}
      +      f(\Lambda)\Lambda^2[{{\bf n}}]_\times\esc    \boldsymbol{\cal D}_0\esc{{\bf n}}
\end{align}


\setcounter{my_equation_counter}{\value{equation}}
  \section{The cross  product}
\setcounter{equation}{\value{my_equation_counter}}
    \label{App:cross}

    We summarize  a
  number of properties  of the cross product matrix  and cross product
  of two vectors, as they are used in many occasions in the paper. For
  arbitrary vectors  $\bf a$, $\bf  b$, and $\bf c$,  scalar $\kappa$,
  and rotation matrix $\boldsymbol{\cal Q}$ one has
\begin{align}
\label{vecprod}
  [{\bf a}]_\times\esc{\bf b}
  &={\bf a}\times{\bf b}\\
  {\bf a}^T\esc [{\bf b}]_\times
  &=({\bf a}\times{\bf b})^T\\
\label{eq:603}
  [{\bf a}]_\times^T
  &=-[{\bf a}]_\times= [-{\bf a}]_\times\\
  [{\bf a}+{\bf b}]_\times
  &=[{\bf a}]_\times+[{\bf b}]_\times\\
  [\kappa{\bf a}]_\times
  &=\kappa[{\bf a}]_\times\\
\label{eq:600}
  [{\bf a}]_\times\esc[{\bf b}]_\times
  &={\bf b} {\bf a}^T - ( {\bf a}\esc {\bf b})\mathbb{1}\\
  [{\bf a}]_\times\esc[{\bf b}]_\times\esc[{\bf c}]_\times
  &={\bf b} {\bf a}^T\esc [{\bf c}]_\times -( {\bf a}\esc {\bf b})[{\bf c}]_\times\\
  &={\bf b} ({\bf a}\times{\bf c})^T-( {\bf a}\esc {\bf b})[{\bf c}]_\times\\
  \label{eq:319}
  [{\bf a}]_\times\esc[{\bf a}]_\times\esc[{\bf a}]_\times
  &=-a^2[{\bf a}]_\times\\
  \label{eq:601}
[\boldsymbol{\cal Q}\esc{\bf a}]_\times
  &=\boldsymbol{\cal Q}\esc[{\bf a}]_\times\esc \boldsymbol{\cal Q}^T\\
  \label{Qaxb}
  \boldsymbol{\cal Q}\esc({\bf a}\times{\bf b})
  &=
      (\boldsymbol{\cal Q}\esc{\bf a}\times\boldsymbol{\cal Q}\esc{\bf b})
\\\label{eq:602}
  \frac{\partial}{\partial {\bf a}^\gamma}[{\bf a}]_\times^{\alpha\beta}
  &=\epsilon_{\alpha\gamma\beta}
\end{align}




\color{black}



\setcounter{my_equation_counter}{\value{equation}}
\section{Results concerning angular velocity, orientation, and gyration tensor}
\setcounter{equation}{\value{my_equation_counter}}
\label{App:DerRot}
In this section, we  derive a  number of  results involving the
derivative   of   the   rotation   matrix  with   respect   to   the
orientation. This allows us to  relate the angular velocity with the
time  derivative  of the  orientation,  and  the derivative  of  the
inertia tensor  with  respect to  the  orientation.  These
results are used in the main text.

\subsection{The derivative of the rotation matrix with respect to the orientation}

We  need an  explicit expression  for the  derivative of  the rotation
matrix  with respect  to  the  orientation.  While  this  can be  done
directly from Rodrigues  formula, we take here  the following somewhat
shorter and  more elegant route.  Using  a formula given by  Wilcox in
\cite{Wilcox1967}, namely
\begin{align}
  \frac{\partial e^{{\bf A}(\lambda)}}{\partial \lambda} &=
\int_0^1 dx e^{(1-x){\bf A}(\lambda)}\esc
  \frac{\partial{\bf A}(\lambda)}{\partial \lambda} \esc
e^{x{\bf A}(\lambda)}
\label{WilcoxLambda}
\end{align}
the derivative of the rotation matrix with respect to the orientation is
\begin{align}
  \label{eq:60}
  \frac{\partial   \left[e^{[\boldsymbol{\Lambda}]_\times}\right]_{\alpha\beta}}{\partial{\Lambda}_\gamma}
  &=\int_0^1 dx \left[e^{(1-x)[\boldsymbol{\Lambda}]_\times}\right]_{\alpha\alpha'}
  \frac{\partial[\boldsymbol{\Lambda}]^{\alpha'\beta'}_\times}{\partial{\Lambda}_\gamma} 
\left[e^{x[\boldsymbol{\Lambda}]_\times}\right]_{\beta'\beta}
    \nonumber\\
  &=\int_0^1 dx \left[e^{(1-x)[\boldsymbol{\Lambda}]_\times}\right]_{\alpha\alpha'}
\epsilon_{\alpha'\gamma\beta'}
    \left[e^{x[\boldsymbol{\Lambda}]_\times}\right]_{\beta'\beta}
    \nonumber\\
  &=\int_0^1 dx \left[e^{(1-x)[\boldsymbol{\Lambda}]_\times}\right]_{\alpha\alpha'}
\epsilon_{\alpha'\gamma\beta'}
    \left[e^{(x-1)[\boldsymbol{\Lambda}]_\times}\right]_{\beta'\delta}
    \left[e^{[\boldsymbol{\Lambda}]_\times}\right]_{\delta\beta}
    \nonumber\\
  &=\int_0^1 dx \left[e^{x[\boldsymbol{\Lambda}]_\times}\right]_{\alpha\alpha'}
\epsilon_{\alpha'\gamma\beta'}
    \left[e^{-x[\boldsymbol{\Lambda}]_\times}\right]_{\beta'\delta}
    \left[e^{[\boldsymbol{\Lambda}]_\times}\right]_{\delta\beta}
\end{align}
where a simple change of variables $1-x\to x$ has been performed in the last equation. Because
the Levi-Civita symbol is rotationally invariant, that is
\begin{align}
  \label{eq:96}
  \left[e^{-x[{\boldsymbol{\Lambda}}]_\times}\right]_{\alpha'\alpha}
  \left[e^{-x[{\boldsymbol{\Lambda}}]_\times}\right]_{\beta'\delta}
  \left[e^{-x[{\boldsymbol{\Lambda}}]_\times}\right]_{\gamma\gamma'}
  \epsilon_{\alpha'\gamma\beta'} &=      \epsilon_{\alpha\gamma'\delta}
\end{align}
we have after some rearrangement
\begin{align}
  \label{eq:98}
      \left[e^{x[{\boldsymbol{\Lambda}}]_\times}\right]_{\alpha\alpha'}
    \left[e^{-x[{\boldsymbol{\Lambda}}]_\times}\right]_{\beta'\delta}
                \epsilon_{\alpha'\gamma\beta'} &=      \epsilon_{\alpha\gamma'\delta} \left[e^{-x[{\boldsymbol{\Lambda}}]_\times}\right]_{\gamma'\gamma}
\end{align}
Substituting (\ref{eq:98}) in (\ref{eq:60}) we obtain the final expression of the
derivative of the rotation matrix with respect to the orientation
\begin{align}
  \label{eq:99}
  \frac{\partial}{\partial{\Lambda}_{\gamma}}
\left[e^{[{\boldsymbol{\Lambda}}]_\times}\right]_{\alpha\beta}
  &=\epsilon_{\alpha\gamma'\beta'}
    {B}_{\gamma'\gamma}^{-1}({\boldsymbol{\Lambda}}) 
    \left[e^{[{\boldsymbol{\Lambda}}]_\times}\right]_{\beta'\beta}
\end{align}
where we have introduced the matrix
\begin{align}
  \label{eq:281}
  {\bf B}^{-1}({\boldsymbol{\Lambda}})  &\equiv\int_0^1dx   e^{x[{\boldsymbol{\Lambda}}]_\times}
\end{align}
This matrix can be explicitly computed by using Rodrigues' formula as
\begin{align}
  \label{eq:138}
  {\bf B}^{-1}({\boldsymbol{\Lambda}})=
  &\int_0^1dx  \left(\mathbb{1}
    +\sin(x{\Lambda}) [{\bf n}]_\times
    +\left(1-\cos(x{\Lambda})\right) [{\bf n}]_\times\esc [{\bf n}]_\times\right)
=\mathbb{1}+a({\Lambda}) [{\bf n}]_\times+b({\Lambda})[{\bf n}]_\times\esc [{\bf n}]_\times 
\end{align}
where 
\begin{align}
  \label{eq:30}
      \nonumber\\
  a({\Lambda})&=\frac{1-\cos{\Lambda}}{{\Lambda}}
     \nonumber\\
  b({\Lambda})&=1-\frac{\sin{\Lambda}}{{\Lambda}}
\end{align}
Using the property (\ref{eq:319}) it easy to check that the inverse
matrix has the same structure as the original matrix
\begin{align}
\label{eq:139}
{\bf B}({\boldsymbol{\Lambda}}) &= \mathbb{1}+p [{\bf n}]_\times+
q[{\bf n}]_\times\esc[{\bf n}]_\times
\end{align}
where
\begin{align}
  \label{eq:140}
p&=\frac{-a}{(1-b)^2+a^2}=-\frac{\Lambda}{2}
     \nonumber\\
  q&=\frac{a^2+b^2-b}{(1-b)^2+a^2}
     =1-\frac{\Lambda}{2}\frac{\sin\Lambda}{(1-\cos\Lambda)}
     =1-\frac{\Lambda}{2}\cot\frac{\Lambda}{2}
\end{align}
{An immediate consequence of the form (\ref{eq:139}) is
  \begin{align}
    \label{eq:224}
    {\bf B}(\boldsymbol{\Lambda})\esc\boldsymbol{\Lambda}=\boldsymbol{\Lambda}
  \end{align}}
The following identities follow from (\ref{eq:99}) and
all express  how the rotation matrix changes with the orientation
\begin{align}
  \label{eq:183}
\frac{\partial}{\partial{\Lambda}_{\gamma}}
\left[e^{[{\boldsymbol{\Lambda}}]_\times}\right]_{\alpha\beta}
{B}_{\gamma\gamma'}({\boldsymbol{\Lambda}})
&=\epsilon_{\alpha\gamma'\beta'} \left[e^{[{\boldsymbol{\Lambda}}]_\times}\right]_{\beta'\beta}\nonumber\\
\frac{\partial}{\partial{\Lambda}_{\gamma}}
\left[e^{-[{\boldsymbol{\Lambda}}]_\times}\right]_{\alpha\beta}
{B}_{\gamma\gamma'}({\boldsymbol{\Lambda}})
&=\epsilon_{\beta\gamma'\beta'} \left[e^{-[{\boldsymbol{\Lambda}}]_\times}\right]_{\alpha\beta'}\nonumber\\
\left[e^{[{\boldsymbol{\Lambda}}]_\times}\right]_{{\beta'\alpha}}
\frac{\partial}{\partial{\Lambda}_{\gamma}}
\left[e^{-[{\boldsymbol{\Lambda}}]_\times}\right]_{{\alpha}\beta}
&=\epsilon_{\beta\gamma'{\beta'}}{B}^{-1}_{\gamma'\gamma}({\boldsymbol{\Lambda}})\nonumber\\
\left(\frac{\partial}{\partial{\Lambda}_{\gamma}}
\left[e^{[{\boldsymbol{\Lambda}}]_\times}\right]_{\alpha\beta}\right)
\left[e^{-[{\boldsymbol{\Lambda}}]_\times}\right]_{\beta\alpha'}
&=\epsilon_{\alpha\gamma'\alpha'} {B}^{-1}_{\gamma'\gamma}({\boldsymbol{\Lambda}})
\end{align}
From these expressions, evaluated at $\boldsymbol{\Lambda}=0$, we obtain the following useful identities.
\begin{align}
\label{eq:457}
  \left . \frac{\partial}{\partial{\Lambda}_{\gamma}}
\left[e^{[{\boldsymbol{\Lambda}}]_\times}\right]_{\alpha\beta}
\right |_{{\boldsymbol{\Lambda}}=0}
&= \epsilon_{\alpha\gamma\beta}\nonumber\\
\left . \frac{\partial}{\partial{\Lambda}_{\gamma}}
\left[e^{-[{\boldsymbol{\Lambda}}]_\times}\right]_{\alpha\beta}
\right |_{{\boldsymbol{\Lambda}}=0}
&= \epsilon_{\beta\gamma\alpha}
\end{align}


\subsection{Properties of ${\bf B}$}

We now  consider some properties  of
${\bf  B}(\boldsymbol{\Lambda})$.  From  (\ref{eq:138}) and
(\ref{eq:139}), we have the symmetry properties
\begin{align}
  \label{eq:216}
  {\bf B}(-{\boldsymbol{\Lambda}})
  &=  {\bf B}^T({\boldsymbol{\Lambda}})
    \nonumber\\
  {\bf B}^{-1}(-{\boldsymbol{\Lambda}})
  &=  {\bf B}^{-T}({\boldsymbol{\Lambda}})
\end{align}
and the interesting property
\begin{align}
  \label{eq:489}
  {\bf B}\esc e^{[{\boldsymbol{\Lambda}}]_\times}={\bf B}^T
\end{align}
whose  proof    is  as  follows.  First,  to  alleviate
notation, write  Rodrigues' formula in the form
\begin{align}
  \label{eq:163}
e^{[{\boldsymbol{\Lambda}}]_\times}&=\mathbb{1}+s[{\bf n}]_\times+v[{\bf n}]_\times\esc[{\bf n}]_\times
\end{align}
where
\begin{align}
  \label{eq:164}
  s&=\sin\Lambda
     \nonumber\\
  v&=1-\cos\Lambda
\end{align}
then starting with the left hand side of (\ref{eq:489}) and using
(\ref{eq:319}) gives
\begin{align}
  \label{BeL}
  {\bf B}\esc e^{[{\boldsymbol{\Lambda}}]_\times}&=
(\mathbb{1}+p[{\bf n}]_\times+q[{\bf n}]_\times\esc[{\bf n}]_\times)\esc
(\mathbb{1}+s[{\bf n}]_\times+v[{\bf n}]_\times\esc[{\bf n}]_\times)\cr
&=\mathbb{1}+(s+p-pv-qs)[{\bf n}]_\times+(v+ps+q-qv)[{\bf n}]_\times\esc[{\bf n}]_\times\cr
&=\mathbb{1}-p[{\bf n}]_\times+q[{\bf n}]_\times\esc[{\bf n}]_\times\cr
&={\bf B}^T
\end{align}
where using (\ref{eq:140}) and (\ref{eq:164}) gives
\begin{align}
  \label{eq:73}
  qv-ps
  &=v
    \nonumber\\
  qs+pv&=s+2p
\end{align}
We will need the derivative of ${\bf B}$ with respect to 
${\boldsymbol{\Lambda}}$, so begin by expressing (\ref{eq:139}) as
\begin{align}
\label{eq:155}
{\bf B}({\boldsymbol{\Lambda}})\overset{(\ref{eq:262})}{=}
\mathbb{1} - \frac{1}{2}[{\boldsymbol{\Lambda}}]_\times +
f(\Lambda) [{\boldsymbol{\Lambda}}]_\times\esc
[{\boldsymbol{\Lambda}}]_\times
\end{align}
Taking the derivative of (\ref{eq:155}) and using (\ref{eq:602}) along with
component expressions in Einstein summation convention where appropriate gives
\begin{align}
\label{eq:156}
\frac{\partial{B}_{\alpha\beta}({\boldsymbol{\Lambda}})}
{\partial{\Lambda}_\gamma}
&=-\frac{1}{2}\epsilon_{\alpha\gamma\beta} +
\frac{df }{d\Lambda}\frac{{\Lambda}_\gamma}{\Lambda}
[{\boldsymbol{\Lambda}}]^{\alpha\alpha'}_\times
[{\boldsymbol{\Lambda}}]^{\alpha'\beta}_\times + f(\Lambda) \left (
\epsilon_{\alpha\gamma\alpha'} [{\boldsymbol{\Lambda}}]^{\alpha'\beta}_\times +
[{\boldsymbol{\Lambda}}]^{\alpha\alpha'}_\times \epsilon_{\alpha'\gamma\beta}\right )\cr
&=-\frac{1}{2}\epsilon_{\alpha\gamma\beta} +
\frac{1}{\Lambda}\frac{df }{d\Lambda}
[{\boldsymbol{\Lambda}}]^{\alpha\alpha'}_\times
[{\boldsymbol{\Lambda}}]^{\alpha'\beta}_\times {\Lambda}_\gamma
+ f(\Lambda) \left (
\epsilon_{\alpha\gamma\alpha'} \epsilon_{\alpha'\beta'\beta}{\Lambda}_{\beta'} +
\epsilon_{\alpha\beta'\alpha'} \epsilon_{\alpha'\gamma\beta}{\Lambda}_{\beta'}\right )\cr
&=-\frac{1}{2}\epsilon_{\alpha\gamma\beta} +
\frac{1}{\Lambda}\frac{df }{d\Lambda}
[{\boldsymbol{\Lambda}}]^{\alpha\alpha'}_\times
[{\boldsymbol{\Lambda}}]^{\alpha'\beta}_\times {\Lambda}_\gamma
+ f(\Lambda)  {\Lambda}_{\beta'} \left (
\delta_{\alpha\beta'}\delta_{\gamma\beta} - \delta_{\alpha\beta}\delta_{\beta'\gamma} +
\delta_{\alpha\gamma}\delta_{\beta'\beta} - \delta_{\alpha\beta}\delta_{\gamma\beta'}
\right )\cr
&=-\frac{1}{2}\epsilon_{\alpha\gamma\beta} +
\frac{1}{\Lambda}\frac{df }{d\Lambda}
\left [[{\boldsymbol{\Lambda}}]_\times\esc [{\boldsymbol{\Lambda}}]_\times
\right ]^{\alpha\beta}{\Lambda}_\gamma + f(\Lambda)  \left (
\delta_{\alpha\gamma}{\Lambda}_\beta +
\delta_{\beta\gamma}{\Lambda}_\alpha -
2\delta_{\alpha\beta}{\Lambda}_\gamma \right )
\end{align}
where the following property of the Levi-Civita symbol has been used as needed
\begin{align}
\label{eq:604}
\epsilon_{ijk}\epsilon_{\ell mn}=
\delta_{i\ell}[\delta_{jm}\delta_{kn}-\delta_{jn}\delta_{km}] -
\delta_{im}[\delta_{j\ell}\delta_{kn}-\delta_{jn}\delta_{k\ell}] +
\delta_{in}[\delta_{j\ell}\delta_{km}-\delta_{jm}\delta_{k\ell}]
\end{align}
Contracting over the indices $\alpha$ and $\gamma$ (or equivalently over $\beta$
and $\gamma$) then gives
\begin{align}
\label{eq:606}
\frac{\partial}{\partial{\boldsymbol{\Lambda}}}\esc
{\bf B}({\boldsymbol{\Lambda}})=
\frac{\partial}{\partial{\boldsymbol{\Lambda}}}\esc
{\bf B}^T({\boldsymbol{\Lambda}}) = 2f(\Lambda)  {\boldsymbol{\Lambda}}=
\left( \frac{2}{\Lambda} - \cot\frac{\Lambda}{2}\right ){\bf n}
\end{align}
Further multiplying this result by ${\bf B}^{-1}$ and using (\ref{eq:138}) gives
%\begin{shaded}
\begin{align}
\label{eq:173}
\left ( \frac{\partial}{\partial{\boldsymbol{\Lambda}}}\esc
{\bf B}({\boldsymbol{\Lambda}})\right ) \esc
{\bf B}^{-1}({\boldsymbol{\Lambda}}) = \left( \frac{2}{\Lambda} -
\cot\frac{\Lambda}{2}\right ){\bf n} =
-\frac{\partial}{\partial {\boldsymbol{\Lambda}}}
\ln\left(\frac{1-\cos\Lambda}{\Lambda^2}\right)
\end{align}
%\end{shaded}
which in component form is
  \begin{align}
    \label{eq:723}
      \frac{\partial {B}_{\alpha\beta}}{\partial{\Lambda}_\alpha}{B}^{-1}_{\beta\gamma}
  & =   \left(\frac{2}{\Lambda}  -\cot\frac{\Lambda}{2}\right){\bf n}_\gamma=-\frac{\partial }{\partial {\Lambda}_\gamma}\ln\left(\frac{1-\cos\Lambda}{\Lambda^2}\right)
  \end{align}



\subsection{The  relationship between the angular  velocity and the time
  derivative of the orientation}
  \label{App:velang}
  From the definition of the angular velocity matrix in terms of the rotation matrix we have
  (we omit hats  denoting phase functions
  for the sake of clarity)
\begin{align}
  \label{eq:501}
  [{\boldsymbol{\omega}}]_\times
  &=\left(\frac{d}{dt}e^{[{\boldsymbol{\Lambda}}]_\times}\right)\esc e^{-[{\boldsymbol{\Lambda}}]_\times}
    =-e^{[{\boldsymbol{\Lambda}}]_\times}\esc \left(\frac{d}{dt}e^{-[{\boldsymbol{\Lambda}}]_\times}\right)
\end{align}
and applying (\ref{WilcoxLambda})
\begin{align}
  \label{eq:22b}
    \frac{d}{dt}e^{[{\boldsymbol{\Lambda}}]_\times}=\int_0^1dx e^{(1-x)[{\boldsymbol{\Lambda}}]_\times}\esc
\left[  \frac{d{\boldsymbol{\Lambda}}}{dt}\right]_\times\esc e^{x[{\boldsymbol{\Lambda}}]_\times}
\end{align}
Using this relation in (\ref{eq:501}) along with a change
of variable $x\rightarrow 1-x$ gives
\begin{align}
  \label{eq:285}
  [{\boldsymbol{\omega}}]_\times
  &=\int_0^1dx   e^{x[{\boldsymbol{\Lambda}}]_\times}\esc
    [\dot{{\boldsymbol{\Lambda}}}]_\times \esc e^{-x[{\boldsymbol{\Lambda}}]_\times}
\overset{(\ref{eq:601})}{=}\int_0^1dx  
    \left[ e^{x[{\boldsymbol{\Lambda}}]_\times}\esc
\dot{{\boldsymbol{\Lambda}}}\right]_\times
    =
    \left[ \int_0^1dx  e^{x[{\boldsymbol{\Lambda}}]_\times}\esc
\dot{{\boldsymbol{\Lambda}}}\right]_\times 
    \overset{(\ref{eq:281})}{=}
    \left[{\bf B}^{-1}(\boldsymbol{\Lambda})\esc
\dot{{\boldsymbol{\Lambda}}}\right]_\times 
    \end{align}
Therefore, we obtain a
neat expression  relating the  angular velocity  \textit{vector} with  the time
derivative of the orientation
\begin{align}
  \label{eq:136}
  {\boldsymbol{\omega}}&=  {\bf B}^{-1}({\boldsymbol{\Lambda}})\esc
\dot{{\boldsymbol{\Lambda}}}\\
  \label{eq:283}
\dot{{\boldsymbol{\Lambda}}}&=  {\bf B}({\boldsymbol{\Lambda}})\esc    {\boldsymbol{\omega}}
\end{align}
Another expression  used in the main text is
\begin{align}
  \label{eq:28}
  e^{-[\boldsymbol{\Lambda}]_\times}\esc\frac{d}{dt}\left(
  e^{[\boldsymbol{\Lambda}]_\times}\right)
  &\overset{(\ref{eq:501})}{=}
    e^{-[\boldsymbol{\Lambda}]_\times}\esc
    [\boldsymbol{\omega}]_\times      e^{[\boldsymbol{\Lambda}]_\times}
\overset{(\ref{eq:601})}{=}
    \left[e^{-[\boldsymbol{\Lambda}]_\times}\esc\boldsymbol{\omega}\right]_\times
    =\left[\boldsymbol{\omega}_{0}\right]_\times
\end{align}

For the sake of completeness, we recover known expressions that relate
angle and  axis variables with  the angular velocity of  the principal
axis. Noting that
$2{{\bf n}} \esc {\dot{{{\bf n}}}}=d{{\bf n}}^2/dt=0$, and by using
\begin{align}
  \label{eq:132}
  [{\bf n}]_\times  \cdot \dot{{\boldsymbol{\Lambda}}}
  &={\Lambda}({{\bf n}}\times\dot{{\bf n}})
    \nonumber\\
  [{\bf n}]_\times\esc[{\bf n}]_\times
  \cdot \dot{{\boldsymbol{\Lambda}}}
  &={\bf n}\times({\bf n}\times\dot{{\boldsymbol{\Lambda}}})
   ={\bf n}({\bf n}\esc\dot{{\boldsymbol{\Lambda}}})                                          -\dot{{\boldsymbol{\Lambda}}}
   ={\bf n}({\bf n}\esc(\dot{\Lambda}{\bf n}+\Lambda\dot{{\bf n}})-\dot{\Lambda}{\bf n}-\Lambda\dot{{\bf n}}
   ={\bf n}\dot{\Lambda}-\dot{\Lambda}{\bf n}-\Lambda\dot{{\bf n}}=-\Lambda\dot{{\bf n}}
\end{align}
in (\ref{eq:136}) with (\ref{eq:138}) we have
\begin{align}
  \label{eq:149}
    {\boldsymbol{\omega}}
    &= {\dot{{\Lambda}}}{\bf n}
      + \sin{\Lambda} {\dot{{\bf n}}}+(1-\cos{\Lambda})({\bf n}\times{\dot{{\bf n}}})
\end{align}
which coincides with Eq. (3.52) of Ref. \cite{Olguin2019}.  On  the other hand,
from (\ref{eq:139}) and (\ref{eq:140}) in (\ref{eq:283}) we obtain
\begin{align}
        \label{eq:64}
  \dot{{\boldsymbol{\Lambda}}}
  &={\boldsymbol{\omega}}
    +\frac{\Lambda}{2} ({\boldsymbol{\omega}}\times {\bf n})
    +\left(1-\frac{\Lambda}{2}\cot\frac{\Lambda}{2}\right)
    {\bf n}\times ({\bf n}\times {\boldsymbol{\omega}})
\end{align}
Since $\dot{\Lambda}={\bf n}\esc\dot{{\boldsymbol{\Lambda}}}$ and from
(\ref{eq:132})
$\dot{{\bf                n}}=-[{\bf               n}]_\times\esc[{\bf
  n}]_\times\cdot\dot{{\boldsymbol{\Lambda}}}/\Lambda$,  (\ref{eq:64})
can be used with (\ref{eq:319}) to show that
\begin{align}
  \label{eq:226}
  \dot{\Lambda}&={\boldsymbol{\omega}}\esc{\bf n}
                \nonumber\\
  \dot{{\bf n}}&=\frac{1}{2}{\boldsymbol{\omega}}\times{\bf n}-\frac{1}{2}\cot\frac{\Lambda}{2}{\bf n}\times({\bf n}\times{\boldsymbol{\omega}})
\end{align}
This coincides with Eq.~(4.11.a) of Ref. \cite{Olguin2019}.


\subsection{Derivative of the inertia tensor  with respect to the orientation}
\label{App:DerI}
The derivative of  the inertia tensor with respect  to the orientation
is a third order tensor given by
\begin{align}
  \label{eq:169}
  \frac{\partial}{\partial{\Lambda}_\gamma}{\bf I}^{-1}_{\alpha\beta}
  &= \frac{\partial}{\partial{\Lambda}_\gamma}
    \left([e^{[\boldsymbol{\Lambda}]_\times}]_{\alpha\alpha'}
    \mathbb{I}^{-1}_{\alpha'\beta'}[e^{-[\boldsymbol{\Lambda}]_\times}]_{\beta'\beta}\right)
   \nonumber\\
&=\left(\frac{\partial[e^{[\boldsymbol{\Lambda}]_\times}]_{\alpha\alpha'}}{\partial{\Lambda}_\gamma}\right)
\mathbb{I}^{-1}_{\alpha'\beta'}[e^{-[\boldsymbol{\Lambda}]_\times}]_{\beta'\beta}
    +[e^{[\boldsymbol{\Lambda}]_\times}]_{\alpha\alpha'}
                 \mathbb{I}^{-1}_{\alpha'\beta'}
\left(\frac{\partial[e^{-[\boldsymbol{\Lambda}]_\times}]_{\beta'\beta}}{\partial{\Lambda}_\gamma}\right)
\nonumber\\
 &=\frac{\partial[e^{[\boldsymbol{\Lambda}]_\times}]_{\alpha\alpha'}}{\partial{\Lambda}_\gamma}
\left[    e^{-[\boldsymbol{\Lambda}]_\times}\right]_{\alpha'\beta'}{\bf I}^{-1}_{\beta'\beta}
    +{\bf I}^{-1}_{\alpha\alpha'}
\left[e^{[\boldsymbol{\Lambda}]_\times}\right]_{\alpha'\beta'}\frac{\partial[e^{-[\boldsymbol{\Lambda}]_\times}]_{\beta'\beta}}{\partial{\Lambda}_\gamma}
\end{align}
We may now use the last two identities in (\ref{eq:183})
to write  (\ref{eq:169}) in the form
\begin{align}
  \label{eq:108}
  \frac{\partial}{\partial{\Lambda}_\gamma}{\bf I}^{-1}_{\alpha\beta}
  &=\epsilon_{\alpha\gamma'\beta'}{B}^{-1}_{\gamma'\gamma}{\bf I}^{-1}_{\beta'\beta}
+{\bf I}^{-1}_{\alpha\alpha'}\epsilon_{\beta\gamma'\alpha'}{B}^{-1}_{\gamma'\gamma}
\end{align}
Note that both the left and right hand sides are symmetric under the permutation of
$\alpha,\beta$.
Now, (\ref{eq:108}) can be used to get the derivative of the rotational kinetic energy as
\begin{align}
  \label{eq:170}
  \frac{\partial K^{\rm rot}}{\partial {\Lambda}_\gamma}
  &=  \frac{\partial}{\partial{\Lambda}_\gamma}
    \left(\frac{1}{2}  {\bf S}^\alpha{\bf I}^{-1}_{\alpha\beta}  {\bf S}^\beta\right)
    =
    \left(\frac{1}{2}  {\bf S}^\alpha\left[\epsilon_{\alpha\gamma'\beta'}{B}^{-1}_{\gamma'\gamma}{\bf I}^{-1}_{\beta'\beta}
+{\bf I}^{-1}_{\alpha\alpha'}\epsilon_{\beta\gamma'\alpha'}{B}^{-1}_{\gamma'\gamma}\right] 
 {\bf S}^\beta\right)
    \nonumber\\
    &=
\frac{1}{2}  \left[{\bf S}^\alpha\epsilon_{\alpha\gamma'\beta'}{B}^{-1}_{\gamma'\gamma}\boldsymbol{\Omega}^{\beta'}
+\boldsymbol{\Omega}^{\alpha'}\epsilon_{\beta\gamma'\alpha'}{B}^{-1}_{\gamma'\gamma}{\bf S}^\beta\right] 
    \nonumber\\
  &=\left(\boldsymbol{\Omega}\times{\bf S}\right)^{\gamma'}{B}^{-1}_{\gamma'\gamma}
\end{align}
or in vector form
%\begin{shaded}
\begin{align}
  \label{eq:192}
    \frac{\partial K^{\rm rot}}{\partial \boldsymbol{\Lambda}} =
    {\bf B}^{-T}( \boldsymbol{\Lambda})\esc
 (\boldsymbol{\Omega}\times{\bf S}) 
\end{align}
%\end{shaded}
Multiplying  (\ref{eq:108}) by  $B_{\gamma\alpha}$ and  summing
over  $\alpha$ and  $\gamma$, and  noting  the full  contraction of  a
product of symmetric and antisymmetric tensors vanishes, gives
\begin{align}
  \label{eq:261}
{B}_{\gamma\alpha}
\frac{\partial{\bf I}^{-1}_{\alpha\beta}}
{\partial{\Lambda}_\gamma}=0
\end{align}
which implies
\begin{align}
  \label{eq:25}
  {B}_{\gamma\alpha}\frac{\partial\boldsymbol{\Omega}^\alpha}{\partial{\Lambda}_\gamma}=0
\end{align}
a result that will be used later.

\newpage
\setcounter{my_equation_counter}{\value{equation}}
\section{Effect of symmetries on orientation and central moments}
\setcounter{equation}{\value{my_equation_counter}}

\label{App:Sym}
\subsection{Translations}
The  gyration tensor  is invariant under a  translation, that
is,     if     ${\bf     r}_{i}'={\bf    r}_{i}+{\bf     a}$,     then
$\hat{\bf  G}(\{{\bf  r}'_{i}\})=\hat{\bf G}(\{{\bf  r}_{i}\})$.  This
implies the orientation is also  invariant under a global translation,
so
\begin{align}
  \hat{\boldsymbol{\Lambda}} (\{{\bf r}_{i }+{\bf a}\})&=
  \hat{\boldsymbol{\Lambda}} (\{{\bf r}_{i }\})
\label{la}
\end{align}
where ${\bf a}$  is an arbitrary translation. As a  consequence, if we
take the derivative of both sides  of (\ref{la}) with respect to ${\bf
  a}$ and then set ${\bf a}=0$, we have
\begin{align}
  \sum_{i }\frac{ \partial \hat{\boldsymbol{\Lambda}} }{\partial {\bf r}_{i }}&=0
\label{trans}
\end{align}
We have a similar expression for translational invariance for the central moments
\begin{align}
  \sum_{i }\frac{ \partial \hat{{\bf M}} }{\partial {\bf r}_{i }}
  &=0                                                                                               % \nonumber\\
\label{trans2}
\end{align}
\subsection{Rotations}
\label{App:Orientation}
  The orientation $\hat{\boldsymbol{\Lambda}}$ does not transform as a
  vector under rotations.  To see  this consider an arbitrary rotation
  matrix  $\boldsymbol{\cal  Q}$  from the  inertial  reference  frame
  $\cal  S$  to  a  frame  ${\cal  S}^\prime$  so  that  the  particle
  coordinates change accordingly
\begin{align}
{\bf r}_i^\prime&=\boldsymbol{\cal Q}\esc{\bf r}_i\cr
{\bf r}_i&=\boldsymbol{\cal Q}^T\esc{\bf r}_i^\prime
\end{align}
Let $\hat{\boldsymbol{\cal R}}$ be  the rotation matrix
from $\cal S$ to the frame  ${\cal S}^0$ which diagonalizes the moment
of  inertia tensor.   Let  $\hat{\boldsymbol{\cal  R}}^\prime$ be  the
rotation  matrix  from  ${\cal  S}^\prime$ to  ${\cal  S}^0$  so  that
$\hat{\boldsymbol{\cal
    R}}^\prime=e^{-[\hat{\boldsymbol{\Lambda}}(\{{\bf
    r}_i^\prime\})]_\times}$.  This  expression follows from  the fact
that $\hat{\boldsymbol{\Lambda}}$  provides the  orientation necessary
to  transform to  ${\cal S}^0$  regardless the  current frame.   Hence
$\hat{\boldsymbol{\Lambda}}(\{{\bf r}_i^\prime\})$  is the orientation
of  ${\cal   S}^0$  relative   to  ${\cal  S}^\prime$.    Because  the
transformation  ${\cal   S}\rightarrow{\cal  S}^0$  is  the   same  as
${\cal S}\rightarrow{\cal S}^\prime\rightarrow{\cal  S}^0$, it follows
that
\begin{align}
  \hat{\boldsymbol{\cal R}}&=\hat{\boldsymbol{\cal R}}^\prime\esc
                               \boldsymbol{\cal Q}\cr
\label{eq:239}
e^{-[\hat{\boldsymbol{\Lambda}}(\{{\bf r}_i\})]_\times}&=
e^{-[\hat{\boldsymbol{\Lambda}}(\{\boldsymbol{\cal Q}\esc{\bf r}_i\})]_\times}\esc \boldsymbol{\cal Q}
\end{align}
Using the series expansion of the exponential and
properties (\ref{eq:601}) we may easily arrive at
\begin{align}
  \label{eq:24b}
  e^{-[\hat{\boldsymbol{\Lambda}}(\{{\bf r}_i\})]_\times}
  &=
    \boldsymbol{\cal Q}\esc
    \boldsymbol{\cal Q}^Te^{-[\hat{\boldsymbol{\Lambda}}(\{\boldsymbol{\cal Q}\esc{\bf r}_i\})]_\times}\esc \boldsymbol{\cal Q}
    \nonumber\\
    &=
    \boldsymbol{\cal Q}\esc
    e^{-\boldsymbol{\cal Q}^T\esc\  [\hat{\boldsymbol{\Lambda}}(\{\boldsymbol{\cal Q}\esc{\bf r}_i\})]_\times\ \esc\ \boldsymbol{\cal Q}}
      \nonumber\\
  &=
    \boldsymbol{\cal Q}\esc
    e^{-[\boldsymbol{\cal Q}^T\esc\ \hat{\boldsymbol{\Lambda}}(\{\boldsymbol{\cal Q}\esc{\bf r}_i\})]_\times}
\end{align}
Should the  matrix $ \boldsymbol{\cal Q}$  multiplying the exponential
on the right hand side of  (\ref{eq:24b}) not be present, this equation
would                            imply                          
$\hat{\boldsymbol{\Lambda}}(\{\boldsymbol{\cal              Q}\esc{\bf
  r}_i\})=\boldsymbol{\cal     Q}\esc\hat{\boldsymbol{\Lambda}}(\{{\bf
  r}_i\})$, which is the way a  vector would transform. Because of the
presence of  $ \boldsymbol{\cal Q}$ multiplying  the exponential, this
is not true, and the orientation does not transform as a vector.

Equation (\ref{eq:239}) allows us  to obtain an important relationship
for  the  orientation.  Take  the  derivative  of (\ref{eq:239})  with
respect  to  the   orientation  ${\bf  A}$  of   the  rotation  matrix
$\boldsymbol{\cal  Q}=e^{[{\bf  A}]_\times}$.    The left  hand  side  does not  depend  on
${\bf A}$ so
\begin{align}
  \label{eq:242}
\frac{\partial}{\partial {\bf A}} \left(e^{-[\hat{\boldsymbol{\Lambda}}(Q r)]_\times}\esc\boldsymbol{\cal Q}\right)&=  0
\end{align}
which can be arranged in the form
\begin{align}
  \label{eq:459}
\left(\frac{\partial}{\partial {\bf A}} \boldsymbol{\cal Q}\right)\esc\ \boldsymbol{\cal Q}^T +
e^{[\hat{\boldsymbol{\Lambda}}(Q r)]_\times}\esc\ 
\frac{\partial}{\partial {\bf A}} \left(e^{-[\hat{\boldsymbol{\Lambda}}(Q r)]_\times}\right)
&=  0
\end{align}
Using (\ref{eq:457}) for the
  derivative of  $\boldsymbol{\cal Q}$ evaluated at ${\bf A}=0$ gives
\begin{align}
  \label{eq:711}
\left.  \frac{\partial \left[e^{-[{\bf A}]_\times}\right]_{\alpha\beta}}{\partial {\bf A}^\gamma}\right|_{{\bf A}=0
  }
  &=  \epsilon_{\beta\gamma\alpha}
\end{align}
Now use (\ref{WilcoxLambda}) for the derivative of  the second
  term in (\ref{eq:459}) and write in component form as
\begin{align}
  \label{eq:276}
\left[e^{[{\boldsymbol{\Lambda}}(Q r)]_\times}\right]_{\alpha\delta}
\frac{\partial}{\partial {\bf A}^\gamma}
\left[e^{-\left[ \hat{\boldsymbol{\Lambda}}(Q r)\right]_\times}\right ]_{\delta\beta} =
  \int_0^1dx \left[e^{x[\hat{\boldsymbol{\Lambda}}(Q r)]_\times}\right]_{\alpha\alpha'}
  \left(  \frac{\partial}{\partial {\bf A}^\gamma}\left[ \hat{\boldsymbol{\Lambda}}(Q r)\right]^{\alpha'\beta'}_\times\right) \left[e^{-x[\hat{\boldsymbol{\Lambda}}(Q r)]_\times}\right]_{\beta'\beta}
\end{align}
The chain rule gives the following expression
\begin{align}
  \label{eq:273}
  \frac{\partial}{\partial{\bf A}^\gamma}  \left[\hat{\boldsymbol{\Lambda}}(Q r)\right]_\times^{\alpha'\beta'}
  &=\epsilon_{\alpha'\gamma'\beta'}\sum_i  \frac{\partial \hat{\Lambda}_{\gamma'}}{\partial{\bf r}_i^{\alpha''}}(Q r) \frac{\partial}{\partial{\bf A}^\gamma}\left (\boldsymbol{\cal Q}^{\alpha''\beta''}\right ){\bf r}_i^{\beta''}
\end{align}
Evaluated at ${\bf A}=0$ and using (\ref{eq:457}) this becomes
\begin{align}
  \label{eq:274}
  \left.    \frac{\partial}{\partial{\bf A}^\gamma}  \left[\hat{\boldsymbol{\Lambda}}(Q r)\right]_\times^{\alpha'\beta'}\right|_{{\bf A}=0}
  &=\epsilon_{\alpha'\gamma'\beta'}\sum_i  \frac{\partial \hat{\Lambda}_{\gamma'}}{\partial{\bf r}_i^{\alpha''}}(r)
    \epsilon_{\alpha''\gamma\beta''}{\bf r}_i^{\beta''}
    =-\epsilon_{\alpha'\gamma'\beta'}
    \hat{O}_{\gamma'\gamma}(r)
\end{align}
where we have introduced
\begin{align}
  \label{eq:407}
    \hat{O}_{\gamma'\gamma}(r)&\equiv\sum_i^N      \frac{\partial \hat{\Lambda}_{\gamma'}}{\partial {\bf r}^{\alpha''}_i} [{\bf r}_i]_\times^{\alpha''\gamma}    
  \end{align}
Putting all these terms in (\ref{eq:459}) and evaluating at ${\bf A}=0$ gives
  \begin{align}
    \label{eq:277}
\int_0^1dx \left[e^{x[\hat{\boldsymbol{\Lambda}}(r)]_\times}\right]_{\alpha\alpha'}\left[
    -\epsilon_{\alpha'\gamma'\beta'}
    \hat{O}_{\gamma'\gamma}\right] \left[e^{-x[\hat{\boldsymbol{\Lambda}}(r)]_\times}\right]_{\beta'\beta}
    &=  -\epsilon_{\beta\gamma\alpha}
  \end{align}
Matching the integrand to that of (\ref{eq:60}) followed by the use of
 (\ref{eq:183}) then gives
  \begin{align}
\hat{O}_{\gamma'\gamma}
\left ( \frac{\partial}{\partial{\Lambda}_{\gamma'}}
\left[e^{[{\boldsymbol{\Lambda}}]_\times}\right]_{\alpha\beta'}\right )
\left[e^{-[\hat{\boldsymbol{\Lambda}}]_\times}\right]_{\beta'\beta}
    &= \epsilon_{\beta\gamma\alpha}\nonumber\\
\hat{O}_{\gamma'\gamma}\epsilon_{\alpha\alpha'\beta}
{B}_{\alpha'\gamma'}^{-1}({\boldsymbol{\Lambda}})
    &= \epsilon_{\beta\gamma\alpha}\nonumber\\
-\epsilon_{\beta\alpha'\alpha}
{B}_{\alpha'\gamma'}^{-1}({\boldsymbol{\Lambda}})
\hat{O}_{\gamma'\gamma}
&= \epsilon_{\beta\gamma\alpha}
  \end{align}
  that implies
  \begin{align}
    \label{eq:160b}
    {B}_{\beta'\gamma'}^{-1}(\hat{\boldsymbol{\Lambda}})\hat{O}_{\gamma'\gamma}=-\delta_{\beta'\gamma}
  \end{align}
In  conclusion, the  transformation  property of  the orientation
(\ref{eq:239}) is reflected into the following mathematical identity
\begin{align}
 \label{eq:255}
  \hat{\bf O}
  &= \sum_i^N      \frac{\partial \hat{\boldsymbol{\Lambda}}}{\partial {\bf r}_i}\esc [{\bf r}_i]_\times
    =-{\bf B}\left(     \hat{\boldsymbol{\Lambda}}\right)
    % =-\left(\mathbb{1}-\frac{\hat{\lambda}}{2} [\hat{\bf n}]_\times
    % +\left(1-\frac{\hat{\lambda}}{2}\cot\frac{\hat{\lambda}}{2}\right)[\hat{\bf n}]_\times[\hat{\bf n}]_\times\right)
\end{align}
This   equation  shows   that   the  matrix   valued  phase   function
$\hat{\bf  O}$  depends on  the  configuration  $r$ only  through  the
orientation.

  Let us
now consider  the transformation  properties of the  central moments
and of the  dilation momentum.  Because the  eigenvalues are invariant
under rotations, we know that
\begin{align}
  \label{eq:47}
\hat{{\bf M}}(Qr)=\hat{{\bf M}}(r)
\end{align}
By   following  an   argument  identical   to  the   one  leading   to
(\ref{eq:255})  we  may  take  the derivative  of  (\ref{eq:47})  with
respect to ${\bf  A}$ and evaluate the final result  at ${\bf A}=0$ to
get the identity
\begin{align}
0=\left.\frac{\partial}{\partial {\bf A}}  \hat{{\bf M}} \left({\cal Q}r\right)\right|_{{\bf A}=0}=
  \left.\sum_{i }  \frac{\partial \hat{{\bf M}} }{\partial {\bf r}_i}\esc\frac{\partial e^{[{\bf A}]_\times}}{\partial {\bf A}}\right|_{{\bf A}=0}\esc {\bf r}_{i }
  \overset{(\ref{eq:711})}{=}
\sum_{i }  [{\bf r}_{i }]_\times\esc\frac{\partial \hat{{\bf M}} }{\partial {\bf r}_i}
\label{dsym}
\end{align}
On the other hand, if we take the derivative of (\ref{eq:47}) and
use the chain rule we get
\begin{align}
  \label{eq:44}
\frac{\partial}{\partial {\bf r}^\alpha_i}  \hat{{\bf M}}(r)= 
  \frac{\partial}{\partial {\bf r}^\alpha_i}  \hat{{\bf M}}(Qr)
  &=\sum_j \frac{\partial \hat{{\bf M}}}{\partial {\bf r}_j^{\beta}}(Qr)
    \boldsymbol{\cal Q}^{\beta \alpha'}\frac{\partial {\bf r}_j^{\alpha'}}{\partial {\bf r}_i^{\alpha}}
    =\frac{\partial \hat{{\bf M}}}{\partial {\bf r}_i^{\beta}}(Qr)
    \boldsymbol{\cal Q}^{\beta \alpha}
\end{align}
 This implies, recalling that $\boldsymbol{\cal Q}^{\alpha\alpha'}\boldsymbol{\cal Q}^{\beta\alpha'}=\delta_{\alpha\beta}$, 
\begin{align}
  \label{eq:49}
  \frac{\partial \hat{{\bf M}}}{\partial {\bf r}_i^{\alpha}}(Qr)
=  \boldsymbol{\cal Q}^{\alpha\beta }\frac{\partial\hat{{\bf M}}}{\partial {\bf r}^\beta_i}  (r)  
\end{align}
As a consequence,  the dilation momentum satisfies
\begin{align}
  \label{eq:51}
  \hat{\boldsymbol{\Pi}}(Qz)
  &=\sum_i\frac{\partial \hat{{\bf M}}}{\partial {\bf r}_i^\alpha}(Qr)
    \boldsymbol{\cal Q}^{\alpha\alpha'}{\bf v}_i^{\alpha'}  =
    \sum_i \boldsymbol{\cal Q}^{\alpha\beta }\frac{\partial\hat{{\bf M}}}{\partial {\bf r}^\beta_i}  (r)
\boldsymbol{\cal Q}^{\alpha\alpha'}{\bf v}_i^{\alpha'}
    =    \sum_i \frac{\partial\hat{{\bf M}}}{\partial {\bf r}^\beta_i}  (r)
{\bf v}_i^{\beta}=\hat{\boldsymbol{\Pi}}(z)
\end{align}
The dilation momentum  is invariant under rotation  of the coordinates
as a consequence  of the central moments also  being invariant under
rotation.   As   a  further   development,  take  the   derivative  of
(\ref{eq:49})  with  respect  to  ${\bf  A}^\gamma$  and  evaluate  at
${\bf A}=0$. The left hand side gives
\begin{align}
  \label{eq:263}
\left .  \frac{\partial}{\partial{\bf A}^\gamma}    \frac{\partial \hat{{\bf M}}}{\partial {\bf r}_i^{\alpha}}(Qr)\right|_{{\bf A}=0}
  &=
    \left . \sum_j   \frac{\partial^2 \hat{{\bf M}}}
    {\partial {\bf r}_i^{\alpha}\partial {\bf r}_j^{\beta}}(Qr)
    \frac{\partial {\boldsymbol{\cal Q}}^{\beta\beta'}{\bf r}_j^{\beta'}}{\partial{\bf A}^\gamma}\right|_{{\bf A}=0}
\overset{(\ref{eq:711})}{=}
    \sum_j   \frac{\partial^2 \hat{{\bf M}}}
    {\partial {\bf r}_i^{\alpha}\partial {\bf r}_j^{\beta}}(r)
    \epsilon_{\beta\gamma\beta'}{\bf r}_j^{\beta'}
    =-\sum_j   \frac{\partial^2 \hat{{\bf M}}}
    {\partial {\bf r}_i^{\alpha}\partial {\bf r}_j^{\beta}}(r)[{\bf r}_j]_\times^{\beta\gamma}
\end{align}
The right hand side gives, using (\ref{eq:711})
\begin{align}
    \label{eq:266}
\left.    \frac{\partial}{\partial{\bf A}^\gamma}\boldsymbol{\cal Q}^{\alpha\beta }
  \right|_{{\bf A}=0}\frac{\partial\hat{{\bf M}}}{\partial {\bf r}^\beta_i}  (r)  
&=\epsilon_{\alpha\gamma\beta}  \frac{\partial\hat{{\bf M}}}{\partial {\bf r}^\beta_i}  (r)  
\end{align}
Equating both sides we obtain the identity
\begin{align}
  \label{eq:267}
  \sum_j   \frac{\partial^2 \hat{{\bf M}}}
  {\partial {\bf r}_i^{\alpha}\partial {\bf r}_j^{\beta}}(r)[{\bf r}_j]_\times^{\beta\gamma}
  &=\epsilon_{\alpha\beta\gamma}  \frac{\partial\hat{{\bf M}}}{\partial {\bf r}^\beta_i}(r)  
\end{align}
This identity reflects rotational invariance at the level of second derivatives of the central moments. 

\newpage

\setcounter{my_equation_counter}{\value{equation}}
\section{Microscopic expressions}
\setcounter{equation}{\value{my_equation_counter}}
\label{App:Mic}
In this section  we consider the explicit form for  a number of phase
functions in terms  of the positions and velocities  of the particles.
These    expressions    are    required     in    the    main    text.
We will need the transformation of coordinates from the lab to the principal axis system
\begin{align}
  \label{rvmic}
      {\bf r}_{0i}&=\hat{\boldsymbol{\cal R}}\esc({\bf r}_i-\hat{\bf R})
                  \nonumber\\
    {\bf v}_{0i}&\equiv\frac{d{\bf r}_{0i}}{dt}=\hat{\boldsymbol{\cal R}}\esc\left( {\bf v}_i-\hat{\bf V}-[\hat{\boldsymbol{\omega}}]_\times\esc({\bf r}_i-\hat{\bf R})\right)
                  \nonumber\\
    {\bf v}_{pi}&=\hat{\boldsymbol{\cal R}}({\bf v}_i-{\bf V})
                  \nonumber\\
    [\hat{\boldsymbol{\omega}}]_\times
                &\equiv-{\hat{\boldsymbol{\cal R}}}^{T}\esc\frac{d{\hat{\boldsymbol{\cal R}}}}{dt}
                  =\frac{d{\hat{\boldsymbol{\cal R}}}^{T}}{dt}\esc{\hat{\boldsymbol{\cal R}}}
                  \nonumber\\
    [\hat{\boldsymbol{\omega}}_0]_\times
                &\equiv{\hat{\boldsymbol{\cal R}}}\esc\frac{d{\hat{\boldsymbol{\cal R}}}^{T}}{dt}
                  =-\frac{d{\hat{\boldsymbol{\cal R}}}}{dt}\esc{\hat{\boldsymbol{\cal R}}}^T
                  =\hat{\boldsymbol{\cal R}}\esc      [\hat{\boldsymbol{\omega}}]_\times\esc \hat{\boldsymbol{\cal R}}^T
  \end{align}



\subsection{Dilational  and  angular  velocities  in  microscopic terms}

The      dilational      $\hat{\boldsymbol{\nu}}$     and      angular
$\hat{\boldsymbol{\omega}}$    velocities    are    closely    related
concepts. In order  to find explicit expressions for both  in terms of
the positions and momenta of  the particles, we consider the following
sequence of identities
  \begin{align}
    \label{eq:6}
    i{\cal L}\hat{\mathbb{G}}
    &=   i{\cal L}\left(\hat{\boldsymbol{\cal R}}\esc\hat{\bf G} \esc \hat{\boldsymbol{\cal R}}^T\right)
\overset{\mbox{\tiny chain rule}}{=} 
\left(i{\cal L}\hat{\boldsymbol{\cal R}}\right)\esc\hat{\bf G} \esc \hat{\boldsymbol{\cal R}}^T
      +\hat{\boldsymbol{\cal R}}\esc
      \left(i{\cal L}\hat{\bf G} \right)\esc \hat{\boldsymbol{\cal R}}^T
      +\hat{\boldsymbol{\cal R}}\esc
      \hat{\bf G}\esc  \left(i{\cal L}\hat{\boldsymbol{\cal R}}^T \right)
      \nonumber\\
&=\left(i{\cal L}\hat{\boldsymbol{\cal R}}\right)\esc \hat{\boldsymbol{\cal R}}^T\esc \hat{\boldsymbol{\cal R}}\esc\hat{\bf G} \esc \hat{\boldsymbol{\cal R}}^T
      +\hat{\boldsymbol{\cal R}}\esc
      \left(i{\cal L}\hat{\bf G} \right)\esc \hat{\boldsymbol{\cal R}}^T
      +\hat{\boldsymbol{\cal R}}\esc
      \hat{\bf G}\esc  \hat{\boldsymbol{\cal R}}^T\esc \hat{\boldsymbol{\cal R}} \left(i{\cal L}\hat{\boldsymbol{\cal R}}^T \right)
      \nonumber\\
&= \left(i{\cal L}\hat{\boldsymbol{\cal R}}\right)\esc\hat{\boldsymbol{\cal R}}^T\esc \hat{\mathbb{G}}
      +\hat{\boldsymbol{\cal R}}\esc
      \left(i{\cal L}\hat{\bf G} \right)\esc \hat{\boldsymbol{\cal R}}^T
      + \hat{\mathbb{G}}\esc \hat{\boldsymbol{\cal R}}\esc \left(i{\cal L}\hat{\boldsymbol{\cal R}}^T \right)
      \nonumber\\
&\overset{(\ref{rvmic})}{=}-[\hat{\boldsymbol{\omega}}_0]_\times \esc \hat{\mathbb{G}}
      +\hat{\boldsymbol{\cal R}}\esc\left(i{\cal L}\hat{\bf G} \right)\esc \hat{\boldsymbol{\cal R}}^T
+\hat{\mathbb{G}}\esc[\hat{\boldsymbol{\omega}}_0]_\times
  \end{align}
  Reordering terms gives
  \begin{align}
    \label{eq:215}
    i{\cal L}\hat{\mathbb{G}}+[\hat{\boldsymbol{\omega}}_0]_\times \esc \hat{\mathbb{G}}-\hat{\mathbb{G}}\esc[\hat{\boldsymbol{\omega}}_0]_\times
    &=\hat{\boldsymbol{\cal R}}\esc\left(i{\cal L}\hat{\bf G} \right)\esc \hat{\boldsymbol{\cal R}}^T
  \end{align}
which in matrix form becomes
  \begin{align}
    \label{eq:219}
    \begin{pmatrix}
      \hat{\Pi}_x
      & (\hat{M}_1-\hat{M}_2)\hat{\boldsymbol{\omega}}_{0}^z
      & (\hat{M}_3-\hat{M}_1)\hat{\boldsymbol{\omega}}_{0}^y \\
(\hat{M}_1-\hat{M}_2)\hat{\boldsymbol{\omega}}_{0}^z 
      &       \hat{\Pi}_y
      & (\hat{M}_2-\hat{M}_3)\hat{\boldsymbol{\omega}}_{0}^x \\
(\hat{M}_3-\hat{M}_1)\hat{\boldsymbol{\omega}}_{0}^y 
      &      (\hat{M}_2-\hat{M}_3)\hat{\boldsymbol{\omega}}_{0}^x 
      &  \hat{\Pi}_z
      \end{pmatrix}      
    &=\hat{\boldsymbol{\cal R}}\esc\left(i{\cal L}\hat{\bf G} \right)\esc \hat{\boldsymbol{\cal R}}^T
  \end{align}
  As the  gyration tensor depends  on particle's positions,  the right
  hand side is explicit in particle positions and velocities giving
  \begin{align}
     \hat{\boldsymbol{\cal R}}\esc
    \left(i{\cal L}\hat{\bf G} \right)\esc \hat{\boldsymbol{\cal R}}^T
    &=
      \hat{\boldsymbol{\cal R}}\esc
      \frac{1}{4}\sum_im_i(({\bf r}_i-\hat{\bf R})({\bf v}_i-\hat{\bf V})^T+({\bf v}_i-\hat{\bf V})({\bf r}_i-\hat{\bf R})^T)\esc \hat{\boldsymbol{\cal R}}^T
      \nonumber\\
    &=
      \frac{1}{4}\sum_im_i({\bf r}_{0i}{\bf v}^T_{pi}+{\bf v}_{pi}{\bf r}^T_{0i})
  \end{align}
Observe that  the diagonal
  elements   in  (\ref{eq:219})   contain   the  dilational   momentum
 and  the  off-diagonal  terms  contain  the  angular
  velocity  in   the  principal   axis.   Therefore,   the  dilational
  velocities are given from the diagonal components
  \begin{align}
\label{eq:260}     \hat{  \nu}_x
    &\equiv\frac{\hat{{\Pi}}_x}{\hat{M}_1}= \frac{1}{2\hat{M}_1} \sum_i m_i {\bf r}_{0i}^{x}{\bf v}_{pi}^{x}
      \nonumber\\
    \hat{  \nu}_y
    &\equiv\frac{\hat{{\Pi}}_y}{\hat{M}_2}= \frac{1}{2\hat{M}_2} \sum_i m_i {\bf r}_{0i}^{y}{\bf v}_{pi}^{y}
      \nonumber\\
    \hat{  \nu}_z
    &\equiv\frac{\hat{{\Pi}}_z}{\hat{M}_3}=  \frac{1}{2\hat{M}_3} \sum_i m_i {\bf r}_{0i}^{z}{\bf v}_{pi}^{z}
  \end{align}
  From (\ref{eq:219}), the off diagonal terms give the angular velocity as
  \begin{align}
     \label{eq:275}
    \hat{\boldsymbol{\omega}}_0^x&=                                    \frac{1}{4\left(\hat{M}_2 - \hat{M}_3\right)}
                                   \sum_i m_i ({\bf v}_{pi}^y {\bf r}_{0i}^z + {\bf r}_{0i}^y {\bf v}_{pi}^z)
                                   =\frac{1}{\left(\hat{I}_3 - \hat{I}_2\right)}
                                   \sum_i m_i({\bf v}_{pi}^y {\bf r}_{0i}^z + {\bf r}_{0i}^y {\bf v}_{pi}^z)
                                   \nonumber\\
    \hat{\boldsymbol{\omega}}_0^y &=
                                    \frac{1}{4\left(\hat{M}_3 - \hat{M}_1\right)}
                                    \sum_i m_i({\bf v}_{pi}^x {\bf r}_{0i}^z + {\bf r}_{0i}^x {\bf v}_{pi}^z)
                                    =\frac{1}{\left(\hat{I}_1 - \hat{I}_3\right)}
                                    \sum_i m_i({\bf v}_{pi}^x {\bf r}_{0i}^z+{\bf r}_{0i}^x {\bf v}_{pi}^z)
\nonumber\\
\hat{\boldsymbol{\omega}}_0^z &=
\frac{1}{4\left(\hat{M}_1-\hat{M}_2\right)}
\sum_i m_i({\bf v}_{pi}^x {\bf r}_{0i}^y + {\bf r}_{0i}^x {\bf v}_{pi}^y)
=\frac{1}{\left(\hat{I}_2 - \hat{I}_1\right)}
\sum_i m_i({\bf v}_{pi}^x {\bf r}_{0i}^y + {\bf r}_{0i}^x {\bf v}_{pi}^y)
\end{align}
Equations  (\ref{eq:260})  and   (\ref{eq:275})  give  the  dilational
momentum  and angular  velocity in  terms  of the  coordinates of  the
particles  and  the principal  moments  and  orientation (through  the
rotation matrix).   As the latter  are CG variables their  presence is
easy to deal with inside conditional expectations.


\subsection{First derivatives of orientation  and central moments}
\label{App:DerOr}
{We seek to evaluate  $\frac{\partial\hat{\boldsymbol{\Lambda}}}{\partial {\bf r}_i}$ and $\frac{\partial\hat{\bf M}}{\partial {\bf r}_i}$.}
In  principle both  the  orientation and  principal  moments are  very
complicated functions of the coordinates  of the particles and a brute
force calculation  seems not  feasible.  However,  a couple  of tricks
allow  us to  find the  explicit dependence  for derivatives  of these
quantities.          From the diagonalization of the gyration tensor,
$    \hat{\bf G}=e^{[\hat{\boldsymbol{\Lambda}}]_\times}\esc
\hat{\mathbb{G}}\esc e^{-[\hat{\boldsymbol{\Lambda}}]_\times}$
      we        know
$\hat{\bf G}(z)={\bf G}(\hat{\boldsymbol{\Lambda}}(z),\hat{\bf M}(z))$
so that using the chain rule gives
\begin{align}
  \label{eq:251}
  \frac{\partial \hat{\bf G}_{\alpha\beta}}{\partial {\bf r}^\gamma_i}
  &=
    \frac{\partial {\bf G}_{\alpha\beta}}{\partial {\Lambda}_\mu}
    \frac{\partial \hat{\Lambda}_\mu}{\partial {\bf r}^\gamma_i}
    +\frac{\partial {\bf G}_{\alpha\beta}}{\partial {M}_\mu}
    \frac{\partial \hat{M}_\mu}{\partial {\bf r}^\gamma_i}
\end{align}
The derivatives of $\hat{\bf G}$ can be found explicitly, so this equation can be solved for 
$    \frac{\partial \hat{\Lambda}_\mu}{\partial {\bf r}^\gamma_i}
,    \frac{\partial \hat{M}_\mu}{\partial {\bf r}^\gamma_i}$.
The derivative in the first term is
\begin{align}
  \label{eq:169b}
    \frac{\partial{\bf G}_{\alpha\beta}}{\partial{\Lambda}_\mu}
  &= \frac{\partial}{\partial{\Lambda}_\mu}
    \left([e^{[\boldsymbol{\Lambda}]_\times}]_{\alpha\alpha'}
    \mathbb{G}^{\alpha'\beta'}[e^{-[\boldsymbol{\Lambda}]_\times}]_{\beta'\beta}\right)
    \nonumber\\
  &=\left(\frac{\partial[e^{[\boldsymbol{\Lambda}]_\times}]_{\alpha\alpha'}}{\partial{\Lambda}_\mu}\right)
    \mathbb{G}^{\alpha'\beta'}[e^{-[\boldsymbol{\Lambda}]_\times}]_{\beta'\beta}
    +[e^{[\boldsymbol{\Lambda}]_\times}]_{\alpha\alpha'}
\mathbb{G}^{\alpha'\beta'}
    \left(\frac{\partial[e^{-[\boldsymbol{\Lambda}]_\times}]_{\beta'\beta}}
    {\partial{\Lambda}_\mu}\right)
    \nonumber\\
    &=\frac{\partial[e^{[\boldsymbol{\Lambda}]_\times}]_{\alpha\alpha'}}{\partial{\Lambda}_\mu}
    \left[    e^{-[\boldsymbol{\Lambda}]_\times}\right]_{\alpha'\beta'}{\bf G}_{\beta'\beta}
    +{\bf G}_{\alpha\alpha'}
    \left[e^{[\boldsymbol{\Lambda}]_\times}\right]_{\alpha'\beta'}
    \frac{\partial[e^{-[\boldsymbol{\Lambda}]_\times}]_{\beta'\beta}}{\partial{\Lambda}_\mu}
\end{align}
We may now use the last  two identities in (\ref{eq:183}) to write (\ref{eq:169b}) in the form
\begin{align}
  \label{eq:108b}
  \frac{\partial{\bf G}_{\alpha\beta}}{\partial{\Lambda}_\mu}
  &=\left[\epsilon_{\alpha\gamma'\beta'}{\bf G}_{\beta'\beta}
    +{\bf G}_{\alpha\alpha'}\epsilon_{\beta\gamma'\alpha'}\right]{B}^{-1}_{\gamma'\mu}
\end{align}
The  derivative in the second term is
\begin{align}
  \label{eq:31}
  \frac{\partial {\bf G}_{\alpha\beta}}{\partial{M}_\mu}
  &=
    \left[e^{[\boldsymbol{\Lambda}]_\times}\right]_{\alpha\alpha'}
    \frac{\partial \mathbb{G}_{\alpha'\beta'}}{\partial{M}_\mu}
\left[e^{-[\boldsymbol{\Lambda}]_\times}\right]_{\beta'\beta}
=\left[e^{[\boldsymbol{\Lambda}]_\times}\right]_{\alpha\underline{\mu}}
\left[e^{-[\boldsymbol{\Lambda}]_\times}\right]_{\underline{\mu}\beta}
\end{align}
A simple way to solve the  system (\ref{eq:251}) is by transforming to
the principal axis system by multiplying both sides from the left with
$e^{-[\hat{\boldsymbol{\Lambda}}]_\times}$  and  from the  right  with
$e^{[\hat{\boldsymbol{\Lambda}}]_\times}$ to give
\begin{align}
  \label{eq:333}
    \left[e^{-[\hat{\boldsymbol{\Lambda}}]_\times}\right]_{\alpha\alpha'}    \frac{\partial \hat{\bf G}_{\alpha'\beta'}}{\partial {\bf r}^\gamma_i}    \left[e^{[\hat{\boldsymbol{\Lambda}}]_\times}\right]_{\beta'\beta}
  &=
    \left[e^{-[\hat{\boldsymbol{\Lambda}}]_\times}\right]_{\alpha\alpha'}    \frac{\partial {\bf G}_{\alpha'\beta'}}{\partial {\Lambda}_\mu}    \left[e^{[\hat{\boldsymbol{\Lambda}}]_\times}\right]_{\beta}
    \frac{\partial \hat{\Lambda}_\mu}{\partial {\bf r}^\gamma_i}
    \nonumber\\
  &+    \left[e^{-[\hat{\boldsymbol{\Lambda}}]_\times}\right]_{\alpha\alpha'}
    \frac{\partial {\bf G}_{\alpha'\beta'}}{\partial {M}_\mu}    \left[e^{[\hat{\boldsymbol{\Lambda}}]_\times}\right]_{\beta'\beta}
    \frac{\partial \hat{M}_\mu}{\partial {\bf r}^\gamma_i}
\end{align}
Let us compute each term in this equation. Using (\ref{eq:108b}) the first term is
\begin{align}
  \label{eq:334}
& \left[e^{-[\hat{\boldsymbol{\Lambda}}]_\times}\right]_{\alpha\alpha'}    \frac{\partial {\bf G}_{\alpha'\beta'}}{\partial {\Lambda}_\mu}    \left[e^{[\hat{\boldsymbol{\Lambda}}]_\times}\right]_{\beta'\beta}
    \nonumber\\
&\overset{(\ref{eq:108b})}{=}
    \left[e^{-[\hat{\boldsymbol{\Lambda}}]_\times}\right]_{\alpha\alpha'}
    \left[\epsilon_{\alpha'\gamma'\beta''}{\bf G}_{\beta''\beta'}
    +    {\bf G}_{\alpha'\alpha''}\epsilon_{\beta'\gamma'\alpha''}\right]
    B^{-1}_{\gamma'\mu}
    \left[e^{[\hat{\boldsymbol{\Lambda}}]_\times}\right]_{\beta'\beta}
    \nonumber\\
  &=
     \left[e^{-[\hat{\boldsymbol{\Lambda}}]_\times}\right]_{\alpha\alpha'}
   \epsilon_{\alpha'\gamma'\beta''}
    {\bf G}_{\beta''\beta'}\left[e^{[\hat{\boldsymbol{\Lambda}}]_\times}\right]_{\beta'\beta}
    B^{-1}_{\gamma'\mu}
  + \left[e^{-[\hat{\boldsymbol{\Lambda}}]_\times}\right]_{\alpha\alpha'}
     {\bf G}_{\alpha'\alpha''}\epsilon_{\beta'\gamma'\alpha''}
     \left[e^{[\hat{\boldsymbol{\Lambda}}]_\times}\right]_{\beta'\beta}
    B^{-1}_{\gamma'\mu}
    \nonumber\\
&=
     \left[e^{-[\hat{\boldsymbol{\Lambda}}]_\times}\right]_{\alpha\alpha'}
     \epsilon_{\alpha'\gamma'\beta''}
     \left[e^{[\hat{\boldsymbol{\Lambda}}]_\times}\right]_{\beta''\beta'}
     \hat{\mathbb{G}}^{\beta'\beta}
     {B}^{-1}_{\gamma'\mu}
     +\hat{\mathbb{G}}^{\alpha\alpha'}     \left[e^{-[\hat{\boldsymbol{\Lambda}}]_\times}\right]_{\alpha'\alpha''}
     \epsilon_{\beta'\gamma'\alpha''}
     \left[e^{[\hat{\boldsymbol{\Lambda}}]_\times}\right]_{\beta'\beta}
     {B}^{-1}_{\gamma'\mu}
    \nonumber\\
&\overset{(\ref{eq:98})}{=}
      \epsilon_{\alpha\gamma''\beta''}
      \hat{\mathbb{G}}^{\beta''\beta}          \left[e^{-[\hat{\boldsymbol{\Lambda}}]_\times}\right]_{\gamma''\gamma'}
      {B}^{-1}_{\gamma'\mu}
      - \hat{\mathbb{G}}^{\alpha\alpha'}     
      \epsilon_{\alpha'\gamma''\beta}\left[e^{-[\hat{\boldsymbol{\Lambda}}]_\times}\right]_{\gamma''\gamma'}
      {B}^{-1}_{\gamma'\mu}
    \nonumber\\
&=
      \epsilon_{\underline{\alpha}\gamma''\underline{\beta}}
\left ( \hat{M}_{\underline{\beta}} - \hat{M}_{\underline{\alpha}}\right )
\left[e^{-[\hat{\boldsymbol{\Lambda}}]_\times}\right]_{\gamma''\gamma'}
      {B}^{-1}_{\gamma'\mu}
    \end{align}
Using (\ref{eq:31}) the second term is
\begin{align}
  \label{eq:339}
  \left[e^{-[\hat{\boldsymbol{\Lambda}}]_\times}\right]_{\alpha\alpha'}
  \frac{\partial {\bf G}_{\alpha'\beta'}}{\partial {M}_\mu}    \left[e^{[\hat{\boldsymbol{\Lambda}}]_\times}\right]_{\beta'\beta}
  &=
  \left[e^{-[\hat{\boldsymbol{\Lambda}}]_\times}\right]_{\alpha\alpha'}
\left[e^{[\boldsymbol{\Lambda}]_\times}\right]_{\alpha'\underline{\mu}}
\left[e^{-[\boldsymbol{\Lambda}]_\times}\right]_{\underline{\mu}\beta'}
  \left[e^{[\hat{\boldsymbol{\Lambda}}]_\times}\right]_{\beta'\beta}
=\delta_{\alpha\underline{\mu}}\delta_{\beta\underline{\mu}}=
    \delta_{\alpha\beta\mu}
\end{align}
where  the three indexed Kronecker delta  $\delta_{\alpha\beta\mu}$    takes    the    value    1    if
$\alpha=\beta=\mu$  and 0  otherwise. Putting  together (\ref{eq:333})
with (\ref{eq:334}) and (\ref{eq:339}) gives
\begin{align}
  \label{eq:126}
  \left[e^{-[\hat{\boldsymbol{\Lambda}}]_\times}\right]_{\alpha\alpha'}
  \frac{\partial    \hat{\bf G}_{\alpha'\beta'}}{\partial {\bf r}^\gamma_i}
  \left[e^{[\hat{\boldsymbol{\Lambda}}]_\times}\right]_{\beta'\beta}
  &=
      \epsilon_{\underline{\alpha}\gamma''\underline{\beta}}
\left ( \hat{M}_{\underline{\beta}} - \hat{M}_{\underline{\alpha}}\right )
    \left[e^{-[\hat{\boldsymbol{\Lambda}}]_\times}\right]_{\gamma''\gamma'}
      {B}^{-1}_{\gamma'\mu}  \frac{\partial \hat{\Lambda}_\mu}{\partial {\bf r}^\gamma_i}
    +\delta_{\alpha\beta\mu}
    \frac{\partial \hat{M}_\mu}{\partial {\bf r}^\gamma_i}
\end{align}
We now  use the  definition of the  gyration tensor in terms  of particle
positions to
express  the left  hand side  of (\ref{eq:126}),
\begin{align}
  \label{eq:286}
       \frac{\partial    \hat{\bf G}_{\alpha'\beta'}}{\partial {\bf r}^\gamma_i}
       &=    
         \sum_j m_j\frac{\partial    }{\partial {\bf r}^\gamma_i}({\bf r}_j^{\alpha'}-\hat{\bf R}^{\alpha'})({\bf r}_j^{\beta'}-\hat{\bf R}^{\beta'})
+            
         \sum_j m_j({\bf r}_j^{\alpha'}-\hat{\bf R}^{\alpha'})\frac{\partial    }{\partial {\bf r}^\gamma_i}({\bf r}_j^{\beta'}-\hat{\bf R}^{\beta'})
         \nonumber\\
            &=    
         \sum_j m_j\left(\delta_{ij}\delta_{\gamma\alpha'}-\frac{m_i}{M}\delta_{\gamma\alpha'}\right)({\bf r}_j^{\beta'}-\hat{\bf R}^{\beta'})
+            
              \sum_j m_j({\bf r}_j^{\alpha'}-\hat{\bf R}^{\alpha'})
              \left(\delta_{ij}\delta_{\gamma\beta'}-\frac{m_i}{M}\delta_{\gamma\beta'}\right)
         \nonumber\\
            &= m_i\delta_{\gamma\alpha'}({\bf r}_i^{\beta'}-\hat{\bf R}^{\beta'})+m_i({\bf r}_i^{\alpha'}-\hat{\bf R}^{\alpha'})\delta_{\gamma\beta'}
     \end{align}
Therefore
\begin{align}
  \label{eq:328}
    \left[e^{-[\hat{\boldsymbol{\Lambda}}]_\times}\right]_{{\alpha}\alpha'}
    \frac{\partial    \hat{\bf G}_{\alpha'\beta'}}{\partial {\bf r}^\gamma_i}
    &\left[e^{[\hat{\boldsymbol{\Lambda}}]_\times}\right]_{\beta'\beta}
      \nonumber\\
  &={\black  \frac{1}{4}}\left[e^{-[\hat{\boldsymbol{\Lambda}}]_\times}\right]_{{\alpha}\gamma}
    m_i     \left({\bf r}^{\beta'}_{i}-  \hat{\bf R}^{\beta'}\right)
    \left[e^{-[\hat{\boldsymbol{\Lambda}}]_\times}\right]_{\beta\beta'}
    +  \frac{1}{4}\left[e^{-[\hat{\boldsymbol{\Lambda}}]_\times}\right]_{{\alpha}\alpha'}
    m_i     \left({\bf r}^{\alpha'}_{i}-  \hat{\bf R}^{\alpha'}\right)
    \left[e^{-[\hat{\boldsymbol{\Lambda}}]_\times}\right]_{\beta\gamma}
    \nonumber\\
  &=\frac{1}{4}\left[e^{-[\hat{\boldsymbol{\Lambda}}]_\times}\right]_{{\alpha}\gamma}
    m_i    {\bf r}^{\beta}_{0i}
    +  \frac{1}{4}
    \left[e^{-[\hat{\boldsymbol{\Lambda}}]_\times}\right]_{\beta\gamma}    m_i    {\bf r}^{\alpha}_{0i}  
\end{align}
where the last line shows the expression in the principal axis frame.  Putting this in (\ref{eq:126}) gives
\begin{align}
  \label{eq:440}
      \epsilon_{\underline{\alpha}\gamma''\underline{\beta}}
\left ( \hat{M}_{\underline{\beta}} - \hat{M}_{\underline{\alpha}}\right )
    \left[e^{-[\hat{\boldsymbol{\Lambda}}]_\times}\right]_{\gamma''\gamma'}
      {B}^{-1}_{\gamma'\mu}  \frac{\partial \hat{\Lambda}_\mu}{\partial {\bf r}^\gamma_i}
    +
    \delta_{\alpha\beta\mu}
  \frac{\partial \hat{M}_\mu}{\partial {\bf r}^\gamma_i}
  &= \frac{1}{4}\left[e^{-[\hat{\boldsymbol{\Lambda}}]_\times}\right]_{{\alpha}\gamma}
    m_i    {\bf r}^{\beta}_{0i}
    +  \frac{1}{4}
  \left[e^{-[\hat{\boldsymbol{\Lambda}}]_\times}\right]_{\beta\gamma}    m_i    {\bf r}^{\alpha}_{0i}
\end{align}
This  is  the  original  system of  equations  (\ref{eq:251})  for  the
unknowns
$    \frac{\partial   \hat{\Lambda}_\mu}{\partial    {\bf
    r}^\gamma_i},               \frac{\partial               \hat{M}_{\alpha}}{\partial  {\bf  r}^\gamma_i}$ written in  the
principal axis system.  In this  reference frame, the system separates
nicely in  its diagonal  and off-diagonal elements,  and allows  us to
find the solution trivially.  When $\alpha=\beta$ we have
  \begin{align}
    \label{eq:425}
    \frac{\partial \hat{M}_\alpha}{\partial {\bf r}^\gamma_i}
    &={\black  \frac{1}{2}}
    \left[e^{-[\hat{\boldsymbol{\Lambda}}]_\times}\right]_{\underline{\alpha}\gamma}
    \left[e^{-[\hat{\boldsymbol{\Lambda}}]_\times}\right]_{\underline{\alpha}\beta'}
    m_i     \left({\bf r}^{\beta'}_{i}-  \hat{\bf R}^{\beta'}\right)
=\frac{1}{2}
\left[e^{-[\hat{\boldsymbol{\Lambda}}]_\times}\right]_{\underline{\alpha}\gamma}
m_i {\bf r}^{\underline{\alpha}}_{0i}
  \end{align}
This result can also be obtained from the diagonalization of the gyration tensor,  by noting $\mathbb{G}$ is
just $\bf G$ expressed in the principal axis system so that
%
\begin{align}
\label{eq:284}
\hat{\mathbb{G}}^{\alpha\beta}=\delta_{\alpha\beta\gamma}\hat{M}_\gamma=
\delta_{\underline{\alpha}\beta}\hat{M}_{\underline{\alpha}}
=\frac{1}{4}\sum_i m_i {\bf r}_{0i}^{\alpha}{\bf r}_{0i}^{\beta}
\end{align}
%
and $\frac{\partial \hat{M}_\alpha}{\partial {\bf r}^\gamma_i}=
\left[e^{-[\hat{\boldsymbol{\Lambda}}]_\times}\right]_{\alpha'\gamma}
\frac{\partial \hat{M}_\alpha}{\partial {\bf r}^{\alpha'}_{0i}}$.
When $\alpha\neq\beta$, the factor of
$\frac{\partial\hat{M}_\mu}{\partial{\bf  r}^\gamma_i}$
in (\ref{eq:440}) vanishes, yielding an equation involving only
$\frac{\partial\hat{\Lambda}_\mu}{\partial{\bf r}^\gamma_i}$ which
in principle can be used to isolate the latter. We do not pursue
this calculation, because the only place where it would be needed is
in the calculation of $\hat{\bf L}^{\mu\nu}$ in (\ref{eq:L}), but this
matrix is computed using a direct argument.


For future reference, we note that (\ref{eq:425}) implies
\begin{align}
  \label{eq:256}
  \sum_i     \frac{\partial \hat{M}_\alpha}{\partial {\bf r}^\mu_i}    ({\bf r}^\nu_i-\hat{\bf R}^\nu)
  &={\black  \frac{1}{2}}
\sum_i
    \left[e^{-[\hat{\boldsymbol{\Lambda}}]_\times}\right]_{\underline{\alpha}\mu}
    \left[e^{-[\hat{\boldsymbol{\Lambda}}]_\times}\right]_{\underline{\alpha}\beta'}
    m_i     \left({\bf r}^{\beta'}_{i}-  \hat{\bf R}^{\beta'}\right)    ({\bf r}^\nu_i-\hat{\bf R}^\nu)
=
    2\left[e^{-[\hat{\boldsymbol{\Lambda}}]_\times}\right]_{\underline{\alpha}\mu}
    \left[e^{-[\hat{\boldsymbol{\Lambda}}]_\times}\right]_{\underline{\alpha}\beta'}
    \hat{\bf G}_{\beta'\nu}
\end{align}



\subsection{The dilational force}
\label{App:Krp}
The dilation momentum can be determined from the 
derivative of (\ref{eq:284}) giving
%
\begin{align}
\label{eq:78}
\hat{{\Pi}}_\alpha & =\frac{d}{dt}\hat{M}_\alpha = \frac{1}{2}
\sum_i m_i {\bf r}_{0i}^{\underline{\alpha}}{\bf v}_{0i}^{\underline{\alpha}}
\end{align}
The dilation force is the time derivative of the
dilation momentum, so using (\ref{eq:78}) gives
%
\begin{align}
\label{eq:42}
\hat{\cal K}_\alpha& =\frac{d}{dt}\hat{  {\Pi}}_\alpha 
= \frac{1}{2} \sum_i m_i\left (
 {\bf v}_{0i}^{\underline{\alpha}}{\bf v}_{0i}^{\underline{\alpha}} +
{\bf r}_{0i}^{\underline{\alpha}}{\bf a}_{0i}^{\underline{\alpha}} \right )
\end{align}
where  ${\bf a}_{0i}$  is  the  acceleration of  particle  $i$ in  the
principal axis  frame, and is  calculated from the time  derivative of
${\bf  v}_{0i}$  in (\ref{rvmic})  as  (remembering  $\hat{\bf V}$  is
constant)
%
\begin{align}
\label{eq:42a}
{\bf a}_{0i}&=\frac{d}{dt}{\bf v}_{0i}\overset{(\ref{eq:601})}{=}
\frac{d}{dt}\left ( e^{-[\hat{\boldsymbol{\Lambda}}]_\times}\esc
\left( {\bf v}_i-\hat{\bf V}\right )- [\hat{\boldsymbol{\omega}}_0]_\times
\esc {\bf r}_{0i}\right )
\nonumber\\
&= \left ( \frac{d}{dt} e^{-[\hat{\boldsymbol{\Lambda}}]_\times}\right )\esc
\left( {\bf v}_i-\hat{\bf V}\right ) +
 e^{-[\hat{\boldsymbol{\Lambda}}]_\times}\esc\frac{d{\bf v}_i}{dt} -
\left [\frac{d\hat{\boldsymbol{\omega}}_0}{dt}\right ]_\times\esc{\bf r}_{0i}
- [\hat{\boldsymbol{\omega}}_0]_\times\esc{\bf v}_{0i}
\nonumber\\
&\overset{(\ref{rvmic})(\ref{eq:601})}{=}
-[\hat{\boldsymbol{\omega}}_0]_\times\esc{\bf v}_{pi} + \frac{1}{m_i}{\bf F}_{0i} -
\left [\frac{d\hat{\boldsymbol{\omega}}_0}{dt}\right ]_\times\esc{\bf r}_{0i}
- [\hat{\boldsymbol{\omega}}_0]_\times\esc\left ( {\bf v}_{pi} -
[\hat{\boldsymbol{\omega}}_0]_\times\esc{\bf r}_{0i}\right )
\nonumber\\
&=\frac{1}{m_i}{\bf F}_{0i} - 2 [\hat{\boldsymbol{\omega}}_0]_\times\esc{\bf v}_{pi} +
 [\hat{\boldsymbol{\omega}}_0]_\times\esc[\hat{\boldsymbol{\omega}}_0]_\times
\esc{\bf r}_{0i} -
\left [\frac{d\hat{\boldsymbol{\omega}}_0}{dt}\right ]_\times\esc{\bf r}_{0i}
\nonumber\\
&\overset{(\ref{rvmic})(\ref{eq:601})}{=}
\frac{1}{m_i}{\bf F}_{0i} - 2 [\hat{\boldsymbol{\omega}}_0]_\times\esc{\bf v}_{0i} -
 [\hat{\boldsymbol{\omega}}_0]_\times\esc[\hat{\boldsymbol{\omega}}_0]_\times
\esc{\bf r}_{0i} -
\left [\frac{d\hat{\boldsymbol{\omega}}_0}{dt}\right ]_\times\esc{\bf r}_{0i}
\end{align}
%
where from (\ref{rvmic}) we have
${\bf v}_0 = {\bf v}_p-[\hat{\boldsymbol{\omega}}_0]_\times\esc{\bf r}_0$ and
%
\begin{align}
\label{eq:42d}
{\bf F}_{0i} = e^{-[\hat{\boldsymbol{\Lambda}}]_\times}\esc{\bf F}_i=
e^{-[\hat{\boldsymbol{\Lambda}}]_\times}\esc\frac{d{\bf p}_i}{dt}
=-e^{-[\hat{\boldsymbol{\Lambda}}]_\times}\esc\frac{\partial \Phi(r)}
{\partial{\bf r}_i}
\end{align}
%
is the force on particle $i$,  due to the other particles, transformed
to the principal axis frame.  From
(\ref{eq:42}),   two   main   effects   drive  the   change   in   the
dilational moments.   The last term  shows forces in  the radial
direction can increase or decrease  the moments, and these forces (see
(\ref{eq:42a}))  arise  from  intermolecular  interactions  among  the
particles, Coriolis and centrifugal effects, and angular acceleration.
However, even  in the absence  of interactions and for  a non-rotating
frame, the first term of (\ref{eq:42}) shows moments will increase due
to the velocities  of particles.  This simply  represents the tendency
of a group of moving, non-interacting particles to disperse over time,
thereby leading to increasing  dilational moments.  Thus, a body
with a stable shape is only possible if ${\bf F}_{0i}$ is large enough
to compensate all the terms causing the dilational moments to increase.  Finally,
observe that
%
\begin{align}
\sum_i m_i {\bf r}^{\underline{\alpha}}_{0i}
\left[\frac{d\hat{\boldsymbol{\omega}}_0}{dt}\right ]_\times^{\underline{\alpha}\alpha'}
{\bf r}^{\alpha'}_{0i}\overset{(\ref{eq:284})}{=}
\left[\frac{d\hat{\boldsymbol{\omega}}_0}{dt}\right ]_\times^{\underline{\alpha}\alpha'}
\hat{\mathbb{G}}^{\underline{\alpha}\alpha'}=
\left[\frac{d\hat{\boldsymbol{\omega}}_0}{dt}\right ]_\times^{\underline{\alpha}\underline{\alpha}}
\hat{M}_{\underline{\alpha}}=0
\end{align}
since  the diagonal  elements of  the cross  product matrix  are zero.
This shows the angular acceleration  in (\ref{eq:42a}) does not affect
the value of $\hat{\cal K}_\alpha$.



\subsection{Some matrices involving first derivatives with respect particle's  positions}
\label{App:TH}
In  this section  we compute  explicitly the  functional form  of the
following matrices  that involve first derivatives  of orientation and
central moments with respect to particle positions

\begin{align}
  \label{eq:L}  \hat{\bf L}^{\mu\nu}
  &\equiv
    {B}^{-1}_{\mu\mu'} 
    {B}^{-1}_{\nu\nu'}  \sum_i\frac{1}{m_i}
    \frac{\partial \hat{\Lambda}_{\mu'}}{\partial {\bf r}^\gamma_i}
    \frac{\partial \hat{\Lambda}_{\nu'}}{\partial {\bf r}^\gamma_i}
\\\label{eq:H}  \hat{\bf H}^{\mu\nu}
  &\equiv\sum_i\frac{1}{m_i}
  \frac{\partial \hat{\Lambda}_\mu}{\partial {\bf r}_i^\gamma}
  \frac{\partial \hat{M}_\nu}{\partial {\bf r}_i^\gamma}
  \\
  \label{eq:T}
    \hat{\bf T}^{\mu\nu}
  &\equiv\sum_i\frac{1}{m_i}\frac{\partial \hat{M}_\mu}{\partial {\bf r}_i^\gamma}
    \frac{\partial \hat{M}_\nu}{\partial {\bf r}_i^\gamma}
\end{align}
To get the explicit form, consider (\ref{eq:440}) with the following two suitable renaming of indices
  \begin{align}
    \label{eq:393}
      \epsilon_{\underline{\alpha}\mu''\underline{\beta}}
\left ( \hat{M}_{\underline{\beta}} - \hat{M}_{\underline{\alpha}}\right )
    \left[e^{-[\hat{\boldsymbol{\Lambda}}]_\times}\right]_{\mu''\mu'}
      {B}^{-1}_{\mu'\mu}  \frac{\partial \hat{\Lambda}_\mu}{\partial {\bf r}^\gamma_i}
    +
\delta_{\alpha\beta\mu}
    \frac{\partial \hat{M}_\mu}{\partial {\bf r}^\gamma_i}
  &=   \frac{1}{4}\left[e^{-[\hat{\boldsymbol{\Lambda}}]_\times}\right]_{{\alpha}\gamma}
    m_i    {\bf r}^{\beta}_{0i}
    +  \frac{1}{4}
  \left[e^{-[\hat{\boldsymbol{\Lambda}}]_\times}\right]_{\beta\gamma}    m_i    {\bf r}^{\alpha}_{0i}
    \nonumber\\
      \epsilon_{\underline{\alpha'}\nu''\underline{\beta'}}
\left ( \hat{M}_{\underline{\beta'}} - \hat{M}_{\underline{\alpha'}}\right )
    \left[e^{-[\hat{\boldsymbol{\Lambda}}]_\times}\right]_{\nu''\nu'}
      {B}^{-1}_{\nu'\nu}  \frac{\partial \hat{\Lambda}_\nu}{\partial {\bf r}^\gamma_i}
    +
\delta_{\alpha'\beta'\nu}
    \frac{\partial \hat{M}_\nu}{\partial {\bf r}^\gamma_i}
  &=     \frac{1}{4}\left[e^{-[\hat{\boldsymbol{\Lambda}}]_\times}\right]_{\alpha'\gamma}
    m_i    {\bf r}^{\beta'}_{0i}
    +  \frac{1}{4}
  \left[e^{-[\hat{\boldsymbol{\Lambda}}]_\times}\right]_{\beta'\gamma}    m_i    {\bf r}^{\alpha'}_{0i}
  \end{align}
  Multiply together  each side  of each  equation, sum  over $\gamma$,
  divide by $m_i$, and sum over all particles $i$ to get
  \begin{align}
    \label{eq:395}
&\sum_i\frac{1}{m_i}
      \left[
      \epsilon_{\underline{\alpha}\mu''\underline{\beta}}
\left ( \hat{M}_{\underline{\beta}} - \hat{M}_{\underline{\alpha}}\right )
    \left[e^{-[\hat{\boldsymbol{\Lambda}}]_\times}\right]_{\mu''\mu'}
      {B}^{-1}_{\mu'\mu}  \frac{\partial \hat{\Lambda}_\mu}{\partial {\bf r}^\gamma_i}
    +
\delta_{\alpha\beta\mu}
    \frac{\partial \hat{M}_\mu}{\partial {\bf r}^\gamma_i}\right]
      \nonumber\\
  &\times\left[
      \epsilon_{\underline{\alpha'}\nu''\underline{\beta'}}
\left ( \hat{M}_{\underline{\beta'}} - \hat{M}_{\underline{\alpha'}}\right )
    \left[e^{-[\hat{\boldsymbol{\Lambda}}]_\times}\right]_{\nu''\nu'}
      {B}^{-1}_{\nu'\nu}  \frac{\partial \hat{\Lambda}_\nu}{\partial {\bf r}^\gamma_i}
    +
    \delta_{\alpha'\beta'\nu}
    \frac{\partial \hat{M}_\nu}{\partial {\bf r}^\gamma_i}\right]&           \nonumber\\
    &=\sum_i \frac{1}{m_i}   \left[   \frac{1}{4}\left[e^{-[\hat{\boldsymbol{\Lambda}}]_\times}\right]_{{\alpha}\gamma}
    m_i    {\bf r}^{\beta}_{0i}
    +  \frac{1}{4}
  \left[e^{-[\hat{\boldsymbol{\Lambda}}]_\times}\right]_{\beta\gamma}    m_i    {\bf r}^{\alpha}_{0i}
\right]\left[     \frac{1}{4}\left[e^{-[\hat{\boldsymbol{\Lambda}}]_\times}\right]_{\alpha'\gamma}
    m_i    {\bf r}^{\beta'}_{0i}
    +  \frac{1}{4}
  \left[e^{-[\hat{\boldsymbol{\Lambda}}]_\times}\right]_{\beta'\gamma}    m_i    {\bf r}^{\alpha'}_{0i}
\right]          
  \end{align}
Now use (\ref{eq:L}), (\ref{eq:H}), (\ref{eq:T}) and (\ref{eq:284}) to
simplify this as
%
\begin{align}
  \label{eq:398}
&\epsilon_{\underline{\alpha}\mu''\underline{\beta}}
\left ( \hat{M}_{\underline{\beta}} - \hat{M}_{\underline{\alpha}}
\right )
\epsilon_{\underline{\alpha'}\nu''\underline{\beta'}}
\left ( \hat{M}_{\underline{\beta'}} - \hat{M}_{\underline{\alpha'}}
\right )
    \left[e^{-[\hat{\boldsymbol{\Lambda}}]_\times}\right]_{\mu''\mu'}
    \left[e^{-[\hat{\boldsymbol{\Lambda}}]_\times}\right]_{\nu''\nu'}
    \hat{\bf L}^{\mu'\nu'}
    \nonumber\\
  &+ \epsilon_{\underline{\alpha}\mu''\underline{\beta}}
\left ( \hat{M}_{\underline{\beta}} - \hat{M}_{\underline{\alpha}}\right )
    \left[e^{-[\hat{\boldsymbol{\Lambda}}]_\times}\right]_{\mu''\mu'}
      {B}^{-1}_{\mu'\mu} \delta_{\alpha'\beta'\nu} \hat{\bf H}^{\mu\nu}
      \nonumber\\
  &+ \epsilon_{\underline{\alpha'}\nu''\underline{\beta'}}
\left ( \hat{M}_{\underline{\beta'}} - \hat{M}_{\underline{\alpha'}}\right )
    \left[e^{-[\hat{\boldsymbol{\Lambda}}]_\times}\right]_{\nu''\nu'}
 {B}^{-1}_{\nu'\nu} \delta_{\alpha\beta\mu} \hat{\bf H}^{\nu\mu}
+
    \delta_{\alpha\beta\mu}
    \delta_{\alpha'\beta'\nu}\hat{\bf T}^{\mu\nu}
                   \nonumber\\
  &=   \frac{1}{4}  \left(
\delta_{\alpha\alpha'}\delta_{\underline{\beta}\beta'}
\hat{M}_{\underline{\beta}}
+  2 \delta_{\alpha\beta'}\delta_{\alpha'\underline{\beta}}
\hat{M}_{\underline{\beta}}
+  \delta_{\beta\beta'}\delta_{\underline{\alpha}\alpha'} 
\hat{M}_{\underline{\alpha}}
  \right)
\end{align}
Quite fortunately, this apparently complicated system of equations  is,  in
fact, trivial, as can be seen by looking at particular components.  If
$\alpha=\beta$ and $\alpha'=\beta'$ then we have
\begin{align}
  \label{eq:399}
\delta_{\alpha\alpha\mu} \delta_{\alpha'\alpha'\nu} \hat{\bf T}^{\mu\nu}
  &=\hat{\bf T}^{\alpha\alpha'}=
\frac{1}{4}\left (
\delta_{\underline{\alpha}\alpha'}\hat{M}_{\underline{\alpha}}
+  2 \delta_{\underline{\alpha}\alpha'}\hat{M}_{\underline{\alpha}}
+  \delta_{\underline{\alpha}\alpha'}\hat{M}_{\underline{\alpha}}\right )
= \delta_{\underline{\alpha}\alpha'}\hat{M}_{\underline{\alpha}}
\end{align}
so that
\begin{align}
  \label{TG}
  \hat{\bf T}&=
               \left(
               \begin{array}{ccc}
                 \hat{M}_1
                 & 0
                 & 0 \\
                 0
                 &  \hat{M}_2
                 & 0  \\
                 0 & 
                   0 &  \hat{M}_3 \\
               \end{array}
  \right)=\hat{\mathbb{G}}
  % 2\left(
  %   \begin{array}{ccc}
  %     2 {M}_1
  %     &    (M_1+M_2-M_3)
  %     &(M_1-M_2+M_3) \\
  %     (M_1+M_2-M_3)
  %     & 2 M_2
  %     &  (-M_1+M_2+M_3) \\
  %     (M_1-M_2+M_3) & 
  %     (-M_1+M_2+M_3) & 2 M_3 \\
  %   \end{array}
  % \right)
\end{align}
Setting $\alpha=\beta$ and $\alpha'\neq\beta'$ in (\ref{eq:398}) gives
\begin{align}
  \label{eq:227}
\epsilon_{\underline{\alpha'}\nu''\underline{\beta'}}
\left ( \hat{M}_{\underline{\beta'}} - \hat{M}_{\underline{\alpha'}}\right )
    \left[e^{-[\hat{\boldsymbol{\Lambda}}]_\times}\right]_{\nu''\nu'}
 {B}^{-1}_{\nu'\nu} \hat{\bf H}^{\nu\alpha}=0
\end{align}
which, provided the moments are different, implies  $\hat{\bf H}$
vanishes identically, that is
\begin{align}
  \label{eq:449}
    \hat{\bf H}^{\mu\nu}&=0
\end{align}

Finally, 
by suitably introducing
\begin{align}
  \label{eq:225}
  \hat{\mathbb{L}}^{\mu\nu}
  &\equiv
    \left[e^{-[\hat{\boldsymbol{\Lambda}}]_\times}\right]_{\mu\mu'}
    \left[e^{-[\hat{\boldsymbol{\Lambda}}]_\times}\right]_{\nu\nu'}
    \hat{\bf L}^{\mu'\nu'}
\end{align}
then (\ref{eq:398}) becomes for $\alpha\neq\beta$ and $\alpha'\neq\beta'$,  
\begin{align}
\label{eq:400}
\epsilon_{\underline{\alpha}\mu\underline{\beta}}
\left ( \hat{M}_{\underline{\beta}} - \hat{M}_{\underline{\alpha}}
\right )
\epsilon_{\underline{\alpha'}\nu\underline{\beta'}}
\left ( \hat{M}_{\underline{\beta'}} - \hat{M}_{\underline{\alpha'}}
\right )
    \hat{\mathbb{L}}^{\mu\nu}
  =   \frac{1}{4}  \left(
\delta_{\alpha\alpha'}\delta_{\underline{\beta}\beta'}
\hat{M}_{\underline{\beta}}
+  2 \delta_{\alpha\beta'}\delta_{\alpha'\underline{\beta}}
\hat{M}_{\underline{\beta}}
+  \delta_{\beta\beta'}\delta_{\underline{\alpha}\alpha'} 
\hat{M}_{\underline{\alpha}}
  \right)
\end{align}
%
Provided the moments are different, (\ref{eq:400}) can be solved by taking
different combinations of components and using (\ref{eq:604}) to simplify the
product of Levi-Civita symbols.  For example, setting
$\alpha\neq\alpha'\neq\beta\neq\beta'$ shows that both sides of (\ref{eq:400})
vanish, giving no useful information about $\hat{\mathbb{L}}$.  Setting
$\alpha=\alpha'$, $\beta\neq\beta'$, $\alpha\neq\beta$ and $\alpha'\neq\beta'$
reduces (\ref{eq:400}) to
%
\begin{align}
-\delta_{\mu\underline{\beta'}}\delta_{\nu\underline{\beta}}
\left ( \hat{M}_{\underline{\beta}} - \hat{M}_{\underline{\alpha}}
\right )
\left ( \hat{M}_{\underline{\beta'}} - \hat{M}_{\underline{\alpha'}}
\right )
    \hat{\mathbb{L}}^{\mu\nu} = 0
\end{align}
%
giving $\hat{\mathbb{L}}^{\beta'\beta}=0$ when $\beta\neq\beta'$. The
matrix $\hat{\mathbb{L}}$  is a diagonal matrix, hence the voided font used to denote it. Finally,
setting $\alpha=\alpha'$, $\beta=\beta'$, $\alpha\neq\beta$ and
$\alpha'\neq\beta'$ reduces (\ref{eq:400}) to
%
\begin{align}
\left(\delta_{\mu\nu}-
\delta_{\mu\underline{\beta}}\delta_{\nu\underline{\beta}}-
\delta_{\mu\underline{\alpha}}\delta_{\nu\underline{\alpha}}\right )
\left ( \hat{M}_{\underline{\beta}} - \hat{M}_{\underline{\alpha}}
\right )^2 \hat{\mathbb{L}}^{\mu\nu} = \frac{1}{4}\left (\hat{M}_\beta+
\hat{M}_\alpha\right )
\end{align}
%
giving the diagonal element of $\hat{\mathbb{L}}$ as
%
\begin{align}
\hat{\mathbb{L}}^{\mu\mu}
-\hat{\mathbb{L}}^{\underline{\alpha}\underline{\alpha}}
-\hat{\mathbb{L}}^{\underline{\beta}\underline{\beta}}
= \frac{\hat{M}_{\underline{\alpha}}+\hat{M}_{\underline{\beta}}}
{4\left ( \hat{M}_{\underline{\beta}} - \hat{M}_{\underline{\alpha}}
\right )^2}
\end{align}
%
where summation over $\mu$ is implied but not over $\alpha$ and $\beta$.
Finally, we have
\begin{align}
  \label{eq:229}
  \hat{\mathbb{L}}=
               \left(
               \begin{array}{ccc}
                \frac{\hat{M}_2+\hat{M}_3}{4(\hat{M}_2-\hat{M}_3)^2}
                 & 0
                 & 0 \\
                 0
                 &  \frac{\hat{M}_1+\hat{M}_3}{4(\hat{M}_1-\hat{M}_3)^2}
                 & 0  \\
                 0 & 
                   0 & \frac{\hat{M}_1+\hat{M}_2}{4(\hat{M}_1-\hat{M}_2)^2}  \\
               \end{array}
  \right)
\end{align}

\setcounter{my_equation_counter}{\value{equation}}
\section{Momentum integrals}
\setcounter{equation}{\value{my_equation_counter}}
\label{App:Submanifold}
\subsection{The general formula}
Here consider  the evaluation of the  following ``sectioned Gaussian''
integral
\begin{align}
  \label{eq:134}
  G(x,J)&=\int dy \delta(\Pi y-x) \exp\left\{-\frac{1}{2}y^T Ay+J^Ty\right\}
\end{align}
where              $y\in\mathbb{R}^n$,             $J\in\mathbb{R}^n$,
$A\in\mathbb{R}^{n\times    n}$   is    symmetric   and    invertible,
$x\in\mathbb{R}^m$,   and   $\Pi\in\mathbb{R}^{m\times    n}$   is   a
rectangular matrix.   The integral  is restricted  by the  Dirac delta
function to the hyperplane defined by $\Pi y=x$. By taking derivatives
with  respect   to  $J$   we  will  generate   moments  of   the  form
(\ref{eq:18}).  Note that  in this section, $\Pi$ is  a parameter and
$G$ a function that should not  be confused with the dilation momentum
and gyration tensor which appear with the same symbols elsewhere.

Begin by introducing the Fourier transform of the
Dirac delta function  in (\ref{eq:134})
\begin{align}
  \label{Jeq:116}
G(x,J)
  &=
    \int \frac{d\lambda}{(2\pi)^m} \exp\left\{-i\lambda^T x\right\}
    \int dy \exp\left\{-\frac{1}{2}y^T Ay +J^Ty\right\} 
      \exp\left\{i\lambda^T \Pi y\right\}
      \nonumber\\
&=\int \frac{d\lambda}{(2\pi)^m} \exp\left\{-i\lambda^T x\right\} \tilde{G}(y,J)
\end{align}
where 
\begin{align}
% \label{eq:238}
  \label{Jeq:115}
    \tilde{G}(\lambda,J)
  &=  \int dy \exp\left\{-\frac{1}{2}y^T Ay+\left(J+i\Pi^T \lambda\right)^Ty\right\}
\nonumber\\
&
=\int dy \exp\left\{ -\frac{1}{2}
\left ( y-A^{-1}(J+i\Pi^T \lambda)\right)^T A
\left ( y-A^{-1}(J+i\Pi^T \lambda)\right)
+\frac{1}{2} \left ( J+i\Pi^T \lambda\right)^T A^{-1}
\left ( J+i\Pi^T \lambda\right)
\right\}
\nonumber\\
&
=\frac{(2\pi)^{n/2}}{\det A^{1/2}}\exp\left\{
\frac{1}{2}\left(J+ i\Pi^T\lambda\right)^TA^{-1}\left(J+i\Pi^T\lambda\right)
\right\}
\end{align}
noting that $(J+ i\Pi^T\lambda)^Ty=y^T(J+ i\Pi^T\lambda)$ and where the
last line was obtained using
 the Gaussian integral
\begin{align}
\int dy' \exp\left\{-\frac{1}{2}y'^T Ay'\right\}
&=\frac{(2\pi)^{n/2}}{\det A^{1/2}}
\label{Gaussian}
\end{align}
Inserting (\ref{Jeq:115})  in (\ref{Jeq:116}) one gets
\begin{align}
  G(x,J)&=\frac{(2\pi)^{n/2}}{\det A^{1/2}}e^{\frac{1}{2}J^TA^{-1}J}
        \int \frac{d\lambda}{(2\pi)^m} \exp\left\{-i\lambda^T (x-\Pi A^{-1}J) 
-\frac{1}{2}\lambda^T\Pi  A^{-1}\Pi^T\lambda\right\}
        \nonumber\\
&=\frac{(2\pi)^{n/2}}{\det A^{1/2}}e^{\frac{1}{2}J^TA^{-1}J}
\int \frac{d\lambda}{(2\pi)^m} \exp \left\{ -\frac{1}{2}
\left (\lambda + i\overline{A}(x-\Pi A^{-1}J)\right )^T \overline{A}^{-1} 
\left (\lambda + i\overline{A}(x-\Pi A^{-1}J)\right )\right\}
\nonumber\\
&\qquad\qquad\qquad\qquad\times\exp\left\{
-\frac{1}{2}(x-\Pi A^{-1}J)^T \overline{A} (x-\Pi A^{-1}J)
\right\}
        \nonumber\\
  &=\frac{(2\pi)^{n/2}}{\det A^{1/2}}
\frac{\det{\overline{A}}^{1/2}}{(2\pi)^{m/2}}
\exp\left\{-\frac{1}{2}(x-\Pi A^{-1}J)^T \overline{A} (x-\Pi A^{-1}J)+\frac{1}{2}J^TA^{-1}J\right\}
\label{apomJ}
\end{align}
where  $\overline{A}\in\mathbb{R}^{m\times m}$ is symmetric and assumed
invertible, and is given by
\begin{align}
  \overline{A}&\equiv \left[\Pi A^{-1}\Pi^T\right]^{-1}
\label{overA}
\end{align}
The calculation of the Gaussian integrals suggests
the change of variables
\begin{align}
  \label{eq:100}
  y'=y-A^{-1} \Pi^T [\Pi  A^{-1} \Pi^T]^{-1} x
\end{align}
that has the property
\begin{align}
  \label{eq:234}
  \Pi y'=\Pi y- x
\end{align}
and, therefore, 
\begin{align}
  \label{eq:240}
  \delta(\Pi y- x)=\delta(\Pi y')
\end{align}
This change of variables will turn to be instrumental in the calculation of the entropy.
% Under this change of variable, the integral
% (\ref{eq:134}) takes the form
% \begin{align}
%   \label{eq:246}
%   G(x,J)
%   &=\int dy' \delta(\Pi y')
%     \exp\left\{-\frac{1}{2}(y'+\overline{y}(x))^T A(y'+\overline{y}(x))+J^T\left(y'+\overline{y}(x)\right)\right\}
%           \nonumber\\
%   &=\int dy' \delta(\Pi y') \exp\left\{-\frac{1}{2}(y'+\overline{y}(x))^T A(y'+\overline{y}(x))+J^T\left(y'+\overline{y}(x)\right)\right\}
% \end{align}
\subsection{The momentum integrals in conditional expectations}
The results  of this  subsection are  used in  the calculation  of the
equilibrium  probability $P^{\rm  rest}_{\cal E}({\bf M},\boldsymbol{\Pi})$
in  (\ref{eq:64b})   and  the  equilibrium
approximation to the angular  diffusion tensor $\boldsymbol{\cal D}_0$.
In  this  subsection we  compute  the  following integrals over momenta
\begin{align}
  G^{(0)}({\bf P},{\bf S},\boldsymbol{\Pi},\beta)
  &=\int dp \Delta(p,{\bf P},{\bf S},\boldsymbol{\Pi},\beta)
    \label{G0}
    \\
{\bf  G}^{(1)}_i({\bf P},{\bf S},\boldsymbol{\Pi},\beta)
  &=\int dp \frac{{\bf p}_{i}}{m_{i}}  \Delta(p,{\bf P},{\bf S},\boldsymbol{\Pi},\beta)
\label{G1}  \\  
{\bf  G}^{(2)}_{ij}({\bf P},{\bf S},\boldsymbol{\Pi},\beta)
  &  =\int dp \frac{{\bf p}_{i}}{m_{i}}\frac{{\bf p}^T_{j}}{m_{j}}
    \Delta(p,{\bf P},{\bf S},\boldsymbol{\Pi},\beta)
\label{G2}\end{align}
where 
\begin{align}
  \label{eq:45}
  \Delta(p,{\bf P},{\bf S},\boldsymbol{\Pi},\beta)
  &\equiv
  \delta\left(\sum_{i}{\bf p}_{i}-{\bf P}\right)
  \delta\left(\sum_{i}[{\bf r}'_{i}]_\times\esc{\bf p}_{i}-{\bf S} \right)
  \delta\left(\sum_{i}\frac{\partial \hat{{\bf M}}}{\partial m_{i}{\bf r}_{i}}\esc{\bf p}_{i}-\boldsymbol{\Pi}\right)
e^{-\beta\sum_i^N\frac{{\bf p}_i^2}{2m_i}+\sum_i{\bf J}_i\esc\frac{{\bf p}_i}{m_i}}
\end{align}
and ${\bf  r}'_{i}={\bf r}_{i}-\hat{\bf R}$.
Observe that
\begin{align}
  \label{eq:18}
  {\bf  G}^{(1)}_i&=\frac{\partial G^{(0)}}{\partial {\bf J}_i}
                    \nonumber\\
    {\bf  G}^{(2)}_i&=\frac{\partial^2 G^{(0)}}{\partial {\bf J}_i\partial {\bf J}_j}
\end{align}
so $G^{(0)}$ acts as a generating function.

\subsection{Calculation of ${G}^0({\bf P},{\bf S},\boldsymbol{\Pi},\beta)$}
Let us translate now the general result (\ref{apomJ}) into momentum
variables by identifying from (\ref{eq:45}) the following
\begin{align}
  n&=3N
\nonumber\\
m&=9
\nonumber\\
x&\to {\bf P},{\bf S},\boldsymbol{\Pi}
\nonumber\\
y&\to {\bf p}_{1},\cdots {\bf p}_{N}
\nonumber\\
J&\to \frac{{\bf J}_{1}}{m_1},\cdots \frac{{\bf J}_{N}}{m_N}
\end{align}
The matrix  $A$ has the diagonal form
\begin{align}
A&\to \beta\left(\begin{array}{ccc}
\frac{1}{m_1}\mathbb{1}_3&\cdots&0\\
\vdots&\ddots&\vdots\\
0&\cdots&\frac{1}{m_N}\mathbb{1}_3
\end{array}\right)  
\end{align}
where $\mathbb{1}_3$ is the  $3\times3$ identity matrix. The inverse and determinant of $A$ are
\begin{align}
A^{-1}&\to \frac{1}{\beta}\left(\begin{array}{ccc}
m_1\mathbb{1}_3&\cdots&0\\
\vdots&\ddots&\vdots\\
0&\cdots&{m_N}\mathbb{1}_3
\end{array}\right)  
,\qquad \det A=\prod_{i}^{3N}\frac{\beta}{m_{i}}
\end{align}
The product
\begin{align}
  \label{eq:90}
  A^{-1}J \to \frac{1}{\beta}\left(\begin{array}{ccc}
m_1\mathbb{1}_3&\cdots&0\\
\vdots&\ddots&\vdots\\
0&\cdots&{m_N}\mathbb{1}_3
\end{array}\right)  \left(\begin{array}{c}
\frac{{\bf J}_1}{m_1}\\
\vdots\\
\frac{{\bf J}_N}{m_N}\\
\end{array}\right) =\frac{1}{\beta} \left(\begin{array}{c}
{\bf J}_1\\
\vdots\\
{\bf J}_N\\
\end{array}\right) 
\end{align}
so that
\begin{align}
  \label{eq:109b}
J^T    A^{-1}J \to \frac{1}{\beta}\sum_i\frac{1}{m_i}{\bf J}^2_i
\end{align}
The rectangular matrix $\Pi$ is best appreciated from its action,
\begin{align}
x=\Pi y &\to \left(
  \begin{array}{c}
{\bf P}\\
    {\bf S} \\
    \boldsymbol{\Pi}
  \end{array}
\right)=
\left(\begin{array}{cccc}
 \mathbb{1}_3  & \mathbb{1}_3& \cdots &\mathbb{1}_3\\
        {[}{\bf r}'_{1}{]}_\times&{[}{\bf r}'_{2}{]}_\times&\cdots &{[}{\bf r}'_{N}{]}_\times\\
        \frac{\partial \hat{{\bf M}}}{\partial m_{1}{\bf r}_{1}}
            &\frac{\partial \hat{{\bf M}}}{\partial m_{2}{\bf r}_{2}}&\cdots
                          &\frac{\partial \hat{{\bf M}}}{\partial m_{N}{\bf r}_{N}}
      \end{array}\right)
\left(
  \begin{array}{c}
{\bf p}_{1}\\
\vdots\\
{\bf p}_{N}
  \end{array}
\right)
\end{align}
Therefore, $\Pi$
depends only on particle positions so is constant  as  far  as  the momentum
variables are concerned. The product
\begin{align}
  \label{eq:50b}
  \Pi A^{-1}J &\to \frac{1}{\beta}
\left(\begin{array}{cccc}
 \mathbb{1}_3  & \mathbb{1}_3& \cdots &\mathbb{1}_3\\
        {[}{\bf r}'_{1}{]}_\times&{[}{\bf r}'_{2}{]}_\times&\cdots &{[}{\bf r}'_{N}{]}_\times\\
        \frac{\partial \hat{{\bf M}}}{\partial m_{1}{\bf r}_{1}}
            &\frac{\partial \hat{{\bf M}}}{\partial m_{2}{\bf r}_{2}}&\cdots
                          &\frac{\partial \hat{{\bf M}}}{\partial m_{N}{\bf r}_{N}}
      \end{array}\right)
\left(
  \begin{array}{c}
{\bf J}_{1}\\
\vdots\\
{\bf J}_{N}
  \end{array}
\right)
= \frac{1}{\beta}\left( \begin{array}{c}
\sum_i{\bf J}_i\\
\sum_i{[}{\bf r}'_{i}{]}_\times\esc{\bf J}_i \\
\sum_i\frac{\partial \hat{{\bf M}}}{\partial m_{i}{\bf r}_{i}}\esc{\bf J}_i
  \end{array}
\right)
\end{align}

The $9\times9$ matrix $ \Pi  A^{-1}\Pi^T$ in (\ref{overA}) is obtained
by block-wise matrix multiplication
\begin{align}
  \Pi A^{-1}\Pi^T&\to  \frac{1}{\beta}                 \left(\begin{array}{cccc}
                           \mathbb{1}_3& \mathbb{1}_3& \cdots &\mathbb{1}_3\\
                           {[}{\bf r}'_{1}{]}_\times&{[}{\bf r}'_{2}{]}_\times&\cdots &{[}{\bf r}'_{N}{]}_\times\\
                           \frac{\partial \hat{{\bf M}}}{\partial m_{1}{\bf r}_{1}}
                                      &\frac{\partial \hat{{\bf M}}}{\partial m_{2}{\bf r}_{2}}&\cdots
                          &\frac{\partial \hat{{\bf M}}}{\partial m_{N}{\bf r}_{N}}
\end{array}\right)\left(\begin{array}{ccc}
m_1\mathbb{1}_3&\cdots&0\\
\vdots&\ddots&\vdots\\
0&\cdots&{m_N}\mathbb{1}_3
\end{array}\right)  
\left(\begin{array}{ccc}
        \mathbb{1}_3  & {[}{\bf r}'_{1}{]}^T_\times&        \left(\frac{\partial \hat{{\bf M}}}{\partial m_{1}{\bf r}_{1}}\right)^T
        \\
\vdots & \vdots& \vdots\\
        \mathbb{1}_3&{[}{\bf r}'_{N}{]^T}_\times&
                \left( \frac{\partial \hat{{\bf M}}}{\partial m_{N}{\bf r}_{N}}\right)^T
\end{array}\right)
\nonumber\\
&=\frac{1}{\beta}\left(\begin{array}{cccc}
 \mathbb{1}_3  & \mathbb{1}_3& \cdots &\mathbb{1}_3\\
{[}{\bf r}'_{1}{]}_\times&{[}{\bf r}'_{2}{]}_\times&\cdots &{[}{\bf r}'_{N}{]}_\times
\\
                                                       \frac{\partial \hat{{\bf M}}}{\partial m_{1}{\bf r}_{1}}
            &\frac{\partial \hat{{\bf M}}'}{\partial m_{2}{\bf r}_{2}}&\cdots
                          &\frac{\partial \hat{{\bf M}}}{\partial m_{N}{\bf r}_{N}}\end{array}\right)
\left(\begin{array}{ccc}
m_1 \mathbb{1}_3  & m_1{[}{\bf r}'_{1}{]}_\times &        \frac{\partial \hat{{\bf M}}}{\partial{\bf r}_{1}}\\
\vdots & \vdots& \vdots\\
m_N\mathbb{1}_3&m_N{[}{\bf r}'_{N}{]}_\times &        \frac{\partial \hat{{\bf M}}}{\partial{\bf r}_{N}}
\end{array}\right)
=\frac{1}{\beta}\left(\begin{array}{ccc}
M \mathbb{1}_3  & 0& 0\\
          0&\hat{\bf I}& 0\\
         0& 0& \hat{\bf T}\end{array}\right)
               \label{Block}
\end{align}
where  (\ref{trans2}) and  (\ref{dsym})  have been  used,  as well  as
(\ref{eq:42b}).   In   fact,  the  definition  of   $\hat{\bf  T}$  in
(\ref{eq:42b})  has its  origin  in (\ref{Block}).  Using this  result
gives $\overline{A}$ and its determinant as
\begin{align}
  \label{eq:30b}
  \overline{A}=\left[  \Pi A^{-1}\Pi^T\right]^{-1}
  &\to
    \beta\left(\begin{array}{ccc}
M^{-1} \mathbb{1}_3  & 0& 0\\
          0&\hat{\bf I}^{-1}& 0\\
         0& 0& \hat{\bf T}^{-1}\end{array}\right)
,\qquad \det\overline{A}=
\frac{\beta^9}{M^{3}\left(\det\hat{\bf I}\right)\left(\det\hat{\bf T}\right)}
\end{align}
and the quadratic form becomes
\begin{align}
  \label{eq:30c}
&(x-\Pi A^{-1}J)^T\overline{A}(x-\Pi A^{-1}J)
   \nonumber\\
  \to &\beta\left(
       \left({\bf P}-\frac{1}{\beta}\sum_i{\bf J}_i\right)^T,
       \left({\bf S}- \frac{1}{\beta}\sum_i{[}{\bf r}'_{i}{]}_\times\esc{\bf J}_i\right)^T,
        \left(\boldsymbol{\Pi}
        -\frac{1}{\beta}   \sum_i\frac{\partial \hat{{\bf M}}}{\partial m_{i}{\bf r}_{i}} \esc{\bf J}_i\right)^T
        \right)
        \nonumber\\
  &\times\left(\begin{array}{ccc}
M^{-1} \mathbb{1}_3  & 0& 0\\
          0&\hat{\bf I}^{-1}& 0\\
        0& 0& \hat{\bf T}^{-1}\end{array}\right)
\left(\begin{array}{c}
{\bf P}-\frac{1}{\beta}\sum_i{\bf J}_i
        \\
{\bf S}- \frac{1}{\beta}\sum_i{[}{\bf r}'_{i}{]}_\times\esc{\bf J}_i
        \\
        \boldsymbol{\Pi}
        - \frac{1}{\beta} \sum_i \frac{\partial \hat{{\bf M}}}{\partial m_{i}{\bf r}_{i}}   \esc{\bf J}_i
      \end{array}\right)
  \nonumber\\
  &=   \beta\left({\bf P}-\frac{1}{\beta}\sum_i{\bf J}_i\right)^T
    M^{-1}    \left({\bf P}-\frac{1}{\beta}\sum_i{\bf J}_i\right)
    +    \beta   \left({\bf S}- \frac{1}{\beta}\sum_i{[}{\bf r}'_{i}{]}_\times\esc{\bf J}_i\right)^T
    \esc\hat{\bf I}^{-1} \esc
    \left({\bf S}- \frac{1}{\beta}\sum_i{[}{\bf r}'_{i}{]}_\times\esc{\bf J}_i\right)
    \nonumber\\
  &+   \beta   \left(\boldsymbol{\Pi}
    - \frac{1}{\beta}\sum_i \frac{\partial \hat{{\bf M}}}{\partial m_{i}{\bf r}_{i}} \esc{\bf J}_i\right)^T
    \esc\hat{\bf T}^{-1}\esc       \left(\boldsymbol{\Pi}
    -  \frac{1}{\beta}       \sum_i \frac{\partial \hat{{\bf M}}}{\partial m_{i}{\bf r}_{i}} \esc{\bf J}_i\right)
\end{align}
Therefore, translating (\ref{apomJ}) gives
\begin{align}
  \label{eq:105}
  G^{(0)}({\bf P},{\bf S},\boldsymbol{\Pi},\beta)
  &=\frac{( \frac{2\pi}{\beta})^{3(N-3)/2}\prod^N_i(m_i)^{3/2}}
    {\sqrt{M^{3}\det\hat{\bf I}\det\hat{\bf T}}}
    \exp\left\{\frac{1}{2\beta}\sum_i\frac{1}{m_i}{\bf J}^2_i\right\}
    \nonumber\\
  & \times  
    \exp\left\{-\frac{\beta}{2}
    \left({\bf P} -\frac{1}{\beta}\sum_i{\bf J}_i\right)^T
    M^{-1}
    \left({\bf P}-\frac{1}{\beta}\sum_i{\bf J}_i\right)\right\}
    \nonumber\\
  & \times  
    \exp\left\{-\frac{\beta}{2}
    \left({\bf S}-  \frac{1}{\beta}\sum_i{[}{\bf r}'_{i}{]}_\times\esc{\bf J}_i\right)^T
    \esc\hat{\bf I}^{-1}\esc
    \left({\bf S}- \frac{1}{\beta}\sum_i{[}{\bf r}'_{i}{]}_\times\esc{\bf J}_i\right)\right\}
    \nonumber\\
  & \times  
    \exp\left\{   -\frac{\beta}{2}
    \left(\boldsymbol{\Pi} - \frac{1}{\beta}\sum_i \frac{1}{m_i}\frac{\partial \hat{{\bf M}}}{\partial{\bf r}_{i}} \esc{\bf J}_i\right)^T
    \esc\hat{\bf T}^{-1}\esc
    \left(\boldsymbol{\Pi} - \frac{1}{\beta} \sum_i \frac{1}{m_i}\frac{\partial \hat{{\bf M}}}{\partial{\bf r}_{i}}\esc{\bf J}_i\right)
\right\}                                                 
\end{align}
As a final remark and for future reference,  the  change   of  variables (\ref{eq:100})
in the present case takes the form
\begin{align}
  \label{eq:52b}
  \left(\begin{array}{c}{\bf p}'_1\\\vdots\\{\bf p}'_N\end{array}\right)
  &=\left(\begin{array}{c}{\bf p}_1\\\vdots\\{\bf p}_N\end{array}\right)
    -\frac{1}{\beta}\left(\begin{array}{ccc}
m_1\mathbb{1}_3&\cdots&0\\
\vdots&\ddots&\vdots\\
0&\cdots&{m_N}\mathbb{1}_3
\end{array}\right)  
\left(\begin{array}{ccc}
        \mathbb{1}_3  & {[}{\bf r}'_{1}{]}_\times^T&        \left(\frac{\partial \hat{{\bf M}}}{\partial m_{1}{\bf r}_{1}}\right )^T
        \\
\vdots & \vdots& \vdots\\
        \mathbb{1}_3&{[}{\bf r}'_{N}{]}_\times^T&
                          \left(\frac{\partial \hat{{\bf M}}}{\partial m_{N}{\bf r}_{N}}\right )^T
\end{array}\right)  \beta\left(\begin{array}{ccc}
M^{-1} \mathbb{1}_3  & 0& 0\\
          0&\hat{\bf I}^{-1}& 0\\
         0& 0& \hat{\bf T}^{-1}\end{array}\right)
\left(  \begin{array}{c}
{\bf P}\\
    {\bf S} \\
    \boldsymbol{\Pi}
  \end{array}
\right)\end{align}
so that
\begin{align}
 {\bf p}_{i}'
  &={\bf p}_{i} - m_{i}{\bf V} + m_{i} ({\bf r}_{i}-{\bf R})\times\hat{\boldsymbol{\Omega}}
    -\frac{\partial \hat{{\bf M}}}{\partial {\bf r}_{i}}\esc\hat{\boldsymbol{\nu}}
\label{BodyFrame}
\end{align}
where the dot product in the last term is over the indices of $\hat{\bf M}$ and $\hat{\boldsymbol{\nu}}$, and the different velocities are
\begin{align}
  {\bf V}
  &=M^{-1}{\bf P}
    \nonumber\\
  \hat{\boldsymbol{\Omega}}
  &=\hat{\bf I}^{-1}\esc{\bf S}
    \nonumber\\
  \hat{\boldsymbol{\nu}}&=\hat{\bf T}^{-1}\esc\boldsymbol{\Pi}
    \label{velos}
\end{align}
This motivated the change of variables (\ref{W:BodyFrame}) introduced in Sec. \ref{App:Entropy}.

\subsection{Calculation of ${\bf G}^{(1)}({\bf P},{\bf S},\boldsymbol{\Pi},\beta)$, ${\bf G}^2({\bf P},{\bf S},\boldsymbol{\Pi},\beta)$}
In component form, (\ref{eq:105}) becomes (repeated index summation convention used)
  \begin{align}
    \label{eq:249}
      G^{(0)}({\bf P},{\bf S},\boldsymbol{\Pi},\beta)
  &=\frac{(\frac{2\pi}{\beta})^{3(N-3)/2}\prod^N_i(m_i)^{3/2}}
    {\sqrt{M^{3}\det\hat{\bf I}\det\hat{\bf T}}}
    \exp\left\{\frac{1}{2\beta }\sum_i\frac{1}{ m_i}{\bf J}^\mu_i{\bf J}^\mu_i\right\}
    \nonumber\\
  & \times  
    \exp\left\{-\frac{\beta}{2M}  
    \left({\bf P}^\mu-\frac{1}{\beta}\sum_i{\bf J}^\mu_i\right)
    \left({\bf P}^\mu-\frac{1}{\beta}\sum_j{\bf J}^\mu_j\right)\right\}
    \nonumber\\
  & \times  
    \exp\left\{-\frac{\beta}{2}
    \left({\bf S}^\mu- \frac{1}{\beta}\sum_i{[}{\bf r}'_{i}{]}_\times^{\mu\sigma}{\bf J}^\sigma_i\right)
    \left[\hat{\bf I}^{-1}\right]^{\mu\nu}
    \left({\bf S}^\nu- \frac{1}{\beta}\sum_j{[}{\bf r}'_{j}{]}^{\nu\sigma'}_\times{\bf J}^{\sigma'}_j\right)
    \right\}
    \nonumber\\
  & \times  
    \exp\left\{   -\frac{\beta}{2}
    \left({\Pi}_\mu -\frac{1}{\beta}\sum_i \frac{1}{m_i}\frac{\partial \hat{M}_\mu}{\partial{\bf r}_{i}^{\sigma}} {\bf J}^\sigma_i\right)
    \left[\hat{\bf T}^{-1}\right]^{\mu\nu}
    \left({\Pi}_\nu- \frac{1}{\beta}\sum_j \frac{1}{m_j}\frac{\partial \hat{M}_\nu}{\partial {\bf r}^{\sigma'}_{j}}{\bf J}^{\sigma'}_j\right)
\right\}                                                 
\end{align}
From (\ref{eq:18}) and (\ref{eq:249}) we have
\begin{align}
  \label{eq:242b}
{\bf  G}^{(1)\gamma}_i
  =\frac{\partial G^{(0)}}{\partial {\bf J}^\gamma_i}
  =     G^{(0)}&\left[
  \frac{1}{\beta m_i}{\bf J}_i^\gamma
   + M^{-1}    \left({\bf P}^\gamma-\frac{1}{\beta}\sum_{i'}{\bf J}^\gamma_{i'}\right)
- {[}{\bf r}'_{i}{]}^{\gamma\mu}_\times \left[\hat{\bf I}^{-1}\right]^{\mu\nu}\left({\bf S}^\nu
    -\frac{1}{\beta}\sum_{i'}{[}{\bf r}'_{i'}{]}^{\nu\sigma'}_\times{\bf J}^{\sigma'}_{i'}\right)
    \right.
    \nonumber\\
  &\left.+ \frac{1}{m_i}
    \frac{\partial \hat{M}_\mu}{\partial{\bf r}^\gamma_{i}}\left[\hat{\bf T}^{-1}\right]^{\mu\nu}
    \left({\Pi}_\nu
    - \frac{1}{\beta} \sum_{i'} \frac{1}{m_{i'}}\frac{\partial \hat{M}_\nu}{\partial {\bf r}^\sigma_{i'}}
    {\bf J}^\sigma_{i'}\right)
\right]
\end{align}
%Observe that this expression fullfils (\ref{eq:243}), as it should.
The second derivative term is
\begin{align}
  \label{eq:271}
  {\bf  G}^{(2)\gamma\gamma'}_{ij}
  &  =\frac{\partial^2G^{(0)}}{\partial {\bf J}^\gamma_i\partial {\bf J}_j^{\gamma'}}
    =\frac{\partial{\bf G}_i^{(1)\gamma}}{\partial {\bf J}_j^{\gamma'}}
      \nonumber\\
&=  \frac{\partial G^{(0)}}{\partial {\bf J}_j^{\gamma'}}   \left[
    \frac{1}{\beta m_i}{\bf J}_i^\gamma
    + M^{-1}    \left({\bf P}^\gamma-\frac{1}{\beta}\sum_{i'}{\bf J}^\gamma_{i'}\right)
    -  {[}{\bf r}'_{i}{]}^{\gamma\mu}_\times \left[\hat{\bf I}^{-1}\right]^{\mu\nu}\left({\bf S}^\nu
    -\frac{1}{\beta}\sum_{i'}{[}{\bf r}'_{i'}{]}^{\nu\sigma'}_\times{\bf J}^{\sigma'}_{i'}\right)
    \right.
    \nonumber\\
  &\left.+\frac{1}{m_i}
    \frac{\partial \hat{M}_\mu}{\partial{\bf r}^\gamma_{i}}\left[\hat{\bf T}^{-1}\right]^{\mu\nu}
    \left({\Pi}_\nu
    - \frac{1}{\beta} \sum_{i'} \frac{1}{m_{i'}}\frac{\partial \hat{M}_\nu}{\partial {\bf r}^\sigma_{i'}}
    {\bf J}^\sigma_{i'}\right)
    \right]
    \nonumber\\
  &+G^{(0)}\frac{\partial}{\partial {\bf J}_j^{\gamma'}}\left[
    \frac{1}{\beta m_i}{\bf J}_i^\gamma
    + M^{-1}    \left({\bf P}^\gamma-\frac{1}{\beta}\sum_{i'}{\bf J}^\gamma_{i'}\right)
    -          {[}{\bf r}'_{i}{]}^{\gamma\mu}_\times \left[\hat{\bf I}^{-1}\right]^{\mu\nu}\left({\bf S}^\nu
    -\frac{1}{\beta}\sum_{i'}{[}{\bf r}'_{i'}{]}^{\nu\sigma'}_\times{\bf J}^{\sigma'}_{i'}\right)
    \right.
    \nonumber\\
  &\left.+ \frac{1}{m_i}
    \frac{\partial \hat{M}_\mu}{\partial{\bf r}^\gamma_{i}}\left[\hat{\bf T}^{-1}\right]^{\mu\nu}
    \left({\Pi}_\nu
    - \frac{1}{\beta} \sum_{i'} \frac{1}{m_{i'}}\frac{\partial \hat{M}_\nu}{\partial {\bf r}^\sigma_{i'}}
    {\bf J}^\sigma_{i'}\right)
    \right]
\end{align}
that is\begin{align}
      \label{eq:282}
  {\bf  G}^{(2)\gamma\gamma'}_{ij}&= G^{(0)}  \left[
    \frac{1}{\beta m_i}{\bf J}_i^\gamma
    + M^{-1}    \left({\bf P}^\gamma-\frac{1}{\beta}\sum_{i'}{\bf J}^\gamma_{i'}\right)
    -{[}{\bf r}'_{i}{]}^{\gamma\mu}_\times \left[\hat{\bf I}^{-1}\right]^{\mu\nu}\left({\bf S}^\nu
    -\frac{1}{\beta}\sum_{i'}{[}{\bf r}'_{i'}{]}^{\nu\sigma'}_\times{\bf J}^{\sigma'}_{i'}\right)
    \right.
    \nonumber\\
  &\left.+\frac{1}{m_i}
    \frac{\partial \hat{M}_\mu}{\partial{\bf r}^\gamma_{i}}\left[\hat{\bf T}^{-1}\right]^{\mu\nu}
    \left({\Pi}_\nu
    - \frac{1}{\beta} \sum_{i'} \frac{1}{m_{i'}}\frac{\partial \hat{M}_\nu}{\partial {\bf r}^\sigma_{i'}}
    {\bf J}^\sigma_{i'}\right)
    \right]
    \nonumber\\
  &\times\left[
 \frac{1}{\beta m_j}{\bf J}_j^{\gamma'}
    + M^{-1}    \left({\bf P}^{\gamma'}-\frac{1}{\beta}\sum_{i'}{\bf J}^{\gamma'}_{i'}\right)
    -          {[}{\bf r}'_{j}{]}^{\gamma'\mu'}_\times \hat{\bf I}^{-1}_{\mu'\nu'}\left({\bf S}^{\nu'}
    -\frac{1}{\beta}\sum_{i'}{[}{\bf r}'_{i'}{]}^{\nu'\sigma''}_\times{\bf J}^{\sigma''}_{i'}\right)
    \right.
    \nonumber\\
  &\left.+ \frac{1}{m_j}
    \frac{\partial \hat{M}_{\mu'}}{\partial {\bf r}^{\gamma'}_{j}}\hat{\bf T}^{-1}_{\mu'\nu'}
    \left({\Pi}_{\nu'}
    - \frac{1}{\beta} \sum_{i'} \frac{1}{m_{i'}}\frac{\partial \hat{M}_{\nu'}}{\partial {\bf r}^{\sigma'}_{i'}}
    {\bf J}^{\sigma'}_{i'}\right)
    \right]
    \nonumber\\
  &+G^{(0)}\left[
    \frac{1}{\beta m_i}\delta_{ij}\delta_{\gamma\gamma'}
   + M^{-1}    \left(-\frac{1}{\beta}\sum_{i'}\frac{\partial}{\partial {\bf J}_j^{\gamma'}}{\bf J}^\gamma_{i'}\right)
    -{[}{\bf r}'_{i}{]}^{\gamma\mu}_\times \left[\hat{\bf I}^{-1}\right]^{\mu\nu}
  \left(-  \frac{1}{\beta}\sum_{i'}{[}{\bf r}'_{i'}{]}^{\nu\sigma'}_\times\frac{\partial}{\partial {\bf J}_j^{\gamma'}}{\bf J}^{\sigma'}_{i'}\right)
    \right.
    \nonumber\\
  &\left.+ \frac{1}{m_i}
    \frac{\partial \hat{M}_\mu}{\partial{\bf r}^\gamma_{i}}\left[\hat{\bf T}^{-1}\right]^{\mu\nu}
    \left(  - \frac{1}{\beta} \sum_{i'} \frac{1}{m_{i'}}\frac{\partial \hat{M}_\nu}{\partial {\bf r}^\sigma_{i'}}
   \frac{\partial}{\partial {\bf J}_j^{\gamma'}} {\bf J}^\sigma_{i'}\right)
    \right]
\end{align}
which simplifies to
\begin{align}
  \label{eq:343}
    {\bf  G}^{(2)\gamma\gamma'}_{ij}
  &=G^{(0)}  \left \{ \left[
    \frac{1}{\beta m_i}{\bf J}_i^\gamma
    + M^{-1}    \left({\bf P}^\gamma-\frac{1}{\beta}\sum_{i'}{\bf J}^\gamma_{i'}\right)
    -{[}{\bf r}'_{i}{]}^{\gamma\mu}_\times \hat{\bf I}^{-1}_{\mu\nu}\left({\bf S}^\nu
    -\frac{1}{\beta}\sum_{i'}{[}{\bf r}'_{i'}{]}^{\nu\sigma'}_\times{\bf J}^{\sigma'}_{i'}\right)
    \right.\right.
    \nonumber\\
  &\left.+ \frac{1}{m_i}
    \frac{\partial \hat{M}_\mu}{\partial{\bf r}^\gamma_{i}}\hat{\bf T}^{-1}_{\mu\nu}
    \left({\Pi}_\nu
    - \frac{1}{\beta} \sum_{i'} \frac{1}{m_{i'}}\frac{\partial \hat{M}_\nu}{\partial {\bf r}^\sigma_{i'}}
    {\bf J}^\sigma_{i'}\right)
    \right]
    \nonumber\\
  &\times\left[
    \frac{1}{\beta m_j}{\bf J}_j^{\gamma'}
    + M^{-1}    \left({\bf P}^{\gamma'}-\frac{1}{\beta}\sum_{i'}{\bf J}^{\gamma'}_{i'}\right)
    -          {[}{\bf r}'_{j}{]}^{\gamma'\mu'}_\times \hat{\bf I}^{-1}_{\mu'\nu'}\left({\bf S}^{\nu'}
    -\frac{1}{\beta}\sum_{i'}{[}{\bf r}'_{i'}{]}^{\nu'\sigma''}_\times{\bf J}^{\sigma''}_{i'}\right)
    \right.
    \nonumber\\
  &\left.+ \frac{1}{m_j}
    \frac{\partial \hat{M}_{\mu'}}{\partial {\bf r}^{\gamma'}_{j}}\hat{\bf T}^{-1}_{\mu'\nu'}
    \left({\Pi}_{\nu'}
    - \frac{1}{\beta} \sum_{i'} \frac{1}{m_{i'}}\frac{\partial \hat{M}_{\nu'}}{\partial {\bf r}^{\sigma'}_{i'}}
    {\bf J}^{\sigma'}_{i'}\right)
    \right]
    \nonumber\\
  &+\left .
    \frac{1}{\beta m_i}\delta_{ij}\delta_{\gamma\gamma'}
    -\frac{1}{\beta} M^{-1}    \delta_{\gamma\gamma'}
    +\frac{1}{\beta}{[}{\bf r}'_{i}{]}^{\gamma\mu}_\times \hat{\bf I}^{-1}_{\mu\nu} {[}{\bf r}'_{j}{]}^{\nu\gamma'}_\times
    -\frac{1}{\beta m_i m_j} 
    \frac{\partial \hat{M}_\mu}{\partial{\bf r}^\gamma_{i}}\hat{\bf T}^{-1}_{\mu\nu}
    \frac{\partial \hat{M}_\nu}{\partial {\bf r}^{\gamma'}_{j}}
    \right \}
\end{align}
Evaluated at ${\bf J}=0$ these quantities become
\begin{align}
  \label{eq:244}
  G^{(0)}({\bf P},{\bf S},\boldsymbol{\Pi},\beta)
  &=\frac{( \frac{2\pi}{\beta})^{3(N-3)/2}\prod^N_i(m_i)^{3/2}}
    {\sqrt{M^{3}\det\hat{\bf I}\det\hat{\bf T}}}
    \exp\left\{-\beta\frac{{\bf P}^2}{2M} -\frac{\beta}{2}        {\bf S}^T \esc\hat{\bf I}^{-1} \esc     {\bf S}  -\frac{\beta}{2}   \boldsymbol{\Pi}^T
    \esc\hat{\bf T}^{-1}\esc    \boldsymbol{\Pi}
    \right\}                                                 
    \nonumber\\
{\bf  G}^{(1)\gamma}_i({\bf P},{\bf S},\boldsymbol{\Pi},\beta)
&  =    G^{(0)}\left[
    M^{-1}    {\bf P}^\gamma
-          {[}{\bf r}'_{i}{]}^{\gamma\mu}_\times \hat{\bf I}^{-1}_{\mu\nu}{\bf S}^\nu
+    \frac{1}{m_i}\frac{\partial \hat{M}_\mu}{\partial{\bf r}^\gamma_{i}}\hat{\bf T}^{-1}_{\mu\nu}
  {\Pi}_\nu
\right]
    \nonumber\\
    {\bf  G}^{(2)\gamma\gamma'}_{ij}({\bf P},{\bf S},\boldsymbol{\Pi},\beta)
  &=G^{(0)}  \left\{ \left[
     M^{-1}    {\bf P}^\gamma
    -{[}{\bf r}'_{i}{]}^{\gamma\mu}_\times \hat{\bf I}^{-1}_{\mu\nu}{\bf S}^\nu
    +\frac{1}{m_i}\frac{\partial \hat{M}_\mu}{\partial{\bf r}^\gamma_{i}}\hat{\bf T}^{-1}_{\mu\nu}
   {\Pi}_\nu
       \right]\right .
    \nonumber\\
  &\times\left[
        M^{-1}    {\bf P}^{\gamma'}
    -          {[}{\bf r}'_{j}{]}^{\gamma'\mu'}_\times \hat{\bf I}^{-1}_{\mu'\nu'}{\bf S}^{\nu'}
    +   \frac{1}{m_j}\frac{\partial \hat{M}_{\mu'}}{\partial{\bf r}^{\gamma'}_{i}}\hat{\bf T}^{-1}_{\mu'\nu'}
    {\Pi}_{\nu'}
        \right]
    \nonumber\\
  &+\left .\frac{1}{\beta}\left[
    \frac{1}{m_i}\delta_{ij}\delta_{\gamma\gamma'}
   - M^{-1}    \delta_{\gamma\gamma'}
    +{[}{\bf r}'_{i}{]}^{\gamma\mu}_\times \hat{\bf I}^{-1}_{\mu\nu} {[}{\bf r}'_{j}{]}^{\nu\gamma'}_\times
    - \frac{1}{m_i m_j}\frac{\partial \hat{M}_\mu}{\partial{\bf r}^\gamma_{i}}\hat{\bf T}^{-1}_{\mu\nu}
    \frac{\partial \hat{M}_\nu}{\partial {\bf r}^{\gamma'}_{j}}
    \right]\right \}
\end{align}
Evaluated at rest, that is ${\bf P}={\bf S}=\boldsymbol{\Pi}=0$, these quantities become
\begin{align}
  \label{eq:220}
  G^{(0)}
  &=\frac{( \frac{2\pi}{\beta})^{3(N-3)/2}\prod^N_i(m_i)^{3/2}}
    {\sqrt{M^{3}\det\hat{\bf I}\det\hat{\bf T}}}
    \nonumber\\
\frac{1}{G^{(0)}}  \int dp {\bf v}^\mu_{i}
    \Delta(p,{\bf 0},{\bf 0},{\bf 0},\beta)
&=  0
                                              \nonumber\\
  \frac{1}{G^{(0)}}  \int dp {\bf v}^\mu_{i}{\bf v}_j^\nu
    \Delta(p,{\bf 0},{\bf 0},{\bf 0},\beta)
&= \frac{1}{\beta}\left[
    \frac{1}{m_i}\delta_{ij}\delta_{\mu\nu}
    - M^{-1}    \delta_{\mu\nu}
    +{[}{\bf r}'_{i}{]}^{\mu\mu'}_\times \hat{\bf I}^{-1}_{\mu'\nu'} {[}{\bf r}'_{j}{]}^{\nu'\nu}_\times
    - \frac{1}{m_i m_j}\frac{\partial \hat{M}_{\mu'}}{\partial{\bf r}^\mu_{i}}\hat{\bf T}^{-1}_{\mu'\nu'}
 \frac{\partial \hat{M}_{\nu'}}{\partial {\bf r}^{\nu}_{j}}
    \right]
\end{align}

\color{black}
\subsection{Sectioned Gaussian integrals for equilibrium averages}
\label{App:Gaussian}
The  results   of  this  subsection   allow  us  to   demonstrate  the
equipartition theorem in  Sec. \ref{App:Equipartition}. We compute
sectioned Gaussians similar  to the those in the  previous section but
with no restriction on the dilational momentum, that is
\begin{align}
  g^{(0)}({\bf P},{\bf S},\beta)
  &=\int dp \Delta(p,{\bf P},{\bf S},\beta)
    \label{G0b}
    \\
{\bf  g}^{(1)}_i({\bf P},{\bf S},\beta)
  &=\int dp \frac{{\bf p}_{i}}{m_{i}}  \Delta(p,{\bf P},{\bf S},\beta)
\label{G1b}  \\  
{\bf  g}^{(2)}_{ij}({\bf P},{\bf S},\beta)
  &  =\int dp \frac{{\bf p}_{i}}{m_{i}}\frac{{\bf p}^T_{j}}{m_{j}}
    \Delta(p,{\bf P},{\bf S},,\beta)
\label{G2b}\end{align}
where
\begin{align}
  \label{eq:45b}
  \Delta(p,{\bf P},{\bf S},\beta)
  &\equiv
  \delta\left(\sum_{i}{\bf p}_{i}-{\bf P}\right)
  \delta\left(\sum_{i}[{\bf r}'_{i}]_\times\esc{\bf p}_{i}-{\bf S} \right)
e^{-\beta\sum_i^N\frac{{\bf p}_i^2}{2m_i}+\sum_i{\bf J}_i\esc\frac{{\bf p}_i}{m_i}}
\end{align}
The results for ${\bf J}=0$ can be obtained by integrating (\ref{eq:244}) over
$\boldsymbol{\Pi}$, so that
%
\begin{align}
g^{(0)}({\bf P},{\bf S},\beta) &=
\int d\boldsymbol{\Pi}\; G^{(0)}({\bf P},{\bf S},\boldsymbol{\Pi},\beta)=
\frac{(\frac{2\pi}{\beta})^{3(N-2)/2}\prod^N_i(m_i)^{3/2}}
{\sqrt{M^{3}\det\hat{\bf I}}}
\exp\left\{-\beta\frac{{\bf P}^2}{2M} -
\frac{\beta}{2}{\bf S}^T \esc\hat{\bf I}^{-1} \esc{\bf S}\right\}
\nonumber\\
{\bf g}^{(1)\gamma}_i({\bf P},{\bf S},\beta) &=
\int d\boldsymbol{\Pi}\;{\bf G}^{(1)\gamma}_i({\bf P},{\bf S},\boldsymbol{\Pi},\beta)
= g^{(0)}\left[ M^{-1}{\bf P}^\gamma -
[{\bf r}'_i]^{\gamma\mu}_\times\hat{\bf I}^{-1}_{\mu\nu}{\bf S}^\nu
\right]
\nonumber\\
{\bf  g}^{(2)\gamma\gamma'}_{ij}({\bf P},{\bf S},\beta) &=
\int d\boldsymbol{\Pi}\;{\bf G}^{(2)\gamma\gamma'}_{ij}({\bf P},{\bf S},\boldsymbol{\Pi},\beta)
=g^{(0)}  \left\{ \left[ M^{-1}{\bf P}^\gamma -
[{\bf r}'_i]^{\gamma\mu}_\times\hat{\bf I}^{-1}_{\mu\nu}{\bf S}^\nu
       \right]\right .
\nonumber\\
&\quad\times\left[ M^{-1}{\bf P}^{\gamma'} -
[{\bf r}'_j]^{\gamma'\mu'}_\times\hat{\bf I}^{-1}_{\mu'\nu'}{\bf S}^{\nu'}
        \right]
    \nonumber\\
  &\quad+\left .\frac{1}{\beta}\left[
\frac{1}{m_i}\delta_{ij}\delta_{\gamma\gamma'} - M^{-1}\delta_{\gamma\gamma'}
+[{\bf r}'_i]^{\gamma\mu}_\times\hat{\bf I}^{-1}_{\mu\nu}[{\bf r}'_j]^{\nu\gamma'}_\times
    \right]\right \}
\nonumber\\
&- g^{(0)} \frac{1}{\beta m_i m_j}\frac{\partial \hat{M}_\mu}{\partial {\bf r}^\gamma_i}
\hat{\bf T}^{-1}_{\mu\nu}
\frac{\partial \hat{M}_\nu}{\partial{\bf r}^{\gamma'}_j}
+\frac{1}{m_i m_j}\frac{\partial\hat{M}_\mu}{\partial{\bf r}^\gamma_{i}}
\hat{\bf T}^{-1}_{\mu\nu}
\frac{\partial \hat{M}_{\mu'}}{\partial{\bf r}^{\gamma'}_i}
\hat{\bf T}^{-1}_{\mu'\nu'}
\int d\boldsymbol{\Pi}\;\boldsymbol{\Pi}_\nu\;\boldsymbol{\Pi}_{\nu'}
G^{(0)}({\bf P},{\bf S},\boldsymbol{\Pi},\beta)
\nonumber\\
&=g^{(0)}  \left\{ \left[ M^{-1}{\bf P}^\gamma -
[{\bf r}'_i]^{\gamma\mu}_\times\hat{\bf I}^{-1}_{\mu\nu}{\bf S}^\nu
\right] \left[ M^{-1}{\bf P}^{\gamma'} -
[{\bf r}'_j]^{\gamma'\mu'}_\times\hat{\bf I}^{-1}_{\mu'\nu'}{\bf S}^{\nu'}
\right]\right .\nonumber\\
&\quad+\left .\frac{1}{\beta}\left[
\frac{1}{m_i}\delta_{ij}\delta_{\gamma\gamma'} - M^{-1}\delta_{\gamma\gamma'}
+[{\bf r}'_i]^{\gamma\mu}_\times\hat{\bf I}^{-1}_{\mu\nu}[{\bf r}'_j]^{\nu\gamma'}_\times
    \right]\right \}
\end{align}
%
where (\ref{Gaussian}) has been used to obtain the first result, terms with a
single factor of $\boldsymbol{\Pi}$ integrate to zero because the integrands
are odd functions, and in the last result, the final two terms cancel after
integration by parts is used in the integral to yield
$g^{(0)}\beta^{-1}\hat{\bf T}^{\nu'\nu}$.


Evaluated at rest, that is ${\bf P}={\bf S}=0$, we obtain
\begin{align}
  \label{eq:315}
    g^{(0)}
  &=\frac{( \frac{2\pi}{\beta})^{3(N-2)/2}\prod^N_i(m_i)^{3/2}}
    {\sqrt{M^{3}\det\hat{\bf I}}}
    \nonumber\\
  \frac{1}{g^{(0)}}  \int dp {\bf v}^\mu_{i}
  \Delta(p,{\bf 0},{\bf 0},\beta)
  &=  0
    \nonumber\\
  \frac{1}{g^{(0)}}  \int dp {\bf v}^\mu_{i}{\bf v}_j^\nu
  \Delta(p,{\bf 0},{\bf 0},\beta)
  &= \frac{1}{\beta}\left[
    \frac{1}{m_i}\delta_{ij}\delta_{\mu\nu}
    - M^{-1}    \delta_{\mu\nu}
    +{[}{\bf r}'_{i}{]}^{\mu\mu'}_\times \hat{\bf I}^{-1}_{\mu'\nu'} {[}{\bf r}'_{j}{]}^{\nu'\nu}_\times
    \right]
\end{align}
\subsection{The equipartition theorem}
\label{App:Equipartition}

The equipartition theorem gives an expression for the rest equilibrium
average  of the  translational  kinetic energy.  The rest  equilibrium
average is

   \begin{align}
  \label{eq:27}
  \llangle K^{\rm trans}\rrangle^{\cal E} =
  \llangle\sum_i\frac{m_i}{2}{\bf v}_i^2\rrangle^{\cal E}_{\rm rest}
  &=\int dz \rho_{\cal E}^{\rm rest}(z)\sum_i\frac{m_i}{2}{\bf v}_i^2
\end{align}
where the microcanonical ensemble is given in (\ref{W:eq:289}). Under equivalence of ensembles we
may use the canonical ensemble (\ref{eq:16}) instead, and then
\begin{align}
  \label{eq:71}
  \llangle K^{\rm trans}\rrangle^{\cal E} =
  \llangle\sum_i\frac{m_i}{2}{\bf v}_i^2\rrangle^{\cal E}_{\rm rest}
  &=\frac{1}{Z(\beta)}\int dz
    \delta\left(\hat{\bf R}(z) \right)
    \delta\left(\hat{\bf P}(z) \right)
    \delta\left(\hat{\bf S}(z) \right)
    e^{-\beta\hat{H}(z)}\sum_i\frac{m_i}{2}{\bf v}_i^2
\end{align}
This integral is of the form (\ref{eq:315}). 
Set $i=j$ and $\mu=\nu$, multiply by $m_i/2$, and sum over $\mu$ and all particles in (\ref{eq:315}) to obtain
\begin{align}
  \label{eq:424}
\frac{1}{g^{(0)}}  \int dp \sum_i\frac{m_i}{2}{\bf v}^\mu_{i}{\bf v}_i^\mu
      \Delta(p,{\bf 0},{\bf 0},\beta)
    &=   \frac{1}{2\beta}
      \left[
      \sum_im_i    \left(\frac{1}{m_i}    -     M^{-1}\right)\delta_{\mu\mu}
      +\sum_im_i     {[}{\bf r}'_{i}{]}_\times^{\mu\mu'}\left[\hat{\bf I}^{-1} \right]^{\mu'\nu'} {[}{\bf r}'_{i}{]}_\times^{\nu'\mu}
\right]
      \nonumber\\
    &=  \frac{1}{2\beta}
      \left(3N-3
      -\left[\hat{\bf I}\right]^{\mu'\nu'}\left[\hat{\bf I}^{-1} \right]^{\mu'\nu'}
\right)
=\frac{3(N-2)}{2\beta}
  \end{align}
where (\ref{eq:603}) have been used to simplify the
expression.
  Therefore, we obtain
\begin{align}
    \label{eq:168}
  \llangle K^{\rm trans}\rrangle^{\cal E}    =
  \llangle \sum_i\frac{m_i}{2}{\bf v}_i^2\rrangle^{\cal E}_{\rm rest}    =\frac{3(N-2)}{2\beta}
       \end{align}  where $\beta=\beta({\cal E})$ is the function given in (\ref{eq:180}). This
  theorem allows us to compute the temperature function $T^{\rm MT}({\cal E})$ in MD simulations.


  \subsection{The conditional covariance of angular velocity}
  \label{Sec:CondCovAng}
  The conditional covariance of the angular velocity $\hat{\boldsymbol{\omega}}$ is 
\begin{align}
  \label{eq:389}
    \llangle\hat{\boldsymbol{\omega}}^\alpha\hat{\boldsymbol{\omega}}^{\beta}
    \rrangle^{\boldsymbol{\Lambda}{{\bf M}}{\cal E}}_{\rm rest}
    \overset{(\ref{eq:136})}{=}
{B}^{-1}_{\alpha\alpha'}{B}^{-1}_{\beta\beta'}
    \llangle
    i{\cal L}\hat{\Lambda}_{\alpha'} i{\cal L}\hat{\Lambda}_{\beta'}
    \rrangle^{\boldsymbol{\Lambda}{{\bf M}}{\cal E}}_{\rm rest}
   ={B}^{-1}_{\alpha\alpha'}{B}^{-1}_{\beta\beta'}
    \llangle
    \sum_{i}\frac{\partial\hat{\Lambda}_{\alpha'}}{\partial{\bf r}_i^{\mu}}{\bf v}_i^\mu
    \sum_{j}\frac{\partial\hat{\Lambda}_{\beta'}}{\partial{\bf r}_j^{\nu}}{\bf v}_j^\nu
    \rrangle^{\boldsymbol{\Lambda}{{\bf M}}{\cal E}}_{\rm rest}
\end{align}
Using the equivalence of ensembles, the momentum part of the integrals
is given by (\ref{eq:220}), so that
\begin{align}
  \label{eq:391}
  &\llangle
    \sum_{i}\frac{\partial\hat{\Lambda}_{\alpha'}}{\partial{\bf r}_i^{\mu}}{\bf v}_i^\mu
    \sum_{j}\frac{\partial\hat{\Lambda}_{\beta'}}{\partial{\bf r}_j^{\nu}}{\bf v}_j^\nu
    \rrangle^{\boldsymbol{\Lambda}{{\bf M}}{\cal E}}_{\rm rest}
    \nonumber\\
  &= k_BT\llangle
    \sum_{ij}\frac{\partial\hat{\Lambda}_{\alpha'}}{\partial{\bf r}_i^{\mu}}
    \frac{\partial\hat{\Lambda}_{\beta'}}{\partial{\bf r}_j^{\nu}}
    \left[
    \frac{1}{m_i}\delta_{ij}\delta_{\mu\nu}
    - M^{-1}    \delta_{\mu\nu}
    +{[}{\bf r}'_{i}{]}^{\mu\mu'}_\times \hat{\bf I}^{-1}_{\mu'\nu'} {[}{\bf r}'_{j}{]}^{\nu'\nu}_\times
    - \frac{\partial \hat{M}_{\mu'}}{\partial m_{i}{\bf r}^\mu_{i}}\hat{\bf T}^{-1}_{\mu'\nu'}
    \frac{\partial \hat{M}_{\nu'}}{\partial m_{i'}{\bf r}^{\nu}_{j}}
    \right]      
    \rrangle^{\boldsymbol{\Lambda}{{\bf M}}{\cal E}}_{\rm rest}
\end{align}
This gives four terms that we discuss separately. The first term is
\begin{align}
  \label{eq:345}
\llangle  \sum_{ij}\frac{\partial\hat{\Lambda}_{\alpha'}}{\partial{\bf r}_i^{\mu}}
  \frac{\partial\hat{\Lambda}_{\beta'}}{\partial{\bf r}_j^{\nu}}
  \frac{1}{m_i}\delta_{ij}\delta_{\mu\nu}
  \rrangle^{\boldsymbol{\Lambda}{{\bf M}}{\cal E}}_{\rm rest}
  &=\llangle
    \sum_{i}     \frac{1}{m_i}
    \frac{\partial\hat{\Lambda}_{\alpha'}}{\partial{\bf r}_i^{\mu}}
    \frac{\partial\hat{\Lambda}_{\beta'}}{\partial{\bf r}_i^{\mu}}
    \rrangle^{\boldsymbol{\Lambda}{{\bf M}}{\cal E}}_{\rm rest}
\end{align}
The second term vanishes due to the translation invariance property (\ref{trans}).
The third term becomes, by virtue of (\ref{eq:255}) and the fact  the inertia tensor is a function of the CG variables,
\begin{align}
  \label{eq:402}
  & \llangle
    \sum_{i}\frac{\partial\hat{\Lambda}_{\alpha'}}{\partial{\bf r}_i^{\mu}}      {[}{\bf r}'_{i}{]}^{\mu\mu'}_\times 
    \sum_j\frac{\partial\hat{\Lambda}_{\beta'}}{\partial{\bf r}_j^{\nu}}{[}{\bf r}'_{j}{]}^{\nu'\nu}_\times
    \hat{\bf I}^{-1}_{\mu'\nu'} \rrangle^{\boldsymbol{\Lambda}{{\bf M}}{\cal E}}_{\rm rest}
    =-{B}_{\alpha'\mu'}{B}_{\beta'\nu'}{\bf I}^{-1}_{\mu'\nu'}
\end{align}
Finally,  we recognize  in the  fourth term  the matrix  $\hat{\bf H}$
defined  in  (\ref{eq:H})  that  vanishes  identically,  according  to
(\ref{eq:449}).
Collecting these results gives
\begin{align}
  \label{eq:392}
    \llangle\hat{\boldsymbol{\omega}}^\alpha\hat{\boldsymbol{\omega}}^{\beta}
    \rrangle^{\boldsymbol{\Lambda}{{\bf M}}{\cal E}}_{\rm rest}
  &=k_BT\left({B}^{-1}_{\alpha\alpha'}{B}^{-1}_{\beta\beta'}
    \llangle
    \sum_{i}     \frac{1}{m_i}
    \frac{\partial\hat{\Lambda}_{\alpha'}}{\partial{\bf r}_i^{\mu}}
    \frac{\partial\hat{\Lambda}_{\beta'}}{\partial{\bf r}_i^{\mu}}      \rrangle^{\boldsymbol{\Lambda}{{\bf M}}{\cal E}}_{\rm rest}  
    -{\bf I}^{-1}_{\alpha\beta}\right)
\end{align}
We recognize the matrix ${\bf L}$ defined in (\ref{eq:L}) in the first
term, and  note from (\ref{eq:225})  and (\ref{eq:229}) that it  is an
explicit function of CG variables, so that
\begin{align}
      \label{eq:415}
    \llangle\hat{\boldsymbol{\omega}}\hat{\boldsymbol{\omega}}^T
    \rrangle^{\boldsymbol{\Lambda}{{\bf M}}{\cal E}}_{\rm rest}
  &{=}k_BT\left({\bf L}-{\bf I}^{-1}\right)
\end{align}
The conditional covariance of the angular velocity in the principal axis frame is
\begin{align}
  \label{eq:00-S}
      \llangle\hat{\boldsymbol{\omega}}_0^\alpha\hat{\boldsymbol{\omega}}_0^{\beta}
    \rrangle^{\boldsymbol{\Lambda}{{\bf M}}{\cal E}}_{\rm rest}
  &=\left[e^{-[\boldsymbol{\Lambda}]_\times}\right]_{\alpha\alpha'}
    \left[e^{-[\boldsymbol{\Lambda}]_\times}\right]_{\beta\beta'}
    \llangle\hat{\boldsymbol{\omega}}^{\alpha'}\hat{\boldsymbol{\omega}}^{\beta'}
    \rrangle^{\boldsymbol{\Lambda}{{\bf M}}{\cal E}}_{\rm rest}
=kT(\mathbb{L}-\mathbb{I}^{-1})
\end{align}
where the last expression was obtained by inserting (\ref{eq:415}) into
(\ref{eq:00-S}), and using  (\ref{eq:225}).


\begin{thebibliography}{2}
\expandafter\ifx\csname natexlab\endcsname\relax\def\natexlab#1{#1}\fi
\expandafter\ifx\csname bibnamefont\endcsname\relax
  \def\bibnamefont#1{#1}\fi
\expandafter\ifx\csname bibfnamefont\endcsname\relax
  \def\bibfnamefont#1{#1}\fi
\expandafter\ifx\csname citenamefont\endcsname\relax
  \def\citenamefont#1{#1}\fi
\expandafter\ifx\csname url\endcsname\relax
  \def\url#1{\texttt{#1}}\fi
\expandafter\ifx\csname urlprefix\endcsname\relax\def\urlprefix{URL }\fi
\providecommand{\bibinfo}[2]{#2}
\providecommand{\eprint}[2][]{\url{#2}}

\bibitem[{\citenamefont{Wilcox}(1967)}]{Wilcox1967}
\bibinfo{author}{\bibfnamefont{R.~M.} \bibnamefont{Wilcox}},
  \bibinfo{journal}{Journal of Mathematical Physics}
  \textbf{\bibinfo{volume}{8}}, \bibinfo{pages}{962} (\bibinfo{year}{1967}).

\bibitem[{\citenamefont{Díaz}(2019)}]{Olguin2019}
\bibinfo{author}{\bibfnamefont{E.~O.} \bibnamefont{Díaz}},
  \emph{\bibinfo{title}{3D Motion of Rigid Bodies A Foundation for Robot
  Dynamics Analysis}} (\bibinfo{publisher}{Springer}, \bibinfo{year}{2019}).

\end{thebibliography}
%\bibliography{library,00Books}
\end{document}
