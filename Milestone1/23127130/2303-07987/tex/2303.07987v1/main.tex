\newif\ifllncs  %
\newif\iffull   %

\fulltrue

\def\shownotes{1}  %

\ifnum\shownotes=1
\newcommand{\authnote}[2]{\textcolor{red}{\textsf{#1 }\textcolor{blue}{ #2}}}
\else
\newcommand{\authnote}[2]{}
\fi
\newcommand{\haozhe}[1]{{\authnote{Haozhe}{#1}}}
\newcommand{\kaiyue}[1]{{\authnote{Kaiyue}{#1}}}
\newcommand{\yilei}[1]{{\authnote{Yilei}{#1}}}



\ifllncs
  \documentclass[runningheads,a4paper]{llncs}
  \let\proof\relax
  \let\endproof\relax
\else
\documentclass[11pt]{article}
\fi

\usepackage
[pdftex,colorlinks=true,pdfpagemode=UseNone,urlcolor=black,linkcolor=black,citecolor=black,pdfstartview=FitH]{hyperref}
\setcounter{secnumdepth}{3}
\usepackage{amsmath,amsfonts, amssymb}
\usepackage{graphicx}
\usepackage{amsthm}
\usepackage{algorithm}
\usepackage{cleveref}
\usepackage{algpseudocode}
\usepackage{tikz}
\usepackage{url}
\usepackage{caption}
\usepackage{subcaption}
\usepackage{booktabs}
\usepackage{diagbox}

\bibliographystyle{alpha}

\iffull
\setlength{\oddsidemargin}{-0.22in}
\setlength{\evensidemargin}{-0.22in}
\setlength{\textwidth}{6.8in}
\setlength{\topmargin}{-0.1in}
\setlength{\textheight}{8.5in}

\setlength{\parindent}{0in}
\setlength{\parskip}{5px}
\fi


\ifllncs\else

\newtheorem{theorem}{Theorem}[section]
\newtheorem{lemma}[theorem]{Lemma}
\newtheorem{definition}[theorem]{Definition}
\newtheorem{example}[theorem]{Example}
\newtheorem{remark}[theorem]{Remark}
\newtheorem{corollary}[theorem]{Corollary}
\newtheorem{proposition}[theorem]{Proposition}
\fi
\newtheorem{fact}[theorem]{Fact}
\newtheorem{setting}[theorem]{Setting}
\newtheorem{construction}[theorem]{Construction}
\newtheorem{Empirical}{Empirical Conclusion}



\newcommand{\bra}[1]{\langle#1|}
\newcommand{\ket}[1]{|#1\rangle}
\newcommand{\bk}[2]{\langle#1|#2\rangle}
\newcommand{\kb}[2]{ \ket{#1}\bra{#2} }

\newcommand{\diag}{\mathrm{diag}  }
\newcommand{\erf}{\mathrm{erf}  }
\newcommand{\gs}[1]{\mathsf{GS} \left( #1\right) }
\newcommand{\ngs}[1]{\mathsf{NGS} \left( #1\right) }


\newcommand{\polylog}{{\sf polylog}}
\newcommand{\poly}{{\sf poly}}
\newcommand{\negl}{{\sf negl}}


\newcommand{\NP}{\mathsf{NP}}
\newcommand{\BQP}{\mathsf{BQP}}
\newcommand{\coAM}{\mathsf{coAM}}
\newcommand{\GAP}{\mathsf{Gap}}
\newcommand{\DGS}{\mathsf{DGS}}
\newcommand{\uSVP}{\mathsf{uSVP}}
\newcommand{\SVP}{\mathsf{SVP}}
\newcommand{\approxSVP}{\mathsf{approx.SVP}}
\newcommand{\SIS}{\mathsf{SIS}}
\newcommand{\SIVP}{\mathsf{SIVP}}
\newcommand{\BDD}{\mathsf{BDD}}
\newcommand{\CVP}{\mathsf{CVP}}
\newcommand{\EDCP}{\mathsf{EDCP}}
\newcommand{\LPN}{\mathsf{LPN}}
\newcommand{\LWE}{\mathsf{LWE}}
\newcommand{\DLWE}{\mathsf{DLWE}}

\newcommand{\zo}{\{0,1\}}
\newcommand{\ir}{\stackrel{\$}{\leftarrow}}
\newcommand{\la}{\leftarrow}
\newcommand{\dist}{ \mathrm{dist} }
\newcommand{\sgn}{ \mathrm{sgn} }

\newcommand{\oFo}{ {}_1{F}_1 }
\newcommand{\tFo}{ {}_2{F}_1 }
\newcommand{\shiind}[5]{ {#1}^{(#5)}_{{#2}, {#4}, {#3}} }
\newcommand{\shind}[4]{ {#1}_{{#2}, {#4}, {#3}} }
\newcommand{\shnd}[3]{ {#1}_{{#2}, {#3}} }
\newcommand{\Ebb}{\mathbb{E}}
\newcommand{\Zp}{\mathbb{Z}_p^\ast}
\newcommand{\pr}{\mbox{Prob}}
\newcommand{\va}{\mathbf{a}}
\newcommand{\expb}[1]{ \exp\left( {#1} \right) }
\newcommand{\expq}[1]{ \omega_q^{#1} }
\newcommand{\Lattice}{L}%
\newcommand{\F}{\mathbb{F}}
\newcommand{\rd}[1]{\left \lfloor #1 \right \rceil}
\newcommand{\upperrounding}[1]{\left \lceil #1 \right \rceil}
\newcommand{\lowerrounding}[1]{\left \lfloor #1 \right \rfloor}
\newcommand{\ipd}[2]{\left \langle {#1}, {#2} \right \rangle}

\newcommand{\mat}[1] { \mathbf{#1} }		%
\newcommand{\ary}[1] { \mathbf{#1} }		%
\newcommand{\pmat}[1]{\begin{pmatrix}#1\end{pmatrix}}

\newcommand{\OOO}{\mathcal{O}}
\newcommand{\C}{\mathbb{C}}
\newcommand{\N}{\mathbb{N}}
\newcommand{\Q}{\mathbb{Q}}
\newcommand{\R}{\mathbb{R}}
\newcommand{\set}[1]{ \left\{ #1 \right\}  }   %
\newcommand{\abs}[1]{ \left| #1 \right|  }   %
\newcommand{\T}{\mathbb{T}}		%
\newcommand{\tr}{\operatorname{tr}}
\newcommand{\Var}{{\mathbb{V}}}
\newcommand{\Z}{\mathbb{Z}}

\newcommand{\QFT}{\mathsf{QFT}}
\newcommand{\HW}{\mathsf{HW}}

\newcommand{\bS}{\mathbb{S}}
\newcommand{\cB}{\mathcal{B}}
\newcommand{\cC}{\mathcal{C}}
\newcommand{\cD}{\mathcal{D}}
\newcommand{\cP}{\mathcal{P}}
\newcommand{\cS}{\mathcal{S}}
\newcommand{\cT}{\mathcal{T}}
\newcommand{\cW}{\mathcal{W}}
\newcommand{\cY}{\mathcal{Y}}

\renewcommand{\l}{\left}
\renewcommand{\r}{\right}
\newcommand{\m}{\middle}

\newcommand{\distribution}{\mathcal{D}}
\newcommand{\datadistribution}{\distribution_{\mathrm{data}}}
\newcommand{\functionclass}{\mathcal{H}}
\newcommand{\dataset}{\mathcal{S}}
\newcommand{\trainingset}{\mathcal{S}_{\mathrm{Train}}}
\newcommand{\boostingset}{\mathcal{S}_{\mathrm{Boost}}}
\newcommand{\trainset}{\trainingset}
\newcommand{\testset}{\mathcal{S}_{\mathrm{Test}}}
\newcommand{\loss}{\ell}
\newcommand{\populoss}{\mathcal{L}_{\distribution}}
\newcommand{\trainloss}{\mathcal{L}_{\mathrm{Train}}}
\newcommand{\testloss}{\mathcal{L}_{\mathrm{Test}}}
\newcommand{\batchloss}{\mathcal{L}_{\mathrm{Batch}}}
\newcommand{\algo}{\mathcal{A}}
\newcommand{\E}{\mathrm{E}}
\newcommand{\nonneg}{\R^{\ge 0}}
\newcommand{\argmin}{\arg \min}
\newcommand{\relu}{\mathrm{ReLU}}
\newcommand{\sigmoid}{\mathrm{Sigmoid}}
\newcommand{\cosine}{\mathrm{Cos}}
\newcommand{\zerooneloss}{\ell_{0-1}}
\newcommand{\logisticloss}{\ell_{\mathrm{log}}}
\newcommand{\maeloss}{\ell_{\mathrm{mae}}}
\newcommand{\mseloss}{\ell_{\mathrm{mse}}}
\newcommand{\one}[1]{\mathrm{1}[#1]}
\newcommand{\model}{\mathcal{M}}
\newcommand{\weight}{W}
\newcommand{\optimizer}{\mathcal{O}}
\newcommand{\sampler}{\mathcal{DS}}
\newcommand{\stopcriterion}{\mathcal{SC}}
\newcommand{\timestopcriterion}{\mathcal{SC}_{\mathrm{time}}}
\newcommand{\stepstopcriterion}{\mathcal{SC}_{\mathrm{step}}}
\newcommand{\accstopcriterion}{\mathcal{SC}_{\mathrm{acc}}}
\newcommand{\regularization}{\mathcal{R}}
\newcommand{\batch}{\mathcal{S}_{\mathrm{Batch}}}
\newcommand{\eps}{\epsilon}
\newcommand{\identity}{\mathcal{I}}
\newcommand{\tw}{\tilde \weight}
\newcommand{\basemodel}{\mathrm{Base}}
\newcommand{\rtime}{t_{\mathrm{time}}}
\newcommand{\step}{t_{\mathrm{step}}}

\algdef{SE}%
[STRUCT]%
{Struct}%
{EndStruct}%
[1]%
{\textbf{struct} \textsc{#1}}%
{\textbf{end struct}}%

\algdef{SE}[VARIABLES]{Variables}{EndVariables}
   {\algorithmicvariables}
   {\algorithmicend\ \algorithmicvariables}
\algnewcommand{\algorithmicvariables}{\textbf{state variables}}

\algdef{SE}[PARAMS]{Params}{EndParams}
   {\algorithmicparams}
   {\algorithmicend\ \algorithmicparams}
\algnewcommand{\algorithmicparams}{\textbf{parameters}}

\begin{document}


\title{Practically Solving LPN in High Noise Regimes Faster \\ Using Neural Networks}


\ifllncs
\titlerunning{Solving LPN Using Neural Networks}
\fi

\author{ 
Haozhe Jiang\thanks{Equal Contribution.}~~\thanks{IIIS, Tsinghua University. Email: \texttt{jianghz20@mails.tsinghua.edu.cn}. } 
\and Kaiyue Wen$^{*}$~\thanks{IIIS, Tsinghua University. Email: \texttt{wenky20@mails.tsinghua.edu.cn}. }
\and Yilei Chen\thanks{IIIS, Tsinghua University, Shanghai Artificial Intelligence Laboratory, and Shanghai Qi Zhi Institute. Email: \texttt{chenyilei@mail.tsinghua.edu.cn}.  
Research supported by Tsinghua University startup funding. }
}

\maketitle

\begin{abstract}
\label{sec:abstract}
We conduct a systematic study of solving the learning parity with noise problem (LPN) using neural networks. Our main contribution is designing families of two-layer neural networks that practically outperform classical algorithms in high-noise, low-dimension regimes. We consider three settings where the numbers of LPN samples are abundant, very limited, and in between. In each setting we provide neural network models that solve LPN as fast as possible. For some settings we are also able to provide theories that explain the rationale of the design of our models. 

Comparing with the previous experiments of Esser, K{\"{u}}bler, and May (CRYPTO 2017), for dimension $n=26$, noise rate $\tau = 0.498$, the ``Guess-then-Gaussian-elimination'' algorithm takes 3.12 days on 64 CPU cores, whereas our neural network algorithm takes 66 minutes on 8 GPUs. Our algorithm can also be plugged into the hybrid algorithms for solving middle or large dimension LPN instances. 
\end{abstract}





\section{Introduction}

The increasing complexity of source code poses a key challenge to the reliability of large-scale software systems. Software bugs in these systems can lead to safety issues~\cite{bug_safety} for users around the world as well as cause non-negligible financial losses~\cite{bug_loss}. As such, developers have to spend a large amount of time and effort on bug fixing. Consequently, \aprfull (\apr), designed to automatically generate patches to fix software bugs, has attracted wide attention from both academia and industry~\cite{long2016prophet, legoues2012genprog, long2015spr, lou2020can, tufano2018empstudy}. 


To achieve \apr, one popular approach is known as Generate-and-Validate (G\&V)~\cite{qi2015gv, ghanbari2019prapr, lou2020can, le2016hdrepair, legoues2012genprog, wen2018capgen, hua2018sketchfix, martinez2016astor, koyuncu2020fixminder, liu2019tbar, liu2019avatar}, which is typically based on the following pipeline: First, fault localization techniques~\cite{wong2016fl, abreu2007ochiai, zhang2013injecting, papadakis2015metallaxis, li2019deepfl, li2017transforming} are applied to determine the suspicious locations in programs where bugs are likely to exist. Then, the buggy locations are used by the \apr tools to generate a list of patches that replace buggy lines with correct lines. Afterward, each patch is validated against the original test suite to identify any \emph{plausible patches} (i.e., passing all tests in the test suite). Finally, to determine the \emph{correct patches}, developers examine the list of plausible patches to see if any of them can correctly fix the bug. 

Traditional \apr tools can mainly be categorized into heuristic-based~\cite{legoues2012genprog, le2016hdrepair, wen2018capgen}, constraint-based~\cite{mechtaev2016angelix, le2017s3, demacro2014nopol, long2015spr} and \template~\cite{ghanbari2019prapr, hua2018sketchfix, martinez2016astor, liu2019tbar, liu2019avatar}. Among these traditional tools, \template \apr tools~\cite{ghanbari2019prapr, liu2019tbar, benton2020effectiveness} have been able to achieve state-of-the-art results. \Template \apr tools typically leverage pre-defined templates (e.g., adding a nullness check) for bug fixing. However, since these fix templates are typically handcrafted, the number and types of bugs they are able to fix can be limited. 



To address the limitations of traditional \apr, researchers have proposed various \learning \apr tools~\cite{li2020dlfix, chen2018sequencer, jiang2021cure, lutellier2020coconut, zhu2021recoder, ye2022rewardrepair} based on the \nmtfull (\nmt) architecture~\cite{sutskever2014mt} where the input is the buggy code snippets and the goal is to translate the buggy code snippets into a fixed version. To accomplish this, \learning \apr tools require supervised training datasets with pairs of both buggy and fixed code snippets in order to learn how to perform this translation step. These training data are usually obtained by mining historical bug fixes using heuristics/keywords~\cite{dallmeier2007benchmark}, which can be imprecise for identifying bug-fixing commits; even the actual bug-fixing commits can include irrelevant code changes, leading to further pollution in the dataset~\cite{xia2022alpharepair}.
% 
Moreover, it can be hard for such \apr tools to generalize and fix bug types unseen during training. 



To better leverage recent advances in \plmfull{s} (\plm{s}), researchers~\cite{xia2022alpharepair, xia2023repairstudy, kolak2022patch, prenner2021codexws} have directly applied \plm{s} to generate patches without bug-fixing datasets. These \llm-based \apr tools work by either directly generating a complete code function~\cite{prenner2021codexws, xia2023repairstudy} or predict/infill the correct code snippet given its surrounding context~\cite{xia2022alpharepair, xia2023repairstudy}. By directly using \llm{s} that are pre-trained on billions of open-source code snippets, \llm-based \apr tools can achieve state-of-the-art performance on many repair datasets~\cite{xia2022alpharepair}. 


% 
%
%

Traditional \apr tools have long used the insight of the \emph{plastic surgery hypothesis}~\cite{barr2014plastic} where it states that the code ingredients to fix a bug already exist within the same project. Traditional \apr tools have manually designed pattern-~\cite{ghanbari2019prapr, saha2017elixir} or heuristic-based~\cite{jiang2018simfix, legoues2012genprog} approaches to finding and using such relevant code ingredients to generate fixes for bugs. However, the plastic surgery hypothesis has been largely ignored in \llm-based \apr. In fact, \llm provides a unique opportunity to fully automate the plastic surgery hypothesis idea via fine-tuning (learning project-specific information via model updates from the buggy project) and prompting (directly providing relevant code ingredients to the model), and make it directly applicable to different languages (since the \llm{s} are typically multi-lingual).%
Moreover, despite the intensive manual efforts involved, traditional \apr tools still cannot fully leverage project-specific information due to large search space for leveraging/composing existing code ingredients. In contrast, the project-specific information can effectively leveraged by \llm{s} due to their power in code understanding/vectorization, e.g., even partial/imprecise information may still guide \llm{s} in correct patch generation!
 To this end, we ask the question: \emph{How useful is the plastic surgery hypothesis in the era of \plm{s}}?








\mypara{Our Work.} To answer the question, we present \ourtech{\xspace} -- a \llm-based approach that automatically utilizes the plastic surgery hypothesis by systematically combining multiple fine-tuning and prompting strategies for \apr. \ourtech fine-tunes \plm{s} using two novel domain-specific training strategies: \textbf{\epfinetune} -- we fine-tune using the original buggy project by aggressively masking out a high percentage of tokens, which allows \plm to learn project-specific code tokens and programming styles; and \textbf{\rofinetune} -- which only masks out a single continuous code sequence per training sample, allowing the model to get used to the final \csapr task of predicting a single continuous code sequence. Furthermore, we directly leverage the ability for \plm{s} to understand natural language instructions and introduce a novel prompting strategy, \textbf{\idprompting}, which uses information retrieval and static analysis to obtain a list of relevant identifiers for the buggy lines. While such relevant identifiers are critical for fixing some difficult bugs, they may not be seen by the \llm during inference due to limited context window size. Through the use of prompting, we directly tell the model to use these extracted identifiers (relevant code ingredients) to generate the correct code. Finally, to perform repair, we combine all four model variants (including the base model, both fine-tuned models and the base model with prompting) for the final repair.





While our insight of leveraging the plastic surgery hypothesis for \llm-based \apr is generalizable across different types of \plm{s}, to implement \ourtech, we choose a recent \plm{\xspace}, \ctfive~\cite{wang2021codet5}, which is pre-trained on millions of open-source code snippets. \ctfive is an encoder-decoder model trained using \mspfull (\msp) objective where a percentage of tokens are masked out and each continuous masked token sequence is referred to as a masked span. Also, although we only extract relevant identifiers from the current buggy project (since this paper focuses on the plastic surgery hypothesis), our work can be easily extended to obtain other code information (such as relevant statements or functions) from other sources, such as  the massive pre-training corpora~\cite{husain2020codesearchnet} or historical bug-fixing datasets~\cite{jiang2019infer}, which can provide more coding knowledge for \llm{s}. Besides, although we mainly focus on using traditional string comparison algorithms for information retrieval in this paper, these techniques can be easily replaced by other frequency-based retrieval~\cite{robertson2009probabilistic} and neural search (or embedding-based search)~\cite{reimers2019sentence}.
  In summary, this paper makes the following contributions:


%


\begin{itemize}[noitemsep, leftmargin=*, topsep=0pt]
    \item \textbf{Dimension.} This paper is the first to revisit the important plastic surgery hypothesis in the era of \llm{s}. It opens up a new dimension for \llm-based \apr to incorporate previously neglected information from the buggy project itself to boost \apr performance. Furthermore, it demonstrates the promising future of retrieval-based prompting for modern \llm-based \apr.
    \item \textbf{Implementation.} We implement \ourtech based on the recent \ctfive model. We augment the model using two novel fine-tuning strategies: \epfinetune and \rofinetune, along with a novel prompting strategy based on information retrieval and static analysis: \idprompting. We combine the patches generated by all four models together and perform patch ranking to speed up \apr.% 
    \item \textbf{Evaluation Study.} We conduct an extensive evaluation against state-of-the-art \apr tools. On the widely studied \dfj 1.2 and 2.0 datasets~\cite{just2014dfj}, \ourtech is able to achieve the new state-of-the-art results of 89 and 44 correct bug fixes (15 and 8 more than best baseline) respectively.  Furthermore, we perform a broad ablation study to justify our design. \ourtech demonstrates for the first time that the plastic surgery hypothesis can substantially boost \llm-based \apr and advance state-of-the-art \apr, while being fully automated and general. Moreover, even partial/imprecise code ingredients may still effectively guide \llm{s} for \apr!
\end{itemize}





\section{Preliminary}\label{sec:prelim}

\paragraph{Notations and terminology.} 
In cryptography, the security parameter is a variable that is used to parameterize the computational complexity of the cryptographic algorithm or protocol, and the adversary's probability of breaking security. 

Let $\R, \Z, \N$ be the set of real numbers, integers, and positive integers. 
For $q\in\N_{\geq 2}$, denote $\Z/q\Z$ by $\Z_q$. 
For $n\in\N$, $[n] := \set{1, ..., n}$. A vector in $\R^n$ (represented in column form by default) is written as a bold lower-case letter, e.g. $\ary{v}$. For a vector $\ary{v}$, the $i^{th}$ component of $\ary{v}$ will be denoted by $v_i$. %
A matrix is written as a bold capital letter, e.g. $\mat{A}$. The $i^{th}$ column vector of $\mat{A}$ is denoted $\ary{a}_i$. 
The length of a vector is the $\ell_p$-norm $\|\ary{v}\|_p := (\sum v_i^p)^{1/p}$, or the infinity norm given by its largest entry $\|\ary v\|_{\infty} := \max_i\{|v_i|\}$. 
The length of a matrix is the norm of its longest column: $\|\mat{A}\|_p := \max_i \|\ary{a}_i\|_p$. 
By default, we use $\ell_2$-norm unless explicitly mentioned. 
For a binary vector $\ary{v}$, let $\HW(\ary{v})$ denote the Hamming weight of $\ary{v}$.
Let $B^n_p$ denote the open unit ball in $\R^n$ in the $\ell_p$ norm.
We will write $x\ e\ y$ as short hands for $x \times 10^y$.

When a variable $v$ is drawn uniformly random from the set $S$ we denote it as $v\la U(S)$. 
When a function $f$ is applied on a set $S$, it means $f(S) := \sum_{x\in S}f(x)$. %




\subsection{Learning Parity with Noise}\label{sec:prelim_coding}


The learning parity with noise problem (LPN) is defined as follows

\begin{definition}[LPN~\cite{DBLP:conf/crypto/BlumFKL93,DBLP:journals/jacm/BlumKW03}] \label{def:lpn}
Let $n \in\N$ be the dimension, $m\in\N$ be the number of samples, $\tau\in(0, 1/2)$ be the error rate. 
Let $\eta_\tau$ be the error distribution that output 1 with probability $\tau$, 0 with probability $1-\tau$. 
A set of $m$ LPN samples is obtained from sampling $\ary{s}\la U(\Z_2^n)$, $\mat{A}\la U(\Z_2^{n\times m})$, $\ary{e}\la\eta_\tau^m$, and outputting $(\mat{A}, \ary{y}^t := \ary{s}^t\mat{A}+\ary{e}^t \mod 2)$.

We say that an algorithm solves $\LPN_{n, m, \tau}$ if it outputs $\ary{s}$ given $\mat{A}$ and $\ary y$ with non-negligible probability. 
\end{definition}


An algorithm solves the decisional version of LPN if it distinguishes the LPN sample $\LPN_{n, m, \tau}$ from random samples over $\Z_2^{n\times m}\times \Z_2^m$ with probability greater than $1/2 + 1/poly(n)$.
The decisional LPN problem is as hard as the search version of LPN~\cite{DBLP:conf/crypto/BlumFKL93}.

The LPN problem reduces to a variant of LPN where the secret is sampled from the error distribution~\cite{DBLP:conf/crypto/ApplebaumCPS09}. The reduction is simple and important for our application so we sketch the theorem statement and the proof here. 

\begin{lemma}
\label{lem:s_sparse}
If $\LPN_{n, m, \tau}$ is hard, then so is the following variant of LPN: we sample each coordinate of the secret $\ary{s}\in\Z_q^n$ from the same distribution as the error distribution, i.e., $\eta_\tau$, and then output $m-n$ LPN samples. 
\end{lemma}
\begin{proof}
Given $m$ standard LPN samples, denoted as $(\mat{A}, \ary{y}^t := \ary{s}^t\mat{A}+\ary{e}^t \mod 2)$. 
Write $\mat{A} = [\mat{A}_1 \mid \mat{A}_2]$ where $\mat{A}_1\in\Z_2^{n\times n}$. Without a loss of generality, assume $\mat{A}_1$ is invertible (if not, pick another block of $n$ full-rank columns from $\mat{A}$ as $\mat{A}_1$). 
Write $\ary{y}^t = [\ary{y}^t_1 \mid \ary{y}^t_2]$ where $\ary{y}_1\in\Z_2^n$.  
Let $\bar{\mat{A}}:= -\mat{A}_1^{-1}\cdot \mat{A}_2$. 
Let $\bar{\ary{y}}^t := \ary{y}^t_1\cdot \bar{\mat{A}} + \ary{y}^t_2$. Then 
$\bar{\ary{y}}^t = (\ary{s}^t\mat{A}_1+\ary{e}^t_1)\cdot (-\mat{A}_1^{-1}\cdot \mat{A}_2) + (\ary{s}^t\mat{A}_2+\ary{e}^t_2) = \ary{e}^t_1\cdot \bar{\mat{A}} + \ary{e}^t_2$, meaning that
$\bar{\mat{A}}, \bar{\ary{y}}^t$ is composed of $m-n$ LPN samples where the secret is sampled from the error distribution.
\end{proof}


\subsection{Machine Learning}

\paragraph{Supervised Learning}The goal of supervised learning is to learn a function that maps inputs to labels. The input $x \in \mathcal{X}$ and the label $y  \in \mathcal{Y}$ are usually assumed to obey a fixed distribution $\distribution$ over $\mathcal{X} \times \mathcal{Y}$. Usually, $\distribution$ is not directly accessible to the learner, instead, another distribution $\datadistribution$, known as empirical distribution, is provided to the learner. This distribution is usually a uniform distribution over a finite set of inputs and labels $\trainingset \triangleq \{ (x_i, y_i)\}_{i \in [1:N]}$. This set $\trainingset$ is usually named \textit{training set} and $(x_i,y_i)$ is assumed to obey $\distribution$ independently.



The goal of \textit{learning} is to choose from a function class $\functionclass$ a function $f: X \to Y$  given $\datadistribution$. To measure the quality of $f$, \textit{loss function} $\ell: Y \times Y \to \nonneg$ is often considered. We now provide some examples of loss functions that will be used in our paper.

\begin{definition}[Zero-one Loss]
    $\zerooneloss(y_1, y_2) = \one{y_1 \neq y_2}$.
\end{definition}

\begin{definition}[Logistic Loss]
\label{def:logistic}
    $\logisticloss(y_1, y_2) = -y_2 \log( 1 - y_1) - (1 - y_2) \log y_1, y_2 \in \{0,1\},y_1 \in [0,1]$.
\end{definition}


\begin{definition}[Mean Absolute Error Loss]
    $\maeloss (y_1, y_2) = |y_1 - y_2|$.
\end{definition}

\begin{definition}[Mean Square Error Loss]
    $\mseloss (y_1, y_2) = |y_1 - y_2|^2$.
\end{definition}

\begin{definition}
    Given a loss function $\ell: Y \times Y \to \nonneg$, the \emph{population loss} $\populoss: \functionclass \to \nonneg$  is defined as
    \begin{align*}
        \populoss(f) = \E_{(x,y) \sim \distribution}[\ell(f(x), y)].
    \end{align*}
\end{definition}

For zero-one loss, $1$ minus the expected loss is also called \textit{accuracy}. We usually abuse this notation when $f$'s co-domain is $[0,1]$ by calling the accuracy of the rounding of $f$ as the accuracy of $f$. The \emph{training accuracy} is defined as accuracy with the underlying population as the uniform distribution over the training set.
The goal of learning can then be rephrased to find $f \in \functionclass$ with low population loss. Two questions then naturally arise, (1) How to evaluate population loss? and (2) How to effectively minimize population loss?

To evaluate the loss, the \textit{test set} $\testset \triangleq \{(x_i, y_i) \}_{i \in [N+1, N+M]}$ is usually considered. The element in $\testset$ is also assumed to obey $\distribution$ independently and is also independent of the elements in $\trainingset$. 
\begin{definition}
    Given a loss function $\ell: Y \times Y \to \nonneg$, the \emph{test loss} $\testloss: \functionclass \to \nonneg$  is defined as
    \begin{align*}
        \testloss(f) = \frac{1}{M}\sum_{i = N+1}^{N+M} \ell(f(x_i), y_i).
    \end{align*}
    The \emph{test accuracy} is defined as accuracy with the underlying population as the uniform distribution over the test set.
\end{definition}
When $f$ is chosen by the learning algorithm given the training data, $\testloss(f)$ can then serve as an unbiased estimator of $\populoss(f)$. In the traditional machine learning community, $\testloss$ is usually only measured once after training, and another set called \textit{validation set} is used to track the performance of the algorithm through the course of the training. However, this boundary is blurred in modern literature and we ignore this subtlety here because our final objective is to utilize machine learning to solve LPN secrets instead of fitting the data. 

To effectively minimize the loss, the learner would use a learning algorithm $\algo$ that maps training distribution to a function $f \in H$ (usually with randomness). As the learner only has access to the data distribution, $\algo$ is usually designed to minimize \emph{training loss}.
\begin{definition}
    Given a loss function $\ell: Y \times Y \to \nonneg$, the \emph{training loss} $\trainloss: \functionclass \to \nonneg$  is defined as
    \begin{align*}
        \trainloss(f) = \frac{1}{M}\sum_{i = 1}^{M} \ell(f(x_i), y_i).
    \end{align*}
\end{definition}


When trying to characterize the gap between the learned function and the best available function in the function class, the following decomposition is common in machine learning literature.
\begin{align}
&\populoss(\algo(\datadistribution)) - \underbrace{\min_{f \in \functionclass} \populoss(f)}_{\mathrm{Representation\ Gap}} \notag\\
=&\underbrace{\populoss(\algo(\datadistribution)) - \trainloss(\algo(\datadistribution))}_{\mathrm{Generalization\ Gap}} + \underbrace{\trainloss(\algo(\datadistribution)) - \trainloss(f^*) }_{\mathrm{Optimization\ Gap}} \notag \\&+ \underbrace{ \trainloss(f^*)  - \populoss(f^*)}_{\mathrm{Stochastic\ Error}}, \quad f^* = \argmin_{f \in \functionclass} \populoss(f). \label{eq:decompose}
\end{align}

The three gaps in the above equation characterize different aspects of machine learning. 
\begin{enumerate}
    \item The function class needs to be chosen to be general enough to minimize the representation gap.
    \item The learning algorithm needs to be chosen to find the best trade-off between the generalization gap and the optimization gap given the function class.
\end{enumerate}
In the recent revolution brought by neural networks, it is shown that choosing the function class as $\{ f \mid f \text{ can be represented by a fixed neural architecture}\}$ and learning algorithm as the gradient-based optimization method can have surprising effects over various domains. We will now briefly introduce neural networks and gradient-based optimization algorithms.

\paragraph{Neural Networks}

Neural networks are defined by \emph{architecture}, which maps differentiable weights to a function from $\mathcal{X}$ to $\mathcal{Y}$. This function is called the \emph{neural network} and the weights are called the \emph{parameterization} of the network. The most simple architecture is \emph{Multi-Layer Perceptron (MLP)}.
\begin{definition}[MLP]
\label{def:mlp}
    Multi-Layer Perceptron is defined as a mapping $\model$ from $\R^{d_1}$ to $\R^{d_{L+1}}$, with
    \begin{align*}
        \model[\theta_1, ..., \theta_L](x) = (\sigma_L \circ T[\theta_L] \circ  ... \circ \sigma_1 \circ T[\theta_1])(x),
    \end{align*}
    where $\theta_i = (W_i, b_i), W_i \in \R^{d_{i+1} \times d_{i}}, b_i \in \R^{d_{i+1}}$ and $T[\theta_i]$ is an affine function with $T[\theta_i](x) = W_ix + b_i$. $\sigma_i: \R^d \to \R^d $ is a function that is applied \emph{coordinate-wise} and is called \emph{activation function}. $L$ and $L - 1$ are called the number of \emph{layers} and \emph{depth} of the MLP, and $\{ d_i \}_{i = 2,..,L}$ is  called the \emph{widths} of the MLP.
\end{definition}

We now provide some examples of activation functions that will be used in our work.

\begin{definition}[ReLU]
$\relu: \R^d \to \R^d$ is defined as $(\relu(x))_i = x_i \one{x_i \ge 0} $.
\end{definition}


\begin{definition}[Sigmoid]
$\sigmoid: \R^d \to \R^d$ is defined as $(\sigmoid(x))_i = \frac{1}{1 + e^{-x_i}} $.  
\end{definition}

\begin{definition}[Cosine]
$\cosine: \R^d \to \R^d$ is defined as $(\cosine(x))_i = \cos(x_i) $.  
\end{definition}

The base model we used in this work is defined as followed.
\begin{definition}[Base Model]
\label{def:basemodel}
    Our base model is defined as MLP with depth $1$ with activation $\sigma_1 = \relu$ and $\sigma_2 = \sigmoid$. We will denote this model as $\basemodel_{d}$ with $d$ specifying $d_2$. We will write $\basemodel$ with activation $\sigma$ to indicate replacing $\sigma_1$ by $\sigma$. 
\end{definition}


\paragraph{Gradient-based Optimization}

It is common to use gradient-based optimization methods to optimize the neural network. A standard template is shown in~\Cref{alg:optim_example}. The unspecified parameters such as $\model, \weight$, and $\ell$ in the algorithm are often called \emph{hyperparameters}. 

\begin{algorithm}
\caption{Gradient-based Optimization}\label{alg:optim_example}
\begin{algorithmic}
\Require $\text{A neural network architecture }\model$
\Require $\text{An initialization parameter for the model }\weight_0$
\Require $\text{A differentiable loss function } \ell$
\Require $\text{A Stop Criterion } \stopcriterion$ 
\Require $\text{A Data Sampler } \sampler$ \Comment{See~\Cref{def:sampler}}
\Require $\text{An Optimizer } \optimizer$ \Comment{See~\Cref{def:optimizer}}
\Require $\text{A Regularization Function } \regularization$  \Comment{See~\Cref{def:regularization}}
\State $step \gets 0$
\While{$\stopcriterion$ \text{ is not reached}} \Comment{See~\Cref{def:stop_criterion}}
\State $f \gets \model[\weight_{step}]$
\State $\batch = \{(x_i, y_i)\}_{i = 1,...,B} \gets \sampler.GetData()$.
\State $\batchloss \gets \frac{1}{B} \sum_{i = 1}^B l(f(x_i), y_i) + \regularization(\weight_{step})$.  \Comment{Calculate regularized loss}
\State  $g_W \gets -\frac{\partial \batchloss}{\partial W} \mid_{W = W_{step}}$. \Comment{Calculate gradient w.r.t the model parameter}
\State $W_{step + 1} \gets \optimizer.GetUpdate(W_{step}, g_W)$.
\State $step \gets step + 1$.
\EndWhile
\State \Return $\model[\weight_{step}]$
\end{algorithmic}
\end{algorithm}

\begin{definition}[Sampler]
\label{def:sampler} 
A \emph{sampler} is a finite-state machine, on each call of method $GetData$, it will return a set of $B$ samples $\{(x_i, y_i)\}$ satisfying $x_i,y_i \sim \datadistribution$. The number $B$ is called \emph{batch size} and the $B$ samples are called a \emph{batch}.
\end{definition}

We hereby provide two examples of samplers that will be used in our papers. \Cref{alg:fix_sample} is the sampler used in Settings~\ref{setting:restricted} and~\ref{setting:moderate} and \Cref{alg:inf_sample} is the sampler used in \Cref{setting:abundant}.

\begin{algorithm}
\caption{Batch Sampler}\label{alg:fix_sample}
\begin{algorithmic}
\Params
 \State $\trainset = \{(x_i, y_i)\}_{i = 1,..,N}$, training set
\State $B \le N$, batch size
\EndParams
\Procedure{GetData}{$\ $}
\State Sample i.i.d from $\{1,2,..,N\}$ $B$ index to form index set $I_{\text{Batch}}$
\State $\batch = \{ (x_i, y_i) \mid i \in  I_{\text{Batch}}\}$.
\State \Return $\batch$.
 \EndProcedure
\end{algorithmic}
\end{algorithm}



\begin{algorithm}
\caption{Oracle Sampler}\label{alg:inf_sample}
\begin{algorithmic}
\Params
 \State $\distribution$, the underlying distribution
\State $B \le N$, batch size
\EndParams
\Procedure{GetData}{$\ $}
\State $\batch = \{ (x_{k}, y_{k}) \}_{k = 1...B}$ with $(x_k, y_k)$ i.i.d sampled from $\distribution$.
\State \Return $\batch$.
 \EndProcedure
\end{algorithmic}
\end{algorithm}


\begin{definition}[Optimizer]
\label{def:optimizer}
An \emph{optimizer} is an automaton, on each call of method $GetUpdate$, it will update states given the current parameter and gradient, and return an updated parameter.
\end{definition}

We now provide some examples of optimizers. The \emph{stochastic gradient descent (SGD) Optimizer} is shown in \Cref{alg:sgd}. When the batch size equals the size of the training set, this algorithm is often called \emph{gradient descent} directly. In our paper, we use a more complicated optimizer named \emph{Adam}, as shown in \Cref{alg:adam}. This optimizer, although poorly understood theoretically, has been widely applied across domains by the current machine-learning community.


\begin{algorithm}[h]
\caption{SGD Optimizer}\label{alg:sgd}
\begin{algorithmic}
\Params
 \State $\eta$, learning rate
 \State $\lambda$, weight decay
\EndParams
\Procedure{GetUpdate}{$W, g_W$}
\State \Return $W - \eta(\lambda W +  g_W)$.
 \EndProcedure
\end{algorithmic}
\end{algorithm}

\begin{algorithm}[t]
\caption{Adam Optimizer}\label{alg:adam}
\begin{algorithmic}
\Variables
\State $m$, first moment, initialized to be $0$.
\State $v$, second moment, initialized to be $0$.
\State $\eps$, a small positive constant, by default $1e-8$
\EndVariables
\Params
 \State $\eta$, learning rate
 \State $\lambda$, weight decay
 \State $\beta_1$, $\beta_2$, moving average factor for moments, by default $0.9, 0.999$.
\EndParams
\Procedure{GetUpdate}{$W, g_W$}
\State  $dW \gets \lambda W + g_W$.
\State  $m \gets \beta_1 m + (1 - \beta_1)dW$.
\State  $v \gets \beta_2 m + (1 - \beta_2)dW^2$.
\State  $\hat m \gets m / (1 - \beta_1)$.
\State  $\hat v \gets v / (1 - \beta_2)$.
\State  \Return $W - \eta \hat m / (\sqrt{\hat v} + \eps)$
 \EndProcedure
\end{algorithmic}
\end{algorithm}

\begin{definition}[Regularization]
\label{def:regularization}
A \emph{regularization function} is defined as a mapping from the parameter space to $\R$. 
\end{definition}

Theoretically and empirically, a proper choice of a regularization function can improve the generalization of the learned model in previous literature.
We hereby provide two examples of regularization functions that will be used in our paper. 
One can easily notice that L2 regularization (\Cref{def:l2}) applied in the gradient-based optimization method is simply another form of weight decay.
L1 regularization (\Cref{def:l1}) applied with the linear model is known as \emph{LASSO} and can induce sparsity in the model parameters (meaning the model parameters contain more zeroes).

\begin{definition}[L2 Regularization]
\label{def:l2}
    L2 Regularization $R_2: (\R^{d_1},... ,\R^{d_k}) \to \R$ with penalty factor $\lambda$ is defined as $R_2(w_1, ..., w_d) = \frac{\lambda}{2} \sum_i \|w_i\|_2^2$.
\end{definition}

\begin{definition}[L1 Regularization]
\label{def:l1}
    L1 Regularization $R_1: (\R^{d_1},... ,\R^{d_k}) \to \R$ with penalty factor $\lambda$ is defined as $R_1(w_1, ..., w_d) = \lambda \sum_i \|w_i\|_1$.
\end{definition}

\begin{definition}[Stop Criterion]
\label{def:stop_criterion}
A \emph{stop criterion} is defined as a function that returns True or False determining whether the procedure should terminate.
\end{definition}
There are typically three kinds of stop criteria, which we list as below.

\begin{definition}[Stop-by-time]
\label{def:stop_time}
A stop-by-time criterion $\timestopcriterion(\rtime)$ returns true if physical running time exceeds threshold $t$.
\end{definition}

\begin{definition}[Stop-by-step]
\label{def:stop_step}
A stop-by-step criterion $\stepstopcriterion(\step)$ returns true if the weight update step exceeds threshold $T$.
\end{definition}

\begin{definition}[Stop-by-accuracy]
\label{def:stop_acc}
A stop-by-accuracy criterion $\accstopcriterion(\dataset, \gamma)$ returns true if the accuracy of the learned function on $\dataset$ exceeds threshold $\gamma$.
\end{definition}

\section{Method}
\label{sec:method}

% \ml{``Inconsistent'' to ``large variation''}

% In this section, we propose our methods based on the observations in Section \ref{sec:motivation}.
In this section, we propose two techniques to further enhance the strong baseline to capture the variation of activation distributions better.
We first introduce spatial re-scaling to adapt the network to pixel-to-pixel variation.
We then propose channel-wise shifting and re-scaling to better capture the channel-to-channel variation.
Meanwhile, as both of the two methods are image-dependent, the image-to-image variation can be captured naturally.
By combining the two methods with our strong baseline, we build our enhanced BNN for SR, named EBSR.

% Because the activation distributions among pixels, channels and images have large variations \red{**are highly inconsistent} in SR networks, we introduce spatial re-scaling to adapt to pixel-wise variations and channel shift and re-scaling to adapt to channel-wise variations. And both of them are image-dependent to adapt to image-wise variations, which means during inference our network re-scales and shifts the distributions of activations flexibly for different input images. Based on these methods, we build an enhanced binary neural network for image super-resolution (EBSR).

% According to [3], the difference of activation magnitudes indicates different scaling factors are needed for each pixel.

\subsection{Spatial Re-scaling}
% It is better to use different scaling factors for different pixels to reduce the quantization error and retain more detailed information for image super-resolution. 

% \ml{In the main method, we do not need to introduce the previous works but can focus on introducing our own method. Channel rescaling in Real-to-binary Net is not relevant in this context.}

% Re-scaling the output of binary convolutions was proposed at the birth of BNN in XNOR-Net \cite{rastegari2016xnor} to reduce quantization error and improve accuracy for image classification tasks.
% It is computed as below:
% \begin{equation}
% \mathcal{A} * \mathcal{W} \approx(\operatorname{sign}(\mathcal{A}) \circledast \operatorname{sign}(\mathcal{W})) \odot \mathcal{K} \alpha
% \label{eq:xnor-net rescale}
% \end{equation}
% where $\circledast$ denotes the binary convolution and $\odot$ denotes the element-wise multiplication.
% $\mathcal{A}$, $\mathcal{W}$, $\alpha$, and $\mathcal{K}$ denote the activation, weight, weight scaling factor, and activation scaling factor, respectively.
%  Later in XNOR-Net++ \cite{bulat2019xnor}, Bulat et al. fuse the activation and weight scaling factors into a single one that is learned end-to-end based on gradients and this improves the classification accuracy on ImageNet dataset.

% % It is computed as Eq.~\ref{eq:xnor-net rescale}, where $\circledast$ denotes 
% %  the binary convolution and $\odot$ denotes the element-wise multiplication. The binary convolution of $\mathcal{A}$ and $\mathcal{W}$ is rescaled by the weight scaling factor $\alpha$ and the activation scaling factor $\mathcal{K}$, both of which are calculated analytically.


% \zc{Similarly, you should explain the meaning of A, W and the operators $\circledast$ in the formula}
% Then in Real-to-binary Net \cite{martinez2020training}, Martinez et al. used a data-driven channel re-scaling module that takes the pre-convolution activations as input to predict the activation scaling factor. Unlike that in XNOR-Net++ \cite{bulat2019xnor}, these scaling factors are not fixed during inference but rather inferred from data. By doing this, they further improved the classification accuracy on ImageNet over XNOR-Net++. 
As is shown in Figure \ref{fig:pixel}, activation distributions have large pixel-to-pixel variation in SR networks
and the difference of activation magnitudes indicates different scaling factors are preferred for different pixels.
Inspired by \cite{martinez2020training}, we propose spatial re-scaling to better adapt the network to the spatial variation
of activation distributions in SR networks.
% fit the various pixel-wise distributions in SR networks.
We take the real-valued activations $A$ before convolution as input and predict pixel-wise scaling factors $S(A)$, which re-scale the binary convolution output. Spatial re-scaling process can be formulated as follows:
\begin{equation}
A * W \approx(\operatorname{sign}(A) \circledast \operatorname{sign}(W)) \odot \alpha \odot S(A)
\label{eq:spatial rescale}
\end{equation}
where $\circledast$ denotes 
the binary convolution and $\odot$ denotes the element-wise multiplication. $A$, $W$, $\alpha$, and $S\left(A\right)$ denote real-valued activations, weights, the scaling factor of weights, and the spatial-wise scaling factor of activations respectively. $S\left(A\right) \in \mathbb{R}^{1\times H\times W}$ can be calculated with a convolution and a sigmoid function.
% as $\sigma\left( CONV\left(A\right)\right)$. 
As shown in Figure \ref{fig:method}(a), real-valued activations first go through a convolution layer,
which has an input channel of $C$ and an output channel of 1, 
and then pass through a sigmoid function to produce the scaling factors $S(A)$ along the spatial dimension.
During inference, the scaling factor will change dynamically according to different input feature maps.
By re-scaling binary convolution output using $S(A)$, we can reduce the quantization error and the original pixel-wise information in FP activation
will be preserved much better.
Spatial re-scaling leads to a large PSNR improvement of 0.24 dB (from 30.30 dB to 31.54 dB) on Set5 and 0.22 dB (from 25.09 dB to 25.31 dB)
on Urban100 compared with our strong baseline. 

\subsection{Channel-wise Shifting and Re-scaling}

\begin{table}[!tb]
\centering
\caption{Comparison between whether to fuse channel-wise shifting and re-scaling or not based on our baseline with spatial re-scaling. }
\label{tab:fusing}

\scalebox{0.65}{
\begin{tabular}{c|cc|cc|cc}
\hline
\multirow{2}{*}{Method}     & \multirow{2}{*}{OPs} & \multirow{2}{*}{Params} & \multicolumn{2}{c|}{Set5} & \multicolumn{2}{c}{Urban100} \\ \cline{4-7} 
                            &                      &                         & PSNR        & SSIM        & PSNR          & SSIM         \\ \hline
Baseline + spatial re-scale & 2.16G                & 0.05M                   & 31.54       & 0.883       & 25.31         & 0.759        \\
+ channel-wise shift and re-scale             & 2.34G                & 0.09M                   & 31.61       & 0.885       & 25.35         & 0.761        \\
+ w/ fusing                   & 2.27G                & 0.08M                   & \textbf{31.64}       & \textbf{0.885}       & \textbf{25.36}         & \textbf{0.761}        \\ \hline
\end{tabular}
}
\end{table}

In SR networks, activation distributions exhibit larger channel-to-channel variation (Figure \ref{fig:chl}).
Both the mean and magnitude of the activation distributions vary significantly across channels.
% Thus we use channel-wise shifting and re-scaling to adapt to various channel-wise distributions. 
\cite{martinez2020training} has proposed the data-driven channel re-scaling, 
but our method differs from them in further introducing data-driven thresholds to handle the channel-wise variation of both mean and magnitude.
Since the blocks to generate the scaling factors and thresholds are very similar, we further propose to fuse them into one module.
% and fusing channel-wise shifting and re-scaling into one module.
We evaluate the effect of fusing the two blocks in Table \ref{tab:fusing}.
With channel-wise shifting and re-scaling fused, our models have fewer operations and parameters overhead and slightly higher performance.

For the specific process, we take the real-valued activations as input and predict different thresholds and scaling factors for each channel. They are also image dependent, e.g., $\beta_{i}$ in Eq.\ref{eq:act_binarize} is no longer fixed during inference but generated according to different input feature maps. Channel-wise shifting and re-scaling can be formulated as follows:
\begin{equation}
A * W \approx(\operatorname{sign}(A-C_s(A)) \circledast \operatorname{sign}(W)) \odot \alpha \odot C_r(A)
\label{eq:channel-wise_shift_and_rescale}
\end{equation}
where $\circledast$ denotes 
the binary convolution and $\odot$ denotes the element-wise multiplication. $C_s(A), C_r(A) \in \mathbb{R}^{C\times1\times1}$ denote the channel-wise threshold and scaling factor, respectively. 
We show the block diagram in Figure \ref{fig:method}(b).
The real-valued input feature map is first squeezed to a ${C\times1\times1}$ vector by a global average pooling (GAP) layer.
The subsequent fully connected layers and ReLU learn the channel-wise information and output a ${2C\times1\times1}$ vector.
Then the ${2C\times1\times1}$ vector is split into two ${C\times1\times1}$ vectors.
We use the first $C$ channels as the channel-wise bias and pass the last $C$ channels through a sigmoid layer 
as the channel-wise scaling factor, which are used to shift the real-valued activations and re-scale the binary convolution output, respectively. 


% \ml{We can mention previously, channel-wise re-scale has been proposed. We propose to fuse them. Add the comparison between fuse v.s. no fuse.}

\begin{figure}[!tbp]%
  \centering
    \includegraphics[width=0.4\textwidth]{fig/methods.png}
  
% \subfloat[channel-wise shifting\&re-scale]{
%     \label{subfig:channel-wise shifting and re-scale}
%     \includegraphics[width=0.2\textwidth]{fig/chl shift and rescale.png}
%   }

  \caption{Block diagram for spatial re-scaling, and channel-wise shifting and re-scaling.} 
  % Input A is the real-valued activation tensor and C, H, and W denote its dimension. GAP stands for global average pooling. The reduction ratio r is set to 16 for a better trade-off between the performance and the number of operations and parameters.}
  \label{fig:method}
\end{figure}


\subsection{Network Structure}

Combining the spatial re-scaling and the channel-wise shifting and re-scaling methods, we construct the enhanced convolution layer (E-Conv).
Then we build our EBSR model based on E-Conv.
In Figure \ref{fig:E-conv}, we compare the binary convolution layer used in the baseline network and our proposed E-Conv.
We use spatial and channel-wise scaling factors to re-scale the binary convolution output,
and use channel-wise shifting to learn appropriate thresholds for each channel before binarization.
The scaling factors and threshold used in E-Conv are learnable and depend on the real-valued input activations.
In this way, our proposed EBSR can adapt to pixel-to-pixel, channel-to-channel, and image-to-image variations
to reduce the large binarization error and preserve more details.
% In this way, our proposed E-Conv reduces the large quantization error caused by binarization and keeps the original information of input feature maps to a large extent.


\begin{figure}[!tb]%
  \centering

    \includegraphics[width=0.5\textwidth]{fig/E-conv.png}

  \caption{Comparison of (a) the binary convolution layer with a skip connection used in our baseline network and (b) the proposed E-Conv.}
  \label{fig:E-conv}
\end{figure}


Figure \ref{fig:network} shows the basic block based on the E-Conv and our EBSR composed of the basic blocks. Following existing works, the convolution layers in the head and tail modules are not binarized. We choose the lightweight EDSR which has 16 basic blocks and 64 channels, and EDSR which has 32 basic blocks and 256 channels as our backbones, which correspond to EBSR-light and EBSR, respectively.

\begin{figure}[!tb]%
  \centering
  {
    \includegraphics[width=0.35\textwidth]{fig/network.png}
  }
  
  \caption{The structure of our proposed EBSR.  Convolution layers in purple are real-valued vanilla 3x3 convolutions.}
  \label{fig:network}
\end{figure}

\section{Experimental Results}
\label{sec:experiments}
\subsection{Training Details}
\cite{Kalantari2017DeepHD} provides the first dataset specifically designed for multi-exposure HDR fusion under large motion. It consists of 74 training sets, which we use to supervise the training of our model. We crop the input images to patches of size \(256 \times 256\) at a step size of 64. This totally generates 20128 training samples. To augment training samples, we randomly rotate and flip the training images. The training adopts Adam optimizer. The learning rate is initialized to \(10^{-4}\) and is reduced to \(10^{-5}\) after 20 epochs. It is observed that 40 epochs are sufficient for the training to converge.    

\subsection{Numerical Evaluation}
We numerically measure the performance of our method on the 15 test sets of \cite{Kalantari2017DeepHD}, by Peak Signal-to-Noise Ratio (PSNR) and Structure Similarity, computed in both tonemapping domain (-\(\mu\)) and HDR linear domain (-L). Visual difference metric HDR-VDP-2 is also adopted, where the parameters are set as same as in previous works \cite{wu2018end} and \cite{niu2021hdrgan}. 

Table \ref{table_metrics} compares our model with state-of-the-art models. For \cite{yan2020nonlocal} and \cite{xiong2021hierarchical}, we use the results reported in the publications. Note that \cite{sen2012robust} and \cite{hu2013hdr} are not machine learning based methods. Moreover,  \cite{Kalantari2017DeepHD} and \cite{wu2018end} apply optical flow and homography transformation to preprocess the input images respectively, and hence entail extra computation. 

Table \ref{table_metrics} shows that our method outperforms competing method in terms of PSNR-L, SSIM-$\mu$, SSIM-L and HDR-VDP-2. It ranks the second best in PSNR-$\mu$, being slightly (0.1dB) inferior to \cite{xiong2021hierarchical}. Note that \cite{xiong2021hierarchical} utilizes a pretrained model to detect ghosting regions for training, whereas our method does not require any pretrained model. The high PSNR and SSIM scores varify that our model has strong HDR reconstruction ability and can accurately restore the radiance and structure of the scene in both tonemapping domain and HDR linear domain. Furthermore, its high performance in term of HDR-VDP-2\cite{mantiuk2011hdr} performance indicates that our method can generate HDR image visually close to the target image.

\begin{table*}[ht]
\centering
\begin{tabular}{l|c|c|c|c|c}
\hline
& PSNR-$\mu$ & PSNR-L & SSIM-$\mu$ & SSIM-L & HDR-VDP-2 \\
\hline
\bfseries Sen & 40.97 & 38.36 & 0.9830 & 0.9746 & 60.60\\
\hline
\bfseries Hu  & 35.65 & 30.80 & 0.9725 & 0.9491 & 58.34\\
\hline
\bfseries Kalantari & 42.69 & 41.22 & 0.9888 & 0.9845 & 65.05\\
\hline
\bfseries DeepHDR& 41.99 & 41.22 & 0.9878 & 0.9859 & \underline{65.91}\\
\hline
\bfseries AHDR & 43.62 & 41.03 & 0.9900  &\underline{0.9883} & 63.85 \\
\hline 
\bfseries NHDRRNet& 42.414 & - & 0.9887 & - & 61.21 \\
\hline 
\bfseries HDR-GAN &43.92 & \underline{41.57} &\underline{0.9905} &0.9865 & 65.45\\
\hline 
\bfseries HFNet & \textbf{44.28} & 41.47 & - & - & - \\
\hline 
\bfseries Ours & \underline{44.18} & \textbf{42.19}&\textbf{0.9912} & \textbf{0.9883}& \textbf{67.07} \\
\hline
\end{tabular}
\caption{Numerical performance of the proposed model, evaluated on the dataset by Kalantari-Ramamoorthi. The best and second best results for each metric are marked in \textbf{bold} and \underline{underlined}, respectively}
\label{table_metrics}
\end{table*}

\subsection{Visual Performance Evaluation}

\begin{figure*}[!htb]
\centering
\includegraphics[width=\textwidth]{experiments/kalantari_test.png}
\caption{Visual comparison on the test set of Kalantari-Ramamoorthi dataset. Zoom-in views of reconstruction by each method are presented on the saturated regions that contain moving objects. Our network built with gated Swin Transformer yields noticeably better visual results than other methods.}
\label{fig_kalantari_test}
\end{figure*}
Fig. \ref{fig_kalantari_test} present the visual performance of our method and comparable methods on two examples from \cite{Kalantari2017DeepHD}. We present the zoom-in views of two challenging cases, where large saturated regions contain substantial non-rigid motion in the reference image. The two patch-based methods do not reconstruct the missing details in the saturated regions, as they heavily rely on the details provided by the reference image for registration. Image reconstructed by the optical flow based method \cite{Kalantari2017DeepHD} suffers motion blur artifacts. This is because the convolutions of DeepHDR and HDR-GAN have limited receptive fields, and hence are hampered to repair missing content in misaligned regions by aligned regions. The gating mechanism of AHDR is only applied to low-level features, so the high-level outliers may deteriorate the HDR fusion. In contrast to comparable methods, our model remarkably overcomes the ghosting artifacts.

\begin{figure}[ht]
\centering
\includegraphics[width=\columnwidth]{experiments/sen_test.pdf}
\caption{Visual performance comparison on example images from the dataset by Sen et al. Zoom in views on challenging areas are presented. Although the ground truth is unavailable, it can be clearly observed that our method visually performs better than comparable methods.}
\label{sen_test}
\end{figure}

\begin{figure}[ht]
\centering
\includegraphics[width=\columnwidth]{experiments/tursun_test.pdf}
\caption{Visual performance comparison on example images from the dataset by Tursun et al. Compared to state of the art methods, our method suffers less ghosting artifact.}
\label{tursun_test}
\end{figure}

Fig.\ref{sen_test} and Fig.\ref{tursun_test} present visual performance of our method on two examples from benchmark datasets \cite{sen2012robust} and \cite{tursun2016objective}. As these test datasets   do not provide ground truth image. we mark the visual difference on the results generated by different methods. It can be seen that our method suffers less artifacts than other methods in various scenes with various motion patterns, achieving better visual results. Our method creates high-quality HDR more robustly and generalizes well. 

\subsection{Ablation Study}

\begin{table}[h]
\centering
\resizebox{\columnwidth}{!}{
\begin{tabular}{l|c|c|c|c|c}
\hline
                         & PSNR-$\mu$ & PSNR-l & SSIM-$\mu$ & SSIM-l & HDR-VDP-2 \\ \hline
restormer(w/o ssim loss) & 44.00  & 41.5   & 0.9906 & 0.9873 & 64.72  \\ \hline
Ours(w/o ssim loss)      & 44.07  & 41.83  & 0.9909 & 0.9879 &  64.78  \\ \hline
Ours                     & 44.18  & 42.19  & 0.9912 & 0.9883 & 67.07      \\ \hline
\end{tabular}
}
\caption{Experimental results of ablation study. We compare using Gated Swin Transformer v.s. Gated Transformer, and the combined loss function v.s. the traditional $l_{1}$ norm loss function.}
\label{table_ablation_block_loss}
\end{table}

We verify various components of our method, including Swin Transformer, loss function, and gating mechanism by ablation study.

\subsubsection{Ablation Study on Block Design}
Our model has similar architecture to Restormer, which uses modified Transformer, whereas we use modified Swin Transformer as the building unit. For comparison, we replace the residual modules in each block in our model with multiple transformer layers as in Restormer, with same number of transformer layers. Table \ref{table_ablation_block_loss} presents the results, which show that using Swin Transformer achieves superior performance in all measures. The reason is that the attention module of Restormer is computed channel-wise, but forgoes the cross-exposure spatial dependency to repair the non-aligned area. 

\subsubsection{Ablation Study on Loss Function}
We trained our model under different loss function configurations, as shown in \ref{table_ablation_block_loss}. The results validate that the SSIM loss benefits detail reconstruction.

\subsubsection{Ablation Study on Gating Mechanism}
\begin{table}[h]
\resizebox{\columnwidth}{!}{
\begin{tabular}{l|c|c|c|c|c}
\hline
           & PSNR-$\mu$ & PSNR-l & SSIM-$\mu$ & SSIM-l & HDR-VDP-2 \\ \hline
w/o gating & 43.14  & 41.03  & 0.9904 & 0.9868 &     64.88      \\ \hline
one gating & 43.44  & 41.42  & 0.9909 & 0.9882 &    67.13   \\ \hline
Ours       & 43.61  & 41.74  & 0.9909 & 0.9881 & 66.96     \\ \hline
\end{tabular}
}
\caption{Ablation experimental results to verify the effectiveness of the gating mechanism}
\label{table_ablation_gating}
\end{table}

The gating mechanism is an important component in our model. Ablation study is conducted in the gating mechanism as follows.

\textbf{w/o gating}: The gating mechanism is not used in the feed forward network of all transformer layers in the model, that it, our GST unit degenerate to the vanilla Swin Transformer.

\textbf{one gating}: The gating mechanism is only used in the first Swin Transformer layers subsequent to the embedding layer, but not used for other layers. 

 Table \ref{table_ablation_gating} shows the results of the ablation experiments, where the model is trained for 20 epochs. By removing the gating mechanism, the network relies on self-attention for image alignment, resulting in the lowest performance. On top of it, adding gates to low level layers notably improves the HDR reconstruction. Furthermore, by integrating the gating mechanism with all Swin Transformer layers, the model effectively inpaints information in non-aligned regions and obtains the highest HDR reconstruction results, thus validates the effectiveness of the gating mechanism in our model.


% !TEX root = ./CauchyCombination.tex
\section{Multiple Hypothesis Testing} \label{secPrelims}

This section introduces the notation on multiple hypothesis testing and the benchmark procedures for addressing the multiple testing problem. 

\subsection{Setting}
Let $H_{i}$ denote the $i^{\text{th}}$ null hypothesis of interest, with $i=1,...,d$,  
and $d$ being the total number of individual hypotheses. To test the $d$ hypotheses, we can use the associated vector of test statistics $\bm{X}=(X_{1},X_{2},\ldots,X_{d})^{^{\prime }}$, one for each hypothesis being tested, or the corresponding raw $p$-values $p_{1},\ldots ,p_{d}$. The test statistics can be independent or % corrected.
correlated. 

%In some cases, like in Section \ref{secApplDriftBurst}, the test statistics are constructed from rolling windows and are extremely serially correlated. 

The first task is to test the global null hypothesis. Let $\mathcal{H}_{0}$ be the collection of null hypotheses of interest. 
The strategy of a classical global test is to abandon the multiplicity issue altogether and replace multiple tests with the global null hypothesis that all elementary hypotheses are true.  The alternative is that at least one elementary hypothesis is false. For example, in high-frequency financial econometrics,  we often need to monitor the presence of certain events (e.g., jumps or drift bursts) within a fixed time period (e.g., within a day). The global null is that there is no occurrence of such an event at all (e.g., 
% in Example 2, none of the stocks has a significant alpha
in Example 1, there is no drift burst within the day or in Example 2, none of the stocks has a significant alpha).
The goal is to get $\alpha$-level control under this global null, i.e., $P_{\mathcal{H}_0} [\text{reject}\, \mathcal{H}_0] \leq \alpha$. The test is conservative when $P_{\mathcal{H}_0} [\text{reject}\, \mathcal{H}_0]$ is strictly less than the theoretical upper bound $\alpha$ and ideal when it is equal to  $\alpha$. 

When, by any  test, the global null $\mathcal{H}_0$ is rejected, the second task is to identify which of the elementary hypotheses $H_{i}$ should be rejected. The set of true hypotheses $\mathcal{T}$, the set of false hypotheses $\mathcal{F}$ and the set of rejected hypotheses $\mathcal{R}$ are defined as: 
\begin{align}
	\begin{split}  \label{eqLocalHypothesis}
		\mathcal{T} &= \{ H_{i}\in \mathcal{H}_0: H_{i} \, \text{is true}\}, \\
		\mathcal{F} &= \{ H_{i}\in \mathcal{H}_0: H_{i} \, \text{is false}\}, \text{and} \\
		\mathcal{R} &= \{H_{i}\in \mathcal{H}_0: H_{i} \, \text{is rejected}\}.
	\end{split}%
\end{align}
The set of true and false hypotheses are unknown. We choose a set of hypotheses to reject. 
on the basis of our data. 
The set of discoveries $\mathcal{R}$ 
should coincide with the set of false hypotheses $\mathcal{F}$ as much as possible.

The goal of various multiple testing corrections is to control the familywise error rate (FWER), defined as the probability of at least one false
rejection in the family, $P[\mathcal{T} \cap \mathcal{R} \neq \varnothing]$,
while retaining the reasonable power in detecting false hypotheses. We want procedures for which the FWER is less than or equal to the upper bound $\alpha$ and ideally as close as possible to the upper bound. We focus on strong control of the FWER, meaning that some of the hypotheses we are testing can be false ($\mathcal{F} \neq \varnothing$), as opposed to the weak FWER control where all hypotheses of interest are true, i.e., $\mathcal{H}_0=\mathcal{T}$.

The probability of falsely rejecting a single hypothesis that is true (i.e., false positive or Type I error) is usually controlled at a nominal $\alpha$-level. However, when the number of tested hypotheses is large, the problem of multiplicity arises: the probability of having at least one false positive conclusion rises well above $\alpha$ if the Type I error of each individual test is controlled at the $\alpha$-level. Numerous controlling procedures have been proposed to deal with this problem. 
We review two classes of controlling procedures: one based on statistical inequalities (Section \ref{ssecOrderdPvals}) and one based on the maximum of the test statistics (Section \ref{ssecMaxTest}).


\subsection{Procedures based on statistical inequalities}
\label{ssecOrderdPvals}

Let us denote by $0 < p_{(1)}\leq p_{(2)}\leq \ldots\leq p_{(d)} < 1$ the set of $d$  ordered (in ascending order) raw $p$-values and $H_{(1)},H_{(2)},\ldots,H_{(d)}$ their corresponding null hypotheses. A single-stage method uses the same rejection 
criterion for all individual hypotheses, like the conservative Bonferroni threshold, while a multi-stage method examines the ordered $p$-values sequentially and adjusts the rejection criterion for each of the individual tests  \citep[e.g.,][]{holm1979simple,hochberg1988sharper,hommel1988stagewise}. 

The Bonferroni method rejects the elementary null hypothesis $H_{(i)}$ if 
$p_{(i)}\leq\alpha/d$ 
for $i=1,\ldots,d$. \citet{holm1979simple} and \citet{hochberg1988sharper} use the same critical values $ \alpha / (d - i + 1)$ depending on the rank of the $p$-value, but reject differently depending on whether they ``step up" or ``step down". The terminologies (``step up" or ``step down") were originally formulated in terms of test statistics which can be confusing when discussing $p$-values.  \citet{holm1979simple} proposes a step-down method that ``steps up" from the smallest $p$-value to the largest one. It is a pessimistic approach: it scans forward and stops as soon as a $p$-value fails to clear its threshold. \citet{hochberg1988sharper} suggests a step-up method that ``steps down" from the largest $p$-value to the	smallest one. It is an optimistic approach: it scans backward and stops as soon as a $p$-value succeeds in clearing its threshold. By construction, Hochberg's procedure will reject as many hypotheses as Holm's procedure. 

\cite{hommel1988stagewise} proposes a more complicated procedure which applies  \citet{simes1986improved}' global test to the $p$-value subset $\left\{ p_{\left(k\right) }\right\} _{ k = i }^{d}$, instead of relying  only on $p_{\left(i\right)}$ to draw inference on $H_{(i)}$ and thus borrows power across hypotheses. 
Hommel's procedure is shown to have higher power than Hochberg's method \citep{hommel1989}. We refer to Appendix \ref{AppOrderedPVals} for more details on the practical implementation of these procedures.


Bonferroni and \citet{holm1979simple}'s method are based on the first-order Bonferroni inequality, which states that given any set of events, the probability of their union is smaller than or equal to the sum of their probabilities. 
Under the null hypothesis, 
the probability that there is at least one hypothesis $H_{(i)}$  for which its raw $p$-value $p_{(i)} \leq \alpha / d$ 
% is not greater than $\alpha$ 
is bounded by $\alpha$: 
\begin{align}  \label{eqIneqPval}
	\Pr\left( \min_{i} p_{(i)} \leq \frac{\alpha}{d} \right) 
	= \Pr\left(\bigcup_{i =
		1}^{d} \left\{ p_{(i)} \leq \frac{\alpha}{d} \right\} \right) 
	&\leq
	\sum^{d}_{i = 1} \Pr \left( p_{(i)} \leq \frac{\alpha}{d}\right) \leq d
	\frac{\alpha}{d} \leq \alpha.
\end{align}
The Bonferroni \eqref{eqIneqPval} inequality 
makes no specific assumption on the dependence between the $p$-values, but protects against the so-called ``worst-case", in which all events are independent and the rejection regions are disjoint (the right half of Equation \eqref{eqIneqPval}) . 
The inequality becomes an equality when all test statistics are independent, and a strict inequality when the hypotheses are dependent. 
In other words, the Bonferroni correction is  conservative when the $p$-values are correlated. 

The methods of \cite{hochberg1988sharper} and \cite{hommel1988stagewise} are based on  \cite{simes1986improved}'s inequality. If a set of hypotheses $H_{(1)}, ...,H_{(d)}$ are all true, the probability of the joint event is: 
\begin{equation}  \label{eqSimes}
	\Pr\left( p_{\left( i\right) }> \frac{i\alpha}{d}, \text{ for all } i=1,\ldots
	,d\right) \geq 1-\alpha.
\end{equation}
\citet{simes1986improved}' inequality was developed for independent uniform $p$-values, and it is applicable for a large family of multivariate distributions. The simulations of \citet{simes1986improved} do show, however, that the test is very conservative for highly correlated multivariate normal statistics, but less so than the classical Bonferroni correction. 



\subsection{Procedures based on the maximum of test statistics}
\label{ssecMaxTest}

Another class of controlling procedures uses the maximum in a group of test statistics: $X_m = \max_{i} \abs{X_{i}}$, with $i = 1, \ldots, d$, to set a stringent critical value. The same critical value can be used for each elementary hypothesis and will control the familywise error rate. In particular, when the individual test statistics are independent and follow the standard normal distribution under the null, the maximum of the test statistics follows a Gumbel distribution when $d$ is large. Quantiles of the Gumbel distribution were used as critical values of the individual tests as a multiple testing correction when, for example, conducting jump tests in high-frequency asset returns \citep[][]{lee2007jumps}. Unfortunately, if the sequence of test statistics exhibits strong correlation, the number of tests severely overstates the effective number of independent copies in a given sample, which makes the Gumbel critical values too conservative \citep[see e.g.,][]{christensen2018drift}. We refer to Appendix \ref{AppOrderedPVals} for more details on the Gumbel distribution. 

Resampling-based methods account for the dependence structure that is specific to the considered dataset, leading to less conservative testing outcomes than the Gumbel-based methods and the inequality-based procedures. Depending on the empirical problem of interest, the resampling can be carried out by bootstrap, permutation, simulation, or randomization \citep[see e.g.,][for detailed discussions on resampling methods and testing procedures]{white2000,romano2005exact,romano2005stepwise,lehmann2005testing}. We refer to Section \ref{secApplDriftBurst} for an example of the resampling-based approach for the drift burst test. 


\section{Cauchy Combination Tests}
\label{secSeqCauchy}

In this section, we first review the global Cauchy combination (GCC) test of \citet{liu2020cauchy} and present our sequential Cauchy combination (SCC) test. While the global test tests the global null hypothesis $\mathcal{H}_0 = \bigcap_{i=1}^{d} H_{i}$ against the alternative hypothesis that at least one of the elementary null hypotheses is false, the sequential test aims at identifying the violations of the elementary null hypotheses while controlling the global error rate. 

\subsection{Global Cauchy combination test}
\label{sec:CC}

The GCC test statistic is constructed from the raw $p$-values of the test statistics $X_i$, which follow a uniform distribution between $0$ and $1$ under the  null hypothesis. The idea of this test is first to transform the uniformly distributed $p$-values into standard Cauchy variates using the formula $\tan \{(0.5-p_{i})\pi \}$ and then construct a new test statistic as the weighted sum of these transformed $p$-values. The new test statistic is denoted by $\tilde{T}$ and defined as: 
\begin{equation}
	\label{eqCauchyStatistic}
	{\normalsize \tilde{T}=\sum_{i=1}^{d}w_{i}\tan \{(0.5-p_{i})\pi \},} 
\end{equation}
in which the $w_{i}$'s are non-negative weights summing to 1. Throughout the paper, the weights $w_{i}$ are set to $1/d$ for $i=1,\ldots,d$ as in \citet{liu2020cauchy}. 

When the raw and hence the transformed $p$-values are independent (resp. perfectly correlated), % under the null hypothesis, 
the new test statistic $\tilde{T}$ is a linear combination of independent (resp. perfectly correlated) Cauchy variates and therefore follows a standard Cauchy distribution because the family of Cauchy densities is closed under convolutions. Although the correlation structure can affect the null distribution of $\tilde{T}$ in the case of general dependence, \citet{liu2020cauchy} show that the impact on the tail is very limited because of the heaviness of the Cauchy tail. Specifically, they prove that: 
\begin{equation}
	\lim_{h\rightarrow \infty }\frac{\Pr\left( \tilde{T}>h\right) }{\Pr\left(
		C>h\right) }=1,  \label{eq:tail}
\end{equation}
in which $C$ is a standard Cauchy random variable, under the null hypothesis $\mathcal{H}_0$ and Assumption \ref{ass1} which requires the test statistics to follow a bivariate zero mean normal distribution.
\begin{assumption}
	\label{ass1} (1) The original test statistics $(X_{i},X_{j})$, for any $1\leq
	i<j\leq d$, follow a bivariate normal distribution; (2) $E\left( \bm{X}%
	\right) =0$, with $\bm{X}=(X_{1},X_{2},\ldots,X_{d})^{^{\prime }}$. 
\end{assumption}
The bivariate normal requirement of Assumption \ref{ass1} is a condition weaker than joint normality, making the procedure applicable for high-dimensional settings. When the dimension $d$ increases at a certain rate with the sample size, the test statistics $\bm{X}$ may not jointly converge to a multivariate normal distribution due to its slower rate of convergence \citep[see][and references therein]{liu2020cauchy} and thus a joint normality assumption is not realistic for those settings. In contrast, the bivariate normality assumption is much weaker and more realistic. There are, of course, applications for which the test statistics are not normally distributed. Through simulations, \citet[][]{liu2020cauchy} show the Cauchy approximation is still accurate when the normality assumption is violated and follows a multivariate Student-$t$ distribution (with 4 degrees of freedom) instead. 
We refer to Section \ref{secApplFan} for a showcase example in finance with test statistics being Student-$t$ distributed. 

The result in  \eqref{eq:tail} suggests that, under the null hypothesis $\mathcal{H}_0$, the tail of the Cauchy combination test statistic is approximately Cauchy under arbitrary dependence structures, so that a $p$-value of the Cauchy combination test, denoted 
$\widetilde{p}$, can  be calculated from the standard Cauchy distribution. Suppose that we observe $\tilde{T}=t_{0}$, then: 
\begin{equation}
	\label{eqCauchyPval}
	\widetilde{p}=\frac{1}{2}-\frac{\arctan t_{0}}{\pi }. 
\end{equation}

Using the GCC $p$-values, the tail result in \eqref{eq:tail} can be equivalently stated as the actual size converging to the nominal size $\alpha$ as the significance level tends to zero:  % \textit{i.e.}, 
\begin{equation}
	\lim_{\alpha\rightarrow 0 }\frac{\Pr\left( \widetilde{p} \leq \alpha\right) }{\alpha}=1, \label{eq:wFWER}
\end{equation} 
The approximation should be particularly accurate for small $\alpha$'s, which are of particular interest in large-scale problems as in Examples 1 and 2. The simulations in \citeauthor{liu2020cauchy} show that when the significance level is moderately small ($\alpha = 10^{-1}, 10^{-2},10^{-3},10^{-4},10^{-5}$), the $p$-value calculation is accurate:  the ratio of the empirical size to the significance level is close to 1 for different types of correlations. 
Put differently, the GCC test achieves the weak familywise error rate control as the empirical size is very close to the nominal size $\alpha$ regardless of the correlation structure. 


Figure \ref{figCauchyPvalsAR} illustrates the fact that while the dependence between the individual test statistics $X_i$ can affect the null distribution of the GCC test statistic, the impact of the dependence is marginal on the tail. We simulate a vector of $d$ test statistics  $\bm{X}$ from a $d$-variate normal distribution with correlation matrix $\bm{\Sigma}$, \textit{i.e.}, $N_d(\bm{0}, \bm{\Sigma})$ with $\bm{\Sigma} = (\sigma_{ij})$ and $d = 300$. The diagonal elements $\sigma_{ii}=1$ for all $i=1,\ldots,d$ and the off-diagonal elements $\sigma_{ij} = \theta^{\abs{i-j}}$ for $i \neq j$, with $\theta = 0.2, 0.4, 0.6,$ $0.8, 0.90, 0.95$. The simulation is repeated $10^7$ times. For each draw, we calculate the GCC test statistic \eqref{eqCauchyStatistic} and its corresponding $p$-value \eqref{eqCauchyPval}. The histogram of the $10^7$ GCC $p$-values is displayed in Figure \ref{figCauchyPvalsAR}. For a low level of autocorrelation (i.e., $\theta=0.2$), the distribution of the $p$-values is close to a uniform distribution. When the level of autocorrelation is higher, there is a pothole in the middle and a bump at the end of the histogram, but whatever the strength of the autoregressive parameter, the percentage of the GCC $p$-values in the first bin is always around $5$\% as is ensured by the limit result in \eqref{eq:wFWER}. 


\begin{figure}[p]
	\caption{The impact of dependence on the tail of the GCC test statistic}
	\label{figCauchyPvalsAR}
	\centering
	
	\par
	
	\subfloat[${\theta} = 0.2$ ]{{\includegraphics[width=.40\textwidth,angle =
			-90,scale=0.70]{Sample_pval_dim_300_rho_0.2_alpha0.05.eps} }} 
	\subfloat[$\theta = 0.4$ ]{{\includegraphics[width=.40\textwidth,angle =
			-90,scale=0.70]{Sample_pval_dim_300_rho_0.4_alpha0.05.eps} }} 
	
	\vspace{0.4cm}
	
	\subfloat[$\theta = 0.6$ ]{{\includegraphics[width=.40\textwidth,angle =
			-90,scale=0.70]{Sample_pval_dim_300_rho_0.6_alpha0.05.eps} }} 
	\subfloat[$\theta = 0.8$ ]{{\includegraphics[width=.40\textwidth,angle =
			-90,scale=0.70]{Sample_pval_dim_300_rho_0.8_alpha0.05.eps} }} 
	
	\vspace{0.4cm}	
	
	\subfloat[$\theta = 0.90$]{{\includegraphics[width=.40\textwidth,angle =
			-90,scale=0.70]{Sample_pval_dim_300_rho_0.9_alpha0.05.eps} }} 
	% 
	\subfloat[$\theta = 0.95$]{{\includegraphics[width=.40\textwidth,angle =
			-90,scale=0.70]{Sample_pval_dim_300_rho_0.95_alpha0.05.eps} }} 
	
	\par
	\begin{minipage}{1.0\linewidth}
		\begin{tablenotes}
			\small
			\item {
				\medskip
				Note: We plot histograms of GCC $p$-values \eqref{eqCauchyPval} for various correlation strengths. The individual test statistics are drawn from a $d$-variate normal distribution $N_d(\bm{0}, \bm{\Sigma})$ with $\bm{\Sigma}= (\sigma_{ij})$ and $d=300$. The diagonal elements of the covariance matrix $\sigma_{ii}=1$ for all $i=1,\ldots,d$ and the off-diagonal elements  $\sigma_{ij} = \theta^{\vert i-j \vert}$ for $i\neq j$, with $\theta = 0.2, 0.4, 0.6,$ $0.8, 0.90, 0.95$. We compute the GCC $p$-value from the test statistic sequence. The simulation is repeated $10^7$ times. The simulated GCC $p$-values are sorted into bins with the bin edges being a sequence of edges from 0 to 1 with a width of  0.05. 				Each bin includes the right edge (right-closed) but does not include the left edge (left-open). We highlight the first bin in black and we
				also add a text note with the probability of $p$-values being in the first bin. }
		\end{tablenotes}
	\end{minipage}
\end{figure}

Interestingly,  \citet{liu2020cauchy} show that the tail property \eqref{eq:tail}  also holds when the number of hypotheses $d$ diverges to infinity at a rate of $o\left(h^{\eta }\right) \ $with $0<\eta <1/2$ and the following additional assumption is satisfied.
\begin{assumption}
	\label{ass2} Let $\mathbf{\Sigma }=corr\left( \bm{X}\right) $. 
	(1) The	largest eigenvalue of the correlation matrix $\lambda _{\max}\left( \mathbf{%
		\Sigma }\right) \leq C_0$ for some constant $C_0>0$; 
	(2) $\max_{1\leq i<j\leq
		d}\left\{ \sigma _{i,j}^{2}\right\} \leq \sigma _{\max }^{2}<1$ for some
	constant $0<\sigma _{\max }^{2}<1$, where $\sigma _{i,j}$ is the $\left(
	i,j\right) $ element of $\mathbf{\Sigma }$.
\end{assumption}
The additional assumptions on the correlation matrix are mild and standard in high dimensional settings and are general enough to incorporate a large class of tests. 


\subsection{Sequential Cauchy Combination Test}

The main contribution of this paper is the sequential Cauchy combination (SCC) test, which unravels the GCC test to make statements on the elementary hypotheses. The raw $p$-values are sorted in ascending order so that  $p_{(1)}\leq p_{(2)}\leq \ldots \leq p_{(d)}$, which is standard for step-down and step-up sequential procedures (see Section \ref{ssecOrderdPvals}). For the inference on hypothesis $H_{\left( i\right) }$ we compute a Cauchy combination test statistic ${\normalsize \tilde{T}}_{\left( i\right) }$ from a subset of $p$-values, running from $p_{(i)}$ to $p_{(d)}$ as:  
\begin{equation}
{\normalsize \ \tilde{T}%
		_{\left( i\right) }=\sum_{j=i}^{d}w_{j}\tan \{(0.5-p_{(j)})\pi \}
	}.
	\label{eq:CC_mt}
\end{equation}
The corresponding $p$-value is: $$ \widetilde{p}_{(i)}=\frac{1}{2}-\frac{\arctan \tilde{T}_{\left(i\right) }}{\pi }.$$ We reject the null hypothesis $H_{(i)}$ if  $\widetilde{p}_{(i)}\leq\alpha$. Like the step-up procedure of  \citet{hommel1988stagewise}, the SCC test also borrows power across hypotheses: the test statistic $\tilde{T}_{(i)}$ is computed from the raw $p$-values associated with $\mathcal{H}_0^{(i)}=\bigcap_{j=i}^{d} H_{(j)}$.


\subsubsection*{Theoretical Properties}
The SCC testing procedure can be viewed as a sequential rejection procedure. Let $\mathcal{R}^{(s)}$ be the collection of rejected hypothesis after step $s$, with $s=\left\{1,2,\ldots,d\right\}$. The hypothesis of interest and decision rules in each step  are illustrated in Table \ref{tabDecisionRule}.
\begin{table}[H]
	\caption{Decision rule in the sequential Cauchy combination test}
	\label{tabDecisionRule}
	\centering
	\begin{tabular}{p{1.cm}p{3.8cm}p{10.5cm}}
		\hline
		Step & Hypothesis & Decision\\ 
		$s=1$ & $\mathcal{H}_0^{\left( d\right) }=H_{(d)}$ & 
		If $\widetilde{p}_{(d)}\leq\alpha$ then reject $\mathcal{H}_0^{\left( d\right) }$ and include $H_{(d)}$ in $\mathcal{R}^{(1)}$
		\\
		$s=2$ & $\mathcal{H}_0^{\left( d-1\right) }=\bigcap_{j=d-1}^d H_{(j)}$  
		& 
		If $\widetilde{p}_{(d-1)}\leq\alpha$ then reject $\mathcal{H}_0^{\left( d-1\right) }$  and include $H_{(d-1)}$ in $\mathcal{R}^{(2)}$  
		\\
		$\ldots$  & $\ldots$  & $\ldots$  \\ 
		$s=d$ & $\mathcal{H}_0^{\left( 1\right) }=\bigcap_{j=1}^d H_{(j)}$ & 
		If $\widetilde{p}_{(1)}\leq\alpha$ then reject $\mathcal{H}_0^{\left( 1\right) }$ and include $H_{(1)}$ in $\mathcal{R}^{(d)}$ \\
		\hline
	\end{tabular}
\end{table}
Let $\mathcal{N}\left(\mathcal{R}^{(s)}\right)$ be the successor function, representing hypotheses to be rejected in the next step given that $\mathcal{R}^{(s)}$ has been rejected. For the SCC test, the successor function is defined as: 
\[
\mathcal{N}\left(\mathcal{R}^{(s)}\right)=\left\{H_{(d-s)} :  \widetilde{p}_{(d-s)} \leq \alpha_{\mathcal{R}^{(s)}}=\alpha\right\}.
\]
The cut-off value is fixed (i.e., $\alpha_{\mathcal{R}^{(s)}}=\alpha$) instead of depending on the rejection set $\mathcal{R}^{(s)}$ like in many other sequential procedures. According to the sequential rejection principle of \cite{goeman2010sequential},  the SCC test  achieves a strong family-wise error rate control if the following two conditions are satisfied. 
\begin{condition}[Monotonicity]
	For every $\mathcal{R}^{(s)}\subseteq \mathcal{R}^{(l)} \subset \mathcal{H}_{0}$, 
	\[
	\mathcal{N}(\mathcal{R}^{(s)}) \subseteq \mathcal{N}(\mathcal{R}^{(l)}) \cup \mathcal{R}^{(l)}
	\]
	almost surely. 
\end{condition}
% By construction, 
The transformed $p$-values of the SCC test are monotonic by construction, with $\widetilde{p}_{(d)}$ being the largest for the smallest set of global null hypotheses $\mathcal{H}_0^{(d)} = H_{(d)}$ and $\widetilde{p}_{(1)}$ being the smallest for the largest set of global nulls $\mathcal{H}_0^{(1)} = \bigcap_{j=1}^{d} H_{(j)}$ (see Figure \ref{figSequentialCauchyIllustration}(e) for an illustration of the monotonic $p$-values). Note that the largest set of global null hypotheses has the same null specification as the GCC test \eqref{eqCauchyStatistic}. It follows that that $\widetilde{p}_{(s)}\geq \widetilde{p}_{(l)}$. Since the cut-off value is fixed, the monotonicity condition of the successor function is satisfied.

\begin{condition}[Single-step condition] \label{SS} 
	When $\mathcal{H}_{0}^{(i)} =\mathcal{T}$, 
	$\Pr\left( \widetilde{p}_{(i)} \leq \alpha\right) \leq \alpha. $
\end{condition}
Condition \ref{SS} requires FWER control of the underlying test of SCC (i.e., the Cauchy combination test) at the ``critical case" in which all hypotheses of interest are true: $\mathcal{H}_{0}^{(i)} =\mathcal{T}$. The condition can be rewritten as $\Pr{\mathcal{N}(\mathcal{F})\subseteq \mathcal{F}} \geq 1-\alpha$ and has been shown to be satisfied by \cite{liu2020cauchy}. In fact,  when $\alpha$ is very small, the familywise false rejection probability of the GCC test under the null is not only bounded by $\alpha$ but also approaches the nominal size $\alpha$, as stated in \eqref{eq:wFWER}, which implies that it is less conservative than tests based on statistical inequalities or tests which impose independence in the presence of correlation. 

The theorem below follows directly from \cite[Theorem 1]{goeman2010sequential} for general sequential rejection procedures, so that Type I control in the critical case is sufficient for overall familywise error control of the sequential procedure. 
\begin{theorem}\label{thm}
	The SCC testing procedure satisfies both the monotonicity and the single-step condition and achieves the strong FWER control:  
	\[
	\lim_{\alpha\rightarrow 0}\Pr\left\{\mathcal{R}^{(d)} \subseteq \mathcal{F} \right\} \geq 1-\alpha, 
	\]
	under Assumption \ref{ass1} if $d$ is fixed and under Assumptions \ref{ass1} and \ref{ass2} if $d\rightarrow \infty$.
\end{theorem}


\subsubsection*{An Illustration}

A more prescriptive description of the SCC testing procedure is as follows: 
\begin{enumerate}
	
	\item Obtain raw $p$-values $p_1, p_2,\ldots, p_d$ corresponding to the null hypotheses $H_{1}, H_{2},\ldots, H_{d} $;%
	
	\item Order the raw $p$-values in ascending order, 	$p_{(1)},p_{(2)},\ldots,p_{(d)}$, with corresponding null ordered hypotheses $H_{(1)},H_{(2)},\ldots,H_{(d)}$;
	
	\item Calculate the SCC test statistic $\tilde T_{(i)}$ and the transformed Cauchy $p$-values $\widetilde{p}_{(i)}$ from a subset of the ordered $p$-values $\left\{p_{(j)}\right\} _{j=i}^{d}$ using \eqref{eq:CC_mt} for $i=1,\ldots,d$;
	
	\item Obtain the rejection set $\mathcal{R}=\left\{H_{\left(i\right)} : \widetilde{p}_{(i)}\leq \alpha\right\}$. 
\end{enumerate}

Figure \ref{figSequentialCauchyIllustration} illustrates the sequential Cauchy combination procedure on a simulated sequence of test statistics. The top row shows the simulated test statistics and their corresponding $p$-values, of which some hypotheses are under the null (grey dots) and some are under the alternative (black dots). The data-generating process is the same as that in Figure \ref{figCauchyPvalsAR} with $\theta=0.9$ and $d=100$. We add constant signals for $5$ out of 100 hypotheses,  with a signal strength equal to $\pm2.806$. The sign of the signal is the same as the sign of the test statistic under the null, such that the signal always amplifies the magnitude of the test statistic. 
The GCC test rejects the global null at $\alpha = 5\%$ for this sequence of $p$-values, which tells us there is at least one signal in the sequence.  
%The estimated first-order autocorrelation of the simulated test statistics is equal to $0.7910$ under the null and is equal to $0.4987$ under the alternative. 

\begin{figure}[p]
	\caption{Rejection procedure of the sequential Cauchy Combination test}
	\label{figSequentialCauchyIllustration}\centering

	\par
	
	\subfloat[Raw test statistics]{{\includegraphics[width=.31\textwidth,angle =
			-90]{1_tstat_d_100_rho_0.9_signal_5_5} }} 
	\subfloat[Raw
	$p$-values]{{\includegraphics[width=.31\textwidth,angle = -90]{2_pvals_d_100_rho_0.9_signal_5_5} }}
	
	\vspace{0.4cm}	
	
	\subfloat[Ordered raw $p$-values]{{\includegraphics[width=.31\textwidth,angle =
			-90]{3_spvals_d_100_rho_0.9_signal_5_5} }} 
	
	\vspace{0.4cm}	
	
	\subfloat[SCC test statistics
	]{{\includegraphics[width=.31\textwidth,angle = -90]{4_ctstats_d_100_rho_0.9_signal_5_5}}}
	\subfloat[SCC 
	$p$-values]{{\includegraphics[width=.31\textwidth,angle = -90]{4_cpvals_d_100_rho_0.9_signal_5_5}}}
	
	
	\begin{minipage}{1.0\linewidth}
		\begin{tablenotes}
			\small
			\item {
				\medskip
				Note: We illustrate the mechanics of the SCC procedure on a simulated test statistic sequence with sparse signals. The top row shows raw test statistics and $p$-values of which some hypotheses are under the null and some are under the alternative. The test statistics are simulated from $N_{d}(\bm{0},\bm{\Sigma})$ as in Figure \ref{figCauchyPvalsAR}. We set $d=100$, $\theta=0.9$ and add $5\%$ signals. The strength of the signal is $\pm2.806$, with its sign identical to that of the test statistic under the null. The horizon line in panel (e) is the 5\% significance level.
			}
		\end{tablenotes}
	\end{minipage}
\end{figure}

The SCC test can tell us which individual $p$-values trigger the rejection of the GCC test. The middle row plots the raw $p$-values in ascending order and the bottom row plots its sequential Cauchy combination test statistics and $p$-values. Specifically, the bottom right panel shows that the SCC $p$-values $\widetilde{p}_{(i)}$ decrease as $i$ moves from $d$ to $1$. In this example, the SCC test rejects three out of the five alternative hypotheses and does not reject under the null hypothesis. The rejections correspond to the 4$^\text{th}$, 29$^\text{th}$ and  46$^\text{th}$ hypotheses in the top row. Note that the smallest SCC $p$-value corresponds to the $p$-value of the GCC test of \citet{liu2020cauchy}, which performs the test on the largest set of hypotheses. 

\bibliography{lpn_ref}




\end{document}
