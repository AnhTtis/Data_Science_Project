\documentclass[lettersize,journal]{IEEEtran}
\usepackage{amsmath,amsfonts}
\usepackage{algorithmic}
\usepackage{algorithm}
\usepackage{array}
%\usepackage[caption=false,font=normalsize,labelfont=sf,textfont=sf]{subfig}
\usepackage{textcomp}
\usepackage{stfloats}
\usepackage{url}
\usepackage{verbatim}
\usepackage{graphicx}
\usepackage[tight,footnotesize]{subfigure}
\usepackage{cite}
\usepackage{booktabs}

%\usepackage{enumitem} %Adding the subitems within the item
\usepackage{color} %Set the word color
%\usepackage{verbatim} %
\hyphenation{op-tical net-works semi-conduc-tor IEEE-Xplore}
% updated with editorial comments 8/9/2021

\begin{document}
%\captionsetup{font={small}}
\title{\fontsize{16}{18}\selectfont
A Fast Path Loss Model for Wireless Channels Considering Environmental Factors}
%\title{A Fast Path Loss Model for Wireless Channels Considering Environmental Factors}

\author{Lingyou~Zhou,~\emph{Student Member, IEEE,} Jie~Zhang,~\emph{Senior Member, IEEE,} Jiliang~Zhang,~\emph{Senior Member, IEEE,} Oktay Cetinkaya,~\emph{Member, IEEE,} and Steve Jubb 
	\vspace{-0.2in}
	% <-this % stops a space
	
	\thanks{
		Lingyou Zhou and Steve Jubb are with the Department of Electronic and Electrical Engineering, the University of Sheffield, Sheffield, S10 2TN, UK.
		
		Jie Zhang is with the Department of Electronic and Electrical Engineering, the University of Sheffield, Sheffield, S10 2TN, UK, and also with Ranplan Wireless Network Design Ltd., Cambridge, CB23 3UY, UK. 
		
		Jiliang Zhang is with the College of Information Science and Engineering, Northeastern University, Shenyang, 110819, China.
		
		Oktay Cetinkaya is with the Department of Engineering Science, University of Oxford, Oxford, OX1 2JD, UK.
		
		(Corresponding author: Jie Zhang)
		%Jiliang Zhang and Jie Zhang are with the Department of Electronic and Electrical Engineering, the University of Sheffield, Sheffield, S10 2TN, UK, and also with Ranplan Wireless Network Design Ltd., Cambridge, CB23 3UY, UK. 
	}% <-this % stops a space
	}


% The paper headers
\markboth{IEEE}%
{Shell \MakeLowercase{\textit{et al.}}: A Sample Article Using IEEEtran.cls for IEEE Journals}

 
 

\maketitle

\begin{abstract}
%Section Checked 09:55 24/01/2023
We present a path loss model that accurately predicts the path loss with low computational complexity considering environmental factors. In the proposed model, the entire area under consideration is recognized and divided into regions from a raster map, and each type of region is assigned with a path loss exponent (PLE) value. We then extract the model parameters via measurement in a suburban area to verify the proposed model. The results show that the root mean square error (RMSE) of the proposed model is 1.4 dB smaller than the widely used log-distance model.
\end{abstract}

\begin{IEEEkeywords}
	Path loss, prediction, propagation.
\end{IEEEkeywords}

\section{Introduction}
%Section Checked 10:14 24/01/2023
\IEEEPARstart{R}{adio} propagation path loss characterization plays a crucial role in wireless network planning and optimization. Currently, deterministic and empirical models are the two most widely used categories of radio propagation models. Deterministic models, such as ray-tracing and ray-launching models \cite{rt1,rt2}, compute rays from the receiver (Rx) to the transmitter (Tx) and from Tx to Rx based on ray optics, respectively. To maintain high prediction accuracy, however, these models require a large amount of computation resources and time. Empirical models, such as Okumura-Hata model \cite{ho1,ho2} and COST-231 Hata model \cite{ep3}, use one characterization with a few parameters to calculate path loss for a typical scenario. Whereas, accuracy is sacrificed for maintaining minimum complexity and the type of environments is constrained to what the models are trained for.   

Meanwhile, several parametrized statistical models for path loss prediction have been proposed, e.g.,\cite{md1,md2,md3,md4,md5,md6,md7,md8,md9}. The parameters in these models can be either extracted from measurement data, or supplied by ray-tracing simulation results. Among these models, the MiWEBA channel model, the mmMAGIC channel model, the METIS channel model, the 5GCMSIG, the 3GPP channel model and the IMT-2020 channel model accommodate propagation characteristics of blockage and gaseous absorption which can be extended to incorporate the high blockage effect due to environmental factors. However, a comprehensive path loss model incorporating environmental factors with low computational complexity is still missing.

In this communication, we present a low computational complexity path loss model considering environmental factors. The initial idea of this model comes from the log-distance model \cite{s2-1}, wherein instead of one PLE, the proposed model contains multiple exponents and each stands for a parametrized environmental factor (PEF). This model first extracts the environment data from a raster map, by recognizing different environmental obstacles, the raster map is divided into multiple regions. To parametrize the environment, each region is labelled with an undefined PEF, which can be further determined via measurement. For calculating path loss between the Tx and Rx, a straight line between them is first generated to record all the regions it intersected. Then, the decibel path loss of each region in the line is accumulated, which leads to the final path loss. We also apply this PEFs model in a suburban area and the results show a good performance of our PEFs model, with an RMSE of 6.78 dB.

In Section II, we present the details of the PEFs path loss model. Section III provides the procedure of extracting PEFs used to construct the PEFs model. Section IV applies the PEFs model in a measurement area to verify its performance, along with the simulation results and analysis. 

\section{The Model}
%Section Checked 11:21 24/01/2023
In this section, we first describe the PEFs path loss model with multiple PLEs, wherein each PLE stands for the parameter of a certain environmental obstacle type (e.g., buildings), and the details of characterizing environment-based path loss will also be given in this section. Then, we reveal the relationship between the PEFs model and the log-distance path loss model under special conditions. 

%By path loss, we mean the transmit power minus the received power in dB scale. % plus the antenna gains and
%the transmit power times the antenna gains divided by the received power.

\subsection{Summary of the Design}
%Subsection Checked 24/01/2023
The PEFs path loss model is derived from the raster map of the measurement area, which contains several types of environmental obstacles with different colours shown in Fig. \ref{DsnExpl}(a). By distinguishing different types of obstacles, the raster map can be divided into multiple regions illustrated in Fig. \ref{DsnExpl}(b). Each region, consisting of a random number of pixels, will be labelled with a PEF. 

Then, a straight line path between Tx and Rx is generated, and the PEF of and the path distance within every region intersected by the line are recorded, as shown in Fig. \ref{DsnExpl}(c). Hence, the path loss between Tx and Rx can be calculated by accumulating the decibel path loss of each region in the line path using the PEFs path loss equation (Reported next). Also, the quantities of PEFs can be derived further from the measurement data.

\begin{figure*}[ht]
	\vspace{-5pt}
	\centering
	\subfigure[]{\includegraphics[width=1.3in]{figure/PrlRtM}}%
	\hfil
	\subfigure[]{\includegraphics[width=1.3in]{figure/PrlRgM}}%
	\hfil
	\subfigure[]{\includegraphics[width=1.3in]{figure/RgMwtSL}}%
	\hfil
	\subfigure[]{\includegraphics[width=2.35in]{figure/Wtdst}}%
	\vspace{-5pt}
	\caption{One example of raster map extraction (a) The original raster map (b) Result of region division (c) Generating a straight-line path between Tx and Rx (d) The distance for each region within the Tx-Rx line path (e.g., $d_{0}$ and $d_{1}$ stand for the close-in distance and the distance of the first region, respectively).}
	\label{DsnExpl}	

\end{figure*}

For the entire characterization of this model is based on the Tx-Rx line path and the information it contains, the details of the Tx-Rx line path are necessary to be mentioned. By generating the line path, a path matrix $\mathbf{S}$ is created as

\begin{equation}
	\label{LnMtx}
	%$\mathbf{\textit{S}}$
    \mathbf{S} =
	\begin{gathered}
		\begin{bmatrix}
			R_0(R_{\mathrm{tx}})  & R_1 & R_2    & R_3    & \cdots  & R_{N}(R_{\mathrm{rx}})\\
			d_0 & d_1 & d_2    & d_3    & \cdots  & d_{N} \\
		\end{bmatrix},
		\quad
	\end{gathered}
\end{equation}
with
\begin{equation}
	\label{PxlCor}
	R_N =
	\begin{gathered}
		\begin{bmatrix}
			x_{1}  & x_{2}    & \cdots  & x_{p}  \\
			y_{1}  & y_{2}    & \cdots  & y_{p}  \\	
		\end{bmatrix}
		\quad
	\end{gathered}
\end{equation}
and
\begin{equation}
	\label{Pxldst}
	d_N =
	\begin{gathered}
		\begin{bmatrix}
			c_{1}  & c_{2}    & \cdots  & c_{p}  \\
		\end{bmatrix},
		\quad
	\end{gathered}
\end{equation}
where the first and second rows in the matrix \eqref{LnMtx} denote the regions and the region distances that the path travelled, respectively. For the distance $d_0$, it is the close-in distance in the line path corresponding to the close-in region $R_0$ (also set as Tx region $R_{\mathrm{tx}}$) \cite{s2-1}, wherein the regions within distance $d_0$ are not counted. That is to say, instead of $N$+$1$ regions, the total number of regions intersected by the Tx-Rx line path is the numerical value $N$.   

As each region contains multiple pixels, the regions and the region distances can be represented by pixels. For the $N$th region in the line path, it is shown as the pixel coordinates in matrix \eqref{PxlCor}, where $(x_{p},y_{p})$ is the $p$th pixel coordinates. Besides, the region distance is shown in matrix \eqref{Pxldst}, where $c_{p}$ is the path distance in the $p$th pixel. Also note that the numerical value $p$ is random as mentioned before, which may differ by region. 

Comprehensively, the straight line, starting from the close-in region $R_{0}$ and ending at the Rx region $R_{N}$, intersects $N$ regions in total (excluding the close-in region $R_0$). Note that $R_0$ and $R_{N}$ are also the regions of Tx ($R_{\mathrm{tx}}$) and Rx ($R_{\mathrm{rx}}$), respectively. One example is shown in Fig. \ref{DsnExpl}(d), wherein five regions in total are intersected by the straight line path. In this case, the region number $N$ is set as five, and the first region is $R_1$ and the last region is $R_5$.

\subsection{PEFs Path Loss Model with Multiple Exponents}
%Subsection Checked 24/01/2023
Based on the above descriptions, the characterization of total path loss is
\begin{equation}
	\label{WtdPL}
	PL = \sum_{a=0}^{N} PL_a+\Psi_{\sigma}
\end{equation}	
with
\begin{equation}
	\label{SubWtdPL}
	\begin{aligned}
		PL_a=
		\left\{
		\begin{array}{ccl}
			C &{a=0}, \\
			10 n_a \log_{10}\left(\frac{\sum_{j=0}^{a} d_j}{\sum_{j=0}^{a-1} d_j}\right)  &{a > 0}, \\
		\end{array} \right.
	\end{aligned}
\end{equation}
where $PL_{a}$, $d_{j}$, $n_{a}$, $C$ and $\Psi_{\sigma}$ are characterized as follows.

$Weighted\,Path\,Loss$: $PL_{a}$ is the sub-path loss value for the $a$th region ($N$ regions in total) intersected by the Tx-Rx line path.

$Weighted\,Distance$: In equation \eqref{SubWtdPL} when $a$ larger than 0, the term in parentheses is the weighted distance, wherein the numerator and the denominator are the distances from the close-in region (Tx region $R_{\mathrm{tx}}$) to $a$th and $a$-1th region, respectively.

$Path\,Loss\,Exponents$: The PLE $n_{a}$ is the PEF in dB scale for the $a$th region, and each region has a corresponding PEF.

$Intercept$: The intercept value $C$ is the decibel path loss at the close-in region with distance $d_0$. 

$Shadow\,Fading\,Term$: The shadow fading term $\Psi_{\sigma}$ is a zero-mean Gaussian random variable of standard deviation $\sigma$, $N[0,{\sigma}^{2}]$.

By summation of the weighted path loss in $N$ regions and a shadow fading term $\Psi_{\sigma}$, the total path loss is derived as \eqref{WtdPL}. Also note that the number of intersected $N$ regions may not equal the total type of regions within the environment (e.g., there are four region types in Fig. \ref{DsnExpl}(d), only three are intersected by the line path). In other words, if there are $I$ region types within the measured environment and each type has a specific PEF value, the formation of \eqref{WtdPL} can be simplified by combining like terms of PEFs and can be then expressed as 
\begin{equation}
	\label{WtdPL_LkTrm}
	PL = C + \sum_{i}^{I} D_in_i+\Psi_{\sigma},
\end{equation} 
where $n_i$ is the $i$th PEF and $D_i$ is the coefficient of $i$th PEF after the combination of like terms. Because the shadow fading term is modeled by Gaussian distribution in dB scale, the path loss in this model can be expressed as a normally distributed value with the same standard deviation along with a mean value, $N[\mu(n_{i},C),{\sigma}^{2}]$, where \cite{sfnd1} 
\begin{equation}
	\label{MeanofWPL}
	\mu(n_{i},C) = C + \sum_{i}^{I} D_in_i.
\end{equation}	 
 
\subsection{Relationship between PEFs and Log-distance Models}
%Subsection Checked 24/01/2023
Based on the above description, the PEFs model is potentially suitable for any environment cases with raster maps. As the number of PEFs is random between different environments, the quantity $I$ can be increased for complicated areas and decreased for simplified areas. Specially, if the measurement area has only one PEF or we just simply set $I$ as one, the characterization in \eqref{WtdPL_LkTrm} will be specialized to the log-distance model, which is
\begin{equation}
	\label{LogdPL}
	PL = C + 10n\log_{10}(d/d_{0})+\Psi_{\sigma}  \qquad     d\ge d_{0},
\end{equation}	
where $n$ is the PLE and $C$ is the intercept value that represents the decibel path loss at close-in distance $d_{0}$. For path loss prediction, this equation is common practice to characterize path loss in dB scale as the power-law of distance $n$ plus a term due to shadow fading \cite{s2-1}. Because the shadow fading term is modeled by Gaussian distribution in dB scale, the path loss in this model can be expressed as a normally distributed value with the same standard deviation along with a distance-dependent mean, $N[\mu(d),{\sigma}^{2}]$, where \cite{sfnd1}
\begin{equation}
	\label{MeanofLgdPL}
	\mu(d) = C + 10n\log_{10}(d/d_{0}).
\end{equation}	 

\section{Extraction of Parametrized Environmental Factors}
%Section Checked 17:13 24/01/2023
\subsection{Region Classification of Raster Map}
%Subsection Checked 24/01/2023

\begin{figure}[t]
	\centering
	\begin{minipage}{.52\linewidth}

		\subfigure[]{\includegraphics[width=1.65in]{figure/totalmap}}
		%Raster Map of Partial Sheffield City
	\end{minipage}%
	\begin{minipage}{.52\linewidth}
	
		\subfigure[]{\includegraphics[width=1.65in]{figure/Classification}}
		%Region Classification Result of the Raster Map
	\end{minipage}
	\vspace{-10pt}
	\caption{The extraction of raster map in partial Sheffield City (a) Raster map of partial Sheffield City (b) Region classification result of the raster map.}
	\label{figrastermapSfd}

\end{figure}

Here we take Sheffield City as an example to help illustrate the process of extracting the PEFs, which is also the area of measured data (Reported in Section IV). The raster map shown in Fig. \ref{figrastermapSfd}(a) is taken from Google maps, which includes several types of environmental obstacles. Different colours on the map stand for different types of obstacles or the same with different using purposes (e.g., open spaces are set as light grey for near residential areas and red for near hospitals, respectively, in Fig. \ref{figrastermapSfd}(a)). Besides, the map scale requires a balance between providing maximum map area and including as much map detail as possible, and thus the map scale of 50 m in Google Maps is chosen. The square raster map shown in Fig. \ref{figrastermapSfd}(a) has a side of 2115 pixels, and the area is around 2.25 km$^{2}$.

To classify the regions, we use k-means clustering by computing the Euclidean distance of red, green and blue (RGB) values within the raster map \cite{km1}. Even though one region type may have two or more colours for multiple using purposes as mentioned before, it is obvious to classify afterwards. The classification results of the raster map are shown in Fig. \ref{figrastermapSfd}(b), concerning five typical regions with new colours, which are: light blue as $Building$, white as $Open\,Space$, yellow as $Lane$, green as $Wooded\,Area$ and pink as $Lake$.   

\subsection{Parameter Extraction from Truncated Data}
%Subsection Checked 24/01/2023
\begin{figure}[t]
\vspace{-12pt}	
	\centering
	\includegraphics[width = 1\linewidth]{figure/CurveFitting}
	\vspace{-22pt}
	\caption{The estimation results of least-squares (LS) and maximum likelihood (ML) for truncated data.} 
	\label{Cf&ML}

\end{figure}

The path loss in Sheffield shows a gradually increasing trend over the log-distance from Tx to Rx. Fig. \ref{Cf&ML} shows the scatter plot of the measured path loss data, along with two curves that stand for different curve fitting methods based on the log-distance path loss model. Initially, we use the least-squares (LS) to fit the curve \cite{s2-1}, which is shown as the blue curve in Fig. \ref{Cf&ML}. However, the PLE $\hat{n}$ and the standard deviation of the shadow fading term $\hat{\sigma}$ estimated by LS are biased. This is because the path loss data beyond 140 dB is missing, caused by the sensitivity limitation of the measurement device \cite{zey22}. To overcome this issue, we use truncated Gaussian distribution beforehand \cite{tcs1} and then re-estimate the fitting curve parameters by maximum likelihood (ML) \cite{ml1}. The red curve shows the result re-estimated by ML, along with the corresponding PLE and the standard deviation of shadow fading.

As stated previously, five typical regions are classified in Sheffield City, therefore, the characterization of the PEFs path loss in \eqref{WtdPL_LkTrm} can be further expressed as
\begin{equation}
	\label{WtdPLforSfd}
	PL = C + \sum_{i}^{5} D_in_i+\Psi_{\sigma},
\end{equation} 
with
\begin{equation}
	\label{MeanofWPLWtI_5}
	\mu_{\mathrm{s}}(n_{i},C) = C + \sum_{i}^{5} D_in_i.
\end{equation}	 
where the total region types $I$ in \eqref{WtdPL_LkTrm} equals five and $\mu_{\mathrm{s}}(n_{i},C)$ is the mean value in this case. Again, to define each PEF within the above equation, the same preprocess of truncated Gaussian distribution and ML are required. Because the PEFs path loss is normally distributed with an expectation of \eqref{MeanofWPL}, the probability density function (PDF) of \eqref{WtdPLforSfd} can be represented as

\begin{footnotesize}
	\vspace{-15pt}
	\begin{equation}
		\label{PDF}
		\begin{aligned}
			P(l_k;\mu_{\mathrm{s}}(n_{i},C),\sigma) =
			\left\{
			\begin{array}{ccl}
				\frac{1}{\sqrt{2\pi}\sigma}\frac{\mathrm{exp}\big(-\big(\frac{l_k-\mu_{\mathrm{s}}(n_{i},C)}{\sqrt{2}\sigma}\big)^{2}\big)}{1-\Phi\big(\frac{\mu_{\mathrm{s}}(n_{i},C)-L}{\sigma}\big)}  &{l_k<L}, \\
				0 & {\text{else}},
			\end{array} \right.
		\end{aligned}
	\end{equation}
\end{footnotesize}where $l_{k}$ is the PEFs path loss for $k$th data point in the measurement data, $\Phi$ is the cumulative distribution function (CDF) of the standard Gaussian distribution, $L$ is the truncated value on the right of path loss PDF, which is 140 dB shown in Fig. \ref{Cf&ML} \cite{tcs1}.

To estimate the quantities of the above equation by ML, the partial derivatives of the likelihood function for all the values are required, including  PEFs $n_i$, intercept $C$ and the standard deviation $\sigma$, which are \cite{ml1}
\begin{equation}
	\label{PD_ni}
	\begin{aligned}
		\frac{\partial F(n_i,C,\sigma)}{\partial n_i} =
		 \sum_{k=1}^{K} D_i& \bigg[ \frac{\mu_{\mathrm{s}}(n_{i},C)-l_k}{\sigma^{2}}\\ &-\frac{\mathrm{exp}\big(-\big(\frac{\mu_{\mathrm{s}}(n_{i},C)-L}{\sqrt{2}\sigma}\big)^{2}\big)}{\sigma\int_{\frac{\mu_{\mathrm{s}}(n_{i},C)-L}{\sigma}}^{\infty} \mathrm{exp}(-\frac{t^{2}}{2})dt} \bigg],
	\end{aligned}
\end{equation}  

\begin{equation}
	\label{PD_C}
	\begin{aligned}
		\frac{\partial F(n_i,C,\sigma)}{\partial C}  = \sum_{k=1}^{K} \bigg[ &\frac{\mu_{\mathrm{s}}(n_{i},C)-l_k}{\sigma^{2}}\\ 
		&-\frac{\mathrm{exp}\big(-\big(\frac{\mu_{\mathrm{s}}(n_{i},C)-L}{\sqrt{2}\sigma}\big)^{2}\big)}{\sigma\int_{\frac{\mu_{\mathrm{s}}(n_{i},C)-L}{\sigma}}^{\infty} \mathrm{exp}(-\frac{t^{2}}{2})dt} \bigg]
	\end{aligned}
\end{equation} 
and
\begin{equation}
	\label{PD_sig}
	\begin{aligned}
		\frac{\partial F(n_i,C,\sigma)}{\partial \sigma} &= \sum_{k=1}^{K} \bigg[ \frac{1}{\sigma}-\frac{(\mu_{\mathrm{s}}(n_{i},C)-l_k)^{2}}{\sigma^{3}} \\
		& + \frac{(\mu_{\mathrm{s}}(n_{i},C)-L) \mathrm{exp}\big(-\big(\frac{\mu_{\mathrm{s}}(n_{i},C)-L}{\sqrt{2}\sigma}\big)^{2}\big)}{ \sigma^{2} \int_{\frac{\mu_{\mathrm{s}}(n_{i},C)-L}{\sigma}}^{\infty} \mathrm{exp}(-\frac{t^{2}}{2})dt} \bigg],
	\end{aligned}
\vspace{10pt}
\end{equation}where $K$ is the number of total data points, the other values in the above equations are mentioned before. Basically, \eqref{PD_ni}-\eqref{PD_sig} are calculated by ML using the PDF in \eqref{PDF}, and the derivation is omitted here for it is not the key part in this design. Finally, we use the gradient descent based on \eqref{PD_ni}-\eqref{PD_sig} with suitable step size to find the optimal PEFs \cite{gd1}. 

\section{Measurement Verification}

In this section, we apply the PEFs model to the measurement area and compare with the log-distance model to verify its performance. We first present the collection method of the measurement data in Sheffield City. Then, the path loss for the given scenario in both the log-distance model and the PEFs model is computed, drawn on the raster map and compared with the measured data. Also, the performance comparison between two models is made and revealed by RMSE.
%, along with the details of the data we measured
\subsection{Collection of Measurement Data}
%To assess the quality of user experience (\textit{network coverage} in our case), 
%with a particular focus on Sheffield City Centre
We mapped Sheffield City Region through “drive test” and “walk test” experiments. Drive tests were performed with equipment, i.e., Rodhe\&Schwarz\textsuperscript{\textregistered}TSMA6 (autonomous mobile network scanner) and LILYGO\textsuperscript{\textregistered} TTGO T-Beam (field test device -FTD), installed on board an Urban Flows Observatory\footnote{Urban Flows Observatory. \url{https://urbanflows.ac.uk/}.}-owned Smart ForFour electric vehicle called Morca. Since walk tests included the FTD only, the analyses in this study focus solely on the FTD measurements. 
%since this study concentrates on the Centre,

At the time of data collection, the signal-to-noise ratio (SNR) and received signal strength indicator (RSSI) values are periodically recorded (at an average height of $1.2$ m from the ground level) and transmitted by the FTD (having an antenna gain of $2$ dBi) together with the respective GPS coordinates, showing the tracks covered during the tests. The signal was transmitted close to 868 MHz, and the measurement data were taken at distances ranging from several meters to 10 km. 

The University of Sheffield (TUoS) had seven gateways scattered around the Region. However, since the constraints of map scale (Reported in Section III), only the connections made via the gateway on TUoS' Hicks Building (53.381029, -1.4864733), serving the area of interest, are considered. The named building has a height of $30$ m, and the gateway it accommodates is a LORIX One, including a $4.15$ dBi omnidirectional antenna. 

For the terrain categories, Sheffield is often considered a hilly place, and the area under map constraints is hilly terrain with moderate-to-heavy tree densities, along with dense buildings. 
 
%Although Sheffield is often considered a hilly place, its Centre is pretty flat with light tree densities, and most of the data were reported from there. The hilly terrain with moderate-to-heavy tree densities around the Centre also has some instances in the whole collection and is therefore considered in the study.

\subsection{Parameter Results}

\begin{table}[ht]
	\caption{The Parameters of Environmental Factors in the Measurement Area}
	\centering
	\begin{tabular}{ccc}
		\toprule
		\textbf{Environmental Factors}  &\textbf{Variables} & \textbf{Values} \\
		\midrule
		Intercept          & $C$ & 74.95\\
		Building           & $n_1$ & 1.12\\
		Open Space         & $n_2$ & 1.74\\
		Lane               & $n_3$ & 2.38\\
		Wooded Area        & $n_4$ & 4.39\\
		Lake               & $n_5$ & 1.08\\
		Standard Deviation & $\sigma$ & 6.80\\
		\bottomrule
	\end{tabular}
	\vspace{-0pt}
\end{table}

As we set the close-in distance $d_0$ as 1 m in \eqref{LnMtx} and applied the PEFs model in the measurement area, the numerical values of PEFs are estimated and shown in Table I. Since the PEFs model divides the whole propagation into multiple subsections, those PEFs values reveal the impact of different environmental factors towards total propagation path loss, and the corresponding analysis are given as follows.

%Within the parameters in Table I, the most and least impact factors are $Wooded\,Area$ and $Lake$, respectively. Also, the analysis of these numerical values based on different environmental factors are given as follows.

%Based on the values in Table I, the environmental factors are divided into small, medium and large impact levels
\begin{itemize}
	\item{$Building$: The quantity $n_1$ is close to the PLE in the 3GPP channel model (InH Path loss with line of sight) and another PLE in a measured in-building case that are 1.73 and 1.57, respectively \cite{md7}, \cite{pefbd1}. }
	%Because the Tx position within the measurement is good for the signal to cover the most area, the quantity $n_1$ in this case is slightly lower than the other two in-building scenarios.
	\item{ $Open\,Space$: The numerical value $n_2$ is similar to the ML re-estimation result of linear PLE, which is 2.12 shown in Fig. \ref{Cf&ML}.}
	\item{$Lane$: For constantly moving vehicles (e.g., cars and buses) on lanes which cause reflection, diffraction and scattering during the measurement \cite{gol05}, the value of $n_3$ is slightly larger than 2.}
	\item{$Wooded\,Area$: The intensive scattering caused by trees leads to severe path loss so that the quantity of $n_4$ goes beyond 4.}
	\item{$Lake$: When propagating through lakes at 868 MHz, the Sommerfeld-Zenneck surface waves are generated which is 10-20 dB stronger than space waves\cite{pet15}, \cite{kin85}. That is, the signal decays as the quantity of $n_5$ because it propagates over the lake surface as a circle instead of radiating through the air as a sphere.} 
	%The 2-ray model with a PLE of 4 dB is considered for water surface, of which is close to the quantity of $n_5$ \cite{peflk1}, \cite{peflk2}.
\end{itemize}

\begin{figure}[t]
	\vspace{-5pt}
	\centering
	\subfigure[Comparison between simulation results of the PEFs model and measured data]{\includegraphics[width=3in]{figure/PLEFsWtMd}%
	}
	\centering
	\subfigure[Comparison between simulation results of the log-distance model and measured data]{\includegraphics[width=3in]{figure/PLLogDWtMd}%
	}
	\vspace{-2pt}
	\caption{Results for the measurement area in Sheffield.}
	\label{PLRst}
	\vspace{-5pt}
\end{figure}

\begin{comment} %Word with red color
\begin{enumerate}[label*=\textbf{\arabic*.},itemsep=0.5em]
	\item[$\bullet$] \textbf{Small Impact}
	\begin{enumerate}[label*=\textbf{\arabic*.}]
		\item [] $Building$: \textcolor[rgb]{1,0,0}{As mentioned in the previous subsection, the Tx position is good for the signal to cover the most area. Meanwhile, most buildings within the measurement are with low heights so that they cannot block the signal.} 
		\item [] $Lake$: \textcolor[rgb]{1,0,0}{When the path go through lakes, the line-of-sight signal and the reflection path caused by water surface dominate the propagation, so the impact of $Lake$ is the least one in this case.}
	\end{enumerate}
	\item[$\bullet$] \textbf{Medium Impact}
	\begin{enumerate}[label*=\textbf{\arabic*.}]
		\item [] $Open\,Space$: \textcolor[rgb]{1,0,0}{Since open spaces have pedestrians and some guide boards, some scattering and reflections may occur. That is, the impact of $Open\,Space$ towards path loss is medium.}
		\item [] $Lane$: \textcolor[rgb]{1,0,0}{For constantly moving vehicles on lanes which cause both scattering and reflection during the measurement, the impact of $Lane$ is a bit larger with respect to other environmental factors.}
	\end{enumerate}
	\item[$\bullet$] \textbf{Large Impact}
    \begin{enumerate}[label*=\textbf{\arabic*.}]
	\item []$Wooded\,Area$: \textcolor[rgb]{1,0,0}{The intensive scattering caused by trees leads to severe path loss so that the $Wooded\,Area$ blocks most signal power.}

    \end{enumerate}

\end{enumerate}
\end{comment} 
 
 \begin{figure}[h]
 	\vspace{-10pt}	
 	\centering
 	\includegraphics[width = 0.9\linewidth]{figure/CDFLogD}
 	\caption{CDF of the absolute error for PEFs and log-distance model with respect to measurement.} 
 	\label{CDF}
 \end{figure}
 
Based on the numerical PEFs in Table I, we simulate the path loss for the entire raster map and compare the predicted values with the measured data points. In Fig. \ref{PLRst}, the path loss of each data point in Sheffield is shown on the raster map. Each square represents one Rx in the measurement area and the white circle in the middle represents the Tx in this dataset. Note that the colour on the map and within the square are the values of predicted path loss and measured path loss, respectively. The simulation result in Fig. \ref{PLRst}(a) shows the predicted path loss when applying the PEFs model with the numerical values in Table I, and the path loss, in this case, increases over distance with different trends as the Tx-Rx line path intersects different environmental factors. When applying the log-distance model to predict the path loss, the results of Rx near the Tx are well predicted as shown in Fig. \ref{PLRst}(b). As the distance goes further, however, the prediction error increases gradually, especially the path loss towards the east, which suffered about a 10 dB difference with the measured data. This issue is mitigated in the PEFs model, wherein the prediction is much closer to the measurement. 

To illustrate the performance difference more visually, the CDF of the absolute error between prediction and measurement is drawn and shown in Fig. \ref{CDF}. Apparently, the CDF results show that the performance of the PEFs model is better than the log-distance model, and the RMSE in dB scale reveals the same, wherein 8.18 dB in the log-distance model and 6.78 dB in the PEFs model.   

\section{Conclusion}
We have presented the PEFs path loss model considering environmental factors. This PEFs model predicts the path loss by one characterization in dB scale with multiple PLEs. We then applied the PEFs model in a suburban area and compare it with the widely used log-distance model. Simulation results of path loss verify the performance of the PEFs model, with a RMSE decrease of 1.4 dB compared with the log-distance model. The PEFs model provides fast prediction results for path loss with low complexity and it is suitable for areas with raster map data. 
 

\begin{thebibliography}{00}
	%Cite Checked 16:15 23/01/2023
	\bibitem{rt1} Z. Yun and M. F. Iskander, “Ray tracing for radio propagation modeling: Principles and applications,” \textit{IEEE Access}, vol. 3, pp. 1089–1100, 2015.
	
	%Cite Checked 16:15 23/01/2023
	\bibitem{rt2} A. Navarro \textit{et al.}, “A proposal to improve ray launching techniques,” \textit{IEEE Antennas Wirel. Propag. Lett.}, vol. 18, no. 1, pp. 143–146, Jan. 2019.
	
	%Cite Checked 16:15 23/01/2023
	\bibitem{ho1} T. Mahmood \textit{et al.}, “The effect of antenna height on the performance of the Okumura/Hata model under different environments propagation,” in \textit{Proc. IEEE International Conference on Intelligent Technologies (CONIT)}, 2021, pp. 1-4.
	
	%Cite Checked 16:15 23/01/2023
	\bibitem{ho2} T. S. Rapport, \textit{Wireless Communications: Principles and Practice}.  New Jersey: Prentice Hall, 1996. 
	
	%Cite Checked 16:15 23/01/2023
	\bibitem{ep3} V. G. Drozdova and R. V. Akhpashev, “Ordinary least squares in COST 231 Hata key parameters optimization base on experimental data,” in \textit{Proc. IEEE International Multi-Conference on Engineering, Computer and Information Sciences (SIBIRCON)}, 2017, pp. 236–238.
	
	%Cite Checked 16:15 23/01/2023
    \bibitem{md1} M. Peter \textit{et al}., \textit{Measurement campaigns and initial channel models for preferred suitable frequency ranges, Deliverable D2.1,} 2016. [Online]. Available: https://ec.europa.eu/research/participants/documents/downlo-$\\$adPublic?documentIds=080166e5a7a6b182$\&$appId=PPGMS

    %Cite Checked 16:15 23/01/2023
    \bibitem{md2} V. Nurmela \textit{et al}., \textit{METIS channel models, Deliverable D1.4,} 2015. [Online]. Available: https://www.metis2020.com/wp-content/uploads/delive-$\\$rables/METIS$\_$D1.4$\_$v1.0.pdf

    %Cite Checked 16:17 23/01/2023
    \bibitem{md3} A. Maltsev \textit{et al}., \textit{MiWEBA D5.1: Channel modeling and characterization, Deliverable D5.1,} 2014. 
    [Online]. Available: https://www.miweba.$\\$eu/wp-content/uploads/2014/07/MiWEBA$\_$D5.1$\_$v1.011.pdf
    
    %Cite Checked 16:17 23/01/2023
    \bibitem{md4} S. Jaeckel \textit{et al.}, \textit{QuaDRiGa—Quasi deterministic radio channel generator, user manual and documentation, V2.0.0}, 2017. [Online]. Available:$\\$ https://quadriga-channel-model.de/wp-content/uploads/2017/08/quadrig-$\\$a$\_$documentation$\_$v2.0.0-664.pdf
    
    %Cite Checked 16:17 23/01/2023
    \bibitem{md5} S. Wu \textit{et al.}, “A general 3-D non-stationary 5G wireless channel model,” \textit{IEEE Trans. Commun.}, vol. 66, no. 7, pp. 3065-3078, Jul. 2018.
    
    %Cite Checked 16:20 23/01/2023
    \bibitem{md6} C. -X. Wang \textit{et al.}, “A survey of 5G channel measurements and models,” \textit{IEEE Commun. Surveys Tuts.}, vol. 20, no. 4, pp. 3142-3168, Aug. 2018.
 
    %Cite Checked 16:21 23/01/2023
    \bibitem{md7} G. R. A. N. W. Group \textit{et al.}, \textit{Study on channel model for frequencies f-$\\$rom 0.5 to 100 GHz, V16.1.0}, 2020. [Online]. Available: https://www.et-$\\$si.org/deliver/etsi$\_$tr/138900$\_$138999/138901/16.01.00$\_$60/tr$\_$138901v1-$\\$60100p.pdf
   
    %Cite Checked 16:22 23/01/2023
    \bibitem{md8} ITU-R SG05, \textit{Draft new report ITU-R m. [IMT-2020.eval]}, 2017. [On-$\\$line]. Available: https://www.itu.int/md/R15-SG05-C-0057/en
    
    %Cite Checked 16:24 23/01/2023
    \bibitem{md9} N. Docomo, \textit{5G channel model for bands up to 100 GHz, V2.0}, 2016. [Online]. Available: http://www.5gworkshops.com/2016/5G$\_$Channel$\_$M$\\$odel$\_$for$\_$bands$\_$up$\_$to100$\_$GHz(2015-12-6).pdf
   
    %Cite Checked 16:24 23/01/2023
    \bibitem{s2-1} V. Erceg \textit{et al.}, “An empirically based path loss model for wireless channels in suburban environments,” \textit{IEEE J. Select. Areas Commun.}, vol. 17, no. 7, pp. 1205–1211, Jul. 1999.
    
    %Cite Checked 16:25 23/01/2023
    \bibitem{sfnd1}  C. Gustafson \textit{et al.}, “Statistical modeling and estimation of censored pathloss data,” \textit{IEEE Wireless Commun. Lett.}, vol. 4, no. 5, pp. 569–572, Oct. 2015.
   
    %Cite Checked 16:33 23/01/2023
    \bibitem{km1} K. P. Sinaga and M. Yang, “Unsupervised K-means clustering algorithm,” \textit{IEEE Access}, vol. 8, pp. 80716–80727, 2020.
    
    %Cited for Outage mentioned in Zeyang
    \bibitem{zey22} Z. Li \textit{et al.}, “Coverage analysis of multiple transmissive RIS-aided outdoor-to-indoor mmWave networks,” \textit{IEEE Trans. Broadcast.}, vol. 68, no. 4, pp. 935–942, Aug. 2022.
    
    %Cite Checked 16:34 23/01/2023
    \bibitem{tcs1} A. C. Cohen, \textit{Truncated and Censored Samples: Theory and Applications.} New York: Marcel Dekker, 1991.
   
    %Cite Checked 16:39 23/01/2023   
    \bibitem{ml1} L. L. Cam, “Maximum likelihood: An introduction,” \textit{Int. Stat. Rev.}, vol. 58, no. 2, pp. 153–171, Aug. 1990.
    
    %Cite Checked 16:39 23/01/2023
    \bibitem{gd1}  S. Ruder, “An overview of gradient descent optimization algorithms,” 2017. [Online]. Available: https://arxiv.org/abs/1609.04747
    
    %Cite Checked 16:41 23/01/2023
    \bibitem{pefbd1} J. Miranda \textit{et al.}, “Path loss exponent analysis in wireless sensor networks: Experimental evaluation,” in \textit{Proc. 11th IEEE International Conference on Industrial Informatics (INDIN)}, 2013, pp. 54-58.
   
    %Cite Checked 16:42 23/01/2023   
    \bibitem{gol05} A. Goldsmith, \textit{Wireless Communications}. Cambridge: Cambridge University Press, 2005.
 
    %Cite Checked 16:43 23/01/2023   
    \bibitem{pet15} G. Peterson, “The application of electromagnetic surface waves to wireless energy transfer,” in \textit{Proc. IEEE Wireless Power Transfer Conference (WPTC)}, 2015, pp. 1-4.

    %Cite Checked 16:45 23/01/2023   
    \bibitem{kin85} R. King, “Electromagnetic surface waves: New formulas and applications,” \textit{IEEE Trans. Antennas Propag.}, vol. 33, no. 11, pp. 1204–1212, Nov. 1985.

\end{thebibliography}


\vfill

\end{document}