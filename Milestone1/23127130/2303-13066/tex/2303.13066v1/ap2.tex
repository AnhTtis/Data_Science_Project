%\documentclass[a4paper,12pt]{article}
\documentclass[hyper]{JHEP}
\pdfoutput=1
%\usepackage{jheppub}
\usepackage[autostyle]{csquotes}
\usepackage{fixltx2e}
\usepackage{float}
\usepackage{graphicx,amssymb,bm,latexsym,amsmath}
\usepackage{subfigure,float,psfrag,rotating}
\usepackage{epstopdf}
\usepackage[section]{placeins}
\usepackage{capt-of}
\usepackage[autostyle]{csquotes}
\MakeOuterQuote{"}
%\usepackage{hyperref}\begin{equation}
%\hypersetup{colorlinks=true}
%\usepackage{color}
\usepackage{caption}
%\usepackage{tabu}
\newcommand{\sn}{{\rm sn}}
\newcommand{\cn}{{\rm cn}}
\newcommand{\dn}{{\rm dn}}
\newtheorem{theorem}{Theorem}[section]
\newtheorem{lemma}[theorem]{Lemma}

%\theoremstyle{definition}
\newtheorem{definition}[theorem]{Definition}
\newtheorem{example}[theorem]{Example}
\newtheorem{xca}[theorem]{Exercise}
\newcommand{\lso}[2]{#1\left(\textcolor{#2}{^{\line(1,0){20}}}\right)}
\newcommand{\lda}[2]{#1\left(\textcolor{#2}{^{_{\bf{------}}}}\right)}
\newcommand{\loo}[2]{#1\left(\textcolor{#2}{^{_{\bf{ooo}}}}\right)}
%\theoremstyle{remark}
\newtheorem{remark}[theorem]{Remark}
%\usepackage{subfig}
%\usepackage[countmax]{subfloat}
%\usepackage{subcaption}
\usepackage[toc,page]{appendix}










%\documentclass[hyper]{JHEP}
%\documentclass[a4paper,12pt]{article}
%\usepackage{jheppub}
%\usepackage{graphicx,amssymb,bm,latexsym,amsmath}
%\usepackage{epstopdf}
%\usepackage{caption}
%\usepackage{mathtools}
%\usepackage{subfig}
%\usepackage[countmax]{subfloat}
%\usepackage{subcaption}
%\usepackage[toc,page]{appendix}
%\newcommand{\bej}[1]{ \begin{equation}\label{#1} }
%\newcommand{\eej}{\end{equation}}
%\newcommand{\beaj}[1]{\begin{eqnarray}\label{#1} }
%\newcommand{\eeaj}{\end{eqnarray}}
%\newcommand{\eq}[1]{(\ref{#1})}
%\def\ZZZ{{\hskip-3pt\hbox{ Z\kern-1.6mm Z}}}
%\def\zzz{{\hskip-3pt\hbox{ z\kern-1mm z}}}
%\newcommand{\rrho}{r}
%\newcommand{\RRR}{{\rm R \hskip -7pt R}}
%\newcommand{\bX}{\bar X}
%\newcommand{\te}{\tilde e}
%\newcommand{\tp}{\tilde{\phi}}
%\newcommand{\ttp}{\tilde{\tilde{\phi}}}
%\newcommand{\tvp}{\tilde \varphi}
%\newcommand{\bx}{\bar x}
%\newcommand{\bw}{\bar w}
%\newcommand{\tx}{\wt x}
%\newcommand{\bF}{\bar F}
%\newcommand{\bbF}{{\bf F}}
%\newcommand{\bN}{{\bf N}}
%\newcommand{\bbb}{{\bar b}}
%\newcommand{\gam}{\tau}
%\newcommand{\eps}{\epsilon}
%\newcommand{\ra}{\rangle}
%\newcommand{\la}{\langle}
%\newcommand{\T}{\chi_{T}(k)}
%\newcommand{\Tm}{\chi_{T}(k')}
%\newcommand{\Cn}{{\cal C}_n}
%\newcommand{\vp}{\varphi}
%\newcommand{\ve}{\varepsilon}
%\newcommand{\vt}{\vartheta}
%\newcommand{\tl}{\lambda}
%\newcommand{\dt}{(\vec \nabla T)^2}
%\newcommand{\hp}{{\wh\Phi}}
%\newcommand{\hq}{{\wh Q_B}}
%\newcommand{\he}{{\wh\eta_0}}
%\newcommand{\ha}{{\wh{A}}}
%\newcommand{\lllb}{\Bigl\langle\Bigl\langle}
%\newcommand{\rrrb}{\Bigr\rangle\Bigr\rangle}
%\newcommand{\tf}{\wt f}
%\newcommand{\sss}{{\cal L}_{av}}
%\newcommand{\we}{\wedge}
%\newcommand{\ot}{\otimes}

%\newcommand{\lm}{\lambda}
%\newcommand{\lb}{\bar \lambda}

%\newcommand{\bra}[1]{\langle #1|}
%\newcommand{\ket}[1]{|#1\rangle}

%\newcommand{\vv} {\bar v}
%\newcommand{\uu} {\bar u}
%\newcommand{\K}{{\rm K_1}}
%\newcommand{\Kt}{{\rm \widetilde K_1}}

%\newcommand{\B}{b'}
%\newcommand{\C}{c'}
%\newcommand{\bB}{\bar b'}
%\newcommand{\Bu}{B_{\vec u}}
%\newcommand{\VV}{{\cal V}}
%\newcommand{\BB}{{\cal B}}
%\newcommand{\II}{{\cal I}}
%\newcommand{\AAA}{{\cal A}}
%\newcommand{\GG}{{\cal G}}
%\newcommand{\KK}{{\cal K}}

%\newcommand{\fff}{{\bf f}}
%\newcommand{\ccc}{{\bf c}}
%\newcommand{\FF}{{\cal F}}
%\newcommand{\HH}{{\cal H}}
%\newcommand{\MM}{{\cal M}}
%\newcommand{\CC}{{\cal C}}
%\newcommand{\DD}{{\cal D}}
%\newcommand{\bC}{{\bf C}}
%\newcommand{\OO}{{\cal O}}
%\newcommand{\QQ}{{\cal Q}}
%\newcommand{\PP}{{\cal P}}
%\newcommand{\EE}{{\cal E}}
%\newcommand{\ZZ}{{\cal Z}}
%\newcommand{\LL}{{\cal L}}
%\def\pint{{-\!\!\!\!\!\!\int}}

%\newcommand{\rrr}{\rangle\rangle}
%\newcommand{\square}{\Box}
%\newcommand{\half}{{1\over 2}}
%\newcommand{\wt}{\widetilde}
%\newcommand{\wh}{\widehat}
%\newcommand{\wc}{\check}
%\newcommand{\wb}{\bar}
%\newcommand{\bd}{\bar{\rm D}}
%\newcommand{\RR}{{\cal R}}
%\newcommand{\NN}{{\cal N}}
%\newcommand{\TT}{{\cal T}}
%\newcommand{\bg}{\bar g}
%\newcommand{\bb}{\bar b}
%\newcommand{\bT}{\bar \Theta}
%\newcommand{\SSS}{{\cal S}}
%\newcommand{\tlx}{\left(\tilde \lambda ; X^0(0) \right)}
%\newcommand{\al}{\alpha}

%\newcommand{\omk}{\omega_n(\vec k)}
%\newcommand{\onk}{\omega^{(N)}_{\vec k_\perp}}
%\newcommand{\N}{\frac{m_{2}}{k_{2}}-\frac{m_{1}}{k_{1}}}
%\newcommand{\NL}{\frac{m_{2}}{k_{2}}+\frac{m_{1}}{k_{1}}}

%\newcommand{\be}{\begin{equation}}
%\newcommand{\ee}{\end{equation}}
%\newcommand{\ben}{\begin{eqnarray}\displaystyle}
%\newcommand{\een}{\end{eqnarray}}
%\newcommand{\refb}[1]{(\ref{#1})}
%\newcommand{\p}{\partial}
%\newcommand{\III}{{\cal I}}
%\newcommand{\sectiono}[1]{\section{#1}\setcounter{equation}{0}}
%\renewcommand{\theequation}{\thesection.\arabic{equation}}
%\renewcommand{\theequation}{\arabic{equation}}
%\def\one{{\hbox{ 1\kern-.8mm l}}}
%\def\zero{{\hbox{ 0\kern-1.5mm 0}}}


%\def \ol{\overline}
%\def\be{\begin{equation}}       
%\def\ee{\end{equation}}         

%\def\bea{\begin{eqnarray}}      
%\def\eea{\end{eqnarray}}
                                
%\def\ba{\begin{array}}
%\def\ea{\end{array}}
%\def\bd{\begin{displaymath}}
%\def\ed{\end{displaymath}}

%\def\eq{\begin{equation}}
%\def\eqe{\end{equation}}
%\def\eqa{\begin{eqnarray}}
%\def\eqae{\end{eqnarray}}
%\def\ena{\end{eqnarray}} 
%\def \ol{\overline}
%\def\ap{{\alpha^{\prime}}}
%\def\al{{\alpha^{\prime}}}
%\def\aap{{a^{\prime}}}
%\def\eg{{\it e.g.~}}
%\def\ie{{\it i.e.~}}
%\def\nn{\nonumber}
%\def\Tr{{\rm Tr}}
%\def\tr{{\rm tr}}

%\def\tz{\tilde{z}}
%\def\ra{\rangle}
%\def\la{\langle}
%\def\ud{{\mathrm{d}}}
%\def\unit{1 \hskip-.3em \raise2pt\hbox{$ \scriptstyle |$ } }

\def\a{\alpha}
\def\b{\beta}
\def\c{\gamma} 
\def\d{\delta}
\def\e{\epsilon}           % Also, \varepsilon
\def\f{\phi}               %      \varphi
\def\vf{\varphi}  \def\tvf{\tilde{\varphi}} 
\def\g{\gamma}
\def\h{\eta}   
\def\i{\iota}
\def\j{\psi}
\def\k{\kappa}                    % Also, \varkappa (see below)
\def\l{\lambda}
\def\m{\mu}
\def\n{\nu}
\def\o{\omega}  \def\w{\omega}
\def\p{\pi}                % Also, \varpi
\def\q{\theta}  \def\th{\theta}                  %     \vartheta
\def\r{\rho}                                     %     \varrho
\def\s{\sigma}                                   %     \varsigma
\def\t{\tau}
\def\u{\upsilon}
\def\x{\xi}
\def\z{\zeta}
\def\D{\Delta}
\def\F{\Phi}
\def\G{\Gamma}
\def\J{\Psi}
\def\L{\Lambda}
\def\O{\Omega}  \def\W{\Omega}
\def\P{\Pi}
\def\Q{\Theta}
\def\S{\Sigma}
\def\U{\Upsilon}
\def\X{\Xi}
\def\del{\partial}              
% overwritten by \nabla

% Calligraphic letters

\def\ca{{\cal A}}
\def\cb{{\cal B}}
\def\cc{{\cal C}}
\def\cd{{\cal D}}
\def\ce{{\cal E}}
\def\cf{{\cal F}}
\def\cg{{\cal G}}
\def\ch{{\cal H}}
\def\ci{{\cal I}}
\def\cj{{\cal J}}
\def\ck{{\cal K}}
\def\cl{{\cal L}}
\def\cm{{\cal M}}
\def\cn{{\cal N}}
\def\co{{\cal O}}
\def\cp{{\cal P}}
\def\cq{{\cal Q}}
\def\car{{\cal R}}
\def\cs{{\cal S}}
\def\ct{{\cal T}}
\def\cu{{\cal U}}
\def\cv{{\cal V}}
\def\cw{{\cal W}}
\def\cx{{\cal X}}
\def\cy{{\cal Y}}
\def\cz{{\cal Z}}

%%%%%%%%%%%%%%%%%%%%%%%%%%%%%%%%%%%%%%%%%%%%%%%%%%%%%%%%%%%%%

% Umut needs

%\def\bd{\begin{displaymath}}
%\def\ed{\end{displaymath}}
%\def\quart{\frac14}
\def\6{\partial}
%\def\N4{{\cal N}=4}
%\def\lab{\label}
%\def\bq{\bar{q}}
%%%%%%%%%%%%%%%%%%%%%%%%%%%%%%%%%%%%%%%%%%%%%%%%%%%%%%%%%%%%%%


%\def\half{{1 \over 2}}

%\def\Bf#1{\mbox{\boldmath $#1$}}       % bold Greek
%\def\Sf#1{\hbox{\sf #1}}               % sans serif "
% Math symbols

%\def\bop#1{\setbox0=\hbox{$#1M$}\mkern1.5mu
        %\vbox{\hrule height0pt depth.04\ht0
        %\hbox{\vrule width.04\ht0 height.9\ht0 \kern.9\ht0
        %\vrule width.04\ht0}\hrule height.04\ht0}\mkern1.5mu}
%\def\Box{{\mathpalette\bop{}}}                        % box
%\def\pa{\partial}                              % curly d
%\def\de{\nabla}                                       % del
%\def\dell{\bigtriangledown} % hi ho the dairy-o
%\def\su{\sum}                                         % summation
%\def\pr{\prod}                                        % product
%\def\iff{\leftrightarrow}                      % <-->
%\def\conj{{\hbox{\large *}}} % complex conjugate
%\def\lconj{{\hbox{\footnotesize *}}}          % little "
%\def\dg{\sp\dagger} % hermitian conjugate
%\def\ddg{\sp\ddagger} % double dagger

%\def\>{\rangle} %right angle

%\def\<{\langle} %left angle
%\def\Dsl{D \hskip-.6em \raise1pt\hbox{$ / $ } }
%\def\to{\rightarrow}
%\def\tf{\tilde{\f}}

%\def\pa{\partial}
%\def\del{\nabla}
%\def\delbar{\bar{\nabla}}
%\def\+{\oplus}


%\def\xx{\times}
%\newcommand{\sm}[1]{\mbox{\scriptsize #1}} 
%\newcommand{\tnnn}[1]{\mbox{\tiny #1}} 
%\def\GN{G_{\mbox{\tnnn N}}}
%\def\de{\mbox{d}} 
%\renewcommand{\theequation}{\thesection.\arabic{equation}}
%\def\nonu{\nonumber \\{}}
%\def\half{{1 \over 2}} 
%\def\Tr{{\rm Tr}\, }
%\def\tB{\tilde{B}}
%\def\tb{\tilde{b}}
%\def\as2{AdS_3\times S^3_1 \times S^3_2}



\title{\boldmath A new formulation of Galilean relativistic Proca theory}
%\preprint{yymm.nnnn[hep-th]}
\author{Rabin Banerjee \footnote{DAE Raja Ramanna fellow}~, ~Soumya Bhattacharya \\
Department of Astrophysics and High energy physics,\\
S. N. Bose National Centre  for Basic Sciences,\\ JD Block, Sector III, Salt Lake City, Kolkata -700 106, India\\
Email: \email{rabin@bose.res.in, soumya557@bose.res.in}}
%\author{Rabin Banerjee${}^{a}$,} 
%\author{Soumya Bhattacharya${}^{a}$,}
%\affiliation{${}^{a}$ Department of Astrophysics and High energy physics\\
%S. N. Bose National Centre 
%for Basic Sciences, JD Block, Sector III, Salt Lake City, Kolkata -700 106, India}
%\emailAdd{soumya557@bose.res.in, soumya557@gmail.com} 
\vskip .2in

\abstract{ In this paper, we discuss Galilean relativistic Proca theory in detail. We first provide a set of mapping relations, derived systematically, that connect the covariant and contravariant vectors in the Lorentz relativistic and Galilean relativistic formulations. Exploiting this map, we construct the two limits of Galilean relativistic Proca theory from usual Proca theory in the potential formalism for both contravariant and covariant vectors which are now distinct entities. An action formalism is thereby derived from which the  field equations are obtained and their internal consistency is shown.  Next we construct Noether currents and show their on-shell conservation. We introduce analogues of Maxwell's electric and magnetic fields and recast the entire analysis in terms of these fields. Explicit invariance under Galilean transformations is shown for both electric/magnetic limits. We then move to discuss Stuckelberg embedded Proca model in the Galilean framework.}

\keywords{Proca theory,  non-relativistic limit}
\begin{document}
\section{Introduction}
\noindent Works on non-Lorentzian physics have gained a considerable attention recently. Non-relativistic limit or Galilean relativistic limit of various field theories and gauge theories have been studied extensively.  These non-relativistic theories have found applications in a wide variety of fields including but not restricted to holography \cite{Taylor}, non-relativistic diffeomorphisms (NRDI) \cite{andreev1, andreev2, jensen, rb1, rb2}, 
condensed matter systems \cite{Pal}, fluid dynamics \cite{Jain, rb3}, gravitational waves \cite{Morand}. This formulation is tricky and quite different from the usual relativistic
case. Covariance in non-relativistic physics is subtle due to the absolute nature of time. The lack of a single non-degenerate metric in the non-relativistic case poses some additional difficulties. The study of Galilean relativistic theories was initiated by Le Bellac and Levy Leblond \cite{Leblond} back in 1970's. Their main focus was on Galilean electromagnetism which had many physical applications \cite{Germain}. The experiments of Rowland, Vasilescu, Karpen, Roentgen, Eichenwald and others \cite{Germain} which were previously interpreted or understood using special relativity found a more eloquent explanation using Galilean electromagnetism since it was perfectly suited to study electrodynamics of continuous media at low velocities. It may also be recalled that the results in \cite{Leblond} were attained primarily by physical principles based on gauge and galilean invariances. In case of non-gauge theories such an approach would be untenable, or at least difficult to formulate.\\
\indent An alternative approach would be to construct the Galilean relativistic version by taking an appropriate limit of the corresponding relativistic theory. Such attempts were done in \cite{Duval}, using group contraction techniques. Likewise in \cite{Mehra1, Mehra3}, scaling relations among potentials were introduced to obtain equations of motion in Galilean electrodynamics from Maxwell electrodynamics. Recently we \cite{Bhattacharya} have developed a method where the Galilean relativistic version of any Lorentz relativistic vector theory is constructed following a structured algorithm. It does not depend whether the vector theory is a gauge or non-gauge theory. \\
\indent In this paper, we are interested in Proca theory which describes a massive spin-1 field in Minkowski spacetime \cite{Proca}. The presence of the mass term actually breaks gauge invariance of the lagrangian. Proca theory plays a key role in many areas of fundamental physics, including experimental physics. For example it plays an important role in theoretical framework for experimental studies aiming at the determination of upper bounds for the mass of the photon \cite{Luo}, \cite{Nieto}, and other different areas of fundamental physics \cite{Sampaio, Dvali, Tomaschitz}. Apart from these there is a recent resurgence of the study of Proca model from different aspects. For example there are studies in the context of gravitational waves \cite{Radu1}, gravitational lensing and black hole (BH) shadows \cite{Radu2}, alternative theories of gravity \cite{Demir}. There has been a prescription to generalise the Proca action which gives rise to generalised Proca model \cite{Heisenberg}. This generalised Proca model has various applications in different areas of fundamental physics \cite{Said, Pichet}. There is also a non-linear Proca model called Proca-Nuevo model \cite{derham}. So given this importance and applications of Proca model we believe the study of non-relativistic limits of various Proca models will be interesting. Our current work is the first step towards that direction.  \\
\indent The construction of the cherished model follows by explicitly providing maps that connect the relativistic with non-relativistic vectors. This is done for both covariant and contravariant vectors since these are distinct entities in the Galilean theory, not being connected by any metric. Also for each component, there are maps corresponding to electric and magnetic limits - the two limits in Galilean relativistic physics. Using these maps the lagrangian for Galilean relativistic Proca model is constructed from its relativistic counterpart, both for electric and magnetic limits, and its consequences are examined. \\
\indent The paper is organised as follows: in section \ref{sec2} we derive mapping relations between relativistic and non-relativistic vectors for electric and magnetic limit for both contravariant and covariant vectors. It should be pointed out that, for contravariant coordinate vectors, only the electric limit can be physically realised since this limit corresponds to large timelike vectors compared to spacelike vectors. For realising the magnetic limit (large spacelike vectors) in this case, recourse has to be taken to a different set of Lorentz transformations introduced by Sengupta \cite{Sengupta}, which is explained here. Also for the covariant sector roles of the electric and magnetic fields are reversed. This has to be kept in mind when using either the conventional Lorentz transformations or that of Sengupta's \cite{Sengupta}. In section \ref{sec3} we derive the non-relativistic lagrangian for both limits and write down the equations of motion. These are shown to be compatible with the equations derived directly from the relativistic Proca equations. In section \ref{sec4} we introduce the galilean electric and magnetic fields and write down the Galilean Proca equations. The Noether currents and their corresponding conservations are discussed in section \ref{sec5}. We now move to discuss the gauge invariant version of Proca theory, obtained by Stuckelberg embedding, and perform its detailed Galilean analysis. This has been discussed in section \ref{sec6}. Finally we conclude in section \ref{sec7}.
\section{Maps relating Lorentz and Galilean vectors} \label{sec2}
\noindent Here we derive a certain scaling between special relativistic and Galilean relativistic quantities. As we know there exists two types of such limits for the vector quantities namely electric and magnetic limits. So first let us consider the contravariant vectors. Let us consider a the generic Lorentz transformation with the boost velocity as $u^i$:
\begin{equation}
    x'^0 = \gamma x^0 - \frac{\gamma u_i}{c} x^i
    \label{t1}
\end{equation}
\begin{equation}
    x'^i = x^i - \frac{\gamma u^i}{c}x^0 + (\gamma -1 )\frac{u^i u_j}{u^2}x^j
    \label{t2}
\end{equation}
where $\gamma = \frac{1}{\sqrt{1-\frac{u^2}{c^2}}}$. Under such Lorentz transformations a contravariant vector changes as
\begin{equation*}
    V'^{\mu} = \frac{\6 x'^{\mu}}{\6 x^{\nu}} V^{\nu}
\end{equation*}
We can write them component-wise as (also considering $u<<c$, so $\gamma \to 1$)
\begin{equation}
    V'^0 = V^0 - \frac{u_j}{c}V^j
    \label{contra1}
\end{equation}
\begin{equation}
    V'^i = V^i - \frac{u^i}{c} V^0
    \label{contra2}
\end{equation}
We next provide a map that relates the Lorentz vectors with their Galilean counterparts. 
\footnote{Notation: Here relativistic vectors are denoted by capital letters ($V^0, V^i$ etc) and Galilean vectors are denoted by lowercase letters ($v^0, v^i$ etc).} 
\begin{equation}
    V^0 = c v^0, \,\,\,\, V^i = v^i
    \label{contrael}
\end{equation}
This particular map corresponds to the case $\frac{V^0}{V^i} = c ~\frac{v^0}{v^i}$ in the $c \to \infty$ limit. This yields largely timelike vectors and is called 'electric limit'.
Now using eqn \ref{contrael} in eqns \ref{contra1} and \ref{contra2} we get
\begin{equation}
v'^0 = v^0
\label{v1}
\end{equation}
\begin{equation}
    v'^i = v^i - u^i v^0
    \label{v2}
\end{equation}
These equations define the usual galilean transformations. To see this in the context of coordinates, we revert to (\ref{t1}, \ref{t2}) and consider the $u^2 << c^2$ limit so that $\gamma \to 1$ and we get,
\begin{equation}
    t' = t, \,\,\, x'^i = x^i - u^i t \label{a1}
\end{equation}
which is the exact analogue of (\ref{v1}, \ref{v2}).\\
\indent Note that we can write (\ref{v1}, \ref{v2}) as a single matrix equation, 
\begin{equation}
\begin{pmatrix}
v'^0 \\ v'^i
\end{pmatrix} 
= \begin{pmatrix}
1 & 0\\ -u^i & 1
\end{pmatrix}
\begin{pmatrix}
v^0 \\ v^i
\end{pmatrix}
\label{V1}
\end{equation}
We next consider the magnetic limit which corresponds to largely spacelike vectors
\begin{equation}
    V^0 = -\frac{v^0}{c}, \,\,\,\, V^i = v^i
    \label{contramag}
\end{equation}
Now using \ref{contramag} in \ref{contra1} and \ref{contra2} we get

\begin{equation}
    v'^0 = v^0 + u_j v^j
    \label{v3}
\end{equation}
\begin{equation}
    v'^i = v^i
    \label{v4}
\end{equation}
which is again a galilean transformation, although the corresponding group is not an invariance group of classical mechanics. The magnetic (or large spacelike) limit cannot be visualised in the context of coordinates obeying the Lorentz transformations (\ref{t1}, \ref{t2}). It can, however, be understood from a different Lorentz transformation introduced by Sengupta \cite{Sengupta}.
\begin{equation}
    x'^i = \Bar{\gamma} \Big( x^i - \frac{c^2  \omega^i }{\omega^2} t \Big), \,\,\,\,\, ~~~~~t' = \Bar{\gamma} \Big(t - \frac{x^i \omega_i}{\omega^2} \Big) \label{a2}
\end{equation}
where $\Bar{\gamma} = \frac{1}{\sqrt{1- \frac{c^2}{\omega^2}}}$. The transformations are meaningful (real) when $\omega^2 > c^2$ and satisfy the Lorentz invariance condition
\begin{equation}
    x'^{i 2} - c^2 t'^2 = x^{i 2} - c^2 t^2
\end{equation}
Taking the $\omega^2 >> c^2$ limit ($\Bar{\gamma} = 1$) in \ref{a2} yields,
\begin{equation}
    x'^i = x^i, \,\,\,\,\, ~~~~~t' = t - \frac{x^i \omega_i}{\omega^2} \label{a4}
\end{equation}
which is the analogue of (\ref{v3}, \ref{v4}). The roles of time and space in \ref{a1} and \ref{a4} have been reversed, exactly as happened among (\ref{v1}, \ref{v2}) and (\ref{v3}, \ref{v4}). Since $\omega^2 > c^2$ the large spacelike (magnetic) limit is naturally implemented while the electric limit is non-existent. This case is also referred as Carollean limit. \\
\indent Furthermore, the two Lorentz transformations (\ref{t1}, \ref{t2}) and \ref{a2} are related by the substitution $u_i \to \frac{\omega_i}{\omega^2} c^2$ or $\omega_i \to \frac{u_i c^2}{u^2}$. Although both $u_i$ and $\omega_i$ have dimensions of velocity, they have different physical interpretations. For instance, $u^i$ is the velocity of a fixed point in space in the primed system measured in the spacetime of the un-primed system. On the contrary, $\omega^i$ is the rate of motion of an event measured in the un-primed system that occurs at a fixed time in the primed system. The genesis of this difference lies in the reversal of roles of space and time in the two cases. \\
\indent We can write eqn \ref{v3} and \ref{v4} as a matrix equation 
\begin{equation}
  \begin{pmatrix}
  v'^0 \\ v'^i 
  \end{pmatrix} 
  = \begin{pmatrix}
  1 & u_j \\ 0 & 1
  \end{pmatrix}
  \begin{pmatrix}
  v^0 \\ v^j
  \end{pmatrix}
  \label{V2}
\end{equation}
We will now consider the covariant vectors. We will write first the reverse transformations of eqn \ref{t1} and \ref{t2} which is 
\begin{equation}
    x^0 = \gamma x'^0 + \frac{\gamma u_i}{c} x'^i
    \label{t3}
\end{equation}
\begin{equation}
    x^i = x'^i + \frac{\gamma u^i}{c}x'^0 + (\gamma -1 )\frac{u^i u_j}{u^2}x'^j
    \label{t4}
\end{equation}
And we know covariant vectors transform as
\begin{equation*}
    V'_{\mu} = \frac{\6 X^{\nu}}{\6 x'^{\mu}} V_{\nu}
\end{equation*}
Componentwise we can again write them as
\begin{equation}
    V'_0 = V_0 + \frac{u^i}{c} V_i
    \label{cov1}
\end{equation}
\begin{equation}
    V'_i = V_i + \frac{u_i}{c} V_0
    \label{cov2}
\end{equation}
Now here we take the  electric limit in the following way, which will soon become clear
\begin{equation}
    V_0 = \frac{v_0}{c}, \,\,\,\, V_i = v_i
    \label{covel}
\end{equation}
Using \ref{covel} in \ref{cov1} and \ref{cov2} we get
\begin{eqnarray}
  v'_0 = v_0 + u^i v_i
  \label{v5}
\end{eqnarray}
\begin{eqnarray}
  v'_i = v_i
  \label{v6}
\end{eqnarray}
which are again Galilean transformations. We can write \ref{v5} and \ref{v6} as a matrix equation as
\begin{equation}
 \begin{pmatrix}
 v'_0 \\ v'_i 
 \end{pmatrix} 
 = \begin{pmatrix}
 1 & u_i \\ 0 & 1
 \end{pmatrix}
 \begin{pmatrix}
 v_0 \\ v_i
 \end{pmatrix}
 \label{V3}
\end{equation}
We will now consider the magnetic limit as 
\begin{equation}
    V_0 = -c v_0, \,\,\,\, V_i = v_i
    \label{covmag}
\end{equation}
Using \ref{covmag} in \ref{cov1} and \ref{cov2} we get 
\begin{equation}
    v_0' = v_0
    \label{v7}
\end{equation}
\begin{equation}
    v'_i = v_i - u_i v_0 
    \label{v8}
\end{equation}
We can write \ref{v7} and \ref{v8} as 
\begin{equation}
    \begin{pmatrix}
    v'_0 \\ v'_i 
    \end{pmatrix}
    = \begin{pmatrix}
    1 & 0 \\ -u_i & 1 
    \end{pmatrix}
    \begin{pmatrix}
    v_0 \\ v_i
    \end{pmatrix}
    \label{V4}
\end{equation}
We can show that the transformation matrix in \ref{V1} and the transpose of the matrix \ref{V3} satisfies
\begin{equation}
 \begin{pmatrix}
1 & 0\\ -v^i & 1
\end{pmatrix}
 \begin{pmatrix}
1 & 0\\ v^i & 1
\end{pmatrix} = \begin{pmatrix}
1 & 0 \\ 0 & 1
\end{pmatrix}
\end{equation}
Similarly the transformation matrix in equation \ref{V2} and the transpose of the transformation matrix in \ref{V4} satisfies
\begin{equation}
 \begin{pmatrix}
  1 & v_j \\ 0 & 1
  \end{pmatrix} 
  \begin{pmatrix}
   1 & -v_j \\ 0 & 1
  \end{pmatrix} = \begin{pmatrix}
   1 & 0 \\ 0 & 1
  \end{pmatrix}
\end{equation}

\noindent To justify the limiting prescriptions even further, we consider the norm preservation for both electric and magnetic limits. Let us first consider the norm in the electric limit
\begin{equation}
    V^0 V_0 + V^i V_i \xrightarrow[\text{limit}]{\text{electric}} \Big(c v^0 \Big)\Big(\frac{v_0}{c} \Big) + \Big(v^i\Big) \Big( v_i\Big) = v^0 v_0 + v^i v_i
\end{equation}
which clearly indicates that under the scaling electric limit the norm is preserved. Now we consider the norm in the magnetic limit
\begin{equation}
    V^0 V_0 + V^i V_i \xrightarrow[\text{limit}]{\text{magnetic}} \Big(-c v_0 \Big)\Big(\frac{-v^0}{c} \Big) + \Big(v^i\Big) \Big( v_i\Big) = v^0 v_0 + v^i v_i
\end{equation}
The norm is again conserved and the role of the minus sign before the scaling is also very important in this context. The mapping relations are summarised in the table \ref{T1}.
\begin{table}
\caption{Mapping relations}\label{T1}
\begin{center}
\begin{tabular}{|c|c|c|} \hline 
${\rm Limit}$  & $ {\rm Contravariant~~mapping}$ & $ {\rm Covariant~~mapping}$  \\ \hline
${\rm Electric ~~limit}$ & $V^0 \to c~v^0, \,\,  V^i \to v^i$ & $V_0 \to \frac{v_0}{c}, \,\, V_i \to v_i$ \\ \hline
${\rm Magnetic ~~limit}$ & $V^0 \to -\frac{v^0}{c},\,\,V^i \to v^i $ & $V_0 \to -c~v_0. \,\, V_i \to v_i$ \\
\hline
\end{tabular}
\label{T1}
\end{center}
\end{table}

\section{Lagrangian and field equations} \label{sec3}
\noindent Now let us start from the relativistic Maxwell theory described by the Lagrangian
\begin{equation}
    \mathcal{L} = -\frac{1}{4}~F_{\mu \nu}F^{\mu \nu} + \frac{k^2}{2} A_\mu A^\mu
    \label{l0}
\end{equation}
where $F_{\mu \nu} = \partial_{\mu} A_{\nu} - \partial_{\nu} A_{\mu}$ and $\eta_{\mu \nu}$ is the flat space metric with signature $\Big(-,+,+,+\Big)$. 
%We can write the Lagrangian as follows:

It gives rise to the following eqns of motion
\begin{eqnarray}
    \6^\mu F_{\mu \nu} + k^2 A_\nu = 0 \label{peom1}
\end{eqnarray}
which imply
\begin{eqnarray}
    \6^{\nu} \Big( \6^\mu F_{\mu \nu} + k^2 A_\nu \Big) = 0 \implies \6^\mu A_\mu = 0 
    \label{peom2}
\end{eqnarray}
Eqn \ref{peom2} is a necessary condition for the relativistic Proca theory, leading to a Klein-Gordon type equation,
\begin{equation}
  \Big(\Box + k^2  \Big) A_\nu = 0  \label{rwe}
\end{equation}
\subsection{Galilean relativistic theory}
We now discuss the derivation of the Galilean invariant Proca model using results of section 2. The two limits are now considered independently. The first step is to open the various terms in \ref{l0} as,
\begin{equation}
    \mathcal{L} = -\frac{1}{4}\Big(2F_{0i}F^{0i} + F_{ij}F^{ij} \Big) + \frac{k^2}{2} \Big(A_0 A^0 + A_i A^i  \Big)
    \label{l1}
\end{equation}

\noindent {\underline {\bf Electric limit:}}\\
Now using the relations given in table \ref{T1} we can write the two terms in \ref{l1} as, 
\begin{eqnarray}
2 F_{0i}F^{0i} = 2 \Big(\frac{1}{c} \partial_t A_i - \partial_i A_0 \Big)\Big(-\frac{1}{c} \partial_t A^i - \partial^i A^0  \Big) \xrightarrow[\text{limit}]{\text{electric}} 2 \Big(\frac{1}{c} \partial_t a_i - \frac{1}{c} \partial_i a_0  \Big)\Big(-\frac{1}{c} \partial_t a^i - c ~\partial^i a^0  \Big) \nonumber \\
\xrightarrow[\text{$c \to \infty$}]{\text{}}  -2 \6^i a^0 \Big( \partial_t a_i - \partial_i a_0  \big) \hspace{1in}
\label{part1}
\end{eqnarray}
 
\begin{eqnarray}
F_{ij}F^{ij} = \Big(\partial_i A_j - \partial_j A_i \Big) \Big(\partial^i A^j - \partial^j A^i \Big) \xrightarrow[\text{limit}]{\text{electric}} \Big(\partial_i a_j - \partial_j a_i \Big) \Big(\partial^i a^j - \partial^j a^i \Big) \nonumber \\
= f_{ij} f^{ij} \hspace{1in}
\end{eqnarray}

\begin{eqnarray}
\frac{k^2}{2} \Big(A_0 A^0 + A_i A^i  \Big)  \xrightarrow[\text{limit}]{\text{electric}}   \frac{k^2}{2} \Big(a_0 a^0 + a_i a^i  \Big) 
\end{eqnarray}

\noindent Here $A^{\mu}$ is the relativistic four potential while $a^0$ and $a^i$ are it's galilean counterpart. Hence the full lagrangian takes the following form
\begin{equation}
    \mathcal{L}_e = \frac{1}{2} \6^i a^0 \Big( \partial_t a_i - \partial_i a_0  \big)  - \frac{1}{4} f_{ij} f^{ij} + \frac{k^2}{2} \Big(a_0 a^0 + a_i a^i  \Big) 
    \label{el1}
\end{equation}


\noindent Now we derive the equations of motion. Varying the Lagrangian \ref{el1} with respect to $a_0,~a_j,~a^0,~a^j$ we get the corresponding equations of motion
\begin{eqnarray}
\6_i \6^i a^0 + k^2 a^0 = 0 
\label{ee1}
\end{eqnarray}
\begin{eqnarray}
\6_t \6^j a^0 + \6_i \6^j a^i - \6_i \6^i a^j - k^2 a^j = 0
\label{ee2}
\end{eqnarray}
\begin{eqnarray}
\6^i \6_t a_i - \6^i \6_i a_0 - k^2 a_0 = 0
\label{ee3}
\end{eqnarray}
\begin{eqnarray}
\6^i \6_i a_j - \6^i \6_j a_i + k^2 a_j = 0 
\label{ee4}
\end{eqnarray}
We now derive these equations directly from the relativistic equations of motion \ref{peom1}, which can be split as
\begin{eqnarray}
\6_i F^{i0} + k^2 A^0 =0 \label{e1}\\
\6_0 F^{0j} + \6_i F^{ij} + k^2 A^j = 0 \label{e2}
\end{eqnarray}
From table \ref{T1} we get in the electric limit
\begin{eqnarray}
\6_i F^{i0} + k^2 A^0  =0  \nonumber \\
\implies \6_i \Big( c \6^i a^0 + \frac{1}{c} \6_t a^i \Big) + c k^2 a^0 = 0 \nonumber \\  \xrightarrow[\text{$c \to \infty$}]{\text{}} \6_i \6^i a^0 + k^2 a^0 = 0 
\label{eo1}
\end{eqnarray}
which reproduces eqn \ref{ee1}. Likewise the Galilean limit of \ref{e2} reproduces \ref{ee2}.

To get the remaining pair of equations \ref{ee3} and \ref{ee4}  we have to interpret  eqns \ref{e1} and \ref{e2} in terms of their covariant components, 
\begin{eqnarray}
\6^i F_{i0} + k^2 A_0 =0 \label{e3}\\
\6^0 F_{0j} + \6^i F_{ij} +k^2 A_j = 0 \label{e4}
\end{eqnarray}
The first of these in the Galilean limit yields
\begin{eqnarray}
 \6^i \Big( \frac{1}{c} \6_i a_0 - \frac{1}{c} \6_t a_i \Big) + \frac{1}{c} k^2 a_0 = 0 \nonumber \xrightarrow[\text{$c \to \infty$}]{\text{}} \6^i \6_i a_0 - \6^i \6_t a_i + k^2 a_0 = 0 
\label{eo3}
\end{eqnarray}
which reproduces eqn \ref{ee3}.
Likewise eqn \ref{e4} yields eqn \ref{ee4} in this limit. 
This shows the consistency of the eqn of motion in Galilean Proca model.\\
\indent Let us now show how \ref{peom2} is manifested in the present analysis. Since covariant and contravariant components are treated separately in the Galilean construction, \ref{peom2} is interpreted as, 
\begin{equation}
    \6^\mu A_\mu = \6_\mu A^\mu = 0
\end{equation}
Now these conditions imply, 
\begin{eqnarray}
    \6^\mu A_\mu = 0 \implies 
    -\frac{1}{c} \6_t \Big(\frac{1}{c} a_0 \Big) + \6^i a_i = 0 \xrightarrow[\text{$c \to \infty$}]{\text{}} \6^i a_i = 0 \label{ae1}
\end{eqnarray}
and,
\begin{eqnarray}
    \6_\mu A^\mu = 0 \implies \6_t a^0 + \6_i a^i = 0 \label{ae2}
\end{eqnarray}
The relation \ref{ae1} follows on acting $\6^j$ on either side of \ref{ee4}. Likewise \ref{ae2} follows on acting $\6_j$ on \ref{ee2} and using \ref{ee1}. We will now discuss the implications of eqn \ref{rwe}. To do that we can write for its  contravariant components, 
\begin{equation}
    \Big(-\frac{1}{c^2} \6_t^2 + \6_j \6^j + k^2 \Big) A^\nu =0 \label{bstr}
    \end{equation}
Considering the electric limits for $\nu = 0 ~\& ~i$ we have following two equations
\begin{eqnarray}
    \6_i \6^i a^0 + k^2 a^0 = 0, \label{ewe1} \\
    \6_i \6^i a^j + k^2 a^j = 0 \label{ewe2}
\end{eqnarray}
Now eqn \ref{ewe1} is nothing but \ref{ee1}. We can re-write eqn \ref{ee2} as
\begin{equation}
    \6^j \Big(\6_t a^0 + \6_i a^i  \Big) - \6_i \6^i a^j - k^2 a^j = 0 \implies \Big(\6_i \6^i + k^2 \Big) a^j = 0 
\end{equation}
which reproduces eqn \ref{ewe2}. The term in the bracket vanishes as a consequence of \ref{ae2}. We can also write the other two equations i.e \ref{ee3} and \ref{ee4} as the the form given by \ref{rwe}. \\
\noindent {\underline {\bf Magnetic limit:}}\\
Here again using the relations given in table \ref{T1} we can write the two terms in \ref{l1} as,
\begin{eqnarray}
2 F_{0i}F^{0i} = 2 \Big(\frac{1}{c} \partial_t A_i - \partial_i A_0 \Big)\Big(-\frac{1}{c} \partial_t A^i - \partial_i A^0  \Big)  \xrightarrow[\text{$c \to \infty$}]{\text{magnetic limit}} -2 \6_i a_0 \Big(\6_t a^i - \6^i a^0 \Big) \hspace{1in}
\end{eqnarray}
\begin{eqnarray}
F_{ij}F^{ij} 
\xrightarrow[\text{$c \to \infty$}]{\text{magnetic limit}} f_{ij}f^{ij} \hspace{1in} 
\end{eqnarray}
\begin{eqnarray}
\frac{k^2}{2} \Big(A_0 A^0 + A_i A^i  \Big)  \xrightarrow[\text{limit}]{\text{magnetic}}   \frac{k^2}{2} \Big(a_0 a^0 + a_i a^i  \Big) 
\end{eqnarray}
So the Lagrangian will take the following form

\begin{equation}
\mathcal{L}_m = \frac{1}{2}  \6_i a_0 \Big(\6_t a^i - \6^i a^0 \Big) - \frac{1}{4} f_{ij}f^{ij} + \frac{k^2}{2} \Big(a_0 a^0 + a_i a^i  \Big)  \label{l2}
\end{equation}
Varying \ref{l2} wrt $a_0,~a_j,~a^0,~a^j$ we get,
\begin{eqnarray}
   \partial_i ~\partial_t~ a^i - \partial^i ~\partial_i ~a^0 - k^2 a^0 = 0 
  \label{em1}
\end{eqnarray}
\begin{eqnarray}
   \6_i \6^i a^j - \6_i \6^j a^i + k^2 a^j = 0
  \label{em2}
  \end{eqnarray}
\begin{eqnarray}
  \6^i \6_i a_0 + k^2 a_0 = 0
  \label{em3}
\end{eqnarray}
\begin{eqnarray}
 \6_t \6_j a_0 + \6^i \6_j a_i - \6^i \6_i a_j - k^2 a_j = 0 \label{em4}
\end{eqnarray}
Here also we can show that the above equations agree with those derived directly from relativistic Maxwell equations by taking the magnetic limit.
From \ref{e1} we have
\begin{eqnarray}
\6_i \Big( -\frac{1}{c}\6^i a^0 + \frac{1}{c} \6_t a^i \Big) - \frac{1}{c} k^2 a^0 = 0 \nonumber  \xrightarrow[\text{$c \to \infty$}]{\text{}} \6_i \6_t a^i - \6_i \6^i a^0 - k^2 a^0 = 0
\label{eo5}
\end{eqnarray}
which reproduces eqn \ref{em1}. 
Similarly \ref{e2} reproduces \ref{em2}
\\
\noindent To get the remaining pair of equations we have to start from the covariant versions (\ref{e3}, \ref{e4}). 
From \ref{e3} we get 
\begin{eqnarray}
\6^i \Big(-c \6_i a_0 - \frac{1}{c} \6_t a_i  \Big) -c k^2 a_0 = 0  \xrightarrow[\text{$c \to \infty$}]{\text{}} \6^i \6_i a_0 + k^2 a_0 = 0 \label{eo7}
\end{eqnarray}
which yields \ref{em3}. 
Likewise eqn \ref{e4} yields \ref{em4} in this limit.

\begin{table}
\caption{Field equations}\label{T2}
\begin{center}
\begin{tabular}{|c|c|c|} \hline 
${\rm Variables}$ & ${\rm Electric ~~limit}$ & ${\rm Magnetic ~~limit}$ \\ \hline
$a^0$  & $\6^i \6_t a_i - \6^i \6_i a_0 -k^2 a_0 = 0$ & $\6^i \6_i a_0 +k^2 a_0 = 0$ \\ \hline
$a^j$ & $\6^i \6_i a_j - \6^i \6_j a_i + k^2 a_j = 0  $ & $\6_t \6_j a_0 + \6^i \6_j a_i - \6^i \6_i a_j - k^2 a_j = 0$ \\ \hline
$a_0$ & $ \partial^i \partial_i a^0 +k^2 a^0 = 0 $ & $\partial_i ~\partial_t~ a^i - \partial^i ~\partial_i ~a^0 - k^2 a^0 = 0 $ \\ \hline
$a_j$ & $\6_t \6^j a^0 + \6_i \6^j a^i - \6_i \6^i a^j - k^2 a^j = 0$ & 
$ \6_i \6^i a^j - \6_i \6^j a^i + k^2 a^j = 0 $ \\ \hline
\end{tabular}
\label{T2}
\end{center}
\end{table}
\noindent The field equations for both limits of Galilean electrodynamics are shown in table \ref{T2}.\\

% We will now see what this condition imply in the non-relativistic limit. For this first we will consider the electric limit and then magnetic limit.\\
%\noindent \underline{\bf Electric limit:}\\


%\noindent \underline{\bf Magnetic limit:}\\
\noindent In magnetic limit the condition \ref{peom2} implies
\begin{eqnarray}
 \6^\mu A_\mu = 0 
\implies -\frac{1}{c} \6_t \Big( -c a_0 \Big) + \6^i a_i =0 \xrightarrow[\text{$c \to \infty$}]{\text{}} \6_t a_0 + \6^i a_i = 0 \label{am1}
\end{eqnarray}
Similarly,
\begin{eqnarray}
    \6_\mu A^\mu = 0 \implies \6_i a^i = 0 \label{am2}
\end{eqnarray}
Here also the relation \ref{am1} follows on acting $\6^j$ on eqn \ref{em4} and using eqn \ref{em3}. Similarly relation \ref{am2} follows on acting $\6_j$ on \ref{em2}.  We will now discuss the implications of eqn \ref{rwe} in the magnetic limit. For the contravariant components the equations are
\begin{eqnarray}
    \Big(\6_i^2 + k^2  \Big) a^0 = 0 \label{mew1}\\
    \Big(\6_i^2 + k^2   \Big) a^i =0 \label{mew2}
\end{eqnarray}
Now from the equations of motion, eqn \ref{em1} implies
\begin{equation}
    \6_t \Big(\6_i a^i \Big) - \6_i \6^i a^0 - k^2 a^0 = 0 \implies \Big( \6_i^2 + k^2 \Big) a^0 =0 \label{mw1}
\end{equation}
In the first line $\6_i a^i = 0 $ from \ref{am2}. Eqn \ref{mw1} reproduces \ref{mew1}. Similarly \ref{em2} reproduces \ref{mew2} by exploiting \ref{am2}. The other two equation \ref{em3} and \ref{em4} can be written in the form of \ref{rwe} by exploiting \ref{am1}. \\
\indent From table \ref{T2} we observe the dual role of electric and magnetic limits vis-a-vis their covariant and contravariant sectors. In other words the electric limit equation found by varying $a^0$ corresponds to the magnetic limit equation obtained by varying $a_0$. The same property holds for other variables. 
\section{Electric and magnetic fields in Galilean Proca theory} \label{sec4}
\noindent Here we discuss Galilean Proca theory in terms of electric and magnetic fields introduced in analogy with Maxwell theory. For this purpose we will discuss contravariant and covariant sectors separately.
\subsection{Contravariant sector}
Relativistic electric and magnetic fields are defined as 
\begin{eqnarray}
E^i = \6^0 A^i - \6^i A^0 \\
B^i = \epsilon^{ij}_{~k} \6_j A^k
\end{eqnarray}
First, we consider the electric limit. \\
\noindent \underline {\bf Electric limit:}\\
\noindent Using the mapping relations given in table \ref{T1} we can write the electric field as 
\begin{eqnarray}
E^i = -\frac{1}{c} \6_t a^i - c \6^i a^0 
\end{eqnarray}
We can define the Galilean electric and magnetic fields as 
\begin{equation}
    e^i = \lim_{c \to \infty} \frac{E^i}{c} = -\6^i a^0, \,\,\,~~~~~ b^i =  \lim_{c \to \infty} B^i = \epsilon^{ij}_{~k} \6_j a^k
\end{equation}

\noindent    Now we write the field equations that we derived in the previous section  in terms of these electric and magnetic fields. From \ref{ee1} we get 
    \begin{eqnarray}
    \6_i \6^i a^0 = -k^2 a^0 \implies \6_i (-e^i) = -k^2 a^0 \implies \vec \nabla . \vec e = k^2 a^0
    \end{eqnarray}
\noindent    Similarly eqn \ref{ee2} implies
    \begin{eqnarray}
   \6_t \6^j a^0 + \6_i \6^j a^i - \6_i \6^i a^j -k^2 a^j= 0 
     \implies  (\vec \nabla \times \vec b)^j = \6_t e^j + k^2 a^j
    \end{eqnarray}
    
\noindent We can see clearly that
\begin{equation}
 \vec \nabla .\vec b =  \6_i b^i = \6_i \epsilon^{ij}_{~k} \6_j a^k =\epsilon^{ij}_{~k} \6_i\6_j a^k = 0  
\end{equation}

\noindent We will now compute $\vec \nabla \times \vec e$,
\begin{eqnarray}
(\nabla \times e)^i = \epsilon^{ij}_{~k} \6_j e^k 
= \epsilon^{ij}_{~k} \6_j (-\6^k a^0) = 0 
\end{eqnarray}
So in electric limit we get the following set of equations
\begin{eqnarray}
\vec \nabla . \vec e = \6_i e^i = k^2 a^0 \\
\vec \nabla . \vec b =\6_i b^i = 0\\ 
(\vec \nabla \times \vec e)^i = \epsilon^{ij}_{~k} \6_j e^k = 0 \label{max1} \\
(\vec \nabla \times \vec b)^i = \epsilon^{ij}_{~k} \6_j b^k = \6_t (\vec e)^i + k^2 a^i\label{max2}
\end{eqnarray}
\noindent \underline {\bf Magnetic limit:}\\
\noindent Electric field can be written in this limit from the mapping relations in table \ref{T1} as 
\begin{equation}
    E^i = -\frac{1}{c} \6_t a^i + \frac{1}{c} \6^i a^0
\end{equation}
We define the galilean electric and magnetic fields as 
\begin{equation}
    e^i = \lim_{c \to \infty} c E^i = -(\6_t a^i - \6^i a^0), \,\,\, ~~~~~ b^i = \lim_{c \to \infty} B^i = \epsilon^{ij}_{~k} \6_j a^k
\end{equation}
From \ref{em1} we get 
\begin{eqnarray}
\6_i\Big(\6_t a^i - \6^i a^0  \Big) -k^2 a^0 = 0 
\implies \6_i (-e^i) = k^2 a^0 \implies \vec \nabla . \vec e = -k^2 a^0
\end{eqnarray}
From \ref{em2} we have
\begin{eqnarray}
\6_i f^{ij} + k^2 a^j = 0 \implies
 -\epsilon^{ji}_{~k} \6_i b^k = -k^2 a^j \implies
\Big( \vec \nabla \times \vec b \Big)^j = k^2 a^j
\end{eqnarray}
Finally, we compute $\vec \nabla \times \vec e$, 
\begin{eqnarray}
(\vec \nabla \times \vec e)^i = \epsilon^{ij}_{~k} \6_j e^k \implies \epsilon^{ij}_{~k} \6_j \Big(\6^k a^0 - \6_t a^k \Big) 
\implies -\6_t \epsilon^{ij}_{~k} \6_j a^k = - \6_t b^i
\end{eqnarray}
So in magnetic limit we get the following set of equations
\begin{eqnarray}
\vec \nabla . \vec e = \6_i e^i = -k^2 a^0 \\
\vec \nabla . \vec b =\6_i b^i = 0\\ 
(\vec \nabla \times \vec e)^i = \epsilon^{ij}_{~k} \6_j e^k = -\6_t (\vec b)^i \\
(\vec \nabla \times \vec b)^i = \epsilon^{ij}_{~k} \6_j b^k  = k^2 a^i
\end{eqnarray}
\subsection{Covariant sector}
Relativistic electric and the magnetic fields are defined as 
\begin{eqnarray}
E_i = - \Big( \6_0 A_i - \6_i A_0\Big) \\
B_i = \epsilon_i^{~jk} \6_j a_k
\end{eqnarray}
First, we consider the electric limit. \\
\noindent \underline {\bf Electric limit:}\\
In this limit the electric field looks like 
\begin{equation}
E_i = -\Big(\frac{1}{c} \6_t a_i + \frac{1}{c} \6_i   a_0\Big)
\end{equation}
The Galilean electric and magnetic fields are given by,  
\begin{equation}
    e_i = \lim_{c \to \infty} c E_i = - (\6_t a_i - \6_i a_0), \,\,\,~~~~~~ b_i = \lim_{c \to \infty} B_i = \epsilon_{i}^{~jk} \6_j a_k
\end{equation}
Using these expressions the Proca equations in the Galilean electric limit are found to be, 
\begin{eqnarray}
\vec \nabla . \vec e = \6^i e_i = -k^2 a_0 \\
\vec \nabla . \vec b = \6^i b_i = 0\\ 
\Big(\vec \nabla \times \vec e\Big)_i = \epsilon_{i}^{~jk}\6_j e_k = - \6_t (\vec b)_i \\
\Big(\vec \nabla \times \vec b \Big)_i= \epsilon_{ij}^{~~k} \6^j b_k = k^2 a_i
\end{eqnarray}
\noindent \underline {\bf Magnetic limit:}\\
In this limit electric field is scaled as 
\begin{equation}
    E_i = -\Big(\frac{1}{c} \6_t a_i + c \6_i a_0 \Big)
\end{equation}
The Galilean electric and magnetic fields are now defined as,
\begin{equation}
    e_i = \lim_{c \to \infty} \frac{E_i}{c} = -\6_i a_0, \,\,\, ~~~~~ b_i = \lim_{c \to \infty} B_i = \epsilon_{i}^{~jk} \6_j a_k
\end{equation}
 Finally, the equations we get in the magnetic limit are given by,
\begin{eqnarray}
\vec \nabla . \vec e = \6^i e_i = k^2 a_0 \\
\vec \nabla . \vec b = \6^i b_i = 0\\ 
\Big(\vec \nabla \times \vec e \Big) = \epsilon_{i}^{~jk} \6_j e_k = 0 \\
\Big(\vec \nabla \times \vec b \Big)_i = \epsilon_{ij}^{~~k} \6^j b_k = \6_t e_i + k^2 a_i
\end{eqnarray}
\subsection{Lagrangian in field formulation and boost invariance}
\noindent In both electric and magnetic limit the lagrangian will take the following form 
\begin{equation}
    \mathcal{L}_e = \mathcal{L}_m =  \frac{1}{2} \Big(e^i e_i - b_i b^i \Big) + \frac{k^2}{2} \Big(a_0 a^0 + a_i a^i  \Big)  \label{lfe}
\end{equation}
In the electric limit $a^0, ~a^i, ~a_0, ~a_i$ will transform under Galilean boost as
\begin{eqnarray}
    a'^0 = a^0, \,\,\, a'^i = a^i - u^i a^0, \,\,\, a'_0 = a_0 + u^i a_i, \,\,\, a'_i =a_i \label{gbt1}
\end{eqnarray}
As pointed out in \cite{Bhattacharya}, the electric and magnetic fields transform in the electric limit
\begin{equation}
    e'^i = e^i, \,\,\, ~~b'^i = b^i - \Big(\vec v \times \vec e \Big)^i, \,\,\, ~~e'_i = e_i + \Big( \vec v \times \vec b \Big)_i, \,\,\, ~~b'_i = b_i \label{ebt}
\end{equation}
Using \ref{ebt} it is evident that the Maxwell part of the lagrangian (\ref{lfe}) remains invariant
\begin{equation}
    \delta \mathcal{L}_{Maxwell} = 0 
\end{equation}

We will show here that remaining terms are also boost invariant. To do so we write
\begin{equation}
    \mathcal{L}_{Mass} =  \frac{k^2}{2} \Big(a_0 a^0 + a_i a^i  \Big) \label{lp}
\end{equation}
Now taking the variation of \ref{lp} and using \ref{gbt1} we have
\begin{equation}
 \delta \mathcal{L}_{Mass} =  \frac{k^2}{2} \Big(\delta a_0 a^0 + a_0 \delta a^0 + \delta a_i a^i + a_i \delta a^i  \Big)  = \frac{k^2}{2} \Big( u^i a_i a^0 - u^i a^0 a_i \Big) = 0
\end{equation}
In the magnetic limit $a^0, ~a^i, ~a_0, ~a_i$ will transform under Galilean boost as
\begin{eqnarray}
    a'^0 = a^0 + u_j a^j, \,\,\, a'^i = a^i, \,\,\, a'_0 = a_0 , \,\,\, a'_i =a_i - u_i a_0 \label{gbt2}
\end{eqnarray}
And using \ref{gbt2} we can write the variation of \ref{lp} as
\begin{equation}
   \delta \mathcal{L}_{Mass} = \frac{k^2}{2} \Big(a_0 u_j a^j - a^i u_i a^0  \Big) =0
\end{equation}
\section{Noether Currents} \label{sec5}
\noindent Contrary to  Maxwell theory there is no conserved current that directly follows from the Noether procedure since there is no gauge invariance. However it is possible to redefine the current so that it is conserved. This happens because although the Lagrangian is not invariant, the action is on-shell invariant. To see this consider the variation,
\begin{equation}
    \delta A_\mu = \6_\mu \alpha
\end{equation}
which modifies the Proca lagrangian to 
\begin{eqnarray}
    \delta \mathcal{L}_{\rm Proca} = k^2 A_\mu \6^\mu \alpha = k^2 \6^\mu \Big( A_\mu \alpha \Big) - k^2 \6^\mu A_\mu \alpha 
\end{eqnarray}
Hence action is on-shell invariant
\begin{equation}
    \delta S = \int \delta \mathcal{L}_{\rm Proca} = 0 
\end{equation}
This indicates the possibility to redefine the current that follows from the Noether procedure such that it is conserved. The usual Noether current is,
\begin{equation}
    J^\mu = \frac{\6 \mathcal{L}}{\6 (\6_\mu A_\nu)} \delta A_\nu = -F^{\mu \nu} \6_\nu \alpha
\end{equation}
Redefining the current as ,
\begin{equation}
    J'^\mu = J^\mu - k^2 A^\mu \alpha
\end{equation}
it is possible to show its on-shell conservation,
\begin{equation}
    \6_\mu J'^\mu = -\6_\mu F^{\mu \nu} \6_\nu \alpha - k^2 A^\mu \6_\mu \alpha - k^2 (\6_\mu A^\mu) \alpha
\end{equation}
Finally using the equations of motion \ref{peom1} and \ref{peom2} we have 
\begin{equation}
    \6_\mu J'^\mu = 0
\end{equation}
Now we see the implications of these currents in the galilean limits. To do this we will start with the electric limit.\\
\noindent \underline{\bf Electric limit:}\\
The map from $J^\mu \to j^\mu$ or $J_\mu \to j_\mu$ just follow the corresponding map for vectors, $A^\mu \to a^\mu$, $A_\mu \to a_\mu$,
\begin{eqnarray}
    J^0 \to c j^0, \,\,\, J^i \to j^i, \,\,\, J_0 \to \frac{j_0}{c}, \,\,\, J_i \to j_i
\end{eqnarray}
Now 
\begin{eqnarray}
    J^0 = -F^{0i} \6_i \alpha \implies j^0 = \Big( \frac{1}{c^2} \6_t a^i + \6^i a^0 \Big) \6_i \alpha\\
    \xrightarrow[\text{$c \to \infty$}]{\text{}} j^0 = \6^i a^0 \6_i \alpha
\end{eqnarray}
Similarly 
\begin{eqnarray}
    J^i = -F^{0i} \6_0 \alpha - F^{ij} \6_j \alpha \\
    \xrightarrow[\text{$c \to \infty$}]{\text{}} j^i = -\6^i a^0 \6_t \alpha - f^{ij} \6_j \alpha
\end{eqnarray}
Now we define the modified currents $J'^\mu$ which in the non relativistic limit gives rise to,
\begin{eqnarray}
  j'^0 = j^0 - k^2 a^0 \alpha, \,\,\,\,\, j'^i = j^i - k^2 a^i \alpha  
\end{eqnarray}
To see current conservation in the galilean limit we compute 
%\begin{eqnarray}
%    \6_\mu J'^\mu = \6_0 J'^0 + \6_i J'^i = \6_t j'^0 + \6_i j'^i 
%\end{eqnarray}
\begin{eqnarray}
    \6_t j'^0 + \6_i j'^i = \6_t j^0 + \6_i j^i - \6_t (k^2 a^0 \alpha) - \6_i (k^2 a^i \alpha) = 0
\end{eqnarray}

%In the above equation we have used eqns \ref{ee1}, \ref{ee2} and \ref{ae1}.

\noindent The covariant currents are derived by adopting a similar approach. From the relativistic expression using the mapping relations given in table \ref{T1},
\begin{equation}
    J_0 = -F_{0i}\6^i \alpha \implies \frac{j_0}{c} = -\frac{1}{c} \Big(\6_t a_i - \6_i a_0  \Big) \6^i \alpha \implies j_0 = -\Big(\6_t a_i - \6_i a_0  \Big) \6^i \alpha
\end{equation}
and
\begin{equation}
    J_i = -F_{ij}\6^j \alpha \implies j_i = -f_{ij} \6^j \alpha 
\end{equation}
The modified currents are given by, 
\begin{eqnarray}
  j'_0 = j_0 - k^2 a_0 \alpha, \,\,\,\,\, j'_i = j_i - k^2 a_i \alpha  
\end{eqnarray}

The conservation equation reduces to 
\begin{eqnarray}
    \6^\mu J'_\mu = \6^0 J'_0 + \6^i J'_i = \6^i j'_i
\end{eqnarray}
Here 
\begin{eqnarray}
    \6^i j'_i =  \6^i j^i - k^2 \6^ia_i \alpha -k^2 a_i \6^i \alpha = 0
\end{eqnarray}
In the above equation we have used eqns \ref{ee4} and \ref{ae1}. \\
\noindent \underline{\bf Magnetic limit:}\\
\noindent In this limit,
\begin{eqnarray}
    J^0 \to -\frac{j^0}{c}, \,\,\, J^i \to j^i, \,\,\, J_0 \to -c j_0, \,\,\, J_i \to j_i
\end{eqnarray}
The contravariant components of the currents are
\begin{eqnarray}
    j^0 = \Big( -\6_t a^i + \6^i a^0 \Big) \6_i \alpha, \,\,\, j^i = -f^{ij} \6_j \alpha 
\label{currents1}
\end{eqnarray}
Conservation of the modified currents imply
\begin{eqnarray}
    \6_i j'^i = \6_i j^i - k^2 (\6_i a^i) \alpha - k^2 a^i \6_i \alpha = 0
\end{eqnarray}
where we have exploited eqns \ref{currents1},  \ref{em2} and \ref{am2}.

\section{Stuckelberg embedded version of Proca model} \label{sec6}
The theory is described by the Lagrangian 
\begin{equation}
   \mathcal{L}_{\rm SProca} = -\frac{1}{4}~F_{\mu \nu}~F^{\mu \nu} + \frac{k^2}{2} \Big(A_\mu + \6_\mu \Phi \Big) \Big(A^\mu + \6^\mu \Phi \Big) \label{sl1}
\end{equation}
where $F_{\mu \nu} = \partial_{\mu} A_{\nu} - \partial_{\nu} A_{\mu}$ and $\Phi$ is a scalar field. We can now see that the Lagrangian is gauge invariant under following transformations,
\begin{equation}
    A_\mu \to A_\mu + \6_\mu \alpha, \,\,\,\,\, \Phi \to \Phi - \alpha
\end{equation}
Now varying the action wrt $A_\nu$ we have the following equations of motion
\begin{equation}
    \6_\mu F^{\mu \nu} + k^2 \Big( A^\nu + \6^\nu \Phi \Big) = 0 \label{seq1}
\end{equation}
Varying the action wrt $\Phi$ will give following equations of motion
\begin{equation}
    \6_\mu \Big( A^\mu + \6^\mu \Phi \Big) = 0 \label{seq2}
\end{equation}
We see that \ref{seq2} is consistent with \ref{seq1}.
%We will now consider the galilean limit of the lagrangian and to do this it is useful to write the lagrangian in eqn \ref{sl1} in the following form
%\begin{equation}
 %   \mathcal{L}_{\rm SProca} =  -\frac{1}{4}~F_{\mu \nu}~F^{\mu \nu} + \frac{k^2}{2} A^\mu A_\mu + \frac{k^2}{2} A_\mu \6^\mu \Phi + \frac{k^2}{2} A^\mu \6_\mu \Phi + \frac{k^2}{2} \6_\mu \Phi \6^\mu \Phi
%\end{equation}
Since our aim is to construct Stuckelberg embedded Galilean Proca model where covariant and contravariant sectors are treated distinctly, it is necessaryto write a more general version of \ref{sl1},
\begin{equation}
   \mathcal{L}_{\rm SProca} = -\frac{1}{4}~F_{\mu \nu}~F^{\mu \nu} + \frac{k^2}{2} \Big(A_\mu + \6_\mu \Phi \Big) \Big(A^\mu + \6^\mu \Psi \Big) \label{sl2}
\end{equation}
which bears an extended gauge invariance
\begin{eqnarray}
    A_\mu \to A_\mu + \6_\mu \alpha, \,\,\,\,~~~~~ \Phi \to \Phi - \alpha \label{G1} \\
    A^\mu \to A^\mu + \6^\mu \beta, \,\,\, \, ~~~~~\Psi \to \Psi - \beta \label{G2}
\end{eqnarray}
The equations of motion are obtained by varying $A_\nu, ~A^\nu, ~\Phi ~\& ~\Psi$
\begin{eqnarray}
    \6_\mu F^{\mu \nu} + k^2 \Big(A^\nu + \6^\nu \Psi   \Big) = 0, \\
     \6^\mu F_{\mu \nu} + k^2 \Big(A_\nu + \6_\nu \Phi   \Big) = 0, \\
     \6_\mu \Big(A^\mu + \6^\mu \Psi   \Big) = 0, \\
     \6^\mu \Big( A_\mu + \6_\mu \Phi \Big) =0
\end{eqnarray}
We will now consider different limits, beginning with the electric case. \\
\noindent \underline{\bf Electric limit:}\\
\noindent In this limit we know the fields scale as 
\begin{eqnarray}
    A^0 \to c a^0, \,\, A^i \to a^i, \,\, A_0 \to \frac{a_0}{c}, \,\, A_i \to a_i, \,\, \Phi \to \phi, \,\, \Psi \to \psi
\end{eqnarray}

\noindent The Lagrangian looks like \ref{sl2} imbibes the form,
\begin{equation}
    \mathcal{L}_{\rm SPe} = \frac{1}{2} \6^i a^0 \Big( \partial_t a_i - \partial_i a_0  \big)  - \frac{1}{4} f_{ij} f^{ij} + \frac{k^2}{2} \Big(a_0 a^0 + a_i a^i  \Big) + \frac{k^2}{2} a_i \6^i \psi + \frac{k^2}{2} \Big( a^0 \6_t \phi + a^i \6_i \phi  \Big) + \frac{k^2}{2} \6_i \phi \6^i \psi \label{sel1}
\end{equation}
\noindent Now we derive the equations of motion. Varying the Lagrangian \ref{sel1} with respect to $a_0,~a_j,~a^0,~a^j$ we get the corresponding equations of motion
\begin{eqnarray}
\6_i \6^i a^0 + k^2 a^0 = 0 
\label{see1}
\end{eqnarray}
\begin{eqnarray}
\6_t \6^j a^0 + \6_i \6^j a^i - \6_i \6^i a^j - k^2 \Big( a^j + \6^j \psi \Big) = 0
\label{see2}
\end{eqnarray}
\begin{eqnarray}
\6^i \6_t a_i - \6^i \6_i a_0 - k^2 \Big( a_0 + \6_t \phi \Big) = 0
\label{see3}
\end{eqnarray}
\begin{eqnarray}
\6^i \6_i a_j - \6^i \6_j a_i + k^2 \Big( a_j + \6_j \phi \Big) = 0 
\label{see4}
\end{eqnarray}
We now derive these equations directly from the equations of motion. The relativistic equations are given by,
\begin{equation}
    \6_{\mu} F^{\mu \nu} + k^2 \Big( A^\nu + \6^\nu \psi \Big) = 0 
\end{equation}
which can be written as
\begin{eqnarray}
\6_i F^{i0} + k^2 \Big( A^0 + \6^0 \psi \Big)  =0 \label{se1}\\
\6_0 F^{0j} + \6_i F^{ij} + k^2 \Big( A^j + \6^j \psi \Big) = 0 \label{se2}
\end{eqnarray}
Using table \ref{T1} and taking appropriate limits, we reproduce \ref{see1} from \ref{se1} and \ref{see2} from \ref{se2}.\\
\indent To get the remaining pair of equations we have to interpret  eqns \ref{se1} and \ref{se2} so that the variables appear in a covariant form,
\begin{eqnarray}
\6^i F_{i0} + k^2 \Big( A_0 + \6_0 \phi \Big) =0 \label{se3}\\
\6^0 F_{0j} + \6^i F_{ij} +k^2 \Big( A_j + \6_j \phi \Big) = 0 \label{se4}
\end{eqnarray}
Following similar steps \ref{se3} reproduces \ref{see3} while \ref{se4} yields \ref{see4} in the Galilean limit.\\
\noindent \underline{\bf Magnetic limit:}\\
\noindent In this limit we know the fields scale as 
\begin{eqnarray}
    A^0 \to -\frac{a^0}{c}, \,\, A^i \to a^i, \,\, A_0 \to -c a_0, \,\, A_i \to a_i, \,\, \Phi \to \phi, \,\, \Psi \to \psi
\end{eqnarray}
Now the Lagrangian looks like 
\begin{equation}
    \mathcal{L}_{\rm SPm} = \frac{1}{2} \6_i a_0 \Big( \partial_t a^i - \partial^i a^0  \big)  - \frac{1}{4} f_{ij} f^{ij} + \frac{k^2}{2} \Big(a_0 a^0 + a_i a^i  \Big) + \frac{k^2}{2} a^i \6_i \phi + \frac{k^2}{2} \Big( a_0 \6_t \psi + a_i \6^i \psi  \Big) + \frac{k^2}{2} \6_i \phi \6^i \psi \label{sml1}
\end{equation}
Varying \ref{sml1} wrt $a_0,~a_j,~a^0,~a^j$ we get,
\begin{eqnarray}
   \partial_i ~\partial_t~ a^i - \partial^i ~\partial_i ~a^0 - k^2 \Big( a^0 + \6_t \psi \Big) = 0 
  \label{sem1}
\end{eqnarray}
\begin{eqnarray}
   \6_i \6^i a^j - \6_i \6^j a^i + k^2 \Big( a^j + \6^j \psi \Big) = 0
  \label{sem2}
  \end{eqnarray}
\begin{eqnarray}
  \6^i \6_i a_0 + k^2 a_0 = 0
  \label{sem3}
\end{eqnarray}
\begin{eqnarray}
 \6_t \6_j a_0 + \6^i \6_j a_i - \6^i \6_i a_j - k^2 \Big( a_j + \6_j \phi \Big) = 0 \label{sem4}
\end{eqnarray}
Here also we can show that the above equations agree with those derived directly from relativistic Maxwell equations by taking the magnetic limit.
From \ref{se1} we have
\begin{eqnarray}
\6_i \Big( -\frac{1}{c}\6^i a^0 + \frac{1}{c} \6_t a^i \Big) + k^2 \Big( - \frac{1}{c} a^0 -\frac{1}{c} \6_t \psi \Big) = 0 \nonumber \\ \xrightarrow[\text{$c \to \infty$}]{\text{}} \6_i \6_t a^i - \6_i \6^i a^0 - k^2 \Big( a^0 + \6_t \psi \Big) = 0
\label{seo5}
\end{eqnarray}
which reproduces eqn \ref{sem1}. 
From \ref{e2} we get 
\begin{eqnarray}
\frac{1}{c} \6_t \Big(-\frac{1}{c} a^j + \frac{a^0}{c} \6^j a^0  \Big) + \6_i \Big(\6^i a^j - \6^j a^i \Big) + k^2 \Big( a^j + \6^j \psi \Big) = 0 \nonumber \\ \xrightarrow[\text{$c \to \infty$}]{\text{}} \6_i \6^i a^j - \6_i \6^j a^i + k^2 \Big( a^j + \6^j \psi \Big) = 0 \label{seo6}
\end{eqnarray}
which reproduces \ref{sem2}. \\
\noindent To get the remaining pair of equations we have to start from the covariant versions (\ref{se3}, \ref{se4}). 
From \ref{se3} we get 
\begin{eqnarray}
\6^i \Big(-c \6_i a_0 - \frac{1}{c} \6_t a_i  \Big) + k^2 \Big( -c a_0 - \frac{1}{c} \6_t \phi \Big) = 0  \xrightarrow[\text{$c \to \infty$}]{\text{}} \6^i \6_i a_0 + k^2 a_0 = 0 \label{seo7}
\end{eqnarray}
which yields \ref{sem3}. 
Likewise eqn \ref{se4} yields \ref{sem4} in this limit.
\begin{table}
\caption{Field equations}\label{T3}
\begin{center}
\begin{tabular}{|c|c|} \hline 
${\rm Electric ~~limit}$ & ${\rm Magnetic ~~limit}$ \\ \hline
  $\6^i \6_t a_i - \6^i \6_i a_0 -k^2 \Big(a_0 + \6_t \phi \Big) = 0$ & $\6^i \6_i a_0 +k^2 a_0 = 0$ \\ \hline
$\6^i \6_i a_j - \6^i \6_j a_i + k^2 \Big(a_j + \6_j \phi \Big) = 0  $ & $\6_t \6_j a_0 + \6^i \6_j a_i - \6^i \6_i a_j - k^2 \Big( a_j + \6_j \phi \Big) = 0$ \\ \hline
$ \partial^i \partial_i a^0 +k^2 a^0 = 0 $ & $\partial_i ~\partial_t~ a^i - \partial^i ~\partial_i ~a^0 - k^2 \Big( a^0 + \6_t \psi \Big)  = 0 $ \\ \hline
$\6_t \6^j a^0 + \6_i \6^j a^i - \6_i \6^i a^j - k^2 \Big( a^j + \6^j \psi \Big) = 0$ & 
$ \6_i \6^i a^j - \6_i \6^j a^i + k^2 \Big( a^j + \6^j \psi \Big)  = 0 $ \\ \hline
\end{tabular}
\label{T3}
\end{center}
\end{table}
From table \ref{T3} we can observe that
\begin{equation*}
{\rm   Contravariant \longleftrightarrow Covariant} \implies {\rm Electric \longleftrightarrow Magnetic} ~~\& ~~\phi \longleftrightarrow \psi
\end{equation*}
\subsection{Gauge invariance}
We have already shown that the relativistic lagrangian is gauge invariant. Here we show that the Galilean counterpart is also gauge invariant. Here we discuss electric and magnetic limit cases separately.\\
\noindent \underline{\bf Electric limit:}\\
\noindent In this limit the relations given in \ref{G1} and \ref{G2} will give rise
\begin{equation}
    a^0 \to a^0, \,\,\, a^i \to a^i + \6^i \beta, \,\,\, a_0 \to a_0 + \6_t \alpha, \,\,\, a_i \to a_i + \6_i \alpha \label{g1}
\end{equation}
Using eqn \ref{g1} we can show that
\begin{equation}
    \mathcal{L}_{\rm SPe} \to \mathcal{L}_{\rm SPe}
\end{equation}
\noindent \underline{\bf Magnetic limit:}\\
\noindent In this limit the relations given in \ref{G1} and \ref{G2} will give rise
\begin{equation}
    a^0 \to a^0 + \6_t \beta, \,\,\, a^i \to a^i + \6^i \beta, \,\,\, a_0 \to a_0, \,\,\, a_i \to a_i + \6_i \alpha \label{g2}
\end{equation}
Using eqn \ref{g2} we can show that
\begin{equation}
    \mathcal{L}_{\rm SPm} \to \mathcal{L}_{\rm SPm}
\end{equation}
\section{Conclusions} \label{sec7}
\noindent Let us now summarise the new significant findings of the paper, comparing with existing results found in the literature. 
\begin{itemize}
    \item An unambiguous construction of the non-relativistic (NR) Proca lagrangian, for both electric and magnetic limits, was given. The non-relativistic analogue of massive spin-1 particle has been discussed in \cite{Leblond2} by using a group theory point of view and a short discussion has been presented in \cite{Santos} which we found incomplete in several ways. For instance in \cite{Leblond2} only the magnetic limit is considered while \cite{Santos} presents a mixed lagrangian with both electric and magnetic sectors. Nowhere a systematic construction of the Galilean invariant lagrangian has been presented so far. Here we have deduced this lagrangian from the standard relativistic lagrangian adopting the dictionary given in section \ref{sec2}. It is expressed either in terms of potentials or fields. Explicit demonstration of boost invariance has been presented.
    \vspace{0.1in}
    \item Here we have considered contravariant and covariant vectors separately as we have done previously in the context of Galilean relativistic version of Maxwell theory \cite{Bhattacharya}. This helps us to discover certain subtleties which were unobserved so far. For example it is observed from table \ref{T2} that if we replace the covariant components by contravariant ones in the electric limit case we will end up with the magnetic limit case and vice-versa.
    \vspace{0.1in}
    \item A central point is the formulation of a dictionary that translates four vectors in the relativistic theory to their corresponding vectors in the non-relativistic theory. Also, we know gauge symmetries play a pivotal role in the understanding of gauge theories. Since there is no gauge symmetry present in the Proca theory it is interesting to observe its consequence in the Galilean invariant theory which has been derived by using the dictionary presented here. We also discuss in section \ref{sec6} how to restore the gauge invariance by using the Stuckelberg embedding. We derive its Galielan counterpart by writing down the lagrangian for both the limits. We show that the Galilean lagrangians for both the limits are gauge invariant. Another interesting thing to observe is the change of contravariant indices by covariant ones induces not only the electric limit to magnetic limit but the Stuckelberg scalars $\phi$ and $\psi$ also get interchanged. 
\end{itemize}
\noindent \underline{\bf Future prospects:}
\begin{itemize}
    \item There have been some observational aspects of Proca models in the context of  gravitational waves \cite{Radu1} and black hole shadows \cite{Radu2}. We believe a further systematic analysis of the non-relativistic Proca theory can shed some important light in this direction.
    \vspace{0.1in}
    \item  Recently its has been observed in \cite{derham} that a certain type of  non-linear Proca field theory which is sometimes called Proca-Nuevo model plays a crucial role in explaining the emergence of half degrees of freedom in Lorentz and parity invariant field theories. It will be interesting to look at the Galilean invariant counterpart and study its further implications. 
    \vspace{0.1in}
    \item Usual Proca model can be generalised by considering derivative
self-interactions with only three propagating degrees of freedom \cite{Heisenberg}. This is sometimes called generalised Proca model. It will be interesting to study its Galilean invariant counterpart. The generalised Proca model can be extended to curved spacetime which gives rise to Horndeski Proca model. We are interested to study the non-relativistic analogue (in a curved background) by adopting the approach of \cite{RB1, RB2}. 
\vspace{0.1in}
 \item Generalised Proca theories of gravity represent an interesting class of vector-tensor theories which can be extended to modified theories of gravity by including torsion \cite{Said}. Considering the recent interests and applications of non-Lorentzian physics in different contexts, we believe the study of the non-relativistic limit will shed some light on various till unknown aspects of generalised Proca model as well as the modified gravity theory.
\vspace{0.1in}
\item Finally we want to look at the Carrollean limit \cite{Duval} of the Proca model. 
\end{itemize}
We expect we can address these issues in the near future.
\section{Acknowledgements}
The authors (RB and SB) acknowledge the support from a DAE Raja Ramanna Fellowship (grant no: $1003/(6)/2021/RRF/R\&D-II/4031$, dated: $20/03/2021$).
%\begin{appendices}
%\section{Noether current calculations}
%\end{appendices}
\begin{thebibliography}{99}
\bibitem{Taylor} M. Taylor, Non-relativistic holography, 0812.0530. 
%\bibitem{Andringa} R. Andringa, E. A. Bergshoeff, S. Panda and M. de Roo, Class. Quantum Grav. 28 105011 (2011).
\bibitem{andreev1}  O. Andreev, M. Haack and S. Hofmann, Phys. Rev. D 89, 064012
(2014) doi:10.1103/PhysRevD.89.064012 [arXiv:1309.7231 [hep-th]].
\bibitem{andreev2}  O. Andreev, Phys. Rev. D 91, no. 2, 024035 (2015)
doi:10.1103/PhysRevD.91.024035 [arXiv:1408.7031 [hep-th]].
\bibitem{jensen}  K. Jensen and A. Karch, JHEP 1504, 155 (2015)
doi:10.1007/JHEP04(2015)155 [arXiv:1412.2738 [hep-th]].
\bibitem{rb1} R. Banerjee and P. Mukherjee, Phys. Rev. D 93, no. 8, 085020 (2016)
doi:10.1103/PhysRevD.93.085020 [arXiv:1509.05622 [gr-qc]].
\bibitem{rb2}  R. Banerjee, S. Gangopadhyay and P. Mukherjee, Int. J. Mod. Phys.
A 32, no. 19n20, 1750115 (2017) doi:10.1142/S0217751X17501159
[arXiv:1604.08711 [hep-th]].
%\bibitem{Son} D.T. Son and M. Wingate, Annals. of. Physics. 321, 197-224 (2006).
\bibitem {Pal} B. Grinstein and S. Pal, Phys. Rev. D 97, no. 12, 125006 (2018).
%\bibitem{Geracie} M. Geracie, arXiv:1611.01198 [hep-th]. 
\bibitem{Jain} A. Jain, Phys. Rev. D 93, no. 6, 065007 (2016).
\bibitem{rb3} R. Banerjee and P. Mukherjee, doi: 10.1016/j.nuclphysb.2018.11.002, [arXiv: 1801.08373].
%\bibitem{Mitra} A. Mitra, Int. J. Mod. Phys. A 32, no. 36, 1750206 (2017).
\bibitem{Morand} K. Morand, Embedding Galilean and Carrollian geometries I. Gravitational waves, Journal of Mathematical Physics 61, 082502 (2020), [arXiv:1811.12681 [hep-th]].
%\bibitem{Read}  J. Read and N. J. Teh, Class. Quant. Grav. 35, no. 18, 18LT01 (2018).
%\bibitem{gkk} G. K. Karananas, doi:10.5075/epfl-thesis-7173.
\bibitem{Leblond} M. L. Bellac and J.-M. Levy-Leblond, Galilean Electromagnetism, Nuovo Cimento. 14B (1973) .
\bibitem{Germain} G. Rousseaux, Forty years of Galilean Electromagnetism (1973–2013), Eur. Phys. J. Plus (2013) 128: 81.
%\bibitem{Bagchi} A. Bagchi and R. Gopakumar, Galilean Conformal Algebras and AdS/CFT, JHEP 07 (2009) 037 [0902.1385].
%\bibitem{Khanna} E. S. Santos, M. de Montigny, F. C. Khanna and A. E. Santana, Galilean covariant Lagrangian models, J. Phys. A 37 (2004) 9771.
\bibitem{Duval} C. Duval, G. W. Gibbons, P. A. Horvathy and P. M. Zhang, Carroll versus Newton and Galilei: two dual non-Einsteinian concepts of time, Class. Quant. Grav. 31 (2014) 085016
[1402.0657].
\bibitem{Mehra1} A. Bagchi, R. Basu and A. Mehra, Galilean Conformal Electrodynamics, JHEP 11 (2014) 061 [1408.0810].

%\bibitem{Mehra2} A. Bagchi, R. Basu, A. Kakkar and A. Mehra, Galilean Yang-Mills Theory, JHEP 04 (2016) 051 [1512.08375].
%\bibitem{Bleeken} D. Van den Bleeken and C. Yunus, Newton-Cartan, Galileo-Maxwell and Kaluza-Klein, Class. Quant. Grav. 33 (2016) 137002 [1512.03799].
%\bibitem{Bergshoeff} E. Bergshoeff, J. Rosseel and T. Zojer, Non-relativistic fields from arbitrary contracting backgrounds, Class. Quant. Grav. 33 (2016) 175010 [1512.06064].
%\bibitem{Festuccia} G. Festuccia, D. Hansen, J. Hartong and N. A. Obers, Symmetries and Couplings of Non-Relativistic Electrodynamics, JHEP 11 (2016) 037 [1607.01753].
%\bibitem{Basu} K. Banerjee, R. Basu and A. Mohan, Uniqueness of Galilean Conformal Electrodynamics and its Dynamical Structure, JHEP 11 (2019) 041 [1909.11993].
\bibitem{Mehra3} A. Mehra and Y. Sanghavi, Galilean electrodynamics: covariant formulation and Lagrangian, JHEP 09 (2021) 078 [2107.08525].
%\bibitem{Mehra4} A. Bagchi, J. Chakrabortty and A. Mehra, Galilean Field Theories and Conformal Structure, JHEP 04 (2018) 144 [1712.05631]. 

%\bibitem{Chapman} S. Chapman, L. Di Pietro, K. T. Grosvenor and Z. Yan, Renormalization of Galilean Electrodynamics, JHEP 10 (2020) 195 [2007.03033].
%\bibitem{Sharma} K. Banerjee and A. Sharma, Quantization of Interacting Galilean Field theories, JHEP 2022, 66 (2022) [arXiv: hep-th/2205.01918].
\bibitem{Bhattacharya} R. Banerjee and S. Bhattacharya, A new formulation of Galilean electrodynamics , [arXiv: hep-th/2211.12023].
\bibitem{Proca} A. Proca, J. Phys. Radium 7, 347 (1936).
\bibitem{Luo} L.C. Tu, J. Luo, G.T. Gillies, The mass of the photon, Rep. Prog. Phys. 68, 77 (2005).
\bibitem{Nieto} A.S. Goldhaber, M.M. Nieto, Photon and graviton mass limits, Rev. Mod. Phys. 82, 939 (2010).
\bibitem{Sampaio} C. Herdeiro, M.O.P. Sampaio, M. Wang, Hawking radiation for a Proca field in $D$ dimensions, Phys. Rev. D 85, 024005 (2012).
\bibitem{Dvali} G. Dvali, M. Papucci, M.D. Schwartz, Infrared Lorentz Violation and Slowly Instantaneous Electricity, Phys. Rev. Lett. 94, 191602 (2005).
\bibitem{Tomaschitz}  R. Tomaschitz, Tachyonic spectral fits of $\gamma$-ray bursts, Europhys. Lett. 89, 39002 (2010).
\bibitem{Radu1}  N. Sanchis-Gual, J. C. Bustillo, C. Herdeiro, E. Radu, J. A. Font, S. H. W. Leong, A. Torres-Forné, Impact of the wave-like nature of Proca stars on their gravitational-wave emission, Phys. Rev. D 106, 124011, [arXiv: 2208.11717].
\bibitem{Radu2}  I. Sengo, P. V. P. Cunha, C. A. R. Herdeiro, E. Radu, Kerr black holes with synchronised Proca hair: lensing, shadows and EHT constraints, JCAP 01 (2023) 047.
\bibitem{Demir}  D. Demir and B. Pulice,  Geometric Proca with Matter in Metric-Palatini Gravity,
 Eur. Phys. J. C 82, 996 (2022).
 \bibitem{Heisenberg} L. Heisenberg, Generalization of the Proca Action, JCAP 1405 (2014) 015 [arXiv: 1402.7026].
 \bibitem{Said}  G. P. Nicosia, J. L. Said and V. Gakis, Generalised Proca Theories in Teleparallel Gravity, EPJP, volume 136, Article number: 191 (2021). 
 \bibitem{Pichet}  J. Sanongkhun and P. Vanichchapongjaroen, On constrained analysis and diffeomorphism invariance of generalised Proca theories, General Relativity and Gravitation volume 52, Article number: 26 (2020).
 \bibitem{derham} C. de Rham, S. Garcia-Saenz, L. Heisenberg, V. Pozsgay and X. Wang,  To Half–Be or Not To Be?, [arXiv: 2303.05354]. 

 
\bibitem{Sengupta}  N. D. Sengupta, On an Analogue of the Galilei Group,  Nuovo Cim. 54 (1966) 512, DOI: 10.1007/BF02740871.
\bibitem{Leblond2} Lévy-Leblond J. M. 1967 Nonrelativistic particles and wave equations Commun. Math. Phys. 6 286–311.
\bibitem{Santos} E. S. Santos, M. de Montigny, F. C. Khanna and A. E. Santana, Galilean covariant Lagrangian models, J. Phys. A 37 (2004) 9771.

\bibitem{RB1} R. Banerjee, A. Mitra and P. Mukherjee, A new formulation of non-relativistic diffeomorphism invariance,  Phys.Lett.B 737 (2014) 369-373, [arXiv: 1404.4491 [gr-qc]].
\bibitem{RB2} R. Banerjee, A. Mitra and P. Mukherjee, Localization of the Galilean symmetry and dynamical realization of Newton-Cartan geometry, Class.Quant.Grav. 32 (2015) 4, 045010, [arXiv: 1407.3617 [hep-th]].




\end{thebibliography}
\end{document}

\section{Inclusion of sources} \label{sec7}
The relativistic Maxwell Lagrangian with source is as follows
\begin{equation}
    \mathcal{L}_{\rm Proca} = -\frac{1}{4} F_{\alpha \beta} F^{\alpha \beta} - A_{\alpha} J^{\alpha} + \frac{k^2}{2} A_\mu A^\mu
\end{equation}
And this Lagrangian gives rise folloing equations of motion
\begin{equation}
    \6_\mu F^{\mu \nu} + k^2 A^\nu = J^\nu
\end{equation}
We can write the Lagrangian in the following form for convenience
\begin{equation}
    \mathcal{L}_{\rm Proca} = -\frac{1}{4} \Big(2 F_{0i}F^{0i} + F_{ij} F^{ij} \Big) - \frac{1}{2} A_{\a} J^{\a} - \frac{1}{2} A^{\a} J_{\a} + \frac{k^2}{2} \Big( A_0 A^0 + A_i A^i \Big)
\end{equation}
We know that the Maxwell theory respects the following gauge transformations
\begin{equation}
    A_\mu \to A_\mu + \6_\mu \Lambda
\end{equation}
The gauge invariance of the Lagrangian demands following condition
\begin{equation}
    \Big(A_\mu + \6_\mu \Lambda \Big) J^\mu = A_\mu J^\mu + \Lambda \6_\mu J^\mu \implies \6_\mu J^\mu = 0 
\end{equation}
\noindent \underline {\bf Electric limit:}\\
\noindent In the electric limit the scaling will be as follows
\begin{eqnarray}
A^0 \to c a^0, \,\, A^i \to a^i, \,\, A_0 \to \frac{a_0}{c}, \,\, A_i \to a_i \nonumber \\ J^0 \to c j^0, \,\, J^i \to j^i, \,\, J_0 \to \frac{j_0}{c}, \,\, J_i \to j_i
\end{eqnarray}
In this limit the Lagrangian looks like 
\begin{equation}
    \mathcal{L}_e = \frac{1}{2} \6^i a^0 \Big(\6_t a_i - \6_i a_0  \Big) -\frac{1}{4} f_{ij} f^{ij} - \frac{1}{2} a_0 j^0 -\frac{1}{2} a_i j^i - \frac{1}{2} a^0 j_0 -\frac{1}{2} a^i j_i + \frac{k^2}{2} \Big( a_0 a^0 + a_i a^i \Big)
\label{ls1}    
\end{equation}
Varying the lagrangian with respect to $a_0, ~a_j, ~a^0, ~a^j$ will give following set of equations
\begin{eqnarray}
 \6^i \6_i a^0 + k^2 a^0 = j^0
 \label{es1}
\end{eqnarray}
\begin{eqnarray}
 \6_t \6^j a^0 + \6_i \6^j a^i - \6_i \6^i a^j - k^2 a^j = - j^j \label{es2}
\end{eqnarray}
\begin{eqnarray}
\6^i \6_t a_i - \6^i \6_i a_0 - k^2 a_0 = -j_0 \label{es3}
\end{eqnarray}
\begin{eqnarray}
\6^i \6_i a_j - \6^i \6_j a_i + k^2 a_j = j_j \label{es4}
\end{eqnarray}
We now derive these equations directly from the equations of motion. The relativistic equations are given by,
\begin{equation}
    \6_{\mu} F^{\mu \nu} + k^2 A^\nu = J^{\nu}
\end{equation}
Which can be written as 
\begin{eqnarray}
\6_i F^{i0} + k^2 A^0 =J^0 \label{ems1} \\
\6_0 F^{0j} + \6_i F^{ij} + k^2 A^j = J^j \label{ems2}
\end{eqnarray}
From the first equation \ref{ems1} we get 
\begin{eqnarray}
\6_i \Big(c \6^i a^0 + \frac{1}{c} \6_t a^i \big) + c k^2 a^0 = c j^0 \nonumber \\\xrightarrow[\text{$c \to \infty$}]{\text{}}
\6_i \6^i a^0 + k^2 a^0 = j^0
\end{eqnarray}
which reproduces eqn \ref{es1}
From the second equation \ref{ems2} we have
\begin{eqnarray}
\frac{1}{c} \6_t \Big(-\frac{1}{c} \6_t a^j - c \6^j a^0 \Big) + \6_i \Big( \6^i a^j - \6^j a^i  \Big) + k^2 a^j = j^j \nonumber \\ \xrightarrow[\text{$c \to \infty$}]{\text{}} \6_t \6^j a^0 + \6_i \6^j a^i - \6_i \6^i a^j - k^2 a^j = -j^j
\end{eqnarray}
which reproduces eqn \ref{es2}. 
To get the remaining pair of equations we have to interpret
eqns \ref{ems1} and \ref{ems2} as 
\begin{eqnarray}
\6^i F_{i0} + k^2 A_0 = J_0 \label{ems3} \\
\6^0 F_{0j} + \6^i F_{ij} + k^2 A_j = J_j \label{ems4}
\end{eqnarray}
The first of them in the galilean limit yields,
 \begin{eqnarray}
 \frac{1}{c} \6^i \Big(\6_i a_0 - \6_t a_i \Big) + \frac{1}{c} k^2 a_0 = \frac{j_0}{c} \\ \xrightarrow[\text{$c \to \infty$}]{\text{}} \6^i \6_t a_i - \6^i \6_i a_0 - k^2 a_0 = -j_0
 \end{eqnarray}
 which reproduces eqn \ref{es3}. Similarly \ref{ems4} yields eqn \ref{es4}. This shows the consistency of the eqn of motion in Galilean electrodynamics with source.\\
 \noindent \underline {\bf Magnetic limit:}\\
Here the scalings are as follows
\begin{eqnarray}
A^0 \to -\frac{a^0}{c},\,\, A^i \to a^i, \, \, A_0 \to -c a_0, \,\, A_i \to a_i \nonumber \\ J^0 \to -\frac{j^0}{c}, \,\, J^i \to j^i, \,\, J_0 \to -c j_0, \,\, J_i \to j_i 
\end{eqnarray}
The Lagrangian in this limit is as follows 
\begin{equation}
    \mathcal{L}_m = \frac{1}{2} \6_i a_0 \Big(\6_t a^i - \6^i a^0 \Big) - \frac{1}{4} f_{ij}f^{ij} - \frac{1}{2} a_0 j^0 -\frac{1}{2} a_i j^i - \frac{1}{2} a^0 j_0 -\frac{1}{2} a^i j_i + \frac{k^2}{2} \Big(a_0 a^0 + a_i a^i  \Big)
\end{equation}
Varying the lagrangian wrt $a_0, ~a_j, ~a^0, ~a^j$ we get the following set of equations
\begin{eqnarray}
 \6_i \6_t a^i - \6_i \6^i a^0 -k^2 a^0 = -j^0 \label{ms1}
\end{eqnarray}
\begin{eqnarray}
 \6_i \6^i a^j - \6_i \6^j a^i + k^2 a^j = j^j \label{ms2}
\end{eqnarray}
\begin{eqnarray}
\6^i \6_i a_0 +k^2 a_0 = j_0 \label{ms3}
\end{eqnarray}
\begin{eqnarray}
\6_t \6_j a_0 - \6^i \6_i a_j + \6^i \6_j a_i - k^2 a_j = - j_j \label{ms4}
\end{eqnarray}
We now derive these equations directly from the equations of motion. Eqn \ref{ems1} will give
\begin{eqnarray}
 \6_i\Big(-\frac{1}{c} \6^i a^0 + \frac{1}{c} \6_t a^i  \Big)-\frac{1}{c}k^2 a^0 = -\frac{j^0}{c} \nonumber \\ \implies \6_i\6_t a^i - \6_i \6^i a^0 -k^2 a^0= -j^0
\end{eqnarray}
Eqn \ref{ems2} yields
\begin{eqnarray}
\frac{1}{c} \Big(-\frac{1}{c} \6_t a^j + \frac{1}{c} \6^j a^0 \Big) + \6_i \6^i a^j - \6_i \6^j a^i + k^2 a^j= j^j \nonumber \\ \xrightarrow[\text{$c \to \infty$}]{\text{}} \6_i \6^i a^j - \6_i \6^j a^i + k^2 a^j = j^j
\end{eqnarray}
Likewise eqns \ref{ems3} and \ref{ems4} yields eqns \ref{ms3} and \ref{ms4} respectively. The field equations for both electric and magnetic limit have been shown in table \ref{T7}.
\begin{table}
\caption{Field equations}\label{T7}
\begin{center}
\begin{tabular}{|c|c|c|} \hline 
${\rm Variables}$ & ${\rm Electric ~~limit}$ & ${\rm Magnetic ~~limit}$ \\ \hline
$a^0$  & $\6^i \6_t a_i - \6^i \6_i a_0 -k^2 a_0 = -j^0$ & $\6^i \6_i a_0 + k^2 a_0= j^0$ \\ \hline
$a^j$ & $\6^i \6_i a_j - \6^i \6_j a_i + k^2 a_j= j_j $ & $\6_t \6_j a_0 + \6^i \6_j a_i - \6^i \6_i a_j - k^2 a_j = -j_j$ \\ \hline
$a_0$ & $ \6^i \6_i a^0 + k^2 a^0= j^0 $ & $\6_i \6_t a^i - \6^i ~\6_i ~a^0 - k^2 a^0= -j^0 $ \\ \hline
$a_j$ & $\6_t \6^j a^0 + \6_i \6^j a ^i - \6_i \6^i a^j - k^2 a^j= -j^j$ & 
$\6_i\6^i a^j - \6_i \6^j a^i + k^2 a^j= j^j $ \\ \hline
\end{tabular}
\label{T7}
\end{center}
\end{table}

\section{Inclusion of sources} \label{sec7}
The relativistic Maxwell Lagrangian with source is as follows
\begin{equation}
    \mathcal{L}_{\rm Proca} = -\frac{1}{4} F_{\alpha \beta} F^{\alpha \beta} - A_{\alpha} J^{\alpha} + \frac{k^2}{2} A_\mu A^\mu
\end{equation}
And this Lagrangian gives rise folloing equations of motion
\begin{equation}
    \6_\mu F^{\mu \nu} + k^2 A^\nu = J^\nu
\end{equation}
We can write the Lagrangian in the following form for convenience
\begin{equation}
    \mathcal{L}_{\rm Proca} = -\frac{1}{4} \Big(2 F_{0i}F^{0i} + F_{ij} F^{ij} \Big) - \frac{1}{2} A_{\a} J^{\a} - \frac{1}{2} A^{\a} J_{\a} + \frac{k^2}{2} \Big( A_0 A^0 + A_i A^i \Big)
\end{equation}
We know that the Maxwell theory respects the following gauge transformations
\begin{equation}
    A_\mu \to A_\mu + \6_\mu \Lambda
\end{equation}
The gauge invariance of the Lagrangian demands following condition
\begin{equation}
    \Big(A_\mu + \6_\mu \Lambda \Big) J^\mu = A_\mu J^\mu + \Lambda \6_\mu J^\mu \implies \6_\mu J^\mu = 0 
\end{equation}
\noindent \underline {\bf Electric limit:}\\
\noindent In the electric limit the scaling will be as follows
\begin{eqnarray}
A^0 \to c a^0, \,\, A^i \to a^i, \,\, A_0 \to \frac{a_0}{c}, \,\, A_i \to a_i \nonumber \\ J^0 \to c j^0, \,\, J^i \to j^i, \,\, J_0 \to \frac{j_0}{c}, \,\, J_i \to j_i
\end{eqnarray}
In this limit the Lagrangian looks like 
\begin{equation}
    \mathcal{L}_e = \frac{1}{2} \6^i a^0 \Big(\6_t a_i - \6_i a_0  \Big) -\frac{1}{4} f_{ij} f^{ij} - \frac{1}{2} a_0 j^0 -\frac{1}{2} a_i j^i - \frac{1}{2} a^0 j_0 -\frac{1}{2} a^i j_i + \frac{k^2}{2} \Big( a_0 a^0 + a_i a^i \Big)
\label{ls1}    
\end{equation}
Varying the lagrangian with respect to $a_0, ~a_j, ~a^0, ~a^j$ will give following set of equations
\begin{eqnarray}
 \6^i \6_i a^0 + k^2 a^0 = j^0
 \label{es1}
\end{eqnarray}
\begin{eqnarray}
 \6_t \6^j a^0 + \6_i \6^j a^i - \6_i \6^i a^j - k^2 a^j = - j^j \label{es2}
\end{eqnarray}
\begin{eqnarray}
\6^i \6_t a_i - \6^i \6_i a_0 - k^2 a_0 = -j_0 \label{es3}
\end{eqnarray}
\begin{eqnarray}
\6^i \6_i a_j - \6^i \6_j a_i + k^2 a_j = j_j \label{es4}
\end{eqnarray}
We now derive these equations directly from the equations of motion. The relativistic equations are given by,
\begin{equation}
    \6_{\mu} F^{\mu \nu} + k^2 A^\nu = J^{\nu}
\end{equation}
Which can be written as 
\begin{eqnarray}
\6_i F^{i0} + k^2 A^0 =J^0 \label{ems1} \\
\6_0 F^{0j} + \6_i F^{ij} + k^2 A^j = J^j \label{ems2}
\end{eqnarray}
From the first equation \ref{ems1} we get 
\begin{eqnarray}
\6_i \Big(c \6^i a^0 + \frac{1}{c} \6_t a^i \big) + c k^2 a^0 = c j^0 \nonumber \\\xrightarrow[\text{$c \to \infty$}]{\text{}}
\6_i \6^i a^0 + k^2 a^0 = j^0
\end{eqnarray}
which reproduces eqn \ref{es1}
From the second equation \ref{ems2} we have
\begin{eqnarray}
\frac{1}{c} \6_t \Big(-\frac{1}{c} \6_t a^j - c \6^j a^0 \Big) + \6_i \Big( \6^i a^j - \6^j a^i  \Big) + k^2 a^j = j^j \nonumber \\ \xrightarrow[\text{$c \to \infty$}]{\text{}} \6_t \6^j a^0 + \6_i \6^j a^i - \6_i \6^i a^j - k^2 a^j = -j^j
\end{eqnarray}
which reproduces eqn \ref{es2}. 
To get the remaining pair of equations we have to interpret
eqns \ref{ems1} and \ref{ems2} as 
\begin{eqnarray}
\6^i F_{i0} + k^2 A_0 = J_0 \label{ems3} \\
\6^0 F_{0j} + \6^i F_{ij} + k^2 A_j = J_j \label{ems4}
\end{eqnarray}
The first of them in the galilean limit yields,
 \begin{eqnarray}
 \frac{1}{c} \6^i \Big(\6_i a_0 - \6_t a_i \Big) + \frac{1}{c} k^2 a_0 = \frac{j_0}{c} \\ \xrightarrow[\text{$c \to \infty$}]{\text{}} \6^i \6_t a_i - \6^i \6_i a_0 - k^2 a_0 = -j_0
 \end{eqnarray}
 which reproduces eqn \ref{es3}. Similarly \ref{ems4} yields eqn \ref{es4}. This shows the consistency of the eqn of motion in Galilean electrodynamics with source.\\
 \noindent \underline {\bf Magnetic limit:}\\
Here the scalings are as follows
\begin{eqnarray}
A^0 \to -\frac{a^0}{c},\,\, A^i \to a^i, \, \, A_0 \to -c a_0, \,\, A_i \to a_i \nonumber \\ J^0 \to -\frac{j^0}{c}, \,\, J^i \to j^i, \,\, J_0 \to -c j_0, \,\, J_i \to j_i 
\end{eqnarray}
The Lagrangian in this limit is as follows 
\begin{equation}
    \mathcal{L}_m = \frac{1}{2} \6_i a_0 \Big(\6_t a^i - \6^i a^0 \Big) - \frac{1}{4} f_{ij}f^{ij} - \frac{1}{2} a_0 j^0 -\frac{1}{2} a_i j^i - \frac{1}{2} a^0 j_0 -\frac{1}{2} a^i j_i + \frac{k^2}{2} \Big(a_0 a^0 + a_i a^i  \Big)
\end{equation}
Varying the lagrangian wrt $a_0, ~a_j, ~a^0, ~a^j$ we get the following set of equations
\begin{eqnarray}
 \6_i \6_t a^i - \6_i \6^i a^0 -k^2 a^0 = -j^0 \label{ms1}
\end{eqnarray}
\begin{eqnarray}
 \6_i \6^i a^j - \6_i \6^j a^i + k^2 a^j = j^j \label{ms2}
\end{eqnarray}
\begin{eqnarray}
\6^i \6_i a_0 +k^2 a_0 = j_0 \label{ms3}
\end{eqnarray}
\begin{eqnarray}
\6_t \6_j a_0 - \6^i \6_i a_j + \6^i \6_j a_i - k^2 a_j = - j_j \label{ms4}
\end{eqnarray}
We now derive these equations directly from the equations of motion. Eqn \ref{ems1} will give
\begin{eqnarray}
 \6_i\Big(-\frac{1}{c} \6^i a^0 + \frac{1}{c} \6_t a^i  \Big)-\frac{1}{c}k^2 a^0 = -\frac{j^0}{c} \nonumber \\ \implies \6_i\6_t a^i - \6_i \6^i a^0 -k^2 a^0= -j^0
\end{eqnarray}
Eqn \ref{ems2} yields
\begin{eqnarray}
\frac{1}{c} \Big(-\frac{1}{c} \6_t a^j + \frac{1}{c} \6^j a^0 \Big) + \6_i \6^i a^j - \6_i \6^j a^i + k^2 a^j= j^j \nonumber \\ \xrightarrow[\text{$c \to \infty$}]{\text{}} \6_i \6^i a^j - \6_i \6^j a^i + k^2 a^j = j^j
\end{eqnarray}
Likewise eqns \ref{ems3} and \ref{ems4} yields eqns \ref{ms3} and \ref{ms4} respectively. The field equations for both electric and magnetic limit have been shown in table \ref{T7}.
\begin{table}
\caption{Field equations}\label{T7}
\begin{center}
\begin{tabular}{|c|c|c|} \hline 
${\rm Variables}$ & ${\rm Electric ~~limit}$ & ${\rm Magnetic ~~limit}$ \\ \hline
$a^0$  & $\6^i \6_t a_i - \6^i \6_i a_0 -k^2 a_0 = -j^0$ & $\6^i \6_i a_0 + k^2 a_0= j^0$ \\ \hline
$a^j$ & $\6^i \6_i a_j - \6^i \6_j a_i + k^2 a_j= j_j $ & $\6_t \6_j a_0 + \6^i \6_j a_i - \6^i \6_i a_j - k^2 a_j = -j_j$ \\ \hline
$a_0$ & $ \6^i \6_i a^0 + k^2 a^0= j^0 $ & $\6_i \6_t a^i - \6^i ~\6_i ~a^0 - k^2 a^0= -j^0 $ \\ \hline
$a_j$ & $\6_t \6^j a^0 + \6_i \6^j a ^i - \6_i \6^i a^j - k^2 a^j= -j^j$ & 
$\6_i\6^i a^j - \6_i \6^j a^i + k^2 a^j= j^j $ \\ \hline
\end{tabular}
\label{T7}
\end{center}
\end{table}
\end{document}





\section{Introduction}
\noindent The formulation of non-relativistic limit of classical field theories received considerable attention recently. It has found applications in  holography \cite{Taylor}, in studying non-relativistic diffeomorphisms (NRDI) \cite{andreev1, andreev2, jensen, rb1, rb2}, 
condensed matter systems \cite{Son, Pal, Geracie}, fluid dynamics \cite{Jain, rb3}, gravitation \cite{Morand, Read} and cosmology \cite{gkk}. This formulation is tricky and markedly different from the relativistic
case. Covariance in non-relativistic physics is subtle due to the absolute nature of time. The lack of a single non-degenerate metric in the non-relativistic limit poses some additional difficulties. Here we are interested in the non-relativistic limit of the Maxwellean electrodynamics which is invariant under galilean transformations. The basic construction of the Galilean electrodynamics was first given by Le Bellac and Levy-Leblond \cite{Leblond} back in 1970's. The more recent works on different aspects of Galilean electrodynamics and gauge theories are \cite{Bagchi, Mehra1, Duval, Mehra2, Bleeken, Bergshoeff, Festuccia, Basu, Mehra3, Mehra4, Khanna, Chapman}. \\
\noindent In this paper we provide a detailed analysis of galilean electrodynamics with and without sources. While earlier findings are reproduced we also find several new results with new interpretations. We know there are two distinct non-relativistic limits possible for electrodynamics known as electric and magnetic limit \cite{Leblond}. We derive these two limits from the Lorentz transformation of an arbitrary four vector. Our derivation of the non-relativistic scaling relations  are consistent with \cite{Leblond}. All previous works so far treated only the contravariant components of any vector quantity but the novelty of our treatment is that we have considered both contravariant and covariant components separately (which are distinct quantities in the non-relativistic case) and hence help us to explore the rich symmetries involved in the theory. We then derive the Lagrangians for both electric and magnetic limits from which the equations of motion are obtained. We also show that Maxwell's equations under non-relativistic limit (electric and magnetic) yield same equations as those we get from the non-relativistic Lagrangians. This implies the internal consistency of the limiting process.  We then define the galilean electric and magnetic fields for contravariant and covariant cases.  Along the way, we discuss the dualities  and point out some of the subtleties involved in the process. Especially we show that the transformations of the electric and magnetic fields under galilean boosts is connected with the familiar duality transformations. Next we move to discuss the gauge symmetry. We have shown that we can choose different gauge parameters for contravariant and covariant four potentials as they represent different entities in the galilean limits. We then compute the galilean version of the Noether currents and explicitly show their on-shell conservation. We discuss shift symmetries which play an important role in the study of low-energy effective lagrangians in the context of Goldstone's theorem. Recenently people have expolored shift symmetries from different aspects \cite{rb4, rb5}. We compute correponding currents and their conservations in this limit. In the end we introduce sources and write down the Lagrangians for appropriate galilean limits (electric and magnetic) and write down the equations of motion just like the sourceless case. \\
\noindent The paper is organised as follows in section \ref{sec2} we derive a mapping relation between relativistic and non-relativistic vectors for electric and magnetic limit for both contravariant and covariant vectors. In section \ref{sec3} we derive the non-relativistic lagrangian for both limits and write down the equations of motion. In section \ref{sec4} we discuss Maxwell's equations in terms of fields and explore the duality relations. In section \ref{sec5} we discuss the gauge symmetry, Noether cureents and their conservations. In section \ref{sec6} we discuss the shift symmetry and it's galilean counterpart, corresponding currents and their conservations. Finally in section \ref{sec7} we discuss galilean electrodynamics by including sources for both contravariant and covariant sectors. 
\section{ Mapping relations} \label{sec2}
\noindent Here we derive a certain scaling between special relativistic and Galilean relativistic quantities. As we know there exists two types of such limits for the vector quantities namely electric and magnetic limits. So first let us consider the contravariant vectors. Let us consider a the generic Lorentz transformation with the boost velocity as $u^i$:
\begin{equation}
    x'^0 = \gamma x^0 - \frac{\gamma u_i}{c} x^i
    \label{t1}
\end{equation}
\begin{equation}
    x'^i = x^i - \frac{\gamma u^i}{c}x^0 + (\gamma -1 )\frac{u^i u_j}{u^2}x^j
    \label{t2}
\end{equation}
Under such Lorentz transformations a contravariant vector changes as
\begin{equation*}
    V'^{\mu} = \frac{\6 x'^{\mu}}{\6 x^{\nu}} V^{\nu}
\end{equation*}
We can write them component-wise as (also considering $u<<c$, so $\gamma \to 1$)
\begin{equation}
    V'^0 = V^0 - \frac{u_j}{c}V^j
    \label{contra1}
\end{equation}
\begin{equation}
    V'^i = V^i - \frac{u^i}{c} V^0
    \label{contra2}
\end{equation}
We next provide a map that relates the Lorentz vectors with their Galilean counterparts. 
\footnote{Notation: Here relativistic vectors are denoted by capital letters ($V^0, V^i$ etc) and Galilean vectors are denoted by lowercase letters ($v^0, v^i$ etc).} 
\begin{equation}
    V^0 = c v^0, \,\,\,\, V^i = v^i
    \label{contrael}
\end{equation}
This particular map corresponds to the case $\frac{V^0}{V^i} = c ~\frac{v^0}{v^i}$ in the $c \to \infty$ limit. This yields largely timelike vectors and is called 'electric limit'.
Now using eqn \ref{contrael} in eqns \ref{contra1} and \ref{contra2} we get
\begin{equation}
v'^0 = v^0
\label{v1}
\end{equation}
\begin{equation}
    v'^i = v^i - u^i v^0
    \label{v2}
\end{equation}
The above two equations define the Galilean transformations. We can write them in a single matrix equation as
\begin{equation}
\begin{pmatrix}
v'^0 \\ v'^i
\end{pmatrix} 
= \begin{pmatrix}
1 & 0\\ -u^i & 1
\end{pmatrix}
\begin{pmatrix}
v^0 \\ v^i
\end{pmatrix}
\label{V1}
\end{equation}
We now consider the magnetic limit which corresponds to largely spacelike vectors
\begin{equation}
    V^0 = -\frac{v^0}{c}, \,\,\,\, V^i = v^i
    \label{contramag}
\end{equation}
Now using \ref{contramag} in \ref{contra1} and \ref{contra2} we get

\begin{equation}
    v'^0 = v^0 + u_j v^j
    \label{v3}
\end{equation}
\begin{equation}
    v'^i = v^i
    \label{v4}
\end{equation}
which is again a galilean transformation. We can write eqn \ref{v3} and \ref{v4} as a matrix equation 
\begin{equation}
  \begin{pmatrix}
  v'^0 \\ v'^i 
  \end{pmatrix} 
  = \begin{pmatrix}
  1 & u_j \\ 0 & 1
  \end{pmatrix}
  \begin{pmatrix}
  v^0 \\ v^j
  \end{pmatrix}
  \label{V2}
\end{equation}
We will now consider the covariant vectors. We will write first the reverse transformations of eqn \ref{t1} and \ref{t2} which is 
\begin{equation}
    x^0 = \gamma x'^0 + \frac{\gamma u_i}{c} x'^i
    \label{t3}
\end{equation}
\begin{equation}
    x^i = x'^i + \frac{\gamma u^i}{c}x'^0 + (\gamma -1 )\frac{u^i u_j}{u^2}x'^j
    \label{t4}
\end{equation}
And we know covariant vectors transform as
\begin{equation*}
    V'_{\mu} = \frac{\6 X^{\nu}}{\6 x'^{\mu}} V_{\nu}
\end{equation*}
Componentwise we can again write them as
\begin{equation}
    V'_0 = V_0 + \frac{u^i}{c} V_i
    \label{cov1}
\end{equation}
\begin{equation}
    V'_i = V_i + \frac{u_i}{c} V_0
    \label{cov2}
\end{equation}
Now here we take the  electric limit in the following way, which will soon become clear
\begin{equation}
    V_0 = \frac{v_0}{c}, \,\,\,\, V_i = v_i
    \label{covel}
\end{equation}
Using \ref{covel} in \ref{cov1} and \ref{cov2} we get
\begin{eqnarray}
  v'_0 = v_0 + u^i v_i
  \label{v5}
\end{eqnarray}
\begin{eqnarray}
  v'_i = v_i
  \label{v6}
\end{eqnarray}
which are again Galilean transformations. We can write \ref{v5} and \ref{v6} as a matrix equation as
\begin{equation}
 \begin{pmatrix}
 v'_0 \\ v'_i 
 \end{pmatrix} 
 = \begin{pmatrix}
 1 & u_i \\ 0 & 1
 \end{pmatrix}
 \begin{pmatrix}
 v_0 \\ v_i
 \end{pmatrix}
 \label{V3}
\end{equation}
We will now consider the magnetic limit as 
\begin{equation}
    V_0 = -c v_0, \,\,\,\, V_i = v_i
    \label{covmag}
\end{equation}
Using \ref{covmag} in \ref{cov1} and \ref{cov2} we get 
\begin{equation}
    v_0' = v_0
    \label{v7}
\end{equation}
\begin{equation}
    v'_i = v_i - u_i v_0 
    \label{v8}
\end{equation}
We can write \ref{v7} and \ref{v8} as 
\begin{equation}
    \begin{pmatrix}
    v'_0 \\ v'_i 
    \end{pmatrix}
    = \begin{pmatrix}
    1 & 0 \\ -u_i & 1 
    \end{pmatrix}
    \begin{pmatrix}
    v_0 \\ v_i
    \end{pmatrix}
    \label{V4}
\end{equation}
We can show that the transformation matrix in \ref{V1} and the transpose of the matrix \ref{V3} satisfies
\begin{equation}
 \begin{pmatrix}
1 & 0\\ -v^i & 1
\end{pmatrix}
 \begin{pmatrix}
1 & 0\\ v^i & 1
\end{pmatrix} = \begin{pmatrix}
1 & 0 \\ 0 & 1
\end{pmatrix}
\end{equation}
Similarly the transformation matrix in equation \ref{V2} and the transpose of the transformation matrix in \ref{V4} satisfies
\begin{equation}
 \begin{pmatrix}
  1 & v_j \\ 0 & 1
  \end{pmatrix} 
  \begin{pmatrix}
   1 & -v_j \\ 0 & 1
  \end{pmatrix} = \begin{pmatrix}
   1 & 0 \\ 0 & 1
  \end{pmatrix}
\end{equation}

\noindent To justify the limiting prescriptions even further, we consider the norm preservation for both electric and magnetic limits. Let us first consider the norm in the electric limit
\begin{equation}
    V^0 V_0 + V^i V_i \xrightarrow[\text{limit}]{\text{electric}} \Big(c v^0 \Big)\Big(\frac{v_0}{c} \Big) + \Big(v^i\Big) \Big( v_i\Big) = v^0 v_0 + v^i v_i
\end{equation}
which clearly indicates that under the scaling electric limit the norm is preserved. Now we consider the norm in the magnetic limit
\begin{equation}
    V^0 V_0 + V^i V_i \xrightarrow[\text{limit}]{\text{magnetic}} \Big(-c v_0 \Big)\Big(\frac{-v^0}{c} \Big) + \Big(v^i\Big) \Big( v_i\Big) = v^0 v_0 + v^i v_i
\end{equation}
The norm is again conserved and the role of the minus sign before the scaling is also very important in this context. The mapping relations are summarised in the table \ref{T1}.
\begin{table}
\caption{Mapping relations}\label{T1}
\begin{center}
\begin{tabular}{|c|c|c|} \hline 
${\rm Limit}$  & $ {\rm Contravariant~~mapping}$ & $ {\rm Covariant~~mapping}$  \\ \hline
${\rm Electric ~~limit}$ & $V^0 \to c~v^0, \,\,  V^i \to v^i$ & $V_0 \to \frac{v_0}{c}, \,\, V_i \to v_i$ \\ \hline
${\rm Magnetic ~~limit}$ & $V^0 \to -\frac{v^0}{c},\,\,V^i \to v^i $ & $V_0 \to -c~v_0. \,\, V_i \to v_i$ \\
\hline
\end{tabular}
\label{T1}
\end{center}
\end{table}

\section{Lagrangian and field equations} \label{sec3}
\noindent Now let us start from the relativistic Maxwell theory described by the Lagrangian
\begin{equation}
    \mathcal{L} = -\frac{1}{4}~F_{\mu \nu}~F^{\mu \nu} = -\frac{1}{4} \eta_{\mu \alpha} \eta_{\nu \beta} F^{\alpha \beta} F^{\mu \nu}
\end{equation}
where $F_{\mu \nu} = \partial_{\mu} A_{\nu} - \partial_{\nu} A_{\mu}$ and $\eta_{\mu \nu}$ is the flat space metric with signature $\Big(-,+,+,+\Big)$. We can write the Lagrangian as follows:
\begin{equation}
    \mathcal{L} = -\frac{1}{4}\Big(2F_{0i}F^{0i} + F_{ij}F^{ij} \Big)
    \label{l1}
\end{equation}
\subsection{Electric limit}
Now using the relations given in table \ref{T1} we can write the two terms in \ref{l1} as, 
\begin{eqnarray}
2 F_{0i}F^{0i} = 2 \Big(\frac{1}{c} \partial_t A_i - \partial_i A_0 \Big)\Big(-\frac{1}{c} \partial_t A^i - \partial^i A^0  \Big) \xrightarrow[\text{limit}]{\text{electric}} 2 \Big(\frac{1}{c} \partial_t a_i - \frac{1}{c} \partial_i a_0  \Big)\Big(-\frac{1}{c} \partial_t a^i - c ~\partial^i a^0  \Big) \nonumber \\
\xrightarrow[\text{$c \to \infty$}]{\text{}}  -2 \6^i a^0 \Big( \partial_t a_i - \partial_i a_0  \big) \hspace{1in}
\label{part1}
\end{eqnarray}
 
\begin{eqnarray}
F_{ij}F^{ij} = \Big(\partial_i A_j - \partial_j A_i \Big) \Big(\partial^i A^j - \partial^j A^i \Big) \xrightarrow[\text{limit}]{\text{electric}} \Big(\partial_i a_j - \partial_j a_i \Big) \Big(\partial^i a^j - \partial^j a^i \Big) \nonumber \\
= f_{ij} f^{ij} \hspace{1in}
\end{eqnarray}

\noindent Here $A^{\mu}$ is the relativistic four potential while $a^0$ and $a^i$ are it's galilean counterpart. So in the electric limit the full lagrangian takes the following form
\begin{equation}
    \mathcal{L}_e = \frac{1}{2} \6^i a^0 \Big( \partial_t a_i - \partial_i a_0  \big)  - \frac{1}{4} f_{ij} f^{ij}
    \label{el1}
\end{equation}

 
%We know for relativistic case varying the action with respect to $A_{\mu}$ we get the equation of motion in the following form 

%\begin{equation}
%    \partial_{\mu}\Big( \frac{\partial \mathcal{L}}{\partial (\partial_{\mu} A_{\nu})} \Big) = 0
%\end{equation}

\noindent Now we derive the equations of motion. Varying the Lagrangian \ref{el1} with respect to $a_0,~a_j,~a^0,~a^j$ we get the corresponding equations of motion
\begin{eqnarray}
\6_i \6^i a^0 = 0 
\label{ee1}
\end{eqnarray}
\begin{eqnarray}
\6_t \6^j a^0 + \6_i \6^j a^i - \6_i \6^i a^j = 0
\label{ee2}
\end{eqnarray}
\begin{eqnarray}
\6^i \6_t a_i - \6^i \6_i a_0 = 0
\label{ee3}
\end{eqnarray}
\begin{eqnarray}
\6^i \6_i a_j - \6^i \6_j a_i = 0 
\label{ee4}
\end{eqnarray}
We now derive these equations directly from the equations of motion. The relativistic equations are given by,
\begin{equation}
    \6_{\mu} F^{\mu \nu} = 0 
\end{equation}
which can be written as
\begin{eqnarray}
\6_i F^{i0} =0 \label{e1}\\
\6_0 F^{0j} + \6_i F^{ij} = 0 \label{e2}
\end{eqnarray}
From table \ref{T1} we get in the electric limit
\begin{eqnarray}
\6_i F^{i0} =0  \nonumber \\
\implies \6_i \Big( c \6^i a^0 + \frac{1}{c} \6_t a^i \Big) = 0 \nonumber \\  \xrightarrow[\text{$c \to \infty$}]{\text{}} \6_i \6^i a^0 = 0 
\label{eo1}
\end{eqnarray}
which reproduces eqn \ref{ee1}. 
From \ref{e2} we get
\begin{eqnarray}
\6_0 F^{0j} + \6_i F^{ij} = 0 \nonumber \\
\implies \frac{1}{c} \6_t \Big(-\frac{1}{c} \6_t a^j - c \6^j a^0  \Big) + \6_i \Big(\6^i a^j - \6^j a^i \Big) = 0 \nonumber \\ \xrightarrow[\text{$c \to \infty$}]{\text{}} \6_t \6^j a^0 - \6_i \6^i a^j + \6_i \6^j a^i = 0 
\label{eo2}
\end{eqnarray}
which reproduces eqn \ref{ee2}.
To get the remaining pair of equations we have to interpret  eqns \ref{e1} and \ref{e2} as
\begin{eqnarray}
\6^i F_{i0} =0 \label{e3}\\
\6^0 F_{0j} + \6^i F_{ij} = 0 \label{e4}
\end{eqnarray}
The first of these in the Galilean limit yields
\begin{eqnarray}
 \6^i \Big( \frac{1}{c} \6_i a_0 - \frac{1}{c} \6_t a_i \Big) = 0 \nonumber \\ \xrightarrow[\text{$c \to \infty$}]{\text{}} \6^i \6_i a_0 - \6^i \6_t a_i = 0 
\label{eo3}
\end{eqnarray}
which reproduces eqn \ref{ee3}.
Likewise eqn \ref{e4} yields eqn \ref{ee4} in this limit. 
This shows the consistency of the eqn of motion in Galilean electrodynamics.
\subsection{Magnetic limit}
Here again using the relations given in table \ref{T1} we can write the two terms in \ref{l1} as,
\begin{eqnarray}
2 F_{0i}F^{0i} = 2 \Big(\frac{1}{c} \partial_t A_i - \partial_i A_0 \Big)\Big(-\frac{1}{c} \partial_t A^i - \partial_i A^0  \Big) \xrightarrow[\text{limit}]{\text{magnetic}} 2 \Big(\frac{1}{c} \partial_t a_i + c \partial_i a_0  \Big)\Big(-\frac{1}{c} \partial_t a^i + \frac{1}{c} ~\partial^i a^0  \Big) \nonumber \\ \xrightarrow[\text{$c \to \infty$}]{\text{}} -2 \6_i a_0 \Big(\6_t a^i - \6^i a^0 \Big) \hspace{1in}
\end{eqnarray}
\begin{eqnarray}
F_{ij}F^{ij} = \Big(\partial_i A_j - \partial_j A_i \Big) \Big(\partial^i A^j - \partial^j A^i \Big) \xrightarrow[\text{limit}]{\text{magnetic}} \Big(\partial_i a_j - \partial_j a_i \Big) \Big(\partial^i a^j - \partial^j a^i \Big) \nonumber \\
\xrightarrow[\text{$c \to \infty$}]{\text{}} f_{ij}f^{ij} \hspace{1in} 
\end{eqnarray}
So the Lagrangian will take the following form

\begin{equation}
\mathcal{L}_m = \frac{1}{2}  \6_i a_0 \Big(\6_t a^i - \6^i a^0 \Big) - \frac{1}{4} f_{ij}f^{ij} \label{l2}
\end{equation}
Varying \ref{l2} wrt $a_0,~a_j,~a^0,~a^j$ we get,
\begin{eqnarray}
   \partial_i ~\partial_t~ a^i - \partial^i ~\partial_i ~a^0 = 0 
  \label{em1}
\end{eqnarray}
\begin{eqnarray}
   \6_j\6^i a^j - \6_j \6^j a^i = 0 
  \label{em2}
  \end{eqnarray}
\begin{eqnarray}
  \6^i \6_i a_0 = 0
  \label{em3}
\end{eqnarray}
\begin{eqnarray}
 \6_t \6_j a_0 + \6^i \6_j a_i - \6^i \6_i a_j = 0 \label{em4}
\end{eqnarray}
Here also we can show that the above equations agree with those derived directly from relativistic Maxwell equations by taking the magnetic limit.
From \ref{e1} we have
\begin{eqnarray}
\6_i \Big( -\frac{1}{c}\6^i a^0 + \frac{1}{c} \6_t a^i \Big) = 0 \nonumber \\ \xrightarrow[\text{$c \to \infty$}]{\text{}} \6_i \6_t a^i - \6_i \6^i a^0 = 0
\label{eo5}
\end{eqnarray}
which reproduces eqn \ref{em1}. 
From \ref{e2} we get 
\begin{eqnarray}
\frac{1}{c} \6_t \Big(-\frac{1}{c} a^j + \frac{a^0}{c} \6^j a^0  \Big) + \6_i \Big(\6^i a^j - \6^j a^i \Big) = 0 \nonumber \\ \xrightarrow[\text{$c \to \infty$}]{\text{}} \6_i \6^i a^j - \6_i \6^j a^i = 0 \label{eo6}
\end{eqnarray}
which reproduces \ref{em2}. \\
\noindent To get the remaining pair of equations we have to start from the covariant versions (\ref{e3}, \ref{e4}). 
From \ref{e3} we get 
\begin{eqnarray}
\6^i \Big(-c \6_i a_0 - \frac{1}{c} \6_t a_i  \Big) = 0  \xrightarrow[\text{$c \to \infty$}]{\text{}} \6^i \6_i a_0 = 0 \label{eo7}
\end{eqnarray}
which yields \ref{em3}. 
Likewise eqn \ref{e4} yields \ref{em4} in this limit.

\begin{table}
\caption{Field equations}\label{T2}
\begin{center}
\begin{tabular}{|c|c|c|} \hline 
${\rm Variables}$ & ${\rm Electric ~~limit}$ & ${\rm Magnetic ~~limit}$ \\ \hline
$a^0$  & $\6^i \6_t a_i - \6^i \6_i a_0 = 0$ & $\6^i \6_i a_0 = 0$ \\ \hline
$a^i$ & $\6^j \6_i a_j - \6^j \6_j a_i = 0 $ & $\6_t \6_i a_0 + \6^j \6_i a_j - \6^j \6_j a_i = 0$ \\ \hline
$a_0$ & $ \partial^i \partial_i a^0 = 0 $ & $\partial_i ~\partial_t~ a^i - \partial^i ~\partial_i ~a^0 = 0 $ \\ \hline
$a_i$ & $\partial_t \partial^i a^0 + \partial_j \partial ^i a ^j - \partial_j \partial^j a^i = 0$ & 
$\6_j\6^i a^j - \6_j \6^j a^i = 0 $ \\ \hline
\end{tabular}
\label{T2}
\end{center}
\end{table}
\noindent The field equations for both the limits of Galilean electrodynamics are shown in table \ref{T2}.
\section{Galilean electric and magnetic fields and dual transformations} \label{sec4}
\noindent Here we introduce the galilean limit of electric and magnetic fields and write down the Maxwell equations. For this purpose we will discuss Contravariant and Covariant sector separately.
\subsection{Contravariant sector}
Relativistic electric and magnetic fields are defined as 
\begin{eqnarray}
E^i = \6^0 A^i - \6^i A^0 \\
B^i = \epsilon^{ij}_{~k} \6_j A^k
\end{eqnarray}
First, we consider the electric limit. \\
\noindent \underline {\bf Electric limit:}\\
\begin{equation*}
    A^0 \to c a^0 \, \, \, \, A^i \to a^i 
\end{equation*}
Using this relation we can write the electric field as 
\begin{eqnarray}
E^i = -\frac{1}{c} \6_t a^i - c \6^i a^0 
\end{eqnarray}
And we can define the Galilean electric field as 
\begin{equation}
    e^i = \lim_{c \to \infty} \frac{E^i}{c} = -\6^i a^0
\end{equation}
and magnetic field as 
\begin{equation}
    b^i =  \lim_{c \to \infty} B^i = \epsilon^{ij}_{~k} \6_j a^k
    \end{equation}
\noindent    Now we write the field equations that we derived in the previous section  in terms of the Galilean electric and magnetic fields. From \ref{ee1} we get 
    \begin{eqnarray}
    \6_i \6^i a^0 = 0 \implies \6_i (-e^i) = 0 \implies \vec \nabla . \vec e = 0 
    \end{eqnarray}
\noindent    Similarly eqn \ref{ee2} implies
    \begin{eqnarray}
   \6_t \6^j a^0 + \6_i \6^j a^i - \6_i \6^i a^j = 0 \nonumber\\
    \implies \6_t(-e^j) + \6_i f^{ji} = 0 \nonumber \\
    \implies -\6_t e^j + \6_i (\epsilon^{ji}_{~k} b^k) = 0 \nonumber \\
     \implies  (\vec \nabla \times \vec b)^j = \6_t e^j
    \end{eqnarray}
    
\noindent We can see clearly that
\begin{equation}
 \vec \nabla .\vec b =  \6_i b^i = \6_i \epsilon^{ij}_{~k} \6_j a^k =\epsilon^{ij}_{~k} \6_i\6_j a^k = 0  
\end{equation}

\noindent We will now compute $\vec \nabla \times \vec e$,
\begin{eqnarray}
(\nabla \times e)^i = \epsilon^{ij}_{~k} \6_j e^k 
= \epsilon^{ij}_{~k} \6_j (-\6^k a^0) = 0 
\end{eqnarray}
So in electric limit we get the following set of equations
\begin{eqnarray}
\vec \nabla . \vec e = \6_i e^i = 0 \\
\vec \nabla . \vec b =\6_i b^i = 0\\ 
(\vec \nabla \times \vec e)^i = \epsilon^{ij}_{~k} \6_j e^k = 0 \\
(\vec \nabla \times \vec b)^i = \epsilon^{ij}_{~k} \6_j b^k = \6_t (\vec e)^i
\end{eqnarray}
We will now consider the magnetic limit.\\
\noindent \underline {\bf Magnetic limit:}\\
\begin{equation*}
    A^0 \to -\frac{a^0}{c} \, \, \, \, A^i \to a^i
\end{equation*}
Electric field can be written in this limit as 
\begin{equation}
    E^i = -\frac{1}{c} \6_t a^i + \frac{1}{c} \6_i a^0
\end{equation}
And we can define the galilean electric field as 
\begin{equation}
    e^i = \lim_{c \to \infty} c E^i = -(\6_t a^i - \6^i a^0)
\end{equation}
And magnetic field as 
\begin{equation}
    b^i = \lim_{c \to \infty} B^i = \epsilon^{ij}_{~k} \6_j a^k
    \end{equation}
From \ref{em1} we get 
\begin{eqnarray}
\6_i\Big(\6_t a^i - \6^i a^0  \Big) = 0 \nonumber \\
\implies \6_i e^i = 0 \implies \vec \nabla . \vec e = 0 
\end{eqnarray}
From \ref{em2} we have
\begin{eqnarray}
\6_i f^{ij} = 
 -\epsilon^{ji}_{~k} \6_i b^k = 
\Big( \vec \nabla \times \vec b \Big)^j = 0 
\end{eqnarray}
Finally, we compute $\vec \nabla \times \vec e$, 
\begin{eqnarray}
(\nabla \times e)^i = \epsilon^{ij}_{~k} \6_j e^k \nonumber \\ \implies \epsilon^{ij}_{~k} \6_j \Big(\6^k a^0 - \6_t a^k \Big) \nonumber \\
\implies -\6_t \epsilon^{ij}_{~k} \6_j a^k = - \6_t b^i
\end{eqnarray}
So in magnetic limit we get the following set of equations
\begin{eqnarray}
\vec \nabla . \vec e = \6_i e^i = 0 \\
\vec \nabla . \vec b =\6_i b^i = 0\\ 
(\vec \nabla \times \vec e)^i = \epsilon^{ij}_{~k} \6_j e^k = -\6_t (\vec b)^i \\
(\vec \nabla \times \vec b)^i = \epsilon^{ij}_{~k} \6_j b^k  = 0
\end{eqnarray}
We can clearly see that equations in electric limit are mapped to those of magnetic limit and vice  versa under the following duality transformations,
\begin{equation}
    e^i \to b^i, \, \, ~~~~ b^i \to -e^i 
\end{equation}
\begin{equation}
    e^i \to -b^i, \, \, ~~~~ b^i \to e^i 
\end{equation}
\noindent This is the analogue of the electromagnetic duality in usual Maxwell's source free theory.
\subsection{Covariant sector}
Relativistic electric and the magnetic fields are defined as 
\begin{eqnarray}
E_i = - \Big( \6_0 A_i - \6_i A_0\Big) \\
B_i = \epsilon_i^{~jk} \6_j a_k
\end{eqnarray}
Now we consider the electric limit. \\
\noindent \underline {\bf Electric limit:}\\
\begin{equation*}
    A_0 \to \frac{a^0}{c} \, \, \, \, A^i \to a^i 
\end{equation*}
In this limit the electric field looks like 
\begin{equation}
E_i = -\Big(\frac{1}{c} \6_t a_i + \frac{1}{c} \6_i   a_0\Big)
\end{equation}
And we can define Galilean electric field as 
\begin{equation}
    e_i = \lim_{c \to \infty} c E_i = - (\6_t a_i - \6_i a_0)
\end{equation}
And magnetic field as 
\begin{equation}
    b_i = \lim_{c \to \infty} B_i = \epsilon_{i}^{~jk} \6_j a_k
    \end{equation}
From eqn \ref{ee3} we get
\begin{eqnarray}
 \6^i \Big( \6_t a_i - \6_i a_0 \Big)
= \6^i (-e_i) = \6^i e_i = \vec \nabla . \vec e =0 
\end{eqnarray}
Eqn \ref{ee4} yields
\begin{eqnarray}
\6^i f_{ij}
= -\epsilon_{ji}^{~~k}\6^i b_k =0 \implies \Big(\vec \nabla \times \vec b \Big)_j = 0
\end{eqnarray}
And finally calculation of $\vec \nabla \times \vec e$ yields,
\begin{eqnarray}
\Big(\vec \nabla \times \vec e \Big)_i = \epsilon_{i}^{~jk} \6_j e_k  = -\epsilon_{i}^{jk}\6_j \Big(\6_t a_k - \6_k a_0  \Big)
= -\6_t \epsilon_{i}^{~jk} \6_j a_k  = - \6_t b_i 
\end{eqnarray}
So the Maxwell equations in the electric limit are 
\begin{eqnarray}
\vec \nabla . \vec e = \6^i e_i = 0 \\
\vec \nabla . \vec b = \6^i b_i = 0\\ 
\Big(\vec \nabla \times \vec e\Big)_i = \epsilon_{i}^{~jk}\6_j e_k = - \6_t (\vec b)_i \\
\Big(\vec \nabla \times \vec b \Big)_i= \epsilon_{ij}^{~~k} \6^j b_k = 0
\end{eqnarray}
\noindent \underline {\bf Magnetic limit:}\\
\begin{equation*}
    A_0 \to -c a_0 \, \, \, \, A_i \to a_i 
\end{equation*}
In this limit electric field is scaled as 
\begin{equation}
    E_i = -\Big(\frac{1}{c} \6_t a_i + c \6_i a_0 \Big)
\end{equation}
And we can define the Galilean electric and magnetic fields as 
\begin{equation}
    e_i = \lim_{c \to \infty} \frac{E_i}{c} = -\6_i a_0
\end{equation}

\begin{equation}
    b_i = \lim_{c \to \infty} B_i = \epsilon_{i}^{~jk} \6_j a_k
    \end{equation}
    From eqn \ref{em3} we get,
    \begin{eqnarray}
    \6^i \6_i a_0 = 0 \implies \6^i e_i = 0 \implies \vec \nabla . \vec e = 0 
    \end{eqnarray}
    Computation of $\vec \nabla \times \vec e $ yields,
    \begin{eqnarray}
    (\nabla \times e)_i = \epsilon_{i}^{~jk} \6_j e_k  = - \epsilon_{i}^{~jk} \6_j \6_k a_0 = 0 
    \end{eqnarray}
    From eqn \ref{em4} we get
    \begin{eqnarray}
    \6_t(\6_i a_0) + \6^j f_{ij} = 0 \implies -\6_t e_i + \epsilon_{ij}^{~~k} \6^j b_k = 0
  \implies -\6_t e_i + (\nabla \times b)_i = 0  
    \end{eqnarray}
    So the equations we get in the magnetic limit are 
\begin{eqnarray}
\vec \nabla . \vec e = \6^i e_i = 0 \\
\vec \nabla . \vec b = \6^i b_i = 0\\ 
\Big(\vec \nabla \times \vec e \Big) = \epsilon_{i}^{~jk} \6_j e_k = 0 \\
\Big(\vec \nabla \times \vec b \Big)_i = \epsilon_{ij}^{~~k} \6^j b_k = \6_t e_i
\end{eqnarray}
Here also we see that electric and magnetic fields satisfy certain duality relations as follows 
\begin{equation}
    e_i \to b_i, \, \, ~~~~ b_i \to -e_i 
\end{equation}
\begin{equation}
    e_i \to -b_i, \, \, ~~~~ b_i \to e_i 
\end{equation}
\begin{table}
\caption{Fields in galilean limit}\label{T05}
\begin{center}
\begin{tabular}{|c|c|c|} \hline 
${\rm Limits}$  & $ {\rm Electric ~field}$ & $ {\rm Magnetic ~field}$  \\ \hline
${\rm Electric ~limit}$ & $E^i \to c e^i, ~E_i \to \frac{e_i}{c}$ & $B^i \to b^i, ~B_i \to b_i$ \\ \hline
${\rm Magnetic ~limit }$ &  $ E^i \to \frac{e^i}{c}, ~E_i \to c e_i$  & $B^i \to b^i, ~B_i \to b_i $ \\
\hline
\end{tabular}
\label{T05}
\end{center}
\end{table}
\subsection{Effect of the dualities at the level of Lagrangian}
\noindent In the electric limit the Lagrangian is represented by eqn \ref{el1} which we re-write here again 
\begin{equation}
    \mathcal{L}_e = \frac{1}{2} \6^i a^0 \Big( \partial_t a_i - \partial_i a_0  \big)  - \frac{1}{4} f_{ij} f^{ij}
\end{equation}
Now in the electric limit the contravariant and covariant electric field is represented as
\begin{equation}
    e^i = -\6^i a^0, \,\,\,\,\, e_i = - (\6_t a_i - \6_i a_0)
\end{equation}
Similarly magnetic fields for the two cases are
\begin{equation}
    b^i = \epsilon^{ij}_{~~k} \6_j a^k, \, \, \,\,\, b_i = \epsilon_{i}^{~jk} \6_j a_k 
\end{equation}
Using these definitions we can write the electric limit Lagrangian in the following form
\begin{equation}
    \mathcal{L}_e =  \frac{1}{2} \Big(e^i e_i - b_i b^i \Big) \label{lfe}
\end{equation}
Similarly in the magnetic limit the Lagrangian is represented by eqn \ref{l2} which we re-write as 
\begin{equation}
\mathcal{L}_m = \frac{1}{2}  \6_i a_0 \Big(\6_t a^i - \6^i a^0 \Big) - \frac{1}{4} f_{ij}f^{ij} 
\end{equation}
In the magnetic limit the electric field is defined as
\begin{equation}
    e^i = -(\6_t a^i - \6^i a^0), \,\,\,\,\, e_i = -\6_i a_0
\end{equation}
And the magnetic field is same as before. So now the magnetic limit Lagrangian takes the following form in terms of the fields
\begin{equation}
    \mathcal{L}_m =  \frac{1}{2} \Big(e^i e_i - b_i b^i \Big) \label{lfm}
\end{equation}

\noindent We observe that both Lagrangians (eqn \ref{lfe} and \ref{lfm}) are identical  
\begin{equation}
    \mathcal{L}_e = \mathcal{L}_m = \mathcal{L}
\end{equation}
In other words, expressed in terms of the gauge invariant fields (electric and magnetic), the lagrangians in the two limits are same. This is to be contrasted with the potential formulation where $\mathcal{L}_e$ and $\mathcal{L}_m$ are different.\\
\noindent We observe there is an overall sign change (i.e $\mathcal{L} \to -\mathcal{L}$) under the duality transformations ($e^i \to b^i, ~ b^i \to -e^i, ~e_i \to b_i, ~ b_i \to -e_i$ or $e^i \to -b^i, ~b^i \to e^i, ~e_i \to -b_i, ~b_i \to e_i$) however the Lagrangians remain invariant (i.e $\mathcal{L} \to \mathcal{L}$) under the twisted duality relations (i.e $e^i \to -b^i, ~ b^i \to e^i, ~e_i \to b_i, ~ b_i \to -e_i$ or $e^i \to b^i, ~ b^i \to -e^i, ~e_i \to -b_i, ~ b_i \to e_i$). This has been shown clearly in table \ref{T03}.
\begin{table}
\caption{Effect of duality on the Lagrangian}\label{T03}
\begin{center}
\begin{tabular}{|c|c|} \hline 
${\rm Duality ~relation}$  & $ {\rm Change ~in ~the ~Lagrangian}$  \\ \hline
$e^i \to b^i, ~b^i \to -e^i, ~e_i \to b_i, ~b_i \to -e_i$ & $\mathcal{L} \to -\mathcal{L}$ \\ \hline
$e^i \to -b^i, ~b^i \to e^i, ~e_i \to -b_i, ~b_i \to e_i$ & $\mathcal{L} \to -\mathcal{L} $ \\ \hline
$e^i \to b^i, ~b^i \to -e^i, ~e_i \to -b_i, ~b_i \to e_i$ & $\mathcal{L} \to \mathcal{L} $ \\ \hline
$e^i \to -b^i, ~b^i \to e^i, ~e_i \to b_i, ~b_i \to -e_i$ & $\mathcal{L} \to \mathcal{L} $ \\ 
\hline
\end{tabular}
\label{T03}
\end{center}
\end{table}

\subsection{Dual transformation of electric and magnetic fields under Galilean boost}
\noindent \underline{\bf Contravariant case:} \\
\noindent The Field transforms as
\begin{eqnarray}
F'^{\mu \nu}(x') = \frac{\6 x'^{\mu}}{\6 x^{\lambda}}\frac{\6 x'^{\nu}}{\6 x^{\rho}} F^{\lambda \rho}(x) 
\label{def1}
\end{eqnarray}
Boost transformations can be written as 
\begin{equation}
    x'^0 = \gamma x^0 - \frac{\gamma v_i}{c} x^i
    \label{t6}
\end{equation}
\begin{equation}
    x'^i= x^i - \frac{\gamma v^i}{c}x^0 + (\gamma -1 )\frac{v^iv_j}{v^2}x^j
    \label{t7}
\end{equation}
From \ref{def1} using \ref{t6} and \ref{t7} we get following relations
\begin{eqnarray}
F'^{0i} = \frac{\6 x'^0}{\6 x^0}\frac{\6 x'^i}{\6 x^j} F^{0j} + \frac{\6 x'^0}{\6 x^j}\frac{\6 x'^i}{\6 x^0} F^{j0} + \frac{\6 x'^0}{\6 x^j}\frac{\6 x'^i}{\6 x^k} F^{jk}  \nonumber \\
%= \gamma \delta^i_j F^{0j} + \Big(-\delta^i_j \frac{\gamma v_k}{c}  \Big)\Big( -\frac{\gamma v^i}{c} \Big) F^{j0} + \Big(-\frac{\gamma v_i \delta^i_j}{c} \Big)\Big(\delta^i_k \Big) F^{jk} \nonumber \\
%=\gamma F^{0i} + \frac{\gamma^2 v^i v_j}{c^2} F^{j0} - \frac{\gamma v_j}{c} F^{ji} \nonumber \\
\implies E'^i = \gamma E^i - \frac{\gamma^2 v^i v_j}{c^2} E^j - \frac{\gamma v_j}{c} F^{ji} \label{e}
\end{eqnarray}
\begin{eqnarray}
F'^{ij} 
= \frac{\6 x'^i}{\6 x^0}\frac{\6 x'^j}{\6 x^k} F^{0k} + \frac{\6 x'^i}{\6 x^k}\frac{\6 x'^j}{\6 x^0} F^{k0} + \frac{\6 x'^i}{\6 x^l}\frac{\6 x'^j}{\6 x^m} F^{lm} \nonumber \\
%=-\frac{\gamma v^i}{c} F^{0j} -\frac{\gamma v^j}{c} F^{i0} + F^{ij} \nonumber \\
\implies F'^{ij} = -\frac{\gamma v^i}{c} E^j + \frac{\gamma v^j}{c} E^i + F^{ij} \label{b}
\end{eqnarray}
\noindent \underline{\bf Electric limit}\\
\noindent From eqn \ref{e} using the electric limit scaling we get
\begin{eqnarray}
%E'^i = \gamma E^i - \frac{\gamma^2 v^i v_j}{c^2} E^j - \frac{\gamma v_j}{c} F^{ji} \nonumber \\
%\xrightarrow[\text{limit}]{\text{electric}}  
c e'^i = \gamma c e^i + \frac{\gamma^2 v^i v_j}{c^2} c e^j - \frac{\gamma v_j}{c} f^{ji} \nonumber \\
\xrightarrow[\text{$c \to \infty$}]{\text{}} e'^i = e^i
\end{eqnarray}
\noindent Similarly, eqn \ref{b} yields
\begin{eqnarray}
%F'^{ij} = -\frac{\gamma v^i}{c} E^j + \frac{\gamma v^j}{c} E^i + F^{ij} \nonumber \\
%\xrightarrow[\text{limit}]{\text{electric}} 
f'^{ij} = -\frac{\gamma v^i}{c} c e^j + \frac{\gamma v^j}{c} c e^i + f^{ij} \nonumber \\
\xrightarrow[\text{$c \to \infty$}]{\text{}} f'^{ij} =- v^i e^j + v^j e^i + f^{ij} \nonumber \\
%\implies \epsilon^{ij}_{~~k} b'^k = -\epsilon_{ij}^{~~k} v^i e^j + \epsilon^{ij}_{~~} b^k \nonumber \\
\implies b'^k =  b^k -\Big(\vec v \times \vec e \Big)^k
\end{eqnarray}
\noindent \underline{\bf Magnetic limit}\\
From eqn \ref{e} using the magnetic limit scaling we get
\begin{eqnarray}
%E'^i = \gamma E^i - \frac{\gamma^2 v^i v_j}{c^2} E^j - \frac{\gamma v_j}{c} F^{ji} \nonumber \\
%\xrightarrow[\text{limit}]{\text{magnetic}} 
\frac{e'^i}{c} = \gamma \frac{e^i}{c} - \frac{\gamma^2 v^i v_j}{c^2} \frac{e^j}{c} - \frac{\gamma v_j}{c} f^{ji} \nonumber \\
\xrightarrow[\text{$c \to \infty$}]{\text{}} e'^i = e^i - v_j f^{ji} \nonumber \\
%\implies e'^i = e^i -\epsilon^{ji}_{~~k} v_j b^k \nonumber \\
%\implies e'^i = e^i + \epsilon^{ij}_{~~k} v_j b^k \nonumber \\
\implies e'^i = e^i + \Big(\vec v \times \vec b \Big)^i
\end{eqnarray}
Similarly from eqn \ref{b} we get
\begin{eqnarray}
%F'^{ij} = -\frac{\gamma v^i}{c} E^j + \frac{\gamma v^j}{c} E^i + F^{ij} \nonumber \\
%\xrightarrow[\text{limit}]{\text{magnetic}} 
f'^{ij} = -\frac{\gamma v^i}{c} \frac{e^j}{c} + \frac{\gamma v^j}{c} \frac{e^i}{c} + f^{ij} \nonumber \\
\xrightarrow[\text{$c \to \infty$}]{\text{}} f'^{ij} = f^{ij} \nonumber \\
\implies \vec b' = \vec b 
\end{eqnarray}
\noindent \underline{\bf Covariant case} \\
\noindent The field transforms as
\begin{equation}
    F'_{\mu \nu}(x') = \frac{\6 x^\lambda}{\6 x'^{\mu}}\frac{\6 x^\rho}{\6 x'^\nu} F_{\lambda \rho} \label{def2}(x)
\end{equation}
We also write the boost transformations as
\begin{equation}
    x^0 = \gamma x'^0 + \frac{\gamma v_i}{c} x'^i
    \label{t8}
\end{equation}
\begin{equation}
    x^i= x'^i + \frac{\gamma v^i}{c}x'^0 + (\gamma -1 )\frac{v^iv_j}{v^2}x'^j
    \label{t9}
\end{equation}
From equation \ref{def2} using \ref{t8} and \ref{t9} we get
\begin{eqnarray}
 F'_{0i} = %\frac{\6 x^\lambda}{\6 x'^0}\frac{\6 x^\rho}{\6 x'^i} F_{\lambda \rho} \nonumber \\
 =  \frac{\6 x^0}{\6 x'^0}\frac{\6 x^k}{\6 x'^i} F_{0k} + \frac{\6 x^k}{\6 x'^0}\frac{\6 x^0}{\6 x'^i} F_{k0} + \frac{\6 x^l}{\6 x'^0}\frac{\6 x^k}{\6 x'^i} F_{lk} \nonumber \\
 -E'_i = -\gamma E_i + \frac{\gamma^2 v_i v^k}{c^2} E_k + \frac{\gamma v^l}{c} F_{li} \label{ec1}
\end{eqnarray}
\begin{eqnarray}
 F'_{ij} = \frac{\6 x^\lambda}{\6 x'^i} \frac{\6 x^\rho}{\6 x'^j}  F_{\lambda \rho} \nonumber \\
 = -\frac{\gamma v_i}{c} E_j + \frac{\gamma v_j}{c} E_i + F_{ij} \label{bc1}
\end{eqnarray}
\noindent \underline{\bf Electric limit}\\
From eqn \ref{ec1} using the electric limit scaling for the covariant case we get
\begin{eqnarray}
 %E'_i = \gamma E_i - \frac{\gamma^2 v_i v^k}{c^2} E_k + \frac{\gamma v^l}{c} F_{li} \nonumber \\ \xrightarrow[\text{limit}]{\text{electric}}
 -\frac{e'_i}{c} = -\frac{\gamma e_i}{c} + \frac{\gamma^2 v_i v^k}{c^2} \frac{e_k}{c} +\frac{\gamma v^l}{c} f_{li} \nonumber \\ \xrightarrow[\text{$c \to \infty$}]{\text{}}
 %-e'_i = -e_i - \epsilon_{ilk} v^l b_k \nonumber \\
  e'_i = e_i + \Big(\vec v \times \vec b  \Big)_i
\end{eqnarray}
Similarly from eqn \ref{bc1} we get
\begin{eqnarray}
 %F'_{ij} = \frac{\gamma v_i}{c} E_j - \frac{\gamma v_j}{c} E_i +F_{ij} \nonumber \\ \xrightarrow[\text{limit}]{\text{electric}} 
 f'_{ij} = f_{ij} %\nonumber \\
 \implies \vec b' = \vec b
\end{eqnarray}
\noindent \underline{\bf Magnetic limit}\\
From eqn \ref{e1} using magnetic limit scaling for covariant case we get
\begin{eqnarray}
%E'_i = \gamma E_i - \frac{\gamma^2 v_i v^k}{c^2} E_k + \frac{\gamma v^l}{c} F_{li} \nonumber \\\xrightarrow[\text{limit}]{\text{magnetic}}
-e'_i = -\gamma e_i + \frac{\gamma^2 v_i v^k}{c^2} e_k + \frac{\gamma v^l}{c} f_{li} \nonumber \\ \xrightarrow[\text{$c \to \infty$}]{\text{}} e'_i = e_i
\end{eqnarray}
Similarly from eqn \ref{bc1} we get
\begin{eqnarray}
%F'_{ij} = \frac{\gamma v_i}{c} E_j - \frac{\gamma v_j}{c} E_i +F_{ij} \nonumber \\ \xrightarrow[\text{limit}]{\text{magnetic}} 
f'_{ij} = - v_i e_j + v_j e_i + f_{ij} \nonumber \\ \implies b'_k = b_k - \Big(\vec v \times \vec e \Big)_k
\end{eqnarray}
The transformed electric and magnetic field under galielan boost has been shown in table \ref{T04}.
\begin{table}
\caption{Transformation of fields under Galilean boost}\label{T04}
\begin{center}
\begin{tabular}{|c|c|c|} \hline 
${\rm Limits}$  & $ {\rm Contravariant ~case}$ & $ {\rm Covariant ~case}$  \\ \hline
${\rm Electric ~limit}$ & $e'^i = e^i, ~~b'^k =  b^k -\Big(\vec v \times \vec e \Big)^k$ & $e'_i = e_i + \Big(\vec v \times \vec b  \Big)_i, ~~\vec b' = \vec b$\\ \hline
${\rm Magnetic ~limit }$ &  $e'^i = e^i + \Big(\vec v \times \vec b \Big)^i, ~~\vec b' = \vec b $  & $e'_i = e_i, ~~b'_k = b_k - \Big(\vec v \times \vec e \Big)_k$ \\
\hline
\end{tabular}
\label{T04}
\end{center}
\end{table}
We can clearly see from table \ref{T04} that under duality transformation ($e^i \to b^i, ~b^i \to -e^i$ and $e_i \to b_i, ~b_i \to -e_i$) electric limit reproduces magnetic limit and vice-versa for both covariant and contravariant cases.
\section{Gauge symmetry} \label{sec5}
We know in the relativistic case the Maxwell lagrangian,
\begin{equation}
  \mathcal{L} = -\frac{1}{4} F_{\mu \nu} F^{\mu \nu}  
\end{equation}
is invariant under the following gauge transformation,
\begin{equation}
    \delta A_{\mu} = \6_{\mu} \alpha, \,\,\,  \delta A^{\mu} = \6^{\mu} \alpha \label{cpot}
\end{equation}
%So,
%\begin{eqnarray}
%\delta F_{\mu \nu} = 0 
%\end{eqnarray}

%So the Maxwell Lagrangian 

\noindent We consider the Galilean version of this gauge invarianvce. \\
 \subsection{Galilean version} \label{gv}
Here we can consider a relatively more general gauge condition,
\begin{equation}
    \delta A_{\mu} = \6_{\mu} \alpha, \,\,\,  \delta A^{\mu} = \6^{\mu} \beta \label{cpot1}
\end{equation}

\noindent In the relativistic theory the covaraint and contravariant vectors are related by a metric implying $\alpha = \beta$. This is mot true in galilean limit. Hence we take $\alpha \neq \beta$ when deriving the galilean version of the gauge transformations. First we consider the electric limit. \\
\noindent \underline {\bf Electric limit} \\
As we discussed before in table \ref{T1}, the potential map as
\begin{eqnarray}
A^0 \to c a^0, \,\, \, \,  A^i \to a^i \nonumber\\ A_0 \to \frac{a_0}{c}, \,\,\,\, A_i \to a_i \label{pot1}
\end{eqnarray}
From eqns \ref{pot1} and \ref{cpot1} we can deduce the following relations
\begin{eqnarray}
\delta A_0 = \6_0 \alpha \implies \frac{1}{c} \delta a_0 = \frac{1}{c} \6_t \alpha %\nonumber \\
\implies \delta a_0 = \6_t \alpha \label{eg1}
\end{eqnarray}
\begin{eqnarray}
\delta A_i = \6_i \alpha \implies \delta a_i = \6_i \alpha \label{eg2}
\end{eqnarray}
\begin{eqnarray}
\delta A^0 = \6^0 \beta \implies c \delta a^0 = -\frac{1}{c} \6_t \beta   
\xrightarrow[\text{$c \to \infty$}]{\text{}} \delta a^0 = 0 \label{eg3}
\end{eqnarray}
\begin{eqnarray}
\delta A^i = \6^i \beta \implies \delta a^i = \6^i \beta \label{eg4}
\end{eqnarray}

\noindent Taking the variation of \ref{el1} in the electric limit,
\begin{eqnarray}
\delta \mathcal{L}_e = \frac{1}{2} \6^i \delta a^0 \Big(\6_t a_i - \6_i a_0  \Big) + \frac{1}{2} \6^i a^0 \Big(\6_t \delta a_i - \6_i \delta a_0  \Big)  = 0 \label{clag}
\end{eqnarray}
on exploiting eqns \ref{eg1}, \ref{eg2}, \ref{eg3}, \ref{eg4}. This shows the invariance of $\mathcal{L}_e$.\\
%In eqn \ref{clag} the first term is zero because $\delta a^0 = 0$ and for the second term it is quite evident. \\
\noindent \underline {\bf Magnetic limit} \\
Here the potential transforms as
\begin{eqnarray}
 A^0 \to -\frac{a^0}{c},\,\, \,\,A^i \to a^i \, \,\,\, \nonumber \\  A_0 \to -c a_0, \,\, A_i \to a_i \label{pot2}
\end{eqnarray}
From eqns \ref{pot2} and \ref{cpot1} we can deduce the following relations,
\begin{eqnarray}
 \delta A^0 = \6^0 \beta \implies -\frac{1}{c} \delta a^0 = -\frac{1}{c} \6_t \beta \implies
 \delta a^0 = \6_t \beta \label{mg1}
\end{eqnarray}
\begin{eqnarray}
 \delta A_0 = \6_0 \alpha \implies -c \delta a_0 = \frac{1}{c}\6_t \alpha 
  \xrightarrow[\text{$c \to \infty$}]{\text{}} \delta a_0 = 0 \label{mg2}
\end{eqnarray}
\begin{eqnarray}
 \delta A^i = \6^i \beta \implies \delta a^i = \6^i \beta \label{mg3}
\end{eqnarray}
\begin{eqnarray}
 \delta A_i = \6_i \alpha \implies \delta a_i = \6_i \alpha \label{mg4}
\end{eqnarray}
Taking the variation of the lagrangian \ref{l2} in the magnetic limit
\begin{eqnarray}
 \delta \mathcal{L}_m = \frac{1}{2} \6_i \delta a_0 \Big(\6_t a^i - \6^i a^0 \Big) + \frac{1}{2} \6_i a_0 \Big(\6_t \delta a^i - \6^i \delta a^0 \Big) 
 = 0 
\end{eqnarray}
on exploiting eqns \ref{mg1}, \ref{mg2}, \ref{mg3}, \ref{mg4}. This shows the invariance of $\mathcal{L}_m$.
\begin{table}
\caption{Variations of the Galilean potentials}\label{T3}
\begin{center}
\begin{tabular}{|c|c|c|} \hline 
${\rm Variable}$  & $ {\rm Electric ~~limit}$ & $ {\rm Magnetic ~~limit}$  \\ \hline
$a^0$ & $\delta a^0 = 0$ & $\delta a^0 = \6_t \beta$\\ \hline
$a^i$ & $\delta a^i = \6^i \beta$  & $\delta a^i = \6^i \beta$ \\ \hline
$a_0$ & $\delta a_0 = \6_t \alpha$ & $\delta a_0 = 0 $\\ \hline
$a_i$ & $\delta a_i = \6_i \alpha$  & $\delta a_i = \6_i \alpha$\\ 
\hline
\end{tabular}
\label{T3}
\end{center}
\end{table}
\subsection{Noether current conservation}
\noindent We know in relativistic classical field theory the Noether current is defined as
\begin{eqnarray}
 J^{\mu}= \frac{\6 \mathcal{L}}{\6 (\6_\mu A_\nu)} \delta A_\nu \label{current1}
\end{eqnarray}
which is conserved on-shell i.e $\6_\mu J^\mu =  0$. Specifically for the 
Maxwell Lagrangian,
\begin{equation*}
  \mathcal{L} = -\frac{1}{4} F_{\mu \nu} F^{\mu \nu}  
\end{equation*}
the current in eqn \ref{current1} has the form, 
\begin{equation}
    J^\mu = -F^{\mu \nu} ~\6_\nu \alpha 
\end{equation}
and is on-shell conserved i.e
\begin{eqnarray}
   \6_\mu J^\mu = -\Big(\6_\mu F^{\mu \nu} \Big)\6_\nu \alpha - F^{\mu \nu} \6_\mu \6_\nu \alpha 
   =0
\end{eqnarray}
The first term is zero because $\6_\mu F^{\mu \nu}= 0 $ and second term is zero because of anti-symmetry of $F^{\mu \nu}$.\\
\noindent We now consider here a suitable Galilean version of this conservation. For this we will directly start from the relativistic definition and substitute the Galilean results in proper limit (electric or magnetic). First we consider the electric limit. \footnote{The details of the Noether current calculations are provided in appendix \ref{ap1}.}\\
\noindent \underline {\bf Electric limit} \\
In this limit the contravariant components of the current are
\begin{eqnarray}
 j^0 = \frac{1}{2} \6^i a^0 \6_i \alpha 
\end{eqnarray}
\begin{eqnarray}
j^i = -\frac{1}{2} \6^i a^0 \6_t \alpha - \frac{1}{2} \Big( \6^i a^j - \6^j a^i\Big) \6_j \alpha
\end{eqnarray}
So we can show the galilean currents are conserved, i.e   
\begin{eqnarray}
 \6_t j^0 + \6_i j^i = 0 \hspace{2in} %\label{conservn1}
\end{eqnarray}
Likewise for the covariant components 
\begin{equation}
    J_\mu = \frac{\6 \mathcal{L}_e}{\6 (\6^\mu A^\nu)}\delta A^\nu
\end{equation}
From this we get following components of the current
\begin{eqnarray}
 j_0 = 0 
\end{eqnarray}
 \begin{eqnarray}
 j_i =  f_{ij} ~\6^j \alpha 
\end{eqnarray} 
Here also we can show that they are conserved
\begin{eqnarray}
\6_t j_0 + \6^i j_i = 0, %\label{conservn2}
\end{eqnarray}

\noindent \underline {\bf Magnetic limit} \\
%The Lagrangian is
%\begin{equation}
%    \mathcal{L}_m =  \frac{1}{2} \6_i a_0 \Big(\6_t a^i - \6^i a^0 \Big) - \frac{1}{4} f_{ij}f^{ij}
%\end{equation}
In this limit the contravariant current components are
\begin{eqnarray}
 j^0 = 0 
\end{eqnarray}
\begin{eqnarray}
 j^i = -\frac{1}{2} f^{ij} \6_j \alpha 
\end{eqnarray}
We can see that they can satisfy following conservation equation. 
\begin{eqnarray}
 \6_t j^0 + \6_i j^i  = 0 
\end{eqnarray}
Similarly for the covariant components,
\begin{eqnarray}
j_0 = \frac{1}{2} \6_i a_0 \6^i \alpha 
\end{eqnarray}
\begin{eqnarray}
 j_i = -\frac{1}{2} \6_i a_0 \6_t \alpha - \frac{1}{2} f_{ij} \6^j \alpha 
\end{eqnarray}
which satisfies
\begin{eqnarray}
 \6_t j_0 + \6^i j_i = 0
\end{eqnarray}
 
\section{Shift symmetry} \label{sec6}
We know that Goldstone's theorem is a crucial input of the study of low-energy effective lagrangians implying that whenever a global symmetry is spontaneously broken, a gapless mode will appear. In relativistic theories this leads to a massless Goldstone particle described by a shift symmetry of
the field
\begin{equation}
    \phi(x) \to \phi(x) + c \label{fi}
\end{equation}
where $c$ is constant and is characterised by the scalar field action
\begin{equation}
    S = \frac{1}{2} \int d^d x \6_\mu \phi \6^\mu \phi
\end{equation}
The above action is invariant under \ref{fi}. Since \ref{fi} is a global transformation, the conserved currents can be found by exploiting Noether's first theorem 
\begin{eqnarray}
J^\mu = \frac{\6 \mathcal{L}}{\6(\6_\mu \phi)} \delta \phi = c ~\6^\mu \phi
\end{eqnarray}
And corresponding conservations are demonstrated as 
\begin{equation}
    \6_\mu J^\mu = c \6_\mu \6^\mu \phi = 0
\end{equation}
Consider a constant shift in the four potential, 
\begin{equation}
    A'_{\mu} = A_{\mu} + C_{\mu}, \, \, \, \, A'^{\mu} = A^{\mu} + D^{\mu}
\end{equation}
that leaves Maxwell lagrangian invariant. We take $C$ and $D$ to be different for reasons stated in section \ref{gv}.  \\
\noindent \underline {\bf Electric limit} \\
We can define following things
\begin{eqnarray}
 \delta A_0 = C_0 \implies \frac{1}{c} \delta a_0 = C_0 \implies \delta a_0 = c C_0 \\
\delta A_i = C_i \implies \delta a_i = C_i 
\end{eqnarray}
Similarly.
\begin{eqnarray}
 \delta A^0 = D^0 \implies c \delta a^0 = D^0 \implies \delta a^0 =  0\\
\delta A^i = D^i \implies \delta a^i = D^i 
\end{eqnarray}
From \ref{el1} Noether currents are found to be
\begin{eqnarray}
j^t = \frac{\6 \mathcal{L}_e}{\6 (\6_t a_0)} \delta a_0 + \frac{\6 \mathcal{L}_e}{\6 (\6_t a_i)} \delta a_i 
= \frac{1}{2} (\6^i a^0) C_i
\end{eqnarray}
\begin{eqnarray}
j^i = \frac{\6 \mathcal{L}_e}{\6 (\6_i a_0)} \delta a_0 + \frac{\6 \mathcal{L}_e}{\6 (\6_i a_j)} \delta a_j 
= -\frac{1}{2} \6^i a^0 C_0 - \frac{1}{2} f^{ij} C_j
\end{eqnarray}
And current conservation can be explicitly demonstrated as 
\begin{eqnarray}
\6_t j^t + \6_i j^i = \frac{1}{2} (\6_t \6^i a^0) C_i - \frac{1}{2} (\6_i \6^i a^0) C_0 - \frac{1}{2} (\6_i f^{ij}) C_j 
= 0 
\end{eqnarray}
The covariant components of the currents are
\begin{eqnarray}
j_t = \frac{\6 \mathcal{L}_e}{\6 (\6_t a^0)} \delta a^0 + \frac{\6 \mathcal{L}_e}{\6 (\6_t a^i)} \delta a_i 
=0 
\end{eqnarray}
\begin{eqnarray}
j_i = \frac{\6 \mathcal{L}_e}{\6 (\6^i a^0)} \delta a_0 + \frac{\6 \mathcal{L}_e}{\6 (\6^i a^j)} \delta a_j 
=  - \frac{1}{2} f_{ij} D^j
\end{eqnarray}
The current conservation gives us 
\begin{eqnarray}
\6_t j_t + \6^i j_i = -\frac{1}{2} (\6^i f_{ij}) D^j = 0 
\end{eqnarray}
\noindent \underline {\bf Magnetic limit} \\
We can define following things
\begin{eqnarray}
\delta A_0 = C_0 \implies -c \delta a_0 = C_0 \implies \delta a_0 = 0 \\
\delta A_i = C_i \implies \delta a_i = C_i 
\end{eqnarray}
Similarly.
\begin{eqnarray}
 \delta A^0 = D^0 \implies -\frac{1}{c} \delta a^0 = D^0 \implies \delta a^0 =  -c D^0\\
\delta A^i = D^i \implies \delta a^i = D^i 
\end{eqnarray}
From \ref{l2} the Noether currents are found to be 
\begin{eqnarray}
j^t = \frac{\6 \mathcal{L}_m}{\6 (\6_i a_0)} \delta a_0 + \frac{\6 \mathcal{L}_m}{\6 (\6_i a_j)} \delta a_j 
=0
\end{eqnarray}
\begin{eqnarray}
j^i = \frac{\6 \mathcal{L}_m}{\6 (\6_i a_0)} \delta a_0 + \frac{\6 \mathcal{L}_m}{\6 (\6_i a_j)} \delta a_j 
= -\frac{1}{2} f^{ij} C_j
\end{eqnarray}
So the current conseravtaions are demonstrated as
\begin{eqnarray}
\6_t j^t + \6_i j^i = -\frac{1}{2} \6_i f^{ij} C_j = 0
\end{eqnarray}
Similarly the covariant current components are 
\begin{eqnarray}
j_t = \frac{\6 \mathcal{L}_m}{\6 (\6_t a^0)} \delta a^0 + \frac{\6 \mathcal{L}_m}{\6 (\6_t a^i)} \delta a^i 
= \frac{1}{2} \6_i a_0 D^i
\end{eqnarray}
\begin{eqnarray}
j_i = \frac{\6 \mathcal{L}_m}{\6 (\6^i a^0)} \delta a^0 + \frac{\6 \mathcal{L}_m}{\6 (\6^i a^j)} \delta a^j 
= -\frac{1}{2} \6_i a_0 D^0 - \frac{1}{2} f_{ij} D^j
\end{eqnarray}
The current conservations give
\begin{eqnarray}
\6_t j_t + \6^i j_i 
= \frac{1}{2} (\6_t \6_i a_0) D^i - \frac{1}{2} (\6^i \6_i a_0) D^0 - \frac{1}{2}\6^i f_{ij} D^j = 0 
\end{eqnarray}

\section{Inclusion of sources} \label{sec7}
The relativistic Maxwell Lagrangian with source is as follows
\begin{equation}
    \mathcal{L} = -\frac{1}{4} F_{\alpha \beta} F^{\alpha \beta} - A_{\alpha} J^{\alpha}
\end{equation}
We can write the Lagrangian in the following form for convenience
\begin{equation}
    \mathcal{L} = -\frac{1}{4} \Big(2 F_{0i}F^{0i} + F_{ij} F^{ij} \Big) - \frac{1}{2} A_{\a} J^{\a} - \frac{1}{2} A^{\a} J_{\a}
\end{equation}
We know that the Maxwell theory respects the following gauge transformations
\begin{equation}
    A_\mu \to A_\mu + \6_\mu \Lambda
\end{equation}
The gauge invariance of the Lagrangian demands following condition
\begin{equation}
    \Big(A_\mu + \6_\mu \Lambda \Big) J^\mu = A_\mu J^\mu + \Lambda \6_\mu J^\mu \implies \6_\mu J^\mu = 0 
\end{equation}
\noindent \underline {\bf Electric limit:}\\
\noindent In the electric limit the scaling will be as follows
\begin{eqnarray}
A^0 \to c a^0, \,\, A^i \to a^i, \,\, A_0 \to \frac{a_0}{c}, \,\, A_i \to a_i \nonumber \\ J^0 \to c j^0, \,\, J^i \to j^i, \,\, J_0 \to \frac{j_0}{c}, \,\, J_i \to j_i
\end{eqnarray}
In this limit the Lagrangian looks like 
\begin{equation}
    \mathcal{L}_e = \frac{1}{2} \6^i a^0 \Big(\6_t a_i - \6_i a_0  \Big) -\frac{1}{4} f_{ij} f^{ij} - \frac{1}{2} a_0 j^0 -\frac{1}{2} a_i j^i - \frac{1}{2} a^0 j_0 -\frac{1}{2} a^i j_i
\label{ls1}    
\end{equation}
Varying the lagrangian with respect to $a_0, ~a_j, ~a^0, ~a^j$ will give following set of equations
\begin{eqnarray}
 \6^i \6_i a^0 = j^0
 \label{es1}
\end{eqnarray}
\begin{eqnarray}
 \6_t \6^j a^0 + \6_i \6^j a^i - \6_i \6^i a^j = - j^j \label{es2}
\end{eqnarray}
\begin{eqnarray}
\6^i \6_t a_i - \6^i \6_i a_0 = -j_0 \label{es3}
\end{eqnarray}
\begin{eqnarray}
\6^i \6_i a_j - \6^i \6_j a_i = j_j \label{es4}
\end{eqnarray}
We now derive these equations directly from the equations of motion. The relativistic equations are given by,
\begin{equation}
    \6_{\mu} F^{\mu \nu} = J^{\nu}
\end{equation}
Which can be written as 
\begin{eqnarray}
\6_i F^{i0} =J^0 \label{ems1} \\
\6_0 F^{0j} + \6_i F^{ij} = J^j \label{ems2}
\end{eqnarray}
From the first equation \ref{ems1} we get 
\begin{eqnarray}
\6_i \Big(c \6^i a^0 + \frac{1}{c} \6_t a^i \big) = c j^0 \nonumber \\\xrightarrow[\text{$c \to \infty$}]{\text{}}
\6_i \6^i a^0 = j^0
\end{eqnarray}
which reproduces eqn \ref{es1}
From the second equation \ref{ems2} we have
\begin{eqnarray}
\frac{1}{c} \6_t \Big(-\frac{1}{c} \6_t a^j - c \6^j a^0 \Big) + \6_i \Big( \6^i a^j - \6^j a^i  \Big) = j^j \nonumber \\ \xrightarrow[\text{$c \to \infty$}]{\text{}} \6_t \6^j a^0 + \6_i \6^j a^i - \6_i \6^i a^j = -j^j
\end{eqnarray}
which reproduces eqn \ref{es2}. 
To get the remaining pair of equations we have to interpret
eqns \ref{ems1} and \ref{ems2} as 
\begin{eqnarray}
\6^i F_{i0} = J_0 \label{ems3} \\
\6^0 F_{0j} + \6^i F_{ij} = J_j \label{ems4}
\end{eqnarray}
 The first of them in the galilean limit yields,
 \begin{eqnarray}
 \frac{1}{c} \6^i \Big(\6_i a_0 - \6_t a_i \Big) = \frac{j_0}{c} \\ \xrightarrow[\text{$c \to \infty$}]{\text{}} \6^i \6_t a_i - \6^i \6_i a_0 = -j_0
 \end{eqnarray}
 which reproduces eqn \ref{es3}. Similarly \ref{ems4} yields eqn \ref{es4}. This shows the consistency of the eqn of motion in Galilean electrodynamics with source.\\
 Taking the variation of the source part of the lagrangian we get 
 \begin{eqnarray}
 \delta \mathcal{L}_e = -\frac{1}{2} \6_t \alpha j^0 - \frac{1}{2} \6_i \alpha j^i - \frac{1}{2} \6^i \beta j_i = \frac{1}{2}\alpha \Big(\6_t j^0 + \6_i j^i \Big) + \frac{1}{2} \beta \6^i j_i = 0
 \end{eqnarray}
Since $\alpha, \beta \neq 0$ we have two conditions
\begin{eqnarray}
\6_t j^0 + \6_i j^i = 0, \, \, \, \, \, \6^i j_i = 0
\end{eqnarray}
The sources as given in eqns (\ref{es1})-(\ref{es4}) satisfy the above conditions. We observe that sources are conserved off-shell.\\
\noindent \underline {\bf Magnetic limit:}\\
Here the scalings are as follows
\begin{eqnarray}
A^0 \to -\frac{a^0}{c},\,\, A^i \to a^i, \, \, A_0 \to -c a_0, \,\, A_i \to a_i \nonumber \\ J^0 \to -\frac{j^0}{c}, \,\, J^i \to j^i, \,\, J_0 \to -c j_0, \,\, J_i \to j_i 
\end{eqnarray}
The Lagrangian in this limit is as follows 
\begin{equation}
    \mathcal{L}_m = \frac{1}{2} \6_i a_0 \Big(\6_t a^i - \6^i a^0 \Big) - \frac{1}{4} f_{ij}f^{ij} - \frac{1}{2} a_0 j^0 -\frac{1}{2} a_i j^i - \frac{1}{2} a^0 j_0 -\frac{1}{2} a^i j_i
\end{equation}
Varying the lagrangian wrt $a_0, ~a_j, ~a^0, ~a^j$ we get the following set of equations
\begin{eqnarray}
 \6_i \6_t a^i - \6_i \6^i a^0 = -j^0 \label{ms1}
\end{eqnarray}
\begin{eqnarray}
 \6_i \6^i a^j - \6_i \6^j a^i = j^j \label{ms2}
\end{eqnarray}
\begin{eqnarray}
\6^i \6_i a_0 = j_0 \label{ms3}
\end{eqnarray}
\begin{eqnarray}
\6_t \6_j a_0 - \6^i \6_i a_j + \6^i \6_j a_i = - j_j \label{ms4}
\end{eqnarray}
We now derive these equations directly from the equations of motion. Eqn \ref{ems1} will give
\begin{eqnarray}
 \6_i\Big(-\frac{1}{c} \6^i a^0 + \frac{1}{c} \6_t a^i  \Big) = -\frac{j^0}{c} \nonumber \\ \implies \6_i\6_t a^i - \6_i \6^i a^0 = -j^0
\end{eqnarray}
Eqn \ref{ems2} yields
\begin{eqnarray}
\frac{1}{c} \Big(-\frac{1}{c} \6_t a^j + \frac{1}{c} \6^j a^0 \Big) + \6_i \6^i a^j - \6_i \6^j a^i = j^j \nonumber \\ \xrightarrow[\text{$c \to \infty$}]{\text{}} \6_i \6^i a^j - \6_i \6^j a^i = j^j
\end{eqnarray}
Likewise eqns \ref{ems3} and \ref{ems4} yields eqns \ref{ms3} and \ref{ms4} respectively. The field equations for both electric and magnetic limit have been shown in table \ref{T7}.
\begin{table}
\caption{Field equations}\label{T7}
\begin{center}
\begin{tabular}{|c|c|c|} \hline 
${\rm Variables}$ & ${\rm Electric ~~limit}$ & ${\rm Magnetic ~~limit}$ \\ \hline
$a^0$  & $\6^i \6_t a_i - \6^i \6_i a_0 = -j^0$ & $\6^i \6_i a_0 = j^0$ \\ \hline
$a^j$ & $\6^i \6_i a_j - \6^i \6_j a_i = j_j $ & $\6_t \6_j a_0 + \6^i \6_j a_i - \6^i \6_i a_j = -j_j$ \\ \hline
$a_0$ & $ \6^i \6_i a^0 = j^0 $ & $\6_i \6_t a^i - \6^i ~\6_i ~a^0 = -j^0 $ \\ \hline
$a_j$ & $\6_t \6^j a^0 + \6_i \6^j a ^i - \6_i \6^i a^j = -j^j$ & 
$\6_i\6^i a^j - \6_i \6^j a^i = j^j $ \\ \hline
\end{tabular}
\label{T7}
\end{center}
\end{table}
\noindent Taking the variation of the source part of the lagrangian we get
\begin{equation}
    \delta \mathcal{L}_m = -\frac{1}{2} \6_i \alpha j^i - \frac{1}{2} \6_t \beta j_0 - \frac{1}{2} \6^i \beta j_i = \frac{1}{2}\alpha \Big(\6_i j^i \Big) + \frac{1}{2} \beta \Big(\6_t j_0 + 6^i j_i \Big) = 0
\end{equation}
Since $\alpha, \beta \neq 0$ we have two conditions
\begin{eqnarray}
\6_i j^i = 0, \, \, \, \, \6_t j_0 + \6^i j_i = 0 
\end{eqnarray}
Here also sources given by eqns (\ref{ms1}) to (\ref{ms4}) satisfy the above off-shell conservation equations. 
\section{Conclusions}
\noindent In this paper we have given a detailed map between the relativistic and galilean vectors under the electric and magnetic limits. The novelty of our approach is that we have given this map for both contravariant and covariant vectors. In previous works only the former sector was considered. We believe our analysis here gives the topic named "Galilean electrodynamics" a complete look and illuminates the underlying deeper and subtle symmetries which remains unobserved so far. We have derived  the non-relativistic Lagrangians for both electric and magnetic limit using the map given in section \ref{sec2}. We have shown that they give consistent equations of motion. 
%We introduce fields in non-relativistic limit and discuss the duality symmetries. 
The potential formulation was extended to the field (electric \& magnetic) formulation. In this set up the duality symmetry was discussed. One can clearly see that in the non-relativistic limit the duality relations are quite non-trivial. On the lagrangian level, duality plays quite a subtle role. We briefly discuss the gauge symmetries in the relativistic case and how they give rise to consistent non-relativistic gauge symmetries. One interesting fact is for the relativistic case the gauge parameters for covarinat and contravariant case are same but that no longer holds true for non-relativistic gauge transformations as the covariant and contravariant quantities are distinct entities (since there is no non-degenerate metric to raise and lower the indices) hence the gauge parameters can be different quantities.  Here we compute the non-relativistic Noether currents and show that they are on-shell conserved. Subsequently we introduced another class of symmetries called shift symmetries which play important role to describe Goldstone's particles. We have translated these symmetries in the non-relativistic limit and show that corresponding non-relativistic Noether currents are conserved on-shell. Finally we include sources in our calculations and derive the equation of motion which again give consistent results. We show that these sources satisfy off-shell conservation. \\
This is quite new research area and many aspects and directions are yet to look at. There is no consistent Hamiltonian formalism for galilean electrodynamics. Another interesting aspect would be to study the ModMax theories which are the maximally symmetric non-linear extension of Maxwell’s theory in four dimensions, in our formalism. Another interesting aspect would be to study the Carollean limit ($c \to 0$).  We hope we can address these issues in near future. 
\section{Acknowledgements}
The authors (RB and SB) acknowledge the support from a DAE Raja Ramanna Fellowship (grant no: $1003/(6)/2021/RRF/R\&D-II/4031$, dated: $20/03/2021$). They also acknowledge Sumit Dey for useful discussions. 
\begin{appendices}
\section{Noether current calculation} \label{ap1}
\noindent \underline {\bf Electric limit} \\
The contravariant components of the current for the relativistic case are
\begin{eqnarray}
J^0 = \frac{\6 \mathcal{L}}{\6 (\6_0 A_\nu)}\delta A_\nu \label{a1}
\end{eqnarray}
\begin{eqnarray}
J^i = \frac{\6 \mathcal{L}}{\6 (\6_i A_0)}\delta A_0 + \frac{\6 \mathcal{L}}{\6 (\6_i A_j)}\delta A_j \label{a2}
\end{eqnarray}
Now using the maps for electric limit given in table \ref{T1} we get
\begin{eqnarray}
 c j^0 = c \frac{\6 \mathcal{L}_e}{\6 (\6_t a_0)}\delta a_0  + c \frac{\6 \mathcal{L}_e}{\6 (\6_t a_i)}\delta a_i 
\implies j^0 = \frac{1}{2} \6^i a^0 \6_i \alpha 
\end{eqnarray}
Similarly, 
\begin{eqnarray}
 j^i = \frac{\6 \mathcal{L}_e}{\6 (\6_i a_0)}\delta a_0 + \frac{\6 \mathcal{L}_e}{\6 (\6_i a_j)}\delta a_j 
= -\frac{1}{2} \6^i a^0 \6_t \alpha - \frac{1}{2} \Big( \6^i a^j - \6^j a^i\Big) \6_j \alpha
\end{eqnarray}
So we can show the conservation of the galilean currents as follows 
\begin{eqnarray}
 \6_t j^0 + \6_i j^i 
= \frac{1}{2} \6_t \6^i a^0 \6_i \alpha +  \6_i [-\frac{1}{2} \6^i a^0 \6_t \alpha - \frac{1}{2} \Big( \6^i a^j - \6^j a^i\Big) \6_j \alpha] \nonumber \\
= -\frac{1}{2} \Big(\6_i \6^i a^0 \Big) \6_t \alpha + \frac{1}{2}\Big(\6_t \6^j a^0 - \6_i f^{ij} \Big) \6_j \alpha + \frac{1}{2} \6^i a^0 \6_t \6_i \alpha - \frac{1}{2} \6^i a^0 \6_i \6_t \alpha - \frac{1}{2} f^{ij} \6_i \6_j \alpha \nonumber \\
= 0 \hspace{2in} \label{conservn1}
\end{eqnarray}
In the second line of eqn \ref{conservn1}, the first and second term is zero from equations of motion eqn \ref{ee1} and \ref{ee2} respectively, third and fourth terms get cancelled and fifth term vanishes because of antisymmetry. \\
The covariant components of the current for the relativistic case are
\begin{eqnarray}
J_0 = \frac{\6 \mathcal{L}_e}{\6 (\6^0 A^\nu)}\delta A^\nu \label{a3}
\end{eqnarray}
\begin{eqnarray}
J_i = \frac{\6 \mathcal{L}_e}{\6 (\6^i A^\nu)}\delta A^\nu \label{a4}
\end{eqnarray}
Now using the map given in tale \ref{T1} we get
\begin{eqnarray}
 j_0 = -c^2 \frac{\6 \mathcal{L}_e}{c \6 (\6_t a^0)} c \delta a^0 - c^2 \frac{\6 \mathcal{L}_e}{\6 (\6_t a^i)}\delta a^i = 0 
\end{eqnarray}
 \begin{eqnarray}
 j_i
= \frac{\6 \mathcal{L}_e}{c \6 (\6^i a^0)} c \delta a^0 + \frac{\6 \mathcal{L}_e}{\6 (\6^i a^j)}\delta a^j 
= f_{ij} ~\6^j \beta 
\end{eqnarray} 
So the conservation looks like
\begin{eqnarray}
\6_t j_0 + \6^i j_i 
= \Big(\6^i f_{ij} \Big) \6^j \beta + f_{ij}\6^i\6^j \beta = 0, \label{conservn2}
\end{eqnarray}
In the conservation equation \ref{conservn2}, first term is zero from equation \ref{ee4} and second term vanishes because of antisymmetry. \\
\noindent \underline {\bf Magnetic limit} \\
Using the maps for magnetic limit given in table \ref{T1} and exploiting eqns \ref{a1} and \ref{a2} we get
\begin{eqnarray}
 j^0 
= -c^2 \frac{\6 \mathcal{L}_m}{(-c)\6 (\6_t a_0)}\delta a_0 -\frac{\6 \mathcal{L}_m}{\6 (\6_t a_i)}\delta a_i 
= 0 
\end{eqnarray}
\begin{eqnarray}
 j^i = 
= \frac{\6 \mathcal{L}_m}{(-c)\6 (\6_i a_0)}(-c)\delta a_0 + \frac{\6 \mathcal{L}_m}{\6 (\6_i a_j)}\delta a_j 
= -\frac{1}{2} f^{ij} \6_j \alpha 
\end{eqnarray}
\begin{eqnarray}
\6_t j^0 + \6_i j^i 
= -\frac{1}{2} \Big(\6_i \6^i a^j - \6_i \6^j a^i \Big) \6_j \alpha = 0 \label{conservn3}
\end{eqnarray}
second line of eqn \ref{conservn3} vanishes from eqn \ref{em2}.
Likewise for the covariant case,
\begin{eqnarray}
 j_0 =  (-c)  \frac{\6 \mathcal{L}_m}{\6 (\6_t a^0)}(-\frac{1}{c})\delta a^0 + \frac{\6 \mathcal{L}_m}{\6 (\6_t a^i)}\delta a^i 
= \frac{1}{2} \6_i a_0 \6^i \beta 
\end{eqnarray}
\begin{eqnarray}
 j_i
= \frac{\6 \mathcal{L}_m}{\6 (\6^i a^0)}\delta a^0 + \frac{\6 \mathcal{L}_m}{\6 (\6^i a^j)}\delta a^j 
= -\frac{1}{2} \6_i a_0 \6_t \alpha - \frac{1}{2} f_{ij} \6^j \beta 
\end{eqnarray}
So the conservations look like
\begin{eqnarray}
\6_t j_0 + \6^i j_i 
=\frac{1}{2} \6_t \6_i a_0 \6^i \beta - \frac{1}{2} (\6^i \6_i a_0 ) \6_t \beta - \frac{1}{2} \6^i f_{ij} \6^j \beta 
= \frac{1}{2} \Big( \6_t \6_j a_0  -  \6^i f_{ij} \Big) \6^j \beta
=0 \hspace{1.0in}\label{conservn4}
\end{eqnarray}
In eqn \ref{conservn4} third line vanishes from eqn \ref{em4}.
\end{appendices}

\begin{thebibliography}{99}
\bibitem{Taylor} M. Taylor, Non-relativistic holography, 0812.0530. 
%\bibitem{Andringa} R. Andringa, E. A. Bergshoeff, S. Panda and M. de Roo, Class. Quantum Grav. 28 105011 (2011).
\bibitem{andreev1}  O. Andreev, M. Haack and S. Hofmann, Phys. Rev. D 89, 064012
(2014) doi:10.1103/PhysRevD.89.064012 [arXiv:1309.7231 [hep-th]].
\bibitem{andreev2}  O. Andreev, Phys. Rev. D 91, no. 2, 024035 (2015)
doi:10.1103/PhysRevD.91.024035 [arXiv:1408.7031 [hep-th]].
\bibitem{jensen}  K. Jensen and A. Karch, JHEP 1504, 155 (2015)
doi:10.1007/JHEP04(2015)155 [arXiv:1412.2738 [hep-th]].
\bibitem{rb1} R. Banerjee and P. Mukherjee, Phys. Rev. D 93, no. 8, 085020 (2016)
doi:10.1103/PhysRevD.93.085020 [arXiv:1509.05622 [gr-qc]].
\bibitem{rb2}  R. Banerjee, S. Gangopadhyay and P. Mukherjee, Int. J. Mod. Phys.
A 32, no. 19n20, 1750115 (2017) doi:10.1142/S0217751X17501159
[arXiv:1604.08711 [hep-th]].
\bibitem{Son} D.T. Son and M. Wingate, Annals. of. Physics. 321, 197-224 (2006).
\bibitem {Pal} B. Grinstein and S. Pal, Phys. Rev. D 97, no. 12, 125006 (2018).
\bibitem{Geracie} M. Geracie, arXiv:1611.01198 [hep-th]. 
\bibitem{Jain} A. Jain, Phys. Rev. D 93, no. 6, 065007 (2016).
\bibitem{rb3} R. Banerjee and P. Mukherjee, doi: 10.1016/j.nuclphysb.2018.11.002, [arXiv: 1801.08373].
%\bibitem{Mitra} A. Mitra, Int. J. Mod. Phys. A 32, no. 36, 1750206 (2017).
\bibitem{Morand} K. Morand, arXiv:1811.12681 [hep-th].
\bibitem{Read}  J. Read and N. J. Teh, Class. Quant. Grav. 35, no. 18, 18LT01 (2018).
\bibitem{gkk} G. K. Karananas, doi:10.5075/epfl-thesis-7173.
\bibitem{Leblond} M. L. Bellac and J.-M. Levy-Leblond, Galilean Electromagnetism, Nuovo Cimento. 14B (1973) .
\bibitem{Bagchi} A. Bagchi and R. Gopakumar, Galilean Conformal Algebras and AdS/CFT, JHEP 07 (2009) 037 [0902.1385].
\bibitem{Mehra1} A. Bagchi, R. Basu and A. Mehra, Galilean Conformal Electrodynamics, JHEP 11 (2014) 061 [1408.0810].
\bibitem{Duval} C. Duval, G. W. Gibbons, P. A. Horvathy and P. M. Zhang, Carroll versus Newton and Galilei: two dual non-Einsteinian concepts of time, Class. Quant. Grav. 31 (2014) 085016
[1402.0657].
\bibitem{Mehra2} A. Bagchi, R. Basu, A. Kakkar and A. Mehra, Galilean Yang-Mills Theory, JHEP 04 (2016) 051 [1512.08375].
\bibitem{Bleeken} D. Van den Bleeken and C. Yunus, Newton-Cartan, Galileo-Maxwell and Kaluza-Klein, Class. Quant. Grav. 33 (2016) 137002 [1512.03799].
\bibitem{Bergshoeff} E. Bergshoeff, J. Rosseel and T. Zojer, Non-relativistic fields from arbitrary contracting backgrounds, Class. Quant. Grav. 33 (2016) 175010 [1512.06064].
\bibitem{Festuccia} G. Festuccia, D. Hansen, J. Hartong and N. A. Obers, Symmetries and Couplings of Non-Relativistic Electrodynamics, JHEP 11 (2016) 037 [1607.01753].
\bibitem{Basu} K. Banerjee, R. Basu and A. Mohan, Uniqueness of Galilean Conformal Electrodynamics and its Dynamical Structure, JHEP 11 (2019) 041 [1909.11993].
\bibitem{Mehra3} A. Mehra and Y. Sanghavi, Galilean electrodynamics: covariant formulation and Lagrangian, JHEP 09 (2021) 078 [2107.08525].
\bibitem{Mehra4} A. Bagchi, J. Chakrabortty and A. Mehra, Galilean Field Theories and Conformal Structure, JHEP 04 (2018) 144 [1712.05631]. 
\bibitem{Khanna} E. S. Santos, M. de Montigny, F. C. Khanna and A. E. Santana, Galilean covariant Lagrangian models, J. Phys. A 37 (2004) 9771.
\bibitem{Chapman} S. Chapman, L. Di Pietro, K. T. Grosvenor and Z. Yan, Renormalization of Galilean Electrodynamics, JHEP 10 (2020) 195 [2007.03033].
\bibitem{rb4}  Rabin Banerjee. Hamiltonian formulation of higher rank symmetric gauge theories. The European Physical Journal C, 82(1):1–12, 2022. [arXiv: 2105.04152 ].
\bibitem{rb5} R. Banerjee and A. Chakraborty, Shift symmetries and duality web in gauge theories,  [arXiv: 2210.12349 ].
\end{thebibliography}
%\end{document}

%%%%%%%%%%%%%%%%%%%%%%%%%%%%%%%%%%%%%%%%%%%%%%%%%%%%%%%%%%%%%%%%%%%%%%%%%%%%%%%%%%%%%%%%%%%%%%%%%%%%%%%%%%%%%%%%%%%%%%%%%%%%%%%%%%%%%%%%%%%%%%%%%%%%








  













