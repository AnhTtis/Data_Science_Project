% CVPR 2023 Paper Template
% based on the CVPR template provided by Ming-Ming Cheng (https://github.com/MCG-NKU/CVPR_Template)
% modified and extended by Stefan Roth (stefan.roth@NOSPAMtu-darmstadt.de)

\documentclass[10pt,twocolumn,letterpaper]{article}

%%%%%%%%% PAPER TYPE  - PLEASE UPDATE FOR FINAL VERSION
\usepackage{cvpr}      % To produce the REVIEW version
%\usepackage{cvpr}              % To produce the CAMERA-READY version
%\usepackage[pagenumbers]{cvpr} % To force page numbers, e.g. for an arXiv version

% Include other packages here, before hyperref.
\usepackage{graphicx}
\usepackage{amsmath}
\usepackage{amssymb}
\usepackage{booktabs}
\usepackage{subcaption}
\usepackage{color}
\usepackage{multirow}
\usepackage{multicol}
\usepackage{floatrow}
\usepackage{algorithm,algorithmic}
\usepackage[accsupp]{axessibility}
%\textcolor[rgb]{r,g,b}{text}
% \newcommand{\ie}{\emph{i.e.}}
% \newcommand{\eg}{\emph{e.g.}}


% It is strongly recommended to use hyperref, especially for the review version.
% hyperref with option pagebackref eases the reviewers' job.
% Please disable hyperref *only* if you encounter grave issues, e.g. with the
% file validation for the camera-ready version.
%
% If you comment hyperref and then uncomment it, you should delete
% ReviewTempalte.aux before re-running LaTeX.
% (Or just hit 'q' on the first LaTeX run, let it finish, and you
%  should be clear).
\usepackage[pagebackref,breaklinks,colorlinks]{hyperref}


% Support for easy cross-referencing
\usepackage[capitalize]{cleveref}
\crefname{section}{Sec.}{Secs.}
\Crefname{section}{Section}{Sections}
\Crefname{table}{Table}{Tables}
\crefname{table}{Tab.}{Tabs.}

\newtheorem{thm}{Theorem}
\newtheorem{prop}{Proposition}
\newtheorem{lemma}{Lemma}
\newtheorem{cor}[thm]{Corollary}
\newtheorem{definition}[thm]{Definition}

\def \y {$$\mathbf{y}$}
\def \E {\mathrm{E}}
\def \x {$\mathbf{x}$}
\def \g {$\mathbf{g}$}
\def \L {\mathcal{L}}
\def \D {\mathcal{D}}
\def \z {$\mathbf{z}$}
\def \u {$\mathbf{u}$}
\def \H {\mathcal{H}}
\def \w {$\mathbf{w}$}
\def \R {\mathbb{R}}
\def \S {\mathcal{S}}
\def \regret {\mbox{regret}}
\def \Uh {\widehat{U}}
\def \Q {\mathcal{Q}}
\def \W {\mathcal{W}}
\def \N {\mathcal{N}}
\def \A {\mathcal{A}}
\def \q {$\mathbf{q}$}
\def \v {$\mathbf{v}$}
\def \M {\mathcal{M}}
\def \c {$\mathbf{c}$}
\def \ph {\widehat{p}}
\def \d {$\mathbf{d}$}
\def \p {$\mathbf{p}$}
\def \q {$\mathbf{q}$}
\def \db {\bar{\d}}
\def \dbb {\bar{d}}
\def \I {\mathcal{I}}
\def \f {$\mathbf{f}$}
\def \a {$\mathbf{a}$}
\def \b {$\mathbf{b}$}
\def \ft {\widetilde{\f}}
\def \bt {\widetilde{\b}}
\def \h {$\mathbf{h}$}
\def \B {$\mathbf{B}$}
\def \bts {\widetilde{b}}
\def \fts {\widetilde{f}}
\def \Gh {\widehat{G}}
\def \bh {\widehat{b}}
\def \fh {\widehat{f}}
\def \vb {\bar{v}}
\def \zt {\widetilde{\z}}
\def \zts {\widetilde{z}}
\def \s {$\mathbf{s}$}
\def \gh {\widehat{\g}}
\def \vh {\widehat{\v}}
\def \Sh {\widehat{S}}
\def \rhoh {\widehat{\rho}}
\def \hh {\widehat{\h}}
\def \C {\mathcal{C}}
\def \V {\mathcal{V}}
\def \t {$\mathbf{t}$}
\def \xh {\widehat{\x}}
\def \Ut {\widetilde{U}}
\def \wt {\widetilde{\w}}
\def \Th {\widehat{T}}
\def \Ot {\tilde{\mathcal{O}}}
\def \X {\mathcal{X}}
\def \nb {\widehat{\nabla}}
\def \K {\mathcal{K}}
\def \P {\mathbb{P}}
\def \T {\mathcal{T}}
\def \F {\mathcal{F}}
\def \ft{\widetilde{f}}
\def \xt {\widetilde{x}}
\def \Rt {\mathcal{R}}
\def \V {\mathcal{V}}
\def \Rb {\bar{\Rt}}
\def \wb {\bar{\w}}
\def \fh {\widehat{f}}
\def \wh {\widehat{\w}}
\def \lh {\widehat{\lambda}}
\def \e {$\mathbf{e}$}
\def \B {\mathcal{B}}
\def \xt {\widetilde{\x}}
\def \Z {\mathcal{Z}}
\def \Y {\mathcal{Y}}
\def \ch {\widehat{c}}
\def \xh {\widehat{x}}
\def \Lh {\widehat{\L}}
\def \wh {\widehat{w}}
\def \eh {\widehat{\ell}}
\def \et {\widetilde{\ell}}
\def \Lt {\widetilde{\L}}
\def \mh {\widehat{\mu}}
\renewcommand{\algorithmicrequire}{ \textbf{Input:}} %Use Input in the format of Algorithm
\renewcommand{\algorithmicensure}{ \textbf{Output:}} %UseOutput in the format of Algorithm


%%%%%%%%% PAPER ID  - PLEASE UPDATE
\def\cvprPaperID{7489} % *** Enter the CVPR Paper ID here
\def\confName{CVPR}
\def\confYear{2023}

% \usepackage{xr}
\makeatletter
\newcommand{\specificthanks}[1]{\@fnsymbol{#1}}
% \newcommand*{\addFileDependency}[1]{
%   \typeout{(#1)}
%   \@addtofilelist{#1}
%   \IfFileExists{#1}{}{\typeout{No file #1.}}
% }
% \makeatother

% \newcommand*{\myexternaldocument}[1]{
%     \externaldocument{#1}
%     \addFileDependency{#1.tex}
%     \addFileDependency{#1.aux}
% }
% \myexternaldocument{appendix}


\begin{document}

%%%%%%%%% TITLE - PLEASE UPDATE
\title{Making Vision Transformers Efficient from A Token Sparsification View}

% \author{Shuning Chang\\
% National University of Singapore\\
% % Institution1 address\\
% {\tt\small changshuning@u.nus.edu}
% % For a paper whose authors are all at the same institution,
% % omit the following lines up until the closing ``}''.
% % Additional authors and addresses can be added with ``\and'',
% % just like the second author.
% % To save space, use either the email address or home page, not both
% \and
% Pichao Wang\\
% Alibaba Group\\
% % First line of institution2 address\\
% {\tt\small pichaowang@gmail.com}
% }

\author{Shuning Chang\textsuperscript{\rm 1}\thanks{Work done during an internship at Alibaba Group.}\quad Pichao Wang\textsuperscript{\rm 2}\thanks{Equal corresponding authors.}\textsuperscript{\, \specificthanks{3}}\quad Ming Lin\textsuperscript{\rm 2}\thanks{Work done at Alibaba Group, and now affiliated with Amazon.} \quad Fan Wang\textsuperscript{\rm 2}\quad David Junhao Zhang\textsuperscript{\rm 1} \\
Rong Jin\textsuperscript{\rm 2}\quad Mike Zheng Shou\textsuperscript{\rm 1}\footnotemark[2] \\ 
\textsuperscript{\rm 1}Show Lab, National University of Singapore \quad \textsuperscript{\rm 2}Alibaba Group \\
\small\{changshuning, junhao.zhang\}@u.nus.edu,\ \{fan.w, jinrong.jr\}@alibaba-inc.com,\ minglamz@amazon.com,\\
{\small\{pichaowang, mike.zheng.shou\}@gmail.com}
}


\maketitle

%%%%%%%%% ABSTRACT
\begin{abstract}
The quadratic computational complexity to the number of tokens limits the practical applications of Vision Transformers (ViTs). Several works propose to prune redundant tokens to achieve efficient ViTs. However, these methods generally suffer from (i) dramatic accuracy drops, (ii) application difficulty in the local vision transformer, and (iii) non-general-purpose networks for downstream tasks. In this work, we propose a novel Semantic Token ViT (STViT), for efficient global and local vision transformers, which can also be revised to serve as backbone for downstream tasks. The semantic tokens represent cluster centers, and they are initialized by pooling image tokens in space and recovered by attention, which can adaptively represent global or local semantic information. Due to the cluster properties, a few semantic tokens can attain the same effect as vast image tokens, for both global and local vision transformers. For instance, only 16 semantic tokens on DeiT-(Tiny,Small,Base) can achieve the same accuracy with more than 100\% inference speed improvement and nearly 60\% FLOPs reduction; on Swin-(Tiny,Small,Base), we can employ 16 semantic tokens in each window to further speed it up by around 20\% with slight accuracy increase. Besides great success in image classification, we also extend our method to video recognition. In addition, we design a STViT-R(ecover) network to restore the detailed spatial information based on the STViT, making it work for downstream tasks, which is powerless for previous token sparsification methods. Experiments demonstrate that our method can achieve competitive results compared to the original networks in object detection and instance segmentation, with over 30\% FLOPs reduction for backbone. Code is available at {\url{https://github.com/changsn/STViT-R}}.
\end{abstract}

\begin{figure*}
  \begin{minipage}[t]{0.5\linewidth}
    \centering
    \includegraphics[scale=0.125]{figure1a.jpeg}
    % \caption{aaa}
    % \label{fig:side:a}
  \end{minipage}%
  \begin{minipage}[t]{0.5\linewidth}
    \centering
    \includegraphics[scale=0.125]{figure1b.jpeg}
  \end{minipage}
  \caption{Left: the attention values of class tokens (normalized and reshaped in image shape) in different self-attention layers. Right: the attention maps in different self-attention layers. Zoom-in for better visibility. \vspace{-3mm}}
  \label{fig1}
\end{figure*}

% revised by Pichao
\section{Introduction}
In contrast to standard Convolutional Neural Networks (CNNs) approaches which process images pixel-by-pixel, Vision Transformers (ViTs)~\cite{dosovitskiy2020vit,touvron2021training,touvron2021going,liu2021swin,wu2021cvt} treat an image as a sequence of patch/image tokens, and have shown promising performance in prevalent visual recognition scenarios. 
% including image classification, object detection, and semantic segmentation, with both supervised and unsupervised (self-supervised) training configurations. 
However, these superior performances do not come for free: the quadratic computational complexity to the number of image tokens limits their application in practice. Previous works~\cite{song2021dynamic,zong2022self} have illustrated the large amount of redundancy in the image tokens and also shown the effect of filtering out unimportant tokens normally according to predefined scoring mechanism. However, these methods face the following challenges. Firstly, the predefined scoring mechanisms for filtering are generally imprecise. In Figure \ref{fig1}, on the left we visualize the class token values in different layers which are commonly used to score the token importance~\cite{xu2022evovit,fayyaz2021ats,liang2022evit}. Different layers have different value distributions, 
% meaning that different layers have different biases for token selection. 
thus using these imprecise scores for filtering would lead to unsatisfactory performance. For example, EViT~\cite{liang2022evit} has an accuracy drop of 1.3\% when saving 50\% FLOPs on DeiT-S~\cite{touvron2021training}. Secondly, the remaining tokens do not distribute evenly in space any more, making them hard to work in local vision transformers\footnote{In this paper, we define the vision transformer with global self-attention (like DeiT) as global vision transformer and the vision transformer with local self-attention (like Swin) as local vision transformer.}. Finally, large-scale token pruning tremendously damages the spatial structure and positional information, and causes difficulties when applied to downstream tasks, which they do not propose a solution to deal with.

To solve these problems, we propose Semantic Token ViT (STViT), for efficient global and local vision transformers, which also can be revised to serve as backbone for downstream tasks. The proposed approach is based on the following observations: (i) unlike local CNNs which learn spatial structure of images, vision transformer discretizes feature map as tokens for global feature exploration, relieving the requirements for maintaining the whole image structure and information;
%(i) unlike CNNs,  depending on the structured neighborhood in space to absorb inductive bias, self-attention mechanism discretizes a feature map as tokens and does not have convolutional inductive bias, which implies that ViTs do not need to hold on the information of a complete feature map like CNNs; 
(ii) discrete tokens are more beneficial for optimization~\cite{wang2021scaled}; (iii)
in Figure~\ref{fig1}, on the right shows the attention maps in different transformer layers, and there are only several vertical lines in the deep layers, which means that only a few tokens with global semantic information matter. Thus, we argue that it is not necessary to maintain massive structured tokens for ViTs, especially in the deep layers. Employing a few discrete tokens with high-level semantic information can potentially achieve both high performance and efficiency.

In STViT, the semantic tokens represent the cluster centers, and the number of them is far less than the original image tokens, significantly reducing the computational cost. Inspired by the fact that multi-head attention can conduct the cluster center recovery (Appendix~\ref{justification}),
%Inspired by the mechanism of class token which can be regarded as a cluster center of the whole image produced by self-attention, 
we only employ the off-the-shelf self-attention to generate the semantic tokens.
Specifically, the first few transformer layers are kept unchanged to obtain the image tokens with low-level features. The image tokens are then fed into our semantic token generation module (STGM) consisting of at least two transformer layers to generate semantic tokens. In each self-attention layer, the semantic tokens are input as queries, and the image tokens are fed as keys and values. The semantic tokens dynamically aggregate image tokens through the attention layers to recover cluster centers. 
% In the first self-attention layer, the semantic tokens are initialized by pooling image tokens in space to achieve discrete and uniform distribution in space. Thanks to the spatial initialization, the semantic tokens mainly incorporate local semantic information. 
In the first attention layer, the semantic tokens are initialized by an intra and inter-window spatial pooling which takes into account incorporating semantic information in each window and maximizing distance between adjacent windows.
Thanks to this spatial initialization, the semantic tokens mainly incorporate local semantic information and achieve discrete and uniform distribution in space. In the following attention layer, besides further clustering, the semantic tokens are equipped with global cluster centers, and the network can adaptively select partial semantic tokens to focus on global semantic information. After the STGM, the original image tokens are discarded, and only semantic tokens are kept for the subsequent transformer layers. Because the generation of semantic tokens is flexible and space-aware, our method can be plugged into both global and local vision transformers. The semantic tokens can be produced in each window for the local vision transformer.

Another property of STViT is its capability to serve as a backbone for downstream tasks, such as object detection and instance segmentation. Discussions have been missing in all previous methods~\cite{xu2022evovit,fayyaz2021ats,liang2022evit,ryoo2021tokenlearner,zong2022self} about how to use them in downstream task under the massive loss of spatial information during the token sparsification process, which actually seriously impedes the application of their method.
Instead, we design a novel STViT-R network based on STViT where a recovery module and dumbbell unit are adopted to periodically restore the full resolution feature map while the intermediate transformer layers continue to use semantic tokens to save computation cost, making our method work in downstream task.

The effectiveness of the proposed method is validated via a comprehensive empirical study on image and video ViT models. Only 16 semantic tokens % are enough for both global and local vision transformers. 
on DeiT-(Tiny, Small, Base) achieve nearly 50\% inference time reduction without any accuracy degradation; on Swin-(Tiny, Small, Base), we also improve the inference throughput by nearly 20\% with slight accuracy increase. 
Moreover, the proposed STViT-R achieves promising results on object detection and instance segmentation. 
To the best of our knowledge, this is one of first works to apply the token sparsification algorithm in local vision transformers, and use the ViTs as backbones in downstream tasks after large-scale token pruning. 
Our findings in ViTs uncover that maintaining the full-size feature map is unnecessary, and a few tokens with high-level semantic representations can achieve both high performance and efficiency. Thanks to its simplicity and general-purpose ability, our method can also serve as a new efficient ViT baseline architecture and a starting point for further research from the token sparsification perspective.


\section{Related work}
% Vanilla transformers have high computational and memory costs because the multi-head self-attention has quadratic computational complexity to the number of image tokens. Recently, various efficient ViTs have been proposed to alleviate this issue. The existing methods mainly focus on reducing the complexity of self-attention or reducing the number of tokens.  Swin Transformer~\cite{liu2021swin} adopts local self-attention, \ie, attending neighboring tokens within a constant window size, achieving a linear computational complexity in the self-attention with high performance. Many subsequent works~\cite{dong2021cswin,yu2022boat,wang2021crossformer,yang2021focal,huang2021shuffle,zhou2021elsa,chu2021twins} follow the local self-attention design to develop variants. 
\paragraph{Vision transformers.}
Vision Transformer (ViT)~\cite{dosovitskiy2020vit} first introduces a pure Transformer backbone for image classification. ViT variants further inspire the applications of transformer to various vision tasks beyond image/video classification~\cite{Yuan_2021_ICCV,touvron2021training,Xu_2021_ICCV,wang2021pyramid,wu2021cvt,touvron2021going,jiang2021all,bertasius2021space,arnab2021vivit,zhang2022morphmlp}, such as object detection~\cite{carion2020end,zhu2020deformable,zheng2020end,dai2021up}, semantic segmentation~\cite{wang2021max,wang2021end,zheng2021rethinking}, and self-supervised learning\cite{chen2021mocov3,caron2021emerging,li2021efficient}. Vanilla transformers have high computational and memory costs because the multi-head self-attention has quadratic computational complexity to the number of image tokens. Recently, various efficient ViTs have been proposed to alleviate this issue. The existing methods mainly focus on reducing the complexity of self-attention or reducing the number of tokens.  Swin Transformer~\cite{liu2021swin} adopts local self-attention, \ie, attending neighboring tokens within a constant window size, achieving a linear computational complexity in the self-attention with high performance. Many subsequent works~\cite{dong2021cswin,yu2022boat,wang2021crossformer,yang2021focal,huang2021shuffle,zhou2021elsa,chu2021twins} 
follow the local self-attention design to develop variants. Token sparsification~\cite{rao2021dynamicvit,wang2021not,pan2021ia,ryoo2021tokenlearner,chen2021chasing,song2021dynamic,yin2022vit,meng2021adavit,fayyaz2021ats,xu2022evovit,zong2022self,tang2021patch,kong2021spvit,liang2022evit} also attracts increasing attention.
\paragraph{Token sparsification.}
Token sparsification methods can be mainly categorized into hard pruning~\cite{rao2021dynamicvit,pan2021ia,chen2021chasing,song2021dynamic,yin2022vit,meng2021adavit,fayyaz2021ats,xu2022evovit,kong2021spvit,liang2022evit,tang2021patch} and soft pruning~\cite{ryoo2021tokenlearner,zong2022self}. Hard pruning methods filter out some unimportant tokens according to a predefined scoring mechanism. DynamicViT~\cite{rao2021dynamicvit}, SPViT~\cite{kong2021spvit}, and AdaViT~\cite{meng2021adavit} introduce additional prediction networks to score the tokens. Evo-ViT\cite{xu2022evovit}, ATS~\cite{fayyaz2021ats}, and EViT~\cite{liang2022evit} utilize the values of class tokens to evaluate the importance of tokens. However, it is difficult to achieve precise scoring as shown in the left of Figure~\ref{fig1}. Therefore, they usually suffer from a significant accuracy drop. For instance, EViT~\cite{liang2022evit} has an accuracy drop of 1.3\% when saving 50\% FLOPs on DeiT-S. Soft pruning methods generate new tokens from image tokens by importing additional attention networks. TokenLearner~\cite{ryoo2021tokenlearner} also argue for a few tokens to replace image tokens. However, its price is a 1.8\% accuracy drop when reducing 44\% FLOPs, which is far inferior to concurrent works. Besides performance degradation, previous methods also have the following disadvantages. First, whether or how to extend the methods to local vision transformers remains unexplored. Second, it has not been discussed about how to serve the downstream tasks like object detection and instance segmentation after the tokens are pruned.

In our method, we apply the off-the-shelf transformer layers to reduce token number. ~\cite{ma2021luna,jaegle2021perceiver,bai2021visual,zhang2019latentgnn,li2019expectation,chen2019graph} adopt similar approaches to achieve efficient non-local relationships. Our method is different from them as below: (i) our method extracts local semantic information instead of non-local relationships; (ii) the semantic tokens are a few cluster centers, which can replace the massive image tokens to achieve image classification; (iii) our method specializes in pruning tokens.
\vspace{-1mm}
\section{Method}
\vspace{-1mm}
The proposed STViT is presented in this section, which aims to construct an efficient and high-performance ViT. STViT is first introduced in Section~\ref{3.1}, followed by how to apply STViT in the local vision transformer in Section~\ref{3.2}. Based on STViT, STViT-R is developed to restore the spatial resolution for downstream tasks in Section~\ref{3.3}. 

\subsection{STViT}
\label{3.1}
\paragraph{Overall architecture.}An overview of STViT architecture is presented in Figure~\ref{arch a}. The patch embedding layer and shallow transformer layers are kept unchanged as a base module in our method. The base module copes with all the image tokens $X\in \mathbb{R}^{N_i\times C}$ to extract low-level features, where $N_i$ is the number of image tokens and $C$ is the number of channels. The image tokens are fed into the semantic token generation module (STGM) to generate $N_s$ semantic tokens $S\in \mathbb{R}^{N_s\times C}$. After the STGM, the image tokens $X$ can be discarded, and only semantic tokens $S$ with high-level semantic information are used in all the subsequent transformers. Due to $N_s \ll N_i$, our method can significantly reduce the computational cost.

\begin{figure*}[t]

    \begin{subfigure}{.55\textwidth}
        \centering
        \includegraphics[width=3in]{stvit.png}
        \caption{STViT}
        \label{arch a}
    \end{subfigure}
    \begin{subfigure}{.44\textwidth}
        \centering
        \includegraphics[width=2in]{stvitr.png}
        \caption{STViT-R}
        \label{arch b}
    \end{subfigure}
\vspace{-3mm}
\caption{The architectures of our STViT and STViT-R.\vspace{-5mm}}
\end{figure*}

\paragraph{Semantic token generation module (STGM).}
The whole image is represented by a few tokens with high-level semantic information through clustering.
%Inspired by the class token which can be regarded as a cluster center of the whole image,
Inspired by the fact that self-attention can conduct cluster center recovery (Appendix~\ref{justification}), we adopt the off-the-shelf self-attention layers to produce the semantic tokens. The STGM consists of at least two transformer layers.

The initial cluster centers $P\in \mathbb{R}^{N_s\times C}$ are generated by an spatial pooling which pools the image tokens into fixed $w_s\times w_s$ tokens, with $N_s = w_s \times w_s$. $w_s$ is generally set as $4$. The spatial pooling can be achieved by a non-parameterized adaptive spatial pooling or a super lightweight network with higher performance, intra and inter-window spatial pooling. The intentions of spatial pooling initialization are three folds. First, the initial cluster centers can distribute uniformly in space, making the generated semantic tokens more discrete and preventing the semantic tokens from collapsing to one point in the following layers. Second, the semantic tokens can be forced to represent more local and distinguished features. Finally, the representation of semantic tokens is associated with the specific spatial locations, which is the basis to allow our method to be applied in local self-attention and downstream tasks. The initial cluster centers $P$ then dynamically integrate the image tokens $X$ according to semantic information by attention mechanism. In the first transformer layer, the processing of the generation of semantic tokens can be written as
\begin{equation}
\begin{gathered}
\label{eq1}
\resizebox{0.85\hsize}{!}{
    $\hat{S^1} = MHA(P, X, X) + P, \quad
    S^1 = FFN(\hat{S^1}) + \hat{S^1}$, }
\end{gathered}
\end{equation}
where $MHA$ and $FFN$ are short for multi-head attention layer and feed-forward network, respectively, and the triplet input of $MHA$ are queries, keys, and values in turn. All the norm layers are omitted in all the equations for brevity. The initial cluster centers are produced in a window by an adaptive spatial pooling layer or an intra and inter-window spatial pooling, while the semantic tokens are generated from a global receptive field by a dynamic attention layer to ensure that they can extract high-level semantic representation. In order to further strengthen the clustering effect, we use the second transformer layer to repeat the clustering operation and guide the semantic tokens to extract global information. In this transformer layer, the semantic tokens are updated as:
\begin{equation}
\begin{gathered}
\label{eq2}
\resizebox{0.87\hsize}{!}{
    $\hat{S^2} = MHA(S^1+G, Concat(S^1, X), Concat(S^1, X)) + S^1$,}\\
    \resizebox{0.35\hsize}{!}{$S^2 = FFN(\hat{S^2}) + \hat{S^2}$},
\end{gathered}
\end{equation}
where $G\in \mathbb{R}^{N_s\times C}$ is the global cluster centers initialized by Gaussian noise and $Concat(\cdot)$ is a concatenation operation. The global cluster centers $G$ are responsible for global semantic information extraction like the class token. 
% The principle of $G$ is similar to the class token. 
Although $S^1$ and $G$ are summed together as the queries, they can be decoupled in the attention computation as:
\begin{equation}
\begin{gathered}
    \resizebox{0.87\hsize}{!}{$A_s = (S^1\cdot W_q)\cdot((S^1+X)\cdot W_k), \quad A_g = (G\cdot W_q)\cdot((S^1+X)\cdot W_k),$}\\
    \resizebox{0.35\hsize}{!}{$A = Softmax(A_s + A_g),$}
\end{gathered}
\end{equation}
where $W_q$ and $W_k$ are the linear projection weights of queries and keys. We can see that $S^1$ and $G$ integrate the keys to generate $A_s$ and $A_g$ independently and just share a Softmax operation to produce the final attention map $A$. Therefore, besides spatial semantic information, the semantic tokens also incorporate global semantic information with a negligible additional overhead. Though very similar, the global cluster centers are actually different from the learned positional encoding. We do not add the global cluster centers to keys. Moreover, it will be shown in the experiments that 
%some tokens take more global features, which is not the function of positional encoding, and 
inserting the actual learned positional encoding will cause an accuracy drop. The number of transformer layers in STGM is flexible. More transformer layers can be associated for clustering. Two transformer layers are employed by default, \ie, $S^2$ is the output of the STGM. Note that the image tokens are not updated in the STGM.

\paragraph{Intra and inter-window spatial pooling.}
Give $H\times W$ feature map $X$, we generate $N_s$($w_s\times w_s$) initial cluster centers. We uniformly split the $X$ into $w_s\times w_s$ windows $[X_w^i]^{N_s}_{i=1}$ with size $\frac{H}{w_s} \times \frac{W}{w_s}$ and each window generates one initial cluster center. To represent abundant semantic information, we take into account intra and inter-window relations, i.e., integrating important tokens in the window and maximizing the distance among  initial cluster centers in different windows. Specifically, we formulate an intra-window function $f_{intra}$ to produce a mask, $M_i=f_{intra}(X_w^i)$, which projects the input window from $\mathbb{R}^{\frac{H}{w_s} \times \frac{W}{w_s}\times C}\rightarrow\mathbb{R}^{\frac{H}{w_s} \times \frac{W}{w_s}}$. The idea is to let the intra-window function $f_{intra}$ adaptively select a combination of informative tokens in $X_w^i$, which is implemented by
\begin{equation}
\label{intra}
\vspace{-0.5mm}
\resizebox{0.9\hsize}{!}{
$M_i = Conv(GeLU(LayerNorm(DepthwiseConv(X_w^i))))$.}
\vspace{-0.5mm}
\end{equation}
Then, we compute each integrated token $\hat{P_i}=Softmax(M_i)\cdot X_w^i$ from each window, and arrange $[\hat{P_i}]^{N_s}_{i=1}$ by spatial structure to form the 2D tensor $\hat{P}\in \mathbb{R}^{w_s\times w_s \times C}$. We adopt an inter-window function $f_{inter}$ to compute the inter-window relations and generate the offset, $O=f_{inter}(\hat{P})$,to revise the mask $M$. The implementation of $f_{inter}$ is similar to Eq. \ref{intra}, except the mapping input from $\mathbb{R}^{w_s \times w_s\times C}\rightarrow\mathbb{R}^{w_s \times w_s \times \frac{HW}{w_s^2}}$, where $\frac{HW}{w_s^2}$ is the number of tokens in each window. For each window, the corresponding $O_i\in \mathbb{R}^\frac{HW}{w_s^2}$ is sliced from $O$ and reshaped to $\mathbb{R}^{\frac{H}{w_s} \times \frac{W}{w_s}}$. The $O_i$ is used to revise $M_i$. The final initial cluster center $P_i$ is computed by
\begin{equation}
\vspace{-0.5mm}
P_i = Softmax(M_i + O_i)\cdot X_w.
\vspace{-0.5mm}
\end{equation}
Our $f_{intra}$ and $f_{intra}$ are super lightweight and the introducing parameters can be negligible. For example, on DeiT-T, the  parameters only increase by 0.05\%.

\subsection{STViT in local vision transformers.}
\label{3.2}
Local self-attention has been widely used in current ViT models to balance efficiency and accuracy. As the generation of semantic tokens in STViT is flexible in space, it can be naturally applied in local self-attention. Suppose each local self-attention layer contains $N_w$ windows with size $w\times w$, we initialize $w_s\times w_s$ cluster centers in each window by our intra and inter-window spatial pooling. The total number of semantic tokens $N_s$ would be $w_s\times w_s\times N_w$. $w$ and $w_s$ are set as $7$ and $3$ by default separately. As a result, our method compresses more than 80\% image tokens in local self-attention. In the STGM, although initial cluster centers are from $w\times w$ windows, we use larger windows with size $w_k\times w_k$ to obtain keys and values in Eq.~\ref{eq1} and Eq.~\ref{eq2} to mitigate the effect of limited window size. Other operations in STGM are kept the same as Section~\ref{3.1}.

In the local ViT models, each local transformer layer is normally followed by a cross-window connection layer, such as a shift window transformer layer following a local transformer layer on Swin Transformer~\cite{liu2021swin}. In our method, 
%after the semantic tokens are produced by the semantic token generation module, 
the attention is computed within $w_s\times w_s$ window in the local self-attention layer,
and the cross-window connection can be achieved by computing self-attention in a larger-size (\eg, $4\times w_s $) sliding window because of the rare number of tokens in each window. For the low-resolution input, our cross-window connection layer is equal to a global self-attention layer.

%\paragraph{Semantic tokens for high-resolution images.}
\vspace{-1mm}
\subsection{STViT for downstream tasks}
\vspace{-1mm}
\label{3.3}
Our method significantly reduces the computation cost by using a small number of semantic tokens, while its side effect is losing nearly all the detailed position information. High-level vision tasks, such as object detection and instance segmentation, are difficult to be executed on this extremely incomplete feature map. This issue also exists in previous works, which hinders the application of token sparsification methods. To solve this issue, we design a STViT-R network based on STViT to restore the original spatial resolution from the semantic tokens.

Our STViT-R shown in Figure~\ref{arch b} has two modifications compared with STViT. First, we adopt a recovery module to restore the spatial resolution from semantic tokens; second, we regroup the transformer layers and construct dumbbell units composed of our STViT-R. 

\vspace{-2mm}
\paragraph{Recovery module.} In the recovery module, only the self-attention layer is employed to restore the spatial resolution without any additional networks. The image tokens $X$ and semantic tokens $S$ are partitioned as $N_w^r$ windows of size $w^r\times w^r$ and $w_s^r \times w_s^r$, respectively. The image tokens in each window aggregate the semantic tokens in the corresponding window, which is represented as:
\begin{equation}
\begin{gathered}
    \hat{X} = MHA(X, S, S) + X,\quad
    X = FFN(\hat{X}) + \hat{X}.
\end{gathered}
\end{equation}
This is a reverse operation of the generation of semantic tokens, using high-level semantic information to boost the image tokens.

\vspace{-2mm}
\paragraph{Dumbbell unit.} The transformer layers are regrouped into multiple dumbbell units in our STViT-R. Each dumbbell unit consists of four parts. The transformers in the first part are responsible for coping with image tokens; the second part is the semantic token generation module; the transformer layers in the third part deal with semantic tokens; the last part is the recovery module.  
Take the application on Swin-S (STViT-R-Swin-S) as an example. One, two, two and one transformer layers are allocated for these four parts, respectively. In total, each dumbbell unit is composed of 6 transformer layers. We concatenate three dumbbell units in Stage 3 of Swin-S.
In each dumbbell unit, the intermediate transformer layers process semantic tokens with high-level semantic information to save computational cost, and the complete spatial resolution is recovered at the end. By repeating multiple dumbbell units, the detailed spatial information will be preserved by the network, which can not only enhance the classification but also serve downstream tasks.

\vspace{-2mm}
\section{Experiments}
STViT will first be applied in two representative ViT models, DeiT~\cite{touvron2021training} and Swin~\cite{liu2021swin} for image classification and video recognition. To validate that our method is effective in downstream tasks, STViT-R is then performed on object detection and instance segmentation tasks.

\vspace{-2mm}
\subsection{Image classification}
\paragraph{Settings.} For image classification, all the models are trained on the ImageNet~\cite{deng2009imagenet} with 1.28M training images and 50K validation images from 1,000 classes. By default, the semantic token generation module (STGM) employs the $5^{th}$ and $6^{th}$ transformer layers of DeiT (with 12 layers in total), employs the $3^{th}$ and $4^{th}$ transformer layers of Stage 3 of Swin-T (with 12 layers in total), and employs the $11^{th}$ and $12^{th}$ transformer layers of Stage 3 of Swin-S and Swin-B (with 24 layers in total). The image resolution in training and inference is $224\times 224$ unless otherwise specified. The batch size is 1,024. All the models are trained from scratch for 300 epochs, and the augmentation and regularization strategies follow the original papers of DeiT and Swin. No knowledge distillation algorithms are used in our experiments. The classification is performed by applying a global average pooling layer on the output tokens of the last transformer layer, followed by a linear classifier. In evaluation, the top-1 accuracy using a single crop is reported. The FLOPs computations of this paper are measured by Fvcore\footnote{\url{https://github.com/facebookresearch/fvcore}}. Throughput is measured with the batch size of 128 on a V100 GPU. 

%On DeiT~\cite{touvron2021training}, we replace the original patch embedding layer containing a $16\times 16$ convolutional layer with four lightweight $3\times 3$ convolutional layers. We set the group of the last convolutional layer as $2$ to make the number of parameters not exceeding the original patch embedding layer for fair comparison. The reason for reconstructing the patch embedding layer is to obtain better low-level features to serve the generation of semantic tokens. Otherwise, we need to apply more transformer layers in the base module, causing a huge overhead on FLOPs.

On Swin~\cite{liu2021swin}, the semantic tokens are generated in Stage 3, and they are not downsampled in Stage 4 due to rare semantic tokens. The patch merging layer between Stage 3 and Stage 4 is replaced with a simple linear layer to double the number of channels. $w_k$ is set as 10 and 14 for two transformer layers of STGM.

\begin{table*} [t]
\small
\begin{center}
\small
%\resizebox{!}{2.28cm}{
\begin{tabular}{c|c|c|cccc}
\toprule
\multirow{2}{4em}{Model} & \multirow{2}{4em}{Metrics} & \multirow{2}{2em}{Base} & \multicolumn{4}{c}{No. of semantic tokens} \\
%\midrule
& & & 16 & 36  & 64 & 100 \\
\midrule
\multirow{3}{*}{STViT-DeiT-T} & Top-1 Acc(\%) & 72.2 & 72.2(+0.0\%) & 72.7(+0.5) & 73.0(+0.8) & 73.2(+1.0)\\
& FLOPs(G) & 1.26 & 0.53(-58\%) & 0.60(-52\%) & 0.71(-44\%) & 0.86(-32\%) \\
& Throughput(img/s) & 2752 & 5511(+101\%) & 4769(+74\%) & 4214(+53\%) & 3551(+29\%) \\
\midrule
\multirow{3}{*}{STViT-DeiT-S} & Top-1 Acc(\%) & 79.8 & 79.8(+0.0) & 80.1(+0.3) & 80.5(+0.7) & 80.6(+0.8)\\
& FLOPs(G) & 4.58 & 1.91(-58\%) & 2.20(-52\%) & 2.62(-43\%) & 3.16(-31\%) \\
& Throughput(img/s) & 1408 & 2891(+105\%) & 2542(+80\%) & 2229(+58\%) & 1837(+30\%) \\
\midrule
\multirow{3}{*}{STViT-DeiT-B} & Top-1 Acc(\%) & 81.8 & 81.8(+0.0) & 82.2(+0.4) & 82.6(+0.8) & 82.7(+0.9)\\
& FLOPs(G) & 17.58 & 7.31(-58\%) & 8.44(-52\%) & 10.04(-43\%) & 12.13(-31\%) \\
& Throughput(img/s) & 626 & 1308(+110\%) & 1150(+85\%) & 1087(+61\%) & 826(+33\%) \\ 
\bottomrule
\end{tabular}
%}
\end{center}
\vspace{-3mm}
\caption{Applying STViT to DeiT-T, DeiT-S, and DeiT-B. The top-1 accuracy, complexity in FLOPs, and throughput are reported for different numbers of semantic tokens.}
\label{cls deit}
\end{table*}

\begin{table*} [t]
\small
\begin{center}
\small
%\resizebox{!}{2.28cm}{
\begin{tabular}{c|c|c|c|ccc}
\toprule
\multirow{2}{4em}{Model} & \multirow{2}{4em}{Metrics} & \multirow{2}{2em}{Base} & Move & \multicolumn{3}{c}{No. of semantic tokens} \\
%\midrule
& & & STGM & 4 & 9  & 16 \\
\midrule
\multirow{3}{*}{STViT-Swin-T} & Top-1 Acc(\%) & 81.3 & 81.0(-0.3\%) & 80.8(-0.5) & 81.5(+0.2) & 81.8(+0.5\%) \\
& FLOPs(G) & 4.5 & 3.14(-30\%) & 2.99(-34\%) & 3.43(-24\%) & 4.06(-10\%) \\
& Throughput(img/s) & 878 & 1124(+29\%) & 1128(+29\%) & 1061(+22\%) & 1008(+15\%)\\
\midrule
\multirow{3}{*}{STViT-Swin-S} & Top-1 Acc(\%) & 83.0 & 82.8(-0.2\%) & 82.4(-0.6\%) & 83.0(-0.0) & 83.1(+0.1\%) \\
& FLOPs(G) & 8.7 & 5.95(-32\%) & 5.95(-32\%) & 6.53(-25\%) & 7.36(-15\%) \\
& Throughput(img/s) & 551 & 739(+35\%) & 732(+34\%) & 691(+26\%) & 657(+20\%) \\
\midrule
\multirow{3}{*}{STViT-Swin-B} & Top-1 Acc(\%) & 83.5 & 83.2(-0.3\%) & 83.0(-0.5) & 83.4(-0.1) & 83.7(+0.2\%) \\
& FLOPs(G) & 15.4 & 10.48(-32\%) & 10.48(-32\%) & 11.51(-25\%) & 12.97(-16\%) \\
& Throughput(img/s) & 415 & 558(+35\%) & 551(+33\%) & 521(+26\%) & 489(+19\%) \\
\bottomrule
\end{tabular}
%}
\end{center}
\vspace{-3mm}
\caption{Applying STViT to Swin-T, Swin-S, and Swin-B. The top-1 accuracy, complexity in FLOPs, and throughput are reported for different numbers of semantic tokens in each window. \textit{Base} indicates the corresponding original Swin model. \textit{Move STGM} indicates changing the default position of STGM.\vspace{-3mm}}
\label{cls swin}
\end{table*}

\vspace{-2mm}
\paragraph{Results.}
One of the advantages of our method is that it can be applied to both global and local vision transformers to reduce computational complexity.
Our main results on DeiT and Swin are summarized in Table \ref{cls deit} and Table \ref{cls swin}, respectively. The results of LV-ViT~\cite{jiang2021all} are illustrated in Appendix \ref{a2} due to limited space. We report the top-1 accuracy, FLOPs, and the throughput under different numbers of semantic tokens. On DeiT, the models with 16 semantic tokens achieve the same accuracy as the DeiT models with 196 tokens and save nearly 60\% FLOPs on DeiT-T, DeiT-S, and DeiT-B. With more semantic tokens, the accuracy can consistently outperform the base models. For instance, STViT-DeiT-B with 36 semantic tokens surpasses DeiT-B by 0.4\% accuracy with 52\% FLOPs reduction. 

The local vision transformer like Swin is already an efficient architecture compared to the global vision transformer, so the reduction of FLOPs on Swin models are smaller than on DeiT models. When 9 semantic tokens are used in each window, STViT-Swin models can reduce 25\% FLOPs with negligible accuracy loss on all the model sizes. If the number of tokens is reduced to 4 in each window (16 in total), a significant accuracy drop will occur, which indicates that local vision transformers need more semantic tokens than global vision transformers. We can move the STGM towards shallow layers to attain a better complexity/accuracy trade-off. In Table~\ref{cls swin}, STGM is moved by one transformer layer on STViT-Swin-T and by two layers on STViT-Swin-S and STViT-Swin-B to save over 30\% FLOPs with only about 0.3\% accuracy drops (column of ``Move STGM"). 

STViT-R equipped with recovery modules is designed for downstream tasks, while it also can perform image classification. The corresponding results are reported in Table~\ref{cls recovery}. On both Swin-S and Swin-B, STViT-R can save 33\% FLOPs with 0.3\% accuracy drop. The hyper-parameters we used in STGM are as same as STViT-Swin.

These results demonstrate that our method achieves both effectiveness and efficiency by employing a few semantic tokens to replace original image tokens. Our method reveals that constructing the tokens with high-level semantic representation is more important than maintaining structured tokens in ViTs. 
% As reflected by the throughput, our method does not have overhead of parameters, memory, or deployment, because no parameterized networks have been introduced.
As reflected by the throughput, our method does not have overhead of memory or deployment. Compared to the total parameters, the additional parameters introduced by intra and inter-window spatial pooling is negligible(less than 0.05\%), so we do not show them.
\vspace{-2mm}

\begin{table} [t]
\small
\begin{center}
\small
\resizebox{!}{0.8cm}{
\begin{tabular}{cccc}
\toprule
% \multicolumn{3}{c}{STViT-R-Swin-S} & \multicolumn{3}{c}{STViT-R-Swin-B} \\
Model & Top-1 Acc(\%) & FLOPs(G) & Throughput \\
\midrule
STViT-R-Swin-S & 82.7(-0.3) & 5.83(-33\%) & 717(+30\%) \\
STViT-R-Swin-B & 83.2(-0.3) & 10.26(-33\%) & 539(+30\%) \\
\bottomrule
\end{tabular}
}
\caption{STViT-R is evaluated on Swin-S and Swin-B on ImageNet. The top-1 accuracy, complexity in FLOPs, and throughput are reported.}
\label{cls recovery}
\end{center}
\vspace{-5mm}
\end{table}

\begin{table}[t]
    %\begin{subtable}{.5\linewidth}
      \centering
      \small
        %\resizebox{!}{1.262cm}{
        \begin{tabular}{cccc}
            \toprule
            Model & Top-1 Acc & FLOPs(G) & $\triangle$ \\
            \midrule
            \multicolumn{4}{c}{DeiT-S} \\
            \midrule
            DynamicViT~\cite{rao2021dynamicvit} & 79.3 & 2.9(-37\%) & -0.5\\
            IA-RED$^2$~\cite{pan2021ia} & 79.1 & 3.2(-30\%) & -0.7\\
            PS-ViT~\cite{tang2021patch} & 79.4 & 2.6(-43\%) & -0.4 \\
            TokenLearner~\cite{ryoo2021tokenlearner} & 76.1 & 1.9(-44\%) & -1.8\\
            DGE*~\cite{song2021dynamic} & 79.7 & 3.1 (-49\%) & -0.6 \\
            A-ViT*~\cite{yin2022vit} & 78.6 & 3.6 (-39\%) & -0.3\\
            Evo-ViT~\cite{xu2022evovit} & 79.4 & 3.0(-35\%) & -0.4 \\
            EViT~\cite{liang2022evit} & 78.5 & 2.3(-50\%) & -1.3 \\
            \textbf{STViT(Ours)} & 79.8 & 1.91(\textbf{-58\%}) & \textbf{-0.0}\\
            \bottomrule
            \multicolumn{4}{c}{DeiT-B} \\
            \midrule
            IA-RED$^2$~\cite{pan2021ia} & 80.3 & 11.8(-33\%) & -1.5\\
            DynamicViT~\cite{rao2021dynamicvit} & 81.3 & 11.2(-36\%) & -0.5\\
            PS-ViT~\cite{tang2021patch} & 81.5 & 9.8(-44\%) & -0.3 \\
            TokenLearner*~\cite{ryoo2021tokenlearner} & 83.7 & 28.7(-48\%) & -1.1\\
            Evo-ViT~\cite{xu2022evovit} & 81.3 & 10.2(-33\%) & -0.5 \\
            % EViT~\cite{liang2022evit} & 81.3 & 11.5(-35\%) & -0.5 \\
            EViT~\cite{liang2022evit} & 80.0 & 8.7(-51\%) & -1.8 \\
            \textbf{STViT(Ours)} & 81.8 & 7.31(\textbf{-58\%}) & \textbf{-0.0} \\
            % \textbf{STViT$^{\star}$(Ours)} & - & 15.75(\textbf{+2\%}) & \textbf{+0.0}\\
            \bottomrule
        \end{tabular}
        %}
        % \caption{DeiT-S\vspace{-2mm}}
    % \end{subtable}%
    % \begin{subtable}{.5\linewidth}
    %   \centering
    %     \resizebox{!}{1.262cm}{
    %     \begin{tabular}{cccc}
    %         \toprule
    %         Model & Top-1 Acc & FLOPs(G) & $\triangle$ \\
    %         \midrule
    %         IA-RED$^2$~\cite{pan2021ia} & 80.3 & 11.8(-33\%) & -1.5\\
    %         DynamicViT~\cite{rao2021dynamicvit} & 81.3 & 11.2(-36\%) & -0.5\\
    %         PS-ViT~\cite{tang2021patch} & 81.5 & 9.8(-44\%) & -0.3 \\
    %         TokenLearner~\cite{ryoo2021tokenlearner} & 83.7 & 28.7(-48\%) & -1.1\\
    %         Evo-ViT~\cite{xu2022evovit} & 81.3 & 10.2(-33\%) & -0.5 \\
    %         % EViT~\cite{liang2022evit} & 81.3 & 11.5(-35\%) & -0.5 \\
    %         EViT~\cite{liang2022evit} & 80.0 & 8.7(-51\%) & -1.8 \\
    %         \textbf{STViT(Ours)} & 81.8 & 7.31(\textbf{-59\%}) & \textbf{-0.0}\\
    %         % \textbf{STViT$^{\star}$(Ours)} & - & 15.75(\textbf{+2\%}) & \textbf{+0.0}\\
    %         \bottomrule
    %     \end{tabular}
    %     }
    %     \caption{DeiT-B\vspace{-2mm}}
    % \end{subtable} 
\caption{Comparisons with the state-of-the-art token sparsification methods on DeiT-S and DeiT-B. $\triangle$ shows the accuracy difference between each model and its base model.$*$: their base models are
not standard DeiT models.\vspace{-5mm}} \label{sota}
\end{table}

% \begin{table} [t]
% \caption{Applying STViT to Swin-T and Swin-S on Kinetics-400. The views are $4\times3$. The top-1 accuracy and  complexity in FLOPs are reported.}
% \small
% \begin{center}
% \small
% \resizebox{!}{0.65cm}{
% \begin{tabular}{cccc|cccc}
% \toprule
% % \multicolumn{3}{c}{STViT-R-Swin-S} & \multicolumn{3}{c}{STViT-R-Swin-B} \\
% Model & Pretrain & Top-1 Acc(\%) & FLOPs(G) & Model & Pretrain & Top-1 Acc(\%) & FLOPs(G) \\
% \midrule
% Swin-T & ImageNet-1K & 78.8 & 88 & Swin-S & ImageNet-1K & 80.6 & 166 \\
% STViT-Swin-T & ImageNet-1K & 78.5(-0.3) & 64.4(-27\%) & STViT-Swin-S & ImageNet-1K & 80.3(-0.3) & 120.5(-27\%) \\
% %\midrule
% %82.7(-0.3) & 5.83(-33\%) & 717(+30\%) & 83.2(-0.3) & 10.26(-33\%) & 539(+30\%) \\
% \bottomrule
% \end{tabular}
% }
% \end{center}
% \label{video}
% \vspace{-4mm}
% \end{table}

\begin{table*} [t]
\small
\begin{center}
\small
%\resizebox{!}{1.cm}{
\begin{tabular}{c|cccc|cccc|c}
\toprule
& AP$^{b}$ & AP$^{b}_{50}$ & AP$^{b}_{75}$ & AP$^{b}_{s}$ & AP$^{m}$ & AP$^{m}_{50}$ & AP$^{m}_{75}$ & AP$^{m}_{s}$ & FLOPs(G)\\
\midrule
Swin-S & 51.8 & 70.4 & 56.3 & 35.2 & 44.7 & 67.9 & 48.5 & 28.8 & 194 \\
STViT-R-Swin-S & 51.8 & 70.6 & 56.1 & 36.7 & 44.7 & 67.8 & 48.6 & 29.0 & 134(-31\%)\\
\midrule
Swin-B & 51.9 & 70.9 & 56.5 & 35.4 & 45.0 & 68.4 & 48.7 & 28.9 & 343\\
STViT-R-Swin-B & 52.2 & 70.8 & 56.8 & 36.5 & 45.2 & 68.3 & 49.1 & 29.5 & 233(-32\%) \\
\bottomrule
\end{tabular}
%}
\end{center}
\vspace{-3mm}
\caption{Results on COCO object detection and instance segmentation under Cascade Mask R-CNN with $3\times$ schedule. The FLOPs are measured for backbones.}
\label{detection}
\end{table*}
\vspace{-1mm}

\paragraph{Comparisons with existing token sparsification methods.}
In Table~\ref{sota}, we compare STViT with the state-of-the-art token sparsification methods on DeiT-S and DeiT-B. Due to different base models used by different methods, we adopt accuracy difference between each model and its base model $\Delta$ to evaluate them for fair. Results indicate that our method achieves the lowest accuracy drop $\Delta$ with the highest FLOPs reduction, outperforming all the state-of-the-art methods in both accuracy and efficiency significantly. 

\subsection{Video recognition}

\paragraph{Setting.}
For video recognition, we apply our STViT to Video Swin~\cite{liu2022video}. All the models are pre-trained on ImageNet-1K and trained on Kinetics-400~\cite{carreira2017quo}. We generate semantic tokens from each frame as illustrated in Section \ref{3.2}. The initialization from pre-trained models and other implementation details are as same as Video Swin~\cite{liu2022video}.

\vspace{-1mm}
\paragraph{Results.}
The results are presented in Table \ref{video}. On Swin-T and Swin-S, STViT-Swin can save about 27\% FLOPs with 0.3\% accuracy drop, which shows that our method works on video recognition.

\subsection{Applications in object detection and instance segmentation}
\paragraph{Settings.} Experiments of object detection and instance segmentation are conducted on COCO 2017~\cite{lin2014microsoft}. We evaluate STViT-R with Swin in Cascade Mask R-CNN~\cite{cai2018cascade,he2017mask} detection frameworks. The $w_s$ is set to $3$. The backbone models are pre-trained on ImageNet-1K and the pre-trained results are presented in Table~\ref{cls recovery}. All the other settings follow Swin~\cite{liu2021swin}.
\vspace{-3mm}
\paragraph{Comparison to Swin Transformer.} The performance of STViT-R-Swin using the Cascade Mask R-CNN framework with $3\times$ schedule is shown in Table~\ref{detection}. Our method achieves better performance with more than 30\% FLOPs reduction for backbone on bath object detection and instance segmentation. This validates that the recovery module and dumbbell unit can restore detailed spatial information, and the global context information integrated from the semantic tokens significantly benefits object detection. Ignoring spatial structure in the intermediate layers does not affect the object detection task, which is a meaningful fact to help design efficient object detection frameworks. Another interesting finding is that our method has a remarkable improvement for small object detection which is a challenging problem in the detection community. For instance, our STViT-R-Swin-S outperforms Swin-S by 1.5\% on $AP_s^b$. 

% \subsection{Applications in semantic segmentation}
% \paragraph{Settings.} ADE20K~\cite{zhou2019semantic} is a widely-used semantic segmentation dataset, including a broad range of 150 semantic classes. It has 25K images in total, with 20K for training, 2K for validation, and 3K for testing. UperNet~\cite{xiao2018unified} in mmseg~\cite{mmseg2020} is utilized as our base framework. The $w_s$ is set to $3$. All the other settings follow the Swin Transformers~\cite{liu2021swin}.
% \vspace{-2mm}
% \paragraph{Comparison to Swin Transformers.}
% Table~\ref{seg} presents the results of STViT-R-Swin on semantic segmentation. With similar FLOPs reduction, the drop on mIoU is larger compared with those in object detection tasks, which shows that our method still has a gap on dense prediction compared to the full-token network, and this is the limitation of this work.

% We analyze the relatively poor performance from two views. First, the STGM strictly prunes more than 80\% tokens by self-attention, which remains the high-level semantic information but loses nearly all the detailed information. Semantic segmentation is a dense pixel-level classification task, and the semantic tokens are difficult to enhance the pixel-level representation. Second, our spatial pooling layer with large kernel size in STGM and self-attention layers can be regarded as low-frequency filters. STGM filters most high-frequency information, which is necessary for semantic segmentation.

\begin{table}
\vspace{-2mm}
%\begin{floatrow}
\center
%\capbtabbox{
\resizebox{!}{1.2cm}{
\begin{tabular}{c|c|c|c}
\toprule
Model & Top-1 Acc(\%) & FLOPs(G) & Speed\\
\midrule
Swin-T & 78.8 & 88 & 779\\
STViT-Swin-T & 78.5(-0.3) & 64.4(-27\%) & 975(+25\%)\\
\midrule
Swin-S & 80.6 & 166 & 456\\
STViT-Swin-S & 80.3(-0.3) & 120.5(-27\%) & 572(+25\%)\\
\bottomrule
\end{tabular}
}
%}{
\vspace{-2mm}
 \caption{Applying STViT to Video Swin (Swin-T and Swin-S) on Kinetics-400. All the models are pre-trained on ImageNet-1K. The views are $4\times3$. The top-1 accuracy and  complexity in FLOPs are reported. Speed is evaluated by FPS.}
 \label{video}
 \small
 \end{table}
%}

\begin{table}
\center
%\capbtabbox{
\resizebox{!}{1.3cm}{
\begin{tabular}{c|c|c|c}
\toprule
Spatial & Global & Learned & Top-1 Acc(\%) \\
\midrule
\checkmark &  & & 79.4 \\
\midrule
& \checkmark & & 78.7 \\
\midrule
\checkmark & \checkmark & & 79.8 \\
\midrule
\checkmark &  & \checkmark & 79.7 \\
\bottomrule
\end{tabular}
}
%}{
\vspace{-2mm}
 \caption{Accuracy with different initialization of STViT. \textit{Spatial}, \textit{Global}, and \textit{Learned} indicate spatial initialization, global initialization, and learned positional encoding methods, respectively.}
 \label{initialization}
 \small
%}
%\end{floatrow}

\end{table}

% \begin{wraptable}{r}{5.3cm}
% \caption{Accuracy with different initialization of STViT. \textit{Spatial}, \textit{Global}, and \textit{Learned} indicate spatial initialization, global initialization, and learned PE, respectively.}
% \resizebox{!}{1.cm}{
% \begin{tabular}{c|c|c|c}
% \toprule
% Spatial & Global & Learned & Top-1 Acc(\%) \\
% \midrule
% \checkmark &  & & 80.0 \\
% \midrule
% & \checkmark & & 79.5 \\
% \midrule
% \checkmark & \checkmark & & 80.5 \\
% \midrule
% \checkmark &  & \checkmark & 80.4 \\
% \bottomrule
% \end{tabular}
% }
% %\label{initialization}
% \end{wraptable}

\begin{figure}
\includegraphics[width=1.\linewidth]{visualization.png}
\centering
\caption{Visualization example of attention maps in the first attention layer of STGM.\vspace{-3mm}}
\vspace{-1mm}
\label{vis}
\end{figure}

\begin{table}
% \begin{floatrow}
%\capbtabbox{
\center
\small
\begin{tabular}{c|cccc}

\toprule
No. of transformers & 2 & 3 & 4 \\
\midrule
Top-1 Acc(\%) & 79.8 & 79.5 & 79.6 \\
FLOPs(G) & 1.91 & 1.97 & 2.03 \\
\bottomrule
\end{tabular}
%}{
\vspace{-2mm}
 \caption{Performance evaluation on different numbers of transformer layers in STGM. Keeping the base module containing four transformer layers unchanged.\vspace{-1mm}}
 \label{number}
 \small
%}
\end{table}

\begin{table}
\center
\small
%\capbtabbox{
\begin{tabular}{c|c|c|c}
\toprule
& STViT-R & w/o DU & Reusing ST \\
\midrule
AP${^b}$ & 51.8 & 51.4 & 51.6 \\
AP${^m}$ & 44.7 & 44.4 & 44.5 \\
\bottomrule
\end{tabular}
%}{
 \caption{Ablation study on STViT-R w/o dumbbell units (\textit{w/o DU}) and reusing semantic tokens (\textit{Reusing ST)}.\vspace{-4mm}}
 \label{dumbbell units}
%}
%\end{floatrow}

\end{table}

\subsection{Ablation study}
All the following ablation experiments of STViT and STViT-R are conducted on the DeiT-S and Swin-S, respectively.
\vspace{-2mm}

\paragraph{Initialization analysis.}
The semantic tokens are the cluster centers recovered by attention layers. The initialization of cluster centers induces the representation of semantic tokens. Spatial and global initialization are adopted in Section~\ref{3.1} to guide the semantic tokens to integrate local and global semantic information separately. We compare different initialization components in Table~\ref{initialization}. When performing single global initialization(3$^{rd}$ row), we replace the spatial initial cluster centers with global initial cluster centers in the first transformer layer. The accuracy of using a single initialization method is far lower than using both, which shows the effectiveness of our initialization strategy. Our global initial cluster centers look similar to learned positional encoding since both use random initialization. To verify their distinction, we experiment with a real learned positional encoding by additionally adding the global initial cluster centers to keys, which causes 0.1\% accuracy drop (the last row of Table~\ref{initialization}). Therefore, our global initialization is different from learned positional encoding.

%To further illustrate the influence of initialization for clustering and the semantic representation of our semantic tokens, 
We visualize the attention maps of the first attention layer in the STGM in Figure~\ref{vis}. 
%Figure 3a is from the single spatial initialization experiment(row 4 in Table \ref{initialization}). 
Because the queries of this layer are spatial initial cluster centers, these attention maps visualize the local semantic information integration by attention. The attention layer groups the semantic information according to the position of initial cluster centers, which ensures to extract fine-grained semantic information and keep the difference among semantic tokens. 
The attention map in the second attention layer is visualized in Appendix~\ref{a3}, which reveals that the network fixedly selects particle semantic tokens to represent global semantic information. We also show the semantic representation of image tokens in the same transformer layer on DeiT in Appendix~\ref{a3}. Compared to original image tokens, our semantic tokens in Figure~\ref{vis} show more high-level semantic information.
\vspace{-2mm}

\paragraph{Number of transformer layers in STGM.}
Two transformer layers are adopted in the STGM by default. The effects of employing different numbers of transformer layers are shown in Table~\ref{number}. The additional  layers are from the ones behind STGM to keep the total number of layers unchanged. More transformer layers do not bring improvement.
\vspace{-2mm}

\paragraph{The effectiveness of the dumbbell unit.}
To verify the effectiveness of our dumbbell unit, we experiment STViT-R without dumbbell units, \ie, STViT equipped with only the recovery module. We employ $6^{th}$ and $7^{th}$ transformer layers to construct STGM and the last transformer layers to construct the recovery module in Stage 3. The FLOPs is the same as the full-model STViT-R for fair comparison. The results on the COCO are reported in Table~\ref{dumbbell units}. Inferior results demonstrate the effectiveness of the dumbbell unit.
\vspace{-2mm}

\paragraph{Reusing semantic tokens in dumbbell units.}
Semantic tokens are generated in each dumbbell unit. If they are produced only once in the first dumbbell unit and reused as initial cluster centers in the subsequent dumbbell units, the result is shown in Table~\ref{dumbbell units} with a slight performance drop.
\vspace{-1mm}

\section{Conclusion}
In this paper, we propose a simple and effective token sparsification method, semantic token vision transformer (STViT). Our method utilizes the clustering property of self-attention to generate a few semantic tokens with high-level information representation to replace the redundant image/video tokens, which can be applied in both global and local vision transformers. By simply configuring the recovery module, our method can be successfully applied to downstream tasks. Extensive experiments demonstrate that our method achieves better accuracy along with less inference time in most cases. The success in downstream tasks significantly boosts the development of token sparsification methods. We hope that this work can inspire more future research to pay much attention to high-level semantic representation in ViTs.

\section*{Acknowledgement}
This project is supported by the National Research Foundation, Singapore under its NRFF Award NRF-NRFF13-2021-0008, and Mike Zheng Shou's Start-Up Grant from NUS. The computational work for this article was partially performed on resources of the National Supercomputing Centre, Singapore. Shuning was supported by Alibaba Research Intern Program.

%%%%%%%%% REFERENCES
% {\small
% \bibliographystyle{ieee_fullname}
% \bibliography{egbib}
% }

{\small
%\bibliographystyle{ieee_fullname}
% \bibliography{egbib}
% This must be in the first 5 lines to tell arXiv to use pdfLaTeX, which is strongly recommended.
\pdfoutput=1
% In particular, the hyperref package requires pdfLaTeX in order to break URLs across lines.

\documentclass[11pt]{article}

% Remove the "review" option to generate the final version.
%\usepackage[review]{ACL2023}
\usepackage{ACL2023}

% Standard package includes
\usepackage{times}
\usepackage{latexsym}

% For proper rendering and hyphenation of words containing Latin characters (including in bib files)
\usepackage[T1]{fontenc}
% For Vietnamese characters
% \usepackage[T5]{fontenc}
% See https://www.latex-project.org/help/documentation/encguide.pdf for other character sets

% This assumes your files are encoded as UTF8
\usepackage[utf8]{inputenc}

% This is not strictly necessary, and may be commented out.
% However, it will improve the layout of the manuscript,
% and will typically save some space.
\usepackage{microtype}

% This is also not strictly necessary, and may be commented out.
% However, it will improve the aesthetics of text in
% the typewriter font.
\usepackage{inconsolata}


% If the title and author information does not fit in the area allocated, uncomment the following
%
%\setlength\titlebox{10cm}
%
% and set <dim> to something 5cm or larger.

%%%%%%%%%%%%%%%%%%%%%%%%%%%%%%%%%%
\usepackage{graphicx}
\usepackage{amsfonts}
\usepackage{amsmath}
\usepackage{bigdelim}
\usepackage{diagbox}
\usepackage{amsthm}
\usepackage{makecell}
\usepackage{mathtools}
\usepackage{booktabs}
\usepackage[shortlabels]{enumitem}
\graphicspath{ {figs/} }

\theoremstyle{remark}
\newtheorem*{question}{Question}

\newcommand{\tk}[1]{\textcolor{blue}{{#1}}}
\newcommand{\sy}[1]{\textcolor{red}{{#1}}}
\newcommand{\mg}[1]{\textcolor{purple}{{#1}}}
\newcommand{\lh}[1]{\textcolor{green}{{#1}}}
\newcommand{\lc}[1]{\textcolor{green}{{#1}}}

% Rounded color box
\definecolor{light_blue}{HTML}{cfdfff}
\usepackage[most]{tcolorbox}
\tcbset{on line, 
        boxsep=1pt, left=0pt,right=0pt,top=0pt,bottom=0pt,
        colframe=white,colback=light_blue,  
        highlight math style={enhanced}
        }

\newcommand{\quash}[1]{}  %Anything in \quash is ignored
\newcommand{\gpt}{\textsc{GPT-2}}
\newcommand{\bert}{\textsc{BERT}}
\newcommand{\bertlarge}{\textsc{BERT-large}}
\newcommand{\mask}{\texttt{[MASK]}}
\newcommand{\cls}{\texttt{[CLS]}}
\newcommand{\sep}{\texttt{[SEP]}}
\newcommand{\mat}{\texttt{mat}}
\newcommand{\id}{\texttt{id}}
\newcommand{\matl}{\texttt{mat}_{\ell \rightarrow \ell'}}
\newcommand{\matattnl}{\texttt{mat\_attn}_{\ell \rightarrow \ell'}}
\newcommand{\matffl}{\texttt{mat\_ffn}_{\ell \rightarrow \ell'}}
\newcommand{\matlnl}{\texttt{mat\_ln1\_ln2}_{\ell \rightarrow \ell'}}
\newcommand{\idl}{\texttt{id}_{\ell \rightarrow \ell'}}
\newcommand{\matlL}{\texttt{mat}_{\ell \rightarrow L}}
\newcommand{\matattnlL}{\texttt{mat\_attn}_{\ell \rightarrow L}}
\newcommand{\matfflL}{\texttt{mat\_ffn}_{\ell \rightarrow L}}
\newcommand{\matlnlL}{\texttt{mat\_ln1\_ln2}_{\ell \rightarrow L}}
\newcommand{\idlL}{\texttt{id}_{\ell \rightarrow L}}

\definecolor{blue(munsell)}{rgb}{0.0, 0.5, 0.69}
%%%%%%%%%%%%%%%%%%%%%%%%%%%%%%%%%%

\title{Jump to Conclusions: Short-Cutting Transformers\\With Linear Transformations}

% Author information can be set in various styles:
% For several authors from the same institution:
% \author{Author 1 \and ... \and Author n \\
%         Address line \\ ... \\ Address line}
% if the names do not fit well on one line use
%         Author 1 \\ {\bf Author 2} \\ ... \\ {\bf Author n} \\
% For authors from different institutions:
% \author{Author 1 \\ Address line \\  ... \\ Address line
%         \And  ... \And
%         Author n \\ Address line \\ ... \\ Address line}
% To start a seperate ``row'' of authors use \AND, as in
% \author{Author 1 \\ Address line \\  ... \\ Address line
%         \AND
%         Author 2 \\ Address line \\ ... \\ Address line \And
%         Author 3 \\ Address line \\ ... \\ Address line}

\author{Alexander Yom Din$^{1}$ ~~~~~ Taelin Karidi$^{1}$ ~~~~~ Leshem Choshen$^{1}$ ~~~~~
Mor Geva$^{2}$ 
\vspace{0.2cm} \\
$^1$Hebrew University of Jerusalem ~~~ $^2$Google Research \\
\small{\texttt{\{alexander.yomdin, taelin.karidi, leshem.choshen\}@mail.huji.ac.il}}, \small{\texttt{pipek@google.com}}}

\quash{
\author{Alexander Yom Din \\
  Hebrew University of Jerusalem \\ \texttt{alexander.yomdin@mail.huji.ac.il} \\\And
  Taelin Karidi \\
  Hebrew University of Jerusalem \\
  \texttt{taelin.karidi@mail.huji.ac.il} \\\And
  Leshem Choshen \\
  Hebrew University of Jerusalem \\ \texttt{leshem.choshen@mail.huji.ac.il} \\\And
  Mor Geva \\
  Google Research \\
  \texttt{pipek@google.com} \\}
}

\begin{document}
\maketitle



\begin{abstract}
% \vspace{-1em}
The diffusion-based generative models have achieved remarkable success in text-based image generation. However, since it contains enormous randomness in generation progress, it is still challenging to apply such models for real-world visual content editing, especially in videos. 
In this paper, we propose \texttt{FateZero}, a zero-shot text-based editing method on real-world videos without per-prompt training or use-specific mask. 
\RM{Specifically, different from a pipeline of two independent inversion and then generation stages, we find the intermediate attention maps during inversions store better structure and motion information. We thus reform them to temporally casual attention and replace them in the generation progress. To further reduce the unnecessary semantic leakage of source video and enhance the editing quality, we then remix the temporally casual attentions via the cross-attention features of the source prompt as the mask.}
To edit videos consistently, we propose several techniques based on the pre-trained models. Firstly, in contrast to the straightforward DDIM inversion technique, our approach captures intermediate attention maps during inversion, which effectively retain both structural and motion information. These maps are directly fused in the editing process rather than generated during denoising. To further minimize semantic leakage of the source video, we then fuse self-attentions with a blending mask obtained by cross-attention features from the source prompt. Furthermore, we have implemented a reform of the self-attention mechanism in denoising UNet by introducing spatial-temporal attention to ensure frame consistency.
Yet succinct, our method is the first one to show the ability of zero-shot text-driven video style and local attribute editing from the trained text-to-image model. We also have a better zero-shot shape-aware editing ability based on the text-to-video model~\cite{tuneavideo}. \RM{Besides video, our unified method also achieves state-of-the-art performance in zero-shot image editing.\chenyang{Need exp or remove the zero-shot image}} Extensive experiments demonstrate our superior temporal consistency and editing capability than previous works.
% The code will be released.
% \chenyang{emphasize: our observation at inversion time} \xiaodong{replacing the bold part to the actual pipeline: \textbf{Specifically, we work on replacing and mixing the attention maps between the inversion and generation since the self-attention map keeps the structure of the original natural image and the cross-attention is semantic-related, after remixing, we replace them in the corresponding generation steps for denoising.}}
% \footnote{Since there is no general video diffusion model is publicly available, we use one-shot video generation method~(Tune-A-Video~\cite{tuneavideo}) as the pretrained video diffusion model for zero-shot video editing\xiaodong{can be removed if we actually zero-shot on video}.}.
\end{abstract}
\section{Introduction}

The ability to reason about plans is critical for performing long-horizon tasks \citep{erol1996hierarchical, sohn2018hierarchical, sharma-etal-2022-skill}, compositional generalization \citep{corona-etal-2021-modular} and generalization to unseen tasks and environments \citep{shridhar2020alfred}.
Consider a simple long-horizon planning scenario where a robot is tasked with preparing a meal and serving it on the table. 
This presents a non-trivial planning problem since the agent needs to understand the sequence of operations required to perform the task and search for the relevant objects in the unfamiliar environment by interacting with various objects. %



Large language models have been recently shown to possess commonsense knowledge about the world such as object affordances and physical dynamics \citep{ouyang2022training,chowdhery2022palm}.
Early approaches considered text based environments and fine-tuned PLMs to predict actions given the history of past observations and actions \citep{jansen-2020-visually,micheli-fleuret-2021-language,yao-etal-2020-keep}.
Recent work has used this ability to reason about plans from text instructions in simulated household environments with simplifying assumptions such as text-only environment observations or feedback \citep{huang2022language,ahn2022can,li2022pre,logeswaran-etal-2022-shot}.


We focus on \emph{visually grounded planning} with PLMs --- the ability to adapt plans based on interaction and visual feedback from the environment.
While PLMs have strong planning commonsense priors, predictions from a PLM may not be directly realizable in the environment since the observation and action spaces are unknown.
This requires \emph{grounding} the PLM in the environment and adapting it to observe visual feedback, which is highly non-trivial.
Some prior works assume the availability of a pre-trained affordance function \citep{ahn2022can} or a success detector \citep{mirchandani2021ella}.
Notably, SayCan \citep{ahn2022can} completely decouples the PLM from observation information by selecting actions that have both high affordability (through a pre-trained affordance model) and high PLM likelihood.
Although this partially addresses the grounding problem, the use of visual feedback for action affordance alone is limited.
Often an agent must choose one of many affordable actions using information from observations.
For example, a driving agent should re-navigate and possibly turn around when encountering a ``road closed'' sign, but both turning around and driving forward are indistinguishable to SayCan because they are both affordable and the PLM is blind to observations.

Another workaround explored in prior work is translating the information in the visual observations to text using a pre-trained captioning system \citep{shridhar2021alfworld,huang2022language}.
However, it can be difficult to faithfully describe an image in words and information is lost in this inherently noisy process, which limits the information available to the planner.



Recent work shows that PLMs can be adapted for various natural language tasks by inserting tunable embeddings or soft prompts at the input of the PLM (also called prompt tuning or prefix tuning)~\citep{li-liang-2021-prefix,lester-etal-2021-power}.
This approach also extends to multi-modal understanding tasks such as image captioning \citep{mokady2021clipcap} and VQA \citep{tsimpoukelli2021multimodal} where images are encoded as soft prompts and finetuned for the target task.
Transformer based architectures have also been successfully applied to offline Reinforcement Learning in recent work \citep{chen2021decision,janner2021offline,li2022pre,reid2022can}.

Taking inspiration from these works, we propose the simple approach of embedding visual observations (`visual prompts') and \textit{directly inserting them as PLM input embeddings}.
The visual encoder and PLM are jointly trained for the target task, an approach we call \textbf{\oursfull}~(\ours).
By teaching the PLM to use observations for planning in an end to end manner, we remove the dependency on external data such as captions and affordability information that was used in prior work.
We show that this simple approach performs better than prior PLM-based planning approaches on two embodied planning benchmarks based on ALFWorld~\citep{shridhar2021alfworld} and Virtualhome~\cite{puig2018virtualhome}.



\section{Related Work}

%Here we summarize prior work on transfer learning and property inference.

%\shortsection{Transfer Learning}
%%Transfer learning reuses features learned by pre-trained models for new tasks, with the pretext that inherent similarities in the generic features will be useful for the downstream tasks and hence reducing their cost of downstream training. Specifically, the downstream model trainer will use a pre-trained upstream model as the starting point for the downstream training, with inclusion of (or replacement with) the task-specific classification layer/module. The downstream model is then trained by either updating all layers of the model (including ones reused from upstream model) or freezing some earlier layers of the reused parts as the ``feature extractor'' and only updating the rest. The latter approach is more popular as the reused feature extractors can already learn useful feature representations and the training cost is also much lower and affordable for individuals with limited computational resources. We study the vulnerability of the latter transfer learning approach in this paper. 


%\shortsection{Transfer Learning} 
Several works have demonstrated risks associated with transfer learning across a variety of attack goals. Wang et al.~\cite{wang2018great} and Yao et al.~\cite{yao2019latent} consider manipulating the upstream model such that the fine-tuned downstream models contain backdoors, misclassifying test inputs that contain predefined backdoor triggers. These transfer manipulations are tailored to their particular attack goals and cannot be applied for the property inference goal considered in this paper. Zou et al.~\cite{zou2020privacy} study the threat of membership inference attacks on transfer learning, but with normally trained upstream models.  
%\dnote{its clear that the goals are different for these attacks, but how similar are the methods?} \ynote{similarity of the methods? more details about the methods? do not know what is expected here}
%In contrast, we investigate the possibility of boosting the effectiveness of property inference by manipulating the upstream model training. % Schuster et al.~\cite{schuster2020humpty} show that the attacker can modify the corpus on which the word embedding is trained such that the downstream NLP models which use that embedding will behave abnormally.

%\shortsection{Property Inference}
The risk of property inference was introduced by Ateniese et al.~\cite{ateniese2015hacking}, % introduces the threat of inferring properties of the training data from pre-trained models, 
and several subsequent works have developed property inference (also known as distribution inference) attacks~\cite{Wang2022GroupPI, suri2022formalizing, Jurez2022BlackBoxAF, Hartmann2022DistributionIR}.
% Ganju et al.~\cite{ganju2018property} and Suri and Evans~\cite{suri2022formalizing} 
These works study property inference against normally trained models, and they launch attacks using a variety of black-box and white-box attacks. All the white-box attacks use meta-classifiers, which take the permutation-invariant representation~\cite{ganju2018property} of the model parameters as the features. We use the state-of-the-art white-box attack~\cite{suri2022formalizing} in our experiments.
%We will use the state-of-the-art white-box method proposed by Ganju et al.~\cite{ganju2018property} and later extended by suri et al.~\cite{suri2022formalizing} in this paper.
%\dnote{do we use these attacks?} 
Melis et al.~\cite{melis2019exploiting} and Zhang et al.~\cite{zhang2021leakage} focus on property inference in distributed training scenarios. In their settings, the attacker is a participant in the global model training and conducts property inference using meta-classifiers that are trained on model outputs or gradients. Similarly, Suri et al.~\cite{suri2022subject} focus on federated learning settings where the attacker is a participant (or the central server) that utilizes black-box attacks for inferring membership of data from particular subjects. %\dnote{if we use black-box attacks, explain which ones, or how ours are related to previous ones} 
For our experiments, We improve the black-box meta-classifier proposed by Zhang et al.~\cite{zhang2021leakage} using the ``query tuning'' technique in Xu et al.~\cite{xu2019detecting}. 

The closest works to ours are Chase et al.~\cite{saeed} and Chaudhari et al.~\cite{Chaudhari2022SNAPEE}, which both consider a scenario where the attacker can manipulate some of the training data of the model to induce a model that significantly increases property inference risk.
% \dnote{it enables precise property inference attacks?}.
These works assume an adversary with the ability to poison the victim's training data, while the adversary in our scenario has no access to the victim's training data, and therefore, their methods are not applicable.
% \dnote{example how different from ours, and why the methods are not applicable}
%Thus, their methods are not applicable to our transfer learning scenario.
%Their methods rely on inducing certain behavior correlated with the properties to be inferred, and thus are not applicable to our transfer learning scenario. \anote{Still a bit unclear why that is the case.}
%
There are also works similar to ours that leverage ``adversarial initializations'' for attack purposes.
% \cite{grosse2019adversarial, boenisch2021curious, wen2022fishing, fowl2021robbing}.
Grosse et al.~\cite{grosse2019adversarial} focus on scenarios where the attacker can control the parameter initialization of a model, and demonstrate that the attacker can use special initializations to damage the performance of the trained model. %This attack is orthogonal to ours.
Other works \cite{boenisch2021curious, wen2022fishing, fowl2021robbing} show that the malicious central server in a federated learning protocol can reconstruct some training samples via falsifying the global model in some training rounds and then analyzing the submitted gradients. These kinds of attacks do not apply to our transfer-learning scenario since the attacker cannot access the downstream gradients, and can only manipulate the upstream training.

\iffalse %%%%%%%%%%%%%%%%%%%%%%%%%%%%%%%%

In this section, we provide the background and also the summary of prior attacks on transfer learning (Section~\ref{sec:transfer_learning}) and property inference (Section~\ref{sec:property_inference}). Then, we introduce the closely related manipulation attacks against machine learning models to boost different privacy risks in Section~\ref{sec:active_inference_attacks}.

%\anote{Do we really need a dedicated section for this? It's barely 2 paragraphs right now.}

%\dnote{the most closely related work to ours are works that attempt to amplify inference attacks by poisoning models, the two most relevant I know of are \url{https://www.computer.org/csdl/proceedings-article/sp/2022/131600b569/1CIO8nmuota} and \url{https://arxiv.org/abs/2204.00032}, but need to look thoroughly for others. We should definitely be describing this and relating it to our work, probably in the introduction. Most of what is here is Background, but should be clear what this section is for (not muddling background and related work)}

\subsection{Transfer Learning} \label{sec:transfer_learning}
Transfer learning reuses features learned by pre-trained models for new tasks, with the pretext that inherent similarities in generic features can be useful for downstream tasks, thus reducing the cost of downstream training. Specifically, the downstream model trainer uses a pre-trained upstream model as the starting point for downstream training, with the inclusion (or replacement) of task-specific classification layers/modules. The downstream model is then trained by either updating all layers of the model (including ones reused from the upstream model) or freezing some earlier layers of the reused parts as the ``feature extractor'' and only updating the rest. The latter approach is more popular as the reused feature extractors can already learn useful feature representations and the training cost is also much lower and affordable for individuals with limited computational resources. We study the vulnerability of the latter transfer learning approach in this paper. 
%mainly in two ways:  1) all the layers (including ones reused from ) and tune the full model; the other one is to freeze some earlier layers of the model as the feature extractor and only tune the rest later layers. The second update strategy could achieve better efficiency since the frozen layers can already produce meaningful feature representations~\cite{wang2018great,yao2019latent}, and we will study the transfer learning using this strategy. 

Recently, various attacks have been proposed for the transfer learning setting, but with different attack goals from ours. Wang et al.~\cite{wang2018great} generate adversarial examples against black-box student models that transfer knowledge from publicly available teacher models without repeated queries. Yao et al.~\cite{yao2019latent} propose to manipulate the upstream model such that the downstream models derived from the upstream model contain backdoors, which would misclassify test inputs that contain some predefined backdoor triggers. Zou et al.~\cite{zou2020privacy} study the threat of membership inference attacks on transfer learning and the upstream models are trained normally. In contrast, we investigate the possibility of boosting the effectiveness of property inference by manipulating the upstream model training. Schuster et al.~\cite{schuster2020humpty} show that the attacker can modify the corpus on which the word embedding is trained such that the downstream NLP models which use that embedding will behave abnormally.

%This additionally allows model trainers to achieve satisfactory performance with limited training samples, leading to reduced computational costs. The most common approach reuses parameters in the earlier layers of the pre-trained model, either by fixing them as the feature extractor or just using them for initialization, to conduct downstream training.

\subsection{Property Inference} \label{sec:property_inference}

\shortsection{Property Inference Attacks} In property inference attacks, the adversary aims to infer some sensitive properties of some data, given a model trained on it. For example, the adversary may be interested in sensitive properties like the presence of people of a specific race in the dataset~\cite{ateniese2015hacking, melis2019exploiting}), or even be curious about the 
the statistics of the training set (e.g, the ratio of people with a specific gender~\cite{saeed, ganju2018property, suri2022formalizing, zhang2021leakage}).


Ateniese et al.~\cite{ateniese2015hacking} were the first to identify the threat of inferring properties of the training data from pre-trained models. Ganju et al.~\cite{ganju2018property} and Suri and Evans~\cite{suri2022formalizing} 
study property inference against normally trained models, and they launch attacks using white-box meta-classifiers, which utilize the permutation-invariance representation~\cite{ganju2018property} of the model parameters, while other works focus on distributed training~\cite{zhang2021leakage} where the attacker is a participant in the global model training and conducts property inference using meta-classifiers trained on model outputs. Similarly, Suri et al.~\cite{suri2022subject} focus on federated learning, where the attacker is a participant (or the central server) that utilizes black-box attacks for inferring membership of data from particular subjects. Chase et al.~\cite{saeed} propose an active property inference attack for data poisoning scenarios, which we will cover and compare to in Section~\ref{sec:active_inference_attacks}.

%The closest work to ours are by Chase et al.~\cite{saeed} and Tramer et al.~\cite{tramer2022truth}. In their work, the attacker can manipulate some of the training data of the model such that a model trained (from scratch) on the poisoned data has an increased inference risk. However, their methods are not applicable to the transfer learning scenario. 
%In this work, we will focus on the property inference in transfer learning scenarios in which the attacker releases the upstream model and infer sensitive properties of the downstream models tuned from that upstream model.
% 

\shortsection{Defenses}
Defending against property inference attacks is an open problem. There are no studies in the current literature on active adversaries, and only a couple on passive ones. Ma et. al.~\cite{ma2021nosnoop} propose a defense against property inference attacks on data batches in the  collaborative learning setting. However, adversaries in the transfer-learning setting do not have access to batch-wise gradients of the downstream trainer. Chen and Ohrimenko~\cite{chen2022protecting} utilize mechanisms that add carefully-crafted noise to features to provide theoretical guarantees against inference adversaries, but focus on query-based access to the underlying dataset, not a machine learning model trained on it. These existing defenses thus do not apply to our threat model.

%propose a framework that reduces property inference to Boolean functions of individual members, posing the ratio of members satisfying the given function in a dataset as the property. These property inference attacks have since then been proposed as distribution inference attacks~\cite{suri2022formalizing}, presenting such attacks as inferring properties of the distributions used to sample datasets, differentiating them from exact inference attacks like dataset inference~\cite{maini2021dataset}. Nearly all property inference attacks use meta-classifiers to perform inference: training models on versions of datasets with and without the target property, followed by training a meta-classifier on top of these classifiers's model representations. These representations can take several forms: using model weights themselves with permutation-invariance~\cite{ganju2018property}, or model activations or logits for a generated set of query points~\cite{xu2019detecting}. However, the capability of such approaches is limited: the most that these attacks have been shown to work is medium-sized convolutional networks on the CelebA dataset~\cite{suri2022formalizing}.


\subsection{Active Privacy Attacks} \label{sec:active_inference_attacks}
% Perhaps the closely related works to ours as ones that proactively enhance the effectiveness of privacy attacks by manipulating the model training process in certain ways~\cite{saeed, melis2019exploiting, nasr2019comprehensive, tramer2022truth}. 
%shown that the adversary can, by using proactive ways, achieve stronger attacks that infer private information from deep learning systems~\cite{nasr2019comprehensive, melis2019exploiting, tramer2022truth, saeed}. In this section, we introduce the ones that are close to ours.

In the decentralized federated learning training, by submitting specially crafted gradients to the central server, malicious agents can increase membership inference risk~\cite{nasr2019comprehensive} and property inference risks~\cite{melis2019exploiting} of other benign agents' training data. However, these attacks do not apply to transfer learning scenario, as the attacker cannot control model gradients of downstream training. In the centralized setting, researchers propose attacks to poison the victim's training data such that the impacts of attribute inference and membership inference~\cite{tramer2022truth} and property inference~\cite{saeed} attacks are amplified on the poisoned model.
The ability to poison the victim's data is a threat model orthogonal to ours, since we have no access to the victim's downstream data. While there is scope to combine such approaches for stronger attacks (albeit with stronger access assumptions), we choose to focus on the scenario with no read/write access to the victim's data.

\fi %%%%%%%%%%%%%%%%%%%%%%%%%%%%%%%%

\section{Linear Shortcut Across Blocks}
\label{sec:layer_jump}

To use a hidden representation from layer $\ell<L$ as a final representation, we propose to cast it using linear regression, while skipping the computation in-between these layers. More generally, this approach can be applied to cast any $\ell$-th hidden representation to any subsequent layer $\ell'>\ell$.


\subsection{Method}
\label{subsec:methodology_linear_shortcut}

Given a source layer $\ell$ and a target layer $\ell'$ such that $0 \leq \ell < \ell' \leq L$, our goal is to learn a mapping
%$A_{\ell', \ell} \in \mathbb{R}^{d_h \times d_h}$
from hidden representations at layer $\ell$ to those at layer $\ell'$. To this end, we first collect a set of corresponding hidden representation pairs $(h^\ell, h^{\ell'})$. Concretely, we run a set $\mathcal{T}$ of input sequences through the model, and for each input $s$, we extract the hidden representations $h_{i_s}^{\ell}, h_{i_s}^{\ell'}$, where $i_s$ is a random position in $s$.
Next, we learn a matrix $A_{\ell', \ell} \in \mathbb{R}^{d_h \times d_h}$ by fitting linear regression over $\mathcal{T}$, i.e., $A_{\ell', \ell}$ is a numerical minimizer for:
$$ A \mapsto \sum_{s \in \mathcal{T}} || A \cdot h_{i_s}^\ell - h_{i_s}^{\ell'} ||^2,$$ 
and define the mapping of a representation $h$ from layer $\ell$ to layer $\ell'$ as:
\begin{equation}
\label{eq:linear_jump}
    \matl{} (h) \coloneqq A_{\ell', \ell} \cdot h.
\end{equation}


\subsection{Baseline}
\label{subsec:baseline}

We evaluate 
% our method against 
the prevalent approach of ``reading'' hidden representations directly, without any transformation. 
Namely, the propagation of a hidden representation from layer $\ell$ to layer $\ell'$ is given by the identity function, dubbed \id{}:

$$ \idl{} (h) \coloneqq h.$$

% Notably, 
This baseline 
assumes that representations at different layers operate in the same linear space.

\subsection{Quality of Fit}
\label{subsec:experiments_r2}

We first evaluate our method by measuring how well the learned linear mappings approximate the representations at the target layer. To this end, we calculate the (coordinate-averaged) $r^2$-score of our mapping's outputs with respect to the representations obtained from a full inference pass, and compare to the same for the \id{} baseline.


\paragraph{Models.}

We use \gpt{} \cite{radford2019language}, a decoder-only auto-regressive LM, with $L = 48$, $d_h = 1600$, and \bert{} \cite{devlin-etal-2019-bert}, an encoder-only model trained with masked language modeling, with $L=24$, $d_h=1024$.
% \footnote{\label{footnote:hf}We use models and data from Huggingface \cite{wolf-etal-2020-transformers,lhoest-etal-2021-datasets}.}
%For masked token prediction, we use a masked LM head pre-trained for our \bert{} model.

% \footnote{Specifically, we use the Huggingface Transformers \cite{wolf-etal-2020-transformers} implementations of all these models.}

%\sy{We use \gpt{} \cite{radford2019language}, a decoder-only auto-regressive LM, coming in four scales; $\texttt{gpt2}$ ($L = 12$, $d_h = 768$), $\texttt{gpt2-medium}$ ($L = 24$, $d_h = 1024$), $\texttt{gpt2-large}$ ($L = 36$, $d_h = 1280$) and $\texttt{gpt2-xl}$ ($L = 48$, $d_h = 1600$). Also, we use \bert{} \cite{devlin-etal-2019-bert}, an encoder-only model trained with masked language modeling, coming in two scales;  \texttt{bert-base-uncased} ($L=12$, $d_h=768$) and \texttt{bert-large-uncased} ($L=24$, $d_h=1024$). For masked token prediction, we use masked LM heads pre-trained for our models. Specifically, we use the Huggingface Transformers \cite{wolf-etal-2020-transformers} implementations of all these models. The plots presented in this section are for $48$-layered \gpt{} and $24$-layered \bert{}.}

%\sy{We use \gpt{} \cite{radford2019language}, a decoder-only auto-regressive LM, in the Huggingface \cite{wolf-etal-2020-transformers} implementation\footnote{\url{https://huggingface.co/gpt2}}, coming in four scales; $\texttt{gpt2}$ ($L = 12$, $d_h = 768$), $\texttt{gpt2-medium}$ ($L = 24$, $d_h = 1024$), $\texttt{gpt2-large}$ ($L = 36$, $d_h = 1280$) and $\texttt{gpt2-xl}$ ($L = 48$, $d_h = 1600$). Also, we use \bert{} \cite{devlin-etal-2019-bert}, an encoder-only model trained with masked language modeling, in the Hugginface implementation, coming in two scales;  \texttt{bert-base-uncased}\footnote{\url{https://huggingface.co/bert-base-uncased}} ($L=12$, $d_h=768$) and \texttt{bert-large-uncased}\footnote{\url{https://huggingface.co/bert-large-uncased}} ($L=24$, $d_h=1024$). For masked token prediction, we use the \texttt{BertForMaskedLM} heads from Huggingface, pretrained for these models. The plots presented in this section are for $48$-layered \gpt{} and $24$-layered \bert{}.}

\paragraph{Data.}
We sample random sentences from Wikipedia,
% \footref{footnote:hf} 
collecting 9,000 (resp. 3,000) sentences for the training set $\mathcal{T}$ (resp. validation set $\mathcal{V}$).\footnote{We use sentences rather than full documents to simplify the analysis.}
%\sy{We use two data sources to evaluate our method. One is Wikiepdia \cite{lhoest-etal-2021-datasets}\footnote{\url{https://huggingface.co/datasets/wikipedia}}; we use \texttt{spaCy}\footnote{\url{https://spacy.io/}} to divide documents into sentences\footnote{We use sentences rather than full documents to simplify the analysis.}\footnote{We pick randomly a Wikipedia document and then pick randomly a sentence ending in a newline character in it. \sy{[maybe this footnote is not needed?]}}, collecting 9,000 (resp. 3,000) random sentences for the training set $\mathcal{T}$ (resp. validation set $\mathcal{V}$). The second is a news article sentences dataset, the 10K English 2020 news sentences corpus
% \footnote{\url{https://downloads.wortschatz-leipzig.de/corpora/eng_news_2020_10K.tar.gz}} from the Leipzig Corpora Collection \cite{goldhahn-etal-2012-building}, which we randomly divide into a training set $\mathcal{T}$ consisting of 9,000 examples and a validation set $\mathcal{V}$ consisting of 1,000 examples.
% We truncate sentences to the maximal token length allowed by the model \mg{do we ever need to truncate? a sentence has about 10 words and the max. input len is thousands} \sy{[I surely did not need to in Leipzig, but discovered (via a transformers runtime warning) that I do need to for some (probably a minority) of the Wikipedia sentences. This probably has to do with that it is not really ``sentences" necessarily, for example, I noticed that it has some listings or something like that (bulleted items)... So some minority might get very long I guess...]}.
For each example $s$, we select a random position $i_s$ and extract the hidden representations $h_{i_s}^{\ell}$ at that position from all the layers.
For \bert{}, we first replace the input token at position $i_s$ with a \mask{} token, as our motivation is interpreting predictions, which are obtained via masked tokens in \bert{} (see \S\ref{subsec:BERT}).
Thus, in this case, the hidden representations we consider
%in the case of \bert{}
are of \mask{} tokens only.
%As we observed highly similar results for the two data sources across all our experiments, throughout the paper we will mainly report results for Wikipedia (except for \S\ref{sec:robustness}, where we cross-validate).


\begin{figure}[t]
\includegraphics[scale=0.2]{figs/r2_scores_48.pdf}
% \includegraphics[width=\columnwidth]{figs/r2_scores_48.pdf}
\caption{The coordinate-averaged $r^2$-score of $\matl{}$ (left) and $\idl{}$ (right) (\gpt{}).}
\label{fig:r2_scores}
\end{figure}


\begin{figure}[t]
\setlength{\belowcaptionskip}{-10pt}
\includegraphics[scale=0.2]{figs/bertmask_r2_scores_24.pdf}
% \includegraphics[width=\columnwidth]{figs/bertmask_r2_scores_24.pdf}
\caption{The coordinate-averaged $r^2$-score of $\matl{}$ (left) and $\idl{}$ (right) (\bert{}).}
\label{fig:bertmask_r2_scores}
\end{figure}



\paragraph{Evaluation.}
For every pair of layers $\ell, \ell'$, such that $0 \leq \ell < \ell' \leq L$, we use the training set $\mathcal{T}$ to fit linear regression as described in \S\ref{subsec:methodology_linear_shortcut}, and obtain a mapping $\matl{}$. 
Next, we evaluate the quality of $\matl{}$ as well as of $\idl{}$ using the $r^2$-coefficient, uniformly averaged over all coordinates. Concretely, we compute the $r^2$-coefficient of each of the predicted representations $\matl{} (h_{i_s}^{\ell})$ and $\idl{} (h_{i_s}^{\ell})$ versus the true representations $h_{i_s}^{\ell'}$
over all $s \in \mathcal{V}$.
%as we vary $s \in \mathcal{V}$.
%for every $s \in \mathcal{V}$.



\paragraph{Results.}
Results for \gpt{} and \bert{} are presented in Figs.~\ref{fig:r2_scores} and~\ref{fig:bertmask_r2_scores}, respectively.
In both models, \mat{} consistently yields better approximations than \id{}, as it obtains higher $r^2$-scores (in blue) across the network. 
This gap between \mat{} and \id{} is especially evident in \bert{}, where \id{} completely fails to map the representations between most layers, suggesting that hidden representations are modified  substantially by every transformer block.
Overall, this highlights the shortcoming of existing practices to inspect representations in the same linear space, and the gains from using our method to approximate future layers.
% in the network.
\section{Linear Shortcut for Language Modeling}
\label{sec:prediction}

We saw that our method approximates future hidden representations substantially better than a naive propagation. 
In this section, we will show that this improvement also translates to better predictive abilities from earlier layers. Specifically, we will use our method to estimate how often intermediate representations encode the final prediction, in the context of two fundamental LM tasks; next token prediction and masked token prediction.

\paragraph{Evaluation Metrics.}
Let $h, h' \in \mathbb{R}^{d_h}$ be a final representation and a substitute final representation obtained by some mapping, and denote by $\delta (h), \delta (h') \in \mathbb{R}^{d_v}$ their corresponding output probability distributions (obtained through projection to the output vocabulary -- see details below). 
We measure the prediction quality of $h'$ with respect to $h$ using two metrics:
\begin{itemize}
[leftmargin=*,topsep=1pt,parsep=1pt]
    \item \textbf{Precision@$k$} ($\uparrow$ is better): This checks whether the token with the highest probability according to $\delta(h')$ appears in the top-$k$ tokens according to $\delta(h)$. Namely, we sort $\delta(h)$ and assign a score of $1$ if $\arg\max(\delta(h'))$ appears in the top-$k$ tokens by $\delta(h)$, and $0$ otherwise.
    
    \item \textbf{Surprisal} ($\downarrow$ is better): We measure the minus log-probability according to $\delta(h)$, of the highest-probability token according to $\delta(h')$. Intuitively, low values mean that the model sees the substitute result as probable and hence not surprising.
\end{itemize}

\noindent We report the average Precision@$k$ and Surprisal over the validation set $\mathcal{V}$.



\subsection{Next Token Prediction}
\label{subsec:next_token_prediction_task}

Auto-regressive LMs output for every position a probability distribution over the vocabulary for the next token. Specifically, the output distribution for every position $i$ is given by $\delta (h_i^L)$, where:
\begin{equation}\label{eq:output_distribution}
    \delta (h) = \texttt{softmax} ( E^\top \cdot h) \in \mathbb{R}^{d_v}
\end{equation}
For some LMs, including \gpt{}, a layer normalization $\texttt{ln\_f}$ is applied to the final layer representation before this conversion (i.e., computing $\delta (\texttt{ln\_f}(h))$ rather than $\delta (h)$).

Recall that our goal is to measure how well this distribution can be estimated from intermediate representations, i.e. estimating $\delta (h_i^L)$ from $\delta (h_i^\ell)$ where $\ell<L$. To this end, we first run examples from the validation set through the model, while extracting for each example $s$ the hidden representation of a random position $i_s$ at every layer. Next, we apply our mappings $\matlL{}$ and the $\idlL{}$ baseline to cast the hidden representations of every layer $\ell$ to final layer substitutes (see \S\ref{sec:layer_jump}). Last, for each layer, we convert its corresponding final-layer substitute to an output distribution (Eq.~\ref{eq:output_distribution}) and compute the average Precision@$k$ (for $k=1,5,10$) and Surprisal scores with respect to the final output distribution, over the validation set.

\paragraph{Results.}
Figs.~\ref{fig:pre} and~\ref{fig:surp} show the average Precision@$k$ and Surprisal scores per layer in $48$-layered \gpt{}, respectively (the plots for the other \gpt{} models are presented in \S\ref{sec:app_scale}). Across all layers, \mat{} outperforms \id{} in terms of both scores, often by a large margin (e.g. till layer $44$ the Precision@$1$ achieved by \mat{} is bigger than that of $\id{}$ by more than $0.2$). 
This shows that linear mappings enable not just better estimation of final layer representations, but also of the predictions they induce. Moreover, the relatively high Precision@$k$ scores of \mat{} in early layers ($0.62$-$0.82$ for $k=10$, $0.52$-$0.74$ for $k=5$, and $0.28$-$0.45$ for $k=1$) suggest that early representations already encode a good estimation of the final prediction. Also, the substantially lower Surprisal scores of \mat{} compared to \id{} imply that our method allows for a more representative reading into the layer-wise prediction-formation of the model than allowed through direct projection to the vocabulary.

\begin{figure}[t]
\centering
\includegraphics[scale=0.4]{figs/pre_48.pdf}
\caption{Precision@$k$ ($k = 1,5, 10$) of $\matlL{}$ and $\idlL{}$ for next token prediction in $48$-layered \gpt{}.}
\label{fig:pre}
\end{figure}

\begin{figure}[t]
\centering
\includegraphics[scale=0.35]{figs/surp_48.pdf}
\caption{Surprisal for $\matlL$ and the baseline $\idlL{}$ ($48$-layered \gpt{} next token prediction task). A 95\% confidence interval surrounds the lines.}
\label{fig:surp}
\end{figure}

\subsection{Masked Token Prediction}
\label{subsec:BERT}

We now conduct the same experiment for the task of masked language modeling, where the model predicts a probability distribution of a masked token in the input rather than the token that follows the input. Unlike next token prediction, where the output distribution is computed from representations of varying input tokens, in masked token prediction the output is always obtained from representations of the same input token (i.e. \texttt{[MASK]}).

For this experiment, we use \bert{}, on top of which we use a pretrained masked language model head $\delta$; given a token sequence $s$, a \mask{} token inside it and its final representation $h$, $\delta (h) \in \mathbb{R}^{d_v}$
 is a probability distribution over tokens giving the model's assessment
 of the likelihood of tokens to be fitting in place of the \mask{} token in $s$.


\begin{figure}[t]
\centering
\includegraphics[scale=0.4]{figs/bertmask_pre_24.pdf}
\caption{Precision@$k$ ($k = 1,5, 10$) for  $\matlL{}$ and the baseline $\idlL{}$ ($24$-layered \bert{} masked token prediction task).}
\label{fig:bertmask_pre}
\end{figure}

\begin{figure}[t]
\centering
\includegraphics[scale=0.35]{figs/bertmask_surp_24.pdf}
\caption{Surprisal for $\matlL{}$ and the baseline $\idlL{}$ ($24$-layered \bert{} masked token prediction task). A 95\% confidence interval surrounds the lines.}
\label{fig:bertmask_surp}
\end{figure}

\paragraph{Results.}
Figs.~\ref{fig:bertmask_pre} and~\ref{fig:bertmask_surp} present the average Precision@$k$ and Surprisal scores per layer in $24$-layered \bert{} (the plots for the $12$-layered \bert{} model are presented in \S\ref{sec:app_scale}), overall showing trends similar to those observed for next token prediction in \gpt{} (\S\ref{subsec:next_token_prediction_task}). This is despite the differences between the two tasks and the considerable architectural differences between \bert{} and \gpt{}.
Notably, the superiority of \mat{} over \id{} in this setting is even more prominent; 
while \mat{}'s precision is between $0.2-0.6$ in the first ten layers (Fig.~\ref{fig:bertmask_pre}), \id{}'s precision for all values of $k$ is close to zero, again strongly indicating that our method allows for better reading into early layer hidden representations. 
More generally, \mat{} improves the Precision@$1$ of \id{} by more than $17\%$ at most layers, and unveils that a substantial amount of predictions ($>25\%$ starting from layer $3$) appear already in the very first layers.
Interestingly, the (rough) divide between the first half of layers and last half of layers for $\id{}$ in Figs.~\ref{fig:bertmask_pre},~\ref{fig:bertmask_surp} seems to align with the two-hump shape of the blue region for $\mat{}$ in Fig.~\ref{fig:bertmask_r2_scores}.

\paragraph{Analysis.}
We manually compare the predictions of our mapping $\matlL{}$ with $\idlL{}$, for a $24$-layered \bert{} model.  Concretely, we select 50 random sentences from the Leipzig dataset. Next, for each layer $\ell$, we manually analyze how many of the top-$5$ tokens according to $\matlL{}$ and $\idlL{}$ fit into context. We consider a token to fit into context if it is grammatically plausible within the sentence (see Tab.~\ref{tab:manual} for concrete examples).
In the resulting $1250$ instances (i.e. $50$ sentences $\times$ $25$ representations), we observe a substantially higher plausibility rate of $85.36\%$ for \mat{} compared to $52.8\%$ for \id{}. In fact, only in less than $4.3\%$ of the instances there are more plausible tokens among the top-$5$ tokens according to \id{} than among the top-$5$ tokens according to \mat{}, further supporting the Surprisal results above.

\begin{table*}
\footnotesize
\setlength{\belowcaptionskip}{-15pt}
\begin{tabular}{p{0.3\linewidth}ccccc}
& $\texttt{id}_{4 \rightarrow 24}$ & $\texttt{mat}_{4 \rightarrow 24}$ & $\texttt{id}_{12 \rightarrow 24}$ & $\texttt{mat}_{12 \rightarrow 24}$ & $\texttt{id}_{24 \rightarrow 24}$ \\ \midrule
\multirow{5}{=}{aldridge had shoulder surgery in \mask{}.} & fellowship & \tcbox{time} & cyclist & \tcbox{2009} & \tcbox{september} \\
& employment & \tcbox{it} & emergencies & \tcbox{2008} & \tcbox{november} \\
& agreement & her & seniors & \tcbox{2010} & \tcbox{december} \\
& \#\#ostal & them & cycling & \tcbox{2006} & \tcbox{august} \\
& \#\#com & work & \tcbox{pennsylvania} & \tcbox{2007} & \tcbox{july} \\ \midrule
\multirow{5}{=}{on your next view you will be asked to \mask{} continue reading.} & \#\#com & be & be & be & \tcbox{please} \\
& accreditation & get & undergo & \tcbox{please} & \tcbox{simply} \\ 
& $	\copyright$ & go & spartans & help & \tcbox{also} \\ 
& fellowship & \tcbox{help} & seniors & \tcbox{simply} & \tcbox{again} \\ 
& summer & have & * & say & \tcbox{immediately} \\ \bottomrule
\end{tabular}
\caption{Examples of top-$5$ predictions at layers $4$, $12$ and $24$, under the mappings $\matlL{}$ and $\idlL{}$, for a $24$-layered \bert{} model. Grammatically plausible predictions (according to a human annotator) are marked in \tcbox{blue}. Note that at layer $24$ the predictions of $\matlL{}$ and $\idlL{}$ are the same (by definition).} 
\label{tab:manual}
\end{table*}

\section{Implication to Early Exiting}
\label{sec:applications}

%The fact that it is often possible to approximate
The possibility of approximating
the final prediction already in the early layers has important implications for efficiency; applying our linear mapping instead of executing transformer blocks of quadratic time complexity, could save a substantial portion of the computation. In this section, we demonstrate this in the context of early exiting.

When 
% performing transformer model inference under 
using an early exit strategy \cite{schwartz-etal-2020-right, xin-etal-2020-deebert, schuster2022confident}, one aims at deciding dynamically at which layer to stop the computation and ``read'' the prediction from the hidden representation of that layer.
More precisely, under a confidence measure paradigm, one decides to stop the computation for a position $i$ at layer $\ell$ based on a confidence criterion, that is derived from casting the hidden representation $h_i^\ell$ as a final-layer representation and converting it to an output probability distribution. Specifically, following \citet{schuster2022confident}, a decision to exit is made if the difference between the highest and the second highest probabilities is bigger than $$ 0.9 \cdot \lambda + 0.1 \cdot {\rm exp} (-4 i / N),$$
where $N$ is the average length of the input until position $i_s$ for $s \in \mathcal{V}$, and $\lambda$ is a hyper-parameter.

\begin{figure}[t]
\setlength{\belowcaptionskip}{-10pt}
\centering
\includegraphics[width=\columnwidth]{figs/ee_gpt2bert.pdf}
\caption{Precision@$1$ with early exit and ``fixed exit'', applied to the $24$-layer \gpt{} for next token prediction (left) and the $24$-layer \bert{} for masked token prediction (right). Varying the confidence parameter $\lambda$, the $x$-coordinate is the average number of layers processed before an early exit decision is reached.}
\label{fig:ee_gpt2bert}
\end{figure}

\quash{
\begin{figure}[t]
\setlength{\belowcaptionskip}{-10pt}
\centering
\includegraphics[scale=0.35]{figs/ee_pre1_24.pdf}
\caption{Precision@$1$ for the various early exit methods, and previous ``fixed exit'' methods for comparison ($24$-layer \gpt{} next token prediction task). Varying the confidence parameter $\lambda$, the $x$-coordinate is the average number of layers processed before an early exit decision is reached.}
\label{fig:ee_pre1}
\end{figure}
}

\paragraph{Experiment.}
We assess the utility of our mapping $\matlL{}$ for early exit as a plug-and-play replacement for $\idlL{}$, through which intermediate representations are cast into final-layer representations.
We use \gpt{} for the next token prediction and \bert{} for masked token prediction (both with 24 layers).
We run each of the models over the validation set examples, while varying the confidence parameter $\lambda$ and using either $\idlL{}$ or $\matlL{}$ for casting intermediate representations.
Furthermore, we compare these early exit variants to the ``fixed exit'' strategy from \S\ref{sec:prediction}, where the computation is stopped after a pre-defined number of layers rather than relying on a dynamic decision.
We evaluate each variant in terms of both prediction's accuracy, using the Precision@$1$ metric (see \S\ref{sec:prediction}), and efficiency, measured as the average number of transformer layers processed during inference.


\paragraph{Results.}
%Figs.~\ref{fig:ee_pre1} and~\ref{fig:bertmask_ee_pre1}
Fig.~\ref{fig:ee_gpt2bert}
plots the average Precision@$1$ score against the average number of layers processed, for $24$-layer \gpt{} and $24$-layer \bert{}. For both models, under an early exit strategy our mapping \mat{} again provides a substantial improvement over \id{}.
For example, aiming at $95\%$ average precision, \mat{} saves $\sim3.3$ ($13.8$\%) layers in \gpt{} compared to only $\sim1.4$ ($5.9$\%) layers by \id{}, and $\sim4.8$ ($20$\%) layers in \bert{} versus $\sim3.5$ ($14.6$\%) layers by \id{}.
These results highlight the potential gains prominent early exit methods can obtain by using our method.
Notably, in both models and for each of the mapping methods, early exit obtains better results than fixed layer exit, as expected. 

\quash{
\begin{figure}[t]
\setlength{\belowcaptionskip}{-10pt}
\centering
\includegraphics[scale=0.35]{figs/bertmask_ee_pre1_24.pdf}
\caption{Precision@$1$ for the various early exit methods, and previous ``fixed exit'' methods for comparison ($24$-layer \bert{} masked token prediction task). Varying the confidence parameter $\lambda$, the $x$-coordinate is the average number of layers processed before an early exit decision is reached.}
\label{fig:bertmask_ee_pre1}
\end{figure}
}
\section{Linear Shortcut Across Sub-Modules}
\label{sec:submodules}

% Our experiments show that
% , despite the commonly-applied simplification by interpretability works, transformer layers do not operate in the same linear space and 
% there is a major gap in approximating future representations using an identity mapping (\S\ref{sec:layer_jump}, \S\ref{sec:prediction}).
% Here, 
In this section, we investigate whether discrepancies across layers result from specific sub-modules or are a general behaviour of all sub-modules in the network.  
This is done by extending our approach to test how well particular components in transformer blocks can be linearly approximated. 


\paragraph{Method.}

Consider \gpt{} for definiteness, then:
% we have 
$$ \texttt{b}_{\ell} = \texttt{b}_{\ell}^{\texttt{ffn}} \circ \texttt{b}_{\ell}^{\texttt{attn}}$$ 
% with
\begin{equation}\label{eq:attn} \texttt{b}^{\texttt{attn}}_{\ell} (H) = \texttt{attn}_{\ell} (\texttt{ln1}_{\ell} (H)) + H,\end{equation} 
where $\texttt{attn}_{\ell}$ is
%a multi-head self-attention
a MHSA
layer and \texttt{ln1} is a layer normalization (LN), and 
$$ \texttt{b}^{\texttt{ffn}}_{\ell} (H) = \texttt{ffn}_{\ell} (\texttt{ln2}_{\ell} (H)) + H,$$  
where $\texttt{ffn}_{\ell}$ is
%a feed-forward network
an FFN
layer and $\texttt{ln2}$ is a LN.
\quash{
Given a block $\texttt{b}_\ell$ and one of its sub-modules $\texttt{ln1}_\ell, \ \texttt{attn}_\ell, \ \texttt{ln2}_\ell$, or $\texttt{ffn}_\ell$, we fit linear regression approximating the output of the sub-module given its input and then use it in order to define mappings, as we now describe.
}
Given a block $\texttt{b}_\ell$ and one of its sub-modules $\texttt{ln1}_\ell, \ \texttt{attn}_\ell, \ \texttt{ln2}_\ell$, or $\texttt{ffn}_\ell$, we fit linear regression approximating the output of the sub-module given its input, and then use it to define mappings $\matattnl{}$, $\matlnl{}$ and $\matffl{}$.
%We provide the definition of $\matattnl{}$ below, and that of the other two in App. \ref{sec:app_submodule_skip_description}.
We provide the formal definitions of these mappings in App. \ref{sec:app_submodule_skip_description}.
\iffalse
\paragraph{$\matattnl{}$.}
%Illustrating this on $\texttt{attn}_\ell$ for definiteness,
For an input $s$, let $v^\ell_{i_s}$ be the vector at position $i_s$ in the output of $\texttt{attn}_\ell (\texttt{ln1}_\ell (H^{\ell - 1}))$. We denote by $A_\ell^{\texttt{attn}} \in \mathbb{R}^{d_h \times d_h}$ the matrix numerically minimizing 
$$ A \mapsto \sum_{s \in \mathcal{T}} || A \cdot \texttt{ln1}_\ell (h^{\ell-1}_{i_s}) - v^\ell_{i_s}||^2,$$
and define an attention sub-module replacement (Eq.~\ref{eq:attn}) by $$
\texttt{b}^{\overline{\texttt{attn}}}_\ell (h) \coloneqq A_{\ell}^{\texttt{attn}} \cdot \texttt{ln1}_\ell (h) + h. $$
We then define a mapping between two layers ${\ell \rightarrow \ell'}$ by:
$$ \matattnl{} (h) \coloneqq $$
$$ \texttt{b}^{\texttt{ffn}}_{\ell'} ( \texttt{b}^{\overline{\texttt{attn}}}_{\ell'} ( \ldots (\texttt{b}^{\texttt{ffn}}_{\ell+1} ( \texttt{b}^{\overline{\texttt{attn}}}_{\ell+1} (h)))\ldots)).$$ 
Namely, when applying each $\ell''$-th block, $\ell < \ell'' \leq \ell'$, we replace its attention sub-module $\texttt{attn}_{\ell''}$ by its linear approximation.
%In an analogous way, we consider the mappings $\matffl{}$ and $\matlnl{}$, where in the latter we perform the linear shortcut both for \texttt{ln1} and for \texttt{ln2} (see~\S\ref{sec:app_submodule_skip_description} for precise descriptions).
Importantly, unlike the original attention module, the approximation $\texttt{b}^{\overline{\texttt{attn}}}_\ell$ operates on each position independently, and therefore applying $\matattnl{}$ disables any contextualization between the layers $\ell$ and $\ell'$. Note that this is not the case for $\matffl{}$ and $\matlnl{}$, which retain the self-attention sub-modules and operate contextually.
\fi

\paragraph{Evaluation.}


We analyze the $24$-layered \gpt{}, and proceed completely analogously to \S\ref{subsec:next_token_prediction_task}, evaluating the Precision@$1$ and Surprisal metrics for the mappings $\matattnlL{}$, $\matfflL{}$ and $\matlnlL{}$.

\begin{figure}[t]
\setlength{\belowcaptionskip}{-0pt}
\centering
%\includegraphics[scale=0.2]
\includegraphics[width=\columnwidth]{figs/parts_presurp_24.pdf}
\caption{Precision@$1$ and Surprisal for the various sub-module linear mappings, and $\matlL{}$ for comparison ($24$-layer \gpt{} next token prediction task). A 95\% confidence interval surrounds the Surprisal lines.}
\label{fig:parts_presurp}
\end{figure}

\quash{
\begin{figure}[t]
\centering
\includegraphics[scale=0.4]{figs/parts_pre1_24.pdf}
\caption{Precision@$1$ for the various sub-module linear shortcut mappings, and the mapping $\matlL{}$ for comparison (\gpt{} next token prediction task).}
\label{fig:parts_pre1}
\end{figure}

\begin{figure}[t]
\centering
\includegraphics[scale=0.35]{figs/parts_surp_24.pdf}
\caption{Surprisal for the various sub-module linear shortcut mappings, and the mapping $\matlL{}$ for comparison (\gpt{} next token prediction task). A 95\% confidence interval surrounds the lines.}
\label{fig:parts_surp}
\end{figure}
}

\paragraph{Results.}
Fig.~\ref{fig:parts_presurp} shows the average Precision@$1$ and Surprisal scores per layer.
From a certain layer (\textasciitilde$7$), all sub-module mappings achieve better results than the full-block mapping $\matlL{}$. Thus, it is not just the cumulative effect of all the sub-modules in the transformer block that is amenable to linear approximation, but also individual sub-modules can be linearly approximated. 
Furthermore, the linear approximation of attention sub-modules is less harmful than that of the FFN or LN sub-modules. 
% Hypothetically, 
A possible reason is that the linear replacement of FFN or LN ``erodes'' the self-attention computation after a few layers. 
Moreover, the good performance of $\matattnlL{}$ suggests that contextualization often exhausts itself in early layers; speculatively, it is only in more delicate cases that the self-attention of late layers adds important information. Last, remark the sharp ascent of the scores for layer normalization in layers $5$-$8$, for which we do not currently see a particular reason. To conclude, we see that the possibility of linear approximation permeates
%the various
transformer components.


\section{Related Work}

Recently, there was a lot of interest in utilizing intermediate representations in transformer-based LMs, both for interpretability and for efficiency.

In the direction of interpretability, one seeks to understand the prediction construction process of the model \cite{tenney-etal-2019-bert, voita-etal-2019-bottom}.

More recent works use mechanistic interpretability and view the inference pass as a residual stream of information \cite{dar2022analyzing,geva-etal-2022-transformer}. Additionally, there are works on probing, attempting to understand what features are stored in the hidden representations \cite{adi2017finegrained, conneau-etal-2018-cram,liu-etal-2019-linguistic}. Our work is different in that it attempts to convert intermediate representations into a final-layer form, which is interpretable by design.

In the direction of efficiency, there is the thread of work on early exit, where computation is cut at a dynamically-decided earlier stage \cite{schwartz-etal-2020-right,xin-etal-2020-deebert,schuster2022confident}. Other works utilize a fixed early stage network to parallelize inference \citep{leviathan2022fast, chen2023accelerating}. However, intermediate representations are directly propagated in these works, which we show is substantially worse than our approach. Moreover, our method requires training considerably less parameters than methods such as \citet{schuster-etal-2021-consistent}, that learn a different output softmax for each intermediate layer.  

More broadly, skipping transformer layers and analyzing the linearity properties of transformer components have been discussed in prior works \cite{Zhao2021of,mickus-etal-2022-dissect,wang-etal-2022-skipbert,lamparth2023analyzing}.


\section{Conclusion and Future Work}

We present a simple and effective method for enhancing utilization of hidden representations in transformer-based LMs, that uses 
pre-fitted context-free and token-uniform linear mappings.
Through a series of experiments on different data sources, model architectures and scales, we show that our method consistently outperforms the prevalent practice of interpreting representations in the final-layer space of the model, yielding better approximations of succeeding representations and the predictions they induce, thus allowing a more faithful interpretation of the model's prediction-formation.
We demonstrate the practicality of our method for improving computation efficiency, saving a substantial amount of compute on top of prominent early exiting approaches. 
Also, by extending our method to sub-modules, 
% more specifically the attention sub-modules, 
we observe that replacing a part of the transformer inference by a non-contextual linear computation often results in a small deterioration of the prediction.
This opens new research directions for improving model efficiency,
% and parallelizability.
% including breaking the computation into several parallelizable tasks.
including breaking the computation into parallel tasks.

\section*{Limitations}

Although we see in this work that there is more linear structure to transformer inference than could be explained solely by the residual connection, we do not elucidate a reason for that. We also do not try to formulate formal criteria according to which to judge, in principle, the quality of ways of short-cutting transformer inference in-between layers. In addition, our experiments cover only English data.


%\section*{Ethics Statement}
%Scientific work published at ACL 2023 must comply with the ACL Ethics Policy.\footnote{\url{https://www.aclweb.org/portal/content/acl-code-ethics}} We encourage all authors to include an explicit ethics statement on the broader impact of the work, or other ethical considerations after the conclusion but before the references. The ethics statement will not count toward the page limit (8 pages for long, 4 pages for short papers).

\section*{Acknowledgements}

We thank Tal Schuster for constructive comments.

% Entries for the entire Anthology, followed by custom entries
\bibliography{anthology,custom}
\bibliographystyle{acl_natbib}

\appendix

\section{Descriptions of $\matattn{}$, $\matff{}$ and $\matln{}$}
\label{sec:app_submodule_skip_description}

Here we detail the definitions of the mappings $\matattnl{}$, $\matffl{}$ and $\matlnl{}$ utilized in \S\ref{sec:submodules}.

\paragraph{Description of $\matattnl{}$.}
%Illustrating this on $\texttt{attn}_\ell$ for definiteness,
For an input $s$, let $v^\ell_{i_s}$ be the vector at position $i_s$ in the output of $\texttt{attn}_\ell (\texttt{ln1}_\ell (H^{\ell - 1}))$. We denote by $A_\ell^{\texttt{attn}} \in \mathbb{R}^{d_h \times d_h}$ the matrix numerically minimizing 
$$ A \mapsto \sum_{s \in \mathcal{T}} || A \cdot \texttt{ln1}_\ell (h^{\ell-1}_{i_s}) - v^\ell_{i_s}||^2,$$
and define an attention sub-module replacement (Eq.~\ref{eq:attn}) by $$
\texttt{b}^{\overline{\texttt{attn}}}_\ell (h) \coloneqq A_{\ell}^{\texttt{attn}} \cdot \texttt{ln1}_\ell (h) + h. $$
We then define a mapping between two layers ${\ell \rightarrow \ell'}$ by:
$$ \matattnl{} (h) \coloneqq $$
$$ \texttt{b}^{\texttt{ffn}}_{\ell'} ( \texttt{b}^{\overline{\texttt{attn}}}_{\ell'} ( \ldots (\texttt{b}^{\texttt{ffn}}_{\ell+1} ( \texttt{b}^{\overline{\texttt{attn}}}_{\ell+1} (h)))\ldots)).$$ 
Namely, when applying each $\ell''$-th block, $\ell < \ell'' \leq \ell'$, we replace its attention sub-module $\texttt{attn}_{\ell''}$ by its linear approximation.
%In an analogous way, we consider the mappings $\matffl{}$ and $\matlnl{}$, where in the latter we perform the linear shortcut both for \texttt{ln1} and for \texttt{ln2} (see~\S\ref{sec:app_submodule_skip_description} for precise descriptions).
Importantly, unlike the original attention module, the approximation $\texttt{b}^{\overline{\texttt{attn}}}_\ell$ operates on each position independently, and therefore applying $\matattnl{}$ disables any contextualization between the layers $\ell$ and $\ell'$. Note that this is not the case for $\matffl{}$ and $\matlnl{}$, which retain the self-attention sub-modules and operate contextually.

\paragraph{Description of $\matffl{}$.}
Let $v^\ell_{i_s}$ be the vector at position $i_s$ in the output of $\texttt{ln2}_{\ell} (\texttt{b}_\ell^{\texttt{attn}} (H^{\ell - 1}))$, for a given input $s$. We denote by $A_\ell^{\texttt{ffn}} \in \mathbb{R}^{d_h \times d_h}$ the matrix numerically minimizing 
$$ A \mapsto \sum_{s \in \mathcal{T}} || A \cdot v^{\ell}_{i_s} - \texttt{ffn}_{\ell} (v^\ell_{i_s})||^2,$$
and define a replacement of the feed-forward sub-module $\texttt{b}_{\ell}^{\texttt{ffn}}$ by $$ \texttt{b}^{\overline{\texttt{ffn}}}_\ell (H) \coloneqq A_{\ell}^{\texttt{ffn}} \cdot \texttt{ln2}_\ell (H) + H.$$
We then define a mapping between two layers ${\ell \rightarrow \ell'}$ by:
$$ \matffl{} (H) \coloneqq $$
$$ \texttt{b}^{\overline{\texttt{ffn}}}_{\ell'} ( \texttt{b}^{\texttt{attn}}_{\ell'} ( \ldots (\texttt{b}^{\overline{\texttt{ffn}}}_{\ell+1} ( \texttt{b}^{\texttt{attn}}_{\ell+1} (H))\ldots)).$$

\paragraph{Description of $\matlnl{}$.}
Let $v^\ell_{i_s}$ be the vector at position $i_s$ in the output of $\texttt{b}^{\texttt{attn}}_{\ell} (H^{\ell - 1})$, for a given input $s$. We denote by $A_\ell^{\texttt{ln1}} \in \mathbb{R}^{d_h \times d_h}$ the matrix numerically minimizing 
$$ A \mapsto \sum_{s \in \mathcal{T}} || A \cdot h^{\ell}_{i_s} - \texttt{ln1}_{\ell} (h^\ell_{i_s})||^2$$ and we denote by $A_\ell^{\texttt{ln2}} \in \mathbb{R}^{d_h \times d_h}$ the matrix numerically minimizing $$ A \mapsto \sum_{s \in \mathcal{T}} || A \cdot v^{\ell}_{i_s} - \texttt{ln2}_{\ell} (v^\ell_{i_s})||^2.$$ We define a replacement of the block $\texttt{b}^{\texttt{attn}}_{\ell}$ by \begin{equation} \texttt{b}^{\overline{\texttt{ln1}}}_\ell (H) \coloneqq \texttt{attn}_{\ell} (A_{\ell}^{\texttt{ln1}} \cdot H) + H\end{equation} and we define a replacement of the block $\texttt{b}^{\texttt{ffn}}_{\ell}$ by \begin{equation} \texttt{b}^{\overline{\texttt{ln2}}}_\ell (H) \coloneqq \texttt{ffn}_{\ell} (A_{\ell}^{\texttt{ln2}} \cdot H) + H.\end{equation}
We then define a mapping between two layers ${\ell \rightarrow \ell'}$ by:
$$ \matlnl{} (H) \coloneqq $$
$$ \texttt{b}^{\overline{\texttt{ln2}}}_{\ell'} ( \texttt{b}^{\overline{\texttt{ln1}}}_{\ell'} ( \ldots (\texttt{b}^{\overline{\texttt{ln2}}}_{\ell+1} ( \texttt{b}^{\overline{\texttt{ln1}}}_{\ell+1} (H))\ldots)).$$


\end{document}

\bibliographystyle{IEEEtran}
%\bibliography{reference}
}

\clearpage
\section{Appendix}

% \begin{multicols}{2}
% \subsection{Detailed Architectures}
% The detailed architectures of our STViT-DeiT and STViT-Swin are shown in Table \ref{arch deit} and Table \ref{arch swin}, where an input image size $224\times 224$ is assumed for all the networks and the default numbers of semantic tokens are 16 and 36 separately. ``Tr(ST)" denotes the transformers processing semantic tokens.

% \begin{table*} [t]
% \small
% \begin{center}
% \small
% \begin{tabular}{m|m|m|m|m}
% \toprule
% & Output size  & STViT-DeiT-T & STViT-DeiT-S & STViT-DeiT-B \\
% \midrule
% \midrule
% \multirow{2}{*}{Base}  & \multirow{2}{*}{14\times 14} & Patch Embedding & Patch Embedding & Patch Embedding \\
% % \cline{3-5}
% & &
% \begin{bmatrix}
% dim\ 192, head\ 3  \\
% \end{bmatrix} $\times 4$
% & 
% \begin{bmatrix}
% dim\ 384, head\ 6  \\
% \end{bmatrix} $\times 4$
% & 
% \begin{bmatrix}
% dim\ 768, head\ 12  \\
% \end{bmatrix} $\times 4$
% \\
% \midrule

% STGM & \multirow{1}{*}{4\times 4}
% &
% \begin{bmatrix}
% dim\ 192, head\ 3  \\
% \end{bmatrix} $\times 2$
% & 
% \begin{bmatrix}
% dim\ 384, head\ 6  \\
% \end{bmatrix} $\times 2$
% & 
% \begin{bmatrix}
% dim\ 768, head\ 12  \\
% \end{bmatrix} $\times 2$
% \\
% \midrule
% Tr(ST) & \multirow{1}{*}{4\times 4}
% &
% \begin{bmatrix}
% dim\ 192, head\ 3  \\
% \end{bmatrix} $\times 6$
% & 
% \begin{bmatrix}
% dim\ 384, head\ 6  \\
% \end{bmatrix} $\times 6$
% & 
% \begin{bmatrix}
% dim\ 768, head\ 12  \\
% \end{bmatrix} $\times 6$
% \\
% \bottomrule
% \end{tabular}
% \vspace{-3mm}
% \end{center}
% \caption{Detailed architecture of STViT-DeiT.}
% \label{arch deit}
% \end{table*}


% \begin{table*} [t]
% \small
% \begin{center}
% %\resizebox{!}{3.28cm}{
% % \small
% \begin{tabular}{m|m|m|m|m|m}
% \toprule
% & & Output size  & STViT-Swin-T & STViT-Swin-S & STViT-Swin-B \\
% \midrule
% \midrule
% \multirow{9}{*}{Base} & \multirow{3}{*}{Stage 1} & \multirow{3}{*}{56\times 56} & Patch Embedding & Patch Embedding & Patch Embedding \\
% % \cline{4-6}
% & & &
% \begin{bmatrix}
% win.\ sz.\ 7\times 7,     \\
% dim\ 96, head\ 3  \\
% \end{bmatrix} $\times 2$
% & 
% \begin{bmatrix}
% win.\ sz.\ 7\times 7,     \\
% dim\ 96, head\ 3  \\
% \end{bmatrix} $\times 2$
% & 
% \begin{bmatrix}
% win.\ sz.\ 7\times 7,     \\
% dim\ 192, head\ 6  \\
% \end{bmatrix} $\times 2$
% \\
% \cline{2-6}
% & \multirow{3}{*}{Stage 2} & \multirow{3}{*}{28\times 28} & Patch Merging & Patch Merging & Patch Merging \\
% % \cline{4-6}
% %\midrule
% & & &
% \begin{bmatrix}
% win.\ sz.\ 7\times 7,     \\
% dim\ 192, head\ 6  \\
% \end{bmatrix} $\times 2$
% & 
% \begin{bmatrix}
% win.\ sz.\ 7\times 7,     \\
% dim\ 192, head\ 6  \\
% \end{bmatrix} $\times 2$
% & 
% \begin{bmatrix}
% win.\ sz.\ 7\times 7,     \\
% dim\ 256, head\ 8  \\
% \end{bmatrix} $\times 2$
% \\
% \cline{2-6}
% & \multirow{3}{*}{Stage 3} & \multirow{3}{*}{14\times 14} & Patch Merging & Patch Merging & Patch Merging \\
% % \cline{4-6}
% & & &
% \begin{bmatrix}
% win.\ sz.\ 7\times 7,     \\
% dim\ 384, head\ 12  \\
% \end{bmatrix} $\times 2$
% & 
% \begin{bmatrix}
% win.\ sz.\ 7\times 7,     \\
% dim\ 384, head\ 12  \\
% \end{bmatrix} $\times 10$
% & 
% \begin{bmatrix}
% win.\ sz.\ 7\times 7,     \\
% dim\ 512, head\ 16  \\
% \end{bmatrix} $\times 10$
% \\
% \midrule

% \multirow{1}{*}{STGM} & \multirow{1}{*}{Stage 3} & \multirow{1}{*}{6\times 6}
% &
% \begin{bmatrix}
% win.\ sz.\ 3\times 3,     \\
% dim\ 384, head\ 12  
% \end{bmatrix} $\times 2$
% & 
% \begin{bmatrix}
% win.\ sz.\ 3\times 3,     \\
% dim\ 384, head\ 12  
% \end{bmatrix} $\times 2$
% & 
% \begin{bmatrix}
% win.\ sz.\ 3\times 3,     \\
% dim\ 512, head\ 16  
% \end{bmatrix} $\times 2$
% \\
% \midrule
% \multirow{3}{*}{Tr(ST)} & \multirow{1}{*}{Stage 3} & \multirow{1}{*}{6\times 6}
% &
% \begin{bmatrix}
% win.\ sz.\ 3\times 3,     \\
% dim\  384, head\ 12  \\
% \end{bmatrix} $\times 2$
% & 
% \begin{bmatrix}
% win.\ sz.\ 3\times 3,     \\
% dim\ 384, head\ 12  \\
% \end{bmatrix} $\times 6$
% & 
% \begin{bmatrix}
% win.\ sz.\ 3\times 3,     \\
% dim\ 512, head\ 16  \\
% \end{bmatrix} $\times 6$
% \\
% \cline{2-6}
% & \multirow{3}{*}{Stage 4} & \multirow{3}{*}{6\times 6} & Linear Layer & Linear Layer & Linear Layer \\
% % \cline{4-6}
% & & &
% \begin{bmatrix}
% win.\ sz.\ 3\times 3,     \\
% dim\ 768, head\ 24  \\
% \end{bmatrix} $\times 2$
% & 
% \begin{bmatrix}
% win.\ sz.\ 3\times 3,     \\
% dim\ 768, head\ 24  \\
% \end{bmatrix} $\times 2$
% & 
% \begin{bmatrix}
% win.\ sz.\ 3\times 3,     \\
% dim\ 1025, head\ 32  \\
% \end{bmatrix} $\times 2$
% \\
% \bottomrule
% \end{tabular}
% %}
% \end{center}
% \caption{Detailed architecture of STViT-Swin.}
% \label{arch swin}
% % \vspace{}
% \end{table*}

\subsection{Computational complexity analysis}
\paragraph{Global vision transformer (DeiT).}
We define that the number of image tokens is N, the number of semantic tokens is M, and their dimension is C. The patch embedding layer is neglected. The computational complexity of a global transformer processing image tokens (IT) is:
\begin{equation}
    \begin{gathered}
    \Omega(MHA(IT)) = 4NC^2 + 2N^2C, \\
    \Omega(FFN(IT)) = 8NC^2.
    \end{gathered}
\end{equation}
The computational complexity of a global transformer processing semantic tokens (ST) is:
\begin{equation}
    \begin{gathered}
    \Omega(MHA(ST)) = 4MC^2 + 2M^2C, \\
    \Omega(FFN(ST)) = 8MC^2.
    \end{gathered}
\end{equation}
The relationships between computational complexity and token number in attention and FFN are quadratic and linear, respectively. Due to the $N\ll M$, our method significantly reduces the cost of transformers, especially the attention.
The computational complexity of STGM is:
\begin{equation}
    \begin{gathered}
    \Omega(STGM) = 2MC^2+2NC^2 + 2MNC.
    \end{gathered}
\end{equation}
In global vision transformers, our STGM is also an efficient module.
The computational complexity of whole DeiT and our STViT-DeiT are:
% STGM (each transformer): $\Omega(STGM) = 2MC^2+2NC^2 + 2MNC$
\begin{equation}
    \begin{gathered}
    \Omega(DeiT)=144NC^2 + 24N^2C, \\
    \Omega(STViT)=52NC^2+12M^2C+76MC^2\\+8N^2C+4MNC.
\end{gathered}
\end{equation}

\begin{table*} [t]
\small
\begin{center}
\small
\begin{tabular}{c|c|c|ccc}
\toprule
\multirow{2}{4em}{Model} & \multirow{2}{4em}{Metrics} & \multirow{2}{2em}{Base} & \multicolumn{3}{c}{No. of semantic tokens} \\
%\midrule
& & & 36  & 49 & 100\\
\midrule
\multirow{3}{*}{STViT-LV-ViT-S} & Top-1 Acc(\%) & 83.3 & 82.7(-0.6) & 82.8(-0.5) & 83.1(-0.2\%)\\
& FLOPs(G) & 6.6 & 3.69(-44\%) & 3.91(-41\%) & 4.62(-30\%)\\
& Throughput(img/s) & 1159 & 2073(+78\%) & 1933(+72\%) & 1592(+37\%)\\
\bottomrule
\end{tabular}
\vspace{-3mm}
\caption{Results of STViT on LV-ViT-S.}
\label{cls lvvit}
\end{center}
% \vspace{}
\end{table*}

\paragraph{Local vision transformer (Swin).}
We define that the number of image tokens (IT) is N, the number of image tokens in each window is W, the number of semantic tokens (ST) in each window is M, and their dimension is C. We only compute the computational complexity in each transformer.
The computational complexity of a local transformer processing image tokens (IT) is:
\begin{equation}
    \begin{gathered}
    \Omega(MHA(IT)) = 4NC^2 + 2W^2NC, \\
    \Omega(FFN(IT)) = 8NC^2.
    \end{gathered}
\end{equation}
The computational complexity of a local transformer processing image tokens (ST) is:
\begin{equation}
    \begin{gathered}
    \Omega(MHA(ST)) = 4(N/W)MC^2 + 2M^2(N/W)C, \\
    \Omega(FFN(ST)) = 8MC^2.
    \end{gathered}
\end{equation}
Swin makes the computational complexity linear to token number, while our method further reduces the computational complexity.
The computational complexity of STGM is:
\begin{equation}
    \begin{gathered}
    \Omega(STGM) = 2(N/W)MC^2+2NC^2 + 2(N/W)M^2C.
    \end{gathered}
\end{equation}

% \paragraph{Parameter.}
% Our STViT and STViT-R use existing transformer layers to achieve our algorithm without any additional parameterized networks. The only introduced parameters are from global initialized tokens G, which can be negligible.

\subsection{The results on LV-ViT}
\label{a2}
\paragraph{Setting.}
In LV-ViT~\cite{jiang2021all}, by default, the STGM employs the $6^{th}$ and $7^{th}$ transformer layers of LV-ViT-S (with 16 layers in total). We downsample the token labels to match the size of our semantic tokens.

\paragraph{Results.}
The main results are shown in Table \ref{cls lvvit}. Token labelling in LV-ViT is not friendly for our method. Token labelling emphasizes the importance of all the output tokens and advocates that each output token should be associated with an individual location-specific label~\cite{jiang2021all}, while our semantic tokens generated by clustering emphasize high-level semantic information. However, we still achieve good performance.
In Table \ref{sota2}, we compare our STViT with the state-of-the-art token sparsification method EViT~\cite{liang2022evit} on LV-ViT-S. Results indicate that our method outperforms it.


\subsection{Applications in semantic segmentation}
\paragraph{Settings.} ADE20K is a widely-used semantic segmentation dataset, including a broad range of 150 semantic classes. It has 25K images in total, with 20K for training, 2K for validation, and 3K for testing. UperNet in mmseg is utilized as our base framework. The $w_s$ is set to $3$. Models are trained for 240K iterations. All the other settings follow the Swin Transformer~\cite{liu2021swin}.
\vspace{-2mm}
\paragraph{Comparison to Swin Transformers.}
Table~\ref{seg} presents the results of STViT-R-Swin on semantic segmentation. With similar FLOPs reduction, the drop on mIoU is larger compared with those in object detection tasks, which shows that our method still has a gap on dense prediction compared to the full-token network.

We analyze the relatively poor performance from two views. First, the STGM strictly prunes more than 80\% tokens by attention, which remains the high-level semantic information but loses nearly all the detailed information. Semantic segmentation is a dense pixel-level classification task, and the semantic tokens are difficult to enhance the pixel-level representation. Second, our spatial pooling layer with large kernel size in STGM and self-attention layers can be regarded as low-frequency filters. STGM filters most high-frequency information, which is necessary for semantic segmentation.
% \paragraph{Additional experiments for semantic segmentation.} 
% We experiment STViT-R with 16 semantic tokens in each window for semantic segmentation. The result is show in Table \ref{ss}.

\subsection{Additional visualization}
\label{a3}
We visualize the attention map of the second attention layer in STGM in Figure \ref{app vis a}. The shape of attention map is $N_s\times (N_s + N_i)$, where $N_s=16$, and $N_i=196$. 
The results of the attention computation between semantic tokens $S^1$ (queries) and semantic tokens $S^1$ (keys) are shown in the most left 16 columns, and the rest columns show the computation between semantic tokens $S^1$ and image tokens $X$. The figure shows that the second semantic token highlights the region of semantic tokens, while other semantic tokens highlight the image tokens. Figure \ref{app vis b} visualizes the attention maps in the self-attention layers after STGM. The second semantic token is incorporated by the majority of semantic tokens. These phenomenons illustrate that the second semantic token focuses on more global semantic information, which further verifies our global cluster center initialization can guide the semantic tokens to extract global semantic information. The phenomenons in Figure \ref{app vis a} and Figure \ref{app vis b} nearly emerge in all the images.

Neglecting the most left 16 columns of Figure \ref{app vis a}, we reshape it into 16 $14\times 14$ attention maps like Figure \ref{vis} and show them in Figure \ref{app vis c}. Thanks to the clustering of second attention and global initialization G, we can see that the semantic information is more accurate and meaningful. 

We visualize the attention maps of semantic tokens with single global initialization in Figure \ref{app vis d}. Without spatial initialization, the response regions are more global and similar. In contrast, our semantics of each semantic token are associated with the specific spatial location, which is the basis to allow our method to be applied in local self-attention and downstream tasks. Additionally, our attention maps contain more recognized and diverse semantic information, reflecting the effectiveness of our spatial initialization.

% To compare the semantic representation between our semantic tokens and original image tokens on DeiT, we uniformly sample 49 image tokens ($7\times 7$) from 196 image tokens ($14 \times 14$) in the $5^{th}$ transformer layer of DeiT-S and show their attention response in Figure ~\ref{image token}. In the same depth, our semantic tokens in Figure ~\ref{vis} capture high-level semantic information with completed foreground and background, while the semantic information of image tokens in Figure ~\ref{image token} is redundant and unclear. High-level semantic information captured by STGM helps our models achieve similar performance with a few tokens.

\begin{table}[t]
\small
\begin{center}
\begin{tabular}{cccc}
            \toprule
            Method & Top-1 Acc & FLOPs(G) \\
            \midrule
            % IA-RED$^2$~\cite{pan2021ia} & 79.1 & 3.2(-30\%) & -0.7\\
            % PS-ViT~\cite{tang2021patch} & 79.4 & 2.6(-43\%) & -0.4 \\
            % TokenLearner~\cite{ryoo2021tokenlearner} & 76.1 & 1.9(-44\%) & -1.8\\
            % Evo-ViT~\cite{xu2022evovit} & 79.4 & 3.0(-35\%) & -0.4 \\
            EViT~\cite{liang2022evit} & 82.5(-0.8) & 3.9(-41\%)\\
            EViT~\cite{liang2022evit} & 83.0(-0.3) & 4.7(-29\%)\\
            \textbf{STViT(Ours)} & 82.7(-0.6) & 3.7(-44\%)\\
            \textbf{STViT(Ours)} & 83.1(-0.2) & 4.6(-30\%)\\
            \bottomrule
        \end{tabular}
\vspace{-3mm}
\end{center}
\caption{Comparisons with the state-of-the-art token sparsification method EViT on LV-ViT-S.\vspace{-3mm}}
\label{sota2}
\end{table}

\subsection{Additional ablation study}
All the following experiments of STViT and STViT-R are conducted on DeiT-S and Swin-S unless otherwise specified, respectively. 
% \paragraph{The details of patch embedding layer in DeiT.} 
% We will introduce more implementation details in this section. 


\begin{figure*}[t]
\begin{subfigure}{1.\textwidth}
\includegraphics[width=1.0\linewidth]{0_5_2.jpeg}
\centering
\caption{The attention map ($16\times 216$) of the second attention layer in STGM.}
\label{app vis a}
\end{subfigure}

\begin{subfigure}{0.9\textwidth}
\includegraphics[width=1.0\linewidth]{vis2.png}
\centering
\caption{The attention maps ($14\times 14$) of 16 semantic tokens in the second attention layer of STGM.}
\label{app vis c}
\end{subfigure}

\begin{subfigure}{0.45\textwidth}
\centering
\includegraphics[width=1.0\linewidth]{app_attn.png}

\caption{Some examples of attention maps ($16\times 16$) of the self-attention layers after STGM.}
\label{app vis b}
\end{subfigure}

\begin{subfigure}{1.\textwidth}
\includegraphics[width=0.9\linewidth]{global.png}
\centering
\caption{The attention maps ($14\times 14$) of 16 semantic tokens generated by single global initialization.}
\label{app vis d}
\end{subfigure}

\centering
\caption{Additional visualization of attention maps.
}
\label{app vis}
\end{figure*}

% \begin{figure*}[t]
% \includegraphics[width=1.\linewidth]{image token.png}
% \centering
% \caption{Attention response of 49 image tokens ($7\times 7$) uniformly sampled from 196 image tokens ($14 \times 14$) in the $5^{th}$ transformer layer of DeiT-S.\vspace{-5mm}}
% \label{image token}
% \end{figure*}

\paragraph{The position of STGM.}
The effects of different position of STGM are shown in Table \ref{pos of stgm}. Two transformer layers are employed in STGM in all the experiments. We can see that the performance achieves improvement with appropriately moving the STGM towards deep layers due to better features of image tokens.

\paragraph{Positional encoding.} We try to apply positional encoding to our semantic tokens. Table \ref{pos} shows comparisons of different positional encoding methods, including learned positional encoding, conditional positional encoding, and relative positional encoding~\cite{liu2021swin}. All the positional encoding methods do not work on DeiT-S and Swin-T, even though relative positional encoding improves Swin-T by 1.2\%. These experiments demonstrate that the interaction between our semantic tokens depends on high-level semantic information and nearly does not use position relationships.

\begin{table}[t]
\small
% \resizebox{!}{1.cm}{
\begin{tabular}{c|c|c|c}
\toprule
Method & Backbone & mIoU & FLOPs(G)\\
\midrule
UperNet & Swin-S & 49.3 & 49 \\
UperNet & STViT-R-Swin-S & 48.3 & 34(-31\%) \\
\midrule
UperNet & Swin-B & 49.7 & 87\\
UperNet & STViT-R-Swin-B & 48.9 & 60(-31\%) \\
\bottomrule
\end{tabular}
\caption{Results of semantic segmentation on the ADE20K val set. A multi-scale inference with resolution $[0.5, 0.75, 1.0, 1.25, 1.5, 1.75]\times$ is applied. FLOPs and latency are measured in backbones with resolution $512 \times 512$.}
\vspace{-2mm}
 \label{seg}
\end{table}

\begin{table} [t]
\small
\begin{center}
\small
\begin{tabular}{c|cccccc}
\toprule
Pos. & 3-5 & 4-6 & 5-7 & 7-9 & 8-10 & 10-12 \\
\midrule
Top-1 Acc(\%) & 79.3 & 79.8 & 79.8 & 80.3 & 80.3 & 79.8\\
FLOPs(G) & 1.56 & 1.91 & 2.25 & 2.95 & 3.30 & 4.00\\
\bottomrule
\end{tabular}
\vspace{-5mm}
\end{center}
\caption{Performance evaluation on the different positions of our STGM.}
\label{pos of stgm}
\end{table}


\begin{table}[t]
\small
\begin{tabular}{c|c|c}
\toprule
 & STViT-DeiT-S Acc & STViT-Swin-T Acc \\
\midrule
%Absolute & &  \\
Learned & 79.6 & 81.5  \\
Conditional & 79.7 & 81.4 \\
Relative & 79.8 & 81.3 \\
No pos. & 79.8 & 81.5 \\
\bottomrule
\end{tabular}
\vspace{-2mm}
\caption{Performance evaluation on different positional encoding methods. \textit{Learned}, \textit{Conditional}, and \textit{Relative} indicate learned positional encoding, conditional positional encoding, and relative positional encoding, respectively.}
\label{pos}
\end{table}
\vspace{-5mm}

\paragraph{Alternative schemes of spatial pooling.}
We use an intra and inter-window spatial pooling in STGM to generate initial cluster centers, which adaptively save meaningful semantic information and avoids overlap between adjacent windows as much as possible. Furthermore, we explore more spatial pooling schemes, including: (i) spatial pooling with large-size kernel and overlap, (ii) multi-scale spatial pooling, and (iii) adaptive spatial pooling. We adopt 25 semantic tokens in these experiments. In (i), the kernel size and overlap are set to 6 and 4, respectively. In (ii), we use two adaptive pooling layers which produce 9 and 16 tokens separately. The results are presented in Table \ref{adaptive spatial pooling} on DeiT-T. We can see that overlap and multiple scales cannot boost the performance, which also demonstrates that discrete semantic tokens with high-level semantic information benefit our method.

\begin{table}[t]
\small
\resizebox{!}{0.55cm}{
\begin{tabular}{c|c|c|c|c}
\toprule
 & Scheme i & Scheme ii & Adaptive spatial pooling & Ours \\
\midrule
%Absolute & &  \\
Top-1 Acc(\%)  & 71.6 & 71.7 & 71.9 & 72.2 \\
\bottomrule
\end{tabular}}
\vspace{-2mm}
\caption{Alternative schemes of spatial pooling.}
\label{adaptive spatial pooling}
\end{table}

% \begin{table}[H]
% \caption{Results of our STViT-R with 16 semantic tokens in each window for semantic segmentation on the ADE20K val set.}
% \label{ss}
% \begin{tabular}{c|c|c|c}
% \toprule
% Method & Backbone & mIoU & FLOPs(G)\\
% \midrule
% UperNet & Swin-S & 49.3 & 49 \\
% UperNet & STViT-R-Swin-S & 48.4 & 40(-20\%)\\
% \midrule
% UperNet & Swin-B & 49.7 & 87 \\
% UperNet & STViT-R-Swin-B & 48.9 & 70(-20\%) \\
% \bottomrule
% \end{tabular}
% \end{table}

% \end{multicols}
\onecolumn
\subsection{Cluster center recovery by self attention}\label{justification}
We present an analysis showing how cluster centers are recovered through the attention mechanism. Let $K$ be the number of clusters. Let $\N(\mu_i, \sigma^2 I/d), i=1, \ldots, K$ be the $K$ Gaussian distributions, with center $\mu_i \in \R^d$ and covariance matrix $\sigma^2 I/d$. Let $x_{i,j} \in \R^d, j=1, \ldots, n,$ be the $n$ data points independently sampled from $\N(\mu_i, \sigma^2 I /d)$. Given data points $\D = \left\{x_{i,j}, i\in[K], j \in [n]\right\}$, of course without knowing the association of each data point to its underlying Gaussian distribution, our goal is to recover the underlying cluster centers $\mu_i, i \in [K]$. We assume that all the center vectors of Gaussian distributions are well separated, i.e. $\langle \mu_j, \mu_k \rangle \leq \gamma$ if $j \neq k$. For the convenience of study, we assume $|\mu_i| = 1, i \in [K]$.


Let $\mh_i \in \R^d, i \in [K]$ the initialized cluster centers, with all the cluster centers being well normalized. Define $\Delta$ as the gap for any initialized $\mh_i$ to the target cluster centers $\mu_i$ than to other clusters $\mu_j$, i.e. 
\[
    \Delta = \min\limits_{i \in [K]} \min\limits_{j \neq i} \langle \mh_i, \mu_i - \mu_j \rangle
\]
The new cluster centers are estimated through the self-attention mechanism, i.e. 
\[
    \mh_k' = \frac{1}{Z_k}\sum_{i=1}^K \sum_{j=1}^n \exp\left(\lambda\langle \mh_k, x_{i,j}\rangle \right)x_{i,j}
\]
where $\lambda > 0$ is a scaling factor and $Z_i$ is defined as
\[
    Z_k = \sum_{i=1}^K \sum_{j=1}^n \exp\left(\lambda\langle \mh_k, x_{i,j}\rangle \right)
\]
\begin{thm}
With sufficiently large $d$ and $n \gg d$, with a probability $1 - O(K/n^2)$, we have
\[
\frac{\langle \mu_k, \mh_k'\rangle}{|\mh_k'|} \geq 1 - O\left(\frac{\log K + \log d}{d\Delta}\right)
\]
\end{thm}
% \begin{proof}
Define $u_{i,j} = x_{i,j} - \mu_i$. We have
\[
\mh_k' = \frac{1}{Z_k}\sum_{i=1}^K \exp\left(\lambda\langle \mu_i, \mh_k \rangle\right)\left\{\left(\sum_{j=1}^n\exp\left(\lambda\langle \mh_k, u_{i,j} \rangle\right)\right)\mu_i + \sum_{j=1}^n \exp\left(\lambda\langle \mh_k, u_{i,j} \rangle\right)u_{i,j}\right\}
\]
We first bound $\sum_{j=1}^n \exp\left(\lambda \langle \mh_k, u_{i,j}\rangle \right)$. Since $u_{i,j} \sim \N(0, \sigma^2 I /d)$ and $|\mh_k| = 1$, we know that $\langle \mh_k, u_{i,j} \rangle \sim \N(0, \sigma^2/d)$. Hence, with a probability $1 - 2\delta$, we have
\[
\left|\sum_{j=1}^n \exp\left(\lambda \langle \mh_k, u_{i,j}\rangle \right) - n\E_{x\sim\N(0,\sigma^2/d)}\left[\exp(\lambda x)\right]\right| \leq 3\exp\left(\lambda\sigma\sqrt{\frac{2}{d}\log\frac{n}{\delta}}\right) + 2\sqrt{n\E_{x\sim\N(0,\sigma^2/d)}\left[\exp(2\lambda x)\right]\log\frac{2}{\delta}}
\]
Since
\[
\E_{x\sim\N(0,\sigma^2/d)}\left[\exp(\lambda x)\right] = \sqrt{\frac{d}{2\pi\sigma}}\int^{+\infty}_{-\infty}\exp\left(\lambda x - \frac{x^2 d}{2\sigma^2}\right) dx = \exp\left(\frac{\lambda^2\sigma^2}{2d}\right)
\]
we have, with a probability $1 - 2\delta$,
\[
\left|\sum_{j=1}^n \exp\left(\lambda \langle \mh_k, u_{i,j}\rangle \right) - n\exp\left(\frac{\lambda^2\sigma^2}{2d}\right)\right| \leq 3\exp\left(\lambda\sigma\sqrt{\frac{2}{d}\log\frac{n}{\delta}}\right) + 2\sqrt{n\exp\left(\frac{2\lambda^2\sigma^2}{d}\right)\log\frac{2}{\delta}}
\]
With large enough $n$, we have
\[
3\exp\left(\lambda\sigma\sqrt{\frac{2}{d}\log\frac{n}{\delta}}\right) + 2\sqrt{n\exp\left(\frac{2\lambda^2\sigma^2}{d}\right)\log\frac{2}{\delta}} \leq C\sqrt{n}\exp\left(\frac{\lambda^2\sigma^2}{2d}\right)
\]
and therefore
\[
    (1 - \tau)n\exp\left(\frac{\lambda^2\sigma^2}{2d}\right)\leq \sum_{j=1}^n \exp\left(\lambda\langle \mh_k, u_{i,j} \rangle\right) \leq (1 + \tau)n\exp\left(\frac{\lambda^2\sigma^2}{2d}\right)
\]
where
\[
    \tau \leq \frac{C}{\sqrt{n}}
\]
Here $C>0$ is a universal constant. 

We second bound $\sum_{j=1}^n \exp\left(\lambda\langle\mh_k, u_{i,j}\rangle\right) u_{i,j}$. We write each $u_{i,j} = u^{\perp}_{i,j} + u^{\parallel}_{i,j}$, where $u^{\parallel}_{i,j} = \langle u_{i,j}, \mh_k\rangle \mh_k$ and $u_{i,j}^{\perp}$ is a $d-1$ dimensional Gaussian vector. We have
\begin{eqnarray*}
\sum_{j=1}^n \exp\left(\lambda\langle\mh_k, u_{i,j}\rangle\right) u_{i,j} = \left(\sum_{j=1}^n \exp\left(\lambda\langle \mh_k, u_{i,j}\rangle\right)\langle \mh_k, u_{i,j}\rangle\right)\mh_k + \sum_{j=1}^n u^{\perp}_{i,j} 
\end{eqnarray*}
Since $u^{\perp}_{i,j}\sim\N\left(0, \sigma^2 I_{d-1}/d\right)$, we have $\sum_{j=1}^n u^{\perp}_{i,j} \sim \N\left(0, n\sigma^2 I_{d-1}/d\right)$. Using the concentration of $\chi^2_{d-1}$ distribution, we have, with a probability $1 - \delta$
\[
\left|\sum_{j=1}^n u^{\perp}_{i,j}\right|^2 \leq \frac{n\sigma^2}{d}\left(d-1 + 2\sqrt{(d-1)\log\frac{1}{\delta}} + 2\log\frac{1}{\delta}\right) \leq n\sigma^2\left(1 + 3\sqrt{\frac{\log(1/\delta)}{d}}\right)
\]
To bound $\sum_{j=1}^n \exp\left(\lambda\langle \mh_k, u_{i,j}\rangle\right)\langle \mh_k, u_{i,j}\rangle$, following the same procedure, we have, with a probability $1 - 2\delta$
\begin{eqnarray*}
\lefteqn{\left|\sum_{j=1}^n \exp\left(\lambda\langle \mh_k, u_{i,j}\rangle\right)\langle \mh_k, u_{i,j}\rangle - n\E_{x\sim\N(0,\sigma^2/d)}\left[\exp(\lambda x)x\right]\right|} \\
& \leq & 3\exp\left(\lambda\sigma\sqrt{\frac{2}{d}\log\frac{n}{\delta}}\right)\sigma\sqrt{\frac{2}{d}\log\frac{n}{\delta}} + \sqrt{n\E_{x\sim\N(0,\sigma^2/d)}\left[\exp(2\lambda x) x^2\right]\frac{2}{\delta}}
\end{eqnarray*}
Since
\[
\E_{x\sim\N(0,\sigma^2/d)}\left[\exp(\lambda x)x\right] = \sqrt{\frac{d}{2\pi\sigma^2}}\int\exp\left(\lambda x - \frac{x^2 d}{2\sigma^2}\right) x dx = \frac{\lambda\sigma^2}{d}\exp\left(\frac{\lambda^2\sigma^2}{2d}\right)
\]
and
\begin{eqnarray*}
\lefteqn{\E_{x\sim\N(0,\sigma^2/d)}\left[\exp(2\lambda x)x^2\right]} \\
& = & \sqrt{\frac{d}{2\pi\sigma^2}}\int\exp\left(2\lambda x - \frac{x^2 d}{2\sigma^2}\right) x^2 dx \\
& = & \sqrt{\frac{d}{2\pi\sigma^2}}\exp\left(\frac{2\lambda^2\sigma^2}{d}\right)\int\exp\left(\frac{ d}{2\sigma^2}\left[x - \frac{\lambda\sigma^2}{d}\right]^2\right) \left(\left[x - \frac{\lambda\sigma^2}{d}\right]^2 + 2\frac{\lambda\sigma^2}{d}\left[x - \frac{\lambda\sigma^2}{d}\right] + \frac{\lambda^2\sigma^4}{d^2}\right) dx \\
& = & \exp\left(\frac{2\lambda^2\sigma^2}{d}\right)\left(\frac{\lambda^2\sigma^4}{d^2} + \left(\frac{2\sigma^2}{d}\right)^{3/2}\E_{x\sim\N(0,1)}\left[x^2\right]\right) \\
& = & \exp\left(\frac{2\lambda^2\sigma^2}{d}\right)\left(\frac{\lambda^2\sigma^4}{d^2} + 2\left(\frac{2\sigma^2}{d}\right)^{3/2}\right) \leq \frac{2\lambda^2\sigma^4}{d^2}\exp\left(\frac{2\lambda^2\sigma^2}{d}\right)
\end{eqnarray*}
We thus have, with a probability $1 - 2\delta$, 
\begin{eqnarray*}
\lefteqn{\left|\sum_{j=1}^n \exp\left(\lambda\langle \mh_k, u_{i,j}\rangle\right)\langle \mh_k, u_{i,j}\rangle - \frac{n\lambda\sigma^2}{d}\exp\left(\frac{\lambda^2\sigma^2}{2d}\right)\right|} \\
& \leq & 3\exp\left(\lambda\sigma\sqrt{\frac{2}{d}\log\frac{n}{\delta}}\right)\sigma\sqrt{\frac{2}{d}\log\frac{n}{\delta}} + \sqrt{\frac{4n\lambda^2\sigma^4}{d^2}\exp\left(\frac{2\lambda^2\sigma^2}{d}\right)\log\frac{2}{\delta}}
\end{eqnarray*}
When $n$ is sufficiently large, we have, with a probability $1 - 2\delta$
\[
\left(1 - \tau\right)\frac{n\lambda\sigma^2}{d}\exp\left(\frac{\lambda^2\sigma^2}{2d}\right) \leq \sum_{j=1}^n \exp\left(\lambda\langle \mh_k, u_{i,j}\rangle\right)\langle \mh_k, u_{i,j}\rangle \leq \left(1 + \tau\right)\frac{n\lambda\sigma^2}{d}\exp\left(\frac{\lambda^2\sigma^2}{2d}\right)
\]
where $\tau \leq C/\sqrt{n}$. By putting them together, with a probability $1 - 4\delta$, we have
\[
\sum_{j=1}^n \exp\left(\lambda \langle \mh_k, u_{i,j}\rangle\right)u_{i,j} = n(1+\beta)\frac{n\lambda\sigma^2}{d}\exp\left(\frac{\lambda^2\sigma^2}{2d}\right)\mh_k + n\nu_i
\]
with $\beta \in [1-\tau, 1+\tau]$ and 
\[
    |\nu_i| \leq \frac{2\sigma}{\sqrt{n}}
\]
Finally, we have, with a probability $1 - 4K\delta$
\[
\mu'_k = \frac{n}{Z_k}\sum_{i=1}^K\exp\left(\lambda\langle \mu_i, \mh_k\rangle\right)\left((1 + \alpha_i)\exp\left(\frac{\lambda^2\sigma^2}{2d^2}\right)\mu_i + (1+\beta_i)\frac{\lambda\sigma^2}{d}\exp\left(\frac{\lambda^2\sigma^2}{2d}\right)\mh_k + \nu\right)
\]
Using the same analysis, we have, with a probability $1 - 4K\delta$,
\[
    \frac{Z_k}{n} = \exp\left(\frac{\lambda^2\sigma^2}{2d^2}\right)\sum_{i=1}^K\exp\left(\lambda\langle\mu_i, \mh_k\rangle\right)(1 + \alpha_i) 
\]
Now, we can bound $|\mh_k' - \mu_k|$. We have
\begin{eqnarray*}
\lefteqn{|\mu_k - \mh_k'|} \\
& \leq & \left|\frac{n}{Z_k}\exp\left(\lambda\langle \mu_k, \mh_k\rangle + \frac{\lambda^2\sigma^2}{2d^2}\right)(1+\alpha_k) - 1\right| + \frac{n}{Z_k}\sum_{i\neq k}\exp\left(\lambda\langle \mu_i, \mh_k\rangle + \frac{\lambda^2\sigma^2}{2d^2}\right)(1+\alpha_i) \\
&  & + \frac{n\lambda\sigma^2|\mu_k - \mh_k|}{Z_k d}\sum_{i=1}^K\exp\left(\lambda\langle \mu_i, \mh_k\rangle + \frac{\lambda^2\sigma^2}{2d^2}\right)(1+\beta_i) + \frac{n}{Z_k}\sum_{i=1}^K\exp\left(\lambda\langle \mu_i, \mh_k\rangle\right)\nu_i
\end{eqnarray*}
To further develop the bound for $|\mu_k - \mh_k'|$, we have
\[
Z_k \geq n\exp\left(\frac{\lambda^2\sigma^2}{2d^2} + \lambda\langle \mu_k, \mh_k\rangle\right)\left(1 - \frac{C}{\sqrt{n}}\right)
\]
and
\[
Z_k \leq  n\exp\left(\frac{\lambda^2\sigma^2}{2d^2} + \lambda\langle \mu_k, \mh_k\rangle\right)\left(1 + \frac{C}{\sqrt{n}}\right)\left(1 + (K-1)\exp\left(-\lambda\Delta\right)\right)
\]
We thus have
\begin{eqnarray*}
\lefteqn{|\mu_k - \mh_k'|} \\
& \leq & \left|\frac{\exp\left(\lambda\Delta\right)}{\exp\left(\lambda\Delta\right) + K - 1}\left(1 - \frac{2C}{\sqrt{n}}\right)^2 - 1\right| + \left(1 + \frac{2C}{\sqrt{n}}\right)^2\frac{K-1}{\exp(\lambda\Delta)}\left(1 + \frac{\lambda\sigma^2}{d}|\mu_k - \mh_k|\right) \\
&  & + \frac{\lambda\sigma^2}{d}|\mu_k - \mh_k| + \left(1 + \frac{2C}{\sqrt{n}}\right)\frac{\sigma}{\sqrt{n}}
\end{eqnarray*}
By choosing $\lambda = (\log d + \log K)/\Delta$, and by assuming $n$ is significantly larger than $d$, we have
\[
|\mu_k - \mh_k'| \leq O\left(\frac{\log d + \log K}{d\Delta}\right)
\]
implying that
\[
    \frac{\langle \mu_k, \mh'_k \rangle}{|\mh_k'|} \geq 1 - O\left(\frac{\log d + \log K}{d\Delta}\right)
\]
% \end{proof}

\end{document}
