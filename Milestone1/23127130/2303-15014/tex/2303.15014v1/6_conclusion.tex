\section{Conclusion}
In this paper, we introduced a novel unsupervised semantic segmentation method by discovering and leveraging two types of hidden positives, global hidden positive~(GHP) and local hidden positive~(LHP), to learn rich semantic information with local consistency. 
First, anchor-dependent GHP comprises task-agnostic and task-specific positive sets which are used to tailor the contrastive learning for the unsupervised semantic segmentation task.
Whereas the task-agnostic features are collected to guide the initial training, task-specific features are progressively engaged to learn the task-specific semantics information.
% Anchor-dependent GHP is constructed with distribution-guided pseudo-positives, to tailor contrastive learning for the unsupervised semantic segmentation task.
% To mitigate the dependence on the pretrained model, we further propose task-specific global hidden positive~(GHP) in addition to the task-agnostic GHP.
Moreover, under the inherent premise that the adjacent patches are likely to be semantically similar, we propagate the loss gradient to the surrounding patches in proportion to their attention scores.
This encourages the semantically similar peripheral patches to have the same objective as the anchor, resulting in semantic consistency between adjacent patches unless they belong to different objects.
%Extensive experiments show that 
Finally, our proposed method achieves new state-of-the-art results in various datasets.
%Through extensive experiments on various datasets, our method achieves the state-of-the-art performance.
%Through extensive experiments on various datasets, we achieve superior performances over the existing state-of-the-art methods.
%Through extensive experiments, we achieve superior performances over the existing state-of-the-art methods.

\vspace{5pt}
\noindent\textbf{Acknowledgements.} This work was supported in part by MSIT/IITP (No. 2022-0-00680, 2019-0-00421, 2020-0-01821, 2021-0-02068), and MSIT\&KNPA/KIPoT (Police Lab 2.0, No. 210121M06).
% , where the augmentation invariance in conventioncal contrastive loss does not provide enough semantic information.