\section{Ablation study on the architecture of TransNet}




Table~\ref{table:transnet_architecture} shows that our fully convolutional network (FCN)-based architecture performs the best compared to widely used fully-connected (FC) network architecture and Transformer~\cite{vaswani2017attention}.
We think this is because our FCN can explicitly utilize the spatial relationship between voxels using the 2.5D heatmap representation.
On the other hand, FC-based and Transformer-based settings take 2.5D coordinates as input.
This result is consistent with previous studies~\cite{moon2020i2l}, which demonstrates the superiority of heatmap representation over coordinate representation.
As proposing better network architecture for TransNet is not our main focus, we believe developing its network architecture can be an interesting future direction.
We used the architecture of Martinez~\etal~\cite{martinez2017simple} for the FC setting, and Zheng~\etal~\cite{zheng20213d} for the Transformer setting.

\begin{table}[t]
\small
\centering
\setlength\tabcolsep{1.0pt}
\def\arraystretch{1.1}
\begin{tabular}{C{3.5cm}|C{2.0cm}|C{2.0cm}}
\specialrule{.1em}{.05em}{.05em}
Settings & HIC~\cite{tzionas2016capturing} & IH2.6M~\cite{moon2020interhand2} \\ \hline
FC & 56.76 & 29.85 \\ 
Transformer & 64.49 & 33.59 \\
\textbf{FCN (Ours)} & \textbf{31.35} & \textbf{29.29} \\ 
\specialrule{.1em}{.05em}{.05em}
\end{tabular}
\vspace*{-3mm}
\caption{MRRPE comparisons between TransNet that have various architectures.}
\label{table:transnet_architecture}
\end{table}
