% CVPR 2023 Paper Template
% based on the CVPR template provided by Ming-Ming Cheng (https://github.com/MCG-NKU/CVPR_Template)
% modified and extended by Stefan Roth (stefan.roth@NOSPAMtu-darmstadt.de)

\documentclass[10pt,twocolumn,letterpaper]{article}

%%%%%%%%% PAPER TYPE  - PLEASE UPDATE FOR FINAL VERSION
%\usepackage[review]{cvpr}      % To produce the REVIEW version
%\usepackage{cvpr}              % To produce the CAMERA-READY version
\usepackage[pagenumbers]{cvpr} % To force page numbers, e.g. for an arXiv version

% Include other packages here, before hyperref.
\usepackage{graphicx}
\usepackage{amsmath}
\usepackage{amssymb}
\usepackage{booktabs}
\usepackage{siunitx}
\usepackage{cite}
\usepackage{ctable}
\usepackage{hhline}
\usepackage{booktabs} 
\usepackage{subcaption}
\usepackage{csquotes}
\usepackage{multirow}
\usepackage{pifont}
\usepackage[super]{nth}
\usepackage{wrapfig}
\usepackage{enumitem}

\definecolor{darkergreen}{RGB}{21, 152, 56}
\definecolor{red2}{RGB}{252, 54, 65}
\newcommand{\cmark}{\textcolor{darkergreen}{\ding{51}}}%
\newcommand{\xmark}{\textcolor{red2}{\ding{55}}}%

\newcolumntype{L}[1]{>{\raggedright\let\newline\\\arraybackslash\hspace{0pt}}m{#1}}
\newcolumntype{C}[1]{>{\centering\let\newline\\\arraybackslash\hspace{0pt}}m{#1}}
\newcolumntype{R}[1]{>{\raggedleft\let\newline\\\arraybackslash\hspace{0pt}}m{#1}}

% It is strongly recommended to use hyperref, especially for the review version.
% hyperref with option pagebackref eases the reviewers' job.
% Please disable hyperref *only* if you encounter grave issues, e.g. with the
% file validation for the camera-ready version.
%
% If you comment hyperref and then uncomment it, you should delete
% ReviewTempalte.aux before re-running LaTeX.
% (Or just hit 'q' on the first LaTeX run, let it finish, and you
%  should be clear).
\usepackage[pagebackref,breaklinks,colorlinks]{hyperref}


% Support for easy cross-referencing
\usepackage[capitalize]{cleveref}
\crefname{section}{Sec.}{Secs.}
\Crefname{section}{Section}{Sections}
\Crefname{table}{Table}{Tables}
\crefname{table}{Tab.}{Tabs.}

% Recommendation from CVPR2023
\usepackage[accsupp]{axessibility}  % Improves PDF readability for those with disabilities.

%%%%%%%%% PAPER ID  - PLEASE UPDATE
\def\confName{CVPR}
\def\confYear{2023}


\begin{document}

%%%%%%%%% TITLE - PLEASE UPDATE
\title{Bringing Inputs to Shared Domains for \\ 3D Interacting Hands Recovery in the Wild}

\author{Gyeongsik Moon\\
Meta Reality Labs \\
{\tt\small mks0601@gmail.com}
}
\maketitle

%%%%%%%%% ABSTRACT
\begin{abstract}
Despite recent achievements, existing 3D interacting hands recovery methods have shown results mainly on motion capture (MoCap) environments, not on in-the-wild (ITW) ones.
This is because collecting 3D interacting hands data in the wild is extremely challenging, even for the 2D data.
We present InterWild, which brings MoCap and ITW samples to shared domains for robust 3D interacting hands recovery in the wild with a limited amount of ITW 2D/3D interacting hands data.
3D interacting hands recovery consists of two sub-problems: 1) 3D recovery of each hand and 2) 3D relative translation recovery between two hands.
For the first sub-problem, we bring MoCap and ITW samples to a shared 2D scale space.
Although ITW datasets provide a limited amount of 2D/3D interacting hands, they contain large-scale 2D single hand data.
Motivated by this, we use a single hand image as an input for the first sub-problem regardless of whether two hands are interacting.
Hence, interacting hands of MoCap datasets are brought to the 2D scale space of single hands of ITW datasets.
For the second sub-problem, we bring MoCap and ITW samples to a shared appearance-invariant space.
Unlike the first sub-problem, 2D labels of ITW datasets are not helpful for the second sub-problem due to the 3D translation's ambiguity.
Hence, instead of relying on ITW samples, we amplify the generalizability of MoCap samples by taking only a geometric feature without an image as an input for the second sub-problem.
As the geometric feature is invariant to appearances, MoCap and ITW samples do not suffer from a huge appearance gap between the two datasets.
The code is publicly available\footnote{\url{https://github.com/facebookresearch/InterWild}}.
\end{abstract}

% main
\section{Introduction}
\label{sec:introduction}
% \begin{itemize}
%     % Diffusion of FL
%     \item {\st{Diffusion of FL}}
%     % Security threats to FL
%     \item {\st{Security threats to FL with particular focus on model poisoning}}
%     % Limitations of existing countermeasures
%     \item {\st{Current countermeasures (e.g., KRUM) and their limitations}}
%     % Proposed method and its advantages
%     \item {\st{Intuitive description of the proposed method and its difference (i.e., advantages) w.r.t. state of the art}}
%     % Main contributions
%     \item {\st{Summary of the main contributions of this work}}
%     % Paper's structure and organization
%     \item {\st{Paper's structure and organization}}
% \end{itemize}

% Diffusion of FL
Recently, {\em federated learning} (FL) has emerged as the leading paradigm for training distributed, large-scale, and privacy-preserving machine learning (ML) systems~\cite{mcmahan2017googleai,mcmahan2017aistats}. 
The core idea of FL is to allow multiple edge clients to collaboratively train a shared, global model without disclosing their local private training data.
%Specifically, an FL system consists of a central server and many edge clients; 
A typical FL round involves the following steps: {\em(i)} the server randomly picks some clients and sends them the current, global model; {\em(ii)} each selected client locally trains its model with its own private data; then, it sends the resulting local model to the server;\footnote{Whenever we refer to global/local model, we mean global/local model {\em parameters}.} {\em(iii)} the server updates the global model by computing an \emph{aggregation function}, usually the average (FedAvg), on the local models received from clients.
% \begin{enumerate}
%     \item[{\em(i)}] the server sends the current, global model to the clients and appoints some of them for training;
%     \item[{\em(ii)}] each selected client locally trains its copy of the global model with its own private data; then, it sends the resulting local model back to the server;\footnote{Whenever we refer to global/local model, we mean global/local model {\em parameters}.}
%     \item[{\em(iii)}] the server updates the global model by computing an \emph{aggregation function} on the local models received from clients (by default, the average, also referred to as FedAvg~\cite{mcmahan2017aistats}).
% \end{enumerate}
This process goes on until the global model converges. %(e.g., after a certain number of rounds or other similar stopping criteria).
%\\
% The advantages of FL over the traditional, centralized learning paradigm are undoubtedly clear in terms of flexibility/scalability (clients can join/disconnect from the FL network dynamically), network communications (only model weights\footnote{We will use \textit{parameters} and \textit{weights} interchangeably.} are exchanged between clients and server), and privacy (each client's private training data is kept local at the client's end and not uploaded to the server).
\\
% Security threats to FL
%However, the growing adoption of FL also raises security concerns~\cite{costa2022covert}, particularly about its confidentiality, integrity, and availability.
Although its advantages over standard ML, FL also raises security concerns~\cite{costa2022covert}. %, particularly about its confidentiality, integrity, and availability~\cite{costa2022covert}.
% OLD, LONG VERSION
% Indeed, some work deals with privacy leakage that may expose the local data of some clients~\cite{melis2019sp}. 
% A large body of work, instead, investigates attacks that usually aim to detriment the predictive accuracy of the learned global model. For instance, \emph{data poisoning} attacks achieve this goal by letting an adversary pollute the training set of some corrupt FL clients with maliciously crafted examples~\cite{jagielski2018sp}.
% Similarly, in \emph{model poisoning} the attacker attempts to tweak the global model weights~\cite{bhagoji2019pmlr} by directly perturbing the local model's weights of some infected FL clients before these are sent to the central server for aggregation, usually via so-called Byzantine attacks. 
% It turns out that Byzantine model poisoning attacks severely impact standard FedAvg; therefore, more robust aggregation functions must be designed to make FL systems secure.
Here, we focus on \emph{untargeted model poisoning} attacks~\cite{bhagoji2019pmlr}, where an adversary attempts to tweak the global model weights %\footnote{We will use the terms \textit{parameters} and \textit{weights} interchangeably.} 
by directly perturbing the local model's parameters of some infected clients before these are sent to the central server for aggregation.
In doing so, the adversary aims to jeopardize the global model \textit{indiscriminately} at inference time.
Such model poisoning attacks severely impact standard FedAvg; therefore, more robust aggregation functions must be designed to secure FL systems.
\\
% In this paper, we focus on designing a novel robust aggregation scheme at the server's end to contrast the effect of Byzantine model poisoning attacks.
%
% Current countermeasures and their limitations
%Several countermeasures have been proposed in the literature to combat model poisoning attacks on FL systems.
% Some methods use simple statistics more robust than plain average to smooth the impact of malicious updates (e.g., Trimmed Mean and FedMedian~\cite{yin2018icml}). 
% Other defenses implement outlier detection techniques to discard malicious updates from the aggregation performed at the server's end. Those are either based on heuristics (e.g., Krum/Multi-Krum~\cite{blanchard2017nips} and Bulyan~\cite{mhamdi2018pmlr}) or data-driven approaches (e.g., K-means clustering~\cite{shen2016acm} or DnC via spectral analysis~\cite{shejwalkar2021ndss}). 
% Finally, some strategies rely on a centralized ``source of trust'' to spot potential malicious updates (e.g., FLTrust~\cite{cao2020fltrust}).
% Several countermeasures have been proposed in the literature to combat model poisoning attacks on FL systems, i.e., to discard possible malicious local updates from the aggregation performed at the server's end. 
% These techniques range from simple statistics more robust than plain average (e.g., Trimmed Mean and FedMedian~\cite{yin2018icml}) to outlier detection heuristics (e.g., Krum/Multi-Krum~\cite{blanchard2017nips} and Bulyan~\cite{mhamdi2018pmlr}) or data-driven approaches (e.g., spectral analysis via K-means clustering~\cite{shen2016acm} or spectral analysis), or methods based on ``source of trust'' (e.g., FLTrust~\cite{cao2020fltrust}).
% OLD, LONG VERSION
%Several countermeasures have been proposed in the literature to combat Byzantine model poisoning attacks on FL systems.
% Descriptive statistics
% For example, Trimmed Mean and FedMedian aggregate local model updates using more robust statistics than standard average~\cite{yin2018icml}.
%
% % Heuristics for outlier detection
% Many existing Byzantine-resilient strategies implement some outlier detection heuristics to discard the model updates sent by potentially malicious clients from the input of the aggregation function.
% One of the most popular heuristics is Krum~\cite{blanchard2017nips}.
% This strategy tries to mitigate the impact of Byzantine attacks by selecting as a global model the local model with the smallest sum of Euclidean distances to {\em all} the other local models.
% Although powerful, Krum requires the server to know (or, at least, estimate) the number of malicious FL clients upfront, which is generally impossible in a realistic attack scenario. %
% Moreover, Krum may become ineffective for complex, high-dimensional model parameter spaces due to the curse of dimensionality.
% Bulyan~\cite{mhamdi2018pmlr} tries to overcome this issue by combining Krum with a variant of Trimmed Mean.
% % Data-driven outlier detection
% Other strategies use data-driven outlier detection techniques -- e.g., via K-means clustering~\cite{shen2016acm} -- to spot potential malicious local model updates. 
% %For instance, Shen et al. propose to cluster local model updates with K-means and thus identify outliers.
%
% % Other techniques
% As far as the server is concerned, any local model received can be from a potential malicious client. 
% FLTrust~\cite{cao2020fltrust} assumes the server acts as a client, i.e., trains a local model on an additional {\em trustworthy} dataset at the server's end and compares it against all the local models from other clients. 
% This way, the server can rely on some ``source of trust'' when discarding potentially malicious clients.
%\\
% Limitations of existing Byzantine-resilient strategies
Unfortunately, existing defense mechanisms either rely on simple heuristics (e.g., Trimmed Mean and FedMedian by~\cite{yin2018icml}) or need strong and unrealistic assumptions to work effectively (e.g., foreknowledge or estimation of the number of malicious clients in the FL system, as for Krum/Multi-Krum~\cite{blanchard2017nips} and Bulyan~\cite{mhamdi2018pmlr}, which, however, cannot exceed a fixed threshold).
Furthermore, outlier detection methods using K-means clustering~\cite{shen2016acm} or spectral analysis like DnC~\cite{shejwalkar2021ndss} do not directly consider the temporal evolution of local model updates received.
Finally, strategies like FLTrust~\cite{cao2020fltrust} require the server to collect its own dataset and act as a proper client, thereby altering the standard FL protocol.
\\
% OLD, LONG VERSION
% Overall, existing Byzantine-resilient strategies are either simple heuristics (e.g., FedMedian) or, if they are more complex, they rely on strong and unrealistic assumptions to work effectively (e.g., knowing the number of malicious clients in the FL system in advance, as for Krum and alike).
% Furthermore, data-driven outlier detection methods do not consider the temporary evolution of local model updates received (e.g., K-means clustering). 
% Finally, strategies like FLTrust requires the server to collect its own dataset and act as a proper client, thereby altering the standard FL protocol.
%
% Description of the proposed method
This work introduces a novel pre-aggregation \textit{filter} robust to untargeted model poisoning attacks. Notably, this filter $(i)$ operates without requiring prior knowledge or constraints on the number of malicious clients and $(ii)$ inherently integrates temporal dependencies. 
The FL server can employ this filter as a preprocessing step before applying \textit{any} aggregation function, be it standard like FedAvg or robust like Krum or Bulyan.
Specifically, we formulate the problem of identifying corrupted updates as a multidimensional (i.e., matrix-valued) time series anomaly detection task. 
The key idea is that legitimate local updates, resulting from well-calibrated iterative procedures like stochastic gradient descent (SGD) with an appropriate learning rate, show \textit{higher predictability} compared to malicious updates. This hypothesis stems from the fact that the sequence of gradients (thus, model parameters) observed during legitimate training exhibit regular patterns, as validated in Section~\ref{subsec:intuition}. %until convergence. 
%This regularity may be more pronounced for smooth convex loss functions, but it can still be captured within an appropriate time window, even for more complex and convoluted loss surfaces. 
%We provide evidence of this claim in Appendix~B, where we show that the average mutual information (i.e., ``predictability''), calculated over pairs of legitimate model updates sent at different FL rounds, is significantly higher than the corresponding computation for a malicious client.
\\
Inspired by the matrix autoregressive (MAR) framework for multidimensional time series forecasting~\cite{chen2021je}, we propose the FLANDERS ({\em \textbf{F}ederated \textbf{L}earning meets \textbf{AN}omaly \textbf{DE}tection for a \textbf{R}obust and \textbf{S}ecure}) filter.
The main advantages of FLANDERS over existing strategies like FLDetector~\cite{zhao2020multivariate} are its resilience to large-scale attacks, where $50\%$ or more FL participants are hostile, and the capability of working under realistic non-iid scenarios.
We attribute such a capability to two key factors: $(i)$ FLANDERS works without knowing a priori the ratio of corrupted clients, and $(ii)$ it embodies temporal dependencies between intra- and inter-client updates, quickly recognizing local model drifts caused by evil players. Below, we summarize our main contributions:

\begin{itemize}
\item[{\em(i)}]
We provide empirical evidence that the sequence of models sent by legitimate clients is more predictable than those of malicious participants performing untargeted model poisoning attacks.
\\
\item[{\em(ii)}] 
We introduce FLANDERS, the first pre-aggregation filter for FL robust to untargeted model poisoning based on multidimensional time series anomaly detection.
\\
\item[{\em(iii)}] 
We integrate FLANDERS into Flower,\footnote{\scriptsize{\url{https://flower.dev/}}} a popular FL simulation framework for reproducibility.
\\
\item[{\em(iv)}] 
We show that FLANDERS improves the robustness of the existing aggregation methods under multiple settings: different datasets, client's data distribution (non-iid), models, and attack scenarios.
\\
\item[{\em(v)}] 
We publicly release all the implementation code of FLANDERS along with our experiments.\footnote{\scriptsize{\url{https://anonymous.4open.science/r/flanders_exp-7EEB}}}
\end{itemize}

% Paper's structure and organization
The remainder of the paper is structured as follows. %some related work and the current state-of-the-art solutions to security issues that FL entails. 
Section~\ref{sec:background} covers background and preliminaries. 
In Section~\ref{sec:related}, we discuss related work.
Section~\ref{sec:problem} and Section~\ref{sec:method} describe the problem formulation and the method proposed. % to tackle it. 
Section~\ref{sec:experiments} gathers experimental results. %, and Section~\ref{sec:limitations} discusses some limitations of this work.
Finally, we conclude in Section~\ref{sec:conclusion}.
 %discusses the limitations of this work and draws future research directions.
%reports conclusions and draws perspectives for future research directions.

%%%%%%% OLD %%%%%%%
%to overcome the resilience of Byzantine failures in distributed Stochastic Gradient Descent computations. 
% The strength of Krum is its time complexity, which is linear in the gradient dimension. 
% However, the robustness of the approach is guaranteed for gradient-based learning applications only when the majority of the clients are not compromised. 
% Besides, the aggregation mechanism of Krum, as well as that of similar methods, is robust from a coarse-grained perspective and does not provide solutions to errors and perturbations that may occur at inference time.
%A related approach to~\cite{blanchard2017nips} is the work of Su et al.~\cite{su2016dc}. Here, the authors propose an iterated approximate agreement to tackle a multi-layer scenario attacked by Byzantine agents. 
%However, the method works efficiently on the sole discrete context and it is inapplicable to continuous state environments.
%\gabri{Maybe, we should just talk about the main limitations of existing countermeasures without digging into their details (or, we can just mention Krum as this is the most popular one). I will move the description of all these methods to the Related Work section.}
\setlength{\tabcolsep}{1.6mm}{
\renewcommand\arraystretch{1.1}
\begin{table}[ht]
  \centering
  \scalebox{0.9}{
  \begin{tabular}{llcccc}
    \toprule
    &\multirow{2}*{Methods} & \multirow{2}*{Sal.} &   \multicolumn{2}{c}{VOC} & MS~COCO \\
    \cmidrule(r){4-5}\cmidrule(r){6-6}
    &&&\texttt{val}&\texttt{test}&\texttt{val}\\
    \hline
    \multirow{13}*{\rotatebox{90}{ResNet-50}}
    &IRN~\cite{irn}          \tiny{CVPR'19}     &              & 63.5       & 64.8          & 42.0  \\
    &LayerCAM~\cite{layercam}\tiny{TIP'21}      &              & 63.0       & 64.5          & -     \\
    &AdvCAM~\cite{advcam}    \tiny{CVPR'21}     &              & 68.1       & 68.0          & 44.2  \\
    &RIB~\cite{rib}          \tiny{NeurIPS'21}  &              & 68.3       & 68.6          & 44.2  \\
    &ReCAM~\cite{recam}      \tiny{CVPR'22}     &              & 68.5       & 68.4          & 42.9  \\
    % \rowcolor{Gray}
    &\cellcolor{Gray}IRN+\texttt{LPCAM}    &\cellcolor{Gray} & \cellcolor{Gray}68.6    & \cellcolor{Gray}68.7      & \cellcolor{Gray}44.5  \\
    &SIPE~\cite{sipe}        \tiny{CVPR'22}     &              & 68.8       & 69.7          & 40.6  \\
    &OOD~\cite{ood}+Adv      \tiny{CVPR'22}     &              & 69.8       & 69.9          & -     \\
    &AMN~\cite{amn}          \tiny{CVPR'22}     &              & 69.5       & 69.6          & 44.7  \\
    &\cellcolor{Gray}AMN+\texttt{LPCAM}    &\cellcolor{Gray} & \cellcolor{Gray}70.1    &\cellcolor{Gray} 70.4      & \cellcolor{Gray}45.5  \\ 
    &ESOL~\cite{esol}        \tiny{NeurIPS'22}  &              & 69.9$^*$   & 69.3$^*$      & 42.6  \\
    &CLIMS~\cite{clims}      \tiny{CVPR'22}     &              & 70.4$^*$   & 70.0$^*$      & -     \\
    &EDAM~\cite{edam}        \tiny{CVPR'21}     &\checkmark    & 70.9$^*$   & 71.8$^*$      & -     \\
    &\cellcolor{Gray}EDAM+\texttt{LPCAM}  &\cellcolor{Gray}\checkmark & \cellcolor{Gray}71.8$^*$ &\cellcolor{Gray} 72.1$^*$& \cellcolor{Gray}42.1\\
    \hline
    \multirow{9}*{\rotatebox{90}{WideResNet-38}}
    &Spatial-BCE~\cite{sbce} \tiny{ECCV'22}     &              & 70.0       & 71.3      & 35.2  \\
    &BDM~\cite{bdm}          \tiny{ACMMM'22}    &\checkmark    & 71.0       & 71.0      & 36.7  \\ 
    &RCA~\cite{rca}+OOA      \tiny{CVPR'22}     &\checkmark    & 71.1       & 71.6      & 35.7  \\
    &RCA~\cite{rca}+EPS      \tiny{CVPR'22}     &\checkmark    & 72.2       & 72.8      & 36.8  \\
    &HGNN~\cite{hgnn}        \tiny{ACMMM'22}    &\checkmark         & 70.5$^*$   & 71.0$^*$  & 34.5  \\ 
    &EPS~\cite{eps}          \tiny{CVPR'21}     &\checkmark         & 70.9$^*$   & 70.8$^*$  & -     \\
    &RPIM~\cite{rpim}        \tiny{ACMMM'22}    &\checkmark         & 71.4$^*$   & 71.4$^*$  & -     \\ 
    &L2G~\cite{l2g}          \tiny{CVPR'22}     &\checkmark         & 72.1$^*$   & 71.7$^*$  & 44.2  \\
    \hline
    \multirow{2}*{\rotatebox{90}{\small{DeiT-S}}}
    &MCTformer~\cite{mctformer}    \tiny{CVPR'22}     &                 & 71.9$^{\dag}$  & 71.6$^{\dag}$   & 42.0  \\
    &\cellcolor{Gray}MCTformer+\texttt{LPCAM}      &\cellcolor{Gray} & \cellcolor{Gray}72.6$^{\dag}$  & \cellcolor{Gray}72.4$^{\dag}$  &\cellcolor{Gray} 42.8 \\
    \bottomrule
  \end{tabular}}
  \vspace{-2mm}
  \caption{The mIoU results (\%) based on DeepLabV2 on VOC and MS~COCO. The side column shows three backbones of multi-label classification model. ``Sal.'' denotes using saliency maps. * denotes the segmentation model is pre-trained on MS~COCO. $^\dag$ denotes the segmentation model is pre-trained on VOC.
  }
  \vspace{-6mm}
  \label{table_related}
\end{table}
}





\section{InterWild}
Fig.~\ref{fig:overall_pipeline} shows the overall pipeline of our InterWild, which consists of DetectNet, SHNet, and TransNet.
DetectNet detects hands from the input image.
Then, SHNet, a network for a single hand, takes each detected hand image as an input and outputs 3D mesh and 2.5D pose of each hand.
The 2.5D poses of the right and left hands are passed to TransNet, which outputs 3D relative translation between two hands.
The final 3D interacting hands are obtained by adding the 3D relative translation to the 3D mesh of the left hand.
DetectNet and SHNet follow architectures of Pose2Pose~\cite{moon2022hand4whole}.
Please refer to the supplementary material for their detailed architectures.

\begin{figure}[t]
\begin{center}
\includegraphics[width=0.7\linewidth]{fig/scale_dist_compare.pdf}
\end{center}
\vspace*{-7mm}
\caption{
Average of each hand's width and height, where the width and height are normalized with the size of the input image.
We extended all hands' boxes to set their aspect ratio to 1 before calculating the scales.
\textcolor[RGB]{242,174,114}{Yellow}: Each hand in the two-hand image of IH2.6M~\cite{moon2020interhand2} when hands are interacting (Previous approach).
\textcolor[RGB]{217,100,89}{Brown}: Each hand in the single-hand image of IH2.6M~\cite{moon2020interhand2} when hands are interacting (Ours).
\textcolor[RGB]{30,216,139}{Green}: Each hand in the single-hand image of MSCOCO~\cite{lin2014microsoft}. 
}
\vspace*{-3mm}
\label{fig:sh_ih_dist_compare}
\end{figure}


\subsection{SHNet}

\noindent\textbf{Input: an image of a single hand.}
SHNet takes a single-hand image regardless of whether two hands are interacting or not, while previous methods take images with two hands when hands are interacting, as shown in Fig.~\ref{fig:shnet_compare_prev}.
Hence, 2D scales of interacting hands are normalized to those of a single hand.
Fig.~\ref{fig:sh_ih_dist_compare} shows that when we crop images to contain two hands when hands are interacting (\textit{i.e.}, previous methods. \textcolor[RGB]{242,174,114}{Yellow} in the figure), 2D scales of each hand in two-hand images have a very different distribution compared to those of each hand in single-hand images (\textcolor[RGB]{30,216,139}{Green} in the figure).
On the other hand, when we crop images to contain a single hand regardless of whether hands are interacting (\textit{i.e.}, ours. \textcolor[RGB]{217,100,89}{Brown} in the figure), 2D scales of each hand in two-hand images have almost the same distribution compared to those of each hand in single-hand images (\textcolor[RGB]{30,216,139}{Green} in the figure).
Such an analysis justifies our design of SHNet to take a single cropped hand.


The single-hand image is cropped and resized from the high-resolution human image using predicted boxes from the DetectNet.
Before cropping the hands, we double the width and height of boxes to prevent hands from missing and provide more surrounding context to SHNet.
The left hand image is horizontally flipped to the right hand; therefore, the input image always represents a right hand image.
The right hand and flipped left hand images are concatenated in the batch dimension and processed in a parallel way by the SHNet.
By taking the right hand and flipped left hand images, SHNet can focus only on learning to process right hand images, which can relieve the burdens of learning to process both right and left hand images.
Also, such flipping is helpful when the two hands are severely interacting so that boxes of two hands are largely overlapped.
For example, let us imagine that most of the left hand is occluded by the right hand.
Then, images from the left and right hand boxes would contain almost the same right hand.
By flipping the left hand image, the right hand in the original left hand image changes to the left hand.
We train SHNet to ignore the left hand in the input image and produce a 3D hand mesh of only the right hand in the input image.
Therefore, the output from the flipped left hand image is a 3D hand mesh of the occluded right hand, which is originally the occluded left hand.
The effectiveness of this flipping is shown in the experimental section.


\noindent\textbf{Output: 3D mesh and 2.5D pose of each hand.}
Using the network architecture of Pose2Pose~\cite{moon2022hand4whole}, our SHNet outputs 3D mesh and 2.5D pose~\cite{sun2018integral} of each hand.
We flip back the outputs of the flipped left hand image.
We denote the 3D mesh of left and right hands by $\mathbf{M}_\text{l}$ and $\mathbf{M}_\text{r}$, respectively.
Each 3D mesh is obtained by forwarding the estimated pose and shape parameters to a MANO~\cite{romero2017embodied} layer.
We subtract 3D meshes from their 3D root joint locations so that the 3D meshes are in the root joint-relative space.
In addition, we denote the 2.5D pose of left and right hands by $\mathbf{P}_\text{l} \in \mathbb{R}^{J \times 3}$ and $\mathbf{P}_\text{r} \in \mathbb{R}^{J \times 3}$, respectively.
$J$ indicates the number of single-hand joints.
The 2.5D pose encodes hand joint locations in 2.5D space.
The $x$- and $y$-axis of the $j$th 2.5D pose represent pixel coordinates of the $j$th joint, where the pixel space is defined in the input image of SHNet (\textit{i.e.}, single-hand image).
The $z$-axis is defined in the root joint-relative depth space.


\subsection{TransNet}~\label{subsec:transnet}
Fig.~\ref{fig:transnet} shows the overall pipeline of TransNet, a network to predict 3D relative translation between two hands.


\noindent\textbf{Input: 2.5D poses of two hands.}
TransNet takes 2.5D poses of two hands, while previous methods take images with two hands, as shown in Fig.~\ref{fig:transnet_compare_prev}.
The 2.5D poses of two hands are from SHNet, which are denoted by $\mathbf{P}_\text{r}$ and $\mathbf{P}_\text{l}$.
Before forwarding them, we apply 2D affine transformations to $\mathbf{P}_\text{r}$ and $\mathbf{P}_\text{l}$, which transform the input space of SHNet (\textit{i.e.}, an image of a single hand) to a union of two-hand boxes space (\textit{i.e.}, an image of two hands).
By warping them to the union hand box space, we can get a relative 2D scale and translation between two hands in the 2D pixel space.
Based on such relative 2D information and pose information, TransNet predicts the 3D relative translation.


For example, when $xy$ distance of two hands' 2.5D pose is small, $(x,y)$ of the 3D relative translation are close to zero.
Also, when one hand takes smaller area in input $xy$ space, that hand might have larger depth; however, not always true as hands are deformable.
When a hand is in neutral pose and the other one is in fist pose, their 3D relative depth can be zero although the hand with fist pose takes smaller area.
Hence, pose is necessary to determine $z$ of the 3D relative translation.
Please note that the 2D affine transformations do not affect the depths of each 2.5D pose; hence, the depths of each 2.5D pose still represent the root joint-relative depths of each hand.
We denote the transformed $\mathbf{P}_\text{r}$ and $\mathbf{P}_\text{l}$ by $\mathbf{P}_\text{r}'$ and $\mathbf{P}_\text{l}'$, respectively.


The 2.5D pose of the right hand $\mathbf{P}_\text{r}'$ and left hand $\mathbf{P}_\text{l}'$ are converted to 2.5D Gaussian heatmaps by making a Gaussian blob around the coordinates.
By converting coordinates to heatmaps, we can exploit the strong feature extraction power of ResNet~\cite{he2016deep} as ResNet takes tensor inputs, not vector inputs.
Then, we concatenate the 2.5D Gaussian heatmap of two hands in a channel dimension, denoted by $\mathbf{H} \in \mathbb{R}^{2J \times D \times H \times W}$.
$D$, $H$, and $W$ represent the depth, height, and width of the 2.5D heatmap, respectively, and we set them to 64.


\noindent\textbf{Output: 3D relative translation between two hands.}
We predict the 3D relative translation between two hands $\mathbf{t} \in \mathbb{R}^3$ from the 2.5D Gaussian heatmap $\mathbf{H}$.
We pass $\mathbf{H}$ to ResNet-18~\cite{he2016deep}, which produces a feature map $\mathbf{F} \in \mathbb{R}^{C \times H/8 \times W/8}$.
$C=512$ represents the channel dimension of $\mathbf{F}$.
We use the original ResNet-18 after dropping the first convolutional block to reduce the downsampling and the last fully-connected layers.
As the 3D relative translation represents a 3D relative location of the left wrist from the right wrist, extracting useful wrist information is a key for accurate 3D relative translation.
However, most existing methods~\cite{moon2020interhand2,rong2021monocular,fan2021learning,zhang2021interacting} perform global average pooling (GAP) to the last feature map of backbones and pass the output to several fully connected layers.
As GAP simply averages the spatial dimension, it might not be effective to capture useful wrist information.
Instead, we perform a bilinear interpolation at the 2D positions of left and right wrists in $\mathbf{F}$, where the 2D wrist positions are from $\mathbf{P}_\text{r}'$ and $\mathbf{P}_\text{l}'$.
The extracted wrist features are concatenated with the 2D wrist coordinates and fed to a linear layer, which finally produces 3D relative translation $\mathbf{t}$.
The effectiveness of our wrist feature extraction compared to previous GAP-based approaches is shown in the experimental section.



\begin{figure}[t]
\begin{center}
\includegraphics[width=\linewidth]{fig/transnet.pdf}
\end{center}
\vspace*{-7mm}
\caption{
The overall pipeline of TransNet.
It applies a 2D affine transformation to the 2.5D pose of each left and right hand to bring them to the union hand box space of the original input image.
Then, wrist features are extracted for the 3D relative translation estimation.
}
\vspace*{-3mm}
\label{fig:transnet}
\end{figure}


\subsection{Final outputs and loss functions}~\label{subsec:loss}
The final 3D interacting hand meshes consist of 1) the 3D mesh of the right hand $\mathbf{M}_\text{r}$ and 2) a summation of the 3D mesh of the left hand $\mathbf{M}_\text{l}$ and the 3D relative translation $\mathbf{t}$.
We train InterWild in an end-to-end manner by minimizing $L1$ distance between predicted and GT boxes, MANO parameters, 3D joint coordinates, and 3D relative translation.
Please note that the 3D relative translation is only supervised by MoCap datasets as ITW datasets do not provide GTs.






\section{Experimental Results}
\label{sec:experiments}
\subsection{Training Details}
\cite{Kalantari2017DeepHD} provides the first dataset specifically designed for multi-exposure HDR fusion under large motion. It consists of 74 training sets, which we use to supervise the training of our model. We crop the input images to patches of size \(256 \times 256\) at a step size of 64. This totally generates 20128 training samples. To augment training samples, we randomly rotate and flip the training images. The training adopts Adam optimizer. The learning rate is initialized to \(10^{-4}\) and is reduced to \(10^{-5}\) after 20 epochs. It is observed that 40 epochs are sufficient for the training to converge.    

\subsection{Numerical Evaluation}
We numerically measure the performance of our method on the 15 test sets of \cite{Kalantari2017DeepHD}, by Peak Signal-to-Noise Ratio (PSNR) and Structure Similarity, computed in both tonemapping domain (-\(\mu\)) and HDR linear domain (-L). Visual difference metric HDR-VDP-2 is also adopted, where the parameters are set as same as in previous works \cite{wu2018end} and \cite{niu2021hdrgan}. 

Table \ref{table_metrics} compares our model with state-of-the-art models. For \cite{yan2020nonlocal} and \cite{xiong2021hierarchical}, we use the results reported in the publications. Note that \cite{sen2012robust} and \cite{hu2013hdr} are not machine learning based methods. Moreover,  \cite{Kalantari2017DeepHD} and \cite{wu2018end} apply optical flow and homography transformation to preprocess the input images respectively, and hence entail extra computation. 

Table \ref{table_metrics} shows that our method outperforms competing method in terms of PSNR-L, SSIM-$\mu$, SSIM-L and HDR-VDP-2. It ranks the second best in PSNR-$\mu$, being slightly (0.1dB) inferior to \cite{xiong2021hierarchical}. Note that \cite{xiong2021hierarchical} utilizes a pretrained model to detect ghosting regions for training, whereas our method does not require any pretrained model. The high PSNR and SSIM scores varify that our model has strong HDR reconstruction ability and can accurately restore the radiance and structure of the scene in both tonemapping domain and HDR linear domain. Furthermore, its high performance in term of HDR-VDP-2\cite{mantiuk2011hdr} performance indicates that our method can generate HDR image visually close to the target image.

\begin{table*}[ht]
\centering
\begin{tabular}{l|c|c|c|c|c}
\hline
& PSNR-$\mu$ & PSNR-L & SSIM-$\mu$ & SSIM-L & HDR-VDP-2 \\
\hline
\bfseries Sen & 40.97 & 38.36 & 0.9830 & 0.9746 & 60.60\\
\hline
\bfseries Hu  & 35.65 & 30.80 & 0.9725 & 0.9491 & 58.34\\
\hline
\bfseries Kalantari & 42.69 & 41.22 & 0.9888 & 0.9845 & 65.05\\
\hline
\bfseries DeepHDR& 41.99 & 41.22 & 0.9878 & 0.9859 & \underline{65.91}\\
\hline
\bfseries AHDR & 43.62 & 41.03 & 0.9900  &\underline{0.9883} & 63.85 \\
\hline 
\bfseries NHDRRNet& 42.414 & - & 0.9887 & - & 61.21 \\
\hline 
\bfseries HDR-GAN &43.92 & \underline{41.57} &\underline{0.9905} &0.9865 & 65.45\\
\hline 
\bfseries HFNet & \textbf{44.28} & 41.47 & - & - & - \\
\hline 
\bfseries Ours & \underline{44.18} & \textbf{42.19}&\textbf{0.9912} & \textbf{0.9883}& \textbf{67.07} \\
\hline
\end{tabular}
\caption{Numerical performance of the proposed model, evaluated on the dataset by Kalantari-Ramamoorthi. The best and second best results for each metric are marked in \textbf{bold} and \underline{underlined}, respectively}
\label{table_metrics}
\end{table*}

\subsection{Visual Performance Evaluation}

\begin{figure*}[!htb]
\centering
\includegraphics[width=\textwidth]{experiments/kalantari_test.png}
\caption{Visual comparison on the test set of Kalantari-Ramamoorthi dataset. Zoom-in views of reconstruction by each method are presented on the saturated regions that contain moving objects. Our network built with gated Swin Transformer yields noticeably better visual results than other methods.}
\label{fig_kalantari_test}
\end{figure*}
Fig. \ref{fig_kalantari_test} present the visual performance of our method and comparable methods on two examples from \cite{Kalantari2017DeepHD}. We present the zoom-in views of two challenging cases, where large saturated regions contain substantial non-rigid motion in the reference image. The two patch-based methods do not reconstruct the missing details in the saturated regions, as they heavily rely on the details provided by the reference image for registration. Image reconstructed by the optical flow based method \cite{Kalantari2017DeepHD} suffers motion blur artifacts. This is because the convolutions of DeepHDR and HDR-GAN have limited receptive fields, and hence are hampered to repair missing content in misaligned regions by aligned regions. The gating mechanism of AHDR is only applied to low-level features, so the high-level outliers may deteriorate the HDR fusion. In contrast to comparable methods, our model remarkably overcomes the ghosting artifacts.

\begin{figure}[ht]
\centering
\includegraphics[width=\columnwidth]{experiments/sen_test.pdf}
\caption{Visual performance comparison on example images from the dataset by Sen et al. Zoom in views on challenging areas are presented. Although the ground truth is unavailable, it can be clearly observed that our method visually performs better than comparable methods.}
\label{sen_test}
\end{figure}

\begin{figure}[ht]
\centering
\includegraphics[width=\columnwidth]{experiments/tursun_test.pdf}
\caption{Visual performance comparison on example images from the dataset by Tursun et al. Compared to state of the art methods, our method suffers less ghosting artifact.}
\label{tursun_test}
\end{figure}

Fig.\ref{sen_test} and Fig.\ref{tursun_test} present visual performance of our method on two examples from benchmark datasets \cite{sen2012robust} and \cite{tursun2016objective}. As these test datasets   do not provide ground truth image. we mark the visual difference on the results generated by different methods. It can be seen that our method suffers less artifacts than other methods in various scenes with various motion patterns, achieving better visual results. Our method creates high-quality HDR more robustly and generalizes well. 

\subsection{Ablation Study}

\begin{table}[h]
\centering
\resizebox{\columnwidth}{!}{
\begin{tabular}{l|c|c|c|c|c}
\hline
                         & PSNR-$\mu$ & PSNR-l & SSIM-$\mu$ & SSIM-l & HDR-VDP-2 \\ \hline
restormer(w/o ssim loss) & 44.00  & 41.5   & 0.9906 & 0.9873 & 64.72  \\ \hline
Ours(w/o ssim loss)      & 44.07  & 41.83  & 0.9909 & 0.9879 &  64.78  \\ \hline
Ours                     & 44.18  & 42.19  & 0.9912 & 0.9883 & 67.07      \\ \hline
\end{tabular}
}
\caption{Experimental results of ablation study. We compare using Gated Swin Transformer v.s. Gated Transformer, and the combined loss function v.s. the traditional $l_{1}$ norm loss function.}
\label{table_ablation_block_loss}
\end{table}

We verify various components of our method, including Swin Transformer, loss function, and gating mechanism by ablation study.

\subsubsection{Ablation Study on Block Design}
Our model has similar architecture to Restormer, which uses modified Transformer, whereas we use modified Swin Transformer as the building unit. For comparison, we replace the residual modules in each block in our model with multiple transformer layers as in Restormer, with same number of transformer layers. Table \ref{table_ablation_block_loss} presents the results, which show that using Swin Transformer achieves superior performance in all measures. The reason is that the attention module of Restormer is computed channel-wise, but forgoes the cross-exposure spatial dependency to repair the non-aligned area. 

\subsubsection{Ablation Study on Loss Function}
We trained our model under different loss function configurations, as shown in \ref{table_ablation_block_loss}. The results validate that the SSIM loss benefits detail reconstruction.

\subsubsection{Ablation Study on Gating Mechanism}
\begin{table}[h]
\resizebox{\columnwidth}{!}{
\begin{tabular}{l|c|c|c|c|c}
\hline
           & PSNR-$\mu$ & PSNR-l & SSIM-$\mu$ & SSIM-l & HDR-VDP-2 \\ \hline
w/o gating & 43.14  & 41.03  & 0.9904 & 0.9868 &     64.88      \\ \hline
one gating & 43.44  & 41.42  & 0.9909 & 0.9882 &    67.13   \\ \hline
Ours       & 43.61  & 41.74  & 0.9909 & 0.9881 & 66.96     \\ \hline
\end{tabular}
}
\caption{Ablation experimental results to verify the effectiveness of the gating mechanism}
\label{table_ablation_gating}
\end{table}

The gating mechanism is an important component in our model. Ablation study is conducted in the gating mechanism as follows.

\textbf{w/o gating}: The gating mechanism is not used in the feed forward network of all transformer layers in the model, that it, our GST unit degenerate to the vanilla Swin Transformer.

\textbf{one gating}: The gating mechanism is only used in the first Swin Transformer layers subsequent to the embedding layer, but not used for other layers. 

 Table \ref{table_ablation_gating} shows the results of the ablation experiments, where the model is trained for 20 epochs. By removing the gating mechanism, the network relies on self-attention for image alignment, resulting in the lowest performance. On top of it, adding gates to low level layers notably improves the HDR reconstruction. Furthermore, by integrating the gating mechanism with all Swin Transformer layers, the model effectively inpaints information in non-aligned regions and obtains the highest HDR reconstruction results, thus validates the effectiveness of the gating mechanism in our model.

\section{Conclusion}\label{sec:conclusion}
In this work, we focus on addressing the fundamental challenge of OOD detection tasks, which is how to fully understand the semantic discrepancy between the ID/OOD samples. We reveal that the key to success in the realistic SCOOD task is to allocate as many ID samples in the unlabeled set correctly as possible. To this end, we propose a novel uncertainty-aware optimal transport scheme that introduces class-specific energy scores as guidance for effective label assignment. Experimental results show that our method achieves better performance than previous state-of-the-art methods on SCOOD benchmarks.

\textbf{Limitations.} In addition to temperature scaling, other techniques such as feature clipping applied in ReAct~\cite{sun2021react} also enhance the performance of energy score, so how to obtain an OOD score that best fits the SCOOD task can be further explored. Moreover, a setting highly related to SCOOD has been proposed in \cite{katz2022training} and formulated as a constrained optimization problem. We will also theoretically analyze these practical OOD settings in our feature work.

% \section*{Acknowledgments}
\textbf{Acknowledgments.} 
This work is supported by National Key R\&D Program of China under Grant 2020AAA0105701, National Natural Science Foundation of China (NSFC) under Grants 61872327, Major Special Science and Technology Project of Anhui, National Natural Science Foundation of China (62033012) and Ant Group through Ant Research Intern Program.


% suppl



\begin{center}
\textbf{\large Supplementary Material for \\ ``Bringing Inputs to Shared Domains for \\ 3D Interacting Hands Recovery in the Wild"}
\end{center}

\setcounter{section}{0}
\setcounter{table}{0}
\setcounter{figure}{0}

\renewcommand{\thesection}{\Alph{section}}   
\renewcommand{\thetable}{\Alph{table}}   
\renewcommand{\thefigure}{\Alph{figure}}


In this supplementary material, we provide more experiments, discussions, and other details that could not be included in the main text due to the lack of pages.
The contents are summarized below:
\begin{enumerate}[nosep, label=\Alph*.]  
    \item Qualitative comparison
    \item Ablation study on the architecture of TransNet
    \item Verification of reproduced results in Table 5
    \item Clarification of 2D-based weak supervision in Table 3
    \item Architecture of DetectNet
    \item Architecture of SHNet
    \item Implementation details
    \item Limitations
\end{enumerate}




\begin{figure*}[t]
\begin{center}
\includegraphics[width=\linewidth]{fig/qualitative_comparison_2.pdf}
\end{center}
\vspace*{-5mm}
\caption{
Qualitative comparison between our InterWild and IntagHand~\cite{li2022interacting} on MSCOCO.
Ours detects hand boxes from the human images, while IntagHand takes the hand images using GT boxes.
}
\label{fig:qualitative_comparison_2}
\end{figure*}

\section{Qualitative comparisons}

Fig.~\ref{fig:qualitative_comparison_2} shows that ours produces much more robust results than IntagHand~\cite{li2022interacting} on in-the-wild images.
Overall, IntagHand produces reasonable 3D hand mesh for a visible hand but fails to recover the other occluded hand.
Also, it suffers from depth ambiguity as the first and fourth rows show, where the 2D error is small but the 3D error is large.
The fourth row also shows that IntagHand fails to recover 3D relative translation between two hands due to the depth ambiguity, while ours successfully recovers.
Finally, we think the reason why IntagHand produces non-hand shape meshes is that IntagHand directly regresses the 3D coordinates of 3D hand meshes.
On the other hand, ours regresses MANO parameters and 3D meshes are obtained by forwarding the parameters to the MANO layer.
\section{Ablation study on the architecture of TransNet}




Table~\ref{table:transnet_architecture} shows that our fully convolutional network (FCN)-based architecture performs the best compared to widely used fully-connected (FC) network architecture and Transformer~\cite{vaswani2017attention}.
We think this is because our FCN can explicitly utilize the spatial relationship between voxels using the 2.5D heatmap representation.
On the other hand, FC-based and Transformer-based settings take 2.5D coordinates as input.
This result is consistent with previous studies~\cite{moon2020i2l}, which demonstrates the superiority of heatmap representation over coordinate representation.
As proposing better network architecture for TransNet is not our main focus, we believe developing its network architecture can be an interesting future direction.
We used the architecture of Martinez~\etal~\cite{martinez2017simple} for the FC setting, and Zheng~\etal~\cite{zheng20213d} for the Transformer setting.

\begin{table}[t]
\small
\centering
\setlength\tabcolsep{1.0pt}
\def\arraystretch{1.1}
\begin{tabular}{C{3.5cm}|C{2.0cm}|C{2.0cm}}
\specialrule{.1em}{.05em}{.05em}
Settings & HIC~\cite{tzionas2016capturing} & IH2.6M~\cite{moon2020interhand2} \\ \hline
FC & 56.76 & 29.85 \\ 
Transformer & 64.49 & 33.59 \\
\textbf{FCN (Ours)} & \textbf{31.35} & \textbf{29.29} \\ 
\specialrule{.1em}{.05em}{.05em}
\end{tabular}
\vspace*{-3mm}
\caption{MRRPE comparisons between TransNet that have various architectures.}
\label{table:transnet_architecture}
\end{table}

\subsection*{Reproducibility Statement}

For reproducibility, we share our code at \url{https://github.com/yifanzhang-pro/Kernel-InfoNCE}. The experiment results can be reproduced following the instructions in the README document. We also provide our experiment details in Appendix \ref{section: experiment_details}.
\section{2D-based weak supervision in Table~\ref{table:transnet_input_weak_sup}}
We describe how the 2D-based weak supervision in Table~\ref{table:transnet_input_weak_sup} of the main manuscript is introduced.
We modified TransNet to output 3D global translation of the right hand and 3D relative translation between two hands from the two hand input.
We observed this produces better results than estimating 3D global translations of both hands.
The 3D global translation of the left hand is obtained by adding the 3D relative translation to the 3D global translation of the right hand.
Then, we translate root joint-relative 3D joint coordinates of each hand (\textit{i.e.}, output of SHNet) using the 3D global translation of each hand.
The translated 3D joint coordinates of two hands are projected to the TransNet's input space (\textit{i.e.}, union of two hand boxes).
Finally, we minimized the L1 distance between the projected and GT 2D coordinates.
Please note that we make the shape parameter of MANO of left and right hands be the same during the weak supervision to minimize the scale ambiguity of each hand.

\section{Architecture of DetectNet}

DetectNet detects left and right hands from an input image $\mathbf{I}_\text{det} \in \mathbb{R}^{3 \times H_\text{det} \times W_\text{det}}$, downsampled from a high-resolution image $\mathbf{I} \in \mathbb{R}^{3 \times 2H_\text{det} \times 2W_\text{det}}$, by predicting two bounding boxes of the left and right hands.
$H_\text{det}=256$ and $W_\text{det}=192$ denote height and width of $\mathbf{I}_\text{det}$, respectively.
The downsampling is necessary to save computational costs.
To this end, we extract the image feature from $\mathbf{I}_\text{det}$ using ResNet-50 and pass the feature to three consecutive deconvolutional layers, which upsample the feature map by 8 times.
We denote the upsampled feature map by $\mathbf{F}_\text{det} \in \mathbb{R}^{C_\text{det} \times H_\text{det}/4 \times W_\text{det}/4}$.
$C_\text{det}=256$ denotes the number of channel of $\mathbf{F}_\text{det}$.
We use the original ResNet-50 after dropping global average pooling (GAP) and following fully-connected layers.
Then, a 1-by-1 convolutional layer takes $\mathbf{F}_\text{det}$ and predicts a 2D heatmap of two hand bounding box centers.
Soft-argmax~\cite{sun2018integral} extracts 2D hand bounding box center coordinates from the 2D heatmap in a differentiable way.
Then, we extract bounding box center features of left and right hands by performing a bilinear interpolation at the box center positions of $\mathbf{F}_\text{det}$.
The extracted bounding box center features of each hand are passed to two fully-connected layers, which produce a scale of the bounding box.
By decoding the bounding box centers and scales, we obtain two bounding boxes of left and right hands.
\section{Architecture of SHNet}

SHNet processes right and left hand images in the same way except for the input and output of the left hand image are flipped to the right hand and flipped back to the left hand, respectively.
Hence, we omit right and left hand notations in the following description. 

SHNet predicts 2.5D joint coordinates $\mathbf{P} \in \mathbb{R}^{J \times 3}$, MANO parameters, and 3D global translation $\mathbf{g} \in \mathbb{R}^3$ from a single hand image $\mathbf{I}_\text{hand} \in \mathbb{R}^{3 \times H_\text{hand} \times W_\text{hand}}$.
$H_\text{hand}=256$ and $W_\text{hand}=256$ denote height and width of $\mathbf{I}_\text{hand}$, respectively.
$J=21$ denotes the number of single hand joints.
MANO parameters include 3D joint rotations $\theta \in \mathbb{R}^{16 \times 3}$ and hand shape parameter $\beta \in \mathbb{R}^{10}$.
The 3D global translation $\mathbf{g}$ is used in two cases: 1) loss calculations and 2) mesh rendering to visualize results.


\noindent\textbf{2.5D joint coordinate estimation.}
ResNet-50 extracts an image feature $\mathbf{F}_\text{hand} \in \mathbb{R}^{C_\text{hand} \times H_\text{hand}/32 \times W_\text{hand}/32}$ from a single hand image $\mathbf{I}_\text{hand}$.
$C_\text{hand}=2048$ denotes the number of channel of $\mathbf{F}_\text{hand}$.
Then, the extracted image feature $\mathbf{F}_\text{hand}$ is passed to a 1-by-1 convolutional layer, which outputs $JD$-dimensional feature map, where $D=8$ denotes discretized depth size.
The feature map is reshaped to the dimension of $\mathbb{R}^{J \times D \times H_\text{hand}/32 \times W_\text{hand}/32}$, which is a 3D heatmap of hand joints.
Soft-argmax~\cite{sun2018integral} extracts 2.5D joint coordinates $\mathbf{P}$ from the 3D heatmap.

\noindent\textbf{MANO parameter regression.}
SHNet firsts reduces the channel dimension of $\mathbf{F}_\text{hand}$ from 2048 to 512 to reduce computational costs.
Then, SHNet extracts joint features by performing a bilinear interpolation at the $(x,y)$ position of the 512-dimensional feature map.
The joint features contain essential articulation information about hand joints.
Finally, a single linear layer outputs MANO pose parameter $\theta$ from a concatenation of the joint features with 2.5D joint coordinates.
The pose parameter $\theta$ is initially estimated in the 6D rotational representation~\cite{zhou2019continuity} and transformed to the axis-angle representation.
The MANO shape parameter $\beta$ and 3D global translation $\mathbf{g}$ are estimated from GAPed $\mathbf{F}_\text{hand}$.

\section{Additional Details}
\label{sec:details}

\subsection{Implementation Details}
\label{sec:imp}
Below we provide all the implementation details of our method, detailed in Section 3 in the main paper.

\subsubsection*{Grid-Based Volumetric Representation}
We use 100 images uniformly sampled from upper hemisphere poses along with corresponding camera intrinsic and extrinsic parameters to train our initial grid. We follow the standard ReLU Fields~\cite{karnewar2022relu} training process using their default settings aside from two modifications: 
\begin{enumerate}
    \item We change the result grid size from the standard $128^3$ to $160^3$ to increase the output render quality.
    \item As detailed in the main paper, we limit the order of spherical harmonics to be zero order only to avoid undesirable view-dependent effects (we further illustrate these effects in Section \ref{sec:sh}). 
\end{enumerate}

 

\subsubsection*{Text-guided Object Editing}
We perform 8000 training iterations during the object editing optimization stage. During each iteration, a random pose is uniformly sampled from an upper hemisphere and an image is rendered from our edited grid $G_e$ according to the sampled pose and the rendering process described in ReLU Fields \cite{karnewar2022relu}. Noise is then added to the rendered image according to the time-step sampled from the fitting distribution. 

We use an annealed SDS loss which gradually decreases the maximal time-step we draw $t$ from. Formally, this annealed SDS loss introduces three additional hyper-parameters to our system: a starting iteration $i_{start}$, an annealing frequency $f_a$ and an annealing factor $\gamma_a$. With these hyper-parameters set, we change our time-step distribution to be:
\begin{equation}
    t \sim U[t_0 + \varepsilon, t_{final}*k_i + \varepsilon],
\end{equation}
\begin{equation}
    k_i = 
    \begin{cases}
    1, & \text{if } i < i_{start} \\
    k_{i-1}*\gamma_a, & \text{else if } i\ \% \ f_a = 0 \\
    k_{i-1}, & \text{otherwise}
    \end{cases}
\end{equation}
In all our experiments, the values we use for $\varepsilon$, $i_{start}$, $f_a$ and $\gamma_a$ are 0.02, 4000, 600, and 0.75. Additionally, we stop annealing the time-step when it reaches a value of 0.35. %
The latent diffusion model we use in our experiments is "StableDiffusion 2.1" by Stability AI\href{https://huggingface.co/stabilityai/stable-diffusion-2-1}. %

We use a weight of $200$ to balance the two terms (multiplying $\mathcal{L}_\text{reg3D}$ by this weight value). The volumetric regularization term operates only on the density features of the editing grid. The optimizer we used in this (and all other stages) is the Adam optimizer~\cite{adamoptimizer} with a learning rate of 0.03 and betas 0.9, 0.999. The resolution of the images rendered from our grid is 266$\times$266. We add a "a render of" prefix to all of our editing prompts as we found that this produced more coherent results (and the images the LDM receives are indeed renders).



\subsubsection*{Spatial Refinement via 3D Cross-Attention}
The diffusion model we use for this stage is \href{ https://huggingface.co/CompVis/stable-diffusion-v1-4 }{"StableDiffusion 1.4" by CompVis} and it consists of several %
cross-attention layers at resolutions 32, 16, and 8. To extract a single attention map for each token we interpolate each cross attention map from each layer and attention head to our image resolution (266x266) and take an average per each token. %
The time-step we use to generate the attention maps is 0.2 (the actual step being 0.2 * $N_{steps}$ = 200). 

The cross-attention grids $A_{e}$ and $A_{obj}$ contain a density feature and an additional one-dimensional feature $a$, which represents the cross-attention value at a given voxel and can be interpreted and rendered as a grayscale luma value. We initialize the density features in these grids to the density features of the editing grid's (the former stage's output) and freeze them.
At each refinement iteration we generate two 2D cross-attention maps from the LDM, one for the object and one for the edit. After obtaining the 2D cross-attention maps, we render gray-scale heatmaps %
from $A_{e}$ and $A_{obj}$ and use $L1$ loss to encourage similarity between the rendered attention images and their corresponding attention maps extracted from the diffusion model. We repeat this process for 1500 iterations, sampling a random upper-hemisphere pose each time. As in the former optimization stage, we use the Adam optimizer with a learning rate of 0.03 and betas 0.9 and 0.999 and generate images in 266$\times$266 resolution.

After obtaining the two grids $A_{e}$ and $A_{obj}$, we perform element-wise softmax on their $a$ values to obtain probabilities for each voxel belonging to either the object, denoted by $P_{obj}(v)$, or the edit, denoted by $P_e(v)$. 
We then proceed to calculate the binary refinement volumetric mask. To do this we define a graph in which each non-zero density voxel in our edited grid $G_e$ is a node. We define "edit" and "object" labels as the \emph{source} and \emph{drain} nodes, such that a node connected to the source node is marked as an "edit" node and a node connected to the drain node is marked as an "object" node. We rank the nodes according to their $P_e(v)$ values and connect the top $N_{init-edit}$ nodes to the source node. We then rank the nodes according to their $P_{obj}(v)$ value and connect the top $N_{init-object}$ nodes to the drain node.
We then connect the non-terminal nodes to each-other in a 6-neighborhood with the capacity of each edge being $w_{pq}$ as detailed in the main paper.


We set the hyper-parameters $N_{init-edit}$ and $N_{init-object}$ to be 300 and 200. %
To perform graph-cut~\cite{boykov2001fast}, we used the \href{ https://github.com/pmneila/PyMaxflow }{PyMaxflow} implementation of the max-flow / min-cut algorithm. %

\subsection{Evaluation Protocol}
To evaluate our results quantitatively, we constructed a test set composed of three scenes: Dog, Cat and Kangaroo, and six editing prompts: (1) A $\left<object\right>$ wearing big sunglasses, (2) A $\left<object\right>$ wearing a Christmas sweater, (3) A $\left<object\right>$ wearing a birthday party hat, (4) A yarn doll of a $\left<object\right>$, (5) A wood carving of a $\left<object\right>$, (6) A claymation $\left<object\right>$. This yields 18 edited scenes in total. We render each edited scene from 100 different poses distributed evenly along a $360^{\circ}$ ring. In addition to these 18 scenes we also render 100 images from the same poses on the initial (reconstruction) grid $G_i$ for each input scene. When comparing our result with other 3D textual editing papers we evaluate our results using two CLIP-based metrics. %
The CLIP model we used for both of these metrics is \href{https://github.com/openai/CLIP}{ViT-B/32} and the input image text prompts used to calculate the directional CLIP metric is ``A render of a $\left<object\right>$". $CLIP_{Dir}$ is calculated for each edited image in relation to the corresponding image in the reconstruction scene.
To quantitatively evaluate ablations we use two additional metrics using FID \cite{Seitzer2020FID}. For this we use the \href{ https://github.com/mseitzer/pytorch-fid }{pytorch implementation} given by the authors with the standard settings. 

\subsubsection*{$360^\circ$ \emph{Real Scenes}} For the $360^\circ$ \emph{Real Scenes} edits we follow the same implementation details as outlined previously, with two modifications: 
\begin{enumerate}
    \item Our input poses are  created in a spherical manner and when rendering we sample linearly in inverse depth rather than in depth as seen in the official implementation of NeRF \href{https://github.com/bmild/nerf}.
    \item We perform 5000 training iterations during the object editing optimization stage and the values we use for $\varepsilon$, $i_{start}$, $f_a$ and $\gamma_a$ are 0.02, 3000, 400, and 0.75.
\end{enumerate}

\subsection{3D Object Editing Techniques} 
Below we provide additional details on the alternative 3D object editing techniques we compare against. All of the techniques we compare against use only an un-textured mesh and an editing prompt as input. As such, we used the meshes our inputs were rendered from as input for the editing methods. Additionally, we tested an additional scenario in which we imported the 'horse' mesh from the \href{https://github.com/threedle/text2mesh/blob/main/data/source_meshes/horse.obj}{Text2Mesh GitHub repository} to blender, added a grey-matte material to it and rendered images of it to use as input for our system. This scenario used four prompts - (1) A wood carving of a horse, (2) A horse wearing a Santa hat, (3) A donkey, (4) A carousel horse, and was used for qualitative comparisons only.

\subsubsection*{Text2Mesh} 
When comparing to Text2Mesh we used the \href{https://github.com/threedle/text2mesh}{code provided by the authors} and the input settings given in the "run\_horse.sh" demo file.



\subsubsection*{SketchShape}
 In this comparison we again use the \href{https://github.com/eladrich/latent-nerf}{code provided by the authors}. And the input parameters used are the default parameters in the 'train\_latent\_nerf.py' script  \href{https://github.com/eladrich/latent-nerf/tree/main/scripts}{'train\_latent\_nerf.py' script} with 10,000 training steps (as opposed to the default 5,000).

\subsubsection*{Latent-Paint}
We compared our method to Latent-Paint only qualitatively as this method outputs edits that transform only the appearance of the input mesh, rather than appearance and geometry. As in SketchShape we used the code provided by the authors and used the default input settings provided for latent paint, which are given in the  \href{https://github.com/eladrich/latent-nerf/tree/main/scripts}{'train\_latent\_paint.py' script}.

\subsection{2D Image Editing Techniques}

When comparing to InstructPix2Pix and SDEdit we constructed two image sets for each scene / prompt combination we wanted to test. Both sets were created by rendering one of our inputs in evenly spaced poses along a $360^{\circ}$ ring, one set was rendered over a white background and the other over a 'realistic' image of a brick wall. We used these sets as input for each 2D editing method along with an editing prompt and compared the results to rendered outputs from our result grids. When comparing to InstructPix2Pix we used the standard \href{https://huggingface.co/docs/diffusers/api/pipelines/stable_diffusion/pix2pix}{InstructPix2Pix pipeline} with 16bit floating point precision, a guidance scale of 1 and 20 inference steps. When giving prompts to InstructPix2Pix we rephrased our prompts as instructions, for example turning "a dog wearing sunglasses" to "put sunglasses on this dog". When comparing to SDEdit we used the \href{https://huggingface.co/docs/diffusers/using-diffusers/img2img}{standard SDEdit pipeline} with guidance scale of 0.75 and a strength of 0.6.



\ignorethis{
\begin{enumerate}
    \item Real-scenes?
    \item cross-attention maps - \textbf{maybe a correction figure?} also, what timesteps are we using, any other important details?
    \item explain ablations and comparisons (all details needed to reproduce experiments)
    \item all hyperparameters
    \item Dataset details - the meshes, all the prompts
\end{enumerate}
}



\section{Limitations and Future Work}

We summarize the limitations we have identified for our method and propose
future research directions.

\textbf{Parallel implementation:} 
With a focus on accuracy and algorithms, our implementation for this work is
serial. Some of the most time-consuming routines in our method can easily
benefit from a parallel implementation, while the same is not obvious for the
SAP solver and the Schur complement computation. Leveraging the power of
parallelization on modern hardware for these computations is an interesting area
for future investigation.

\textbf{Rotational invariance:} 
As with all other linear constitutive models, our linearized model with lagged
rotational component is not rotationally invariant. Thus it is not suitable for
simulation of extreme deformations using large time steps. For those scenarios,
we fall back to traditional nonlinear models with Hessian positive definite
corrections proposed in \cite{bib:teran2005robust}.

\textbf{Self-contact:} 
We do not consider self-contact at the moment due to the lack of support by our
geometry engine. Self-contact can be incorporated into our method by updating the
geometry engine to augment the set of contacts reported.

\textbf{Tunneling at high speeds:} Though our method has a lower computational
cost, it could benefit from continuous collision detection strategies
\cite{bib:li2020ipc} to provide constraints before contact is established. This
would allow to mitigate issues such as objects tunneling past each other at high
speeds. Efficient solution to mitigate this issue is a topic of active research
for the authors.

\textbf{Redundant constraints:} Our geometry engine often introduces a large
number of constraints to resolve contact. Similarly, welding a large number of
deformable mesh vertices to a rigid body (as done in Section
\ref{sec:bubble_gripper}) introduces many constraints. Even though our SAP
solver \cite{bib:castro2022unconstrained} provides existence and uniqueness
guarantees, a large number of constraints hurts performance as can be observed
in the \emph{Soft-bubble} example. We are currently investigating strategies to
significantly reduce the number of constraints without sacrificing accuracy.


\clearpage

%%%%%%%%% REFERENCES
{\small
\bibliographystyle{ieee_fullname}
\bibliography{bib}
}


\end{document}
