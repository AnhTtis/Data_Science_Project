\documentclass[twoside,11pt]{article}

% Any additional packages needed should be included after ewrl_2022.
% Note that ewrl_2022.sty includes epsfig, amssymb, natbib and graphicx,
% and defines many common macros, such as 'proof' and 'example'.
%
% It also sets the bibliographystyle to plainnat; for more information on
% natbib citation styles, see the natbib documentation, a copy of which
% is archived at http://www.jmlr.org/format/natbib.pdf


% To compile for anonymous submission
\usepackage[accepted]{ewrl_2022}

% Our imports
\usepackage{amsfonts, amsmath, amssymb}
\usepackage{xfrac}
\usepackage{graphicx, xcolor, colortbl}
\usepackage{footnote}
% \usepackage{tablefootnote}
\usepackage{natbib}
\usepackage{dsfont}
\usepackage{bbm}
\usepackage{algorithm}
%\usepackage{algorithmic}
\usepackage{algpseudocode}
\usepackage{import}
\usepackage{mathtools}
\usepackage{bm}
\usepackage{xspace}
\usepackage{thmtools,thm-restate} %to restate lemma
\usepackage{accents}
\usepackage{framed}
\usepackage[inline]{enumitem}
\usepackage{booktabs}
\usepackage{color, colortbl}
\usepackage{natbib}
\usepackage{subcaption}

\definecolor{LightGray}{gray}{0.9}


% For  footnote reuse (https://tex.stackexchange.com/questions/35043/reference-different-places-to-the-same-footnote)
% \usepackage{scrextend}

% for multirow in table
\usepackage{multirow}

%\newtheorem{lemma}{Lemma}
% \newtheorem{theorem}{Theorem}
% \newtheorem{corollary}{Corollary}
%
% \theoremstyle{definition}
% \newtheorem{property}{Property}
% \newtheorem{proposition}{Proposition}
% \newtheorem{definition}{Definition}
% \newtheorem{assumption}{Assumption}
%\newtheorem{remark}{Remark}
% \newtheorem{example}{Example}

\usepackage{sty/notations}


% working together
\usepackage{todonotes}
\newcommand{\todoMariana}[1]{\todo[color=violet!40, inline]{\small Mariana: #1}}
\newcommand{\todoPi}[1]{\todo[color=yellow!40, inline]{\small #1}}


% for table of contents (Appendix only)
\usepackage{minitoc}
\renewcommand \thepart{}
\renewcommand \partname{}


% Definitions of handy macros can go here

\newcommand{\dataset}{{\cal D}}
\newcommand{\fracpartial}[2]{\frac{\partial #1}{\partial  #2}}

\ewrlheading{}{}{}{Vargas Vieyra and M\'enard}


\ShortHeadings{Generative Models with Reinforcement Learning}{M. Vargas Vieyra and P. M\'enard}
\firstpageno{1}

\begin{document}

\title{Learning Generative Models with Goal-conditioned Reinforcement Learning}

\author{\name Mariana Vargas Vieyra \email mariana.vargas-vieyra@inria.fr \\
       \addr Inria, France
       \AND
       \name Pierre M\'enard \email pierre.menard@ens-lyon.fr \\
       \addr ENS Lyon, France}


\maketitle

\begin{abstract}%   <- trailing '%' for backward compatibility of .sty file
  We present a novel, alternative framework for learning generative models with goal-conditioned reinforcement learning. We define two agents, a \textit{goal conditioned agent} (GC-agent) and a \textit{supervised agent} (S-agent). Given a user-input initial state, the GC-agent learns to reconstruct the training set. In this context, elements in the training set are the \textit{goals}. During training, the S-agent learns to imitate the GC-agent while remaining agnostic of the goals. At inference we generate new samples with the S-agent. Following a similar route as in variational auto-encoders, we derive an upper bound on the negative log-likelihood that consists of a reconstruction term and a divergence between the GC-agent policy and the (goal-agnostic) S-agent policy. We empirically demonstrate that our method is able to generate diverse and high quality samples in the task of image synthesis.
\end{abstract}

\begin{keywords}
  Generative Models, Goal-conditioned Reinforcement Learning.
\end{keywords}


\section{Introduction}
\label{sec:introduction}
% \begin{itemize}
%     % Diffusion of FL
%     \item {\st{Diffusion of FL}}
%     % Security threats to FL
%     \item {\st{Security threats to FL with particular focus on model poisoning}}
%     % Limitations of existing countermeasures
%     \item {\st{Current countermeasures (e.g., KRUM) and their limitations}}
%     % Proposed method and its advantages
%     \item {\st{Intuitive description of the proposed method and its difference (i.e., advantages) w.r.t. state of the art}}
%     % Main contributions
%     \item {\st{Summary of the main contributions of this work}}
%     % Paper's structure and organization
%     \item {\st{Paper's structure and organization}}
% \end{itemize}

% Diffusion of FL
Recently, {\em federated learning} (FL) has emerged as the leading paradigm for training distributed, large-scale, and privacy-preserving machine learning (ML) systems~\cite{mcmahan2017googleai,mcmahan2017aistats}. 
The core idea of FL is to allow multiple edge clients to collaboratively train a shared, global model without disclosing their local private training data.
%Specifically, an FL system consists of a central server and many edge clients; 
A typical FL round involves the following steps: {\em(i)} the server randomly picks some clients and sends them the current, global model; {\em(ii)} each selected client locally trains its model with its own private data; then, it sends the resulting local model to the server;\footnote{Whenever we refer to global/local model, we mean global/local model {\em parameters}.} {\em(iii)} the server updates the global model by computing an \emph{aggregation function}, usually the average (FedAvg), on the local models received from clients.
% \begin{enumerate}
%     \item[{\em(i)}] the server sends the current, global model to the clients and appoints some of them for training;
%     \item[{\em(ii)}] each selected client locally trains its copy of the global model with its own private data; then, it sends the resulting local model back to the server;\footnote{Whenever we refer to global/local model, we mean global/local model {\em parameters}.}
%     \item[{\em(iii)}] the server updates the global model by computing an \emph{aggregation function} on the local models received from clients (by default, the average, also referred to as FedAvg~\cite{mcmahan2017aistats}).
% \end{enumerate}
This process goes on until the global model converges. %(e.g., after a certain number of rounds or other similar stopping criteria).
%\\
% The advantages of FL over the traditional, centralized learning paradigm are undoubtedly clear in terms of flexibility/scalability (clients can join/disconnect from the FL network dynamically), network communications (only model weights\footnote{We will use \textit{parameters} and \textit{weights} interchangeably.} are exchanged between clients and server), and privacy (each client's private training data is kept local at the client's end and not uploaded to the server).
\\
% Security threats to FL
%However, the growing adoption of FL also raises security concerns~\cite{costa2022covert}, particularly about its confidentiality, integrity, and availability.
Although its advantages over standard ML, FL also raises security concerns~\cite{costa2022covert}. %, particularly about its confidentiality, integrity, and availability~\cite{costa2022covert}.
% OLD, LONG VERSION
% Indeed, some work deals with privacy leakage that may expose the local data of some clients~\cite{melis2019sp}. 
% A large body of work, instead, investigates attacks that usually aim to detriment the predictive accuracy of the learned global model. For instance, \emph{data poisoning} attacks achieve this goal by letting an adversary pollute the training set of some corrupt FL clients with maliciously crafted examples~\cite{jagielski2018sp}.
% Similarly, in \emph{model poisoning} the attacker attempts to tweak the global model weights~\cite{bhagoji2019pmlr} by directly perturbing the local model's weights of some infected FL clients before these are sent to the central server for aggregation, usually via so-called Byzantine attacks. 
% It turns out that Byzantine model poisoning attacks severely impact standard FedAvg; therefore, more robust aggregation functions must be designed to make FL systems secure.
Here, we focus on \emph{untargeted model poisoning} attacks~\cite{bhagoji2019pmlr}, where an adversary attempts to tweak the global model weights %\footnote{We will use the terms \textit{parameters} and \textit{weights} interchangeably.} 
by directly perturbing the local model's parameters of some infected clients before these are sent to the central server for aggregation.
In doing so, the adversary aims to jeopardize the global model \textit{indiscriminately} at inference time.
Such model poisoning attacks severely impact standard FedAvg; therefore, more robust aggregation functions must be designed to secure FL systems.
\\
% In this paper, we focus on designing a novel robust aggregation scheme at the server's end to contrast the effect of Byzantine model poisoning attacks.
%
% Current countermeasures and their limitations
%Several countermeasures have been proposed in the literature to combat model poisoning attacks on FL systems.
% Some methods use simple statistics more robust than plain average to smooth the impact of malicious updates (e.g., Trimmed Mean and FedMedian~\cite{yin2018icml}). 
% Other defenses implement outlier detection techniques to discard malicious updates from the aggregation performed at the server's end. Those are either based on heuristics (e.g., Krum/Multi-Krum~\cite{blanchard2017nips} and Bulyan~\cite{mhamdi2018pmlr}) or data-driven approaches (e.g., K-means clustering~\cite{shen2016acm} or DnC via spectral analysis~\cite{shejwalkar2021ndss}). 
% Finally, some strategies rely on a centralized ``source of trust'' to spot potential malicious updates (e.g., FLTrust~\cite{cao2020fltrust}).
% Several countermeasures have been proposed in the literature to combat model poisoning attacks on FL systems, i.e., to discard possible malicious local updates from the aggregation performed at the server's end. 
% These techniques range from simple statistics more robust than plain average (e.g., Trimmed Mean and FedMedian~\cite{yin2018icml}) to outlier detection heuristics (e.g., Krum/Multi-Krum~\cite{blanchard2017nips} and Bulyan~\cite{mhamdi2018pmlr}) or data-driven approaches (e.g., spectral analysis via K-means clustering~\cite{shen2016acm} or spectral analysis), or methods based on ``source of trust'' (e.g., FLTrust~\cite{cao2020fltrust}).
% OLD, LONG VERSION
%Several countermeasures have been proposed in the literature to combat Byzantine model poisoning attacks on FL systems.
% Descriptive statistics
% For example, Trimmed Mean and FedMedian aggregate local model updates using more robust statistics than standard average~\cite{yin2018icml}.
%
% % Heuristics for outlier detection
% Many existing Byzantine-resilient strategies implement some outlier detection heuristics to discard the model updates sent by potentially malicious clients from the input of the aggregation function.
% One of the most popular heuristics is Krum~\cite{blanchard2017nips}.
% This strategy tries to mitigate the impact of Byzantine attacks by selecting as a global model the local model with the smallest sum of Euclidean distances to {\em all} the other local models.
% Although powerful, Krum requires the server to know (or, at least, estimate) the number of malicious FL clients upfront, which is generally impossible in a realistic attack scenario. %
% Moreover, Krum may become ineffective for complex, high-dimensional model parameter spaces due to the curse of dimensionality.
% Bulyan~\cite{mhamdi2018pmlr} tries to overcome this issue by combining Krum with a variant of Trimmed Mean.
% % Data-driven outlier detection
% Other strategies use data-driven outlier detection techniques -- e.g., via K-means clustering~\cite{shen2016acm} -- to spot potential malicious local model updates. 
% %For instance, Shen et al. propose to cluster local model updates with K-means and thus identify outliers.
%
% % Other techniques
% As far as the server is concerned, any local model received can be from a potential malicious client. 
% FLTrust~\cite{cao2020fltrust} assumes the server acts as a client, i.e., trains a local model on an additional {\em trustworthy} dataset at the server's end and compares it against all the local models from other clients. 
% This way, the server can rely on some ``source of trust'' when discarding potentially malicious clients.
%\\
% Limitations of existing Byzantine-resilient strategies
Unfortunately, existing defense mechanisms either rely on simple heuristics (e.g., Trimmed Mean and FedMedian by~\cite{yin2018icml}) or need strong and unrealistic assumptions to work effectively (e.g., foreknowledge or estimation of the number of malicious clients in the FL system, as for Krum/Multi-Krum~\cite{blanchard2017nips} and Bulyan~\cite{mhamdi2018pmlr}, which, however, cannot exceed a fixed threshold).
Furthermore, outlier detection methods using K-means clustering~\cite{shen2016acm} or spectral analysis like DnC~\cite{shejwalkar2021ndss} do not directly consider the temporal evolution of local model updates received.
Finally, strategies like FLTrust~\cite{cao2020fltrust} require the server to collect its own dataset and act as a proper client, thereby altering the standard FL protocol.
\\
% OLD, LONG VERSION
% Overall, existing Byzantine-resilient strategies are either simple heuristics (e.g., FedMedian) or, if they are more complex, they rely on strong and unrealistic assumptions to work effectively (e.g., knowing the number of malicious clients in the FL system in advance, as for Krum and alike).
% Furthermore, data-driven outlier detection methods do not consider the temporary evolution of local model updates received (e.g., K-means clustering). 
% Finally, strategies like FLTrust requires the server to collect its own dataset and act as a proper client, thereby altering the standard FL protocol.
%
% Description of the proposed method
This work introduces a novel pre-aggregation \textit{filter} robust to untargeted model poisoning attacks. Notably, this filter $(i)$ operates without requiring prior knowledge or constraints on the number of malicious clients and $(ii)$ inherently integrates temporal dependencies. 
The FL server can employ this filter as a preprocessing step before applying \textit{any} aggregation function, be it standard like FedAvg or robust like Krum or Bulyan.
Specifically, we formulate the problem of identifying corrupted updates as a multidimensional (i.e., matrix-valued) time series anomaly detection task. 
The key idea is that legitimate local updates, resulting from well-calibrated iterative procedures like stochastic gradient descent (SGD) with an appropriate learning rate, show \textit{higher predictability} compared to malicious updates. This hypothesis stems from the fact that the sequence of gradients (thus, model parameters) observed during legitimate training exhibit regular patterns, as validated in Section~\ref{subsec:intuition}. %until convergence. 
%This regularity may be more pronounced for smooth convex loss functions, but it can still be captured within an appropriate time window, even for more complex and convoluted loss surfaces. 
%We provide evidence of this claim in Appendix~B, where we show that the average mutual information (i.e., ``predictability''), calculated over pairs of legitimate model updates sent at different FL rounds, is significantly higher than the corresponding computation for a malicious client.
\\
Inspired by the matrix autoregressive (MAR) framework for multidimensional time series forecasting~\cite{chen2021je}, we propose the FLANDERS ({\em \textbf{F}ederated \textbf{L}earning meets \textbf{AN}omaly \textbf{DE}tection for a \textbf{R}obust and \textbf{S}ecure}) filter.
The main advantages of FLANDERS over existing strategies like FLDetector~\cite{zhao2020multivariate} are its resilience to large-scale attacks, where $50\%$ or more FL participants are hostile, and the capability of working under realistic non-iid scenarios.
We attribute such a capability to two key factors: $(i)$ FLANDERS works without knowing a priori the ratio of corrupted clients, and $(ii)$ it embodies temporal dependencies between intra- and inter-client updates, quickly recognizing local model drifts caused by evil players. Below, we summarize our main contributions:

\begin{itemize}
\item[{\em(i)}]
We provide empirical evidence that the sequence of models sent by legitimate clients is more predictable than those of malicious participants performing untargeted model poisoning attacks.
\\
\item[{\em(ii)}] 
We introduce FLANDERS, the first pre-aggregation filter for FL robust to untargeted model poisoning based on multidimensional time series anomaly detection.
\\
\item[{\em(iii)}] 
We integrate FLANDERS into Flower,\footnote{\scriptsize{\url{https://flower.dev/}}} a popular FL simulation framework for reproducibility.
\\
\item[{\em(iv)}] 
We show that FLANDERS improves the robustness of the existing aggregation methods under multiple settings: different datasets, client's data distribution (non-iid), models, and attack scenarios.
\\
\item[{\em(v)}] 
We publicly release all the implementation code of FLANDERS along with our experiments.\footnote{\scriptsize{\url{https://anonymous.4open.science/r/flanders_exp-7EEB}}}
\end{itemize}

% Paper's structure and organization
The remainder of the paper is structured as follows. %some related work and the current state-of-the-art solutions to security issues that FL entails. 
Section~\ref{sec:background} covers background and preliminaries. 
In Section~\ref{sec:related}, we discuss related work.
Section~\ref{sec:problem} and Section~\ref{sec:method} describe the problem formulation and the method proposed. % to tackle it. 
Section~\ref{sec:experiments} gathers experimental results. %, and Section~\ref{sec:limitations} discusses some limitations of this work.
Finally, we conclude in Section~\ref{sec:conclusion}.
 %discusses the limitations of this work and draws future research directions.
%reports conclusions and draws perspectives for future research directions.

%%%%%%% OLD %%%%%%%
%to overcome the resilience of Byzantine failures in distributed Stochastic Gradient Descent computations. 
% The strength of Krum is its time complexity, which is linear in the gradient dimension. 
% However, the robustness of the approach is guaranteed for gradient-based learning applications only when the majority of the clients are not compromised. 
% Besides, the aggregation mechanism of Krum, as well as that of similar methods, is robust from a coarse-grained perspective and does not provide solutions to errors and perturbations that may occur at inference time.
%A related approach to~\cite{blanchard2017nips} is the work of Su et al.~\cite{su2016dc}. Here, the authors propose an iterated approximate agreement to tackle a multi-layer scenario attacked by Byzantine agents. 
%However, the method works efficiently on the sole discrete context and it is inapplicable to continuous state environments.
%\gabri{Maybe, we should just talk about the main limitations of existing countermeasures without digging into their details (or, we can just mention Krum as this is the most popular one). I will move the description of all these methods to the Related Work section.}
%!TEX root = ../generative_by_rl.tex

\section{Inference as Goal-Conditioned Reinforcement Learning}
\label{sec:inference_as_GCRL}

We assume that we have access to a training set $\{\tx^1,\ldots,\tx^N\}=:\cD\subseteq\cX$ of samples from some unknown distribution. Our goal is to perform inference on this training set where the candidates probability distributions are the final state distribution of a policy in a certain episodic MDP.

\paragraph{Markov Decision Problem (MDP)} We consider a loss-free episodic MDP $\cM= (\cX,\cY,H,p)$ where $\cX$ is the set of states (that also contains the training set $\cD$), $\cY$ the action set, $H$ the number of steps, $p_h(x'|x,y)$ is the transition probability from state~$x$ to state~$x'$ by taking the action $y$ at step $h\in[H]$, where $[H] := \{1,\dots,H\}$.

\paragraph{Policy, reach probability and value function}  A policy $\pi$ is a collection of functions $\pi_h : \cX \to \Delta(\cY)$ for all $h\in [H]$, where every $\pi_h$  maps each state to a probability distribution over actions.

Under policy $\pi$ a trajectory is generated as follows: an initial state $x_1 \sim p_0$ is sampled. Then for $h\in[H]$, given the current sate $x_h$, the agent samples an action $y_h\sim\pi_h(\cdot|x_h)$ and the next state $x_{h+1}\sim p_h(\cdot|x_h,y_h)$ is generated according to the transition probability. We denote by $p_h^\pi(x)$ the probability to reach the state $x$ in the MDP $\cM$ at step $h$ under the policy $\pi$. Similarly, we denote by $p^\pi(\tau)$ the probability distribution of a trajectory $\tau = x_1,a_1,\ldots,x_H,a_H, x_{H+1}$
under the policy $\pi$. Given two policies $\pi$ and $\pi'$ the Kullback-Leibler divergence between $p^\pi$ and $p^{\pi'}$ is, by the chain rule,
\[
\KL(p^\pi, p^{\pi'}) = \E^\pi \left[ \sum_{h=1}^H \KL\!\big(\pi(x_h),\pi'(x_h)\big) \right]\,,
\]
where $\E_\pi$ is the expectation under $p^\pi$.


% , receives a loss $\ell_h = \ell_h(x_h,y_y)$
% The value functions of policy $\pi$ is
% \[
% V^\pi = \E_\pi\left[ \sum_{h=1}^H  \ell_h\right]\,,
% \]
% where $\E_\pi$ is the expectation over trajectory generated with the policy $\pi$ in the MDP $\cM$.

\paragraph{Upper bound the negative log-likelihood} We assume as probabilistic model for the training set the reach probability of the last step $p_{H+1}^\pi(\cdot)$ \emph{parameterized by the policy} $\pi$. We then want to find the policy that minimizes the negative log-likelihood of the training set $L(\pi)$ where
\[
L(\pi) := \frac{1}{N}\sum_{\tx\in\cD} \log \frac{1}{p_{H+1}^\pi(\tx)}\,.
\]
Solving this optimization problem is difficult because the probability $p_{H+1}^\pi(\tx)$ is typically intractable. Similarly to variational inference~\citep{blei2017}, we will instead minimize \emph{an upper bound} on the negative log-likelihood.
For a fixed element $\tx\in\cD$, conditioned on the state-action pair $(x_H,y_H)$ it holds that
\begin{align}
    \log \frac{1}{p_{H+1}^\pi(\tx)} &= -\log\Big( \E_\pi \big[p_{H}(\tx|x_{H}, y_{H}) \big] \Big)\nonumber\\
     &\leq \!\min_{\pi'} \E_{\pi'}\! \left[\log\frac{1}{p_{H}(\tx|x_{H}, y_{H})}\right]\! +\KL(p^{\pi'},p^{\pi})  \,. \label{eq:ub_negloglikelihood}
\end{align}
Equation~\eqref{eq:ub_negloglikelihood} follows from the variational
formula for the moment generating function (see Lemma 1 in Appendix B) and it defines an upper-bound on the negative log-likelihood of $\tx$.

We can then define a surrogate loss parameterized by a policy $\pi$ and a family of goal-conditioned policies $(\pi^{\tx})_{\tx\in\cD}$, indexed by an element of the training set as:
\begin{align}
  \Lub(\pi, (\pi^{\tx})_{\tx\in\cD} ) &:= \frac{1}{N}\sum_{\tx\in\cD}  \E_{\pi^{\tx}} \left[\log \frac{1}{p_{H}(\tx|x_{H}, y_{H})}\right]+ \KL(p^{\pi^{\tx}},p^{\pi})\label{eq:def_surrogate}\,.
\end{align}
Using the fact that $\Lub$ is an upper-bound on the loss $L$ our problem becomes:
\begin{align*}
\min_{\pi}L(\pi)&\leq \min_{\pi}\min_{(\pi^{\tx})_{\tx\in\cD}} \Lub(\pi,(\pi^x)_{x\in\cX}) \\
&= \min_{\pi} \frac{1}{N}\sum_{\tx\in\cD}  \min_{\pi^{\tx}} \E_{\pi^{\tx}} \left[\log \frac{1}{p_{H}(\tx|x_{H}, y_{H})}\right]+ \KL(p^{\pi^{\tx}},p^{\pi})\,.
\end{align*}
% Our goal in this paper is to minimize this surrogate loss $\Lub$.
At a high level the minimization of $\Lub$ goes as follows. For each point $\tx\in\cD$ in the training set with goal-conditioned RL we learn a policy $\pi^{\tx}$ not too far from $\pi$ that leads to $\tx$ with high probability. Simultaneously we train the policy $\pi$ to reproduce trajectories from the policies $\pi^{\tx}$.


\paragraph{Inner minimization: goal-conditioned RL} If we fix the policy $\pi$ and a state $\tx\in\cD$ in the training set then solving~\eqref{eq:ub_negloglikelihood} is equivalent to solving a regularized \emph{goal-conditioned RL problem}. That is, we want to find a policy $\pi^{\tx}$ close to $\pi$ that lead to the goal $\tx$ with high probability.
Let
\begin{equation}
  \label{eq:def_loss}
  \ell_h^{\tx}(x,y) := \begin{cases}
  \log\dfrac{1}{p_{H}(\tx|x, y)} &\text{ if }h=H\\
  0 &\text{otherwise}
\end{cases}\,,
\end{equation}
be a loss function parameterized by the goal $\tx$, and let $\cM^{\tx} = (\cX,\cY,H,p,\ell^{\tx})$ be an MDP.
Then the minimization of~\eqref{eq:ub_negloglikelihood} is equivalent to solving the MDP $\cM^{\tx}$ regularized by the policy $\pi$. Precisely, we have
\begin{align*}
\min_{\pi^{\tx}}\E_{\pi^{\tx}} \left[\log\frac{1}{p_{H}(\tx|x_{H}, y_{H})}\right]\! +\!\KL(\pi^{\tx},p^{\pi})
\!= \min_{\pi^{\tx}} V^{\pi^{\tx},\tx}+ \KL(\pi^{\tx},p^{\pi})\,,
\end{align*}
where $V^{\pi^{\tx},\tx}$ is the value of the goal-conditioned policy $\pi^{\tx}$ in the MDP $\cM^{\tx}$. Although this goal-conditioned RL problem has already been studied by \citet{rudner2021outcome} and \citet{attias03a}, it is the first time to the best of our knowledge that this formulation is used as an intermediate task to build a generative model.
The link with variational inference is clear: we seek a policy $\pi'$ that approximates well the posterior distribution of a trajectory generated by $\pi$ conditioned on the fact that this trajectory reaches the goal $\tx$.

\paragraph{Outer minimization: supervised learning} Now, if we fix the family of goal-conditioned policies, minimizing the surrogate loss $\Lub(\cdot,(\pi^{\tx})_{\tx\in\cD})$ over the policy $\pi$ amounts to minimizing a convex combination of Kullback-Leibler divergences
\begin{equation}
  \label{eq:supervised_loss}
\argmin_\pi \Lub(\pi, (\pi^{\tx})_{\tx\in\cD}) = \argmin_\pi  \frac{1}{N}\sum_{\tx\in\cD} \KL(p^{\pi^{\tx}},p^{\pi})
\end{equation}
This optimization problem can be efficiently solved with supervised learning.
First, we sample goals according the the empirical distribution of the training set.
Then, for each goal, we generate a trajectory with the goal-conditioned policy, and collect state-action pairs.
We then use these state-action pairs to supervise the policy $\pi$.

\paragraph{Variational agent} In the sequel we call the pair $(\pi,(\pi^{\tx})_{\tx\in\cD})$ a variational agent (V-agent) made of
$\pi$ a supervised agent (S-agent) and $(\pi^{\tx})_{\tx\in\cD}$ a goal-conditioned agent (GC-agent).

\begin{algorithm}
\footnotesize
    \caption{$k$-SALSA}
    \label{alg:overall}
    \textbf{Input:} Private dataset $X=(x_1,\dots,x_n)$, auxiliary dataset $X_0$ for GAN model training, integer $k>1$ (assume $n = mk$ for integer $m$ without loss of generality), number of iterations $T$, loss ratio parameter $\lambda$ \\
    \textbf{Output:} Synthetic dataset $\tilde{X}$ of size $m$ with $k$-anonymity

    \begin{algorithmic}[1]
        \State Train a GAN generator $G$ and a GAN inversion encoder $E$ on $X_0$
        \State Obtain latent code $w_i = E(x_i)$ for each $i\in [n]$ and let $W=\{w_i\}_{i=1}^n$
        \State $(C_1,\dots,C_m)= \textsf{SameSizeClustering}(W, k)$  \Comment{$C_j\subset W$, $|C_j|=k$, $|C_j\cap C_{j'\neq j}|=0$, $\forall j$}
      
        \State Initialize $\tilde{X} = \emptyset$
        \For {each cluster $j\in [m]$}
        \State Let $C_j = (w_1',\dots,w_k')$, and $x_i'$ the original image of $w_i'$ for each $i$
        \State Compute $w_0 = \frac{1}{k} \sum_{i=1}^{k} w'_i$ and generate $x_0 = G(w_0)$
        \State Initialize $w_\text{avg}^{(0)} = w_0$
        \For {each iteration $t \in [T]$}
            \State Generate $x_\text{avg}^{(t-1)} = G(w_\text{avg}^{(t-1)})$
            \State Compute content loss $\mathcal{L}_\text{content}(x_0, x_\text{avg}^{(t-1)})$ using Eq.~\ref{eq:loss-content} 
            \State Compute local style alignment loss $\mathcal{L}_\text{style}((x'_1,\dots,x'_k), x_\text{avg}^{(t-1)})$ using Eq.~\ref{eq:loss-style}
            \State Compute total loss $\mathcal{L}_\text{total}=\lambda \mathcal{L}_\text{content}+(1-\lambda)\mathcal{L}_\text{style}$
            \State Update $w_{\text{avg}}^{(t)}$ using $w_{\text{avg}}^{(t-1)}$ and the gradient $\nabla_{w_\text{avg}^{(t-1)}} \mathcal{L}_\text{total}$
        \EndFor
        \State Add $G(w_{\text{avg}}^{(T)})$ to $\tilde{X}$
        \EndFor
    \State    \Return $\tilde{X}$

    \end{algorithmic}
\end{algorithm}

We present in section~\ref{ssec:faces} an application of PnP-HVAE on face images, using a pretrained state-of-the-art hierarchical VAE. 
Next, we study the application of our framework to natural images. To that end, we introduce  in section~\ref{ssec:patchVDVAE}  a patch hierachical VAE architecture, that is able to model natural images of different resolutions. In section~\ref{ssec:app_nat}, we provide deblurring, super-resolution and inpainting experiments to demonstrate the relevance of the proposed method.

Additional results are presented in Appendix~\ref{app:add}. All experiments can be reproduced using the code available at \url{https://github.com/jprost76/PnP-HVAE}.



\subsection{Face Image restoration (FFHQ)}\label{ssec:faces}
We first demonstrate the effectiveness of PnP-HVAE on highly structured data, by performing face image restoration.
Latent variable generative models can accurately model structured images such as face images \cite{karras2019style,vahdat2020nvae,child2021very,kingma2018glow}, and then be used to produce high quality restoration of such data. 
In our experiments, we use the VDVAE model of~\cite{child2021very}, pre-trained on the FFHQ dataset~\cite{karras2019style}, as our hierarchical VAE prior.
VDVAE has $L=66$ latent variable groups in its hierarchy and generates images at resolution $256\times256$.

We compare PnP-HVAE with the intermediate layer optimization algorithm (ILO)~\cite{daras2021intermediate} that is based on a different class of generative models than HVAE. ILO is a GAN inversion method which optimizes the image latent code along with the intermediate layer representation of a StyleGAN to generate an image consistent with a degraded observation.
We use the official implementation of ILO, along with a StyleGAN2 model~\cite{karras2020analyzing, stylegan2pytorch}, that was trained for 550k iterations on images of resolution $256\times256$ from FFHQ.  
As VDVAE and StyleGAN models are not trained on the same train-test split of FFHQ, we chose to evaluate the methods on a subset of 100 images from the CelebA dataset~\cite{liu2018large}. 
For super-resolution, the degradation model corresponds to the application of a gaussian low-pass filter followed by a $\times 4$ sub-sampling, and the addition of a gaussian white noise with $\sigma=3$.
For the deblurring, we considered motion blur and  gaussian kernels, both with a noise level $\sigma=8$. %

We provide quantitative comparisons in table~\ref{table:comp_ILO}, along with a visual comparison of the results in figure~\ref{fig:face_restoration}.
PnP-HVAE has the best  PSNR and SSIM results for all the considered restoration tasks, while ILO provides better results  for the perceptual distance.
By jointly optimizing the image and its latent variable, PnP-HVAE provides  results that are both realistic and consistent with the degraded observation.
On the other hand,  ILO  only optimizes on an extended latent space. This method generates  sharp and realistic images with better LPIPS scores,   
but the results lack  of consistency with respect to the observation, which explains the overall lower PSNR performance. 






\subsection{PatchVDVAE: a HVAE for natural images}\label{ssec:patchVDVAE}
Available generative models in the literature operate on images of  fixed resolutions and
are either restrained to datasets of limited diversity, or even to registered face images~\cite{kingma2018glow,child2021very, vahdat2020nvae, karras2019style}, or requiring additional class information~\cite{brock2018large, dhariwal2021diffusion, song2020score, luhman2022optimizing}.
Fitting an unconditional model on natural images appears to be a more difficult task, as their resolution can change, and their content is highly diverse.
The complexity of the problem can be reduced by learning a prior model on patches of reduced dimension. 
For image restoration problems, the patch model can be reused on images of higher dimensions~\cite{zoran2011learning,prost2021learning,altekruger2022patchnr}. When the model is a full CNN, the prior on the set of the  patches can  be computed efficiently by applying the network on the full image~\cite{prost2021learning}.

We thus introduce  patchVDVAE, a fully convolutional hierarchical VAE.
Contrary to existing HVAE models whose resolution is constrained by the constant tensor at the input of the top-down block, patchVDVAE can generate images of different resolutions by controlling the dimension of the input latent. 
This amounts to defining a prior on patches whose dimension corresponds to the receptive field of the VAE. A similar model is used for image denoising in~\cite{prakash2021interpretable}.

 
For PatchVDVAE architecture, we use the same bottom-up and top-down blocks as VDVAE~\cite{child2021very}, and replace the constant trainable input in the first top-down block by a latent variable, to make the model fully convolutional (details on the  architecture are given in Appendix~\ref{app:details}). 
The training dataset is composed of $128\times 128$ patches extracted from a combination of DIV2K~\cite{agustsson2017ntire} and Flickr2K~\cite{Lim_2017_CVPR_workshops} datasets.
We perform data augmentation by extracting  patches at $3$ resolutions: HR-images and $\times 2$ and $\times 4$ downscaled images. 
The model is trained for $7.10^5$ iterations with a batch size of $64$. Following the recommendation of~\cite{hazami2022efficient}, we use Adamax optimizer with an exponential moving average and gradient smoothing of the variance.
We set the decoder model to be a gaussian with diagonal covariance, as in~\cite{luhman2022optimizing}.
PatchVDVAE is fully convolutional and can generate images of dimension that are multiples of $64$ as illustrated by
figure~\ref{fig:vdvae}.

\newlength{\patchwidth}
\setlength{\patchwidth}{0.135\columnwidth}
\begin{figure}[!ht]
    \centering
    \begin{subfigure}[t]{.34\columnwidth}\hspace{0.1cm}
        \setlength{\tabcolsep}{0.02pt}
\renewcommand{\arraystretch}{0}
        \begin{tabular}{*{2}{p{1.03\patchwidth}}}
            \includegraphics[width=\patchwidth]{figures_arxiv/patchVDVAE/samples/generated/64x64/setup-5-image-0018.png} &
            \includegraphics[width=\patchwidth]{figures_arxiv/patchVDVAE/samples/generated/64x64/setup-5-image-0016.png} \\
            \includegraphics[width=\patchwidth]{figures_arxiv/patchVDVAE/samples/generated/64x64/setup-5-image-0008.png} &
            \includegraphics[width=\patchwidth]{figures_arxiv/patchVDVAE/samples/generated/64x64/setup-5-image-0019.png}   
        \end{tabular}
    \end{subfigure}\hspace{-0.15cm}
    \begin{subfigure}[t]{.64\columnwidth}
\begin{tabular}{cc}\vspace{-0.1cm}
\includegraphics[width=2\patchwidth]{figures_arxiv/patchVDVAE/samples/generated/256x256/setup-2-image-0009.png}&
        \includegraphics[width=2\patchwidth]{figures_arxiv/patchVDVAE/samples/generated/256x256/setup-2-image-0002.png}\end{tabular}

    \end{subfigure}
    \caption{\label{fig:vdvae} Left: $64\times64$ patches samples from our patchVDVAE model trained on patches from natural images.
    Right: PatchVDVAE is fully convolutional and it can generate images of higher resolution (here: $128\times128$).\vspace{-0.2cm}}
\end{figure}

\subsection{Natural images restoration}\label{ssec:app_nat}
We  evaluate PnP-HVAE on natural image restoration.
For each task, we report the average value of the PSNR, the SSIM, and the LPIPS metrics on $20$ images from the test set of the BSD dataset~\cite{MartinFTM01}.\\


\noindent
{\bf Image deblurring.}
In the experiments, we consider $2$ gaussian kernels and $2$ motion blur kernels from~\cite{levin2009understanding}, with $3$ different noise levels 
$\sigma \in \{2.55, 7.65, 12.75\}$.
As a baseline we consider  EPLL~\cite{zoran2011learning}, which learns a prior on image patches with a gaussian mixture model.
We also compare PnP-HVAE  with PnP-MMO and GS-PnP, $2$ competing convergent Plug-and-Play methods based on CNN denoisers.
PnP-MMO~\cite{pesquet2021learning} restricts the denoiser to be contraction in order to guarantee the convergence of the PnP forward-backard algorithm. GS-PnP~\cite{hurault2022gradient} considers a gradient step denoiser and reaches state-of-the-art performances of non converging methods~\cite{zhang2021plug}.
We set the temperature $\tau$  in our method as $0.95$, $0.8$ and $0.6$ for noise levels $2.55$, $7.65$ and $12.75$ respectively, and we let it run for a maximum of $50$ iterations. 
For the three compared methods we use the official implementations and pre-trained models provided by the respective authors. 
Details on the choice of hyperparameters for the concurrent methods are provided in the Appendix~\ref{app:details}
Figure~\ref{fig:deblurring_bsd} illustrates that our method provides correct deblurring results. 

According to table~\ref{tab:deb}, the performance of PnP-HVAE is between those of EPLL and GS-PnP and it outperforms PnP-MMO for large noise levels.\\

\begin{table}
\begin{center}\footnotesize
    \begin{tabular}{>{\centering}m{.3cm}*{5}{c}}
    $\sigma$ &Method & PSNR$\uparrow$ & SSIM$\uparrow$ & LPIPS$\downarrow$  \\ 
    \hline
    \multirow{4}{*}{\vcell{$2.55$}}
    & PnP-HVAE & $27.75$ & $0.79$ & $0.31$\\
    & GS-PNP \cite{hurault2022gradient} & $\mathbf{29.59}$ & $\mathbf{0.84}$ & $\mathbf{0.22}$\\
    & EPLL \cite{zoran2011learning} & $26.49$ & $0.71$ & $0.36$\\ 
    & PnP-MMO \cite{pesquet2021learning} & $\underbar{29.50}$ & $\underbar{0.83}$ & $\underbar{0.20}$ \\ \hline
    \multirow{4}{*}{\vcell{$7.65$}}
    & PnP-HVAE & $\underbar{26.36}$ & $\underbar{0.72}$ & $\underbar{0.40}$\\
    & GS-PNP \cite{hurault2022gradient} & $\mathbf{27.33}$ & $\mathbf{0.77}$ & $\mathbf{0.31}$\\
    & EPLL \cite{zoran2011learning} & $24.04$ & $0.66$ & $0.45$ \\ 
    & PnP-MMO \cite{pesquet2021learning} & $25.34$ & $0.69$ & $0.34$\\
    \hline
    \multirow{4}{*}{\vcell{$12.75$}}
    & PnP-HVAE & $\underbar{25.12}$ & $\mathbf{0.73}$ & $\underbar{0.47}$\\
    & GS-PNP \cite{hurault2022gradient} & $\mathbf{26.32}$ & $\mathbf{0.73}$ & $\mathbf{0.37}$\\
    & EPLL \cite{zoran2011learning} & $23.28$ & $0.61$ & $0.51$ \\ 
    & PnP-MMO \cite{pesquet2021learning} & $22.42$ & $0.53$& $0.54$ \\
    \hline
    &\vspace*{-.3cm}\\
            \multicolumn{2}{c}{Blur and motion kernels}& \multicolumn{3}{c}{
        \includegraphics*[scale=1]{figures_arxiv/kernels/4.png}\;\includegraphics*[scale=1]{figures_arxiv/kernels/7.png}\;\includegraphics*[scale=1]{figures_arxiv/kernels/9.png}\;\includegraphics*[scale=1]{figures_arxiv/kernels/11.png}} 
    \end{tabular}
        \caption{\label{tab:deb}Comparison  of PnP-HVAE  and other restoration methods on deblurring. Results are averaged on $4$ kernels.\vspace{-0.2cm}}% on image deblurring.}
    \end{center}
\end{table}

\begin{figure}
    
    \begin{subfigure}[h]{\linewidth}
        \centering
        \includegraphics*[width=\columnwidth]{figures_arxiv/deb_s255_k7.pdf}\vspace{-0.1cm}
        \caption{Gaussian blur, $\sigma=2.55$}
    \end{subfigure}
    \begin{subfigure}[h]{\linewidth}
        \centering
        \includegraphics*[width=\columnwidth]{figures_arxiv/deb_s765_k11.pdf}\vspace{-0.1cm}
        \caption{Motion blur, $\sigma=7.65$}
    \end{subfigure}\vspace*{-0.1cm}
    \caption{\label{fig:deblurring_bsd} Natural image deblurring\vspace{-0.1cm}}
\end{figure}

\noindent {\bf Effect of the temperature.}
PnP-HVAE gives control on the temperature of the prior over the latent space.
In figure~\ref{fig:temp_effect}, we illustrate that reducing the temperature increases the strength of the regularization prior. In this example the tuning $\tau=0.7$ produces the best performance.\\
\begin{figure}[!ht]
   
    \includegraphics[width=\columnwidth]{figures_arxiv/demo_temp.pdf}\vspace{-0.15cm}
    \caption{ \label{fig:temp_effect} Effect of the temperature in PnP-VAE on a deblurring problem, with $\sigma=7.65$.\vspace{-0.15cm}}
\end{figure}


\noindent
{\bf Image inpainting.}
Next we consider the task of noisy image inpainting. 
We compose a test-set of 10 images from the validation set of BSD~\cite{MartinFTM01} and we create masks
  by occluding diverse objects of small size in the images. 
A gaussian white noise with $\sigma=3$ is added to the images.
As a comparaison, we still consider GS-PnP and EPLL.
For PnP-HVAE, the temperature is set to $\tau=0.6$, and the algorithm is run for a maximum of $200$ iterations, unless the residual $||\x_{k+1}-\x_k||$ is on a plateau.
We provide on Table~\ref{tab:inpainting_bsd} the distortion metrics with the ground truth, as well as a visual
\begin{table}



\begin{center}
    \begin{tabular}{cccc}
        & PSNR$\uparrow$ & SSIM$\uparrow$ &LPIPS$\downarrow$ \\\hline
        PnP-HVAE  & $\mathbf{29.54}$ & $\mathbf{0.93}$ & $\mathbf{0.06}$\\
        GS-PNP & $28.52$ & $\mathbf{0.93}$ & $0.09$\\
        EPLL & $\underline{29.16}$ & $\mathbf{0.93}$ & $\mathbf{0.06}$\\
    \end{tabular}
    \caption{\label{tab:inpainting_bsd}Quantitative evaluation for inpainting on BSD.}
    \end{center}
\end{table}
comparison on figure~\ref{fig:inpainting_bsd}. 
With its hierarchical structure,  PnP-HVAE outperforms the compared methods. \vspace{0.05cm}



\begin{figure}[!h]
    \includegraphics[width=\columnwidth]{figures_arxiv/demo_inp_bsd2.pdf}\vspace{-0.1cm}
    \caption{\label{fig:inpainting_bsd}Natural image inpainting\vspace{-0.3cm}}
\end{figure}











\section{Conclusion}\label{sec:conclusion}
In this work, we focus on addressing the fundamental challenge of OOD detection tasks, which is how to fully understand the semantic discrepancy between the ID/OOD samples. We reveal that the key to success in the realistic SCOOD task is to allocate as many ID samples in the unlabeled set correctly as possible. To this end, we propose a novel uncertainty-aware optimal transport scheme that introduces class-specific energy scores as guidance for effective label assignment. Experimental results show that our method achieves better performance than previous state-of-the-art methods on SCOOD benchmarks.

\textbf{Limitations.} In addition to temperature scaling, other techniques such as feature clipping applied in ReAct~\cite{sun2021react} also enhance the performance of energy score, so how to obtain an OOD score that best fits the SCOOD task can be further explored. Moreover, a setting highly related to SCOOD has been proposed in \cite{katz2022training} and formulated as a constrained optimization problem. We will also theoretically analyze these practical OOD settings in our feature work.

% \section*{Acknowledgments}
\textbf{Acknowledgments.} 
This work is supported by National Key R\&D Program of China under Grant 2020AAA0105701, National Natural Science Foundation of China (NSFC) under Grants 61872327, Major Special Science and Technology Project of Anhui, National Natural Science Foundation of China (62033012) and Ant Group through Ant Research Intern Program.




% Acknowledgements should go at the end, before appendices and references

\acks{This work was granted access to the HPC/AI resources of IDRIS under the allocation 2022-AD011011232R2  made by GENCI. We gratefully acknowledge the support of the Centre Blaise Pascal's IT test platform at ENS de Lyon (Lyon, France) The platform operates the SIDUS solution \citep{quemener2013sidus} developed by Emmanuel Quemener. Pierre M\'enard acknowledges the support of the Chaire SeqALO (ANR-20-CHIA-0020-01).}

\vskip 0.2in
\bibliography{generative_by_rl-bib}

% Manual newpage inserted to improve layout of sample file - not
% needed in general before appendices/bibliography.

\newpage
\appendix
%!TEX root = ../generative_by_rl.tex

\section{Derivations of Section~\ref{sec:inference_as_GCRL}}
\label{app:proofs_derivations}

In this appendix we detail the missing derivations of Section~\ref{sec:inference_as_GCRL}.

\begin{lemma}
For any policy $\pi$, state $x$,
\label{lem:ub_neg_loglikelihood}
\[-\log\Big( \E_\pi \big[p_{H}(x|x_{H}, y_{H}) \big] \Big)
 \leq  \min_{\pi' \text{ policy}} \E_{\pi'} \left[\log\frac{1}{p_{H}(x|x_{H}, y_{H})}\right] +\KL(p^{\pi'},p^{\pi})\,.\]
\end{lemma}
\begin{proof}
Thanks to the Donsker-Varadhan's formula \citep{donsker1983} it holds
\[
-\log\Big( \E_\pi \big[p_{H}(x|x_{H}, y_{H}) \big] \Big) = \min_{q\in \Delta(\mathrm{Traj})} \E_q\left[\log\frac{1}{p_{H}(x|x_{H}, y_{H})} \right] + \KL(q,p^\pi)\,,
\]
where $\Delta(\mathrm{Traj})$ is the set of probability distributions supported on the trajectory $\tau = (x_1,y_1,\ldots,x_H,y_H)$ up to step $H$ with $x_h\in\cX$ and $y_h\in\cY$. Restricting the infimum to probability distributions $q=p^{\pi'}$ induced by some policy $\pi'$ allows us to conclude.
\end{proof}

\newpage
%!TEX root = ../generative_by_rl.tex

\section{Network architectures}
\label{app:net_arch}
In this appendix we describe the architecture of the proposal, Q-value and selection networks.

\subsection{Proposal network}

The proposal network is parameterized as a four-layer \UNET\footnote{The \UNET implementation is publicly available at https://github.com/milesial/Pytorch-UNet.}.
Each layer has two blocks consisting of a convolution layer followed by group normalization and a ReLU activation function.

The different networks used in our model also depends on discrete time-step $h\in[H]$. We incorporate this information in the model as follows. At each time-step $h$ the proposal network receives an input of the form $(x_{h},h,a_h)$. The current state $x_h$ is of shape $w \times t \times c$, where $w$ is the width,
$t$ is the height, and $c$ is the number of channels
(for example, $c = 3$ for RGB images). The selected index $a_h$ indicates which of the $A$ available proposals is to be
chosen at the current time-step $h$.
We first calculate an embedding $\mathbf{h}_{a,h}$ representing $h$ and $a$ through an Embedding layer to
obtain a tensor the same size as $x_{h}$. We then concatenate $x_{h}$ and $\mathbf{h}_{a,h}$ on the channel axis to obtain a tensor of shape $w\times t\times 2c$ that is fed to the \UNET.
Finally, the \UNET outputs the tensor $x_{h+1}$ of shape $w\times t\times c$.
This process is described as follows
\begin{equation}
  \label{eq:propnet}
  \begin{aligned}
    h'_{a,h} &= Ah + a \\
    \mathbf{h}_{a,h} &= \mathtt{Emb}(h'_{a,h}) \\
    x_{h+1} &= \UNET([x_h, \mathbf{h}_{a,h}])\,,
  \end{aligned}
\end{equation}
where $[\cdot,\cdot]$ represents concatenation of 3D tensors on the channel axis. At each pass the network generates the proposal that corresponds to the specified index.
The output of each pass will then have the same shape as the inputs. We provide a detailed architecture of the network in Figure~\ref{fig:pnet}.

Note that, with this embedding of the time-step and selection the proposal network can memorize at most $H\times A$ distinct images. Thus combining proposals by choosing the selections in a trajectory is essential to be able to model a rich distribution.

\subsection{Selection network}

The selection network architecture is the same for both agents.
It consists of three blocks of convolution layers followed by a group normalization and a ReLU activation function.
The network outputs a tensor of size $\texttt{batch\_size}\times A$ corresponding to the selected actions for each
element in the batch. The embedding of the time-step follows the same method as in the proposal network.
Figure~\ref{fig:snet} presents a sketch of the architecture we used for our experiments.
The batch size was set to $128$, proposal size $A = 16$ and number of steps $H = 16$.

\subsection{Q-value selection network}

The architecture of the Q-value network is the same as the one of the selection network except for the last 
layer where the output is not normalized to obtain a probability distribution.
For the last layer we use a Dueling Network architecture \citep{wang2016dueling}.


\begin{figure}
  \includegraphics[width=\textwidth]{figures/propnet_torchviz}
  \caption{Proposal network architecture (shared between both agents) with batch size $128$ and number of steps $H = 16$.}
  \label{fig:pnet}
\end{figure}

\newpage
\begin{figure}
  \includegraphics[width=\textwidth]{figures/qnet_torchviz}
  \caption{Selection network architecture for the GC-agent and S-agent with batch size $128$ and selection size $A = 16$.}
  \label{fig:snet}
\end{figure}


\end{document}
