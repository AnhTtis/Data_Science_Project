For simplicity, we will focus on a box-shaped object with only one out of three dimensions that is within the gripper's graspable width. This is because it'd be easier to design an algorithm for a box as opposed to an irregularly shaped object. 

The pivoting task is divided into three sub-tasks. The first section covers visual object pose detection where the 6D pose of the box is estimated. The second part contains the generation of the grasp pose, which is be placed on a point along the top edge of the box based on a user input. Finally, closed loop gripper width control is used to grasp the object, and further regulates slip during the pivot motion. A force-based end-effector position controller is also used to amend a pre-planned end-effector trajectory, aiming to maintain contact between the grasped box and the surface. The control loop runs until robot is able to complete the pivoting task, or when all waypoints of the pre-planned path has been executed. This overall structure of the system is illustrated in Fig.\ref{fig: module}.


% \begin{figure*}[ht]
%     \centering\includegraphics[scale=0.28]{Pics/flowchart.png}
%     \caption{Flowchart of the system}
%     %the robot will generate a grasp pose from visual image captured by a camera, then try to pivot the object in an arc from its current pose up to a certain degrees either clockwise or anti-clockwise (depends on where the robot will grasp the object). The radius of the arc is determined from the object's dimensions.}
%     \label{fig: flowchart}
% \end{figure*}
\begin{figure}[ht!]
    \centering
    \includegraphics[width=1\linewidth]{Pics/sysdiag_akan.pdf}
    \caption{System diagram}
    \label{fig: module}
\end{figure}

\subsection{Object Pose Estimation}
\label{sec:box_pose_estimation}

An RGB-D side camera is used to detect the ArUco marker placed in the center of the largest surface of the box. Based on this marker detection as well as the box dimensions, the box pose can be inferred by offsetting the marker's pose by half of the box's respective dimension. The 6D pose of the box's axis of origin, is defined at its centre point with an orientation that aligns the axes so that they are parallel with the length, width and height of the box. 

\subsection{Grasp Pose Synthesis} 

From the box's pose, a feasible grasp pose will be generated using the known dimensions of the box. The gripper will grasp the box such that the gripper is perpendicular to the table surface, i.e. the gripper is pointing down into the table. The grasp pose is also positioned ensuring that all pillars of the tactile sensors would make contact with the box. Given that there is only one graspable dimension, the robot can only grasp along one surface of the box. The user is able to input a number to choose which end of the box should be grasped.  

The placement of the grasp pose on the top corner of the box prevents the box from being stopped by the gripper palm during the pivoting motion, and allows it to fit through the gap between the fingers. The grasp point relative to the box, as well as the orientation of the gripper, is maintained during the pivoting action. For a given grasp point, the pivot direction is always chosen such that the grasp point and pivot point are on opposite sides of the box. This is demonstrated in Fig.\ref{fig:intro}.
    


\subsection{Closed Loop Control}\label{closed_loop}
To successfully pivot the box we employ a robot end-effector position controller and a gripper controller. These controllers will adjust both the robot end-effector trajectory and the gripper width respectively throughout the pivoting motion.

%\par\vspace{\baselineskip}
%\begin{enumerate}
    %\item 
    \noindent\textbf{Initial Gripper Width:} Determining an adequately loose grip width is essential to enabling rotational slip to allow the box to rotate in-hand. To grasp the box, an initial grip width is first determined using tactile sensors and the graspable dimension of the box. This is calculated by dividing the length of the graspable dimension by the distance between the two tactile sensors. Additionally, to ensure that the gripper does not experience translational slip initially, the static friction force applied by the gripper must be greater than the force of the box on the gripper at the point where the robot pivots the box $d\theta$. 
    \begin{equation}
        F\textsubscript{static} = \mu\textsubscript{s}F\textsubscript{N}
    \end{equation}
    \begin{equation}
        F\textsubscript{static} > F(d\theta) 
    \end{equation}
    The force at $d\theta$ is taken as this is the instantaneous force on the gripper as the robot starts pivoting the box. The normal reaction force is found by the tactile sensor reading. This will ensure that the gripper is tight enough to start pivoting the box. Until this condition is met, the gripper will continue to close itself.
    
    % This can be seen in further detail in Fig.\mytodo{[TO DO]: Insert figure for gripper free body diagram}. 
     
    
    \noindent\textbf{Force-based Position Controller:} The robot force-based position controller controls and updates the arm's trajectory throughout the pivoting motion. This controller utilises the force/torque wrist sensor which provides force readings, and the RGB side camera which detects and estimates the box's rotation angle as the robot pivots the box. 
    
    We represent a complete pivoting motion with the pivot point of the object remaining in contact with the surface by an ideal force profile. This profile is derived through analysis of the forces applied to the object during pivoting. It varies with respect to the progress of the pivot, as indicated by the angle of rotation. The controller will then maintain contact by tracking this profile. The force exerted by the robot is controlled by applying a vertical offset to a pre-planned arc trajectory for the end-effector. 




    
    Initially, a Cartesian path consisting of 50 waypoints is generated using the MoveIt motion planning framework, that instructs the robot to move with an arc trajectory parameterised by the box's dimensions. 50 waypoints are used as it strikes a balance between time taken to complete the movement as well as maintaining the trajectory resolution to keep the arc shape. as a circular arc with radius as the distance between the grasp point and the pivot point. For a box grasped at the top corner, the radius would become its diagonal, as seen in Fig.\ref{fig:arc}. 

    \begin{figure}[!ht]
        \centering\includegraphics[width=0.7\linewidth]{Pics/arc4.pdf}
        \caption{Path Analysis: the box is pivoting from the solid line position to dashed line position, O is the pivoting center, P is the grasp point at the beginning of the pivoting and Q is the corresponding point of P after pivoting.}
        \label{fig:arc}
    \end{figure}
    
    To pivot the box from point $P$ to point $Q$, the position of that point's trajectory is:
    \begin{align}
        Position_x &= L - r \cdot cos(\theta)\\
        Position_y &= r \cdot sin(\theta) 
    \end{align}
    where $L$ is the length of the box, the arc radius $r$ is the diagonal of the box and $\theta$ can be calculated from the box's dimension. 
    
    % The robot will move according to this pre-generated plan, while the controller continuously checks that the box's rotation angle does not exceed the goal rotation angle, which is 90 degrees for a complete pivot, to ensure the box stops appropriately.  

    The analytical force profile was derived assuming equilibrium conditions during the pivot action, corresponding to a stationary configuration, or a constant speed rotation. In such a case, as illustrated in Fig.\ref{fig: force_profile}, torque applied on the box by the gravitational force $F_g$, with the centre of mass assumed to be at the centre of the box, is balanced by the forces applied to it from the pivoting motion $F_p$. 
    
    \begin{figure}[h]
        \centering\includegraphics[width=0.3\columnwidth]{Pics/force_profile_symbols.png}
        \caption{Free-body diagram of a pivoting box under equilibrium conditions}
        \label{fig: force_profile}
    \end{figure}
    
    The force that would be measured by the wrist sensor in the vertical (z) direction is derived as the the vertical component of pivoting force $F_p$, $f_p^z$:

    \begin{equation}
        f_p^z = \frac{F_g sin(\alpha)cos(\beta)}{2}
    \end{equation}

    The above equation can then be expressed with respect to the angle of rotation of the box, $\phi$, and specific properties of the box, including the angle between its base and diagonal, $\theta$, its mass $m$, as well as the gravitational constant $g$:

    \begin{equation}
        f_p^z = \frac{mg sin(\frac{\pi}{2} - \phi - \theta)cos(\phi + \theta)}{2}
    \end{equation}

    % To close the loop for dynamic pivot action, we use an analytical force profile in the vertical dimension to detect and ensure that the box remains in contact with the surface. 
    From empirical tests, the relationship between the height that the gripper lifts the box up to and the force measured by the wrist sensor during surface-contact pivoting was found to be roughly linear. Thus, to control the force applied to the box, a vertical offset is applied to the trajectory for every waypoint. 

    Using the above information, the position controller is implemented, and described in Alg.\ref{alg:traj}. At each waypoint, the ideal force $f_{ideal}$ is predicted using the rotation angle of the box, estimated using vision. The measured force $f_{real}$ is then compared to the ideal force to produce an error, and accumulated in $error_{acc}$. The vertical offset that should be applied is then calculated from the force error using a Proportional-Integral (PI) control scheme. A proportional constant $K_p$ is applied to the instantaneous error, and an integral constant $K_i$ is applied to the accumulated error. Both constants are empirically tuned. The offset is applied to all waypoints, and adjusted incrementally during each iteration. This aims to maintain the offset that minimises the error between the two force values. The pre-planned waypoint is updated by applying this offset to the vertical dimension, and re-planned so that the robot will move in accordance to the adjustment. If at any point the rotation angle is estimated to be greater than or equal to 90\degree, it is assumed that the pivot is complete and the program will terminate. 

    \begin{algorithm}
    \caption{Trajectory force-based position controller, using z as the vertical dimension}\label{alg:traj}
    \begin{algorithmic}
    \State $\mathit{offset} \gets 0$ 
    \State $error_{acc} \gets 0$
    \FOR{\textit{waypoint} \textbf{in} \textit{trajectory}}
        \State $\phi \gets $ vision\_detection
        \State $f_{real} \gets $ force\_torque\_sensor
        \State $f_{ideal} \gets $ ideal\_force\_profile($\phi$)
        \IF{$\phi < 90\degree$}
            \State $error \gets |f\_ideal - f\_real|$
            \State $error_{acc} \gets error_{acc} + error$
            \State $\mathit{offset} \gets \mathit{offset} + K_p \times error + K_i \times error_{acc}$
            \State $waypoint.z \gets waypoint.z + \mathit{offset}$
            \State move\_robot($waypoint$)
        \ELSE
            \State \textbf{break}
        \ENDIF
    \ENDFOR
    \end{algorithmic}
    \end{algorithm}
    

    Due to the assumption of equilibrium and the replanning of waypoints for each iteration, the robot is required to pause momentarily before each consecutive movement to the next waypoint. This causes a discontinuous, slow motion that becomes slower with more waypoints. 

    \textbf{Gripper Controller:} The slip of the object will be controlled to perform the pivoting movement. We first estimate the slip type by detecting either translational or rotational slip. We ignore the no slip case as for boxes that are heavier than a certain weight, the gripper cannot grasp the box without experiencing rotational or translational slip. During the process of pivoting, the desired slip is rotational slip and translational slip should be avoided. 
    
    To detect the type of slip, the 3D displacement data and contact estimation from the tactile sensors on the gripper fingers is used. We specifically utilise displacement in the direction parallel to the force of gravity on the box. If there is translational slip, the box will move in that direction relative to the tactile sensor. Again, we focus on the vertical dimension as gravity is the major factor causing slip. When all pillars that are detected to be in contact with the box show displacement values in the same direction (i.e. their displacement values share the same sign), the box is labelled to be in translational slip. This is in contrast to when the box undergoes rotational slip, where pillars would be displaced in different directions with opposing signs in displacement values, as the centre of rotation is roughly at the centre of the sensor. 
    
    If the tactile sensors detect translational slip, the gripper width will close slightly, whereas if the tactile sensors detect rotational slip, then the gripper width is unaltered. The gripper will also loosen its grip if at least one of the pillars experience a displacement of over 8mm, to prevent damage to the sensors and the box. 
 
%\end{enumerate}
