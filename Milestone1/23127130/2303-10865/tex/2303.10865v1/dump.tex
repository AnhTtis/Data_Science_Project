% intro
From the Approach,
\begin{enumerate}   
    \item Machine learning model name for slip detection
    \item Add one picture for the set up include the position of the second camera
    \item Not "Object segmentation anymore", use Aruco marker and tf transform to detect where is the object.
    \item State estimation, either from calculation or edge detection
    \item "Manipulation planner" is actually supposed to be motion planner?
\end{enumerate}

% A dragging action can be roughly divided into two distinct sub-tasks, with their respective sub-problems:
% \begin{enumerate}
%     \item \textbf{Lifting and pivoting:} When lifting the object, the gripper needs to apply enough force to grasp and lift the object, while ensuring the object remains in contact with the surface to prevent excessive load on the joint motors. Such contact can be maintained by allowing rotational slip to change the grasp on the object as the gripper lifts up. 
%     \begin{itemize}
%         \item Contact estimation: The contact status between the object and the surface need to be inferred from available sensor data. The area in contact, whether that is a surface, an edge or a vertex of the object can also assist the trajectory planning of the lifting motion. The primary tool would involve force data from a tactile sensor array, which has been widely used to derive information about the grasped object, such as that of contact and slip \cite{luo2017robotic, li2020review, toskov2022hand}. Visual and depth data can also be 
        
%         \item Adaptive lifting: It has been shown that by choosing the grasp position and surface contact point to be on opposite sides of the center of mass (CoM) of the object being pivoted, a stable configuration can be achieved \cite{hou2018fast}. Thus, to generalise the lifting algorithm to potentially unseen objects with an unknown mass distribution, estimation of its CoM by interaction with the object can assist the robot in reaching the pivoted position.  
%     \end{itemize}
    
%     \item \textbf{Drag planning:} The lifted object will be moved by motion planning with the robot arm's end effector. The same height will be kept throughout, confining the motion to 2D space. Obstacles in the environment also need to be considered and avoided by both the robot arm and the now rigidly grasped object.  
% \end{enumerate}


% related works


% Estimating the normal contact force of the object and the surface is useful for determining how much force is necessary for the gripper to apply to the object so that the robot can hold the object in a position ideal for dragging. Hou \textit{et al.} developed a 2D model that estimates the normal contact force when the object is pivoted, assuming quasi-static conditions \cite{hou2018fast}. Initially, it will also be useful to determine binary contact with the surface as for testing, objects that are lighter than the maximum rated payload of the robot will be used. Aiyama \textit{et al.} employed vision sensors to estimate friction qualitatively, by assigning different types of friction as events based on the object's reaction to the force applied by the robot fingers and the lifting of the object onto its edge \cite{aiyama1993pivoting}. For the purposes of this study, the detail of the contact is not as important. 


% Identifying the location of the center of mass is necessary for lifting the object stably. Most works use sensor-based methods to find the center of mass using vision, range and tactile sensors \cite{liu2022center}, \cite{kanoulas2018center}, \cite{feng2020center}. In our study, we will also use tactile sensors to determine the center of mass.


% Route planning is required if the robot is asked to efficiently move the object from the start position towards the end position.
% Unlike normal route planning, drag planning is about dragging an object from one position to another. This could potentially be one of the approaches for lifting heavy objects that exceeds the robot's payload. For heavy object manipulation, Zhang \textit{et al} proposed a new hierarchical motion planner to successfully move a bulky object using a dual-arm robot\cite{zhang2021manipulation}. Their work mainly focused on the reorienting of an object instead of moving it. In terms of the path planning with obstacles, many approaches such as heuristic functions\cite{tournassoud1987regrasping} and neural networks \cite{abdi2021novel} have been applied while most of the work is about manipulating and grasping regular weighted objects, few of them tested on pivoting the objects. In our work, we will experiment the combination of two problems, pivoting and dragging the object which has not been done by many researchers.



% refs

% @article{liu2022center,
%   title={Center-of-Mass-based Robust Grasp Pose Adaptation Using RGBD Camera and Force/Torque Sensing},
%   author={Liu, Shang and Wei, Xiaobao and Wang, Lulu and Zhang, Jing and Li, Boyu and Yue, Haosong},
%   journal={arXiv preprint arXiv:2205.01048},
%   year={2022}
% }

% @article{kanoulas2018center,
%   title={Center-of-mass-based grasp pose adaptation using 3d range and force/torque sensing},
%   author={Kanoulas, Dimitrios and Lee, Jinoh and Caldwell, Darwin G and Tsagarakis, Nikos G},
%   journal={International Journal of Humanoid Robotics},
%   volume={15},
%   number={04},
%   pages={1850013},
%   year={2018},
%   publisher={World Scientific}
% }

% @article{lumelsky1987effect,
%   title={Effect of kinematics on motion planning for planar robot arms moving amidst unknown obstacles},
%   author={Lumelsky, VLADIMIRJ},
%   journal={IEEE Journal on Robotics and Automation},
%   volume={3},
%   number={3},
%   pages={207--223},
%   year={1987},
%   publisher={IEEE}
% }
