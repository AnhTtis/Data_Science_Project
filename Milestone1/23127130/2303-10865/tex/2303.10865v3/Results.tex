



\subsection{Setup}
The proposed pivoting system was tested using a UR5 robot arm with the Robotiq 2F-85 gripper and a FT-300 Force/Torque sensor attached to the end effector. The fingers of the gripper have two Contactile PapillArray 3x3 tactile sensor arrays attached, capable of measuring 3D displacement, force, and torque \cite{khamis2018papillarray}. In some experiments involving high-force grasping, the tactile sensors are replaced with flat rubberised fingers to avoid damaging the sensors. The gripper has the tactile sensors mounted unless specified otherwise. The camera used is the Intel RealSense D435i, placed to the side of the robot. The hardware setup is shown in Fig. \ref{fig: set up}. 




\begin{figure}[t]
  \centering   
  \begin{overpic}[trim={0 520 0 0}, clip, width=0.96\linewidth]{Pics/Scene.jpeg}
     \put(2,16){\includegraphics[trim={220 300 200 0}, clip, frame, cfbox=white,  width=0.4\linewidth]{Pics/gripper.jpeg}}  
  \end{overpic}
\caption{Experimental Setup}
\label{fig: set up}
\end{figure}

Cardboard boxes with varying dimensions are used for the experiment, shown in Fig. \ref{fig: boxes}. Their mass and mass distributions are adjustable by adding iron weights inside.

\begin{figure}[t]
    \centering\includegraphics[trim={0 450 0 450}, clip, width=1\linewidth]{Pics/boxes.jpg}
    \caption{The cardboard boxes used for pivoting. From left to right: Small box (18x11x4cm, 1.27kg), Large box (23x16x5cm, 0.88kg), Long box (28x12x5cm, 1.72kg)}
    \label{fig: boxes}
\end{figure}

% \begin{figure}[t!]
%     \centering\includegraphics[trim={0 0 0 0}, clip, width=0.485\linewidth]{Pics/Scene.jpeg}
%     \includegraphics[trim={0 0 0 0}, clip, width=0.485\linewidth]{Pics/Scene.jpeg}
%     \caption{Experimental Setup}
%     \label{fig: set up}
% \end{figure}




\subsection{Procedure} 

A series of experiments are designed to verify the validity of our approach.  
The goal of all experiments is to complete a $90\degree$ rotation of the box without lifting it off the surface or letting the box slip out of the gripper's grasp. A simple pick-and-place method is  performed as a baseline. We compare five different methods, including the application of controllers introduced in Section \ref{closed_loop}, under four different conditions. Each applicable combination was repeated 10 times for each of the three boxes.

% In the baseline approach, the robot first lifts up the object completely and then rotates $90 \degree$ of the front joint, and completes pivoting after lowering it down. The gripper will be set to close as hard as possible for cases that don't have the gripper controller. 

\subsection{Methods}

\noindent\textbf{Pick\&Place:} Picking up a box from the centre of its top edge, the robot first lifts up the box completely and then rotates it in the air by $90\degree$, and completes pivoting after lowering it down. The gripper is set to close as hard as possible, with the tactile sensors replaced with flat, rubberised fingers. The lifted height before pivoting is set just high enough so that the robot can complete pivoting without hitting the surface, to represent the lowest effort situation using this method.

\noindent\textbf{Open Loop:} No controllers are used. The gripper is set to close as hard as possible and the tactile sensors are also replaced with flat, rubberised fingers. The arm's trajectory is planned as an ideal arc based on the box's dimensions.
 
\noindent\textbf{Vision:} A vision-based controller is used. The gripper is set to close as hard as possible with the tactile sensors replaced with flat, rubberised fingers. The arm's trajectory is planned as an ideal arc based on the box's dimensions and adjusted with offsets. However, the vertical offset is calculated from the estimated height of the box from the surface. 

\noindent\textbf{Gripper:} Only the gripper controller is used, both to grasp the object and to control slip during pivoting. The arm's trajectory is planned as an ideal arc, not updated during the movement.

\noindent\textbf{Force:} Only the force-based position controller is used. The gripper is set to close as hard as possible with the tactile sensors replaced with flat, rubberised fingers. The arm's path is planned as an ideal arc and updated during the movement using the error between measured force and predicted force. 

\noindent\textbf{Force+Gripper+Vision:} All three sensors are used. The gripper controller is used to grasp the object and control slip. Vision is used to track the rotation angle between the surface and the box's lowest edge for force estimation. The arm's path is planned as an ideal arc and updated during the movement by the force-based position controller, and the gripper width is updated by gripper controller. 


% \subsection{Conditions}
% \begin{enumerate}
%     \item \textbf{Long-to-Short Pivot:} The robot pivots a box from standing on its longer edge to standing on its shorter edge. 
%     % The box will start on its longest graspable surface. The arm's trajectory planned is ideal. 
%     \item \textbf{Long-to-Short Pivot with added noise:} Using the same pivoting method, the robot will pivot the box from standing on its longer edge. Noise is added to the initially planned arc trajectory by adding 5cm to the base dimension of the box.
%     % However, the arm's trajectory planned is not ideal, i.e. parameters will be fed into the arc trajectory planning that will offset it from an ideal trajectory. 
%     \item \textbf{Short-to-Long Pivot:} The robot pivots a box from standing on its shorter edge to standing on its longer edge.
%     %The arm's trajectory planned is ideal.
%     \item\textbf{Short-to-Long Pivot with added noise:} Using the same pivoting method, the robot will pivot the box from standing on its shorter edge. Noise is added to the initially planned arc trajectory by adding 5cm to the base dimension of the box.
% \end{enumerate}



\subsection{Independent Variables}
\begin{enumerate}
    \item \textbf{Pivoting types:} The robot performs two types of pivoting. It either pivots a box from standing on its longer edge to standing on its shorter edge, or from its shorter edge to its longer edge. 
    \item \textbf{Noise:} A 5cm noise value for the base dimension of the box can be added, causing the arc trajectory to have a larger radius. No noise represents the ideal arc trajectory.
\end{enumerate}







\subsection{Evaluation Metrics} 

To assess the performance of the robot in each experiment, five quantitative measures are established:

\begin{itemize}
    \item \textbf{Success Rate (\%)}: The percentage of all attempts where the robot fully completes the $90 \degree$ rotation.
    \item \textbf{Slip Off Rate (\%)}: The percentage of all attempts where the box slips off from the gripper during the movement.
    \item \textbf{Lift Up Rate (\%)}: The percentage of all attempts where the box is lifted off the surface during the movement, regardless of whether contact was re-established later during the motion.
    \item \textbf{Time Taken (s)}: The total time it takes the robot to move the object from the start to end position.
    \item \textbf{Work (Joules)}: The amount of both translational work and rotational work done applied by the robot on the object to move the object from the start to goal pose. 
    
    % Work is defined as the energy transferred to or from an object via the application of force along a displacement. The measurements required to determine work done will be retrieved from the PapillArray tactile sensors and the FT-300 Force/Torque sensor. This criteria will demonstrate how much effort is required of the robot to complete the task.
\end{itemize}

Slip off and lift up events are observed by the experimenter, while the work is calculated from force measurements from the wrist mounted force/torque sensor, and displacement derived from the object pose estimation. 

A pivot action is considered failed when the robot cannot complete the rotation. Lift up and slip off rates are used to distinguish between perfect success (i.e. neither leaving nor being pushed too hard into the surface) and conditional success, which is less efficient in terms of effort and has a higher risk of damaging the object or environment.


\subsection{Results}
The detailed results for all experiments can be found in Table \ref{table:box_1_results}, with the aggregate results presented in Table \ref{table:all_results}. The combined controller approach achieves 100\% rate with 0 lift up and 0 slip off events. This approach also requires the least work among all approaches while the pick\&place method required the highest work. However, it lags behind other methods in terms of completion time. Instead, the open loop approach requires the least time to complete the movement. 








\begin{table*}[htbp]
\centering
\caption{Experiment results summary.`NA' indicates no such cases were tested since no dimensions of the boxes are used in the pick\&place method. `-' is used for cases where the robot failed to do pivoting in all experiments.}
\begin{tabular}{|l|ccccc|ccccc||ccccc|ccccc|}
\hline

& \multicolumn{10}{c||}{Short-to-Long Pivot} & \multicolumn{10}{c|}{Long-to-Short Pivot}\\ \hline
& \multicolumn{5}{c|}{Without Noise} & \multicolumn{5}{c||}{Added Noise} & \multicolumn{5}{c|}{Without Noise} & \multicolumn{5}{c|}{Added Noise}\\ \hline

& $\%$ & $\%$ & $\%$ & sec & J & $\%$ & $\%$ & $\%$ & sec & J & $\%$ & $\%$ & $\%$ & sec & J & $\%$ & $\%$ & $\%$ & sec & J\\ \hline
\rowcolor{LightCyan}
& Succ. & Lift & Slip & Time & Work & Succ. & Lift & Slip & Time & Work & Succ. & Lift & Slip & Time & Work & Succ. & Lift & Slip & Time & Work\\ \hline

\multicolumn{21}{|c|}{Small Box (Dimensions: $18 \times 11 \times 4 $ cm, Weight: $1.27$ kg) }\\ \hline
Pick\&Place           & 100     & 100 & 0 & \textbf{16.4} & 8.0 & NA & NA & NA & NA & NA & 100 & 100 & 0 & 18.2 & 8.0 & NA & NA & NA & NA & NA\\ \hline 
\rowcolor{LightGray}
Open Loop           & 0     & 100 & 0 & - & - & 0 & 100 & 0 & - & - & 100 & 0 & 0 & \textbf{10.4} & 2.7 & 0 & 100 & 0 & - & -\\ \hline
Vision            & 100     & 100 & 0 & 22.7 & 4.4 & 100 & 100 & 0 & \textbf{23.2} & 4.7 & 100 & 100 & 0 & 24.1 & 2.2 & 100 & 100 & 0 & 24.3 & 2.3\\ \hline
\rowcolor{LightGray} 
Gripper           & 100     & 0 & 0 & 27.0 & 1.6 & 0 & 100 & 0 & - & - & 100 & 0 & 0 & 29.6 & 1.9 & 0 & 90 & 10 & - & -\\ \hline
Force            & 100     & 100 & 0 & 22.9 & 6.4 & 100 & 100 & 0 & 23.5 & 5.8 & 100 & 0 & 0 & 23.8 & 3.1 & 100 & 80 & 0 & \textbf{21.7} & 2.1\\ \hline
\rowcolor{LightGray}
Gripper+Force+Vision           & \textbf{100}     & \textbf{0} & \textbf{0} & 25.3 & \textbf{1.1} & \textbf{100} & \textbf{0} & \textbf{0} & 24.4 & \textbf{1.7} & \textbf{100} & \textbf{0} & \textbf{0} & 26.4 & \textbf{1.5} & \textbf{100} & \textbf{0} & \textbf{0} & 24.9 & \textbf{1.8}\\ \hline

\hline
\multicolumn{21}{|c|}{Large Box (Dimensions: $23 \times 16 \times 5 $ cm, Weight: $0.88$ kg) }\\ \hline
Pick\&Place           & 100     & 100 & 0 & \textbf{16.1} & 6.6 & NA & NA & NA & NA & NA & 100 & 100 & 0 & 16.6 & 5.9 & NA & NA & NA & NA & NA\\ \hline 
\rowcolor{LightGray}
Open Loop           & 0     & 100 & 0 & - & - & 0 & 100 & 0 & - & - & 70 & 0 & 0 & \textbf{10.8} & 4.0 & 0 & 100 & 0 & - & -\\ \hline
Vision            & 0     & 100 & 0 & - & - & 0 & 100 & 0 & - & - & 100 & 90 & 0 & 24.4 & 5.5 & 100 & 100 & 0 & 28.1 & 3.1 \\ \hline
\rowcolor{LightGray} 
Gripper           & 100     & 90 & 0 & 26.8 & 2.4 & 30 & 100 & 10 & 28.3 & 4.3 & 100 & 0 & 0 & 26.7 & 1.9 & 100 & 100 & 0 & 34.0 & 2.5\\ \hline
Force            & 0     & 100 & 0 & - & - & 0 & 100 & 0 & - & - & 100 & 0 & 0 & 24.4 & 5.4 & 100 & 80 & 0 & \textbf{25.1} & 3.3\\ \hline
\rowcolor{LightGray}
Gripper+Force+Vision           & \textbf{100}     & \textbf{0} & \textbf{0} & 25.5 & \textbf{1.8} & \textbf{100} & \textbf{0} & \textbf{0} & \textbf{25.8} & \textbf{1.9} & \textbf{100} & \textbf{0} & \textbf{0} & 26.2 & \textbf{2.3} & \textbf{100} & \textbf{0} & \textbf{0} & 26.9 & \textbf{1.9}\\ \hline

\hline
\multicolumn{21}{|c|}{Long Box (Dimensions: $28 \times 12 \times 5 $ cm, Weight: $1.72$ kg)}\\ \hline
Pick\&Place           & 0     & 100 & 0 & - & - & NA & NA & NA & NA & NA & 0 & 100 & 0 & - & - & NA & NA & NA & NA & NA\\ \hline 
\rowcolor{LightGray}
Open Loop           & 90  & 10 & 0 & \textbf{9.3} & 8.7 & 0 & 100 & 0 & - & - & 100 & 0 & 0 & \textbf{10.8} & 12.7 & 0 & 100 & 0 & - & - \\ \hline
Vision            & 100     & 100 & 0 & 24.6 & 4.1 & 100 & 100 & 0 & 28.2 & 4.0 & 100 & 100 & 0 & 28.7 & 5.3 & 100 & 100 & 0 & 31.0 & 5.2\\ \hline
\rowcolor{LightGray} 
Gripper           & 90 & 0 & 10 & 28.7 & 3.6 & 90 & 100 & 10 & 35.1 & 9.8 & 100 & 0 & 0 & 29.8 & 7.0 & 0 & 100 & 50 & - & -\\ \hline
Force            & 100 & 0 & 0 & 23.9 & 4.5 & 100 & 100 & 0 & \textbf{24.9} & 4.9 & 100 & 0 & 0 & 26.1 & 7.1 & 100 & 0 & 0 & \textbf{27.4} & 4.8\\ \hline
\rowcolor{LightGray}
Gripper+Force+Vision           & \textbf{100}     & \textbf{0} & \textbf{0} & 29.3 & \textbf{2.9} & \textbf{100} & \textbf{0} & \textbf{0} & 30.4 & \textbf{3.3} & \textbf{100} & \textbf{0} & \textbf{0} & 30.6 & \textbf{3.9} & \textbf{100} & \textbf{0} & \textbf{0} & 33.1 & \textbf{3.7}\\ \hline

\end{tabular}
\label{table:box_1_results}
\vspace{-.3cm}
\end{table*}




\begin{table}[htbp]
\center
\caption{Aggregate results from 60 trials for the Pick\&Place method (Pick\&Place was not run for the noisy dimension condition), and 120 for each of the other methods.}
\begin{tabular}{|l|ccccc|}
\hline
& $\%$ & $\%$ & $\%$ & sec & J\\ \hline
\rowcolor{LightCyan}
& Succ. & Lift & Slip & Time & Work\\ \hline
Pick$\&$Place           & 66.7  & 100 & 0 & 16.8 & 7.1\\ \hline 
\rowcolor{LightGray}
Open Loop & 30  & 67.5 & 0 & \textbf{10.3} & 7.0\\ \hline 
Vision           & 83.3  & 99.2 & 0 & 25.9 & 4.1\\ \hline 
\rowcolor{LightGray}
Gripper           & 67.5  & 56.7 & 7.5 & 29.6 & 3.9\\ \hline 
Force           & 83.3  & 55 & 0 & 24.4 & 4.7\\ \hline 
\rowcolor{LightGray}
Force + Gripper + Vision          & \textbf{100}  & \textbf{0} & \textbf{0} & 27.4 & \textbf{2.3}\\ \hline 
\end{tabular}
\label{table:all_results}
\end{table}






%%%%%%%%%%%%%%%%%%%%%%%%%%%%%%%%%%%%%%%%%%%%%%%%%%%%%%%%%%%%%%%%%%%%%%%%

% \begin{table*}[htbp]
% \centering
% \caption{Small box experiment results summary.}
% \begin{tabular}{|l|ccccc|ccccc|ccccc|ccccc|}
% \hline
% & \multicolumn{20}{c|}{Small Box (Dimensions: $18 \times 11 \times 4 $ cm, Weight: $1.27$ kg) }\\ \hline
% & \multicolumn{10}{c|}{Short-to-Long Pivot} & \multicolumn{10}{c|}{Long-to-Short Pivot}\\ \hline
% & \multicolumn{5}{c|}{Without Noise} & \multicolumn{5}{c|}{Added Noise} & \multicolumn{5}{c|}{Without Noise} & \multicolumn{5}{c|}{Added Noise}\\ \hline

% & $\%$ & $\%$ & $\%$ & sec & J & $\%$ & $\%$ & $\%$ & sec & J & $\%$ & $\%$ & $\%$ & sec & J & $\%$ & $\%$ & $\%$ & sec & J\\ \hline
% \rowcolor{LightCyan}
% & Succ. & Lift & Slip & Time & Work & Succ. & Lift & Slip & Time & Work & Succ. & Lift & Slip & Time & Work & Succ. & Lift & Slip & Time & Work\\ \hline
% Pick\&Place           & 100     & 100 & 0 & \textbf{16.4} & 8.0 & NA & NA & NA & NA & NA & 100 & 100 & 0 & 18.2 & 8.0 & NA & NA & NA & NA & NA\\ \hline 
% \rowcolor{LightGray}
% Open Loop           & 0     & 100 & 0 & - & - & 0 & 100 & 0 & - & - & 100 & 0 & 0 & \textbf{10.4} & 2.7 & 0 & 100 & 0 & - & -\\ \hline
% Vision            & 100     & 100 & 0 & 22.7 & 4.4 & 100 & 100 & 0 & \textbf{23.2} & 4.7 & 100 & 100 & 0 & 24.1 & 2.2 & 100 & 100 & 0 & 24.3 & 2.3\\ \hline
% \rowcolor{LightGray} 
% Gripper           & 100     & \textbf{0} & 0 & 27.0 & 1.6 & 0 & 100 & 0 & - & - & 100 & 0 & 0 & 29.6 & 1.9 & 0 & 90 & 10 & - & -\\ \hline
% Force            & 100     & 100 & 0 & 22.9 & 6.4 & 100 & 100 & 0 & 23.5 & 5.8 & 100 & 0 & 0 & 23.8 & 3.1 & 100 & 80 & 0 & \textbf{21.7} & 2.1\\ \hline
% \rowcolor{LightGray}
% Gripper+Force+Vision           & \textbf{100}     & 0 & \textbf{0} & 25.3 & \textbf{1.1} & \textbf{100} & \textbf{0} & \textbf{0} & 24.4 & \textbf{1.7} & \textbf{100} & \textbf{0} & \textbf{0} & 26.4 & \textbf{1.5} & \textbf{100} & \textbf{0} & \textbf{0} & 24.9 & \textbf{1.8}\\ \hline
% \end{tabular}
% \label{table:box_1_results}
% \vspace{-.3cm}
% \end{table*}



% \begin{table*}[htbp]
% \centering
% \caption{Large Box experiment results summary.}
% \begin{tabular}{|l|ccccc|ccccc|ccccc|ccccc|}
% \hline
% & \multicolumn{20}{c|}{Large Box (Dimensions: $23 \times 16 \times 5 $ cm, Weight: $0.88$ kg) }\\ \hline
% & \multicolumn{10}{c|}{Short-to-Long Pivot} & \multicolumn{10}{c|}{Long-to-Short Pivot}\\ \hline
% & \multicolumn{5}{c|}{Without Noise} & \multicolumn{5}{c|}{Added Noise} & \multicolumn{5}{c|}{Without Noise} & \multicolumn{5}{c|}{Added Noise}\\ \hline

% & $\%$ & $\%$ & $\%$ & sec & J & $\%$ & $\%$ & $\%$ & sec & J & $\%$ & $\%$ & $\%$ & sec & J & $\%$ & $\%$ & $\%$ & sec & J\\ \hline
% \rowcolor{LightCyan}
% & Succ. & Lift & Slip & Time & Work & Succ. & Lift & Slip & Time & Work & Succ. & Lift & Slip & Time & Work & Succ. & Lift & Slip & Time & Work\\ \hline
% Pick\&Place           & 100     & 100 & 0 & \textbf{16.1} & 6.6 & NA & NA & NA & NA & NA & 100 & 100 & 0 & 16.6 & 5.9 & NA & NA & NA & NA & NA\\ \hline 
% \rowcolor{LightGray}
% Open Loop           & 0     & 100 & 0 & - & - & 0 & 100 & 0 & - & - & 70 & 0 & 0 & \textbf{10.8} & 4.0 & 0 & 100 & 0 & - & -\\ \hline
% Vision            & 0     & 100 & 0 & - & - & 0 & 100 & 0 & - & - & 100 & 90 & 0 & 24.4 & 5.5 & 100 & 100 & 0 & 28.1 & 3.1 \\ \hline
% \rowcolor{LightGray} 
% Gripper           & 100     & 90 & 0 & 26.8 & 2.4 & 30 & 100 & 10 & 28.3 & 4.3 & 100 & 0 & 0 & 26.7 & 1.9 & 100 & 100 & 0 & 34.0 & 2.5\\ \hline
% Force            & 0     & 100 & 0 & - & - & 0 & 100 & 0 & - & - & 100 & 0 & 0 & 24.4 & 5.4 & 100 & 80 & 0 & \textbf{25.1} & 3.3\\ \hline
% \rowcolor{LightGray}
% Gripper+Force+Vision           & \textbf{100}     & \textbf{0} & \textbf{0} & 25.5 & \textbf{1.8} & \textbf{100} & \textbf{0} & \textbf{0} & \textbf{25.8} & \textbf{1.9} & \textbf{100} & \textbf{0} & \textbf{0} & 26.2 & \textbf{2.3} & \textbf{100} & \textbf{0} & \textbf{0} & 26.9 & \textbf{1.9}\\ \hline

% \end{tabular}
% \label{table:box_2_results}
% \end{table*}

% \begin{table*}[htbp]
% \centering
% \caption{Long box experiment results summary. `NA' indicates no such cases were tested since no dimensions of the boxes are used in the pick\&place method. `-' is used for cases where the robot failed to do pivoting in all experiments.}
% \begin{tabular}{|l|ccccc|ccccc|ccccc|ccccc|}
% \hline
% & \multicolumn{20}{c|}{Long Box (Dimensions: $28 \times 12 \times 5 $ cm, Weight: $1.72$ kg)}\\ \hline
% & \multicolumn{10}{c|}{Short-to-Long Pivot} & \multicolumn{10}{c|}{Long-to-Short Pivot}\\ \hline
% & \multicolumn{5}{c|}{Without Noise} & \multicolumn{5}{c|}{Added Noise} & \multicolumn{5}{c|}{Without Noise} & \multicolumn{5}{c|}{Added Noise}\\ \hline

% & $\%$ & $\%$ & $\%$ & sec & J & $\%$ & $\%$ & $\%$ & sec & J & $\%$ & $\%$ & $\%$ & sec & J & $\%$ & $\%$ & $\%$ & sec & J\\ \hline
% \rowcolor{LightCyan}
% & Succ. & Lift & Slip & Time & Work & Succ. & Lift & Slip & Time & Work & Succ. & Lift & Slip & Time & Work & Succ. & Lift & Slip & Time & Work\\ \hline
% Pick\&Place           & 0     & 100 & 0 & - & - & NA & NA & NA & NA & NA & 0 & 100 & 0 & - & - & NA & NA & NA & NA & NA\\ \hline 
% \rowcolor{LightGray}
% Open Loop           & 90  & 10 & 0 & \textbf{9.3} & 8.7 & 0 & 100 & 0 & - & - & 100 & 0 & 0 & \textbf{10.8} & 12.7 & 0 & 100 & 0 & - & - \\ \hline
% Vision            & 100     & 100 & 0 & 24.6 & 4.1 & 100 & 100 & 0 & 28.2 & 4.0 & 100 & 100 & 0 & 28.7 & 5.3 & 100 & 100 & 0 & 31.0 & 5.2\\ \hline
% \rowcolor{LightGray} 
% Gripper           & 90 & 0 & 10 & 28.7 & 3.6 & 90 & 100 & 10 & 35.1 & 9.8 & 100 & 0 & 0 & 29.8 & 7.0 & 0 & 100 & 50 & - & -\\ \hline
% Force            & 100 & 0 & 0 & 23.9 & 4.5 & 100 & 100 & 0 & \textbf{24.9} & 4.9 & 100 & 0 & 0 & 26.1 & 7.1 & 100 & 0 & 0 & \textbf{27.4} & 4.8\\ \hline
% \rowcolor{LightGray}
% Gripper+Force+Vision           & \textbf{100}     & \textbf{0} & \textbf{0} & 29.3 & \textbf{2.9} & \textbf{100} & \textbf{0} & \textbf{0} & 30.4 & \textbf{3.3} & \textbf{100} & \textbf{0} & \textbf{0} & 30.6 & \textbf{3.9} & \textbf{100} & \textbf{0} & \textbf{0} & 33.1 & \textbf{3.7}\\ \hline




%%%%%%%%%%%%%%%%%%%%%%%%%%%%%%%%%%%%%%%%%%%%%%%%%%%%%%%%%%%%%






% Successful pivoting is our first priority goal, time and work evaluation metrics are recorded as '-' only when the pivoting failure happens. 


% \subsubsection{Vision Controller}
% Compared with the open loop approach, vision controller has the advantage of updating trajectory which can be reflected from the success rate. However, the vision controller almost always lifts the object which case unintended work consumption since it only tries to fix the trajectory after the trajectory goes wrong. It will be limited if the grasped object is too light for causing enough rotational slip as in the case of box 2.  

% \subsubsection{Gripper Controller}
% In terms of the gripper controller, the success rate increases when we used the correct dimensions for the object.





% \begin{table}[htbp]
%     \centering
%     \caption{Time taken for each approaches}
%     \label{tab: time}
%     \begin{tabular}{cccc}
%     \hline
%     \textit{Controllers} & \textit{box 1} & \textit{box 2} & \textit{box 3}\\ \hline
%         Baseline & 17.28 & 16.33  & / \\ \hline
%         \textbf{Open Loop} & \textbf{10.07} & \textbf{11.18} & \textbf{10.72} \\ \hline
%         Vision  & 23.56 & 24.97 & 28.14\\ \hline
%         Gripper & 28.14 & 28.92 & 32.28 \\ \hline
%         Force & 22.96 & 24.95 & 25.57 \\ \hline
%         Gripper, Force  & 25.25 & 26.10 & 30.84\\ \hline
%     \end{tabular}
% \end{table}

% \begin{table}[htbp]
%     \centering
%     \caption{Success Rate}
%     \label{tab: rate}
%     \begin{tabular}{cccc}
%     \hline
%     \textit{Controllers} & \textit{box 1} & \textit{box 2} & \textit{box 3}\\ \hline
%         Baseline & 1/1 & 1/1  & 0/0 \\ \hline
%         Open Loop & (1,0)/(0,0) & 11.18 & 10.72 \\ \hline
%         Vision  & 23.56 & 24.97 & 28.14\\ \hline
%         Gripper & 28.14 & 28.92 & 32.28 \\ \hline
%         Force & 22.96 & 24.95 & 25.57 \\ \hline
%         Gripper, Force  & 25.25 & 26.10 & 30.84\\ \hline
%     \end{tabular}
% \end{table}






