% !TeX root = ../main.tex

\section{Experiment}
\label{sec:experiment}\label{subsec: mesh_eval}

We conduct ablations on \datasetname to verify our design choices (\secref{subsec:ablation}) and to compare with baselines (\secref{subsec:comparison}).
%
We consider the following metrics for reconstruction evaluation: volumetric IoU (mIoU) [\%], Chamfer distance ({C-}$L_2$) [cm], point-to-surface distance (P2S) [cm], and normal consistency (NC) [\%]. 

\begin{figure}
       \centering
        \setlength{\tabcolsep}{1pt}
        {\scriptsize
        \begin{tabular}{c c c c c c c }
            { Original } &
            \multicolumn{2}{c}{  } &
            \multicolumn{4}{c}{$\longleftarrow$ Object level variations $\longrightarrow$} \\
            \includegraphics[width=0.185\linewidth]{images/ablation/chair.jpg} &
            \multicolumn{2}{c}{  } &
            \includegraphics[width=0.185\linewidth]{images/ablation/1_only_prompt_mixing/bench.jpg} &
            \includegraphics[width=0.185\linewidth]{images/ablation/1_only_prompt_mixing/stool.jpg} &
            \includegraphics[width=0.185\linewidth]{images/ablation/1_only_prompt_mixing/armchair.jpg} &
            \includegraphics[width=0.185\linewidth]{images/ablation/1_only_prompt_mixing/saddle.jpg} \\
            \multicolumn{3}{c}{  } &
            \multicolumn{4}{c}{ Only Prompt Mixing } \\
            \multicolumn{3}{c}{ } &
            \includegraphics[width=0.185\linewidth]{images/ablation/2_with_self_attn_injection/bench.jpg} &
            \includegraphics[width=0.185\linewidth]{images/ablation/2_with_self_attn_injection/stool.jpg} &
            \includegraphics[width=0.185\linewidth]{images/ablation/2_with_self_attn_injection/armchair.jpg} &
            \includegraphics[width=0.185\linewidth]{images/ablation/2_with_self_attn_injection/saddle.jpg} \\
            \multicolumn{3}{c}{  } &
            \multicolumn{4}{c}{ + Attention-Based Shape Localization } \\
            \multicolumn{3}{c}{ } &
            \includegraphics[width=0.185\linewidth]{images/ablation/3_background_blending/bench.jpg} &
            \includegraphics[width=0.185\linewidth]{images/ablation/3_background_blending/stool.jpg} &
            \includegraphics[width=0.185\linewidth]{images/ablation/3_background_blending/armchair.jpg} &
            \includegraphics[width=0.185\linewidth]{images/ablation/3_background_blending/saddle.jpg} \\
            \multicolumn{3}{c}{  } &
            \multicolumn{4}{c}{ + Controllable Background Preservation } \\
        \end{tabular}
        }
    \vspace{1mm}
    \captionof{figure}{
    Ablating our full object variations pipeline. Original image was crated using the prompt ``A \emph{chair} with a dog on it''. 
    }
    \vspace{-10pt}
    \label{fig:ablation}
\end{figure}

\subsection{Ablation Study}
\label{subsec:ablation}

%%Figures of the Sec.
\begin{figure}[!h]
  \centering
    \includegraphics[width=0.4\textwidth]{Figures/sec07-experiments/pose_only.pdf}
  \caption{\textbf{Qualitative ablation (shape refinement).} The shape refinement stage better models the contact-aware deformation.}
  \label{fig: pose_only}
\end{figure}

\begin{figure}[!h]
  \centering
    \includegraphics[width=0.4\textwidth]{Figures/sec07-experiments/joint_opt.pdf}
  \caption{\textbf{Qualitative ablation (alternating optimization).} Optimizing poses and shape networks alternatingly improves results in areas with heavy contact.}
  \label{fig: joint_opt}
\end{figure}


\begin{figure}[!h]
  \centering
    \includegraphics[width=0.43\textwidth]{Figures/sec07-experiments/collision.pdf}
  \caption{\textbf{Importance of collision loss.} 
  Instance meshes intersect each other in contact areas if we remove the collision loss term.}
  \label{fig: ablation_interp}
  \vspace{-2mm}
\end{figure}


%% Tables of next sec
\begin{table*}[t]
\centering
\small
\begin{tabular}{ccccccccc}
\hline
\textbf{Setting}  & \textbf{Method}  & \textbf{MPJPE}  $\downarrow$ & \textbf{MVE} $\downarrow$ & \textbf{NMJE} $\downarrow$ & \textbf{NMVE} $\downarrow$ & \textbf{F1} $\uparrow$& $\textbf{PCDR}^{0.10}$ $\uparrow$& \textbf{CD} $\downarrow$\\ \hline
\multicolumn{1}{c}{\multirow{3}{*}{Monocular}} & PARE   \cite{kocabas2021pare}          &   $87.6$    &  $106.5$   &  $95.3$    &    $115.9$  &  $0.919$  &   $0.610$   &      $297.7$         \\
\multicolumn{1}{c}{}                           & ROMP  \cite{sun2021romp}           &   $93.0$    &  $116.2$    &  $93.2$    &  $116.4$    &  $0.998$  &   $0.613$   &      $338.0$         \\
\multicolumn{1}{c}{}                           & BEV    \cite{sun2022bev}          &   $92.5$    &  $113.7$   &   $92.6$   &    $113.8$   &  $0.999$    &  $0.745$   &      $295.8$         \\ \hline
\multicolumn{1}{c}{\multirow{2}{*}{Multi-view}} & MVPose (4-views) \cite{dong2019mvpose} &   $61.3$     &   $78.3$  &    $67.0$   &   $85.4$    &  $0.917$  &   $0.957$   &    $234.8$           \\
\multicolumn{1}{c}{}    & MVPose (8-views) \cite{dong2019mvpose} &   $50.3$     &   $61.8$  &    $51.8$   &   $63.6$    &  $0.971$  &   $0.972$   &    $166.8$           \\
\hline
\end{tabular}
\caption{\textbf{SMPL estimation.} Results of monocular and multi-view SMPL estimation methods on \datasetname  (\cf Sec. \ref{subsec: smpl_baseline} and Fig. \ref{fig: smpl_estimation}).}
\label{tab: smpl_estimation}
\end{table*}

\begin{figure*}[]
  \centering
    \includegraphics[width=\textwidth]{Figures/sec08-benchmark/mesh_baseline_img.pdf}
  \caption{\textbf{Detailed geometry reconstruction.} 
  Results of monocular and multi-view methods together with the GT of \datasetname (\cf Sec. \ref{subsec: mesh_baseline}).
  }
  \label{fig: mesh_reconstruction}
\end{figure*} 


\noindent\textbf{Shape Refinement.}
We compare our full optimization pipeline to a version without the shape refinement stage (\secref{subsec: shape_refine}). From \figref{fig: pose_only}, we observe that the details in the cloth are not accurately modelled if we do not further refine the shape network weights. Moreover, without the shape refinement stage, the method fails to model the contact-aware cloth deformations (\eg the right hand of the blue colored person is occluded by the other person's cloth). Note that such deformations only occur with physical contact between people and cannot be learned from individual scans.
The quantitative results in \tabref{tab: quant_exp} also support the benefits of the shape network refinement.

\noindent \textbf{Alternating Optimization.}
 We compare our alternating optimization with an approach where poses and shape networks are optimized concurrently.
We observe that in this case, it is hard to disambiguate body parts of different subjects in the contact area (\cf \figref{fig: joint_opt}).
Our alternating pipeline can better disentangle the effects of the pose and shape network.
The quantitative results of \tabref{tab: quant_exp} also confirm this.

\noindent\textbf{Collision Loss.}
Without penalizing the collision of the two occupancy fields, one person’s mesh might be partially intersected by the other person in the contact area, as we see from \figref{fig: ablation_interp}. 
We quantitatively measure the interpenetration by calculating the intersection volume between the individual segmented meshes. With the collision loss term defined in \equref{eq: collision}, the average intersection volume decreases from \SI{11.62e-4}{\meter^3} to \SI{5.82e-4}{\meter^3} by $49.91 \%$.  


\subsection{Comparison Study}
\label{subsec:comparison}

\noindent\textbf{SMPL+D Baseline.} 
A straightforward baseline for our task is to directly track multiple clothed SMPL body template meshes. We define this baseline as the SMPL+D (\cf \cite{bhatnagar2020ipnet, bhatnagar2020loopreg}) tracking baseline. It tracks the 3D geometry of close interacting people at each frame by estimating the individual displacement of each SMPL vertex and each subject. 
We optimize these displacement fields and the SMPL body parameters in a similar alternating manner as we do in our proposed method.
For more implementation details please refer to the Supp. Mat.
Quantitatively, our proposed method with personalized priors outperforms the SMPL+D baseline on all metrics (\tabref{tab: quant_exp}). The optimized SMPL+D models do not have any personalized prior knowledge from which we can infer when substantial instance ambiguity exists. 
We show qualitative results in the Supp. Mat.





