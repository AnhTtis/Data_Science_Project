\section{Conclusion}
\label{sec:conclusion}
In this paper, we have introduced an evidence collection dataset specific to fact-checking in the context of sub-editing news articles. We have shown that conditional text generation approaches can improve on rule-based baselines on their ability to produce the correct search results. We have also explored the effect of ensembling several of these approaches, and shown that this outperforms any single approach. 
We have also analysed the shortcomings of different generation methods, and found areas of potential improvement. In the future, there is scope to experiment with methods of including additional context in the automated model input. There is also scope to experiment with more involved ways of combining different methods' input, e.g., the output of rule-based baselines could be used to enhance the inputs for the automated generation methods. Finally, further data annotation or data augmentation methods might be used to overcome the challenges of having a dataset which is relatively limited.
