\section{Dataset}
\label{sec:dataset}

In this section, we describe the dataset that we have compiled and annotated in this research. The dataset is composed of 390 claims containing verifiable information. Each claim is accompanied by a target URL, an evidence page containing the information required to verify the claim, along with a target search query, which was used to find the target URL. The source of claims is an archive of news articles published in The Times and Sunday Times, and includes content from 20 different sections. A random
sample of 100 articles published between May and June of 2021 were selected from this
collection for annotation.

During the annotation process, articles were first split into sentences using the Spacy\footnote{https://spacy.io/} library. Second, annotators read through the articles sentence by sentence, assessing whether each sentence contained verifiable information. Only sentences labelled as verifiable progressed to the next annotation steps. Third, annotators looked for relevant evidence that would allow the verification of claims. In order to do this, they used the Bing search engine to find relevant web pages. The motivation for this process was to follow the usual approach of a fact-checker looking for information based on a claim, and not place any artificial constraints on the format of the query used. The only limitation set was that the chosen web page should be found on the first page of search results. Annotators could construct the search queries using information contained in the claim sentence as well as additional context from other sentences of the article if necessary. If no relevant page was found with their initial query, they could modify them and retry as many times as needed. Annotators would review results and select the highest-ranking page that contained the necessary information to verify the claim.  Once an evidence web page was found, the claim sentence would be annotated with the final search query used to find it. Finally, they would be annotated with the URL of the selected evidence web page. 
Working with historic article data presented some challenges, including the fact that many of the claims are echoed through other media outlets, often saturating the search results provided by search engines. Throughout the annotation process, annotators were instructed to prioritise first-hand sources, and avoid using other news/media sources where possible. If news/media sources were the only apparent verification, priority was given to those sources which had earlier publication dates than the checked article.  

During our experiments, we found that search results from the Bing search engine were variable, even when using the same query. The original dataset contained more than the current 390 claims. Repeated executions of the target search queries produced varying results, and the target URLs were not always present in the search results. This was a concern since this would be one of the metrics used to evaluate the quality of the generated queries. In order to ensure that the dataset would only include samples where the target URL was able to be found using the corresponding target search query, we executed all target search queries on Bing a total of 12 times on 3 different days and collected the results. Only the samples where the target URL was found in the search results in 50\% or more of the 12 executions were kept in the dataset. This resulted in 390 of the samples remaining in the dataset. All fine-tuning and evaluation are done on these 390 samples. Table \ref{tab:ec-dataset-examples} shows some examples of claims, each with their corresponding target search query and target URL.

\begin{table*}
%\small
\centering
\begin{tabular}{p{0.35\textwidth} p{0.2\textwidth} p{0.35\textwidth}}
\cline{1-3}
Claim & Target Search Query & Target URL \\
\cline{1-3}
Liz Truss, the international trade secretary, said that membership of the CPTPP represented "a huge opportunity for Britain". & Liz Truss international trade secretary & https://en.wikipedia.org/wiki/\newline liz\_truss \\

The OECD raised its global growth forecast from 5.6 per cent to 5.75 per cent for this year and from 4 per cent to nearly 4.5 per cent for 2022. & oecd global gdp growth forecast may 2021 & https://www.oecd.org/economic-outlook/may-2021 \\

Galileo, the European system, operates at a height of 23,000km, but Britain is being evicted from Galileo because of Brexit. & galileo operating height & https://en.wikipedia.org/wiki/\newline galileo\_(satellite\_navigation) \\

\end{tabular}
\caption{Sample of claims, target search queries and corresponding target URL in the dataset.}
\label{tab:ec-dataset-examples}
\end{table*}