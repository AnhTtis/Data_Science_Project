%                                                              aa.dem
% AA vers. 9.1, LaTeX class for Astronomy & Astrophysics
% demonstration file
%                                                       (c) EDP Sciences
%-----------------------------------------------------------------------
%
%\documentclass[referee]{aa} % for a referee version
%\documentclass[onecolumn]{aa} % for a paper on 1 column  
%\documentclass[longauth]{aa} % for the long lists of affiliations 
%\documentclass[letter]{aa} % for the letters 
%\documentclass[bibyear]{aa} % if the references are not structured 
%                              according to the author-year natbib style
%
\documentclass{aa}  
\usepackage{amssymb}
%
\usepackage{xcolor}
\usepackage{graphicx}
\usepackage{hyperref}
\newcommand*{\bba}{$^{\scriptstyle 3\mathrm{D}}$B{\sc {arolo}}}
\newcommand*{\galfit}{G{\sc {alfit}}}
\newcommand*{\cannubi}{C{\sc {annubi}}}
\newcommand*{\stardust}{S{\sc {tardust}}}
\newcommand*{\fast}{F{\sc {ast}}}
\newcommand*{\casa}{C{\sc {asa}}}
\newcommand*{\grizli}{G{\sc {rizli}}}
\def\CIi{[CI]($^3P_1 - ^3P_0$)}
\def\CIii{[CI]($^3P_2  - ^3P_1$)}

\def\CII{\hbox{[C~$\scriptstyle\rm II $]}}

\newcommand{\quotes}[1]{``#1''}

\def\msun{{\rm M}_{\odot}}
%%%%%%%%%%%%%%%%%%%%%%%%%%%%%%%%%%%%%%%%
\usepackage{txfonts}
%%%%%%%%%%%%%%%%%%%%%%%%%%%%%%%%%%%%%%%%
%\usepackage[options]{hyperref}
% To add links in your PDF file, use the package "hyperref"
% with options according to your LaTeX or PDFLaTeX drivers.
%
\begin{document} 


   \title{The ALMA-ALPAKA survey I}
   \subtitle{High-resolution CO and [CI] kinematics of star-forming galaxies at $z= 0.5$ - 4}


\author{F. Rizzo
        \inst{1, 2}
        \and
        F. Roman-Oliveira
        \inst{3}
        \and
        F. Fraternali
        \inst{3}
        \and
        D. Frickmann
        \inst{1, 2}
        \and
        F. M. Valentino
        \inst{1, 2, 4}
        \and
        G. Brammer
        \inst{1, 2}
        \and
        friends
        }  

   \institute{Cosmic Dawn Center (DAWN)
   \and Niels Bohr Institute, University of Copenhagen, Jagtvej 128, 2200 Copenhagen N, Denmark\\
              \email{francesca.rizzo@nbi.ku.dk}
              \and
             Kapteyn Astronomical Institute, University of Groningen, Landleven 12, 9747 AD, Groningen, The Netherlands
             \and
             European Southern Observatory, Karl-Schwarzschild-Str. 2, D-85748 Garching bei Munchen, Germany
             }

% 5 {} token are mandatory 
  \abstract
    % context heading (optional)
     %leave it empty if necessary  
   {Spatially-resolved studies of the kinematics of galaxies provide crucial insights into their assembly and evolution, providing the means to infer the properties of the dark matter halos, derive the impact of feedback on the interstellar medium (ISM), measure and characterize the outflow motions. To date, most of the kinematic studies at $z = 0.5$ - 4 were obtained using emission lines from warm gas (e.g., H$\alpha$, [OII], [OIII], which are shown to give a partial view of the dynamics of galaxies and the properties of the ISM. Complementary insights on the cold gas kinematics are therefore needed.}
  %aims heading (mandatory)
   {We present ALPAKA, a project aimed at providing high-resolution observations of CO and [CI] emission lines from star-forming galaxies at $z = 0.5$ - 4. ALPAKA is based on the mining of the Atacama Large Millimiter Array (ALMA) public archive and the selection of galaxies with high data quality observations, necessary for robustly recovering the motions of the gas. With $\approx$ 147 hours of total integration time, ALPAKA provides $\sim$ 0.25" observations for 28 star-forming galaxies, the largest sample with spatially-resolved CO or [CI] kinematics at $z \gtrsim 0.5$, over 7 Gyr of cosmic history.}
  % methods heading (mandatory)
   {We describe the selection of targets, the observations, data reduction and derivation of global properties, and kinematics of the ALPAKA targets. }
  % results heading (mandatory)
   {Using multi-wavelength ground- and space-based ancillary data, we found that ALPAKA probes the massive ($M_{\star} \gtrsim 10^{10}$ M$_{\odot}$), actively star-forming (SFR $\approx$ 10 - 3000 M$_{\odot}$\,yr$^{-1}$) part of the population of galaxies at $z \sim 0.5$ - 4. Further, a large fraction of 19/28 ALPAKA galaxies lie in overdensity regions (clusters, groups, and protoclusters). By performing a kinematic analysis of the ALPAKA sample, we divided the targets into three kinematic classes: mergers, disks, disturbed disks. The latter show signatures of non-circular motions and disturbances driven by interactions, environmental mechanisms, and outflows. The two disk subsamples have velocity dispersion values that are typically larger in the inner than in the outermost regions, with average values across their disks from 10 to 167 km s$^{-1}$. The ratios of ordered-to-random motion ($V/\sigma$) of the ALPAKA disks range between 2.6 and 19, with a median value of 10$^{+7}_{-5}$ and with no evidence of evolution with redshift. The reduced data cubes and the derived kinematic profiles (rotation velocity and velocity dispersion) are made publicly available. }
  % conclusions heading (optional), leave it empty if necessary 
   {}

   \keywords{Galaxies: evolution -- 
   Galaxies: high-redshift --
   Galaxies: ISM --
    Galaxies: kinematics and dynamics --
    Galaxies: photometry --
    Galaxies: structure}

   \maketitle
%
%-------------------------------------------------------------------
%This overleaf will be used to share the outputs of the kinematic analysis and references for the photometry. Any feedback and comments are appreciated.
\section{Introduction}
According to the current paradigm of galaxy formation and evolution, the assembly of galaxies is regulated by a variety of physical processes: interplay between dark and baryonic matter, gas accretion, galaxy mergers, star formation, and stellar and active galactic nuclei (AGN) feedback \citep{Mo_2010, Cimatti_2019, Vogelsberger_2020}. From the theoretical perspective, state-of-the-art cosmological simulations reproduce most of the global properties of galaxies at different cosmic epochs \citep[e.g.,][]{Schaye_2015, Nelson_2018, Pillepich_2018, Roca_2021}. However, this success does not imply that we understand how galaxies form and evolve in detail, as the modeling of processes acting on scales below the resolution of simulations (e.g., stellar and AGN feedback, star formation) relies on simplistic assumptions for the so-called sub-grid models \citep{Vogelsberger_2020}.
As a consequence, the kpc and sub-kpc spatial distributions of the baryonic and dark matter within galaxies strongly vary with the adopted models \citep{Kim_2016, Roca_2021}. 

From the observational perspective, the role played by the various processes in driving galaxy evolution is still unclear, even in redshift ranges that have been widely studied in the last decades (e.g., $z = 0.5 - 4$
). Understanding the assembly of galaxies requires, indeed, simultaneous high-resolution, multi-wavelength studies of their morphologies and dynamics. The detailed morphological analysis of the distribution of stars \citep[e.g.,][]{Lang_2014, vanderwel_2014, Mowla_2019, Ferreira_2022, Kartaltepe_2022} and gas/dust \citep[e.g.,][]{Calistro_2018, Gullberg_2019, Hodge_2019, Rujopakarn_2019, Kaasinen_2020, Tadaki_2020, Ikeda_2022} within galaxies allows one to trace the build up of structures.
%using multiwavelength, spatially resolved observations of their stars, gas and dust distributions and diffent kinematic gas tracers, re. While morphological studies of the stellar components have now been obtained for 1000s galaxies up to $z \sim 5$ - 6 \citep[e.g.,][]{Lang_2014, vanderwel_2014, Mowla_2019, Ferreira_2022, Kartaltepe_2022}, the morphology of the gas and dust components is still limited to 10s galaxies \citep[e.g.,][]{Calistro_2018, Gullberg_2019, Hodge_2019, Rujopakarn_2019, Kaasinen_2020, Tadaki_2020, Ikeda_2022}. Similarly, detailed multiwavelength kinematic studies are challenging, even with state-of-the-art facilities, as they require long integration times to spatially and spectrally resolve typical $z \gtrsim 1$ star-forming galaxies (see details below). On the other hand, the 
Galaxy dynamics, on the other hand, provide constraints on the impact of mergers, outflows, gas accretion, and environmental mechanisms on the growth of galaxies \citep[e.g.,][]{Loiacono_2019, Concas_2022, Bacchini_2023, Roman_2023}. For instance, the prevalence of circular motions within galaxies indicates that the build up of galaxies is mainly driven by smooth gas accretion \citep{Pillepich_2019, Ginzburg_2022}. On the contrary, the prevalence of non-circular motions is ascribed to outflows, merger events, or counter-rotating gas streams. These mechanisms can be potentially disruptive as they destroy or prevent the rebuilding of disks on timescales comparable to or longer than the dynamical times of a galaxy \citep{Gurvich_2022, Kretschmer_2022}. In addition, for galaxy disks, measurements of gas rotation allow for inferring the content and distribution of dark matter within galaxies \citep[e.g.,][]{Straatman_2017, Price_2021}. In contrast, measurements of gas velocity dispersion provide key insights into the mechanisms injecting turbulence into the interstellar medium \citep[ISM; e.g.,][]{Krumholz_2018, Ubler_2019, Kohandel_2020, Rizzo_2021, Jimenez_2022, Rathjen_2022}. 
%Deriving the gas kinematics from both warm and cold gas tracers is, therefore, crucial for comprehensively understanding the galaxy assembly and the role played by feedback. 

To date, most of the information on galaxy kinematics at $z \sim 0.5$ - 4 are derived from observations of optical emission lines tracing warm ionized gas \citep[e.g., H$\alpha$, {[OIII]}, {[OII]};][]{Forster_2006, Stott_2016, Turner_2017, Forster_2018, Wisnioski_2019}. However, such studies are still in their infancy and limited in statistics. We note, indeed, that even though integral field unit (IFU) observations of hundreds of $z \sim 0.5$ - 3 galaxies are now available, they are taken in seeing-limited mode (i.e., $\gtrsim 0.6$", spatial scale $\gtrsim 5$ kpc at $z = 2$). Therefore, most galaxies are barely resolved, with 1.5 - 2 angular resolution elements covering their emission. Under such observational conditions, identifying any non-circular motions is, therefore, challenging. Using adaptive optics observations (angular resolutions of 0.17") of 35 $z \sim 2$ massive galaxies, \citet{Forster_2018} show that most galaxies that were previously identified as regular rotating disks have non-circular motions driven by outflows and mergers. This finding is consistent with results from recent observational and theoretical studies that suggest that the choice of the kinematic tracer can have a strong impact on the dynamical studies as the warm gas might be strongly affected by the presence of gas in outflows or in extraplanar layers \citep{Girard_2021, Ejdetjarn_2022, Kretschmer_2022, Rathjen_2022}. Should this be the case, the rotation velocity and velocity dispersion from warm gas might be highly uncertain as these two quantities are typically derived assuming that the gas moves in circular orbits, while radial or vertical motions due to outflows could strongly bias their measurements. Overall, considering the caveat described above, a common result of observational studies at $z \sim 0.5$ - 3 using warm gas tracers is that star-forming galaxies  have a more turbulent ISM than local galaxies \citep[i.e., larger velocity dispersion $\sigma$ and lower rotation-to-random ratios $V/\sigma$]{Forster_2006, Stott_2016, Turner_2017, Forster_2018, Wisnioski_2019}. %This progressive increase of chaotic, random motions with redshift is also predicted by most of the models and cosmological simulations and it is explained as resulting from the vigorous gas accretion, both from the cosmic web and from minor mergers \citep[e.g.,][]{Pillepich_2019, Kretschmer_2022, Gurvich_2022}. 

At $z \gtrsim 4$, galaxy kinematics have been derived for tens of sources, using Atacama Large Millimeter Array (ALMA) observations targeting the [CII]-158 $\mu$m emission line, a cold gas tracer \citep[e.g.,][]{Neeleman_2020, Rizzo_2020, Fraternali_2021, Jones_2021,  Tsukui_2021}. For around 15 of these galaxies, the angular resolution of [CII] observations is comparable or even better than the one obtained with adaptive optics at lower redshifts. However, only in a few cases, strong kinematic anomalies ascribed to outflows and mergers are identified \citep{Tadaki_2020, Roman_2023}. Further, the main finding from these studies is that a large fraction of star-forming galaxies at $z \gtrsim 4$ are dinamically cold disks with $V/\sigma$ ratios as high as 10 \citep[e.g.,][]{Rizzo_2020, Lelli_2021, Rizzo_2021}. 

An accurate comparison between dynamical properties of galaxies in these two redshift ranges, $z \sim 0.5$ - 4 and $z \gtrsim 4$, requires high-resolution kinematic measurements obtained using similar ISM tracers.
%The use of A possible explanation for this discrepancy is that, in these two redshift ranges, the kinematic properties are derived from emission lines tracing different phases of the ISM: warm gas (e.g., H$\alpha$, [OII], [OIII]) at $z \sim 1$ - 3 and cold gas (e.g., [CII], CO) at $z \gtrsim 4$.In other words, there are indications that galaxy disks are dynamically colder at 
Unfortunately, because of the sensitivity and frequency coverage of state-of-the-art facilities, high-resolution [CII] observations at $z \lesssim 4$ are challenging or not feasible, due to the low atmospheric transmission \citep{Carilli_2013}. To study the $z \sim 0.5$ - 4 kinematics using cold gas tracers, one can target carbon monoxide (CO) transitions or fine structure lines from neutral atomic carbon (CI). However, emission lines from CO and CI are typically fainter than [CII] \citep{Carilli_2013, Bernal_2022}. In addition, to date, the only instrument able to achieve the combined sensitivity and angular resolution requirements of $\lesssim 0.3$" needed for studying the CO (or [CI]) kinematics at $z \gtrsim 0.5$ is ALMA. However, 
even with ALMA, CO and [CI] observations are extremely time consuming. For this reason, studies of the kinematics using cold gas tracers at cosmic noon have been obtained only for a handful of galaxies \citep[e.g.,][]{Molina_2019, Noble_2019, Xiao_2022, Lelli_2023}. Finally, since the published kinematic measurements are derived using different algorithms and assumptions, a systematic compilation and comparison of the results is not straightforward. 

Here we present the project ``Archival Large Program to Advance Kinematic Analysis" (ALPAKA), aimed at filling all the above gaps by collecting high data quality emission line observations of $z = 0.5 - 4$ galaxies from the ALMA public archive. We note that the use of archival data does not have an impact on the originality of the present work as, to date, the ALMA observations for $\approx 67\%$ of the ALPAKA galaxies have not been published in any studies. With an increase in sample size by a factor of 3 compared to previous works, ALPAKA will be used to obtain the first high-resolution characterization of the morpho-kinematic properties of galaxies at $z \sim 0.5$ - 4 using millimiter to optical observations.

In this paper, the first of a series, we present the ALPAKA sample and discuss the sample selection (Sect.\,\ref{sec:sample}) and describe the ALMA and Hubble Space Telescope data analyzed in this work (Sect.\,\ref{sec:observations}). In Sect.\,\ref{sec:gloabl}, we derive the global properties of the ALPAKA targets using ancillary data. The kinematic modelling and assumptions, and the identification of the disks is presented in Sect.\,\ref{sec:kinematic}. In Sect.~\ref{sec:details}, we describe the kinematics of each ALPAKA target. In Sect.\,\ref{sec:discussion}, we discuss the potential bias due to selection effects and we describe the dynamical properties of the subsample of ALPAKA disk galaxies. Finally, in Sect.\,\ref{sec:conclusion}, we summarize the main findings.

Throughout this paper, we assume a $\Lambda$CDM cosmology with parameters from \citet{Planck} and a Chabrier Initial Mass Function \citep{Chabrier}.


%This paper is the first of a series on the spatially-resolved morpho-kinematic properties of the ALPAKA galaxies. Here, we have not previous observations are In addition to deriving the kinematic properties of the galaxies using the same homogeneous , we will provide the first systematic characterization of the cold ISM at cosmic noon. This paper is the first 
%However, the choice of the observational tracer can have a strong impact on the obtained results. The velocity dispersion of galaxies at peak star ormation may be vastly overestimated when using ionised gas as a proxy for the total gas mass 
%the general gas kinematics are traced differently by different gas tracers, and relying solely on the warm ionised phase (which is traced primarily by H) can lead to an overestimation of the total turbulent energy in the gas.

%On the other hand, despite current and future generations of telescopes are bringing exceptional progress in the observational domain, resolving kpc and sub-kpc scales even at intermediate redshift can be extremely timely expensive. In this context, the 

\section{Sample description} \label{sec:sample}
\subsection{Selection criteria}
The ALPAKA project is designed to study the kinematic and dynamical properties of galaxies at $z = 0.5$ - 4. To this end, we collected the sample by selecting data publicly available from the ALMA archive and with spectral setup covering CO and/or [CI] emission lines. We queried the database at the end of August 2022 to select galaxies with spectroscopic redshift in the range 0.5 - 4, with angular resolution $\lesssim 0.5\arcsec$, spectral resolution $\lesssim 50$ km s$^{-1}$ and medium to high signal-to-noise ratio (SNR), i.e., data cubes with a SNR $\gtrsim$ 3 per channel in at least 5 spectral channels. These requirements guarantee data quality sufficient to infer robust kinematic parameters \citep{Rizzo_2022}. The resulting sample consists of 28 galaxies, whose ID, name, coordinates, and redshifts are reported in Table \ref{tab:tab1}. In Fig.~\ref{fig:zdistr} (upper panel), we show the histrogram with the redshift distribution of the ALPAKA sample.

\begin{figure}[th!]
    \begin{center}  \includegraphics[width=0.95\columnwidth]{figures/jzhist.pdf}
        \caption{Distribution of the redshift (upper panel) and the CO and [CI] transitions (bottom panel) of ALPAKA.}  
        \label{fig:zdistr}
    \end{center}
\end{figure}

\subsection{Target characterization}
High-resolution ALMA observations of high-$z$ galaxies require long integration times. Therefore, to optimize the observing strategy, such observations are usually performed on sources for which the fluxes of the target emission lines are known through previous coarse and shallow campaigns. Despite the fact that our targets were selected using an agnostic approach regarding the multi-wavelength coverage, we find that all ALPAKA galaxies %For this reason, all ALPAKA targets have spectroscopic redshifts based on previous optical or FIR/sub-mm spectroscopic campaigns (column 4 in Table \ref{tab:tab1}). 
lie in broadly studied survey area. In Table \ref{tab:tab1}, we list the main features of the targets as reported in the literature: %the spectral range in which they are firstly discovered and 
 the fields in which they lie, and information regarding the environment and the presence of AGN. The latter are based on previous diagnostic features (e.g., rest-frame UV or mid-infrared spectra, X-ray brightness) depending on spectroscopic coverage of the galaxies in different fields. 
Given that the selection criteria we adopt are only based on the data quality of archive data, the ALPAKA sample is heterogeneous: 68\% (19 out of 28) of the targets lie in overdense regions (clusters, groups and protocluster); 7\% (2 out of 28) were previously identified as merging systems and 1 of the two is in group; 25\% (7 out of 28) host an AGN. Among the AGN hosts, 3 sources are protocluster members. A discussion on the potential impact of AGN feedback on the ISM kinematics properties of the host galaxies is provided in Sect.\ref{sec:discussion}.

\begin{table*}[h!] 
\begin{center}
\begin{tabular}{ccccccc}
\\
\hline
\\
ID  &  Name & RA (deg) & Dec (deg) & redshift & Field/Survey & Notes\\
\\
\hline
\\
1 & GOODS-S 15503 & 53.08205 & -27.83995 & 0.561 & GOODS-S & - \\ 
2 & COSMOS 2989680 & 150.43186 & 2.80261 & 0.625 & COSMOS & Minor merger\\
3 & COSMOS 1648673 & 149.98144 & 2.25321 & 1.448 & COSMOS & AGN(?) \\
4 & ALMA.03  & 333.99393 & -17.62988 & 1.453 & XCS & Cluster \\
5 & ALMA.010 & 333.98853 & -17.63149 & 1.454 & XCS & Cluster \\ %-co5field0
  %-co4field0 and co4field1
6 & ALMA.08 & 333.99266 & -17.63950 & 1.457 & XCS & Cluster\\ % -co3field0
7 & ALMA.01 & 333.99233 & -17.63737 & 1.466 & XCS & Cluster\\%
8 & ALMA.06 & 333.99880 & -17.63306 & 1.467 & XCS & Cluster \\% -co2field1-co2field0
9 & ALMA.013 & 333.99909 & -17.63797 & 1.471 & XCS & Cluster\\
10 & SHiZELS-19 & 149.79817 & 2.39008 & 1.484 & COSMOS & - \\
11 & SpARCS J0225-371 & 36.44191 & -3.92436 & 1.599 & SpARCS & Cluster \\%
12 & SpARCS J0224-159 & 36.11320 & -3.40037 & 1.635 & SpARCS & Cluster \\%
13 & COSMOS 3182 & 150.07594 & 2.21182 & 2.1028 & COSMOS & Protocluster \\
14 & Q2343-BX610 & 356.53934 & 12.82202 & 2.210 & SINS & AGN(?)\\
15 & GS30274 & 53.13114 & -27.77319 & 2.225 & GOODS-S & AGN \\
16 & HXMM01-a & 35.06938 & -6.02830 & 2.308 & HerMES & Group\\
17 & HXMM01-b+c & 35.06907 & -6.02904 & 2.308 & HerMES &  Group, merger\\
18 & HATLAS J084933-W & 132.38994 & 2.24573 & 2.407 & H-ATLAS & AGN, protocluster\\
19 & CLJ1001-131077 & 150.23728 & 2.33813 & 2.494 & COSMOS & Cluster \\ %CO_1_3_3/CO_1_4_3 - SB1
20 & CLJ1001-130949 & 150.23691 & 2.33579 & 2.503 & COSMOS & Cluster\\ %CO_1_3_2/CO_1_4_4 - MS1
21 & CLJ1001-130891 & 150.23986 & 2.33646 & 2.512 & COSMOS & Cluster\\ %CO_1_4_1/CO_1_3_1 -SB2
22 & Gal3 & 150.33139 & 2.16239 & 2.935 & COSMOS & -\\
23 & ADF22.1 & 334.38507 & 0.29551 & 3.089 & SSA2 & AGN, protocluster \\%CO_1
24 & ADF22.5 & 334.38416 & 0.29323 & 3.089 & SSA2 & Protocluster  \\%CO_3
25 & ADF22.7 & 334.38120 & 0.29946 & 3.094 & SSA2 & AGN, protocluster  \\%CO_2
26 & Gal5 & 149.87724 & 2.28388 & 3.341 & COSMOS & -\\
27 & Gal4  & 150.27827 & 2.25887 & 3.431 & COSMOS & -\\
28 & W0410-0913 & 62.54425 & -9.21812 & 3.631 & WISE & AGN, protocluster\\
\\
\hline
\end{tabular}
\end{center}
\caption{The ALPAKA sample. Columns 1 and 2: ALPAKA ID and mostly used name for the galaxy. Column 3 and 4: coordinates of the center used for the kinematic fitting in Sect.\ref{sec:kinematic}. Column 5: field or survey where the galaxies lie. Column 6: indication about the environment and the presence of an AGN. The presence of a "?" indicates that for the corresponding galaxy there are hints of the presence of an AGN (see Sect.~\ref{sec:details} for details).
}
\label{tab:tab1}
\end{table*}

\section{Observations}\label{sec:observations}
 \subsection{ALMA data}\label{sec:alma}
 In Table \ref{tab:tab2}, we list the main properties of the ALMA datasets: ALMA project ID, frequency coverage and emission line for each target. When multiple observations at similar angular resolutions are available (e.g., ID10 and 14), we combine them to increase the SNR. Due to the selection criteria, and to the ALMA sensitivity and frequency coverage, the ALPAKA galaxies have observations of various emission lines: 18 targets have low-J CO transitions observations -- 10 with CO(2-1) and 8 with CO(3-2) --, while for 10 targets only high-J CO transitions or CI emission lines are available -- 1 with CO(4-3), 4 with CO(5-4), 1 with CO(6-5), 2 with CO(7-6), 1 with CI(1-0) and 1 with CI(2-1) (see bottom panel in Fig.~\ref{fig:zdistr}). The on-source integration times for each target are listed in Table \ref{tab:tab2} and range from 22 minutes to 14 hours, for a total of $\approx$ 90 hours. The total integration time, including overhead, is instead of 147 hours, corresponding to the duration of an ALMA Large Program.\\
In this paper, we make use of the calibrated measurement sets provided by the European ALMA Regional Centre \citep{Hatziminaoglou_2015}, that calibrated the raw visibility data using the standard pipeline script delivered with the raw observation sets. All of the post-processing steps were handled in the Common Astronomy Software Applications (CASA) suite \citep{McMullin_2007}, version 6. The calibrated data were first inspected to confirm the quality of the pipeline calibration and that no further flagging was required. For data sets containing one single target, we then subtracted the continuum from the line spectral windows using \texttt{UVCONTSUB}. Most of the data are averaged into groups of between two to six channels in order to increase the overall SNR. This procedure results in channels with typical velocity widths ranging from 18 to $\approx$ 50 km s$^{-1}$. The continuum and spectral lines were imaged using the \texttt{TCLEAN} routine in the  CASA package, assuming a natural weighting of the visibilities to maximise the SNR. Targets with high SNR are imaged using a Briggs weighting of the visibilities \citep[robust parameter set equal to 0.5][]{Briggs_1995}, in order to enhance the angular resolution of the output images, without significantly degrading their effective sensitivity. The \texttt{CLEAN} algorithm is run down to a flux threshold of 2 $\times$ RMS, where RMS is the root mean square of the data measured within the dirty data cubes. For datasets containing multiple targets, to better account for the source-to-source variation of the continuum signal, we perform the continuum subtraction using the \texttt{IMCONTSUB} task. In Table \ref{tab:tab2}, we present the main properties of the ALPAKA data cubes: beam size, channel widths, and RMS per spectral channel. The beam of the observations of the ALPAKA targets ranges from 0.1" to 0.5" (median value of 0.25") and the corresponding resolution in physical units varies between 1 kpc and 4 kpc (median value of 2 kpc).

\subsection{HST data}
Spatially resolved observations of the stellar continuum of galaxies are important for the kinematic analysis as they allow us to derive geometrical parameters (e.g., inclination angle). As discussed in detail in Sect.\ref{sec:geometry}, accurate measurements of the inclination and position angles provide, indeed, robust velocity estimate and kinematic characterization. The comparison between the stellar and gas morphology and the gas kinematics of a galaxy allows for identifying any merger or outflow features (see Sect. \ref{sec:kinematic}). Further, combined measurements of the stellar and gas distributions and the rotation curves provide a unique means to infer the dark matter content within galaxies. \\
For 23 ALPAKA galaxies, HST observations are publicly available and are taken from the Complete Hubble Archive for Galaxy Evolution (CHArGE). The latter performs uniform processing of all archival HST imaging and slitless spectroscopy observations of high-$z$ galaxies. In CHArGE, the data were processed with the \texttt{GRIZLI} pipeline \citep{Brammer_2021}, which creates mosaics for all filter exposures that cover a given area of the sky.
All exposures are aligned to each other using different techniques \citep[see][for details]{Kokorev_2022} resulting in a typical astrometric precision $<$ 100 mas.

For each source we select the reddest HST filter (see Table \ref{tab:mstar}) in which the source is detected to minimize biases in the determination of the structural parameters due to the irregular morphology of the galaxies that host UV-bright star-forming regions \citep{Guo_2018, Zanella_2019}. In Table \ref{tab:mstar} we report the central rest-frame wavelength covered by the selected HST filter. For 17 galaxies the HST data cover the rest-frame optical emission ($\gtrsim 4000 \AA$), while 6 galaxies are covered only in the near-UV range. Fig.~\ref{fig:hstalma1} shows the HST images for the 23 ALPAKA targets, while the white contours show the integrated CO or [CI] emission lines from ALMA data.




\begin{table*}[h!]
\begin{center}
\caption{Description of the ALMA observations and datasets. For targets ID4 - 9, the observation consists of three mosaic pointings; the corresponding integration time refers to the total value summed over all pointings. For targets with multiple observations (i.e., ID10 and 14), we image the CO after combining the corresponding measurement sets and we list the resulting properties of the cubes (beam, channel width, RMS). 
}
\begin{tabular}{ccccccccc}
\\
\hline
\\
ID   & Project ID & Line & Frequency Range & Beam & Channel Width & RMS & Integration time  \\
& & &  GHz & (" $\times$ ") & (km s$^{-1}$) & (mJy/beam) & hours\\
\\
\hline
\\
1 & 2017.1.01659.S & CO(2-1) & 146.72 - 148.59 & 0.49 $\times$ 0.39 & 39 & 0.056 & 10.48\\ 
2 & 2016.1.00624.S & CO(3-2) & 212.07 - 231.00 & 0.20 $\times$ 0.16 & 32 & 0.23 & 0.72\\
3 & 2016.1.01426.S & CO(5-4) & 217.00 - 236.54 & 0.14 $\times$ 0.10 & 30 & 0.12 & 1.5 \\
4 & 2017.1.00471.S & CO(2-1) & 92.06 - 93.93 & 0.50 $\times$ 0.34 & 25 & 0.096 & 12.8\\ %-co6field0
5 & 2017.1.00471.S & CO(2-1) & 92.06 - 93.93 & 0.50 $\times$ 0.34 & 25 & 0.094 & 12.8\\ %-co5field0
6 & 2017.1.00471.S & CO(2-1) & 92.06 - 93.93 & 0.50 $\times$ 0.34 & 25 & 0.1 & 12.8\\  %-co4field0 and co4field1
7 & 2017.1.00471.S & CO(2-1) & 92.06 - 93.93 & 0.50 $\times$ 0.35 & 25 & 0.1 & 12.8\\ % -co3field0
8 & 2017.1.00471.S & CO(2-1) & 92.06 - 93.93 & 0.50 $\times$ 0.34 & 25 & 0.1 & 12.8\\% -co2field1
9 & 2017.1.00471.S & CO(2-1) & 92.06 - 93.93 & 0.51 $\times$ 0.35 & 25 & 0.1 & 12.8\\% -co2field0
10 & 2017.1.01674.S &  CO(2-1) & 91.42 - 106.72 & 0.30 $\times$ 0.28 & 25 & 0.069 & 3.65\\
& 2015.1.00862.S &  CO(2-1) & 91.77 - 107.49 & comb. & comb. & comb. & 2.49\\
11 & 2017.1.01228.S & CO(2-1) & 85.90 - 101.81 & 0.53 $\times$ 0.42 & 26 & 0.1 & 2.69\\
%
12 & 2018.1.00974.S & CO(2-1) & 86.72 - 102.53 & 0.39 $\times$ 0.30 & 39 & 0.1 & 2.91\\
13 & 2017.1.00413.S & CI(2-1) & 243.03 - 261.99 & 0.19 $\times$ 0.16 & 18 & 0.15 & 1.09\\
14 & 2019.1.01362.S  & CO(4-3) & 140.58 - 156.27 & 0.14 $\times$ 0.13 & 16 & 0.043 &  13.82\\
& 2017.1.01045.S & CO(4-3) & 140.80 - 156.38 & comb. & comb. & comb. & 3.89\\
& 2013.1.00059.S & CO(4-3) & 140.48 - 156.10 & comb. & comb. & comb. & 1.56 \\
15 & 2018.1.00543.S &  CO(3-2) & 92.34 - 108.18 & 0.28 $\times$ 0.23 & 32 & 0.058 & 5.95\\
%15 & UDS 35673 & 34.2722419 & -5.1571733 & 2.23 & CI(1-0) & 2017.1.00413 & 3DHST/UDS\\
16 & 2015.1.00723.S &  CO(7-6) & 226.21 - 245.50 & 0.22 $\times$ 0.20 & 19 & 0.11 & 2.55\\
17 & 2015.1.00723.S &  CO(7-6) & 226.21 - 245.50 & 0.22 $\times$ 0.20 & 19 & 0.11 & 2.55\\
18 & 2018.1.01146.S &  CI(1-0) & 86.51 - 102.39 & 0.25 $\times$ 0.19 & 32 & 0.082 & 1.33\\
19 & 2016.1.01155.S & CO(3-2) & 85.75-100.61 & 0.35 $\times$ 0.29 & 35 & 0.12 & 3.32\\ %CO_1_3_3/CO_1_4_3 - SB1
20 & 2016.1.01155.S & CO(3-2) & 85.75-100.61 & 0.35 $\times$ 0.29 & 35 & 0.12 & 3.32\\ %CO_1_3_2/CO_1_4_4 - MS1
21 & 2016.1.01155.S & CO(3-2) & 85.75-100.61 & 0.36 $\times$ 0.30 & 35 & 0.13 & 3.32\\ %CO_1_4_2 - MS2
22 & 2017.1.01677.S & CO(5-4) & 133.53 - 149.39 & 0.19 $\times$ 0.16 & 32 & 0.13 & 2.31\\
23 & 2018.1.01306.S & CO(3-2) & 84.09 - 99.78 & 0.24 $\times$ 0.20 & 28 & 0.069 & 5.71\\%CO_1
24 & 2018.1.01306.S & CO(3-2) & 84.09 - 99.78 & 0.24 $\times$ 0.20 & 28 & 0.075 & 5.71\\%CO_2
25 & 2018.1.01306.S & CO(3-2) & 84.09 - 99.78 & 0.24 $\times$ 0.20 & 28 & 0.074 & 5.71\\%CO_3
%29 & [CLG2020] Gal1 & 150.346592 & 2.607239 & 3.1 & CO(5-4) & 2017.1.01677.S & 3\\
26 & 2017.1.01677.S & CO(5-4) & 129.50 - 144.99 &  0.28 $\times$ 0.21 & 35 & 0.11 & 3.40 \\
27 & 2017.1.01677.S & CO(5-4) & 129.50 - 144.99 & 0.29 $\times$ 0.21 & 36 & 0.12 & 3.40\\
28 & 2017.1.00908.S & CO(6-5) & 135.76 - 151.74 & 0.18 $\times$ 0.15 & 31 & 0.06 & 3.61\\
\\
\hline
\end{tabular}
\tablefoot{Here, we provide the list of principal investigators (PIs) for the ALMA projects employed in this work: 2017.1.01659.S, PI: ...; 2016.1.00624.S, PI: ...}
\end{center}
\label{tab:tab2}
\end{table*}

\section{Global properties} \label{sec:gloabl}
Being in well characterized survey areas, extensive studies of the global, unresolved properties (e.g., stellar masses, $M_{\star}$, and star-formation rate, SFR) of the ALPAKA galaxies are already available. However, these parameters are derived using different algorithms and assumptions. 
For this reason, we collect UV-to-radio photometric data using public multi-wavelength catalogues \citep{Fu_2013, Ivison_2013, Magnelli_2013, Hayashi_2018, Liu_2019, Weaver_2022} with the aim of fitting the spectral-energy distribution (SED) in a consistent way. The only exceptions are ID11 and ID12, for which photometric data are not publicly available. Throughout the rest of the paper, for these two galaxies, we will refer to the stellar masses and SFR reported in \citet{Noble_2017} using assumptions consistent with those used for the rest of our sample. \citet{Noble_2017} derived the stellar masses using \texttt{FAST} \citep{Kriek_2009}, while the SFR were estimated by fitting the far-infrared/sub-mm data using dust templates from \citet{Chary_2001}.%A summary of the photometric data for the ALPAKA galaxies is provided in the Master Thesis by D. Reimer Frickmann ( ). \\

To derive the stellar masses and SFR for the ALPAKA galaxies, we perform the SED fitting using \texttt{STARDUST} \citep{Kokorev_2021}. \texttt{STARDUST} performs a multi-component fit that linearly combines three classes of templates: stellar libraries from an updated version of the Stellar Population Synthesis models described in \citet{Brammer_2008}; AGN torus templates from \citet{Mullaney_2011}; infrared (IR) models of dust emission arising from star formation from \citet{Draine_2007, Draine_2014}. These three components  are fitted simultaneously but independently from each other, i.e., without assuming an energy balance between the absorbed UV/optical radiation and the IR emission. This approach allows one to account for spatial offsets between the stellar and dust distributions within a galaxy. For fitting the ALPAKA targets, we include the AGN templates only for galaxies that are previously identified as AGN hosts (Table \ref{tab:mstar}). 

In Table~\ref{tab:mstar}, we show the best-fit stellar masses, SFR, and the star-formation infrared luminosity ($\mathrm{L_{IR}}$) from \texttt{STARDUST}. We note that, for AGN-host galaxies, in addition to star-formation, the dusty tori contribute to the total infrared luminosity. For all ALPAKA galaxies, we consider only the dust-obscured SFR, traced by $\mathrm{L_{IR}}$, and we neglect the contribution from UV emission. For three galaxies --- ID16, 17, 24 --- due to the lack of good coverage in the optical/near-infrared, the uncertainties on the stellar masses are much larger than the parameter values itself. Therefore, the respective estimates of the stellar mass are not stastically meaningful. In addition, due to the lack of a good far-infrared coverage, measuring the SFR for ID24 is challenging. In Fig. \ref{fig:ms}, we show the distribution of the ALPAKA sample in the SFR-$M_{\star}$ plane for the 25 galaxies with a good estimate (i.e., uncertainties smaller than the value) of both parameters. We divide our targets in three redshift bins and compare them with the main-sequence relation at the corresponding redshift. For the latter, we use the parametrization of \citet[][solid lines]{Schreiber_2015} using a Chabrier IMF and show the 0.3 dex scatter at the average redshifts of the galaxies in the three bins, $z = 1.3, 2.2, 3$ (dashed lines). The dot-dashed lines in Fig. \ref{fig:ms} show the empirical lines, located 4 times above the SFR of main sequence that is usually used to divide main-sequence from starburst galaxies \citep{Rodighiero_2011}. According to this definition, a large fraction of ALPAKA galaxies are starbursts (16/25 or 64\%; see column 4 in Table~\ref{tab:mstar}). In the low redshift bin ($z =$0.5 - 1.5), 50\% of ALPAKA galaxies lie within the $\pm1\sigma$ scatter of the main sequence relation, while this fraction falls to 14\% at $z \sim 3$. Further, the ALPAKA sample covers high stellar mass galaxies, $\gtrsim 10^{10} M_{\odot}$ with SFR ranging from 8 to $\sim$3000 M$_{\odot}$\,yr$^{-1}$. This can be ascribed to a selection effect: being discovered as bright sources in the infrared or sub-mm wavelength, ALPAKA galaxies have high SFRs and gas fractions. To visualize this, in Fig.~\ref{fig:lir}, we show the distribution of the ALPAKA sample in the emission line-infrared luminosity planes, an observational proxy of the Kennicutt-Schmidt relation \citep{Schmidt_1959, Kennicutt_1998}, with respect to compilations of local and high-$z$ galaxies from the literature \citep[e.g.,][]{Liu_2015, Silverman_2018, Valentino_2020, Boogaard_2020, Birkin_2021, Valentino_2021}. The CO fluxes and respective luminosities for the ALPAKA sample, as listed in Table~\ref{tab:mstar}, are derived by summing the flux above 3 $\times$ RMS in the high-resolution moment-0 map data presented in Sect.~\ref{sec:alma}. We check that these values are within $\pm20\%$ from the ones obtained by fitting the moment-0 maps with the \texttt{IMFIT} tool within \texttt{CASA}. %. after fitting the velocity-integrated intensity map (Moment 0) with a 2D Gaussian. 
In almost all cases, these fluxes are consistent with previous estimates, mostly obtained through non-resolved observations, showing that we are not resolving out some flux in these high-resolution observations. The only exceptions are ID18, and 27, for which only 67, and 57\% of the fluxes, respectively, are recovered with respect the one found at 1" resolution \citep{Ivison_2013, Cassata_2020}. \textcolor{red}{I will check that by imaging these cubes with natural weighting and/or including the 1" ms, I recover the total flux} To derive the infrared luminosities listed in Table~\ref{tab:mstar}, we integrated the best-fit dust model employed by \texttt{STARDUST} in the range 8 - 1000 $\mu$m. The samples of galaxies from the literature comprise: the compilation in \citet{Valentino_2020} consisting of 30 main-sequence galaxies at $z \sim 1$ and 65 submillimiter galaxies (SMGs) and quasars at $z \sim 2.5$ \citep{Walter_2011, Alaghband-Zadeh_2013, Canameras_2018, Yang_2017, Andreani_2018} and 146 local starbursts from \citet{Liu_2015}; 12 starbursts at $z \sim 1.6$ \citep{Silverman_2018}; 22 main-sequence and starburst galaxies at $z \sim 1.4$ from the ALMA Spectroscopic Survey in the Hubble Ultra Deep Field survey \citep{Boogaard_2020}; 47 SMGs at a median $z$ of 2.5 from \citet{Birkin_2021}. Considering the substantial effort in obtaining high SNR spatially resolved ALMA observations at $z \sim 0.5$ - 4, the ALPAKA sample covers the brightest part of the distributions in each panel of Fig.\ref{fig:lir}, with a median luminosity of $2 \times 10^{10}$ K\,km\,s$^{-1}$ pc$^{2}$. ALPAKA galaxies are, therefore, biased towards the most actively star-forming galaxy populations (see discussion in Sect. \ref{sec:discussion}). 

%\section{Notes}
%Source 2 is in the PHIBBS2 survey (Freundlich et al. 2019).\\
%Source 3 is known as PACS819. This galaxy is classified as a clumpy galaxy by Silverman 2015

\begin{table*}[h!] 
\begin{center}
\begin{tabular}{ccccccccc}
\\
\hline\hline
\\
ID  &  M$_{\star}$ & SFR & Type & L$_{\mathrm{IR}}$ & I$_{\mathrm{CO/[CI]}}$ &  L'$_{\mathrm{CO/[CI]}}$ &  HST Filter & $\lambda_{\mathrm{rest, eff}}$\\
 & $10^{10}$M$_{\odot}$ & M$_{\odot}$ yr$^{-1}$ & & $10^{12}$ L$_{\odot}$ & Jy km/s & $10^{10}$ K km s$^{-1}$ pc$^{2}$ & & \AA\\
\hline
\\
1 & 1.1 $\pm$ 0.3 & 8 $\pm$ 2 & MS & 0.8 $\pm$ 0.2 & 0.32 $\pm$ 0.07 & 0.14 $\pm$ 0.03 & F160W & 9794\\ 
2 & 5.7 $\pm$ 0.5 & 28 $\pm$ 9 & MS & 2.8 $\pm$ 0.9 & 1.54 $\pm$ 0.05 & 0.37 $\pm$ 0.01 & F814W & 4921\\
3 & 3.2 $\pm$ 1.1 & 281 $\pm$ 18 & SB & 28 $\pm$ 2 & 3.2 $\pm$ 0.7 & 1.5 $\pm$ 0.3 & F814W & 3254\\
4 & 9 $\pm$ 4 & 220 $\pm$ 50 &  SB & 22 $\pm$ 5 & 0.88 $\pm$ 0.07 & 2.5 $\pm$ 0.2 & F160W & 6236\\
5 & 3.6 $\pm$ 0.5 & 92 $\pm$ 24 & SB & 9 $\pm$ 2 & 0.67 $\pm$ 0.05 & 1.9 $\pm$ 0.1 & F160W & 6236\\ 
6 & 5.8 $\pm$ 1.0 & 67 $\pm$ 20 &  MS & 6.7 $\pm$ 2.0  & 0.46 $\pm$ 0.07 & 1.3 $\pm$ 0.2 & F160W & 6210\\ 
7 & 7.6 $\pm$ 0.9 & 170 $\pm$ 30 &  SB & 17 $\pm$ 3 & 0.75 $\pm$ 0.02 & 2.22 $\pm$ 0.06 & F160W & 6185\\%
8 & 11 $\pm$ 3 & 141 $\pm$ 33 &  MS & 14 $\pm$ 3 & 0.9 $\pm$ 0.2 & 2.6 $\pm$ 0.7 & F160W & 6185\\
9 & 3.9 $\pm$ 0.9 & 48 $\pm$ 11 &  MS & 4.8 $\pm$ 1.2 & 0.48 $\pm$ 0.05 & 1.4 $\pm$ 0.1 & F160W & 6185\\
10 & 4.7 $\pm$ 0.4 & 229 $\pm$ 15 & SB & 23 $\pm$ 2 & 0.72 $\pm$ 0.09 & 2.15 $\pm$ 0.27 & F160W & 6160\\
11 & 6.3 $\pm$ 0.8 & 174 $\pm$ 78 & SB & - & 1.19 $\pm$ 0.13 & 4.07 $\pm$ 0.44 & F160W & 5899\\%
12 & 5.9 $\pm$ 1.8 & 217 $\pm$ 82 & SB & - & 0.5 $\pm$ 0.2 & 1.99 $\pm$ 0.72 & F160W & 4759\\
13 & 12 $\pm$ 3 & 230 $\pm$ 56  & MS & 24 $\pm$ 10 & 1.96 $\pm$ 0.72 & 0.90 $\pm$ 0.33 & F160W & 5092\\
14 & 11 $\pm$ 3 & 441 $\pm$ 77 & SB & 44 $\pm$ 17 & 1.83 $\pm$ 0.06 & 2.8 $\pm$ 0.4 & F140W & 4279\\
15 & 25 $\pm$ 5 & 215 $\pm$ 40 & MS & 21.5 $\pm$ 5.2 & 0.8 $\pm$ 0.02 & 2.23 $\pm$ 0.06 & F160W & 4737\\
16 & - & 730 $\pm$ 300 & - & 73 $\pm$ 13 & 6.3 $\pm$ 0.8 & 3.4 $\pm$ 0.4 & F110W & 3384\\
17 & - & 1360 $\pm$ 680 & - & 136 $\pm$ 25 &  2.4 $\pm$ 0.5 & 1.3 $\pm$ 0.3 & F110W & 3384\\
18 & 12 $\pm$ 3 & 3150$\pm$830  & SB & 315 $\pm$ 83 & 2.7 $\pm$ 0.2 &  $\pm$  & F110W & 3285\\
19 & 8 $\pm$ 2 & 1100$\pm$150  & SB & 110 $\pm$ 15 & 1.2 $\pm$ 0.1 & 4.1 $\pm$ 0.3 & F160W & 4377\\ 
20 & 32$\pm$8 & 130$\pm$170  & MS & 13 $\pm$ 17 & 0.5 $\pm$ 0.1 & 1.8 $\pm$ 0.3 & F160W  & 4365\\ 
21 & 21$\pm$6 & 323$\pm$70  & SB & 32 $\pm$ 7 & 0.6 $\pm$ 0.1 & 2.2 $\pm$ 0.5 & F160W & 4352\\ 
22 & 7$\pm$2 & 804$\pm$313  & SB & 80 $\pm$ 31 & 1.2 $\pm$ 0.4 & 1.9 $\pm$ 0.7 & F160W & 3887\\
23 & 53 $\pm$ 30 & 898$\pm$590  & MS & 90 $\pm$ 59 & 1.0 $\pm$ 0.5 & 4.9 $\pm$ 3.6 & - & - \\%CO_1
24 & - & - & - & - & - & 1.7 $\pm$ 1.3 & - & -\\%CO_3
25 & 11 $\pm$ 5 & 704 $\pm$ 400 & SB & 70 $\pm$ 40 & 0.6 $\pm$ 0.2 & 2.8 $\pm$ 1.0 & - & - \\%CO_2
26 & 7 $\pm$ 4 & 916 $\pm$ 70 &  SB & 92 $\pm$ 7 & 0.86 $\pm$ 0.06 & 1.7 $\pm$ 0.1 & - & -\\
27 & 8.5 $\pm$ 4.0 & 2150 $\pm$ 210 &  SB & 215 $\pm$ 21 & 2.90 $\pm$ 0.08 & 5.9 $\pm$ 0.2 & - & -\\
28 & 13 $\pm$ 6 & 1848 $\pm$ 900 & SB & 184 $\pm$ 90 & 5.20 $\pm$ 0.05 & 8.16 $\pm$ 0.08 & F160W & 3299
\\
\hline
\end{tabular}
\end{center}
\caption{Properties of the ALPAKA sample and description of the HST dataset. Column 4 indicates whether the galaxy is a main-sequence (MS) or a starburst (SB). The last column shows the rest-frame effective wavelength probed by the HST data.
%F140W: 13734.66; F160W: 15278.47, F110W: 11200.52; F814W: 7973.39
}
\label{tab:mstar}
\end{table*}

\begin{figure*}[h!]
    \begin{center}
        \includegraphics[width=1\textwidth]{figures/main_seq.pdf}
        \caption{Distribution of the ALPAKA galaxies in the stellar mass (M$_{\star}$) - SFR plane, divided in three redshift bins. The solid line in the three panels show the empirical main-sequence relations from \citet{Schreiber_2015} at $z = 1.3, 2.2, 3$, that are the average redshifts of ALPAKA targets in the three bins. The dashed and dot-dashed lines show the $\pm 1\sigma$ scatter and the line dividing the main-sequence and starburst galaxies \citep{Rodighiero_2011}. We note that only the 25 ALPAKA targets with good estimates of both M$_{\star}$ and SFR are shown.}  
        \label{fig:ms}
    \end{center}
\end{figure*}


\begin{figure*}[h!]
    \begin{center}
        \includegraphics[width=1\textwidth]{figures/lum_co.pdf}
        \caption{Distribution of the ALPAKA galaxies and samples of local and high-$z$ galaxies from the literature \citep[e.g.,][]{Liu_2015, Silverman_2018, Valentino_2020, Boogaard_2020, Birkin_2021} in the emission line-infrared luminosity planes.}  
        \label{fig:lir}
    \end{center}
\end{figure*}


%WFC3/F160W
%WFC3/F140W
\begin{figure*}[h!] 
    \begin{center}
        \includegraphics[width=\textwidth]{figures/almahst_1.pdf}
        \caption{HST images with contours of the CO or [CI] moment-0 maps from ALMA for the 23 ALPAKA galaxies with HST observations. The CO transitions and HST filters shown here are listed in Tables \ref{tab:tab2} and \ref{tab:mstar}, respectively. The white bar in the bottom left of each panel shows a scale of 1".}    
        \label{fig:hstalma1}
    \end{center}
\end{figure*}



\section{Kinematic analysis}\label{sec:kinematic}
In this section, we describe how we analyze the kinematics of the ALPAKA galaxies by fitting the data using rotating disk models (Sec.\,\ref{sec:barolo}). This allows us to provide a first-order description of the gas motions within the ALPAKA sample. In Sect. \ref{sec:kinclassification}, we show how we identify any deviations from pure circular orbits, likely due to radial and vertical motions driven by outflows and mergers. The kinematic properties of each ALPAKA target are described in Sect.~\ref{sec:details}.
\subsection{Disk modeling}\label{sec:barolo}
We fitted the kinematics of the ALPAKA galaxies using the software \bba\ \citep{DiTeodoro_2015}. \bba\ produces three dimensional (two spatial, one spectral axis) realizations of a so-called tilted-ring model \citep{Rogstad_1974}. The latter consists of a disk divided into a series of concentric circular rings, each with its kinematic (i.e., systemic velocity $V_{\mathrm{sys}}$, rotation velocity $V_{\mathrm{rot}}$ and velocity dispersion $\sigma$) and geometric properties (i.e., center, inclination angle $i$ and position angle $PA$).
For a thin disk model, the line-of-sight velocity $V_{\mathrm{los}}$ at a radius $R$ is given by
\begin{equation}
V_{\mathrm{los}}(R) = V_{\mathrm{sys}} + V_{\mathrm{rot}}(R) \cos \phi \sin i,
\label{eq:vlos}
\end{equation}
where $\phi$ is the azimuthal angle in the disk plane. \\
The best-fit model is obtained by means of a least-square minimization. At each step of the model optimization and before calculating the residuals between the data and the model, \bba\ convolves the model disk with a Gaussian kernel with sizes and position angle equal to the beam of the corresponding observation. In the case of the ALMA observations, this is set to be equal to the synthesized beam of the cleaned image. This approach allows for a robust recovery of the rotation velocity and velocity dispersion profiles, since it largely mitigates the effects of beam smearing also in the case of data with relatively low angular resolution \citep{DiTeodoro_2015, Rizzo_2022}.

\subsubsection{Geometrical parameters from morphological fitting} \label{sec:geometry}
\bba\ can estimate the geometrical parameters, namely center, inclination, and position angle of each tilted ring component. However, due to the relatively small number of resolution elements covering the CO/[CI] line emission, we prefer to reduce the number of free parameters by fixing the center and inclination. In particular, estimating the inclination of the galaxies is crucial. Correcting for it can in fact account for a large fraction of the rotation velocity if the galaxies are seen at low inclinations, due to the $\sin i$ dependence of Eq. \ref{eq:vlos}. When dealing with low-resolution observations of low-$z$ and high-$z$ galaxies, the inclination is usually fixed to the one estimated from the optical images \citep{deBlok_1996, Lelli_2016, Wisnioski_2019}. However, since only a fraction of ALPAKA galaxies have HST data covering the rest-frame optical emission, we use two methods for estimating their inclinations:
\begin{itemize}
    \item GALFIT on HST data. For the 23 galaxies with HST data, we used \texttt{GALFIT} \citep{Peng_2002} to model their 2D surface brightness using one or two Sersic components (see details in Appendix \ref{sec:galfit}). \texttt{GALFIT} fits the center, the three parameters describing the Sersic profile (total magnitude, Sersic index, effective radius), the position angle $PA_{\mathrm{HST}}$ and the axis ratio $b/a$ between the projected major and minor axis. The latter allows for computing the inclination angles, $i_{\mathrm{HST}} = \arccos(b/a)$. In Table \ref{tab:PA}, we report the best-fit geometrical parameters, $PA_{\mathrm{HST}}$, $i_{\mathrm{HST}}$. In Figs. \ref{fig:galfit1} and \ref{fig:galfit2}, we show the HST data and the corresponding GALFIT models and residuals.
    \item CANNUBI on ALMA data. \texttt{CANNUBI} is a Monte Carlo Markov Chain algorithm that models the geometry of galaxies without assuming parametric descriptions of the surface brightness distribution. \texttt{CANNUBI} fits the moment-0 maps using resolution-matched 3D tilted-ring models of rotating disks \citep[see details in][]{Roman_2023}. The free parameters of the fit are the center of the disk, its radial extent, the position and inclination angles ($PA_{\mathrm{ALMA}}$, $i_{\mathrm{ALMA}}$). 
\end{itemize}
To perform a brief check of the independence of our results on the specific method used to model each data set, we repeat the analysis of ALMA data using \texttt{GALFIT}. Since \texttt{GALFIT} is optimized for dealing with optical/near-infrared images (e.g., units in counts, magnitude zero-points, Poissonian error in each pixel), we applied some arbitrary conversion factors to fit the moment-0 maps obtained from ALMA data. The resulting best-fit inclinations are consitent within 5\% with the ones obtained with \texttt{CANNUBI}. 


\begin{table}[h!]
\begin{center}
\begin{tabular}{lllllll}%{ccccccc}
\\
\hline
\\
ID   & $PA_{\mathrm{HST}}$ & $i_{\mathrm{HST}}$ & $PA_{\mathrm{ALMA}}$ & $i_{\mathrm{ALMA}}$ & $PA_{\mathrm{kin}}$ & Kinematic\\
& & & & & & class\\
\\
\hline
\\
1 & 253 & 53 & 244 $^{+9}_{-9}$ & 47 $^{+6}_{-8}$ & 259 $\pm$7 & D\\ 
2 & 223 & 60 & 210 $^{+5}_{-6}$    & 77 $^{+5}_{-5}$ & 215 $\pm$8 & DD\\
3 & 129 & 84 & 116 $^{+11}_{-13}$ & 28 $^{+7}_{-6}$ & 134 $\pm$11 & DD\\
4 & 167 & 36 & 168 $^{+30}_{-34}$ & 52 $^{+15}_{-24}$ & 158 $\pm$10 & M\\
5 & 358 & 37 &  290 $^{+51}_{-60}$ & 26 $^{+11}_{-8}$ & 328 $\pm$20 & M\\
6 & 216 & 45 & 199 $^{+31}_{-29}$  & 43 $^{+21}_{-19}$ & 213 $\pm$8 & DD\\
7 & 272 & 37 &  319 $^{+55}_{-59}$ & 35 $^{+18}_{-13}$ & 294 $\pm$3 & DD\\
8 & 206 & 37 & 201 $^{+21}_{-27}$ & 48 $^{+13}_{-21}$ & 216 $\pm$10 & DD\\
9 & 128 & 46 & 130 $^{+22}_{-17}$ & 44 $^{+13}_{-17}$ & 120 $\pm$3 & DD\\
10 & 103 & 47 & 92 $^{+35}_{-43}$ & 30 $^{+12}_{-11}$ & 97 $\pm$7 & M\\
11 & 113 & 26 & 118 $^{+30}_{-37}$ & 31 $^{+12}_{-11}$ & 111 $\pm$9 & DD\\
12 & 95 & 66 & 102 $^{+38}_{-27}$ & 37 $^{+14}_{-15}$ & 103 $\pm 10$ & DD\\
13 & 344 & 36 & 10 $^{+26}_{-23}$ & 24 $^{+8}_{-6}$ & 37 $\pm$4 & D\\
14 & 193 & 53 &  323 $^{+14}_{-11}$  & 58 $^{+14}_{-8}$ & 308 $\pm$10 & DD\\
15 & 92 & 60 & 112 $^{+21}_{-16}$ & 42 $^{+10}_{-15}$ & 120 $\pm$20  & DD\\
16 & 17 & 79 & 13 $^{+3}_{-3}$  & 73 $^{+4}_{-4}$ & 15 $\pm$10  & M\\
17 & 354 & 66 & 359 $^{+7}_{-6}$  & 65 $^{+8}_{-11}$ & 1 $\pm$20  & M\\
18 & 267 & 26 & 267 $^{+30}_{-30}$ & 27 $^{+9}_{-8}$ & 248 $\pm$15 & D\\
19 & 32 & 60 & 359 $^{+5}_{-5}$ &  71 $^{+3}_{-3}$ & 23 $\pm$15 & DD\\
20 & 154 & 45 & 144 $^{+11}_{-11}$ & 56 $^{+9}_{-12}$ & 141 $\pm$8 & DD\\
21 & 13 & 26 & 52 $^{+51}_{-52}$  & 36 $^{+19}_{-14}$ & 13 $\pm$5 & M\\
22 & 107 & 66 & 173 $^{+32}_{-39}$ & 26 $^{+9}_{-8}$ & 130 $\pm$6 & M\\
23 & - & - & 86 $^{+3}_{-2}$ & 70 $^{+3}_{-3}$ & 97 $\pm$9 & D\\
24 & - & - & 16 $^{+4}_{-3}$ & 75 $^{+6}_{-6}$ & 19 $\pm$4 & D\\
25 & - & - & 76 $^{+4}_{-4}$ & 74 $^{+5}_{-4}$ & 75 $\pm$8 & M\\
26 & - & - & 317 $^{+51}_{-57}$ & 38 $^{+19}_{-15}$ &  5 $\pm$5 & M\\
27 & - & - & 142 $^{+4}_{-4}$ & 62 $^{+3}_{-4}$ & 97 $\pm$19 & M\\
28 & 93 & 53 & 133 $^{+13}_{-12}$ & 52 $^{+13}_{-18}$ & 140 $\pm$5 & DD\\
\\
\hline
\end{tabular}
\end{center}
\caption{Columns 2 - 5: values of the position and inclination angles, as derived by \texttt{GALFIT} using the HST data (when available) and \texttt{CANNUBI} using the ALMA moment-0 map. Column 3: kinematic position angles derived by \bba\. Column 4: kinematic class of each ALPAKA source (D: disk, DD: disturbed disk, M: merger).
}
\label{tab:PA}
\end{table}

Fig.~\ref{fig:ipa} shows the distribution of the difference between the inclination angles found with the two methods (upper left panel). For 21 out of the 23 ALPAKA galaxies with HST data, $i_{\mathrm{HST}} - i_{\mathrm{ALMA}}$ values are within 20 $\deg$. By comparing the inclination angles obtained with \texttt{GALFIT} and \texttt{CANNUBI} (bottom left panel in Fig.\ref{fig:ipa}), we find that for 65\% of the sample, they are consistent within the $\pm1\sigma$ uncertainties. The only galaxies with a difference between the two inclination angles $\gtrsim 20$ $\deg$ are ID3 and ID22, with $i_{\mathrm{HST}}-i_{\mathrm{ALMA}}$ of 54 and 42 $\deg$, respectively. However, the HST data for both ID3 and ID22 cover only their rest-frame near UV emission and they may be not representative of the bulk of the stellar population. ID3 has, in fact, two UV bright clumps, clearly visible in Fig.\ref{fig:galfit1}, while the CO emission has a smooth distribution and its center is located between them (Fig.\ref{fig:hstalma1}). ID22 is one of the galaxies that we classify as a merger based on its kinematics (see Sect.\ref{sec:kinclassification}). To summarize, for most of the galaxies with HST data, the value of $i_{\mathrm{HST}}$ are consistent with $i_{\mathrm{ALMA}}$, despite that these two values are derived by fitting two different components of the galaxy (i.e., stellar continuum and gas emission line) and using different tools and assumptions. 

For the rest of this subsection, we attribute to each ALPAKA galaxy the value of $i_{\mathrm{HST}}$ if HST images are available, and $i_{\mathrm{ALMA}}$ otherwise. The median value of this inclination distribution is 53$\pm 16 \deg$, that is, smaller, although within 1$\sigma$, than the average inclination of 60 $\deg$ expected from the observation of randomly oriented galaxies  \citep{Romanowsky_2012}. A potential reason for this discrepancy can be due to a bias towards low-inclination galaxies. \citet{Kohandel_2019} showed, indeed, that the line detection of galaxies is more challenging for edge-on rather than face-on galaxies. At fixed line luminosity, the Full Width at Half Maximum is larger in the edge-on case, pushing the peak flux below the detection limit. Since most ALPAKA observations were obtained as high-resolution follow-up of line-detected samples \citep[e.g.,][]{Brisbin_2019, Ivison_2019, Cassata_2020}, it could be that they suffer from this bias. To check this, we compare the distribution of the CO and [CI] fluxes with unresolved samples observed with ALMA in Fig.~\ref{fig:flux}. The ALPAKA galaxies with data from low-J CO transitions -- CO(2-1) and CO(3-2) -- have fluxes consistent with the samples from the literature, while ALPAKA targets with high-J transitions measurements are the brightest among the galaxies observed with ALMA. In other words, ALPAKA galaxies detected in CO(2-1) and CO(3-2) are representative of galaxies with fluxes close to the detection limit. For this reason, they are likely biased towards low inclinations. 
To visualize this comparison, we bin the distribution from the literature and the ALPAKA sample into two bins: "low-J" that contains the fluxes from CO(2-1) and CO(3-2) and "high-J" with the remaining CO transitions and the CI fluxes (Fig.~\ref{fig:flux}).

\begin{figure}[h!] 
    \begin{center}
        \includegraphics[scale=0.8]{figures/flux_co_targets.pdf}
        \caption{CO and [CI] emission-line fluxes of ALPAKA galaxies and high-$z$ galaxy samples observed with ALMA. \textit{Upper panel:} ALPAKA galaxies with low-J transitions -- CO(2-1) and CO(3-2) -- measurements have fluxes which are consistent with the bulk of the other samples from the literature. \textit{Bottom panel:} at high-J transitions, ALPAKA galaxies cover, instead, the brightest part of the distribution.} 
        \label{fig:flux}
    \end{center}
\end{figure}

Another potential reason for the low inclination of the ALPAKA sample may be due to our assumption of the thickness of the disks. By assuming the the disks are thin, we derived the inclination angles from the measurements of the axis ratio $b/a$. However, if galaxies have a thick disk, these inclination angles are underestimated. A common assumption for deriving the inclination of thick disks is to estimate the inclination using the following equation, 
\begin{equation}
    \cos i = \sqrt{\frac{(b/a)^2 - Q^2}{1- Q^2}}
    \label{eq:cos}
\end{equation}

where $Q$ is the intrinsic axis ratio that is assumed equal to 0.2 \citep[e.g.,][]{Forster_2018}. If we use eq.~\ref{eq:cos} to compute the inclination angles, the values change by 2 to 7\%, with a median value of the distribution of 54 degrees, consistent with the one obtained with the thin disk assumption. Further, we note that a  disk thickness equal to 0.2 times the effective radii of the line emission distribution of ALPAKA galaxies falls below the resolution of these observations and therefore they cannot be constrained.


\subsubsection{Assumptions for the kinematic fitting}
In this section, we describe the assumptions we made to run \bba\ and fit the kinematics of the ALPAKA sample:
\begin{itemize}
    \item \textit{Mask.} Before fitting the data, \bba\ uses the source finder derived from \texttt{DUCHAMP} \citep{Whiting_2012} to build a mask on the regions that are identified as containing the emission from the target. Within \bba, different parameters can be used to define how to build the mask. For our sample, we selected the parameters \texttt{SNRCUT} and \texttt{GROWTHCUT} that define the primary and secondary SNR cuts applied to the data.
    Once the emission pixels with flux above a threshold defined by \texttt{SNRCUT} are identified, the algorithm increases the detection area by adding nearby pixels that are above some secondary threshold defined by \texttt{GROWTHCUT} and not already part of the detected object. For ALPAKA galaxies, we used \texttt{SNRCUT} values of 2.5 - 4 and \texttt{GROWTHCUT} of 2 - 3 depending on the quality of each data cube.
    \item \textit{Radial separation.} To keep the number of modeled rings as close as possible to the number of resolution elements, we used a radial separation between rings close to $0.5$ - 1 times the angular resolution (see Sect.\ref{sec:disks}). %Only in XX galaxies we had to employ radial separations that are a factor of XX smaller than the resolution of observations in order to have the minimum number of 2 rings, necessary for fitting a disk model. 
    This assumption ensures that the rotation curves and the velocity dispersion profiles are sampled with almost independent points.  
    \item \textit{Center.} We fixed the galactic centers to the values obtained with \texttt{CANNUBI}. In some cases, we changed them by less than 2 pixels\footnote{The ALMA data are imaged using pixel size equal to 1/5 of the beam size.} after visually inspecting the position-velocity diagrams (PVDs). The latter are cuts of the cubes along the major and minor axis (see Sect.\ref{sec:kinclassification} for details). The coordinates of the centers are listed in Table \ref{tab:tab1}.
    \item \textit{Inclination angle.} As discussed in Sect.\ref{sec:geometry}, for the 23 galaxies with HST data, we fixed the inclination angles to the ones found by GALFIT fitting, $i_{\mathrm{HST}}$, while for the remaining 5 galaxies, we fixed them to the values $i_{\mathrm{ALMA}}$ found by \texttt{CANNUBI}.
\end{itemize}
Using the assumptions described above, we run \bba\ and we fit the rotation velocity $V_{\mathrm{rot}}$, the velocity dispersion $\sigma$ and the position angles. For each galaxy, the number of free parameters is equal to $N \times V_{\mathrm{rot}} + N \times \sigma + PA_{\mathrm{kin}}$, where $N$ is the number of rings over which the galaxy disk is divided. With the angular resolution of the ALPAKA galaxies, $N$ range from 2 to 5.

\begin{figure*}[h!]
    \begin{center}      
    \includegraphics[width=\textwidth]{figures/pa_i.pdf}
         \caption{\textit{Left panels:} comparison between the inclination angles derived from the morphological fitting of HST and ALMA data using \texttt{GALFIT} and \texttt{CANNUBI}, respectively. \textit{Central panels:} comparison between the position angles derived from the morphological fitting of HST data with \texttt{GALFIT} and kinematic fitting of ALMA data with \bba\. \textit{Right panels:} comparison between the position angles derived from the morphological fitting of ALMA data with \texttt{CANNUBI} and the corresponding kinematic fitting. In the bottom panels the gray line shows the 1:1 relation.}    
        \label{fig:ipa}
    \end{center}
\end{figure*}

\subsubsection{Outputs}
For each galaxy, we show in Fig. \ref{fig:pv1} and Figs. \ref{fig:pv2} - \ref{fig:pv4}, the moment-0 and 1 maps and the PVDs along the major and minor axis for the data (dashed black contours) and the model (red solid contours). In Appendix \ref{sec:channels}, we show 7 representative channel maps of the data, model, and residuals for each ALPAKA galaxy. The moment maps of the models are not shown here because, as extensively discussed in \citet{Rizzo_2022}, these are not as informative as the data cubes. In fact, moment maps are obtained by projecting the data cubes to 2D space and masking low SNR pixels in each spectral channel. On the other hand, as discussed in Sect. \ref{sec:kinclassification}, the PVDs are instead the most informative way to visualize the kinematics of galaxies.\\
The best-fit $PA_{\mathrm{kin}}$ as fit by \bba\ are listed in Table \ref{tab:PA}. In Fig. \ref{fig:ipa}, we compare these values with $PA_{\mathrm{HST}}$ and $PA_{\mathrm{ALMA}}$. As shown in the histograms and the scatter plots in the central panels, for 90\% of the 23 galaxies with HST data, the difference between the morphological PA and the kinematic ones are within 30 $\deg$. Instead, ID13, 14 and 28 have $PA_{\mathrm{HST}} - PA_{\mathrm{kin}}$ of 54, 65 and 47 $\deg$. The difference between $PA_{\mathrm{ALMA}}$ and $PA_{\mathrm{kin}}$ is $\lesssim 30 \deg$ for 23 out of the 28 ALPAKA galaxies, while ID5, 21, 22, 26, 27 have differences $\gtrsim 38 \deg$. In Sect. \ref{sec:kinclassification}, we discuss potential explanations of the discrepancies between $PA_{\mathrm{kin}}$, $PA_{\mathrm{ALMA}}$ and $PA_{\mathrm{HST}}$ for this subsample of ALPAKA galaxies. A detailed description of the best-fit rotation velocity and velocity dispersion values is, instead, given in Sect. \ref{sec:discussion}. 
%ID13: galaxy with PAhst - Pakin of 115\\
%ID12: galaxy with PAhst - Pakin of 48\\
%ID28: galaxy with PAhst - Pakin of 44\\
%\\
%PAalma - pakin\\
%ID5: -38.0 m \\
%ID18: -64.0 m\\
%ID 21`; 39
%ID22: 41.0 m\\
%ID26: -41.0 m\\
%ID27: 45.0 m

\begin{figure*}[h!]
    \begin{center}        \includegraphics[width=0.95\textwidth]{figures/pv1.pdf}
        \caption{For each target with ID in the upper left, we show from top to bottom: the moments-0 and 1 maps and the major and minor-axis PVDs. In the moment-0 maps, the first external contour is a "pseudo-contour" (see Appendix \ref{sec:pseudo} for details) at 4 RMS. In the moment-1 map (or velocity field), the black lines show the iso-velocity contours, with thickest one indicating the systemic velocity. The black dashed and gray dotted lines are the kinematic and morphological position angles, respectively. For the morphological position angle, we show the one obtained from fitting HST data when available and the moment-0 map otherwise. The beam is shown in the bottom left. In the PVDs, the y-axis shows the line-of-sight velocities centred on the systemic velocity and the x-axis shows the distance with respect to the center. The contours for the data (solid black) and the model (red) are at [1, 2, 4, 8, 16, 32] $\times$ 2.5 RMS. The gray dotted contours are at -2.5 RMS. The white circles are the best-fit line-of-sight rotation velocities. The width of the boundaries in the PVDs indicate the kinematic class of the corresponding target: disk (thin, e.g., ID1), disturbed disk (medium, e.g., ID2 and 3), merger (thick, ID4). The arrows show kinematic anomalies that are described and discussed in detail in Sect.~\ref{sec:details}.} 
        \label{fig:pv1}
    \end{center}
\end{figure*}

\subsection{Kinematic classification} \label{sec:kinclassification}
The moment-1 maps of the ALPAKA galaxies are characterized by a smooth gradient. Therefore, we could, in principle, conclude that all ALPAKA galaxies are smooth rotating disks. However, as shown in \citet{Simons_2019} and \citet{Rizzo_2022}, the velocity map of a merging system can be similar to that of a smooth rotating disk as, due to the angular resolution of observations, the irregularity and asymmetries are smoothed out. Further, even outflows result in velocity gradients that can be erroneously interpreted as rotation \citep{Loiacono_2019}. In this Section, we will discuss how we identify any non-circular motions in the ALPAKA targets and build a subsample of galaxies where the presence of a rotating disk can be considered robust.
\subsubsection{Visual inspection} \label{sec:visclass} 
For rotating disks, the PVDs have specific features: the major-axis PV has an S-shape profile, and the minor-axis PV has a diamond shape, symmetric with respect to the axes defining the center and the systemic velocity. Using mock ALMA data of simulated galaxies, \citet{Rizzo_2022} show that these features are imprinted in the data even for barely resolved observations. At the typical resolutions of high-$z$ observations, the flux distribution along the major-axis PVD has two symmetric brightest emission regions in the approaching and receding sides, along the horizontal parts of the S-shape. In addition, using geometrical arguments, it can be shown that for an axisymmetric rotating disk, the kinematic position angle should be aligned with its projected morphological major axis. Differences between these two angles of $\gtrsim 30 \deg$ can be ascribed to a 
 variety of reasons -- e.g., presence of outflows \citep[e.g.,][]{Lelli_2018, Hogarth_2021, Loiacono_2019} or non-axisymmetric structures \citep[e.g., bar or interaction features;][]{Krajnovic_2011}. By visually inspecting the PVDs of the data and models, the channel maps and comparing the morphological and kinematic position angles, we identify three classes of galaxies:
\begin{itemize}
    \item \textbf{Disks} These are galaxies with PVDs typical of rotating disks, no features indicating the presence of disturbances and alignment between the kinematic and position angles, with $PA_{\mathrm{kin}} - PA_{\mathrm{HST/ALMA}} \lesssim 30 \deg$. We identified 5 ALPAKA galaxies that are regular rotating disks (ID1, 13, 18, 23, 24), see Sect. \ref{sec:details} for details). 
    \item \textbf{Disturbed disks} To this class belong galaxies for which 
    the disk models reproduce the bulk of the emission of the galaxies except for a few features indicating the presence of kinematic anomalies. These galaxies have major-axis PVD with 
    the S-shape profile but they have either asymmetries along the minor axis (e.g., ID3, 8, 19) or emission at high velocities in the inner regions (e.g., ID 28). The former can be ascribed to disturbances driven by environmental effects (e.g., ram pressure stripping) or minor mergers; the latter are likely due to emission from outflows or inflows. A large fraction of ALPAKA galaxies, 13 out of the 28 are disturbed disks.
    \item \textbf{Mergers} These are galaxies for which the rotating disk model does not reproduce the PVD and strong systematic residuals are present in the channel maps. In some cases the nuclei of the two interacting systems can be easily detected in the major-axis PVD. We note that 8 out of the 10 ALPAKA galaxies that we identify as mergers have a misalignment between the morphological and kinematic position angles larger than 40 $\deg$, supporting our classification as non-dynamically relaxed and virialized systems. 
\end{itemize}
The kinematic class for each ALPAKA galaxy is listed in Table~\ref{tab:PA}. In Fig.~\ref{fig:pv1} and Figs.~\ref{fig:pv2} - \ref{fig:pv4}, the width of the boundaries of the panels with the PVDs is thin, medium and thick, depending on whether the galaxy is a disk, a disturbed disk or a merger.

\subsubsection{PVsplit analysis}
Most of the kinematic classification methods rely on the analysis of the moment maps. However, recent studies show that at the typical resolution and SNR of current observations, the success rate of these methods can be as low as 10\%. On the other hand, \citet{Rizzo_2022} present a new classification method -- "PVsplit''-- that relies on the morphological and symmetric properties of the major-axis PVDs, quantitatively defined by three parameters: $P_{\mathrm{major}}$, $P_{\mathrm{V}}$, and $P_{\mathrm{R}}$. The parameter $P_{\mathrm{major}}$ quantifies the symmetry of the PVD with respect to the systemic velocity: a rotating disk with an S-shape profile should have a completely symmetric PVD and a values of $P_{\mathrm{major}} = 0$. The parameters $P_{\mathrm{V}}$ and $P_{\mathrm{R}}$ define the position of the centroid of the brightest regions in the major-axis PVDs with respect to the line-of-sight velocity and center position, respectively. The PVsplit method has been tested using both low-$z$ systems and ALMA mock data of simulated galaxies \citep{Pallottini_2022}, known to be disks, disturbed disks, and mergers. \citet{Rizzo_2022} find that disks and mergers occupy different locations in the 3D space defined by the PVsplit parameters (gray circles and squares in Fig.~\ref{fig:pvsplit}) and \citet{Roman_2023} define a plane that divides these two kinematic classes (red plane in Fig.~\ref{fig:pvsplit}). The disturbed disks analyzed in \citet{Rizzo_2022} are located both in the disk and merger PVsplit sections. \\
Despite the high quality of the ALPAKA data allowing for the accurate derivation of kinematic properties and identification of merger features, we applied the PVsplit method to get a further quantitative confirmation that the three classes described in Sect.\ref{sec:visclass} are reliable. In Fig.~\ref{fig:pvsplit}, we show the location of the galaxies that we classified as disks (green circles), disturbed disks (blue diamonds) and mergers (yellow squares), according to the visual inspection of the spectral channels, PVDs, and residuals. With the exception of ID18 (see Sect.~\ref{sec:details}), all disks and mergers lies in the corresponding regions of the PVsplit diagram. ALPAKA galaxies classified as disturbed disks (blue diamonds in Fig.~\ref{fig:pvsplit}) lie in both the disk and merger sections, similarly to the simulated perturbed disks analyzed in \citet{Rizzo_2022}. %Most (11/13) of the ALPAKA galaxies that we classified as disturbed disks fall in the disk section, with the exception of ID3 and 12. ID3 has better data quality (high SNR and good angular resolution, see Fig.~\ref{fig:pv1}) than the others, allowing for mapping its perturbations and disturbances at high significance. For ID12, the axisymmetric tail along its major axis results in a large $P_{\mathrm{major}}$ parameter.

%Mergers with PVsplit: 4, 10, 15, 16, 18, 21 22, 26, 27, 28.\\
%Disks with a difference in the HST - kinematic position angles $\gtrsim 20$ are ID12, 13, 17, 19. ALMA - kin $\gtrsim 21$ are ID6, 7, 12, 17. ID6, 7, 12 and 17 are face-on, ID19 is clumpy. Check also the rest-frame HST\\Mergers with HST - kin $\gtrsim 21$: ID5, 22. ALMA - kin $\gtrsim 21$ are ID5, 22, 26, 27.diff larger than 30 for [12 13 19.] [ 5 22 26 27]

\begin{figure*}[h!] 
    \begin{center}
        \includegraphics[width=\textwidth]{figures/pvsplit_new.pdf}
        \caption{Distribution of the ALPAKA galaxies in the PVsplit parameter space. The gray circles and squares show simulated disks and mergers, respectively from \citet{Rizzo_2022}. The red plane divides the regions occupied by the two kinematic classes. The colored markers show ALPAKA galaxies classified as disks, disturbed disks (left panel) and mergers (right panel) based on the visual inspection of their PVDs and channel maps. Two different projections are shown for better visualizing the position of all ALPAKA gaalxies.} 
        \label{fig:pvsplit}
    \end{center}
\end{figure*}
\section{ALPAKA in detail} \label{sec:details}
In this Section, we summarize the main physical properties of each ALPAKA target based on previous results from the literature, along with a description of the kinematic fitting and properties. 

\subsection{ID1}
ID1 is part of the IMAGES survey that investigated the [OII]3726, 3729 $\AA$ kinematics using the FLAMES-GIRAFFE multi-object IFU \citep{Yang_2008}. Based on their kinematic analysis, \citep{Yang_2008} classified ID1 as a regular disk. \\
The HST data for this galaxy clearly show that ID1 is a spiral galaxy (Fig.~\ref{sec:galfit}). The large residuals in the HST fitting are due, indeed, to the presence of spiral arms (Fig.~\ref{sec:galfit}). The channel maps and PVDs from the CO(2-1) ALMA data have the typical features of a rotating disk (e.g., see the S-shape profile in Fig.~\ref{fig:pv1}). With our analysis, we find a rotation velociyu for ID1 of 200 km\,s$^{-1}$, consistent within the $1\sigma$ uncertainties with the value of $230\pm 33$ km\,s$^{-1}$ found by the IMAGES collaboration \citep{Puech_2008}. The low velocity dispersions of 10 km\,s$^{-1}$ are consistent with $\sigma$ values of local spiral galaxies \citep{Bacchini_2020}.
%no kinematics

\subsection{ID2}
This galaxy is part of the PHIBBS2 survey, with ID: XV53. Based on the morphology derived by visually inspecting the HST \textit{I}-band (HST/ACS F814W) images, ID2 is classified as disk dominated with asymmetric features \citep{Freundlich_2019}. The HST image shows, in fact, a smooth disk and a clump in the south east direction (Fig.~\ref{fig:galfit1}). \citet{Lackner_2014} interpreted the presence of these two clumps as due to two interacting systems and classify this galaxy as a late-stage minor merger. Here, we model the HST image using two off-center Sérsic components. The CO(3-2) emission is compact and does not extend to the south-east peak. However, the iso-velocity contours on the receding side of the velocity fields are more distorted than the approaching side. This is also visible in the PVDs: along the major-axis, there are two bright peaks at positive velocities, one symmetric with respect to the negative side at $\sim 250$ km\,s$^{-1}$ and the other located at $\sim$ 50 km\,s$^{-1}$, likely corresponding to the interacting companion (black arrow in Fig.~\ref{fig:pv1}); the minor axis PVD is asymmetric. Therefore, we classify ID2 as a disturbed disk. Considering the disturbances on one side of ID2, we fitted  $V_{\mathrm{rot}}$ and $\sigma$ only using the approaching side of the galaxy, that is less disturbed by the interaction and for which the assumption of non-circular motions are, therefore, straightforward. To perform such a fitting, we fixed the $PA_{\mathrm{kin}}$ and the systemic velocity to the values obtained by using both sides of the galaxy. 
%no kinematics

\subsection{ID3}
This galaxy, also know as PACS819 is a Hershel detected galaxy \citep{Rodighiero_2011} whose global properties are extensively studied in the far-infrared wavelength range \citep{Silverman_2015, Silverman_2018, Chang_2020}. In addition, by examining the BPT diagram \citep{Baldwin_1981}, \citep{Silverman_2015} show that PACS-819 is close to the line separating star-forming and AGN galaxies due to the strong [NII] emission line \citep{Kewley_2013}. In the rest-frame UV HST image, ID3 shows two bright clumps, while the center of the CO(5-4) emission is located between them. Due to the different morphology between the CO and UV data, for ID3 we fixed the inclination angle to the one obtained with \texttt{CANNUBI}. 

The iso-velocity contours and the PVDs show that ID3 is kinematically lopsided, meaning that the velocity gradient in its approaching and receding sides are different from each other \citep{Swaters_1999, Bacchini_2023}. We therefore fit the approaching and the receding sides separately (note that in Fig.~\ref{fig:pv1} we show the receding-side model). The rotating disk models reproduce the bulk of the emission from ID3. However, the disturbances of the two external contours at 2.5 and 5$\sigma$ along the minor and major axis PVDs (see arrows in Fig.~\ref{fig:pv1}), as well as the residuals in the channel maps (Fig.~\ref{fig:channel1}), indicate the presence of non-circular motions, in particular at negative velocities. In Fig.~\ref{fig:pv1}, we show with a red arrow the gas that is moving at lower velocities than those predicted by the model. Such kinematic features are usually attributed to radial motions driven by a bar. The black arrow shows, instead, gas that is moving faster than predicted by the model, in particular in the inner regions, and it is likely due to outflows. The minor-axis PVD is not well reproduced by the model: the contours of the data and the model do not overlap and the shape of the model appears more rounded than the data. This is because \bba\ tries to fit the features at anomalous velocities by increasing the velocity dispersion values. Therefore, for ID3, we consider the $\sigma$ values recovered by \bba\ as upper limits.  
%no kinematics

\subsection{ID4 - 9}
These galaxies are part of the XMMXCS J2215.9-1738 cluster detected in the XMM Cluster Survey \citep{Romer_2001}. Global and spatially resolved properties are studied using multi-wavelength observations \citep{Hayashi_2017, Hayashi_2018, Ikeda_2022}. Based on our kinematic analysis, we classify ID4 and 5 as mergers and ID6 - 9 as disturbed disks. ID 4 and 5 have, in fact, strong asymmetric features in the PVDs and ID4 has distorted iso-velocity contours. In the major-axis PVD, ID4 exhibits bright emission only at negative velocities (e.g., see the comparison between the first inner contours of the data and the model). In ID5, the brightest regions of the major axis PVD are located at 0 and 100 km\,s$^{-1}$, respectively; they may be the nuclei of two unresolved interacting galaxies (see black arrows in Fig.~\ref{fig:pv2}). The minor axis PVD of ID5 is strongly asymmetric and not reproduced by the rotating disk model. The HST data of ID4 show a clump in the north-west direction, while no clumpy structures are identified in ID5.\\ 
Galaxies 6 - 9 have, instead, the typical features of rotating disks. However, they show some asymmetries, especially along the minor-axis PVD (see arrows in Fig.~\ref{fig:pv2}), and distorted iso-velocity contours in the velocity fields that may be driven by environmental effects \citep[e.g., ram pressure stripping][]{Lee_2017, Zabel_2019, Cramer_2020, Bacchini_2023} or gravitational interactions \citep{Boselli_2022}. Due to the quality of the current data, discriminating between these two mechanisms is challenging. For instance, at low-$z$, a clear indication of ram-pressure stripping is a distortion of the gas distribution with respect to the stellar component \citep{Lee_2017, Boselli_2022}. On the contrary, it is expected that gravitational interactions have an impact on both the gas and stellar components. For ID6 - 9, the moment-0 maps appears smooth in all cases, likely because of the resolution of the observations, and aligned with the stellar distribution (Fig.~\ref{fig:hstalma1}). The HST data of ID7 and 8 show, instead, close-by satellites (ID7 and 8) and companions (ID8) that are not visible in the ALMA data (Fig.~\ref{fig:hstalma1}). Interestingly, for ID7, the asymmetric features visible in the velocity field and in the minor-axis PVD are in the same direction of the satellite, indicating that the disturbances of the ID7 disk are due to gravitational interactions. Similarly, the strong asymmetries at positive velocities in the minor-axis PVD of ID8 are aligned to the south-east direction of the HST companion. Both for ID7 and 8, \bba\ tries to reproduce these kinematic perturbations by inflating the velocity dispersion (see the comparison between the models and the data in the minor-axis PVDs). For this reason, in Sect.~\ref{sec:discussion}, we consider the $\sigma$ values derived by \bba\ for these two galaxies as upper limits. %Therefore, we classified ID 7 - 9 as disturbed disk.
%no kinematics

\subsection{ID10}
This galaxy is part of the SHiZELS survey \citep{Swinbank_2012, Molina_2017, Gillman} that map the H$\mathrm{\alpha}$ emission line with VLT/SINFONI. The ALMA observations of the CO(2-1) line used here were previously analyzed by \citet{Molina_2019}, who classify ID10 as a disturbed dispersion-dominated disk. Here, we confirm the presence of a smooth gradient in the velocity field, but the strong disturbances and asymmetries in both PVDs and the presence of bright emission only on one side of the major-axis (see arrow in Fig.~\ref{fig:pv2}) lead us to classify ID10 as a merging system. 

\subsection{ID11 - 12}
ID11 and 12 are members of two galaxy clusters that are recently studied in \citet{Cramer_2022} (see also \citet{Noble_2019} for ID11). Both galaxies have the typical features of a rotating disk (e.g., S-shape major-axis PVD) but they show asymmetries in the minor-axis PVDs -- despite at low significance -- and residuals in the channel maps, likely due to tidal disturbances or ram-pressure stripping. %ID11 was classified as a rotating disk by \citet{Noble_2019} and \citet{Cramer_2022}. On the contrary, these authors classified ID12

\subsection{ID13}
This galaxy is in a protocluster. Its global properties (stellar masses, SFR, dust content) are studied in numerous papers \citep{Casey_2016, Hung_2016, Zavala_2019}. In the HST cutout, two additional sources are visible, one located in the north east and the other in the south east directions. However, no dust or CO emissions are detected in correspondence of these two galaxies. The PVDs of ID13 show the typical features of a rotating disk. %The latter report a redshift for this galaxy of 2.1021, as found by using H$\alpha$ emission line from MOSDEF redshift catalogue \citep{Kriek_2015}. However, by using the CO(7-6) and CI(2-1), we find here a redshift of 2.1033. A potential reason for the difference between these two redshifts can be due to the close proximity ($\approx 1$ arcsec) with another source, not identified in the dust continuum and emission lines but visible in the HST data (extended source in the south west). These two galaxies can not be considered as interacting as they are .. km/s apart.\\
%no kinematics

 \subsection{ID14}
ID14, also known as BX610, has been extensively studied both at short \citep{Forster_2009, Forster_2014, Tacchella_2018} and long wavelength \citep{Aravena_2014, Bolatto_2015, Brisbin_2019}. Its kinematics was previously analyzed using H$\alpha$ data from VLT/SINFONI survey \citep{Forster_2018}. Based on H$\alpha$, ID14 is described as one of the best examples of a rotating disk containing massive star-forming clumps \citep{Forster_2011, Forster_2018}. It also exhibits signatures of gas outflows driven by a weak or obscured AGN in the center and by star formation at the location of the bright southern clump visible both in H$\alpha$ and UV \citep{Forster_2014}. The CO(4-3) data analyzed here clearly show the presence of non-circular motions: the major axis PVD has the brightest emission only in the central regions; the minor-axis PVD is strongly axisymmetric, the iso-velocity contours are distorted in the south-east direction where a peak in the CO(4-3) distribution is visible. The presence of non-circular motions, likely driven by outflows or mergers, supports the indication of shock excitation in BX610 as found by measurements of the CO(7-6)/L$_{\mathrm{IR}}$ ratio \citep{Brisbin_2019} and the location in the BPT diagram \citep{Forster_2014}.
%The integrated and nuclear ratios for Q2343- BX610 lie in the so-called “composite region” encompassing the range between the boundaries of local normal SFGs and AGN-dominated systems
Further, we note that ID14 is the galaxy with a large difference of 65 $\deg$ between $PA_{\mathrm{HST}}$ and $PA_{\mathrm{kin}}$. This misalignment is likely due to the presence of outflow motions.
%no kinematics

\subsection{ID15}
ID15 is known as the prototypical example of a compact star-forming galaxy that is rapidly consuming its gas reservoir and is expected to evolve into a quiescent galaxy \citep{Popping_2017}. The kinematics of the CO(6-5) emission line is studied in \citet{Talia_2018} that identify the presence of a rotating disk with a velocity $V = 320^{+92}_{-53}$ km\,s$^{-1}$. Further, \citet{Loiacono_2019} analyzed the H$\alpha$ and [OIII] kinematics, finding evidence of rotation motions, with $V \sim 200 ^{+17}_{-20} $ km\,s$^{-1}$ for an inclination of 75 $\deg$. The CO(3-2) data employed here clearly show the presence of a compact rotating disk. However, the distorted iso-velocity contours (Fig.~\ref{fig:pv3}) and the residuals in the channel maps indicate the presence of non-circular motions, likely driven by outflows. \textcolor{red}{I will expand the discussion to the comparison of the inclination angles and V} %see also Wisnioski 2018
%no kinematics
\subsection{ID16 - 17}
ID16 and 17 are two galaxies, separated by 3" and connected by a bridge of gas and dust \citep{Fu_2013}. These sources are mildly magnified by two foreground galaxies, with magnification factors of 1.5. Due to their close proximity, ID15 and 16 were first identified as a unique bright SMG, HXMM01, in the \textit{Hershel} Multi-tiered Extragalctic survey \citep{Oliver_2012}. Subsequently, higher-angular resolution observations resolve HXMM01 into three sources, ID16, 17 and its companion, visible in the moment-0 map in Fig.~\ref{fig:hstalma1} \citep{Fu_2013}. The bright clump visible in the HST image at the southern end of ID17 has an SED that is consistent with either a less obscured galaxy at $z = 2.3$ or a physically unrelated contaminating source \citep{Fu_2013}. The ALMA cube employed here contain two emission lines, CO(7-6) and CI(2-1) that slightly overlap along the frequency axis. A kinematic analysis of this complex system was already presented in \citet{Xue_2018} who interpreted ID16 and 17 as two merging rotating disks. Despite ID16 showing some symmetric features along the minor axis PVD (Fig.~\ref{fig:pv3}), the major axis PVD is strongly asymmetric, being bright only at negative velocities. Similarly, ID17 is strongly asymmetric as it is interacting with a companion, clearly visible in the moment-0 map (Fig.~\ref{fig:hstalma1}) and located at 1" in the north direction with respect to the ID17 center. Due to the presence of strong disturbances, we classify both ID16 and 17 as merging systems. 

\subsection{ID18} 
ID18 lies in a well-studied protocluster. This structure was originally selected with \textit{Hershel} and later found to host an exceptionally high SFR $\sim 7000$ M$_{\odot}$\,yr$^{-1}$, spread over 5 galaxies (W, T, C, M, E), confined within a scale of $\sim$100 kpc \citep{Ivison_2013, Ivison_2019}. The ALMA data set used here map galaxy W and T at high resolution and SNR, while the data quality is not sufficient for performing a kinematic analysis on galaxies C, M, and E. Further, galaxy T is not included in the ALPAKA sample as it is a strongly gravitationally lensed source.\\
The CO(4-3) data analyzed in this paper indicates that ID18 is kinematically lopsided as evident from the different shapes of the emission in the major-axis PVD at positive and negative velocities (Fig.~\ref{fig:pv3}). Because of these asymmetries, its PVsplit parameter $P_\mathrm{major}$ is higher than for typical rotating disks (see Fig.~\ref{fig:pvsplit}) and ID18 falls, therefore, into the PVsplit parameter space of the merging systems. However, considering the S-shape of the major-axis PVD and the symmetric minor-axis PVD, we classify ID18 as a rotating disk.

\subsection{ID19 - 21} 
These galaxies are members of the forming cluster CLJ1001 at $z = 2.51$ \citep{Wang_2016, Gomez_2019, Champagne_2021}. The CO(3-2) kinematics of ID19 - 21, along with another galaxy member of the same cluster (130901), are recently studied by \citet{Xiao_2022} using the same data employed here. We exclude 130901 from our analysis due to the low SNR. \\
For the kinematic fitting, \citet{Xiao_2022} used \texttt{GALPAK$^{\mathrm{3D}}$} \citep{Bouche_2015} and they fit simultaneously the kinematic and geometric parameters, finding values of the inclinations of 24, 45 and 19 $\deg$ and position angles of 12, 148 and 10 $\deg$. For fitting the kinematics with \bba, we used, instead, inclination angles from HST data modeling of 60, 45 and 30 $\deg$. The kinematic position angles (32$\pm$15, 133$\pm$8, 13$\pm$5 $\deg$) are consistent with the values reported by \citet{Xiao_2022}. The value of the velocity dispersion for ID19 is consistent with the one found by \citet{Xiao_2022}, while we find a value of rotation velocity which is a factor of 1.5 smaller than the value found by \citet{Xiao_2022}. This discrepancy is due to the different inclination angles used to recover the intrinsic rotation velocity. On the contrary, for ID20 we find rotation velocities consistent with those found in \citet{Xiao_2022} and velocity dispersions which are a factor of 2 smaller than the ones in \citet{Xiao_2022}.

ID19 and 20 have disturbed PVDs, likely driven by environmental effects or interactions with companions (see discussion for ID4 - 9). In particular, the major-axis PVD of ID19 is asymmetric and the external contour of the minor-axis PVD has tidal features that may indicate the presence of ram-pressure stripping (see arrow in Fig.~\ref{fig:pv3}). However, the low-significance of this contour does not allow us to draw a conclusion on the mechanisms causing the perturbation of ID19 disk. Similarly to ID7, also in this case the \bba\ model is overestimating the velocity dispersion, which we, therefore, consider as an upper limit. 
Despite the comparison between the \bba\ model and the data indicates that the bulk of the motions in ID20 are reproduced by a rotating disk, the relatively low SNR of ID20 does not allow us to robustly check the presence of any perturbations in the minor-axis PVD. 
Further, ID19 and ID20 have close-by companions and clumps that are clearly visible in the HST data (see Figs.~\ref{fig:hstalma1} and \ref{fig:galfit2}). Based on this, and the kinematic anomalies described above we classify both galaxies as disturbed disks. We classify, instead, ID21 as a merger due to the lack of bright features typical of rotating disks in the major-axis PVD, the asymmetries in the minor-axis PVD and the strong systematic residuals in the channel maps. \citet{Xiao_2022} interpreted, instead, the gradient in the velocity map of ID21, also visible in Fig.~\ref{fig:pv3}, as due to the presence of a rotating disk. 

\subsection{ID22, 26 - 27}
These galaxies are recently studied at long wavelength using ALMA observations at 0.6" \citep{Cassata_2020}. The 0.17" observations of the CO(5-4) used here allow us to resolve well their kinematic structures. In all cases, a rotating disk is not a good description of the data: the PVDs are strongly asymmetric and systematic residuals appear in all channel maps. These are among the galaxies with the largest difference between $PA_{\mathrm{ALMA}}$ and $PA_{\mathrm{kin}}$, another indication that they are non-resolved merging systems.

\subsection{ID23 - 25} 
These galaxies are members of the SSA 22 protocluster \citep{Steidel_1998} that was extensively studied with ALMA \citep{Umehata_2015, Umehata_2017, Umehata_2018}. \citet{Lehmer_2009} identified X-ray luminous AGNs in ID 23 and 25 using observations from the Chandra Space telescope. Extended Lyman-$\alpha$ emission from multiple filaments between galaxies within SSA 22 are identified using MUSE observations \citep{Umehata_2019} and are thought to be responsible for the accretion of gas within protocluster and the growth of galaxies and their supermassive black holes. 

The high-angular resolution observations employed here allow us to clearly see the typical features of rotating disks in ID 23 and 24. ID23, being almost edge-on, has a very extended S-shape major-axis PVD (Fig.~\ref{fig:pv4}). ID24 has, instead, a major-axis PVD with an X-shape pattern, typical of a bar seen at high inclination along the line of sight \citep[e.g.,][]{Alatalo_2013, Hogarth_2021}. The gas emission at $-400$ km s$^{-1}$ along the minor-axis PVD (see arrow in Fig.~\ref{fig:pv4}), despite at low significance, may be due to non-circular motions driven by the bar. ID25 is, instead, likely a merger due to the strong asymmetries of both the smooth distribution and the brightest regions (at negative and positive velocities, see arrows in Fig.~\ref{fig:pv4}) along the major-axis PVD. 

\subsection{ID28}
This is an hyper-luminous dust-obscured AGN that was identified by WISE and extensively studied using ALMA observations \citep[e.g.,][]{Fan_2018, Diaz_2021, Ginolfi_2022}. Recently, \citet{Ginolfi_2022} report the presence of 24 Lyman-$\alpha$ emitting galaxies on projected physical scales of 400 kpc around ID28. 

The kinematics of ID28 is peculiar as it shows the typical features of a rotating disk (e.g., S-shape along the major-axis PV, diamond shape along the minor axis PV) but also strong emission in the inner 1 kpc regions (see arrows in the PVDs, Fig.~\ref{fig:pv4}) that indicates the presence of gas moving at a line-of-sight velocity of 900 km s$^{-1}$ relative to the systemic velocity. This emission, not reproduced by the symmetric rotating disk model, can be explained in two ways. The first possibility is that the CO distribution is asymmetric between the approaching and the receding sides and there is an inner rise of the rotation curve caused by the presence of a compact bulge. The second possibility is that this emission is due to non-circular motions driven by outflows. An additional indication in favor of the presence of outflows is the misalignment of 48 $\deg$ between the two morphological position angles (i.e., from HST and ALMA) and the kinematic one. As discussed in Sect.~\ref{sec:kinclassification}, such strong misalignment typically indicates the presence of asymmetric structures, radial or vertical motions. The $V$ and $\sigma$ values for these galaxy should thus be taken with caution as the emission from the disk may be strongly contaminated by the one from the outflow. We note that strong outflow motions were already identified in ID28 from the analysis of rest-frame UV spectrum \citep{Ginolfi_2022}. By analyzing the same CO(6-5) data presented in this paper, \citet{Ginolfi_2022} find kinematic profiles consistent with those found here (see Sect.~\ref{sec:discussion} for further details).  

\section{Discussion} \label{sec:discussion}
\subsection{Kinematics of the disk subsamples} \label{sec:disks}
In this subsection, we discuss the kinematic properties of the two disk subsamples (i.e., disk and disturbed disk) consisting of 18 ALPAKA galaxies in total. The rotation velocities and velocity dispersions of mergers are, instead, unreliable as they are derived under the assumption of galaxies being virialized systems. As such, we decide not show nor discuss them. In Fig.\ref{fig:vprofile}, we show the profiles of the rotation velocity and velocity dispersion as derived by \bba\ (Sect.\ref{sec:barolo}). The visual inspection of the rotation velocity profiles show a variety of shapes: flat (e.g., ID12, 19), slowly increasing and then flattening (e.g., ID1, 2), declining in the inner regions and then flattening (e.g., ID13, 28). We caution the reader that the velocity profiles in Fig.\ref{fig:vprofile} show the rotation and not the circular velocity. The latter is a direct proxy of the gravitational potential and can be derived from $V_{\mathrm{rot}}$ after applying the asymmetric drift correction which is a function of the velocity dispersion. Such estimates as well as the derivation of the gravitational potential will be part of a future paper of the ALPAKA series. The velocity dispersion profiles have $\sigma$ values in the inner regions that are typically higher (up to a factor of 2) than the ones measured at the outermost radius. This decrease of $\sigma$ with radii is also observed in the CO profiles of local disk galaxies \citep{Bacchini_2020}, and it is usually ascribed to the radial change of the mechanisms sustaining the turbulence within galaxies (e.g., supernova feedback; the detailed study of such mechanisms is deferred to the second paper of the ALPAKA series). The analysis of the ALPAKA galaxies shows, therefore, that caution must be taken when fitting the velocity dispersion profile assuming a constant value across the disks \citep[e.g.,][]{Ubler_2018, Forster_2018}. This assumption may result in the estimation of velocity dispersion $\sigma$ to be mainly influenced by the emission from the bright regions at the center of a galaxy, in turn resulting in inflated $\sigma$ measurements compared to the actual values.

\begin{figure*}[th!]
    \begin{center}
        \includegraphics[width=0.95\textwidth]{figures/vrot.pdf}
        \caption{Rotation velocity and velocity dispersion profiles for the ALPAKA galaxies classified as disks or disturbed disks.} 
        \label{fig:vprofile}
    \end{center}
\end{figure*}

To facilitate the comparison with previous work, in Tab.~\ref{tab:vsigma}, we refactor our result in terms of the kinematic parameters mostly common in the literature: the maximum values of the rotation velocity $V_{\mathrm{max}}$; the radial average values of the velocity dispersion $\sigma_{\mathrm{m}}$; the values of velocities and dispersions computed by averaging the two outermost values in their profiles, $V_{\mathrm{ext}}$ and $\sigma_{\mathrm{ext}}$; the ratios $V_{\mathrm{max}}/\sigma_{\mathrm{m}}$ and $V_{\mathrm{ext}}/\sigma_{\mathrm{ext}}$. The latter are used to define the rotational support of galaxies: $V/\sigma \lesssim 2$ indicate that the galaxy is dispersion dominated while rotation-dominated galaxies have $V/\sigma \gtrsim 2$ \citep{Forster_2018, Wisnioski_2019}. In Fig.\ref{fig:vsigma}, we show $\sigma_{\mathrm{m}}$ and $V_{\mathrm{max}}/\sigma_{\mathrm{m}}$ as a function of redshift for the ALPAKA disks (blue diamonds) and disturbed disks (green circles). In this figure, galaxies hosting an AGN are indicated with a black dots. Considering the limited number of galaxies in the disk subsample (5/28; see Sect.\ref{sec:selectioneffect} for details), a comparison between the kinematic properties of disks and disturbed disks is not straightforward. Further, as discussed in Sect.\ref{sec:kinclassification}, the values of the velocity dispersions for the disturbed disks can be considered upper limits in some cases. Despite this, and considering the bias of the ALPAKA sample towards galaxies hosting very energetic mechanisms (e.g., stellar and AGN feedback) that are expected to either boost the velocity dispersion values or completely destroy the disk structures, we finnd that $\sigma_{\mathrm{m}}$ range between 10 and 48 km\,s$^{-1}$ for all galaxies, but ID 15 and 28, two AGN hosts with values of $\sigma_{\mathrm{m}}$ of 110 and 163 km\,s$^{-1}$, respectively. ID15 and 28 along with ID14, are the only three ALPAKA galaxies with $V_{\mathrm{max}}/\sigma_{\mathrm{m}} \sim 3$. For the rest of the sample, the $V_{\mathrm{max}}/\sigma_{\mathrm{m}}$ values range between 6 and 19, with a median value of 10$^{+7}_{-5}$. The latter is obtained after excluding the lower limits for the 4 galaxies for which the \bba\ velocity dispersions are overestimated (see Sect.~\ref{sec:details}). In the second paper of this series, we will compare the distribution of $\sigma$ and $V/\sigma$ of the ALPAKA sample with redshift-matched samples of galaxies with warm gas kinematics, and we will study the evolution of the cold gas kinematics across cosmic time.

\begin{figure}[th!]
    \begin{center}
        \includegraphics[width=0.95\columnwidth]{figures/vsigma_z.pdf}
        \caption{Distribution of the ALPAKA disks and disturbed disks in the velocity dispersion - redshift plane (upper panel) and rotation-to-velocity dispersion ratio - redshift plane (bottom panel). The values of $\sigma_{\mathrm{m}}$ and $V_{\mathrm{max}}/\sigma_{\mathrm{m}}$ (Table~\ref{tab:vsigma}) are plotted here. The redshifts of ID3, ID6 - 9, and ID19 are  shifted by $|\Delta z| \gtrsim 0.25$ for a better visualization of all the data points.}       
    \end{center}
    \label{fig:vsigma}
\end{figure}

\begin{table}[h!]
\begin{center}
\begin{tabular}{ccccccc}
\\
\hline
\\
ID   & $V_{\mathrm{max}}$ & $\sigma_{\mathrm{m}}$ & $V_{\mathrm{ext}}$ & $\sigma_{\mathrm{ext}}$ & $V_{\mathrm{max}}/\sigma_{\mathrm{m}}$ & $V_{\mathrm{ext}}/\sigma_{\mathrm{ext}}$\\
 & km\,s$^{-1}$ &  km\,s$^{-1}$ & km\,s$^{-1}$ & km\,s$^{-1}$\\
\\
\hline
\\
1 & $ 204 \substack{+ 9 \\ -10 }$ &$ 10 \substack{+ 4 \\- 4 }$ &$ 202 \substack{+ 5 \\- 6 }$ &$ 9 \substack{+ 5 \\- 5 }$ &$ 19 \substack{+ 14 \\- 6 }$ &$ 21 \substack{+ 24 \\- 8 }$ \\
2 & $ 259 \substack{+ 11 \\ -12 }$ &$ 15 \substack{+ 4 \\- 4 }$ &$ 257 \substack{+ 7 \\- 7 }$ &$ 13 \substack{+ 6 \\- 6 }$ &$ 17 \substack{+ 7 \\- 4 }$ &$ 19 \substack{+ 15 \\- 6 }$ \\
%3 & $ 366 \substack{+ 34 \\ -29 }$ &$ 49 \substack{+ 6 \\- 5 }$ &$ 346 \substack{+ 22 \\- 18 }$ &$ 58 \substack{+ 8 \\- 7 }$ &$ 7 \substack{+ 1 \\- 1 }$ &$ 6 \substack{+ 1 \\- 1 }$ \\
3 & $366 \substack{+ 34 \\ -29 }$ & $\lesssim 49$ & $346 \substack{+ 22 \\- 18 }$ & $\lesssim 58$ & $\gtrsim 7$ & $\gtrsim 6$ \\
6 & $273 \substack{+ 23 \\ -25 }$ & $42 \substack{+ 8 \\- 7 }$ &$ 266 \substack{+ 15 \\- 17 }$ &$ 42 \substack{+ 8 \\- 7 }$ &$ 6 \substack{+ 1 \\- 1 }$ &$ 6 \substack{+ 1 \\- 1 }$ \\
%7 & $ 300 \substack{+ 10 \\ -12 }$ &$ 52 \substack{+ 6 \\- 6 }$ &$ 274 \substack{+ 24 \\- 26 }$ &$ 52 \substack{+ 9 \\- 9 }$ &$ 6 \substack{+ 1 \\- 1 }$ &$ 5 \substack{+ 1 \\- 1 }$ \\
7 & $ 300 \substack{+ 10 \\ -12 }$ &$ \lesssim 52$ &$ 274 \substack{+ 24 \\- 26 }$ &$ \lesssim 52$ & $ \gtrsim 6$ & $ \gtrsim 5$ \\
8 & $ 401 \substack{+ 42 \\ -53 }$ &$ \lesssim 65$ &$ 369 \substack{+ 30 \\- 35 }$ &$ \lesssim 60$ &$ \gtrsim 6$ &$ \gtrsim 6 $ \\
9 & $ 343 \substack{+ 35 \\ -44 }$ &$ 30 \substack{+ 10 \\- 12 }$ &$ 342 \substack{+ 22 \\- 27 }$ &$ 30 \substack{+ 10 \\- 12 }$ &$ 12 \substack{+ 8 \\- 3 }$ &$ 12 \substack{+ 8 \\- 3 }$ \\
11 & $ 421 \substack{+ 21 \\ -22 }$ &$ 30 \substack{+ 7 \\- 5 }$ &$ 395 \substack{+ 24 \\- 24 }$ &$ 32 \substack{+ 9 \\- 7 }$ &$ 14 \substack{+ 3 \\- 2 }$ &$ 12 \substack{+ 4 \\- 2 }$ \\
12 & $ 283 \substack{+ 19 \\ -26 }$ &$ 30 \substack{+ 6 \\- 6 }$ &$ 275 \substack{+ 15 \\- 17 }$ &$ 35 \substack{+ 8 \\- 7 }$ &$ 9 \substack{+ 2 \\- 2 }$ &$ 8 \substack{+ 2 \\- 1 }$ \\
13 & $ 263 \substack{+ 14 \\ -17 }$ &$ 32 \substack{+ 5 \\- 6 }$ &$ 236 \substack{+ 32 \\- 30 }$ &$ 33 \substack{+ 7 \\- 8 }$ &$ 8 \substack{+ 2 \\- 1 }$ &$ 7 \substack{+ 2 \\- 2 }$ \\
14 & $ 165 \substack{+ 19 \\ -22 }$ &$ 48 \substack{+ 4 \\- 4 }$ &$ 153 \substack{+ 11 \\- 12 }$ &$ 46 \substack{+ 6 \\- 7 }$ &$ 3.4 \substack{+ 0.5 \\- 0.5 }$ &$ 3.4 \substack{+ 0.6 \\- 0.5 }$ \\
15 & $ 432 \substack{+ 22 \\ -28 }$ &$ 110 \substack{+ 12 \\- 11 }$ &$ 369 \substack{+ 22 \\- 21 }$ &$ 110 \substack{+ 12 \\- 11 }$ &$ 3.8 \substack{+ 0.5 \\- 0.4 }$ &$ 3.3 \substack{+ 0.4 \\- 0.4 }$ \\
18 & $ 509 \substack{+ 27 \\ -27 }$ &$ 40 \substack{+ 10 \\- 8 }$ &$ 466 \substack{+ 35 \\- 30 }$ &$ 30 \substack{+ 17 \\- 14 }$ &$ 13 \substack{+ 4 \\- 2 }$ &$ 14 \substack{+ 11 \\- 5 }$ \\
%19 & $ 288 \substack{+ 28 \\ -35 }$ &$ 45 \substack{+ 8 \\- 9 }$ &$ 286 \substack{+ 15 \\- 19 }$ &$ 45 \substack{+ 8 \\- 9 }$ &$ 6 \substack{+ 2 \\- 1 }$ &$ 6 \substack{+ 2 \\- 1 }$ \\
19 & $ 288 \substack{+ 28 \\ -35 }$ &$ \lesssim 45 $ &$ 286 \substack{+ 15 \\- 19 }$ &$ \lesssim 45 $ &$ \gtrsim 6 $ &$ \gtrsim 6 $ \\
20 & $ 316 \substack{+ 29 \\ -25 }$ &$ 45 \substack{+ 10 \\- 10 }$ &$ 301 \substack{+ 26 \\- 24 }$ &$ 45 \substack{+ 10 \\- 10 }$ &$ 7 \substack{+ 2 \\- 1 }$ &$ 7 \substack{+ 2 \\- 1 }$ \\
23 & $ 508 \substack{+ 16 \\ -20 }$ &$ 25 \substack{+ 6 \\- 6 }$ &$ 489 \substack{+ 12 \\- 15 }$ &$ 26 \substack{+ 9 \\- 9 }$ &$ 20 \substack{+ 6 \\- 4 }$ &$ 19 \substack{+ 10 \\- 5 }$ \\
24 & $ 362 \substack{+ 34 \\ -45 }$ &$ 36 \substack{+ 6 \\- 6 }$ &$ 303 \substack{+ 16 \\- 18 }$ &$ 39 \substack{+ 9 \\- 9 }$ &$ 10 \substack{+ 2 \\- 2 }$ &$ 8 \substack{+ 2 \\- 2 }$ \\
28 & $ 428 \substack{+ 65 \\ -52 }$ &$ 167 \substack{+ 10 \\- 11 }$ &$ 398 \substack{+ 40 \\- 35 }$ &$ 85 \substack{+ 12 \\- 13 }$ &$ 2.6 \substack{+ 0.4 \\- 0.4 }$ &$ 5 \substack{+ 1 \\- 1 }$ \\
\\
\hline
\end{tabular}
\end{center}
\caption{Global kinematic parameters of the ALPAKA galaxies classified as disks and disturbed disks.
}
\label{tab:vsigma}
\end{table}

\subsection{Selection effects and the impact of energetic mechanisms on the disk properties} \label{sec:selectioneffect}
The selection criteria used to build the ALPAKA sample are not based on the global physical properties of the sources, but on the quality of the available data in the ALMA archive. This, combined with the intrinsic faintness of CO transitions and the limited sensitivities of ALMA, results in a sample that is biased towards massive main-sequence or starburst galaxies, mostly in overdensity regions. Numerous studies showed, indeed, that the gas-to-stellar mass ratio are typically higher in cluster than in field galaxies \citep{Noble_2017, Hayashi_2018, Tadaki_2019}. Further, 7 out of the 28 ALPAKA galaxies host an AGN. According to current models of galaxy formation and evolution, the growth of high-$z$ massive star-forming galaxies is mainly driven by galaxy mergers and intense gas accretion which drive large amounts of gas towards their centers, boosting the SFR and the growth of supermassive black holes.  

Among the 10 mergers in ALPAKA, we do find that all of them are starbursts, confirming that intense bursts of star formation may be triggered by interactions and merging activities. In addition, 4 out of the 6 remaining starbursts are disturbed disks in a post-merger phase or with irregularities driven by counter-rotating gas streams or outflows. The dynamical time scale of these galaxies is $\approx$ 10 Myr \footnote{The dynamical timescales are computed as $R_{\mathrm{ext}}/V_{\mathrm{ext}}$, where $R_{\mathrm{ext}}$ is the outermost radius for which we measured the rotation velocity.}, a factor of 10 smaller than the typical depletion timescales ($\sim$ 100s Myr) of starbursts galaxies at these redshifts \citep{Scoville_2017, Liu_2019}. Despite based on small statistics, the presence of disks among the ALPAKA starbursts suggests that after a merger or an intense accretion event, galaxies have time to transition into a stable dynamical stage and form a disk with $V/\sigma \sim 10$, before consuming their gas reservoirs. Another potential explanation could be that the burst of star-formation in these galaxies was caused by intense gas accretion of co-planar, co-rotating gas via cold cosmic-web streams \citep{Kretschmer_2022}.
Among the subsample of 7 AGN-host galaxies, only one is a merger (ID 25), while the others are disks (2/7) and disturbed disks (4/7). This result indicates that the AGN feedback does not prevent the formation of a rotating disks and it may have also a limited impact on the ISM properties of the host galaxies. With the exception of ID28 and ID15 (see Sect.\ref{sec:disks}), whose kinematic properties may be contaminated by outflows, all AGN hosts have, indeed, velocity dispersions similar to rest of the disk subsamples.  

Most ALPAKA galaxies are in overdensity regions, clusters or groups (13/28) or protoclusters (6/28). In these dense environments, galaxy mergers and tidal interactions could be, efficiently enhanced with respect to the field \citep[e.g.,][]{Merritt_1983, Moore_1996, Kronberger_2006, Cortese_2021}. Further, recent studies showed that environmental effects, like ram-pressure stripping, are responsible for the kinematic asymmetries of the molecular gas distribution and kinematics in cluster galaxies \citep[e.g.,][]{Lee_2017, Cramer_2020, Bacchini_2023}. Among the 13 ALPAKA galaxies in clusters and groups, 5 of them are mergers and the remaining 8 are disturbed disks. On the contrary, among the 5 protocluster galaxies, 3 are disks, 1 is a disk with outflow contamination (ID28) and 1 is a merger. This finding suggests that environmental mechanisms may not have a relevant impact on the ISM of galaxies during the early stages of cluster formation. 

\section{Summary and conclusion} \label{sec:conclusion}
Studying the kinematics of galaxies using cold gas tracers is crucial for gaining insight into the formation and evolution of structures across cosmic time. %In the last decade, such kinematic investigations were performed on hundreds of galaxies at $z \sim 0.5$ - 3.5 mainly using $\gtrsim 0.6$" observations from warm gas tracers. However, there are hints that the kinematic and dynamic properties inferred from emission lines tracing warm gas (T $\approx$ 10$^4$ K) differ from the ones obtained from cold gas tracers (T $\lesssim$ 10$^2$ K, e.g., CO transitions). 
Nevertheless, before the advent of the Next Generation Very Large Array \citep[ngVLA;][]{ngvla} in 2030s, high-resolution observations of CO transitions at $z \gtrsim 0.5$ are feasible only on a few bright targets, due to the long integration times needed for achieving sensitivities sufficient for recovering robust kinematic measurements. For instance, a Milky-Way progenitor with stellar and gas masses of $1 \times 10^{10}~\mathrm{M_{{\odot}}}$ at $z \sim 2.2$ \citep{vandokkum_2013} is expected to have a CO(3-2) luminosity of 10$^9$ K\,km\,s $^{-1}$\,pc$^2$ and a flux of 0.08 Jy km\,s $^{-1}$, that is a factor of $\approx 10$ smaller than the CO(3-2) fluxes of ALPAKA galaxies at similar redshifts (e.g., ID19-21). Assuming the on-source integration time of 3 hours for ID19 - 21, this means that we should observe this Milky-Way progenitor for 300 hours (on-source) for obtaining CO(3-2) data with SNR and angular resolution comparable to the ALPAKA data sets. Considering the current 50 hours limit for normal ALMA observing programs, such observations are, therefore, not feasible. For this reason, to date, there are no systematic studies of CO or [CI] kinematics of $z \gtrsim$ 0.5, but only observational campaigns and studies on single, luminous galaxies. In this context, the ALPAKA project aims at alleviating this limitation by providing systematic kinematic properties of a sample of 28 star-forming galaxies at $z \sim$ 0.5 - 4 (median $z$ of 1.8). Data for the ALPAKA galaxies were obtained by collecting ALMA archive high-data quality observations (median angular resolution of 0.25") from CO and [CI] transitions. In this manuscript, we present the ALPAKA galaxies and derive their global ($M_{\star}$, SFR), morphological and kinematic properties. We found that ALPAKA galaxies have high stellar masses ($M_{\star} \gtrsim 10^{10}$ M$_{\odot}$) and SFR that range from 8 to 3000 M$_{\odot}$ yr$^{-1}$. A large fraction of ALPAKA, 16 out of the 25 galaxies with good estimates of $M_{\star}$ and SFR, are starbursts, while the remaining are main-sequence. Further, 18/28 ALPAKA galaxies lie in overdensity regions. By performing 3D kinematic analysis on the ALPAKA data cubes, we found the following results:
\begin{itemize}
    \item 10/28 are mergers, 5/28 are disks and 13/28 are disturbed disks. The latter have features that indicate the presence of perturbations driven by outflows, minor mergers, and environmental effects.
    \item We derive the rotation velocity and velocity dispersion profiles of the two disk subsamples (i.e., disk and disturbed disk). The samplings of these profiles are quite diverse: for 5/18 galaxies, the kinematic profiles are sampled only by 2 resolution elements, while for the remaing 13 there are up to 4 - 5. In all cases, the rotation curves show a variety of shapes but they are flat in the outermost regions. For almost all ALPAKA galaxies, the velocity dispersion profiles are not constant and they are typically larger in the inner regions and decline up to a factor of 2 at outermost radii, indicating that there is a radial change in the nature or intensity of the mechanisms driving turbulence. The median $\sigma_{\mathrm{m}}$ for the ALPAKA disk subsamples (excluding the upper limits) is 33$_{-7}^{+14}$ km s$^{-1}$. A systematic comparison with kinematic measurements from warm gas at similar redshifts will be discussed in the second paper of the ALPAKA series. The $V/\sigma$ values for ALPAKA galaxies range between 3 and 20. The median $V/\sigma$ for the ALPAKA disk subsamples is 10$_{-5}^{+7}$ and there is no evidence of evolution with redshift.
\end{itemize}
This work is the first of a series that will allow for fully exploiting the ALMA data presented here to characterize the ALPAKA galaxies. In particular, we plan to use the kinematic analysis described in this manuscript to: systematically compare the dynamics from warm and cold gas tracers; investigate the evolution of velocity dispersion with redshift and study the energy sources of turbulence; infer the dark matter halo properties and content within the ALPAKA sample; derive the dynamic scaling relations and study their evolution. 



\newpage





\bibliographystyle{aa}
\bibliography{ms.bib}



\appendix
\section{GALFIT fitting} \label{sec:galfit}
\textcolor{red}{Here I will describe the assumption for the GALFIT fitting. See Figs.~\ref{fig:galfit1} and \ref{fig:galfit2} for the output.}
\begin{figure*} 
    \begin{center}
        \includegraphics[width=0.8\textwidth]{figures/galfit_tot_1.pdf}
        \caption{GALFIT model and residuals. For each target, we show in the upper panel the contours of the HST data (black) and GALFIT model (green). The bottom panel shows the residual normalized to the noise map. The red cross shows the center of the main galaxy and the orange crosses show the center of additional components that were added to the fitting, when necessary. }    
        \label{fig:galfit1}
    \end{center}
\end{figure*}

\begin{figure*} 
    \begin{center}
        \includegraphics[width=0.8\textwidth]{figures/galfit_tot_2.pdf}
        \caption{Same as Fig.~\ref{fig:galfit1}} 
        \label{fig:galfit2}
    \end{center}
\end{figure*}

\section{Pseudo contours} \label{sec:pseudo}

\section{Output of the kinematic fitting} \label{sec:pvfigures}

\begin{figure*}
    \begin{center}
        \includegraphics[width=0.88\textwidth]{figures/pv2.pdf}
        \caption{Same as Fig.~\ref{fig:pv1}, but for ID5 - 12.}  
        \label{fig:pv2}
    \end{center}
\end{figure*}

\begin{figure*}
    \begin{center}
        \includegraphics[width=0.8\textwidth]{figures/pv3.pdf}
        \caption{Same as Fig.~\ref{fig:pv1}, but for ID13 - 20.} 
        \label{fig:pv3}
    \end{center}
\end{figure*}

\begin{figure*}
    \begin{center}
        \includegraphics[width=0.8\textwidth]{figures/pv4.pdf}
        \caption{Same as Fig.~\ref{fig:pv1}, but for ID21 - 28.}  
        \label{fig:pv4} 
    \end{center}
\end{figure*}

\section{Channel maps}\label{sec:channels}
\begin{figure*}
    \begin{center}
        \includegraphics[width=0.8\textwidth]{figures/channel_res.pdf}
        \caption{For each ALPAKA target, we show 7 representative channel maps for the data (black contours) and the rotating disk model (red contours) in the upper panels and the residuals in the bottom panels. The contours of the data and the models are at [1, 2, 4, 8, 16, 32] $\times$ 2.5 RMS. For the residuals, we show the emission over a scale of $\pm 5$ RMS. }  
        \label{fig:channel1} 
    \end{center}
\end{figure*}

\begin{figure*}
    \begin{center}
        \includegraphics[width=\textwidth]{figures/channel_res2.pdf}
        \caption{Same as Fig.~\ref{fig:channel1}, but for ID6 - 10.}  
        \label{fig:channel2} 
    \end{center}
\end{figure*}

\begin{figure*}
    \begin{center}
        \includegraphics[width=\textwidth]{figures/channel_res3.pdf}
        \caption{Same as Fig.~\ref{fig:channel1}, but for ID11 - 15.}  
        \label{fig:channel2} 
    \end{center}
\end{figure*}

\begin{figure*}
    \begin{center}
        \includegraphics[width=\textwidth]{figures/channel_res4.pdf}
        \caption{Same as Fig.~\ref{fig:channel1}, but for ID16 - 20.}  
        \label{fig:channel3} 
    \end{center}
\end{figure*}

\begin{figure*}
    \begin{center}
        \includegraphics[width=\textwidth]{figures/channel_res5.pdf}
        \caption{Same as Fig.~\ref{fig:channel1}, but for ID21 - 25.}  
        \label{fig:channel4} 
    \end{center}
\end{figure*}

\begin{figure*}
    \begin{center}
        \includegraphics[width=\textwidth]{figures/channel_res6.pdf}
        \caption{Same as Fig.~\ref{fig:channel1}, but for ID26 - 28.}  
        \label{fig:channel5} 
    \end{center}
\end{figure*}



\end{document}


%%%% End of aa.dem
