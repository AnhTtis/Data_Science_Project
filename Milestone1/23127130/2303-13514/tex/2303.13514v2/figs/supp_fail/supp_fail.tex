\begin{figure*}
\begin{center}    
    \begin{picture}(\textwidth, 10)
    \put(0.05\textwidth,2){Input}
    \put(0.15\textwidth,2){Pose}
    \put(0.30\textwidth,2){Reconstruction}
    \put(0.50\textwidth,2){Parts}
    \put(0.73\textwidth,2){MagicPony~\cite{wu2022magicpony}}
    \end{picture}
    
    \includegraphics[height=2.3cm, trim=5 0 5 0, clip]{figs/supp_fail/0.png}
    \includegraphics[height=2.3cm, trim=5 0 5 0, clip]{figs/supp_fail/0_mp_rec.png}
    \includegraphics[width=2.3cm, trim=5 0 5 0, clip]{figs/supp_fail/0_mp_rec_text.png}
    \includegraphics[width=2.3cm, trim=30 10 15 10, clip]{figs/supp_fail/0_mp_rec_o.png}
    
    \includegraphics[height=2.3cm, trim=5 0 5 0, clip]{figs/supp_fail/1.png}
    \includegraphics[height=2.3cm, trim=5 0 5 0, clip]{figs/supp_fail/1_mp_rec.png}
    \includegraphics[width=2.3cm, trim=5 0 5 0, clip]{figs/supp_fail/1_mp_rec_tex.png}
    \includegraphics[width=2.3cm, trim=30 10 15 10, clip]{figs/supp_fail/1_mp_rec_o.png}
 
    \vspace{-5pt}
   
\end{center} 
\vspace{-10pt}
\caption{Failure cases on cows. On the left we see SAOR-101 predictions (estimated pose, original viewpoint reconstruction, different view, and estimated parts). 
On the right we display MagicPony~\cite{wu2022magicpony} (original viewpoint reconstruction, textured reconstruction, different view). When the pose is very different than the typical ones present in the training set (top) or there is too much occlusion (bottom) our method fails to produce a sensible shape estimate. For the first example, MagicPony fails to capture the articulation of the head, and for the second occluded example it predicts an average template shape with the wrong pose.
}
\vspace{-5pt}
\label{fig:supp_fail}
\end{figure*}

