\begin{figure}
\begin{center}    
    \begin{picture}(\columnwidth, 10)
    \put(0.4\columnwidth,5){A-CSM~\cite{kulkarni2020acsm}}
    \end{picture}
    \vspace{8pt}
    \includegraphics[height=2.5cm, trim=0 0 0 0, clip]{figs/acsm/0_acsm.png}
    \includegraphics[height=2.5cm, trim=0 0 0 0, clip]{figs/acsm/1_acsm.png}
    \includegraphics[height=2.5cm, trim=0 0 0 0, clip]{figs/acsm/2_acsm.png}
    
    \begin{picture}(\columnwidth, 10)
    \put(0.45\columnwidth,5){Ours}
    \end{picture}
    \includegraphics[height=2.5cm, trim=5 5 0 10, clip]{figs/acsm/0_saor_uv.png}
    \includegraphics[height=2.5cm, trim=10 25 0 25, clip]{figs/acsm/1_saor_uv.png}
    \includegraphics[height=2.5cm, trim=0 0 0 0, clip]{figs/acsm/2_saor_uv.png}
    
\end{center} \vspace{-12pt}
\caption{Comparison with A-CSM~\cite{kulkarni2020acsm} on horses using example images from their paper. 
Even though A-CSM uses a 3D template with pre-defined fixed parts, it still maps left and right legs to the same leg in the template and the legs are not consistent across viewpoints (\ie the part assignment is different in the top row depending on whether the horse is facing left or right. 
In contrast, despite not using any 3D object priors at training time, our method is much more consistent in its assignment.  
However, it does mistake one of the left for the the horse's tail in the final column. }
\label{fig:acsm}
\end{figure}
