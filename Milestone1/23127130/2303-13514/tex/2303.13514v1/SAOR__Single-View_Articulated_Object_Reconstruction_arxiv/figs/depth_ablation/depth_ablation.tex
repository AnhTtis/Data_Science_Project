\begin{figure*}
\begin{center}    
    \begin{picture}(0.49\textwidth, 5)
    \put(0.03\textwidth,0){Input}
    \put(0.12\textwidth,0){Pose}
    \put(0.24\textwidth,0){Reconstruction}
    \put(0.41\textwidth,0){Parts}
    \end{picture}
    \begin{picture}(0.49\textwidth, 5)
    \put(0.035\textwidth,0){Input}
    \put(0.13\textwidth,0){Pose}
    \put(0.24\textwidth,0){Reconstruction}
    \put(0.41\textwidth,0){Parts}
    \end{picture}
    \vspace{-5pt}
    \includegraphics[width=0.4975\textwidth, trim=5 0 5 0, clip]{figs/depth_ablation/d_037.png}
    \includegraphics[width=0.4975\textwidth, trim=5 0 5 0, clip]{figs/depth_ablation/d_296.png}
    \begin{tikzpicture}
    \draw (0,0) -- (\textwidth,0);
    \end{tikzpicture}
    \vspace{-5pt}
    \includegraphics[width=0.4975\textwidth, trim=5 0 5 0, clip]{figs/depth_ablation/nod_037.png}
    \includegraphics[width=0.4975\textwidth, trim=5 0 5 0, clip]{figs/depth_ablation/nod_296.png}
    
   

\end{center} \vspace{-12pt}
\caption{Comparison of models trained with relative depth supervision (top) and without (bottom). Our model trained without depth also estimates detailed 3D shapes with the correct viewpoint. However, the 3D predictions are marginally worse as it produces slightly wider 3D shapes.}
\label{fig:depth_ablation}
\end{figure*}
