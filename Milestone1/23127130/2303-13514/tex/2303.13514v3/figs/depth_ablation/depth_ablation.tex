\begin{figure*}
\begin{center}    
    \begin{picture}(0.49\textwidth, 5)
    \put(0.03\textwidth,0){Input}
    \put(0.12\textwidth,0){Pose}
    \put(0.24\textwidth,0){Reconstruction}
    \put(0.41\textwidth,0){Parts}
    \end{picture}
    \begin{picture}(0.49\textwidth, 5)
    \put(0.035\textwidth,0){Input}
    \put(0.13\textwidth,0){Pose}
    \put(0.24\textwidth,0){Reconstruction}
    \put(0.41\textwidth,0){Parts}
    \end{picture}
    \vspace{-5pt}
    \includegraphics[width=0.4975\textwidth, trim=5 0 5 0, clip]{figs/depth_ablation/09_d.png}
    \includegraphics[width=0.4975\textwidth, trim=5 0 5 0, clip]{figs/depth_ablation/00_d.png}
    \begin{tikzpicture}
    \draw (0,0) -- (\textwidth,0);
    \end{tikzpicture}
    \vspace{-5pt}
    \includegraphics[width=0.4975\textwidth, trim=5 0 5 0, clip]{figs/depth_ablation/09_nod.png}
    \includegraphics[width=0.4975\textwidth, trim=5 0 5 0, clip]{figs/depth_ablation/00_nod.png}
    

\end{center} \vspace{-12pt}
\caption{Comparison of models trained with relative depth supervision (top) and without (bottom). Our model trained without depth also estimates detailed 3D shapes with the correct viewpoint. However, the 3D predictions are marginally worse as the model without depth produces slightly wider 3D shapes. Please note that part assignment and pose orientation are different since the two models started from different random initializations.}
\label{fig:depth_ablation}
\end{figure*}
