\section{Preliminaries} \label{sec:preliminaries}


\subsection{System Model} \label{sec:System_model}

Let us consider a graph $\mathcal{G} = (\mathcal{V}, \mathcal{E})$ where we denote $\mathcal{V}$ as the set of $N_{\mathrm{a}}$ mobile agents and the set $\mathcal{E} \subset \mathcal{V} \times \mathcal{V}$ defines edge connections between agents. Each agent $i = 1,2,\dots,N_{\mathrm{a}}$ is assumed to have dynamics that can be represented in a discrete-time LTI state space form:
\begin{align} \label{eq:dynamical_model}
    \bm{\mathrm{x}}_{i}^{(k+1)} &= \bm{\mathrm{A}} \bm{\mathrm{x}}_i^{(k)} + \bm{\mathrm{B}} \bm{\mathrm{u}}_{i}^{(k)} + \bm{\nu}_i^{(k)} \\
    \label{eq:output_vector}
    \bm{\mathrm{y}}_i^{(k)} &= \bm{\mathrm{Cx}}_i^{(k)} + \bm{\eta}_i^{(k)}
\end{align}
with state $\bm{\mathrm{A}}$, input $\bm{\mathrm{B}}$, and output $\bm{\mathrm{C}}$ matrices consisting of appropriate dimensions, the state vector $\bm{\mathrm{x}}_i^{(k)} \hspace{-.3pt} \in \hspace{-.3pt} \R^n$, control input $\bm{\mathrm{u}}_{i}^{(k)} \hspace{-.3pt} \in \hspace{-.3pt} \R^{N_{\mathrm{m}}}$, and output vector $\bm{\mathrm{y}}_i^{(k)} \hspace{-.3pt} \in \hspace{-.3pt} \R^{N_{\mathrm{s}}}$. Within the state vector are the position coordinates $\bm{\mathrm{p}}_i^{(k)} \hspace{-.2pt} \in \hspace{-.3pt} \R^D$ in a $D$-dimensional Euclidean space. Process and measurement noises are described as i.i.d. zero-mean Gaussian distributions $\bm{\nu}_i^{(k)} \hspace{-.2pt} \sim \hspace{-.2pt} \mathcal{N}(\bm{0},\bm{\mathrm{Q}}) \hspace{-.2pt} \in \hspace{-.2pt} \R^{n}$ and $\bm{\eta}_i^{(k)} \hspace{-.3pt} \sim \hspace{-.2pt} \mathcal{N}(\bm{0},\bm{\mathrm{R}}) \hspace{-.2pt} \in \hspace{-.2pt} \R^{N_{\mathrm{s}}}$ with covariance matrices $\bm{\mathrm{Q}} > 0$ and $\bm{\mathrm{R}} > 0$, respectively. The use of a Kalman Filter (KF) provides state estimates $\hat{\bm{\mathrm{x}}}_i^{(k|k)} \in \R^n$ and predictions $\hat{\bm{\mathrm{x}}}_i^{(k+1|k)} \in \R^n$ for each agent $i$. 

In order for the multi-agent swarm to cooperatively maintain a desired proximity-based formation, the agents exchange their state estimate information. Each $i$th agent follows a control consensus $\mathcal{U}(\cdot,\cdot,\cdot)$ by
\begin{equation} \label{eq:consensus_control}
\bm{\mathrm{u}}_i^{(k)} = \mathcal{U} \big( \hat{\bm{\mathrm{x}}}_i^{(k|k)} , \hat{\bm{\mathrm{x}}}_j^{(k|k)}, \bm{\mathrm{x}}_{\mathrm{ref}}^{(k)} \big)
\end{equation}
to maintain a proximity-based formation while navigating within an environment, given $i \ne j$ and $j \in \mathcal{S}_i$ where $\mathcal{S}_i \subset \mathcal{V}$ is the neighbor set used for control purposes by an agent $i$ and $\bm{\mathrm{x}}_{\mathrm{ref}}^{(k)}$ is a reference state to follow.

\begin{definition}[Control Graph \cite{Paul_TRO}] \label{def:control_graph}
    Given each agent $i$ in the set $\mathcal V$ having a neighbor set for control $\mathcal{S}_i \subset \mathcal V$, we define the graph $\mathcal{G}_{\mathcal{U}} = ( \mathcal{V}, \mathcal{E}_{\mathcal{U}} )$ with the edge set,
    \begin{equation} \label{eq:control_edges}
        \mathcal{E}_{\mathcal{U}} = \big\{ (i,j) \; \big| \; j \in \mathcal{S}_i, \forall i \in \mathcal{V} \big\}
    \end{equation}
    as the \textit{control graph} of the agent set $\mathcal{V}$.
\end{definition}


\subsection{Threat Model} \label{sec:Attack_model}

During operations, we assume agents may experience cyber attacks or faults to on-board positioning sensors. With the loss of reliable position sensing, localization within the environment is compromised, thus degradation of the control performance occurs within proximity-based formations. Without loss of generality, the output vector from \eqref{eq:output_vector} is formalized in terms of position measurements $\bm{\mathrm{y}}_{i,[1:D]}^{(k)}$ (i.e., measuring the $D$-dimensional position) and sensor measurements of other states (if applicable) that are deemed non-vulnerable $\bm{\mathrm{y}}_{i,[(D+1):N_{\mathrm{s}}]}^{(k)}$ as $\bm{\mathrm{y}}_{i}^{(k)}~=~\big[ ( \bm{\mathrm{y}}_{i,[1:D]}^{(k)} )^{\mathsf{T}} \; ( \bm{\mathrm{y}}_{i,[(D+1):N_{\mathrm{s}}]}^{(k)} )^{\mathsf{T}} \big]^{\mathsf{T}}$. 

\vspace{2pt}
We denote the compromised position measurement vector due to cyber attacks or sensor faults as
\begin{equation} \label{eq:attacked_sensor}
    \widetilde{\bm{\mathrm{y}}}_{i,[1:D]}^{(k)} = \bm{\mathrm{C}}_{[1:D]} \bm{\mathrm{x}}_i^{(k)} + \bm{\eta}_{i,[1:D]}^{(k)} + \bm{\xi}_{i}^{(k)}
\end{equation}
where the vector $\bm{\xi}_{i}^{(k)} \in \R^D$ represents altered position measurements from the nominal behavior on an agent $i \in \mathcal{V}$ and $\bm{\mathrm{C}}_{[1:D]}$ indicates the first $D$ rows of the output matrix corresponding to position. When $\bm{\xi}_{i}^{(k)} \ne 0$ is satisfied (i.e., $\widetilde{\bm{\mathrm{y}}}_{i,[1:D]}^{(k)} \ne \bm{\mathrm{y}}_{i,[1:D]}^{(k)}$), this indicates the presence of cyber attacks manipulating position measurements or a faulty sensor.



\subsection{Communication Model} \label{sec:Communication_model}

\begin{figure}[b!]
\vspace{-10pt}
\hspace{-3pt} \subfloat[\label{fig:pathloss} ]{\includegraphics[width = 0.238\textwidth]{Figures/PathLossModel.png}}
\hspace{3pt} \subfloat[\label{fig:estimationError} ]{\includegraphics[width = 0.238\textwidth]{Figures/EstimationError.png}}
\vspace{-1pt}
\caption{An example of the path loss model and the incurred distance estimation error magnitude from RSSI measurements as distance increases.}
\label{fig:PathLossModel}
\vspace{-10pt}
\end{figure}
To overcome malicious cyber attacks or faults to position sensors, agents measure noisy RSSI from the received communications of nearby agents. A commonly-used path loss model is the log-normal shadowing model \cite{goldsmith_2005} defined by:
\vspace{-7pt}
\begin{equation} \label{eq:comm_model}
    P_{ij,[\text{rx}]}^{(k)} = P_{[\text{tx}]} - PL(\mathrm{d}_0) - 10 \beta \log \frac{\mathrm{d}_{ij}^{(k)}}{\mathrm{d}_0} + \Lambda
\end{equation}
where $P_{ij,[\text{rx}]}^{(k)}$ is the measured received power by an agent $i$ of an agent $j$, $PL(\mathrm{d}_0)$ is the power loss (in dB) from a reference distance $\mathrm{d}_0 \in \R_{>0}$, and $\mathrm{d}_{ij}^{(k)} = \| \bm{\mathrm{p}}_i^{(k)} - \bm{\mathrm{p}}_j^{(k)} \|$ denotes the true distance between agents $i$ and $j$. The channel shadowing $\Lambda \sim \mathcal{N}(0,\sigma^2_{\Lambda})$ is modeled as a zero-mean Gaussian noise and $\beta$ is the path loss exponent. It is assumed that all agents have the same transmitting power $P_{[\text{tx}]}$ which is known by the agents beforehand. Fig.~\ref{fig:PathLossModel} provides an example of received signal strength that follows the assumed path loss model with shadowing \eqref{eq:comm_model} and the impact to the corresponding distance estimation as distance between agents increases.

\begin{definition}[Communication Graph \cite{Paul_TRO}] \label{def:comm_graph}
    Given the $N_{\mathrm{a}}$ agents in set $\mathcal V$ with a known maximum communication range $\delta_{\mathrm{c}} \in \R_{>0}$, we define the graph $\mathcal{G}_{\mathcal{C}} = ( \mathcal{V}, \mathcal{E}_{\mathcal{C}} )$ with the edge set represented by
    \begin{equation} \label{eq:comm_edges}
        \mathcal{E}_{\mathcal{C}} = \big\{ (i,j) \; \big| \; \big\| \bm{\mathrm{p}}_i^{(k)} - \bm{\mathrm{p}}_j^{(k)} \big\| \leq \delta_{\mathrm{c}}, \; i,j \in \mathcal{V} \big\}
    \end{equation}
    as the \textit{communication graph} of the agent set $\mathcal{V}$. The set $\mathcal{C}_i = \{ j \in \mathcal{V} \; | \; (i,j) \in \mathcal{E}_{\mathcal{C}} \}$ represents any mobile agent $j$ that is within communication range of an agent $i$ to receive its broadcast information signal.
\end{definition}


\subsection{Problem Formulation} \label{sec:problem}

Given the multi-agent formation topology described by the control graph $\mathcal{G}_{\mathcal{U}}$ that can suffer from degraded control performance due to cyber attacks or faults on individual agent's localization sensors (i.e., $\bm{\xi}_{i}^{(k)} \ne 0$), we are interested in solving the following problems:
\begin{problem} \label{problem1}
\textit{(Detection and Sensor Reconfiguration)} Create a policy such that any agent $i \in \mathcal{V}$ that detects anomalous position sensor measurement behavior can reconfigure its sensor model in the $D$-dimensional space to satisfy:
\begin{equation} \label{eq:reconfiguration}
    \widetilde{\bm{\mathrm{y}}}_{i,[1:D]}^{(k)} \longrightarrow \bar{\bm{\mathrm{y}}}_{i,[1:D]}^{(k)}
\end{equation}
by leveraging the known communication model to provide reliable position measurements $\bar{\bm{\mathrm{y}}}_{i,[1:D]}^{(k)}$ to re-localize itself.
\end{problem}
\vspace{1pt}

Upon detection of sensor attacks/faults and sensor reconfiguration, we want to improve state estimation performance to accommodate the updated sensor measurement model.
\begin{problem} \label{problem2}
\textit{(Estimation Error Minimization)} Create a policy $\mathcal{P}$ where an agent $i \hspace{-1pt} \in \hspace{-1pt} \mathcal{V}$ adaptively estimates the unknown covariance $\bar{\bm{\mathrm{R}}}_i^{(k)}$ for the RSSI-based position measurements that is robust to an unreliable on-board state estimate. Given the updated sensor model in Problem \ref{problem1}, the policy $\mathcal{P}$ follows:
\begin{equation} \label{prob:minimize_er}
    \mathcal{P}\big( \bar{\bm{\mathrm{R}}}_i^{(k)} \big) \longrightarrow \min \Big( \big( \bm{\mathrm{e}}_i^{(k)} \big)^{\mathsf{T}} \bm{\mathrm{e}}_i^{(k)} \Big)
\end{equation}
to minimize its state estimation error $\bm{\mathrm{e}}_i^{(k)} \hspace{-.4pt} = \bm{\mathrm{x}}_i^{(k)} \hspace{-.4pt} - \hat{\bm{\mathrm{x}}}_i^{(k)}$ to provide robust estimation performance within MAS formations.
\end{problem}
