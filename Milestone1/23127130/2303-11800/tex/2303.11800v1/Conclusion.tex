\section{Conclusions} \label{sec:conclusion}

This paper provides a decentralized framework for multi-agent systems to resiliently navigate in the presence of cyber attacks and/or faults to on-board positioning sensors within open or unknown environments that lack identifiable landmarks (i.e., anchors) and also operate beyond distance/range sensing of other nearby agents. Upon detection of anomalous sensor behavior, an agent performs sensor reconfiguration to leverage RSSI-based measurements from the nearby agents (i.e., mobile landmarks) as a replacement for the original position sensor. An adaptive Kalman Filtering method accommodates the updated position sensor by estimating its unknown measurement covariance to reduce estimation error for improved control performance within the swarm. From here, future work includes improving the robustness of our framework when the assumed path loss model does not hold due to instability of RSSI signals, for example because of the presence of cluttered environments that create multi-path behavior in the communication broadcasts.

% In our future work, we plan to investigate improving the robustness of our framework when the assumed path loss model does not hold due to instability of RSSI signals for example because of the presence of cluttered environments that create multi-path behavior in the communication broadcasts.


\section*{Acknowledgements}

This work is based on research sponsored by the National Science Foundation under grants 1816591 and 1916760. 