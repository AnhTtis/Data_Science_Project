\section{Results} \label{sec:results}

Our approach is validated with MATLAB simulations on swarms of $N_{\mathrm{a}}=12$ mobile agents modeled with double integrator dynamics satisfying \eqref{eq:dynamical_model}. The MAS performs a go-to-goal operation within a 2-dimensional plane (i.e., $D=2$) while experiencing cyber attacks and faults to on-board positioning sensors. As a case study, we employ a virtual spring-damper mesh \cite{Paul_TRO} for decentralized proximity-based formation control to validate the RSSI-based localization framework for resilient formation control. For all simulations, the agents begin with randomized initial positions and are tasked to perform a go-to-goal operation while maintaining $l^{\mathrm{des}} = 8$m distance from any neighboring agent. Additionally, at time $k=350$, five agents are randomly chosen to suffer from a cyber attack and two experience sensor faults to on-board positioning sensors. The communication model has a path loss exponent $\beta = 2$ and shadowing noise of $\Lambda = \mathcal{N}(0,2)$, while the forgetting parameter used is $\gamma = 0.01$ for measurement covariance estimation.

\begin{figure}[tb!]
\vspace{-1pt}
\hspace{-3pt} \subfloat[\label{fig:first_sim} ]{\setlength{\fboxsep}{0pt}\fbox{\includegraphics[width = 0.238\textwidth]{Figures/sim1_NoDetect2.png}}}
\hspace{-1pt} \subfloat[\label{fig:second_sim} ]{\setlength{\fboxsep}{0pt}\fbox{\includegraphics[width = 0.238\textwidth]{Figures/sim2_NoDetect2.png}}} \\[3.5pt]
\hspace{-1pt} \subfloat[\label{fig:third_sim} ]{\setlength{\fboxsep}{0pt}\fbox{\includegraphics[width = 0.238\textwidth]{Figures/sim3_NoDetect2.png}}}
\hspace{-1pt} \subfloat[\label{fig:fourth_sim} ]{\setlength{\fboxsep}{0pt}\fbox{\includegraphics[width = 0.238\textwidth]{Figures/sim4_NoDetect2.png}}}
\vspace{-1pt}
\caption{An unprotected system of $N_{\mathrm{a}} = 12$ agents compromised by cyber attacks and faults to on-board position sensors on seven agents (red disks) resulting in them being diverted away from the goal (green region).}
\label{fig:Simulation}
\vspace{-4pt}
\end{figure}

\begin{figure}[tb!]
\hspace{-3pt} \subfloat[\label{fig:first_sim2} ]{\setlength{\fboxsep}{0pt}\fbox{\includegraphics[width = 0.238\textwidth]{Figures/sim1_Detect2.png}}}
\hspace{-1pt} \subfloat[\label{fig:second_sim2} ]{\setlength{\fboxsep}{0pt}\fbox{\includegraphics[width = 0.238\textwidth]{Figures/sim2_Detect2.png}}} \\[3.5pt]
\hspace{-1pt} \subfloat[\label{fig:third_sim2} ]{\setlength{\fboxsep}{0pt}\fbox{\includegraphics[width = 0.238\textwidth]{Figures/sim3_Detect2.png}}}
\hspace{-1pt} \subfloat[\label{fig:fourth_sim2} ]{\setlength{\fboxsep}{0pt}\fbox{\includegraphics[width = 0.238\textwidth]{Figures/sim4_Detect2.png}}}
\vspace{-1pt}
\caption{An MAS leveraging our framework resiliently navigates to the desired goal point while experiencing cyber attacks and faults to on-board position sensors (recovered agents in yellow).}
\label{fig:Simulation2}
\vspace{-12pt}
\end{figure}

In the first simulation presented in Fig.~\ref{fig:Simulation}, we show a sequence of snapshots of an unprotected multi-agent system navigating toward a goal region (green disk). The implemented cyber attacks occur simultaneously on the five compromised agents with the intent of diverting their true positions toward the undesired region (red region). Beginning in Fig.~\ref{fig:first_sim}, all agents are uncompromised (blue disks) and perform in a nominal manner; however in Fig.~\ref{fig:second_sim}, the compromised agents are subject to malicious cyber attacks and faults to their positioning sensors. In Figs.~\ref{fig:third_sim}-\subref{fig:fourth_sim}, the true states of the compromised agents (red disks) are driven to undesired regions in the environment, while the remaining uncompromised agents (blue disks) along with the corresponding unreliable (i.e., incorrect) position estimates of the compromised agents (empty disks) continue navigating toward the goal.
In Fig.~\ref{fig:Simulation2} we perform the same simulation as in Fig.~\ref{fig:Simulation}, but this time the MAS is utilizing our framework for resiliency. As shown in the series of snapshots, all seven agents are able to: 1) detect the anomalous on-board positioning sensor behavior and 2) perform sensor reconfiguration to RSSI-based position measurements to resiliently maintain desired performance within the MAS such that all agents safely reach the desired goal.


We provide a comparison between various MAS scenarios by showing the true proximity error between neighboring agents $(i,j) \hspace{-.9pt} \in \hspace{-.9pt}  \mathcal{E}$. We highlight the scenarios, which include: 1) no detection and recovery, 2) a non-robust noise covariance update method of $\bm{\mathrm{R}}_i^{(k)}$ in \cite{AdaptiveR_EKF}, 3) detection and recovery without updating $\bm{\mathrm{R}}_i^{(k)}$, and 4) our proposed robust method for updating $\bm{\mathrm{R}}_i^{(k)}$. Fig.~\ref{fig:FormationError} shows the average formation proximity error over $400$ simulations with randomized initial positions for each scenario. At each time step $k$, the formation proximity error is computed as $\mathrm{E}^{(k)} =  \frac{1}{|\mathcal{E}|} \sum_{\forall (i,j) \in \mathcal{E}} \big| \| \bm{\mathrm{p}}_i^{(k)} - \bm{\mathrm{p}}_j^{(k)} \| - l^{\text{des}} \big| $. In Table~\ref{table:PositionError}, the results of position estimation error $\bm{\mathrm{e}}_{i,x/y}$ in the 2-dimensional coordinate frame are presented for any compromised agent $i$. Our proposed method to adaptively update the unknown RSSI-based position measurement covariance matrix reduces estimation error for more desirable control performance within the swarm.


\begin{figure}[tb!]
\centering
\includegraphics[width = 0.485\textwidth]{Figures/FormationError2.png}
\vspace{-13pt}
\caption{A comparison of inter-agent proximity error within the formation.}
\label{fig:FormationError}
\vspace{-10pt}
\end{figure}

%\vspace{-5pt}
%\small
\begin{table}[htb!]
\caption{Position Estimation Error.}
\vspace{-4pt}
\centering
%\begin{tabular}{ p{2.1cm}||p{1.8cm}|p{1.8cm} }
\begin{tabular}{ p{2.4cm}||p{2cm}|p{2cm} }
 \rule{0pt}{1.2\normalbaselineskip} & \centering No Update & \hspace*{2pt} Robust Update \\[1pt]
 \hline
 \rule{0pt}{.9\normalbaselineskip} \centering Variance $[\mathrm{e}_{i,x/y}]$ & \centering $0.249/0.253$ & \hspace*{3pt} $0.151/0.144$ \\[.2pt]
% \hline
\end{tabular}
\label{table:PositionError}
%\vspace{-10pt}
\end{table}
\vspace{-4pt}
%\normalsize
