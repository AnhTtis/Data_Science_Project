\section{Introduction} \label{sec:introduction}

One of the essential capabilities for a mobile autonomous system is to localize itself within an environment. The ability to perform an accurate and robust localization allows unmanned systems to achieve truly autonomous operations. These operations can be accomplished in various ways, by relying on positioning sensors like global positioning system (GPS), odometry, and initial measurement unit or through the use of range sensors such as LiDAR, infrared, and camera systems. The sensing information can then be leveraged via localization methods such as Particle filters and Simultaneous Localization and Mapping (SLAM) techniques. 

When considering multi-agent system (MAS) applications, for example robotic swarms, consensus algorithms are typically considered where agents share their states to attain coordinated behaviors in a decentralized fashion to accomplish a desired goal \cite{formation_control1}. When information being exchanged is incorrect, the MAS can be hijacked and lead to unsafe conditions \cite{7822915}. A variety of issues can cause undesirable information to be exchanged between agents, such as cyber attacks or faults to on-board sensors. If known landmarks are present in the operating space, range sensors can be utilized for localization or to determine if the system is performing as expected. However, landmarks may not be available if an MAS is navigating in an open terrain, thus leaving compromised agents unable to reliably localize themselves.


With such premises, the focus on this paper resides on the problem of resilient coordination in MASs that leverage control consensus schemes when one or more agents lose localization capabilities as on-board positioning sensors become unreliable. More specifically, agents are tasked to maintain desired formations within open or unknown environments that do not offer identifiable landmarks and also operate beyond sensing range of nearby agents. To deal with this, the proposed framework enables compromised agents to leverage Received Signal Strength Indication (RSSI) and received position information from neighboring agents for localization (i.e., mobile landmarks). Multilateration is performed using the noisy RSSI measurements and received neighboring agent's positions to provide RSSI-based position measurements in replacement of the unreliable on-board position sensors. To minimize the RSSI-based position measurement error, a weighted least squares method is used. Moreover, the RSSI-based position measurements have an unknown covariance which differ from the nominal on-board position sensor. To improve state estimation performance, compromised agents leverage an adaptive Kalman Filtering method that estimates the RSSI-based position measurement covariance matrix at runtime. The proposed framework is introduced in a generalized manner that may be used on any formation control technique for swarms of homogeneous linear time-invariant (LTI) modeled agents. As a specific case study in this paper, a virtual spring-damper physics model \cite{Paul_TRO} for proximity-based formation control is considered on MASs in a $2$-dimensional coordinate frame. However, our MAS framework can be expanded to heterogeneous systems \cite{hetergeneous_MAS}, non-linear modeled agents \cite{Nonlinear_MAS}, and higher dimension coordinate frames \cite{positioning_3D}. 
\vspace{-2pt}

\subsection{Related Literature}

The topic of system resilience has received significant attention recently, notably in the area of localization within single- and multi-agent system operations \cite{MAS_survey}. Various time-based measurement techniques used for localization have been proposed, such as Time of Arrival (ToA), Time Difference on Arrival (TDoA), Angle of Arrival (AoA), and Time of Flight (ToF) \cite{localization_survey}. Another technique to obtain ranges are from measuring the Received Signal Strength Indicator (RSSI). As the name suggests, RSSI-based techniques rely on measuring the strength of the received radio frequency signal over the communication channel. 

Much effort has been placed on leveraging RSSI to aid in localization within an environment. In particular, numerous articles have leveraged anchor nodes with known positions to aid in localization within indoor environments \cite{survey_LocalizationRSSI}. One example is found in \cite{hsu2016particle}, the proposed framework utilizes a particle filter on the RSSI measurements from known anchor nodes and then fused the position estimate with the remaining system states for improved localization capabilities. In \cite{RSSI_DVhop}, the Weighted Distance Vector Hop algorithm using RSSI is combined with a weighted hyperbolic localization algorithm to estimate the position of any nearby agents in MASs while utilizing the known locations of anchor nodes in the environment. Authors in \cite{RSSI_NarrowCorridor} proposed a method for robotic swarms deployed in indoor environments to effectively navigate through narrow passageways by allocating specific roles to robots to ensure localization accuracy. An approach \cite{RSSI_mixed} was proposed to provide robustness in localization performance within swarms when nearby agents satisfy both Line-of-Sight and Non-Line-of-Sight conditions. In \cite{Mostofi}, RSSI signals are leveraged to estimate AoA of signal sources (i.e., transmitters) and humans/robots for target tracking. 
%In \cite{RSSI_SAR}, theoretical results for an adaptive framework are presented that positions a team of robotic agents acting as mobile routers to provide communication coverage to a remaining subset of client robots, when the knowledge of the clients' positions are unknown. The multi-robot system is able to position the robots behaving as routers to satisfy the client robots' demands, while adapting to changes in wireless signals (i.e., leveraging directionality of signal strength) and the dynamic (and potentially unmapped) environment. 

Similar to our work, authors in \cite{oliveira2014rssi} proposed an RSSI-based localization algorithm for multi-robot teams within anchor-less environments. Their approach combines a Kalman Filter and the Floyd-Warshall algorithm to compute smooth distance estimates between agents, then multidimensional scaling is utilized to estimate relative positions of nearby agents. Differing from \cite{oliveira2014rssi}, we assume that each agent has a nominal localization sensor that is vulnerable to cyber attacks and faults. As such scenarios arise, our framework allows for any compromised agent to perform sensor reconfiguration by removing the nominal position sensor in favor of RSSI-based position sensing capabilities. Moreover, our decentralized approach does not suffer from scalability issues as in \cite{oliveira2014rssi} where authors claim their framework is effective on mobile robot teams of ``approximately up to 10". 

Our work utilizes a similar principal to previous literature that have characterized adaptive adjustments to estimate unknown noise covariances in dynamic systems \cite{AdaptiveR_EKF}, \cite{AdaptiveR_UKF}. In these works, measurement residuals are used to adaptively update estimates of the noise covariance matrices. However, the authors assumed that measurement noise always follows a zero-mean Gaussian distribution (i.e., attack/fault-free conditions), hence the measurement residuals used for noise covariance estimation are also zero-mean Gaussian random variables. In the presence of compromised sensor measurements, state estimates are unreliable; thus compromising the covariance matrix estimation process.

The contributions of this work include: i) a decentralized framework to provide resiliency to MAS formations in the presence of cyber attacks and faults to critical on-board positioning sensors when operations occur in open/unknown environments that lack known landmarks and agents are beyond range sensing of other neighboring agents. Compromised agents perform sensor reconfiguration to leverage RSSI from the nearby agents (i.e., mobile landmarks) to aid in re-localization in replacement of its compromised on-board position sensor, and ii) a novel adaptive Kalman Filtering approach to update the unknown RSSI-based position measurement covariance at runtime that is robust to unreliable state estimates (improving upon \cite{AdaptiveR_EKF,AdaptiveR_UKF}), to optimize position estimation of compromised agents.

