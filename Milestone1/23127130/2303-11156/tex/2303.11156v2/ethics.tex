\section{Discussion}

Recent advancements in NLP show that LLMs can generate human-like texts for a various number of tasks \citep{gpt4}. However, this can create several challenges. LLMs can potentially be misused for plagiarism, spamming, or even social engineering to manipulate the public. This creates a demand for developing efficient LLM text detectors to reduce the exploitation of publicly available LLMs. Recent works propose a variety of AI text detectors using watermarking \citep{kirchenbauer2023watermark}, zero-shot methods \citep{mitchell2023detectgpt}, retrieval-based methods \citep{krishna2023paraphrasing} and trained neural network-based classifiers \citep{openaidetectgpt2}. 
In this paper, we show both theoretically and empirically, that the state-of-the-art detectors cannot reliably detect LLM outputs in practical scenarios. Our experiments show that paraphrasing the LLM outputs helps evade these detectors effectively. 
Attackers can generate and spread misinformation using LLMs and use neural network-based paraphrasers without getting detected. 
Moreover, our theory demonstrates that for a sufficiently advanced language model, even the best detector can only perform marginally better than a random classifier. This means that for a detector to have both low type-I and type-II errors, it will have to trade off the LLM's performance. We also empirically show that watermarking and retrieval-based detectors can be spoofed to make human-composed text detected as AI-generated. 
For example, we show that it is possible for an attacker to learn the soft watermarking scheme in \citep{kirchenbauer2023watermark}. Using this information, an adversary can launch a spoofing attack where adversarial humans generate texts that are detected to be watermarked. 
Spoofing attacks can lead to the generation of derogatory passages detected as AI-generated that might affect the reputation of the LLM detector developers.




With the release of GPT-4 \citep{gpt4}, the applications of LLMs are endless. This also calls for the need for more secure methods to prevent their misuse. Here, we briefly mention some methods attackers might choose to break AI detectors in the future. As we demonstrated in this paper, the emergence of improved paraphrasing models can be a severe threat to AI text detectors. Moreover, advanced LLMs might be vulnerable to attacks based on {\it smart prompting}. For example, attackers could input a prompt that starts with ``Generate a sentence in active voice and present tense using only the following set of words that I provide...''. High-performance LLMs would have a low entropy output space (less number of likely output sequences) for this prompt, making it harder to add a strong LLM signature in their output for detection. The soft watermarking scheme in \cite{kirchenbauer2023watermark} is vulnerable to this attack. If the logits of the LLM have low entropy over the vocabulary, soft watermarking scheme samples the token with the highest logit score (irrespective of the green list tokens) to preserve model perplexity. Furthermore, in the future, we can expect more open-source LLMs to be available to attackers. This could help attackers leverage these models to design transfer attacks to target a larger LLM. Adversarial input prompts could be designed using transfer attacks such that the target LLM is encouraged to have a low entropy output space. Future research on AI text detectors must be cautious about these vulnerabilities.

A detector should ideally be helpful in reliably flagging AI-generated texts to prevent the misuse of LLMs. However, the cost of misidentification by a detector can itself be huge. If the false positive rate of the detector is not low enough, humans could get wrongly accused of plagiarism. Moreover, a disparaging passage falsely detected to be AI-generated could affect the reputation of the LLM's developers. As a result, the practical applications of AI-text detectors can become unreliable and invalid. Security methods need not be foolproof. However, we need to make sure that it is not an easy task for an attacker to break these security defenses. Thus, analyzing the risks of using current detectors can be vital to avoid creating a false sense of security. We hope that the results presented in this work can encourage an open and honest discussion in the community about the ethical and trustworthy applications of generative LLMs. 

%Recent works on AI-text detection have raised concerns about our theoretical results regarding the constraints on detection performance. For instance,

Recent follow-up work by \citet{chakraborty2023possibilities} has argued that AI-text detection is almost always possible, even when the total variation between human and AI-generated distributions is low, through boosting a detector's performance via {\it independent} and identically distributed text samples.
However, this assumption of independence may not hold in most real-world applications as human-written text often has correlations (e.g., the second sentence in a paragraph is usually related to the first one, the third sentence is related to the first two, etc). If a detector makes a mistake on one sample, it will likely make the same mistake on the other samples as well due to such correlations. 
It may also be infeasible, in many real-world applications, to obtain several samples of the entire text. For instance, it would be unreasonable to expect a student to submit several versions of their essay just to determine whether it has been written using AI or not. Also, an adversary could mix both types of samples. For example, an AI Twitter bot could evade detection by alternating between AI and human text. 

%using standard results in information theory. However, the underlying assumption of their result is that several {\it independent} samples are available to the detector from either distribution.
% This may not be a practical assumption since sentences in a document are often correlated with each other.
% Such correlations may even open up more possibilities to evade detection.

%their underlying i.i.d. assumption is not practical and opens up more possibilities for an adversary to evade detection. 

Some other works (\citet{detectgpt_openreview} and~\citet{kirchenbauer2023reliability}) argue that the outputs of most existing LLMs are very different from any particular human being.
%Another line of counter-arguments, presented in~\cite{detectgpt_openreview} and~\cite{kirchenbauer2023reliability}, is that the outputs of most existing LLMs are very different from any particular human being.
This could be due to special training methods like reinforcement learning with human feedback, or sampling procedures like the greedy sampler.
This could cause the total variation distance between the human and AI-generated text distributions to be high.
However, the central message of our impossibility results is that if an adversary seeks to evade detection, their ability to do so improves as language models advance.
Most publicly available language models today are not designed to be stealthy.
However, there are several services out there whose primary goal is to not get detected\footnote{Undetectable AI: \url{https://undetectable.ai/}, StealthGPT: \url{https://www.stealthgpt.ai/}}.
Their ability to produce high-quality text that does not get detected will most likely increase in the future.
A detection mechanism that works by making assumptions about the training or sampling procedure of the language model -- or any assumption at all about the model for that matter -- may fail against an adversary that deliberately violates them to avoid detection.

In addition to reliability issues of detectors that we have studied in this paper, a recent work by \citet{liang2023gpt} has shown that these detectors can be  biased against non-native English writers. Thus, having a small {\it average} type I and II errors may not be sufficient to deploy a detector in practice: such a detector may have very large errors within a sub-population of samples such as essays written by non-native English writers or essays on a particular topic or with a particular writing style. 

We hope that results and discussions presented in this work can initiate an honest discussion within the community concerning the ethical and dependable utilization of AI-generated text in practice.

