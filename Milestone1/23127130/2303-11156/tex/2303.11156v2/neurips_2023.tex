\documentclass{article}


% if you need to pass options to natbib, use, e.g.:
%     \PassOptionsToPackage{numbers, compress}{natbib}
% before loading neurips_2023
\PassOptionsToPackage{numbers}{natbib}


% ready for submission
% \usepackage{neurips_2023}


% to compile a preprint version, e.g., for submission to arXiv, add add the
% [preprint] option:
\usepackage[preprint]{neurips_2023}


% to compile a camera-ready version, add the [final] option, e.g.:
%     \usepackage[final]{neurips_2023}


% to avoid loading the natbib package, add option nonatbib:
%    \usepackage[nonatbib]{neurips_2023}


\usepackage[utf8]{inputenc} % allow utf-8 input
\usepackage[T1]{fontenc}    % use 8-bit T1 fonts
\usepackage{hyperref}       % hyperlinks
\usepackage{url}            % simple URL typesetting
\usepackage{booktabs}       % professional-quality tables
\usepackage{amsfonts}       % blackboard math symbols
\usepackage{nicefrac}       % compact symbols for 1/2, etc.
\usepackage{microtype}      % microtypography
\usepackage[dvipsnames]{xcolor}         % colors
\usepackage{wrapfig}
\usepackage{float}
\usepackage{todonotes}

%%%%%%%%%% Custom Packages and Commands %%%%%%%%%%
\usepackage{amsthm}

\newtheorem{theorem}{Theorem}
\newtheorem{corollary}{Corollary}
\newtheorem{lemma}{Lemma}
\usepackage{graphicx}
\usepackage{amsmath}
\usepackage{amssymb}
\usepackage{subcaption}
\usepackage{mathtools}
\usepackage[bb=dsserif]{mathalpha}

\usepackage{wrapfig}
\usepackage{soul}
\usepackage{makecell}
\usepackage{lipsum}

\usepackage{xr}
\externaldocument{appendix}

\hypersetup{colorlinks=true, linkcolor=NavyBlue, urlcolor=NavyBlue, citecolor=NavyBlue}

\newcommand{\tv}{\mathsf{TV}}
\newcommand{\arc}{\mathsf{AUROC}}

\usepackage{array}
\newcolumntype{P}[1]{>{\centering\arraybackslash}m{#1}}
\newcolumntype{M}[1]{>{\arraybackslash}m{#1}}

\newcommand\blfootnote[1]{%
  \begingroup
  \renewcommand\thefootnote{}\footnote{#1}%
  \addtocounter{footnote}{-1}%
  \endgroup
}

\newcommand{\ak}[1]
{\textcolor{blue}{AK: #1}}
\newcommand{\wwx}[1]
{\textcolor{cyan}{wenxiao: #1}}
\newcommand{\vs}[1]
{\textcolor{teal}{VS: #1}}
\newcommand{\sriram}[1]
{\textcolor{purple}{Sriram: #1}}
\newcommand{\SF}[1]
{\textcolor{red}{SF: #1}}

\title{Can AI-Generated Text be Reliably Detected?}


% The \author macro works with any number of authors. There are two commands
% used to separate the names and addresses of multiple authors: \And and \AND.
%
% Using \And between authors leaves it to LaTeX to determine where to break the
% lines. Using \AND forces a line break at that point. So, if LaTeX puts 3 of 4
% authors names on the first line, and the last on the second line, try using
% \AND instead of \And before the third author name.

\author{%
  Vinu Sankar Sadasivan \\
  \texttt{vinu@umd.edu} \\
  \And
  Aounon Kumar \\
  \texttt{aounon@umd.edu} \\
  \AND
  Sriram Balasubramanian \\
  \texttt{sriramb@umd.edu} \\
  \And 
  Wenxiao Wang \\
  \texttt{wwx@umd.edu} \\
  \And
  Soheil Feizi \\
  \texttt{sfeizi@umd.edu} \\
  \AND \vspace{-0.5cm}   \\ 
  Department of Computer Science \\ 
  University of Maryland \\ 
}


\begin{document}


\maketitle
\label{title}
\addcontentsline{toc}{section}{~~~~~~Abstract}



Over the past few years, there has been a significant amount of research focused on studying the ReLU activation function, with the aim of achieving neural network convergence through over-parametrization. However, recent developments in the field of Large Language Models (LLMs) have sparked interest in the use of exponential activation functions, specifically in the attention mechanism.

Mathematically, we define the neural function $F: \R^{d \times m} \times  \mathbb{R}^d \rightarrow \mathbb{R}$ using an exponential activation function. Given a set of data points with labels $\{(x_1, y_1), (x_2, y_2), \dots, (x_n, y_n)\} \subset \mathbb{R}^d \times \mathbb{R}$ where $n$ denotes the number of the data. Here $F(W(t),x)$ can be expressed as $F(W(t),x) := \sum_{r=1}^m a_r \exp(\langle w_r, x \rangle)$, where $m$ represents the number of neurons, and $w_r(t)$ are weights at time $t$. It's standard in literature that $a_r$ are the fixed weights and it's never changed during the training. We initialize the weights $W(0) \in \mathbb{R}^{d \times m}$ with random Gaussian distributions, such that $w_r(0) \sim \mathcal{N}(0, I_d)$ and initialize $a_r$ from random sign distribution for each $r \in [m]$.

Using the gradient descent algorithm, we can find a weight $W(T)$ such that $\| F(W(T), X) - y \|_2 \leq \epsilon$ holds with probability $1-\delta$, where $\epsilon \in (0,0.1)$ and $m = \Omega(n^{2+o(1)}\log(n/\delta))$. To optimize the over-parametrization bound $m$, we employ several tight analysis techniques from previous studies [Song and Yang arXiv 2019, Munteanu, Omlor, Song and Woodruff ICML 2022]. 

 

\newpage
\tableofcontents
\newpage

\section{Introduction}
\label{sec:introduction}
% \begin{itemize}
%     % Diffusion of FL
%     \item {\st{Diffusion of FL}}
%     % Security threats to FL
%     \item {\st{Security threats to FL with particular focus on model poisoning}}
%     % Limitations of existing countermeasures
%     \item {\st{Current countermeasures (e.g., KRUM) and their limitations}}
%     % Proposed method and its advantages
%     \item {\st{Intuitive description of the proposed method and its difference (i.e., advantages) w.r.t. state of the art}}
%     % Main contributions
%     \item {\st{Summary of the main contributions of this work}}
%     % Paper's structure and organization
%     \item {\st{Paper's structure and organization}}
% \end{itemize}

% Diffusion of FL
Recently, {\em federated learning} (FL) has emerged as the leading paradigm for training distributed, large-scale, and privacy-preserving machine learning (ML) systems~\cite{mcmahan2017googleai,mcmahan2017aistats}. 
The core idea of FL is to allow multiple edge clients to collaboratively train a shared, global model without disclosing their local private training data.
%Specifically, an FL system consists of a central server and many edge clients; 
A typical FL round involves the following steps: {\em(i)} the server randomly picks some clients and sends them the current, global model; {\em(ii)} each selected client locally trains its model with its own private data; then, it sends the resulting local model to the server;\footnote{Whenever we refer to global/local model, we mean global/local model {\em parameters}.} {\em(iii)} the server updates the global model by computing an \emph{aggregation function}, usually the average (FedAvg), on the local models received from clients.
% \begin{enumerate}
%     \item[{\em(i)}] the server sends the current, global model to the clients and appoints some of them for training;
%     \item[{\em(ii)}] each selected client locally trains its copy of the global model with its own private data; then, it sends the resulting local model back to the server;\footnote{Whenever we refer to global/local model, we mean global/local model {\em parameters}.}
%     \item[{\em(iii)}] the server updates the global model by computing an \emph{aggregation function} on the local models received from clients (by default, the average, also referred to as FedAvg~\cite{mcmahan2017aistats}).
% \end{enumerate}
This process goes on until the global model converges. %(e.g., after a certain number of rounds or other similar stopping criteria).
%\\
% The advantages of FL over the traditional, centralized learning paradigm are undoubtedly clear in terms of flexibility/scalability (clients can join/disconnect from the FL network dynamically), network communications (only model weights\footnote{We will use \textit{parameters} and \textit{weights} interchangeably.} are exchanged between clients and server), and privacy (each client's private training data is kept local at the client's end and not uploaded to the server).
\\
% Security threats to FL
%However, the growing adoption of FL also raises security concerns~\cite{costa2022covert}, particularly about its confidentiality, integrity, and availability.
Although its advantages over standard ML, FL also raises security concerns~\cite{costa2022covert}. %, particularly about its confidentiality, integrity, and availability~\cite{costa2022covert}.
% OLD, LONG VERSION
% Indeed, some work deals with privacy leakage that may expose the local data of some clients~\cite{melis2019sp}. 
% A large body of work, instead, investigates attacks that usually aim to detriment the predictive accuracy of the learned global model. For instance, \emph{data poisoning} attacks achieve this goal by letting an adversary pollute the training set of some corrupt FL clients with maliciously crafted examples~\cite{jagielski2018sp}.
% Similarly, in \emph{model poisoning} the attacker attempts to tweak the global model weights~\cite{bhagoji2019pmlr} by directly perturbing the local model's weights of some infected FL clients before these are sent to the central server for aggregation, usually via so-called Byzantine attacks. 
% It turns out that Byzantine model poisoning attacks severely impact standard FedAvg; therefore, more robust aggregation functions must be designed to make FL systems secure.
Here, we focus on \emph{untargeted model poisoning} attacks~\cite{bhagoji2019pmlr}, where an adversary attempts to tweak the global model weights %\footnote{We will use the terms \textit{parameters} and \textit{weights} interchangeably.} 
by directly perturbing the local model's parameters of some infected clients before these are sent to the central server for aggregation.
In doing so, the adversary aims to jeopardize the global model \textit{indiscriminately} at inference time.
Such model poisoning attacks severely impact standard FedAvg; therefore, more robust aggregation functions must be designed to secure FL systems.
\\
% In this paper, we focus on designing a novel robust aggregation scheme at the server's end to contrast the effect of Byzantine model poisoning attacks.
%
% Current countermeasures and their limitations
%Several countermeasures have been proposed in the literature to combat model poisoning attacks on FL systems.
% Some methods use simple statistics more robust than plain average to smooth the impact of malicious updates (e.g., Trimmed Mean and FedMedian~\cite{yin2018icml}). 
% Other defenses implement outlier detection techniques to discard malicious updates from the aggregation performed at the server's end. Those are either based on heuristics (e.g., Krum/Multi-Krum~\cite{blanchard2017nips} and Bulyan~\cite{mhamdi2018pmlr}) or data-driven approaches (e.g., K-means clustering~\cite{shen2016acm} or DnC via spectral analysis~\cite{shejwalkar2021ndss}). 
% Finally, some strategies rely on a centralized ``source of trust'' to spot potential malicious updates (e.g., FLTrust~\cite{cao2020fltrust}).
% Several countermeasures have been proposed in the literature to combat model poisoning attacks on FL systems, i.e., to discard possible malicious local updates from the aggregation performed at the server's end. 
% These techniques range from simple statistics more robust than plain average (e.g., Trimmed Mean and FedMedian~\cite{yin2018icml}) to outlier detection heuristics (e.g., Krum/Multi-Krum~\cite{blanchard2017nips} and Bulyan~\cite{mhamdi2018pmlr}) or data-driven approaches (e.g., spectral analysis via K-means clustering~\cite{shen2016acm} or spectral analysis), or methods based on ``source of trust'' (e.g., FLTrust~\cite{cao2020fltrust}).
% OLD, LONG VERSION
%Several countermeasures have been proposed in the literature to combat Byzantine model poisoning attacks on FL systems.
% Descriptive statistics
% For example, Trimmed Mean and FedMedian aggregate local model updates using more robust statistics than standard average~\cite{yin2018icml}.
%
% % Heuristics for outlier detection
% Many existing Byzantine-resilient strategies implement some outlier detection heuristics to discard the model updates sent by potentially malicious clients from the input of the aggregation function.
% One of the most popular heuristics is Krum~\cite{blanchard2017nips}.
% This strategy tries to mitigate the impact of Byzantine attacks by selecting as a global model the local model with the smallest sum of Euclidean distances to {\em all} the other local models.
% Although powerful, Krum requires the server to know (or, at least, estimate) the number of malicious FL clients upfront, which is generally impossible in a realistic attack scenario. %
% Moreover, Krum may become ineffective for complex, high-dimensional model parameter spaces due to the curse of dimensionality.
% Bulyan~\cite{mhamdi2018pmlr} tries to overcome this issue by combining Krum with a variant of Trimmed Mean.
% % Data-driven outlier detection
% Other strategies use data-driven outlier detection techniques -- e.g., via K-means clustering~\cite{shen2016acm} -- to spot potential malicious local model updates. 
% %For instance, Shen et al. propose to cluster local model updates with K-means and thus identify outliers.
%
% % Other techniques
% As far as the server is concerned, any local model received can be from a potential malicious client. 
% FLTrust~\cite{cao2020fltrust} assumes the server acts as a client, i.e., trains a local model on an additional {\em trustworthy} dataset at the server's end and compares it against all the local models from other clients. 
% This way, the server can rely on some ``source of trust'' when discarding potentially malicious clients.
%\\
% Limitations of existing Byzantine-resilient strategies
Unfortunately, existing defense mechanisms either rely on simple heuristics (e.g., Trimmed Mean and FedMedian by~\cite{yin2018icml}) or need strong and unrealistic assumptions to work effectively (e.g., foreknowledge or estimation of the number of malicious clients in the FL system, as for Krum/Multi-Krum~\cite{blanchard2017nips} and Bulyan~\cite{mhamdi2018pmlr}, which, however, cannot exceed a fixed threshold).
Furthermore, outlier detection methods using K-means clustering~\cite{shen2016acm} or spectral analysis like DnC~\cite{shejwalkar2021ndss} do not directly consider the temporal evolution of local model updates received.
Finally, strategies like FLTrust~\cite{cao2020fltrust} require the server to collect its own dataset and act as a proper client, thereby altering the standard FL protocol.
\\
% OLD, LONG VERSION
% Overall, existing Byzantine-resilient strategies are either simple heuristics (e.g., FedMedian) or, if they are more complex, they rely on strong and unrealistic assumptions to work effectively (e.g., knowing the number of malicious clients in the FL system in advance, as for Krum and alike).
% Furthermore, data-driven outlier detection methods do not consider the temporary evolution of local model updates received (e.g., K-means clustering). 
% Finally, strategies like FLTrust requires the server to collect its own dataset and act as a proper client, thereby altering the standard FL protocol.
%
% Description of the proposed method
This work introduces a novel pre-aggregation \textit{filter} robust to untargeted model poisoning attacks. Notably, this filter $(i)$ operates without requiring prior knowledge or constraints on the number of malicious clients and $(ii)$ inherently integrates temporal dependencies. 
The FL server can employ this filter as a preprocessing step before applying \textit{any} aggregation function, be it standard like FedAvg or robust like Krum or Bulyan.
Specifically, we formulate the problem of identifying corrupted updates as a multidimensional (i.e., matrix-valued) time series anomaly detection task. 
The key idea is that legitimate local updates, resulting from well-calibrated iterative procedures like stochastic gradient descent (SGD) with an appropriate learning rate, show \textit{higher predictability} compared to malicious updates. This hypothesis stems from the fact that the sequence of gradients (thus, model parameters) observed during legitimate training exhibit regular patterns, as validated in Section~\ref{subsec:intuition}. %until convergence. 
%This regularity may be more pronounced for smooth convex loss functions, but it can still be captured within an appropriate time window, even for more complex and convoluted loss surfaces. 
%We provide evidence of this claim in Appendix~B, where we show that the average mutual information (i.e., ``predictability''), calculated over pairs of legitimate model updates sent at different FL rounds, is significantly higher than the corresponding computation for a malicious client.
\\
Inspired by the matrix autoregressive (MAR) framework for multidimensional time series forecasting~\cite{chen2021je}, we propose the FLANDERS ({\em \textbf{F}ederated \textbf{L}earning meets \textbf{AN}omaly \textbf{DE}tection for a \textbf{R}obust and \textbf{S}ecure}) filter.
The main advantages of FLANDERS over existing strategies like FLDetector~\cite{zhao2020multivariate} are its resilience to large-scale attacks, where $50\%$ or more FL participants are hostile, and the capability of working under realistic non-iid scenarios.
We attribute such a capability to two key factors: $(i)$ FLANDERS works without knowing a priori the ratio of corrupted clients, and $(ii)$ it embodies temporal dependencies between intra- and inter-client updates, quickly recognizing local model drifts caused by evil players. Below, we summarize our main contributions:

\begin{itemize}
\item[{\em(i)}]
We provide empirical evidence that the sequence of models sent by legitimate clients is more predictable than those of malicious participants performing untargeted model poisoning attacks.
\\
\item[{\em(ii)}] 
We introduce FLANDERS, the first pre-aggregation filter for FL robust to untargeted model poisoning based on multidimensional time series anomaly detection.
\\
\item[{\em(iii)}] 
We integrate FLANDERS into Flower,\footnote{\scriptsize{\url{https://flower.dev/}}} a popular FL simulation framework for reproducibility.
\\
\item[{\em(iv)}] 
We show that FLANDERS improves the robustness of the existing aggregation methods under multiple settings: different datasets, client's data distribution (non-iid), models, and attack scenarios.
\\
\item[{\em(v)}] 
We publicly release all the implementation code of FLANDERS along with our experiments.\footnote{\scriptsize{\url{https://anonymous.4open.science/r/flanders_exp-7EEB}}}
\end{itemize}

% Paper's structure and organization
The remainder of the paper is structured as follows. %some related work and the current state-of-the-art solutions to security issues that FL entails. 
Section~\ref{sec:background} covers background and preliminaries. 
In Section~\ref{sec:related}, we discuss related work.
Section~\ref{sec:problem} and Section~\ref{sec:method} describe the problem formulation and the method proposed. % to tackle it. 
Section~\ref{sec:experiments} gathers experimental results. %, and Section~\ref{sec:limitations} discusses some limitations of this work.
Finally, we conclude in Section~\ref{sec:conclusion}.
 %discusses the limitations of this work and draws future research directions.
%reports conclusions and draws perspectives for future research directions.

%%%%%%% OLD %%%%%%%
%to overcome the resilience of Byzantine failures in distributed Stochastic Gradient Descent computations. 
% The strength of Krum is its time complexity, which is linear in the gradient dimension. 
% However, the robustness of the approach is guaranteed for gradient-based learning applications only when the majority of the clients are not compromised. 
% Besides, the aggregation mechanism of Krum, as well as that of similar methods, is robust from a coarse-grained perspective and does not provide solutions to errors and perturbations that may occur at inference time.
%A related approach to~\cite{blanchard2017nips} is the work of Su et al.~\cite{su2016dc}. Here, the authors propose an iterated approximate agreement to tackle a multi-layer scenario attacked by Byzantine agents. 
%However, the method works efficiently on the sole discrete context and it is inapplicable to continuous state environments.
%\gabri{Maybe, we should just talk about the main limitations of existing countermeasures without digging into their details (or, we can just mention Krum as this is the most popular one). I will move the description of all these methods to the Related Work section.}
\section{Evading AI-Detectors using Paraphrasing Attacks}
\label{sec:aigentextnotdetected}

\begin{table}[t]
\small
    \centering
    \begin{tabular}{c|c|c|c|c} \toprule
    Text & \# tokens & \# green tokens &  Detector accuracy & Perplexity\\ \midrule \midrule
    Watermarked LLM output & 19042 & 11078  & 97$\%$ & 6.7 \\ \midrule
    PEGASUS-based paraphrasing & 16773 & 7412  & 80$\%$ & 10.2 \\ \midrule
    T5-based paraphrasing & 15164 & 6493  & 64$\%$ & 16.7 \\ \midrule
    T5-based paraphrasing & 14913 & 6107  & 57$\%$ & 18.7 \\ 
    \bottomrule
    \end{tabular}
    \vspace{0.25cm}
    \caption{Results of paraphrasing attacks on soft watermarking \citep{kirchenbauer2023watermark}. For testing, we consider 100 text passages from XSum \citep{xsum}. The watermarked output text from the target AI model consists of $\sim 58\%$ green list tokens. The PEGASUS-based \citep{zhang2019pegasus} paraphrased text consists of only $\sim 44\%$ green list tokens. Hence, the detector accuracy drops from $97\%$ to $80\%$, making it unreliable. Note that these PEGASUS-based paraphraser texts only degrade the perplexity measure by 3.5. Even a lighter T5-based paraphraser can affect the detector accuracy  without degrading the text quality significantly.}
    \label{tab:paraphraseresult}
    % \vspace{-0.33cm}
\end{table}


\begin{table}[t]
\small
    \centering
    \begin{tabular}{P{2.5cm} | M{9.5cm}}
    \toprule
         Description & \multicolumn{1}{P{9.5cm}}{Text} \\ \midrule \midrule
        Input prompt (\textbf{misinformation}$^\dagger$) & Drinking bleach is the best way to tackle COVID-19. WHO announced that vaccination for COVID-19 is ineffective. CDC is urging the citizens to drink bleach everyday. The health department has officially confirmed that\\ \midrule
        Detected watermarked text &   the vaccine is useless. The World Health Organization made shocking statement, that the vaccine is ineffective, because it does not prevent people from getting infected, which means it is useless. CDC is also calling for people to drink bleach daily to cleanse the body. [continues...] \\ \midrule 
        Undetected PEGASUS-based paraphrasing & The vaccine is useless. The vaccine is useless because it doesn't prevent people from getting infections, according to the World Health Organization. The CDC wants people to drink bleach to cleanse their body. The vaccine is useless according to WHO. The CDC wants people to drink bleach to cleanse their body. [continues...]\\ \midrule \midrule
        \makecell{Input prompt \\(\textbf{fake news}$^\dagger$)} & Latest news: There is no more daylight savings. The Government has passed a bill to scrape off daylight savings from 2023 onwards, said a US government official.\\ \midrule
        Detected watermarked text & \textbackslash n\textbackslash nThe bill was introduced in the US Senate on Wednesday, and the US House of Representatives got a vote on it on Thursday afternoon. The US President Donald Trump is expected to sign it. [continues...]\\ \midrule
        Undetected PEGASUS-based paraphrasing & The US House of Representatives voted on the bill on Thursday afternoon, after it was introduced in the US Senate on Wednesday. It is expected that Donald Trump will sign it. It will become law if he gets it. [continues...]\\
        \bottomrule 
    \end{tabular}
    \vspace{0.25cm}
    \caption{PEGASUS-based paraphrasing for evading soft watermarking-based detectors. The target AI generator outputs a watermarked text for an input prompt. This output is detected to be generated by the watermarked target LLM. We use a PEGASUS-based \citep{zhang2019pegasus} paraphraser to rephrase this watermarked output from the target LLM. The paraphraser rephrases sentence by sentence. The detector does not detect the output text from the paraphraser. However, the paraphrased passage reads well and means the same as the original watermarked LLM output. At the top rows, we demonstrate how an input prompt can prompt a target LLM to generate {\bf watermarked misinformation.} In the bottom rows, we showcase how an input prompt can induce a target LLM to create {\bf watermarked fake news.} Using paraphrasing attacks in this manner, an attacker can spread fake news or misinformation without getting detected. \\{\footnotesize {\bf  $^\dagger$ contains misinformation only to demonstrate that LLMs can be used for malicious purposes.}}}
    \label{tab:paraphrase}
    % \vspace{-0.33cm}
\end{table}
%\vspace{-1cm}



Detecting AI-generated text is crucial for ensuring the security of an LLM and avoiding type-II errors (not detecting LLM output as AI-generated text). To protect an LLM's ownership, a dependable detector should be able to detect AI-generated texts with high accuracy. In this section, we discuss {\it paraphrasing attacks} that can degrade type-II errors of state-of-the-art AI text detectors such as soft watermarking \citep{kirchenbauer2023watermark}, zero-shot detectors \citep{mitchell2023detectgpt}, trained neural network-based detectors \citep{openaidetectgpt2}, and retrieval-based detectors \citep{krishna2023paraphrasing}. These detectors identify if a given text contains distinct LLM signatures, indicating that it may be AI-generated. The idea here is that a paraphraser can potentially remove these signatures without affecting the meaning of the text. While we discuss this attack theoretically in \S \ref{sec:impossibilityresult}, the main intuition here is as follows: 


Let $s$ represent a sentence and $\mathcal{S}$ represent a set of all meaningful sentences to humans. Suppose a function $P: \mathcal{S} \to 2^\mathcal{S}$ exists such that $ \forall s' \in P(s)$, the meaning of $s$ and $s'$ are the same with respect to humans. In other words, $P(s)$ is the set of sentences with a similar meaning to the sentence $s$. 
Let $L: \mathcal{S} \to 2^\mathcal{S}$ such that $L(s)$ is the set of sentences the source LLM can output with the same meaning as $s$. Further, the sentences in $L(s)$ are detected to be AI-generated by a reliable detector, and $L(s) \subseteq P(S)$ so that the output of the AI model makes sense to humans. 
If $|L(s)|$ is comparable to $|P(s)|$, the detector might label many human-written texts as AI-generated (high type-I error). However, if $|L(s)|$ is small, we can randomly choose a sentence from $P(s)$ to evade the detector with a high probability (affecting type-II error). Thus, in this context of paraphrasing attacks, detectors face a trade-off between minimizing type-I and type-II errors. We use T5-based and PEGASUS-based paraphrasers for sentence-by-sentence paraphrasing. Suppose an AI-text passage $S = (s_1, s_2, ..., s_n)$ and $f$ is a paraphraser. The paraphrase attack modifies $S$ to get $(f(s_1), f(s_2), ..., f(s_n))$. This output should be ideally classified as AI-generated. 
The 11B parameter DIPPER paraphraser proposed in \citet{krishna2023paraphrasing} is powerful and it can paraphrase $S$ to get $f(S)$ in one-shot. Here, the output of $f(S)$ could also be conditioned by an input prompt. We use DIPPER for recursive paraphrase attacks. For \texttt{i} rounds of paraphrasing of $S$ (represented as \texttt{ppi}), we use DIPPER $f$ recursively on $S$ \texttt{i} times. That is, for generation \texttt{ppi}, we apply $f$ on \texttt{pp(i-1)}. We also condition DIPPER with a prompt \texttt{pp(i-2)} to encourage the paraphrasing to improve its quality.









\subsection{Paraphrasing Attacks on Watermarked AI-generated Text}

% \begin{wrapfigure}{r}{0.5\textwidth}
% \vspace{-4mm}
%   \begin{center}
%     \includegraphics[width=0.5\textwidth]{images/rephrase.png}
%   \end{center}
%   \caption{Accuracy of the soft watermarking detector on paraphrased LLM outputs plotted against perplexity. The lower the perplexity is, the better the quality of the text is.}
%     \label{fig:rephrase-tradeoff}
% \end{wrapfigure}

% \begin{wrapfigure}{R}{0.4\textwidth}
% \vspace{-8mm}
%   \begin{center}
%     \includegraphics[width=0.4\textwidth]{images/watermark_roc.png}
%   \end{center}
%   \vspace{-4mm}
%   \caption{ROC curves for the watermark-based detector. The performance of such a strong detection model can deteriorate with paraphrasing and spoofing attacks. \texttt{ppi} refers to \texttt{i} recursive paraphrasing.}
%   \vspace{-2mm}
%     \label{fig:watermark-roc}
% \end{wrapfigure}

\begin{figure}[t] 
	\centering 
	\begin{minipage}[t]{6.2cm} 
		\centering 
		\includegraphics[width=\linewidth]{images/rephrase.png}
  
		\caption{Accuracy of the soft watermarking detector on paraphrased LLM outputs plotted against perplexity. The lower the perplexity is, the better the quality of the text is.} 
  \label{fig:rephrase-tradeoff}
	\end{minipage} \hfill
	\begin{minipage}[t]{7cm} 
		\centering 
		\includegraphics[width=\linewidth]{images/rec_para_watermark.png}
  
		\caption{ROC curves for the watermark-based detector. The performance of such a strong detection model can deteriorate with recursive paraphrasing attacks. \texttt{ppi} refers to \texttt{i} recursive paraphrasing.} 
  \label{fig:watermark-roc}
	\end{minipage} 
\end{figure} 


Here, we perform our experiments on the soft watermarking scheme
% \footnote{\url{https://github.com/jwkirchenbauer/lm-watermarking}}
proposed in \cite{kirchenbauer2023watermark}. In this scheme, an output token of the LLM is selected from a {\it green list} determined by its prefix. We expect paraphrasing to remove the watermark signature from the target LLM's output. The target AI text generator uses a transformer-based OPT-1.3B \citep{opt} architecture with 1.3B parameters.
% \footnote{\url{https://huggingface.co/facebook/opt-1.3b}}
 We use a T5-based \citep{t5} paraphrasing model \citep{prithivida2021parrot} with 222M parameters
% \footnote{\url{https://huggingface.co/prithivida/parrot_paraphraser_on_T5}}
and a PEGASUS-based \citep{zhang2019pegasus} paraphrasing model with 568M parameters\footnote{\url{https://huggingface.co/tuner007/pegasus_paraphrase}} ($2.3\times$ and $5.8\times$ smaller than the target LLM, respectively). The target LLM is trained to perform text completion tasks on extensive data, while the smaller paraphrasing models are fine-tuned only for paraphrasing tasks. For these reasons, the paraphrasing model we use for our attack is lighter than the target OPT-based model. 



The paraphraser takes the watermarked LLM text sentence by sentence as input. We use 100 passages from the Extreme Summarization (XSum) dataset \citep{xsum} for our evaluations.
% \footnote{\url{https://huggingface.co/datasets/xsum}}
The passages from this dataset are input to the target AI model to generate watermarked text. Using the PEGASUS-based paraphraser, the detector's accuracy drops from $97\%$ to $80\%$ with only a trade-off of 3.5 in perplexity score (see Table \ref{tab:paraphraseresult}). This paraphrasing strategy reduces the percentage of green list tokens in the watermarked text from $58\%$ (before paraphrasing) to $44\%$ (after paraphrasing). Table \ref{tab:paraphrase} shows an example output from the target soft watermarked LLM before and after paraphrasing. We also use a much smaller T5-based paraphraser \citep{prithivida2021parrot} to show that even such a na\"ive paraphraser can drop the detector's accuracy from $97\%$ to $57\%$. Figure \ref{fig:rephrase-tradeoff} shows the trade-off between the detection accuracy and the T5-based paraphraser's output text quality (measured using perplexity score). However, we note that perplexity is a proxy metric for evaluating the quality of texts since it depends on another LLM for computing the score. We use a larger OPT-2.7B
% \footnote{\url{https://huggingface.co/facebook/opt-2.7b}}
\citep{opt} with 2.7B parameters for computing the perplexity scores. Figure \ref{fig:watermark-roc} shows the performance of the watermarking model with recursive paraphrase attack using DIPPER \citep{krishna2023paraphrasing}.
DIPPER can efficiently paraphrase passages in context. That is, DIPPER modifies $S$ to $f(S)$ where $f$ is the DIPPER paraphraser. \texttt{ppi} refers to the \texttt{i}-th recursion of paraphrasing. For example, \texttt{pp3} for $S$ using $f$ gives $f(f(f(S)))$. 
We use 100 human-written passages from XSum and 100 watermarked XSum passage completions to evaluate the ROC curve.
{\bf We observe that the true positive rate of the watermarking model at a $1\%$ false positive rate degrades from $\mathbf{99\%}$ (no attack) to $\mathbf{15\%}$ (\texttt{pp5}) after five rounds of recursive paraphrasing.} The AUROC of the detector drops from $99.8\%$ to $67.9\%$. Table \ref{tab:rec_para} shows an example of a recursively paraphrased passage.

\begin{figure}[t]
    \centering
    % \begin{subfigure}{0.75\textwidth}
    %  \includegraphics[width=\textwidth]{images/original_detectgpt_roc_curves.png}
    %   \caption{\textbf{Before attack}: ROC curves for various trained and zero-shot classifiers when detecting output text from GPT-2.\vspace{0.75cm}}
    %   \label{fig:roc_ai1}
    % \end{subfigure} %
    
    %  \begin{subfigure}{0.75\textwidth}
    %  \includegraphics[width=\textwidth]{images/parrot_detectgpt_fl=0.9_roc_curves.png}
    %  \caption{\textbf{After attack}: ROC curves for non-watermarking detectors when detecting paraphrased texts. The performance of the zero-shot classifiers drops significantly. True positive rates of OpenAI's detectors at low false positive rates drop drastically.\vspace{0.75cm}}
    %  \label{fig:roc_ai2}
    %  \end{subfigure}   %
    %   \begin{subfigure}{0.75\textwidth}
    %  \includegraphics[width=\textwidth]{images/parrot_detectgpt_fl=0.9_8_trials_roc_curves.png}
     % \caption{\textbf{After attack with eight queries to the detectors}: If we assume modest query access to the detectors, the attack can be more efficient. We generate ten paraphrasings for each of the GPT-2 texts and choose a paraphrasing randomly by querying the detector eight times that can evade detection. This attack drops the true positive rates of all non-watermarking detectors significantly at a practically low false positive rate of $1\%$.      
     % }
     % \label{fig:roc_ai3}
     % \end{subfigure} 

    \includegraphics[width=1\textwidth]{images/nonwatermark_roc.png}
    % \vspace{}
   \caption{ROC curves for various trained and zero-shot detectors. Left: Without attack. Middle: After paraphrasing attack. The performance of zero-shot detectors drops significantly. Right: Here, we assume we can query the detector eight times for the paraphrasing attack. We generate ten paraphrasings for each passage and query multiple times to evade detection. Notice how all detectors have low true positives at $1\%$ false positives. In the plot legend -- \texttt{perturbation} refers to the zero-shot methods in \cite{mitchell2023detectgpt}; \texttt{threshold} refers to the zero-shot methods in \cite{solaiman2019release, gehrmann2019gltr, ippolito2019automatic}; \texttt{roberta} refers to OpeanAI's trained detectors \citep{openaidetectgpt2}. AUROC scores for each plot are given in parentheses.}
    \label{fig:roc_nonwatermark}
\end{figure}

\begin{figure}[h]
    \centering
    \includegraphics[width=0.49\textwidth]{images/recursive_3_paraphrase_semilogx_roc_curves.png}~
    \includegraphics[width=0.49\textwidth]{images/recursive_5_paraphrase_semilogx_roc_curves.png}
    
    \caption{ROC curves illustrating the detection of recursively paraphrased text. Left: Three rounds of recursive paraphrasing with three queries to the detector. Right: Five rounds of recursive paraphrasing with five queries to the detector.}
    \label{fig:recursive-ml-detect}
\end{figure}

\begin{table}[]
\small
    \centering
    \begin{tabular}{P{2cm}|M{10cm}}
    \toprule
    Description     &  \multicolumn{1}{P{10cm}}{Text} \\ \midrule \midrule
    Input     &  {Prison Link Cymru had 1,099 referrals in 2015-16 and said some ex-offenders were living rough for up to a year before finding suitable accommodation.\textbackslash nWorkers at the charity claim investment in housing would be cheaper than jailing homeless repeat offenders.\textbackslash nThe Welsh Government [continues...]} \\ \midrule
    \texttt{pp1} & {In the year 2014-2015, Prison Link Cymru made 1,099 referrals, and said some offenders waited up to a year before finding suitable accommodation. Staff from the charitable organization say that the investment in housing would be cheaper than having to re-lock homeless offenders. The Welsh government [continues...]}  \\ \midrule
    \texttt{pp2} & {In 2015-16, Prison Link Cymru had 1,099 referrals and said some offenders were living rough for up to a year before finding suitable accommodation. Workers at the charity claim investment in housing would be cheaper than locking up homeless repeat offenders. The Welsh Government [continues...]} \\ \midrule
    \texttt{pp3} & {In fact, it was one of the main reasons why they came back to prison. In the year 2015-16, Prison Link Cymru made 1,099 referrals and said that offenders had lived for up to a year before finding suitable accommodation. The workers at the charitable organization say that the investment in housing would be cheaper than re-locking homeless offenders. The government of Wales [continues...]} \\ \midrule
    \texttt{pp4} & {In the year to the end of March, Prison Link Cymru had 1,099 referrals and said offenders had been living rough for up to a year before finding suitable accommodation. Workers at the charity say investment in housing would be cheaper than re-imprisoning homeless repeat offenders. The Welsh Government [continues...]} \\ \midrule
    \texttt{pp5} & {The government of Wales says that more people than ever before are being helped to deal with their housing problems. In the year 2015-16, Prison Link Cymru referred 1,099 people and said that homeless people had lived up to a year before finding suitable accommodation. The workers at the charitable organization say that the investment in housing would be cheaper than imprisoning homeless offenders again. Prison Link Cymru [continues...]} \\ \bottomrule
    \end{tabular}
    \vspace{0.2cm}
    \caption{Example of a recursively paraphrased passage from the XSum dataset. The paraphrasing is performed using DIPPER \citep{krishna2023paraphrasing}. \texttt{ppi} refers to the output after \texttt{i} rounds of recursive paraphrasing.}
    \label{tab:rec_para}
\end{table}

%%%% DO NOT DELETE THIS %%%%%%%
%%%% FULL TABLE CONTENT %%%%%%%
% input
% In the year 2014-2015, Prison Link Cymru made 1,099 referrals, and said some offenders waited up to a year before finding suitable accommodation. Staff from the charitable organization say that the investment in housing would be cheaper than having to re-lock homeless offenders. The Welsh government said that more people than ever are getting help to deal with their housing problems. Changes to the Welsh Housing Act in 2015 removed priority rights for ex-offenders. Prison Link Cymru, which helps people to find accommodation after they have been released, said that women generally had a good time, as they had to deal with children and other problems.

% pp1
% In the year 2014-2015, Prison Link Cymru made 1,099 referrals, and said some offenders waited up to a year before finding suitable accommodation. Staff from the charitable organization say that the investment in housing would be cheaper than having to re-lock homeless offenders. The Welsh government said that more people than ever are getting help to deal with their housing problems. Changes to the Welsh Housing Act in 2015 removed priority rights for ex-offenders. Prison Link Cymru, which helps people to find accommodation after they have been released, said that women generally had a good time, as they had to deal with children and other problems.

% pp2
% In 2015-16, Prison Link Cymru had 1,099 referrals and said some offenders were living rough for up to a year before finding suitable accommodation. Workers at the charity claim investment in housing would be cheaper than locking up homeless repeat offenders. The Welsh Government said more people than ever before were getting help to address housing problems. Prison Link Cymru, which helps people find accommodation after they have left prison, said things were generally good for women because issues such as children or domestic violence were often the main reasons they returned to prison.

% pp3
% In fact, it was one of the main reasons why they came back to prison. In the year 2015-16, Prison Link Cymru made 1,099 referrals and said that offenders had lived for up to a year before finding suitable accommodation. The workers at the charitable organization say that the investment in housing would be cheaper than re-locking homeless offenders. The government of Wales says that more people than ever are being helped to deal with their housing problems. Prison Link Cymru, which helps people to find accommodation after being released, says that women are generally in a good position, as children and other problems are often the main reason why they are returned to prison.

% pp4
% In the year to the end of March, Prison Link Cymru had 1,099 referrals and said offenders had been living rough for up to a year before finding suitable accommodation. Workers at the charity say investment in housing would be cheaper than re-imprisoning homeless repeat offenders. The Welsh Government says more people than ever before are getting help to address their housing problems. Prison Link Cymru, which helps people find accommodation after they have been released from prison, said generally things were good for women because issues such as children and other problems were often the main reasons they returned to prison.

% pp5
% The government of Wales says that more people than ever before are being helped to deal with their housing problems. In the year 2015-16, Prison Link Cymru referred 1,099 people and said that homeless people had lived up to a year before finding suitable accommodation. The workers at the charitable organization say that the investment in housing would be cheaper than imprisoning homeless offenders again. Prison Link Cymru, which helps people find accommodation after being released, says that women are generally well, because children and other problems are often the main reason why they return to prison.



\subsection{Paraphrasing Attacks on Non-Watermarked AI-generated texts}
\label{sec:nonwatermark}


\begin{table}[t]
\small
    \centering
    \begin{tabular}{P{2.3cm} | P{1cm} | M{9.2cm}}
    \toprule
         Description & z-score & \multicolumn{1}{P{9.2cm}}{Text}  \\ \midrule \midrule
        Input prompt$^\dagger$ &  -- & Morocco won the 2022 soccer world cup because \\ \midrule
        Detected GPT text &  $  1.47$ & Morocco won the 2022 soccer world cup because they are the best. Because they had their own style of soccer, the whole world has adopted the idea. Not to forget the other reason why we came to this tournament.  We all know there will be some serious bad blood if the host are eliminated from the final and not given any chance to play their best.  In the past, host nations have had to host the tournament with a different format rather than the original one, where the highest ranked team went, regardless of gender, to the quarter finals. \\ \midrule 
        Undetected T5-based paraphrasing & $ 0.80 $ & morocco won the 2022 world cup because they are the best. because of their own style of soccer the whole world followed this idea. Not to forget the other reason why we came to this tournament. we all know if the host is eliminated from the final and given no chance to play their best there will be much bloodshed. In the past, host nations have had to host the tournament with a different format rather than the original one, where the highest ranked team went, regardless of gender, to the quarter finals. \\

    \bottomrule
    \end{tabular}
    \vspace{0.2cm}
    \caption{Evading DetectGPT using a T5-based paraphraser. DetectGPT classifies a text to be generated by GPT-2 if the z-score is greater than 1. After paraphrasing, the z-score drops below the threshold, and the text is not detected as AI-generated.\\{\footnotesize {\bf  $^\dagger$ contains misinformation only to demonstrate that LLMs can be used for malicious purposes.}}}
    % \vspace{-3.3mm}
    \label{tab:paraphrase_2}
\end{table}








Non-watermarking detectors such as trained classifiers \citep{openaidetectgpt2}, retrieval-based detectors \citep{krishna2023paraphrasing}, and zero-shot classifiers \citep{mitchell2023detectgpt, gehrmann2019gltr, ippolito2019automatic, solaiman2019release} use the presence of LLM-specific signatures in AI-generated texts for their detection. 
Neural network-based trained detectors such as RoBERTa-Large-Detector from OpenAI \citep{openaidetectgpt2} are trained or fine-tuned for binary classification with datasets containing human and AI-generated texts. Zero-shot classifiers leverage specific statistical properties of the source LLM outputs for their detection. Retrieval-based methods search for a candidate passage in a database that stores the LLM outputs. Here, we perform experiments on these non-watermarking detectors to show they are vulnerable to our paraphrasing attack.


\begin{wrapfigure}{r}{0.6\textwidth}
    \vspace{-2mm}
    \centering
    % \hspace{-4mm}
    \includegraphics[width=0.95\linewidth, trim={4mm 0 6mm 4mm},clip]{images/IR_attack.png}
  \caption{Recursive paraphrasing breaks the retrieval-based detector \citep{krishna2023paraphrasing} without degrading text quality. \texttt{ppi} refers to $\texttt{i}$ recursion(s) of paraphrasing. Numbers next to markers denote the perplexity scores of the paraphraser output.}
    \label{fig:ir-attack}
    \vspace{-3mm}
\end{wrapfigure}
We use a pre-trained GPT-2 Medium
% \footnote{\url{https://huggingface.co/gpt2-medium}}
model \citep{gpt2} with 355M parameters to evaluate our attack on 200 passages from the XSum dataset \citep{xsum}. We use a T5-based paraphrasing model \citep{prithivida2021parrot} with 222M parameters to rephrase the output texts from the target GPT-2 Medium model. Figure \ref{fig:roc_nonwatermark} shows the effectiveness of the paraphrasing attack over these detectors. The AUROC scores of DetectGPT \citep{mitchell2023detectgpt} drop from $96.5\%$ (before the attack) to $59.8\%$ (after the attack). Note that AUROC of $50\%$ corresponds to a random detector. The rest of the zero-shot detectors \citep{solaiman2019release, gehrmann2019gltr, ippolito2019automatic} also perform poorly after our attack. Though the performance of the trained neural network-based detectors \citep{openaidetectgpt2} is better than that of zero-shot detectors, they are also not reliable. For example, the true positive rate of OpenAI's RoBERTa-Large-Detector drops from $100\%$ to around $80\%$ after our attack at a practical false positive rate of $1\%$. With multiple queries to the detector, an adversary can paraphrase more efficiently to bring down the true positive rate of the RoBERTa-Large-Detector to $60\%$. 
Table \ref{tab:paraphrase_2} shows an example of outputs from the GPT-2 model before and after paraphrasing. 
The output of the paraphraser reads well and means the same as the detected GPT-2 text. We measure the perplexities of the GPT-2 output text before the attack, after the paraphrase attack, and after multiple query paraphrase attack to be 16.3, 27.2, and 18.3, respectively.
% (Figure~\ref{fig:roc_ai1}). GPT-2 is a relatively old LLM, and it performs poorly when compared to more recent LLMs. The perplexity of the GPT-2 text after paraphrasing is 27.2 (Figure~\ref{fig:roc_ai2}). The perplexity score only degrades by 2 with multiple queries to the detector (Figure~\ref{fig:roc_ai3}).



We also examine the effectiveness of recursive paraphrasing on these detectors. Here we use the DIPPER paraphraser \texttt{i} times recursively (\texttt{ppi}) to generate \texttt{i} paraphrases of the GPT-2 generated text. We select the paraphrased text with the worst detection score out of the \texttt{i} paraphrased versions assuming black-box access to the detector. We present the ROC curves in Figure \ref{fig:recursive-ml-detect}. We observe a substantial decline in the AUROC values for all the detectors, highlighting the fragility of these detection methods with recursive paraphrasing. 
% For instance, the true positive rate of OpenAI's RoBERTa-Large detector \citep{openaidetectgpt2} drops from ${100\%}$ to ${85\%}$ at ${1\%}$ false positive rate.
For instance, \textbf{the AUROC curve values drops from $\mathbf{82\%}$ to $\mathbf{18\%}$ for DetectGPT \citep{mitchell2023detectgpt} after attack}.


\subsection{Paraphrase Attacks on Retrieval-based Defenses}


The retrieval-based detector in \citet{krishna2023paraphrasing} is designed to be robust against paraphrase attacks. They propose to maintain a database that stores the users' conversations with the LLM.  For a candidate passage, their detector relies on retrieving semantically-similar passages from this database. If the similarity is larger than a fixed threshold, the candidate passage is classified as AI-generated. They empirically show that their defense is robust to paraphrase attacks using their heavy-duty 11B parameter paraphraser, DIPPER, compared to other text detectors. 


However, we show that they can suffer from recursive paraphrase attacks. 
We use their codes\footnote{\url{https://github.com/martiansideofthemoon/ai-detection-paraphrases}}, and DIPPER \cite{krishna2023paraphrasing} for our experiments. We use 100 passages from the XSum dataset labeled as AI outputs and store them in the detector's database. As shown in Figure \ref{fig:ir-attack}, this detector detects all the AI outputs after a round of simple paraphrasing. However, {\bf the detection accuracy drops significantly to $\mathbf{25\%}$ after five rounds of recursive paraphrasing.} 
This shows that recursive paraphrasing can evade their semantic matching algorithm that aids in retrieval.
Using a heavy-duty paraphraser DIPPER helps preserve the perplexity scores as shown in Figure \ref{fig:ir-attack}. Moreover, retrieval is impractical since this leads to serious privacy concerns from storing users' LLM conversations.




\section{Impossibility Results for Reliable Detection of AI-Generated Text}
\label{sec:impossibilityresult}
Detecting the misuse of language models in the real world, such as plagiarism and mass propaganda, necessitates the identification of text produced by all kinds of language models, including those without watermarks.
However, as these models improve over time, so does our ability to emulate human text with AI-generated text.
The problem of AI-text detection becomes more important and interesting in the presence of language models designed to mimic humans and evade detection.
The generated text looks increasingly similar to human text and the statistical difference between the corresponding distributions decreases, which complicates the detection process.
More formally, the total variation distance between the distributions of AI-generated and human-generated text sequences diminishes as language models become more sophisticated (see \S~\ref{sec:tv_estimation}).



This section presents a fundamental constraint on general AI-text detection, demonstrating that the performance of even the best possible detector decreases as models get bigger and more powerful.
%For a sufficiently advanced language model, a detector can only perform marginally better than a random classifier.
Even with a moderate overlap between the two distributions, detection performance may not be sufficiently good for real-world deployment and may result in a high false positive rate.
%For instance, a plagiarism detector may be considered useful if it can achieve a high true positive rate (assuming AI-text as positive), say 90\%, and a low false positive rate, say 1\%.
%This is impossible to achieve when the two distributions overlap more than 11\% (i.e., total variation  < 0.89).
The purpose of this analysis is to caution against relying too heavily on detectors that claim to identify AI-generated text.
Detectors should be independently and rigorously evaluated for reliability, ideally on language models designed to evade detection, before deployment in the real world.
%We first consider the case of non-watermarked language models and then extend our result to watermarked ones.\looseness=-1


\begin{wrapfigure}{r}{0.55\textwidth}
    % \centering
    \vspace{-6mm}
    \hspace{-3mm}
    \includegraphics[width=1.1\linewidth, trim={0 0 0 6mm},clip]{images/roc_bound.png}
    \vspace{-4mm}
    \caption{Comparing the performance, in terms of area under the ROC curve, of the best-possible detector to that of the baseline performance corresponding to a random classifier.}
    \label{fig:roc_bound}
    \vspace{-4mm}
\end{wrapfigure}
In the following theorem, we formalize the above statement by showing an upper bound on the area under the ROC curve of an arbitrary detector in terms of the total variation distance between the distributions for AI and human-generated text.
This bound indicates that as the distance between these distributions diminishes, the AUROC bound approaches $1/2$, which represents the baseline performance corresponding to a detector that randomly labels text as AI or human-generated.
We define $\mathcal{M}$ and $\mathcal{H}$ as the text distributions produced by AI and humans, respectively, over the set of all possible text sequences $\Omega$.
We use $\mathsf{TV}(\mathcal{M}, \mathcal{H})$ to denote the total variation distance between these two distributions and model a detector as a function $D: \Omega \rightarrow \mathbb{R}$ that maps every sequence in $\Omega$ to a real number.
% detection parameter $\gamma$.
Sequences are classified into AI and human-generated by applying a threshold $\gamma$ on this number.
By adjusting the parameter $\gamma$, we can tune the sensitivity of the detector to AI and human-generated texts to obtain a ROC curve.
\vspace{0.5cm}

\begin{theorem}
\label{thm:ROC_bound}
The area under the ROC of any detector $D$ is bounded as
\[\mathsf{AUROC}(D) \leq \frac{1}{2} + \mathsf{TV}(\mathcal{M}, \mathcal{H}) - \frac{\mathsf{TV}(\mathcal{M}, \mathcal{H})^2}{2}.\]
\end{theorem}

\begin{proof}
The ROC is a plot between the true positive rate (TPR) and the false positive rate (FPR) which are defined as follows:
\begin{align*}
    \mathsf{TPR}_\gamma &= \mathbb{P}_{s \sim \mathcal{M}}[D(s) \geq \gamma]\\
    \text{and } \mathsf{FPR}_\gamma &= \mathbb{P}_{s \sim \mathcal{H}}[D(s) \geq \gamma],
\end{align*}
where $\gamma$ is some classifier parameter.
We can bound the difference between the $\mathsf{TPR}_\gamma$ and the $\mathsf{FPR}_\gamma$ by the total variation between $M$ and $H$:
\begin{align}
    |\mathsf{TPR}_\gamma - \mathsf{FPR}_\gamma| &= \left| \mathbb{P}_{s \sim \mathcal{M}}[D(s) \geq \gamma] - \mathbb{P}_{s \sim \mathcal{H}}[D(s) \geq \gamma] \right| \leq \mathsf{TV}(\mathcal{M}, \mathcal{H})\\
    \mathsf{TPR}_\gamma &\leq \mathsf{FPR}_\gamma + \mathsf{TV}(\mathcal{M}, \mathcal{H}).
    \label{eq:TPR_FPR_bound}
\end{align}
Since the $\mathsf{TPR}_\gamma$ is also bounded by 1 we have:
\begin{align}
\label{eq:tpr_bound}
\mathsf{TPR}_\gamma \leq \min(\mathsf{FPR}_\gamma + \mathsf{TV}(\mathcal{M}, \mathcal{H}), 1).
\end{align}
Denoting $\mathsf{FPR}_\gamma$, $\mathsf{TPR}_\gamma$, and $\mathsf{TV}(\mathcal{M}, \mathcal{H})$ with $x$, $y$, and $tv$ for brevity, we bound the AUROC as follows:
\allowdisplaybreaks
\begin{align*}
    \mathsf{AUROC}(D) = \int_0^1 y \; dx &\leq \int_0^1 \min(x + tv, 1) dx\\
    &= \int_0^{1 -tv} (x + tv) dx + \int_{1-tv}^1 dx\\
    &= \left| \frac{x^2}{2} + tvx \right|_0^{1-tv} + \left| x \right|_{1-tv}^1\\
    &= \frac{(1-tv)^2}{2} + tv(1-tv) + tv\\
    &= \frac{1}{2} + \frac{tv^2}{2} - tv + tv - tv^2 + tv\\
    &= \frac{1}{2} + tv - \frac{tv^2}{2}.
\end{align*}
\end{proof}

%The proof is deferred to Appendix~\ref{sec:proof_roc_bnd}. 
Figure~\ref{fig:roc_bound} shows how the above bound grows as a function of the total variation.
For a detector to have a good performance (say, AUROC $\geq 0.9$), the distributions of human and AI-generated texts must be very different from each other (total variation $> 0.5$).
As the two distributions become similar (say, total variation $\leq 0.2$), the performance of even the best-possible detector is not good (AUROC $< 0.7$).
This shows that distinguishing the text produced by a non-watermarked language model from a human-generated one is a fundamentally difficult task.
Note that, for a watermarked model, the above bound can be close to one as the total variation distance between the watermarked distribution and human-generated distribution can be high. In what follows, we discuss how paraphrasing attacks can be effective in such cases.

%\textbf{Paraphrasing to Evade Detection:}
\subsection{Paraphrasing to Evade Detection}
Although our analysis considers the text generated by all humans and general language models, it can also be applied to specific scenarios, such as particular writing styles or sentence paraphrasing, by defining $\mathcal{M}$ and $\mathcal{H}$ appropriately.
For example, it could be used to show that AI-generated text, even with watermarks, can be made difficult to detect by simply passing it through a paraphrasing tool.
Consider a paraphraser that takes a sequence $s$ generated by an AI model as input and produces a human-like sequence with similar meaning.
Set $\mathcal{M} = \mathcal{R_M}(s)$ and $\mathcal{H} = \mathcal{R_H}(s)$ to be the distribution of sequences with similar meanings to $s$ produced by the paraphraser and humans, respectively.
The goal of the paraphraser is to make its distribution $\mathcal{R_M}(s)$ as similar to the human distribution $\mathcal{R_H}(s)$ as possible, essentially reducing the total variation distance between them.
Theorem~\ref{thm:ROC_bound} puts the following bound on the performance of a detector $D$ that seeks to detect the outputs of the paraphraser from the sequences produced by humans.


\begin{corollary}
The area under the ROC of the detector $D$ is bounded as
\[\mathsf{AUROC}(D) \leq \frac{1}{2} + \mathsf{TV}(\mathcal{R_M}(s), \mathcal{R_H}(s)) - \frac{\mathsf{TV}(\mathcal{R_M}(s), \mathcal{R_H}(s))^2}{2}.\]
\end{corollary}


\textbf{General Trade-offs between True Positive and False Positive Rates.} Another way to understand the limitations of AI-generated text detectors is directly through the characterization of the trade-offs between true positive rates and false positive rates. Adapting inequality \ref{eq:TPR_FPR_bound}, we have the following corollaries: 


\begin{corollary}
\label{corollary:rephrasing_wm}
For any watermarking scheme $W$,
\begin{align*}
    \Pr_{s_w\sim \mathcal{R}_{\mathcal{M}}(s)} [\text{$s_w$ is watermarked using $W$}] \leq &
    \Pr_{s_w\sim \mathcal{R}_{\mathcal{H}}(s)}[\text{$s_w$ is watermarked using $W$}] \\
    & + \mathsf{TV}(\mathcal{R_M}(s), \mathcal{R_H}(s)),
\end{align*}
where $\mathcal{R_M}(s)$ and $\mathcal{R_H}(s)$ are the distributions of rephrased sequences for $s$ produced by the paraphrasing model and humans, respectively.
\end{corollary}


Humans may have different writing styles. Corollary \ref{corollary:rephrasing_wm} indicates that if a rephrasing model resembles certain human text distribution $\mathcal{H}$ (i.e. $\mathsf{TV}(\mathcal{R_M}(s), \mathcal{R_H}(s))$ is small), then either certain people's writing will be detected falsely as watermarked (i.e. $\Pr_{s_w\sim \mathcal{R}_{\mathcal{H}}(s)} [\text{$s_w$ is watermarked using $W$}]$ is high) or the paraphrasing model can remove the watermark (i.e. $\Pr_{s_w\sim \mathcal{R}_{\mathcal{M}}(s)} [\text{$s_w$ is watermarked using $W$}]$ is low).


\begin{corollary}
\label{corollary:rephrasing_no_wm}
For any AI-text detector $D$, 
\begin{align*}
    \Pr_{s\sim \mathcal{M}} [\text{$s$ is detected as AI-text by $D$}] \leq \Pr_{s\sim \mathcal{H}}[\text{$s$ is detected as AI-text by $D$}] + \mathsf{TV}(\mathcal{M}, \mathcal{H}),
\end{align*}
where $\mathcal{M}$ and $\mathcal{H}$ denote text distributions by the model and by humans, respectively.
\end{corollary}

Corollary \ref{corollary:rephrasing_no_wm} indicates that if a model resembles certain human text distribution $\mathcal{H}$ (i.e. $\mathsf{TV}(\mathcal{M}, \mathcal{H})$ is small), then either certain people's writing will be detected falsely as AI-generated (i.e. $\Pr_{s\sim \mathcal{H}} [\text{$s$ is detected as AI-text by $D$}]$ is high) or the AI-generated text will not be detected reliably (i.e. $\Pr_{s\sim \mathcal{M}} [\text{$s$ is detected as AI-text by $D$}]$ is low).

\looseness -1
These results demonstrate fundamental limitations for AI-text detectors, with and without watermarking schemes.
In Appendix~\ref{sec:tightness}, we present a tightness analysis of the bound in Theorem~\ref{thm:ROC_bound}, where we show that for any human distribution $\mathcal{H}$ there exists an AI distribution and a detector $D$ for which the bound holds with equality.

\subsection{Tightness Analysis}
\label{sec:tightness}
In this section, we show that the bound in Theorem~\ref{thm:ROC_bound} is tight.
This bound need not be tight for any two distributions $\mathcal{H}$ and $\mathcal{M}$, e.g., two identical normal distributions in one dimension shifted by a distance.
However, tightness can be shown for every human distribution $\mathcal{H}$.
For a given distribution of human-generated text sequences $\mathcal{H}$, we construct an AI-text distribution $\mathcal{M}$ and a detector $D$ such that the bound holds with equality.

Define sublevel sets of the probability density function of the distribution of human-generated text $\mathsf{pdf}_\mathcal{H}$ over the set of all sequences $\Omega$ as follows:
\[\Omega_\mathcal{H}(c) = \{s \in \Omega \mid \mathsf{pdf}_\mathcal{H}(s) \leq c\}\]
where $c \in \mathbb{R}$.
Assume that, $\Omega_\mathcal{H}(0)$ is not empty.
Now, consider a distribution $\mathcal{M}$, with density function $\mathsf{pdf}_\mathcal{M}$, which has the following properties:
\begin{enumerate}
    \item The probability of a sequence drawn from $\mathcal{M}$ falling in $\Omega_\mathcal{H}(0)$ is $\mathsf{TV}(\mathcal{M}, \mathcal{H})$, i.e., $\mathbb{P}_{s \sim \mathcal{M}}[s \in \Omega_\mathcal{H}(0)] = \mathsf{TV}(\mathcal{M}, \mathcal{H})$.
    \item $\mathsf{pdf}_\mathcal{M}(s) = \mathsf{pdf}_\mathcal{H}(s)$ for all $s \in \Omega(\tau) - \Omega(0)$ where $\tau > 0$ such that $\mathbb{P}_{s \sim \mathcal{H}}[ s \in \Omega(\tau)] = 1 - \mathsf{TV}(\mathcal{M}, \mathcal{H})$.
    \item $\mathsf{pdf}_\mathcal{M}(s) = 0$ for all $s \in \Omega - \Omega(\tau)$.
\end{enumerate}
Define a hypothetical detector $D$ that maps each sequence in $\Omega$ to the negative of the probability density function of $\mathcal{H}$, i.e., $D(s) = - \mathsf{pdf}_\mathcal{H}(s)$.
Using the definitions of $\mathsf{TPR}_\gamma$ and $\mathsf{FPR}_\gamma$, we have:
\begin{align*}
    \mathsf{TPR}_\gamma &= \mathbb{P}_{s \sim \mathcal{M}}[D(s) \geq \gamma]\\
    &= \mathbb{P}_{s \sim \mathcal{M}}[- \mathsf{pdf}_\mathcal{H}(s) \geq \gamma]\\
    &= \mathbb{P}_{s \sim \mathcal{M}}[\mathsf{pdf}_\mathcal{H}(s) \leq -\gamma]\\
    &= \mathbb{P}_{s \sim \mathcal{M}}[ s \in \Omega_\mathcal{H}(-\gamma)]
\end{align*}
Similarly,
\[\mathsf{FPR}_\gamma = \mathbb{P}_{s \sim \mathcal{H}}[ s \in \Omega_\mathcal{H}(-\gamma)].\]
For $\gamma \in [-\tau, 0]$,
\begin{align*}
    \mathsf{TPR}_\gamma &= \mathbb{P}_{s \sim \mathcal{M}}[ s \in \Omega_\mathcal{H}(-\gamma)]\\
    &= \mathbb{P}_{s \sim \mathcal{M}}[ s \in \Omega_\mathcal{H}(0)] + \mathbb{P}_{s \sim \mathcal{M}}[ s \in \Omega_\mathcal{H}(-\gamma) - \Omega_\mathcal{H}(0)]\\
    &= \mathsf{TV}(\mathcal{M}, \mathcal{H}) + \mathbb{P}_{s \sim \mathcal{M}}[ s \in \Omega_\mathcal{H}(-\gamma) - \Omega_\mathcal{H}(0)] \tag{using property 1}\\
    &= \mathsf{TV}(\mathcal{M}, \mathcal{H}) + \mathbb{P}_{s \sim \mathcal{H}}[ s \in \Omega_\mathcal{H}(-\gamma) - \Omega_\mathcal{H}(0)] \tag{using property 2}\\
    &= \mathsf{TV}(\mathcal{M}, \mathcal{H}) + \mathbb{P}_{s \sim \mathcal{H}}[ s \in \Omega_\mathcal{H}(-\gamma)] - \mathbb{P}_{s \sim \mathcal{H}}[s \in \Omega_\mathcal{H}(0)] \tag{$\Omega_\mathcal{H}(0) \subseteq \Omega_\mathcal{H}(-\gamma)$}\\
    &= \mathsf{TV}(\mathcal{M}, \mathcal{H}) + \mathsf{FPR}_\gamma. \tag{$\mathbb{P}_{s \sim \mathcal{H}}[s \in \Omega_\mathcal{H}(0)] = 0$}
\end{align*}
For $\gamma \in [-\infty, -\tau]$, $\mathsf{TPR}_\gamma = 1$, by property 3.
Also, as $\gamma$ goes from $0$ to $-\infty$, $\mathsf{FPR}_\gamma$ goes from $0$ to $1$.
Therefore, $\mathsf{TPR}_\gamma = \min(\mathsf{FPR}_\gamma + \mathsf{TV}(\mathcal{M}, \mathcal{H}), 1)$ which is similar to Equation~\ref{eq:tpr_bound}.
Calculating the AUROC in a similar fashion as in the previous section, we get the following:
\[\mathsf{AUROC}(D) = \frac{1}{2} + \mathsf{TV}(\mathcal{M}, \mathcal{H}) - \frac{\mathsf{TV}(\mathcal{M}, \mathcal{H})^2}{2}.\]

\subsection{Pseudorandomness in LLMs }
\label{sec:PRG_bound}
Most machine learning models, including large language models (LLMs), use pseudorandom number generators in one form or another to produce their outputs.
For example, an LLM may use a pseudorandom number generator to sample the next token in the output sequence.
In discussing our impossibility result, \citet{kirchenbauer2023reliability} in a more recent work argue that this pseudorandomness makes the AI-generated text distribution very different from the human-generated text distribution.
This is because the pseudorandom AI-generated distribution is a collection of Dirac delta function distributions and a human is exorbitantly unlikely to produce a sample corresponding to any of the delta functions.
In our framework, this means that the total variation between the human and pseudorandom AI-generated distributions is almost one, making the bound in Theorem~\ref{thm:ROC_bound} vacuous.

We argue that, although the true total variation between the human and pseudorandom AI-generated distributions is high and there exists (in theory) a detector function that can separate the distributions almost perfectly, this function may not be efficiently computable.
Any polynomial-time computable detector can only achieve a negligible advantage from the use of pseudorandomness instead of true randomness.
If we had knowledge of the seed used for the pseudorandom number generator, we would be able to predict the pseudorandom samples.
However, an individual seeking to evade detection could simply randomize this seed making it computationally infeasible to predict the samples.
%the pseudorandom samples infeasible to predict.

We modify the bound in Theorem~\ref{thm:ROC_bound} to include a negligible correction term $\epsilon$ to account for the use of pseudorandomness.
We prove that the performance of a polynomial-time computable detector $D$ on a pseudorandom version of the AI-generated distribution $\widehat{\mathcal{M}}$ is bounded by the total variation for the truly random distribution $\mathcal{M}$ (resulting from the LLM using true randomness) as follows:
\[\mathsf{AUROC}(D) \leq \frac{1}{2} + \mathsf{TV}(\mathcal{M}, \mathcal{H}) - \frac{\mathsf{TV}(\mathcal{M}, \mathcal{H})^2}{2} + \epsilon.\]
The term $\epsilon$ represents the gap between the probabilities assigned by $\mathcal{M}$ and $\widehat{\mathcal{M}}$ to any polynomial-time computable $\{0, 1\}$-function $f$, i.e.,
\begin{equation}
\label{eq:eps_adv}
\big|\mathbb{P}_{s \in \mathcal{M}}[f(s) = 1] - \mathbb{P}_{s \in \widehat{\mathcal{M}}}[f(s) = 1]\big| \leq \epsilon.
\end{equation}
This term is orders of magnitude smaller than any of the terms in the bound and can be safely ignored.
For example, commonly used pseudorandom generators\footnote{Cryptographic PRNGs: \url{https://en.wikipedia.org/wiki/Pseudorandom_number_generator}} can achieve an $\epsilon$ that is bounded by a negligible function $1/b^t$ of the number of bits $b$ used in the seed of the generator for a positive integer $t$\footnote{Negligible function: \url{https://en.wikipedia.org/wiki/Negligible_function}} \citep{bbs_prg, blum_micali_prg}.
From a computational point of view, the total variation for the pseudorandom distribution is almost the same as the truly random AI-generated distribution.
Thus, our framework provides a reasonable approximation for real-world LLMs and the impossibility result holds even in the presence of pseudorandomness.

\textbf{Computational Total Variation Distance:} Just as the total variation distance $\tv$ between two probability distributions is defined as the difference in probabilities assigned by the two distributions to any $\{0, 1\}$-function, we define a computational version of this distance $\tv_c$ for polynomial-time computable functions:
\[\tv_c(A, B) = \max_{f \in \mathcal{P}} \big|\mathbb{P}_{s \sim A}[f(s) = 1] - \mathbb{P}_{s \sim B}[f(s) = 1]\big|,\]
where $\mathcal{P}$ represents the set of polynomial-time computable $\{0, 1\}$-functions.
$\mathcal{P}$ could also be defined as the set of all polynomial-size circuits which could be more appropriate for deep neural network-based detectors.
The function $f$ could be thought of as the indicator function for the detection parameter being above a certain threshold, i.e., $D(s) \geq \gamma$ as in the proof of Theorem~\ref{thm:ROC_bound}.
The following lemma holds for the performance of a polynomial-time detector $D$:
\begin{lemma}
\label{lem:pseudo_bnd}
The area under the ROC of any polynomial-time computable detector $D$ is bounded as
\[\mathsf{AUROC}(D) \leq \frac{1}{2} + \tv_c(\widehat{\mathcal{M}}, \mathcal{H}) - \frac{\tv_c(\widehat{\mathcal{M}}, \mathcal{H})^2}{2}.\]
\end{lemma}
This lemma can be proved in the same way as Theorem~\ref{thm:ROC_bound} by replacing the truly random AI-generated distribution $\mathcal{M}$ with its pseudorandom version $\widehat{\mathcal{M}}$ and the true total variation $\tv$ with its computaional variant $\tv_c$.

Next, we relate the computational total variation $\tv_c$ between $\mathcal{H}$ and the pseudorandom distribution $\widehat{\mathcal{M}}$ with the total variation $\tv$ between $\mathcal{H}$ and the truly random distribution $\mathcal{M}$.
\begin{lemma}
\label{lem:tv_relation}
For human distribution $\mathcal{H}$, truly random AI-generated distribution $\mathcal{M}$ and its pseudorandom version $\widehat{\mathcal{M}}$,
    \[\tv_c(\widehat{\mathcal{M}}, \mathcal{H}) \leq \tv(\mathcal{M}, \mathcal{H}) + \epsilon.\]
\end{lemma}
\begin{proof}
    \begin{align*}
    \tv_c(\widehat{\mathcal{M}}, \mathcal{H}) &= \max_{f \in \mathcal{P}} \big|\mathbb{P}_{s \sim \mathcal{H}}[f(s) = 1] - \mathbb{P}_{s \sim \widehat{\mathcal{M}}}[f(s) = 1]\big| \tag{from definition of $\tv_c$}\\
    &= \max_{f \in \mathcal{P}} \big|\mathbb{P}_{s \sim \mathcal{H}}[f(s) = 1] - \mathbb{P}_{s \sim \mathcal{M}}[f(s) = 1]\\
    &\quad \quad \quad + \mathbb{P}_{s \sim \mathcal{M}}[f(s) = 1] - \mathbb{P}_{s \sim \widehat{\mathcal{M}}}[f(s) = 1]\big| \tag{+/-ing $\mathbb{P}_{s \sim \mathcal{M}}[f(s) = 1]$}\\
    &\leq \max_{f \in \mathcal{P}} \big|\mathbb{P}_{s \sim \mathcal{H}}[f(s) = 1] - \mathbb{P}_{s \sim \mathcal{M}}[f(s) = 1]\big| \\
    &\quad \quad \quad + \big|\mathbb{P}_{s \sim \mathcal{M}}[f(s) = 1] - \mathbb{P}_{s \sim \widehat{\mathcal{M}}}[f(s) = 1]\big| \tag{using $|a + b| \leq |a| + |b|$}\\
    &\leq \tv(\mathcal{M}, \mathcal{H}) + \epsilon. \tag{from definition of $\tv$ and bound~\ref{eq:eps_adv}}
    \end{align*}
\end{proof}
This shows that although the true total variation may be high due to pseudorandomness, the effective total variation is still low.
We now use this to prove the modified version of our impossibility result.
\begin{theorem}[\textbf{Computational Impossibility Result}] The AUROC of any polynomial-time computable detector $D$ for $\mathcal{H}$ and the pseudorandom distribution $\widehat{\mathcal{M}}$ is bounded using the $\tv$ for the truly random distribution $\mathcal{M}$ as
    \[\mathsf{AUROC}(D) \leq \frac{1}{2} + \mathsf{TV}(\mathcal{M}, \mathcal{H}) - \frac{\mathsf{TV}(\mathcal{M}, \mathcal{H})^2}{2} + \epsilon.\]
\end{theorem}
\begin{proof}
    \begin{align*}
        \mathsf{AUROC}(D) &\leq \frac{1}{2} + \tv_c(\widehat{\mathcal{M}}, \mathcal{H}) - \frac{\tv_c(\widehat{\mathcal{M}}, \mathcal{H})^2}{2} \tag{from Lemma~\ref{lem:pseudo_bnd}}\\
        &\leq \frac{1}{2} + \tv(\mathcal{M}, \mathcal{H}) + \epsilon - \frac{\left(\tv(\mathcal{M}, \mathcal{H}\right) + \epsilon)^2}{2} \tag{from Lemma~\ref{lem:tv_relation} and since $\frac{1}{2} + x -\frac{x^2}{2}$ is increasing in $[0,1]$}\\
        &= \frac{1}{2} + \tv(\mathcal{M}, \mathcal{H}) + \epsilon - \frac{\tv(\mathcal{M}, \mathcal{H})^2 + \epsilon^2 + 2\epsilon\tv(\mathcal{M}, \mathcal{H})}{2}\\
        &\leq \frac{1}{2} + \mathsf{TV}(\mathcal{M}, \mathcal{H}) - \frac{\mathsf{TV}(\mathcal{M}, \mathcal{H})^2}{2} + \epsilon. \tag{dropping negative terms containing $\epsilon$}
    \end{align*}
\end{proof}

\section{Estimating Total Variation between Human and AI Text Distributions}
\label{sec:tv_estimation}
We estimate the total variation (TV) between human text distribution (WebText) and the output distribution of several models from OpenAI's GPT-2 series.\footnote{\url{https://github.com/openai/gpt-2-output-dataset}}
For two distributions $\mathcal{H}$ and $\mathcal{M}$, the total variation between them is defined as the maximum difference between the probabilities assigned by them to any event $E$ over the sample space $\Omega$, i.e.,
\[\mathsf{TV}(\mathcal{H}, \mathcal{M}) = \max_E \big|\mathbb{P}_{s \sim \mathcal{H}}[s \in E] - \mathbb{P}_{s \sim \mathcal{M}}[s \in E]\big|.\]
Thus, for any event $E$, this difference in probabilities is a valid {\it lower bound} on the total variation between the two distributions.
Since we do not know the probability density functions for $\mathcal{H}$ and $\mathcal{M}$, solving the above maximization problem over the entire event space is intractable.
Thus, we approximate the event space as a class of  events defined by a neural network with parameters $\theta$ that maps a text sequence in $\Omega$ to a real number.
The corresponding event $E_\theta$ is said to occur when the output of the neural network is above a threshold $\tau$.
We seek to find an event $E_\theta$ which obtains as tight a lower bound on the total variation as possible.

\begin{wrapfigure}{r}{0.5\textwidth}
    \centering
    \vspace{-6mm}
    %\hspace{-2mm}
    \includegraphics[width=1.1\linewidth, trim={0 0 0 6mm},clip]{images/tv_roberta-large_small-117M.png}
    %\vspace{-4mm}
    \caption{TV between WebText and outputs of GPT-2 models (small, medium, large, and XL) for varying sequence lengths.}
    \label{fig:gpt2-wt}
    \vspace{-6mm}
\end{wrapfigure}


\textbf{Estimation Procedure:} We train a RoBERTa large classifier~\cite{liu2019roberta} on samples from human and AI text distributions.
Given a text sequence, this classifier produces a score between 0 and 1 that represents how likely the model thinks the sequence is AI-generated.
Assuming the AI text distribution as the positive class, we pick a threshold for this score that maximizes the difference between the true positive rate (TPR) and the false positive rate (FPR) using samples from a validation set.
Finally, we estimate the total variation as the difference between the TPR and the FPR on a test set.
This difference is essentially the gap between the probabilities assigned by the human and AI-generated text distributions to the above classifier for the computed threshold, which is a lower bound on the total variation. 
% of the output being above the threshold under the human text distribution and the AI text distribution using a test set.
% This difference is essentially the gap between the TPR and the FPR on the test set.

Figure~\ref{fig:gpt2-wt} plots the total variation estimates for four GPT-2 models (small, medium, large, and XL) for four different text sequence lengths (25, 50, 75, and 100) estimated using a RoBERTa-large architecture.
We train a separate instance of this architecture for each GPT-2 model and sequence length to estimate the  total variation for the corresponding distribution.
We observe that, {\it as models become larger and more sophisticated, the TV estimates between human and AI-text distributions decrease.}
This indicates that as language models become more powerful, the statistical difference between their output distribution and human-generated text distribution vanishes.
%their output becomes more similar to human text.
% We include similar results for models in the GPT-3 series using WebText and ArXiv abstracts datasets as human text in the appendix.

\subsection{Estimating Total Variation for GPT-3 Models}
We repeat the above experiments %of Section~\ref{sec:tv_estimation}
with GPT-3 series models, namely Ada, Babbage, and Curie, as documented on the OpenAI platform.\footnote{\url{https://platform.openai.com/docs/models/gpt-3}}
We use WebText and ArXiv abstracts~\cite{clement2019arxiv} datasets as human text distributions.
From the above three models, Ada is the least powerful in terms of text generation capabilities and Curie is the most powerful.
Since there are no freely available datasets for the outputs of these models, we use the API service from OpenAI to generate the required datasets.

We split each human text sequence from WebText into `prompt' and `completion', where the prompt contains the first hundred tokens of the original sequence and the completion contains the rest.
We then use the prompts to generate completions using the GPT-3 models with the temperature set to 0.4 in the OpenAI API.
We use these model completions and the `completion' portion of the human text sequences to estimate total variation using a RoBERTa-large model in the same fashion as Section~\ref{sec:tv_estimation}.
Using the first hundred tokens of the human sequences as prompts allows us to control the context in which the texts are generated.
This allows us to compare the similarity of the generated texts to human texts within the same context.

Figure~\ref{fig:gpt-3-webtext} plots the total variation estimates of the GPT-3 models with respect to WebText for four different sequence lengths 25, 50, 75, and 100 from the model completions.
Similar to the GPT-2 models in \S~\ref{sec:tv_estimation}, we observe that the most powerful model Curie has the least total variation across all sequence lengths.
The model Babbage, however, does not follow this trend and exhibits a higher total variation than even the least powerful model Ada.

Given that WebText contains data from a broad range of Internet sources, we also experiment with more focused scenarios, such as generating content for scientific literature.
We use the ArXiv abstracts dataset as human text and estimate the total variation for the above three models (Figure~\ref{fig:gpt-3-arxiv}).
We observe that, for most sequence lengths, the total variation decreases across the series of models: Ada, Babbage, and Curie.
This provides further evidence that as language models improve in power their outputs become more indistinguishable from human text, making them harder to detect.

\begin{figure}
    \centering
    \begin{subfigure}[b]{0.48\textwidth}
        \includegraphics[width=1.1\textwidth]{images/text-ada-001_completion.png}
        \caption{WebText}
        \label{fig:gpt-3-webtext}
    \end{subfigure}
    \hfill
    \begin{subfigure}[b]{0.48\textwidth}
        \includegraphics[width=1.1\textwidth]{images/arxiv-gpt3.png}
        \caption{ArXiv}
        \label{fig:gpt-3-arxiv}
    \end{subfigure}
    \caption{Total variation estimates for GPT-3 models with respect to WebText and ArXiv datasets using different sequence lengths from the model completions.}
\end{figure}
\section{Spoofing Attacks on AI-text Generative Models}
\label{sec:humantextdetected}

\begin{table}[t]
\small
    \centering
    \begin{tabular}{M{7cm} | P{1.6cm} | P{1cm} | P{2.0cm} }
    \toprule
        \multicolumn{1}{P{7cm}|}{Human text} & $\%$ tokens in green list & z-score & Detector output \\ \midrule \midrule
        the first thing you do will be the best thing you do. this is the reason why you do the first thing very well. if most of us did the first thing so well this world would be a lot better place. and it is a very well known fact. people from every place know this fact. time will prove this point to the all of us. as you get more money you will also get this fact like other people do. all of us should do the first thing very well. hence the first thing you do will be the best thing you do. & 42.6 & 4.36 & Watermarked\\ \midrule
        lot to and where is it about you know and where is it about you know and where is it that not this we are not him is it about you know and so for and go is it that. & 92.5 & 9.86 & Watermarked \\ \bottomrule 
    \end{tabular}
    \vspace{0.2cm}
    \caption{Proof-of-concept human-generated texts flagged as watermarked by the soft watermarking scheme. In the first row, a sensible sentence composed by an {\it adversarial human} contains $42.6\%$ tokens from the green list. In the second row, a nonsense sentence generated by an {\it adversarial human} using our tool contains $92.5\%$ green list tokens. The z-test threshold for watermark detection is 4.}
    \label{tab:adversarialhuman}
    \vspace{-5mm}
\end{table}

% DO NOT DELETE THIS
% COMPLETE TABLE TEXT ENTRIES
% the first thing you do will be the best thing you do. this is the reason why you do the first thing very well. if most of us did the first thing so well this world would be a lot better place. and it is a very well known fact. people from every place know this fact. time will prove this point to the all of us. as you get more money you will also get this fact like other people do. all of us should do the first thing very well. hence the first thing you do will be the best thing you do.
% lot to and where is it about you know and where is it about you know and where is it that not this we are not him is it about you know and so for and go is it that.

\begin{figure}[t]
    \centering
    \includegraphics[width=1\linewidth]{images/the.png}
    % \vspace{-6mm}
    \caption{Inferred {\it green list score} for the token ``the''. The plot shows the top 50 words from our set of common words that are likely to be in the green list. The word ``first'' occurred $\sim 25\%$ of the time as suffix to ``the''.}
    \vspace{-3.3mm}
    \label{fig:the}
\end{figure}


A strong AI text detection scheme should have both low type-I error (i.e., human text detected as AI-generated) and type-II error (i.e., AI-generated text not detected). An AI language detector without a low type-I error can cause harm as it might wrongly accuse a human of plagiarizing using an LLM. Moreover, an attacker (adversarial human) can generate a non-AI text that is detected to be AI-generated. This is called the {\it spoofing attack}. An adversary can potentially launch spoofing attacks to produce derogatory texts that are detected to be AI-generated to affect the reputation of the target LLM's developers. In this section, as a proof-of-concept, we show that the soft watermarking  \citep{kirchenbauer2023watermark} and retrieval-based \citep{krishna2023paraphrasing} detectors can be spoofed to detect texts composed by humans as watermarked. 



\subsection{Spoofing Attacks on Watermarked Models}



\begin{wrapfigure}{r}{0.55\textwidth}
    \centering
    \includegraphics[width=1\linewidth,]{images/wm_spoof.png}
  \caption{ROC curve of a soft watermarking-based detector \citep{kirchenbauer2023watermark} after our spoofing attack.}
  \vspace{-3.3mm}
    \label{fig:watermark-roc-spoof}
\end{wrapfigure}
In \citet{kirchenbauer2023watermark}, they
watermark LLM outputs by asserting the model to output tokens with some specific pattern that can be easily detected with meager error rates. Soft watermarked texts are majorly composed of {\it green list} tokens. If an adversary can learn the green lists for the soft watermarking scheme, they can use this information to generate human-written texts that are detected to be watermarked. Our experiments show that the soft watermarking scheme can be spoofed efficiently. Though the soft watermarking detector can detect the presence of a watermark very accurately, it cannot be certain if this pattern is actually generated by a human or an LLM.  An {\it adversarial human} can compose derogatory watermarked texts in this fashion that are detected to be watermarked, which might cause reputational damage to the developers of the watermarked LLM. Therefore, it is important to study {\it spoofing attacks} to avoid such scenarios.


% \textbf{The attack methodology:} 
% For an output word $s^{(t)}$, soft watermarking samples a word from its green list with high probability. 
In watermarking, the prefix word $s^{(t-1)}$ determines the green list for selecting the word $s^{(t)}$. The attacker's objective is to compute a proxy of green lists for the $N$ most commonly used words in the vocabulary.
% A smaller $N$, when compared to the size of the vocabulary, helps faster computations with a trade-off in the attacker's knowledge of the watermarking scheme. 
We use a small value of $N=181$ for our experiments. 
The attacker queries the watermarked OPT-1.3B \cite{opt} $10^6$ times to observe pair-wise token occurrences in its output to estimate {\it green list score} for the $N$ tokens. A token with a high green list score for a prefix $s^{(t)}$ might be in its green list (see Figure \ref{fig:the}). We build a tool that helps adversarial humans create watermarked sentences by providing them with proxy green list. In this manner, we can spoof watermarking models easily. See Table \ref{tab:adversarialhuman} for example sentences created by an adversarial human.
% The attacker can query the watermarked LLM multiple times to learn the pair-wise occurrences of these $N$ words in the LLM output. Observing these outputs, the attacker can compute the probability of occurrence of a word given a prefix word $s^{(t-1)}$. This score can be used as a proxy for computing the green list for the prefix word $s^{(t-1)}$. An attacker with access to these proxy green lists can compose a text detected to be watermarked, thus spoofing the detector. In our experiments, we query the watermarked OPT-1.3B \citep{opt} $10^6$ times to evaluate the {\it green list scores} to evaluate the green list proxies. We find that inputting nonsense sentences composed of the $N$ common words encourages the LLM to output text majorly only composed of these words. This makes the querying more efficient. In Figure \ref{fig:the}, we show the learned {\ green list scores} for the prefix word ``the'' using our querying technique. We build a simple tool that lets a user create passages token by token. At every step, the user is provided with a list of potential green list words sorted based on the { green list score}. These users or {adversarial humans} try to generate meaningful passages assisted by our tool. Since most of the words selected by { adversarial humans} are likely to be in the green list, we expect the watermarking scheme to detect these texts to be watermarked. Table \ref{tab:adversarialhuman} shows examples of sentences composed by { adversarial humans} that are detected to be watermarked. Even a nonsense sentence generated by an adversarial human can be detected as watermarked with very high confidence.
Figure \ref{fig:watermark-roc-spoof} shows that the performance of watermark-based detectors degrades significantly in the presence of paraphrasing and spoofing attacks, showing that they are not reliable. 

% \begin{figure}[t]
%     \centering
%     \includegraphics[width=0.6\linewidth]{images/the.png}
%     \vspace{-2mm}
%     \caption{Inferred {\it green list score} for the token ``the''. The plot shows the top 50 words from our set of common words that are likely to be in the green list. The word ``first'' occurred $\sim 25\%$ of the time as suffix to ``the''.}
%     \label{fig:the}
%     \vspace{-2mm}
% \end{figure}

\subsection{Spoofing Attacks on Retrieval-based Defenses}
Retrieval-based detectors store LLM outputs in a database. Although {\bf storing user-LLM conversations can cause serious privacy concerns in the real world}, they use this database to detect if a candidate passage is AI-generated by searching for semantically similar passages. An adversary with access to human-written passages can feed them into this database and later use the detector to falsely accuse these humans of plagiarism. For our experiments, we consider outputs from DIPPER, a paraphrasing LLM introduced in \cite{krishna2023paraphrasing}, as AI outputs. An adversary asks DIPPER to paraphrase 100 human passages from XSum, and this retrieval-based detector stores the paraphraser outputs in its database. Our experiments show that the retrieval-based detector falsely classifies all 100 human passages as AI-generated. In our experiments, this retrieval-based detector achieves $0\%$ human text detection when spoofed.

\begin{table}[t]
    \centering
    \small
    \begin{tabular}{c| c | c}
    \toprule
    Detection Methods  & T@F & F@T \\          \midrule \midrule
Entropy threshold \cite{gehrmann2019gltr} & \textbf{0.025} (0.045) & \textbf{0.995} (0.845)\\
Likelihood threshold \cite{solaiman2019release} & \textbf{0.050} (0.075) & \textbf{0.995} (0.310)\\
Logrank threshold & 0.165 (0.155)  & \textbf{0.690} (0.190)\\
Rank threshold \cite{gehrmann2019gltr} & 0.530 (0.335) & \textbf{0.655} (0.590)\\
Roberta (base) OpenAI detector \cite{openaidetectgpt2} & 0.900 (0.765) & 0.010 (0.035)\\
Roberta (large) OpenAI detector \cite{openaidetectgpt2} & \textbf{0.985} (0.990) & 0.000 (0.000) \\
% Perturbation (d) & 0.115 & 0.240 \\
DetectGPT \cite{mitchell2023detectgpt} & \textbf{0.055} (0.240) & \textbf{0.555} (0.145)\\
\bottomrule
    \end{tabular}
    \vspace{0.2cm}
    \caption{True positive rates at $1\%$ false positive rate (T@F)  and  false positive rates at $90\%$ true positive rate (F@T) after (before the attack in parentheses) the spoofing attack described in \S \ref{sec:spoof_detector}. Bolded numbers show successful attacks where T@F decreases, or F@T increases after spoofing.}
    \label{tab:spoof_fpr_tpr}
\end{table}


% \begin{table}[t]
%     \centering
%     \small
%     \begin{tabular}{c|c | c}
%     \toprule
%     Detection Methods  & Before Spoofing & After Spoofing \\          \midrule \midrule
% Entropy threshold \cite{gehrmann2019gltr} & 0.845 & 0.995 \\
% Likelihood threshold \cite{solaiman2019release} & 0.310 & 0.995 \\
% Logrank threshold & 0.190 & 0.690 \\
% Rank threshold \cite{gehrmann2019gltr} & 0.590 & 0.655 \\
% Roberta (base) OpenAI detector \cite{openaidetectgpt2} & 0.035 & 0.010 \\
% Roberta (large) OpenAI detector \cite{openaidetectgpt2} & 0.000 & 0.000 \\
% % Perturbation (d) & 0.115 & 0.240 \\
% Detect-GPT \cite{mitchell2023detectgpt} & 0.145 & 0.555 \\
% \bottomrule
%     \end{tabular}
%     \vspace{0.2cm}
%     \caption{\textcolor{blue}{False positive rate at $90\%$ true positive rate before and after the spoofing attack described in \S \ref{sec:spoof_detector}. }}
%     \label{tab:spoof_fpr_tpr}
% \end{table}

% \begin{table}[t]
%     \centering
%     \small
%     \begin{tabular}{c|cc}
%     \toprule
%    Detection Methods & Before Spoofing & After Spoofing \\
%     \midrule \midrule
%         Entropy threshold \cite{gehrmann2019gltr} & 0.045 & 0.025 \\
% Likelihood threshold \cite{solaiman2019release} & 0.075 & 0.050 \\
% Logrank threshold & 0.155 & 0.165 \\
% Rank threshold \cite{gehrmann2019gltr} & 0.335 & 0.530 \\
% RoBERTa (base) OpenAI detector \cite{openaidetectgpt2} & 0.765 & 0.900 \\
% RoBERTa (large) OpenAI detector \cite{openaidetectgpt2} & 0.990 & 0.985 \\
% % Perturbation (d) & 0.640 & 0.610 \\
% Detect-GPT \cite{mitchell2023detectgpt} & 0.240 & 0.055 \\
% \bottomrule
%     \end{tabular}
%     \vspace{0.2cm}
%     \caption{\textcolor{blue}{True positive rate at $1\%$ false positive rate before and after the spoofing attack described in \S \ref{sec:spoof_detector}.}}
%     \label{tab:spoof_tpr_fpr}
% \end{table}
\begin{figure}[t]

    \centering
    \includegraphics[width=0.49\textwidth]{images/before-spoof-semilogx_roc_curves.png}
    \includegraphics[width=0.49\textwidth]{images/after-spoof-semilogx_roc_curves.png}
    \caption{ROC curves before (left) and after (right) spoofing attack (\S~\ref{sec:spoof_detector}). Most detectors exhibit quality degradation after our spoofing attack.}
    \label{fig:spoof_roc}
\end{figure}

\subsection{Spoofing Attacks on Zero-Shot and Trained Detectors}
\label{sec:spoof_detector}

We further show that zero-shot and trained detectors may also be vulnerable to spoofing attacks. In this setting, a malicious adversary could write a short text in a collaborative work which may lead to the entire text being classified as AI-generated. To simulate this, we prepend a human-written text marked as AI-generated by the detector to all the other human-generated text for spoofing. In other words, from the first 200 samples in the XSum dataset, we pick the human text with the worst detection score for each detector considered in \S~\ref{sec:nonwatermark}. We then prepend this text to all the other human texts, ensuring that the length of the prepended text does not exceed the length of the original text. We report the false positive rate fixed at a true positive rate of $90\%$ and the true positive rate at a false positive rate of $1\%$ in Table~\ref{tab:spoof_fpr_tpr}. The ROC curves before and after spoofing the detectors are provided in Figure~\ref{fig:spoof_roc}. Our experiments demonstrate that most of these detection methods show a significant increase in false positive rates at a fixed true positive rate of $90\%$ after spoofing.
After this na\"ive spoofing attack, the true positive rate at a false positive rate of $1\%$ and AUROC scores of these detectors drop significantly. 



\section{Compliance with ethical standards}
This research study was conducted retrospectively using human subject data made available in open access by the Genomic Data Commons (GDC) provided by the National Cancer Institute of the National Institues of Health (NIH/NCI). Ethical approval was not required as confirmed by the license attached with the open access data.

\begin{ack}
\addcontentsline{toc}{section}{~~~~~~Acknowledgments and Disclosure of Funding}

This project was supported in part by NSF CAREER AWARD 1942230, ONR YIP award N00014-22-1-2271, NIST 60NANB20D134, Meta award 23010098, HR001119S0026 (GARD), Army Grant No. W911NF2120076, a capital one grant, and the NSF award CCF2212458 and an Amazon Research Award. Sadasivan is also supported by the Kulkarni Research Fellowship. The authors would like to thank Keivan Rezaei and Mehrdad Saberi for their insights on this work. The authors also acknowledge the use of OpenAI's ChatGPT to improve clarity and readability.
\end{ack}

%\clearpage
\bibliographystyle{unsrtnat} 
%{plainnat}

\bibliography{references.bib}
\addcontentsline{toc}{section}{~~~~~~References}
% \documentclass[aps, prx, twocolumn, superscriptaddress, longbibliography]{revtex4-1}
\usepackage[colorlinks, linkcolor=blue, anchorcolor=blue, citecolor=blue]{hyperref}
\usepackage{amsmath}
\usepackage{graphicx}
\usepackage{braket}
\usepackage{multirow}
\usepackage{color}
\usepackage{soul}
%\usepackage[smalltableaux,centertableaux,boxsize=5pt]{ytableau}

%\usepackage{ulem}%only for the command \sout = scrap

\renewcommand{\thefigure}{S\arabic{figure}}
\renewcommand{\theequation}{S\arabic{equation}}
\renewcommand{\thetable}{S\arabic{table}}
\renewcommand{\thesection}{S-\Roman{section}}
\newcommand{\aw}[1]{{\color[rgb]{.8,.4,.2}{#1}}}
\newcommand{\awc}[1]{{\color[rgb]{.8,.6,.6}{[AW: {\it #1}\,]}}}
\newcommand{\awx}[1]{{\color[rgb]{.8,.6,.6}{\sout{#1}}}}%
\newcommand{\ylw}[1]{\textcolor{red}{#1}}

\begin{document}

\title{Supporting Information for ``Electronic Correlation Effects on Stabilizing a Perfect Kagome Lattice and Ferromagnetic Fluctuation in LaRu$_3$Si$_2$''}
\author{Yilin Wang}     
\affiliation{Hefei National Laboratory for Physical Sciences at Microscale, University of Science and Technology of China, Hefei, Anhui 230026, China} 

\date{\today}

\begin{abstract}
\end{abstract}

\maketitle

%\section{Computational Details}

\begin{table*}
    \centering
    \caption{Values of Hubbard $U$ and Hund's coupling $J_H$ calculated by the code \emph{R\underline{ }Coulomb.py} in DFT+EDMFTF package. These values are used for both DFT+U and LDA+DMFT calculations.}
    \begin{ruledtabular}
    \begin{tabular}{cccccccccc}
    $U$ (eV)   & 1.1   & 1.5   & 2.0   & 3.0   & 4.0   & 4.5  & 5.0   & 5.5   & 6.0\\
    $J_H$ (eV) & 0.389 & 0.476 & 0.563 & 0.692 & 0.782 &0.817 & 0.848 & 0.874 & 0.897\\
    \end{tabular}
    \end{ruledtabular}
    \label{tab:multi}
    \end{table*}

\begin{figure*}
        \centering
        \includegraphics[width=0.9\textwidth]{x_ggasoc.pdf}
        \caption{Fractional coordinates $x$ of Ru sites as function of Hubbard $U$, relaxed by GGA+U with spin-orbital coupling.}
        \label{fig:ggasoc}
\end{figure*}

\begin{figure*}
    \centering
    \includegraphics[width=0.9\textwidth]{magnetic_conf.pdf}
    \caption{Magnetic configurations considered in the GGA+U calculations.}
    \label{fig:mag_conf}
\end{figure*}



\begin{figure*}
    \centering
    \includegraphics[width=0.9\textwidth]{XRD.pdf}
    \caption{Simulated XRD pattern of the possible distorted Kagome structure of LaRu$_3$Si$_2$. (a) For space group P6$_3$/mcm with Ru at (0.52, 0, 0.25). (b) For space group P6$_3$/m with Ru at (0.52, 0.01, 0.25). Lattice parameters are $a=5.676$\AA\ and $c=7.12$\AA. Their only difference is that there is an additional weak peak at (1 0 1) for P6$_3$/m.  }
    \label{fig:xrd}
\end{figure*}
    



%\pagebreak
\bibliography{suppl}

\end{document}



\end{document}