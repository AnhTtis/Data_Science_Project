\section{Discussion}

Recent advancements in NLP show that LLMs can generate human-like texts for a various number of tasks \citep{gpt4}. However, this can create several challenges. LLMs can potentially be misused for plagiarism, spamming, or even social engineering to manipulate the public. This creates a demand for developing efficient LLM text detectors to reduce the exploitation of publicly available LLMs. Recent works propose a variety of AI text detectors using watermarking \citep{kirchenbauer2023watermark}, zero-shot methods \citep{mitchell2023detectgpt}, and trained neural network-based classifiers \citep{openaidetectgpt2}. In this paper, we show both theoretically and empirically, that these state-of-the-art detectors cannot reliably detect LLM outputs in practical scenarios. Our experiments show that paraphrasing the LLM outputs helps evade these detectors effectively. 
%Attackers can generate and spread misinformation using LLMs and use neural network-based paraphrasers without getting detected. 
Moreover, our theory demonstrates that for a sufficiently advanced language model, even the best detector can only perform marginally better than a random classifier. This means that for a detector to have both low type-I and type-II errors, it will have to trade off the LLM's performance. We also empirically show that watermarking-based detectors can be spoofed to make human-composed text detected as watermarked. We show that it is possible for an attacker to learn the soft watermarking scheme in \citep{kirchenbauer2023watermark}. Using this information, an adversary can launch a spoofing attack where adversarial humans generate texts that are detected to be watermarked. Spoofing attacks can lead to the generation of watermarked derogatory passages that might affect the reputation of the watermarked LLM developers.


 

With the release of GPT-4 \citep{gpt4}, the applications of LLMs are endless. This also calls for the need for more secure methods to prevent their misuse. Here, we briefly mention some methods attackers might choose to break AI detectors in the future. As we demonstrated in this paper, the emergence of improved paraphrasing models can be a severe threat to AI text detectors. Moreover, advanced LLMs might be vulnerable to attacks based on {\it smart prompting}. For example, attackers could input a prompt that starts with ``Generate a sentence in active voice and present tense using only the following set of words that I provide...''. High-performance LLMs would have a low entropy output space (less number of likely output sequences) for this prompt, making it harder to add a strong LLM signature in their output for detection. The soft watermarking scheme in \cite{kirchenbauer2023watermark} is vulnerable to this attack. If the logits of the LLM have low entropy over the vocabulary, soft watermarking scheme samples the token with the highest logit score (irrespective of the green list tokens) to preserve model perplexity. Furthermore, in the future, we can expect more open-source LLMs to be available to attackers. This could help attackers leverage these models to design transfer attacks to target a larger LLM. Adversarial input prompts could be designed using transfer attacks such that the target LLM is encouraged to have a low entropy output space. Future research on AI text detectors must be cautious about these vulnerabilities.

A detector should ideally be helpful in reliably flagging AI-generated texts to prevent the misuse of LLMs. However, the cost of misidentification by a detector can itself be huge. If the false positive rate of the detector is not low enough, humans could get wrongly accused of plagiarism. Moreover, a disparaging passage falsely detected to be AI-generated could affect the reputation of the LLM's developers. As a result, the practical applications of AI-text detectors can become unreliable and invalid. Security methods need not be foolproof. However, we need to make sure that it is not an easy task for an attacker to break these security defenses. Thus, analyzing the risks of using current detectors can be vital to avoid creating a false sense of security. We hope that the results presented in this work can encourage an open and honest discussion in the community about the ethical and trustworthy applications of generative LLMs. 