
\documentclass[10pt]{article} % For LaTeX2e
\usepackage[preprint]{tmlr}
% If accepted, instead use the following line for the camera-ready submission:
%\usepackage[accepted]{tmlr}
% To de-anonymize and remove mentions to TMLR (for example for posting to preprint servers), instead use the following:
% \usepackage[preprint]{tmlr}

% Optional math commands from https://github.com/goodfeli/dlbook_notation.
\newcommand{\bbox}{\text{bbox}}
\newcommand{\alphapck}{\alpha_\bbox}
\newcommand{\kcycle}{\text{k-CyPCK}}
\newcommand{\cycle}{\text{-CyPCK}}

\newcommand{\I}{\mathbf{I}}
\newcommand{\Ia}{\I^\text{a}}
\newcommand{\Ib}{\I^\text{b}}
\newcommand{\Iatob}{\I^\text{a $\rightarrow$ b}}
\newcommand{\F}{\mathbf{F}}
\newcommand{\Fa}{\F^\text{a}}
\newcommand{\Fb}{\F^\text{b}}
\newcommand{\f}{\mathbf{f}}
\newcommand{\fa}{\f^\text{a}}
\newcommand{\fb}{\f^\text{b}}
\newcommand{\p}{\mathbf{p}}
\newcommand{\pa}{\p^\text{a}}
\newcommand{\pb}{\p^\text{b}}
\newcommand{\A}{\boldsymbol{\Phi}_\text{align}}
\newcommand{\G}{\mathbf{G}}
\newcommand{\C}{\mathbf{C}}
\newcommand{\Ca}{\C^\text{a}}
\newcommand{\Cb}{\C^\text{b}}
\newcommand{\cc}{\mathbf{c}}
\newcommand{\cca}{\cc^\text{a}}
\newcommand{\ccb}{\cc^\text{b}}
\newcommand{\Irec}{\I_\text{Recon}}
\newcommand{\M}{\mathbf{M}}
\newcommand{\Mrec}{\M_\text{Recon}}
\newcommand{\loss}{\mathcal{L}}
\newcommand{\T}{\mathcal{T}}
\newcommand{\W}{\mathcal{W}}
\newcommand{\Id}{\mathcal{I}}


\usepackage{hyperref}
\usepackage{url}


\usepackage{xcolor}     
\hypersetup{colorlinks=true, linkcolor=purple, urlcolor=purple, citecolor=purple}
\usepackage[utf8]{inputenc} % allow utf-8 input
\usepackage[T1]{fontenc}    % use 8-bit T1 fonts
\usepackage{booktabs}       % professional-quality tables
\usepackage{amsfonts}       % blackboard math symbols
\usepackage{nicefrac}       % compact symbols for 1/2, etc.
\usepackage{microtype}      % microtypography
\usepackage{wrapfig}
\usepackage{float}
\usepackage{todonotes}
\usepackage{amsthm}
\usepackage{adjustbox}
\newtheorem{theorem}{Theorem}
\newtheorem{apptheorem}{Theorem}
\newtheorem{corollary}{Corollary}
\newtheorem{lemma}{Lemma}
\usepackage{graphicx}
\usepackage{amsmath}
\usepackage{amssymb}
\usepackage{subcaption}
\usepackage{mathtools}
\usepackage[bb=dsserif]{mathalpha}
\usepackage{soul}
\usepackage{makecell}
\usepackage{lipsum}
\usepackage{multirow}
\usepackage{array}
\newcolumntype{P}[1]{>{\centering\arraybackslash}m{#1}}
\newcolumntype{M}[1]{>{\arraybackslash}m{#1}}
\newcommand{\tv}{\mathsf{TV}}
\newcommand{\arc}{\mathsf{AUROC}}
\newcommand\blfootnote[1]{%
  \begingroup
  \renewcommand\thefootnote{}\footnote{#1}%
  \addtocounter{footnote}{-1}%
  \endgroup
}
\newcommand{\newcontent}[1]{\textcolor{black}{#1}}
\newcommand{\revision}[1]{\textcolor{black}{#1}}
\newcommand{\SF}[1]{\textcolor{red}{SF: #1}}




\title{Can AI-Generated Text be Reliably Detected?
}

% Authors must not appear in the submitted version. They should be hidden
% as long as the tmlr package is used without the [accepted] or [preprint] options.
% Non-anonymous submissions will be rejected without review.

\author{\name Vinu Sankar Sadasivan \email vinu@umd.edu \\
      \addr Department of Computer Science,
      University of Maryland
      \AND
      \name Aounon Kumar \email aokumar@hbs.edu \\
      \addr Department of Computer Science,
      Harvard University
      \AND
      \name Sriram Balasubramanian \email sriramb@umd.edu \\
      \addr Department of Computer Science,
      University of Maryland
      \AND
      \name Wenxiao Wang \email wwx@umd.edu \\
      \addr Department of Computer Science,
      University of Maryland
      \AND
      \name Soheil Feizi \email sfeizi@umd.edu \\
      \addr Department of Computer Science,
      University of Maryland}

% The \author macro works with any number of authors. Use \AND 
% to separate the names and addresses of multiple authors.

\newcommand{\fix}{\marginpar{FIX}}
\newcommand{\new}{\marginpar{NEW}}

\def\month{01}  % Insert correct month for camera-ready version
\def\year{2025} % Insert correct year for camera-ready version
\def\openreview{\url{https://openreview.net/forum?id=OOgsAZdFOt}} % Insert correct link to OpenReview for camera-ready version


\begin{document}


\maketitle

% \begin{abstract}
% The rapid progress of Large Language Models (LLMs) has made them capable of performing astonishingly well on various tasks, including document completion and question answering. The unregulated use of these models, however, can potentially lead to malicious consequences such as plagiarism, generating fake news, spamming, etc. Therefore, reliable detection of AI-generated text can be critical to ensure the responsible use of LLMs. Recent works attempt to tackle this problem either using certain model signatures present in the generated text outputs or by applying watermarking techniques that imprint specific patterns onto them. In this paper, we show that these detectors can break in \newcontent{several} scenarios \newcontent{in the presence of an adversary}. In particular, we develop a {\it recursive paraphrasing} attack to \newcontent{stress test these AI text detectors}, which can break a whole range of detection schemes, including the ones using the watermarking schemes as well as neural network-based detectors, zero-shot classifiers, and retrieval-based detectors. Our experiments include passages around 300 tokens in length, showing the sensitivity of the detectors even in the case of relatively long passages. We also observe that our {\it recursive paraphrasing} only degrades text quality slightly --- measured via human studies, and metrics such as perplexity scores and accuracy on text benchmarks --- \newcontent{making it a practical attack for many social engineering applications such as spamming and spreading fake news}. 
% Additionally, we show that even LLMs protected by watermarking schemes can be vulnerable to spoofing attacks adversarially aimed to mislead detectors to classify human-written text as AI-generated, potentially causing reputational damages to the developers. In particular, we show that an adversary can infer hidden AI text signatures of the LLM outputs without having white-box access to the detection method. Finally, we provide a theoretical connection between the AUROC of the best possible detector and the Total Variation distance between human and AI text distributions that can be used to study the fundamental hardness of the reliable detection problem for advanced language models.
% \end{abstract}

\begin{abstract}
\newcontent{
Large Language Models (LLMs) can perform impressively well in various applications, such as document completion and question-answering. 
However, the potential for misuse of these models in activities such as plagiarism, generating fake news, and spamming has raised concerns about their responsible use.  
Consequently, the reliable detection of AI-generated text has become a critical area of research.
Recent works have attempted to address this challenge through various methods, including the identification of model signatures in generated text outputs and the application of watermarking techniques to detect AI-generated text.
These detectors have shown to be effective under their specific settings.
In this paper, we stress-test the robustness of these AI text detectors in the presence of an attacker.
We introduce {\it recursive paraphrasing} attack to stress test a wide range of detection schemes, including the ones using the watermarking as well as neural network-based detectors, zero-shot classifiers, and retrieval-based detectors.
Our experiments conducted on passages, each approximately 300 tokens long, reveal the varying sensitivities of these detectors to our attacks.
We also observe that these paraphrasing attacks add slight degradation to the text quality.
We analyze the trade-offs between our attack strength and the resulting text quality, measured through human studies, perplexity scores, and accuracy on text benchmarks. 
Our findings indicate that while our recursive paraphrasing method can significantly reduce detection rates, it only slightly degrades text quality in many cases, highlighting potential vulnerabilities in current detection systems in the presence of an attacker.
Additionally, we investigate the susceptibility of watermarked LLMs to spoofing attacks aimed at misclassifying human-written text as AI-generated. 
We demonstrate that an attacker can infer hidden AI text signatures without white-box access to the detection method, potentially leading to reputational risks for LLM developers.
Finally, we provide a theoretical framework connecting the AUROC of the best possible detector to the Total Variation distance between human and AI text distributions. 
This analysis offers insights into the fundamental challenges of reliable detection as language models continue to advance.
Our code is publicly available at \url{https://github.com/vinusankars/Reliability-of-AI-text-detectors}.}
\end{abstract}





\section{Introduction}
\label{sec:introduction}
% \begin{itemize}
%     % Diffusion of FL
%     \item {\st{Diffusion of FL}}
%     % Security threats to FL
%     \item {\st{Security threats to FL with particular focus on model poisoning}}
%     % Limitations of existing countermeasures
%     \item {\st{Current countermeasures (e.g., KRUM) and their limitations}}
%     % Proposed method and its advantages
%     \item {\st{Intuitive description of the proposed method and its difference (i.e., advantages) w.r.t. state of the art}}
%     % Main contributions
%     \item {\st{Summary of the main contributions of this work}}
%     % Paper's structure and organization
%     \item {\st{Paper's structure and organization}}
% \end{itemize}

% Diffusion of FL
Recently, {\em federated learning} (FL) has emerged as the leading paradigm for training distributed, large-scale, and privacy-preserving machine learning (ML) systems~\cite{mcmahan2017googleai,mcmahan2017aistats}. 
The core idea of FL is to allow multiple edge clients to collaboratively train a shared, global model without disclosing their local private training data.
%Specifically, an FL system consists of a central server and many edge clients; 
A typical FL round involves the following steps: {\em(i)} the server randomly picks some clients and sends them the current, global model; {\em(ii)} each selected client locally trains its model with its own private data; then, it sends the resulting local model to the server;\footnote{Whenever we refer to global/local model, we mean global/local model {\em parameters}.} {\em(iii)} the server updates the global model by computing an \emph{aggregation function}, usually the average (FedAvg), on the local models received from clients.
% \begin{enumerate}
%     \item[{\em(i)}] the server sends the current, global model to the clients and appoints some of them for training;
%     \item[{\em(ii)}] each selected client locally trains its copy of the global model with its own private data; then, it sends the resulting local model back to the server;\footnote{Whenever we refer to global/local model, we mean global/local model {\em parameters}.}
%     \item[{\em(iii)}] the server updates the global model by computing an \emph{aggregation function} on the local models received from clients (by default, the average, also referred to as FedAvg~\cite{mcmahan2017aistats}).
% \end{enumerate}
This process goes on until the global model converges. %(e.g., after a certain number of rounds or other similar stopping criteria).
%\\
% The advantages of FL over the traditional, centralized learning paradigm are undoubtedly clear in terms of flexibility/scalability (clients can join/disconnect from the FL network dynamically), network communications (only model weights\footnote{We will use \textit{parameters} and \textit{weights} interchangeably.} are exchanged between clients and server), and privacy (each client's private training data is kept local at the client's end and not uploaded to the server).
\\
% Security threats to FL
%However, the growing adoption of FL also raises security concerns~\cite{costa2022covert}, particularly about its confidentiality, integrity, and availability.
Although its advantages over standard ML, FL also raises security concerns~\cite{costa2022covert}. %, particularly about its confidentiality, integrity, and availability~\cite{costa2022covert}.
% OLD, LONG VERSION
% Indeed, some work deals with privacy leakage that may expose the local data of some clients~\cite{melis2019sp}. 
% A large body of work, instead, investigates attacks that usually aim to detriment the predictive accuracy of the learned global model. For instance, \emph{data poisoning} attacks achieve this goal by letting an adversary pollute the training set of some corrupt FL clients with maliciously crafted examples~\cite{jagielski2018sp}.
% Similarly, in \emph{model poisoning} the attacker attempts to tweak the global model weights~\cite{bhagoji2019pmlr} by directly perturbing the local model's weights of some infected FL clients before these are sent to the central server for aggregation, usually via so-called Byzantine attacks. 
% It turns out that Byzantine model poisoning attacks severely impact standard FedAvg; therefore, more robust aggregation functions must be designed to make FL systems secure.
Here, we focus on \emph{untargeted model poisoning} attacks~\cite{bhagoji2019pmlr}, where an adversary attempts to tweak the global model weights %\footnote{We will use the terms \textit{parameters} and \textit{weights} interchangeably.} 
by directly perturbing the local model's parameters of some infected clients before these are sent to the central server for aggregation.
In doing so, the adversary aims to jeopardize the global model \textit{indiscriminately} at inference time.
Such model poisoning attacks severely impact standard FedAvg; therefore, more robust aggregation functions must be designed to secure FL systems.
\\
% In this paper, we focus on designing a novel robust aggregation scheme at the server's end to contrast the effect of Byzantine model poisoning attacks.
%
% Current countermeasures and their limitations
%Several countermeasures have been proposed in the literature to combat model poisoning attacks on FL systems.
% Some methods use simple statistics more robust than plain average to smooth the impact of malicious updates (e.g., Trimmed Mean and FedMedian~\cite{yin2018icml}). 
% Other defenses implement outlier detection techniques to discard malicious updates from the aggregation performed at the server's end. Those are either based on heuristics (e.g., Krum/Multi-Krum~\cite{blanchard2017nips} and Bulyan~\cite{mhamdi2018pmlr}) or data-driven approaches (e.g., K-means clustering~\cite{shen2016acm} or DnC via spectral analysis~\cite{shejwalkar2021ndss}). 
% Finally, some strategies rely on a centralized ``source of trust'' to spot potential malicious updates (e.g., FLTrust~\cite{cao2020fltrust}).
% Several countermeasures have been proposed in the literature to combat model poisoning attacks on FL systems, i.e., to discard possible malicious local updates from the aggregation performed at the server's end. 
% These techniques range from simple statistics more robust than plain average (e.g., Trimmed Mean and FedMedian~\cite{yin2018icml}) to outlier detection heuristics (e.g., Krum/Multi-Krum~\cite{blanchard2017nips} and Bulyan~\cite{mhamdi2018pmlr}) or data-driven approaches (e.g., spectral analysis via K-means clustering~\cite{shen2016acm} or spectral analysis), or methods based on ``source of trust'' (e.g., FLTrust~\cite{cao2020fltrust}).
% OLD, LONG VERSION
%Several countermeasures have been proposed in the literature to combat Byzantine model poisoning attacks on FL systems.
% Descriptive statistics
% For example, Trimmed Mean and FedMedian aggregate local model updates using more robust statistics than standard average~\cite{yin2018icml}.
%
% % Heuristics for outlier detection
% Many existing Byzantine-resilient strategies implement some outlier detection heuristics to discard the model updates sent by potentially malicious clients from the input of the aggregation function.
% One of the most popular heuristics is Krum~\cite{blanchard2017nips}.
% This strategy tries to mitigate the impact of Byzantine attacks by selecting as a global model the local model with the smallest sum of Euclidean distances to {\em all} the other local models.
% Although powerful, Krum requires the server to know (or, at least, estimate) the number of malicious FL clients upfront, which is generally impossible in a realistic attack scenario. %
% Moreover, Krum may become ineffective for complex, high-dimensional model parameter spaces due to the curse of dimensionality.
% Bulyan~\cite{mhamdi2018pmlr} tries to overcome this issue by combining Krum with a variant of Trimmed Mean.
% % Data-driven outlier detection
% Other strategies use data-driven outlier detection techniques -- e.g., via K-means clustering~\cite{shen2016acm} -- to spot potential malicious local model updates. 
% %For instance, Shen et al. propose to cluster local model updates with K-means and thus identify outliers.
%
% % Other techniques
% As far as the server is concerned, any local model received can be from a potential malicious client. 
% FLTrust~\cite{cao2020fltrust} assumes the server acts as a client, i.e., trains a local model on an additional {\em trustworthy} dataset at the server's end and compares it against all the local models from other clients. 
% This way, the server can rely on some ``source of trust'' when discarding potentially malicious clients.
%\\
% Limitations of existing Byzantine-resilient strategies
Unfortunately, existing defense mechanisms either rely on simple heuristics (e.g., Trimmed Mean and FedMedian by~\cite{yin2018icml}) or need strong and unrealistic assumptions to work effectively (e.g., foreknowledge or estimation of the number of malicious clients in the FL system, as for Krum/Multi-Krum~\cite{blanchard2017nips} and Bulyan~\cite{mhamdi2018pmlr}, which, however, cannot exceed a fixed threshold).
Furthermore, outlier detection methods using K-means clustering~\cite{shen2016acm} or spectral analysis like DnC~\cite{shejwalkar2021ndss} do not directly consider the temporal evolution of local model updates received.
Finally, strategies like FLTrust~\cite{cao2020fltrust} require the server to collect its own dataset and act as a proper client, thereby altering the standard FL protocol.
\\
% OLD, LONG VERSION
% Overall, existing Byzantine-resilient strategies are either simple heuristics (e.g., FedMedian) or, if they are more complex, they rely on strong and unrealistic assumptions to work effectively (e.g., knowing the number of malicious clients in the FL system in advance, as for Krum and alike).
% Furthermore, data-driven outlier detection methods do not consider the temporary evolution of local model updates received (e.g., K-means clustering). 
% Finally, strategies like FLTrust requires the server to collect its own dataset and act as a proper client, thereby altering the standard FL protocol.
%
% Description of the proposed method
This work introduces a novel pre-aggregation \textit{filter} robust to untargeted model poisoning attacks. Notably, this filter $(i)$ operates without requiring prior knowledge or constraints on the number of malicious clients and $(ii)$ inherently integrates temporal dependencies. 
The FL server can employ this filter as a preprocessing step before applying \textit{any} aggregation function, be it standard like FedAvg or robust like Krum or Bulyan.
Specifically, we formulate the problem of identifying corrupted updates as a multidimensional (i.e., matrix-valued) time series anomaly detection task. 
The key idea is that legitimate local updates, resulting from well-calibrated iterative procedures like stochastic gradient descent (SGD) with an appropriate learning rate, show \textit{higher predictability} compared to malicious updates. This hypothesis stems from the fact that the sequence of gradients (thus, model parameters) observed during legitimate training exhibit regular patterns, as validated in Section~\ref{subsec:intuition}. %until convergence. 
%This regularity may be more pronounced for smooth convex loss functions, but it can still be captured within an appropriate time window, even for more complex and convoluted loss surfaces. 
%We provide evidence of this claim in Appendix~B, where we show that the average mutual information (i.e., ``predictability''), calculated over pairs of legitimate model updates sent at different FL rounds, is significantly higher than the corresponding computation for a malicious client.
\\
Inspired by the matrix autoregressive (MAR) framework for multidimensional time series forecasting~\cite{chen2021je}, we propose the FLANDERS ({\em \textbf{F}ederated \textbf{L}earning meets \textbf{AN}omaly \textbf{DE}tection for a \textbf{R}obust and \textbf{S}ecure}) filter.
The main advantages of FLANDERS over existing strategies like FLDetector~\cite{zhao2020multivariate} are its resilience to large-scale attacks, where $50\%$ or more FL participants are hostile, and the capability of working under realistic non-iid scenarios.
We attribute such a capability to two key factors: $(i)$ FLANDERS works without knowing a priori the ratio of corrupted clients, and $(ii)$ it embodies temporal dependencies between intra- and inter-client updates, quickly recognizing local model drifts caused by evil players. Below, we summarize our main contributions:

\begin{itemize}
\item[{\em(i)}]
We provide empirical evidence that the sequence of models sent by legitimate clients is more predictable than those of malicious participants performing untargeted model poisoning attacks.
\\
\item[{\em(ii)}] 
We introduce FLANDERS, the first pre-aggregation filter for FL robust to untargeted model poisoning based on multidimensional time series anomaly detection.
\\
\item[{\em(iii)}] 
We integrate FLANDERS into Flower,\footnote{\scriptsize{\url{https://flower.dev/}}} a popular FL simulation framework for reproducibility.
\\
\item[{\em(iv)}] 
We show that FLANDERS improves the robustness of the existing aggregation methods under multiple settings: different datasets, client's data distribution (non-iid), models, and attack scenarios.
\\
\item[{\em(v)}] 
We publicly release all the implementation code of FLANDERS along with our experiments.\footnote{\scriptsize{\url{https://anonymous.4open.science/r/flanders_exp-7EEB}}}
\end{itemize}

% Paper's structure and organization
The remainder of the paper is structured as follows. %some related work and the current state-of-the-art solutions to security issues that FL entails. 
Section~\ref{sec:background} covers background and preliminaries. 
In Section~\ref{sec:related}, we discuss related work.
Section~\ref{sec:problem} and Section~\ref{sec:method} describe the problem formulation and the method proposed. % to tackle it. 
Section~\ref{sec:experiments} gathers experimental results. %, and Section~\ref{sec:limitations} discusses some limitations of this work.
Finally, we conclude in Section~\ref{sec:conclusion}.
 %discusses the limitations of this work and draws future research directions.
%reports conclusions and draws perspectives for future research directions.

%%%%%%% OLD %%%%%%%
%to overcome the resilience of Byzantine failures in distributed Stochastic Gradient Descent computations. 
% The strength of Krum is its time complexity, which is linear in the gradient dimension. 
% However, the robustness of the approach is guaranteed for gradient-based learning applications only when the majority of the clients are not compromised. 
% Besides, the aggregation mechanism of Krum, as well as that of similar methods, is robust from a coarse-grained perspective and does not provide solutions to errors and perturbations that may occur at inference time.
%A related approach to~\cite{blanchard2017nips} is the work of Su et al.~\cite{su2016dc}. Here, the authors propose an iterated approximate agreement to tackle a multi-layer scenario attacked by Byzantine agents. 
%However, the method works efficiently on the sole discrete context and it is inapplicable to continuous state environments.
%\gabri{Maybe, we should just talk about the main limitations of existing countermeasures without digging into their details (or, we can just mention Krum as this is the most popular one). I will move the description of all these methods to the Related Work section.}
\section{Evading AI-Detectors using Paraphrasing Attacks}
\label{sec:aigentextnotdetected}

In this section, we first present the experimental setup for our paraphrasing attacks in \S\ref{sec:exp-setup}. We also provide experiments in \S\ref{sec:quality} showing that our novel recursive paraphrasing attacks only lead to a slight degradation in text quality. \S\ref{sec:watermark_evade} and \S\ref{sec:paraphrase_nonwatermark} show the effect of the paraphrasing attacks on watermarking and non-watermarking detectors, respectively.


\subsection{Setup and Paraphrasing Methods}
\label{sec:exp-setup}


\begin{wrapfigure}{r}{0.35\textwidth}
    \centering
    \vspace{-6mm}
    \hspace{5mm}
    \includegraphics[width=0.6\linewidth]{images/rec-para.png}
    \vspace{-2mm}
    \caption{Recursive paraphrasing}
    \label{fig:rec-para}
    % \vspace{-2mm}
\end{wrapfigure}
We use the ``document'' features of the XSum dataset \citep{xsum} containing 1000 long news articles ($\sim$300 tokens in length) for our experiments. In Appendix~\ref{app:moreexps}, we perform experiments with additional datasets -- a medical text dataset PubMedQA \citep{jin-etal-2019-pubmedqa} and a dataset with articles from 10 different domains Kafkai \citep{kafkai}. 
As target LLMs, we use OPT-1.3B and OPT-13B \citep{opt} language models with 1.3B and 13B parameters, respectively.
In Appendix~\ref{app:moreexps}, we also evaluate our attacks with GPT-2 Medium \citep{gpt2} as the target model.
We use three different neural network-based paraphrasers -- DIPPER with 11B parameters \citep{krishna2023paraphrasing}, LLaMA-2-7B-Chat with 7B parameters \citep{llama}, and T5-based paraphraser \citep{prithivida2021parrot} with 222M parameters.
Suppose a passage $S = (s_1, s_2, ..., s_n)$ where $s_i$ is the $i^{th}$ sentence. DIPPER and LLaMA-2-7B-Chat paraphrase $S$ to be $S' = f_{strong}(S)$ in one-shot while the light-weight T5-based paraphraser would output $S' = (f_{weak}(s_1), f_{weak}(s_2), ..., f_{weak}(s_n))$ where they can only paraphrase sentence-by-sentence. DIPPER also has the ability to input a context prompt text $C$ to generate higher-quality paraphrasing $S' = f_{strong}(S, C)$. We can also vary two different hyperparameters of DIPPER to generate a diverse number of paraphrases for a single input passage.





\begin{table}[t]
    \centering\renewcommand\cellalign{lc}
    \setcellgapes{3pt}\makegapedcells
    \begin{adjustbox}{max width=\textwidth}
    \begin{tabular}{c|c||c|c|c|c|c||c}
    \toprule
    \multicolumn{2}{c||}{\textbf{ppi}} & i=1 & i=2 & i=3 & i=4 & i=5 & All ppi \\ \midrule \midrule
    \multirow{2}{*}{\makecell{\textbf{Content}\\\textbf{preservation}}} & Avg. rating & {$4.0\pm0.8$} & {$4.1\pm0.8$} & {$3.9\pm0.9$} & {$4.2\pm0.9$} & {$3.7\pm1.1$} & {$4.0\pm0.9$} \\ \cline{2-8}
    {} & Ratings 5\&4 & {$70.2\%$} & {$77.2\%$} & {$63.2\%$} & {$80.0\%$} & {$61.4\%$} & {{$70.4\%$}} \\ \midrule
    \multirow{2}{*}{\makecell{\textbf{Grammar or}\\\textbf{text quality}}} & Avg. rating & {$4.28 \pm 0.67$} & {$4.12 \pm 0.50$} & {$4.12 \pm 0.53$} & {$4.11 \pm 0.64$} & {$4.07 \pm 0.53$} & {{$ 4.14\pm 0.58$}}\\ \cline{2-8}
    {} & Ratings 5\&4 & {$87.72\%$} & {$92.98\%$} & {$91.23\%$} & {$84.21\%$} & {$89.47\%$} & {{$89.12\%$}}\\ 
    \bottomrule
    \end{tabular}
    \end{adjustbox}
    \caption{Summary of the MTurk human evaluation study on content preservation and grammar or text quality of the recursive paraphrases with DIPPER that we use for our attacks. Ratings are on a Likert scale of 1 to 5. See Appendix~\ref{app:human_study} for details.}
    \label{tab:summary-mturk}
    
    \vspace{2mm}
    \centering\renewcommand\cellalign{lc}
    \setcellgapes{3pt}\makegapedcells
    \begin{adjustbox}{max width=\textwidth}
    \begin{tabular}{c|c||c|c|c|c|c||c}
    \toprule
    \multicolumn{2}{c||}{\textbf{ppi}} & i=1 & i=2 & i=3 & i=4 & i=5 & {All ppi} \\ \midrule \midrule
    \multirow{2}{*}{\makecell{\textbf{Content}\\\textbf{preservation}}} & Avg. rating & {$4.37\pm0.63$} & {$4.18\pm0.67$} & {$3.93\pm0.71$} & {$3.9\pm0.75$} & {$3.85\pm0.78$} & {{$4.05\pm0.2$}} \\ \cline{2-8}
    {} & Ratings 5\&4 & {$91.67\%$} & {$85.0\%$} & {$80.0\%$} & {$78.3\%$} & {$80.0\%$} & {{$83.0\%$}} \\ \midrule
    \multirow{2}{*}{\makecell{\textbf{Grammar or}\\\textbf{text quality}}} & Avg. rating & {$4.62 \pm 0.55$} & {$4.28\pm 0.73$} & {$4.26 \pm 0.65$} & {$4.22 \pm 0.64$} & {$4.17\pm0.74$} & {{$ 4.31\pm 0.35$}}\\ \cline{2-8}
    {} & Ratings 5\&4 & {$96.67\%$} & {$83.33\%$} & {$88.33\%$} & {$88.3\%$} & {$83.33\%$} & {{$88.0\%$}}\\ 
    \bottomrule
    \end{tabular}
    \end{adjustbox} 
    \caption{
    Summary of the MTurk human evaluation study of the recursive paraphrases with LLaMA-2-7B-Chat.}
    \label{tab:summary-mturk-llama}

    \vspace{2mm}
    \begin{adjustbox}{max width=.76\textwidth}
    \begin{tabular}{l|l|c|c|c|c|c|c}
    \toprule
    \multicolumn{1}{c|}{\textbf{Paraphraser}} &
      \multicolumn{1}{c|}{\textbf{Evaluation}} &
      \textbf{AI text} &
      \textbf{pp1} &
      \textbf{pp2} &
      \textbf{pp3} &
      \textbf{pp4} &
      \textbf{pp5} \\ \midrule \midrule
    \multirow{2}{*}{DIPPER}     & Mean perplexity       & 5.2  & 7.7  & 7.8  & 8.5  & 7.7  & 8.7    \\ \cmidrule{2-8} 
                                & QA performance & 97\% & 97\% & 96\% & 96\% & 96\% & 95.5\% \\ \midrule
    \multirow{2}{*}{LLaMA-2-7B-Chat} & Mean perplexity       & 5.2  & 8.1  & 9.3  & 9.0  & 10.3 & 10.5   \\ \cmidrule{2-8} 
                                & QA Performance & 97\% & 97\% & 97\% & 97\% & 97\% & 97\%   \\ \bottomrule
    \end{tabular}
    \end{adjustbox}
    \caption{
    Automated evaluation of the text quality of recursive paraphrases using perplexity measures with respect to OPT-13B and question-answering benchmark accuracy.}
    \label{tab:quality-eval}
    
\end{table}




We use DIPPER and LLaMA-2-7B-Chat for recursive paraphrasing attacks since they provide high-quality paraphrasing when compared to the 222M parameter T5 model.
Let an LLM $L$ generate AI text output $S = L(C)$ for an input prompt $C$. DIPPER or LLaMA-2-7B-Chat can be used to generate a paraphrase \texttt{pp1} $=f_{strong}(S, C)$. This paraphrasing can be performed in recursion (see Figure~\ref{fig:rec-para}). That is, \texttt{ppi} $= f_{strong}($\texttt{pp(i-1)}$, C)$. 

While DIPPER is explicitly trained to be a paraphraser, LLaMA-2-7B-Chat is an instruction-tuned model for chat purposes.
We design a system prompt (see Appendix~\ref{app:systemprompts}) with the LLaMA-2-7B-Chat model to use it as a paraphraser.
In \S\ref{sec:watermark_evade} and \S\ref{sec:paraphrase_nonwatermark}, we show that recursive paraphrasing is effective in evading the strong watermark and retrieval-based detectors when compared to a single round of paraphrasing. 
Using human and other automated evaluation techniques in \S\ref{sec:quality}, we show that our recursive paraphrasing method only degrades the text quality slightly.

\subsection{Quality of the Paraphrases}
\label{sec:quality}

% \noindent{\bf Quality of the paraphrased passages.}
In order to reliably study the quality of the recursive paraphrases we use in our experiments using DIPPER and LLaMA-2-7B-Chat, we perform human evaluations using MTurk and other automated techniques. 
The AI-text used in this study is generated using a watermarked OPT-13B model. 
Tables~\ref{tab:summary-mturk}~and~\ref{tab:summary-mturk-llama} provide a summary of the study.  We investigate the content preservation and text quality or grammar of the recursive paraphrases with respect to the AI-generated texts (see Tables~\ref{tab:mturk-content}-\ref{tab:mturk-grammar-llama} in Appendix \ref{app:human_study} for more details). 
{\bf In terms of content preservation with DIPPER, \bm{$70\%$} of the paraphrases were rated high quality and \bm{$23\%$} somewhat equivalent. In terms of text quality or grammar, \bm{$89\%$} of the paraphrases were rated high quality.} 
On a Likert scale of 1 to 5, the DIPPER paraphrases that we use received an average rating of $4.14 \pm 0.58$ for text quality or grammar and $4.0 \pm 0.9$ for content preservation. 
\textbf{Similarly, \bm{$83\%$} of the recursive paraphrases we obtain with LLaMA-2-7B-Chat were rated high quality.}
See Appendix \ref{app:human_study} for more details on the human study.


For automated text quality evaluations, we use perplexity measures and a question-answering (QA) benchmark in Table~\ref{tab:quality-eval}.  
We measure the perplexity scores using OPT-13B. 
As shown in the table, the perplexity scores degrade slightly from 5.5 to 8.7 and 10.5, respectively, for DIPPER and LLaMA-2-7B-Chat after 5 rounds of paraphrasing.
We also use a QA benchmark SQuAD-v2 \citep{squadv2} to evaluate the effect of recursive paraphrasing. For this, we use two hundred data points from SQuAD-v2. Each data point consists of a context passage, a question, and an answer.
We evaluate a QA model on the SquAD-v2 benchmark to observe that it achieves $97\%$ accuracy in the QA task.
For the QA model, we use the LLaMA-2-13B-Chat model with a carefully written system prompt (see Appendix~\ref{app:systemprompts}).
To evaluate the quality of paraphrases, we paraphrase the context passages recursively and use these to evaluate the QA accuracy with the QA model.
If the QA model can answer the question correctly based on the paraphrased context, then the information is preserved in the paraphrase.
As we observe, the QA performance with recursive paraphrasing is similar to that with the clean context passage.
These results confirm that AI text detectors can be effectively attacked using recursive paraphrasing with only a slight degradation in text quality. 




\begin{figure}[h!]
\centering
\begin{subfigure}{.495\textwidth}
  \centering
  \adjustimage{width=1\linewidth, left}{images/wm_opt13b_dipper.png}
  % \vspace{-2mm}
  \caption{Recursive paraphrasing with DIPPER}
  \label{fig:watermark-opt13-dipper}
\end{subfigure}% 
\hfill
\begin{subfigure}{.495\textwidth}
  \centering
  \adjustimage{width=1\linewidth, right}{images/wm_opt13b_llama.png}
  % \vspace{-2mm}
  \caption{Recursive paraphrasing with LLaMA-2-7B-Chat}
  \label{fig:watermark-opt13-llama}
\end{subfigure}
% \vspace{-2mm}
\caption{ROC plots for soft watermarking with recursive paraphrasing attacks. AUROC, TPR@1$\%$FPR, and perplexity scores measured using OPT-13B are given in the legend. The target LLM OPT-13B is used to generate watermarked output that are 300 tokens in length.}
\label{fig:watermarkroc-op13b}
\vspace{-0mm}
\end{figure}

\begin{figure}[h!]
\centering
\begin{subfigure}{.495\textwidth}
  \centering
  \adjustimage{width=1\linewidth, left}{images/rebuttal_wm_rec_paraphrase.png}
  % \vspace{-2mm}
  \caption{Watermarked text with mean token length 300}
  \label{fig:watermarkroc1}
\end{subfigure}% 
\hfill
\begin{subfigure}{.495\textwidth}
  \centering
  \adjustimage{width=1\linewidth, right}{images/rebuttal_wm_rec_paraphrase_vary_tokens.png}
  % \vspace{-2mm}
  \caption{Watermarked text with varying token lengths}
  \label{fig:watermarkroc2}
\end{subfigure}
% \vspace{-2mm}
\caption{ROC plots for soft watermarking with recursive paraphrasing attacks. AUROC, TPR@1$\%$FPR, and perplexity scores measured using OPT-13B are given in the legend. The target LLM is OPT-1.3B. (a) Even for 300 tokens long watermarked passages, recursive paraphrasing is effective. As paraphrasing rounds proceed, detection rates degrade significantly with a slight trade-off in text quality. (b) Attacking watermarked passages become easier as their length reduces.}
\label{fig:watermarkroc}
\vspace{-0mm}
\end{figure}

\subsection{Paraphrasing Attacks on Watermarked AI Text}
\label{sec:watermark_evade}

In this section, we evaluate our recursive paraphrasing attacks on the soft watermarking scheme proposed in \cite{kirchenbauer2023watermark}.  Soft watermarking encourages LLMs to output token $s^{(t)}$ at time-step $t$ that belongs to a ``green list''. The green list for $s^{(t)}$ is created using a private pseudo-random generator that is seeded with the prior token $s^{(t-1)}$. A watermarked output from the LLM is designed to have tokens that are majorly selected from the green list. Hence, a watermark detector with the pseudo-random generator checks the number of \textit{green} tokens in a candidate passage to detect whether it is watermarked or not. 
Here, we target watermarked OPT-13B with 13B parameters in Figure~\ref{fig:watermarkroc-op13b} and watermarked OPT-1.3B in Figure~\ref{fig:watermarkroc} for our experiments. In Appendix~\ref{app:moreexps_wm}, we also evaluate our attacks on GPT-2 Medium \citep{gpt2} and other datasets -- PubMedQA \citep{jin-etal-2019-pubmedqa} and Kafkai \citep{kafkai}. 

\noindent{\bf Dataset.} We perform our experiments on 2000 text passages that are around 300 tokens in length \textcolor{black}{(1000 passages per human and AI text classes)}. We pick 1000 long news articles from the XSum ``document'' feature. For each article, the first $\sim$300 tokens are input to the target OPT-1.3B \textcolor{black}{to generate 1000 watermarked AI text passages that are each $\sim$300 tokens in length. The second 300 tokens from the 1000 news articles in the dataset are treated as baseline human text.} We note that our considered dataset has more and longer passages compared to the experiments in \cite{kirchenbauer2023watermark}.

\noindent{\bf Detection results after paraphrasing attack.}
Weaker paraphrasing attacks discussed in \cite{kirchenbauer2023watermark} are not effective in removing watermarks. They perform ``span replacement'' by replacing random tokens (in-place) using a language model. 
However, after a single round of paraphrasing (\texttt{pp1}) with a watermarked OPT-13B as the target LLM, TPR@$1\%$FPR of watermark detector degrades from $99.8\%$ to $80.7\%$ and $54.6\%$, respectively, with DIPPER and LLaMA-2-7B-Chat paraphrasers. 
We also show that our stronger recursive paraphrasing attacks can effectively evade watermark detectors with only a slight degradation in text quality. 
As shown in Figures~\ref{fig:watermarkroc-op13b}-\ref{fig:watermarkroc}, the recursive paraphrase attack further degrades the detection rate of the detector to below $20\%$ after 5 rounds of paraphrasing (\texttt{pp5}). 
Note that in all the settings \texttt{pp2} or 2 rounds of paraphrasing is sufficient to degrade TPR@$1\%$FPR to below $50\%$. 
As shown in Figure~\ref{fig:watermarkroc-op13b}, DIPPER shows a clearer and more consistent trend in improving attack performance over recursions of paraphrasing in comparison to LLaMA-2. This is because DIPPER is trained explicitly to be a paraphraser with hyperparameters that can control the quality of paraphrasing. Therefore, we mainly employ DIPPER for our recursive paraphrase attacks.
\texttt{Best of ppi} in the figure refers to the method where, for each passage, we select the paraphrase out of all the \texttt{ppi}'s that has the worst detector score. For \texttt{Best of ppi} with OPT-1.3B,  the detection rate reduces drastically from $99.8\%$ to $4.0\%$ with only a trade-off of $1.5$ in the perplexity score (Figure~\ref{fig:watermarkroc1}). 
\texttt{Best of ppi}, unlike the \texttt{ppi} attacks, assume black box query access to the detector.
Figure \ref{fig:watermarkroc2} shows that the watermarking detector becomes weaker as the length of the watermarked text reduces. Note that for watermarked texts that are 50 or 100 tokens long, the detection performance after the recursive paraphrasing attack is similar to that of a random detector. We provide examples of paraphrased text that we use for our attacks in Appendix \ref{app:example-paraphrases}.



\subsection{Paraphrasing Attacks on Non-Watermarked AI Text}
\label{sec:paraphrase_nonwatermark}

Neural network-based trained detectors such as RoBERTa-Large-Detector from OpenAI \citep{openaidetectgpt2} are trained or fine-tuned for binary classification with datasets containing human and AI-generated texts. Zero-shot classifiers leverage specific statistical properties of the source LLM outputs for their detection. Retrieval-based methods search for a candidate passage in a database that stores the LLM outputs. Here, we perform experiments on these non-watermarking detectors to show they are vulnerable to our paraphrasing attack.\looseness=-1


\begin{figure}[t]
    \centering
    \includegraphics[width=1\textwidth]{images/nonwatermark_roc_new.png}
    \vspace{-2mm}
   \caption{ROC curves for various trained and zero-shot detectors. \textbf{Left:} Without attack. \textbf{Middle:} After paraphrasing attack using T5-based paraphraser. The performance of zero-shot detectors drops significantly. \textbf{Right:} Here, we assume we can query the detector ten times for the paraphrasing attack. We generate ten paraphrasings for each passage and query multiple times to evade detection. Notice how all detectors have low TPR@$1\%$FPR. In the plot legend -- \texttt{perturbation} refers to the zero-shot methods in \cite{mitchell2023detectgpt}; \texttt{threshold} refers to the zero-shot methods in \cite{solaiman2019release, gehrmann2019gltr, ippolito2019automatic}; \texttt{roberta} refers to OpenAI's trained detectors \citep{openaidetectgpt2}. The \newcontent{TPR@$1\%$FPR} scores of different detectors before the attack, after the attack, and after the attack with multiple queries, respectively, are provided in the plot legend.}
    \label{fig:roc_nonwatermark}
    \vspace{-5mm}
\end{figure}



\begin{wrapfigure}{r}{0.5\textwidth}
        \centering
        \vspace{-8mm}
        \includegraphics[width=1.03\linewidth]{images/IR_attack.png}
        \caption{\textcolor{black}{Recursive paraphrasing breaks the retrieval-based detector \citep{krishna2023paraphrasing} with only slight degradation in text quality. \texttt{ppi} refers to $\texttt{i}$ recursion(s) of paraphrasing. Numbers next to markers denote the perplexity scores of the paraphraser output.}}
        \vspace{-4mm}
    \label{fig:ir-attack}
\end{wrapfigure}
\textbf{Trained and Zero-shot detectors.} We use a pre-trained GPT-2 Medium
% \footnote{\url{https://huggingface.co/gpt2-medium}}
model \citep{gpt2} with 355M parameters as the target LLM to evaluate our attack on \textcolor{black}{1000} long passages from the XSum dataset \citep{xsum}. We use the T5-based paraphrasing model \citep{prithivida2021parrot} with 222M parameters to rephrase the \textcolor{black}{1000} output texts generated using the target GPT-2 Medium model. Figure~\ref{fig:roc_nonwatermark} shows the effectiveness of the paraphrasing attack over these detectors. {\bf The AUROC scores of DetectGPT \citep{mitchell2023detectgpt} drop from \bm{$96.5\%$} (before the attack) to \bm{$59.8\%$} (after the attack).} Note that AUROC of $50\%$ corresponds to a random detector. The rest of the zero-shot detectors \citep{solaiman2019release, gehrmann2019gltr, ippolito2019automatic} also perform poorly after our attack. Though the performance of the trained neural network-based detectors \citep{openaidetectgpt2} is better than that of zero-shot detectors, they are also not reliable. For example,\newcontent{ TPR@$1\%$FPR of OpenAI's RoBERTa-Large-Detector drops from $100\%$ to around $92\%$ after our attack.}






In another setting, we assume the attacker may have multiple access to the detector. That is, the attacker can query the detector with an input AI text passage, and the detector would reveal the detection score to the attacker. For this scenario, we generate ten different paraphrases for an input passage and query the detector for the detection scores. For each AI text passage, we then select the paraphrase with the worst detection score for evaluating the ROC curves. As shown in Figure~\ref{fig:roc_nonwatermark}, \newcontent{\bf with multiple queries to the detector, an adversary can paraphrase more efficiently to bring down TPR@\bm{$1\%$}FPR of the RoBERTa-Large-Detector from \bm{$100\%$} to \bm{$80\%$}.} In Appendix~\ref{app:moreexps_zs}, we show more experiments with more datasets and target LLMs.



\textbf{Retrieval-based detectors.} Detector in \citet{krishna2023paraphrasing} is designed to be robust against paraphrase attacks. However, we show that they can suffer from the recursive paraphrase attacks that we develop using DIPPER. 
\textcolor{black}{We use 2000 passages (1000 generated by OPT-1.3B and 1000 human passages) from the XSum dataset. AI outputs are stored in the AI database by the detector. As shown in Figure \ref{fig:ir-attack}, this detector detects almost all of the AI outputs even after a round of paraphrasing. However, \textbf{the detection accuracy drops below \bm{$\sim60\%$} after five rounds of recursive paraphrasing.} As marked in the plot, the perplexity score of the paraphrased text only degrades by $1.7$ at a detection accuracy of $\sim60\%$.} 
Moreover, retrieval-based detectors are concerning since they might lead to \textbf{serious privacy issues} from storing users' LLM conversations. In Appendix~\ref{app:moreexps_retrieval}, we show more experiments with more datasets and target LLMs.
\section{Spoofing Attacks on Generative AI-text Models}
\label{sec:humantextdetected}


An AI language detector without a low type-I error can cause harm as it might wrongly accuse a human of plagiarizing using an LLM. Moreover, an attacker ({\it adversarial human}) can generate a non-AI text to be detected as AI-generated. This is called the {\it spoofing attack}. An adversary can potentially launch spoofing attacks to produce derogatory texts to damage the reputation of the target LLM's developers. In this section, as a proof-of-concept, we show that current text detectors can be spoofed to detect texts composed by adversarial humans as AI-generated. More details on the spoofing experiments are presented in Appendix \ref{app:spoofing}.

\begin{wrapfigure}{r}{0.5\textwidth}
    \centering
    \vspace{-6mm}
    % \hspace{5mm}
   \includegraphics[width=1\linewidth,]{images/wm_spoof.png}
   \vspace{-8mm}
    \caption{ROC curve of a soft watermarking-based detector \citep{kirchenbauer2023watermark} after our spoofing attack.}
    \label{fig:watermark-roc-spoof}
    \vspace{-4mm}
\end{wrapfigure}
\textbf{Soft watermarking.} As discussed in \S \ref{sec:aigentextnotdetected}, soft watermarked LLMs \citep{kirchenbauer2023watermark} generate tokens from the ``green list'' that are determined by a pseudo-random generator seeded by the prefix token. Though the pseudo-random generator is private, an attacker can estimate the green lists by observing multiple token pairs in the watermarked texts from the target LLM. An adversarial human can then leverage the estimated green lists to compose texts by themselves that are detected to be watermarked. In our experiments, we estimate the green lists for 181 most commonly used words in the English vocabulary. We query the target watermarked OPT-1.3B model one million times to observe the token pair distributions within this smaller vocabulary subset we select. 
Note that this attack on a watermarked model only needs to be performed once to learn the watermarking pattern to spoof it thereafter. 
Based on the frequency of tokens that follow a prefix token in the observed generative outputs, we estimate green lists for each of the 181 common words. We build a tool that helps adversarial humans create watermarked
sentences by providing them with the proxy green list we learn.
We observe that the \textbf{soft watermarking scheme can be spoofed to degrade its detection AUROC from \bm{$99.8\%$} to \bm{$1.3\%$}} (see Figure~\ref{fig:watermark-roc-spoof}).



\textbf{Retrieval-based detectors.} \cite{krishna2023paraphrasing} use a database to store LLM outputs to detect AI-text by retrieval. We find in our experiments (see Figure~\ref{fig:irapp}) that an \textbf{adversary can spoof this detector \bm{$100\%$} of the time, even if the detector maintains a private database}. Suppose an adversary, say a teacher, has access to a human written document $S$, say a student's essay. The adversary can prompt the target LLM to paraphrase $S$ to get $S'$. This results in the LLM, by design, storing its output $S'$ in its private database for detection purposes. Now, the detector would classify the original human text $S$ as AI-generated since a semantically similar copy $S'$ is present in its database. In this manner, a teacher can purposefully allege an innocent student to have plagiarised using the retrieval-based detector. 
Note that manipulating retrieval-based detectors is easier using this approach compared to watermarking techniques. 
This observation implies a practical tradeoff between type-I and type-II errors. 
When a detector is strengthened against type-II errors, it tends to result in a deterioration of its performance in terms of type-I errors.



\textbf{Zero-shot and neural network-based detectors.} In this setting, a malicious adversary could write a short text in a collaborative work, which may lead to the entire text being classified as AI-generated. To simulate this, we prepend a human-written
text marked as AI-generated by the detector to all the other human-generated text for spoofing. In other words, from 200 long passages in the XSum dataset, we pick the human text with the worst detection score for each detector considered in \S \ref{sec:paraphrase_nonwatermark}. We then prepend this text to all the other human
texts, ensuring that the length of the prepended text does not exceed the length of the original text. Our experiments show that the \textbf{AUROC of all these detectors drops after spoofing} (see plots in Appendix \ref{app:spoofing}). After this
na\"ive spoofing attack, the TPR@$1\%$FPR of most of these detectors drop significantly.
% !TEX root = ./CauchyCombination.tex
\section{Multiple Hypothesis Testing} \label{secPrelims}

This section introduces the notation on multiple hypothesis testing and the benchmark procedures for addressing the multiple testing problem. 

\subsection{Setting}
Let $H_{i}$ denote the $i^{\text{th}}$ null hypothesis of interest, with $i=1,...,d$,  
and $d$ being the total number of individual hypotheses. To test the $d$ hypotheses, we can use the associated vector of test statistics $\bm{X}=(X_{1},X_{2},\ldots,X_{d})^{^{\prime }}$, one for each hypothesis being tested, or the corresponding raw $p$-values $p_{1},\ldots ,p_{d}$. The test statistics can be independent or % corrected.
correlated. 

%In some cases, like in Section \ref{secApplDriftBurst}, the test statistics are constructed from rolling windows and are extremely serially correlated. 

The first task is to test the global null hypothesis. Let $\mathcal{H}_{0}$ be the collection of null hypotheses of interest. 
The strategy of a classical global test is to abandon the multiplicity issue altogether and replace multiple tests with the global null hypothesis that all elementary hypotheses are true.  The alternative is that at least one elementary hypothesis is false. For example, in high-frequency financial econometrics,  we often need to monitor the presence of certain events (e.g., jumps or drift bursts) within a fixed time period (e.g., within a day). The global null is that there is no occurrence of such an event at all (e.g., 
% in Example 2, none of the stocks has a significant alpha
in Example 1, there is no drift burst within the day or in Example 2, none of the stocks has a significant alpha).
The goal is to get $\alpha$-level control under this global null, i.e., $P_{\mathcal{H}_0} [\text{reject}\, \mathcal{H}_0] \leq \alpha$. The test is conservative when $P_{\mathcal{H}_0} [\text{reject}\, \mathcal{H}_0]$ is strictly less than the theoretical upper bound $\alpha$ and ideal when it is equal to  $\alpha$. 

When, by any  test, the global null $\mathcal{H}_0$ is rejected, the second task is to identify which of the elementary hypotheses $H_{i}$ should be rejected. The set of true hypotheses $\mathcal{T}$, the set of false hypotheses $\mathcal{F}$ and the set of rejected hypotheses $\mathcal{R}$ are defined as: 
\begin{align}
	\begin{split}  \label{eqLocalHypothesis}
		\mathcal{T} &= \{ H_{i}\in \mathcal{H}_0: H_{i} \, \text{is true}\}, \\
		\mathcal{F} &= \{ H_{i}\in \mathcal{H}_0: H_{i} \, \text{is false}\}, \text{and} \\
		\mathcal{R} &= \{H_{i}\in \mathcal{H}_0: H_{i} \, \text{is rejected}\}.
	\end{split}%
\end{align}
The set of true and false hypotheses are unknown. We choose a set of hypotheses to reject. 
on the basis of our data. 
The set of discoveries $\mathcal{R}$ 
should coincide with the set of false hypotheses $\mathcal{F}$ as much as possible.

The goal of various multiple testing corrections is to control the familywise error rate (FWER), defined as the probability of at least one false
rejection in the family, $P[\mathcal{T} \cap \mathcal{R} \neq \varnothing]$,
while retaining the reasonable power in detecting false hypotheses. We want procedures for which the FWER is less than or equal to the upper bound $\alpha$ and ideally as close as possible to the upper bound. We focus on strong control of the FWER, meaning that some of the hypotheses we are testing can be false ($\mathcal{F} \neq \varnothing$), as opposed to the weak FWER control where all hypotheses of interest are true, i.e., $\mathcal{H}_0=\mathcal{T}$.

The probability of falsely rejecting a single hypothesis that is true (i.e., false positive or Type I error) is usually controlled at a nominal $\alpha$-level. However, when the number of tested hypotheses is large, the problem of multiplicity arises: the probability of having at least one false positive conclusion rises well above $\alpha$ if the Type I error of each individual test is controlled at the $\alpha$-level. Numerous controlling procedures have been proposed to deal with this problem. 
We review two classes of controlling procedures: one based on statistical inequalities (Section \ref{ssecOrderdPvals}) and one based on the maximum of the test statistics (Section \ref{ssecMaxTest}).


\subsection{Procedures based on statistical inequalities}
\label{ssecOrderdPvals}

Let us denote by $0 < p_{(1)}\leq p_{(2)}\leq \ldots\leq p_{(d)} < 1$ the set of $d$  ordered (in ascending order) raw $p$-values and $H_{(1)},H_{(2)},\ldots,H_{(d)}$ their corresponding null hypotheses. A single-stage method uses the same rejection 
criterion for all individual hypotheses, like the conservative Bonferroni threshold, while a multi-stage method examines the ordered $p$-values sequentially and adjusts the rejection criterion for each of the individual tests  \citep[e.g.,][]{holm1979simple,hochberg1988sharper,hommel1988stagewise}. 

The Bonferroni method rejects the elementary null hypothesis $H_{(i)}$ if 
$p_{(i)}\leq\alpha/d$ 
for $i=1,\ldots,d$. \citet{holm1979simple} and \citet{hochberg1988sharper} use the same critical values $ \alpha / (d - i + 1)$ depending on the rank of the $p$-value, but reject differently depending on whether they ``step up" or ``step down". The terminologies (``step up" or ``step down") were originally formulated in terms of test statistics which can be confusing when discussing $p$-values.  \citet{holm1979simple} proposes a step-down method that ``steps up" from the smallest $p$-value to the largest one. It is a pessimistic approach: it scans forward and stops as soon as a $p$-value fails to clear its threshold. \citet{hochberg1988sharper} suggests a step-up method that ``steps down" from the largest $p$-value to the	smallest one. It is an optimistic approach: it scans backward and stops as soon as a $p$-value succeeds in clearing its threshold. By construction, Hochberg's procedure will reject as many hypotheses as Holm's procedure. 

\cite{hommel1988stagewise} proposes a more complicated procedure which applies  \citet{simes1986improved}' global test to the $p$-value subset $\left\{ p_{\left(k\right) }\right\} _{ k = i }^{d}$, instead of relying  only on $p_{\left(i\right)}$ to draw inference on $H_{(i)}$ and thus borrows power across hypotheses. 
Hommel's procedure is shown to have higher power than Hochberg's method \citep{hommel1989}. We refer to Appendix \ref{AppOrderedPVals} for more details on the practical implementation of these procedures.


Bonferroni and \citet{holm1979simple}'s method are based on the first-order Bonferroni inequality, which states that given any set of events, the probability of their union is smaller than or equal to the sum of their probabilities. 
Under the null hypothesis, 
the probability that there is at least one hypothesis $H_{(i)}$  for which its raw $p$-value $p_{(i)} \leq \alpha / d$ 
% is not greater than $\alpha$ 
is bounded by $\alpha$: 
\begin{align}  \label{eqIneqPval}
	\Pr\left( \min_{i} p_{(i)} \leq \frac{\alpha}{d} \right) 
	= \Pr\left(\bigcup_{i =
		1}^{d} \left\{ p_{(i)} \leq \frac{\alpha}{d} \right\} \right) 
	&\leq
	\sum^{d}_{i = 1} \Pr \left( p_{(i)} \leq \frac{\alpha}{d}\right) \leq d
	\frac{\alpha}{d} \leq \alpha.
\end{align}
The Bonferroni \eqref{eqIneqPval} inequality 
makes no specific assumption on the dependence between the $p$-values, but protects against the so-called ``worst-case", in which all events are independent and the rejection regions are disjoint (the right half of Equation \eqref{eqIneqPval}) . 
The inequality becomes an equality when all test statistics are independent, and a strict inequality when the hypotheses are dependent. 
In other words, the Bonferroni correction is  conservative when the $p$-values are correlated. 

The methods of \cite{hochberg1988sharper} and \cite{hommel1988stagewise} are based on  \cite{simes1986improved}'s inequality. If a set of hypotheses $H_{(1)}, ...,H_{(d)}$ are all true, the probability of the joint event is: 
\begin{equation}  \label{eqSimes}
	\Pr\left( p_{\left( i\right) }> \frac{i\alpha}{d}, \text{ for all } i=1,\ldots
	,d\right) \geq 1-\alpha.
\end{equation}
\citet{simes1986improved}' inequality was developed for independent uniform $p$-values, and it is applicable for a large family of multivariate distributions. The simulations of \citet{simes1986improved} do show, however, that the test is very conservative for highly correlated multivariate normal statistics, but less so than the classical Bonferroni correction. 



\subsection{Procedures based on the maximum of test statistics}
\label{ssecMaxTest}

Another class of controlling procedures uses the maximum in a group of test statistics: $X_m = \max_{i} \abs{X_{i}}$, with $i = 1, \ldots, d$, to set a stringent critical value. The same critical value can be used for each elementary hypothesis and will control the familywise error rate. In particular, when the individual test statistics are independent and follow the standard normal distribution under the null, the maximum of the test statistics follows a Gumbel distribution when $d$ is large. Quantiles of the Gumbel distribution were used as critical values of the individual tests as a multiple testing correction when, for example, conducting jump tests in high-frequency asset returns \citep[][]{lee2007jumps}. Unfortunately, if the sequence of test statistics exhibits strong correlation, the number of tests severely overstates the effective number of independent copies in a given sample, which makes the Gumbel critical values too conservative \citep[see e.g.,][]{christensen2018drift}. We refer to Appendix \ref{AppOrderedPVals} for more details on the Gumbel distribution. 

Resampling-based methods account for the dependence structure that is specific to the considered dataset, leading to less conservative testing outcomes than the Gumbel-based methods and the inequality-based procedures. Depending on the empirical problem of interest, the resampling can be carried out by bootstrap, permutation, simulation, or randomization \citep[see e.g.,][for detailed discussions on resampling methods and testing procedures]{white2000,romano2005exact,romano2005stepwise,lehmann2005testing}. We refer to Section \ref{secApplDriftBurst} for an example of the resampling-based approach for the drift burst test. 


\section{Cauchy Combination Tests}
\label{secSeqCauchy}

In this section, we first review the global Cauchy combination (GCC) test of \citet{liu2020cauchy} and present our sequential Cauchy combination (SCC) test. While the global test tests the global null hypothesis $\mathcal{H}_0 = \bigcap_{i=1}^{d} H_{i}$ against the alternative hypothesis that at least one of the elementary null hypotheses is false, the sequential test aims at identifying the violations of the elementary null hypotheses while controlling the global error rate. 

\subsection{Global Cauchy combination test}
\label{sec:CC}

The GCC test statistic is constructed from the raw $p$-values of the test statistics $X_i$, which follow a uniform distribution between $0$ and $1$ under the  null hypothesis. The idea of this test is first to transform the uniformly distributed $p$-values into standard Cauchy variates using the formula $\tan \{(0.5-p_{i})\pi \}$ and then construct a new test statistic as the weighted sum of these transformed $p$-values. The new test statistic is denoted by $\tilde{T}$ and defined as: 
\begin{equation}
	\label{eqCauchyStatistic}
	{\normalsize \tilde{T}=\sum_{i=1}^{d}w_{i}\tan \{(0.5-p_{i})\pi \},} 
\end{equation}
in which the $w_{i}$'s are non-negative weights summing to 1. Throughout the paper, the weights $w_{i}$ are set to $1/d$ for $i=1,\ldots,d$ as in \citet{liu2020cauchy}. 

When the raw and hence the transformed $p$-values are independent (resp. perfectly correlated), % under the null hypothesis, 
the new test statistic $\tilde{T}$ is a linear combination of independent (resp. perfectly correlated) Cauchy variates and therefore follows a standard Cauchy distribution because the family of Cauchy densities is closed under convolutions. Although the correlation structure can affect the null distribution of $\tilde{T}$ in the case of general dependence, \citet{liu2020cauchy} show that the impact on the tail is very limited because of the heaviness of the Cauchy tail. Specifically, they prove that: 
\begin{equation}
	\lim_{h\rightarrow \infty }\frac{\Pr\left( \tilde{T}>h\right) }{\Pr\left(
		C>h\right) }=1,  \label{eq:tail}
\end{equation}
in which $C$ is a standard Cauchy random variable, under the null hypothesis $\mathcal{H}_0$ and Assumption \ref{ass1} which requires the test statistics to follow a bivariate zero mean normal distribution.
\begin{assumption}
	\label{ass1} (1) The original test statistics $(X_{i},X_{j})$, for any $1\leq
	i<j\leq d$, follow a bivariate normal distribution; (2) $E\left( \bm{X}%
	\right) =0$, with $\bm{X}=(X_{1},X_{2},\ldots,X_{d})^{^{\prime }}$. 
\end{assumption}
The bivariate normal requirement of Assumption \ref{ass1} is a condition weaker than joint normality, making the procedure applicable for high-dimensional settings. When the dimension $d$ increases at a certain rate with the sample size, the test statistics $\bm{X}$ may not jointly converge to a multivariate normal distribution due to its slower rate of convergence \citep[see][and references therein]{liu2020cauchy} and thus a joint normality assumption is not realistic for those settings. In contrast, the bivariate normality assumption is much weaker and more realistic. There are, of course, applications for which the test statistics are not normally distributed. Through simulations, \citet[][]{liu2020cauchy} show the Cauchy approximation is still accurate when the normality assumption is violated and follows a multivariate Student-$t$ distribution (with 4 degrees of freedom) instead. 
We refer to Section \ref{secApplFan} for a showcase example in finance with test statistics being Student-$t$ distributed. 

The result in  \eqref{eq:tail} suggests that, under the null hypothesis $\mathcal{H}_0$, the tail of the Cauchy combination test statistic is approximately Cauchy under arbitrary dependence structures, so that a $p$-value of the Cauchy combination test, denoted 
$\widetilde{p}$, can  be calculated from the standard Cauchy distribution. Suppose that we observe $\tilde{T}=t_{0}$, then: 
\begin{equation}
	\label{eqCauchyPval}
	\widetilde{p}=\frac{1}{2}-\frac{\arctan t_{0}}{\pi }. 
\end{equation}

Using the GCC $p$-values, the tail result in \eqref{eq:tail} can be equivalently stated as the actual size converging to the nominal size $\alpha$ as the significance level tends to zero:  % \textit{i.e.}, 
\begin{equation}
	\lim_{\alpha\rightarrow 0 }\frac{\Pr\left( \widetilde{p} \leq \alpha\right) }{\alpha}=1, \label{eq:wFWER}
\end{equation} 
The approximation should be particularly accurate for small $\alpha$'s, which are of particular interest in large-scale problems as in Examples 1 and 2. The simulations in \citeauthor{liu2020cauchy} show that when the significance level is moderately small ($\alpha = 10^{-1}, 10^{-2},10^{-3},10^{-4},10^{-5}$), the $p$-value calculation is accurate:  the ratio of the empirical size to the significance level is close to 1 for different types of correlations. 
Put differently, the GCC test achieves the weak familywise error rate control as the empirical size is very close to the nominal size $\alpha$ regardless of the correlation structure. 


Figure \ref{figCauchyPvalsAR} illustrates the fact that while the dependence between the individual test statistics $X_i$ can affect the null distribution of the GCC test statistic, the impact of the dependence is marginal on the tail. We simulate a vector of $d$ test statistics  $\bm{X}$ from a $d$-variate normal distribution with correlation matrix $\bm{\Sigma}$, \textit{i.e.}, $N_d(\bm{0}, \bm{\Sigma})$ with $\bm{\Sigma} = (\sigma_{ij})$ and $d = 300$. The diagonal elements $\sigma_{ii}=1$ for all $i=1,\ldots,d$ and the off-diagonal elements $\sigma_{ij} = \theta^{\abs{i-j}}$ for $i \neq j$, with $\theta = 0.2, 0.4, 0.6,$ $0.8, 0.90, 0.95$. The simulation is repeated $10^7$ times. For each draw, we calculate the GCC test statistic \eqref{eqCauchyStatistic} and its corresponding $p$-value \eqref{eqCauchyPval}. The histogram of the $10^7$ GCC $p$-values is displayed in Figure \ref{figCauchyPvalsAR}. For a low level of autocorrelation (i.e., $\theta=0.2$), the distribution of the $p$-values is close to a uniform distribution. When the level of autocorrelation is higher, there is a pothole in the middle and a bump at the end of the histogram, but whatever the strength of the autoregressive parameter, the percentage of the GCC $p$-values in the first bin is always around $5$\% as is ensured by the limit result in \eqref{eq:wFWER}. 


\begin{figure}[p]
	\caption{The impact of dependence on the tail of the GCC test statistic}
	\label{figCauchyPvalsAR}
	\centering
	
	\par
	
	\subfloat[${\theta} = 0.2$ ]{{\includegraphics[width=.40\textwidth,angle =
			-90,scale=0.70]{Sample_pval_dim_300_rho_0.2_alpha0.05.eps} }} 
	\subfloat[$\theta = 0.4$ ]{{\includegraphics[width=.40\textwidth,angle =
			-90,scale=0.70]{Sample_pval_dim_300_rho_0.4_alpha0.05.eps} }} 
	
	\vspace{0.4cm}
	
	\subfloat[$\theta = 0.6$ ]{{\includegraphics[width=.40\textwidth,angle =
			-90,scale=0.70]{Sample_pval_dim_300_rho_0.6_alpha0.05.eps} }} 
	\subfloat[$\theta = 0.8$ ]{{\includegraphics[width=.40\textwidth,angle =
			-90,scale=0.70]{Sample_pval_dim_300_rho_0.8_alpha0.05.eps} }} 
	
	\vspace{0.4cm}	
	
	\subfloat[$\theta = 0.90$]{{\includegraphics[width=.40\textwidth,angle =
			-90,scale=0.70]{Sample_pval_dim_300_rho_0.9_alpha0.05.eps} }} 
	% 
	\subfloat[$\theta = 0.95$]{{\includegraphics[width=.40\textwidth,angle =
			-90,scale=0.70]{Sample_pval_dim_300_rho_0.95_alpha0.05.eps} }} 
	
	\par
	\begin{minipage}{1.0\linewidth}
		\begin{tablenotes}
			\small
			\item {
				\medskip
				Note: We plot histograms of GCC $p$-values \eqref{eqCauchyPval} for various correlation strengths. The individual test statistics are drawn from a $d$-variate normal distribution $N_d(\bm{0}, \bm{\Sigma})$ with $\bm{\Sigma}= (\sigma_{ij})$ and $d=300$. The diagonal elements of the covariance matrix $\sigma_{ii}=1$ for all $i=1,\ldots,d$ and the off-diagonal elements  $\sigma_{ij} = \theta^{\vert i-j \vert}$ for $i\neq j$, with $\theta = 0.2, 0.4, 0.6,$ $0.8, 0.90, 0.95$. We compute the GCC $p$-value from the test statistic sequence. The simulation is repeated $10^7$ times. The simulated GCC $p$-values are sorted into bins with the bin edges being a sequence of edges from 0 to 1 with a width of  0.05. 				Each bin includes the right edge (right-closed) but does not include the left edge (left-open). We highlight the first bin in black and we
				also add a text note with the probability of $p$-values being in the first bin. }
		\end{tablenotes}
	\end{minipage}
\end{figure}

Interestingly,  \citet{liu2020cauchy} show that the tail property \eqref{eq:tail}  also holds when the number of hypotheses $d$ diverges to infinity at a rate of $o\left(h^{\eta }\right) \ $with $0<\eta <1/2$ and the following additional assumption is satisfied.
\begin{assumption}
	\label{ass2} Let $\mathbf{\Sigma }=corr\left( \bm{X}\right) $. 
	(1) The	largest eigenvalue of the correlation matrix $\lambda _{\max}\left( \mathbf{%
		\Sigma }\right) \leq C_0$ for some constant $C_0>0$; 
	(2) $\max_{1\leq i<j\leq
		d}\left\{ \sigma _{i,j}^{2}\right\} \leq \sigma _{\max }^{2}<1$ for some
	constant $0<\sigma _{\max }^{2}<1$, where $\sigma _{i,j}$ is the $\left(
	i,j\right) $ element of $\mathbf{\Sigma }$.
\end{assumption}
The additional assumptions on the correlation matrix are mild and standard in high dimensional settings and are general enough to incorporate a large class of tests. 


\subsection{Sequential Cauchy Combination Test}

The main contribution of this paper is the sequential Cauchy combination (SCC) test, which unravels the GCC test to make statements on the elementary hypotheses. The raw $p$-values are sorted in ascending order so that  $p_{(1)}\leq p_{(2)}\leq \ldots \leq p_{(d)}$, which is standard for step-down and step-up sequential procedures (see Section \ref{ssecOrderdPvals}). For the inference on hypothesis $H_{\left( i\right) }$ we compute a Cauchy combination test statistic ${\normalsize \tilde{T}}_{\left( i\right) }$ from a subset of $p$-values, running from $p_{(i)}$ to $p_{(d)}$ as:  
\begin{equation}
{\normalsize \ \tilde{T}%
		_{\left( i\right) }=\sum_{j=i}^{d}w_{j}\tan \{(0.5-p_{(j)})\pi \}
	}.
	\label{eq:CC_mt}
\end{equation}
The corresponding $p$-value is: $$ \widetilde{p}_{(i)}=\frac{1}{2}-\frac{\arctan \tilde{T}_{\left(i\right) }}{\pi }.$$ We reject the null hypothesis $H_{(i)}$ if  $\widetilde{p}_{(i)}\leq\alpha$. Like the step-up procedure of  \citet{hommel1988stagewise}, the SCC test also borrows power across hypotheses: the test statistic $\tilde{T}_{(i)}$ is computed from the raw $p$-values associated with $\mathcal{H}_0^{(i)}=\bigcap_{j=i}^{d} H_{(j)}$.


\subsubsection*{Theoretical Properties}
The SCC testing procedure can be viewed as a sequential rejection procedure. Let $\mathcal{R}^{(s)}$ be the collection of rejected hypothesis after step $s$, with $s=\left\{1,2,\ldots,d\right\}$. The hypothesis of interest and decision rules in each step  are illustrated in Table \ref{tabDecisionRule}.
\begin{table}[H]
	\caption{Decision rule in the sequential Cauchy combination test}
	\label{tabDecisionRule}
	\centering
	\begin{tabular}{p{1.cm}p{3.8cm}p{10.5cm}}
		\hline
		Step & Hypothesis & Decision\\ 
		$s=1$ & $\mathcal{H}_0^{\left( d\right) }=H_{(d)}$ & 
		If $\widetilde{p}_{(d)}\leq\alpha$ then reject $\mathcal{H}_0^{\left( d\right) }$ and include $H_{(d)}$ in $\mathcal{R}^{(1)}$
		\\
		$s=2$ & $\mathcal{H}_0^{\left( d-1\right) }=\bigcap_{j=d-1}^d H_{(j)}$  
		& 
		If $\widetilde{p}_{(d-1)}\leq\alpha$ then reject $\mathcal{H}_0^{\left( d-1\right) }$  and include $H_{(d-1)}$ in $\mathcal{R}^{(2)}$  
		\\
		$\ldots$  & $\ldots$  & $\ldots$  \\ 
		$s=d$ & $\mathcal{H}_0^{\left( 1\right) }=\bigcap_{j=1}^d H_{(j)}$ & 
		If $\widetilde{p}_{(1)}\leq\alpha$ then reject $\mathcal{H}_0^{\left( 1\right) }$ and include $H_{(1)}$ in $\mathcal{R}^{(d)}$ \\
		\hline
	\end{tabular}
\end{table}
Let $\mathcal{N}\left(\mathcal{R}^{(s)}\right)$ be the successor function, representing hypotheses to be rejected in the next step given that $\mathcal{R}^{(s)}$ has been rejected. For the SCC test, the successor function is defined as: 
\[
\mathcal{N}\left(\mathcal{R}^{(s)}\right)=\left\{H_{(d-s)} :  \widetilde{p}_{(d-s)} \leq \alpha_{\mathcal{R}^{(s)}}=\alpha\right\}.
\]
The cut-off value is fixed (i.e., $\alpha_{\mathcal{R}^{(s)}}=\alpha$) instead of depending on the rejection set $\mathcal{R}^{(s)}$ like in many other sequential procedures. According to the sequential rejection principle of \cite{goeman2010sequential},  the SCC test  achieves a strong family-wise error rate control if the following two conditions are satisfied. 
\begin{condition}[Monotonicity]
	For every $\mathcal{R}^{(s)}\subseteq \mathcal{R}^{(l)} \subset \mathcal{H}_{0}$, 
	\[
	\mathcal{N}(\mathcal{R}^{(s)}) \subseteq \mathcal{N}(\mathcal{R}^{(l)}) \cup \mathcal{R}^{(l)}
	\]
	almost surely. 
\end{condition}
% By construction, 
The transformed $p$-values of the SCC test are monotonic by construction, with $\widetilde{p}_{(d)}$ being the largest for the smallest set of global null hypotheses $\mathcal{H}_0^{(d)} = H_{(d)}$ and $\widetilde{p}_{(1)}$ being the smallest for the largest set of global nulls $\mathcal{H}_0^{(1)} = \bigcap_{j=1}^{d} H_{(j)}$ (see Figure \ref{figSequentialCauchyIllustration}(e) for an illustration of the monotonic $p$-values). Note that the largest set of global null hypotheses has the same null specification as the GCC test \eqref{eqCauchyStatistic}. It follows that that $\widetilde{p}_{(s)}\geq \widetilde{p}_{(l)}$. Since the cut-off value is fixed, the monotonicity condition of the successor function is satisfied.

\begin{condition}[Single-step condition] \label{SS} 
	When $\mathcal{H}_{0}^{(i)} =\mathcal{T}$, 
	$\Pr\left( \widetilde{p}_{(i)} \leq \alpha\right) \leq \alpha. $
\end{condition}
Condition \ref{SS} requires FWER control of the underlying test of SCC (i.e., the Cauchy combination test) at the ``critical case" in which all hypotheses of interest are true: $\mathcal{H}_{0}^{(i)} =\mathcal{T}$. The condition can be rewritten as $\Pr{\mathcal{N}(\mathcal{F})\subseteq \mathcal{F}} \geq 1-\alpha$ and has been shown to be satisfied by \cite{liu2020cauchy}. In fact,  when $\alpha$ is very small, the familywise false rejection probability of the GCC test under the null is not only bounded by $\alpha$ but also approaches the nominal size $\alpha$, as stated in \eqref{eq:wFWER}, which implies that it is less conservative than tests based on statistical inequalities or tests which impose independence in the presence of correlation. 

The theorem below follows directly from \cite[Theorem 1]{goeman2010sequential} for general sequential rejection procedures, so that Type I control in the critical case is sufficient for overall familywise error control of the sequential procedure. 
\begin{theorem}\label{thm}
	The SCC testing procedure satisfies both the monotonicity and the single-step condition and achieves the strong FWER control:  
	\[
	\lim_{\alpha\rightarrow 0}\Pr\left\{\mathcal{R}^{(d)} \subseteq \mathcal{F} \right\} \geq 1-\alpha, 
	\]
	under Assumption \ref{ass1} if $d$ is fixed and under Assumptions \ref{ass1} and \ref{ass2} if $d\rightarrow \infty$.
\end{theorem}


\subsubsection*{An Illustration}

A more prescriptive description of the SCC testing procedure is as follows: 
\begin{enumerate}
	
	\item Obtain raw $p$-values $p_1, p_2,\ldots, p_d$ corresponding to the null hypotheses $H_{1}, H_{2},\ldots, H_{d} $;%
	
	\item Order the raw $p$-values in ascending order, 	$p_{(1)},p_{(2)},\ldots,p_{(d)}$, with corresponding null ordered hypotheses $H_{(1)},H_{(2)},\ldots,H_{(d)}$;
	
	\item Calculate the SCC test statistic $\tilde T_{(i)}$ and the transformed Cauchy $p$-values $\widetilde{p}_{(i)}$ from a subset of the ordered $p$-values $\left\{p_{(j)}\right\} _{j=i}^{d}$ using \eqref{eq:CC_mt} for $i=1,\ldots,d$;
	
	\item Obtain the rejection set $\mathcal{R}=\left\{H_{\left(i\right)} : \widetilde{p}_{(i)}\leq \alpha\right\}$. 
\end{enumerate}

Figure \ref{figSequentialCauchyIllustration} illustrates the sequential Cauchy combination procedure on a simulated sequence of test statistics. The top row shows the simulated test statistics and their corresponding $p$-values, of which some hypotheses are under the null (grey dots) and some are under the alternative (black dots). The data-generating process is the same as that in Figure \ref{figCauchyPvalsAR} with $\theta=0.9$ and $d=100$. We add constant signals for $5$ out of 100 hypotheses,  with a signal strength equal to $\pm2.806$. The sign of the signal is the same as the sign of the test statistic under the null, such that the signal always amplifies the magnitude of the test statistic. 
The GCC test rejects the global null at $\alpha = 5\%$ for this sequence of $p$-values, which tells us there is at least one signal in the sequence.  
%The estimated first-order autocorrelation of the simulated test statistics is equal to $0.7910$ under the null and is equal to $0.4987$ under the alternative. 

\begin{figure}[p]
	\caption{Rejection procedure of the sequential Cauchy Combination test}
	\label{figSequentialCauchyIllustration}\centering

	\par
	
	\subfloat[Raw test statistics]{{\includegraphics[width=.31\textwidth,angle =
			-90]{1_tstat_d_100_rho_0.9_signal_5_5} }} 
	\subfloat[Raw
	$p$-values]{{\includegraphics[width=.31\textwidth,angle = -90]{2_pvals_d_100_rho_0.9_signal_5_5} }}
	
	\vspace{0.4cm}	
	
	\subfloat[Ordered raw $p$-values]{{\includegraphics[width=.31\textwidth,angle =
			-90]{3_spvals_d_100_rho_0.9_signal_5_5} }} 
	
	\vspace{0.4cm}	
	
	\subfloat[SCC test statistics
	]{{\includegraphics[width=.31\textwidth,angle = -90]{4_ctstats_d_100_rho_0.9_signal_5_5}}}
	\subfloat[SCC 
	$p$-values]{{\includegraphics[width=.31\textwidth,angle = -90]{4_cpvals_d_100_rho_0.9_signal_5_5}}}
	
	
	\begin{minipage}{1.0\linewidth}
		\begin{tablenotes}
			\small
			\item {
				\medskip
				Note: We illustrate the mechanics of the SCC procedure on a simulated test statistic sequence with sparse signals. The top row shows raw test statistics and $p$-values of which some hypotheses are under the null and some are under the alternative. The test statistics are simulated from $N_{d}(\bm{0},\bm{\Sigma})$ as in Figure \ref{figCauchyPvalsAR}. We set $d=100$, $\theta=0.9$ and add $5\%$ signals. The strength of the signal is $\pm2.806$, with its sign identical to that of the test statistic under the null. The horizon line in panel (e) is the 5\% significance level.
			}
		\end{tablenotes}
	\end{minipage}
\end{figure}

The SCC test can tell us which individual $p$-values trigger the rejection of the GCC test. The middle row plots the raw $p$-values in ascending order and the bottom row plots its sequential Cauchy combination test statistics and $p$-values. Specifically, the bottom right panel shows that the SCC $p$-values $\widetilde{p}_{(i)}$ decrease as $i$ moves from $d$ to $1$. In this example, the SCC test rejects three out of the five alternative hypotheses and does not reject under the null hypothesis. The rejections correspond to the 4$^\text{th}$, 29$^\text{th}$ and  46$^\text{th}$ hypotheses in the top row. Note that the smallest SCC $p$-value corresponds to the $p$-value of the GCC test of \citet{liu2020cauchy}, which performs the test on the largest set of hypotheses. 
% We provide some comments on the growth conditions which constituted the majority of our analysis in sections \ref{sec:Hmixing} and \ref{sec:Hsigma}. In the simplest cases of Lemma \ref{lemma:unstableGrowth}, growth was established in an analogous fashion to the old one-step expansion condition (\ref{eq:oldOneStepExpansion}), finding the relevant Jacobians $M_j$ and checking that their expansion factors $K(M_j)$ satisfy
\begin{equation}
    \label{eq:discussionOneStep}
    \sum_j \frac{1}{K(M_j)} <1.
\end{equation}
For the more complicated cases, the inductive method used to establish growth near the accumulation points in Lemma \ref{lemma:unstableGrowth} and the weakened one-step expansion condition (\ref{eq:oneStep}) both address the same fundamental issue: the splitting of unstable curves by singularities into an unbounded number of small components. They circumvent this obstacle in rather different ways, however. While (\ref{eq:oneStep}) generalises (\ref{eq:discussionOneStep}) to ensure an growth of unstable curves `on average' (see \cite{chernov_statistical_2009} for a precise statement), our inductive method is a more direct adaptation of (\ref{eq:discussionOneStep}), using it to generate contradictory geometric conditions which a hypothetical non-growing unstable curve must satisfy. It may be possible to prove Theorem \ref{sec:Hmixing} using (\ref{eq:oneStep}) as the basis for growth. Since we required (\ref{eq:oneStep}) anyway for proving Theorem \ref{thm:HsigmaExp}, this could potentially condense our analysis, but only to a minor extent. A convenience of the method used in section \ref{sec:Hmixing} is that, by way of the `simple intersection' property, it naturally gives geometric information on the images of manifolds, useful for proving the property \textbf{(M)} of Theorem \ref{thm:katok-strelcyn}.

We expect that essentially analogous analysis can be applied to establish mixing properties in a wide class of piecewise linear non-uniformly hyperbolic maps, including those (like the OTM) which sit on the boundary of ergodicity and beyond. While we have relied on the precise partition structure of $H_\sigma$, its fundamental feature (self-similar sequences of elements $A^k$, sharing boundaries with its neighbours $A^{k-1},A^{k+1}$ and accumulating onto some point $p$) is quite typical to return map systems. See, for example, those of various stadium billiards \cite{chernov_chaotic_2006,chernov_improved_2008,chernov_statistical_2009} and LTMs \cite{springham_polynomial_2014}. Indeed, the same method can be used to prove the Bernoulli property for non-monotonic LTMs \cite{myers_hill_mixing_2022}, where monotonicity of the manifold images cannot be assumed and the classical argument \cite{sturman_mathematical_2006} fails. The OTM is the pointwise limit of these maps as the boundary shrinks to null measure. It further has utility in proving growth conditions for maps which are uniformly hyperbolic but possess regions $A_j$ where the hyperbolicity is very weak, signified by $K(M_j) \approx 1$, so that (\ref{eq:discussionOneStep}) fails. Typically this leads to suboptimal bounds on mixing windows, see e.g. \cite{wojtkowski_model_1981,przytycki_ergodicity_1983,myers_hill_family_2022}. The map $H_{(\eta,\eta)}$ for $\eta \approx 1/2$ is another example, possessing weak hyperbolicity over $A_2, A_3$. Letting $\varepsilon = |\eta-1/2|>0$, there is an upper bound $N = N(\varepsilon)$ on escape times from the intersections $A_2\cap \sigma, A_3 \cap \sigma$. The growth lemma then follows by applying the inductive step roughly $N$ times and can be established for arbitrarily small $\varepsilon$, opening the door to establishing optimal mixing windows.

The above gives two examples of piecewise linear perturbations to $H$ where mixing with respect to Lebesgue is preserved and our methods can be applied. Nonlinear perturbations to the shear profiles complicate the analysis in several ways. Firstly as the map's Jacobians takes on a broader range of values, cone invariance becomes an increasingly harder condition to establish. Cones must be widened, giving looser bounds on expansion factors, which may already be weak due to new regions of weaker stretching. This, together with the change from polygonal to curvilinear return time partition elements and nonlinear local manifolds, adds some complexity to showing growth conditions. This does not rule out certain (small) nonlinear perturbations however. There is some leeway in the inequalities which govern cone invariance and growth of local manifolds, the latter of which is not too dissimilar from the piecewise linear setting (see Lemmas \ref{lemma:piecewiseApprox}, \ref{lemma:componentLength}). Certain small perturbations would not alter the \emph{topological} structure of the return time partition, i.e. which elements share boundaries, the key information needed for setting up the induction. Finally while the partition elements would no longer be polygonal, only coarse geometric information is required for verifying each inductive step. Following the above, a potential perturbation could be to replace the linear portions of each shear by a cubic, perturbing the tent profile
\[  f(t) = \begin{cases} 2t & 0 \leq t \leq 1/2, \\ 2(1-t) & 1/2 \leq t \leq 1 ,\end{cases} \]
of the OTM shears to
\[  f_a(t) = \begin{cases} \frac{1}{8} t \left(16 - a + 6at - 8at^{2} \right) & 0 \leq t \leq 1/2, \\ \frac{1}{8}\left(1-t\right)\left( 16 - a + 6a\left(1-t\right) - 8a\left(1-t\right)^{2}\right)  & 1/2 \leq t \leq 1, \end{cases}   \]
for $a>0$. For small enough $a$ the gradient range $f'(t)$ is restricted to small neighbourhoods of $\{ 2, -2\}$ and the escape time partition retains a similar structure. We illustrate this in Figure \ref{fig:perturbations}, showing escapes from the square $S_3$ under the map $G \circ F$, equivalent to escapes from the perturbed $A_3$ under the $G \circ F$, but with a cleaner geometry for comparison. When $a$ is too large the analogy to the OTM breaks down. At $a=16$ the map is twice differentiable everywhere and features a new source of slowed mixing, the Jacobian is the identity at the corner points $x,y \in \{  0, 1/2 \}$ giving locally parabolic behaviour (visible in the escape time partition). 

\begin{figure}
    \centering
    \includegraphics[width=0.24 \linewidth]{0.png}
    \includegraphics[width=0.24 \linewidth]{4.png}
    \includegraphics[width=0.24 \linewidth]{8.png}
    \includegraphics[width=0.24 \linewidth]{16.png}
    \caption{Partition of escape times from $S_3$ under the mapping $F \circ G$ for $a= 0,4,8,16$. }
    \label{fig:perturbations}
\end{figure}

% \subsubsection*{Broader Impact Statement}
% \subsubsection*{Author Contributions}
% \subsubsection*{Acknowledgments}

\section*{Acknowledgments and Disclosure of Funding}

This project was supported in part by NSF CAREER AWARD 1942230, ONR YIP award N00014-22-1-2271, NIST 60NANB20D134, Meta award 23010098, HR001119S0026 (GARD), Army Grant No. W911NF2120076, a capital one grant, and the NSF award CCF2212458 and an Amazon Research Award. Sadasivan is also supported by the Kulkarni Research Fellowship. The authors would like to thank Keivan Rezaei and Mehrdad Saberi for their insights on this work. The authors also acknowledge the use of OpenAI's ChatGPT to improve clarity and readability.

\bibliography{main}
\bibliographystyle{tmlr}

\appendix
\documentclass[aps, prx, twocolumn, superscriptaddress, longbibliography]{revtex4-1}
\usepackage[colorlinks, linkcolor=blue, anchorcolor=blue, citecolor=blue]{hyperref}
\usepackage{amsmath}
\usepackage{graphicx}
\usepackage{braket}
\usepackage{multirow}
\usepackage{color}
\usepackage{soul}
%\usepackage[smalltableaux,centertableaux,boxsize=5pt]{ytableau}

%\usepackage{ulem}%only for the command \sout = scrap

\renewcommand{\thefigure}{S\arabic{figure}}
\renewcommand{\theequation}{S\arabic{equation}}
\renewcommand{\thetable}{S\arabic{table}}
\renewcommand{\thesection}{S-\Roman{section}}
\newcommand{\aw}[1]{{\color[rgb]{.8,.4,.2}{#1}}}
\newcommand{\awc}[1]{{\color[rgb]{.8,.6,.6}{[AW: {\it #1}\,]}}}
\newcommand{\awx}[1]{{\color[rgb]{.8,.6,.6}{\sout{#1}}}}%
\newcommand{\ylw}[1]{\textcolor{red}{#1}}

\begin{document}

\title{Supporting Information for ``Electronic Correlation Effects on Stabilizing a Perfect Kagome Lattice and Ferromagnetic Fluctuation in LaRu$_3$Si$_2$''}
\author{Yilin Wang}     
\affiliation{Hefei National Laboratory for Physical Sciences at Microscale, University of Science and Technology of China, Hefei, Anhui 230026, China} 

\date{\today}

\begin{abstract}
\end{abstract}

\maketitle

%\section{Computational Details}

\begin{table*}
    \centering
    \caption{Values of Hubbard $U$ and Hund's coupling $J_H$ calculated by the code \emph{R\underline{ }Coulomb.py} in DFT+EDMFTF package. These values are used for both DFT+U and LDA+DMFT calculations.}
    \begin{ruledtabular}
    \begin{tabular}{cccccccccc}
    $U$ (eV)   & 1.1   & 1.5   & 2.0   & 3.0   & 4.0   & 4.5  & 5.0   & 5.5   & 6.0\\
    $J_H$ (eV) & 0.389 & 0.476 & 0.563 & 0.692 & 0.782 &0.817 & 0.848 & 0.874 & 0.897\\
    \end{tabular}
    \end{ruledtabular}
    \label{tab:multi}
    \end{table*}

\begin{figure*}
        \centering
        \includegraphics[width=0.9\textwidth]{x_ggasoc.pdf}
        \caption{Fractional coordinates $x$ of Ru sites as function of Hubbard $U$, relaxed by GGA+U with spin-orbital coupling.}
        \label{fig:ggasoc}
\end{figure*}

\begin{figure*}
    \centering
    \includegraphics[width=0.9\textwidth]{magnetic_conf.pdf}
    \caption{Magnetic configurations considered in the GGA+U calculations.}
    \label{fig:mag_conf}
\end{figure*}



\begin{figure*}
    \centering
    \includegraphics[width=0.9\textwidth]{XRD.pdf}
    \caption{Simulated XRD pattern of the possible distorted Kagome structure of LaRu$_3$Si$_2$. (a) For space group P6$_3$/mcm with Ru at (0.52, 0, 0.25). (b) For space group P6$_3$/m with Ru at (0.52, 0.01, 0.25). Lattice parameters are $a=5.676$\AA\ and $c=7.12$\AA. Their only difference is that there is an additional weak peak at (1 0 1) for P6$_3$/m.  }
    \label{fig:xrd}
\end{figure*}
    



%\pagebreak
\bibliography{suppl}

\end{document}


\end{document}
