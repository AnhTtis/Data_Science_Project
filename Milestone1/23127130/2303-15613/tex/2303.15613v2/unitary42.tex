\documentclass{amsart}[12pt,article]

\usepackage{amsmath,amssymb,amscd,amsthm,indentfirst}
\usepackage[all]{xy}
\usepackage{enumerate}
%\usepackage{hyperref}
\usepackage{graphicx}
\usepackage{multirow}
\usepackage{pgf,tikz,pgfplots}
\pgfplotsset{compat=1.15}
\usepackage{mathrsfs}
\usetikzlibrary{arrows}
\usepackage{wrapfig}
\usepackage{cutwin}
\usepackage{lipsum}
\usepackage{accents}
\usepackage{yhmath}
\usepackage{import}
\usepackage{pdfpages}

\newcommand{\incfig}[1]{
    \import{./}{#1.pdf_tex}
}

\newtheorem{introtheorem}{Theorem}
\renewcommand*{\theintrotheorem}{\Alph{introtheorem}}
\newtheorem{thm}{Theorem}[section]
\newtheorem{prop}[thm]{Proposition}
\newtheorem{cor}[thm]{Corollary}
\newtheorem{lem}[thm]{Lemma}
\newtheorem{conjecture}[thm]{Conjecture}

\theoremstyle{definition}
\newtheorem{defn}[thm]{Definition}
\newtheorem{remark}[thm]{Remark}
\newtheorem{hypothesis}[thm]{Hypothesis}
\newtheorem{notation}[thm]{Notation}
\newtheorem{exa}[thm]{Example}

\newenvironment{dedication}
   {\vspace{0ex}\begin{quotation}\begin{center}\begin{em}}
   {\par\end{em}\end{center}\end{quotation}}
\newcommand{\bigzero}{\mbox{\normalfont\Large\bfseries 0}}
\newcommand{\rvline}{\hspace*{-\arraycolsep}\vline\hspace*{-\arraycolsep}}


\def\A{\ensuremath{\mathcal{A}}}
\def\C{\ensuremath{\mathcal{C}}}
\def\D{\ensuremath{\mathcal{D}}}
\def\Q{\ensuremath{\mathcal{Q}}}
\def\P{\ensuremath{\mathcal{P}}}
\def\SS{\ensuremath{\mathcal{S}}}
\def\Sym{\operatorname{Sym}}
\def\FF{\ensuremath{\mathbb{F}}}
\def\NN{\ensuremath{\mathbb{N}}}
\def\ZZ{\ensuremath{\mathbb{Z}}}
\def\CC{\ensuremath{\mathbb{C}}}
\def\QQ{\ensuremath{\mathbb{Q}}}
\def\GG{\ensuremath{\mathbb{G}}}
\def\TT{\ensuremath{\mathbb{T}}}
\def\aa{\mathbf a}
\def\cc{\mathbf c}
\def\ii{\mathbf i}
\def\jj{\mathbf j}
\def\kk{\mathbf k}
\def\ll{\mathbf l}
\def\mm{\mathbf m}
\def\ss{\mathbf s}
\def\xx{\mathbf x}
\def\dd{\boldsymbol{\delta}}
\def\ee{\boldsymbol{\epsilon}}
\def\CF{\ensuremath{\mathcal{F}}}
\def\rk{\operatorname{rk}}
\def\ch{\operatorname{char}}
\def\lcm{\operatorname{lcm}}
\def\Hom{\mathrm{Hom}}
\def\Out{\mathrm{Out}}
\def\End{\operatorname{End}}
\def\Iso{\mathrm{Iso}}
\def\Isom{\mathrm{Isom}}
\def\Aut{\mathrm{Aut}}
\def\ker{\operatorname{Ker}}
\def\id{\textrm{id}}
\def\im{\operatorname{Im}}
\def\Res{\operatorname{Res}}
\def\sgn{\operatorname{sgn}}
\def\chr{\operatorname{char}}
\def\length{\operatorname{length}}
\def\gcd{\operatorname{gcd}}
\def\diag{\operatorname{diag}}
\def\adiag{\operatorname{antidiag}}
\def\tuples{\mathcal{T}}
\def\subsetN{\mathcal{N}}
\def\sequences{\mathcal{M}}
\def\ord{\operatorname{ord}}
\def\Tr{\operatorname{Tr}}
\def\sup{\operatorname{sup}}
\def\barsub{\mathbf{C}}


\newcommand{\Syl}{\operatorname{Syl}\nolimits}
\newcommand{\GL}{\operatorname{GL}\nolimits}
\newcommand{\SL}{\operatorname{SL}\nolimits}
\newcommand{\PSL}{\operatorname{PSL}\nolimits}
\newcommand{\GU}{\operatorname{GU}\nolimits}
\newcommand{\SU}{\operatorname{SU}\nolimits}
\newcommand{\PSU}{\operatorname{PSU}\nolimits}
\newcommand{\PGU}{\operatorname{PGU}\nolimits}
\newcommand{\PGL}{\operatorname{PGL}\nolimits}
\newcommand{\Sp}{\operatorname{Sp}\nolimits}
\newcommand{\SO}{\operatorname{SO}\nolimits}
\renewcommand{\O}{\operatorname{O}\nolimits}
\newcommand{\PSO}{\operatorname{PSO}\nolimits}
\newcommand{\transpose}{{\operatorname{T}\nolimits}}
\newcommand{\OSAVPHI}{{\operatorname{(OAVS)}\nolimits}}




\newcommand{\ls}[2]{{}^{{#1}\!}{#2}}
\def\wth{\widetilde H}
\newcommand{\qbox}[1]{\quad\hbox{#1}\quad}
\newcommand{\ovl}[1]{\overline{#1}}
\newcommand*{\dt}[1]{% 
   \accentset{\mbox{\large\bfseries .}}{#1}}
\renewcommand{\subjclassname}{%
  \textup{2010} Mathematics Subject Classification} 

\date{\today}
\title{Quillen's conjecture and unitary groups}


\author{Antonio D\'{i}az Ramos}
\address{Departamento de {\'A}lgebra, Geometr{\'\i}a y Topolog{\'\i}a,
Universidad de M{\'a}\-la\-ga, Apdo correos 59, 29080 M{\'a}laga,
Spain.}
\email{adiazramos@uma.es}



\begin{document}

\subjclass[2010]{55U10; 05E45, 20J99.}
\maketitle

\begin{dedication}
To Mariola and María.
\end{dedication}

\begin{abstract}
We prove Quillen's conjecture for finite simple unitary groups and obtain, as a consequence of the Aschbacher-Smith work, Quillen's conjecture for all odd primes. 
\end{abstract}



\smallskip

\section{Introduction}
\label{section:introduction}

Let $G$ be a finite group and let $p$ be a prime. Denote by $O_p(G)$ the largest normal $p$-subgroup of $G$ and by $|\A_p(G)|$ the topological realization of the poset $\A_p(G)$, which consists of the non-trivial elementary abelian $p$-subgroups of $G$. Quillen's conjecture states that $O_p(G)\neq 1$ if and only if $|\A_p(G)|$ is contractible. The work \cite{AS1993} is so far the greatest advance towards proving the conjecture in full generality and, in \cite{Diaz2016} and \cite{DiazMazza2020}, a new geometric method is introduced and employed to prove most of the results in \cite[Theorem 3.1]{AS1993}. Unitary groups are known to satisfy the conjecture by \cite{AK1990}, but it has remained open whether they have non-zero top dimensional homology, see \cite[Definition p.474]{AS1993} and Definition \ref{def:QDp}. In the current work, we expand the aforementioned geometric method and show that this is indeed the case, as conjectured by Aschbacher and Smith in \cite[Conjecture 4.1]{AS1993}. 

\begin{introtheorem}\label{thm:main}
Assume $q\neq 2$, $p$ is odd and divides $q+1$, $G$ is one of the groups $\PSU_n(q)$, $\PGU_n(q)$, or $\PGU_n(q)$ extended by field automorphisms, where we exclude the case $(p,q)=(3,8)$ in the latter case, and set $r=\rk_p(G)$. Then we have $H_{r-1}(|\A_p(G)|;\ZZ)\neq 0$ and hence Quillen's conjecture holds for $G$.
\end{introtheorem}

\begin{figure}[h!]
\centering
\includegraphics{triangulationPSLPSUZ}
\caption{Triangulation of the sphere $S^2$ arising from $\PSL_4(q)$ (left) and from $\PSU_4(q)$ (right).}
\label{fig:triangulationPSLPSUr=3}
\end{figure}

Here it is a rough preview of the construction of the non-trivial homology class of Theorem \ref{thm:main} for the case of $\PSU_n(q)$: given an elementary abelian subgroup $E$ of rank $r$, a simplicial chain $\barsub_E$ is constructed that  corresponds to the barycentric subdivision of an $(r-1)$-simplex with $E$ as barycenter, see Definition \ref{def:ZEa} and Remark \ref{rmk:barycentric_subdivision} for an illustration. The goal is then to find a suitable ``sphere'' of elements $X$ of $G$, i.e,  a linear combination of the conjugates ${}^X\barsub_E$ which is a cycle and such that the conjugates ${}^XE$ are the barycenters of the faces of a sphere $S^{r-1}$. See Definition \ref{def:Xhsphere} for a set of sufficient conditions for the existence of a ``sphere''. The elements in $X$ are products of certain elements $x_1,x_2,\ldots,x_r$ of $G$ and, while in much of the previous literature, e.g., \cite[Theorem 11.2(b)]{Quillen1978}, \cite{Diaz2016}, \cite{DiazMazza2020}, the elements $x_i$'s commute, in this work they satisfy the braid relations, in particular, they commute only if their indices are not consecutive. In the former case, the sphere $S^{r-1}$ is the standard $r$-folded join of $S^0$ and has $2^r$ $(r-1)$-simplices. In the current work, the corresponding triangulation of the sphere $S^{r-1}$ is the Coxeter complex of the symmetric group on $r+1$ letters and has $(r+1)!$ $(r-1)$-simplices, see Remark \ref{rmk:Coxeter_complex} for more details. These facts are exemplified in Figures \ref{fig:triangulationPSLPSUr=3} and \ref{fig:triangulationPSLPSUr=2} for the cases $r=3$ and $r=2$ respectively, and for $\PSL_n(q)$ with commutative $X$ and $\PSU_n(q)$ with no commutative $X$. In both figures, the chain  $\barsub_E$ has been shadowed and, in  Figure \ref{fig:triangulationPSLPSUr=2}, each conjugated ${}^x \barsub_E$ is labelled with the conjugating element $x\in X$. For further details, see Definition \ref{def:definition_of_xi} for the definition of the elements $x_i\in \SU_n(q)$,  Proposition \ref{prop:relationsinSU} for the proof of the braid relations, and Remark \ref{rmk:comparison_previous_works} for a precise comparison to previous work. 

\begin{figure}[h!]
\centering
\incfig{triangulationPSLPSUZr2}
\caption{Triangulation of the sphere $S^1$ arising from $\PSL_3(q)$ (left) and from $\PSU_3(q)$ (right).}
\label{fig:triangulationPSLPSUr=2}
\end{figure}


While the case of $\PGU_n(q)$ is similar to that of $\PSU_n(q)$, the case with field automorphisms is substantially different. Geometrically, the resulting complex is now a suspension or ``jester hat'' with base a sphere $S^{n-2}$ and $2n$ apexes (only $2$ apexes for $n=2$), see Examples \ref{example:n2_quasireflections} and \ref{example:n3_quasireflections} and the figures there. Regarding the methods, while for $\PSU_n(q)$ and $\PGU_n(q)$ the elements $x_i$'s are transvections and $E$ is a diagonal subgroup, now the elements $x_i$'s are permutation matrices and the subgroup $E$ is spanned by quasi-reflections and the field automorphism itself. Recall that transvections generate $\SU_n(q)$ and together with quasi-reflections generate $\GU_n(q)$, see \cite[Chapter 11]{Grove} and Remarks \ref{rmk:transvections} and \ref{rmk:quasi-reflections}. The proof of Theorem \ref{thm:main} is contained in Theorems \ref{thm:QDp_for_SUn}, \ref{thm:QDp_for_PSUn}, \ref{thm:QDp_for_PGUn}, and \ref{thm:QDp_for_PGUnfield}. By the result \cite[Corollary, p. 475]{AS1993} by Aschbacher and Smith and its extension to all odd primes in \cite[Theorem 1.1]{PS2022} by Smith and Piterman, we conclude the following result.
 
\begin{introtheorem}\label{thm:Quillen}
Let $G$ be a finite group and $p$ be an odd prime such that $O_p(G)=1$. Then $\widetilde{H}_*(|\A_p(G)|;\QQ)\neq 0$. In particular, Quillen's conjecture holds for odd primes.
\end{introtheorem}


\textbf{Acknowledgements:} I am sincerely grateful to Stephen Smith and Kevin Piterman for careful reading  and for many helpful comments which have improved both the mathematics and the presentation of this work. 

\textbf{Notation:} 
We denote conjugation by $g$ as ${}^gx=gxg^{-1}$ and we write $[g,h]=g^{-1}h^{-1}gh$ for the commutator of $g$ and $h$. We denote the order of an element $g$ by $\ord(g)$.
%By $\adiag(a,b,c,\ldots)$ we denote the anti-diagonal matrix with $a$ at the leftmost bottom entry.
%Along the paper, we employ the following notations: greatest common divisor $gcd(\cdot,\cdot)$, cyclic group $\ZZ_n$ of order $n$, diagonal matrix with entries $d_1,\ldots,d_n$, $\diag(d_1,\ldots,d_n)$, conjugation morphism by the element $g$, $c_g(x)=gxg^{-1}$, identity matrix of size $n\times n$, $I_n$, $\FF_q$ for the field of size $q=r^m$ for some prime number $r$ and natural number $m$, $\sgn$ for signature of permutation or signature of tuple (Definition \ref{def:sequenceandsignature}), and $(\cdot)_p$ for the $p$-share of an integer.

%%%%%%%%%%%%%%%%%%%%%%%%%%%%

%%%%%%%%%%%%%%%%%%%%%%%%%%%%%%%%%%%%%
%%%%%%%%%%%%%%%%%%%%%%%%%%%%


\section{Preliminaries}\label{section:preliminaries}

In this section, we introduce the necessary preliminaries to discuss certain classes in the reduced homology groups $\widetilde H_*(|\A_p(G)|;R)$ with trivial coefficients in the \emph{unital ring $R$}. Recall that $|\A_p(G)|$ is the simplicial complex with $n$-simplices the chains of length $n$ in $\A_p(G)$,
\[
P_0<P_1<\ldots<P_n,\text{where $P_i\in \A_p(G)$ for $0\leq i\leq n$.}
\]

The following notions of tuple and signature will be employed to algebraically manipulate certain barycentric subdivision (see Definition \ref{def:ZEa} and Remark \ref{rmk:barycentric_subdivision}) as well as  the boundary map of linear combinations of these subdivisions (see Theorem \ref{thm:noncontractiblemoregeneral}).

\begin{defn}\label{def:sequences}
Let $r$ be a positive integer. We define a \emph{tuple for $r$} to be an ordered sequence of $r-1$ elements $\ii=[i_1,\ldots,i_{r-1}]$ with $1\leq i_j\leq r$
and no repetition. By $\tuples^r$ we denote the set of all
tuples for $r$.
\end{defn}

\begin{defn}\label{def:sequenceandsignature}
For a tuple $\ii=[i_1,\ldots,i_{r-1}]\in \tuples^r$, we define
the {\em signature of $\ii$} as
\[
\sgn(\ii)=(-1)^{n+m}\text{, where}
\]
\begin{itemize}
\item $n$ is the number of transpositions
we need to apply to the tuple $\ii$ to rearrange it in increasing order
$[j_1,\ldots,j_{r-1}]$, and\item $m$ is the number of positions in which
$[j_1,\ldots,j_{r-1}]$ differ from $[1,\ldots,r-1]$.\end{itemize}
\end{defn}
Note that in Definition~\ref{def:sequenceandsignature}, the number $n$
of transpositions is not uniquely defined, but its parity is. For instance, if $\ii=[1,4,2]\in \tuples^4$, then $n=1$, since we need to apply $(2,4)$
to reorder $\ii$ as $[1,2,4]$, and $m=1$, since $[1,2,4]$ differs from
$[1,2,3]$ only in one place. Thus $\sgn(\ii)=1$.

\begin{prop}\label{prop:right_signature}
Let $\ii=[i_1,\ldots,i_{r-1}]$, $\jj=[j_1,\ldots,j_{r-1}]\in \tuples^r$ and define $1\leq i,j\leq r$ by $\{i_1,\ldots,i_{r-1},i\}=\{j_1,\ldots,j_{r-1},j\}=\{1,\ldots,r\}$. Let $\sigma$ be the permutation on $r$ letters with $\sigma(j_l)=i_l$ for $i=1,\ldots,r-1$ and with $\sigma(j)=i$. Then $\sgn(\sigma)=\sgn(\jj)\sgn(\ii)$.
\end{prop}
\begin{proof}
Define $\widehat \sigma_\ii$ as the permutation on $r$ letters that arranges $[i_1,\ldots,i_{r-1}]$ in increasing order $[1,2,\ldots,i-1,i+1,\ldots,r]$. Then $\sgn(\ii)=\sgn(\widehat\sigma_\ii)(-1)^{r-i}$ by Definition \ref{def:sequenceandsignature}. Define $\sigma_\ii$ as the permutation on $r$ letters that arranges $[i_1,\ldots,i_{r-1},i]$ in increasing order $[1,\ldots,i-1,i,i+1,\ldots,r]$. Then we clearly have
\[
\sigma_\ii=(r, i)\circ(r-1,i)\circ\ldots\circ(i+1,i)\circ \widehat \sigma_\ii.
\]
Hence, $\sgn(\sigma_\ii)=\sgn(\widehat\sigma_\ii)(-1)^{r-i}=\sgn(\ii)$.
Analogously, we define $\sigma_\jj$ with $\sgn(\jj)=\sgn(\sigma_\jj)$. Then it is clear that 
$\sigma_\ii\sigma\sigma^{-1}_\jj=1$ and hence the result.
\end{proof}


\begin{defn}\label{def:sequencesubspace}
Let $E=\langle e_1,\dots,e_r \rangle$ be an elementary abelian $p$-subgroup of
$G$ of rank $r$. For each tuple $\ii=[i_1,\ldots,i_{r-1}]\in \tuples^r$ and each $0\leq l<r$, set
\[
E_{[i_1,\ldots,i_l]}=
\langle e_1,\ldots,\widehat{e_{i_1}},\ldots,\widehat{e_{i_l}},\ldots,e_r\rangle,
\]
for the subgroup of $E$ generated by all the $e_i$, except
$e_{i_1},\dots,e_{i_l}$. 
\end{defn}

In the definition above, for $l=0$, we have $E_{\emptyset}=E$, for $l=1$, we obtain the hyperplane $E_{i_1}=E_{[i_1]}$ of $E$, and, for $l=r-1$,  $E_{[i_1,\ldots,i_{r-1}]}$ is a cyclic subgroup of order $p$. Next, we define certain simplicial chains in $C_{r-1}(|\A_p(G)|;R)$.

\begin{defn}\label{def:ZEa}
Let $E=\langle e_1,\dots,e_r \rangle$ be an elementary abelian $p$-subgroup of $G$ of rank $r$. For $\ii=[i_1,\ldots,i_{r-1}]\in \tuples^r$, we define the
$(r-1)$-simplex 
\[
\sigma_{\ii}=\big(E_{[i_1,\ldots,i_{r-1}]}<E_{[i_1,\dots,i_{r-2}]}<
\dots<E_{i_1}<E\big)\in|\A_p(E)|,
\]
and the chain
\[
\barsub_E=\sum_{\ii\in \tuples^r}\sgn(\ii)\sigma_{\ii}
\qbox{in}C_{r-1}(|\A_p(E)|;R).
\]
\end{defn}



Its differential $d(\barsub_E)\in C_{r-2}(|\A_p(E)|;R)$ is described in the next result, cf.  \cite[Proposition 3.2]{Diaz2016}, \cite[Proposition 3.2]{DiazMazza2020}.
\begin{defn}\label{def:flagtau}
With the notation above and for $\ii=[i_1,\ldots,i_{r-1}]\in \tuples^r$, we define the
$(r-2)$-simplex 
\[
\tau_{\ii}=\big(E_{[i_1,\ldots,i_{r-1}]}<E_{[i_1,\dots,i_{r-2}]}<\dots<E_{i_1}\big).
\]
\end{defn}

\begin{prop}\label{prop:dz}
With the above notation,
\[
d(\barsub_E)=(-1)^{r-1}\sum_{\ii\in \tuples^r}\sgn(\ii)\tau_{\ii}.
\]
\end{prop}

\begin{proof}  
Recall that $d(\barsub_E)=\sum_{j=0}^{r-1}(-1)^jd_j(\barsub_E)$, where $d_j$ removes the $(j+1)$-th leftmost term of a given $(r-1)$-simplex. Suppose that $d_k(\sigma_{\ii})=d_l(\sigma_{\jj})$ for some
$\ii,\jj\in \tuples^r$ and $0\leq k,l\leq r-1$, i.e.,
\begin{multline*}
\big(E_{[i_1,\ldots,i_{r-1}]}<\dots<E_{[i_1,\dots,i_{r-k}]}<
E_{[i_1,\dots,i_{r-k-2}]}<\dots<E_{i_1}<E\big)=\\
=\big(E_{[j_1,\ldots,j_{r-1}]}<\dots<E_{[j_1,\dots,i_{r-l}]}<
E_{[j_1,\dots,j_{r-l-2}]}<\dots<E_{j_1}<E\big).
\end{multline*}
Comparing the sizes of the elementary abelian $p$-subgroups occurring in these two chains, it is clear that we must have $k=l$. If $0<k=l<r-1$, then the tuples $\ii$ and $\jj$ are
identical but for $\{ i_{r-k-1}, i_{r-k} \}=\{ j_{r-k-1}, j_{r-k}\}$.
So either $\ii=\jj$, or
$\ii$ and $\jj$ differ by one transposition and hence
$\sgn(\ii)=-\sgn(\jj)$. In the latter case, the
corresponding summands $\sgn(\ii)d_k(\sigma_\ii)$ and $\sgn(\jj)
d_l(\sigma_\jj)$ add up to zero. Assume now that $k=l=0$. Then $[j_1,\dots,j_{r-2}]=[i_1,\dots,i_{r-2}]$ and either $\ii=\jj$ or
$j_{r-1}\neq i_{r-1}$. In the latter case, by \cite[Lemma 2.4]{Diaz2016}, $\sgn(\ii)=-\sgn(\jj)$ and again the two
terms cancel each other out. Finally, if $k=l=r-1$, the tuples
$\ii$ and $\jj$ are identical and the terms contribute to the sum in
the statement of the lemma. 
\end{proof}

\begin{remark}\label{rmk:barycentric_subdivision}
The chain $\barsub_E$ represents the barycentric subdivision of an $(r-1)$-simplex $\sigma$. The $(r-1)$ simplex $\sigma_\ii$ for each tuple $\ii\in \tuples^r$ corresponds to one piece of the subdivision of $\sigma$, and the $(r-2)$-simplex $\tau_\ii$ to one piece in the subdivision of the boundary of $\sigma$. For instance, for $r=3$, we have the subdivision of a triangle shown in Figure \ref{fig:barycentric_subdivision}. For each $\ii\in \tuples^2=\{[1,2],[2,1],[1,3],[3,1],[2,3],[3,2]\}$, $\sigma_\ii$ is one of the $6$ small triangles, and $\tau_\ii$ is one of the $6$ segments in the boundary of the large triangle. The $7$ vertices correspond to the subgroups $E_{[i_1,\ldots,i_l]}$ for $l=0,1,2$ and, for $l=0$, $E$ is located at the barycenter of the triangle.
\begin{figure}[h]
\centering
\incfig{barycentricsubdivision}
\caption{Barycentric subdivision of a $2$-simplex.}\label{fig:barycentric_subdivision}
\end{figure}

\end{remark}

Recall that $G$ acts by conjugation on $C_{r-1}(|\A_p(G)|;R)$ and that $C_{r-1}(|\A_p(E)|;R)\subseteq C_{r-1}(|\A_p(G)|;R)$. Then, for $x\in G$ and $a\in R$, the element
\[
\ls x(a\barsub_E)=a\sum_{\ii\in \tuples^r}\sgn(\ii)
\ls x{\sigma_{\ii}}=
a\sum_{\ii\in \tuples^r}\sgn(\ii)
\big(\ls xE_{[i_1,\ldots,i_{r-1}]}<\dots<\ls xE_{i_1}<\ls xE\big)
\]
belongs to $C_{r-1}(|\A_p({}^xE)|;R)\subseteq C_{r-1}(|\A_p(G)|;R)$. 
\begin{defn}\label{def:ZGXA}
Let $X\subseteq G$ be a non-empty subset of $G$ and let $h\colon X\to R$ be a function. Define the chain in $C_{r-1}(|\A_p(G)|;R)$,
\[
\barsub_{E,X,h}=\sum_{x\in X}\ls{x}(h(x)\barsub_E)=
\sum_{x\in X}h(x)\sum_{\ii\in \tuples^r}
\sgn(\ii)\ls{x}{\sigma_{\ii}}.
\]
\end{defn}

Now, by Proposition \ref{prop:dz},
\[
d(\barsub_{E,X,h})=(-1)^{r-1}\sum_{x\in X}h(x)\sum_{\ii\in \tuples^r}
\sgn(\ii)\ls{x}{\tau_{\ii}}.
\]
Hence, given $x\in X$ and $\ii\in \tuples^r$, the coefficient of
$\ls{x}{\sigma_{\ii}}$ in $\barsub_{E,X,h}$ and the coefficient of $\ls{x}{\tau_{\ii}}$ in $d(\barsub_{E,X,h})$ are, respectively:
\begin{align}
C_{x,\ii}&=\sum_{(y,\jj)\in\C(x,\ii)}h(y)\sgn(\jj)
\qbox{and}
\label{equ:Cxii}\\ 
D_{x,\ii}&=(-1)^{r-1}\sum_{(y,\jj)\in\D(x,\ii)}h(y)\sgn(\jj),
\label{equ:Dxii}
\end{align}
where
\begin{align}
\C(x,\ii)&=\{(y,\jj)\in X\times \tuples^r~|~\ls{y}\sigma_\jj=\ls{x}\sigma_\ii\}\qbox{and}
\label{equ:CCxii}\\ 
\D(x,\ii)&=\{(y,\jj)\in X\times \tuples^r~|~\ls{y}\tau_\jj=\ls{x}\tau_\ii\}.
\label{equ:DDXii}
\end{align}
If we further assume that $E$ is a \emph{maximal} elementary abelian
$p$-subgroup of $G$, we immediately obtain the following result, see \cite[Theorem 3.3]{DiazMazza2020}.

\begin{thm}\label{thm:noncontractiblemoregeneral}
Let $E=\langle e_1,\dots,e_r\rangle$ be a \emph{maximal} elementary
abelian $p$-subgroup of rank $r$ of the group $G$. Consider $X\subseteq G$ and $h\colon X\to R$ satisfying that
\begin{enumerate}[\hspace{1cm}(a)]
\item $C_{x,\ii}\neq 0$ for some $x\in X$ and some
$\ii\in \tuples^r$,
\item  $D_{x,\ii}=0$ for all $x\in X$ and all
$\ii\in \tuples^r$.
\end{enumerate}
Then
\[
0\neq [\barsub_{E,X,h}]\in \widetilde H_{r-1}(|\A_p(G)|;R).
\]
\end{thm}
The maximality condition on $E$ is needed to ensure that the non-trivial cycle $[\barsub_{E,X,h}]$ is not a boundary. This maximality conditions holds, for instance, if $E$ is of rank $r=\rk_p(G)$. The conclusion of the theorem implies that  $|\A_p(G)|$ is not contractible and, in particular, that Quillen's conjecture holds for $G$. In the case $r=\rk_p(G)$ and $R=\ZZ,\QQ$, the theorem shows that the $\Q\D_p$ property holds for $G$.
\begin{defn} [$\Q\D_p$] \label{def:QDp}
The finite group $G$ with $r=\rk_p(G)$ has the \emph{Quillen dimension at $p$ property}, $\Q\D_p$ for short, if 
\[
\widetilde H_{r-1}(|\A_p(G)|;\QQ)\neq 0.
\]
\end{defn}

The case $R=\ZZ$ implies $\Q\D_p$ as top dimension integral homology is a free abelian group. We finish this section with a particular case of Theorem \ref{thm:noncontractiblemoregeneral}.

\begin{defn}\label{def:Xhsphere}
Let $G$ be a finite group.  A \emph{Q-sphere of dimension $r$} for $G$ is a tuple $(E,\{e_i\}_{i=1}^r,X,\{x_i\}_{i=1}^r,h)$, where $E=\langle e_1,\ldots_,e_r\rangle$ is an elementary abelian $p$-subgroup of rank $r$, $X$ is a subset of $G$, $x_i\in G$ for $1\leq i\leq r$, and $h$ is a function $h\colon X\to \ZZ^*$ satisfying the following:
\begin{enumerate}[\hspace{1cm}(a)]
\item If $x',x\in X$, $n\in N_G(E)$, $x'=xn$, then $x'=x$.
\item For $1\leq i\leq r$, $x_i\in C_G(E_i)$.
\item For $1\leq i\leq r$ and $x\in X$, $xx_i^\epsilon\in X$ for exactly one value $\epsilon\in \{-1,+1\}$, and $h(x)+h(xx_i^\epsilon)=0$. 
\end{enumerate}
\end{defn}

\begin{remark}
In most of the applications of Section \ref{section:unitary}, the set X we employ to define the chain in Definition \ref{def:ZGXA} satisfies that $1\in X$ and that $x_i\in X$. Note that these conditions are not imposed in Definition \ref{def:Xhsphere}.
\end{remark}

We warn the reader that the term Q-sphere will be employed throughout the paper to refer to Definition \ref{def:Xhsphere} and slight variations of it. See Remarks \ref{rmk:sphere_even_characteristic} and \ref{rmk:sphere_even_characteristic_applications} as well as Theorem \ref{thm:QDp_for_PGUnfield}. As we show below, the existence of a top-dimensional Q-sphere implies non-zero top-dimensional homology.

\begin{thm}\label{thm:sphere_implies_QDp}
If $G$ has a Q-sphere of dimension $r=\rk_p(G)$, then we have $\widetilde H_{r-1}(|\A_p(G)|;\ZZ)\neq 0$ and hence $\Q\D_p$ holds for $G$.
\end{thm}
\begin{proof}
Let the Q-sphere be given by $E=\langle e_1,\ldots_,e_r\rangle$, $X\subseteq G$, $h\colon X\to \ZZ^*\subseteq \ZZ=R$, and elements $x_1,\ldots,x_r$. We consider the chain $\barsub_{E,X,h}$ of Definition \ref{def:ZGXA} and we show that the hypotheses of Theorem \ref{thm:noncontractiblemoregeneral} hold. For take $x\in X$ and $\ii\in \tuples^r$. If $(y,\jj)\in \C(x,\ii)$ then $\ls{y}\sigma_\jj=\ls{x}\sigma_\ii$ and, in particular, $\ls{y}E=\ls{x}E$. Hence $x^{-1}y\in N_G(E)$ and, by condition (a) in Definition \ref{def:Xhsphere}, we must have $y=x$. Consequently, we must have $\ii=\jj$. So $\C(x,\ii)=\{(x,\ii)\}$ and $C(x,\ii)=\pm h(x)\neq 0$.

Next, decompose $\D(x,\ii)=\cup_{\jj\in \tuples_r} \D(x,\ii,\jj)\times \{\jj\}$ with 
\[
\D(x,\ii,\jj)=\{y\in X~|~\ls{y}\tau_\jj=\ls{x}\tau_\ii\}.
\]
Then $D(x,\ii)=(-1)^{r-1}\sum_{\jj\in \tuples^r}\sgn(\jj)D(x,\ii,\jj)$ with $D(x,\ii,\jj)=\sum_{y\in \D(x,\ii,\jj)} h(y)$. Hence, to check that $D(x,\ii)=0$, it is enough to check that $D(x,\ii,\jj)=0$ for all $\jj$.

So let $\jj=[j_1,\ldots,j_{r-1}]\in \tuples_r$. Then, by (b) in Definition \ref{def:Xhsphere}, the element $x_{j_1}$ belongs to $C_G(E_{j_1})=C_G(\tau_\jj)$. For any element $y\in \D(x,\ii,\jj)$ we have, by (c) in  Definition \ref{def:Xhsphere}, that $yx_{j_1}^\epsilon\in X$ for $\epsilon$ equal to $1$ or $-1$. As $x_{j_1}\in C_G(\tau_\jj)$, we also have that $yx_{j_1}^\epsilon\in \D(x,\ii,\jj)$. So, if we define, 
\begin{align*}
\D^+(x,\ii,\jj)=&\{y\in \D(x,\ii,\jj)~|~yx_{j_1} \in \D(x,\ii,\jj)\}\text{ and }\\
\D^-(x,\ii,\jj)=&\{y\in \D(x,\ii,\jj)~|~yx_{j_1}^{-1} \in \D(x,\ii,\jj)\},
\end{align*}
we have $\D(x,\ii,\jj)=\D^+(x,\ii,\jj)\cup \D^-(x,\ii,\jj)$. Also, if $y\in \D^+(x,\ii,\jj)\cap \D^-(x,\ii,\jj)$, then both $yx_{j_1}$ and $yx_{j_1}^{-1}$ belong to $X$, contradicting the uniqueness of $\epsilon$ in (c) of Definition \ref{def:Xhsphere}. So this intersection is empty. In addition, the maps $\D^+(x,\ii,\jj)\to \D^-(x,\ii,\jj)$, $y\mapsto yx_{j_1}$, and $\D^-(x,\ii,\jj)\to \D^+(x,\ii,\jj)$, $y\mapsto yx_{j_1}^{-1}$, are inverse to each other and hence bijections. Therefore, $\D(x,\ii,\jj)=\cup_{y\in \D^+(x,\ii,\jj)} \{y,yx_{i_1}\}$ and 
\[
D(x,\ii,\jj)=\sum_{y\in \D^+(x,\ii,\jj)} h(y)+h(yx_{j_1})=0,
\]
where we have used by (c) in Definition \ref{def:Xhsphere}.
\end{proof}

We will need the following result to construct spheres in the applications.

\begin{lem}\label{lem:homomorphic_image_of_sphere}
Let $S=(E,\{e_i\}_{i=1}^r,X,\{x_i\}_{i=1}^r,h)$ be a Q-sphere of dimension $r$ and let $\varphi\colon G\to H$ be a group homomorphism such that, with $K=\ker(\varphi)$,
\begin{enumerate}
\item $K\leq N_G(E)$, $K\cap E=1$, and  $N_G(E)$ surjects onto $N_{\langle \varphi(X)\rangle}(\varphi(E))$, and
\item if $1\leq i\leq r$, $x\in X$, then $xx_i^\epsilon\in XK$ for exactly one value $\epsilon\in \{\pm 1\}$.
\end{enumerate}  
Then 
\[
\varphi(S)=(\varphi(E),\{\varphi(e_i)\}_{i=1}^r,\varphi(X),\{\varphi(x_i)\}_{i=1}^r,\varphi(h))
\]
is a Q-sphere of dimension $r$ for $H$, where $\varphi(h)\colon \varphi(X)\to \{\pm 1\}$ is given by 
\[
\varphi(h)(\varphi(x))=h(x)
\]
for $x\in X$.
\end{lem}
\begin{proof}
Note that $\varphi$ restricted to $X$ is injective: If $\varphi(x)=\varphi(x')$ with $x,x'\in X$, then $x'=xk$ with $k\in K\leq N_G(E)$, and then by condition (a) of Definition \ref{def:Xhsphere} for the Q-sphere $S$, we have that $x=x'$. In particular, $\varphi(h)$ is well defined. In addition, becuase of the injectivity of $\varphi$ restricted to $E$, we have that $\varphi(E)$ is an elementary abelian $p$-subgroup of rank $r$ of $H$ generated by the $\{\varphi(e_i)\}_{i=1}^r$.

Now we check conditions (a), (b) and (c) for $\varphi(S)$. If $\varphi(x')=\varphi(x)m$ with $m\in N_H(\varphi(E))$ then, by hypothesis (1), there exists $n\in N_G(E)$ with $\varphi(n)=m$. Thus, $x'=xnk$ with $k\in K\leq N_G(E)$ and, by condition (a) for $S$, we deduce that $x'=x$. Condition (b) is straightforward as $\varphi(C_G(E_i))\leq C_H(\varphi(E_i))$. For condition (c), assume, without loss of generality, that $xx_i\in X$ and that $\varphi(xx_i^{-1})\in \varphi(X)$. Then, $xx_i^{-1}=x'k$ with $x'\in X$ and $k\in K$, and we are done by hypothesis (2).
\end{proof}

\begin{remark}\label{rmk:sphere_even_characteristic}
In the applications, we will employ Theorem \ref{thm:sphere_implies_QDp} with $G$ a finite simple or almost simple unitary group in cross-characteristic $s$. For $s=2$ or for the case with field automorphisms, the elements $x_i$'s we construct are involutions, and it is readily checked that condition (c) in Definition \ref{def:Xhsphere} cannot be satisfied by involutory elements $x_i$'s. Hence, in theses cases, the Q-sphere we construct will satisfy (b) in Definition \ref{def:Xhsphere} together with the following variant,
\begin{enumerate}[\hspace{1cm}(c2)]\label{def:c2}
\item For $1\leq i\leq r$ and $x\in X$, $x_i$ has order $2$, $xx_i\in X$, and $h(x)+h(xx_i)=0$. 
\end{enumerate}
Then Theorem \ref{thm:sphere_implies_QDp} is valid in this context, i.e., we still have that $\widetilde H_{r-1}(|\A_p(G)|;\ZZ)\neq 0$ for $r=\rk_p(G)$. The arguments are identical but for considering, in the last part, the non-trivial involution $\D(x,\ii,\jj)\to \D(x,\ii,\jj)$, $y\mapsto yx_{j_1}$, and the corresponding partition into subsets of the form $\{y,yx_{j_1}\}$. Moreover, hypothesis (1) of Lemma \ref{lem:homomorphic_image_of_sphere} alone implies that the image by $\varphi$ of a sphere that satisfies (a), (b) and (c2), satisfies these three conditions too.
\end{remark}



\begin{remark}
The hypotheses of Theorem \ref{thm:noncontractiblemoregeneral} imply that $O_p(G)=1$. This is a consequence of  Quillen's result \cite[Proposition 2.4]{Quillen1978} that asserts that $O_p(G)\neq 1\Rightarrow |\A_p(G)|\simeq *$.
\end{remark}

\begin{remark}\label{rmk:comparison_previous_works}
Theorem \ref{thm:sphere_implies_QDp} improves the results of the earlier works \cite{Diaz2016} and \cite{DiazMazza2020} as explained next. There, a maximal elementary abelian $p$-subgroup $E=\langle e_1,\ldots_,e_r\rangle$ of $G$ was considered, together with a family of elements $x_1,\ldots,x_r$ of $G$ satisfying that $x_i\in C_G(E_i)\setminus N_G(\langle e_i\rangle)$ and that $[x_i,x_j]=1$ for all $1\leq i,j\leq r$. We may then define the subset $X$ of $G$ by
\[
X=\{x_1^{\delta_1}\cdots x_r^{\delta_r},\delta_i\in \{0,1\}\text{ for $1\leq i\leq r$}\},
\]
and the function $h\colon X\to \ZZ^*$ by 
\[
h(x_1^{\delta_1}\cdots x_r^{\delta_r})=(-1)^{\delta_1+\ldots+\delta_r}.
\]

Then $(E,\{e_i\}_{i=1}^r,X,\{x_i\}_{i=1}^r,h)$ is a Q-sphere for $G$ in the sense of Definition \ref{def:Xhsphere} by \cite[Lemma 2.6]{DiazMazza2020}, and Theorem \ref{thm:sphere_implies_QDp} implies \cite[Theorem 5.1]{Diaz2016} and \cite[Theorem 3.5]{DiazMazza2020}. The conditions in Definition \ref{def:Xhsphere} are laxer as the  constraint of commutativity is dropped.
\end{remark}

%%%%%%%%%%%%%%%%%%%%%%%%%%%%%%%%%%%%%

\section{Symmetric, linear, unitary and braid groups.}
\label{section:preliminaries_groups}

In this section, we discuss some properties of the symmetric group on $n$ letters $S_n$, Artin's braid group on $n$ strings $B_n$, as well as the special linear group $\SL_n(q)$ and the special unitary group $\SU_n(q)$ over the field of $q$ elements and $q^2$ elements respectively.

\subsection{Symmetric and braid groups.}
\label{subsection:symmetricandbraidgroups}
The groups $S_n$ and $B_n$ have the following presentations,
\begin{align*}
S_n&=\langle s_1,\ldots,s_{n-1}\quad|\quad s_is_{i+1}s_i=s_{i+1}s_is_{i+1}\text{ , }s_is_j=s_js_i\text{ if $|i-j|>1$, }s_i^2=1\rangle,\\
B_n&=\langle \sigma_1,\ldots,\sigma_{n-1}\quad|\quad\sigma_i\sigma_{i+1}\sigma_i=\sigma_{i+1}\sigma_i\sigma_{i+1}\text{ , }\sigma_i\sigma_j=\sigma_j\sigma_i\text{ if $|i-j|>1$}\rangle.
\end{align*}
The element $s_i$ corresponds to the transposition $(i,i+1)$ and $\sigma_i$ represents the geometric braid in which strings $i$ and $i+1$ cross once. There is a homomorphism 
\[
\pi\colon B_n\to S_n
\]
that takes a braid to the underlying permutation of the strings and such that $\pi(\sigma_i)=s_i$. 

We denote by $B_n^+$ the monoid of positive braids, i.e., braids that may be written as a word in positive powers of the generators $\sigma_i$'s. The set $S_n^B$ of \emph{permutation braids} in $B_n^+$ is usually defined as the prefixes or suffixes of the Garside braid. Here we follow a different approach via the length functions  $l\colon S_n\to \NN\cup \{0\}$ \label{equ:lengthSn} and $l\colon B^+_n\to \NN\cup \{0\}$, defined as the length of any \emph{reduced expression}, i.e., as the minimal number of generators $s_i$'s or $\sigma_i$'s, respectively,  needed to express a given permutation or positive braid. Moreover, if $w=s_{i_1}s_{i_2}\ldots s_{i_r}$ with $i_j\in\{1,\ldots,n-1\}$ is a reduced expression for $w\in S_n$, we set $\rho(w)=\sigma_{i_1}\sigma_{i_2}\ldots \sigma_{i_r}\in B_n^+$. This construction gives an injective well defined map, which is not a group homomorphism, 
\[
\rho\colon S_n\to B_n^+,
\]
such that $\pi\circ \rho=1_{S_n}$, see \cite[Section 6.5.2]{KasselTuraev}. Lemmas 6.21 and 6.24 of that section prove the following characterization of $S_n^B$.

\begin{prop}\label{prop:permutatiobraids_characterization}
Let $b\in B_n^+$. Then the following are equivalent.
\begin{enumerate}[\hspace{1cm}(a)]
\item $b\in S_n^B$.
\item $l(\pi(b))=l(b)$.
\item $b\in \im(\rho)$
\end{enumerate}
\end{prop}

This result implies that $\rho\colon S_n\to S_n^B$ is a bijection and that $S^B_n$ is closed under taking inverses. For a permutation $w\in S_n$, we define the \emph{finishing set} $F(w)$  as the following subset of $\{1,\ldots,n-1\}$,
\[
F(w)=\{\quad i\quad|\quad \text{some reduced expression for $w$ ends with $s_i$}\}.
\]

The following result is a consequence of \cite[Corollary 1.4.6, Proposition 1.5.3]{BjornerBrenti}.
\begin{prop}\label{prop:finishing_sets}
For $w\in S_n$ and $i\in \{1,\ldots,n-1\}$, the following are equivalent.
\begin{enumerate}[\hspace{1cm}(a)]
\item $i\in F(w)$.
\item $w(i)>w(i+1)$.
\item $l(ws_i)<l(w)$.
\end{enumerate}
\end{prop}

We finish this subsection with a canonical form for permutations of $S_n$ in which $s_{n-1}$ appears at most once, see \cite[Section 4.1, Lemma 4.3, Exercise 4.1.3]{KasselTuraev}.

\begin{prop}[Normal form in $S_n$]\label{prop:normal_form_Sn}
For each permutation $w\in S_n$, there exist unique elements $w_i\in \{1,s_i,s_is_{i-1},\ldots,s_is_{i-1}\cdots s_2s_1\}$ for $i=1,\ldots,n-1$ such that
\[
w=w_1w_2\cdots w_{n-1},
\]
and such expression is a reduced expression for $w$.
\end{prop}

\subsection{Special linear group.}
\label{subsection:lineargroups}
Let $\GL_n(q)$ and $\SL_n(q)$ denote the general and special linear groups over the field $\FF_q$ of $q$ elements respectively. We consider the following subgroups of $\GL_n(q)$, 
\begin{align*}
B_{\GL_n(q)}&=\text{upper triangular matrices,}\\
U_{\GL_n(q)}&=\text{upper triangular matrices with $1$'s along the diagonal,}\\
T_{\GL_n(q)}&=\text{diagonal matrices,}\\
N_{\GL_n(q)}&=\text{monomial matrices, }
\end{align*}
where a matrix is monomial if it has exactly one non-zero entry in each row and column. We also consider the corresponding subgroups of $\SL_n(q)$,
\begin{align*}
B=B_{\GL_n(q)}\cap \SL_n(q)&\text{, }U=U_{\GL_n(q)}\cap \SL_n(q)\\
T=T_{\GL_n(q)}\cap \SL_n(q)&\text{, and }N=N_{\GL_n(q)}\cap \SL_n(q).
\end{align*}
We denote the elements of $T$ by $t=\diag(t_1,\ldots,t_n)$ with $t_i\in \FF_q^*$. Note that $t_1\cdots t_n=1$ and hence $T\cong \ZZ_{q-1}^{n-1}$. Moreover, $B\cap N=T$, $U\cap T=1$, $B=TU$, $T$ is normal in $N$, and $N/T\cong S_n$, the symmetric group on $n$ letters. This is the standard split $BN$-pair for $\SL_n(q)$. If $w\in S_n$, we denote by $\dt{w}$ a fixed element of $N$ that is mapped to $w$ by the quotient map $N\twoheadrightarrow  S_n$. 

\begin{remark}\label{rmk:normalizer_is_monomial}
By \cite[Exercise (b), p. 19]{Steinberg}, if $q\geq 4$, then $N_{\SL_n(q)}(T)=N$.
\end{remark}


To introduce a \emph{normal form} for elements of $\SL_n(q)$, we need the following preliminaries: By $e_{ij}$ we denote the matrix that has $1$ in position $(i,j)$ and $0$ elsewhere. For $i\neq j$, define $X_{ij}$ as the following abelian subgroup of order $q$,
\[
X_{ij}=\{I_n+\alpha e_{ij}\text{ , }\alpha\in \FF_q\}.
\]
Finally, given $w\in S_n$, consider the following subgroups of $U$,
\begin{align*}
U^+_w&=\langle X_{ij} \quad|\quad i<j\text{ and }w(i)<w(j)\rangle,\\
U^-_w&=\langle X_{ij} \quad|\quad i<j\text{ and }w(i)>w(j)\rangle.
\end{align*}

The following normal form is a refinement of the Bruhat decomposition,
\[
\SL_n(q)=\bigsqcup_{w\in S_n} B\dt{w}B,
\]
see \cite[Theorem 4']{Steinberg} and \cite[5.10 Theorem]{Taylor}.

\begin{prop}[Normal form in $\SL_n(q)$]\label{prop:normal_form_SLnq}
Each element $x\in \SL_n(q)$ can be expressed uniquely in the form $x=b\dt{w}u$, where $b\in B$, $w\in S_n$ and $u\in U^{-}_w$.
\end{prop}

If we map $x\in \SL_n(q)$ to the unique element $w\in S_n$ that appears in the decomposition \ref{prop:normal_form_SLnq}, we obtain a map of sets,
\[
\pi_b\colon \SL_n(q)\to S_n,
\]
which is not a group homomorphism. We will need the following details about normal forms and normal forms of product of double cosets. These are a refinement of the standard formula for product of double cosets.

\begin{prop}\label{prop:particular_normal_forms}
For $1\leq i\leq n-1$ we have $U^-_{s_i}=X_{i,i+1}$ and, for $w\in S_n$, 
\begin{enumerate}[\hspace{1cm}(a)]
\item If $x=b\dt{w}u$ is in normal form and $l(ws_i)=l(w)+1$, then $u_{i,i+1}=0$.
\item If $x=b_x\dt{w}u_x$, $y=b_y\dt{s_i}u_y$ are in normal form and $l(ws_i)=l(w)+1$, then the normal form $xy=b\dt{(ws_i)}u$ satisfies that $u_{i,i+1}={u_y}_{i,i+1}$.
\end{enumerate}
\end{prop}
\begin{proof}
That $U^-_{s_i}=X_{i,i+1}$ is a direct computation from the definition of $U^-_{s_i}$. To prove (a)  and (b), we note first that the condition $l(ws_i)=l(w)+1$ implies, by Proposition \ref{prop:finishing_sets}, that $w(i+1)>w(i)$ or, equivalently, that $(ws_i)(i)>(ws_i)(i+1)$. From here we deduce that 
\begin{equation}\label{equ:particular_normal_forms_basic}
X_{i,i+1}\nleq U^-_w\text{, }X_{i,i+1}\leq U^-_{ws_i}\text{ and }X_{i,i+1}\nleq U^+_{ws_i}.
\end{equation} 
Then (a) is consequence of the first inequality above. To prove (b), we use that for any permutation $z\in S_n$ we have, by \cite[5.9 Lemma]{Taylor},
\begin{equation}\label{equ:U^+U^-}
U=U^+_zU^-_z\text{, }U^+_z\cap U^-_z=1\text{, and }\dt{z}U^+_z\dt{z}^{-1}=U^+_{z^{-1}}\leq B.
\end{equation}
In particular, we note that if $u\in U$ decomposes as $u=u^+u^-$ with $u^{\pm}\in U^\pm_z$, then $u_{j,j+1}=u^+_{j,j+1}+u^-_{j,j+1}$  for all $j$. Now we have,
\[
xy=b_x\dt{w}u_xb_y\dt{s_{i}}u_y=b'\dt{w}u'\dt{s_{i}}u_y,
\]
where $b'\in B$ and we get the element $u'\in U^-_{w}$ after applying Proposition \ref{prop:normal_form_SLnq} to the element $\dt{w}u_xb_y$ that belongs to the double coset $B\dt{w}B$. Next we write
\[
b'\dt{w}u'\dt{s_{i}}u_y=b'\dt{w}\dt{s_i}\dt{s_i}^{-1}u'\dt{s_{i}}u_y=b'\dt{w}\dt{s_i}u''u_y,
\]
with $u''=\dt{s_i}^{-1}u'\dt{s_{i}}$. By part (a), we have that $u'_{i,i+1}=0$, and an easy computation shows then that $u''\in U$ (and also that $u''_{i,i+1}=0$). Hence by Equation \eqref{equ:U^+U^-} applied to $z=ws_i$, we can write $u''=u''^{+}u''^{-}$ with $u''^{\pm}\in U^\pm_{ws_i}$. Thus, applying Proposition \ref{prop:normal_form_SLnq} to $\dt{w}\dt{s_i}u''^{+}$ we get to
\[
b'\dt{w}\dt{s_i}u''u_y=b\dt{w}\dt{s_i}u''^{-}u_y
\]
for some $b\in B$. By Equation \eqref{equ:particular_normal_forms_basic}, we have that $X_{i,i+1}\nleq U^+_{ws_i}$, and hence $u''^{+}_{i,i+1}=0$. As $u''_{i,i+1}=0=u''^{+}_{i,i+1}+u''^{-}_{i,i+1}$, we deduce that $u''^{-}_{i,i+1}=0$. Then $(u''^{-}u_y)_{i,i+1}=u''^{-}_{i,i+1}+{u_y}_{i,i+1}={u_y}_{i,i+1}$. To finish, by Equation \eqref{equ:particular_normal_forms_basic} again, we have that $U^-_{s_i}=X_{i,i+1}\leq  U^-_{ws_i}$ and hence $u=u''^{-}u_y$ belongs to $U^-_{ws_i}$.
\end{proof}

\subsection{Special unitary group.}
\label{subsection:unitarygroups}
Recall that the field $\FF_{q^2}$ of $q^2$ elements has an involution given by $\alpha\mapsto \overline{\alpha}=\alpha^q$, so that we recover the field of $q$ elements as $\FF_q=\{\alpha\in \FF_{q^2}|\alpha=\overline{\alpha}\}$. Moreover, over an $n$-dimensional vector space over $\FF_{q^2}$, we have the sesquilinear form
\[
f((u_1,\ldots,u_n),(v_1,\ldots,v_n))=u_1\overline{v_1}+\ldots+u_n\overline{v_n}.
\]
If $f(u,u)=0$ we say that $u$ is \emph{isotropic} and otherwise we say that $u$ is \emph{anisotropic}. We define the special unitary group $\SU_n(q)$ as the subgroup of \emph{orthonormal} matrices of $\SL_n(q^2)$, i.e., $x\in \SL_n(q^2)$ such that $x\overline{x}^\transpose=I_n$, where $\overline{x}$ is obtained from $x$ by applying $\overline{\cdot}$ to each entry, and $^\transpose$ denotes transpose. We consider the following subgroups of $\SU_n(q)$, 
\begin{align*}
T_U&=T\cap \SU_n(q),\\
N_U&=N\cap \SU_n(q).
\end{align*}
where $T,N\leq \SL_n(q^2)$ were defined in Subsection \ref{subsection:lineargroups}. Note that an element $t=\diag(t_1,\ldots,t_n)$ belongs to $T_U$ if and only if $t_1\cdots t_n=1$ and $t_i^{q+1}=1$ for all $i$, and hence $T_U\cong \ZZ_{q+1}^{n-1}$. A monomial matrix belongs to $N_U$ if and only if its determinant is $1$ and each non-zero entry raised to the $(q+1)$-th power equals $1$. It is clear that $T_U$ is normal in $N_U$.
\begin{remark}\label{rmk:normalizer_is_monomial_unitary}
By Remark \ref{rmk:normalizer_is_monomial}, as $q^2\geq 4$, then $N_{\SL_n(q^2)}(T)=N$, and hence $N_{\SU_n(q)}(T_U)=N_U$.
\end{remark}
 For $n=2$, a generic element $x$ of $\SU_2(q)$ has the form
\[
x(\alpha,\beta)=\begin{pmatrix}
\alpha &\beta\\
-\overline{\beta}& \overline{\alpha}
\end{pmatrix},
\]
for $\alpha,\beta\in \FF_{q^2}$ such that $\alpha\overline{\alpha}+\beta\overline{\beta}=1$. Moreover, for each $1\leq i\leq n-1$, there is a copy $\SU_2(q)_i$ of $\SU_2(q)$ in $\SU_n(q)$ consisting of the matrices $x_i(\alpha,\beta)$,
\begin{scriptsize}
\begin{equation}\label{equ:defx_ialphabeta}
x_i(\alpha,\beta)=
\begin{pmatrix}
1 &  & & & & && \\
  & \ddots & & && &\\
  &  &1\\
  &  & & \alpha &  \beta & & &\\
  &  & & -\overline{\beta} & \overline{\alpha}& & &\\
  &  & & &&  1& &\\
  &  & &      &  && \ddots\\
  &  & & & & & &1&
\end{pmatrix},
\end{equation}
\end{scriptsize} 
where $\alpha$ is at position $(i,i)$. The following centralizer computation is easy and it is related to later verifying Definition \ref{def:Xhsphere}(b).

\begin{lem}\label{lem:centralizers}
Let $\alpha,\beta\in \FF_{q^2}$ be such that $\alpha\overline{\alpha}+\beta\overline{\beta}=1$. Then the centralizer $C_{T_U}(x_i(\alpha,\beta))$ is equal to $T_U$ if $\beta=0$ and equal to the following subgroup if $\beta\neq 0$,
\[
\{\diag(t_1,t_2,\ldots,t_n)\in T_U|t_i=t_{i+1}\}.
\]
\end{lem}


As $\SU_n(q)\leq \SL_n(q^2)$, it makes sense to consider the normal form of elements from $\SU_n(q)$ by applying Proposition \ref{prop:normal_form_SLnq} to the group $\SL_n(q^2)$. Next, we describe the normal form of the elements $x_i(\alpha,\beta)$'s. 

\begin{prop}\label{prop:normalformx_i}
For $s_i\in S_n$ and $\dt{s_i}=x_i(0,1)$, we have, for any $\alpha,\beta\in \FF_{q^2}^*$ with $\alpha\overline{\alpha}+\beta\overline{\beta}=1$ and $\beta\neq 0$, that the normal form of $x_i(\alpha,\beta)$ is  
\[
x_i(\alpha,\beta)=b\dt{s_i}u,
\]
where $b\in B$ and $u\in U^{-}_{s_i}$ are given by
\begin{scriptsize}
\[
b=
\begin{pmatrix}
1 &  & & & & && \\
  & \ddots & & && &\\
  &  &1\\
  &  & & \overline{\beta}^{-1} &  -\alpha & & &\\
  &  & &  & \overline{\beta}& & &\\
  &  & & &&  1& &\\
  &  & &      &  && \ddots\\
  &  & & & & & &1&
\end{pmatrix}\text{, }
u=
\begin{pmatrix}
1 &  & & & & && \\
  & \ddots & & && &\\
  &  &1\\
  &  & & 1 &  -\overline{\alpha}\overline{\beta}^{-1} & & &\\
  &  & &  & 1& & &\\
  &  & & &&  1& &\\
  &  & &      &  && \ddots\\
  &  & & & & & &1&
\end{pmatrix}.
\]\end{scriptsize} 
\end{prop}
\begin{proof}
By Proposition \ref{prop:particular_normal_forms}, $U^-_{s_i}=X_{i,i+1}$. Then it is clear that the element $u$ defined in the statement belongs to $U^-_{s_i}$. Also, $b$ is an upper triangular matrix of determinant $1$, $x_i(0,1)\in N$ goes to $s_i$ via the quotient map $N\twoheadrightarrow  S_n$, and it easy to check that the triple product $b\dt{s_i}u$ is exactly  $x_i(\alpha,\beta)$.
\end{proof}

\section{A subgroup of $\SU_n(q)$.}\label{section:subgroupofSU}

In this section we define a subgroup of $\SU_n(q)$ that is a quotient of the braid group $B_n$. We start by fixing $\zeta,\eta\in \FF_{q^2}^*$ such that

\begin{equation}\label{equ:choice_of_zeta_and_eta}
\overline{\zeta}=-\zeta\text{ and }\overline{\eta}=-\eta^{-1}.
\end{equation}

\begin{remark}\label{rmk:zetanotone}
The trace map $\Tr\colon \FF_{q^2}\to \FF_q=\{\alpha\in \FF_{q^2}|\overline{\alpha}=\alpha\}$ is defined by $\Tr(\alpha)=\overline{\alpha}+\alpha$ and it is $\FF_q$-linear and surjective with kernel $K$ a $\FF_q$-vector space of dimension $1$. Thus, if $q\geq 4$, we can assume that $\zeta\neq 0,1,-1$.
\end{remark}

\begin{remark}\label{rmk:etasqaurenotone}
The norm map $N\colon \FF^*_{q^2}\to \FF^*_q$ is defined by $N(\alpha)=\alpha\overline{\alpha}$ and it is surjective. Hence there exists $\eta\neq 0$ as above.
\end{remark}

If we define $\alpha=1+\zeta$ and $\beta=-\zeta\eta^{-1}$, then we have
\[
\alpha\overline{\alpha}+\beta\overline{\beta}=(1+\zeta)(1-\zeta)+(-\zeta\eta^{-1})(-\zeta\eta)=1-\zeta^2+\zeta^2=1.
\]
Hence, the following element belongs to $\SU_2(q)$,
\[
x(\alpha,\beta)=x(1+\zeta,-\zeta\eta^{-1})=\begin{pmatrix}
1+\zeta&-\zeta\eta^{-1}\\
\zeta\eta& 1-\zeta
\end{pmatrix}.
\]
As $\overline{\zeta^{-1}}=(\overline{\zeta})^{-1}=-\zeta^{-1}$, the next element is also in $\SU_2(q)$,
\[
x(1+\zeta^{-1},-\zeta^{-1}\eta^{-1})=\begin{pmatrix}
1+\zeta^{-1}&-\zeta^{-1}\eta^{-1}\\
\zeta^{-1}\eta& 1-\zeta^{-1}
\end{pmatrix}.
\]
\begin{defn}\label{def:definition_of_xi}
For $1\leq i\leq n-1$, we define $x_i$ as the following element of  $\SU_n(q)$,
\[
x_i=\begin{cases}
x_i(1+\zeta,-\zeta\eta^{-1}),&\text{for $i$ odd,}\\
x_i(1+\zeta^{-1},-\zeta^{-1}\eta^{-1}),&\text{for $i$ even}.
\end{cases}
\]
\end{defn}

\begin{remark}[Transvections]\label{rmk:transvections}
An isotropic vector $v$ and an element $\mu\in \FF^*_{q^2}$ with $\Tr(\mu)=0$, determine an element of $\SU_n(q)$, called \emph{transvection}, as follows,
\[
X_{v,\mu}(u)=u+\mu f(u,v)v.
\]
The element $x_i\in \SU_n(q)$ considered above is the transvection corresponding  to the vector
\[
v_i=(0,\ldots,0, 1, \eta, 0,\ldots, 0)^\transpose,
\]
where the $1$ is at position $i$, and to the scalar $\zeta$ for $i$ odd and to the scalar $\zeta^{-1}$ for $i$ even. The following properties are easy to prove and all but the last one are well-known, 
\begin{subequations}
\begin{align}
X_{\gamma v,\mu}&=X_{v,\gamma\overline{\gamma}\mu}\text{ for all $\gamma\in \FF_{q^2}$,}\label{equ:transvectionsa}\\
X_{v,\mu}X_{v,\mu'}&=X_{v,\mu+\mu'},\label{equ:transvectionsb}\\
X_{v,\mu}X_{v',\mu'}&=X_{v',\mu'}X_{v,\mu}\Leftrightarrow f(v,v')=0,\text{ and}\label{equ:transvectionsc}\\
\ls{x}X_{v,\mu}&=X_{x(v),\mu},\text{ for any element $x\in \SU_n(q)$,}\label{equ:transvectionsd}\\
X_{v,\mu}(u)&=u\text{ if $f(u,v)=0$, and}\label{equ:transvectionse}\\
X_{v,\mu}X_{v',\mu'}X_{v,\mu}&=X_{v',\mu'}X_{v,\mu}X_{v',\mu'}\Leftrightarrow 1+\mu\mu'f(v,v')f(v',v)=0.\label{equ:transvectionsf}
\end{align}
\end{subequations}
\end{remark}

\begin{prop}\label{prop:relationsinSU}
The elements $x_1,x_2,\ldots,x_{n-1}$ of $\SU_n(q)$ have order $s=\ch(q)$ and satisfy the following relations
\[
x_ix_{i+1}x_i=x_{i+1}x_ix_{i+1}\text{ , }x_ix_j=x_jx_i\text{ if $|i-j|>1$}.
\]
\end{prop}
\begin{proof}
By Remark \ref{rmk:transvections}, we have $x_i=X_{v_i,\zeta^{\pm 1}}$. Then the claim about the order is consequence of \eqref{equ:transvectionsb}. The commuting relation for $|i-j|>1$ is a consequence of $x_i$ and $x_j$ being defined over non-overlapping diagonal blocks or a consequence of that $v_i$ and $v_j$ are orthogonal vectors and \eqref{equ:transvectionsc}. The braid relation is a consequence of \eqref{equ:transvectionsf},
\[
1+\zeta\cdot \zeta^{-1} \cdot \eta\cdot \overline{\eta}=1+\eta\overline{\eta}=1-1=0.
\]
One can also perform the computation $x_1x_2x_1=x_2x_1x_2$ in $\SU_3(q)$:
\[
x_1=\begin{pmatrix}
1+\zeta&-\zeta\eta^{-1}&0\\
\zeta\eta& 1-\zeta&0\\
0&0&1\end{pmatrix}\text{, }
x_2=\begin{pmatrix}
1&0&0\\
0&1+\zeta^{-1}&-\zeta^{-1}\eta^{-1}\\
0&\zeta^{-1}\eta& 1-\zeta^{-1}
\end{pmatrix},
\]
and
\[
x_1x_2x_1=x_2x_1x_2=\begin{pmatrix}
1+\zeta& -\eta^{-1}(1+\zeta)&\eta^{-2}\\
\eta(1+\zeta)&-\zeta-1+\zeta^{-1}&\eta^{-1}(1-\zeta^{-1})\\
\eta^2&-\eta(1-\zeta^{-1})&1-\zeta^{-1}
\end{pmatrix}.
\]
\end{proof}

\begin{defn}\label{def:subgroupofSU}
We define the group $B^U_n$ as the subgroup of $\SU_n(q)$ generated by the elements $x_1,x_2,\ldots,x_{n-1}$.
\end{defn}

The group $B^U_n$ is clearly a quotient of braid group $B_n$, and we denote by
\[
\psi\colon B_n\to B^U_n
\]
the group homomorphism sending $\sigma_i$ to $x_i$. 
\begin{remark}\label{rmk:generalizedCoxetergroups}
The group $B^U_n$ is also quotient of the group $G_n$ with presentation
\[
\langle R_1,\ldots,R_{n-1}\quad|\quad R_iR_{i+1}R_i=R_{i+1}R_iR_{i+1}\text{ , }R_iR_j=R_jR_i\text{ if $|i-j|>1$, }R_i^s=1\rangle,
\]
where $s=\ch(q)$. Such a presentation is known as a generalized Coxeter system \cite[Definition 4.1]{ReinerRipollStump} and, by the work of Coxeter in \cite[Section 12.1]{Coxeter}, it is associated to the symbol $s[3]s[3]\ldots s[3]s$ and the generalized Coxeter graph,
\[
\xymatrix@R=0pt@C=10pt{
\bullet\ar@{-}[rr]&&\bullet\ar@{-}[rr]&&\bullet\ar@{-}[r]&\ldots\ar@{-}[r]&\bullet\ar@{-}[rr]&&\bullet.\\
s&3&s&3&s&\ldots&s&3&s
}
\]
If $s=2$, then $G_n$ is the Coxeter group $S_n$. Moreover, Coxeter showed that Shephard groups, i.e., symmetry groups of regular complex polytopes, have similar presentations to that of $G_n$.
\end{remark}

\subsection{Unitary permutation braids}
\label{subsection:unitarypermutationbraids}
So far, we have studied the following groups and maps between them,
\[
\xymatrix{
S_n\ar[r]^\rho&B_n\ar[r]^\pi\ar[d]^\psi&S_n\ar@{=}[d]\\
&B^U_n\ar[r]^{\pi_b}\ar@{^(->}[d]& S_n\ar@{=}[dd]\\
&\SU_n(q)\ar@{^(->}[d]&\\
&\SL_n(q^2)\ar[r]^{\pi_b}&S_n.
}
\]

The maps $\rho$ and $\pi$ in the top row were defined in Subsection \ref{subsection:symmetricandbraidgroups}, $\rho$ is not a group homomorphism, $\pi$ is a group homomorphism, and $\pi\circ \rho=1_{S_n}$. The map $\pi_b$ in second and last rows was defined in Subsection \ref{subsection:lineargroups} and it is not a group homomorphism. Finally, the group $B^U_n$ and the group homomorphism $\psi$ were defined in Section \ref{section:subgroupofSU}. The bottom rectangle is clearly commutative and we show in Proposition \ref{prop:normalformunitarypermutationbraid}(a) below that the top square is commutative when restricted to $\rho(S_n)$. Recall that $\rho(S_n)=S_n^B\subseteq B_n^+\subseteq B_n$ are exactly the permutation braids. We define, analogously, the set of ``unitary permutation braids'' in $B^U_n$.

\begin{defn}\label{def:unitarypermutationbraids}
The set of \emph{unitary permutation braids} is $S_n^U=\psi(S_n^B)\subseteq B^U_n$.
\end{defn}

The next two technical results describe some properties of the subgroup $B^U_n$ and of unitary permutation braids and their normal forms.

\begin{lem}\label{lem:generalized_weighted_row_and_column_sums}
If $x=(x_{ij})_{i,j=1}^n\in B^U_n$, then the following weighted row sum is equal to $1$ for all $1\leq i\leq n$,
\[
\sum_{j=1}^n \eta^{j-i}x_{ij}=1,
\]
and the following weighted column sum is equal to $1$ for all $1\leq j\leq n$,
\[
\sum_{i=1}^n \eta^{j-i}(-1)^{j-i}x_{ij}=1.
\]
\end{lem}
\begin{proof}
Note that the identity matrix satisfies the conditions on the weighted row and column sums of the statement. Moreover, because $B^U_n=\langle x_1,x_2,\ldots,x_{n-1}\rangle$, it is enough to show that, if $M$ is a matrix of size $n\times n$ with coefficients in $\FF_{q^2}$ whose weighted row and column sums are equal to $1$, then so is $N=Mx_l^k$, where $1\leq l\leq n-1$ and $k\in \ZZ$. By  \eqref{equ:transvectionsa}, $x_l^k$ is of the form $x_l(1+\zeta',-\zeta'\eta^{-1})$ for some $\zeta'\in \FF^*_{q^2}$. Writing $M=(m_{ij})_{i,j=1}^n$ and $N=(n_{ij})_{i,j=1}^n$, the weighted row sum for the $i$-th row is,
\[
\sum_{j=1}^n \eta^{j-i}n_{ij}=\sum_{j=1}^{l-1} \eta^{j-i}m_{ij}+\eta^{l-i}n_{il}+\eta^{l+1-i}n_{i(l+1)}+\sum_{j=l+1}^n \eta^{j-i}m_{ij},
\]
where the sum of the two middle summands is as follows for $l$ odd,
\[
\eta^{l-i}((1+\zeta')m_{il}+\zeta'\eta m_{i(l+1)})+\eta^{l+1-i}(-\zeta'\eta^{-1}m_{il}+(1-\zeta')m_{i(l+1)}),
\]
which equals 
\[
\eta^{l-i}((1+\zeta')m_{il}-\zeta' m_{il})+\eta^{l+1-i}(\zeta' m_{i(l+1)}+(1-\zeta')m_{i(l+1)})=\eta^{l-1}m_{il}+\eta^{l+1-i}m_{i(l+1)}.
\]
The case with $l$ even is done analogously, and the argument for the weighted column sums is similar.
\end{proof}


\begin{prop}\label{prop:normalformunitarypermutationbraid}
Assume $q\geq 4$. If $x=\psi(\rho(w))$, $w\in S_n$, is a unitary permutation braid, then 
\begin{enumerate}[\hspace{1cm}(a)]
\item $\pi_b(x)=w$ and, if $x=b\dt{w}u$ with $b\in B$, $u\in U^-_{w}$, is the normal form of $x$, then,
\[
u_{i,i+1}=\begin{cases}
0,&\text{$i\notin F(w)$,}\\
(\zeta^{-1}-1)\eta^{-1},&\text{$i\in F(w)$ and $i$ odd,}\\
(\zeta-1)\eta^{-1},&\text{$i\in F(w)$ and $i$ even}.
\end{cases}
\]
\item if $w=w_1w_2\ldots w_{n-1}$ is the canonical form of $w$ in Proposition \ref{prop:normal_form_Sn} and 
\[
1\neq w_l=s_ls_{l-1}\cdots s_k \text{ and } w_{l+1}=\ldots=w_{n-1}=1,
\]
then the matrix $x$ has block decomposition $\begin{pmatrix}  A & \rvline & 0\\\hline   0 & \rvline & I \end{pmatrix}$, where $A=(a_{ij})_{i,j=1}^{l+1}$, $I$ is the identity matrix,  $a_{(l+1)j}=0$ for $1\leq j<k$, and $(a_{(l+1)j})_{j=k}^{l+1}$ is given as follows,
\begin{small}
\begin{align*}
&&&&&\begin{pmatrix}
\zeta\eta^{l-k+1}&\eta^{l-k}(1-\zeta)&\eta^{l-k-1}(\zeta-1)&\cdots&\eta(\zeta-1)&1-\zeta
\end{pmatrix}\\
&&&&&\begin{pmatrix}
\eta^{l-k+1}&\eta^{l-k}(\zeta^{-1}-1)&\eta^{l-k-1}(1-\zeta^{-1})&\cdots&\eta(\zeta^{-1}-1)&1-\zeta^{-1}
\end{pmatrix}\\
&&&&&\begin{pmatrix}
\eta^{l-k+1}&\eta^{l-k}(\zeta-1)&\eta^{l-k-1}(1-\zeta)&\cdots&\eta(\zeta-1)&1-\zeta
\end{pmatrix}\\
&&&&&\begin{pmatrix}
\zeta^{-1}\eta^{l-k+1}&\eta^{l-k}(1-\zeta^{-1})&\eta^{l-k-1}(\zeta^{-1}-1)&\cdots&\eta(\zeta^{-1}-1)&1-\zeta^{-1}
\end{pmatrix},
\end{align*}
\end{small}
where the rows correspond from top to bottom to the cases $l$ odd and $k$ odd, $l$ even and $k$ odd, $l$ odd and $k$ even, and $l$ even and $k$ even, respectively.
\item with the notations of (b), if for some $1\leq i,k'\leq l+1$, we have that $a_{ij}=0$ for all $1\leq j\leq k'$, then $k'\leq k-1$.
\end{enumerate}
\end{prop}

\begin{proof}
For part (a), we do induction on the common length $l(w)=l(\rho(w))$ of the permutation $w$ and the permutation braid $\rho(w)$, see Proposition \ref{prop:permutatiobraids_characterization}. If $w=1$ the claim is obvious. Let $w=s_{i_1}s_{i_2}\ldots s_{i_r}$ with $i_j\in\{1,\ldots,n-1\}$ and $r\geq 1$ be a reduced expression for $w$, so that $i_r\in F(w)$. Then $w'=s_{i_1}s_{i_2}\ldots s_{i_{r-1}}$ is also a reduced expression. Hence, $\theta=\rho(w)=\sigma_{i_1}\sigma_{i_2}\ldots \sigma_{i_r}$ and $\theta'=\rho(w')=\sigma_{i_1}\sigma_{i_2}\ldots \sigma_{i_{r-1}}$ are permutation braids. Then we have the unitary permutation braids $x=\psi(\theta)=x_{i_1}x_{i_2}\ldots x_{i_r}$ and $x'=\psi(\theta')=x_{i_1}x_{i_2}\ldots x_{i_{r-1}}$ and, by induction, $\pi(x')=w'$, i.e., 
\[
x'=x_{i_1}x_{i_2}\ldots x_{i_{r-1}}=b'\dt{w'}u'
\]
for $b'\in B$ and $u'\in U^-_{w'}$, see Proposition \ref{prop:normal_form_SLnq}. Therefore, 
\[
x=x_{i_1}x_{i_2}\ldots x_{i_{r-1}}x_{i_r}=b'\dt{w'}u'x_{i_r}=b'\dt{w'}u'b''\dt{s_{i_r}}u'',
\]
where $b''\in B$ and $u''\in U^-_{s_i}$ are given by Proposition \ref{prop:normalformx_i}. In particular, $u''_{i,i+1}$ is $(\zeta^{-1}-1)\eta^{-1}$ for $i$ odd and $(\zeta-1)\eta^{-1}$ for $i$ even. As $w=w's_{i_r}=s_{i_1}s_{i_2}\ldots s_{i_r}$ is a reduced expression we have that $l(w's_{i_r})=l(w')+1$ and, by (b) in Proposition \ref{prop:particular_normal_forms}, we know that $x=b\dt{(w's_{i_r})}u$ for some $b\in B$ and $u\in U^{-}_w$ with $u_{i,i+1}=u''_{i,i+1}$. This shows that $\pi(x)=w$ and that $u_{i,i+1}$ is as described in the statement. If $i\notin F(w)$, we apply (a) of Proposition \ref{prop:particular_normal_forms} to conclude that $u_{i,i+1}=0$.

For part (b), note that the element $w_i$ in Proposition \ref{prop:normal_form_Sn} is a reduced expression and may involve only the generators $s_1,\ldots,s_i$. Thus $\psi(\rho(w_i))$ may involve only the generators $x_1,x_2,\ldots,x_i$. Then we have a block decomposition 
\[
\begin{pmatrix}  A & \rvline & 0\\\hline   0 & \rvline & I \end{pmatrix}
\]
for $x=\psi(\rho(w_1w_2\ldots w_l))=\psi(\rho(w_1))\psi(\rho(w_2))\cdots \psi(\rho(w_l))$ with $A$ of size $(l+1)\times (l+1)$, where we have used that $w_1w_2\ldots w_l$ is a reduced expression. For the same reasons, there is a similar block decomposition for $\psi(\rho(w_1w_2\ldots w_{l-1}))$ with left top matrix of size $(l+1)\times(l+1)$ equal to
\[
\begin{pmatrix}  B & \rvline & 0\\\hline   0 & \rvline & 1 \end{pmatrix},
\]
where $B$ has size $l\times l$, and a similar block decomposition for $\psi(\rho(w_l))$ with left top matrix $C$ of size $(l+1)\times (l+1)$. From the expression
\[
x=\psi(\rho(w_1w_2\ldots w_l))=\psi(\rho(w_1w_2\ldots w_{l-1}))\psi(\rho(w_l)),
\]
we deduce that 
\begin{equation}\label{equ:A=BC}
A= \begin{pmatrix}  B & \rvline & 0\\\hline   0 & \rvline & 1 \end{pmatrix} \cdot C.
\end{equation}
Hence the $(l+1)$-th rows of $A$ and $C$ are identical, and it is enough to prove that $C$ satisfies the conditions in part (b) of the theorem. To that aim, we denote by $C^m=(c^m_{ij})_{i,j=1}^{l+1}$ the left top matrix of size $(l+1)\times (l+1)$ for the product $\psi(\rho(s_ms_{m-1}\ldots s_k))$ with $k\leq m\leq l$, and we show that $C^m$ satisfies the following two conditions, regarding the $(m+1)$-th row and $k$-th column of $C^m$ respectively,
\begin{equation}\label{equ:bottom_row}
(c^m_{(m+1)j})_{j=1}^{m+1}\text{ satisfies part (b) of the theorem,}
\end{equation}
and, for all $i$ with $k\leq i\leq m$,
\begin{equation}\label{equ:left_column}
c^m_{ik}=\begin{cases}
(1+\zeta)\eta^{i-k},&\text{if $k$ is odd,}\\
(1+\zeta^{-1})\eta^{i-k},&\text{if $k$ is even.}\\
\end{cases}
\end{equation}
The case $m=l$ in \eqref{equ:bottom_row} proves part (b) of the theorem, and \eqref{equ:left_column} will be used later to prove part (c) of the theorem. Clearly, \eqref{equ:bottom_row} and \eqref{equ:left_column} hold for $m=k$ by the definition of $x_k$ in \ref{def:definition_of_xi}. Now, assume both equations hold for $C^{m-1}$ and note that
\[
C^m=\widehat{x_m}\cdot C^{m-1},
\]
where $\widehat{x_m}$ consists of the first $l+1$ rows and columns of the matrix $x_m$. Note that  $C^{m-1}$ has decomposition
\begin{equation}\label{equ:Cm-1_decomposition}
C^{m-1}= \begin{pmatrix}  I & \rvline & 0 & \rvline & 0\\\hline   0 & \rvline & \ast & \rvline & 0 \\\hline 0 & \rvline & 0 & \rvline & I \end{pmatrix},
\end{equation}
where the top left, middle, and bottom right squares have sizes $(k-1)\times (k-1)$, $(m-k+1)\times (m-k+1)$, and $(l-m+1)\times (l-m+1)$ respectively. Then, the $(m+1)$-th row of $C^m$ is given by 
\[
c^m_{(m+1)j}=\begin{cases} \zeta\eta c^{m-1}_{mj},&\text{$1\leq j<m+1$,}\\
1-\zeta,&\text{$j=m+1$,}
\end{cases}
\]
if $m$ is odd, and as follows if $m$ is even,
\[
c^m_{(m+1)j}=\begin{cases} \zeta^{-1}\eta c^{m-1}_{mj},&\text{$1\leq j<m+1$,}\\
1-\zeta^{-1},&\text{$j=m+1$.}
\end{cases}
\]
We clearly have $c^m_{(m+1)j}=0$ for $1\leq j<k$ and it is left to check that, considering the four rows in the statement of part (b) of the theorem, the first row times $\zeta^{-1}$ gives the second row but for the last term,  the second row times $\zeta$ gives the first row but for its last term, and an analogous statement for the third and fourth rows. This is readily checked and hence \eqref{equ:bottom_row} is valid.

From \eqref{equ:Cm-1_decomposition} again, the $k$-th column of $C^m$ is identical to that of $C^{m-1}$ for the rows $i$ in the range $k\leq i\leq m-1$, and 
\[
c^m_{mk}=\begin{cases}
c^{m-1}_{mk}(1+\zeta),&\text{if $m$ is odd,}\\
c^{m-1}_{mk}(1+\zeta^{-1}),&\text{if $m$ is even.}
\end{cases}
\]
From part (b) of this theorem, we have that
\[
c^{m-1}_{mk}=\begin{cases}
\zeta\eta^{m-k},&\text{$m$ even, $k$ odd,}\\
\eta^{m-k},&\text{$m$ odd, $k$ odd,}\\
\eta^{m-k},&\text{$m$ even, $k$ even,}\\
\zeta^{-1}\eta^{m-k},&\text{$m$ odd, $k$ even.}
\end{cases}
\]
Checking the different cases, it follows that \eqref{equ:left_column} is valid.


Now we prove part (c) of the theorem, and we keep the same notation as for the proof of part (b). Assume that $a_{ij}=0$ for all $1\leq j\leq k'$, where $1\leq i,k'\leq l+1$. If $i=l+1$, we obtain that $k'\leq k-1$ from part (b) and Remarks \ref{rmk:zetanotone} and \ref{rmk:etasqaurenotone}. Thus we can assume that $1\leq i\leq l$ and $k'\geq k$, so that $a_{ij}=0$ for all $1\leq j\leq k$. From Equation \eqref{equ:A=BC} and decomposition \eqref{equ:Cm-1_decomposition} for $C=C^l$, this is equivalent to
\begin{equation}\label{equ:bij_is_zero}
\text{ $b_{ij}=0$ for all $1\leq j\leq k-1$}
\end{equation}
and $\sum_{j=k}^{l} b_{ij}c_{jk}=0$, which, by \eqref{equ:left_column}, can be rewritten as
\begin{equation}\label{equ:product_bijcjk_is_zero}
\sum_{j=k}^{l} b_{ij}(1+\zeta^\delta)\eta^{j-k}=0,
\end{equation}
where $\delta=(-1)^{k+1}$. Because $B$ if the top left corner of $\psi(\rho(w_1w_2\ldots w_{l-1}))$, and by Lemma \ref{lem:generalized_weighted_row_and_column_sums}, we obtain the following contradiction,
\[
1=\sum_{j=1}^l \eta^{j-i}b_{ij}=\sum_{j=1}^{k-1}\eta^{j-i}b_{ij}+\sum_{j=k}^{l}\eta^{j-i}b_{ij}=\frac{\eta^{k-i}}{1+\zeta^\delta}\sum_{j=k}^{l}(1+\zeta^\delta)\eta^{j-k}b_{ij}=0,
\]
where we have used Equations \eqref{equ:bij_is_zero} and \eqref{equ:product_bijcjk_is_zero} and Remark  \ref{rmk:zetanotone}.
\end{proof}



Part (a) of this result implies that $\psi\colon S^B_n\to S^U_n$ is a bijection between the permutation braids and the unitary permutation braids, and hence we also have a bijection between permutations and unitary permutation braids 
\[
S_n\stackrel{\rho}\to S^B_n\stackrel{\psi}\to S^U_n.
\] 
In particular, the following notion of length is well defined.

\begin{defn}\label{def:length_of_unitary_permutation_braid}
If $x=\psi(\rho(w))$ is a unitary permutation braid with $w\in S_n$, we defined the \emph{length} of $x$ as $l(x)=l(w)$.
\end{defn}

Because of Proposition \ref{prop:permutatiobraids_characterization}(b), we could have also written $l(x)=l(\rho(w))$ in the definition above.

\begin{exa}\label{exa:unitarypermutationbraids_SU3}
In this example we describe the unitary permutation braids  in $\SU_3(q)$ and their normal forms. The following table contains the non-trivial unitary permutation braids $x=\psi(\rho(w))\in S^U_3$ with $w\in S_3$. The last column gives the element $u_x\in U^-_w$ for the normal form $x=b_x\dt{w}u_x$ of the unitary permutation braid $x$ such that  $\pi(x)=w\in S_3$.
\begin{center}
\begin{tabular}[h]{|c|c|c|c|}
\hline
$x\in S^U_3$ & $x\in \SL_3(q^2)$ & $u_x$\\
\hline
$x_1$&{\scriptsize $\begin{pmatrix}
1+\zeta&-\zeta\eta^{-1}&0\\
\zeta\eta& 1-\zeta&0\\
0&0&1\end{pmatrix}$}&{\scriptsize $\begin{pmatrix}
1&(\zeta^{-1}-1)\eta^{-1}&0\\
0& 1&0\\
0&0&1\end{pmatrix}$}\\ 
\hline
$x_2$&{\scriptsize $\begin{pmatrix}
1&0&0\\
0&1+\zeta^{-1}&-\zeta^{-1}\eta^{-1}\\
0&\zeta^{-1}\eta& 1-\zeta^{-1}
\end{pmatrix}$}&{\scriptsize $\begin{pmatrix}
1&0&0\\
0&1&(\zeta-1)\eta^{-1}\\
0&0&1\end{pmatrix}$}\\
\hline
$x_1x_2$&{\scriptsize $\begin{pmatrix}
1+\zeta&-(1+\zeta)\eta^{-1}&\eta^{-2}\\
\zeta\eta&\zeta^{-1}-\zeta&(1-\zeta^{-1})\eta^{-1}\\
0&\zeta^{-1}\eta& 1-\zeta^{-1}
\end{pmatrix}$}& {\scriptsize $\begin{pmatrix}
1&0&(\zeta-1)\eta^{-2}\\
0&1&(\zeta-1)\eta^{-1}\\
0&0&1\end{pmatrix}$}\\
\hline
$x_2x_1$&{\scriptsize $\begin{pmatrix}
1+\zeta&-\zeta\eta^{-1}&0\\
(1+\zeta)\eta&\zeta^{-1}-\zeta&-\zeta^{-1}\eta^{-1}\\
\eta^2&(\zeta^{-1}-1)\eta& 1-\zeta^{-1}
\end{pmatrix}$}&
{\scriptsize $\begin{pmatrix}
1&(\zeta^{-1}-1)\eta^{-1}&(1-\zeta^{-1})\eta^{-2}\\
0&1&0\\
0&0&1\end{pmatrix}$}\\
\hline
$x_1x_2x_1$&{\scriptsize $\begin{pmatrix}
1+\zeta& -\eta^{-1}(1+\zeta)&\eta^{-2}\\
\eta(1+\zeta)&-\zeta-1+\zeta^{-1}&\eta^{-1}(1-\zeta^{-1})\\
\eta^2&(\zeta^{-1}-1)\eta&1-\zeta^{-1}
\end{pmatrix}$}&
{\scriptsize $\begin{pmatrix}
1&(\zeta^{-1}-1)\eta^{-1}&(1-\zeta^{-1})\eta^{-2}\\
0&1&(\zeta-1)\eta^{-1}\\
0&0&1\end{pmatrix}$}\\
\hline
\end{tabular}
\end{center}
Recall that $s_2s_1s_2=s_1s_2s_1$, $\sigma_2\sigma_1\sigma_2=\sigma_1\sigma_2\sigma_1$  and $x_2x_1x_2=x_1x_2x_1$.
\end{exa}

\subsection{Properties of unitary permutation braids.}
\label{subsection:properties_of_unitary_permutation_braids}

In this subsection, we study further properties of unitary permutation braids that are consequence of Lemma \ref{lem:generalized_weighted_row_and_column_sums} and Proposition \ref{prop:normalformunitarypermutationbraid}. The results here are related to the notion of Q-sphere in Definition \ref{def:Xhsphere} and, in particular, the following theorem will be employed later to verify Definition \ref{def:Xhsphere}(a).

\begin{thm}\label{thm:X_n_faithful_on_torus}
Assume $q\geq 4$. Let $w,w'\in S_n$ be permutations and consider the unitary permutation braids $x=\psi(\rho(w))$ and $x'=\psi(\rho(w'))$. If $x'=x\cdot g$ for some $g\in N_{\GL_n(q^2)}$, then $x'=x$ and $g=1$.
\end{thm}
\begin{proof}
From $x'=xg$, we have  $x'^{-1}=g^{-1}x^{-1}$, where $g^{-1}\in  N_{\GL_n(q^2)}$ and $x^{-1}=\psi(\rho(w^{-1}))$ and $x'^{-1}=\psi(\rho(w'^{-1}))$ are unitary permutation braids. Decompose $g^{-1}=td$, where $t\in N_{\GL_n(q^2)}$ is a permutation matrix, i.e., $t$ has entries $0$'s and $1$'s and represents the (row) permutation $\sigma\in S_n$, and where $d=\diag(d_1,\ldots,d_n)$ is a diagonal matrix. By Proposition \ref{lem:generalized_weighted_row_and_column_sums}, the weighted row sums of $x^{-1}$ are equal to $1$. Note that the $i$-th row of $dx^{-1}$ is the $i$-th row of $x^{-1}$ multiplied by $d_i$. Moreover, the rows of $tdx^{-1}$ are the rows of $dx^{-1}$ in the ordered determined by $\sigma$. So the weighted row sum of the $\sigma(i)$-th row of $tdx^{-1}$ is exactly $d_i\eta^{i-\sigma(i)}$. As $tdx^{-1}=x'^{-1}$ and the weighted row sums of $x'^{-1}$ are all equal to $1$, we conclude that $d_i=\eta^{\sigma(i)-i}$ for $1\leq i\leq n$.

Now, by Proposition \ref{prop:normal_form_Sn}, we can write 
\[
w'^{-1}=w'_1w'_2\ldots w'_{n-1}\text{ and }w^{-1}=w_1w_2\ldots w_{n-1}
\]
for some permutations $w_i$, $w'_i$, $1\leq i\leq n-1$. Assume that $w_{l'}\neq 1$, $w_{l'+1}=w_{l'+2}=\ldots=w'_{n-1}=1$ and that $w_l\neq 1$, $w_{l+1}=w_{l+2}=\ldots=w_{n-1}=1$ for some integers $0\leq l,l',\leq n-1$. Then 
\begin{equation}\label{equ:unitarypbs_sphere_aux}
w'^{-1}=w'_1w'_2\ldots w'_{l'}\text{ and }w^{-1}=w_1w_2\ldots w_l.
\end{equation}
We can also write
\[
w'_{l'}=s_{l'}s_{l'-1}\ldots s_{k'}\text{ and }w_{l}=s_{l}s_{l-1}\ldots s_{k}
\]
for $1\leq k\leq l$ and $1\leq k'\leq l'$. By Proposition \ref{prop:normalformunitarypermutationbraid}(b), the matrices $x'^{-1}$ and $x^{-1}$ have block decompositions,
\[
\begin{pmatrix}  A' & \rvline & 0\\\hline   0 & \rvline & I \end{pmatrix}
\text{ and }
\begin{pmatrix}  A & \rvline & 0\\\hline   0 & \rvline & I \end{pmatrix},
\]
with $A'$ of size $(l'+1)\times (l'+1)$, $A$ of size $(l+1)\times (l+1)$, and where the bottom rows of $A'$ and $A$ are described in that proposition. Note that, as $x'^{-1}=tdx^{-1}$, the rows of $x'^{-1}$ are the rows of $x^{-1}$ multiplied by $d$ and reordered by the permutation matrix $t$. If $l'<l$ then the $(l+1)$-th row of $dx^{-1}$ must be sent to a row of $x'^{-1}$ such that either it has either $0$ in the column $(l+1)$, or it has $1$ in the column $(l+1)$ and $0$ in the column $l$. But, by Proposition \ref{prop:normalformunitarypermutationbraid}(b), the  $l$-th and $(l+1)$-th columns of the $(l+1)$-th row of $dx^{-1}$ are given by the following matrix times $\eta^{\sigma(l+1)-(l+1)}$,
\[
\begin{pmatrix} \eta(\zeta-1)&(1-\zeta)\end{pmatrix}\text{ or }\begin{pmatrix} \zeta\eta &(1-\zeta)\end{pmatrix}\text{ or }\begin{pmatrix} \eta &(1-\zeta)\end{pmatrix}
\]
if $l$ is odd, and by 
\[
\begin{pmatrix} \eta(\zeta^{-1}-1)&(1-\zeta^{-1})\end{pmatrix}\text{ or }\begin{pmatrix} \zeta^{-1}\eta &(1-\zeta^{-1})\end{pmatrix}\text{ or }\begin{pmatrix} \eta &(1-\zeta^{-1})\end{pmatrix}
\]
if $l$ is even. Checking cases, we would obtain then that either $\eta=0$ or $\zeta=0$ or $\zeta=1$, which are contradictions with Remarks \ref{rmk:zetanotone} or \ref{rmk:etasqaurenotone}. Thus $l'\geq l$. By considering $x^{-1}=t^{-1}{}^t(d^{-1})x'^{-1}$, a similar argument shows that $l\geq l'$. Hence $l=l'$. 

As $l=l'$, from the block decompositions above, we conclude that $t$ must fix the rows $l+2$, $l+3$,.., $n$ and that, in particular, $\sigma(l+1)\leq l+1$. Assume that $k'<k$. Then the entries of the $(l+1)$-th row of $dx^{-1}$  with column indexes $1,2,\ldots,k'$ are zero, and this row it is sent to a row of $x'^{-1}$ with row index $\sigma(l+1)\leq l+1=l'+1$. This is impossible by Proposition \ref{prop:normalformunitarypermutationbraid}(c) and hence $k'\geq k$.  A similar argument with the equation $x^{-1}=t^{-1}{}^t(d^{-1})x'^{-1}$ shows that $k\leq k'$. Hence $k=k'$ and the bottom rows of $x'^{-1}$ and $x^{-1}$ are identical. Moreover, if $\sigma(l+1)< l+1$, a multiple of the $(l+1)$-row of $x'^{-1}$ equals the $\sigma(l+1)$-th row of $x'^{-1}$. This implies that the determinant of $x'^{-1}$ is zero, which is a contradiction as $x'\in \SU_n(q)$. Hence, $\sigma(l+1)=l+1$ and $d_i=1$ for $i=l+1,\ldots,n$.

As $l=l'$ and $k=k'$, we have that $w'_{l'}=w_{l}$ and hence, as \eqref{equ:unitarypbs_sphere_aux} are reduced expressions, we have,
\begin{align*}
x'^{-1}=tdx^{-1}&\Leftrightarrow \psi(\rho(w'^{-1})) = td \psi(\rho(w^{-1}))\\&\Leftrightarrow\psi(\rho(w'_1w'_2\ldots w'_{l'-1}))\psi(\rho(w'_{l'})) = td \psi(\rho(w_1w_2\ldots w_{l-1}))\psi(\rho(w_l))\\
&\Leftrightarrow\psi(\rho(w'_1w'_2\ldots w'_{l'-1}))= td \psi(\rho(w_1w_2\ldots w_{l-1}))
\end{align*}
Then we can repeat the arguments from the the last two paragraphs to show that $w'_{l'-1}=w_{l-1}$, $w'_{l'-2}=w_{l-2}$, ..., $w'_2=w_2$, and $w'_1=w_1$. Thus we conclude that $x'=x$ and that $g=d=t=1$.
\end{proof}

The next result will be employed to later verifying Definition \ref{def:Xhsphere}(c).

\begin{thm}\label{thm:X_n_no_squares}
Assume $s=\ch(q)$ is odd. Let $x=\psi(\rho(w))$ with $w\in S_n$ be a unitary permutation braid and $1\leq i\leq n-1$. Then $xx_i^\epsilon\in S^U_n$ for exactly one value $\epsilon \in\{-1,+1\}$. Moreover, $l(xx_i^\epsilon)=l(x)+\epsilon$.
\end{thm}
\begin{proof}
Let $w=s_{i_1}\ldots s_{i_r}$ be a reduced expression for $w$. Then,
\begin{enumerate}
\item If $l(ws_i)=l(w)+1$, then $s_{i_1}\ldots s_{i_r}s_i$ is a reduced expression by Proposition \ref{prop:finishing_sets} and $x'=xx_i=x_{s_{i_1}}\ldots x_{s_{i_r}}x_i=\psi(\rho(s_{i_1}\ldots s_{i_r}s_i))$ is a unitary permutation braid. Moreover, by Definition \ref{def:length_of_unitary_permutation_braid},
\[
l(xx_i)=l(x')=l(s_{i_1}\ldots s_{i_r}s_i)=l(s_{i_1}\ldots s_{i_r})+1=l(x)+1.
\]
\item If $l(ws_i)=l(w)-1$, then $w=s_{i_1}\ldots s_{i_r}=s_{j_1}\ldots s_{j_{r-1}}s_i$ for some reduced expression $s_{j_1}\ldots s_{j_{r-1}}s_i$ by Proposition \ref{prop:finishing_sets}. Then $x''=\psi(\rho(s_{j_1}\ldots s_{j_{r-1}}))=x_{j_1}\ldots x_{j_{r-1}}\in S^U_n$ and $x=\psi(\rho(s_{j_1}\ldots s_{j_{r-1}}s_i))=x_{j_1}\ldots x_{j_{r-1}}x_i$. Hence $xx_i^{-1}=x''\in S^U_n$. Moreover, by Definition \ref{def:length_of_unitary_permutation_braid},
\[
l(xx_i^{-1})=l(x'')=l(s_{j_1}\ldots s_{j_{r-1}})=l(s_{j_1}\ldots s_{j_{r-1}}s_i)-1=l(x)-1.
\]
\end{enumerate}
Assume now that both $x'=xx_i$ and $x''=xx_i^{-1}$ are unitary permutation braids. Then $x'=xx_i=x''x_i^2$ is a unitary permutation braid.  Consider the following normal forms,
\[
x=b\dt{w}u\text{, }x'=b'\dt{w'}u'\text{, }x''=b''\dt{w''}u''\text{ and  }x_i^2=b_2\dt{s_i}u_2.
\]
As $x_i=x_i(1+\zeta',-\zeta'\eta^{-1})$ with $\zeta'\in \{\zeta,\zeta^{-1}\}$ and  $x_i^2=x_i(1+2\zeta',-2\zeta'\eta^{-1})$, we have $(u_2)_{i,i+1}=(2\zeta'-1)\eta^{-1}$ by Proposition \ref{prop:normalformx_i}. Then:
\begin{enumerate}
\item If $l(w''s_i)=l(w'')+1$, then the normal form of $x'=x''x_i^2$ satisfies that $u'_{i,i+1}=((2\zeta')^{-1}-1)\eta^{-1}$ and that $w'=w''s_i$ by (b) in Proposition \ref{prop:particular_normal_forms}. Also, by Proposition \ref{prop:finishing_sets}, we have that $i\in F(w')$, and hence,  by Proposition \ref{prop:normalformunitarypermutationbraid}(a), we have $u'_{i,i+1}=(\zeta'^{-1}-1)\eta^{-1}$. Thus, $\zeta'=0$. This is a contradiction as $\zeta'\in \FF_{q^2}^*$.
\item If $l(w''s_i)=l(w'')-1$, then there is a reduced expression $w''=w'''s_i$. Hence $x'''=\psi(\rho(w'''))\in S^U_n$ and $x''=x'''x_i$, so
\[
xx_i=x'=x''x_i^2=x'''x_ix_i^2=x'''x_i^3,
\]
from where $x=x'''x_i^2$. Now we have $l(w'''s_i)=l(w'')=l(w''s_i)+1=l(w''')+1$. Then the normal form of $x=x'''x_i^2$ satisfies,  by (b) in   Proposition \ref{prop:particular_normal_forms} again, that $w=w'''s_i$ and that $u_{i,i+1}=((2\zeta')^{-1}-1)\eta^{-1}$. But then  $i\in F(w)$, and from Proposition \ref{prop:normalformunitarypermutationbraid}(a), we have that $u_{i,i+1}=(\zeta'^{-1}-1)\eta^{-1}$. Thus, $\zeta'=0$, and this is a contradiction as $\zeta'\in \FF_{q^2}^*$.
\end{enumerate}
\end{proof}

\begin{remark}\label{rmk:oneepsilon_with_centre}
If $x$ is a permutation braid in $S^U_n$ and $1\leq i\leq n-1$, then it also holds the following stronger version of Theorem \ref{thm:X_n_no_squares}, if $s=\ch(q)$ is odd,
\[
xx_i\in S^U_n\Rightarrow xx^{-1}_iz\notin S^U_n\text{ and }xx_i^{-1}\in S^U_n\Rightarrow xx_iz\notin S^U_n
\]
for all $z\in Z(\GU_n(q))$. This result will be employed to check in the applications that hypothesis (2) in Lemma \ref{lem:homomorphic_image_of_sphere} holds. Assume $xx_i^{-1}z=x'\in S^U_n$ with $z=\diag(\lambda,\ldots,\lambda)$. Then the weighted row sum of this element must be, by Lemma \ref{lem:generalized_weighted_row_and_column_sums}, equal to $\lambda$ by using the left-hand expression, and equal to $1$ by using the right-hand expression. So $z=1$ and we may apply Theorem \ref{thm:X_n_no_squares}. 
\end{remark}

\section{Unitary groups}
\label{section:unitary}

In this section, we start by combining the geometric construction of non-trivial homology classes in Section \ref{section:preliminaries} with the properties of the subset of unitary permutation braids $S^U_n\subseteq \SU_n(q)$ defined in Section \ref{section:subgroupofSU} in order to prove Quillen's conjecture for unitary groups. 

\begin{remark}\label{rmk:sphere_even_characteristic_applications}
For $s=\ch(q)=2$, the spheres we construct in Theorems \ref{thm:QDp_for_SUn}, \ref{thm:QDp_for_PSUn}, and \ref{thm:QDp_for_PGUn}  satisfy (c2) of Remark \ref{rmk:sphere_even_characteristic} instead of (c) from Definition \ref{def:Xhsphere}. Then Theorem \ref{thm:X_n_no_squares} and Remark \ref{rmk:oneepsilon_with_centre} do not apply, but we still have non-zero top dimensional homology as explained in Remark \ref{rmk:sphere_even_characteristic}. In Theorem \ref{thm:QDp_for_PGUnfield}, (c2) of Remark \ref{rmk:sphere_even_characteristic} is also employed.
\end{remark}

\begin{thm}\label{thm:QDp_for_SUn}
If $q\neq 2$, $p$ is odd, $p\mid (q+1)$, $p\nmid n$, then $\widetilde H_{n-2}(|\A_p(\SU_n(q))|;\ZZ)\neq 0$ and $\Q\D_p$ holds for $\SU_n(q)$.
\end{thm}
\begin{proof}
As $p\mid (q+1)$, there is an element $u$ of order $p$ in $\FF_{q^2}^*$, and hence $u$ satisfies that $u^{q+1}=1$. In addition, by \cite[4.10.3(a)]{GLSIII}, $\rk_p(\SU_n(q))=n-1$. Consider the maximal elementary abelian $p$-subgroup of $\SU_n(q)$, $E=\langle e_1,\ldots,e_{n-1}\rangle$, where
\[
e_i=\diag(u^{i-n},\ldots,u^{i-n},u^i,\ldots,u^i),
\]
where $u^{i-n}$ appears $i$ times and $u^i$ appears $n-i$ times. Note that, setting $e_0=1$, we have for $1\leq i\leq n-1$, that
\[
e_ie_{i-1}^{-1}=\diag(u,\ldots,u,u^{1-n},u,\ldots,u),
\]
where $u^{1-n}$ appears at the $i$-th entry. We show below $(E,\{e_i\}_{i=1}^{n-1},X,\{x_i\}_{i=1}^{n-1},h)$ is a Q-sphere of dimension $n-1$ for $\SU_n(q)$ (Definition \ref{def:Xhsphere}), where $e_i$ is defined above, $X=S^U_n \subseteq \SU_n(q)$ is the set of unitary permutation braids (Definition \ref{def:unitarypermutationbraids}), the elements $x_1,\ldots,x_{n-1}$ are defined in  \ref{def:definition_of_xi}, and $h\colon X\to \{\pm 1\}$, $x\mapsto (-1)^{l(x)}$, with $l(x)$ given by Definition \ref{def:length_of_unitary_permutation_braid}. Then Theorem \ref{thm:sphere_implies_QDp} proves the thesis of this theorem.  

\textbf{(a) in Definition \ref{def:Xhsphere} holds:} Let $x,x'\in X$ and $n=N_{\SU_n(q)}(E)$ such that $x'=xn$. As $N_{\SU_n(q)}(E)=N_{\SU_n(q)}(T_U)$, we have, by Remark \ref{rmk:normalizer_is_monomial_unitary}, that $N_{\SU_n(q)}(E)=N_U\leq N\leq N_{\GL_n(q^2)}$. Then $x'=x$ by Theorem \ref{thm:X_n_faithful_on_torus}, and this theorem may be applied as the hypotheses $q\neq 2$, $p$ odd, $p\mod (q+1)$, imply that $q\geq 4$.

\textbf{(b) in Definition \ref{def:Xhsphere} holds:} Fix $1\leq i\leq n-1$ and recall that the subgroup $E_i$ is given by 
\[
\langle e_1,\ldots,\widehat{e_i},\ldots,e_{n-1}\rangle.
\]
In addition, $x_i$ equals $x_i=x_i(1+\zeta,-\zeta\eta^{-1})$ or $x_i(1+\zeta^{-1},-\zeta^{-1}\eta^{-1})$, where $-\zeta\eta^{-1}\neq 0$ and $\zeta^{-1}\eta^{-1}\neq 0$. Then, by the computation in Lemma \ref{lem:centralizers}, $x_i\in C_{\SU_n(q)}(e_j)$ for $j\neq i$, and hence $x_i\in C_{\SU_n(q)}(E_i)$.

\textbf{(c) in Definition \ref{def:Xhsphere} holds:} This is exactly the content of Theorem \ref{thm:X_n_no_squares}.
\end{proof}

\begin{remark}\label{rmk:Coxeter_complex}
The bijection $\psi\circ \rho$ from $S_n$ to $S^U_n$ has additional properties: Recall that the reduced words in the generators $s_i$'s of $S_n$ modulo the braid relations are in bijection with the elements of $S_n$. Moreover, the braid relations are also satisfied by the elements $x_i$'s by Proposition \ref{prop:relationsinSU}. Then it is an exercise to check that, for $x=\psi(\rho(w))$, $x'=\psi(\rho(w'))$, $w,w'\in S_n$, and $1\leq i\leq n-1$, we have
\begin{align*}
xx_i=x'\Leftrightarrow ws_i=w'\text{ and }l(ws_i)>l(w'),\\
x=x'x_i\Leftrightarrow w=w's_i\text{ and }l(ws_i)<l(w').
\end{align*}
In particular, if $w$ and $w'$ are adjacent in the Coxeter complex of $S_n$, then the two maximal faces ${}^x\barsub_E$ and  ${}^{x'}\barsub_E$ of the chain constructed in the previous theorem are adjacent. Note that we do not claim that the opposite implication holds.
\end{remark}

Next, we work out the case of the simple unitary group. Recall that $\PSU_n(q)=\SU_n(q)/Z$ with 
\[
Z=Z(SU_n(q))=\{\diag(z,\ldots,z)|z^n=z^{q+1}=1\}=\ZZ_{gcd(n,q+1)}.
\]
We write $[x]\in \PSU_n(q)$ for the class of $x=(x_{i,j})_{i,j=1}^n\in \SU_n(q)$  and $[X]=\{[x],x\in X\}$ if $X$ is a subgroup or a subset of $\SU_n(q)$. We write $(\cdot)_p$ for the $p$-share of an integer.

\begin{thm}\label{thm:QDp_for_PSUn}
If $p$ is odd, $q\neq 2$ and $p\mid (q+1)$, then
\begin{enumerate}[\hspace{1cm}(a)]
\item if $p\nmid n$, then $\widetilde H_{n-2}(|\A_p(\PSU_n(q))|;\ZZ)\neq 0$
\item if $p\mid n$, $(n)_p<(q+1)_p$, then $\widetilde H_{n-2}(|\A_p(\PSU_n(q))|;\ZZ)\neq 0$
\item if $p\mid n$, $(n)_p\geq (q+1)_p$, then $\widetilde H_{n-3}(|\A_p(\PSU_n(q))|;\ZZ)\neq 0$
\end{enumerate}
In particular, $\Q\D_p$ holds for $\PSU_n(q)$.
\end{thm}
\begin{proof}
Assume first that $p\nmid n$ and let $(E,\{e_i\}_{i=1}^{n-1},X,\{x_i\}_{i=1}^{n-1},h)$ be the Q-sphere of dimension $n-1$ of the proof of Theorem \ref{thm:QDp_for_SUn}. Note that the kernel $Z$ of the projection $\SU_n(q)\to \PSU_n(q)$ is a $p'$-subgroup contained in $N_{\SU_n(q)}(E)$ so that hypothesis (1) in Lemma \ref{lem:homomorphic_image_of_sphere} holds. Moreover, by Remark \ref{rmk:oneepsilon_with_centre}, hypothesis (2) also holds and then, by that result, $([E],\{[e_i]\}_{i=1}^r,[X],\{[x_i]\}_{i=1}^r,[h])$ is a Q-sphere of dimension $n-1$ for $\PSU_n(q)$ and we are done by Theorem \ref{thm:sphere_implies_QDp}. Now assume that $p\mid n$. Then, by \cite[I.10-6(1)]{GL}, the rank of $\PSU_n(q)$ is as follows,
\[
\rk_p(\PSU_n(q))=\begin{cases}
n-1\text{, if $(n)_p<(q+1)_p$,}\\
n-2\text{, if $(n)_p\geq(q+1)_p$ and $n>3$.}\\
\end{cases}
\]

Suppose first that $(n)_p\geq(q+1)_p$ and hence that $\rk_p(\PSU_n(q))=n-2$. Then we embed $\SU_{n-1}(q)$ in $\PSU_n(q)$ as the following subgroup,
\[
\left(\begin{array}{c|c}\SU_{n-1}(q)&
\begin{matrix}0\\\vdots\\0\end{matrix}\\\hline
0\ \dots\ 0&1\end{array}\right).
\]
Then $p\nmid (n-1)$ and, if $S=(E,\{e_i\}_{i=1}^{n-2},X,\{x_i\}_{i=1}^{n-2},h)$ is the Q-sphere of dimension $n-2$ of $\SU_{n-1}(q)$ given by the proof of Theorem \ref{thm:QDp_for_SUn}, then $S$ is also a Q-sphere for $\PSU_n(q)$ by Lemma \ref{lem:homomorphic_image_of_sphere} applied to the inclusion $\SU_{n-1}(q)\to \PSU_n(q)$. So we are done by Theorem \ref{thm:sphere_implies_QDp}. 

Assume now that $(n)_p<(q+1)_p$ holds and choose elements $z,u\in \FF_{q^2}^*$ such that $z$ has order $(n)_p$ and $u^p=z$. Define $e=\diag(z,\ldots,z)\in Z$ and set
\[
e_i=\diag(u^{i-n},\ldots,u^{i-n},u^i,\ldots,u^i),
\]
where $u^{i-n}$ appears $i$ times and $u^i$ appears $n-i$ times. Then $[e_i]^p=[e_i^p]=[e^i]=1$ and $E=\langle [e_1],\ldots,[e_{n-1}]\rangle$ is an elementary abelian $p$-subgroup of $\PSU_n(q)$ of rank $n-1$. Moreover,  condition (1) of Theorem \ref{lem:homomorphic_image_of_sphere} is satisfied but for $Z\cap E=1$, and condition (2) of that theorem is satisfied by Remark \ref{rmk:oneepsilon_with_centre}. The same arguments employed in that result show then that $(E,\{[e_i]\}_{i=1}^{n-1},[X],\{[x_i]\}_{i=1}^{n-1},h)$ is a Q-sphere of dimension $n-1$ for $\PSU_n(q)$, where $e_i$ is defined above, $X$ is the set of unitary permutation braids $X=S^U_n \subseteq \SU_n(q)$, the elements $x_1,\ldots,x_{n-1}$ are defined in \ref{def:definition_of_xi}, and $h\colon X\to \{\pm 1\}$, $x\mapsto (-1)^{l(x)}$, with $l(x)$ given by Definition \ref{def:length_of_unitary_permutation_braid}. Theorem \ref{thm:sphere_implies_QDp} finishes the proof.
\end{proof}

In the rest of this section we deal with $\PGU_n(q)$ and its extension by field-automorphisms.

\begin{thm}\label{thm:QDp_for_PGUn}
If $p$ is odd, $q\neq 2$, and $p\mid (q+1)$, then $H_{n-2}(|\A_p(\PGU_n(q))|;\ZZ)\neq 0$ and $\Q\D_p$ holds for $\PGU_n(q)$.
\end{thm}
\begin{proof}
As $p\mid (q+1)$, there is an element $u$ of order $p$ in $\FF_{q^2}^*$. Recall that $\rk_p(\PGU_n(q))=n-1$ and  consider the elementary abelian $p$-subgroup of $\GU_n(q)$, $E=\langle e_1,\ldots,e_{n-1}\rangle$ with
\[
e_i=\diag(u,\ldots,u,1,\ldots,1),
\]
where $u$ appears $i$ times. Then, the same arguments as in Theorem \ref{thm:QDp_for_SUn}, show that $(E,\{e_i\}_{i=1}^{n-1},X,\{x_i\}_{i=1}^{n-1},h)$ is a Q-sphere of dimension $n-1$ for $\GU_n(q)$, where $e_i$ is defined above, $X$ is the set of unitary permutation braids $X=S^U_n \subseteq \SU_n(q)\subseteq \GU_n(q)$, the elements $x_1,\ldots,x_{n-1}$ are defined in \ref{def:definition_of_xi}, and $h\colon X\to \{\pm 1\}$, $x\mapsto (-1)^{l(x)}$, with $l(x)$ given by Definition \ref{def:length_of_unitary_permutation_braid}. Note that the kernel of the surjection $\GU_n(q)\to \PGU_n(q)$ is contained in $N_{\GU_n(q)}(E)$ and trivially intersects the subgroup $E$. Using also Remark \ref{rmk:oneepsilon_with_centre}, we may apply Lemma \ref{lem:homomorphic_image_of_sphere} and obtain that $([E],\{[e_i]\}_{i=1}^{n-1},[X],\{[x_i]\}_{i=1}^{n-1}r,[h])$ is a Q-sphere of dimension $n-1$ for $\PGU_n(q)$. Hence we are done by Theorem \ref{thm:sphere_implies_QDp}.
\end{proof}

The proof for the case with field automorphisms is appropriately described in terms of quasi-reflections. Our construction is a ``degenerated'' Q-sphere in the sense that (a) in Definition \eqref{def:Xhsphere} will no longer hold. In particular, there will be non-trivial identifications among simplices, see Examples \ref{example:n2_quasireflections} and \ref{example:n3_quasireflections} for more details. There exists a minor technical difference between the methods for $n=2$ and $n\geq 3$.

\begin{remark}[Quasi-reflections]\label{rmk:quasi-reflections}
An anisotropic vector $v$ and an element $\mu\in \FF^*_{q^2}$ with $\mu\overline{\mu}=1$, determine an element of $\GU_n(q)$, called \emph{quasi-reflection}, as follows,
\[
Y_{v,\mu}(u)=u+(\mu-1)\frac{f(u,v)}{f(v,v)}v.
\]
A simple computation shows the following properties, all of which are well-known, 
\begin{subequations}
\begin{align}
Y_{\gamma v,\mu}&=Y_{v,\mu}\text{ for all $\gamma\in \FF^*_{q^2}$,}\label{equ:quasi-reflectionsa}\\
Y_{v,\mu}Y_{v,\mu'}&=Y_{v,\mu\mu'},\label{equ:quasi-reflectionsb}\\
Y_{v,\mu}Y_{v',\mu'}&=Y_{v',\mu'}Y_{v,\mu}\Leftrightarrow f(v,v')=0\text{ or $v'=\gamma v$ with $\gamma\in \FF_{q^2}$,}\label{equ:quasi-reflectionsc}\\
\ls{y}Y_{v,\mu}&=Y_{y(v),\mu},\text{ for any element $y\in \GU_n(q)$,}\label{equ:quasi-reflectionsd}\\
Y_{v,\mu}(v)&=\mu v\text{, }Y_{v,\mu}(u)=u\text{ if $f(u,v)=0$.}\label{equ:quasi-reflectionse}
\end{align}
\end{subequations}
\end{remark}

\begin{thm}\label{thm:QDp_for_PGUnfield}
If $p$ is odd, $p\mid (q+1)$, $n\geq 2$, $(p,q)\neq (3,8)$, and $G$ is a $p$-extension of $\PGU_n(q)$ by field automorphisms, then $H_{n-1}(|\A_p(G)|;\ZZ)\neq 0$ and $\Q\D_p$ holds for $G$.
\end{thm}
\begin{proof}
If $q=s^e$ for a prime $s$, then the field automorphisms of $\PGU_n(q)$ have order $2e$. As $p$ is odd, if there exists a field automorphism of order $p$, we must have $e=pl$ for some natural number $l$, and thus $q=s^{pl}$. Denote by $\phi$ the field automorphism of $\PGU_n(q)$ that acts on each entry of a matrix as $\FF_s\to \FF_s$, $\lambda\mapsto \lambda^s$. Then $\phi$ has order $2pl$, $\Phi=\phi^{2l}$ has order $p$, and $\FF_{s^{2l}}$ is the fixed field of $\Phi$. We write $G=\PGU_n(q)\langle \Phi\rangle$ and reserve the notation $\overline{x}=x^q$ for conjugation in $\GU_n(q)$. As preparation, we need the following two arithmetic digressions, where for $m\in \ZZ$,  $p^{\ord_p(m)}$ is the largest power of $p$ dividing $m$ and, for $m,m'\in \ZZ$, $(m,m')$ denotes its greatest common divisor. We write $Z=Z(\GU_n(q))$.


\textbf{Choice of $\alpha$.} We show that there exist $\alpha\in \FF_{s^{2l}}$ such that 
\begin{equation}\label{equ:alpha}
\alpha+\overline{\alpha}+(n-2)=0\text{, }\alpha\overline{\alpha}+(n-1)\neq 0\text{, and }\alpha\neq 1.
\end{equation}
In fact, the leftmost equation has $q$ different solutions over $\FF_{q^2}$ and, if $\tilde{\alpha}$ is such a solution, then
\[
\alpha=\frac{\tilde{\alpha}+\Phi(\tilde{\alpha})+\ldots +\Phi^{p-1}(\tilde{\alpha})}{p}
\]
is a solution that belongs to $\FF_{s^{2l}}$. If $\alpha\overline{\alpha}+(n-1)=0$ we get to
\[
\alpha\overline{\alpha}-\alpha-\overline{\alpha}+1=(\alpha-1)(\overline{\alpha}-1)=0
\]
and to $\alpha=1$. Hence, it is enough to find $\alpha$ satisfying the leftmost and rightmost conditions in \eqref{equ:alpha}. Note that, for $0\neq \beta\in \FF_{s^{2l}}$ with trace $0$, i.e., $\beta+\overline{\beta}=0$, $\alpha+\beta$ is also a solution of the leftmost equation in \eqref{equ:alpha} and, if $\alpha=1$, then $\alpha+\beta\neq 1$. Thus, is is enough to find such a value $\beta$. We first recall that the equation $\gamma^{q+1}=1$  has at least $p$ different solutions over $\FF_{s^{2l}}$ because, by Fermat's little theorem, $p$ divides 
\[
(q+1,s^{2l}-1)=(s^{pl}+1,(s^l-1)(s^l+1)).
\]
For such a solution $\gamma\neq -1$, we may define $\beta=\frac{\gamma-1}{\gamma+1}\in \FF_{s^{2l}}$ and compute
\[
\beta+\overline{\beta}=\frac{\gamma-1}{\gamma+1}+\overline{\frac{\gamma-1}{\gamma+1}}=\frac{\gamma-1}{\gamma+1}+\frac{\gamma^{-1}-1}{\gamma^{-1}+1}=\frac{\gamma-1}{\gamma+1}+\frac{1-\gamma}{1+\gamma}=0.
\]
Thus, to find $\beta$ with the required properties is enough to find $\gamma\neq 1,-1$ and, as $p\geq 3$, this is indeed the case.

\textbf{Choice of $\lambda$ and $\Lambda$.} As $p$ is odd, we must have $\ord_p(s^{2pl}-1)=\ord_p(s^{pl}+1)$ and $\ord_p(s^{2l}-1)=\ord_p(s^l+1)$. By the results of Artin in \cite{ARTIN}, we have $\ord_p(s^{2lp}-1) = \ord_p(s^{2l}-1)+1$, and thus
\[
\ord_p(s^{pl}+1)=\ord_p(s^{2l}-1)+1.
\]
Next, consider the equalities,
\[
(s^{2pl}-1)=(s^{pl}-1)(s^{pl}+1)=(s^l-1)\gamma(s^{pl}+1)=(s^{2l}-1)\gamma'=(s^l-1)(s^l+1)\gamma',
\]
where $\gamma=s^{l(p-1)}+s^{l(p-2)}+\ldots+s^l+1$ and $\gamma'=s^{2l(p-1)}+s^{2l(p-2)}+\ldots+s^{2l}+1$, and deduce that
\[
(s^{pl}+1)\gamma=(s^l+1)\gamma'.
\]
By pairing the $(p-1)$ elements different from $1$ in these sums, it is easy to see that $\gamma$ and $\gamma'$ are odd numbers. It follows that
\[
\ord_2(s^{pl}+1)=\ord_2(s^l+1).
\]
So we have $\ord_p(s^{pl}+1)>\ord_p(s^{2l}-1)$, $\ord_2(s^{pl}+1)\leq \ord_2(s^{2l}-1)$, and we show that $p$ cannot be the only prime $r\neq 2$ satisfying $\ord_r(s^{pl}+1)>\ord_r(s^{2l}-1)$. For in that case, we have
\[
\frac{s^{pl}+1}{s^{2l}-1}\leq p,
\]
and close examination of this inequality shows that this cannot be the case unless $(p,q)=(3,8)$, a case excluded by assumption. Thus, there exists a prime $r\neq 2,p$ with $\ord_r(s^{pl}+1)>\ord_r(s^{2l}-1)$, and, for $\lambda\in \FF^*_{q^2}$ with $\ord(\lambda)= r^{\ord_r(s^{2l}-1)+1}$, 
\begin{equation}\label{equ:twistedsphere_lambda}
{\lambda}^{q+1}=1\text{ and }\Lambda=\lambda^{1-s^{2l}}\text{ has order }\ord(\Lambda)=r\neq 2,p.
\end{equation}


\textbf{Choice of vectors $v_i$'s and quasi-reflections $Y_{v_i,u}$'s.} For $1\leq i\leq n$, consider the vector
\[
v_i=(1, \ldots ,1, \alpha, 1, \ldots, 1)^\transpose,
\]
where $\alpha$ is at the $i$-th position, and note that
\[
f(v_i,v_j)=\begin{cases}
\alpha+\overline{\alpha}+(n-2),&\text{for $i\neq j$,}\\
\alpha\overline{\alpha}+(n-1),&\text{for $i=j$.}
\end{cases}
\]
Thus, by Equation \eqref{equ:alpha}, $v_i$ is anisotropic and orthogonal to $v_j$ for all $j\neq i$. As $p|(s^l+1)$, there is an element $u$ of order $p$ in $\FF_{s^{2l}}$. Then, by \eqref{equ:quasi-reflectionsb} and \eqref{equ:quasi-reflectionsc}, the following quasi-reflections have order $p$ and commute pairwise,
\[
Y_{v_i,u}\text{ for $1\leq i\leq n$.}
\]
In addition, these quasi-reflections have the following additional property,
\begin{equation}\label{equ:quasi_reflections_uniqueness_of_product}
Y_{v_1,u^{i_1}}\cdots Y_{v_n,u^{i_n}}=Y_{v_1,u^{j_1}}\cdots Y_{v_n,u^{j_n}}\Rightarrow i_1=j_1,\ldots,i_n=j_n\mod p.
\end{equation}
This is easily obtained as, by \eqref{equ:quasi-reflectionse}, evaluation at $v_l$ gives
\[
u^{i_l}v_l=u^{j_l}v_l\Rightarrow u^{i_l}=u^{j_l}\Rightarrow i_l=j_l\mod p.
\]
\textbf{Choice of subgroup $E$.} 
For $1\leq i\leq n-1$, consider the following commutative product,
\[
e_i=Y_{v_{i+1},u}Y_{v_{i+2},u}\cdots Y_{v_{n-1},u}Y_{v_n,u}.
\]
and note that $Y_{v_1,u}$ is not involved. The elements $e_i$'s have order $p$ and pairwise commute. Moreover, by construction, $\alpha\in \FF_{s^{2l}}$, so the following is an elementary abelian $p$-subgroup of $\GU_n(q)\langle \Phi\rangle$,
\[
E=\langle e_1,\ldots,e_{n-1},e_n\rangle,
\]
where $e_n=\Phi$. By \eqref{equ:quasi_reflections_uniqueness_of_product}, this subgroup has rank $n$, and we show that the projection $E\mapsto [E]$ in the quotient $G$ is injective, i.e., that $E\cap Z=1$. In fact, assume that $e\Phi^i=z$ with $z=\diag(\mu,\ldots,\mu)\in Z$ and $e\in \langle e_1,\ldots,e_{n-1}\rangle$ given by 
\[
e=Y_{v_2,u^{i_2}}Y_{v_3,u^{i_3}}\cdots Y_{v_{n-1},u^{i_{n-1}}}Y_{v_n,u^{i_n}}.
\]
Then it's immediate that $i=0$, $e=z$, and,  by \eqref{equ:quasi-reflectionse}, evalution at $v_1$ gives
\[
e(v_1)=v_1=z(v_1)=\mu v_1\Rightarrow \mu=1\Rightarrow e=1.
\]

\textbf{Choice of subgroup $S_n$.} For $i=1,\ldots,n-1$, let $x_i=\dt{s_i}$ be the usual permutation matrix of $\GU_n(q)$ that lifts the corresponding generator of the symmetric group, so that $\langle x_1,\ldots,x_{n-1}\rangle\cong S_n$. It is immediate that, for $1\leq i\leq n-1$, $1\leq j\leq n$, we have 
\[
x_i(v_j)=\begin{cases}
v_{i+1}&\text{for $j=i$,}\\
v_{i}&\text{for $j=i+1$,}\\
v_j&\text{otherwise,}\\
\end{cases}
\]
so that, by Equation \eqref{equ:quasi-reflectionsd}, for any $\mu\in \FF_{s^{2l}}$ with $\mu\overline{\mu}=1$, we have
\begin{equation}\label{equ:x_i_conjugates_qrs}
\ls{x_i}Y_{v_j,\mu}=\begin{cases}
Y_{v_{i+1},\mu}&\text{for $j=i$,}\\
Y_{v_i,\mu}&\text{for $j=i+1$,}\\
Y_{v_j,\mu}&\text{otherwise.}\\
\end{cases}
\end{equation}
In particular, $[x_i,e_j]=1$ if $1\leq i\leq n-1$, $1\leq j\leq n$, $i\neq j$, and hence
\begin{equation}\label{equ:xi_centralizes_Ei}
x_i\in C_G(E_i)\text{ for $i=1,\ldots,n-1$.}
\end{equation} 

\textbf{Choice of subset $X$.} If $n\geq 3$, consider the following set,
\[
X=\{xY^\delta_{v_j,\lambda}\mid x\in \langle x_1,\ldots,x_{n-1}\rangle\text{, }1\leq j\leq n\text{, }\delta=-1,1\},
\]
and, if $n=2$, the following one,
\[
X=\{xY^\delta_{v_j,\lambda}\mid x\in \langle x_1\rangle\text{, }1\leq j\leq 2\text{, }\delta=1\},
\]
and note the difference between the allowable values for $\delta$. We show first that the projection $X\mapsto [X]$ in the quotient $G$ is injective. For assume that $xY^\delta_{v_j,\lambda}=zx'Y^{\delta'}_{v_{j'},\lambda}$ with $z\in Z$. Then, $z^{-1}x'^{-1}x=Y^{\delta'}_{v_{j'},\lambda}Y^{-\delta}_{v_j,\lambda}$ and, as $[x',\Phi]=[x,\Phi]=1$, if we let this element act on $\Phi$, we get to
\[
\diag(\mu,\ldots,\mu)=Y^{\delta'}_{v_{j'},\Lambda}Y^{-\delta}_{v_j,\Lambda}
\] 
for some $\mu\in \FF^*_{q^2}$. If $j\neq j'$ and $n=2$, then evaluating at $v_j$ and $v_{j'}$ we obtain 
\[
\mu v_j=\Lambda^{-\delta} v_j,\mu v_{j'}=\Lambda^{\delta'}v_{j'}\Rightarrow \mu=\Lambda^{-\delta}=\Lambda^{\delta'}\Rightarrow \Lambda^{\delta+\delta'}=1,
\]
and this contradicts \eqref{equ:twistedsphere_lambda} as $\delta+\delta'=2$. If $j\neq j'$ and $n\geq 3$, evaluating at $v_j,v_{j'},v_k$ with $k\neq j,j'$, we obtain $1=\mu=\Lambda^{\delta'}=\Lambda^{-\delta}$, and this contradicts \eqref{equ:twistedsphere_lambda} as $\delta,\delta'\neq 0$. If $j=j'$, we evaluate at $v_j,v_k$ with $k\neq j$ and obtain $1=\mu=\Lambda^{\delta'-\delta}$ and hence $\delta=\delta'$. Thus, we have that $x=zx'$ and then it is clear that $z=1$ and $x=x'$ and we are done. 

Finally, by \eqref{equ:quasi-reflectionsc}, $[Y_{v_i,\lambda},e_j]=1$ for $1\leq i\leq n$, $1\leq j\leq n-1$, and hence
\begin{equation}\label{equ:Y_vjmu_centralizes_e1...en-1}
Y_{v_i,\lambda}\in C_G(E_n)\text{ for $i=1,\ldots,n$.}
\end{equation}

\textbf{Choice of simplicial chain.} For $n\geq 3$, consider the map $h\colon X\to \ZZ$ given by 
\[
h(xY^\delta_{v_j,\lambda})=(-1)^{l(x)+\frac{1+\delta}{2}},
\]
where $l(x)$ is the length of $x\in S_n$, and, for $n=2$, the map $h\colon X\to \ZZ$ given by
\[
h(xY^\delta_{v_j,\lambda})=(-1)^j.
\]
We also consider the simplicial chain given by Definition \ref{def:ZGXA},
\[
\barsub_{[E],[X],[h]}.
\]
In order to apply Theorem \ref{thm:noncontractiblemoregeneral} to this chain and conclude the proof, we show below that both hypotheses of this result hold.

\textbf{Hypothesis $(a)$ in Theorem \ref{thm:noncontractiblemoregeneral} holds.} It is enough to show that, for $x\in \langle x_1,\ldots,x_{n-1}\rangle$, $1\leq j,j'\leq n$, $\delta,\delta'$, and the two $(n-1)$-simplices
\[
\sigma=\langle [e_{n-1}]\rangle<\langle [e_{n-1}],[e_{n-2}]\rangle<\ldots<\langle [e_{n-1}],\ldots,[e_1]\rangle<[E]
\]
and
\[
\sigma'=\langle [e_{j_1}]\rangle<\langle [e_{j_1}],[e_{j_2}]\rangle<\ldots<\langle [e_{j_1}],\ldots,[e_{j_{n-1}}]\rangle<[E],
\]
the condition 
\[
\ls{[xY^{\delta'}_{v_{j'},\lambda}Y^{-\delta}_{v_j,\lambda}]}\sigma=\sigma'
\]
implies that $x=1$, $j_l=n-l$ for $1\leq l\leq n-1$, and that $\delta\neq 0$ or $\delta'\neq 0$ implies that $j=j'$ and $\delta=\delta'$. Write $x=w_1w_2\ldots w_{n-1}$ as in Proposition \ref{prop:normal_form_Sn}, and define, in view of the forthcoming discussion, $w_n=1$. Then we prove inductively on $i=n,\ldots,1$ that 
\begin{equation}\label{equ:induction_for_hypotehsis_a}
w_i=1\text{ and }j_l=n-l\text{ for $l=1,\ldots,n-i$.}
\end{equation}
The base case $i=n$ is vacuous and thus we assume that \eqref{equ:induction_for_hypotehsis_a} holds for $n,n-1,\ldots,i+1$. By Equation \eqref{equ:Y_vjmu_centralizes_e1...en-1}, we deduce that 
\[
\ls{[x]}\langle [e_{n-1}],\ldots,[e_{i+1}],[e_{i}]\rangle=\langle [e_{j_1}],\ldots,[e_{j_{n-i}}]\rangle,
\]
and employing the induction hypothesis we obtain that
\[
\ls{[x]}[e_i]\in \langle [e_{n-1}],\ldots,[e_{i+1}],[e_{j_{n-i}}]\rangle
\]
and that $j_{n-i}\leq i$. Thus, for some $z\in Z$ and integers $k_{n-1},\ldots,k_{i+1},k_i$, we have
\[
\prod^n_{k=i+1} Y_{x(v_k),u}=z\prod_{l=n-1}^{i+1} \prod_{k=l+1}^n Y_{v_k,u^{k_l}} \prod_{k=j_{n-i}+1}^n Y_{v_k,u^{k_i}}.
\]
Evaluating at $v_1$ we obtain that $z=1$ and note that, as $w_{n-1}=\ldots=w_{i+1}=1$, we have $x(v_k)=v_k$ for all $i+2\leq k\leq n$ and $x(v_{i+1})=v_{k'}$ with $k'\leq i+1$. Then, from \eqref{equ:quasi_reflections_uniqueness_of_product}, we obtain the contradiction $k_i=0$ unless $j_{n-i}=i$, $k_i=1$, and $k'=i+1$. We deduce that $w_i=1$, $k_{n-1}=\ldots=k_{i+1}=0$, $\ls{x}e_i=e_i$, and \eqref{equ:induction_for_hypotehsis_a} holds for $i$.

Thus, we are left with coping with $\delta$, $\delta'$, $j$ and $j'$. By hypothesis we have 
\[
\ls{[Y^{\delta'}_{v_{j'},\lambda}Y^{-\delta}_{v_j,\lambda}]}\Phi\in [E],
\]
and hence
\[
Y^{\delta'}_{v_{j'},\Lambda}Y^{-\delta}_{v_j,\Lambda}=ze
\]
for some $z=\diag(\mu,\ldots,\mu)\in Z$  and $e\in \langle e_1,\ldots,e_{n-1}\rangle$ given by 
\[
e=Y_{v_2,u^{i_2}}Y_{v_3,u^{i_3}}\cdots Y_{v_{n-1},u^{i_{n-1}}}Y_{v_n,u^{i_n}}
\]
for integers $i_2,\ldots,i_n$. For the upcoming discussion, define $i_1=0$. If $j\neq j'$ and $n=2$,  evaluating at $v_j$ and $v_{j'}$ we obtain 
\[
\mu u^{i_j} v_j=\Lambda^{-\delta} v_j,\mu u^{i_{j'}} v_{j'}=\Lambda^{\delta'}v_{j'}\Rightarrow \Lambda^{\delta+\delta'}=u^{i_{j'}-i_j}\Rightarrow \Lambda^{p(\delta+\delta')}=1,
\]
and this contradicts \eqref{equ:twistedsphere_lambda} because $\delta+\delta'=2$. If $j\neq j'$ and $n\geq 3$, we evaluate at $v_j,v_{j'},v_k$ with $k\neq j,j'$ and obtain $\Lambda^{p\delta}=\Lambda^{p\delta'}=1$, $\delta=\delta'=0$, contradiction.  If $j=j'$, we evaluate at $v_j$ and at $v_k$ with $k\neq j$ and obtain $\Lambda^{p(\delta'-\delta)}=1$ and $\delta=\delta'$.

\textbf{Hypothesis $(b)$ in Theorem \ref{thm:noncontractiblemoregeneral} holds.} For tuples $\ii,\jj\in \tuples^n$ and $y\in [X]$, define, as in Theorem \ref{thm:sphere_implies_QDp},
\[
\D(y,\ii,\jj)=\{z\in [X]~|~\ls{z}\tau_\jj=\ls{y}\tau_\ii\}\text{ and }D(y,\ii,\jj)=\sum_{z\in \D(y,\ii,\jj)} h(z).
\]
Then we want to prove that $D(y,\ii,\jj)=0$. Write $\jj=[j_1,\ldots,j_{r-1}]\in \tuples_r$ and assume first that $1\leq j_1\leq n-1$. Note that, by \eqref{equ:xi_centralizes_Ei}, $(b)$ in Definition \ref{def:Xhsphere} is satisfied for such $j_1$. In addition, if $[xY^\delta_{v_j}]\in \D(y,\ii,\jj)$, from Equation \eqref{equ:x_i_conjugates_qrs}, we have 
\begin{equation}\label{equ:horizontal_gluing}
[xY^\delta_{v_j,\lambda}][x_{j_1}]=[xx_{j_1}Y^\delta_{x_{j_1}(v_j),\lambda}]\in \D(y,\ii,\jj)
\end{equation}
and, checking the definition of $h$ for the cases $n\geq 3$ and $n=2$, we also have
\[
h([xY^\delta_{v_j,\lambda}])+h([xx_{j_1}Y^\delta_{x_{j_1}(v_j),\lambda}])=0.
\]
Thus, $(c2)$ in Remark \ref{rmk:sphere_even_characteristic} holds for such $j_1$ and, by the arguments in that remark, $D(y,\ii,\jj)=0$ in this case. To deal with the case $j_1=n$, consider the following map $\Psi\colon \D(y,\ii,\jj)\to \D(y,\ii,\jj)$,
\begin{equation}\label{equ:vertical_gluing}
z=[xY^\delta_{v_j,\lambda}]\mapsto \Psi(z)=\begin{cases}
[xY^{-\delta}_{v_j,\lambda}]&\text{ for $n\geq 3$,}\\
[xY^{\delta}_{v_{3-j},\lambda}]&\text{ for $n=2$.}
\end{cases}
\end{equation}
This map is well defined by \eqref{equ:Y_vjmu_centralizes_e1...en-1}, is a non-trivial involution, and we  have $h(z)+h(\Psi(z))=0$. Thus, we obtain $D(y,\ii,\jj)=0$ and we are done.
\end{proof}

\begin{remark}
A consequence of the preceding proof is that the simplicial chain 
\[
\barsub_{\langle [e_1],\ldots,[e_{n-1}]\rangle,\langle [x_1],\ldots,[x_{n-1}]\rangle,[h]}
\]
i a non-trivial cycle in $\widetilde H_{n-2}(|\A_p(\PGU_n(q))|;\ZZ)$, thus giving an alternative proof to Theorem \ref{thm:QDp_for_PGUn} if $q=s^{pl}$.
\end{remark}

\begin{remark}\label{rmk:some_nonidentified_simplices}
We have shown that simplex $\sigma_\ii$ for $\ii=[n,1,2,\ldots,n-2]$ (see Definition \ref{def:ZEa}) is not identified to any other simplex. In fact, the same arguments prove that this is the case for any $X$-conjugate of $\sigma_\ii$ for the tuples $\ii$ of the form $[1,2,\ldots,i,n,i+1,\ldots,n-2]$ for $0\leq i\leq n-2$ and for the tuple $[1,2,\ldots,n-1]$.
\end{remark}

\begin{figure}[h!]
\centering
\incfig{triangulationPSUPhin2}
\caption{Complex for $\PSU_2(q)\langle \Phi\rangle$.}
\label{fig:triangulationPSUPhin2}
\end{figure}

\begin{exa}\label{example:n2_quasireflections}
For the case of $\PGU_2(q)$ with field automorphisms, we have, for any $\mu\in\FF_{s^{2l}}$ with $\mu\overline{\mu}=1$, that
\[
v_1=(\alpha,1)^\transpose\text{, }Y_{v_1,\mu}=\frac{1}{1+\alpha\overline{\alpha}}\begin{pmatrix} \alpha\overline{\alpha}+\mu & (\mu-1)\overline{\alpha}\\
(\mu-1)\alpha& 1+\mu\alpha\overline{\alpha}
\end{pmatrix}\text{, }x_1=\begin{pmatrix} 0&1\\1&0\end{pmatrix},
\]
and $v_2$ and $Y_{v_2,\mu}$ are obtained as $x_1(v_1)$ or $\ls{x_1}Y_{v_1,\mu}$ respectively. In addition, we have $e_1=Y_{v_2,u}$, the set $X$ is described below, where we write $Y_i=Y_{v_i,\lambda}$,
\[
X=\{Y_1,Y_2,x_1Y_1,x_1Y_2\},
\]
and the map $h$ is given by $h(Y_1)=h(x_1Y_1)=-1$ and $h(Y_2)=h(x_1Y_2)=+1$. By Remark \ref{rmk:some_nonidentified_simplices}, there are no identifications among simplices and Figure \ref{fig:triangulationPSUPhin2} depicts the constructed chain, where we have labelled all conjugating elements and some vertices. 
\end{exa}

\begin{figure}[h!]
\centering
\incfig{triangulationPSUPhin3bisbis}
\caption{The complex for $\PSU_3(q)\langle \Phi\rangle$ is obtained identifying the three equators, the edges $a,\ldots,l$ in pairs via \eqref{equ:horizontal_gluing}, and a similar gluing pattern in the lower part. For example, edges d's are glued via $Y_1\mapsto Y_1x_1=x_1Y_2$. The vertical symmetry is due to \eqref{equ:vertical_gluing}.}
\label{fig:triangulationPSUPhin3}
\end{figure}


\begin{exa}\label{example:n3_quasireflections}
For $\PGU_3(q)$ with field automorphisms and $\mu\in\FF_{s^{2l}}$ with $\mu\overline{\mu}=1$ we have
\[
v_1=(\alpha,1,1)^\transpose\text{, }Y_{v_1,\mu}=\frac{1}{2+\alpha\overline{\alpha}}\begin{pmatrix}
2+\mu\alpha\overline{\alpha}&(\mu-1)\alpha&(\mu-1)\alpha\\
(\mu-1)\overline{\alpha}&\alpha\overline{\alpha}+\mu+1&\mu-1\\
(\mu-1)\overline{\alpha}&\mu-1&\alpha\overline{\alpha}+\mu+1
\end{pmatrix}
\]
and $v_2,v_{3},Y_{v_2,\mu},Y_{v_3,\mu}$ are obtained by multiplying or conjugating by the appropriate product of the elements
\[
x_1=\begin{pmatrix} 0&1&0\\1&0&0\\0&0&1\end{pmatrix}\text{ and }x_2=\begin{pmatrix}1&0&0\\0&0&1\\0&1&0\end{pmatrix}.
\]
We also have $e_1=Y_{v_2,u}Y_{v_3,u}$ and $e_2=Y_{v_3,u}$ and the set $X$ consists of the following 36 elements, where we write $Y_i=Y_{v_i,\lambda}$ for $i=1,2,3$,
\[
\{Y^\delta_i,x_1Y_i^\delta,x_2Y_i^\delta,x_1x_2Y_i^\delta,x_2x_1Y_i^\delta,x_1x_2x_1Y_i^\delta\text{, for $i=1,2,3$, $\delta=-1,1$}\}.
\]
The resulting complex is shown in Figure \ref{fig:triangulationPSUPhin3}, which includes labels for some conjugating elements and some vertices. For the equators, compare to Figure \ref{fig:triangulationPSLPSUr=2} (right). In the barycentric subdivisions considered, see Remark \ref{rmk:barycentric_subdivision}, the white $2$-simplices are the $Y^{-1}_1$-conjugates of the following ones, and they are not identified to any other simplices by Remark \ref{rmk:some_nonidentified_simplices},
\[
\langle [e_2]\rangle < \langle [e_2],[e_1]\rangle< [E]\text{, }\langle [e_2]\rangle< \langle [e_2],[\Phi]\rangle < [E]\text{, and }\langle [\Phi]\rangle< \langle [\Phi],[e_2]\rangle <[E].
\]
The black $2$-simplexes are equal as $x_2$ centralizes the $2$-simplex
\[
\ls{Y_1^{-1}}{\big(\langle [e_1]\rangle < \langle [e_1],[e_2]\rangle< E\big)}.
\]
\end{exa}

%%%%%%%%%%%%%%%%%%%%%%%%%%%%%%%%%%%%%%%%%%%%%%%%%



\begin{thebibliography}{99}

\bibitem{ARTIN} E.Artin, {\it The orders of the linear groups}, Comm. Pure Appl. Math., 8 (1955), 355--365.
 
\bibitem{AK1990} M.~Aschbacher, P.B.~Kleidman, {\it On a conjecture of Quillen and a lemma of Robinson}, Arch. Math. (Basel), 55 (1990), no.3, 209--217.  

\bibitem{AS1993} M.~Aschbacher, S.~Smith, {\it On Quillen's conjecture for the p-groups complex}, Ann. of Math. (2) 137 (1993), no. 3, 473--529.  
  
\bibitem{BjornerBrenti} A. Bjorner, F. Brenti, {\em Combinatorics of Coxeter Groups}, Graduate Text in Mathematics, volume 231, Springer, 2005.

\bibitem{Coxeter} H.S.M.~Coxeter, {\em Regular complex polytopes}, Second edition. Cambridge University Press, Cambridge, 1991.

\bibitem{Diaz2016} A. D\'iaz Ramos, {\it On Quillen's conjecture for p-solvable groups}, Journal of Algebra, 513, 1, 2018, 246--264.

\bibitem{DiazMazza2020} A. D\'iaz Ramos, N. Mazza, {\it A geometric approach to Quillen's conjecture}, J. Group Theory 25 (2022), 91--112.

\bibitem{GL} D. Gorenstein, R. Lyons, {\em The local structure of
  finite groups of characteristic 2 type}. Mem. Amer. Math. Soc. 42
  (1983), no. 276. 
  
\bibitem{GLSIII} D.~Gorenstein, R.~Lyons, R.~Solomon, {\em The classification of the finite simple groups}. Number 3. Mathematical Surveys and Monographs, Volume 40, Number 3, AMS, 1998.

\bibitem{Grove} L.C.Grove, {\em Classical Groups and Geometric Algebra}, Graduate Studies in Mathematics, Volume 39, AMS, 2002.

\bibitem{KasselTuraev} C.~Kassel, V.~Turaev, {\em Braid Groups}, Graduate Text in Mathematics, volume 247, Springer, 2008.

\bibitem{PS2022} K.I. Piterman, S.D. Smith, {\em Some results on Quillen's Conjecture via equivalent-poset techniques}, arXiv:2204.13055 

\bibitem{Quillen1978} D.~Quillen, {\it Homotopy properties of the poset of nontrivial p-subgroups of a group},  Adv. Math. 28 (1978), no. 2,   101--128. 

\bibitem{Steinberg} R.~Steinberg, {\em Lectures on Chevalley groups}, Yale University 1967.

\bibitem{Taylor} D.E.~Taylor, {\em The geometry of the classical groups}, Heldermann Verlag, 1998.

\bibitem{ReinerRipollStump} V.~Reiner, V.~Ripoll, C.~Stump, {\em On non-conjugate Coxeter elements in well-generated reflection groups}, Math. Z. 285 (2017), no. 3-4, 1041–1062.

\end{thebibliography}
\end{document}

%%%%%%%%%%%%%%%%%%%%%%%%%%%%%%%%%%%%%5
%
%
%
%
%    END DOCUMENT !!!!!!!!!!!!!!
%    END DOCUMENT !!!!!!!!!!!!!!
%    END DOCUMENT !!!!!!!!!!!!!!
%    END DOCUMENT !!!!!!!!!!!!!!
%
%
%
%
%%%%%%%%%%%%%%%%%%%%%%%%%%%%%%%%%%%%%

