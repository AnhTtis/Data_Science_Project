\documentclass[aps,prd,twocolumn,a4paper,final,superscriptaddress,longbibliography]{revtex4-1}

%\usepackage{package}
%\usepackage{gentombow}

%\usepackage[dvipdfmx]{graphicx}
%\usepackage[dvipdfmx]{hyperref}

%arXiV%
\usepackage{graphicx}
%\usepackage{hyperref}

%\usepackage[utf8]{inputenc}
\usepackage [latin1]{inputenc}

\usepackage{amsmath,amssymb}
\usepackage{bm} 
\usepackage{mathtools}
\usepackage[caption=false]{subfig}
\usepackage{booktabs}
\usepackage{mathrsfs}

\usepackage{cases}

\usepackage{color}
\usepackage{ulem}
\usepackage{comment}
\usepackage{cleveref}

\usepackage{natbib}


\graphicspath{{./Fig/}}


\def\Vec#1{\mbox{\boldmath $#1$}}

\def\tr{\mathrm{tr}}

\def\ie{\textit{i.e., }}
\def\etc{\textit{etc. }}


\begin{document}

\title{A baby-Skyrme model with the anisotropic DM interaction: the compact skyrmions revisited}

\author{Funa Hanada}
\email{funa87@gmail.com}

\author{Nobuyuki Sawado}
\email{sawadoph@rs.tus.ac.jp}
\affiliation{Department of Physics, Tokyo University of Science, Noda, Chiba 278-8510, Japan}


\begin{abstract}
We consider a baby-Skyrme model with a Dzyaloshinsikii-Moriya interaction (DMI) and two types of potential terms. 
The model has a close connection with the vacuum functional of fermions coupled with $O(3)$ nonlinear 
$\bm{n}$-fields and with a constant $SU(2)$ gauge background. 
It can be derived from the heat-kernel expansion for the fermion determinant. 
The model possesses normal skyrmions with topological charge $Q=1$. 
The restricted version of the model also possesses both weak-compacton 
(at the boundary, not continuously differentiable) and genuine-compacton (continuously differentiable).  
We also show that the model consists of only the Skyrme term and 
the DMI provides the soliton solutions, which are so-called \textit{skyrmions without a potential}.

\end{abstract}



\maketitle

\section{Introduction}

The Skyrme model, a (3+1)-dimensional nonlinear field theory of pions, is a model of hadrons and is 
supposed to be the  most promising and long-lived effective model in the low-energy domain of 
quantum-chromodynamics (QCD). The skyrmions, the topological solitons in the Skyrme model, well describe 
not only the standard hadrons and nuclei, but also structures of the dense nuclear matter
~\cite{Adam:2010fg,Adam:2013tda,Adam:2015lra,Ferreira:2021ryf} and the neutron star~\cite{Adam:2014dqa,Adam:2015lpa}. 

The Skyrme model in a (2+1)-dimensions recently has grown much attention.
Especially, magnetic skyrmions have been increasing interest in both theoretical aspects of topological matter 
and also in many applications of spintronics, quantum computing, dense magnetic nanodevices, and so on. 
Magnetic skyrmions are derived from a model comprised of the Dzyaloshinsikii-Moriya interaction (DMI)
~\cite{DZYALOSHINSKY1958241,Moriya:1960}. 
The DMI and a potential break in the scale invariance of the model successfully evade the Derrick's theorem. 
The Skyrme field $\bm{n}=(n_1,n_2,n_3)$ with $\bm{n}\cdot\bm{n}=1$, realizes maps: $S^2\to S^2$, 
and are characterized by the homotopy group $\Pi_2(S^2)=\mathbb{Z}$.
The energy density is defined as~\cite{Bogdanov:1989,BOGDANOV:1994, Barton-Singer:2018dlh,Schroers:2019hhe}
\begin{align}
&\mathcal{E}_{\textrm{DM}}=\kappa_2(\partial_i \bm{n} )^2 +\kappa_1\bm{n}\cdot \nabla\times \bm{n} +V[\bm{n}]\,, ~~i=1,2,
\end{align}
where $\kappa_2,\kappa_1$ are constants with positive sign. 
The second differential term (the kinetic term) is a scale-invariant term and the DMI has a negative contribution 
to the energy and then the solution may exist in the Derrick's theorem. 

The baby-Skyrme model is a direct mimic of the (3+1)-Skyrme model; 
the model consists of an $O(3)$ nonlinear sigma model (the kinetic term), 
a 4th differential order term (the Skyrme term) and a Zeeman or the other types of potential terms.
As it is known that the Skyrme and the potential terms are responsible for the Derrick's theorem. 
The energy density of the baby-Skyrme model is defined by~\cite{Piette:1994ug} 
\begin{align}
&\mathcal{E}_{\textrm{bS}}=\kappa_2(\partial_i \bm{n} )^2 +\kappa_4\left(\partial_i \bm{n } \times \partial_j \bm{n }\right)^2 +V[\bm{n}]\,,
~~i,j=1,2,
\label{babyskyrme}
\end{align}
where $\kappa_4$ is a positive constant. 
The baby-skyrmions have their applications in the quantum Hall effects
~\cite{Sondhi:1993,Neubauer:2009_1,Neubauer:2009_2,Balram:2015,Jiang:2017}, the nematic crystals~\cite{Bogdanov:2003,Fukuda:2011,Leonov:2014,Ackerman:2017,Matteis:2018,Matteis:2022}, 
the superconducting materials~\cite{Zyuzin:2017}, 
and the brane-world scenarios~\cite{Kodama:2008xm,Brihaye:2010nf,Delsate:2011aa,Delsate:2012hz}, so on. 
The baby-Skyrme model without the kinetic term, named 
the restricted baby-Skyrme model~\cite{Gisiger:1996vb,Adam:2010jr,Andrade:2022wiv} 
has a significant feature; it possesses analytical Bogomol'nyi-Prasad-Sommerfield (BPS) solutions. The baby-Skyrme model and the restricted model have the solutions of 
compacton. Compactons possess a distinct character among other solutions of standard field theory models. 
Namely, the field takes its vacuum values outside this support and the energy, as well as the charge, are always
concentrated on the compact support~\cite{Arodz:2005gz,Arodz:2008jk}. 
There have been several studies of compact skyrmions in the baby-Skyrme model
~\cite{Gisiger:1996vb,Adam:2009px, Speight:2010sy, Adam:2010jr,Ashcroft:2015jwa,Casana:2022bei}. 
For finding compactons, the baby-Skyrme model requires a special non-analytical potential called V-shaped potential.
While in the restricted model, there are more general choices for the potential. 
There is a prominent challenge to the modification of the model. 
The baby-Skyrme model with fractional power of the kinetic term with no potential term successfully evades the 
Derricks's theorem, and has compact and non-compact skyrmion solutions~\cite{Ashcroft:2015jwa}. 


A natural question arises: are both models able to combine to describe some kind of phenomenology? 
For the moment, we have no clear evidence for both interactions should coexist. 
From a theoretical point of view, however, 
it may be worthful to consider a combined model and find novel solutions. 
In this paper, we examine such models and find several types of solutions including compactons. 
For simplicity, we concentrate on the circular symmetric solutions 
but if one lifts the constraint, certainly a wide variety of structures will emerge. 

The paper is organized as follows. In Section \ref{sec:2} we present the fermionic model. We begin with 
a fermionic vacuum functional and obtain the Skyrme-like model with both the DMI 
and the Skyrme term in terms of the heat-kernel expansion. It justifies the existence of the model. 
Section \ref{sec:3} is a brief explanation of our model, including the energy functional and the Euler equation. 	
We present several analytical and numerical solutions to the model in Section \ref{sec:4}. 
We describe a novel combined model which has no potential term and give the solutions in Section \ref{sec:5}.
Conclusions and remarks are presented in the last Section.

%%%%%%%%%%%%%%%%%%%%%%%%%%%%%%%%%%%%%%%%%%%%%%%%%%%%%%%%%%%%%%%%%%%%%%%%%%%%%%%%%
\section{\label{sec:2}A fermionic soliton model and an extension of the baby-Skyrme model}
%%%%%%%%%%%%%%%%%%%%%%%%%%%%%%%%%%%%%%%%%%%%%%%%%%%%%%%%%%%%%%%%%%%%%%%%%%%%%%%%%

In this section, we construct a baby-Skyrme model, the DMI and also potentials 
from a model of fermions coupled with the Skyrme field $\bm{n}$ 
and also a background $SU(2)$ constant gauge field. 
In~\cite{Jaroszewicz:1985ip,Abanov:2000ea,Abanov:2001iz,Amari:2019tgs}, 
the authors investigated the $O(3)$ nonlinear sigma model lagrangian and also their topological terms 
based on the derivative expansion of the lagrangian of the fermions coupled with the Skyrme field 
via $\partial_\mu \bm{n}$. There are some recent theoretical studies about the fermions with 
the baby-skyrmions~\cite{Perapechka:2018yux} and the magnetic skyrmions~\cite{Perapechka:2019upv}, 
taking into account the backreaction from the fermionic fields. 

We begin with the following vacuum functional:
\begin{align}
\mathcal{Z}= \int \mathscr{D}\psi\mathscr{D}\bar{\psi}e^{S_{\rm E}}
\label{vacuumfunctional}
\end{align}
where the Euclidean action is
\begin{align}
S_{\rm E}=\int d\tau\int d^2x \biggl[\bar{\psi}\Bigl(i\gamma_\mu (\partial_\mu-i\bm{A}_\mu)-m\bm{\tau}\cdot\bm{n}\Bigr)\psi\biggr]\,.
\end{align}
The Euclidean time component $\tau$ is defined by the Wick-rotation $t=x_0=-i\tau$. 
The gamma matrices are defined as $\gamma_\mu:=-i\sigma_\mu, \mu=1,2,3$, which satisfy the 
Clifford algebra $\{\gamma_\mu,\gamma_\nu\}=-2\delta_{\mu\nu}$. 
The DMI term emerges introducing a constant
background gauge field $\bm{A}_\mu=A_\mu^a\tau_a/2$~\cite{Schroers:2019hhe, Amari} defined as
\begin{align}
&A_1^a=(-D,0,0),~~A_2^a=(0,-D,0),
\nonumber \\
&\hspace{1cm}\textrm{all the others are zero}\,.
\label{DMgauge}
\end{align}
Performing the integration (\ref{vacuumfunctional}), we obtain the effective action $\omega(\bm{n})$
\begin{align}
\mathcal{Z}=\det i\mathcal{D}\equiv \exp[\omega(\bm{n})],~~
\omega(\bm{n}):=\textrm{Tr}\log (i\mathcal{D})\,,
\label{efaction1}
\end{align}
where the Dirac operator is given by
\begin{align}
i\mathcal{D}:=i\gamma_\mu (\partial_\mu-i\bm{A}_\mu)-m\bm{\tau}\cdot\bm{n}\,.
\label{Diracop}
\end{align}
In Euclidean space, the effective action is generally a complex quantity 
$\omega(\bm{n}):=\omega_R(\bm{n})+i\omega_I(\bm{n})$, where
\begin{align}
&\omega_R(\bm{n})=\frac{1}{2}\textrm{Tr}\log\mathcal{D}^\dagger\mathcal{D}\,,~~
\label{efactionr} \\
&\omega_I(\bm{n})=\frac{1}{2i}\textrm{Tr}\log(\mathcal{D}^\dagger)^{-1}\mathcal{D}\,.
\label{efactioni}
\end{align}
Our main concern here is its real part because it generates an effective soliton model 
similar to a baby-Skyrme type action. We expand $\omega_R(\bm{n})$
in terms of the derivatives of $\bm{n}$ field, {\it i.e.}, $\partial_\mu\bm{n}$. 
Here, we perform the expansion based on the heat kernel 
method~\cite{Ebert:1985kz, Reinhardt:1989st}, which directly investigate the static 
energy of the model. From the Dirac operator~(\ref{Diracop}), 
we define the hamiltonian $h$ 
\begin{align}
&i\mathcal{D}=\sigma_3(-\partial_\tau-h),
\\
&h:=-\sigma_3\sigma_k\Bigl(\partial_k+iD\frac{\tau^k}{2}\Bigr)+\sigma_3 m \bm{\tau}\cdot\bm{n},~~k=1,2\,.
\end{align}
One finds that a baby-Skyrme model with the DMI emerges 
after subtracting the gauged (\ref{DMgauge}) vacuum state. 
We define the vacuum hamiltonian with $\bm{n}_0=(0,0,1)$, 
\begin{align}
h_0=-\sigma_3\sigma_k\Bigl(\partial_k+iD\frac{\tau^k}{2}\Bigr)+\sigma_3 m\tau_3\,.
\label{hamiltonian0}
\end{align}
The choice for the gauge field (\ref{DMgauge}) and the vacuum hamiltonian (\ref{hamiltonian0}) 
violate the $SU(2)$ symmetry of the theory. 

Here, we similar analysis for the (3+1)-QCD effective model~\cite{Dhar:1985gh,Ebert:1985kz, Reinhardt:1989st,Wakamatsu:1990ud}, 
where a regularized action must be introduced because the action is generally divergent. 
According to \cite{Ebert:1985kz, Reinhardt:1989st}, we define the proper-time regularized action given by
\begin{align}
\omega_R(\bm{n})\to -\frac{1}{2}\int_{1/\Lambda^2}ds s^{-1}\textrm{Tr}\exp(-s\mathcal{D}^\dagger\mathcal{D}) \,.
\end{align}
There is a large difference between the (2+1)- and the (3+1)-models. Because 
in the (2+1)-model, (\ref{efactionr}) becomes finite after suitably subtracting the vacuum contribution, 
and no need to introduce the ultraviolet cutoff. 
When we consider the Dirac sea contribution to the total energy, 
the cutoff significantly improves the numerical convergence, so for the moment we leave it in the 
formulation. In this paper, we shall examine the resulting Skyrme models found by this expansion, 
so we finally set $\Lambda\to\infty$. The energy is given by 
$\omega_R(\bm{n})=-\int_0^\infty d\tau E_0$ and then
\begin{align}
E_0=\frac{1}{4\sqrt{\pi}}\int_{1/\Lambda^2}^\infty ds s^{-3/2}\textrm{Tr} K(s),~~K(s):=\exp(-sh^2)\,.
\end{align}
For the heat kernel expansion, we write the proper-time kernel by defining
\begin{align}
\mathcal{H}=\mathcal{H}_0+\mathcal{V}\,,~~\mathcal{H}:=h^2\,,~~\mathcal{H}_0:=h_0^2\,,
\end{align}
and 
\begin{align}
K(s):=K_0(s)K_1(s),
\end{align} 
where
\begin{align}
K_0(s)=\exp(-s\mathcal{H}_0)\,,
\end{align}
and the interaction part is
\begin{align}
K_1(s)=\textrm{T}\exp\biggl[-\int_0^s ds'K_0(-s')\mathcal{V}K_0(s')\biggr]\,,
\end{align}
where T denotes the proper-time ordering. 
The interaction part satisfies the differential equation 
\begin{align}
[\partial_s+K_0(-s)\mathcal{V}K_0(s)]K_1(s)=0,~~K_1(s=0)=1\,.
\label{heatequation}
\end{align}
It has the heat expansion
\begin{align}
K_1(s)=\sum_{n=0}^\infty s^nb_n,~~b_0=1\,.
\label{heatexpansion}
\end{align}
The heat coefficients $b_n$ can be easily obtained by plugging (\ref{heatexpansion}) into 
(\ref{heatequation}); the first a few terms are then
\begin{align}
&b_1=-\mathcal{V},~~b_2=\frac{1}{2}\mathcal{V}^2-\frac{1}{2}[\mathcal{H}_0,\mathcal{V}],
\nonumber \\
&b_3=-\frac{1}{6}\mathcal{V}^3+\frac{1}{2}[\mathcal{H}_0,\mathcal{V}]\mathcal{V}-\frac{1}{6}[\mathcal{H}_0,[\mathcal{H}_0,\mathcal{V}]],
\nonumber \\
&b_4=\frac{1}{24}\mathcal{V}^4-\frac{1}{4}[\mathcal{H}_0,\mathcal{V}]\mathcal{V}^2+\frac{1}{6}[\mathcal{H}_0,[\mathcal{H}_0,\mathcal{V}]]\mathcal{V}
\nonumber \\
&~~~~+\frac{1}{8}[\mathcal{H}_0,\mathcal{V}]^2-\frac{1}{24}[\mathcal{H}_0,[\mathcal{H}_0,[\mathcal{H}_0,\mathcal{V}]]]\,.
\end{align}
The energy in the heat kernel expansion is
\begin{align}
E_0=\frac{1}{4\sqrt{\pi}}\int_{1/\Lambda^2}^\infty ds s^{-3/2}\sum_{n=0}^\infty \textrm{Tr} (K_0(s)b_n)\,.
\end{align}
For evaluating the trace Tr, it takes the Lorentz, the flavor (isospin) and also the 
plain wave 
\begin{align}
h_{\rm plain}|\phi_\nu^0\rangle =\epsilon_\nu^0|\phi_\nu^0\rangle\,,~~
h_{\rm plain}=-\sigma_3\sigma_k\partial_k+\sigma_3 m\,,
\end{align}
the energy $E_0$ becomes
\begin{align}
E_0=\frac{1}{2}\sum_{n=0}^\infty\sum_\nu |\epsilon_\nu^0|^{1-2n}\Gamma\biggl(n-\frac{1}{2},\Bigl(\frac{\epsilon_\nu^0}{\Lambda}\Bigr)^2\biggr)
\langle\phi_\nu^0|b_n|\phi_\nu^0\rangle\,.
\end{align}
The explicit form of $\mathcal{V}$ can be
\begin{align}
\mathcal{V}&=m\sigma_k\bigl(\bm{\tau}\cdot\partial_k\bm{n}-i[\bm{A}_k,\bm{\tau}\cdot\bm{n}-\tau_3]\bigr)
\nonumber \\
&=m\sigma_k\bigl\{\bm{\tau}\cdot\partial_k\bm{n}
+D\bigl((\bm{\tau}\times\bm{n})_k-\tau_3\bigr)\bigr\}\,.
\end{align}
The first nonzero contribution to the energy is the second order term: $n=2$
\begin{align}
&E_0^{(2)}=\kappa_2\int d^2x \Bigl\{(\partial_i\bm{n})^2+2D\bm{n}\cdot(\nabla\times\bm{n})
\nonumber \\
&\hspace{1cm}-2D(\partial_1n_2-\partial_2n_1)
\nonumber \\
&\hspace{1cm}+2D^2(1-n_3)+D^2(1-n_3)^2
\Bigr\}\,,
\label{E2}
\\
&\kappa_2:=\frac{m}{8\pi^{3/2}}\Gamma\biggl(\frac{1}{2},\Bigl(\frac{m}{\Lambda}\Bigr)^2\biggr)
\underset{\Lambda\to\infty}{\longrightarrow} \frac{m}{8\pi}\,.
\end{align}
  
The calculations for the higher order contributions to the energy are almost straightforward but 
the results are cumbersome. The inclusion of the DMI is highly nontrivial for these 
terms. 
Therefore, for $n\geq 3$, we set $D=0$ and restrict ourselves to the case of no DMI to the Skyrme or higher order corrections.
The results of the subsequent order, $n=4$ is
\begin{align}
&E_0^{(4)}=\kappa_4\int d^2x \Bigl\{2(\partial_i\bm{n}\times\partial_j\bm{n})^2
+(\partial_i\bm{n})^2(\partial_j\bm{n})^2\Bigr\}\,, \\
&\kappa_4:=\frac{1}{96\pi^{3/2}m}\Gamma\biggl(\frac{5}{2},\Bigl(\frac{m}{\Lambda}\Bigr)^2\biggr)
\underset{\Lambda\to\infty}{\longrightarrow}
\frac{1}{128\pi m}\,.
\label{E4}
\end{align}

The imaginary part of the action (\ref{efaction1}) tells us the statistical property of the model. 
For that, we consider the $U(1)$ gauged model of (\ref{Diracop})
\begin{align}
i\mathcal{D}:=i\gamma_\mu (\partial_\mu-i\bm{A}_\mu-ia_\mu)-m\bm{\tau}\cdot\bm{n}
\end{align}
where $a_\mu$ is an external electromagnetic potential. After some efforts for the expansion of $a_\mu$ 
that it supplies the following contribution to the effective action~\cite{Abanov:2000ea,Abanov:2001iz}
\begin{align}
\omega_I(\bm{n})=-\int d^3x a_\mu J_\mu
\end{align}
where the topologiral current is
\begin{align}
&J_\mu=\frac{1}{16\pi i}\epsilon_{\mu\nu\delta}\textrm{tr}(u D_\nu uD_\delta u)\,,
\\
&D_\mu u:=\partial_\mu u-i[A_\mu,u],~~~~u:=\bm{\tau}\cdot\bm{n}\,,
\end{align}
and the third component becomes 
\begin{align}
J_3=\frac{1}{4\pi}\biggl(\epsilon_{abc}n_a\partial_1n_b\partial_2n_c
+D(\partial_1n_2-\partial_2n_1)+D^2n_3\biggr)\,.
\label{topologicaldensity}
\end{align}
The first term defines the well-known topological charge 
\begin{align}
Q&=\frac{1}{4\pi}\int d^2 x q(\bm{x})\nonumber\\
    &=\frac{1}{4\pi}\int d^2 x \bm{n}(\bm{x}) \cdot \left\{\partial_1 \bm{n}(\bm{x}) \times \partial_2 \bm{n}(\bm{x})\right\}\,.
\end{align}
If one employ the standard circular symmetric ansatz for the field $\bm{n}$ 
\begin{align}
\bm{n}=\left(\sin f(r) \cos \varphi, \sin f(r) \sin \varphi, \cos f(r) \right)
\end{align}
with the boundary condition 
\begin{align}
f(0)=\pi,~~f(\infty)=0\,,
\label{bc}
\end{align}
we easy to verify $Q=1$.

A topological vortex strength appears in the third term in the energy (\ref{E2}) and 
also in the topological charge density (\ref{topologicaldensity}).
Its integration becomes zero with the boundary condition (\ref{bc}). 
As \cite{Barton-Singer:2018dlh, Schroers:2019hhe} pointed out when the relevant integral is not well-defined
for some reason, it might be restricted to compact subsets and the contribution may be finite. 
The last term in (\ref{topologicaldensity}) can be rewritten as $\textrm{tr}[\bm{F}_{12}u]$ which is the form
of $SU(2)$ gauge invariance. This gauge invariance is broken in terms of the special gauge choice~(\ref{DMgauge}).
It is a topological quantity because it does not depend on the complex structure
~\cite{Schroers:2019hhe}.

In the (3+1)-QCD effective models~\cite{Diakonov:1987ty,Meissner:1989kq,Wakamatsu:1990ud,Christov:1995vm,Alkofer:1995mv}.
the effective fermionic action itself has its own soliton solutions. 
The model in (3+1)-dimensions is called the chiral quark soliton model or the semi-bosonized NJL soliton model. 
The energy comprises the valence quarks and the infinite sum of the Dirac sea fermions. The skyrmions and 
the quark wave functions are obtained in terms of the self-consistent manner. 
For our fermionic model (\ref{vacuumfunctional}), or (\ref{efactionr}), we also find the soliton solutions in the 
self-consistent analysis.  
Note that, though there are solutions in the model, 
no stable soliton solutions exist in the Skyrme-like 
model obtained by the derivative expansion~\cite{Aitchison:1985rta}. 
However, unlike in the (3+1)-model, our (2+1)-model may possibly have
the solutions. In this paper, we shall explore the solutions in the model. 

As a result, we obtain a 
Skyrme type model with the DMI from the fermionic model~(\ref{vacuumfunctional}) via  
a derivative expansion. We summerize the terms 
\begin{align}
    E & = \int d^2 x \Bigl\{\kappa_2 \left(\partial_i \bm{n} \right)^2 
\nonumber \\
    &+ \kappa_1 \bm{n} \cdot \left(\nabla \times \bm{n} \right)
    -\kappa_1 (\partial_1n_2-\partial_2n_1)
\nonumber \\
    & + \kappa_{4a} \left(\partial_i \bm{n} \times \partial_j \bm{n} \right)^2 + \kappa_{4b} \left(\partial_i \bm{n} \right)^2 \left(\partial_j \bm{n} \right)^2 
\nonumber\\
    & +\kappa_{0a}\left(1-n_3 \right)+\kappa_{0b}\left(1-n_3 \right)^2 \Bigr\}\,,
    \label{eq:edens}
\end{align}
where the each term corresponds to
\begin{align}
&\textrm{(i) the kinetic}:~
E_2:=\kappa_2\int d^2 x \left(\partial_i \bm{n} \right)^2\,,
\nonumber \\
&\textrm{(ii) the DMI}:~
E_1:=\kappa_1\int d^2x \bm{n} \cdot \left(\nabla \times \bm{n} \right)\,,
\nonumber \\
&\textrm{(iii) the Skyrme}:~
E_{4a}:=\kappa_{4a}\int d^2x \left(\partial_i \bm{n} \times \partial_j \bm{n} \right)^2\,,
\nonumber \\
&\textrm{(iv) an extended 4th}:~
\nonumber \\
&\hspace{2.5cm}E_{4b}:=\kappa_{4b}\int d^2x \left(\partial_i \bm{n} \right)^2 \left(\partial_j \bm{n} \right)^2\,,
\nonumber \\
&\textrm{(v) the Zeeman}:~
E_{0a}:=\kappa_{0a}\int d^2x \left(1-n_3 \right)\,,
\nonumber \\
&\textrm{(vi) a squared Zeeman}:~
E_{0b}:=\kappa_{0b}\int d^2x  \left(1-n_3 \right)^2\,.
\end{align} 
Note that the integration of the vortex strength is zero. 

One can fix the above coefficients for corresponding to (\ref{E2}) and (\ref{E4}) such as
\begin{align}
&\kappa_1:=2\kappa_2D,~~\kappa_{4a}=2\kappa_4,~~\kappa_{4b}:=\kappa_4\,,
\nonumber \\
&\kappa_{0a}:=2\kappa_2D^2,~~\kappa_{0b}:=\kappa_2D^2\,.
\label{eq:kappa}
\end{align}

In this paper, we do not restrict our analysis to the above relations (\ref{eq:kappa}) 
and freely choose these parameters in order to see how far a wide range of solutions exists.  





%%%%%%%%%%%%%%%%%%%%%
\section{\label{sec:3}The model}
%%%%%%%%%%%%%%%%%%%%%

The configuration space of the model has comprised maps from the plane $\mathbb{R}^2$ to the target space $S^2$. 
Taking coordinates $\Theta, \Phi$ on the target sphere (corresponding to the usual spherical polar coordinates), 
the best-known solution is the rotationally symmetric solution given by 
\begin{align}
    \Theta =f(r),~~\Phi=\varphi\,,
\end{align}
where $r,\varphi$ are the usual polar coordinates on the plane. 
As a result, the configuration giving rise to a baby-skyrmion with topological charge $n$ is defined by
\begin{align}
    \bm{n}=\left(\sin f(r) \cos \left(n \varphi+\gamma\right), \sin f(r) \sin  \left(n \varphi+\gamma\right), \cos f(r) \right)\,,
    \label{eq:ansatz}
\end{align}
where the phase $\gamma$ describes the internal orientation of the solution. 
Notably, the energy of the magnetic skyrmion depends on $\gamma$ and it takes minimal value with $\gamma=\pi/2$. 
Further, rotationally invariant configuration \eqref{eq:ansatz} exists only for $n=1$. 
Substituting (\ref{eq:ansatz}) into (\ref{eq:edens}), we define the energy density $\varepsilon [f]$
\begin{align}
    &\varepsilon[f]=\kappa_2 \left(f'^2+ \frac{\sin^2 f}{r^2} \right)
\nonumber \\
    &~~+\kappa_1 \left(f'+\frac{\sin 2f}{2r}\right)\sin \gamma 
	-\kappa_1\biggl(\cos f f'+\frac{\sin f}{r}\biggr)\sin \gamma
\nonumber \\ 
    &~~+\kappa_{4a} \frac{2\sin^2 ff'^2}{r^2}
    +\kappa_{4b}\left(f'^4 + \frac{2\sin^2 ff'^2}{r^2}+\frac{\sin^4 f}{r^4}\right)\nonumber\\
    &~~+\kappa_{0a}\left(1-\cos f\right)+\kappa_{0b}\left(1-\cos f\right)^2\,,
    \label{eq:edens-f}
\end{align}
where $f':=\dfrac{df(r)}{dr}$.
The function $f(r)$ satisfies the Euler equation,  non-linear second order ordinary differential equation
\begin{align}
   &\kappa_2\left(rf''+f'-\frac{\sin 2f}{2r}\right)+\kappa_1 \sin^2 f \sin \gamma
\nonumber \\
   &+\kappa_{4a}\left(\frac{2\sin^2 f}{r} f''+\frac{\sin 2f}{r} f'^2 -\frac{2\sin^2 f}{r^2} f' \right)
\nonumber \\
   &+\kappa_{4b}\Biggl\{ \left( 6r f'^2+\frac{2\sin^2 f}{r} \right)f'' 
\nonumber \\
   &+\left( 2f'^2 +\frac{\sin 2f}{r}f' -\frac{2\sin^2 f}{r^2}\right)f' 
   -\frac{\sin 2f}{2r^2}+\frac{\sin 4f}{4r^2}\Biggr\}
\nonumber \\
   &-\frac{\kappa_{0a}}{2}r\sin f -\frac{\kappa_{0b}}{2}r\left(2\sin f -\sin 2f \right)=0\,.
   \label{Eulereq}
\end{align}

In the following, we sometimes refer to the model in terms of its parameters 
setting: $[\kappa_2, \kappa_1 ,\kappa_{4a}, \kappa_{4b}, \kappa_{0a}, \kappa_{0b}]$. 


%%%%%%%%%%%%%%%%%%%%%%%%%%%%%%%%%%%%%%%%%%%%%%%%%%%%%%%%%%%%%%%%%%%%%%Fig1
\begin{figure*}[t]
  \begin{minipage}[b]{0.5\linewidth}
    \centering
    \includegraphics[keepaspectratio,scale=1.2,bb=0 0 193 137]{fig1_left.pdf}
  \end{minipage}\hspace{-0.5cm}
  \begin{minipage}[b]{0.5\linewidth}
    \centering
    \includegraphics[keepaspectratio,scale=1.2,bb=0 0 193 137]{fig1_right.pdf}
  \end{minipage}
  \caption{The skyrmions with $[\kappa_2, 0.0, 0.0, 1.0, 1.0, 1.0]$. The profile functions $f(r)$ (left) and the 
energy density $\varepsilon (r) $ (right). The model has the genuine-compacton solution for $\kappa_2=0.0$.}
  \label{fig:ho-pot}
\end{figure*}
%%%%%%%%%%%%%%%%%%%%%%%%%%%%%%%%%%%%%%%%%%%%%%%%%%%%%%%%%%%%%%%%%%%%%Fig1

%%%%%%%%%%%%%%%%%%%%%%%%%%%%%%%%%%%%%%%%%%%%%%%%%%%%%%%%%%%%%%%%%%%%Fig2
\begin{figure}[t]
    \centering
    \includegraphics[keepaspectratio,scale=1.2,bb=0 0 193 137]{fig2.pdf}
  \caption{We plot the compacton shown in Fig. \ref{fig:ho-pot}: 
the profile function and its derivatives $f(r),f'(r),f''(r)$~(the blue, the red, the black lines), 
which clearly shows that the first derivative is continuous at the boundary $r=R=5.231$.}
  \label{fig:ho-pot-dfdr}
\end{figure}
%%%%%%%%%%%%%%%%%%%%%%%%%%%%%%%%%%%%%%%%%%%%%%%%%%%%%%%%%%%%%%%%%%%%Fig2


%%%%%%%%%%%%%%%%%%%%%%%%%%%%%%%%%%%%%%%%%%%%%%%%%%%%%%%%%%%%%%%%%%%%%%%%%%%%%%%%%%%Fig3
\begin{figure*}[t]
  \begin{minipage}[b]{0.5\linewidth}
    \centering
    \includegraphics[keepaspectratio,scale=1.2,bb=0 0 193 137]{fig3_left.pdf}
  \end{minipage}\hspace{-0.5cm}
  \begin{minipage}[b]{0.5\linewidth}
    \centering
    \includegraphics[keepaspectratio,scale=1.2,bb=0 0 193 137]{fig3_right.pdf}
  \end{minipage}
  \caption{The skyrmions with $[\kappa_2, 1.0, 1.0, 0.0, 1.0, 0.0]$ of $\kappa_2=0.0,0.5,1.0,5.0$. 
The profile functions $f(r)$ (left) and the energy density $\varepsilon (r)$ (right).
The restricted model $\kappa_2=0.0$ has the compacton solution (the blue line). }
  \label{fig:dm-sk-ze}
\end{figure*}
%%%%%%%%%%%%%%%%%%%%%%%%%%%%%%%%%%%%%%%%%%%%%%%%%%%%%%%%%%%%%%%%%%%%%%%%%%%%%%%%%%%Fig3

%%%%%%%%%%%%%%%%%%%%%%%%%%%%%%%%%%%%%%%%%%%%%%%%%%%%%%%%%%%%%%%%%%%%%%%%%%%%%%%%%%%Fig4
\begin{figure}[t]
    \centering
    \includegraphics[keepaspectratio,scale=1.2,bb=0 0 193 137]{fig4.pdf}
  \caption{We plot the compacton shown in Fig. \ref{fig:dm-sk-ze}: 
the profile function $f(r)$~(the blue line) and its derivative $f'(r)$ (the red line), 
which clearly shows that the derivative is not continuous at the 
boundary $r=R=3.396$.}
  \label{fig:dm-sk-ze-dfdr}
\end{figure}
%%%%%%%%%%%%%%%%%%%%%%%%%%%%%%%%%%%%%%%%%%%%%%%%%%%%%%%%%%%%%%%%%%%%%%%%%%%%%%%%%%%Fig4


%%%%%%%%%%%%%%%%%%%%%%%%%%%%%%%%%%%%%%%%%%%%%%%%%%%%%%%%%%%%%%%%%%%%%%%%%%%%%%%%%%%%%%%%%%%%%%%
\section{\label{sec:4}The solutions: the compactons emerged from the DMI term and the 4th order terms}
%%%%%%%%%%%%%%%%%%%%%%%%%%%%%%%%%%%%%%%%%%%%%%%%%%%%%%%%%%%%%%%%%%%%%%%%%%%%%%%%%%%%%%%%%%%%%%%

\subsection{The models with the Skyrme term and the extended 4th term}

The model: $[0, 0, \kappa_{4a}, 0, \kappa_{0a},0]$ 
is known as the restricted baby-Skyrme model which possesses the solutions so-called compacton.
Compactons are solutions reach their vacuum value $f\sim 0, (f'\sim 0)$ with finite radius $r=R$.  
It is thought the compacton has an advantageous forming of the skyrmion lattice because it could smoothly 
connect to the neighbors~\cite{Gudnason:2022aig}. 

We give the following classification for the compactons whether the 
function is continuously differentiable at the boundary or not 
\begin{itemize}
\item \textrm{{\it Genuine}-compactons:}~~$f(R)=0,~\dfrac{df(r)}{dr}\biggr|_{r=R}=0$\,.
\item \textrm{{\it Weak}-compactons:}~~$f(R)=0,~\dfrac{df(r)}{dr}\biggr|_{r=R}\neq 0$\,.
\end{itemize}
Note that, for the weak-comacton case, even if the profile function is not differentiable, 
the energy density is still continuous because of the term $\sin^2f$. 
Also, Speight~\cite{Speight:2010sy} also gave a classification in his paper for slightly 
different purpose. 

We consider a slightly generalized restricted model such as
$[0, 0, \kappa_{4a}, \kappa_{4b}, \kappa_{0a}, \kappa_{0b}]$. 

\subsubsection{$[0, 0, \kappa_{4a} ,0, \kappa_{0a}, \kappa_{0b}]$}
\label{sec:skyrme}

The model is the restricted baby-Skyrme model. 
Gisiger and Paranjape~\cite{Gisiger:1996vb} found the compacton in the model with the Zeeman potential 
($\kappa_{0b}=0$) solving the Euler equations. Also Adam et.al.~\cite{Adam:2010jr} found the compacton and 
also the non-compacton solutions in the models with different potential terms by solving the BPS equations. 
These potential terms~\cite{Adam:2010jr} are, for example, the Zeeman potential $V=(1-n_3)$, 
the new-baby potential $V=(1-n_3^2)$, and the squared Zeeman potential $V=(1-n_3)^2$. 
Here we solve the Euler equations of the model with two potential terms; the Zeeman and the squared Zeeman potential. 
We examine the mixed potential of the vacuum structure. From the boundary condition (\ref{bc}), the potential 
should take the minimum at $n_3=1$. We rewrite the potential
\begin{align}
    V[n_3]&=\kappa_{0a}(1-n_3)+\kappa_{0b}(1-n_3)^2 
\nonumber\\
&=(\kappa_{0a}+\kappa_{0b})(1-n_3)\biggl(1-\frac{\kappa_{0b}}{\kappa_{0a}+\kappa_{0b}}n_3\biggr) \label{eq:pot1}
\\
&= \kappa_{0b}\left(n_3-\frac{\kappa_{0a}+2\kappa_{0b}}{2\kappa_{0b}}\right)^2 - \frac{\kappa_{0a}^2}{4\kappa_{0b}} \label{eq:pot2},
\end{align}
where the parameters are set $\kappa_{0a},\kappa_{0b}\neq 0$.
For (\ref{eq:pot1}), when the parameters are $\kappa_{0b}/(\kappa_{0a}+\kappa_{0b})=\pm 1$, 
i.e. $\kappa_{0a}=0$ or $\kappa_{0b}=-\kappa_{0a}/2$, these potentials become the squared Zeeman potential term or the new-baby potential term$V=(1-n_3^2)$. 
In the case of the squared Zeeman potential, the model has no compacton solutions. 
In the case of the new-baby potential, Adam et.al. solves the Bogomol'nyi equation and obtain the weak-compacton.
Here, we would like to get the new compacton for $\kappa_{0a}\neq 0 $ and $\kappa_{0b}\neq -\kappa_{0a}/2 $.
Also according to (\ref{eq:pot2}) in the case of $\kappa_{0b}>0$, when the parameters 
satisfy $(\kappa_{0a}+2\kappa_{0b})/(2\kappa_{0b})\geq 1$, e.g., $\kappa_{0a}\geq 0$, the potential is always
positive and take the minimum: $V=0$ at $n_3=1$.
Also, for $\kappa_{0b}<0$, when the parameters are $(\kappa_{0a}+2\kappa_{0b})/(2\kappa_{0b})\leq 0$, e.g., 
$\kappa_{0a}\geq -2 \kappa_{0b}$, the potential is always positive and takes the minimum at $n_3=1$. 
As a result, in these conditions, the potential is suitable for finding the soliton solutions in the model. 
 
The solution found in \cite{Gisiger:1996vb} is apparently the weak-compacton. It is directly verified 
by examining the analytical behavior at the compacton boundary $r=R$, 
where the profile function can be smoothly connected to the vacuum.
We assume the series expansion around $r=R$
\begin{align}
    f(r)=\sum_{s=0}^\infty A_s(R-r)^s\,.
\label{expansion}
\end{align}
The smoothness of the energy $f(R)=0$ suggests the expansion starts with $s>0$. Here, it is enough 
to consider the lowest order term so we substitute $f(r)\sim A_s (R-r)^s$ 
into the Euler equation and obtain the relation for the lowest-order contribution
\begin{align}
    \frac{2\kappa_{4a}}{r}A_s^3 s (2s-1) (R-r)^{3s-2}-\frac{\kappa_{0a}}{2}r A_s (R-r)^s=0\,.
\end{align}
Obviously, it has a solution $s=1$. 
This implies that there is a standard linear approach to the vacuum - 
a typical feature for compactons in the restricted baby-Skyrme model.

For the case of the two potentials, the Zeeman and the squared Zeeman potential coexist, 
we can obtain the analytical solution in a similar way. 
According to~\cite{Gisiger:1996vb}, we separate the equation \eqref{Eulereq} into
\begin{subnumcases}{}
\kappa_{4a} \biggl( 2 f''-\dfrac{2}{r}f'+2\cot f f'^2\biggr)
-\dfrac{\kappa_{0a}}{2}r^2\textrm{csc} f
\nonumber \\
\hspace{1cm}-\kappa_{0b}r^2(\textrm{csc} f-\cot f)  =0,\hspace{1cm}r\leq R,
\label{GPa} \\
\sin f=0,\hspace{4.5cm} r>R.
\label{GPb}
\end{subnumcases}
For simplicity, we employ the rescaling of the parameters such as
$\kappa_{0a}/\kappa_{4a}\to \kappa_{0a}, \kappa_{0b}/\kappa_{4a}\to \kappa_{0b}$. 
We define a new field 
\begin{align}
  \mathcal{F}(r):=\cos f(r) -\frac{\kappa_{0a}+2\kappa_{0b}}{2\kappa_{0b}}
\end{align}
and the equation (\ref{GPa}) becomes a very simple form
\begin{align}
  \frac{d^2 \mathcal{F}}{dr^2}-\frac{1}{r}\frac{d\mathcal{F}}{dr}-\frac{\kappa_{0b}}{2}r^2\mathcal{F}=0\,.
\label{Eulereq3}
\end{align}
From the boundary condition (\ref{bc}), we have
\begin{align}
&\mathcal{F}(r=0)=-1-\frac{\kappa_{0a}+2\kappa_{0b}}{2\kappa_{0b}},
\nonumber \\
&\mathcal{F}(r=R)=1-\frac{\kappa_{0a}+2\kappa_{0b}}{2\kappa_{0b}}\,.
\end{align}
The equation (\ref{Eulereq3}) in $\kappa_{0b}>0$ can be solved analytically. 
The solution is
\begin{align}
&\mathcal{F}(r)=-\frac{\kappa_{0a}+4\kappa_{0b}}{2\kappa_{0b}}\cosh  
\left(\frac{\sqrt{\kappa_{0b}}r^2}{2\sqrt{2}}\right)
\nonumber \\
&\hspace{0.8cm}+\sqrt{\frac{2(\kappa_{0a}+2\kappa_{0b})}{\kappa_{0b}}} \sinh \left(\frac{\sqrt{\kappa_{0b}}r^2}{2\sqrt{2}}\right),\nonumber\\
& r\in \left[0,R=\frac{2^{3/4}}{\kappa_{0b}^{1/4}} \sqrt{\textrm{arccosh}\left( \frac{\kappa_{0a}+4\kappa_{0b}}{\kappa_{0a}}\right)}\right].
\end{align}

In $\kappa_{0b}<0$ and $\kappa_{0a}\geq 2|\kappa_{0b}|$, the solution is
\begin{align}
    &\mathcal{F}(r) = \frac{\kappa_{0a}-4\kappa_{0b}}{2\kappa_{0b}} \cos \left(\frac{\sqrt{\kappa_{0b}}r^2}{2\sqrt{2}}\right)\nonumber\\
    &\hspace{0.8cm}+\sqrt{\frac{2(\kappa_{0a}-2\kappa_{0b})}{\kappa_{0b}}}\sin \left(\frac{\sqrt{\kappa_{0b}}r^2}{2\sqrt{2}}\right),\nonumber\\
    &r\in \left[0,R=\frac{2^{3/4}}{\kappa_{0b}^{1/4}}\sqrt{\arccos \left(\frac{\kappa_{0a}-4\kappa_{0b}}{\kappa_{0a}}\right)}\right].
\end{align}

%%%%%%%%%%%%%%%%%%%%%%%%%%%%%%%%%%%%%%%%%%%%%%%%%%%%%%%%%%%%%%%%%%%%%%%%%%%%%%%%%%%%%%%%Fig5
\begin{figure*}[t]
  \begin{minipage}[b]{0.5\linewidth}
    \centering
    \includegraphics[keepaspectratio,scale=1.2,bb=0 0 193 137]{fig5_left.pdf}
  \end{minipage}\hspace{-0.5cm}
  \begin{minipage}[b]{0.5\linewidth}
    \centering
    \includegraphics[keepaspectratio,scale=1.2,bb=0 0 193 137]{fig5_right.pdf}
  \end{minipage}
  \caption{The skyrmions with $[\kappa_2, 1.0, 1.0, 0.0, 0.0, 1.0]$. The profile functions $f(r)$~(left) and the 
energy density $\varepsilon (r)$~(right). The restricted model $\kappa_2=0.0$ (the blue line) has no compacton. }
  \label{fig:sk-dm-sq}
\end{figure*}
%%%%%%%%%%%%%%%%%%%%%%%%%%%%%%%%%%%%%%%%%%%%%%%%%%%%%%%%%%%%%%%%%%%%%%%%%%%%%%%%%%%%%%%%Fig5




\subsubsection{$[0, 0, 0, \kappa_{4b} ,\kappa_{0a}, \kappa_{0b}]$}	
\label{sec:quartic}

Interestingly, if we replace the Skyrme term with
the extended 4th term, we can obtain the genuine-compacton. 
First, we show that if there is a compacton in
the equation, it always becomes the genuine-compacton. 
The following discussion is valid for the potentials: the Zeeman potential and the
mixtures of the Zeeman and the squared Zeeman potential. 
We assume at the boundary $f(R) = 0$, thus
the Euler equation at the boundary $r = R$ becomes

\begin{align}
    2 f'(R)^2 \left\{3R  f''(R) +f'(R)\right\} =0\,,
\label{EeqaR}
\end{align}
has the solutions
\begin{align}
  \textrm{(i)}~f'(R)=0,~~\textrm{(ii)}~f''(R)=-\frac{f'(R)}{3R}.
\end{align}
First, we examine case (ii). If $f''(R)=-f'(R)/(3R)\neq 0$, the energy density is not continuous at $r=R$
that is not what we aim for. If $f'(R)=0$, the energy becomes continuous and then becomes $f(R)=f'(R)=f''(R)=0$.
It connects to the trivial vacuum solution $f(r)=0,r\in [0,R]$.
Therefore, case (i) $f'(R)=0$ should be employed for finding the nontrivial, the genuine-compacton solutions. 
As a result, the second derivative $f''(r)$ is not continuous at $r=R$. 

This situation is again easy to confirm in terms of the expansion~(\ref{expansion}).
For the lowest order, we obtain
\begin{align}
    6 \kappa_{4b}r A_s^3 s^3 (s-1)r (R-r)^{3s-4}-\frac{\kappa_{0a}}{2}r A_s (R-r)^s=0\,.
\label{exeq2}
\end{align}
Now we have a solution $s=2$ for (\ref{exeq2}). 
This implies that there is a standard parabolic approach to the vacuum - a typical feature for the genuine-compacton.

In this case,  the analytical solution has been yet not found, so we numerically solve the Euler equation. 
We use the Newton-Raphson method with $N=1000$ mesh points. 
We employ the standard rescaling scheme to the coordinate
\begin{align}
x=\frac{r}{1+r},~~x\in [0,1)\,.
\end{align}
The relative numerical errors of order $10^{-7}$.
Note that we always solve the Euler equation for the entire radial coordinate $x$ (not in the compact subset) 
even for the compactons, which means the compacton naturally arises in our numerical computation. 
We present our results for the Zeeman potential in Fig.\ref{fig:ho-pot} . As increasing $\kappa_2$, 
the tail of the profile function extends and the maximum of the energy density is higher. 
This is because the kinetic term $\kappa_2 (\partial_i \bm{n} )^2>0$ exists in the energy density. 
Fig.\ref{fig:ho-pot-dfdr} shows the compacton solution and also the derivatives $f'(r), f''(r)$. 
One can easily see that it is the genuine-compacton and the $f''(r)$ has discontinuity around $r=R$.

Let the thing be clearer that the above discussion does not directly mean there always 
exists the compacton in the model. If we employ the squared Zeeman potential, 
the solution becomes the normal skyrmions. 
It corresponds to the solution found in~\cite{Adam:2009px}, where the model is composed of the 
Skyrme term and the squared Zeeman potential. 





\subsection{The model of the DMI}\label{sec:DMI}

Here, we would like to find the DMI-mediated compactons of our model. 
First, according to the baby-skyrmions case, we set $\kappa_2=0$, in other words the restricted model. 
In the case, we omit the Zeeman energy in order to get a solution. The parameter set is $[0,\kappa_1,0,0,0,\kappa_{0b}]$.
From~(\ref{Eulereq}), the equation is 
\begin{align}
    \kappa_1 \sin^2 f-\kappa_{0b}r\sin f(1-\cos f)=0\,.
\end{align}
For the nontrivial solutions $\sin f \neq 0$, except at the boundaries, then
\begin{align}
    \kappa_1 \sin f=\kappa_{0b}r(1-\cos f)\,.
\end{align}
By squaring both sides, we finally get
\begin{align}
(\kappa_{0b}^2 r^2 +\kappa_{1}^2)\cos^2 f -2\kappa_{0b}^2r^2 \cos f +\kappa_{0b}^2 r^2    +\kappa_{1}^2=0\,.
\end{align}
We obtain the nontrivial solution of the form
\begin{align}
    \cos f =\frac{\kappa_{0b}^2r^2-\kappa_{1}^2}{\kappa_{0b}^2r^2+\kappa_1^2}\,.
\end{align}
This is exactly the solution found by Schroer in the equation of motion and also of a first order 
Bogomol'nyi equation~\cite{Barton-Singer:2018dlh}. 
It apparently is not the compacton. 

It can be confirmed in terms of analysis of the expansion at the boundary (\ref{expansion}).
At the lowest order, we obtain
\begin{align}
    \kappa_1 A_s^2 (R-r)^{2s}-\frac{\kappa_{0a}}{2}r A_s (R-r)^s=0
\label{exeq3}
\end{align}
where the solution is $s=0$. 
This implies that the profile function is a constant at $r=R$ and it corresponds to 
the normal skyrmion solution. 

%%%%%%%%%%%%%%%%%%%%%%%%%%%%%%%%%%%%%%%%%%%%%%%%%%%%%%%%%%%%%%%%%%%%Fig6
\begin{figure*}[t]
  \begin{minipage}[b]{0.5\linewidth}
    \centering
    \includegraphics[keepaspectratio,scale=1.2,bb=0 0 193 137]{fig6_left.pdf}
  \end{minipage}\hspace{-0.5cm}
  \begin{minipage}[b]{0.5\linewidth}
    \centering
    \includegraphics[keepaspectratio,scale=1.2,bb=0 0 193 137]{fig6_right.pdf}
  \end{minipage}
  \caption{The skyrmions with $[\kappa_2, 1.0, 0.0, 1.0, 1.0, 0.0]$. 
The profile functions (left) and the energy density (right). The restricted model: $\kappa_2=0.0$ 
is the compacton solution (the blue line).}
  \label{fig:dm-4th-ze}
\end{figure*}
%%%%%%%%%%%%%%%%%%%%%%%%%%%%%%%%%%%%%%%%%%%%%%%%%%%%%%%%%%%%%%%%%%%%%%Fig6


%%%%%%%%%%%%%%%%%%%%%%%%%%%%%%%%%%%%%%%%%%%%%%%%%%%%%%%%%%%%%%%%%%%%%%Fig7
\begin{figure}[t]
    \centering
    \includegraphics[keepaspectratio,scale=1.2,bb=0 0 193 137]{fig7.pdf}
  \caption{We plot the compacton shown in Fig. \ref{fig:dm-4th-ze}: 
the profile function and the derivatives $f(r),f'(r)$~(the blue, the red lines), 
which clearly shows that the derivatives are continuous at the boundary $r=R=5.969$.}
  \label{fig:dm-4th-ze-dfdr}
\end{figure}
%%%%%%%%%%%%%%%%%%%%%%%%%%%%%%%%%%%%%%%%%%%%%%%%%%%%%%%%%%%%%%%%%%%%%%Fig7

%%%%%%%%%%%%%%%%%%%%%%%%%%%%%%%%%%%%%%%%%%%%%%%%%%%%%%%%%%%%%%%%%%%%%%%%%%%%%%%%%%%%Fig8
\begin{figure*}[t]
  \begin{minipage}[b]{0.5\linewidth}
    \centering
 \includegraphics[keepaspectratio,scale=0.3,bb=0 0 772 591]{fig8_left.pdf}
  \end{minipage}\hspace{-0.5cm}
  \begin{minipage}[b]{0.5\linewidth}
    \centering
    \includegraphics[keepaspectratio,scale=0.3,bb=0 0 772 591]{fig8_right.pdf}
  \end{minipage}
  \caption{For the Derrick's theorem, we compute the ratio $-E_1/(2E_{0a})$ 
of the Skyrme-DMI-Zeeman model:$[0.0, \kappa_1, \kappa_{4a} ,0.0 ,1.0 ,0.0]$~(left) 
and the ratio $-E_1/(2E_{0b})$ of the Skyrme-DMI-squared Zeeman model:
$[0.0, \kappa_1, \kappa_{4a}, 0.0, 0.0, 1.0]$~(right)
for various values of $(\kappa_1,\kappa_{4a})$.}
  \label{fig:sk-dm-ze-derrick}
\end{figure*}
%%%%%%%%%%%%%%%%%%%%%%%%%%%%%%%%%%%%%%%%%%%%%%%%%%%%%%%%%%%%%%%%%%%%%%%%%%%%%%%%%%%%Fig8





\subsection{Inclusion of the DMI term and the 4th order terms}\label{sec:DMI-quartic}

Next, in order to get a compacton, we add the 4th order terms as well.  
The model is constructed based on the term in the restricted baby-Skyrme model. 
When both the DMI and the 4th order terms are included, again no analytical solutions 
exist and we have to solve the equation numerically. 
We treat the model with the parameter set: $[0,\kappa_1,\kappa_{4a},0,\kappa_{0a},\kappa_{0b}]$.
The model contains two types of derivative terms: the DMI and the Skyrme term, which break 
the scale invariance. The magnetic Skyrme model and the baby-Skyrme model possess soliton 
solutions for these terms and usually do not need to combine them for stability. 
We shall look at Derrick's argument for the model. 
The energy applying the spatial rescaling $x\mapsto \mu x$ can be written as
\begin{align}
e(\mu)=E_2+\mu^{-1}E_1+\mu^2E_{4a}+\mu^{-2}(E_{0a}+E_{0b})\,. 
\end{align}
Taking the derivative with $\mu$, we get 
\begin{align}
&\frac{de(\mu)}{d\mu}\biggr|_{\mu=1}=-E_1+2E_{4a}-2(E_{0a}+E_{0b})=0\label{eq:e1e4}\,.
\end{align}
For evading the Derrick's argument, the potential energy should satisfy $E_{0a}+E_{0b}>0$. 
It is realized with the parameters $\kappa_{0a}>0,\kappa_{0b}>0,$ or $\kappa_{0a}\geq -2\kappa_{0b},\kappa_{0b}<0$
 (\ref{sec:skyrme}). 
We divide both sides of (\ref{eq:e1e4}) by $E_{0a}+E_{0b}$, we obtain
\begin{align}
    \frac{-E_1}{2(E_{0a}+E_{0b})}+\frac{E_4}{E_{0a}+E_{0b}}=1\,.
\end{align}
Since $-E_1,E_{4a}\geq 0$ 
\begin{align}
0\leq \frac{-E_1}{2(E_{0a}+E_{0b})},~~\frac{E_4}{E_{0a}+E_{0b}} \leq 1\,.
\end{align} 
In the magnetic Skyrme model, the Derrick's theorem supports $-E_1/(2(E_{0a}+E_{0b}))=1$, while 
in the baby-Skyrme model, $E_{4a}/(E_{0a}+E_{0b})=1$. The ratio thus tells us which terms have 
the dominant role in the stability of the solitons. We shall discuss it in the next subsection. 

For the Skyrme term and the Zeeman potential; $[0, \kappa_1, \kappa_{4a}, 0, \kappa_{0a}, 0]$, 
the solutions are plotted in Fig.\ref{fig:dm-sk-ze} also with the non-compacton solutions $\kappa_2\neq 0$. 
As increasing $\kappa_2$, the tail of the profile function extends and changes the convex from upward to downward. 
The maximum of the energy density approaches to the origin. 
This is because when the convex is downward, the range of $\pi/2<f\leq \pi$ reduces. 
At that time, the energy density of the kinetic term and the Skyrme term increases, while the contribution of the DMI term is negative. 
As a result, the energy density enhances in the vicinity of the origin. 
In Fig.\ref{fig:dm-sk-ze-dfdr}, we focus on a characteristic behavior of this solution;
it exhibits $f'(R)\neq 0$ at the boundary $r=R$ which looks similar with the compactons found by 
Gisiger and Paranjape~\cite{Gisiger:1996vb}. It can be easy to verify how the feature is realized 
as the following. If one plug the boundary condition $f(R)=0$ into the Euler equation (\ref{Eulereq}), 
at the boundary it has
\begin{align}
    \left.\frac{df(r)}{dr}\right|_{r=R} = \sqrt{\frac{\kappa_{0a}}{4\kappa_{4a}}}R\,.
\end{align}
It is easy to see that when the boundary is far from the origin $R\to \infty$, $df/dr\to \infty$, 
and then the energy density (\ref{eq:edens-f}) becomes divergent. 
Therefore, in order to avoid such singularity of the energy, the model has to choose a solution with a concrete finite radius, i.e., the compacton.



When a different potential term is chosen, such as the squared  Zeeman potential term; $[0, \kappa_1, \kappa_{4a}, 0, 0, \kappa_{0b}]$,  
the solutions are not compactons (see Fig.\ref{fig:sk-dm-sq}). 
As increasing $\kappa_2$, the tail of the profile function extends and the maximum of energy density becomes higher and closer to the origin. 
Compared with Fig.\ref{fig:dm-sk-ze}, the change is moderate. 
If we assume $f(R)=0$, the Euler equation at the boundary now becomes 
\begin{align}
    \left.\frac{df(r)}{dr}\right|_{r=R}=0\,,
\end{align}
and we are not able to say anything about the compactness from this point of view. 
In the next subsection, we will present the new compacton solution satisfying this condition. 


\subsection{The gunuine DMI compacton: $[0,\kappa_1,0,\kappa_{4b},\kappa_{0a},0]$}

So far, only in the restricted cases $\kappa_ 2= 0$ 
the compactons emerged, while for $\kappa_ 2\neq 0$ the solutions became normal skyrmions. 
Therefore, the kinetic term simply extends the tail of solutions.
For the 4th order terms, the models with $\kappa_{4a}\neq 0 $ and $\kappa_{4b}=0$ 
have the weak-compactons, while the models with $\kappa_{4a}=0$ and $\kappa_{4b}\neq 0$ 
have the genuine-compactons. Therefore, the extended 4th term has a role
in constructing the genuine-compactons.
For our potentials, in the squared Zeeman potential case  ($\kappa_{0a}=0$ and $\kappa_{0b}\neq 0$), 
the solutions are the baby-skyrmions, 
and in the Zeeman potential case ($\kappa_{0a}\neq 0$ $\kappa_{0b}=0$) the solutions become 
compactons.

We saw that the DMI has less effective for constructing the compactons. 
The reason is as follows: The DMI and the potential terms are the ones 
that do not have the derivatives in the Euler equation and the major 
difference between the DMI and the potential terms are the dimensions; 
the potential is multiplied by $r$. 
The compacton radius $R$ is determined in terms of the behavior of 
the solutions at the large $r$ and apparently, the potential is 
dominated than the DMI. That is the reason why the compactons are 
supported via mainly the potentials, not the DMI.
From another angle, we easy to confirm the situation by the series expansion~(\ref{expansion}).
The condition of the leading order is
\begin{align}
&\kappa_1 A_s^2(R-r)^{2s}
+6\kappa_{4b}rA_s^3s^3(s-1)(R-r)^{3s-4}
\nonumber \\
&-\frac{\kappa_{0a}}{2}rA_s(R-r)^s=0\,.
\label{}
\end{align}
Therefore, the DMI term does not contribute to the lowest order terms, and then, 
the condition coincides one without the DMI term (\ref{exeq2}). 

In the next section, we shall examine the new model where the DMI plays an important role for compactons.
 


We consider the effect of the DMI and the Skyrme term concerning the stability (the existence) 
of the solutions for the point of view of the Derrick's argument. 
We examine the value $-E_1/(2E_0)$ for several parameters strength for the models: 
$[0,\kappa_1, \kappa_{4a},0,\kappa_{0a},0]$ and $[0,\kappa_1, \kappa_{4a},0,0,\kappa_{0b}]$ 
If the solutions are obtained by the DMI, it goes to 1, while by the Skyrme term it closes to 0.  
Fig.\ref{fig:sk-dm-ze-derrick} is the result for the Zeeman and the squared Zeeman potential, respectively. 
They are reasonable results: 
for $\kappa_{4a}\to 0$, $-E_1/(2E_0)$ approaches $1$
and for $\kappa_1 \to 0$, $-E_1/(2E_0)$ approaches $0$.
Fig.\ref{fig:sk-dm-ze-derrick} shows there are no solutions in two regions: 
\begin{itemize}
\item[(i)] at $\kappa_{4a} \to 0$ for all $\kappa_1$, 
\item[(ii)] the lower right: $\kappa_1 \to 1$ with the small $\kappa_4$. 
\end{itemize}
(i) Without the Skyrme term, the restricted model of the Zeeman potential 
only has a half-skyrmion and no soliton solution.
(ii) As increasing $\kappa_{1}$, the model substantially moves to case (i). 
So, the blank grows as $\kappa_1$ increases.

In Fig.\ref{fig:sk-dm-ze-derrick}(right), the solution exhibits no such limiting behavior. 


%%%%%%%%%%%%%%%%%%%%%%%%%%%%%%%%%%%%%%%%%%%%%%%%%%%%%%%%%%%%%%%%%%%%%%%%%%%%Fig9
\begin{figure*}[t]
  \begin{minipage}[b]{0.5\linewidth}
    \centering
    \includegraphics[keepaspectratio,scale=1.2,bb=0 0 193 137]{fig9_left.pdf}
  \end{minipage}\hspace{-0.5cm}
  \begin{minipage}[b]{0.5\linewidth}
    \centering
    \includegraphics[keepaspectratio,scale=1.2,bb=0 0 193 137]{fig9_right.pdf}
  \end{minipage}
  \caption{The skyrmions without potential~$[\kappa_2, 1.0, 1.0 ,0 ,0 ,0]$.  
The profile function $f(r)$ (left) and the energy density $\varepsilon (r)$ (right). 
The restricted model: $\kappa_2=0.0$ is the compacton solution (the blue line).}
  \label{fig:kin-dm-sk}
\end{figure*}
%%%%%%%%%%%%%%%%%%%%%%%%%%%%%%%%%%%%%%%%%%%%%%%%%%%%%%%%%%%%%%%%%%%%%%%%%%%%Fig9

%%%%%%%%%%%%%%%%%%%%%%%%%%%%%%%%%%%%%%%%%%%%%%%%%%%%%%%%%%%%%%%%%%%%%%Fig10
\begin{figure}[t]
    \centering
    \includegraphics[keepaspectratio,scale=1.2,bb=0 0 193 137]{fig10.pdf}
  \caption{We plot the compactons of~(\ref{restrictdm-sk}). 
the profile function and the derivatives $f(r),f'(r)$~(the blue, the red lines), 
which clearly shows that the derivatives are continuous at the boundary $r=R=5.231, 4.168,2.431~(\bar{\kappa}=0.5, 1.0, 5.0)$.}
  \label{fig:mdm-sk_dfdr}
\end{figure}
%%%%%%%%%%%%%%%%%%%%%%%%%%%%%%%%%%%%%%%%%%%%%%%%%%%%%%%%%%%%%%%%%%%%%%Fig10

%%%%%%%%%%%%%%%%%%%%%%%%%%%%%%%%%%%%%%%%%%%%%%%%%%%%%%%%%%%%%%%%%%%
\section{\label{sec:5}The solutions of the models without the potential terms}
%%%%%%%%%%%%%%%%%%%%%%%%%%%%%%%%%%%%%%%%%%%%%%%%%%%%%%%%%%%%%%%%%%%

In this paper, we have studied `the normal' models which always possess potential terms. 
The kinetic term has no role in the Derrick's theorem, and the 
4th order terms of the baby-Skyrme model, and the DMI term of the 
magnetic Skyrme model together with the potential terms are 
responsible for the existence of the soliton solutions.  

According to \cite{Ashcroft:2015jwa}, there is a new 
type of baby-Skyrme model without any potential term. 
The model is composed of the kinetic and the Skyrme 
terms with the integer or fractional power of $\alpha,\beta$. 
The range of these parameters is examined so as to ensure stability 
with respect to rescaling.  

We propose a model that comprises the Skyrme and the DMI without potential. 
The energy applying the spatial rescaling $x\mapsto \mu x$ can be written as
\begin{align}
e(\mu)=E_2+\mu^{-1}E_1+\mu^2E_{4a}\,.
\end{align}
There no stational point with $e(\mu)$ because $E_{4a}>0, E_1<0$ for  $\gamma=\pi/2$. 
However, when we set $\gamma=-\pi/2$,  it can take extremum at
\begin{align}
\mu=\sqrt[3]{\frac{E_1}{2E_{4a}}},~~E_1,E_{4a}>0\,,
\end{align}
then we may have a soliton solution to the model. 

We consider the model with $[\kappa_2, \kappa_1, \kappa_{4a} ,0 ,0 ,0]$. 
We present the results in Fig.\ref{fig:kin-dm-sk}. 
As increasing in $\kappa_2$, the tail of the solution extends and the maximum of the energy density enhances at the origin, 
because the gradient of the solution grows. 
In the case of $\kappa_2=0$, the solution becomes the compacton. 
For the restricted model ($\kappa_2=0$), the Euler equation is the following simple one-parameter equation
\begin{align}
    2rf''-2f'+2r\cot f f'^2 -\bar{\kappa}r^2=0
\label{restrictdm-sk}
\end{align}
where $\bar{\kappa}:=\kappa_1/\kappa_{4a}$. 
Fig.\ref{fig:mdm-sk_dfdr} plots the $f(r),f'(r),f''(r)$ for several $\bar{\kappa}$. 
The $f,f'$ simultaneously become zero at $r=R$, where the $f''(R)$ remains to be finite, 
which is likely the genuine-compacton. Analytically, we can check it 
by substituting $f(R)=0$ intto (\ref{Eulereq}), we obatain $f'(R)=0$. 
Of course, it does not mean that there is a compacton solution in the model. 
But,  we can say if there is a compacton, it should be the genuine-compacton. 
As increasing $\bar{\kappa}$, the compacton radius $R$ moves to the origin. 

In this model, there is symmetry with respect to the inversion of the coefficient. 
Our model is (a)~$\gamma=-\pi/2, \kappa_{4a}>0$. The model
(b)~$\gamma=\pi/2, \kappa_{4a}<0,$ becomes the same equation, 
where the energy density reverses the sign.
Here, we would like to consider the symmetry really exist or not. 
For that, we add the kinetic term to the model, and (a)~has the solution but (b) is not. 
The result of the heat-kernel expansion~(\ref{eq:edens}) tell us that, 
the kinetic term, the DMI term, and the Skyrme term are the same sign. 
Also, it is straightforward to verify that the model with $\kappa_2>0, \kappa_4<0$ is always unstable 
in the quantum stability analysis. Therefore, the above symmetry does not exist 
and just an artifact for the restricted model. 

In terms of the series expansion at the compacton boundary (\ref{expansion}), we have the condition
for the lowest order
\begin{align}
    \frac{2}{r}A_s^3 s (2s-1) (R-r)^{3s-2}-\bar{\kappa} A_s^2 (R-r)^{2s}=0\,,
\end{align}
which has the solution $s=2$. 
This implies that there is a standard parabolic approach to the vacuum for the genuine-compacton.





%%%%%%%%%%%%%%%%%%%%
\section{Summary}
%%%%%%%%%%%%%%%%%%%%

In this paper, we have studied a generalization of the baby-Skyrme model with the inclusion 
of the Dzyaloshinsikii-Moriya interaction (DMI). The model has been derived from 
the vacuum functional of fermions coupled with $O(3)$ nonlinear $\bm{n}$-fields and 
with a constant $SU(2)$ gauge background. Integrating out the fermionic fields, we obtained
the effective action defined by the fermion determinant. 
The heat-kernel expansion for the determinant, we obtained the baby-Skyrme type model with the 
DMI and the two potential terms. 

In terms of the circular symmetric ansatz for $\bm{n}$-fields, we have obtained several normal
soliton solutions. For the restricted model where the kinetic term is omitted, 
the compact skyrmions are obtained. The compactons are the solutions with 
finite radius and there are two types of solutions: the weak-compacton and the gunuine-compacton. 
These compactons are defined by a number of the differentiability at the boundary.  
The weak-compacton is not continuously differentiable and the genuine-compacton is 
one time differentiable. 
These are successfully obtained in terms of the choice of the 4th order terms. 
The DMI has less effect for constructing compacton in this restricted model 
because the potential terms tend to dominate in the vicinity of the compacton radius. 
We proposed a new type of model for compactons without potential terms, which comprises 
just the Skyrme term and the DMI term with opposite chirality. The solution becomes the 
genuine-compactons.

This paper constitutes an initial step for the construction of soliton solutions for our new model. 
The following problems have to be solved in order:
\begin{itemize}
\item All our results were on the circular symmetric ansatz and 
it would certainly be interesting to lift this symmetry. The higher topological charge,  
non-circular solutions certainly exist in the model. 
\item Since the magnetic skyrmion often forms various platonic lattice structures, 
so in our model, some novel structure might appear by the conjunction or 
competition between the DMI and the 4th order terms.   
\item The fermionic vacuum functional has its own soliton solutions for the model 
where the energy density comprises the sum of the valence fermions and an 
infinite sum of the Dirac sea fermions. The well-known Atiyah-Patodi-Singer 
index theorem implies the existence of such soliton solutions. The analysis is a 
little complicated but the results are certainly interesting. 
\end{itemize}

We shall report on these issues in separate papers.





\vskip 0.5cm\noindent
\begin{center}
{\bf Acknowledgment}
\end{center}

The authors would like to thank Pawe\l~Klimas
for his careful reading of this manuscript and many useful advices. 
We also appreciate Yuki Amari for helpful advice regarding the constant gauge field~(\ref{DMgauge}).
We also thank Atsushi Nakamula, Yakov Shnir, Kouichi Toda, Shota Yanai for fruitful discussions and comments. 
N.S. is supported in part by JSPS KAKENHI Grant Number JP20K03278.




\bibliography{restskyrme}

\end{document}


