\documentclass[12pt,pdflatex]{article}
%\documentclass[aps,prd,preprint,amsmath,amssymb,showpacs,showkeys]{revtex4-1}% APS journal style
\usepackage{graphicx}
\usepackage{subfigure}
\usepackage{dcolumn}% Align table columns on decimal point
\usepackage{hyperref}% add hypertext capabilities
%\usepackage[retainorgcmds]{IEEEtrantools}
\usepackage[T1]{fontenc}
\usepackage[T1]{fontenc}
\usepackage[utf8]{inputenc} %
%\usepackage[symbol]{footmisc}
\usepackage[english]{babel}
\usepackage{amsmath,bm}
\usepackage{amstext,amsthm,amscd}
\usepackage{amsfonts,txfonts}
\usepackage{amssymb}
\usepackage{multirow, array}
\usepackage{mathrsfs}
\hyphenation{Feyn-Arts process--indepen-dent}
\usepackage[bottom]{footmisc}% places footnotes at page bottom
\usepackage[affil-it]{authblk}
%\usepackage[english]{babel}
\usepackage[left=2cm,right=2cm,top=2cm,bottom=2cm]{geometry}
\usepackage[usenames]{color}
\newcommand{\cor}[1]{{\color{red} #1}}
\newcommand{\cob}[1]{{\color{blue} #1}}
\newcommand{\cog}[1]{{\color{green} #1}}
\newcommand{\cov}[1]{{\color{violet} #1}}
%\usepackage{feynmf}
% To use eps figures
\usepackage{epstopdf}
\epstopdfsetup{update}
\newcommand{\Lagr}{\mathcal{L}}


%%%% writing comments
\definecolor{MHcolor}{rgb}{.5,.5, 1}
\newcommand*{\MH}[1]{{\color{MHcolor}#1}}
%%%%%%%%%%%

\title{Top quark production through flavor violating neutral currents in $pp$ and $lp$ collisions within the 2HDM type-III }
%THDM-III FCNC quark top production via DIS}
%\tilte{Opportunities for Indirect Searches of Flavor Violation Higgs in Future DIS}

\author[1]{M. Gómez-Bock\footnote{melina.gomez@upaep.mx}}

\author[2]{W. González-Olivares\footnote{w\_gonzalez\_olivares@hotmail.com}}
%\affiliation{IFUAP}
%\email{rosado@ifuap.buap.mx}
\author[3]{M. Hentschinski\footnote{martin.hentschinski@udlap.mx}}
%\affiliation{Depto. de Actuar\'ia, F\'isica y Matem\'aticas, Universidad de las Américas Puebla, Ex. hda. Sta. Catarina m\'artir s/n San Andr\'es Cholula, C.P.72810, Pue., México}
%\email{martin.hentschinski@udlap.mx}
\author[2]{A. Rosado\footnote{rosado@ifuap.buap.mx}}
%\affiliation{IFUAP}
%\email{rosado@ifuap.buap.mx}
\author[4]{S. Rosado-Navarro\footnote{sebastian.rosado@proton.mail}}
%\email{}
%\affiliation{}
\affil[1]{\'Area de Matem\'aticas, Universidad Popular Aut\'onoma del Estado de Puebla, 
%21 sur 1103, Barrio de Santiago C.P. 72410, 
Puebla,Pue., México}
\affil[2]{Instituto de Física de la Benemérita Universidad Autónoma de Puebla, Puebla, Pue. México.}
\affil[3]{Dpto. de Actuar\'ia, F\'isica y Matem\'aticas, Universidad de las Américas Puebla, San Andr\'es Cholula
%Ex. hda. Sta. Catarina m\'artir s/n San Andr\'es Cholula, C.P.72810, 
Pue., México}
\affil[4]{Centro Interdisciplinario de Investigación y Ense\~nanza de la Ciencia, Benemérita Universidad Autónoma de Puebla, Puebla, Pue. México.}
\begin{document}
\maketitle
%\tableofcontents
%$\mathscr{ABCDEFGHIJKLMNOPQRSTUVWXYZ}$

\begin{abstract}

We explore flavor changing neutral currents which originate from the flavor violating Two Higgs Doublet Model for both LHC and future lepton-hadron colliders.

Processes of Flavor Changing Neutral Currents are present at tree
  level if dynamics beyond the Standard Model is included.  To examine the scenarios that could exhibit
  such events, we perform a search in the parameter space to
  determine the possible same sign top pair production at the $pp$ LHC and also  single top quark production in $e p$ Deep
  Inelastic Scattering at a future LHeC, for energies
  given by current and future colliders. We show the order of
  magnitude for the model parameter
  $\vert\tilde{\chi}_{ij}^{f}\vert$ that would allow for the
  observation of flavor violation through the scalar sector. In addition we explore the order of magnitude of possible single top production via $\mu p$, at the proposed muon-proton colliders. 
  We found scattering values for these exotic processes of  $\mathcal{O}(10^{-2})pb$.
  \end{abstract}
%%% also needs our analysis of LHC data etc\end{abstract}


%\pacs{}
%\keywords{}
\maketitle
\section{Introduction}
%The Standard Model (SM) is based on symmetries, ...
The original phenomenological structure of the Standard Model of Elementary Particle Physics  is based on the use of conserved and discrete quantum numbers which determine the allowed dynamical processes.  They exhibit a set of restricted Flavor Changing Charged Currents (FCCCs) and suppressed Flavor Changing Neutral Currents (FCNCs).  
This structure helps to identify possible processes to be observed experimentally and reduces furthermore the structure of the model.
Nevertheless, {\it Flavour Physics} has shown a much more richer phenomenology than  the one predicted by the Standard Model (SM); in particular evidence for neutrino mixing  \cite{heeger2005evidence} and possibly non-universality of weak leptonic meson decay \cite{falkowski2015lepton}. Moreover, the recent results on the  mass of the $W$ boson may suggest beyond the SM weak structure \cite{CDF:2022hxs}.  These observations could in principle affect the original flavor structure of the SM, which has been constructed as an effective theory from the available low energy experimental data. They therefore seem to indicate the limits of the SM 
 and hint at the need to extend the flavor structure of the SM.\\

%We now know that more complex flavour structure is present in neutral particles with the evidence on neutrino mixing \cite{}. 
%These recent experimental results rise the question whether could it be there a more enriched flavour structure in the quark and/or in the charged leptonic sectors, through the scalar sector. The current results suggests that we should at least inquire and explore the possibility as beyond SM structures for NP.\\

Within the Standard Model (SM), quarks acquire their mass through
spontaneous symmetry breaking \cite{Wells:2009kq}, while the mechanism cannot explain
the mass spectrum and the resulting mass hierarchy.  While the coefficients are
directly related to the mass values, the Yukawa couplings are only
determined experimentally, see \cite{Sirunyan:2018lzm} for a recent analysis. In order to account for this enriched flavor structure and to explain it, it is needed to  extend the SM.  
A minimal extension of the SM which allows for flavor changing neutral currents at the level of the Lagrangian is provided by the Two Doublet Higgs Model with flavour violation at tree level (THDM-III), which we will discuss in more detail in the next section. \\  %, should be tasted with different FV processes in the current and future experimental data. \\

 In order to know with certainty if we are in extended model domains,  an excellent candidate to study such processes is  given by events which are associated with the production of a top quark, since its mass  is  the largest of the known elementary particles.  Top processes within the 2HDM-II and III models have been explored and compared in \cite{Arhrib:2016tni}, where a bound for flavor violating top process has been established as $BR(t\to c, h(H))<10^{-3}$. A so far unexplored channel in this context is the production of a pair of top or anti-top quarks at the LHC \cite{single}, which allows to constrain a potential transition of a charm quark into a top quark via a t-channel exchange of a flavor violating Higgs boson. 
Single top production at an electron-proton collider would on the other hand serve to  constrain the coupling of a flavor violating Higgs to the leptonic sector, which is hard to control at an hadron-hadron collider.  A first bound was already established at the HERA experiments, where $\sigma(ep\to etX)< 0.25pb$  for an energy of$\sqrt{s}=319$~GeV,  \cite{H1:2009yuy}. We could expect an improvement of this bound at an future LHeC \cite{Gao:2021plf}. 
 %$\sigma_{t+\bar{t}}=245 \, pb $ with proton energy of $7 \,TeV$ and electron energy $60GeV$.} 
A muon-proton collider, as proposed in \cite{Acosta:2022ejc},  would on the other hand give access to the muon-to-tau transition through $t$-channel exchange of a flavor violating Higgs boson and in this way further constrain the content of the model. \\

%In particular, the THDM-III we found bounds on parameters 
%$\tan\beta$ and $\cos(\alpha-\beta)$.}
The paper is organized as follows. In section 2, we present the  FCNC stemming from Yukawa interactions within the 2HDM-III.
In section 3, we calculate the cross sections for  double top quark production at the LHC and single top quark production at $ep$ collider, via Deep Inelastic Scattering (DIS),  through FCNC in the context of the THDM-III, using the Parton Model \cite{} with five scheme PDFs \cite{}. In section 4, we perform numerical calculation for the processes giving constraints on the parameters of the model due to experimental bounds. Finally, we summarize our conclusions.


%%%% comment Martin Jan 17, 23: https://cds.cern.ch/record/2689091/files/ATL-PHYS-PROC-2019-090.pdf has a SM preduction sigma(tt tbar tbar) 9.2fb at NLO
%where 1206.3064 [hep-ph] is cited

%%%%%%%%%%%%%%%%%%%%%%%%%%%%%%%%%%%%%%%%%%%%%%%%%%%%%%%%%%%%%%%%%%%%%%%%%%%%%%%%%%%%%%%%%

\section{Flavor violating processes within the 2HDM-III model}
Flavor violating processes have been widely studied in the literature, in particular for the so called exotic decays  via a neutral Higgs boson, using an effective Lagrangian framework \cite{Goldouzian:2014nha}, through higher dimension operators in the context of THDM \cite{Buschmann:2016uzg}, and directly from the Lagrangian in the specific scenario of THDM-III  \cite{Das:2015kea, Arhrib:2016tni}.
Extensions on the Higgs mechanism with more than one Higgs doublet are still widely explored in the literature, {\it e.g.} \cite{Bento:2017eti}.
%Extensions on the scalar sector has been and still is widely explored, performing  \cite{Bento:2017eti}. Gracias, si
Adding another Higgs doublet to the Standard Model is the simplest extension possible of the SM Higgs sector. It leads to the Two Higgs Doublets Models (THDM), and is also the lowest possible structure for the scalar sector  in the context of Supersymmetry. THDM models differ among each other,
%The difference in models type will be
based on a discrete symmetry added to the model in order to avoid FCNC, as the masses of the fermions originate from two doublets in contrast to one doublet in the Standard Model. To establish the precise model one uses direct search of new scalar particles, but also  indirect searches such as rare processes of flavor violation. \\
%In order to arrive at a better understanding of possible new physics scenarios, we therefore need to consider couplings is part of the model.

The actual possibility of having FCNC has been discussed in the literature. The initially studied versions of the Higgs doublet model completely excluded this possibility as in 2HDM type I and II \cite{arhrib2016two}, the specific 2HDM with FV tree level structure was studied in \cite{Diaz-Cruz:2004wsi,melina}, in the so-called THDM-III. In this article we claim that while those flavor mixing couplings must be small, as is reported in other processes \cite{das2016flavor}, they do not need to be zero, as in the SM.\\

In the following we review the relevant elements of the 2HDM-III model for our study. 
The 2HDM-III Lagrangian of the Yukawa sector has the form
\begin{equation}
L_{Y}=\sum_{a,i}Y_{a}^{i} \overline{F}_{L}^{i}\Phi_a
f_{R}^{i}+h.c.,
\end{equation}
where $F_{L}$ denotes the fermion doublet (left-handed), $f_{R}$ is the fermion singlet (right-handed), and $\Phi_a$ are the Higgs doublets $(a=1,2)$. Considering three generations, the coefficient $Y_{a}^{i}$ can be expressed as a $3\times3$ matrix and tagged as $Y_{a}^{l}, Y_{a}^{u}, Y_{a}^{d}$, for leptons, $u$ and $d$ type quarks respectively, while we take neutrinos to be massless.  
For type III models, the two Higgs doublets  couple to  two quarks with different isopin \cite{DiazCruz:2004tr}.  This is a particularly interesting  feature of this model, since it allows to explain in a more natural way the mass hierarchy of the Standard Model fermions.
In the 2HDM-III we have three types of free parameters that must be determined from experiment, \emph{i.e.} scalar masses, Yukawa couplings and the ratio of vacuum expectation values. Then we have the following Yukawa couplings to fermions:

\begin{equation}
\begin{split}
L_{Y}^{q} &=Y_{1}^{u}\overline{Q}^{\prime}_{L}\tilde{\Phi}_1 u_{R}^{\prime}+Y_{2}^{u}\overline{Q}_{L}^{\prime}\tilde{\Phi}_2 u_{R}^{\prime}+Y_{1}^{d}\overline{Q}_{L}^{\prime}\Phi_1 d_{R}^{\prime}+Y_{2}^{d}\overline{Q}_{L}^{\prime}\Phi_2 d_{R}^{\prime}+h.c.,\\
%L_{Y}^{l} &=Y_{1}^{l}\overline{L}_{L}^{\prime}\Phi_1 l_{R}^{\prime}+Y_{2}^{l}\overline{L}_{L}^{\prime}\Phi_2 l_{R}^{\prime}+h.c.,
\end{split}
\end{equation}
where $\tilde{\Phi}_{1,2}=i\sigma_2\Phi^{*}_{1,2}$  and $\sigma_2$ is the Pauli matrix. The charged leptonic sector has a   similar form to the one of
%sthe ame as 
the $d-type$ quark, and is obtained from the latter by replacing
$d_{i}\rightarrow\,l_{i}$, including the masses.
After spontaneous symmetry breaking, each of the two doublets acquire vacuum expectation values (vevs),  $v_{1,2}$. Expanding the scalar complex fields over this vevs, one obtains:
\begin{align}
\Phi_1&=\begin{pmatrix}
\phi^{+}_{1} \\
\frac{v_1}{\sqrt{2}}+\phi^{0}_{1}
\end{pmatrix}
; &
\Phi_2&=\begin{pmatrix}
\phi^{+}_{2} \\
\frac{v_2}{\sqrt{2}}+\phi^{0}_{2}
\end{pmatrix},
\end{align}
which yields the following form of the mass matrices:\\
%Yukawa couplings 
%\begin{eqnarray}
%\begin{split}
%L^{q}_{y}&=&\bar{u}_{Li}\left[\frac{v_2}{\sqrt{2}}Y^{u}_{2ij}+\frac{v_1}{\sqrt{2}}Y^{u}_{1ij}\right]u_{Rj}+\bar{d}_{Li}\left[\frac{v_2}{\sqrt{2}}Y^{d}_{2ij}+\frac{v_1}{\sqrt{2}}Y^{d}_{1ij}\right]d_{Rj}\notag
%\\
%&&+\bar{u}_{Li}\left[\phi^{0*}_{2}Y^{u}_{2ij}+\phi^{0*}_{1}Y^{u}_{1ij}\right]u_{Rj}+\bar{d}_{Li}\left[\phi^{0}_{2}Y^{d}_{2ij}+\phi^{0}_{1}Y^{d}_{1ij}\right]d_{Rj}\notag
%\\
%&&+
%\underbrace{
%\bar{u}_{Li}\left[\phi^{+}_{2}Y^{u}_{2ij}+\phi^{+}_{1}Y^{u}_{1ij}\right]u_{Rj}+\bar{d}_{Li}\left[\phi^{-}_{2}Y^{d}_{2ij}+\phi^{-}_{1}Y^{d}_{1ij}\right]d_{Rj},
%}%_{terminos\,\, cargados}
%\end{split}
%\label{Yukawa}
%\end{eqnarray}
%with
\begin{equation}
M_f=\frac{1}{\sqrt{2}}(v_{1}Y_{1}^{f}+v_{2}.
Y_{2}^{f}),\hspace{0.5cm} f=u,d,l.
\end{equation}
In the physical basis, $M_f$ is diagonal but not necessary are each of the two Yukawa matrices. In order to diagonalize  analytically, we reduce the possible $3\times3$ flavor fermion mass matrices by a proposed {\it ansatz} with a hierarchical structure, which is  based on a textures form (zero for some flavor mixing elements guided by experimental data), as was first studied in \cite{antaramian1992flavor,sher1991rare}. For the two hermitian
Yukawa matrices, we  consider a more general flavor structure with a 4-texture, considering both types of quarks. Following \cite{DiazCruz:2004tr}, we have 
\begin{eqnarray}
M_f=
\begin{pmatrix}
0 & C_l &0 \\
C^*_l & \tilde{B}_l & B_l \\
0 & B^*_l & A_l \\
\end{pmatrix} ; &
Y_1=
\begin{pmatrix}
0 & C_1 &0 \\
C^*_1 & \tilde{B}_1 & B_1 \\
0 & B^*_1 & A_1 \\
\end{pmatrix} ; &
Y_2=
\begin{pmatrix}
0 & C_2 &0 \\
C^*_2 & \tilde{B}_2 & B_2 \\
0 & B^*_2 & A_2 \\
\end{pmatrix} ,
\label{textures}
\end{eqnarray}
where the elements of the matrix should be taken with a hierarchy imposed by the measured fermion masses,  $|A_l| \gg |\tilde{B}_l|, |B_l|, |C_l|$, which
can be diagonalized through
\begin{eqnarray}
 \bar{M}^{diag}_{u}=V^{u}_{L}M_{u}V^{u \dagger}_{R}&\,\,\;
 \bar{M}^{diag}_{d}=V^{d}_{L}M_{d}V^{d \dagger}_{R}&\,\,\;
 \bar{M}^{diag}_{l}=O^{l}_{L}M_{l}O^{l \dagger}_{R} .
\end{eqnarray}
%where $V$ and $O$ have the same structure but different values 
Here, as usual, $V_{CKM}=V^{u}_{L}V^{d\dagger}_{L}$, and 
\begin{eqnarray}
\tilde{Y}^{q}_{1,2}=V^{q}_{L}Y^{q}_{1,2}V^{q \dagger}_{R}\; \text {and}  &
\tilde{Y}^{l}_{1,2}=O^{l}_{L}Y^{l}_{1,2}O^{l \dagger}_{R}
%\tilde{Y}^{d}_{1,2}=V^{d}_{L}Y^{d}_{1,2}V^{d^{\dagger}}_{R}
\end{eqnarray}
yields the CKM matrix 
and $q=u,b$.
We further note the following relation between the two Yukawa matrices:
\begin{eqnarray}
\displaystyle
\tilde{Y}_{1}^{d}=
\frac{\sqrt{2}}{v\cos\beta}\bar{M}_{d}-\tan\beta\tilde{Y}_{2}^{d} & \text{ and } &
\tilde{Y}_{1}^{l}=\frac{\sqrt{2}}{v\cos\beta}\bar{M}_{l}-\tan\beta\tilde{Y}_{2}^{l} \\
\tilde{Y}_{2}^{u}=\frac{\sqrt{2}}{v\sin\beta}\bar{M}_{u}-\cot\beta\tilde{Y}_{1}^{u} & &
\label{rotyukawas}
\end{eqnarray}

%\subsection{Neutral Higgs Yukawa Lagrangian}
Having an extra Higgs  scalar doublet in the THDM, 
%(\MH{we introduce})
%the mass basis for neutral Higgs bosons, 
as it is known \cite{Gunion:1984yn,Gomez-Bock:2007azi}, $\alpha$ is the rotation angle for CP-even physical neutral Higgs bosons $h^0$ and $H^0$ states, and $\beta$ is the angle associated with the Goldstone states basis,  $\tan \beta= v_2/v_1$. In the physical basis, omitting Goldstone contributions, we have for the fermions couplings with neutral scalars:
\begin{eqnarray}
{\cal{L}}_Y^{q} & = &\frac{g}{2}
\left\{ \bar{u}_{i}
\left[\left(\frac{m_{u_{i}}}{m_W}\right)\frac{\cos\alpha}{\sin\beta} \delta_{ij}- \frac{\sqrt{2} \,
\cos(\alpha - \beta)}{g \, \sin\beta}
(\tilde{Y}_1^u)_{ij}\right]u_{j}h^{0} \right.
\nonumber \\
&+& \left.\bar{d}_{i}
\left[-\left(\frac{m_{d_{i}}}{m_W}\right)\frac{\sin\alpha}{\cos\beta} \delta_{ij}+ \frac{\sqrt{2} \,
\cos(\alpha - \beta)}{g \, \cos\beta}(\tilde{Y}_2^d)_{ij}\right]d_{j}
h^{0}\right. \nonumber \\
& & \left. + \bar{u}_{i}\left[\left(\frac{m_{u{i}}}{m_W}\right)\frac{ \, \sin\alpha}{\sin\beta}\delta_{ij}-\frac{\sqrt{2} \, \sin(\alpha - \beta)}{g \, \sin\beta}(\tilde{Y}_1^u)_{ij}\right]u_{j}H^{0}
\right.
\nonumber \\
&+& \left.\bar{d_{i}}\left[\left(\frac{m_{d_{i}}}{m_W}\right)\frac{ \, \cos\alpha}{\cos\beta}\delta_{ij}+
\frac{\sqrt{2} \, \sin(\alpha - \beta)}{g \, \cos\beta}(\tilde{Y}_2^d)_{ij}\right]d_{j}H^{0}\right. \nonumber \\
%&& \quad+ 
& &\left. 
%+ \bar{u}_{i}
%\left[\left(\frac{m_{u_{i}}}{m_W}\right)\frac{\cos\alpha}{\sin\beta} \delta_{ij}- \frac{\sqrt{2} \,
%\cos(\alpha - \beta)}{g \, \sin\beta}(\tilde{Y}_1^u)_{ij}\right]u_{j}h^{0}
+i\bar{u}_{i}\left[-\left(\frac{m_{u_{i}}}{m_W}\right)\cot\beta \delta_{ij} + \frac{\sqrt{2} }{g \, \sin\beta}
(\tilde{Y}_1^u)_{ij}\right]\gamma^{5}u_{j} A^{0}\right. 
\nonumber \\
&+& \left.i\bar{d}_{i}\left[-\left(\frac{m_{d_{i}}}{m_W}\right)\tan\beta \delta_{ij}+  \frac{\sqrt{2} }{g \, \cos\beta}(\tilde{Y}_2^d)_{ij}\right]
\gamma^{5}d_{j} A^{0} \right\}.
\label{lageigenstates}
\end{eqnarray}
%\marginpar{ Eq, (8) añadir parte leptónica}
%\cor{DAR  referencias al DESARROLLO. PASAR $Y^u_2$ A $Y^u_2$  
%Question: is eq.8 the same as eq. 3 (different symbol); I think so, but not sure about it. YES\\
%We probably should explain/mention the origin
%of $\alpha, \beta$, since later on we make certain choices for these parameters and our predictions depend a lot on these choices. SURE}
%\subsection{Taken from somewhere down}
%The quarks couplings with the neutral Higgs bosons ~\cite{melina} can be obtained from the the following Lagrangian:  
%\begin{equation}
%\label{eq:2}
%\begin{split}
%\mathcal{L}^{q}_{Y}= &\frac{g}{2}\left(\frac{m_{d_{i}}}{M_{W}}\right)\bar{d}_{i}\left[\frac{\cos{\alpha}}{\cos{\beta}}\delta_{ij}+\frac{\sqrt{2}\sin({\alpha\,-\beta})}{g\cos{\beta}}\left(\frac{m_{W}}{m_{d_{i}}}\right)(\tilde{Y}^{d}_{2})_{ij}\right]d_{j}H^{0}\\
%& +\frac{g}{2}\left(\frac{m_{d_{i}}}{M_{W}}\right)\bar{d}_{i}\left[-\frac{\sin{\alpha}}{\cos{\beta}}\delta_{ij}+\frac{\sqrt{2}\cos({\alpha\,-\beta})}{g\cos{\beta}}\left(\frac{m_{W}}{m_{d_{i}}}\right)(\tilde{Y}^{d}_{2})_{ij}\right]d_{j}h^{0}\\
%& +i\frac{g}{2}\left(\frac{m_{d_{i}}}{M_{W}}\right)\bar{u}_{i}\left[-\tan{\beta}\delta_{ij}+\frac{\sqrt{2}}{g\cos{\beta}}\left(\frac{m_{W}}{m_{d_{i}}}\right)(\tilde{Y}^{d}_{2})_{ij}\,\right]\gamma^{5}d_{j}A^{0}\\
%&+\frac{g}{2}\left(\frac{m_{u_{i}}}{M_{W}}\right)\bar{u}_{i}\left[\frac{\sin{\alpha}}{\sin{\beta}}\delta_{ij}+\frac{\sqrt{2}\sin({\alpha\,-\beta})}{g\sin{\beta}}\left(\frac{m_{W}}{m_{u_{i}}}\right)(\tilde{Y}^{u}_{1})_{ij}\right]u_{j}H^{0}\\
%&+\frac{g}{2}\left(\frac{m_{u_{i}}}{M_{W}}\right)\bar{u}_{i}\left[\frac{\cos{\alpha}}{\sin{\beta}}\delta_{ij}+\frac{\sqrt{2}\cos({\alpha\,-\beta})}{g\sin{\beta}}\left(\frac{m_{W}}{m_{u_{i}}}\right)(\tilde{Y}^{u}_{1})_{ij}\right]u_{j}h^{0}\\
%& +i\frac{g}{2}\left(\frac{m_{u_{i}}}{M_{W}}\right)\bar{u}_{i}\left[-\cot{\beta}\delta_{ij}+\frac{\sqrt{2}}{g\sin{\beta}}\left(\frac{m_{W}}{m_{u_{i}}}\right)(\tilde{Y}^{d}_{2})_{ij}\,\right]\gamma^{5}u_{j}A^{0}\\
%\end{split}
%\end{equation}
The leptonic part is obtained by replacing
$d_{i}\rightarrow\,l_{i}$. We see that for $h^0$ to be the SM-like Higgs, one needs to impose  $\alpha -\beta = \pi /2$ as the decoupling limit \cite{Gunion_2003}.

Using the Cheng-Sher ansatz to reproduce
the mass hierarchy \cite{cheng1987mass}, the Yukawa couplings from the previous Lagrangian
can be described in terms of dimensionless experimental parameters
$\tilde{\chi}_{ij}$ which could have a complex phase, as the matrices in Eq.(\ref{textures}) are Hermitian. In particular negative  $\tilde{\chi}_{ij}$ is possible. Then, the Yukawa matrix elements are 
\begin{equation}
\label{eq:1}
\begin{split}
\left(\tilde{Y}_{2}^{d,l}\right)_{ij} &=\frac{\sqrt{m_{i}^{d,l}m_{j}^{d,l}}}{v}\tilde{\chi}_{ij}^{d,l}\\
\left(\tilde{Y}_{1}^{u,\nu_l}\right)_{ij} &=\frac{\sqrt{m_{i}^{u,\nu_l}m_{j}^{u,\nu_l}}}{v}\tilde{\chi}_{ij}^{u,\nu_l}.
\end{split}
\end{equation}
Similar to  Eq.~\eqref{eq:1},  a large number of proposals to achieve specific fermion mass matrices are possible,  see for instance reference~\cite{3}. The four zero texture matrix fits however quite well with the quark mixing data. It is worth to point out that, as a consequence of regarding $\tilde{\chi}_{ij}^{f}$ as experimental parameters, we will be  able to define a range where it would be feasible to measure this  FCNC processes. 
For type III models  the two doublets are furthermore coupled to the two types of quarks (up and down) \cite{DiazCruz:2004tr}.  These features of the model would explain in a more natural way the mass hierarchy of the SM fermions.\\
%\cob{According to Eqs.~\eqref{eq:2} and~\eqref{eq:1}, the vertex factor at the leptonic line reads:
%\begin{equation}
%\mathcal{L}_{\bar{l}lh^0}=\frac{g}{2}\,\bar{l}_{i}\left[-\left(\dfrac{m_{l_{i}}}{m_{W}}\right)\dfrac{\sin{\alpha}}{\cos{\beta}}\delta_{ij}+\dfrac{\cos{(\alpha\,-\beta)}}{\sqrt{2}\cos{\beta}}\left(\dfrac{\sqrt{m_{l_{i}}m_{l_{j}}}}{m_{W}}\tilde{\chi}^{l}_{ij}\right)l_{j}\right]h^{0},
% \label{fvl}
%\end{equation}
%while one has for  up type quarks: 
%\begin{equation}
%\mathcal{L}_{\bar{u}uh^0}=\frac{g}{2}\,\bar{u}_{i}\left[\left(\frac{m_{u_{i}}}{m_{W}}\right)\frac{\cos{\alpha}}{\sin{\beta}}\delta_{ij}-\frac{\cos(\alpha-\beta)}{\sqrt{2}\sin{\beta}}\left(\dfrac{\sqrt{m_{u_{i}}m_{u_{j}}}}{m_{W}}\tilde{\chi}^{u}_{ij}\right)u_{j}\right]h^{0}.
 %\label{fvq}
%\end{equation}
%}

%\subsection{Charged Higgs Yukawa Lagrangian}
%In the intraction basis, the charge part of the Higgs potential is given as following
%\begin{equation}
%\begin{split}
%L^{H^{\pm}}_{y}= \bar{u}_{Li}\left[\phi^{+}_{2}Y^{d}_{2ij}+\phi^{+}_{1}Y^{d}_{1ij}\right]d_{Rj}-\bar{d}_{Li}\left[\phi^{-}_{2}Y^{u}_{2ij}+\phi^{-}_{1}Y^{u}_{1ij}\right]u_{Rj}+h.c. ,
%\label{Lcharged}
%\end{split}
%\end{equation}
%where in general the fileds $\bar{u}_{Li}, u_{Rj}, \bar{d}_{Li}, d_{Rj}$ should be rotated in order to  diagonalizarse the mass matrices for quarks. In general we have transformation in both type of quarks
%\begin{eqnarray}
%\bar{q}^{'}_{L}=\bar{q}_{L}V^{q^{\dagger}}_{L} & \text{and}&
%q^{'}_{R}=V^{q}_{R}q_{R}, \; \text{with}\; q=u,d.
%\label{Vq-transf}
%\end{eqnarray}
%where $V^{u,d}$ are unitarity transfomations
% \begin{equation}
% V^{u}_{L}V^{u^{\dagger}}_{L}=\mathbbm{1}=V^{d}_{L}V^{d^{\dagger}}_{L}
%\end{equation}

%For the THDM we would have
%\begin{equation}
 %\bar{M}^{diag}_{u}=V^{u}_{L}M_{u}V^{u^{\dagger}}_{R}
%\end{equation}
%\begin{equation}
 %\bar{M}^{diag}_{d}=V^{d}_{L}M_{d}V^{d^{\dagger}}_{R}
%\end{equation}
%con
%\begin{equation}
%M_q=\frac{1}{\sqrt{2}}(v_{1}Y_{1}^{q}+v_{2}
%Y_{2}^{q})
%\end{equation}
%and after rotation we would have 
%$\tilde{Y}_{1,2}^{q}=V_{qL}^{\dagger}Y_{1,2}^{q}V_{qR}$
%for the Yukawa matrices 
%\begin{eqnarray}
%\tilde{Y}^{u}_{1,2}=V^{u}_{L}Y^{u}_{1,2}V^{u^{\dagger}}_{R} &
%\tilde{Y}^{d}_{1,2}=V^{d}_{L}Y^{d}_{1,2}V^{d^{\dagger}}_{R}
%\end{eqnarray}
%As in th SM, having this transformations in the Lagrangian will lead to a mixed matriz de Cabibbo-Kobayashi-Maskawa
%\begin{eqnarray}\label{YCKM}
%Y^{d}_{1,2CKM}&=&V^{u}_{L}Y^{d}_{1,2}V^{d^{\dagger}}_{R} \\
%\end{equation}
%\begin{equation}
%Y^{u}_{1,2CKM}&=&V^{d}_{L}Y^{u}_{1,2}V^{u^{\dagger}}_{R}
%\end{eqnarray}
%where the Cabibbo-Kobayashi-Maskawa is given by $V_{CKM}=V^{u}_{L}V^{d\dagger}_{L}$. Despite the transformation given in (\ref{Vq-transf}) tere is a common treatment as a convention where only the $d-type$ quarks are rotated by the whole $V_{CKM} $ matrix and the {\it u-type} is already in mass eigenstate. 

%After SSB and in the mass Higgs eigenstates basis we have
%\begin{equation}
%\begin{split}
%\Lagr^{H^{\pm}}_{Yuk}= \bar{u}^{'}_{Li}\left[(Y^{d}_{1CKM})_{ij}\cos\beta+(Y^{d}_{2CKM})_{ij}\sin\beta\right]G^{+}d^{'}_{Rj}-\bar{d}^{'}_{Li}\left[(y^{u}_{1CKM})_{ij}\cos\beta+(Y^{u}_{2CKM})_{ij}\sin\beta\right]G^{-}u^{'}_{Rj}
%\\
%+\bar{u}^{'}_{Li}\left[(Y^{d}_{2CKM})_{ij}\cos\beta-(Y^{d}_{1CKM})_{ij}\sin\beta\right]H^{+}d^{'}_{Rj}-\bar{d}^{'}_{Li}\left[(Y^{u}_{2CKM})_{ij}\cos\beta-(Y^{u}_{1CKM})_{ij}\sin\beta\right]H^{-}u^{'}_{Rj}+H.C.
%\end{split}
%\end{equation}

%\begin{equation}
%\Lagr^{H^{\pm}}_{Yuk}=G^{\pm}_w\left[\bar{u}^{'}_{i}A^{d}_{G}\frac{1}{2}(1+\gamma_5)d^{'}_{j}-\bar{u}^{'}_{i}A^{u}_{G}\frac{1}{2}(1-\gamma^5)d^{'}_{j}\right]+h.c.\\
%+H^{\pm}\left[\bar{u}^{'}_{i}A^{d}\frac{1}{2}(1+\gamma^5)d^{'}_{j}-\bar{u}^{'}_{i}A^{u}\frac{1}{2}(1-\gamma_5)d^{'}_{j}\right]+h.c.
%\end{equation} 
%Donde $A^{d,u}$ are $3\times3$ matrices given as:
%\begin{eqnarray}
%A^{d,u}_{G}=Y^{d,u}_{1CKM}\cos\beta+Y^{d,u}_{2CKM}\sin\beta \notag\\
%\end{equation}
%\begin{equation}
%A^{d,u}=Y^{d,u}_{2CKM}\cos\beta-Y^{d,u}_{1CKM}\sin\beta \\
%\end{eqnarray}
%And using eq. (\ref{YCKM}) we get that the elements are given as:
%\begin{eqnarray}
%A^{d}_{ij}&=&(Y^{d}_{2CKM})_{ij}\cos\beta-(Y^{d}_{1CKM})_{ij}\sin\beta\notag \\
%A^{d}_{ij}
%&=&\sum_{k,l}\left\lbrace V^{u}_{Lik}Y^d_{2kl} V_{Rlj}^{d\dagger} \cos\beta-V^{u}_{Lik}Y^d_{1kl} V_{Rlj}^{d\dagger}\sin\beta \right\rbrace
%\end{eqnarray}
%\begin{eqnarray}
%A^{u}_{ij}&=&(Y^{u}_{2CKM})_{ij}\cos\beta-(Y^{u}_{1CKM})_{ij}\sin\beta\notag \\
%A^{d}_{ij}
%&=&\sum_{k,l}\left\lbrace V^{d}_{Lik}Y^u_{2kl} V_{Rlj}^{u\dagger} \cos\beta-V^{d}_{Lik}Y^u_{1kl} V_{Rlj}^{u\dagger}\sin\beta \right\rbrace
%\end{eqnarray}
%Now, using the mass matrix to solve for only one of the Yukawa matrix in the THDM-III, 
%\begin{eqnarray}
%\tilde{Y}_{1}^{d}&=&
%\frac{\sqrt{2}}{v\cos\beta}\bar{M}_{d}-\tan\beta\tilde{Y}_{2}^{d}\nonumber \\
%\tilde{Y}_{2}^{u}&=&\frac{\sqrt{2}}{v\sin\beta}\bar{M}_{u}-\cot\beta\tilde{Y}_{1}^{u}
%\label{rotyukawas}
%\end{eqnarray}
%Then , finally we get for the charged scalar Lagrangian
%\begin{eqnarray}
%\Lagr^{H^{\pm}}_{Yuk}&=&\frac{H^+}{2}\bar{u}^{\prime}_i\sum_k\left\lbrace 
%\sec\beta \left[(V_{CKM})_{ik}(\tilde{Y}_{2}^d)_{kj}-(V_{CKM}^{\dagger})_{ik}(\tilde{Y}_{2}^u)_{kj}\right]-\frac{\sqrt{2}}{v}%\tan\beta\left[m_dj\delta_{kj}(V_{CKM})_{ik}-m_ui\delta_{ki}(V_{CKM})_{ik}\right]\right.\notag\\
%&+&\left. \sec\beta \left[(V_{CKM})_{ik}(\tilde{Y}_{2}^d)_{kj}+(V_{CKM}^{\dagger})_{ik}(\tilde{Y}_{2}^u)_{kj}\right]\gamma_5-\frac{\sqrt{2}}{v}\tan\beta\left[m_dj\delta_{kj}(V_{CKM})_{ik}+m_ui\delta_{ki}(V_{CKM})_{ik}\right]\gamma_5\right\rbrace d_j^{\prime}+h.c.\notag\\
%\end{eqnarray}


%%%%%%%%%%%%%%%%%%%%%%%%%%%%%%%%%%%%%%%%%%%%%%%

%\cor{
%\subsection{Free parameters of the model and so-far constraints}
%
%\begin{itemize}
%    \item Why tilde on $\chi, Y$? THE INFO IS COMPLETE, NOW.
%    \item what parametrize alpha and beta, are there constraints on their possible values? THE ANGLES ARE DEFINED, BUT WE SHOULD GIVE RANGE OF VALUES.
%    \item how is k related to $\chi$; how can this be parametrized? what is "k"? WE SHOULD NOT USE PARAMETER k ANYMORE
%\end{itemize}
%\marginpar{Quitar tabla y tomar datos para escribir parrafo}
%}
\section{Flavor violating top production within the 2HDM-III model}

 
As explained in the previous section, within the 2HDM-III model the Lagrangian Eq. (\ref{lageigenstates}) yields  couplings between quarks and leptons of different families, but  with identical isospin. In this work we search for small but non zero flavor changing neutral currents (FCNC) at tree level, which are caused by the extended Higgs sector. In the following we explore experimental signals of such non-zero couplings, assuming that the light Higgs boson of the model $h^0$ is the Higgs boson found at LHC with mass $m_h = 125$~GeV/$c^2$. 
%In particular within  the  2HDM-III model, we can search for small but non zero flavor changing neutral currents at tree level, which are caused by the extended Higgs sector. 
In the following we will explore this possibility, for both FV transitions between quarks, as well as for FV in leptons. While the former are naturally explored at the Large Hadron Collider, the latter can be accessed in lepton-proton  Deep Inelastic Scattering ($lp$ DIS). In both cases one makes use of the limitation in phase space for the production of two top quarks (hadron-hadron collisions) and a single top quark (hadron-lepton collisions), which leads to a strong suppression of such signals within the Standard Model. 

%Rare top processes could be a key analysis to establish the model structure of beyond the Standard Model physics. This is  in particular true for the study of Yukawa couplings and degrees of freedom which allow for flavor violating. 
%One of these interesting processes is the production of two top quarks or two anti-top quarks in proton-proton collisions. 


Since the center of mass energy of the LHC is not high enough to allow for top production through QCD evolution, 
the top quark cannot appear in the initial state of a partonic process at the LHC. It must therefore be produced via  weak interactions through a process  -- which involves the exchange of W bosons --  or via strong interaction  -- which requires the production of two top-antitop pairs. In both cases the cross-section is strongly suppressed. For weak interactions the exchange of 2 $W$ bosons will have the corresponding weak coupling factors reduction. In the strong interactions, it would be due to the limitations of phase space. For a numerical study of Standard Model process using CalcHEP, we found that the dominant sub-process for the SM cross-section for two $t\bar{t}$ pairs production in  proton-proton collision, as obtained from gluon fusion of the order of $0.146 fb$, the up-quark contribution is one order of magnitude lower, whereas the charm contribution is down by three orders of magnitude with respect to gluon gluon. 

%see  Tab.~\ref{SMsigma4tops}. 
%\begin{table}[t]
%\centering
%\begin{tabular}{|c|c|c||}
%\hline
%subprocess &  $\sigma (pb)$ & unncertainty ($\%$)\\
%\hline
%$g g\to t, \bar{t}, t, \bar{t}$  & $1.460\times 10^{-4}$ & 0.172\\
%\hline
%$u\bar{u}\to t, \bar{t}, t, \bar{t}$ &  $1.642\times10^{-5} $& 0.080 % \\
%\hline
%$\bar{u}u\to t, \bar{t}, t, \bar{t}$ &  $1.640\times10^{-5} $& 0.086 % \\
%\hline
%$c\bar{c}\to t, \bar{t}, t, \bar{t}$ &  $1.115\times10^{-7} $& 0.076 % \\
%\hline
%$\bar{c}c\to t, \bar{t}, t, \bar{t}$ &  $1.112\times10^{-7} $& 0.076 % \\
%\hline
%\end{tabular}
%\caption{Cross-section for dominant Standard Model processes for the %production of two $t\bar{t}$ pairs in  proton-proton collision, as %obtained from CalcHEP.}
%\label{SMsigma4tops}
%\end{table}
%\cor{ 
A similar observation holds for the case of lepton-hadron collisions where the center-of-mass energies are in general significantly lower than in hadron-hadron collisions. For a collider setup this in general  due to a significantly lower energy of the leptonic beam in comparison to the hadron beam, which leads to strong reduction in the center-of-mass energy. As a consequence, production of a top-antitop pair is not possible or at least strongly suppressed in such reactions, see Tab.~\ref{SMsigma}
%EXPLAIN THE TABLE}. 
For the LHeC, single top quark production via charged-current DIS is dominant in all the top quark production channels \cite{Gao:2021plf}. Here we use CalHEP to calculate the possible top production within the SM as background, see table \ref{SMsigma}

\begin{table}[h]
\centering
\begin{tabular}{||c|c|c||}
\hline
process  & $\sigma (pb)$ & uncertainty ($\%$)\\
\hline
$e p\to e+1jet$  & 42.6 & 0.00846\\
\hline
$e p\to e+2jet$  & 8.876 & 0.0566  \\
\hline
$e p\to e+2jet+t$  & 0.1702 & 0.8305 \\
\hline
$e p\to \nu_e+W^-+1jet\to \nu_e + \mu+\bar{\nu}_{\mu} + 1jet $  & 2.87 & 0.03053 \\
\hline
\end{tabular}
\caption{Total cross section at LHeC ($\sqrt{s}=1.296TeV$) for SM background processes regarding the tree level single top production through t-channel. The jets include the top quarks.}
\label{SMsigma}
\end{table}

%\cob{
%{ \color{Blue} We calculate the possible SM background although the number of final particles gives a clear distinction. The alike SM processes are calculated using CalcHEP  and the PDF CT10 through a range of angle form $1^\circ$ to $179^\circ$  avoiding the central production, with the intitial particle momentum given as $80\, GeV$ and $7\, TeV$. The first row in table \ref{SMsigma} represent the only channel with same number of particle production provided theer is no flavour violation in leptonic sector, this precess includes all possible quark interation to produce a jet which in turn represent the topo quark production. The second and third rows are extra jet processes which we give here for completeness propouses. Finally, the only proccess whithin the SM production of muons imply no single top production and is given in forth rowof the table.  }}\\





This limitations of phase space is on the other hand absent within the 2HDM-III model, which provides a  direct coupling of up and charm quark to the top  quark through the extended Higgs sector. The generic process is depicted in Fig.~\ref{fig:mesh1}. 
\begin{figure}[t]
\centering
\includegraphics[width=.5\textwidth, height=4.5cm]{fv_higgs.pdf}
\caption{Flavor violating process transmitted through a $t$-channel exchange of a neutral Higgs boson of an extended Higgs sector. The final state fermions are either two top quarks (hadron-hadron collisions) or a lepton in combination with a top quark (lepton-hadron collisions).}
\label{fig:mesh1}
\end{figure}
For the Large Hadron Collider, incoming fermions are both up and charm quarks and anti-quarks, where contributions due up (anti-) quarks are strongly suppressed in comparison to the charm contribution
due to the quark mass, which appear in the Higgs couplings  Eq.~\eqref{lageigenstates},  with the suppression factor of the order of  $2.4 \times 10^{-4}$, even if the enhancement due to parton distribution functions is taken into account.  We therefore focus on the contribution due to an initial charm quarks, in absence of partonic top quark and due to  conservation of isospin in our model.
 We will therefore study for hadron-hadron colliders the processes
\begin{align}
    c(p) + c(q) & \to t(p') + t(q') ,\notag \\
    \bar{c}(p) + \bar{c}(q) & \to \bar{t}(p') + \bar{t}(q') ,
\end{align}
which allow to constrain the flavor violating coupling between charm and to quarks in the Higgs sector. For lepton-hadron reactions, one of the incoming quark is replaced by a lepton, {\it i.e.} an electron or a muon. We therefore consider in that case
\begin{align}
    l(p) + c(q) & \to l'(p') + t(q') ,\notag \\
    \bar{l}(p) + \bar{c}(q) & \to \bar{l}'(p') + \bar{t}(q') ,
\end{align}
where $l$  is either an electron or a muon and and $l'$ an electron, a muon, or a tau. As far as past and future collider projects are concerned, such a reaction could be observed at the HERA collider ($\sqrt{s} =314$~GeV) as well as at the LHeC ($\sqrt{s} =1.3$~TeV) and at the proposed muon-ion collider (muon-IC)\cite{acosta2022muon}. The center-of-mass energy of the future Electron Ion Collider ($\sqrt{s} =20-140$~GeV) is on the other hand too low for the production of a top quark, and the observation the proposed process will not be possible. From the experimental side,  exclusive same sign double top production, with no other final states, has so far been searched for by  both the CMS \cite{CMS:2011gff} and  the ATLAS \cite{ATLAS:2012iws} collaborations, where the ATLAS collaboration establishes  $\sigma(tt) < 1.7$~pb  as an upper limit on this production cross-section with 95\% confidence level at $\sqrt{s}=7\,TeV$. \\

Since the process does not involve exchange of color, the color averaged scattering amplitude is identical for quark-quark and quark-lepton scattering and reads:
\begin{align}
\vert\mathscr{M}\vert^{2}(ab \to a'b')&=\dfrac{g^{4}}{64}C_{aa'}(\alpha, \beta) C_{bb'}(\alpha, \beta) 
\cdot 
\frac{m_{a}m_{b}m_{a'}m_{b'}}{m_{W}^{4}} 
\cdot
\frac{\left[t-(m_{a} -m_{a'})^{2}\right]\left[t-(m_{b}-m_{b'})^{2}\right]}{(t-m_{h}^{2})^{2}},
\label{Mampli}
\end{align}
%with
%\begin{eqnarray}
%C(\alpha, \beta)=4\dfrac{\cos^{4}(\alpha\,-\beta)}{\sin^{2}(2\beta)}
%\end{eqnarray}
 where quark and lepton masses are neglected against the top mass, whenever both are summed up. We defined the flavor violation couplings for leptonic (or b-type quarks) and t-type quark, respectively as:
\begin{align}
        C_{aa'}(\alpha, \beta)  & = \frac{\cos^2(\alpha - \beta)}{\cos^2(\beta)} |\tilde{\chi}_{ll'}|^{2}, 
    &
     C_{bb'}(\alpha, \beta)  & = \frac{\cos^2(\alpha - \beta)}{\sin^2(\beta)}|\tilde{\chi}_{qq'}|^{2}.
     \label{Cs}
\end{align}
For lepton-hadron process $C_{aa'}$ will come from the leptonic coupling, and $C_{bb'}$ from the t-type quark coupling, while for hadron-hadron collider both will be t-type quark-Higgs coupling.\\

%\begin{align}
%        C_{aa'} & = \frac{\cos^2(\alpha - \beta)}{\sin^2(\beta)} |\tilde{\chi}_{q_{a}q_{a'}}|^{2}, 
%    &
%     C_{bb'} & = \frac{\cos^2(\alpha - \beta)}{\sin^2(\beta)}|\tilde{\chi}_{q_{b}q_{b'}}|^{2}
%     \label{Cs2}
%\end{align}

Note that $q\neq q'$ always; however, we could have $l=l'$ in the case of the muon-proton collider. In such case we should also consider the SM coupling as is given in Eq. (\ref{lageigenstates}). Then, for the process $\mu c \to \mu t$, we have:
\begin{align}
\vert\mathscr{M}\vert^{2}(\mu c \to \mu t)&=\dfrac{g^{4}}{64}C_{\mu\mu}(\alpha, \beta) C_{32}(\alpha, \beta) 
\cdot 
\frac{m_{\mu}^{2}m_{c}m_{t}}{m_{W}^{4}} 
\cdot
\frac{t\left[t-(m_{c}-m_{t})^{2}\right]}{(t-m_{h}^{2})^{2}},
\label{Mampli2}
\end{align}
with
\begin{align}
        C_{\mu\mu} (\alpha, \beta) & =\left|-\frac{\sqrt{2}\sin\alpha}{\cos\beta}+ \frac{\cos(\alpha - \beta)}{\cos(\beta)} \tilde{\chi}_{22}^{l}\right|^{2}, 
    &
     C_{32}(\alpha, \beta)  & = \left|\frac{\cos(\alpha - \beta)}{\sin\beta}\tilde{\chi}_{32}^{u}\right|^{2}
     \label{Cs3}
\end{align}

%\cob{
Before we continue, we must note equation(\ref{Mampli2}) could be zero for certain values of the parameters of the model, since $\tilde{\chi}_{22}^{l}$ is complex, and its complex phase could change the sign, which in turn will imply that the whole process is not viable at LO for 2HDM-III. In this previous case, the relation within parameters of the model takes the following form:
\begin{equation}
 -\frac{\sqrt{2}\sin\alpha}{\cos\beta}+ \frac{\cos(\alpha - \beta)}{\cos\beta} \tilde{\chi}_{22}^{l}=0;  
\end{equation}
taking the extreme complex values for $\tilde{\chi}_{22}^{l} \sim\mathcal{O}(1)$, {\it i.e.} $\chi_{22}^{l}=\pm 1$, which will hold, respectively, the restrictions for $\alpha$ and $\beta$ angles as follow:

\begin{align}
    \sqrt{2}\sin\alpha=\pm \cos(\alpha - \beta) & \Rightarrow 
    \tan\alpha = \frac{\cos\beta}{\mp\sqrt{2}-\sin\beta},
%    &      \frac{\sqrt{2}\sin\alpha}{\cos\beta}& = \frac{\cos^2(\alpha - \beta)}{\cos^2(\beta)}.
     \label{b-arelation}
\end{align}
%}
which in turns would reflect the possibility of having tree level flavor violation coupling, nevertheless, the combination of mixing angles values given for neutral Higgs sector, will suppress the flavor violation scattering amplitude process.
%We obtain equivalence in process as shown in table \ref{tb:s-fv}1
%, so different particles involved imply only substitution of the particle mass value.
%\begin{table}[hbt]
%\renewcommand{\arraystretch}{1.5}
%\begin{center}
%\begin{tabular}{|l|c|c|c|}
%\hline
%$ep \to \mu t $  & $\Longleftrightarrow$ & $\mu p \to e t $  \\
%\hline
%$\mu p \to \tau t $  & $\Longleftrightarrow$ & $\tau p \to \mu t $  \\
%\hline
%\end{tabular}
%\renewcommand{\arraystretch}{1.2}
%\caption[]{\label{tb:s-fv} \it 
%Equivalence in Flavor Violation in a lepton hadron scattering processes through neutral Higgs bososn s-channel. 
%neutral Higgs bosons to pair of fermions.}
%\end{center}
%\end{table}
Having discussed the structure of the couplings, we are able to construct now the analytical hadronic cross-sections, which read
\begin{align}
%    \label{eq:Xsec:lh}
    \sum_{f=t,\bar{t}}\sigma(pl \to fl')& = \frac{1}{16 \pi} \int_{x_{min}}^1  dx \,   \int_{t_-}^{t^+} dt \,  \left[\frac{|\mathscr{M}|^2 (lc \to l't) }{(xs)^2} \cdot f_c(x, \mu_F)  + \frac{|\mathscr{M}|^2 (l\bar{c} \to l'\bar{t})}{(xs)^2} \cdot f_{\bar{c}}(x, \mu_F)  \right]
    \label{sigmapl}
\end{align}
 for the case of lepton-hadron scattering the center-of-mass energy squared ven as $\hat{s}=xs$, with $x_{min} = \frac{(m_{l'} + m_t)^2}{s}\approx \frac{m_t^2}{s} $,  and 
\begin{align}
    \label{eq:Xsec:hh}
  \sum_{f=t,\bar{t}}    \sigma(pp \to ff) & = \frac{1}{16 \pi} \int_{\frac{4 m_t^2}{s}}^1  dy \,  \cdot \int_{t_-}^{t^+} dt \, \left[ \frac{ |\mathscr{M}|^2 (cc \to tt) }{(ys)^2} {\cal L}_{cc}(y, \mu_F) + \frac{ |\mathscr{M}|^2 (\bar{c}\bar{c} \to \bar{t}\bar{t}) }{(ys)^2} {\cal L}_{\bar{c}\bar{c}}(y, \mu_F)  \right]
\end{align}
for hadron-hadron scattering, where
\begin{equation}
t_{\pm}{(ab;a'b')}=m_a^2+m_{a'}^2-\frac{1}{2\hat{s}}\left[ (\hat{s}+m_a^2-m_{b}^2)(\hat{s}+m_{a'}^2-m_{b'}^2)\mp \lambda^{1/2}(\hat{s},m_a^2,m_{b}^2)\lambda^{1/2}(\hat{s},m_{a '}^2,m_{b '}^2)\right],
\end{equation} 
where
$\lambda(x,y,z)=(x-y-z)^2-4yz$,  and partonic center-of-mass energy squared $\hat{s}=ys$  with
\begin{align}
        {\cal L}_{ab}(y, \mu_F) & = \int_y^1 \frac{dx}{x} f_a(x, \mu_F)f_b(y/x, \mu_F),
\end{align}
and $f_q(x, \mu_F)$, $q= c, \bar{c}$  the (anti-)charm distribution function in the proton, evaluated at the factorization scale $\mu_F$ which we identify in the following with the top quark mass. For numerical studies we use the leading order MMHT2014 PDF set \cite{Harland-Lang:2014zoa}.
%In order to have a consistent description with CalcHEP 3.8.7 \cite{Belyaev:2012qa}, used in the following the estimate potential SM background in inclusive reactions, we use the the CT10 PDF set, \cor{\bf cite reference; maybe we should also redo this with an more up-to-date PDF}.\\

\section{Numerical analysis}
\label{sec:num}

As our first approach to possible top production through neutral Higgs flavor violation processes, we perform a numerical analysis considering  $qq$ initial states, searching for same sign top pair production at LHC, with $0<\beta-\alpha<\frac{\pi}{2}$ as discussed in \cite{Gunion_2003},  and a relatively loose restriction on  $\tan \beta$, $\mathcal{O} (10^{-2})<\tan \beta<\mathcal{O} (10^{1})$ in order to observe the production behavior on this free parameters of the model, see Figure \ref{fig:exclude}. At this point we want to state the following remark: too close to zero values on $\tan\beta$ would lead to an unwanted enhancement on u-quark type Higgs couplings which are proportional to $\cot\beta$ and, on the other hand, higher values on $\tan\beta$ would lead to an enhancement on d-quark type Higgs couplings, which in turn are proportional to $\tan\beta$. Our numerical analysis will show the behavior of the former case, having increasing values for $\sigma(pp \to tt, \bar{t}\bar{t})$ for low $\tan\beta$.

Furthermore, we calculate the scattering amplitude for the single top production within a flavor violation context, with a future perspective and contribution for proposed  lepton-proton colliders. Specifically, for the initial states $eq, \mu q$ and taking into account the neutral Higgs coupling in this extended model, we will get, $\mu \, t$ and $\tau \, t$ as part of the final states. Due to the form of the neutral Higgs boson couplings, these processes depend on all the masses of particles involved, we can see that for electron in the final state will greatly reduce the scattering amplitude.

The flavor violation coupling parameter, given in the matrix amplitude (\ref{Cs}), will be taking in general as $|\tilde{\chi}^{q,l}_{ij}|\sim \mathcal{O}(1)$, where $i,j$ stand for fermionic flavor\footnote{As can be seen in the analytical expression for the flavor violation cross sections, they are directly proportional to $|\tilde{\chi}^{q,l}_{ij}|^2$, any other values will change the cross section in this proportionality.}.  In calculating the cross section $ep\rightarrow l X_t X$ working in the context of the parton model, with the parton distribution function given in reference \cite{Pumplin:2002vw}, we observe that the charm-top transition is dominant over the up-top contribution, as is evident in Figure \ref{Fig.uvsc}. Hence, we can safely explore only the charm-top $\tilde{\chi}^u_{23}$ for hadronic part. 

%List of questions
%\begin{itemize}
%    \item What is the k value? is this defined somewhere? We do not use it anymore, \cob{it was from the rough calculations. We should take all k's out.} 
%    \item For the numerical study at LHC: what shall be used as a starting value for the chis? 
 
 
 %By performing our computations,we work in the context of the parton model  \cite{Pumplin:2002vw}.
%C            January 24, 2002
%CC
%C   Ref: "New Generation of Parton Distributions with
%C         Uncertainties from Global QCD Analysis"
%C   By: J. Pumplin, D.R. Stump, J.Huston, H.L. Lai, P. Nadolsky, W.K. Tung
%C       hep-ph/0201195
%C
%C   This package contains 3 standard sets of CTEQ6 PDF's and 40 up/down sets
%C   with respect to CTEQ6M PDF's. Details are:
%C %---------------------------------------------------------------------------
%C  Iset   PDF        Description       Alpha_s(Mz)**Lam4  Lam5   Table_File


 
 
 
%We consider  $qq, eq, \mu q$ initial states, due to the interest on top processes at current collider and future ones; in combination with $\mu \, t$ and $\tau \, t$ as part of the final states. Due to the form of the neutral Higgs boson couplings, these processes depend on all the masses of particles involved, we can see that for $e$ in the final state will greatly reduce the scattering amplitude.
In order to make our numerical calculations we will take the masses of the fermions involved in this work
 as follows: $m_e=0.511$ MeV, $m_{\mu}=105.66$ MeV, $m_{\tau}=1.777$ GeV, $M_p=0.938$ GeV, $m_u=2.16$ MeV, $m_c=1.27$ GeV and $m_t=172.69$ GeV. Finally, we take the $W{^\pm}$, $Z^0$ and $h_1^0$ as $m_{W^\pm}=80.44$ GeV, $m_{Z^0}=91.2$ and $m_{h_1^0}=0.125$ GeV \cite{ParticleDataGroup:2022pth}. 
  
%    \item can we have a discussion somewhere of meaningful values of the parameters alpha, beta and k (which is somehow contained in the chi's) 
 %   \cob{we discuss this in the model section and in this section the parameters of the model relation}
 
   % \item background reaction for ep (why jets)
    %\item Which reactions for lepton-hadron? initial e, mu, final e, mu, tau? 

%    \item HERA limit? \cob{is already in the introduction}
%\end{itemize}
\subsection{Constraints on charm-top transitions from LHC data}
 %\cob{The center of mass dependence for hadron hadron is tricky, since also the background goes up (moving away from threshold)...\\}
%\cog{
In order to calculate the same sign pair top production via flavor violation at tree level, working in the context of 2HDM-III, 
%} 
we evaluate Eq.~\eqref{eq:Xsec:hh} for $\sqrt{s} = 7$~TeV, 
%\cog{using the Drell-Yang model with the} 
MMHT2014 leading order PDFs   \cite{Harland-Lang:2014zoa}, we  obtain the following numerical result,
\begin{align}
   \sum_{f=t,\bar{t}}    \sigma(pp \to ff)  & = C(\alpha, \beta) \cdot |\tilde{\chi}^u_{23}|^4 \cdot  9.37 \times 10^{-6}~\text{pb},
\end{align}
%\cog{
with  $C(\alpha, \beta)=\cos^4(\alpha-\beta)/\sin^4 (\beta)$.
%containing only the $\alpha$ and $\beta$ parameters coming from the couplings. 

From the upper limit given by ATLAS, we have $\sum_{f=t,\bar{t}}    \sigma(pp \to ff)  < 1.7$~pb, this allow us to establish a restriction on the model parameter's which is rather wide
%}
\begin{align}
    C(\alpha, \beta) \cdot  |\tilde{\chi}^u_{23}|^4 < 1.81\times 10^{5}
\end{align}
For the current energy reached at LHC, $\sqrt{s} = 14$~TeV, the cross-sections for dedicated choices of the parameters $\alpha, \beta$, are given in Tab.~\ref{LHC_ab}
\begin{table}[h]
  \begin{center}
\begin{tabular}{|c|c|c|c|c|}
\hline
 $\beta-\alpha$  & $\tan\beta$& $\sqrt{s}$ (TeV)  &$ \displaystyle \sum_{f=t,\bar{t}}    \sigma(pp \to tt, \bar{t}\bar{t})$   \\
 \hline
 $\pi/3$ &$1/50$& 14  &  $5.05$~pb \\
 \hline
$\pi/3$ & $1$& 14  &  $3.3 \cdot 10^{-6}$~pb \\
 \hline
 $\pi/3$ & $50$& 14  &  $8.1\cdot 10^{-7}$~pb \\
 \hline
$\pi/4$ & $1/50$& 14  &  $20$~pb \\
 \hline
$\pi/4$ & $1$& 14  &  $1.3\cdot 10^{-5}$~pb \\
\hline
$\pi/4$ & $50$& 14  &  $3.2\cdot 10^{-6}$~pb \\
\hline
\end{tabular}
\label{LHC_ab}
\caption{Cross-section for dedicated choices (some illustrating values along the range) of the model parameters $\alpha, \beta$ with $|\tilde{\chi}^u_{23}| =1$}
\end{center}
\end{table}
 %\cog{
We can see in  this Table that as $\beta-\alpha$ is closer to $\pi/2$ the cross-section diminishes, recovering the SM prediction in the limit $\beta-\alpha \to \pi/2$, while the cross-section increases for low $\tan\beta <1$. Fig.~\ref{fig:exclude} illustrates the  parameter space region excluded by $7$~TeV LHC data, where we observe that $\tan\beta \lesssim 5\times 10^{-2}$ gets strongly suppressed from experimental data for the case of $\beta-\alpha\lesssim \pi/2$, seen as a excluded blue part in the Figure \ref{fig:exclude}. 
%The upper an lower limits on $\tan\beta$, inversely imply that the up and down -type quark coupling with neutral Higgs boson, which are proportional to $\cot\beta$, will be undesirable large for close to zero values of $\tan\beta$.
Whereas, for $\beta-\alpha \sim \pi/2$ this restriction on $\tan\beta$ disappears, although the cross section diminishes as is suggested from Figure \ref{fig:sigmapp2tt}.
%}
%\cor{(values of $\beta-\alpha$ close to $\pm \pi \; (\pi/2$  lead to a significant enhancement of the cross-section). 
%The same is true for values of $\beta$ close to $0$)}.
%It is then only those scenarios which are constrained by current LHC data. 

\begin{figure}[h]
     \centering
     \includegraphics[width=0.6\textwidth]{exclusion_7TeVMar10.pdf}
     \caption{In Blue the range of parameters excluded by current $7$~TeV LHC data through imposing the requirement $\sigma(pp \to tt, \bar{t}\bar{t})<1.7$~pb.}
     \label{fig:exclude}
 \end{figure}
 \begin{figure}[h]
     \centering
      \includegraphics[width=0.7\textwidth]{ttatlhcMMHT14Mar10.png}
     \caption{Logarithm of the two (anti-)top production cross-section at the LHC with $\sqrt{s} = 14$~TeV through the $t$-channel exchange of a flavor violating Higgs boson with mass $m_H = 125$~GeV/$c^2$.}
     \label{fig:sigmapp2tt}
 \end{figure}
%\cog{
In Figure \ref{fig:sigmapp2tt}, we can observe the possible values of the top pair production cross section at $\sqrt{s}=14$ TeV in the 2HDM parameter space. Having an enhancement for $\beta-\alpha \sim 0$ reaching the order of picobarn as a maximal value. Although, as we will discuss in next section, there is a constrain on a relation within $\cos(\beta-\alpha), \, \tilde{\chi}^u_{23}$ and $\sin\beta$, see Eq.(\ref{tchBound})  coming from experimentally bounded FV top decay $t\to c h^0$, disallowing the maximal values of these cross section to be reached.
%}

%%%%%%%%%%%%%%%%% needs to be edited 


%In the next section, we will discuss furthermore the consequences of the parameters 
%\cor{of the $k$ value {\bf what iws the k value?}} on the total cross section given by Eq.~\eqref{sigmaT}.
%\subsection{$p p \to lepton + t \to l + X $ }
%\subsection{$p p \to t+ t \to X + X $ }
%\subsection{$p p \to \bar{t}+ t \to X + X $ 
%From the LHC data for sigle top production....
%$\vert\tilde{\chi}_{i,j}^{u,d,l}\vert^{2}=1$ \cor{Poner tabla con los parametros}

\subsection{FCNC single top production through $e q\to l+X_t +X$ }

The next process we analyze is given as $ep\to l X_t X $, where $l$ is a charged lepton, $X_t$ corresponds to a jet coming from top production and $X$ could be anything.
From the analytical expression given in previous section we are in a position to discuss the parameter space for electron-proton collisions related to the production of a single top through FCNC. 
Also we explore the process for a potential muon-proton collider \cite{https://doi.org/10.48550/arxiv.2203.06258} which will be decisive for this FCNC.

Observed that, as we said before, the processes are via Higgs boson exchange, the amplitude of the processes depend on the masses of the external particles involved, as shown in eq. (\ref{Mampli}) and the parameters of the model eq. (\ref{Cs}). 
Three processes considered here are dependent through the different parameters $\tilde{\chi}^l_{ij}$ and the masses of the leptons involved. 
Each cross section, $\sigma(l_i q_{k}\rightarrow l'_j t X)$ with  ($i\neq j$) have one leptonic parameter $\tilde{\chi}^{l}_{ij}$ and in principle, two quark contributions $\tilde{\chi}^{u}_{k3}$,  with $k=1,2$. Although there is no relation between processes with different leptons, the parameters are independent from one another, we will show in our numerical analysis, the charm quark contribution $|\tilde{\chi}^{u}_{23}|^2$ dominates, then we can factorize the parameters, then
we can express these relations as follows:
\begin{eqnarray}
    \sigma (ep \to \tau q X) \approxeq \frac{m_\tau}{m_\mu}\frac{|\tilde{\chi}^l_{13}|}{|\tilde{\chi}^l_{12}|}\sigma (ep \to \mu q X) \nonumber \\
   % \sigma (\mu p \to \mu q X)  \approxeq  \frac{m_\mu}{m_e}\frac{|\chi^l_{22}|}{|\chi^l_{12}|}\sigma (ep \to \mu q X) \nonumber \\
    \sigma (\mu p \to \tau q X)  \approxeq  \frac{m_\tau}{m_e}\frac{|\tilde{\chi}^l_{23}|}{|\tilde{\chi}^l_{12}|}\sigma(ep \to \mu q X);
    \label{sigmaRel}
\end{eqnarray}
The differences could come from the kinematic relations, nevertheless in this paper, for the numerical calculation purpose, we consider in the kinematics, all fermion massless except for the quark top. Hence, in this approximation for the kinematics we reach the equality on Eq. (\ref{sigmaRel}) when all lepton masses are taken equal to zero, and only in the dynamics we consider the masses involved different from zero.
 


For the numerical calculation of the sub-processes $l_iq\rightarrow l_j t $ cross section, we consider first a range for $\cos(\beta-\alpha)\in [0,1]$ and $\tan \beta \in [1,50]$ (discussions about the THDM and parameter space  can be found in \cite{Gunion_2003,Branco_2012}), for energies in the center-of-mass running from $\sqrt{s} \in [1.3,50]$ TeV.

The results we obtained for processes with lepton flavor violation are given in general  as
$$\sigma(ep\rightarrow \mu X_t X)=\mathcal{O}(10^{-5})|\tilde{\chi}^{l}_{12}|^2|\tilde{\chi}^{u}_{23}|^2$$
The other processes of this type will be obtained by Eq. (\ref{sigmaRel}).

As seen in Fig. \ref{Fig.3Dep-mut}, the cross sections  will increase to its maximal value for $\cos(\beta-\alpha)\approx 1$ and high values of $\tan\beta\sim 50$. In this case, the order of magnitude of the parameters are taken as $\vert\tilde{\chi}_{ij}^{u,d,l}\vert^{2}=1$.

\begin{figure}[h!]
\centering
\includegraphics[width=0.8\textwidth]{GRAFICAS/Gr3D-epmutS.png}
\caption{The total cross section to the process $e p\rightarrow \mu+ X_t+X$, with center of mass energies $\sqrt{s}\in[1.3, 50]$ TeV. Is is shown its dependence on the wide range of free parameters $\tan\beta \in [1,50]$, $\cos(\beta-\alpha)\in [0,1]$ and $|\tilde{\chi}_{12}^l|=|\tilde{\chi}_{ij}^u|=1$. }
\label{Fig.3Dep-mut}
\end{figure}

As we already said, we see from our results in Fig. \ref{Fig.uvsc} that the parton contribution from u-quark is small compared with the dominating c-quark contribution, so we can safely explore the single top production process exclusively from the charm quark contribution.
\begin{figure}[h!]
\centering
\includegraphics[width=0.8\textwidth]{GRAFICAS/Gra-epmut-sS.png}
%\includegraphics[width=0.5\textwidth]{GRAFICAS/Gra-epmut-s.png}
\caption {The contribution from each quark, $u$(green), $c$(blue), to the total cross section  to the process $e q\rightarrow \mu t$, with $q$ stands for $u,c$. Considering the values of the parameters as $\tan\beta \in [1,50]$, $\cos(\beta-\alpha)\in [0,1]$ and $|\tilde{\chi}_{12}^l|=|\tilde{\chi}_{13}^u|=|\tilde{\chi}_{23}^u|=1$.}
%The masses of each quark are involved but also the parton distribution functions associated with each parton.}
\label{Fig.uvsc}
\end{figure}

%\cor{Using this
%approximations, the cross section gets reduced in a notable way due to
%the value of $W$ which has an order of magnitude of about $10^{-13}$,
%making the observation of a flavor violating process highly
%unlikely. }
%\marginpar{Rehacer gráficas con un rango mayor de energías 1200GeV a 100TeV (DECIR QUE para energías del EIC (20GeV) NO SE ABRE EL UMBRAL DE LA PRODUCCI[ON DE SINGLE TOP). poner etiquetas a las gráficas, poner unidades}
%\cor{On the other hand if we had an scenario with $k=1$} then the
%cross section $\sigma^{ep}$ would be experimentally observable. 




%\begin{figure}[H]
%\centering
%\includegraphics[width=0.7\textwidth]{GRAFICAS/GraficaeQs-mu+top.png}
%\caption {The contribution from each quark, $u$(yellow), $c$(red), to the total cross section  to the process $e p\rightarrow \tau+t$. The masses of each quark are involved but also the parton distribution functions associated with each parton.}
%\end{figure}

\begin{figure}[h!]
\centering
\includegraphics[width=0.8\textwidth]{GRAFICAS/Gra-eplept-ba.png}
\caption{The total cross section for $ep$ t-channel FCNC top production via DIS with two different processes: $e p\rightarrow \tau (\mu)+t $, blue (yellow). With center of mass energies $\sqrt{s}\in[1.3, 50]$ TeV, applying the restriction on top FV neutral Higgs boson decay. }
\label{ep2lep-t-cna}
\end{figure}
Considering the experimental bounds over flavor violations process as top decay $\Gamma(t\to c h)$, the parameter would be restricted as is reported in \cite{arhrib2016two},
\begin{eqnarray}
%\displaystyle
 \Gamma(t \to c h)&=&\frac{1}{32\pi m_t^2}\left( \frac{\cos(\beta -\alpha)\tilde{\chi}^u_{23}}{\sin \beta} \right)^2\left(
 (m_c+m_t)^2-m_h^2\right) \nonumber \\
 &\times &\sqrt{(m_t^2-(m_h-m_c)^2)(m_t^2-(m_h+m_c)^2)}
\end{eqnarray}

At the LHC, ATLAS and CMS have searched for top FCNCs and they set a limit on the flavor violating branching fraction $Br(t \to ch) < 0.82\%$ for ATLAS \cite{TheATLAScollaboration:2013nbo} and $Br(t \to ch) < 0.56 \%$ for CMS \cite{CMS:2014qxa}. We will use the bound given for CMS, as is the more stringent because they take into account more processes in the search. 

Then, using the CMS results to bound the free FV parameters of the model we get
\begin{eqnarray}
 \left( \frac{\cos(\beta -\alpha)\chi^u_{23}}{\sin \beta} \right)< 0.36 & \text{  for  } & Br(t\to c h)< 5.6\times 10^{-3}
 \label{tchBound}
\end{eqnarray}
In general, the experimental bound for a flavor violation decay $y^{Exp}_{ij}$ will generate a relation with the free parameters $\alpha, \beta $ and $\tilde{\chi}^f_{ij}$ as follows
\begin{eqnarray}
    \left|\frac{|\tilde{\chi}_{ij}|^2 \cos \alpha}{y^{Exp}_{ij}-\sin\alpha|\tilde{\chi}_{ij}|^2}\right|<\tan \beta
\end{eqnarray}
%where $\chi_{23}=1$

%\marginpar{poner etiquetas a las gráficas}
From figure \ref{Fig.uvsc} we can see that the dependence on the center on mass energy $\sqrt{s}$ is not significant, the relevance will come from improvement on luminescence.

%\vspace{0.3cm}
%\cor{put this result on equation \ref{sigmaep} since PDFs are distribution functions then they have a value smaller than one order of magnitude and so this is not an important contribution}
%\begin{figure}[t]
%\centering
%\includegraphics[width=.7\textwidth]{GRAFICAS/Gra-eplept-ba.png}
%\centerline{ a)}
%\centerline{ b)}
%\includegraphics[width=.4\textwidth]{GRAFICAS/beSigLOGep2mutop4.png}
%\centerline{c)}
%\caption{\normalsize{Cross section for t-channel FCNC top production via DIS $ec \to \mu (\tau) t$ in green (black), with center of mass energies $\sqrt{s}\in[1, 50]$ TeV, considering the $t \to c h$ restriction on $\cos(\beta -\alpha)$.}}
%\label{fig:results2}
%\end{figure}

%\begin{figure}[ht!]
%\centering
%\includegraphics[width=.8\textwidth]{Graf-tB30.PNG}
%\includegraphics[width=.7\textwidth]{GRAFICAS/GrafTan50.PNG}
%\caption{Total cross section for t-channel muon and top production via DIS ep, considering $eu+ec$ as initial state with value of $\tan\beta=50$ and three different values of $\beta -\alpha=0,\pi/3,\pi/4$.}
%\label{fig:2}
%\end{figure}
%\begin{figure}[h!]
%\centering
%\includegraphics[width=0.7\textwidth]{GRAFICAS/beSigLOGep2mutop4.png}
%\caption{The total cross section to the process $e p\rightarrow \mu+t$. The masses of each quark are involved but also the parton distribution functions associated with each parton. \cor{Habrá que ver si esta funciona es la misma que las dos anteriores}}
%\end{figure}
%\marginpar{Hacer gráficas de mu+p para complementar las de e+p->mu, tau+top}
Our results for $e p\rightarrow \tau (\mu)+t $ are shown in Fig.~\ref{Fig.uvsc} and Fig.~\ref{ep2lep-t-cna}. 
Fig.~\ref{Fig.uvsc} shows  electron and up quark ($eu$) and electron and charm quark ($ec$) at the initial state; note that convolutions with the corresponding parton distribution functions have been taken into account. 
% These processes are more the likely to occur because even if there are flavour changing, light Higgs can not change charge,

%\cor{Each figure display two cases, $k=1$ where $\vert\tilde{\chi}_{i,j}^{u,d,lf}%\vert^{2}\sim10^{3}$ and   $k\neq\,1$ where $\vert\tilde{\chi}_{12}^{f}\vert^{2}\sim10^{1}$. The center of mass energies used for calculation are: HERA $\sqrt{s} = 0.310$~TeV, LHeC $\sqrt{s} = 1.3$~TeV, LHeC-he $\sqrt{s} = 1.9$~TeV, and FCC-he $\sqrt{s} = 3.5$~TeV (see also \cite{energias})
%}

%\end{multicols}



%\subsection{$p p \to \bar{t}+ t \to X + X $ }
%\subsection{$e p \to \mu+ t \to \mu + X $ }
%Figura 8
%\begin{figure}[h!]
%\centering
%\includegraphics[width=0.7\textwidth]{GRAFICAS/GePtautop0.png}
%\caption{The contribution }
%\end{figure}

In Fig.~\ref{ep2lep-t-cna} we present the comparison of the production of single top with muon, $e p \to \mu + X_t + X $ and the single top production accompanied with a tau, in the process $ e p \to  \tau + X_t +X $, at $\sqrt{s}=1.3 - 50$~TeV; considering the restriction on the parameters given in Eq. (\ref{tchBound}). 
As expected according to Eq.(\ref{sigmaRel}), the difference is given by $m_\tau/m_\mu$.

%\subsection{$ e p \to \tau + t \to \tau + X $ }
\subsection{FCNC single top production through $\mu p\to l+X_t +X$ }
For completeness, in this subsection we include the analysis of single top production in a possible $\mu p$ scattering processes. The muon collider has been studied in terms of the energies and the kinematics achieved \cite{acosta2022muon}, then we explore possible exotic flavor processes.
%This is our case. 
We find that in this kind of colliders, the flavor violation single top production, through DIS scattering will improve the probability to detect this exotic signal.
Moreover, as we explored the analytical expressions in section 3, the lepton-lepton-Higgs coupling has the special structure that includes SM and also 2HDM contributions, which in turn could have opposite signs, and then enhance or diminish the scattering amplitude having no direct proportionality with the masses and the $|\tilde{\chi}^l_{ii}|^2$ values, this is why detailed numerical exploration is more relevant in this case.

\begin{figure}[hbt!]
\centering
\includegraphics[width=0.8\textwidth]{GRAFICAS/Gra-mupmut-bap01110.png}
%\includegraphics[width=.45\textwidth]{GRAFICAS/Gra-mupmut-ba.png}
%\centerline{ a)}
%\includegraphics[width=.45\textwidth]{GRAFICAS/Gra-muptaut-ba.png}
%\centerline{ b)}
%\includegraphics[width=.4\textwidth]{GRAFICAS/beSigLOGep2mutop4.png}
%\centerline{c)}
\caption{\normalsize{Cross section for t-channel FCNC  top production via DIS $\mu c \to \mu t $ comparing different orders of FV parameter: $\tilde{\chi}^l_{22}=0.1,1,10$ in purple, black and gray, respectively. Within a range for the With center of mass energies $\sqrt{s}\in[1.3, 50]$ TeV, considering the $t \to c h$ restriction.}}
\label{fig:resultsmu2p}
\end{figure}
As we discuss in section 3, the flavor violation parameter could be complex, for purpose of illustration we take extreme values for the complex  phases, positive and negative, then we would have possibilities of constructive or destructive interference with the SM term, see Eq. (\ref{Cs3}). These results are shown in Fig. \ref{fig:resultsmu2p}, for positive values, and in Fig. \ref{fig:resultsmu2n} for negative values. An aspect that is worth to notice in the positive case is that for $\tilde{\chi}^l_{22}=0.1$ we would have grater values for the cross section than those for $\tilde{\chi}^l_{22}=1$, implying that for the last case, the destructive interference is greater. For a high value of the parameter, $\tilde{\chi}^l_{22}=10$ we reach 0.01 pb for the single top production cross section, and the 2HDM contribution being the dominant one. 
For negative values of the parameter, Fig. \ref{fig:resultsmu2n}, we observe a general constructive interference with the SM contribution.

In Fig. \ref{fig:resultsmu2mutau}, as is expected, the single top production with muon-tau flavor violation process increase the cross section value. It is also seen here, that for negative values of the parameter, the interference is constructive, compared with the positive one of the same order.

\begin{figure}[hbt!]
\centering
\includegraphics[width=0.8\textwidth]{GRAFICAS/Gra-mupmut-ban01110.png}
%\includegraphics[width=.45\textwidth]{GRAFICAS/Gra-mupmut-ba.png}
%\centerline{ a)}
%\includegraphics[width=.45\textwidth]{GRAFICAS/Gra-muptaut-ba.png}
%\centerline{ b)}
%\includegraphics[width=.4\textwidth]{GRAFICAS/beSigLOGep2mutop4.png}
%\centerline{c)}
\caption{\normalsize{Cross section for t-channel FCNC  top production via DIS $\mu c \to \mu t $ comparing different orders of FV parameter: $\tilde{\chi}^l_{22}=-0.1,-1,-10$ in purple, black and gray, respectively. Within a range for the With center of mass energies $\sqrt{s}\in[1.3, 50]$ TeV, considering the $t \to c h$ restriction.}}
\label{fig:resultsmu2n}
\end{figure}


\begin{figure}[hbt!]
\centering
\includegraphics[width=.8\textwidth]{GRAFICAS/Gra-muplept-banp1.png}
%\includegraphics[width=.45\textwidth]{GRAFICAS/Gra-mupmut-ba.png}
%\centerline{ a)}
%\includegraphics[width=.45\textwidth]{GRAFICAS/Gra-muptaut-ba.png}
%\centerline{ b)}
%\includegraphics[width=.4\textwidth]{GRAFICAS/beSigLOGep2mutop4.png}
%\centerline{c)}
\caption{Cross section for t-channel FCNC  top production via DIS comparing $\mu c \to \mu  t $ in orange and black, and for $\mu c \to \tau  t $ in purple, taking $\tilde{\chi}^q_{23}= 1$ and $\tilde{\chi}^l_{ij}= \mathcal{O}(1)$, within a range for the center of mass energies $\sqrt{s}\in[1.3, 50]$ TeV, considering the $t \to c h$ restriction.}
\label{fig:resultsmu2mutau}
\end{figure}
%\begin{figure}[hbt!]
%\centering
%\includegraphics[width=.45\textwidth]{GRAFICAS/Gra-mupmut-Xbapn1.png}
%\includegraphics[width=.45\textwidth]{GRAFICAS/Gra-muptaut-bac.png}
%\centerline{ a)}
%\includegraphics[width=.45\textwidth]{GRAFICAS/Gra-muptaut-ba.png}
%\centerline{ b)}
%\includegraphics[width=.4\textwidth]{GRAFICAS/beSigLOGep2mutop4.png}
%\centerline{c)}
%\caption{\cor{Cross section for t-channel FCNC  top production via DIS $\mu c \to \mu (\tau) t $ in orange (black), with center of mass energies $\sqrt{s}\in[1, 50]$ TeV, considering the $t \to c h$ restriction on $\cos(\beta -\alpha)$.}}
%\label{fig:results2}
%\end{figure}

%\begin{figure}[h!]
%\centering
%\includegraphics[width=0.7\textwidth]{GRAFICAS/Gra-muplept-ba.png}
%\caption{Cross section for t-channel FCNC top production via DIS muon-proton to the process $\mu p\rightarrow \tau %+t+X$ (red) and $\mu p \rightarrow \muon +t+X$ (blue), considering the $t \to c h$ restriction on $\cos(\beta %-\alpha)$.}
%\end{figure}
%\subsection{$\mu p\to \mu+t\to\mu+X$ }
%\subsection{$\mu p\to \tau+t\to\tau+X$ }



%\section{t-channel single top production via $eP$ DIS with FCCC} 
%\subsection{$e p \to \nu_{e} + t \to \nu_{e}+X $}
% \section{\cor{t-channel doble same charge top production via $eP$ DIS}}
%\subsection{\cor{$pp\to t+t$ }}
%\subsection{$pp\to \mu+t\to\mu+X$ }
%%%%%%%%%%%%%%%%%%%%%%%%%%%%%%%%%%%%%%%%%%%%%%%%%%%%%%%%%
%\section{Results}

%For a general process $eq\rightarrow\,eq$ based on a flavor  neutral ({\it i.e.} SM like) Higgs  boson,  flavor violation is absent,   $\tilde{\chi}_{ij}^{u,d,l}=0$. 
% independent cross section and having direct proportion to the product of the square masses of the interacting lepton and quark, however in this framework the process is constricted and we can just obtain same incident and 
%Processes such as $eu\rightarrow\,et$ or $eu\rightarrow\,\mu\,t$ are only present at one and two loops respectively.
%We therefore consider the  process $eu\rightarrow\,\mu\,t$ actually the analysis are able to be extended to a general reaction $lq\rightarrow\,l'q'$ for any incident lepton $l$ and quark $q$ and any outgoing lepton $l'$ and quark $q'$ due couplings are the same order of magnitude. \\
\section{Conclusions}
In this paper we discussed the possibility to observe effects of FCNC through flavor violating neutral Higgs boson accompanied by either same sign top pair production for pp collisions, or single top production at lepton-proton colliders. While we found that cross-sections are small, $\sigma (f q \to f' t)\lesssim \mathcal{O}(10^{-1})pb$, and its observation would be challenging. On the other hand, also means that the consideration of BSM FV couplings in the Higgs sector at tree level could not be ruled out from current experiments, a need for future colliders lepton-proton results that would be more decisive.



We performed the complete analytical calculation for exotic flavor violation process in a 2HDM in proton-proton and lepton-proton collisions at LO.
In general, FCNC
%(Flavour Changing Neutral Current 
interactions,  where neutral Higgs boson and top quark are involved, are an excellent source to seek beyond Standard Model signals.
The analytic structure for each of these processes were generated through the corresponding sub-processes $lq\to l't$ and  $qq\to tt$, with $q=u,c$. Considering that top quark does not contribute to the sea quark, due to its heavy mass,  we use PDFs based in a five light quark scheme.
%since we do not have top pairs produced in the quark-gluon sea. 
%, as the top signal is a clear one, although we need to reach its threshold production.
In addition, according to 
%\textit{Higgs Boson Flavor-Changing Neutral Decays into Top Quark in a General Two-Higgs-Doublet Model}~
\cite{4} those kind of processes have not been fully reviewed so far, then, continuing the exploration for different scenarios would be relevant to establish BSM possibilities. 
%\begin{itemize}
%\item
In order to search for clean signals of new physics, which would directly confirm or discard this FV process through Higgs neutral currents, we determined a specific non-SM analogous tree level process through t-channel in Deep Inelastic Scattering  that could be tested at current and future colliders.

For the analysis of the FV process, we first proceed calculating same sign pair top production at a proton-proton collider, $p+p\to t+t+X$ which establish a first parameter restriction from LHC, given by both 7TeV and 14TeV LHC data for 2HDM parameter space. We got a restriction for 2HDM parameters, $\tan\beta \gtrapprox 5\times 10^{-2}$ with $\beta-\alpha \lesssim \pi/3$.
We found that the upper limit is very loose as allows even the FV coupling parameter to be up to $\tilde{\chi}_{ij} \sim \mathcal{O}(10)$. Then, we focus on single top production processes, for electron-proton collider $l+p\to l'+t+X$ and use the bound given by HERA, although its energy reach is low. The future EIC collider will not have enough energy to produce a top quark as it would have variable center of mass energies from $\sim 20 \to \sim 100$ GeV, up gradable to $\sim 140$ GeV, \cite{Accardi:2012qut,AbdulKhalek:2021gbh}.

It is clear from the Feynman diagrams of the sub-processes that there is no counterpart within the SM, these are necessarily BSM processes.
From the structure of the analytical scattering amplitude and also corroborating with a numerical analysis which includes the PDFs, we found that the c-quark contribution dominates the process over the u-quark, because the masses are relevant directly on couplings than the valence nature of the quark.  

We also found that the process will not be sensible to the energy scale. There is no difference for the calculated cross section for an energy range from 1.3 TeV to 50 TeV.  Rather the luminosity will be the most important factor that may enhance the possibility of measure this exotic processes.

We extended our calculation to illustrate the enhancement in a muon-proton collider, because the particle masses are involved in the couplings. The $\mu q\to \mu t$ process most include the SM-type muon vertex and the FV contribution from 2HDM. The analytical structure differs from the doble flavor violation in lepton and quark couplings. We found in this case, that as the  complex structure of $\tilde{\chi}_{ij}$ is relevant, we explore negative sign possibilities obtaining an enhancement due to the fact that the SM-type coupling is negative. \\

 We showed the model parameter space dependence  for the  scattering cross section. Enhancing the cross section $\sigma(pp\to t t (\bar{t}\bar{t})+X)$ up to $\sim \mathcal{O}(10) pb$, for low $\tan\beta\sim \mathcal{O}(10^{-2})$ and for  $\cos(\alpha-\beta)\sim 1$. And  $\sigma(ep\to \mu t+X)\sim 10^{-5}pb$, for large $\tan\beta\approx 50$ and $\cos(\alpha-\beta)\approx 1$. The cross section is enhanced by two orders of magnitude when we consider muon-proton collider and tau involved.
The experimental bound on $\Gamma(t\to c h)$ from ATLAS and CMS restricts the values of the 2HDM FV parameters. We used the most restrictive experimental value which coming from CMS, restricting the values of $\cos(\beta-\alpha)\lesssim 0.35 $ considering $|\chi_{ij}|^2\approx 1$.
Even if the process is not measured, due to its small value, bounds on FV parameter of the model, $\tilde{\chi}_{ff'}$, are achieved as the analytical expression is directly proportional to this parameter.











%\cob
%{
%One ninth of the energy of the colliding protons should  be %at least twice the top mass in order to produce a top pair. %As each of the protons have three quarks which are the %possible partons to which the collision is produce at DIS.
%Experimental detection might be possible, especially for cases which enhance this FCNC
%as the highest center of mass energies... 
%For this situation the parameters $\vert\tilde{\chi}_{ij}^{u,d,l}\vert$ must be an order of magnitude equal about $\sim3$ which is a feasible result because those are free parameters free of the experiment.\\

 
%In fact it is the goal of our work to know which conditions would lead us to a  $k=1$ scenario.
%Se puede observar que en el caso kigualuno la seccion eficaz si podria ser detectada, como lo muestran los puntos rojor, para esta situacion los parametros ki deben ser un orden de magnitud igual a 3 lo cual es un resultado factible debido a que estos son parametros libres del experimento.Por otro lado, en el caso de kdif1 las quis y C tienen un orden de magnitud igual a uno mientras que W  uno igula a menos 13 por lo que k seria de un orden de magnitud de a la menos 11 lo que hace imperceptible a la seccion efizas total.
%For $\vert\tilde{\chi}_{12}^{f}\vert^{2}\sim10^{1}$ the cross section is on the ot herh and  too small and an experimental observation  appears hardly possible. It seems therefore not possible to  exclude this kind of scenarios through Deep Inelastic Scattering processes.
%\end{itemize}
%}

\section*{Acknowledgements}

We want to thank Cristian Baldenegro for his useful discussion on experimental results. W. Gonzalez gratefully acknowledges the scholarship from CONACyT which allowed development of the present work and also thankful to Dr. A. Rosado for for financial support at the project. Support by  Consejo Nacional de Ciencia y Tecnolog\'ia grant number
A1 S-43940 (CONACYT-SEP Ciencias B\'asicas) as well as funds assigned
by the Decanato de Investigaci\'on y Posgrado UDLAP are gratefully
acknowledged. 
MGB dedicate this work to Sandra Rodriguez Pe\~na, for her great animosity towards this work, and essentially towards life altogether.%We thank the organizers for inviting us to participate in this important eventc.%\vspace{1cm}


\appendix

%\section{Removed material}
%Combining Eqs.~\eqref{sigmaep} and \eqref{integral} we find {\bf check factors of $1/s$; for hadron hadron we have $1/s^2$, for the lepton-hadron only one factor of $1/s$. Dimensions suggest $1/s^2$.}
%\begin{small}\sigma^{eq}=\dfrac{\vert\tilde{\chi}_{12}^{l}\vert^{2}\vert\tilde{\chi}_{13}^{u}\vert^{2}}{16\pi(x's)^{2}}\dfrac{\cos^{4}(\alpha\,-\beta)}{\cos^{2}\beta\sin^{2}\beta}(3.67\times10^{-13})(4.83\times10^{6}) \label{sigmaeq}
%\begin{align}
%\sigma^{ep}(ep\rightarrow\,\mu\,t)&=\dfrac{1}{16\pi}\sum_{i}
%\int_{0}^{1}dx'\dfrac{f_{i}(x^{\prime},\tilde{Q}^{2})}{(x^{\prime}%\,s)^2} 
%\notag \\
%& \hspace{2cm} \cdot 
%\int_{t^{-}}^{t^{+}}dt\vert\mathscr{M}\vert^{2}
%\dfrac{\left[t-(m_{\mu}-m_{e})^{2}\right]\left[t-(m_{u}-m_{t})^{2}%%%\right]}{(t-m_{h}^{2})^{2}}.\label{sigmaT}
%\end{align}
%\end{small}

%For the following analysis we further define


%\section{Rare t-channel double top production in the  $p p$ collisions with FCNC}
%\cor{
%\begin{enumerate}
%    \item Production of two same sign tops at the LHC within the SM, background, the order of magnitude of these sub-processes that contribute to the cross section. 
 %   \item Production of two same sign tops at the LHC within the 2HDM-III
  %  \item Discussion and comparison of data.
%\end{enumerate}
%}




%Scalar extensions of the SM will also contribute to this process, in particular for models that allow FCNC, as is the 2HDM-III. 


%\cob{The cross section for double top production is the same as the cross section of double anti-top production, although the $uu$ contribution is different considering $uu$ or $\bar{u}\bar{u}$, this contribution is negligible, therefore both double production should be the same.} 
%%%%% Martin: I would not write this .... 

%\cob{Added by Martin:} Within the 2HDM-III model, double top and double anti-top production is obtained in the following way: With the incoming quarks partonic, {\it i.e.} massless we have for the scattering amplitude for the transition of two quarks of flavor $a, b$ to 2 top quarks the following expression
%\begin{align}
 %   |{\cal M}|^2(q_a q_b \to tt \to) & = \frac{g^4}{16} CM %|\chi_{a3}|^2  |\chi_{b3}|^2 \frac{[t-m_t^2]^2}{(t-m_h^2)^2},
%\end{align}
%where a similar expression holds for the transition of two antiquarks into antitops and 
%\begin{align}
%    C & = \frac{\cos^4(\alpha-\beta)}{\cos^2 \beta \sin^2 \beta} && \text{and} & M & = \frac{m_a m_b m_t^2}{m_W^2},
%\end{align}
%where we use for the factor $M$ the corresponding pdg values for quark masses. The partonic cross-section is then obtained as 
%\begin{align}
%    \hat{\sigma}_{q_a q_b \to tt}(\hat{s}) & = \frac{\pi^2}{2 %\hat{s}^2 (2 \pi)^3} \int_{t_-}^{t_+} d t \, |{\cal M}|^2(q_a q_b %\to tt ) ,
%\end{align}
%with 
%\begin{align}5    t_{\pm} = -\frac{\hat{s}}{2} \left(1 - 2 \frac{m_t^2}{\hat{s}} \mp \sqrt{1 - \frac{4 m_t^2}{\hat{s}}} \right).
%\end{align}




%\%section{more taken ouot}



%We see that the process $q\,q\to t\,t$ is not symmetrical when we consider the contributions $q=u, c$, because the $cc(\bar{c}\bar{c})$  contribution are the same, whereas the $uu(\bar{u}\bar{u})$ are not the same because the PDFs are not the same for $u$ and $\bar{u}$, nevertheless the up quark contribution can be neglected \cor{(is $\mathcal{O}(1pb)$)} as the charm quark contribution is greater and hence dominant. \cor{Then we safely consider, in this approximation, where the two process $qq(\bar{q}\bar{q}) \to tt(\bar{t}\bar{t})$ are symmetrical, they have the same probability. }
%However, this processes could appear in the context of FCNC in allowed models as the $2HDM-III$, see Figure (cc -> tt).
%Henceforth, our interest to calculate this process in t-channel to compare with analysis of experimental data. This rare process has the advantage to be a clear signal of physics beyond the SM and precision measurements for Yukawa couplings.


%In order to achieve an analysis that could include the LHC results we performed an approximated calculation in the following way. First, we calculate the t-channel of $\sigma(cc\to cc)$ in the SM using CalcHEP program; here we also confirmed that $\sigma(uu \to uu)<< \sigma(cc\to cc)$ . 
%Using this result for the kinematics, we change only the quark-Higgs couplings $cch^0$ by the flavor violating couplings from the THDM-III, $cth^0$. In this way we obtain an approximate value for the flavor violation sub-processes cross section $\sigma(cc \to t t)$ .   
%In order to isolate only the SM Higgs boson, we consider the other Higgs boson masses in the 2HDM-III larger than 1TeV.

%Because we are holding only to the kinematics, this result will be valid only when $pp$ energy collision would be much more larger than the top mass (at least two orders of magnitude higher), so the top mass could be neglected as we do it for the charm quark mass. This means we should be in energies higher than $14$TeV (lower energies would increase the error attached to this calculation), in order for this rough calculation to give justified results to compare with the LHC data.

%\cor{ 
%Firstly, we calculate the sub-process \cor{
%$c \, c \to c \,c$ or
%$c\bar{c}\to c\bar{c}$}
%in the context of the SM. We perform these calculations by using %the program CalcHEP , taking only the contribution of the diagram %in which a Higgs boson is exchanged in the t-channel, avoiding the %$Z$ contribution, to identify the scalar exchange process %exclusively. For the FCNC process 
%\cor{
%$cc\to tt$ ó
%$c\bar{c}\to t\bar{t}$}
% we compare with the dynamics of the prevous SM process %$c\bar{c}\to c\bar{c}$ with no FV and change the couplings %associated with FV. \cob{In order to identify de parameters of the %coupling in the SM}, we neglect the light fermion masses to keep %the kinematics structure (we do not change the PDFs). We then %change only the dynamical as given in the 2HDM-III Lagrangian %(\ref{lageigenstates}). Now, for $c \, c \to t \, t$ we calculate %the process for energies larger than 20 TeV, for which we can %neglect the top mass as well. As usual the light neutral Higgs %boson $h^{0}$ of the 2HDM-III is taken as the SM Higgs with %mass,$m_{h_{SM}} \sim 125$GeV.  The heavier neutral Higgs boson %$H^0$ and the pseudoscalar $A^0$ masses are consider to be larger %%than 1 TeV, in order to neglect these contributions.  
%Now, for $c \bar{c}\to t \bar{t}$ we calculate the process in %energies larger than 20 TeV, neglecting the top mass as well. So we %can use the results obtained for $cc \to cc$ in the frame of the SM %in order to calculate  $c \, c\to t \, t$, we only need to make the %appropriate change of the vertices of the SM for those of the 2HDM-%III with change of flavor. In this form we obtain a reasonable estimation of the process $pp\to t \, t$.
%}
 
 %\cor{[Nuestro interés en calcular estos procesos es para ver establecer la posibilidad de la detección experimental y su posible detección. Debido a los parámetros libres del modelo, se busca establecer de que ´rden deben ser estos parámetros para que sean detectables experimentalmente.]}
 
 %\begin{figure}[h]
 %    \centering
 %    \includegraphics{pptanB50.png}
 %    \caption{Caption}
 %    \label{fig:my_label}
 %\end{figure}
 
 
 
%\section{Rare t-channel single top production via $eP$ collisions with FCNC} 

%Se analiza el background de posibles señales  asociadas en el SM %considerando estados finales de lepton mas jets.

%\subsection{Single top in $ep$  collisions}
%\cor{[DESCRIBIR AQUI LOS PROCESOS DEL SM PARA UN SOLO TOP. PORQUE %ES UNA SEÑAL INTERESANTE.\\
%AUNQUE HAY PROCESOS DE DOS ANTI TOPS Y DOS TOPS, TOPS DEL MISMO %SIGNO.}\\
%%e also have the process $c\bar{c}\to t\bar{t}$
%We calculate the possible SM background although the number of %final particles gives a clear distinction. The alike SM processes are calculated using CalcHEP  and the PDF CT10 through a range of angle form $1^\circ$ to $179^\circ$  avoiding the central production, with the intitial particle momentum given as $80\, GeV$ and $7\, TeV$. The first row in table \ref{SMsigma} repres...

%The two same sign tops production is not possible within the SM. The possible processes in SM requires two pairs of tops and anti-tops as final state. The sub-process associated with the same initial state from proton-proton collision as the one we are studying are given in table \ref{SMsigma4tops}, where the dominant sub-process is from gluon-gluon.




%As the complete QCD calculus is cumbersome we perform an approximation to only obtain an order of magnitude of this possible process. By using CalcHep we estimate the value of the cross section $pp\to t \, t$ \cob{within the FV model 2HDM-III} as follows.\\

%Figure diagram..\\


%To describe the  cross-section  of electron $+$ proton $\to $ 2 fermions via neutral Higgs boson exchange, we make use of factorization of strong interaction matrix elements in the collinear limit, where  the hard scales $M \gg \Lambda_{\text{QCD}}$ is given by the mass of the produced top quark $M = m_t = 173.0$~GeV. We therefore find for  cross-section the following factorized expression,
%According Parton Model, the electron proton process can be described via the electron quark subprocess
%\begin{equation}
%\sigma(ep\to l' q')=\sum_{i}\int_{0}^{1}dxf_{i}(x,M^{2})%%\,\hat{\sigma}^{eq}, \label{sigmaep}
%\end{equation}\\
%\vspace{1cm},
%\int_{t^{-}}^{t^{+}}\dfrac{d\sigma}{dt}^{eq}dt
%where $\hat{\sigma}^{eq}$ denotes the partonic electron-quark ($eq$) cross sections with  $i$ the quark  flavor index and $f_{i}(x',M^{2})$  the Parton Distribution Function of quark with flavor $i$. For the actual implementation  we make use of of the pdf set MMHT2014 NLO 120 set \cite{pdf}. The partonic cross-section $\hat{\sigma}^{eq}$  is given
%\begin{equation}
%\hat{\sigma}^{eq}=\dfrac{1}{16\pi(x^{\prime}s)^{2}}%\int_{t^{-}}^{t^{+}}\vert\mathcal{M}(t)\vert\,^{2}dt \label{integral},
%\end{equation}
%where $\vert\mathcal{M}(t)\vert\,^{2}$ is the squared matrix element
%of the process $ep \to l'q'$ and $x$ denote the proton momentum
%fraction.  It is important to keep in mind that we are looking for a
%single top production via neutral Higgs boson exchange (but with
%flavor exchange) at tree level. The reaction of interest process is
%therefore $ep\rightarrow\,\mu\,t$; for the  relevant Feynman Diagram  see Fig.~\ref{fig:mesh1}\\

%%%%%%%%%%%%%%%%%%%%


%To obtain the corresponding terms for down quarks, we perform the substitution $l\,\rightarrow\,d$ in Eq.~\eqref{fvl}. Hence for the case shown in the Fig.~\ref{fig:mesh1} we obtain:
%\begin{small}
%\begin{equation}
%\vert\,\mathscr{M}\vert\,^{2}=\dfrac{1}{\left[(p-p')^{2}-m_{h}^{2}%\right]^{2}}\mathscr{M}_{l}^{\theta\nu}\mathscr{M}_{\theta\nu}^{q} ,
%\end{equation}
%\end{small}   
%with
%\begin{align}
%\mathscr{M}_{l}^{\theta\nu} & =\displaystyle\sum_{s}\bar{\mu}(p')\left[\frac{g}{2}\dfrac{\cos{(\alpha-\beta)}}{\sqrt{2}\cos{\beta}}\dfrac{\sqrt{m_{e}m_{\mu}}}{m_{W}}\tilde{\chi}_{12}^{l}\right]e(p)\bar{e}(p)\left[\dfrac{g}{2}\dfrac{\cos{(\alpha-\beta)}}{\sqrt{2}\cos{\beta}}\dfrac{\sqrt{m_{e}m_{\mu}}}{m_{W}}\left(\tilde{\chi}_{12}^{l}\right)^{\ast}\right]\mu\,(p'),\\
%\mathscr{M}^{q}_{\theta\nu} & =\displaystyle\sum_{s^{\prime}}\bar{t}(q')\left[-\dfrac{g}{2}\,\dfrac{\cos{(\alpha-\beta)}}{\sqrt{2}\sin{\beta}}\dfrac{\sqrt{m_{u}m_{t}}}{m_{W}}\tilde{\chi}_{13}^{u}\right]u(q)\bar{u}(q)\left[-\dfrac{g}{2}\,\dfrac{\cos{(\alpha-\beta)}}{\sqrt{2}\sin{\beta}}\dfrac{\sqrt{m_{q}m_{t}}}{m_{W}}\left(\tilde{\chi}_{13}^{u}\right)^{\ast}\right]t(q').
%\end{align}

%By simplifying, we have
%\begin{small}
%\cor{REVISAR\\
%\begin{equation}
%\vert\mathscr{M}\vert^{2}=\dfrac{g^{4}}{16}CM\vert\tilde{\chi}_{12}^{l}\vert^{2}\vert\tilde{\chi}_{13}^{u}\vert^{2}\dfrac{\left[t-(m_{\mu}-m_{e})^{2}\right]\left[t-(m_{u}-m_{t})^{2}\right]}{(t-m_{h}^{2})^{2}},
%\end{equation}
%where
%\begin{eqnarray}
%C=4\dfrac{\cos^{4}(\alpha\,-\beta)}{\sin^{2}(2\beta)}\,\,\,& \text{and}&\,\,\,
%M=\dfrac{m_{e}m_{\mu}m_{u}m_{t}}{m_{W}^{4}}.
%\nonumber
%\end{eqnarray}
%Combining Eqs.~\eqref{sigmaep} and \eqref{integral} we find
%\begin{small}\sigma^{eq}=\dfrac{\vert\tilde{\chi}_{12}^{l}\vert^{2}\vert\tilde{\chi}_{13}^{u}\vert^{2}}{16\pi(x's)^{2}}\dfrac{\cos^{4}(\alpha\,-\beta)}{\cos^{2}\beta\sin^{2}\beta}(3.67\times10^{-13})(4.83\times10^{6}) \label{sigmaeq}
%\begin{align}
%\sigma^{ep}(ep\rightarrow\,\mu\,t)&=\dfrac{1}{16\pi}\sum_{i}
%\int_{0}^{1}dx'\dfrac{f_{i}(x^{\prime},\tilde{Q}^{2})}{x^{\prime}\,s} 
%\notag \\
%& \hspace{2cm} \cdot 
%\int_{t^{-}}^{t^{+}}dt\vert\mathscr{M}\vert^{2}
%\dfrac{\left[t-(m_{\mu}-m_{e})^{2}\right]\left[t-(m_{u}-m_{t})^{2}%\right]}{(t-m_{h}^{2})^{2}}.\label{sigmaT}
%\end{align}
%\end{small}

%For the following analysis we further define
%where
%\begin{equation}
%t_{\pm}=m_e^2+m_{\mu}^2-\frac{1}{2\hat{s}}\left[ (\hat{s}+m_e^2-m_{u}^2)(\hat{s}+m_{\mu}^2-m_{t}^2)\mp \lambda^{1/2}(\hat{s},m_e^2,m_{u}^2)\lambda^{1/2}(\hat{s},m_{\mu}^2,m_{t}^2)\right]
%k=\dfrac{g^{4}}{256\pi}CM\vert\tilde{\chi}_{12}^{l}\vert^{2}\vert\tilde{\chi}_{13}^{u}\vert^{2},
%\end{equation} 
%with the usual defined function
%$\lambda(x,y,z)=(x-y-z)^2-4yz$ and $\hat{s}=x's$ for $0<x'<1$; and in the high energy limit $\lambda(x,y,z)\approx x's$
%which collects the model dependent coupling constants.
%\\
 %%or explicitly\\ 
%\vspace{0.3cm}
%
%\begin{equation}
%k=\dfrac{\cos^{4}(\alpha\,-\beta)}{\cos^{2}\beta\sin^{2}\beta}\dfrac{m_{e}m_{\mu}m_{u}m_{t}}{m_{W}^{4}}\vert\tilde{\chi}_{12}^{l}\vert^{2}\vert\tilde{\chi}_{13}^{u}\vert^{2} \label{k}
%\end{equation}
%
%\cob{ REESCRIBIR\\}
%\cor{especificar el umbral cinem[atico 1.3TeVs, se pone a mano porque las masas de las particulas finales e iniciales se esta aproximando a cero. Solo se considera completamente la parte din[amica}

 
 
\bibliographystyle{JHEP} 
%\bibliographystyle{unsrt}
\bibliography{references}


%\addbibresource{referencias.bib}
%\printindex
%\printbibliography



\end{document}
