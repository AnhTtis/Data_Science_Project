\section{Perceptual Study}
In Figure \ref{fig:user_UI} we show an example of our survey question. 
The input video is on the left; the middle and the right columns are two edited results for comparison, one of which is \methodname while the other one is randomly chosen from 4 baselines: per-frame editing, Jamriska et al.~\cite{Jamriska19-SIG}, 
Text2Live \cite{bar2022text2live} and a concurrent method Tune-a-Video \cite{wu2022tuneavideo}.
Two edited results are placed randomly in the middle or the right to avoid any bias.
We ask two questions: (i)~Which one better represents the prompt (shown on the top)? (ii)~Which one do you prefer?
In the second question we do not ask users to pay particular attention to any attribute, e.g., temporal smoothness or realism, as we aim to evaluate generally the perceptual quality of a video.

% Fig.~\ref{fig:user_ratio} shows how often a method is chosen when presented together with other methods. 

\begin{figure}[b]
     \centering
    % \begin{subfigure}[t]{\columnwidth}
         \includegraphics[width=\columnwidth]{figures/user_format.png}
         % \caption{Question design}
         
     % \end{subfigure}     
     % \begin{subfigure}[t]{\columnwidth}
     %     \centering
     %     \includegraphics[trim={0 0 0 -0.5cm},clip,width=\linewidth]{figures/user_overall.png}
     %     \caption{Chosen frequency (\%) := obtained votes / total occurrence}
     %     \label{fig:user_ratio}
     % \end{subfigure}     
     \caption{
     \textbf{Question design of the user evaluation. }
        }
        \label{fig:user_UI}
\end{figure}