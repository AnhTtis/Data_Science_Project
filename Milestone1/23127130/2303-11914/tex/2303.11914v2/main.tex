%%%%%%%%%%%%%%%%%%%%%%%%%%%%%%%%%%%%%%%%%%%%%%%%%%%%%%%%%%%%%%%%%%%%%%%%
%    INSTITUTE OF PHYSICS PUBLISHING                                   %
%                                                                      %
%   `Preparing an article for publication in an Institute of Physics   %
%    Publishing journal using LaTeX'                                   %
%                                                                      %
%    LaTeX source code `ioplau2e.tex' used to generate `author         %
%    guidelines', the documentation explaining and demonstrating use   %
%    of the Institute of Physics Publishing LaTeX preprint files       %
%    `iopart.cls, iopart12.clo and iopart10.clo'.                      %
%                                                                      %
%    `ioplau2e.tex' itself uses LaTeX with `iopart.cls'                %
%                                                                      %
%%%%%%%%%%%%%%%%%%%%%%%%%%%%%%%%%%
%
%
% First we have a character check
%
% ! exclamation mark    " double quote  
% # hash                ` opening quote (grave)
% & ampersand           ' closing quote (acute)
% $ dollar              % percent       
% ( open parenthesis    ) close paren.  
% - hyphen              = equals sign
% | vertical bar        ~ tilde         
% @ at sign             _ underscore
% { open curly brace    } close curly   
% [ open square         ] close square bracket
% + plus sign           ; semi-colon    
% * asterisk            : colon
% < open angle bracket  > close angle   
% , comma               . full stop
% ? question mark       / forward slash 
% \ backslash           ^ circumflex
%
% ABCDEFGHIJKLMNOPQRSTUVWXYZ 
% abcdefghijklmnopqrstuvwxyz 
% 1234567890
%
%%%%%%%%%%%%%%%%%%%%%%%%%%%%%%%%%%%%%%%%%%%%%%%%%%%%%%%%%%%%%%%%%%%
%
\documentclass[10pt]{iopart}
%Uncomment next line if AMS fonts required
\usepackage{iopams}  
\usepackage{cite}
\usepackage{hyperref}
\expandafter\let\csname equation*\endcsname\relax
\expandafter\let\csname endequation*\endcsname\relax
\usepackage{amsmath}
%\usepackage{eqnarray}
\usepackage{txfonts}
%\usepackage{orcidlink}
\bibliographystyle{iopart-num}

\begin{document}

\title{Stronger EPR-steering criterion based on inferred Schr\"odinger-Robertson uncertainty relation}

\author{Laxmi Prasad Naik$^{1,}$$^2$, Rakesh Mohan Das$^3$ and Prasanta K. Panigrahi$^2$}

\address{$^1$ Indian Institute of Science Education And Research Kolkata, Mohanpur, Nadia - 741 246, West Bengal, India}
\address{$^2$ Indian Institute of Technology Delhi, Hauz Khas - 1100016, New Delhi, India}
\address{$^3$ Kalinga Institute of Industrial Technology, Bhubaneswar - 751024, Odisha, India}
\ead{laxmiprasadnaik5897@gmail.com,rakesh.dasfpy@kiit.ac.in and pprasanta@iiserkol.ac.in}
\vspace{10pt}
%\begin{indented}
%\item[]January 2023
%\end{indented}

\begin{abstract}
Steering is one of the three in-equivalent forms of nonlocal correlations intermediate between Bell nonlocality and entanglement. Schr\"odinger-Robertson uncertainty relation (SRUR), has been widely used to detect entanglement and steering. However, the steering criterion in earlier works, based on SRUR, did not involve complete inferred-variance uncertainty relation. In this paper, by considering the local hidden state model and Reid’s formalism, we derive a complete inferred-variance EPR-steering criterion based on SRUR in the bipartite scenario. %in one-sided, two-measurement and two-outcome scenarios.% 
Furthermore, we check the effectiveness of our steering criterion to discrete variable bipartite two qubit, two qutrit and two ququart isotropic states.
\end{abstract}

%
% Uncomment for keywords
\vspace{2pc}
\noindent{\it Keywords}: EPR-steering, Schr\"odinger-Robertson uncertainty relation
%
% Uncomment for Submitted to journal title message
%\submitto{\JPA}
%
% Uncomment if a separate title page is required
%\maketitle
% 
% For two-column output uncomment the next line and choose [10pt] rather than [12pt] in the \documentclass declaration
\ioptwocol
\section{Introduction}
EPR-steering is a nonlocal correlation intermediate between Bell nonlocality and quantum entanglement \cite{einstein_1935,einstein_1936,wiseman2007steering}. It is the ability to remotely affect or \emph{steer} a shared entangled quantum state by a single party's (say Alice's) arbitrary choice of local measurements without violating the no-signalling principle \cite{simon2001no}. Wiseman \textit{et al.} gave an operational definition of steering as a task between Alice and Bob. Alice prepares an entangled state and sends one part to Bob. Here, Bob does not trust Alice, and by performing local measurements, she has to convince him that the state is entangled \cite{wiseman2007steering}. If Bob's steered quantum state cannot be explained by a local hidden state (LHS) model, then the state is said to exhibit steering. In contrast to Bell's nonlocality and entanglement, steering demonstrates asymmetric behaviour in which one party can steer the other party, but vice versa is not always permitted \cite{midgley2010asymmetric,bowles2014one,bowles2016sufficient,reid2013monogamy}. Moreover, not every entangled state exhibits steering, and not every steerable state violates Bell inequality \cite{wiseman2007steering}.
EPR-steering has a wide range of applications in many quantum information processing tasks e.g. in one-sided device-independent quantum key distribution \cite{branciard2012one,gehring2015implementation,walk2016experimental}, quantum networking tasks \cite{Huang2018SecuringQN,Armstrong2014MultipartiteES,Cavalcanti2014DetectionOE}, subchannel discrimination \cite{Piani2015NecessaryAS,Chen2016NaturalFF, Sun2018DemonstrationOE}, quantum secret sharing \cite{Kogias2016UnconditionalSO, Xiang2016MultipartiteGS}, quantum teleportation \cite{Reid2013SignifyingQB,fan2022quantum}, randomness certification \cite{passaro2015optimal,curchod2017unbounded,Mattar2016ExperimentalME}, and random number generation \cite{joch2022certified} to mention a few. Recently it has also been demonstrated to be a useful resource in noisy and lossy quantum network systems \cite{Qu2021RetrievingHQ,Srivastav2022QuickQS}.\newline\newline
Effective detection of steering exhibited by quantum states is crucial to realise applications of steerable quantum states. Uncertainty relations (UR) can be experimentally verified because it involves measurement of observables. Assuming the description of quantum mechanics is correct, EPR's condition of locality and sufficient condition of reality are satisfied, UR's become an important tool for determining steering criteria. Many criteria in this direction e.g., using the Heisenberg uncertainty relation (HUR) \cite{reid1989demonstration} and later involving a broader class of uncertainty relations, have been proposed \cite{bialynicki1975uncertainty, Deutsch1983UncertaintyIQ,Chowdhury2013EinsteinPodolskyRosenSU,schneeloch2013einstein}. Additionally, under different measurement scenarios, more optimal steering criteria \cite{pramanik2014fine, maity2017tighter} were obtained using fine-grained uncertainty relations \cite{Oppenheim2010TheUP, Chowdhury2015StrongerSC} and sum uncertainty relations \cite{Maccone2014StrongerUR, maity2017tighter}. \newline \newline
The criterion for experimental demonstration of steering was first proposed by Reid \cite{reid1989demonstration}, which is based on inferred-variances. Recently, a steering criterion using Schr\"odinger-Robertson uncertainty relation (SRUR) was also proposed. However, these earlier works did not involve the inferred-means in the lower bound \cite{sasmal2018tighter}. A recent work involves inferred-variance based product and sum uncertainty relations in the presence of entanglement \cite{bagchi2022inferred}. We aim to derive a steering criterion based on SRUR involving inferred-means and inferred-variances following up the analysis in \cite{cavalcanti2009experimental}. \newline\newline
SRUR is a generalized relation and reduces to HUR when the covariance between two operators is zero, which is not always true. Hence, SRUR provides a stronger bound than HUR. It is shown that SRUR, using positive partial transpose provides a stricter entanglement condition \cite{NhaPhysRevA.76.014305,goswami2017uncertainty} for non-Gaussian entangled states\cite{NhaKimPhysRevA.74.012317} and covariance matrix entanglement criterion (which can be formulated in the form of SRUR) for Gaussian entangled states\cite{SimonPhysRevLett.84.2726}, which was later generalized by V Tripathi \emph{et al} \cite{Tripathi_2020}. \newline\newline
The states satisfying the minimum condition of HUR are called ordinary intelligent states (OIS) and the states satisfying the minimum condition of SRUR are called generalized intelligent states (GIS). The application of OIS and GIS are studied in the context of squeezing in quantum precision measurements \cite{OIS&GISPhysRevA.76.053834}.\newline\newline
In the next section, we briefly discuss steering and the EPR-Reid criterion. In Sec.3, we derive a steering criterion based on the SRUR. We check the efficiency of the steering criterion in Sec.4, using it on two-qubit, two-qutrit and two-ququart isotropic states for which a steering inequality is obtained. The paper ends with a conclusion and an appendix.
\section{Preliminaries}
\subsection{EPR-Steering}\label{EPR Steering}
Consider a general unfactorizable bipartite pure state shared by two distant parties, Alice and Bob
\begin{eqnarray}\label{eqn1}
    |\Psi\rangle = \sum_{n} c_n|u_n\rangle|v_n\rangle = \sum_{n} d_n|\psi_n\rangle|\phi_n\rangle
\end{eqnarray}
where, $\{|u_n \rangle \}(\{|\psi_n\rangle\})$ and $\{|v_n \rangle \}(\{|\phi_n\rangle\})$ denote two different orthonormal bases in Alice’s and
Bob’s system, respectively. This property of inseparability is called entanglement, which is one of the most useful resources in quantum information processing that has been studied extensively in the literature \cite{horodecki,Bhaskara_QINP_2017,roy2021geometric,mahanti2022classification,mishra2022geometric}. In this scenario, Alice chooses to measure in the $|u_n\rangle\left(|\psi_n\rangle\right)$ basis, then Bob's state will be projected into $|v_n\rangle (|\phi_n\rangle)$ basis. The ability of Alice to influence (steer)
Bob’s state, nonlocally, was termed as steering by Schr\"odinger \cite{einstein1935can,einstein_1935,einstein_1936}.\newline\newline
Consider the following situation, Alice and Bob share an entangled quantum state, described by density matrix $\hat{\rho}$. The generalised local measurements of Alice and Bob are denoted by ${\hat{M}_{a|A}}$ and ${\hat{M}_{b|B}}$ ($M_{a(b)|A(B)} \geq 0, \sum_{a(b)}M_{a(b)|A(B)} = 1 \hspace{3mm}\forall A(B)$), where $a$ and $b$ denote the outcomes corresponding to the measurement operators $\hat{M}_{a|A}$ and $\hat{M}_{b|B}$. $A$ and $B$ are Alice's and Bob's measurement settings, respectively. The quantum probability of their joint measurement is given as follows
\begin{eqnarray}\label{eqn2}
    P(a,b) = \textnormal{Tr}[\hat{\rho}(\hat{M}_{a|A} \otimes \hat{M}_{b|B})]
\end{eqnarray}
where, $P(a,b)$ is the joint probability of obtaining outcomes $a$ and $b$. If and only if for all measurements $\hat{M}_{a(b)|A(B)}$, the joint probability distribution $P(a,b)$ for Alice's and Bob's measurements can be explained using a LHS model for Bob and a local hidden variable (LHV) model for Alice, then Alice and Bob would fail to demonstrate steering i.e. the joint probability distributions can be written as
\begin{eqnarray}\label{eqn3}
    P(a,b) = \sum_{\eta}p(\eta)P(a|\eta)P_{Q}(b|\eta)
\end{eqnarray}
where $\eta$ is a local hidden variable having probability distribution $p(\eta)$, satisfying $p(\eta) \geq 0$ and $\sum_{\eta}p(\eta) = 1$. $P(a|\eta)$ is the probability distribution for outcome $a$ determined by the local hidden variable $\eta$ and $P_{Q}$ is the quantum probability distribution for outcome $b$; $P_{Q}(b|\eta)= \textnormal{Tr}_{B}[(\hat{M}_{b|B}\hat{\rho}_{\eta}]$ ($Q$ stands for quantum), corresponding to a local hidden quantum state described by $\hat{\rho}_{\eta}$, which is unaffected by local measurements of Alice. The use of LHS to explain steering is a clear implication of the consistency of EPR's condition of locality. Any constraint that can be obtained obeying \eref{eqn3} is called an \emph{EPR-steering criterion}, violation of which will demonstrate steering. The joint probability distribution and the state is said to admit an LHS model if \eref{eqn2}) can be expressed having a decomposition of the form \eref{eqn3} for all the choice of Alice's and Bob's measurements respectively. \newline\newline
It is important to note that Alice and Bob perform single measurements in each run of experiment on the identically prepared states. And, Alice arbitrarily chooses to perform different measurements in each of the runs of the experiment. The probability of the result of a measurement is independent of the other measurement results. Mathematically, the probability $P(a_{i}|a_{j}) = P(a_{i})$ and $P(a_{j}|a_{i}) = P(a_{j})$ where $a_{i}$ and $a_{i}$ are the results which correspond to different single measurements performed in each run by Alice. \newline\newline
The interpretation can be given in terms of Bob's reduced state i.e. Bob's state assemblages. Prior to all experiments, Bob asks Alice to announce the set of possible ensembles of states into which Alice would like to project Bob's state into i.e. $\{E^{\hat{A}}: \forall \hat{A} \}$, where $\hat{A}$ stands for Alice's operators. \newline \newline
For every run of the experiment, Alice prepares an entangled state and sends one part to Bob. Bod randomly picks an ensemble $E^{\hat{A}}$ from the announced set and asks Alice to prepare it. Alice performs the measurement $\hat{A}$ and announces to Bob corresponding to the result $a$ about his collapsed state $\hat{\sigma}^{\hat{A}}_{a}$ that she has prepared. This experiment is performed over many runs and Bob confirms the probability $\textnormal{Tr} [\hat{\sigma}^{\hat{A}}_{a}]$ of its occurrence.\\ \\
Now Alice can cheat Bob by adopting a strategy. In this strategy, instead of sending one part of an entangled state to Bob, Alice picks a state at random from some prior ensemble of states $\textnormal{R} = \{p_{\eta}\sigma_{\eta}\}$ with $\sigma = \sum_{\eta}p_{\eta}\sigma_{\eta}$. Now Alice announces to Bob about the state that is prepared from her knowledge of $\eta$ and any stochastic map from $\eta$ to $a$, where $a$ is one of the results of the measurement $\hat{A}$ which Bob had picked from the announced ensemble. Alice will fail to convince Bob that the state sent is entangled if and only if, Alice, for all her choice of measurements $\hat{A}$ and for all eigenvalues of $\hat{A}$ one can find an ensemble R and a stochastic map from $\eta$ to $a$ such that the Bob's state can be explained by local hidden state model,
\begin{eqnarray}\label{eqn4}
    \hat{\sigma}_{a}^{\hat{A}} = \sum_{\eta}p_{\eta}p(a|A,\eta)\sigma_{\eta}
\end{eqnarray}
For the state assemblages of Bob, if Alice cannot find an ensemble R i.e., a local hidden state model $\sigma_{\eta}$ and a stochastic map $p(a|\hat{A},\eta)$ then the state is steerable. Alice cannot affect Bob's unconditioned state $\textnormal{Tr}_{A}[\hat{\rho}]$, because that would violate superluminal communication \cite{simon2001no}.
\newline\newline
In this strategy, we observe that in each run of the experiment, while Bob picks one of the ensembles $E^{\hat{A}}$ announced by Alice and asks her to announce back to Bob about his collapsed state. Alice performs this step of the experiment through a stochastic map from $\eta$ to a, corresponding to one of the results a of the measurement $\hat{A}$. The choice of which outcome of the measurement should be mapped from $\eta$ to a, is completely arbitrary and random, i.e. $P(a_{i}|a_{j}) = P(a_{i})$, $\forall a_{i},a_{j}$ corresponding to different results of Alice's different measurements.\newline\newline
In a local hidden state model , the choices of measurements of Alice i.e. $\hat{A}$ to infer the values of the corresponding measurements of Bob are arbitrary. The reason being that, Alice’s probabilities are allowed to depend arbitrarily on the variables $\eta$.
\newline\newline
The local hidden state model description of steering uses the property of states. However, in an experimental situation one is not concerned about what type of state is used to demonstrate steering. We only rely on the measured data. As in the case of bell nonlocality and entanglement detection techniques they just rely on the measured data rather than the state properties i.e local hidden variables and local hidden states respectively. Therefore an experimental EPR steering criterion should not depend on any assumption about the type of state being prepared rather depend only on the measured data.
\newline\newline
Alice can attempt to infer different outcomes of Bob corresponding to different observables. Assuming that the LHV-LHS model. Since Bob's state corresponds to a local hidden quantum state, uncertainty relations can be used for Bob's measurements. This was first realized by Reid \cite{reid1989demonstration}, who proposed an experimental EPR-steering criterion using HUR in continuous variable systems. Therefore we aim to derive an EPR-steering criterion using SRUR because it involves the covariance of the observables, which captures stronger correlations.
\subsection{EPR-Reid criterion}\label{EPR Reid}
Reid proposed a modified version of EPR's sufficient condition of reality, which states that if without in any way disturbing a system, we can predict with some specified uncertainty the value of a physical quantity, there exists a stochastic element of physical reality which determines this physical quantity with atmost that specific uncertainty, called as \emph{Reid's extension of EPR's sufficient condition of reality}. This is attributed to the intrinsic stochastic nature exhibited in the preparation and detection of quantum states \cite{reid1989demonstration,cavalcanti2009experimental}.\newline\newline
Consider two parties, Alice and Bob sharing an entangled state. Now Alice makes a local measurement $\hat{Y}$ and makes an estimate $\hat{X}^{est}(\hat{Y})$ for the result of Bob's measurement $\hat{X}$ observing the outcomes of her own measurement $\hat{Y}$. The idea of estimation is implemented to incorporate EPR's sufficient condition of reality. Therefore the average inferred-variance of $\hat{X}$ for an estimate $\hat{X}^{est}(\hat{Y})$ is given as follows
\begin{eqnarray}\label{eqn5}
    \Delta_{inf}^{2}\hat{X}^{2} = \langle(\hat{X} - \langle\hat{X}^{est}(\hat{Y})\rangle)^{2}\rangle.
\end{eqnarray}
Alice's estimate for Bob's measurement is given by $\hat{X}^{est}(\hat{Y}) = g\hat{Y}$, where the choice of $g$ should be such that it gives the minimum error, i.e., $g = \frac{\langle\hat{X}\hat{Y}\rangle}{{\langle \hat{Y}^{2} \rangle}}$ gives the optimal inferred-variance. Using EPR's condition of locality, Reid's extension of EPR's sufficient condition for reality and completeness of quantum mechanics, a limit on the product of inferred-variances based on HUR for two noncommuting quadrature phase amplitude observables $\hat{X}_{1}$ and $\hat{X}_{2}$ on Bob's side is \cite{reid1989demonstration}
\begin{eqnarray}\label{eqn6}
\Delta_{inf}^{2}\hat{X}_{1}\Delta_{inf}^{2}\hat{X}_{2} \geq 1.
\end{eqnarray}
This is known as \emph{EPR-Reid criterion}. A state will show steering if \eref{eqn7} is violated, which has also been verified experimentally \cite{cavalcanti2009experimental}.
\section{EPR-steering criterion using Schr\"odinger-Robertson uncertainty relation}
Our derivation of EPR-steering is based on the works of \cite{reid1989demonstration,cavalcanti2007uncertainty,cavalcanti2009experimental}. Here, we use a different notation for the outcomes of measurement. Consider the outcomes $A$ and $B$, corresponding to observables $\hat{A}$ and $\hat{B}$, for Alice's and Bob's measurements respectively. \newline \newline 
Alice and Bob perform the measurements $\hat{A}$ and $\hat{B}$ respectively. Here, Alice tries to infer the outcome of Bob's measurement based on the result of her measurement. $B^{est}(A)$ is Alice's estimate of Bob's measurement outcome based on her outcome $A$.
Using the EPR-Reid criterion the inferred-variance is written as
\begin{eqnarray}\label{eqn7}
    \Delta_{inf}^{2}B = \langle\left(B - B^{est}(A)\right)^{2}\rangle.
\end{eqnarray}
The average is calculated by taking the average over all outcomes $A$ and $B$. The estimates involving different measurements is done following the EPR Reid criterion, by conducting many runs of the experiment, where each run involves a single pair of measurement on Alice's and Bob's side. This scenario is similar to the detection of Bell nonlocality and entanglement. The inferred-variance $\Delta_{inf}^{2}B$ is minimized (optimized) when $B^{est}(A) = \langle B \rangle_{A}$. So the minimized inferred-variance $ \Delta^{2}_{min}B$ is as follows
\begin{eqnarray}\label{eqn8}
     \Delta^{2}_{min}B &= \langle (B  - \langle B 
 \rangle_{A})^{2}\rangle = \sum_{A,B} P(A,B)(\langle B - \langle B \rangle_{A})^{2} \nonumber \\
                            &= \sum_{A} P(A) \sum_{B} P(B|A)(B - \langle B \rangle_{A})^{2} \nonumber \\
                            &= \sum_{A} P(A) \Delta^{2} (B|A).
\end{eqnarray}
where, $\Delta^{2} (B|A)$ is calculated from the conditional probability distribution $P(B|A)$, stands for the conditional variance of Bob's measurement outcome $B$ provided, the outcome A of Alice's measurement is known. So we have the following condition
\begin{eqnarray}\label{eqn9}
    \Delta^{2}_{inf}B \geq \Delta^{2}_{min}B.
\end{eqnarray}
Assuming that the statistics of the experimental outcomes of Alice's and Bob's measurements can be described by a LHS model \eref{eqn3}, so assuming that model
the conditional probability distribution $P(B|A)$ can be written as follows
\begin{eqnarray}\label{eqn10}
    P(B|A) &= \frac{P(A,B)}{P(A)} = \sum_{\eta} \frac{P(\eta) P(A|\eta)}{P(A)} P_{Q}(B|\eta) \nonumber \\
                       &= \sum_{\eta} P(\eta|A)P_{Q}(B|\eta)
\end{eqnarray}
Here, $\eta$ is a classical random variable such that, $P(\eta) \geq 0$ and $\sum_{\eta}P(\eta) = 1$. Moreover, we can observe that the basic essence of adopting the LHS model is statistical independence of probabilities, which is one of the most important prescriptions in the local hidden variable (LHV) theory by Bell \cite{bell1964einstein}. If $P(u)$ is a classical probability distribution, which has a convex decomposition i.e. $P(u) = \sum_{v} P(v)P(u|v)$, then the variance $\Delta^{2} u$ corresponding to the probability distribution $P(u)$ is bounded by the average of the variances $\Delta^{2}(u|v)$ over the conditional distribution $P(u|v)$, i.e. $\Delta^{2} u \geq \sum_{v}P(v)\Delta^{2}(u|v)$. Therefore, from \eref{eqn9}, the variance of the conditional measurement outcomes $B|A$ is given as 
\begin{eqnarray}\label{eqn11}
    \Delta^{2}(B|A) \geq \sum_{\eta}P(\eta|A)\Delta^{2}_{Q}(B|\eta
    )
\end{eqnarray}
where, the variance $\Delta^{2}_{Q}(B|\eta)$ is calculated using the conditional quantum probability distribution $P_{Q}(B|\eta) = \textnormal{Tr}[\hat{B}\hat{\rho}_{\eta}]$. The average of the measurement operator $\hat{B}$, specified by its outcome $B$ is calculated corresponding to a local quantum hidden state described by $\hat{\rho}_{\eta}$. Therefore the bound for $\Delta_{min}^{2}B$, using \eref{eqn9},\eref{eqn12} is given by

\begin{eqnarray}\label{eqn12}
\Delta^{2}_{min}B &\geq \sum_{A} P(A) \Delta^{2} (B|A) \nonumber \\ &\geq \sum_{A} P(A) \sum_{\eta}P(\eta|A)\Delta^{2}_{Q}(B|\eta) \nonumber \\ &\geq \sum_{A,\eta} P(A,\eta)\Delta^{2}_{Q}(B|\eta) \nonumber \\ &\geq \sum_{\eta} P(\eta)\Delta^{2}_{Q}(B|\eta).
\end{eqnarray}
Consider Bob's arbitrary local measurement operators $\hat{B_{1}},\hat{B}_{2}$ with their corresponding outcomes $B_{1}, B_{2}$. These operators then satisfy the SRUR \cite{robertson1934indeterminacy}
\begin{eqnarray}\label{eqn13}
\langle\Delta^{2}\hat{B}_{1}\rangle \langle\Delta^{2} B_2\rangle \geq \frac{1}{4}\left|\langle[\hat{B}_{1},\hat{B}_{2}]\rangle\right|^2 \nonumber + \\
\hspace{3cm} \left(\frac{\langle\{\hat{B}_1,\hat{B}_2\} \rangle}{2} - \langle \hat{B}_1\rangle \langle \hat{B}_2 \rangle\right)^2
\end{eqnarray}
where, $\{\hat{B_1},\hat{B_2}\}$ is the anticommutator, $[\hat{B}_{1},\hat{B}_{2}]$ is the commutator.  $\langle\Delta_{Q}^{2}\hat{B_i}\rangle_{\hat{\rho}}$ is the variance and $\langle\hat{B_i}\rangle_{\hat{\rho}}$ is the average calculated for a quantum state. The above eqnarray can be written in terms of the outcomes of Bob given by
\begin{eqnarray}\label{eqn14}
    \langle\Delta_{Q}^{2}B_{1}\rangle \langle\Delta_{Q}^{2} B_2\rangle \geq \frac{1}{4}\left|\langle[B_{1},B_{2}]\rangle_{Q}\right|^{2} + \nonumber \\ \hspace{2.5cm} \left(\frac{\langle\{\hat{B}_1,\hat{B}_2\} \rangle_{Q}}{2} - \langle \hat{B}_1\rangle_{Q} \langle \hat{B}_2 \rangle_{Q}\right)^2.
\end{eqnarray}
For any two vectors $\textbf{u}$ and $\textbf{v}$ in a linear vector space, the Cauchy-Schwartz inequality is given by
\begin{eqnarray}\label{eqn15}
 ||\textbf{u}||^{2}_{2}||\textbf{v}||^{2}_{2} \geq |\langle\textbf{u},\textbf{v}\rangle|^2   
\end{eqnarray}
where $||.||_{2}$ is L2 norm, $\langle.\rangle$ is inner product and $|.|$ is the modulus in the linear vector space.
Using \eref{eqn12} the  vectors \textbf{u} and \textbf{v} can be defined as
\begin{eqnarray}\label{eqn16}
   \textbf{u} &\equiv \{\sqrt{P(\eta_1)}\Delta_{Q}(B_1|\eta_1), \sqrt{P(\eta_2)}\Delta_{Q}(B_1|\eta_2), ...\} \nonumber \\
   \textbf{v} &\equiv \{\sqrt{P(\eta_1)}\Delta_{Q}(B_2|\eta_1), \sqrt{P(\eta_2)}\Delta_{Q}(B_2|\eta_2), ...\}.
\end{eqnarray}
From \eref{eqn12} and comparing \eref{eqn16}, we have, $\Delta^{2}_{min}B_{1} \geq ||\textbf{u}||_{2}^{2}$ and $\Delta^{2}_{min}B_{2} \geq ||\textbf{v}||_{2}^{2}$. Hence, we employ \eref{eqn15} for the two outcomes $\hat{B}_{1}$ and $\hat{B}_{2}$, and obtain a lower bound for the product of variances $\Delta^{2}_{min}B_{1}$ and $\Delta^{2}_{min}B_{2}$ which is given by,
\begin{eqnarray}\label{eqn17}
   \Delta^{2}_{min}B_{1}\Delta^{2}_{min}B_{2} &\geq ||\textbf{u}||^{2}_{2}||\textbf{v}||^{2}_{2} \geq |\langle\textbf{u},\textbf{v}\rangle|^{2} \nonumber
   \\ &\geq \sum_{\eta}P(\eta)\Delta_{Q}^{2}(B_{1}|\eta)\Delta_{Q}^{2}(B_{2}|\eta).
\end{eqnarray}
Using \eref{eqn14}, the above inequality can be written as,
\begin{eqnarray}\label{eqn18}
    \Delta^{2}_{min}B_{1}\Delta^{2}_{min}B_{2} \geq
    \frac{1}{4}\sum_{\eta}P(\eta)[|\langle B_{3} \rangle_{\eta}|^{2} + \\ \hspace{3.5cm} \left(
\langle\{B_{1},B_{2}\}\rangle_{\eta} - 2\langle B_{1}\rangle_{\eta}\langle B_{2}\rangle_{\eta}\right)^{2}] \nonumber
\end{eqnarray}
where, $\langle B_{i} \rangle_{\eta}$ is the mean w.r.t probability distribution $P_{Q}(B_{i}|\eta)$. Using the properties of convex functions and Jensen's inequality $|\alpha|^{2}$, where $\alpha$ is a random variable, we have $\sum_{\alpha}P(\alpha)|\alpha|^{2}  \geq  |\sum_{\alpha}P(\alpha)\alpha|^{2}$ for a given probability distribution $P(\alpha)$, the RHS of (\ref{eqn18}) in terms of inferred-variances and averages is written as (refer Appendix for details),
\begin{eqnarray}\label{eqn19}
    \Delta^{2}_{min}B_{1}\Delta^{2}_{min}B_{2} \geq  \frac{1}{4}\left | \langle[B_{1},B_{2}]\rangle \right |^{2}_{inf} \\ \hspace{2cm}
 + \left( \left( \frac{\langle \{B_{1},B_{2}\}\rangle}{2} - \langle B_{1}\rangle\langle B_{2}\rangle \right)^{2}\right)_{inf}.\nonumber
\end{eqnarray}
From \ref{eqn9}, using the condition, $\Delta^{2}_{inf}B \geq \Delta^{2}_{min}B$ and employing in \ref{eqn19}, we have the EPR-steering criterion based on SRUR which can be written as follows 
\begin{eqnarray}\label{eqn20}
    \Delta^{2}_{inf}B_{1}\Delta^{2}_{inf}B_{2} \geq \frac{1}{4}|\langle[B_{1},B_{2}]\rangle|^{2}_{inf} + \\  \hspace{3cm} \left( \left( \frac{\langle \{B_{1},B_{2}\}\rangle}{2} - \langle B_{1}\rangle\langle B_{2}\rangle \right)^{2}\right)_{inf}. \nonumber
\end{eqnarray}
\section{Results and Discussion}
The family of isotropic states in $C_{d}\otimes C_{d}$, parameterized by $p \in \mathbb{R}$, is given by
\begin{eqnarray}\label{eqn21}
    W^{p}_{d} = \left(1-p\right)\frac{\mathbb{\textbf{I}}}{d^{2}} + p|\Psi_{+}\rangle\langle\Psi_{+}|
\end{eqnarray}
where, $0\leq p \leq1$, $|\Psi_{+}\rangle = \sum_{i=1}^{d} \frac{|ii\rangle}{\sqrt{d}}$ and $\mathbb{\textbf{I}}$ is the Identity operator. Here, we calculate the steerability of isotropic states for dimension d = 2,3,4 using \eref{eqn20}.\newline \newline
For dimension $\textbf{d=2}$, two qubit isotropic state, our choice for Bob's measurement operators $\hat{B}_{1}$ and $\hat{B}_{2}$ are spin half observables, i.e., $\hat{B}_{1} = \hat{S}_{B_x}$,$\hat{B}_{2} = \hat{S}_{B_y}$,$\hat{B}_{3} = \hat{S}_{B_z}$ with their corresponding outcomes $S_{B_{x}}$, $S_{B_{y}}$ and $S_{B_{z}}$respectively. We obtain a steering inequality for two-qubit isotropic state in one-sided three-measurement scenarios. Therefore \eref{eqn20} for spin observables can be written in the following way
\begin{eqnarray}\label{eqn22}
\Delta^{2}_{inf}S_{B_x}\Delta^{2}_{inf}S_{B_y} \geq \frac{1}{4}|\langle [S_{B_{x}},S_{B_{y}}]\rangle|^{2}_{inf} +  \\ \hspace{1.9cm} \left(\frac{\langle \{S_{B_x}, S_{B_y}\}\rangle}{2} - \langle S_{B_x}\rangle\langle S_{B_y}\rangle\right)^{2}_{inf} \nonumber
\end{eqnarray}
\newline
where  $\Delta^{2}_{inf} S_{B_i} = \langle(S_{B_i} - S_{B_i}^{est}(S_{A_i}))^{2}\rangle = \langle(S_{B_i} - g_{i}S_{A_i})^{2}\rangle, g_i = \frac{\langle S_{A_i} \otimes S_{B_i}\rangle}{\langle S_{A_i}^{2}\rangle}$, $i = x,y$ 
 and $\langle S_{B_j}\rangle_{inf} = \sum_{S_{A_j}}P(S_{A_j})\langle S_{B_j}\rangle_{S_{A_j}}$, $j = x,y,z$.\newline\newline
Calculation of inferred-variance and inferred-mean for two-qubit isotropic state gives $\Delta^{2}_{inf} S_{B_x} = \Delta^{2}_{inf} S_{B_y} = \frac{1}{4}(1 - p^{2})$, $\langle\{S_{B_x}, S_{B_y}\}\rangle = 0$, $\langle S_{B_x}\rangle_{inf} = \langle S_{B_y}\rangle_{inf} = \langle S_{B_z}\rangle_{inf} = \frac{p}{2}$. Using these values in \eref{eqn22}, we obtain the following condition
 \begin{eqnarray}\label{eqn23}
     p \leq \frac{1}{\sqrt{3}}.
 \end{eqnarray}
 Violation of \eref{eqn23}, i.e. $p > \frac{1}{\sqrt{3}}$, detect steerable two qubit isotropic states for three measurements. The optimal condition for steerability of two qubit isotropic state for an infinitely large number of measurements is $p > \frac{1}{2}$ \cite{wiseman2007steering}. This is very difficult to achieve in experiments (i.e. for a finite number of measurements). It was shown that two-qubit Werner states violate steering inequality for $p > \frac{1}{\sqrt{3}}$ for three measurements \cite{cavalcanti2009spin}. The two qubit isotropic state also show same steerability condition as two qubit Werner state \cite{wiseman2007steering}. \newline\newline
For dimension \textbf{ d = 3}, two qutrit isotropic state, our choice of Bob's operators $\hat{B}_{1}$ and $\hat{B}_{2}$, following the commutation relation $[\hat{B}_{1},\hat{B}_{2}]=\iota\hat{B}_{3}$ is given as follows \cite{ZhenPhysRevA.93.012108},
 \begin{eqnarray}
 \hspace{0.5cm}B_{1} = 
     \begin{pmatrix}
         0 & \frac{1}{\sqrt{2}} & 0 \\
         \frac{1}{\sqrt{2}} & 0 & 0 \\
         0 & 0 & 0
     \end{pmatrix}
     %
\hspace{0.5cm}B_{2} = 
     \begin{pmatrix}
         0 & 0 & \frac{1}{\sqrt{2}} \\
         0 & 0 & 0 \\
         \frac{1}{\sqrt{2}} & 0 & 0
     \end{pmatrix}\nonumber
 \end{eqnarray}
\begin{eqnarray}
 \hspace{2.2cm}B_{3} = 
     \begin{pmatrix}
         0 & 0 & 0 \\
         0 & 0 & \frac{-\iota}{\sqrt{2}} \\
         \frac{\iota}{\sqrt{2}} & 0 & 0
     \end{pmatrix}
     \nonumber
\end{eqnarray}
 We calculate the inferred variances and inferred means of the above operators gives $\Delta_{inf}^{2}B_{1} = \frac{1-p^{2}}{3} = \Delta_{inf}^{2}B_{2}$, $\langle B_{1} \rangle_{inf} = \langle B_{2} \rangle_{inf} = \langle [B_{1},B_{2}] \rangle_{inf} = \langle\{ B_{1},B_{2} \}\rangle_{inf} = 0$, $\langle [B_{1},B_{2}] \rangle ^{2}_{inf} = \langle\{ B_{1},B_{2} \}\rangle_{inf}^{2} = \frac{p^{2}}{6}, \langle \hat{B}_{1} \rangle^{2}_{inf} =  \langle \hat{B}_{1} \rangle^{2}_{inf} = \frac{p^{2}}{3}$ and obtain the following steering condition by using \eref{eqn20}
 \begin{eqnarray}
     p \leq \sqrt{\frac{4}{11}}
 \end{eqnarray}
Violating of  the above condition i.e. $p > \sqrt{\frac{4}{11}}$ detect steerable two qutrit isotropic states for three measurements. The optimal condition for steerability of two qutrit isotropic state is $p > \frac{5}{12}$ for infinitely large number of measurements.\newline
For dimension \textbf{d = 4}, two-ququart isotropic state, our choice for Bob's operators $\hat{B}_{1}$ and $\hat{B}_{2}$ with the commutation relation $[\hat{B}_{1},\hat{B}_{2}]=\iota\hat{B}_{3}$ are \cite{ZhenPhysRevA.93.012108}
\begin{eqnarray}
\hspace{0.2cm} \hat{B}_{1} = 
    \begin{pmatrix}
        0 & \frac{1}{\sqrt{2}} & 0 & 0 \\
        \frac{1}{\sqrt{2}} & 0 & 0 & 0 \\
        0 & 0 & 0 & 0 \\
        0 & 0 & 0 & 0
    \end{pmatrix}
    %
    \hspace{0.2cm} \hat{B}_{2} = 
    \begin{pmatrix}
        0 & 0 & \frac{1}{\sqrt{2}} & 0 \\
        0 & 0 & 0 & 0 \\
        \frac{1}{\sqrt{2}} & 0 & 0 & 0 \\
        0 & 0 & 0 & 0
    \end{pmatrix}
\nonumber 
\end{eqnarray}
\begin{eqnarray}
\hspace{2.2cm}\hat{B}_{3} =
     \begin{pmatrix}
        0 & 0 & 0 & 0 \\
        0 & 0 & \frac{-\iota}{\sqrt{2}} & 0 \\
        0 & \frac{\iota}{\sqrt{2}} & 0 & 0 \\
        0 & 0 & 0 & 0
    \end{pmatrix}\nonumber
\end{eqnarray}
Calculation of inferred variances and inferred means of the above operators, we obtain the following expressions $\Delta_{inf}^{2}B_{1} = \frac{1-p^{2}}{4} = \Delta_{inf}^{2}B_{2}$, $\langle B_{1} \rangle_{inf} = \langle B_{2} \rangle_{inf} = \langle [B_{1},B_{2}] \rangle_{inf} = \langle\{ B_{1},B_{2} \}\rangle_{inf} = 0$, $\langle [B_{1},B_{2}] \rangle ^{2}_{inf} = \langle\{ B_{1},B_{2} \}\rangle_{inf}^{2} = \frac{p^{2}}{8}$. Plugging these expression into the eqnarray we obtain the steering inequality for two ququart Isotropic states
\begin{eqnarray}
    p \leq \frac{1}{\sqrt{3}}.
\end{eqnarray}
If this inequality is violated i.e. $p > \frac{1}{\sqrt{3}}$ then the state show steerability for four measurements. Two ququart isotropic state is optimally steerable when $p > \frac{13}{36}$ for infinitely large measurements.\newline\newline
An important point to note is that experimental detection of steerable states depends crucially on the choice of measurements. Hence, the quest for an optimal choice of measurements is a big challenge. Moreover, with increase in the dimensions of the systems a large number of measurements is required for detecting steerability of higher dimensional systems.
\section{Conclusion}
We utilized the Local Hidden State (LHS) model and Reid's criterion in Schr\"odinger-Robertson Uncertainty Relation (SRUR) and obtained the EPR-steering criterion for bipartite systems \cite{einstein_1935,einstein_1936,reid1989demonstration} which can be experimentally verified. We then used our EPR-Steering criterion to detect steerability for different dimensional i.e., d = 2,3,4 systems. We observed that  with our choice of observables, the steerability of the states could be detected, however the bounds can be further improved by selecting optimal measurement operators. This is because the construction of optimal set of observables which would give optimal condition of steerability for a given set of measurements is very important. Hence, construction of operators which would detect steerability optimally is a major challenge. Adopting an LHS model and Reid criterion one could obtain a completely inferred variance based steering criterion using generalized $n$ operator variance based \cite{multi,Tripathi_2020}, sum \cite{Sum} uncertainty relation and higher order SRUR \cite{higherorderSRURlee2014inseparability}. It is tempting to look for stricter bounds of steering in continuous variable systems. Furthermore, this criterion can be implemented using measurements corresponding to positive operator-valued measures (POVM). One of the works shows the correspondence between joint measurability and steering \cite{Quintino2014JointME}. Therefore, it would be interesting to look for the correspondence between steering and uncertainty relations involving incompatible POVMs.
\section{Acknowledgments}
We would like to thank Mr Abhinash Kumar Roy, Mr Prabhuda Roy, Mr Sumit Mukherjee for numerous enlightening discussions. PKP acknowledge the support from DST, India, through Grant No. DST/ICPS/QuST/Theme-1/2019/2020-21/01
\appendix
\section*{Appendix}
\setcounter{section}{1}
 Consider the inequality developed in \eref{eqn18}. For Bob's two measurement outcomes $B_1$ and $B_2$, corresponding to operators $\hat{B}_{1}$ and $\hat{B}_{2}$, we have the following relation
\begin{eqnarray}\label{a1}
     \Delta^{2}_{min}B_{1}\Delta^{2}_{min}B_{2} \geq 
     \frac{1}{4}\sum_{\eta}P(\eta)|\langle B_3\rangle_{\eta}|^{2} + \\ 
     \hspace{2cm}\sum_{\eta}P(\eta)\left(\frac{\langle\{B_{1}, B_{2}\}\rangle_{\eta}}{2} - \langle B_{1}\rangle_{\eta}\langle B_{2}\rangle_{\eta}\right)^{2}.\nonumber
\end{eqnarray}
Here we witness one of the important applications of Local Hidden Variable (LHV) theory i.e. the sufficient condition of reality. It is very important to note that in the LHS model, the choice of the operators $\{\hat{A_1},\hat{A_2},\hat{A_3}\}$ for Alice is completely arbitrary. In this model the operators are allowed to  depend arbitrarily on the hidden variables $\eta$ and hence plays no role in the LHS model of Bob. For all the measurement outcomes $A_{i}$ corresponding to the operators $\hat{A}_{i}$ that exhaust Alice's measurement set of operators, the RHS of \eref{a1} is as follows
\begin{eqnarray}\label{a2}
\frac{1}{4}\sum_{A_{3},\eta}P(A_{3},\eta)|\langle B_3\rangle_{\eta}|^{2} + \frac{1}{4}\sum_{A_{4},\eta}P\left(A_{4},\eta\right)\langle\{B_{4}\}\rangle_{\eta}^{2} \nonumber \\ \hspace{2cm} + \sum_{A_{1},\eta}P(A_{1},\eta)\langle B_{1}\rangle_{\eta}^{2}\langle B_{2}\rangle_{\eta}^{2} \\ \hspace{3cm}-  \sum_{A_{1},\eta}P(A_{1},\eta)\langle B_{4}\rangle_{\eta}\langle B_{2}\rangle_{\eta}\langle B_{1}\rangle_{\eta} \nonumber
\end{eqnarray}
where, $\{\hat{A}_1,\hat{A}_2\} = \hat{A}_4$, $\{\hat{B}_1,\hat{B}_2\} = \hat{B}_4$, all the $P$'s are the joint probability distribution for different outcomes $A_{i}$ and classical random variable $\eta$. $P(A_{i})$ for $i = 1,2,3,4$ corresponds to the measurement observables $\hat{A}_{i}$.
For a real variable $u$, $|u|^{2}$ and $u^{2}$ are convex functions. Hence, by Jensen's inequality $\sum_{u}P(u)|u|^{2} \geq |\sum_{u}P(u)u|^{2}$. Therefore, \eref{a2} is lower bounded which is given as follows
\begin{eqnarray}\label{a3}
    \geq \frac{1}{4} \sum_{A_{3}}P(A_{3})\left|\sum_{\eta}P(\eta|A_{3})\langle B_{3}\rangle_{\eta}\right|^{2} \\ \hspace{0.6cm} + \frac{1}{4}\sum_{A_{4}}P(A_{4})\left(\sum_{\eta}P(\eta|A_{4})\langle\{B_{4}\}\rangle_{\eta}\right)^{2} \nonumber \\ \hspace{1.4cm} + \sum_{\eta,A_{1}}P(\eta|A_{1})P(A_{1})\langle B_{1}\rangle_{\eta}^{2}\langle B_{2}\rangle_{\eta}^{2} \nonumber \\ \hspace{2.1cm}-  \sum_{\eta,A_{1}}P(\eta|A_{1})P(A_{1})\langle B_{1}\rangle_{\eta}\langle B_{2}\rangle_{\eta}\langle B_{4}\rangle_{\eta}\nonumber
\end{eqnarray}
Here, we use the marginal probabilities in the derivation below i.e. $P(A_{1}) = \sum_{A_{2}}P(A_{1},A_{2})$. The \eref{a3} can be written as 
\begin{eqnarray}
    \geq \frac{1}{4} \sum_{A_{3}}P(A_{3})\left|\sum_{\eta}P(\eta|A_{3})\langle B_{3}\rangle_{\eta}\right|^{2} \\ \hspace{0.4cm} + \frac{1}{4}\sum_{A_{4}}P(A_{4})\left(\sum_{\eta}P(\eta|A_{4})\langle\{B_{4}\}\rangle_{\eta}\right)^{2} \nonumber \\ \hspace{0.9cm} + \sum_{\eta,A_{1}}P(\eta|A_{1})\langle B_{1}\rangle_{\eta}^{2}\sum_{A_{2}}P(A_{1},A_{2})\langle B_{2}\rangle_{\eta}^{2} \nonumber \\ \hspace{1.2cm}-  \sum_{\eta,A_{1}}P(\eta|A_{1})\langle B_{1}\rangle_{\eta}\sum_{A_{2}}P(A_{1},A_{2})\langle B_{2}\rangle_{\eta}\langle B_{4}\rangle_{\eta}\nonumber
\end{eqnarray}
We utilize the Bayes rule, $P(A_{i},A_{j}) = P(A_{i}|A_{j})P(A_{j})$ and the marginal distribution $P(A_{i}) = \sum_{\eta}P(\eta,A_{i})$ where, $A_{i}$, $A_{j}$ are outcomes corresponding to different measurements $\hat{A_{i}}$ and $\hat{A_{j}}$ of Alice. And $\eta$ is the local hidden variable correlated to the local hidden state $\sigma_{\eta}$.
\begin{eqnarray}
    \geq \frac{1}{4} \sum_{A_{3}}P(A_{3})\left|\sum_{\eta}P(\eta|A_{3})\langle B_{3}\rangle_{\eta}\right|^{2} \\ \hspace{1cm} +\frac{1}{4}\sum_{A_{4}}P(A_{4})\left(\sum_{\eta}P(\eta|A_{4})\langle\{B_{4}\}\rangle_{\eta}\right)^{2} \nonumber \\ + \sum_{\eta,A_{1}}P(\eta|A_{1})\langle B_{1}\rangle_{\eta}^{2}\sum_{A_{2}}P(A_{1}|A_{2})\sum_{\eta}P(\eta,A_{2})\langle B_{2}\rangle_{\eta}^{2} \nonumber \\-  \sum_{\eta,A_{1}}P(\eta|A_{1})\langle B_{1}\rangle_{\eta}\sum_{A_{2}}P(A_{1}|A_{2})P(A_{2})\langle B_{2}\rangle_{\eta}\langle B_{4}\rangle_{\eta}\nonumber
\end{eqnarray}
We use the concept explained in \ref{EPR Steering}. The probabilities are calculated by running the experiment many time where Alice and Bob perform single set of measurements in each run. Here, Alice and Bob arbitrarily chooses to perform different measurements in each of the run of the experiment. The probability of the result of Alice's a particular measurement is independent of the probability of the result of Alice's any other measurement, because, the measurements are done in different runs in identically prepared states.\newline \newline Moreover, from the interpretation of local hidden state model, Alice sends a local hidden state $\sigma^{\hat{A}}_{a}$ to Bob, by picking from an ensemble $\textnormal{R} = \sum_{\eta}p_{\eta}\sigma_{\eta}$ and doing a stochastic map from LHV $\eta$ to the result $a$ of the measurement $\hat{A}$ in each run of the experiment. Here, Alice randomly chooses any one of the results of the measurement $\hat{A}$ for mapping, into which she wants Bob's state to collapse into. Hence the probability of choosing one result is independent of the other i.e. $P(a_{i}|a_{j}) = P(a_{i})$ for all outcomes $a_{i}$ and $a_{j}$. However, these outcomes depend arbitrarily on the LHV $\eta$. Therefore, we use this statistical independence of outcomes in the step below i.e. $P(A_{1}|A_{2}) = P(A_{1})$
\begin{eqnarray}
    \geq \frac{1}{4} \sum_{A_{3}}P(A_{3})\left|\sum_{\eta}P(\eta|A_{3})\langle B_{3}\rangle_{\eta}\right|^{2} \\ \hspace{1.5cm} +\frac{1}{4}\sum_{A_{4}}P(A_{4})\left(\sum_{\eta}P(\eta|A_{4})\langle\{B_{4}\}\rangle_{\eta}\right)^{2} \nonumber \\+ \sum_{A_{1}}P(A_{1})\sum_{\eta}P(\eta|A_{1})\langle B_{1}\rangle_{\eta}^{2} \nonumber \\ \nonumber \hspace{3.7cm} \sum_{A_{2}}P(A_{2})\sum_{\eta}P(\eta|A_{2})\langle B_{2}\rangle_{\eta}^{2} \nonumber \\-\sum_{A_{1}}P(A_{1}) \sum_{\eta}P(\eta|A_{1})\langle B_{1}\rangle_{\eta} \nonumber \\
    \hspace{3.4cm}\sum_{A_{2},A_{4}}P(A_{2}|A_{4})\langle B_{2}\rangle_{\eta}P(A_{4})\langle B_{4}\rangle_{\eta}\nonumber
\end{eqnarray}
In the above eqnarray, we used the marginal distribution and bayes rule in fourth term i.e. $P(A_{2}) = \sum_{A_{4}}P(A_{2},A_{4}) = P(A_{2}|A_{4})P(A_{4})$ and the marginal distribution $P(A_{4}) = \sum_{\eta}P(\eta,A_{4})$.
\begin{eqnarray}
    \geq \frac{1}{4} \sum_{A_{3}}P(A_{3})\left|\sum_{\eta}P(\eta|A_{3})\langle B_{3}\rangle_{\eta}\right|^{2} \\  \hspace{1.5cm} + \frac{1}{4}\sum_{A_{4}}P(A_{4})\left(\sum_{\eta}P(\eta|A_{4})\langle\{B_{4}\}\rangle_{\eta}\right)^{2} \nonumber \\ + \sum_{A_{1}}P(A_{1})\left(\sum_{\eta}P(\eta|A_{1})\langle B_{1}\rangle_{\eta}\right)^{2} \nonumber \\ \hspace{3.1cm}\sum_{A_{2}}P(A_{2})\left(\sum_{\eta}P(\eta|A_{2})\langle B_{2}\rangle_{\eta}\right)^{2} \nonumber\\ -\sum_{A_{1}}P(A_{1}) \sum_{\eta}P(\eta|A_{1})\langle B_{1}\rangle_{\eta} \nonumber \\ \hspace{2.2cm} \sum_{A_{2}}P(A_{2})\sum_{\eta}P(\eta|A_{2})\langle B_{2}\rangle_{\eta} \nonumber \\ \hspace{3.8cm}\sum_{A_{4}}P(A_{4})\sum_{\eta}P(\eta|A_{4})\langle B_{4}\rangle_{\eta}\nonumber
\end{eqnarray}
We obtain averages of Bob's measurements which are inferred from Alice's measurements.
\begin{eqnarray}\label{a5}
    \frac{1}{4}\sum_{A_{3}}P(A_{3})|\langle B_{3}\rangle_{A_{3}}|^{2}+ \frac{1}{4}\sum_{A_{4}}P(A_{4})\langle\{B_{4}\}\rangle_{A_{4}}^{2} \\ \hspace{1.5cm} + \sum_{A_{1}}P(A_{1})\langle B_{1}\rangle_{A_{1}}^{2}
    \sum_{A_{2}}P(A_{2})\langle B_{2}\rangle_{A_{2}}^{2}\nonumber \\-  \sum_{A_{1}}P(A_{1})\langle B_{1}\rangle_{A_{1}}\sum_{A_{2}}P(A_{2})\langle B_{2}\rangle_{A_{2}}\sum_{A_{4}}P(A_{4})\langle B_{4}\rangle_{A_{4}} \nonumber
\end{eqnarray}
We obtain the inferred steering criterion based on Schr\"odinger-Robertson Uncertainty Relation
\begin{eqnarray}\label{a8}
    \frac{1}{4} \left|\langle B_{3}\rangle\right|^{2}_{inf} + \frac{1}{4}\left(\left(\langle \{B_{1},B_{2}\} \rangle - 2\langle B_{1}\rangle \langle B_{2}\rangle\right)^2\right)_{inf}\nonumber\\
\end{eqnarray}
\section*{References}
\bibliography{main}
\end{document}

