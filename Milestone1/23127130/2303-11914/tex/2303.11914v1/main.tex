%%%%%%%%%%%%%%%%%%%%%%%%%%%%%%%%%%%%%%%%%%%%%%%%%%%%%%%%%%%%%%%%%%%%%%%%
%    INSTITUTE OF PHYSICS PUBLISHING                                   %
%                                                                      %
%   `Preparing an article for publication in an Institute of Physics   %
%    Publishing journal using LaTeX'                                   %
%                                                                      %
%    LaTeX source code `ioplau2e.tex' used to generate `author         %
%    guidelines', the documentation explaining and demonstrating use   %
%    of the Institute of Physics Publishing LaTeX preprint files       %
%    `iopart.cls, iopart12.clo and iopart10.clo'.                      %
%                                                                      %
%    `ioplau2e.tex' itself uses LaTeX with `iopart.cls'                %
%                                                                      %
%%%%%%%%%%%%%%%%%%%%%%%%%%%%%%%%%%
%
%
% First we have a character check
%
% ! exclamation mark    " double quote  
% # hash                ` opening quote (grave)
% & ampersand           ' closing quote (acute)
% $ dollar              % percent       
% ( open parenthesis    ) close paren.  
% - hyphen              = equals sign
% | vertical bar        ~ tilde         
% @ at sign             _ underscore
% { open curly brace    } close curly   
% [ open square         ] close square bracket
% + plus sign           ; semi-colon    
% * asterisk            : colon
% < open angle bracket  > close angle   
% , comma               . full stop
% ? question mark       / forward slash 
% \ backslash           ^ circumflex
%
% ABCDEFGHIJKLMNOPQRSTUVWXYZ 
% abcdefghijklmnopqrstuvwxyz 
% 1234567890
%
%%%%%%%%%%%%%%%%%%%%%%%%%%%%%%%%%%%%%%%%%%%%%%%%%%%%%%%%%%%%%%%%%%%
%
\documentclass[10pt]{iopart}
%Uncomment next line if AMS fonts required
\usepackage{iopams}  
\usepackage{cite}
\usepackage{hyperref}
%\usepackage{orcidlink}
\bibliographystyle{iopart-num}

\begin{document}

\title{Stronger EPR-steering criterion based on Schr\"odinger-Robertson uncertainty relation}

\author{Laxmi Prasad Naik$^{1,}$$^2$, Rakesh Mohan Das$^3$ and Prasanta K. Panigrahi$^2$}

\address{$^1$ Indian Institute of Science Education And Research Kolkata, Mohanpur, Nadia - 741 246, West Bengal, India}
\address{$^2$ Indian Institute of Technology Delhi, Hauz Khas - 1100016, New Delhi, India}
\address{$^3$ Kalinga Institute of Industrial Technology, Bhubaneswar - 751024, Odisha, India}
\ead{laxmiprasadnaik5897@gmail.com,rakesh.dasfpy@kiit.ac.in and pprasanta@iiserkol.ac.in}
\vspace{10pt}
\begin{indented}
\item[]January 2023
\end{indented}

\begin{abstract}
Steering is one of the three in-equivalent forms of nonlocal correlations intermediate between Bell nonlocality and entanglement. Schr\"odinger-Robertson uncertainty relation (SRUR), has been widely used to detect entanglement and steering. However, the steering criterion in earlier works, based on SRUR, did not involve complete inferred-variance uncertainty relation. In this paper, by considering the local hidden state model and Reid’s formalism in SRUR, we derive a complete inferred-variance steering criterion for bipartite systems in one-sided, two-measurement and two-outcome scenarios. Furthermore,  our steering criterion, when applied to bipartite discrete variable case, provides a stricter range for two-qubit Werner states.
\end{abstract}

%
% Uncomment for keywords
\vspace{2pc}
\noindent{\it Keywords}: EPR-steering, Schr\'odinger-Robertson uncertainty relation
%
% Uncomment for Submitted to journal title message
%\submitto{\JPA}
%
% Uncomment if a separate title page is required
%\maketitle
% 
% For two-column output uncomment the next line and choose [10pt] rather than [12pt] in the \documentclass declaration
\ioptwocol
\section{Introduction}
EPR-steering is a nonlocal correlation intermediate between Bell nonlocality and quantum entanglement \cite{einstein_1935,einstein_1936,wiseman2007steering}. It is the ability to remotely affect or \emph{steer} a shared entangled quantum state by a single party's (say Alice's) arbitrary choice of local measurements without violating the no-signalling principle \cite{simon2001no}. Wiseman \textit{et al.} gave an operational definition of steering as a task between Alice and Bob. Alice prepares an entangled state and sends one part to Bob. Here, Bob does not trust Alice, and by performing local measurements, she has to convince him that the state is entangled \cite{wiseman2007steering}. If Bob's steered quantum state cannot be explained by a local hidden state (LHS) model, then the state is said to exhibit steering. In contrast to Bell's nonlocality and entanglement, steering demonstrates asymmetric behaviour in which one party can steer the other party, but vice versa is not always permitted \cite{midgley2010asymmetric,bowles2014one,bowles2016sufficient,reid2013monogamy}. Moreover, not every entangled state exhibits steering, and not every steerable state violates Bell inequality \cite{wiseman2007steering}.
EPR-steering has a wide range of applications in many quantum information processing tasks e.g. in one-sided device-independent quantum key distribution \cite{branciard2012one,gehring2015implementation,walk2016experimental}, quantum networking tasks \cite{Huang2018SecuringQN,Armstrong2014MultipartiteES,Cavalcanti2014DetectionOE}, subchannel discrimination \cite{Piani2015NecessaryAS,Chen2016NaturalFF, Sun2018DemonstrationOE}, quantum secret sharing \cite{Kogias2016UnconditionalSO, Xiang2016MultipartiteGS}, quantum teleportation \cite{Reid2013SignifyingQB,fan2022quantum}, randomness certification \cite{passaro2015optimal,curchod2017unbounded,Mattar2016ExperimentalME}, and random number generation \cite{joch2022certified} to mention a few. Recently it has also been demonstrated to be a useful resource in noisy and lossy quantum network systems \cite{Qu2021RetrievingHQ,Srivastav2022QuickQS}.\newline\newline
Effective detection of steering exhibited by quantum states is crucial to realise applications of steerable quantum states. Uncertainty relations (UR) can be experimentally verified because it involves measurement of observables. Assuming the description of quantum mechanics is correct, EPR's condition of locality and sufficient condition of reality are satisfied, UR's become an important tool for determining steering criteria. Many criteria in this direction e.g., using the Heisenberg uncertainty relation (HUR) \cite{reid1989demonstration} and later involving a broader class of uncertainty relations, have been proposed \cite{bialynicki1975uncertainty, Deutsch1983UncertaintyIQ,Chowdhury2013EinsteinPodolskyRosenSU,schneeloch2013einstein}. Additionally, under different measurement scenarios, more optimal steering criteria \cite{pramanik2014fine, maity2017tighter} were obtained using fine-grained uncertainty relations \cite{Oppenheim2010TheUP, Chowdhury2015StrongerSC} and sum uncertainty relations \cite{Maccone2014StrongerUR, maity2017tighter}. \newline \newline
The criterion for experimental demonstration of steering was first proposed by Reid \cite{reid1989demonstration}, which is based on inferred-variances. Recently, a steering criterion using Schr\"odinger-Robertson uncertainty relation (SRUR) was also proposed. However, these earlier works did not involve the inferred-means in the lower bound \cite{sasmal2018tighter}. A recent work involves inferred-variance based product and sum uncertainty relations in the presence of entanglement \cite{bagchi2022inferred}. We aim to derive a steering criterion based on SRUR involving inferred-means and inferred-variances for one-sided two-measurement and
two-outcome scenarios, following up the analysis in \cite{cavalcanti2009experimental}. The steering criterion’s efficiency is evaluated for two-qubit Werner states \cite{werner1989quantum} which provides a steering
inequality that was obtained for one-sided three-measurement as well as two-sided two-measurement settings. \cite{cavalcanti2009spin,Saunders2009ExperimentalEU, Chowdhury2015StrongerSC,costa2018steering}. \newline\newline
In the next section, we briefly discuss steering and the EPR-Reid criterion. In Sec.3, we derive a steering criterion based on the SRUR. We check the efficiency of the steering criterion in Sec.4, using it on two-qubit Werner states for which a steering inequality is obtained and discuss the improvement of our result compared to other measurement scenarios. The paper ends with a conclusion and an appendix.
\section{Preliminaries}
\subsection{EPR-Steering}
Consider a general unfactorizable bipartite pure state shared by two distant parties, Alice and Bob
\begin{equation}\label{eqn1}
    |\Psi\rangle = \sum_{n} c_n|u_n\rangle|v_n\rangle = \sum_{n} d_n|\psi_n\rangle|\phi_n\rangle
\end{equation}
where, $\{|u_n \rangle \}(\{|v_n\rangle\})$ and $\{|\psi_n\rangle\}(\{|\phi_n\rangle\})$ denote two different orthonormal bases in Alice’s and
Bob’s system, respectively. This property of inseparability is called entanglement, which is one of the most useful resources in quantum information processing that has been studied extensively in the literature \cite{horodecki,Bhaskara_QINP_2017,roy2021geometric,mahanti2022classification,mishra2022geometric}. In this scenario, Alice chooses to measure in the ${|u_n\rangle}\left(\{|v_n\rangle\}\right)$ basis, then Bob's state will be projected into ${|\psi_n\rangle} (\{|\phi_n\rangle\})$ basis. The ability of Alice to influence (steer)
Bob’s state, nonlocally, was termed as steering by Schr\"odinger \cite{einstein1935can,einstein_1935,einstein_1936}.\newline\newline
Consider the following situation, Alice and Bob share an entangled quantum state, described by density matrix $\hat{\rho}$. The generalised local measurements of Alice and Bob are denoted by ${\hat{M}_{a|A}}$ and ${\hat{M}_{b|B}}$ ($M_{a(b)|A(B)} \geq 0, \sum_{a(b)}M_{a(b)|A(B)} = 1 \hspace{3mm}\forall A(B)$) respectively, where $a$ and $b$ denote the outcomes corresponding to the measurement operators $\hat{M}_{a|A}$ and $\hat{M}_{b|B}$. $A$ and $B$ are Alice's and Bob's measurement settings, respectively. In a Steering task, Bob does not trust Alice and wants to verify whether the shared state is entangled or not. So he asks Alice to perform a measurement $\hat{M}_{a|A}$ and classically communicate its outcome $a$ to him. The quantum probability of their joint measurement is given as follows
\begin{equation}\label{eqn2}
    P(a,b) = \textnormal{Tr}[\hat{\rho}(\hat{M}_{a|A} \otimes \hat{M}_{b|B})]
\end{equation}
where, $P(a,b)$ is the joint probability of obtaining outcomes $a$ and $b$. If and only if for all measurements $\hat{M}_{a(b)|A(B)}$, Eq.(\ref{eqn2}) can be defined as
\begin{equation}\label{eqn3}
    P(a,b) = \sum_{\eta}p(\eta)P(a,\eta)P_{Q}(b,\eta)
\end{equation}
where $\eta$ is a classical random variable having probability distribution $p(\eta)$, satisfying $p(\eta) \geq 0$ and $\sum_{\eta}p(\eta) = 1$. $P(a,\eta)$ is joint probability distribution between $\eta$ and outcome $a$. Quantum probability distribution $P_{Q}$ between $\eta$ and outcome $b$ is $P_{Q}(b,\eta)= \textnormal{Tr}_{B}[(\hat{M}_{b|B}\hat{\rho}_{\eta}]$ ($Q$ stands for quantum), corresponding to a local hidden quantum state described by $\hat{\rho}_{\eta}$, which is unaffected by local measurements of Alice. The use of LHS to explain steering is a clear implication of the consistency of EPR's condition of locality. Any constraint that can be obtained obeying Eq.(\ref{eqn3}) is called an \emph{EPR-steering criterion}, violation of which will demonstrate steering. The joint probability distribution and the state is said to admit an LHS model if Eq.(\ref{eqn2}) can be expressed having a decomposition of the form Eq.(\ref{eqn3}) for all the choice of Alice's and Bob's measurements respectively. In other words, if and only if, Alice, for all her choice of measurements $\hat{M}_{a|A}$ could steer Bob's system into a conditioned state which is given as
\begin{equation}\label{eqn4}
    \hat{\sigma}_{a|A} = \textnormal{Tr}_{A}[(\hat{M}_{a|A} \otimes I)\hat{\rho}]
\end{equation}
where $\textnormal{Tr}_{A}$ is partial trace over Alice's system and if Bob cannot express $\hat{\sigma}_{a|A}$ in the following form
\begin{equation}\label{eqn5}
   \hat{\sigma}_{a|A} = \int p(\eta)P(a,\eta)\hat{\rho}_{\eta}\,d\eta
\end{equation}
then, the state is said to be steerable. However, Alice cannot affect Bob's unconditioned state $\textnormal{Tr}_{A}[\hat{\rho}]$, because that would violate superluminal communication \cite{simon2001no}.
\newline\newline
Since Bob's state corresponds to a local hidden quantum state, uncertainty relations can be used for Bob's measurements. This was first realized by Reid \cite{reid1989demonstration}, who proposed an experimental EPR-steering criterion using HUR in continuous variable systems. Therefore we aim to derive an EPR-steering criterion using SRUR because it involves the covariance of the observables, which captures stronger correlations.
\subsection{EPR-Reid criterion}
Reid proposed a modified version of EPR's sufficient condition of reality, which states that if without in any way disturbing a system, we can predict with some specified uncertainty the value of a physical quantity, there exists a stochastic element of physical reality which determines this physical quantity with atmost that specific uncertainty, called as \emph{Reid's extension of EPR's sufficient condition of reality}. This is attributed to the intrinsic stochastic nature exhibited in the preparation and detection of quantum states \cite{reid1989demonstration,cavalcanti2009experimental}.\newline\newline
Consider two parties, Alice and Bob sharing an entangled state. Now Alice makes a local measurement $\hat{Y}$ and makes an estimate $\hat{X}^{est}(\hat{Y})$ for the result of Bob's measurement $\hat{X}$ observing the outcomes of her own measurement $\hat{Y}$. The idea of estimation is implemented to incorporate EPR's sufficient condition of reality. Therefore the average inferred-variance of $\hat{X}$ for an estimate $\hat{X}^{est}(\hat{Y})$ is given as follows
\begin{equation}\label{eqn6}
    \Delta_{inf}^{2}\hat{X}^{2} = \langle(\hat{X} - \langle\hat{X}^{est}(\hat{Y})\rangle)^{2}\rangle.
\end{equation}
Alice's estimate for Bob's measurement is given by $\hat{X}^{est}(\hat{Y}) = g\hat{Y}$, where the choice of $g$ should be such that it gives the minimum error, i.e., $g = \frac{\langle\hat{X}\hat{Y}\rangle}{{\langle \hat{Y}^{2} \rangle}}$ gives the optimal inferred-variance. Using EPR's condition of locality, Reid's extension of EPR's sufficient condition for reality and completeness of quantum mechanics, a limit on the product of inferred-variances based on HUR for two noncommuting quadrature phase amplitude observables $\hat{X}_{1}$ and $\hat{X}_{2}$ on Bob's side is \cite{reid1989demonstration}
\begin{equation}\label{eqn7}
\Delta_{inf}^{2}\hat{X}_{1}\Delta_{inf}^{2}\hat{X}_{2} \geq 1.
\end{equation}
This is known as \emph{EPR-Reid criterion}. A state will show steering if Eq.(\ref{eqn7}) is violated, which has also been verified experimentally \cite{cavalcanti2009experimental}.
\section{EPR-steering criterion using Schr\"odinger-Robertson uncertainty relation}
Our derivation of EPR-steering is based on the works of \cite{reid1989demonstration,cavalcanti2007uncertainty,cavalcanti2009experimental}. Here, we use a different notation for the outcomes of measurement. Consider the outcomes $A$ and $B$, corresponding to observables $\hat{A}$ and $\hat{B}$, for Alice's and Bob's measurements respectively. Using the EPR-Reid criterion and the LHS model for $A$ and $B$, the inferred-variance is written as
\begin{equation}\label{eqn8}
    \Delta_{inf}^{2}B = \langle\left(B - B^{est}(A)\right)^{2}\rangle.
\end{equation}
The inferred-variance $\Delta_{inf}^{2}B$ is minimized (optimized) when $B^{est}(A) = \langle B \rangle_{A}$. So the minimized inferred-variance $ \Delta^{2}_{min}B$ is as follows
\begin{eqnarray}\label{eqn9}
     \Delta^{2}_{min}B &= \langle (B  - \langle B 
 \rangle_{A})^{2}\rangle = \sum_{A,B} P(A,B)(\langle B - \langle B \rangle_{A})^{2} \nonumber \\
                            &= \sum_{A} P(A) \sum_{B} P(B|A)(B - \langle B \rangle_{A})^{2} \nonumber \\
                            &= \sum_{A} P(A) \Delta^{2} (B|A).
\end{eqnarray}
where $\Delta^{2} (B|A)$ is calculated from the conditional probability distribution $P(B|A)$, stands for the conditional variance of Bob's measurement outcome $B$ provided, the outcome A of Alice's measurement is known. So we have the following condition
\begin{equation}\label{eqn10}
    \Delta^{2}_{inf}B \geq \Delta^{2}_{min}B.
\end{equation}
Assuming the LHS model Eq.(\ref{eqn3},\ref{eqn5}), the conditional probability distribution $P(B|A)$ can be written as follows
\begin{eqnarray}\label{eqn11}
    P(B|A) &= \frac{P(A,B)}{P(A)} = \sum_{\eta} \frac{P(\eta) P(A|\eta)}{P(A)} P_{Q}(B|\eta) \nonumber \\
                       &= \sum_{\eta} P(\eta|A)P_{Q}(B|\eta)
\end{eqnarray}
Here, $\eta$ is a classical random variable such that, $P(\eta) \geq 0$ and $\sum_{\eta}P(\eta) = 1$. Moreover, we can observe that the basic essence of adopting the LHS model is statistical independence of probabilities, which is one of the most important prescriptions in the local hidden variable (LHV) theory by Bell \cite{bell1964einstein}. If $P(u)$ is a classical probability distribution, which has a convex decomposition i.e. $P(u) = \sum_{v} P(u)P(u|v)$, then the variance $\Delta^{2} u$ corresponding to the probability distribution $P(u)$ is bounded by the average of the variances $\Delta^{2}(u|v)$ over the conditional distribution $P(u|v)$, i.e. $\Delta^{2} u \geq \sum_{u}P(u)\Delta^{2}(u|v)$. Therefore, from Eq.(\ref{eqn9}), the variance of the conditional measurement outcomes $B|A$ is given as 
\begin{equation}\label{eqn12}
    \Delta^{2}(B|A) \geq \sum_{\eta}P(\eta|A)\Delta^{2}_{Q}(B|\eta
    )
\end{equation}
where, the variance $\Delta^{2}_{Q}(B|\eta)$ is calculated using the conditional quantum probability distribution $P_{Q}(B|\eta) = \textnormal{Tr}[\hat{B}\hat{\rho}_{\eta}]$. The average of the measurement operator $\hat{B}$, specified by its outcome $B$ is calculated corresponding to a local quantum hidden state described by $\hat{\rho}_{\eta}$. Therefore the bound for $\Delta_{min}^{2}B$, using Eq.(\ref{eqn9},\ref{eqn12}) is given by

\begin{eqnarray}\label{eqn13}
\Delta^{2}_{min}B &\geq \sum_{A} P(A) \Delta^{2} (B|A) \nonumber \\ &\geq \sum_{A} P(A) \sum_{\eta}P(\eta|A)\Delta^{2}_{Q}(B|\eta) \nonumber \\ &\geq \sum_{A,\eta} P(A,\eta)\Delta^{2}_{Q}(B|\eta) \nonumber \\ &\geq \sum_{\eta} P(\eta)\Delta^{2}_{Q}(B|\eta).
\end{eqnarray}
Consider Bob's arbitrary local measurement operators $\hat{B_{1}},\hat{B}_{2}$ with their corresponding outcomes $B_{1}, B_{2}$. These operators then satisfy the SRUR \cite{robertson1934indeterminacy}
\begin{eqnarray}\label{eqn14}
\langle\Delta^{2}\hat{B}_{1}\rangle \langle\Delta^{2} B_2\rangle \geq \frac{1}{4}|\langle[\hat{B}_{1},\hat{B}_{2}]\rangle|^2 \nonumber + \\
\hspace{3cm} \frac{1}{4}(\langle\{\hat{B}_1,\hat{B}_2\} \rangle - 2\langle \hat{B}_1\rangle \langle \hat{B}_2 \rangle)^2
\end{eqnarray}
where, $\{\hat{B_1},\hat{B_2}\}$ is the anticommutator, $[\hat{B}_{1},\hat{B}_{2}]$ is the commutator.  $\langle\Delta_{Q}^{2}\hat{B_i}\rangle_{\hat{\rho}}$ is the variance and $\langle\hat{B_i}\rangle_{\hat{\rho}}$ is the average calculated for a quantum state.
The above equation can be written in terms of the outcomes of Bob given by,
\begin{eqnarray}\label{eqn15}
    \langle\Delta_{Q}^{2}B_{1}\rangle \langle\Delta_{Q}^{2} B_2\rangle \geq \frac{1}{4}|\langle[B_{1},B_{2}]\rangle_{Q}|^2 + \nonumber \\ \hspace{2.5cm} \frac{1}{4}(\langle\{B_1,B_2\}\rangle_{Q} -  2\langle B_1 \rangle_{Q} \langle B_2 \rangle_{Q})^2 .
\end{eqnarray}
For any two vectors $\textbf{u}$ and $\textbf{v}$ in a linear vector space, the Cauchy-Schwartz inequality is given by
\begin{equation}\label{eqn16}
 ||\textbf{u}||^{2}_{2}||\textbf{v}||^{2}_{2} \geq |\langle\textbf{u},\textbf{v}\rangle|^2   
\end{equation}
where $||.||_{2}$ is L2 norm, $\langle.\rangle$ is inner product and $|.|$ is the modulus in the linear vector space.
Using Eq.(\ref{eqn12},\ref{eqn13}) the  vectors \textbf{u} and \textbf{v} can be defined as
\begin{eqnarray}\label{eqn17}
   \textbf{u} &\equiv \{\sqrt{P(\eta_1)}\Delta_{Q}(B_1|\eta_1), \sqrt{P(\eta_2)}\Delta_{Q}(B_1|\eta_2), ...\} \nonumber \\
   \textbf{v} &\equiv \{\sqrt{P(\eta_1)}\Delta_{Q}(B_2|\eta_1), \sqrt{P(\eta_2)}\Delta_{Q}(B_2|\eta_2), ...\}.
\end{eqnarray}
From Eq.(\ref{eqn13}) and comparing Eqn(\ref{eqn17}), we have, $\Delta^{2}_{min}B_{1} \geq ||\textbf{u}||_{2}^{2}$ and $\Delta^{2}_{min}B_{2} \geq ||\textbf{v}||_{2}^{2}$. Hence, we employ Eq.(\ref{eqn13},\ref{eqn14},\ref{eqn16}) for the two outcomes $\hat{B}_{1}$ and $\hat{B}_{2}$, and obtain a lower bound for the product of multiplicative variances $\Delta^{2}_{min}B_{1}$ and $\Delta^{2}_{min}B_{2}$ which is given by,
\begin{eqnarray}\label{eqn18}
   \Delta^{2}_{min}B_{1}\Delta^{2}_{min}B_{2} &\geq ||\textbf{u}||^{2}_{2}||\textbf{v}||^{2}_{2} \geq |\langle\textbf{u},\textbf{v}\rangle|^{2} \nonumber
   \\ &\geq \sum_{\eta}P(\eta)\Delta_{Q}^{2}(B_{1}|\eta)\Delta_{Q}^{2}(B_{2}|\eta).
\end{eqnarray}
Using Eq.(\ref{eqn15}), the above inequality can be written as,
\begin{eqnarray}\label{eqn19}
    \Delta^{2}_{min}B_{1}\Delta^{2}_{min}B_{2} \geq
    \frac{1}{4}\sum_{\eta}P(\eta)[|\langle B_{3} \rangle_{\eta}|^{2} + \\ \hspace{3.5cm} \left(\langle\{B_{1},B_{2}\}\rangle_{\eta} - 2\langle B_{1}\rangle_{\eta}\langle B_{2}\rangle_{\eta}\right)^{2}] \nonumber
\end{eqnarray}
where, $\langle B_{i} \rangle_{\eta}$ is the mean w.r.t probability distribution $P_{Q}(B_{i}|\eta)$. Using the properties of convex functions and Jensen's inequality $|\alpha|^{2}$, where $\alpha$ is a random variable, we have $\sum_{\alpha}P(\alpha)|\alpha|^{2}  \geq  |\sum_{\alpha}P(\alpha)\alpha|^{2}$ for a given probability distribution $P(\alpha)$, the RHS of Eq.(\ref{eqn18}) in terms of inferred-variances and averages is written as (refer Appendix for details),
\begin{eqnarray}\label{eqn20}
    \Delta^{2}_{min}B_{1}\Delta^{2}_{min}B_{2} \geq  \frac{1}{4}|\langle[B_{1},B_{2}]\rangle_{inf}|^{2} \\ \hspace{2cm}
 + \frac{1}{4}\left(\langle \{B_{1},B_{2}\}\rangle_{inf} - 2\langle B_{1}\rangle_{inf}\langle B_{2}\}\rangle_{inf}\right)^{2}.\nonumber
\end{eqnarray}
From Eq.(\ref{eqn10}), using the condition, $\Delta^{2}_{inf}B \geq \Delta^{2}_{min}B$ and employing in Eq.(\ref{eqn19}), the EPR-steering criterion based on SRUR can be written as follows 
\begin{eqnarray}\label{eqn21}
    \Delta^{2}_{inf}B_{1}\Delta^{2}_{inf}B_{2} \geq \frac{1}{4}|\langle[B_{1},B_{2}]\rangle_{inf}|^{2} + \\  \hspace{2cm} \frac{1}{4}\left(\langle \{B_{1},\hat{B}_{2}\}\rangle_{inf} - 2\langle B_{1}\rangle_{inf}\langle B_{2}\rangle_{inf}\right)^{2}. \nonumber
\end{eqnarray}
\section{Results and Discussion}
Our choice for Bob's measurement operators are spin observables, and we check the effectiveness of our steering criterion Eq.(\ref{eqn21}) i.e. $\hat{B}_{1} = \hat{S}_{B_x}$,$\hat{B}_{2} = \hat{S}_{B_y}$ with their corresponding outcomes $S_{B_{x}}$, $S_{B_{y}}$ respectively. We obtain a steering inequality for two-qubit Werner state in one-sided two-measurement two-outcome scenarios. Therefore, Eq.(\ref{eqn20}) for spin observables can be written in the following way
\begin{eqnarray}\label{eqn22}
\Delta^{2}_{inf}S_{B_x}\Delta^{2}_{inf}S_{B_y} \geq \frac{1}{4}|\langle S_{B_{z}}\rangle_{inf}|^{2} +  \\ \hspace{1.8cm} \frac{1}{4}\left(\langle \{S_{B_x}, S_{B_y}\}\rangle_{inf} - 2\langle S_{B_x}\rangle_{inf}\langle S_{B_y}\rangle_{inf}\right)^{2} \nonumber
\end{eqnarray}
\newline
where  $\Delta^{2}_{inf} S_{B_i} = \langle(S_{B_i} - S_{B_i}^{est}(S_{A_i}))^{2}\rangle = \langle(S_{B_i} - g_{i}S_{A_i})^{2}\rangle, g_i = \frac{\langle S_{A_i}S_{B_i}\rangle}{\langle S_{A_i}^{2}\rangle}$, $i = x,y$ 
 and $\langle S_{B_j}\rangle_{inf} = \sum_{S_{A_j}}P(S_{A_j})\langle S_{B_j}\rangle_{S_{A_j}}$, $j = x,y,z$.
Calculation of inferred-variance and inferred-mean for two-qubit Werner state $\hat{\rho}_I = \eta|\Psi^-\rangle\langle\Psi^-| + \frac{1-\eta}{4}\hat{I}$, where $|\Psi^{-}\rangle = \frac{1}{\sqrt{2}}(|01\rangle - |10\rangle)$ is given by $\Delta^{2}_{inf} S_{B_x} = \Delta^{2}_{inf} S_{B_y} = \frac{1}{4}(1 - \eta^{2})$, $\langle\{S_{B_x}, S_{B_y}\}\rangle = 0$, $\langle S_{B_x}\rangle_{inf} = \langle S_{B_y}\rangle_{inf} = \langle S_{B_z}\rangle_{inf} = \frac{\eta}{2}$. Using these values in (\ref{eqn21}), we obtain the following condition
 \begin{equation}\label{eqn23}
     \eta \leq \frac{1}{\sqrt{3}}.
 \end{equation}
 Violation of Eq.(\ref{eqn22}), i.e. $\eta > \frac{1}{\sqrt{3}}$, detect steerable two qubit Werner states for $\eta \in [0,1]$. Werner state was shown to be steerable in theory with an infinite number of measurements for $\eta > \frac{1}{2}$ [29]. This is not achievable in experiments (i.e. for a finite number of measurements). In [30], it was shown that two-qubit Werner states violate steering inequality for $\eta > \frac{1}{\sqrt{3}}$ for three measurement settings \cite{cavalcanti2009spin}. However, we obtain this result in one-side, two-measurement and two-outcome scenarios.
\section{Conclusion}
 In summary, utilizing the LHS model and Reid's criterion in SRUR, we derive the EPR-steering criterion for bipartite systems \cite{einstein_1935,einstein_1936,reid1989demonstration} for one-sided, two-measurement and two-outcome scenarios. Interestingly, this steering condition yields a stricter bound compared to earlier works \cite{cavalcanti2009experimental,Chowdhury2013EinsteinPodolskyRosenSU,sasmal2018tighter,Saunders2009ExperimentalEU,maity2017tighter}. Moreover, in the context of discrete variable systems, the steering criterion gives a stronger violation, $\eta > \frac{1}{\sqrt{3}}$, for two-qubit Werner state which was earlier obtained for one-sided three-measurement and four-measurement cases. Moreover, the steering criterion obtained can be utilized to obtain stricter bound for higher dimensional states in the discrete variable case. It is also tempting to look for stricter bounds of steering in continuous variable systems. Furthermore, this criterion can be implemented using measurements corresponding to positive operator-valued measures (POVM). One of the works shows the correspondence between joint measurability and steering \cite{Quintino2014JointME}. Therefore, it would be interesting to look for the correspondence between steering and uncertainty relations involving not jointly measurable POVMs.
\section{Acknowledgments}
We would like to thank Mr Abhinash Kumar Roy, Mr Prabhuda Roy, Mr Sumit Mukherjee, Mr Arman and Mr Rajiuddin for numerous enlightening discussions. PKP acknowledge the support from DST, India, through Grant No. DST/ICPS/QuST/Theme-1/2019/2020-21/01
\appendix
\section*{Appendix}
\setcounter{section}{1}
 Consider the inequality developed in Eq.(\ref{eqn18}). For Bob's two measurement outcomes $B_1$ and $B_2$, corresponding to operator $\hat{B}_{1}$ and $\hat{B}_{2}$, we have the following relation
\begin{eqnarray}\label{a1}
     \Delta^{2}_{min}B_{1}\Delta^{2}_{min}B_{2} \geq 
     \frac{1}{4}\sum_{\eta}P(\eta)|\langle B_3\rangle_{\eta}|^{2} + \\ 
     \hspace{2cm}\frac{1}{4}\sum_{\eta}P(\eta)\left(\langle\{B_{1}, B_{2}\}\rangle_{\eta} - 2\langle B_{1}\rangle_{\eta}\langle B_{2}\rangle_{\eta}\right)^{2}.\nonumber
\end{eqnarray}
The RHS of the above equation is given by,
\begin{eqnarray}\label{a2}
      \sum_{\eta}P(\eta)\left[|\langle B_{3}\rangle_{\eta}|^{2} + \left(\{B_{1}, B_{2}\}\rangle_{\eta} - 2\langle B_{1} \rangle_{\eta} \langle B_{2}\rangle_{\eta}\right)^{2}\right] = 
      \nonumber \\\sum_{\eta} P(\eta) |\langle B_3\rangle_{\eta}|^{2} + \langle\{B_{1},B_{2}\}\rangle_{\eta}^{2}
      + 4\langle B_{1} \rangle_{\eta}^{2} \langle B_{2}\rangle_{\eta}^{2}  \\ \hspace{4cm}
      - 4\langle\{ B_{1},B_{2}\}\rangle_{\eta} \langle B_{1}\rangle_{\eta} \langle B_2\rangle_{\eta}. \nonumber
 \end{eqnarray}
For all the measurement outcomes $A_{i}$ corresponding to the operators $\hat{A}_{i}$ that exhaust Alice's measurement setting, the RHS of Eq.(\ref{a2}) is as follows
\begin{eqnarray}\label{a3}
\sum_{A_{3},\eta}P(A_{3},\eta)|\langle B_3\rangle_{\eta}|^{2} + \sum_{A_{1}A_{2},\eta}P\left(A_{1}A_{2},\eta\right)\langle\{B_{1}, B_{2}\}\rangle_{\eta}^{2} \nonumber \\ \hspace{0.5cm} + 4 \sum_{A_{1},\eta}P(A_{1},\eta)\langle B_{1}\rangle_{\eta}^{2}\sum_{A_{2},\eta}P(A_{2},\eta)\langle B_{2})\rangle_{\eta}^{2} \nonumber \\ \hspace{1.5cm}- 4 \sum_{A_{1}A_{2},\eta}P(A_{1}A_{2},\eta)\langle\{B_{1},B_{2}\}\rangle_{\eta} \\ \hspace{2.5cm}\sum_{A_{2},\eta}P(A_{2},\eta)\langle B_{2}\rangle_{\eta}\sum_{A_{1},\eta}P(A_{1},\eta)\langle B_{1}\rangle_{\eta} \nonumber
\end{eqnarray}
where, $P(A_{i},\eta)$, for $i = 1,2$ is the joint probability distribution of outcomes $A_{i}$ and classical random variable $\eta$. $P(A_{i}A_{j},\eta)$ for $i,j = 1,2$ is the joint probability distribution between $\eta$ and the joint outcomes $A_{1}A_{2}$ of the joint measurement observables $\hat{A}_{1}\hat{A}_{2}$. $P(A_{i})$ for $i = 1,2,3$ is probability distribution for outcomes $A_{i}$, corresponding to the measurement observables $\hat{A}_{i}$. $|u|^{2}$ and $u^{2}$ are convex functions for a real variable $u$. Hence, by Jensen's inequality $\sum_{u}P(u)|u|^{2} \geq |\sum_{u}P(u)u|^{2}$. Therefore, Eq.(\ref{a3}) is lower bounded, i.e.,
\begin{eqnarray}\label{a4}
    \geq \sum_{A_{3}}P(A_{3})\left|\sum_{\eta}P(\eta|A_{3})\langle B_{3}\rangle_{\eta}\right|^{2} + \\ \hspace{1cm}\sum_{A_{1}A_{2}}P(A_{1}A_{2})\left(\sum_{\eta}P(\eta|A_{1}A_{2})\langle\{B_{1}, B_{2}\}\rangle_{\eta}\right)^{2}+ \nonumber\\ \hspace{3.5cm}\sum_{A_{1}}P(A_{1})\left(\sum_{\eta}P(\eta|A_{1})\langle B_{1}\rangle_{\eta}\right)^{2} \nonumber \\ \sum_{A_{2}}P(A_{2})\left(\sum_{\eta}P(\eta|A_{2})\langle B_{2}\rangle_{\eta})\right)^{2} - \nonumber \\ \hspace{1cm} 4 \sum_{A_{1}}P(A_{1})\left(\sum_{\eta}P(\eta|A_{1})\langle B_{1}\rangle_{\eta}\right) \nonumber \\ \hspace{2cm}\sum_{{A_2}}P(A_{2})\left(\sum_{\eta}P(\eta|A_{2})\langle B_{2}\rangle_{\eta}\right) \nonumber \\
    \hspace{1.5cm}\sum_{A_{1},A_{2}}P(A_{1}A_{2}) \left(\sum_{\eta}P(\eta|A_{1}A_{2})\langle\{B_{1},B_{2}\}\rangle_{\eta} \right) \nonumber
\end{eqnarray}
Here, the conditional probability distributions of $\eta$; $P(\eta|A_{i})$, where, $i = 1,2,3$, for measurement outcomes $A_{i}$ corresponds to measurement operators $\hat{A}_{i}$ and $P(\eta|A_{1}A_{2})$ with the given measurement outcomes $A_{1}A_{2}$ for joint measurement operators $\hat{A}_{1}\hat{A}_{2}$. Considering the probabilities $P(\eta)$ for all values of $\eta$, we have the following equation independent of $\eta$ which is given as
\begin{eqnarray}\label{a5}
  = \sum_{A_{3}}P(A_{3}) |\langle B_{3}\rangle_{A_{3}}|^{2} +  \sum_{A_{1}A_{2}}P(A_{1}A_{2})\langle \{B_{1},B_{2}\}\rangle_{A_{1}A_{2}}^{2}\nonumber\\ 
  \hspace{1.2cm} - 4 \sum_{A_{1}}P(A_{1})\langle B_{1}\rangle^{2}_{A_{1}}\sum_{A_{2}}P(A_{2})\langle B_{2}\rangle^{2}_{A_{2}} \nonumber \\ - 4 \sum_{A_{1}A_{2}}P(A_{1}A_{2})\langle \{B_{1},B_{2}\}\rangle_{A_{1}A_{2}}\sum_{A_{2}}P(A_{2})\langle B_{2}\rangle_{A_{2}} \nonumber \\
  \hspace{4.7cm}\sum_{A_{1}}P(A_{1})\langle B_{1}\rangle_{a_{1}}.
\end{eqnarray}
The terms in (\ref{a5}) are called as inferred-averages because these are calculated under conditional probability distributions. Hence, (\ref{a5}) is given as follows
\begin{eqnarray}\label{a6}
     |\langle B_{3}\rangle_{inf}|^{2} + \langle \{B_{1},B_{2}\}\rangle_{inf}^{2} + 4\langle B_{1}\rangle_{inf}^{2}\langle B_{2}\rangle_{inf}^{2}
     \\
     \hspace{3.5cm} - 4\langle B_{1},B_{2}\}\rangle_{inf}\langle B_{1}\rangle_{inf}\langle B_{2}\}\rangle_{inf} \nonumber \\
     = |\langle B_{3}\rangle_{inf}|^{2} + \left(\langle \{B_{1},B_{2}\}\rangle_{inf} - 2\langle B_{1}\rangle_{inf}\langle B_{2}\}\rangle_{inf}\right)^{2}\nonumber
\end{eqnarray}\newline
Using Eq.(\ref{a1}), we have the following bound for the product of minimum inferred-variances.
\begin{eqnarray}
         \Delta^{2}_{min}B_{1}\Delta^{2}_{min}B_{2} \geq \frac{1}{4}|\langle[B_{1},B_{2}]\rangle_{inf}|^{2} + \\ \hspace{2.2cm}\frac{1}{4}\left(\langle \{B_{1},\hat{B}_{2}\}\rangle_{inf} - 2\langle B_{1}\rangle_{inf}\langle B_{2}\rangle_{inf}\right)^{2}. \nonumber
\end{eqnarray}
\section*{References}
\bibliography{iopart-num}
\end{document}

