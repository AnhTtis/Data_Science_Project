\begin{table*}[t!]
    \subfloat[
        \textbf{View generation.}Views produced by our enhanced data augmentation are stronger than the original RGB-D frames. Scene-level views can significantly speed up pre-training and make contrastive learning more effective. The performance can be further boosted with additional masked point modeling.
        \vspace{1mm}
        \label{tab:ablation_view_generation}
    ]{
        \begin{minipage}{0.98\linewidth}{\begin{center}
            \tablestyle{6pt}{1.02}
            \begin{tabular}{lccccccccc}
\toprule
View generation methods &Pre-train data &Storage &Batch size &Iters &Epochs &FT mIoU (\%) &Hours (h) &Speedup \\
\midrule
Frame matching (PointContrast~\cite{xie2020pointcontrast}) &ScanNet Raw &500G &32 &100k &5 &74.0 &48 & 1.0$\times$ \\
Scene augmentation w/o mask (ours) &ScanNet v2 &20G &32 &30k &600 &74.4 &\textbf{11} &\textbf{4.4$\times$} \\
\cellcolor[HTML]{efefef}Scene augmentation w mask (ours) &ScanNet v2 &20G &32 &30k &600 &\textbf{75.0} &14 &3.4$\times$ \\
\bottomrule
\end{tabular}

        \end{center}}\end{minipage}
    } \\
    \centering
    \subfloat[
        \textbf{Number of positive pairs.} A larger amount of sampled positive pairs are necessary for scene-level views.
        \label{tab:ablation_ft_bs}
    ]{
        \begin{minipage}{0.29\linewidth}{\begin{center}
            \tablestyle{4pt}{1.02}
            \begin{tabular}{cccc}\toprule
\#Pos pairs &PC~\cite{xie2020pointcontrast} &MSC~(ours) \\\midrule
1024 &73.8 &74.3 \\
2048 &74.0 &74.5 \\
4098 &73.7 &74.9 \\
\cellcolor[HTML]{efefef}8192 &73.9 &\textbf{75.0} \\
\bottomrule
\end{tabular}
        \end{center}}\end{minipage}
    }
    \hspace{5mm}
    \subfloat[
        \textbf{Data augmentation.} The combination of spatial and photometric augmentation makes the view generation pipeline comes to work.
        \label{tab:ablation_augmentation}
    ]{
        \begin{minipage}{0.29\linewidth}{\begin{center}
            \tablestyle{4pt}{1.02}
            \begin{tabular}{cccc}\toprule
Spatial &Photometric &FT mIoU (\%) \\\midrule
w/o aug &w/o aug &72.1 \\
w aug &w/o aug &73.4 \\
w/o aug &w aug &72.8 \\
\cellcolor[HTML]{efefef}w aug &\cellcolor[HTML]{efefef}w aug &\textbf{74.4} \\
\bottomrule
\end{tabular}
        \end{center}}\end{minipage}
    }
    \hspace{5mm}
    \subfloat[
        \textbf{View mixing.} Randomly mixing query views while leaving key views unmixed is a sweet point.
        \label{tab:ablation_mix}
    ]{
        \begin{minipage}{0.29\linewidth}{\begin{center}
            \tablestyle{4pt}{1.02}
            \begin{tabular}{cccc}\toprule
Query view &Key view &FT mIoU (\%) \\\midrule
w/o mix &w/o mix &74.1 \\
\cellcolor[HTML]{efefef}w mix &\cellcolor[HTML]{efefef}w/o mix &\textbf{74.4} \\
w/o mix &w mix &74.2 \\
w mix &w mix &73.7 \\
\bottomrule
\end{tabular}
        \end{center}}\end{minipage}
    } \\
    \centering
    \vspace{1.5mm}
    \subfloat[
        \textbf{Cross mask.} Masks containing shared masked patches have a negative influence on contrastive learning.
        \label{tab:ablation_cross_mask}
    ]{
        \begin{minipage}{0.29\linewidth}{\begin{center}
            \tablestyle{4pt}{1.02}
            \begin{tabular}{cccc}\toprule
Mask &Task &FT mIoU (\%) \\\midrule
w/o cross &w/o contrast &74.1 \\
w cross &w/o contrast &74.4 \\
w/o cross &w contrast &74.7 \\
\cellcolor[HTML]{efefef}w cross &\cellcolor[HTML]{efefef}w contrast &\textbf{75.0} \\
\bottomrule
\end{tabular}
        \end{center}}\end{minipage}
    }
    \hspace{5mm}
    \subfloat[
        \textbf{Mask grid size.} Our design works with mask patches with a grid size larger than 0.1m, and we consider 0.15m as a default setting.
        \label{tab:ablation_mask_grid_size}
    ]{
        \begin{minipage}{0.29\linewidth}{\begin{center}
            \tablestyle{4pt}{1.02}
            \begin{tabular}{ccc}\toprule
Mask grid size (m) &FT mIoU (\%) \\\midrule
0.05 &74.3 \\
0.1 &75.0 \\
\cellcolor[HTML]{efefef}0.15 &\textbf{75.0} \\
0.2 &74.8 \\
\bottomrule
\end{tabular}
        \end{center}}\end{minipage}
    }
    \hspace{5mm}
    \subfloat[
        \textbf{Reconstruction target.} Both targets have a positive effect, while color reconstruction has a dominant impact on indoor scenes.
        \label{tab:ablation_reconstruction_target}
    ]{
        \begin{minipage}{0.29\linewidth}{\begin{center}
            \tablestyle{4pt}{1.02}
            \begin{tabular}{cccc}\toprule
Color &Normal &FT mIoU (\%) \\\midrule
w/o &w/o &74.4 \\
w &w/o &74.9 \\
w/o &w &74.6 \\
\cellcolor[HTML]{efefef}w &\cellcolor[HTML]{efefef}w &\textbf{75.0} \\
\bottomrule
\end{tabular}
        \end{center}}\end{minipage}
    }
    \vspace{-2mm}
    \caption{\textbf{Ablation experiments.} We adopt \textit{SparseUNet} and \textit{efficient} pre-training on ScanNet~\cite{dai2017scannet} point cloud data to ablate our designs. We report fine-turning (FT) mIoU (\%) results on ScanNet 20 classes semantic segmentation as the default metric. If not specified, the default setting is as follows: the pre-training period is 600 epochs, the masking ratio is 30\% and the masked patch has a grid size of 0.15m in the real-world space, view mixing probability is 0.8. All of our designs are enabled by default. Default settings are marked in \colorbox{gray}{gray}.} 
    \label{tab:ablation}
    \vspace{-3mm}
\end{table*}
