\begin{table}[t!]
    \centering
    \subfloat[
        \textbf{Semantic segmentation.} We conduct pre-training on SparseUNet and compare semantic segmentation mIoU~(\%) results on ScanNet and ScanNet200~\cite{rozenberszki2022language} validation set.
        \vspace{1mm}
        \label{tab:results_sem_seg}
    ]{
        \begin{minipage}{0.95\linewidth}{\begin{center}
            \tablestyle{2.5pt}{1.02}
            \begin{tabular}{cc|ccccc}\toprule
\multirow{2}{*}{Datasets} &\multirow{2}{*}{Backbones} &\multicolumn{4}{c}{Semantic Seg. (mIoU)} \\\cmidrule{3-6}
& &SC &PC~\cite{xie2020pointcontrast} &CSC~\cite{hou2021exploring} &\cellcolor[HTML]{efefef}MSC~(ours) \\\midrule
ScanNet &SparseUNet &72.2 &74.1~\tiny{\textcolor{darkgray}{(+1.9)}} &73.8~\tiny{\textcolor{darkgray}{(+1.6)}} &\textbf{75.5}~\tiny{\textcolor{darkgreen}{(+3.3)}} \\
ScanNet200 &SparseUNet &25.0 &26.2~\tiny{\textcolor{darkgray}{(+1.2)}} &26.4~\tiny{\textcolor{darkgray}{(+1.4)}}&\textbf{28.8}~\tiny{\textcolor{darkgreen}{(+3.8)}} \\
\bottomrule
\end{tabular}

        \end{center}}\end{minipage}
    }
    \\
    \centering
    \subfloat[
        \textbf{Instance segmentation.} We conduct pre-training on SparseUNet and compare instance segmentation mAP@0.5~(\%) results driven by \textit{PointGroup}~\cite{jiang2020pointgroup} on ScanNet and ScanNet200~\cite{rozenberszki2022language} validation set.
        \label{tab:results_ins_seg}
    ]{
        \begin{minipage}{0.95\linewidth}{\begin{center}
            \tablestyle{2.5pt}{1.02}
            \begin{tabular}{cc|ccccc}\toprule
\multirow{2}{*}{Datasets} &\multirow{2}{*}{Backbones} &\multicolumn{4}{c}{Instance Seg. (mAP@0.5)} \\\cmidrule{3-6}
& &SC &PC~\cite{xie2020pointcontrast} &CSC~\cite{hou2021exploring} &\cellcolor[HTML]{efefef}MSC~(ours) \\\midrule
ScanNet &SparseUNet &56.9 &58.0~\tiny{\textcolor{darkgray}{(+1.1)}} &59.4~\tiny{\textcolor{darkgray}{(+2.5)}} &\textbf{59.6}~\tiny{\textcolor{darkgreen}{(+2.7)}}\\
ScanNet200 &SparseUNet &24.5 &24.9~\tiny{\textcolor{darkgray}{(+0.4)}} &25.2~\tiny{\textcolor{darkgray}{(+0.7)}} &\textbf{26.8}~\tiny{\textcolor{darkgreen}{(+2.3)}}\\
\bottomrule
\end{tabular}

        \end{center}}\end{minipage}
    }
    \\% Data efficient
    \vspace{1mm}
    \subfloat[
        \textbf{Data efficiency.} We follow the ScanNet Data Efficient benchmark and compare the validation results SparseUNet with previous methods.
        \label{tab:results_data_efficient}
    ]{
        \begin{minipage}{0.95\linewidth}{\begin{center}
            \tablestyle{2.4pt}{1.02}
            \begin{tabular}{c|ccccc|c|ccccc}\toprule
LR &\multicolumn{4}{c}{Semantic Seg.} & &LA &\multicolumn{4}{c}{Semantic Seg.} \\\cmidrule{1-5}\cmidrule{6-11}
Pct. &SC &CSC &VIBUS &\cellcolor[HTML]{efefef}MSC & &Pts. &SC &CSC &VIBUS &\cellcolor[HTML]{efefef}MSC \\\midrule
100\% &72.2 &73.8 &- &\textbf{75.3} & &Full &72.2 &73.8 &- & \textbf{75.3} \\
1\% &26.0 &28.9 &28.6 &\textbf{29.2} & &20 &41.9 &55.5 &61.0 & \textbf{61.2}\\
5\% &47.8 &49.8 &47.4 &\textbf{50.7} & &50 &53.9 &60.5 &65.6 & \textbf{66.8}\\
10\% &56.7 &59.4 &60.5 &\textbf{61.0} & &100 &62.2 &65.9 &68.9 & \textbf{69.7}\\
20\% &62.9 &64.6 &64.8 &\textbf{64.9} & &200 &65.5 &68.2 &69.6 & \textbf{70.7}\\
\bottomrule
\end{tabular}
        \end{center}}\end{minipage}
    }
    \vspace{-2mm}
    \caption{\textbf{Results comparison.} We adopt cross-dataset pre-training utilizing ScanNet and ArkitScenes point cloud scenes for comparison of downstream task results. The pre-training setting is fixed as the default described in \tabref{tab:ablation}. More specific pre-training details are available in the Appendix. \textit{SC} denotes train from scratch.}
    \vspace{-5mm}
    \label{tab:results}
\end{table}