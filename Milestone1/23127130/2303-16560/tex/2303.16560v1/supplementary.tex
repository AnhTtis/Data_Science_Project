\documentclass[aps,prb,amsmath,amssymb]{revtex4}

\usepackage{graphicx}% Include figure files
\usepackage{dcolumn}% Align table columns on decimal point
\usepackage{bm}% bold math
\usepackage[caption=false]{subfig}
\usepackage{amssymb}
\usepackage{amsmath}
%\usepackage{commath}
\usepackage{graphicx}
\usepackage{verbatim}
%SSI
\usepackage[usenames, dvipsnames]{color}
%SSI
\usepackage{units}
\usepackage[export]{adjustbox}

\usepackage{caption}

\captionsetup[table]{name=Supplementary Table}
\captionsetup[figure]{name=Supplementary Figure}

\usepackage{hyperref}
\usepackage{color}
\newcommand{\lp}[1]{\textcolor{red}{#1}}
\def\beq{\begin{equation}}
\def\eeq{\end{equation}}
\def\vk{{\bf k}}
\def\vQ{{\bf Q}}
\def\vR{{\bf R}}
\def\vr{{\bf r}}
\def\vq{{\bf q}}
\def\vs{{\bf s}}
\def\vp{{\bf p}}
\def\vF{{\bf F}}


\newcommand{\red}[1]{\textcolor[rgb]{1.0, 0.0, 0.2}{#1}}


%\renewcommand{\thefigure}{S\arabic{figure}}


\begin{document}
	
	
\title{Supplementary material for 
'The Mott transition in the 5d$^1$ compound Ba$_2$NaOsO$_6:$ a DFT+DMFT study with PAW non-collinear projectors'}
%--------------------------------------------------------------------

\author{Dario Fiore Mosca}
\affiliation{Centre de Physique Th\'eorique, Ecole Polytechnique, CNRS, Institut Polytechnique de Paris,
  91128 Palaiseau Cedex, France}
\affiliation{Coll\`ege de France, 11 place Marcelin Berthelot, 75005 Paris, France}

\author{Hermann Schnait}
\affiliation{Institute of Theoretical and Computational Physics, Graz University ofTechnology, NAWI Graz, 8010 Graz, Austria}

\author{Lorenzo Celiberti}
\affiliation{University of Vienna, Faculty of Physics and Center for Computational Materials Science, Vienna, Austria}

\author{Markus Aichhorn}
\affiliation{Institute of Theoretical and Computational Physics, Graz University ofTechnology, NAWI Graz, 8010 Graz, Austria}

\author{Cesare Franchini}
\affiliation{University of Vienna, Faculty of Physics and Center for Computational Materials Science, Vienna, Austria}
\affiliation{Department of Physics and Astronomy "Augusto Righi", Alma Mater Studiorum - Universit\`a di Bologna, Bologna, 40127 Italy}

	
\date{\today}	
\maketitle


\section{Density of States}

The following figure shows the comparison between the total DFT density of states, and the one obtained from the PAW non-collinear projectors from TRIQS.

\begin{figure}[h]
    \centering
    \includegraphics[width=0.6\linewidth]{suppl_dos.pdf}
    \caption{Comparison of the DOS from DFT and from PAW projectors.}
    \label{fig:my_label}
\end{figure}

\clearpage

\section{DFT Computational Details}

This section provides a summary of the computational methodology used for calculating the metallic and insulating density of states discussed in the main text. To perform the calculations, we utilized the VASP program and the Perdew-Burke-Ernzerhof approximation of the exchange correlation functional, with an 8 f.u. supercell of dimensions  $\sqrt{2}a\times \sqrt{2} \times a$, where $a$ is equal to 8.287 \AA. We incorporated Dudarev's DFT+U scheme, applying a $U_{eff} = U-J = 3.4$,eV, consistent with previous research~\onlinecite{mosca2021interplay}. We further deactivated all symmetries and activated the non-collinear routine, while using an energy cutoff of 600,eV with a k-mesh of $6\times6\times4$. 
Non-collinear magnetic orderings were implemented through a penalty energy functional as described in a related work of Dudarev~\onlinecite{dudarev_parametrization_2019} with a value of the penalty energy constant $\lambda$ = 10, which allowed calculations of precision of $10^{-4}$ eV. The "canted Antiferromagnetic Configuration" referred to in the main text was initiated following the guidelines of Ref.~\onlinecite{mosca2021interplay}. In case of Jahn-Teller distortions calculations, we allowed all atoms to relax and assume their optimal positions. 


\subsection{DFT + $U$ + AFM phase}

To provide a comprehensive overview, we present an additional insulating solution that does not involve the spin-orbit coupling effect. The sole distinction from the previous calculations is the adoption of a type-I antiferromagnetic phase along the [001] direction with magnetic moments oriented in the $z$-direction and without structural distortions.

\begin{figure}[h]
\begin{center}
    \includegraphics[width=0.6\linewidth]{suppl_dos_nosoc_cafm.pdf}
    \caption{Density of states of the canted AFM configuration with DFT+$U$ at $U= 3.4$\,eV.}
\end{center}
\end{figure}

\clearpage
\section{VASP DMFT computational details}

VASP employs a universal reference frame for the spin and orbital components. In this case, the spin quantization axis was aligned with the global z-axis direction. As a result, we had to implement a rotation to align the local reference frame of the Os-O octahedra with the global reference frame. To achieve this alignment, we derived the transformation using the diagonalization of the effective atomic levels. 
\footnotesize
\begin{equation}
\begin{pmatrix}
 0 & 0 & 0 & -0.817 & 0 &  0 &  0 & -0.577 &  0 &  0 \\
 0 & 0 & 0.817 &  0 & 0 &  0 &  0 &  0 &  0 &  0.577 \\
 0 & 0 & 0 &  0 & 0.577 &  0 &  0.817 &  0 &  0 &  0 \\
 0 & 0 & 0 &  0 & 0 & -0.577 &  0 &  0 & -0.817 &  0 \\
 1.000 & 0 & 0 &  0 & 0 &  0 &  0 &  0 &  0 &  0 \\
 0 & 1.000 & 0 &  0 & 0 &  0 &  0 &  0 &  0 &  0 \\
 0 & 0 & 0 & -0.577 & 0 &  0 &  0 &  0.817 &  0 &  0 \\
 0 & 0 & 0.577 &  0 & 0 &  0 &  0 &  0 &  0 & -0.817 \\
 0 & 0 & 0 &  0 & 0.817 &  0 & -0.577 &  0 &  0 &  0 \\
 0 & 0 & 0 &  0 & 0 & -0.817 &  0 &  0 &  0.577 &  0 \\
\end{pmatrix}
\end{equation}
\normalsize
which leads to the following order of the orbitals: 
$\big( d_{xy,\uparrow},d_{xy,\downarrow},d_{yz,\uparrow},d_{yz,\downarrow},d_{z^{2},\uparrow},d_{z^{2},\downarrow},d_{xz,\uparrow},d_{xz,\downarrow},d_{x^{2}-y^{2},\uparrow},d_{x^{2}-y^{2},\downarrow} \big)$. 
In TRIQS/operators the construction of the four-index U matrix follows the conventional order: $\big( d_{xy,\uparrow},d_{yz,\uparrow},d_{z^{2},\uparrow},d_{xz,\uparrow},d_{x^{2}y^{2},\uparrow},d_{xy,\downarrow},d_{yz,\downarrow},d_{z^{2},\downarrow},d_{xz,\downarrow},d_{x^{2}-y^{2},\downarrow} \big)$. A mapping between the two basis sets was adopted to make the interacting Hamiltonian of the Anderson Impurity problem consistent. 
       


\section{Wien2k DFT computational details}

To validate the results of our non-collinear projectors from VASP, we performed a similar DFT(+SOC) calculation using the LAPW basis set as implemented in Wien2k.\cite{Blaha_2020_PAPER}
In contrast to VASP, Wien2k allows the inclusion of SOC in collinear calculations by using a variational approach.
Wien2k has been used for DFT+DMFT calculations in the past, both using maximally localized Wannier function as well as projective Wannier functions.
For comparison with our non-collinear projectors from VASP, we used projective Wannier functions as implemented in dmftproj in TRIQS/DFTtools.\cite{Aichhorn2016} 
In the non-relativistic calculation, the projection was performed onto the Os t$_{2g}$ manifold, utilizing the symmetry operations employed by Wien2k.
In the relativistic case, e$_g$ and t$_{2g}$ are no longer irreducible representations of the $d$-shell, thus the projection was performed onto the full $d$ manifold.

The Wien2k DFT calculations have been performed using $5000$ k-points in the irreducible Brillouin zone using the PBE functional. After convergence was reached, the states within a energy window of $[-1, 5.8]~\text{eV}$ ($[-0.8, 6.8]~\text{eV}$ for the relativistic case) around the Fermi level were projected onto the t$_{2g}$ ($d$) shell of the Os atom.



\bibliography{biblio}


\end{document}