\documentclass[aps,prb,amsmath,amssymb]{revtex4}

\usepackage{graphicx}% Include figure files
\usepackage{dcolumn}% Align table columns on decimal point
\usepackage{bm}% bold math
\usepackage[caption=false]{subfig}
\usepackage{amssymb}
\usepackage{amsmath}
%\usepackage{commath}
\usepackage{graphicx}
\usepackage{verbatim}
%SSI
\usepackage[usenames, dvipsnames]{color}
%SSI
\usepackage{units}
\usepackage[export]{adjustbox}

\usepackage{caption}

\captionsetup[table]{name=Supplementary Table}
\captionsetup[figure]{name=Supplementary Figure}

\usepackage{hyperref}
\usepackage{color}
\newcommand{\lp}[1]{\textcolor{red}{#1}}
\def\beq{\begin{equation}}
\def\eeq{\end{equation}}
\def\vk{{\bf k}}
\def\vQ{{\bf Q}}
\def\vR{{\bf R}}
\def\vr{{\bf r}}
\def\vq{{\bf q}}
\def\vs{{\bf s}}
\def\vp{{\bf p}}
\def\vF{{\bf F}}


\newcommand{\red}[1]{\textcolor[rgb]{1.0, 0.0, 0.2}{#1}}


%\renewcommand{\thefigure}{S\arabic{figure}}


\begin{document}
	
	
\title{Supplementary material for 
'The Mott transition in the 5d$^1$ compound Ba$_2$NaOsO$_6:$ a DFT+DMFT study with PAW non-collinear projectors'}
%--------------------------------------------------------------------

\author{Dario Fiore Mosca}
\affiliation{Centre de Physique Th\'eorique, Ecole Polytechnique, CNRS, Institut Polytechnique de Paris,
  91128 Palaiseau Cedex, France}
\affiliation{Coll\`ege de France, 11 place Marcelin Berthelot, 75005 Paris, France}

\author{Hermann Schnait}
\affiliation{Institute of Theoretical and Computational Physics, Graz University ofTechnology, NAWI Graz, 8010 Graz, Austria}

\author{Lorenzo Celiberti}
\affiliation{University of Vienna, Faculty of Physics and Center for Computational Materials Science, Vienna, Austria}

\author{Markus Aichhorn}
\affiliation{Institute of Theoretical and Computational Physics, Graz University ofTechnology, NAWI Graz, 8010 Graz, Austria}

\author{Cesare Franchini}
\affiliation{University of Vienna, Faculty of Physics and Center for Computational Materials Science, Vienna, Austria}
\affiliation{Department of Physics and Astronomy "Augusto Righi", Alma Mater Studiorum - Universit\`a di Bologna, Bologna, 40127 Italy}

	
\date{\today}	
\maketitle


\section{Density of States}

The following figure shows the comparison between the total DFT density of states, and the one obtained from the PAW non-collinear projectors from TRIQS.

\begin{figure}[h]
    \centering
    \includegraphics[width=0.6\linewidth]{suppl_dos.pdf}
    \caption{Comparison of the DOS from DFT and from PAW projectors.}
    \label{fig:my_label}
\end{figure}

\clearpage

\section{DFT Computational Details}

This section provides a summary of the computational methodology used for calculating the metallic and insulating density of states discussed in the main text. To perform the calculations, we utilized the VASP program and the Perdew-Burke-Ernzerhof approximation of the exchange correlation functional, with an 8 f.u. supercell of dimensions  $\sqrt{2}a\times \sqrt{2} \times a$, where $a$ is equal to 8.287 \AA. We incorporated Dudarev's DFT+U scheme, applying a $U_{eff} = U-J = 3.4$,eV, consistent with previous research~\onlinecite{mosca2021interplay}. We further deactivated all symmetries and activated the non-collinear routine, while using an energy cutoff of 600\,eV with a k-mesh of $6\times6\times4$. 
Non-collinear magnetic orderings were implemented through a penalty energy functional as described in a related work of Dudarev~\onlinecite{dudarev_parametrization_2019} with a value of the penalty energy constant $\lambda$ = 10, which allowed calculations of precision of $10^{-4}$ eV. The "canted Antiferromagnetic Configuration" referred to in the main text was initiated following the guidelines of Ref.~\onlinecite{mosca2021interplay}. 

\textcolor{black}{In case of the Jahn-Teller distorted structure, we allowed all atoms to relax to their optimal positions. For this purpose we used a a quasi-Newton algorithm with a step width of 0.5 \AA~ and an energy convergence of 10$^{-2}$ eV. The obtained structure in cif format is reported at the end of the supplementary materials. It is important to clarify that, although JT distortions are present as a consequence of the orbital degeneracy and electron occupancy of the $t_{2g}$ states~\cite{kugel1982jahn}, the solution derived under these conditions is not physically representative.  The omission of SOC effect prevents the spin-orbital entanglement, which is critical for the shift of the system from being characterized by real cubic harmonics to spherical complex ones. Particularly in the context of a 5d$^1$ configuration,  the inclusion of SOC is expected to reduce the amplitude of the JT modes~\cite{PhysRevX.10.031043, PhysRevB.105.205142}.}


\subsection{DFT + $U$ + AFM phase}

To provide a comprehensive overview, we present an additional insulating solution that does not involve the spin-orbit coupling effect. The sole distinction from the previous calculations is the adoption of a type-I antiferromagnetic phase along the [001] direction with magnetic moments oriented in the $z$-direction and without structural distortions.

\begin{figure}[h]
\begin{center}
    \includegraphics[width=0.6\linewidth]{suppl_dos_nosoc_cafm.pdf}
    \caption{Density of states of the canted AFM configuration with DFT+$U$ at $U= 3.4$\,eV.}
\end{center}
\end{figure}

\subsection{DFT polymorphous representation using a paramagnetic supercell}

To further assess the role of electronic correlations in this compound, we have explored the use of a supercell approach that has recently been employed for obtaining paramagnetic and insulating solutions in compounds that would be otherwise metallic within the standard DFT framework~\cite{PhysRevB.100.035119, PhysRevB.102.045112}.

\noindent We have computed the Density of States for both DFT and DFT+SOC calculations using a 16 f.u. super cell with lattice constants $  a\sqrt{2} \times a \sqrt{2} \times 2a$, consisting of 160 atoms. This cell is shown in Figure~\ref{fig:paramagnetic}. 
The paramagnetic phase was here simulated with both PBE as well as SCAN potentials like in previous theoretical works~\cite{PhysRevB.102.045112, PhysRevMaterials.5.104410}. We applied the following procedure: 

\begin{enumerate}
    \item The overall shape of the supercell is kept fixed to the microscopically observed lattice symmetry Fm-3m. 
    \item A spin-paramagnetic configuration was simulated by initializing randomly the direction of the magnetic moments, such that the overall sum in the unit cell is zero. We preferred this implementation with respect to a spin-up spin-down only configuration, as our calculations were performed in a non-collinear setup~\cite{PhysRevB.92.054428}. 

    In our case 14 magnetic moments were randomly initialized, and the remaining 2 were adapted in order to recover the net zero overall magnetic moment of the supercell. A constrained magnetic moment approach on the direction of the moments was also enforced (see Figure~\ref{fig:paramagnetic} for the visualization of the magnetic moments). 
    \item The atoms were allowed to relax to their optimal configuration, while retaining the overall cubic symmetry of the cell. We further tested the case of a fully relaxed unit cell, without finding significant differences.

    A conjugate gradient algorithm was employed for this purposes, with an accuracy on the relaxation of 10$^{-3}$. 
    \item The occupation of the degenerate occupied orbitals was not forced to be the same, also by means of switching all symmetries off. 
    In such manner, there is no symmetry constraint applied for the DFT calculations. We further highlight that, especially in presence of SOC effect, the random initialization of magnetic moments produces a nudge of the atomic displacements and orbital occupations, as a consequence of the coupling between the spin and orbital momenta. 
\end{enumerate}
A k-mesh of $2\times2\times1$ was employed for the structural minimization, while a k-mesh of $3\times3\times2$ for the Density of States calculation. An energy cutoff on the plane wave expansion of 600 eV was adopted, together with a convergence criterion for the energy of 10$^{-5}$ eV.  We have not tested larger supercell structures due to the computational  cost associated with this type of calculation. 

\noindent Our results show that in absence of +U corrections, the DOS is always metallic, both with and without SOC, as well as for PBE and SCAN potentials (see figures ~\ref{fig:scan_supercell} and ~\ref{fig:supercell}). The onsite Coulomb interaction seems to be critical for the occurrence of the paramagnetic insulating phase in this compound.  

\begin{figure}
    \centering
    \includegraphics[width=0.4\linewidth]{supplementary_supercell.png}
    \caption{Supercell containing 16 f.u. with magnetic moments initialised randomly used for the DFT without and with SOC (and without U). The magnetic moments on the in-equivalent osmium sites are also thereby shown.}
    \label{fig:paramagnetic}
\end{figure}

\begin{figure}
    \centering
    \includegraphics[width=0.7\linewidth]{supplementaty_scan_paramagnetic.pdf}
    \caption{Density of States for the DFT paramagnetic calculation without and with SOC effect with SCAN potential. Both DOS provide a metallic solution.}
    \label{fig:scan_supercell}
\end{figure}

\begin{figure}
    \centering
    \includegraphics[width=0.7\linewidth]{supplementaty_paramagnetic.pdf}
    \caption{Density of States for the DFT paramagnetic calculation without and with SOC effect using PBE potential. Both DOS provide a metallic solution.}
    \label{fig:supercell}
\end{figure}


\clearpage
\section{VASP DMFT computational details}

VASP employs a universal reference frame for the spin and orbital components. In this case, the spin quantization axis was aligned with the global z-axis direction. As a result, we had to implement a rotation to align the local reference frame of the Os-O octahedra with the global reference frame. To achieve this alignment, we derived the transformation using the diagonalization of the effective atomic levels. 
\footnotesize
\begin{equation}
\begin{pmatrix}
 0 & 0 & 0 & -0.817 & 0 &  0 &  0 & -0.577 &  0 &  0 \\
 0 & 0 & 0.817 &  0 & 0 &  0 &  0 &  0 &  0 &  0.577 \\
 0 & 0 & 0 &  0 & 0.577 &  0 &  0.817 &  0 &  0 &  0 \\
 0 & 0 & 0 &  0 & 0 & -0.577 &  0 &  0 & -0.817 &  0 \\
 1.000 & 0 & 0 &  0 & 0 &  0 &  0 &  0 &  0 &  0 \\
 0 & 1.000 & 0 &  0 & 0 &  0 &  0 &  0 &  0 &  0 \\
 0 & 0 & 0 & -0.577 & 0 &  0 &  0 &  0.817 &  0 &  0 \\
 0 & 0 & 0.577 &  0 & 0 &  0 &  0 &  0 &  0 & -0.817 \\
 0 & 0 & 0 &  0 & 0.817 &  0 & -0.577 &  0 &  0 &  0 \\
 0 & 0 & 0 &  0 & 0 & -0.817 &  0 &  0 &  0.577 &  0 \\
\end{pmatrix}
\end{equation}
\normalsize
which leads to the following order of the orbitals: 
$\big( d_{xy,\uparrow},d_{xy,\downarrow},d_{yz,\uparrow},d_{yz,\downarrow},d_{z^{2},\uparrow},d_{z^{2},\downarrow},d_{xz,\uparrow},d_{xz,\downarrow},d_{x^{2}-y^{2},\uparrow},d_{x^{2}-y^{2},\downarrow} \big)$. 
In TRIQS/operators the construction of the four-index U matrix follows the conventional order: $\big( d_{xy,\uparrow},d_{yz,\uparrow},d_{z^{2},\uparrow},d_{xz,\uparrow},d_{x^{2}y^{2},\uparrow},d_{xy,\downarrow},d_{yz,\downarrow},d_{z^{2},\downarrow},d_{xz,\downarrow},d_{x^{2}-y^{2},\downarrow} \big)$. A mapping between the two basis sets was adopted to make the interacting Hamiltonian of the Anderson Impurity problem consistent. 
       


\section{Wien2k DFT computational details}

To validate the results of our non-collinear projectors from VASP, we performed a similar DFT(+SOC) calculation using the LAPW basis set as implemented in Wien2k.\cite{Blaha_2020_PAPER}
In contrast to VASP, Wien2k allows the inclusion of SOC in collinear calculations by using a variational approach.
Wien2k has been used for DFT+DMFT calculations in the past, both using maximally localized Wannier function as well as projective Wannier functions.
For comparison with our non-collinear projectors from VASP, we used projective Wannier functions as implemented in dmftproj in TRIQS/DFTtools.\cite{Aichhorn2016} 
In the non-relativistic calculation, the projection was performed onto the Os t$_{2g}$ manifold, utilizing the symmetry operations employed by Wien2k.
In the relativistic case, e$_g$ and t$_{2g}$ are no longer irreducible representations of the $d$-shell, thus the projection was performed onto the full $d$ manifold.

The Wien2k DFT calculations have been performed using $5000$ k-points in the irreducible Brillouin zone using the PBE functional. After convergence was reached, the states within a energy window of $[-1, 5.8]~\text{eV}$ ($[-0.8, 6.8]~\text{eV}$ for the relativistic case) around the Fermi level were projected onto the t$_{2g}$ ($d$) shell of the Os atom.


\section{JT-distorted Structure}
\footnotesize

\begin{verbatim}
    
    

#======================================================================
# CRYSTAL DATA
#----------------------------------------------------------------------
data_VESTA_phase_1

_chemical_name_common                  'BNOO                                  '
_cell_length_a                         11.719587
_cell_length_b                         11.719587
_cell_length_c                         8.287000
_cell_angle_alpha                      90.000000
_cell_angle_beta                       90.000000
_cell_angle_gamma                      90.000000
_cell_volume                           1138.208861
_space_group_name_H-M_alt              'P 1'
_space_group_IT_number                 1

loop_
_space_group_symop_operation_xyz
   'x, y, z'

loop_
   _atom_site_label
   _atom_site_occupancy
   _atom_site_fract_x
   _atom_site_fract_y
   _atom_site_fract_z
   _atom_site_adp_type
   _atom_site_U_iso_or_equiv
   _atom_site_type_symbol
   Na1        1.0     0.000000     0.500000     0.500000    Uiso  ? Na
   Na2        1.0     0.500000     0.500000     0.500000    Uiso  ? Na
   Na3        1.0     0.500000     0.000000     0.500000    Uiso  ? Na
   Na4        1.0     0.750000     0.750000     0.000000    Uiso  ? Na
   Na5        1.0     0.750000     0.250000     0.000000    Uiso  ? Na
   Na6        1.0     0.000000     0.000000     0.500000    Uiso  ? Na
   Na7        1.0     0.250000     0.250000     0.000000    Uiso  ? Na
   Na8        1.0     0.250000     0.750000     0.000000    Uiso  ? Na
   Os1        1.0     0.000000     0.000000     0.000000    Uiso  ? Os
   Os2        1.0     0.500000     0.000000     0.000000    Uiso  ? Os
   Os3        1.0     0.500000     0.500000     0.000000    Uiso  ? Os
   Os4        1.0     0.750000     0.250000     0.500000    Uiso  ? Os
   Os5        1.0     0.750000     0.750000     0.500000    Uiso  ? Os
   Os6        1.0     0.000000     0.500000     0.000000    Uiso  ? Os
   Os7        1.0     0.250000     0.750000     0.500000    Uiso  ? Os
   Os8        1.0     0.250000     0.250000     0.500000    Uiso  ? Os
   Ba1        1.0     0.000000     0.250000     0.750000    Uiso  ? Ba
   Ba2        1.0     0.500000     0.250000     0.750000    Uiso  ? Ba
   Ba3        1.0     0.500000     0.750000     0.750000    Uiso  ? Ba
   Ba4        1.0     0.750000     0.500000     0.250000    Uiso  ? Ba
   Ba5        1.0     0.750000     0.000000     0.250000    Uiso  ? Ba
   Ba6        1.0     0.000000     0.750000     0.750000    Uiso  ? Ba
   Ba7        1.0     0.250000     0.000000     0.250000    Uiso  ? Ba
   Ba8        1.0     0.250000     0.500000     0.250000    Uiso  ? Ba
   Ba9        1.0     0.000000     0.750000     0.250000    Uiso  ? Ba
   Ba10       1.0     0.500000     0.750000     0.250000    Uiso  ? Ba
   Ba11       1.0     0.500000     0.250000     0.250000    Uiso  ? Ba
   Ba12       1.0     0.750000     0.000000     0.750000    Uiso  ? Ba
   Ba13       1.0     0.750000     0.500000     0.750000    Uiso  ? Ba
   Ba14       1.0     0.000000     0.250000     0.250000    Uiso  ? Ba
   Ba15       1.0     0.250000     0.500000     0.750000    Uiso  ? Ba
   Ba16       1.0     0.250000     0.000000     0.750000    Uiso  ? Ba
   O1         1.0     0.862463     0.363592     0.500435    Uiso  ? O
   O2         1.0     0.364074     0.366082     0.499509    Uiso  ? O
   O3         1.0     0.362344     0.864628     0.500641    Uiso  ? O
   O4         1.0     0.615139     0.615254    -0.000986    Uiso  ? O
   O5         1.0     0.612465     0.114801     0.999247    Uiso  ? O
   O6         1.0     0.865633     0.864649     0.499233    Uiso  ? O
   O7         1.0     0.114833     0.115444     0.000289    Uiso  ? O
   O8         1.0     0.112369     0.613739     1.000283    Uiso  ? O
   O9         1.0     0.137398     0.635498     0.499586    Uiso  ? O
   O10        1.0     0.634649     0.635193     0.500958    Uiso  ? O
   O11        1.0     0.637695     0.136352     0.499242    Uiso  ? O
   O12        1.0     0.884890     0.884793     0.999805    Uiso  ? O
   O13        1.0     0.887509     0.386214     0.999933    Uiso  ? O
   O14        1.0     0.135695     0.134000     0.500057    Uiso  ? O
   O15        1.0     0.385302     0.384383     1.000576    Uiso  ? O
   O16        1.0     0.387835     0.885087     0.000450    Uiso  ? O
   O17        1.0     0.134729     0.363365     0.501657    Uiso  ? O
   O18        1.0     0.636574     0.365338     0.499360    Uiso  ? O
   O19        1.0     0.635664     0.864791     0.501321    Uiso  ? O
   O20        1.0     0.886863     0.615576     0.999249    Uiso  ? O
   O21        1.0     0.885819     0.114033     0.001684    Uiso  ? O
   O22        1.0     0.137115     0.864938     0.499564    Uiso  ? O
   O23        1.0     0.386451     0.114591    -0.000263    Uiso  ? O
   O24        1.0     0.385738     0.613600     0.001800    Uiso  ? O
   O25        1.0     0.864090     0.635211     0.498257    Uiso  ? O
   O26        1.0     0.363054     0.635232     0.500405    Uiso  ? O
   O27        1.0     0.365530     0.136665     0.498932    Uiso  ? O
   O28        1.0     0.613340     0.885199     1.000378    Uiso  ? O
   O29        1.0     0.614159     0.386414     0.998278    Uiso  ? O
   O30        1.0     0.863199     0.134419     0.500748    Uiso  ? O
   O31        1.0     0.113559     0.384815     0.000728    Uiso  ? O
   O32        1.0     0.114365     0.885972     0.998383    Uiso  ? O
   O33        1.0     0.999800     0.499344     0.769849    Uiso  ? O
   O34        1.0     0.500616     0.498966     0.773016    Uiso  ? O
   O35        1.0     0.501490     0.000246     0.770145    Uiso  ? O
   O36        1.0     0.750347     0.749191     0.274021    Uiso  ? O
   O37        1.0     0.750104     0.250340     0.269686    Uiso  ? O
   O38        1.0     0.001625     0.998978     0.773430    Uiso  ? O
   O39        1.0     0.251094     0.248512     0.273970    Uiso  ? O
   O40        1.0     0.250814     0.749734     0.269732    Uiso  ? O
   O41        1.0     0.999645     0.499699     0.230182    Uiso  ? O
   O42        1.0     0.498909     0.501705     0.226946    Uiso  ? O
   O43        1.0     0.498722     0.000677     0.229843    Uiso  ? O
   O44        1.0     0.749310     0.751557     0.725862    Uiso  ? O
   O45        1.0     0.750283     0.250448     0.730243    Uiso  ? O
   O46        1.0     0.998770     1.000394     0.226588    Uiso  ? O
   O47        1.0     0.249200     0.250740     0.726130    Uiso  ? O
   O48        1.0     0.249005     0.749251     0.730328    Uiso  ? O
\end{verbatim}


\bibliography{biblio}


\end{document}