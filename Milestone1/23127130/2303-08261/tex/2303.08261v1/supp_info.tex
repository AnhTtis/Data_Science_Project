

%%%%%%%%%% Merge with supplemental materials %%%%%%%%%%
%%%%%%%%%% Prefix a "S" to all equations, figures, tables and reset the counter %%%%%%%%%%


\renewcommand{\thesection}{Supplementary Information \arabic{section}}    %%%% but here
\renewcommand{\theequation}{S\arabic{equation}}
\renewcommand{\thefigure}{S\arabic{figure}}
\renewcommand{\thetable}{S\arabic{table}}
\renewcommand{\bibnumfmt}[1]{[S#1]}
%\renewcommand{~\citeumfont}[1]{S#1}
%%%%%%%%%% Prefix a "S" to all equations, figures, tables and reset the counter %%%%%%%%%%

\setcounter{equation}{0}
\setcounter{figure}{0}
\setcounter{table}{0}
\setcounter{page}{1}
% \makeatletter
\setcounter{section}{0}

\widetext
\begin{center}
\textbf{\large Supplemental Materials: Ca-dimers and solvent layering determine electrochemically active species in Ca(BH$_4$)$_2$ in THF}
\end{center}



\begin{figure}[h!]
\centering
\includegraphics[width=0.9\linewidth]{images/SI/si-bulk-paths.png}
\caption{Bulk free-energy surface of dimerization from ionic and neutral species (a). Bulk disproportionation (b) versus direct dissociation (c) pathways. }
\label{fig:si_bulk-paths}
\end{figure}




\begin{figure}[h!]
\centering
\includegraphics[width=0.9\linewidth]{images/SI/si-dimer-minima.png}
\caption{ \textbf{Determination of short and long dimers equilibrium RT distances. }Free-energy of dimerization (distance Ca-Ca) as a function of the Ca-B(H$_4$) coordination number (left) and Ca-O(THF) coordination number (r$_0$ = 3.6~\AA, right).  One-dimensional profiles obtained through thermodynamic integration over each coordination number allow us to interpolate the equilibrium distances of the short and long dimer.   }
\label{fig:si-dimer-minima}
\end{figure}




 \begin{figure}[ht]
\centering
\includegraphics[width=0.7\linewidth]{images/SI/perpendicular_dimer_coordination.pdf}
\caption{ \textbf{Solvent coordination of perpendicular dimers.} Orientation of the Ca-Ca axis with respect to the surface normal of structures of the isomer SD 5 THF - 4 at the DL (see Fig. 3 in the main text) depending on the THF-Ca coordination of the Ca that resides in the DL. At strongly perpendicular orientations (larger that 150 $^{\circ}$ with respect to the surface normal), two-fold solvent coordination is favored. }
\label{fig:si_perpendicular_dimer_coordination}
\end{figure}



\begin{figure}[h!]
\centering
\includegraphics[width=0.9\linewidth]{images/SI/bh4_layering_thf.pdf}
\caption{\textbf{ Free-energy of BH$_4^-$ at the (biased) interface. }The free-energy profile of a single THF molecule in THF (top) at RT with respect to distance from a graphite interface (red) and the corresponding oxygen density profile (black), with slight adjustment (dashed lines) at negative bias. The Dense Layer (DL), Intermediate Density Layer (IDL) and bulk regions are color-coded hereafter in red, blue and gray. The bottom figure shows the analogous free-energy profile of a single BH$_4^-$ molecule in THF at RT with respect to distance from a graphite interface (red) and the corresponding boron density profile (black). }
\label{fig:si-bh4-layering}
\end{figure}

\newpage

\section{Dipole moments of Ca(BH$_4$)$_2$ isomers} 
\label{sec:si-dipole}

In order to compute the dipole moment of specific Ca(BH$_4$)$_2$ isomer, an average structure of the whole group was calculated (including THF solvating molecules). Then, using Density Functional Theory as implemented in Terachem,\cite{Ufimtsev2008} the atomic positions were optimized to obtain the dipole moment. Typically, a decrease in the dipole moment of about 1 D is observed upon minimization. The B3LYP\cite{b3lyp} exchange-correlation functional with a dispersion correction (D3)\cite{grimme2010} and a 6-311G basis set were used. 


\section{Fermi energy changes for charged surfaces} 
\label{sec:si-esm_graphene}

Estimations of Fermi energy changes with electronic charge were carried out on a model graphene supercell of dimensions 0.426 x 0.246~nm to mimic the surface. Electronic structure calculations using Density functional theory (DFT) as implemented in SIESTA\cite{Soler2002}, with the PBE functional\cite{Perdew1996} and a double-zeta with polarization (DZP) basis set. In addition, orbitals-confinement was reduced by using a PAO Energy shift of 0.001 Ry in order to capture the electron density decay at the surface. A 10x10x1 k-point grid together with a 400 Ry charge density grid cutoff were used to ensure numerical convergence. Additionally, the Effective Screening Medium (ESM)\cite{Hamada2013,Otani2006} model was used to ensure a common reference vacuum level. In the ESM, the slab is placed between two semi-inifinite mediums that can be vacuum or metal-like (to mimic an electrode or electrolyte). We have compared two screening setups: one in which the slab is sandwiched between metal mediums (known as bc2) or between a metal and a vacuum media (bc3). Both cases yield the same result: a change in workfunction of 0.56 and 0.74~eV for a surface charge of 0.065 and 0.13~e$^-$~nm$^{-2}$, respectively.

\begin{figure}
\begin{subfigure}{.5\textwidth}
  \centering
  \includegraphics[width=.9\linewidth]{images/SI/esm_dos.pdf}
  \caption{}
\end{subfigure}%
\begin{subfigure}{.5\textwidth}
  \centering
  \includegraphics[width=.9\linewidth]{images/SI/esm_pot.pdf}
  \caption{}
\end{subfigure}
\caption{ \textbf{Fermi energies of charged graphene layers.} Aligned total density of states (DOS) of graphene with neutral and negative charge, with each Fermi level energy marked by a vertical line (a). The corresponding ESM potential as a function of the distance from the graphene (at z=0~\AA) in the Metal $\vert$ slab $\vert$ Metal ESM mode (b). }
\label{fig:si-esm_graphene}
\end{figure}




\begin{table}[h!]
\begin{center}
\begin{tabular}{|  c | c | c | c | c    | c | c | c | c | c   | c | c | c | c |} 
\hline

Fig. & THF & Ca$^{2+}$ & BH$_4^-$ & G  & Box pars.~(\AA)  & CVs  & q$_G$ (e$^-$) & W & H & Freq & Reps. & RepFreq &  Time (ns) \\

 \hline
 Bulk  &200&2& 4& -  & 30.3 x 30.3 x30.3  &  dCa [0:16]~\AA    & - & 0.05 & 0.02 & 200  & 20 & 50000 & 1660\\

\hline
 1 &   & &  &    &                    & CN(Ca-B) [0-4]        &   &  0.05 &     &      &    &       &          \\

\hline
Neutral  &336&2& 4& 2  & 33.9 x 31.8 x 49.8  &  dCa [0:16]~\AA   & G1 = G2 = 0 & 0.2 & 0.02 & 200  & 70 & 50000 & 2082\\
2,3,  &   & &  &    &                    & CN(Ca-B) [0-4]            &   &  0.2 &     &      &    &       &          \\
5  &   & &  &    &                    & dZ [0:25]                 &   &  0.5 &     &      &    &       &          \\

\hline
2  &336  &-& -& 2  & 33.9 x 31.8 x 49.8 &  dZ [0:25]~\AA   & G1 = G2 = 0 & 0.05 & 0.02 & 200  & 15 & 50000 & 1049 \\ 
\hline

2  &336  &-& -& 2 & 33.9 x 31.8 x 49.8& dZ [0:25] & G1 = -1.4;   G2 = 0 &  0.05 & 0.02 & 200  & 10 & 50000 & 620  \\ 
\hline


\hline
CS1 &514&2& 4& 2  & 33.9 x 31.8 x 71.8  &  CN(Ca-Ca) [0:1]~\AA    & G1 = -0.7;  & 0.2 & 0.04 & 200  & 70 & 50000 & 1438 \\

  &   & &  &    &                    & CN(Ca-B) [0-4]            &  G2 = +0.7  &  0.2 &     &      &    &       &          \\
  &   & &  &    &                    & dZ [2:70]                 &   &  0.25 &     &      &    &       &          \\
\hline
CS2 &514&2& 4& 2  & 33.9 x 31.8 x 71.8  &  CN(Ca-Ca) [0:1]~\AA    & G1 = -1.4;  & 0.2 & 0.04 & 200  & 70 & 50000 & 1464 \\
 2,4, &   & &  &    &                    & CN(Ca-B) [0-4]            & G2 = +1.4   &  0.2 &     &      &    &       &          \\
5  &   & &  &    &                    & dZ [2:70]                 &   &  0.25 &     &      &    &        & 
 \\ 
\hline

\end{tabular}
\caption{\textbf{ Metadynamics simulation details.} Figure number or label, box composition (Ca$^{2+}$, BH$_4^-$ and graphene) and equilibrated lattice parameters, collective variables and their ranges, surface charges, and metadynamics parameters (hill width and height, frequency, number of replicals, replica update frequency, and total simulation time).  }
\label{tab:si_systems}
\end{center}

\end{table}




%\subsection{Negatively charged surface}

%Simulations with a negatively charged interface were set up to keep the interface neutral. The assumptions behind this choice are as follows. We are modelling a surface with an overall negative charge surface density, near a bath of electrolyte solution. Charged ions from the bulk can arrive to the interface in a short enough time to equilibrate its charge, making it neutral overall. Charge transfer events are rare[], and we assume that in our simulations time- and space-scale, the interface returns to equilibrium in between those events (e.g., in a real interface, any charge imbalance that places the interface out of equilibrium can be rapidly compensated by an ion brought in from the reservoir in the bulk). While it is not computationally feasible to carry out metadynamics simulations with a big enough cell to allow for enough size/species to create an explicit ion reservoir, we can model this situation by making the whole interface neutral. It's an "after the fact" model. 




 \begin{figure}[ht]
\centering
\includegraphics[width=0.9\linewidth]{images/SI/1e_interface_pops.pdf}
\caption{ \textbf{Species free-energy and populations in charged state 1.} Populations
of electrolyte species derived from their respective free energies $\Delta$G (kT), which were obtained by integration of the free-energy surface sampled using metadynamics with respect to Ca-surface distance, Ca-Ca distance and Ca-B(H$_4$) coordination number with a surface charge of 0.065 e nm$^{-2}$. At this charge, the dominating species at the IDL are still dimers, but at the DL, the neutral monomer Ca(BH$_4$)$_2$ becomes more favorable. }
\label{fig:si_1e_interface_pops}
\end{figure}




 \begin{figure}[ht]
\centering
\includegraphics[width=0.9\linewidth]{images/SI/1_4_dimers_free_energy.pdf}
\caption{ \textbf{Interfacial free-energy of [1,4] dimers.} Free energy pathways as a function of distance from the neutral surface for dimers with [1,4] Ca-B(H$_4$)$^-$  coordination numbers at different Ca-Ca distances. The most favorable, with d(Ca-Ca) = 4.9-5.1~\AA{}, preferably sits far away from the layered interface, at $\sim$ 17.5~\AA{}.}
\label{fig:si_1_4_dimer_freeenergies}
\end{figure}

