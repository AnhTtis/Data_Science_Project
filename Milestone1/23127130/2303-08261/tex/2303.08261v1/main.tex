\documentclass[%
 reprint,
 superscriptaddress,
 amsmath,amssymb,
 aps,
 % linenumbers
]{revtex4-2}

\usepackage{natbib}
\setcitestyle{super,sort&compress,comma}

% \usepackage[switch, modulo]{lineno}


% \usepackage{refcheck}
\usepackage{threeparttable}
\usepackage{float}
\restylefloat{table}
\usepackage{subcaption}

\usepackage{dcolumn}% Align table columns on decimal point
\usepackage{bm}% bold math
\usepackage{braket} %braket notation

\usepackage[utf8]{inputenc}
\usepackage{graphicx}
\graphicspath{ {images/} }

% add support for chemical formula
\usepackage[version=4]{mhchem}

% \usepackage{xr}

\begin{document}

\title{Ca-dimers and solvent layering determine electrochemically active species in Ca(BH$_4$)$_2$ in THF}

\author{Ana Sanz-Matias}
\affiliation{%
 Joint Center for Energy Storage Research, Lawrence Berkeley National Laboratory, Berkeley, CA 94720, USA
}%
\affiliation{The Molecular Foundry, Lawrence Berkeley National Laboratory, Berkeley, CA 94720, USA}
\author{Fabrice Roncoroni}
\affiliation{%
 Joint Center for Energy Storage Research, Lawrence Berkeley National Laboratory, Berkeley, CA 94720, USA
}%
\affiliation{The Molecular Foundry, Lawrence Berkeley National Laboratory, Berkeley, CA 94720, USA}
\author{Siddharth Sundararaman}
\affiliation{%
 Joint Center for Energy Storage Research, Lawrence Berkeley National Laboratory, Berkeley, CA 94720, USA
}%
\affiliation{The Molecular Foundry, Lawrence Berkeley National Laboratory, Berkeley, CA 94720, USA}
\author{David Prendergast}%
\affiliation{%
 Joint Center for Energy Storage Research, Lawrence Berkeley National Laboratory, Berkeley, CA 94720, USA
}%
\affiliation{The Molecular Foundry, Lawrence Berkeley National Laboratory, Berkeley, CA 94720, USA}
 \email{dgprendergast@lbl.gov}
 
            

\date{\today}% It is always \today, today,
             %  but any date may be explicitly specified
            

\begin{abstract}
 Divalent ions, such as Mg, Ca, and Zn, are being considered as competitive, safe, and earth-abundant alternatives to Li-ion electrochemistry. However, the challenge remains to match electrode and electrolyte materials that stably cycle with these new formulations, based primarily on controlling interfacial phenomena. We explore the formation of electroactive species in the electrolyte Ca(BH$_4$)$_2$ in THF through molecular dynamics simulation. Free-energy analysis indicates that this electrolyte has a majority population of neutral Ca dimers and monomers, albeit with diverse molecular conformations as revealed by unsupervised learning techniques, but with an order of magnitude lower concentration of possibly electroactive charged species, such as the monocation, CaBH$_4^+$, which we show is produced via disproportionation of neutral Ca(BH$_4$)$_2$ complexes. Dense layering of THF molecules within 1~nm of the electrode surface  (modeled here using graphite) hinders the approach of reducible species to within 0.6~nm and instead enhances the local concentration of species in a narrow intermediate-density layer from 0.7-0.9~nm. A dramatic increase in the monocation population in this intermediate layer is induced at negative bias, supplied by local dimer disproportionation. We see no evidence to support any functional role of fully-solvated Ca$^{2+}$ in the electrochemical activity of this electrolyte. The consequences for performance and alternative formulations are discussed in light of this molecular-scale insight.
\end{abstract}


\flushbottom
\maketitle
\thispagestyle{empty}

\section*{Introduction}

To increase the rate of conversion to renewable energy sources, electrification of various energy-intensive aspects of society is underway. The concomitant demand for electrochemical energy storage solutions increasingly highlights the limits of Li-ion technologies with respect to performance, safety and sustainability. Multivalent ions such as Mg$^{2+}$, Ca$^{2+}$, Zn$^{2+}$, or even Al$^{3+}$, offer more earth-abundant alternatives, some with higher theoretical specific capacity and reduced safety concerns due to self-passivation of metal anodes.\cite{Dong2020, Mohtadi2021, Lv2022, GuWu2017, Arroyo-deDompablo2020} 

However, realizing performant electrochemical cells using these ions is hindered due to various issues driven by interfacial phenomena:  low power output and charging rates due to large overpotentials and associated electrolyte decomposition and interphase growth.\cite{McClary2022, Liang2020, Deng2022, LeeYu2023}

At issue is our lack of understanding of the specific complex nature of solvation in suitable electrolytes for multivalent ions and the identification of which of these species are active at the electrode-electrolyte interface and why.\cite{forer-saboya2022} We may be biased by familiarity with aqueous solutions and the ability of water to generate perfect electrolytes, with fully dissociated and solvated ions, for many salts. However, organic solvents (required for a sufficiently wide window of electrochemical stability in batteries), with typically lower dielectric constants and larger molecular sizes, have increased residence times for coordinating highly charged cations and have relatively little interaction with anions, unlike ambipolar water molecules which more easily solvate both charges.

Molecular dynamics provides a window to the inner workings of electrolytes, revealing details of coordination of cations by solvent molecules and anions. However, in the study of highly-charged species in poor dielectrics, we must take care to avoid sampling only a limited set of coordination states due to their long lifetimes and unavoidable limitations in computing time and complexity. Free-energy sampling allows us the opportunity to pose fundamental questions regarding the chemical composition of a poor electrolyte and the mechanisms by which its solvated species interconvert.\cite{Roy2016, Baer2016, Joswiak2018, Baskin2019, baskin2020, Silvestri2022} Furthermore, mining the large quantities of compositional and conformational data produced by these simulations presents its own challenge. Here we rely on recently developed unsupervised learning approaches\cite{Roncoroni2023} to provide a faster path to extracting molecular-scale insight and guidance for future experiments to validate our predictions.

We study Ca(BH$_4$)$_2$ in THF as a promising electrolyte candidate,\cite{Wang2018} spurred on by recent studies of the bulk electrolyte~\cite{hahn2020} and the anode interface.\cite{McClary2022, Bodin2023} We can also contrast its behavior with Mg(TFSI)$_2$ in THF, studied recently using free energy sampling.\cite{baskin2021} Ref.~\cite{hahn2020} proposed that neutral Ca dimers (Ca$_2$(BH$_4$)$_4$) facilitate the disproportionation of neutral monomers into (active) monocations CaBH$_4^+$ and anions Ca(BH$_4$)$_3^-$. In the same work, molecular cluster calculations (embedded in a polarizable continuum model) indicated that the dimer is the second most stable conformation after the neutral monomer,
%in cluster caclulations dimers are more stable than ionic species but still less than the neutral species; experiments detect dimers
but lacked specific solvent interactions beyond the first coordination shell and any Debye screening from finite ion concentrations. Interfacial characterization reveals the ready formation of solid-electrolyte interphases incorporating oxidized boron and even embedded calcium hydride.\cite{McClary2022} And debate continues as to the presence or electrochemical relevance of the fully-solvated Ca$^{2+}$ dication.~\cite{Wang2018, ta2019,melemed2020, jieTan202, Liepinya2021, rev-Lu2021}

\begin{figure*}[ht!]
\centering
\includegraphics[width=\linewidth]{images/bulk.png}
\caption{\textbf{Analysis of the bulk electrolyte.} (\textbf{a}) One time step sampled from our molecular dynamics model of the bulk electrolyte showing two Ca$^{2+}$ dications and 4 BH$_4^-$ anions in THF at room temperature (RT). (\textbf{b}) 
Populations of electrolyte species derived from their respective free energies $\Delta G$ (kT), which were obtained by integration of (\textbf{c}) the free-energy surface sampled using metadynamics with respect to Ca-Ca distance and Ca-B(H$_4$) coordination number. Minimum energy pathways for dimer disproportionation from the neutral species, Ca(BH$_4$)$_2$ are indicated by dashed lines. (\textbf{d}) 
Population analysis with respect to anion and THF coordination and conformation about Ca ions obtained via unsupervised learning, indicating the diversity of species umbrella-sampled from points on the free-energy surface corresponding to the neutral monomer (CN(Ca-B) = 1.9 and d(Ca-Ca) = 11~\AA) and the long (L) and short (S) dimers (d(Ca-Ca) $<$ 7~\AA). 
Atomic structures of dominant structures (populations larger than 7.5\%) are shown.}
\label{fig:bulk}
\end{figure*}

In this work, we reveal intricate details of the population of species in the bulk of this electrolyte and striking differences within a nanometer of its electrode interfaces, which are further exaggerated by potential differences. We discuss the consequences of these predictions for functioning cells.


\section*{Results}

\subsection*{In the bulk electrolyte}

We employ free-energy analysis for a model of the bulk electrolyte at room temperature (RT), comprising two Ca$^{2+}$ ions (and four borohydride anions) dissolved in THF (Fig.~\ref{fig:bulk} a), using empirical force fields (see Methods and Supporting Information). The free energy surface (Fig.~\ref{fig:bulk} c) is sampled using metadynamics\cite{Laio2002} with respect to two collective variables: the Ca-B coordination number, which controls the charge of complexes, and the Ca-Ca distance, which for a system with only two Ca ions, distinguishes between monomers and dimers. Integration over the free energy surface indicates that the most favorable species in solution are Ca-Ca dimers (Fig.~\ref{fig:bulk} b). These are neutral complexes with the formula Ca$_2$(BH$_4$)$_4$ and come in two varieties: long dimers (36\%) bridged by a single BH$_4^-$ and short dimers (21\%) bridged by two anions (see below for more detail). The next most common species is the neutral monomer complex, Ca(BH$_4$)$_2$ (40\%). It is notable that only a small percentage (3\%) of the solvated population is predicted to be charged, the monocation, CaBH$_4^+$, with its complementary anion, Ca(BH$_4$)$_3^-$. The free energy for the formation of fully solvated dications, Ca$^{2+}$, in this model, is too high ($\sim$19~kT) to support a significant population. These results provide some reordering with respect to static estimates which predict the neutral monomer as most favorable, followed by a single (short) dimer conformation, the monocation and the anion.~\cite{hahn2020} From the relative populations of each species, it is clear that neglecting the long dimer would lead to this conclusion. Furthermore, it is clear from RT sampling that there are multiple possible conformations for the dimer at finite temperature (as we explore below). However, both sets of calculations agree that the fully solvated dication Ca$^{2+}$ is the least favored in this set of possibilities (1-1.2~eV higher in energy by static estimates~\cite{hahn2020}, 0.48~eV from free-energy sampling at RT).  

Our 2D free energy surface (Fig.~\ref{fig:bulk} c) reveals that direct interconversion of these solvated objects, by removal or addition of borohydride anions, is prevented by quite steep free energy barriers (24.4~kT or $\sim$0.6~eV). The easier path to disproportionation (forming charged species from neutral monomers) is through the formation of dimers, with exchange of borohydride anions before dissociation into the complex monocation and anion (with activation energies of 2.4 - 10~kT, Fig.~\ref{fig:bulk} c and Fig.~\ref{fig:si_bulk-paths}):

$$ \ce{2Ca(BH4)2 -> Ca2(BH4)4 -> CaBH4^{+} + Ca(BH4)3^{-} } $$


At a minimum, this explains the prevalence of dimers in spectroscopic analysis of the bulk electrolyte (using EXAFS and Raman)\cite{hahn2020} and the observation of saturating ionic conductivity with increasing concentration as more neutral dimers form (prior to precipitation).~\cite{hahn2020,Hahn2021}

Strikingly, we find that bulk populations of each complex are conformationally quite diverse. Our metadynamics simulations project the full free energy landscape onto only a few collective variables, however, multiple molecular conformations may satisfy these constraints (Ca-B coordination number and Ca-Ca distance, in this case).
Data-mining techniques (dimensionality reduction, hierarchical clustering and permutation-invariant alignment~\cite{Roncoroni2023} -- details in Methods and SI) applied to selective umbrella sampling of local minima in the free energy landscape reveal a rich variety of solvated isomeric structures, summarized in Fig.~\ref{fig:bulk} d. 

For example, the neutral monomer, Ca(BH$_4$)$_2$, when additionally coordinated by three THF molecules ($\sim$~60\% of its population) adopts mostly bent borohydride arrangements with a large dipole moment (8 D, see SI). In addition, we found a significant portion ($\sim$~30\%) of monomers coordinated by 4 THF molecules with an axial borohydride arrangement and low dipole moment (2 D). These two structures had been proposed separately as the minimum energy structure from quantum-chemical cluster calculations~\cite{hahn2020} and molecular dynamics simulations,~\cite{Hahn2021} respectively. 

By contrast, the monocation occurs mostly with 5 coordinating THF molecules, in agreement with previous reports.\cite{hahn2020}

We find that dimers are always neutral (with four borohydride anions) and are split in two main spatial configurations characterized as short (SD) and long (LD) dimers with average Ca-Ca equilibrium distances of 4.48 and 5.25~\AA{}, respectively (Fig.~\ref{fig:si-dimer-minima}). Furthermore, each was found to have sub-populations with 4-7 solvent molecules, and, within those, several stereoisomers (Fig.~\ref{fig:bulk} d). Key differences between the long and short dimers are the presence of a single-anion or double-anion bridge and predominant 6 THF coordination or mixed 5-6 THF coordination, respectively. Short dimers are in excellent agreement with a previously proposed double-bridged dimer structure with a 4.4~\AA{}Ca-Ca distance (from EXAFS fitting data\cite{hahn2020}).
The set of structures shown here expands upon and underscores the configurational flexibility of calcium.~\cite{hahn2020}
And, based on our understanding of the efficient disproportionation pathways to form charged species  via dimerization (discussed above), it makes sense that the dominant dimer conformations form from combinations of bent and axial borohydride arrangements of the neutral monomers (e.g., long dimer isomers 1, 3 and 7 with 6 THF molecules in Fig.~\ref{fig:bulk} d are bent-bent, axial-bent and axial-bent combinations, respectively).

Although all dimers here are neutral, each Ca ion within a given dimer may be locally coordinated by 1 to 4 anions, with 1 (long) or 2 (short) shared between them. Most commonly we observe  [3,2] or [3,3] anion arrangements for long or short dimers, respectively. Small populations of [1,4] dimers are found and are key to some interfacial disproportionation processes discussed below and in Fig.~\ref{fig:disproportionation}.

\begin{figure*}[ht!]
\centering
\includegraphics[width=\linewidth]{images/interface_layering.png}
\caption{\textbf{Effect of solvent layering at the interface}. (\textbf{a}) The free-energy profile (red) of a single THF molecule in THF at RT with respect to distance from a graphite interface and the corresponding oxygen density profile (black), with slight adjustment (dashed lines) at negative bias. 
The Dense Layer (DL), Intermediate Density Layer (IDL) and bulk regions are color-coded hereafter in red, blue and gray. The inset shows a snapshot of the interfacial region sampled from molecular dynamics indicating THF molecules lying flat against the graphite surface and somewhat constrained at the DL-IDL interface. (\textbf{b}) The 3D free energy landscape derived from metadynamics with respect to Ca-Ca distance, C-B coordination number, and distance from the graphite surface for the neutral interface. Minimum free-energy paths for (\textbf{c}) neutral and (\textbf{d}) charged species in the electrolyte arriving at the interface from the bulk under zero (solid) and negative (dashed) bias conditions indicated by the graphite surface charge density, $\sigma$. Sudden jumps in the free-energy profiles reflect variations with respect to other collective variables in the free-energy landscape. Under negative bias the long and short dimers are not distinguished in our sampling. In general, negative bias stabilizes charged species at the interface.} 
\label{fig:interface_layering}
\end{figure*}

 \begin{figure*}[ht!]
\centering
\includegraphics[width=0.9\linewidth]{images/neutral_interface.png}
\caption{\textbf{Distribution of species at the neutral interface.} Molecular model of the simulation box, obtained from a snapshot of the equilibrated MD trajectory (\textbf{a}), showing a dimer at the IDL (in blue), near the DL (in red). Integrated free-energy and corresponding population per layer (\textbf{b}), color-coded as red, blue and gray for the DL, IDL and bulk, respectively. Unsupervised clustering analysis of umbrella sampling trajectories at the DL and IDL (z = 5.75 and  8.25~\AA{}) show the relative populations of dimer isomers per layer (\textbf{c,d}). Representative structures have discrete orientations, which depend on the layer (insets in \textbf{e,f}). 
Color lines show the orientation of the Ca-Ca axis with respect to the surface normal, where 90\textdegree{} is parallel to the surface, as shown in IDL dominating dimer LD 6 THF - 7  (also in a), while DL dimer SD 5 THF - 4 is mostly  perpendicular. Dipoles have also discrete orientations with respect to the surface normal for given isomers (shown in black, dotted lines) and are roughly aligned to the Ca-Ca axis in long, flat dimers (IDL) or perpendicular, as in DL dimer SD 6 THF - 2. } 
\label{fig:neutral_interface}
\end{figure*}

\subsection*{At the electrode-electrolyte interface}

What happens to species in the electrolyte as they approach an interface, such as the electrode surface? 
Firstly, as expected from simple statistical mechanics modeling of molecules at hard interfaces,~\cite{Henderson1976} the solvent, THF, adopts a layered molecular structure~\cite{Qiao2010,baskin2021} near the surface with a dense layer (DL) at 3-6~\AA{}, followed by a low-density `gap', and an intermediate density layer (IDL) at $\sim$~7-10~\AA{} \ from the surface,  with 2.5, 0.3 and 1.5 times the bulk THF density, respectively (Fig.~\ref{fig:interface_layering} a). A third collective coordinate, calcium distance from the interface, sampled with metadynamics allows us to obtain the interfacial, 3D free-energy landscape (Fig.~\ref{fig:interface_layering} b), of which the most distant slice is the 2D bulk free-energy landscape in Fig.~\ref{fig:bulk} c. Dissolved species in the electrolyte near the electrode surface respect this underlying solvent structure. Now, free-energy minima are distributed between the bulk, IDL, and DL, and separated by barriers, as indicated by minimum energy pathways of neutral and charged species approaching the interface in the 3D landscape (Fig.~\ref{fig:interface_layering} c and d).

A key observation is that the IDL defines an attractive basin for most species, especially dimers and monocations, with minima in free energy that are lower than in the bulk, implying that this is a narrow interfacial region for enhanced concentration of solutes. Conversely, the DL defines a region from which solutes may be excluded due to additional free energy costs, without assistance of some applied bias (see below). This has some generally important consequences for electrochemistry in this electrolyte at a non-interacting electrode (such as graphite). Negligible populations of species that might be specifically adsorbed or adjacent to the electrode would imply the absence of inner-sphere electron transfer events during reduction. Similarly, if the electroactive species are likely to be highly coordinated in the IDL, then the necessary outer-sphere electron transfer events may very well lead to reductive decomposition of the coordinating species (solvent or anions) leading to low Coulombic efficiency and the formation of a solid-electrolyte interphase (SEI). Based on this study, our claim is that this inefficiency is as much a consequence of the strong solvent layering at the interface as it is of the salt that defines the electrolyte.

Specifically, we find that, next to an unbiased non-interacting electrode (Fig.~\ref{fig:neutral_interface} a), the IDL is dominated ($\sim$~75\%) by dimers (long dimers in particular, 55\%), with reduced populations of neutral monomers ($\sim$~10\%) and a very slight increase to 8\% in the monocation population (Fig.~\ref{fig:neutral_interface} b). Only 6\% of the interfacial population is in the DL, according to thermodynamic integration of the free energy. In the DL, the monocation population decreases and dimers dominate ($\sim$~85\%) -- especially short dimers, in contrast to bulk and IDL populations.


Unsupervised clustering analysis of structures from umbrella-sampling trajectories at the IDL and DL (z = 8.25 and 5.75~\AA{}, respectively) reveals a reduced number of favored dimer isomers compared to the bulk, with one given isomer making $>$20~\% of the population in each case  (Fig.~\ref{fig:neutral_interface} c-d). In the IDL, this is a bent-axial long dimer with 6 solvating THF molecules (indexed as Isomer 7 or LD 6 THF - 7), already a favored species in the bulk. In the DL, on the other hand, a short dimer also with 6 THF molecules (SD 6 THF - 2) dominates. Furthermore, we find that the orientation of dimers at the interface is discretized (Fig.~\ref{fig:neutral_interface} e,f). Favored IDL dimers have mostly flat orientations of the Ca-Ca vector relative to the graphite surface, with two dense layer THF molecules involved in solvation. Similarly, the dominant dimer orientation in the DL is flat (i.e., in a plane parallel to the surface), with both calcium ions embedded in that dense region, and with two THF molecules from the IDL contributing to solvation. Additionally, perpendicular dimers form at least 11\% of the DL population. Isomer 4 of the short dimer with 5 THF molecules (SD THF 5 - 4) is an asymmetric dimer, with calcium ions solvated by 2 and 3 THF molecules that sit in the DL and on the edge of the IDL, respectively (Fig.~\ref{fig:si_perpendicular_dimer_coordination}). 
Note that the specific conformation of these coordination complexes defines their effective dipole moment, which may, in some cases, be orthogonal to the Ca-Ca vector of the dimer. An example in the DL is the favored isomer SD 6 THF - 2, with its Ca-Ca vector parallel to the surface but three BH$_4^-$ sitting closer to the interface, near the corresponding DL free energy minimum for isolated borohydride in THF next to graphite (z= $\sim$4~\AA{}, see Fig.~\ref{fig:si-bh4-layering}). Dipole orientation will be discussed in more detail below in the context of biased interfaces.

As observed in the bulk electrolyte, the absence of fully solvated dications is more pronounced at the interface. This speaks to the rigidity of the THF coordination sphere around this multivalent ion (similar to that around Mg as seen in Ref.~\cite{baskin2021}) whose size (an effective radius $\sim$ 5 ~\AA{}) prevents the close approach of this species to the electrode. The preferred, relatively flat orientation of THF molecules in the dense layer is incompatible with the preferred radial coordination of the cation and likely forces a reduction in coordination number that raises the free energy of fully-solvated Ca$^{2+}$ (reducing its relative population) relative to complexes coordinated by the more compact borohydride anions. 





\subsection*{Biased Interfaces}

So far, it seems that Ca(BH$_4$)$_2$ in THF is a poor electrolyte, with only a small fraction of the salt concentration (3\%) present as charged species in the bulk, albeit with a noticeable enhancement to 10\% in the intermediate density layer (IDL). Otherwise this electrolyte is dominated by neutral species (monomers and dimers). This is consistent with previous discussion of  undissociated neutral species as dominant in Ca-based electrolytes with boron-containing anions\cite{forer-saboya2022}. However, the bulk free energy landscape (Fig.~\ref{fig:bulk} c) indicates that interconversion of species is possible (more details below) and some equilibrium exists between charged and neutral species. We explore the impact of biased/charged electrodes on the population of solutes by evenly distributing opposing charges on either face of the two-layer graphite electrode model, which, under periodic boundary conditions, polarizes the electrolyte. We considered two specific charge states with (1) 0.065 and (2) 0.13~e/nm$^2$. We estimated that these surface charge densities raise the Fermi level of a graphene model by 0.56 and 0.74 eV, respectively (see Sec.~\ref{sec:si-esm_graphene}). Screening at the interface by the solvent and electric double layer will result in a much smaller potential difference across the interface. A lower bound for this bias is 0.038 and 0.065 V, respectively, provided by the Grahame equation, assuming a simulated bulk concentration of 0.12~M.\cite{HJButt2006} (Note that at higher bulk concentrations, with a shorter Debye screening length, this estimated potential difference should be even smaller -- at 1.65~M~\cite{McClary2022} the calculation would indicate 0.01~V and 0.02~V, respectively). The larger of these electrode surface charge densities is not enough to draw monocations into the dense layer, which are held outside this region by a barrier of approximately 0.25 eV. 


 \begin{figure*}[ht!]
\centering
\includegraphics[width=0.9\linewidth]{images/negative_interface.png}
\caption{ \textbf{Populations at a biased interface.} Snapshot obtained from the equilibrated trajectory of charged state 2 ($\sigma$ = $\pm$ 0.13 e nm$^{-2}$) with highlighted layering (red/blue for DL/IDL) at the negative interface showing the IDL solvated monocation (\textbf{a}). Free-energy and population distribution ((\textbf{b}) at the negatively charged interface is dominated by the monocation, most favored at the IDL.   Unsupervised clustering analysis of Umbrella Sampling trajectories of the monocation at the IDL, close to the edge of the DL (z = 5.8~\AA{}) and close to the center of the DL (z = 4.8~\AA{}) (\textbf{c}) classifies the structures into two isomers of the five-fold and four-fold THF coordinated monocation, which differ by slight changes in the local geometry of the coordinating THF molecules. The corresponding Ca-B dipole orientation with respect to the surface normal  ($\vec{\mu}$), with 180$^{\circ}$ being the B pointing away from the surface, indicate discrete orientation at the interface (\textbf{d}). Representative structures of the main isomers at each layer in their most likely orientation, with the surface on the  left side (\textbf{e}). }
\label{fig:negative_interface}
\end{figure*}

From Fig.~\ref{fig:interface_layering} (a) we see that the solvent layering remains practically unchanged upon charging the electrode. Dimers are still favored in the intermediate density layer (IDL) and further from the interface in charge state 1 (Fig.~\ref{fig:si_1e_interface_pops}). However, in charge state 2 some charged species become strongly stabilized at the interface. 
Specifically, at this larger negative bias the most favored species within the intermediate (IDL) and dense(DL) layers is now the monocation, CaBH$_4^+$, which almost entirely displaces the neutral dimers and monomers (Fig.~\ref{fig:negative_interface} a, b).  Due to the higher free energy of species in the DL, the interfacial population is largely limited to the IDL (99.4\%), according to thermodynamic integration.
At this potential, the approach of the monocation to the electrode, through the DL, would have to be an endergonic process, requiring 5.7~kT of free energy, overcoming a barrier of 7.5~kT. 
Although, this is some improvement over the case of the neutral electrode, which presents a barrier of 9.9~kT to enter the DL, with a required input of 9.1~kT of free energy.

We can understand these free energy costs by following the evolution in solvation and associated dipole orientation of the monocation. Unsupervised learning analysis of various umbrella sampling trajectories: in the IDL, at the edge of the DL and in the DL (z = 8.25, 5.8 and 4.8~\AA{}, Fig.~\ref{fig:negative_interface} c-e), reveals that five-fold THF coordination dominates the IDL population, with the BH$_4^-$ anion pointing away from the surface -- as one might expect given the direction of the electric field at the negative electrode. 
However, this dipole reorients at the edge of the DL, likely due to mixed DL/IDL solvent coordination, and a small four-fold coordinated population appears. Ultimately, in the center of the DL, four-fold coordination dominates, with the dipole of the predominant isomer pointing away from the surface again. 

This complicated and costly path for the monocation to reach the electrode further emphasizes the important role of the THF solvent layering and specific coordination in determining the electrochemical activity. By the same token, the fully-solvated dication, Ca$^{2+}$, with its somewhat rigid first solvation shell, is still too unfavorable to define a noticeable population at the negative electrode, despite its higher electrostatic charge. In fact, we find that the dication is only slightly more stable than at the neutral interface (Fig.~\ref{fig:interface_layering} d).





 \begin{figure*}[ht]
\centering
\includegraphics[width=1.\linewidth]{images/disproportionation.png}
\caption{\textbf{Disproportionation pathways} Scheme depicting the main two steps of dimer disproportionation (\textbf{a}): reorganization (\textbf{1-5}), followed by separation into ions (\textbf{6}) through distinctive paths A and B. 
Minimum energy pathways of disproportionation at the neutral  IDL and bulk (\textbf{b}) and at the charge state 2 IDL and bulk (\textbf{c}), obtained from the 3D free-energy landscapes.  Since each point can be labelled according to the sampled collective variables, we have color-coded the distance to the surface in the IDL and bulk tones (blue, gray).} 
\label{fig:disproportionation}
\end{figure*}


\subsection*{Generation of active species}

Thus far, it seems that the monocation, CaBH$_4^+$, is the strongest candidate for the electroactive species in this electrolyte, given its high population in the vicinity of the electrode upon charging to a sufficiently high potential difference. We do not explore reduction potentials in this work. However, as we have seen, the appearance of the monocation with close proximity to the interface, in the IDL, indicates that it is a likely candidate for an outer-sphere electron transfer process. With increasing potential difference, to overcome the free energy costs associated with penetration of the dense THF layer (DL) and reduced solvation, we would predict an inner-sphere reduction process. However, the fact remains that this poor electrolyte (only 3\% of species in solution are charged, as mentioned above) must supply the IDL with monocations in the first place, via some disproportionation mechanism from neutral species (most likely dimers), and replenish the same species while electroreduction and electrode deposition consumes them. Any barriers in this supply chain should be evident in the kinetics of the electrochemistry as an observed deposition overpotential or associated rate limitations in charging cells with this electrolyte.

To shed light on the underlying processes, we approach the generation of charged species (monocations) as a two-step process, involving dimer reorganization and disproportionation. As we show below, the most stable or prevalent dimer species are not readily disposed to disproportionate. Some molecular rearrangement of borohydride anions and solvent molecules is first required. The subsequent disproportionation follows two major pathways, which can occur in the bulk electrolyte or at the interface, and which we have investigated both at neutral and negative biases.

In the neutral cell, increased dimer concentrations in the intermediate density layer (IDL) are readily explained through analysis of the free energy landscape (Fig.~\ref{fig:interface_layering}). 
The minimum energy path (MEP) for disproportionation in the presence of the interface indicates that neutral species in the bulk can easily flow, without a significant free energy barrier, into the IDL. Migration from the bulk to the IDL is essentially barrierless for all species except Ca$^{2+}$. As summarized in Fig.~\ref{fig:disproportionation} a, to prepare for disproportionation of the most favorable dimers, some reorganization into an intermediate (less stable) dimer is required and ultimately Ca-Ca separation into ionic species follows two main paths, A or B.

For Path A, the dominant long dimer configuration (\textbf{1} in Fig.~\ref{fig:disproportionation} a) can undergo a low-barrier reorganization, through a short dimer (\textbf{2}), to another long dimer with similar coordination of Ca ions with borohydrides (\textbf{3}). Then, the Ca-Ca distance increases (\textbf{4}) up to  6.7~\AA{} at the transition state so that the nascent monocation Ca ion can increase its number of coordinating THF molecules, from the original 3-4 to the preferred 5, upon full dissociation. Path B branches from the short (\textbf{2}) or intermediate long (\textbf{3}) conformations, through further borohydride and solvent reorganization, to a higher energy dimer (\textbf{5}), with a [1,4] anion coordination, which then disproportionates to produce the monocation (\textbf{6}).   

Figures~\ref{fig:disproportionation} b and c outline the MEP for disproportionation at the electrode interface or in the bulk electrolyte under neutral or negative bias. Overall, at the electrode interface, in the intermediate density layer (IDL), disproportionation follows Path A, whereas Path B is preferred in the bulk, likely due to configuration \textbf{5} being more favorable in the bulk than at the interface
(Fig.~\ref{fig:si_1_4_dimer_freeenergies}). 

We find that interfacial (IDL) disproportionation at zero bias via Path A is favored, since it has slightly lower barriers (E$_{a, 3\rightarrow4}$ = 8.4~kT and E$_{a, 4\rightarrow 6}$ = 9.8~kT) and a lower free energy cost than bulk disproportionation via Path B (Fig.~\ref{fig:disproportionation} b). This is in agreement with the slight increase in monocation population observed at the IDL in Fig.~\ref{fig:neutral_interface} (Slight variations in barriers between bulk and interface models can be due to differences in the collective variables and grid-spacing employed in our metadynamics simulations, SI Table 1). 

At the negatively charged electrode, we have seen (Fig.~\ref{fig:negative_interface}) that the monocation is the most favored species in the IDL, completely displacing the previously dominant dimers with increasing negative charge on the electrode (charge state 2). Due to the size limits of our simulations, we may well expect that the bulk thermodynamics are somewhat different from those under neutral conditions, however, disproportionation still follows Path B in our simulations (Fig.~\ref{fig:disproportionation} c), albeit with an additional step involving the formation of a long dimer conformation (\textbf{3}). Similarly, Path A is still preferred in the IDL. Although disproportionation occurs in the bulk, barrier-less pathways to the IDL suggest that dimers can approach the charged IDL and undergo disproportionation quite favorably. 
Therefore, upon negative charging, the concentration of monocations should increase in the IDL, based on a favorable free energy and necessary supply from the local (IDL) dimer population via interfacial disproportionation. 

This free-energy analysis shows little activity in the dense layer (DL) due to its low population of dissolved species. However, as noted above, the slight stabilization of the monocation in the DL at negative bias results in a slight increase within the overall interfacial (DL plus IDL) population of this species --
from $\sim$0.2\% in the neutral DL to $\sim$0.6\% in the DL at charge state 2. Structural analysis indicates that these monocations have their anionic end pointing away from the negative surface and have lower solvent coordination. We have stated before that electrochemical activity is a strong (exponentially decaying) function of distance of the reducible species from the electrode. It is also very likely enhanced by reduced cation coordination. The question remains whether these benefits outweigh the low population of such species in the net electron transfer rate at similar potential differences.



\section*{Discussion}

Based on our analysis of the Ca(BH$_4$)$_2$|THF electrolyte and its interfacial speciation, we can propose the following phenomenology that may explain existing observation and provide guidance and interpretation of future characterization efforts. 

First and foremost, the solvent, THF, and its strong solvation and interfacial layering dictate much of what we have observed. The strong solvation of the dication essentially prevents it from taking part in electrochemistry. And the monocation requires a threshold negative bias to begin to dominate as an interfacial species, albeit confined to the intermediate density layer (IDL), 7-10~\AA{} from the electrode surface. The dense solvent layer (DL) is only sparsely populated with solvated or partially solvated species.

That negative bias can begin to drive the interfacial population of charged species significantly above that present in the bulk indicates that this poor electrolyte is ``activated'' upon charging. Therefore, meaningful characterization of the electrochemical activity of this electrolyte requires operando measurements that are sensitive to within a nanometer of the interface. The rich isomer subpopulations with distinct orientations in the IDL make it an excellent playground for interfacially-sensitive, polarization-dependent spectroscopies that can capture these differences. Furthermore, the bias-dependence switch in local population from oriented dimers to monocations should be observable with chemically-sensitive vibrational\cite{Lu2019, Yang2022, He2022} and electronic probes.\cite{velasco2014, Wu2018, Prabhakaran2023}


The required potential differences to enrich the IDL with charged species and, presumably, the disruption and population of the DL with more charged species at even higher potentials, may be consistent with known overpotentials for Ca deposition. Plating and stripping of Ca using this electrolyte carries a first, short duration overpotential of $\sim$250~mV, followed by a $\sim$100~mV overpotential in subsequent cycles.\cite{Wang2018} The remaining free-energy cost (5.7~kT) for monocations to access the DL in our higher charge state simulations (estimated to be at 0.065~V),  implies that the final potential difference to enrich the DL with monocations for reduction is $\sim$~0.2~V. It remains to be determined how a more strongly interacting electrode surface (which may be present after the first cycle) alters the THF solvent layering and, thereby, the thermodynamics of electrolyte species within the first nanometer of the evolving electrode surface.
Strong solvent and anion coordination of electrochemically active species, which has dominated our analysis, is very likely the source of solid-electrolyte interphase (SEI) formation due to electroreduction and decomposition of these ligands at the interface.


Although all small molecule polar solvents should present dense layers at interfaces, the inaccessibility of the DL is also, in part, determined by the choice of anion, whose electrode surface affinity, solvophobicity, size and cation coordination strength may facilitate overcoming the molecular packing in the DL and reduce associated overpotentials, ideally without the production of a thick interphase due to its own electrochemical instability.
An example of a solvophobic anion that can bridge the THF dense layer is TFSI$^-$, which, according to previous free-energy studies, is more favorable in the DL and IDL regions than in the bulk solvent ($\sim$2~kT)\cite{baskin2021}. This contrasts with BH$_4^-$ (Fig.~\ref{fig:interface_layering} b) which has reduced populations at the interface, especially in the DL. We have seen that Mg, another divalent cation, which when fully-solvated dication is also hindered from close approach to the neutral electrode, but when combined with TFSI$^-$ can arrive stably at the interface, effectively in the dense layer (barrier $<$4~kT, $\Delta$G $<$~3~kT, z=6.3~\AA{}). A non-sparsely populated dense layer with significant monocation population seems likely in the MgTFSI$_2$|THF electrolyte. Yet, we know, TFSI is both inherently and electrochemically unstable, both for Mg\cite{Yu2017} and for Ca.\cite{hahn2022} 

The instability of TFSI may be driven by its tight coordination of cations through its sulfone groups. Increased electroreductive stability in large anions that simultaneously span the small solvent molecule dense layer may be afforded by considering non-cation-coordinating species. For example, closoboranes have been studied with Mg in tetraglyme.\cite{Jay2019} These bulkier anions may also lead to better electrolytes overall (for example, preventing the formation of dimers), readily producing  charged electroactive species. Although, strong solvent coordination, as we have seen for the fully-solvated Ca$^{2+}$, may still lead to significant overpotentials for electrodeposition and associated solvent decomposition and SEI growth.

From the computational perspective, we have highlighted the value of free-energy exploration and unsupervised learning to reveal the complexity of this nominally simple electrolyte and tried to connect our simulations to observed electrochemical behavior and characterization. However, as in all theoretical models, our study has some inherent limitations. The complexity of the system and the time-scales required to explore different coordination complexes required the use of empirical force fields rather than ab initio methods. The finite number of dissolved species in our simulations mimics only low concentrations, with limited Debye screening, and our force-fields employ static partial charges which must be scaled appropriately to attempt to reproduce dielectric screening due to solvent or anion polarizability.


The notable outcome of this study is a reinforcement of the notion that performant nonaqueous multivalent electrolytes (with high ionic conductivity and low overpotential) have competing requirements for multivalent ion coordination: to be strongly coordinating to keep the salt dissolved and the ionic conductivity high while not so strongly coordinating that the ion cannot break free from solvation during electrodeposition. The need for strong solvent coordination is driven by competition with favorable ionic bonds between counterions. If we want to maintain the advantages of earth-abundant, multivalent ions, then one option would be to switch to larger anions without specific coordinating moieties. However, this study highlights that there may be other options to consider that sideline the isolated multivalent ion altogether, which may even be irrelevant in terms of electrochemical activity. Firstly, that coordination with counterions may actually help bring active species closer to the electrode -- that the ability of the species to respect or disrupt the solvent inhomogeneous structure at the interface can dictate which species approaches the electrode closest. Secondly, that incomplete dissolution and the activity of oligomers (dimers in this case) could be key to improved electrochemical activity albeit balanced by some power limitations due to reduced bulk ionic conductivity and the need to regenerate electroactive species through a disproportionation equilibrium. Clearly we have more work to do, both in terms of understanding the reduction of these clusters and their dissociation in the reduced state leading to Ca deposition; the potential negative side-effects of reductive instability of the coordinating species, already connected to SEI formation;\cite{McClary2022} and the ultimate origins of measured overpotentials and currents in experiments. However, we have highlighted the importance of free energy sampling to attack such complex problems, even within relatively ideal conditions and contexts, and look forward to seeing more studies of this kind in the future.


\section*{Methods}

Metadynamics sampling (MTD) with a classical force-field was used to obtain free-energy landscapes as a function of $n$ collective variables. Equilibrium populations at critical points of a given landscape were then collected with Umbrella Sampling (US). Structural analysis of the US trajectories was then performed with a python-based unsupervised learning protocol.

\paragraph{Free-energy sampling.} Metadynamics free-energy sampling\cite{Laio2002} was carried out using the COLVARS module\cite{colvars} implemented in LAMMPS.\cite{lammps} Systems (see Table S1 for a full list) were generated using Packmol\cite{martinez2009}. Concentration values were chosen to avoid forcing aggregation. Dimerization free energy surfaces show that at Ca-Ca distances larger than $\sim$ 10 ~\AA{}, the free energy converges with respect to Ca-Ca separation. That is the cutoff we consider between "dissociated" and "aggregated", giving an effective radius of 5 ~\AA{} for the first coordination shell of the dication. Additionally, g(r) shows that the second solvation shell (O-THF) settles at around 6 ~\AA{}. Assuming an optimal close-packing of ions, we obtain that SSIPs would be unavoidable at concentrations below 0.5 M for a 6 ~\AA{} radius and 0.7 M for a 5 ~\AA{} radius. Hence, we selected concentrations $\leq$ 0.03M, well below these limits. System equilibration consisted of conjugate-gradient minimization to avoid steric clashes, followed by a short NVT warm-up to room temperature  (298 K) using a 1 fs timestep. Box-size equilibration was achieved by continuing the trajectory under NPT conditions at 1 atm with a 2 fs timestep. A final NVT step with the equilibrated lattice parameters (~ 20 ns) was sufficient to bring the systems to equilibrium. Force-field parameters\cite{samba2009, andrade2002} were validated in our previous work on the same system.\cite{Roncoroni2023} The graphene was frozen in place by setting the forces to zero in order to ensure neutral and charged simulations were comparable. These MD setup was kept for the MTD and US simulations. Metadynamics calculations parameters -- the width of the grid along a collective variable (W), height of the Gaussian 'hills' used to bias the potential (H) and the frequency at which they are added (Freq) -- can be found in Table~\ref{tab:si_systems} and were chosen to ensure convergence, namely, that the simulation reached diffusive regime in the given collective variable space.\cite{baskin2021} Faster completion times were achieved by taking advantage of multiple-walker metadynamics, which allows to parallelize sampling among several trajectories (replicas) that update their biased potential at a given frequency (RepFreq) with the total biased potential. Despite this, the grid used in the three-dimensional MTD calculations was necessarily coarser. The minima explored here tend to be separated by more than 0.7 A (e.g., between the long and short dimer, or between solvent layers, which is larger than the coarser grid resolution. In order to keep cell neutrality and consistence with the neutral simulation, charged states 1 and 2 were generated by adding equal and opposite charges ($\pm$ 2 $\mu$C/cm$^2$) distributed evenly among two graphene layers, emulating a positively charged and a negatively charged electrode. 

Free-energy sampling was performed with combinations of the following collective variables: the distance between the calcium and the center of mass of the top graphene layer (dZ); the coordination number between the calcium and the boron atom in a given BH$_4$ (CN(Ca-B), r$_0$ = 3.8~\AA{}); and, to track dimerization, the distance between two calcium atoms (dCa) or a Ca-Ca coordination number (CN(Ca-Ca), r$_0$ = 6.5~\AA{}).

Note that the state free-energies in Fig.~\ref{fig:bulk} b) were obtained by thermodynamic integration of our 2D free energy surface (Fig.~\ref{fig:bulk} c) over regions delimited by given collective variable values  (e.g., for the Ca(BH$_4$)$_4^{2-}$ state, a Ca-Ca distance larger that 7~\AA{} and a CN(Ca-B) larger than 3.5).  Collective variables apply to only one of the calcium atoms, and hence the free energy is averaged over all (remaining) possible coordinations/distances  for the other atom. Hence, the room-temperature Boltzmann probability speaks of the likelihood of finding a state formed by the constrained species (e.g., Ca(BH$_4$)$_4^{2-}$) in the environment of the remaining, unconstrained species. Since our system contains two calciums and four borohydrides, the un-constrained space is different depending on the value of the collective variables. In the Ca(BH$_4$)$_4^{2-}$ basin, the un-constrained Ca can only be fully solvated. On the other hand, in the Ca${2+}$ basin, the other calcium can exist in five different ion coordination states (Ca$^{2+}$, Ca(BH$_1$)$^{+}$, Ca(BH$_4$)$_2$, Ca(BH$_4$)$_3^{-}$, and Ca(BH$_4$)$_4^{2-}$). This is the reason why no  symmetry is expected on the free energy surface along the coordination axis, e.g., between CN=0 and CN=4; and CN=1 and CN=3.


\paragraph{Population sampling} The structures and equilibrium population of points of interest in the free energy surface were collected using Umbrella Sampling. Initial structures were obtained from metadynamics trajectory snapshots at the desired point in the CV space. The collective variable coordinates were restrained with a harmonic potential centered at the desired value, with force constant 1/w$^2$ where w is the width of the collective variable grid (see Table~\ref{tab:si_systems}). Then, trajectories of 150-200 ns were calculated under similar MD parameters as the final equilibration step and MTD run. 

\paragraph{Data analysis.} Umbrella sampling trajectories were analyzed with an unsupervised learning methodology recently developed by us.\cite{Roncoroni2023} 
In this protocol, between 5000 and 10000 local atomic arrangements were extracted from each US trajectory, which when there aligned while taking into account possible permutations between similar elements (e.g. THF - O).\cite{Gunde2022,Griffiths2017} 
Classification based on their structural similarity was carried out using dimensionality reduction\cite{McInnes2018,Becht2019}  and clustering\cite{McInnes2017} algorithms, in an ASE-compatible\cite{Larsen_2017} environment. 
The $n$-dimensional free energy landscapes obtained from the MTD sampling were explored with a Jupyter-adapted version of the MEPSAnd module\cite{Marcos-Alcalde2019} in order to find critical points and minimum energy pathways. 

 \section*{Data availability}
Additional computational details and free-energy information referenced in the text can be found in the Supporting Information. The datasets generated and analysed during the current study are available from the corresponding author on reasonable request.

\section*{Author Contributions}


 ASM performed the calculations, analyzed the
data and wrote the original draft with support from DP. ASM and FR developed and tested the clustering
algorithm. SS contributed to force field
development. DP supervised and managed the project. All authors contributed to editing the manuscript. 

\section*{Conflicts of interest}
There are no conflicts to declare.

\section*{Acknowledgements}
This work was fully supported by the Joint Center for Energy Storage Research (JCESR), an Energy Innovation Hub funded by the U.S. Department of Energy, Office of Science, Basic Energy Sciences.  The theoretical analysis in this work was supported by a User Project at The Molecular Foundry and its computing resources, managed by the High Performance Computing Services Group at Lawrence Berkeley National Laboratory (LBNL), supported by the
Director, Office of Science, Office of Basic Energy Sciences, of
the United States Department of Energy under Contract DE-AC02-05CH11231.
\bibliography{refs} 


% \nocite{*}


\newpage

\newpage




%%%%%%%%%% Merge with supplemental materials %%%%%%%%%%
%%%%%%%%%% Prefix a "S" to all equations, figures, tables and reset the counter %%%%%%%%%%


\renewcommand{\thesection}{Supplementary Information \arabic{section}}    %%%% but here
\renewcommand{\theequation}{S\arabic{equation}}
\renewcommand{\thefigure}{S\arabic{figure}}
\renewcommand{\thetable}{S\arabic{table}}
\renewcommand{\bibnumfmt}[1]{[S#1]}
%\renewcommand{~\citeumfont}[1]{S#1}
%%%%%%%%%% Prefix a "S" to all equations, figures, tables and reset the counter %%%%%%%%%%

\setcounter{equation}{0}
\setcounter{figure}{0}
\setcounter{table}{0}
\setcounter{page}{1}
% \makeatletter
\setcounter{section}{0}

\widetext
% \begin{center}
% \textbf{\large Supplemental Materials: Ca-dimers, solvent layering, and dominant  electrochemically active species in Ca(BH$_4$)$_2$ in THF}
% \end{center}


\section*{Supplementary Information: Ca-dimers, solvent layering, and dominant  electrochemically active species in Ca(BH$_4$)$_2$ in THF}






\begin{figure}[h!]
\centering
\includegraphics[width=0.9\linewidth]{images/SI/si-bulk-paths.png}
\caption{(a) Bulk free-energy surface of dimerization of ionic and neutral species, controlled by borohydride coordination number. (b) Free energy profiles for bulk disproportionation pathways with colors corresponding to the same indicated on the free-energy surface in (a). (c) Direct dissociation pathways by removal of borohydride obtained from vertical slices through the free-energy surface at specific Ca-Ca separations.}
\label{fig:si_bulk-paths}
\end{figure}




\begin{figure}[h!]
\centering
\includegraphics[width=0.9\linewidth]{images/SI/si-dimer-minima.png}
\caption{ \textbf{Determination of short and long dimers equilibrium RT distances. }Free-energy of dimerization (distance Ca-Ca) as a function of the Ca-B(H$_4$) coordination number (left) and Ca-O(THF) coordination number (r$_0$ = 3.6~\AA, right).  One-dimensional profiles obtained through thermodynamic integration over each coordination number allow us to interpolate the equilibrium distances of the short and long dimer.   }
\label{fig:si-dimer-minima}
\end{figure}




 \begin{figure}[ht]
\centering
\includegraphics[width=0.7\linewidth]{images/SI/perpendicular_dimer_coordination.pdf}
\caption{ \textbf{Solvent coordination of perpendicular dimers.} Orientation of the Ca-Ca axis with respect to the surface normal of structures of the isomer SD 5 THF - 4 at the DL (see Fig. 3 in the main text) depending on the THF-Ca coordination of the Ca that resides in the DL. At strongly perpendicular orientations (larger that 150 $^{\circ}$ with respect to the surface normal), two-fold solvent coordination is favored. }
\label{fig:si_perpendicular_dimer_coordination}
\end{figure}



\begin{figure}[h!]
\centering
\includegraphics[width=0.9\linewidth]{images/SI/bh4_layering_thf.pdf}
\caption{\textbf{ Free-energy of BH$_4^-$ at the (biased) interface. }The free-energy profile of a single THF molecule in THF (top) at RT with respect to distance from a graphite interface (red) and the corresponding oxygen density profile (black), with slight adjustment (dashed lines) at negative bias. The Dense Layer (DL), Intermediate Density Layer (IDL) and bulk regions are color-coded hereafter in red, blue and gray. The bottom figure shows the analogous free-energy profile of a single BH$_4^-$ molecule in THF at RT with respect to distance from a graphite interface (red) and the corresponding boron density profile (black). }
\label{fig:si-bh4-layering}
\end{figure}

\newpage

\newpage



\begin{table}[h!]
\begin{center}
\begin{tabular}{|  c | c | c | c | c    | c | c | c | c | c   | c | c | c | c |} 
\hline

Fig. & THF & Ca$^{2+}$ & BH$_4^-$ & G  & Box pars.~(\AA)  & CVs  & q$_G$ (e$^-$) & W & H & Freq & Reps. & RepFreq &  Time (ns) \\

 \hline
 Bulk  &200&2& 4& -  & 30.3 x 30.3 x30.3  &  dCa [0:16]~\AA    & - & 0.05 & 0.02 & 200  & 20 & 50000 & 1660\\

\hline
 1 &   & &  &    &                    & CN(Ca-B) [0-4]        &   &  0.05 &     &      &    &       &          \\

\hline
Neutral  &336&2& 4& 2  & 33.9 x 31.8 x 49.8  &  dCa [0:16]~\AA   & G1 = G2 = 0 & 0.2 & 0.02 & 200  & 70 & 50000 & 2082\\
2,3,  &   & &  &    &                    & CN(Ca-B) [0-4]            &   &  0.2 &     &      &    &       &          \\
5  &   & &  &    &                    & dZ [0:25]                 &   &  0.5 &     &      &    &       &          \\

\hline
2  &336  &-& -& 2  & 33.9 x 31.8 x 49.8 &  dZ [0:25]~\AA   & G1 = G2 = 0 & 0.05 & 0.02 & 200  & 15 & 50000 & 1049 \\ 
\hline

2  &336  &-& -& 2 & 33.9 x 31.8 x 49.8& dZ [0:25] & G1 = -1.4;   G2 = 0 &  0.05 & 0.02 & 200  & 10 & 50000 & 620  \\ 
\hline


\hline
CS1 &514&2& 4& 2  & 33.9 x 31.8 x 71.8  &  CN(Ca-Ca) [0:1]~\AA    & G1 = -0.7;  & 0.2 & 0.04 & 200  & 70 & 50000 & 1438 \\

  &   & &  &    &                    & CN(Ca-B) [0-4]            &  G2 = +0.7  &  0.2 &     &      &    &       &          \\
  &   & &  &    &                    & dZ [2:70]                 &   &  0.25 &     &      &    &       &          \\
\hline
CS2 &514&2& 4& 2  & 33.9 x 31.8 x 71.8  &  CN(Ca-Ca) [0:1]~\AA    & G1 = -1.4;  & 0.2 & 0.04 & 200  & 70 & 50000 & 1464 \\
 2,4, &   & &  &    &                    & CN(Ca-B) [0-4]            & G2 = +1.4   &  0.2 &     &      &    &       &          \\
5  &   & &  &    &                    & dZ [2:70]                 &   &  0.25 &     &      &    &        & 
 \\ 
\hline

\end{tabular}
\caption{\textbf{ Metadynamics simulation details.} Figure number or label, box composition (Ca$^{2+}$, BH$_4^-$ and graphene) and equilibrated lattice parameters, collective variables and their ranges, surface charges, and metadynamics parameters (hill width and height, frequency, number of replicas, replica update frequency, and total simulation time).  }
\label{tab:si_systems}
\end{center}

\end{table}



 \begin{figure}[ht]
\centering
\includegraphics[width=0.9\linewidth]{images/SI/1_4_dimers_free_energy.pdf}
\caption{ \textbf{Interfacial free-energy of [1,4] dimers.} Free energy pathways as a function of distance from the neutral surface for dimers with [1,4] Ca-B(H$_4$)$^-$  coordination numbers at different Ca-Ca distances. The most favorable, with d(Ca-Ca) = 4.9-5.1~\AA{}, preferably sits far away from the layered interface, at $\sim$ 17.5~\AA{}.}
\label{fig:si_1_4_dimer_freeenergies}
\end{figure}


\section{Dipole moments of Ca(BH$_4$)$_2$ isomers} 
\label{sec:si-dipole}

In order to compute the dipole moment of specific Ca(BH$_4$)$_2$ isomer, an average structure of the whole group was calculated (including THF solvating molecules). Then, using Density Functional Theory as implemented in Terachem,\cite{Ufimtsev2008} the atomic positions were optimized to obtain the dipole moment. Typically, a decrease in the dipole moment of about 1 D is observed upon minimization. The B3LYP\cite{b3lyp} exchange-correlation functional with a dispersion correction (D3)\cite{grimme2010} and a 6-311G basis set were used. 




\section{Reduction potentials}
\label{sec:si-reduction potential}
We have estimated the onset potential for  reduction, $\mathcal{E}$, as follows: 

\begin{align}
    \mathcal{E} < \ \mathcal{E}^0 + [\ln n_O(z) dV + \mu_O^{\mathrm{excess}}(z)]/e \nonumber \\
    + \phi(z) + [\Phi_O(z) - \Phi_R(z)]/e \label{eq:reductionpotential}
\end{align}

where $\mathcal{E}^0 = -\Delta G^{\mathrm{red}}/nF $, the standard reduction potential, is approximated as $-[E_{\mathrm{solv}}[R] - E_{\mathrm{solv}}[O] - \epsilon_F]/nF$ for an $n$-electron process ($F$ is Faraday's constant, $\epsilon_F$ is the metal electrode work function). In this work we focus on single-electron transfer events only (n=1) for electrolyte species. This term was obtained individually for each isomer structure using Density Functional Theory (we used the QCHEM code{~\cite{qchem}} through the PyQchem interface.{~\cite{pyqchem}}  The B3LYP{~\cite{b3lyp}} exchange-correlation functional with a dispersion correction (D3){~\cite{grimme2010}} and a 6-311G basis set were used, with implicit solvation modeled using PCM.{~\cite{pcm}}
The activity-dependence in this expression is captured by two terms related to: (1) the concentration profile of each species before reduction, $n_O (z)$ ($dV$ is the smallest volume element in our continuum model, set equal to the effective volume of the smallest species); and (2) the excess chemical potential, $\mu^{excess}_O(z)$, which includes non-ideal concentration contributions mostly due to volume exclusion. The electrostatic potential profile is indicated by $\phi(z)$, as function of distance from the electrode. These profiles were obtained self-consistently with our continuum model, employing as inputs the specific adsorption potential profiles of the oxidized and reduced species, $\Phi_O(z)$ and $\Phi_R(z)$, which were computed with molecular-dynamics-based free-energy sampling in the dilute limit.


\section{Thermodynamic reduction potential}\label{sec:si-thermo_red_pot}  In order to estimate the overpotential, we computed the thermodynamic reduction potential of the 2$e^-$-reduction-deposition process of a dication in the bulk electrolyte, with reference to Ca$^{2+}$ in vacuum so as to enable comparisons with the interfacial reduction potentials, as follows:

\begin{align}
 \Delta G^{red-dep} =  \Delta G^{red}_{(g)} -  \Delta G_{(sol)} - E_{at} - n(\epsilon_F - \varphi(0)  )  
\end{align}

Then, the potential at the electrode required for thermodynamically favorable reduction-deposition ($\Delta G^{red-dep}  \le 0 $) can then be defined as:

\begin{align}
 \varphi (0)^* \leq \epsilon_F - ( \Delta G^{red}_{(g)} -  \Delta G_{(sol)} - E_{at}) /n    
\end{align}

We assume a calcium metal electrode, with Fermi energy $\epsilon_F$ = -2.84 V, calcium atomization energy E$_{at}$ = 1.845 V,{~\cite{Oxtobyc2002}} and n=2 (averaging over  two electron transfer steps). We obtain  $\Delta G^{red}_{(g)}$ and  $\Delta G_{(sol)}$ from DFT calculations on Ca$^{2+}$(THF)$_6$ isomers favored in the bulk (see Fig.\ref{fig:si_mono_dication_full_solvent_distr}, bottom right). A thermodynamic average of the values of $\varphi$ (0)* for each isomer using bulk populations yields a bulk thermodynamic reduction potential for Ca$^{2+}$ in THF of -2.25~V. 



\begin{figure}
    \centering
    \includegraphics[width=0.6\paperwidth]{images/SI/ET_contribution_0.12M.png}
    \caption{Contributions to the electron transfer rate integrated throughout the interface for various overpotentials as a function of the dielectric spacer width (expressed as a fraction of the tunneling decay length $z_0$ at 0.12~M.}
    \label{fig:si_low_conc_ET_rates}
\end{figure}


\begin{figure}
    \centering
    \includegraphics[width=0.8\paperwidth]{images/SI/si_monocation_dication_interface_full_solvent.png}
    \caption{Population distribution of monocation (top) and dication (bottom) isomers by a biased electrode (CS2). Whole solvent and BH$_4$ molecules are included in the coordination sphere.  For the monocation, dipole orientation distribution of isomers with population larger than 10\% per layer are shown at the top. The angle is calculated with respect to the surface normal, with 180 being the BH$_4$ pointing away from the electrode. The bottom histograms show isomer populations obtained from unsupervised learning analysis  close to the center of the DL (z = 4.8 \AA),  close to the edge of the DL (z = 5.8 \AA) , in the IDL (z = 8.25 \AA), and in the bulk.   Representative atomic structures of the dominating isomers are shown in the inset in their preferred orientation, with the surface on the left.  Note the similar population distributions in the bulk and IDL, and the `flattened' DL isomers for monocations (4 THF - 1, 4 THF - 2, 5 THF - 1, and 5 THF - 5) and dications (5 THF - 0 and 1, and 6 THF - 0, 4 and 8). }
    \label{fig:si_mono_dication_full_solvent_distr}
\end{figure}

\bibliography{refs} 





\end{document}

