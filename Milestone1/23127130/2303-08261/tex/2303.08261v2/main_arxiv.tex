\documentclass[%
 reprint,
 superscriptaddress,
 amsmath,amssymb,
 aps,
 % linenumbers
]{revtex4-2}


\usepackage{natbib}

\setcitestyle{super,sort&compress,comma}
\renewcommand{\bibsection}{\section*{References}}
% \usepackage[switch, modulo]{lineno}
\usepackage{color,soulutf8}

% \usepackage{refcheck}
\usepackage{threeparttable}
\usepackage{float}
\restylefloat{table}
\usepackage{subcaption}

\usepackage{dcolumn}% Align table columns on decimal point
\usepackage{bm}% bold math
\usepackage{braket} %braket notation

\usepackage[utf8]{inputenc}
\usepackage{graphicx}
\graphicspath{ {images/} }

% add support for chemical formula
\usepackage[version=4]{mhchem}

%Highlight equations:
\usepackage{empheq}
\usepackage[many]{tcolorbox}
\tcbset{
  highlight math style={
    colback=yellow,
    arc=0pt,
    outer arc=0pt,
    boxrule=0pt,
    top=0pt,
    bottom=0pt,
    left=0pt,
    right=0pt,
  }
}
% \usepackage{xr}
\bibliographystyle{naturemag}

\begin{document}

\title{Ca-dimers, solvent layering, and dominant  electrochemically active species in Ca(BH$_4$)$_2$ in THF}

\author{Ana Sanz Matias}
% \email{asanzmatias@lbl.gov}
\affiliation{%
 Joint Center for Energy Storage Research, Lawrence Berkeley National Laboratory, Berkeley, CA 94720, USA
}%
\affiliation{The Molecular Foundry, Lawrence Berkeley National Laboratory, Berkeley, CA 94720, USA}
\author{Fabrice Roncoroni}
\affiliation{%
 Joint Center for Energy Storage Research, Lawrence Berkeley National Laboratory, Berkeley, CA 94720, USA
}%
\affiliation{The Molecular Foundry, Lawrence Berkeley National Laboratory, Berkeley, CA 94720, USA}
\author{Siddharth Sundararaman}
\affiliation{%
 Joint Center for Energy Storage Research, Lawrence Berkeley National Laboratory, Berkeley, CA 94720, USA
}%
\affiliation{The Molecular Foundry, Lawrence Berkeley National Laboratory, Berkeley, CA 94720, USA}
\author{David Prendergast}%
\email{dgprendergast@lbl.gov}
\affiliation{%
 Joint Center for Energy Storage Research, Lawrence Berkeley National Laboratory, Berkeley, CA 94720, USA
}%
\affiliation{The Molecular Foundry, Lawrence Berkeley National Laboratory, Berkeley, CA 94720, USA}



% 150 words
\begin{abstract}
Divalent ions (Mg, Ca, and Zn) are being considered as competitive, safe, and earth-abundant alternatives to Li-ion electrochemistry, but present challenges for stable cycling due to undesirable interfacial phenomena. We explore the formation of electroactive species in the electrolyte Ca(BH$_4$)$_2$$|$THF using molecular dynamics coupled with a continuum model of bulk and interfacial speciation. Free-energy analysis and unsupervised learning indicate a majority population of neutral Ca dimers and monomers with diverse molecular conformations and an order of magnitude lower concentration of the primary electroactive charged species -- the monocation, CaBH$_4^+$ -- produced via disproportionation of neutral complexes. Dense layering of THF molecules within $\sim$~1~nm of the electrode surface strongly modulates local electrolyte species populations. A dramatic increase in monocation population in this interfacial zone is induced at negative bias. We see no evidence for electrochemical activity of fully-solvated Ca$^{2+}$. The consequences for performance are discussed in light of this molecular-scale insight.

\end{abstract}

\flushbottom
\maketitle
\thispagestyle{empty}

\section*{Introduction}

To increase the rate of conversion to renewable energy sources, electrification of various energy-intensive aspects of society is underway. The concomitant demand for electrochemical energy storage solutions increasingly highlights the limits of Li-ion technologies with respect to performance, safety and sustainability. Multivalent ions such as Mg$^{2+}$, Ca$^{2+}$, Zn$^{2+}$, or even Al$^{3+}$, offer more earth-abundant alternatives, some with higher theoretical specific capacity and reduced safety concerns due to self-passivation of metal anodes.~\cite{Dong2020, Mohtadi2021, Lv2022, GuWu2017, Arroyo-deDompablo2020} 

However, realizing performant electrochemical cells using these ions is hindered due to various issues driven by interfacial phenomena:  low power output and charging rates due to large overpotentials and associated electrolyte decomposition and interphase growth.{~\cite{McClary2022, Liang2020,Deng2022,LeeYu2023,Hosein2021}} At issue is our lack of understanding of the specific complex nature of solvation in suitable electrolytes for multivalent ions and the identification of which of these species are active at the electrode-electrolyte interface and why.~\cite{forer-saboya2022} We may be biased by familiarity with aqueous solutions and the ability of water to generate perfect electrolytes, with fully dissociated and solvated ions, for many salts. However, organic solvents (required for a sufficiently wide window of electrochemical stability in batteries), with typically lower dielectric constants and larger molecular sizes, have increased residence times for coordinating highly charged cations and have relatively little interaction with anions, unlike ambipolar water molecules which more easily solvate both charges.

Molecular dynamics provides a window to the inner workings of electrolytes, revealing details of coordination of cations by solvent molecules and anions.{~\cite{YaoZhang2022, ChenZhang2020}} However, in the study of highly-charged species in poor dielectrics, we must take care to avoid sampling only a limited set of coordination states due to their long lifetimes and unavoidable limitations in computing time and complexity. Free-energy sampling allows us the opportunity to pose fundamental questions regarding the chemical composition of a poor electrolyte and the mechanisms by which its solvated species interconvert.~\cite{Roy2016, Baer2016, Joswiak2018, Baskin2019, baskin2020, Silvestri2022} Mining the large quantities of compositional and conformational data produced by these simulations presents its own challenge. Here we rely on recently developed unsupervised learning approaches~\cite{Roncoroni2023} to provide a faster path to extracting molecular-scale insight and guidance for future experiments to validate our predictions.

Calcium is a competitive multivalent candidate largely because its redox potential is just 0.2~V above Li (lower than Mg, Zn or Al). Despite decades of research,{~\cite{WeiLiu2022}} only in 2016 was reversible plating and stripping on calcium metal achieved.{~\cite{Ponrouch2016}} Room-temperature reversibility followed swiftly, in 2018, using Ca/BH$_4^-$/THF.{~\cite{Wang2018}}
Hydride-terminated anions such as BH$_4^-$   offer high efficiency without significant passivation{~\cite{Wang2018}} partly due to solvent and salt stability against the anode{~\cite{Liepinya2021}} in comparison with typical Li-ion salts and solvents.{~\cite{WangCheng2018}}  Newer, more complex, hydride-borane based electrolytes, such as carba-closoborane in mixed solvents (DME/THF),{~\cite{Kisu2021}} with similar performance and reversibility, exhibit a wider electrochemical stability window and have recently shown good long-term operation characteristics.{~\cite{Kisu2023}} This promise makes it even more important to understand the key aspects of hydride-based{~\cite{Hahn2023}} anion electrolytes. Hence, we study Ca(BH$_4$)$_2$ in THF as a promising electrolyte candidate  for room-temperature, reversible calcium deposition  with best-in-class efficiency and minimum parasitic reactions -- with complex bulk and largely unknown interfacial solvation environments.{~\cite{Wang2018, hahn2020, jieTan202, Liepinya2021, ta2019, melemed2020, McClary2022, YangTrahey2023, Melemed2022, Melemed_Oct2023}}

 We can also contrast its behavior with Mg(TFSI)$_2$ in THF, studied recently using free energy sampling.{~\cite{baskin2021}} Recent investigations of the complexation in the bulk electrolyte{~\cite{hahn2020, Hahn2021}} detected Ca-dimers  (Ca$_2$(BH$_4$)$_4$) experimentally. Dimers were suggested to facilitate the disproportionation of neutral monomers  into anions Ca(BH$_4$)$_3^-$ and monocations CaBH$_4^+$, proposed to be the main electroactive species:


\begin{align}
       \ce{ 2 Ca(BH4)2 & <=>  Ca2(BH4)4 <=>}\nonumber \\
    \ce{&CaBH4^{+} + Ca(BH4)3^{-}}   \label{eq:dimer_eq}
\end{align}


In the same work, molecular cluster calculations (embedded in a polarizable continuum model) indicated that the dimer is the second most stable conformation after the neutral monomer, but lacked specific solvent interactions beyond the first coordination shell and any Debye screening from finite ion concentrations.

Interfacial characterization reveals the ready formation of solid-electrolyte interphases incorporating oxidized boron and even embedded calcium hydride.~\cite{McClary2022} And debate continues as to the presence or electrochemical relevance of the fully-solvated Ca$^{2+}$ dication.{~\cite{Wang2018, ta2019,melemed2020, jieTan202, Liepinya2021, rev-Lu2021, Hahn2023}}


\begin{figure*}[ht!]
\centering
\includegraphics[width=\linewidth]{images/bulk.png}
\caption{\textbf{Analysis of the bulk electrolyte.} (\textbf{a}) One time step sampled from our molecular dynamics model of the bulk electrolyte showing two Ca$^{2+}$ dications (green) and 4 BH$_4^-$ anions (B atoms pink) in THF (O atoms red) at room temperature (RT).  (\textbf{b}) Free-energy surface sampled using metadynamics with respect to Ca-Ca distance and Ca-B(H$_4$) coordination number. Minimum energy pathways for dimer disproportionation from the neutral species, Ca(BH$_4$)$_2$ are indicated by dashed lines, and were used to parametrize a continuum model to obtain (\textbf{c})  equilibrium bulk populations of neutral and charged electrolyte species as a function of concentration. (\textbf{d}) Population analysis with respect to anion and THF coordination and conformation about Ca ions obtained via unsupervised learning, indicating the diversity of species umbrella-sampled from points on the free-energy surface corresponding to the neutral monomer (CN(Ca-B) = 1.9 and d(Ca-Ca) = 11~\AA) and the long (L) and short (S) dimers (d(Ca-Ca) $<$ 7~\AA). Atomic structures of dominant structures (populations larger than 7.5\%) are shown.}
\label{fig:bulk}
\end{figure*}

In this work, we reveal intricate details of the population of species in the bulk of this electrolyte and striking differences within a nanometer of its electrode interfaces, which are further exaggerated by negative potential differences. We discuss the consequences of these predictions for functioning cells.


\section*{Results}

\subsection*{In the bulk electrolyte}

We employ free-energy analysis for a model of the bulk electrolyte at room temperature (RT), comprising two Ca$^{2+}$ ions (and four borohydride anions) dissolved in THF (Fig.~1 a), using empirical force fields (see Methods and Supporting Information). The free energy surface (Fig.~1 b) is sampled using metadynamics~\cite{Laio2002} with respect to two collective variables: the Ca-B coordination number, which controls the charge of complexes, and the Ca-Ca distance, which for a system with only two Ca ions, distinguishes between monomers and dimers.

This projection of the free energy landscape indicates a preference for dimer formation in competition with the entropy gains associated with free monomers. From analysis of our molecular dynamics (MD) trajectories, Ca-Ca dimers are neutral complexes with the formula Ca$_2$(BH$_4$)$_4$ and come in two varieties: long dimers (63\%) bridged by a single BH$_4^-$ and short dimers (37\%) bridged by two anions (see below for more detail). 
Sampling local minima in the free energy surface allows us to parametrize a continuum model for chemical equilibration at finite concentration that includes a model of configurational entropy{~\cite{Wu2018, Baskin_2017, Baskin_2019_JCP}}, as shown in Fig.~1(c). In what follows, we will make explicit comparisons between the moderate (0.12 M) effective concentration of our MD simulations and a high concentration (1.65 M, close to saturation) which led to improved electrochemical performance in experiment.{~\cite{hahn2020, Hahn2021, YangTrahey2023}} 
The relative population of Ca-Ca dimers increases with concentration and is the most favored species above 2~M (although this is likely above the solubility limit). 
Conversely, the population of the neutral monomer complex, Ca(BH$_4$)$_2$, decreases with concentration. This might be expected from the perspective of solvent entropy -- the need to coordinate more dissolved species at higher concentrations, which reduces entropy, can be somewhat offset by the formation of solute oligomers, such as contact ion complexes, dimers, etc. which require less solvent coordination.

It is notable that only a small percentage of the solvated population is predicted to be charged ($<$2\% at low concentrations), dominated by the monocation, CaBH$_4^+$, with its complementary anion, Ca(BH$_4$)$_3^-$. Both bulk populations decrease with concentration. The free energy for the formation of fully solvated dications, Ca$^{2+}$, in this model, is too high ($\sim$19~kT) to support a significant bulk population at any concentration. These results provide some reordering with respect to static estimates which predict the neutral monomer as most favorable, followed by a single (short) dimer conformation, the monocation and the anion.~\cite{hahn2020} From the relative populations of each species, it is clear that neglecting the long dimer would lead to this conclusion. Furthermore, it is clear from RT sampling that there are multiple possible conformations for the dimer at finite temperature (as we explore below). However, both sets of calculations agree that the fully solvated dication Ca$^{2+}$ is the least favored in this set of possibilities (1-1.2~eV higher in energy by static estimates~\cite{hahn2020}, 0.48~eV from free-energy sampling at RT  at an effective concentration of 0.12 M).  

Our 2D free energy surface (Fig.~1 b) reveals that direct interconversion of these solvated objects, by removal or addition of borohydride anions, is prevented by quite steep free energy barriers (24.4~kT or $\sim$0.6~eV). The easier path to disproportionation (forming charged species from neutral monomers) is through the formation of dimers, with exchange of borohydride anions before dissociation into the complex monocation and anion as per Eq.~1 (with activation energies of 2.4 - 10~kT, Fig.~1 b and Fig.~S1).

At a minimum, this explains the prevalence of dimers in spectroscopic analysis of the bulk electrolyte (using EXAFS and Raman spectroscopy)\cite{hahn2020} and possibly the observation of saturating ionic conductivity with increasing concentration as more neutral dimers form (prior to precipitation, Fig.~1(c)).~\cite{hahn2020,Hahn2021} 

The predicted concentration-dependence of neutral species is in great agreement with experimental reports;{~\cite{Hahn2021}} and although we obtain a similar order of magnitude for the monocation, our results indicate that the monocation population actually decreases with concentration, opposite to experimental results. However, as we will see below, bulk concentrations of charged species may have little to do with the electrochemical performance which is dominated by interfacial phenomena.

Strikingly, we find that bulk populations of each complex are conformationally quite diverse. Our metadynamics simulations project the full free energy landscape onto only a few collective variables, however, multiple molecular conformations may satisfy these constraints (Ca-B coordination number and Ca-Ca distance, in this case).
Data-mining techniques (dimensionality reduction, hierarchical clustering and permutation-invariant alignment~\cite{Roncoroni2023} -- details in Methods and SI) applied to selective umbrella sampling of local minima in the free energy landscape reveal a rich variety of solvated isomeric structures, summarized in Fig.~1 d. 

For example, the neutral monomer, Ca(BH$_4$)$_2$, when additionally coordinated by three THF molecules ($\sim$~60\% of its population) adopts mostly bent borohydride arrangements with a large dipole moment ($\sim$8 Debye, see  Fig.~1 d, isomer 3 THF - 1 and SI Section 1). In addition, we found a significant portion ($\sim$~30\%) of monomers coordinated by 4 THF molecules with an axial borohydride arrangement and low dipole moment ($\sim$2 Debye, see Fig.~1 d, isomer 4 THF - 2)). These two structures had been proposed separately as the minimum energy structure from quantum-chemical cluster calculations~\cite{hahn2020} and molecular dynamics simulations,~\cite{Hahn2021} respectively.  By contrast, the monocation occurs mostly with 5 coordinating THF molecules (see below), in agreement with previous reports.\cite{hahn2020}

We find that dimers are always neutral (with four borohydride anions) and are split in two main spatial configurations characterized as short (SD) and long (LD) dimers with average Ca-Ca equilibrium distances of 4.48 and 5.25~\AA{}, respectively (Fig.~S2). Furthermore, each was found to have sub-populations with 4-7 solvent molecules, and, within those, several stereoisomers (Fig.~1 d). Key differences between the long and short dimers are the presence of a single-anion or double-anion bridge and predominant 6 THF coordination or mixed 5-6 THF coordination, respectively. Short dimers are in excellent agreement with a previously proposed double-bridged dimer structure with a 4.4~\AA{} Ca-Ca distance (from fitting to EXAFS  data).~\cite{hahn2020}
The set of structures shown here expands upon and underscores the configurational flexibility of calcium.~\cite{hahn2020}
And, based on our understanding of the efficient disproportionation pathways to form charged species  via dimerization (discussed above), it makes sense that the dominant dimer conformations form from combinations of bent and axial borohydride arrangements of the neutral monomers (e.g., long dimer isomers 1, 3 and 7 with 6 THF molecules in Fig.~1 d are bent-bent, axial-bent and axial-bent combinations, respectively).

Although all dimers here are neutral, each Ca ion within a given dimer may be locally coordinated by 1 to 4 anions, with 1 (long) or 2 (short) shared between them. Most commonly we observe  [3,2] or [3,3] anion arrangements for long or short dimers, respectively. Small populations of [1,4] dimers are found and are key to some interfacial disproportionation processes discussed below and in Section `Generation of active species'.


\begin{figure*}[ht!]
\centering
\includegraphics[width=\linewidth]{images/interface_layering.png}
\caption{\textbf{Effect of solvent layering at the interface}. (\textbf{a}) The free-energy profile (red) of a single THF molecule in THF at RT with respect to distance from a graphite interface and the corresponding oxygen density profile (black), with slight adjustment (dashed lines) at negative bias. 
The Dense Layer (DL), Intermediate Density Layer (IDL) and bulk regions are color-coded hereafter in red, blue and gray. The inset shows a snapshot of the interfacial region sampled from molecular dynamics indicating THF molecules lying flat against the graphite surface and somewhat constrained at the DL-IDL interface. (\textbf{b}) The 3D free energy landscape derived from metadynamics with respect to Ca-Ca distance, Ca-B coordination number, and distance from the graphite surface for the neutral interface (at 0.12 M). Minimum free-energy paths for (\textbf{c}) neutral and (\textbf{d}) charged species in the electrolyte arriving at the interface from the bulk under zero (solid) and negative (dashed) bias conditions.  Sudden jumps in the free-energy profiles reflect variations with respect to other collective variables in the free-energy landscape. Under negative bias the long and short dimers are not distinguished in our sampling. In general, negative bias stabilizes charged species at the interface.} 
\label{fig:interface_layering}
\end{figure*}

 \begin{figure*}[ht!]
\centering
\includegraphics[width=0.9\linewidth]{images/neutral_interface.png}
\caption{\textbf{Distribution of species at the neutral interface.} Molecular model of the simulation box, obtained from a snapshot of the equilibrated MD trajectory (\textbf{a}), showing a dimer at the IDL (in blue), near the DL (in red). Relative concentration of neutral and charged species as a function of distance from the electrode(\textbf{b}) calculated with the continuum model for total bulk concentrations 0.12~M and 1.65~M. The DL and IDL regions are marked with red and blue rectangles. Unsupervised clustering analysis of umbrella sampling trajectories at the DL and IDL (z = 5.75 and  8.25~\AA{}) show the relative populations of dimer isomers per layer (\textbf{c,d}). Representative structures have discrete orientations, which depend on the layer (insets in \textbf{e,f}). 
Color lines show the orientation of the Ca-Ca axis with respect to the surface normal, where 90\textdegree{} is parallel to the surface, as shown in IDL dominating dimer LD 6 THF - 7  (also in \textbf{a}), while DL dimer SD 5 THF - 4 is mostly  perpendicular. Dipoles also have discrete orientations with respect to the surface normal for given isomers (shown in black, dotted lines) and are roughly aligned to the Ca-Ca axis in long, flat dimers (IDL) or perpendicular, as in DL dimer SD 6 THF - 2. }
\label{fig:neutral_interface}
\end{figure*}

\subsection*{At the electrode-electrolyte interface}

What happens to species in the electrolyte as they approach an interface, such as the electrode surface? 
Firstly, as expected from simple statistical mechanics modeling of molecules at hard interfaces,~\cite{Henderson1976} the solvent, THF, adopts a layered molecular structure~\cite{Qiao2010,baskin2021} near the surface with a dense layer (DL) at 3-6~\AA{}, followed by a low-density `gap', and an intermediate density layer (IDL) at $\sim$~7-10~\AA{} \ from the surface,  with 2.5, 0.3 and 1.5 times the bulk THF density, respectively (Fig.~2 a). A third collective coordinate, calcium distance from the interface, sampled with metadynamics allows us to obtain the interfacial, 3D free-energy landscape (Fig.~2 b), of which the most distant slice is the 2D bulk free-energy landscape in Fig.~1 c. Dissolved species in the electrolyte near the electrode surface respect this underlying solvent layering with their free-energy minima distributed between the bulk, IDL, and DL, and separated by barriers, as indicated by minimum energy pathways of neutral and charged species approaching the interface in the 3D landscape (Fig.~2 c and d).

A key observation is that the IDL defines an attractive basin for most species, especially dimers and monocations, with minima in free energy that are lower than in the bulk, implying that this is a narrow interfacial region for enhanced concentration of solutes. Conversely, the DL defines a region from which solutes may be excluded due to additional free energy costs, without assistance of some applied bias (see below). This has some striking consequences for electrochemistry that  will be apparent when discussing reductive processes (see below).

Expanding our continuum model to include the presence of an electrode and the sampled interfacial free-energy hypersurface{~\cite{Wu2018, Baskin_2017, Baskin_2019_JCP}} allows us to calculate the potential of zero charge (PZC), which is -1.8 meV at 0.12 M, with negligible variations in interfacial concentration between open circuit potential (OCP) and PZC conditions. The metadynamics simulations used in this section, with zero excess charge, are thus representative of an open-circuit system.


Using our re-parametrized continuum model, we find that, next to this unbiased non-interacting electrode (Fig.~3 a), concentration effects are even stronger than in the bulk.  At 1.65 M,  the IDL is dominated ($\sim$~86\%) by dimers (long dimers in particular, 55\%), with reduced populations of neutral monomers ($\sim$~11\%) and a very slight increase to 2\% in the monocation population (Fig.~3 b). Around 30\% of the interfacial population is in the DL, where monocation population rapidly declines and dimers dominate ($\sim$~92\%) -- especially short dimers, in contrast to bulk and IDL populations. At low concentration, 0.12 M, only 13\% of the population is in the DL; neutral monomers dominate the DL and IDL; dimer population is limited to 15 and 18\% respectively; and there is a slight accumulation of monocation at the IDL (7\%).


Unsupervised clustering analysis of structures from umbrella-sampling trajectories at the IDL and DL (z = 8.25 and 5.75~\AA{}, respectively) reveals a reduced number of favored dimer isomers compared to the bulk, with one given isomer making $>$20~\% of the population in each case  (Fig.~3 c-d). In the IDL, this is a bent-axial long dimer with 6 solvating THF molecules (indexed as Isomer 7 or LD 6 THF - 7), already a favored species in the bulk. In the DL, on the other hand, a short dimer also with 6 THF molecules (SD 6 THF - 2) dominates. Furthermore, we find that the orientation of dimers at the interface is discretized (Fig.~3 e,f). Favored IDL dimers have mostly flat orientations of the Ca-Ca vector relative to the graphite surface, with two dense layer THF molecules involved in solvation. Similarly, the dominant dimer orientation in the DL is flat (i.e., in a plane parallel to the surface), with both calcium ions embedded in that dense region, and with two THF molecules from the IDL contributing to solvation. Additionally, perpendicular dimers form at least 11\% of the DL population. Isomer 4 of the short dimer with 5 THF molecules (SD THF 5 - 4) is an asymmetric dimer, with calcium ions solvated by 2 and 3 THF molecules that sit in the DL and on the edge of the IDL, respectively (Fig.~S3). 
Note that the specific conformation of these coordination complexes defines their effective dipole moment, which may, in some cases, be orthogonal to the Ca-Ca vector of the dimer. An example in the DL is the favored isomer SD 6 THF - 2, with its Ca-Ca vector parallel to the surface but three BH$_4^-$ sitting closer to the interface, near the corresponding DL free energy minimum for isolated borohydride in THF next to graphite (z= $\sim$4~\AA{}, see Fig.~S4). Dipole orientation will be discussed in more detail below in the context of biased interfaces.

The closest approach of fully solvated ions (as either dimers or neutral monomers at OCP) takes place in the IDL, which by some definitions is analogous to the Outer Helmholtz plane (OHP). Similarly, the DL is a close analogue of the inner Helmholtz plane (IHP). One key distinction from textbook definitions of the IHP (usually for aqueous electrolytes) is that THF layering in the dense layer leads to a flat orientation of THF molecules with respect to the electrode surface, and we did not observe any large reorientation of its dipole with changes in surface charge (below). In this electrolyte only the dipole moments of dissolved complexes exhibit reorientation in the DL. Finally, lack of solvent layering more than 12~\AA{} from the electrode maps onto the diffuse layer.


\subsection*{Negatively biased interfaces}

So far, it seems that Ca(BH$_4$)$_2$ in THF is a poor electrolyte, with only a small fraction of the salt concentration (2\% for both 0.12~M and 1.65~M) present as charged species in the bulk, albeit with a noticeable enhancement to 7\% in the  IDL for a concentration of 0.12 M at OCP. Otherwise this electrolyte is dominated by neutral species (monomers and dimers). This is consistent with previous discussion of  undissociated neutral species as dominant in Ca-based electrolytes with boron-containing anions~\cite{forer-saboya2022}. However, the bulk free energy landscape (Fig.~1 c) indicates that interconversion of species is possible (more details below) and some equilibrium exists between charged and neutral species. We explore the impact of biased/charged electrodes on the population of solutes by evenly distributing opposing charges on either face of the two-layer graphite electrode model, which, under periodic boundary conditions, polarizes the electrolyte. We considered two specific charge states with (1) 0.065 and (2) 0.13~e/nm$^2$ (labeled CS1 and CS2, respectively). These surface charge densities  correspond to potential differences of $-$0.26~V (CS1) and $-$0.42~V (CS2), calculated using our continuum model (see Methods) assuming a simulated bulk concentration of 0.12 M. (Note that at higher bulk concentrations, with a shorter Debye screening length, this estimated potential difference should be even smaller -- at 1.65 M: $-$0.20~V (CS1) and $-$0.35~V (CS2).)  These potentials are sufficient to draw monocations into the dense layer.



 \begin{figure*}[ht!]
\centering
\includegraphics[width=0.9\linewidth]{images/negative_interface.png}
\caption{ \textbf{Populations at a biased interface.} Snapshot obtained from the equilibrated trajectory of charged state 2 ($\varphi(0)$ = $-$0.35 V) with highlighted layering (red/blue for DL/IDL) at the negative interface showing the IDL solvated monocation (\textbf{a}). Relative concentration of neutral and charged species as a function of distance from the negatively charged electrode (\textbf{b}) at potential differences corresponding to CS 1 and CS 2 for a bulk concentration of 1.65~M as calculated using the continuum model. Again, the DL and IDL regions are marked with red and blue rectangles. Unsupervised clustering analysis of Umbrella Sampling trajectories of the monocation at the IDL, close to the edge of the DL (z = 5.8~\AA{}) and close to the center of the DL (z = 4.8~\AA{}) (\textbf{c}) classifies the structures into two isomers of the five-fold and four-fold THF coordinated monocation, which differ by slight changes in the local geometry of the coordinating THF molecules. The corresponding Ca-B dipole orientation with respect to the surface normal  ($\vec{\mu}$), with 180$^{\circ}$ corresponding to the B pointing away from the surface, indicate discrete orientation at the interface (\textbf{d}). Representative structures of the main isomers at each layer in their most likely orientation, with the surface on the  left side (\textbf{e}). }
\label{fig:negative_interface}
\end{figure*}


From Fig.~2 (a) we see that the solvent layering remains practically unchanged upon charging the electrode.
At high concentration (1.65 M), a third of the interfacial population is in the DL. Dimers are still the most favored species in the DL and IDL in both charge states (Fig.~2 a, b). With increase in the bias potential,  some charged species become strongly stabilized at the interface. Specifically, the monocation, CaBH$_4^+$, partly displaces the neutral dimers and monomers (Fig.4 b)  to become up to 27 and 21\% of each layer's population, respectively (compared to 0.3 and 2\% in the unbiased electrode). The concentration dependence already seen at the unbiased electrode becomes far more striking at negative biases. Low concentration (0.12 M) leads to a sparser DL, which constitutes only 16\% of the total interfacial population. Both DL and IDL are completely dominated by the monocation  (98 and 92 \%  respectively, for CS2) while dimer population declines less than 1\% in both. As in the unbiased electrode, the bulk population at low concentration consists mostly of neutral monomers (95\%).

Notably, the peak concentration of monocation in the DL sits near its external edge, at $\sim$~6~\AA{}. At this potential, the approach of the monocation to the electrode, through the DL, would have to be an endergonic process, requiring 5.7~kT of free energy, overcoming a barrier of 7.5~kT. Although, this is some improvement over the case of the neutral electrode, which presents a barrier of 9.9~kT to enter the DL, with a required input of 9.1~kT of free energy.

We can understand these free energy costs by following the evolution in solvation and associated dipole orientation of the monocation. Unsupervised learning analysis of various umbrella sampling trajectories: in the IDL, at the edge of the DL and in the DL (z = 8.25, 5.8 and 4.8~\AA{}, Fig.~4 c-e), reveals that five-fold THF coordination dominates the IDL population, with the BH$_4^-$ anion pointing away from the surface -- as one might expect given the direction of the electric field at the negative electrode. 
However, this dipole reorients at the edge of the DL, likely due to mixed DL/IDL solvent coordination, and a small four-fold 
THF coordinated population appears. Ultimately, in the center of the DL, four-fold coordination dominates, with the dipole of the predominant isomer pointing away from the surface again. 

This complicated and costly path for the monocation to reach the electrode further emphasizes the important role of the THF solvent layering and specific coordination in determining the electrochemical activity. By the same token, the fully-solvated dication, Ca$^{2+}$, with its somewhat rigid first solvation shell, is still too unfavorable to define a noticeable population at these bias potentials, despite its higher electrostatic charge. Stabilization of Ca$^{2+}$ species to reach non-negligible concentrations is only achieved at relatively large (yet still non-reductive) potential differences in the inner DL ($<$~-1.1~V at 1.65~M), where preferred isomer structures are undercoordinated and flattened (see below).


 \begin{figure*}[!htb]
\centering
\includegraphics[width=1.\linewidth]{images/disproportionation.png}
\caption{\textbf{Disproportionation pathways} Scheme depicting the main two steps of dimer disproportionation (\textbf{a}): reorganization (\textbf{1-5}), followed by separation into ions (\textbf{6}) through distinctive paths A and B. 
Minimum energy pathways for disproportionation in the IDL and bulk for (\textbf{b}) neutral and (\textbf{c}) negatively charged (CS2) electrodes, obtained from the 3D free-energy landscapes.  Color-coding indicates distinctions between processes in the IDL (blue) and the bulk electrolyte (gray).} 
\label{fig:disproportionation}
\end{figure*}

\subsection*{Generation of active species}

Thus far, it seems that the monocation, CaBH$_4^+$, is the strongest candidate for the electroactive species in this electrolyte, given its high population in the vicinity of the electrode upon  negative charging. 
The fact remains that this poor electrolyte (only 2\% of species in solution are charged, as mentioned above) must supply the DL and IDL with monocations in the first place, via some disproportionation mechanism from neutral species (most likely dimers), and replenish the same species while electroreduction and electrode deposition consumes them. Any barriers in this supply chain should be evident in the kinetics of the electrochemistry as an observed deposition overpotential or associated rate limitations in charging cells with this electrolyte. The much smaller concentration of dimers in low concentration electrolytes may be partly behind the loss of performance observed in experiments at concentrations below 0.5~M -- which exhibit lower current densities and lower Coulombic efficiencies.{~\cite{hahn2020, YangTrahey2023}}  

To shed light on the underlying processes, we approach the generation of charged species (monocations) as a two-step process, involving dimer reorganization and disproportionation. As we show below, the most stable or prevalent dimer species are not readily disposed to disproportionate. Some molecular rearrangement of borohydride anions and solvent molecules is first required. The subsequent disproportionation follows two major pathways, which can occur in the bulk electrolyte or at the interface, and which we have investigated both at neutral and negative biases.

In the neutral cell, increased dimer concentrations in the IDL are readily explained through analysis of the free energy landscape (Fig.~2). 
The minimum energy path (MEP) for disproportionation in the presence of the interface indicates that neutral species in the bulk can easily flow, without a significant free energy barrier, into the IDL. Migration from the bulk to the IDL is essentially barrierless for all species except Ca$^{2+}$. As summarized in Fig.~5 a, to prepare for disproportionation of the most favorable dimers, some reorganization into an intermediate (less stable) dimer is required and ultimately Ca-Ca separation into ionic species follows two main paths, A or B.

For Path A, the dominant long dimer configuration (\textbf{1} in Fig.~5 a) can undergo a low-barrier reorganization, through a short dimer (\textbf{2}), to another long dimer with similar coordination of Ca ions with borohydrides (\textbf{3}). Then, the Ca-Ca distance increases (\textbf{4}) up to  6.7~\AA{} at the transition state so that the nascent monocation Ca ion can increase its number of coordinating THF molecules, from the original 3-4 to the preferred 5, upon full dissociation. Path B branches from the short (\textbf{2}) or intermediate long (\textbf{3}) conformations, through further borohydride and solvent reorganization, to a higher energy dimer (\textbf{5}), with a [1,4] anion coordination, which then disproportionates to produce the monocation (\textbf{6}).   

Figures~5 b and c outline the MEP for disproportionation at the electrode interface or in the bulk electrolyte under neutral or negative bias. Overall, at the electrode interface, in the IDL, disproportionation follows Path A, whereas Path B is preferred in the bulk, likely due to configuration \textbf{5} being more favorable in the bulk than at the interface
(Fig.~S5). 

We find that interfacial (IDL) disproportionation at zero bias via Path A is favored, since it has slightly lower barriers (E$_{a, 3\rightarrow4}$ = 8.4~kT and E$_{a, 4\rightarrow 6}$ = 9.8~kT) and a lower free energy cost than bulk disproportionation via Path B (Fig.~5 b). This is in agreement with the slight increase in monocation population observed at the IDL in Fig.~3 (Slight variations in barriers between bulk and interface models can be due to differences in the collective variables and grid-spacing employed in our metadynamics simulations, SI Table S1). 

At the negatively charged electrode, we have seen (Fig.4) that the monocation is favored in the DL and IDL, displacing the previously dominant dimers  (neutral monomers) at high (low) concentration  with increasing negative charge on the electrode (CS2). Due to the size limits of our metadynamics simulations, we may well expect that the bulk thermodynamics are somewhat different from those under neutral conditions, however, disproportionation still follows Path B in our simulations (Fig.~5 c), albeit with an additional step involving the formation of a long dimer conformation (\textbf{3}). Similarly, Path A is still preferred in the IDL. Although disproportionation occurs in the bulk, barrier-less pathways to the IDL suggest that dimers can approach the charged IDL and undergo disproportionation quite favorably. 
Therefore, upon negative charging, the concentration of monocations should increase in the IDL, based on a favorable free energy and necessary -- highly concentration-dependent-- supply from the local (IDL) dimer population via interfacial disproportionation. 


 \begin{figure*}[!htb]
\centering
\includegraphics[width=1.\linewidth]{images/reduction_interface.png}
\caption{\textbf{Reduction at the negative electrode}  (\textbf{a}) Interfacial concentration profiles of monocation and dication as a function of overpotential at 1.65~M for the bare electrode (top) and with dielectric spacers of width $d_{spacer}$ = 5, 10 and 15~\AA{}.  (\textbf{b}) Contributions to the electron transfer rate integrated over the interfacial region for specific overpotentials as a function of spacer width (expressed as a fraction of the tunneling decay length, $z_0=$~1~nm). The experimentally observed overpotential on the first charge is 0.25~V{~\cite{Wang2018}} - which we assume corresponds to our bare electrode model (without dielectric spacer). A band of similar ET rate magnitudes to that at 0.25~V is indicated by a horizontal hatched pattern, highlighting that lower overpotentials combined with finite dielectric spacer widths can lead to similar ET rates. (\textbf{c}) A closer view of this effective ET rate regime indicating the relative contributions of individual dication and monocation isomers in the DL at the corresponding overpotentials (colored dots). IDL contributions are negligible and thus not plotted. The inset on the right shows the structure of the dominant contributing isomer: a flattened, undercoordinated monocation complex, 4 - THF 2, as obtained via unsupervised learning analysis including the entire solvation sphere.} 
\label{fig:reduction}
\end{figure*}

\subsection*{Reductive processes and the seeds of the SEI}

The   stabilization of the monocation in the DL at negative bias results in a significant increase within the overall interfacial (DL plus IDL) population of this species -- from $\sim$1\% (6\%) in the unbiased electrode to $\sim$23\% (94)\% for CS2 at 1.65~M (0.12~M). Structural analysis indicates that these monocations have their anionic end pointing away from the negative surface and have lower solvent coordination.

Using our interfacial continuum model, we explore increasingly negative electrode potentials at or slightly above the thermodynamic Ca deposition potential ($-$2.25 V, see Supplementary Information 3).
 We limit our discussion here to an electrolyte concentration of 1.65~M. Strikingly, a highly-localized dication population appears in the inner DL (3 to 5~\AA{}, Fig.~6 a). Unsupervised learning analysis, this time including whole solvent molecules in the coordination sphere, reveals that dication coordination at these short distances consists mostly of flattened, under-solvated (5 THF) structures. The more dominant monocation population is pushed slightly outwards by that first dication layer. This begs the question as to which species contributes the most to the electron transfer rate.

Electrochemical activity is a strong (exponentially decaying) function of distance of the reducible species from the electrode. It is also very likely enhanced by reduced cation coordination, i.e., favoring reduction of species that are less thermodynamically stable.{~\cite{Baskin2019}} However, this increased thermodynamic cost would also lower the local concentration of the same species. Additionally, the magnitude of the required reduction potential of a given species can increase with proximity to the interface, due to the stabilization of positively charged species by the negative electrode potential. This trade-off between species availability (concentration), distance to the electrode and reductive stability is key in determining each species contribution to the electron transfer rate.

The reduction potential of each isomer was calculated using a combination of quantum-chemical calculations, free-energy sampling and potential-dependent concentration profiles from the continuum model. Contributions to an effective electron-transfer (ET) rate were estimated based on distance dependent tunneling decay, reduction potential and isomer concentration (see Methods). Note that the concentration (more precisely, activity) dependence of the ET rate effectively follows the Nernst equation, although we are not modeling an equilibrium process here. Based on our calculations, we see 100- to 10,000-fold increases in the ET rate as we increase the electrolyte concentration from 0.12~M to 1.65~M (see Fig.~S6). In what follows, we focus on results for 1.65~M electrolytes. 

We further simulated the effect of SEI growth on ET rate within the continuum model by including a progressively wider dielectric spacer, under the assumption that electrons could tunnel through it with the same decay constant. In Fig.~6(a) we can see that the effect of holding incoming positively charged species farther away from the negatively charged electrode is to reduce their effective concentrations in the interfacial (dielectric-electrolyte) region, due to the decreasing potential difference in this region with increased spacer width. Most noticeably, this removes the interfacial population of dications that we observed at the pristine electrode. 

Increases in the magnitude of the negative electrode potential above the thermodynamic deposition potential (labeled as overpotential here) have only minor effects on these concentration profiles. However, the species-dependent reduction potentials and the combined estimate of the ET rate are strong functions of this overpotential. In fact, the presence of a dielectric spacer that is close in width to the ET decay length (approximated as 1~nm here) appears to lower the required overpotential to reach the same effective ET rate, as shown in Fig.~6(b). For example, the pristine electrode exhibits an effective ET rate of $\sim 10^2$ at a 0.25~V overpotential, while the same or higher rate is possible with dielectric spacers of 0.5 to 1.5 times the ET decay length at a smaller 0.10~V overpotential. Note that the overpotentials discussed here are thermodynamic in origin. 

With no dielectric spacer, the main contributing species to the ET rate are monocations coordinated by 4 THF molecules in two specific arrangements (4 THF 0 and 4 THF 2), nominally undercoordinated and flattened with respect to their bulk electrolyte coordination, located at the outer edge of the DL (beyond 5~\AA{}). A representative molecular structure is provided in the inset of Fig.~6 (c) (see the rest in Fig.~S7). Interestingly, these monocations are not as close to the electrode as they could be, because of the presence of dications stabilized by the strong negative potential difference close to the electrode. However, the same dications require a larger overpotential to be reduced due to the stabilization provided by this potential difference, and thus do not contribute significantly to the effective ET rate.

With a 5~\AA{} spacer, the potential difference at the electrolyte interface is now lower, and the less thermodynamically favorable dication population disappears. The electroactive species in the DL region are now exclusively monocations, which can occupy the inner DL and be reduced at lower overpotentials due to a decrease in thermodynamic stabilization by the smaller potential difference this far from the electrode.

The final outcome of this analysis is that the under-coordinated monocations in the DL are the dominant contributors to the electrochemical activity, with negligible contributions from the IDL, as shown in Fig.~6(c). Therefore, we would conclude that Ca electrodeposition is defined by inner sphere processes in this electrolyte. And, surprisingly, that the presence of a thin SEI layer may reduce the required (thermodynamic) overpotential for electrodeposition, indicating the importance of activating such electrodes during the first charge.


\section*{Discussion}

Based on our analysis of the Ca(BH$_4$)$_2|$THF electrolyte and its interfacial speciation, we can propose the following phenomenology that may explain existing observation and provide guidance and interpretation of future characterization efforts. 

First and foremost, competition between the solvent, THF, and the anions, BH$_4^-$, for coordination of the dication, Ca$^{2+}$ leads to an electrolyte dominated by neutral species, mostly monomers, Ca(BH$_4$)$_2$, with an increasing prevalence of neutral dimers, Ca$_2$(BH$_4$)$_4$, with increasing concentration. The formation of charged species is facilitated by disproportionation of dimers rather than the more thermodynamically expensive direct dissociation of neutral monomers.

Strong solvent layering defines two interfacial regions within the first nanometer of the electrolyte: a dense layer inside 6~\AA{} and an intermediate density layer, from 7--10~\AA{}. These layers present mildly attractive or repulsive thermodynamic potentials that modulate the interfacial population of all species relative to the bulk electrolyte, even in the absence of electrode biasing. With increasing concentration, we see enhancements in the populations of various species in these interfacial zones, which has important consequences both for the availability of these species for electrochemical processes (the monocation CaBH$_4^+$) and their replenishment (via dimer disproportionation).

Overall, this is a poor electrolyte, with very low concentrations of charged species in bulk solution. However, it is ``activated'' at biased electrodes within this narrow interfacial region, wherein large relative populations of charged species emerge, especially with increasing concentration.
Therefore, meaningful characterization of the electrochemical activity of this electrolyte requires operando measurements that are sensitive to within a nanometer of the interface. The rich isomer subpopulations with distinct orientations in the IDL make it an excellent playground for interfacially-sensitive, polarization-dependent spectroscopies that can capture these differences. Furthermore, the bias-dependent switch in local population from oriented dimers to monocations should be observable with chemically-sensitive vibrational~\cite{Lu2019, Yang2022, He2022} and electronic probes.~\cite{velasco2014, Wu2018, Prabhakaran2023}

With increasing negative potentials up to and exceeding the thermodynamic potential for Ca deposition, we find that the pristine electrode may require a significant overpotential to register a deposition current. However, the presence of a thin SEI (introduced here as a dielectric spacer) can reduce this overpotential significantly, due to decreased electrostatic stabilization of charged species.  
This may explain the observation in previous work, that electrochemical activity, viz.  plating of Ca using this electrolyte carries a first, short duration overpotential of $\sim$250~mV, followed by a $\sim$100~mV overpotential in subsequent cycles.~\cite{Wang2018}

In addition, we find strong dependence in the electron transfer rate on the overall bulk concentration, primarily due to a decrease in the overpotential required for each species as its relative concentration increases (as seen in the Nernst equation).

Strong solvent and anion coordination of electrochemically active species, which has dominated our analysis, is very likely the source of solid-electrolyte interphase (SEI) formation due to electroreduction and decomposition of these ligands at the interface, as highlighted recently through electron microscopy.{~\cite{McClary2022}} Similarly, for other multivalent ion electrolytes,  we know, TFSI is both inherently and electrochemically unstable, for Mg~\cite{Yu2017} and for Ca.~\cite{hahn2022} 

Increased electroreductive stability in large anions may be afforded by considering non-cation-coordinating species. For example, closoboranes have been studied with Mg in tetraglyme.~\cite{Jay2019} These bulkier anions may also lead to better electrolytes overall (for example, preventing the formation of dimers), readily producing charged electroactive species. Although, strong solvent coordination, as we have seen for the fully-solvated Ca$^{2+}$, may still lead to significant overpotentials for electrodeposition and associated solvent decomposition and SEI growth.

From the computational perspective, we have highlighted the value of free-energy exploration, through combined molecular dynamics and continuum modeling, and unsupervised learning to reveal the complexity of this nominally simple electrolyte and tried to connect our simulations to observed electrochemical behavior and characterization. However, as in all theoretical models, our study has some inherent limitations. The complexity of the system and the time-scales required to explore different coordination complexes required the use of empirical force fields rather than ab initio methods. The finite number of dissolved species in our MD simulations mimics only low concentrations, which we extend to higher concentrations only at the level of our continuum model. Accurate inclusion of Debye screening from finite ionic concentrations, and dielectric screening due to solvent or anion polarizability{~\cite{Bedrov2019}} (which we approximate only by charge rescaling) should be considered in future studies.

The notable outcome of this study is a reinforcement of the notion that performant nonaqueous multivalent electrolytes (with high ionic conductivity and low overpotential) have competing requirements for multivalent ion coordination: to be strongly coordinating to keep the salt dissolved and the ionic conductivity high while not so strongly coordinating that the ion cannot break free from solvation during electrodeposition. The need for strong solvent coordination is driven by competition with favorable ionic bonds between counterions. If we want to maintain the advantages of earth-abundant, multivalent ions, then one option would be to switch to larger anions without specific coordinating moieties. However, this study highlights that there may be other options to consider that sideline the isolated multivalent ion altogether, which may even be irrelevant in terms of electrochemical activity. Firstly, that coordination with counterions may actually help bring active species closer to the electrode and that the strength of solvent coordination can dictate which species approaches the electrode closest. Secondly, that incomplete dissolution and the involvement of oligomers (dimers in this case) could be key to improved electrochemical activity albeit balanced by some power limitations due to reduced bulk ionic conductivity and the need to regenerate electroactive species through a disproportionation equilibrium. Clearly we have more work to do, both in terms of understanding the specifics of reduction of these clusters and their dissociation in the reduced state leading to Ca deposition; the potential negative side-effects of reductive instability of the coordinating species, already connected to SEI formation;~\cite{McClary2022} and the ultimate origins of measured overpotentials and currents in experiments. However, we have highlighted the importance of free energy sampling to attack such complex problems, even within relatively ideal conditions and contexts, and look forward to seeing more studies of this kind in the future.


\section*{Methods}

Metadynamics sampling (MTD) with a classical force-field was used to obtain free-energy landscapes as a function of $n$ collective variables. Equilibrium populations at critical points of a given landscape were then collected with Umbrella Sampling (US). Structural analysis of the US trajectories was then performed with a python-based unsupervised learning protocol. In order to explore the effects of finite concentration and bias potential at the electrode, a continuum model was developed within the modified Poisson-Boltzmann equation framework with inclusion of chemical equilibrium parameterized from the sampled free-energy differences. Finally, reduction potentials were calculated using the results of the continuum model and quantum-chemical calculations of the structures obtained from the unsupervised learning analysis. 

\paragraph{Free-energy sampling.} Metadynamics free-energy sampling~\cite{Laio2002} was carried out using the COLVARS module~\cite{colvars} implemented in LAMMPS.~\cite{lammps} Systems (see Table S1 for a full list) were generated using Packmol~\cite{martinez2009}. Concentration values were chosen to avoid forcing aggregation. Dimerization free energy surfaces show that at Ca-Ca distances larger than $\sim$ 10~\AA{}, the free energy converges with respect to Ca-Ca separation. That is the cutoff we consider between ``dissociated" and ``aggregated", giving an effective radius of 5~\AA{} for the first coordination shell of the dication. Additionally, g(r) shows that the second solvation shell (O-THF) settles at around 6~\AA{}. Assuming an optimal close-packing of ions, we deduce that solvent-separated ion pairs  would be unavoidable at concentrations above 0.5~M for a 6~\AA{} radius and 0.7~M for a 5~\AA{} radius. Hence, we selected concentrations $\leq$ 0.03M, well below these limits. System equilibration consisted of conjugate-gradient minimization to avoid steric clashes, followed by a short NVT warm-up to room temperature  (298 K) using a 1 fs timestep. Box-size equilibration was achieved by continuing the trajectory under NPT conditions at 1 atm with a 2 fs timestep. A final NVT step with the equilibrated lattice parameters (~ 20 ns) was sufficient to bring the systems to equilibrium. Force-field parameters~\cite{samba2009, andrade2002} were validated in our previous work on the same system.~\cite{Roncoroni2023} The graphene was frozen in place by setting the forces to zero in order to ensure neutral and charged simulations were comparable. These MD setup was kept for the MTD and US simulations. Metadynamics calculation parameters -- the width of the grid along a collective variable (W), height of the Gaussian 'hills' used to bias the potential (H) and the frequency at which they are added (Freq) -- can be found in SI Table S1 and were chosen to ensure convergence, namely, that the simulation reached  the diffusive regime in the given collective variable space.~\cite{baskin2021} Faster completion times were achieved by taking advantage of multiple-walker metadynamics, which allows to parallelize sampling among several trajectories (replicas) that update their biased potential at a given frequency (RepFreq) with the total biased potential. Despite this, the grid used in the three-dimensional MTD calculations was necessarily coarser. The minima explored here tend to be separated by more than 0.7~\AA{} (e.g., between the long and short dimer, or between solvent layers), which is larger than the coarser grid resolution. In order to keep cell neutrality and consistence with the neutral simulation, charge states 1 (CS1) and 2 (CS2) were generated by adding equal and opposite charges ($\pm$ 2 $\mu$C/cm$^2$) distributed evenly among two graphene layers, emulating a positively charged and a negatively charged electrode. 

Free-energy sampling was performed with combinations of the following collective variables: the distance between the calcium and the center of mass of the top graphene layer (dZ); the coordination number between the calcium and the boron atom in a given BH$_4$ (CN(Ca-B), r$_0$ = 3.8~\AA{}); and, to track dimerization, the distance between two calcium atoms (dCa) or a Ca-Ca coordination number (CN(Ca-Ca), r$_0$ = 6.5~\AA{}).

Note that the state free-energies in Fig.~1 b) were obtained by thermodynamic integration of our 2D free energy surface (Fig.~1 c) over regions delimited by given collective variable values  (e.g., for the Ca(BH$_4$)$_4^{2-}$ state, a Ca-Ca distance larger that 7~\AA{} and a CN(Ca-B) larger than 3.5).  Collective variables apply to only one of the calcium atoms, and hence the free energy is averaged over all (remaining) possible coordinations/distances  for the other atom. Hence, the room-temperature Boltzmann probability speaks of the likelihood of finding a state formed by the constrained species (e.g., Ca(BH$_4$)$_4^{2-}$) in the environment of the remaining, unconstrained species. Since our system contains two calciums and four borohydrides, the un-constrained space is different depending on the value of the collective variables. In the Ca(BH$_4$)$_4^{2-}$ basin, the un-constrained Ca can only be fully solvated. 
On the other hand, in the Ca${2+}$ basin, the other calcium can exist in five different ion coordination states (Ca$^{2+}$, Ca(BH$_4$)$^{+}$, Ca(BH$_4$)$_2$, Ca(BH$_4$)$_3^{-}$, and Ca(BH$_4$)$_4^{2-}$). This is the reason why no  symmetry is expected on the free energy surface along the coordination axis, e.g., between CN=0 and CN=4; and CN=1 and CN=3.


\paragraph{Population sampling} The structures and equilibrium population of points of interest in the free energy surface were collected using Umbrella Sampling. Initial structures were obtained from metadynamics trajectory snapshots at the desired point in the CV space. The collective variable coordinates were restrained with a harmonic potential centered at the desired value, with force constant 1/w$^2$ where w is the width of the collective variable grid (see SI Table S1). Then, trajectories of 150-200 ns were calculated under similar MD parameters as the final equilibration step and MTD run. 

\paragraph{Data analysis.} Umbrella sampling trajectories were analyzed with an unsupervised learning methodology recently developed by us.~\cite{Roncoroni2023} 
In this protocol, between 5000 and 10000 local atomic arrangements were extracted from each US trajectory, and aligned while taking into account possible permutations between similar elements (e.g. THF - O).~\cite{Gunde2022,Griffiths2017} 
Classification based on their structural similarity was carried out using dimensionality reduction~\cite{McInnes2018,Becht2019}  and clustering~\cite{McInnes2017} algorithms, in an ASE-compatible~\cite{Larsen_2017} environment. 
The $n$-dimensional free energy landscapes obtained from the MTD sampling were explored with a Jupyter-adapted version of the MEPSAnd module~\cite{Marcos-Alcalde2019} in order to find critical points and minimum energy pathways. 

\paragraph{Continuum Model.} The free-energy of a system formed by a finite concentration of multiple (charged) species by a (charged) electrode was minimized using a beyond Gouy-Chapman model based on a modified Poisson-Boltzmann equation.{\cite{Baskin_2017, Wu2018, Baskin_2019_JCP}} This continuum model accounts for the electrostatic potential of the electrode as well as each species' specific adsorption (the so-called Frumkin effects),{\cite{Bard2022, White2014, Willard2020, PEndergast_2023}}, chemical potential and entropic (volume exclusion) effects, and chemical re-equilibration between species; parameterised from free-energy sampling results.

\paragraph{Electron Transfer Rate Estimates.} The electron transfer rate estimate $I$ was obtained by integrating the individual contributions of isomers over the interface:

\begin{equation}
    I \propto \sum_{s} \int_{z} dz  C_{s} (z) e^{-z/z_0} e^{-\Delta G_s/ k_B T}
\end{equation}
 where $C_{s}(z)$, each isomer's concentration profile, was obtained by renormalizing the species' concentration profiles of the continuum model  with the US/UL isomer layer populations. $z_0$ is the electron tunneling decay length, set to 1~nm here. The reduction potential for a given isomer, $\Delta G_s$, was obtained using Density Functional Theory (see SI Section 2).

 \section*{Data availability}
Additional computational details and free-energy information referenced in the text can be found in the Supporting Information. Source data for Figures 1-6 are provided with this paper in the Supplementary Data file.. The raw datasets generated and analysed during the current study are also available from the corresponding author on reasonable request. 

\section*{Code availability}

The continuum model code is available upon request to the corresponding author. 

\begin{filecontents}{references.bib}

@article{Roncoroni2023,
author ="Roncoroni, Fabrice and Sanz-Matias, Ana and Sundararaman, Siddharth and Prendergast, David",
title  ="Unsupervised learning of representative local atomic arrangements in molecular dynamics data",
journal  ="Phys. Chem. Chem. Phys.",
year  ="2023",
volume  ="25",
issue  ="19",
pages  ="13741-13754",
publisher  ="The Royal Society of Chemistry",
ref_doi  ="10.1039/D3CP00525A",
ref_url  ="http://dx.doi.org/10.1039/D3CP00525A",
}

@Article{Yang2022,
author={Yang, Shanshan
and Zhao, Xiao
and Lu, Yi-Hsien
and Barnard, Edward S.
and Yang, Peidong
and Baskin, Artem
and Lawson, John W.
and Prendergast, David
and Salmeron, Miquel},
title={Nature of the Electrical Double Layer on Suspended Graphene Electrodes},
journal={Journal of the American Chemical Society},
year={2022},
month={Jul},
day={27},
publisher={American Chemical Society},
volume={144},
number={29},
pages={13327-13333},
ref_note={doi: 10.1021/jacs.2c03344},
issn={0002-7863},
ref_url={https://doi.org/10.1021/jacs.2c03344}
}


@Article{Lu2019,
author={Lu, Yi-Hsien
and Larson, Jonathan M.
and Baskin, Artem
and Zhao, Xiao
and Ashby, Paul D.
and Prendergast, David
and Bechtel, Hans A.
and Kostecki, Robert
and Salmeron, Miquel},
title={Infrared Nanospectroscopy at the Graphene-Electrolyte Interface},
journal={Nano Letters},
year={2019},
month={Aug},
day={14},
publisher={American Chemical Society},
volume={19},
number={8},
pages={5388-5393},
ref_note={doi: 10.1021/acs.nanolett.9b01897},
issn={1530-6984},
ref_url={https://doi.org/10.1021/acs.nanolett.9b01897}
}

@article{Henderson1976,
author = { Douglas   Henderson  and  Farid F.   Abraham  and  John A.   Barker },
title = {The Ornstein-Zernike equation for a fluid in contact with a surface},
journal = {Molecular Physics},
volume = {31},
number = {4},
pages = {1291-1295},
year  = {1976},
publisher = {Taylor & Francis},
ref_doi = {10.1080/00268977600101021},

ref_url = { 
    
        https://doi.org/10.1080/00268977600101021
    
    

},
ref_eprint = { 
    
        https://doi.org/10.1080/00268977600101021
    
    

}

}



@article{Perdew1996,
  title = {Generalized Gradient Approximation Made Simple},
  author = {Perdew, John P. and Burke, Kieron and Ernzerhof, Matthias},
  journal = {Phys. Rev. Lett.},
  volume = {77},
  issue = {18},
  pages = {3865--3868},
  numpages = {0},
  year = {1996},
  month = {Oct},
  publisher = {American Physical Society},
  ref_doi = {10.1103/PhysRevLett.77.3865},
  ref_url = {https://link.aps.org/doi/10.1103/PhysRevLett.77.3865}
}


@article{Soler2002,                                                          
ref_doi = {10.1088/0953-8984/14/11/302},
ref_url = {https://dx.doi.org/10.1088/0953-8984/14/11/302},
year = {2002},
month = {mar},
publisher = {},
volume = {14},
number = {11},
pages = {2745},
author = {José M Soler and  Emilio Artacho and  Julian D Gale and  Alberto García and  Javier Junquera and  Pablo Ordejón and  Daniel Sánchez-Portal},
title = {The SIESTA method for ab initio order-N
materials simulation},
journal = {Journal of Physics: Condensed Matter},
}


@article{Otani2006,
  title = {First-principles calculations of charged surfaces and interfaces: A plane-wave nonrepeated slab approach},
  author = {Otani, M. and Sugino, O.},
  journal = {Phys. Rev. B},
  volume = {73},
  issue = {11},
  pages = {115407},
  numpages = {11},
  year = {2006},
  month = {Mar},
  publisher = {American Physical Society},
  ref_doi = {10.1103/PhysRevB.73.115407},
  ref_url = {https://link.aps.org/doi/10.1103/PhysRevB.73.115407}
}


@article{Hamada2013,
  title = {Improved modeling of electrified interfaces using the effective screening medium method},
  author = {Hamada, Ikutaro and Sugino, Osamu and Bonnet, Nic\'ephore and Otani, Minoru},
  journal = {Phys. Rev. B},
  volume = {88},
  issue = {15},
  pages = {155427},
  numpages = {4},
  year = {2013},
  month = {Oct},
  publisher = {American Physical Society},
  ref_doi = {10.1103/PhysRevB.88.155427},
  ref_url = {https://link.aps.org/doi/10.1103/PhysRevB.88.155427}
}


@article{lammps,
title = {LAMMPS - a flexible simulation tool for particle-based materials modeling at the atomic, meso, and continuum scales},
journal = {Computer Physics Communications},
volume = {271},
pages = {108171},
year = {2022},
issn = {0010-4655},
ref_doi = {https://doi.org/10.1016/j.cpc.2021.108171},
ref_url = {https://www.sciencedirect.com/science/article/pii/S0010465521002836},
author = {Aidan P. Thompson and H. Metin Aktulga and Richard Berger and Dan S. Bolintineanu and W. Michael Brown and Paul S. Crozier and Pieter J. {in 't Veld} and Axel Kohlmeyer and Stan G. Moore and Trung Dac Nguyen and Ray Shan and Mark J. Stevens and Julien Tranchida and Christian Trott and Steven J. Plimpton},
}



@article{colvars,
author = {Giacomo Fiorin and Michael L. Klein and Jérôme Hénin},
title = {Using collective variables to drive molecular dynamics simulations},
journal = {Molecular Physics},
volume = {111},
number = {22-23},
pages = {3345-3362},
year  = {2013},
publisher = {Taylor & Francis},
ref_doi = {10.1080/00268976.2013.813594},

ref_url = { 
    
        https://doi.org/10.1080/00268976.2013.813594
    
    

},
ref_eprint = { 
    
        https://doi.org/10.1080/00268976.2013.813594
    
    

}

}

@article{Larsen_2017,
ref_doi = {10.1088/1361-648X/aa680e},
ref_url = {https://dx.doi.org/10.1088/1361-648X/aa680e},
year = {2017},
month = {jun},
publisher = {IOP Publishing},
volume = {29},
number = {27},
pages = {273002},
author = {Ask Hjorth Larsen and Jens Jørgen Mortensen and Jakob Blomqvist and Ivano E Castelli and Rune Christensen and Marcin Dułak and Jesper Friis and Michael N Groves and Bjørk Hammer and Cory Hargus and Eric D Hermes and Paul C Jennings and Peter Bjerre Jensen and James Kermode and John R Kitchin and Esben Leonhard Kolsbjerg and Joseph Kubal and Kristen Kaasbjerg and Steen Lysgaard and Jón Bergmann Maronsson and Tristan Maxson and Thomas Olsen and Lars Pastewka and Andrew Peterson and Carsten Rostgaard and Jakob Schiøtz and Ole Schütt and Mikkel Strange and Kristian S Thygesen and Tejs Vegge and Lasse Vilhelmsen and Michael Walter and Zhenhua Zeng and Karsten W Jacobsen},
title = {The atomic simulation environment—a Python library for working with atoms},
journal = {Journal of Physics: Condensed Matter},
abstract = {The atomic simulation environment (ASE) is a software package written in the Python programming language with the aim of setting up, steering, and analyzing atomistic simulations. In ASE, tasks are fully scripted in Python. The powerful syntax of Python combined with the NumPy array library make it possible to perform very complex simulation tasks. For example, a sequence of calculations may be performed with the use of a simple ‘for-loop’ construction. Calculations of energy, forces, stresses and other quantities are performed through interfaces to many external electronic structure codes or force fields using a uniform interface. On top of this calculator interface, ASE provides modules for performing many standard simulation tasks such as structure optimization, molecular dynamics, handling of constraints and performing nudged elastic band calculations.}}


@Article{Qiao2010,
author ="Feng, Guang and Huang, Jingsong and Sumpter, Bobby G. and Meunier, Vincent and Qiao, Rui",
title  ="Structure and dynamics of electrical double layers in organic electrolytes",
journal  ="Phys. Chem. Chem. Phys.",
year  ="2010",
volume  ="12",
issue  ="20",
pages  ="5468-5479",
publisher  ="The Royal Society of Chemistry",
doi  ="10.1039/C000451K",
ref_url  ="http://dx.doi.org/10.1039/C000451K",
abstract  ="The organic electrolyte of tetraethylammonium tetrafluoroborate (TEABF4) in the aprotic solvent of acetonitrile (ACN) is widely used in electrochemical systems such as electrochemical capacitors. In this paper{,} we examine the solvation of TEA+ and BF4− in ACN{,} and the structure{,} capacitance{,} and dynamics of the electrical double layers (EDLs) in the TEABF4–ACN electrolyte using molecular dynamics simulations complemented with quantum density functional theory calculations. The solvation of TEA+ and BF4− ions is found to be much weaker than that of small inorganic ions in aqueous solutions{,} and the ACN molecules in the solvation shell of both types of ions show only weak packing and orientational ordering. These solvation characteristics are caused by the large size{,} charge delocalization{,} and irregular shape (in the case of TEA+ cation) of the ions. Near neutral electrodes{,} the double-layer structure in the organic electrolyte exhibits a rich organization: the solvent shows strong layering and orientational ordering{,} ions are significantly contact-adsorbed on the electrode{,} and alternating layers of cations/anions penetrate ca. 1.1 nm into the bulk electrolyte. The significant contact adsorption of ions and the alternating layering of cation/anion are new features found for EDLs in organic electrolytes. These features essentially originate from the fact that van der Waals interactions between organic ions and the electrode are strong and the partial desolvation of these ions occurs easily{,} as a result of the large size of the organic ions. Near charged electrodes{,} distinct counter-ion concentration peaks form{,} and the ion distribution cannot be described by the Helmholtz model or the Helmholtz + Poisson–Boltzmann model. This is because the number of counter-ions adsorbed on the electrode exceeds the number of electrons on the electrode{,} and the electrode is over-screened in parts of the EDL. The computed capacitances of the EDLs are in good agreement with that inferred from experimental measurements. Both the rotations (ACN only) and translations of interfacial ACN and ions are found to slow down as the electrode is electrified. We also observe an asymmetrical dependence of these motions on the sign of the electrode charge. The rotation/diffusion of ACN and the diffusion of ions in the region beyond the first ACN or ion layer differ only weakly from those in the bulk."}


@article{Marcos-Alcalde2019,
    author = {Marcos-Alcalde, Iñigo and López-Viñas, Eduardo and Gómez-Puertas, Paulino},
    title = "{MEPSAnd: minimum energy path surface analysis over n-dimensional surfaces}",
    journal = {Bioinformatics},
    volume = {36},
    number = {3},
    pages = {956-958},
    year = {2019},
    month = {08},
    abstract = "{n-dimensional energy surfaces are becoming computationally accessible, yet interpreting their information is not straightforward. We present minimum energy path surface analysis over n-dimensional surfaces (MEPSAnd), an open source GUI-based program that natively calculates minimum energy paths across energy surfaces of any number of dimensions. Among other features, MEPSAnd can compute the path through lowest barriers and automatically provide a set of alternative paths. MEPSAnd offers distinct plotting solutions as well as direct python scripting.MEPSAnd is freely available (under GPLv3 license) at: http://bioweb.cbm.uam.es/software/MEPSAnd/.Supplementary data are available at Bioinformatics online.}",
    issn = {1367-4803},
    ref_doi = {10.1093/bioinformatics/btz649},
    ref_url = {https://doi.org/10.1093/bioinformatics/btz649},
    ref_eprint = {https://academic.oup.com/bioinformatics/article-pdf/36/3/956/32369715/btz649.pdf},
}


@Article{He2022,
author={He, Xin
and Larson, Jonathan M.
and Bechtel, Hans A.
and Kostecki, Robert},
title={In situ infrared nanospectroscopy of the local processes at the \ce{Li}/polymer electrolyte interface},
journal={Nature Communications},
year={2022},
volume={13},
number={1},
pages={1398},
abstract={Solid-state batteries possess the potential to significantly impact energy storage industries by enabling diverse benefits, such as increased safety and energy density. However, challenges persist with physicochemical properties and processes at electrode/electrolyte interfaces. Thus, there is great need to characterize such interfaces in situ, and unveil scientific understanding that catalyzes engineering solutions. To address this, we conduct multiscale in situ microscopies (optical, atomic force, and infrared near-field) and Fourier transform infrared spectroscopies (near-field nanospectroscopy and attenuated total reflection) of intact and electrochemically operational graphene/solid polymer electrolyte interfaces. We find nanoscale structural and chemical heterogeneities intrinsic to the solid polymer electrolyte initiate a cascade of additional interfacial nanoscale heterogeneities during Li plating and stripping; including Li-ion conductivity, electrolyte decomposition, and interphase formation. Moreover, our methodology to nondestructively characterize buried interfaces and interphases in their native environment with nanoscale resolution is readily adaptable to a number of other electrochemical systems and battery chemistries.},
issn={2041-1723},
ref_url={https://doi.org/10.1038/s41467-022-29103-z}
}



@article{velasco2014,
author = {Juan-Jesus Velasco-Velez  and Cheng Hao Wu  and Tod A. Pascal  and Liwen F. Wan  and Jinghua Guo  and David Prendergast  and Miquel Salmeron },
title = {The structure of interfacial water on gold electrodes studied by x-ray absorption spectroscopy},
journal = {Science},
volume = {346},
number = {6211},
pages = {831-834},
year = {2014},
ref_doi = {10.1126/science.1259437},
ref_url = {https://www.science.org/doi/abs/10.1126/science.1259437},
ref_eprint = {https://www.science.org/doi/pdf/10.1126/science.1259437},
abstract = {The molecular structure of the electrical double layer determines the chemistry in all electrochemical processes. Using x-ray absorption spectroscopy (XAS), we probed the structure of water near gold electrodes and its bias dependence. Electron yield XAS detected at the gold electrode revealed that the interfacial water molecules have a different structure from those in the bulk. First principles calculations revealed that ~50\% of the molecules lie flat on the surface with saturated hydrogen bonds and another substantial fraction with broken hydrogen bonds that do not contribute to the XAS spectrum because their core-excited states are delocalized by coupling with the gold substrate. At negative bias, the population of flat-lying molecules with broken hydrogen bonds increases, producing a spectrum similar to that of bulk water. The water double-layer structure at an electrode changed from ordered to disordered when the applied bias was switched. The structure of water within a nanometer of an electrode surface is known as the electrical double layer. This layer creates a strong electrical field that can affect electrochemical reactions. Velasco-Velez et al. explored the structure of the electrical double layer at a bare gold electrode. With no applied potential and at positive potentials, the layer is highly structured (resembling ice) with few dangling hydrogen bonds. However, at negative potentials, the layer was more like bulk water, and half of the water molecules lie flat on the surface. Science, this issue p. 831}}

@Article{Wu2018,
author={Wu, Cheng Hao
and Pascal, Tod A.
and Baskin, Artem
and Wang, Huixin
and Fang, Hai-Tao
and Liu, Yi-Sheng
and Lu, Yi-Hsien
and Guo, Jinghua
and Prendergast, David
and Salmeron, Miquel B.},
title={Molecular-Scale Structure of Electrode-Electrolyte Interfaces: The Case of Platinum in Aqueous Sulfuric Acid},
journal={Journal of the American Chemical Society},
year={2018},
month={Nov},
day={28},
publisher={American Chemical Society},
volume={140},
number={47},
pages={16237-16244},
ref_note={doi: 10.1021/jacs.8b09743},
issn={0002-7863},
ref_url={https://doi.org/10.1021/jacs.8b09743}
}



@Article{Prabhakaran2023,
author={Prabhakaran, Venkateshkumar
and Agarwal, Garvit
and Howard, Jason D.
and Wi, Sungun
and Shutthanandan, Vaithiyalingam
and Nguyen, Dan-Thien
and Soule, Luke
and Johnson, Grant E.
and Liu, Yi-Sheng
and Yang, Feipeng
and Feng, Xuefei
and Guo, Jinghua
and Hankins, Kie
and Curtiss, Larry A.
and Mueller, Karl T.
and Assary, Rajeev S.
and Murugesan, Vijayakumar},
title={Coordination-Dependent Chemical Reactivity of \ce{TFSI} Anions at a \ce{Mg} Metal Interface},
journal={ACS Applied Materials \& Interfaces},
year={2023},
month={Feb},
day={08},
publisher={American Chemical Society},
volume={15},
number={5},
pages={7518-7528},
ref_note={doi: 10.1021/acsami.2c18477},
issn={1944-8244},
ref_url={https://doi.org/10.1021/acsami.2c18477}
}

%%%For aggregates, not sure when to cite yet

@article{samuel2017,
author = {Samuel, Devon and Steinhauser, Carl and Smith, Jeffrey G. and Kaufman, Aaron and Radin, Maxwell D. and Naruse, Junichi and Hiramatsu, Hidehiko and Siegel, Donald J.},
title = {Ion Pairing and Diffusion in Magnesium Electrolytes Based on Magnesium Borohydride},
journal = {ACS Applied Materials \& Interfaces},
volume = {9},
number = {50},
pages = {43755-43766},
year = {2017},
ref_doi = {10.1021/acsami.7b15547},
    ref_note ={PMID: 29134805},

ref_url = { 
        https://doi.org/10.1021/acsami.7b15547
    
},
ref_eprint = { 
        https://doi.org/10.1021/acsami.7b15547
    
}

}


@article{martinez2009,
author = {Martinez, L. and Andrade, R. and Birgin, E. G. and Martinez, J. M.},
title = {PACKMOL: A package for building initial configurations for molecular dynamics simulations},
journal = {Journal of Computational Chemistry},
volume = {30},
number = {13},
pages = {2157-2164},
keywords = {initial configuration, molecular dynamics, packing, large-scale optimization, Packmol},
ref_doi = {https://doi.org/10.1002/jcc.21224},
ref_url = {https://onlinelibrary.wiley.com/doi/abs/10.1002/jcc.21224},

abstract = {Abstract Adequate initial configurations for molecular dynamics simulations consist of arrangements of molecules distributed in space in such a way to approximately represent the system's overall structure. In order that the simulations are not disrupted by large van der Waals repulsive interactions, atoms from different molecules must keep safe pairwise distances. Obtaining such a molecular arrangement can be considered a packing problem: Each type molecule must satisfy spatial constraints related to the geometry of the system, and the distance between atoms of different molecules must be greater than some specified tolerance. We have developed a code able to pack millions of atoms, grouped in arbitrarily complex molecules, inside a variety of three-dimensional regions. The regions may be intersections of spheres, ellipses, cylinders, planes, or boxes. The user must provide only the structure of one molecule of each type and the geometrical constraints that each type of molecule must satisfy. Building complex mixtures, interfaces, solvating biomolecules in water, other solvents, or mixtures of solvents, is straightforward. In addition, different atoms belonging to the same molecule may also be restricted to different spatial regions, in such a way that more ordered molecular arrangements can be built, as micelles, lipid double-layers, etc. The packing time for state-of-the-art molecular dynamics systems varies from a few seconds to a few minutes in a personal computer. The input files are simple and currently compatible with PDB, Tinker, Molden, or Moldy coordinate files. The package is distributed as free software and can be downloaded from http://www.ime.unicamp.br/∼martinez/packmol/. © 2009 Wiley Periodicals, Inc. J Comput Chem, 2009},
year = {2009}
}


@misc{McInnes2018,
      title={\ce{UMAP}: Uniform Manifold Approximation and Projection for Dimension Reduction}, 
      author={Leland McInnes and John Healy and James Melville},
      year={2020},
      eprint={https://arxiv.org/abs/1802.03426},
      archivePrefix={arXiv},
      primaryClass={stat.ML}
}

@article{Laio2002,

author = {Alessandro Laio  and Michele Parrinello },
title = {Escaping free-energy minima},
journal = {Proceedings of the National Academy of Sciences},
volume = {99},
number = {20},
pages = {12562-12566},
year = {2002},
ref_doi = {10.1073/pnas.202427399},
ref_url = {https://www.pnas.org/doi/abs/10.1073/pnas.202427399},
ref_eprint = {https://www.pnas.org/doi/pdf/10.1073/pnas.202427399},
abstract = {We introduce a powerful method for exploring the properties of the multidimensional free energy surfaces (FESs) of complex many-body systems by means of coarse-grained non-Markovian dynamics in the space defined by a few collective coordinates. A characteristic feature of these dynamics is the presence of a history-dependent potential term that, in time, fills the minima in the FES, allowing the efficient exploration and accurate determination of the FES as a function of the collective coordinates. We demonstrate the usefulness of this approach in the case of the dissociation of a NaCl molecule in water and in the study of the conformational changes of a dialanine in solution.}}


@book{HJButt2006,

title= "Physics and Chemistry of Interfaces",
author = {Hans-Jürgen Butt and Karlheinz Graf and  Michael Kappl},
isbn = { 978-3-527-40629-6},
year = " 2006 ",
publisher = " Wiley-VCH",
address = "Germany",

pages = {45,44,56,76-83},

}


@Article{Becht2019,
author={Becht, Etienne
and McInnes, Leland
and Healy, John
and Dutertre, Charles-Antoine
and Kwok, Immanuel W. H.
and Ng, Lai Guan
and Ginhoux, Florent
and Newell, Evan W.},
title={Dimensionality reduction for visualizing single-cell data using UMAP},
journal={Nature Biotechnology},
year={2019},
volume={37},
number={1},
pages={38-44},
abstract={A benchmarking analysis on single-cell RNA-seq and mass cytometry data reveals the best-performing technique for dimensionality reduction.},
issn={1546-1696},
ref_url={https://doi.org/10.1038/nbt.4314}
}
@article{Gunde2022,
title = {Iterative Rotations and Assignments (IRA): A shape matching algorithm for atomic structures},
journal = {Software Impacts},
volume = {12},
pages = {100264},
year = {2022},
issn = {2665-9638},
ref_doi = {https://doi.org/10.1016/j.simpa.2022.100264},
ref_url = {https://www.sciencedirect.com/science/article/pii/S2665963822000240},
author = {Miha Gunde and Nicolas Salles and Anne Hémeryck and Layla Martin-Samos},
keywords = {Shape matching, Structural superposition, Linear assignment problem, Pattern recognition, Off-lattice kinetic Monte Carlo},
abstract = {IRA is a Fortran library that solves the shape matching problem for atomic structures, stored as sets of points representing the atomic positions. In the case of exact- and near-congruence, IRA provides the optimal rigid transformation between the structures, given by the atomic assignments, the rotation/reflection matrix, and the translation vector. IRA is also able to operate on structures containing a non-equal number of atoms, i.e. matching a structure to any of its fragments. Any application that requires the solution of a shape matching problem could benefit from IRA.}
}




@article{Griffiths2017,
author = {Griffiths, Matthew and Niblett, Samuel P. and Wales, David J.},
title = {Optimal Alignment of Structures for Finite and Periodic Systems},
journal = {Journal of Chemical Theory and Computation},
volume = {13},
number = {10},
pages = {4914-4931},
year = {2017},
ref_doi = {10.1021/acs.jctc.7b00543},
    note ={PMID: 28841314},

ref_url = { 
        https://doi.org/10.1021/acs.jctc.7b00543
    
},
ref_eprint = { 
        https://doi.org/10.1021/acs.jctc.7b00543   
}
}


@article{McInnes2017, 
ref_doi = {10.21105/joss.00205}, ref_url = {https://doi.org/10.21105/joss.00205}, 
year = {2017}, 
publisher = {The Open Journal}, 
volume = {2}, number = {11}, 
pages = {205}, 
author = {Leland McInnes and John Healy and Steve Astels}, 
title = {hdbscan: Hierarchical density based clustering}, 
journal = {Journal of Open Source Software} } 



%%%%%%%%%%%%%%%%%%%%%%%%%%%%%%%%%%%%%%%%%%%%%%



@Article{Arroyo-deDompablo2020,
author={Arroyo~de~Dompablo, M. Elena
and Ponrouch, Alexandre
and Johansson, Patrik
and Palac\'{i}n, M. Rosa},
title={Achievements, Challenges, and Prospects of Calcium Batteries},
journal={Chemical Reviews},
year={2020},
month={Jul},
day={22},
publisher={American Chemical Society},
volume={120},
number={14},
pages={6331-6357},
ref_note={doi: 10.1021/acs.chemrev.9b00339},
issn={0009-2665},
ref_url={https://doi.org/10.1021/acs.chemrev.9b00339}
}


@article{Mohtadi2021,
title = {The metamorphosis of rechargeable magnesium batteries},
journal = {Joule},
volume = {5},
number = {3},
pages = {581-617},
year = {2021},
issn = {2542-4351},
ref_doi = {https://doi.org/10.1016/j.joule.2020.12.021},
ref_url = {https://www.sciencedirect.com/science/article/pii/S2542435120306206},
author = {Rana Mohtadi and Oscar Tutusaus and Timothy S. Arthur and Zhirong Zhao-Karger and Maximilian Fichtner},
keywords = {energy storage, beyond Li-ion battery, magnesium},
abstract = {Summary
The need for energy storage technologies that meet the demands for safety, sustainability, and high energy density has spurred increased interests in rechargeable Mg batteries. However, verification of their potential remains hampered by the absence of practical components. Recently, an unconventional research direction that deviates from the mainstream path has emerged and is redefining the boundaries of what can be accomplished in Mg batteries. Herein, we analyze this direction with a focus on fundamental phenomena with an aim to reveal new opportunities that can overcome long-standing challenges and unveil unaddressed issues. The topics covered encompass: (1) realizing efficient single salt Mg electrolytes that mimic those used in typical Li- and Na-ion batteries, (2) discovery of important concomitant mechanisms and previously unanticipated interfacial phenomena, and (3) identification of often overlooked bottlenecks in Mg cathodes and alternative anodes. In closing, we put forward our proposal for R&D focuses to help realize practical Mg batteries.}}


@Article{Ponrouch2016,
author={Ponrouch, A.
and Frontera, C.
and Barde\`{e}, F.
and Palac\'{i}n, M. R.},
title={Towards a calcium-based rechargeable battery},
journal={Nature Materials},
year={2016},
volume={15},
number={2},
pages={169-172},
abstract={Although rechargeable batteries that use light electropositive metal anodes are attractive, electrodeposition of calcium has proved difficult. Calcium plating at moderate temperatures using conventional organic electrolytes has now been demonstrated.},
issn={1476-4660},
ref_url={https://doi.org/10.1038/nmat4462}
}

@Article{Ufimtsev2008,
author={Ufimtsev, Ivan S.
and Martinez, Todd J.},
title={Quantum Chemistry on Graphical Processing Units. 1. Strategies for Two-Electron Integral Evaluation},
journal={Journal of Chemical Theory and Computation},
year={2008},
month={Feb},
day={01},
publisher={American Chemical Society},
volume={4},
number={2},
pages={222-231},
ref_note={doi: 10.1021/ct700268q},
issn={1549-9618},
ref_url={https://doi.org/10.1021/ct700268q}
}










@article{grimme2010,
author = {Grimme,Stefan  and Antony,Jens  and Ehrlich,Stephan  and Krieg,Helge },
title = {A consistent and accurate ab initio parametrization of density functional dispersion correction (DFT-D) for the 94 elements H-Pu},
journal = {The Journal of Chemical Physics},
volume = {132},
number = {15},
pages = {154104},
year = {2010},
ref_doi = {10.1063/1.3382344},

ref_url = { 
        https://doi.org/10.1063/1.3382344
    
},
ref_eprint = { 
        https://doi.org/10.1063/1.3382344
    
}

}



@article{b3lyp,
author = {Stephens, P. J. and Devlin, F. J. and Chabalowski, C. F. and Frisch, M. J.},
title = {Ab Initio Calculation of Vibrational Absorption and Circular Dichroism Spectra Using Density Functional Force Fields},
journal = {The Journal of Physical Chemistry},
volume = {98},
number = {45},
pages = {11623-11627},
year = {1994},
ref_doi = {10.1021/j100096a001},

ref_url = { 
        https://doi.org/10.1021/j100096a001
    
},
ref_eprint = { 
        https://doi.org/10.1021/j100096a001
    
}

}


@article{forer-saboya2022,
author = {Forero-Saboya, Juan D. and Tchitchekova, Deyana S. and Johansson, Patrik and Palac\'{i}n, M. Rosa and Ponrouch, Alexandre},
title = {Interfaces and Interphases in \ce{Ca} and \ce{Mg} Batteries},
journal = {Advanced Materials Interfaces},
volume = {9},
number = {8},
pages = {2101578},
keywords = {calcium and magnesium batteries, interface, plating, SEI, solvation},
ref_doi = {https://doi.org/10.1002/admi.202101578},
ref_url = {https://onlinelibrary.wiley.com/doi/abs/10.1002/admi.202101578},

abstract = {Abstract The development of high energy density battery technologies based on divalent metals as the negative electrode is very appealing. Ca and Mg are especially interesting choices due to their combination of low standard reduction potential and natural abundance. One particular problem stalling the technological development of these batteries is the low efficiency of plating/stripping at the negative electrode, which relates to several factors that have not yet been looked at systematically; the nature/concentration of the electrolyte, which determines the mass transport of electro-active species (cation complexes) toward the electrode; the possible presence of passivation layers, which may hinder ionic transport and hence limit electrodeposition; and the mechanisms behind the charge transfer leading to nucleation/growth of the metal. Different electrolytes are investigated for Mg and Ca, with the presence/absence of chlorides in the formulation playing a crucial role in the cation desolvation. From a R\&D point-of-view, proper characterization alongside modeling is crucial to understand the phenomena determining the mechanisms of the plating/stripping processes. The state-of-the-art is here presented together with a short perspective on the influence of the cation solvation also on the positive electrode and finally an attempt to define guidelines for future research in the field.},
year = {2022}
}

@Article{Hahn2021,
author={Hahn, Nathan T.
and Self, Julian
and Han, Kee Sung
and Murugesan, Vijayakumar
and Mueller, Karl T.
and Persson, Kristin A.
and Zavadil, Kevin R.},
title={Quantifying Species Populations in Multivalent Borohydride Electrolytes},
journal={The Journal of Physical Chemistry B},
year={2021},
month={Apr},
day={15},
publisher={American Chemical Society},
volume={125},
number={14},
pages={3644-3652},
ref_note={doi: 10.1021/acs.jpcb.1c00263},
issn={1520-6106},
ref_url={https://doi.org/10.1021/acs.jpcb.1c00263}
}



@article{Araujo2021,
title = {Towards novel calcium battery electrolytes by efficient computational screening},
journal = {Energy Storage Materials},
volume = {39},
pages = {89-95},
year = {2021},
issn = {2405-8297},
ref_doi = {https://doi.org/10.1016/j.ensm.2021.04.015},
ref_url = {https://www.sciencedirect.com/science/article/pii/S2405829721001574},
author = {Rafael B. Araujo and Vigneshwaran Thangavel and Patrik Johansson},
keywords = {Calcium batteries, Electrolyte, Ca-salt solubility, Screening strategy, COSMO-RS},

}

@Article{Baskin2019,
author={Baskin, Artem and Prendergast, David},
title={`Ion Solvation Spectra': Free Energy Analysis of Solvation Structures of Multivalent Cations in Aprotic Solvents},
journal={The Journal of Physical Chemistry Letters},
year={2019},
month={Sep},
day={05},
publisher={American Chemical Society},
volume={10},
number={17},
pages={4920-4928},
ref_note={doi: 10.1021/acs.jpclett.9b01569},
ref_url={https://doi.org/10.1021/acs.jpclett.9b01569}
}


@article{baskin2020,
author = {Baskin, Artem and Prendergast, David},
title = {Ion Solvation Engineering: How to Manipulate the Multiplicity of the Coordination Environment of Multivalent Ions},
journal = {The Journal of Physical Chemistry Letters},
volume = {11},
number = {21},
pages = {9336-9343},
year = {2020},
ref_doi = {10.1021/acs.jpclett.0c02682},
    ref_note ={PMID: 33090799},

ref_url = { 
        https://doi.org/10.1021/acs.jpclett.0c02682
    
},
ref_eprint = { 
        https://doi.org/10.1021/acs.jpclett.0c02682
    
}

}

@article{baskin2021,
author = {Baskin, Artem and Lawson, John W. and Prendergast, David},
title = {Anion-Assisted Delivery of Multivalent Cations to Inert Electrodes},
journal = {The Journal of Physical Chemistry Letters},
volume = {12},
number = {18},
pages = {4347-4356},
year = {2021},
ref_doi = {10.1021/acs.jpclett.1c00943},
    ref_note ={PMID: 33929859},

ref_url = { 
        https://doi.org/10.1021/acs.jpclett.1c00943
    
},
ref_eprint = { 
        https://doi.org/10.1021/acs.jpclett.1c00943
    
}

}

@Article{Yu2017,
author={Yu, Yi
and Baskin, Artem
and Valero-Vidal, Carlos
and Hahn, Nathan T.
and Liu, Qiang
and Zavadil, Kevin R.
and Eichhorn, Bryan W.
and Prendergast, David
and Crumlin, Ethan J.},
title={Instability at the Electrode/Electrolyte Interface Induced by Hard Cation Chelation and Nucleophilic Attack},
journal={Chemistry of Materials},
year={2017},
month={Oct},
day={10},
publisher={American Chemical Society},
volume={29},
number={19},
pages={8504-8512},
ref_note={doi: 10.1021/acs.chemmater.7b03404},
issn={0897-4756},
ref_url={https://doi.org/10.1021/acs.chemmater.7b03404}
}



@article{Lv2022,
author = {Lv, Yanqun and Xiao, Ying and Ma, Longtao and Zhi, Chunyi and Chen, Shimou},
title = {Recent Advances in Electrolytes for `Beyond Aqueous' Zinc-Ion Batteries},
journal = {Advanced Materials},
volume = {34},
number = {4},
pages = {2106409},
ref_doi = {https://doi.org/10.1002/adma.202106409},
ref_url = {https://onlinelibrary.wiley.com/doi/abs/10.1002/adma.202106409},

year = {2022},
}


@article{Bodin2023,
author = {Bodin, Charlotte and Forero Saboya, Juan and Jankowski, Piotr and Radan, Kristian and Foix, Dominique and Courrèges, Cécile and Yousef, Ibraheem and Dedryvère, Rémi and Davoisne, Carine and Lozinšek, Matic and Ponrouch, Alexandre},
title = {Boron-Based Functional Additives Enable Solid Electrolyte Interphase Engineering in Calcium Metal Battery},
journal = {Batteries \& Supercaps},
volume = {6},
number = {1},
pages = {e202200433},
keywords = {boron-based adducts, calcium batteries, electrolyte additive, solid electrolyte interphase},
ref_doi = {https://doi.org/10.1002/batt.202200433},
ref_url = {https://chemistry-europe.onlinelibrary.wiley.com/doi/abs/10.1002/batt.202200433},
ref_eprint = {https://chemistry-europe.onlinelibrary.wiley.com/doi/pdf/10.1002/batt.202200433},
abstract = {Abstract Calcium-metal batteries have received growing attention recently after several studies reporting successful metal plating and stripping with organic electrolytes. Given the low redox potential of metallic calcium, its surface is commonly covered by a passivation layer grown by the accumulation of electrolyte decomposition products. The presence of borate species in this layer has been shown to be a key parameter allowing for Ca2+ migration and favoring Ca electrodeposition. Here, boron-based additives are evaluated in order to tune the SEI composition, morphology, and properties. The decomposition of a BF3-based additive is studied at different potentiostatic steps and the resulting SEI layer was thoroughly characterized. SEI growth mechanism is proposed based on both experimental data and DFT calculations pointing at the formation of boron-crosslinked polymeric matrices. Several boron-based adducts are explored as SEI-forming additives for calcium-metal batteries paving the way to very rich chemistry leading to Ca2+ conducting SEI.},
year = {2023}
}




@article{Deng2022,
title = {Anode chemistry in calcium ion batteries: A review},
journal = {Energy Storage Materials},
volume = {53},
pages = {467-481},
year = {2022},
issn = {2405-8297},
ref_doi = {https://doi.org/10.1016/j.ensm.2022.09.033},
ref_url = {https://www.sciencedirect.com/science/article/pii/S2405829722005207},
author = {Xianming Deng and Linyuan Li and Guobin Zhang and Xu Zhao and Jing Hao and Cuiping Han and Baohua Li},
keywords = {Calcium ion batteries, Calcium metals, Electrolytes, Anode materials},
abstract = {Large-scale energy storage and scientific research rapidly promote the research and exploration of calcium ion batteries (CIBs) due to the abundant reservation of calcium and the competitive redox potential of Ca/Ca2+. However, several critical issues hindered its development, especially the unsatisfactory performance of anode materials due to the poor plating/stripping reversibility Ca metal and sluggish de-solvation kinetics of Ca2+ ion in the electrolyte. This review aims to provide a timely access to the state-of-art advance in anode chemistry of CIBs. Firstly, an overview on the development history of CIBs is presented. Then, a comprehensive understanding on the challenges facing CIBs is elaborated. The advancements on the development of appropriate salts and electrolytes to achieve highly reversible plating/stripping of calcium metal anode are systematically discussed. As alternative to metallic calcium anode, suitable anode candidates for hosting calcium ions are summarized, including alloying, intercalation, and organic electrodes systems. Finally, critical challenges for metallic calcium and alternative anode materials are envisioned, followed by their important future directions for the development and application in CIBs.}
}

@article{GuWu2017,
title = {Confirming reversible \ce{Al^3+} storage mechanism through intercalation of \ce{Al^3+} into \ce{V2O5} nanowires in a rechargeable aluminum battery},
journal = {Energy Storage Materials},
volume = {6},
pages = {9-17},
year = {2017},
issn = {2405-8297},
ref_doi = {https://doi.org/10.1016/j.ensm.2016.09.001},
ref_url = {https://www.sciencedirect.com/science/article/pii/S2405829716301842},
author = {Sichen Gu and Huali Wang and Chuan Wu and Ying Bai and Hong Li and Feng Wu},
keywords = {Rechargeable aluminum battery, Al storage, VO nanowire, Intercalation reaction, Phase transition reaction},
abstract = {As a new type of multi-electron transfer device, rechargeable aluminum batteries are promising post-lithium ion batteries owing to their high theoretical energy density. However, it is unknown whether Al3+ can be reversibly stored in the lattice of the host electrode material because of its small cation diameter and high valence state, thus trapping it tightly in lattice or defect sites. Here, we report the reversible storage of Al3+ in V2O5 nanowires. It is found that Al3+ intercalates into crystalized V2O5 nanowires in the first discharge. Meanwhile, this electrochemical intercalation leads to the reduction of V5+ and the formation of an amorphous layer on the edge of nanowires. In the subsequent cycling, a new phase forms along the nanowires’ edges and a two-phase transition reaction occurs. Our findings demonstrate clearly for the first time that it is possible that Al3+ can be inserted into the metal oxide and stored reversibly through intercalation and a phase-transition reaction, which is expected to inspire more comprehensive investigations for rechargeable aluminum batteries.}
}





@article{LeeYu2023,
author = {Lee, Eun-Seo and Huh, Sung-Ho and Lee, Si-Hwan and Yu, Seung-Ho},
title = {On the Road to Stable Electrochemical Metal Deposition in Multivalent Batteries},
journal = {ACS Sustainable Chemistry \& Engineering},
volume = {11},
number = {6},
pages = {2014-2032},
year = {2023},
ref_doi = {10.1021/acssuschemeng.2c05173},

ref_url = { 
        https://doi.org/10.1021/acssuschemeng.2c05173
    
},
ref_eprint = { 
        https://doi.org/10.1021/acssuschemeng.2c05173
    
}

}



@article{Hou2021,
author = {Singyuk Hou  and Xiao Ji  and Karen Gaskell  and Peng-fei Wang  and Luning Wang  and Jijian Xu  and Ruimin Sun  and Oleg Borodin  and Chunsheng Wang },
title = {Solvation sheath reorganization enables divalent metal batteries with fast interfacial charge transfer kinetics},
journal = {Science},
volume = {374},
number = {6564},
pages = {172-178},
year = {2021},
ref_doi = {10.1126/science.abg3954},
ref_url = {https://www.science.org/doi/abs/10.1126/science.abg3954},
ref_eprint = {https://www.science.org/doi/pdf/10.1126/science.abg3954},
abstract = {Divalent rechargeable metal batteries such as those based on magnesium and calcium are of interest because of the abundance of these elements and their lower tendency to form dendrites, but practical demonstrations are lacking. Hou et al. used methoxyethyl amine chelants in which the ligands attach to the metal atom in more than one place, modulating the solvation structure of the metal ions to enable a facile charge-transfer reaction (see the Perspective by Zuo and Yin). In full battery cells, these components lead to high efficiency and energy density. Theoretical calculations were used to understand the solvation structures. —MSL Chelating ligands promote fast charge-transfer kinetics for Mg and Ca batteries with substantially lowered overpotentials. Rechargeable magnesium and calcium metal batteries (RMBs and RCBs) are promising alternatives to lithium-ion batteries because of the high crustal abundance and capacity of magnesium and calcium. Yet, they are plagued by sluggish kinetics and parasitic reactions. We found a family of methoxyethyl-amine chelants that greatly promote interfacial charge transfer kinetics and suppress side reactions on both the cathode and metal anode through solvation sheath reorganization, thus enabling stable and highly reversible cycling of the RMB and RCB full cells with energy densities of 412 and 471 watt-hours per kilogram, respectively. This work provides a versatile electrolyte design strategy for divalent metal batteries.}}



@Article{Liang2020,
author={Liang, Yanliang
and Dong, Hui
and Aurbach, Doron
and Yao, Yan},
title={Current status and future directions of multivalent metal-ion batteries},
journal={Nature Energy},
year={2020},
volume={5},
number={9},
pages={646-656},
abstract={Batteries based on multivalent metals have the potential to meet the future needs of large-scale energy storage, due to the relatively high abundance of elements such as magnesium, calcium, aluminium and zinc in the Earth{\^a}??s crust. However, the complexity of multivalent metal-ion chemistries has led to rampant confusions, technical challenges, and eventually doubts and uncertainties about the future of these technologies. In this Review, we clarify the key strengths as well as common misconceptions of multivalent metal-based batteries. We then examine the growth behaviour of metal anodes, which is crucial for their safety promises but hitherto unestablished. We further discuss scrutiny of anode efficiency and cathode storage mechanism pertaining to complications arising from electrolyte solutions. Finally, we critically review existing cathode materials and discuss design strategies to enable genuine multivalent metal-ion-based energy storage materials with competitive performance.},
issn={2058-7546},
ref_url={https://doi.org/10.1038/s41560-020-0655-0}
}



@Article{Dong2020,
author={Dong, Hui
and Tutusaus, Oscar
and Liang, Yanliang
and Zhang, Ye
and Lebens-Higgins, Zachary
and Yang, Wanli
and Mohtadi, Rana
and Yao, Yan},
title={High-power \ce{Mg} batteries enabled by heterogeneous enolization redox chemistry and weakly coordinating electrolytes},
journal={Nature Energy},
year={2020},
volume={5},
number={12},
pages={1043-1050},
abstract={Magnesium batteries have long been pursued as potentially low-cost, high-energy and safe alternatives to Li-ion batteries. However, Mg2+ interacts strongly with electrolyte solutions and cathode materials, leading to sluggish ion dissociation and diffusion, and consequently low power output. Here we report a heterogeneous enolization chemistry involving carbonyl reduction (C=O{\^a}??C{\^a}??O{\^a}??), which bypasses the dissociation and diffusion difficulties, enabling fast and reversible redox processes. This kinetically favoured cathode is coupled with a tailored, weakly coordinating boron cluster-based electrolyte that allows for dendrite-free Mg plating/stripping at a current density of 20{\^a}??mA{\^a}??cm{\^a}??2. The combination affords a Mg battery that delivers a specific power of up to 30.4{\^a}??kW{\^a}??kg{\^a}??1, nearly two orders of magnitude higher than that of state-of-the-art Mg batteries. The cathode and electrolyte chemistries elucidated here propel the development of magnesium batteries and would accelerate the adoption of this low-cost and safe battery technology.},
issn={2058-7546},
ref_url={https://doi.org/10.1038/s41560-020-00734-0}
}

@Article{Baer2016,
author={Baer, Marcel D.
and Mundy, Christopher J.},
title={Local Aqueous Solvation Structure Around \ce{Ca$^{2+}$} During \ce{Ca$^{2+}$-Cl$^-$} Pair Formation},
journal={The Journal of Physical Chemistry B},
year={2016},
month={Mar},
day={03},
publisher={American Chemical Society},
volume={120},
number={8},
pages={1885-1893},
ref_note={doi: 10.1021/acs.jpcb.5b09579},
issn={1520-6106},
ref_url={https://doi.org/10.1021/acs.jpcb.5b09579}
}

@article{Dominko2020,
title = {Magnesium batteries: Current picture and missing pieces of the puzzle},
journal = {Journal of Power Sources},
volume = {478},
pages = {229027},
year = {2020},
issn = {0378-7753},
ref_doi = {https://doi.org/10.1016/j.jpowsour.2020.229027},
ref_url = {https://www.sciencedirect.com/science/article/pii/S0378775320313240},
author = {Robert Dominko and Jan Bitenc and Romain Berthelot and Magali Gauthier and Gioele Pagot and Vito {Di Noto}},
keywords = {Magnesium, Electrolyte, Cathode, Anode, Review, Energy density},
abstract = {Rechargeable magnesium batteries are gaining a lot of interest due to promising electrochemical features, which, at least in theory, are comparable than those of Li-ion batteries. Such performance metrics can be achieved by using thin metal foils or high-capacity alloys coupled with suitable electrolytes enabling a high Coulombic efficiency and use of a high energy density cathode materials. All three components significantly influence electrochemical characteristics and energy density of rechargeable magnesium batteries. Although there are many reports showing progress in the cyclability and stability of different systems, only few cathode materials promise possible commercialization. Remaining issues with efficiency, magnesium anode processing and electrolyte compatibility with cell housing are preventing faster development of technology with high possible impact on the future battery landscape. In the given perspective paper a critical overview on electrolytes, anode materials and three different classes of cathode materials is reported. Different rechargeable magnesium battery configurations were assumed and their dependence of volumetric energy densities on gravimetric energy densities are provided assuming realistic conditions with optimized electrode thicknesses and loadings, electrode porosity and optimized electrolyte quantity. Although calculated values are attractive, further experimental steps are needed in order to prove these numbers on the lab-scale and small prototype cells.}}


@Article{Roy2016,
author={Roy, Santanu
and Baer, Marcel D.
and Mundy, Christopher J.
and Schenter, Gregory K.},
title={Reaction Rate Theory in Coordination Number Space: An Application to Ion Solvation},
journal={The Journal of Physical Chemistry C},
year={2016},
month={Apr},
day={14},
publisher={American Chemical Society},
volume={120},
number={14},
pages={7597-7605},
ref_note={doi: 10.1021/acs.jpcc.6b00443},
issn={1932-7447},
ref_url={https://doi.org/10.1021/acs.jpcc.6b00443}
}




@article{Joswiak2018,
author = {Mark N. Joswiak  and Michael F. Doherty  and Baron Peters },
title = {Ion dissolution mechanism and kinetics at kink sites on NaCl surfaces},
journal = {Proceedings of the National Academy of Sciences},
volume = {115},
number = {4},
pages = {656-661},
year = {2018},
ref_doi = {10.1073/pnas.1713452115},
ref_url = {https://www.pnas.org/doi/abs/10.1073/pnas.1713452115},
ref_eprint = {https://www.pnas.org/doi/pdf/10.1073/pnas.1713452115},
abstract = {Desolvation barriers are present for solute–solvent exchange events, such as ligand binding to an enzyme active site, during protein folding, and at battery electrodes. For solution-grown crystals, desolvation at kink sites can be the rate-limiting step for growth. However, desolvation and the associated kinetic barriers are poorly understood. In this work, we use rare-event simulation techniques to investigate attachment/detachment events at kink sites of a NaCl crystal in water. We elucidate the desolvation mechanism and present an optimized reaction coordinate, which involves one solute collective variable and one solvent collective variable. The attachment/detachment pathways for Na+ and Cl− are qualitatively similar, with quantitative differences that we attribute to different ion sizes and solvent coordination. The attachment barriers primarily result from kink site desolvation, while detachment barriers largely result from breaking ion–crystal bonds. We compute ion detachment rates from kink sites and compare with results from an independent study. We anticipate that the reaction coordinate and desolvation mechanism identified in this work may be applicable to other alkali halides.}}


@Article{Silvestri2022,
author={Silvestri, Alessandro
and Raiteri, Paolo
and Gale, Julian D.},
title={Obtaining Consistent Free Energies for Ion Binding at Surfaces from Solution: Pathways versus Alchemy for Determining Kink Site Stability},
journal={Journal of Chemical Theory and Computation},
year={2022},
month={Oct},
day={11},
publisher={American Chemical Society},
volume={18},
number={10},
pages={5901-5919},
ref_note={doi: 10.1021/acs.jctc.2c00787},
issn={1549-9618},
ref_url={https://doi.org/10.1021/acs.jctc.2c00787}
}


@article{rajput2015,
author = {Rajput, Nav Nidhi and Qu, Xiaohui and Sa, Niya and Burrell, Anthony K. and Persson, Kristin A.},
title = {The Coupling between Stability and Ion Pair Formation in Magnesium Electrolytes from First-Principles Quantum Mechanics and Classical Molecular Dynamics},
journal = {Journal of the American Chemical Society},
volume = {137},
number = {9},
pages = {3411-3420},
year = {2015},
ref_doi = {10.1021/jacs.5b01004},
    ref_note ={PMID: 25668289},

ref_url = { 
        https://doi.org/10.1021/jacs.5b01004
    
},
ref_eprint = { 
        https://doi.org/10.1021/jacs.5b01004
    
}

}


@article{McClary2022,
author = {McClary, Scott A. and Long, Daniel M. and Sanz-Matias, Ana and Kotula, Paul G. and Prendergast, David and Jungjohann, Katherine L. and Zavadil, Kevin R.},
title = {A Heterogeneous Oxide Enables Reversible Calcium Electrodeposition for a Calcium Battery},
journal = {ACS Energy Letters},
volume = {7},
number = {8},
pages = {2792-2800},
year = {2022},
ref_doi = {10.1021/acsenergylett.2c01443},

ref_url = { 
        https://doi.org/10.1021/acsenergylett.2c01443
    
},

}




@article{Driscoll_2020,
	ref_doi = {10.1149/1945-7111/abc8e3},
	ref_url = {https://doi.org/10.1149/1945-7111/abc8e3},
	year = 2020,
	month = {dec},
	publisher = {The Electrochemical Society},
	volume = {167},
	number = {16},
	pages = {160512},
	author = {Darren M. Driscoll and Naveen K. Dandu and Nathan T. Hahn and Trevor J. Seguin and Kristin A. Persson and Kevin R. Zavadil and Larry A. Curtiss and Mahalingam Balasubramanian},
	title = {Rationalizing Calcium Electrodeposition Behavior by Quantifying Ethereal Solvation Effects on Ca2$\mathplus$ Coordination in Well-Dissociated Electrolytes},
	journal = {Journal of The Electrochemical Society},
	abstract = {Ca-ion electrochemical systems have been pushed to the forefront of recent multivalent energy storage advances due to their use of earth-abundant redox materials and their high theoretical specific densities in relation to monovalent or even other more widely explored multivalent-charge carriers. However, significant pitfalls in metal plating and stripping arise from electrolyte decomposition and can be related to the coordination environment around Ca2+ with both the negatively charged anion and the organic–aprotic solvent. In this study, we apply multiple spectroscopic techniques in conjunction with density functional theory to evaluate the coordination environment of Ca2+ across a class of ethereal solvents. Through the combination of X-ray absorption fine structure and time-dependent density functional theory, descriptive measures of the local geometry, coordination, and electronic structure of Ca–ethereal complexes provide distinct structural trends depending on the extent of the Ca2+–solvent interaction. Finally, we correlate these findings with electrochemical measurements of calcium tetrakis(hexafluoroisopropoxy)borate (CaBHFIP2) salts dissolved within this class of solvents to provide insight into the preferred structural configuration of Ca2+ electrolytic solutions for optimized electrochemical plating and stripping.}}

@Article{hahn2020,
author ="Hahn, Nathan T. and Self, Julian and Seguin, Trevor J. and Driscoll, Darren M. and Rodriguez, Mark A. and Balasubramanian, Mahalingam and Persson, Kristin A. and Zavadil, Kevin R.",
title  ="The critical role of configurational flexibility in facilitating reversible reactive metal deposition from borohydride solutions",
journal  ="J. Mater. Chem. A",
year  ="2020",
volume  ="8",
issue  ="15",
pages  ="7235-7244",
publisher  ="The Royal Society of Chemistry",
doi  ="10.1039/D0TA02502J",
ref_url  ="http://dx.doi.org/10.1039/D0TA02502J",
abstract  ="Development of calcium metal batteries has been historically frustrated by a lack of electrolytes capable of supporting reversible calcium electrodeposition. In this paper{,} we report the study of an electrolyte consisting of Ca(BH4)2 in tetrahydrofuran (THF) to gain important insight into the role of the liquid solvation environment in facilitating the reversible electrodeposition of this highly reactive{,} divalent metal. Through interrogation of the Ca2+ solvation environment and comparison with Mg2+ analogs{,} we show that an ability to reversibly electrodeposit metal at reasonable rates is strongly regulated by dication charge density and polarizability. Our results indicate that the greater polarizability of Ca2+ over Mg2+ confers greater configurational flexibility{,} enabling ionic cluster formation via neutral multimer intermediates. Increased concentration of the proposed electroactive species{,} CaBH4+{,} enables rapid and stable delivery of Ca2+ to the electrode interface. This work helps set the stage for future progress in the development of electrolytes for calcium and other divalent metal batteries."}


@Article{Jay2019,
author={Jay, Rahul
and Tomich, Anton W.
and Zhang, Jian
and Zhao, Yifan
and De Gorostiza, Audrey
and Lavallo, Vincent
and Guo, Juchen},
title={Comparative Study of \ce{Mg(CB11H12)2} and \ce{Mg(TFSI)2} at the Magnesium/Electrolyte Interface},
journal={ACS Applied Materials \& Interfaces},
year={2019},
month={Mar},
day={27},
publisher={American Chemical Society},
volume={11},
number={12},
pages={11414-11420},
ref_note={doi: 10.1021/acsami.9b00037},
issn={1944-8244},
ref_url={https://doi.org/10.1021/acsami.9b00037}
}



@article{andrade2002,
author = {de Andrade, Jones and Böes, Elvis S. and Stassen, Hubert},
title = {Computational Study of Room Temperature Molten Salts Composed by 1-Alkyl-3-methylimidazolium CationsForce-Field Proposal and Validation},
journal = {The Journal of Physical Chemistry B},
volume = {106},
number = {51},
pages = {13344-13351},
year = {2002},
ref_doi = {10.1021/jp0216629},

ref_url = { 
        https://doi.org/10.1021/jp0216629
    
},
ref_eprint = { 
        https://doi.org/10.1021/jp0216629
    
}

}

@Article{hahn2022,
author={Hahn, Nathan T.
and McClary, Scott A.
and Landers, Alan T.
and Zavadil, Kevin R.},
title={Efficacy of Stabilizing Calcium Battery Electrolytes through Salt-Directed Coordination Change},
journal={The Journal of Physical Chemistry C},
year={2022},
month={Jun},
day={30},
publisher={American Chemical Society},
volume={126},
number={25},
pages={10335-10345},
ref_note={doi: 10.1021/acs.jpcc.2c02587},
issn={1932-7447},
ref_url={https://doi.org/10.1021/acs.jpcc.2c02587}
}
             


@Article{Liepinya2021,
author={Liepinya, Diana
and Smeu, Manuel},
title={A Computational Comparison of Ether and Ester Electrolyte Stability on a \ce{Ca Metal} Anode},
journal={Energy Material Advances},
year={2021},
publisher={AAAS},
volume={2021},
pages={9769347},
ref_url={https://doi.org/10.34133/2021/9769347}
}

@Article{WeiLiu2022,
author ="Wei, Qianshun and Zhang, Liping and Sun, Xiaohua and Liu, T. Leo",
title  ="Progress and prospects of electrolyte chemistry of calcium batteries",
journal  ="Chem. Sci.",
year  ="2022",
volume  ="13",
issue  ="20",
pages  ="5797-5812",
publisher  ="The Royal Society of Chemistry",
doi  ="10.1039/D2SC00267A",
ref_url  ="http://dx.doi.org/10.1039/D2SC00267A",
abstract  ="The increasing energy storage demand of portable devices{,} electric vehicles{,} and scalable energy storage has been driving extensive research for more affordable{,} more energy dense battery technologies than Li ion batteries. The alkaline earth metal{,} calcium (Ca){,} has been considered an attractive anode material to develop the next generation of rechargeable batteries. Herein{,} the chemical designs{,} electrochemical performance{,} and solution and interfacial chemistry of Ca2+ electrolytes are comprehensively reviewed and discussed. In addition{,} a few recommendations are presented to guide the development and evaluation of Ca2+ electrolytes in future."}





@article{jieTan202,
author = {Jie, Yulin and Tan, Yunshu and Li, Linmei and Han, Yehu and Xu, Shutao and Zhao, Zhenchao and Cao, Ruiguo and Ren, Xiaodi and Huang, Fanyang and Lei, Zhanwu and Tao, Guohua and Zhang, Genqiang and Jiao, Shuhong},
title = {Electrolyte Solvation Manipulation Enables Unprecedented Room-Temperature Calcium-Metal Batteries},
journal = {Angewandte Chemie International Edition},
volume = {59},
number = {31},
pages = {12689-12693},
keywords = {calcium plating, Coulombic efficiency, cycle life, solvation structure},
ref_doi = {https://doi.org/10.1002/anie.202002274},
ref_url = {https://onlinelibrary.wiley.com/doi/abs/10.1002/anie.202002274},

abstract = {Abstract Calcium-metal batteries (CMBs) provide a promising option for high-energy and cost-effective energy-storage technology beyond the current state-of-the-art lithium-ion batteries. Nevertheless, the development of room-temperature CMBs is significantly impeded by the poor reversibility and short lifespan of the calcium-metal anode. A solvation manipulation strategy is reported to improve the plating/stripping reversibility of calcium-metal anodes by enhancing the desolvation kinetics of calcium ions in the electrolyte. The introduction of lithium salt changes the electrolyte structure considerably by reducing coordination number of calcium ions in the first solvation shell. As a result, an unprecedented Coulombic efficiency of up to 99.1 \% is achieved for galvanostatic plating/stripping of the calcium-metal anode, accompanied by a very stable long-term cycling performance over 200 cycles at room temperature. This work may open up new opportunities for development of practical CMBs.},
year = {2020}
}


@article{samba2009,
author = {Sambasivarao, Somisetti V. and Acevedo, Orlando},
title = {Development of OPLS-AA Force Field Parameters for 68 Unique Ionic Liquids},
journal = {Journal of Chemical Theory and Computation},
volume = {5},
number = {4},
pages = {1038-1050},
year = {2009},
ref_doi = {10.1021/ct900009a},
    ref_note ={PMID: 26609613},

ref_url = { 
        https://doi.org/10.1021/ct900009a
    
},
ref_eprint = { 
        https://doi.org/10.1021/ct900009a
    
}

}


@Article{Wang2018,
author={Wang, Da
and Gao, Xiangwen
and Chen, Yuhui
and Jin, Liyu
and Kuss, Christian
and Bruce, Peter G.},
title={Plating and stripping calcium in an organic electrolyte},
journal={Nature Materials},
year={2018},
volume={17},
number={1},
pages={16-20},
abstract={Although multivalent cation batteries based on magnesium, calcium or aluminium are technologically attractive, the metal anode still represents a challenge. It is now demonstrated that significant quantities of calcium can be plated and stripped at room temperature with low polarization.},
issn={1476-4660},
ref_url={https://doi.org/10.1038/nmat5036}
}



@article{prabakar2019,
author = {Richard Prabakar, S. J. and Ikhe, Amol Bhairuba and Park, Woon Bae and Chung, Kee-Choo and Park, Hwangseo and Kim, Ki-Jeong and Ahn, Docheon and Kwak, Joon Seop and Sohn, Kee-Sun and Pyo, Myoungho},
title = {Graphite as a Long-Life Ca2+-Intercalation Anode and its Implementation for Rocking-Chair Type Calcium-Ion Batteries},
journal = {Advanced Science},
volume = {6},
number = {24},
pages = {1902129},
keywords = {Ca-ion batteries, calcium intercalation, cointercalation, graphite anodes, organic cathodes},
abstract = {Abstract Herein, graphite is proposed as a reliable Ca2+-intercalation anode in tetraglyme (G4). When charged (reduced), graphite accommodates solvated Ca2+-ions (Ca-G4) and delivers a reversible capacity of 62 mAh g−1 that signifies the formation of a ternary intercalation compound, Ca-G4·C72. Mass/volume changes during Ca-G4 intercalation and the evolution of in operando X-ray diffraction studies both suggest that Ca-G4 intercalation results in the formation of an intermediate phase between stage-III and stage-II with a gallery height of 11.41 Å. Density functional theory calculations also reveal that the most stable conformation of Ca-G4 has a planar structure with Ca2+ surrounded by G4, which eventually forms a double stack that aligns with graphene layers after intercalation. Despite large dimensional changes during charge/discharge (C/D), both rate performance and cyclic stability are excellent. Graphite retains a substantial capacity at high C/D rates (e.g., 47 mAh g−1 at 1.0 A g−1 s vs 62 mAh g−1 at 0.05 A g−1) and shows no capacity decay during as many as 2000 C/D cycles. As the first Ca2+-shuttling calcium-ion batteries with a graphite anode, a full-cell is constructed by coupling with an organic cathode and its electrochemical performance is presented.},
year = {2019}
}

@article{ta2019,
author = {Ta, Kim and Zhang, Ruixian and Shin, Minjeong and Rooney, Ryan T. and Neumann, Elizabeth K. and Gewirth, Andrew A.},
title = {Understanding \ce{Ca} Electrodeposition and Speciation Processes in Nonaqueous Electrolytes for Next-Generation \ce{Ca}-Ion Batteries},
journal = {ACS Applied Materials \& Interfaces},
volume = {11},
number = {24},
pages = {21536-21542},
year = {2019},

}


@ARTICLE{rev-Lu2021,
  
AUTHOR={Lu, Yi-Ting and Neale, Alex R. and Hu, Chi-Chang and Hardwick, Laurence J.},   
	 
TITLE={Divalent Nonaqueous Metal-Air Batteries},      
	
JOURNAL={Frontiers in Energy Research},      
	
VOLUME={8},      

PAGES={357},     
	
YEAR={2021},         
	
ISSN={2296-598X},   
   
ABSTRACT={In the field of secondary batteries, the growing diversity of possible applications for energy storage has led to the investigation of numerous alternative systems to the state-of-the-art lithium-ion battery. Metal-air batteries are one such technology, due to promising specific energies that could reach beyond the theoretical maximum of lithium-ion. Much focus over the past decade has been on lithium and sodium-air, and, only in recent years, efforts have been stepped up in the study of divalent metal-air batteries. Within this article, the opportunities, progress, and challenges in nonaqueous rechargeable magnesium and calcium-air batteries will be examined and critically reviewed. In particular, attention will be focused on the electrolyte development for reversible metal deposition and the positive electrode chemistries (frequently referred to as the “air cathode”). Synergies between two cell chemistries will be described, along with the present impediments required to be overcome. Scientific advances in understanding fundamental cell (electro)chemistry and electrolyte development are crucial to surmount these barriers in order to edge these technologies toward practical application.}
}


@article{melemed2020,
author = {Melemed, Aaron M. and Khurram, Aliza and Gallant, Betar M.},
title = {Current Understanding of Nonaqueous Electrolytes for Calcium-Based Batteries},
journal = {Batteries \& Supercaps},
volume = {3},
number = {7},
pages = {570-580},
keywords = {anode, battery, calcium, energy storage, organic electrolyte, solvent effects},
abstract = {Abstract Calcium metal batteries are receiving growing research attention due to significant breakthroughs in recent years that have indicated reversible Ca plating/stripping with attractive Coulombic efficiencies (90–95 \%), once thought to be out of reach. While the Ca anode is often described as being surface film-controlled, the ability to access reversible Ca electrochemistry is highly electrolyte-dependent in general, which affects both interfacial chemistry on plated Ca along with more fundamental Ca2+/Ca redox properties. This minireview describes recent progress towards a reversible Ca anode from the point of view of the most successful electrolyte chemistries identified to date. This includes, centrally, what is currently known about the Ca2+ solvation environment in these systems. Experimental (physico-chemical and spectroscopy) and computational results are summarized for the two major solvent classes – carbonates and ethers – that have yielded promising results so far. Current knowledge gaps and opportunities to improve fundamental understanding of Ca2+/Ca redox are also identified.},
year = {2020}
}
@article{nielson2020,
author = {Nielson, Kevin V. and Luo, Jian and Liu, T. Leo},
title = {Optimizing Calcium Electrolytes by Solvent Manipulation for Calcium Batteries},
journal = {Batteries \& Supercaps},
volume = {3},
number = {8},
pages = {766-772},
keywords = {calcium batteries, calcium electrolytes, energy storage, solvation, weakly coordination anions},
abstract = {Abstract Calcium is a highly attractive metal anode because of its high earth abundance and low reduction potential. However, the lack of calcium electrolytes for reversible calcium deposition significantly hampers the development of Ca rechargeable batteries. Herein, the calcium deposition/stripping behaviors of a calcium salt electrolyte, Ca[B(hfip)4]2 ([B(hfip)4]−=tetrakis(hexafluoroisopropyloxy)borate) were systematically studied using different working electrodes (GC, Pt, Cu, and Al) and different solvents including tetrahydrofuran (THF), dimethoxyethane (DME), and diglyme (DGM). It was found that the Ca[B(hfip)4]2/DGM electrolyte demonstrated the highest reversibility and stability in cyclic voltammetry and symmetric Ca/Ca half-cell studies. The Ca[B(hfip)4]2/DGM electrolyte was further employed to demonstrate a 3.4 V Ca battery using a FePO4 cathode with a discharge capacity of 120 Ah/mg.},
year = {2020}
}

%%%%%%%%%%%%%%%%%%%%%%%%%%%%%%%%%%%%%%%%%%%%%%%%%%%%%%%%%%%%%%%%%%%%%%%%%%%%%%%%%
%###########################################New references:
%%%%%%%%%%%%%%%%%%%%%%%%%%%%%%%%%%%%%%%%%%%%%%%%%%%%%%%%%%%%%%%%%%%%%%%%%%%%%%%%%

@Article{Kisu2021,
author={Kisu, Kazuaki
and Kim, Sangryun
and Shinohara, Takara
and Zhao, Kun
and Z{\~A}{\OE}ttel, Andreas
and Orimo, Shin-ichi},
title={Monocarborane cluster as a stable fluorine-free calcium battery electrolyte},
journal={Scientific Reports},
year={2021},
volume={11},
number={1},
pages={7563},
abstract={High-energy-density and low-cost calcium (Ca) batteries have been proposed as {\^a}??beyond-Li-ion{\^a}?? electrochemical energy storage devices. However, they have seen limited progress due to challenges associated with developing electrolytes showing reductive/oxidative stabilities and high ionic conductivities. This paper describes a calcium monocarborane cluster salt in a mixed solvent as a Ca-battery electrolyte with high anodic stability (up to 4{\^A}?V vs. Ca2+/Ca), high ionic conductivity (4 mS{\^A}?cm{\^a}??1), and high Coulombic efficiency for Ca plating/stripping at room temperature. The developed electrolyte is a promising candidate for use in room-temperature rechargeable Ca batteries.},
issn={2045-2322},
}


@article{Kisu2023,
author = {Kisu, Kazuaki and Mohtadi, Rana and Orimo, Shin-ichi},
title = {Calcium Metal Batteries with Long Cycle Life Using a Hydride-Based Electrolyte and Copper Sulfide Electrode},
journal = {Advanced Science},
volume = {10},
number = {22},
pages = {2301178},
keywords = {calcium metal anodes, copper sulfide, hydride-based electrolytes, long-term cycling, rechargeable calcium batteries},
abstract = {Abstract As potential alternatives to Li-ion batteries, rechargeable Ca metal batteries offer advantageous features such as high energy density, cost-effectiveness, and natural elemental abundance. However, challenges, such as Ca metal passivation by electrolytes and a lack of cathode materials with efficient Ca2+ storage capabilities, impede the development of practical Ca metal batteries. To overcome these limitations, the applicability of a CuS cathode in Ca metal batteries and its electrochemical properties are verified herein. Ex situ spectroscopy and electron microscopy results show that a CuS cathode comprising nanoparticles that are well dispersed in a high-surface-area carbon matrix can serve as an effective cathode for Ca2+ storage via the conversion reaction. This optimally functioning cathode is coupled with a tailored, weakly coordinating monocarborane-anion electrolyte, namely, Ca(CB11H12)2 in 1,2-dimethoxyethane/tetrahydrofuran, which enables reversible Ca plating/stripping at room temperature. The combination affords a Ca metal battery with a long cycle life of over 500 cycles and capacity retention of 92\% based on the capacity of the 10th cycle. This study confirms the feasibility of the long-term operation of Ca metal anodes and can expedite the development of Ca metal batteries.},
year = {2023}
}



@Article{WangCheng2018,
author={Wang, Meng
and Jiang, Chunlei
and Zhang, Songquan
and Song, Xiaohe
and Tang, Yongbing
and Cheng, Hui-Ming},
title={Reversible calcium alloying enables a practical room-temperature rechargeable calcium-ion battery with a high discharge voltage},
journal={Nature Chemistry},
year={2018},
volume={10},
number={6},
pages={667-672},
abstract={Calcium-ion batteries (CIBs) are attractive candidates for energy storage because Ca2+ has low polarization and a reduction potential ({\^a}??2.87{\^a}??V versus standard hydrogen electrode, SHE) close to that of Li+ ({\^a}??3.04{\^a}??V versus SHE), promising a wide voltage window for a full battery. However, their development is limited by difficulties such as the lack of proper cathode/anode materials for reversible Ca2+ intercalation/de-intercalation, low working voltages (<2{\^a}??V), low cycling stability, and especially poor room-temperature performance. Here, we report a CIB that can work stably at room temperature in a new cell configuration using graphite as the cathode and tin foils as the anode as well as the current collector. This CIB operates on a highly reversible electrochemical reaction that combines hexafluorophosphate intercalation/de-intercalation at the cathode and a Ca-involved alloying/de-alloying reaction at the anode. An optimized CIB exhibits a working voltage of up to 4.45{\^a}??V with capacity retention of 95\% after 350 cycles.},
issn={1755-4349},
}



@article{YangTrahey2023,
author = {Yang, Zhenzhen and Leon, Noel J. and Liao, Chen and Ingram, Brian J. and Trahey, Lynn},
title = {Effect of Salt Concentration on the Interfacial Solvation Structure and Early Stage of Solid–Electrolyte Interphase Formation in \ce{Ca(BH4)2/THF} for Ca Batteries},
journal = {ACS Applied Materials \& Interfaces},
volume = {15},
number = {20},
pages = {25018-25028},
year = {2023},
}
@article{Melemed2022,
author = {Melemed, Aaron M. and Skiba, Dhyllan A. and Gallant, Betar M.},
title = {Toggling Calcium Plating Activity and Reversibility through Modulation of \ce{Ca^2+} Speciation in Borohydride-Based Electrolytes},
journal = {The Journal of Physical Chemistry C},
volume = {126},
number = {2},
pages = {892-902},
year = {2022},
}


@article{Melemed_Oct2023,
author = {Melemed, Aaron M. and Skiba, Dhyllan A. and Steinberg, Katherine J. and Kim, Kyeong-Ho and Gallant, Betar M.},
title = {Impact of Differential \ce{Ca^2+} Coordination in Borohydride-Based Electrolyte Blends on Calcium Electrochemistry and SEI Formation},
journal = {The Journal of Physical Chemistry C},
volume = {0},
number = {0},
pages = {null},
year = {0},
}



@article{Baskin_2017,
year = {2017},
month = {jun},
publisher = {The Electrochemical Society},
volume = {164},
number = {11},
pages = {E3438},
author = {Artem Baskin and David Prendergast},
title = {Improving Continuum Models to Define Practical Limits for Molecular Models of Electrified Interfaces},
journal = {Journal of The Electrochemical Society},
abstract = {We develop a continuum theory of electrolyte solutions in contact with a metal electrode, based on a generalized free energy functional, and use it to explore the structure of the electric double layer at different electrochemical conditions. The model captures the effects of specific adsorption of ions and solvent polarization, and can be applied on the same footing to cases with non-zero faradaic current, beyond the classical double-layer regime. These advances permit the prediction of peculiar character in the ion profiles at electrode potentials near the redox level, exploration of the electrochemical stability of the interface, and differentiation between the mechanisms of electron and ion transport and associated time scales. The developed methodology enables us to self-consistently determine the fundamental limits for a microscopic description of biased interfaces, in terms characteristic sizes and time scales of relevant processes, within atomistic and ab initio molecular dynamics simulations.}
}

@article{Baskin_2019_JCP, 

    author = {Baskin, Artem and Prendergast, David},
    title = "{Exploring chemical speciation at electrified interfaces using detailed continuum models}",
    journal = {The Journal of Chemical Physics},
    volume = {150},
    number = {4},
    pages = {041725},
    year = {2019},
    month = {01},
    abstract = "{We present a local free-energy functional-based generic continuum model for material interfaces with a specific emphasis on electrified solid/liquid interfaces. The model enables a description of multicomponent phases at interfaces and includes the effects of specific non-electrostatic interactions (specific adsorption), ion size disparity, and the explicit presence of neutral species. In addition to the optimization of electrostatic, non-electrostatic, and steric forces, the model can be easily modified to explore the effects of other channels for equilibration, including local chemical transformations driven by equilibrium constants and electrochemical reactions driven by the electrode potential. In this way, we show that, upon accounting for these effects, local speciation in the vicinity of the interface can be drastically different from what is expected from restricted models and minor species (from the bulk perspective) may become dominant due to the effects of local pH. We evaluate the ionic contribution to the surface tension at the interface and show how this could impact the structure of air/liquid interfaces. On the same footing, an attempt to describe electrochemical metal dissolution is made. The model allows estimates of the mutual population of newly produced ions depending on their charge and size and could be useful for interpretation of electrochemical and spectroscopic measurements if the dissolution involves different metal ions (species). With these advances, the proposed model may be used as an ingredient within a hybrid ab initio-continuum methodology to model biased interfaces.}",
    issn = {0021-9606},
}



@article{Hosein2021,
author = {Hosein, Ian D.},
title = {The Promise of Calcium Batteries: Open Perspectives and Fair Comparisons},
journal = {ACS Energy Letters},
volume = {6},
number = {4},
pages = {1560-1565},
year = {2021},

}

@Book{Oxtobyc2002,
author={Oxtoby, David W.
and Gillis, H. P.
and Nachtrieb, Norman H.},
title={Principles of modern chemistry},
year={c2002},
publisher={Thomson/Brooks/Cole [Pacific Grove, CA]},
address={[Pacific Grove, CA]},
isbn={0030353734; 9780030353734}
}

@Book{Bard2022,
author={Bard, Allen J.
and Faulkner, Larry R.
and White, Henry S.},
title={Electrochemical methods : fundamentals and applications},
year={2022},
publisher={John Wiley \& Sons, Ltd. Hoboken, NJ},
address={Hoboken, NJ},
isbn={9781119334064; 1119334063}
}


@article{White2014,
author = {Fan, Lixin and Liu, Yuwen and Xiong, Jiewen and White, Henry S. and Chen, Shengli},
title = {Electron-Transfer Kinetics and Electric Double Layer Effects in Nanometer-Wide Thin-Layer Cells},
journal = {ACS Nano},
volume = {8},
number = {10},
pages = {10426-10436},
year = {2014}
}


@article{Willard2020,
author = {Limaye, Aditya M. and Willard, Adam P.},
title = {Modeling Interfacial Electron Transfer in the Double Layer: The Interplay between Electrode Coupling and Electrostatic Driving},
journal = {The Journal of Physical Chemistry C},
volume = {124},
number = {2},
pages = {1352-1361},
year = {2020},
}

@misc{pyqchem, 
name="PyQchem",
title="PyQchem",
url="https://pyqchem.readthedocs.io/en/master/",
author="Carreras, Abel and Casanova, David",
year = "2023",

}

@article{qchem,
author = {Yihan Shao and Zhengting Gan and Evgeny Epifanovsky and Andrew T.B. Gilbert and Michael Wormit and Joerg Kussmann and Adrian W. Lange and Andrew Behn and Jia Deng and Xintian Feng and Debashree Ghosh and Matthew Goldey and Paul R. Horn and Leif D. Jacobson and Ilya Kaliman and Rustam Z. Khaliullin and Tomasz Kuś and Arie Landau and Jie Liu and Emil I. Proynov and Young Min Rhee and Ryan M. Richard and Mary A. Rohrdanz and Ryan P. Steele and Eric J. Sundstrom and H. Lee Woodcock III and Paul M. Zimmerman and Dmitry Zuev and Ben Albrecht and Ethan Alguire and Brian Austin and Gregory J. O. Beran and Yves A. Bernard and Eric Berquist and Kai Brandhorst and Ksenia B. Bravaya and Shawn T. Brown and David Casanova and Chun-Min Chang and Yunqing Chen and Siu Hung Chien and Kristina D. Closser and Deborah L. Crittenden and Michael Diedenhofen and Robert A. DiStasio Jr. and Hainam Do and Anthony D. Dutoi and Richard G. Edgar and Shervin Fatehi and Laszlo Fusti-Molnar and An Ghysels and Anna Golubeva-Zadorozhnaya and Joseph Gomes and Magnus W.D. Hanson-Heine and Philipp H.P. Harbach and Andreas W. Hauser and Edward G. Hohenstein and Zachary C. Holden and Thomas-C. Jagau and Hyunjun Ji and Benjamin Kaduk and Kirill Khistyaev and Jaehoon Kim and Jihan Kim and Rollin A. King and Phil Klunzinger and Dmytro Kosenkov and Tim Kowalczyk and Caroline M. Krauter and Ka Un Lao and Adèle D. Laurent and Keith V. Lawler and Sergey V. Levchenko and Ching Yeh Lin and Fenglai Liu and Ester Livshits and Rohini C. Lochan and Arne Luenser and Prashant Manohar and Samuel F. Manzer and Shan-Ping Mao and Narbe Mardirossian and Aleksandr V. Marenich and Simon A. Maurer and Nicholas J. Mayhall and Eric Neuscamman and C. Melania Oana and Roberto Olivares-Amaya and Darragh P. O’Neill and John A. Parkhill and Trilisa M. Perrine and Roberto Peverati and Alexander Prociuk and Dirk R. Rehn and Edina Rosta and Nicholas J. Russ and Shaama M. Sharada and Sandeep Sharma and David W. Small and Alexander Sodt and Tamar Stein and David Stück and Yu-Chuan Su and Alex J.W. Thom and Takashi Tsuchimochi and Vitalii Vanovschi and Leslie Vogt and Oleg Vydrov and Tao Wang and Mark A. Watson and Jan Wenzel and Alec White and Christopher F. Williams and Jun Yang and Sina Yeganeh and Shane R. Yost and Zhi-Qiang You and Igor Ying Zhang and Xing Zhang and Yan Zhao and Bernard R. Brooks and Garnet K.L. Chan and Daniel M. Chipman and Christopher J. Cramer and William A. Goddard III and Mark S. Gordon and Warren J. Hehre and Andreas Klamt and Henry F. Schaefer III and Michael W. Schmidt and C. David Sherrill and Donald G. Truhlar and Arieh Warshel and Xin Xu and Alán Aspuru-Guzik and Roi Baer and Alexis T. Bell and Nicholas A. Besley and Jeng-Da Chai and Andreas Dreuw and Barry D. Dunietz and Thomas R. Fref_urlani and Steven R. Gwaltney and Chao-Ping Hsu and Yousung Jung and Jing Kong and Daniel S. Lambrecht and WanZhen Liang and Christian Ochsenfeld and Vitaly A. Rassolov and Lyudmila V. Slipchenko and Joseph E. Subotnik and Troy Van Voorhis and John M. Herbert and Anna I. Krylov and Peter M.W. Gill and Martin Head-Gordon},
title = {Advances in molecular quantum chemistry contained in the Q-Chem 4 program package},
journal = {Molecular Physics},
volume = {113},
number = {2},
pages = {184-215},
year = {2015},
publisher = {Taylor & Francis},
}

@article{pcm,
author = {J. Tomasi, B. Mennucci and E. Cancès},
title ={The IEF Version of the PCM Solvation Method: An Overview of a New Method Addressed to Study Molecular Solutes at the QM Ab Initio Level},
journal = {Molecular Physics},
year = {1999},
volume={464}, 
pages = {211-226},
}


@article{PEndergast_2023,
author = {Pendergast, Andrew D. and White, Henry S.},
title = {Double-Layer Inhibition of Peroxydisulfate Reduction at Mercury Ultramicroelectrodes. A Quantitative Analysis of the Frumkin Effect Including Molecular Transport and Long-Range Electron Transfer},
journal = {The Journal of Physical Chemistry C},
volume = {127},
number = {23},
pages = {11283-11297},
year = {2023},
}

@article{Hahn2023,
author = {Landers, Alan T. and Self, Julian and McClary, Scott A. and Fritzsching, Keith J. and Persson, Kristin A. and Hahn, Nathan T. and Zavadil, Kevin R.},
title = {Calcium Cosalt Addition to Alter the Cation Solvation Structure and Enhance the \ce{Ca} Metal Anode Performance},
journal = {The Journal of Physical Chemistry C},
volume = {127},
number = {49},
pages = {23664-23674},
year = {2023},
}

@article{YaoZhang2022,
author = {Yao, Nan and Chen, Xiang and Fu, Zhong-Heng and Zhang, Qiang},
title = {Applying Classical, Ab Initio, and Machine-Learning Molecular Dynamics Simulations to the Liquid Electrolyte for Rechargeable Batteries},
journal = {Chemical Reviews},
volume = {122},
number = {12},
pages = {10970-11021},
year = {2022},


}







@article{ChenZhang2020,
author = {Chen, Xiang and Zhang, Qiang},
title = {Atomic Insights into the Fundamental Interactions in Lithium Battery Electrolytes},
journal = {Accounts of Chemical Research},
volume = {53},
number = {9},
pages = {1992-2002},
year = {2020},


}

@article{Bedrov2019,
author = {Bedrov, Dmitry and Piquemal, Jean-Philip and Borodin, Oleg and MacKerell, Alexander D. Jr. and Roux, Benoît and Schröder, Christian},
title = {Molecular Dynamics Simulations of Ionic Liquids and Electrolytes Using Polarizable Force Fields},
journal = {Chemical Reviews},
volume = {119},
number = {13},
pages = {7940-7995},
year = {2019},

}
\end{filecontents}

\bibliography{references} 

\section*{Acknowledgements}
This work was primarily supported by the Joint Center for Energy Storage Research (JCESR), an Energy Innovation Hub funded by the U.S. Department of Energy, Office of Science, Basic Energy Sciences.  The theoretical analysis in this work by ASM was supported by a User Project at The Molecular Foundry and its computing resources, managed by the High Performance Computing Services Group at Lawrence Berkeley National Laboratory (LBNL), supported by the
Director, Office of Science, Office of Basic Energy Sciences, of
the United States Department of Energy under Contract DE-AC02-05CH11231. We thank Dr. Artem Baskin and the Reviewers for inspiration.

\section*{Author Contributions}

 ASM performed the calculations, analyzed the
data and wrote the original draft with support from DP. ASM and FR developed and tested the clustering
algorithm. SS contributed to force field
development. DP supervised and managed the project and developed and wrote the continuum model framework. All authors contributed to editing the manuscript. 

\section*{Competing Interests}
The Authors declare no competing interests.

% 

%%%%%%%%%% Merge with supplemental materials %%%%%%%%%%
%%%%%%%%%% Prefix a "S" to all equations, figures, tables and reset the counter %%%%%%%%%%


\renewcommand{\thesection}{Supplementary Information \arabic{section}}    %%%% but here
\renewcommand{\theequation}{S\arabic{equation}}
\renewcommand{\thefigure}{S\arabic{figure}}
\renewcommand{\thetable}{S\arabic{table}}
\renewcommand{\bibnumfmt}[1]{[S#1]}
%\renewcommand{~\citeumfont}[1]{S#1}
%%%%%%%%%% Prefix a "S" to all equations, figures, tables and reset the counter %%%%%%%%%%

\setcounter{equation}{0}
\setcounter{figure}{0}
\setcounter{table}{0}
\setcounter{page}{1}
% \makeatletter
\setcounter{section}{0}

\widetext
% \begin{center}
% \textbf{\large Supplemental Materials: Ca-dimers, solvent layering, and dominant  electrochemically active species in Ca(BH$_4$)$_2$ in THF}
% \end{center}


\section*{Supplementary Information: Ca-dimers, solvent layering, and dominant  electrochemically active species in Ca(BH$_4$)$_2$ in THF}






\begin{figure}[h!]
\centering
\includegraphics[width=0.9\linewidth]{images/SI/si-bulk-paths.png}
\caption{(a) Bulk free-energy surface of dimerization of ionic and neutral species, controlled by borohydride coordination number. (b) Free energy profiles for bulk disproportionation pathways with colors corresponding to the same indicated on the free-energy surface in (a). (c) Direct dissociation pathways by removal of borohydride obtained from vertical slices through the free-energy surface at specific Ca-Ca separations.}
\label{fig:si_bulk-paths}
\end{figure}




\begin{figure}[h!]
\centering
\includegraphics[width=0.9\linewidth]{images/SI/si-dimer-minima.png}
\caption{ \textbf{Determination of short and long dimers equilibrium RT distances. }Free-energy of dimerization (distance Ca-Ca) as a function of the Ca-B(H$_4$) coordination number (left) and Ca-O(THF) coordination number (r$_0$ = 3.6~\AA, right).  One-dimensional profiles obtained through thermodynamic integration over each coordination number allow us to interpolate the equilibrium distances of the short and long dimer.   }
\label{fig:si-dimer-minima}
\end{figure}




 \begin{figure}[ht]
\centering
\includegraphics[width=0.7\linewidth]{images/SI/perpendicular_dimer_coordination.pdf}
\caption{ \textbf{Solvent coordination of perpendicular dimers.} Orientation of the Ca-Ca axis with respect to the surface normal of structures of the isomer SD 5 THF - 4 at the DL (see Fig. 3 in the main text) depending on the THF-Ca coordination of the Ca that resides in the DL. At strongly perpendicular orientations (larger that 150 $^{\circ}$ with respect to the surface normal), two-fold solvent coordination is favored. }
\label{fig:si_perpendicular_dimer_coordination}
\end{figure}



\begin{figure}[h!]
\centering
\includegraphics[width=0.9\linewidth]{images/SI/bh4_layering_thf.pdf}
\caption{\textbf{ Free-energy of BH$_4^-$ at the (biased) interface. }The free-energy profile of a single THF molecule in THF (top) at RT with respect to distance from a graphite interface (red) and the corresponding oxygen density profile (black), with slight adjustment (dashed lines) at negative bias. The Dense Layer (DL), Intermediate Density Layer (IDL) and bulk regions are color-coded hereafter in red, blue and gray. The bottom figure shows the analogous free-energy profile of a single BH$_4^-$ molecule in THF at RT with respect to distance from a graphite interface (red) and the corresponding boron density profile (black). }
\label{fig:si-bh4-layering}
\end{figure}

\newpage

\newpage



\begin{table}[h!]
\begin{center}
\begin{tabular}{|  c | c | c | c | c    | c | c | c | c | c   | c | c | c | c |} 
\hline

Fig. & THF & Ca$^{2+}$ & BH$_4^-$ & G  & Box pars.~(\AA)  & CVs  & q$_G$ (e$^-$) & W & H & Freq & Reps. & RepFreq &  Time (ns) \\

 \hline
 Bulk  &200&2& 4& -  & 30.3 x 30.3 x30.3  &  dCa [0:16]~\AA    & - & 0.05 & 0.02 & 200  & 20 & 50000 & 1660\\

\hline
 1 &   & &  &    &                    & CN(Ca-B) [0-4]        &   &  0.05 &     &      &    &       &          \\

\hline
Neutral  &336&2& 4& 2  & 33.9 x 31.8 x 49.8  &  dCa [0:16]~\AA   & G1 = G2 = 0 & 0.2 & 0.02 & 200  & 70 & 50000 & 2082\\
2,3,  &   & &  &    &                    & CN(Ca-B) [0-4]            &   &  0.2 &     &      &    &       &          \\
5  &   & &  &    &                    & dZ [0:25]                 &   &  0.5 &     &      &    &       &          \\

\hline
2  &336  &-& -& 2  & 33.9 x 31.8 x 49.8 &  dZ [0:25]~\AA   & G1 = G2 = 0 & 0.05 & 0.02 & 200  & 15 & 50000 & 1049 \\ 
\hline

2  &336  &-& -& 2 & 33.9 x 31.8 x 49.8& dZ [0:25] & G1 = -1.4;   G2 = 0 &  0.05 & 0.02 & 200  & 10 & 50000 & 620  \\ 
\hline


\hline
CS1 &514&2& 4& 2  & 33.9 x 31.8 x 71.8  &  CN(Ca-Ca) [0:1]~\AA    & G1 = -0.7;  & 0.2 & 0.04 & 200  & 70 & 50000 & 1438 \\

  &   & &  &    &                    & CN(Ca-B) [0-4]            &  G2 = +0.7  &  0.2 &     &      &    &       &          \\
  &   & &  &    &                    & dZ [2:70]                 &   &  0.25 &     &      &    &       &          \\
\hline
CS2 &514&2& 4& 2  & 33.9 x 31.8 x 71.8  &  CN(Ca-Ca) [0:1]~\AA    & G1 = -1.4;  & 0.2 & 0.04 & 200  & 70 & 50000 & 1464 \\
 2,4, &   & &  &    &                    & CN(Ca-B) [0-4]            & G2 = +1.4   &  0.2 &     &      &    &       &          \\
5  &   & &  &    &                    & dZ [2:70]                 &   &  0.25 &     &      &    &        & 
 \\ 
\hline

\end{tabular}
\caption{\textbf{ Metadynamics simulation details.} Figure number or label, box composition (Ca$^{2+}$, BH$_4^-$ and graphene) and equilibrated lattice parameters, collective variables and their ranges, surface charges, and metadynamics parameters (hill width and height, frequency, number of replicas, replica update frequency, and total simulation time).  }
\label{tab:si_systems}
\end{center}

\end{table}



 \begin{figure}[ht]
\centering
\includegraphics[width=0.9\linewidth]{images/SI/1_4_dimers_free_energy.pdf}
\caption{ \textbf{Interfacial free-energy of [1,4] dimers.} Free energy pathways as a function of distance from the neutral surface for dimers with [1,4] Ca-B(H$_4$)$^-$  coordination numbers at different Ca-Ca distances. The most favorable, with d(Ca-Ca) = 4.9-5.1~\AA{}, preferably sits far away from the layered interface, at $\sim$ 17.5~\AA{}.}
\label{fig:si_1_4_dimer_freeenergies}
\end{figure}


\section{Dipole moments of Ca(BH$_4$)$_2$ isomers} 
\label{sec:si-dipole}

In order to compute the dipole moment of specific Ca(BH$_4$)$_2$ isomer, an average structure of the whole group was calculated (including THF solvating molecules). Then, using Density Functional Theory as implemented in Terachem,\cite{Ufimtsev2008} the atomic positions were optimized to obtain the dipole moment. Typically, a decrease in the dipole moment of about 1 D is observed upon minimization. The B3LYP\cite{b3lyp} exchange-correlation functional with a dispersion correction (D3)\cite{grimme2010} and a 6-311G basis set were used. 




\section{Reduction potentials}
\label{sec:si-reduction potential}
We have estimated the onset potential for  reduction, $\mathcal{E}$, as follows: 

\begin{align}
    \mathcal{E} < \ \mathcal{E}^0 + [\ln n_O(z) dV + \mu_O^{\mathrm{excess}}(z)]/e \nonumber \\
    + \phi(z) + [\Phi_O(z) - \Phi_R(z)]/e \label{eq:reductionpotential}
\end{align}

where $\mathcal{E}^0 = -\Delta G^{\mathrm{red}}/nF $, the standard reduction potential, is approximated as $-[E_{\mathrm{solv}}[R] - E_{\mathrm{solv}}[O] - \epsilon_F]/nF$ for an $n$-electron process ($F$ is Faraday's constant, $\epsilon_F$ is the metal electrode work function). In this work we focus on single-electron transfer events only (n=1) for electrolyte species. This term was obtained individually for each isomer structure using Density Functional Theory (we used the QCHEM code{~\cite{qchem}} through the PyQchem interface.{~\cite{pyqchem}}  The B3LYP{~\cite{b3lyp}} exchange-correlation functional with a dispersion correction (D3){~\cite{grimme2010}} and a 6-311G basis set were used, with implicit solvation modeled using PCM.{~\cite{pcm}}
The activity-dependence in this expression is captured by two terms related to: (1) the concentration profile of each species before reduction, $n_O (z)$ ($dV$ is the smallest volume element in our continuum model, set equal to the effective volume of the smallest species); and (2) the excess chemical potential, $\mu^{excess}_O(z)$, which includes non-ideal concentration contributions mostly due to volume exclusion. The electrostatic potential profile is indicated by $\phi(z)$, as function of distance from the electrode. These profiles were obtained self-consistently with our continuum model, employing as inputs the specific adsorption potential profiles of the oxidized and reduced species, $\Phi_O(z)$ and $\Phi_R(z)$, which were computed with molecular-dynamics-based free-energy sampling in the dilute limit.


\section{Thermodynamic reduction potential}\label{sec:si-thermo_red_pot}  In order to estimate the overpotential, we computed the thermodynamic reduction potential of the 2$e^-$-reduction-deposition process of a dication in the bulk electrolyte, with reference to Ca$^{2+}$ in vacuum so as to enable comparisons with the interfacial reduction potentials, as follows:

\begin{align}
 \Delta G^{red-dep} =  \Delta G^{red}_{(g)} -  \Delta G_{(sol)} - E_{at} - n(\epsilon_F - \varphi(0)  )  
\end{align}

Then, the potential at the electrode required for thermodynamically favorable reduction-deposition ($\Delta G^{red-dep}  \le 0 $) can then be defined as:

\begin{align}
 \varphi (0)^* \leq \epsilon_F - ( \Delta G^{red}_{(g)} -  \Delta G_{(sol)} - E_{at}) /n    
\end{align}

We assume a calcium metal electrode, with Fermi energy $\epsilon_F$ = -2.84 V, calcium atomization energy E$_{at}$ = 1.845 V,{~\cite{Oxtobyc2002}} and n=2 (averaging over  two electron transfer steps). We obtain  $\Delta G^{red}_{(g)}$ and  $\Delta G_{(sol)}$ from DFT calculations on Ca$^{2+}$(THF)$_6$ isomers favored in the bulk (see Fig.\ref{fig:si_mono_dication_full_solvent_distr}, bottom right). A thermodynamic average of the values of $\varphi$ (0)* for each isomer using bulk populations yields a bulk thermodynamic reduction potential for Ca$^{2+}$ in THF of -2.25~V. 



\begin{figure}
    \centering
    \includegraphics[width=0.6\paperwidth]{images/SI/ET_contribution_0.12M.png}
    \caption{Contributions to the electron transfer rate integrated throughout the interface for various overpotentials as a function of the dielectric spacer width (expressed as a fraction of the tunneling decay length $z_0$ at 0.12~M.}
    \label{fig:si_low_conc_ET_rates}
\end{figure}


\begin{figure}
    \centering
    \includegraphics[width=0.8\paperwidth]{images/SI/si_monocation_dication_interface_full_solvent.png}
    \caption{Population distribution of monocation (top) and dication (bottom) isomers by a biased electrode (CS2). Whole solvent and BH$_4$ molecules are included in the coordination sphere.  For the monocation, dipole orientation distribution of isomers with population larger than 10\% per layer are shown at the top. The angle is calculated with respect to the surface normal, with 180 being the BH$_4$ pointing away from the electrode. The bottom histograms show isomer populations obtained from unsupervised learning analysis  close to the center of the DL (z = 4.8 \AA),  close to the edge of the DL (z = 5.8 \AA) , in the IDL (z = 8.25 \AA), and in the bulk.   Representative atomic structures of the dominating isomers are shown in the inset in their preferred orientation, with the surface on the left.  Note the similar population distributions in the bulk and IDL, and the `flattened' DL isomers for monocations (4 THF - 1, 4 THF - 2, 5 THF - 1, and 5 THF - 5) and dications (5 THF - 0 and 1, and 6 THF - 0, 4 and 8). }
    \label{fig:si_mono_dication_full_solvent_distr}
\end{figure}

\bibliography{refs} 





\end{document}

