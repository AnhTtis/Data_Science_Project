\documentclass[11pt,reqno,oneside]{amsart}
%\pagestyle{plain}
%\usepackage[brazilian]{babel}
\usepackage[latin1]{inputenc}
\usepackage[T1]{fontenc}
\usepackage{amsmath,amsfonts,amscd,amssymb,amsthm}
\usepackage{mathtools}
\usepackage{dsfont}
\usepackage{color}
\usepackage[mathscr]{euscript}
\usepackage{comment}
\usepackage{graphicx}
\usepackage{placeins}
\usepackage{extsizes}
\usepackage{float}
\usepackage{multirow}
\usepackage{pdflscape}
\usepackage{caption}
\usepackage{subcaption}
\usepackage{tikz}
\usepackage{nicefrac}
\usepackage{longtable}
\usetikzlibrary{decorations.pathreplacing}
\definecolor{color1bg}{HTML}{f73d28}%FA8072}
\definecolor{color2bg}{HTML}{FA8072}
\definecolor{bblue}{HTML}{00BFFF}
\definecolor{bblue2}{HTML}{00ffff}


%\numberwithin{equation}{section}
%\usepackage[notcite,notref]{showkeys}
\newcommand*\circled[1]{\tikz[baseline=(char.base)]{
	\node[shape=circle,draw,inner sep=2pt] (char) {#1};}}
\usetikzlibrary{arrows,calc}
\tikzset{
%Define standard arrow tip
>=stealth',
%Define style for different line styles
help lines/.style={dashed, thick},
axis/.style={<->},
important line/.style={thick},
connection/.style={thick, dotted},
}

\usetikzlibrary{shadows}
\usetikzlibrary{backgrounds}
\tikzset{
diagonal fill/.style 2 args={fill=#2, path picture={
		\fill[#1, sharp corners] (path picture bounding box.south west) -|
		(path picture bounding box.north east) -- cycle;}},
reversed diagonal fill/.style 2 args={fill=#2, path picture={
		\fill[#1, sharp corners] (path picture bounding box.north west) |- 
		(path picture bounding box.south east) -- cycle;}}
}
\usetikzlibrary{arrows.meta}

\newcommand\irregularcircle[2]{% radius, irregularity
\pgfextra {\pgfmathsetmacro\len{(#1)+rand*(#2)}}
+(0:\len pt)
\foreach \a in {10,20,...,350}{
	\pgfextra {\pgfmathsetmacro\len{(#1)+rand*(#2)}}
	-- +(\a:\len pt)
} -- cycle
}

\usepackage{rotate}
%\usepackage[noend]{algorithmic}
%\usepackage{algorithm}% http://ctan.org/pkg/algorithms
%\usepackage{setspace}
%\usepackage{extsizes}
%\usepackage{hyperref}

%\usepackage[notcite,notref]{showkeys}

\makeatletter
\@addtoreset{equation}{section}
\makeatother
\newcounter{as}[section]
\renewcommand{\theas}{\thesection.\Alph{as}}
\newcommand{\newas}[1]{\refstepcounter{as}\label{#1}}
\newcommand{\useas}[1]{\ref{#1}}

\renewcommand\theequation{\thesection.\arabic{equation}}
\renewcommand\thefigure{\arabic{figure}}
\renewcommand\thetable{\thesection.\arabic{table}}

\newcommand{\<}{\langle}
\renewcommand{\>}{\rangle}
\DeclareMathOperator*{\argminA}{arg\,min}
\DeclareMathOperator*{\argmaxA}{arg\,max}


\title[Distribution-free Deviation Bounds]{Distribution-free Deviation Bounds and The Role of Domain Knowledge in Learning via Model Selection with Cross-validation Risk Estimation}

%\date{\today}
\author{Diego Marcondes}
\address{Department of Computer Science, Institute of Mathematics and Statistics, Universidade de S\~{a}o
	Paulo, R. do Mat\~{a}o, 1010 - Butant\~{a}, S\~{a}o Paulo - SP,
	05508-090, Brazil. \\
	e-mail: \texttt{dmarcondes@ime.usp.br}}

\author{Cl�udia Peixoto}
\address{Department of Applied Mathematics, Institute of Mathematics and Statistics, Universidade de S\~{a}o
	Paulo, R. do Mat\~{a}o, 1010 - Butant\~{a}, S\~{a}o Paulo - SP,
	05508-090, Brazil.}

\allowdisplaybreaks
\newtheorem{theorem}{Theorem}[section]
\newtheorem{remark}[theorem]{Remark}
%\newenvironment{proof}{\paragraph{Proof:}}{\hfill$\blacksquare$}

\newtheorem{definition}[theorem]{Definition}
\newtheorem{conjecture}{Conjecture}
\newtheorem{corollary}[theorem]{Corollary}
\newtheorem{lemma}[theorem]{Lemma}
\newtheorem{proposition}[theorem]{Proposition}
\newtheorem{example}[theorem]{Example}
\newtheorem{notation}[theorem]{Notation}


\newcommand{\mc}[1]{{\mathcal #1}}
\newcommand{\mf}[1]{{\mathfrak #1}}
\newcommand{\mb}[1]{{\mathbf #1}}
\newcommand{\bb}[1]{{\mathbb #1}}
\newcommand{\bs}[1]{{\boldsymbol #1}}
\newcommand{\ms}[1]{{\mathscr #1}}
\newcommand{\bbf}[1]{{\mathbf #1}}
\newcommand{\bl}[1]{{\color{blue} #1}}

\usepackage{lineno}
%\linenumbers





\begin{document}
	\maketitle
	
\begin{abstract}
	Cross-validation techniques for risk estimation and model selection are widely used in statistics and machine learning. However, the understanding of the theoretical properties of learning via model selection with cross-validation risk estimation is quite low in face of its widespread use. In this context, this paper presents learning via model selection with cross-validation risk estimation as a general systematic learning framework within classical statistical learning theory and establishes distribution-free deviation bounds in terms of VC dimension, giving detailed proofs of the results and considering both bounded and unbounded loss functions. In particular, we investigate how the generalization of learning via model selection may be increased by modeling the collection of candidate models. We define the Learning Spaces as a class of candidate models in which the partial order by inclusion reflects the models complexities, and we formalize a manner of defining them based on domain knowledge. We illustrate this modeling in a worst-case scenario of learning a classifier with finite domain and a typical scenario of linear regression.  Through theoretical insights and concrete examples, we aim to provide guidance on selecting the family of candidate models based on domain knowledge to increase generalization.
	
	\noindent \textbf{Keywords:} deviation bounds, cross-validation, model selection, statistical learning theory, unbounded loss function, empirical risk minimization
\end{abstract}

\section{Introduction}

In learning problems, properly choosing a hypotheses space is critical to achieving high generalization, which is more likely when it contains hypotheses with low risk and is of limited complexity relative to the sample size, so overfitting ought to be avoided, and the low-risk hypotheses can be learned. Properly selecting a hypotheses space passes through translating prior information and domain knowledge into properties that low-risk hypotheses should satisfy, and then designing a space containing hypotheses with these properties. This can be done, for example, by identifying invariances, such as a group invariance, associated with the learning problem and considering only hypotheses which respect them \cite{marcondesback}.

If prior information was wrong or weak, or if it could not be properly converted into a relatively simple hypotheses space, then generalization may be low. In order to mitigate these issues, one may select the hypotheses space from data, in what is known in the literature as model selection \cite{raschka2018,ding2018,massart2007}. In model selection, one fixes a collection $\{\mathcal{M}_{1},\dots,\mathcal{M}_{n}\}$ of hypotheses spaces, or models, and an empirical error $\hat{L}(\mathcal{M}_{i})$ for each one, which is often given by an independent validation sample, cross-validation, or complexity penalization. A hypotheses space is selected by minimizing the empirical error, and a hypothesis is learned from it.

If the candidate models are nested, i.e., $\mathcal{M}_{1} \subset \cdots \subset \mathcal{M}_{n}$, then a method based on the Structured Risk Minimization (SRM) Inductive Principle, in which the resubstitution error of the estimated hypothesis of $\mathcal{M}_{i}$ is penalized by its complexity, may be applied to solve this problem (see \cite[Chapter~4]{vapnik2000} for more details and \cite{anguita2012} for an example). Actually, model selection may be generally performed by penalizing the resubstitution error by the complexity of each model, in both nested and non-nested frameworks (see \cite{massart2007} for an in-depth presentation of model selection by penalization and \cite{koltchinskii2001,koltchinskii2011,arlot2011,bartlett2008} for more specific results). Moreover, the classical problem of variable selection \cite{guyon2003,john1994} constitutes another framework for model selection, in which a partially ordered structured family of constrained hypotheses spaces is generated through the elimination of variables.

Although there is rich literature about model selection, the methods usually have two shortcomings. First, the collection of candidate models is heuristically selected, and strong domain knowledge is not considered. For example, nested candidate models are selected due to their increasing complexity and not necessarily because their hypotheses have some properties that are believed to be satisfied by low-risk hypotheses. Second, even when domain knowledge is considered, the collection of candidate models does not have a rich structure that could be leveraged to enhance the computational efficiency of model selection. For example, a collection of nested models does not have as many algebraic properties as a Boolean lattice, which is isomorphic to the collection of candidate models in variable selection. Although variable selection is an example that considers domain knowledge and its collection of candidate models has a lattice structure, it is specific to problems in which low-risk hypotheses do not depend on all variables, and it cannot be readily adapted to other scenarios.

In this paper, we analyze how domain knowledge and prior information can be leveraged to increase generalization by learning via model selection. We focus on model selection with cross-validation risk estimation, in which a hypothesis is learned on the selected model with an independent sample or by reusing the sample used to select the model. We formalize model selection within the statistical learning framework and deduce insightful distribution-free deviation bounds for learning via model selection in this instance. These insights are then illustrated in case studies, and practical ways of increasing generalization by inserting domain knowledge into the family of candidate models are discussed.

\section{Background}

The classical framework of Statistical Learning Theory is a triple $(\mathcal{H},\mathbb{A},\mathcal{D}_{N})$, composed by a set $\mathcal{H}$ of hypotheses $h$, called hypotheses space, and a learning algorithm $\mathbb{A}(\mathcal{H},\mathcal{D}_{N})$, which searches $\mathcal{H}$ seeking to minimize a risk measure based on a training sample $\mathcal{D}_{N} = \{Z_{1},\dots,Z_{N}\}$ of a random vector $Z$, with range $\mathcal{Z} \subset \mathbb{R}^{d}, d \geq 1,$ and unknown probability distribution $P$.

Let $\ell: \mathcal{Z} \times \mathcal{H} \mapsto  \mathbb{R}_{+}$ be a loss function. The risk of a hypothesis $h \in \mathcal{H}$ is an expected value of the local loss $\ell(z,h), z \in \mathcal{Z}$. If the expectation is the sample mean of $\ell(z,h)$ under $\mathcal{D}_{N}$, we have the empirical risk $L_{\mathcal{D}_{N}}(h)$, while if the expectation of $\ell(Z,h)$ is under the distribution $P$, we then have the out-of-sample risk $L(h)$. A target hypothesis $h^{\star} \in \mathcal{H}$ is such that its out-of-sample risk is minimum in $\mathcal{H}$, i.e., $L(h^{\star}) \leq L(h), \forall h \in \mathcal{H}$, while an empirical risk minimization (ERM) hypothesis $\hat{h}$ is such that its empirical risk is minimum, i.e., $L_{\mathcal{D}_{N}}(\hat{h}) \leq L_{\mathcal{D}_{N}}(h), \forall h \in \mathcal{H}$.

In this context, learning via model selection is a two-step framework in which first a model is selected from a collection of candidates, and then a hypothesis is learned on it. Fix a hypotheses space $\mathcal{H}$, a collection of candidate subsets, or models,
\begin{align*}
	\mathbb{C}(\mathcal{H}) = \{\mathcal{M}_{1},\dots,\mathcal{M}_{n}\} & & \text{ such that } & & \mathcal{H} = \bigcup_{i=1}^{n} \mathcal{M}_{i},
\end{align*}
and an empirical risk estimator $\hat{L}: \mathbb{C}(\mathcal{H}) \mapsto \mathbb{R}_{+}$, which for each candidate model attributes a risk estimated from data. The first step of learning via model selection is to select a minimizer of $\hat{L}$ in $\mathbb{C}(\mathcal{H})$:
\begin{equation*}
	\hat{\mathcal{M}} \coloneqq \argminA_{\mathcal{M} \in \mathbb{C}(\mathcal{H})} \hat{L}(\mathcal{M}).
\end{equation*}
Once a model is selected, one employs a data-driven algorithm $\mathbb{A}$ to learn a hypothesis $\hat{h}^{\mathbb{A}}_{\hat{\mathcal{M}}}$ in $\hat{\mathcal{M}}$.

An intrinsic characteristic of model selection frameworks is a bias-variance trade off, which is depicted in Figure \ref{fig_error}. On the one hand, when one selects from data a model $\hat{\mathcal{M}}$ among the candidates to learn on, he adds a bias to the learning process if $h^{\star}$ is not in $\hat{\mathcal{M}}$, since the best hypotheses $h^{\star}_{\hat{\mathcal{M}}}$ one can learn on $\hat{\mathcal{M}}$ may have a greater risk than $h^{\star}$, i.e., $L(h^{\star}_{\hat{\mathcal{M}}}) > L(h^{\star})$. Hence, even when the sample size is great, the learned hypothesis may not well generalize relative to $h^{\star}$. We call this bias type III estimation error, as depicted in Figure \ref{fig_error}.

On the other hand, learning within $\hat{\mathcal{M}}$ may have a lesser error than on the whole $\mathcal{H}$, in the sense of $L(\hat{h}^{\mathbb{A}}_{\hat{\mathcal{M}}})$ being close to $L(h^{\star}_{\hat{\mathcal{M}}})$. We call their difference type II estimation error, as depicted in Figure \ref{fig_error}. The actual error of learning in this instance is the difference between $L(\hat{h}^{\mathbb{A}}_{\hat{\mathcal{M}}})$ and $L(h^{\star})$, also depicted in Figure \ref{fig_error} as type IV estimation error. 

A successful framework for model selection should be such that $L(\hat{h}^{\mathbb{A}}_{\hat{\mathcal{M}}}) < L(\hat{h}^{\mathbb{A}})$. In other words, by adding a bias (III), the learning variance (II) within $\hat{\mathcal{M}}$ should be low enough, so the actual error (IV) committed when learning is lesser than the one committed by learning on the whole space $\mathcal{H}$, that is $L(\hat{h}^{\mathbb{A}}) - L(h^{\star})$.

\begin{figure}[ht]
	\centering
	\includegraphics[width=0.75\linewidth]{Diagram_GErrors.png}
	\caption{Types II, III, and IV estimation errors when learning on $\hat{\mathcal{M}}$, in which $\hat{h}_{\hat{\mathcal{M}}} \equiv \hat{h}_{\hat{\mathcal{M}}}^{\mathbb{A}}$. These errors are formally defined in Section \ref{SecErrors}.}
	\label{fig_error}
\end{figure}

In this paper, we study distribution-free asymptotics of learning via model selection, that means bounding the estimation errors in Figure \ref{fig_error}. We define the target model $\mathcal{M}^{\star}$ among the candidates and establish convergence rates of $\hat{\mathcal{M}}$ to $\mathcal{M}^{\star}$. We focus on frameworks where the risk of each candidate model is estimated via a cross-validation procedure and learning on $\hat{\mathcal{M}}$ is performed with an independent sample. We briefly discuss the case in which the same sample that was used to selected $\hat{\mathcal{M}}$ is reused to learn on it.

\subsection{Related work}

Cross-validation techniques for risk estimation and model selection are widely spread in statistics and machine learning, and their asymptotic behavior have been fairly studied. The asymptotic optimality of specific cross-validation methods has been studied in \cite{andrews1991asymptotic,li1987asymptotic,dudoit2005asymptotics} and so-called oracle inequalities have been established for cross-validation methods in \cite{van2006oracle,lecue2012oracle}. Moreover, some straightforward deviation-bounds in distribution-free scenarios can be found in \cite[Chapter~4]{mohri2018foundations}. However, the understanding of the theoretical properties of cross-validation is quite low in face of its widespread use. This fact has been noted by the recent works \cite{austern2020asymptotics,maillard2021local} which study the asymptotics of cross-validation for specific classes of models. To the best of our knowledge, a systematic study of model selection with cross-validation risk estimation within a distribution-free framework has not been done before, and may bring some insights into the benefits of model selection and have an impact on practical applications.

In this paper, we apply some techniques to prove the results that have been applied before in the context of model selection via complexity penalization. For instance, results in  \cite[Chapter~4]{mohri2018foundations} present deviation bounds for SRM depending on the VC dimension of the target model, as do ours. However, our results also depend on the minimum discrimination error that is, informally, the difference between the risk of the target model and the second best.

We note that a lot of work has been done in model selection in distribution dependent settings, specially for complexity penalization, as very well presented in \cite{massart2007}. In particular, we outline the work in \cite{mendelson2004importance} which, in a distribution-dependent framework, obtains some insights related to that of this paper about the effect of the complexity of the learned model on the generalization quality of learning via model selection. Distribution-dependent bounds are out of the scope of this paper, but the results presented here might be adapted to distribution-dependent scenarios, a topic we will leave for future studies.

Despite the rich literature about model selection, an important facet of it is rarely treated from a statistical perspective: how the family of candidate models should be chosen. This question, when addressed, is usually done so from a computational perspective by fixing a family which can be efficiently searched. In particular, how generalization can be increased by properly selecting a penalization is a highly studied topic, but how it can be increased by properly selecting the family of candidate models is not. This paper is specially concerned with this neglected topic and how to leverage domain knowledge to increase generalization.

\subsection{Main contributions}

We present learning via model selection with cross-validation risk estimation as a general systematic learning framework within classical Statistical Learning Theory, and establish distribution-free deviation bounds for the estimation errors in terms of VC dimension, giving detailed proofs of the results and considering both bounded and unbounded loss functions. In order to treat the case of bounded loss functions we extend the classical Statistical Learning Theory \cite{vapnik1998,devroye1996} to learning via model selection, while to treat the case of unbounded loss functions we apply and extend recent results of \cite{cortes2019}.

In \cite{cortes2019}, bounds for the tails probabilities of
\begin{align*}
	\sup\limits_{h \in \mathcal{H}} \frac{L(h) - L_{\mathcal{D}_{N}}(h)}{\sqrt[p]{(L^{p}(h))^{p}} + \varsigma}
\end{align*}
are established for unbounded loss functions, for any $\varsigma > 0$, in which $L^{p}(h)$ is the $p$-norm of $\ell(Z,h)$ and $p > 1$ is such that $\sup_{h \in \mathcal{H}} L^{p}(h) < \infty$. We extend this result for $p = 1$ to obtain bounds for the relative type I estimation error
\begin{align*}
	\sup\limits_{h \in \mathcal{H}} \frac{|L(h) - L_{\mathcal{D}_{N}}(h)|}{L(h) + \varsigma}
\end{align*}
so that we can carry out an analysis of model selection with unbounded loss functions. The details about this extension are in Section \ref{ApUnbounded} in Appendix \ref{apVCtheory}.

We also define the Learning Spaces as a class of candidate models in which the partial order by inclusion reflects the models complexities. In particular, we focus on Learning Spaces that have a lattice structure and formalize a manner of defining them through Learning Space generators. We present some examples and discuss how the Learning Space can be modeled to reflect domain knowledge and prior information about the target hypotheses. We illustrate this modeling in a \textit{worst-case} scenario of learning a classifier with finite domain and a \textit{typical} scenario of linear regression. We expect with the theoretical results and these concrete examples to guide practitioners and applied researchers on how to choose the family of candidate models based on domain knowledge to increase generalization.

\subsection{Paper structure}

In Section \ref{SecPreliminaries}, we present the main concepts of Statistical Learning Theory, and define the framework of learning via model selection with cross-validation risk estimation as a systematic general learning framework. In Sections \ref{boundedL} and \ref{SecUnbounded}, we establish deviation bounds for learning via model selection with independent sample for bounded and unbounded loss functions, respectively. In Section \ref{SecReuse}, we briefly discuss learning via model selection by reusing. In Section \ref{SecEnhanceGen}, we introduce the Learning Spaces, discuss how they can be built based on prior information and study two concrete examples to better understand the effect on the generalization quality of quantities present in the established deviation-bounds. In particular, we discuss how domain knowledge can be leveraged to increase generalization by learning via model selection and briefly present considerations about computational aspects of learning via model selection. We discuss the main results and perspectives of this paper in Section \ref{FinalRemarks}. In Section \ref{SecProof}, we present the proof of the results. In order to increase the self-containment of this paper, enhancing its accessibility to readers not expert in VC theory, we present in Appendix \ref{apVCtheory} an overview of its main results necessary to this paper.

\section{Model selection in Statistical Learning}
\label{SecPreliminaries}

Let $Z$ be a random vector defined on a probability space $(\Omega,\mathcal{S},\mathbb{P})$, with range $\mathcal{Z} \subset \mathbb{R}^{d}, d \geq 1$. Denote $P(z) \coloneqq \mathbb{P}(Z \leq z)$ as the probability distribution of $Z$ at point $z \in \mathcal{Z}$, which we assume unknown, but fixed. Define a sample $\mathcal{D}_{N} = \{Z_{1}, \dots, Z_{N}\}$ as a sequence of independent and identically distributed random vectors, defined on $(\Omega,\mathcal{S},\mathbb{P})$, with distribution $P$.

Let $\mathcal{H}$ be a general set, whose typical element we denote by $h$, which we call hypotheses space. We denote subsets of $\mathcal{H}$ by $\mathcal{M}_{i}$, indexed by the positive integers, i.e., $i \in \mathbb{Z}_{+}$. We may also denote a subset of $\mathcal{H}$ by $\mathcal{M}$ to ease notation. We consider \textit{model} and \textit{subset of} $\mathcal{H}$ as synonyms.

Let $\ell: \mathcal{Z} \times \mathcal{H} \mapsto  \mathbb{R}_{+}$ be a, possibly unbounded, loss function, which represents the loss $\ell(z,h)$ that incurs when one applies a hypothesis $h \in \mathcal{H}$ to \textit{explain} a feature of point $z \in \mathcal{Z}$. Denoting $\ell_{h}(z) \coloneqq \ell(z,h)$ for $z \in \mathcal{Z}$, we assume that, for each $h \in \mathcal{H}$, the composite function $\ell_{h} \circ Z$ is $(\Omega,\mathcal{S})$-measurable.

The risk of a hypothesis $h \in \mathcal{H}$ is defined as
\begin{equation*}
	L(h) \coloneqq \mathbb{E}[\ell_{h}(Z)] =  \int_{\mathcal{Z}} \ell_{h}(z) \ dP(z),
\end{equation*}
in which $\mathbb{E}$ means expectation under $\mathbb{P}$. We define the empirical risk on sample $\mathcal{D}_{N}$ as
\begin{equation*}
	L_{\mathcal{D}_{N}}(h) \coloneqq \frac{1}{N} \sum_{i=1}^{N} \ell_{h}(Z_{i}),
\end{equation*}
that is the empirical mean of $\ell_{h}(Z)$.

We denote the set of target hypotheses of $\mathcal{H}$ as
\begin{equation*}
	h^{\star} \coloneqq \argminA\limits_{h \in \mathcal{H}} L(h),
\end{equation*}
that are the hypotheses that minimize $L$ in $\mathcal{H}$, and the set of the target hypotheses of subsets of $\mathcal{H}$ by
\begin{align*}
	h^{\star}_{i} \coloneqq \argminA\limits_{h \in \mathcal{M}_{i}} L(h) & & h^{\star}_{\mathcal{M}} \coloneqq \argminA\limits_{h \in \mathcal{M}} L(h),
\end{align*}
depending on the subset.

The set of hypotheses which minimize the empirical risk in $\mathcal{H}$ is defined as 
\begin{align}
	\label{ERMH}
	\hat{h}^{\mathcal{D}_{N}} \coloneqq \argminA\limits_{h \in \mathcal{H}} L_{\mathcal{D}_{N}}(h),
\end{align}	
while the ones that minimize it in models are denoted by
\begin{align}
	\label{ERMSub}
	\hat{h}_{i}^{\mathcal{D}_{N}} \coloneqq \argminA\limits_{h \in \mathcal{M}_{i}} L_{\mathcal{D}_{N}}(h) & & \hat{h}_{\mathcal{M}}^{\mathcal{D}_{N}} \coloneqq \argminA\limits_{h \in \mathcal{M}} L_{\mathcal{D}_{N}}(h).
\end{align}
We assume the minimum of $L$ and $L_{\mathcal{D}_{N}}$ is achieved in $\mathcal{H}$, and in all subsets of it that we consider throughout this paper, so the sets above are not empty. To ease notation, we may simply denote $\hat{h}$ as a hypothesis that minimizes the empirical risk in $\mathcal{H}$. In general, we denote hypotheses estimated via an algorithm $\mathbb{A}$ in $\mathcal{H}$ and its subsets by $\hat{h}^{\mathbb{A}}, \hat{h}_{i}^{\mathbb{A}}$ and $\hat{h}_{\mathcal{M}}^{\mathbb{A}}$. In the special case when $\mathbb{A}$ is given by empirical risk minimization in sample $\mathcal{D}_{N}$ we have the hypotheses defined in \eqref{ERMH} and \eqref{ERMSub}.

We denote a collection of candidate models by $\mathbb{C}(\mathcal{H}) = \{\mathcal{M}_{i}: i \in \mathcal{J}\}$, for $\mathcal{J} \subset \mathbb{Z}_{+}, \text{\textbar}\mathcal{J}\text{\textbar} < \infty$, and assume that it covers $\mathcal{H}$:
\begin{equation*}
	\mathcal{H} = \bigcup_{i \in \mathcal{J}} \mathcal{M}_{i}.
\end{equation*}
We define the VC dimension under loss function $\ell$ of such a collection as
\begin{equation*}
	\label{VCdimCand}
	d_{VC}(\mathbb{C}(\mathcal{H}),\ell) \coloneqq \max\limits_{i \in \mathcal{J}} d_{VC}(\mathcal{M}_{i},\ell)
\end{equation*}
and assume that $d_{VC}(\mathbb{C}(\mathcal{H}),\ell) < \infty$. Since every model in $\mathbb{C}(\mathcal{H})$ is a subset of $\mathcal{H}$ it follows that $d_{VC}(\mathbb{C}(\mathcal{H}),\ell) \leq d_{VC}(\mathcal{H},\ell)$. In Appendix \ref{apVCtheory} we review the main concepts of VC theory \cite{vapnik1998}. In particular, we define the VC dimension of a hypotheses space under loss function $\ell$ (cf. Definition \ref{VCdimension}). When the loss function is clear from the context, or not relevant to our argument, we denote the VC dimension simply by $d_{VC}(\mathcal{M})$ for $\mathcal{M} \subseteq \mathcal{H}$ and $d_{VC}(\mathbb{C}(\mathcal{H}))$.

\subsection{Model risk estimation}
\label{esti_Lhat}

The risk of a model in $\mathbb{C}(\mathcal{H})$ is defined as
\begin{equation*}
	L(\mathcal{M}) \coloneqq  \min\limits_{h \in \mathcal{M}} L(h) = L(h^{\star}_{\mathcal{M}}),
\end{equation*}
for $\mathcal{M} \in \mathbb{C}(\mathcal{H})$, and we consider estimators $\hat{L}(\mathcal{M})$ for $L(\mathcal{M})$ based on cross-validation. We assume that $\hat{L}$ is of the form
\begin{align}
	\label{form_Lhat}
	\hat{L}(\mathcal{M}) = \frac{1}{m} \ \sum_{j=1}^{m} \ \hat{L}^{(j)}(\hat{h}^{(j)}_{\mathcal{M}}), & & \mathcal{M} \in \mathbb{C}(\mathcal{H}),
\end{align}
in which there are $m$ pairs of independent training and validation samples, $\hat{L}^{(j)}$ is the empirical risk under the $j$-th validation sample, and $\hat{h}^{(j)}_{\mathcal{M}}$ is a hypothesis that minimizes the empirical risk in $\mathcal{M}$ under the $j$-th training sample, denoted by $\mathcal{D}_{N}^{(j)}$. We assume independence between samples within a pair $j$, but there may exist dependence between samples of distinct pairs $j,j^{\prime}$. All training and validation samples are subsets of $\mathcal{D}_{N}$. We assume all training samples have a size $N_{t}$ and the validation samples a size $N_{v}$.

In order to exemplify the results obtained in this paper, we consider two estimators of the form \eqref{form_Lhat}, obtained with a validation sample and k-fold cross validation, which we formally define below. When there is no need to specify which estimator of $L(\mathcal{M})$ we are referring to, we denote simply $\hat{L}(\mathcal{M})$ to mean an arbitrary estimator with form \eqref{form_Lhat}.

\subsubsection{Validation sample}

Fix a sequence $\{V_{N}: N \geq 1\}$ such that $\lim\limits_{N \rightarrow \infty} V_{N} = \lim\limits_{N \to \infty} N - V_{N} = \infty$, and let 
\begin{align*}
	\mathcal{D}_{N}^{(\text{train})} = \{Z_{l}: 1 \leq l \leq N - V_{N}\} & & \mathcal{D}_{N}^{(\text{val})} = \{Z_{l}: N - V_{N} < l \leq N\}
\end{align*}
be a split of $\mathcal{D}_{N}$ into a training and validation sample. The two samples $\mathcal{D}_{N}^{(\text{train})}$ and $\mathcal{D}_{N}^{(\text{val})}$ are independent. The estimator under the validation sample is given by
\begin{equation*}
	\label{VALLhat}
	\hat{L}_{\text{val}}(\mathcal{M}) \coloneqq L_{\mathcal{D}_{N}^{(\text{val})}}(\hat{h}_{\mathcal{M}}^{(\text{train})}) =  \frac{1}{V_{N}} \sum_{N- V_{N} < l \leq N}  \ell\big(Z_{l},\hat{h}_{\mathcal{M}}^{(\text{train})}\big),
\end{equation*}
in which
\begin{equation*}
	\hat{h}_{\mathcal{M}}^{(\text{train})} = \argminA_{h \in \mathcal{M}} L_{\mathcal{D}_{N}^{(\text{train})}}(h)
\end{equation*}
minimizes the empirical risk in $\mathcal{M}$ under $\mathcal{D}_{N}^{(\text{train})}$.

\subsubsection{K-fold cross-validation}

Fix $k \in \mathbb{Z}_{+}$ and assume $N \coloneqq kn$, for a $n \in \mathbb{Z}_{+}$. Then, let 
\begin{align*}
	\mathcal{D}_{N}^{(j)} \coloneqq \{Z_{l}: (j-1)n < l \leq jn\}, & & j = 1,\dots,k,
\end{align*}
be a partition of $\mathcal{D}_{N}$: 
\begin{align*}
	\mathcal{D}_{N} = \bigcup_{j=1}^{k} \mathcal{D}_{N}^{(j)} & & \text{ and } & & \mathcal{D}_{N}^{(j)} \cap \mathcal{D}_{N}^{(j^{\prime})} = \emptyset \text{ if } j \neq j^{\prime}.
\end{align*}
We define
\begin{equation*}
	\hat{h}^{(j)}_{\mathcal{M}} \coloneqq \argminA_{h \in \mathcal{M}} \ L_{\mathcal{D}_{N}\setminus\mathcal{D}_{N}^{(j)}}(h) = \argminA_{h \in \mathcal{M}} \ \frac{1}{(k-1)n} \sum_{\substack{l \leq (j-1)n \\ \cup \ l > jn}} \ell(Z_{l},h)
\end{equation*}
as the hypothesis which minimizes the empirical risk in $\mathcal{M}$ under the sample $\mathcal{D}_{N}\setminus\mathcal{D}_{N}^{(j)}$, that is the sample composed by all folds, but the $j$-th, and
\begin{equation*}
	\hat{L}_{\text{cv(k)}}^{(j)}(\mathcal{M}) \coloneqq L_{\mathcal{D}_{N}^{(j)}}(\hat{h}^{(j)}_{\mathcal{M}}) =  \frac{1}{n} \sum_{(j-1)n < l \leq jn} \ell(Z_{l},\hat{h}^{(j)}_{\mathcal{M}}),
\end{equation*}
as the validation risk of the $j$-th fold.

The k-fold cross-validation estimator of $L(\mathcal{M})$ is then given by
\begin{equation*}
	\label{CVLhat}
	\hat{L}_{\text{cv(k)}}(\mathcal{M}) \coloneqq \frac{1}{k} \ \sum_{j=1}^{k} \hat{L}_{\text{cv(k)}}^{(j)}(\mathcal{M}),
\end{equation*}
that is the average validation risk over the folds. 

\subsection{Target model and estimation errors}
\label{SecErrors}

The motivation for learning via model selection is presented in Figure \ref{paradigms}, in which the ellipses represent some models in $\mathbb{C}(\mathcal{H})$, and their area is proportional to their complexity, given for example by VC dimension. Assume that $\mathcal{H}$ is all we have to learn on, and we are not willing to consider any hypothesis outside $\mathcal{H}$. Then, if we could choose, we would like to learn on $\mathcal{M}^{\star}$: \textit{the least complex model in $\mathbb{C}(\mathcal{H})$ which contains a target hypothesis $h^{\star}$}. We call $\mathcal{M}^{\star}$ the target model.

In order to formally define the target model, we need to consider equivalence classes of models, as it is not possible to differentiate some models with the concepts of Statistical Learning Theory. Define in $\mathbb{C}(\mathcal{H})$ the equivalence relation given by
\begin{align}
	\label{equiv_class}
	\mathcal{M}_{i} \sim \mathcal{M}_{j} \text{ if, and only if, } d_{VC}(\mathcal{M}_{i}) = d_{VC}(\mathcal{M}_{j}) \text{ and } L(\mathcal{M}_{i}) = L(\mathcal{M}_{j}),
\end{align}
for $\mathcal{M}_{i}, \mathcal{M}_{j} \in \mathbb{C}(\mathcal{H})$: two models in $\mathbb{C}(\mathcal{H})$ are equivalent if they have the same VC dimension and risk. Let 
\begin{equation*}
	\mathcal{L}^{\star} = \argminA\limits_{\mathcal{M} \in \ \nicefrac{\mathbb{C}(\mathcal{H})}{\sim}} L(\mathcal{M})
\end{equation*}
be the equivalence classes which contain a target hypothesis of $\mathcal{H}$, so their risk is minimum. We define the target model $\mathcal{M}^{\star} \in \nicefrac{\mathbb{C}(\mathcal{H})}{\sim}$ as
\begin{equation*}
	\mathcal{M}^{\star} = \argminA\limits_{\mathcal{M} \in \mathcal{L}^{\star}} d_{VC}(\mathcal{M}),
\end{equation*}
which is the class of the smallest models in $\mathbb{C}(\mathcal{H})$, in the VC dimension sense, that are not disjoint with $h^{\star}$. The target model has the lowest complexity among the unbiased models in $\mathbb{C}(\mathcal{H})$.

The target model is dependent on both $\mathbb{C}(\mathcal{H})$ and the data generating distribution, so we cannot establish beforehand, without looking at data, on which model of $\mathbb{C}(\mathcal{H})$ to learn. Hence, in this context, a model selection procedure should, based on data, learn a model $\hat{\mathcal{M}}$ among the candidates $\mathbb{C}(\mathcal{H})$ as an estimator of $\mathcal{M}^{\star}$.

\begin{figure}[ht]
	\centering
	\includegraphics[width=0.75\linewidth]{New_Paradigm.png}
	\caption{Decomposition of $\mathcal{H}$ by a $\mathbb{C}(\mathcal{H})$. We omitted some models for a better visualization, since $\mathbb{C}(\mathcal{H})$ should cover $\mathcal{H}$.}
	\label{paradigms}
\end{figure}

However, when learning on a model $\hat{\mathcal{M}} \in \mathbb{C}(\mathcal{H})$ selected based on data, one commits three types of errors:
\begin{align*}
	\label{ee23}
	\textbf{(II)} \ \ L(\hat{h}_{\hat{\mathcal{M}}}^{\mathbb{A}}) - L(h^{\star}_{\hat{\mathcal{M}}}) & & \textbf{(III)} \ \ L(h^{\star}_{\hat{\mathcal{M}}}) - L(h^{\star}) & & \textbf{(IV)} \ \ L(\hat{h}_{\hat{\mathcal{M}}}^{\mathbb{A}}) - L(h^{\star}),
\end{align*}
that we call types II, III, and IV estimation errors\footnote{We define type I estimation error in Section \ref{SecLearnOn} (cf. \eqref{typeIe}).}, which are illustrated in Figure \ref{fig_error}. In a broad sense, type III estimation error would represent the bias of learning on $\hat{\mathcal{M}}$, while type II would represent the variance within $\hat{\mathcal{M}}$, and type IV would be the error, with respect to $\mathcal{H}$, committed when learning on $\hat{\mathcal{M}}$ with algorithm $\mathbb{A}$. 

Indeed, type III estimation error compares a target hypothesis $h_{\hat{\mathcal{M}}}^{\star}$ of $\hat{\mathcal{M}}$ with a target hypothesis $h^{\star}$ of $\mathcal{H}$, hence any difference between them would be a systematic bias of learning on $\hat{\mathcal{M}}$ when compared to learning on $\mathcal{H}$. Type II estimation error compares the loss of the estimated hypothesis $\hat{h}_{\hat{\mathcal{M}}}^{\mathbb{A}}$ of $\hat{\mathcal{M}}$ and the loss of its target, assessing how much the estimated hypothesis varies from a target of $\hat{\mathcal{M}}$, while type IV is the effective error committed, since compares the estimated hypothesis of $\hat{\mathcal{M}}$ with a target of $\mathcal{H}$. 

As is often the case, there will be a bias-variance trade-off that should be minded when learning on $\hat{\mathcal{M}}$, so it is important to guarantee that, when the sample size increases, all the estimation errors tend to zero. Furthermore, it is desired that the learned model $\hat{\mathcal{M}}$ converges to the target model $\mathcal{M}^{\star}$ with probability one, so learning is asymptotically optimal. The proposed learning framework via model selection defined in the next section will take these desires into account.

\subsection{Learning hypotheses via model selection}
\label{LearningFramework}

Learning via model selection is composed of two steps: first learn a model $\hat{\mathcal{M}}$ from $\mathbb{C}(\mathcal{H})$ and then learn a hypothesis on $\hat{\mathcal{M}}$. In this section, we define $\hat{\mathcal{M}}$ and two algorithms $\mathbb{A}$ to learn on it.

\subsubsection{Learning model $\hat{\mathcal{M}}$}

Model selection is performed by applying a $(\Omega,\mathcal{S})$-measurable function $\mathbb{M}_{\mathbb{C}(\mathcal{H})}$, dependent on $\mathbb{C}(\mathcal{H})$, satisfying
\begin{equation}
	\label{diagram}
	\omega \in \Omega \xrightarrow{(\mathcal{D}_{N},\hat{L})} (\mathcal{D}_{N}(\omega),\hat{L}(\omega)) \xrightarrow{\ \ \mathbb{M}_{\mathbb{C}(\mathcal{H})} \ \ } \mathcal{\hat{M}}(\omega) \in \mathbb{C}(\mathcal{H}),
\end{equation}
which is such that, given $\mathcal{D}_{N}$ and an estimator $\hat{L}$ of the risk of each candidate model, learns a $\mathcal{\hat{M}} \in \mathbb{C}(\mathcal{H})$. Note from (\ref{diagram}) that $\mathcal{\hat{M}}$ is a $(\Omega,\mathcal{S})$-measurable $\mathbb{C}(\mathcal{H})$-valued function, as it is the composition of measurable functions, i.e., $\mathcal{\hat{M}} \coloneqq \mathcal{\hat{M}}_{\mathcal{D}_{N},\hat{L},\mathbb{C}(\mathcal{H})} = \mathbb{M}_{\mathbb{C}(\mathcal{H})}\big(\mathcal{D}_{N},\hat{L}\big)$. Even though $\mathcal{\hat{M}}$ depends on $\mathcal{D}_{N}, \hat{L}$ and $\mathbb{C}(\mathcal{H})$, we drop the subscripts to ease notation. 

We are interested in a $\mathbb{M}_{\mathbb{C}(\mathcal{H})}$ such that
\begin{equation}
	\label{consistent_LM}
	\mathcal{\hat{M}} = \mathbb{M}_{\mathbb{C}(\mathcal{H})}\big(\mathcal{D}_{N},\hat{L}\big) \xrightarrow{N \rightarrow \infty} \mathcal{M}^{\star} \text{ with probability one}.
\end{equation}
Actually, it is desired the model learned by $\mathbb{M}_{\mathbb{C}(\mathcal{H})}$ to be as simple as it can be under the restriction that it converges to the target model. 

A $\mathbb{M}_{\mathbb{C}(\mathcal{H})}$ which satisfies (\ref{consistent_LM}) may be defined by mimicking the definition of $\mathcal{M}^{\star}$, but employing the estimated risk $\hat{L}$ instead of $L$. Define in $\mathbb{C}(\mathcal{H})$ the equivalence relation given by
\begin{align*}
	\mathcal{M}_{i} \hat{\sim} \mathcal{M}_{j} \text{ if, and only if, } d_{VC}(\mathcal{M}_{i}) = d_{VC}(\mathcal{M}_{j}) \text{ and } \hat{L}(\mathcal{M}_{i}) = \hat{L}(\mathcal{M}_{j}),
\end{align*}
for $\mathcal{M}_{i}, \mathcal{M}_{j} \in \mathbb{C}(\mathcal{H})$, which is a random $(\Omega,\mathcal{S})$-measurable equivalence relation. Let
\begin{equation*}
	\hat{\mathcal{L}} = \argminA\limits_{\mathcal{M} \in \ \nicefrac{\mathbb{C}(\mathcal{H})}{\hat{\sim}}} \hat{L}(\mathcal{M})
\end{equation*}
be the classes in $\nicefrac{\mathbb{C}(\mathcal{H})}{\hat{\sim}}$ with the least estimated risk. We call the classes in $\hat{\mathcal{L}}$ the global minimums of $\mathbb{C}(\mathcal{H})$. Then, $\mathbb{M}_{\mathbb{C}(\mathcal{H})}$ selects
\begin{equation}
	\label{Ghat}
	\hat{\mathcal{M}} = \argminA\limits_{\mathcal{M} \in \mathcal{\hat{L}}} d_{VC}(\mathcal{M}),
\end{equation} 
the simplest class among the global minimums.

\subsubsection{Learning hypotheses on $\hat{\mathcal{M}}$}
\label{SecLearnOn}

Once $\hat{\mathcal{M}}$ is selected, we need to learn hypotheses on it. In this paper, we focus on learning with an independent sample and briefly discuss learning by reusing, as follows. 

Let $\tilde{\mathcal{D}}_{M} = \{\tilde{Z}_{l}: 1 \leq l \leq M\}$ be a sequence of $M$ independent and identically distributed random vectors with distribution $P$, independent of $\mathcal{D}_{N}$. When learning with an independent sample, we consider
\begin{align*}
	%\label{learn_inde}
	\hat{h}_{\hat{\mathcal{M}}}^{\tilde{\mathcal{D}}_{M}} \coloneqq \argminA\limits_{h \in \hat{\mathcal{M}}} L_{\tilde{\mathcal{D}}_{M}}(h),
\end{align*}
that are the hypotheses which minimize the empirical risk under $\tilde{\mathcal{D}}_{M}$ on $\hat{\mathcal{M}}$. 

Another straightforward manner of learning on $\hat{\mathcal{M}}$ is to simply consider
\begin{align}
	\label{learn_reuse}
	\hat{h}_{\hat{\mathcal{M}}}^{\mathcal{D}_{N}} \coloneqq \argminA\limits_{h \in \hat{\mathcal{M}}} L_{\mathcal{D}_{N}}(h),
\end{align}
that are the hypotheses which minimize the empirical error under $\mathcal{D}_{N}$ on $\hat{\mathcal{M}}$. We call this framework \textit{learning by reusing}. Figure \ref{learn_hyp} summarizes these systematic frameworks of learning via model selection.

\begin{figure*}[ht]
	\centering
	\includegraphics[width=\linewidth]{learn_hyp1}	
	\caption{The systematic frameworks for learning hypotheses via model selection.(a) A sample of size $N+M$ is split into two, one of size $N$ that is used to estimate $\hat{\mathcal{M}}$ by the minimization of $\hat{L}$ on $\mathbb{C}(\mathcal{H})$, and another of size $M$ is used to learn a hypothesis on $\hat{\mathcal{M}}$ by the minimization of the empirical risk. (b) The whole sample of size $N$ is used for estimating $\hat{\mathcal{M}}$ by the minimization of $\hat{L}$ on $\mathbb{C}(\mathcal{H})$, and to estimate hypotheses on $\hat{\mathcal{M}}$ via ERM.}
	\label{learn_hyp}
\end{figure*}

We define the type I estimation error as
\begin{align}
	\label{typeIe}
	\textbf{(I)} \begin{cases}
		\sup\limits_{h \in \hat{\mathcal{M}}} \left|L_{\tilde{\mathcal{D}}_{M}}(h) - L(h)\right| & \text{if learning with independent sample}\\
		\sup\limits_{h \in \hat{\mathcal{M}}} \left|L_{\mathcal{D}_{N}}(h) - L(h)\right| & \text{if learning by reusing}
	\end{cases},
\end{align}
which represents how well one can estimate the loss uniformly on $\hat{\mathcal{M}}$ by the empirical risk under $\tilde{\mathcal{D}}_{M}$ and $\mathcal{D}_{N}$, respectively.

\begin{remark}
	We assume that the supremum in \eqref{typeIe} are $(\Omega,\mathcal{S})$-measurable, so it is meaningful to calculate probabilities of events which involve them. We also assume throughout this paper that these supremum, over any fixed $\mathcal{M} \in \mathbb{C}(\mathcal{H}),$ are also $(\Omega,\mathcal{S})$-measurable.
\end{remark}

\subsection{Deviation bounds and main results}

In classical learning theory, or VC theory, there are two kinds of estimation errors, whose tail probabilities are
\begin{align}
	\label{GE1}
	\mathbb{P}&\left(\sup\limits_{h \in \mathcal{H}} \text{\textbar}L_{\mathcal{D}_{N}}(h) - L(h)\text{\textbar} > \epsilon\right) 
\end{align}
and
\begin{align}
	\label{GE2}
	\mathbb{P}&\left(L(\hat{h}^{\mathcal{D}_{N}}) - L(h^{\star}) > \epsilon\right),
\end{align}
for $\epsilon > 0$. In the terminology of this paper, they are called, respectively, type I and II estimation error, when the target hypotheses of $\mathcal{H}$ are estimated by minimizing the empirical risk under sample $\mathcal{D}_{N}$.

When the loss function is bounded, the rate of convergence of \eqref{GE2} to zero is decreasing on the VC dimension of $\mathcal{H}$. This is the main result of VC theory, which may be stated as follows, and is a consequence of Corollaries \ref{cor3TypeI} and \ref{cor1TypeII}. Observe that the bounds do not depend on $P$, and are valid for any distribution $Z$ may have, that is, are distribution-free. A result analogous to Proposition \ref{propVC} will be stated for unbounded loss functions in Section \ref{SecUnbounded}. In what follows, a.s. stands for almost sure convergence or convergence with probability one.

\begin{proposition}
	\label{propVC}
	Assume the loss function is bounded. Fixed a hypotheses space $\mathcal{H}$ with $d_{VC}(\mathcal{H}) < \infty$, there exist sequences $\{B^{I}_{N,\epsilon}: N \geq 1\}$ and $\{B^{II}_{N,\epsilon}: N \geq 1\}$ of positive real-valued increasing functions with domain $\mathbb{Z}_{+}$ satisfying
	\begin{equation*}
		\lim\limits_{N \to \infty} B^{I}_{N,\epsilon}(k) = \lim\limits_{N \to \infty} B^{II}_{N,\epsilon}(k) = 0,
	\end{equation*}
	for all $\epsilon > 0$ and $k \in \mathbb{Z}_{+}$ fixed, such that
	\begin{align*}
		\mathbb{P}&\left(\sup\limits_{h \in \mathcal{H}} \text{\textbar}L_{\mathcal{D}_{N}}(h) - L(h)\text{\textbar} > \epsilon\right) \leq B^{I}_{N,\epsilon}(d_{VC}(\mathcal{H}))\\
		\mathbb{P}&\left(L(\hat{h}^{\mathcal{D}_{N}}) - L(h^{\star}) > \epsilon\right) \leq B^{II}_{N,\epsilon}(d_{VC}(\mathcal{H})).
	\end{align*}	
	Furthermore, the following holds:
	\begin{align*}
		&\sup\limits_{h \in \mathcal{H}} \text{\textbar}L_{\mathcal{D}_{N}}(h) - L(h)\text{\textbar} \xrightarrow[N \to \infty]{a.s.} 0 & & L(\hat{h}^{\mathcal{D}_{N}}) - L(h^{\star}) \xrightarrow[N \to \infty]{a.s.} 0.
	\end{align*}
\end{proposition}

The sequences $\{B^{I}_{N,\epsilon}: N \geq 1\}$ and $\{B^{II}_{N,\epsilon}: N \geq 1\}$ are what we call the deviation bounds of learning via empirical risk minimization in $\mathcal{H}$. The main results of this paper are the convergence of $\hat{\mathcal{M}}$ to $\mathcal{M}^{\star}$ with probability one and deviation bounds for types I, II, III, and IV estimation errors when learning via model selection with bounded or unbounded loss function.

In particular, it will follow that the established deviation bounds for the type IV estimation error of learning via model selection may be tighter than that for the type II estimation error of learning directly on $\mathcal{H}$ via empirical risk minimization, and hence one may have a lower risk by learning via model selection. This means that by introducing a bias III, which converges to zero, we may decrease the variance II of the learning process, so it is more efficient to learn via model selection. In Section \ref{SecEnhanceGen}, we further discuss when this is the case and how domain knowledge can be leveraged to choose a class of candidate models in which learning via model selection may be, in general, better than learning via ERM in $\mathcal{H}$.

In the following sections, we treat the cases of bounded and unbounded loss functions.

\begin{remark}
	Throughout this paper, we assume that functions such as $B_{N,\epsilon}^{I}(k)$ and $B_{N,\epsilon}^{II}(k)$ are decreasing on $\epsilon$ and $N$ for $k$ fixed.
\end{remark}

\begin{remark}
	The results of this paper for bounded loss functions hold for any distribution-free complexity, such that Proposition \ref{propVC} remains true. In fact, the results hold by assuming the existence of bounds $B_{N,\epsilon}^{I}(d(\mathcal{H}))$ and $B_{N,\epsilon}^{II}(d(\mathcal{H}))$ for \eqref{GE1} and \eqref{GE2} depending on a distribution-free complexity $d(\mathcal{H})$. For instance, bounds based on the Radamacher and Gaussian complexities \cite{bartlett2002rademacher} or the fat-shattering dimension \cite{bartlett1994fat} could be considered. We stated the results for the VC-dimension to be consistent with the unbounded loss results, since in that case, we could only show that similar bounds hold for the VC dimension (cf. Proposition \ref{propVC2}) by extending recent results \cite{cortes2019}. Showing analogous bounds for other complexities is an open problem in that case.
\end{remark}

\section{Learning via model selection with bounded loss functions}
\label{boundedL}

In Section \ref{SecConvTM}, we show the convergence of $\hat{\mathcal{M}}$ to the target model $\mathcal{M}^{\star}$ with probability one, and in Section \ref{SecConvOnMhat} we establish deviation bounds for the estimation errors of learning via model selection when the loss function is bounded. From now on, we assume there exists a constant $C > 0$ such that
\begin{align*}
	0 \leq \ell(z,h) \leq C & & \text{ for all } z \in \mathcal{Z}, h \in \mathcal{H}.
\end{align*}
In this section, we focus on learning with an independent sample and in Section \ref{SecReuse} we briefly present analogous results for learning by reusing.

\subsection{Convergence to the target model}
\label{SecConvTM}

We start by studying a result weaker than the convergence of $\hat{\mathcal{M}}$ to $\mathcal{M}^{\star}$, that is, the convergence of $L(\hat{\mathcal{M}})$ to $L(\mathcal{M}^{\star})$.

In order to have $L(\hat{\mathcal{M}}) = L(\mathcal{M}^{\star})$, one does not need to know exactly $L(\mathcal{M})$ for all $\mathcal{M} \in \mathbb{C}(\mathcal{H})$, i.e., one does not need $\hat{L}(\mathcal{M}) = L(\mathcal{M})$, for all $\mathcal{M} \in \mathbb{C}(\mathcal{H})$. We argue that it suffices to have $\hat{L}(\mathcal{M})$ close enough to $L(\mathcal{M})$, for all $\mathcal{M} \in \mathbb{C}(\mathcal{H})$, so the global minimums of $\mathbb{C}(\mathcal{H})$ will have the same risk as $\mathcal{M}^{\star}$, even if it is not possible to properly estimate their risk. This ``close enough'' depends on $P$, hence is not distribution-free, and is given by the \textit{maximum discrimination error} (MDE) of $\mathbb{C}(\mathcal{H})$ under $P$, which we define as
\begin{equation*}
	\epsilon^{\star} = \epsilon^{\star}(\mathbb{C}(\mathcal{H}),P) \coloneqq \min\limits_{\substack{\mathcal{M} \in \mathbb{C}(\mathcal{H})\\L(\mathcal{M}) > L(\mathcal{M}^{\star})}} L(\mathcal{M}) - L(\mathcal{M}^{\star}).
\end{equation*}

The MDE is the minimum difference between the out-of-sample risk of a target hypothesis and the best hypothesis in a model which does not contain a target. In other words, it is the difference between the risk of the best model $\mathcal{M}^{\star}$ and the second best. The meaning of $\epsilon^{\star}$ is depicted in Figure \ref{epsilonstar}.

\begin{figure}[ht]
	\centering
	\includegraphics[width=\linewidth]{epsilonstar}
	\caption{The risks of the equivalence classes (cf. \eqref{equiv_class}) of $\mathbb{C}(\mathcal{H})$ in ascending order. The MDE $\epsilon^{\star}$ is the difference between the risk of the target class $\mathcal{M}^{\star}$, and the second best $\mathcal{M}_{2}$. The colored intervals represent a distance of $\epsilon^{\star}/2$ from the out-of-sample risk of each model, and the colored estimated risks $\hat{L}$ illustrate a case such that the estimated risk is within $\epsilon^{\star}/2$ of the out-of-sample risk for all models. The class $\mathcal{M}_{1}$ has the same risk as $\mathcal{M}^{\star}$, but has a smaller estimated risk, and, by the definition of $\mathcal{M}^{\star}$, greater VC dimension. Note from the representation that, if one can estimate $\hat{L}$ within a margin of error of $\epsilon^{\star}/2$, then $\hat{\mathcal{M}}$ will be a model with the same risk as $\mathcal{M}^{\star}$, in this case $\mathcal{M}_{1}$ (cf. Proposition \ref{proposition_principal}).} \label{epsilonstar}
\end{figure}

The MDE is defined only if there exists at least one $\mathcal{M} \in \mathbb{C}(\mathcal{H})$ such that $h^{\star} \cap \mathcal{M} = \emptyset$, i.e., there is a subset in $\mathbb{C}(\mathcal{H})$ which does not contain a target hypothesis. If  $h^{\star} \cap \mathcal{M} \neq \emptyset$ for all $\mathcal{M} \in \mathbb{C}(\mathcal{H})$, then type III estimation error is zero, and type IV reduces to type II. From this point, we assume that $\epsilon^{\star}$ is well-defined.

The terminology MDE is used for we can show that a fraction of $\epsilon^{\star}$ is the greatest error one can commit when estimating $L(\mathcal{M})$ by $\hat{L}(\mathcal{M})$, for all $\mathcal{M} \in \mathbb{C}(\mathcal{H})$, in order for $L(\mathcal{\hat{M}})$ to be equal to $L(\mathcal{M}^{\star})$. This is the result of the next proposition.

\begin{proposition}
	\label{proposition_principal}
	Assume there exists $\delta > 0$ such that
	\begin{equation}
		\label{cond_prop_principal}
		\mathbb{P}\left(\max\limits_{i \in \mathcal{J}} \text{\textbar}L(\mathcal{M}_{i}) - \hat{L}(\mathcal{M}_{i})\text{\textbar} < \epsilon^{\star}/2\right) \geq 1 - \delta.
	\end{equation}
	Then
	\begin{equation}
		\label{prob_equal}
		\mathbb{P}\left(L(\mathcal{\hat{M}}) = L(\mathcal{M}^{\star})\right) \geq 1-\delta.
	\end{equation}	
\end{proposition}


\begin{remark}
	Since there may exist $\mathcal{M} \in \ \nicefrac{\mathbb{C}(\mathcal{H})}{\sim}$ with $L(\mathcal{M}) = L(\mathcal{\mathcal{M}^{\star}})$ and $d_{VC}(\mathcal{M}) > d_{VC}(\mathcal{M}^{\star})$, condition \eqref{cond_prop_principal} guarantees only that the estimated risk of both $\mathcal{M}$ and $\mathcal{M}^{\star}$ is lesser than the estimated risk of any model with risk greater than theirs, but it may happen that $\hat{L}(\mathcal{M}) < \hat{L}(\mathcal{M}^{\star})$ (see Figure \ref{epsilonstar} for an example). In this instance, we have $\hat{\mathcal{M}} = \mathcal{M}$ and $L(\hat{\mathcal{M}}) = L(\mathcal{M}^{\star})$.
\end{remark}

Recall we are assuming that $\hat{L}$ is of the form \eqref{form_Lhat}. In this case, we may obtain a bound for \eqref{prob_equal} depending on $\epsilon^{\star}$, on $d_{VC}(\mathbb{C}(\mathcal{H}))$, and on bounds for tail probabilities of type I estimation error under each validation and training sample (cf. Proposition \ref{propVC}). These bounds also depend on the number of maximal models of $\mathbb{C}(\mathcal{H})$, which are models in
\begin{align*}
	\text{Max } \mathbb{C}(\mathcal{H}) = \left\{\mathcal{M} \in \mathbb{C}(\mathcal{H}): \text{ if } \mathcal{M} \subset \mathcal{M}^{\prime} \in \mathbb{C}(\mathcal{H}) \text{ then } \mathcal{M} = \mathcal{M}^{\prime} \right\},
\end{align*}
that are models not contained in any element in $\mathbb{C}(\mathcal{H})$ besides themselves. We denote
\begin{equation*}
	\mathfrak{m}(\mathbb{C}(\mathcal{H})) = \text{\textbar}\text{Max } \mathbb{C}(\mathcal{H})\text{\textbar},
\end{equation*}
the number of maximal models in $\mathbb{C}(\mathcal{H})$. We have the following rate of convergence of $L(\hat{\mathcal{M}})$ to $L(\mathcal{M}^{\star})$, and condition for $\hat{\mathcal{M}}$ to converge to $\mathcal{M}^{\star}$ with probability one.

\begin{theorem}
	\label{theorem_principal_convergence}
	Assume the loss function is bounded. For each $\epsilon > 0$, let $\{B_{N,\epsilon}: N \geq 1\}$ and $\{\hat{B}_{N,\epsilon}: N \geq 1\}$ be sequences of positive real-valued increasing functions with domain $\mathbb{Z}_{+}$ satisfying
	\begin{equation*}
		\lim\limits_{N \to \infty} B_{N,\epsilon}(k) = \lim\limits_{N \to \infty} \hat{B}_{N,\epsilon}(k) = 0,
	\end{equation*}
	for all $\epsilon > 0$ and $k \in \mathbb{Z}_{+}$ fixed, and such that
	\begin{align*}
		\max_{j} \mathbb{P}\left(\sup\limits_{h \in \mathcal{M}} \text{\textbar}L_{\mathcal{D}_{N}^{(j)}}(h) - L(h)\text{\textbar} > \epsilon \right) \leq B_{N_{t},\epsilon}(d_{VC}(\mathcal{M})) & \text{ and } &\\ \nonumber 
		\max_{j} \mathbb{P}\left(\sup\limits_{h \in \mathcal{M}} \text{\textbar}\hat{L}^{(j)}(h) - L(h)\text{\textbar} > \epsilon \right) \leq \hat{B}_{N_{v},\epsilon}(d_{VC}(\mathcal{M})), & & 
	\end{align*}
	for all $\mathcal{M} \in \mathbb{C}(\mathcal{M})$, recalling that $L_{\mathcal{D}_{N}^{(j)}}$ and $\hat{L}^{(j)}$ represent the empirical risk under the $j$-th training and validation samples, respectively. Let $\mathcal{\hat{M}} \in \mathbb{C}(\mathcal{H})$ be a random model learned by $\mathbb{M}_{\mathbb{C}(\mathcal{H})}$. Then,
	\begin{align}
		\label{bound_pM} \nonumber
		\mathbb{P}\left(L(\hat{\mathcal{M}}) \neq L(\mathcal{M}^{\star})\right) &\leq m \sum_{\mathcal{M} \in \text{Max } \mathbb{C}(\mathcal{H})} \left[B_{N_{t},\epsilon^{\star}/8}(d_{VC}(\mathcal{M})) + \hat{B}_{N_{v},\epsilon^{\star}/4}(d_{VC}(\mathcal{M}))\right]\\
		&\leq m \ \mathfrak{m}(\mathbb{C}(\mathcal{H})) \left[B_{N_{t},\epsilon^{\star}/8}(d_{VC}(\mathbb{C}(\mathcal{H}))) + \hat{B}_{N_{v},\epsilon^{\star}/4}(d_{VC}(\mathbb{C}(\mathcal{H})))\right],
	\end{align}
	in which $m$ is the number of pairs considered to calculate \eqref{form_Lhat}. Furthermore, if
	\begin{align}
		\label{as_conv}
		\max_{\mathcal{M} \in \mathbb{C}(\mathcal{H})} \max_{j} \sup\limits_{h \in \mathcal{M}} \text{\textbar}L_{\mathcal{D}_{N}^{(j)}}(h) - L(h)\text{\textbar} \xrightarrow{\text{a.s.}} 0 & \text{ and } &\\ \nonumber
		\max_{\mathcal{M} \in \mathbb{C}(\mathcal{H})} \max_{j} \sup\limits_{h \in \mathcal{M}} \text{\textbar}\hat{L}^{(j)}(h) - L(h)\text{\textbar} \xrightarrow{\text{a.s.}} 0, & & 
	\end{align}
	then
	\begin{equation*}
		\lim_{N \rightarrow \infty} \mathbb{P}\left(\hat{\mathcal{M}} = \mathcal{M}^{\star}\right) = 1.
	\end{equation*}
\end{theorem}

A bound for $\mathbb{P}(L(\hat{\mathcal{M}}) \neq L(\mathcal{M}^{\star}))$, and the almost sure convergence of $\hat{\mathcal{M}}$ to $\mathcal{M}^{\star}$ in the case of k-fold cross validation, follow from Proposition \ref{propVC}, recalling that the sample size in each training and validation sample is $N_{t} = (k-1)n$ and $N_{v} = n$, respectively, with $N = kn$. Analogously, we may obtain a bound when an independent validation sample is considered. This result is stated in the next theorem.

\begin{theorem}
	\label{CVModelconvergence}
	Assume the loss function is bounded. If $\hat{L}$ is given by k-fold cross-validation or by an independent validation sample, then $\hat{\mathcal{M}}$ converges with probability one to $\mathcal{M}^{\star}$.
\end{theorem}

Since $B_{N,\epsilon}(k)$ and $\hat{B}_{N,\epsilon}(k)$ are decreasing on $\epsilon$ and $N$ for $k$ fixed, it follows from \eqref{bound_pM} that a tighter bound for $\mathbb{P}(L(\hat{\mathcal{M}}) \neq L(\mathcal{M}^{\star}))$ is obtained if the training sample size is greater than the validation sample size ($N_{t} > N_{v}$) since the functions $B_{N_{t},\epsilon^{\star}/8}$ and $\hat{B}_{N_{v},\epsilon^{\star}/4}$ appear on the bound. Observe this is the case in k-fold cross-validation.

Moreover, from this bound it follows that, with a fixed sample size, we can have a tighter bound for $\mathbb{P}(L(\hat{\mathcal{M}}) \neq L(\mathcal{M}^{\star}))$ by choosing a family of candidate models with small $d_{VC}(\mathbb{C}(\mathcal{H}))$ and few maximal elements, while attempting to increase $\epsilon^{\star}$. Of course, there is a trade-off between $d_{VC}(\mathbb{C}(\mathcal{H}))$ and the number of maximal elements of $\mathbb{C}(\mathcal{H})$, the only known free quantities in bound $\eqref{bound_pM}$, since the sample size is fixed and $\epsilon^{\star}$ is unknown.

\subsection{Deviation bounds for estimation errors on $\hat{\mathcal{M}}$}
\label{SecConvOnMhat}

Bounds for types I and II estimation errors when learning on a random model with a sample independent of the one employed to compute such random model, may be obtained when there is a bound for them on each $\mathcal{M} \in \mathbb{C}(\mathcal{H})$ under the independent sample. This is the content of Theorem \ref{bound_constant}.

\begin{theorem}
	\label{bound_constant}	
	Fix a bounded loss function. Assume we are learning with an independent sample $\tilde{\mathcal{D}}_{M}$, and that for each $\epsilon > 0$ there exist sequences $\{B^{I}_{M,\epsilon}: M \geq 1\}$ and $\{B^{II}_{M,\epsilon}: M \geq 1\}$ of positive real-valued increasing functions with domain $\mathbb{Z}_{+}$ satisfying
	\begin{equation*}
		\lim\limits_{M \to \infty} B^{I}_{M,\epsilon}(k) = \lim\limits_{M \to \infty} B^{II}_{M,\epsilon}(k) = 0,
	\end{equation*}
	for all $\epsilon > 0$ and $k \in \mathbb{Z}_{+}$ fixed, such that
	\begin{align}
		\label{bound_theoremBC}
		\mathbb{P}\left(\sup\limits_{h \in \mathcal{M}} \text{\textbar}L_{\tilde{\mathcal{D}}_{M}}(h) - L(h) \text{\textbar} > \epsilon \right) \leq B^{I}_{M,\epsilon}(d_{VC}(\mathcal{M})) & \text{ and } &\\ \nonumber %\\ \nonumber
		\mathbb{P}\left(L(\hat{h}_{\mathcal{M}}^{\tilde{\mathcal{D}}_{M}}) - L(h^{\star}_{\mathcal{M}}) > \epsilon \right) \leq B^{II}_{M,\epsilon}(d_{VC}(\mathcal{M})), & & 
	\end{align}
	for all $\mathcal{M} \in \mathbb{C}(\mathcal{H})$. Let $\mathcal{\hat{M}} \in \mathbb{C}(\mathcal{H})$ be a random model learned by $\mathbb{M}_{\mathbb{C}(\mathcal{H})}$. Then, for any $\epsilon > 0$,
	\begin{align*}
		\textbf{(I)} \ \mathbb{P}&\left(\sup\limits_{h \in \mathcal{\hat{M}}} \text{\textbar}L_{\tilde{\mathcal{D}}_{M}}(h) - L(h) \text{\textbar} > \epsilon \right) \leq \mathbb{E}\Big[B^{I}_{M,\epsilon}(d_{VC}(\mathcal{\hat{M}}))\Big] \leq B^{I}_{M,\epsilon}\left(d_{VC}(\mathbb{C}(\mathcal{H}))\right)
	\end{align*}
	and
	\begin{align*}
		\textbf{(II)} \ \mathbb{P}\left(L(\hat{h}_{\mathcal{\hat{M}}}^{\tilde{\mathcal{D}}_{M}}) - L(h^{\star}_{\mathcal{\hat{M}}}) > \epsilon \right) \leq \mathbb{E}\Big[B^{II}_{M,\epsilon}(d_{VC}(\mathcal{\hat{M}}))\Big] \leq B^{II}_{M,\epsilon}\left(d_{VC}(\mathbb{C}(\mathcal{H}))\right),
	\end{align*}
	in which the expectations are over all samples $\mathcal{D}_{N}$, from which $\hat{\mathcal{M}}$ is calculated. Since $d_{VC}(\mathbb{C}(\mathcal{H})) < \infty$, both probabilities above converge to zero when $M \to \infty$.
\end{theorem}

Our definition of $\mathcal{\hat{M}}$ ensures that it is going to have the smallest VC dimension under the constraint that it is a global minimum of $\mathbb{C}(\mathcal{H})$. As the quantities inside the expectations of Theorem \ref{bound_constant} are increasing functions of VC dimension, fixed $\epsilon$ and $M$, we tend to have smaller expectations, thus tighter bounds for types I and II estimation errors. Furthermore, it follows from Theorem \ref{bound_constant} that the sample complexity needed to learn on $\hat{\mathcal{M}}$ is at most that of $d_{VC}(\mathbb{C}(\mathcal{H}))$. This implies that this complexity is at most that of $\mathcal{H}$, but may be much lesser if $d_{VC}(\mathbb{C}(\mathcal{H})) \ll d_{VC}(\mathcal{H})$. 

A bound for type III estimation error may be obtained using methods similar to that we employed to prove Theorem \ref{theorem_principal_convergence}. As in that theorem, the bound for type III estimation error depends on $\epsilon^{\star}$, on bounds for type I estimation error under each training and validation sample, and on $\mathbb{C}(\mathcal{H})$, more specifically, on its VC dimension and number of maximal elements. To ease notation, we denote $\epsilon \vee \epsilon^{\star} \coloneqq \max \{\epsilon,\epsilon^{\star}\}$ for any $\epsilon > 0$.

\begin{theorem}
	\label{theorem_tipeIII}
	Assume the premises of Theorem \ref{theorem_principal_convergence} are in force. Let $\mathcal{\hat{M}} \in \mathbb{C}(\mathcal{H})$ be a random model learned by $\mathbb{M}_{\mathbb{C}(\mathcal{H})}$. Then, for any $\epsilon > 0$,
	\begin{align*}
		\textbf{(III)} &\ \mathbb{P}\left(L(h_{\hat{\mathcal{M}}}^{\star}) - L(h^{\star}) > \epsilon\right) \\
		& \leq m \sum_{\mathcal{M} \in \text{Max } \mathbb{C}(\mathcal{H})} \left[ B_{N_{t},(\epsilon \vee \epsilon^{\star})/8}(d_{VC}(\mathcal{M})) + \hat{B}_{N_{v},(\epsilon \vee \epsilon^{\star})/4}(d_{VC}(\mathcal{M}))\right]\\
		&\leq m \ \mathfrak{m}(\mathbb{C}(\mathcal{H})) \left[ B_{N_{t},(\epsilon \vee \epsilon^{\star})/8}(d_{VC}(\mathbb{C}(\mathcal{H}))) + \hat{B}_{N_{v},(\epsilon \vee \epsilon^{\star})/4}(d_{VC}(\mathbb{C}(\mathcal{H})))\right].
	\end{align*}
	In particular,
	\begin{align*}
		\lim_{N \rightarrow \infty} \mathbb{P}\left(L(h_{\hat{\mathcal{M}}}^{\star}) - L(h^{\star}) > \epsilon\right) = 0,
	\end{align*}
	for any $\epsilon > 0$.
\end{theorem}

\begin{remark}
	\label{remReuse}
	Type III estimation error, and its bound presented in Theorem \ref{theorem_tipeIII}, do not depend on the algorithm $\mathbb{A}$ employed to learn on $\hat{\mathcal{M}}$, hence this theorem is true for both frameworks in Figure \ref{learn_hyp}, holding also when learning by reusing.
\end{remark}

On the one hand, by definition of $\epsilon^{\star}$, if $\epsilon < \epsilon^{\star}$, then type III estimation error is lesser than $\epsilon$ if, and only if, $L(\hat{\mathcal{M}}) = L(\mathcal{M}^{\star})$, so this error is actually zero, and the result of Theorem \ref{theorem_principal_convergence} is a bound for type III estimation error in this case. On the other hand, if $\epsilon > \epsilon^{\star}$, one way of having type III estimation error lesser than $\epsilon$ is to have the estimated risk of each $\mathcal{M}$ at a distance at most $\epsilon/2$ from its out-of-sample risk and, as can be inferred from the proof of Theorem \ref{theorem_principal_convergence}, this can be accomplished if one has type I estimation error not greater than a fraction of $\epsilon$ under each training and validation sample considered, so a modification of Theorem \ref{theorem_principal_convergence} applies to this case.

Finally, as the tail probability of type IV estimation error may be bounded by the following inequality, involving the tail probabilities of types II and III estimation errors,
\begin{align}
	\label{triangle} \nonumber
	\textbf{(IV)} \ \mathbb{P}&\left(L(\hat{h}_{\mathcal{\hat{M}}}^{\tilde{\mathcal{D}}_{M}}) - L(h^{\star}) > \epsilon\right)\\& \leq \mathbb{P}\left(L(\hat{h}_{\mathcal{\hat{M}}}^{\tilde{\mathcal{D}}_{M}}) - L(h^{\star}_{\mathcal{\hat{M}}}) > \epsilon/2\right) + \mathbb{P}\left(L(h^{\star}_{\mathcal{\hat{M}}}) - L(h^{\star}) > \epsilon/2\right),
\end{align}
a bound for \eqref{triangle} is a direct consequence of Theorems \ref{bound_constant} and \ref{theorem_tipeIII}.

\begin{corollary}
	\label{cor_typeIV}
	Assume the premises of Theorems \ref{theorem_principal_convergence} and \ref{bound_constant} are in force. Let $\mathcal{\hat{M}} \in \mathbb{C}(\mathcal{H})$ be a random model learned by $\mathbb{M}_{\mathbb{C}(\mathcal{H})}$. Then, for any $\epsilon > 0$,
	\begin{align*}
		&\textbf{(IV)} \ \mathbb{P}\left(L(\hat{h}_{\mathcal{\hat{M}}}^{\tilde{\mathcal{D}}_{M}}) - L(h^{\star}) > \epsilon\right)\\
		& \leq \mathbb{E}\Big[B^{II}_{M,\epsilon/2}(d_{VC}(\mathcal{\hat{M}}))\Big] \\
		&+ m \ \sum_{\mathcal{M} \in \text{Max } \mathbb{C}(\mathcal{H})} \left[B_{N_{t},(\epsilon/2 \vee \epsilon^{\star})/8}(d_{VC}(\mathcal{M})) + \hat{B}_{N_{v},(\epsilon/2 \vee \epsilon^{\star})/4}(d_{VC}(\mathcal{M}))\right]\\
		&\leq  B^{II}_{M,\epsilon/2}(d_{VC}(\mathbb{C}(\mathcal{H}))) \\
		&+ m \ \mathfrak{m}(\mathbb{C}(\mathcal{H})) \left[B_{N_{t},(\epsilon/2 \vee \epsilon^{\star})/8}(d_{VC}(\mathbb{C}(\mathcal{H}))) + \hat{B}_{N_{v},(\epsilon/2 \vee \epsilon^{\star})/4}(d_{VC}(\mathbb{C}(\mathcal{H})))\right].
	\end{align*}
	In particular,
	\begin{align*}
		\lim_{\substack{N \rightarrow \infty \\ M \rightarrow \infty}} \mathbb{P}\left(L(\hat{h}_{\mathcal{\hat{M}}}^{\tilde{\mathcal{D}}_{M}}) - L(h^{\star}) > \epsilon\right) = 0,
	\end{align*}
	for any $\epsilon > 0$.
\end{corollary}

Comparing the bounds of Corollary \ref{cor_typeIV} with that of type II estimation error of learning via ERM in $\mathcal{H}$ for a sample size of $N + M$ (cf. Proposition \ref{propVC}), we see that the former can be tighter. This is the case when the VC dimension of the maximal models is smaller than $d_{VC}(\mathcal{H})$, $\epsilon^{\star}$ is great or $d_{VC}(\mathcal{M}^{\star})$ is small and $d_{VC}(\hat{\mathcal{M}}) \approx d_{VC}(\mathcal{M}^{\star})$ with high probability. This fact evidences that by properly modeling the set of candidate models, it may be possible to increase generalization without increasing the sample size. This fact will be further explored in Section \ref{SecEnhanceGen}. This relation between the bounds also holds for unbounded functions (cf. Corollary \ref{cor_typeIV2}) and when learning by reusing (cf. Section \ref{SecReuse}).

\section{Learning via model selection with unbounded loss functions}
\label{SecUnbounded}

When the loss function is unbounded, we need to consider relative estimation errors and make assumptions about the tail weight of $P$. Heavy tail distributions are classically defined as those with a tail heavier than that of exponential distributions \cite{foss2011}. Nevertheless, in the context of learning, the tail weight of $P$ should take into account the loss function $\ell$. Hence, for $1 < p < \infty$ and a fixed hypotheses space $\mathcal{H}$, we measure the weight of the tails of distribution $P$ by
\begin{equation*}
	\tau_{p} \coloneqq \sup\limits_{h \in \mathcal{H}} \frac{\left(\int_{\mathcal{Z}} \ell^{p}(z,h) \ dP(z)\right)^{\frac{1}{p}}}{\int_{\mathcal{Z}} \ell(z,h) \ dP(z)} = \sup\limits_{h \in \mathcal{H}} \frac{L^{p}(h)}{L(h)},
\end{equation*}
in which $L^{p}(h) \coloneqq \left(\int_{\mathcal{Z}} \ell^{p}(z,h) \ dP(z)\right)^{\frac{1}{p}}$. We omit the dependence of $\tau_{p}$ on $\ell$, $P$ and $\mathcal{H}$ to simplify notation, since they will be clear from context. The weight of the tails of distribution $P$ may be defined based on $\tau_{p}$, as follows. Our presentation is analogous to \cite[Section~5.7]{vapnik1998} and is within the framework of \cite{cortes2019}. In this section, we again focus on learning with an independent sample and in Section \ref{SecReuse} we present analogous results for learning by reusing.

\begin{definition}
	\label{def_tails1}
	We say that distribution $P$ on $\mathcal{H}$ under $\ell$ has:
	\begin{itemize}
		\item Light tails, if there exists a $p > 2$ such that $\tau_{p} < \infty$;
		\item Heavy tails, if there exists a $1 < p \leq 2$ such that $\tau_{p} < \infty$, but $\tau_{p} = \infty$ for all $p > 2$;
		\item Very heavy tails, if $\tau = \infty$ for all $p > 1$.
	\end{itemize}
\end{definition}

We assume that $P$ has at most heavy tails, which means there exists a $p > 1$, that can be lesser than 2, with
\begin{align}
	\label{tauStar}
	\tau_{p} < \tau^{\star} < \infty,
\end{align}
that is, $P$ is in a class of distributions for which bound \eqref{tauStar} holds. From now on, fix a $p > 1$ and a $\tau^{\star}$ such that \eqref{tauStar} holds.

Besides the constraint \eqref{tauStar}, we also assume that the loss function is greater or equal to one: $\ell(z,h) \geq 1$ for all $z \in \mathcal{Z}, h \in \mathcal{H}$. This is done to ease the presentation, and without loss of generality, since it is enough to sum 1 to any unbounded loss function to have this property and, in doing so, not only the minimizers of $L_{\mathcal{D}_{N}}$ and $L$ in each model in $\mathbb{C}(\mathcal{H})$ remain the same, but also $\epsilon^{\star}$ does not change. Hence, by summing one to the loss, the estimated model $\hat{\mathcal{M}}$ and learned hypotheses from it do not change, and the result of the model selection framework is the same. We refer to Remark \ref{remark_geq1} for the technical reason we choose to consider loss functions greater than one.

Finally, we assume that $\ell$ has a finite moment of order $p$, under $P$ and under the empirical measure, for all $h \in \mathcal{H}$. That is, defining\footnote{We elevate \eqref{LNp} to the $1/p$ power to be consistent with the theory presented in Appendix \ref{apVCtheory}.}
\begin{equation}
	\label{LNp}
	L_{\mathcal{D}_{N}}^{p}(h) \coloneqq \left(\frac{1}{N} \sum_{i=1}^{N} \ell^{p}(Z_{i},h)\right)^ {\frac{1}{p}},
\end{equation}
we assume that
\begin{align}
	\label{finite_moments_text}
	\sup\limits_{h \in \mathcal{H}} L_{\mathcal{D}_{N}}^{p}(h)  < \infty & & \text{ and } & & \sup\limits_{h \in \mathcal{H}} L^{p}(h) < \infty,
\end{align}
in which the first inequality should hold with probability one, for all possible samples $\mathcal{D}_{N}$. Since the moments $L^{p}(h)$ are increasing in $p$, \eqref{finite_moments_text} actually implies \eqref{tauStar}, so \eqref{finite_moments_text} is the non-trivial constraint in distribution $P$.

Although this is a deviation from the distribution-free framework, it is a mild constraint in distribution $P$. On the one hand, the condition on $L^{p}$ is usually satisfied for distributions observed in real data (see \cite[Section~5.7]{vapnik1998} for examples with Normal, Uniform, and Laplacian distributions under the quadratic loss function). On the other hand, the condition on $L_{\mathcal{D}_{N}}^{p}$ is more a feature of the loss function, than of the distribution $P$, and can be guaranteed if one excludes from $\mathcal{H}$ some hypotheses with arbitrarily large loss in a way that $h^{\star}$ and $d_{VC}(\mathcal{H})$ remain the same (see Lemma \ref{lemma_norm} and Remark \ref{remark_finite_moments} for more details).

When the loss function is unbounded, besides the constraints in the moments of $\ell$, under $P$ and the empirical measure, we also have to consider variants of the estimation errors. Instead of the estimation errors, we consider the relative estimation errors:
\begin{align*}
	&\textbf{(I)} \sup\limits_{h \in \hat{\mathcal{M}}} \ \frac{\text{\textbar} L(h) - L_{\tilde{\mathcal{D}}_{M}}(h) \text{\textbar}}{L(h)} & & \textbf{(II)} \ \frac{L(\hat{h}_{\hat{\mathcal{M}}}^{\mathbb{A}}) - L(h^{\star}_{\hat{\mathcal{M}}})}{L(\hat{h}_{\hat{\mathcal{M}}}^{\mathbb{A}})}\\
	&\textbf{(III)} \ \frac{L(h^{\star}_{\hat{\mathcal{M}}}) - L(h^{\star})}{L(h^{\star}_{\hat{\mathcal{M}}})} & & \textbf{(IV)} \ \frac{L(\hat{h}_{\hat{\mathcal{M}}}^{\mathbb{A}}) - L(h^{\star})}{L(\hat{h}_{\hat{\mathcal{M}}}^{\mathbb{A}})}
\end{align*}
where algorithm $\mathbb{A}$ is dependent on the estimation technique once $\hat{\mathcal{M}}$ is selected.

In this section, we prove analogues of Theorems \ref{theorem_principal_convergence}, \ref{bound_constant} and \ref{theorem_tipeIII}. Before starting the study of the convergence of $\hat{\mathcal{M}}$ to $\mathcal{M}^{\star}$, we state a result analogous to Proposition \ref{propVC} about deviation bounds of relative type I and II estimation errors on $\mathcal{H}$, which are a consequence of Corollaries \ref{convergence_relativeTI} and \ref{cor2TypeII}. These are novel results of this paper which extend that of \cite{cortes2019}. 

\begin{proposition}
	\label{propVC2}
	Assume the loss function is unbounded and $P$ is such that \eqref{finite_moments_text} holds. Fixed a hypotheses space $\mathcal{H}$ with $d_{VC}(\mathcal{H}) < \infty$, there exist sequences $\{B^{I}_{N,\epsilon}: N \geq 1\}$ and $\{B^{II}_{N,\epsilon}: N \geq 1\}$ of positive real-valued increasing functions with domain $\mathbb{Z}_{+}$ satisfying
	\begin{equation*}
		\lim\limits_{N \to \infty} B^{I}_{N,\epsilon}(k) = \lim\limits_{N \to \infty} B^{II}_{N,\epsilon}(k) = 0,
	\end{equation*}
	for all $\epsilon > 0$ and $k \in \mathbb{Z}_{+}$ fixed, such that
	\begin{align*}
		\mathbb{P}&\left(\sup\limits_{h \in \mathcal{H}} \frac{\text{\textbar}L_{\mathcal{D}_{N}}(h) - L(h)\text{\textbar}}{L(h)} > \epsilon\right) \leq B^{I}_{N,\epsilon}(d_{VC}(\mathcal{H})) \text{ and }\\
		\mathbb{P}&\left(\frac{L(\hat{h}^{\mathcal{D}_{N}}) - L(h^{\star})}{L(\hat{h}^{\mathcal{D}_{N}})} > \epsilon\right) \leq B^{II}_{N,\epsilon}(d_{VC}(\mathcal{H})).
	\end{align*}
	Furthermore, the following holds:
	\begin{align*}
		\sup\limits_{h \in \mathcal{H}} \frac{\text{\textbar}L_{\mathcal{D}_{N}}(h) - L(h)\text{\textbar}}{L(h)} \xrightarrow[N \to \infty]{a.s.} 0 & & \text{ and } & & \frac{L(\hat{h}^{\mathcal{D}_{N}}) - L(h^{\star})}{L(\hat{h}^{\mathcal{D}_{N}})} \xrightarrow[N \to \infty]{a.s.} 0.
	\end{align*}
\end{proposition}

The results of this section seek to obtain insights about the asymptotic behavior of learning via model selection in the case of unbounded loss functions, rather than obtain the tightest possible bounds. Hence, in some results, the simplicity of the bounds is preferred over its tightness, and tighter bounds may be readily obtained from the proofs.

\subsection{Convergence to the target model}

We start by showing a result similar to Theorem \ref{theorem_principal_convergence}.

\begin{theorem}
	\label{theorem_principal_convergence_unbounded}
	Assume the loss function is unbounded and $P$ is such that \eqref{finite_moments_text} holds. For each $\epsilon > 0$, let $\{B_{N,\epsilon}: N \geq 1\}$ and $\{\hat{B}_{N,\epsilon}: N \geq 1\}$ be sequences of positive real-valued increasing functions with domain $\mathbb{Z}_{+}$ satisfying
	\begin{equation*}
		\lim\limits_{N \to \infty} B_{N,\epsilon}(k) = \lim\limits_{N \to \infty} \hat{B}_{N,\epsilon}(k) = 0,
	\end{equation*}
	for all $\epsilon > 0$ and $k \in \mathbb{Z}_{+}$ fixed, and such that
	\begin{align*}
		\max_{j} \mathbb{P}\left(\sup\limits_{h \in \mathcal{M}} \frac{\text{\textbar}L(h) - L_{\mathcal{D}_{N}^{(j)}}(h)\text{\textbar}}{L(h)} > \epsilon \right) \leq B_{N_{t},\epsilon}(d_{VC}(\mathcal{M})) & \text{ and } &\\ \nonumber \\ \nonumber
		\max_{j} \mathbb{P}\left(\sup\limits_{h \in \mathcal{M}} \frac{\text{\textbar}L(h) - \hat{L}^{(j)}(h)\text{\textbar}}{L(h)} > \epsilon \right) \leq \hat{B}_{N_{v},\epsilon}(d_{VC}(\mathcal{M})), & & 
	\end{align*}
	for all $\mathcal{M} \in \mathbb{C}(\mathcal{M})$, recalling that $L_{\mathcal{D}_{N}^{(j)}}$ and $\hat{L}^{(j)}$ represent the empirical risk under the $j$-th training and validation samples, respectively. Let $\mathcal{\hat{M}} \in \mathbb{C}(\mathcal{H})$ be a random model learned by $\mathbb{M}_{\mathbb{C}(\mathcal{H})}$. Then,
	\begin{align*}
		%\label{bound_pM2}
		\mathbb{P}\left(L(\hat{\mathcal{M}}) \neq L(\mathcal{M}^{\star})\right) & \leq 2m \sum_{\mathcal{M} \in \text{Max } \mathbb{C}(\mathcal{H})} \left[\hat{B}_{N_{v},\frac{\delta(1-\delta)}{2}}(d_{VC}(\mathcal{M})) + B_{N_{t},\frac{\delta(1-\delta)}{4}}(d_{VC}(d_{VC}(\mathcal{M})))\right]\\
		&\leq 2 m \ \mathfrak{m}(\mathbb{C}(\mathcal{H})) \left[\hat{B}_{N_{v},\frac{\delta(1-\delta)}{2}}(d_{VC}(\mathbb{C}(\mathcal{H}))) + B_{N_{t},\frac{\delta(1-\delta)}{4}}(d_{VC}(\mathbb{C}(\mathcal{H})))\right],
	\end{align*}
	in which $m$ is the number of pairs considered to calculate \eqref{form_Lhat} and 
	\begin{equation*}
		\delta \coloneqq \frac{\epsilon^{\star}}{2 \max\limits_{i \in \mathcal{J}} L(\mathcal{M}_{i})}.
	\end{equation*}
	Furthermore, if
	\begin{align}
		\label{as_conv2}
		\max_{\mathcal{M} \in \mathbb{C}(\mathcal{H})} \max_{j} \sup\limits_{h \in \mathcal{M}} \frac{\text{\textbar}L(h) - L_{\mathcal{D}_{N}^{(j)}}(h)\text{\textbar}}{L(h)} \xrightarrow[N \to \infty]{\text{a.s.}} 0 & \text{ and } &\\ \nonumber \\ \nonumber
		\max_{\mathcal{M} \in \mathbb{C}(\mathcal{H})} \max_{j} \sup\limits_{h \in \mathcal{M}} \frac{\text{\textbar}L(h) - \hat{L}^{(j)}(h)\text{\textbar}}{L(h)} \xrightarrow[N \to \infty]{\text{a.s.}} 0, & & 
	\end{align}
	then
	\begin{equation*}
		\lim_{N \rightarrow \infty} \mathbb{P}\left(\hat{\mathcal{M}} = \mathcal{M}^{\star}\right) = 1.
	\end{equation*}
\end{theorem}

In this instance, a bound for $\mathbb{P}(L(\hat{\mathcal{M}}) \neq L(\mathcal{M}^{\star}))$, and the almost sure convergence of $\hat{\mathcal{M}}$ to $\mathcal{M}^{\star}$, in the case of k-fold cross validation and independent validation sample, follow from Proposition \ref{propVC2} in a manner analogous to Theorem \ref{CVModelconvergence}. We state the almost sure convergence in Theorem \ref{CVModelconvergence2}, whose proof is analogous to that of Theorem \ref{CVModelconvergence}, and follows from Corollary \ref{convergence_relativeTI}. 

\begin{theorem}
	\label{CVModelconvergence2}
	Assume the loss function is unbounded and $P$ is such that \eqref{finite_moments_text} holds. If $\hat{L}$ is given by k-fold cross-validation or by an independent validation sample, then $\hat{\mathcal{M}}$ converges with probability one to $\mathcal{M}^{\star}$.
\end{theorem}

\subsection{Convergence of estimation errors on $\hat{\mathcal{M}}$}

The results stated here are rather similar to the case of bounded loss functions, with some minor modifications. Hence, we state the analogous results and present a proof only when it is different from the respective result in Section \ref{SecConvOnMhat}.

Bounds for relative types I and II estimation errors, when learning on a random model with a sample independent of the one employed to compute such random model, may be obtained as in Theorem \ref{bound_constant}. In fact, the proof of the following bounds are the same as in that theorem, with the respective changes from estimation errors to relative estimation errors. Hence, we state the results without a proof.

\begin{theorem}
	\label{bound_constant2}	
	Fix an unbounded loss function and assume $P$ is such that \eqref{finite_moments_text} holds. Assume we are learning with an independent sample $\tilde{\mathcal{D}}_{M}$, and that for each $\epsilon > 0$ there exist sequences $\{B^{I}_{M,\epsilon}: M \geq 1\}$ and $\{B^{II}_{M,\epsilon}: M \geq 1\}$ of positive real-valued increasing functions with domain $\mathbb{Z}_{+}$ satisfying
	\begin{equation*}
		\lim\limits_{M \to \infty} B^{I}_{M,\epsilon}(k) = \lim\limits_{M \to \infty} B^{II}_{M,\epsilon}(k) = 0,
	\end{equation*}
	for all $\epsilon > 0$ and $k \in \mathbb{Z}_{+}$ fixed, such that
	\begin{align*}
		\mathbb{P}\left(\sup\limits_{h \in \mathcal{M}} \frac{\text{\textbar}L_{\tilde{\mathcal{D}}_{M}}(h) - L(h)\text{\textbar}}{L(h)}  > \epsilon \right) \leq B^{I}_{M,\epsilon}(d_{VC}(\mathcal{M})) & \text{ and } &\\
		\mathbb{P}\left(\frac{L(\hat{h}_{\mathcal{M}}^{\tilde{\mathcal{D}}_{M}}) - L(h^{\star}_{\mathcal{M}})}{L(\hat{h}_{\mathcal{M}}^{\tilde{\mathcal{D}}_{M}})} > \epsilon \right) \leq B^{II}_{M,\epsilon}(d_{VC}(\mathcal{M})), & & 
	\end{align*}
	for all $\mathcal{M} \in \mathbb{C}(\mathcal{H})$. Let $\mathcal{\hat{M}} \in \mathbb{C}(\mathcal{H})$ be a random model learned by $\mathbb{M}_{\mathbb{C}(\mathcal{H})}$. Then, for any $\epsilon > 0$,
	\begin{align*}
		\textbf{(I)} \ \mathbb{P}&\left(\sup\limits_{h \in \mathcal{\hat{M}}} \frac{\text{\textbar}L_{\tilde{\mathcal{D}}_{M}}(h) - L(h)\text{\textbar}}{L(h)}  > \epsilon \right) \leq \mathbb{E}\Big[B^{I}_{M,\epsilon}(d_{VC}(\mathcal{\hat{M}}))\Big] \leq B^{I}_{M,\epsilon}\left(d_{VC}(\mathbb{C}(\mathcal{H}))\right)
	\end{align*}
	and
	\begin{align*}
		\textbf{(II)} \ \mathbb{P}\left(\frac{L(\hat{h}_{\mathcal{\hat{M}}}^{\tilde{\mathcal{D}}_{M}}) - L(h^{\star}_{\mathcal{\hat{M}}})}{L(\hat{h}_{\mathcal{\hat{M}}}^{\tilde{\mathcal{D}}_{M}})} > \epsilon \right) \leq \mathbb{E}\Big[B^{II}_{M,\epsilon}(d_{VC}(\mathcal{\hat{M}}))\Big] \leq B^{II}_{M,\epsilon}\left(d_{VC}(\mathbb{C}(\mathcal{H}))\right),
	\end{align*}
	in which the expectations are over all samples $\mathcal{D}_{N}$, from which $\hat{\mathcal{M}}$ is calculated. Since $d_{VC}(\mathbb{C}(\mathcal{H})) < \infty$, both probabilities above converge to zero when $M \to \infty$.
\end{theorem}

The convergence to zero of relative type III estimation error may be obtained, as in Theorem \ref{theorem_tipeIII}, by the methods used to prove Theorem \ref{theorem_principal_convergence_unbounded}. We state and prove this result, since its proof is slightly different from that of Theorem \ref{theorem_tipeIII}.

\begin{theorem}
	\label{theorem_tipeIII2}
	Assume the premises of Theorem \ref{theorem_principal_convergence_unbounded} are in force. Let $\mathcal{\hat{M}} \in \mathbb{C}(\mathcal{H})$ be a random model learned by $\mathbb{M}_{\mathbb{C}(\mathcal{H})}$. Then, for any $\epsilon > 0$,
	\begin{align*}
		\textbf{(III)} &\ \mathbb{P}\left(\frac{L(h_{\hat{\mathcal{M}}}^{\star}) - L(h^{\star})}{L(h_{\hat{\mathcal{M}}}^{\star})} > \frac{\epsilon}{L(\mathcal{M}^{\star})}\right) \\
		& \leq 2m \sum_{\mathcal{M} \in \text{Max } \mathbb{C}(\mathcal{H})} \left[\hat{B}_{N_{v},\frac{\delta^\prime(1-\delta^\prime)}{2}}(d_{VC}(\mathcal{M})) + B_{N_{t},\frac{\delta^\prime(1-\delta^\prime)}{4}}(d_{VC}(d_{VC}(\mathcal{M})))\right]\\
		& \leq 2m \ \mathfrak{m}(\mathbb{C}(\mathcal{H})) \left[\hat{B}_{N_{v},\frac{\delta^\prime(1-\delta^\prime)}{2}}(d_{VC}(\mathbb{C}(\mathcal{H}))) + B_{N_{t},\frac{\delta^\prime(1-\delta^\prime)}{4}}(d_{VC}(\mathbb{C}(\mathcal{H})))\right]
	\end{align*}
	in which
	\begin{equation*}
		\delta^\prime \coloneqq \frac{\epsilon \vee \epsilon^{\star}}{2 \max\limits_{i \in \mathcal{J}} L(\mathcal{M}_{i})}.
	\end{equation*}
	In particular,
	\begin{align*}
		\lim_{N \rightarrow \infty} \mathbb{P}\left(\frac{L(h_{\hat{\mathcal{M}}}^{\star}) - L(h^{\star})}{L(h_{\hat{\mathcal{M}}}^{\star})} > \epsilon\right) = 0,
	\end{align*}
	for any $\epsilon > 0$.
\end{theorem}

Finally, a bound on the rate of convergence of type IV estimation error to zero is a direct consequence of Theorems \ref{bound_constant2} and \ref{theorem_tipeIII2}, and the following inequality
\begin{align*}
	&\textbf{(IV)} \ \mathbb{P}\left(\frac{L(\hat{h}_{\mathcal{\hat{M}}}^{\tilde{\mathcal{D}}_{M}}) - L(h^{\star})}{L(\hat{h}_{\mathcal{\hat{M}}}^{\tilde{\mathcal{D}}_{M}})} > \frac{\epsilon}{L(\mathcal{M}^{\star})}\right)\\
	&\leq \mathbb{P}\left(\frac{L(\hat{h}_{\mathcal{\hat{\mathcal{M}}}}^{\tilde{\mathcal{D}}_{M}}) - L(h^{\star}_{\hat{\mathcal{M}}})}{L(\hat{h}_{\mathcal{\hat{M}}}^{\tilde{\mathcal{D}}_{M}})} > \frac{\epsilon}{2L(\mathcal{M}^{\star})}\right) + \mathbb{P}\left(\frac{L(h^{\star}_{\hat{\mathcal{M}}}) - L(h^{\star})}{L(h^{\star}_{\mathcal{\hat{\mathcal{M}}}})} > \frac{\epsilon}{2L(\mathcal{M}^{\star})}\right),
\end{align*}
which is true since $L(\hat{h}_{\mathcal{\hat{M}}}^{\tilde{\mathcal{D}}_{M}}) \geq L(h^{\star}_{\mathcal{\hat{M}}})$.

\begin{corollary}
	\label{cor_typeIV2}
	Assume the premises of Theorems \ref{theorem_principal_convergence_unbounded} and \ref{bound_constant2} are in force. Let $\mathcal{\hat{M}} \in \mathbb{C}(\mathcal{H})$ be a random model learned by $\mathbb{M}_{\mathbb{C}(\mathcal{H})}$. Then, for any $\epsilon > 0$,
	\begin{align*}
		&\mathbb{P}\left(\frac{L(\hat{h}_{\mathcal{\hat{M}}}^{\tilde{\mathcal{D}}_{M}}) - L(h^{\star})}{L(\hat{h}_{\mathcal{\hat{M}}}^{\tilde{\mathcal{D}}_{M}})} > \frac{\epsilon}{L(\mathcal{M}^{\star})}\right)\\
		& \leq \mathbb{E}\Big[B^{II}_{M,\frac{\epsilon}{2L(\mathcal{M}^{\star})}}(d_{VC}(\mathcal{\hat{M}}))\Big] \\
		&+ 2m \ \sum_{\mathcal{M} \in \text{Max } \mathbb{C}(\mathcal{H})} \left[\hat{B}_{N_{v},\frac{\delta^\prime(1-\delta^\prime)}{2}}(d_{VC}(\mathcal{M})) + B_{N_{t},\frac{\delta^\prime(1-\delta^\prime)}{4}}(d_{VC}(\mathcal{M}))\right]\\
		&\leq  B^{II}_{M,\frac{\epsilon}{2L(\mathcal{M}^{\star})}}(d_{VC}(\mathbb{C}(\mathcal{H}))) \\
		&+ 2m \ \mathfrak{m}(\mathbb{C}(\mathcal{H})) \left[\hat{B}_{N_{v},\frac{\delta^\prime(1-\delta^\prime)}{2}}(d_{VC}(\mathbb{C}(\mathcal{H}))) + B_{N_{t},\frac{\delta^\prime(1-\delta^\prime)}{4}}(d_{VC}(\mathbb{C}(\mathcal{H})))\right]
	\end{align*}
	with
	\begin{equation*}
		\delta^\prime \coloneqq \frac{\epsilon/2 \vee \epsilon^{\star}}{2 \max\limits_{i \in \mathcal{J}} L(\mathcal{M}_{i})}.
	\end{equation*}
	In particular,
	\begin{align*}
		\lim_{\substack{N \rightarrow \infty \\ M \rightarrow \infty}} \mathbb{P}\left(\frac{L(\hat{h}_{\mathcal{\hat{M}}}^{\tilde{\mathcal{D}}_{M}}) - L(h^{\star})}{L(\hat{h}_{\mathcal{\hat{M}}}^{\tilde{\mathcal{D}}_{M}})}  > \epsilon\right) = 0,
	\end{align*}
	for any $\epsilon > 0$.
\end{corollary}

\section{Learning by reusing}
\label{SecReuse}

In this section, we present bounds for types I, II, and consequently IV, estimation errors for learning by reusing for bounded and unbounded loss functions. When learning by reusing, one is employing the same sample points to both estimate $\hat{\mathcal{M}}$ and learn a hypothesis $\hat{h}^{\mathcal{D}_{N}}_{\hat{\mathcal{M}}} \in \hat{\mathcal{M}}$ from it, so there is a dependence between types I and II estimation errors and the events $\{\hat{\mathcal{M}} = \mathcal{M}\}, \mathcal{M} \in \mathbb{C}(\mathcal{H})$. For instance, the bounds for types I and II estimation errors in Theorem \ref{bound_constant} depend on an equality (cf. \eqref{cond_independence} in the proof of Theorem \ref{bound_constant}) that does not hold when learning by reusing since
\begin{align*}
	%\label{cond_independence2}
	\mathbb{P}\left(\sup\limits_{h \in \hat{\mathcal{M}}} \big|L_{\mathcal{D}_{N}}(h) - L(h) \big| > \epsilon \Big|\mathcal{\hat{M}} = \mathcal{M}\right) \neq \mathbb{P}\left(\sup\limits_{h \in \mathcal{M}} \big|L_{\mathcal{D}_{N}}(h) - L(h) \big| > \epsilon\right).
\end{align*}
Conditioned on $\{\hat{\mathcal{M}} = \mathcal{M}\}$, not only the distribution of each sample point $(X_{i},Y_{i}), i = 1,\dots,N$, changes, but also these points are now dependent: they must be such that $\hat{\mathcal{M}} = \mathcal{M}$, hence, cannot be independent. Therefore, the argument of the proof of Theorem \ref{bound_constant} does not hold in this instance.

Nevertheless, since $\hat{\mathcal{M}}$ converges with probability one to $\mathcal{M}^{\star}$ by Theorem \ref{theorem_principal_convergence}, we may obtain a bound for types I and II estimation errors when learning by reusing which depends on such bounds in $\mathcal{M}^{\star}$, and on the rate of convergence of $\hat{\mathcal{M}}$ to $\mathcal{M}^{\star}$.

\begin{theorem}
	\label{bound_constant_reusing}	
	Fix a bounded loss function. Assume we are learning by reusing and that, for each $\epsilon > 0$, there exist sequences $\{B^{I}_{N,\epsilon}: N \geq 1\}$ and $\{B^{II}_{N,\epsilon}: N \geq 1\}$ of positive real-valued increasing functions with domain $\mathbb{Z}_{+}$ satisfying
	\begin{equation*}
		\lim\limits_{N \to \infty} B^{I}_{N,\epsilon}(k) = \lim\limits_{N \to \infty} B^{II}_{N,\epsilon}(k) = 0,
	\end{equation*}
	for all $\epsilon > 0$ and $k \in \mathbb{Z}_{+}$ fixed, such that
	\begin{align*}
		\label{bound_theoremBC2}
		\mathbb{P}\left(\sup\limits_{h \in \mathcal{M}} \big|L_{\mathcal{D}_{N}}(h) - L(h) \big| > \epsilon \right) \leq B^{I}_{N,\epsilon}(d_{VC}(\mathcal{M})) & \text{ and } &\\ \nonumber
		\mathbb{P}\left(L(\hat{h}_{\mathcal{M}}^{\mathcal{D}_{N}}) - L(h^{\star}_{\mathcal{M}}) > \epsilon \right) \leq B^{II}_{N,\epsilon}(d_{VC}(\mathcal{M})), & &
	\end{align*}
	for all $\mathcal{M} \in \mathbb{C}(\mathcal{H})$. Let $\mathcal{\hat{M}} \in \mathbb{C}(\mathcal{H})$ be a random model learned by $\mathbb{M}_{\mathbb{C}(\mathcal{H})}$. Then, for any $\epsilon > 0$,
	\begin{align*}
		\textbf{(I)} \ \mathbb{P}\left(\sup\limits_{h \in \mathcal{\hat{M}}} \big|L_{\mathcal{D}_{N}}(h) - L(h) \big| > \epsilon \right) \leq B^{I}_{N,\epsilon}(d_{VC}(\mathcal{M}^{\star})) + \mathbb{P}\left(\hat{\mathcal{M}} \neq \mathcal{M}^{\star}\right)
	\end{align*}
	and
	\begin{align*}
		\textbf{(II)} \ \mathbb{P}\left(L(\hat{h}_{\mathcal{\hat{M}}}^{\mathcal{D}_{N}}) - L(h^{\star}_{\mathcal{\hat{M}}}) > \epsilon \right) \leq B^{II}_{N,\epsilon}(d_{VC}(\mathcal{M}^{\star})) + \mathbb{P}\left(\hat{\mathcal{M}} \neq \mathcal{M}^{\star}\right).
	\end{align*}
	If conditions \eqref{as_conv} of Theorem \ref{theorem_principal_convergence} are satisfied, both probabilities above converge to zero when $N \to \infty$.
\end{theorem}

\begin{remark}
	Theorem \ref{bound_constant_reusing} also holds when learning with independent sample by exchanging $\mathcal{D}_{N}$ with $\tilde{\mathcal{D}}_{M}$.
\end{remark}

From inequality \eqref{triangle} and the bound for type III estimation error established in Theorem \ref{theorem_tipeIII}, that is also true when learning by reusing (cf. Remark \ref{remReuse}), follow that the tail probability of type IV estimation error converges to zero as $N$ tends to infinity, a result analogous to Corollary \ref{cor_typeIV}.

For unbounded loss functions, when learning by reusing, a result analogous to Theorem \ref{bound_constant_reusing}, together with Theorem \ref{theorem_principal_convergence_unbounded} and a result analogous to Corollary \ref{cor_typeIV2}, will imply the convergence of the estimation errors to zero. We state this result without proof.

\begin{theorem}
	\label{bound_constant_reusing2}	
	Fix an unbounded loss function and assume $P$ is such that \eqref{finite_moments_text} holds. Assume we are learning by reusing and that, for each $\epsilon > 0$, there exist sequences $\{B^{I}_{N,\epsilon}: N \geq 1\}$ and $\{B^{II}_{N,\epsilon}: N \geq 1\}$ of positive real-valued increasing functions with domain $\mathbb{Z}_{+}$ satisfying
	\begin{equation*}
		\lim\limits_{N \to \infty} B^{I}_{N,\epsilon}(k) = \lim\limits_{N \to \infty} B^{II}_{N,\epsilon}(k) = 0,
	\end{equation*}
	for all $\epsilon > 0$ and $k \in \mathbb{Z}_{+}$ fixed, such that
	\begin{align*}
		\mathbb{P}\left(\sup\limits_{h \in \mathcal{M}} \left|\frac{L_{\mathcal{D}_{N}}(h) - L(h)}{L(h)} \right| > \epsilon \right) \leq B^{I}_{N,\epsilon}(d_{VC}(\mathcal{M})) & \text{ and } &\\ \nonumber
		\mathbb{P}\left(\frac{L(\hat{h}_{\mathcal{M}}^{\mathcal{D}_{N}}) - L(h^{\star}_{\mathcal{M}})}{L(\hat{h}_{\mathcal{M}}^{\mathcal{D}_{N}})} > \epsilon \right) \leq B^{II}_{N,\epsilon}(d_{VC}(\mathcal{M})), & & 
	\end{align*}
	for all $\mathcal{M} \in \mathbb{C}(\mathcal{H})$. Let $\mathcal{\hat{M}} \in \mathbb{C}(\mathcal{H})$ be a random model learned by $\mathbb{M}_{\mathbb{C}(\mathcal{H})}$. Then, for any $\epsilon > 0$,
	\begin{align*}
		\textbf{(I)} \ \mathbb{P}\left(\sup\limits_{h \in \mathcal{\hat{M}}} \left|\frac{L_{\mathcal{D}_{N}}(h) - L(h)}{L(h)} \right| > \epsilon \right) \leq B^{I}_{N,\epsilon}(d_{VC}(\mathcal{M}^{\star})) + \mathbb{P}\left(\hat{\mathcal{M}} \neq \mathcal{M}^{\star}\right)
	\end{align*}
	and
	\begin{align*}
		\textbf{(II)} \ \mathbb{P}\left(\frac{L(\hat{h}_{\mathcal{\hat{M}}}^{\mathcal{D}_{N}}) - L(h^{\star}_{\mathcal{\hat{M}}})}{L(\hat{h}_{\mathcal{\hat{M}}}^{\mathcal{D}_{N}})} > \epsilon \right) \leq B^{II}_{N,\epsilon}(d_{VC}(\mathcal{M}^{\star})) + \mathbb{P}\left(\hat{\mathcal{M}} \neq \mathcal{M}^{\star}\right).
	\end{align*}
	If conditions \eqref{as_conv2} of Theorem \ref{theorem_principal_convergence_unbounded} are satisfied, both probabilities above converge to zero when $N \to \infty$.
\end{theorem}

\section{Increasing generalization by learning via model selection}
\label{SecEnhanceGen}

In this section, we analyze concrete examples to better understand how generalization may be increased by learning via model selection. In particular, we illustrate the role of $\epsilon^{\star}$, $d_{VC}(\mathcal{M}^{\star})$ and $d_{VC}(\mathbb{C}(\mathcal{H}))$ on the generalization and how it can be increased by inserting prior information about $h^{\star}$ into the family of candidate models.

In Sections \ref{SecLS} and \ref{SecExamplesLS}, we present a class of candidate models in which the partial order by inclusion reflects the models complexity. We call this class Learning Spaces and focus our analyses on a subclass that has a lattice structure. In Section \ref{SecPLLS}, we analyze a worst-case scenario of learning a classifier with finite domain, focusing on the effect of $\epsilon^{\star}$ and $d_{VC}(\mathbb{C}(\mathcal{H}))$ on the generalization. In Section \ref{SecRegression}, we analyze a typical case in linear regression, focusing on the effect of $d_{VC}(\mathcal{M}^{\star})$ and prior information about $h^{\star}$ on generalization. In Section \ref{SecPIEnhance}, we further discuss how generalization may be increased by inserting domain knowledge into the family of candidate models and in Section \ref{SecComp} we briefly discuss computational aspects of model selection.

\subsection{Learning Spaces}
\label{SecLS}

Let $\mathbb{L}(\mathcal{H}) \coloneqq \{\mathcal{M}_{i}: i \in \mathcal{J} \subset \mathbb{Z}_{+}\}$ be a finite subset of the power set of $\mathcal{H}$, i.e., $\mathbb{L}(\mathcal{H}) \subset \mathcal{P}(\mathcal{H})$ and $|\mathcal{J}| < \infty$. We say that the partially ordered set (poset) $(\mathbb{L}(\mathcal{H}),\subset)$ is a Learning Space if
\begin{itemize}
	\item[] (i) $\bigcup\limits_{i \in \mathcal{J}} \mathcal{M}_{i} = \mathcal{H}$
	\item[] (ii) $\mathcal{M}_{1}, \mathcal{M}_{2} \in \mathbb{L}(\mathcal{H})$ and $\mathcal{M}_{1} \subset \mathcal{M}_{2}$ implies $d_{VC}(\mathcal{M}_{1}) < d_{VC}(\mathcal{M}_{2})$.
\end{itemize}

Learning Spaces are families of candidate models that cover $\mathcal{H}$ and such that any element $\mathcal{M} \in \mathbb{L}(\mathcal{H})$ is complexity maximal in the sense that there does not exist $\mathcal{M}' \in \mathbb{L}(\mathcal{H}), \mathcal{M}' \neq \mathcal{M},$ such that $d_{VC}(\mathcal{M}') = d_{VC}(\mathcal{M})$ and $\mathcal{M}' \subset \mathcal{M}$. This condition guarantees that if $\mathcal{M}_{1} \subset \mathcal{M}_{2}$ then the complexity of $\mathcal{M}_{2}$ is greater than that of $\mathcal{M}_{1}$. This implies that the poset $(\mathbb{L}(\mathcal{H}),\subset)$ reflects the complexity of its models. We note that one could choose $\{\mathcal{M}_{1},\dots,\mathcal{M}_{n}\}$ without thinking of it as a decomposition of a hypotheses space $\mathcal{H}$. Nevertheless, if condition (ii) is satisfied, then it would be a Learning Space of $\mathcal{H} = \cup_{i} \mathcal{M}_{i}$, so taking $\mathcal{H}$ as this union, the only non-trivial condition is (ii).

Learning Spaces are not unique, i.e., there are multiple subsets of $\mathcal{P}(\mathcal{H})$ which are Learning Spaces, and the main class of Learning Spaces are the Lattice Learning Spaces, which have a complete lattice structure. A complete lattice $(\mathbb{L}(\mathcal{H}),\subset,\wedge,\vee,\mathcal{O},\mathcal{I})$ is the poset $(\mathbb{L}(\mathcal{H}),\subset)$ with binary relations meet ($\wedge$) and join ($\vee$) representing the greatest lower bound and least upper bound of two elements. A lattice is complete if every subset of $\mathbb{L}(\mathcal{H})$ has a meet and a join in $\mathbb{L}(\mathcal{H})$. In particular, a complete lattice has a least element ($\mathcal{O}$) and greatest element ($\mathcal{I}$) that are the meet and join of $\mathbb{L}(\mathcal{H})$, so $Max \ \mathbb{L}(\mathcal{H}) = \{\mathcal{I}\}$ and $\mathfrak{m}(\mathbb{L}(\mathcal{H})) = 1$.

\begin{remark}
	Although we consider the VC-dimension, other complexity measures of hypotheses spaces could be used to define the Learning Spaces, such as the fat-shattering dimension \cite{bartlett1994fat} and the Rademacher and Gaussian complexities \cite{bartlett2002rademacher}. We note that the value of the VC dimension is not of importance to the algebraic aspect of the Learning Space definition, but only the fact that it increases when we consider nested models. Hence, any other complexity measure such that this increase is also observed for the chosen poset of models would generate a valid Learning Space.
\end{remark}

The first step in building a Learning Space is fixing an algebraic parametric representation of the hypotheses in $\mathcal{H}$. The algebraic structure of $(\mathbb{L}(\mathcal{H}),\subset)$ may be defined from the learning model and algebraic parametric representation fixed, in the following manner. Let $(\mathcal{F},\leq)$ be a poset, in which $\mathcal{F}$ is an arbitrary set with finite cardinality. Moreover, let $\mathcal{R}: \mathcal{F} \mapsto Im(\mathcal{R}) \subset \mathcal{P}(\mathcal{H})$ be a lattice isomorphism from set $(\mathcal{F},\leq)$ to $(Im(\mathcal{R}),\subset)$, a subset of the power set of $\mathcal{H}$ partially ordered by inclusion. This means that $\mathcal{R}$ is bijective, and if $a,b \in \mathcal{F}, a \leq b$, then $\mathcal{R}(a) \subset \mathcal{R}(b)$, so $\mathcal{R}$ preserves the partial order $\leq$ on $\mathcal{F}$ as the partial order on $Im(\mathcal{R})$ given by inclusion. Then, if 
\begin{itemize}
	\item[] (i) $\bigcup\limits_{a \in \mathcal{F}} \mathcal{R}(a) = \mathcal{H}$ and
	\item[] (ii) $a,b \in \mathcal{F}, a \leq b,$ implies $d_{VC}(\mathcal{R}(a)) < d_{VC}(\mathcal{R}(b))$,
\end{itemize}
we may define $\mathbb{L}(\mathcal{H}) \coloneqq Im(\mathcal{R})$ as a Learning Space of $\mathcal{H}$. We call isomorphisms which satisfy these conditions Learning Space generators. Since the generator $\mathcal{R}$ is an isomorphism, it preserves properties of $(\mathcal{F},\leq)$, hence, for instance, by applying $\mathcal{R}$ to a complete lattice we obtain a Lattice Learning Space.

A Learning Space is completely defined by a triple $(\mathcal{F},\leq,\mathcal{R})$, in which the elements of $\mathcal{F}$ may be interpreted as sets of parameters which describe a subset of hypotheses, i.e., the hypotheses in $\mathcal{R}(a), a \in \mathcal{F},$ are represented by the parameters $a$, so that, in particular, $\mathcal{F}$ generates a parametric representation of the functions in $\mathcal{H}$. For this reason, we call $(\mathcal{F},\leq)$ a parametric poset of $\mathcal{H}$. Therefore, in general, to build a Learning Space of $\mathcal{H}$ we apply a generator to a parametric poset of its hypotheses. 

\subsection{Examples of Learning Spaces}
\label{SecExamplesLS}

We present some examples of Learning Spaces completely defined by a triple $(\mathcal{F},\leq,\mathcal{R})$.

\begin{example}[Variable Selection]
	\label{feature_lattice} \normalfont
	Assume that $\mathcal{H}$ is a space of functions with domain $\mathcal{X} \subset \mathbb{R}^{d}, d > 1$. Let $\mathcal{F} = \mathcal{P}(\{1,\dots,d\})$ be the power set of $\{1,\dots,d\}$ partially ordered by inclusion, so that $(\mathcal{F},\subset,\cap,\cup,\emptyset,\{1,\dots,d\})$ is a complete Boolean lattice. Consider the Learning Space generator $\mathcal{R}: \mathcal{F} \mapsto Im(\mathcal{R}) \subset \mathcal{P}(\mathcal{H})$ given by
	\begin{equation*}
		\mathcal{R}(a) = \Big\{h \in \mathcal{H}: h(x) = h(z), \text{ if } x \equiv_{a} z\Big\},
	\end{equation*} 
	in which $a = \{a_{1},\dots,a_{j}\} \in \mathcal{F}$ and $x = (x_{1},\dots,x_{d}) \equiv_{a} z = (z_{1},\dots,z_{d})$ if, and only if, $x_{a_{i}} = z_{a_{i}}$ for $i = 1,\dots,j$, so $\mathcal{R}(a)$ contains the hypotheses which depend solely on variables in $a$. The lattice isomorphism $\mathcal{R}$ satisfies the condition (i) and often satisfies (ii), as in many applications the VC dimension is an increasing function of the number of variables, so $Im(\mathcal{R})$ is often a Learning Space. This is the usual collection of candidate models for variable selection problems.
	
	\hfill$\square$
\end{example}

\begin{example}[Partition Lattice]
	\label{partition_lattice} \normalfont
	Let $\mathcal{X}$ be an arbitrary set with $\text{\textbar}\mathcal{X}\text{\textbar} < \infty$ and let $\mathcal{H} = \{h: \mathcal{X} \mapsto \{0,1\}\}$ be the set of all functions from $\mathcal{X}$ to $\{0,1\}$. Assuming $Z = (X,Y)$ and $\mathcal{Z} = \mathcal{X} \times \{0,1\}$, under the simple loss function $\ell((x,y),h) = \mathds{1}\{h(x) \neq y\}$, it follows that $L(h) = \mathbb{P}(h(X) \neq Y)$ is the classification error.
	
	Denote $\Pi = \{\pi: \pi \text{ is a partition of } \mathcal{X}\}$ as the set of all partitions of $\mathcal{X}$. A partition $\pi = \{p_{1},\dots,p_{k}\}$ is a collection of subsets of $\mathcal{X}$, called blocks or parts, such that
	\begin{align*}
		\bigcup_{i=1}^{k} p_{i} = \mathcal{X} & & p_{i} \cap p_{i^\prime} = \emptyset \text{ if } i \neq i^\prime.
	\end{align*}
	Each partition $\pi \in \Pi$ creates an equivalence class in $\mathcal{X}$ with equivalence between points in the same block of partition $\pi$:
	\begin{equation*}
		x \equiv_{\pi} y \iff \exists p \in \pi \text{ such that } \{x,y\} \subset p.
	\end{equation*}
	Consider in $\Pi$ the partial order $\leq$ defined as
	\begin{equation*}
		\pi_{1} \leq \pi_{2} \text{ if, and only if, } x \equiv_{\pi_{2}} z \text{ implies } x \equiv_{\pi_{1}} z
	\end{equation*}
	for $\pi_{1}, \pi_{2} \in \Pi$, which turns it into a complete lattice $(\Pi,\leq,\wedge,\vee,\{\mathcal{X}\},\mathcal{X})$. This partial order is equivalent to $\pi_{1} \leq \pi_{2}$ if, and only if, for every $p_{2} \in \pi_{2}$ there exists a $p_{1} \in \pi_{1}$ such that $p_{2} \subset p_{1}$. See Figure \ref{partitionL} for an example of a partition lattice. By applying the generator $\mathcal{R}: \mathcal{F} \mapsto Im(\mathcal{R}) \subset \mathcal{P}(\mathcal{H})$ given by
	\begin{equation*}
		\mathcal{R}(\pi) = \mathcal{M}_{\pi} \coloneqq \Big\{h \in \mathcal{H}: h(x) = h(z) \text{ if } x \equiv_{\pi} z \Big\},
	\end{equation*}
	for $\pi \in \Pi$, we obtain the partition lattice Learning Space $\mathbb{L}(\mathcal{H}) \coloneqq Im(\mathcal{R})$. The set $\mathcal{R}(\pi)$ is formed by all hypotheses which classify the points inside a block of $\pi$ in the same category; those are the hypotheses which respect $\pi$. This lattice is indeed a Leaning Space since $\mathcal{H} \in \mathbb{L}(\mathcal{H})$ and $d_{VC}(\mathcal{M}_{\pi}) = |\pi|$. In this case, the parameters of the functions $h \in \mathcal{H}$ are the elements in their domain $\mathcal{X}$, in contrast, for example, to the variables they depend on, as in Example \ref{feature_lattice}.
	
	For any distribution $P$, $d_{VC}(\mathcal{M}^{\star}) \leq 2$ since $h^{\star} \in \mathcal{M}_{\pi^{\star}}$ in which $\pi^{\star} = \{\{x: h^{\star}(x) = 0\},\{x: h^{\star}(x) = 1\}\}$ is the partition generated by $h^{\star}$.
	
	\hfill$\square$
	
	\begin{figure}[ht]
		\centering
		\includegraphics[width=\textwidth]{reticulado.pdf}
		\caption{The set $\Pi$ of all partitions of $\mathcal{X} = \{1,2,3,4\}$. The tables present the hypotheses in selected models $\mathcal{M}_{\pi_{1}}, \mathcal{M}_{\pi_{2}}$.}
		\label{partitionL}
	\end{figure}
	%\end{landscape}
\end{example}

\begin{example}[Linear Regression]
	\label{parametric_lattice} \normalfont
	Let $\mathcal{H}$ be given by the linear functions in $\mathbb{R}^{d}, d \geq 1$:
	\begin{equation*}
		\mathcal{H} = \Bigg\{h_{a}(x) = a_{0} + \sum_{i=1}^{d} a_{i}x_{i}: a_{i} \in \mathbb{R}\Bigg\},
	\end{equation*}
	in which $x = (x_{1},\dots,x_{d}) \in \mathbb{R}^{d}$ and $h_{a}$ is the function indexed by its parameters $a = (a_{0},\dots,a_{d}) \in \mathbb{R}^{d+1}$. Assume $Z = (X,Y)$ and $\mathcal{Z} = \mathbb{R}^{d} \times \mathbb{R}$, and consider the square loss function $\ell((x,y),h) = (h(x) - y)^{2}$ so $L(h) = \mathbb{E}[(h(X) - Y)^{2}]$ is the mean squared error.
	
	Denoting $\mathcal{A} = \{1,\dots,d\}$, we consider two distinct Learning Space generators: from the Boolean lattice $(\mathcal{P}(\mathcal{A}),\subset,\cap,\cup,\{\emptyset\},\mathcal{A})$ and from the partition lattice $(\Pi_{\mathcal{A}},\leq,\wedge,\vee,\{\mathcal{A}\},\mathcal{A})$ of $\mathcal{A}$, in which $\mathcal{P}(\mathcal{A})$ is the power set of $\mathcal{A}$ and $\Pi_{\mathcal{A}}$ is the set of all partitions of $\mathcal{A}$. The partition lattice is represented in Figure \ref{partitionL} for $d = 4$.
	
	Define $\mathcal{R}_{1}: \mathcal{P}(\mathcal{A}) \mapsto \mathcal{P}(\mathcal{H})$ as
	\begin{equation*}
		\mathcal{R}_{1}(A) = \Big\{h_{a} \in \mathcal{H}: a_{j} = 0  \text{ if } j \notin A \cup \{0\}\Big\},
	\end{equation*}
	for $A \in \mathcal{P}(\mathcal{A})$ as a variable selection generator, and define $\mathcal{R}_{2}: \Pi_{\mathcal{A}} \mapsto \mathcal{P}(\mathcal{H})$ as
	\begin{equation*}
		\mathcal{R}_{2}(\pi) = \Big\{h_{a} \in \mathcal{H}: a_{j} = a_{k} \text{ if } j \equiv_{\pi} k\},
	\end{equation*}
	for $\pi \in \Pi_{\mathcal{A}}$ as a generator that equals parameters, i.e., create equivalence classes in $\mathcal{A}$. Both $\mathcal{R}_{1}, \mathcal{R}_{2}$ clearly satisfy (i) and (ii). Therefore, these lattice isomorphisms generate two distinct Lattice Learning Spaces for the same hypotheses space $\mathcal{H}$. Isomorphic Learning Spaces also apply to the hypotheses space of linear classifiers.	
	\hfill$\square$
\end{example}

\subsection{Partition Lattice Learning Space}
\label{SecPLLS}

In this section, we compare learning via ERM on the whole hypotheses space with learning via model selection in the partition lattice Learning Space of Example \ref{partition_lattice}. We consider the \textit{worst-case} distribution for a fixed value of $\epsilon^{\star}$, that is, when the joint distribution of $(X,Y)$ has maximum conditional entropy. Clearly this happens when $\mathbb{P}(X = x) = 1/|\mathcal{X}|$ for all $x \in \mathcal{X}$ and
\begin{align}
	\label{cond_max_entropy}
	|\mathbb{P}(Y = 1|X = x) - \mathbb{P}(Y = 0|X = x)| = |\mathcal{X}|\epsilon^{\star}, \ \text{ for all } x \in \mathcal{X}.
\end{align}
There are $2^{|\mathcal{X}|}$ distributions that satisfy \eqref{cond_max_entropy} and we choose the one such that $\mathbb{P}(Y = 1|X = x) > \mathbb{P}(Y = 0|X = x)$ if $x$ is odd and $\mathbb{P}(Y = 1|X = x) < \mathbb{P}(Y = 0|X = x)$ if $x$ is even. For any value of $\epsilon^{\star}$, $h^{\star}(x)$ equals one for $x$ odd and zero for $x$ even, and $\mathcal{M}^{\star} = \mathcal{M}_{\pi}$ in which $\pi$ is the partition of even and odd numbers with $d_{VC}(\mathcal{M}^{\star}) = 2$.

For each $\epsilon^{\star} \in \{0.0025,0.005,0.0125,0.025,0.05,0.075,0.1\}$ we simulate $1,000$ samples of size $n \in \{16,32,64,128,256\}$ considering $\mathcal{X} = \{1,\dots,8\}$ and the worst-case distribution described above. The distributions considered are in Table \ref{jointDist} in Appendix \ref{AppResults}. For each sample, we learn by ERM with the whole sample, learn via model selection with independent sample by dividing the sample into 50\% - 50\% for training and independent sample, and learn via model selection by reusing. Model risks are estimated by k-fold cross validation with $k = 4$ and when the solution is not unique, i.e., $\hat{\mathcal{M}}$ is an equivalence class with more than one model, learning is performed by minimizing the respective empirical risk in the union of the models.

Figure \ref{fig_PLLS1} presents for each scenario the number of simulated samples in which the risk of the estimated hypothesis via model selection was greater, lesser or equal to the risk of the ERM hypothesis. We see that for all $\epsilon^{\star}$, the number of samples in which learning via model selection is better decreases with $n$, illustrating that learning via model selection is more beneficial for small sample sizes. For all sample sizes, the number of samples in which learning via model selection is better decreases with $\epsilon^{\star}$, and for great values of $\epsilon^{\star}$ ($>= 0.05$) learning via model selection is as good as via ERM in the majority of cases for mild to great sample sizes. Finally, learning via model selection by reusing is, in general, slightly better than with independent sample relative to learning via ERM.

The results in Figure \ref{fig_PLLS1} illustrate the effect of $\epsilon^{\star}$ in the generalization of learning via model selection. On the one hand, if $\epsilon^{\star}$ is too great, then the learning problem is actually quite \textit{easy} and model selection is not necessary. For example, the conditional distribution for $\epsilon^{\star} = 0.1$ is such that $\mathbb{P}(Y = 1|X = 1) = 0.9$ (cf. Table \ref{jointDist}) so we do not need many samples with $X = 1$ to figure out that $h^{\star}(1) = 1$. Therefore, with or without model selection, we can properly learn $h^{\star}$.

On the other hand, when $\epsilon^{\star}$ is small, the learning problem becomes \textit{harder}. For example, the conditional distribution for $\epsilon^{\star} = 0.0025$ is such that $\mathbb{P}(Y = 1|X = 1) = 0.51$ so more samples with $X = 1$ are needed to establish that $h^{\star}(1) = 1$. However, if we could somehow figure out that, say, $h^{\star}(1) = h^{\star}(3)$ then we would pool the samples with $X = 1$ and $X = 3$ together to figure the value of $h^{\star}$ at these points. Therefore, by knowing a partition of the domain with points in the same block having the same value of $h^{\star}$, we need fewer points in the independent sample to properly learn. In the limit case, when the partition selected is that of the even and odd numbers, learning with independent sample will be performed in a (\textit{good}) hypotheses space with VC dimension 2 and few samples will suffice.

\begin{figure}[ht]
	\centering
	\includegraphics[width=\linewidth]{res_partition_simulation.pdf}
	\caption{Number of simulations in which learning via model selection in the whole partition lattice Learning Space is better, worse and as good as learning via ERM with respect to the risk of the estimated hypothesis.} \label{fig_PLLS1}
\end{figure}

\begin{figure}[ht]
	\centering
	\includegraphics[width=\linewidth]{mean_partition_simulation.pdf}
	\caption{The average with its 95\% confidence interval of the ERM hypothesis bias and types II, III and IV estimation errors over the 1,000 simulated samples for each case when learning via model selection in the whole partition lattice Learning Space. These results are also presented in Table \ref{res_PLLS} in Appendix \ref{AppResults}.} \label{fig_PLLS2}
\end{figure}

Figure \ref{fig_PLLS2} presents the average bias of the ERM hypothesis and types II, III and IV estimation errors over the 1,000 simulated samples for each case. For $\epsilon^{\star} \geq 0.05$, the average estimation errors decrease with the sample size, while the average bias of the ERM hypothesis decreases with $n$ for all $\epsilon^{\star} \geq 0.005$. For $\epsilon^{\star} \leq 0.025$, the average of the estimation errors first increase with $n$, then remains stable, until attaining a sample size in which it rapidly decreases, specially for learning with independent sample. In particular, the average of the bias of the ERM hypothesis is, in general, greater than the average of type IV estimation error for small sample sizes.

The increase in the estimation errors from sample size $16$ to $32$ can be explained by the fact that with $16$ samples the model of the constant functions, related to the partition $\pi = \{\{\mathcal{X}\}\}$, which has VC dimension 1, is often selected: between 22-33\% for small values of $\epsilon^{\star}$ (see Figure \ref{fig_PLLS4} in Appendix \ref{AppResults}). When this model is selected, then the learned hypothesis has a risk equal to $L(h^{\star}) + 4\epsilon^{\star}$, that is the risk of the constant hypotheses. When more sample are available, models with VC dimension 2 are selected more often and the risk can be as great as $L(h^{\star}) + 8\epsilon^{\star}$, so it is reasonable that the average of estimation errors slightly increase.

The rapid decrease, specially for type II estimation error, is due to a phase transition which is a feature of, for example, the bounds in Theorems \ref{theorem_tipeIII} and \ref{theorem_tipeIII2} and Corollaries \ref{cor_typeIV} and \ref{cor_typeIV2}, for example (due to the term $\epsilon \vee \epsilon^{\star}$). When there are not enough samples to properly estimate the risks with a precision of $\epsilon^{\star}/2$ with high probability, it is unlikely that a model \textit{close}, in the type III estimation error sense, to $\mathcal{M}^{\star}$ is selected. But once there are enough samples, so type III estimation error is low, then learning should be better and types II and IV estimation errors should significantly decrease.

We end this section by analyzing the case of learning via model selection, considering only the models with VC dimension 2 in the partition lattice Learning Space. This family of candidate models has VC dimension two and $2^{7} - 1$ maximal elements, since all of its elements are maximal, but it has the same $\epsilon^{\star}$ as the whole partition lattice Learning Space. The comparison with learning via ERM for this case is in Figure \ref{fig_PLLS3}. The performance of model selection relative to learning via ERM improves a lot for this family of candidate models when compared to the whole partition lattice. For instance, learning via ERM is better than learning via model selection by reusing in no more than 8.4\% of the simulated samples for all sample sizes, and learning via model selection is strictly better in as much as two thirds of the samples for small sample sizes. However, as $\epsilon^{\star}$ and the sample size increases, learning via ERM becomes as good as learning via model selection. More details about this simulation can be found in Appendix \ref{AppResults}.

\begin{figure}[ht]
	\centering
	\includegraphics[width=\linewidth]{res_partition_simulation_d2.pdf}
	\caption{Number of simulations in which learning via model selection in the models of the partition lattice Learning Space with VC dimension 2 is better, worse and as good as learning via ERM with respect to the risk of the estimated hypothesis.} \label{fig_PLLS3}
\end{figure}

\subsection{Domain knowledge in linear regression}
\label{SecRegression}

In this section, we compare learning via ERM on the whole hypotheses space with learning via model selection in linear regression leveraging domain knowledge. We consider two prior information about $h^{\star}$: that it  is sparse and that the effect of some input variables is the same. Denoting $h^{\star}(x) = a_{0}^{\star} + \sum_{i=1}^{d} a_{i}^{\star} x_{i}$, sparsity means that $a_{i}^{\star} = 0$ for $i$ in a subset of $\{1,\dots,d\}$ and two input variables $x_{i}, x_{j}$ having the same effect on the output means that $a_{i}^{\star} = a_{j}^{\star}$.

We consider two cases in our analysis with the following target hypothesis:
\begin{align*}
	h^{\star}_{1}(x) = a_{0}^{\star} + \sum_{i=1}^{d/2} a_{2i}^{\star} x_{2i} & & \text{ and } & & h_{2}^{\star}(x) = \alpha^{\star} \sum_{i=1}^{d/2} x_{2i} + \beta^{\star} \sum_{i=1}^{d/2} x_{2i - 1}
\end{align*}
for $\alpha^{\star}, \beta^{\star} \in \mathbb{R}\setminus\{0\}$ and distinct values of $a_{2}^{\star},a_{4}^{\star},\dots,a_{d}^{\star}$. We assume that $d$ is even for simplification. When the target hypothesis is $h_{1}^{\star}$ the sparsity property holds and when the target hypothesis is $h_{2}^{\star}$ the property that many inputs have the same effect holds.

We consider a \textit{typical} scenario. Let $X$ be a random vector uniformly distributed in $[-1,1]^{d}$ and define the random variables $Y^{(1)}$ and $Y^{(2)}$ as
\begin{align*}
	Y^{(1)} = h^{\star}_{1}(X) + \varepsilon & & \text{ and } & & Y^{(2)} = h^{\star}_{2}(X) + \varepsilon
\end{align*}
in which $\varepsilon$ is independent of $X$ and has mean zero and variance $\sigma^{2} > 0$. Denote the distributions of $(X,Y^{(1)})$ and $(X,Y^{(2)})$ as $P_{1}$ and $P_{2}$, respectively. Clearly $h_{1}^{\star}$ is the target hypothesis under $P_{1}$ and $h_{2}^{\star}$ is the target hypothesis under $P_{2}$. Considering the square loss function, the risk of these target hypotheses are $L_{1}(h_{1}^{\star}) = L_{2}(h_{2}^{\star}) = \sigma^{2}$ in which $L_{i}$ is the expected loss under distribution $P_{i}$.

First, we analyze the variable selection Learning Space generated by $\mathcal{R}_{1}$ from Example \ref{parametric_lattice}. Under distribution $P_{1}$, $\mathcal{M}^{\star,(1)}_{1} = \mathcal{R}_{1}(\{2,4,...,d-2,d\})$ is the linear functions that depend only on the even inputs and
\begin{align*}
	\epsilon^{\star,(1)}_{1} = \min_{i \neq 0} \frac{1}{2} \int_{-1}^{1} (a^{\star}_{2i} x)^{2} \ dx = \frac{1}{3} \min_{i \neq 0} (a^{\star}_{2i})^{2}
\end{align*}
Under distribution $P_{2}$, $\mathcal{M}^{\star,(2)}_{1} = \mathcal{H} = \mathcal{R}_{1}(\{1,...,d\})$ is all linear functions and
\begin{align*}
	\epsilon^{\star,(2)}_{1} = \frac{1}{3} \min\{\alpha^{2},\beta^{2}\}.
\end{align*}
In this case, $d_{VC}(\mathcal{M}^{\star,(1)}_{1}) < d_{VC}(\mathcal{M}^{\star,(2)}_{1})$ and we assume the parameters of $h_{1}^{\star}$ and $h_{2}^{\star}$ are such that $\epsilon^{\star,(1)}_{1} = \epsilon^{\star,(2)}_{1}$.

Now consider the partition lattice Learning Space generated by $\mathcal{R}_{2}$ from Example \ref{parametric_lattice}. Under distribution $P_{1}$, $\mathcal{M}^{\star,(1)}_{2} = \mathcal{R}_{2}(\{\{2\},\{4\},...,\{d-2\},\{d\},\{1,3,...,d-3,d-1\}\})$ is the linear functions in which the odd parameters have the same value and
\begin{align*}
	\epsilon^{\star,(1)}_{2} &= \frac{1}{3} \min\left\{\min_{i \neq j}  (a_{2i}^{\star} - a_{2j}^{\star})^{2},\min_{i} (a_{2_{i}}^{\star})^{2}\right\}
\end{align*}
Under distribution $P_{2}$, $\mathcal{M}^{\star,(2)}_{2} = \mathcal{R}_{2}(\{\{2,4,...,d-2,d\},\{1,3,...,d-3,d-1\}\})$ is the linear functions in which the even parameters and the odds parameters have each a same value and
\begin{align*}
	\epsilon^{\star,(2)}_{2} = \frac{1}{3} (\alpha - \beta)^{2}.
\end{align*}
Now, $d_{VC}(\mathcal{M}^{\star,(2)}_{2}) < d_{VC}(\mathcal{M}^{\star,(1)}_{2})$ and we assume $\epsilon^{\star,(1)}_{2} = \epsilon^{\star,(2)}_{2}$.

We analyze the case $d = 4$ with $a_{0} = 0.1, a_{2}^{\star} = 0.025, a_{4}^{\star} = 0.05, \alpha = 0.025$ and $\beta = 0.05$. In this case, $\epsilon^{\star,(1)}_{1} = \epsilon^{\star,(2)}_{1} = \epsilon^{\star,(1)}_{2} = \epsilon^{\star,(2)}_{2} = 0.0002083$. For each $\sigma \in \{0.01,0.025,0.05\}$ we simulate $1,000$ samples of size $n \in \{16,32,64,128,256\}$ from $P_{1}$ and $P_{2}$ considering a Gaussian distribution for $\varepsilon$. For each sample, we learn by ERM with the whole sample, learn with independent sample in both Learning Spaces by dividing the sample into 50 - 50\% for training and independent sample, and learn by reusing in both Learning Spaces. Model risks are estimated by k-fold cross validation with $k = 4$.

Figure \ref{fig_reg1} presents for each scenario the number of simulated samples in which the risk of the estimated hypothesis via model selection was greater, lesser or equal to the risk of the ERM hypothesis. We first note that learning by reusing is, in general, considerably better than with independent sample. When the target hypothesis is $h_{2}^{\star}$ learning via model selection by reusing in the partition lattice Learning Space is overwhelmingly better than learning via ERM for $\sigma = 0.01$, specially for greater sample sizes. When $\sigma$ increases, the benefit of learning via model selection in this case decreases, although it is still beneficial for greater sample sizes. Now, when the target is $h_{2}^{\star}$, learning via model selection by reusing in the variable selection Learning Space is, in general, as good as learning via ERM. This happens because $h^{\star}_{2}$ is only in the greatest model in this Learning Space, that is $\mathcal{M}^{\star,(2)}_{1} = \mathcal{H}$, and when this model is selected, learning via model selection by reusing is equivalent to ERM. Therefore, there are no benefits on learning via variable selection when the target is $h_{2}^{\star}$.

When the target is $h^{\star}_{1}$, learning via model selection by reusing in the variable selection Learning Space is overwhelmingly better than learning via ERM for $\sigma = 0.01$, specially for greater sample sizes. Again, as $\sigma$ increases, the benefit of learning via model selection in this case decreases, but it is still better than ERM for greater sample sizes. When the target is $h^{\star}_{1}$, learning via model selection by reusing in the partition lattice Learning Space is not as good compared to ERM as doing so in the variable selection Learning Space. Recall that the VC dimension of the target model is lesser on the variable selection Learning Space in this case, and hence, when it is selected, fewer samples are needed to properly learn on it when compared to the partition lattice Learning Space.

\begin{figure}[ht]
	\centering
	\includegraphics[width=\linewidth]{res_regression_simulation.pdf}
	\caption{Number of simulations for each sample size and Learning Space in which learning via model selection is better, worse and as good as learning via ERM with respect to the risk of the estimated hypothesis.} \label{fig_reg1}
\end{figure}

Figure \ref{fig_reg2} presents the average ERM hypothesis bias and types II, III and IV estimation errors over the 1,000 simulated samples for each case when learning via model selection by reusing in each Learning Space. Unlike the example in Section \ref{SecPLLS}, there is no phase transition in learning via model selection and the average of all estimation errors decrease with the sample size. When the target hypothesis is $h^{\star}_{2}$, the average type IV estimation error is smaller in the partition lattice Learning Space, specially for greater values of $\sigma$. When the target hypothesis is $h^{\star}_{1}$, the average type IV estimation error is smaller in the variable selection Learning Space. This fact illustrates that learning is more efficient when the Learning Space is \textit{compatible} with the target hypothesis, in the sense of $d_{VC}(\mathcal{M}^{\star})$ being smaller.

\begin{figure}[ht]
	\centering
	\includegraphics[width=\linewidth]{mean_regression_simulation.pdf}
	\caption{The average with its 95\% confidence interval of the ERM hypothesis bias and types II, III and IV estimation errors over the 1,000 simulated samples for each case when learning via model selection by reusing in each Learning Space. } \label{fig_reg2}
\end{figure}


\subsection{Inserting domain knowledge into Learning Spaces increases generalization}
\label{SecPIEnhance}

The example in Section \ref{SecRegression} is a special case in which generalization may be increased by inserting prior information into the Learning Space. If it is known that the target hypothesis is sparse, then the variable selection Learning Space should be chosen to increase generalization, while if it is known that some input variables have the same effect, the partition lattice Learning space should be chosen.

In ERM, in order to increase generalization, much stronger domain knowledge must be available to consider $\mathcal{H}$ as a simpler hypotheses space without adding a great bias. For instance, knowing that the target is sparse is not enough, and it is necessary to know exactly on what variables it depends on. Moreover, knowing that some variables have a same effect is not enough, and it is necessary to know which variables have a same effect. Conversely, the Learning Spaces can leverage weaker domain knowledge to increase generalization, as illustrated in Section \ref{SecRegression}.

Domain knowledge can be specially leveraged to decrease the complexity of $\mathcal{M}^{\star}$, what should enable greater generalization. Indeed, if it is too complex, then there would only be the possibility of learning a target hypothesis if a highly complex model was selected, and then it was learned from it. This scenario would require a great independent sample to properly learn the target hypothesis once a highly complex model that contains it is selected; actually, a great sample size might also be necessary to select such a complex model. On the other hand, if $\mathcal{M}^{\star}$ has low complexity, then it might be more likely to select it, or a low-complex model which contains it, and fewer samples are necessary to both select a suitable model and learn a hypothesis from it. The results in Figure \ref{fig_reg1} illustrate this fact.

Figure \ref{candidates} presents some examples of how the quality of prior information can be associated with the complexity of $\mathcal{M}^{\star}$. In Figure \ref{candidates} (a), the complexity of $\mathcal{M}^{\star}$ is low, so generalization might be higher. The partition lattice Learning Space of linear functions in Section \ref{SecRegression} when the target is $h_{2}^{\star}$ is an example of this scenario. In Figure \ref{candidates} (b), the complexity of $\mathcal{M}^{\star}$ is not as low, but generalization might be higher than in the scenario in Figure \ref{candidates} (c), in which $\mathcal{M}^{\star}$ is highly complex. The partition lattice Learning Space of linear functions in Section \ref{SecRegression} when the target is $h_{1}^{\star}$ is an example of Figure \ref{candidates} (b) and the variable selection Learning Space when the target is $h^{\star}_{2}$ is an example of Figure \ref{candidates} (c).

\begin{figure*}[ht]
	\centering
	\includegraphics[width=\linewidth]{prior_info}
	\caption{Three examples illustrating how the quality of prior information is associated with the complexity of $\mathcal{M}^{\star}$. The circles represent some models in $\mathbb{L}(\mathcal{H})$ and their area is proportional to the model complexity.} \label{candidates}
\end{figure*}

The reasoning illustrated in Figure \ref{candidates} also argues \textit{against} considering nested models as the family of candidates from a statistical perspective. Nested models represent a chain $\mathcal{M}_{1} \subset \dots \mathcal{M}_{r}$ in the lattice, and the \textit{amount} of domain knowledge has to be \textit{great} to choose the \textit{right} chain. This is illustrated in Figure \ref{candidates} (a) in which $h^{\star}$ is in a simple model of one chain, but a complex model of the other chain. In this case, domain knowledge would need to be spot on so that the \textit{right} chain could be selected. In the example of Section \ref{SecRegression} this would mean that only knowing that some variables have the same effect would not be sufficient and more prior information would have to lead one to select a chain that passes through the partition of the even and odds numbers. Moreover, we argue that lattice Learning Spaces also have advantages from a computational perspective, specially if one looks for suboptimal solutions.

\subsection{Computational aspects of Learning Spaces}
\label{SecComp}

A lattice of candidate models can also aid in computing $\hat{\mathcal{M}}$ by minimizing the empirical error $\hat{L}$ over $\mathbb{L}(\mathcal{H})$. On the one hand, due to the lattice structure, there may exist non-exhaustive algorithms to minimize the empirical error. On the other hand, suboptimal solutions may be suitable since, due to the inclusion of models, a target hypothesis is contained in multiple models, as illustrated in Figure \ref{candidates}, and learning from many of them could yield low generalization.

There exist specialized algorithms for minimizing functions in lattices. In particular, there are the U-curve algorithms \cite{u-curve1,u-curve3,ucurveParallel,reis2018} for minimizing U-shaped functions in Boolean lattices and the more general stochastic lattice descent algorithm (SLDA) \cite{marcondes2023discrete,marcondes2024lattice} which are more efficient for computing suboptimal solutions. In \cite{marcondes2024algorithm}, a two-step SLDA is applied to solve a problem in binary image processing via model selection. 

The partition lattice Learning Space is highly flexible and can be constrained according to prior information to solve specific problems, by identifying equivalences on the parameters that represent hypotheses. Therefore, developing multipurpose optimization algorithms to minimize functions in subsets of the partition lattice is an interesting topic for future research. These algorithms may be applied in several instances when distinct levels of prior information is available. For instance, the algorithm proposed in \cite{yu2023information,yu2023group} to learn a partition with the purpose of explaining the rules of a signal could be adapted to model selection. We are currently working on algorithms for learning via model selection in subsets of partition lattices.

\section{Final Remarks}
\label{FinalRemarks}

Model selection techniques have been historically sought to increase generalization. The main concern of these methods has been to control the complexity of the hypotheses space to avoid overfitting. This is explicitly done in model selection by complexity penalization, and in variable selection, since the complexity of the model is related to the number of variables. However, selecting a low-complexity model is only one step to achieving high generalization, since it is also necessary to have low-risk hypotheses in such a model; otherwise high generalization is unfeasible. In this paper, we investigated how modeling the collection of candidate models based on domain knowledge and prior information may increase generalization.

We presented learning via model selection with cross-validation risk estimation as a systematic data-driven framework consisting of selecting the simplest global minimum of a family of candidate models, and then learning a hypothesis on it with an independent sample or by reusing, seeking to approximate a target hypothesis of $\mathcal{H}$. We studied the distribution-free asymptotics of such framework by showing the convergence of the estimated model to the target one, and of the estimation errors to zero, for both bounded and unbounded loss functions. The case of bounded loss functions was treated with the usual tools of VC theory, while the case of unbounded loss functions required some new technical results, which are an extension of those in \cite{cortes2019}. 

We introduced the maximum discrimination error $\epsilon^{\star}$, formalized the concept of target model, and evidenced the possibility of better learning with a fixed sample size by properly modeling the family of candidate models with a couple of simulations. We argued that by modeling the collection of candidate models based on domain knowledge, it is possible to increase generalization. In particular, we introduced the Learning Spaces and discussed how generalization might be increased by designing them in such a way that the target hypothesis is contained in simple models. The theoretical results also support this assertion, as follows.

First, since $\hat{\mathcal{M}}$ converges to $\mathcal{M}^{\star}$ with probability one, 
\begin{equation*}
	\mathbb{E}(G(\mathcal{\hat{M}})) \xrightarrow{N \rightarrow \infty} G(\mathcal{M}^{\star}),
\end{equation*}
in which $G: \mathbb{C}(\mathcal{H}) \mapsto  \mathbb{R}$ is any real-valued function. The convergence of $\mathbb{E}(G(\mathcal{\hat{M}}))$ ensures that the expectations of functions of $\hat{\mathcal{M}}$ on the right-hand side of inequalities in Theorems \ref{bound_constant} and \ref{bound_constant2} and Corollaries \ref{cor_typeIV} and \ref{cor_typeIV2} tend to the same functions evaluated at $\mathcal{M}^{\star}$ when $N$ tends to infinity. Hence, if one was able to isolate $h^{\star}$ within a model $\mathcal{M}^{\star}$ with small VC dimension, the bounds for types I, II and IV estimation errors will tend to be tighter for a fixed sample size. Furthermore, tighter bounds are obtained in cases in which $d_{VC}(\hat{\mathcal{M}}) \approx d_{VC}(\mathcal{M}^{\star})$, or $\hat{\mathcal{M}} = \mathcal{M}^{\star}$, with high probability (see also Theorems \ref{bound_constant_reusing} and \ref{bound_constant_reusing2}).

Second, if the MDE of $\mathbb{C}(\mathcal{H})$ under $P$ is great, then we need less precision when estimating $L(\mathcal{M})$ for $L(\hat{\mathcal{M}})$ to be equal to $L(\mathcal{M}^{\star})$, and for types III and IV estimation errors to be lesser than a $\epsilon \ll \epsilon^{\star}$ with high probability, so fewer samples are needed to learn a model as good as $\mathcal{M}^{\star}$ and to have lesser types III and IV estimation errors. Moreover, the sample complexity to learn this model is that of the most complex model in $\mathbb{C}(\mathcal{H})$, hence is at most the complexity of a model with VC dimension $d_{VC}(\mathbb{C}(\mathcal{H}))$, which may be lesser than that of $\mathcal{H}$. However, there may be a trade-off between $d_{VC}(\mathbb{C}(\mathcal{H}))$ and the number $\mathfrak{m}(\mathbb{C}(\mathcal{H}))$ of maximal elements in $\mathbb{C}(\mathcal{H})$, as can be seen on the established bounds.

Since the bounds were developed for a general hypotheses space with finite VC dimension in a distribution-free framework, they are not the tightest possible in specific cases, hence an interesting topic for future research would be to apply the methods used here to obtain tighter bounds for restricted classes of hypotheses spaces, candidate models and/or data generating distributions. The results of this paper may be extended when distribution-dependent bounds for types I and II estimation errors are available in the framework of Propositions \ref{propVC} and \ref{propVC2}. In particular, a theoretical study of the estimation errors in the examples in Section \ref{SecEnhanceGen} should be straightforward, so they ought to be the first distribution-dependent settings to be investigated in future studies. Bounding the estimation errors in the framework of \cite{mendelson2004importance} is also a promising line of research in this context.

The established bounds are not useful to determine the number of folds in cross-validation, since we have not made use of the dependence between the samples in different folds and just applied a union bound to consider each pair of samples separately. The bounds could be improved and insights about the optimal number of folds could be obtained if this dependence was considered, but it would be necessary to restrict the study to a specific class of models, since such a result should not be possible in a general framework.

Although outside the scope of this paper, the computational cost of computing $\hat{\mathcal{M}}$ by solving optimization problem \eqref{Ghat} should also be considered when choosing family $\mathbb{C}(\mathcal{H})$. As discussed in Section \ref{SecComp}, this family of candidate models should have some structure that allows an efficient computation of \eqref{Ghat}, or the computation of a suboptimal solution with satisfactory practical results. 

The framework of this paper could be studied considering other risk estimators besides cross-validation. For example, one could consider modeling the family of candidate models based on domain knowledge together with penalization methods, by taking the risk $\hat{L}$ as a penalization of the empirical error. This approach, in both a distribution-free and dependent scenario, could lead to tighter bounds and better practical methods for specific problems.

From an applied perspective, there is a great perspective on investigating how specific domain knowledge may be leveraged to design a Learning Space to solve specific learning problems. The examples in Section \ref{SecEnhanceGen} illustrate how this can be done, but it is necessary to further study not only how to translate prior information into a suitable Learning Space, but also the effect of this modeling on the generalization in other applied problems. This can be investigated from a theoretical perspective, by developing deviation-bounds in specific scenarios, and empirically by comparing learning via model selection based on domain knowledge with other methods.

\section{Proof of results}
\label{SecProof}

\subsection{Results of Section \ref{boundedL}}

We start with a lemma.

\begin{lemma}
	\label{lemma0}
	\begin{equation*}
		\label{inclusion_star}
		\left\{\max\limits_{i \in \mathcal{J}} \text{\textbar}L(\mathcal{M}_{i}) - \hat{L}(\mathcal{M}_{i})\text{\textbar} < \epsilon^{\star}/2\right\} \subset \left\{L(\hat{\mathcal{M}}) = L(\mathcal{M}^{\star})\right\},
	\end{equation*}
\end{lemma}
\begin{proof}
	If
	\begin{equation*}
		\max\limits_{i \in \mathcal{J}} \text{\textbar}L(\mathcal{M}_{i}) - \hat{L}(\mathcal{M}_{i})\text{\textbar} < \epsilon^{\star}/2
	\end{equation*}
	then, for any $i \in \mathcal{J}$ such that $L(\mathcal{M}_{i}) > L(\mathcal{M}^{\star})$, we have
	\begin{align}
		\label{ineq11}
		\hat{L}(\mathcal{M}_{i}) - \hat{L}(\mathcal{M}^{\star}) > L(\mathcal{M}_{i}) - L(\mathcal{M}^{\star}) - \epsilon^{\star} \geq 0,
	\end{align}
	in which the last inequality follows from the definition of $\epsilon^{\star}$. From \eqref{ineq11} it follows that the global minimum of $\nicefrac{\mathbb{C}(\mathcal{H})}{\hat{\sim}}$ with the least VC dimension, that is $\hat{\mathcal{M}}$, is such that $L(\hat{\mathcal{M}}) = L(\mathcal{M}^{\star})$. Indeed, from \eqref{ineq11} it follows that $\hat{L}(\mathcal{M}) > \hat{L}(\mathcal{M}^{\star})$ for all $\mathcal{M} \in \mathbb{C}(\mathcal{H})$ such that $L(\mathcal{M}) > L(\mathcal{M}^{\star})$. Hence, since $\hat{L}(\hat{\mathcal{M}}) \leq \hat{L}(\mathcal{M}^{\star})$, we must have $L(\hat{\mathcal{M}}) = L(\mathcal{M}^{\star})$ implying the desired inclusion of events.
\end{proof}

\begin{proof}[\textbf{Proof of Proposition \ref{proposition_principal}}]
	The result is a direct consequence of Lemma \ref{lemma0}.
\end{proof}

We state and prove a lemma that will aid the proof of Theorem \ref{theorem_principal_convergence}

\begin{lemma}
	\label{lemma1}
	Assume the premises of Theorem \ref{theorem_principal_convergence} are in force. Then, for any $\epsilon > 0$ it holds
	\begin{align*}
		\mathbb{P}&\left(\max\limits_{i \in \mathcal{J}} \text{\textbar}L(\mathcal{M}_{i}) - \hat{L}(\mathcal{M}_{i})\text{\textbar} \geq \epsilon/2\right) \leq \\
		&\leq m \sum_{\mathcal{M} \in \text{ Max } \mathbb{C}(\mathcal{H})} \left[B_{N_{t},\epsilon/8}(d_{VC}(\mathcal{M})) + \hat{B}_{N_{v},\epsilon/4}(d_{VC}(\mathcal{M}))\right]\\
		&\leq m \ \mathfrak{m}(\mathbb{C}(\mathcal{H})) \left[B_{N_{t},\epsilon/8}(d_{VC}(\mathbb{C}(\mathcal{H}))) + \hat{B}_{N_{v},\epsilon/4}(d_{VC}(\mathbb{C}(\mathcal{H})))\right].
	\end{align*}
\end{lemma}
\begin{proof}
	Fix $\epsilon > 0$. Denoting $\hat{h}_{i}^{(j)} \coloneqq \hat{h}_{\mathcal{M}_{i}}^{(j)}$,
	\begin{align}
		\label{dp1} \nonumber
		\mathbb{P}&\left(\max\limits_{i \in \mathcal{J}} \text{\textbar}L(\mathcal{M}_{i}) - \hat{L}(\mathcal{M}_{i})\text{\textbar} \geq \epsilon/2\right) \leq \mathbb{P}\left(\max\limits_{i \in \mathcal{J}} \sum_{j=1}^{m} \frac{1}{m} \text{\textbar}L(\mathcal{M}_{i}) - \hat{L}^{(j)}(\hat{h}^{(j)}_{i})\text{\textbar} > \epsilon/2\right)\\ \nonumber
		&\leq \mathbb{P}\left(\max_{j} \max\limits_{i \in \mathcal{J}} \text{\textbar}L(\mathcal{M}_{i}) - \hat{L}^{(j)}(\hat{h}^{(j)}_{i})\text{\textbar} > \epsilon/2\right)\\ \nonumber
		&\leq \mathbb{P}\left(\bigcup_{j=1}^{m} \left\{\max\limits_{i \in \mathcal{J}} \text{\textbar}L(\mathcal{M}_{i}) - \hat{L}^{(j)}(\hat{h}^{(j)}_{i})\text{\textbar} > \epsilon/2\right\}\right)\\ \nonumber
		&\leq \sum_{j=1}^{m} \mathbb{P}\left(\max\limits_{i \in \mathcal{J}} \text{\textbar}L(\mathcal{M}_{i}) - \hat{L}^{(j)}(\hat{h}^{(j)}_{i})\text{\textbar} > \epsilon/2\right)\\ \nonumber
		&= \sum_{j=1}^{m} \mathbb{P}\left(\max\limits_{i \in \mathcal{J}} \text{\textbar}L(\mathcal{M}_{i}) - L(\hat{h}^{(j)}_{i}) + L(\hat{h}^{(j)}_{i}) - \hat{L}^{(j)}(\hat{h}^{(j)}_{i})\text{\textbar} > \epsilon/2\right)\\ \nonumber
		&\leq \sum_{j=1}^{m} \mathbb{P}\left(\max\limits_{i \in \mathcal{J}} L(\hat{h}^{(j)}_{i}) - L(\mathcal{M}_{i}) + \max\limits_{i \in \mathcal{J}} \text{\textbar}L(\hat{h}^{(j)}_{i}) - \hat{L}^{(j)}(\hat{h}^{(j)}_{i})\text{\textbar} > \epsilon/2\right)\\ \nonumber
		&\leq \sum_{j=1}^{m} \mathbb{P}\left(\max\limits_{i \in \mathcal{J}} L(\hat{h}^{(j)}_{i}) - L(\mathcal{M}_{i}) > \epsilon/4\right) + \mathbb{P}\left(\max\limits_{i \in \mathcal{J}} \text{\textbar}L(\hat{h}^{(j)}_{i}) - \hat{L}^{(j)}(\hat{h}^{(j)}_{i})\text{\textbar} > \epsilon/4\right)\\
		&\leq \sum_{j=1}^{m} \mathbb{P}\left(\max\limits_{i \in \mathcal{J}} L(\hat{h}^{(j)}_{i}) - L(\mathcal{M}_{i}) > \epsilon/4\right) + \mathbb{P}\left(\max\limits_{i \in \mathcal{J}} \sup_{h \in \mathcal{M}_{i}} \text{\textbar}\hat{L}^{(j)}(h) - L(h)\text{\textbar} > \epsilon/4\right)
	\end{align}
	in which in the first inequality we applied the definition of $\hat{L}(\mathcal{M})$. For each $j$, the first probability in \eqref{dp1} is equal to
	\begin{align*}
		\mathbb{P}&\left(\max\limits_{i \in \mathcal{J}} L(\hat{h}^{(j)}_{i}) - L_{\mathcal{D}_{N}^{(j)}}(\hat{h}^{(j)}_{i}) + L_{\mathcal{D}_{N}^{(j)}}(\hat{h}^{(j)}_{i}) - L(\mathcal{M}_{i}) > \epsilon/4\right)\\
		&\leq \mathbb{P}\left(\max\limits_{i \in \mathcal{J}} L(\hat{h}^{(j)}_{i}) - L_{\mathcal{D}_{N}^{(j)}}(\hat{h}^{(j)}_{i}) + L_{\mathcal{D}_{N}^{(j)}}(h_{i}) - L(\mathcal{M}_{i}) > \epsilon/4\right)\\
		&\leq \mathbb{P}\left(\left\{\max\limits_{i \in \mathcal{J}} \text{\textbar}L(\hat{h}^{(j)}_{i}) - L_{\mathcal{D}_{N}^{(j)}}(\hat{h}^{(j)}_{i})\text{\textbar} > \epsilon/8\right\} \bigcup \left\{\max\limits_{i \in \mathcal{J}} \text{\textbar} L_{\mathcal{D}_{N}^{(j)}}(h_{i}) - L(\mathcal{M}_{i}) \text{\textbar} > \epsilon/8\right\}\right)\\
		&\leq \mathbb{P}\left(\max\limits_{i \in \mathcal{J}} \sup_{h \in \mathcal{M}_{i}} \text{\textbar}L_{\mathcal{D}_{N}^{(j)}}(h) - L(h)\text{\textbar} > \epsilon/8\right),
	\end{align*}
	in which the first inequality follows from the fact that $L_{\mathcal{D}_{N}^{(j)}}(\hat{h}^{(j)}_{i}) \leq L_{\mathcal{D}_{N}^{(j)}}(h_{i})$, and the last follows since $L(\mathcal{M}_{i}) = L(h_{i})$. We conclude that
	\begin{align*}
		&\mathbb{P}\left(\max\limits_{i \in \mathcal{J}} \text{\textbar}L(\mathcal{M}_{i}) - \hat{L}(\mathcal{M}_{i})\text{\textbar} \geq \epsilon/2\right) \\
		&\leq \sum_{j=1}^{m} \mathbb{P}\left(\max\limits_{i \in \mathcal{J}} \sup_{h \in \mathcal{M}_{i}} \text{\textbar}L_{\mathcal{D}_{N}^{(j)}}(h) - L(h)\text{\textbar} > \epsilon/8\right) + \mathbb{P}\left(\max\limits_{i \in \mathcal{J}} \sup_{h \in \mathcal{M}_{i}} \text{\textbar}\hat{L}^{(j)}(h) - L(h)\text{\textbar} > \epsilon/4\right).
	\end{align*}	
	If $\mathcal{M}_{1} \subset \mathcal{M}_{2}$ then, for any $\epsilon > 0$ and $j = 1, \dots, m$, we have the following inclusion of events
	\begin{align*}
		&\left\{\sup_{h \in \mathcal{M}_{1}} \text{\textbar}\hat{L}^{(j)}(h) - L(h)\text{\textbar} > \epsilon\right\} \subset \left\{\sup_{h \in \mathcal{M}_{2}} \text{\textbar}\hat{L}^{(j)}(h) - L(h)\text{\textbar} > \epsilon\right\}\\
		&\left\{\sup_{h \in \mathcal{M}_{1}} \text{\textbar}L_{\mathcal{D}_{N}^{(j)}}(h) - L(h)\text{\textbar} > \epsilon\right\} \subset \left\{\sup_{h \in \mathcal{M}_{2}} \text{\textbar}L_{\mathcal{D}_{N}^{(j)}}(h) - L(h)\text{\textbar} > \epsilon\right\},
	\end{align*}
	hence it is true that
	\begin{align*}
		&\left\{\max\limits_{i \in \mathcal{J}} \sup_{h \in \mathcal{M}_{i}} \text{\textbar}\hat{L}^{(j)}(h) - L(h)\text{\textbar} > \epsilon/4\right\} \subset \left\{\max_{\mathcal{M} \in \text{ Max } \mathbb{C}(\mathcal{H})} \sup_{h \in \mathcal{M}} \text{\textbar}\hat{L}^{(j)}(h) - L(h)\text{\textbar} > \epsilon/4\right\}\\
		&\left\{\max\limits_{i \in \mathcal{J}} \sup_{h \in \mathcal{M}_{i}} \text{\textbar}L_{\mathcal{D}_{N}^{(j)}}(h) - L(h)\text{\textbar} > \epsilon/8\right\} \subset \left\{\max_{\mathcal{M} \in \text{ Max } \mathbb{C}(\mathcal{H})} \sup_{h \in \mathcal{M}} \text{\textbar}L_{\mathcal{D}_{N}^{(j)}}(h) - L(h)\text{\textbar} > \epsilon/8\right\},
	\end{align*}
	which yields
	\begin{align}
		\label{conlusion_cond} \nonumber
		\mathbb{P}&\left(\max\limits_{i \in \mathcal{J}} \text{\textbar}L(\mathcal{M}_{i}) - \hat{L}(\mathcal{M}_{i})\text{\textbar} \geq \epsilon/2\right) \\  \nonumber
		&\leq \sum_{j=1}^{m} \sum_{\mathcal{M} \in \text{ Max } \mathbb{C}(\mathcal{H})} \mathbb{P}\left(\sup_{h \in \mathcal{M}} \text{\textbar}L_{\mathcal{D}_{N}^{(j)}}(h) - L(h)\text{\textbar} > \epsilon/8\right) + \mathbb{P}\left(\sup_{h \in \mathcal{M}} \text{\textbar}\hat{L}^{(j)}(h) - L(h)\text{\textbar} > \epsilon/4\right)\\ \nonumber
		&\leq m \sum_{\mathcal{M} \in \text{ Max } \mathbb{C}(\mathcal{H})} \left[B_{N_{t},\epsilon/8}(d_{VC}(\mathcal{M})) + \hat{B}_{N_{v},\epsilon/4}(d_{VC}(\mathcal{M}))\right]\\
		&\leq m \ \mathfrak{m}(\mathbb{C}(\mathcal{H})) \left[B_{N_{t},\epsilon/8}(d_{VC}(\mathbb{C}(\mathcal{H}))) + \hat{B}_{N_{v},\epsilon/4}(d_{VC}(\mathbb{C}(\mathcal{H})))\right],
	\end{align}
	in which the last inequality follows from the fact that both $\hat{B}_{N_{v},\epsilon/4}$ and $B_{N_{t},\epsilon/8}$ are increasing functions, and $d_{VC}(\mathbb{C}(\mathcal{H})) = \max_{\mathcal{M} \in \mathbb{C}(\mathcal{H})} d_{VC}(\mathcal{M})$.
\end{proof}

\begin{proof}[\textbf{Proof of Theorem \ref{theorem_principal_convergence}}]
	It follows from Lemma \ref{lemma1} that
	\begin{align*}
		\mathbb{P}&\left(\max\limits_{i \in \mathcal{J}} \text{\textbar}L(\mathcal{M}_{i}) - \hat{L}(\mathcal{M}_{i})\text{\textbar} \geq \epsilon^{\star}/2\right) \\  \nonumber
		&\leq m \sum_{\mathcal{M} \in \text{ Max } \mathbb{C}(\mathcal{H})} \left[B_{N_{t},\epsilon^{\star}/8}(d_{VC}(\mathcal{M})) + \hat{B}_{N_{v},\epsilon^{\star}/4}(d_{VC}(\mathcal{M}))\right]\\
		&\leq m \ \mathfrak{m}(\mathbb{C}(\mathcal{H})) \left[B_{N_{t},\epsilon^{\star}/8}(d_{VC}(\mathbb{C}(\mathcal{H}))) + \hat{B}_{N_{v},\epsilon^{\star}/4}(d_{VC}(\mathbb{C}(\mathcal{H})))\right],
	\end{align*}
	so the result follows from Proposition \ref{proposition_principal} since
	\begin{equation*}
		\{L(\hat{\mathcal{M}}) \neq L(\mathcal{M}^{\star})\} \subset \left\{\max\limits_{i \in \mathcal{J}} \text{\textbar}L(\mathcal{M}_{i}) - \hat{L}(\mathcal{M}_{i})\text{\textbar} \geq \epsilon^{\star}/2\right\}.
	\end{equation*}	
	
	If the almost sure convergences \eqref{as_conv} hold, then
	\begin{equation}
		\label{as_proof}
		\hat{L}(\mathcal{M}) \xrightarrow[N \to \infty]{\text{a.s.}} L(\mathcal{M})
	\end{equation}
	for all $\mathcal{M} \in \mathbb{C}(\mathcal{H})$, since, if $L(h) = \hat{L}^{(j)}(h) = L_{\mathcal{D}_{N}}^{(j)}(h)$ for all $j = 1,\dots,m$ and $h \in \mathcal{H}$, then $\hat{L}(\mathcal{M}) = L(\mathcal{M})$ for all $\mathcal{M} \in \mathbb{C}(\mathcal{H})$. Observe that
	\begin{align}
		\label{incl}
		\left\{\max_{\mathcal{M} \in \mathbb{C}(\mathcal{H})} \text{\textbar}L(\mathcal{M}) - \hat{L}(\mathcal{M})\text{\textbar} = 0\right\} \subset \left\{\hat{\mathcal{M}} = \mathcal{M}^{\star}\right\},
	\end{align}
	since, if the estimated risk $\hat{L}$ is equal to the out-of-sample risk $L$, then the definitions of $\hat{\mathcal{M}}$ and $\mathcal{M}^{\star}$ coincide. As the probability of the event on the left hand-side of \eqref{incl} converges to one if \eqref{as_proof} is true, we conclude that, if \eqref{as_conv} holds, then $\hat{\mathcal{M}}$ converges to $\mathcal{M}^{\star}$ with probability one.
\end{proof}

\begin{proof}[\textbf{Proof of Theorem \ref{CVModelconvergence}}]
	We need to show that \eqref{as_conv} holds in these instances. For any $\epsilon > 0$, by Corollary \ref{cor3TypeI},	
	\begin{align*}
		&\mathbb{P}\left(\max_{\mathcal{M} \in \mathbb{C}(\mathcal{H})} \max_{j} \sup\limits_{h \in \mathcal{M}} \text{\textbar}L_{\mathcal{D}_{N}^{(j)}}(h) - L(h)\text{\textbar} > \epsilon\right) \leq \sum_{j=1}^{m} \mathbb{P}\left(\sup\limits_{h \in \mathcal{H}} \text{\textbar}L_{\mathcal{D}_{N}^{(j)}}(h) - L(h)\text{\textbar} > \epsilon\right)\\
		&\leq m \ 8 \exp\left\{d_{VC}(\mathcal{H}) \left(1 + \ln \frac{N_{t}}{d_{VC}(\mathcal{H})} - N_{t}\frac{\epsilon^{2}}{32C^{2}}\right)\right\}.
	\end{align*}
	By the inequality above, and Borel-Cantelli Lemma \cite[Theorem~4.3]{billingsley2008}, the first convergence in \eqref{as_conv} holds. The second convergence holds since the inequality above is also true, but with $L_{\mathcal{D}_{N}^{(j)}}$ and $N_{t}$ interchanged by $\hat{L}^{(j)}$ and $N_{v}$, the empirical risk and size of the $j$-th validation sample.
\end{proof}

\begin{proof}[\textbf{Proof of Theorem \ref{bound_constant}}]
	We first note that
	\begin{align}
		\label{Sum1} \nonumber
		\mathbb{P}&\left(\sup\limits_{h \in \mathcal{\hat{M}}} \text{\textbar}L_{\tilde{\mathcal{D}}_{M}}(h) - L(h) \text{\textbar} > \epsilon \right)  = \mathbb{E} \Bigg(\mathbb{P}\left(\sup\limits_{h \in \mathcal{\hat{M}}} \text{\textbar}L_{\tilde{\mathcal{D}}_{M}}(h) - L(h) \text{\textbar} > \epsilon \text{\textbar}\mathcal{\hat{M}}\right)\Bigg)\\ \nonumber
		& = \sum_{i \in \mathcal{J}} \mathbb{P}\left(\sup\limits_{h \in \mathcal{\hat{M}}} \text{\textbar}L_{\tilde{\mathcal{D}}_{M}}(h) - L(h) \text{\textbar} > \epsilon \text{\textbar}\mathcal{\hat{M}} = \mathcal{M}_{i}\right) \mathbb{P}(\mathcal{\hat{M}} = \mathcal{M}_{i})\\
		& = \sum_{i \in \mathcal{J}} \mathbb{P}\left(\sup\limits_{h \in \mathcal{M}_{i}} \text{\textbar}L_{\tilde{\mathcal{D}}_{M}}(h) - L(h) \text{\textbar} > \epsilon \text{\textbar}\mathcal{\hat{M}} = \mathcal{M}_{i}\right) \mathbb{P}(\mathcal{\hat{M}} = \mathcal{M}_{i}).
	\end{align}
	Fix $\mathcal{M} \in \mathbb{C}(\mathcal{H})$ with $\mathbb{P}(\mathcal{\hat{M}} = \mathcal{M}) > 0$. We claim that
	\begin{align}
		\label{cond_independence}
		\mathbb{P}\left(\sup\limits_{h \in \mathcal{M}} \text{\textbar}L_{\tilde{\mathcal{D}}_{M}}(h) - L(h) \text{\textbar} > \epsilon \text{\textbar}\mathcal{\hat{M}} = \mathcal{M}\right) = \mathbb{P}\left(\sup\limits_{h \in \mathcal{M}} \text{\textbar}L_{\tilde{\mathcal{D}}_{M}}(h) - L(h) \text{\textbar} > \epsilon\right).
	\end{align}
	Indeed, since $\tilde{\mathcal{D}}_{M}$ is independent of $\mathcal{D}_{N}$, the event 
	\begin{equation*}
		\left\{\sup_{h \in \mathcal{M}} \text{\textbar}L_{\tilde{\mathcal{D}}_{M}}(h) - L(h) \text{\textbar} > \epsilon\right\}
	\end{equation*}
	is independent of $\{\mathcal{\hat{M}} = \mathcal{M}\}$, as the former depends solely on $\tilde{\mathcal{D}}_{M}$, and the latter solely on $\mathcal{D}_{N}$. Hence, by applying bound $\eqref{bound_theoremBC}$ to each positive probability in the sum \eqref{Sum1}, we obtain that
	\begin{align*}
		\mathbb{P}\left(\sup\limits_{h \in \mathcal{\hat{M}}} \text{\textbar}L_{\tilde{\mathcal{D}}_{M}}(h) - L(h) \text{\textbar} > \epsilon \right) & \leq \sum_{i \in \mathcal{J}} B_{M,\epsilon}^{I}(d_{VC}(\mathcal{M}_{i}))  \mathbb{P}(\mathcal{\hat{M}} = \mathcal{M}_{i})\\
		& = \mathbb{E} \left(B_{M,\epsilon}^{I}(d_{VC}(\mathcal{\hat{M}}))\right) \leq B_{N,\epsilon}^{I}(d_{VC}(\mathbb{C}(\mathcal{H}))),
	\end{align*}
	as desired, in which the last inequality follows from the fact that $B_{M,\epsilon}^{I}$ is an increasing function and $d_{VC}(\mathbb{C}(\mathcal{H})) = \max_{\mathcal{M} \in \mathbb{C}(\mathcal{H})} d_{VC}(\mathcal{M})$.
	
	The bound for type II estimation error may be obtained similarly, since
	\begin{align*}
		\mathbb{P}&\left(L(\hat{h}_{\mathcal{\hat{M}}}^{\tilde{\mathcal{D}}_{M}}) - L(h^{\star}_{\mathcal{\hat{M}}}) > \epsilon \right) = \mathbb{E} \Bigg(\mathbb{P}\left(L(\hat{h}_{\mathcal{\hat{M}}}^{\tilde{\mathcal{D}}_{M}}) - L(h^{\star}_{\mathcal{\hat{M}}}) > \epsilon \text{\textbar} \mathcal{\hat{M}} \right)\Bigg)\\
		& = \sum_{i \in \mathcal{J}} \mathbb{P}\left(L(\hat{h}_{\mathcal{\hat{M}}}^{\tilde{\mathcal{D}}_{M}}) - L(h^{\star}_{\mathcal{\hat{M}}}) > \epsilon \text{\textbar} \mathcal{\hat{M}} = \mathcal{M}_{i} \right) \mathbb{P}(\mathcal{\hat{M}} = \mathcal{M}_{i})\\
		& = \sum_{i \in \mathcal{J}} \mathbb{P}\left(L(\hat{h}_{\mathcal{M}_{i}}^{\tilde{\mathcal{D}}_{M}}) - L(h^{\star}_{\mathcal{M}_{i}}) > \epsilon \text{\textbar} \mathcal{\hat{M}} = \mathcal{M}_{i} \right) \mathbb{P}(\mathcal{\hat{M}} = \mathcal{M}_{i})\\
		& = \sum_{i \in \mathcal{J}} \mathbb{P}\left(L(\hat{h}_{\mathcal{M}_{i}}^{\tilde{\mathcal{D}}_{M}}) - L(h^{\star}_{\mathcal{M}_{i}}) > \epsilon \right) \mathbb{P}(\mathcal{\hat{M}} = \mathcal{M}_{i}),
	\end{align*}
	and $B^{II}_{M,\epsilon}(d_{VC}(\mathcal{M}_{i}))$ is a bound for the probabilities inside the sum by \eqref{bound_theoremBC}. The assertion that types I and II estimation errors are asymptotically zero when $d_{VC}(\mathbb{C}(\mathcal{H})) < \infty$ is immediate from the established bounds.
\end{proof}

\begin{proof}[\textbf{Proof of Theorem \ref{theorem_tipeIII}}]
	We first show that
	\begin{align}
		\label{lemma_inside}
		\mathbb{P}\left(L(h_{\hat{\mathcal{M}}}^{\star}) - L(h^{\star}) > \epsilon\right) \leq \mathbb{P}\left(\max\limits_{i \in \mathcal{J}} \text{\textbar}\hat{L}(\mathcal{M}_{i}) - L(\mathcal{M}_{i})\text{\textbar} > (\epsilon \vee \epsilon^{\star})/2 \right).
	\end{align}
	If $\epsilon \leq \epsilon^{\star}$ then, by Lemma \ref{lemma0}, we have that
	\begin{align}
		\label{incl1}
		\left\{\max\limits_{i \in \mathcal{J}} \text{\textbar}\hat{L}(\mathcal{M}_{i}) - L(\mathcal{M}_{i})\text{\textbar} < (\epsilon \vee \epsilon^{\star})/2\right\} \subset \left\{L(\hat{\mathcal{M}}) = L(\mathcal{M}^{\star})\right\} \subset \left\{L(h_{\hat{\mathcal{M}}}^{\star}) - L(h^{\star}) < \epsilon\right\},
	\end{align}
	since $L(h^{\star}_{\hat{\mathcal{M}}}) = L(\hat{\mathcal{M}})$ and $L(h^{\star}_{\mathcal{M}^{\star}}) = L(\mathcal{M}^{\star})$, so \eqref{lemma_inside} follows in this case.
	
	Now, if $\epsilon > \epsilon^{\star}$ and $\max\limits_{i \in \mathcal{J}} \text{\textbar}\hat{L}(\mathcal{M}_{i}) - L(\mathcal{M}_{i})\text{\textbar} < \epsilon/2$, then
	\begin{align*}
		L(\hat{\mathcal{M}}) - L(\mathcal{M}^{\star}) &= [L(\hat{\mathcal{M}}) - \hat{L}(\mathcal{M}^{\star})] - [L(\mathcal{M}^{\star}) - \hat{L}(\mathcal{M}^{\star})]\\
		&\leq [L(\hat{\mathcal{M}}) - \hat{L}(\hat{\mathcal{M}})] - [L(\mathcal{M}^{\star}) - \hat{L}(\mathcal{M}^{\star})]\\
		&\leq \epsilon/2 + \epsilon/2 = \epsilon,
	\end{align*}
	in which the first inequality follows from the fact that the minimum of $\hat{L}$ is attained at $\hat{\mathcal{M}}$, and the last inequality follows from $\max\limits_{i \in \mathcal{J}} \text{\textbar}\hat{L}(\mathcal{M}_{i}) - L(\mathcal{M}_{i})\text{\textbar} < \epsilon/2$. Since $L(\hat{\mathcal{M}}) - L(\mathcal{M}^{\star}) = L(h_{\hat{\mathcal{M}}}^{\star}) - L(h^{\star})$, we also have the inclusion of events
	\begin{align}
		\label{incl2}
		\left\{\max\limits_{i \in \mathcal{J}} \text{\textbar}\hat{L}(\mathcal{M}_{i}) - L(\mathcal{M}_{i})\text{\textbar} < (\epsilon \vee \epsilon^{\star})/2\right\} \subset \left\{L(h_{\hat{\mathcal{M}}}^{\star}) - L(h^{\star}) < \epsilon\right\},
	\end{align}
	when $\epsilon > \epsilon^{\star}$. From \eqref{incl1} and \eqref{incl2} follows \eqref{lemma_inside}, as desired.
	
	It follows from Lemma \ref{lemma1} that
	\begin{align}
		\label{conclusion2} \nonumber
		\mathbb{P}&\left(\max\limits_{i \in \mathcal{J}} \text{\textbar}L(\mathcal{M}_{i}) - \hat{L}(\mathcal{M}_{i})\text{\textbar} \geq (\epsilon \vee \epsilon^{\star})/2\right) \leq\\ \nonumber
		&\leq m \sum_{\mathcal{M} \in \text{Max } \mathbb{C}(\mathcal{H})} \left[B_{N_{t},(\epsilon \vee \epsilon^{\star})/8}(d_{VC}(\mathcal{M})) + \hat{B}_{N_{v},(\epsilon \vee \epsilon^{\star})/4}(d_{VC}(\mathcal{M}))\right]\\
		& m \ \mathfrak{m}(\mathbb{C}(\mathcal{H})) \left[B_{N_{t},(\epsilon \vee \epsilon^{\star})/8}(d_{VC}(\mathbb{C}(\mathcal{H}))) + \hat{B}_{N_{v},(\epsilon \vee \epsilon^{\star})/4}(d_{VC}(\mathbb{C}(\mathcal{H})))\right].
	\end{align}
	The result follows combining \eqref{lemma_inside} and \eqref{conclusion2}.	
\end{proof}

\subsection{Results of Section \ref{SecUnbounded}}

\begin{proof}[\textbf{Proof of Theorem \ref{theorem_principal_convergence_unbounded}}]
	We claim that
	\begin{equation}
		\label{implication1}
		1 - \delta < \frac{\hat{L}(\mathcal{M}_{i})}{L(\mathcal{M}_{i})} < 1 + \delta, \ \forall i \in \mathcal{J} \implies \max\limits_{i \in \mathcal{J}} \ \text{\textbar}\hat{L}(\mathcal{M}_{i}) - L(\mathcal{M}_{i})\text{\textbar} < \frac{\epsilon^{\star}}{2}.
	\end{equation}
	Indeed, the left-hand side of \eqref{implication1} implies
	\begin{equation*}
		\begin{cases}
			L(\mathcal{M}_{i}) - \hat{L}(\mathcal{M}_{i}) < \frac{\epsilon^{\star} L(\mathcal{M}_{i})}{2 \max\limits_{i \in \mathcal{J}} L(\mathcal{M}_{i})} < \frac{\epsilon^{\star}}{2}\\
			\hat{L}(\mathcal{M}_{i}) - L(\mathcal{M}_{i}) < \frac{\epsilon^{\star} L(\mathcal{M}_{i})}{2 \max\limits_{i \in \mathcal{J}} L(\mathcal{M}_{i})} < \frac{\epsilon^{\star}}{2}\\
		\end{cases} \ \forall i \in \mathcal{J},
	\end{equation*}
	as desired. In particular, it follows from Lemma \ref{lemma0} that
	\begin{equation}
		\label{writeP}
		\mathbb{P}\left(L(\hat{\mathcal{M}}) \neq L(\mathcal{M}^{\star})\right) \leq \mathbb{P}\left(\min\limits_{i \in \mathcal{J}} \frac{\hat{L}(\mathcal{M}_{i})}{L(\mathcal{M}_{i})} \leq 1 - \delta\right) + \mathbb{P}\left(\max\limits_{i \in \mathcal{J}} \frac{\hat{L}(\mathcal{M}_{i})}{L(\mathcal{M}_{i})} \geq 1 + \delta\right)
	\end{equation}
	hence it is enough to bound both probabilities on the right-hand side of the expression above.
	
	The first probability in \eqref{writeP} may be written as
	\begin{align}	
		\label{writeP2}
		\mathbb{P}\left(\max\limits_{i \in \mathcal{J}} \ \frac{L(\mathcal{M}_{i}) - \hat{L}(\mathcal{M}_{i})}{L(\mathcal{M}_{i})} \geq \delta\right) \leq \sum_{j=1}^{m} \mathbb{P}\left(\max\limits_{i \in \mathcal{J}} \ \frac{L(\mathcal{M}_{i}) - \hat{L}^{(j)}(\hat{h}^{(j)}_{i})}{L(\mathcal{M}_{i})} \geq \delta\right),
	\end{align}
	in which the inequality follows from a union bound. Since $x \mapsto  \frac{x - \alpha}{x}$ is increasing, and $L(\mathcal{M}_{i}) \leq L(\hat{h}^{(j)}_{i})$ for every $j = 1,\dots,m$, each probability in \eqref{writeP2} is bounded by
	\begin{equation}
		\label{res1}
		\mathbb{P}\left(\max\limits_{i \in \mathcal{J}} \ \frac{L(\hat{h}^{(j)}_{i}) - \hat{L}^{(j)}(\hat{h}^{(j)}_{i})}{L(\hat{h}^{(j)}_{i})} \geq \delta\right) \leq \mathbb{P}\left(\max\limits_{i \in \mathcal{J}} \sup\limits_{h \in \mathcal{M}_{i}} \ \frac{\text{\textbar}L(h) - \hat{L}^{(j)}(h)\text{\textbar}}{L(h)} \geq \delta\right).
	\end{equation}
	
	We turn to the second probability in \eqref{writeP} which can be written as
	\begin{equation}
		\label{writeP3}
		\mathbb{P}\left(\max\limits_{i \in \mathcal{J}} \ \frac{\hat{L}(\mathcal{M}_{i}) - L(\mathcal{M}_{i})}{L(\mathcal{M}_{i})} \geq \delta\right) \leq \sum_{j=1}^{m} \mathbb{P}\left(\max\limits_{i \in \mathcal{J}} \ \frac{\hat{L}^{(j)}(\hat{h}^{(j)}_{i}) - L(\mathcal{M}_{i})}{L(\mathcal{M}_{i})} \geq \delta\right),
	\end{equation}
	in which again the inequality follows from a union bound. In order to bound each probability in \eqref{writeP3} we intersect its event with
	\begin{equation*}
		\max\limits_{i \in \mathcal{J}} \frac{L(\hat{h}_{i}^{(j)})}{L(\mathcal{M}_{i})} \leq \frac{1}{1 - \delta} \iff \max\limits_{i \in \mathcal{J}} \frac{L(\hat{h}_{i}^{(j)}) - L(\mathcal{M}_{i})}{L(\hat{h}_{i}^{(j)})} \leq \delta,
	\end{equation*}
	and its complement, to obtain
	\begin{align}
		\label{res2} \nonumber
		&\mathbb{P}\left(\max\limits_{i \in \mathcal{J}} \ \frac{\hat{L}^{(j)}(\hat{h}^{(j)}_{i}) - L(\mathcal{M}_{i})}{L(\mathcal{M}_{i})} \geq \delta\right) \leq \mathbb{P}\left(\max\limits_{i \in \mathcal{J}} \frac{L(\hat{h}_{i}^{(j)}) - L(\mathcal{M}_{i})}{L(\hat{h}_{i}^{(j)})} \geq \delta\right)\\ \nonumber
		&+ \mathbb{P}\left(\max\limits_{i \in \mathcal{J}} \ \left(\frac{L(\hat{h}_{i}^{(j)})}{L(\mathcal{M}_{i})}\right) \frac{\hat{L}^{(j)}(\hat{h}^{(j)}_{i}) - L(\mathcal{M}_{i})}{L(\hat{h}_{i}^{(j)})} \geq \delta,\max\limits_{i \in \mathcal{J}} \frac{L(\hat{h}_{i}^{(j)})}{L(\mathcal{M}_{i})} \leq \frac{1}{1 - \delta}\right)\\ \nonumber
		&\leq \mathbb{P}\left(\max\limits_{i \in \mathcal{J}} \frac{L(\hat{h}_{i}^{(j)}) - L(\mathcal{M}_{i})}{L(\hat{h}_{i}^{(j)})} \geq \delta\right) + \mathbb{P}\left(\max\limits_{i \in \mathcal{J}} \ \frac{\hat{L}^{(j)}(\hat{h}^{(j)}_{i}) - L(\mathcal{M}_{i})}{L(\hat{h}_{i}^{(j)})} \geq \delta(1-\delta)\right)\\
		&\leq \mathbb{P}\left(\max\limits_{i \in \mathcal{J}} \sup\limits_{h \in \mathcal{M}_{i}} \frac{\text{\textbar}L_{\mathcal{D}_{N}}^{(j)}(h) - L(h)\text{\textbar}}{L(h)} \geq \frac{\delta}{2}\right) + \mathbb{P}\left(\max\limits_{i \in \mathcal{J}} \ \frac{\hat{L}^{(j)}(\hat{h}^{(j)}_{i}) - L(\mathcal{M}_{i})}{L(\hat{h}_{i}^{(j)})} \geq \delta(1-\delta)\right)
	\end{align}
	in which the last inequality follows from Lemma \ref{lemmaTypeItoII}.
	
	It remains to bound the second probability in \eqref{res2}. We have that it is equal to
	\begin{align}
		\label{res3} \nonumber
		&\mathbb{P}\left(\max\limits_{i \in \mathcal{J}} \ \frac{\hat{L}^{(j)}(\hat{h}^{(j)}_{i}) - L(\hat{h}^{(j)}_{i}) + L(\hat{h}^{(j)}_{i}) - L(\mathcal{M}_{i})}{L(\hat{h}_{i}^{(j)})} \geq \delta(1-\delta)\right)\\ \nonumber
		&\leq \mathbb{P}\left(\max\limits_{i \in \mathcal{J}} \ \frac{\hat{L}^{(j)}(\hat{h}^{(j)}_{i}) - L(\hat{h}^{(j)}_{i})}{L(\hat{h}_{i}^{(j)})} \geq \frac{\delta(1-\delta)}{2}\right) + \mathbb{P}\left(\max\limits_{i \in \mathcal{J}} \ \frac{L(\hat{h}^{(j)}_{i}) - L(\mathcal{M}_{i})}{L(\hat{h}_{i}^{(j)})} \geq \frac{\delta(1-\delta)}{2}\right)\\
		&\leq \mathbb{P}\left(\max\limits_{i \in \mathcal{J}} \sup\limits_{h \in \mathcal{M}_{i}} \ \frac{\text{\textbar}\hat{L}^{(j)}(h) - L(h)\text{\textbar}}{L(h)} \geq \frac{\delta(1-\delta)}{2}\right) + \mathbb{P}\left(\max\limits_{i \in \mathcal{J}} \sup\limits_{h \in \mathcal{M}_{i}} \ \frac{\text{\textbar}L(h) - L_{\mathcal{D}_{N}}(h)\text{\textbar}}{L(h)} \geq \frac{\delta(1-\delta)}{4}\right),
	\end{align}
	in which the last inequality follows again from Lemma \ref{lemmaTypeItoII}. From (\ref{writeP}-\ref{res3}), it follows that
	\begin{align*}
		\mathbb{P}\left(L(\hat{\mathcal{M}}) \neq L(\mathcal{M}^{\star})\right)\leq &2 \sum_{j=1}^{m} \Bigg[\mathbb{P}\left(\max\limits_{i \in \mathcal{J}} \sup\limits_{h \in \mathcal{M}_{i}} \ \frac{\text{\textbar}\hat{L}^{(j)}(h) - L(h)\text{\textbar}}{L(h)} \geq \frac{\delta(1-\delta)}{2}\right) +\\
		&\mathbb{P}\left(\max\limits_{i \in \mathcal{J}} \sup\limits_{h \in \mathcal{M}_{i}} \ \frac{\text{\textbar}L(h) - L_{\mathcal{D}_{N}}(h)\text{\textbar}}{L(h)} \geq \frac{\delta(1-\delta)}{4}\right)\Bigg]\\
		&\leq 2 \sum_{j=1}^{m} \sum_{\mathcal{M} \in \text{ Max } \mathbb{C}(\mathcal{H})} \Bigg[\mathbb{P}\left(\sup\limits_{h \in \mathcal{M}} \ \frac{\text{\textbar}\hat{L}^{(j)}(h) - L(h)\text{\textbar}}{L(h)} \geq \frac{\delta(1-\delta)}{2}\right) +\\
		&\mathbb{P}\left(\sup\limits_{h \in \mathcal{M}} \ \frac{\text{\textbar}L(h) - L_{\mathcal{D}_{N}}(h)\text{\textbar}}{L(h)} \geq \frac{\delta(1-\delta)}{4}\right)\Bigg],
	\end{align*}
	in which the inequality holds by the same arguments as in \eqref{conlusion_cond}, hence
	\begin{equation*}
		\mathbb{P}\left(L(\hat{\mathcal{M}}) \neq L(\mathcal{M}^{\star})\right) \leq 2m \ \mathfrak{m}(\mathbb{C}(\mathcal{H})) \left[\hat{B}_{N_{v},\frac{\delta(1-\delta)}{2}}(d_{VC}(\mathbb{C}(\mathcal{H}))) + B_{N_{t},\frac{\delta(1-\delta)}{4}}(d_{VC}(\mathbb{C}(\mathcal{H})))\right].
	\end{equation*}
	
	If the almost sure convergences \eqref{as_conv2} hold, then $L(h) = L_{\mathcal{D}_{N}}^{(j)}(h) = \hat{L}^{(j)}(h)$ for all $j$ and $h \in \mathcal{H}$, and the definitions of $\mathcal{M}^{\star}$ and $\hat{\mathcal{M}}$ coincide.
\end{proof}

\begin{proof}[\textbf{Proof of Theorem \ref{theorem_tipeIII2}}]
	We show that
	\begin{align}
		\label{implication2}
		\mathbb{P}\left(L(h_{\hat{\mathcal{M}}}^{\star}) - L(h^{\star}) > \epsilon\right) \geq \mathbb{P}\left(\frac{L(h_{\hat{\mathcal{M}}}^{\star}) - L(h^{\star})}{L(h_{\hat{\mathcal{M}}}^{\star})} > \frac{\epsilon}{L(\mathcal{M}^{\star})}\right),
	\end{align}
	so from \eqref{lemma_inside} and \eqref{implication1} will follow that
	\begin{equation*}
		\mathbb{P}\left(\frac{L(h_{\hat{\mathcal{M}}}^{\star}) - L(h^{\star})}{L(h_{\hat{\mathcal{M}}}^{\star})} > \frac{\epsilon}{L(\mathcal{M}^{\star})}\right) \leq \mathbb{P}\left(\min\limits_{i \in \mathcal{J}} \frac{\hat{L}(\mathcal{M}_{i})}{L(\mathcal{M}_{i})} \leq 1 - \delta^\prime\right) + \mathbb{P}\left(\max\limits_{i \in \mathcal{J}} \frac{\hat{L}(\mathcal{M}_{i})}{L(\mathcal{M}_{i})} \geq 1 + \delta^\prime\right),
	\end{equation*}
	and the result is then direct from the proof of Theorem \ref{theorem_principal_convergence_unbounded}. But \eqref{implication2} is clearly true since
	\begin{align*}
		\frac{L(h_{\hat{\mathcal{M}}}^{\star}) - L(h^{\star})}{L(h_{\hat{\mathcal{M}}}^{\star})} > \frac{\epsilon}{L(\mathcal{M}^{\star})} \implies L(h_{\hat{\mathcal{M}}}^{\star}) - L(h^{\star}) > \epsilon \frac{L(h_{\hat{\mathcal{M}}}^{\star})}{L(\mathcal{M}^{\star})} \geq \epsilon.
	\end{align*}		
\end{proof}

\subsection{Results of Section \ref{SecReuse}}

\begin{proof}[Proof of Theorem \ref{bound_constant_reusing}]
	The bound for type I estimation error follows from the inequality
	\begin{align*}
		&\mathbb{P}\left(\sup\limits_{h \in \hat{\mathcal{M}}} \left|L_{\mathcal{D}_{N}}(h) - L(h)\right| > \epsilon\right) \\
		& = \mathbb{P}\left(\sup\limits_{h \in \hat{\mathcal{M}}} \left|L_{\mathcal{D}_{N}}(h) - L(h)\right| > \epsilon,\hat{\mathcal{M}} = \mathcal{M}^{\star}\right) + \mathbb{P}\left(\sup\limits_{h \in \hat{\mathcal{M}}} \left|L_{\mathcal{D}_{N}}(h) - L(h)\right| > \epsilon,\hat{\mathcal{M}} \neq \mathcal{M}^{\star}\right)\\
		&\leq \mathbb{P}\left(\sup\limits_{h \in \mathcal{M}^{\star}} \left|L_{\mathcal{D}_{N}}(h) - L(h)\right| > \epsilon\right) + \mathbb{P}\left(\hat{\mathcal{M}} \neq \mathcal{M}^{\star}\right),
	\end{align*}
	by noting that $B_{N,\epsilon}^{I}(d_{VC}(\mathcal{M}^{\star}))$ is a bound for the first probability. With a similar argument, we have the bound for type II estimation error.
\end{proof}

\FloatBarrier

\appendix
	\section{Vapnik-Chervonenkis theory}
	\label{apVCtheory}
	
	In this appendix, we present the main ideas and results of classical Vapnik-Chervonenkis (VC) theory, the stone upon which the results in this paper are built. The presentation of the theory is a simplified merge of \cite{vapnik1998}, \cite{vapnik2000}, \cite{devroye1996} and \cite{cortes2019}, where the simplicity of the arguments is preferred over the refinement of the bounds. Hence, we present results which support those in this paper and outline the main ideas of VC theory, even though are not the tightest available bounds. We omit the proofs, and note that refined versions of the results presented here may be found at one or more of the references.
	
	This appendix is a review of VC theory, except for novel results presented in Section \ref{ApUnbounded} for the case of unbounded loss functions, where we obtain new bounds for relative type I estimation error by extending the results in \cite{cortes2019}. We start defining the shatter coefficient and VC dimension of a hypotheses space under loss function $\ell$.
	
	\begin{definition}[Shatter coefficient]
		\label{shatter} 
		Let $\mathcal{G} = \{I: \mathcal{Z} \mapsto  \{0,1\}\}$ be a set of binary functions with domain $\mathcal{Z}$. The $N$-shatter coefficient of $\mathcal{G}$ is defined as
		\begin{equation*}
			S(\mathcal{G},N) = \max\limits_{(z_{1},\dots,z_{N}) \in \mathcal{Z}^{N}} \text{\textbar}\big\{\big(I(z_{1}),\dots,I(z_{N})\big): I \in \mathcal{G}\big\}\text{\textbar},
		\end{equation*}
		for $N \in \mathbb{Z}_{+}$, in which $\text{\textbar}\cdot\text{\textbar}$ is the cardinality of a set.
	\end{definition}
	
	\begin{definition}[Vapnik-Chervonenkis dimension]
		\label{VCdimension} 
		Fixed a hypotheses space $\mathcal{H}$ and a loss function $\ell$, set
		\begin{align*}
			C = \sup\limits_{\substack{z \in \mathcal{Z} \\ h \in \mathcal{H}}} \ell(z,h),
		\end{align*}
		in which $C$ can be infinity. Consider, for each $h \in \mathcal{H}$ and $\beta \in (0,C)$, the binary function $I(z;h,\beta) = \mathds{1}\{\ell(z,h) \geq \beta\}$, for $z \in \mathcal{Z}$, and denote
		\begin{align*}
			\mathcal{G}_{\mathcal{H},\ell} = \Big\{I(\cdot;h,\beta): h \in \mathcal{H}, \beta \in (0,C)\Big\}.
		\end{align*}
		We define the shatter coefficient of $\mathcal{H}$ under loss function $\ell$ as
		\begin{equation*}
			S(\mathcal{H},\ell,N) \coloneqq S(\mathcal{G}_{\mathcal{H},\ell},N).
		\end{equation*}
		The Vapnik-Chervonenkis (VC) dimension of $\mathcal{H}$ under loss function $\ell$ is the greatest integer $k \geq 1$ such that $S(\mathcal{H},\ell,k) = 2^{k}$, and is denoted by $d_{VC}(\mathcal{H},\ell)$. If $S(\mathcal{H},\ell,k) = 2^{k}$, for all integer $k \geq 1$, we denote $d_{VC}(\mathcal{H},\ell) = \infty$.
	\end{definition} 
	
	\begin{remark}
		If there is no confusion about which loss function we are referring to, or when it is not of importance to our argument, we omit $\ell$ and denote the shatter coefficient and VC dimension simply by $S(\mathcal{H},N)$ and $d_{VC}(\mathcal{H})$. We note that if the hypotheses in $\mathcal{H}$ are binary valued functions and $\ell$ is the simple loss function $\ell((x,y),h) = \mathds{1}\{h(x) \neq y\}$, then $\mathcal{H} = \mathcal{G}_{\mathcal{H},\ell}$, and its $N$-th shatter coefficient is actually the maximum number of dichotomies that can be generated by the functions in $\mathcal{H}$ with $N$ points.
	\end{remark}
	
	\subsection{Generalized Glivenko-Cantelli Problems}
	
	The main results of VC theory are based on a generalization of the Glivenko-Cantelli Theorem, which can be stated as follows. Recall that $\mathcal{D}_{N} = \{Z_{1},\dots,Z_{N}\}$ is a sequence of independent random vectors with a same distribution $P(z) \coloneqq \mathbb{P}(Z \leq z)$, for $z \in \mathcal{Z} \subset \mathbb{R}^{d}$, defined in a probability space $(\Omega,\mathcal{S},\mathbb{P})$. 
	
	In order to ease notation, we assume, without loss of generality, that $\Omega = \mathbb{R}^{d}$, $\mathcal{S}$ is the Borel $\sigma$-algebra of $\mathbb{R}^{d}$, the random vector $Z$ is the identity $Z(\omega) = \omega$, for $\omega \in \Omega$, and $\mathbb{P}$ is the unique probability measure such that $\mathbb{P}(\{\omega:\omega \leq z\}) = P(z)$, for all $z \in \mathbb{R}^{d}$. Define
	\begin{align*}
		P_{\mathcal{D}_{N}}(z) \coloneqq \frac{1}{N} \sum_{i=1}^{N} \mathds{1}\{Z_{i} \leq z\}, & & z \in \mathcal{Z}
	\end{align*}
	as the empirical distribution of $Z$ under sample $\mathcal{D}_{N}$.
	
	The assertion of the theorem below is that of \cite[Theorem~12.4]{devroye1996}. Its bottom line is that the empirical distribution of random variables converges uniformly to $P$ with probability one.
	
	\begin{theorem}[Glivenko-Cantelli Theorem]
		\label{glivenko_cantelli}
		Assume $d = 1$ and $\mathcal{Z} = \mathbb{R}$. Then, for a fixed $\epsilon > 0$ and $N$ great enough,
		\begin{equation}
			\label{GCbound}
			\mathbb{P}\left(\sup\limits_{z \in \mathbb{R}} \text{\textbar}P(z) - P_{\mathcal{D}_{N}}(z)\text{\textbar} > \epsilon\right) \leq 8(N+1) \exp\left\{-N \frac{\epsilon^2}{32}\right\}.
		\end{equation}
		Applying Borel-Cantelli Lemma \cite[Theorem~4.3]{billingsley2008} to \eqref{GCbound} yields
		\begin{equation*}
			\lim\limits_{N \to \infty} \sup\limits_{z \in \mathbb{R}} \text{\textbar}P(z) - P_{\mathcal{D}_{N}}(z)\text{\textbar} = 0 \text{ with probability one.}
		\end{equation*}
		In other words, $P_{\mathcal{D}_{N}}$ converges uniformly almost surely to $P$.
	\end{theorem}
	
	Theorem \ref{glivenko_cantelli} has the flavor of VC theory results: a rate of uniform convergence of the empirical probability of a class of events to their real probability, which implies the almost sure convergence. Indeed, letting $\mathcal{S}^{\star} \subset \mathcal{S}$ be a class of events, that is not necessarily a $\sigma$-algebra, and denoting
	\begin{equation*}
		\mathbb{P}_{\mathcal{D}_{N}}(A) = \frac{1}{N} \sum_{i=1}^{N} \mathds{1}\{Z_{i} \in A\},
	\end{equation*}
	as the empirical probability of event $A \in \mathcal{S}$ under sample $\mathcal{D}_{N}$, the probability in \eqref{GCbound} can be rewritten as
	\begin{equation}
		\label{partial_convergence}
		\mathbb{P}\left(\sup\limits_{A \in \mathcal{S}^{\star}} \text{\textbar}\mathbb{P}(A) - \mathbb{P}_{\mathcal{D}_{N}}(A)\text{\textbar} > \epsilon\right),
	\end{equation}
	in which $\mathcal{S}^{\star} = \{A_{z}:z \in \mathbb{R}\}$ with $A_{z} = \{\omega \in \Omega: \omega \leq z\}$. If probability \eqref{partial_convergence} converges to zero when $N$ tends to infinity for a class $\mathcal{S}^{\star} \subsetneq \mathcal{S}$, we say there exists a \textit{partial uniform convergence} of the empirical measure to $\mathbb{P}$.
	
	Observe that in \eqref{partial_convergence} not only the class $\mathcal{S}^{\star}$ is fixed, but also the probability measure $\mathbb{P}$, hence partial uniform convergence is dependent on the class and the probability. Nevertheless, in a distribution-free framework, such as that of learning (cf. Section \ref{SecPreliminaries}), the convergences of interest should hold for any data generating distribution, which is the case, for example, of Glivenko-Cantelli Theorem, that presents a rate of convergence \eqref{GCbound} which does not depend on $P$, holding for any probability measure and random variable $Z$. Therefore, once a class $\mathcal{S}^{\star}$ of interest is fixed, partial uniform convergence should hold for any data generating distribution, a problem which can be stated as follows.
	
	Let $\mathcal{P}$ be the class of all possible probability distributions of a random variable with support in $\mathcal{Z}$, and let $\mathcal{S}^{\star}$ be a class of events. The \textit{generalized Glivenko-Cantelli problem} (GGCP) is to find a positive constant $a$ and a function $b: \mathbb{Z}_{+} \mapsto \mathbb{R}_{+}$, such that $\lim\limits_{N \to \infty} b(N)/\exp cN = 0, \forall c > 0$, satisfying, for $N$ great enough,\footnote{In the presentation of \cite[Chapter~2]{vapnik1998} it is assumed that $b$ is a positive constant, not depending on sample size $N$. Nevertheless, having $b$ as a function of $N$ of an order lesser than exponential does not change the qualitative behavior of this convergence, that is, also guarantees the almost sure converge due to Borel-Cantelli Lemma.}
	\begin{equation}
		\label{Gen_GCbound}
		\sup\limits_{P \in \mathcal{P}} \mathbb{P}\left(\sup\limits_{A \in \mathcal{S}^{\star}} \text{\textbar}\mathbb{P}(A) - \mathbb{P}_{\mathcal{D}_{N}}(A)\text{\textbar} > \epsilon\right) \leq b(N) \exp\{-a\epsilon^{2}N\},
	\end{equation}
	in which $\mathbb{P}$ is to be understood as dependent on $P$, since it is the unique probability measure on the Borel $\sigma$-algebra that equals $P$ on the events $\{\omega \in \Omega:\omega \leq z\}, z \in \mathbb{R}^{d}$. If the events are of the form $A = \{w \in \Omega: Z(w) \leq z\}, z \in \mathbb{R}$, then \eqref{Gen_GCbound} is equivalent to \eqref{GCbound}, although in the latter it is implicit that it holds for any distribution $P$.
	
	The investigation of GGCP revolves around deducing necessary and sufficient conditions on the class $\mathcal{S}^{\star}$ for \eqref{Gen_GCbound} to hold. We will study these conditions in order to establish the almost sure convergence to zero of type I estimation error (cf. \eqref{GE1}) when the loss function is binary, what may be stated as a GGCP.
	
	\subsection{Convergence to zero of type I estimation error}
	\label{ApTypeI}
	
	\subsubsection{Binary loss functions}
	
	Fix a hypotheses space $\mathcal{H}$, a binary loss function $\ell$, and consider the class $\mathcal{S}^{\star} = \{A_{h}: h \in \mathcal{H}\}$, such that $\mathds{1}\{z \in A_{h}\} = \ell(z,h) \in \{0,1\}, z \in \mathcal{Z}, h \in \mathcal{H}$, that is, if $z \in A_{h}$ the loss is one, and otherwise it is zero. For example, if $Z = (X,Y)$, the hypotheses in $\mathcal{H}$ are functions from the range of $X$ to that of $Y$, and $\ell$ is the simple loss function, then $A_{h}$ may be explicitly written as
	\begin{equation*}
		A_{h} = \{\omega: h(X(\omega)) \neq Y(\omega)\}.
	\end{equation*}
	In this instance, the probability in the left-hand side of \eqref{Gen_GCbound} may be written as
	\begin{equation}
		\label{typeIexp}
		\mathbb{P}\left(\sup\limits_{h \in \mathcal{H}} \text{\textbar}\mathbb{E}(\ell(Z,h)) - \mathbb{E}_{\mathcal{D}_{N}}(\ell(Z,h))\text{\textbar} > \epsilon\right),
	\end{equation}
	in which $\mathbb{E}$ is expectation with respect to $\mathbb{P}$ and $\mathbb{E}_{\mathcal{D}_{N}}$ is the empirical mean under $\mathcal{D}_{N}$. With the notation of Section \ref{SecPreliminaries}, this last probability equals
	\begin{equation*}
		\label{typeIappendix}
		\mathbb{P}\left(\sup\limits_{h \in \mathcal{H}} \text{\textbar}L(h) - L_{\mathcal{D}_{N}}(h)\text{\textbar} > \epsilon\right),
	\end{equation*}
	the tail probability of type I estimation error in $\mathcal{H}$.
	
	For each fixed $h \in \mathcal{H}$, we are comparing in \eqref{typeIexp} the mean of a binary function with its empirical mean, so we may apply Hoeffding's inequality \cite{hoeffding1963} to obtain 
	\begin{equation*}
		\mathbb{P}\left(\text{\textbar}\mathbb{E}(\ell(Z,h)) - \mathbb{E}_{\mathcal{D}_{N}}(\ell(Z,N))\text{\textbar} > \epsilon\right) \leq 2 \exp\{-2\epsilon^{2}N\},
	\end{equation*}
	from which follows a solution of type I estimation error GGCP when the cardinality of $\mathcal{H}$ is finite, by applying an elementary union bound:
	\begin{align*}
		\mathbb{P}\Bigg(\sup\limits_{h \in \mathcal{H}} \text{\textbar}\mathbb{E}(\ell(Z,h)) -\mathbb{E}_{\mathcal{D}_{N}}(\ell(Z,N))\text{\textbar} > \epsilon\Bigg) &\leq \sum_{h \in \mathcal{H}} \mathbb{P}\left(\text{\textbar}\mathbb{E}(\ell(Z,h)) - \mathbb{E}_{\mathcal{D}_{N}}(\ell(Z,N))\text{\textbar} > \epsilon\right) \\
		&\leq 2 \text{\textbar}\mathcal{H}\text{\textbar} \exp\{-2\epsilon^{2}N\},
	\end{align*}
	what establishes the almost sure convergence to zero of type I estimation error when $\mathcal{H}$ is finite and $\ell$ is binary.
	
	In order to treat the case when $\mathcal{H}$ has infinitely many hypotheses, we rely on a modification of Glivenko-Cantelli Theorem, which depends on the shatter coefficient of a class $\mathcal{S}^{\star} \subset \mathcal{S}$ of events in the Borel $\sigma$-algebra of $\mathbb{R}^{d}$, defined below.
	
	\begin{definition}
		\label{shatter_borel}
		Fix $\mathcal{S}^{\star} \subset \mathcal{S}$ and let
		\begin{equation*}
			\mathcal{G}_{\mathcal{S}^\star} = \{h_{A}(z) = \mathds{1}\{z \in A\}: A \in \mathcal{S}^\star\}
		\end{equation*}
		be the characteristic functions of the sets in $\mathcal{S}^{\star}$. We define the shatter coefficient of $\mathcal{S}^{\star}$ as
		\begin{equation*}
			S(\mathcal{S}^{\star},N) \coloneqq S(\mathcal{G}_{\mathcal{S}^{\star}},N),
		\end{equation*}
		in which $S(\mathcal{G}_{\mathcal{S}^{\star}},N)$ is the shatter coefficient of $\mathcal{G}_{\mathcal{S}^{\star}}$ (cf. Definition \ref{shatter}). From this definition, it follows that
		\begin{equation*}
			d_{VC}(\mathcal{S}^{\star}) = d_{VC}(\mathcal{G}_{\mathcal{S}^{\star}}).
		\end{equation*}
	\end{definition}
	
	The shatter coefficient and VC dimension of a class $\mathcal{S}^{\star}$ are related to the dichotomies this class can build with $N$ points by considering whether a point is in each set or not. From a simple modification of the proof of Theorem \ref{glivenko_cantelli} presented in \cite[Theorem~12.4]{devroye1996} follows a result due to \cite{vapnik1971uniform}.
	
	\begin{theorem}
		\label{theorem_GGCP}
		For any probability measure $\mathbb{P}$ and class of sets $\mathcal{S}^{\star} \subset \mathcal{S}$, for fixed $N \in \mathbb{Z}$ and $\epsilon > 0$, it is true that
		\begin{equation*}
			\mathbb{P}\left(\sup\limits_{A \in \mathcal{S}^{\star}} \text{\textbar}\mathbb{P}(A) - \mathbb{P}_{\mathcal{D}_{N}}(A)\text{\textbar} > \epsilon\right)  \leq 8 S(\mathcal{S}^{\star},N) \exp\left\{-N \frac{\epsilon^2}{32}\right\}.
		\end{equation*}
	\end{theorem}
	
	From this theorem follows a bound for tail probabilities of type I estimation error when $\ell$ is binary.
	
	\begin{corollary}
		\label{cor1TypeI}
		Fix a hypotheses space $\mathcal{H}$ and a loss function $\ell: \mathcal{Z} \times \mathcal{H} \mapsto \{0,1\}$. Let $\mathcal{S}^{\star} = \{A_{h}: h \in \mathcal{H}\}$, with
		\begin{equation*}
			\mathds{1}\{z \in A_{h}\} = \ell(z,h), z \in \mathcal{Z}, h \in \mathcal{H}.
		\end{equation*}
		Then,
		\begin{equation}
			\label{Ap1Eq4}
			\mathbb{P}\left(\sup\limits_{h \in \mathcal{H}} \text{\textbar}L(h) - L_{\mathcal{D}_{N}}(h)\text{\textbar} > \epsilon\right) \leq 8 \ S(\mathcal{H},N) \exp\left\{-N \frac{\epsilon^2}{32}\right\},
		\end{equation}
		with
		\begin{equation*}
			S(\mathcal{H},N) \coloneqq S(\mathcal{S}^{\star},N).
		\end{equation*}
	\end{corollary}
	
	\begin{remark}
		We remark that $S(\mathcal{S}^{\star},N) = S(\mathcal{G}_{\mathcal{H},\ell},N)$, as defined in Definition \ref{VCdimension} when the loss $\ell$ is binary. Observe that $\mathcal{S}^{\star}$ depends on $\ell$, although we omit the dependence to ease notation.
	\end{remark}
	
	The calculation of the quantities on the right-hand side of \eqref{Ap1Eq4} is not straightforward, since the shatter coefficient is not easily determined for arbitrary $N$. Nevertheless, the shatter coefficient may be bounded by a quantity depending on the VC dimension of $\mathcal{H}$. This is the content of \cite[Theorem~4.3]{vapnik1998}.
	
	\begin{theorem}
		\label{theorem_shaterDVC}
		If $d_{VC}(\mathcal{H}) < \infty$, then
		\begin{equation*}
			\ln S(\mathcal{H},N) \begin{cases}
				= N \ln 2, & \text{ if } N \leq d_{VC}(\mathcal{H})\\
				\leq d_{VC}(\mathcal{H}) \left(1 + \ln \frac{N}{d_{VC}(\mathcal{H})}\right), & \text{ if } N > d_{VC}(\mathcal{H}) 
			\end{cases}.
		\end{equation*}
	\end{theorem}
	
	\begin{remark}
		Theorem \ref{theorem_shaterDVC} is true for any loss function $\ell$, not only binary.
	\end{remark}
	
	Combining this theorem with Corollary \ref{cor1TypeI}, we obtain the following result.
	
	\begin{corollary}
		\label{cor2TypeI}
		Under the hypotheses of Corollary \ref{cor1TypeI} it holds
		\begin{equation}
			\label{Ap1Eq5}
			\mathbb{P}\left(\sup\limits_{h \in \mathcal{H}} \text{\textbar}L(h) - L_{\mathcal{D}_{N}}(h)\text{\textbar} > \epsilon\right) \leq 8 \ \exp\left\{d_{VC}(\mathcal{H}) \left(1 + \ln \frac{N}{d_{VC}(\mathcal{H})}\right) - N \frac{\epsilon^2}{32}\right\}.
		\end{equation}
		In particular, if $d_{VC}(\mathcal{H}) < \infty$, not only \eqref{Ap1Eq5} converges to zero, but also
		\begin{equation*}
			\sup\limits_{h \in \mathcal{H}} \text{\textbar}L(h) - L_{\mathcal{D}_{N}}(h)\text{\textbar} \xrightarrow[N \to \infty]{} 0,
		\end{equation*}
		with probability one by Borel-Cantelli Lemma.
	\end{corollary}
	
	From Corollary \ref{cor2TypeI} follows the convergence to zero of type I estimation error when the loss function is binary and $d_{VC}(\mathcal{H})$ is finite. We now extend this result to real-valued bounded loss functions.
	
	\subsubsection{Bounded loss functions}
	
	Assume the loss function is bounded, that is, for all $z \in \mathcal{Z}$ and $h \in \mathcal{H}$,
	\begin{equation}	
		\label{bounded}
		0 \leq \ell(z,h) \leq C < \infty,
	\end{equation}
	for a positive constant $C \in \mathbb{R}_{+}$. Throughout this section, a constant $C$ satisfying \eqref{bounded} is fixed.
	
	For any $h \in \mathcal{H}$, by definition of Lebesgue-Stieltjes integral, we have that
	\begin{equation*}
		L(h) = \int_{\mathcal{Z}} \ell(z,h) \ dP(z) = \lim\limits_{n \to \infty} \sum_{k=1}^{n-1} \frac{C}{n} \mathbb{P}\left(\ell(Z,h) > \frac{kC}{n}\right),
	\end{equation*}
	recalling that $Z$ is a random variable with distribution $P$. In the same manner, we may also write the empirical risk under $\mathcal{D}_{N}$ as
	\begin{equation*}
		L_{\mathcal{D}_{N}}(h) = \frac{1}{N} \sum_{i=1}^{N} \ell(Z_{i},h) = \lim\limits_{n \to \infty} \sum_{k=1}^{n-1} \frac{C}{n} \mathbb{P}_{\mathcal{D}_{N}}\left(\ell(Z,h) > \frac{kC}{n}\right),
	\end{equation*}
	recalling that $\mathbb{P}_{\mathcal{D}_{N}}$ is the empirical measure according to $\mathcal{D}_{N}$.
	
	From the representation of $L$ and $L_{\mathcal{D}_{N}}$ described above, we have that, for each $h \in \mathcal{H}$ fixed,
	\begin{align*}
		\text{\textbar}L(h) - L_{\mathcal{D}_{N}}(h)\text{\textbar} &= \text{\textbar}\lim\limits_{n \to \infty} \sum_{k=1}^{n-1} \frac{C}{n} \left(\mathbb{P}\left(\ell(Z,h) > \frac{kC}{n}\right) - \mathbb{P}_{\mathcal{D}_{N}}\left(\ell(Z,h) > \frac{kC}{n}\right)\right) \text{\textbar}\\
		&\leq \text{\textbar}\lim\limits_{n \to \infty} \sum_{k=1}^{n-1} \frac{C}{n} \sup\limits_{0 \leq \beta \leq C} \left(\mathbb{P}\left(\ell(Z,h) > \beta\right) - \mathbb{P}_{\mathcal{D}_{N}}\left(\ell(Z,h) > \beta\right)\right) \text{\textbar}\\
		&\leq \lim\limits_{n \to \infty} \sum_{k=1}^{n-1} \frac{C}{n} \sup\limits_{0 \leq \beta \leq C} \text{\textbar}\mathbb{P}\left(\ell(Z,h) > \beta\right) - \mathbb{P}_{\mathcal{D}_{N}}\left(\ell(Z,h) > \beta\right)\text{\textbar}\\
		&= C \sup\limits_{0 \leq \beta \leq C} \text{\textbar}\int_{\mathcal{Z}} \mathds{1}\{\ell(z,h) > \beta\} \ dP(z) - \frac{1}{N} \sum_{i=1}^{N} \ \mathds{1}\{\ell(Z_{i},h) > \beta\}\text{\textbar}.
	\end{align*}
	We conclude that
	\begin{align*}
		&\mathbb{P}\left(\sup\limits_{h \in \mathcal{H}} \text{\textbar}L(h) - L_{\mathcal{D}_{N}(h)}\text{\textbar} > \epsilon\right) \\
		&\leq \mathbb{P}\left(\sup\limits_{\substack{h \in \mathcal{H}\\ 0 \leq \beta \leq C}} \text{\textbar}\int_{\mathcal{Z}} \mathds{1}\{\ell(z,h) > \beta\} \ dP(z) - \frac{1}{N} \sum_{i=1}^{N} \ \mathds{1}\{\ell(Z_{i},h)\}\text{\textbar} > \frac{\epsilon}{C}\right).
	\end{align*}
	
	Since the right-hand side of the expression above is a GGCP with
	\begin{equation*}
		\mathcal{S}^{\star} = \{\{z \in \mathcal{Z}:\ell(z,h) > \beta\}: h \in \mathcal{H},0 \leq \beta \leq C\}
	\end{equation*}
	and, recalling the definition of shatter coefficient of $\mathcal{H}$ under a real-valued loss function $\ell$ (cf. Definition \ref{shatter}), we note that
	\begin{align*}
		S(\mathcal{G}_{\mathcal{H},\ell},N)  = S(\mathcal{S}^{\star},N) & & \text{ hence } & & d_{VC}(\mathcal{H}) = d_{VC}(\mathcal{S}^{\star}) 
	\end{align*}
	so a bound for the tail probabilities of type I estimation error when the loss function is bounded follows immediately from Theorems \ref{theorem_GGCP} and \ref{theorem_shaterDVC}.
	
	\begin{corollary}
		\label{cor3TypeI}
		Fix a hypotheses space $\mathcal{H}$ and a loss function $\ell: \mathcal{Z} \times \mathcal{H} \mapsto \mathbb{R}_{+}$, with $0 \leq \ell(z,h) \leq C$ for all $z \in \mathcal{Z}, h \in \mathcal{H}$. Then,
		\begin{equation}
			\label{Ap1Eq6}
			\mathbb{P}\left(\sup\limits_{h \in \mathcal{H}} \text{\textbar}L(h) - L_{\mathcal{D}_{N}}(h)\text{\textbar} > \epsilon\right) \leq 8 \ \exp\left\{d_{VC}(\mathcal{H}) \left(1 + \ln \frac{N}{d_{VC}(\mathcal{H})}\right) - N \frac{\epsilon^2}{32C^{2}}\right\}.
		\end{equation}
		In particular, if $d_{VC}(\mathcal{H}) < \infty$, not only \eqref{Ap1Eq6} converges to zero, but also
		\begin{equation*}
			\sup\limits_{h \in \mathcal{H}} \text{\textbar}L(h) - L_{\mathcal{D}_{N}}(h)\text{\textbar} \xrightarrow[N \to \infty]{} 0,
		\end{equation*}
		with probability one by Borel-Cantelli Lemma.
	\end{corollary}
	
	It remains to treat the case of unbounded loss functions, which requires a different approach.
	
	\subsubsection{Unbounded loss functions}
	\label{ApUnbounded}
	
	In this section, we establish conditions on $P$ and $\mathcal{H}$ for the convergence in probability to zero of type I relative estimation error when $\ell$ is unbounded. The framework treated here is that described at the beginning of Section \ref{SecUnbounded}.
	
	In order to ease notation, we denote
	\begin{equation*}
		\ell(\mathcal{D}_{N},h) \coloneqq \left(\ell(Z_{1},h),\dots,\ell(Z_{N},h)\right) \in \mathbb{R}^{N}\setminus\{0\},
	\end{equation*}
	the vector sample point loss, so it follows from \eqref{LNp} that, for $1 \leq q \leq p$,
	\begin{equation}
		\label{def_sample_pmoment}
		L_{\mathcal{D}_{N}}^{q}(h) \coloneqq \frac{\lVert \ell(\mathcal{D}_{N},h) \rVert_{q}}{N^{\frac{1}{q}}},
	\end{equation}
	in which $\lVert \cdot \rVert_{q}$ is the $q$-norm in $\mathbb{R}^{N}$. Recall that we assume that $P$ has at most heavy tails, which means there exists a $p > 1$, that can be lesser than 2, with
	\begin{align}
		\label{tauStarAp}
		\tau_{p} < \tau^{\star} < \infty,
	\end{align}
	and that
	\begin{align}
		\label{finite_moments}
		\sup\limits_{h \in \mathcal{H}} L_{\mathcal{D}_{N}}^{p}(h)  < \infty & & \text{ and } & & \sup\limits_{h \in \mathcal{H}} L^{p}(h) < \infty,
	\end{align}
	in which the first inequality should hold with probability one. 
	
	The first condition in \eqref{finite_moments} is more a feature of the loss function, than of distribution $P$. Actually, one can bound $L_{\mathcal{D}_{N}}^{p}(h)$ by a quantity depending on $N$ and $L_{\mathcal{D}_{N}}^{q}(h)$ with $1 \leq q < p$, for any sample $\mathcal{D}_{N}$ of any distribution $P$. This is the content of the next lemma, which will be useful later on, and that implies the following: if $\sup_{h \in \mathcal{H}} L_{\mathcal{D}_{N}}^{1}(h) = \sup_{h \in \mathcal{H}} L_{\mathcal{D}_{N}}(h) < \infty$, then $\sup_{h \in \mathcal{H}} L_{\mathcal{D}_{N}}^{p}(h) < \infty$ for any $1 < p < \infty$, for $N$ and $\mathcal{H}$ fixed. 
	
	\begin{lemma}
		\label{lemma_norm}
		For fixed $\mathcal{H}$, $N \geq 1$ and $1 \leq q < p$, it follows that
		\begin{equation*}
			1 \leq \frac{L_{\mathcal{D}_{N}}^{p}(h)}{L_{\mathcal{D}_{N}}^{q}(h)} \leq N^{\frac{1}{q} - \frac{1}{p}}
		\end{equation*}
		for all $h \in \mathcal{H}$.
	\end{lemma}
	\begin{proof}
		Recalling definition \eqref{def_sample_pmoment}, we have that
		\begin{equation*}
			\frac{L_{\mathcal{D}_{N}}^{p}(h)}{L_{\mathcal{D}_{N}}^{q}(h)} = N^{\frac{1}{q} - \frac{1}{p}} \frac{\lVert \ell(\mathcal{D}_{N},h) \rVert_{p}}{\lVert \ell(\mathcal{D}_{N},h) \rVert_{q}},
		\end{equation*}
		so it is enough to show that
		\begin{equation*}
			N^{\frac{1}{p} - \frac{1}{q}} \leq \frac{\lVert \ell(\mathcal{D}_{N},h) \rVert_{p}}{\lVert \ell(\mathcal{D}_{N},h) \rVert_{q}} \leq 1.
		\end{equation*}
		
		Now, the right inequality above is clear, since if $w \in \mathbb{R}^{N}$ is such that $\lVert w \rVert_{q} = 1$, then
		\begin{align*}
			\lVert w \rVert_{p}^{p} = \sum_{i=1}^{N} \text{\textbar}w_{i}\text{\textbar}^{p} \leq \sum_{i=1}^{N} \text{\textbar}w_{i}\text{\textbar}^{q} = 1,
		\end{align*}
		so the result follows when $\lVert w \rVert_{q} = 1$ by elevating both sided to the $1/p$ power. To conclude the proof it is enough to see that, for any $w \in \mathbb{R}^{N}\setminus\{0\}$,
		\begin{equation*}
			\lVert w \rVert_{p} = \lVert w \rVert_{q} \lVert \frac{w}{\lVert w \rVert_{q}} \rVert_{p} \leq \lVert w \rVert_{q} \lVert \frac{w}{\lVert w \rVert_{q}} \rVert_{q} = \lVert w \rVert_{q}.
		\end{equation*}
		
		The left inequality is a consequence of H\"older's inequality, since, for $w \in \mathbb{R}^{N}$,
		\begin{align*}
			\sum_{i=1}^{N} \text{\textbar}w_{i}\text{\textbar}^{q} \cdot 1 \leq \left(\sum_{i=1}^{N} \text{\textbar}w_{i}\text{\textbar}^{p}\right)^{\frac{q}{p}} N^{1 - \frac{q}{p}},
		\end{align*}
		and the result follows by taking the $1/q$ power on both sides.
	\end{proof}
	
	For unbounded losses, rather than considering the convergence of type I estimation error to zero, we will consider the convergence of the relative type I estimation error, defined as
	\begin{equation}	
		\label{Ap1Eq9}
		\textbf{(I)} \ \ \sup\limits_{h \in \mathcal{H}} \frac{\text{\textbar}L(h) - L_{\mathcal{D}_{N}}(h)\text{\textbar}}{L(h)} .
	\end{equation}
	In order to establish bounds for the tail probabilities of \eqref{Ap1Eq9} when \eqref{tauStarAp} and \eqref{finite_moments_text} hold, we rely on the following novel technical theorem.
	
	\begin{theorem}
		\label{change_denominator}
		Let $q = \sqrt{p}$. For any hypotheses space $\mathcal{H}$, loss function $\ell$ satisfying $\ell(h,z) \geq 1$, and $0 < \epsilon < 1$, it holds
		\begin{align*}
			&\mathbb{P}\left(\sup\limits_{h \in \mathcal{H}} \frac{\text{\textbar}L(h) - L_{\mathcal{D}_{N}}(h)\text{\textbar}}{L(h)}  > \tau^{\star} \epsilon\right) \leq \mathbb{P}\left(\sup\limits_{h \in \mathcal{H}} \frac{L(h) - L_{\mathcal{D}_{N}}(h)}{L^{p}(h)} > \epsilon\right)\\
			&+ \mathbb{P}\left(\sup\limits_{h \in \mathcal{H}} \frac{L_{\mathcal{D}_{N}}^{\prime}(h) - L^{\prime}(h)}{L_{\mathcal{D}_{N}}^{\prime q}(h)} > \frac{\epsilon}{N^{\frac{1}{q} - \frac{1}{p}}}\right) + \mathbb{P}\left(\sup\limits_{h \in \mathcal{H}} \frac{L_{\mathcal{D}_{N}}(h) - L(h)}{L_{\mathcal{D}_{N}}^{p}(h)} > \frac{\epsilon(1-\epsilon)}{N^{\frac{1}{q} - \frac{1}{p}}}\right)
		\end{align*}
		in which $L^{\prime},L_{\mathcal{D}_{N}}^{\prime}$ and $L_{\mathcal{D}_{N}}^{\prime k}$ are the respective risks and $k$ moments of loss function $\ell^{\prime}(z,h) \coloneqq (\ell(z,h))^{q}$.
	\end{theorem}
	\begin{proof}
		We first note that
		\begin{align*}
			&\sup\limits_{h \in \mathcal{H}} \frac{L_{\mathcal{D}_{N}}(h) - L(h)}{L(h)} > \tau^{\star}\epsilon \implies \sup\limits_{h \in \mathcal{H}} \left(\frac{L^{q}(h)}{L(h)} \frac{1}{\tau^{\star}}\right) \frac{L_{\mathcal{D}_{N}}(h) - L(h)}{L^{q}(h)} > \epsilon\\
			&\implies \sup\limits_{h \in \mathcal{H}} \frac{L_{\mathcal{D}_{N}}(h) - L(h)}{L^{q}(h)} > \epsilon
		\end{align*}
		since
		\begin{align*}
			\sup\limits_{h \in \mathcal{H}} \left(\frac{L^{q}(h)}{L(h)} \frac{1}{\tau^{\star}}\right) \leq \sup\limits_{h \in \mathcal{H}} \left(\frac{L^{p}(h)}{L(h)} \frac{1}{\tau^{\star}}\right) \leq 1
		\end{align*}
		by \eqref{tauStarAp}. With an analogous deduction, it follows that
		\begin{equation*}
			\sup\limits_{h \in \mathcal{H}} \frac{L(h) - L_{\mathcal{D}_{N}}(h)}{L(h)} > \tau^{\star}\epsilon \implies \sup\limits_{h \in \mathcal{H}} \frac{L(h) - L_{\mathcal{D}_{N}}(h)}{L^{p}(h)} > \epsilon.
		\end{equation*}
		Hence, the probability on the left hand-side of the statement is lesser or equal to
		\begin{align}
			\label{Ap1Eq10}
			\mathbb{P}\left(\sup\limits_{h \in \mathcal{H}} \frac{L(h) - L_{\mathcal{D}_{N}}(h)}{L^{p}(h)} > \epsilon\right) + \mathbb{P}\left(\sup\limits_{h \in \mathcal{H}} \frac{L_{\mathcal{D}_{N}}(h) - L(h)}{L^{q}(h)} > \epsilon\right),
		\end{align}
		so it is enough to properly bound the second probability in \eqref{Ap1Eq10}. 	
		
		In order to do so, we will intersect the event inside the probability with the following event, and its complement:
		\begin{align*}
			\sup\limits_{h \in \mathcal{H}} \frac{L_{\mathcal{D}_{N}}^{q}(h) - \delta L_{\mathcal{D}_{N}}^{p}(h)}{L^{q}(h)} \leq 1 \iff \sup\limits_{h \in \mathcal{H}} \frac{L_{\mathcal{D}_{N}}^{q}(h) - L^{q}(h)}{L_{\mathcal{D}_{N}}^{p}(h)} \leq \delta,
		\end{align*}
		in which
		\begin{equation*}
			\delta \coloneqq \frac{\epsilon}{N^{\frac{1}{q} - \frac{1}{p}}}.
		\end{equation*}
		Proceeding in this way, we conclude that
		\begin{align}
			\label{Ap1Eq12} \nonumber
			&\mathbb{P}\left(\sup\limits_{h \in \mathcal{H}} \frac{L_{\mathcal{D}_{N}}(h) - L(h)}{L^{q}(h)} > \epsilon\right) \leq \mathbb{P}\left(\sup\limits_{h \in \mathcal{H}} \frac{L_{\mathcal{D}_{N}}^{q}(h) - L^{q}(h)}{L_{\mathcal{D}_{N}}^{p}(h)} > \delta\right)\\ \nonumber
			&+ \mathbb{P}\left(\sup\limits_{h \in \mathcal{H}} \left(\frac{L_{\mathcal{D}_{N}}^{q}(h) - \delta L_{\mathcal{D}_{N}}^{p}(h)}{L^{q}(h)}\right) \frac{L_{\mathcal{D}_{N}}(h) - L(h)}{L_{\mathcal{D}_{N}}^{q}(h) - \delta L_{\mathcal{D}_{N}}^{p}(h)} > \epsilon,\sup\limits_{h \in \mathcal{H}} \frac{L_{\mathcal{D}_{N}}^{q}(h) - \delta L_{\mathcal{D}_{N}}^{p}(h)}{L^{q}(h)} \leq 1\right)\\
			&\leq \mathbb{P}\left(\sup\limits_{h \in \mathcal{H}} \frac{L_{\mathcal{D}_{N}}^{q}(h) - L^{q}(h)}{L_{\mathcal{D}_{N}}^{p}(h)} > \delta\right) + \mathbb{P}\left(\sup\limits_{h \in \mathcal{H}} \frac{L_{\mathcal{D}_{N}}(h) - L(h)}{L_{\mathcal{D}_{N}}^{q}(h) - \delta L_{\mathcal{D}_{N}}^{p}(h)} > \epsilon\right).
		\end{align}
		To bound the first probability above, we recall the definition of $L_{\mathcal{D}_{N}}^{p}(h)$ and note that $a^{\frac{1}{q}} - b^{\frac{1}{q}} \leq a - b$ if $q > 1$ and $1 \leq b \leq a$, so that
		\begin{equation}
			\label{Ap1Eq11}
			\mathbb{P}\left(\sup\limits_{h \in \mathcal{H}} \frac{L_{\mathcal{D}_{N}}^{q}(h) - L^{q}(h)}{L_{\mathcal{D}_{N}}^{p}(h)} > \delta\right) \leq \mathbb{P}\left(\sup\limits_{h \in \mathcal{H}} \frac{\left(L_{\mathcal{D}_{N}}^{q}(h)\right)^{q} - \left(L^{q}(h)\right)^{q}}{N^{-\frac{1}{p}} \lVert \ell(\mathcal{D}_{N},h)\rVert_{p}} > \delta\right).
		\end{equation}
		Define the loss function $\ell^{\prime}(z,h) \coloneqq (\ell(z,h))^{q}$, and let $L^{\prime},L^{\prime k},L_{\mathcal{D}_{N}}^{\prime}$ and $L_{\mathcal{D}_{N}}^{\prime k}$ be the risks and $k$ moments according to this new loss function. Then, the probability in \eqref{Ap1Eq11} can be written as
		\begin{equation}
			\label{Ap1Eq13}
			\mathbb{P}\left(\sup\limits_{h \in \mathcal{H}} \frac{L_{\mathcal{D}_{N}}^{\prime}(h) - L^{\prime}(h)}{\left(N^{-1}\lVert \ell^{\prime}(\mathcal{D}_{N},h) \rVert_{\frac{p}{q}}^{\frac{p}{q}}\right)^{\frac{1}{p}}} > \delta\right) \leq \mathbb{P}\left(\sup\limits_{h \in \mathcal{H}} \frac{L_{\mathcal{D}_{N}}^{\prime}(h) - L^{\prime}(h)}{L_{\mathcal{D}_{N}}^{\prime q}(h)} > \delta\right)
		\end{equation}
		since $\frac{p}{q} = q$ and $N^{\frac{1}{p}} \leq N^{\frac{1}{q}}$.
		
		We turn to the second probability in \eqref{Ap1Eq12}. By applying Lemma \ref{lemma_norm}, and recalling the definition of $\delta$, we have that $L_{\mathcal{D}_{N}}^{q}(h) - \delta L_{\mathcal{D}_{N}}^{p}(h)$ is equal to
		\begin{align*}
			L_{\mathcal{D}_{N}}^{p}(h) \left(\frac{L_{\mathcal{D}_{N}}^{q}(h)}{L_{\mathcal{D}_{N}}^{p}(h)} - \delta\right) \geq L_{\mathcal{D}_{N}}^{p}(h) \left(\frac{1}{N^{\frac{1}{q} - \frac{1}{p}}} - \frac{\epsilon}{N^{\frac{1}{q} - \frac{1}{p}}}\right) = \frac{L_{\mathcal{D}_{N}}^{p}(h)}{N^{\frac{1}{q} - \frac{1}{p}}} \left(1 - \epsilon\right),
		\end{align*}
		from which follows
		\begin{equation}
			\label{Ap1Eq14}
			\mathbb{P}\left(\sup\limits_{h \in \mathcal{H}} \frac{L_{\mathcal{D}_{N}}(h) - L(h)}{L_{\mathcal{D}_{N}}^{q}(h) - \delta L_{\mathcal{D}_{N}}^{p}(h)} > \epsilon\right) \leq \mathbb{P}\left(\sup\limits_{h \in \mathcal{H}} \frac{L_{\mathcal{D}_{N}}(h) - L(h)}{L_{\mathcal{D}_{N}}^{p}(h)} > \frac{\epsilon(1-\epsilon)}{N^{\frac{1}{q} - \frac{1}{p}}}\right).
		\end{equation}
		The result follows by combining \eqref{Ap1Eq10}, \eqref{Ap1Eq12}, \eqref{Ap1Eq13} and \eqref{Ap1Eq14}.
	\end{proof}
	
	An exponential bound for relative type I estimation error \eqref{Ap1Eq9} depending on $p$, $\tau^{\star}$ and $d_{VC}(\mathcal{H})$ is a consequence of Theorems \ref{theorem_shaterDVC} and \ref{change_denominator}, and results in \cite{cortes2019}, which we now state. Define, for a $0 < \varsigma < 1$ fixed,
	\begin{align*}
		\Gamma(p,\epsilon) &= \frac{p-1}{p}(1+\varsigma)^{\frac{1}{p}} + \frac{1}{p} \left(\frac{p}{p-1}\right)^{p-1} \left(1 + \left(\frac{p-1}{p}\right)^{p} \varsigma^{\frac{1}{p}}\right)^{\frac{1}{p}} \left[1 + \frac{\ln(1/\epsilon)}{\left(\frac{p}{p-1}\right)^{p-1}}\right]^{\frac{p-1}{p}},		
	\end{align*}
	for $0 < \epsilon < 1, 1 < p \leq 2$, and
	\begin{align*}
		\Lambda(p) = \left(\frac{1}{2}\right)^{\frac{2}{p}} \left(\frac{p}{p-2}\right)^{\frac{p-1}{p}} + \frac{p}{p-1} \varsigma^{\frac{p-2}{2p}}
	\end{align*}
	for $p > 2$.
	
	\begin{theorem}
		\label{theorem_unbounded}
		Fix a hypotheses space $\mathcal{H}$ and an unbounded loss function $\ell$, and assume that \eqref{finite_moments} is in force. Then, the following holds:
		\begin{itemize}
			\item If $P$ has light tails, so that \eqref{tauStarAp} holds for a $p > 2$ fixed, then		
			\begin{align*}
				\mathbb{P}\Bigg(\sup\limits_{h \in \mathcal{H}} & \frac{L(h) - L_{\mathcal{D}_{N}}(h)}{\sqrt[p]{\left(L^{p}(h)\right)^{p} + \varsigma}} > \Lambda(p) \epsilon\Bigg)\\& < 4 \exp\left\{d_{VC}(\mathcal{H})\left(1 + \ln \frac{2N}{d_{VC}(\mathcal{H})}\right) - \frac{\epsilon^{2}N}{4}\right\}
			\end{align*}
			and
			\begin{align*}
				\mathbb{P}\Bigg(\sup\limits_{h \in \mathcal{H}} & \frac{L_{\mathcal{D}_{N}}(h) - L(h)}{\sqrt[p]{\left(L_{\mathcal{D}_{N}}^{p}(h)\right)^{p} + \varsigma}} > \Lambda(p) \epsilon\Bigg) \\&< 4 \exp\left\{d_{VC}(\mathcal{H})\left(1 + \ln \frac{2N}{d_{VC}(\mathcal{H})}\right) - \frac{\epsilon^{2}N}{4}\right\},
			\end{align*}
			for $0 < \epsilon < 1$ and $0 < \varsigma < \epsilon^{2}$.
			
			\item If $P$ has heavy tails, so that \eqref{tauStarAp} holds only for a $1 < p \leq 2$ fixed, then
			\begin{align*}
				\mathbb{P}\Bigg(\sup\limits_{h \in \mathcal{H}} & \frac{L(h) - L_{\mathcal{D}_{N}}(h)}{\sqrt[p]{\left(L^{p}(h)\right)^{p} + \varsigma}} > \Gamma(p,\epsilon) \epsilon\Bigg) \\&< 4 \exp\left\{d_{VC}(\mathcal{H})\left(1 + \ln \frac{2N}{d_{VC}(\mathcal{H})}\right) - \frac{\epsilon^{2}N^{\frac{2(p-1)}{p}}}{2^{\frac{p+2}{2}}}\right\}
			\end{align*}
			and
			\begin{align*}
				\mathbb{P}\Bigg(\sup\limits_{h \in \mathcal{H}} & \frac{L_{\mathcal{D}_{N}}(h) - L(h)}{\sqrt[p]{\left(L_{\mathcal{D}_{N}}^{p}(h)\right)^{p} + \varsigma}} > \Gamma(p,\epsilon) \epsilon\Bigg) \\&< 4 \exp\left\{d_{VC}(\mathcal{H})\left(1 + \ln \frac{2N}{d_{VC}(\mathcal{H})}\right) - \frac{\epsilon^{2}N^{\frac{2(p-1)}{p}}}{2^{\frac{p+2}{2}}}\right\},
			\end{align*}
			for $0 < \epsilon < 1$ and $0 < \varsigma^{\frac{p-1}{p}} < \epsilon^{\frac{p}{p-1}}$.
		\end{itemize}
	\end{theorem}
	
	Theorem \ref{theorem_unbounded}, together with Theorem \ref{change_denominator}, imply the following corollary, which is an exponential bound for relative type I estimation error when $P$ has heavy or light tails. The value of $\varsigma$ in the definitions of $\Lambda(p)$ and $\Gamma(p,\epsilon)$ below can be arbitrarily small.
	
	\begin{corollary}
		\label{convergence_relativeTI}
		Fix a hypotheses space $\mathcal{H}$, an unbounded loss function $\ell$ and $\epsilon > 0$. The following holds:
		\begin{itemize}
			\item If \eqref{tauStarAp} holds for a $p \geq 4$ fixed, then
			\begin{align*}
				\mathbb{P}&\Bigg(\sup\limits_{h \in \mathcal{H}} \frac{\text{\textbar}L(h) - L_{\mathcal{D}_{N}}(h)\text{\textbar}}{L(h)} > \tau^{\star} \Lambda(\sqrt{p}) \epsilon\Bigg) \\
				&< 12 \exp\left\{d_{VC}(\mathcal{H})\left(1 + \ln \frac{2N}{d_{VC}(\mathcal{H})}\right) - \frac{\epsilon^{2}(1-\epsilon)^{2}N^{1 - \frac{2}{\sqrt{p}} + \frac{2}{p}}}{4}\right\}
			\end{align*}
			
			\item If \eqref{tauStarAp} holds for a $1 < p < 4$ fixed, then
			\begin{align*}
				\mathbb{P}&\Bigg(\sup\limits_{h \in \mathcal{H}} \frac{\text{\textbar}L(h) - L_{\mathcal{D}_{N}}(h)\text{\textbar}}{L(h)} > \tau^{\star} \Gamma\left(\sqrt{p},\frac{\epsilon}{N^{\frac{1}{\sqrt{p}} - \frac{1}{p}}}\right)\epsilon\Bigg) \\
				&< 12 \exp\left\{d_{VC}(\mathcal{H})\left(1 + \ln \frac{2N}{d_{VC}(\mathcal{H})}\right) - \frac{\epsilon^{2}N^{\frac{2(\sqrt{p} - 1)}{\sqrt{p}} - \frac{2}{\sqrt{p}} + \frac{2}{p}}}{2^{\frac{\sqrt{p} + 2}{2}}} \right\}.
			\end{align*}
		\end{itemize}
		In both cases, if $d_{VC}(\mathcal{H}) < \infty$, then, by Borel-Cantelli Lemma,
		\begin{equation*}
			\sup\limits_{h \in \mathcal{H}} \frac{\text{\textbar}L(h) - L_{\mathcal{D}_{N}}(h)\text{\textbar}}{L(h)} \xrightarrow[N \to \infty]{} 0,
		\end{equation*}
		with probability one.
	\end{corollary}
	
	Corollary \ref{convergence_relativeTI} establishes the convergence to zero of relative type I estimation error, and concludes our study of type I estimation error convergence in classical VC theory.
	
	\begin{remark}
		We simplified the bounds in Corollary \ref{convergence_relativeTI} since, by combining Theorems \ref{change_denominator} with \ref{theorem_unbounded}, we obtain a bound with three terms of different orders in $N$, where the exponential in each of them is multiplied by four. The term of the greatest order is that we show in Corollary \ref{convergence_relativeTI}, with the exponential multiplied by twelve, since we can bound the two terms of lesser order by the one of the greatest order. This worsens the bound for fixed $N$, but eases notation and has the same qualitative effect of presenting an exponential bound for relative type I estimation error, which implies its almost sure convergence to zero.
	\end{remark}
	
	\begin{remark}
		Observe that $d_{VC}(\mathcal{M},\ell) = d_{VC}(\mathcal{M},\ell^{\prime})$, in which $\ell^{\prime}(z,h) = \left(\ell(z,h)\right)^{q}, q = \sqrt{p},$ as defined in Theorem \ref{change_denominator}, hence we can bound all three terms of the inequality in this theorem by $d_{VC}(\mathcal{M},\ell)$, as is done in Corollary \ref{convergence_relativeTI}. The VC dimension equality is true since $\mathcal{G}_{\mathcal{M},\ell} = \mathcal{G}_{\mathcal{M},\ell^{\prime}}$ (cf. Definition \ref{VCdimension}), as each function $g_{\beta,h}(z) = \mathds{1}\{\ell(z,h) \geq \beta\}$ in $\mathcal{G}_{\mathcal{M},\ell}$ has a correspondent $g^{\prime}_{\beta^{q},h}(z) = \mathds{1}\{\ell^{\prime}(z,h) \geq \beta^{q}\}$ in $\mathcal{G}_{\mathcal{M},\ell^{\prime}}$ that is such that $g \equiv g^{\prime}$. 
	\end{remark}
	
	\begin{remark}
		\label{remark_finite_moments}
		Condition \eqref{finite_moments} is not actually satisfied by many $\mathcal{H}$, for instance it does not hold for linear regression under the quadratic loss function. However, one can actually consider a $\mathcal{M} \subset \mathcal{H}$ such that \eqref{finite_moments} is true, with $d_{VC}(\mathcal{M}) = d_{VC}(\mathcal{H})$ and $L(h^{\star}) = L(h^{\star}_{\mathcal{M}})$, without loss of generality. For example, in linear regression one could consider only hypotheses with parameters bounded by a \textit{very large} constant $\gamma$, excluding hypotheses that are unlikely to be the target one. Observe that, in this example, it is better to consider the bounds for relative type I estimation error of unbounded loss functions, rather than consider that the loss function is bounded by a very large constant $C = \mathcal{O}(\gamma^2)$, which would generate bad bounds when applying Corollary \ref{cor3TypeI}. The results for unbounded loss functions hold for bounded ones, with $p$ arbitrarily large.
	\end{remark}
	
	\begin{remark}
		\label{remark_geq1}
		The main reason we assume that $\ell(z,h) \geq 1$, for all $z \in \mathcal{Z}$ and $h \in \mathcal{H}$, is to simplify the argument before \eqref{Ap1Eq11}, which could fail if the losses were lesser than one. We believe this assumption could be dropped at the cost of more technical results. Nevertheless, the results in Corollary \ref{convergence_relativeTI} present an exponential bound for
		\begin{equation*}
			\mathbb{P}\left(\sup\limits_{h \in \mathcal{H}} \frac{\text{\textbar}L(h) - L_{\mathcal{D}_{N}}(h)}{L(h) + 1}\text{\textbar} > \epsilon\right)
		\end{equation*}
		for any unbounded loss function $\ell$. We note that, if we had not imposed this constraint in the loss function, we would have to deal with the denominators in the estimation errors, which could then be zero. This could have been easily accomplished by summing a constant to the denominators and then making it go to zero after the bounds are established, that is, find bounds for
		\begin{equation*}
			\mathbb{P}\left(\sup\limits_{h \in \mathcal{H}} \frac{\text{\textbar}L(h) - L_{\mathcal{D}_{N}}(h)}{L(h) + \varsigma}\text{\textbar} > \epsilon\right),
		\end{equation*}
		and then make $\varsigma \to 0$. This is done in \cite{cortes2019}. By considering loss functions greater or equal to one, we have avoided the need to have heavier notations and more technical details when establishing the convergence of relative estimation errors. 
	\end{remark}
	
	\subsection{Convergence to zero of type II estimation error}
	
	A bound for type II estimation error \eqref{GE2} and relative type II estimation error, defined as
	\begin{equation*}
		\textbf{(II)} \ \ \frac{L(\hat{h}^{\mathcal{D}_{N}}) - L(h^{\star})}{L(\hat{h}^{\mathcal{D}_{N}})}
	\end{equation*}
	follow immediately from a bound obtained for the respective type I error. This is a consequence of the following elementary inequality, which can be found in part in \cite[Lemma~8.2]{devroye1996}.
	
	\begin{lemma}
		\label{lemmaTypeItoII}
		For any hypotheses space $\mathcal{H}$ and possible sample $\mathcal{D}_{N}$,
		\begin{align}
			\label{first_inequality}
			L(\hat{h}^{\mathcal{D}_{N}}) - L(h^{\star}) \leq 2 \ \sup\limits_{h \in \mathcal{H}} \text{\textbar}L(h) - L_{\mathcal{D}_{N}}(h)\text{\textbar},
		\end{align}
		and, if $\ell(z,h) \geq 1$, for all $z \in \mathcal{Z}$ and $h \in \mathcal{H}$, then
		\begin{equation}
			\label{second_inequality}
			\frac{L(\hat{h}^{\mathcal{D}_{N}}) - L(h^{\star})}{L(\hat{h}^{\mathcal{D}_{N}})} \leq 2 \ \sup\limits_{h \in \mathcal{H}} \frac{\text{\textbar}L(h) - L_{\mathcal{D}_{N}}(h)\text{\textbar}}{L(h)}.
		\end{equation}
		These inequalities yield
		\begin{align}	
			\label{firstPinequality}
			\mathbb{P}\left(L(\hat{h}^{\mathcal{D}_{N}}) - L(h^{\star}) > \epsilon\right) &\leq \mathbb{P}\left(\sup\limits_{h \in \mathcal{H}} \text{\textbar}L(h) - L_{\mathcal{D}_{N}}(h)\text{\textbar} > \epsilon/2\right)
		\end{align}
		and
		\begin{equation}	
			\label{secondPinequality}
			\mathbb{P}\left(\frac{L(\hat{h}^{\mathcal{D}_{N}}) - L(h^{\star})}{L(\hat{h}^{\mathcal{D}_{N}})} > \epsilon\right) \leq \mathbb{P}\left(\sup\limits_{h \in \mathcal{H}} \frac{\text{\textbar}L(h) - L_{\mathcal{D}_{N}}(h)\text{\textbar}}{L(h)} > \epsilon/2\right).
		\end{equation}
	\end{lemma}
	\begin{proof}
		The first inequality follows from
		\begin{align*}
			L(\hat{h}^{\mathcal{D}_{N}}) - L(h^{\star}) &= L(\hat{h}^{\mathcal{D}_{N}}) - L_{\mathcal{D}_{N}}(\hat{h}^{\mathcal{D}_{N}}) + L_{\mathcal{D}_{N}}(\hat{h}^{\mathcal{D}_{N}}) - L(h^{\star})\\
			&\leq  L(\hat{h}^{\mathcal{D}_{N}}) - L_{\mathcal{D}_{N}}(\hat{h}^{\mathcal{D}_{N}}) + L_{\mathcal{D}_{N}}(h^{\star}) - L(h^{\star})\\
			&\leq 2 \ \sup\limits_{h \in \mathcal{H}} \text{\textbar}L(h) - L_{\mathcal{D}_{N}}(h)\text{\textbar}.
		\end{align*}
		For the second one, analogous to the deduction above, we have that
		\begin{align*}
			\frac{L(\hat{h}^{\mathcal{D}_{N}}) - L(h^{\star})}{L(\hat{h}^{\mathcal{D}_{N}})} &\leq  \frac{L(\hat{h}^{\mathcal{D}_{N}}) - L_{\mathcal{D}_{N}}(\hat{h}^{\mathcal{D}_{N}})}{L(\hat{h}^{\mathcal{D}_{N}})} + \frac{L_{\mathcal{D}_{N}}(h^{\star}) - L(h^{\star})}{L(h^{\star})}\\
			&\leq 2 \ \sup\limits_{h \in \mathcal{H}} \frac{\text{\textbar}L(h) - L_{\mathcal{D}_{N}}(h)\text{\textbar}}{L(h)},
		\end{align*}
		since $L(\hat{h}^{\mathcal{D}_{N}}) \geq L(h^{\star})$. The inequalities \eqref{firstPinequality} and \eqref{secondPinequality} are direct from \eqref{first_inequality} and \eqref{second_inequality}.
	\end{proof}
	
	Combining Lemma \ref{lemmaTypeItoII} with Corollaries \ref{cor2TypeI} and \ref{cor3TypeI} we obtain the consistency of type II estimation error, when $d_{VC}(\mathcal{H}) < \infty$ and the loss function is bounded, what also concerns binary loss functions.
	
	\begin{corollary}
		\label{cor1TypeII}
		Fix a hypotheses space $\mathcal{H}$ and a loss function $\ell: \mathcal{Z} \times \mathcal{H} \mapsto \mathbb{R}_{+}$, with $0 \leq \ell(z,h) \leq C$ for all $z \in \mathcal{Z}, h \in \mathcal{H}$. Then,
		\begin{equation}
			\label{Ap1Eq7}
			\mathbb{P}\left(L(\hat{h}^{\mathcal{D}_{N}}) - L(h^{\star}) > \epsilon\right) \leq 8 \ \exp\left\{d_{VC}(\mathcal{H}) \left(1 + \ln \frac{N}{d_{VC}(\mathcal{H})}\right) - N \frac{\epsilon^2}{128C^{2}}\right\}.
		\end{equation}
		In particular, if $d_{VC}(\mathcal{H}) < \infty$, not only \eqref{Ap1Eq7} converges to zero, but also
		\begin{equation*}
			L(\hat{h}^{\mathcal{D}_{N}}) - L(h^{\star}) \xrightarrow[N \to \infty]{} 0,
		\end{equation*}
		with probability one by Borel-Cantelli Lemma.
	\end{corollary}
	
	Finally, combining Lemma \ref{lemmaTypeItoII} with Corollary \ref{convergence_relativeTI}, we obtain the consistency of relative type II estimation error when $d_{VC}(\mathcal{H}) < \infty$, the loss function is unbounded, and $P$ satisfies \eqref{tauStarAp}.
	
	\begin{corollary}
		\label{cor2TypeII}
		Fix a hypotheses space $\mathcal{H}$, an unbounded loss function $\ell$ and $\epsilon > 0$. The following holds:
		\begin{itemize}
			\item If \eqref{tauStarAp} holds for a $p \geq 4$ fixed, then
			\begin{align*}
				\mathbb{P}\Bigg(&\frac{L(\hat{h}^{\mathcal{D}_{N}}) - L(h^{\star})}{L(\hat{h}^{\mathcal{D}_{N}})} > \tau^{\star} \Lambda(\sqrt{p}) \epsilon\Bigg) \\
				&< 12 \exp\left\{d_{VC}(\mathcal{H})\left(1 + \ln \frac{2N}{d_{VC}(\mathcal{H})}\right) - \frac{\epsilon^{2}(1-\epsilon/2)^{2}N^{1 - \frac{2}{\sqrt{p}} + \frac{2}{p}}}{16}\right\}
			\end{align*}
			
			\item If \eqref{tauStarAp} holds for a $1 < p < 4$ fixed, then
			\begin{align*}
				\mathbb{P}\Bigg(&\frac{L(\hat{h}^{\mathcal{D}_{N}}) - L(h^{\star})}{L(\hat{h}^{\mathcal{D}_{N}})} > \tau^{\star} \Gamma\left(\sqrt{p},\frac{\epsilon}{N^{\frac{1}{\sqrt{p}} - \frac{1}{p}}}\right)\epsilon\Bigg) \\
				&< 12 \exp\left\{d_{VC}(\mathcal{H})\left(1 + \ln \frac{2N}{d_{VC}(\mathcal{H})}\right) - \frac{\epsilon^{2}N^{\frac{2(\sqrt{p} - 1)}{\sqrt{p}} - \frac{2}{\sqrt{p}} + \frac{2}{p}}}{2^{\frac{\sqrt{p} + 6}{2}}} \right\}.
			\end{align*}
		\end{itemize}
		In any case, if $d_{VC}(\mathcal{H}) < \infty$, then, by Borel-Cantelli Lemma,
		\begin{equation*}
			\sup\limits_{h \in \mathcal{H}} \frac{L(\hat{h}^{\mathcal{D}_{N}}) - L(h^{\star})}{L(\hat{h}^{\mathcal{D}_{N}})} \xrightarrow[N \to \infty]{} 0,
		\end{equation*}
		with probability one.
	\end{corollary}
	
	This ends the study of type II estimation error convergence.
	
	\section{Results of the experiments}
	\label{AppResults}
	
	% latex table generated in R 4.2.2 by xtable 1.8-4 package
	% Mon May  6 12:07:08 2024
	
	\begin{table}[ht]
		\centering
		\caption{Joint distributions considered in each example in Section \ref{SecPLLS}.} \label{jointDist}
		\resizebox{\linewidth}{!}{\begin{tabular}{|c|cc|cc|cc|cc|cc|cc|cc|}
				\hline
				\multirow{2}{*}{$x$} & \multicolumn{2}{c|}{Example 1} & \multicolumn{2}{c|}{Example 2}  & \multicolumn{2}{c|}{Example 3}  & \multicolumn{2}{c|}{Example 4} & \multicolumn{2}{c|}{Example 5} & \multicolumn{2}{c|}{Example 6} & \multicolumn{2}{c|}{Example 7}  \\ %\cline{2-15}
				& $p(0,x)$ & $p(1,x)$ & $p(0,x)$ & $p(1,x)$ & $p(0,x)$ & $p(1,x)$ & $p(0,x)$ & $p(1,x)$ & $p(0,x)$ & $p(1,x)$ & $p(0,x)$ & $p(1,x)$ & $p(0,x)$ & $p(1,x)$ \\ 
				\hline
				1 & 0.0612 & 0.0638 & 0.0600 & 0.0650 & 0.0563 & 0.0688 & 0.0500 & 0.0750 & 0.0375 & 0.0875 & 0.0250 & 0.1000 & 0.0125 & 0.1125 \\ 
				2 & 0.0638 & 0.0612 & 0.0650 & 0.0600 & 0.0688 & 0.0563 & 0.0750 & 0.0500 & 0.0875 & 0.0375 & 0.1000 & 0.0250 & 0.1125 & 0.0125 \\ 
				3 & 0.0612 & 0.0638 & 0.0600 & 0.0650 & 0.0563 & 0.0688 & 0.0500 & 0.0750 & 0.0375 & 0.0875 & 0.0250 & 0.1000 & 0.0125 & 0.1125 \\ 
				4 & 0.0638 & 0.0612 & 0.0650 & 0.0600 & 0.0688 & 0.0563 & 0.0750 & 0.0500 & 0.0875 & 0.0375 & 0.1000 & 0.0250 & 0.1125 & 0.0125 \\ 
				5 & 0.0612 & 0.0638 & 0.0600 & 0.0650 & 0.0563 & 0.0688 & 0.0500 & 0.0750 & 0.0375 & 0.0875 & 0.0250 & 0.1000 & 0.0125 & 0.1125 \\ 
				6 & 0.0638 & 0.0612 & 0.0650 & 0.0600 & 0.0688 & 0.0563 & 0.0750 & 0.0500 & 0.0875 & 0.0375 & 0.1000 & 0.0250 & 0.1125 & 0.0125 \\ 
				7 & 0.0612 & 0.0638 & 0.0600 & 0.0650 & 0.0563 & 0.0688 & 0.0500 & 0.0750 & 0.0375 & 0.0875 & 0.0250 & 0.1000 & 0.0125 & 0.1125 \\ 
				8 & 0.0638 & 0.0612 & 0.0650 & 0.0600 & 0.0688 & 0.0563 & 0.0750 & 0.0500 & 0.0875 & 0.0375 & 0.1000 & 0.0250 & 0.1125 & 0.0125 \\ 
				\hline $\epsilon^{\star}$ & \multicolumn{2}{c|}{0.0025}  & \multicolumn{2}{c|}{0.005}  & \multicolumn{2}{c|}{0.0125}  & \multicolumn{2}{c|}{0.025}  & \multicolumn{2}{c|}{0.050}  & \multicolumn{2}{c|}{0.075}  & \multicolumn{2}{c|}{0.100}  \\ 
				CE & \multicolumn{2}{c|}{0.693}  & \multicolumn{2}{c|}{0.692}  & \multicolumn{2}{c|}{0.688}  & \multicolumn{2}{c|}{0.673}  & \multicolumn{2}{c|}{0.611}  & \multicolumn{2}{c|}{0.500}  & \multicolumn{2}{c|}{0.325}  \\ 
				$L(h^{\star})$ & \multicolumn{2}{c|}{0.490}  & \multicolumn{2}{c|}{0.480}  & \multicolumn{2}{c|}{0.450}  & \multicolumn{2}{c|}{0.400}  & \multicolumn{2}{c|}{0.300}  & \multicolumn{2}{c|}{0.200}  & \multicolumn{2}{c|}{0.100}  \\ 
				\hline
				\multicolumn{15}{l}{\footnotesize $p(0,x) = \mathbb{P}(Y = 0,X = x)$, $p(1,x) = \mathbb{P}(Y = 1,X = x)$, CE: Conditional Entropy}
		\end{tabular}}
	\end{table}
	
	\begin{figure}[ht]
		\centering
		\includegraphics[width=\linewidth]{VC_dim.pdf}
		\caption{The frequency of $d_{VC}(\hat{\mathcal{M}})$ over the 1,000 samples in each case when learning via model selection in the whole partition lattice Learning Space in the example of Section \ref{SecPLLS}.} \label{fig_PLLS4}
	\end{figure}
	
	\begin{figure}[ht]
		\centering
		\includegraphics[width=\linewidth]{mean_partition_simulation_d2.pdf}
		\caption{The average bias and 95\% confidence interval of the ERM hypothesis and types II, III and IV estimation error over the 1,000 simulated samples for each case when learning via model selection in the models with VC dimension 2 in the partition lattice Learning Space in the example of Section \ref{SecPLLS}. These results are also presented in Table \ref{res_PLLS2}.} \label{fig_PLLS5}
	\end{figure}
	
	\footnotesize
	\begin{longtable}{lllllll}	
		\caption{\normalsize Average and standard error of the ERM hypothesis bias and estimation errors over the 1,000 samples simulated in Section \ref{SecPLLS} for learning in the whole partition lattice Learning Space.} \label{res_PLLS} \\
		\hline
		$\epsilon^{\star}$ & Learning & Size & Bias ERM & Type II & Type III & Type IV \\ 
		\hline
		\multirow{10}{*}{0.0025} & \multirow{5}{*}{Independent} & 16 & 0.0097 (1e-04) & 0.003 (1e-04) & 0.0059 (1e-04) & 0.0089 (1e-04) \\ 
		&  & 32 & 0.0097 (1e-04) & 0.0036 (1e-04) & 0.0059 (1e-04) & 0.0095 (1e-04) \\ 
		&  & 64 & 0.0094 (1e-04) & 0.0032 (1e-04) & 0.0063 (1e-04) & 0.0095 (1e-04) \\ 
		&  & 128 & 0.0094 (1e-04) & 0.0031 (1e-04) & 0.0066 (1e-04) & 0.0097 (1e-04) \\ 
		&  & 256 & 0.0092 (1e-04) & 0.0029 (1e-04) & 0.0067 (1e-04) & 0.0097 (1e-04) \\  \cline{2-7}
		& \multirow{5}{*}{Reusing} & 16 & 0.0097 (1e-04) & 0.0025 (1e-04) & 0.0059 (1e-04) & 0.0084 (1e-04) \\ 
		&  & 32 & 0.0097 (1e-04) & 0.0028 (1e-04) & 0.0062 (1e-04) & 0.009 (1e-04) \\ 
		&  & 64 & 0.0094 (1e-04) & 0.0027 (1e-04) & 0.0065 (1e-04) & 0.0092 (1e-04) \\ 
		&  & 128 & 0.0094 (1e-04) & 0.0025 (1e-04) & 0.0068 (1e-04) & 0.0093 (1e-04) \\ 
		&  & 256 & 0.0092 (1e-04) & 0.0024 (1e-04) & 0.0068 (1e-04) & 0.0093 (1e-04) \\ 
		\hline
		\multirow{10}{*}{0.005} & \multirow{5}{*}{Independent} & 16 & 0.0192 (2e-04) & 0.0064 (2e-04) & 0.0112 (2e-04) & 0.0177 (2e-04) \\ 
		&  & 32 & 0.019 (2e-04) & 0.007 (2e-04) & 0.0116 (2e-04) & 0.0185 (2e-04) \\ 
		&  & 64 & 0.0181 (2e-04) & 0.0067 (2e-04) & 0.0124 (2e-04) & 0.0191 (2e-04) \\ 
		&  & 128 & 0.0175 (2e-04) & 0.0062 (2e-04) & 0.013 (2e-04) & 0.0192 (2e-04) \\ 
		&  & 256 & 0.0164 (2e-04) & 0.0057 (2e-04) & 0.0131 (2e-04) & 0.0188 (2e-04) \\  \cline{2-7}
		& \multirow{5}{*}{Reusing} & 16 & 0.0192 (2e-04) & 0.0048 (2e-04) & 0.0116 (2e-04) & 0.0164 (2e-04) \\ 
		&  & 32 & 0.019 (2e-04) & 0.0056 (2e-04) & 0.0122 (2e-04) & 0.0178 (2e-04) \\ 
		&  & 64 & 0.0181 (2e-04) & 0.0046 (2e-04) & 0.013 (2e-04) & 0.0177 (2e-04) \\ 
		&  & 128 & 0.0175 (2e-04) & 0.0043 (2e-04) & 0.0133 (2e-04) & 0.0176 (2e-04) \\ 
		&  & 256 & 0.0164 (2e-04) & 0.0035 (2e-04) & 0.0134 (2e-04) & 0.017 (2e-04) \\ 
		\hline
		\multirow{10}{*}{0.0125} & \multirow{5}{*}{Independent} & 16 & 0.0446 (5e-04) & 0.0154 (5e-04) & 0.0279 (5e-04) & 0.0433 (4e-04) \\ 
		&  & 32 & 0.0426 (5e-04) & 0.0166 (6e-04) & 0.028 (4e-04) & 0.0446 (5e-04) \\ 
		&  & 64 & 0.0396 (6e-04) & 0.0154 (5e-04) & 0.0297 (4e-04) & 0.0451 (5e-04) \\ 
		&  & 128 & 0.0349 (5e-04) & 0.0141 (5e-04) & 0.0307 (4e-04) & 0.0448 (5e-04) \\ 
		&  & 256 & 0.0287 (5e-04) & 0.0136 (5e-04) & 0.0293 (4e-04) & 0.0429 (5e-04) \\  \cline{2-7}
		& \multirow{5}{*}{Reusing} & 16 & 0.0446 (5e-04) & 0.0102 (5e-04) & 0.0283 (5e-04) & 0.0385 (5e-04) \\ 
		&  & 32 & 0.0426 (5e-04) & 0.0104 (5e-04) & 0.0297 (4e-04) & 0.04 (5e-04) \\ 
		&  & 64 & 0.0396 (6e-04) & 0.0087 (4e-04) & 0.0305 (4e-04) & 0.0392 (5e-04) \\ 
		&  & 128 & 0.0349 (5e-04) & 0.0059 (4e-04) & 0.0309 (4e-04) & 0.0368 (5e-04) \\ 
		&  & 256 & 0.0287 (5e-04) & 0.0042 (3e-04) & 0.0282 (4e-04) & 0.0324 (5e-04) \\ 
		\hline
		\multirow{10}{*}{0.025} & \multirow{5}{*}{Independent} & 16 & 0.0804 (0.001) & 0.0287 (0.0011) & 0.0535 (0.0011) & 0.0822 (9e-04) \\ 
		&  & 32 & 0.0682 (0.001) & 0.0288 (0.0011) & 0.0538 (9e-04) & 0.0825 (0.001) \\ 
		&  & 64 & 0.0575 (0.001) & 0.0283 (0.0011) & 0.0535 (9e-04) & 0.0819 (0.0011) \\ 
		&  & 128 & 0.0413 (9e-04) & 0.024 (0.001) & 0.046 (9e-04) & 0.07 (0.0012) \\ 
		&  & 256 & 0.0255 (7e-04) & 0.0148 (9e-04) & 0.0365 (9e-04) & 0.0513 (0.0012) \\  \cline{2-7}
		& \multirow{5}{*}{Reusing} & 16 & 0.0804 (0.001) & 0.0165 (8e-04) & 0.0527 (0.001) & 0.0692 (0.0011) \\ 
		&  & 32 & 0.0682 (0.001) & 0.0119 (7e-04) & 0.0537 (9e-04) & 0.0656 (0.001) \\ 
		&  & 64 & 0.0575 (0.001) & 0.0093 (6e-04) & 0.0508 (9e-04) & 0.0601 (0.001) \\ 
		&  & 128 & 0.0413 (9e-04) & 0.0052 (4e-04) & 0.0434 (9e-04) & 0.0486 (9e-04) \\ 
		&  & 256 & 0.0255 (7e-04) & 0.0024 (2e-04) & 0.0334 (8e-04) & 0.0358 (9e-04) \\ 
		\hline
		\multirow{10}{*}{0.05} & \multirow{5}{*}{Independent} & 16 & 0.1169 (0.0019) & 0.0464 (0.002) & 0.095 (0.0022) & 0.1414 (0.0022) \\ 
		&  & 32 & 0.085 (0.0018) & 0.044 (0.002) & 0.0787 (0.002) & 0.1227 (0.0025) \\ 
		&  & 64 & 0.0475 (0.0015) & 0.0321 (0.0018) & 0.0616 (0.0017) & 0.0937 (0.0025) \\ 
		&  & 128 & 0.0215 (0.001) & 0.0124 (0.0012) & 0.0384 (0.0014) & 0.0508 (0.002) \\ 
		&  & 256 & 0.0054 (5e-04) & 0.0023 (5e-04) & 0.0158 (9e-04) & 0.0181 (0.0011) \\  \cline{2-7}
		& \multirow{5}{*}{Reusing} & 16 & 0.1169 (0.0019) & 0.0169 (0.001) & 0.0836 (0.002) & 0.1005 (0.0021) \\ 
		&  & 32 & 0.085 (0.0018) & 0.0146 (9e-04) & 0.0703 (0.0018) & 0.0849 (0.0019) \\ 
		&  & 64 & 0.0475 (0.0015) & 0.0068 (6e-04) & 0.0562 (0.0016) & 0.063 (0.0017) \\ 
		&  & 128 & 0.0215 (0.001) & 0.0026 (4e-04) & 0.0352 (0.0012) & 0.0378 (0.0013) \\ 
		&  & 256 & 0.0054 (5e-04) & 3e-04 (1e-04) & 0.0155 (9e-04) & 0.0159 (9e-04) \\ 
		\hline
		\multirow{10}{*}{0.075} & \multirow{5}{*}{Independent} & 16 & 0.1273 (0.0025) & 0.0552 (0.0029) & 0.106 (0.0034) & 0.1612 (0.0039) \\ 
		&  & 32 & 0.0677 (0.002) & 0.0403 (0.0025) & 0.0679 (0.0025) & 0.1082 (0.0036) \\ 
		&  & 64 & 0.0241 (0.0013) & 0.0124 (0.0015) & 0.0377 (0.0018) & 0.05 (0.0024) \\ 
		&  & 128 & 0.0036 (5e-04) & 0.0013 (4e-04) & 0.0148 (0.001) & 0.0161 (0.0012) \\ 
		&  & 256 & 1e-04 (1e-04) & 0 (0) & 0.0029 (5e-04) & 0.0029 (5e-04) \\  \cline{2-7}
		& \multirow{5}{*}{Reusing} & 16 & 0.1273 (0.0025) & 0.0173 (0.0011) & 0.0842 (0.0029) & 0.1014 (0.003) \\ 
		&  & 32 & 0.0677 (0.002) & 0.0119 (9e-04) & 0.0575 (0.0022) & 0.0695 (0.0023) \\ 
		&  & 64 & 0.0241 (0.0013) & 0.0038 (5e-04) & 0.0352 (0.0016) & 0.0389 (0.0017) \\ 
		&  & 128 & 0.0036 (5e-04) & 8e-04 (2e-04) & 0.0146 (0.001) & 0.0153 (0.001) \\ 
		&  & 256 & 1e-04 (1e-04) & 0 (0) & 0.0029 (5e-04) & 0.0029 (5e-04) \\ 
		\hline
		\multirow{10}{*}{0.1} & \multirow{5}{*}{Independent} & 16 & 0.1014 (0.0028) & 0.0383 (0.0029) & 0.0864 (0.0043) & 0.1247 (0.0051) \\ 
		&  & 32 & 0.0316 (0.0017) & 0.0105 (0.0015) & 0.0382 (0.0024) & 0.0487 (0.003) \\ 
		&  & 64 & 0.0056 (7e-04) & 0.0016 (5e-04) & 0.0144 (0.0013) & 0.016 (0.0014) \\ 
		&  & 128 & 1e-04 (1e-04) & 0 (0) & 0.0013 (4e-04) & 0.0013 (4e-04) \\ 
		&  & 256 & 0 (0) & 0 (0) & 1e-04 (1e-04) & 1e-04 (1e-04) \\  \cline{2-7}
		& \multirow{5}{*}{Reusing} & 16 & 0.1014 (0.0028) & 0.0128 (0.0011) & 0.053 (0.003) & 0.0658 (0.0032) \\ 
		&  & 32 & 0.0316 (0.0017) & 0.0038 (6e-04) & 0.0296 (0.0017) & 0.0334 (0.0018) \\ 
		&  & 64 & 0.0056 (7e-04) & 0.0013 (4e-04) & 0.0137 (0.0011) & 0.015 (0.0012) \\ 
		&  & 128 & 1e-04 (1e-04) & 0 (0) & 0.0013 (4e-04) & 0.0013 (4e-04) \\ 
		&  & 256 & 0 (0) & 0 (0) & 1e-04 (1e-04) & 1e-04 (1e-04) \\ 
		\hline
	\end{longtable}
	
	\begin{longtable}{lllllll}
		\caption{\normalsize Average and standard error of the ERM hypothesis bias and estimation errors over the 1,000 samples simulated in Section \ref{SecPLLS} for learning in the partition lattice Learning Space restricted to the models with VC dimension 2.} \label{res_PLLS2} \\
		\hline
		$\epsilon^{\star}$ & Learning & Size & Bias ERM & Type II & Type III & Type IV \\ 
		\hline
		\multirow{10}{*}{0.0025} & \multirow{5}{*}{Independent} & 16 & 0.0097 (1e-04) & 0.0048 (1e-04) & 0.0032 (1e-04) & 0.0079 (1e-04) \\ 
		&  & 32 & 0.0097 (1e-04) & 0.0046 (1e-04) & 0.0045 (1e-04) & 0.0091 (1e-04) \\ 
		&  & 64 & 0.0094 (1e-04) & 0.0039 (1e-04) & 0.0054 (1e-04) & 0.0093 (1e-04) \\ 
		&  & 128 & 0.0094 (1e-04) & 0.0038 (1e-04) & 0.0058 (1e-04) & 0.0096 (1e-04) \\ 
		&  & 256 & 0.0092 (1e-04) & 0.0035 (1e-04) & 0.0062 (1e-04) & 0.0096 (1e-04) \\ \cline{2-7}
		& \multirow{5}{*}{Reusing} & 16 & 0.0097 (1e-04) & 0.0031 (1e-04) & 0.0044 (1e-04) & 0.0075 (1e-04) \\ 
		&  & 32 & 0.0097 (1e-04) & 0.003 (1e-04) & 0.0052 (1e-04) & 0.0082 (1e-04) \\ 
		&  & 64 & 0.0094 (1e-04) & 0.0026 (1e-04) & 0.0057 (1e-04) & 0.0084 (1e-04) \\ 
		&  & 128 & 0.0094 (1e-04) & 0.0028 (1e-04) & 0.006 (1e-04) & 0.0088 (1e-04) \\ 
		&  & 256 & 0.0092 (1e-04) & 0.0026 (1e-04) & 0.0063 (1e-04) & 0.0088 (1e-04) \\ 
		\hline
		\multirow{10}{*}{0.005} & \multirow{5}{*}{Independent} & 16 & 0.0192 (2e-04) & 0.0096 (2e-04) & 0.0061 (1e-04) & 0.0157 (2e-04) \\ 
		&  & 32 & 0.019 (2e-04) & 0.0089 (2e-04) & 0.0087 (2e-04) & 0.0177 (2e-04) \\ 
		&  & 64 & 0.0181 (2e-04) & 0.0081 (2e-04) & 0.0106 (2e-04) & 0.0187 (2e-04) \\ 
		&  & 128 & 0.0175 (2e-04) & 0.0075 (2e-04) & 0.0114 (2e-04) & 0.019 (2e-04) \\ 
		&  & 256 & 0.0164 (2e-04) & 0.0066 (2e-04) & 0.012 (2e-04) & 0.0185 (2e-04) \\  \cline{2-7}
		& \multirow{5}{*}{Reusing} & 16 & 0.0192 (2e-04) & 0.006 (2e-04) & 0.0087 (2e-04) & 0.0147 (2e-04) \\ 
		&  & 32 & 0.019 (2e-04) & 0.0055 (2e-04) & 0.0106 (2e-04) & 0.016 (2e-04) \\ 
		&  & 64 & 0.0181 (2e-04) & 0.0045 (2e-04) & 0.0116 (2e-04) & 0.0161 (2e-04) \\ 
		&  & 128 & 0.0175 (2e-04) & 0.0044 (2e-04) & 0.0118 (2e-04) & 0.0162 (2e-04) \\ 
		&  & 256 & 0.0164 (2e-04) & 0.0034 (2e-04) & 0.0121 (2e-04) & 0.0155 (2e-04) \\ 
		\hline
		\multirow{10}{*}{0.0125} & \multirow{5}{*}{Independent} & 16 & 0.0446 (5e-04) & 0.0224 (5e-04) & 0.0155 (3e-04) & 0.0379 (5e-04) \\ 
		&  & 32 & 0.0426 (5e-04) & 0.0212 (6e-04) & 0.0214 (4e-04) & 0.0427 (5e-04) \\ 
		&  & 64 & 0.0396 (6e-04) & 0.0181 (5e-04) & 0.0256 (4e-04) & 0.0437 (5e-04) \\ 
		&  & 128 & 0.0349 (5e-04) & 0.0168 (5e-04) & 0.0266 (4e-04) & 0.0434 (5e-04) \\ 
		&  & 256 & 0.0287 (5e-04) & 0.0152 (5e-04) & 0.0266 (4e-04) & 0.0418 (5e-04) \\  \cline{2-7}
		& \multirow{5}{*}{Reusing} & 16 & 0.0446 (5e-04) & 0.0115 (5e-04) & 0.0215 (4e-04) & 0.033 (5e-04) \\ 
		&  & 32 & 0.0426 (5e-04) & 0.009 (5e-04) & 0.0258 (4e-04) & 0.0349 (5e-04) \\ 
		&  & 64 & 0.0396 (6e-04) & 0.0074 (5e-04) & 0.0277 (4e-04) & 0.0351 (5e-04) \\ 
		&  & 128 & 0.0349 (5e-04) & 0.0057 (4e-04) & 0.0268 (4e-04) & 0.0325 (5e-04) \\ 
		&  & 256 & 0.0287 (5e-04) & 0.0035 (4e-04) & 0.0238 (4e-04) & 0.0273 (5e-04) \\ 
		\hline
		\multirow{10}{*}{0.025} & \multirow{5}{*}{Independent} & 16 & 0.0804 (0.001) & 0.0404 (0.001) & 0.0294 (7e-04) & 0.0698 (0.0011) \\ 
		&  & 32 & 0.0682 (0.001) & 0.036 (0.0011) & 0.0404 (8e-04) & 0.0763 (0.0011) \\ 
		&  & 64 & 0.0575 (0.001) & 0.0329 (0.0011) & 0.0454 (8e-04) & 0.0783 (0.0011) \\ 
		&  & 128 & 0.0413 (9e-04) & 0.0255 (0.001) & 0.0418 (8e-04) & 0.0673 (0.0012) \\ 
		&  & 256 & 0.0255 (7e-04) & 0.0154 (9e-04) & 0.0348 (8e-04) & 0.0502 (0.0012) \\  \cline{2-7}
		& \multirow{5}{*}{Reusing} & 16 & 0.0804 (0.001) & 0.0159 (9e-04) & 0.0414 (8e-04) & 0.0573 (0.001) \\ 
		&  & 32 & 0.0682 (0.001) & 0.0099 (8e-04) & 0.0448 (8e-04) & 0.0547 (0.001) \\ 
		&  & 64 & 0.0575 (0.001) & 0.0049 (6e-04) & 0.0436 (9e-04) & 0.0485 (0.001) \\ 
		&  & 128 & 0.0413 (9e-04) & 0.0026 (4e-04) & 0.0339 (8e-04) & 0.0364 (9e-04) \\ 
		&  & 256 & 0.0255 (7e-04) & 3e-04 (1e-04) & 0.0221 (7e-04) & 0.0224 (7e-04) \\ 
		\hline
		\multirow{10}{*}{0.05} & \multirow{5}{*}{Independent} & 16 & 0.1169 (0.0019) & 0.0591 (0.0021) & 0.0522 (0.0013) & 0.1113 (0.0023) \\ 
		&  & 32 & 0.085 (0.0018) & 0.0514 (0.002) & 0.0598 (0.0015) & 0.1112 (0.0024) \\ 
		&  & 64 & 0.0475 (0.0015) & 0.034 (0.0018) & 0.057 (0.0016) & 0.091 (0.0024) \\ 
		&  & 128 & 0.0215 (0.001) & 0.0127 (0.0012) & 0.0366 (0.0013) & 0.0493 (0.0019) \\ 
		&  & 256 & 0.0054 (5e-04) & 0.0023 (5e-04) & 0.0156 (9e-04) & 0.0179 (0.0011) \\  \cline{2-7}
		& \multirow{5}{*}{Reusing} & 16 & 0.1169 (0.0019) & 0.0126 (0.0012) & 0.0626 (0.0015) & 0.0752 (0.0018) \\ 
		&  & 32 & 0.085 (0.0018) & 0.0037 (6e-04) & 0.0565 (0.0015) & 0.0602 (0.0017) \\ 
		&  & 64 & 0.0475 (0.0015) & 0.0013 (4e-04) & 0.0341 (0.0013) & 0.0354 (0.0013) \\ 
		&  & 128 & 0.0215 (0.001) & 2e-04 (1e-04) & 0.0161 (9e-04) & 0.0162 (9e-04) \\ 
		&  & 256 & 0.0054 (5e-04) & 0 (0) & 0.0041 (4e-04) & 0.0041 (4e-04) \\ 
		\hline
		\multirow{10}{*}{0.075} & \multirow{5}{*}{Independent} & 16 & 0.1273 (0.0025) & 0.0662 (0.0029) & 0.0584 (0.0019) & 0.1246 (0.0034) \\ 
		&  & 32 & 0.0677 (0.002) & 0.0423 (0.0025) & 0.0563 (0.0019) & 0.0986 (0.0034) \\ 
		&  & 64 & 0.0241 (0.0013) & 0.0127 (0.0015) & 0.0355 (0.0016) & 0.0482 (0.0023) \\ 
		&  & 128 & 0.0036 (5e-04) & 0.0013 (4e-04) & 0.0148 (0.001) & 0.0161 (0.0012) \\ 
		&  & 256 & 1e-04 (1e-04) & 0 (0) & 0.0029 (5e-04) & 0.0029 (5e-04) \\  \cline{2-7}
		& \multirow{5}{*}{Reusing} & 16 & 0.1273 (0.0025) & 0.0065 (0.0011) & 0.0605 (0.002) & 0.0671 (0.0022) \\ 
		&  & 32 & 0.0677 (0.002) & 7e-04 (3e-04) & 0.0372 (0.0016) & 0.0379 (0.0016) \\ 
		&  & 64 & 0.0241 (0.0013) & 0 (0) & 0.014 (0.001) & 0.014 (0.001) \\ 
		&  & 128 & 0.0036 (5e-04) & 0 (0) & 0.0026 (4e-04) & 0.0026 (4e-04) \\ 
		&  & 256 & 1e-04 (1e-04) & 0 (0) & 1e-04 (1e-04) & 1e-04 (1e-04) \\ 
		\hline
		\multirow{10}{*}{0.1} & \multirow{5}{*}{Independent} & 16 & 0.1014 (0.0028) & 0.0444 (0.003) & 0.0384 (0.0018) & 0.0828 (0.0038) \\ 
		&  & 32 & 0.0316 (0.0017) & 0.0111 (0.0016) & 0.0304 (0.0017) & 0.0415 (0.0025) \\ 
		&  & 64 & 0.0056 (7e-04) & 0.0016 (5e-04) & 0.0138 (0.0011) & 0.0154 (0.0013) \\ 
		&  & 128 & 1e-04 (1e-04) & 0 (0) & 0.0013 (4e-04) & 0.0013 (4e-04) \\ 
		&  & 256 & 0 (0) & 0 (0) & 1e-04 (1e-04) & 1e-04 (1e-04) \\  \cline{2-7}
		& \multirow{5}{*}{Reusing} & 16 & 0.1014 (0.0028) & 0.001 (5e-04) & 0.0334 (0.0018) & 0.0344 (0.0019) \\ 
		&  & 32 & 0.0316 (0.0017) & 0 (0) & 0.0106 (0.001) & 0.0106 (0.001) \\ 
		&  & 64 & 0.0056 (7e-04) & 0 (0) & 0.0028 (5e-04) & 0.0028 (5e-04) \\ 
		&  & 128 & 1e-04 (1e-04) & 0 (0) & 0 (0) & 0 (0) \\ 
		&  & 256 & 0 (0) & 0 (0) & 0 (0) & 0 (0) \\
		\hline
	\end{longtable}
	\normalsize 
	
	\begin{figure}[ht]
		\centering
		\includegraphics[width=\linewidth]{VC_dim_regression.pdf}
		\caption{The frequency of the number of variables/parts of the model selected over the 1,000 samples in each case when learning via model selection in the partition lattice and variable selection Learning Space in the example of Section \ref{SecRegression}.} \label{fig_reg3}
	\end{figure}

\section*{Acknowledgments}

D. Marcondes was funded by grant \#2022/06211-2, S�o Paulo Research Foundation (FAPESP), during the writing of this paper. We thank J. Barrera and A. Simonis for many fruitful conversations about Learning Spaces.

\bibliographystyle{plain}
\bibliography{ref}


\end{document}

