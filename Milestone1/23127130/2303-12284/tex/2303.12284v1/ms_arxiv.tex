% ****** Start of file apssamp.tex ******
%
%   This file is part of the APS files in the REVTeX 4.2 distribution.
%   Version 4.2a of REVTeX, December 2014
%
%   Copyright (c) 2014 The American Physical Society.
%
%   See the REVTeX 4 README file for restrictions and more information.
%
% TeX'ing this file requires that you have AMS-LaTeX 2.0 installed
% as well as the rest of the prerequisites for REVTeX 4.2
%
% See the REVTeX 4 README file
% It also requires running BibTeX. The commands are as follows:
%
%  1)  latex apssamp.tex
%  2)  bibtex apssamp
%  3)  latex apssamp.tex
%  4)  latex apssamp.tex
%
\documentclass[%
 reprint,
%superscriptaddress,
%groupedaddress,
%unsortedaddress,
%runinaddress,
%frontmatterverbose, 
%preprint,
%preprintnumbers,
%nofootinbib,
%nobibnotes,
%bibnotes,
 amsmath,amssymb,
 aps,
 prl,
%pra,
%prb,
%rmp,
%prstab,
%prstper,
%floatfix,
]{revtex4-2}

\usepackage{aas_macros}
\usepackage{graphicx}% Include figure files
\usepackage{dcolumn}% Align table columns on decimal point
\usepackage{bm}% bold math
\usepackage{hyperref}% add hypertext capabilities
\usepackage{color}
\usepackage{natbib}
\usepackage{CJK}
\usepackage[normalem]{ulem}
%\usepackage[mathlines]{lineno}% Enable numbering of text and display math
%\linenumbers\relax % Commence numbering lines
\usepackage[english]{babel}

%\usepackage[showframe,%Uncomment any one of the following lines to test 
%%scale=0.7, marginratio={1:1, 2:3}, ignoreall,% default settings
%%text={7in,10in},centering,
%%margin=1.5in,
%%total={6.5in,8.75in}, top=1.2in, left=0.9in, includefoot,
%%height=10in,a5paper,hmargin={3cm,0.8in},
%]{geometry}

%%%%%%%%%%%%% Journals %%%%%%%%%%%%%%%%%%%
\newcommand{\epja}{Eur.~Phys.~J.~A}
\newcommand{\prx}{Phys.~Rev.~X}
\newcommand{\rmph}{Rev.~Mod.~Phys.~}
\newcommand{\cqg}{{Class.~Quant.~Grav.}~}
%\newcommand{\apj}{{Astrophys.~J.}}
%\newcommand{\apjl}{{Astrophys. J. Lett.}}
%\newcommand{\apjs}{{Astrophys. J. Suppl. Ser.}}
\newcommand{\jcph}{{J.~Comput.~Phys.}}
\newcommand{\jcapp}{{J.~Cosmol.~Astropart.~Phys.}}
%\newcommand{\physrep}{{Phys. Rep.}}
%\newcommand{\mnras}{{Mon. Not. R. Astron. Soc.}}
%\newcommand{\aap}{Astron. Astrophys.}
%\newcommand{\pasp}{Pub. Astron. Soc. Pac.}
\newcommand{\pthphys}{Prog.~Theor.~Phys.}
\newcommand{\lrr}{Living.~Rev.~Relativ.}

%%%%%%%%%% Custom commands %%%%%%%%%%%%%%%%%%%%5
\newcommand{\dms}[1]{\textcolor{red}{#1}}
\newcommand{\luc}[1]{\textcolor{blue}{[[#1]]}}


\begin{document}
\begin{CJK*}{UTF8}{gbsn} % Use default fonts from CJK (see below)
\preprint{APS/123-QED}

%\title{A magnetar-powered jet as the origin of blue kilonovae.}

\title{Jets from neutron-star merger remnants and massive blue kilonovae}
%\title{Magnetically-driven outflows from hypermassive neutron star remnants: \\ GRMHD simulations of binary neutron star mergers with weak interactions}
% Force line breaks with \\
%\thanks{A footnote to the article title}%

\author{Luciano Combi$^{1,2,3}$}
\altaffiliation{CITA National Fellow}

\author{Daniel M.~Siegel$^{4,1,2}$}

\affiliation{$^1$Perimeter Institute for Theoretical Physics, Waterloo, Ontario N2L 2Y5, Canada\\
$^2$Department of Physics, University of Guelph, Guelph, Ontario N1G 2W1, Canada\\
$^3$Instituto Argentino de Radioastronom\'ia (IAR, CCT La Plata, CONICET/CIC)\\
C.C.5, (1984) Villa Elisa, Buenos Aires, Argentina\\
$^4$Institute of Physics, University of Greifswald, D-17489 Greifswald, Germany}

%\collaboration{MUSO Collaboration}
%\noaffiliation

%\date{\today}% It is always \today, today,
             %  but any date may be explicitly specified

\begin{abstract} 
We perform high-resolution three-dimensional general-relativistic magnetohydrodynamic simulations with neutrino transport of binary neutron star (BNS) mergers resulting in a long-lived remnant neutron star, with properties typical of galactic BNS and consistent with those inferred for the first observed BNS merger GW170817. We demonstrate self-consistently that within $\lesssim\!30$\,ms post-merger magnetized ($\sigma\sim 5-10$) twin polar jets emerge with asymptotic Lorentz factor $\Gamma\sim 5-10$, which successfully break out from the merger debris within $\lesssim\!20$\,ms. A fast ($v\lesssim 0.6c$), magnetized ($\sigma\sim 0.1$) wind surrounds the jet core and generates a UV/blue kilonova precursor on timescales of hours, similar to the precursor signal due to free neutron decay in fast dynamical ejecta. Post-merger ejecta are quickly dominated by MHD-driven outflows from an accretion disk. We demonstrate that within only 50\,ms post-merger, $\gtrsim 2\times 10^{-2}M_\odot$ of lanthanide-free, quasi-spherical ejecta with velocity $\sim\!0.1c$ is launched, yielding a kilonova signal consistent with GW170817 on timescales of $\lesssim\!5$\,d.
\end{abstract}

%\keywords{Suggested keywords}%Use showkeys class option if keyword
\maketitle
\end{CJK*}


%%%%%%%%% Introduction %%%%%%%%%

\textit{Introduction.---}The astrophysical origin of about half of the elements heavier than iron, created via rapid neutron capture (the r-process), remains an open question 
\cite{cowan2021Origin,siegel2022r}. The first observed binary neutron-star merger, detected via gravitational waves \cite{abbott2017GW170817}, referred to as GW170817, was followed by quasi-thermal emission \cite{abbott2017Multimessenger,coulter2017Swope,soares-santos2017Electromagnetic} consistent with radioactive heating from the nucleosynthesis of r-process elements in the merger debris---a kilonova \cite{metzger2010electromagnetic}. Based on inferred event rates of BNS mergers, the ejecta yield as well as the photometric and spectroscopic properties \cite{villar2017Combined,kasen2017Origin,smartt2017Kilonova,pian2017Spectroscopic,chornock2017Electromagnetic,tanaka2020Systematic,watson2019identification,kasliwal2022spitzer}, this transient not only provided strong evidence for the production of both light (atomic mass number $A\lesssim 135$) and heavy (lanthanide-bearing; $A\gtrsim 136$) r-process elements in this particular event, but also for BNS mergers being a potentially dominant production site of r-process elements in the Universe.

Whereas the origin of the lanthanide-bearing red and infrared emission of the GW170817 kilonova peaking on timescales of a week is naturally explained by MHD-driven outflows from a massive, self-regulated, neutrino-cooled accretion disk around a final remnant black hole \cite{siegel2017ThreeDimensional,kasen2017Origin,fernandez2019Longterm,christie2019role,li2021Neutrino}, the origin of the early ($\sim$day) blue and ultraviolet (UV) emission remained more elusive. Both the blue and red GW170817 kilonova emission are difficult to explain by dynamical debris from the collision itself \cite{siegel2019GW170817,metzger2020Kilonovae,radice2020Dynamics} (but see Ref.~\cite{kawaguchi2018Radiative} for a corner case); therefore, several alternative mechanisms for the origin of the blue emission post-merger have been considered, including a combination of magnetically driven and neutrino-driven winds from a remnant neutron star \cite{metzger2018Magnetar,ciolfi2020Magnetically,mosta2020Magnetar}, additional turbulent viscosity in the neutron-star remnant \cite{fujibayashi2020postmerger, radice2018ViscousDynamical}, spiral waves driven into a post-merger accretion disk by non-axisymmetric modes of a remnant neutron star \cite{nedora2019Spiralwave}, and outflows from a post-merger accretion disk both around a remnant neutron star \cite{fahlman2018Hypermassive} and a black hole \cite{miller2019Full}.

Here, we demonstrate by means of self-consistent ab-initio simulations of the merger and post-merger phases that the inferred properties of the blue GW170817 kilonova emission can arise naturally from mass ejection within only $\lesssim\!50$\,ms post-merger due to MHD-driven winds from an accretion disk, aided by non-linear hydrodynamic effects. The emergence of a jet from the remnant neutron star generates a UV/blue precursor signal and, upon collapse of the remnant, `seeds' an ultra-relativistic jet to generate a short gamma-ray burst.



%%%%%%%%% Computational setup %%%%%%%%%

\textit{Computational setup.---}We solve Einstein's equations coupled to the general-relativistic magneto-hydrodynamic (GRMHD) equations with weak interactions using an enhanced version \cite{siegel2018Threedimensional,combi2023grmhd} of the flux-conservative code \texttt{GRHydro} \citep{baiotti2003New, mosta2013GRHydro}, which is part of the \texttt{EinsteinToolkit} \footnote{\url{http://einsteintoolkit.org}} open-source code framework \cite{goodale2003Cactus,schnetter2004Evolutions,thornburg2004Fast,loffler2012Einstein, babiuc2019einstein}. We use the numerical setup presented in detail in Ref.~\cite{combi2023grmhd} (henceforth CS23); in particular, we implement the recovery of primitive variables of Ref.~\cite{siegel2018Recovery}, which provides support for tabulated nuclear equations of state (EOS), weak interactions, and neutrino radiation transport via a one-moment approximation of the general-relativistic Boltzmann equation \cite{radice2016Dynamical,radice2018Binary}.
A fixed Cartesian grid hierarchy composed of six nested refinement boxes is used. The finest mesh covers $\simeq\!76$\,km in diameter with a resolution of $\Delta x=180$\,m. The largest box has an extend of 3000\,km. Reflection symmetry across the orbital plane is employed for computational efficiency. A comparison run without imposing the symmetry at slightly lower resolution shows that our conclusions are not affected by this setup (see CS23). The GRMHD scheme uses an atmosphere floor of $\rho_{\rm atm} \sim 5 \times 10^2\,{\rm g}\,{\rm cm}^{-3}$. 

The initial data consists of two cold, $\beta$-equilibrated, equal-mass neutron stars of radius 11.6\,km and mass $1.35M_\odot$ in quasi-circular orbit at a separation of 45\,km. We build initial data with the elliptical solver LORENE \cite{gourgoulhon2001Quasiequilibrium}, employing the APR EOS \cite{akmal1998Equation} in finite-temperature, tabulated form \cite{schneider2019AkmalPandharipandeRavenhall}. This EOS generates cold non-rotating neutron stars with a maximum mass of $2.2M_\odot$ and a radius of 11.6\,km for a $1.4M_\odot$ star, well within the ballpark of currently allowed values \cite{landry2020nonparametric,cromartie2020relativistic}. While similar post-merger phenomenology (see below) is observed for the other, stiffer EOS configurations of CS23, we focus here on the APR configuration, which results in the longest-lived remnant neutron star. After setting the hydrodynamical variables, we initialize a weak poloidal magnetic seed field confined to the interior of each star with maximum strength of $B_{\rm max} = 3 \times 10^{15}$\,G at the center and total energy of $10^{48}$\,erg. The initial magnetic field is energetically and dynamically insignificant. 

%%%%%%%%% Magnetic field evolution and jet emergence %%%%%%%%%

\begin{figure}[tb!]
  \centering
  \includegraphics[width=\columnwidth]{avg_Bt_rm8_vs_t.pdf}
  \includegraphics[width=1\columnwidth]{toroidal_field.pdf}
  \caption{Top: Average toroidal field in the remnant neutron star as a function of time after merger (blue solid line). The dashed line represents the expected amplification by the Kelvin-Helmholtz instability, the dot-dashed line the turbulent amplification by large-scale vortices as a result of core bounces, and the dotted line the expected linear growth by magnetic winding. Bottom: 3D rendering of magnetic field lines at 15\,ms (left) and 50\,ms (right), showing the conversion of turbulent fields into large-scale toroidal structures that give rise to twin jets (a magnetic tower; \cite{Lynden-Bell1996}).}
  \label{fig-bTvst-3db}
\end{figure}

\textit{Magnetic field evolution and jet formation.---}During inspiral, the magnetic seed field remains buried inside the stars. Starting at the merger (referred to as $t=0$\,ms), the field is amplified exponentially by the Kelvin-Helmholtz instability (KHI) for $\approx\!2$\,ms, as evidenced by the toroidal field in Fig.~\ref{fig-bTvst-3db}. Since the KHI has no characteristic spatial scale, the finite resolution of our simulation is only able to capture partial amplification of the average toroidal magnetic field from essentially zero G to $\approx\!5\times 10^{15}$\,G. In the first $15$\,ms post-merger, the generation of large-scale eddies in the remnant's interior by quasi-radial core bounces further amplifies the toroidal field by an order of magnitude and roughly $\propto\!t^2$. These currents are associated with the formation of vortices around the inner core \cite{kastaun2021numerical, kastaun2015Propertiesa, ciolfi2017General} which dissipate kinetic into magnetic energy. 

The toroidal field eventually continues to grow at the expected linear growth rate of magnetic winding due to differential rotation inside the star. Amplification proceeds mainly in the slowly rotating core with positive angular velocity gradient interior to $r \approx 8$\,km. At larger radii, a nearly rotationally supported `Keplerian envelope' is established that gradually transitions into an accretion disk formed by merger debris, a well-established quasi-universal configuration \cite{kastaun2015Properties, ciolfi2017General, kastaun2021numerical, fujibayashi2018Mass}. Larger to infinite resolution would lead to earlier saturation of the turbulent magnetic field via the KHI \cite{kiuchi2014High,palenzuela2021turbulent}, only making the subsequent amplification processes obsolete, but unlikely altering the qualitative evolution of the system. 

At $\approx\!25$\,ms after merger, thanks to magnetic winding providing an inverse turbulent cascade, the small-scale, turbulent field has been wound up into a large-scale toroidal structure (Fig.~\ref{fig-bTvst-3db}). Owing to their magnetic buoyancy, toroidal fields eventually rise to the stellar surface in the polar regions, break out of the star and form a magnetic tower \cite{Lynden-Bell1996} (Figs.~\ref{fig-bTvst-3db}, \ref{fig-rho-betaqnet}). 

\begin{figure}[tb!]
  \centering
  \includegraphics[width=0.9\columnwidth]{double_beta_qnet.pdf}
   %\includegraphics[width=\columnwidth]{avg_rhoz_rm8_vs_t.pdf}
  \caption{
  %Top: 
  Meridional snapshot along the rotational axis showing the magnetic-to-fluid pressure ratio (upper half-plane) and the net specific neutrino heating rate (lower half-plane) once a stationary jet structure has emerged.}
  \label{fig-rho-betaqnet}
\end{figure}
%\textit{Magnetized wind outflow properties.---}Prior to the
\textit{Outflow properties.---}Prior to the emergence of the magnetic tower structure, strong neutrino radiation ($L_{\nu_{\rm e}} \approx 1.7 \times 10^{52}\,{\rm erg}\,{\rm s}^{-1}$) from the hot merger remnant drives a neutrino-driven wind \cite{dessart2009Neutrino,perego2014Neutrinodriven,desai2022three} of unbound material in polar regions with typical velocities $v\lesssim 0.1\,c$ and electron fraction $Y_e\approx 0.5$ via absorption of neutrinos in a gain layer above the stellar surface that extends out to $\lesssim\!50$\,km (Fig.~\ref{fig-rho-betaqnet}). In this gain layer, which is similar to that above proto-NSs in core-collapse supernovae, the net absorbed energy per unit time as seen by the Eulerian observer $\dot{Q}_{\rm net}=\int_{\theta<30^{\circ}} \dot{q}_{\rm net} \rho \Gamma \sqrt{\gamma}\,d^3x$ in a polar volume (polar angle $\theta<30^\circ)$ roughly equals the kinetic power $\dot{E}_{\rm k}= \int_{\theta<30^{\circ}}  \sqrt{\gamma}h \rho \Gamma (\alpha v^i-\beta^i)\,dA_i$ of the wind leaving that volume along the polar axis. Here, $\dot{q}_{\rm net}$ is the net specific neutrino heating rate, $h$ the specific enthalpy, $\Gamma =-n_\mu u^\mu$ is the Lorentz factor of the plasma with three-velocity $v^i$ and four-velocity $u^\mu$, $dA_i$ is the surface element, and $\alpha$, $\mathbf{\beta}$, and $\gamma$ denote the lapse, shift, and determinant of the three-metric of the adopted 3+1 foliation of spacetime with normal vector $n^\nu$, which defines the Eulerian observer. A steady-state wind profile $\rho\propto r^{-2}$ emerges in polar regions, as expected from mass conservation, $\dot{M} = \Delta\Omega\, r^2 \rho \Gamma v$, for a wind opening solid angle $\Delta\Omega$, constant mass-loss rate $\dot{M}$, and four-velocity $\Gamma v$ set by neutrino absorption at the base of the wind. % (Fig.~\ref{fig-rho-betaqnet}).

As the buoyant toroidal magnetic field structures break out of the star and neutrino absorption helps to form a magnetic tower along the rotational axis, a strongly magnetically dominated outflow with magnetic-to-fluid pressure ratio of up to $p_{\rm B}/p_{\rm f} \approx 10^{2}$ is established (Fig.~\ref{fig-rho-betaqnet}). A steady-state wind profile $\rho\propto r^{-2}$ quickly emerges (Fig.~\ref{fig-rho-betaqnet}), with a steady-state mass-loss rate $\dot{M}$ enhanced by roughly one order of magnitude---consistent with the wind solutions of Ref.~\cite{metzger2018Magnetar}. Within the same time window of 5--10\,ms, the kinetic power of the outflow increases by more than an order of magnitude and comes into equipartition with neutrino heating and the (dominant) Poynting luminosity of the magnetic structure, $\dot{E}_{\rm k}\approx \dot{Q}_{\rm net}+L_{\rm EM}\approx L_{\rm EM}$. The magnetization, $\sigma = L_{\rm EM}/\dot{M}$, in the polar 
%jet
region ($\theta\lesssim 30^\circ$ or specific entropy $s > 25 k_{\rm B}$\,baryon$^{-1}$), rapidly increases to $\sigma \approx 0.1$ (Fig.~\ref{fig-fluxes}). Through strong neutrino heating at the base, shock-heating at the jet head, and in internal shocks, the plasma outflow attains high specific entropy $\gtrsim\!(25-100) k_{\rm B}$\,baryon$^{-1}$ (Fig.~\ref{fig-vr-s}). Magneto-centrifugal forces accelerate the plasma in the polar funnel from average velocities of $v \lesssim 0.1c$ (neutrino-driven wind) to $v \approx (0.25-0.6)c$. For acceleration of the wind primarily by magnetic fields, the flow should acquire a four-velocity $ u = v^{r} \Gamma \approx c\sigma^{1/3}$ when it reaches the fast magnetosonic surface \cite{michel1969relativistic}. This limit with the density-averaged value of $\langle u \rangle \approx 0.35c$ is indeed reached in the polar region (Fig.~\ref{fig-fluxes}).

\begin{figure}[tb]
  \centering 
  \includegraphics[width=0.9\columnwidth]{double_vr_s.pdf}
  \caption{Meridional snapshot along the rotational axis, showing the successful break-out of a high-velocity, high-entropy jet structure from the surrounding merger debris (dynamical and post-merger wind ejecta). We show the radial velocity (top) and specific entropy (bottom). The white contour contains bound material according to the Bernoulli criterion (the accretion disk). Black contours contain dynamically unbound ejecta (geodesic criterion). The rest of the outflow domain is unbound according to the Bernoulli criterion. %Dot-dashed blue lines indicate the Alfv\'en surface.
  }
  \label{fig-vr-s}
\end{figure}
A magnetized ($\sigma \sim 5-10$) jet structure emerges in the polar funnel consisting of exclusively dynamically unbound material (geodesic criterion; $-u_0 < 1$) with an half-opening angle of $\approx 20^\circ$ at the base, and high entropy, $s \gtrsim 50 k_{\rm B}$\,baryon$^{-1}$. The jet head propagates with $v\gtrsim 0.6c$ through and breaks out from the envelope of merger debris (Fig.~\ref{fig-vr-s}). The jet is stabilized by subdominant
%($\sigma\ll 1$)
toroidal magnetic fields in the core, which reduce instabilities at the jet-envelope boundary layer \cite{gottlieb2020structure} while avoiding global (kink) instabilities that can develop in strongly magnetized jets ($\sigma \gg 1$) \cite{bromberg2016relativistic}. The terminal velocity of the jet outflow can be further boosted by neutrino pair-annihilation (expected to be subdominant; not included here) and dissipation of magnetic and thermal energy into kinetic energy at larger spatial scales, up to $\Gamma \lesssim - u_0 (h/h_{\infty} + b^2/\rho) \approx 5-10$. Here, $h_\infty$ denotes the EOS-specific asymptotic value of $h$ and $b$ is the co-moving magnetic field strength.

During the first 50\,ms post-merger, the system ejects $\gtrsim 2\times 10^{-2}M_\odot$ of matter at a time-averaged rate of $\dot{M} \sim 0.5 M_{\odot}\,{\rm s}^{-1}$. Before the emergence of the jet structure ($t\lesssim 30$\,ms), spiral waves propagating outward in the accretion disk \citep{nedora2019Spiralwave, nedora2021Dynamical} dominate angular momentum transport and mass ejection of the system of $\lesssim\!1 \times 10^{-2} M_{\odot}$ with mass-averaged velocity $\langle v \rangle\approx 0.1 c$. These waves and associated mass ejection, generated mainly by $m=2$ and $m=1$ non-axisymmetric density modes of the remnant through hydrodynamical 
instabilities \cite{lehner2016Instability,radice2016Onearmed, east2016Relativistic}, are visible as concentric waves in the radial velocity (Fig.~\ref{fig-vr-s}), paralleled by oscillations of the unbound mass flux (Fig.~\ref{fig-fluxes}). As the vicinity of the remnant becomes strongly magnetized and the jet structure emerges ($t\approx 30$\,ms), the unbound mass flux increases by an order of magnitude to $\dot{M}\approx 1 M_\odot\,{\rm s}^{-1}$ (Fig.~\ref{fig-fluxes}), quickly dominating the total mass unbound post-merger. Within only $\approx 15$\,ms a total of $\gtrsim 1\times 10^{-2}M_\odot$ of material ($\gtrsim 50\%$ of the cumulative mass ejected post-merger) is unbound with mass-averaged velocity $\langle v\rangle\approx 0.1 c$. 

As accretion blocks outward radial mass flux in equatorial regions over timescales of interest, the neutrino and magnetically-driven wind from the remnant totaling $1 \times 10^{-3}M_\odot$ escapes in polar directions ($ \theta \lesssim \!30^{\circ}$) with only $0.2 \times 10^{-3} M_\odot$ being launched within the jet core. The dominant contribution to ejected material, however, is launched as winds from the accretion disk. Disk winds intensify after $\approx\!30$\,ms when angular momentum transport by spiral waves through the compact bound merger debris, magnetic stresses in the vicinity of the star, and the onset of magnetohydrodynamic turbulence driven by the magnetorotational instability have established and enlarged the accretion disk to a radius of $\gtrsim\!150$\,km with approximate inflow--outflow equilibrium. The onset of strongly enhanced disk winds also coincides with the first cycles of an emerging dynamo as evident from a `butterfly diagram' similar to that obtained in previous work \cite{siegel2018Threedimensional}. Despite intense neutrino irradiation from the remnant, the disk then settles into a self-regulated state of moderate electron degeneracy $\mu_e/k_{\rm B}T\sim 1$ \cite{siegel2017ThreeDimensional,siegel2018Threedimensional}, which implies high neutron-richness of $Y_e\approx 0.1-0.15$ \cite{beloborodov2003nuclear,chen2007Neutrinocooled,siegel2017ThreeDimensional} (Fig.~\ref{fig-ye-eta}). The mass averaged $Y_e$ of the disk indeed shifts from $\approx\!0.25$ ($t<30$ ms) to $\approx\!0.15$ as it approaches a quasi-stationary state with an accretion rate of $\gtrsim\! 1 M_\odot\,{\rm s}^{-1}$ and mass of $\approx\!0.19 M_\odot$.

\begin{figure}[tb!]
  \centering
\includegraphics[width=0.9\columnwidth]{fluxes_vs_t2_polar.pdf}
  \caption{Total unbound mass flux $\dot{M}_{\rm tot}$ through a spherical shell with radius 300\,km and associated density averaged local four-velocity $u =v^r \Gamma$, magnetization $\sigma$, and expected velocity at the magnetosonic surface, $\sigma^{1/3}$, in polar regions ($\theta \lesssim 30^\circ$).}
  \label{fig-fluxes}
\end{figure}
\begin{figure}[tb]
  \centering
\includegraphics[width=0.9\columnwidth]{double_ye_etae_late.pdf}
  \caption{Meridional snapshot along the rotational axis showing the electron fraction (top) and electron degeneracy $\eta=\mu_e/k_{\rm B} T$ (bottom) with density contours at $\rho = [10^{7.5},10^{8},10^{9.75},10^{13}]\,{\rm g}\,{\rm cm}^{-3}$ as yellow, black, purple, and white solid lines, respectively, $\approx\!50$\,ms post-merger. %}
  The accretion disk is in a self-regulated state of moderate degeneracy ($\eta\sim 1$), which implies high neutron-richness ($Y_e\approx 0.15$).} % Strong neutrino irradiation from the remnant (cf.~Fig.~\ref{fig-rho-betaqnet}) protonizes the highly neutron-rich disk winds to $Y_e > 0.2$.}
  \label{fig-ye-eta}
\end{figure}
%%%%%%%%% Nucleosynthesis \& kilonova %%%%%%%%%

\textit{Nucleosynthesis \& kilonova.---}Figure~\ref{fig-distro} shows properties of unbound outflows at the onset of neutron capture reactions ($T\approx\!5$\,GK) as sampled by multiple families of $\approx 2\times 10^{4}$ unbound passive tracer particles injected into the simulation domain (see CS23 for details of tracer placement). Fast outflow speeds $>0.2c$ are almost exclusively associated with polar outflows. As a result of neutrino absorption but high outflow speeds due to magnetic fields, material ejected from the highly neutron-rich degenerate surface layer of the star is protonized to asymptotic values of $Y_{\rm e} \approx 0.3-0.4$, much lower than $Y_e\approx 0.5$ as in purely neutrino-driven winds of hot proto-neutron stars, even in the presence of fast rotation \cite{desai2022three}. Outflows from the self-regulated neutron-rich reservoir of the disk are protonized by absorption of intense neutrino radiation from the remnant (cf.~Figs.~\ref{fig-rho-betaqnet}, \ref{fig-ye-eta}) to a mass averaged value of $\langle Y_e \rangle\approx 0.3$ at 5\,GK.


\begin{figure}[tb!]
  \centering
  \includegraphics[width=1\columnwidth]{tracers_histo_polar_all.pdf}
  \includegraphics[width=1\columnwidth]{abundance_polar_all.pdf}
  \caption{Top: Unbound mass distribution in terms of electron fraction (left) and asymptotic expansion velocity (right) as sampled by tracer particles, normalized by the total ejected mass, with separate histrograms for the polar outflows.
  %associated with the jet. 
  $Y_e$ is extracted at the onset of neutron-capture reactions at 5\,GK. Bottom: Final nucleosynthetic abundances at $10^9$\,s from reaction network calculations for post-merger ejecta, compared to observed solar abundances \cite{arnould2007Rprocess} (arbitrary normalization).}
  \label{fig-distro}
\end{figure}

Nucleosynthesis calculations based on the unbound tracer particles are conducted with the nuclear reaction network \texttt{SkyNet} \cite{lippuner2017SkyNet} using 7843 nuclides and 140 000 nuclear reactions with the setup described in Refs.~\cite{li2021Neutrino} and CS23. They start in nuclear statistical equilibrium at a temperature of $T=7$\,GK and take neutrino irradiation into account using neutrino fluxes directly extracted from our simulation as in Ref.~\cite{li2021Neutrino}. Final abundances at $t=10^9$\,s are shown in Fig.~\ref{fig-distro}. Elements beyond the 2nd r-process peak ($A \approx 130$) are entirely suppressed, similar to results of long-term 2D hydrodynamical simulations of similar systems \cite{fujibayashi2020mass}, but different from Ref.~\cite{curtis2021Process}, which may result in part from differences in the treatment of neutrino absorption in the nuclear network calculation.

We compute kilonova light curves based on angular-dependent ejecta mass profiles extracted from the simulation, using the method presented in CS23. The resulting kilonova signal from post-merger ejecta is consistent with observations of GW170817 in the UV and blue bands up to several days (Fig.~\ref{fig-kn}). Underestimation on timescales $\gtrsim\!5$\,d can be explained by additional `redder' (lanthanide bearing) components \cite{villar2017Combined} not included here, which can be generated by neutron-richer accretion disk winds upon collapse of the remnant into a BH \cite{siegel2017ThreeDimensional,siegel2018Threedimensional,kasen2017Origin,li2021Neutrino}. The $\sim\!{\rm day}$ kilonova is determined by the disk outflows. Fast material from the jet region carries most of the kinetic energy but only 10\% of the total ejected mass; the latter dominates the signal on a few-hours timescale and can boost the $\sim\!{\rm hr}$ UV/blue signal depending on the line of sight. In direction of the jet, the signal is enhanced by up to 1.5 magnitudes, reaching similar luminosities to the kilonova precursor signal from free neutron decay in fast dynamical ejecta \cite{metzger2015Neutronpowered} when the boost due to relativistic effects is taken into account (CS23).

\begin{figure}[tb!]
  \centering
\includegraphics[width=1\columnwidth]{aprpostmerger_obsdata.pdf}
  \caption{Kilonova light curves of the $\approx\!2\times 10^{-2}M_\odot$ post-merger ejecta in various UV and blue bands, compared to observed data (dots) and upper limits (triangles) of the GW170817 kilonova \cite{villar2017Combined}. The kilonova lightcurves are computed for a distance of 40\,Mpc and an observer angle of $35^\circ$ wrt. the rotational axis, as inferred for GW170817 \cite{abbott2017GW170817, mooley2018superluminal}.
  }
  \label{fig-kn}
\end{figure}

%%%%%%%%% Conclusion %%%%%%%%%

\textit{Conclusion.---}These results provide strong evidence for massive ($\gtrsim 10^{-2}M_\odot$) kilonovae such as GW170817 with early ($\sim\!{\rm day}$) blue and late-time ($\sim\!{\rm week}$) red emission being dominated by post-merger disk outflows. This provides additional support to the conjecture of Ref.~\cite{siegel2022r} that outflows from accretion disks are the main astrophysical site of the Galactic r-process. In particular, we show that binaries consistent with GW170817 and typical of galactic BNS require a remnant lifetime of only $\approx\!50$\,ms to generate a lanthanide-poor blue kilonova component of $\gtrsim\!2\times 10^{-2}\,M_\odot$ with expansion velocity $v\approx 0.1c$ and lightcurves consistent with GW170817. While we find elements of previously proposed mechanisms such as magnetized winds from the remnant \cite{metzger2018Magnetar} and spiral waves \cite{nedora2019Spiralwave}, bulk mass ejection here is due to a combination of a magnetic jet ($\sigma\sim 5-10$) that emerges $\lesssim 30$\,ms post-merger, associated global magnetic stresses, and the onset of magnetohydrodynamic turbulence, which reconfigure the accretion disk, enhancing outflows that quickly dominate the cumulative post-merger mass ejected. At $\gtrsim\!50$\,ms post-merger the accretion disk with mass of $\approx\!0.19M_\odot$ and accretion rate of $\gtrsim\! 1M_\odot\,{\rm s}^{-1}$ is in a self-regulated neutrino-cooled state with properties in good agreement with initial conditions of previous work \cite{siegel2017ThreeDimensional,siegel2018Threedimensional,li2021Neutrino}. We conclude that upon collapse of the remnant and its neutrino irradiation, lanthanide-bearing outflows of $\gtrsim0.05\,M_\odot$ ($\sim\!30\%$ of the remaining disk mass \cite{siegel2017ThreeDimensional,siegel2018Threedimensional,de2021Igniting,fujibayashi2020postmerger,fernandez2019Longterm,christie2019role}) consistent with the red kilonova emission of GW170817 are generated over the subsequent few hundred milliseconds \cite{siegel2017ThreeDimensional,siegel2018Threedimensional,li2021Neutrino,fernandez2019Longterm}.

The rapid and self-consistent emergence of a weakly magnetized ($\sigma \sim 0.1$), mildly relativistic ($v\lesssim\!0.6c$) wind from a merger remnant reported here leads to a $\sim\!{\rm hr}$ UV/blue kilonova signal that can be degenerate with the kilonova precursor signal from free neutron decay in the fast tail of dynamical merger ejecta. This novel precursor provides an additional discriminant to distinguish between BNS and NS--black-hole mergers and highlights the importance of early UV and optical follow-up observations of future merger events. The successful break-out of the jet from the surrounding merger debris found here may have additional non-thermal emission signatures.

The results also suggest a novel formation mechanism for the central engine of short gamma-ray bursts. Upon collapse of the remnant, the magnetic jet `seeds' the black hole with magnetic flux and forms a strongly magnetized ($\sigma=L_{\rm EM}/\dot{M} \gg 1$), highly-relativistic jet, owing to the absence of baryon loading from the stellar wind. The emergence of a jet structure from small-scale turbulent fields in the final BH--disk configuration of binary systems with long-lived remnants has so far remained elusive in numerical simulations. The results here suggest that jets from the remnant could `seed' gamma-ray burst jets for remnant lifetimes of $\gtrsim\,30$\,ms. %Supplemental Material for this Letter is available.

%%%%%%%%%%%%%%%%%%%%%%%%%%%%%%%%%%%%%%%%%
%% ACKNOWLEDGEMENTS 
%%%%%%%%%%%%%%%%%%%%%%%%%%%%%%%%%%%%%%%%%

\acknowledgments

The authors thank B.~Metzger and D.~Desai for discussions and comments, and W.~Kastaun for providing visualization tools including the \texttt{PyCactus} package (\href{https://github.com/wokast/PyCactus}{https://github.com/wokast/PyCactus}). This research was enabled in part by support provided by SciNet (www.scinethpc.ca) and Compute Canada (www.computecanada.ca). LC is a CITA National fellow and acknowledges the support by the Natural Sciences and Engineering Research Council of Canada (NSERC), funding reference DIS-2022-568580. DMS acknowledges the support of the Natural Sciences and Engineering Research Council of Canada (NSERC), funding reference number RGPIN-2019-04684. Research at Perimeter Institute is supported in part by the Government of Canada through the Department of Innovation, Science and Economic Development Canada and by the Province of Ontario through the Ministry of Colleges and Universities.


\section{Supplementary Material}
\subsection{Magnetic field evolution in the remnant and the accretion disk}

Figure \ref{fig-avg-Bs} illustrates the evolution of the poloidal and toroidal magnetic field components after the merger ($t=0$). During the first $\approx\!2$\,ms post-merger, small-scale turbulence due to the Kelvin-Helmholtz instability (KHI) exponentially amplifies the average toroidal field strength from its initial value of essentially zero G to $5\times 10^{15}$\,G and the average poloidal field strength by at least an order of magnitude to $\approx\!10^{16}$\,G. Whereas the poloidal component has roughly saturated, amplification by large-scale eddies over the subsequent $\approx\!20$\,ms mainly acts on the toroidal component. Magnetic winding starts do dominate magnetic field growth of the average toroidal component starting at $t\approx 25$\,ms. The associated inverse turbulent cascade converts small-scale turbulent fields into large-scale structures that eventually break out of the star (Fig.~\ref{fig-B-structure} and Fig.~1 of the main text).

\begin{figure}[tb]
  \centering
  \includegraphics[width=\columnwidth]{avg_BtBp_vs_t.pdf}
  \caption{Post-merger evolution of volume averaged poloidal (P) and toroidal (T) magnetic field within the remnant neutron star (similar to Fig.~1 of the main text).}
  \label{fig-avg-Bs}
\end{figure}

\begin{figure}[tb!]
  \centering
\includegraphics[width=1\columnwidth]{omega_apr.pdf}
  \caption{Azimuthally averaged angular velocity of the remnant neutron star and surrounding accretion disk in the orbital plane at different instances in time (between 11 ms (blue) and 60 \,ms (red)), showing the emergence of a slowly rotating core and a nearly Keplerian envelope with a smooth transition into a rotationally supported accretion disk.}
  \label{fig-omega}
\end{figure}

Initially, post-KHI amplification of the average toroidal field mainly proceeds in the core ($r\lesssim 8$ km, where ${\rm d}\Omega/{\rm d}r > 0$; Fig.~\ref{fig-omega}), and only later spreads into the nearly-Keplerian envelope of the star ($r\gtrsim 8$ km, where ${\rm d}\Omega/{\rm d}r <0$; see Fig.~\ref{fig-omega}). As the `wound-up' toroidal magnetic field becomes buoyant, it rises toward the stellar surface and eventually breaks out of the star to form a magnetic tower structure resulting in twin polar jets (cf.~upper panel of Fig.~\ref{fig-B-structure}). The large-scale structure of the poloidal and toroidal fields after break-out is shown in the lower panel of Fig.~\ref{fig-B-structure}.  The poloidal field in the jet core helps in stabilizing the jet structure against kink instabilities. The jet head propagates at mildly relativistic speeds through the envelope of previously ejected material (dynamical ejecta as discussed in Ref.~\cite{combi2023grmhd} and post-merger disk ejecta) and successfully breaks out from this envelope around $t\approx 50$\,ms. Figure \ref{fig-e-rho} shows the jet once it has successfully drilled through the ejecta structure.

\begin{figure}[tb!]
  \centering
  \includegraphics[width=\columnwidth]{avg_Btz_rm8_vs_t.pdf}  
  \includegraphics[width=\columnwidth]{double_bt_bp.pdf}
  \caption{Upper panel: Toroidal magnetic field as a function of height $z$ above the orbital plane, radially averaged within a cylindrical radius $\varpi_{\rm cyl} \le 12$\,km, at early ($t=9$\,ms; blue lines) to late ($t=60$\,ms; red lines) times, with a frequency of $\approx\!1.5$\,ms. The time sequence shows the break-out of the toroidal structures from the stellar surface and the emergence of a $\propto z^{-1}$ magnetic tower structure above the stellar surface ($z\approx 10$\,km). Lower panel: large-scale view of the toroidal (upper half of domain) and poloidal (lower half of domain) magnetic field in the meridional plane at $\approx\!55$\,ms post-merger, showing a mildly relativistic, moderately magnetized ($\sigma \sim 1$) jet structure.}
  \label{fig-B-structure}
\end{figure}

\begin{figure}[tb!]
  \centering
  \includegraphics[width=\columnwidth]{double_e_rho.pdf}
  \caption{Jet structure in terms of the specific internal energy (top) once it has successfully broken out of the ejecta envelope (represented by rest-mass density; bottom).}
  \label{fig-e-rho}
\end{figure}

Magnetic field amplification in the accretion disk that forms from bound merger debris upon circularization around the remnant neutron star proceeds via turbulent amplification by the magnetorotational instability (MRI). The MRI is well resolved by our numerical setup throughout the MRI-unstable part of the computational domain (${\rm d}\Omega/{\rm d}r < 0$), with typically $\gtrsim 10-100$ grid points per fastest-growing unstable MRI mode (see lower panel of Fig.~\ref{fig-trphi-lambda}). MRI-driven MHD turbulence and associated MHD dynamo activity emerge $\approx\!20$\,ms post-merger once the accretion stream has circularized (Fig.~\ref{fig-dynamo}); it erodes the spiral wave patterns, drives radial spreading of the accretion disk and greatly enhances outflows from the system (see below; Fig.~\ref{fig-outflow}). Magnetic stresses (Maxwell stress) associated with MRI-driven turbulence in the disk and the expulsion of magnetic fields from the remnant become comparable to or larger than the total hydrodynamic stresses (Reynolds stress and advective stresses; upper panel of Fig.~\ref{fig-trphi-lambda}) at about 30\,ms, which leads to strong radial spreading and reconfiguration of the accretion disk within only $10-15$\,ms. Fully developed, steady-state MHD turbulence in the disk and an associated dynamo with a cycle of a few ms emerge by $\approx\!40$\,ms as illustrated by the emerging `butterfly' pattern in Fig.~\ref{fig-dynamo}.

\begin{figure}[ht!]
  \centering
  \includegraphics[width=1.01\columnwidth]{butterfly.png}
  \caption{Spacetime diagram of the $x$-component (azimuthal/toroidal component) of the magnetic field in the Eulerian frame, radially averaged between 25 and 60\,km from the rotation axis in the meridional ($yz$) plane, as a function of height $z$ relative to the disk midplane. A stationary dynamo and strongly enhanced mass outflows emerge at around 40\,ms as indicated by the `butterfly' pattern.}
  \label{fig-dynamo}
\end{figure}

\begin{figure}[tb!]
  \centering
  \includegraphics[width=\columnwidth]{trphi_mri.pdf}
  \caption{Upper panel: relative strength of total hydrodynamic stress (Reynolds and advective stresses) and magnetic stresses (Maxwell stress) acting on the post-merger plasma at $t\approx 50$\,ms. Lower panel: number of grid points per wavelength $\lambda_{\rm MRI}$ of the fastest-growing unstable MRI mode at the same time instance, indicating that the MRI is well resolved.}
  \label{fig-trphi-lambda}
\end{figure}

\subsection{Comparison with previous work including magnetic fields in the post-merger phase}

The post-merger evolution of systems with (meta-)stable remnant neutron stars as obtained from numerical simulations is sensitive to the microphysics included in the simulations. The inclusion of weak interactions allows the plasma to cool via neutrino emission and thus significantly reduces baryon pollution in the vicinity of the merger remnant immediately after merger. Due to fallback flows becoming less `puffy', neutrino cooling enables the formation of a massive accretion disk around the remnant. Previous work without weak interactions found hot magnetized material surrounding the remnant, which inflates rapidly \citep{ciolfi2017General,ciolfi2020Collimated, ciolfi2019First}. The emergence of a large-scale toroidal field then spreads quasi-isotropically, forming a `magnetic bubble' that expands and drives low-velocity winds to large scales \citep{siegel2014Magnetically,ciolfi2020Magnetically}. Here, neutrino cooling helps to reduce baryon pollution in polar regions and neutrino absorption overcomes the ram pressure of low-angular momentum fallback material in the polar `funnel' and generates a neutrino-driven outflow in polar directions. 
Neutrino absorption and high magnetic pressure help to establish and stabilize a magnetic tower structure. In this wind environment, the jet head is able to propagate and to successfully break out of the merger debris.

Ref.~\cite{mosta2020Magnetar} place a large-scale dipolar field structure onto a neutron-star merger remnant and its surroundings, obtained by purely hydrodynamical merger evolution 17\,ms after merger. The authors find that this global magnetic field structure rapidly leads to the formation of a polar jet structure somewhat similar to that obtained in the present paper. Here, the emergence of twin polar jets is obtained self-consistently by turbulent amplification of an initially vanishing toroidal magnetic field pre-merger and magnetic winding without ad-hoc prescriptions for the magnetic field post-merger. Previous work found a timescale for the emergence of global magnetic field structures and associated winds to require $\gtrsim\!170$\,ms \cite{ciolfi2020Collimated}. Here, we demonstrate by including all relevant microphysical processes that twin jets can emerge by only a few tens of ms post-merger.

%\cite{mosta2020Magnetar,curtis2021Process}

%field manage to pull through the evacuated funnel creating a collimated structure. The jet velocity is considerably larger than magnetic winds found in simulations without weak interactions (e.g., \cite{ciolfi2020Magnetically}), but more similar to those of Ref.~\cite{mosta2020Magnetar}, which includes neutrino transport, but assumes a large-scale poloidal magnetic field in the post-merger.

\subsection{Global outflow properties}

Spiral waves driven by non-axisymmetric modes in the remnant neutron star, similar to those observed in previous hydrodynamic simulations \cite{nedora2019Spiralwave,nedora2021Dynamical}, are imprinted in and are associated with the dominant mass outflow during the first $\approx\!30$\,ms post-merger (Fig.~\ref{fig-outflow} and Fig.~5 of the main text). %can be clearly observed in the spacetime plot of Fig.\ref{fig-outflow} measured at radius of 300 km. 
Once the accretion disk spreads radially due to MHD effects and a steady-state dynamo emerges (see above), the outflow enhances drastically (around $t\approx 35$\,ms), quickly dominating the total cumulative mass ejection. Furthermore, the plasma outflow becomes more spherical in this latter phase (cf.~Fig.~\ref{fig-outflow}). %enhancement of mass outflows when the disk settles can also be seen at later times.

Roughly 95\% of the total cumulative ejected mass of $\gtrsim\!2\times 10^{-2}M_\odot$ by $t\gtrsim 55$\,ms is carried by disk winds (Fig.~\ref{fig-masses}), which have a narrow velocity profile of $v\approx (0.05-0.2)c$ (cf.~Fig.~6 of the main text). The kinetic energy of the outflow, however, is dominated by the polar component including the jet (polar angle $\theta \lesssim 30^\circ$), which extends to asymptotic velocities of $v \gtrsim 0.6c$ (see Fig.~6 of the main text). Only about 1\% of the total outflows originate from within the jet core ($\theta \lesssim 20^\circ$).
 
\begin{figure}[tb!]
  \centering
  \includegraphics[width=\columnwidth]{mdot_phth.png}
  \caption{Spacetime diagrams of the mass outflow through a spherical shell with radius 300\,km averaged in the azimuthal (left) and polar (right) direction as a function of time and polar and azimuthal angle, respectively.}
  \label{fig-outflow}
\end{figure}

\begin{figure}[tb!]
  \centering
\includegraphics[width=1\columnwidth]{masses_vs_t.pdf}
  \caption{Total (dynamical + post-merger) cumulative ejected mass within the polar regions including the jet ($\theta \lesssim 30^\circ$; green solid line) and total solid angle (jet and disk outflows; red solid line) as a function of time.}
  \label{fig-masses}
\end{figure}


\subsection{Polar outflow}

%Figure \ref{fig-energies} shows the total rotational, internal, and magnetic energy within a box of side length 200\,km. The magnetic energy saturates at $\approx\!10^{51}$\,erg at about 35--40\,ms after merger, when a quasi-stationary jet and accretion disk have been established. The total internal energy remains approximately constant during our simulation run, while the rotational energy slowly decreases, following a linear trend. Indeed, rotational energy is tapped by the magnetic field and the non-axisymmetric density modes to release the magnetized jet and the spiral wave winds.



%\begin{figure}[tb!]
%  \centering
%  \includegraphics[width=\columnwidth]{Emag_vs_t.pdf}
%  \caption{Internal, magnetic, and rotational evolution contained in a box of radius 100 km.}
%  \label{fig-energies}
%\end{figure}

Figure \ref{fig-lum} reports several luminosities in the polar region where outflows dominate over accretion in the immediate vicinity of the stellar surface ($\theta \lesssim 30^{\circ}$). Prior to the jet emergence $t\lesssim 30$\,ms, the kinetic power of the outflow roughly equals the total power absorbed by neutrinos, as expected for a neutrino-driven wind. As the jet emerges the kinetic power rises by more than an order of magnitude and approaches the Poynting luminosity of the emergent jet, which extracts rotational energy from the rotating remnant neutron star.

Figure \ref{fig-magwind-profiles} shows rest-mass density profiles along the jet axis at various epochs. Prior to and after the emergence of the jet a $\rho \propto r^{-2}$ wind profile is established as expected from mass conservation for a steady state wind with mass-loss rate and velocity set in the gain region close to the neutron-star surface (see the main text).

The magnetized polar wind with enhanced mass-loss rate of $\dot{M}\approx 1 \times 10^{-2}M_\odot\,{\rm s}^{-1}$ (enhanced by approximately one order of magnitude relative to the prior purely neutrino-driven wind), a poloidal field strength of $\sim\!{\rm few}\times 10^{14}$\,G (Fig.~\ref{fig-B-structure}), a neutrino luminosity of $L_\nu \sim {\rm few} \times 10^{52}$\,erg\,s$^{-1}$, mass-averaged speed $\langle u \rangle \approx c \sigma^{1/3}$ ($\sigma \approx 0.1$; see the main text), and mass-averaged $Y_e\approx 0.3-0.4$ is in broad agreement with the 1D wind solutions of Ref.~\cite{metzger2018Magnetar}.


\begin{figure}[tb!]
  \centering
\includegraphics[width=1\columnwidth]{lum_vs_t.pdf}
  \caption{Various luminosities extracted in the polar region $\theta \le 30^\circ$. Shown are the electromagnetic (Poynting) luminosity $L_{\rm EM}$ and the kinetic power $\dot{E}_{\rm k}$ extracted at a radius of 25\,km, as well as the total absorbed neutrino power $\dot{Q}_{\rm net}$ in the corresponding gain layer (a volume between $r=6$\,km and $r=40$\,km). %Dashed lines indicate a linear  extrapolation at early times where we do not have available data for this quantity.
  }
  \label{fig-lum}
\end{figure}

\begin{figure}[tb!]
  \centering
   \includegraphics[width=\columnwidth]{avg_rhoz_rm8_vs_t.pdf}
  \caption{Rest-mass density profiles along the rotational (jet) axis, obtained by averaging over a cylindrical volume $\varpi_{\rm cyl} < 12$\,km, for different times with a frequency of $\approx 1.5$ ms. The blue (early times; neutrino-driven wind) to red (late times; neutrino and magnetically driven outflow) solid lines show the emergence of a stationary ($\propto z^{-2}$) wind profile in both wind regimes. The associated mass-loss rate increases by an order of magnitude as the wind becomes strongly magnetized.}
  \label{fig-magwind-profiles}
\end{figure}

\subsection{Kilonova light curves from polar and disk winds}

Figure~\ref{fig-kn-angle} illustrates the contributions of polar outflows versus disk outflows to the kilonova emission. When observed near the polar axis, the fast polar outflows associated with the jet can greatly boost the emission by an order of magnitude in the UV and blue bands on timescales of a few hours after merger. Dissipation of magnetic energy into heat not considered here may further enhance the emission. Also shown are the kilonova precursor signals due to free neutron decay in fast dynamical ejecta computed in CS23. The polar outflow component peaks somewhat later than the $\lesssim\!1$\,h peak timescale of the neutron precursor, but with similar or higher magnitudes than the neutron precursor. Both emission components strongly overlap in time and create a prolonged precursor signal $\lesssim\!1-\text{few}$\,h.

For calculating our light curves, we use the geometrical multi-angle axisymmetric kilonova approach with the detailed %of Ref. \cite{perego2017AT2017gfo}, combined with the more realistic 
heating rates of Ref. \cite{hotokezaka2020Radioactive} as implemented in CS23. We calculate the $\beta$-decay, $\alpha$-decay, and fission heating rates taking into account elements up to $A=135$ and abundance distributions consistent with our nuclear reaction network calculations. We use opacity values consistent with the ones obtained in Ref.~\cite{banerjee2020simulations} for the early stages ($\approx\!0.1-1$\,day) of a lanthanide-free kilonova. In particular, as a minimal assumption, we use $\kappa = 0.5$\,cm$^2$g$^{-1}$ for $v > 0.2c$, corresponding to the jet component, and $\kappa = 10$\,cm$^2$g$^{-1}$ for $v < 0.2$, corresponding to the disk wind. We have also checked that our results are largely insensitive to using a more detailed, $Y_e$-binned opacity prescription as employed in CS23. 

\begin{figure}[tb!]
  \centering
\includegraphics[width=1\columnwidth]{aprpostmerger_angles.pdf}
  \caption{Kilonova light curves in the UV/B bands generated by the post-merger ejecta, viewed along the polar axis (thick lines) and the equatorial plane (dashed lines) at a distance of 40\,Mpc. Also shown for comparison is the kilonova precursor signal due to free neutron decay in fast dynamical ejecta (dot-dashed lines) as computed for this merger setup in CS23.}
  \label{fig-kn-angle}
\end{figure}



\bibliographystyle{apsrev4-2}
\bibliography{ms}% Produces the bibliography via BibTeX.

\end{document}
%
% ****** End of file apssamp.tex ******