% ****** Start of file apssamp.tex ******
%
%   This file is part of the APS files in the REVTeX 4.2 distribution.
%   Version 4.2a of REVTeX, December 2014
%
%   Copyright (c) 2014 The American Physical Society.
%
%   See the REVTeX 4 README file for restrictions and more information.
%
% TeX'ing this file requires that you have AMS-LaTeX 2.0 installed
% as well as the rest of the prerequisites for REVTeX 4.2
%
% See the REVTeX 4 README file
% It also requires running BibTeX. The commands are as follows:
%
%  1)  latex apssamp.tex
%  2)  bibtex apssamp
%  3)  latex apssamp.tex
%  4)  latex apssamp.tex
%

\documentclass[%
 reprint,
%superscriptaddress,
%groupedaddress,
%unsortedaddress,
%runinaddress,
%frontmatterverbose, 
%preprint,
%preprintnumbers,
%nofootinbib,
%nobibnotes,
%bibnotes,
 amsmath,amssymb,
 aps,
 prl,
%pra,
%prb,
%rmp,
%prstab,
%prstper,
%floatfix,
]{revtex4-2}

\usepackage{aas_macros}
\usepackage{graphicx}% Include figure files
\usepackage{dcolumn}% Align table columns on decimal point
\usepackage{bm}% bold math
\usepackage{hyperref}% add hypertext capabilities
\usepackage{color}
\usepackage{natbib}
\usepackage{CJK}
%\usepackage[mathlines]{lineno}% Enable numbering of text and display math
%\linenumbers\relax % Commence numbering lines

\usepackage[english]{babel}

%\usepackage[showframe,%Uncomment any one of the following lines to test 
%%scale=0.7, marginratio={1:1, 2:3}, ignoreall,% default settings
%%text={7in,10in},centering,
%%margin=1.5in,
%%total={6.5in,8.75in}, top=1.2in, left=0.9in, includefoot,
%%height=10in,a5paper,hmargin={3cm,0.8in},
%]{geometry}


\newcommand{\epja}{Eur.~Phys.~J.~A}
\newcommand{\prx}{Phys.~Rev.~X}
\newcommand{\rmph}{Rev.~Mod.~Phys.~}
\newcommand{\cqg}{{Class.~Quant.~Grav.}~}
%\newcommand{\apj}{{Astrophys.~J.}}
%\newcommand{\apjl}{{Astrophys. J. Lett.}}
%\newcommand{\apjs}{{Astrophys. J. Suppl. Ser.}}
\newcommand{\jcph}{{J.~Comput.~Phys.}}
\newcommand{\jcapp}{{J.~Cosmol.~Astropart.~Phys.}}
%\newcommand{\physrep}{{Phys. Rep.}}
%\newcommand{\mnras}{{Mon. Not. R. Astron. Soc.}}
%\newcommand{\aap}{Astron. Astrophys.}
%\newcommand{\pasp}{Pub. Astron. Soc. Pac.}
\newcommand{\pthphys}{Prog.~Theor.~Phys.}
\newcommand{\lrr}{Living.~Rev.~Relativ.}

\newcommand{\xl}{\textbf{red}}
\newcommand{\dms}[1]{\textcolor{red}{#1}}
\newcommand{\luc}[1]{\textcolor{blue}{[[#1]]}}
\newcommand{\rvw}[1]{\textbf{#1}}


\begin{document}
\begin{CJK*}{UTF8}{gbsn} % Use default fonts from CJK (see below)
\preprint{APS/123-QED}

\title{Supplementary Material:\\
Jets from neutron-star merger remnants and massive blue kilonovae}

\author{Luciano Combi$^{1,2,3}$}
\altaffiliation{CITA National Fellow}

\author{Daniel M.~Siegel$^{4,1,2}$}

\affiliation{$^1$Perimeter Institute for Theoretical Physics, Waterloo, Ontario N2L 2Y5, Canada\\
$^2$Department of Physics, University of Guelph, Guelph, Ontario N1G 2W1, Canada\\
$^3$Instituto Argentino de Radioastronom\'ia (IAR, CCT La Plata, CONICET/CIC)\\
C.C.5, (1984) Villa Elisa, Buenos Aires, Argentina\\
$^4$Institute of Physics, University of Greifswald, D-17489 Greifswald, Germany}

%\collaboration{MUSO Collaboration}
%\noaffiliation

%\date{\today}% It is always \today, today,
             %  but any date may be explicitly specified



%\keywords{Suggested keywords}%Use showkeys class option if keyword
\maketitle
\end{CJK*}

%\section{Supplemental Material}
%\section{Supplementary Material}
\subsection{Magnetic field evolution in the remnant and the accretion disk}

Figure \ref{fig-avg-Bs} illustrates the evolution of the poloidal and toroidal magnetic field components after the merger ($t=0$). During the first $\approx\!2$\,ms post-merger, small-scale turbulence due to the Kelvin-Helmholtz instability (KHI) exponentially amplifies the average toroidal field strength from its initial value of essentially zero G to $5\times 10^{15}$\,G and the average poloidal field strength by at least an order of magnitude to $\approx\!10^{16}$\,G. Whereas the poloidal component has roughly saturated, amplification by large-scale eddies over the subsequent $\approx\!20$\,ms mainly acts on the toroidal component. Magnetic winding starts do dominate magnetic field growth of the average toroidal component starting at $t\approx 25$\,ms. The associated inverse turbulent cascade converts small-scale turbulent fields into large-scale structures that eventually break out of the \rvw{remnant NS} (Fig.~\ref{fig-B-structure} and Fig.~1 of the main text).

\begin{figure}[tb]
  \centering
  \includegraphics[width=\columnwidth]{avg_BtBp_vs_t.pdf}
  \caption{Post-merger evolution of volume averaged poloidal (P) and toroidal (T) magnetic field within the remnant NS (similar to Fig.~1 of the main text).}
  \label{fig-avg-Bs}
\end{figure}

\begin{figure}[tb!]
  \centering
\includegraphics[width=1\columnwidth]{omega_apr.pdf}
  \caption{Azimuthally averaged angular velocity of the remnant NS and surrounding accretion disk in the orbital plane at different instances in time (between 11 ms (blue) and 60 \,ms (red)), showing the emergence of a slowly rotating core and a nearly Keplerian envelope with a smooth transition into a rotationally supported accretion disk.}
  \label{fig-omega}
\end{figure}

Initially, post-KHI amplification of the average toroidal field mainly proceeds in the core ($r\lesssim 8$ km, where ${\rm d}\Omega/{\rm d}r > 0$; Fig.~\ref{fig-omega}), and only later spreads into the nearly-Keplerian envelope of the \rvw{remnant NS} ($r\gtrsim 8$ km, where ${\rm d}\Omega/{\rm d}r <0$; see Fig.~\ref{fig-omega}). As the `wound-up' toroidal magnetic field becomes buoyant, it rises toward the stellar surface and eventually breaks out of the \rvw{remnant NS} to form a magnetic tower structure resulting in twin polar jets (cf.~Fig.~\ref{fig-B-structure}). The large-scale structure of the poloidal and toroidal fields after break-out is shown in the lower panel of Fig.~\ref{fig-B-structure}.  The poloidal field in the jet core helps in stabilizing the jet structure against kink instabilities. The jet head propagates at mildly relativistic speeds through the envelope of previously ejected material (dynamical ejecta as discussed in Ref.~\cite{combi2023grmhd} and post-merger disk ejecta) and successfully breaks out from this envelope around $t\approx 50$\,ms. Figure \ref{fig-e-rho} shows the jet once it has successfully drilled through the ejecta structure.

\begin{figure}[tb!]
  \centering
  \includegraphics[width=\columnwidth]{avg_Btz_rm8_vs_t.pdf}  
  \includegraphics[width=\columnwidth]{double_bt_bp.pdf}
  \caption{Upper panel: Toroidal magnetic field as a function of height $z$ above the orbital plane, radially averaged within a cylindrical radius $\varpi_{\rm cyl} \le 12$\,km, at early ($t=9$\,ms; blue lines) to late ($t=60$\,ms; red lines) times, with a frequency of $\approx\!1.5$\,ms. The time sequence shows the break-out of the toroidal structures from the stellar surface and the emergence of a $\propto z^{-1}$ magnetic tower structure above the stellar surface ($z\approx 10$\,km). Lower panel: large-scale view of the toroidal (upper half of domain) and poloidal (lower half of domain) magnetic field in the meridional plane at $\approx\!55$\,ms post-merger, showing a mildly relativistic, moderately magnetized ($\sigma \sim 1$) jet structure.}
  \label{fig-B-structure}
\end{figure}

\begin{figure}[tb!]
  \centering
  \includegraphics[width=\columnwidth]{double_rho_hut.pdf}
  \caption{Jet structure in terms of the specific energy (bottom) once it has successfully broken out of the bulk ejecta envelope (represented by rest-mass density; top).}
  \label{fig-e-rho}
\end{figure}

Magnetic field amplification in the accretion disk that forms from bound merger debris upon circularization around the remnant NS proceeds via turbulent amplification by the magnetorotational instability (MRI). The MRI is well resolved by our numerical setup throughout the MRI-unstable part of the computational domain (${\rm d}\Omega/{\rm d}r < 0$), with typically $\gtrsim 10-100$ grid points per fastest-growing unstable MRI mode (see lower panel of Fig.~\ref{fig-trphi-lambda}). MRI-driven MHD turbulence and associated MHD dynamo activity emerge $\approx\!20$\,ms post-merger once the accretion stream has circularized (Fig.~\ref{fig-dynamo}); it erodes the spiral wave patterns, drives radial spreading of the accretion disk and greatly enhances outflows from the system (see below; Fig.~\ref{fig-outflow}). Magnetic stresses (Maxwell stress) associated with MRI-driven turbulence in the disk and the expulsion of magnetic fields from the remnant NS become comparable to or larger than the total hydrodynamic stresses (Reynolds stress and advective stresses; upper panel of Fig.~\ref{fig-trphi-lambda}) at about 30\,ms, which leads to strong radial spreading and reconfiguration of the accretion disk within only $10-15$\,ms. Fully developed, steady-state MHD turbulence in the disk and an associated dynamo with a cycle of a few ms emerge by $\approx\!40$\,ms as illustrated by the emerging `butterfly' pattern in Fig.~\ref{fig-dynamo}. Figure \ref{fig-ye-eta} shows that the disk then also enters a self-regulated state characterized by moderate electron degeneracy $\eta_e =\mu_e/k_{\rm B}T \sim 1$ and corresponding low $Y_e\approx 0.1-0.15$ \cite{beloborodov2003nuclear,chen2007Neutrinocooled,siegel2017ThreeDimensional,siegel2018Threedimensional}.

\begin{figure}[ht!]
  \centering
  \includegraphics[width=1.01\columnwidth]{butterfly.png}
  \caption{Spacetime diagram of the $x$-component (azimuthal/toroidal component) of the magnetic field in the Eulerian frame, radially averaged between 25 and 60\,km from the rotation axis in the meridional ($yz$) plane, as a function of height $z$ relative to the disk midplane. A stationary dynamo and strongly enhanced mass outflows emerge at around 40\,ms as indicated by the `butterfly' pattern.}
  \label{fig-dynamo}
\end{figure}

\begin{figure}[tb!]
  \centering
  \includegraphics[width=\columnwidth]{trphi_mri.pdf}
  \caption{Upper panel: relative strength of total hydrodynamic stress (Reynolds and advective stresses) and magnetic stresses (Maxwell stress) acting on the post-merger plasma at $t\approx 50$\,ms. Lower panel: number of grid points per wavelength $\lambda_{\rm MRI}$ of the fastest-growing unstable MRI mode at the same time instance, indicating that the MRI is well resolved.}
  \label{fig-trphi-lambda}
\end{figure}

\begin{figure}[tb]
  \centering
\includegraphics[width=\columnwidth]{double_ye_etae_late.pdf}
  \caption{Meridional snapshot along the rotational axis showing the electron fraction (top) and electron degeneracy $\eta=\mu_e/k_{\rm B} T$ (bottom) with density contours at $\rho = [10^{7.5},10^{8},10^{9.75},10^{13}]\,{\rm g}\,{\rm cm}^{-3}$ as yellow, black, purple, and white solid lines, respectively, $\approx\!50$\,ms post-merger. %}
  The accretion disk is in a self-regulated state of moderate degeneracy ($\eta\sim 1$), which implies high neutron-richness ($Y_e\approx 0.15$).} % Strong neutrino irradiation from the remnant (cf.~Fig.~\ref{fig-rho-betaqnet}) protonizes the highly neutron-rich disk winds to $Y_e > 0.2$.}
  \label{fig-ye-eta}
\end{figure}


\subsection{Comparison with other work}

The post-merger evolution of systems with (meta-)stable remnant NSs as obtained from numerical simulations is sensitive to the microphysics included in the simulations. The inclusion of weak interactions allows the plasma to cool via neutrino emission and thus significantly reduces baryon pollution in the vicinity of the merger remnant immediately after merger. Due to fallback flows becoming less `puffy', neutrino cooling enables the formation of a massive accretion disk around the remnant. Previous work without weak interactions found hot magnetized material surrounding the remnant NS, which inflates rapidly \citep{ciolfi2017General,ciolfi2020Collimated, ciolfi2019First}. The emergence of a large-scale toroidal field then spreads quasi-isotropically, forming a `magnetic bubble' that expands and drives low-velocity winds to large scales \citep{ciolfi2020Magnetically}. Here, neutrino cooling helps to reduce baryon pollution in polar regions and neutrino absorption overcomes the ram pressure of low-angular momentum fallback material in the polar `funnel' and generates a neutrino-driven outflow in polar directions. 
Neutrino absorption and high magnetic pressure help to establish and stabilize a magnetic tower structure. In this wind environment, the jet head is able to propagate and to successfully break out of the merger debris.

\rvw{Ref.~\cite{ciolfi2020Collimated} reports the formation of a collimated outflow powered by a remnant NS, which requires $\gtrsim\!170$\,ms to emerge. Here, we demonstrate by including all relevant microphysical processes that twin incipient jets can emerge by only a few tens of ms post-merger. Furthermore, Ref.~\cite{ciolfi2020Collimated} finds outflows solely in polar directions starting $\gtrsim\!170$\,ms, and they differ quite significantly from our simulations: These outflows are slow with typical velocities of $0.2$c, more similar to our disk outflows, they become unbound only $\gtrsim\!500$\,km above the engine, and they are only loosely collimated as a result of intense baryon pollution.}

Ref.~\cite{mosta2020Magnetar}, place a large-scale dipolar field structure onto a BNS merger remnant and its surroundings (obtained by purely hydrodynamical merger evolution) 17\,ms after merger. The authors find that this imposed global magnetic field structure rapidly leads to the formation of a polar jet-like structure, somewhat similar to that obtained in the present paper. Here, the emergence of twin polar jets is obtained self-consistently by turbulent amplification of an initially vanishing toroidal magnetic field pre-merger and magnetic winding without ad-hoc prescriptions for the magnetic field post-merger. \rvw{After submission of this manuscript, Ref.~\cite{curtis2023outflows} followed the same approach as in Refs.~\cite{curtis2023r, mosta2020Magnetar}, but using a more advanced M1 neutrino transport scheme \cite{radice2021New} instead of a leakage based cooling and heating scheme, finding higher-$Y_e$ outflows and strongly suppressed lanthanides. A central difference of Refs.~\cite{mosta2020Magnetar, curtis2023outflows} to our work is that the former claim polar outflows within the jet to be dominant and, over much longer time spans of hundreds of ms, be consistent with blue kilonovae such as GW170817. The reason for apparent absence of inevitable equatorial outflows due to neutrino-driven winds from the remnant NS and accretion disk winds remains unclear.}

\rvw{At the time of submission of this paper, Ref.~\cite{most2023flares} presented GRMHD simulations with neutrino cooling of BNS mergers using a subgrid model for an $\alpha$-dynamo, which, by construction and depending on the assumed dynamo parameter, can generate an ultra-strong \emph{dominantly poloidal} field of $\gtrsim\!10^{17}$\,G within a few milliseconds after the merger. Strongly amplified magnetic flux structures become buoyant and break out of the remnant NS. Magnetic flux loops anchored to the remnant's surface twist, reconnect, and power flares due to the fact that the remnant rotates differentially. Eventually, a collimated, magnetically dominated outflow forms in the polar directions. %The authors explore whether the associated Poynting luminosity shows similar periodicities to the initial oscillations of the remnant NS. Ref.~\cite{chiarenti} speculates that such oscillations could be responsible for periodicities that they claim are present in the prompt emission of some short GRBs. 
In our simulations, the generation of a jet is based on self-consistent turbulent amplification and winding of a \emph{dominantly toroidal} magnetic field.
%toroidal field is instead amplified by winding, and so it takes a few tens of milliseconds to reach high mangnetizations to break through and form the outflow. Although the velocities are comparable to our simulation, we do not find any periodicities as expected, since the remnant oscillations already died out. Moreover, we do not see flares because of baryon background neutrino driven wind.
}

\rvw{After submission of this paper, Ref.~\cite{aguilera2023role} find with very-high resolution ($\Delta x = 60$\,m) GRMHD simulations using a Large-Eddy subgrid model a dominantly toroidal, turbulently amplified, small-scale magnetic field post-merger, which becomes more coherent due to magnetic winding around $\sim\!30$\,ms post-merger, similar to what we find here. One may speculate that the fact that a large-scale helical, loosely collimated structure emerges from the remnant NS only at late times ($\gtrsim\!100$\,ms post-merger) may be related to the absence of weak interactions and realistic EOSs (cf.~the discussion above related to Ref.~\cite{ciolfi2020Collimated}).}

\rvw{Finally, after submission of this paper, Ref.~\cite{kiuchi2023large} find with ``extremely-high'' resolution ($\Delta x = 12$\,m) GRMHD simulations including weak interactions that a similar toroidally dominated magnetic field and collimated outflow is produced after merger over similar timescales of tens of milliseconds, broadly validating our results. They attribute the formation of a large-scale field to an MRI-driven dynamo acting within the envelope of the star and the inner accretion disk (similar to our Fig.~\ref{fig-dynamo}), and not to the stellar interior as we find here. However, in this proposed MRI scenario, the actual role of the $\alpha$-$\Omega$ dynamo in generating the large-scale flux observed in the simulation remains somewhat uncertain; it remains also unclear what drives a similar growth of toroidal \textit{and} poloidal fields in their low-resolution run, where this MRI-driven dynamo is not resolved.} %, and whether this is independent of the initial magnetic field strengths.}


%\cite{mosta2020Magnetar,curtis2021Process}

%field manage to pull through the evacuated funnel creating a collimated structure. The jet velocity is considerably larger than magnetic winds found in simulations without weak interactions (e.g., \cite{ciolfi2020Magnetically}), but more similar to those of Ref.~\cite{mosta2020Magnetar}, which includes neutrino transport, but assumes a large-scale poloidal magnetic field in the post-merger.

\subsection{Global outflow properties}

Spiral waves driven by non-axisymmetric modes in the remnant NS, similar to those observed in previous hydrodynamic simulations \cite{nedora2019Spiralwave,nedora2021Dynamical}, are imprinted in and are associated with the dominant mass outflow during the first $\approx\!30$\,ms post-merger (Fig.~\ref{fig-outflow} and Fig.~5 of the main text). %can be clearly observed in the spacetime plot of Fig.\ref{fig-outflow} measured at radius of 300 km. 
Once the accretion disk spreads radially due to MHD effects and a steady-state dynamo emerges (see above), the outflow enhances drastically (around $t\approx 35$\,ms), quickly dominating the total cumulative mass ejection. Furthermore, the plasma outflow becomes more spherical in this latter phase (cf.~Fig.~\ref{fig-outflow}). %enhancement of mass outflows when the disk settles can also be seen at later times.

Roughly 95\% of the total cumulative ejected mass of $\gtrsim\!2\times 10^{-2}M_\odot$ by $t\gtrsim 55$\,ms is carried by disk winds (Fig.~\ref{fig-masses}), which have a narrow velocity profile of $v\approx (0.05-0.2)c$ (cf.~Fig.~6 of the main text). The kinetic energy of the outflow, however, is dominated by the polar component including the jet (polar angle $\theta \lesssim 30^\circ$), which extends to asymptotic velocities of $v \gtrsim 0.6c$ (see Fig.~6 of the main text). Only about 1\% of the total outflows originate from within the jet core ($\theta \lesssim 20^\circ$).
 
\begin{figure}[tb!]
  \centering
  \includegraphics[width=\columnwidth]{mdot_phth.png}
  \caption{Spacetime diagrams of the mass outflow through a spherical shell with radius 300\,km averaged in the azimuthal (left) and polar (right) direction as a function of time and polar and azimuthal angle, respectively.}
  \label{fig-outflow}
\end{figure}

\begin{figure}[tb!]
  \centering
\includegraphics[width=1\columnwidth]{masses_vs_t.pdf}
  \caption{Total (dynamical + post-merger) cumulative ejected mass within the polar regions including the jet ($\theta \lesssim 30^\circ$; green solid line) and total solid angle (jet and disk outflows; red solid line) as a function of time.}
  \label{fig-masses}
\end{figure}


\subsection{Polar outflow}

%Figure \ref{fig-energies} shows the total rotational, internal, and magnetic energy within a box of side length 200\,km. The magnetic energy saturates at $\approx\!10^{51}$\,erg at about 35--40\,ms after merger, when a quasi-stationary jet and accretion disk have been established. The total internal energy remains approximately constant during our simulation run, while the rotational energy slowly decreases, following a linear trend. Indeed, rotational energy is tapped by the magnetic field and the non-axisymmetric density modes to release the magnetized jet and the spiral wave winds.



%\begin{figure}[tb!]
%  \centering
%  \includegraphics[width=\columnwidth]{Emag_vs_t.pdf}
%  \caption{Internal, magnetic, and rotational evolution contained in a box of radius 100 km.}
%  \label{fig-energies}
%\end{figure}

Figure \ref{fig-lum} reports several luminosities in the polar region where outflows dominate over accretion in the immediate vicinity of the stellar surface ($\theta \lesssim 30^{\circ}$). Prior to the jet emergence $t\lesssim 30$\,ms, the kinetic power of the outflow roughly equals the total power absorbed by neutrinos, as expected for a neutrino-driven wind. As the jet emerges the kinetic power rises by more than an order of magnitude and approaches the Poynting luminosity of the emergent jet, which extracts rotational energy from the rotating remnant NS.

Figure \ref{fig-magwind-profiles} shows rest-mass density profiles along the jet axis at various epochs. Prior to and after the emergence of the jet a $\rho \propto r^{-2}$ wind profile is established as expected from mass conservation for a steady state wind with mass-loss rate and velocity set in the gain region close to the NS surface (see the main text).

The magnetized polar wind with enhanced mass-loss rate of $\dot{M}\approx 1 \times 10^{-2}M_\odot\,{\rm s}^{-1}$ (enhanced by approximately one order of magnitude relative to the prior purely neutrino-driven wind), a poloidal field strength of $\sim\!{\rm few}\times 10^{14}$\,G (Fig.~\ref{fig-B-structure}), a neutrino luminosity of $L_\nu \sim {\rm few} \times 10^{52}$\,erg\,s$^{-1}$, mass-averaged speed $\langle u \rangle \approx c \sigma^{1/3}$ ($\sigma \approx 0.1$; see the main text), and mass-averaged $Y_e\approx 0.3-0.4$ is in broad agreement with the 1D wind solutions of Ref.~\cite{metzger2018Magnetar}.


\begin{figure}[tb!]
  \centering
\includegraphics[width=1\columnwidth]{lum_vs_t.pdf}
  \caption{Various luminosities extracted in the polar region $\theta \le 30^\circ$. Shown are the electromagnetic (Poynting) luminosity $L_{\rm EM}$ and the kinetic power $\dot{E}_{\rm k}$ extracted at a radius of 25\,km, as well as the total absorbed neutrino power $\dot{Q}_{\rm net}$ in the corresponding gain layer (a volume between $r=6$\,km and $r=40$\,km). %Dashed lines indicate a linear  extrapolation at early times where we do not have available data for this quantity.
  }
  \label{fig-lum}
\end{figure}

\begin{figure}[tb!]
  \centering
   \includegraphics[width=\columnwidth]{avg_rhoz_rm8_vs_t.pdf}
  \caption{Rest-mass density profiles along the rotational (jet) axis, obtained by averaging over a cylindrical volume $\varpi_{\rm cyl} < 12$\,km, for different times with a frequency of $\approx 1.5$ ms. The blue (early times; neutrino-driven wind) to red (late times; neutrino and magnetically driven outflow) solid lines show the emergence of a stationary ($\propto z^{-2}$) wind profile in both wind regimes. The associated mass-loss rate increases by an order of magnitude as the wind becomes strongly magnetized.}
  \label{fig-magwind-profiles}
\end{figure}

\subsection{Kilonova light curves from polar and disk winds}

Figure~\ref{fig-kn-angle} illustrates the contributions of polar outflows versus disk outflows to the kilonova emission. When observed near the polar axis, the fast polar outflows associated with the jet can greatly boost the emission by an order of magnitude in the UV and blue bands on timescales of a few hours after merger. Dissipation of magnetic energy into heat not considered here may further enhance the emission. Also shown are the kilonova precursor signals due to free neutron decay in fast dynamical ejecta computed in CS23. The polar outflow component peaks somewhat later than the $\lesssim\!1$\,h peak timescale of the neutron precursor, but with similar or higher magnitudes than the neutron precursor. Both emission components strongly overlap in time and create a prolonged precursor signal $\lesssim\!1-\text{few}$\,h.

For calculating our light curves, we use the geometrical multi-angle axisymmetric kilonova approach with the detailed %of Ref. \cite{perego2017AT2017gfo}, combined with the more realistic 
heating rates of Ref. \cite{hotokezaka2020Radioactive} as implemented in CS23. We calculate the $\beta$-decay, $\alpha$-decay, and fission heating rates taking into account elements up to $A=135$ and abundance distributions consistent with our nuclear reaction network calculations. We use opacity values consistent with the ones obtained in Ref.~\cite{banerjee2020simulations} for the early stages ($\approx\!0.1-1$\,day) of a lanthanide-free kilonova. In particular, as a minimal assumption \rvw{and similar to \cite{metzger2015Neutronpowered}}, we use $\kappa = 0.5$\,cm$^2$g$^{-1}$ for $v > 0.2c$, corresponding to the jet component \rvw{with opacity similar to electron-scattering}, and $\kappa = 10$\,cm$^2$g$^{-1}$ for $v < 0.2$, corresponding to the disk wind \rvw{with mostly lanthanide-free, d-block elements}. We have also checked that our results are largely insensitive to using a more detailed, $Y_e$-binned opacity prescription \rvw{(e.g., Ref.~\cite{wu2021Radiation})} as employed by CS23.

\begin{figure}[tb!]
  \centering
\includegraphics[width=1\columnwidth]{aprpostmerger_angles.pdf}
  \caption{Kilonova light curves in the UV/B bands generated by the post-merger ejecta, viewed along the polar axis (thick lines) and the equatorial plane (dashed lines) at a distance of 40\,Mpc. Also shown for comparison is the kilonova precursor signal due to free neutron decay in fast dynamical ejecta (dot-dashed lines) as computed for this merger setup in CS23.}
  \label{fig-kn-angle}
\end{figure}

\bibliographystyle{apsrev4-2}
\bibliography{ms}% Produces the bibliography via BibTeX.

\end{document}
%
% ****** End of file apssamp.tex ******
