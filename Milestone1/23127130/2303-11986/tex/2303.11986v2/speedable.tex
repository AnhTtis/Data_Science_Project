% !TeX spellcheck = en_US
\section{Non-speedable non-randoms}

In this final section we prove our third main result by showing that there exists a left-computable number that is neither speedable nor Martin-Löf random; thereby giving a negative answer to the question of Merkle and Titov~\cite{MT2020} whether among the left-computable numbers the Martin-Löf randoms are characterized by being non-speedable. The understanding of the relationship between near computability and regaining approximability that we gained in the last section will be instrumental for this.
We begin by establishing a more convenient characterization of speedability.
\begin{lem}\label{lem:characterisierung-speedable}
	A left-computable number is speedable if and only if there exists a constant $\rho \in (0, 1) $ and a computable increasing sequence $(x_n)_n$ of rational numbers converging to $x$ such that there are infinitely many $n \in \IN$ with $\frac{x_{n+1} - x_n}{x - x_n} \geq \rho$.
\end{lem}
\begin{proof}
	Suppose that $x$ is speedable, that is, by definition, there exists a constant $\rho' \in (0, 1)$ and a computable increasing sequence $(x_n)_n$ converging to $x$ with $\frac{x - x_{n+1}}{x - x_n} \leq \rho'$ for infinitely many $n \in \IN$. 
	If we let $\rho := 1 - \rho'$, this is equivalent to $\frac{x_{n+1} - x_n}{x - x_n} = 1 - \frac{x - x_{n+1}}{x - x_n} \geq \rho$ for infinitely many $n \in \IN$.
\end{proof}
Thus, a left-computable number $x$ is non-speedable if and only if for every computable increasing sequence of rational numbers $(x_n)_n$ converging to~$x$ the sequence $\left(\frac{x_{n+1} - x_n}{x - x_n}\right)_n$ converges to zero. Before we use this characterization we first examine the relationship between speedable and regainingly approximable numbers.
\begin{prop}\label{prop:aufholend-approximierbar-impliziert-beschleunigbar}
	Every regainingly approximable number is speedable.
\end{prop}
\begin{proof}
	Let $x$ be regainingly approximable. Then, by definition, there exists a computable non-decreasing sequence of rational numbers $(x_n)_n$ converging to $x$ with $x - x_n < 2^{-n}$ for infinitely many $n \in \IN$. Define the sequence $(y_n)_n$ by $y_n := x_n - 2^{-n}$ for all $n \in \IN$. This sequence is computable, increasing and also converges to $x$. Then, for every $n \in \IN$ with $x - x_n < 2^{-n}$, we have
	\begin{equation*}
		\frac{y_{n+1} - y_n}{x - y_n} = \frac{\left(x_{n+1} - x_n\right) + 2^{-(n+1)}}{\left(x - x_n\right) + 2^{-n}} > \frac{2^{-(n+1)}}{2^{-n} + 2^{-n}} = \frac{1}{4}.\qedhere
	\end{equation*}
\end{proof}
The converse is not true as the two following known results imply.
\begin{prop}[Merkle, Titov \cite{MT2020}]
	Every strongly left-computable number is speedable.
\end{prop}
\begin{theorem}[Hertling, Hölzl, Janicki \cite{HHJ2023}]\label{satz:SLBNAA}
	There exists a strongly left-computable number that is not regainingly approximable.
\end{theorem}
\begin{kor}\label{dfsdjhhdbfajfkjgsdjfhdasnfbsafevcvssda}
	There exists a speedable number that is not regainingly approximable.\qed 
\end{kor}

Thus, in general, the regainingly approximable numbers are a proper subset of the speedable numbers. However, as we will show now, the two notions become equivalent once we restrict ourselves to nearly computable numbers. To demonstrate this, we employ the following two convenient characterizations of the left-computable nearly computable and of the left-computable regainingly approximable numbers.
\begin{prop}[Hertling, Janicki~\cite{HJ2023}]\label{prop:characterisierung-lnc}
	The following statements are equivalent for a left-computable number~$x$:
	\begin{enumerate}
		\item $x$ is nearly computable.
		\item For every computable non-decreasing sequence of rational numbers $(x_n)_n$ converging to $x$, the sequence $(x_{n+1} - x_n)_n$ converges computably to zero.
	\end{enumerate}
\end{prop}
\begin{prop}[Hertling, Hölzl, Janicki \cite{HHJ2023}]\label{prop:characterization-reg-app}
		For a left-computable number~$x$ the following are equivalent:
		\begin{enumerate}
			\item $x$ is regainingly approximable.
			\item There exists a computable non-decreasing sequence of rational numbers $(x_n)_n$ converging to $x$ and a computable non-decreasing and unbounded function $f \colon \IN \to \IN$ with
			$x - x_n \leq 2^{-f(n)}$
			for infinitely many $n \in \IN$.
		\end{enumerate}
	\end{prop}
\begin{theorem}\label{dsfjknsdkjldfgljsdfdfgdfterter}
	Let $x$ be a speedable number which is nearly computable. Then $x$ is  regainingly approximable.
\end{theorem}
\begin{proof}
	Choose a constant $\rho \in (0,1)$ and a computable increasing sequence of rational numbers $(x_n)_n$ converging to $x$ that witness the speedability of $x$ in the sense of Lemma~\ref{lem:characterisierung-speedable}.
	
	Since $x$ is nearly computable, the sequence $(x_{n+1} - x_n)_n$ converges computably to zero, as witnessed by some modulus of convergence $f \colon \IN \to \IN$ that is computable, non-decreasing and unbounded.
	Define $g \colon \IN \to \IN$ via
	\begin{equation*}
		g(n) := \begin{cases}
			0 &\text{if $f(0) > n$}, \\
			\max\{k \in \IN \colon f(k) \leq n\} &\text{otherwise},
		\end{cases} 
	\end{equation*}
	for all $n \in \IN$. Clearly, $g$ is computable, non-decreasing and unbounded, and we have $x_{n+1} - x_n < 2^{-g(n)}$ for all $n \geq f(0)$.
	Fix some $k \in \IN$ with $\nicefrac{1}{\rho} \leq 2^k$ and define $h \colon \IN \to \IN$ via
	\begin{equation*}
		h(n) := \max\{0, g(n) - k\}
	\end{equation*}
	 for all $n \in \IN$. Again, $h$ is computable, non-decreasing and unbounded.
	  
	Let $m := \min\{i \geq f(0) \colon g(i) \geq k \}$ and consider any of the infinitely many $n \geq m$ for which $\frac{x_{n+1} - x_n}{x - x_n} \geq \rho$ holds by choice of $\rho$ and of $(x_n)_n$. For all of these $n$ we have
	\begin{equation*}
		x - x_n \leq \nicefrac{1}{\rho} \cdot \left(x_{n+1} - x_n\right) < 2^k \cdot 2^{-g(n)} = 2^{-h(n)}.
	\end{equation*}
	Hence, using Proposition~\ref{prop:characterization-reg-app}, 	
	$x$ is regainingly approximable.
\end{proof}

Putting together Proposition~\ref{prop:aufholend-approximierbar-impliziert-beschleunigbar} and Theorem~\ref{dsfjknsdkjldfgljsdfdfgdfterter}, we obtain the following corollary.

\goodbreak

\begin{kor}
		Let $x$ be a left-computable number that is nearly computable. Then the following statements are equivalent:
		\begin{enumerate}
			\item $x$ is regainingly approximable.
			\item $x$ is speedable.\qed
		\end{enumerate}
	\end{kor}
	
	\goodbreak

Together with Theorem~\ref{satz:FBNAA} we obtain the next corollary.
\begin{kor}\label{kor:LCNCNS}
	There exists a left-computable number which is nearly computable but not speedable.\qed
\end{kor}
Stephan and Wu~\cite{SW2005} showed that a left-computable number which is nearly computable cannot be Martin-Löf random. Together with Corollary~\ref{kor:LCNCNS} this implies our final main result, a negative answer to the question of Merkle and Titov~\cite{MT2020}.
\begin{kor}
	There exists a left-computable number which is not speedable and not Martin-Löf random.\qed
\end{kor}

\section{Acknowledgments}

The authors would like to thank Ivan Titov for helpful discussions.