% !TeX spellcheck = en_US

\section{Preliminaries}

Let $\Sigma := \{0, 1\}$ be the binary alphabet, let $\Sigma^*$ be the set of all (finite) binary words, and let $\Sigma^\IN$ be the set of all (infinite) sequences over $\Sigma$. Write $\Gamma := \Sigma^* \cup \Sigma^\IN$. For any $\sigma \in \Sigma^*$ and $\tau \in \Gamma$, we say that $\sigma$ is a \emph{prefix} of $\tau$ (written $\sigma \prefixeq \tau$) if there exists an $\rho \in \Gamma$ with $\sigma \rho = \tau$. Obviously, the empty word $\lambda$ is a prefix of every other word. If $\sigma \prefixeq \tau$ and $\sigma \neq \tau$ hold, we write $\sigma \prefix \tau$ and say that $\sigma$ is a \emph{proper prefix} of $\tau$. Restricted to $\Sigma^*$, the prefix relation $\prefixeq$ is a partial order.

In addition, we introduce the following strict order: For any $\sigma, \tau \in \Gamma$ we say that $\sigma$ is \emph{lexicographically smaller} than $\tau$ (written $\sigma <_L \tau$) if there exists a $\rho \in \Sigma^*$ with $\rho 0 \prefixeq \sigma$ and $\rho 1 \prefixeq \tau$. Note that this relation is a strict total order only if it is restricted to $\Sigma^\IN$. We explicitly point out that according to this definition two distinct elements are incomparable if and only if one is a proper prefix of the other.

For every $\sigma \in \SigmaS$, we denote with $\left|\sigma\right|$ the length of $\sigma$. Finally, we define the function $\nu\colon \SigmaS \to \IN$, where $\nu(\sigma)$ is $\sigma$'s position in the length-lexicographic order of all strings in $\SigmaS$.

\begin{defi}\label{def:true-path}
	Let $(\truepath[t])_t$ be a sequence in $\SigmaS$. We recursively define the sequence $\truepath \in \SigmaN$, called the \emph{true path} of $(\truepath[t])_t$, as follows:
	\begin{align*}
		\mathcal{T}(0) &:= \begin{cases}
			0 &\text{if $0 \prefixeq \mathcal{T}[t]$ for infinitely many $t \in \IN$} \\
			1 &\text{otherwise}
		\end{cases} \\
		\mathcal{T}(e+1) &:= \begin{cases}
			0 &\text{if $\mathcal{T}(0) \dots \mathcal{T}(e) 0 \prefixeq \mathcal{T}[t]$ for infinitely many $t \in \IN$} \\
			1 &\text{otherwise}
		\end{cases}
	\end{align*}
\end{defi}
Note that the true path is also defined when the sequence $(\left|\truepath[t]\right|)_t$ is bounded.

\begin{fact}\label{prop:truepath-properties}
	Let $(\truepath[t])_t$ be a sequence in $\SigmaS$ and let $\truepath$ be its true path. Then the following statements hold:
	\begin{enumerate}
		\item For all $\sigma \in \SigmaS$ with $\sigma <_L \mathcal{T}$, there exist only finitely many $t\in \IN$ with $\sigma \prefixeq \mathcal{T}[t]$.
		\item If the sequence $(\left|\truepath[t]\right|)_t$ is bounded from above, then $L:= \limsup_{t\to\infty} \left|\truepath[t]\right|$ is a natural number and we have $\truepath(e) = 1$ for all $e \geq L$.
		\item If the sequence $(\left|\truepath[t]\right|)_t$ is unbounded, then for all $\sigma \prefix \truepath$ there exist infinitely many $t \in \IN$ with $\sigma \prefixeq \truepath[t]$. \qed
	\end{enumerate}
\end{fact}

Finally, we fix some standard enumeration $(\varphi_e)_e$ of all computable partial functions from $\IN$ to $\IN$. As usual, the notation $\varphi_{e}(n)[t]{\downarrow}$ denotes that the $e$-th Turing machine, which computes $\varphi_{e}$, stops after at most $t$ steps on input $n$. In this article we will use a uniform modification of this enumeration by enforcing the implication $\varphi_e(n)[t]{\downarrow} \Rightarrow \varphi_e(n) \leq t$ for all $e, n, t \in \IN$. \label{enumeration_convention} Note that this modification does not change the property of being a standard enumeration.

\begin{defi}\label{def:laengenfunktionen}
	Let 
	 $\ell\colon \IN^2 \rightarrow \IN$, defined for all $(e,t)\in \IN^2$ via
	\begin{equation*}
		\ell(e)[t] := \max(\{l \leq t \mid \forall k < l \colon \varphi_e(k)[t]{\downarrow} < \varphi_e(k+1)[t]{\downarrow}\} \cup \{-1\}),
	\end{equation*}
be the {\em length function for increasing functions}.
\end{defi}
Clearly, for all $e \in \IN$, the sequence $(\ell(e)[t])_t$ is non-decreasing, and it tends to infinity if and only if $\varphi_{e}$ is total and increasing.