% !TeX spellcheck = en_US
\section{Non-speedable non-randoms}

In this final section we prove our third main result by showing that there exists a left-computable number that is neither speedable nor Martin-Löf random; thereby giving a negative answer to the question of Merkle and Titov~\cite{MT2020} whether among the left-computable numbers the Martin-Löf randoms are characterized by being non-speedable. The understanding of the relationship between near computability and regaining approximability that we gained in the last section will be instrumental for this.
We begin by establishing a more convenient characterization of speedability.
\begin{lem}\label{lem:characterisierung-speedable}
	A left-computable number is speedable if and only if there exists a constant $\rho \in \left] 0, 1 \right[ $ and a computable increasing sequence $(x_n)_n$ of rational numbers converging to $x$ such that there are infinitely many $n \in \IN$ with $\frac{x_{n+1} - x_n}{x - x_n} \geq \rho$.
\end{lem}
\begin{proof}
	Suppose that $x$ is speedable, that is, by definition, there exists a constant $\rho' \in \left] 0, 1\right[$ and a computable increasing sequence $(x_n)_n$ converging to $x$ with $\frac{x - x_{n+1}}{x - x_n} \leq \rho'$ for infinitely many $n \in \IN$. 
	If we let $\rho := 1 - \rho'$, this is equivalent to $\frac{x_{n+1} - x_n}{x - x_n} = 1 - \frac{x - x_{n+1}}{x - x_n} \geq \rho$ for infinitely many $n \in \IN$.
\end{proof}
Thus, a left-computable number $x$ is non-speedable if and only if for every computable increasing sequence of rational numbers $(x_n)_n$ converging to~$x$ the sequence $\left(\frac{x_{n+1} - x_n}{x - x_n}\right)_n$ converges to zero. Before we use this characterization we first examine the relationship between speedable and regainingly approximable numbers.
\begin{prop}\label{prop:aufholend-approximierbar-impliziert-beschleunigbar}
	Every regainingly approximable number is speedable.
\end{prop}
\begin{proof}
	Let $x$ be regainingly approximable. Then, by definition, there exists a computable non-decreasing sequence of rational numbers $(x_n)_n$ converging to $x$ with $x - x_n < 2^{-n}$ for infinitely many $n \in \IN$. Define the sequence $(y_n)_n$ by $y_n := x_n - 2^{-n}$ for all $n \in \IN$. This sequence is computable, increasing and also converges to $x$. Then, for every $n \in \IN$ with $x - x_n < 2^{-n}$, we have
	\begin{equation*}
		\frac{y_{n+1} - y_n}{x - y_n} = \frac{\left(x_{n+1} - x_n\right) + 2^{-(n+1)}}{\left(x - x_n\right) + 2^{-n}} > \frac{2^{-(n+1)}}{2^{-n} + 2^{-n}} = \frac{1}{4}.\qedhere
	\end{equation*}
\end{proof}
The converse is not true as the two following known results imply.
\begin{prop}[Merkle, Titov \cite{MT2020}]
	Every strongly left-computable number is speedable.
\end{prop}
\begin{theorem}[Hertling, Hölzl, Janicki \cite{HHJ2023}]\label{satz:SLBNAA}
	There exists a strongly left-computable number that is not regainingly approximable.
\end{theorem}
\begin{kor}\label{dfsdjhhdbfajfkjgsdjfhdasnfbsafevcvssda}
	There exists a speedable number that is not regainingly approximable.\qed 
\end{kor}

Thus, in general, the regainingly approximable numbers are a proper subset of the speedable numbers. However, as we will show next, the two notions become equivalent once we restrict ourselves to nearly computable numbers.
The proof of this is not straightforward and needs some preparation.
Let us first recall a convenient characterization of the left-computable, nearly computable numbers.
\begin{prop}[Hertling, Janicki~\cite{HJ2023}]\label{prop:characterisierung-lnc}
	The following statements are equivalent for a left-computable number~$x$:
	\begin{enumerate}
		\item $x$ is nearly computable.
		\item For every computable non-decreasing sequence of rational numbers $(x_n)_n$ converging to $x$, the sequence $(x_{n+1} - x_n)_n$ converges computably to zero.
	\end{enumerate}
\end{prop}
As this characterization employs the concept of 
computable convergence, it will be helpful to have a way of expressing regaining approximability in a similar vein.
\begin{prop}\label{prop:aufholend-approximierbar-aequivalenz}
	Let $x$ be a left-computable number. Let $(x_n)_n$ be a computable non-decreasing sequence of rational numbers converging to $x$ and let $g \colon \IN \to \IN$ be its optimal modulus of convergence. Then the following statements are equivalent:
	\begin{enumerate}
		\item $x$ is regainingly approximable.
		\item There exists a computable function $f \colon \IN \to \IN$ such that $f(n) \geq g(n)$ for infinitely many $n \in \IN$.
	\end{enumerate}
\end{prop}
\begin{proof} 
	``$(1) \Rightarrow (2)$'': Suppose that $x$ is regainingly approximable. Due to Proposition~\ref{prop:charakterisierung-aa}, there is a computable increasing function $f\colon \IN \to \IN$ with 
	${x - x_{f(n)} < 2^{-n}}$
	for infinitely many $n \in \IN$. By definition of an optimal modulus of convergence we must have $f(n) \geq g(n)$ for these $n$.
	
	\smallskip
	
	``$(2) \Rightarrow (1)$'': Suppose that there exists a computable function $f\colon \IN \to \IN$ with $f(n) \geq g(n)$ for infinitely many $n \in \IN$ and assume w.l.o.g.\ that $f$ is increasing. Then for every $n \in \IN$ with $f(n) \geq g(n)$ we have
		$x - x_{f(n)} \leq x - x_{g(n)} < 2^{-n}$.
\end{proof}
As usual we say for functions $f, g\colon \IN \to \IN$ that \emph{$f$ dominates $g$} if ${f(n) \geq g(n)}$ for almost all $n \in \IN$.
Now recall that in Theorem~\ref{satz:FBNAA} we have shown that there are left-computable numbers $x$ that are nearly computable but not regainingly approximable. Then, informally speaking, Proposition~\ref{prop:aufholend-approximierbar-aequivalenz} 
states that every computable non-decreasing sequence of rational numbers $(x_n)_n$ converging to such an~$x$ must converge very slowly in the sense that its optimal modulus of convergence dominates every computable function.
In contrast to this, since $x$ is nearly computable, for every such $(x_n)_n$, the sequence $(x_{n+1} - x_n)_n$ converges computably to zero, thus its optimal modulus of convergence is dominated by a computable function.

Putting these observations together suggests 
that, for every $x$ as in Theorem~\ref{satz:FBNAA} and any computable increasing sequence of rational numbers $(x_n)_n$ converging to~$x$, the sequence $\left(\frac{x_{n+1} - x_n}{x - x_n}\right)_n$ ought to converge to zero, due to its numerator converging much faster than its denominator. The following lemma confirms this intuition. 
\begin{lem}\label{lem:quotient-von-zwei-nullfolgen-konvergenzmodule}
	Let $(a_n)_n$ be a sequence and $(b_n)_n$ be a decreasing sequence, both converging to zero. Let $f \colon  \IN \to \IN$ be a non-decreasing and unbounded function that is a modulus of convergence of $(a_n)_n$, and let $g \colon  \IN \to \IN$ be the optimal modulus of convergence of $(b_n)_n$. Suppose that for all $k \in \IN$ there exists an $m \in \IN$ such that for all $n \geq m$ the inequality $f(n+k) < g(n)$ holds. 
	Then \[\lim_{n\rightarrow\infty}\left(\frac{a_n}{b_n}\right)_{\!\!n}\!=0.\] 
\end{lem}
\begin{proof}
	Let $\varepsilon > 0$. Fix a $k \in \IN$ with $2^{-k} < \frac{\varepsilon}{2}$ and an $m \in \IN$ such that the inequality $f(n+k) < g(n)$ is satisfied for all $n \geq m$. Choose $l := f(m+k)$, and let $i \geq l$. Since $f$ is non-decreasing and unbounded, there exists a number $n \in \IN$ with $f(n+k) \leq i \leq f(n+1+k)$. Then, since $f$ is a modulus of convergence of~$(a_n)_n$, we have $|a_i|<2^{-(n+k)}$. Furthermore $i < g(n+1)$, and as $(b_n)_n$ is a decreasing sequence converging to zero and $g$ its optimal modulus of convergence, we also have $b_i \geq 2^{-(n+1)}$. Putting everything together we obtain
\[
		\left|\frac{a_i}{b_i}\right|  = \frac{\left|a_i\right|}{b_i}                                                            < 
        \frac{2^{-(n+k)}}{2^{-(n+1)}} 
        = 2 \cdot 2^{-k}                                                                             < 2 \cdot \frac{\varepsilon}{2}                                                           = \varepsilon.
\]
	Thus, $\left(\frac{a_n}{b_n}\right)_n$\! converges to zero.
\end{proof}

After this preparation we are ready to prove our final theorem.
\begin{theorem}\label{dsfjknsdkjldfgljsdfdfgdfterter}
	Let $x$ be a left-computable number that is nearly computable but not regainingly approximable. Then $x$ is not speedable.
\end{theorem}
\begin{proof}
	Let $(x_n)_n$ be a computable increasing sequence of rational numbers converging to $x$. Then the sequence $(x_{n+1} - x_n)_n$ converges to zero and, as $x$ is nearly computable, has a computable modulus of convergence $f \colon  \IN  \to \IN$.
	Since $x$ is not regainingly approximable it is in particular non-computable, and thus $f$ must be unbounded; w.l.o.g.\ we also assume it to be non-decreasing.
	Let $g \colon  \IN \to \IN$ be the optimal modulus of convergence of $(x_n)_n$. Since $x$ is not regainingly approximable there is no computable function $h \colon  \IN \to \IN$ with $h(n) \geq g(n)$ for infinitely many~$n\in \IN$, due to Proposition \ref{prop:aufholend-approximierbar-aequivalenz}. 
	In particular, for every constant $k \in \IN$ there exists a number $m \in \IN$ such that for all $n \geq m$ the inequality $f(n+k) < g(n)$ holds.
	Note that $(x - x_n)_n$ is a decreasing sequence that converges to zero and $g$ is its optimal modulus of convergence. Thus, the premises of Lemma~\ref{lem:quotient-von-zwei-nullfolgen-konvergenzmodule} are satisfied, and $\left(\frac{x_{n+1} - x_n}{x - x_n}\right)_n$ converges to zero. Then, by Lemma~\ref{lem:characterisierung-speedable}, $x$~is not speedable.
\end{proof}
Putting together Proposition~\ref{prop:aufholend-approximierbar-impliziert-beschleunigbar} and Theorem~\ref{dsfjknsdkjldfgljsdfdfgdfterter}, we obtain the following corollary.
\begin{kor}
	Let $x$ be a left-computable number that is nearly computable. Then the following statements are equivalent:
	\begin{enumerate}
		\item $x$ is regainingly approximable.
		\item $x$ is speedable.\qed
	\end{enumerate}
\end{kor}
Together with Theorem~\ref{satz:FBNAA} we obtain the next corollary.
\begin{kor}\label{kor:LCNCNS}
	There exists a left-computable number which is nearly computable but not speedable.\qed
\end{kor}
Stephan and Wu~\cite{SW2005} showed that a left-computable number which is nearly computable cannot be Martin-Löf random. Together with Corollary~\ref{kor:LCNCNS} this implies our final main result, a negative answer to the question of Merkle and Titov~\cite{MT2020}.
\begin{kor}
	There exists a left-computable number which is not speedable and not Martin-Löf random.\qed
\end{kor}

\section{Acknowledgments}

The authors would like to thank Ivan Titov for helpful discussions.