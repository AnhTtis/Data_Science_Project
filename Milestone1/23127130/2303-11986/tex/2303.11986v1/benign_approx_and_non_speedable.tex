% !TeX spellcheck = en_US
\documentclass[12pt,reqno]{amsart}

\usepackage[utf8]{inputenc}
\usepackage[]{tikz}
\usepackage[a4paper]{geometry}
\usepackage{hyphenat}
\usepackage[english]{babel}

\usepackage{amsthm}
\usepackage{amssymb}
\usepackage{amsmath}
\usepackage{amsaddr} 
\usepackage{mathtools}
\usepackage{calc}
\usepackage{comment}

\usepackage{listliketab}

\usepackage{hyperref}

\newtheorem{defi}{Definition}
\newtheorem{theorem}[defi]{Theorem}
\newtheorem{lem}[defi]{Lemma}
\newtheorem{fact}[defi]{Fact}
\newtheorem{prop}[defi]{Proposition}
\newtheorem{kor}[defi]{Corollary}
\newtheorem{remark}[defi]{Remark}

%% Definitionen von Zahlenklassen
\def\IN{\mathbb{N}} % natürliche Zahlen
\def\IZ{\mathbb{Z}} % ganze Zahlen
\def\IQ{\mathbb{Q}} % rationale Zahlen
\def\IR{\mathbb{R}} % reelle Zahlen

\def\EC{\mathbf{EC}} % berechenbare Zahlen
\def\LC{\mathbf{LC}} % links-berechenbare Zahlen
\def\WC{\mathbf{WC}} % schwach berechenbare Zahlen
\def\DBC{\mathbf{DBC}} % schwach berechenbare Zahlen
\def\CA{\mathbf{CA}} % rekursiv approximierbare Zahlen

\def\LNC{\mathbf{LNC}} % fast berechenbare links-berechenbare Zahlen
\def\DLNC{\mathbf{DLNC}} % Differenz von fast berechenbaren links-berechenbaren Zahlen
\def\WNC{\mathbf{WNC}} % fast berechenbare schwach berechenbare Zahlen
\def\NC{\mathbf{NC}} % fast berechenbare Zahlen

\def\SigmaS{\Sigma^{\ast}} %Sigma^*
\def\SigmaN{\Sigma^{\IN}} %Sigma^*
\def\prefix{\sqsubset} %echtes Präfix
\def\prefixeq{\sqsubseteq} %Präfix
\def\suffix{\sqsupset} %echtes Suffix
\def\suffixeq{\sqsupseteq} %Suffix
\def\truepath{\mathcal{T}} %tatsächlicher Pfad, True Path

\def\id{\textnormal{id}} % Identität
\def\dom{\textnormal{dom}} % Definitionsbereich


\parindent=0em

\begin{document}

\allowdisplaybreaks

\title[Benign approximations and non-speedability]{Benign approximations and non-speedability\\{\scriptsize (Draft; \MakeUppercase\today)}}

\author{Rupert H\"olzl and Philip Janicki}
\address{University of the Bundeswehr Munich}
\email{r@hoelzl.fr}
\email{philip.janicki@unibw.de}

\begin{abstract}
A left-computable number $x$ is called  {\em regainingly approximable} if there is a computable increasing sequence $(x_n)_n$ of rational numbers converging to $x$ such that $x - x_n < 2^{-n}$ for infinitely many $n \in \IN$; and it is called {\em nearly computable} if there is such an~$(x_n)_n$ such that for every computable increasing function $s \colon  \IN \to \IN$ the sequence ${(x_{s(n+1)} - x_{s(n)})_n}$ converges computably to~$0$.
In this article we study the relationship between both concepts by constructing on the one hand a non-computable number that is both regainingly approximable and nearly computable, and on the other hand a left-computable number that is nearly computable but not regainingly approximable; it then easily follows that the two notions are  incomparable with non-trivial intersection. With this relationship  clarified, we then hold the keys to answering an open question of Merkle and Titov: they studied {\em speedable} numbers, that is, left-computable numbers whose approximations can be sped up in a certain sense, and asked whether, among the left-computable numbers, being Martin-Löf random is equivalent to being non-speedable. As we show that the concepts of speedable and regainingly approximable numbers are equivalent within the nearly computable numbers, our second construction provides a negative answer.
\end{abstract}

\maketitle

\section{Introduction}
\label{sec:introduction}
% \begin{itemize}
%     % Diffusion of FL
%     \item {\st{Diffusion of FL}}
%     % Security threats to FL
%     \item {\st{Security threats to FL with particular focus on model poisoning}}
%     % Limitations of existing countermeasures
%     \item {\st{Current countermeasures (e.g., KRUM) and their limitations}}
%     % Proposed method and its advantages
%     \item {\st{Intuitive description of the proposed method and its difference (i.e., advantages) w.r.t. state of the art}}
%     % Main contributions
%     \item {\st{Summary of the main contributions of this work}}
%     % Paper's structure and organization
%     \item {\st{Paper's structure and organization}}
% \end{itemize}

% Diffusion of FL
Recently, {\em federated learning} (FL) has emerged as the leading paradigm for training distributed, large-scale, and privacy-preserving machine learning (ML) systems~\cite{mcmahan2017googleai,mcmahan2017aistats}. 
The core idea of FL is to allow multiple edge clients to collaboratively train a shared, global model without disclosing their local private training data.
%Specifically, an FL system consists of a central server and many edge clients; 
A typical FL round involves the following steps: {\em(i)} the server randomly picks some clients and sends them the current, global model; {\em(ii)} each selected client locally trains its model with its own private data; then, it sends the resulting local model to the server;\footnote{Whenever we refer to global/local model, we mean global/local model {\em parameters}.} {\em(iii)} the server updates the global model by computing an \emph{aggregation function}, usually the average (FedAvg), on the local models received from clients.
% \begin{enumerate}
%     \item[{\em(i)}] the server sends the current, global model to the clients and appoints some of them for training;
%     \item[{\em(ii)}] each selected client locally trains its copy of the global model with its own private data; then, it sends the resulting local model back to the server;\footnote{Whenever we refer to global/local model, we mean global/local model {\em parameters}.}
%     \item[{\em(iii)}] the server updates the global model by computing an \emph{aggregation function} on the local models received from clients (by default, the average, also referred to as FedAvg~\cite{mcmahan2017aistats}).
% \end{enumerate}
This process goes on until the global model converges. %(e.g., after a certain number of rounds or other similar stopping criteria).
%\\
% The advantages of FL over the traditional, centralized learning paradigm are undoubtedly clear in terms of flexibility/scalability (clients can join/disconnect from the FL network dynamically), network communications (only model weights\footnote{We will use \textit{parameters} and \textit{weights} interchangeably.} are exchanged between clients and server), and privacy (each client's private training data is kept local at the client's end and not uploaded to the server).
\\
% Security threats to FL
%However, the growing adoption of FL also raises security concerns~\cite{costa2022covert}, particularly about its confidentiality, integrity, and availability.
Although its advantages over standard ML, FL also raises security concerns~\cite{costa2022covert}. %, particularly about its confidentiality, integrity, and availability~\cite{costa2022covert}.
% OLD, LONG VERSION
% Indeed, some work deals with privacy leakage that may expose the local data of some clients~\cite{melis2019sp}. 
% A large body of work, instead, investigates attacks that usually aim to detriment the predictive accuracy of the learned global model. For instance, \emph{data poisoning} attacks achieve this goal by letting an adversary pollute the training set of some corrupt FL clients with maliciously crafted examples~\cite{jagielski2018sp}.
% Similarly, in \emph{model poisoning} the attacker attempts to tweak the global model weights~\cite{bhagoji2019pmlr} by directly perturbing the local model's weights of some infected FL clients before these are sent to the central server for aggregation, usually via so-called Byzantine attacks. 
% It turns out that Byzantine model poisoning attacks severely impact standard FedAvg; therefore, more robust aggregation functions must be designed to make FL systems secure.
Here, we focus on \emph{untargeted model poisoning} attacks~\cite{bhagoji2019pmlr}, where an adversary attempts to tweak the global model weights %\footnote{We will use the terms \textit{parameters} and \textit{weights} interchangeably.} 
by directly perturbing the local model's parameters of some infected clients before these are sent to the central server for aggregation.
In doing so, the adversary aims to jeopardize the global model \textit{indiscriminately} at inference time.
Such model poisoning attacks severely impact standard FedAvg; therefore, more robust aggregation functions must be designed to secure FL systems.
\\
% In this paper, we focus on designing a novel robust aggregation scheme at the server's end to contrast the effect of Byzantine model poisoning attacks.
%
% Current countermeasures and their limitations
%Several countermeasures have been proposed in the literature to combat model poisoning attacks on FL systems.
% Some methods use simple statistics more robust than plain average to smooth the impact of malicious updates (e.g., Trimmed Mean and FedMedian~\cite{yin2018icml}). 
% Other defenses implement outlier detection techniques to discard malicious updates from the aggregation performed at the server's end. Those are either based on heuristics (e.g., Krum/Multi-Krum~\cite{blanchard2017nips} and Bulyan~\cite{mhamdi2018pmlr}) or data-driven approaches (e.g., K-means clustering~\cite{shen2016acm} or DnC via spectral analysis~\cite{shejwalkar2021ndss}). 
% Finally, some strategies rely on a centralized ``source of trust'' to spot potential malicious updates (e.g., FLTrust~\cite{cao2020fltrust}).
% Several countermeasures have been proposed in the literature to combat model poisoning attacks on FL systems, i.e., to discard possible malicious local updates from the aggregation performed at the server's end. 
% These techniques range from simple statistics more robust than plain average (e.g., Trimmed Mean and FedMedian~\cite{yin2018icml}) to outlier detection heuristics (e.g., Krum/Multi-Krum~\cite{blanchard2017nips} and Bulyan~\cite{mhamdi2018pmlr}) or data-driven approaches (e.g., spectral analysis via K-means clustering~\cite{shen2016acm} or spectral analysis), or methods based on ``source of trust'' (e.g., FLTrust~\cite{cao2020fltrust}).
% OLD, LONG VERSION
%Several countermeasures have been proposed in the literature to combat Byzantine model poisoning attacks on FL systems.
% Descriptive statistics
% For example, Trimmed Mean and FedMedian aggregate local model updates using more robust statistics than standard average~\cite{yin2018icml}.
%
% % Heuristics for outlier detection
% Many existing Byzantine-resilient strategies implement some outlier detection heuristics to discard the model updates sent by potentially malicious clients from the input of the aggregation function.
% One of the most popular heuristics is Krum~\cite{blanchard2017nips}.
% This strategy tries to mitigate the impact of Byzantine attacks by selecting as a global model the local model with the smallest sum of Euclidean distances to {\em all} the other local models.
% Although powerful, Krum requires the server to know (or, at least, estimate) the number of malicious FL clients upfront, which is generally impossible in a realistic attack scenario. %
% Moreover, Krum may become ineffective for complex, high-dimensional model parameter spaces due to the curse of dimensionality.
% Bulyan~\cite{mhamdi2018pmlr} tries to overcome this issue by combining Krum with a variant of Trimmed Mean.
% % Data-driven outlier detection
% Other strategies use data-driven outlier detection techniques -- e.g., via K-means clustering~\cite{shen2016acm} -- to spot potential malicious local model updates. 
% %For instance, Shen et al. propose to cluster local model updates with K-means and thus identify outliers.
%
% % Other techniques
% As far as the server is concerned, any local model received can be from a potential malicious client. 
% FLTrust~\cite{cao2020fltrust} assumes the server acts as a client, i.e., trains a local model on an additional {\em trustworthy} dataset at the server's end and compares it against all the local models from other clients. 
% This way, the server can rely on some ``source of trust'' when discarding potentially malicious clients.
%\\
% Limitations of existing Byzantine-resilient strategies
Unfortunately, existing defense mechanisms either rely on simple heuristics (e.g., Trimmed Mean and FedMedian by~\cite{yin2018icml}) or need strong and unrealistic assumptions to work effectively (e.g., foreknowledge or estimation of the number of malicious clients in the FL system, as for Krum/Multi-Krum~\cite{blanchard2017nips} and Bulyan~\cite{mhamdi2018pmlr}, which, however, cannot exceed a fixed threshold).
Furthermore, outlier detection methods using K-means clustering~\cite{shen2016acm} or spectral analysis like DnC~\cite{shejwalkar2021ndss} do not directly consider the temporal evolution of local model updates received.
Finally, strategies like FLTrust~\cite{cao2020fltrust} require the server to collect its own dataset and act as a proper client, thereby altering the standard FL protocol.
\\
% OLD, LONG VERSION
% Overall, existing Byzantine-resilient strategies are either simple heuristics (e.g., FedMedian) or, if they are more complex, they rely on strong and unrealistic assumptions to work effectively (e.g., knowing the number of malicious clients in the FL system in advance, as for Krum and alike).
% Furthermore, data-driven outlier detection methods do not consider the temporary evolution of local model updates received (e.g., K-means clustering). 
% Finally, strategies like FLTrust requires the server to collect its own dataset and act as a proper client, thereby altering the standard FL protocol.
%
% Description of the proposed method
This work introduces a novel pre-aggregation \textit{filter} robust to untargeted model poisoning attacks. Notably, this filter $(i)$ operates without requiring prior knowledge or constraints on the number of malicious clients and $(ii)$ inherently integrates temporal dependencies. 
The FL server can employ this filter as a preprocessing step before applying \textit{any} aggregation function, be it standard like FedAvg or robust like Krum or Bulyan.
Specifically, we formulate the problem of identifying corrupted updates as a multidimensional (i.e., matrix-valued) time series anomaly detection task. 
The key idea is that legitimate local updates, resulting from well-calibrated iterative procedures like stochastic gradient descent (SGD) with an appropriate learning rate, show \textit{higher predictability} compared to malicious updates. This hypothesis stems from the fact that the sequence of gradients (thus, model parameters) observed during legitimate training exhibit regular patterns, as validated in Section~\ref{subsec:intuition}. %until convergence. 
%This regularity may be more pronounced for smooth convex loss functions, but it can still be captured within an appropriate time window, even for more complex and convoluted loss surfaces. 
%We provide evidence of this claim in Appendix~B, where we show that the average mutual information (i.e., ``predictability''), calculated over pairs of legitimate model updates sent at different FL rounds, is significantly higher than the corresponding computation for a malicious client.
\\
Inspired by the matrix autoregressive (MAR) framework for multidimensional time series forecasting~\cite{chen2021je}, we propose the FLANDERS ({\em \textbf{F}ederated \textbf{L}earning meets \textbf{AN}omaly \textbf{DE}tection for a \textbf{R}obust and \textbf{S}ecure}) filter.
The main advantages of FLANDERS over existing strategies like FLDetector~\cite{zhao2020multivariate} are its resilience to large-scale attacks, where $50\%$ or more FL participants are hostile, and the capability of working under realistic non-iid scenarios.
We attribute such a capability to two key factors: $(i)$ FLANDERS works without knowing a priori the ratio of corrupted clients, and $(ii)$ it embodies temporal dependencies between intra- and inter-client updates, quickly recognizing local model drifts caused by evil players. Below, we summarize our main contributions:

\begin{itemize}
\item[{\em(i)}]
We provide empirical evidence that the sequence of models sent by legitimate clients is more predictable than those of malicious participants performing untargeted model poisoning attacks.
\\
\item[{\em(ii)}] 
We introduce FLANDERS, the first pre-aggregation filter for FL robust to untargeted model poisoning based on multidimensional time series anomaly detection.
\\
\item[{\em(iii)}] 
We integrate FLANDERS into Flower,\footnote{\scriptsize{\url{https://flower.dev/}}} a popular FL simulation framework for reproducibility.
\\
\item[{\em(iv)}] 
We show that FLANDERS improves the robustness of the existing aggregation methods under multiple settings: different datasets, client's data distribution (non-iid), models, and attack scenarios.
\\
\item[{\em(v)}] 
We publicly release all the implementation code of FLANDERS along with our experiments.\footnote{\scriptsize{\url{https://anonymous.4open.science/r/flanders_exp-7EEB}}}
\end{itemize}

% Paper's structure and organization
The remainder of the paper is structured as follows. %some related work and the current state-of-the-art solutions to security issues that FL entails. 
Section~\ref{sec:background} covers background and preliminaries. 
In Section~\ref{sec:related}, we discuss related work.
Section~\ref{sec:problem} and Section~\ref{sec:method} describe the problem formulation and the method proposed. % to tackle it. 
Section~\ref{sec:experiments} gathers experimental results. %, and Section~\ref{sec:limitations} discusses some limitations of this work.
Finally, we conclude in Section~\ref{sec:conclusion}.
 %discusses the limitations of this work and draws future research directions.
%reports conclusions and draws perspectives for future research directions.

%%%%%%% OLD %%%%%%%
%to overcome the resilience of Byzantine failures in distributed Stochastic Gradient Descent computations. 
% The strength of Krum is its time complexity, which is linear in the gradient dimension. 
% However, the robustness of the approach is guaranteed for gradient-based learning applications only when the majority of the clients are not compromised. 
% Besides, the aggregation mechanism of Krum, as well as that of similar methods, is robust from a coarse-grained perspective and does not provide solutions to errors and perturbations that may occur at inference time.
%A related approach to~\cite{blanchard2017nips} is the work of Su et al.~\cite{su2016dc}. Here, the authors propose an iterated approximate agreement to tackle a multi-layer scenario attacked by Byzantine agents. 
%However, the method works efficiently on the sole discrete context and it is inapplicable to continuous state environments.
%\gabri{Maybe, we should just talk about the main limitations of existing countermeasures without digging into their details (or, we can just mention Krum as this is the most popular one). I will move the description of all these methods to the Related Work section.}
\section{Notation and Preliminaries}\label{sec_prel}
Let $\mathbb{Z}_{>0}$ denote the set of positive integers and let $\mathbb{Z}_{[a,b]}$ denote the set of integers in the interval $[a,b]$. The $m\times m$ identity matrix is denoted by $I_m$ and its columns by $e_i$ for $i\in\mathbb{Z}_{[1,m]}$. We use $\mathbf{0}$ to denote a vector or a matrix of zeros of appropriate dimensions. For a sequence $\{z_k\}_{k=0}^{N-1}$ with $z_k\in\mathbb{R}^\eta$, we denote its stacked vector as $z = \begin{bmatrix}z_0^\top &z_1^\top & \dots & z_{N-1}^\top\end{bmatrix}^\top$ and a stacked window of it as $z_{[l,j]} = \begin{bmatrix}z_l^\top &z_{l+1}^\top & \dots & z_{j}^\top\end{bmatrix}^\top$ with $0\leq l<j$.\par
Persistence of excitation of a sequence and its extension to multiple sequences \cite{vanWaarde20} are defined as follows.
\begin{definition} The sequence \(\{z_k\}_{k=0}^{N-1}\), $z_k\in\mathbb{R}^{\eta}$, is said to be persistently exciting of order \(L\) if \(\textup{rank}(\mathscr{H}_{L}(z))=\eta L\), where $\mathscr{H}_L(z) = \begin{bmatrix}
		z_{[0,L-1]} & z_{[1,L]} & \cdots & z_{[N-L,N-1]}
	\end{bmatrix}$.
	\label{def_PE}
\end{definition}
\begin{definition}[\cite{vanWaarde20}]\label{def_cPE}
	The sequences $\{z_k^{(j)}\}_{k=0}^{N_j-1}$, with $z_k^{(j)}\in\mathbb{R}^\eta$ and $j\in\mathbb{Z}_{[1,r]}$, are said to be \textit{collectively persistently exciting} of order $L$ if rank$(\mathcal{H}_L(\mathscr{Z}))=\eta L$, where $\mathscr{Z} = \begin{bmatrix}
		(z^{(1)})^\top & \cdots & (z^{(r)})^\top
	\end{bmatrix}^\top,$ and
	\begin{equation*}
		\mathcal{H}_L(\mathscr{Z}) = \begin{bmatrix}
			\mathscr{H}_L(z^{(1)}) & \cdots & \mathscr{H}_L(z^{(r)})
		\end{bmatrix}.
	\end{equation*}
\end{definition}
 \begin{figure}[t]
        \centering
            \begin{subfigure}{0.48\linewidth}
            \centering
            {\includegraphics[width=0.95\linewidth]{figures/qual_wo.png}}
            \caption{\method w/o TA}\label{fig:quala}
           \end{subfigure}
           \begin{subfigure}{0.48\linewidth}
            \centering
            {\includegraphics[width=0.95\linewidth]{figures/qual_w.png}}
            \caption{\method w/ TA}\label{fig:qualb}
           \end{subfigure}\vspace{-7pt}
        \caption{Visualization of attention maps (a) without temporal adaptation (TA) and (b) with temporal adaptation for the action 'Spinning [something] that quickly stops spinning' in SSv2~\cite{ssv2}.}\vspace{-5pt}
    \label{fig:qual}
    \end{figure}
% !TeX spellcheck = en_US

\section{Incomparability}

In this section we prove that  the regainingly approximable and the nearly computable numbers are incomparable within the left-computable numbers. One of the separations is provided by the following proposition.
\begin{prop}\label{sadjasjhkrtjhqwejhxvbbmasdgh}
	There exists a regainingly approximable number which is not nearly computable.
\end{prop}
\begin{proof}
	A result of Downey, Hirschfeldt, and LaForte~\cite[Theorem~2.15]{DHL01} implies that every strongly left-computable number that is nearly computable is, in fact, computable.
	But Hertling, Hölzl, and Janicki~\cite{HHJ2023} established the existence of regainingly approximable numbers that are strongly left-computable without being computable. Such a number can therefore not be nearly computable.
\end{proof}

The other separation is the second main result of this article. 
\begin{theorem}\label{satz:FBNAA}
	There exists a left-computable number which is nearly computable but not regainingly approximable.
\end{theorem}
While the proof of Theorem~\ref{satz:FBNAA} will be structurally similar to that of Theorem~\ref{satz:fast-berechenbar-aufholend-approximierbar}, there are important differences that, while seemingly subtle, vastly impact the dynamics of the construction. In these introductory explanations we will thus focus on the differences between both constructions, but will then proceed with the construction and verification in full detail.
We again prove the theorem by constructing a computable non-decreasing sequence of rational numbers $(x_t)_t$ which converges nearly computably to some $x$.
To guarantee that $x$ is not regainingly approximable, we can use the following characterization.
\begin{lem}[Hertling, Hölzl, Janicki~\cite{HHJ2023}]\label{prop:charakterisierung-aa}
	The following statements are equivalent for a left-computable number~$x$:
	\begin{enumerate}
		\item $x$ is regainingly approximable.
		\item For every computable non-decreasing sequence of rational numbers $(x_n)_n$ converging to $x$ there exists a computable increasing function $s \colon \IN \to \IN$ with $x - x_{s(n)} < 2^{-n}$ for infinitely many $n \in \IN$.
	\end{enumerate} 
\end{lem}
It is thus sufficient to ensure that for the sequence $(x_t)_t$ that we are constructing there is no computable function $s$ as in the second item. Therefore, for every~$e \in \IN$, we define the following negative requirement:
\begin{equation*}
	\mathcal{N}_e \colon \varphi_e \text{ total and increasing } \Rightarrow \left(\exists m \in \IN\right) \left(\forall n \geq m\right) \;  x - x_{\varphi_e(n)} \geq 2^{-n}
\end{equation*}
Notice how the quantifiers differ from those in the negative requirements used to prove Theorem~\ref{satz:fast-berechenbar-aufholend-approximierbar}; here, a negative requirement can be threatened infinitely often, which will need to be reflected in the details of the construction below.

To ensure that $x$ is nearly computable, we use the same positive requirements as before; namely, for every $e \in \IN$:
\begin{equation*}
	\mathcal{P}_e \colon \varphi_e \text{ total and increasing }  \Rightarrow \left(x_{\varphi_e(t+1)} - x_{\varphi_e(t)}\right)_t
	\text{ converges computably to } 0
\end{equation*}
As in the last proof, the two types of requirements seemingly contradict each other. We will again resolve this apparent conflict by splitting large jumps into small jumps that will be scheduled and delayed for later execution.

We point out that, despite the more demanding negative requirements used here, the proof of this theorem is somewhat easier than that of Theorem~\ref{satz:fast-berechenbar-aufholend-approximierbar}. This is because in that proof, unlike here, in order to establish regaining approximability, much effort had to be devoted to ensuring  the existence of the special cut-off stages.

\smallskip

We again work with an infinite injury priority construction on an infinite binary tree of strategies~$\sigma \in \Sigma^*$ that each are responsible for satisfying both requirements $\mathcal{N}_{\left|\sigma\right|}$ and $\mathcal{P}_{\left|\sigma\right|}$, and define 
the \emph{parameters} of such a strategy as the following four functions from $\SigmaS \times \IN$ to $\IN$, defined for every $\sigma \in \Sigma^*$ and every~$t\in\IN$:
\begin{itemize}
	\item a \textit{counter} $c(\sigma)[t]$
	\item a \emph{pause flag} $p(\sigma)[t]$
	\item a \textit{restraint} $r(\sigma)[t]$
	\item a \textit{witness} $w(\sigma)[t]$		
\end{itemize}
As before, by construction, $w$ and $r$ will be non-decreasing in $t$ for each $\sigma$. In the construction, some negative requirements will require attention infinitely often. The pause flag will be used to prevent  high priority negative requirements from precluding other requirements from ever receiving attention.

\smallskip

After these informal remarks, we proceed with the full proof of the theorem.	
\begin{proof}[Proof of Theorem~\ref{satz:FBNAA}]
	We recursively define a computable non-decreasing sequence of rational numbers $(x_t)_t$ starting with $x_0 := 0$. At the same time, we also recursively define four functions ${c, p, r, w \colon \SigmaS \times \IN \to \IN}$
	starting with
\begin{align*}
	c(\sigma)[0]	&:= 0, \\
	p(\sigma)[0]	&:= 0, \\
	r(\sigma)[0]	&:= 0, \\
	w(\sigma)[0]	&:= \nu(\sigma),
\end{align*}
for all $\sigma \in \SigmaS$.
As in the proof of Theorem~\ref{satz:fast-berechenbar-aufholend-approximierbar}, the construction proceeds in stages consisting of at most $t + 1$ substages. 
At the end of each stage $t\in \IN$, we will fix a finite binary string $\truepath[t]$ and say that stage~$t$ {\em settles on}~$\truepath[t]$. Again, this string is determined in the substages as follows: In the first substage of each stage we {\em apply} 
strategy~$\lambda$; and in each substage when some $\sigma$ is applied, we can choose whether we want to {\em apply} $\sigma0$ or $\sigma1$ in the following substage or whether we want to let the current stage end right after the current substage. 
The strategy applied in the last substage of a stage is then the string that that stages settles on.

We define the same terms as before, but with the different conditions needed here:
\begin{itemize}
	\item  \emph{Initializing} a strategy~${\tau \in \SigmaS}$ at a stages $t$ means setting
\begin{align*}
	c(\tau)[t+1] &:= 0, \\
	w(\tau)[t+1] &:= \nu(\tau) + t + 2.
\end{align*}
	\item We say that a strategy $\sigma\in \SigmaS$ is {\em threatened at stage $t$} if the  conditions 
	\begin{itemize}
	\item $p(\sigma)[t] = 0$,
	\item $\ell(e)[t] \geq w(\sigma)[t]$,
	\item $x_{t} - x_{\varphi_{e}(\ell(e)[t])} < 2^{-w(\sigma)[t]}$
\end{itemize}
	are satisfied. As before, to defeat this threat, we would like to react by making some large jump.
	
	\item We say that a strategy $\sigma\in \SigmaS$ is \emph{expansionary at stage $t$} if the conditions
	\begin{itemize}
		\item $\ell(e)[t]\geq 0$ where $e:=|\sigma|$,
		\item $x_{t} - x_{\varphi_{e}(\ell(e)[t])} < 2^{-r(\sigma)[t]}$
	\end{itemize}
	are satisfied. As before this means that $\varphi_{e}$ has made some progress towards looking like a total, increasing function. 
\end{itemize}

We complete the description of the construction by giving the details of what we do in a substage of stage~$t$ when a strategy $\sigma \in \SigmaS$ is applied; note the significantly different initialization behaviour compared with the proof of Theorem~\ref{satz:fast-berechenbar-aufholend-approximierbar}:
\begin{enumerate}
	\item Let $e := \left|\sigma\right|$. If we have $e = t$, then this is the last substage, we set
	\begin{equation*}
		x_{t+1} := x_t,
	\end{equation*}
	we initialize all strategies $\tau \in \SigmaS$ with $\sigma <_L \tau$, and terminate stage~$t$. Otherwise we continue with~(2).	

	\item \emph{Negative requirement:}
	If $\sigma$ is not threatened at stage $t$, set
	\[p(\sigma)[t+1] := 0,\] meaning in particular that if strategy $\sigma$ was paused before, it is now unpaused again. Then jump directly to~(3).
	
	\smallskip
		
 Otherwise, that is if $\sigma$ is threatened at stage $t$, check if there is a $\gamma \in \Sigma^*$ with $\gamma 0 \sqsubseteq \sigma$ and $r(\gamma)[t+1] \geq w(\sigma)[t]$. If not, then let
	\begin{align*}
		x_{t+1} &:= x_{t} + 2^{-w(\sigma)[t]}.
	\end{align*}
If yes, then choose the longest such $\gamma$ and set
	\begin{align*}
		c(\gamma)[t+1] 	&:= \left\langle \sigma, 2^{r(\gamma)[t+1] - w(\sigma)[t]}\right\rangle, \\
		x_{t+1} &:= x_t.
	\end{align*}
	In either case, this will be the last substage of this stage. We pause strategy~$\sigma$ by setting
	\begin{align*}
		p(\sigma)[t+1] &:= 1;		
	\end{align*}
	this prevents $\sigma$ from being threatened in the next stage and thus gives other, lower priority strategies a chance to act. We also set 
\begin{align*}
	w(\sigma)[t+1] &:= w(\sigma)[t] + 1
\end{align*}
	which, unlike in the proof of Theorem~\ref{satz:fast-berechenbar-aufholend-approximierbar}, is necessary due to the infinitary nature of the negative requirements. We then initialize all strategies $\tau \in \SigmaS$ with $\sigma <_L \tau$, and terminate stage $t$.
	
	\item \emph{Positive requirement:}
	If $\sigma$ is not expansionary at stage $t$, continue with the next substage $e+1$ applying~$\sigma 1$. As before, this
 means that we currently have no reason to believe that $\varphi_e$ is a total and increasing function.
		
		Otherwise, if $\sigma$ is expansionary at stage $t$, we check if $c(\sigma)[t] = 0$. If this is the case, then we set
	\begin{equation*}
		r(\sigma)[t+1] := r(\sigma)[t] + 1
	\end{equation*} 
	and continue with the next substage $e+1$ applying~$\sigma 0$. As before, the ``0'' documents that we
	have made our restraint on future jump sizes stricter because we currently consider $\varphi_e$ a viable candidate for being a total and increasing function.
	
	Otherwise there exist a strategy $\alpha \in \SigmaS$ and a number $k \in \IN$ with $c(\sigma)[t] = \left\langle \alpha, k+1\right\rangle$, meaning that we still have scheduled jumps to execute. Thus check if there exists a $\gamma \in \Sigma^*$ with $\gamma 0 \sqsubseteq \sigma$. If not, then set 
	\begin{equation*}
		x_{t+1} := x_{t} + 2^{-r(\sigma)[t]}.
	\end{equation*}
	If yes, choose the longest such $\gamma$. The argument used to prove Fact~\ref{restrait_monotony} works exactly in the same way for the present construction; thus we have $r(\gamma)[t+1] \geq r(\sigma)[t]$ and we can set
	\begin{align*}
		c(\gamma)[t+1] &:= \left\langle \alpha, 2^{r(\gamma)[t+1] - r(\sigma)[t]}\right\rangle, \\
		x_{t+1} &:= x_t.
	\end{align*}
	In either case, this will be the last substage of this stage. As one of the scheduled jumps has been taken care of, we can decrement the corresponding counter by setting
	\begin{equation*}
			c(\sigma)[t+1] := \begin{cases}
				0 &\text{if $k = 0$,} \\
				\left\langle \alpha, k\right\rangle &\text{otherwise}.
			\end{cases}
	\end{equation*}
	Then we initialize all strategies $\tau \in \SigmaS$ with $\sigma0 <_L \tau$, and terminate stage~$t$.
\end{enumerate}

	If some of the parameters $c(\sigma)[t+1]$, $p(\sigma)[t+1]$, $r(\sigma)[t+1]$, or $w(\sigma)[t+1]$ have not been explicitly set to some new value during stage $t$, then we set them to preserve their last respective values $c(\sigma)[t]$, $p(\sigma)[t]$, $r(\sigma)[t]$, or $w(\sigma)[t]$. 

\medskip

We proceed with the verification. The following properties of the restraint and the witness functions follow directly from their definitions; we omit the proofs.
\goodbreak
\begin{fact}\label{satz:FBNAA:lem04}
For all $\sigma, \tau \in \Sigma^*$ and all $t \in \IN$ we have
\begin{itemize}
	\item \makebox[\widthof{$w(\sigma)[t]$}][r]{$r(\sigma)[t]$} $\leq t$,
	\item \makebox[\widthof{$w(\sigma)[t]$}][r]{$r(\sigma)[t]$} $\leq r(\sigma)[t+1]$,
	\item $w(\sigma)[t]$ $\leq w(\sigma)[t+1]$.\qed
\end{itemize}
\end{fact}

	It is easy to verify that  Fact~\ref{asdjltzhjkqwehk} and Fact~\ref{fact:counters-on-expansionary-stages} carry over to this construction with literally the same proofs.
	The next two statements hold because the pause function is not affected by initializations.
	\begin{fact}\label{sdfjhasdfkjlersdvsdfdfgdfssd}
		Let $t_1 < t_2$ be two consecutive stages at which some strategy $\sigma \in \SigmaS$ is applied. Then $p(\sigma)[t_1]=0$ or $p(\sigma)[t_2]=0$.\qed
	\end{fact}
	\begin{fact}\label{sdfjhasdfkjldfghjkasjlersdvsdf}
	Let $t_1 < t_2$ be two consecutive stages at which some strategy $\sigma \in \SigmaS$ is applied. Assume that $\sigma$ is threatened at $t_1$. Then $\sigma$ is not threatened at $t_2$.
	\end{fact}
\begin{proof}
	Since $\sigma$ is threatened at $t_1$, by construction we have both $p(\sigma)[t_1] = 0$ and $p(\sigma)[t_1+1] = 1$. Then, by construction, we still have $p(\sigma_l)[t_2] = 1$, and thus $\sigma$ cannot be threatened at~$t_2$.
\end{proof}
The proof of Lemma~\ref{satz:FBNBAA:lem02} has to be modified as follows to reflect the different dynamics of the present construction.
	\begin{lem}\label{fghsdfdhdfsdgdfghgfsfsdf}
		Let $\sigma \in \Sigma^*$. The following two properties are equivalent:
		\begin{enumerate}
			\item The strategy $\sigma$ is applied and expansionary at infinitely many stages.
			\item The strategy $\sigma$ is applied, not threatened, and expansionary at infinitely many stages.
			\item Infinitely many stages settle on some extension of $\sigma0$.
		\end{enumerate}
	\end{lem}
	\begin{proof}
		``$(1) \Rightarrow (2)$'': Let $t_1$ be any stage at which $\sigma$ is applied and expansionary. We claim there is a stage $t_2 \geq t_1$ at which $\sigma$ is applied, not threatened, and expansionary. If $\sigma$ is not threatened at $t_1$ we can let $t_2:=t_1$. Otherwise let $t_2$ be the next stage at which $\sigma$ is applied. In this case, to see that $t_2$ is as needed, note that by construction and by the definition of an expansionary stage, $\sigma$ is still expansionary at $t_2$. But by Fact~\ref{sdfjhasdfkjldfghjkasjlersdvsdf}, $\sigma$ cannot be threatened at~$t_2$.
		
		\smallskip
		
		``$(2) \Rightarrow (3)$'':  Suppose that there are infinitely many stages at which $\sigma$ is applied, not threatened, and expansionary. Let $t_0 \in \IN$ be an arbitrary  such stage. 
		We claim that there must be a stage $t^\ast \geq t_0$ where $\sigma0$ is applied. 
		If $c(\sigma)[t_0] = 0$, then $t^\ast=t_0$ by construction. 
		Otherwise fix $\alpha \in \SigmaS$ and $k \geq1$ with $c(\sigma)[t_0] = \left\langle \alpha, k\right\rangle$. Then let $t_1, \dots, t_k \in \IN$ denote the $k$ consecutive stages where $\sigma$~is applied, not threatened, and expansionary that immediately follow $t_0$; we claim that  we can find~$t^\ast$ among them. Namely, if between $t_0$ and $t_k$ an initialization of $\sigma$ occurs at some stage $\widehat t$, then let $t^\ast$ be the smallest element of $\{t_0,\dots, t_k\}$ that is greater than~$\widehat t$. Otherwise, if no such initialization occurs, then we must have  $c(\sigma)[t_k] = 0$ by construction, and $t^\ast = t_k$.
		
		\smallskip
		
		``$(3) \Rightarrow (1)$'': Suppose that infinitely many stages settle on some extension of $\sigma0$. By construction $\sigma$ must be applied and expansionary at these stages.
	\end{proof}
	
	Define $J$ and $u$ as in the proof of Theorem~\ref{satz:fast-berechenbar-aufholend-approximierbar}. We continue by noting that Fact~\ref{sdlakjldfgahrjehrfhksbfasjd234}, Lemma~\ref{satz:FBNBAA:lem:spruenge-expansionary}, Fact~\ref{remark:spruenge-expansionary-weak}, Lemma~\ref{satz:FBNBAA:lem:spruenge-threatened}, Fact~\ref{remark:spruenge-threatened-weak}, and 
	Corollary~\ref{jsjhfarjslkdgssfg} again carry over to this construction with literally the same proofs. The proof of Lemma~\ref{sdfjkasdjlkfgjkleknjjxvc} needs to be modified as follows, because here we use different rules for increasing witnesses.	
	\begin{lem}\label{sdfjkasdjlkfgjkleknjjxvcdfdfgdfg}
		Let $I \subseteq J$, let $\sigma \in \SigmaS$ and let $t_0 \leq \min u(I)$. Then we have
		\begin{equation*}
			\sum_{\mathclap{t' \in u(I)\colon \truepath[t']=\sigma}}
			\; 2^{-w(\sigma)[t']} \leq 2^{-w(\sigma)[t_0]+1}.
		\end{equation*}
	\end{lem}
	\begin{proof}
		By definition of $u$, all the numbers $t'$ that appear in the sum on the left-hand side are stages where $\sigma$ is applied and threatened. By construction, between two such stages,  the value of $\sigma$'s witness grows by at least $1$.
		Thus,	
		\begin{equation*}
			\sum_{\mathclap{t' \in u(I)\colon \truepath[t']=\sigma}}
			\; 2^{-w(\sigma)[t']}
			\leq \sum_{k=0}^{\infty} \;
			2 ^{-(w(\sigma)[t_0]+k)}
			\leq 2^{-w(\sigma)[t_0]+1}. \qedhere
		\end{equation*}
	\end{proof}
	Using these tools, we can prove the following statement.
	\begin{prop}
		The sequence $(x_t)_t$ is computable, non-decreasing, and bounded from above. Thus, its limit $x := \lim_{t\to\infty} x_t$ is a left-computable number.
	\end{prop}
	
	\begin{proof}
		The proof is literally the same as that of Proposition~\ref{satz:FBNBAA:prop:konvergenz}, except with Lemma~\ref{sdfjkasdjlkfgjkleknjjxvcdfdfgdfg} in place of Lemma~\ref{sdfjkasdjlkfgjkleknjjxvc}.
	\end{proof}

The following is analogous to Proposition~\ref{sdfjlw3ljsdadfgjrfasdfdfg}, but allows for an easier proof.
\begin{prop}\label{satz:FBNAA:lem03}
For every $l \in \IN$, there exists a binary string $\sigma_l \in \Sigma^{l}$ satisfying the following conditions:
\begin{enumerate}
	\item There exist infinitely stages in which $\sigma_l$ is applied.
	\item There exist only finitely many stages in which $\sigma_l$ is initialized.
\end{enumerate}
In particular, the map $t \mapsto \left|\truepath[t]\right|$ is unbounded, and if we let 
$\truepath$ denote the true path of $(\truepath[t])_t$ then  conditions (1) and (2) hold for every $\sigma \prefix \truepath$.
\end{prop}
\begin{proof} 
	We proceed by induction. The claim trivially holds for $l = 0$ since $\lambda$ is applied in every stage but never initialized.
	Assume that for some fixed $l \in \IN$ there exists a strategy $\sigma_l \in \Sigma^l$ satisfying both conditions. Let $t_0 \in \IN$ be the earliest stage after which $\sigma_l$ is no longer initialized. 
	There are two cases:
	\begin{itemize}
		\item {\em $\sigma_l0$ is applied infinitely often:}
		We claim $\sigma_l0$ is never initialized after $t_0$; this is because by construction any initialization of $\sigma_l0$ would have to occur in a stage where some $\gamma <_L \sigma_l0$ is applied; this would also initialize $\sigma_l$, contradiction. Thus we can choose $\sigma_{l+1} := \sigma_l 0$.
		
		\item {\em $\sigma_l0$ is applied only finitely often:}
		Due to Lemma~\ref{fghsdfdhdfsdgdfghgfsfsdf} in this case there is a stage $t_1 \geq t_0$ after which $\sigma_l$ is never again applied and expansionary. Then, by construction, $\sigma_l1$ is never again initialized after $t_1$.
 We also claim that for every $t_2 \geq t_1$ at which $\sigma_l$ is applied there exists a stage $t_3\geq t_2$ at which $\sigma_l1$ is applied. If $\sigma_l$ is not threatened at $t_2$ then by construction $\sigma_l1$ is applied at $t_2$, and we are done. Otherwise, if $\sigma_l$ is threatened at $t_2$, then by Fact~\ref{sdfjhasdfkjldfghjkasjlersdvsdf} we have that $\sigma_l$ cannot be threatened at the next stage $t_3 > t_2$ where $\sigma_l$ is applied. Thus $\sigma_l1$ is applied at $t_3$. 
Consequently, we can choose $\sigma_{l+1} := \sigma_l 1$.
	\end{itemize}
The second part of the proposition is proven in literally the same way as in the proof of Proposition~\ref{sdfjlw3ljsdadfgjrfasdfdfg}.
\end{proof}

The following lemma demonstrates that the dynamics that result from the way the construction was set up work as intended.
\begin{lem}\label{dfjkhasdjldfgmjasddh}
	Let $e$ be such that $\varphi_e$ is total and increasing and let $\sigma \prefix\truepath$ with $|\sigma|=e$. Then the following statements hold:
	\begin{enumerate}
		\item There exist infinitely many stages at which $\sigma$ is applied and threatened, and in particular, $(w(\sigma)[t])_t$ tends to infinity. Furthermore, if $t_0$ is a stage after which $\sigma$ is never again initialized, then for ${m := w(\sigma)[t_0]}$ we have that every $k \geq m$ is an element of $(w(\sigma)[t])_t$.
		
		\item There exist infinitely many stages at which $\sigma$ is applied and expansionary.
		In particular, $(r(\sigma)[t])_t$ tends to infinity.
	\end{enumerate}
\end{lem}
\begin{proof} \; \nopagebreak
\begin{enumerate}
	\item 	 By Proposition~\ref{satz:FBNAA:lem03}, $\sigma$ is applied infinitely often and there is a stage $t_0$ after which $\sigma$ is not initialized anymore. Then, for $t> t_0$ we can only have $w(\sigma)[t+1] > w(\sigma)[t]$ if $\sigma$ is applied and threatened at $t$. 
	Recall that the assumption that $\varphi_e$ is total and increasing is equivalent to the statement that 
$(\ell(e)[t])_t$ tends to infinity. 
	Thus, by the definition of a threatened stage, by the fact that $(x_t)_t$ converges, and using Fact~\ref{sdfjhasdfkjlersdvsdfdfgdfssd}, for every $t_1> t_0$ there exists a $t_2 \geq t_1$ at which $\sigma$ is applied and threatened and such that $w(\sigma)[t_1+1] = w(\sigma)[t_1]+1$. This is enough to establish all three claims.
	
	\item Let $t_1 \in \IN$ be arbitrary. 
	By (1), there are infinitely many stages at which $\sigma$~is applied and threatened. Thus, by Fact~\ref{sdfjhasdfkjldfghjkasjlersdvsdf}, there must also be infinitely many stages, where $\sigma$ is applied and not threatened. 
	Therefore, and since $(x_t)_t$ converges, there is a smallest $t_2 \geq t_1$ such that for all 
	$t \geq t_2$ we have 
	${x_t - x_{\varphi_{e}(\ell(e)[t])} < 2^{-r(\sigma)[t_1]}}$
	and at which $\sigma$ is applied, not threatened, and by by definition expansionary.
	Thus, since $t_1$ was arbitrary, there are infinitely many stages at which $\sigma$ is applied and expansionary. Then by Lemma~\ref{fghsdfdhdfsdgdfghgfsfsdf}, $\sigma0$ is applied at infinitely many stages, and by construction, $(r(\sigma)[t])_t$ tends to infinity.\qedhere
\end{enumerate}	
\end{proof}

Noting that Lemma~\ref{satz:FBNBAA:lem07} also holds for this construction with literally the same proof as before, we are ready to prove that all negative requirements are satisfied.
\begin{prop}
	$\mathcal{N}_e$ is statisfied for all $e \in \IN$.
\end{prop}
\begin{proof} 
	Let $e$ be such that $\varphi_e$ is total and increasing, and let $\sigma \prefix\truepath$ with $|\sigma|=e$. 	Let $t_0 \in \IN$ be the earliest stage after which $\sigma$ is not initialized anymore. Let $m$ be as in Lemma~\ref{dfjkhasdjldfgmjasddh}~(1), and let $n \geq m$ be arbitrary. Then there exists a uniquely determined stage $t_1 \geq t_0$ at which $\sigma$ is applied and threatened and such that $w(\sigma)[t_1] = n$. Let $t_2 > t_1$ be the next stage at which $\sigma$ is applied. Then 
	\[
		x - x_{\varphi_{e}(n)} = x - x_{\varphi_{e}(w(\sigma)[t_1])} \\
		\geq x_{t_2} - x_{\varphi_{e}(\ell(e)[t_1])}  \\
		\geq   2^{-w(\sigma)[t_1]} \\
		= 2^{-n};
	\]
	here, the first inequality uses the fact that 
	$\ell(e)[t_1] \geq w(\sigma)[t_1]$  by definition of a threatened stage, and the second inequality uses Lemma~\ref{satz:FBNBAA:lem07}.

	In summary, there exists a number $m \in \IN$ such that for all $n \geq m$ we have $x - x_{\varphi_{e}(n)} \geq 2^{-n}$. Therefore,  $\mathcal{N}_e$ is satisfied.
\end{proof}
\begin{kor} 
	The number $x$ is not regainingly approximable.\qed
\end{kor}

It remains to prove that $x$ is nearly computable. 
Write 
\[S := \{e \in \IN \mid \varphi_{e} \text{ is total and increasing}\}.\] The following lemma is analogous to Lemma~\ref{satz:FBNBAA:lem08}; however, the infinitary nature of the negative requirements and the different initialization strategy used in this construction are reflected in the higher complexity of its statement and proof.
\begin{lem}\label{satz:FBNAA:lem09}
	 Let $\sigma \prefix \truepath$ and $t_0 \in \IN$ be the earliest stage after which $\sigma$ is no longer initialized and after which there are no more stages at which any $\tau \prefixeq \sigma$ with $\left|\tau\right| \notin S$ is applied and threatened. Let $t_1, t_2 \in \IN$ with $t_0 \leq t_1< t_2$ be two consecutive stages at which 
	$\sigma$ is applied and expansionary. Then we have
	\begin{equation*}
		x_{t_2} - x_{t_1} \leq 2^{-r(\sigma)[t_1] + 1} + \sum_{\mathclap{\substack{\phantom{,}\tau \prefixeq \sigma, \\ \left|\tau\right| \in S}}} 2^{-w(\tau)[t_1]+1}.
	\end{equation*}
\end{lem}
As in the proof of Lemma~\ref{satz:FBNBAA:lem08}, we need to analyze all possible causes for jumps being made between stages $t_1$ and $t_2$. Again there are essentially two such causes: we may still have jumps scheduled from before stage $t_1$ that await execution, and new threats may arise after $t_1$ leading to further jumps. 
The most important difference to Lemma~\ref{satz:FBNBAA:lem08} is however that here initial segments $\tau$ of $\sigma$ with $\left|\tau\right| \in S$ may also get threatened --- possibly multiple times --- leading to additional jumps that need to be accounted for by the sum in the statement of the lemma.
\begin{proof}
	We again distinguish between obligations to make jumps that may already stand at the moment when we enter stage $t_1$, and new obligations to make jumps that are created between stages $t_1$ and $t_2$.
	First, to quantify jumps due to potential standing obligations, note that in substages $0$ up to $|\sigma|-1$ of stage $t_1$, no jumps are made by construction. 
	Now consider the following facts:
	\begin{enumerate}
		\item No strategy $\tau <_L \sigma$ will be applied again after stage $t_1$, as otherwise this would lead to initialization of $\sigma$, which contradicts our assumptions. Thus, if any jumps are still scheduled for any such $\tau$ (that is, if $c(\tau)[t_1]>0$), we need not take them into account.
		
		\item For strategies $\tau \in \SigmaS$ with 
		$\tau 0 \prefixeq \sigma$ and $|\tau|\notin S$
		we must have  $c(\tau)[t_1] = 0$ due to Fact~\ref{fact:counters-on-expansionary-stages}. Furthermore,  $\tau$ will never be applied and threatened between stages $t_1$ and $t_2$ by our assumptions. Thus we need not consider these strategies in the remainder of the proof.
		
		\item For strategies $\tau \in \SigmaS$ with 
		$\tau 0 \prefixeq \sigma$ and $|\tau|\in S$
		we must have ${c(\tau)[t_1] = 0}$, as well. However, such a $\tau$ may get threatened between stages $t_1$ and $t_2$ which might lead to jumps that we need to take into account.
				
		\item While for strategies $\tau \in \SigmaS$ with $\tau1 \prefixeq \sigma$ we might have $c(\tau)[t_1]>0$, none of these scheduled jumps will ever be executed from $t_1$ onwards. This is because
		such a jump could only be executed during a stage at which $\tau$ is applied and expansionary. This would by construction lead to the initialization of all $\rho$ with $\tau0 <_L \rho$, which in particular include $\sigma$; contradiction. Thus, as before, we need not take such $\tau$'s into account.
		
		\item No strategy $\tau$ with $ \sigma0 <_L \tau$ can be applied at stage $t_1$ by construction. While such $\tau$'s may be applied during some stage $t_1 < t < t_2$, note that we set $c(\tau)[t_1+1]=0$ for all of them when they are initialized at the end of stage $t_1$.
		
		\item For a strategy $\tau$ with $\sigma0 \prefixeq \tau$ we may indeed have $c(\tau)[t_1] > 0$.
		
		\item Similarly, it might be the case that $c(\sigma)[t_1] > 0$.
	\end{enumerate}
	Thus, any jumps made in stages $t_1$ up to $t_2 -1$ are either caused by some strategy as in~(6) or~(7), or must be due to \textit{new} threats to strategies of type~(3) or~(5) that occur strictly between substage~$|\sigma|$ of stage~$t_1$ and stage $t_2$. We establish upper bounds for all of these cases:
	\begin{itemize}
		\item Concerning strategies of type~(6), by construction, if we make any jump at all while applying them, then these can only occur immediately at stage $t_1$ in substage $|\tau|$.
		We claim that such a jump can not occur due to  $\tau$ being applied and expansionary at $t_1$ with ${c(\tau)[t_1] > 0}$. This is because such a jump would not be made at $t_1$, but would by construction instead be split and scheduled for later execution by some strategy $\rho$ with $\sigma0 \prefixeq \rho0 \prefixeq \tau$, and this strategy $\rho$ will not be applied before stage $t_2$ by construction.
		Thus, the only reason to make a jump in this case is if
		$\tau$ is applied and threatened at $t_1$. Then, by construction the jump made would be of size 
		\[2^{-w(\tau)[t_1]} < 2^{-r(\sigma)[t_1+1]} \leq 2^{-r(\sigma)[t_1]}.\]
		
		\item Concerning~(7), in this case, by construction, we would like to make a jump of size $2^{-r(\sigma)[t_1]}$. The jump may be made immediately, or split and scheduled for later execution by some strategies that are prefixes of $\sigma$; but in either case the total sum of all jumps resulting from this is bounded by $2^{-r(\sigma)[t_1]}$ by Lemma~\ref{satz:FBNBAA:lem:spruenge-expansionary}.
		
		\item  Consider new threats to strategies $\tau$ of type~(5) that occur strictly between substage $|\sigma|$ of stage $t_1$ and stage $t_2$.
		Since these $\tau$'s were all initialized at stage $t_1$, we have $w(\tau)[t_1+1] = \nu(\tau) + t_1 + 2$ for each of them.
		Thus, 
		applying Lemma~\ref{sdfjkasdjlkfgjkleknjjxvcdfdfgdfg} to each such $\tau$ and then summing over all of them, the total sum of all jumps made or scheduled when such $\tau$'s are applied in stages between $t_1$ and $t_2$ can be at most $2^{-t_1}$, which by Fact~\ref{satz:FBNBAA:lem03} is at most~$2^{-r(\sigma)[t_1]}$.	
		
		\item Finally, consider new threats to strategies $\tau$ of type~(3) that occur at some stage $t$ strictly between substage $|\sigma|$ of stage $t_1$ and stage $t_2$. By Fact~\ref{remark:spruenge-threatened-weak} 		
		each such threat can only contribute jumps totaling at most $2^{-w(\tau)[t]}$. And, by construction, whenever this occurs, we set $w(\tau)[t+1]=w(\tau)[t]+1$. Thus, even if the same $\tau$ is threatened multiple times between $t_1$ and $t_2$, the total sum of all jumps associated with these threats sums to at most~${2^{-w(\tau)[t_1]+1}}$.		
	\end{itemize}
By construction, cases~(6) and~(7) exclude each other, thus the jumps caused by cases (5)--(7) combined contribute at most $2^{-r(\sigma)[t_1]+1}$. Together with the maximally possible contribution of all $\tau$'s of type~(3) we obtain \begin{equation*}
	x_{t_2} - x_{t_1} \leq 2^{-r(\sigma)[t_1] + 1} + \sum_{\mathclap{\substack{\phantom{,}\tau \prefixeq \sigma, \\ \left|\tau\right| \in S}}} 2^{-w(\tau)[t_1]+1}.\qedhere
\end{equation*}
\end{proof}

	After these preparations, we can show that all positive requirements are satisfied.
	\begin{prop}
		$\mathcal{P}_e$ is statisfied 	for all $e \in \IN$.
	\end{prop}
	\begin{proof} 
		Let $e \in \IN$ and $\sigma \prefix \mathcal{T}$ with $\left| \sigma \right| = e$. Suppose that $\varphi_{e}$ is  total and increasing. Then, by Lemma~\ref{dfjkhasdjldfgmjasddh}~(2), $(r(\sigma)[t])_{t}$ tends to infinity. Let $t_0 \in \IN$ be the earliest stage after which $\sigma$ is no longer initialized and such that, for all $\tau \prefixeq \sigma$ with $\left|\tau\right| \notin S$, there are no more stages at which $\tau$ is applied and threatened.
		For $n\in\IN$, write  
		\[t(n):=\min\left\{t \geq t_0\colon\; 
			\parbox{9cm}{\centering
				$r(\sigma)[t] \geq n+3$ and $\sum_{\tau \prefixeq \sigma, \left|\tau\right| \in S} 2^{-w(\tau)[t] + 1} \leq 2^{-(n+1)}$\\
				and $\sigma$ is applied and expansionary at stage $t$
			}
		\right\}.\]
		
		Note that for each $\tau \prefixeq \sigma$ with $\left|\tau\right| \in S$ we have that $(w(\tau)[t])_t$ tends to infinity by Lemma~\ref{dfjkhasdjldfgmjasddh}~(1), and thus $t(n)$ is defined for every $n$.
		Thus, we can define a function $v \colon  \IN \to \IN$ via ${v(n) = \ell(e)[t(n)]}$ for all $n\in\IN$. Clearly, $t$ and $v$ are computable. 
		
	 We claim that $v$ is a modulus of convergence of the sequence $(x_{\varphi_{e}(t+1)} - x_{\varphi_{e}(t)})_t$. 
	 To see this, let $n\in \IN$ and $i \geq v(n)$. Let $t_2 > t_1 \geq t(n)$ be the two consecutive stages at which $\sigma$ is applied and expansionary with $\ell(e)[t_1] \leq i$ and $\ell(e)[t_2] \geq i+1$. Applying Lemma~\ref{satz:FBNAA:lem09} and the assumption that $\sigma$ is expansionary at $t_1$, we obtain	 
		\begin{align*}
			x_{\varphi_{e}(i+1)} - x_{\varphi_{e}(i)} &\leq
			x_{\varphi_{e}(\ell(e)[t_2])} - x_{\varphi_{e}(\ell(e)[t_1])} \\
			&= \underbrace{\left( x_{\varphi_{e}(\ell(e)[t_2])} - x_{t_2} \right)}_{\leq 0} + \left( x_{t_2} - x_{t_1} \right) + \left( x_{t_1} - x_{\varphi_{e}(\ell(e)[t_1])} \right) \\
			&\leq  \left(2^{-r(\sigma)[t_1]+1} + \sum_{\substack{\tau \prefixeq \sigma, \\ \left|\tau\right| \in S}} 2^{-w(\tau)[t_1]+1}\right) 
			+ 2^{-r(\sigma)[t_1]} \\
			&\leq  2^{-r(\sigma)[t(n)]+1} + 2^{-(n+1)} + 2^{-r(\sigma)[t(n)]} \\
			&< 2^{-r(\sigma)[t(n)]+2} + 2^{-(n+1)} \\
			&\leq 2^{-n}.
		\end{align*}
		Thus $(x_{\varphi_{e}(t+1)} - x_{\varphi_{e}(t)})_t$ converges computably to zero, and $\mathcal{P}_e$ is satisfied.	
	\end{proof}
	
	\begin{kor}
		The number $x$ is nearly computable.\qed
	\end{kor}
	
This completes 
	the proof of the existence of a left-computable, nearly computable number that is not regainingly approximable.
	\phantom\qedhere
\end{proof}
% !TeX spellcheck = en_US
\section{Non-speedable non-randoms}

In this final section we prove our third main result by showing that there exists a left-computable number that is neither speedable nor Martin-Löf random; thereby giving a negative answer to the question of Merkle and Titov~\cite{MT2020} whether among the left-computable numbers the Martin-Löf randoms are characterized by being non-speedable. The understanding of the relationship between near computability and regaining approximability that we gained in the last section will be instrumental for this.
We begin by establishing a more convenient characterization of speedability.
\begin{lem}\label{lem:characterisierung-speedable}
	A left-computable number is speedable if and only if there exists a constant $\rho \in \left] 0, 1 \right[ $ and a computable increasing sequence $(x_n)_n$ of rational numbers converging to $x$ such that there are infinitely many $n \in \IN$ with $\frac{x_{n+1} - x_n}{x - x_n} \geq \rho$.
\end{lem}
\begin{proof}
	Suppose that $x$ is speedable, that is, by definition, there exists a constant $\rho' \in \left] 0, 1\right[$ and a computable increasing sequence $(x_n)_n$ converging to $x$ with $\frac{x - x_{n+1}}{x - x_n} \leq \rho'$ for infinitely many $n \in \IN$. 
	If we let $\rho := 1 - \rho'$, this is equivalent to $\frac{x_{n+1} - x_n}{x - x_n} = 1 - \frac{x - x_{n+1}}{x - x_n} \geq \rho$ for infinitely many $n \in \IN$.
\end{proof}
Thus, a left-computable number $x$ is non-speedable if and only if for every computable increasing sequence of rational numbers $(x_n)_n$ converging to~$x$ the sequence $\left(\frac{x_{n+1} - x_n}{x - x_n}\right)_n$ converges to zero. Before we use this characterization we first examine the relationship between speedable and regainingly approximable numbers.
\begin{prop}\label{prop:aufholend-approximierbar-impliziert-beschleunigbar}
	Every regainingly approximable number is speedable.
\end{prop}
\begin{proof}
	Let $x$ be regainingly approximable. Then, by definition, there exists a computable non-decreasing sequence of rational numbers $(x_n)_n$ converging to $x$ with $x - x_n < 2^{-n}$ for infinitely many $n \in \IN$. Define the sequence $(y_n)_n$ by $y_n := x_n - 2^{-n}$ for all $n \in \IN$. This sequence is computable, increasing and also converges to $x$. Then, for every $n \in \IN$ with $x - x_n < 2^{-n}$, we have
	\begin{equation*}
		\frac{y_{n+1} - y_n}{x - y_n} = \frac{\left(x_{n+1} - x_n\right) + 2^{-(n+1)}}{\left(x - x_n\right) + 2^{-n}} > \frac{2^{-(n+1)}}{2^{-n} + 2^{-n}} = \frac{1}{4}.\qedhere
	\end{equation*}
\end{proof}
The converse is not true as the two following known results imply.
\begin{prop}[Merkle, Titov \cite{MT2020}]
	Every strongly left-computable number is speedable.
\end{prop}
\begin{theorem}[Hertling, Hölzl, Janicki \cite{HHJ2023}]\label{satz:SLBNAA}
	There exists a strongly left-computable number that is not regainingly approximable.
\end{theorem}
\begin{kor}\label{dfsdjhhdbfajfkjgsdjfhdasnfbsafevcvssda}
	There exists a speedable number that is not regainingly approximable.\qed 
\end{kor}

Thus, in general, the regainingly approximable numbers are a proper subset of the speedable numbers. However, as we will show next, the two notions become equivalent once we restrict ourselves to nearly computable numbers.
The proof of this is not straightforward and needs some preparation.
Let us first recall a convenient characterization of the left-computable, nearly computable numbers.
\begin{prop}[Hertling, Janicki~\cite{HJ2023}]\label{prop:characterisierung-lnc}
	The following statements are equivalent for a left-computable number~$x$:
	\begin{enumerate}
		\item $x$ is nearly computable.
		\item For every computable non-decreasing sequence of rational numbers $(x_n)_n$ converging to $x$, the sequence $(x_{n+1} - x_n)_n$ converges computably to zero.
	\end{enumerate}
\end{prop}
As this characterization employs the concept of 
computable convergence, it will be helpful to have a way of expressing regaining approximability in a similar vein.
\begin{prop}\label{prop:aufholend-approximierbar-aequivalenz}
	Let $x$ be a left-computable number. Let $(x_n)_n$ be a computable non-decreasing sequence of rational numbers converging to $x$ and let $g \colon \IN \to \IN$ be its optimal modulus of convergence. Then the following statements are equivalent:
	\begin{enumerate}
		\item $x$ is regainingly approximable.
		\item There exists a computable function $f \colon \IN \to \IN$ such that $f(n) \geq g(n)$ for infinitely many $n \in \IN$.
	\end{enumerate}
\end{prop}
\begin{proof} 
	``$(1) \Rightarrow (2)$'': Suppose that $x$ is regainingly approximable. Due to Proposition~\ref{prop:charakterisierung-aa}, there is a computable increasing function $f\colon \IN \to \IN$ with 
	${x - x_{f(n)} < 2^{-n}}$
	for infinitely many $n \in \IN$. By definition of an optimal modulus of convergence we must have $f(n) \geq g(n)$ for these $n$.
	
	\smallskip
	
	``$(2) \Rightarrow (1)$'': Suppose that there exists a computable function $f\colon \IN \to \IN$ with $f(n) \geq g(n)$ for infinitely many $n \in \IN$ and assume w.l.o.g.\ that $f$ is increasing. Then for every $n \in \IN$ with $f(n) \geq g(n)$ we have
		$x - x_{f(n)} \leq x - x_{g(n)} < 2^{-n}$.
\end{proof}
As usual we say for functions $f, g\colon \IN \to \IN$ that \emph{$f$ dominates $g$} if ${f(n) \geq g(n)}$ for almost all $n \in \IN$.
Now recall that in Theorem~\ref{satz:FBNAA} we have shown that there are left-computable numbers $x$ that are nearly computable but not regainingly approximable. Then, informally speaking, Proposition~\ref{prop:aufholend-approximierbar-aequivalenz} 
states that every computable non-decreasing sequence of rational numbers $(x_n)_n$ converging to such an~$x$ must converge very slowly in the sense that its optimal modulus of convergence dominates every computable function.
In contrast to this, since $x$ is nearly computable, for every such $(x_n)_n$, the sequence $(x_{n+1} - x_n)_n$ converges computably to zero, thus its optimal modulus of convergence is dominated by a computable function.

Putting these observations together suggests 
that, for every $x$ as in Theorem~\ref{satz:FBNAA} and any computable increasing sequence of rational numbers $(x_n)_n$ converging to~$x$, the sequence $\left(\frac{x_{n+1} - x_n}{x - x_n}\right)_n$ ought to converge to zero, due to its numerator converging much faster than its denominator. The following lemma confirms this intuition. 
\begin{lem}\label{lem:quotient-von-zwei-nullfolgen-konvergenzmodule}
	Let $(a_n)_n$ be a sequence and $(b_n)_n$ be a decreasing sequence, both converging to zero. Let $f \colon  \IN \to \IN$ be a non-decreasing and unbounded function that is a modulus of convergence of $(a_n)_n$, and let $g \colon  \IN \to \IN$ be the optimal modulus of convergence of $(b_n)_n$. Suppose that for all $k \in \IN$ there exists an $m \in \IN$ such that for all $n \geq m$ the inequality $f(n+k) < g(n)$ holds. 
	Then \[\lim_{n\rightarrow\infty}\left(\frac{a_n}{b_n}\right)_{\!\!n}\!=0.\] 
\end{lem}
\begin{proof}
	Let $\varepsilon > 0$. Fix a $k \in \IN$ with $2^{-k} < \frac{\varepsilon}{2}$ and an $m \in \IN$ such that the inequality $f(n+k) < g(n)$ is satisfied for all $n \geq m$. Choose $l := f(m+k)$, and let $i \geq l$. Since $f$ is non-decreasing and unbounded, there exists a number $n \in \IN$ with $f(n+k) \leq i \leq f(n+1+k)$. Then, since $f$ is a modulus of convergence of~$(a_n)_n$, we have $|a_i|<2^{-(n+k)}$. Furthermore $i < g(n+1)$, and as $(b_n)_n$ is a decreasing sequence converging to zero and $g$ its optimal modulus of convergence, we also have $b_i \geq 2^{-(n+1)}$. Putting everything together we obtain
\[
		\left|\frac{a_i}{b_i}\right|  = \frac{\left|a_i\right|}{b_i}                                                            < 
        \frac{2^{-(n+k)}}{2^{-(n+1)}} 
        = 2 \cdot 2^{-k}                                                                             < 2 \cdot \frac{\varepsilon}{2}                                                           = \varepsilon.
\]
	Thus, $\left(\frac{a_n}{b_n}\right)_n$\! converges to zero.
\end{proof}

After this preparation we are ready to prove our final theorem.
\begin{theorem}\label{dsfjknsdkjldfgljsdfdfgdfterter}
	Let $x$ be a left-computable number that is nearly computable but not regainingly approximable. Then $x$ is not speedable.
\end{theorem}
\begin{proof}
	Let $(x_n)_n$ be a computable increasing sequence of rational numbers converging to $x$. Then the sequence $(x_{n+1} - x_n)_n$ converges to zero and, as $x$ is nearly computable, has a computable modulus of convergence $f \colon  \IN  \to \IN$.
	Since $x$ is not regainingly approximable it is in particular non-computable, and thus $f$ must be unbounded; w.l.o.g.\ we also assume it to be non-decreasing.
	Let $g \colon  \IN \to \IN$ be the optimal modulus of convergence of $(x_n)_n$. Since $x$ is not regainingly approximable there is no computable function $h \colon  \IN \to \IN$ with $h(n) \geq g(n)$ for infinitely many~$n\in \IN$, due to Proposition \ref{prop:aufholend-approximierbar-aequivalenz}. 
	In particular, for every constant $k \in \IN$ there exists a number $m \in \IN$ such that for all $n \geq m$ the inequality $f(n+k) < g(n)$ holds.
	Note that $(x - x_n)_n$ is a decreasing sequence that converges to zero and $g$ is its optimal modulus of convergence. Thus, the premises of Lemma~\ref{lem:quotient-von-zwei-nullfolgen-konvergenzmodule} are satisfied, and $\left(\frac{x_{n+1} - x_n}{x - x_n}\right)_n$ converges to zero. Then, by Lemma~\ref{lem:characterisierung-speedable}, $x$~is not speedable.
\end{proof}
Putting together Proposition~\ref{prop:aufholend-approximierbar-impliziert-beschleunigbar} and Theorem~\ref{dsfjknsdkjldfgljsdfdfgdfterter}, we obtain the following corollary.
\begin{kor}
	Let $x$ be a left-computable number that is nearly computable. Then the following statements are equivalent:
	\begin{enumerate}
		\item $x$ is regainingly approximable.
		\item $x$ is speedable.\qed
	\end{enumerate}
\end{kor}
Together with Theorem~\ref{satz:FBNAA} we obtain the next corollary.
\begin{kor}\label{kor:LCNCNS}
	There exists a left-computable number which is nearly computable but not speedable.\qed
\end{kor}
Stephan and Wu~\cite{SW2005} showed that a left-computable number which is nearly computable cannot be Martin-Löf random. Together with Corollary~\ref{kor:LCNCNS} this implies our final main result, a negative answer to the question of Merkle and Titov~\cite{MT2020}.
\begin{kor}
	There exists a left-computable number which is not speedable and not Martin-Löf random.\qed
\end{kor}

\section{Acknowledgments}

The authors would like to thank Ivan Titov for helpful discussions.

\bibliography{cca}
\bibliographystyle{abbrv}

\end{document}
