% \documentclass{article}

% to compile a preprint version, e.g., for submission to arXiv, add add the
% [preprint] option:
% \usepackage[preprint]{neurips_2019}
\documentclass[runningheads]{llncs}
% \usepackage[subtle]{savetrees}
\usepackage{graphicx}
\usepackage{booktabs}
\usepackage{adjustbox}
\usepackage[colorlinks=true,citecolor=blue,anchorcolor=purple]{hyperref}\usepackage{url}
\usepackage{multirow}
\usepackage{amsmath}
\usepackage{amssymb}
\usepackage{mathtools}
\usepackage{caption}
\usepackage{wrapfig}
\usepackage{lipsum}
\usepackage{amsfonts}
\usepackage{pifont}
\usepackage{fdsymbol}
\newcommand{\cmark}{\ding{51}}%
\newcommand{\xmark}{\ding{55}}%
\def \Em{{\mathbb{E}}}
\def \Rm{{\mathbb{R}}}

\def \Im{{\mathbb{I}}}
\def \abf{{\mathbf a}}
\def \Abf{{\mathbf A}}
\def \bbf{{\mathbf b}}
\def \Bbf{{\mathbf B}}
\def \cbf{{\mathbf c}}
\def \Cbf{{\mathbf C}}
\def \dbf{{\mathbf d}}
\def \Dbf{{\mathbf D}}
\def \ebf{{\mathbf e}}
\def \Ebf{{\mathbf E}}
\def \fbf{{\mathbf f}}
\def \Fbf{{\mathbf F}}
\def \gbf{{\mathbf g}}
\def \Gbf{{\mathbf G}}
\def \hbf{{\mathbf h}}
\def \Hbf{{\mathbf H}}
\def \ibf{{\mathbf i}}
\def \Ibf{{\mathbf I}}
\def \jbf{{\mathbf j}}
\def \Jbf{{\mathbf J}}
\def \kbf{{\mathbf k}}
\def \Kbf{{\mathbf K}}
\def \lbf{{\mathbf l}}
\def \Lbf{{\mathbf L}}
\def \mbf{{\mathbf m}}
\def \Mbf{{\mathbf M}}
\def \nbf{{\mathbf n}}
\def \Nbf{{\mathbf N}}
\def \obf{{\mathbf o}}
\def \Obf{{\mathbf O}}
\def \pbf{{\mathbf p}}
\def \Pbf{{\mathbf P}}
\def \qbf{{\mathbf q}}
\def \Qbf{{\mathbf Q}}
\def \rbf{{\mathbf r}}
\def \Rbf{{\mathbf R}}
\def \sbf{{\mathbf s}}
\def \Sbf{{\mathbf S}}
\def \tbf{{\mathbf t}}
\def \Tbf{{\mathbf T}}
\def \ubf{{\mathbf u}}
\def \Ubf{{\mathbf U}}
\def \vbf{{\mathbf v}}
\def \Vbf{{\mathbf V}}
\def \wbf{{\mathbf w}}
\def \Wbf{{\mathbf W}}
\def \xbf{{\mathbf x}}
\def \Xbf{{\mathbf X}}
\def \ybf{{\mathbf y}}
\def \Ybf{{\mathbf Y}}
\def \zbf{{\mathbf z}}
\def \Zbf{{\mathbf Z}}
\def \Hbf{{\mathbf H}}
\def \0bf{{\mathbf 0}}

\def \Emean{\mathbb{E}}

\def \Sibf{{\mathbf \Sigma}}
\def \xbbf{\mathbf{\bar{x}}}
\def \etr{\mbox{etr}}
\def \tr{\mbox{tr}}
\def \Tr{\mbox{Tr}}
\def \Cov{\mbox{Cov}}
\def \cost{\mbox{cost}}
\def \diag{\mbox{diag}}
\def \Lambf{{\mathbf{\Lambda}}}
\def \Gambf{{\mathbf{\Gamma}}}
\def \Sigbf{{\mathbf \Sigma}}
\newcommand{\rhobf}{\ensuremath{\boldsymbol{\rho}}}
\newcommand{\lambf}{\ensuremath{\boldsymbol{\lambda}}}
\newcommand{\nubf}{\ensuremath{\boldsymbol{\nu}}}
%\newcommand{\alphabf}{\ensuremath{\boldsymbol{\alpha}}} 
\def \alphabf{{\boldsymbol{\alpha}}}

\def \Ncal{{\mathcal N}}
\def \Pcal{{\mathcal P}}
\def \Fcal{{\mathcal F}}
\def \Ecal{{\mathcal E}}
\def \Scal{{\mathcal S}}
\def \Qcal{{\mathcal Q}}
\def \Gcal{{\mathcal G}}
\def \Lcal{{\mathcal L}}
\def \KLD{\text{KLD}}
\def \KL{\text{KL}}

% \begin{titlepage}
\title{Operating critical machine learning models in resource constrained regimes}
% \end{titlepage}


% The \author macro works with any number of authors. There are two commands
% used to separate the names and addresses of multiple authors: \And and \AND.
%
% Using \And between authors leaves it to LaTeX to determine where to break the
% lines. Using \AND forces a line break at that point. So, if LaTeX puts 3 of 4
% authors names on the first line, and the last on the second line, try using
% \AND instead of \And before the third author name.

 % index{Last Name, First Name}
 \author{Raghavendra Selvan\inst{1,2} \and Julian Schön\inst{1} \and Erik B Dam~\inst{1} }
 % \author{ Raghavendra Selvan \inst{1,2} \and Nikhil Bhagwat\inst{3} \and Lasse F. Wolff Anthony\inst{1} \and {Benjamin Kanding}\inst{1} \and {Erik B. Dam}\inst{1}}
 
\authorrunning{Selvan, Schön and Dam}
% \authorrunning{First author et al.}
\titlerunning{Operating critical ML with resource constraints}
% First names are abbreviated in the running head.
% If there are more than two authors, 'et al.' is used.
%
\institute{Department of Computer Science, University of Copenhagen \and
Department of Neuroscience, University of Copenhagen
\\
\email{raghav@di.ku.dk}
}

\begin{document}

\maketitle

\begin{abstract} 

	The accelerated development of machine learning methods, primarily deep learning, are causal to the recent breakthroughs in medical image analysis and computer aided intervention. The resource consumption of deep learning models in terms of amount of training data, compute and energy costs are known to be massive. These large resource costs can be barriers in deploying these models in clinics, globally. To address this, there are cogent efforts within the machine learning community to introduce notions of resource efficiency. For instance, using quantisation to alleviate memory consumption. While most of these methods are shown to reduce the resource utilisation, they could come at a cost in performance. In this work, we probe into the trade-off between resource consumption and performance, specifically, when dealing with models that are used in critical settings such as in clinics.\footnote{Source Code: \url{https://github.com/raghavian/redl}} 

\keywords{Resource efficiency \and Image Classification \and Deep Learning}

\end{abstract}

\section{Introduction}
Every third person on the planet does not {\em yet} have universal health coverage according to the estimates from the United Nations\cite{sdg2022}. And improving access to universal health coverage is one of the UN Sustainable Development Goals. The use of machine learning (ML) based methods could be instrumental in achieving this objective. Reliable ML in clinical decision support could multiply the capabilities of healthcare professionals by reducing the time spent on several tedious and time-consuming steps. This can free up the precious time and effort of healthcare professionals to cater to the people in need\cite{yu2018artificial}. 
% {\em The universal health coverage improved from a global average of 45 out of 100 in 2000 to 64 in 2015 and then 67 in 2019.}\cite{sdg2022}

The deployment of ML in critical settings like clinical environments is, however, currently in a nascent state. In addition to the fundamental challenges related to multi-site/multi-vendor validation, important issues pertaining to fairness, bias and ethics of these methods are still being investigated\cite{ricci2022addressing}. The demand for expensive material resources, in terms of computational infrastructure and energy requirements along with the climate impact of ML are also points of concern\cite{selvan2022carbon}.

In this work, we focus on the question of improving the resource efficiency of developing and deploying deep learning models for resource constrained environments~\cite{sze2017efficient,bartoldson2022compute}. Resource constraints in real world scenarios could be manifested as the need for: low latency (corresponding to urgency), non-specialized hardware (instead of requiring GPU/similar infrastructure) and low-energy requirements (if models are deployed on portable settings). The case for resource efficiency when {\em deploying} ML models is quite evident due to these aforementioned factors. In this work, we {\em also} argue that the training resource efficiency is important, as ML systems will be most useful in the clinical settings if they are able to continually learn~\cite{baweja2018towards,vokinger2021continual}. We validate the usefulness of several methods for improving resource efficiency of deep learning models using detailed experiments. We show that resource efficiency -- in terms of reducing compute time, GPU memory usage, energy consumption -- can be achieved without degradation in performance for some classes of methods.


% \begin{enumerate}
%     \item Not as critical in some applications
%     \item AI safety
%     \item High risk and can we reduce the accuracy knowingly
%     \item Helping to make decisions; not making the decisions
%     \item Get away with some performance drop
%     \item Risk/reliability measures
%     \item Guidelines are missing from FDA; if you want to update a model, before changing the model
%     \item Precision/recall scores
%     \item Regulatory perspective you are not harming anyone
%     \item Olympus colon detection system; detect the same polyps
%     \item Caution the trade-off
%     \item Recommendation: Don't use the systems; more of a political issue
%     \item Mammography screening
%     \item No feature on portable device; in ICUs, ambulances, time critical
%     \item More stationary; 
%     \item Clinical trial;
%     \item Too expensive to have a stationary system, with lot of down time
%     \item Accessibility
%     \item High sensitivity could burden the systems
%     \item Why DK won't use AI in Healthcare \footnote{\url{https://behandlingsraadet.dk/find-evalueringer-analyser/analyser/2021/kunstig-intelligens-som-beslutningsstotte-ved-koloskopiske-undersogelser-til-diagnosticering-af-neoplastisk-sygdom}} \footnote{\url{https://ugeskriftet.dk/nyhed/behandlingsradet-kommer-fra-start-med-gange-nej}}
% \end{enumerate}
% \section{Related Work}
% \subsection{Data frugality}
% \cite{zhaodataset}

\section{Methods for Resource Efficiency}
\label{sec:methods}
% Training and deploying of deep learning models incurs large resource costs which is also further amplified when dealing with large amounts of high resolution images as encountered in medical image analysis. 
In this work we mainly consider reducing the overall memory footprint of deep learning models, which in turn could reduce computation time and the corresponding energy costs. We do not explicitly study the influence of model selection using efficient neural architecture search~\cite{tan2019efficientnet} or model compression using pruning or similar methods~\cite{cheng2020compression}, as these techniques require additional design choices outside of the normal model development and deployment phases. By focusing on strategies that can be easily incorporated, we also place emphasis on the possible ease of clinical adoption.

Denote a neural network, $f_\Wbf(\cdot): \Xbf \rightarrow \Ybf$ which acts on inputs $\Xbf$ to predict the output $\Ybf$ and has trainable parameters $\Wbf$. For recent deep neural networks, the number of trainable parameters, $|\Wbf|$, can be quite large~\cite{sevilla2022compute} in $\mathcal{O}(100M)$. 

During training, the weights, $\Wbf$, are updated using some form of a gradient-based update rule. This requires the computation of the gradients of the loss $\Lcal(\cdot)$ with respect to $\Wbf$ at iteration $t$, $\gbf_t=\frac{\partial{\Lcal}}{\partial{\Wbf}}$. In the case of stateful optimisers that use momentum-based optimisation, such as Adam~\cite{kingma2014adam}, the first and second order statistics of the gradient over time, $\mbf_t,\vbf_t$, respectively, are maintained for improved convergence. Thus, for a neural network with $|\Wbf|$ parameters, at any point during the training there are an additional $\approx 3\cdot |\Wbf|$ variables stored in memory. Further, the intermediate activations at layer $l$, $\hbf_l$ are also stored in memory to efficiently perform backpropagation. Note that all the scalar entries of $\Wbf, \gbf_t, \mbf_t, \vbf_t, \hbf_l \in \Rm$. On most computers, these real numbers are discretised into floating point-32 (FP32) format; wherein each variable requires 32 bits.

Resource efficiency in this work is primarily addressed by reducing the precision of these variables by quantisation~\cite{fiesler1990weight,sze2017efficient}. The implication of using quantised or low-precision variables is that it not only reduces memory footprint but can also reduce the compute cost of matrix multiplications~\cite{hubara2017quantized} at the expense of loss in precision. Note, however, that the overhead of performing quantisation in some cases might outweigh the gains in computation.

In this work, we investigate a combination of the following three quantisation strategies:
\begin{enumerate}
    \item {\bf Gradients and intermediate activations}: Drastic quantisation of $\hbf_l, \gbf_t$ have been studied extensively in  literature~\cite{hubara2017quantized,chakrabarti2019backprop}. For instance, in~\cite{hubara2017quantized} the gradients and activations were quantised to 1-bit and still yielded reasonable performance compared to the 32-bit models. However, not all operations are well suited for quantisation. In frameworks like Pytorch~\cite{paszke2019pytorch}, the precision of $\hbf_l, \gbf_t$ are automatically cast into 16-bit precision, and recast to 32-bit when necessary. This technique known as {\tt automatic mixed precision} (AMP) can yield considerable reduction of memory costs~\cite{micikevicius2018mixed}.
    \item {\bf Optimiser states}: Until recently, the main focus of quantisation in deep learning has been on compressing $\gbf_t,\hbf_l,\Wbf$. In a recent work in~\cite{dettmers8}, it was shown that the optimiser states, $\mbf_t,\vbf_t$ can use up to 75\% of memory. By quantising optimiser states to 8-bits using a dynamic quantisation scheme, this work reduces the memory consumption during model training. 
    \item {\bf Model weights}: The resolution of the hypothesis space is controlled by the precision of the weights, $\Wbf$. Using lower precision weights for models can reduce their expressive power. This is most evident when a model trained in full precision is then cast into lower precision. In such cases, it is a common strategy to fine tune the quantised model for a few epochs to recover the loss in performance~\cite{nagel2021white}. However, when a model is trained from the outset in low-precision modes, they can achieve similar performances to models with full precision weights~\cite{sun2020ultra}.
\end{enumerate}

\section{Data \& Experiments}

Nation-wide screening for early detection of cancer is being attempted in many countries. For instance, early breast cancer detection using mammography screening has shown positive impact, both for the subjects and in allocating healthcare resources~\cite{daysal2022economic}. To simulate a scenario where ML models are deployed in such screening programs, we use a publicly available mammography screening dataset for the task of recommending follow-up scans. We study the influence of using resource optimised models and their impact on the predictions in the mammography screening.
%
\\
{\bf Datasets}: The experiments in this work were primarily performed using a subset of the data available as part of the Radiological Society of North America (RSNA) Mammography Breast Cancer Detection challenge\cite{rsna-breast-cancer-detection}\footnote{\url{https://www.kaggle.com/competitions/rsna-breast-cancer-detection/overview}. Accessed on 08/03/2023}. Subjects under the age of $55$ yr in the training set were used as the cohort of interest for early detection. This yielded a dataset comprising $3199$ unique subjects. As each subject has multiple scans (also from different views), the final dataset consists of $11856$ mammography scans. While the dataset provides multiple markers which could be used as labels, we focus on the BI-RADS\footnote{BI-RADS:Breast Imaging Reporting \& Data System} score\cite{orel1999birads}. The dataset provides three grades of the BI-RADS score: 0 if the breast required follow-up, 1 if the breast was rated as negative for cancer, and 2 if the breast was rated as normal. We combine the classes 1 and 2 to make a binary classification task of predicting if a follow-up scan was required or not. This resulted in the following label distribution: [$7903$, $3953$], corresponding to requiring follow-up (class-1) and no follow-up (class-0). The dataset was randomly split into training, validation and test sets based on subject identities. This resulted in a split of [1919,640,640] subjects in the three sets, respectively. In terms of mammography images, the resulting split was [7113,2371,2372] for the training, validation and test sets. Sample images from the dataset and the age distribution of the subjects is shown in Figure~\ref{fig:data}. We acknowledge that this classification task is a simplification of the mammography reading done during screening, but depending on the screening setup (e.g. number of readers), even this simple task could supplement the workflow. 

\begin{figure}[t]
    
    
\begin{minipage}{0.55\textwidth}
    \centering
\includegraphics[width=0.79\textwidth]{images/data.jpg}

(A)
\end{minipage}
\begin{minipage}{0.4\textwidth}
% \begin{figure}
    \centering
    \includegraphics[width=0.65\textwidth]{images/age_dist.pdf}
    
    (B)
\end{minipage}
    \vspace{-0.35cm}
\caption{(A): Sample mammography images from the RSNA training set showing the diverse acquisition and anatomical variations. (B): Age distribution of the subjects in the dataset used. Maximum age of $55$ yr was used. }
    \label{fig:data}
    \vspace{-0.5cm}
\end{figure}

Additional experiments on a second dataset for lung nodule detection are also reported for further validation. We use the LIDC-IDRI dataset which consists of 1018 thoracic CT scans and predict the presence or absence of tumours~\cite{armato2004lung}. Further details of the LIDC classification dataset are provided in the Supplementary Material.
%
\\
{\bf Classification models}: To evaluate the impact of resource constraints, different classes of deep learning models are used. Specifically, we use Densenet~\cite{huang2017densely}, Efficientnet~\cite{tan2019efficientnet}, Vision transformer~\cite{dosovitskiyimage} and Swin Transformer~\cite{liu2021swin}. All models were pretrained on ImageNet and fine tuned on the two datasets\footnote{Pretrained weights for the models were obtained from \url{https://timm.fast.ai/}}. %Specific details of model architectures are provided in Supplementary material. 
%
\\
{\bf Experimental design}: All the models were trained for a maximum of 50 epochs, with an early stopping patience of 5 epochs using batch size of 32. All the experiments were performed on a desktop workstation with Intel i7 processor, 32GB memory and Nvidia RTX 3090 GPU with 24 GB memory. All experiments were performed in
Pytorch (v1.7.1)~\cite{paszke2019pytorch}, and unless otherwise specified using the Adam~\cite{kingma2014adam} optimizer with learning rate of $10^{-5}$. Measurements of run time, energy and carbon consumption were performed using CarbonTracker~\cite{anthony2020carbontracker}. All models were trained three times with different random seeds. The training of models in this work used 25.0 kWh of electricity contributing to 4.7 kg of CO2eq.
% This is equivalent to 10.801 km travelled by car
%
\\
{\bf Experiments}: Resource constraints in real world scenarios could be due to: compute time (corresponding to low latency), specialized hardware (access to GPU infrastructure) and energy consumption (if models are deployed on portable devices). Using these and the test set performance as the axes, we run all the classification models with- and without- any resource optimisation. The three main optimisation techniques we use are related to the methods described in Section~\ref{sec:methods}. Firstly, we perform automatic mixed precision training of the models in order to obtain low compute and memory models~\cite{micikevicius2018mixed}. This is implemented using the {\tt automatic mixed precision} (AMP) feature in Pytorch. Secondly, we minimize the resource utilisation of the optimizer using an {\tt 8-bit optimiser} based on the implementation\footnote{8-bit optimiser from \href{https://github.com/TimDettmers/bitsandbytes}{bitsandbytes}} from~\cite{dettmers8}. Finally, we also investigate training of a low precision model by casting all the model weights to {\tt half precision} (float-16). We then compare the performance of all the classification models without any explicit resource optimisation and with the combination of the mixed precision training/half-precision models and the 8-bit optimiser. Note that AMP cannot be combined with Half Precision models, as the model itself is in low precision. We use binary accuracy on the test set ($P_T$), average training time to convergence ($T_c$), maximum GPU memory required, average inference time on test set ($T$) and the total energy consumption ($E$)  as the measures to compare the models and different resource efficiency techniques.
% NHS dataset \footnote{\url{https://digital.nhs.uk/data-and-information/publications/statistical/breast-screening-programme/england---2020-21}}
% VinDr:Mammo\footnote{\url{https://vindr.ai/datasets/mammo}}
\\
%
\begin{table}[h]
\vspace{-0.5cm}
\caption{Quantitative comparison of the different classification methods reported over three random initialisations for RSNA and LIDC datasets. The use of 8-bit optimizer (8bit), automatic mixed precision (AMP), and half precision model (Half) are marked. Number of parameters: $|\Wbf|$, mean accuracy with standard deviation over three seeds, average GPU memory required (in GB), average inference time, and average energy consumption of different settings reported when predicting on test set. Models that diverged at least during one of the random initialisations are marked with $^*$ in $P_T$ column.}
\vspace{0.1cm}
\label{tab:results}
\centering
% \scriptsize
\tiny
\begin{tabular}{clrccccrrrr}
\toprule
        {\bf Dataset} &{\bf Method} &   $|\Wbf|$(M) &  {\bf 8bit} &  {\bf AMP} &   {\bf Half} & {$\mathbf P_T$}  &    {\bf GPU} &     $\mathbf T_{c}$(s) & {\bf T}(s) & {\bf E}(Wh) \\
\midrule
\multirow{16.5}{*}{\bf RSNA\cite{rsna-breast-cancer-detection}}&\multirow{5}{*}{Densenet\cite{huang2017densely}} &   \multirow{5}{*}{6.9}&    \xmark &    \xmark & \xmark & 0.738$\pm$0.005 &    6.6 &  361.3 & 2.7 &  104.2 \\
    &&&    \xmark &    \cmark & \xmark & 0.740$\pm$0.005 &  3.3 &  296.0 & 2.7 &  59.2 \\
    &&&    \cmark &    \xmark & \xmark &0.743$\pm$0.003 &  6.5 &  240.0 & 2.7 & 58.1 \\
    &&&    \cmark &    \xmark & \cmark & {\bf 0.745$\pm$0.004} &  
    {\bf 3.2} &  {\bf 170.3} & {\bf 2.1} & 34.6\\
    &&&    \cmark &    \cmark & \xmark & {\bf 0.744$\pm$0.006} &  
    3.2 &  191.3 & 2.7&  42.3 \\
            \cmidrule{2-11}
 &       \multirow{5}{*}{Swin Trans.\cite{liu2021swin}} & \multirow{5}{*}{86.9}&   \xmark &    \xmark & \xmark & 0.739$\pm$0.008  & 15.3 &  617.6 & 7.5 & 100.1 \\
         &&&    \xmark &    \cmark & \xmark & 0.735$\pm$0.006  & 10.7 &  497.0 & 7.5& 109.8 \\
         &&&    \cmark &    \xmark & \xmark & 0.739$\pm$0.012  & 14.3 &  732.3 & 7.5& 86.1 \\
         &&&    \cmark &    \xmark & \cmark & 0.733$\pm$0.013  &  
         7.5 &  316.0 & 4.3& 118.5 \\ 
         &&&    \cmark &    \cmark & \xmark & 0.712$\pm$0.01$^*$  &  
         9.7 &  588.6 & 7.5&  95.8 \\ 
         \cmidrule{2-11}
 &         \multirow{5}{*}{ViT\cite{dosovitskiyimage}} &  \multirow{5}{*}{116.7}&     \xmark &    \xmark & \xmark & 0.728$\pm$0.001  &  5.6 &  322.3 & 1.9 & 52.0 \\
            &&&    \xmark &    \cmark & \xmark & 0.728$\pm$0.006  &  5.6 &  249.0 &1.9 & 39.8 \\
            &&&    \cmark &    \xmark & \xmark & 0.716$\pm$0.009  &  4.1 &  497.0 & 1.9& 81.0 \\
            &&&    \cmark &    \xmark & \cmark & 0.690$\pm$0.02$^*$  
            &  3.5 &  {69.6} & { 1.0} &10.7 \\
            &&&    \cmark &    \cmark & \xmark & 0.665$\pm$0.02$^*$  
            &  4.7 &  132.6 & 1.9& 21.0 \\    
\midrule


\midrule
\multirow{16.5}{*}{\bf LIDC\cite{armato2004lung}}&\multirow{5}{*}{Densenet\cite{huang2017densely}} &   \multirow{5}{*}{6.9}&    \xmark &    \xmark & \xmark & 0.656$\pm$0.005 &    6.6 &  674.3 & 3.4 &  104.2 \\
    &&&    \xmark &    \cmark & \xmark & 0.655$\pm$0.010 &  3.3 &  425.7 & 3.4 &  59.2 \\
    &&&    \cmark &    \xmark & \xmark &0.677$\pm$0.008 &  6.5 &  370.7 & 3.4 & 58.1 \\
    &&&    \cmark &    \xmark & \cmark & { 0.675$\pm$0.007} &  
    {\bf 3.2} &  {\bf 234.3} & { 2.7} & 34.6\\
    &&&    \cmark &    \cmark & \xmark & {0.676$\pm$0.005} &  
    3.2 &  291.3 & 2.7&  42.3 \\
            \cmidrule{2-11}
 &       \multirow{5}{*}{Swin Trans.\cite{liu2021swin}} & \multirow{5}{*}{86.9}&   \xmark &    \xmark & \xmark & 0.697$\pm$0.018  & 15.3 &  678.7 & 9.6 & 109.8 \\
         &&&    \xmark &    \cmark & \xmark & 0.696$\pm$0.018  & 10.8 &  544.0 & 9.6& 86.1 \\
         &&&    \cmark &    \xmark & \xmark & 0.684$\pm$0.016  & 14.4 &  727.3 & 9.6&118.5 \\
         &&&    \cmark &    \xmark & \cmark & {\bf 0.704$\pm$0.007}  &  9.9 &  401.0 & 5.6&  64.1 \\ 
         &&&    \cmark &    \cmark & \xmark & 0.684$\pm$0.020  &  7.5 &  615.3 & 9.6& 99.9 \\ 
         \cmidrule{2-11}
 &         \multirow{5}{*}{ViT\cite{dosovitskiyimage}} &  \multirow{5}{*}{116.7}&     \xmark &    \xmark & \xmark & 0.610$\pm$0.041  &  5.6 &  231.7 & 2.4 & 37.2 \\
            &&&    \xmark &    \cmark & \xmark & 0.608$\pm$0.043  &  5.5 &  166.3 &2.4 & 25.8 \\
            &&&    \cmark &    \xmark & \xmark & 0.636$\pm$0.012  &  4.3 &  415.3 & 2.4& 67.1 \\
            &&&    \cmark &    \xmark & \cmark & 0.564$\pm$0.072  &  3.4 &  123.0 & {\bf 1.3}& 15.2 \\
            &&&    \cmark &    \cmark & \xmark & 0.561$\pm$0.076  &  4.5 &  {99.0} & 2.4 &19.1 \\
\bottomrule
\end{tabular}
\vspace{-0.5cm}
\end{table}
{\bf Results}: Experiments with the various resource optimisation settings and methods for both datasets are reported comprehensively in Table~\ref{tab:results}. For each method, five settings are reported. We treat the model with no additional resource optimisation as the {\em baseline}, reported in the first row for each method. For each model, we then report the following settings: [AMP, 8bit optimizer, AMP+ 8bit optimizer, Half precision + 8bit optimiser] yielding five settings for each model. The models are sorted based on the total number of trainable parameters, reported in the third column. 

At the outset, it is worth noting that the baseline model performance on the test set of all Densenet and the transformer-based Swin- and Vision- Transformers are comparable ($\approx 0.74$) in the RSNA mammography dataset. 

The use of AMP during training does not degrade the performance of any of the models, while reducing the maximum GPU memory utilised (up to 50\% in most cases). The run time, $T_c$ for all the models is also decreased when using AMP.

The use of 8-bit optimiser reduces the GPU memory utilised and the convergence time. The more interesting observation is that in almost all cases (except ViT), it also converges to a better solution, yielding a small performance improvement. This is consistent with the observations reported in~\cite{dettmers8}, where this behaviour is attributed to the elimination of outliers (which otherwise could lead to exploding gradients) during quantisation and dequantisation of the optimiser states.

Setting the model weights to half precision (16-bit instead of 32-bit) immediately gives a reduction in the GPU memory utilisation. In our experiments, however, there were stability issues when the half precision model was trained using the standard Adam optimiser. Using the 8-bit optimiser which is quantisation-aware helped convergence~\cite{dettmers8}. 

The best performing configuration across all the models for RSNA dataset is the Densenet with 8-bit optimiser either when used with AMP or the half precision model. This is an interesting result showing that these techniques not only reduce resource utilisation but could also improve model convergence.

The observations about the influence of AMP, 8-bit optimiser and half-precision hold within each model category for the LIDC dataset. The best performing configuration across all the models for LIDC dataset is the Swin Transformer with 8-bit optimiser when used with the half precision model.

The performance trends in Table~\ref{tab:results} are captured visually in the radar plots presented in Figure~\ref{fig:results}-A for Densenet. The mean performance measures of Densenet over the three random initialisations, for the five settings, are shown in different colours. The four axes are used to report four key measures:[$P_T, E, \text{GPU},T$]. The optimal model would have larger footprint along $P_T$ and smaller values on the other axes. We notice that the baseline model (in blue) spans a larger region over all axes, wheres the half precision model trained with 8-bit optimiser (in red) has smaller span. These trends are also visualised for the Vision transformer model in Figure~\ref{fig:results}-B, and for other models in the Supplementary Material.
%% Standard deviation
% \begin{tabular}{lrrrrrrr}
% \toprule
%         method &  bnb &  amp &   acc &  param &   gpu &    time &  energy \\
% \midrule
%    densenet121 &    0 &    0 & 0.005 &  0.000 & 0.000 &  42.724 &   6.911 \\
%    densenet121 &    0 &    1 & 0.005 &  0.000 & 0.000 &  27.514 &   4.491 \\
%    densenet121 &    1 &    0 & 0.003 &  0.000 & 0.000 &  17.436 &   2.761 \\
%    densenet121 &    1 &    1 & 0.004 &  0.000 & 0.000 &  26.274 &   4.307 \\
% efficientnetv2 &    0 &    0 & 0.026 &  0.000 & 0.000 & 121.837 &  19.255 \\
% efficientnetv2 &    0 &    1 & 0.005 &  0.000 & 0.000 & 381.373 &  60.202 \\
% efficientnetv2 &    1 &    0 & 0.022 &  0.000 & 0.000 & 122.748 &  20.272 \\
% efficientnetv2 &    1 &    1 & 0.008 &  0.000 & 0.001 & 409.619 &  61.790 \\
%            mlp &    0 &    0 & 0.196 &  0.000 & 0.000 &  16.166 &   1.381 \\
%            mlp &    0 &    1 & 0.197 &  0.000 & 0.000 &   2.000 &   0.123 \\
%            mlp &    1 &    0 & 0.206 &  0.000 & 0.000 &  19.053 &   1.520 \\
%            mlp &    1 &    1 & 0.200 &  0.000 & 0.000 &  29.687 &   2.403 \\
%         swinv2 &    0 &    0 & 0.008 &  0.000 & 0.000 &  48.336 &   8.080 \\
%         swinv2 &    0 &    1 & 0.006 &  0.000 & 0.000 &  39.850 &   6.452 \\
%         swinv2 &    1 &    0 & 0.012 &  0.000 & 0.000 & 118.158 &  19.830 \\
%         swinv2 &    1 &    1 & 0.013 &  0.000 & 0.000 &  30.022 &   4.813 \\
%            vit &    0 &    0 & 0.001 &  0.000 & 0.000 &  33.606 &   5.488 \\
%            vit &    0 &    1 & 0.006 &  0.000 & 0.001 &  24.880 &   4.051 \\
%            vit &    1 &    0 & 0.009 &  0.000 & 0.000 & 129.904 &  21.592 \\
%            vit &    1 &    1 & 0.025 &  0.000 & 0.001 &  45.214 &   7.468 \\
% \bottomrule
% \end{tabular}

\begin{figure}[t]
\begin{minipage}{0.32\textwidth}
% \begin{figure}[t]
    \centering
    \includegraphics[width=0.99\textwidth]{images/densenet121_radar.pdf}
    % \caption{}
% \end{figure}

(A): Densenet
\end{minipage}
\begin{minipage}{0.32\textwidth}
% \begin{figure}
    \centering
    \includegraphics[width=0.99\textwidth]{images/vit_radar.pdf}
    
    (B): ViT
\end{minipage}
\begin{minipage}{0.3\textwidth}
% \begin{figure}
    \centering
    \includegraphics[width=0.95\textwidth]{images/downsample.pdf}
    
    (C)
\end{minipage}
\caption{(A \& B): Radar plots showing the mean metrics for performance ($P_T, E, \text{GPU},T$) reported in Table~\ref{tab:results} for the five settings, shown for Densenet~\cite{huang2017densely} and Vision transformer~\cite{dosovitskiyimage}. (C): Influence of different scales of downsampling on the test performance for three configurations of Densenet models on the RSNA dataset. For each, the mean and the standard deviation of the three runs is shown as the curve and the shaded area, respectively.}
    \label{fig:results}
    \vspace{-0.5cm}
\end{figure}


\vspace{-0.25cm}
\section{Discussion}
\vspace{-0.25cm}
{\bf Data Quality}: One common method to manage the resource constraints during training or inference of deep learning models is the use of reduced quality data. For instance, in image analysis, downsampling of images is a commonly used strategy as this can have a direct impact on the size of the models. Such reduction in data quality are undoubtedly lossy. We illustrate this for the mammography screening task for the Densenet models in Figure~\ref{fig:results}-C. We reduce the quality of the images both during training and inference by downsampling the image at different scales: $S=[2,4,8]$. We observe a clear performance degradation compared to the models trained with full resolution ($S=1$). While this might be unsurprising, we argue, in lieu of the results in Table~\ref{tab:results} that optimising models instead of degrading data quality can offer a better trade-off between performance and resource efficiency.
%
\\
{\bf Resource Efficiency and Model Complexity}: The most common measure of model complexity is the number of parameters. The experiments and results reported in Table~\ref{tab:results} reveal slightly surprising trends. The best performing model for the RSNA dataset (Densenet) had the fewest number of parameters (6.9M), and ViT which was the largest model (116.7M) was also the fastest at inference time ($<$2s). This is due to the fact that during training the resource overhead for deep learning models is due to factors other than model weights, such as the optimiser states or other derived data like gradients/activation maps as discussed in Section~\ref{sec:methods}. We have demonstrated that resource efficiency can also be achieved by optimising these other axes.
\\
%
% {\bf Influence of resource optimisation methods}: 
% Focus on training performance for online learning models which can be useful in clinical settings.
% Once the models are trained, it is their inference time that is of most 
% \\
%
% {\bf Compounding gains in some instances}
% \\
% {\bf Performance degradation?}
{\bf Resource Efficiency of Vision Transformers}: The use of AMP or 8-bit optimiser settings degrades the performance of both the transformer-based models to some extent in both datasets. This is in contrast to the CNN models, where there is no large performance degradation. One reason for the degradation in performance in transformers when using the 8-bit optimiser could be instability during training. The authors in~\cite{dettmers8} suggest the use of a {\tt stable embedding} layer when training transformers for NLP tasks. For a fairer comparison with the CNN models, we do not use the stable embedding layer for the two transformer models. Our experiments indicate that CNN models could be less sensitive when compared to transformer-based models in low/mixed precision or quantised settings.
\\
% The baseline performance of the two transformer-based models were close to the baseline performance of Densenet as seen in Table~\ref{tab:results}. We expect this performance could be improved slightly if the hyperparameters were fine-tuned for each class of models. Further, 
% \begin{figure}[t]
% % {r}{0.4\textwidth}
% % \vspace{-1.2cm}
%   \begin{center}
%     \includegraphics[width=0.4\textwidth]{images/downsample.pdf}
%   \end{center}
%   \vspace{-0.5cm}
%   \caption{Influence of different scales of downsampling on the Densenet models.}
%   \label{fig:downsample}
%   \vspace{-0.5cm}
% \end{figure}
% %
{\bf QALY-like metrics}: Quality Adjusted Life Year (QALY) is a utilitarian index used to estimate the cost-effectiveness of a health treatment; it attempts to arithmetically relate quality and quantity of life~\cite{prieto2003problems}.  In our experiments, we did not notice degradation in performance for some of the model classes. However, when there is a drop in performance, the tolerance levels in clinical settings would require formulating a Pareto optimisation of performance and resources~\cite{deb2011multi}. It is in this context that a metric similar to QALY could be useful, when balancing resource constraints with critical performance guarantees. 
% In principle, it would be possible to balance the QALYs gained from deploying the ML method directly against the QALYs lost by introducing additional energy consumption and thereby additional emission. 
\\
%
{\bf Limitations}: This work focused on optimising resource efficiency using only a few of the techniques (mixed precision training, half precision models, 8-bit optimisers). There are several other methods which could further optimise the resource consumption. For instance, formulating model selection that can yield inherently efficient models within neural architecture search (NAS) framework could be a possibility. However, performing NAS in itself is prohibitively expensive~\cite{elsken2019neural}. The other main limitation of the work is the gap in validating these methods in actual, resource constrained setting which the authors did not have access to. Testing these resource optimisation measures in diverse settings such as on different edge devices, for instance, could be an interesting direction for future work.

\vspace{-0.35cm}
\section{Conclusion}
\vspace{-0.35cm}
\label{sec:conc}
Large deep learning models are advancing medical image analysis and will undoubtedly accelerate clinical workflows. For these models to make the most impact, globally, resource efficiency is an important factor. In this work, we have evaluated a subset of resource optimisation methods that can reduce memory consumption, latency, and energy consumption, without any performance degradation (Densenet in RSNA, Swin Trans. in LIDC). In contrast to our initial hypothesis, surprisingly, there is no performance degradation across the board; only systematic ones (like with Transformers which can be handled). This prompts us to recommend these resource optimisation methods to be part of the standard procedures when developing and deploying deep learning methods.
% We studied the impact of quantising gradients, intermediate activations, optimiser states and weights, using existing methods. 
% \\
% \\
% %
\clearpage
\flushleft
{\bf Acknowledgments}
The authors would like to thank Christian Igel (UCPH) and Lee Lassen (Ambu A/S) for fruitful discussions, and Pedram Bakhtiarifard (UCPH) for their public repository for making radar plots. RS and JS are partly funded by European Union’s HorizonEurope research and innovation programme under grant agreements No. 101070284 and No. 101070408.


\bibliographystyle{splncs04}
\bibliography{references}

\section{Appendix for Proofs}

\paragraph{Proof of Theorem \ref{thm:main}.}

\begin{proof}
\label{proof:main}
Our proof has two steps. In Step 1, we will show that SimCLR is equivalent to minimizing the cross entropy loss defined in Eqn.~(\ref{eqn:cross-entropy}). 
In Step 2, we will show  that minimizing the cross-entropy loss 
is equivalent to spectral clustering on $\bfpi$. 
Combining the two steps together, we have proved our theorem. 

\textbf{Step 1: } SimCLR is equivalent to minimizing the cross entropy loss.

The cross-entropy loss takes expectation over 
$\bfW_\bfX\sim \mathbb{P}(\cdot ; \bfpi)$, 
which means $\bfW_\bfX$ has exactly one non-zero entry in each row $i$. By Lemma~\ref{lem:multinomial}, we know every row $i$ of $\bfW_\bfX$ is independent of other rows. Moreover, 
$\bfW_{\bfX,i}\sim \mathcal{M}(1, \bfpi_i/\sum_j \bfpi_{i,j})=\mathcal{M}(1, \bfpi_i)$, because $\bfpi_i$ itself is a probability distribution.
Similarly, we know $\bfW_\bfZ$ also has the row-independent property by sampling over $\mathbb{P}(\cdot;\bfK_\bfZ)$.
Therefore, by Lemma~\ref{lem:cross_split}, we know Eqn.~(\ref{eqn:cross-entropy}) is equivalent to:
\[
 -\sum_{i=1}^n \mathbb{E}_{\bfW_{\bfX,i}}[\log \mathbb{P}(\bfW_{\bfZ,i}=\bfW_{\bfX,i};\bfK_\bfZ)],
\]

This expression takes expectation over $\bfW_{\bfX,i}$ for the given row $i$. Notice that 
$\bfW_{\bfX,i}$ has exactly one non-zero entry, which equals $1$ (same for $\bfW_{\bfZ,i}$). 
As a result
we expand the above expression to be:
\begin{equation}
 -\sum_{i=1}^n \sum_{j\neq i} \Pr(\bfW_{\bfX,i,j}=1)\log \Pr(\bfW_{\bfZ,i,j}=1).
\label{eqn:detailed-expansion}    
\end{equation}


By Lemma~\ref{lem:multinomial}, $\Pr(\bfW_{\bfZ,i,j}=1)=\bfK_{\bfZ,i,j}/\|\bfK_{\bfZ,i}\|_1$ for $j\neq i$. Recall that $\bfK_\bfZ=(k(\bfZ_i-\bfZ_j))_{(i,j)\in[n]^2}$, which means 
$\bfK_{\bfZ,i,j}/\|\bfK_{\bfZ,i}\|_1=\frac{\exp(-\|\bfZ_i-\bfZ_j\|^2/{2\tau})}{\sum_{k\neq i}
\exp(-\|\bfZ_i-\bfZ_k\|^2/{2\tau})
}$ for $j\neq i$, when $k$ is the Gaussian kernel with variance $\tau$. 

Notice that $\bfZ_i=f(\bfX_i)$, so we know
\begin{equation}
-\log \Pr(\bfW_{\bfZ,i,j}=1)=
-\log \frac{\exp(-\|f(\bfX_i)-f(\bfX_j)\|^2/{2\tau})}{\sum_{k\neq i}
\exp(-\|f(\bfX_i)-f(\bfX_k)\|^2/{2\tau}),
}
\label{eqn:infonce-equivalence}    
\end{equation}


The right hand side is exactly the InfoNCE loss defined in Eqn.~(\ref{eqn:infonce}).
Inserting Eqn.~(\ref{eqn:infonce-equivalence}) into Eqn.~(\ref{eqn:detailed-expansion}), we get the SimCLR algorithm, which first samples augmentation pairs $(i,j)$ with $\Pr(\bfW_{\bfX,i,j}=1)$ for each row $i$, and then optimize the InfoNCE loss. 

\textbf{Step 2: } minimizing the cross entropy loss 
is equivalent to spectral clustering on $\bfpi$.


By Lemma~\ref{lem:convert_to_spectral}, we may further convert the loss to 
\begin{equation}
\label{eqn:main-theorem-repul-attr}
\min_{\bfZ}
-\sum_{(i,j)\in [n]^2} \mathbf{P}_{i,j}
\log k (\bfZ_i-\bfZ_j)+\log \mathbf{R}(\bfZ).
\end{equation}
Since $k$ is the Gaussian kernel, this reduces to \[
\min_\bfZ \mathrm{tr}(\bfZ^\top \mathbf{L}(\bfpi) \bfZ)
+\log \mathbf{R}(\bfZ),
\]

where we use the fact that $\mathbb{E}_{\bfW_\bfX\sim \mathbb{P}(\cdot; \bfpi)}[\mathbf{L}(\bfW_\bfX)]
=\mathbf{L}(\bfpi)
$, because the Laplacian operator is linear and $
\mathbb{E}_{\bfW_\bfX\sim \mathbb{P}(\cdot; \bfpi)}(\bfW_\bfX)=\bfpi
$.
\end{proof}

\paragraph{Proof of Theorem \ref{thm:clip}.}
\begin{proof}
Since $\bfW_\bfX\sim \mathbb{P}(\cdot;\bfpi_{\mathbf{A}, \mathbf{B}})$, we know 
$\bfW_\bfX$ has exactly one non-zero entry in each row, denoting the pair that got sampled. 
A notable difference compared to the previous proof is we now have $n_\mathcal{A}+n_\mathcal{B}$ objects in our graph. CLIP deals with this by taking a mini-batch of size $2N$, 
such that $n_\mathcal{A}=n_\mathcal{B}=N$, and adding the $2N$ InfoNCE losses together. We label the objects in $\mathcal{A}$ as $[n_\mathcal{A}]$, and the objects in $\mathcal{B}$ as $\{n_\mathcal{A}+1, \cdots, n_\mathcal{A}+n_\mathcal{B}\}$. 

Notice that $\bfpi_{\mathbf{A}, \mathbf{B}}$ is a bipartite graph, so the edges of objects in $\mathcal{A}$ will only connect to object in $\mathcal{B}$ and vice versa. We can define the similarity matrix in $\cZ$ as $\bfK_\bfZ$, 
where $\bfK_\bfZ(i, j+n_\mathcal{A})=\bfK_\bfZ(j+n_\mathcal{A},i)= k(\bfZ_i-\bfZ_j)$ for $i\in [n_\mathcal{A}], j\in [n_\mathcal{B}]$, and otherwise we set $\bfK_\bfZ(i,j)=0$. 
The rest is same as the previous proof. 
\end{proof}

\paragraph{Proof of Theorem \ref{thm:exponential}.}

\begin{proof}
\label{proof:exponential}
Since the objective function consists of a linear term combined with an entropy regularization, which is a strongly concave function, the maximization problem is a convex optimization problem. Owing to the implicit constraints provided by the entropy function, the problem is equivalent to having only the equality constraint. We then introduce the Lagrangian multiplier $\lambda$ and obtain the following relaxed problem:

$$
\widetilde{E}(\boldsymbol{\alpha})=\psi_{1}-\sum_{i=1}^n \alpha_{i} \psi_{i}+\tau \sum_{i=1}^n \alpha_{i}\log \alpha_{i}+\lambda\left(\boldsymbol{\alpha}^{\top} \mathbf{1}_n-1\right).
$$

As the relaxed problem is unconstrained, taking the derivative with respect to $\alpha_{i}$ yields

$$
\frac{\partial \widetilde{E}(\boldsymbol{\alpha})}{\partial \alpha_{i}}=-\psi_{i}+\tau\left(\log \alpha_{i}+\alpha_{i} \frac{1}{\alpha_{i}}\right)+\lambda=0.
$$

Solving the above equation implies that $\alpha_{i}$ takes the form
$
\alpha_{i}=\exp \left(\frac{1}{\tau} \psi_{i}\right) \exp \left(\frac{-\lambda}{\tau}-1\right).
$ Since $\alpha_{i}$ lies on the probability simplex, the optimal $\alpha_{i}$ is explicitly given by
$
\alpha^{*}_{i}=\frac{\exp \left(\frac{1}{\tau} \psi_{i}\right)}{\sum_{i^{\prime}=1}^n \exp \left(\frac{1}{\tau} \psi_{i^{\prime}}\right)} .
$ Substituting the optimal point into the objective function, we obtain
$$
\begin{aligned}
E\left(\boldsymbol{\alpha}^*\right)  &=\psi_1-\sum_{i=1}^n \frac{\exp \left(\frac{1}{\tau} \psi_{i}\right)}{\sum_{i^{\prime}=1}^n \exp \left(\frac{1}{\tau} \psi_{i^{\prime}}\right)} \psi_{i}+\tau \sum_{i=1}^n \frac{\exp \left(\frac{1}{\tau} \psi_{i}\right)}{\sum_{i^{\prime}=1}^n \exp \left(\frac{1}{\tau} \psi_{i^{\prime}}\right)}\log \frac{\exp \left(\frac{1}{\tau} \psi_{i}\right)}{\sum_{i^{\prime}=1}^n \exp \left(\frac{1}{\tau} \psi_{i^{\prime}}\right)} \\
& =\psi_1 - \tau \log \left(\sum_{i=1}^n \exp \left(\frac{1}{\tau} \psi_{i}\right)\right).
\end{aligned}
$$
Thus, the Lagrangian dual function is given by
\begin{equation*}
-E\left(\boldsymbol{\alpha}^*\right)= -\tau \log \frac{\exp \left(\frac{1}{\tau} \psi_{1}\right)}{\sum_{i=1}^n \exp \left(\frac{1}{\tau} \psi_{i}\right)}.\qedhere
\end{equation*}
\end{proof}



\section{More on Experiments} \label{section: experiment_details}

\paragraph{CIFAR-10 and CIFAR-100} CIFAR-10 ~\citep{krizhevsky2009learning} and CIFAR-100 ~\citep{krizhevsky2009learning} are well-known classic image classification datasets. Both CIFAR-10 and CIFAR-100 contain a total of 60k $32 \times 32$ labeled images of different classes, with 50k for training and 10k for testing. CIFAR-10 is similar to CIFAR-100, except there are 10 different classes in CIFAR-10 and 100 classes in CIFAR-100.

\paragraph{TinyImageNet} TinyImageNet ~\citep{le2015tiny} is a subset of ImageNet ~\citep{deng2009imagenet}. There are 200 different object classes in TinyImageNet, with 500 training images, 50 validation images, and 50 test images for each class. All the images in TinyImageNet are colored and labeled with a size of $64 \times 64$.

\textbf{Pseudo-code.} Algorithm \ref{alg:Training Procedure} presents the pseudo-code for our empirical training procedure.

\begin{algorithm}[!htbp]
\caption{Training Procedure}
\label{alg:Training Procedure}
\begin{algorithmic}[1]
\REQUIRE trainable encoder network $f$, batch size $N$, augmentation strategy \textit{aug}, loss function $L$ with hyperparameters \textit{args}
\FOR {sampled minibatch ${x_i}_{i=1}^N$}
\FORALL{$i \in { 1, ..., N }$}
\STATE draw two augmentations $t_i = \textit{aug}\left(x_i\right) $, $t_i' = \textit{aug}\left(x_i\right) $
\STATE $z_i = f\left(t_i\right)$, $z_i' = f\left(t_i'\right)$
\ENDFOR
\STATE compute loss $\mathcal{L} = L(N, z, z', \textit{args})$
\STATE update encoder network $f$ to minimize $\mathcal{L}$
\ENDFOR
\STATE \textbf{Return} encoder network $f$
\end{algorithmic}
\end{algorithm}

We also provide the pseudo-code for our core loss function used in the training procedure in Algorithm \ref{alg:Core loss}. The pseudo-code is almost identical to SimCLR's loss function, with the exception of an extra parameter $\gamma$.

\begin{algorithm}[!htbp]
\caption{Core loss function $\mathcal{C}$}
\label{alg:Core loss}
\begin{algorithmic}[1]
\REQUIRE batch size $N$, two encoded minibatches $z_1, z_2$, $\gamma$, temperature $\tau$
\STATE $z = \textit{concat}\left(z_1, z_2\right)$
\FOR {$i \in {1, ..., 2N }, j \in {1, ..., 2N}$ }
\STATE $s_{i,j} = \Vert z_i - z_j \Vert_2^{\gamma}$
\ENDFOR
\STATE \textbf{define} $l(i, j)$ \textbf{as} $l(i, j) = - \log \frac{exp\left(s_{i,j}/\tau \right)}{\sum_{k=1}^{2N} \mathbf{1}{[k \ne i]} exp\left(s{i, j} / \tau \right)} $
\STATE \textbf{Return} $\frac{1}{2N} \sum_{k=1}^N\left[l(i, i+N) + l(i+N, i)\right]$
\end{algorithmic}
\end{algorithm}

Utilizing the core loss function $\mathcal{C}$, we can define all kernel loss functions used in our experiments in Table \ref{table: loss definition}. For all $z_i \in z$ with even dimensions $n$, we define $z_{L_i} = z_i\left[0:n/2\right]$ and $z_{R_i} = z_i\left[n/2:n\right]$.

\begin{table}[ht]
\centering
\begin{tabular}{{@{}l|l@{}}}
Kernel  &  Loss function \\ \midrule
Laplacian & $\mathcal{C}\left(N, z, z', \gamma=1, \tau\right)$\\ \midrule
Sum       & $\lambda * \mathcal{C}\left(N, z, z', \gamma=1, \tau_1\right) + (1-\lambda) * \mathcal{C}\left(N, z, z', \gamma=2, \tau_2\right)$  \\ \midrule
Concatenation Sum&$\lambda * \mathcal{C}\left(N, z_L, z'_L, \gamma=1, \tau_1\right) + (1-\lambda) * \mathcal{C}\left(N, z_R, z'_R, \gamma=2, \tau_2\right)$\\ \midrule
$\gamma = 0.5$ & $\mathcal{C}\left(N, z, z', \gamma=0.5, \tau\right)$          \\ 

\end{tabular}

\caption{Definition of kernel loss functions in our experiments}
\label {table: loss definition}
\end{table}

\textbf{Baselines.} We reproduce the SimCLR algorithm using PyTorch Lightning~\citep{PytorchLightning}.

\textbf{Encoder details.}
The encoder $f$ consists of a backbone network and a projection network. We employ ResNet50~\citep{ResNet} as the backbone and a 2-layer MLP (connected by a batch normalization~\citep{ioffe2015batch} layer and a ReLU \cite{nair2010rectified} layer) with hidden dimensions 2048 and output dimensions 128 (or 256 in the concatenation kernel case).

\textbf{Encoder hyperparameter tuning.}
For each encoder training case, we randomly sample 500 hyperparameter groups (sample details are shown in Table \ref{table: Hyperparameter sample}) and train these samples simultaneously using Ray Tune ~\citep{RayTune}, with the ASHA scheduler~\citep{li2018massively}. Ultimately, the hyperparameter group that maximizes the online validation accuracy (integrated in PyTorch Lightning) within 5000 validation steps is chosen for the given encoder training case.

\begin{table}[ht]
\centering

\begin{tabular}{@{}l|l|l@{}}
\midrule
Hyperparameter  & Sample Range & Sample Strategy \\ \midrule
start learning rate & $\left[10^{-2}, 10\right]$ & log uniform \\ \midrule
$\lambda$       & $\left[0, 1\right]$ & uniform \\ \midrule
$\tau$, $\tau_1$, $\tau_2$ & $\left[0, 1\right]$ & log uniform \\ \midrule
\end{tabular}

\caption{Hyperparameters sample strategy}
\label {table: Hyperparameter sample}
\end{table}

\textbf{Encoder training.} 
We train each encoder using the LARS optimizer~\citep{LARSOptimizer}, LambdaLR Scheduler in PyTorch, momentum 0.9, weight decay $10^{-6}$, batch size 256, and the aforementioned hyperparameters for 400 epochs on a single A-100 GPU.

\textbf{Image transformation.} The image transformation strategy, including augmentation, is identical to the default transformation strategy provided by PyTorch Lightning.

\textbf{Linear evaluation.}
The linear head is trained using the SGD optimizer with a cosine learning rate scheduler, batch size 64, and weight decay $10^{-6}$ for 100 epochs. The learning rate starts at $0.3$ and ends at $0$.

\textbf{Moco Experiments.} We also tested our method based on MoCo~\citep{he2019moco}. The results are summarized in Table \ref{tab:results-moco}. Here we choose ResNet18~\citep{ResNet} as the backbone and set a temperature of $0.1$ as default. For our simple sum kernel, we set $\lambda=0.8$. The results show that our method outperforms the original MoCo method.

\begin{table}[thb]
\centering
\caption{MoCo Experiment Results on CIFAR-10 and CIFAR-100.}
\label{tab:results-moco}
\resizebox{\textwidth}{!}{%
\begin{tabular}{@{}c|ccc|ccc@{}}
\toprule
\multirow{3}{*}{Method} & \multicolumn{3}{c|}{CIFAR-10} & \multicolumn{3}{c}{CIFAR-100} \\ \cmidrule(lr){2-4} \cmidrule(lr){5-7} 
                        & 200 epochs & 400 epochs    & 1000 epochs   & 200 epochs & 400 epochs & 1000 epochs         \\ \midrule
MoCo (repro.)         & $76.41 \pm 0.12$    & $80.01 \pm 0.15$          & $84.45 \pm 0.08$    & $\mathbf{47.02 \pm 0.11}$ & $52.50 \pm 0.07$ & $57.62 \pm 0.15$            \\
\midrule
Laplacian Kernel        & ${78.09 \pm 0.10}$    & $\mathbf{83.85 \pm 0.09}$          & $\mathbf{88.34 \pm 0.16}$    & $46.12 \pm 0.22$   & $53.44 \pm 0.17$ & $59.10 \pm 0.14$        \\
Simple Sum Kernel & $\mathbf{78.12 \pm 0.15}$   & $83.23 \pm 0.18$ & $87.50 \pm 0.20$ & $46.65 \pm 0.06$ & $\mathbf{53.62 \pm 0.19}$ & $\mathbf{59.83 \pm 0.12}$\\
\bottomrule
\end{tabular}
}
\end{table}



\section{More Experiments on Synthetic Data}


Consider a scenario with $n$ clusters, each containing $k$ vertices. Let the probability of vertices $u$ and $v$ from the same cluster belonging to $\bfpi$ be $p$. Conversely, for vertices $u$ and $v$ from different clusters, let the probability of belonging to $\pi$ be $q$. We generate the graph $\bfpi$ randomly, based on $p$ and $q$. We experiment with values of $k=100$ and $n=6$ for ease of visualization, embedding all points in a two-dimensional space. Each vertex's initial position originates from a normal distribution. In each iteration, we sample a subgraph of $\bfpi$ uniformly, ensuring each vertex has an out-degree of $1$. We then optimize the corresponding vectors using InfoNCE loss with an SGD optimizer and iterate until convergence. Our experimental setup consists of an SGD learning rate of $1$, an InfoNCE loss temperature of $0.5$, and a batch size of $50$. We evaluate two scenarios with different $p$ and $q$ values: $p=1$, $q=0$, and $p=0.75$, $q=0.2$. The results of these experiments are visualized in Figure \ref{fig:vis-spectral-cluster}. The obtained embeddings exhibit the hallmark pattern of spectral clustering of graph $\bfpi$.

\begin{figure}[!tb]
\centering
\subfigure{
\includegraphics[width=1\textwidth]{Figures/cluster_pi.png}
\label{fig:vis-cluster}
}
\subfigure{
\includegraphics[width=1\textwidth]{Figures/noised_cluster_pi.png}
\label{fig:vis-noised-cluster}
}
\caption{Visualizations of the optimization process using InfoNCE Loss on the vectors corresponding to $\bfpi$. Points of identical color belong to the same cluster within $\bfpi$. To showcase the internal structure of $\bfpi$, we randomly select 10 vertices from each cluster to display the edge distribution of $\bfpi$.}
\label{fig:vis-spectral-cluster}
\end{figure}


\end{document}
