
\clearpage
\appendix
\section{Additional Results}

\begin{figure}[h]
\vspace{-1.0cm}
    \centering
    \includegraphics[width=0.69\textwidth]{images/lidc.png}
    \caption{LIDC-IDRI dataset comprises 1018 thoracic CT images with lesions annotated by four radiologists~\cite{armato2004lung}. Patches of 128x128 px are extracted from the 2D slices, yielding a total of 15096 patches~\cite{lidc}. Each patch has annotations from four raters marking the tumour regions. These segmentation masks were converted into binary labels indicating the presence (if $\geq 2$ radiologists marked a tumour) or absence of tumours (if $<2$ raters marked tumours), resulting in a fairly balanced
dataset. All image intensities are normalised to be in [0, 1]. Training, validation and test splits are made following a [0.6,0.2,0.2] ratio.}
\vspace{-0.5cm}
\end{figure}
\vspace{-0.5cm}

\begin{table}[h]
\caption{Quantitative comparison of MLP and EfficientNet~\cite{sze2017efficient} reported over three random initialisations for the RSNA dataset. The use of 8-bit optimizer (8b\_Opt), automatic mixed precision (AMP) and half precision model (Half) are marked. Number of parameters: $|\theta|$, mean accuracy with standard deviation over three seeds, average GPU memory required (in GB), average inference time and average energy consumption of different settings reported when predicting on test set.}
\vspace{0.5cm}
\label{tab:app_results}
\centering
\scriptsize
\begin{tabular}{lrcccccrrrr}
\toprule
        {\bf Method} &   {\bf 
        $|\theta|$}(M) &  {\bf 8b\_Opt} &  {\bf AMP} &   {\bf Half} & {\bf Acc.}  &    {\bf GPU} &     {\bf T}$_{c}$(s) & {\bf T}(s) & {\bf E} (Wh) \\
\midrule

    \multirow{5}{*}{MLP} &   \multirow{5}{*}{8.4} &     \xmark &    \xmark & \xmark & 0.685$\pm$0.010 &   0.4 &   85.0 & 0.2 &  7.3 \\
            &&    \xmark &    \cmark & \xmark & 0.687$\pm$0.005  &     0.4 &   92.3 & 0.2&   8.1 \\
            &&    \cmark &    \xmark & \xmark & 0.693$\pm$0.007 &      0.2 &   41.3 & 0.2 & 3.4 \\
            &&    \cmark &    \cmark & \xmark & {\bf 0.694$\pm$0.005} &      0.2 &   55.3 & 0.2 & 4.7 \\
            &&    \cmark &    \xmark & \cmark & 0.680$\pm$0.005 &      0.1 &   27.3 & 0.2 & 2.4 \\
            \midrule
\multirow{5}{*}{Efficientnet\cite{tan2019efficientnet}} &   \multirow{5}{*}{51.1}&  \xmark &    \xmark & \xmark &  0.679$\pm$0.026 & 14.6 &  666.7 & 4.9 & 104.3 \\
 &&    \xmark &    \cmark & \xmark & 0.692$\pm$0.005  &  7.8 & 4027.3 & 4.9 & 618.4 \\
 &&    \cmark &    \xmark & \xmark & 0.715$\pm$0.022  & 14.2 &  466.0 & 4.9& 74.4 \\
 &&    \cmark &    \cmark & \xmark & {\bf 0.717$\pm$0.008}  &  7.4 & 2769.0 & 4.9& 427.2 \\ 
 &&    \cmark &    \xmark & \cmark & {\bf 0.717$\pm$0.008}  &  7.4 & 2769.0 & 3.8 & 427.2 \\ 
\bottomrule
\end{tabular}
% \vspace{-0.5cm}
\end{table}

\begin{figure}[h]
\centering
\begin{minipage}{0.3\textwidth}
% \begin{figure}[t]
    \centering
    \includegraphics[width=0.95\textwidth]{images/mlp_radar.pdf}
    % \caption{}
% \end{figure}

(A): MLP
\end{minipage}
\begin{minipage}{0.3\textwidth}
% \begin{figure}
    \centering
    \includegraphics[width=0.95\textwidth]{images/efficientnetv2_radar.pdf}
    
    (B): EfficientNet~\cite{sze2017efficient}
\end{minipage}
% \begin{minipage}{0.3\textwidth}
% % \begin{figure}
%     \centering
%     \includegraphics[width=0.95\textwidth]{images/swinv2_radar.pdf}
    
%     (C)
% \end{minipage}
\caption{(A \& B): Radar plots for multi-layered perceptron (MLP) and EfficientNet~\cite{sze2017efficient} showing the mean metrics for performance ($P_T, E, \text{GPU},T$) reported in Table~\ref{tab:app_results} for the five settings.}
    \label{fig:app_rsna}
\end{figure}

% AMP increases run time for Efficientnet:This is particularly pronounced for the Efficientnet models. We speculate that the architecture of Efficientnet, which was discovered automatically using efficient NAS, is highly optimised and performing AMP on such models is ineffective from a computation time perspective. 

\begin{figure}
\begin{minipage}{0.3\textwidth}
% \begin{figure}[t]
    \centering
    \includegraphics[width=0.95\textwidth]{images/densenet121_radar_lidc.pdf}
    % \caption{}
% \end{figure}

(A)
\end{minipage}
\begin{minipage}{0.3\textwidth}
% \begin{figure}
    \centering
    \includegraphics[width=0.95\textwidth]{images/swinv2_radar_lidc.pdf}
    
    (B)
\end{minipage}
\begin{minipage}{0.3\textwidth}
% \begin{figure}
    \centering
    \includegraphics[width=0.95\textwidth]{images/vit_radar_lidc.pdf}
    
    (C)
\end{minipage}

\caption{Radar plots for DenseNet~\cite{huang2017densely}, Swin Transformer~\cite{liu2021swin} and Vision Transformer~\cite{dosovitskiyimage} on the LIDC dataset, showing the mean metrics for performance ($P_T, E, \text{GPU},T$) reported in Table~\ref{tab:results} for the five settings.}

    \label{fig:app_lidc}
\end{figure}

