221206
- Intro의 EGO 문단을 UQ in engineering의 활용 방안정도만 강조하는 식으로 간략하게 정리
- 그리고 UQ의 필요성을 EGO 뿐만 아니라 interpretability in decision making - e.g., risk assessment로 해석하기. 예를 들어 CFD 에서 데이터가 나오고 이를 바탕으로 regression했다고 하자. --> 이 때 이러한 regression을 바탕으로 한 예측에서는 불확실성이 데이터로부터 나오는 uncertainty, 근사모델에서 나오는 uncertainty가 있는데 이를 구분하고 각각이 얼만큼인지 알아야 이후 reasonable, realistic한 의사 결정 가능
- GPR의 내용도 v1에 비해 줄임. 세부적인 원리 지움.
- BNN은 내용을 더줄여야 됨. 그냥 BNN이 ~한 한계를 가져서 approximate Bayesian이 필요하다 이정도만 얘기해주면되지 v1처럼 BNN에 대해 많이 언급할 필요가 없음.