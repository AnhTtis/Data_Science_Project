\documentclass{article} % For LaTeX2e
\usepackage{mefomo_2023,times}

\usepackage{amsmath,amsfonts,amssymb,amsthm}
\newtheorem{definition}{Definition}
\usepackage{microtype}
\usepackage{graphicx}
\usepackage{subfig}

\usepackage[symbol]{footmisc}
\renewcommand{\thefootnote}{\fnsymbol{footnote}}

\iclrfinalcopy

\newcommand{\bbox}{\text{bbox}}
\newcommand{\alphapck}{\alpha_\bbox}
\newcommand{\kcycle}{\text{k-CyPCK}}
\newcommand{\cycle}{\text{-CyPCK}}

\newcommand{\I}{\mathbf{I}}
\newcommand{\Ia}{\I^\text{a}}
\newcommand{\Ib}{\I^\text{b}}
\newcommand{\Iatob}{\I^\text{a $\rightarrow$ b}}
\newcommand{\F}{\mathbf{F}}
\newcommand{\Fa}{\F^\text{a}}
\newcommand{\Fb}{\F^\text{b}}
\newcommand{\f}{\mathbf{f}}
\newcommand{\fa}{\f^\text{a}}
\newcommand{\fb}{\f^\text{b}}
\newcommand{\p}{\mathbf{p}}
\newcommand{\pa}{\p^\text{a}}
\newcommand{\pb}{\p^\text{b}}
\newcommand{\A}{\boldsymbol{\Phi}_\text{align}}
\newcommand{\G}{\mathbf{G}}
\newcommand{\C}{\mathbf{C}}
\newcommand{\Ca}{\C^\text{a}}
\newcommand{\Cb}{\C^\text{b}}
\newcommand{\cc}{\mathbf{c}}
\newcommand{\cca}{\cc^\text{a}}
\newcommand{\ccb}{\cc^\text{b}}
\newcommand{\Irec}{\I_\text{Recon}}
\newcommand{\M}{\mathbf{M}}
\newcommand{\Mrec}{\M_\text{Recon}}
\newcommand{\loss}{\mathcal{L}}
\newcommand{\T}{\mathcal{T}}
\newcommand{\W}{\mathcal{W}}
\newcommand{\Id}{\mathcal{I}}

\newcommand{\new}{\textcolor{red}}

\title{A Tale of Two Circuits: Grokking as Competition of Sparse and Dense Subnetworks}


\author{William Merrill\thanks{Equal contribution} \hspace{0.1mm}, Nikolaos Tsilivis\footnotemark[1] \hspace{0.1mm} \& Aman Shukla \\
New York University
}

% \usepackage[latin1]{inputenc}
\usepackage[british]{babel}
\usepackage[all]{xy}
\usepackage{amscd}
\usepackage{amssymb}
\usepackage{amsthm}
\usepackage{enumitem}
\usepackage{mathrsfs,bbm}
\usepackage{xcolor,graphicx}
\usepackage{graphics}
\usepackage{soul}
\usepackage{comment}
\usepackage[all]{xy}
\usepackage{amscd}
\usepackage{amssymb,amsmath,latexsym}
\usepackage{amsthm}
\usepackage{enumitem}
\usepackage{mathrsfs,bbm}
\usepackage{dsfont}
\usepackage{tikz-cd}
\usepackage[T1]{fontenc}
\usepackage[utf8]{inputenc}  
 %
%%%%%%%%%%%%%%%%%%%%%%%%%%%%%%%%%%
%pagestyle
%%%%%%%%%%%%%%%%%%%%%%%%%%%%%%%%%%
%\pagestyle{plain}
\textwidth=430pt
\headsep=.7cm
\evensidemargin=15pt
\oddsidemargin=15pt
\leftmargin=0cm
\rightmargin=0cm
%%
%%%%%%%%%%%%%%%%%%%%%%%
\newcommand*\fixitem {\item[]%
  \refstepcounter{enumi}\hskip-\leftmargin\labelenumi\hskip\labelsep}
\newtheorem*{mainthm}{Main Theorem}
\newtheorem*{mainthm1}{Theorem}
\newtheorem*{maincor}{Corollary}
\usepackage[colorlinks=true]{hyperref}
\DeclareMathOperator{\Forall}{\forall}
\DeclareMathOperator{\Exists}{\exists}
\DeclareMathOperator{\ord}{ord}
\newcommand{\phiD}{\varphi_D}
\newcommand{\phiDI}{\varphi_{\mathbf{D}_I}}
\newcommand{\phiDIj}{\varphi_{\mathbf{D}_I (j)}}
\newcommand{\phiH}{\varphi_H}
\newcommand{\phiTimes}{\phiD \otimes \phiH}
\newcommand{\phiTimesDI}{\varphi_{\mathbf{D}_I} \otimes \phiH}
\newcommand{\R}{\mathscr{A}}
\newcommand{\X}{\mathscr{X}}
\newcommand{\Xf}{\mathscr{X}_{(k_0 ,i)}[r_0]}
\newcommand{\Xfr}{\mathscr{X}_{(k_0,i)}[r]}
\newcommand{\hotimes}{\widehat{\otimes}}
\newcommand{\C}{\mathbb{C}_p}
\newcommand{\V}{\mathscr{V}}
\newcommand{\B}{\mathscr{B}}
\newcommand{\dualD}{\mathfrak{D}}
\newcommand{\Dg}{\mathbf{D}}
\newcommand{\DD}{\mathcal{D}^0}
\newcommand{\DDg}{\mathcal{D}}
\newcommand{\DV}{\mathcal{D}}
\newcommand{\W}{\mathscr{W}_N}
\newcommand{\Ao}{\mathbf{A}^\circ}
\newcommand{\AoK}{\mathbf{A}^\circ_{\K}}
\newcommand{\AK}{\mathbf{A}_{/\K}}
\newcommand{\OOO}{\mathscr{A}^\circ}
\newcommand{\K}{\mathcal{K}} 
\newcommand{\OK}{\mathcal{O}_{\K}}
\newcommand{\varprojlog}[1]{\underleftarrow{\log\!^{#1}}}
\newcommand{\T}{\mathscr{T}}
\newcommand{\TT}{\mathbf{T}}
\newcommand{\VV}{\mathbf{V}}
\newcommand{\HH}{\mathcal{H}}
\newcommand{\hh}{\mathcal{H}^+}
\newcommand{\HG}[2]{\mathcal{H}_{#1}(#2)}
\newcommand{\hhl}{\mathcal{H}^{+,[l]}}
\newcommand{\hhj}{\mathcal{H}^{+,[j]}}
\newcommand{\hhjj}{\mathcal{H}^{+,[l,l']}}
\newcommand{\GS}{G_{\mathbb{Q},S}}
\newcommand{\Rf}{R_{(k_0 ,i)}[r_0]}
\newcommand{\Rfr}{R_{(k_0 ,i)}[r]}
\newcommand{\parT}{\langle T\rangle}
\newcommand{\Zf}{Z_{(k_0 ,i)}[r_0]}
\newcommand{\Zfr}{\mathscr{Z}_{(k_0 ,i)}[r]}
\newcommand{\ZFf}{\mathscr{Z}_{(k_0 ,i)}[r_0]}
\newcommand{\ZFfr}{\mathscr{Z}_{(k_0 ,i)}[r]}
\newcommand{\ZF}{\mathscr{Z}}

\begin{document}

\maketitle


\begin{abstract}
Grokking is a phenomenon where a model trained on an algorithmic task first overfits but, then, after a large amount of additional training, undergoes a phase transition to generalize perfectly. We empirically study the internal structure of networks undergoing grokking on the sparse parity task, and find that the grokking phase transition corresponds to the emergence of a sparse subnetwork that dominates model predictions. On an optimization level, we find that this subnetwork arises when a small subset of neurons undergoes rapid norm growth, whereas the other neurons in the network decay slowly in norm. Thus, we suggest that the grokking phase transition can be understood to emerge from \emph{competition} of two largely distinct subnetworks: a dense one that dominates before the transition and generalizes poorly, and a sparse one that dominates afterwards.

% There has been recent interest in explaining how a model learns to overcome overfitting with no disambiguating evidence in the data.
% One family of hypotheses involve compression: the original phase of grokking corresponds to learning a large memorization circuit, whereas the second phase transition corresponds simplifying this memorized circuit to something that generalizes better.\will{Is this compression hypothesis something hinted at in the original paper? If no one has really made the claim, framing the paper this way may be a bit of a strawman}. \nikos{agreed. sparsity, I think, is one of our contributions.}
% While this seems appealing, we find that compression is not a good explanation of grokking
% by focusing on the case study of the sparse parity task and analyzing a model's activated subcircuit over the course of training.
% While the generalization-phase circuit is simpler than the memorization-phase circuit, we find there is limited neuron overlap between the two circuits.
% This suggests that grokking is not caused by compression of a memorizing subcircuit to a generalizing one, but from competition between generalizing and memorizing subcircuits, where the generalizing subcircuit eventually wins out.
\end{abstract}

\section{Introduction}
\renewcommand*{\thefootnote}{\arabic{footnote}}

% \will{This claim from the mechanistic interpretability paper seems worth responding to: ''Surprisingly, the sudden transition to perfect test accuracy in grokking occurs during cleanup, after the generalizing mechanism is learned. These results show that grokking, rather than being a sudden shift, arises from the gradual amplification of structured mechanisms encoded in the weights, followed by the later removal of memorizing components.'' We are finding something similar: grokking happens when gen. overtakes mem. circuit, and, furthermore, the gen. circuit is not a subset of the mem. circuit?}
% \will{Perhaps we could start by summarizing their story as a hypothesis, then frame our hypotheses as two finer-grained versions of theirs. We can then frame our contributions as 1) testing their hypothesis and 2) seeing which extended version is more realistic}
% \nikos{I don't like this idea that much.}


Grokking \citep{powers2022grokking,barak2022hidden} is a curious generalization trend for neural networks trained on certain algorithmic tasks. Under grokking, the network's accuracy (and loss) plot displays two phases. Early in training, the training accuracy goes to $100\%$, while the generalization accuracy remains near chance. Since the network appears to be simply memorizing the data in this phase, we refer to this as the \emph{memorization phase}. Significantly later in training, the generalization accuracy spikes suddenly to $100\%$, which we call the \emph{grokking transition}.

This mysterious pattern defies conventional machine learning wisdom: after initially overfitting, the model is somehow learning the correct, generalizing behavior without any disambiguating evidence from the data. Accounting for this strange behavior motivates developing a theory of grokking rooted in optimization.
Moreover, grokking resembles so-called emergent behavior in large language models \citep{barret2022emergent}, where performance on some (often algorithmic) capability remains at chance below a critical scale threshold, but, with enough scale, shows roughly monotonic improvement.
We thus might view grokking as a controlled test bed for emergence in large language models, and hope that understanding the dynamics of grokking could lead to hypotheses for analyzing such emergent capabilities. Ideally, an effective theory for such phenomena should be able to understand the causal mechanisms behind the phase transitions, predict on which downstream tasks they could happen, and disentangle the statistical (number of data) from the computational (compute time, size of network) aspects of the problem.

% Grokking, as defined above, is a purely behavioral phenomenon, but we would like to understand how the network internals change over time to bring about the phase transition.
While grokking was originally identified on algorithmic tasks, \citet{liu2023omnigrok} show it can be induced on natural tasks from other domains with the right hyperparameters. Additionally, grokking-like phase transitions have long been studied in the statistical physics community \citep{EnVa01}, albeit in a slightly different setting (online gradient descent, large limits of model parameters and amount of data etc.).
Past work analyzing grokking has reverse-engineered the network behavior in Fourier space \citep{nanda2023progress} and found measures of progress towards generalization before the grokking transition \citep{barak2022hidden}.
\citet{thilak2022the} observe a ``slingshot'' pattern during grokking: the final layer weight norm follows a roughly sigmoidal growth trend around the grokking phase transition.
This suggests grokking is related to the magnitude of neurons within the network, though without a clear theoretical explanation or account of individual neuron behavior.

% In this work, we demonstrate a connection between grokking and emergent \textit{sparsity} related to targeted norm growth of specific neurons.
In this work, we aim to better understand grokking on sparse parity \citep{barak2022hidden} by studying the \textit{sparsity} and \textit{computational structure} of the model over time. We empirically demonstrate a connection between grokking, emergent sparsity, and competition between different structures inside the model (\Cref{fig:three_phases}).
We first show that, after grokking, network behavior is controlled by a sparse subnetwork (but by a dense one before the transition).
Aiming to better understand this sparse subnetwork, we then demonstrate that the grokking phase transition corresponds to accerelated norm growth in a \emph{specific} set of neurons, and norm decay elsewhere.
After this norm growth, we find that the targeted neurons quickly begin to dominate network predictions, leading to the emergence of the sparse subnetwork.
We also find that the size of the sparse subnetwork corresponds to the size of a disjunctive normal form circuit for computing parity, suggesting this may be what the model is doing.
Taken together, our results suggest grokking arises from targeted norm growth of specific neurons within the network. This targeted norm growth sparsifies the network, potentially enabling generalizing discrete behavior that is useful for algorithmic tasks.

\begin{figure}
    \centering
    \includegraphics[scale=0.5]{figures/document_F107_11308.pdf}
    \caption{An illustration of the structure of a neural network during training in algorithmic tasks. Neural networks often exhibit a memorization phase that corresponds to a dense network, followed by the generalization phase where a sparse, largely disjoint to the prior one, model takes over.}
    \label{fig:three_phases}
\end{figure}

% \will{Should somewhere mention that conventional wisdom is that small norm $\implies$ sparse, but targeted large norm growth can have the same effect.}

% We therefore analyze the internals of the network during each phase of grokking, focusing on two key synthetic tasks where grokking has been documented in the literature. We make the following observations about these networks:

% \will{We don't fully have result 2 (and definitely don't have 3), but if we did, that would be a nice story.}

% \begin{enumerate}
%     \item The network is sparser during the generalization phase than during the memorization phase. \label{item:sparsity}
%     \item The sparse subnetwork active during the generalization phase is largely disjoint from the dense subnetwork active during the memorization phase. \label{item:disjoint}
%     \item Masking everything besides the sparse subnetwork leads to nontrivial and monotonically improving generalization accuracy \emph{before} the grokking transition. \label{item:progress}
% \end{enumerate}

% These observations point to a speculative understanding of grokking on a mechanistic level. The grokking phase transition in loss corresponds to sparsification of the network (\Cref{item:sparsity}). It may be tempting to assume sparsification means compression of the memorization subnetwork into a smaller subnetwork approximating it. However, because the two subnetworks are disjoint (\Cref{item:disjoint}), it seems more likely that disjoint sparse and dense subnetworks exist independently throughout the course of training.
% \Cref{item:progress} suggests that the sparse subnetwork makes progress on solving the task in a way that generalizes throughout training, but that it only becomes visible once the sparse subnetwork ``takes over'' network predictions after the grokking transition. Together, our empirical observations suggest that grokking may arise from the competition of disjoint dense and sparse subnetworks, and that the grokking phase transition corresponds to a switch in which network controls network behavior at large.

\section{Tasks, Models, and Methods} \label{sec:methods}

\paragraph{Sparse Parity Function.} We focus on analyzing grokking in the problem of learning a sparse $(n, k)$-parity function \citep{barak2022hidden}.
A $(n, k)$-parity function takes as input a string $x \in \{ \pm 1 \}^n$ returns $\pi(x) = \prod_{i \in S} x_i \in \{ \pm 1 \}$, where $S$ is a fixed, \textit{hidden} set of $k$ indices. The training set consists of $N$ i.i.d. samples of $x$ and $\pi(x)$.
We call the $(n, k)$-parity problem \emph{sparse} when $k \ll n$, which is satisfied by our choice of $n = 40$, $k=3$, and $N=1000$. 
% \subsection{Learning Binary Operation Tables with Transformers}
% \citet{powers2022grokking} initially observed grokking training $2$-layer transformer language models on strings encoding entries in a binary operation table. That is, if the cell $a, b$ has the value $c$, the input string to the language model would be $a \circ b = c$.

\paragraph{Network Architecture.}
Following \citet{barak2022hidden}, we use a $1$-layer ReLU net:
\begin{align*}\label{eq:arch}
    f(x) &= u^\intercal \sigma (Wx + b) \\
    \Tilde{y} &= \mathrm{sgn}\left (f(x)\right) ,
\end{align*}
where $\sigma (x) = \max(0, x)$ is ReLU, $u \in \mathbb{R}^p, W \in \mathbb{R}^{p \times n}$, and $ b \in \mathbb{R}^p$.  We minimize the hinge loss $\ell(x, y) = \max(0, 1-f(x) y)$, using stochastic gradient descent (batch size $B$) with constant learning rate $\eta$ and (potential) weight decay of strength $\lambda$ (that is, we minimize the regularized loss $\ell(x, y) + \lambda \| \theta \|_2$, where $\theta$ denotes all the parameters of the model). 
Unless stated otherwise, we use weight decay $\lambda = 0.01$, learning rate $\eta = 0.1$ batch size $B = 32$, and hidden size $p = 1000$.
We train each network 5 times, varying the random seed for generating the train set and training the model, but keeping the test set of $100$ points fixed.

\subsection{Active Subnetworks and Effective Sparsity} \label{sec:sparsity}
% \paragraph{Active Subnetwork Identification.}
We use a variant of weight magnitude pruning \citep{MoSm89} to find active subnetworks that control the full network predictions. The method assumes a given support of input data $\mathcal X$. Let $f$ be the network prediction function and $f_{k}$ be the prediction where the $p-k$ neurons with the least-magnitude incoming edges have been pruned. We define the \emph{active subnetwork} of $f$ as $f_k$ where $k$ is minimal such that, for all $x \in \mathcal X$, $f(x) = f_{k}(x)$.

We will use the active subnetwork to identify important structures within a network during grokking. We can also naturally use it to measure sparsity: we define the \emph{effective sparsity} of $f$ as the number of neurons in the hidden layer of the active subnetwork of $f$.

% \begin{definition}
% Let $f(\cdot;W,b,u)$ be a neural network as in \ref{eq:arch}. Let $w_1, \ldots, w_p$ be an ordering of the rows of $W$ as per their norm (in decreasing order), that is $\|w_1\| > \|w_2\| > \ldots > \|w_p\|$. We define the \textit{effective sparsity} of $f(\cdot;W,b,u)$ as follows
% \begin{equation}
%     \mathrm{sp}(f(\cdot;W, b, u) = \min_{k \leq p} \{  \},
% \end{equation}
% and accordingly we call that model the \textit{effective subnetwork} of $f(\cdot;W,b,u)$.
% \end{definition}

% \begin{definition}
%     Memorizing subnetwork is the effective subnetwork of the model at the first epoch during training that the model reaches $1 - \epsilon$ training accuracy. $\epsilon$ is set to $0.02$ in our experiments.
% \end{definition}

% \begin{definition}
%     Generalizing subnetwork is the effective subnetwork of the model at the end of training (provided that it actually generalizes to the test set).
% \end{definition}

\section{Results}

% In this section, we describe our main experimental findings. Unless stated otherwise, we set input dimension $n=40$, parity size $k=3$, weight decay $\lambda = 0.01$, number of training points $N = 1000$, learning rate $\eta = 0.1$, batch size $B = 32$, width $p = 1000$ and refer to the Appendix for additional configurations.

We see in \Cref{fig:accloss_spars} that our sparse parity task indeed displays grokking, both in accuracy and loss. We now turn to analyzing the internal network structure before, during, and after grokking. We refer to Appendix \ref{app:other_configs} for additional configurations (smaller weight decay or larger parity size) that support our findings\footnote{Code available on \url{https://github.com/Tsili42/parity-nn}}.

\begin{figure}
    \centering
    \includegraphics[scale=0.27]{figures/n40_k3_N1000_lr01_wd001_width1000/combined1.pdf}
    \caption{Accuracy (left), Average Loss (middle) and Effective Sparsity (right) during training of an FC network on $(40, 3)$ parity. Generalization accuracy suddenly jumps from random chance to flawless prediction concurrent with sparsification of the model. Shaded areas show randomness over the training dataset sampling, model initialization, and stochasticity of SGD (5 random seeds).}
    \label{fig:accloss_spars}
\end{figure}

\subsection{Grokking Corresponds to Sparsification}

% We define the effective sparsity of architecture~(\ref{eq:arch}), as the minimum number of neurons $k \leq p$ (ordered by norm) that are needed for a pruned network to match the performance of the full network on the training dataset.
\Cref{fig:accloss_spars} (right) shows the effective sparsity~(number of active neurons; cf. \Cref{sec:sparsity}) of the network over time. Noticeably, it becomes orders of magnitude sparser as it starts generalizing to the test set, and crucially, this phase transition happens at the same time as the loss phase transition. This can be directly attributed to the norm regularization being applied in the loss function, as it kicks in right after we reach (almost) zero in the data-fidelity part of the loss. Interestingly, this phase transition can be calculated solely from the training data but correlates with the phase transition in the test accuracy.

\citet{nanda2023progress} observe sparsity in the Fourier domain after grokking, whereas we have found it in the conventional network structure as well.
Motivated by the discovery of this sparse subnetwork, we now turn our attention to understanding why this subnetwork emerges and its structure.

% \begin{figure}
%     \centering
%     \includegraphics[scale=0.4]{figures/n40_k3_N1000_lr01_wd001_width1000/sparsity.pdf}
%     \caption{Sparsity over time}
%     \label{fig:sparsity}
% \end{figure}

% \subsection{The Grokking Phases Use Different Subnetworks}
% \subsection{Norm Growth and Competition During Grokking Phases}
\subsection{Selective Norm Growth Induces Sparsity During Grokking}

Having identified a sparse subnetwork of neurons that emerges to control network behavior after the grokking transition, we study the properties of these neurons throughout training (before the formation of the sparse subnetwork). \Cref{fig:subnetworks} (left) plots the average neuron norm for three sets of neurons: the neurons that end up in the sparse subnetwork, the complement of those neurons, and a set of random neurons with the same size as the sparse subnetwork.
We find that the 3 networks have similar average norm up to a point slightly before the grokking phase transition, at which the generalizing subnetwork norm begins to grow rapidly.
% \footnote{Somewhat similarly, \citet{barak2022hidden} observe selective norm growth for networks learning sparse parity functions with modified sinusoidal networks in the online setting.}

In \Cref{fig:subnetworks} (right), we measure the faithfulness of the neurons in the sparse subnetwork over time: in other words, the ability of these neurons alone to reconstruct the full network predictions on the test set, measured as accuracy. The grokking phase transition corresponds to these networks emerging to fully explain network predictions, and we believe this is likely a causal effect of norm growth.\footnote{Conventional machine learning wisdom associates small weight norm with sparsity, so it may appear counterintuitive that \emph{growing} norm induces sparsity. We note that the growth of selective weights can lead to effective sparsity because the large weights dominate linear layers \citep{merrill-etal-2021-effects}.}
The fact that the performance of its complement degrades after grokking supports the conclusion that the sparse network is ``competing'' with another network to inform model predictions, and that the grokking transition corresponds to a sudden switch where the sparse network dominates model output.

The element of competition between the different circuits is further evident when plotting the norm of individual neurons over time. \Cref{fig:ind_neurons} in the Appendix shows that neurons active during the memorization phase slightly grow in norm before grokking but then ``die out", while the the neurons of sparse subnetwork are inactive during memorization and then explode in norm. The fact that the model is overparameterized allows this kind of competition to take place.

\subsection{Subnetwork Computational Structure}

\paragraph{Sparse Subnetwork.} Across $5$ runs, the sparse subnetwork has size $\{6, 6, 6, 8, 8\}$. This suggests that the network may be computing the parity via a representation resembling disjunctive normal form (DNF), via the following argument.
A standard DNF construction uses $2^k =8$ neurons to compute the parity of $k=3$ bits (\Cref{thm:dnf-general}). We also derive a modified DNF net that uses only $6$ neurons to compute the parity of $3$ bits (\Cref{thm:dnf-special}). Since our sparse subnetwork always contains either $6$ or $8$ neuron, we speculate it may always be implementing a variant of these constructions. However, there is an even smaller network computing an $(n, 3)$-parity with only 4 neurons via a threshold-gate construction, but it does not appear to be found by our networks (\Cref{thm:threshold}).

\paragraph{Dense Subnetwork.} The network active during the so-called memorization phase is not exactly memorizing. Evidence for this claim comes from observing grokking on the binary operator task of \citet{powers2022grokking}. For the originally reported division operator task, the network obtains near zero generalization prior to grokking (\Cref{fig:memorization_evidence}, right). However, switching the operator to addition, the generalization accuracy is above chance before grokking (\Cref{fig:memorization_evidence}, left). We hypothesize this is because the network, even pre-grokking, can generalize to unseen data since addition (unlike division) is commutative. In this sense, it is not strictly memorizing the training data.

% \paragraph{Weight Decay.}
% In our toy learning problem, this sparsity property can be directly attributed to the weight decay that was added in the optimization objective (since it happens after the data-dependent part of the loss becomes 0). \will{Nikos should clarify this, idk exactly what the claim is}

% We measure the norm of each individual neuron, i.e the $i$-th row of $W$ for $i \in \{1, \ldots, p\}$. \Cref{fig:neuron_norm} (left) shows these norms over training. We observe that many neurons are driven to zero (eventually dying out), while a small set ``wakes up" during or right before grokking happens. On \Cref{fig:neuron_norm} (right) we single out two neurons that have the following properties: they had the largest norm on the onset of either (a) the \textit{memorization}, or (b) \textit{generalization} phase. We see exactly opposite behaviors. The memorizing neuron grows in norm as the model fits more and more the training data, while the generalizing one shrinks. However, at a certain point, its norm starts exploding (mimicking the global accuracy curve), and the fellow neuron goes to zero. This very interesting trend hints already on the existence of two difference subcircuits on the network: the \textit{memorizing} one that overfits the training data, and a mutually-exclusive one, the \textit{generalizing} one that emerges during and after grokking, becoming the dominant one.
% \aman{Do we have evidence for mutual exclusivity? Are we implying that there is no common neuron between the two sub-circuits. I'm not sure if that's the case.}

% Claims supported by parity analysis:
% \begin{itemize}
% \item Each grokking phase transition corresponds to the emergence of a different saturated (\nikos{more evidence}) subcircuit
% \item On parity, we found two distinct subcircuits are responsible by grokking. On OpenAI, we need to order neurons by norm and do binary search over them. Then measure saturation on each circuit alone.
% \item Should justify that lots of neurons in circuit 2 are not in neuron 1
% \end{itemize}

% \will{Make a separate claim about \Cref{fig:neuron_norm}? We get a really clear pattern here, worth emphasizing. The current section title could be improved. Then again, maybe it just belongs with claim 3.1}

\begin{figure}
    \centering
    \includegraphics[scale=0.3]{figures/n40_k3_N1000_lr01_wd001_width1000/combined_faithfulness.pdf}
    \caption{Left: Average norm of different subnetworks during training. Right: Agreement between the predictions of a subnetwork and the full network on the test set. The \textit{generalizing} subnetwork is the final sparse net, the \textit{complementary} subnetwork is its complement, and the \textit{control} subnetwork is a random network with the same size as the generalizing one.}
    \label{fig:subnetworks}
\end{figure}

\section{Conclusion}

We have shown empirically that the grokking phase transition, at least in a specific setting, arises from competition between a sparse subnetwork and the (dense) rest of the network.
% In this work, we provided empirical evidence that phase transitions that occur during training of neural networks are accompanied by gradual sparsification and, also, competition between different components of the network.
Moreover, grokking seems to arise from selective norm growth of this subnetwork's neurons.
As a result, the sparse subnetwork is largely inactive during the memorization phase, but soon after grokking, fully controls model prediction.
% Along the lines of the lottery ticket hypothesis \citep{FrCa19}, we hypothesize that the specific structure of this sparse subnetwork leads to better generalization, which could be tested in future work.

We speculate that norm growth and sparsity may facilitate emergent behavior in large language models similar to their role in grokking. 
As preliminary evidence, \citet{merrill-etal-2021-effects} observed monotonic norm growth of the parameters in T5, leading to ``saturation'' of the network in function space.\footnote{Saturation measures the discreteness of the network function, but may relate to effective sparsity.} More promisingly, \citet{dettmers2022gptint} observe that a targeted subset of weights in pretrained language models have high magnitude, and that these weights overwhelmingly explain model predictions.
% Based on our analysis of grokking, future work could investigate whether the emergence of this high-norm subnetwork in language models corresponds to novel emergent behaviors. 
It would also be interesting to extend our analysis of grokking to large language models: specifically, does targeted norm growth subnetworks of large language models \citep{dettmers2022gptint} facilitate emergent behavior?
% Speculatively, lottery ticket networks may have a special sparse structure that is amenable to solving the task they are trained on, and the training process could thus be seen as searching for them.

\section{Acknowledgements}

This material is based upon work supported by the National Science Foundation under NSF Award 1922658.

\bibliography{references.bib}
\bibliographystyle{iclr2023_conference}

\appendix
\section{Binary Operator Experiments}

We trained a decoder only transformer with 2 layers, width 128, and 4 attention heads \citep{powers2022grokking}. In both operator settings, we used the AdamW optimizer, with a learning rate of $10^{-3}$, $\beta_1 = 0.9$ and $\beta_2 = 0.98$, weight decay equal to 1, batch size equal to 512, 9400 sample points and an optimization limit of $10^5$ updates. We repeated the experiments for both operators with 3 random seeds and aggregated the results.  

\begin{figure}[H]
    \centering
    \includegraphics[scale=0.4]{figures/commutative_compare.pdf}
    \caption{Accuracy curves for addition (left) and division (right). For the addition operator, the dashed line represents the \% of dataset that can be solved by commuting test points and then looking them up in the memorized training set.
    The generalization accuracy before grokking matches this level, suggested that the network has learned to generalize the commutative property of addition before it learns to generalize fully.}
    \label{fig:memorization_evidence}
\end{figure}

\section{Additional plots}

\begin{figure}[H]
    \centering
    \includegraphics[scale=0.42]{figures/n40_k3_N1000_lr01_wd001_width1000/combined3.pdf}
    \caption{Weight norm of individual neurons during training. Left: Evolution of the dominant neurons during the memorization epoch (first time we hit $>$ 98\% train accuracy) and final epoch (that corresponds to the generalizing subnetwork). Right: Weight norm over time for all neurons. Notice that most of them are driven to 0.}
    \label{fig:ind_neurons}
\end{figure}

\section{Additional Configurations}\label{app:other_configs}

We provide accuracy, loss, sparsity, subnetwork norm and subnetwork faithfulness plots for smaller weight decay (Figures \ref{fig:wd0001_acc} and \ref{fig:wd0001_sub}), and for larger parity size (Figures \ref{fig:k4_acc} and \ref{fig:k4_sub}). The experimental observations are consistent with those of the main body of the paper.

\begin{figure}
    \centering \includegraphics[scale=0.27]{figures/n40_k3_N1000_lr01_wd0001_width1000/combined1.pdf}
    \caption{Reproduction of Figure \ref{fig:accloss_spars} for smaller weight decay $\lambda=0.001$ (the rest of the hyperparameters are the same as in the standard setup). Accuracy (left), Average Loss (middle) and Effective Sparsity (right) during training of an FC network on $(40, 3)$-parity.}
    \label{fig:wd0001_acc}
\end{figure}

\begin{figure}
    \centering
    \includegraphics[scale=0.3]{figures/n40_k3_N1000_lr01_wd0001_width1000/combined_faithfulness.pdf}
    \caption{Reproduction of Figure \ref{fig:subnetworks} for smaller weight decay $\lambda=0.001$ (the rest of the hyperparameters are the same as in the standard setup). Left: Average norm of different subnetworks during training. Right: Agreement between the predictions of a subnetwork and the full network on the test set.}
    \label{fig:wd0001_sub}
\end{figure}

\begin{figure}
    \centering
    \includegraphics[scale=0.27]{figures/n40_k4_N1000_lr01_wd001_width1000/combined1.pdf}
    \caption{Reproduction of Figure \ref{fig:accloss_spars} for larger parity size $k=4$ (the rest of the hyperparameters are the same as in the standard setup). Accuracy (left), Average Loss (middle) and Effective Sparsity (right) during training of an FC network on $(40, 4)$ parity.}
    \label{fig:k4_acc}
\end{figure}

\begin{figure}
    \centering
    \includegraphics[scale=0.3]{figures/n40_k4_N1000_lr01_wd001_width1000/combined_faithfulness.pdf}
    \caption{Reproduction of Figure \ref{fig:subnetworks} for larger parity size $k=4$ (the rest of the hyperparameters are the same as in the standard setup). Left: Average norm of different subnetworks during training. Right: Agreement between the predictions of a subnetwork and the full network on the test set.}
    \label{fig:k4_sub}
\end{figure}


\section{Computing Parity with Neural Nets} \label{sec:parity}

We say that a neural net of the form defined in \autoref{sec:methods} computes parity iff its output is positive when the parity is $1$ and negative otherwise.

We first show a general way to represent parity in ReLU networks for any parity size $k$. This construction requires $2^k$ hidden neurons.

\begin{proposition} \label{thm:dnf-general}
For any $n$, there exists a 1-layer ReLU net with $2^k$ neurons that computes $(n, k)$-parity.
\end{proposition}

\begin{proof}
We use each $2^k$ neurons to match a specific configuration of the $k$ parity bits by using the first affine transformation to implement an AND gate (note that the bias term is crucial here). In the output layer, we add positive weight on edges from neurons corresponding to configurations with parity $1$ and negative weight for neurons corresponding to configurations with parity $-1$.
\end{proof}

In the case where $k=3$, we show that there is a simpler construction with $6$ neurons.

\begin{proposition} \label{thm:dnf-special}
For any $n$, there exists a 1-layer ReLU net with $6$ neurons that computes $(n, 3)$-parity.
\end{proposition}

\begin{proof}
Let $x_1, x_2, x_3$ be the $3$ parity bits. We construct $\mathbf h \in \mathbb R^6$ as follows, where $\sigma$ is ReLU:
\begin{align*}
    h_1 &= \sigma(-x_1 + -x_2 + 10 x_3 - 9) \\
    h_2 &= \sigma(-x_1 - x_2 + x_3 - 2) \\
    h_3 &= \sigma(x_1 + x_2 + x_3 - 2) \\
    h_4 &= \sigma(x_1 - x_2 - 10 x_3 - 9) \\
    h_5 &= \sigma(-x_1 + x_2 - x_3 - 2) \\
    h_6 &= \sigma(x_1 - x_2 - x_3 - 2) .
\end{align*}
In the final layer, we assign $h_1$ and $h_4$ a weight of $-1$, and $h_2, h_3, h_5$, and $h_6$ a weight of $+10$.

To show correctness, we first characterize the logical condition that each neuron encodes:
\begin{align*}
    h_1 > 0 &\iff (x_1 = -1 \vee x_2 = -1) \wedge x_3 = 1 \\
    h_2 > 0 &\iff x_1 = -1 \wedge x_2 = -1 \wedge x_3 = 1 \\
    h_3 > 0 &\iff x_1 = 1 \wedge x_2 = 1 \wedge x_3 = 1 \\
    h_4 > 0 &\iff (x_1 = 1 \vee x_2 = -1) \wedge x_3 = -1 \\
    h_5 > 0 &\iff x_1 = -1 \wedge x_2 = 1 \wedge x_3 = -1 \\
    h_6 > 0 &\iff x_1 = 1 \wedge x_2 = -1 \wedge x_3 = -1 .
\end{align*}
In the final layer, $h_1$ and $h_4$ contribute a weight of $-1$ whenever the parity is negative (and in two other cases).
But in the four cases when the true parity is positive, one of the other neurons contributes a positive weight of $+10$.
Thus, the sign of the network output is correct in all $8$ cases. We conclude that this $6$-neuron network correctly computes the parity of $x_1, x_2$, and $x_3$.
\end{proof}

However, there is a $4$-neuron construction computing parity,\footnote{We thank anonymous reviewer 3kdq for demonstrating this construction.} which, interestingly, our networks do not find:

\begin{proposition} \label{thm:threshold}
    For any $n$, there exists a $1$-layer ReLU net with $4$ neurons that computes $(n, 3)$-parity.
\end{proposition}

\begin{proof}
    Let $X = x_1 + x_2 + x_3$ be the sum of the parity bits. We construct $\mathbf h \in \mathbb R^4$ as follows:
    \begin{align*}
        h_1 &= 1 \\
        h_2 &= \sigma(X - 1) \\
        h_3 &= \sigma(X + 1) \\
        h_4 &= \sigma(-X - 1) .
    \end{align*}
    In the final layer, we assign $h_1$ a weight of $1$, $h_2$ a weight of $2$, and $h_3$ and $h_4$ a weight of $-1$. We proceed by cases over the possible values of $X \in \{\pm 1, \pm 3\}$, which uniquely determines the parity:
    \begin{enumerate}
        \item \underline{$X = -3$:} Then there are three input bits with value $-1$, so the parity is $-1$. We see that $h_1 = 1$, $h_2 = 0$, $h_3 = 0$, and $h_4 = 2$. So the output is $h_1 - h_4 = -1$.
        \item \underline{$X = -1$:} Then there are two input bits with value $-1$, so the parity is $1$. We see that $h_1 = 1$, $h_2 = 0$, $h_3 = 0$, and $h_4 = 0$. So the output is $h_1 = 1$.
        \item \underline{$X = 1$:} Then there is one input with value $-1$, so the parity is $-1$. We see that $h_1 = 1$, $h_2 = 0$, $h_3 = 2$, and $h_4 = 0$. So the output is $h_1 - h_3 = -1$.
        \item \underline{$X = 3$:} Then there are no inputs with value $-1$, so the parity is $1$. We see that $h_1 = 1$, $h_2 = 2$, $h_3 = 4$, and $h_3 = 0$. So the output is $h_1 + 2h_2 - h_3 = 1$.
    \end{enumerate}
    We conclude that this $4$-neuron network correctly computes the parity of $x_1$, $x_2$, and $x_3$.
\end{proof}

\end{document}
