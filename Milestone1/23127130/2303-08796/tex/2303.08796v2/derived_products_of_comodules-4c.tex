\documentclass{amsart}
\usepackage{shortsalch}
%\usepackage{MnSymbol}
\usepackage{xy}
%\usepackage{csquotes}
%\usepackage{stmaryrd}
%\usepackage{tikz,placeins}
\usepackage{mathtools}
\xyoption{all}
\usepackage{cleveref}
\title[Derived limit in comodules over the dual Steenrod algebra]{Derived functors of product and limit in the category of comodules over the dual Steenrod algebra.}
%\date{February 2021}
\author{A. Salch}
\begin{document}
\begin{abstract}
In the 2000s, Sadofsky constructed a spectral sequence which converges to the mod $p$ homology groups of a homotopy limit of a sequence of spectra. The input for this spectral sequence is the derived functors of sequential limit in the category of graded comodules over the dual Steenrod algebra. Since then, there has not been an identification of those derived functors in more familiar or computable terms. Consequently there have been no calculations using Sadofsky's spectral sequence except in cases where these derived functors are trivial in positive cohomological degrees. 

In this paper, we prove that the input for the Sadofsky spectral sequence is simply the graded local cohomology of the Steenrod algebra, taken with appropriate (quite computable) coefficients. 
This turns out to require both some formal results, like some general results on torsion theories and local cohomology of noncommutative non-Noetherian rings, and some decidedly non-formal results, like a 1985 theorem of Steve Mitchell on some very specific duality properties of the Steenrod algebra not shared by most finite-type Hopf algebras. Along the way there are a few results of independent interest, such as an identification of the category of graded $A_*$-comodules with the full subcategory of graded $A$-modules which are torsion in an appropriate sense.
\end{abstract}
\maketitle
\tableofcontents


\section{Introduction.}

Let $k$ be a field.
It is well-known that the category $\Comod(\Gamma)$ of comodules over a $k$-coalgebra $\Gamma$ is abelian and has generally agreeable properties, like being complete and co-complete and having enough injectives, but unlike the category of modules over a ring, $\Comod(\Gamma)$ does not necessarily satisfy Grothendieck's axiom $AB4^*$ \cite{MR0102537}. Recall that $AB4^*$ is the axiom that states that infinite products of exact sequences are exact. Consequently, given a set $\{ M_i: i\in I\}$ of $\Gamma$-comodules, we may have nonvanishing higher right-derived functors $R^n\prod_{i\in I}^{\Gamma} M_i$ of the categorical product $\prod^{\Gamma}$ in the category of $\Gamma$-comodules. One consequence is that, given a sequence $\dots \rightarrow M_2 \rightarrow M_1 \rightarrow M_0$ of $\Gamma$-comodule homomorphisms, we may have nonvanishing right-derived functors $R^n\lim^{\Gamma}_i M_i$ for $n>1$, unlike the familiar situation in categories of modules; see \cite{MR132091} and \cite{MR2197371} for this implication.

There are important topological consequences. Let $p$ be a prime number, suppose that the ground field $k$ is the field $\mathbb{F}_p$ with $p$ elements, and suppose that the coalgebra $\Gamma$ is the $p$-primary dual Steenrod algebra. In the influential but unpublished 2001 preprint \cite{sadofsky2001homology}, H. Sadofsky constructed a spectral sequence 
\begin{align}
\label{sadofsky ss 1} E_2^{*,*} \cong R^*\lim^{\Gamma}_i H_*(X_i; \mathbb{F}_p) &\Rightarrow H_*\left( \holim_i X_i; \mathbb{F}_p\right)\end{align}
for a sequence $\dots \rightarrow X_2 \rightarrow X_1 \rightarrow X_0$ of $H\mathbb{F}_p$-nilpotently complete spectra. Since then, Sadofsky's spectral sequence has been written about, and the details of its construction are available in the published literature: see the appendix of \cite{MR4080481}, for example, or \cite{MR2337861} for the case of products of spectra, where one has a spectral sequence
\begin{align}
\label{sadofsky ss 2} E_2^{*,*} \cong R^*\prod^{\Gamma}_i H_*(X_i; \mathbb{F}_p) &\Rightarrow H_*\left( \prod_i X_i; \mathbb{F}_p\right)\end{align}
for any set $\{ X_i: i\in I\}$ of spectra.
We will refer to \eqref{sadofsky ss 1} as the {\em Sadofsky spectral sequence,} and \eqref{sadofsky ss 2} as the {\em Hovey-Sadofsky spectral sequence.}
Countable products can of course be treated as a special case of sequential limits, so it is easy to see the countable case of the Hovey-Sadofsky spectral sequence as a special case of the Sadofsky spectral sequence.

It is a classical theorem of Adams (Theorem III.15.2 of \cite{MR1324104}) that mod $p$ homology commutes with homotopy limits of {\em uniformly bounded-below} sequences of spectra. That is, $H_*\left( \holim_i X_i;\mathbb{F}_p\right) \rightarrow \lim_i H_*(X_i;\mathbb{F}_p)$ is an isomorphism if there is a {\em uniform} lower bound on the degrees of the nonvanishing homotopy groups of the spectra $X_0, X_1, X_2, \dots$. Sadofsky's spectral sequence is the only available general tool for calculating homology of homotopy limits of spectra in the absence of a uniform lower bound on their homotopy groups. 

However, to date there has been little known about the input of the Sadofsky or Hovey-Sadofsky spectral sequences, because it has been unclear whether there could be some practical means of calculating the derived functors of sequential limit, or of product, in categories of graded comodules. As Behrens and Rezk write in \cite{MR4094969} about the Sadofsky spectral sequence, ``[t]he $E_2$-term of this spectral sequence is in general quite mysterious'', and as Hovey writes in \cite{MR2337861} about the Hovey-Sadofsky spectral sequence, ``[a]lmost nothing is known about these right derived functors''. 
The purpose of this paper is to develop some general theory and some practical tools for calculating derived functors of sequential limits and of products in the category of graded comodules over a graded coalgebra, with a particular emphasis on the dual Steenrod algebras as our motivating examples. The basic strategy is to use the {\em covariant} embedding of the category of graded $\Gamma$-comodules into the category of graded $\Gamma^*$-modules, in order to identify the derived functors of product in the former category with some more familiar cohomological invariant in the latter category.

Below, we survey the results in this paper, but first we present the most topologically compelling results, which appear as Corollaries \ref{derived products are local cohomology} and \ref{steenrod alg seq cor}:
\begin{unnumberedcorollary}
Let $p$ be a prime, let $\Gamma^*$ be the mod $p$ Steenrod algebra, and let $\Gamma$ be its dual coalgebra. Let $\{ M_i: i\in I\}$ be a set of bounded-above\footnote{To be clear, the grading on all modules and comodules in this paper is the {\em cohomological} grading. Hence, if $M_i$ is the homology of a spectrum---the most relevant case for the Sadofsky and Hovey-Sadofsky spectral sequence---then the relevant assumption is that the spectrum should be bounded {\em below.}

Furthermore, the assumption here is only that {\em each} graded $\Gamma$-comodule is bounded above. There does not need to be {\em uniform} upper bound on these comodules. In the case where there is a {\em uniform} upper bound on the comodules, none of the work we do in this paper is necessary, because it is easy to show that the comodule product is exact.} graded $\Gamma$-comodules. Then the $n$th derived functor $R^n\prod_{i\in I}^{\Gamma} M_i$ of the product of the $M_i$ in the category of graded $\Gamma$-comodules is isomorphic, as an abelian group, to \[ \underset{j\rightarrow\infty}{\colim} \Ext_{\Gamma^*}^n\left( \Gamma^*/I_j,\prod_{i\in I} \iota M_i\right),\] where $I_j$ is the ideal of the Steenrod algebra $\Gamma^*$ generated by all homogeneous elements of degree $\geq j$, and where $\iota M_i$ is $M_i$ regarded as a graded $\Gamma^*$-module via the adjoint $\Gamma^*$-action. That is, the derived functors of product in the category of graded $\Gamma$-comodules are given, on families of bounded-above comodules, by the graded local cohomology of the Steenrod algebra $\Gamma^*$ with coefficients in the product of the adjoint $\Gamma^*$-modules.
\end{unnumberedcorollary}
\begin{unnumberedcorollary}
Suppose that $p$ is a prime number. Write $\Gamma$ for the dual $p$-primary Steenrod algebra. Let \begin{equation*}%\label{seq 430949598c}
\dots\rightarrow M_2\rightarrow M_1\rightarrow M_0\end{equation*} be a sequence of bounded-above graded $\Gamma$-comodules such that $R^1\lim$ vanishes on the sequence of graded $\Gamma^*$-modules 
$\dots\rightarrow \iota M_2\rightarrow \iota M_1\rightarrow \iota M_0$.
Then we have an isomorphism of graded $\Gamma^*$-modules
\[ \iota R^*\lim^{\Gamma}_i M_i  \cong \underset{j\rightarrow\infty}{\colim} \Ext^*_{\Gamma^*}\left(\Gamma^*/I_j,\lim_i \iota M_i\right)\]
where $I_j$ is the ideal of the $p$-primary Steenrod algebra $\Gamma^*$ generated by all homogeneous elements of degree $\geq j$.
That is, if $R^1\lim$ vanishes on the sequence of adjoint $\Gamma^*$-modules, then the derived functors of sequential limit in the category of graded $\Gamma$-comodules are given, on sequences of bounded-above comodules, by the graded local cohomology of the Steenrod algebra $\Gamma^*$ with coefficients in the limit of the adjoint $\Gamma^*$-modules.
\end{unnumberedcorollary}
For bounded-above comodules, these results reduce the calculation of the input of the Sadofsky and Hovey-Sadofsky spectral sequences to the calculation of graded local cohomology of the Steenrod algebra. Graded local cohomology is already a relatively familiar and well-studied theory, and there are already some computational tools for it, so we regard this identification of the Sadofsky and Hovey-Sadofsky spectral sequence $E_2$-terms as the main selling point of this paper.

We owe the reader some explanation of the peculiarities of {\em graded} local cohomology. Classically, given a commutative ring $R$, an ideal $I$ of $R$, and an $R$-module $M$, the local cohomology of $M$ at the ideal $I$ is defined by \begin{equation}\label{local coh old} H^*_I(M) := \underset{j\rightarrow\infty}{\colim}\Ext_R(R/I^j,M).\end{equation} In the literature (see \cite{MR1428799}, for example), when $R$ is a nonnegatively-graded Noetherian ring, not necessarily commutative, the {\em graded} local cohomology of a left $R$-module $M$ is defined as the colimit 
\begin{equation}\label{graded local coh}
 \underset{j\rightarrow\infty}{\colim}\Ext_R(R/I_j,M),
\end{equation}
where $I_j$ is the ideal of $R$ generated by all homogeneous elements of degree $\geq j$. The Noetherian hypothesis yields the isomorphism of \eqref{graded local coh} with $\underset{j\rightarrow\infty}{\colim}\Ext_R(R/I^j,M)$, where $I = I_1$ is the ideal of all elements of $R$ of positive degree. The Steenrod algebra, however, is graded but not Noetherian, and there is no guarantee that \eqref{graded local coh} must coincide with $\underset{j\rightarrow\infty}{\colim}\Ext_R(R/I^j,M)$. In this paper, we adopt \eqref{graded local coh}, rather than \eqref{local coh old}, as the definition of graded local cohomology for non-Noetherian graded rings like the Steenrod algebra.

Here are the other main results in this paper:
\begin{enumerate}
\item \Cref{Local cohomology functors associated...} covers elementary notions of local cohomology associated to a set of (left) ideals in a (not necessarily commutative) ring $R$. While local cohomology is well-studied over commutative rings, for the purposes of this paper we must consider local cohomology of {\em non}-commutative rings like the Steenrod algebra. Some familiar properties of local cohomology of commutative rings fail in surprising and treacherous ways in the absence of commutativity. Consequently we have to spend a large portion of this paper developing some necessary theory of local cohomology of noncommutative rings.

Given a set $S$ of ideals in $R$, we can consider the group \[ h^0_S(M)\coloneqq \underset{I\in S}{\colim} \hom_R(R/RI,M)\] of elements of an $R$-module $M$ which are $I$-torsion with respect to some ideal $I\in S$, or we can consider the $R$-submodule $H^0_S(M)$ of $M$ generated by the subgroup $h^0_S(M)$ of $M$. When $R$ is commutative, the functors $h^0_S$ and $H^0_S$ coincide, but for some noncommutative rings and some choices of $S$, they do not agree. Examples are given in Remark \ref{remark on ideal sets}. While $h^0_S$ is more familiar and computationally accessible, it is $H^0_S$ that carries the essential data about the relationship between comodules and modules, as we describe below.
\item \Cref{Comodules as a pretorsion...} reviews the well-known covariant embedding (via the adjoint action) of $\Gamma$-comodules into $\Gamma^*$-modules, and then reviews well-known ideas from torsion theory, and proves some preliminary results on the relationship between the two. A $\Gamma^*$-module $M$ is called {\em rational} if $M$ is in the image of this embedding. 
The main idea in \cref{Comodules as a pretorsion...} is that, given a coalgebra $\Gamma$ over a field, one can define a certain set $\dist(\Gamma)$ of ideals in $\Gamma^*$, the {\em distinguished ideals}, such that a $\Gamma^*$-module is rational if and only if the natural map $H^0_{\dist(\Gamma)}(M) \rightarrow M$ is an isomorphism. This is the content of Theorem \ref{theta-rationality and torsion}. 
One consequence is Corollary \ref{H0 is left exact}, which establishes that the functor $H^0_{\dist(\Gamma)}$ is left exact. (It is also true, but much easier to prove, that $h^0_{\dist(\Gamma)}$ is left exact.) Consequently we have some familiar homological tools for dealing with the right derived functors $H^*_{\dist(\Gamma)} \coloneqq R^*H^0_{\dist(\Gamma)}$, which we call {\em distinguished local cohomology.} Since the covariant embedding of comodules into modules is full and faithful, Theorem \ref{theta-rationality and torsion} characterizes $\Gamma$-comodules in terms of $\Gamma^*$-modules: the category of $\Gamma$-comodules is equivalent to the full subcategory of the $\Gamma^*$-modules generated by those $\Gamma^*$-modules $M$ such that $H^0_{\dist(\Gamma)}(M) \rightarrow M$ is an isomorphism.

I do not know of anywhere where Theorem \ref{theta-rationality and torsion} already appears in the literature, and I have never heard or seen others mention the idea. Still I am uneasy about calling it a terribly novel result: the work involved is a question of developing some (largely formal) theory and using it alongside ideas in sections 7 and 41 of \cite{MR2012570}. I regard \cref{Comodules as a pretorsion...} as a section devoted to background and review and developing some formal theory, and the hard ``non-formal'' work in this paper does not really begin until \cref{Distinguished local cohomology...}.
\item With \cref{Distinguished local cohomology...}, we begin to prove new results, at the cost of having to narrow the level of generality. Under the assumption that $\Gamma$ is a finite-type, connected coalgebra over a field, satisfying the condition that every uniformly bounded-below product of free graded $\Gamma^*$-modules is graded-projective\footnote{This condition is satisfied when $\Gamma^*$ is the mod $p$ Steenrod algebra, for any prime $p$.}, Theorem \ref{main thm 1} shows that the rational graded $\Gamma^*$-modules (i.e., the graded $\Gamma$-comodules) form a hereditary pretorsion class in the category of graded $\Gamma^*$-modules. The same theorem furthermore shows that the higher distinguished local cohomology groups vanish on bounded-above graded $\Gamma^*$-modules, and that the bounded-above graded $\Gamma^*$-modules are rational. 

Most importantly, Theorem \ref{main thm 1} shows (under the same assumptions) that for any integer $n$ and any set $\{ M_i: i\in I\}$ of bounded-above graded $\Gamma$-comodules, the $n$th right-derived functor $R^n\prod_{i\in I}^{\Gamma} M_i$ of product in the category of graded $\Gamma$-comodules agrees (as a graded $\Gamma^*$-module via the adjoint action) with the distinguished local cohomology $H^n_{\dist(\Gamma)}\left(\prod_{i\in I} \iota(M_i)\right)$, where $\iota$ is the covariant embedding of comodules into modules. The point is that $\prod_{i\in I} \iota(M_i)$ is the {\em ordinary, familiar} Cartesian product of graded $\Gamma^*$-modules, not the more obscure categorical product in comodules. Consequently, if we reduce the distinguished local cohomology groups $H^n_{\dist(\Gamma)}$ to some familiar homological invariants, like a colimit of $\Ext$-groups, then we have a practical means of calculating $R^n\prod_{i\in I}^{\Gamma} M_i$, and consequently of calculating the homology groups of infinite products of spectra, via the Hovey-Sadofsky spectral sequence. Much of the rest of the paper is devoted to proving that $H^n_{\dist(\Gamma)}$ is indeed a straightforward colimit of $\Ext$-groups in sufficient generality to apply when $\Gamma$ is the dual Steenrod algebra at any prime.

One consequence of Theorem \ref{main thm 1} is Theorem \ref{structure thm}: if $\Gamma^*$ is finite-type and connected over a field, and if every uniformly bounded-below product of graded free $\Gamma^*$-modules is graded-projective, then for each integer $n$, every graded $\Gamma^*$-module $M$ is canonically an extension of an $n$-co-connected rational graded $\Gamma^*$-module by an $n$-connective graded $\Gamma^*$-module $\conn_n(M)$. Furthemore, the higher distinguished local cohomology groups of $M$ depend only on those of $\conn_n(M)$. Under the same hypotheses, we then get Corollary \ref{modules are limits of comodules}, which states that every graded $\Gamma^*$-module is a limit of a Mittag-Leffler sequence of rational $\Gamma^*$-modules. We also get Corollary \ref{uniqueness of module cat}, which states that the only full subcategory of $\gr\Mod(\Gamma^*)$ which contains the rational $\Gamma^*$-modules and which is closed under kernels and countable products is $\gr\Mod(\Gamma^*)$ itself.
\item \Cref{Mitchell coalgebras...} introduces the notion of a {\em Mitchell coalgebra}, a coalgebra over a field admitting a certain kind of decomposition which is intended to resemble the decomposition of the dual Steenrod algebra $A_*$ as the limit of the coalgebras $A(n)_*$. In \cite{MR793186}, S. Mitchell identified certain self-duality and compatibility properties of this decomposition of $A_*$, and we include analogous properties as part of the definition of a Mitchell coalgebra in Definition \ref{def of mitchell condition}, because we find in Theorem \ref{main thm on mitchell coalgebras} that precisely these properties can be used to show that the two local cohomology functors $h^*_{\dist(\Gamma)}$ and $H^*_{\dist(\Gamma)}$ coincide over the dual of a Mitchell coalgebra $\Gamma$. 
As a consequence, the distinguished local cohomology $H^*_{\dist(\Gamma)}$---which, by Theorem \ref{main thm 1}, computes the derived functors of product in the category of $\Gamma$-comodules---has the good computational properties of $h^*_{\dist(\Gamma)}$, and in particular, it is isomorphic to a colimit of $\Ext$-groups.

Since Mitchell's results in \cite{MR793186} establish that the dual Steenrod algebra is what we call a Mitchell coalgebra, we get the main results of this paper: if $\Gamma$ is the dual Steenrod algebra at any prime, then Corollary \ref{main cor 10} establishes that, for any graded $\Gamma^*$-module $M$, the distinguished local cohomology $H^*_{\dist(\Gamma)}(M)$ is isomorphic to the graded local cohomology $\underset{j\rightarrow\infty}{\colim} \Ext_{\Gamma^*}^*\left( \Gamma^*/I_j,M\right)$ of the Steenrod algebra $\Gamma^*$. A consequence is Corollary \ref{derived products are local cohomology}: if $\{ M_i: i\in I\}$ is a set of bounded-above graded $\Gamma$-comodules, then the $n$th derived functor $R^n\prod^{\Gamma}_{i\in I} M_i$ of product in the category of graded $\Gamma$-comodules is isomorphic, as an abelian group, to $\underset{j\rightarrow\infty}{\colim} \Ext_{\Gamma^*}^n\left( \Gamma^*/I_j,\prod_i M_i\right)$. Here $\prod_i M_i$ is the product (in the category of {\em modules}, i.e., the Cartesian product) of the graded $\Gamma^*$-modules $M_i$ with the adjoint action of $\Gamma^*$.
\item An abelian category satisfies Grothendieck's axiom $AB4^*$ if and only if products exist and are exact in that category. Weakenings of $AB4^*$ have been studied: given an integer $n$, an abelian category $\mathcal{C}$ is said to satisfy axiom $AB4^*\mhyphen (n)$ if and only if the $m$th derived functor of product in $\mathcal{C}$ vanishes for all $m > n$. The condition $AB4^*\mhyphen (0)$ is equivalent to $AB4^*$. When $\Gamma^*$ is the Steenrod algebra at some prime, one knows (e.g. from experience with the Adams spectral sequence) that $\Ext_{\Gamma^*}^{n}(M,N)$ is capable of being nonzero for arbitrarily large $n$, but this does not rule out the possibility of a finite bound on the integers $n$ such that $\underset{j\rightarrow\infty}{\colim} \Ext_{\Gamma^*}^{n}(\Gamma^*/I_j,N)$ is nonzero for some $N$. In other words, it seems plausible that the category of graded $\Gamma$-comodules satisfies axiom $AB4^*\mhyphen (n)$ for some $n$, and consequently that there is a uniform horizontal vanishing line in the $E_2$-term for the Hovey-Sadofsky spectral sequence calculating the mod $p$ homology of infinite products of spectra.

The purpose of \cref{Bounds on distinguished...} is to show that we are not, in fact, so lucky: Corollary \ref{no ab4n for dual steenrod alg} shows that the category of comodules over the dual Steenrod algebra, at any prime, cannot satisfy axiom $AB4^*\mhyphen (n)$ for any $n$ at all. This is a consequence of Theorem \ref{main thm 4}, which shows that, if $\Gamma$ is a coalgebra over a field satisfying some appropriate hypotheses (satisfied for the dual Steenrod algebras), then the category of bounded-above graded $\Gamma$-comodules satisfies $AB4^*\mhyphen (n)$ for some $n$ if and only if it satisfies $AB4^*$. The key to applying Theorem \ref{main thm 4} to the dual Steenrod algebra is a theorem of Margolis: Margolis has proven that the Steenrod algebras (more generally, the ``$\mathcal{P}$-algebras'' in Margolis' sense) have the properties assumed in the hypotheses of Thoerem \ref{main thm 4}. We also provide \cref{Review of Margolis...}, which is devoted to review of Margolis' basic theorems on graded modules over $\mathcal{P}$-algebras.
\begin{comment} (THIS IS ABOUT A SECTION WHICH IS CURRENTLY COMMENTED-OUT.)
\item One consequence of Theorem \ref{main thm 1} is that higher distinguished local cohomology vanishes on rational modules. However, it is a bit too glib to express this by a slogan like ``Distinguished local cohomology detects the failure of a module to be a comodule,'' since there are also some modules with vanishing distinguished local cohomology which still fail to be rational. Bounded-below free graded modules over the Steenrod algebra are an example of this phenomenon. See the remarks preceding Theorem \ref{structure thm} for further discussion. 

Consequently, one wants a classification of the modules whose higher distinguished local cohomology vanishes. Such a classification is the purpose of \cref{Characterization of the modules...}, whose main result is Theorem \ref{decomposition of acyclics}: if $\Gamma$ is a Mitchell coalgebra over a countable field $k$, then a graded $\Gamma^*$-module $M$ has vanishing higher distinguished local cohomology if and only if $M$ admits graded $\Gamma^*$-submodules $M^{\prime\prime}\subseteq M^{\prime}\subseteq M$ such that $M^{\prime\prime}$ is bounded below and distinguished-torsion, such that $M^{\prime}/M^{\prime\prime}$ is bounded below and has the property that $\Ext^n_{\Gamma^*}(k,M^{\prime}/M^{\prime\prime})$ vanishes for all $n$, and such that $M/M^{\prime}$ is bounded above.
\end{comment}
\item While Theorem \ref{main thm 1} shows that $R^n\prod_{i\in I}^{\Gamma} M_i$ agrees with the distinguished local cohomology group $H^n_{\dist}(\prod_{i\in I}\iota M_i)$, one would like to have a similar theorem for calculating more general derived limits, not just derived products, in comodule categories. \Cref{Derived functors of seq...} addresses that problem, at least for sequential limits. Theorem \ref{derived sequential lim thm} establishes that, when $\Gamma$ is a finite-type coalgebra over a field such that $\Gamma^*$ is connected and such that uniformly bounded-below products of free graded $\Gamma^*$-modules are graded-projective, the derived functors of sequential limit $R^*\lim^{\Gamma}_iM_i$ in the category of graded $\Gamma$-comodules agree with the distinguished local cohomology groups $H^*_{\dist(\Gamma)}(\lim_i \iota M_i)$ as long as the sequence of graded $\Gamma^*$-modules $\dots \rightarrow \iota M_2 \rightarrow \iota M_1 \rightarrow \iota M_0$ has vanishing $\lim^1$, and as long as each comodule $M_i$ is bounded-above. It is an important and convenient point that this $\lim^1$-vanishing condition is checked in the {\em module} category, not in the {\em comodule} category, so it is relatively easy to check: one can simply verify that the Mittag-Leffler condition holds, for example.

Consequently, we have Corollary \ref{steenrod alg seq cor}: given a sequence $\dots \rightarrow M_2 \rightarrow M_1 \rightarrow M_0$ of bounded-above graded comodules over the dual Steenrod algebra at any prime, if the sequence of adjoint modules $\dots \rightarrow \iota M_2 \rightarrow \iota M_1 \rightarrow \iota M_0$ has vanishing $\lim^1$, then the derived limit $R^*\lim^{\Gamma}_i M_i$ in the category of comodules over the dual Steenrod algebra agrees with the graded local cohomology $\underset{j\rightarrow\infty}{\colim}\Ext^*_{\Gamma^*}\left(\Gamma^*/I_j,\lim_i \iota M_i\right)$ of the Steenrod algebra $\Gamma^*$.
%\item \Cref{...generalization of Megibben...} is devoted to proving a graded generalization of a 1982 theorem of C. Megibben. Megibben's theorem, from \cite{MR633266}, establishes that every countable injective module $M$ has the property that every direct sum of copies of $M$ is also injective. Over a Noetherian ring, this is a trivial consequence of the Bass-Papp theorem, but over non-Noetherian rings, Megibben's theorem has some ``teeth.'' The failure of the Steenrod algebra (and the duals of various other coalgebras which are treated by Theorem \ref{main thm 1}) to be Noetherian means the Bass-Papp theorem is not sufficient to establish that direct sums of copies of the Steenrod algebra are injective. As a consequence, at one point in the proof of Theorem \ref{main thm 1}, we need a version of Megibben's theorem which applies to graded-injective modules over graded rings. That graded version of Megibben's theorem is our Theorem \ref{graded megibbens thm}. See the remarks at the beginning of \cref{...generalization of Megibben...} for why the original (ungraded) theorem of Megibben is not enough for our purposes. \Cref{...generalization of Megibben...} does not logically depend on anything earlier in the paper, so it does not constitute a logical problem for the proof of Theorem \ref{main thm 1} to use a result (namely, Theorem \ref{graded megibbens thm}) in \cref{...generalization of Megibben...}.
\item \Cref{Review of Margolis...} reviews some results of Margolis, from \cite{MR738973}, on module theory over $\mathcal{P}$-algebras such as the Steenrod algebras. Those results of Margolis are used in several proofs in this paper, so it is useful to have an appendix dedicated to their review. There are no new results in \cref{Review of Margolis...}. 
\end{enumerate}


\begin{remark}
Many cases of the results of this paper can be interpreted as having stack-theoretic content. Suppose we are given a commutative Hopf algebra $\Gamma$ over a commutative ring $A$. If $\Gamma$ is smooth over $A$, then the category of $\Gamma$-comodules is equivalent to the category of quasicoherent $\mathcal{O}_{B\mathbb{G}}$-modules, where $\mathcal{O}_{B\mathbb{G}}$ is the structure sheaf of the fppf site of the Artin stack $B\mathbb{G}$ of $\mathbb{G}$-torsors. Here $\mathbb{G}$ is the group scheme represented by $\Spec\Gamma$. In light of this, the question ``For what $n$ does the $n$th right derived functor $R^n\prod_I^{\Gamma}: \Comod(\Gamma)^I\rightarrow\Comod(\Gamma)$ vanish?'' becomes the question ``For what $n$ does the $n$th right derived functor $R^n\prod_I: QC\Mod(B\mathbb{G})^I\rightarrow QC\Mod(B\mathbb{G})$ vanish?'' %at least when $\Gamma$ is a commutative Hopf algebra smooth over $A$. 
This latter question was investigated for Deligne-Mumford stacks in \cite{hogadixu2009}, so the present paper could be seen as, in part, handling for certain Artin stacks the same problem that was investigated for Deligne-Mumford stacks in \cite{hogadixu2009}. But of course our motivations and main applications in this paper are really topological, rather than algebro-geometric. 

Artin stacks are more general than Deligne-Mumford stacks, so the results in this paper are not special cases of those in \cite{hogadixu2009}. Indeed, the results and the methods obtained in this paper are entirely different from the results and the methods involved in \cite{hogadixu2009}. 
\end{remark}

\begin{remark}
This paper examines the derived functors of products in categories of comodules over a coalgebr{\em a}, not a more general coalgebr{\em oid}, for example a Hopf algebroid. The Hovey-Sadofsky spectral sequence has a version for a generalized homology theory $E_*$ whose input is the derived functor of product in the category of graded $(E_*,E_*E)$-comodules, where now $(E_*,E_*E)$ is a Hopf algebroid and not typically a Hopf algebra (or coalgebra). So there is good topological motivation to try to prove Hopf algebroid analogues of the results in the present paper. A similar remark is also true with sequential limits in place of products throughout. %In particular, letting $E$ be the Morava $E$-theory spectrum of a height $1$ one-dimensional formal group over a finite field (for example, $p$-complete periodic complex $K$-theory is weakly equivalent to such a Morava $E$-theory spectrum), I imagine that an appropriate Hopf-algebroid-theoretic analogue of the final part of Theorem \ref{main thm 1} would establish that the derived functors of the product in the category of graded $(E_*,E_*E)$-comodules are isomorphic to appropriate local cohomology groups in a certain category of Iwasawa modules; ...
\end{remark}

\begin{conventions}\label{conventions}\leavevmode
\begin{itemize}
\item Unless otherwise specified, our modules will be left modules, and our comodules will be right comodules.
\item All gradings in this paper are $\mathbb{Z}$-gradings whenever not otherwise stated.
\item When speaking of a {\em graded} module $M$, if we say that $M$ is injective, we mean that $M$ is injective in the category of {\em graded} modules. This is weaker than saying that $M$ is injective in the category of {\em ungraded} modules: see for example Remarks 3.3.11 in section I.3 of \cite{MR551625}. 
\item 
Throughout, when we have a coalgebra over a commutative ring, we use the standard notations from the theory of Hopf algebroids (from the influential first appendix of \cite{MR860042}, for example): we write $A$ for the commutative ring, and we write $\Gamma$ for the coalgebra. 
\item When we have a ring $R$ and a left ideal $I$ of $R$, we will write $RI$ for the ideal $I$ regarded as a left $R$-module. Consequently $R/RI$ denotes the left $R$-module given by the cokernel of the inclusion $RI \hookrightarrow R$. While writing $R/RI$ rather than $R/I$ may seem annoying to some readers, it is a common convention, e.g. as in \cite{lam2013first}, and this convention avoids some notational ambiguities: for example, when $A$ is the Steenrod algebra, if we were to write $\hom_A(A/(\Sq^1,\Sq^2,\dots),M)$, it could leave the reader uncertain whether $A/(\Sq^1,\Sq^2,\dots)$ means $A$ modulo the two-sided ideal generated by $\{\Sq^1,\Sq^2,\dots\}$, or the much larger quotient module given by $A$ modulo only the {\em left} ideal generated $\{\Sq^1,\Sq^2,\dots\}$. It is convenient to have the notation $A/A(\Sq^1,\Sq^2,\dots)$ reserved for the latter meaning.
\item At many places in this paper, we give arguments involving annihilators over noncommutative rings, such as Steenrod algebras. These arguments require a bit of care, because some of the nice behavior of annihilators in commutative algebra does not carry over to the noncommutative setting. 
We will follow the notational conventions for annihilators from \cite{lam2013first}: if $R$ is a ring and $M$ is a (left) $R$-module, then $\ann(M)$ denotes the set $\{ r\in R: rm=0\ \forall m\in M\}$, which is a {\em two-sided} ideal in $R$. If $S$ is a subset of $M$, we write $\ann_{\ell}(S)$ for the subset $\{ r\in R: rs=0\ \forall s\in S\}$, which is in general only a {\em left} ideal of $R$. Of course this distinction is only necessary when $R$ is noncommutative. 

In particular, if $M$ is a cyclic left $R$-module, then $M$ is isomorphic to $R/R\ann_{\ell}(g)$, where $g$ is a generator for $M$. We have an equality $R/R\ann(M) = R/R\ann_{\ell}(M)$, but $R/R\ann(M)$ and $R/R\ann_{\ell}(M)$ are {\em not} necessarily isomorphic to $R/R\ann_{\ell}(g)$, hence not necessarily isomorphic to $M$. The discussion in section 2.4 of \cite{lam2013first} is an excellent textbook treatment of this issue. %(I apologize for belaboring this elementary point, but I suspect that readers of this paper are likely to have thought far more about annihilators over commutative rings, and so this distinction that occurs in the noncommutative case---which plays an important role in the proof of Proposition \ref{filteredness prop}, for example---is liable to be missed.)
\item Given a graded ring $R$ and graded $R$-modules $M$ and $N$, we write $\hom_R(M,N)$ for the degree-preserving $R$-linear morphisms $M\rightarrow N$, and we write $\underline{\hom}_R(M,N)$ for the graded abelian group whose degree $n$ summand\footnote{Some references, e.g. \cite{brunerprimer}, use the opposite grading on $\underline{\hom}_R$---hence the need to give our grading conventions explicitly. The argument for our choice of gradings is that it is the unique one so that $\hom_R(L,\underline{\hom}_R(M,N)) \cong \hom_R(L\otimes_R M,N)$.} is $\hom_R(\Sigma^n M,N)$.
\item We always write $\Gamma$ for a coalgebra, $A$ for its ground ring, and $\Gamma^*$ for the $A$-linear dual algebra of $\Gamma$. This includes our discussions of the Steenrod algebra as the motivating example for the theory: we will write $\Gamma$ for the {\em dual} Steenrod algebra, and consequently $\Gamma^*$ for the Steenrod algebra itself.

When we have a finite-type graded coalgebra $\Gamma$, we {\em always} set up the grading so that its dual algebra $\Gamma^*$ is connected\footnote{``Connected'' is a standard term, but to avoid any possible misunderstanding, we include its definition here: to say that the finite-type algebra $\Gamma^*$ is connected is to say that, for each $n$, the degree $n$ summand of $\Gamma^*$ (equivalently, $\Gamma$) is a finite dimensional $A$-vector space, that $\Gamma^*$ is one-dimensional in degree $0$, and that $\Gamma^*$ is trivial in negative degrees.}. Whenever possible, we avoid direct manipulation of the grading on $\Gamma$ itself, in order to completely avoid any ambiguity or confusion about the effect of dualization on the gradings. 

One consequence is that $\iota(\Gamma)$ is concentrated in non{\em positive} degrees, where $\iota$ is the {\em covariant} embedding of graded $\Gamma$-comodules into graded $\Gamma^*$-modules. This fact becomes very useful, starting in Theorem \ref{main thm 1}.
\item We write $\prod$ for a product of graded modules, and $\prod^{\Gamma}$ for the product in the category of graded $\Gamma$-comodules.
\end{itemize}
\end{conventions}


\section{Local cohomology functors associated to sets of ideals.}
\label{Local cohomology functors associated...}
This section consists of elementary notions about generalized local cohomology functors associated to sets of ideals in some (not necessarily commutative) ring. We do not claim originality for the ideas in this section, but we do not know anywhere where this sequence of ideas already appears in the literature.

\begin{definition}\label{def of ideal sets}
Let $R$ be a graded ring. 
\begin{itemize}
\item By an {\em ideal set in $R$} we mean a set $S$ of homogeneous proper left ideals of $R$. %which is filtered under inclusion, that is, if $I,J\in S$, then there exists an element of $S$ contained in both $I$ and $J$.
\item We will say that an ideal set $S$ in $R$ is {\em connected} if $S$ is connected as a partially ordered set under inclusion, i.e., if $I,J\in S$, then there exists a finite sequence $I = I_0, J_0, I_1, J_1, \dots ,I_{n-1},J_{n-1},I_n,J_n = J$ of elements of $S$ such that $I_h \subseteq J_h$ and $I_{h+1}\subseteq J_h$ for all $h$.
\item We will say that an ideal set $S$ in $R$ is {\em filtered} %if $S$ is filtered as a partially-ordered set under inclusion, i.e., if 
if, for each $I,J\in S$, there exists an element of $S$ contained in both $I$ and $J$.
\item If $S,S^{\prime}$ are ideal sets in $R$, we write $S\leq S^{\prime}$ if, for every element $I$ of $S$, there exists some element $J\in S^{\prime}$ such that $I\supseteq J$. (Intuitively, this says that $S\leq S^{\prime}$ iff $S^{\prime}$ is ``finer'' than $S$.) The relation $\leq$ on ideal sets is transitive and reflexive, hence the collection of ideal sets in $R$ is a preorder. We write $\idealsets(R)$ for this preorder.
\item We say that ideal sets $S$ and $S^{\prime}$ are {\em equivalent} if $S\leq S^{\prime}$ and $S^{\prime}\leq S$.
\item Given an ideal set $S$ in $R$, let $\overline{S}$ denote the set of all intersections\footnote{To be clear: an element of $\overline{S}$ is a homogeneous left ideal in $R$ which is the intersection of finitely many homogeneous left ideals which are members of $S$.} of finite sets of members of $S$. We call $\overline{S}$ the {\em filtered closure} of $S$. The filtered closure of $S$ is a filtered ideal set in $R$, and $S\leq \overline{S}$. Furthermore, $\overline{S}$ is minimal (up to equivalence) with that property. That is, if $T$ is any filtered ideal set in $R$ such that $S\leq T$, then $\overline{S}\leq T$.
\item Given an ideal set $S$, we have two associated degree zero local cohomology functors: first, we have the functor
\begin{align*}
 h^0_S: \gr\Mod(R) & \rightarrow \gr\Ab \\
        h^0_S(M) &= \underset{I\in S}{\colim}\ \underline{\hom}_R(R/RI,M). \end{align*}
In particular, if $S$ is filtered, then $h^0_S(M)$ is the graded subgroup of $M$ generated by all homogeneous elements which are $I$-torsion for some element $I$ of $S$. 

We write $h^n_S$ for the $n$th right derived functor $R^nh^0_S$ of $h^0_S$.

Meanwhile, when $S$ is filtered, we also have the functor
\begin{align*}
 H^0_S: \gr\Mod(R) & \rightarrow \gr\Mod(R) \end{align*}
given by letting $H^0_S(M)$ be the graded left $R$-submodule of $M$ generated by the subgroup $h^0_S(M)$ of $M$.
We write $H^n_S$ for the $n$th right derived functor $R^nH^0_S$ of $H^0_S$.
\end{itemize}
\end{definition}
Note that, if $S\leq S^{\prime}$ and $M$ is a graded $R$-module, then every homogeneous element of $M$ which is $I$-torsion for some $I\in S$ is also $J$-torsion for some $J\in S^{\prime}$. If $S^{\prime}$ is also assumed to be filtered, then we have a well-defined map from $\underline{\hom}_R(R/RI,M)$ to $\underset{J\in S^{\prime}}{\colim}\ \underline{\hom}_R(R/RJ,M)$. Consequently, if $S\leq S^{\prime}$ and $S^{\prime}$ is filtered, we get a canonical choice of natural transformation $h^0_S \rightarrow h^0_{S^{\prime}}$.

\begin{remark} \label{remark on ideal sets}
Here are some examples, non-examples, and observations to illustrate the various notions defined in Definition \ref{def of ideal sets}.
\begin{description}
\item[Why connectedness matters]
If an ideal set $S$ is not connected, we can still make the definition $h^0_S(M) \coloneqq \underset{I\in S}{\colim}\ \underline{\hom}_R(R/RI,M)$, but the resulting abelian group $h^0_S(M)$ can fail to be a subgroup of $M$. For example, let $k$ be a field, and let $R = k[x,y]$. Let $S$ be the ideal set $\left\{ (x),(y)\right\}$. Then $h^0_S(M)$ is the direct sum of the $x$-torsion subgroup of $M$ and the $y$-torsion subgroup of $M$. This is a subgroup of $M\oplus M$, but it is not a subgroup of $M$ itself in any natural way.

The filtered closure $\overline{S}$ of $S$ is $\{ (x), (y), (xy)\}$, so $h^0_{\overline{S}}(M)$ is the $(xy)$-torsion subgroup of $M$, i.e., the set of elements $m\in M$ such that $xym=0$. This {\em is} a subgroup of $M$, not merely a subgroup of $M\oplus M$. What we saw in this example is one case of a general phenomenon: a filtered partially-ordered set is automatically connected, so $h^0_{\overline{S}}(M)$ is a subgroup of $M$ for any ideal set $S$. 

This is also an example of an ideal set $S$ such that $h^0_S$ is not isomorphic to $h^0_{\overline{S}}$. In general, there is no reason to expect $h^0_S$ to coincide with $h^0_{\overline{S}}$.
\item[When $h^0_S$ and $H^0_S$ differ]
Suppose $S$ is filtered. 
When $R$ is commutative, then $\underline{\hom}_R(R/RI,M)$ is in fact an $R$-submodule of $M$, and not only a subgroup. Put another way, when $R$ is commutative, the $I$-torsion in a given $R$-module is a submodule and not merely a subgroup. Consequently the natural transformation $h^0_S\rightarrow H^0_S$ is an isomorphism for all filtered ideal sets $S$, when $R$ is commutative.

On the other hand, when $R$ is noncommutative, $h^0_S$ may differ from $H^0_S$. Here is an explicit example where $h^0_{S}(M)$ fails to be an $R$-submodule of $M$, and consequently $h^0_{S}(M)$ fails to agree with $H^0_{S}(M)$. Let $R$ be the subalgebra $A(1)$ of the mod $2$ Steenrod algebra generated by $\Sq^1$ and $\Sq^2$. Let $S$ be the left ideal in $A(1)$ generated by $\Sq^1$. Then $h^0_{S}(M)$ is simply the graded subgroup of $M$ consisting of the elements of $M$ annihilated by $\Sq^1$. 
In the case $M = A(1)$, we have $\Sq^1\in h^0_{S}(A(1))$, since $\Sq^1\Sq^1 = 0$ in $A(1)$. However, $\Sq^2\Sq^1 \notin h^0_{S}(A(1))$, since $\Sq^1\Sq^2\Sq^1 \neq 0$ in $A(1)$. So $h^0_{S}(A(1))$ is not closed under left $A(1)$-multiplication in $A(1)$, i.e., $h^0_{S}(A(1))$ is not a left $A(1)$-submodule of $A(1)$. Hence $h^0_{S}(A(1))$ fails to coincide with $H^0_{S}(A(1))$: the latter contains $\Sq^2\Sq^1$, while the former does not.
\item[Why equivalence matters] 
If filtered ideal sets $S,S^{\prime}$ in $R$ satisfy $S\leq S^{\prime}$, then the natural monomorphism $h^0_S(M)\hookrightarrow M$ factors uniquely through the natural monomorphism $h^0_{S^{\prime}}(M)\hookrightarrow M$. Consequently we get natural transformations $h^n_S\rightarrow h^n_{S^{\prime}}$ and $H^n_S\rightarrow H^n_{S^{\prime}}$ for each $n$.

If the filtered ideal sets $S$ and $S^{\prime}$ are equivalent, then we have $h^0_S(M) = h^0_{S^{\prime}}(M)$ as subgroups of $M$, so we get natural isomorphisms $h^n_S \cong h^n_{S^{\prime}}$ and $H^n_S \cong H^n_{S^{\prime}}$ for all $n$ as well.
\item[Left exactness of $h^0_S$ and $H^0_S$]
We have much better tools for understanding and calculating the right derived functors $h^*_S$ of $h^0_S$ when we know that $h^0_S$ is left exact. If the ideal set $S$ is filtered, then colimits over $S$ are exact, so
$h^0_S(-) = \underset{I\in S}{\colim}\ \underline{\hom}_R(R/RI,-)$ is indeed left exact. 

It is much less obvious that $H^0_S$ is left exact. This is a topic we take up later, using tools from torsion theory, in Corollary \ref{H0 is left exact}.
\end{description}
\end{remark}

\begin{example} \label{examples of ideal sets} 
{\bf (Important examples of ideal sets.)}
\begin{itemize}
\item The famous case of $H^0_{S}(M)$ is the case where $R$ is commutative and concentrated in degree zero, $I$ is an ideal of $R$, and $S$ is the set $S = \{ I, I^2, I^3, \dots\}$ of powers of $I$. In that case, $H^0_{S}(M)$ is simply the classical local cohomology $H^0_I(M)$ of $M$ at the ideal $I$, in the sense of \cite{MR0222093} and \cite{MR3014449}, among many other references. 

The reason for introducing $h^0_{S}$ and $H^0_{S}$ in Definition \ref{def of ideal sets} is that, when generalizing local cohomology from the classical setting to a setting where $R$ is not necessarily commutative, $h^0_{S}$ and $H^0_{S}$ can differ. This may lead to confusion: if, for example, $R$ is noncommutative and $S$ is the set $\{ I,I^2,I^3, \dots\}$ of powers of some left ideal $I$ of $R$, then $h^0_{S}(M) = \underset{n\rightarrow\infty}{\colim}\ \underline{\hom}_R(R/RI^n,M)$ is generally only a subgroup of $M$, not necessarily an $R$-submodule of $M$, so some familiar arguments from classical local cohomology no longer work. For example, it no longer even makes sense to ask whether $h^0_{S}:\gr\Mod(R)\rightarrow\gr\Ab$ is idempotent. On the other hand, it {\em does} make sense to ask whether $H^0_{S}:\gr\Mod(R)\rightarrow\gr\Mod(R)$ is idempotent. The functor $H^0_S$ has some very desirable properties (see Theorem \ref{theta-rationality and torsion}, for example), but the derived functors $H^n_S$ (unlike $h^n_S$) are not generally isomorphic to colimits of $\Ext$-groups, so it can be much less clear how to actually calculate them. 
\item The following example is trivial, but helps to build intuition for the preorder $\idealsets(R)$. 
The preorder $\idealsets(R)$ has a maximal element, which can be taken to be the ideal set $\{ (0)\}$ consisting of simply the zero ideal of $R$. This ideal set is equivalent to any other ideal set in $R$ containing the zero ideal, for example the ideal set consisting of all proper homogeneous left ideals of $R$. For all graded $R$-modules $M$, we have $h^0_{\{(0)\}}(M) = M = H^0_{\{(0)\}}(M)$, so $h^n_{\{(0)\}}(M) = 0 = H^n_{\{(0)\}}(M)$ for all $n>0$. 

The point is that the maximal element in $\idealsets(R)$ is the element whose associated local cohomology theory has $h^0(M)$ consisting of {\em all of $M$}. The partial order $\leq$ on $\idealsets(R)$ has the property that {\em smaller filtered elements $S$ of $\idealsets(R)$ make $h^0_S(M)$ a smaller subgroup of $M$.}
\item Let $R$ be a graded ring, and let $\Theta$ be a graded left $R$-module. Let $\homog(\Theta)$ be the set of homogeneous elements of $\Theta$, and for each $m\in \homog(\Theta)$, let $\ann_{\ell}(m)$ be the left annihilator of $m$. Let $\dist(\Theta)$ be the ideal set $\{ \ann_{\ell}(m) : 0\neq m \in \homog(\Theta)\}$ of $R$. We refer to the members of $\dist(\Theta)$ as the {\em strongly $\Theta$-distinguished ideals of $R$}, and we refer to the members of the filtered closure $\overline{\dist(\Theta)}$ as the {\em $\Theta$-distinguished ideals of $R$.}
\item Let $R$ be an $\mathbb{N}$-graded ring, and let $\grad$ be the ideal set $\{ I_1, I_2, I_3, \dots\}$ in $R$, where $I_n$ is the left ideal (equivalently, two-sided ideal) in $R$ generated by all homogeneous elements of degree $\geq n$. Note that $\grad$ is filtered. The ideal set $\grad$ will play a pivotal role in the proof of the most important result in this paper, Theorem \ref{main thm on mitchell coalgebras}.
\end{itemize}
\end{example}

\section{Comodules as a (pre)torsion class in modules.}
\label{Comodules as a pretorsion...}

This section covers preliminary notions on the covariant embedding of $\Gamma$-comodules into $\Gamma^*$-modules, as well as preliminary notions on (pre)torsion theories. 

\subsection{Review of the embedding of the category of $\Gamma$-comodules into the category of $\Gamma^*$-modules.}

Let $A$ be a commutative ring, and let $\Gamma$ be a graded $A$-coalgebra which is flat as an $A$-module. Let $\Gamma^*$ denote the $A$-linear dual graded algebra $\underline{\hom}_A(\Gamma,A)$ of $\Gamma$. %, and let $\Gamma^{\vee}$ denote $\Gamma^*$ with the gradings reversed, i.e., the degree $n$ summand $(\Gamma^{\vee})^n$ of $\Gamma^{\vee}$ is the degree $-n$ summand $(\Gamma^*)^n$ of $\Gamma^*$.
 We have a well-known functor
\begin{equation}
 \label{comodule inclusion functor}
 \iota:\gr\Comod(\Gamma) \rightarrow \gr\Mod(\Gamma^*)
\end{equation}
given by sending a graded $\Gamma$-comodule $M$ to the graded $A$-module $M$, equipped with the ``adjoint'' $\Gamma^*$-action. The adjoint left $\Gamma^*$-action\footnote{To avoid possible confusion, we remark that when $\Gamma$ is not only a coalgebra but a finite-type Hopf algebra, then $\Gamma\cong \Gamma^{**}$ is also a graded $\Gamma^*$-module via the {\em contragredient} action, which differs from the adjoint action by an application of the antipode of $\Gamma^*$.} is the action $\Gamma^*\otimes_A M \rightarrow M$ given by sending $f\otimes m$ to the image of $m$ under the composite map 
\[ M \stackrel{\psi_M}{\longrightarrow} M\otimes_A \Gamma \stackrel{M\otimes f}{\longrightarrow} M\otimes_A A \stackrel{\cong}{\longrightarrow} M.\]
The graded $\Gamma^*$-modules in the essential image of the functor $\iota$ are called {\em rational} modules in the literature, e.g. in the textbook \cite{MR2012570}. The use of the term ``rational'' here is well-established, so we use that term in this paper\footnote{There is some risk of confusion here: topologists, like the author of this paper, are used to the term ``rational'' being reserved for abelian groups which are vector spaces over $\mathbb{Q}$, or spectra which are modules over $H\mathbb{Q}$, or other small variations on this theme. That use of the term ``rational'' is simply unlike the use of the term ``rational'' in the coalgebra literature and in this paper. %, that is, a module which is the contragredient dual of a $\Gamma$-comodule. 
Rational (in the sense of coalgebra) modules over the Steenrod algebra are one of the main objects of study in this paper, but they do not arise as the homology or cohomology of rational (in the sense of topology) spaces or spectra, except in trivial cases.}. 

The following theorem summarizes some established properties of the functor \eqref{comodule inclusion functor}; see 4.1, 4.3, 4.7, 7.1, and 20.1 of \cite{MR2012570} for a comprehensive treatment.
\begin{theorem}\label{review thm} The following claims are each true:
\begin{enumerate}
\item The functor $\iota$ is faithful and its image lies in the full subcategory of $\gr\Mod(\Gamma^*)$ consisting of all graded $\Gamma^*$-modules $M$ such that $M$ is a graded submodule of a graded quotient module of a coproduct of copies of suspensions of $\iota\Gamma$. This full subcategory of $\gr\Mod(\Gamma^*)$ is denoted $\sigma[{}_{\Gamma^*}\Gamma]$.
\item The resulting functor $\gr\Comod(\Gamma)\rightarrow \sigma[{}_{\Gamma^*}\Gamma]$ is full if and only if $\Gamma$ is locally projective as an $A$-module. Consequently, when $\Gamma$ is locally projective as an $A$-module, we may regard $\gr\Comod(\Gamma)$ as (via the functor $\iota$) a full subcategory of $\gr\Mod(\Gamma^*)$. 
\item If $\Gamma$ is locally projective as an $A$-module, then the functor $\iota$ is full, faithful, and admits a right adjoint $\tr: \gr\Mod(\Gamma^*)\rightarrow \gr\Comod(\Gamma)$, called the ``rational functor'' or the ``trace functor,'' and which is given on a graded $\Gamma^*$-module $M$ by letting $\tr(M)$ be the graded $\Gamma^*$-submodule $M$ generated by the homogeneous elements $m$ such that $m$ is in the image of a graded $\Gamma^*$-module homomorphism from a rational graded $\Gamma^*$-module to $M$.
\item The functor $\iota: \gr\Comod(\Gamma) \rightarrow \gr\Mod(\Gamma^*)$ is an equivalence of categories if and only if $\Gamma$ is finitely generated and projective as an $A$-module. 
\item In particular, suppose that $\Gamma$ is projective as an $A$-module. Then the rational graded $\Gamma^*$-modules form a full abelian subcategory of the graded $\Gamma^*$-modules, closed under kernels, cokernels, and coproducts. Furthermore, the following are equivalent:
\begin{itemize}
\item Every graded $\Gamma^*$-module is rational.
\item $\Gamma$ is finitely generated as an $A$-module. 
\end{itemize}
\end{enumerate}
\end{theorem}
%We have the functor $\Rat: \gr\Mod(\Gamma^*) \rightarrow \gr\Comod(\Gamma)$ and the functor $\iota: \gr\Comod(\Gamma)\hookrightarrow \gr\Mod(\Gamma^*)$.
%While I do not know where it is explicitly stated in the literature, it is easy to see that the composite $\iota\circ \Rat$ is naturally isomorphic to the identity, while the composite $\Rat\circ\iota$ is idempotent.


In Proposition \ref{rationals closed under quots and subs}, we offer a proof of part of Theorem \ref{review thm}, in order to make it clear (as explained in Remark \ref{rationals closed under quots and subs 2}) that the proof generalizes from rational modules to $\Theta$-rational modules.
\begin{prop} \label{rationals closed under quots and subs}
Suppose that $\Gamma^*$ is projective as an $A$-module. Then the following claims are each true:
\begin{enumerate}
\item 
Every coproduct of rational graded $\Gamma^*$-modules is rational.
\item 
Every graded submodule of a rational graded $\Gamma^*$-module is rational.
\item
Every graded quotient of a rational graded $\Gamma^*$-module is rational.
\end{enumerate}
\end{prop}
\begin{proof}\leavevmode
\begin{itemize}
\item[(1) and (2)]
It follows immediately from the definition of $\sigma[{}_{\Gamma^*}\Gamma]$ that it is closed under coproducts and graded submodules. By the first part of Theorem \ref{review thm}, $\sigma[{}_{\Gamma^*}\Gamma]$ agrees with the category of rational graded $\Gamma^*$-modules.
\item[(3)]
Suppose that $Q$ is a graded quotient of a rational graded $\Gamma^*$-module $M$. Since $M$ is rational, it embeds into a quotient module $F^{\prime}$ of a coproduct $F$ of suspensions of copies of $\iota\Gamma$. We have the diagram of graded $\Gamma^*$-modules
\begin{equation}\label{diag 195949}\xymatrix{
 & M\ar@{->>}[ld] \ar@{^{(}->}[rd] & & F \ar@{->>}[ld] \\
 Q & & F^{\prime} &
}\end{equation}
and, taking the pushout of the subdiagram of \eqref{diag 195949} containing $M$ and $Q$ and $F^{\prime}$, we get a graded $\Gamma^*$-module into which $Q$ embeds, and which is also a quotient of $F$, since epimorphisms and monomorphisms are each stable under pushouts in a category of modules over a ring. Hence $Q$ is in $\sigma[{}_{\Gamma^*}\Gamma]$, hence $Q$ is rational.
\end{itemize}
\end{proof}

\begin{definition}
Let $R$ be a graded ring, and let $\Theta$ be a graded left $R$-module. We will call a graded left $R$-module $M$ {\em $\Theta$-rational} if $M$ is a graded submodule of a graded quotient module of a coproduct of copies of suspensions of $\Theta$.
\end{definition}
\begin{remark}\label{rationals closed under quots and subs 2}
Note that Proposition \ref{rationals closed under quots and subs} remains true, with the same proof, if we replace ``rational'' with ``$\Theta$-rational'' throughout.
\end{remark}


\subsection{Distinguished ideals and distinguished torsion.}
\label{Distinguished ideals and...}

Suppose that $\Gamma$ is projective over $A$. 
Then, as a consequence of Theorem \ref{review thm}, the graded $\Gamma$-comodules form a particularly nice subcategory of the graded $\Gamma^*$-modules. Here ``nice'' means, in particular, full and coreflective and abelian. It is not obvious, at a glance, how to tell if a given graded $\Gamma^*$-module actually lives in that nice subcategory. 
Consequently, one would like to have a purely module-theoretic characterization of that subcategory: that is, given a graded $\Gamma^*$-module $M$, we would like to be able to determine whether $M$ is rational by means of some criterion which refers only to the $\Gamma^*$-action on $M$. 
One of our tasks (completed in Theorem \ref{main thm 1}) in this section is to show that there is indeed such a criterion: it is the property of being {\em distinguished-torsion}, which we now define.

\begin{prop}\label{prop-def of distinguished}
Let $R$ be a graded ring, and let $\Theta$ be a graded left $R$-module. 
Let $I$ be a homogeneous left ideal of $R$. Then the following conditions are equivalent:
\begin{itemize}
%\item $\Gamma$, regarded as a graded $\Gamma^*$-module via $\iota$, contains a homogeneous element $\gamma$ annihilated by the right action of $I$. (NO!)
\item There exists an exact sequence of graded left $R$-modules
\[ 0 \rightarrow I \stackrel{f}{\longrightarrow} R \rightarrow \Sigma^n \Theta \]
for some integer $n$, where $f$ is the canonical inclusion of $I$ into $R$.
\item $\Theta$ contains a graded $R$-submodule which is isomorphic to a suspension of $R/RI$.
\item $I$ is a member of the ideal set $\dist(\Theta)$ defined in Example \ref{examples of ideal sets}.
\end{itemize}
%We write $H^0_{\Theta}$ for the functor $\gr\Mod(R)\rightarrow\gr\Mod(R)$ given by letting $H^0_{\Theta}(M)$ be the graded $R$-submodule of $M$ generated by all homogeneous elements whose annihilator is $\Theta$-distinguished.
%There is a natural monomorphism of graded abelian groups
%\begin{equation}\label{nat map 1} h^0_{\Theta}(M)\hookrightarrow M\end{equation} induced by the composite $\underline{\hom}_{R}\left( R/RI,M\right) \rightarrow \underline{\hom}_{R}\left( R,M\right) \stackrel{\cong}{\longrightarrow} M$, and its image lands in the $R$-submodule $H^0_{\Theta}(M)$ of $M$; in particular, the graded $R$-module of $M$ generated by the subgroup $h^0_{\Theta}(M)$ of $M$ is precisely $H^0_{\Theta}(M)$.
\end{prop}
\begin{proof}
Elementary.%, and the ungraded analogues are well-known, although not expressed in the language of ideal sets. See for example chapter 3 section 1.2 of \cite{MR594432}, or sections 7.2 and 7.3 of chapter 1 of \cite{MR2012570}.
%This is very elementary, but we provide the proof anyway. Given an element $\gamma\in \Gamma$ annihilated by $I$, the $\Gamma^*$-module homomorphism $\Gamma^*\rightarrow M$ sending $1$ to $\gamma$ factors through the projection $\Gamma^* \rightarrow \Gamma^*/I$, and the kernel of the resulting $\Gamma^*$-module homomorphism $\Gamma^*/I \rightarrow \Gamma$ is a sub-$\Gamma^*$-module of $\Gamma^*/I$, i.e., a left ideal $J$ of $\Gamma^*$ containing $I$. The resulting factor map $\Gamma^*/J\rightarrow \Gamma$ is injective. Conversely, if $\Gamma^*/J \rightarrow \Gamma$ is injective, then the image of $1$ in $\Gamma$ is $J$-torsion. (NO!)
\end{proof}
\begin{definition}\label{def of distinguished}
Let $R,\Theta$ be as in Proposition \ref{prop-def of distinguished}.
\begin{itemize}
\item
As in Example \ref{examples of ideal sets}, we call a left ideal $I$ of $R$ {\em strongly $\Theta$-distinguished} if it is homogeneous and satisfies the equivalent conditions of Proposition \ref{prop-def of distinguished}. Similarly, we call a homogeneous left ideal $I$ {\em $\Theta$-distinguished} if it contains the intersection of a finite set of strongly $\Theta$-distinguished left ideals. 
\item Let $\dist(\Theta)$ be the ideal set of $\Theta$-distinguished left ideals of $R$. As in Definition \ref{def of ideal sets}, we write $\overline{\dist{\Theta}}$ for the filtered closure of $\dist{\Theta}$. 
We say that a left $R$-module $M$ is {\em $\Theta$-distinguished-torsion} if the inclusion $H^0_{\overline{\dist(\Theta)}}(M) \hookrightarrow M$ is an isomorphism. 
\item
The most important case of these notions is when $R = \Gamma^*$, the dual $A$-algebra of an $A$-coalgebra $\Gamma$ which is projective as an $A$-module, and $\Theta = \iota\Gamma$. In that case we write {\em distinguished ideal}, {\em distinguished torsion}, $h^0_{\dist}$, and $H^0_{\dist}$ as shorthand for $\iota\Gamma$-distinguished ideal, $\iota\Gamma$-distinguished torsion, $h^0_{\overline{\dist(\iota\Gamma)}}$, and $H^0_{\overline{\dist(\iota\Gamma)}}$, respectively.
\end{itemize}
\end{definition}

%Suppose that $I,J$ are left ideals of $\Gamma^*$ such that $I \subseteq J$. It is not difficult to come up with examples where $I$ is strongly distinguished and $J$ is not, and also examples where $J$ is strongly distinguished and $I$ is not. So the property of being strongly distinguished is not generally inherited by sub-ideals or super-ideals, although being distinguished is clearly inherited by super-ideals.

% Given a left $\Gamma^*$-module $M$, let $\Cyc(M)$ denote the partially-ordered (by inclusion) set of cyclic left $\Gamma^*$-submodules of $M$. 
% Write $<x>$ for the submodule generated by an element $x$ of $M$.
% If $M = \Gamma$, then we have $sx = x'$ iff $(s\tensor x')\Delta(x)=1$.
% So $<x>$ contains $<x'>$ if $(s\tensor x')\Delta(x)=1$ for some $s$.
% So if $\Delta(x)$ has a summand with an x' on the right, then $<x>$ contains $<x'>$.
% If $\Delta(x)$ and $\Delta(y)$ contain terms $1\otimes x$ 
%  and $1\otimes y$ respectively, then
%  $\Delta(x+y)$ has a summand with an $x$ on the right and also a summand with
%   $y$ on the right, as long as $x,y$ are linearly independent over the ground ring $A$.
%  So $<x+y>$ contains both $<x>$ and $<y>$.
%  Isn't the sum of the cyclic modules $<x>$ and $<y>$ simply $<x+y>$?
%  <x> + <y> = { rx + sy }
%  <x+y>     = { rx + ry }.
%  Maybe not. But $<x> + <y>$ contains $<x+y>$,
%   so $<x>+<y> \cong \Gamma^*/(I\cap J)$ contains $<x+y>$.
%   where here $I = Ann(<x>)$ and $J = Ann(<y>)$.
%  So $Ann(<x+y>)$ contains $I\cap J$.
% Wait, this is silly: we should directly compare $\Ann(x)\cap \Ann(y)$ to $\Ann(x+y)$.
% We just showed that $\Ann(x)\cap \Ann(y) \subseteq \Ann(x+y)$.
%  Does the converse containment hold? Not necessarily!

\begin{theorem}\label{theta-rationality and torsion}
Let $R$ be a graded ring and let $\Theta$ be a graded left $R$-module. Let $M$ be a graded left $R$-module. Then the following are equivalent:
\begin{enumerate}
\item $M$ is $\Theta$-rational.
\item For every homogeneous $m\in M$, the annihilator of $m$ is a $\Theta$-distinguished left ideal of $R$.
\item $M$ is $\Theta$-distinguished torsion.
\end{enumerate}
\end{theorem}
\begin{proof}
It is easy to see from the definition of distinguished torsion that conditions 2 and 3 are equivalent, so the rest of this proof consists of showing that conditions 1 and 2 are equivalent.
\begin{itemize}
\item 
Suppose that $M$ is $\Theta$-rational, and that $m$ is a homogenous element of $M$.
Embed $M$ into a quotient $Q$ of a coproduct of copies of $\Theta$ via a map $f: M \rightarrow Q$.
Since $f$ is injective, we have $\ann_l(f(m)) = \ann_l(m)$.
Now choose a surjective left $R$-module map $\sigma: F \rightarrow Q$ where $F$ is a coproduct of
copies of $\Theta$, and choose a homogeneous element $m^{\prime}$ in $F$ such that 
$\sigma(m^{\prime}) = f(m)$.
Since coproducts in categories of modules are direct sums, the element $m^{\prime}\in F$ has only finitely many nonzero components in summands $\Theta$ of $F$. Write $m^{\prime}_1, \dots ,m^{\prime}_n$ for these homogeneous nonzero elements of $\Theta$. 
We now have the chain of equalities and containments
\begin{align*}
 \bigcap_{i=1}^n \ann_l(m^{\prime}_i) &= \ann_l(m^{\prime}) \\
  &\subseteq \ann_l(\sigma(m^{\prime})) \\
  &= \ann_l(f(m)) \\
  &= \ann_l(m),
\end{align*}
so $\ann_l(m)$ contains the intersection $\bigcap_{i=1}^n \ann_l(m^{\prime}_i)$ of the left annihilators of a finite set of homogeneous elements of $\Theta$, i.e., $\ann_l(m)$ is a $\Theta$-distinguished left ideal of $R$. This holds for every homogeneous $m\in M$, so $M$ is $\Theta$-distinguished torsion.
\item
Suppose conversely that the annihilator of every homogeneous element of $M$ is a $\Theta$-distinguished left ideal of $R$. Let $\homog(M)$ denote the set of homogeneous elements of $M$, and consider the graded $R$-module morphism \[ \epsilon_M: \coprod_{m\in \homog(M)} \Sigma^{\left| m\right|} R/R\ann_l(m) \rightarrow M\]
which sends the summand $\Sigma^{\left| m\right|} R/R\ann_l(m)$ corresponding to $m\in \homog(M)$ to $M$ via the map $\Sigma^{\left| m\right|} R/R\ann_l(m) \rightarrow M$ sending $1$ to $m$.
For each $m\in \homog(M)$, choose a finite set $\gamma_{m,1}, \dots ,\gamma_{m,n_m}$ of homogeneous elements of $\Theta$ such that $\ann_l(m)$ contains $\cap_{i=1}^{n_m}\ann_l(\gamma_{m,i})$.
Then $R/R\left( \cap_{i=1}^{n_m}\ann_l(\gamma_{m,i})\right)$ surjects on to $R/R\ann_l(m)$.
We also have the graded $R$-module monomorphism 
\begin{equation}\label{map 30499949ff} R/R\left(\cap_{i=1}^{n_m}\ann_l(\gamma_{m,i}) \right)
  \rightarrow \coprod_{i=1}^{n_m} R/R\ann_l(\gamma_{m,i}) \end{equation}
which sends $x\in R/R\left(\cap_{i=1}^{n_m}\ann_l(\gamma_{m,i})\right)$ to the 
element 
\[ \left( x\mod \ann_l(\gamma_{m,1}),\ \ x\mod \ann_l(\gamma_{m,2}), \dots ,\ \ x\mod \ann_l(\gamma_{m,n_m}) \right)\]
of $\coprod_{i=1}^{n_m} R/R\ann_l(\gamma_{m,i})$. Since each $\ann_l(\gamma_{m,i})$ is strongly $\Theta$-distinguished, each $R/R\ann_l(\gamma_{m,i})$ is $\Theta$-rational. Since \eqref{map 30499949ff} is monic, Remark \ref{rationals closed under quots and subs 2} gives us that $R/R\left(\cap_{i=1}^{n_m}\ann_l(\gamma_{m,i})\right)$ is also $\Theta$-rational. Since $R/R\ann_l(m)$ is a quotient of $R/R\left(\cap_{i=1}^{n_m}\ann_l(\gamma_{m,i})\right)$, Remark \ref{rationals closed under quots and subs 2} gives us the $\Theta$-rationality of $R/R\ann_l(m)$, and it also gives us that $\coprod_{m\in \homog(M)} \Sigma^{\left| m\right|} R/R\ann_l(m)$ is $\Theta$-rational. Finally, one more application of Remark \ref{rationals closed under quots and subs 2} gives us that $M$ is $\Theta$-rational, since $\epsilon_M$ is surjective.
\end{itemize}
\end{proof}

\begin{corollary}\label{rational iff dist-torsion}
Let $\Gamma$ be a coalgebra which is projective over a commutative ring $A$, and let $M$ be a graded left $\Gamma^*$-module. Then $M$ is rational if and only if $M$ is distinguished-torsion.
\end{corollary}

\begin{comment} (REDUNDANT, SINCE SUMMANDS ARE A CASE OF SUBMODULES)
Here is another, less important, corollary of Theorem \ref{theta-rationality and torsion}. This corollary is not new, and can also be proven in more elementary ways; we only bother to state it here because it will be used in the proof of Theorem \ref{main thm 1}.
\begin{corollary}\label{rationals closed under summands}
Let $R$ be a graded ring and let $\Theta$ be a graded left $R$-module. Then every summand of a $\Theta$-rational graded $R$-module is $\Theta$-rational.
\end{corollary}
\begin{proof}
Let $M,N$ be graded left $R$-modules, let $N$ be $\Theta$-rational, and let $f: M\rightarrow N$ be a split monomorphism with splitting map $g: N \rightarrow M$.
We have the commutative diagram
\[\xymatrix{
 H^0_{\overline{\dist(\Theta)}}(M) \ar[r]^{H^0_{\overline{\dist(\Theta)}}(f)} \ar@{^{(}->}[d] & H^0_{\overline{\dist(\Theta)}}(N) \ar[r]^{H^0_{\overline{\dist(\Theta)}}(g)}\ar[d]^{\cong} & H^0_{\overline{\dist(\Theta)}}(M) \ar@{^{(}->}[d] \\
 M \ar[r]^f & N \ar[r]^g & M .
 }\]
Since $g$ is epic, so is the composite $H^0_{\overline{\dist(\Theta)}}(N)\rightarrow M$, and consequently so is the map $H^0_{\overline{\dist(\Theta)}}(M) \hookrightarrow M$, which is also automatically monic. So $M$ is $\Theta$-distinguished torsion, so $M$ is $\Theta$-rational by Theorem \ref{theta-rationality and torsion}.
\end{proof}
\end{comment}

\begin{comment} (THIS IS TRUE, BUT TOO OBVIOUS TO STATE.)
\begin{prop}\label{dist gamma is filtered}
Suppose we are given a graded ring $R$, a graded left $R$-module $\Theta$, and $\Theta$-distinguished left ideals $I,J$ of $R$. Then the %following claims are both true:
%\begin{enumerate}\item The 
intersection $I\cap J$ is $\Theta$-distinguished.
%\item The product left ideal $IJ$ is $\Theta$-distinguished. (THIS LAST CLAIM IS COMMENTED OUT SINCE IT'S PERHAPS NOT TRUE?)\end{enumerate}
\end{prop}
\begin{proof}
Let $x_1, \dots ,x_m$ and $y_1, \dots ,y_n$ be homogeneous elements of $\Theta$ such that $I$ contains $\cap_{i=1}^m \ann_l(x_i)$ and $J$ contains $\cap_{j=1}^n \ann_l(y_j)$. Then $I\cap J$ contains $\left(\cap_{i=1}^m \ann_l(x_i)\right)\cap\left(\cap_{j=1}^n \ann_l(y_j)\right)$.
\end{proof}
\end{comment}


\begin{remark}\label{remark on h0 and H0}
Corollary \ref{rational iff dist-torsion} establishes the fundamental importance of $H^*_{\dist}$: the category of graded $\Gamma$-comodules, sitting inside the category of graded $\Gamma^*$-modules, consists precisely of those $\Gamma^*$-modules $M$ such that $H^0_{\dist}(M) \rightarrow M$ is an isomorphism. On the other hand, it is $h^*_{\dist}$, not $H^*_{\dist}$, which is defined in terms of a colimit of $\Ext$ groups. It is not clear how to calculate $H^*_{\dist}$ except in cases where it coincides with $h^*_{\dist}$. Consequently it is a matter of some importance to know, for a given coalgebra $\Gamma$, whether $h^*_{\dist}$ coincides with $H^*_{\dist}$. We have $h^*_{\dist} = H^*_{\dist}$ for all co-commutative coalgebras $\Gamma$, by the same argument as given in Remark \ref{remark on ideal sets}. But the motivating example for this paper is case where $\Gamma$ is the mod $p$ dual Steenrod algebra, which is not co-commutative for any prime $p$!
 
By a significantly less trivial argument, we prove in Theorem \ref{main thm on mitchell coalgebras} that a certain family of graded coalgebras, the finite-type {\em Mitchell coalgebras}, also have the property that $h^*_{\dist} = H^*_{\dist}$. The dual Steenrod algebras are finite-type Mitchell coalgebras, so we get $h^*_{\dist} = H^*_{\dist}$ in the case of the dual Steenrod algebras.
\end{remark}


\subsection{Review of (pre)torsion theories, (pre)torsion classes, and stability.}

For any graded ring $R$ and any graded left $R$-module $\Theta$, it is easy to use standard ideas to see that the functor $h^0_{\overline{\dist(\Theta)}}: \gr\Mod(R) \rightarrow\gr\Ab$ is left exact: since $\overline{\dist(\Theta)}$ is filtered, $h^0_{\overline{\dist(\Theta)}} = \underset{I\in \overline{\dist(\Theta)}}{\colim}\ \underline{\hom}_R(R/RI,-)$ is a composite of left-exact functors, namely $\underline{\hom}_R(R/RI,-)$ and $\underset{I\in\overline{\dist(\Theta)}}{\colim}$. We hope we do not try the reader's patience by pointing out again that it is much less obvious that $H^0_{\overline{\dist(\Theta)}}$ is left exact: recall that $H^0_{\overline{\dist(\Theta)}}(M)$ is the $R$-submodule of $M$ generated by the subgroup $h^0_{\overline{\dist(\Theta)}}(M)$ of $M$. This is not a very commonplace way to construct a functor, and so it is not immediately obvious what kinds of ideas might allow us to see that $H^0_{\overline{\dist(\Theta)}}$ is left exact. 

It turns out that the relevant ideas are those from {\em torsion theory}: it is a standard result (given below in Theorem \ref{stenstrom thm}) that the preradical associated to a hereditary pretorsion class is left exact, and in Proposition \ref{rational modules are hereditary pretorsion} we prove that the $\Theta$-rational modules are a hereditary pretorsion class, whose associated preradical---namely, $H^0_{\overline{\dist(\Theta)}}$---is consequently left exact. 

In this subsection, we give a ``crash course'' in the basic ideas from torsion theory that are used in our proof that $H^0_{\overline{\dist(\Theta)}}$ is left exact. Introductory accounts of torsion theories include sections 1.12 and 1.13 of \cite{MR1313497}, the entirety of \cite{MR880019}, \cite{MR3565424}, chapter VI of \cite{MR0389953}, and the original paper that introduced torsion theories, \cite{MR191935}. The second of those references restricts attention to the abelian category of left $R$-modules for a ring $R$, while the third deals more generally with Grothendieck categories, including graded $R$-modules, which is the desired level of generality for the applications in this paper. The fifth reference, \cite{MR191935}, assumes that the ambient abelian category is well-generated (i.e., each object has a set, rather than a proper class, of subobjects), which is also satisfied in all applications in this paper. All five are excellent references. 

Here are the relevant basics. %, following Stenstr\"{o}m's treatment in chapter VI of \cite{MR191935}:
The following definition appears in section VI.1 of \cite{MR0389953}:
\begin{definition}
Let $\mathcal{C}$ be a complete, co-complete abelian category. 
\begin{itemize}
\item
A {\em preradical on $\mathcal{C}$} is a functor $r: \mathcal{C} \rightarrow \mathcal{C}$ equipped with a natural transformation $\eta: r \rightarrow \id_{\mathcal{C}}$ such that $\eta X: rX \rightarrow X$ is a monomorphism for all objects $X$ of $\mathcal{C}$.
\item
A preradical $r$ is called a {\em radical} if $r(X/rX)$ vanishes for all objects $X$ of $\mathcal{C}$.
\item
A {\em pretorsion class in $\mathcal{C}$} is a class of objects of $\mathcal{C}$ which is closed under coproducts and quotients.
\end{itemize}
\end{definition}

\begin{definition-proposition}\label{def-prop on torsion theories}
We continue to assume that the abelian category $\mathcal{C}$ is complete and co-complete.
\begin{itemize}
\item
Suppose $\mathcal{C}$ is well-generated. A {\em torsion theory on $\mathcal{C}$} is a pair $(\mathcal{T},\mathcal{F})$ of full replete\footnote{A subcategory is said to be {\em replete} if it is contains every object isomorphic to one of its own objects.} subcategories of $\mathcal{C}$ such that:
\begin{enumerate}
\item if $X\in\ob\mathcal{T}$ and $Z\in\ob\mathcal{F}$, then $\hom_{\mathcal{C}}(X,Z) = 0$, and
\item if $Y\in \ob\mathcal{C}$, then there exists a short exact sequence
\[ 0 \rightarrow X \rightarrow Y \rightarrow Z \rightarrow 0\]
in $\mathcal{C}$ such that $X\in\ob\mathcal{T}$ and $Z\in\ob\mathcal{F}$.
\end{enumerate}
\item A class $\mathcal{T}$ of objects of $\mathcal{C}$ is called a {\em torsion class} if there exists a class $\mathcal{F}$ of objects of $\mathcal{C}$ such that $(\mathcal{T},\mathcal{F})$ is a torsion theory on $\mathcal{C}$. Equivalently (see Theorem 2.3 of \cite{MR191935}), a torsion class in $\mathcal{C}$ is a class of objects of $\mathcal{C}$ closed under images, coproducts, and extensions.
\item A torsion theory $(\mathcal{T},\mathcal{F})$ is called {\em hereditary} if every subobject of every object in $\mathcal{T}$ is also in $\mathcal{T}$. A pretorsion class is called {\em hereditary} if it is closed under subobjects. Consequently, a torsion class is hereditary if and only if it is the torsion class of a hereditary torsion theory.
\item Suppose furthermore that $\mathcal{C}$ is Grothendieck, i.e., $\mathcal{C}$ satisfies $AB5$ (that is, $\mathcal{C}$ has all small coproducts, and filtered colimits of exact sequences are exact) and has a generator. Then every object $X$ in $\mathcal{C}$ has a canonical injective hull $E(X)$. %, and we say that $X$ is {\em $(\mathcal{T},\mathcal{F})$-local} if $X$ and the quotient $E(X)/X$ are each in $\mathcal{F}$. We write $\mathcal{C}/\mathcal{T}$ for the full subcategory of $\mathcal{C}$ generated by the $(\mathcal{T},\mathcal{F})$-local objects.
%\item It is proven in \cite{MR0232821} that the inclusion functor $\mathcal{C}/\mathcal{T}\rightarrow \mathcal{C}$ has an exact left adjoint, which we call $Q$. We write $L$ for the composite $L: \mathcal{C}\rightarrow \mathcal{C}$ of $Q: \mathcal{C}\rightarrow \mathcal{C}/\mathcal{T}$ with the inclusion $\mathcal{C}/\mathcal{T} \rightarrow \mathcal{C}$. The functor $L$ is called the {\em localization} associated to the hereditary torsion theory $(\mathcal{T},\mathcal{F})$. We say that the hereditary torsion theory $(\mathcal{T},\mathcal{F})$ is {\em exact} if the localization functor $L$ is exact.
\item 
Still assuming that $\mathcal{C}$ is Grothendieck, a hereditary pretorsion theory $(\mathcal{T},\mathcal{F})$ in $\mathcal{C}$ is called {\em stable} if $\mathcal{T}$ is closed under taking injective hulls; that is, if $X\in\ob\mathcal{T}$, then $E(X)\in\ob\mathcal{T}$. 
\end{itemize}
\end{definition-proposition}

%We follow a common convention in the subject: from now on, in this paper {\bf all torsion theories will be assumed to be hereditary.}

\begin{comment} (FINE, BUT NOT CURRENTLY USED)
The following proposition is well-known: see e.g. Proposition 60.13 for this result in the setting of modules over a ring, or Lemma 3.2(i) in \cite{MR3565424} for this result in the setting of Grothendieck categories.
\begin{prop}\label{dickson thm on stability}
Let $\mathcal{C}$ be a Grothendieck category, and let $(\mathcal{T},\mathcal{F})$ be a stable torsion theory on $\mathcal{C}$. Let $H^0$ denote the right adjoint to the inclusion $\mathcal{T}\rightarrow\mathcal{C}$. Let $n$ be a positive integer. Then the $n$th right derived functor $R^nH^0$ of $H^0$ vanishes on all objects $M$ of $\mathcal{T}$.
\end{prop}
\end{comment}


The following useful result combines Propositions 1.4, 1.7, 2.3, and 3.1 and Corollary 1.8 in chapter VI of \cite{MR0389953}:
\begin{theorem}\label{stenstrom thm}
Let $\mathcal{C}$ be a complete, co-complete abelian category. Given a preradical $r$ on $\mathcal{C}$, let $\mathcal{T}_r$ denote the collection of all objects $X$ of $\mathcal{C}$ such that $\eta X$ is an isomorphism. 
\begin{itemize}
\item If $r$ is idempotent, then $r$ and $\mathcal{T}_r$ determine one another, and consequently there is a one-to-one correspondence between idempotent preradicals on $\mathcal{C}$ and pretorsion classes in $\mathcal{C}$.
\item 
The following are equivalent:
\begin{itemize}
\item $r$ is left exact.
\item $r$ is idempotent and $\mathcal{T}_r$ is hereditary.
\end{itemize}
Consequently there is a one-to-one correspondence between left exact idem preradicals\footnote{Equivalently, left exact idempotent preradicals, since the left exact preradicals are all idempotent.} on $\mathcal{C}$ and hereditary pretorsion classes in $\mathcal{C}$.
\item Furthermore, $r$ is an idempotent radical if and only if $\mathcal{T}_r$ is a torsion class. Consequently we get a one-to-one correspondence between idempotent radicals on $\mathcal{C}$ and torsion theories on $\mathcal{C}$. 
\item %If we furthermore assume that the abelian category $\mathcal{C}$ is Grothendieck, then 
Combining the above results, $r$ is a left exact radical if and only if $\mathcal{T}_r$ is a hereditary torsion class. Consequently we get a one-to-one correspondence between left exact radicals\footnote{Equivalently, left exact idempotent radicals.} on $\mathcal{C}$ and hereditary torsion theories on $\mathcal{C}$.
\end{itemize}
\end{theorem}



\subsection{The hereditary pretorsion class associated to a module $\Theta$.}


\begin{comment}
\begin{lemma} (COMMENTED OUT SINCE NOT CURRENTLY USED.)
Suppose we are given a graded ring $R$, a graded left $R$-module $\Theta$, a $\Theta$-rational left ideal $I$ of $R$, and a $\Theta$-rational graded left $R$-module $M$. Suppose that $m\in M$ is a homogeneous element of $M$. Then both of the following claims are true:
\begin{enumerate}
\item If $im\in h^0_{\dist}(M)\subseteq M$ for all $i\in I$, then $m\in h^0_{\dist}(M)$.
%\item If $im\in H^0_{\dist}(M)\subseteq M$ for all $i\in I$, then $m\in H^0_{\dist}(M)$. (COMMENTED OUT SINCE PERHAPS NOT TRUE.)
\end{enumerate}
\end{lemma}
\begin{proof}\leavevmode
\begin{enumerate}
\item 
Since $im \in h^0_{\dist}(M)$, there exists some distinguished left ideal $J$ of $R$ such that $jim=0$ for all $j\in J$. 
...(UNFINISHED)
\end{enumerate}
\end{proof}
\end{comment}

\begin{definition}
Let $R$ be a graded ring, and let $S$ be a connected ideal set in $R$. 
\begin{itemize}
\item A graded left $R$-module $M$ is {\em $S$-torsion} if the natural map $h^0_{S}(M)\hookrightarrow M$ is an isomorphism. That is, $M$ is $S$-torsion if and only if every homogeneous element of $M$ is $I$-torsion for some member $I$ of $S$.
\item A graded left $R$-module $M$ is {\em $S$-rational} if the natural map $H^0_{S}(M)\hookrightarrow M$ is an isomorphism. That is, $M$ is $S$-rational if and only if every homogeneous element of $M$ is a homogeneous $R$-linear combination of homogeneous elements of $R$, each of which is $I$-torsion for some $I\in S$.
\end{itemize}
\end{definition}
Of course every $S$-torsion module is $S$-rational, and the converse is true when $R$ is commutative. When $R$ is non-commutative, there can exist $S$-rational modules which fail to be $S$-torsion: an example is given in Examples \ref{torsion class examples}.

\begin{prop}\label{rational modules are hereditary pretorsion}
Let $R$ be a graded ring, and let $S$ be a connected ideal set in $R$. Then the following claims are true:
\begin{enumerate}
\item The $S$-torsion modules in $\gr\Mod(R)$ are a hereditary pretorsion class. If $h^1_S(M)$ vanishes on all $S$-torsion graded $R$-modules $M$, then the $S$-torsion modules in $\gr\Mod(R)$ are a hereditary torsion class.
\item The $S$-rational modules in $\gr\Mod(R)$ are a pretorsion class. 
\item If we furthermore have that $S = \overline{\dist(\Theta)}$ for some graded left $R$-module $\Theta$, then the $S$-rational modules in $\gr\Mod(R)$ are a hereditary pretorsion class. 
\end{enumerate}
\end{prop}
\begin{proof}
First, from Definition \ref{def of distinguished}, we know that $H^0_{S}$ is a preradical\footnote{Note that, since $h^0_{S}$ does not necessarily take values in $\gr\Mod(R)$ but only in $\gr\Ab$, $h^0_{S}$ is not generally a preradical.} on $\gr\Mod(R)$.
\begin{enumerate}
\item  
By Theorem \ref{stenstrom thm}, to know that the $\Theta$-torsion modules form a hereditary pretorsion class, we need to show that the $S$-torsion modules are closed under coproducts, quotients, and submodules.
We begin with coproducts. Given a graded left $R$-module $M$, let $\eta_M: h^0_S(M)\rightarrow M$ denote the natural monomorphism.
Then, given a set $\{ M_j: j\in J\}$ of $S$-torsion modules in $\gr\Mod(R)$, we have natural maps fitting into a commutative diagram
\begin{equation}\label{diag 0340405} \xymatrix{
 \underset{I\in S}{\colim}\ \underline{\hom}_R\left(R/RI,\underset{j}{\coprod} M_j \right) \ar[r]^(.6){\eta_{\underset{j}{\coprod} M_j}} 
  & \underset{j}{\coprod} M_j \\
 \underset{i\in S}{\colim}\ \underset{j}{\coprod} \underline{\hom}_R(R/RI,M_j) \ar[u] 
  & \underset{j}{\coprod} \underset{I\in S}{\colim}\ \underline{\hom}_R(R/RI,M_j) \ar[l]^{\cong} \ar[u]^{\cong}_{\underset{j}{\coprod} \eta_{M_j}} .
}\end{equation}
Since $\underset{j}{\coprod} \eta_{M_j}$ is an isomorphism and hence epic, the last map in the composite depicted in diagram \ref{diag 0340405} must also be epic. That map is $\eta_{\underset{j}{\coprod} M_j}$, which is also monic since $S$ is connected. Hence $\eta_{\underset{j}{\coprod} M_j}$ is an isomorphism, i.e., $\underset{j}{\coprod}M_j$ is $S$-torsion.

We also need to show that the $S$-torsion modules are closed under quotients. This is a consequence of an easy diagram chase along the lines of the argument just given for closure under coproducts. Consequently the $S$-torsion modules form a pretorsion class. A similar diagram chase suffices to prove that, if $h^1_S(M)$ vanishes for all $S$-torsion $M$, then the $S$-torsion modules are closed under extensions in $\gr\Mod(R)$, and consequently form a torsion class.

Finally, we also need to show that the $S$-torsion modules are closed under submodules. This follows easily from the understanding of the $S$-torsion modules as those graded $R$-modules in which every homogeneous element is $I$-torsion for some $I\in S$.
\item 
We need to show that the $S$-rational modules are closed under coproducts. Given a set $\{ M_j: j\in J\}$ of $S$-rational modules in $\gr\Mod(R)$, any given homogeneous element $m$ of the coproduct $\coprod_{j\in J}M_j$ is a homogeneous $R$-linear combination of elements which are each concentrated in a single summand $M_j$ of $\coprod_{j\in J}M_j$. Each of those elements, in turn, is a homogeneous $R$-linear combination of homogeneous elements which are $I$-torsion for various ideals $I$ of $S$. Consequently $m$ is a homogeneous $R$-linear combination of homogeneous elements which are $I$-torsion for various ideals $I$ of $S$. Consequently $\coprod_{j\in J} M_j$ is $S$-rational.

Showing that the $S$-rational graded $R$-modules are closed under quotients is a simple matter of a diagram chase. Consequently the $S$-rational modules form a pretorsion class in graded $R$-modules. %Another diagram chase easily shows that, if $H^1_S(M)$ vanishes for all $S$-rational $M$, then the $S$-rational modules are closed under extensions, and consequently the $S$-rational modules form a torsion class. 
\begin{comment} (MUCH SHORTER ARGUMENT USED INSTEAD)
\begin{description}
\item[Closure under coproducts] 
Recall, from the proof of Theorem \ref{theta-rationality and torsion}, the notation $\homog(M)$ for the set of homogeneous elements of a graded left $R$-module $M$. In that proof we considered the graded left $R$-module 
\begin{equation}\label{def of C(M)}\coprod_{m\in \homog(M)} \Sigma^{\left| m\right|} R/R\ann_l(m),\end{equation} and a surjective graded module map $\epsilon_M$ from that module to $M$, whose image is $H^0_{\Theta}(M)$.
We will write $C(M)$ for the graded module \eqref{def of C(M)}.

Let $\{ M_i: i\in I\}$ be a set of $\Theta$-rational modules. We have an inclusion $\cup_{i\in I}\homog(M_i)\subseteq \homog(\coprod_{i\in I}M_i)$, and consequently
morphisms
\begin{align*}
 \coprod_{i\in I} C(M_i) 
  &\stackrel{=}{\longrightarrow} \coprod_{i\in I} \coprod_{m\in \homog(M_i)} \Sigma^{\left| m\right|} R/R\ann_l(m) \\
  &\stackrel{\cong}{\longrightarrow} \coprod_{m\in \cup_{i\in I}\homog(M_i)} \Sigma^{\left| m\right|} R/R\ann_l(m) \\
  &\stackrel{}{\hookrightarrow} \coprod_{m\in \homog(\coprod_{i\in I}M_i)} \Sigma^{\left| m\right|} R/R\ann_l(m) \\
  &\stackrel{=}{\longrightarrow}  C\left( \coprod_{i\in I} M_i\right)
}\]
which commute with the $\epsilon$ morphisms, i.e., we have a commutative diagram of graded left $R$-modules
\[\xymatrix{
 \coprod_{i\in I} C(M_i) \ar[rd]_{\coprod_{i\in I}\epsilon_{M_i}} \ar@{^{(}->}[r] &
  C\left( \coprod_{i\in I} M_i\right) \ar[d]^{\epsilon_{\coprod_{i\in I} M_i}}\\
 & \coprod_{i\in I}M_i.
}\]
The conclusion here is that, if $\coprod_{i\in I}\epsilon_{M_i}$ is epic, so is $\epsilon_{\coprod_{i\in I} M_i}$; i.e., if each of the modules $M_i$ is $\Theta$-rational, then so is $\coprod_{i\in I}M_i$.
\item[Closure under quotients] 
Given a graded $R$-module epimorphism $M \rightarrow N$, we get an epimorphism $\homog(M) \rightarrow \homog(N)$, and consequently a commutative diagram
\[\xymatrix{
 C(M) \ar[d]_{\epsilon_M}\ar[r] & C(N) \ar[d]^{\epsilon_N} \\ 
 M \ar@{->>}[r] & N.
}\]
Consequently if $\epsilon_M$ is surjective, then so is $\epsilon_N$; i.e., if $M$ is $\Theta$-rational, then so is $N$.
\item[Closure under subobjects]
Given a graded $R$-module monomorphism $f: M \rightarrow N$ such that $N$ is $\Theta$-rational, 
we get an monomorphism $\homog(M) \rightarrow \homog(N)$, and consequently a commutative diagram
\begin{equation}\label{comm diag059599}\xymatrix{
 C(M) \ar[d]_{\epsilon_M}\ar[r] & C(N) \ar[d]^{\epsilon_N} \\ 
 M \ar@{^{(}->}[r] & N.
}\end{equation}
Since $f$ is monic, the annihilator of an element $m\in M$ coincides with the annihilator of $f(m)$, so the map $C(M)\rightarrow C(N)$ in diagram \eqref{comm diag059599} is monic. 
Given a homogeneous element $x\in M$, choose an element $\sum_{n\in \homog(N)}r_n\in C(N)$ such that $\epsilon_N\left( \sum_{n\in \homog(N)}r_n\right) = f(x)$, i.e., $\sum_{n\in\homog(N)}r_n\cdot n = f(x)$. The set of elements $n\in \homog(N)$ such that $r_n$ is nonzero is finite, so choose a total ordering $n_1, \dots ,n_N$ on that set. Then we have $f(x) = \sum_{i=1}^N r_{n_i}\cdot n_i$. If $n_1\notin \im f$, then ...(UNFINISHED)
\end{comment}
%This is, however, an immediate consequence of Proposition \ref{rationals closed under quots and subs} together with Theorem \ref{theta-rationality and torsion}, so all that remains is to prove that $H^0_{\Theta}$ is idempotent.
%Applying $h^0_{\Theta}$ to the $R$-module inclusion $H^0_{\Theta}(M)\hookrightarrow M$, we get an isomorphism $h^0_{\Theta}H^0_{\Theta}(M)\stackrel{\cong}{\longrightarrow} h^0_{\Theta}M$, so the $R$-submodule of $H^0_{\Theta}(M)$ generated by $h^0_{\Theta}H^0_{\Theta}(M)$ coincides with the $R$-submodule of $M$ generated by $h^0_{\Theta}(M)$. That is, $H^0_{\Theta}H^0_{\Theta}(M)\rightarrow H^0_{\Theta}(M)$ is an isomorphism.
\item Given a graded left $R$-module $\Theta$, it is a consequence of Theorem \ref{theta-rationality and torsion} that $\overline{\dist(\Theta)}$-rationality---that is, $\Theta$-rationality---is inherited by submodules. 
\end{enumerate}
\end{proof}
\begin{corollary}\label{H0 is left exact}
Let $R$ be a graded ring, and let $\Theta$ be a graded $R$-module. Then $H^0_{\overline{\dist(\Theta)}}: \gr\Mod(R) \rightarrow \gr\Mod(R)$ is left exact.
\end{corollary}
\begin{proof}
Immediate consequence of Theorem \ref{stenstrom thm}
and the third part of Proposition \ref{rational modules are hereditary pretorsion}.
\end{proof}

\begin{corollary}\label{H0 rat and iota}
Let $A$ be a commutative ring, let $\Gamma$ be an $A$-coalgebra projective over $A$, and let $R = \Gamma^*$ be the $A$-linear dual algebra of $\Gamma$. Then $H^0_{\dist}: \gr\Mod(R)\rightarrow \gr\Mod(R)$ is naturally isomorphic to the composite $\iota\circ\tr$ of the rational functor $\tr: \gr\Mod(\Gamma^*)\rightarrow \gr\Comod(\Gamma)$ with the inclusion $\iota: \gr\Comod(\Gamma)\rightarrow\gr\Mod(\Gamma^*)$ of the comodules into the modules.
\end{corollary}

\begin{comment} (NOT CURRENTLY USED)
\begin{corollary}\label{H1 vanishing and torsion}
Let $R$ be a graded ring, and let $\Theta$ be a graded $R$-module. If $H^1_{\overline{\dist(\Theta)}}(M)$ vanishes for all $\Theta$-rational modules $M$ in $\gr\Mod(R)$, then the $\Theta$-rational modules in $\gr\Mod(R)$ form a torsion class.
\end{corollary}
\begin{proof}
By Proposition \ref{rational modules are hereditary pretorsion}, the $S$-rational modules are a pretorsion class, hence are closed under quotients and coproducts. Since Corollary \ref{H0 is left exact} establishes that $H^0_{\Theta}$ is left exact, vanishing of $H^1_{\Theta}$ on the $S$-rational modules ensures that the $S$-rational modules are also closed under extensions, and consequently they form a torsion class.
\end{proof}
\end{comment}

%\begin{remark} As far as the author knows, the $\Theta$-rational modules in $\gr\Mod(R)$ do not necessarily form a torsion class. By Theorem \ref{stenstrom thm}, the $\Theta$-rational modules in $\gr\Mod(R)$ form a torsion class if and only if $H^0_{\Theta}$ is a radical. The most straightforward approach to proving that $H^0_{\Theta}$ is a radical requires that the product of distinguished left ideals also be distinguished, but in general there seems to be no reason to expect that distinguished ideals are closed under finite products of ideals.\end{remark}

\begin{examples}\label{torsion class examples}
Here are some examples to demonstrate the importance of the hypotheses involved in Proposition \ref{rational modules are hereditary pretorsion}.
\begin{itemize}
\item 
There exist rings $R$ and filtered (hence connected) ideal sets $S$ in $R$ such that the pretorsion class of $S$-torsion modules fails to be a torsion class, and also such that the pretorsion class of $S$-rational modules fails to be a torsion class. Here is a simple example of both: let $R = \mathbb{Z}/4\mathbb{Z}$, and let $S$ be $\{ (2)\}$. Then $h^0_S(M) = H^0_S(M)$ is the $2$-torsion submodule of a $\mathbb{Z}/4\mathbb{Z}$-module $M$. We have the short exact sequence of $\mathbb{Z}/4\mathbb{Z}$-modules
\[ 
 0 
  \rightarrow \mathbb{Z}/2\mathbb{Z} 
  \rightarrow \mathbb{Z}/4\mathbb{Z} 
  \rightarrow \mathbb{Z}/2\mathbb{Z} 
  \rightarrow 0\]
in which the left- and right-hand nonzero modules are $S$-torsion and also $S$-rational, but the middle module, $\mathbb{Z}/4\mathbb{Z}$, is neither $S$-torsion nor $S$-rational.
\item 
When the connected ideal set $S$ is not equivalent to $\overline{\dist(\Theta)}$ for some graded left $R$-module $\Theta$, it is not necessarily the case that the pretorsion class consisting of the $S$-rational modules is hereditary. Here is an example: let the ring $R$ be the subalgebra $A(1)$ of the $2$-primary Steenrod algebra generated by $\Sq^1$ and $\Sq^2$. Let $S$ be $\{ A(1)\Sq^1\}$, that is, $S$ has just one member, and that member is the left ideal of $A(1)$ generated by $\Sq^1$. 

Let $M$ be the graded left $A(1)$-submodule of $A(1)$ generated by the element $\Sq^1$. For clarity, write $x$ for this generator, so that $M$ is a four-dimensional $\mathbb{F}_2$-vector space with homogeneous $\mathbb{F}_2$-linear basis \[ \{ x,\ \Sq^2x,\ \Sq^1\Sq^2x,\ \Sq^2\Sq^1\Sq^2x\}.\] The action of the ideal $A(1)\Sq^1$ on the generator $x\in M$ is trivial, since $\Sq^1x=0$. However, the action of $A(1)\Sq^1$ on $M$ is not trivial: the element $\Sq^1\in A(1)\Sq^1$ acts nontrivially on $\Sq^2x$. 

The upshot is that $H^0_S(M) = M$, but $h^0_S(M)$ is strictly smaller than $M$. In particular, $h^0_S(M)$ is the $\mathbb{F}_2$-linear span of $x$, $\Sq^1\Sq^2x,$ and $\Sq^2\Sq^1\Sq^2x$. Hence $M$ is $S$-rational but not $S$-torsion. 

Furthermore, let $M^{\prime}$ be the $A(1)$-submodule of $M$ generated by $\Sq^2x$. Since $h^0_S(M^{\prime})$ is $\mathbb{F}_2$-linearly spanned by $\Sq^1\Sq^2x$ and $\Sq^2\Sq^1\Sq^2x$, we have that $h^0_S(M^{\prime}) = H^0_S(M^{\prime}) \neq M^{\prime}$, so $M^{\prime}$ is neither $S$-rational nor $S$-torsion, despite being a submodule of a $S$-rational module. Hence the pretorsion class of $S$-rational $A(1)$-modules is not hereditary.
\end{itemize}
\end{examples}

\begin{comment} 

\subsection{Graded comodules as a torsion class in graded modules.}

Suppose we are given a coalgebra $\Gamma$ over a field $A$. The $\Gamma$-comodules $\Comod(\Gamma)$ form a pretorsion class in the category $\Mod(\Gamma^*)$ of $\Gamma^*$-modules, but it is not always true $\Comod(\Gamma)$ is a {\em torsion} class in $\Mod(\Gamma^*)$. Equivalently, an extension of rational $\Gamma^*$-modules is not always rational. This problem has been studied at least since Taft's 1972 paper (gx CITE TAFT ``REFLEXIVITY OF ALGEBRAS AND COALGEBRAS'') and Radford's 1973 paper (gx CITE RADFORD ``COREFLEXIVE COALGEBRAS''). Necessary and sufficient conditions for $\Comod(\Gamma)$ to be a torsion class in $\Mod(\Gamma^*)$ can be found in (gx CITE IOVANOV ``ON EXTENSIONS OF RATIONAL MODULES''), in (gx CITE CUADRA ``EXTENSIONS OF RATIONAL MODULES''), and in section 7.6 of chapter 1 of \cite{MR2012570}. 

For the purposes of this paper, we need to know that {\em graded} $\Gamma$-comodules form a torsion class in {\em graded} $\Gamma^*$-modules. Happily, we do not need a thoroughgoing development of the theory: we need only find sufficient conditions, general enough to be satisfied by the dual Steenrod algebras and other Mitchell coalgebras, for $\gr\Comod(\Gamma)$ to be a torsion class in $\gr\Mod(\Gamma^*)$. The following theorem suffices for our purposes:
\begin{theorem}\label{hereditary torsion class thm}
Let $A$ be a field, and let $\Gamma$ be a connected, finite-type graded coalgebra over $A$. Then the graded $\Gamma$-comodules form a hereditary torsion class in the graded $\Gamma^*$-modules.
\end{theorem}
\begin{proof}
Corollary \ref{rational iff dist-torsion} together with Proposition \ref{rational modules are hereditary pretorsion} establishes that the rational modules form a hereditary pretorsion class, so we need only prove that this pretorsion class is a torsion class. That is, we need to show that the rational graded $\Gamma^*$-modules are closed under extensions.

In Corollary 3.17 of (gx CITE CUADRA ``EXTENSIONS OF RATIONAL MODULES''), Cuadra shows that ggxx


\end{proof}

\begin{remark}
Under the hypotheses of Theorem \ref{hereditary torsion class thm}, $\gr\Comod(\Gamma)$ is a hereditary torsion class in $\gr\Mod(\Gamma^*)$, but it is not necessarily a {\em stable} torsion class (see Definition-Proposition \ref{def-prop on torsion theories}). That is, it is not necessarily the case that every rational graded $\Gamma^*$-module embeds into a graded $\Gamma^*$-module which is both gr-injective and rational. This phenomenon arises when the functor $\iota: \gr\Comod(\Gamma)\rightarrow\gr\Mod(\Gamma^*)$ fails to preserve gr-injectivity. Indeed, this bad behavior arises for $\Gamma$ the dual Steenrod algebras, the motivating case for this paper. We take up these issues in the preprint \cite{selfinjectivitypreprint}.
\end{remark}

\end{comment}

\section{Distinguished local cohomology of products.}
\label{Distinguished local cohomology...}
\subsection{Derived products of comodules are given by distinguished local cohomology.}

%\begin{definition} Let $R$ be a graded ring, and let $\Theta$ be a graded left $R$-module. By {\em $\Theta$-distinguished local cohomology} we mean the right-derived functors of the $\Theta$-distinguished torsion functor $H^0_{\Theta}: \gr\Mod(R)\rightarrow \gr\Mod(R)$. We write $H^n_{\Theta}$ for the $n$th $\Theta$-distinguished local cohomology functor, i.e., $H^n_{\Theta}  = R^nH^0_{\Theta}$.

%The most important case, for our purposes, is when $A$ is a commutative ring, $\Gamma$ is an $A$-coalgebra projective over $A$, $R = \Gamma^*$ is the $A$-linear dual algebra of $\Gamma$, and $\Theta = \Gamma$ with the adjoint action of $\Gamma^*$. In that case, we refer to $\Theta$-distinguished local cohomology as simply {\em distinguished local cohomology,} and we write $H^n_{\dist}$ rather than $H^n_{\Theta}$.\end{definition}

\begin{theorem}\label{main thm 1}
Suppose that $\Gamma$ is a graded $A$-coalgebra which is projective as an $A$-module.
Then the following claim is true:
\begin{enumerate}
\item For each nonnegative integer $n$ and each graded left $\Gamma^*$-module $M$, we have an isomorphism $H^n_{\dist}(M) \cong \iota\left( R^n\tr(M)\right)$, natural in the variable $M$.
\end{enumerate}
Suppose furthermore that $A$ is a field, suppose that $\Gamma^*$ is finite-type and connected, and suppose that uniformly bounded-below products of free graded $\Gamma^*$-modules are graded-projective\footnote{These conditions are all satisfied in the case that $\Gamma^*$ is a Steenrod algebra, or more generally, any $P$-algebra in the sense of Margolis; see Theorem 12 in section 13.3 of \cite{MR738973}.}. Then the following claims are also each true:
\begin{enumerate}
\item[(2)] If $M$ is a bounded-above injective graded $\Gamma$-comodule, then the graded $\Gamma^*$-module $\iota(M)$ is injective. 
\item[(3)] The rational graded $\Gamma^*$-modules are a hereditary %stable 
pretorsion class\footnote{We do {\em not} claim that this pretorsion class is a torsion class, or that it is stable. Indeed, $\iota$ can fail to preserve injectivity, and rational graded modules which are not bounded above can fail to embed into any injective rational graded modules. We take up these issues in the preprint \cite{selfinjectivitypreprint}.} in $\gr\Mod(\Gamma^*)$. 
\item[(4)] If $M$ is a bounded-above graded $\Gamma^*$-module, then the distinguished local cohomology groups $H^n_{\dist}(M)$ vanish for all $n>0$. 
\item[(5)] Every bounded-above graded $\Gamma^*$-module is rational.
\item[(6)] Let $I$ be a set, and suppose that, for each $i\in I$, we have a bounded-above\footnote{To be clear: each comodule $M_i$ is assumed to be bounded above, but we do {\em not} assume that the whole collection of comodules $\{ M_i: i\in I\}$ is {\em uniformly} bounded above!} graded $\Gamma$-comodule $M_i$. Then, for each nonnegative integer $n$, the $n$th distinguished local cohomology module $H^n_{\dist}\left(\prod_{i\in I} \iota(M_i)\right)$ of the product of the graded $\Gamma^*$-modules $\iota(M_i)$ is isomorphic to $\iota$ applied to the $n$th right derived functor $R^n\prod^{\Gamma}_{i\in I}\{ M_i\}$ of the product functor $\gr\Comod(\Gamma)^I\stackrel{\prod^{\Gamma}}{\longrightarrow} \gr\Comod(\Gamma)$. That is, we have an isomorphism
\[ \iota\left( R^{n}\prod^{\Gamma}_i\left(\left\{ M_i:i\in I\right\}\right)\right)
 \cong H^n_{\dist}\left( \prod_{i\in I}\iota( M_i)\right).\]
\end{enumerate}
\end{theorem}
\begin{proof}
\begin{enumerate}
\item Since $\tr$ has an exact left adjoint (namely, $\iota$), $\tr$ sends injectives to injectives. Consequently we have a Grothendieck spectral sequence $R^s\iota \left(R^t\tr(M)\right) \Rightarrow R^{s+t}(\iota\circ\tr)(M)$, and it collapses to the $s=0$ line since $\iota$ is exact. We consequently have isomorphisms $\iota \left( R^t\tr(M)\right)\cong R^t(\iota\circ \tr)(M)\cong H^t_{\dist}(M)$ due to Corollary \ref{H0 rat and iota}.
\begin{comment} (NO--CAN'T USE MEGIBBEN'S THEOREM IN THIS WAY.)
\item 
We begin with a few observations, and in the last paragraph of this argument we string these observations together into a simple proof that $\iota$ sends injectives to injectives.

Since $A$ is a field, the injective objects in $\gr\Comod(\Gamma)$ are precisely the comodules whose coaction map is a split monomorphism of comodules; see 1.3.18 of \cite{MR2012570}, for example. That is, the injective objects in $\gr\Comod(\Gamma)$ are the retracts of extended comodules. Because $A$ is a field, every graded $A$-module is a coproduct of suspensions of $A$, so every extended $\Gamma$-comodule is a coproduct of copies of suspensions of $\Gamma$. 

Since $\iota$ is a left adjoint, it preserves coproducts, so if $N$ is an extended $\Gamma$-comodule, then $\iota(N)$ is a coproduct of copies of suspensions of $\iota(\Gamma)$. It is well-known that the linear dual of a connected algebra $R$ over a field, regarded as a graded $R$-module via the adjoint action, is injective; see e.g. Proposition 12 in section 11.3 of \cite{MR738973}, for example. Since we have assumed that $\Gamma$ is finite-type, the linear dual comodule $\Gamma^{**}$ of $\Gamma^*$ is isomorphic to $\Gamma$, and so the graded $\Gamma^*$-module $\iota(\Gamma)$ is injective.

To know that $\iota(N)$ is injective, we must know that a coproduct of copies of $\iota(\Gamma)$ is injective\footnote{The well-known Bass-Papp theorem (see Theorem 3.46 of \cite{MR1653294} for a textbook treatment) establishes that a ring $R$ is left Noetherian if and only if every coproduct of injective left $R$-modules is injective. Nothing in our setup requires $\Gamma^*$ to be Noetherian, however, and our main motivating examples, the Steenrod algebras, are not Noetherian. (More generally, $\mathcal{P}$-algebras are never Noetherian on either side, by Proposition 2 of section 13.1 of \cite{MR738973}.) So there exist coproducts of injective modules over the Steenrod algebra which fail to be injective; we must show that coproducts of copies of $\iota(\Gamma)$ are not among them.}. Split $\iota(N)$ as the graded $\Gamma^*$-module coproduct \[ \iota(N) \cong \coprod_{n\in\mathbb{Z}} \coprod_{J_n} \Sigma^n \iota(\Gamma),\] where $J_n$ is simply the set of summands $\Sigma^n\iota(\Gamma)$ of $\iota(N)$. Since $\iota(\Gamma)$ is {\em co-}connective\footnote{The co-connectivity of $\iota(\Gamma)$ is a critical point here: the same argument does not show that an arbitrary coproduct of suspensions of the {\em connective} countable injective $\Gamma^*$-module $\Gamma^*$ is injective, and indeed, such a coproduct is not injective unless it is uniformly bounded below.}, $\iota(N)$ agrees in each degree with the {\em limit} 
\[ \lim_{m\rightarrow -\infty} \coprod_{n\geq m} \Sigma^n \iota(\Gamma).\]
%There is a graded version of the Bass-Papp theorem (which unfortunately does not appear in the literature anywhere that I am aware of) which depends only on the degree zero subring of an $\mathbb{N}$-graded ring $R$, but it only establishes that {\em uniformly bounded below} coproducts of injectives are injective. In our case, this is not enough: we need to be able to accommodate non-bounded-below coproducts of injectives. There are other possible graded versions of the Bass-Papp theorem (e.g. Proposition 2.3 of \cite{MR3023485}) but none which apply in our situation.
%
%We conclude that Megibben's theorem really is the right tool, here: the various forms of the Bass-Papp theorem are not able to show that coproducts of copies of $\iota(\Gamma)$ are injective.}. 
Now we invoke Theorem \ref{graded megibbens thm}, our graded version of Megibben's theorem \cite{MR633266}: every coproduct of a set of suspensions of a countable injective module, over any $\mathbb{N}$-graded ring, with a uniform lower bound on the degrees of the suspensions, is injective. %Recall that a module over a ring is said to be ``$\Sigma$-injective'' if every coproduct of copies of that module is injective. 
So each of the coproducts $\coprod_{n\geq m}\Sigma^n \iota(\Gamma)$ is injective, so $\iota(N)$ is a limit of injectives, hence injective.

Finally we are ready to prove the claim. If $M$ is an injective graded $\Gamma$-comodule, then $M$ is a summand of the extended comodule $\Gamma\otimes_A M$, so $\iota(M)$ is a summand of $\iota(\Gamma\otimes_AM)$, which we have just shown is injective. Summands of injectives are injective, so $\iota(M)$ is injective, as desired.
\end{comment}
\item The injective graded $\Gamma$-comodule are retracts of the extended graded $\Gamma$-comodules. Since $A$ is a field, the extended graded $\Gamma$-comodules are the coproducts of suspensions of $\Gamma$ itself. Consequently, if we can show that $\iota(M)$ is injective when $M$ is a bounded-above coproduct of suspensions of $\Gamma$, then $\iota(M)$ must be injective for all bounded-above injective $\Gamma$-comodules $M$.

To that end, we suppose that $M$ is the graded $\Gamma$-comodule $\coprod_{s\in S} \Sigma^{d(s)} \Gamma$, where $S$ is a set, and where $d: S\rightarrow\mathbb{Z}$ is a function whose image in $\mathbb{Z}$ is bounded above. Since $\iota$ is a left adjoint, it preserves coproducts, so we have $\iota(M) \cong \coprod_{s\in S} \Sigma^{d(s)} \iota(\Gamma)$. Since $\Gamma$ is finite-type, it is isomorphic to its own double dual. Consequently the $\mathbb{F}_p$-linear dual $A$-module $(\iota(M))^*$ is isomorphic to $\prod_{s\in S} \Sigma^{-d(s)} \Gamma^*$, a uniformly bounded-below product of suspensions of $\Gamma^*$. By our assumptions on $\Gamma^*$, $(\iota(M))^*\cong \prod_{s\in S} \Sigma^{-d(s)} \Gamma^*$ is consequently graded-projective. Now we have that $\iota(M)$ is graded-injective, since its dual is graded-projective; see Proposition 12 of section 11.3 of \cite{MR738973} for this classical fact. 
\item[(3)] Corollary \ref{rational iff dist-torsion} together with Proposition \ref{rational modules are hereditary pretorsion} establishes that the rational modules form a hereditary pretorsion class.
\item[(4, part 1)] We will first prove that $H^n_{\dist}(M)$ vanishes for $n>0$ for all bounded-above graded {\em rational} $\Gamma^*$-modules $M$. We then return to this proof, after proving (5), in order to lift the rationality assumption on $M$.

Let $M$ be a bounded-above rational graded $\Gamma^*$-module. Then the extended graded $\Gamma$-comodule on the underlying graded $\mathbb{F}_p$-vector space of $\tr(M)$ is also bounded-above. Hence $\tr(M)$ admits a resolution by bounded-above graded-injective $\Gamma$-comodules. Since we have already shown that $\iota$ sends bounded-above injectives to bounded-above injectives, $M$ admits a resolution $\mathcal{I}^{\bullet}$ by rational, graded-injective $\Gamma^*$-modules. Consequently $(\iota\circ\tr)(\mathcal{I}^{\bullet})\simeq \mathcal{I}^{\bullet}$ is acyclic.

Since $\tr$ sends injectives to injectives, we have the Grothendieck spectral sequence
\begin{align*}
 (R^s\iota\circ R^t\tr)(M) &\Rightarrow R^{s+t}(\iota\circ\tr)(M).
\end{align*}
Its $E_2^{s,t}$-term $(R^s\iota\circ R^t\tr)(M)$ is isomorphic to $\iota(R^t\tr(M))$, since $\iota$ is exact, so the spectral sequence collapses to the $s=0$-line with no differentials. Its abutment $R^{*}(\iota\circ\tr)(M)$ vanishes in degrees $*>0$, by the acyclicity argument in the previous paragraph. Hence $\iota R^t\tr(M)$, i.e., $H^t_{\dist}(M)$, vanishes for $t>0$.
\item[(5)]
Suppose that $M$ is a graded $\Gamma^*$-module. For each integer $n$, write $\conn_n(M)$ for the graded sub-$\Gamma^*$-module of $M$ generated by all homogeneous elements of degree $\geq n$. Then we have the sequence of monomorphisms \begin{equation}\label{filt 13}\dots \hookrightarrow \conn_n(M) \hookrightarrow \conn_{n-1}(M) \hookrightarrow \conn_{n-2}(M) \hookrightarrow\dots\end{equation}
of graded $\Gamma^*$-modules, and its colimit is $M$. If $M$ is bounded-above, then each of the submodules $\conn_n(M)$ is bounded (i.e., both bounded-above and bounded-below). 

Consequently every bounded-above graded $\Gamma^*$-module is a colimit of bounded graded $\Gamma^*$-modules.
By Proposition \ref{rationals closed under quots and subs}, the rational $\Gamma^*$-modules are closed under cokernels and coproducts in $\gr\Mod(\Gamma^*)$, hence closed under all small colimits in $\gr\Mod(\Gamma^*)$. Hence, if we can show that the bounded graded $\Gamma^*$-modules are rational, then all the bounded-above graded $\Gamma^*$-modules will be rational.

So we suppose that $N$ is a bounded graded $\Gamma^*$-module. We carry out an induction to prove that $H^n_{\dist}(N)$ vanishes for $n>0$. The initial step in the induction is the case in which $N$ is concentrated in a single degree. Since $\Gamma^*$ is connected and $A$ is a field, we have that $N$ splits (as a $\Gamma^*$-module) as a coproduct of copies of $\Gamma^*/\conn_1(\Gamma^*) \cong A$. %, where $\conn_1(\Gamma^*)$ denotes the homogeneous ideal in $\Gamma^*$ generated by all homogeneous elements in positive degrees. 
It is straightforward to see that the graded $\Gamma^*$-module $\Gamma^*/\conn_1(\Gamma^*)$ must consequently be rational. By Proposition \ref{rationals closed under quots and subs}, a coproduct of rational modules is rational, so $N$ is rational, as desired. This completes the initial step in the induction.

The inductive step is as follows: suppose $m$ is a nonnegative integer, and suppose that we have already shown that $N^{\prime}$ is rational for all graded $\Gamma^*$-modules $N^{\prime}$ which are trivial except in $\leq m$ consecutive degrees. (That is, the inductive hypothesis is that, if $r$ is an integer and $N^{\prime}$ is a graded $\Gamma^*$-module which is trivial except in grading degrees $r, r+1, \dots ,r+m-1$, then $N^{\prime}$ is rational.) 
Suppose that $N$ is a graded $\Gamma^*$-module which is trivial except in $m+1$ consecutive degrees. Let $r$ be the lowest degree in which $N$ is nontrivial. Let $N^{\geq r+1}$ be the graded $A$-submodule of $N$ generated by all homogeneous elements of degree $\geq r+1$.
Then $N^{\geq r+1}$ is a graded $\Gamma^*$-submodule of $N$, since $\Gamma^*$ is connective. We consequently have a short exact sequence of graded $\Gamma^*$-modules
\begin{equation}\label{ses 04040} 0 \rightarrow N^{\geq r+1} \rightarrow N \rightarrow N/N^{\geq r+1} \rightarrow 0\end{equation}
and $N^{\geq r+1}$ is trivial except in $\leq m$ consecutive degrees, hence is rational by the inductive hypothesis, and consequently $H^n_{\dist}(N^{\geq r+1})$ vanishes for all $n>0$, by the previous part of this theorem.
Consequently (and using the left-exactness of $H^0_{\dist}$, proven in Corollary \ref{H0 is left exact}), applying $H^*_{\dist}$ to \eqref{ses 04040} yields exactness of the top row in the commutative diagram
\begin{equation}\label{comm diag 0595959}
\xymatrix{
 0 \ar[r] \ar[d] &
  H^0_{\dist}(N^{\geq r+1}) \ar[r]\ar[d]^{\cong} &
  H^0_{\dist}(N) \ar[r]\ar[d] &
  H^0_{\dist}(N/N^{\geq r+1}) \ar[r] \ar[d]^{\cong} &
  0 \ar[d] \\
 0 \ar[r] &
  N^{\geq r+1} \ar[r] &
  N \ar[r] &
  N/N^{\geq r+1} \ar[r] &
 0.}\end{equation}
The vertical maps indicated with the symbol $\cong$ in diagram \eqref{comm diag 0595959} are isomorphisms by the initial step (for $N/N^{\geq r+1}$, since it is concentrated in a single degree), and by the inductive hypothesis (for $N^{\geq r+1}$). %The usual Five Lemma from homological algebra then gives us that 
Si $H^0_{\dist}(N)\rightarrow N$ is also an isomorphism, i.e., $N$ is also rational, completing the inductive step. So every bounded graded $\Gamma^*$-module is rational, as desired.
\item[(4, part 2)] Now it is easy to finish the proof of claim (4): we have already shown that $H^n_{\dist}(M)$ vanishes for all bounded-above graded rational $\Gamma^*$-modules $M$, and we have just shown that every bounded-above $\Gamma^*$-module is rational. So $H^n_{\dist}$ vanishes for $n>0$ on any bounded-above $\Gamma^*$-module.
\item[(6)]
Since products of injectives are injective, the functor $\prod_I:\gr\Mod(\Gamma^*)^I \rightarrow\gr\Mod(\Gamma^*)$ preserves injectives. We have shown that $\iota$ also preserves bounded-above injectives, and an object of the functor category $\gr\Mod(\Gamma^*)^I$ is injective if and only if it is objectwise injective. So, if $M_i$ is a bounded-above graded-injective $\Gamma$-comodule for each $i\in I$, then the product $\prod_{i\in I} \iota(M_i)$ is a graded-injective $\Gamma^*$-module.
So we get a Grothendieck spectral sequence
\begin{align*}
 E_2^{s,t} \cong R^s\tr R^t\left(\prod_{i\in I}\circ\iota\right)\left(\left\{ M_i:i\in I\right\}\right) &\Rightarrow R^{s+t}\left( \tr\circ\prod_{i\in I}\circ\iota\right)\left(\left\{ M_i:i\in I\right\}\right) 
\\ &\cong R^{s+t}\left( \prod^{\Gamma}_i\circ \tr\circ\iota\right)\left(\left\{ M_i:i\in I\right\}\right) 
\\ &\cong R^{s+t} \prod^{\Gamma}_i\left(\left\{ M_i:i\in I\right\}\right).
\end{align*}
This spectral sequence collapses to the $t=0$ line at its $E_2$-page, since $\iota$ and $\prod_I$ are each exact and so their composite $\prod_I\circ\iota$ is exact. 
Consequently we have isomorphisms
\begin{align}
\nonumber \iota R^{s}\prod^{\Gamma}_i\left(\left\{ M_i:i\in I\right\}\right)
  &\cong \iota R^s\tr \left(\prod_{i\in I}\iota( M_i)\right) \\
\label{iso 00094912}  &\cong H^s_{dist}\left( \prod_{i\in I}\iota( M_i)\right) ,
\end{align}
with isomorphism \eqref{iso 00094912} due to the first part of this theorem.
\end{enumerate}
\end{proof}

Theorem \ref{main thm 1} proves that, among other things, the higher distinguished local cohomology groups vanish on the bounded-above graded modules which come (via $\iota$) from comodules.
One might try to think of this result as telling us that distinguished local cohomology $H^*_{\dist}(M)$ is a cohomology theory that tells us how far the module $M$ is from being a comodule. This requires a bit of care. For example, distinguished local cohomology {\em also} vanishes in positive degrees on some $\Gamma^*$-modules which {\em don't} come from comodules. In Theorem 12 of section 13.3 of \cite{MR738973}, Margolis proves that the Steenrod algebras are self-injective, and more generally, that if $\Gamma^*$ is a $\mathcal{P}$-algebra in the sense of Margolis, then $\Gamma^*$ is self-injective. Letting $\Gamma^*$ be the Steenrod algebra, all the hypotheses of Theorem \ref{main thm 1} are satisfied, but not only does $H^n_{\dist}(\Gamma^*)$ vanish for all $n>0$, we also have that $H^0_{\dist}(\Gamma^*)$ vanishes. %(See Lemma \ref{dist ideals identification 2} for a proof of this claim.) 
So $\Gamma^*$ cannot be in the image of $\iota$, since $0\cong H^0_{\dist}(\Gamma^*) \cong \iota(\Rat(\Gamma^*))$.

So while one can truthfully say that $H^*_{\dist}$ ``detects the failure of a bounded-above $\Gamma^*$-module to be a $\Gamma$-comodule,'' it is not the case that the $\Gamma$-comodules are {\em precisely} those $\Gamma^*$-modules on which $H^*_{\dist}$ vanishes. Still, there are satisfying structural relationships between the category of $\Gamma$-comodules and the category of $\Gamma^*$-modules, such as the following theorem:
\begin{theorem}\label{structure thm}
Suppose that $A$ is a field, $\Gamma$ a graded $A$-coalgebra, and that $\Gamma^*$ is finite-type and connected. Suppose that uniformly bounded-below products of free graded $\Gamma^*$-modules are graded-projective. Let $n$ be an integer.
Then every graded $\Gamma^*$-module is an extension of an $n$-co-connected rational graded $\Gamma^*$-module by an $n$-connective graded $\Gamma^*$-module.

Rephrased, with more detail: for each graded $\Gamma^*$-module $M$, we have 
a short exact sequence of graded $\Gamma^*$-modules
\begin{equation}\label{ses 40909090} 0 \rightarrow \conn_n(M) \rightarrow M \rightarrow \iota(\comod_n(M)) \rightarrow 0,\end{equation}
where $\conn_n(M)$ is a {\em $n$-connective} graded $\Gamma^*$-module (i.e., $\conn_n(M)$ is trivial in degrees $<n$), and $\comod_n(M)$ is an {\em $n$-co-connected} graded $\Gamma$-comodule (i.e., $\comod_n(M)$ is trivial in degrees $\geq n$). This sequence is natural in the variable $M$.

Furthermore, the higher distinguished local cohomology of $M$ depends only on $\conn_n(M)$. That is, $H^i_{\dist}(\conn_n(M)) \rightarrow H^i_{\dist}(M)$ is an isomorphism for all $n$ and for all $i>0$.
\end{theorem}
\begin{proof}
Let $\conn_n(M)$ simply be the graded $\Gamma^*$-submodule of $M$ generated by all elements in degrees $\geq n$. The quotient $M/\conn_n(M)$ is then $n$-co-connected, hence bounded above, hence is $\iota$ of a graded $\Gamma$-comodule by Theorem \ref{main thm 1}. 
\end{proof}
It is easy to see that the sequence \eqref{ses 40909090} is natural in the integer $n$, in the sense that we have a commutative diagram of graded $\Gamma^*$-modules
\begin{equation}\label{comm diag 40990909}\xymatrix{
\vdots \ar[d] &
 \vdots \ar[d] &
 \vdots \ar[d] &
 \vdots \ar[d] &
 \vdots \ar[d] \\
0 \ar[r] \ar[d] & 
 \conn_{n+1}(M) \ar[r]\ar@{^{(}->}[d] & 
 M \ar[r]\ar[d]^{\id} & 
 \iota(\comod_{n+1}(M))\ar[r]\ar@{->>}[d] & 0 \ar[d] \\
0 \ar[r] \ar[d] & 
 \conn_n(M) \ar[r]\ar@{^{(}->}[d] & 
 M \ar[r]\ar[d]^{\id} & 
 \iota(\comod_n(M))\ar[r]\ar@{->>}[d] & 0 \ar[d] \\
0 \ar[r] \ar[d] & 
 \conn_{n-1}(M) \ar[r]\ar@{^{(}->}[d] & 
 M \ar[r]\ar[d]^{\id} & 
 \iota(\comod_{n-1}(M))\ar[r]\ar@{->>}[d] & 0 \ar[d] \\
\vdots &
 \vdots &
 \vdots &
 \vdots &
 \vdots 
}\end{equation}
with exact rows.
Taking the limit of each column in \eqref{comm diag 40990909} yields:
\begin{corollary}\label{modules are limits of comodules}
Suppose that $A$ is a field, and that $\Gamma^*$ satisfies the hypotheses stated in Theorem \ref{structure thm}. Then every graded $\Gamma^*$-module is the limit of a Mittag-Leffler sequence of rational graded $\Gamma^*$-modules. 
\end{corollary}
\begin{proof}
The sequence of monomorphisms (i.e., the left-hand nonzero column) in \eqref{comm diag 40990909} is eventually constant in each grading degree, hence is a Mittag-Leffler sequence in each grading degree. Consequently $R^1\underset{n\rightarrow\infty}{\lim}\left(\conn_n(M)\right)$ vanishes in the category $\gr\Mod(\Gamma^*)$. (This argument does not show that $R^1\underset{n\rightarrow\infty}{\lim}\left( \conn_n(M)\right)$ vanishes in the ungraded category $\Mod(\Gamma^*)$: the grading is important here.) Consequently we have the isomorphism $M\stackrel{\cong}{\longrightarrow}\underset{n\rightarrow\infty}{\lim}\iota(\comod_n(M))$ in $\gr\Mod(\Gamma^*)$.
\end{proof}
It would be nice to have a rigorous way to interpret Theorem \ref{structure thm} as stating that ``the category of graded $\Gamma^*$-modules is an extension of the category of co-connected $\Gamma$-comodules by the category of connective $\Gamma^*$-modules,'' but there does not seem to be a notion of ``extension of abelian categories'' in the literature which is of the right kind of generality to include the situation of Theorem \ref{structure thm} as an example. Marmaridis's notion of ``extensions of abelian categories,'' from \cite{MR1213784}, do not suffice, since $\gr\Mod(\Gamma^*)$ is not monadic or comonadic over connective $\Gamma^*$-modules or over co-connected $\Gamma$-comodules (see Theorem 2.6 of \cite{MR1780016} for the relationship between (co)monadicity and Marmaridis's extension theory). The situation of Theorem \ref{structure thm} is not a special case of a ``deformation of abelian categories'' in the sense of \cite{MR2238922} either.

\begin{corollary}\label{uniqueness of module cat}
Let $A,\Gamma^*$ be as in Corollary \ref{modules are limits of comodules}. Then the only full subcategory of $\gr\Mod(\Gamma^*)$ which contains the rational $\Gamma^*$-modules and which is closed under kernels and countable products is $\gr\Mod(\Gamma^*)$ itself.
\end{corollary}
\begin{proof}
Sequential limits in $\gr\Mod(\Gamma^*)$ are kernels of maps between countable products, so this follows from Corollary \ref{modules are limits of comodules}.
\end{proof}


\section{Mitchell coalgebras.}
\label{Mitchell coalgebras...}

Throughout this section, and for the rest of this paper, we assume that the ground ring $A$ is a field, so that we may freely use Theorem \ref{main thm 1}.

A technically complicated, but nevertheless important, example of a filtered ideal set arises from a coalgebra satisfying the {\em Mitchell condition,} which we define in Definition \ref{def of mitchell condition}. It requires that we first recall (from chapter 13 of \cite{MR738973}) the definition of a  ``$\mathcal{P}$-algebra'':
\begin{definition}\label{def of p-alg}
A {\em $\mathcal{P}$-algebra} is a union of a sequence of subalgebras $B(0) \subsetneq B(1) \subsetneq \dots$ such that each $B(n)$ is a Poincar\'{e} algebra, and each $B(n+1)$ is flat over $B(n)$. Here a ``Poincar\'{e} algebra,'' as in \cite{MR335572}, is a finite-dimensional graded connected %\footnote{``Connected'' here means that $A$ is trivial in negative grading degrees and that the unit map $k\rightarrow A^0$ is an isomorphism.} 
algebra $A$ over a field $k$ such that there exists a map of graded $k$-modules $e: A\rightarrow \Sigma^{-n} k$, for some integer $n$, such that the pairing $A^q\otimes_k A^{n-q}\stackrel{\nabla}{\longrightarrow} A^n \stackrel{\Sigma^n e}{\longrightarrow} k$ is nonsingular.
\end{definition}
Of course the most important examples of $\mathcal{P}$-algebras are the Steenrod algebras: the mod $p$ Steenrod algebra is a $\mathcal{P}$-algebra for every prime $p$, by Proposition 7 from section 15.1 of \cite{MR738973}.

\begin{definition}\label{def of mitchell condition}
Suppose $\Gamma$ is a graded coalgebra over a field $A$. 
\begin{itemize}
\item By a {\em Mitchell decomposition of $\Gamma$} we mean the following data:
\begin{itemize} 
\item a sequence $\dots \rightarrow \Gamma(2)\rightarrow \Gamma(1) \rightarrow\Gamma(0)$ of surjective graded $A$-coalgebra morphisms, with each $\Gamma(n)$ a graded quotient $A$-coalgebra of $\Gamma$, 
%$\Gamma^*(0) \subsetneq \Gamma^*(1) \subsetneq \dots$ of graded $A$-subalgebras of $\Gamma^*$ 
such that each of the dual algebras $\Gamma^*(n)$ is a Poincar\'{e} algebra, and such that the left $\Gamma^*(n)$-action on $\Gamma^*(n+1)$ arising from the dual map $\Gamma^*(n)\rightarrow\Gamma^*(n+1)$ of $\Gamma(n+1)\rightarrow\Gamma(n)$ makes $\Gamma^*(n+1)$ flat over $\Gamma^*(n)$.
\item For each nonnegative integer $n$, a homogeneous element $\omega_n \in \Gamma(n)$ whose associated map of graded $A$-modules $e:\Gamma^*(n)\rightarrow \Sigma^{-\left| \omega_n\right|} A$ has the property that the pairing 
\begin{equation}\label{pairing 0439} \Gamma^*(n)^q\otimes_A \Gamma^*(n)^{\left| \omega_n\right|-q}\stackrel{\nabla}{\longrightarrow} \Gamma^*(n)^{\left| \omega_n\right|} \stackrel{\Sigma^{\left| \omega_n\right|} e}{\longrightarrow} A\end{equation} is nonsingular.
%\item For each nonnegative integer $n$, a graded $A$-coalgebra $\Gamma(n)$ equipped with a surjective graded $A$-coalgebra homomorphism $\sigma_n: \Gamma\rightarrow\Gamma(n)$.
%\item For each nonnegative integer $n$, a surjective graded $A$-coalgebra homomorphism $\sigma_{n,n+1}: \Gamma(n+1) \rightarrow\Gamma(n)$. These maps are required to satisfy the condition that $\sigma_{n,n+1}\circ\sigma_{n+1} = \sigma_n$ for all $n$.
%\item A sequence of homogeneous elements $(S^0, S^1, S^2,\dots)$ of $\Gamma^*$ such that, for each nonnegative integer $n$, the graded $A$-subalgebra $\Gamma^*(n)$ of $\Gamma^*$ generated by $\{ S^0, \dots ,S^n\}$ coincides with the $A$-linear dual of the quotient coalgebra $\Gamma(n)$ of $\Gamma$.
\item For each nonnegative integer $n$, an extension of the natural graded left $\Gamma^*(n)$-module structure of $\Gamma^*(n)$ to a graded left action of $\Gamma^*$ on $\Gamma^*(n)$, together with graded left $\Gamma^*(n)$-module homomorphisms $\sigma_n: \Gamma^*\rightarrow \Gamma^*(n)$ and $\sigma_{n,n+1}: \Gamma^*(n+1) \rightarrow \Gamma^*(n)$ such that 
\begin{enumerate}
\item the composite of the $A$-algebra injection $\Gamma^*(n)\hookrightarrow \Gamma^*$ with the left $A$-module morphism $\sigma_n: \Gamma^* \rightarrow\Gamma^*(n)$ is the identity on $\Gamma^*(n)$,
\item the duality isomorphism $\Gamma^*(n)\stackrel{\cong}{\longrightarrow}\Sigma^{\left| \omega_n\right|}\Gamma^*(n)^*$ of graded left $\Gamma^*(n)$-modules, adjoint to \eqref{pairing 0439}, is in fact an isomorphism of graded left $\Gamma^*$-modules,
\item $\sigma_{n,n+1}\circ\sigma_{n+1} = \sigma_n$ for all $n$, 
\item and the resulting graded left $\Gamma^*$-module homomorphism $\Gamma^* \rightarrow \lim_{n\rightarrow\infty}\Gamma^*(n)$ is an isomorphism.
\end{enumerate}
\end{itemize}
\item A {\em Mitchell coalgebra} is a graded coalgebra which admits a Mitchell decomposition. 
\item Given a Mitchell coalgebra $\Gamma$, by a {\em $\mathcal{P}$-sequence} we mean a sequence $\Gamma^*(0)\subseteq\Gamma^*(1)\subseteq\dots$ of graded $A$-subalgebras of $\Gamma^*$ as in the definition of a Mitchell decomposition. 
\item Given a Mitchell coalgebra and a choice of $\mathcal{P}$-sequence, by an {\em orientation sequence} we mean a sequence $\omega_0,\omega_1, \omega_2, \dots$ of elements as in the definition of a Mitchell decomposition.
\item Given a Mitchell coalgebra $\Gamma$ equipped with a choice of Mitchell decomposition with orientation sequence $\omega_0,\omega_1, \omega_2, \dots$, let $\Mit$ be the set of all intersections of finite collections of homogeneous left ideals of $\Gamma^*$ of the form $\ann_{\ell}(\omega_n)$. That is, a homogeneous left ideal $I$ of $\Gamma^*$ is a member of $\Mit$ if and only if there exists a finite set $N$ of nonnegative integers such that $I = \underset{n\in N}{\bigcap}\ann_{\ell}(\omega_n)$.
\end{itemize}
\end{definition}

The main theorem of \cite{MR793186}, expressed in the language just introduced, is that, for each prime $p$, the $p$-primary dual Steenrod algebra is a Mitchell coalgebra.
%(ABOUT ADJOINTS AND CONTRAGREDIENTS: it works out fine, look at what Mitchell actually proves! Include a comment anyway about how, given a finite-type Hopf $A$-algebra $\Gamma$, if we regard $\Gamma \cong \Gamma^{**}$ as a left $\Gamma^*$-module via the {\em contragredient} $\Gamma^*$-action, then the fact that the antipode $\chi: \Gamma^*\rightarrow\Gamma^*$ is an antihomomorphism makes $\Gamma$ also into a right $\Gamma^*$-module, and this right $\Gamma^*$-module agrees with $\Gamma$ with the {\em adjoint} $\Gamma^*$-action.)

\begin{lemma}\label{mitchell maps are surj}
Let $\Gamma$ be a graded coalgebra over a field $A$. Suppose $\Gamma$ is equipped with a choice of Mitchell decomposition. Then, for each nonnegative integer $n$, the graded left $\Gamma^*(n)$-module map $\sigma_{n,n+1}: \Gamma^*(n+1)\rightarrow \Gamma^*(n)$ is surjective.
\end{lemma}
\begin{proof}
The map $\sigma_n: \Gamma^* \rightarrow \Gamma^*(n)$ is a split $\Gamma^*(n)$-module epimorphism, and it factors as the composite $\sigma_{n,n+1}\circ \sigma_{n+1}$. Hence $\sigma_{n,n+1}$ is also surjective.
\end{proof}

Recall from Example \ref{examples of ideal sets} that $\grad$ is the filtered ideal set $\{ I_1, I_2, I_3, \dots\}$ in $\Gamma^*$, where $I_n$ is the left ideal (equivalently, two-sided ideal) in $\Gamma^*$ generated by all homogeneous elements of degree $\geq n$.
\begin{prop}\label{grad dist mit}
Suppose $A$ is a field and $\Gamma$ is a graded $A$-coalgebra concentrated in nonpositive degrees. Then we have $\dist\leq \grad$ in the preorder of filtered ideal sets in $\Gamma^*$. 

If $\Gamma$ is a finite-type Mitchell coalgebra, then we also have $\grad \leq\Mit\leq \dist$, and consequently $\grad$ and $\dist$ are equivalent in the preorder of filtered ideal sets in $\Gamma^*$. 
\end{prop}
\begin{proof}\leavevmode\begin{itemize}
\item 
Recall from Definition \ref{def of distinguished} that a homogeneous left ideal $I$ of $\Gamma^*$ is said to be {\em strongly distinguished} if the rational $\Gamma^*$-module $\iota(\Gamma)$ contains a graded $\Gamma^*$-submodule isomorphic to a suspension of $\Gamma^*/\Gamma^*I$.
Given a strongly distinguished homogeneous left ideal $I$ of $\Gamma^*$, let $\gamma\in \iota(\Gamma)$ be a homogeneous element whose left annihilator ideal $\ann_{\ell}(\gamma)$ is $I$. Let $n$ be the degree of the homogeneous element $\gamma$. Since $\Gamma$ is coconnective, $n$ must be nonpositive, so every element of $\Gamma^*$ of degree $>n$ annihilates $\gamma$, that is, $I_{n+1}\subseteq I$. 

Now by the definition of a distinguished ideal (Definition \ref{def of distinguished}), every element $J$ of $\dist$ contains an intersection $\cap_{j=1}^m J_j$ of a finite set $J_1, \dots ,J_m$ of strongly distinguished homogeneous left ideals of $\Gamma^*$. For each $j=1, \dots,m$, choose a nonnegative integer $f(j)$ such that $I_{f(j)}\subseteq J_j$. Then we have $J\supseteq \cap_{j=1}^m J_j\supseteq \cap_{j=1}^m I_{f(j)} = I_{\max\{ f(1), \dots,f(m)\}}$, so every element $J$ of $\dist$ contains an element $I_{\max\{ f(1), \dots,f(m)\}}$ of $\grad$.
So $\dist\leq \grad$ in the preorder of filtered ideal sets in $\Gamma^*$.
\item
If $\Gamma$ is a finite-type Mitchell coalgebra and $\omega_0,\omega_1,\omega_2,\dots$ is an orientation sequence for $\Gamma$, then by applying the dual $\sigma_n^*: \Gamma(n)\rightarrow \Gamma$ of the left $\Gamma^*$-module map $\sigma_n: \Gamma^*\rightarrow \Gamma^*(n)$ to $\omega_n\in \Gamma(n)$, we get a sequence of elements $\sigma_0^*(\omega_0),\sigma_1^*(\omega_1),\sigma_2^*(\omega_2),\dots$ of $\Gamma$. Since $\sigma_n$ is surjective, its dual $\sigma_n^*$ is injective, so $\ann_{\ell}(\sigma_n^*(\omega_n)) = \ann_{\ell}(\omega_n)$ for each $n$. Consequently each of the left ideals $\ann_{\ell}(\omega_n)$ is strongly distinguished, so the ideal set $\Mit$ is contained in the ideal set $\dist$, and consequently $\Mit\leq \dist$.

To show that $\grad\leq \Mit$, choose some nonnegative integer $n$. We need to show that there exists some member $I$ of $\Mit$ such that $I_n\supseteq I$. That is, we need to find a member $I$ of $\Mit$ which has no nonzero elements of degree $<n$. At this point, we know that $\Gamma^*$ is finite-type, that the maps 
\begin{equation}\label{seq 34094f} \dots \rightarrow \Gamma^*(2) \rightarrow \Gamma^*(1) \rightarrow \Gamma^*(0)\end{equation}
are all grading-preserving and surjective (as a consequence of Lemma \ref{mitchell maps are surj}), and the map $\Gamma^*\rightarrow \lim_m\Gamma^*(m)$ is an isomorphism. Consequently, although the sequence \eqref{seq 34094f} is not necessarily eventually constant, in any single given degree it {\em is} eventually constant. 

In particular, there exists some positive integer $q_n$ such that the map $\Gamma^*\rightarrow \Gamma^*(m)$ is an isomorphism in degrees $\leq n$ for all $m\geq q_n$. By self-duality of $\Gamma^*(m)$, the left annihilator of $\omega_m$ contains no nonzero elements of $\Gamma^*(m)$ itself, so if $m\geq q_n$, $\ann_{\ell}(\omega_m)$ contains no nonzero elements of degree $<n$, as desired.
\end{itemize}
\end{proof}


To a filtered ideal set $S$ in a graded ring $R$, we have the associated local-cohomology-like functor $h_0^S$ given by $\underset{I\in S}{\colim}\ \underline{\hom}_R\left( R/RI,-\right) : \gr\Mod(R) \rightarrow \gr\Ab$. The functor $h_0^S$ comes equipped with a natural transformation $h_0^S \hookrightarrow F$, where $F$ is the forgetful functor $F: \gr\Mod(R)\rightarrow\gr\Ab$. %Put simply, $\underset{I\in S}{\colim}\underline{\hom}_R\left( R/RI,M\right)$ is the subgroup of $M$ consisting of the homogeneous elements of $M$ that are $I$-torsion for some $I\in S$.  For brevity, we refer to elements of a left $R$-module $M$ that are $I$-torsion elements for some $I\in S$ as {\em $S$-torsion elements} of $M$.

Sometimes we are fortunate, and the image of the natural injection of abelian groups $h_0^S(M)\hookrightarrow M$ is actually an $R$-submodule of $M$. That is, when we are lucky, the set of $S$-torsion elements of $M$ is closed under left $R$-scalar multiplication, so that $h^0_S(M) \hookrightarrow H^0_S(M)$ is an isomorphism for all graded $R$-modules $M$. We will call the ideal set $S$ {\em closed} when this condition is satisfied. 

%\begin{remark}
If $R$ is commutative, then every filtered ideal set in $R$ is closed. On the other hand, if $R$ is noncommutative, then $R$ may have some nonclosed filtered ideal sets. One example was given in Remark \ref{remark on ideal sets}. %Another example is as follows: let $k$ be a field, and let $R$ be the free associative $k$-algebra on two degree $1$ generators $x$ and $y$. Let $\Theta = R/Rx$. Then $1\in \Theta$ is in the subgroup $h^0_{\dist(\Theta)}(R)$ of $\Theta$, but $y$ isn't, so $h^0_{\dist(\Theta)}(R)$ isn't an $R$-submodule of $R$, and consequently $h^0_{\dist(\Theta)}(R)$ fails to agree with $H^0_{\dist(\Theta)}(R)$, i.e., the ideal set $\dist(\Theta)$ is not closed. (DOUBLE CHECK THIS EXAMPLE---NOT SURE I BUY THAT DIST(THETA) REALLY IS JUST THE LEFT IDEAL GENERATED BY x.)
%\end{remark}

\begin{lemma}\label{closure closed under equivalence}
Let $R$ be a graded ring.
If two filtered ideal sets $S,S^{\prime}$ in $R$ are equivalent, and if $S$ is closed, then $S^{\prime}$ is also closed.
\end{lemma}
\begin{proof}
Easy consequence of the definitions.
\end{proof}

%Recall the ideal set $\grad$, defined in Example \ref{examples of ideal sets}: it is the set of ideals $\{ I_1, I_2, I_3, \dots\}$ in a graded ring $R$, where $I_n$ is the left ideal in $R$ generated by all homogeneous elements of degree $\geq n$.
\begin{prop}\label{grad is closed}
Suppose $R$ is a connected graded algebra over a field. Then the filtered ideal set $\grad$ is closed.
\end{prop}
\begin{proof}
Elementary.
%Given a graded left $R$-module $M$, a homogeneous element $m\in M$ is $\grad$-torsion if and only if there exists some integer $n(m)$ such that $rm=0$ for all homogenous $r\in R$ of degree $>n(m)$. If $r^{\prime}\in R$ is homogeneous, then for all homogeneous $r\in R$ of degree $>n(m)$, we have that $rr^{\prime}$ is also homogeneous of degree $>n(m)$, and consequently $rr^{\prime}m=0$. So the set of $\grad$-torsion homogeneous elements of $M$ is closed under left multiplication by homogeneous elements of $R$.
\end{proof}

\begin{theorem}\label{main thm on mitchell coalgebras}
Suppose $A$ is a field and $\Gamma$ is a Mitchell coalgebra over $A$. Then the following statements are each true:
\begin{enumerate}
\item The filtered ideal sets $\dist$ and $\grad$ in $\Gamma^*$ are equivalent. In particular, $\dist$ is closed.
\item The functor $h^0_{\dist}: \gr\Mod(R)\rightarrow\gr\Ab$ is equivalent to the composite of $H^0_{\dist}: \gr\Mod(R)\rightarrow\gr\Mod(R)$ with the forgetful functor $\gr\Mod(R)\rightarrow\gr\Ab$.
\item For every integer $n$, the distinguished local cohomology $H^n_{\dist}(M)$ is naturally isomorphic, as an abelian group, to $\underset{I\in \dist(\Gamma)}{\colim}\Ext_{\Gamma^*}\left( \Gamma^*/I,M\right)$.
\item For every integer $n$, the distinguished local cohomology $H^n_{\dist}(M)$ is naturally isomorphic, as an abelian group, to $\underset{j\rightarrow\infty}{\colim}\Ext_{\Gamma^*}\left( \Gamma^*/I_j,M\right)$, where $I_j$ is as in the definition of $\grad$, i.e., $I_j$ is the left ideal of $\Gamma^*$ generated by all homogeneous elements of degree $\geq j$.
%\item Let $\Gamma^*(0)\subseteq\Gamma^*(1)\subseteq\dots$ be a $\mathcal{P}$-sequence in $\Gamma^*$, and for each nonnegative integer $j$, let $I(j)$ denote the homogeneous left ideal of $\Gamma^*$ generated by the homogeneous elements of $\Gamma^*(j)$. Then, for every integer $n$, the distinguished local cohomology functor $H^n_{\dist}$ is naturally equivalent to $\underset{j\rightarrow\infty}{\colim}\underline{\hom}_{\Gamma^*}\left( \Gamma^*/I(j),-)$.
\end{enumerate}
\end{theorem}
\begin{proof}\leavevmode
\begin{enumerate}
\item Immediate from Proposition \ref{grad dist mit}.
\item 
By Proposition \ref{grad is closed}, $\grad$ is closed.
By the previous part of this theorem, $\dist$ and $\grad$ are equivalent. By Lemma \ref{closure closed under equivalence}, $\dist$ is consequently also closed, so $h^0_{\dist}\cong H^0_{\dist}$.
\item Since $\dist$ is filtered, the colimit $\underset{I\in \dist}{\colim}$ is exact, so we have 
\begin{align*}
 H^n_{\dist}(M) 
  &\cong R^nH^0_{\dist}(M) \\
  &\cong R^nh^0_{\dist}(M) \\
  &\cong R^n\left(\underset{I\in \dist}{\colim}\ \underline{\hom}_{\Gamma^*}\left( \Gamma^*/\Gamma^*I,-\right)\right)(M) \\
%  &\cong \underset{I\in \dist}{\colim}\ R^n\left(\underline{\hom}_{\Gamma^*}\left( \Gamma^*/\Gamma^*I,-\right)\right)(M) \\
  &\cong \underset{I\in \dist}{\colim} \Ext_R^n\left( \Gamma^*/\Gamma^*I,M\right).
\end{align*}
\item Immediate from the preceding part of this theorem together with the equivalence of the filtered ideal sets $\grad$ and $\dist$.
%\item From Proposition \ref{grad dist mit} we have that the filtered ideal sets $\Mit$ and $\dist$ are equivalent, so using the filteredness of $\Mit$ we get an isomorphism $H^*_{\dist}(M) \cong \colim_{I\in\Mit}\Ext^*_{\Gamma^*}(\Gamma^*/\Gamma^*I,M)$....?
\end{enumerate}
\end{proof}

See Remark \ref{remark on h0 and H0} for some discussion of why Theorem \ref{main thm on mitchell coalgebras} matters.

%In the Steenrod algebra $\Gamma^*$, the ideal set $\grad = \{ I_1, I_2, \dots\}$ is equivalent to the ideal set of powers $\{ I, I^2, I^3, \dots\}$ of the augmentation ideal of $I$. This is because, for every positive integer $n$, we have $I_n\supseteq I^n$, and because there exists a positive integer $j$ such that $I^n\supseteq I_j$ (otherwise May's filtration on the Steenrod algebra, by powers of the augmentation ideal, would not be a separated/Hausdorff filtration). (NO, I DON'T BUY THE LAST CONTAINMENT!) Consequently we have:
\begin{corollary}\label{main cor 10}
Let $p$ be a prime number, and let $\Gamma^*$ be the mod $p$ Steenrod algebra. Let $M$ be a graded $\Gamma^*$-module. Then, for all integers $n$, the distinguished local cohomology group $H^n_{\dist}(M)$ is isomorphic to the graded local cohomology group $\underset{j\rightarrow\infty}{\colim} \Ext_{\Gamma^*}^n\left( \Gamma^*/I_j,M\right)$, where $I_j$ is the ideal of the Steenrod algebra generated by all homogeneous elements of degree $\geq j$. 
\end{corollary}

\begin{corollary}\label{derived products are local cohomology}
Let $p$ be a prime, let $\Gamma$ be the mod $p$ dual Steenrod algebra, and let $\{ M_i: i\in I\}$ be a set of bounded-above\footnote{To be clear, the comodules $M_i$ do not need to be {\em uniformly} bounded above.} graded $\Gamma$-comodules. Then the $n$th derived functor $R^n\prod^{\Gamma}_{i\in I} M_i$ of product in the category of graded $\Gamma$-comodules is isomorphic, as an abelian group, to the graded local cohomology $\underset{j\rightarrow\infty}{\colim} \Ext_{\Gamma^*}^n\left( \Gamma^*/I_j,\prod_i M_i\right)$. Here $I_j$ is as in Corollary \ref{main cor 10}, and $\prod_i M_i$ is the Cartesian product of the graded $\Gamma^*$-modules $M_i$ with the adjoint action of $\Gamma^*$.
\end{corollary}

As far as the author knows, the original reference on graded local cohomology is \cite{MR494707}, which only considered commutative Noetherian rings. In the noncommutative case, \cite{MR1428799} seems to be the first reference, although it still assumes a Noetherian condition on the graded noncommutative ring. Unfortunately we have not found any results in the literature on local cohomology which appear to be useful in the case where $R$ is a Steenrod algebra, since as far as we have been able to determine, none of the existing literature on local cohomology treats rings which are both noncommutative and non-Noetherian. With the topological applications provided by Corollary \ref{derived products are local cohomology} and the Sadofsky and Hovey-Sadofsky spectral sequences, there is good reason to generalize the existing computational tools for local cohomology to the case of graded noncommutative non-Noetherian rings. That task lies beyond the scope of this paper, though.


\section{Bounds on distinguished local-cohomological dimension.}
\label{Bounds on distinguished...}

%\subsection{Infinite distinguished-cohomological dimension of the category of modules.}
%\label{Infinite distinguished-cohomological...}

The most fundamental and well-known vanishing theorem in classical local cohomology is this: if $I$ is an ideal in a Noetherian commutative ring, then the classical local cohomology groups $H^n_I(M)$ vanish for all $R$-modules $M$ whenever $n$ is greater than the least number of generators for $I$. One wants a generalization of this result which applies to distinguished local cohomology. However, $h^*_{\dist}$ is given by a colimit of $\Ext$ groups associated to a set $S$ of ideals which is not generally the sequence of powers of any single ideal $I$, so it is not clear whether one ought to expect $h^n_{\dist}(M)$ to vanish for all $n$ larger than some particular integer. It is even less clear what to expect from $H^n_{\dist}$ except when the relevant coalgebra $\Gamma$ is either co-commutative, or finite-type and Mitchell, so that $H^n_{\dist}$ agrees with $h^n_{\dist}$.

The main result of this section is Theorem \ref{main thm 4}, which establishes that, for a certain class of graded coalgebras $\Gamma$ (including those whose dual is a $\mathcal{P}$-algebra, so in particular, including the dual Steenrod algebras), there is no such vanishing theorem for distinguished local cohomology. As a consequence we get that the graded comodule category over such a coalgebra fails to be $AB4^*\mhyphen (n)$ for any $n$ whatsoever.

\begin{theorem}\label{main thm 4}
Suppose that $A$ is a field, and that $\Gamma^*$ is finite-type and connected.
Suppose that uniformly bounded-below products of free graded $\Gamma^*$-modules are graded-projective. 
Suppose furthermore that every bounded-below free graded $\Gamma^*$-module is injective in $\gr\Mod(\Gamma^*)$.
Then the following are equivalent:
\begin{enumerate}
\item The category of graded $\Gamma^*$-modules has distinguished local cohomological dimension zero. That is, $H^m_{\dist}(M)$ vanishes for all positive $m$ and all graded $\Gamma^*$-modules $M$.
\item The category of graded $\Gamma^*$-modules has finite distinguished local cohomological dimension. That is, there exists some integer $n$ such that $H^m_{\dist}(M)$ vanishes for all $m>n$ and all graded $\Gamma^*$-modules $M$.
\item The category of bounded-above graded $\Gamma$-comodules satisfies axiom $AB4^*\mhyphen (n)$ for some nonnegative integer $n$. 
That is, there exists some integer $n$ such that $R^m\prod^{\Gamma}_i\left( \left\{ M_i\right\}\right)$ vanishes for all $m>n$, all countable sets $I$, and all sets $\{ M_i:i\in I\}$ of bounded-above graded $\Gamma$-comodules.
\item The category of bounded-above graded $\Gamma$-comodules satisfies Grothendieck's property $AB4^*$. That is, countable products are exact on bounded-above graded $\Gamma$-comodules.
\end{enumerate}
\end{theorem}
\begin{proof}\leavevmode
\begin{description}
\item[(1) implies (2)] Immediate.
\item[(2) implies (1)] Let $M$ be a graded $\Gamma^*$-module. By Theorem \ref{structure thm}, $M$ has the same distinguished local cohomology in positive degrees as the graded $\Gamma^*$-module $\conn_0(M)$ of $M$ generated by all homogeneous elements of nonnegative degree. Since $\conn_0(M)$ is trivial in negative degrees, there exists a connective free graded $\Gamma^*$-module $F$ and an epimorphism $\epsilon: F \rightarrow \conn_0(M)$. Since $F$ is injective, % (by Theorem \ref{thm from margolis 2}), 
we have $H^{i+1}_{\dist}(\ker \epsilon)\cong H^{i}_{\dist}(\conn_0(M))\cong H^{i}_{\dist}(M)$ for all $i\geq 1$. So if $H^i_{\dist}$ is nonzero on some graded $\Gamma^*$-module for some positive $i$, then $H^{i+1}_{\dist}$ is also nonzero on some graded $\Gamma^*$-module. Consequently, under the stated hypotheses, the only way to have a finite cohomological bound on distinguished local cohomology is for that bound to be zero.
\item[(2) implies (3)] Special case of Theorem \ref{main thm 1}.
\item[(1) implies (4)] Special case of Theorem \ref{main thm 1}.
\item[(4) implies (3)] Immediate, since $AB4^*$ is the same condition as $AB4^*\mhyphen (0)$.
\item[(3) implies (2)] Suppose that the category of bounded-above graded $\Gamma$-comodules satisfies axiom $AB4^*\mhyphen (n)$ for some nonnegative integer $n$. Suppose that $M$ is a graded $\Gamma^*$-module. By Corollary \ref{modules are limits of comodules}, $M$ is isomorphic to the limit of a Mittag-Leffler sequence $\dots \rightarrow N_2 \rightarrow N_1 \rightarrow N_0$ of rational, bounded-above graded $\Gamma^*$-modules. Consequently we have a short exact sequence of graded $\Gamma^*$-modules
\begin{equation}\label{ses 30-204949} 0 \rightarrow M \rightarrow \prod_{i\in I} N_i \stackrel{\id - T}{\longrightarrow} \prod_{i\in I} N_i \rightarrow 0\end{equation} with each of the modules $N_0,N_1,N_2,\dots$ bounded-above.
Since each of the $N_i$ are bounded-above, Theorem \ref{main thm 1} gives us an isomorphism of abelian groups \begin{align*} H^*_{\dist}\left( \prod_{i\in I} N_i\right) &\cong R^*\prod^{\Gamma}_i\left( \{ N_i :i\in\mathbb{N}\}\right),\end{align*} so the $AB4^*\mhyphen (n)$ assumption on the comodule category gives us that \linebreak $H^j_{\dist}\left( \prod_{i\in I} N_i\right)$ vanishes for all $j>n$.
The long exact sequence obtained by applying $H^*_{\dist}$ to \eqref{ses 30-204949} consequently gives us that $H^j_{\dist}(M)$ vanishes for all $j>n+1$.
That is, the category of graded $\Gamma^*$-modules has distinguished local cohomological dimension at most $n+1$.
\end{description}
\end{proof}

\begin{corollary}\label{main cor 4a}
Suppose that $A$ is a field, and suppose that the $A$-algebra $\Gamma^*$ is a finite-type $\mathcal{P}$-algebra. Then {\em exactly one} of the two following statements is true:
\begin{enumerate}
\item The category of bounded-above graded $\Gamma$-comodules satisfies condition $AB4^*$. That is, products are exact in the category of graded $\Gamma$-comodules.
\item For each integer $n$, the category of bounded-above graded $\Gamma$-comodules fails to satisfy condition $AB4^*\mhyphen (n)$. That is, for each integer $n$, there exists a countably infinite set $\{ M_0, M_1, M_2, \dots\}$ of graded $\Gamma$-comodules such that $R^m\prod^{\Gamma}_i \left( \{ M_i\}\right)$ is nonzero for some $m>n$.
\end{enumerate}
\end{corollary}
\begin{proof}
By Theorem \ref{thm from margolis 2} from the appendix on Margolis' results on $\mathcal{P}$-algebras, every bounded-below free graded $\Gamma^*$-module is injective. Consequently the hypotheses of Theorem \ref{main thm 4} are satisfied.
\end{proof}

\begin{corollary}\label{no ab4n for dual steenrod alg}
Let $p$ be a prime number, and let $\Gamma$ denote the mod $p$ dual Steenrod algebra. Then, for each integer $n$, the category of bounded-above graded $\Gamma$-comodules fails to satisfy condition $AB4^*\mhyphen (n)$.
That is, there exists a countably infinite set $M_0, M_1, M_2, \dots$ of bounded-above graded $\Gamma$-comodules such that the derived product $R^m\prod^{\Gamma}_i\left(\left\{ M_i: i\in \mathbb{N}\right\}\right)$ is nonzero for some $m>n$.
\end{corollary}
\begin{proof}
It is already well-known (but we still give some explanation in the rest of this proof) that the category of graded comodules over the dual Steenrod algebra does not have exact products, so the claim is a corollary of Corollary \ref{main cor 4a}.

If the category of bounded-above graded comodules over the dual Steenrod algebra had exact products, this would imply that, for any countable set $\{ X_i\}$ of bounded-below $H\mathbb{F}_p$-nilpotently complete spectra, the Hovey-Sadofsky spectral sequence 
\[ E_2^{*,*} \cong R^*\prod_{i\in I}^{\Gamma}\left( \{ H_*(X_i;\mathbb{F}_p)\}\right) \Rightarrow H_*\left( \prod_{i\in I} X_i;\mathbb{F}_p\right)\] collapses on to the $R^0\prod^{\Gamma}$-line at the $E_2$-page. One can consult Sadofsky's unpublished preprint \cite{sadofsky2001homology} for the original construction of this spectral sequence (which is given there more generally for sequential limits, not just countable products), Hovey's paper \cite{MR2337861} for a published account (which, however, discusses convergence only when $H_*(-;\mathbb{F}_p)$ is replaced with a Morava $E$-theory), and the appendix of \cite{MR4080481} for a published account which discusses convergence of the spectral sequence in the case at hand, i.e., the case of mod $p$ homology. Collapse of the spectral sequence for all $\{ X_i: i\in\mathbb{N}\}$ would imply that mod $p$ homology commutes with all countable products of bounded-below $H\mathbb{F}_p$-nilpotently complete spectra, and consequently that mod $p$ homology commutes with all sequential homotopy limits of bounded-below $H\mathbb{F}_p$-nilpotently complete spectra, which is well-known to be untrue without some additional hypothesis (e.g. that the spectra are {\em uniformly} bounded below). 
%for example, by the chromatic convergence theorem of Ravenel and Hopkins, the $p$-local sphere spectrum is weakly equivalent to the homotopy limit of the sphere spectrum localized at each of the Johnson-Wilson homology theories, i.e., $L_{(p)}S \simeq \holim_n L_{E(n)}S$. The $E(n)$-local sphere spectrum $L_{E(n)}S$ is the homotopy limit of an $E(n)$-Adams resolution of the sphere, by the proof (by Devinatz-Hopkins-Smith) that the sphere spectrum is $E(n)$-prenilpotent in the sense of Bousfield. See the book \cite{MR737778} for sustained discussion of these ideas. Every stage in an $E(n)$-Adams resolution is an $E(n)$-module spectrum, hence is $H\mathbb{F}_p$-acyclic since $E(n)$ is $H\mathbb{F}_p$-acyclic. If mod $p$ homology were to commute with countable products, then it would also commute with sequential homotopy limits, and consequently $L_{E(n)}S$ would also be $H\mathbb{F}_p$-acyclic, and so $L_{(p)}S$ would be $H\mathbb{F}_p$-acyclic too. But this is not true: $H_*(L_{(p)}S; \mathbb{Z}_{(p)}) \cong H_*(S; \mathbb{Z}_{(p)})\cong \mathbb{Z}_{(p)}$, and consequently $H_*(L_{(p)}S; \mathbb{F}_p)\cong \mathbb{F}_p$. (THIS EXAMPLE IS BAD, SINCE THE $E(n)$-LOCAL SPHERE ISN'T $HF_p$-LOCAL!)
%An explicit algebraic example of the failure of products to be exact in the category of graded comodules over the dual Steenrod algebra can also be found following Proposition 2.13 in \cite{sadofsky2001homology}.
\end{proof}

\begin{comment} (GOOD, BUT NEEDS TO BE REVISITED: THIS SECTION IS REALLY ABOUT DISTINGUISHED LOCAL COHOMOLOGY IN THE CASE WHERE IT COINCIDES WITH LOCAL COHOMOLOGY WITH RESPECT TO A SINGLE IDEAL. SO IT'S RELEVANT TO NOETHERIAN NONCOMMUTATIVE GRADED LOCAL COHOMOLOGY, BUT NOT THE NON-NOETHERIAN CASE, LIKE THE STEENROD ALGEBRA.)
\section{Characterization of the modules acyclic with respect to distinguished local cohomology.}
\label{Characterization of the modules...}

\begin{prop}\label{ext and Hn triviality prop}
Let $A$ be a field and let $\Gamma$ be a graded $A$-coalgebra such that the ideal set $\dist(\Gamma)$ is equivalent to the set of powers of some graded two-sided ideal $I$ of $\Gamma^*$ such that the ring $\Gamma^*/I$ is graded-semisimple. (For example, we can take $\Gamma$ to be any Mitchell coalgebra, such as the dual Steenrod algebra, by Theorem \ref{main thm on mitchell coalgebras}. Then $I$ can be taken to be the ideal generated by all homogeneous element of positive degree.) Then, for a given graded $R$-module $M$, the following are equivalent:
\begin{enumerate}
\item For all positive integers $n$, $H^n_{\dist}(M)$ is trivial.
\item For all integers $n$,  $\Ext^n_{\Gamma^*}(\Gamma^*/I,M/H^0_{\dist}(M))$ is trivial.
\end{enumerate}
\end{prop} 
\begin{proof}\leavevmode
\begin{description}
\item[1 implies 2] Since $I$ is a two-sided ideal, $\underline{\hom}_{\Gamma^*}(\Gamma^*/I^n,M)$ is a graded $\Gamma^*$-submodule of $M$ for each positive integer $n$ and each graded $\Gamma^*$-module $M$. Consequently the ideal set of powers of $I$ is closed, and so we have the natural isomorphism $H^*_{\dist}\cong h^*_{\dist}$ and consequently isomorphisms 
\begin{align*}
 H^n_{\dist}(M) 
  &\cong h^n_{\dist}(M) \\
  &\cong \colim_{m}\Ext^n_{\Gamma^*}\left( \Gamma^*/I^m,M \right) 
\end{align*}
for each integer $n$.
The short exact sequence of graded $\Gamma^*$-modules
\begin{equation}\label{ses 0404330493057} 0 \rightarrow I^m/I^{m+1} \rightarrow \Gamma^*/I^{m+1}\rightarrow \Gamma^*/I^m \rightarrow 0\end{equation}
induces an injection $\underline{\hom}_{\Gamma^*}(\Gamma^*/I^m,M) \hookrightarrow \underline{\hom}_{\Gamma^*}(\Gamma/I^{m+1},M)$ for each $M$. Consequently the vanishing of $H^0_{\dist}(M/H^0_{\dist}(M)))$ implies the vanishing of $\underline{\hom}_{\Gamma^*}(\Gamma^*/I^m,M/H^0_{\dist}(M))$ for each $m$. 

Now we set up an induction. Suppose that $N$ is some positive integer, and suppose that we have already proven that $\Ext_{\Gamma^*}^n(\Gamma/I,M/H^0(M))$ is trivial for all $n<N$.
Since $\Gamma^*/I$ is graded-semisimple, the graded $\Gamma^*$-module $I^m/I^{m+1}$ splits as a coproduct of suspensions of summands of $\Gamma^*/I$ for each integer $m$. Consequently the vanishing of $\Ext^{N-1}_{\Gamma^*}(\Gamma^*/I,M/H^0_{\dist}(M))$ yields the vanishing of $\Ext^{N-1}_{\Gamma^*}(I^m/I^{m+1},M/H^0_{\dist}(M))$ for each $m$.  The long exact sequence induced in $\Ext^*_{\Gamma^*}(-,M/H^0_{\dist}(M))$ by \eqref{ses 0404330493057} then yields that \[\Ext^N_{\Gamma^*}(\Gamma^*/I^m,M/H^0_{\dist}(M))\rightarrow\Ext^N_{\Gamma^*}(\Gamma^*/I^{m+1},M/H^0_{\dist}(M))\] 
is injective for each $m$. Theorem \ref{main thm 1} gives us that $H^N_{\dist}(M/H^0_{\dist}(M))\cong H^N_{\dist}(M)$, which vanishes by assumption. Since $H^N_{\dist}(M/H^0_{\dist}(M)) \cong \colim_{m}\Ext^N_{\Gamma^*}\left( \Gamma^*/I^m,M/H^0_{\dist}(M) \right)$, the colimit of a sequence of injections, the only way that this colimit can vanish is if each term is trivial, and in particular, the first term $\Ext^N_{\Gamma^*}\left( \Gamma^*/I,M/H^0_{\dist}(M) \right)$ must be trivial, completing the inductive step. Hence $\Ext^N_{\Gamma^*}\left( \Gamma^*/I,M/H^0_{\dist}(M) \right)$ must vanish for all $N$.
\item[2 implies 1] By Theorem \ref{main thm 1}, for positive $n$ we have isomorphisms
\begin{align}
\nonumber H^n_{\dist}(M) 
  &\cong H^n_{\dist}(M/H^0_{\dist}(M)) \\
\nonumber  &\cong h^n_{\dist}(M/H^0_{\dist}(M)) \\
\label{iso 43094995}  &\cong \colim_m\Ext_{\Gamma^*}^n(\Gamma^*/I^m,M/H^0_{\dist}(M)).\end{align}
Graded-semisimplicity of $\Gamma^*/I$ again yields that each quotient $I^m/I^{m+1}$ splits as a coproduct of suspensions of summands of $\Gamma^*/I$, hence 
$\Ext_{\Gamma^*}^n(I^m/I^{m+1},M/H^0_{\dist}(M))$ vanishes for all $m$ and all $n$. Since $\Gamma^*/I^m$ admits a finite filtration by the quotients by lower powers of $I^m$, we have vanishing of $\Ext_{\Gamma^*}^n(\Gamma^*/I^m,M/H^0_{\dist}(M))$ for all $m$ and all $n$. Isomorphism \eqref{iso 43094995} then yields the desired vanishing of $H^n_{\dist}(M)$.
\end{description}\end{proof}

%\begin{lemma} Suppose that $R$ is a connected graded algebra over a field $A$. If $M$ is a bounded below graded $R$-module such that $\Ext^n_R(A,M)\cong 0$ for all $n$, then $M$ is injective. (IS THIS TRUE?) \end{lemma}

We say that a graded $\Gamma^*$-module $M$ is ``$H^0_{\dist}$-acyclic'' if $H^n_{\dist}(M)$ vanishes for all $n>0$.
\begin{theorem}\label{decomposition of acyclics}
Let $A$ be a field and let $\Gamma$ be a Mitchell coalgebra over $A$. Then a graded $\Gamma^*$-module $M$ is $H^0_{\dist}$-acyclic if and only if $M$ admits graded $\Gamma^*$-submodules
\begin{equation}\label{filt 20940} M^{\prime\prime}\subseteq M^{\prime}\subseteq M\end{equation}
such that:
\begin{itemize}
\item $M^{\prime\prime}$ is bounded below and distinguished-torsion,
\item $M^{\prime}/M^{\prime\prime}$ is bounded below and has the property that $\Ext^n_{\Gamma^*}(A,M^{\prime}/M^{\prime\prime})$ vanishes for all $n$,
\item and $M/M^{\prime}$ is bounded above.\end{itemize}
\end{theorem}
\begin{proof}\leavevmode
\begin{itemize}
\item Suppose that $M$ admits submodules as in \eqref{filt 20940}. Then $M^{\prime\prime}$ (by Theorem \ref{main thm 1}) and $M^{\prime}/M^{\prime\prime}$ (by Proposition \ref{ext and Hn triviality prop}) each have trivial higher distinguished local cohomology, and $M^{\prime}$ is an extension of the latter by the former, so $M^{\prime}$ is $H^0_{\dist}$-acyclic.
Similarly, Theorem \ref{main thm 1} gives us that $M/M^{\prime}$ is rational and hence $H^0_{\dist}$-acyclic since it is bounded above. So $M$ is $H^0_{\dist}$-acyclic.
\item Conversely, suppose that $M$ is $H^0_{\dist}$-acyclic. Let $M^{\prime}$ be the graded $\Gamma^*$-submodule of $M$ generated by all elements in degrees $\geq 0$, and let $M^{\prime\prime}$ be $H^0_{\dist}(M^{\prime})$. Then $M/M^{\prime}$ is certainly bounded above (by zero), and $M^{\prime\prime}$ is certainly distinguished-torsion and bounded below (by zero). It is clear that $M^{\prime}/M^{\prime\prime}$ is bounded below (by zero), and the vanishing of $H^n_{\dist}(M)$ for $n>0$ implies the vanishing of $H^n_{\dist}(M^{\prime})$ for $n>0$ as well. By Proposition \ref{ext and Hn triviality prop}, $\Ext^n_{\Gamma^*}(A,M^{\prime}/M^{\prime\prime})$ vanishes for all $n$, as claimed.
\end{itemize}
\end{proof}
\end{comment}

\section{Derived functors of sequential limit in comodule categories.}
\label{Derived functors of seq...}

Let $\Gamma$ be a graded coalgebra over a field $A$. We continue to write $\iota: \gr\Comod(\Gamma) \rightarrow \gr\Mod(\Gamma^*)$ for the inclusion of comodules into modules via the adjoint $\Gamma^*$-action, and we continue to write $\tr: \gr\Mod(\Gamma^*) \rightarrow \gr\Comod(\Gamma)$ for the right adjoint of $\iota$.

%It takes a bit of work to calculate derived sequential imits in comodule categories. The classical construction of sequential $\lim_i M_i $ and $R^1\lim_i M_i$ as the kernel and cokernel of a map $\prod_i M_i \rightarrow \prod_i M_i$ fails in a category with nonvanishing higher sequential derived limit groups, like comodule categories.

Let $\mathcal{D}$ be a small category. 
For any functor $\mathcal{F}:\mathcal{D}\rightarrow\gr\Comod(\Gamma)$, the Bousfield-Kan construction of $\iota\circ\mathcal{F}$ is the cosimplicial graded $\Gamma^*$-module
\begin{equation}\label{bk construction} \xymatrix{ 
\underset{x\in N\mathcal{D}_0}{\prod} \iota\mathcal{F}(\cod x) \ar@<1ex>[r] \ar@<-1ex>[r] & 
 \underset{x\in N\mathcal{D}_1}{\prod} \iota\mathcal{F}(\cod x)  \ar[l]\ar@<2ex>[r]\ar[r]\ar@<-2ex>[r] & 
 \underset{x\in N\mathcal{D}_2}{\prod} \iota\mathcal{F}(\cod x)  \ar@<1ex>[l] \ar@<-1ex>[l] \ar@<3ex>[r] \ar@<1ex>[r] \ar@<-1ex>[r] \ar@<-3ex>[r] & \dots \ar@<2ex>[l]\ar[l]\ar@<-2ex>[l] , }\end{equation}
The notation is as follows: $N\mathcal{D}_{\bullet}$ is the nerve of $\mathcal{D}$, so for $i>0$, $N\mathcal{D}_{i}$ is the set of composable $i$-tuples of morphisms of $\mathcal{D}$, while $N\mathcal{D}_{0}$ is the set of objects of $\mathcal{D}$. The notation $\cod x$ is supposed to denote the final object appearing in a composable $i$-tuple of morphisms $x$; in the case $i=1$, this is simply the codomain of a morphism.
We write $C^{\bullet}(\iota\mathcal{F})$ for the Moore complex (i.e., alternating sum cochain complex) of \eqref{bk construction}, so that $H^i(C^{\bullet}(\iota\mathcal{F})) \cong (R^i\lim)(\iota\mathcal{F})$ by Proposition XI.6.2 of \cite{MR0365573}.


In Theorem \ref{derived sequential lim thm}, we write $\lim^{\Gamma}$ for a limit taken in the category of graded $\Gamma$-comodules.
\begin{theorem}\label{derived sequential lim thm}
Suppose that $A$ is a field and that $\Gamma$ is a finite-type $A$-coalgebra such that $\Gamma^*$ is connected. Suppose that uniformly bounded-below products of free graded $\Gamma^*$-modules are graded-projective.
Let \begin{equation}\label{seq 430949598}\dots\rightarrow M_2\rightarrow M_1\rightarrow M_0\end{equation} be a sequence of bounded-above\footnote{Again, to be clear: these comodules do not need to be {\em uniformly} bounded above.} graded $\Gamma$-comodules such that $R^1\lim$ vanishes on the sequence of graded $\Gamma^*$-modules 
$\dots\rightarrow \iota M_2\rightarrow \iota M_1\rightarrow \iota M_0$.
Then there is an isomorphism of $\Gamma^*$-modules
\begin{align}\label{iso 09994} 
 \iota R^s\lim^{\Gamma}_iM_i &\cong H^s_{\dist}(\lim_i \iota M_i)\end{align}
for each $s$,
and consequently a long exact sequence
\begin{equation}\label{les 239094}\xymatrix{
 0 \ar[r] & 
  \iota \lim^{\Gamma}_i M_i \ar[r] & 
  H^0_{\dist}\left( \prod_i \iota M_i\right) \ar[r] &
  H^0_{\dist}\left( \prod_i \iota M_i\right) \ar`r_l[ll] `l[dll] [dll] \\
 & \iota R^1\lim^{\Gamma}_i M_i \ar[r] &
  H^1_{\dist}\left( \prod_i \iota M_i\right) \ar[r] &
  H^1_{\dist}\left( \prod_i \iota M_i\right) \ar`r_l[ll] `l[dll] [dll] \\
 & \iota R^2\lim^{\Gamma}_i M_i \ar[r] &
  H^2_{\dist}\left( \prod_i \iota M_i\right) \ar[r] &
  \dots .}\end{equation}
\end{theorem}
\begin{proof}
Consider the natural numbers $\mathbb{N}$ under their usual ordering, and let $\mathbb{N}^{\op}$ be the opposite category of the resulting partially-ordered set. Then the sequence \eqref{seq 430949598} is a functor $\mathcal{F}: \mathbb{N}^{\op}\rightarrow\gr\Comod(\Gamma)$.
Since $\tr$ is a right adjoint, it is left exact, so we have the hypercohomology spectral sequences
\begin{align*}
 {}^{\prime}E_1^{s,t} 
  & \cong R^t\tr(C^s\iota \mathcal{F}) \\
 {}^{\prime\prime}E_2^{s,t} 
  &\cong R^s\tr(H^t C^{\bullet}\iota\mathcal{F}) \\
  &\cong R^s\tr(R^t\lim \iota\mathcal{F}) ,
\end{align*}
each of which converges to the cohomology of the total complex of $\tr$ applied to $C^{\bullet}\iota\mathcal{F}$. 
Since $R^t\lim \iota\mathcal{F}$ is a derived functor of sequential limit in a category of modules, it is only capable of being nontrivial in degrees $t=0$ and $t=1$. Furthermore, $R^1\lim \iota\mathcal{F}$ vanishes by assumption. Hence the spectral sequence ${}^{\prime\prime}E_r^{*,*}$ collapses at the ${}^{\prime\prime}E_2$-page, with ${}^{\prime\prime}E_2^{s,0}\cong R^s\tr(\lim\iota\mathcal{F})$ and with ${}^{\prime\prime}E_2^{s,t}$ trivial for $t>0$.

Meanwhile, ${}^{\prime}E_1^{*,t}$ is precisely the Moore complex $C^{\bullet}\left((R^t\tr) \iota\mathcal{F}\right)$ of the Bousfield-Kan construction of $(R^t\tr) \circ \iota\circ \mathcal{F}$, since $\tr$ commutes with products and hence $R^t\tr$ commutes with products. So 
\begin{align}
\nonumber {}^{\prime}E_2^{s,t} &\cong R^s\lim^{\Gamma} (R^t\tr (\iota\mathcal{F}))\\ 
\label{iso 534084}&\cong R^s\lim^{\Gamma}_i (R^t\tr (\iota M_i)),\end{align}
with \eqref{iso 534084} due to exactness of $\iota$.
Now ${}^{\prime}E_2^{s,t}$ is trivial for $t>0$, since $\iota$ is exact and faithful and since $\iota R^t\tr (\iota M_i) \cong H^t_{\dist}(\iota M_i) \cong 0$ for $t>0$, as distinguished local cohomology vanishes on bounded-above modules, by Theorem \ref{main thm 1}. 
Hence ${}^{\prime}E_2^{s,0} \cong R^s\lim^{\Gamma}( \tr\iota\mathcal{F}) \cong R^s\lim^{\Gamma} \mathcal{F}$, and ${}^{\prime}E_2^{s,t}$ is trivial for $t\neq 0$.

Since both spectral sequences converge to the cohomology of the same total complex, we have $R^n\lim^{\Gamma} \mathcal{F} \cong R^n\tr(\lim\iota\mathcal{F})$ for all $n$, and on $\iota$ and using Theorem \ref{main thm 1}, we have the desired isomorphism
\begin{align}\label{iso 40059454} \iota R^n\lim_i^{\Gamma}(M_i) &\cong H^n_{\dist}\left(\lim_i\iota M_i\right).\end{align}

Finally, long exact sequence \eqref{les 239094} arises from applying $H^n_{\dist}$ to the Milnor-type short exact sequence of graded $\Gamma^*$-modules
\[ 0 \rightarrow \lim_i \iota M_i \rightarrow \prod_i \iota M_i \rightarrow \prod_i \iota M_i \rightarrow 0 \]
and using isomorphism \eqref{iso 40059454}.
\end{proof}


\begin{corollary}
Suppose that $A$ is a field and that $\Gamma$ is a Mitchell coalgebra over $A$. Suppose that uniformly bounded-below products of free graded $\Gamma^*$-modules are graded-projective. Let $\dots\rightarrow M_2\rightarrow M_1\rightarrow M_0$ be a sequence of bounded-above graded $\Gamma$-comodules such that $R^1\lim$ vanishes on the sequence of graded $\Gamma^*$-modules 
$\dots\rightarrow \iota M_2\rightarrow \iota M_1\rightarrow \iota M_0$.
Then we have an isomorphism of abelian groups
\[ R^*\lim^{\Gamma}_i M_i  \cong \underset{j\rightarrow\infty}{\colim}\Ext^*_{\Gamma^*}\left(\Gamma^*/I_j,\lim_i \iota M_i\right)\]
where $I_j$ is the ideal of $\Gamma^*$ generated by all homogeneous elements of degree $\geq j$.
\end{corollary}
\begin{proof}
Consequence of Theorems \ref{derived sequential lim thm} and \ref{main thm on mitchell coalgebras}.
\end{proof}

\begin{corollary}\label{steenrod alg seq cor}
Suppose that $p$ is a prime number. Write $\Gamma$ for the dual $p$-primary Steenrod algebra. Let $\dots\rightarrow M_2\rightarrow M_1\rightarrow M_0$ be a sequence of bounded-above graded $\Gamma$-comodules such that $R^1\lim$ vanishes on the sequence of graded $\Gamma^*$-modules 
$\dots\rightarrow \iota M_2\rightarrow \iota M_1\rightarrow \iota M_0$.
Then we have an isomorphism of abelian groups
between $R^*\lim^{\Gamma}_i M_i$ and the graded local cohomology $\underset{j\rightarrow\infty}{\colim}\Ext^*_{\Gamma^*}\left(\Gamma^*/I_j,\lim_i \iota M_i\right)$ of the Steenrod algebra $\Gamma^*$.
\end{corollary}


\appendix
\section{Review of Margolis' basic results on $\mathcal{P}$-algebras.}
\label{Review of Margolis...}


Margolis' fundamental results on graded module theory over $\mathcal{P}$-algebras are Theorem 5 from section 13.2 and Theorem 12 from section 13.3 of Margolis's book \cite{MR738973}. We state those two results:
\begin{theorem}{\bf (Margolis.)} \label{thm from margolis} Let $B$ be a $\mathcal{P}$-algebra, and let $B(0) \subsetneq B(1) \subsetneq \dots$ be a sequence of Poincar\'{e} subalgebras of $B$, as in Definition \ref{def of p-alg}. Let $M$ be a graded left $B$-module. Then the following are equivalent:
\begin{enumerate}
\item The projective dimension of $M$ is $\leq 1$.
\item $M$ is flat.
\item The injective dimension of $M$ is $\leq 1$.
\item $M$ is free over $B(n)$ for each $n$.
\end{enumerate}
Furthermore, if $M$ does not satisfy these (equivalent) conditions, then the projective dimension, weak dimension, and injective dimension of $M$ are each infinite.
\end{theorem}
\begin{theorem}{\bf (Margolis.)} \label{thm from margolis 2} Let $B$ be a $\mathcal{P}$-algebra, and let $B(0) \subsetneq B(1) \subsetneq \dots$ be a sequence of Poincar\'{e} subalgebras of $B$, as in Definition \ref{def of p-alg}. Let $M$ be a {\em bounded-below} graded left $B$-module. Then the following are equivalent:
\begin{enumerate}
\item $M$ is free.
\item $M$ is projective.
\item $M$ is flat.
\item $M$ is injective.
\item $M$ is free over $B(n)$ for each $n$.
\end{enumerate}
Furthermore, if $M$ does not satisfy these (equivalent) conditions, then the projective dimension, weak dimension, and injective dimension of $M$ are each infinite.
\end{theorem}



\bibliography{/home/asalch/texmf/tex/salch}{}
\bibliographystyle{plain}
\end{document}

\subsection{Infinite distinguished-cohomological dimension of individual modules, and the case of the Steenrod algebras.}

In \cref{Infinite distinguished-cohomological...} we showed that there is no finite bound on the distinguished local-cohomological dimension of the graded comodules over the dual of an infinite-dimensional $\mathcal{P}$-algebra over a countable field, like the Steenrod algebra. This leaves open the question of whether there may be some bound on the distinguished local-cohomological dimension of the graded comodules which have finite distinguished local-cohomological dimension.

This question is easier to understand if one compares to the classical notion of finitistic dimension, as in the foundational papers \cite{MR86822} and \cite{MR0086071} and \cite{MR157984}. Let $R$ be a commutative Artin algebra. The {\em finitistic dimension of $R$} is the supremum of the integers $n$ such that there exists a finitely generated $R$-module of projective dimension $n$. For example, Frobenius algebras have finitistic dimension zero, since every finitely generated module over a Frobenius algebra has projective dimension either $0$ or $\infty$. More generally, to say that $R$ has finitistic dimension $\leq n$ is to say that the possible projective dimensions of finitely generated $R$-modules are included in the set $\{ 0, 1, \dots ,n,\infty\}$.

These ideas motivate the following definition:
\begin{definition} \leavevmode
\begin{itemize}
\item Let $M$ be a graded $\Gamma^*$-module. By the {\em distinguished local-cohomological dimension of $M$} we mean the least integer $n$ such that $H^m_{\dist}(M)$ vanishes for all $m>n$, or $\infty$ if there is no such integer $n$.
\item By the {\em distinguished local-cohomological finitistic dimension of $\Gamma$} we mean the supremum of all the integers $n$ such that $n$ is the distinguished local-cohomological dimension of some graded $\Gamma^*$-module, or $\infty$ if there are no such integers $n$.
\item Similarly, by the {\em bounded-below distinguished local-cohomological finitistic dimension of $\Gamma$} we mean the supremum of all the integers $n$ such that $n$ is the distinguished local-cohomological dimension of some bounded-below graded $\Gamma^*$-module, or $\infty$ if there are no such integers $n$.
\end{itemize}
\end{definition}

It follows from Theorem \ref{thm from margolis} that, $\Gamma^*$ is a $\mathcal{P}$-algebra and $M$ is a bounded-below graded $\Gamma^*$-module, then the projective dimension of $M$ is either $0$ or $\infty$, and similarly the injective dimension of $M$ is either $0$ or $\infty$. So if one restricts to only the bounded-below modules, the finitistic dimension of $\Gamma^*$ is $0$, and indeed this is true in an especially strong sense (no restriction to finitely generated modules is necessary, and the role of projective dimension can be replaced with injective dimension without changing the result). 

At a glance it seems plausible that distinguished local-cohomological finitistic dimension could agree with ordinary finitistic dimension (or perhaps its bounded-below variant). In other words, when $\Gamma^*$ is a finite-type, infinite-dimensional $\mathcal{P}$-algebra like the Steenrod algebra, it seems plausible that $\Gamma^*$ could perhaps have bounded-below distinguished local-cohomological finitistic dimension zero.

The purpose of this section is to show that this plausible guess is, in fact, wrong: for every prime $p$, the Steenrod algebra has {\em infinite} distinguished local-cohomological finitistic dimension. That is, for every integer $n$, there exists a graded $\Gamma^*$-module whose distinguished local-cohomological dimension is finite but nevertheless greater than $n$. The same is true in the bounded-below setting as well. Perhaps the same is also true, more generally, for every infinite-dimensional finite-type $\mathcal{P}$-algebra (not only for the Steenrod algebras), but I do not know how to prove that.
\begin{comment}
\begin{lemma}
Suppose the field $A$ is countable, and suppose that the $A$-algebra $\Gamma^*$ is a finite-type $\mathcal{P}$-algebra. Let $M$ be a finite-type, bounded-below free graded $\Gamma^*$-module. Then $H^n_{\dist}(M)$ is trivial for all integers $n$.
\end{lemma}
\begin{proof}
(THIS MIGHT ACTUALLY BE TRUE, BUT IT TURNS OUT I DON'T KNOW HOW TO PROVE IT.)
\end{proof}
\end{comment}

The relevant results are Theorem \ref{inf finitistic dim thm} and Corollary \ref{inf finitistic dim cor}. First we need a couple of lemmas:
\begin{lemma}\label{dist ideals identification}
Let $\Gamma$ denote the $p$-primary dual Steenrod algebra, for some $p$. The following claims are each true:
\begin{enumerate}
\item If $p=2$, then for every positive integer $n$, the two-sided ideal $(\Sq^{2^n},\Sq^{2^{n+1}},\Sq^{2^{n+2}},\dots)$ of $\Gamma^*$ is distinguished\footnote{To be clear: distinguished ideals are always {\em right} ideals, by our conventions in Definition \ref{def of distinguished}. We are claiming here that the two-sided ideal of the Steenrod algebra $\Gamma^*$ generated by $\{\Sq^{2^n},\Sq^{2^{n+1}},\Sq^{2^{n+2}},\dots\}$, when regarded as a right ideal, is distinguished.}
.%\footnote{Since $\Gamma^*$ is finite-type, there is a unique graded coalgebra of which it is the dual algebra (namely, $\Gamma^{**}$), so we can unambiguously speak of distinguished ideals of $\Gamma^*$ without specifying the coalgebra of which $\Gamma^*$ is the dual.}.
\item For every odd prime $p$ and every positive integer $n$, the right ideal $(P^{p^n},P^{p^{n+1}},P^{p^{n+2}},\dots)$ of $\Gamma^*$ is distinguished.
\item If $p=2$, then every distinguished right ideal of $\Gamma^*$ contains $(\Sq^{2^n},\Sq^{2^{n+1}},\Sq^{2^{n+2}},\dots)$ for some $n$.
\item For every odd prime $p$, every distinguished right ideal of $\Gamma^*$ contains $(P^{p^n},P^{p^{n+1}},P^{p^{n+2}},\dots)$ for some $n$.
\end{enumerate}
\end{lemma}
\begin{proof}\leavevmode
\begin{enumerate}
\item Fix a nonnegative integer $n$, and let $A(n)$ be the usual subalgebra of $\Gamma^*$ generated by $\Sq^1, \Sq^2, \dots ,\Sq^{2^n}$. The linear dual $A(n)^*$ is isomorphic to $\mathbb{F}_2[\overline{\xi}_1,\overline{\xi}_2, \dots]/(\overline{\xi}_1^{2^{n+1}},\overline{\xi}_2^{2^{n}}, \overline{\xi}_3^{2^{n-1}}, \dots ,\overline{\xi}_{n+1}^{2})$, hence has a unique top-degree element $\overline{\xi}_1^{2^{n+1}-1}\overline{\xi}_2^{2^{n}-1}\overline{\xi}_3^{2^{n-1}-1}\dots \overline{\xi}_{n+1}$. Lift that element to the element with the same name (i.e., $\overline{\xi}_1^{2^{n+1}-1}\overline{\xi}_2^{2^{n}-1}\overline{\xi}_3^{2^{n-1}-1}\dots \overline{\xi}_{n+1}$) in the dual Steenrod algebra $\Gamma$; call that resulting element $T_n$. 

We claim that the annihilator of $T_n$, under the contragredient right action of $\Gamma^*$ on $\Gamma$, is the two-sided ideal $(\Sq^{2^{n+1}},\Sq^{2^{n+2}}, \dots)$. First, since $A(n)$ is self-dual (i.e., $A(n)$ is a Poincar\'{e} algebra, in the sense of Definition \ref{def of p-alg}), none of the nonzero elements of $A(n)$ itself annihilate $T_n$. The two-sided ideal $(\Sq^{2^{n+1}},\Sq^{2^{n+2}}, \dots)$ is precisely the two-sided ideal of $A$ gggxxxx (ALTERNATIVE APPROACH, MAYBE MUCH EASIER: SHOW THAT IT'S ENOUGH TO TAKE A COLIMIT OVER A SYSTEM OF IDEALS SUCH THAT EVERY DIST IDEAL IS NESTED BETWEEN TWO IDEALS IN THE SYSTEM, EVEN IF THE IDEALS IN THE SYSTEM AREN'T DISTINGUISHED THEMSELVES!)
Observe that the 



The submodule of $\Gamma$ (with the contragredient right $\Gamma^*$-action) generated by the dual class of the top-degree element of $A(n-1)$ is annihilated by $(\Sq^{2^n},\Sq^{2^{n+1}},\dots)$, for degree reasons (ggggxxx NOT JUST FOR DEGREE REASONS! FIX THIS), while the intersection of $A(n-1)$ with the annihilator of that class is trivial, since $A(n-1)$ is self-dual (i.e., $A(n-1)$ is a Poincar\'{e} algebra, in the sense of Definition \ref{def of p-alg}). Consequently that class generated a graded right $\Gamma^*$-submodule of $\Gamma$ isomorphic to $\Gamma^*/(\Sq^{2^n},\Sq^{2^{n+1}},\dots)$.
% How does Sq4 act on the dual of the top class Sq2 Sq2 Sq2 in A(1)?
% Let t denote the dual of Sq2 Sq2 Sq2.
% A^* -> A^* \otimes A^* -> k \otimes A^* -> A^*
% 
\item Similar to the $2$-primary proof just given.
\item If $x$ is a homogeneous element of $\Gamma$, then the $\Gamma^*$-submodule of $\Gamma$ generated by $x$ has the property that $\Sq^nx$ for all sufficiently high $x$, since $\Gamma$ is bounded above. 
\item Similar to the $2$-primary proof just given.
\end{enumerate}
\end{proof}

\begin{lemma}\label{dist ideals identification 2}
Let $\Gamma$ denote the $p$-primary dual Steenrod algebra, for some $p$. 
Then the distinguished local cohomology groups $H^n_{\dist}(\Gamma^*)$ vanish for all $n$. 
\end{lemma}
\begin{proof}
Since $\Gamma^*$ is bounded-below, it is injective by Theorem \ref{thm from margolis 2}. Hence $H^n_{\dist}(M)$ vanishes for $n>0$. Suppose that $x$ is a homogeneous element of $\Gamma^*$. Write $x$ as a $\mathbb{F}_p$-linear combination of elements in the Cartan-Serre admissible basis (I believe \cite{MR60234} and \cite{MR0087934} are the correct original references, but a textbook reference like \cite{MR0145525} is probably more convenient for a modern reader to find and consult). We use the convention that, at odd primes $p$, the symbol $\Sq_p^{2k(p-1)}$ denotes the $k$th Steenrod power $P^k$, and the symbol $\Sq_p^{2k(p-1)+1}$ denotes $\beta P^k$, the B ockstein times the $k$th Steenrod power.
There are only finitely many terms in that expression of $x$ as a linear combination of admissible monomials, and in each term, there are only finitely many $\Sq_p^n$ factors, so there exists a largest $n$ such that $\Sq_p^{n}$ appears in a monomial in our linear combination expressing $x$. Choose any $m$ divisible by $p-1$, and which is larger than $p\cdot n$. Then, since the admissible basis is a basis, multiplying each of the monomials in our linear combination expressing $x$ by $\Sq_p^m$ on the left yields a set of monomials which is still admissible. Hence that set of monomials is still linearly independent. Consequently $\Sq_p^m\cdot x$ is nonzero. Since this argument works for any sufficiently large $m$ divisible by $p-1$, 
\begin{itemize}
\item if $p=2$, it cannot be the case that $x$ is $(\Sq^{2^{\ell}},\Sq^{2^{\ell+1}},\Sq^{2^{\ell+2}},\dots)$-torsion for any $\ell$ at all, and
\item if $p>2$, it cannot be the case that $x$ is $(P^{p^{\ell}},P^{p^{\ell+1}},P^{p^{\ell+2}},\dots)$-torsion for any $\ell$ at all.
\end{itemize}
Hence, by Proposition \ref{dist ideals identification}, $x$ cannot be distinguished-torsion.
\end{proof}

\begin{theorem}\label{inf finitistic dim thm}
Suppose the field $A$ is countable, and suppose that the $A$-algebra $\Gamma^*$ is a finite-type $\mathcal{P}$-algebra. Suppose furthermore that $H^0_{\dist}(\Gamma^*)$ vanishes. Then the distinguished local-cohomological finitistic dimension of $\Gamma$, and the bounded-below distinguished local-cohomological finitistic dimension of $\Gamma$, are each infinite. 

That is, for each integer $n$, there exists some bounded-below graded $\Gamma^*$-module $M$ whose distinguished local-cohomological dimension is finite but greater than $n$. This module can furthermore be chosen to be finite-type.
%\item (NOT SURE IF THIS CLAIM IS TRUE, OR HOW TO PROVE IT.) For each integer $n$, there exists a sequence $M_0, M_1, M_2, \dots$ of bounded-below graded $\Gamma^*$-modules and an integer $m>n$ such that $R^m\prod^{\Gamma}_i \{ M_i\} $ is nonzero but $R^{\ell}\prod^{\Gamma}_i\{ M_i\}$ is trivial for all $\ell>m$. \end{enumerate}
\end{theorem}
\begin{proof}
Let $\mathfrak{m}$ denote the ideal of $\Gamma^*$ generated by all homogeneous elements of positive degree (i.e., $\mathfrak{m}$ is ``irrelevant ideal'' in the sense of the term from projective geometry). Then $\Gamma^*/\mathfrak{m}$ is bounded-above, and consequently is distinguished-torsion, by Theorem \ref{main thm 1}. Hence $H^n_{\dist}(\Gamma^*/\mathfrak{m})$ vanishes for $n>0$, by Corollary \ref{vanishing of higher dist coh on comodules}.
The fact that $\Gamma^*/\mathfrak{m}$ is distinguished-torsion also yields that $H^0_{\dist}(\Gamma^*/\mathfrak{m}) \cong \Gamma^*/\mathfrak{m}$, a one-dimensional vector-space over the ground field $A$.

 Write $M_0$ for $\Gamma^*/\mathfrak{m}$, and write $M_1$ for the kernel of the surjection $\Gamma^* \rightarrow \Gamma^*/\mathfrak{m}$, so that we have a short exact sequence
\begin{equation}
\label{ses 0400430}
 0 \rightarrow M_1 \rightarrow \Gamma^* \rightarrow M_0\rightarrow 0\end{equation}
of bounded-below graded $\Gamma^*$-modules.
The long exact sequence induced in $H^*_{\dist}$ by \eqref{ses 0400430}, together with the vanishing of $H^n_{\dist}(\Gamma^*)$ for all $n$ by Lemma \ref{dist ideals identification 2}, gives us that $H^n_{\dist}(M_1) \cong H^{n-1}_{\dist}(M_0)$ for all $n$. In particular, we have $H^1_{\dist}(M_1)\cong A$, and $H^n_{\dist}(M_1)$ vanishes for all $n\geq 1$. Note also that $M_1$, like $M_0$, is bounded-below and finite-type.

Now we carry out an inductive definition: suppose we have already constructed a bounded-below finite-type graded $\Gamma^*$-module $M_i$ such that $H^i_{\dist}(M_i)\cong A$ and such that $H^n_{\dist}(M_i)\cong 0$ for all $n\neq i$.
Choose a minimal set of homogeneous generators for $M_i$ as a graded $\Gamma^*$-module. Such a set of generators yields a surjective graded $\Gamma^*$-module homomorphism from some {\em bounded-below} finite-type free graded $\Gamma^*$-module $F_i$ to $M_i$. Let $M_{i+1}$ denote the kernel of that surjection $F_i \rightarrow M_i$.
Then the short exact sequence
\begin{equation*}
\label{ses 0400431}
 0 \rightarrow M_{i+1} \rightarrow F_i \rightarrow M_i\rightarrow 0\end{equation*}
of bounded-below graded finite-type $\Gamma^*$-modules induces a long exact sequence in $H^*_{\dist}$ and, by vanishing of $H^*_{\dist}(F_i)$ in all degrees due to Lemma \ref{dist ideals identification 2}, we have an isomorphism $H^n_{\dist}(M_{i+1})\cong H^{n-1}_{\dist}(M_i)$ for all $n$. 

That completes the induction. Consequently, for every nonnegative integer $i$, we have a bounded-below finite-type graded $\Gamma^*$-module $M_i$ such that $H^i_{\dist}(M_i)\cong A$ and such that $H^n_{\dist}(M_i)\cong 0$ for all $n\neq i$.
\end{proof}

\begin{comment}(THIS WAS PART OF A PROOF THAT HIGHER DISTINGUISHED LOCAL COHOMOLOGY VANISHES ON FINITE TYPE FREE MODULES, BUT PROBABLY ISN'T NECESSARY TO SPELL OUT.)
By Theorem \ref{thm from margolis 2}, $M$ is injective, hence $H^n_{\dist}(M)$ vanishes for $n>0$. We still need to show vanishing of $H^0_{\dist}(M)$, which we do as follows:  since $M$ is finite-type and free, $M$ is isomorphic to $\coprod_n \Sigma^{d(n)} \Gamma^*$ for some function $d: \mathbb{Z}\rightarrow\mathbb{N}$. Since $M$ is bounded-below, $d(n)$ vanishes for $n<<0$, and consequently 
we have isomorphisms
\begin{align} 
\label{iso 04040} \underline{\hom}_{\Gamma^*}(\Gamma^*/I,M)
  &\cong \underline{\hom}_{\Gamma^*}\left(\Gamma^*/I,\coprod_n \Sigma^{d(n)}\Gamma^*\right)\\
\nonumber  &\cong \coprod_n \Sigma^{d(n)}\underline{\hom}_{\Gamma^*}\left(\Gamma^*/I,\Gamma^*\right)
\end{align}
for all distinguished right ideals $I$ of $\Gamma^*$. (Note that, since $\Gamma^*$ is not %necessarily 
Noetherian, the natural map 
\[\coprod_i\underline{\hom}_{\Gamma^*}\left(\Gamma^*/I,M_i\right)\rightarrow \underline{\hom}_{\Gamma^*}\left(\Gamma^*/I,\coprod_i M_i\right)\] is not necessarily an isomorphism for an arbitrary set $\{ M_i\}$ of graded $\Gamma^*$-modules, even though $\Gamma^*/I$ is finitely generated.) 

Taking the colimit of \eqref{iso 04040} over all distinguished right ideals $I$ of $\Gamma^*$, we have the isomorphism $H^0_{\dist}(M)\cong \coprod_n\Sigma^{d(n)}H^0_{\dist}(\Gamma^*)$. Consequently all that remains to be shown is that $H^0_{\dist}(\Gamma^*)$ vanishes, that is, that there are no nonzero homogeneous distinguished-torsion elements of the Steenrod algebra. 
\end{comment}

\begin{corollary}\label{inf finitistic dim cor}
Let $\Gamma$ be the mod $p$ dual Steenrod algebra, for some prime $p$. 
Suppose the field $A$ is countable, and suppose that the $A$-algebra $\Gamma^*$ is a finite-type $\mathcal{P}$-algebra. Suppose furthermore that $H^0_{\dist}(\Gamma^*)$ vanishes.
Then the distinguished local-cohomological finitistic dimension of $\Gamma$, and the bounded-below distinguished local-cohomological finitistic dimension of $\Gamma$, are each infinite. 

That is, for each integer $n$, there exists some bounded-below graded $\Gamma^*$-module $M$ whose distinguished local-cohomological dimension is finite but greater than $n$. This module can furthermore be chosen to be finite-type.
\end{corollary}
\begin{proof}
Consequence of Lemma \ref{dist ideals identification 2} and Theorem \ref{inf finitistic dim thm}.
\end{proof}


Lemma \ref{dist ideals identification} also has the following consequence:
\begin{prop}
Let $A$ be the $p$-primary Steenrod algebra, for some prime $p$.
Then the following claims are each true:
\begin{itemize}
\item Let $p=2$. Then 
the functor $H^0_{\dist}:\gr\Mod(A)\rightarrow\grMod(A)$ is isomorphic to $\colim_{n\rightarrow\infty} \underline{\hom}_A\left( A/(\Sq^{2^n}, \Sq^{2^{n+1}}, \dots ),-\right)$. Consequently we have an isomorphism
\[ H^m_{\dist}(M) \cong \colim_{n\rightarrow\infty}\Ext^{m,*}_A\left( A/(\Sq^{2^n},\Sq^{2^{n+1}},\dots),M\right),\]
natural in the variable $M$.
\item Let $p>2$. Then 
the functor $H^0_{\dist}:\gr\Mod(A)\rightarrow\grMod(A)$ is isomorphic to $\colim_{n\rightarrow\infty} \underline{\hom}_A\left( A/(P^{p^n}, P^{p^{n+1}}, \dots ),-\right)$. Consequently we have an isomorphism
\[ H^m_{\dist}(M) \cong \colim_{n\rightarrow\infty}\Ext^{m,*}_A\left( A/(P^{p^n},P^{p^{n+1}},\dots),M\right),\]
natural in the variable $M$.
\item Let $p=2$. Then, given a set $\{ M_i:i\in I\}$ of graded $A^*$-comodules, the $n$th right derived functor $R^*\prod_{i\in I}^{A_*} \left(\left\{ M_i\right\}\right)$ of the product in the category of graded $A_*$-comodules is naturally isomorphic to 
\[\colim_{n\rightarrow\infty}\Ext^{m,*}_A\left( A/(\Sq^{2^n},\Sq^{2^{n+1}},\dots),\prod_{i\in I} \iota(M_i)\right), \]
where $\iota(M_i)$ is the graded $A$-module given by $M_i$ with the contragredient $A$-action, and $\prod_{i\in I} \iota(M_i)$ is the ordinary product in the category of graded $A$-modules.
\item Let $p>2$. Then, given a set $\{ M_i:i\in I\}$ of graded $A^*$-comodules, the $n$th right derived functor $R^*\prod_{i\in I}^{A_*} \left(\left\{ M_i\right\}\right)$ of the product in the category of graded $A_*$-comodules is naturally isomorphic to 
\[\colim_{n\rightarrow\infty}\Ext^{m,*}_A\left( A/(P^{p^n},P^{p^{n+1}},\dots),\prod_{i\in I} \iota(M_i)\right), \]
where again $\iota(M_i)$ is the graded $A$-module given by $M_i$ with the contragredient $A$-action, and $\prod_{i\in I} \iota(M_i)$ is the ordinary product in the category of graded $A$-modules.
\end{itemize}
\end{prop}
\begin{proof}
Let $p=2$. By Proposition \ref{dist ideals identification}, the ideals of the form $(\Sq^{2^n}, \Sq^{2^{n+1}}, \dots )$ are final in the partially-ordered set $\dist(M)$ of distinguished right ideals of $M$. Consequently the colimit over such ideals is isomorphic to the colimit over $\dist(M)$, yielding the first claim. The odd primary argument for the second claim is entirely similar. The third and fourth claims then follow from the first two claims, along with Theorem \ref{main thm 1}.
\end{proof}



\subsection{Derived products in comodules as colimits of $\Ext$ groups.}

Recall, from the discussion from \cref{Distinguished ideals and...}, that unlike the classical local cohomology $H^*_I(M)\cong\colim_{n\rightarrow\infty}\Ext^n_R(R/I^n,M)$, distinguished local cohomology is the derived functors of $H^0_{\dist}$, while $R^*h^0_{\dist}$ is a colimit of $\Ext$-groups, but when the coalgebra $\Gamma$ is not cocommutative, $H^0_{\dist}$ does not necessarily coincide with $h^0_{\dist}$. While $H^*_{\dist}$ has highly desirable properties (most importantly Theorem \ref{main thm 1}), its failure to agree with $R^*h^0_{\dist}$ makes it more difficult to compute. Our goal in this subsection is to find conditions on the coalgebra $\Gamma$ which yield a description of $H^*_{\dist}$ as a colimit of $\Ext$ groups. Of course our conditions need to be satisfied in the case where $\Gamma^*$ is the dual Steenrod algebra at any prime, and indeed, the conditions are satisfied in that case: see ggxx for the main result.

---

\begin{examples} Here are some of the examples of filtered ideal sets we will be concerned with.
\begin{itemize}
\item
Given a left $R$-module $\Theta$, the set $\dist(\Theta)$ defined in Definition-Proposition \ref{def of distinguished} is a filtered ideal set in $R$.  As a special case, when $R = \Gamma^*$ for some coalgebra $\Gamma$, we have the filtered ideal set $\dist \coloneqq \dist(\iota(\Gamma))$. 
\item 
When $R$ is concentrated in nonnegative degrees, we have the filtered ideal set $\grad = \{ I_0,I_1,I_2,I_3,I_4, \dots \}$, where $I_n$ is the homogeneous left ideal in $R$ generated by the homogeneous elements of degree $\geq n$. 
\end{itemize}
\end{examples}

\bibliography{/home/asalch/texmf/tex/salch}{}
\bibliographystyle{plain}
\end{document}




\subsection{The torsion theory associated to a set of homogeneous ideals.}

I have not seen the results in this subsection in any reference on torsion theories, but the ideas are elementary, and I suspect they are quite well-known to those who work with torsion theories, so I do not claim any kind of originality for the material in this subsection.
\begin{definition}\label{def of cofiltered}
Let $S$ be a partially-ordered set.
\begin{itemize}
\item We will say that $S$ is {\em cofiltered} if for every pair of elements $s_1,s_2$ in $S$, there exists an element $s \in S$ such that $s\leq s_1$ and $s\leq s_2$.
\item We will say that two elements $s_1,s_2$ of $S$ are {\em connectable} if there exists a finite sequence $t_1, t_2, \dots t_n$ of elements of $S$ such that $s_1 \geq t_1$ and $t_n\geq s_2$ and such that, for every odd $i\in \{ 1, \dots ,n-1\}$, we have $t_i \leq t_{i+1}$, and such that, for every even $i\in \{1, \dots ,n-1\}$, we have $t_i \geq t_{i+1}$. (In other words, $s_1$ and $s_2$ are connectable if there exists a finite zigzag of elements of $S$ connecting $s_1$ and $s_2$.)
\item %Finite connectability is an equivalence relation on $S$, and by the {\em finite-connectable components of $S$} we mean the equivalence classes under this equivalence relation. 
We will say that $S$ is {\em connected} if every pair of elements of $S$ is connectable.
%\item We will say that $S$ is {\em weakly cofiltered} if, for every connectable pair of elements $s_1,s_2$ of $S$, there exists an element $s \in S$ such that $s\leq s_1$ and $s\leq s_2$.
\end{itemize}
\end{definition}

\begin{definition}
Let $R$ be a graded ring, and let $\mathcal{I}$ be a set of homogeneous proper left ideals of $R$. Equip $\mathcal{I}$ with the partial ordering by inclusion. 
\begin{itemize}
\item We will say that $\mathcal{I}$ is {\em cofiltered} if $\mathcal{I}$, regarded as a partially-ordered set, is cofiltered.
\item Let $r_{\mathcal{I}}$ denote the functor $\gr\Mod(R)\rightarrow\gr\Mod(R)$ given by $r_{\mathcal{I}}(M) = \colim_{I \in\mathcal{I}} \underline{\hom}_R(R/RI,M)$. 
\item If $\mathcal{I},\mathcal{J}$ are sets of homogeneous left ideals of $R$ such that $\mathcal{J}\subseteq\mathcal{I}$, we get a natural transformation $r_{\mathcal{J}} \rightarrow r_{\mathcal{I}}$ given by left Kan extension. 
We will say that $\mathcal{I}$ and $\mathcal{J}$ are {\em torsion-equivalent} if $r_{\mathcal{J}} \rightarrow r_{\mathcal{I}}$ is a natural isomorphism. 
\end{itemize}
\end{definition}

\begin{prop}
Let $\mathcal{I}$ be a connected set of homogeneous proper left ideals of a graded ring $R$. Then $r_{\mathcal{I}}$ is a preradical. Furthermore, $r_{\mathcal{I}}$ is idempotent if and only if $\mathcal{I}$ is connected.
\end{prop}
\begin{proof}
Given a graded $R$-module $M$, an element $x$ of $\colim_{I\in\mathcal{I}}\underline{\hom}_R(R/RI,M) = r_{\mathcal{I}}(M)$ consists of a choice of homogeneous left $I$-torsion element $x_I$ of $M$ for each element $I$ in some cofinal subset of $\mathcal{I}$, such that, if $I_1,I_2\mathcal{I}$ satisfy $I_1\leq I_2$, then $x_{I_1} = x_{I_2}$. Consequently, whenever $I_1$ and $I_2$ are connectable, we have $x_{I_1} = x_{I_2}$. Since $\mathcal{I}$ was assumed to be connected, the situation simplifies: an element $x$ of $r_{\mathcal{I}}(M)$ corresponds to a homogeneous element $\overline{x}$ of $M$ which is $I$-torsion for every ideal $I$ in a cofinal subset of $\mathcal{I}$. We have an obvious inclusion of graded left $R$-modules $r_{\mathcal{I}}(M)\hookrightarrow M$, so $r_{\mathcal{I}}$ is a preradical. 

% Is $r_{\mathcal{I}}(M)$ really a left $R$-submodule of $M$?
%  Given $x\in r_{\mathcal{I}}(M)$ and homogeneous $r\in R$,
%  choose a cofinal subset $S$ of $\mathcal{I}$ such that
%  $x$ is $I$-torsion for all $I\in S$.
% So $ix = 0$ for all homogeneous $i \in I$.
% If $R(ir)$ is contained in $R(i)$ for all homogeneous i,r in R,
%  then $irx = six = 0$ for some homogeneous $s \in R$.
%
% If I is dist-torsion, and ix = 0 for all i in I,
%  choose some homogeneous gamma in Gamma such that I = ann_l(gamma).
%  For each r in Gamma^*, we need to know that 
%   there is some homogeneous gamma' of Gamma
%   such that ann_l(gamma') is contained in I,
%   and such that rx is ann_l(gamma')-torsion.
%   How about letting gamma' = r* gamma?
%   If s \in ann_l(r* gamma), is sx = 0?
%   s (r* gamma) = ....
%   gggggxxxx
\end{proof}



\begin{prop}
Let $R$ be a graded ring. Then every connected set of homogeneous left ideals of $R$ is torsion-equivalent to a cofiltered set of homogeneous left ideals of $R$.
\end{prop}
\begin{proof}
Suppose $\mathcal{I}$ is a connected set of homogeneous left ideals of $R$. Let $\tilde{\mathcal{I}}$ be the set of homogeneous left ideals given by
\[ \tilde{\mathcal{I}} = \left\{ \cup_{I \in X}  I: X\mbox{\ is\ a\ finite\ subset\ of\ } \mathcal{I}\right\}.\]
That is, $\tilde{\mathcal{I}}$ is the set of all intersections of finite subsets of $\mathcal{I}$. It is easy to see that $\mathcal{I}\subseteq \tilde{\mathcal{I}}$, and that $\tilde{\mathcal{I}}$ is cofiltered. 

It takes only a bit more thought to see that $\tilde{\mathcal{I}}$ is torsion-equivalent to $\mathcal{I}$. The argument is as follows: given a graded $R$-module $M$, an element $x$ of $r_{\mathcal{I}}(M)$ consists of a choice of homogeneous left $I$-torsion element $x_I$ of $M$ for each element $I$ in some cofinal subset of $\mathcal{I}$, such that, if $I_1,I_2\mathcal{I}$ satisfy $I_1\leq I_2$, then $x_{I_1} = x_{I_2}$. Consequently, whenever $I_1$ and $I_2$ are finitely connectable, we have $x_{I_1} = x_{I_2}$. Since $\mathcal{I}$ was assumed to be finitely connected, the situation simplifies: an element $x$ of $r_{\mathcal{I}}(M)$ corresponds to a homogeneous element $\overline{x}$ of $M$ which is $I$-torsion for every ideal $I$ in a cofinal subset of $\mathcal{I}$. The image of $x$ in $r_{\tilde{\mathcal{I}}}(M)$ then corresponds to the same homogeneous element $\phi(\overline{x})$ of $M$. Consequently the map $\phi: r_{\mathcal{I}}(M)\rightarrow r_{\tilde{\mathcal{I}}}(M)$ is injective. To see surjectivity of $\phi$, choose an element $x\in r_{\mathcal{I}}(M)$, and let $\overline{x}$ be the corresponding homogeneous element of $M$ and $S$ a corresponding cofinal subset of $\mathcal{I}$ such that $\overline{x}$ is $I$-torsion for 
every $I\in S$. 

% Call a subset $T$ of $\mathcal{I}$ {\em $S$-minimal} if $\cap_{I\in T} I \in S$ but no proper subset $T^{\prime}$ of $T$ has the property that $\cap_{I\in T^{\prime}} I\in S$.
% Let $\tilde{S}$ be the subset of $\mathcal{I}$ consisting of each element $I\in\mathcal{I}$ which is a member of an $S$-minimal subset of $\mathcal{I}$. 
% Suppose that $x$ is $I$-torsion for all $I\in S$. Given $J\in \tilde{S}$, let $\{ J,J_1, \dots ,J_n\}$ be an $S$-minimal subset of $\mathcal{I}$. Then $x$ is $(J\cap \bigcap_n J_n)$-torsion. 


% Suppose that $x$ is $I\cap J$-torsion but not $I$-torsion. Choose some homogeneous $i\in I$
%  such that $ix\neq 0$. Since $I,J$ are members of a cofinal subset of $S$, there exists some 
%  ideal $I'$ such that $I' \subseteq I$ and such that $x$ is $I'$-torsion. 
%  

--

Suppose $\mathcal{I}$ is a set of homogeneous left ideals of $R$. Let $\mathcal{I}^{\prime}$ be the set of homogeneous left ideals given by
\[ \mathcal{I}^{\prime} = \left\{ I\cap J : I,J\in \mathcal{I} \mbox{\ are\ finitely\ connectable}\right\}.\]
Clearly we have $\mathcal{I}\subseteq \mathcal{I}^{\prime}$, since $I = I\cap I$ for every $I\in\mathcal{I}$. We claim that $\mathcal{I}$ and $\mathcal{I}^{\prime}$ are torsion-equivalent. The argument is as follows: given a graded $R$-module $M$, an element $x$ of $r_{\mathcal{I}}(M)$ consists of a choice of homogeneous left $I$-torsion element $x_I$ of $M$ for each element $I$ in some cofinal subset of $\mathcal{I}$, such that, if $I_1,I_2\mathcal{I}$ satisfy $I_1\leq I_2$, then $x_{I_1} = x_{I_2}$. Consequently, whenever $I_1$ and $I_2$ are finitely connectable, we have $x_{I_1} = x_{I_2}$. Hence, given an element of $r_{\mathcal{I}}(M)$, we can construct an element $y$ of $r_{\mathcal{I}^{\prime}}(M)$ by letting $y_{I\cap J} = x_I = x_J$ for each $I,J$ in the same cofinal subset of $\mathcal{I}$, since $x_I$ is an $I$-torsion element and it is equal to the $J$-torsion element $x_J$, so $x_I = x_J$ is in fact an $I\cap J$-torsion element. Restricting $y$ to the given cofinal subset of $\mathcal{I}$ recovers $x$, so the map 
\begin{equation}\label{map 304940945} r_{\mathcal{I}^{\prime}}(M) \rightarrow r_{\mathcal{I}}(M)\end{equation} is surjective. ggggggxxxxxxxxx MAPS GOING THE CORRECT WAY?

It is straightforward to show that the map \eqref{map 304940945} is injective: for an element $x\in r_{\mathcal{I}^{\prime}}(M)$ to map to zero under \eqref{map 304940945} is equivalent to there being some cofinal subset $S$ of $\mathcal{I}$ such that $x_I$ is trivial for all $I\in S$. This in turn implies that $x_I$ vanishes for all $I$ in every finitely connectable component of $\mathcal{I}$, and consequently that $x_I$ vanishes for all $I$ in every finitely connectable component of $\mathcal{I}^{\prime}$, since by construction $\mathcal{I}^{\prime}$ has the same finitely connectable components as $\mathcal{I}$. 

Consequently we have a sequence of inclusions
\[ \mathcal{I} \subseteq \mathcal{I}^{\prime} \subseteq 
 \mathcal{I}^{\prime\prime} \subseteq 
 \mathcal{I}^{\prime\prime\prime} \subseteq \dots\]
each of which is a torsion-equivalence. Our claim is that the union of the sequence, $\cup_m \mathcal{I}^{(m)}$, is weakly cofiltered and torsion-equivalent to $\mathcal{I}$. This claim follows immediately from two sub-claims:
\begin{enumerate}
\item Suppose that \begin{equation}\label{seq 04999494}\mathcal{I}_0 \subseteq \mathcal{I}_1 \subseteq \mathcal{I}_2 \subseteq \dots\end{equation} is a sequence of torsion-equivalences of sets $\mathcal{I}_0, \mathcal{I}_1, \mathcal{I}_2,\dots $ of left homogeneous ideals of $R$, and suppose that, for each positive integer $n$, there exists a positive integer $n^{\prime}\geq n$ such that $\mathcal{I}_{n^{\prime}}$ contains $\mathcal{I}_n^{\prime}$. Then the union $\cup_m \mathcal{I}_m$ is weakly cofiltered. 
\item If we have a sequence of torsion-equivalences as in \eqref{seq 04999494}, then 
$\mathcal{I}_0$ is torsion-equivalent to the union $\cup_m \mathcal{I}_m$ of the sequence.
\end{enumerate}
We prove these claims in turn:
\begin{enumerate}
\item Given a finitely connectable pair $I,J\in \cup_m\mathcal{I}_m$, there exists an integer $N_I$ such that $I\in \mathcal{I}_{N_I}\subseteq \cup_m\mathcal{I}_m$, and an integer $N_J$ such that $J\in \mathcal{I}_{N_b}\subseteq \cup_m\mathcal{I}_m$. 
Since $I,J$ are finitely connectable, there exists finite sequence $t_1, \dots ,t_n$ in $\cup_m\mathcal{I}_m$ as in Definition \ref{def of cofiltered}. For each $i\in \{ 1, \dots ,n\}$, choose an integer $N_i$ such that $t_i \in \mathcal{I}_{N_i}\subseteq \cup_m\mathcal{I}_m$. Finally, let $M$ denote the maximum $M = \max\{ N_I,N_J, N_1, N_2, \dots ,N_n\}$. Then each of the elements $I,J,t_1, \dots ,t_n$ exists in $\mathcal{I}_{M}$, and the sequence $t_1, \dots ,t_n$ gives us that $I$ and $J$ are finitely connectable in $\mathcal{I}_M$. By the definition of $\mathcal{I}_M^{\prime}$, we have that $I$ and $J$ have an upper bound in $\mathcal{I}_{M}^{\prime} = \mathcal{I}_{M^{\prime}}$, so $I,J$ also have an upper bound in $\cup_n\mathcal{I}_n$.
\item For any graded $R$-module $X$, we have natural (in the variable $X$) isomorphisms
\begin{align}
\nonumber r_{\cup_n\mathcal{I}_n}(X) 
  &= \colim_{I\in \cup_n\mathcal{I}_n} \underline{\hom}_R(R/RI,X) \\
\nonumber  &= \colim_{I\in \colim_n \mathcal{I}_n} \underline{\hom}_R(R/RI,X) \\
\label{colims commute w colims}  &= \colim_n \left( \colim_{I\in \mathcal{I}_n} \underline{\hom}_R(R/RI,X)\right) \\
\nonumber  &= \colim_n r_{\mathcal{I}_n}(X),
\end{align}
where \eqref{colims commute w colims} (ggggxxxxx HANDLE THIS MORE CAREFULLY! THE MAPS ARE NOT OBVIOUSLY GOING IN THE CORRECT DIRECTION FOR THIS ARGUMENT) is simply due to the classical fact, from category theory, that colimits commute with colimits. The sequence of graded $R$-module homomorphisms
\[ r_{\mathcal{I}_0}(X) \rightarrow ggxx
\end{enumerate}

All that is left is to see that $\mathcal{I}^{\prime}$ is weakly cofiltered. gggggxxxxx




\end{proof}


\bibliography{/home/asalch/texmf/tex/salch}{}
\bibliographystyle{plain}
\end{document}




\subsection{Comodules are the torsion class in a hereditary torsion theory on modules.}

A graded $\Gamma^*$-module $M$ is distinguised-torsion (as defined in Definition-Proposition \ref{def of distinguished}) if, for every homogeneous $m \in M$, there exists some distinguished left ideal $I$ of $\Gamma^*$ such that $im=0$ for all $i\in I$.
It is easy to see that $H^0_{\dist}$ is an idempotent preradical and that $\mathcal{T}_{H^0_{\dist}}$ is simply the category of distinguished-torsion graded $\Gamma^*$-modules, which is certainly closed under subobjects in $\gr\Mod(\Gamma^*)$. Consequently, by Theorem \ref{stenstrom thm}, $H^0_{\dist}$ is left exact. It takes only a bit more thought to verify that $H^0_{\dist}$ is not only a preradical but in fact a radical, that is, for any graded $\Gamma^*$-module $M$, there are no nonzero distinguished-torsion elements of the quotient $M/H^0_{\dist}(M)$. Definition-Proposition \ref{def of distinguished} then gives us that the distinguished-torsion graded $\Gamma^*$-modules are the torsion class of a hereditary torsion theory on graded $\Gamma^*$-modules. 

\begin{comment}
\subsection{Filteredness of $\dist(\iota\Gamma)$ and left-exactness of $H^0_{\dist}$.}

In order to prove that $H^0_{\dist}$ is left exact, we must prove that the partially-ordered set $\dist(\iota\Gamma)$ is filtered. To that end, we introduce an auxiliary definition, Definition-Proposition \ref{def of polydist ideals}. This definition is used only for the sake of proving the filteredness of $\dist(\iota\Gamma)$ in Proposition\ref{dist is filtered}, and the reader who does not care to verify that the proof of that proposition is correct can safely ignore the whole idea of ``polydistinguished ideals.''
\begin{definition-proposition}\label{def of polydist ideals} Let $I$ be a homogeneous left ideal of $\Gamma^*$. Then the following conditions are equivalent:
\begin{itemize}ggxxx (UPDATE W/ GABRIEL TOPOLOGIES)
\item There exists an exact sequence of graded left $\Gamma^*$-modules
\[ 0 \rightarrow I \stackrel{f}{\longrightarrow} \Gamma^* \rightarrow \Sigma^n \iota\Gamma \]
for some integer $n$, where $f$ is the canonical inclusion of $I$ into $\Gamma^*$.
\item $\iota\Gamma$ contains a graded $\Gamma^*$-submodule which is isomorphic to a suspension of a quotient of $\Gamma^*/\Gamma^*I$.
\end{itemize}
We will call a left ideal $I$ of $\Gamma^*$ {\em distinguished} if it is homogeneous and satisfies the above equivalent conditions.
We write $\dist(\iota\Gamma)$ for the set of distinguished left ideals of $\Gamma^*$, partially ordered by inclusion. We write $H^0_{\dist}$ for the\footnote{When the coalgebra $\Gamma$ is co-commutative, then $H^0_{\dist}(M)$ is not only a graded abelian group but also has an action of $\Gamma^*$, and so $H^0_{\dist}$ is in fact a functor $\gr\Mod(\Gamma^*)\rightarrow\gr\Mod(\Gamma^*)$. It will be clear from context whether each particular instance of the symbol $H^0_{\dist}$ is the functor $H^0_{\dist}: \gr\Mod(\Gamma^*)\rightarrow\gr\Mod(\Gamma^*)$ or the functor $H^0_{\dist}: \gr\Mod(\Gamma^*)\rightarrow\gr\Ab$.} functor $\gr\Mod(\Gamma^*)\rightarrow\gr\Mod(\Gamma^*)$ given by $H^0_{\dist}(M) = \colim_{I\in \dist(\iota\Gamma)}\underline{\hom}_{\Gamma^*}\left( \Gamma^*/\Gamma^*I,M\right)$.

There is a natural monomorphism of graded abelian groups
\begin{equation}\label{nat map 1} H^0_{\dist}(M)\hookrightarrow M\end{equation} induced by the composite $\underline{\hom}_{\Gamma^*}\left( \Gamma^*/\Gamma^* I,M\right) \rightarrow \underline{\hom}_{\Gamma^*}\left( \Gamma^*,M\right) \stackrel{\cong}{\longrightarrow} M$. The map \eqref{nat map 1} is a graded $\Gamma^*$-module homomorphism if $\Gamma$ is co-commutative. We will say that $M$ is {\em distinguished-torsion} if the map \eqref{nat map 1} is an isomorphism.
\end{definition-proposition}

\begin{definition}\label{def of polydist ideals}
gx

\end{definition}

\end{comment}

\begin{comment}
\subsection{Property $\Cyc$ and left-exactness of $H^0_{\dist}$.}

Most of this paper is, either directly or indirectly, about the right-derived functors $H^n_{\dist} \vcentcolon= R^nH^0_{\dist}$ of $H^0_{\dist}$.
If $H^0_{\dist}$ is not left exact, then its right-derived functors are much more difficult to compute or prove anything about. Consequently we want to know that $H^0_{\dist}$ is left exact. Of course $\underline{\hom}_{\Gamma^*}(\Gamma^*/\Gamma^*I,-)$ is left exact for each $I\in \dist(\Gamma)$, so the natural way to prove that $H^0_{\dist} = \colim_{I\in \dist(\Gamma)}\underline{\hom}_{\Gamma^*}(\Gamma^*/\Gamma^*I,-)$ is left exact is to show that the set $\dist(\Gamma)$ of distinguished left ideals of $\Gamma^*$, partially-ordered by inclusion, is filtered. Filteredness of $\dist(\Gamma)$ then implies that $\colim_{I\in \dist(\Gamma)}$ is exact.

%(CAREFUL ABOUT THESE ANNIHILATORS!)
%It is nontrivial to show that $\dist(\Gamma)$ is filtered. Consider the case where $\Gamma$ is the $2$-primary dual Steenrod algebra, and where the distinguished left ideals are the annihilator $\Ann(\xi_2^2)$ of the cyclic $\Gamma^*$-submodule $\langle \xi_2^2\rangle$ of $\Gamma$, and the annihilator $\Ann(\xi_3)$ of the cyclic $\Gamma^*$-submodule $\langle \xi_3\rangle$ of $\Gamma$. One might imagine that the cyclic submodule of $\Gamma^*$ generated by $\xi_2^2\xi_3$ has annihilator contained in $\Ann(\xi_2^2)\cap \Ann(\xi_3)$, but an explicit calculation yields that $(\xi_2^2)^*(\xi_2^2\xi_3) = \xi_3 + \xi_1\xi_2^2$, while $S(\xi_2^2\xi_3)=0$ for all degree $6$ elements $S$ of the Steenrod algebra which are not multiples of $\xi_2^2\xi_3$. Consequently $\Ann(\xi_2^2\xi_3)$ fails to be contained in $\Ann(\xi_3)$. This counterexample rules out several na\"{i}ve approaches to trying to show that $\dist(\Gamma)$ is filtered.

It is indeed the case that $\dist(\Gamma)$ is filtered, where $\Gamma$ is the dual Steenrod algebra, but it takes a bit of work %a slightly less na\"{i}ve approach is necessary in order 
to prove this. %the most naive approaches to proving this are not effective, as the above small example indicates. 
Our relevant positive result is Proposition \ref{filteredness prop}: if $\Gamma$ is a coalgebra with property $\Cyc$ (defined in Definition \ref{def of cyc}), then the partially-ordered set $\dist(\Gamma)$ is filtered, and so $H^0_{\dist}$ is left exact. The dual Steenrod algebra at any prime does indeed have property $\Cyc$, as we show later, in Corollary \ref{comm hopf algebras with p-alg duals are cyc}.

I do not know of any specific example of a coalgebra which does not have property $\Cyc$, or whose distinguished-torsion functor $H^0_{\dist}$ fails to be left exact. Perhaps no such examples exist. %Since the purpose of this paper is to develop the theory as far as is needed for pressing topological applications, I have not investigated certain broader questions related to the ideas presented here, such as whether there are any coalgebras without property $\Cyc$. 


\begin{definition}\label{def of cyc}\leavevmode
\begin{itemize}
\item Given a homogeneous element $n\in \Gamma$, we will say that a homogeneous element $m\in \Gamma$ {\em co-annihilates $n$} if, whenever $\gamma\in\Gamma^*$ is a homogeneous element such that $\gamma m = 0$, we have $\gamma n = 0$.
\item We will say that a cyclic graded $\Gamma^*$-module $M$ is {\em self-co-annihilating} if it is generated by a homogeneous element which co-annihilates every homogeneous element of $M$.
\item We will say that the graded $A$-coalgebra $\Gamma$ {\em has property $\Cyc$} if, for each finitely generated graded $\Gamma^*$-submodule $M$ of $\iota\Gamma$, there exists a self-co-annihilating cyclic graded $\Gamma^*$-submodule of $\iota\Gamma$ containing $M$.
\end{itemize}
\end{definition}

\begin{remark}
The notions of ``co-annihilating'' and ``self-co-annihilating'' are totally unnecessary when $\Gamma^*$ is commutative, since in that case, any element of $\Gamma^*$ that annihilates a generator of a cyclic $\Gamma^*$-module also annihilates every other element of that module. See the remarks on annihilators in Conventions \ref{conventions} for how the situation is more complicated when $\Gamma^*$ is noncommutative.

Unfortunately, our main motivating examples of coalgebras $\Gamma$ in this paper are the dual Steenrod algebras, and in those examples, $\Gamma^*$ is noncommutative. So it seems we cannot dispense with the notions of ``co-annihilating,'' ``self-co-annihilating,'' etc., which play an important role in the proof of part 1 of Proposition \ref{filteredness prop}, which in turn plays a pivotal role in everything that follows.
\end{remark}

\begin{observation}\label{steenrod alg has cyc}
By Proposition \ref{comm hopf algebras with p-alg duals are cyc}, if $\Gamma$ is a co-commutative Hopf algebra over a field and its dual is a finite-type $\mathcal{P}$-algebra (in the sense of chapter 13 of Margolis' book \cite{MR738973}), then $\Gamma$ has property $\Cyc$. Consequently the dual Steenrod algebras have property $\Cyc$, as stated in Corollary \ref{comm hopf algebras with p-alg duals are cyc}.

Similarly, by Proposition \ref{cocomm hopf algebras with p-alg duals are cyc}, if $\Gamma$ is a commutative Hopf algebra over a field and its dual is a finite-type $\mathcal{P}$-algebra, then $\Gamma$ has property $\Cyc$. Consequently, for each prime $p$, the subalgebra of the mod $p$ Steenrod algebra generated by the Milnor primitives $Q_0,Q_1,Q_2,\dots$ has property $\Cyc$, and so does its dual.
\end{observation}
\end{comment}

\begin{lemma}\label{subobjs quots and exts lemma}
Let $\mathcal{C}$ be an abelian category, let $U: \mathcal{C}\rightarrow\gr\Ab$ be a faithful exact functor, let $H: \mathcal{C}\rightarrow\gr\Ab$ be a left exact functor, and let $\eta: H \rightarrow U$ be a natural transformation such that $\eta M: HM \rightarrow UM$ is monic for all objects $M$ of $\mathcal{C}$. We refer to an object $M$ of $\mathcal{C}$ as {\em $\eta$-complete} if $\eta M: HM \rightarrow UM$ is an isomorphism. Then each of the following are true:
\begin{enumerate}
\item Every subobject of an $\eta$-complete object is $\eta$-complete.
\item Every quotient object of an $\eta$-complete object is $\eta$-complete.
\item If we furthermore assume that $U$ commutes with coproducts, then every coproduct of $\eta$-complete objects is $\eta$-complete.
\item If we furthermore have that the first right-derived functor $R^1H(M)$ vanishes for all $\eta$-complete objects $M$, then every extension of an $\eta$-complete object by an $\eta$-complete object is $\eta$-complete.
\end{enumerate}
\end{lemma}
\begin{proof}\leavevmode
\begin{description}
\item[1 and 2]
Given a short exact sequence 
\[ 0 \rightarrow A^{\prime} \rightarrow A \rightarrow A^{\prime\prime}\rightarrow 0\]
in $\mathcal{C}$, we get a commutative diagram with exact rows
\begin{equation}\label{comm diag 094393}
\xymatrix{
0 \ar[r]\ar[d] 
 & HA^{\prime} \ar@{^{(}->}[d]_{\eta A^{\prime}} \ar[r] 
 & HA \ar@{^{(}->}[d]_{\eta A} \ar[r]
 & HA^{\prime\prime} \ar@{^{(}->}[d]_{\eta A^{\prime\prime}} 
 & \\
0 \ar[r] 
 & UA^{\prime} \ar[r] 
 & UA^{} \ar[r] 
 & UA^{\prime\prime} \ar[r] 
 & 0.
}\end{equation}
If $A$ is $\eta$-complete, then an easy diagram chase in \eqref{comm diag 094393} shows that $A^{\prime\prime}$ is $\eta$-complete, and an application of the Four Lemma, from elementary homological algebra, to \eqref{comm diag 094393} then yields that $A^{\prime}$ is $\eta$-complete.
\item[3] Let $\{ M_i: i\in I\}$ be a set of objects of $\mathcal{C}$. We have the commutative square of graded abelian groups
\begin{equation}\label{comm sq 043040}\xymatrix{
 \coprod_i HM_i \ar@{^{(}->}[r]_{\coprod_i \eta M_i} \ar[d] 
  & \coprod_i UM_i \ar[d] \\
 H\coprod_i M_i \ar@{^{(}->}[r]_{\eta \coprod_iM_i} 
  & U\coprod_i M_i .
}\end{equation}
The right-hand vertical map in \eqref{comm sq 043040} is an isomorphism since $U$ is assumed to commute with coproducts, while $\coprod_i \eta M_i$ is an isomorphism since each $M_i$ is $\eta$-complete. So the top composite in square \eqref{comm sq 043040} is an isomorphism, and in particular, surjective, hence $\eta \coprod_iM_i$ is surjective, hence an isomorphism, %By the two-out-of-three property for isomorphisms, the left-hand vertical map in \eqref{comm sq 043040} is an isomorphism, 
i.e., $\coprod_iM_i$ is $\eta$-complete.
\item[4] If, on the other hand, $A^{\prime}$ and $A^{\prime\prime}$ are $\eta$-complete, then the vanishing of $R^1HA^{\prime}$ together with the Five Lemma, applied to \eqref{comm diag 094393}, yields that $A$ is also $\eta$-complete.
\end{description}
\end{proof}

ggxx
\begin{prop}\label{filteredness prop}
Suppose that the $A$-coalgebra $\Gamma$ is projective. % as an $A$-module and has property $\Cyc$. 
Then the following statements are all true:
\begin{enumerate}
%\item Suppose we are given distinguished left ideals $I,J$ of $\Gamma^*$. Then there exists a distinguished left ideal $K$ of $\Gamma^*$ which is contained in both $I$ and $J$.
%\item The partially-ordered set $\dist(\Gamma)$ is filtered. ggxx
\item For each integer $n$, we have an isomorphism of functors between $H^n_{\dist}$ and $\colim_{I\in\dist(\Gamma)}\Ext^n_{\Gamma^*}(\Gamma^*/\Gamma^* I,-)$. 
\item The functor $H^0_{\dist}$ is left exact.
\item If $\Gamma$ is co-commutative and $\Gamma^*$ is Noetherian, then the functor $H^0_{\dist}: \gr\Mod(\Gamma^*)\rightarrow\gr\Mod(\Gamma^*)$ is idempotent.
\item If $M$ is a distinguished-torsion graded $\Gamma^*$-module, then every graded $\Gamma^*$-submodule of $M$ is also distinguished-torsion.
\item If $M$ is a distinguished-torsion graded $\Gamma^*$-module, then every graded $\Gamma^*$-module quotient of $M$ is also distinguished-torsion.
\item Every coproduct of distinguished-torsion graded $\Gamma^*$-modules is distinguished-torsion.
\item Suppose furthermore that $H^1_{\dist}$ vanishes on every distinguished-torsion graded $\Gamma^*$-module\footnote{See Corollary \ref{vanishing of higher dist coh on comodules} for some reasonable conditions on $\Gamma$ so that $H^1_{\dist}$ does indeed vanish on every distinguished-torsion graded $\Gamma^*$-module. These conditions are general enough to include, for example, the mod $p$ Steenrod algebra for every prime $p$.}. If $M$ is an extension of a distinguished-torsion graded $\Gamma^*$-module by a distinguished-torsion graded $\Gamma^*$-module, then $M$ is also distinguished-torsion.
\end{enumerate}
\end{prop} 
\begin{proof}\leavevmode
\begin{enumerate}
%\item Since $I$ and $J$ are distinguished, we can choose homogeneous elements $x_I$ and $x_J$ of $\Gamma$ such that the left $\Gamma^*$-submodule $\langle x_I\rangle$ of $\Gamma$ generated by $x_I$ is isomorphic to $\Gamma^*/\Gamma^* I$, and such that the left $\Gamma^*$-submodule $\langle x_J\rangle$ of $\Gamma$ generated by $x_J$ is isomorphic to $\Gamma^*/\Gamma^* J$. 
%Since $\Gamma$ has property $\Cyc$, there exists a co-annihilating cyclic graded $\Gamma^*$-submodule of $\Gamma$ containing the $\Gamma^*$-submodule sum $\langle x_I \rangle + \langle x_J\rangle \subseteq \Gamma$. (Recall that the sum of two submodules $M_1$ and $M_2$ is the intersection of all submodules containing both $M_1$ and $M_2$.)
%Since $M$ is a cyclic graded module, it has a homogeneous self-co-annihilating generator; choose such a generator $m \in M$.
%We have a containment 
%\begin{equation}\label{containment 1} \ann_{\ell}(m) \supseteq \ann_{\ell}(M)\end{equation} simply from the definition of $\ann_{\ell}$ (as in Conventions \ref{conventions}). Since $m$ is self-co-annihilating, any element of $\Gamma^*$ that annihilates an element of $M$ also annihilates $m$; that is, \eqref{containment 1} is, in fact, an equality.
%Consequently we have containments
%\begin{align}
%\nonumber \ann_{\ell}(m) 
% &= \ann_{\ell}(M)
%\label{containment 2}  &\subseteq \ann_{\ell}(\langle x_I\rangle) \cap \ann_{\ell}(\langle x_J\rangle) \\
%\nonumber &= I \cap J,
%\end{align}
%with \eqref{containment 2} due to the containments $M\supseteq \langle x_I\rangle$ and $M\supseteq \langle x_J\rangle$.
%So $\ann_{\ell}(m)$ is a homogeneous left ideal of $\Gamma^*$ contained in $I$ and $J$, and the ideal $\ann_{\ell}(m)$ has the property that $\Gamma^*/\ann_{\ell}(m)$ is isomorphic to a graded cyclic $\Gamma^*$-submodule of $\Gamma$ (namely, $M$), as desired.
%\item Follows immediately from the previous part of this proposition.
\item This is routine, but we give a proof anyway, to show that the result does not require any condition on $\underline{\hom}_{\Gamma^*}(\Gamma^*/\Gamma^* I,-)$ commuting with filtered colimits, which would require additional Noetherian-like hypotheses. 
Write $H^0_{\dist}: \gr\Mod(\Gamma^*) \rightarrow \gr\Ab$ as the composite of the functors 
\begin{align*}
 \mathcal{h}: \gr\Mod(\Gamma^*) &\rightarrow \gr\Ab^{\dist(\Gamma)}  \\
 \mathcal{h}(M)(I) &= \underline{\hom}_{\Gamma^*}\left( \Gamma^*/\Gamma^* I,M\right) ,\mbox{\ and} \\
 \colim: \gr\Ab^{\dist(\Gamma)} &\rightarrow \gr\Ab
\end{align*}
Here we are using the standard notation $\mathcal{C}^{\mathcal{B}}$ to denote the category of functors $\mathcal{B}\rightarrow\mathcal{C}$, for given categories $\mathcal{B}$ and $\mathcal{C}$.

% Let $\mathcal{A},\mathcal{B}$ be posets. We have the functor categories $\mathcal{C}^{\mathcal{A}}$ and $\mathcal{C}^{\mathcal{B}}$. We say that {\em $\mathcal{A}$ refines $\mathbb{B}$} if there exists a functor ggxx

Since we have shown that $\dist(\Gamma)$ is filtered (NOT ANYMORE!), we have that the functor $\colim$ is exact, and in particular, left-exact. Exactness of $\colim$ furthermore tells us that $\mathcal{h}$ sends injectives to $\colim$-acyclics, so the Grothendieck spectral sequence
\[ E_2^{s,t}\cong R^s\colim (R^t\mathcal{h}(M)) \Rightarrow R^{s+t}(\colim\circ \mathcal{h})(M) \cong (R^{s+t}H^0_{\dist})(M)\]
exists and furthermore collapses to the $(s=0)$-line, given by \[ E_2^{0,t}\cong \colim_{I\in\dist(\Gamma)}\left(R^t\underline{\hom}_{\Gamma}(\Gamma^*/\Gamma^* I,-)\right)(M),\] as desired.
\item The functor $\underline{\hom}_{\Gamma^*}(\Gamma^*/\Gamma^* I,-)$ is left exact, and a filtered colimit of left exact functors $\gr\Mod(\Gamma^*)\rightarrow \gr\Ab$ is left exact, since the category of graded $\Gamma^*$-modules satisfies Grothendieck's axiom AB5.
\item When $\Gamma^*$ is commutative and Noetherian, we have isomorphisms
\begin{align}
\nonumber H^0_{\dist}H^0_{\dist}(M) 
  &\cong \colim_{I\in\dist(\Gamma)}\underline{\hom}_{\Gamma^*}\left( \Gamma^*/I, \colim_{J\in\dist(\Gamma)}\underline{\hom}_{\Gamma^*}\left( \Gamma^*/J, M\right)\right) \\
\label{iso 409949}  &\cong \colim_{I\in\dist(\Gamma)}\colim_{J\in\dist(\Gamma)}\underline{\hom}_{\Gamma^*}\left( \Gamma^*/I, \underline{\hom}_{\Gamma^*}\left( \Gamma^*/J, M\right)\right) \\
\nonumber  &\cong \colim_{(I,J)\in\dist(\Gamma)\times\dist(\Gamma)}\underline{\hom}_{\Gamma^*}\left( (\Gamma^*/I) \otimes_{\Gamma^*} (\Gamma^*/J),M\right) \\
\nonumber  &\cong \colim_{(I,J)\in\dist(\Gamma)\times\dist(\Gamma)}\underline{\hom}_{\Gamma^*}\left( \Gamma^*/(I+J),M\right) \\
\nonumber  &\cong \colim_{I\in\dist(\Gamma)\times\dist(\Gamma)}\underline{\hom}_{\Gamma^*}\left( \Gamma^*/I,M\right) \\
\nonumber  &\cong H^0_{\dist}(M),
\end{align}
natural in $M$. Isomorphism \eqref{iso 409949} is the one that relies on $\Gamma^*$ being Noetherian, so that $\Gamma^*/I$ is finitely presented, and not merely finitely generated.)
\item[(6 through 9)] Consequence of Lemma \ref{subobjs quots and exts lemma}, using $H^0_{\dist}$ for $H$ and the forgetful functor $\gr\Mod(\Gamma^*)\rightarrow\gr\Ab$ for $U$.
\end{enumerate}
\end{proof}
In light of the left-exactness of $H^0_{\dist}$, we refer to the $n$th right derived functor $H^n_{\dist} = R^nH^0_{\dist}: \gr\Mod(\Gamma^*) \rightarrow \gr\Ab$ as the {\em $n$th distinguished local cohomology group}.


\begin{comment}
\section{Appendix: self-enrichment of comodule categories.}

\begin{prop}
Let $\Gamma$ be a graded co-commutative coalgebra over a graded commutative ring $A$. Then the category $\gr\Comod(\Gamma)$ of graded $\Gamma$-comodules has a self-enrichment given by equipping the graded $A$-module $\underline{\hom}_{\Comod(\Gamma)}(M,N)$ of graded $\Gamma$-comodule homomorphisms $M\rightarrow N$ with the $\Gamma$-coaction 
\begin{align*}
 \psi: \underline{\hom}_{\Comod(\Gamma)}(M,N) &\rightarrow \Gamma\otimes_A \underline{\hom}_{\Comod(\Gamma)}(M,N)
\end{align*}
.... (IN FACT THIS ISN'T NECESSARY YET!)
\end{prop}
\end{comment}


\section{Comodules as a torsion class in modules.}

\begin{theorem}\label{torsion class thm}
 Suppose that the $A$-coalgebra $\Gamma$ is projective as an $A$-module and has property $\Cyc$. 
Then the distinguished-torsion graded $\Gamma^*$-modules are a torsion class in % a hereditary torsion theory on 
the category of graded $A$-modules.
The corresponding torsion theory is stable if we also assume that $A$ is a countable field and that $\Gamma^*$ is finite-type and connected.
\end{theorem}
\begin{proof}
That the distinguished-torsion modules form a torsion class is a consequence of Proposition \ref{filteredness prop} and Definition-Proposition \ref{def-prop on torsion theories}.
Stability is a consequence of claims (1) and (3) in Theorem \ref{main thm 1}: we identify graded $\Gamma$-comodules with distinguished-torsion graded $\Gamma^*$-modules, so that the fact that $\iota$ preserves injectives gives us that each distinguished-torsion graded $\Gamma^*$-module embeds in an injective graded $\Gamma^*$-module which is distinguished-torsion, so in particular, the injective hull of that module is also distinguised-torsion.
\end{proof}

Theorem \ref{torsion class thm} together with Proposition \ref{dickson thm on stability} yield the following corollary:
\begin{corollary}\label{vanishing of higher dist coh on comodules}
Suppose that $A$ is a countable field, and that $\Gamma^*$ is finite-type and connected and has property $\Cyc$.
Then, for all graded $\Gamma$-comodules $M$ and all positive integers $n$, we have $H^n_{\dist}(\iota(M)) \cong 0$.
\end{corollary}



\subsection{Duals of commutative Hopf $\mathcal{P}$-algebras and co-commutative Hopf $\mathcal{P}$-algebras have property $\Cyc$.}

In Definition \ref{def of strong p-alg} we introduce {\em strong $\mathcal{P}$-algebras}, which are $\mathcal{P}$-algebras satisfying an additional axiom. It is easy to see that a graded exterior algebra on an infinite sequence of generators, only finitely many in each degree, is a strong $\mathcal{P}$-algebra. A much more nontrivial example is given by the Steenrod algebra, at every prime: Proposition 7 from section 15.1 of \cite{MR738973} establishes that the Steenrod algebras are $\mathcal{P}$-algebras, while Theorem A from \cite{MR793186} establishes that the Steenrod algebras also satisfy the extra condition in the definition of a {\em strong} $\mathcal{P}$-algebra.

The notion of a strong $\mathcal{P}$-algebra is used only in the proof of Lemma \ref{partial dual orientations lemma}, but that lemma plays an essential role in our argument (in Proposition \ref{comm hopf algebras with p-alg duals are cyc}) that the Steenrod algebras have property $\Cyc$. I do not know any example of a $\mathcal{P}$-algebra which fails to be strong. Perhaps it is possible to show that all $\mathcal{P}$-algebras are strong, or to show that one can prove Lemma \ref{partial dual orientations lemma} without strongness of a $\mathcal{P}$-algebra; either would let us do away with the idea of strongness of a $\mathcal{P}$-algebra entirely.
\begin{definition}\label{def of strong p-alg}
A {\em strong $\mathcal{P}$-algebra} is a union $B$ of a sequence of subalgebras $B(0) \subsetneq B(1) \subsetneq \dots$ such that:
\begin{itemize}
\item each $B(n)$ is a Poincar\'{e} algebra,
\item each $B(n+1)$ is flat over $B(n)$,
\item and each $B(n)$ admits a compatible self-dual\footnote{Since $B(n)$ is a Poincar\'{e} algebra, we have an isomorphism of graded $B(n)$-modules $f: \Sigma^d \underline{\hom}_{B(n)}(B(n),A) \stackrel{\cong}{\longrightarrow} B(n)$, where $d$ is the top nontrivial degree in $B(n)$. Saying that a given graded $B$-module structure on $B(n)$ is {\em self-dual} is saying that the isomorphism $f$ extends to an isomorphism of graded $B$-modules $\Sigma^d \underline{\hom}_{B}(B(n),A) \stackrel{\cong}{\longrightarrow} B(n)$. {\em Compatibility} of the self-dual $B$-module structures is the condition that $B \cong \lim_n B(n)$ as graded $B$-modules. See \cite{MR793186} for this notion of compatible self-duality.} graded $B$-module structure extending its $B(n)$-module structure.
\end{itemize}
\end{definition}

\begin{definition}
Suppose that $A$ is a field and the dual algebra %$\Gamma$ is a graded Hopf algebra over $A$ whose dual 
$\Gamma^*$ of $\Gamma$ is a $\mathcal{P}$-algebra. We will say that a homogeneous element $x\in \Gamma$ is {\em a partial dual orientation class} if there exists a 
sequence of subalgebras \begin{equation}\label{seq 04040} B(0) \subsetneq B(1) \subsetneq \dots\end{equation} of $\Gamma^*$, as in Definition \ref{def of p-alg}, and an integer $i$ such that the image of $x$ under the canonical surjection $\Gamma\rightarrow B(i)^*$ is dual to the top degree class\footnote{Since $B(i)$ is a Poincar\'{e} algebra, in its top degree it is a one-dimensional vector space, so when we write ``the top degree class in $B(i)$,'' the element so described is well-defined up to scalar multiplication.} in $B(i)$. 
\end{definition}


\begin{lemma}\label{partial dual orientations lemma}
Suppose that $A$ is a field and the dual algebra %$\Gamma$ is a graded Hopf algebra over $A$ whose dual 
$\Gamma^*$ of $\Gamma$ is a strong $\mathcal{P}$-algebra. Let $j$ be an integer. Then there exists some partial dual orientation class $x\in \Gamma$ such that:
\begin{itemize}
\item the graded $\Gamma^*$-submodule $\langle x\rangle$ of $\iota\Gamma$ generated by $x$ contains every homogeneous element of $\iota\Gamma$ of degree $\geq j$, and
\item $x$ co-annihilates every homogeneous element of $\langle x\rangle$. (gggggxxxxx THIS ISN'T TRUE, AND THIS APPROACH WON'T WORK AS STATED!)
\end{lemma}
\begin{proof}
Since $\iota\Gamma$ is trivial in positive degrees, the case of $j$ negative is the important case.
Since the union of the sequence \eqref{seq 04040} is $\Gamma^*$ and since $\Gamma^*$ is finite-type, there exists some integer $h$ such that every homogeneous element of $\Gamma^*$ of degree $\leq -j$ is contained in $B(h)$.
Write $d$ for the top nonvanishing degree of $B(h)$, and choose any nonzero homogeneous element $\alpha$ in degree $d$ in the $A$-linear dual $B(h)^*$ of $B(h)$.
Since $B(h)$ is a Poincar\'{e} algebra, $B(h)$ is a one-dimensional $A$-vector space in its top nonvanishing degree, and so $\alpha$ is unique up to $A$-linear multiplication.ggxx

The compatibility condition in Definition \ref{def of strong p-alg} guarantees that we have an isomorphism $\colim_h B(h)^* \rightarrow \iota\Gamma$ of graded $\Gamma^*$-modules. Write $\overline{\alpha}$ for the image of $\alpha\in B(h)^*$ under the resulting composite
\[ B(h)^* \rightarrow \colim_h B(h)^* \stackrel{\cong}{\longrightarrow} \iota\Gamma.\]
The graded $\Gamma^*$-submodule $\langle \overline{\alpha}\rangle$ of $\iota\Gamma$ generated by $\overline{\alpha}$ is the image, in $\iota\Gamma$, of the graded $\Gamma^*$-submodule of $B(h)^*$ generated by $\alpha$. This last graded $\Gamma^*$-submodule is simply $B(h)^*$ itself, by self-duality. So every element of $\iota\Gamma$ of degree $\geq j$ is in $\langle\overline{\alpha}\rangle$. The composite $B(h)^*\rightarrow \iota\Gamma \rightarrow B(h)^*$ is the identity on $B(h)^*$, so $\overline{\alpha}$ is a partial dual orientation class, as desired.

We still must show that $\overline{\alpha}$ co-annihilates every homogeneous element of $\langle \overline{\alpha}\rangle$. Suppose that $m$ is a homogeneous element of $\langle \overline{\alpha}\rangle$, i.e., $m = \tilde{m} \overline{\alpha}$ for some homogeneous $\tilde{m}\in \Gamma^*$, and suppose that $\gamma \overline{\alpha} = 0$. By self-duality, we have the isomorphism of $\Gamma^*$-modules $\langle \overline{\alpha}\rangle \cong B(h)$, so $\gamma\overline{\alpha} = 0$ implies that $\gamma\cdot 1 = 0$ in the $\Gamma^*$-module structure on $B(h)$ given by Definition \ref{def of strong p-alg}.
% So here's what we need to know: in Mitchell's $\Gamma^*$-module structure on $B(h)$,
%  if $\gamma\in \Gamma^*$ and $\gamma\cdot 1 = 0$ in $B(h)$,
%  is $\gamma\cdot x = 0$ for all $x\in B(h)$?
%  Given a ring $R$, say that a graded $R$-module $M$ has property F if $M$ has a
%   finite set $S$ of homogeneous generators such that, whenever $\gamma s = 0$ for all $s\in S$,
%   we have $\gamma m = 0$ for all $m\in M$.
%  Suppose $f: M \rightarrow N$ is a surjective graded $R$-module homomorphism, and suppose that
%   $M$ has property F. Does $N$ also have property F? Not necessarily!
%   Let $f(S)$ denote the image of $S$ under $f$. Then $f(S)$ is a set of homogeneous
%    $R$-module generators for $N$. Suppose that $\gamma f(s) = 0$ for all $s\in S$.
%    Then $f(\gamma s) = 0$ for all $s\in S$, ....gggggxxxxx
\end{proof}

\begin{prop}\label{comm hopf algebras with p-alg duals are cyc}
Suppose that $A$ is a field and that the $A$-coalgebra $\Gamma$ has the structure of a commutative Hopf algebra such that $\Gamma^*$ is a finite-type $\mathcal{P}$-algebra. Then $\Gamma$ has property $\Cyc$.
\end{prop}
\begin{proof}
Let $M$ be a finitely generated graded $\Gamma^*$-submodule of $\iota\Gamma$. By Lemma \ref{partial dual orientations lemma}, there exists a sequence of subalgebras $B(0) \subsetneq B(1) \subsetneq \dots$ as in Definition \ref{def of p-alg} and a partial dual orientation class $x\in \Gamma$ such that the graded $\Gamma^*$-submodule $\langle x\rangle$ of $\iota\Gamma$ generated by $x$ contains every homogeneous element of $\iota\Gamma$ of degree $\geq j$. We need to know that $x$ co-annihilates every homogeneous element of $\langle x \rangle$. To that end, suppose that $\gamma$ is a homogeneous element of $\Gamma^*$ such that $\gamma x = 0$, and suppose that $n$ is a homogeneous element of $\langle x\rangle$. 
Then $n = gx$ for some homogeneous $g\in \Gamma^*$. Writing $\nabla$ for the multiplication map $\nabla: \Gamma^* \otimes_A \Gamma^*\rightarrow\Gamma^*$ and $\tau$ for the factor swap map $\tau: \Gamma^* \otimes_A \Gamma^*\rightarrow \Gamma^* \otimes_A \Gamma^*$, we have
\begin{align}
\label{cocomm eq 0}  \gamma n
 &= \gamma gx \\
\nonumber  &= \nabla(\gamma \otimes g)(\Delta(x)) \\
\label{cocomm eq 1}  &= (\nabla)(\gamma \otimes g)(\tau\circ \Delta(x)) \\
\nonumber  &= (\nabla\circ \tau )(g\otimes \gamma)(\Delta(x)) \\ ggggxxxxx FIX THIS ISSUE! LOOK AT MILNOR IDENTITY AT BOTTOM OF PG 156
  &= (\nabla)(g \otimes \gamma)(\Delta(x)) \\
\label{cocomm eq 2}  &= g\gamma x
\nonumber  &= 0,
\end{align}
where \eqref{cocomm eq 1} is due to the co-commutativity of $\Gamma^*$. So $\gamma n=0$, as desired, so $\langle x\rangle$ is self-co-annihilating, as desired.
\end{proof}

\begin{corollary}\label{comm hopf algebras with p-alg duals are cyc}
For every prime $p$, the mod $p$ dual Steenrod algebra has property $\Cyc$.
\end{corollary}


\begin{prop}\label{cocomm hopf algebras with p-alg duals are cyc}
Suppose that $A$ is a field and that the $A$-coalgebra $\Gamma$ has the structure of a co-commutative Hopf algebra such that $\Gamma^*$ is a finite-type $\mathcal{P}$-algebra. Then $\Gamma$ has property $\Cyc$.
\end{prop}
\begin{proof}
Same argument as in Proposition \ref{comm hopf algebras with p-alg duals are cyc}, but the equalities between \eqref{cocomm eq 0} and \eqref{cocomm eq 2} are unnecessary, since we have $\gamma gx = g\gamma x = 0$ by the commutativity of $\Gamma^*$.
\end{proof}

\begin{corollary}\label{cocomm hopf algebras with p-alg duals are cyc}
For every prime $p$, the subalgebra of the mod $p$ Steenrod algebra generated by Milnor's primitives $Q_0,Q_1,\dots$ has property $\Cyc$.
\end{corollary}




\bibliography{/home/asalch/texmf/tex/salch}{}
\bibliographystyle{plain}
\end{document}

\begin{prop}
Suppose that $k$ is a field, and suppose that $\Gamma$ is a $k$-coalgebra such that the graded ring $\Gamma^*$ is connected, that is, $\Gamma^*$ is concentrated in nonnegative degrees, and isomorphic to $k$ in degree zero\footnote{Equivalently, by our grading conventions, $\Gamma$ is concentrated in nonpositive grading degrees, and its degree zero summand is a one-dimensional $k$-vector space.}. Then the following claims are each true:
\begin{enumerate}
\item The double-dual $\Gamma^{**}$ is injective as a graded right $\Gamma^*$-module.
\item If $\Gamma$ is finite-type\footnote{That is, for each integer $n$, the degree $n$ summand of $\Gamma$ is a finite-dimensional $k$-vector space.}, then $\Gamma$ is injective as a graded $\Gamma^*$-module.
\item Every bounded-above graded $\Gamma^*$-module is rational.
\end{enumerate}
\end{prop}
\begin{proof}\leavevmode
\begin{enumerate}
\item It is standard and straightforward to show that, if $A$ is a graded $k$-algebra concentrated in nonnegative degrees, then the dual graded $A$-module $A^*$ is injective; see Proposition 12 in section 11.3 of \cite{MR738973}, for example. In this case we simply let $A = \Gamma^*$.
\item When $\Gamma$ is finite-type, the canonical map $\Gamma \rightarrow \Gamma^{**}$ is an isomorphism of graded $\Gamma^*$-modules, so this part of the proposition follows from the previous part of this proposition.
% gggxxx Is it true that \Gamma^* = (\iota\Gamma)^* as graded \Gamma^*-modules?
\item First, a lemma: let $M$ be a bounded (i.e., bounded both above and below) graded $\Gamma^*$-module, and let $I$ be an injective graded $\Gamma^*$-module which contains gggggggxxxxxxxx
% Suppose we embed \Gamma^*/J into a finite direct sum of suspensions of copies of \Gamma:
% f: \Gamma^*/J \rightarrow \coprod_{i=1}^j \Sigma^{e(i)} \Gamma
% The composite % So we have a finite set J_1, ..., J_j of 
%  dist. right ideals of \Gamma^*, and 
%  J_1 ggxx

%  \Gamma^* ->  \Gamma^*/J
%           ->  \coprod_{i=1}^j \Sigma^{e(i)} \Gamma
%           ->> \Sigma^{e(i)} \Gamma.
% has some kernel J_i containing J, and the image of 1\in \Gamma^* is some element t_i of \Sigma^{e(i)}\Gamma. 
% The intersection of J_1,...,J_n seems to be J?
% So, by property \Cyc, there exists a cyclic graded \Gamma^*-submodule of \Gamma containing all the t_i.
% Write t for a generator of that cyclic submodule.
% The kernel of the map \Gamma^* -> \Gamma sending 1 to t
%  is a dist. ideal, and is contained in each of the J_i, hence contained in the
%  intersection of J_1, ..., J_n, hence contained in J.
%  So J contains a dist. ideal.ggxx
. Let $\underline{m}$ denote the ``irrelevant ideal'' of $\Gamma^*$, that is, the ideal of $\Gamma^*$ generated by all homogeneous elements of positive degree. For each nonnegative integer $n$, write $M[\underline{m}^n]$ for the graded sub-$\Gamma^*$-module of $M$ generated by the homogeneous $\underline{m}^n$-torsion elements of $M$. In other words, $M[\underline{m}^n]$ is simply $\underline{\hom}_{\Gamma^*}(\Gamma^*/\underline{m}^n,M)$. 

Since $M$ is assumed bounded, there exists some $n$ such that $\mathfrak{m}^{n+1}M = 0$.
Consequently $\mathfrak{m}^nM$ is a $\Gamma^*/\underline{m}$-module. Since $\Gamma^*$ is connected, $\Gamma^*/\underline{m}\cong k$, ggxx

gggxxx
By contrapositive: 
let $M$ be a bounded-above graded right $\Gamma^*$-module, and suppose that $M$ is not distinguished-torsion. Then there exists a homogeneous element $m\in M$ such that $Im \neq 0$ for all distinguished right ideals $I$ of $\Gamma^*$.
% Write $n$ for an integer such that $M$ vanishes in degree $i$ for all $i>n$, and write $j$ for the degree of $m$.
% Then $gm=0$ for all homogeneous $g\in\Gamma^*$ of degree $>n-j$, so each distinguished right ideal $I$ of $\Gamma^*$ must contain some nonzero homogeneous element of degree $\leq n-j$. 
% Write $I^{>n-j}$ for the right ideal of $\Gamma^*$ generated by all homogeneous elements of degree $>n-j$. Then $Im = 0$, so there cannot be a distinguished right ideal of $\Gamma^*$ which is contained in $I^{>n-j}$.


% Suppose that $I$ is a particular distinguished right ideal of $\Gamma^*$, and let $i$ be a nonzero homogeneous element of $I$ of degree $\leq n-j$. Suppose that $\gamma\in \Gamma$ is a homogeneous element whose annihilator ideal $\Ann(\gamma)$ is equal to $I$. (Such an element $\gamma$ exists by the definition of a distinguished ideal.) Then $i\gamma =0$ggxx

% We have the evaluation pairing \Gamma^* \times \Gamma -> k.
% Choose a homogeneous basis B for \Gamma^{\geq n-j}, and write B^* for the dual homogeneous basis for (\Gamma^*)^{\leq n-j}.
% Each element b\in B generates a cyclic graded $\Gamma^*$-submodule <b> of $\Gamma$.
% The annihilator of $b$ is a distinguished right ideal of $\Gamma^*$, hence must contain some nonzero homogeneous element of degree $\leq n-j$. Write $e_b$ for such an element.
% So we have: e_b b = e_b(b) = 0,
%        and  b^* b = 1, so:
%             e_b ((e_b)^*) = 1, and
%             e_{f^*}(f^*) = ggxxx
% If B has at least two elements, b1 and b2:
%             e_{b1} (e_{b1})^*
\end{enumerate}
\end{proof}


%\item Suppose $I,J$ are homogeneous right ideals of $\Gamma^*$ with $I\subseteq J$. If $I$ is distinguished, then so is $J$. (THIS IS PROBABLY TRUE UNDER APPROPRIATE HYPOTHESES.)
%
% ---
%
% When Gamma = k{t_0, t_1, ...} and Gamma^* = k[x],
%  the reason that xt_1 = t_0 is that (x \tensor t_0)\Delta(t_1) = 1.
% Similarly, (x \tensor t_n)\Delta(t_{n+1}) = 1.
% For general Gamma, if t,t' are in Gamma and s is in Gamma^*, 
%  then st = t' iff (s\tensor t')\Delta(t) = 1.
%  BETTER WAY TO PUT THAT: st is the element of Gamma whose value (st)s' on 
%   an element s' of Gamma^* is (s tensor s') Delta(t).
%  Is that correct?
% So suppose that the ideal (s) is distinguished, 
%  and t\in Gamma generates a copy of Gamma^*/(s) in Gamma.
%  Let s' be an element of Gamma^*.
%  We want to know if we can find an element t' of Gamma 
%   so that s' t' = t, since then perhaps (ss') is distinguished.
%  s' t' = t is equivalent to (s' \tensor t)\Delta(t') = 1.
%  How to produce t'? 
%  
% ---
%
% In A^*:
%  Sq^n 1 = 0 for all n>0.
%  So A^* has a sub-A-module iso to A/(Sq^1,Sq^2,Sq^3,...)
%      gen. by 1.
%
%  \Delta(xi_1) = xi_1 \tensor 1 + 1 \tensor xi_1,
%  so  Sq^1 xi_1 = 1,
%  and Sq^n xi_1 = 0 for all n>0.
%  So A^* has a sub-A-module iso to A/(Sq^2,Sq^3,...)
%      gen. by xi_1.
%  
%  \Delta(xi_1^2) = xi_1^2 \tensor 1 + 1 \tensor xi_1^2,
%  so  Sq^2 (xi_1^2) = 1,
%  and Sq^n xi_1 = 0 for all n \neq 2.
%  So A^* has a sub-A-module iso to A/(Sq^1,Sq^3,...)
%      gen. by (xi_1)^2.
%  
%  In general: 
%      Sq^{2^n} (xi_1^{2^n}) = 1,
%  and Sq^{2^m} xi_1^{2^n} = 0 for all m \neq n.
%  So A^* has a sub-A-module iso to A/(every Sq^{2^m} except Sq^{2^n})
%      gen. by (xi_1)^{2^{n-1}}.
%  
%  \Delta(xi_2) = xi_2 \tensor 1 + xi_1^2 \tensor xi_1 + 1 \tensor xi_2,
%  so Sq^2 xi_2 = xi_1,
%     (Sq^2 Sq^1 + Sq^1 Sq^2) xi_2 = 1,
%  and consequently Sq^1 xi_2 = 0.
%  
%  ...



The following theorem appears as Corollary 3.3 in \cite{MR3565424}, and---restricted to the category of $R$-modules for a ring $R$---as Proposition 60.22 in \cite{MR880019}.
\begin{theorem}\label{stable exact tors theories are quasistable}
If $(\mathcal{T},\mathcal{F})$ is a stable exact %hereditary
 torsion theory on a Grothendieck category $\mathcal{C}$, then the inclusion $\mathcal{T}\hookrightarrow\mathcal{C}$ has a right adjoint, and the composite $\Gamma: \mathcal{C}\rightarrow\mathcal{T}\rightarrow\mathcal{C}$ is left exact, with right derived functors $R^n\Gamma: \mathcal{C}\rightarrow\mathcal{C}$ vanishing for all $n>1$.
\end{theorem}

The following theorem appears as Proposition 26.3 in \cite{MR880019}:
\begin{theorem}
If $R$ is a ring and $(\mathcal{T},\mathcal{F})$ is a stable %hereditary
torsion theory on the category of $R$-modules, then its associated localization functor $L$ is given by $L(M) \cong \colim_{I} \hom_R(I,M)$, where the colimit is taken over all right ideals $I$ of $R$ such that $R/RI$ is in $\mathcal{T}$.
\end{theorem}

% If $\Gamma^*$ is self-injective and a $P$-algebra, is $(\mathcal{T},\mathcal{F})$ stable?
% Let $M$ be a distinguished-torsion $\Gamma^*$-module. 
% Let $E(M)$ be an injective hull for $M$.
% Old construction of $E(M)$: let E(M) = \prod_{f\in \hom_R(M,\hom_Z(R,Q/Z))} \hom_Z(R,Q/Z),
%  e(M): M  -> E(M)
%        m |-> \prod_f f(m).
% But that's not a hull!

The above results offer the conclusions which we claim hold for comodules over the dual Steenrod algebra, but their hypotheses are not quite right for our setting, since the torsion class of distinguished-torsion modules over the Steenrod algebra is not stable (see gx INSERT INTERNAL REF). A weakening of the stability condition was considered in \cite{MR3849888} and in \cite{MR4026630}: there, a torsion theory $(\mathcal{T},\mathcal{F})$, on the category of modules over a ring $R$, is called {\em weakly stable} if, for every injective $R$-module $E$, the $R$-module $\Gamma_{\mathcal{T}}(E)$ is $\Gamma_{\mathcal{T}}$-acyclic. Here $\Gamma_{\mathcal{T}}:\mathcal{C}\rightarrow\mathcal{C}$ is the $\mathcal{T}$-torsion functor given by the composite of the inclusion $\mathcal{T}\hookrightarrow\mathcal{C}$ with its right adjoint $\mathcal{C}\rightarrow\mathcal{T}$. In Lemma 3.12 of \cite{MR4026630}, it is proven that, if the $n$th right-derived functor $R^n\Gamma_{\mathcal{T}}$ vanishes for $n>1$ for a given torsion theory $(\mathcal{T},\mathcal{F})$, then $(\mathcal{T},\mathcal{F})$ is weakly stable. It follows from (gx INSERT INTERNAL REF) that the distinguished-torsion modules over the Steenrod algebra form a weakly stable torsion class. However, it seems that there is no easy way to use known results on weakly stable torsion classes---or any other weakening of the stability condition on torsion classes---to prove the results of this paper on vanishing of $H^n_{\dist}$ for $n>0$ for modules over the Steenrod algebra.

Finally, here is a highly convenient result of Dickson's, a special case of Theorem 2 from \cite{MR191935} (also see Example 60.1 of \cite{MR880019} for a statement of the result in terms close to what we state here):
\begin{theorem}\label{dickson acyclicity}
Let $R$ be a ring, let $(\mathcal{T},\mathcal{F})$ be a torsion theory on the category of $R$-modules, and let $M$ be an object of $\mathcal{T}$. Then $R^n\Gamma_{\mathcal{T}}(M)$ vanishes for all $n>0$.
\end{theorem}



\section{Distinguished local cohomology and derived functors of products in comodule categories.}


\begin{definition-proposition}
Suppose that $\Gamma$ is projective as an $A$-module and has property $\Cyc$. By the {\em ideal transform of $\Gamma$} we mean the left exact functor $D: \gr\Mod(\Gamma^*)\rightarrow\gr\Mod(\Gamma^*)$ which sends a graded $\Gamma^*$-module $M$ to $\colim_{I\in \dist(\Gamma)}\underline{\hom}_{\Gamma^*}(I,M)$. 
\end{definition-proposition}
\begin{proof}
The claim that $D$ is left exact is proven by the same argument as used to prove left exactness of $H^0_{\dist}$ in Proposition \ref{filteredness prop}.
\end{proof}
The term ``ideal transform,'' and the notation $D$, appears in \cite{MR3014449}, in the context of classical local cohomology (as opposed to the local cohomology associated to a general torsion theory, of which $H^*_{\dist}$ is a special case). This is not the same thing as the localization functor associated to a torsion theory, denoted $Q$ in \cite{MR880019}: in our setting, the localization functor $Q$ would instead be $\colim_{I\in \dist(\Gamma)}\underline{\hom}_{\Gamma^*}\left(I,M/H^0_{\dist}(M)\right)$. Golan shows in \cite{MR880019} that $Q$ and $D$ coincide for a stable torsion theory, but in our setting we cannot rely on the torsion theory whose torsion class is the distinguished-torsion modules being a stable torsion theory, as shown in Proposition \ref{instability}. 

\begin{prop}
Suppose that $\Gamma$ is projective as an $A$-module and has property $\Cyc$. Then $D$ is left exact, and 
for any graded $\Gamma^*$-module $M$ we have an exact sequence 
\begin{equation}\label{exact seq 16} 0 \rightarrow H^0_{\dist}(M) \rightarrow M \rightarrow D(M) \rightarrow H^1_{\dist}(M) \rightarrow 0,\end{equation}
and an isomorphism 
\begin{equation}\label{iso 17} H^n_{\dist}(M) \cong R^{n-1}D(M)\end{equation} for all $n\geq 2$. Both \eqref{exact seq 16} and \eqref{iso 17} are natural in the variable $M$.
\end{prop}
\begin{proof}
Straightforward consequence of choosing an injective resolution $I^{\bullet}$ for $M$, applying $\underline{\hom}_{\Gamma^*}(-,I^{\bullet})$ to the short exact sequence $0 \rightarrow I \rightarrow \Gamma^*\rightarrow \Gamma^*/I\rightarrow 0$, and taking the (filtered) colimit over all $I\in \dist(\Gamma)$.
\end{proof}

\begin{corollary}
Suppose that $\Gamma$ is projective as an $A$-module and has property $\Cyc$. Let $\mathcal{T}$ be the torsion class consisting of the distinguished-torsion graded $\Gamma^*$-modules. Then the torsion theory $(\mathcal{T},\mathcal{F})$ is quasistable if and only if the ideal transform $D: \gr\Mod(\Gamma^*)\rightarrow \gr\Mod(\Gamma^*)$ is exact.
\end{corollary}


\begin{comment}
\section{Dominant dimension relative to a torsion theory.}


\begin{definition}
Let $(\mathcal{T},\mathcal{F})$ be a torsion theory on a Grothendieck abelian category $\mathcal{C}$. Let $n$ be a nonnegative integer, and let $M$ be an object of $\mathcal{C}$. We say that {\em $M$ has $(\mathcal{T},\mathcal{F})$-dominant dimension $\geq n$}
if $M$ admits an injective resolution
\[ 0 \rightarrow M \rightarrow E^0 \rightarrow E^1 \rightarrow E^2 \rightarrow \dots\]
such that $E^0, E^1, \dots ,E^{n-1}$ are each in $\mathcal{F}$. 

Naturally, we then say that {\em $M$ has $(\mathcal{T},\mathcal{F})$-dominant dimension $n$} if $M$ has $(\mathcal{T},\mathcal{F})$-dominant dimension $\geq n$ but $M$ fails to have $(\mathcal{T},\mathcal{F})$-dominant dimension $\geq n+1$; and we say that {\em $M$ has $(\mathcal{T},\mathcal{F})$-dominant dimension $\infty$} if $M$ has $M$ has $(\mathcal{T},\mathcal{F})$-dominant dimension $\geq n$ for all positive integers $n$.
\end{definition}

The following result is Theorem 60.10 from Golan's book \cite{MR880019}:
\begin{theorem} Let $R$ be a ring, and 
let $(\mathcal{T},\mathcal{F})$ be a torsion theory on the category of right $R$-modules. Let $n$ be a nonnegative integer, and let $M$ be a right $R$-module. Then the following conditions are equivalent:
\begin{itemize}
\item The $(\mathcal{T},\mathcal{F})$-dominant dimension of $M$ is $\geq n$.
\item The derived functors $R^iT(M)$ of the torsion functor $T: \Mod(R)\rightarrow\mathcal{T}\rightarrow\Mod(R)$ vanish for all $0\leq i<n$.
\item $\Ext^i_R(R/I,M)$ is trivial for all $0\leq i<n$ and all right ideals $I$ of $R$ such that $R/I$ is in $\mathcal{T}$.
\end{itemize}
\end{theorem}

% Suppose that M has $\tau$-dominant dimension $\geq 2$.
% Then $\Ext^i_{\Gamma^*}(\Gamma^*/I,M)$ is trivial for $i=0,1$ and for all 
%  distinguished right ideals $I$ of $\Gamma^*$.
% Then does $M$ have infinite $\tau$-dominant dimension?
%  gggxxx
%
% Also: if $\Ext^i_{\Gamma^*}(\Gamma^*/I,M)$ is trivial for $I = (Sq^1,Sq^2,...)$ and for all $0\leq i < n$,
%  then is $\Ext^i_{\Gamma^*}(\Gamma^*/I,M)$ trivial for all distinguished right ideals $I$ of $\Gamma^*$ and all $0\leq i < n$?
%
% Let I_m = (Sq^{2^m},Sq^{2^{m+1}},...).
%  We have a SES:
%   0 -> I_{m-1}/I_m -> A/I_m -> A/I_{m-1} -> 0.
%   If Ext^i_A(A/I_{m-1},M) is trivial for all 0\leq i < n:
%   
%
%   0 -> I_0/I_1   -> A/I_1       -> A/I_0 -> 0
%   0 -> F_2{Sq^1} -> F_2{1,Sq^1} ->   F_2 -> 0
%   0 -> I_1/I_2   -> A/I_2       -> A/I_1 -> 0
%   0 -> F_2{Sq^2,Sq^2Sq^1,Sq^1Sq^2,Sq^2Sq^2 = Sq^1Sq^2Sq^1} -> A(1) -> A(0) -> 0
%  Alternatively:
%   0 -> A(m-1) -> A(m) -> DE(m) -> 0
%   If Ext^i_A(A(m-2),M) is trivial for all 0\leq i < n:
%  ggggxxxx

\end{comment}

\begin{comment} (GOOD BUT NOT CURRENTLY USED.)

\section{Review of Buchweitz's approach to maximal Cohen-Macaulay modules.}

Let $R$ be a left and right Noetherian (but not necessarily commutative) ring such that $R$, as a left $R$-module, admits a finite injective resolution. Such rings are more general than ordinary Gorenstein local rings (which are assumed to be commutative), but they are what Buchweitz calls ``Gorenstein rings'' in the extremely influential preprint \cite{buchweitz1987maximal}. For that reason, we make the following definition:
\begin{definition}\label{gorenstein in buchweitz sense}
A (not necessarily commutative) ring $R$ is {\em Gorenstein in the sense of Buchweitz} if $R$ is left and right Noetherian, and $R$ is of finite injective resolution both as a left module and as a right module over itself.
\end{definition}

Buchweitz then defines a maximal Cohen-Macaulay module as follows:
\begin{definition}
Suppose $R$ is a ring $R$ which is Gorenstein in the sense of Buchweitz.
\begin{itemize}
\item A finitely generated left $R$-module $M$ is said to be {\em maximal Cohen-Macaulay} if $\Ext^n_R(M,R)$ vanishes for all $n>0$.
\item The {\em stable category of maximal Cohen-Macaulay left $R$-modules} $\Ho(MCM(R))$ is just as in stable module theory: that is, $\Ho(MCM(R))$ has the same objects as the full subcategory of $\Mod(R)$ generated by the maximal Cohen-Macaulay modules, and the set of morphisms $M \rightarrow N$ in $\Ho(MCM(R))$ is the set of equivalence classes of left $R$-module morphisms $M \rightarrow N$ under the equivalence relation in which we let $f\sim g$ if and only if $f-g$ factors through a projective left $R$-module.
\end{itemize}
\end{definition}

%APC(R) = the homotopy category of unbounded acyclic complexes of finitely generated projective R-modules
%\Omega_0(C^{\bullet}) = cokernel of d: C^{-1} -> C^0
%

The following is the main result, Theorem 4.4.1, from Buchweitz's highly influential paper \cite{buchweitz1987maximal}:
\begin{theorem} {\bf (Buchweitz.)}
Suppose $R$ is a ring $R$ which is Gorenstein in the sense of Buchweitz.
\begin{enumerate}
\item Let $\Ho(APC(R))$ be the chain-homotopy category of unbounded acyclic complexes of finitely generated projective $R$-modules. Then the functor $\Omega^0:\Ho(APC(R)) \rightarrow \Ho(MCM(R))$, given on the chain-homotopy equivalence class of an unbounded complex $C^{\bullet}$ by letting $\Omega^0\left( [C^{\bullet}]\right)$ be the stable equivalence class of $\coker d: C^{-1}\rightarrow C^0$, is well-defined, and is furthermore an equivalence of categories. Consequently $\Ho(MCM(R))$ is, like $\Ho(APC(R))$, triangulated.
\item Let $D^b(R)$ be the category of bounded chain complexes of {\em finitely-generated} $R$-modules modulo quasi-isomorphisms, and let $D^b_{perf}(R)$ be its subcategory generated by the perfect complexes. Then we have the triangulated category given by the Verdier quotient $D^b(R)/D^b_{perf}(R)$, and there is a well-defined functor $\iota_R: \Ho(MCM(R)) \rightarrow D^b(R)/D^b_{perf}(R)$ given by letting $\iota_R(M)$ be the equivalence class of the chain complex consisting of $M$ concentrated in homological degree zero. This functor $\iota_R$ is an equivalence of categories.
\item The triangulated structure on $\Ho(MCM(R))$ induced by the equivalence $\Ho(APC(R)) \simeq  \Ho(MCM(R))$ agrees with the triangulated structure on $\Ho(MCM(R))$ induced by the equivalence $D^b(R)/D^b_{perf}(R) \simeq  \Ho(MCM(R))$.
\end{enumerate}
\end{theorem}

We recall one more fundamental result in this area, parts (a) and (b) of Theorem A from \cite{MR1044344}:
\begin{theorem}\label{auslander-buchweitz thm} {\bf (Auslander-Buchweitz.)}
Let $R$ be a commutative, local Noetherian ring which admits a dualizing module\footnote{For example, $R$ could be a Gorenstein ring in the classical sense, but since $R$ is assumed commutative, the assumptions of Theorem \ref{auslander-buchweitz thm} are more restrictive than assuming that $R$ is Gorenstein in the sense of Buchweitz. As far as I know, it is still an open conjecture that the commutative local Noetherian rings which admit dualizing modules are precisely the Cohen-Macaulay quotients of Gorenstein rings.}. Then, for any finitely generated $R$-module $M$, there exists a short exact sequence of finitely generated $R$-modules
\[ 0 \rightarrow M^{\prime}\rightarrow M \rightarrow M^{\prime\prime} \rightarrow 0\]
such that $M^{\prime}$ is MCM and such that $M^{\prime\prime}$ has finite injective dimension.
\end{theorem}


\section{Gorenstein-Margolis rings.}

\subsection{Basic definitions.}

A ring is called {\em quasi-Frobenius} if its injective modules coincide with its projective modules; equivalently, when its projective modules have injective dimension $0$. In the literature one finds a wealth of generalizations of the condition of being quasi-Frobenius: the paper (gx INSERT CITATION TO THRALL ``SOME GENERALIZATIONS...'') seems to have begun that line of investigation, while (gx INSERT CITATION TO COLBY-RUTTER ``GENERALIZATIONS...'') introduced and studied a further range of generalizations. (It would be a serious detour to survey this area of ring theory and cite a respectable range of its significant papers, so we hope the reader will forgive us for citing only two papers which were particularly early and influential, and which introduced many of the definitions later studied in that area.) 

However, as far as I know, there is no already-existing term in the literature for a ring $R$ with the property that every projective $R$-module has injective dimension $\leq d$. Here $d$ is some fixed integer. When $R$ is commutative, Noetherian, and local, this condition is equivalent to the condition that $R$ is a $d$-dimensional Gorenstein local ring. The case $d=0$ of a $d$-dimensional Gorenstein local ring is the case of a Frobenius ring. Quasi-Frobenius rings were originally introduced (see gx INSERT CITATION) to generalize the homological properties of Frobenius rings to the case of rings that are not necessarily commutative, not necessarily Noetherian, and not necessarily local. 

Just as quasi-Frobenius rings generalize Frobenius rings (i.e., $0$-dimensional Gorenstein rings) to the non-commutative non-Noetherian non-local case, we are in need of a definition that generalizes $d$-dimensional Gorenstein local rings to the non-commutative non-Noetherian non-local case. This definition is given in Definition \ref{def of gorenstein-margolis ring}, the definition of a ``Gorenstein-Margolis ring.''\footnote{Many generalizations of Gorenstein rings have been considered in the literature, e.g. noncommutative (but still Noetherian) Gorenstein rings in (gx INSERT BUCHWEITZ REF), ``almost Gorenstein rings'' (which are still commutative, local, and Noetherian) in (gx CITE BARUCCI-FROEHBERG), ``quasi-Gorenstein rings'' (which are still commutative, local, and Noetherian) in (gx CITE TAVANFAR AND TOUSI), ``weakly Gorenstein rings'' (which are Artinian) in (gx CITE RINGEL AND ZHANG ``GORENSTEIN-PROJECTIVE AND SEMI-GORENSTEIN-PROJECTIVE...''), and non-Noetherian Gorehstein rings (which are still Noetherian) in (gx CITE LIVIA MILLER'S THESIS). None of these are sufficiently general to include the main motivating examples for this paper, the Steenrod algebras! Since  all the usual adjectives (``weakly,'' ``almost,'' etc.) have already been put before the phrase ``Gorenstein ring,'' and since it is a result of Margolis (Theorems \ref{thm from margolis} and (gx INSERT MARGOLIS CITATION)) that the Steenrod algebras have the weakened Gorenstein property we are after, it seemed better to use a term like ``Gorenstein-Margolis ring'' in Definition \ref{def of gorenstein-margolis ring}, rather than something like ``quasi-almost-weakly-Gorenstein ring.''}
Here is a schematic diagram to illustrate the relationships between these definitions:
\begin{equation}\label{sketch diag 1}\xymatrix{
\left\{ \mbox{Frobenius local rings}\right\} \ar@{^{(}=>}[r]\ar@{^{(}=>}[d] & \left\{ \mbox{quasi-Frobenius rings}\right\} \ar@{^{(}=>}[d]\\
\left\{ \mbox{$d$-dim'l Gorenstein local rings}\right\} \ar@{^{(}=>}[r] & \left\{ \mbox{$d$-dim'l Gorenstein-Margolis rings}\right\} 
}\end{equation}
The left-hand column in diagram \eqref{sketch diag 1} is the commutative Noetherian local case of the right-hand column, while the top row in diagram \eqref{sketch diag 1} is the $d=0$ case of the bottom row.

\begin{definition}\label{def of gorenstein-margolis ring}
Let $R$ be a graded ring concentrated in nonnegative grading degrees.
\begin{itemize}
\item
We will say that $R$ is a {\em Gorenstein-Margolis ring} if there exists an upper bound on the injective dimension of graded projective $R$-modules. 
\item
If $R$ is a Gorenstein-Margolis ring and if $d$ is the least upper bound on the injective dimension of graded projective $R$-modules, we will say that $R$ is a {\em $d$-dimensional Gorenstein-Margolis ring.}
\item
If $R$ is a $d$-dimensional Gorenstein-Margolis ring and all bounded-below free $R$-modules are injective, we will say that $R$ is a {\em $d$-dimensional Margolis ring.}
\end{itemize}
\end{definition}

\begin{remark}
Suppose $R$ is a $d$-dimensional Gorenstein local ring (in the classical sense: so $R$ is Noetherian and commutative). Then, by giving $R$ the trivial grading in which $R$ is concentrated in degree zero, $R$ is also a $d$-dimensional Gorenstein-Margolis ring. So Gorenstein-Margolis rings generalize Gorenstein local rings. 

However, $R$ (with the trivial grading) will not be a $d$-dimensional Margolis ring unless $R$ is also quasi-Frobenius. So $d$-dimensional Margolis rings generalize $0$-dimensional Gorenstein local rings, but they do not generalize $d$-dimensional Gorenstein local rings if $d>0$. The Margolis rings behave homologically somewhat as though they are an ``infinitesimal thickening'' of the quasi-Frobenius rings, so that they are ``in between'' $0$-dimensional Gorenstein local rings and $1$-dimensional Gorenstein local rings.
\end{remark}

\subsection{Co-stable equivalence over Gorenstein-Margolis rings.}
\label{Co-stable equivalence...}

In this subsection we prove Theorem \ref{auslander-buchweitz thm 2}, a result dual to the Auslander-Buchweitz result reproduced above as Theorem \ref{auslander-buchweitz thm}, but without the restriction to finitely-generated modules. The ideas and results in this subsection are straightforward, and our purpose here is really to lay the groundwork for the next subsection, \cref{Distinguished equivalence...}.

\begin{definition}\label{def of co-stable equiv}
Let $R$ be a graded ring.
\begin{itemize}
\item Given graded right $R$-modules $M,N$, we say that two graded $R$-module maps $f,g: M \rightarrow N$ are {\em co-stably equivalent} if $f-g$ factors through an injective right $R$-module.
\item We say that a map $f: M \rightarrow N$ of graded $R$-modules is a {\em co-stable equivalence} if $f$ admits an inverse up to co-stable equivalence; that is, if there exists a graded $R$-module map $i: N \rightarrow M$ such that $f\circ i$ is co-stably equivalent to $\id_N$ and $i\circ f$ is co-stably equivalent to $\id_M$.
\item Given a graded right $R$-module $M$, by the {\em loop module of $M$,} written $\Omega M$, we mean the cokernel of any injective right $R$-module morphism $M\rightarrow I$ with $I$ injective\footnote{This definition has the unpleasant feature of making $0 \rightarrow M \rightarrow I \rightarrow \Omega M \rightarrow 0$ a short exact sequence with $I$ co-stably trivial, which is the opposite of what one is accustomed to in homotopy theory, where one has the homotopy fiber sequence $\Omega X \rightarrow PX \rightarrow X$ with $PX$ contractible. Writing suspension $\Sigma$ rather than looping $\Omega$ would seem to be more in line with conventions from homotopy theory. Our writing $\Omega$ rather $\Sigma$ is simply because, in $R$-modules up to co-stable equivalence, the functor $\Ext_R^n(N,-)$ plays a role analogous to homotopy groups $\pi_*$ in classical homotopy theory, and we have $\Ext_R^n(N,\Omega M) \cong \Ext_R^{n+1}(N,M)$ for $n>1$, that is, looping drops the cohomological invariants of a module by one dimension, just as looping a space or spectrum drops the homotopy invariants by one dimension.}. Of course $\Omega M$ is not well-defined in $\Mod(R)$, but it is at least well-defined up to co-stable equivalence.
\item Given a positive integer $n$, of course we write $\Omega^n$ for the $n$-fold iterate of $\Omega$.
\item We say that a graded right $R$-module $M$ is {\em deloopable} if there exists a graded right $R$-module $N$ and a co-stable equivalence $\Omega N \simeq M$. 
\item Similarly, we say that a graded right $R$-module $M$ is {\em $n$-fold deloopable} if there exists a graded right $R$-module $N$ and a co-stable equivalence $\Omega^n N \simeq M$. 
\item We say that a graded right $R$-module $M$ is {\em infinitely deloopable} if there exists a sequence of graded right $R$-modules $M = N_0,N_1, N_2, N_3, \dots$ and a co-stable equivalence $\Omega N_i \simeq N_{i-1}$ for each $i$.
\end{itemize}
\end{definition}

\begin{theorem}\label{auslander-buchweitz thm 2}
Let $d$ be a nonnegative integer, and let $R$ be a $d$-dimensional Gorenstein-Margolis ring. Then the following claims are each true:
\begin{enumerate}
\item Each graded right $R$-module $M$ fits into a short exact sequence of graded right $R$-modules
\begin{equation}\label{ses 300}
0 \rightarrow M \rightarrow M^{\prime} \rightarrow M^{\prime\prime}\rightarrow 0\end{equation}
in which $M^{\prime}$ is deloopable and $M^{\prime\prime}$ has finite injective dimension.
Furthermore, if $M$ is bounded-below, then $M^{\prime}$ can be chosen to be the loop module of a bounded-below graded $R$-module.
\item In particular, every bounded-below graded right $R$-module is co-stably equivalent to the loop module of a bounded-below graded right $R$-module.
\item If $R$ is a Margolis ring, then every bounded-below graded right $R$-module is infinitely deloopable\footnote{To be clear: when $R$ is Margolis, every bounded-below $R$-module is {\em itself} infinitely deloopable, not merely co-stably equivalent to an infinitely deloopable module.}.
\end{enumerate}
\end{theorem}
\begin{proof}
\begin{enumerate}
\item Choose a free graded right $R$-module $F$ and a surjective graded $R$-module homomorphism $\epsilon: F \rightarrow M$ and an injective graded $R$-module homomorphism $\iota: F\rightarrow I$ with $I$ injective. We then have the commutative diagram with exact rows and columns
\begin{equation}\label{comm diag 300}\xymatrix{
 & & 0 \ar[r]\ar[d] & 0 \ar[d] & \\
 0 \ar[r] \ar[d] &
  \ker \epsilon\ar[d]^{\id} \ar[r] &
  F \ar[r]^{\epsilon}\ar[d]^{\iota} &
  M \ar[r] \ar[d] &
  0 \ar[d] \\
 0 \ar[r] \ar[d] &
  \ker \epsilon\ar[d]^{\id} \ar[r] &
  I \ar[r]^{}\ar[d]^{} &
  M\coprod_F I \ar[r] \ar[d] &
  0 \ar[d] \\
 && \coker\iota \ar[r]^{\cong}\ar[d] & 
  \coker\iota \ar[r]^{\cong}\ar[d] & 
  0 \\
 && 0 \ar[r] & 0 & 
}\end{equation}
The right-hand nontrivial column in diagram \eqref{comm diag 300} is the desired short exact sequence \eqref{ses 300}. The middle row in diagram \eqref{comm diag 300} expresses $M\coprod_F I$ as the loop module of $\ker \epsilon$---that is, $M\coprod_F I$ is deloopable---while the middle column of diagram \eqref{comm diag 300} expresses that $\coker \epsilon$ has injective dimension one less than that of $F$, which is finite since $R$ is Gorenstein-Margolis. 

If $M$ is furthermore bounded-below, then since $R$ is connective, $F$ can be chosen to be bounded-below as well, and consequently $\ker \epsilon$ (our delooping of $M\coprod_F I$) is also bounded-below, as desired.
\item Immediate consequence of the previous part.
\item Let $M$ be a bounded-below graded right $R$-module. Since $R$ is concentrated in nonnegative grading degrees, we can find a bounded-below free graded $R$-module $F$ and a surjective graded $R$-module homomorphism $\epsilon: F \rightarrow M$. Since $R$ is a Margolis ring, $F$ is also injective. The kernel of $\epsilon$ is also bounded-below, so by iterating this procedure, we get a exact sequence of graded $R$-modules
\[ \dots 
 \stackrel{\delta^{-1}}{\longrightarrow} I^{-1} 
 \stackrel{\delta^0}{\longrightarrow} I^0 
 \rightarrow M \rightarrow 0\]
such that each $I^n$ is injective, that is, $M = \coker \delta^0$ and $\Omega (\coker \delta^n) \simeq \coker \delta^{n-1}$, and consequently $\coker \delta^n$ is an $n$-fold delooping of $M$.
\end{enumerate}
\end{proof}


\subsection{Distinguished equivalence over Gorenstein-Margolis rings.}
\label{Distinguished equivalence...}

Now we prove results similar to that of the previous subsection, \cref{Co-stable equivalence...}, with the following twist: rather than ``co-stable equivalence,'' in which two modules $M,N$ are equivalent if right-derived functors of left-exact functors agree on $M$ and $N$ in sufficiently high degrees, we consider ``distinguished equivalence,'' in which two modules $M,N$ are equivalent if their distinguished local cohomology groups agree in sufficiently high degrees. This gains force from Theorem \ref{dickson acyclicity}, which establishes that the rational modules are acyclic with respect to distinguished local cohomology. Consequently, as a loose slogan: given a $k$-coalgebra $\Gamma$, when we consider $\Gamma^*$-modules up to distinguished equivalence, {\em we are considering the homological algebra of $\Gamma^*$-modules modulo the homological algebra of $\Gamma$-comodules.}

Now we begin to make the above slogan precise:
\begin{definition}
Let $k$ be a field, and let $\Gamma$ be a graded $k$-coalgebra.
\begin{itemize}
\item Given graded right $\Gamma^*$-modules $M,N$, we say that two graded $\Gamma^*$-module maps $f,g: M \rightarrow N$ are {\em distinguished-equivalent} if $f-g$ factors through a rational right $R$-module.
\item We say that a map $f: M \rightarrow N$ of graded $R$-modules is a {\em distinguished equivalence} if $f$ admits an inverse up to distinguished equivalence. %; that is, if there exists a graded $R$-module map $i: N \rightarrow M$ such that $f\circ i$ is distinguished-equivalent to $\id_N$ and $i\circ f$ is distinguished-equivalent to $\id_M$.
%\item  ``Loop modules'' and ``deloopable modules'' remain defined by Definition \ref{def of co-stable equiv}.
\end{itemize}
\end{definition}

\begin{theorem}
Let $k$ be a field, and let $\Gamma$ be a graded $k$-coalgebra such that $\Gamma^*$ is a $d$-dimensional Gorenstein-Margolis ring. Then every bounded-above graded right $\Gamma^*$-module is rational.
Furthermore, the following are equivalent:
\begin{enumerate}
\item There exists an integer $n$ such that the functor $H^i_{\dist}: \gr\Mod(\Gamma^*)\rightarrow\gr\Mod(\Gamma^*)$ vanishes for all $i>n$.
\item There exists an integer $n$ such that, for all products $M$ of rational graded $R$-modules, the functor $H^i_{\dist}(M)$ vanishes for all $i>n$.
\item $d=0$.
\end{enumerate}
\end{theorem}
\begin{proof}
Let $n$ be an integer, and suppose that $M$ is a graded right $\Gamma^*$-module which vanishes in grading degree $i$ for all $i>n$.
Since $\Gamma^*$ is connective (since connectivity is part of Definition \ref{def of gorenstein-margolis ring}), the annihilator $\Ann(m)$ of any given homogeneous element $m\in M$ of degree $j$ is a homogeneous right ideal of $\Gamma^*$ which contains every homogeneous element of $\Gamma^*$ of degree greater than $i-j$. Consequently $\Gamma^*/\Ann(m)$ is isomorphic to ggggggxxxxxxx

% Let $M$ be a bounded-above graded right $\Gamma^*$-module, and suppose that $M$ is not distinguished-torsion. Then there exists a homogeneous element $m\in M$ such that $Im \neq 0$ for all distinguished right ideals $I$ of $\Gamma^*$.
% Write $n$ for an integer such that $M$ vanishes in degree $i$ for all $i>n$, and write $j$ for the degree of $m$.
% Then $gm=0$ for all homogeneous $g\in\Gamma^*$ of degree $>n-j$, so each distinguished right ideal $I$ of $\Gamma^*$ must contain some nonzero homogeneous element of degree $\leq n-j$. 
% Suppose that $I$ is a particular distinguished right ideal of $\Gamma^*$, and let $i$ be a nonzero homogeneous element of $I$ of degree $\leq n-j$. Suppose that $\gamma\in \Gamma$ is a homogeneous element whose annihilator ideal $\Ann(\gamma)$ is equal to $I$. (Such an element $\gamma$ exists by the definition of a distinguished ideal.) Then 

 and $I$ is a distinguished right ideal of $\Gamma^*$, ggxx

\begin{description}
\item[1 implies 2] Immediate.
\item[2 implies 3] We handle this by contrapositive: ggxx

\end{description}
\end{proof}

ggxx
\end{comment}

\begin{comment}
\section{Exactness results.}

\begin{lemma}\label{nice injective resolutions}
Let $R$ be a graded ring, and let $N\stackrel{f}{\longrightarrow} M \stackrel{g}{\longrightarrow} J$ be morphisms of graded $R$-modules, and suppose that $J$ is injective and that $f$ is surjective. Then there exists an injective graded $R$-module $I$ and a commutative diagram
\begin{equation}\label{comm diag 11} \xymatrix{
N \ar[r] \ar[d]^f & I \ar[d] \\ M \ar[r]^g & J
}\end{equation}
such that the map $I \rightarrow J$ is surjective and has injective kernel, and such that the map $N \rightarrow I$ is injective.
\end{lemma}
\begin{proof}
Choose an injective graded $R$-module $I^{\prime}$ into which $\ker f$ embeds, and an injective graded $R$-module $I^{\prime\prime}$ into which $M$ embeds. Then the Horseshoe Lemma yields an embedding of $N$ into $I = I^{\prime}\oplus I^{\prime\prime}$, such that the diagram \eqref{comm diag 11} commutes and such that $I\rightarrow J$ is surjective with kernel $I^{\prime}$.
\end{proof}

Proposition \ref{sseq prop 1} establishes that the Grothendieck spectral sequence for the composite of the sequential $\lim$ functor with a left exact functor $D$ vanishing on all free modules converges to an especially convenient $E_{\infty}$-term when $D$ is defined on the category of modules over a $\mathcal{P}$-algebra.
\begin{prop}\label{sseq prop 1}
Let $B$ be a $\mathcal{P}$-algebra, let $R$ be a ring, and let $D: \gr\Mod(B)\rightarrow \gr\Mod(R)$ be a left exact functor. Suppose that $R^nD(M)$ vanishes for all $n>0$ for all free $B$-modules $M$. 
Let \begin{equation}\label{seq 110} \dots \rightarrow M_2 \rightarrow M_1 \rightarrow M_0\end{equation} be a sequence of surjective morphisms of graded $B$-modules. 
Then there exists a strongly convergent spectral sequence
\begin{align*}
%\label{sseq 16} 
E_2^{s,t} \cong R^s\lim_n R^tD(M_n) & \Rightarrow \lim_n R^{s+t}D(M_n) \\
%\nonumber 
d_r: E_r^{s,t} &\rightarrow E_r^{s+r,t+r-1}.
\end{align*}
\end{prop}
\begin{proof}
Use Lemma \ref{nice injective resolutions} to inductively build a commutative diagram of graded $B$-modules 
\begin{equation}\label{diag 111} \xymatrix{
\vdots \ar[d] & \vdots \ar[d] & \vdots \ar[d] & \vdots \ar[d] & \\
0 \ar[r] \ar[d] & M_2 \ar[r] \ar[d] & I_2^0 \ar[r]\ar[d] & I_2^1\ar[r] \ar[d] & \dots \\
0 \ar[r] \ar[d] & M_1 \ar[r] \ar[d] & I_1^0 \ar[r]\ar[d] & I_1^1\ar[r] \ar[d] & \dots \\
0 \ar[r]  & M_0 \ar[r]  & I_0^0 \ar[r] & I_0^1\ar[r]  & \dots \\
}\end{equation}
whose rows are exact and whose vertical maps are all surjective.
We ``truncate,'' omitting each resolved module $M_n$, and we then write $I_n^{\bullet}$ for the resulting injective resolution of each $M_n$.
We then have a sequence 
\begin{equation}\label{seq of cochain cplxs 1} \dots I_2^{\bullet} \dots I_1^{\bullet} \dots I_0^{\bullet}  \end{equation}
of levelwise epimorphisms in the category of connective\footnote{That is, concentrated in nonnegative cohomological degrees.} cochain complexes of graded $B$-modules.
Levelwise epimorphisms are fibrations in the injective model structure on connective cochain complexes, so \eqref{seq of cochain cplxs 1} is a fibrant object in the category of inverse sequences of connective cochain complexes of graded $B$-modules; this is just a way of recognizing the limit of \eqref{seq of cochain cplxs 1} is a homotopy limit, and so the Bousfield-Kan spectral sequence with $E_2$-term $E_2^{s,t} \cong R^s\lim_n H^t(I_n^{\bullet})$ converges to $H^{s+t}(\lim_n I_n^{\bullet})$. Since each $I_n^{\bullet}$ is acyclic with $H^0$ isomorphic to $M_n$, and since the maps in \eqref{seq 110} are surjective and consequently \eqref{seq 110} is Mittag-Leffer, this Bousfield-Kan spectral sequence collapses at $E_2$, yielding that the cochain complex $\lim_nI^{\bullet}_n$ is acyclic, with $H^0(\lim_nI^{\bullet}_n)\cong \lim_nM_n$.

Furthermore, for each $m$, the usual analysis of sequential limits in AB4 abelian categories yields an exact sequence
\[ 0 \rightarrow \lim_nI_n^m \rightarrow \prod_n I_n^m \stackrel{\id - T}{\longrightarrow} \prod_n I_n^m \rightarrow \lim^1_nI_n^m \rightarrow 0.\]
Since the vertical maps in \eqref{diag 111} are surjective, $\lim^1_nI_n^m$ vanishes, and so\begin{equation}\label{seq 12} 0 \rightarrow \lim_nI_n^m \rightarrow \prod_n I_n^m \stackrel{\id - T}{\longrightarrow} \prod_n I_n^m \rightarrow 0\end{equation}
is short exact. Products of injectives are injective, and so \eqref{seq 12} shows that $\lim_nI_n^m$ has injective dimension $\leq 1$. Now Margolis's result, reproduced above as \ref{thm from margolis}, gives us that each $\lim_nI_n^m$ also has projective dimension $\leq 1$.

If a module $M$ has projective dimension $\leq 1$, then we can choose a short exact sequence \begin{equation*}%\label{les 14} 
0 \rightarrow P_1 \rightarrow P_0 \rightarrow M \rightarrow 0\end{equation*} of graded $B$-modules, and applying $R^*D$ to that short exact sequence yields a long exact sequence in which $R^iD(P_1)$ and $R^iD(P_0)$ must each vanish for all $i>0$, since $P_0$ and $P_1$ are each summands of free $B$-modules and $D$ was assumed to be acyclic, and hence $R^iD(M)$ also vanishes for all $i>0$. 

So $R^iD(\lim_nI_n^m)$ vanishes for all $i>0$. Consequently $\lim_nI_n^{\bullet}$ is a $D$-acyclic resolution of $\lim_n M_n$. 

We now apply $D$ to the entire sequence of cochain complexes \eqref{seq of cochain cplxs 1} to yield a sequence of morphisms of cochain complexes of abelian groups
\begin{equation}\label{seq of cochain cplxs 2} \dots D(I_2^{\bullet}) \dots D(I_1^{\bullet}) \dots D(I_0^{\bullet}).  \end{equation}
Since each vertical map in \eqref{diag 111} is surjective with injective kernel, the long exact sequence induced by applying $D$ to each vertical map in \eqref{diag 111} is in fact simply a short exact sequence, i.e., the morphisms of cochain complexes in the sequence \eqref{seq of cochain cplxs 2} are in fact levelwise surjective. 
So we get another Bousfield-Kan spectral sequence
\begin{align}
\label{sseq 15} E_2^{s,t} \cong R^s\lim_n H^t(D(I_n^{\bullet})) & \Rightarrow H^{s+t}(\lim_n D(I_n^{\bullet})) \\
\nonumber d_r: E_r^{s,t} &\rightarrow E_r^{s+r,t+r-1}
\end{align}
and an\footnote{At this point it is probably clear that we this Bousfield-Kan spectral sequence is simply the Grothendieck spectral sequence relating the right-derived functors of $\lim$, $D$, and $\lim\circ D$. But simply citing a textbook reference for the Grothendieck spectral sequence would not yield the better handle of what the spectral sequence is converging to that we gained from our careful construction of the spectral sequence.} isomorphism $E_2^{s,t} \cong R^s\lim_n R^tD(M_n)$.
Levelwise surjectivity of the maps in \eqref{seq of cochain cplxs 2} also gives us (via the Mittag-Leffler condition) an isomorphism $H^i(\lim_n D(I_n^{\bullet})) \cong \lim_nH^i(D(I_n^{\bullet})$, i.e., spectral sequence \eqref{sseq 15} converges to $\lim_n H^{s+t}(D(I_n^{\bullet})) \cong \lim_n R^{s+t}D(M_n)$. 
\end{proof}
\end{comment}

\begin{comment} (NOT CURRENTLY USED)
\section{Mitchell sequences.}

\begin{definition}
Let $\Gamma$ be a $k$-coalgebra.
By a {\em Mitchell sequence in $\Gamma$} we mean a sequence of $k$-subcoalgebras
\begin{equation}\label{mitchell seq} \Gamma_0 \subseteq \Gamma_1 \subseteq \Gamma_2 \subseteq \dots \end{equation}
of $\Gamma$ such that ....?
\end{definition}

\begin{definition}
Let $\Gamma$ be a graded $k$-coalgebra.
\begin{itemize}
\item
Given a graded $k$-subcoalgebra $\Gamma^{\prime}$ of $\Gamma$, we say that a homogeneous right ideal $I$ of $\Gamma^*$ is {\em $\Gamma^{\prime}$-distinguished} if $\Gamma^{\prime}$, regarded as a graded $\Gamma^*$-module, contains a graded $\Gamma^*$-submodule which is isomorphic to a suspension of a quotient of $\Gamma^*/I$.
\item 
We write $\dist_{\Gamma^{\prime}}(\Gamma)$ for the set of $\Gamma^{\prime}$-distinguished right ideals of $\Gamma^*$, partially ordered by inclusion. We write $H^0_{\Gamma^{\prime}-\dist}$ for the functor $\gr\Mod(\Gamma^*)\rightarrow\gr\Mod(\Ab)$ given by $H^0_{\Gamma^{\prime}-\dist}(M) = \colim_{I\in \dist_{\Gamma^{\prime}}(\Gamma)}\underline{\hom}\left( \Gamma^*/I,M\right)$.
\item 
There is a natural monomorphism of graded $\Gamma^*$-modules 
\begin{equation}\label{nat map 1a} H^0_{\dist,\Gamma^{\prime}}(M)\hookrightarrow M\end{equation} induced by the composite $\underline{\hom}\left( \Gamma^*/I,M\right) \rightarrow \underline{\hom}\left( \Gamma^*,M\right) \stackrel{\cong}{\longrightarrow} M$. We will say that $M$ is {\em $\Gamma^{\prime}$-distinguished-torsion} if the natural map \eqref{nat map 1a} is an isomorphism.
\item
We write $\mathcal{T}_{\Gamma^{\prime}}$ for the torsion class\footnote{It is formal ggxx} consisting of the $\Gamma^{\prime}$-distinguished-torsion $\Gamma^*$-modules. 
\end{itemize}
\end{definition}

\begin{prop}
Let $\Gamma$ be a $k$-coalgebra, and let \eqref{mitchell seq} be a Mitchell sequence in $\Gamma$. Then ggxx
\end{prop}
\begin{proof}

% \lim_n \colim_{I\in \dist_{\Gamma(n)}(\Gamma)}\hom_{\Gamma^*}(\Gamma^*/I,-)
% How does this work for Gamma = k<t_0, t_1, ...>, Gamma^* = k[x]?
% We have Gamma(n) = k<t_0, t_1, ..., t_{n-1}>, Gamma(n)^* = k[x]/x^n.
% 
% 0 -> (mult by x seq. on k[x]) -> (id seq on k[x]) -> (seq of quotients k[x]/x^n) -> 0
%
% 0 -> (seq of ideals           -> (id seq on A)    -> (seq of quotients           -> 0
%       (Sq^2,Sq^4,Sq^8,...),                            A/(Sq^2,Sq^4,Sq^8,...),
%            (Sq^4,Sq^8,...),                                 A/(Sq^4,Sq^8,...),
%                 (Sq^8,...),                                      A/(Sq^8,...),
%                  ... )                                              ... )
% ggxx

\end{proof}
\end{comment}


\section{Applications to the Steenrod algebra.}

Throughout this section, let $p$ be a prime number, let $A$ be the mod $p$ Steenrod algebra, and let $A^*$ be the dual mod $p$ Steenrod algebra. While we grade $A$ in the usual way (so that $\Sq^n$ is in degree $n$, $P^n$ is in degree $2n$, and $\beta$ is in degree $1$), we emphasize that, contrary to common practice in topology, {\em we grade the dual Steenrod algebra $A^*$ so that it is concentrated in nonpositive degrees,} so that $\xi_n$ is in degree $2^n-1$ for $p=2$ and $2(p^n-1)$ for $p>2$, and $\tau_n$ is in degree $2p^n-1$. This uncommon grading convention is necessary in order to consider $A^*$ as a graded $A$-module with the contragredient $A$-action.

--

\begin{lemma}
The dual Steenrod algebra $A^*$ is injective when regarded as a graded right $A$-module via the contragredient $A$-action.
\end{lemma}
\begin{proof}
This is a special case of the well-known classical (given, for example, as Lemma 3.37 in \cite{MR2455920}) that, given a ring $R$ and an $R$-module $M$, the $R$-module $\hom_{\mathbb{Z}}(R,\mathbb{Q}/\mathbb{Z})$, with the contragredient $R$-action, is injective.
\end{proof}

\begin{prop}\label{instability}
Let $\mathcal{T}$ be the distinguised-torsion graded right $A$-modules. Then 
the %hereditary
 torsion class $(\mathcal{T},\mathcal{F})$ on $\gr\Mod(A)$ is {\em not} stable.
\end{prop}
\begin{proof}
Given a left $A^*$-comodule $M$, the extended $A^*$-comodule $EM = A^*\otimes_{\mathbb{F}_p} M$ ggxx
\end{proof}

\begin{theorem}
Let $(\mathcal{T},\mathcal{F})$ be the torsion theory on graded right $A$-modules in which $\mathcal{T}$ consists of the distinguished-torsion $A$-modules. Then $(\mathcal{T},\mathcal{F})$ is quasistable.
\end{theorem}
\begin{proof}
Let $M$ be a graded right $A$-module, and let $\mathbb{\Gamma}(M)$ be its rational submodule, i.e., $\mathbb{\Gamma}(M) \hookrightarrow M$ is the counit of the adjunction given by $\Rat: \gr\Mod(\Gamma^*) \rightarrow \gr\Comod(\Gamma)$ and $\iota: \gr\Comod(\Gamma)\rightarrow \gr\Mod(\Gamma^*)$. We have the commutative diagram in $\gr\Mod(\Gamma^*)$ with exact rows and columns
\[\xymatrix{
 & 0 \ar[d] &  0 \ar[d] &  0 \ar[d] & \\
0 \ar[r] & \mathbb{\Gamma}(M) \ar[r]\ar[d] & M \ar[r]\ar[d] & M/\mathbb{\Gamma}(M) \ar[r]\ar[d] & 0 \\
0 \ar[r] & \iota\left(A_*\otimes_{\mathbb{F}_p}\Rat(M)\right) & & ggxx &
}\]
Since $A_*\otimes_{\mathbb{F}_p}\Rat(M)$ is an extended $A_*$-comodule, it is injective, and so $\iota(A_*\otimes_{\mathbb{F}_p}\Rat(M))$ is $\Gamma$-injective, i.e., $\mathcal{T}$-acyclic, by 42.20 of \cite{MR2012570}. 

IF WE CAN SHOW THAT $M/\mathbb{\Gamma}(M)$ EMBEDS INTO A PRODUCT OF COPIES OF $A$:
we use Mitchell's theorem, that $A \cong \lim_n A(n)$ as $A$-modules, to get 
$\prod_I A \cong \lim_n \prod_I A(n)$, i.e., $H^i_{\dist}(\lim_n \prod_I A(n)) \cong \colim_{I\in \dist(A_*)} \Ext^i_A(A/I,\lim_n\prod_IA(n))$. If $M$ is a quotient of $\prod_I A$, then:
$M \rightarrow \lim_n (A(n)\otimes_A M)$ is...ggxxx

IS $M/\mathbb{\Gamma}(M)$ EMBEDDABLE INTO A PRODUCT OF COPIES OF $A$?
Given a graded $A$-module $N$, we have the map $N \rightarrow \prod_{f: N \rightarrow A} A$.

%A_* is Gamma-injective, hence T-acyclic.
%A is injective, hence Gamma-injective, hence T-acyclic.
%Products of injectives are injective, hence Gamma-injective, hence T-acyclic.
%Products of Gamma-injectives are injective, hence T-acyclic.
%In fact H^0_{dist}(A) also vanishes!
% 0 -> M -> A -> A/M -> 0 yields that H^{n+1}(M) = H^n(A/M) for all n, including n=0.
% 0 -> H^0(A/M) -> A/M -> (A/M)/H^0(A/M) -> 0
% H^2(M) = H^1(A/M) = H^0(A/((A/M)/H^0(A/M)))
% So apparently we need to show that A/M is dist-torsion? (No, we don't.)
%
% Uh oh: suppose that (T,F) is quasistable, and suppose that H^1(H^0 M) vanishes 
%  for all n>0 and all M in C.
% Then for every M in C, we have 0 -> M -> I -> I/M -> 0 for some injective I in C, 
%  so I/M is T-acyclic by quasistability.
% Applying H^0 to 0 -> H^0(I/M) -> I/M -> (I/M)/H^0(I/M) -> 0 and using the vanishing of
%  H^1(H^0 I/M) gives us that H^0((I/M)/H^0(I/M)) vanishes. 
%  ...no, this seems ok.
%



ggxx
\end{proof}


ggxx

\bibliography{/home/asalch/texmf/tex/salch}{}
\bibliographystyle{plain}
\end{document}


\section{Finite cohomological dimension for $H^*_{\dist}$.}


This final section is of a speculative nature, explaining some questions (Questions \ref{question 1} and \ref{question 2}) that I do not yet know answers to, why these questions are algebraically natural, and some topological consequences for a positive answer to Question \ref{question 2}.
Classically, given a commutative Noetherian local ring $R$ of Krull dimension $d$, the local cohomology $H^n_{\mathfrak{m}}(M)$ at the maximal ideal $\mathfrak{m}$ of $R$ vanishes for all $R$-modules $M$ whenever $n>d$; while this result is classical for $R$ Cohen-Macaulay, in this greater level of generality it is the content of Corollary 3.2 of \cite{MR1172439}. It is natural to ask whether, when $\Gamma$ is a $k$-coalgebra whose dual algebra is Gorenstein-Margolis of finite dimension $d$, we might also have a vanishing theorem, establishing that the distinguished-local cohomology $H^n_{\dist}(M)$ vanishes for all $n>d$, or at least for all $n$ greater than some reasonable bound depending on $d$. I do not know whether any such vanishing theorem might hold, except in the  case $d=0$, where such a vanishing theorem holds for trivial reasons. 

In \cref{Quasistable torsion theories.} we give some discussion of existing ideas about torsion classes (like stability and the weak stability of \cite{MR3849888} and \cite{MR4026630}) which are related to such vanishing theorems. In \cref{...on products of rational...} we state some natural questions about such vanishing theorems, and topological motivations for those questions.

\subsection{Quasistable torsion theories.}
\label{Quasistable torsion theories.}
Throughout this section, let $(\mathcal{T},\mathcal{F})$ be a %hereditary
 torsion theory on an abelian category $\mathcal{C}$ with enough injectives.

\begin{definition}
An object $X$ of $\mathcal{C}$ is said to be {\em $\mathcal{T}$-acyclic} if $R^n\mathbb{\Gamma}(X)$ vanishes for all $n>0$, where $\mathbb{\Gamma} :\mathcal{C}\rightarrow\mathcal{C}$ is the composite of the inclusion $\mathcal{T}\rightarrow\mathcal{C}$ with the right adjoint to the inclusion $\mathcal{T}\hookrightarrow\mathcal{C}$.
\end{definition}

\begin{prop}\label{equiv quasistable conditions}
The following conditions are equivalent:
\begin{enumerate}
\item For each object $X$ of $\mathcal{C}$, there exists an injective object $I$ of $\mathcal{C}$ and a monomorphism $\eta: X\rightarrow I$ in $\mathcal{C}$ such that the cokernel $\coker \eta$ is $\mathcal{T}$-acyclic. 
\item For each object $X$ of $\mathcal{C}$, there exists a $\Gamma$-acyclic object $I$ of $\mathcal{C}$ and a monomorphism $\eta: X\rightarrow I$ in $\mathcal{C}$ such that the cokernel $\coker \eta$ is $\mathcal{T}$-acyclic. 
\item The functor $\mathbb{\Gamma}: \mathcal{C}\rightarrow\mathcal{C}$ has cohomological dimension $\leq 1$. That is, for every object $X$ of $\mathcal{C}$ and every integer $n>1$, $R^n\Gamma(X)$ vanishes.
\end{enumerate}
\end{prop}
\begin{proof}\leavevmode
\begin{description}
\item[1 implies 2] Immediate since injectives are $\mathcal{T}$-acyclic.
\item[2 implies 3] The long exact sequence obtained by applying $R^*\mathbb{\Gamma}$ to the long exact sequence $0 \rightarrow X \rightarrow I \rightarrow I/X\rightarrow 0$ reduces to the exact sequence
\[ 0 \rightarrow \mathbb{\Gamma} X \rightarrow \mathbb{\Gamma} I \rightarrow \mathbb{\Gamma} (I/X) \rightarrow R^1\mathbb{\Gamma}(X) \rightarrow 0 \]
and the isomorphisms $R^{n+1}\mathbb{\Gamma}(X) \cong R^{n}\mathbb{\Gamma}(I/X) \cong 0$ for all $n>0$. 
\item[3 implies 1] Choose an injective object $I$ of $\mathcal{C}$ and a monomorphism $\eta: X \rightarrow I$. Since $R^n\mathbb{\Gamma}(X)$ vanishes for all $n>1$, the long exact sequence obtained by applying $R^*\mathbb{\Gamma}$ to $0 \rightarrow X \rightarrow I \rightarrow I/X\rightarrow 0$ gives us that $R^n\mathbb{\Gamma}(I/X)$ must vanish for all $n>0$, i.e., $I/X$ is $\mathcal{T}$-acyclic, as desired.
\end{description}
\end{proof}

\begin{definition}
We will say that the torsion theory $(\mathcal{T},\mathcal{F})$ is {\em quasistable} if $(\mathcal{T},\mathcal{F})$ satisfies the equivalent conditions of Proposition \ref{equiv quasistable conditions}.
\end{definition}

\begin{prop}\label{stable quasistable and weakly stable}
Every exact stable torsion theory is quasistable. Furthermore, every quasistable torsion theory on a category of modules over a ring is weakly stable.
\end{prop}
\begin{proof}
That every exact stable torsion theory is quasistable follows from Theorem \ref{stable exact tors theories are quasistable} and Proposition \ref{equiv quasistable conditions}. That every quasistable torsion theory on a module category is also weakly stable follows from Proposition \ref{equiv quasistable conditions} together with Lemma 3.12 from \cite{MR4026630}: if a torsion theory on a category of modules over a ring has the property that $R^n\mathbb{\Gamma}$ vanishes for all $n>1$, then the torsion theory is quasistable.
\end{proof}
Proposition \ref{stable quasistable and weakly stable} probably admits some degree of generalization. It seems likely that quasistable torsion theories are stable, without any need to restrict to module categories, but we have not attempted to prove this. We have also not attempted to find out whether there are non-exact stable torsion theories which fail to be quasistable. 

\subsection{$H^*_{\dist}$ on products of rational modules.}
\label{...on products of rational...}

In gx, we showed that, for each nonnegative integer $n$, there exists a graded module $M$ over the Steenrod algebra such that $H^n_{dist}(M)$ is nontrivial. Consequently the distinguished torsion class on the category of graded $A$-modules fails to be quasistable. However, the modules $M$ in that construction are not constructed as products of rational modules, so that nonvanishing result for $H^n_{dist}(M)$ for $n>>0$ does not imply that $H^n_{dist}\left(\prod_{i\in I} M_i\right)$ can be nontrivial for $n>>0$ when all the $A$-modules $M_i$ are rational. Since (by gx) only the local cohomology groups $H^n_{dist}\left(\prod_{i\in I} M_i\right)$ for $M_i$ rational have a topological interpretation, it remains an important open question to determine whether $H^n_{dist}\left(\prod_{i\in I} M_i\right)$ can be nontrivial for large $n$ when all $M_i$ are rational. 

We now give more detail to that question. This begins with a definition:
\begin{definition}\label{def of prodrat}
Let $k$ be a field and let $\Gamma$ be a graded $k$-coalgebra such that the dual algebra $\Gamma^*$ is Gorenstein-Margolis of positive finite dimension. Let $\ProdRat(\Gamma^*)$ denote the smallest subcategory of $\gr\Mod(\Gamma^*)$ which:
\begin{enumerate}
\item contains all rational graded $\Gamma^*$-modules,
\item is closed under kernels and cokernels in $\gr\Mod(\Gamma^*)$,
\item and is closed under products and coproducts in $\gr\Mod(\Gamma^*)$.
\end{enumerate}
\end{definition} 
If $\{ \mathcal{C}_i: i\in I\}$ is a set of subcategories of $\gr\Mod(\Gamma^*)$ such that each $\mathcal{C}_i$ satisfies the three conditions in Definition \ref{def of prodrat}, then the intersection $\cap_{i\in I}\mathcal{C}_i$ also satisfies those three conditions, which is why we can speak of {\em the smallest} subcategory of $\gr\Mod(\Gamma^*)$ satisfying those conditions.

It is standard (see, for example, Theorem 3.41 of \cite{MR2050440}) that a subcategory of an abelian category $\mathcal{C}$ closed under kernels, cokernels, and biproducts is abelian, and its inclusion into $\mathcal{C}$ is exact. Consequently $\ProdRat(\Gamma^*)$ is an abelian subcategory of $\gr\Mod(\Gamma^*)$, and the inclusion $\ProdRat(\Gamma^*)\hookrightarrow \gr\Mod(\Gamma^*)$ is exact.

Recall that $\sigma[{}_{\Gamma^*}\Gamma]$ is the full subcategory of $\gr\Mod(\Gamma^*)$ generated by all submodules of quotients of graded free $\Gamma^*$-modules; see Theorem \ref{review thm} for more detail, and section 41 of \cite{MR2012570} for much more. While $\sigma[{}_{\Gamma^*}\Gamma]$ is equivalent to the category of graded $\Gamma$-comodules, it is in general not closed under products in $\gr\Mod(\Gamma^*)$, while $\ProdRat(\Gamma^*)$ is. So in the chain of inclusions
\[ \sigma[{}_{\Gamma^*}\Gamma] \subseteq \ProdRat(\Gamma^*)\subseteq \gr\Mod(\Gamma^*)\]
the inclusions are typically proper inclusions. 

\begin{question}\label{question 1}
Let $k$ be a field and let $\Gamma$ be a graded $k$-coalgebra such that the dual algebra $\Gamma^*$ is Gorenstein-Margolis of positive finite dimension.
Does there exist some integer $n$ such that $H^m_{\dist}(M)\cong 0$ for all $m>n$ and all $M\in\ob\ProdRat(\Gamma^*)$?
\end{question}
One must exercise a bit of care in making sense of Question \ref{question 1}: I know of no reason why Definition \ref{def of prodrat} ought to imply that $\ProdRat(\Gamma^*)$ has enough injectives, so it is not clear that we can define $H^m_{\dist}(M)$ to be the $n$th right derived functor of $H^0_{\dist}(M)$ in the category $\ProdRat(\Gamma^*)$. Instead, if $M$ is an object of $\ProdRat(\Gamma^*)$, then by $H^m_{\dist}(M)$ we simply mean $H^m_{\dist}$ applied to the image of $M$ under the inclusion 
\begin{equation}\label{incl functor 1} \ProdRat(\Gamma^*) \hookrightarrow\gr\Mod(\Gamma^*).\end{equation} This definition is quite reasonable, since \eqref{incl functor 1} is exact.

The most important case of Question \ref{question 1} is the case when $d=1$, since the Steenrod algebras are one-dimensional Gorenstein-Margolis algebras. Since mod $p$ homology does not always commute with infinite products of spectra, it is not possible that $H^m_{\dist}(M)$ for all $M\in\ProdRat(A)$ and all $m>0$, where $A$ is the mod $p$ Steenrod algebra. Instead, the least plausible value of $n$ such that $H^m_{\dist}(M)$ for all $M\in\ProdRat(A)$ and all $m>n$ is $n=1$. This suggests the question:
\begin{question}\label{question 2}
Let $k$ be a field and let $\Gamma$ be a graded $k$-coalgebra such that the dual algebra $\Gamma^*$ is a one-dimensional Gorenstein-Margolis ring.
Is the distinguished torsion class on $\ProdRat(\Gamma^*)$ quasistable? 

More precisely\footnote{In \cref{Quasistable torsion theories.} we have only defined and discussed quasistable torsion classes on Grothendieck categories, and as discussed above, I know of no reason to think that $\ProdRat(\Gamma^*)$ has enough injectives, hence no reason to expect $\ProdRat(\Gamma^*)$ to be Grothendieck. A suitable generalization of the ideas from \cref{Quasistable torsion theories.} yields this straightforward notion of quasistability for the distinguished torsion class on $\ProdRat(\Gamma^*)$.}, does $H^m_{\dist}(M)$ vanish for all $m>1$ and all $M\in\ob\ProdRat(\Gamma^*)$?
\end{question}

If Question \ref{question 2} has an affirmative answer---even just in the case that $\Gamma = A_*$, the dual mod $p$ Steenrod algebra---then the Hovey-Sadofsky spectral sequence 
\begin{align*}
 E^2_{s,t} &\cong R^s\prod_{i\in I}^{A_*} \left(H_*\left( X_i; \mathbb{F}_p\right)\right)_t \\
 d^r_{s,t}: E^r_{s,t} &\rightarrow E^r_{s+r,t+r-1}
\end{align*}
has a horizontal vanishing line at the $E_2$-page, and consequently converges strongly to the homology of the product, $H_{t-s}\left( \prod_{i\in I} X_i; \mathbb{F}_p\right)$, as long as each of the spectra $X_i$ is $H\mathbb{F}_p$-nilpotently complete. (gx INSERT CITATIONS) Consequently, if Question \ref{question 2} has an affirmative answer, then we get a rather Milnor-like short exact sequence
\begin{equation}\label{ses 23000} 0 \rightarrow \left( H^1_{\dist}\prod_{i\in I} H_*\left(X_i; \mathbb{F}_p\right)\right)_{n+1}
 %R^1\prod_{i\in I}^{A_*}\left(H_*\left( X_i; \mathbb{F}_p\right)\right)_{n+1}
 \rightarrow
 H_n\left( \prod_{i\in I} X_i; \mathbb{F}_p\right) \rightarrow
 \prod_{i\in I} H_n\left( X_i; \mathbb{F}_p\right) \rightarrow 0 \end{equation}
whenever each spectrum $X_i$ is $H\mathbb{F}_p$-nilpotently complete. The exact sequence \eqref{ses 23000} is very appealing in its simplicity and would be a far more general and effective way to calculate homology of non-bounded-below infinite products of spectra than any other method currently known. This is the motivation for posing Question \ref{question 2}.



% H^0 M^* 
%  = \colim_{I\in \dist(\Gamma)} \hom_{\Gamma^*}(\Gamma^*/I,\hom_k(M,k))
%  = \colim_{I\in \dist(\Gamma)} \hom_{k}(\Gamma^*/I \otimes_{\Gamma^*} M,k))
%  = \colim_{I\in \dist(\Gamma)} (M/IM)^*
%
% (H^0 M)^*
%  = \hom_k(\colim_{I\in \dist(\Gamma)} \hom_{\Gamma^*}(\Gamma^*/I,M),k)
%  = \lim_{I\in \dist(\Gamma)} \hom_k(\hom_{\Gamma^*}(\Gamma^*/I,M),k)
%  = \lim_{I\in \dist(\Gamma)} (\hom_{\Gamma^*}(\Gamma^*/I,M)^*)
%
% (H^0 M^*)^*
%  = \lim_{I\in \dist(\Gamma)} (\hom_{\Gamma^*}(\Gamma^*/I,\hom_k(M,k))^*)
%  = \lim_{I\in \dist(\Gamma)} (\hom_k(\Gamma^*/I \otimes_{\Gamma^*}M,k)^*)
%  = \lim_{I\in \dist(\Gamma)} ((M/IM)^**)
%
% Double-dualization is exact.
% Does it preserve injectives?
% If I is an inj. A-module, 
% 0 -> M -> N       is equiv. to:      0 -> M\otimes_k I* -> N\otimes_k I*
%      |                                     |
%      v                                     v
% I -> I**                                   k
%
% If double-dualization is indeed exact, then given a SES
%  0 -> M' -> M -> M'' -> 0
%  of bounded-below graded left A-modules,
%  0 -> M'** -> M** -> M''** -> 0
%  is also exact, so:
% ggxx








