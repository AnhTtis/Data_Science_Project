\documentclass{amsart}
\usepackage{shortsalch}
%\usepackage{MnSymbol}
\usepackage{xy}
%\usepackage{csquotes}
%\usepackage{stmaryrd}
%\usepackage{tikz,placeins}
\usepackage{mathtools}
\xyoption{all}
\usepackage{cleveref}
\title[Derived limit in comodules over the dual Steenrod algebra]{Derived functors of product and limit in the category of comodules over the dual Steenrod algebra.}
%\date{February 2021}
\author{A. Salch}
\begin{document}
\begin{abstract}
In the 2000s, Sadofsky constructed a spectral sequence which converges to the mod $p$ homology groups of a homotopy limit of a sequence of spectra. The input for this spectral sequence is the derived functors of sequential limit in the category of graded comodules over the dual Steenrod algebra. Since then, there has not been an identification of those derived functors in more familiar or computable terms. Consequently there have been no calculations using Sadofsky's spectral sequence except in cases where these derived functors are trivial in positive cohomological degrees. 

In this paper, we prove that the input for the Sadofsky spectral sequence is simply the local cohomology of the Steenrod algebra, taken with appropriate (quite computable) coefficients. 
This turns out to require both some formal results, like some general results on torsion theories and local cohomology of noncommutative non-Noetherian rings, and some decidedly non-formal results, like a 1985 theorem of Steve Mitchell on some very specific duality properties of the Steenrod algebra not shared by most finite-type Hopf algebras. Along the way there are a few results of independent interest, such as an identification of the category of graded $A_*$-comodules with the full subcategory of graded $A$-modules which are torsion in an appropriate sense.
\end{abstract}
\maketitle
\tableofcontents


\section{Introduction.}

Let $k$ be a field.
It is well-known that the category $\Comod(\Gamma)$ of comodules over a $k$-coalgebra $\Gamma$ is abelian and has generally agreeable properties, like being complete and co-complete and having enough injectives, but unlike the category of modules over a ring, $\Comod(\Gamma)$ does not necessarily satisfy Grothendieck's axiom $AB4^*$ \cite{MR0102537}. (Recall that $AB4^*$ is the axiom that states that infinite products of exact sequences are exact.) Consequently, given a set $\{ M_i: i\in I\}$ of $\Gamma$-comodules, we may have nonvanishing higher right-derived functors $R^n\prod_{i\in I}^{\Gamma} M_i$ of the categorical product $\prod^{\Gamma}$ in the category of $\Gamma$-comodules. One consequence is that, given a sequence $\dots \rightarrow M_2 \rightarrow M_1 \rightarrow M_0$ of $\Gamma$-comodule homomorphisms, we may have nonvanishing right-derived functors $R^n\lim^{\Gamma}_i M_i$ for $n>1$, unlike the familiar situation in categories of modules; see \cite{MR132091} and \cite{MR2197371} for this implication.

There are important topological consequences. Let $p$ be a prime number, suppose that the ground field $k$ is the field $\mathbb{F}_p$ with $p$ elements, and suppose that the coalgebra $\Gamma$ is the $p$-primary dual Steenrod algebra. In the influential but unpublished 2001 preprint \cite{sadofsky2001homology}, H. Sadofsky constructed a spectral sequence 
\begin{align}
\label{sadofsky ss 1} E_2^{*,*} \cong R^*\lim^{\Gamma}_i H_*(X_i; \mathbb{F}_p) &\Rightarrow H_*\left( \holim_i X_i; \mathbb{F}_p\right)\end{align}
for a sequence $\dots \rightarrow X_2 \rightarrow X_1 \rightarrow X_0$ of $H\mathbb{F}_p$-nilpotently complete spectra. Since then, Sadofsky's spectral sequence has been written about, and the details of its construction are available in the published literature: see the appendix of \cite{MR4080481}, for example, or \cite{MR2337861} for the case of products of spectra, where one has a spectral sequence
\begin{align}
\label{sadofsky ss 2} E_2^{*,*} \cong R^*\prod^{\Gamma}_i H_*(X_i; \mathbb{F}_p) &\Rightarrow H_*\left( \prod_i X_i; \mathbb{F}_p\right)\end{align}
for any set $\{ M_i: i\in I\}$ of spectra.
We will refer to \eqref{sadofsky ss 1} as the {\em Sadofsky spectral sequence,} and \eqref{sadofsky ss 2} as the {\em Hovey-Sadofsky spectral sequence.}
Countable products can of course be treated as a special case of sequential limits, so it is easy to see the countable case of the Hovey-Sadofsky spectral sequence as a special case of the Sadofsky spectral sequence.

It is a classical theorem of Adams (Theorem III.15.2 of \cite{MR1324104}) that mod $p$ homology commutes with homotopy limits of {\em uniformly bounded-below} sequences of spectra, i.e., $H_*\left( \holim_i X_i;\mathbb{F}_p\right) \rightarrow \lim_i H_*(X_i;\mathbb{F}_p)$ is an isomorphism if there is a {\em uniform} lower bound on the degrees of the nonvanishing homotopy groups of the spectra $X_0, X_1, X_2, \dots$. Sadofsky's spectral sequence is the only available general tool for calculating homology of homotopy limits of spectra in the absence of a uniform lower bound on their homotopy groups. 

However, to date there has been little known about the input of the Sadofsky or Hovey-Sadofsky spectral sequences, because it has been unclear whether there could be some practical means of calculating the derived functors of sequential limit, or of product, in categories of graded comodules. As Behrens and Rezk write in \cite{MR4094969} about the Sadofsky spectral sequence, ``[t]he $E_2$-term of this spectral sequence is in general quite mysterious'', and as Hovey writes in \cite{MR2337861} about the Hovey-Sadofsky spectral sequence, ``[a]lmost nothing is known about these right derived functors''. 
The purpose of this paper is to develop some general theory and some practical tools for calculating derived functors of sequential limits and of products in the category of graded comodules over a graded coalgebra, with a particular emphasis on the dual Steenrod algebras as our motivating examples.
Below, we survey the results in this paper, but first we present the most topologically compelling results, which appear as Corollaries \ref{derived products are local cohomology} and \ref{steenrod alg seq cor}:
\begin{unnumberedcorollary}
Let $p$ be a prime, let $\Gamma$ be the $p$-primary dual Steenrod algebra, and let $\{ M_i: i\in I\}$ be a set of graded $\Gamma$-comodules. Then the $n$th derived functor $R^n\prod_{i\in I}^{\Gamma} M_i$ of the product of the $M_i$ in the category of graded $\Gamma$-comodules is isomorphic, as an abelian group, to $\colim_m \Ext_{\Gamma^*}^n\left( \Gamma^*/I^m,\prod_{i\in I} \iota M_i\right)$, where $I$ is the augmentation ideal of the Steenrod algebra $\Gamma^*$, and where $\iota M_i$ is $M_i$ regarded as a graded $\Gamma^*$-module via the adjoint $\Gamma^*$-action. That is, the derived functors of product in the category of graded $\Gamma$-comodules are given by the local cohomology of the Steenrod algebra $\Gamma^*$ with coefficients in the product of the adjoint $\Gamma^*$-modules.
\end{unnumberedcorollary}
\begin{unnumberedcorollary}
Suppose that $p$ is a prime number. Write $\Gamma$ for the dual $p$-primary Steenrod algebra. Let \begin{equation*}%\label{seq 430949598c}
\dots\rightarrow M_2\rightarrow M_1\rightarrow M_0\end{equation*} be a sequence of graded $\Gamma$-comodules such that $R^1\lim$ vanishes on the sequence of graded $\Gamma^*$-modules 
$\dots\rightarrow \iota M_2\rightarrow \iota M_1\rightarrow \iota M_0$.
Then we have an isomorphism of graded $\Gamma^*$-modules
\[ \iota R^*\lim^{\Gamma}_i M_i  \cong \underset{n\rightarrow\infty}{\colim} \Ext^*_{\Gamma^*}\left(\Gamma^*/I^n,\lim_i \iota M_i\right)\]
where $I$ is the augmentation ideal of the $p$-primary Steenrod algebra $\Gamma^*$.
That is, if $R^1\lim$ vanishes on the sequence of adjoint $\Gamma^*$-modules, then the derived functors of sequential limit in the category of graded $\Gamma$-comodules are given by the local cohomology of the Steenrod algebra $\Gamma^*$ with coefficients in the limit of the adjoint $\Gamma^*$-modules.
\end{unnumberedcorollary}
These results reduce the calculation of the input of the Sadofsky and Hovey-Sadofsky spectral sequences to the calculation of local cohomology of the Steenrod algebra. Local cohomology is already a relatively familiar and well-studied theory, and there are already some computational tools for it, so we regard this identification of the Sadofsky and Hovey-Sadofsky spectral sequence $E_2$-terms as the main selling point of this paper.

Here are the other main results in this paper:
\begin{enumerate}
\item \Cref{Local cohomology functors associated...} covers elementary notions of local cohomology associated to a set of (left) ideals in a (not necessarily commutative) ring $R$. While local cohomology is well-studied over commutative rings, for the purposes of this paper we must consider local cohomology of {\em non}-commutative rings like the Steenrod algebra. Some familiar properties of local cohomology of commutative rings fail in surprising and treacherous ways in the absence of commutativity. Consequently we have to spend a large portion of this paper developing some necessary theory of local cohomology of noncommutative rings.

Given a set $S$ of ideals in $R$, we can consider the group $h^0_S(M)\coloneqq \colim_{I\in S} \hom_R(R/RI,M)$ of elements of an $R$-module $M$ which are $I$-torsion with respect to some ideal $I\in S$, or we can consider the $R$-submodule $H^0_S(M)$ of $M$ generated by the subgroup $h^0_S(M)$ of $M$. When $R$ is commutative, the functors $h^0_S$ and $H^0_S$ coincide, but for some noncommutative rings and some choices of $S$, they do not agree. Examples are given in Remark \ref{remark on ideal sets}. While $h^0_S$ is more familiar and computationally accessible, it is $H^0_S$ that carries the essential data about the relationship between comodules and modules, as we describe below.
\item \Cref{Comodules as a pretorsion...} reviews the well-known covariant embedding (via the adjoint action) of $\Gamma$-comodules into $\Gamma^*$-modules, and then reviews well-known ideas from torsion theory, and proves some preliminary results on the relationship between the two. A $\Gamma^*$-module $M$ is called {\em rational} if $M$ is in the image of this embedding. 
The main idea in \cref{Comodules as a pretorsion...} is that, given a coalgebra $\Gamma$ over a field, one can define a certain set $\dist(\Gamma)$ of ideals in $\Gamma^*$, the {\em distinguished ideals}, such that a $\Gamma^*$-module is rational if and only if the natural map $H^0_{\dist(\Gamma)}(M) \rightarrow M$ is an isomorphism. This is the content of Theorem \ref{theta-rationality and torsion}. 

One consequence is Corollary \ref{H0 is left exact}, which establishes that the functor $H^0_{\dist(\Gamma)}$ is left exact. (It is also true, but much easier to prove, that $h^0_{\dist(\Gamma)}$ is left exact.) Consequently we have some familiar homological tools for dealing with the right derived functors $H^*_{\dist(\Gamma)} \coloneqq R^*H^0_{\dist(\Gamma)}$, which we call {\em distinguished local cohomology.} Since the covariant embedding of comodules into modules is full and faithful, Theorem \ref{theta-rationality and torsion} characterizes $\Gamma$-comodules in terms of $\Gamma^*$-modules: the category of $\Gamma$-comodules is equivalent to the full subcategory of the $\Gamma^*$-modules generated by those $\Gamma^*$-modules $M$ such that $H^0_{\dist(\Gamma)}(M) \rightarrow M$ is an isomorphism.

I do not know of anywhere where Theorem \ref{theta-rationality and torsion} already appears in the literature, and I have never heard or seen others mention the idea. Still I am uneasy about calling it a terribly novel result: the work involved is a question of developing some (largely formal) theory and using it alongside ideas in sections 7 and 41 of \cite{MR2012570}. I regard \cref{Comodules as a pretorsion...} as a section devoted to background and review and developing some formal theory, and the hard ``non-formal'' work in this paper does not really begin until \cref{Distinguished local cohomology...}.
\item With \cref{Distinguished local cohomology...}, we begin to prove new results, at the cost of having to narrow the level of generality. Under the assumption that $\Gamma$ is a finite-type, connected coalgebra over a countable field, Theorem \ref{main thm 1} shows that the rational graded $\Gamma^*$-modules (i.e., the graded $\Gamma$-comodules) form a hereditary stable torsion class in graded $\Gamma^*$-modules. The same theorem furthermore shows that the higher distinguished local cohomology groups vanish on rational graded $\Gamma^*$-modules, and that the bounded-above $\Gamma^*$-modules are rational. 

Most importantly, Theorem \ref{main thm 1} shows (under the same assumptions) that for any integer $n$ and any set $\{ M_i: i\in I\}$ of graded $\Gamma$-comodules, the $n$th right-derived functor $R^n\prod_{i\in I}^{\Gamma} M_i$ of product in the category of graded $\Gamma$-comodules agrees (as a graded $\Gamma^*$-module via the adjoint action) with the distinguished local cohomology $H^n_{\dist(\Gamma)}\left(\prod_{i\in I} \iota(M_i)\right)$, where $\iota$ is the covariant embedding of comodules into modules. The point is that $\prod_{i\in I} \iota(M_i)$ is the {\em ordinary, familiar} Cartesian product of graded $\Gamma^*$-modules, not the more obscure categorical product in comodules. Consequently, if we reduce the distinguished local cohomology groups $H^n_{\dist(\Gamma)}$ to some familiar homological invariants, like a colimit of $\Ext$-groups, then we have a practical means of calculating $R^n\prod_{i\in I}^{\Gamma} M_i$, and consequently of calculating the homology groups of infinite products of spectra, via the Hovey-Sadofsky spectral sequence. Much of the rest of the paper is devoted to proving that $H^n_{\dist(\Gamma)}$ is indeed a straightforward colimit of $\Ext$-groups in sufficient generality to apply when $\Gamma$ is the dual Steenrod algebra at any prime.

One consequence of Theorem \ref{main thm 1} is Theorem \ref{structure thm}: if $\Gamma^*$ is finite-type and connected over a countable field, then for each integer $n$, every graded $\Gamma^*$-module $M$ is canonically an extension of a $n$-co-connected rational graded $\Gamma^*$-module by an $n$-connective graded $\Gamma^*$-module $\conn_n(M)$, and the higher distinguished local cohomology groups of $M$ depend only on those of $\conn_n(M)$. Under the same hypotheses, we then get Corollary \ref{modules are limits of comodules}, which states that every graded $\Gamma^*$-module is a limit of a Mittag-Leffler sequence of rational $\Gamma^*$-modules. We also get Corollary \ref{uniqueness of module cat}, which states that the only full subcategory of $\gr\Mod(\Gamma^*)$ which contains the rational $\Gamma^*$-modules and which is closed under kernels and countable products is $\gr\Mod(\Gamma^*)$ itself.
\item \Cref{Mitchell coalgebras...} defines the notion of a {\em Mitchell coalgebra}, a coalgebra over a field admitting a certain kind of decomposition which is intended to resemble the decomposition of the dual Steenrod algebra $A_*$ as the limit of the coalgebras $A(n)_*$. In \cite{MR793186}, S. Mitchell identified certain self-duality and compatibility properties of this decomposition of $A_*$, and we include analogous properties as part of the definition of a Mitchell coalgebra in Definition \ref{def of mitchell condition}, because we find in Theorem \ref{main thm on mitchell coalgebras} that precisely these properties can be used to show that the two local cohomology functors $h^*_{\dist(\Gamma)}$ and $H^*_{\dist(\Gamma)}$ coincide over the dual of a Mitchell coalgebra $\Gamma$. 
As a consequence, the distinguished local cohomology $H^*_{\dist(\Gamma)}$---which, by Theorem \ref{main thm 1}, computes the derived functors of product in the category of $\Gamma$-comodules---has the good computational properties of $h^*_{\dist(\Gamma)}$, and in particular, it is isomorphic to a colimit of $\Ext$-groups.

Since Mitchell's results in \cite{MR793186} establish that the dual Steenrod algebra is what we call a Mitchell coalgebra, we get the main results of this paper: if $\Gamma$ is the dual Steenrod algebra at any prime, then Corollary \ref{main cor 10} establishes that, for any graded $\Gamma^*$-module $M$, the distinguished local cohomology $H^*_{\dist(\Gamma)}(M)$ is isomorphic to $\colim_m \Ext_{\Gamma^*}^*\left( \Gamma^*/I^m,M\right)$, where $I$ is the augmentation ideal of the Steenrod algebra $\Gamma^*$. A consequence is Corollary \ref{derived products are local cohomology}: if $\{ M_i: i\in I\}$ is a set of graded $\Gamma$-comodules, then the $n$th derived functor $R^n\prod^{\Gamma}_{i\in I} M_i$ of product in the category of graded $\Gamma$-comodules is isomorphic to $\colim_m \Ext_{\Gamma^*}^n\left( \Gamma^*/I^m,\prod_i M_i\right)$, where $\prod_i M_i$ is the product (in the category of {\em modules}, i.e., the Cartesian product) of the graded $\Gamma^*$-modules $M_i$ with the adjoint action of $\Gamma^*$.
\item An abelian category satisfies Grothendieck's axiom $AB4^*$ if and only if products exist and are exact in that category. Weakenings of $AB4^*$ have been studied: given an integer $n$, an abelian category $\mathcal{C}$ is said to satisfy axiom $AB4^*\mhyphen (n)$ if and only if the $m$th derived functor of product in $\mathcal{C}$ vanishes for all $m > n$. The condition $AB4^*\mhyphen (0)$ is equivalent to $AB4^*$. When $\Gamma^*$ is the Steenrod algebra at some prime, one knows (e.g. from experience with the Adams spectral sequence) that $\Ext_{\Gamma^*}^{n}(M,N)$ is capable of being nonzero for arbitrarily large $n$, but this does not rule out the possibility of a finite bound on the integers $n$ such that $\colim_m \Ext_{\Gamma^*}^{n}(\Gamma^*/I^m,N)$ is nonzero for some $N$. In other words, it seems plausible that the category of graded $\Gamma$-comodules satisfies axiom $AB4^*\mhyphen (n)$ for some $n$, and consequently that there is a uniform horizontal vanishing line in the $E_2$-term for the Hovey-Sadofsky spectral sequence calculating the mod $p$ homology of infinite products of spectra.

The purpose of \cref{Bounds on distinguished...} is to show that we are not, in fact, so lucky: Corollary \ref{no ab4n for dual steenrod alg} shows that the category of comodules over the dual Steenrod algebra, at any prime, cannot satisfy axiom $AB4^*\mhyphen (n)$ for any $n$ at all. This is a consequence of Theorem \ref{main thm 4}, which shows that, if $\Gamma$ is a coalgebra over a countable field such that $\Gamma^*$ is finite-type and connected and such that every bounded-below free graded $\Gamma^*$-module is graded-injective, then the category of graded $\Gamma$-comodules satisfies $AB4^*\mhyphen (n)$ for some $n$ if and only if it satisfies $AB4^*$. The key to applying Theorem \ref{main thm 4} to the dual Steenrod algebra is a theorem of Margolis: Margolis has proven that the Steenrod algebras (more generally, the ``$\mathcal{P}$-algebras'' in Margolis' sense) have the property that their bounded-below free graded modules are graded-injective. We also provide \cref{Review of Margolis...}, which is devoted to review of Margolis' basic theorems on graded modules over $\mathcal{P}$-algebras.
\item One consequence of Theorem \ref{main thm 1} is that higher distinguished local cohomology vanishes on rational modules. However, it is a bit too glib to express this by a slogan like ``Distinguished local cohomology detects the failure of a module to be a comodule,'' since there are also some modules with vanishing distinguished local cohomology which still fail to be rational. Bounded-below free graded modules over the Steenrod algebra are an example of this phenomenon. See the remarks preceding Theorem \ref{structure thm} for further discussion. 

Consequently, one wants a classification of the modules whose higher distinguished local cohomology vanishes. Such a classification is the purpose of \cref{Characterization of the modules...}, whose main result is Theorem \ref{decomposition of acyclics}: if $\Gamma$ is a Mitchell coalgebra over a countable field, then a graded $\Gamma^*$-module $M$ has vanishing higher distinguished local cohomology if and only if $M$ admits graded $\Gamma^*$-submodules $M^{\prime\prime}\subseteq M^{\prime}\subseteq M$ such that $M^{\prime\prime}$ is bounded below and distinguished-torsion, such that $M^{\prime}/M^{\prime\prime}$ is bounded below and has the property that $\Ext^n_{\Gamma^*}(A,M^{\prime}/M^{\prime\prime})$ vanishes for all $n$, and such that $M/M^{\prime}$ is bounded above.
\item While Theorem \ref{main thm 1} shows that $R^n\prod_{i\in I}^{\Gamma} M_i$ agrees with the distinguished local cohomology group $H^n_{\dist}(\prod_{i\in I}\iota M_i)$, one would like to have a similar theorem for calculating more general derived limits, not just derived products, in comodule categories. \Cref{Derived functors of seq...} addresses that problem, at least for sequential limits. Theorem \ref{derived sequential lim thm} establishes that, when $\Gamma$ is a finite-type coalgebra over a countable field such that $\Gamma^*$ is connected, the derived functors of sequential limit
$R^*\lim^{\Gamma}_iM_i$ in the category of graded $\Gamma$-comodules agree with the distinguished local cohomology groups $H^*_{\dist(\Gamma)}(\lim_i \iota M_i)$ as long as the sequence of graded $\Gamma^*$-modules $\dots \rightarrow \iota M_2 \rightarrow \iota M_1 \rightarrow \iota M_0$ has vanishing $\lim^1$. It is an important and convenient point that this $\lim^1$-vanishing condition is checked in the {\em module} category, not in the {\em comodule} category, so it is relatively easy to check: one can simply verify that the Mittag-Leffler condition holds, for example.

Consequently, we have Corollary \ref{steenrod alg seq cor}: given a sequence $\dots \rightarrow M_2 \rightarrow M_1 \rightarrow M_0$ of graded comodules over the dual Steenrod algebra at any prime, if the sequence of adjoint modules $\dots \rightarrow \iota M_2 \rightarrow \iota M_1 \rightarrow \iota M_0$ has vanishing $\lim^1$, then the derived limit $R^*\lim^{\Gamma}_i M_i$ in the category of comodules over the dual Steenrod algebra agrees with the colimit of $\Ext$-groups $\colim_{n\rightarrow\infty}\Ext^*_{\Gamma^*}\left(\Gamma^*/I^n,\lim_i \iota M_i\right)$ over the Steenrod algebra $\Gamma^*$.
\item \Cref{...generalization of Megibben...} is devoted to proving a graded generalization of a 1982 theorem of C. Megibben. Megibben's theorem, from \cite{MR633266}, establishes that every countable injective module $M$ has the property that every direct sum of copies of $M$ is also injective. Over a Noetherian ring, this is a trivial consequence of the Bass-Papp theorem, but over non-Noetherian rings, Megibben's theorem has some ``teeth.'' The failure of the Steenrod algebra (and the duals of various other coalgebras which are treated by Theorem \ref{main thm 1}) to be Noetherian means the Bass-Papp theorem is not sufficient to establish that direct sums of copies of the Steenrod algebra are injective. As a consequence, at one point in the proof of Theorem \ref{main thm 1}, we need a version of Megibben's theorem which applies to graded-injective modules over graded rings. That graded version of Megibben's theorem is our Theorem \ref{graded megibbens thm}. See the remarks at the beginning of \cref{...generalization of Megibben...} for why the original (ungraded) theorem of Megibben is not enough for our purposes. \Cref{...generalization of Megibben...} does not logically depend on anything earlier in the paper, so it does not constitute a logical problem for the proof of Theorem \ref{main thm 1} to use a result (namely, Theorem \ref{graded megibbens thm}) in \cref{...generalization of Megibben...}.
\item \Cref{Review of Margolis...} reviews some results of Margolis, from \cite{MR738973}, on module theory over $\mathcal{P}$-algebras such as the Steenrod algebras. Those results of Margolis are used in several proofs in this paper, so it is useful to have an appendix dedicated to their review. There are no new results in \cref{Review of Margolis...}. 
\end{enumerate}


\begin{remark}
Many cases of the results of this paper can be interpreted as having stack-theoretic content. Suppose we are given a commutative Hopf algebra $\Gamma$ over a commutative ring $A$. If $\Gamma$ is smooth over $A$, then the category of $\Gamma$-comodules is equivalent to the category of quasicoherent $\mathcal{O}_{B\mathbb{G}}$-modules, where $\mathcal{O}_{B\mathbb{G}}$ is the structure sheaf of the fppf site of the Artin stack $B\mathbb{G}$ of $\mathbb{G}$-torsors. Here $\mathbb{G}$ is the group scheme represented by $\Spec\Gamma$. In light of this, the question ``For what $n$ does the $n$th right derived functor $R^n\prod_I^{\Gamma}: \Comod(\Gamma)^I\rightarrow\Comod(\Gamma)$ vanish?'' becomes the question ``For what $n$ does the $n$th right derived functor $R^n\prod_I: QC\Mod(B\mathbb{G})^I\rightarrow QC\Mod(B\mathbb{G})$ vanish?'' %at least when $\Gamma$ is a commutative Hopf algebra smooth over $A$. 
This latter question was investigated for Deligne-Mumford stacks in \cite{hogadixu2009}, so the present paper could be seen as, in part, handling for certain Artin stacks the same problem that was investigated for Deligne-Mumford stacks in \cite{hogadixu2009}. But of course our motivations and main applications in this paper are really topological, rather than algebro-geometric. 

Artin stacks are more general than Deligne-Mumford stacks, so the results in this paper are not special cases of those in \cite{hogadixu2009}. Indeed, the results and the methods obtained in this paper are entirely different from the results and the methods involved in \cite{hogadixu2009}. 
\end{remark}

\begin{remark}
This paper examines the derived functors of products in categories of comodules over a coalgebr{\em a}, not a more general coalgebr{\em oid}, for example a Hopf algebroid. The Hovey-Sadofsky spectral sequence has a version for a generalized homology theory $E_*$ whose input is the derived functor of product in the category of graded $(E_*,E_*E)$-comodules, where now $(E_*,E_*E)$ is a Hopf algebroid and not typically a Hopf algebra (or coalgebra). So there is good topological motivation to try to prove Hopf algebroid analogues of the results in the present paper. A similar remark is also true with sequential limits in place of products throughout. %In particular, letting $E$ be the Morava $E$-theory spectrum of a height $1$ one-dimensional formal group over a finite field (for example, $p$-complete periodic complex $K$-theory is weakly equivalent to such a Morava $E$-theory spectrum), I imagine that an appropriate Hopf-algebroid-theoretic analogue of the final part of Theorem \ref{main thm 1} would establish that the derived functors of the product in the category of graded $(E_*,E_*E)$-comodules are isomorphic to appropriate local cohomology groups in a certain category of Iwasawa modules; ...
\end{remark}

\begin{conventions}\label{conventions}\leavevmode
\begin{itemize}
\item Unless otherwise specified, our modules will be left modules, and our comodules will be right comodules.
\item 
Throughout, when we have a coalgebra over a commutative ring, we use the standard notations from the theory of Hopf algebroids (from the influential first appendix of \cite{MR860042}, for example): we write $A$ for the commutative ring, and we write $\Gamma$ for the coalgebra. 
\item When we have a ring $R$ and a left ideal $I$ of $R$, we will write $RI$ for the ideal $I$ regarded as a left $R$-module. Consequently $R/RI$ denotes the left $R$-module given by the cokernel of the inclusion $RI \hookrightarrow R$. While writing $R/RI$ rather than $R/I$ may seem annoying to some readers, it is a common convention, e.g. as in \cite{lam2013first}, and this convention avoids some notational ambiguities: for example, when $A$ is the Steenrod algebra, if we were to write $\hom_A(A/(\Sq^1,\Sq^2,\dots),M)$, it could leave the reader uncertain whether $A/(\Sq^1,\Sq^2,\dots)$ means $A$ modulo the two-sided ideal generated by $\{\Sq^1,\Sq^2,\dots\}$, or the much larger quotient module given by $A$ modulo only the {\em left} ideal generated $\{\Sq^1,\Sq^2,\dots\}$. It is convenient to have the notation $A/A(\Sq^1,\Sq^2,\dots)$ reserved for the latter meaning.
\item At many places in this paper, we give arguments involving annihilators over noncommutative rings, such as Steenrod algebras. These arguments require a bit of care, because some of the nice behavior of annihilators in commutative algebra does not carry over to the noncommutative setting. 
We will follow the notational conventions for annihilators from \cite{lam2013first}: if $R$ is a ring and $M$ is a (left) $R$-module, then $\ann(M)$ denotes the set $\{ r\in R: rm=0\ \forall m\in M\}$, which is a {\em two-sided} ideal in $R$. If $S$ is a subset of $M$, we write $\ann_{\ell}(S)$ for the subset $\{ r\in R: rs=0\ \forall s\in S\}$, which is in general only a {\em left} ideal of $R$. Of course this distinction is only necessary when $R$ is noncommutative. 

In particular, if $M$ is a cyclic left $R$-module, then $M$ is isomorphic to $R/R\ann_{\ell}(g)$, where $g$ is a generator for $M$. We have equalities $R/R\ann(M) = R/R\ann_{\ell}(M)$, but $R/R\ann(M)$ and $R/R\ann_{\ell}(M)$ are {\em not} necessarily isomorphic to $R/R\ann_{\ell}(g)$, hence not necessarily isomorphic to $M$. The discussion in section 2.4 of \cite{lam2013first} is an excellent textbook treatment of this issue. %(I apologize for belaboring this elementary point, but I suspect that readers of this paper are likely to have thought far more about annihilators over commutative rings, and so this distinction that occurs in the noncommutative case---which plays an important role in the proof of Proposition \ref{filteredness prop}, for example---is liable to be missed.)
\item All gradings in this paper are $\mathbb{Z}$-gradings whenever not otherwise stated.
\item Given a graded ring $R$ and graded $R$-modules $M$ and $N$, we write $\hom_R(M,N)$ for the degree-preserving $R$-linear morphisms $M\rightarrow N$, and we write $\underline{\hom}_R(M,N)$ for the graded abelian group whose degree $n$ summand\footnote{Some references, e.g. \cite{brunerprimer}, use the opposite grading on $\underline{\hom}_R$---hence the need to give our grading conventions explicitly. The argument for our choice of gradings is that it is the unique one so that $\hom_R(L,\underline{\hom}_R(M,N)) \cong \hom_R(L\otimes_R M,N)$.} is $\hom_R(\Sigma^n M,N)$.
\end{itemize}
\end{conventions}

\begin{comment}
(gggxxxxx GO THROUGH AGAIN AND REPHRASE TO BE CAREFUL ABOUT ANNIHILATORS: WHEN $R$ IS NONCOMMUTATIVE, THE ANNIHILATOR OF $R/RI$ IS THE LARGEST TWO-SIDED IDEAL CONTAINED IN THE LEFT IDEAL $I$!)

(gggxxxxx GO THROUGH AGAIN AND REPHRASE TO BE CAREFUL ABOUT THE DISTINCTION BETWEEN ADJOINT ACTION AND CONTRAGREDIENT ACTION: CONTRAGREDIENT ONLY MAKES SENSE OVER A HOPF ALGEBRA, SINCE YOU NEED AN ANTIPODE.)

(gggxxxxx GO THROUGH AGAIN AND MAKE SURE WE DON'T USE EXTENDED COMODULES EXCEPT WHEN THE COALGEBRA ACTUALLY HAS A UNIT MAP, I.E., IS ACTUALLY A BIALGEBRA.)
\end{comment}

\section{Local cohomology functors associated to sets of ideals.}
\label{Local cohomology functors associated...}
This section consists of elementary notions about generalized local cohomology functors associated to sets of ideals in some (not necessarily commutative) ring. We do not claim originality for the ideas in this section, but we do not know anywhere where this sequence of ideas already appears in the literature.

\begin{definition}\label{def of ideal sets}
Let $R$ be a graded ring. 
\begin{itemize}
\item By an {\em ideal set in $R$} we mean a set $S$ of homogeneous proper left ideals of $R$. %which is filtered under inclusion, that is, if $I,J\in S$, then there exists an element of $S$ contained in both $I$ and $J$.
\item We will say that an ideal set $S$ in $R$ is {\em connected} if $S$ is connected as a partially ordered set under inclusion, i.e., if $I,J\in S$, then there exists a finite sequence $I = I_0, J_0, I_1, J_1, \dots ,I_{n-1},J_{n-1},I_n,J_n = J$ of elements of $S$ such that $I_h \subseteq J_h$ and $I_{h+1}\subseteq J_h$ for all $h$.
\item We will say that an ideal set $S$ in $R$ is {\em filtered} %if $S$ is filtered as a partially-ordered set under inclusion, i.e., if 
if, for each $I,J\in S$, there exists an element of $S$ contained in both $I$ and $J$.
\item If $S,S^{\prime}$ are ideal sets in $R$, we write $S\leq S^{\prime}$ if, for every element $I$ of $S$, there exists some element $J\in S^{\prime}$ such that $I\supseteq J$. (Intuitively, this says that $S\leq S^{\prime}$ iff $S^{\prime}$ is ``finer'' than $S$.) The relation $\leq$ on ideal sets is transitive and reflexive, hence the collection of ideal sets in $R$ is a preorder. We write $\idealsets(R)$ for this preorder.
\item We say that ideal sets $S$ and $S^{\prime}$ are {\em equivalent} if $S\leq S^{\prime}$ and $S^{\prime}\leq S$.
\item Given an ideal set $S$ in $R$, let $\overline{S}$ denote the set of all intersections of finite sets of members of $S$. We call $\overline{S}$ the {\em filtered closure} of $S$. The filtered closure of $S$ is a filtered ideal set in $R$, and $S\leq \overline{S}$. Furthermore, $\overline{S}$ is minimal (up to equivalence) with that property. That is, if $T$ is any filtered ideal set in $R$ such that $S\leq T$, then $\overline{S}\leq T$.
\item Given an ideal set $S$, we have two associated degree zero local cohomology functors: first, we have the functor
\begin{align*}
 h^0_S: \gr\Mod(R) & \rightarrow \gr\Ab \\
        h^0_S(M) &= \colim_{I\in S} \underline{\hom}_R(R/RI,M). \end{align*}
In particular, if $S$ is filtered, then $h^0_S(M)$ is the graded subgroup of $M$ generated by all homogeneous elements which are $I$-torsion for some element $I$ of $S$. 

We write $h^n_S$ for the $n$th right derived functor $R^nh^0_S$ of $h^0_S$.

Meanwhile, when $S$ is filtered, we also have the functor
\begin{align*}
 H^0_S: \gr\Mod(R) & \rightarrow \gr\Mod(R) \end{align*}
given by letting $H^0_S(M)$ be the graded left $R$-submodule of $M$ generated by the subgroup $h^0_S(M)$ of $M$.
We write $H^n_S$ for the $n$th right derived functor $R^nH^0_S$ of $H^0_S$.
\end{itemize}
\end{definition}
Note that, if $S\leq S^{\prime}$ and $M$ is a graded $R$-module, then every homogeneous element of $M$ which is $I$-torsion for some $I\in S$ is also $J$-torsion for some $J\in S^{\prime}$. If $S^{\prime}$ is also assumed to be filtered, then we have a well-defined map from $\underline{\hom}_R(R/RI,M)$ to $\colim_{J\in S^{\prime}}\underline{\hom}_R(R/RJ,M)$. Consequently, if $S\leq S^{\prime}$ and $S^{\prime}$ is filtered, we get a canonical choice of natural transformation $h^0_S \rightarrow h^0_{S^{\prime}}$.

\begin{remark} \label{remark on ideal sets}
Here are some examples, non-examples, and observations to illustrate the various notions defined in Definition \ref{def of ideal sets}.
\begin{description}
\item[Why connectedness matters]
If an ideal set $S$ is not connected, we can still make the definition $h^0_S(M) \coloneqq \colim_{I\in S} \underline{\hom}_R(R/RI,M)$, but the resulting abelian group $h^0_S(M)$ can fail to be a subgroup of $M$. For example, let $k$ be a field, and let $R = k[x,y]$. Let $S$ be the ideal set $\left\{ (x),(y)\right\}$. Then $h^0_S(M)$ is the direct sum of the $x$-torsion subgroup of $M$ and the $y$-torsion subgroup of $M$. This is a subgroup of $M\oplus M$, but it is not a subgroup of $M$ itself in any natural way.

The filtered closure $\overline{S}$ of $S$ is $\{ (x), (y), (xy)\}$, so $h^0_{\overline{S}}(M)$ is the $(xy)$-torsion subgroup of $M$, i.e., the set of elements $m\in M$ such that $xym=0$. This {\em is} a subgroup of $M$. What we saw in this example is one case of a general phenomenon: a filtered partially-ordered set is automatically connected, so $h^0_{\overline{S}}(M)$ is a subgroup of $M$ for any ideal set $S$. 

This is also an example of an ideal set $S$ such that $h^0_S$ is not isomorphic to $h^0_{\overline{S}}$. In general, there is no reason to expect $h^0_S$ to coincide with $h^0_{\overline{S}}$.
\item[When $h^0_S$ and $H^0_S$ differ]
Suppose $S$ is filtered. 
When $R$ is commutative, then $\underline{\hom}_R(R/RI,M)$ is in fact an $R$-submodule of $M$, and not only a subgroup. Put another way, when $R$ is commutative, the $I$-torsion in a given $R$-module is a submodule and not merely a subgroup. Consequently the natural transformation $h^0_S\rightarrow H^0_S$ is an isomorphism for all filtered ideal sets $S$, when $R$ is commutative.

On the other hand, when $R$ is noncommutative, $h^0_S$ may differ from $H^0_S$. Here is an explicit example where $h^0_{S}(M)$ fails to be an $R$-submodule of $M$, and consequently $h^0_{S}(M)$ fails to agree with $H^0_{S}(M)$. Let $R$ be the subalgebra $A(1)$ of the mod $2$ Steenrod algebra generated by $\Sq^1$ and $\Sq^2$. Let $S$ be the left ideal in $A(1)$ generated by $\Sq^1$. Then $h^0_{S}(M)$ is simply the graded subgroup of $M$ consisting of the elements of $M$ annihilated by $\Sq^1$. 
In the case $M = A(1)$, we have $\Sq^1\in h^0_{S}(A(1))$, since $\Sq^1\Sq^1 = 0$ in $A(1)$. However, $\Sq^2\Sq^1 \notin h^0_{S}(A(1))$, since $\Sq^1\Sq^2\Sq^1 \neq 0$ in $A(1)$. So $h^0_{S}(A(1))$ is not closed under left $A(1)$-multiplication in $A(1)$, i.e., $h^0_{S}(A(1))$ is not a left $A(1)$-submodule of $A(1)$. Hence $h^0_{S}(A(1))$ fails to coincide with $H^0_{S}(A(1))$: the latter contains $\Sq^2\Sq^1$, while the former does not.
\item[Why equivalence matters] 
If filtered ideal sets $S,S^{\prime}$ in $R$ satisfy $S\leq S^{\prime}$, then the natural monomorphism $h^0_S(M)\hookrightarrow M$ factors uniquely through the natural monomorphism $h^0_{S^{\prime}}(M)\hookrightarrow M$. Consequently we get natural transformations $h^n_S\rightarrow h^n_{S^{\prime}}$ and $H^n_S\rightarrow H^n_{S^{\prime}}$ for each $n$.

If the filtered ideal sets $S$ and $S^{\prime}$ are equivalent, then we have $h^0_S(M) = h^0_{S^{\prime}}(M)$ as subgroups of $M$, so we get natural isomorphisms $h^n_S \cong h^n_{S^{\prime}}$ and $H^n_S \cong H^n_{S^{\prime}}$ for all $n$ as well.
\item[Left exactness of $h^0_S$ and $H^0_S$]
We have much better tools for understanding and calculating the right derived functors $h^*_S$ of $h^0_S$ when we know that $h^0_S$ is left exact. If the ideal set $S$ is filtered, then colimits over $S$ are exact, so
$h^0_S(-) = \colim_{I\in S}\underline{\hom}_R(R/RI,-)$ is indeed left exact. 

It is much less obvious that $H^0_S$ is left exact. This is a topic we take up later, using tools from torsion theory, in Corollary \ref{H0 is left exact}.
\end{description}
\end{remark}

\begin{example} \label{examples of ideal sets} 
{\bf (Important examples of ideal sets.)}
\begin{itemize}
\item The famous case of $H^0_{S}(M)$ is the case where $R$ is commutative and concentrated in degree zero, $I$ is an ideal of $R$, and $S$ is the set $S = \{ I, I^2, I^3, \dots\}$ of powers of $I$. In that case, $H^0_{S}(M)$ is simply the classical local cohomology $H^0_I(M)$ of $M$ at the ideal $I$, in the sense of \cite{MR0222093} and \cite{MR3014449}, among many other references. 

The reason for introducing $h^0_{S}$ and $H^0_{S}$ in Definition \ref{def of ideal sets} is that, when generalizing local cohomology from the classical setting to a setting where $R$ is not necessarily commutative, $h^0_{S}$ and $H^0_{S}$ can differ. This may lead to confusion: if, for example, $R$ is noncommutative and $S$ is the set $\{ I,I^2,I^3, \dots\}$ of powers of some left ideal $I$ of $R$, then $h^0_{S}(M) = \colim_n \underline{\hom}_R(R/RI^n,M)$ is generally only a subgroup of $M$, not necessarily an $R$-submodule of $M$, so some familiar arguments from classical local cohomology no longer work. For example, it no longer even makes sense to ask whether $h^0_{S}:\gr\Mod(R)\rightarrow\gr\Ab$ is idempotent. On the other hand, we can ask whether $H^0_{S}:\gr\Mod(R)\rightarrow\gr\Mod(R)$ is idempotent, and $H^0_S$ has some very desirable properties (see Theorem \ref{theta-rationality and torsion}, for example), but the derived functors $H^n_S$ (unlike $h^n_S$) are not generally isomorphic to colimits of $\Ext$-groups, so it can be much less clear how to actually calculate them. 
\item The following example is trivial, but helps to build intuition for $\idealsets(R)$. 
The preorder $\idealsets(R)$ has a maximal element, which can be taken to be the ideal set $\{ (0)\}$ consisting of simply the zero ideal of $R$. This ideal set is equivalent to any other ideal set in $R$ containing the zero ideal, for example the ideal set consisting of all proper homogeneous left ideals of $R$. For all graded $R$-modules $M$, we have $h^0_{\{(0)\}}(M) = M = H^0_{\{(0)\}}(M)$, so $h^n_{\{(0)\}}(M) = M = H^n_{\{(0)\}}(M)$ for all $n>0$. 

The point is that the maximal element in $\idealsets(R)$ is the element whose associated local cohomology theory has $h^0(M)$ consisting of {\em all of $M$}. The partial order $\leq$ on $\idealsets(R)$ has the property that {\em smaller filtered elements $S$ of $\idealsets(R)$ make $h^0_S(M)$ a smaller subgroup of $M$.}
\item Let $R$ be a graded ring, and let $\Theta$ be a graded left $R$-module. Let $\homog(\Theta)$ be the set of homogeneous elements of $\Theta$, and for each $m\in \homog(\Theta)$, let $\ann_{\ell}(m)$ be the left annihilator of $m$. Let $\dist(\Theta)$ be the ideal set $\{ \ann_{\ell}(m) : 0\neq m \in \homog(\Theta)\}$ of $R$. We refer to the members of $\dist(\Theta)$ as the {\em strongly $\Theta$-distinguished ideals of $R$}, and we refer to the members of the filtered closure $\overline{\dist(\Theta)}$ as the {\em $\Theta$-distinguished ideals of $R$.}
\item Let $R$ be an $\mathbb{N}$-graded ring, and let $\grad$ be the ideal set $\{ I_1, I_2, I_3, \dots\}$ in $R$, where $I_n$ is the left ideal (equivalently, two-sided ideal) in $R$ generated by all homogeneous elements of degree $\geq n$. Note that $\grad$ is filtered.
\end{itemize}
\end{example}

\section{Comodules as a pretorsion class in modules.}
\label{Comodules as a pretorsion...}

This section covers preliminary notions on the covariant embedding of $\Gamma$-comodules into $\Gamma^*$-modules, as well as preliminary notions on (pre)torsion theories. Our goal is to show that $\Gamma^*$-modules are a torsion class in $\Gamma^*$-modules, perhaps a torsion class with particularly good properties. Indeed this is true (at least when the ground ring $A$ is a countable field), but we do not prove it until the start of the next section, in Theorem \ref{main thm 1}.

\subsection{Review of the embedding of the category of $\Gamma$-comodules into the category of $\Gamma^*$-modules.}

Let $A$ be a commutative ring, and let $\Gamma$ be a graded $A$-coalgebra which is flat as an $A$-module. Let $\Gamma^*$ denote the $A$-linear dual graded algebra $\underline{\hom}_A(\Gamma,A)$ of $\Gamma$. %, and let $\Gamma^{\vee}$ denote $\Gamma^*$ with the gradings reversed, i.e., the degree $n$ summand $(\Gamma^{\vee})^n$ of $\Gamma^{\vee}$ is the degree $-n$ summand $(\Gamma^*)^n$ of $\Gamma^*$.
 We have a well-known functor
\begin{equation}
 \label{comodule inclusion functor}
 \iota:\gr\Comod(\Gamma) \rightarrow \gr\Mod(\Gamma^*)
\end{equation}
given by sending a graded $\Gamma$-comodule $M$ to the graded $A$-module $M$, equipped with the ``adjoint'' $\Gamma^*$-action. The adjoint left $\Gamma^*$-action\footnote{To avoid possible confusion, we remark that when $\Gamma$ is not only a coalgebra but a finite-type Hopf algebra, then $\Gamma\cong \Gamma^{**}$ is also a graded $\Gamma^*$-module via the {\em contragredient} action, which differs from the adjoint action by an application of the antipode of $\Gamma^*$.} is the action $\Gamma^*\otimes_A M \rightarrow M$ given by sending $f\otimes m$ to the image of $m$ under the composite map 
\[ M \stackrel{\psi_M}{\longrightarrow} M\otimes_A \Gamma \stackrel{M\otimes f}{\longrightarrow} M\otimes_A A \stackrel{\cong}{\longrightarrow} M.\]
The graded $\Gamma^*$-modules in the essential image of the functor $\iota$ are called {\em rational} modules in the literature, e.g. in the textbook \cite{MR2012570}. The use of the term ``rational'' here is well-established, so we use that term in this paper\footnote{There is some risk of confusion here: topologists, like the author of this paper, are used to the term ``rational'' being reserved for abelian groups which are vector spaces over $\mathbb{Q}$, or spectra which are modules over $H\mathbb{Q}$, or other small variations on this theme. That use of the term ``rational'' is simply unlike the use of the term ``rational'' in the coalgebra literature and in this paper. %, that is, a module which is the contragredient dual of a $\Gamma$-comodule. 
Rational (in the sense of coalgebra) modules over the Steenrod algebra are one of the main objects of study in this paper, but they do not arise as the homology or cohomology of rational (in the sense of topology) spaces or spectra, except in trivial cases.}. 

The following theorem summarizes some established properties of the functor \eqref{comodule inclusion functor}; see 4.1, 4.3, 4.7, 7.1, and 20.1 of \cite{MR2012570} for a textbook treatment.
\begin{theorem}\label{review thm} The following claims are each true:
\begin{enumerate}
\item The functor $\iota$ is faithful and its image lies in the full subcategory of $\gr\Mod(\Gamma^*)$ consisting of all graded $\Gamma^*$-modules $M$ such that $M$ is a graded submodule of a graded quotient module of a coproduct of copies of suspensions of $\iota\Gamma$. This full subcategory of $\gr\Mod(\Gamma^*)$ is denoted $\sigma[{}_{\Gamma^*}\Gamma]$.
\item The resulting functor $\gr\Comod(\Gamma)\rightarrow \sigma[{}_{\Gamma^*}\Gamma]$ is full if and only if $\Gamma$ is locally projective as an $A$-module. Consequently, when $\Gamma$ is locally projective as an $A$-module, we may regard $\gr\Comod(\Gamma)$ as (via the functor $\iota$) a full subcategory of $\gr\Mod(\Gamma^*)$. 
\item If $\Gamma$ is locally projective as an $A$-module, then the functor $\iota$ is full, faithful, and admits a right adjoint $\tr: \gr\Mod(\Gamma^*)\rightarrow \gr\Comod(\Gamma)$, called the ``rational functor'' or the ``trace functor,'' and which is given on a graded $\Gamma^*$-module $M$ by letting $\tr(M)$ be the graded $\Gamma^*$-submodule $M$ generated by the homogeneous elements $m$ such that $m$ is in the image of a graded $\Gamma^*$-module homomorphism from a graded module in the image of $\iota$ to $M$.
\item The functor $\iota: \gr\Comod(\Gamma) \rightarrow \gr\Mod(\Gamma^*)$ is an equivalence of categories if and only if $\Gamma$ is finitely generated and projective as an $A$-module. 
\item In particular, suppose that $\Gamma$ is projective as an $A$-module. Then the rational graded $\Gamma^*$-modules form a full abelian subcategory of the graded $\Gamma^*$-modules, closed under kernels, cokernels, and coproducts. Furthermore, the following are equivalent:
\begin{itemize}
\item Every graded $\Gamma^*$-module is rational.
\item $\Gamma$ is finitely generated as an $A$-module. 
\end{itemize}
\end{enumerate}
\end{theorem}
%We have the functor $\Rat: \gr\Mod(\Gamma^*) \rightarrow \gr\Comod(\Gamma)$ and the functor $\iota: \gr\Comod(\Gamma)\hookrightarrow \gr\Mod(\Gamma^*)$.
%While I do not know where it is explicitly stated in the literature, it is easy to see that the composite $\iota\circ \Rat$ is naturally isomorphic to the identity, while the composite $\Rat\circ\iota$ is idempotent.


In Proposition \ref{rationals closed under quots and subs}, we offer a proof of part of Theorem \ref{review thm}, in order to make it clear (as explained in Remark \ref{rationals closed under quots and subs 2}) that the proof generalizes from rational modules to $\Theta$-rational modules.
\begin{prop} \label{rationals closed under quots and subs}
Suppose that $\Gamma^*$ is projective as an $A$-module. Then the following claims are each true:
\begin{enumerate}
\item 
Every coproduct of rational graded $\Gamma^*$-modules is rational.
\item 
Every graded submodule of a rational graded $\Gamma^*$-module is rational.
\item
Every graded quotient of a rational graded $\Gamma^*$-module is rational.
\end{enumerate}
\end{prop}
\begin{proof}\leavevmode
\begin{itemize}
\item[(1) and (2)]
It follows immediately from the definition of $\sigma[{}_{\Gamma^*}\Gamma]$ that it is closed under coproducts and graded submodules. By Theorem \ref{review thm}, $\sigma[{}_{\Gamma^*}\Gamma]$ agrees with the category of rational graded $\Gamma^*$-modules.
\item[(3)]
Suppose that $Q$ is a graded quotient of a rational graded $\Gamma^*$-module $M$. Write $F$ for a coproduct of suspensions of $\Gamma$ into which $M$ embeds. We have the diagram of graded $\Gamma^*$-modules
\[\xymatrix{
 M \ar@{^{(}->}[r] \ar@{->>}[d] & F \\ Q & 
}\]
and, taking the pushout of the diagram, we get a graded $\Gamma^*$-module into which $Q$ embeds, and which is a quotient of $F$, since epimorphisms and monomorphisms are each stable under pushouts in a category of modules over a ring. Hence $Q$ is in $\sigma[{}_{\Gamma^*}\Gamma]$, hence $Q$ is rational.
\end{itemize}
\end{proof}

\begin{definition}
Let $R$ be a graded ring, and let $\Theta$ be a graded left $R$-module. We will call a graded left $R$-module $M$ {\em $\Theta$-rational} if $M$ is a graded submodule of a graded quotient module of a coproduct of copies of suspensions of $\Theta$.
\end{definition}
\begin{remark}\label{rationals closed under quots and subs 2}
Note that Proposition \ref{rationals closed under quots and subs} remains true, with the same proof, if we replace ``rational'' with ``$\Theta$-rational'' throughout.
\end{remark}


\subsection{Distinguished ideals and distinguished torsion.}
\label{Distinguished ideals and...}

Suppose that $\Gamma$ is projective over $A$. 
Then, as a consequence of Theorem \ref{review thm}, the graded $\Gamma$-comodules form a particularly nice subcategory of the graded $\Gamma^*$-modules: here ``nice'' means, in particular, full and coreflective and abelian. It is not obvious, at a glance, how to tell if a given graded $\Gamma^*$-module actually lives in that nice subcategory. 
Consequently, one would like to have a purely module-theoretic characterization of that subcategory: that is, given a graded $\Gamma^*$-module $M$, we would like to be able to determine whether $M$ is rational by means of some criterion which refers only to the $\Gamma^*$-action on $M$. 
One of our tasks (completed in Theorem \ref{main thm 1}) in this section is to show that there is indeed such a criterion: it is the property of being {\em distinguished-torsion}, which we now define.

\begin{definition-proposition}\label{def of distinguished}
Let $R$ be a graded ring, and let $\Theta$ be a graded left $R$-module. 
Let $I$ be a homogeneous left ideal of $R$. Then the following conditions are equivalent:
\begin{itemize}
%\item $\Gamma$, regarded as a graded $\Gamma^*$-module via $\iota$, contains a homogeneous element $\gamma$ annihilated by the right action of $I$. (NO!)
\item There exists an exact sequence of graded left $R$-modules
\[ 0 \rightarrow I \stackrel{f}{\longrightarrow} R \rightarrow \Sigma^n \Theta \]
for some integer $n$, where $f$ is the canonical inclusion of $I$ into $R$.
\item $\Theta$ contains a graded $R$-submodule which is isomorphic to a suspension of $R/RI$.
\item $I$ is a member of the ideal set $\dist(\Theta)$ defined in Example \ref{examples of ideal sets}.
\end{itemize}
As in Example \ref{examples of ideal sets}, we call a left ideal $I$ of $R$ {\em strongly $\Theta$-distinguished} if it is homogeneous and satisfies the above equivalent conditions, and we call a homogeneous left ideal $I$ {\em $\Theta$-distinguished} if it contains the intersection of a finite set of strongly $\Theta$-distinguished left ideals. 
%We write $H^0_{\Theta}$ for the functor $\gr\Mod(R)\rightarrow\gr\Mod(R)$ given by letting $H^0_{\Theta}(M)$ be the graded $R$-submodule of $M$ generated by all homogeneous elements whose annihilator is $\Theta$-distinguished.
%There is a natural monomorphism of graded abelian groups
%\begin{equation}\label{nat map 1} h^0_{\Theta}(M)\hookrightarrow M\end{equation} induced by the composite $\underline{\hom}_{R}\left( R/RI,M\right) \rightarrow \underline{\hom}_{R}\left( R,M\right) \stackrel{\cong}{\longrightarrow} M$, and its image lands in the $R$-submodule $H^0_{\Theta}(M)$ of $M$; in particular, the graded $R$-module of $M$ generated by the subgroup $h^0_{\Theta}(M)$ of $M$ is precisely $H^0_{\Theta}(M)$.

Let $\dist(\Theta)$ be the ideal set of $\Theta$-distinguished left ideals of $R$. 
We say that a left $R$-module $M$ is {\em $\Theta$-distinguished-torsion} if the inclusion $H^0_{\overline{\dist(\Theta)}}(M) \hookrightarrow M$ is an isomorphism. 

The most important case of these notions is when $R = \Gamma^*$, the dual $A$-algebra of an $A$-coalgebra $\Gamma$ which is projective as an $A$-module, and $\Theta = \iota\Gamma$. In that case we write {\em distinguished ideal}, {\em distinguished torsion}, $h^0_{\dist}$, and $H^0_{\dist}$ as shorthand for $\iota\Gamma$-distinguished ideal, $\iota\Gamma$-distinguished torsion, $h^0_{\overline{\dist(\iota\Gamma)}}$, and $H^0_{\overline{\dist(\iota\Gamma)}}$, respectively.
\end{definition-proposition}
\begin{proof}
These claims are elementary.
%This is very elementary, but we provide the proof anyway. Given an element $\gamma\in \Gamma$ annihilated by $I$, the $\Gamma^*$-module homomorphism $\Gamma^*\rightarrow M$ sending $1$ to $\gamma$ factors through the projection $\Gamma^* \rightarrow \Gamma^*/I$, and the kernel of the resulting $\Gamma^*$-module homomorphism $\Gamma^*/I \rightarrow \Gamma$ is a sub-$\Gamma^*$-module of $\Gamma^*/I$, i.e., a left ideal $J$ of $\Gamma^*$ containing $I$. The resulting factor map $\Gamma^*/J\rightarrow \Gamma$ is injective. Conversely, if $\Gamma^*/J \rightarrow \Gamma$ is injective, then the image of $1$ in $\Gamma$ is $J$-torsion. (NO!)
\end{proof}

%Suppose that $I,J$ are left ideals of $\Gamma^*$ such that $I \subseteq J$. It is not difficult to come up with examples where $I$ is strongly distinguished and $J$ is not, and also examples where $J$ is strongly distinguished and $I$ is not. So the property of being strongly distinguished is not generally inherited by sub-ideals or super-ideals, although being distinguished is clearly inherited by super-ideals.

% Given a left $\Gamma^*$-module $M$, let $\Cyc(M)$ denote the partially-ordered (by inclusion) set of cyclic left $\Gamma^*$-submodules of $M$. 
% Write $<x>$ for the submodule generated by an element $x$ of $M$.
% If $M = \Gamma$, then we have $sx = x'$ iff $(s\tensor x')\Delta(x)=1$.
% So $<x>$ contains $<x'>$ if $(s\tensor x')\Delta(x)=1$ for some $s$.
% So if $\Delta(x)$ has a summand with an x' on the right, then $<x>$ contains $<x'>$.
% If $\Delta(x)$ and $\Delta(y)$ contain terms $1\otimes x$ 
%  and $1\otimes y$ respectively, then
%  $\Delta(x+y)$ has a summand with an $x$ on the right and also a summand with
%   $y$ on the right, as long as $x,y$ are linearly independent over the ground ring $A$.
%  So $<x+y>$ contains both $<x>$ and $<y>$.
%  Isn't the sum of the cyclic modules $<x>$ and $<y>$ simply $<x+y>$?
%  <x> + <y> = { rx + sy }
%  <x+y>     = { rx + ry }.
%  Maybe not. But $<x> + <y>$ contains $<x+y>$,
%   so $<x>+<y> \cong \Gamma^*/(I\cap J)$ contains $<x+y>$.
%   where here $I = Ann(<x>)$ and $J = Ann(<y>)$.
%  So $Ann(<x+y>)$ contains $I\cap J$.
% Wait, this is silly: we should directly compare $\Ann(x)\cap \Ann(y)$ to $\Ann(x+y)$.
% We just showed that $\Ann(x)\cap \Ann(y) \subseteq \Ann(x+y)$.
%  Does the converse containment hold? Not necessarily!

\begin{theorem}\label{theta-rationality and torsion}
Let $R$ be a graded ring and let $\Theta$ be a graded left $R$-module. Let $M$ be a graded left $R$-module. Then the following are equivalent:
\begin{enumerate}
\item $M$ is $\Theta$-rational.
\item For every homogeneous $m\in M$, the annihilator of $m$ is a $\Theta$-distinguished left ideal of $R$.
\item $M$ is $\Theta$-distinguished torsion.
\end{enumerate}
\end{theorem}
\begin{proof}
It is easy to see from the definition of distinguished torsion that conditions 2 and 3 are equivalent, so the rest of this proof consists of showing that conditions 1 and 2 are equivalent.
\begin{itemize}
\item 
Suppose that $M$ is $\Theta$-rational, and that $m$ is a homogenous element of $M$.
Embed $M$ into a quotient $Q$ of a coproduct of copies of $\Theta$ via a map $f: M \rightarrow Q$.
Since $f$ is injective, we have $\ann_l(f(m)) = \ann_l(m)$.
Now choose a surjective left $R$-module map $\sigma: F \rightarrow Q$ where $F$ is a coproduct of
copies of $\Theta$, and choose a homogeneous element $m^{\prime}$ in $F$ such that 
$\sigma(m^{\prime}) = f(m)$.
Since coproducts in categories of modules are direct sums, the element $m^{\prime}\in F$ has only finitely many nonzero components in summands $\Theta$ of $F$. Write $m^{\prime}_1, \dots ,m^{\prime}_n$ for these homogeneous nonzero elements of $\Theta$. 
We now have the chain of equalities and containments
\begin{align*}
 \bigcap_{i=1}^n \ann_l(m^{\prime}_i) &= \ann_l(m^{\prime}) \\
  &\subseteq \ann_l(\sigma(m^{\prime})) \\
  &= \ann_l(f(m)) \\
  &= \ann_l(m),
\end{align*}
so $\ann_l(m)$ contains the intersection $\bigcap_{i=1}^n \ann_l(m^{\prime}_i)$ of the left annihilators of a finite set of homogeneous elements of $\Theta$, i.e., $\ann_l(m)$ is a distinguished left ideal of $R$.
\item
Suppose conversely that the annihilator of every homogeneous element of $M$ is a distinguished left ideal of $R$. Let $\homog(M)$ denote the set of homogeneous elements of $M$, and consider the graded $R$-module morphism \[ \epsilon_M: \coprod_{m\in \homog(M)} \Sigma^{\left| m\right|} R/R\ann_l(m) \rightarrow M\]
which sends the summand $\Sigma^{\left| m\right|} R/R\ann_l(m)$ corresponding to $m\in \homog(M)$ to $M$ via the map $\Sigma^{\left| m\right|} R/R\ann_l(m) \rightarrow M$ sending $1$ to $m$.
For each $m\in \homog(M)$, choose a finite set $\gamma_{m,1}, \dots ,\gamma_{m,n_m}$ of homogeneous elements of $\Theta$ such that $\ann_l(m)$ contains $\cap_{i=1}^{n_m}\ann_l(\gamma_{m,i})$.
Then $R/R\cap_{i=1}^{n_m}\ann_l(\gamma_{m,i})$ surjects on to $R/R\ann_l(m)$.
We also have the graded $R$-module monomorphism 
\begin{equation}\label{map 30499949ff} R/R\cap_{i=1}^{n_m}\ann_l(\gamma_{m,i}) 
  \rightarrow \coprod_{i=1}^{n_m} R/R\ann_l(\gamma_{m,i}) \end{equation}
which sends $x\in R/R\cap_{i=1}^{n_m}\ann_l(\gamma_{m,i})$ to the 
element 
\[ \left( x\mod \ann_l(\gamma_{m,1}), x\mod \ann_l(\gamma_{m,2}), \dots , x\mod \ann_l(\gamma_{m,n_m}) \right)\]
of $\coprod_{i=1}^{n_m} R/R\ann_l(\gamma_{m,i})$. Since each $\ann_l(\gamma_{m,i})$ is strongly distinguished, each $R/R\ann_l(\gamma_{m,i})$ is $\Theta$-rational. Since \eqref{map 30499949ff} is monic, Remark \ref{rationals closed under quots and subs 2} gives us that $R/R\cap_{i=1}^{n_m}\ann_l(\gamma_{m,i})$ is also $\Theta$-rational. Since $R/R\ann_l(m)$ is a quotient of $R/R\cap_{i=1}^{n_m}\ann_l(\gamma_{m,i})$, Remark \ref{rationals closed under quots and subs 2} gives us the $\Theta$-rationality of $R/R\ann_l(m)$, and it also gives us that $\coprod_{m\in \homog(M)} \Sigma^{\left| m\right|} R/R\ann_l(m)$ is $\Theta$-rational. Finally, one more application of Remark \ref{rationals closed under quots and subs 2} gives us that $M$ is $\Theta$-rational, since $\epsilon_M$ is surjective.
\end{itemize}
\end{proof}

\begin{corollary}\label{rational iff dist-torsion}
Let $\Gamma$ be a coalgebra projective over a commutative ring $A$, and let $M$ be a graded left $\Gamma^*$-module. Then $M$ is rational if and only if $M$ is distinguished-torsion.
\end{corollary}

Here is another, less important, corollary of Theorem \ref{theta-rationality and torsion}. This corollary is not new, and can also be proven in more elementary ways; we only bother to state it here because it will be used in the proof of Theorem \ref{main thm 1}.
\begin{corollary}\label{rationals closed under summands}
Let $R$ be a graded ring and let $\Theta$ be a graded left $R$-module. Then every summand of a $\Theta$-rational graded $R$-module is $\Theta$-rational.
\end{corollary}
\begin{proof}
Let $M,N$ be graded left $R$-modules, let $N$ be $\Theta$-rational, and let $f: M\rightarrow N$ be a split monomorphism with splitting map $g: N \rightarrow M$.
We have the commutative diagram
\[\xymatrix{
 H^0_{\overline{\dist(\Theta)}}(M) \ar[r]^{H^0_{\overline{\dist(\Theta)}}(f)} \ar@{^{(}->}[d] & H^0_{\overline{\dist(\Theta)}}(N) \ar[r]^{H^0_{\overline{\dist(\Theta)}}(g)}\ar[d]^{\cong} & H^0_{\overline{\dist(\Theta)}}(M) \ar@{^{(}->}[d] \\
 M \ar[r]^f & N \ar[r]^g & M .
 }\]
Since $g$ is epic, so is the composite $H^0_{\overline{\dist(\Theta)}}(N)\rightarrow M$, and consequently so is the map $H^0_{\overline{\dist(\Theta)}}(M) \hookrightarrow M$, which is also automatically monic. So $M$ is $\Theta$-distinguished torsion, so $M$ is $\Theta$-rational by Theorem \ref{theta-rationality and torsion}.
\end{proof}

\begin{comment} (THIS IS TRUE, BUT TOO OBVIOUS TO STATE.)
\begin{prop}\label{dist gamma is filtered}
Suppose we are given a graded ring $R$, a graded left $R$-module $\Theta$, and $\Theta$-distinguished left ideals $I,J$ of $R$. Then the %following claims are both true:
%\begin{enumerate}\item The 
intersection $I\cap J$ is $\Theta$-distinguished.
%\item The product left ideal $IJ$ is $\Theta$-distinguished. (THIS LAST CLAIM IS COMMENTED OUT SINCE IT'S PERHAPS NOT TRUE?)\end{enumerate}
\end{prop}
\begin{proof}
Let $x_1, \dots ,x_m$ and $y_1, \dots ,y_n$ be homogeneous elements of $\Theta$ such that $I$ contains $\cap_{i=1}^m \ann_l(x_i)$ and $J$ contains $\cap_{j=1}^n \ann_l(y_j)$. Then $I\cap J$ contains $\left(\cap_{i=1}^m \ann_l(x_i)\right)\cap\left(\cap_{j=1}^n \ann_l(y_j)\right)$.
\end{proof}
\end{comment}

\begin{remark}\label{remark on h0 and H0}
Corollary \ref{rational iff dist-torsion} establishes the fundamental importance of $H^*_{\dist}$: the category of graded $\Gamma$-comodules, sitting inside the category of graded $\Gamma^*$-modules, consists precisely of those $\Gamma^*$-modules $M$ such that $H^0_{\dist}(M) \rightarrow M$ is an isomorphism. On the other hand, it is $h^*_{\dist}$ which is defined in terms of a colimit of $\Ext$ groups. It is not clear how to calculate $H^*_{\dist}$ except in cases where it coincides with $h^*_{\dist}$. Consequently it is a matter of some importance to know, for a given coalgebra $\Gamma$, whether $h^*_{\dist}$ coincides with $H^*_{\dist}$. We have $h^*_{\dist} = H^*_{\dist}$ for co-commutative coalgebras $\Gamma$, by the same argument as given in Remark \ref{remark on ideal sets}. By a significantly less trivial argument, we prove in Theorem \ref{main thm on mitchell coalgebras} that a certain family of graded coalgebras, the finite-type {\em Mitchell coalgebras}, also have the property that $h^*_{\dist} = H^*_{\dist}$. While the dual Steenrod algebras are not co-commutative, they are finite-type Mitchell coalgebras, so we get $h^*_{\dist} = H^*_{\dist}$ in the case of the dual Steenrod algebras.
\end{remark}

\subsection{Review of (pre)torsion theories, (pre)torsion classes, and stability.}

For any graded ring $R$ and any graded left $R$-module $\Theta$, it is easy to use standard ideas to see that the functor $h^0_{\overline{\dist(\Theta)}}: \gr\Mod(R) \rightarrow\gr\Ab$ is left exact: since the filtered closure $\overline{\dist(\Theta)}$ of $\dist(\Theta)$ is filtered, $h^0_{\overline{\dist(\Theta)}} = \colim_{I\in \overline{\dist(\Theta)}}\underline{\hom}_R(R/RI,-)$ is a composite of left-exact functors, namely $\underline{\hom}_R(R/RI,-)$ and $\colim_{I\in\overline{\dist(\Theta)}}$. On the other hand, it is much less obvious that $H^0_{\overline{\dist(\Theta)}}$ is left exact: recall that $H^0_{\overline{\dist(\Theta)}}(M)$ is the $R$-submodule of $M$ generated by the subgroup $h^0_{\overline{\dist(\Theta)}}(M)$ of $M$. This is not a very commonplace way to construct a functor, and so it is not immediately obvious what kinds of ideas might allow us to see that $H^0_{\overline{\dist(\Theta)}}$ is left exact. 

It turns out that the right ideas are those from {\em torsion theory}: it is a standard result (given below in Theorem \ref{stenstrom thm}) that the preradical associated to a hereditary pretorsion class is left exact, and in Proposition \ref{rational modules are hereditary pretorsion} we prove that the $\Theta$-rational modules are a hereditary pretorsion class, whose associated preradical---namely, $H^0_{\overline{\dist(\Theta)}}$---is consequently left exact. 

In this subsection, we give a ``crash course'' in the basic ideas from torsion theory that are used in our proof that $H^0_{\overline{\dist(\Theta)}}$ is left exact. Introductory accounts of torsion theories include sections 1.12 and 1.13 of \cite{MR1313497}, the entirety of \cite{MR880019}, \cite{MR3565424}, chapter VI of \cite{MR0389953}, and the original paper that introduced torsion theories, \cite{MR191935}. The second of those references restricts attention to the abelian category of left $R$-modules for a ring $R$, while the third deals more generally with arbitrary Grothendieck categories, including graded $R$-modules, which is the desired level of generality for the applications in this paper. The fourth reference, \cite{MR191935}, assumes that the ambient abelian category is well-generated (i.e., each object has a set, rather than a proper class, of subobjects), which is also satisfied in all applications in this paper. All five are excellent references. 

Here are the relevant basics. %, following Stenstr\"{o}m's treatment in chapter VI of \cite{MR191935}:
The following definition appears in section VI.1 of \cite{MR0389953}:
\begin{definition}
Let $\mathcal{C}$ be a complete, co-complete abelian category. 
\begin{itemize}
\item
A {\em preradical on $\mathcal{C}$} is a functor $r: \mathcal{C} \rightarrow \mathcal{C}$ equipped with a natural transformation $\iota: r \rightarrow \id_{\mathcal{C}}$ such that $\iota X: rX \rightarrow X$ is a monomorphism for all objects $X$ of $\mathcal{C}$.
\item
A preradical $r$ is called a {\em radical} if $r(X/rX)$ vanishes for all objects $X$ of $\mathcal{C}$.
\item
A {\em pretorsion class in $\mathcal{C}$} is a class of objects of $\mathcal{C}$ which is closed under coproducts and quotients.
\end{itemize}
\end{definition}

\begin{definition-proposition}\label{def-prop on torsion theories}\leavevmode
\begin{itemize}
\item
Given a well-generated abelian category $\mathcal{C}$, a {\em torsion theory on $\mathcal{C}$} is a pair $(\mathcal{T},\mathcal{F})$ of full replete\footnote{A subcategory is said to be {\em replete} if it is contains every object isomorphic to one of its own objects.} subcategories of $\mathcal{C}$ such that:
\begin{enumerate}
\item if $X\in\ob\mathcal{T}$ and $Z\in\ob\mathcal{F}$, then $\hom_{\mathcal{C}}(X,Z) = 0$, and
\item if $Y\in \ob\mathcal{C}$, then there exists a short exact sequence
\[ 0 \rightarrow X \rightarrow Y \rightarrow Z \rightarrow 0\]
in $\mathcal{C}$ such that $X\in\ob\mathcal{T}$ and $Z\in\ob\mathcal{F}$.
\end{enumerate}
\item A class $\mathcal{T}$ of objects of $\mathcal{C}$ is called a {\em torsion class} if there exists a class $\mathcal{F}$ of objects of $\mathcal{C}$ such that $(\mathcal{T},\mathcal{F})$ is a torsion theory on $\mathcal{C}$. Equivalently (see Theorem 2.3 of \cite{MR191935}), a torsion class in $\mathcal{C}$ is a class of objects of $\mathcal{C}$ closed under images, coproducts, and extensions.
\item A torsion theory $(\mathcal{T},\mathcal{F})$ is called {\em hereditary} if every subobject of every object in $\mathcal{T}$ is also in $\mathcal{T}$. A pretorsion class is called {\em hereditary} if it is closed under subobjects. Consequently, a torsion class is hereditary if and only if it is the torsion class of a hereditary torsion theory.
\item Suppose furthermore that $\mathcal{C}$ is Grothendieck, i.e., $\mathcal{C}$ satisfies $AB5$ (that is, $\mathcal{C}$ has all small coproducts, and filtered colimits of exact sequences are exact) and has a generator. Then every object $X$ in $\mathcal{C}$ has a canonical injective hull $E(X)$. %, and we say that $X$ is {\em $(\mathcal{T},\mathcal{F})$-local} if $X$ and the quotient $E(X)/X$ are each in $\mathcal{F}$. We write $\mathcal{C}/\mathcal{T}$ for the full subcategory of $\mathcal{C}$ generated by the $(\mathcal{T},\mathcal{F})$-local objects.
%\item It is proven in \cite{MR0232821} that the inclusion functor $\mathcal{C}/\mathcal{T}\rightarrow \mathcal{C}$ has an exact left adjoint, which we call $Q$. We write $L$ for the composite $L: \mathcal{C}\rightarrow \mathcal{C}$ of $Q: \mathcal{C}\rightarrow \mathcal{C}/\mathcal{T}$ with the inclusion $\mathcal{C}/\mathcal{T} \rightarrow \mathcal{C}$. The functor $L$ is called the {\em localization} associated to the hereditary torsion theory $(\mathcal{T},\mathcal{F})$. We say that the hereditary torsion theory $(\mathcal{T},\mathcal{F})$ is {\em exact} if the localization functor $L$ is exact.
%\item 
Still assuming that $\mathcal{C}$ is Grothendieck, a hereditary pretorsion theory $(\mathcal{T},\mathcal{F})$ in $\mathcal{C}$ is called {\em stable} if $\mathcal{T}$ is closed under taking injective hulls; that is, if $X\in\ob\mathcal{T}$, then $E(X)\in\ob\mathcal{T}$.
\end{itemize}
\end{definition-proposition}

%We follow a common convention in the subject: from now on, in this paper {\bf all torsion theories will be assumed to be hereditary.}

\begin{comment} (FINE, BUT NOT CURRENTLY USED)
The following proposition is well-known: see e.g. Proposition 60.13 for this result in the setting of modules over a ring, or Lemma 3.2(i) in \cite{MR3565424} for this result in the setting of Grothendieck categories.
\begin{prop}\label{dickson thm on stability}
Let $\mathcal{C}$ be a Grothendieck category, and let $(\mathcal{T},\mathcal{F})$ be a stable torsion theory on $\mathcal{C}$. Let $H^0$ denote the right adjoint to the inclusion $\mathcal{T}\rightarrow\mathcal{C}$. Let $n$ be a positive integer. Then the $n$th right derived functor $R^nH^0$ of $H^0$ vanishes on all objects $M$ of $\mathcal{T}$.
\end{prop}
\end{comment}


The following useful result combines Propositions 1.4, 1.7, 2.3, and 3.1 and Corollary 1.8 in chapter VI of \cite{MR0389953}:
\begin{theorem}\label{stenstrom thm}
Let $\mathcal{C}$ be a complete, co-complete abelian category. Given a preradical $r$ on $\mathcal{C}$, let $\mathcal{T}_r$ denote the collection of all objects $X$ of $\mathcal{C}$ such that $\iota X$ is an isomorphism. 
\begin{itemize}
\item Then $r$ and $\mathcal{T}_r$ determine one another, and consequently there is a one-to-one correspondence between idempotent preradicals on $\mathcal{C}$ and pretorsion classes in $\mathcal{C}$.
\item 
The following are equivalent:
\begin{itemize}
\item $r$ is left exact.
\item $r$ is idempotent and $\mathcal{T}_r$ is hereditary.
\end{itemize}
Consequently there is a one-to-one correspondence between left exact idempotent preradicals on $\mathcal{C}$ and hereditary pretorsion classes in $\mathcal{C}$.
\item Furthermore, $r$ is an idempotent radical if and only if $\mathcal{T}_r$ is a torsion class. Consequently we get a one-to-one correspondence between idempotent radicals on $\mathcal{C}$ and torsion theories on $\mathcal{C}$. 
\item %If we furthermore assume that the abelian category $\mathcal{C}$ is Grothendieck, then 
Combining the above results, $r$ is a left exact idempotent radical if and only if $\mathcal{T}_r$ is a hereditary torsion class. Consequently we get a one-to-one correspondence between left exact idempotent radicals on $\mathcal{C}$ and hereditary torsion theories on $\mathcal{C}$.
\end{itemize}
\end{theorem}



\subsection{The hereditary pretorsion class associated to a module $\Theta$.}


\begin{comment}
\begin{lemma} (COMMENTED OUT SINCE NOT CURRENTLY USED.)
Suppose we are given a graded ring $R$, a graded left $R$-module $\Theta$, a $\Theta$-rational left ideal $I$ of $R$, and a $\Theta$-rational graded left $R$-module $M$. Suppose that $m\in M$ is a homogeneous element of $M$. Then both of the following claims are true:
\begin{enumerate}
\item If $im\in h^0_{\dist}(M)\subseteq M$ for all $i\in I$, then $m\in h^0_{\dist}(M)$.
%\item If $im\in H^0_{\dist}(M)\subseteq M$ for all $i\in I$, then $m\in H^0_{\dist}(M)$. (COMMENTED OUT SINCE PERHAPS NOT TRUE.)
\end{enumerate}
\end{lemma}
\begin{proof}\leavevmode
\begin{enumerate}
\item 
Since $im \in h^0_{\dist}(M)$, there exists some distinguished left ideal $J$ of $R$ such that $jim=0$ for all $j\in J$. 
...(UNFINISHED)
\end{enumerate}
\end{proof}
\end{comment}

\begin{definition}
Let $R$ be a graded ring, and let $S$ be a connected ideal set in $R$. 
\begin{itemize}
\item A graded left $R$-module $M$ is {\em $S$-torsion} if the natural map $h^0_{S}(M)\hookrightarrow M$ is an isomorphism. That is, $M$ is $S$-torsion if and only if every homogeneous element of $M$ is $I$-torsion for some member $I$ of $S$.
\item A graded left $R$-module $M$ is {\em $S$-rational} if the natural map $H^0_{S}(M)\hookrightarrow M$ is an isomorphism. That is, $M$ is $S$-torsion if and only if every homogeneous element of $M$ is a homogeneous $R$-linear combination of homogeneous elements of $R$, each of which is $I$-torsion for some $I\in S$.
\end{itemize}
\end{definition}
Of course every $S$-torsion module is $S$-rational, and the converse is true when $R$ is commutative. When $R$ is commutative, there can exist $S$-rational modules which fail to be $S$-torsion: an example is given in Examples \ref{torsion class examples}.

\begin{prop}\label{rational modules are hereditary pretorsion}
Let $R$ be a graded ring, and let $S$ be a connected ideal set in $R$. Then the following claims are true:
\begin{enumerate}
\item The $S$-torsion modules in $\gr\Mod(R)$ are a hereditary pretorsion class. If $h^1_S(M)$ vanishes on all $S$-torsion graded $R$-modules $M$, then the $S$-torsion modules in $\gr\Mod(R)$ are a hereditary torsion class.
\item The $S$-rational modules in $\gr\Mod(R)$ are a pretorsion class. 
\item If we furthermore have that $S = \overline{\dist(\Theta)}$ for some graded left $R$-module $\Theta$, then the $S$-rational modules in $\gr\Mod(R)$ are a hereditary pretorsion class. 
\end{enumerate}
\end{prop}
\begin{proof}
First, from Definition-Proposition \ref{def of distinguished}, we know that $H^0_{S}$ is a preradical\footnote{Note that, since $h^0_{S}$ does not necessarily take values in $\gr\Mod(R)$ but only in $\gr\Ab$, $h^0_{S}$ is not generally a preradical.} on $\gr\Mod(R)$.
\begin{enumerate}
\item  
By Theorem \ref{stenstrom thm}, to know that the $\Theta$-torsion modules form a hereditary pretorsion class, we need to show that the $S$-torsion modules are closed under coproducts, quotients, and submodules.
We begin with coproducts. Given a graded left $R$-module $M$, let $\eta_M: h^0_S(M)\rightarrow M$ denote the natural monomorphism.
Then, given a set $\{ M_j: j\in J\}$ of $S$-torsion modules in $\gr\Mod(R)$, we have natural maps fitting into a commutative diagram
\begin{equation}\label{diag 0340405} \xymatrix{
 \colim_{I\in S} \underline{\hom}_R\left(R/RI,\coprod_j M_j \right) \ar[r]^{\eta_{\coprod_j M_j}} 
  & \coprod_j M_j \\
 \colim_{i\in S} \coprod_j \underline{\hom}_R(R/RI,M_j) \ar[u] 
  & \coprod_j \colim_{I\in S} \underline{\hom}_R(R/RI,M_j) \ar[l]^{\cong} \ar[u]^{\cong}_{\coprod_j \eta_{M_j}} .
}\end{equation}
Since $\coprod_j \eta_{M_j}$ is an isomorphism and hence epic, the last map in the composite depicted in diagram \ref{diag 0340405} must also be epic. That map is $\eta_{\coprod_j M_j}$, which is also monic since $S$ is connected. Hence $\eta_{\coprod_j M_j}$ is an isomorphism, i.e., $\coprod_jM_j$ is $S$-torsion.

We also need to show that the $S$-torsion modules are closed under quotients. This is a consequence of an easy diagram chase along the lines of the argument just given for closure under coproducts. Consequently the $S$-torsion modules form a pretorsion class. A similar diagram chase suffices to prove that, if $h^1_S(M)$ vanishes for all $S$-torsion $M$, then the $S$-torsion modules are closed under extensions in $\gr\Mod(R)$, and consequently form a torsion class.

Finally, we also need to show that the $S$-torsion modules are closed under submodules. This follows easily from the understanding of the $S$-torsion modules as those graded $R$-modules in which every homogeneous element is $I$-torsion for some $I\in S$.
\item 
We need to show that the $S$-rational modules are closed under coproducts. Given a set $\{ M_j: j\in J\}$ of $S$-rational modules in $\gr\Mod(R)$, any given homogeneous element $m$ of the coproduct $\coprod_{j\in J}M_j$ is a homogeneous $R$-linear combination of elements which are each concentrated in a single summand $M_j$ of $\coprod_{j\in J}M_j$. Each of those elements, in turn, is a homogeneous $R$-linear combination of homogeneous elements which are $I$-torsion for various ideals $I$ of $S$. Consequently $m$ is a homogeneous $R$-linear combination of homogeneous elements which are $I$-torsion for various ideals $I$ of $S$. Consequently $\coprod_{j\in J} M_j$ is $S$-rational.

Showing that the $S$-rational graded $R$-modules are closed under quotients is a simple matter of a diagram chase. Consequently the $S$-rational modules form a pretorsion class in graded $R$-modules. Another diagram chase easily shows that, if $H^1_S(M)$ vanishes for all $S$-rational $M$, then the $S$-rational modules are closed under extensions, and consequently the $S$-rational modules form a torsion class. 
\begin{comment} (MUCH SHORTER ARGUMENT USED INSTEAD)
\begin{description}
\item[Closure under coproducts] 
Recall, from the proof of Theorem \ref{theta-rationality and torsion}, the notation $\homog(M)$ for the set of homogeneous elements of a graded left $R$-module $M$. In that proof we considered the graded left $R$-module 
\begin{equation}\label{def of C(M)}\coprod_{m\in \homog(M)} \Sigma^{\left| m\right|} R/R\ann_l(m),\end{equation} and a surjective graded module map $\epsilon_M$ from that module to $M$, whose image is $H^0_{\Theta}(M)$.
We will write $C(M)$ for the graded module \eqref{def of C(M)}.

Let $\{ M_i: i\in I\}$ be a set of $\Theta$-rational modules. We have an inclusion $\cup_{i\in I}\homog(M_i)\subseteq \homog(\coprod_{i\in I}M_i)$, and consequently
morphisms
\begin{align*}
 \coprod_{i\in I} C(M_i) 
  &\stackrel{=}{\longrightarrow} \coprod_{i\in I} \coprod_{m\in \homog(M_i)} \Sigma^{\left| m\right|} R/R\ann_l(m) \\
  &\stackrel{\cong}{\longrightarrow} \coprod_{m\in \cup_{i\in I}\homog(M_i)} \Sigma^{\left| m\right|} R/R\ann_l(m) \\
  &\stackrel{}{\hookrightarrow} \coprod_{m\in \homog(\coprod_{i\in I}M_i)} \Sigma^{\left| m\right|} R/R\ann_l(m) \\
  &\stackrel{=}{\longrightarrow}  C\left( \coprod_{i\in I} M_i\right)
}\]
which commute with the $\epsilon$ morphisms, i.e., we have a commutative diagram of graded left $R$-modules
\[\xymatrix{
 \coprod_{i\in I} C(M_i) \ar[rd]_{\coprod_{i\in I}\epsilon_{M_i}} \ar@{^{(}->}[r] &
  C\left( \coprod_{i\in I} M_i\right) \ar[d]^{\epsilon_{\coprod_{i\in I} M_i}}\\
 & \coprod_{i\in I}M_i.
}\]
The conclusion here is that, if $\coprod_{i\in I}\epsilon_{M_i}$ is epic, so is $\epsilon_{\coprod_{i\in I} M_i}$; i.e., if each of the modules $M_i$ is $\Theta$-rational, then so is $\coprod_{i\in I}M_i$.
\item[Closure under quotients] 
Given a graded $R$-module epimorphism $M \rightarrow N$, we get an epimorphism $\homog(M) \rightarrow \homog(N)$, and consequently a commutative diagram
\[\xymatrix{
 C(M) \ar[d]_{\epsilon_M}\ar[r] & C(N) \ar[d]^{\epsilon_N} \\ 
 M \ar@{->>}[r] & N.
}\]
Consequently if $\epsilon_M$ is surjective, then so is $\epsilon_N$; i.e., if $M$ is $\Theta$-rational, then so is $N$.
\item[Closure under subobjects]
Given a graded $R$-module monomorphism $f: M \rightarrow N$ such that $N$ is $\Theta$-rational, 
we get an monomorphism $\homog(M) \rightarrow \homog(N)$, and consequently a commutative diagram
\begin{equation}\label{comm diag059599}\xymatrix{
 C(M) \ar[d]_{\epsilon_M}\ar[r] & C(N) \ar[d]^{\epsilon_N} \\ 
 M \ar@{^{(}->}[r] & N.
}\end{equation}
Since $f$ is monic, the annihilator of an element $m\in M$ coincides with the annihilator of $f(m)$, so the map $C(M)\rightarrow C(N)$ in diagram \eqref{comm diag059599} is monic. 
Given a homogeneous element $x\in M$, choose an element $\sum_{n\in \homog(N)}r_n\in C(N)$ such that $\epsilon_N\left( \sum_{n\in \homog(N)}r_n\right) = f(x)$, i.e., $\sum_{n\in\homog(N)}r_n\cdot n = f(x)$. The set of elements $n\in \homog(N)$ such that $r_n$ is nonzero is finite, so choose a total ordering $n_1, \dots ,n_N$ on that set. Then we have $f(x) = \sum_{i=1}^N r_{n_i}\cdot n_i$. If $n_1\notin \im f$, then ...(UNFINISHED)
\end{comment}
%This is, however, an immediate consequence of Proposition \ref{rationals closed under quots and subs} together with Theorem \ref{theta-rationality and torsion}, so all that remains is to prove that $H^0_{\Theta}$ is idempotent.
%Applying $h^0_{\Theta}$ to the $R$-module inclusion $H^0_{\Theta}(M)\hookrightarrow M$, we get an isomorphism $h^0_{\Theta}H^0_{\Theta}(M)\stackrel{\cong}{\longrightarrow} h^0_{\Theta}M$, so the $R$-submodule of $H^0_{\Theta}(M)$ generated by $h^0_{\Theta}H^0_{\Theta}(M)$ coincides with the $R$-submodule of $M$ generated by $h^0_{\Theta}(M)$. That is, $H^0_{\Theta}H^0_{\Theta}(M)\rightarrow H^0_{\Theta}(M)$ is an isomorphism.
\item Given a graded left $R$-module $\Theta$, it is a consequence of Theorem \ref{theta-rationality and torsion} that $\overline{\dist(\Theta)}$-rationality---that is, $\Theta$-rationality---is inherited by submodules. 
\end{enumerate}
\end{proof}
\begin{corollary}\label{H0 is left exact}
Let $R$ be a graded ring, and let $\Theta$ be a graded $R$-module. Then $H^0_{\Theta}: \gr\Mod(R) \rightarrow \gr\Mod(R)$ is left exact.
\end{corollary}
\begin{proof}
Immediate consequence of Theorem \ref{stenstrom thm}
and Proposition \ref{rational modules are hereditary pretorsion}.
\end{proof}

\begin{corollary}\label{H0 rat and iota}
Let $A$ be a commutative ring, let $\Gamma$ be an $A$-coalgebra projective over $A$, and let $R = \Gamma^*$ be the $A$-linear dual algebra of $\Gamma$. Then $H^0_{\dist}: \gr\Mod(R)\rightarrow \gr\Mod(R)$ is naturally isomorphic to the composite $\iota\circ\tr$ of the rational functor $\tr: \gr\Mod(\Gamma^*)\rightarrow \gr\Comod(\Gamma)$ with the inclusion $\iota: \gr\Comod(\Gamma)\rightarrow\gr\Mod(\Gamma^*)$ of the comodules into the modules.
\end{corollary}

\begin{corollary}\label{H1 vanishing and torsion}
Let $R$ be a graded ring, and let $\Theta$ be a graded $R$-module. If $H^1_{\Theta}(M)$ vanishes for all $\Theta$-rational modules $M$ in $\gr\Mod(R)$, then the $\Theta$-rational modules in $\gr\Mod(R)$ form a torsion class.
\end{corollary}
\begin{proof}
By Proposition \ref{rational modules are hereditary pretorsion}, the $S$-rational modules are a pretorsion class, hence are closed under quotients and coproducts. Since Corollary \ref{H0 is left exact} establishes that $H^0_{\Theta}$ is left exact, vanishing of $H^1_{\Theta}$ on the $S$-rational modules ensures that the $S$-rational modules are also closed under extensions, and consequently they form a torsion class.
\end{proof}


%\begin{remark} As far as the author knows, the $\Theta$-rational modules in $\gr\Mod(R)$ do not necessarily form a torsion class. By Theorem \ref{stenstrom thm}, the $\Theta$-rational modules in $\gr\Mod(R)$ form a torsion class if and only if $H^0_{\Theta}$ is a radical. The most straightforward approach to proving that $H^0_{\Theta}$ is a radical requires that the product of distinguished left ideals also be distinguished, but in general there seems to be no reason to expect that distinguished ideals are closed under finite products of ideals.\end{remark}

\begin{examples}\label{torsion class examples}
Here are some examples to demonstrate the importance of the hypotheses involved in Proposition \ref{rational modules are hereditary pretorsion}.
\begin{itemize}
\item 
There exist rings $R$ and filtered (hence connected) ideal sets $S$ in $R$ such that the pretorsion class of $S$-torsion modules fails to be a torsion class, and also such that the pretorsion class of $S$-rational modules fails to be a torsion class. Here is a simple example of both: let $R = \mathbb{Z}/4\mathbb{Z}$, and let $S$ be $\{ (2)\}$. Then $h^0_S(M) = H^0_S(M)$ is the $2$-torsion submodule of a $\mathbb{Z}/4\mathbb{Z}$-module $M$. We have the short exact sequence of $\mathbb{Z}/4\mathbb{Z}$-modules
\[ 
 0 
  \rightarrow \mathbb{Z}/2\mathbb{Z} 
  \rightarrow \mathbb{Z}/4\mathbb{Z} 
  \rightarrow \mathbb{Z}/2\mathbb{Z} 
  \rightarrow 0\]
in which the left- and right-hand nonzero modules are $S$-torsion and also $S$-rational, but the middle module, $\mathbb{Z}/4\mathbb{Z}$, is neither $S$-torsion nor $S$-rational.
\item
When the connected ideal set $S$ is not equivalent to $\overline{\dist(\Theta)}$ for some graded left $R$-module $\Theta$, it is not necessarily the case that the the pretorsion class consisting of the $S$-rational modules is hereditary. Here is an example: let $R$ be the subalgebra $A(1)$ of the $2$-primary Steenrod algebra generated by $\Sq^1$ and $\Sq^2$. Let $S$ be $\{ A(1)\Sq^1\}$, that is, $S$ has just one member, and that member is the left ideal of $A(1)$ generated by $\Sq^1$. Let $M$ be the graded $A(1)$-submodule of the cohomology $H^*(\mathbb{R}P^4; \mathbb{F}_2)$ of four-dimensional real projective space generated by $\Sq^1\cdot 1\in H^1(\mathbb{R}P^4;\mathbb{F}_2)$. Then $M$ is a three-dimensional $\mathbb{F}_2$-vector space with homogeneous $\mathbb{F}_2$-linear basis $\{ \Sq^1, \Sq^2\Sq^1, \Sq^1\Sq^2\Sq^1\}$. The subgroup $h^0_S(M)$ of $M$ is the $\mathbb{F}_2$-linear span of $\Sq^1$ and $\Sq^1\Sq^2\Sq^1$, and the $A(1)$-submodule of $M$ generated by $h^0_S(M)$ is $M$ itself. Hence $M$ is $S$-rational but not $S$-torsion.

Let $M^{\prime}$ be the $A(1)$-submodule of $M$ generated by $\Sq^2\Sq^1$. Since $h^0_S(M^{\prime})$ is $\mathbb{F}_2$-linearly spanned by $\Sq^1\Sq^2\Sq^1$, we have that $h^0_S(M^{\prime}) = H^0_S(M^{\prime}) \neq M^{\prime}$, so $M^{\prime}$ is neither $S$-rational nor $S$-torsion, despite being a submodule of a $S$-rational module.
\end{itemize}
\end{examples}


\section{Distinguished local cohomology of products.}
\label{Distinguished local cohomology...}
\subsection{Derived products of comodules are given by distinguished local cohomology.}

%\begin{definition} Let $R$ be a graded ring, and let $\Theta$ be a graded left $R$-module. By {\em $\Theta$-distinguished local cohomology} we mean the right-derived functors of the $\Theta$-distinguished torsion functor $H^0_{\Theta}: \gr\Mod(R)\rightarrow \gr\Mod(R)$. We write $H^n_{\Theta}$ for the $n$th $\Theta$-distinguished local cohomology functor, i.e., $H^n_{\Theta}  = R^nH^0_{\Theta}$.

%The most important case, for our purposes, is when $A$ is a commutative ring, $\Gamma$ is an $A$-coalgebra projective over $A$, $R = \Gamma^*$ is the $A$-linear dual algebra of $\Gamma$, and $\Theta = \Gamma$ with the adjoint action of $\Gamma^*$. In that case, we refer to $\Theta$-distinguished local cohomology as simply {\em distinguished local cohomology,} and we write $H^n_{\dist}$ rather than $H^n_{\Theta}$.\end{definition}

\begin{theorem}\label{main thm 1}
Suppose that $\Gamma$ is a graded $A$-coalgebra which is projective as an $A$-module.
Then the following claim is true:
\begin{enumerate}
\item For each nonnegative integer $n$ and each graded left $\Gamma^*$-module $M$, we have an isomorphism $H^n_{\dist}(M) \cong \iota\left( R^n\tr(M)\right)$, natural in the variable $M$.
\end{enumerate}
Suppose furthermore that $A$ is a countable field and that $\Gamma^*$ is finite-type and connected\footnote{``Connected'' is a standard term, but to avoid any possible misunderstanding, we include its definition here: to say that the finite-type algebra $\Gamma^*$ is connected is to say that, for each $n$, the degree $n$ summand of $\Gamma^*$ (equivalently, $\Gamma$) is a finite dimensional $A$-vector space, that $\Gamma^*$ is one-dimensional in degree $0$, and that $\Gamma^*$ is trivial in negative degrees.}. Then the following claims are also each true:
\begin{enumerate}
\item[(2)] If $M$ is an injective graded $\Gamma$-comodule, then the graded $\Gamma^*$-module $\iota(M)$ is injective.
\item[(3)] The rational graded $\Gamma^*$-modules are a hereditary stable %pretorsion
torsion class in $\gr\Mod(\Gamma^*)$. 
\item[(4)] If $M$ is a rational graded $\Gamma^*$-module, then the distinguished local cohomology groups $H^n_{\dist}(M)$ vanish for all $n>0$. 
\item[(5)] Every bounded-above graded $\Gamma^*$-module is rational.
\item[(6)] Let $I$ be a set, and let $F: I \rightarrow \gr\Comod(\Gamma)$ be a functor. (That is, $F$ consists of a choice of graded $\Gamma$-comodule $F(i)$ for each $i\in I$, and no other data.) 
Then, for each nonnegative integer $n$, the $n$th distinguished local cohomology group $H^n_{\dist}\left(\prod_{i\in I} \iota(M_i)\right)$ of the product, over $i\in I$, of the graded $\Gamma^*$-modules $\iota(M_i)$, is isomorphic to $\iota$ applied to the $n$th right derived functor $R^n\prod^{\Gamma}_{i\in I}\{ M_i\}$ of the product functor $\gr\Comod(\Gamma)^I\stackrel{\prod^{\Gamma}}{\longrightarrow} \gr\Comod(\Gamma)$. That is, we have an isomorphism
\[ \iota\left( R^{n}\prod^{\Gamma}_i\left(\left\{ M_i:i\in I\right\}\right)\right)
 \cong H^n_{\dist}\left( \prod_{i\in I}\iota( M_i)\right).\]
\end{enumerate}
\end{theorem}
\begin{proof}
\begin{enumerate}
\item Since $\tr$ has an exact left adjoint (namely, $\iota$), $\tr$ sends injectives to injectives. Consequently we have a Grothendieck spectral sequence $R^s\iota \left(R^t\tr(M)\right) \Rightarrow R^{s+t}(\iota\circ\tr)(M)$, and it collapses to the $s=0$ line since $\iota$ is exact. We consequently have isomorphisms $\iota \left( R^t\tr(M)\right)\cong R^t(\iota\circ \tr)(M)\cong H^t_{\dist}(M)$ due to Corollary \ref{H0 rat and iota}.
\item 
Since $A$ is a field, the injective objects in $\gr\Comod(\Gamma)$ are precisely the comodules whose coaction map is a split monomorphism of comodules; see 1.3.18 of \cite{MR2012570}, for example. That is, the injective objects in $\gr\Comod(\Gamma)$ are the retracts of extended comodules. Again, since $A$ is a field, every graded $A$-module is a coproduct of suspensions of $A$, so every extended $\Gamma$-comodule is a coproduct of copies of suspensions of $\Gamma$. 

Since $\iota$ is a left adjoint, it preserves coproducts, so if $N$ is an extended $\Gamma$-comodule, then $\iota(N)$ is a coproduct of copies of suspensions of $\iota(\Gamma)$. It is well-known that the linear dual of a connected algebra $R$ over a field, regarded as a graded $R$-module via the adjoint action, is injective; see e.g. Proposition 12 in section 11.3 of \cite{MR738973}, for example. Since we have assumed that $\Gamma$ is finite-type, the linear dual comodule $\Gamma^{**}$ of $\Gamma^*$ is isomorphic to $\Gamma$, and so the graded $\Gamma^*$-module $\iota(\Gamma)$ is injective.

To know that $\iota(N)$ is injective, we must know that a coproduct of copies of $\iota(\Gamma)$ is injective\footnote{The well-known Bass-Papp theorem (see Theorem 3.46 of \cite{MR1653294} for a textbook treatment) establishes that a ring $R$ is left Noetherian if and only if every coproduct of injective left $R$-modules is injective. Nothing in our setup requires $\Gamma^*$ to be Noetherian, however, and our main motivating examples, the Steenrod algebras, are not Noetherian. (More generally, $\mathcal{P}$-algebras are never Noetherian on either side, by Proposition 2 of section 13.1 of \cite{MR738973}.) So there exist coproducts of injective modules over the Steenrod algebra which fail to be injective; we must show that coproducts of copies of $\iota(\Gamma)$ are not among them.

There is a graded version of the Bass-Papp theorem (which unfortunately does not appear in the literature anywhere that I am aware of) which depends only on the degree zero subring of an $\mathbb{N}$-graded ring $R$, but it only establishes that {\em uniformly bounded below} coproducts of injectives are injective. In our case, this is not enough: we need to be able to accommodate non-bounded-below coproducts of injectives. There are other possible graded versions of the Bass-Papp theorem (e.g. Proposition 2.3 of \cite{MR3023485}) but none which apply in our situation.

We conclude that Megibben's theorem really is the right tool, here: the various forms of the Bass-Papp theorem are not able to show that coproducts of copies of $\iota(\Gamma)$ are injective.}. Now we invoke the main theorem of \cite{MR633266}: every countable injective module, over any ring, is $\Sigma$-injective. Recall that a module over a ring is said to be ``$\Sigma$-injective'' if every coproduct of copies of that module is injective. So Megibben's theorem gives us that $\iota(N)$, a coproduct of copies of the countable injective $\Gamma^*$-module $\iota(\Gamma)$, is also injective. See Theorem \ref{graded megibbens thm}, in the appendix, for the graded generalization of Megibben's theorem that we are in fact using here.

If $M$ is an injective graded $\Gamma$-comodule, then $M$ is a summand of the extended comodule $\Gamma\otimes_A M$, so $\iota(M)$ is a summand of $\iota(\Gamma\otimes_AM)$, which we have just shown is injective. Summands of injectives are injective, so $\iota(M)$ is injective, as desired.
\item[(3, part 1)] Corollary \ref{rational iff dist-torsion} together with Proposition \ref{rational modules are hereditary pretorsion} establishes that the rational modules form a hereditary pretorsion class, so we need only prove that this pretorsion class is stable. 
Suppose $M$ is a rational graded $\Gamma^*$-module. Regard $M$ as a graded $\Gamma$-comodule $\tilde{M}$. Embed $\tilde{M}$ into an injective graded $\Gamma$-comodule $I$; a canonical way to do this is to simply let $I$ be the extended comodule $\tilde{M}\otimes_k \Gamma$, for example. Applying $\iota$ yields an injective graded $\Gamma^*$-module homomorphism $M = \iota(\tilde{M}) \hookrightarrow \iota(I)$, and by the previous part of this theorem, $\iota(I)$ is an injective graded $\Gamma^*$-module. The minimality condition on the injective hull $E(M)$ of $M$ guarantees that the embedding $M\hookrightarrow \iota(I)$ factors as the embedding $M\hookrightarrow E(M)$ followed by an embedding $E(M)\hookrightarrow \iota(I)$, so by the universal property of an injective, $E(M)$ is a {\em summand} of the rational $\Gamma^*$-module $\iota(I)$. By Corollary \ref{rationals closed under summands}, $E(M)$ is rational. So the rational graded $\Gamma^*$-modules, as a pretorsion class in graded $\Gamma^*$-modules, is stable. 

So far, we have shown that the rational graded $\Gamma^*$-modules form a hereditary stable pretorsion class in the graded $\Gamma^*$-modules. We will prove that this pretorsion class is in fact a torsion class after first proving claim \#4.
\item[(4)] Let $M$ be a rational graded $\Gamma^*$-module. %The functor $\Rat$ is left exact, since it is a right adjoint, and $\iota$ is exact, and 
We have just shown that $\iota$ sends injectives to injectives. %Consequently we have the Grothendieck spectral sequence $R^s\iota \left(R^t\Rat(M)\right) \Rightarrow R^{s+t}(\iota\circ\Rat)(M)$, and it collapses to the $s=0$ line since $\iota$ is exact. 
The functor $\tr$ has an exact left adjoint, so $\tr$ also sends injectives to injectives. Consequently, if we choose an injective resolution $I^{\bullet}$ for $\tr(M)$ in the category of graded $\Gamma^*$-comodules, then exactness of $\iota$ yields that $J^{\bullet}\coloneqq\iota(I^{\bullet})$ is an injective resolution of $\iota(\tr(M))\cong M$ in graded $\Gamma^*$-comodules with the property that $\tr(J^{\bullet}) \cong I^{\bullet}$. The point is that, if we use the injective resolution $J^{\bullet}$ to calculate the right-derived functors of $\tr$ applied to $M$, we get $R^t\tr(M) \cong H^t(\tr(J^{\bullet})) \cong H^t(I^{\bullet})\cong 0$ for all $t>0$.

Since $\iota$ and $\tr$ are left exact and $\tr$ sends injectives to injectives, we get the Grothendieck spectral sequence $R^s\iota(R^t\tr(M)) \Rightarrow R^{s+t}(\iota\circ\tr)(M)$. By Corollary \ref{H0 rat and iota}, $R^{s+t}(\iota\circ \tr)$ coincides with the distinguished local cohomology functor $H^{s+t}_{\dist}$, so the exactness of $\iota$ and the vanishing of $R^t\tr(M)$ for $t>0$ for $M$ rational gives us that $H^n_{\dist}(M)$ vanishes for all $n>0$ and all rational $M$.
\item[(3, part 2)]
We have just shown that, in particular, $H^1_{\dist}(M)$ vanishes for rational $M$. Consequently Corollary \ref{H1 vanishing and torsion} yields that the rational modules are in fact a torsion class, not only a pretorsion class, completing the proof of claim (3).
\item[(5)]
Suppose that $M$ is a graded $\Gamma^*$-module. For each integer $n$, write $\conn_n(M)$ for the graded sub-$\Gamma^*$-module of $M$ generated by all homogeneous elements of degree $\geq n$. Then we have the sequence of monomorphisms \begin{equation}\label{filt 13}\dots \hookrightarrow \conn_n(M) \hookrightarrow \conn_{n-1}(M) \hookrightarrow \conn_{n-2}(M) \hookrightarrow\dots\end{equation}
of graded $\Gamma^*$-modules, and its colimit is $M$. If $M$ is bounded-above, then each of the submodules $\conn_n(M)$ is bounded (i.e., both bounded-above and bounded-below). 

Consequently every bounded-above graded $\Gamma^*$-module is a colimit of bounded graded $\Gamma^*$-modules.
By Proposition \ref{rationals closed under quots and subs}, the rational $\Gamma^*$-modules are closed under cokernels and coproducts in $\gr\Mod(\Gamma^*)$, hence closed under all small colimits in $\gr\Mod(\Gamma^*)$. Hence, if we can show that the bounded graded $\Gamma^*$-modules are rational, then all the bounded-above graded $\Gamma^*$-modules will be rational.

So we suppose that $N$ is a bounded graded $\Gamma^*$-module. We carry out an induction to prove that $H^n_{\dist}(N)$ vanishes for $n>0$. The initial step in the induction is the case in which $N$ is concentrated in a single degree. Since $\Gamma^*$ is connected, $N$ splits (as a $\Gamma^*$-module) as a coproduct of copies of $\Gamma^*/\conn_1(\Gamma^*) \cong k$, where $\conn_1(\Gamma^*)$ denotes the homogeneous ideal in $\Gamma^*$ generated by all homogeneous elements in positive degrees. It is straightforward to see that the graded $\Gamma^*$-module $\Gamma^*/\conn_1(\Gamma^*)$ is rational. By Proposition \ref{rationals closed under quots and subs}, a coproduct of rational modules is rational, so $N$ is rational, as desired. This completes the initial step in the induction.

The inductive step is as follows: suppose $m$ is a nonnegative integer, and suppose that we have already shown that $N^{\prime}$ is rational for all graded $\Gamma^*$-modules $N^{\prime}$ which are trivial except in $\leq m$ consecutive degrees. (That is, the inductive hypothesis is that, if $r$ is an integer and $N^{\prime}$ is a graded $\Gamma^*$-module which is trivial except in grading degrees $r, r+1, \dots ,r+m-1$, then $N^{\prime}$ is rational.) 
Suppose that $N$ is a graded $\Gamma^*$-module which is trivial except in $m+1$ consecutive degrees. Let $r$ be the lowest degree in which $N$ is nontrivial. Let $N^{\geq r+1}$ be the graded $A$-submodule of $N$ generated by all homogeneous elements of degree $\geq r+1$.
Then $N^{\geq r+1}$ is a graded $\Gamma^*$-submodule of $N$, since $\Gamma^*$ is connective. We consequently have a short exact sequence of graded $\Gamma^*$-modules
\begin{equation}\label{ses 04040} 0 \rightarrow N^{\geq r+1} \rightarrow N \rightarrow N/N^{\geq r+1} \rightarrow 0\end{equation}
and $N^{\geq r+1}$ is trivial except in $\leq m$ consecutive degrees, hence is rational by the inductive hypothesis, and consequently $H^n_{\dist}(N^{\geq r+1})$ vanishes for all $n>0$, by the previous part of this theorem.
Consequently (and using the left-exactness of $H^0_{\dist}$, proven in Corollary \ref{H0 is left exact}), applying $H^*_{\dist}$ to \eqref{ses 04040} yields that the top row in the commutative diagram
\begin{equation}\label{comm diag 0595959}
\xymatrix{
 0 \ar[r] \ar[d] &
  H^0_{\dist}(N^{\geq r+1}) \ar[r]\ar[d]^{\cong} &
  H^0_{\dist}(N) \ar[r]\ar[d] &
  H^0_{\dist}(N/N^{\geq r+1}) \ar[r] \ar[d]^{\cong} &
  0 \ar[d] \\
 0 \ar[r] &
  N^{\geq r+1} \ar[r] &
  N \ar[r] &
  N/N^{\geq r+1} \ar[r] &
 0.}\end{equation}
The vertical maps indicated with the symbol $\cong$ in diagram \eqref{comm diag 0595959} are isomorphisms by the initial step (for $N/N^{\geq r+1}$, since it is concentrated in a single degree), and by the inductive hypothesis (for $N^{\geq r+1}$). The usual Five Lemma from homological algebra then given us that $H^0_{\dist}(N)\rightarrow N$ is also an isomorphism, i.e., $N$ is also rational, completing the inductive step. So every bounded graded $\Gamma^*$-module is rational, as desired.
\item[(6)]
Since products of injectives are injective, the functor $\prod_I:\gr\Mod(\Gamma^*)^I \rightarrow\gr\Mod(\Gamma^*)$ preserves injectives. Since we have shown that $\iota$ also preserves injectives, so does the composite $\prod_I\circ\iota$, so we get a Grothendieck spectral sequence
\begin{align*}
 E_2^{s,t} \cong R^s\tr R^t\left(\prod_{i\in I}\circ\iota\right)\left(\left\{ M_i:i\in I\right\}\right) &\Rightarrow R^{s+t}\left( \tr\circ\prod_{i\in I}\circ\iota\right)\left(\left\{ M_i:i\in I\right\}\right) 
\\ &\cong R^{s+t}\left( \prod^{\Gamma}_i\circ \tr\circ\iota\right)\left(\left\{ M_i:i\in I\right\}\right) 
\\ &\cong R^{s+t} \prod^{\Gamma}_i\left(\left\{ M_i:i\in I\right\}\right),
\end{align*}
which collapses to the $t=0$ line, since $\iota$ and $\prod_I$ are each exact and so their composite $\prod_I\circ\iota$ is exact. 
Consequently we have isomorphisms
\begin{align}
\nonumber \iota R^{s}\prod^{\Gamma}_i\left(\left\{ M_i:i\in I\right\}\right)
  &\cong \iota R^s\tr \left(\prod_{i\in I}\iota( M_i)\right) \\
\label{iso 00094912}  &\cong H^s\left( \prod_{i\in I}\iota( M_i)\right) ,
\end{align}
with isomorphism \eqref{iso 00094912} due to the first part of this theorem.
\end{enumerate}
\end{proof}

Theorem \ref{main thm 1} proves that, among other things, the higher distinguished local cohomology groups vanish on the modules which come (via $\iota$) from comodules.
One might try to think of this result as telling us that distinguished local cohomology $H^*_{\dist}(M)$ is a cohomology theory that tells us how far the module $M$ is from being a comodule. This requires a bit of care, because distinguished local cohomology {\em also} vanishes in positive degrees on some $\Gamma^*$-modules which {\em don't} come from comodules. For example, in Theorem 12 of section 13.3 of \cite{MR738973}, Margolis proves that the Steenrod algebras are self-injective, and more generally, that if $\Gamma^*$ is a $\mathcal{P}$-algebra in the sense of Margolis, then $\Gamma^*$ is self-injective. The Steenrod algebras satisfy all the hypotheses of Theorem \ref{main thm 1}, but not only does $H^n_{\dist}(\Gamma^*)$ vanish for all $n>0$, we also have that $H^0_{\dist}(\Gamma^*)$ vanishes. %(See Lemma \ref{dist ideals identification 2} for a proof of this claim.) 
So $\Gamma^*$ cannot be in the image of $\iota$, since $0\cong H^0_{\dist}(\Gamma^*) \cong \iota(\Rat(\Gamma^*))$.

So while one can truthfully say that $H^*_{\dist}$ ``detects the failure of a $\Gamma^*$-module to be a $\Gamma$-comodule,'' it is not the case that the $\Gamma$-comodules are {\em precisely} those $\Gamma^*$-modules on which $H^*_{\dist}$ vanishes. Still, there are satisfying structural relationships between the category of $\Gamma^*$-comodules and the category of $\Gamma$-modules, such as the following theorem:
\begin{theorem}\label{structure thm}
Suppose that $A$ is a countable field, $\Gamma$ a graded $A$-coalgebra, and that $\Gamma^*$ is finite-type and connected. Let $n$ be an integer.
Then every graded $\Gamma^*$-module is an extension of a $n$-co-connected rational graded $\Gamma^*$-module by an $n$-connective graded $\Gamma^*$-module.

Rephrased, with more detail: for each graded $\Gamma^*$-module $M$, we have 
a short exact sequence of graded $\Gamma^*$-modules
\begin{equation}\label{ses 40909090} 0 \rightarrow \conn_n(M) \rightarrow M \rightarrow \iota(\comod_n(M)) \rightarrow 0,\end{equation}
where $\conn_n(M)$ is a {\em $n$-connective} graded $\Gamma^*$-module (i.e., $\conn_n(M)$ is trivial in degrees $<n$), and $\comod_n(M)$ is an {\em $n$-co-connected} graded $\Gamma$-comodule (i.e., $\comod_n(M)$ is trivial in degrees $\geq n$). This sequence is natural in the variable $M$.

Furthermore, the higher distinguished local cohomology of $M$ depends only on $\conn_n(M)$. That is, $H^i_{\dist}(\conn_n(M)) \rightarrow H^i_{\dist}(M)$ is an isomorphism for all $n$ and for all $i>0$.
\end{theorem}
\begin{proof}
Let $\conn_n(M)$ simply be the graded $\Gamma^*$-submodule of $M$ generated by all elements in degrees $\geq n$. The quotient $M/\conn_n(M)$ is then $n$-co-connected, hence bounded above, hence is $\iota$ of a graded $\Gamma$-comodule by Theorem \ref{main thm 1}. 
\end{proof}
It is easy to see that the sequence \eqref{ses 40909090} is natural in the integer $n$, in the sense that we have a commutative diagram of graded $\Gamma^*$-modules
\begin{equation}\label{comm diag 40990909}\xymatrix{
\vdots \ar[d] &
 \vdots \ar[d] &
 \vdots \ar[d] &
 \vdots \ar[d] &
 \vdots \ar[d] \\
0 \ar[r] \ar[d] & 
 \conn_{n+1}(M) \ar[r]\ar@{^{(}->}[d] & 
 M \ar[r]\ar[d]^{\id} & 
 \iota(\comod_{n+1}(M))\ar[r]\ar@{->>}[d] & 0 \ar[d] \\
0 \ar[r] \ar[d] & 
 \conn_n(M) \ar[r]\ar@{^{(}->}[d] & 
 M \ar[r]\ar[d]^{\id} & 
 \iota(\comod_n(M))\ar[r]\ar@{->>}[d] & 0 \ar[d] \\
0 \ar[r] \ar[d] & 
 \conn_{n-1}(M) \ar[r]\ar@{^{(}->}[d] & 
 M \ar[r]\ar[d]^{\id} & 
 \iota(\comod_{n-1}(M))\ar[r]\ar@{->>}[d] & 0 \ar[d] \\
\vdots &
 \vdots &
 \vdots &
 \vdots &
 \vdots 
}\end{equation}
with exact rows.
Taking the limit of each column in \eqref{comm diag 40990909} yields:
\begin{corollary}\label{modules are limits of comodules}
Suppose that $A$ is a countable field, and that $\Gamma^*$ is finite-type and connected. Then every graded $\Gamma^*$-module is a limit of a Mittag-Leffler sequence of rational graded $\Gamma^*$-modules. 
\end{corollary}
\begin{proof}
The sequence of monomorphisms (i.e., the left-hand nonzero column) in \eqref{comm diag 40990909} is eventually constant in each grading degree, hence is a Mittag-Leffler sequence in each grading degree. Consequently $R^1\lim_n \conn_n(M)$ vanishes in the ungraded category $\gr\Mod(\Gamma^*)$. (This argument does not show that $R^1\lim_n \conn_n(M)$ vanishes in the category $\Mod(\Gamma^*)$: the grading is important here.) Consequently we have the isomorphism $M\stackrel{\cong}{\longrightarrow}\lim_n \iota(\comod_n(M))$ in $\gr\Mod(\Gamma^*)$.
\end{proof}
It would be nice to have a rigorous way to interpret Theorem \ref{structure thm} as stating that ``the category of graded $\Gamma^*$-modules is an extension of the category of co-connected $\Gamma$-comodules by the category of connective $\Gamma^*$-modules,'' but there does not seem to be a notion of ``extension of abelian categories'' in the literature which is of the right kind of generality to include the situation of Theorem \ref{structure thm} as an example. Marmaridis's notion of ``extensions of abelian categories,'' from \cite{MR1213784}, do not suffice, since $\gr\Mod(\Gamma^*)$ is not monadic or comonadic over connective $\Gamma^*$-modules or over co-connected $\Gamma$-comodules (see Theorem 2.6 of \cite{MR1780016} for the relationship between (co)monadicity and Marmaridis's extension theory). The situation of Theorem \ref{structure thm} is not a special case of a ``deformation of abelian categories'' in the sense of \cite{MR2238922} either.

\begin{corollary}\label{uniqueness of module cat}
Let $A,\Gamma^*$ be as in Corollary \ref{modules are limits of comodules}. Then the only full subcategory of $\gr\Mod(\Gamma^*)$ which contains the rational $\Gamma^*$-modules and which is closed under kernels and countable products is $\gr\Mod(\Gamma^*)$ itself.
\end{corollary}
\begin{proof}
Sequential limits in $\gr\Mod(\Gamma^*)$ are kernels of maps between countable products, so this follows from Corollary \ref{modules are limits of comodules}.
\end{proof}


\section{Mitchell coalgebras.}
\label{Mitchell coalgebras...}

Throughout this section, and for the rest of this paper, we assume that the ground ring $A$ is a countable field, so that we may freely use Theorem \ref{main thm 1}.

A more involved, but nevertheless important, example of a filtered ideal set arises from a coalgebra satisfying the {\em Mitchell condition,} which we define in Definition \ref{def of mitchell condition}. It requires that we first recall (from chapter 13 of \cite{MR738973}) the definition of a  ``$\mathcal{P}$-algebra'':
\begin{definition}\label{def of p-alg}
A {\em $\mathcal{P}$-algebra} is a union of a sequence of subalgebras $B(0) \subsetneq B(1) \subsetneq \dots$ such that each $B(n)$ is a Poincar\'{e} algebra, and each $B(n+1)$ is flat over $B(n)$. Here a ``Poincar\'{e} algebra,'' as in \cite{MR335572}, is a finite-dimensional graded connected %\footnote{``Connected'' here means that $A$ is trivial in negative grading degrees and that the unit map $k\rightarrow A^0$ is an isomorphism.} 
algebra $A$ over a field $k$ such that there exists a map of graded $k$-modules $e: A\rightarrow \Sigma^{-n} k$, for some integer $n$, such that the pairing $A^q\otimes_k A^{n-q}\stackrel{\nabla}{\longrightarrow} A^n \stackrel{\Sigma^n e}{\longrightarrow} k$ is nonsingular.
\end{definition}
Of course the most important examples of $\mathcal{P}$-algebras are the Steenrod algebras: the mod $p$ Steenrod algebra is a $\mathcal{P}$-algebra for every prime $p$, by Proposition 7 from section 15.1 of \cite{MR738973}.

\begin{definition}\label{def of mitchell condition}
Suppose $\Gamma$ is a graded coalgebra over a field $A$. 
\begin{itemize}
\item By a {\em Mitchell decomposition of $\Gamma$} we mean the following data:
\begin{itemize} 
\item a sequence $\dots \rightarrow \Gamma(2)\rightarrow \Gamma(1) \rightarrow\Gamma(0)$ of surjective graded $A$-coalgebra morphisms, with each $\Gamma(n)$ a graded quotient $A$-coalgebra of $\Gamma$, 
%$\Gamma^*(0) \subsetneq \Gamma^*(1) \subsetneq \dots$ of graded $A$-subalgebras of $\Gamma^*$ 
such that each $\Gamma^*(n)$ is a Poincar\'{e} algebra, and such that the left $\Gamma^*(n)$-action on $\Gamma^*(n+1)$ arising from the dual map $\Gamma^*(n)\rightarrow\Gamma^*(n+1)$ of $\Gamma(n+1)\rightarrow\Gamma(n)$ makes $\Gamma^*(n+1)$ flat over $\Gamma^*(n)$.
\item For each nonnegative integer $n$, a homogeneous element $\omega_n \in \Gamma(n)$ whose associated map of graded $A$-modules $\Gamma^*(n)\rightarrow \Sigma^{-\left| n\right|} A$ has the property that the pairing 
\begin{equation}\label{pairing 0439} \Gamma^*(n)^q\otimes_A \Gamma^*(n)^{\left| n\right|-q}\stackrel{\nabla}{\longrightarrow} \Gamma^*(n)^{\left| n\right|} \stackrel{\Sigma^{\left| n\right|} e}{\longrightarrow} A\end{equation} is nonsingular.
%\item For each nonnegative integer $n$, a graded $A$-coalgebra $\Gamma(n)$ equipped with a surjective graded $A$-coalgebra homomorphism $\sigma_n: \Gamma\rightarrow\Gamma(n)$.
%\item For each nonnegative integer $n$, a surjective graded $A$-coalgebra homomorphism $\sigma_{n,n+1}: \Gamma(n+1) \rightarrow\Gamma(n)$. These maps are required to satisfy the condition that $\sigma_{n,n+1}\circ\sigma_{n+1} = \sigma_n$ for all $n$.
%\item A sequence of homogeneous elements $(S^0, S^1, S^2,\dots)$ of $\Gamma^*$ such that, for each nonnegative integer $n$, the graded $A$-subalgebra $\Gamma^*(n)$ of $\Gamma^*$ generated by $\{ S^0, \dots ,S^n\}$ coincides with the $A$-linear dual of the quotient coalgebra $\Gamma(n)$ of $\Gamma$.
\item For each nonnegative integer $n$, an extension of the natural graded left $\Gamma^*(n)$-module structure of $\Gamma^*(n)$ to a graded left action of $\Gamma^*$ on $\Gamma^*(n)$, together with graded left $\Gamma^*(n)$-module homomorphisms $\sigma_n: \Gamma^*\rightarrow \Gamma^*(n)$ and $\sigma_{n,n+1}: \Gamma^*(n+1) \rightarrow \Gamma^*(n)$ such that
\begin{itemize}
\item the composite of the $A$-algebra injection $\Gamma^*(n)\hookrightarrow \Gamma^*$ with the left $A$-module morphism $\sigma_n: \Gamma^* \rightarrow\Gamma^*(n)$ is the identity on $\Gamma^*(n)$,
\item the duality isomorphism $\Gamma^*(n)\stackrel{\cong}{\longrightarrow}\Sigma^{\left| n\right|}\Gamma^*(n)^*$ of graded left $\Gamma^*(n)$-modules, adjoint to \eqref{pairing 0439}, is in fact an isomorphism of graded left $\Gamma^*$-modules,
\item $\sigma_{n,n+1}\circ\sigma_{n+1} = \sigma_n$ for all $n$, 
\item and the resulting graded left $\Gamma^*$-module homomorphism $\Gamma^* \rightarrow \lim_{n\rightarrow\infty}\Gamma^*(n)$ is an isomorphism.
\end{itemize}
\end{itemize}
\item A {\em Mitchell coalgebra} is a graded coalgebra which admits a Mitchell decomposition. 
\item Given a Mitchell coalgebra $\Gamma$, by a {\em $\mathcal{P}$-sequence} we mean a sequence $\Gamma^*(0)\subseteq\Gamma^*(1)\subseteq\dots$ of graded $A$-subalgebras of $\Gamma^*$ as in the definition of a Mitchell decomposition. 
\item Given a Mitchell coalgebra and a choice of $\mathcal{P}$-sequence, by an {\em orientation sequence} we mean a sequence $\omega_0,\omega_1, \omega_2, \dots$ of elements as in the definition of a Mitchell decomposition.
\item Given a Mitchell coalgebra $\Gamma$ equipped with a choice of Mitchell decomposition, let $\Mit$ be the set of all intersections of finite collections of homogeneous left ideals of $\Gamma^*$ of the form $\ann_{\ell}(\omega_n)$. That is, a homogeneous left ideal $I$ of $\Gamma^*$ is a member of $\Mit$ if and only if there exists a finite set $N$ of nonnegative integers such that $I = \cap_{n\in N}\ann_{\ell}(\omega_n)$.
\end{itemize}
\end{definition}

The main theorem of \cite{MR793186}, expressed in the language just introduced, is that, for each prime $p$, the $p$-primary dual Steenrod algebra is a Mitchell coalgebra.
%(ABOUT ADJOINTS AND CONTRAGREDIENTS: it works out fine, look at what Mitchell actually proves! Include a comment anyway about how, given a finite-type Hopf $A$-algebra $\Gamma$, if we regard $\Gamma \cong \Gamma^{**}$ as a left $\Gamma^*$-module via the {\em contragredient} $\Gamma^*$-action, then the fact that the antipode $\chi: \Gamma^*\rightarrow\Gamma^*$ is an antihomomorphism makes $\Gamma$ also into a right $\Gamma^*$-module, and this right $\Gamma^*$-module agrees with $\Gamma$ with the {\em adjoint} $\Gamma^*$-action.)

Recall from Example \ref{examples of ideal sets} that $\grad$ is the filtered ideal set $\{ I_1, I_2, I_3, \dots\}$ in $\Gamma^*$, where $I_n$ is the left ideal (equivalently, two-sided ideal) in $\Gamma^*$ generated by all homogeneous elements of degree $\geq n$.
\begin{prop}\label{grad dist mit}
Suppose $A$ is a field and $\Gamma$ is a graded $A$-coalgebra concentrated in nonpositive degrees. Then we have $\dist\leq \grad$ in the preorder of filtered ideal sets in $\Gamma^*$. 

If $\Gamma$ is a finite-type Mitchell coalgebra, then we also have $\grad \leq\Mit\leq \dist$, and consequently $\grad$ and $\dist$ are equivalent in the preorder of filtered ideal sets in $\Gamma^*$. 
\end{prop}
\begin{proof}\leavevmode\begin{itemize}
\item
Given a strongly distinguished (see Definition-Proposition \ref{def of distinguished} for this definition) homogeneous left ideal $I$ of $\Gamma^*$, let $\gamma\in \iota(\Gamma)$ be a homogeneous element whose left annihilator ideal $\ann_{\ell}(\gamma)$ is $I$. Let $n$ be the degree of the homogeneous element $\gamma$. Since $\Gamma$ is coconnective, $n$ must be nonpositive, so every element of $\Gamma^*$ of degree $>n$ annihilates $\gamma$, that is, $I_{n+1}\subseteq I$. 

Now by definition (see Definition-Proposition \ref{def of distinguished}), every element $J$ of $\dist$ contains an intersection $\cap_{j=1}^m J_j$ of a finite set $J_1, \dots ,J_m$ of strongly distinguished homogeneous left ideals of $\Gamma^*$. For each $j=1, \dots,m$, choose a nonnegative integer $f(j)$ such that $I_{f(j)}\subseteq J_j$. Then we have $J\supseteq \cap_{j=1}^m J_j\supseteq \cap_{j=1}^m I_{f(j)} = I_{\max\{ f(1), \dots,f(m)\}}$, so every element $J$ of $\dist$ contains an element $I_n$ of $\grad$.
So $\dist\leq \grad$ in the preorder of filtered ideal sets in $\Gamma^*$.
\item
If $\Gamma$ is a finite-type Mitchell coalgebra and $\omega_0,\omega_1,\omega_2,\dots$ is an orientation sequence for $\Gamma$, then by applying the dual $\sigma_n^*: \Gamma(n)\rightarrow \Gamma$ of the left $\Gamma^*$-module map $\sigma_n: \Gamma^*\rightarrow \Gamma^*(n)$ to $\omega_n\in \Gamma(n)$, we get a sequence of elements $\sigma_0^*(\omega_0),\sigma_1^*(\omega_1),\sigma_2^*(\omega_2),\dots$ of $\Gamma$. Since $\sigma_n$ is surjective, its dual $\sigma_n^*$ is injective, so $\ann_{\ell}(\sigma_n^*(\omega_n)) = \ann_{\ell}(\omega_n)$ for each $n$. Consequently each of the left ideals $\ann_{\ell}(\omega_n)$ is a strongly distinguished, so the ideal set $\Mit$ is contained in the ideal set $\dist$, and consequently $\Mit\leq \dist$.

To show that $\grad\leq \Mit$, choose some nonnegative integer $n$. We need to show that there exists some member $I$ of $\Mit$ such that $I_n\supseteq I$. That is, we need to find a member $I$ of $\Mit$ which has no nonzero elements of degree $<n$. Since $\Gamma^*$ is finite-type and since the map $\Gamma^*\rightarrow \lim_m\Gamma^*(m)$ is an isomorphism, there exists some positive integer $q$ such that the map $\Gamma^*\rightarrow \Gamma^*(m)$ is an isomorphism in degrees $\leq n$ for all $m\geq q$. By self-duality of $\Gamma^*(m)$, the left annihilator of $\omega_m$ contains no nonzero elements of $\Gamma^*(m)$ itself, so if $m\geq q$, $\ann_{\ell}(\omega_m)$ contains no nonzero elements of degree $<n$, as desired.
\end{itemize}
\end{proof}


To a filtered ideal set $S$ in a graded ring $R$, we have an associated local-cohomology-like functor $\colim_{I\in S}\underline{\hom}_R\left( R/RI,-\right) : \gr\Mod(R) \rightarrow \gr\Ab$ equipped with a natural transformation $\colim_{I\in S}\underline{\hom}_R\left( R/RI,-\right) \hookrightarrow F$, where $F$ is the forgetful functor $F: \gr\Mod(R)\rightarrow\gr\Ab$. Put simply, $\colim_{I\in S}\underline{\hom}_R\left( R/RI,M\right)$ is the subgroup of $M$ consisting of the homogeneous elements of $M$ that are $I$-torsion for some $I\in S$.  For brevity, we refer to elements of a left $R$-module $M$ that are $I$-torsion elements for some $I\in S$ as {\em $S$-torsion elements} of $M$.

Sometimes we are fortunate, and the image of the natural injection of abelian groups $\colim_{I\in S}\underline{\hom}_R\left( R/RI,M\right)\hookrightarrow M$ is actually an $R$-submodule of $M$. That is, when we are lucky, the set of $S$-torsion elements of $M$ is closed under left $R$-scalar multiplication, so that $h^0_S(M) \cong H^0_S(M)$, for all $R$-modules $M$. We will call $S$ {\em closed} when this condition is satisfied. 
\begin{remark}
If $R$ is commutative, then every filtered ideal set in $R$ is closed. On the other hand, if $R$ is noncommutative, then $R$ may have some nonclosed filtered ideal sets. One example was given in Remark \ref{remark on ideal sets}. %Another example is as follows: let $k$ be a field, and let $R$ be the free associative $k$-algebra on two degree $1$ generators $x$ and $y$. Let $\Theta = R/Rx$. Then $1\in \Theta$ is in the subgroup $h^0_{\dist(\Theta)}(R)$ of $\Theta$, but $y$ isn't, so $h^0_{\dist(\Theta)}(R)$ isn't an $R$-submodule of $R$, and consequently $h^0_{\dist(\Theta)}(R)$ fails to agree with $H^0_{\dist(\Theta)}(R)$, i.e., the ideal set $\dist(\Theta)$ is not closed. (DOUBLE CHECK THIS EXAMPLE---NOT SURE I BUY THAT DIST(THETA) REALLY IS JUST THE LEFT IDEAL GENERATED BY x.)
\end{remark}

\begin{lemma}\label{closure closed under equivalence}
Let $R$ be a graded ring.
If two filtered ideal sets $S,S^{\prime}$ in $R$ are equivalent, and if $S$ is closed, then $S^{\prime}$ is also closed.
\end{lemma}
\begin{proof}
Easy consequence of the definitions.
\end{proof}

\begin{prop}\label{grad is closed}
Suppose $R$ is a graded ring concentrated in nonnegative degrees. Then the filtered ideal set $\grad$ is closed.
\end{prop}
\begin{proof}
Given a graded left $R$-module $M$, a homogeneous element $m\in M$ is $\grad$-torsion if and only if there exists some integer $n(m)$ such that $rm=0$ for all homogenous $r\in R$ of degree $>n(m)$. If $r^{\prime}\in R$ is homogeneous, then for all homogeneous $r\in R$ of degree $>n(m)$, we have that $rr^{\prime}$ is also homogeneous of degree $>n(m)$, and consequently $rr^{\prime}m=0$. So the set of $\grad$-torsion homogeneous elements of $M$ is closed under left multiplication by homogeneous elements of $R$.
\end{proof}

\begin{theorem}\label{main thm on mitchell coalgebras}
Suppose $A$ is a field and $\Gamma$ is a Mitchell coalgebra over $A$. Then the following statements are each true:
\begin{enumerate}
\item The filtered ideal sets $\dist$ and $\grad$ in $\Gamma^*$ are equivalent. In particular, $\dist$ is closed.
\item The functor $h^0_{\dist}: \gr\Mod(R)\rightarrow\gr\Ab$ is equivalent to the composite of $H^0_{\dist}: \gr\Mod(R)\rightarrow\gr\Mod(R)$ with the forgetful functor $\gr\Mod(R)\rightarrow\gr\Ab$.
\item For every integer $n$, the distinguished local cohomology functor $H^n_{\dist}$ is naturally equivalent to $\colim_{I\in \dist(\Gamma)}\Ext_{\Gamma^*}\left( \Gamma^*/\Gamma^*/I,-\right)$.
\item For every integer $n$, the distinguished local cohomology functor $H^n_{\dist}$ is naturally equivalent to $\colim_{j\rightarrow\infty}\Ext_{\Gamma^*}\left( \Gamma^*/\Gamma^*/I_j,-\right)$, where $I_j$ is as in the definition of $\grad$, i.e., $I_j$ is the left ideal of $\Gamma^*$ generated by all homogeneous elements of degree $\geq j$.
%\item Let $\Gamma^*(0)\subseteq\Gamma^*(1)\subseteq\dots$ be a $\mathcal{P}$-sequence in $\Gamma^*$, and for each nonnegative integer $j$, let $I(j)$ denote the homogeneous left ideal of $\Gamma^*$ generated by the homogeneous elements of $\Gamma^*(j)$. Then, for every integer $n$, the distinguished local cohomology functor $H^n_{\dist}$ is naturally equivalent to $\colim_{j\rightarrow\infty}\underline{\hom}_{\Gamma^*}\left( \Gamma^*/I(j),-)$.
\end{enumerate}
\end{theorem}
\begin{proof}\leavevmode
\begin{enumerate}
\item Immediate from Proposition \ref{grad dist mit}.
\item 
By Proposition \ref{grad is closed}, $\grad$ is closed.
By the previous part of this theorem, $\dist$ and $\grad$ are equivalent. By Lemma \ref{closure closed under equivalence}, $\dist$ is consequently also closed, so $h^0_{\dist}\cong H^0_{\dist}$.
\item Since $\dist$ is filtered, the colimit $\colim_{I\in \dist}$ is exact, so we have 
\begin{align*}
 H^n_{\dist}(M) 
  &\cong R^nH^0_{\dist}(M) \\
  &\cong R^nh^0_{\dist}(M) \\
  &\cong R^n\left(\colim_{I\in \dist}\underline{\hom}_{\Gamma^*}\left( \Gamma^*/\Gamma^*I,-\right)\right)(M) \\
  &\cong \colim_{I\in \dist}R^n\left(\underline{\hom}_{\Gamma^*}\left( \Gamma^*/\Gamma^*I,-\right)\right)(M) \\
  &\cong \colim_{I\in \dist}\Ext_R^n\left( \Gamma^*/\Gamma^*I,M\right).
\end{align*}
\item Immediate from the preceding part of this theorem together with the equivalence of the filtered ideal sets $\grad$ and $\dist$.
%\item From Proposition \ref{grad dist mit} we have that the filtered ideal sets $\Mit$ and $\dist$ are equivalent, so using the filteredness of $\Mit$ we get an isomorphism $H^*_{\dist}(M) \cong \colim_{I\in\Mit}\Ext^*_{\Gamma^*}(\Gamma^*/\Gamma^*I,M)$....?
\end{enumerate}
\end{proof}

See Remark \ref{remark on h0 and H0} for some discussion of why Theorem \ref{main thm on mitchell coalgebras} matters.

\begin{corollary}\label{main cor 10}
Let $p$ be a prime number, and let $\Gamma^*$ be the $p$-primary Steenrod algebra. Let $M$ be a graded $\Gamma^*$-module. Then, for all integers $n$, the distinguished local cohomology $H^n_{\dist}(M)$ is isomorphic to $\colim_m \Ext_{\Gamma^*}^n\left( \Gamma^*/I^m,M\right)$, where $I$ is the augmentation ideal of $\Gamma^*$. That is, $H^n_{\dist}(M) \cong h^n_{\dist}(M)\cong H^n_I(M)$.
\end{corollary}

\begin{corollary}\label{derived products are local cohomology}
Let $p$ be a prime, let $\Gamma$ be the dual $p$-primary Steenrod algebra, and let $\{ M_i: i\in I\}$ be a set of graded $\Gamma$-comodules. Then the $n$th derived functor $R^n\prod^{\Gamma}_{i\in I} M_i$ of product in the category of graded $\Gamma$-comodules is isomorphic to $\colim_m \Ext_{\Gamma^*}^n\left( \Gamma^*/I^m,\prod_i M_i\right)$, where $I$ is the augmentation ideal of the $p$-primary Steenrod algebra $\Gamma^*$, and where $\prod_i M_i$ is the product (in the category of modules, i.e., the Cartesian product) of the graded $\Gamma^*$-modules $M_i$ with the adjoint action of $\Gamma^*$.
\end{corollary}

Given an $\mathbb{N}$-graded ring $R$, write $I$ for its ideal generated by all homogeneous elements in positive degrees. The functor $\colim_m \Ext_R^n\left( R/I^m,-\right)$ has been studied under the name of {\em graded local cohomology}: as far as I know, the original reference is \cite{MR494707}, which only considered commutative Noetherian rings, while in the noncommutative case, \cite{MR1428799} seems to be the first reference, although it still assumes a Noetherian condition on the graded noncommutative ring. In this language, Corollary \ref{derived products are local cohomology} establishes that derived products of comodules over the dual Steenrod algebra are isomorphic to graded local cohomology of the Steenrod algebra with coefficients in the Cartesian product of those comodules, with the adjoint action. Unfortunately I have not found any results in the literature on noncommutative graded local cohomology which appear to be useful in the case where $R$ is a Steenrod algebra.



\section{Bounds on distinguished local-cohomological dimension.}
\label{Bounds on distinguished...}

%\subsection{Infinite distinguished-cohomological dimension of the category of modules.}
%\label{Infinite distinguished-cohomological...}

The most fundamental and well-known vanishing theorem in classical local cohomology is this: if $I$ be an ideal in a Noetherian commutative ring, then the classical local cohomology groups $H^n_I(M)$ vanish for all $R$-modules $M$ whenever $n$ is greater than the least number of generators for $I$. One wants a generalization of this result which applies to distinguished local cohomology. However, $h^*_{\dist}$ is given by a colimit of $\Ext$ groups associated to a set $S$ of ideals which is not generally the sequence of powers of any single ideal $I$, so it is not clear whether one ought to expect $h^n_{\dist}(M)$ to vanish for all $n$ larger than some particular integer. It is even less clear what to expect from $H^n_{\dist}$ except when the relevant coalgebra $\Gamma$ is either co-commutative, or finite-type and Mitchell, so that $H^n_{\dist}$ agrees with $h^n_{\dist}$.

The main result of this section is Theorem \ref{main thm 4}, which establishes that, for a certain class of coalgebras $\Gamma$ (including those whose dual is a $\mathcal{P}$-algebra, so in particular, including the dual Steenrod algebras), there is no such vanishing theorem for distinguished local cohomology. As a consequence we get that the graded comodule category over such a coalgebra fails to be $AB4^*\mhyphen (n)$ for any $n$ whatsoever.

\begin{theorem}\label{main thm 4}
Suppose that $A$ is a countable field, and that $\Gamma^*$ is finite-type and connected.
Suppose furthermore that every bounded-below free graded $\Gamma^*$-module is injective in $\gr\Mod(\Gamma^*)$.
Then the following are equivalent:
\begin{enumerate}
\item The category of graded $\Gamma^*$-modules has distinguished local cohomological dimension zero. That is, $H^m_{\dist}(M)$ vanishes for all positive $m$ and all graded $\Gamma^*$-modules $M$.
\item The category of graded $\Gamma^*$-modules has finite distinguished local cohomological dimension. That is, there exists some integer $n$ such that $H^m_{\dist}(M)$ vanishes for all $m>n$ and all graded $\Gamma^*$-modules $M$.
\item The category of graded $\Gamma$-comodules satisfies axiom $AB4^*\mhyphen (n)$ for some nonnegative integer $n$. 
That is, there exists some integer $n$ such that $R^m\prod^{\Gamma}_i\left( \left\{ M_i\right\}\right)$ vanishes for all $m>n$, all countable sets $I$, and all sets $\{ M_i:i\in I\}$ of graded $\Gamma$-comodules.
\item The category of graded $\Gamma$-comodules satisfies Grothendieck's property $AB4^*$. That is, countable products are exact in the category of graded $\Gamma$-comodules.
\end{enumerate}
\end{theorem}
\begin{proof}\leavevmode
\begin{description}
\item[(1) implies (2)] Immediate.
\item[(2) implies (1)] Let $M$ be a graded $\Gamma^*$-module. By Theorem \ref{structure thm}, $M$ has the same distinguished local cohomology in positive degrees as the graded $\Gamma^*$-module $\conn_0(M)$ of $M$ generated by all homogeneous elements of nonnegative degree. Since $\conn_0(M)$ is trivial in negative degrees, there exists a connective free graded $\Gamma^*$-module $F$ and an epimorphism $\epsilon: F \rightarrow \conn_0(M)$. Since $F$ is injective, % (by Theorem \ref{thm from margolis 2}), 
we have $H^{i+1}_{\dist}(\ker \epsilon)\cong H^{i}_{\dist}(\conn_0(M))\cong H^{i}_{\dist}(M)$ for all $i\geq 1$. So if $H^i_{\dist}$ is nonzero on some graded $\Gamma^*$-module for some positive $i$, then $H^{i+1}_{\dist}$ is also nonzero on some graded $\Gamma^*$-module. Consequently, under the stated hypotheses, the only way to have a finite cohomological bound on distinguished local cohomology is for that bound to be zero.
\item[(2) implies (3)] Special case of Theorem \ref{main thm 1}.
\item[(1) implies (4)] Special case of Theorem \ref{main thm 1}.
\item[(4) implies (3)] Immediate, since $AB4^*$ is the same condition as $AB4^*\mhyphen (0)$.
\item[(3) implies (2)] Suppose that the category of graded $\Gamma$-comodules satisfies axiom $AB4^*\mhyphen (n)$ for some nonnegative integer $n$. Suppose that $M$ is a graded $\Gamma^*$-module. By Corollary \ref{modules are limits of comodules}, $M$ is isomorphic to the limit of a Mittag-Leffler sequence $\dots \rightarrow N_2 \rightarrow N_1 \rightarrow N_0$ of rational graded $\Gamma^*$-modules. Consequently we have a short exact sequence of graded $\Gamma^*$-modules
\begin{equation}\label{ses 30-204949} 0 \rightarrow M \rightarrow \prod_{i\in I} N_i \stackrel{\id - T}{\longrightarrow} \prod_{i\in I} N_i \rightarrow 0\end{equation} with each of the modules $N_0,N_1,N_2,\dots$ rational. 
Since each of the $N_i$ are rational, Theorem \ref{main thm 1} gives us an isomorphism of abelian groups $H^*_{\dist}\left( \prod_{i\in I} N_i\right) \cong R^*\prod^{\Gamma}_i\left( \{ N_i :i\in\mathbb{N}\}\right)$, so the $AB4^*\mhyphen (n)$ assumption on the comodule category gives us that $H^j_{\dist}\left( \prod_{i\in I} N_i\right)$ vanishes for all $j>n$.
The long exact sequence obtained by applying $H^*_{\dist}$ to \eqref{ses 30-204949} consequently gives us that $H^j_{\dist}(M)$ vanishes for all $j>n+1$.
That is, the category of graded $\Gamma^*$-modules has distinguished local cohomological dimension at most $n+1$.
\end{description}
\end{proof}

\begin{corollary}\label{main cor 4a}
Suppose the field $A$ is countable, and suppose that the $A$-algebra $\Gamma^*$ is a finite-type $\mathcal{P}$-algebra. Then {\em exactly one} of the two following statements is true:
\begin{enumerate}
\item The category of graded $\Gamma$-comodules satisfies condition $AB4^*$. That is, products are exact in the category of graded $\Gamma$-comodules.
\item For each integer $n$, the category of graded $\Gamma$-comodules fails to satisfy condition $AB4^*\mhyphen (n)$. That is, for each integer $n$, there exists a countably infinite set $\{ M_0, M_1, M_2, \dots\}$ of graded $\Gamma$-comodules such that $R^m\prod^{\Gamma}_i \left( \{ M_i\}\right)$ is nonzero for some $m>n$.
\end{enumerate}
\end{corollary}
\begin{proof}
By Theorem \ref{thm from margolis 2} from the appendix on Margolis' results on $\mathcal{P}$-algebras, every bounded-below free graded $\Gamma^*$-module is injective. Consequently the hypotheses of Theorem \ref{main thm 4} are satisfied.
\end{proof}

\begin{corollary}\label{no ab4n for dual steenrod alg}
Let $p$ be a prime number, and let $\Gamma$ denote the mod $p$ dual Steenrod algebra. Then, for each integer $n$, the category of graded $\Gamma$-comodules fails to satisfy condition $AB4^*\mhyphen (n)$.
That is, there exists a countably infinite set $M_0, M_1, M_2, \dots$ of graded $\Gamma$-comodules such that the derived product $R^m\prod^{\Gamma}_i\left(\left\{ M_i: i\in \mathbb{N}\right\}\right)$ is nonzero for some $m>n$.
\end{corollary}
\begin{proof}
It is already well-known (but we still give some explanation in the rest of this proof) that the category of graded comodules over the dual Steenrod algebra does not have exact products, so the claim is a corollary of Corollary \ref{main cor 4a}.

If the category of graded comodules over the dual Steenrod algebra had exact products, this would imply that, for any countable set $\{ X_i\}$ of $H\mathbb{F}_p$-nilpotently complete, the Hovey-Sadofsky spectral sequence 
\[ E_2^{*,*} \cong R^*\prod_{i\in I}^{\Gamma}\left( \{ H_*(X_i;\mathbb{F}_p)\}\right) \Rightarrow H_*\left( \prod_{i\in I} X_i;\mathbb{F}_p\right)\] collapses on to the $R^0\prod_I^{\Gamma}$-line at the $E_2$-page. One can consult Sadofsky's unpublished preprint \cite{sadofsky2001homology} for the original construction of this spectral sequence (which is given there more generally for sequential limits, not just countable products), Hovey's paper \cite{MR2337861} for a published account (which, however, discusses convergence only when $H_*(-;\mathbb{F}_p)$ is replaced with a Morava $E$-theory), and the appendix of \cite{MR4080481} for a published account which discusses convergence of the spectral sequence in the case at hand, i.e., the case of mod $p$ homology. Collapse of the spectral sequence for all $\{ X_i: i\in\mathbb{N}\}$ would imply that mod $p$ homology commutes with all countable products of $H\mathbb{F}_p$-local spectra, and consequently that mod $p$ homology commutes with all sequential homotopy limits of $H\mathbb{F}_p$-nilpotently complete spectra, which is well-known to be untrue without some additional hypothesis (e.g. that the spectra are uniformly bounded below).
%for example, by the chromatic convergence theorem of Ravenel and Hopkins, the $p$-local sphere spectrum is weakly equivalent to the homotopy limit of the sphere spectrum localized at each of the Johnson-Wilson homology theories, i.e., $L_{(p)}S \simeq \holim_n L_{E(n)}S$. The $E(n)$-local sphere spectrum $L_{E(n)}S$ is the homotopy limit of an $E(n)$-Adams resolution of the sphere, by the proof (by Devinatz-Hopkins-Smith) that the sphere spectrum is $E(n)$-prenilpotent in the sense of Bousfield. See the book \cite{MR737778} for sustained discussion of these ideas. Every stage in an $E(n)$-Adams resolution is an $E(n)$-module spectrum, hence is $H\mathbb{F}_p$-acyclic since $E(n)$ is $H\mathbb{F}_p$-acyclic. If mod $p$ homology were to commute with countable products, then it would also commute with sequential homotopy limits, and consequently $L_{E(n)}S$ would also be $H\mathbb{F}_p$-acyclic, and so $L_{(p)}S$ would be $H\mathbb{F}_p$-acyclic too. But this is not true: $H_*(L_{(p)}S; \mathbb{Z}_{(p)}) \cong H_*(S; \mathbb{Z}_{(p)})\cong \mathbb{Z}_{(p)}$, and consequently $H_*(L_{(p)}S; \mathbb{F}_p)\cong \mathbb{F}_p$. (THIS EXAMPLE IS BAD, SINCE THE $E(n)$-LOCAL SPHERE ISN'T $HF_p$-LOCAL!)
An explicit algebraic example of the failure of products to be exact in the category of graded comodules over the dual Steenrod algebra can also be found following Proposition 2.13 in \cite{sadofsky2001homology}.
\end{proof}

\section{Characterization of the modules acyclic with respect to distinguished local cohomology.}
\label{Characterization of the modules...}

\begin{prop}\label{ext and Hn triviality prop}
Let $A$ be a field and let $\Gamma$ be a graded $A$-coalgebra such that the ideal set $\dist(\Gamma)$ is equivalent to the set of powers of some graded two-sided ideal $I$ of $\Gamma^*$ such that the ring $\Gamma^*/I$ is graded-semisimple. (For example, we can take $\Gamma$ to be any Mitchell coalgebra, such as the dual Steenrod algebra, by Theorem \ref{main thm on mitchell coalgebras}. Then $I$ can be taken to be the ideal generated by all homogeneous element of positive degree.) Then, for a given graded $R$-module $M$, the following are equivalent:
\begin{enumerate}
\item For all positive integers $n$, $H^n_{\dist}(M)$ is trivial.
\item For all integers $n$,  $\Ext^n_{\Gamma^*}(\Gamma^*/I,M/H^0_{\dist}(M))$ is trivial.
\end{enumerate}
\end{prop} 
\begin{proof}\leavevmode
\begin{description}
\item[1 implies 2] Since $I$ is a two-sided ideal, $\underline{\hom}_{\Gamma^*}(\Gamma^*/I^n,M)$ is a graded $\Gamma^*$-submodule of $M$ for each positive integer $n$ and each graded $\Gamma^*$-module $M$. Consequently the ideal set of powers of $I$ is closed, and so we have the natural isomorphism $H^*_{\dist}\cong h^*_{\dist}$ and consequently isomorphisms 
\begin{align*}
 H^n_{\dist}(M) 
  &\cong h^n_{\dist}(M) \\
  &\cong \colim_{m}\Ext^n_{\Gamma^*}\left( \Gamma^*/I^m,M \right) 
\end{align*}
for each integer $n$.
The short exact sequence of graded $\Gamma^*$-modules
\begin{equation}\label{ses 0404330493057} 0 \rightarrow I^m/I^{m+1} \rightarrow \Gamma^*/I^{m+1}\rightarrow \Gamma^*/I^m \rightarrow 0\end{equation}
induces an injection $\underline{\hom}_{\Gamma^*}(\Gamma^*/I^m,M) \hookrightarrow \underline{\hom}_{\Gamma^*}(\Gamma/I^{m+1},M)$ for each $M$. Consequently the vanishing of $H^0_{\dist}(M/H^0_{\dist}(M)))$ implies the vanishing of $\underline{\hom}_{\Gamma^*}(\Gamma^*/I^m,M/H^0_{\dist}(M))$ for each $m$. 

Now we set up an induction. Suppose that $N$ is some positive integer, and suppose that we have already proven that $\Ext_{\Gamma^*}^n(\Gamma/I,M/H^0(M))$ is trivial for all $n<N$.
Since $\Gamma^*/I$ is graded-semisimple, the graded $\Gamma^*$-module $I^m/I^{m+1}$ splits as a coproduct of suspensions of summands of $\Gamma^*/I$ for each integer $m$. Consequently the vanishing of $\Ext^{N-1}_{\Gamma^*}(\Gamma^*/I,M/H^0_{\dist}(M))$ yields the vanishing of $\Ext^{N-1}_{\Gamma^*}(I^m/I^{m+1},M/H^0_{\dist}(M))$ for each $m$.  The long exact sequence induced in $\Ext^*_{\Gamma^*}(-,M/H^0_{\dist}(M))$ by \eqref{ses 0404330493057} then yields that \[\Ext^N_{\Gamma^*}(\Gamma^*/I^m,M/H^0_{\dist}(M))\rightarrow\Ext^N_{\Gamma^*}(\Gamma^*/I^{m+1},M/H^0_{\dist}(M))\] 
is injective for each $m$. Theorem \ref{main thm 1} gives us that $H^N_{\dist}(M/H^0_{\dist}(M))\cong H^N_{\dist}(M)$, which vanishes by assumption. Since $H^N_{\dist}(M/H^0_{\dist}(M)) \cong \colim_{m}\Ext^N_{\Gamma^*}\left( \Gamma^*/I^m,M/H^0_{\dist}(M) \right)$, the colimit of a sequence of injections, the only way that this colimit can vanish is if each term is trivial, and in particular, the first term $\Ext^N_{\Gamma^*}\left( \Gamma^*/I,M/H^0_{\dist}(M) \right)$ must be trivial, completing the inductive step. Hence $\Ext^N_{\Gamma^*}\left( \Gamma^*/I,M/H^0_{\dist}(M) \right)$ must vanish for all $N$.
\item[2 implies 1] By Theorem \ref{main thm 1}, for positive $n$ we have isomorphisms
\begin{align}
\nonumber H^n_{\dist}(M) 
  &\cong H^n_{\dist}(M/H^0_{\dist}(M)) \\
\nonumber  &\cong h^n_{\dist}(M/H^0_{\dist}(M)) \\
\label{iso 43094995}  &\cong \colim_m\Ext_{\Gamma^*}^n(\Gamma^*/I^m,M/H^0_{\dist}(M)).\end{align}
Graded-semisimplicity of $\Gamma^*/I$ again yields that each quotient $I^m/I^{m+1}$ splits as a coproduct of suspensions of summands of $\Gamma^*/I$, hence 
$\Ext_{\Gamma^*}^n(I^m/I^{m+1},M/H^0_{\dist}(M))$ vanishes for all $m$ and all $n$. Since $\Gamma^*/I^m$ admits a finite filtration by the quotients by lower powers of $I^m$, we have vanishing of $\Ext_{\Gamma^*}^n(\Gamma^*/I^m,M/H^0_{\dist}(M))$ for all $m$ and all $n$. Isomorphism \eqref{iso 43094995} then yields the desired vanishing of $H^n_{\dist}(M)$.
\end{description}\end{proof}

%\begin{lemma} Suppose that $R$ is a connected graded algebra over a field $A$. If $M$ is a bounded below graded $R$-module such that $\Ext^n_R(A,M)\cong 0$ for all $n$, then $M$ is injective. (IS THIS TRUE?) \end{lemma}

We say that a graded $\Gamma^*$-module $M$ is ``$H^0_{\dist}$-acyclic'' if $H^n_{\dist}(M)$ vanishes for all $n>0$.
\begin{theorem}\label{decomposition of acyclics}
Let $A$ be a field and let $\Gamma$ be a Mitchell coalgebra over $A$. Then a graded $\Gamma^*$-module $M$ is $H^0_{\dist}$-acyclic if and only if $M$ admits graded $\Gamma^*$-submodules
\begin{equation}\label{filt 20940} M^{\prime\prime}\subseteq M^{\prime}\subseteq M\end{equation}
such that:
\begin{itemize}
\item $M^{\prime\prime}$ is bounded below and distinguished-torsion,
\item $M^{\prime}/M^{\prime\prime}$ is bounded below and has the property that $\Ext^n_{\Gamma^*}(A,M^{\prime}/M^{\prime\prime})$ vanishes for all $n$,
\item and $M/M^{\prime}$ is bounded above.\end{itemize}
\end{theorem}
\begin{proof}\leavevmode
\begin{itemize}
\item Suppose that $M$ admits submodules as in \eqref{filt 20940}. Then $M^{\prime\prime}$ (by Theorem \ref{main thm 1}) and $M^{\prime}/M^{\prime\prime}$ (by Proposition \ref{ext and Hn triviality prop}) each have trivial higher distinguished local cohomology, and $M^{\prime}$ is an extension of the latter by the former, so $M^{\prime}$ is $H^0_{\dist}$-acyclic.
Similarly, Theorem \ref{main thm 1} gives us that $M/M^{\prime}$ is rational and hence $H^0_{\dist}$-acyclic since it is bounded above. So $M$ is $H^0_{\dist}$-acyclic.
\item Conversely, suppose that $M$ is $H^0_{\dist}$-acyclic. Let $M^{\prime}$ be the graded $\Gamma^*$-submodule of $M$ generated by all elements in degrees $\geq 0$, and let $M^{\prime\prime}$ be $H^0_{\dist}(M^{\prime})$. Then $M/M^{\prime}$ is certainly bounded above (by zero), and $M^{\prime\prime}$ is certainly distinguished-torsion and bounded below (by zero). It is clear that $M^{\prime}/M^{\prime\prime}$ is bounded below (by zero), and the vanishing of $H^n_{\dist}(M)$ for $n>0$ implies the vanishing of $H^n_{\dist}(M^{\prime})$ for $n>0$ as well. By Proposition \ref{ext and Hn triviality prop}, $\Ext^n_{\Gamma^*}(A,M^{\prime}/M^{\prime\prime})$ vanishes for all $n$, as claimed.
\end{itemize}
\end{proof}


\section{Derived functors of sequential limit in comodule categories.}
\label{Derived functors of seq...}

Let $\Gamma$ be a graded coalgebra over a field $A$, and let $\mathcal{D}$ be a small category. We write $\iota: \gr\Comod(\Gamma) \rightarrow \gr\Mod(\Gamma^*)$ for the inclusion of comodules into modules via the adjoint $\Gamma^*$-action, and we write $\tr: \gr\Mod(\Gamma^*) \rightarrow \gr\Comod(\Gamma)$ for the right adjoint of $\iota$.
For any functor $\mathcal{F}:\mathcal{D}\rightarrow\gr\Comod(\Gamma)$, we can form the Bousfield-Kan construction 
\begin{equation}\label{bk construction} \xymatrix{ 
\prod_{x\in N\mathcal{D}_0} \iota\mathcal{F}(\cod x) \ar@<1ex>[r] \ar@<-1ex>[r] & 
 \prod_{x\in N\mathcal{D}_1} \iota\mathcal{F}(\cod x)  \ar[l]\ar@<2ex>[r]\ar[r]\ar@<-2ex>[r] & 
 \prod_{x\in N\mathcal{D}_2} \iota\mathcal{F}(\cod x)  \ar@<1ex>[l] \ar@<-1ex>[l] \ar@<3ex>[r] \ar@<1ex>[r] \ar@<-1ex>[r] \ar@<-3ex>[r] & \dots \ar@<2ex>[l]\ar[l]\ar@<-2ex>[l] , }\end{equation}
a cosimplicial object of $\gr\Mod(\Gamma^*)$. The notation is as follows: $N\mathcal{D}_{\bullet}$ is the nerve of $\mathcal{D}$, so for $i>0$, $N\mathcal{D}_{i}$ is the set of composable $i$-tuples of morphisms of $\mathcal{D}$, while $N\mathcal{D}_{0}$ is the set of objects of $\mathcal{D}$. The notation $\cod x$ is supposed to be the final object appearing in a composable $i$-tuple of morphisms $x$; in the case $i=1$, this is simply the codomain of a morphism.
We write $C^{\bullet}(\iota\mathcal{F})$ for the Moore complex (i.e., alternating sum cochain complex) of \eqref{bk construction}, so that $H^i(C^{\bullet}(\iota\mathcal{F})) \cong (R^i\lim)(\iota\mathcal{F})$ by Proposition XI.6.2 of \cite{MR0365573}.

\begin{theorem}\label{derived sequential lim thm}
Suppose that $A$ is a countable field and that $\Gamma$ is a finite-type $A$-coalgebra such that $\Gamma^*$ is connected.
Let \begin{equation}\label{seq 430949598}\dots\rightarrow M_2\rightarrow M_1\rightarrow M_0\end{equation} be a sequence of graded $\Gamma$-comodules such that $R^1\lim$ vanishes on the sequence of graded $\Gamma^*$-modules 
$\dots\rightarrow \iota M_2\rightarrow \iota M_1\rightarrow \iota M_0$.
Then there is an isomorphism 
\begin{align}\label{iso 09994} 
 \iota R^s\lim^{\Gamma}_iM_i &\cong H^s_{\dist}(\lim_i \iota M_i)\end{align}
for each $s$,
and consequently a long exact sequence
\begin{equation}\label{les 239094}\xymatrix{
 0 \ar[r] & 
  \iota \lim^{\Gamma}_i M_i \ar[r] & 
  H^0_{\dist}\left( \prod_i \iota M_i\right) \ar[r] &
  H^0_{\dist}\left( \prod_i \iota M_i\right) \ar`r_l[ll] `l[dll] [dll] \\
 & \iota R^1\lim^{\Gamma}_i M_i \ar[r] &
  H^1_{\dist}\left( \prod_i \iota M_i\right) \ar[r] &
  H^1_{\dist}\left( \prod_i \iota M_i\right) \ar`r_l[ll] `l[dll] [dll] \\
 & \iota R^2\lim^{\Gamma}_i M_iA_2 \ar[r] &
  H^2_{\dist}\left( \prod_i \iota M_i\right) \ar[r] &
  \dots .}\end{equation}
\end{theorem}
\begin{proof}
Consider the natural numbers $\mathbb{N}$ under their usual ordering, and let $\mathbb{N}^{\op}$ be the opposite category of the resulting partially-ordered set. Then the sequence \eqref{seq 430949598} is a functor $\mathcal{F}: \mathbb{N}^{\op}\rightarrow\gr\Comod(\Gamma)$.
Since $\tr$ is a right adjoint, it is left exact, so we have the hypercohomology spectral sequences
\begin{align*}
 {}^{\prime}E_1^{s,t} 
  & \cong R^t\tr(C^s\iota \mathcal{F}) \\
 {}^{\prime\prime}E_2^{s,t} 
  &\cong R^s\tr(H^t C^{\bullet}\iota\mathcal{F}) \\
  &\cong R^s\tr(R^t\lim \iota\mathcal{F}) ,
\end{align*}
each of which converges to the cohomology of the total complex of $\tr$ applied to $C^{\bullet}\iota\mathcal{F}$. 
Since $R^t\lim \iota\mathcal{F}$ is a derived functor of sequential limit in a category of modules, it is only capable of being nontrivial in degrees $t=0$ and $t=1$, and $R^1\lim \iota\mathcal{F}$ vanishes by assumption. Hence the spectral sequence ${}^{\prime\prime}E_r^{*,*}$ collapses at the ${}^{\prime\prime}E_2$-page, with ${}^{\prime\prime}E_2^{s,0}\cong R^s\tr(\lim\iota\mathcal{F})$ and with ${}^{\prime\prime}E_2^{s,t}$ trivial for $t>0$.

Meanwhile, ${}^{\prime}E_1^{*,t}$ is precisely the Bousfield-Kan construction $C^{\bullet}(R^t\tr \iota\mathcal{F})$, since $\tr$ commutes with products and sends injectives to injectives (since it has an exact left adjoint, namely $\iota$). So ${}^{\prime}E_2^{s,t} \cong R^s\lim^{\Gamma} (R^t\tr (\iota\mathcal{F}))\cong R^s\lim^{\Gamma}_i (R^t\tr (\iota M_i))$, which is trivial for $t>0$, since $\iota$ is exact and faithful and since $\iota R^t\tr (\iota M_i) \cong H^t_{\dist}(\iota M_i) \cong 0$ for $t>0$, as distinguished local cohomology vanishes on rational modules.
Hence ${}^{\prime}E_2^{s,0} \cong R^s\lim^{\Gamma}( \tr\iota\mathcal{F}) \cong R^s\lim^{\Gamma} \mathcal{F}$, and ${}^{\prime}E_2^{s,t}$ is trivial for $t=0$.

Since both spectral sequences converge to the cohomology of the same total complex, we have $R^n\lim^{\Gamma} \mathcal{F} \cong R^n\tr(\lim\iota\mathcal{F})$ for all $n$, and on applying $\iota$ to each side, we get the isomorphism 
\begin{align}\label{iso 40059454} H^n_{\dist}\left(\lim_i^{\Gamma}M_i\right) &\cong H^n_{\dist}\left(\lim_i\iota M_i\right)\end{align}
from Theorem \ref{main thm 1}.

Finally, long exact sequence \eqref{les 239094} arises from applying $H^n_{\dist}$ to the Milnor-type short exact sequence of graded $\Gamma^*$-modules
\[ 0 \rightarrow \lim_i \iota M_i \rightarrow \prod_i \iota M_i \rightarrow \prod_i \iota M_i \rightarrow 0 \]
and using isomorphism \eqref{iso 40059454}.
\end{proof}


\begin{corollary}
Suppose that $A$ is a countable field and that $\Gamma$ is a Mitchell coalgebra over $A$. Let \begin{equation*}\label{seq 430949598a}\dots\rightarrow M_2\rightarrow M_1\rightarrow M_0\end{equation*} be a sequence of graded $\Gamma$-comodules such that $R^1\lim$ vanishes on the sequence of graded $\Gamma^*$-modules 
$\dots\rightarrow \iota M_2\rightarrow \iota M_1\rightarrow \iota M_0$.
Then we have an isomorphism of graded $\Gamma^*$-modules
\[ \iota R^*\lim^{\Gamma}_i M_i  \cong \colim_{n\rightarrow\infty}\Ext^*_{\Gamma^*}\left(\Gamma^*/I^n,\lim_i \iota M_i\right)\]
where $I$ is the ideal of $\Gamma^*$ generated by all homogeneous elements of positive degree.
\end{corollary}
\begin{proof}
Consequence of Theorems \ref{derived sequential lim thm} and \ref{main thm on mitchell coalgebras}.
\end{proof}

\begin{corollary}\label{steenrod alg seq cor}
Suppose that $p$ is a prime number. Write $\Gamma$ for the dual $p$-primary Steenrod algebra. Let \begin{equation*}\label{seq 430949598b}\dots\rightarrow M_2\rightarrow M_1\rightarrow M_0\end{equation*} be a sequence of graded $\Gamma$-comodules such that $R^1\lim$ vanishes on the sequence of graded $\Gamma^*$-modules 
$\dots\rightarrow \iota M_2\rightarrow \iota M_1\rightarrow \iota M_0$.
Then we have an isomorphism of graded $\Gamma^*$-modules
\[ \iota R^*\lim^{\Gamma}_i M_i  \cong \colim_{n\rightarrow\infty}\Ext^*_{\Gamma^*}\left(\Gamma^*/I^n,\lim_i \iota M_i\right)\]
where $I$ is the augmentation ideal of the $p$-primary Steenrod algebra $\Gamma^*$.
\end{corollary}

\appendix
\section{A graded generalization of Megibben's theorem (``Countable injective modules are $\Sigma$-injective'').}
\label{...generalization of Megibben...}

Given a ring $R$, a left $R$-module $M$ is called {\em $\Sigma$-injective} if every direct sum of copies of $M$ is injective. Every $\Sigma$-injective $R$-module is injective, and a theorem of Bass and Papp states that the converse is true if and only if $R$ is left Noetherian.
See Theorem 3.46 of \cite{MR1653294} for a textbook treatment of the Bass-Papp theorem.

Consequently, it is only over non-Noetherian rings that $\Sigma$-injectivity is a particularly interesting notion. A very useful theorem in that context is a result of C. Megibben from a 1982 paper, \cite{MR633266}, whose title---``Countable injective modules are $\Sigma$-injective''---states the result quite clearly.

However, despite our best efforts, we have been unable to locate in the literature any version of Megibben's theorem for graded rings and graded modules. It is not difficult to write down the proper generalization of Megibben's result and its proof, but it does require a bit of care, since a sufficiently na\"{i}ve approach will lead to a wrong result. For example, while it is true (and we prove, in Theorem \ref{graded megibbens thm}, below) that a direct sum of copies of a countable injective graded module over a graded ring is also injective, it is not true that a direct sum of copies of {\em suspensions} of a countable injective graded module over a graded ring is also injective. A nice example is the Steenrod algebra $\Gamma^*$ at any prime, as a module over itself: any {\em bounded-below} direct sum of copies of suspensions of $\Gamma^*$ is injective as a graded $\Gamma^*$-module, but an unbounded direct sum of copies of suspensions of $\Gamma^*$ can fail to be injective. See Theorems \ref{thm from margolis} and \ref{thm from margolis 2} of the second appendix for some highly relevant results, and chapter 13 of \cite{MR738973} for a very useful sustained discussion of these ideas. 

It is also not enough to expect the graded result to follow from the ungraded result, by simply forgetting the grading: for example, while the Steenrod algebra $\Gamma^*$ is injective in the category of graded $\Gamma^*$-modules, it is not injective in the category of ungraded $\Gamma^*$-modules, as one can derive from Theorems \ref{thm from margolis} and \ref{thm from margolis 2}. What we need is an analogue of Megibben's theorem which tells us about when an arbitrary coproduct of copies of a graded module $M$ is injective {\em as a graded module,} even when the underlying ungraded module of $M$ may fail to be injective.

We first recall the graded analogue of the Baer criterion, e.g. from Lemma I.2.4 of \cite{nastasescu2011graded}:
\begin{lemma}\label{graded baer criterion}
Let $R$ be a $\mathbb{N}$-graded ring, and let $M$ be a $R$-module. Then $M$ is injective in the category of graded $R$-modules if and only if, for every graded left ideal $I$ of $R$ and every diagram 
\[\xymatrix{
 \Sigma^n I \ar[r]\ar[rd] & \Sigma^n R \ar@{-->}[d] \\ & M
}\]
in the category of graded $R$-modules in which the top horizontal map is the canonical inclusion, a map of graded $R$-modules exists which fills in the dotted arrow and makes the diagram commute.
\end{lemma}

Here is our graded generalization of Megibben's theorem:
\begin{theorem}\label{graded megibbens thm}
Let $R$ be a $\mathbb{N}$-graded ring, and let $M$ be a countable graded $R$-module which is injective in the category of graded $R$-modules. Then every direct sum of copies of $M$ is also injective in the category of graded $R$-modules\footnote{To be clear: our claim is about injectivity of a direct sum of copies of $M$ itself, {\em not} copies of suspensions of $M$. So, for example, $\coprod_{n\in\mathbb{Z}} \Sigma^n M$ may fail to be injective in the category of graded modules.}.
\end{theorem}
\begin{proof}
This argument combines some arguments made in section 3 of Faith's paper \cite{MR193107} and arguments made in \cite{MR633266} in a straightforward way. We adopt some notation from Faith's paper: given a graded $R$-module $M$ and a set $X$ of homogeneous elements of $R$, we write $X^{\perp}$ for the set of homogeneous elements $m\in M$ such that $xm=0$ for all $x\in X$. Given a set $Y$ of homogeneous elements of $M$, we write $Y^{\perp}$ for the set of homogeneous elements $r\in R$ such that $ry=0$ for all $y\in Y$. 

Now suppose $M$ is countable and injective in the category of graded $R$-modules.
\begin{enumerate}
\item We first claim that, for every sequence $Y_1, Y_2, Y_3, \dots$ of sets of homogeneous elements of $M$ such that the ideals $Y_1^{\perp}, Y_2^{\perp}, \dots$ form an ascending chain 
\begin{equation}\label{chain 1} Y_1^{\perp}\subseteq Y_2^{\perp} \subseteq Y_3^{\perp}\subseteq\dots,\end{equation} that ascending chain stabilizes at some finite stage. 
We prove this claim by contrapositive, following (a graded analogue of) the argument from the main theorem from \cite{MR633266}. Suppose that we have a chain as in \eqref{chain 1} which does {\em not} stabilize. Then there exists some integer $d$ such that the degree $d$ summands $ \left( Y_1^{\perp}\right)^d\subseteq \left( Y_2^{\perp}\right)^d \subseteq \left( Y_3^{\perp}\right)^d\subseteq\dots$ fail to stabilize. Fix such an integer $d$.

We have the descending chain 
\begin{equation}\label{chain 2} Y_1^{\perp\perp} \supseteq Y_2^{\perp\perp} \supseteq Y_3^{\perp\perp}\supseteq \dots\end{equation} of sets of homogeneous elements of $M$. Since $M$ is countable, choose an enumeration $y_1, y_2, y_3, \dots$ of the set of the homogeneous elements of $M$ of degree $d$. Without loss of generality, we can choose this enumeration so that the homogeneous elements $m\in M$ in $\cap_nY_n^{\perp\perp}$ form a terminal subsequence, i.e., if $m^{\prime}\in M$ is a homogeneous element not in $\cap_nY_n^{\perp\perp}$, then $m^{\prime}$ occurs in the enumeration before every homogeneous element $m\in M$ such that $m\in \cap_nY_n^{\perp\perp}$. 
We are about to carry out an inductive proof; here is the initial step. Choose a positive integer $i_1$ such that there exists an element $z_1\in Y_{1}^{\perp\perp}$ 
such that $z_1 - y_1$ is not in $\cap_n Y_n^{\perp\perp}$. (If this is not possible, then every element $z\in Y_1^{\perp\perp}$, we have $z - y_{1}\in \cap_n Y_n^{\perp\perp}$. In particular, the case $z=0$ implies $y_{1}\in \cap_n Y_n^{\perp\perp}$. The assumption on how we have ordered the elements $y_1, y_2, \dots$ now implies that all the elements $y_1, y_2, \dots$ are in $\cap_n Y_n^{\perp\perp}$, and consequently the sequence \eqref{chain 2} stabilizes immediately in degree $d$, and consequently the sequence \eqref{chain 1} stabilizes immediately in degree $d$, as desired.) Since $z_1 - y_{1} \notin \cap_n Y_n^{\perp\perp} = \left( \cup_n Y_n^{\perp}\right)^{\perp}$, there exists some homogeneous $b_1\in \cup_n Y_n^{\perp}$ such that $b_1\cdot (z_1 - y_{1})\neq 0$. Let $f_1: \Sigma^{\left| d\right|} R(b_1) \rightarrow M$ be the graded $R$-module homomorphism given by the composite of the inclusion $\Sigma^d R(b_1) \hookrightarrow \Sigma^d R$ with the graded $R$-module map $\Sigma^d R\rightarrow M$ sending $1$ to $z_1$. Then $f_1(r) = rz_1$ for all $r\in R(b_1)$, and $f_1$ has the property that $f_1(b_1) =  b_1z_1 \neq b_1 y_1$. Here our notation $R(b_1)$ denotes the (homogeneous) left ideal of $R$ generated by $b_1$.

That was the initial step in the induction. We set up the inductive step as follows: the inductive hypothesis is that we have already chosen %some monotone strictly increasing sequence $i_1, \dots, i_n$ of positive integers, 
some sequence $b_1, \dots ,b_n$ of homogeneous elements of $\cup_n Y_n^{\perp}$, and some graded $R$-module homomorphism $f_n: \Sigma^d R(b_1, \dots ,b_n) \rightarrow M$ such that $f_n(b_j) \neq b_jy_{j}$ for all $j=1, \dots ,n$. 
Since $M$ is graded-injective, there exists a graded $R$-module map $g_{n}: \Sigma^d R \rightarrow M$ filling in the dotted arrow in the diagram
\[\xymatrix{
 \Sigma^d R(b_1, \dots ,b_n)\ar@{^{(}->}[d] \ar[r]^{f_n} & M \\
 \Sigma^d R \ar@{-->}[ur]_{g_{n}} &
}\]
and making the resulting diagram commute. Let $z_n = g_n(1)$. 

Choose some integer $m$ such that $b_1, \dots ,b_n\in Y_m^{\perp}\subseteq \cup_n Y_n^{\perp}$.
Choose %the least integer $i_{n+1}>i_n$ such that there exists 
a homogeneous element $z_{n+1}\in Y_m^{\perp\perp}$ of degree $d$ such that $z_{n+1} + z_n - y_{n+1}\notin \cap_n Y_n^{\perp\perp}$. (If no such %$i_{n+1}$ and 
$z_{n+1}$ exists, then the particular case $z_{n+1} = z_n$ yields that $y_{n+1}$ is in $\cap_n Y_n^{\perp\perp}$, so by our assumption on the ordering of the sequence $y_1, y_2, \dots$, all but a finite set of terms in that sequence (namely, $y_1, \dots ,y_{n}$) are members of $\cap_nY_n^{\perp\perp}$. Consequently the sequence \eqref{chain 2} stabilizes in degree $d$ at a finite stage, and consequently the sequence \eqref{chain 1} stabilizes in degree $d$ at a finite stage, as desired.) 
Then there exists some homogeneous $b_{n+1}\in R$ such that $b_{n+1}(z_{n+1} + z_n - y_{n+1}) \neq 0$. Let $f_{n+1}: \Sigma^d R(b_1, \dots ,b_{n+1})$ denote the graded $R$-module homomorphism given by the composite of the inclusion $\Sigma^d R(b_1,\dots ,b_{n+1}) \hookrightarrow \Sigma^d R$ with the graded $R$-module map $\Sigma^d R\rightarrow M$ sending $1$ to $z_n + z_{n+1}$. Then for all $j\in \{ 1, \dots ,n\}$, we have 
$f_{n+1}(b_j) = b_j(z_n + z_{n+1}) = f_n(b_j) + b_jz_{n+1} = f_n(b_j)$, since $z_{n+1}\in Y_m^{\perp\perp}$ and $b_j\in Y_m^{\perp}$.
Consequently $f_{n+1}$, with its domain restricted to $\Sigma^dR(b_1, \dots ,b_n)\subseteq \Sigma^d R(b_1, \dots ,b_{n+1})$, coincides with $f_n$.
We also have $f_{n+1}(b_{n+1}) = b_{n+1}(z_n + z_{n+1}) \neq b_{n+1}y_{n+1}$, completing the inductive step.

The conclusion of this induction is that we have a sequence of graded $R$-modules
\[
 \Sigma^d R(b_1) \hookrightarrow \Sigma^d R(b_1,b_2)\hookrightarrow\dots\]
and a compatible morphism of graded $R$-modules $f_n: \Sigma^d R(b_1, \dots ,b_n) \rightarrow M$ for each $n$, such that $f_n(b_n) \neq b_n y_{n}$ for each $n$. By the universal property of the colimit, we get a graded $R$-module homomorphism $f:\Sigma^dR(b_1, b_2, \dots) \rightarrow M$ such that $f(b_n) = f_n(b_n) \neq b_ny_n$ for each $n$.Since $M$ is graded-injective, there exists a graded $R$-module map $g: \Sigma^d R \rightarrow M$ filling in the dotted arrow in the diagram
\[\xymatrix{
 \Sigma^d R(b_1, b_2, \dots)\ar@{^{(}->}[d] \ar[r]^{f} & M \\
 \Sigma^d R\ar@{-->}[ur]_{g} &
}\]
and making the resulting diagram commute. Now $g(1)$ is a homogeneous element of $M$ of degree $d$ such that $b_n g(1) = f_n(b_n) \neq b_ny_n$ for each $n$, i.e., $g(1) \neq y_n$ for each $n$. Hence $g(1)$ cannot be any of the elements $y_1, y_2, \dots$, which were assumed to exhaust the homogeneous degree $d$ elements of $M$, a contradiction, as desired.
\item
We next claim that, for each homogeneous left ideal $I$ of $R$, there exists a finitely generated homogeneous left ideal $I_1$ of $R$ contained in $I$ such that $I^{\perp} = I_1^{\perp}$. The proof follows that of Proposition 3.1 in \cite{MR193107}: consider the collection $\mathcal{I}(I)$ of all finitely generated homogeneous left ideals of $R$ contained in $I$. Preorder this collection by letting $J_1\geq J_2$ if and only if $J_1^{\perp}\subseteq J_2^{\perp}$. 
Let $\pi_0\mathcal{I}(I)$ be the partially-ordered set of equivalence classes in the preordered set $\mathcal{I}(I)$.
Since $M$ is assumed injective, the previous part of this proof shows that every ascending sequence in $\pi_0\mathcal{I}(I)$ stabilizes. Consequently, by Zorn's Lemma, $\pi_0\mathcal{I}(I)$ has a maximal element. Let $I_1$ be an element of $\mathcal{I}(I)$ representing a maximal element of $\pi_0\mathcal{I}(I)$. If $x$ is a homogeneous element of $I$, then $I_1 + Rx$ is a member of $\mathcal{I}(I)$ containing $I_1$, and by maximality of $I_1$, we must have $(I_1 + Rx)^{\perp} = I_1^{\perp}$. Since $x\in I_1 + Rx$, we must have $xm=0$ for all $m\in \left( I_1 + Rx\right)^{\perp} = I_1^{\perp}$. This argument applies for all homogeneous $x\in I$, so we have $I^{\perp}\subseteq I_1^{\perp}$. The reverse containment $I^{\perp}\supseteq I_1^{\perp}$ follows from $I_1$ being a subideal of $I$, so we have $I^{\perp} = I_1^{\perp}$, as desired.
\item Finally, we aim to show that every coproduct of copies of $M$ is injective. This argument follows that of Proposition 3.3 of \cite{MR193107}. Let $S$ be a set, let $\coprod_{s\in S}M$ be the coproduct in the category of graded $R$-modules, and let $I$ be a homogeneous left ideal in $R$. If, for every diagram of graded $R$-modules
\begin{equation}\label{diag 340949}\xymatrix{
 \Sigma^n I \ar@{^{(}->}[d] \ar[r]^f & \coprod_{s\in S} M \\
 \Sigma^n R \ar@{-->}_g,
}\end{equation}
we can produce a map $g$ filling in the dotted arrow and making the diagram commute, then Lemma \ref{graded baer criterion} implies that $\coprod_{s\in S} M$ is injective in $\gr\Mod(R)$.
The construction of $g$ is simple: since it is a general result from category theory that a product of injectives is injective, we have a morphism $\tilde{g}: \Sigma^n R \rightarrow \prod_{s\in S} M$ such that the diagram of graded $R$-modules
\[\xymatrix{
 \Sigma^n I \ar@{^{(}->}[d] \ar[r]^f & \coprod_{s\in S} M \ar[d]^{i}\\
 \Sigma^n R \ar[r]_{\tilde{g}} & \prod_{s\in S} M,
}\]
commutes, where $i$ is the canonical inclusion of the direct sum into the Cartesian product. 
Using the previous part of this proof, let $I_1$ be a finitely generated homogeneous left subideal of $I$ such that $I_1^{\perp} = I^{\perp}$. Let $i_1, \dots ,i_m$ be a set of homogeneous generators for $I_1$. Since that set is finite, there exists a {\em finite} subset $S^{\prime}$ of $S$ such that, for each $j\in \{ 1, \dots ,m\}$ and for each $s\in S$, $f(i_j)_s$ is zero unless $s\in S^{\prime}$. Here (and elsewhere), we write $x_s$ for the component of an element $x\in \coprod_{s\in S}M$ in the summand corresponding to $s\in S$.
Let $\hat{g}: \Sigma^n R \rightarrow \coprod_{s\in S} M$ be the morphism of graded $R$-modules given by, for each $s\in S$, letting $\hat{g}(1)_s$ be zero if $s\notin S^{\prime}$, and letting $\hat{g}(1)_s = \tilde{g}(1)_s$ if $s\in S^{\prime}$. Then $i\hat{g}(1) = i\tilde{g}(1)$ for each $i$ in a set of generators for $I_1$, and consequently $i\hat{g}(1) = i\tilde{g}(1)$ for all $i\in I_1$. Hence $\hat{g}(1)_s - \tilde{g}(1)_s \in I_1^{\perp} = I^{\perp}$ for all $s\in S$. Consequently $i\hat{g}(1)_s = i\tilde{g}(1)_s$ for all homogeneous $i\in I$ and all $s\in S$, and consequently $f(i) = i\hat{g}(1) = i\tilde{g}(1)$ for all homogeneous $i\in I$, i.e., $\tilde{g}$ is the desired map $g$ filling in the dotted arrow in diagram \eqref{diag 340949}.
\end{enumerate}
\end{proof}

\section{Review of Margolis' basic results on $\mathcal{P}$-algebras.}
\label{Review of Margolis...}


Margolis' fundamental results on graded module theory over $\mathcal{P}$-algebras are Theorem 5 from section 13.2 and Theorem 12 from section 13.3 of Margolis's book \cite{MR738973}. We state those two results:
\begin{theorem}{\bf (Margolis.)} \label{thm from margolis} Let $B$ be a $\mathcal{P}$-algebra, and let $B(0) \subsetneq B(1) \subsetneq \dots$ be a sequence of Poincar\'{e} subalgebras of $B$, as in Definition \ref{def of p-alg}. Let $M$ be a graded left $B$-module. Then the following are equivalent:
\begin{enumerate}
\item The projective dimension of $M$ is $\leq 1$.
\item $M$ is flat.
\item The injective dimension of $M$ is $\leq 1$.
\item $M$ is free over $B(n)$ for each $n$.
\end{enumerate}
Furthermore, if $M$ does not satisfy these (equivalent) conditions, then the projective dimension, weak dimension, and injective dimension of $M$ are each infinite.
\end{theorem}
\begin{theorem}{\bf (Margolis.)} \label{thm from margolis 2} Let $B$ be a $\mathcal{P}$-algebra, and let $B(0) \subsetneq B(1) \subsetneq \dots$ be a sequence of Poincar\'{e} subalgebras of $B$, as in Definition \ref{def of p-alg}. Let $M$ be a {\em bounded-below} graded left $B$-module. Then the following are equivalent:
\begin{enumerate}
\item $M$ is free.
\item $M$ is projective.
\item $M$ is flat.
\item $M$ is injective.
\item $M$ is free over $B(n)$ for each $n$.
\end{enumerate}
Furthermore, if $M$ does not satisfy these (equivalent) conditions, then the projective dimension, weak dimension, and injective dimension of $M$ are each infinite.
\end{theorem}



\bibliography{/home/asalch/texmf/tex/salch}{}
\bibliographystyle{plain}
\end{document}
