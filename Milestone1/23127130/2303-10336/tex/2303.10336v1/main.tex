\pdfoutput=1
%%
%% This is file `sample-authordraft.tex',
%% generated with the docstrip utility.
%%
%% The original source files were:
%%
%% samples.dtx  (with options: `authordraft')
%% 
%% IMPORTANT NOTICE:
%% 
%% For the copyright see the source file.
%% 
%% Any modified versions of this file must be renamed
%% with new filenames distinct from sample-authordraft.tex.
%% 
%% For distribution of the original source see the terms
%% for copying and modification in the file samples.dtx.
%% 
%% This generated file may be distributed as long as the
%% original source files, as listed above, are part of the
%% same distribution. (The sources need not necessarily be
%% in the same archive or directory.)
%%
%% The first command in your LaTeX source must be the \documentclass command.

% \documentclass[sigconf, manuscript, anonymous]{acmart}
\documentclass[authorversion, nonacm]{acmart}

%% NOTE that a single column version may be required for 
%% submission and peer review. This can be done by changing
%% the \doucmentclass[...]{acmart} in this template to 
%% \documentclass[manuscript,screen,review]{acmart}
%% 
%% To ensure 100% compatibility, please check the white list of
%% approved LaTeX packages to be used with the Master Article Template at
%% https://www.acm.org/publications/taps/whitelist-of-latex-packages 
%% before creating your document. The white list page provides 
%% information on how to submit additional LaTeX packages for 
%% review and adoption.
%% Fonts used in the template cannot be substituted; margin 
%% adjustments are not allowed.
%%
%% \BibTeX command to typeset BibTeX logo in the docs
\AtBeginDocument{%
  \providecommand\BibTeX{{%
    \normalfont B\kern-0.5em{\scshape i\kern-0.25em b}\kern-0.8em\TeX}}}

%% Rights management information.  This information is sent to you
%% when you complete the rights form.  These commands have SAMPLE
%% values in them; it is your responsibility as an author to replace
%% the commands and values with those provided to you when you
%% complete the rights form.
% \setcopyright{acmcopyright}
% \copyrightyear{2018}
% \acmYear{2018}
% \acmDOI{10.1145/1122445.1122456}

%% These commands are for a PROCEEDINGS abstract or paper.
% \acmConference[Woodstock '18]{Woodstock '18: ACM Symposium on Neural
  % Gaze Detection}{June 03--05, 2018}{Woodstock, NY}
% \acmBooktitle{Woodstock '18: ACM Symposium on Neural Gaze Detection,
  % June 03--05, 2018, Woodstock, NY}
% \acmPrice{15.00}
% \acmISBN{978-1-4503-XXXX-X/18/06}


\usepackage{graphics} 
\usepackage{caption}
\usepackage{dirtytalk}
\usepackage{subfigure}
\usepackage{amsmath}
\usepackage{wrapfig}

\newcommand{\TM}{\textsuperscript{TM}}


%%
%% Submission ID.
%% Use this when submitting an article to a sponsored event. You'll
%% receive a unique submission ID from the organizers
%% of the event, and this ID should be used as the parameter to this command.
%%\acmSubmissionID{123-A56-BU3}

%%
%% The majority of ACM publications use numbered citations and
%% references.  The command \citestyle{authoryear} switches to the
%% "author year" style.
%%
%% If you are preparing content for an event
%% sponsored by ACM SIGGRAPH, you must use the "author year" style of
%% citations and references.
%% Uncommenting
%% the next command will enable that style.
%%\citestyle{acmauthoryear}

%%
%% end of the preamble, start of the body of the document source.
\begin{document}

%%
%% The "title" command has an optional parameter,
%% allowing the author to define a "short title" to be used in page headers.
% \title{Recognizing Complex Gestures on Minimalistic Knitted Sensors}

\title [Recognizing Complex Gestures on Minimalistic Knitted Sensors] {Recognizing Complex Gestures on Minimalistic Knitted Sensors: Toward Real--World Interactive Systems}

%%
%% The "author" command and its associated commands are used to define
%% the authors and their affiliations.
%% Of note is the shared affiliation of the first two authors, and the
%% "authornote" and "authornotemark" commands
%% used to denote shared contribution to the research.
\author{Denisa Qori McDonald}
\email{denisaqori@gmail.com}
\affiliation{%
    \institution{College of Computing and Informatics, Drexel University}
    \streetaddress{3675 Market St.}
    \city{Philadelphia}
    \state{Pennsylvania}
    \country{USA}
}

\author{Richard Vallett}
\email{rjvallett@drexel.edu}
\affiliation{%
    \institution{Center for Functional Fabrics, Drexel University}
    \streetaddress{3101 Market St.}
    \city{Philadelphia}
    \state{Pennsylvania}
    \country{USA}
}

\author{Lev Saunders}
\email{levsau00@gmail.com }
\affiliation{%
  \institution{College of Computing and Informatics, Drexel University}
    \streetaddress{3675 Market St.}
    \city{Philadelphia}
    \state{Pennsylvania}
    \country{USA}
}

\author{Genevi\`{e}ve Dion}
\affiliation{%
    \institution{Center for Functional Fabrics, Drexel University}
    \streetaddress{3101 Market St.}
    \city{Philadelphia}
    \state{Pennsylvania}
    \country{USA}
}

\author{Ali Shokoufandeh}
\affiliation{%
    \institution{College of Computing and Informatics, Drexel University}
    \streetaddress{3675 Market St.}
    \city{Philadelphia}
    \state{Pennsylvania}
    \country{USA}
}



%%
%% By default, the full list of authors will be used in the page
%% headers. Often, this list is too long, and will overlap
%% other information printed in the page headers. This command allows
%% the author to define a more concise list
%% of authors' names for this purpose.
\renewcommand{\shortauthors}{McDonald, et al.}

%%
%% The abstract is a short summary of the work to be presented in the
%% article.
\begin{abstract}
Developments in touch-sensitive textiles have enabled many novel interactive techniques and applications. Our digitally-knitted capacitive active sensors can be manufactured at scale with little human intervention. Their sensitive areas are created from a single conductive yarn, and they require only few connections to external hardware. This technique increases their robustness and usability, while shifting the complexity of enabling interactivity from the hardware to computational models. This work advances the capabilities of such sensors by creating the foundation for an interactive gesture recognition system. It uses a novel sensor design, and a neural network-based recognition model to classify 12 relatively complex, single touch-point gesture classes with 89.8\% accuracy, unfolding many possibilities for future applications. We also demonstrate the system's applicability and robustness to real-world conditions through its performance while being worn and the impact of washing and drying on the sensor's resistance. 

\end{abstract}

\begin{CCSXML}
<ccs2012>
   <concept>
       <concept_id>10010147.10010257.10010293.10010294</concept_id>
       <concept_desc>Computing methodologies~Neural networks</concept_desc>
       <concept_significance>500</concept_significance>
       </concept>
   <concept>
       <concept_id>10010147.10010257.10010339</concept_id>
       <concept_desc>Computing methodologies~Cross-validation</concept_desc>
       <concept_significance>300</concept_significance>
       </concept>
   <concept>
       <concept_id>10010147.10010257.10010258.10010259</concept_id>
       <concept_desc>Computing methodologies~Supervised learning</concept_desc>
       <concept_significance>300</concept_significance>
       </concept>
   <concept>
       <concept_id>10010147.10010257.10010258.10010259.10010263</concept_id>
       <concept_desc>Computing methodologies~Supervised learning by classification</concept_desc>
       <concept_significance>500</concept_significance>
       </concept>
   <concept>
       <concept_id>10010147.10010257.10010258.10010260</concept_id>
       <concept_desc>Computing methodologies~Unsupervised learning</concept_desc>
       <concept_significance>500</concept_significance>
       </concept>
   <concept>
       <concept_id>10003120.10003121.10003128.10011755</concept_id>
       <concept_desc>Human-centered computing~Gestural input</concept_desc>
       <concept_significance>500</concept_significance>
       </concept>
   <concept>
       <concept_id>10003120.10003121.10003125</concept_id>
       <concept_desc>Human-centered computing~Interaction devices</concept_desc>
       <concept_significance>300</concept_significance>
       </concept>
   <concept>
       <concept_id>10003120.10003138.10003139.10010904</concept_id>
       <concept_desc>Human-centered computing~Ubiquitous computing</concept_desc>
       <concept_significance>300</concept_significance>
       </concept>
   <concept>
       <concept_id>10003120.10003121.10003122.10003334</concept_id>
       <concept_desc>Human-centered computing~User studies</concept_desc>
       <concept_significance>300</concept_significance>
       </concept>
 </ccs2012>
\end{CCSXML}

\ccsdesc[500]{Human-centered computing~Gestural input}
\ccsdesc[300]{Human-centered computing~Interaction devices}
\ccsdesc[500]{Computing methodologies~Neural networks}
\ccsdesc[300]{Computing methodologies~Cross-validation}
\ccsdesc[300]{Computing methodologies~Supervised learning by classification}
\ccsdesc[300]{Human-centered computing~Ubiquitous computing}
\ccsdesc[300]{Human-centered computing~User studies}


%%
%% Keywords. The author(s) should pick words that accurately describe
%% the work being presented. Separate the keywords with commas.
\keywords{gesture detection, knitted sensors, wearable sensor, CNN networks, LSTM networks, neural networks, evaluation study, interactive system, real-time system, real-world conditions}


%% A "teaser" image appears between the author and affiliation
%% information and the body of the document, and typically spans the
%% page.

\begin{teaserfigure}
    \subfigure[]
    {\includegraphics[height=1.25in]{src/figures/Lev_Touchpad_Frame_Overlay.png}\label{fig:sensor_electrodes}}
    \hfill
    \subfigure[]
    {\includegraphics[height=1.25in]{src/figures/system_components-min.pdf}\label{fig:sensor_stretch}}
    \hfill
    \subfigure[]
    {\includegraphics[height=1.25in]{src/figures/wearing_sensor-min.pdf}\label{fig:wearable_sensor}} 
    \caption{Knitted sensors for gesture recognition, constructed with carbon-coated nylon yarn (darker rectangular regions) and polyester yarn. \emph{(a)}: A knitted sensor with four electrodes attached capturing gesture information. \emph{(b)}: Components to construct future real-time gesture recognition systems: an NVIDIA® Jetson Xavier™ NX Developer Kit hosting a trained model, a micro-controller for signal generation and processing, and the knitted sensor for gesture input. The knitted component is a similarly designed and constructed sensor as in \emph{(a)}, but with three smaller conductive areas. \emph{(c)}: The same knitted sensor shown in \emph{(b)} being worn.} 
    \Description{Three photos labelled (a), (b) and (c) show images of a knitted sensor under different circumstances. Image (a) shows a sensor attached to four electrodes and a user touching it with their finger. In (b) a similarly constructed knitted sensor to the one in (a) is shown alongside a micro-controller and an NVIDIA system-on-module computer. In (c) a user is wearing the same knitted sensor shown in (b) on their forearm, an touching it with one finger.}
\end{teaserfigure}

% 

%%
%% This command processes the author and affiliation and title
%% information and builds the first part of the formatted document.
\maketitle

\begin{table}[t]
    \footnotesize
    \NineColors{saturation=high}
    \caption[Variables list]{Selection of variables used in this paper, grouped in their respective categories.}
    \begin{adjustbox}{width=\linewidth}
    \begin{tblr}
    	{
    		%caption = {Selection of variables used in this paper, grouped in categories.},
    		colspec = {cll|ll},
    		vlines = {white},
    		hlines = {white},
    		vline{4} = {2}{-}{white},
    		cell{1}{2-5} = {red3, fg=yellow9, font=\bfseries},
    		cell{2-14}{1} = {red3, fg=yellow9, font=\bfseries},
    		%cell{1-10}{2-5} = {red3, fg=yellow9, font=\bfseries},
    		cell{3,5,7,9,11,13,15}{2-5} = {red9},
    		%cell{3-6}{1}={red3, fg=yellow9},
    		%cell{3-6}{1} = {font=\bfseries},
    	}
    	&Symbol&Description&Symbol&Description\\
    	\multirow{4}{*}{\rotatebox{90}{Acq. model}}&
    	$\sigma$ & Wavenumbers &$\mathbf{s}$&Focal plane coordinates\\
    	&$\bm{\omega}=(\theta^{[i]},\phi^{[i]})$& Incident angle&$\{\Omega_j\}_{\range{j}{1}{N_p}}$& Solid angle of incidence\\
   		&$\mathcal{L}(\sigma,\bm{\omega})$ & Input spectral radiance&$\{\Phi_{jk}\}_{\range{j}{1}{N_p},\range{k}{1}{N_i}}$&Received flux\\
   		&$\{S_k\}_{\range{k}{1}{N_i}}$&Entrance pupil surface&$\{d_k\}_{\range{k}{1}{N_i}}$&Interferometer thickness\\
    	\multirow{4}{*}{\rotatebox{90}{Parameters}}&
    	$\bm{\delta}=\{\delta_i\}_{\range{i}{1}{N_i}}$& \glsentryshortpl{opd} & $\varphi^{ }_0$&Phase shift\\
    	&$\mathcal{A}(\sigma)$&Gain &$\mathbf{a}=\{a_m\}_{\range{m}{0}{N_d}}$& Gain coefficients\\
    	&$\mathcal{R}(\sigma)$&Surface reflectivity &$\mathbf{r}=\{r_m\}_{\range{m}{0}{N_d}}$& Reflectivity coefficients\\
    	&$\bm{\beta}=\{\beta_m\}_{\range{m}{1}{N_m}}$& Vector of parameters &$\hat{\bm{\beta}}=\{\hat{\beta}_m\}_{\range{m}{1}{N_m}}$& Estimated parameters \\
    	\multirow{3}{*}{\rotatebox{90}{Acq. vectors}} &$\bm{\sigma}=\{\sigma_i\}_{\range{i}{1}{N_a}}$& Central wavenumbers& $T_{\bm{\beta}}(\sigma_i)=\{t_i\}_{\range{i}{1}{N_a}}$ & Transfer function\\
    	&$\mathbf{y}=\{y_i\}_{\range{i}{1}{N_a}}$ & Single pixel acquisition& $\mathbf{w}=\{w_i\}_{\range{i}{1}{N_a}}$ & Flat field pixel statistic \\
    	& $\mathbf{u}=\{u_i\}_{\range{i}{1}{N_a}}$ & Neighborhood mean&$\mathbf{v}=\{v_i\}_{\range{i}{1}{N_a}}$ & Scaled neighborhood mean\\
    	\multirow{3}{*}{\rotatebox{90}{Amount}} &$N_a$ & Acquisitions & $N_i$ & Interferometers\\
    	& $N_p$ & Pixels per interferometer & $W$ & Waves\\
    	& $N_d$ & Degree & $N_m$ & Parameters\\
%    	
 	\end{tblr}
 	\end{adjustbox}
    \label{tab:variables}
\end{table}

\section{Introduction}
\label{sec:introduction}
% \begin{itemize}
%     % Diffusion of FL
%     \item {\st{Diffusion of FL}}
%     % Security threats to FL
%     \item {\st{Security threats to FL with particular focus on model poisoning}}
%     % Limitations of existing countermeasures
%     \item {\st{Current countermeasures (e.g., KRUM) and their limitations}}
%     % Proposed method and its advantages
%     \item {\st{Intuitive description of the proposed method and its difference (i.e., advantages) w.r.t. state of the art}}
%     % Main contributions
%     \item {\st{Summary of the main contributions of this work}}
%     % Paper's structure and organization
%     \item {\st{Paper's structure and organization}}
% \end{itemize}

% Diffusion of FL
Recently, {\em federated learning} (FL) has emerged as the leading paradigm for training distributed, large-scale, and privacy-preserving machine learning (ML) systems~\cite{mcmahan2017googleai,mcmahan2017aistats}. 
The core idea of FL is to allow multiple edge clients to collaboratively train a shared, global model without disclosing their local private training data.
%Specifically, an FL system consists of a central server and many edge clients; 
A typical FL round involves the following steps: {\em(i)} the server randomly picks some clients and sends them the current, global model; {\em(ii)} each selected client locally trains its model with its own private data; then, it sends the resulting local model to the server;\footnote{Whenever we refer to global/local model, we mean global/local model {\em parameters}.} {\em(iii)} the server updates the global model by computing an \emph{aggregation function}, usually the average (FedAvg), on the local models received from clients.
% \begin{enumerate}
%     \item[{\em(i)}] the server sends the current, global model to the clients and appoints some of them for training;
%     \item[{\em(ii)}] each selected client locally trains its copy of the global model with its own private data; then, it sends the resulting local model back to the server;\footnote{Whenever we refer to global/local model, we mean global/local model {\em parameters}.}
%     \item[{\em(iii)}] the server updates the global model by computing an \emph{aggregation function} on the local models received from clients (by default, the average, also referred to as FedAvg~\cite{mcmahan2017aistats}).
% \end{enumerate}
This process goes on until the global model converges. %(e.g., after a certain number of rounds or other similar stopping criteria).
%\\
% The advantages of FL over the traditional, centralized learning paradigm are undoubtedly clear in terms of flexibility/scalability (clients can join/disconnect from the FL network dynamically), network communications (only model weights\footnote{We will use \textit{parameters} and \textit{weights} interchangeably.} are exchanged between clients and server), and privacy (each client's private training data is kept local at the client's end and not uploaded to the server).
\\
% Security threats to FL
%However, the growing adoption of FL also raises security concerns~\cite{costa2022covert}, particularly about its confidentiality, integrity, and availability.
Although its advantages over standard ML, FL also raises security concerns~\cite{costa2022covert}. %, particularly about its confidentiality, integrity, and availability~\cite{costa2022covert}.
% OLD, LONG VERSION
% Indeed, some work deals with privacy leakage that may expose the local data of some clients~\cite{melis2019sp}. 
% A large body of work, instead, investigates attacks that usually aim to detriment the predictive accuracy of the learned global model. For instance, \emph{data poisoning} attacks achieve this goal by letting an adversary pollute the training set of some corrupt FL clients with maliciously crafted examples~\cite{jagielski2018sp}.
% Similarly, in \emph{model poisoning} the attacker attempts to tweak the global model weights~\cite{bhagoji2019pmlr} by directly perturbing the local model's weights of some infected FL clients before these are sent to the central server for aggregation, usually via so-called Byzantine attacks. 
% It turns out that Byzantine model poisoning attacks severely impact standard FedAvg; therefore, more robust aggregation functions must be designed to make FL systems secure.
Here, we focus on \emph{untargeted model poisoning} attacks~\cite{bhagoji2019pmlr}, where an adversary attempts to tweak the global model weights %\footnote{We will use the terms \textit{parameters} and \textit{weights} interchangeably.} 
by directly perturbing the local model's parameters of some infected clients before these are sent to the central server for aggregation.
In doing so, the adversary aims to jeopardize the global model \textit{indiscriminately} at inference time.
Such model poisoning attacks severely impact standard FedAvg; therefore, more robust aggregation functions must be designed to secure FL systems.
\\
% In this paper, we focus on designing a novel robust aggregation scheme at the server's end to contrast the effect of Byzantine model poisoning attacks.
%
% Current countermeasures and their limitations
%Several countermeasures have been proposed in the literature to combat model poisoning attacks on FL systems.
% Some methods use simple statistics more robust than plain average to smooth the impact of malicious updates (e.g., Trimmed Mean and FedMedian~\cite{yin2018icml}). 
% Other defenses implement outlier detection techniques to discard malicious updates from the aggregation performed at the server's end. Those are either based on heuristics (e.g., Krum/Multi-Krum~\cite{blanchard2017nips} and Bulyan~\cite{mhamdi2018pmlr}) or data-driven approaches (e.g., K-means clustering~\cite{shen2016acm} or DnC via spectral analysis~\cite{shejwalkar2021ndss}). 
% Finally, some strategies rely on a centralized ``source of trust'' to spot potential malicious updates (e.g., FLTrust~\cite{cao2020fltrust}).
% Several countermeasures have been proposed in the literature to combat model poisoning attacks on FL systems, i.e., to discard possible malicious local updates from the aggregation performed at the server's end. 
% These techniques range from simple statistics more robust than plain average (e.g., Trimmed Mean and FedMedian~\cite{yin2018icml}) to outlier detection heuristics (e.g., Krum/Multi-Krum~\cite{blanchard2017nips} and Bulyan~\cite{mhamdi2018pmlr}) or data-driven approaches (e.g., spectral analysis via K-means clustering~\cite{shen2016acm} or spectral analysis), or methods based on ``source of trust'' (e.g., FLTrust~\cite{cao2020fltrust}).
% OLD, LONG VERSION
%Several countermeasures have been proposed in the literature to combat Byzantine model poisoning attacks on FL systems.
% Descriptive statistics
% For example, Trimmed Mean and FedMedian aggregate local model updates using more robust statistics than standard average~\cite{yin2018icml}.
%
% % Heuristics for outlier detection
% Many existing Byzantine-resilient strategies implement some outlier detection heuristics to discard the model updates sent by potentially malicious clients from the input of the aggregation function.
% One of the most popular heuristics is Krum~\cite{blanchard2017nips}.
% This strategy tries to mitigate the impact of Byzantine attacks by selecting as a global model the local model with the smallest sum of Euclidean distances to {\em all} the other local models.
% Although powerful, Krum requires the server to know (or, at least, estimate) the number of malicious FL clients upfront, which is generally impossible in a realistic attack scenario. %
% Moreover, Krum may become ineffective for complex, high-dimensional model parameter spaces due to the curse of dimensionality.
% Bulyan~\cite{mhamdi2018pmlr} tries to overcome this issue by combining Krum with a variant of Trimmed Mean.
% % Data-driven outlier detection
% Other strategies use data-driven outlier detection techniques -- e.g., via K-means clustering~\cite{shen2016acm} -- to spot potential malicious local model updates. 
% %For instance, Shen et al. propose to cluster local model updates with K-means and thus identify outliers.
%
% % Other techniques
% As far as the server is concerned, any local model received can be from a potential malicious client. 
% FLTrust~\cite{cao2020fltrust} assumes the server acts as a client, i.e., trains a local model on an additional {\em trustworthy} dataset at the server's end and compares it against all the local models from other clients. 
% This way, the server can rely on some ``source of trust'' when discarding potentially malicious clients.
%\\
% Limitations of existing Byzantine-resilient strategies
Unfortunately, existing defense mechanisms either rely on simple heuristics (e.g., Trimmed Mean and FedMedian by~\cite{yin2018icml}) or need strong and unrealistic assumptions to work effectively (e.g., foreknowledge or estimation of the number of malicious clients in the FL system, as for Krum/Multi-Krum~\cite{blanchard2017nips} and Bulyan~\cite{mhamdi2018pmlr}, which, however, cannot exceed a fixed threshold).
Furthermore, outlier detection methods using K-means clustering~\cite{shen2016acm} or spectral analysis like DnC~\cite{shejwalkar2021ndss} do not directly consider the temporal evolution of local model updates received.
Finally, strategies like FLTrust~\cite{cao2020fltrust} require the server to collect its own dataset and act as a proper client, thereby altering the standard FL protocol.
\\
% OLD, LONG VERSION
% Overall, existing Byzantine-resilient strategies are either simple heuristics (e.g., FedMedian) or, if they are more complex, they rely on strong and unrealistic assumptions to work effectively (e.g., knowing the number of malicious clients in the FL system in advance, as for Krum and alike).
% Furthermore, data-driven outlier detection methods do not consider the temporary evolution of local model updates received (e.g., K-means clustering). 
% Finally, strategies like FLTrust requires the server to collect its own dataset and act as a proper client, thereby altering the standard FL protocol.
%
% Description of the proposed method
This work introduces a novel pre-aggregation \textit{filter} robust to untargeted model poisoning attacks. Notably, this filter $(i)$ operates without requiring prior knowledge or constraints on the number of malicious clients and $(ii)$ inherently integrates temporal dependencies. 
The FL server can employ this filter as a preprocessing step before applying \textit{any} aggregation function, be it standard like FedAvg or robust like Krum or Bulyan.
Specifically, we formulate the problem of identifying corrupted updates as a multidimensional (i.e., matrix-valued) time series anomaly detection task. 
The key idea is that legitimate local updates, resulting from well-calibrated iterative procedures like stochastic gradient descent (SGD) with an appropriate learning rate, show \textit{higher predictability} compared to malicious updates. This hypothesis stems from the fact that the sequence of gradients (thus, model parameters) observed during legitimate training exhibit regular patterns, as validated in Section~\ref{subsec:intuition}. %until convergence. 
%This regularity may be more pronounced for smooth convex loss functions, but it can still be captured within an appropriate time window, even for more complex and convoluted loss surfaces. 
%We provide evidence of this claim in Appendix~B, where we show that the average mutual information (i.e., ``predictability''), calculated over pairs of legitimate model updates sent at different FL rounds, is significantly higher than the corresponding computation for a malicious client.
\\
Inspired by the matrix autoregressive (MAR) framework for multidimensional time series forecasting~\cite{chen2021je}, we propose the FLANDERS ({\em \textbf{F}ederated \textbf{L}earning meets \textbf{AN}omaly \textbf{DE}tection for a \textbf{R}obust and \textbf{S}ecure}) filter.
The main advantages of FLANDERS over existing strategies like FLDetector~\cite{zhao2020multivariate} are its resilience to large-scale attacks, where $50\%$ or more FL participants are hostile, and the capability of working under realistic non-iid scenarios.
We attribute such a capability to two key factors: $(i)$ FLANDERS works without knowing a priori the ratio of corrupted clients, and $(ii)$ it embodies temporal dependencies between intra- and inter-client updates, quickly recognizing local model drifts caused by evil players. Below, we summarize our main contributions:

\begin{itemize}
\item[{\em(i)}]
We provide empirical evidence that the sequence of models sent by legitimate clients is more predictable than those of malicious participants performing untargeted model poisoning attacks.
\\
\item[{\em(ii)}] 
We introduce FLANDERS, the first pre-aggregation filter for FL robust to untargeted model poisoning based on multidimensional time series anomaly detection.
\\
\item[{\em(iii)}] 
We integrate FLANDERS into Flower,\footnote{\scriptsize{\url{https://flower.dev/}}} a popular FL simulation framework for reproducibility.
\\
\item[{\em(iv)}] 
We show that FLANDERS improves the robustness of the existing aggregation methods under multiple settings: different datasets, client's data distribution (non-iid), models, and attack scenarios.
\\
\item[{\em(v)}] 
We publicly release all the implementation code of FLANDERS along with our experiments.\footnote{\scriptsize{\url{https://anonymous.4open.science/r/flanders_exp-7EEB}}}
\end{itemize}

% Paper's structure and organization
The remainder of the paper is structured as follows. %some related work and the current state-of-the-art solutions to security issues that FL entails. 
Section~\ref{sec:background} covers background and preliminaries. 
In Section~\ref{sec:related}, we discuss related work.
Section~\ref{sec:problem} and Section~\ref{sec:method} describe the problem formulation and the method proposed. % to tackle it. 
Section~\ref{sec:experiments} gathers experimental results. %, and Section~\ref{sec:limitations} discusses some limitations of this work.
Finally, we conclude in Section~\ref{sec:conclusion}.
 %discusses the limitations of this work and draws future research directions.
%reports conclusions and draws perspectives for future research directions.

%%%%%%% OLD %%%%%%%
%to overcome the resilience of Byzantine failures in distributed Stochastic Gradient Descent computations. 
% The strength of Krum is its time complexity, which is linear in the gradient dimension. 
% However, the robustness of the approach is guaranteed for gradient-based learning applications only when the majority of the clients are not compromised. 
% Besides, the aggregation mechanism of Krum, as well as that of similar methods, is robust from a coarse-grained perspective and does not provide solutions to errors and perturbations that may occur at inference time.
%A related approach to~\cite{blanchard2017nips} is the work of Su et al.~\cite{su2016dc}. Here, the authors propose an iterated approximate agreement to tackle a multi-layer scenario attacked by Byzantine agents. 
%However, the method works efficiently on the sole discrete context and it is inapplicable to continuous state environments.
%\gabri{Maybe, we should just talk about the main limitations of existing countermeasures without digging into their details (or, we can just mention Krum as this is the most popular one). I will move the description of all these methods to the Related Work section.}
\section{Related Work}
This section starts by providing some context regarding fabric touch sensors produced using different fabric construction or embellishment techniques, as well as different sensing methods, focusing on capacitive sensing upon which this system is built. Then, the discussion transitions to work that has focused on gesture recognition on fabric sensors, which have been produced through various techniques. Subsequently, a discussion of the neural network architectures upon which our model relies, provides an overview of their working and the reasons for our choice. 


\subsection{Smart Textile Construction Processes}

\begin{figure*}[h]
    \centering
    \subfigure[]
    {\includegraphics[width=0.18\textwidth]{src/figures/Embroidered_Sensor.pdf}\label{fig:Embroidered_Sensor}}
    \hfill
    \subfigure[]
    {\includegraphics[width=0.18\textwidth]{src/figures/Capacitive_Sensing_Matrix.pdf}\label{fig:Capacitive_Sensing_Matrix}}
    \hfill
    \subfigure[]
    {\includegraphics[width=0.18\textwidth]{src/figures/Reflectometry.pdf}\label{fig:Reflectometry}}
    \hfill
    \subfigure[]
    {\includegraphics[width=0.18\textwidth]{src/figures/Knitted_Sensor.pdf}\label{fig:Knitted_Sensor}}
    \hfill
    \caption{Illustrations of sensing techniques and electrode layouts used in textile touch sensors. \emph{(a)}: Embroidered capacitive buttons. \emph{(b)}: Woven or embroidered capacitive sensing matrix. \emph{(c)}: \emph{Reflectometry} measured using two parallel conductors. \emph{(d)}: Single weft knitted conductor measured using \emph{differential capacitive sensing}.}
    \label{fig:Designs}
    
    \Description{The four photos, (a), (b), (c), and (d) show different sensing strategies used for fabric sensors. On each figure, there is an illustrated hand touching on an area of the illustrated fabric sensor. In (a), there are nine buttons shown, from 1-9, with all of them colored black, except for number 2, which is red, illustrating the button being touched. Figure (b) shows a matrix of vertical and horizontal lines. There is a vertical blue line and red horizontal line intersecting on the point of touch. In (c), there are two parallel lines forming a shape together. One of the lines is red up to the point of touch on the fabric, and black after it; the other line is completely black. Figure (d) shows a serpentine line, where the top part of the line, which is before the point of touch, is blue, while the bottom one, after the point of touch is red.}
\end{figure*}

Textile assembly or embellishing methods, such as embroidery, weaving, and knitting have been used to create touch-sensitive textile devices using conductive yarn. \emph{Embroidery} is a textile embellishing method that inserts thread into a fabric substrate to create arbitrary patterns (Figure~\ref{fig:Embroidered_Sensor}), and has been used to create several interactive textile devices~\cite{Post2000a,Gilliland2010a,Hamdan2016a,Hamdan2018a,aigner2020embroidered,mlakar2020design}. Embroidered threads are not part of the original textile, therefore they can be overlaid to form a variety of sensing shapes on any textile, if conductive yarn in used. Large-area distributed touch sensing can also be achieved through integrating conductive yarns into the textile structure during formation, eliminating post processing. {\em Weaving} and {\em knitting} are textile manufacturing processes that can combine conductive and non-conductive yarns to form fabric circuits.

In {\em weaving}, a textile is formed by interlacing perpendicular yarns~(Figure~\ref{fig:Capacitive_Sensing_Matrix}). It is a widely-used method to produce distributed textile touch sensors~\cite{DeRossi2002a,Hasegawa2007a,Takamatsu2011a,Agcayazi2016a,Poupyrev2016a,wu2020zebrasense} due to its similarity to standard capacitive and resistive sensing matrices. Woven textiles use the perpendicular alignment of numerous horizontal and vertical yarns to form a sensing grid and detect touch at yarn intersection points, supporting a high-resolution sensing area. However, connecting a matrix to external sensing electronics requires an equal number of fabric-to-wire connections---a difficult and time-consuming post production process. These connections often exist at millimeter-scales and utilize techniques like soldering, which create brittle joints that reduce the textile's durability, and potentially real-world practical use. 

\emph{Knitting}, the technique used to construct our sensor, produces fabric by employing a continuous yarn or set of yarns to form a series of interlocking loops. Knitting has been recently explored in producing interactive textiles, such as force-sensing knitted structures~\cite{Pointner2020KnitedRESi}, knitted electronic musical controllers~\cite{Wicaksono2020KnittedKeyboard}, and knitted dynamic displays~\cite{devendorf2016don}. Other research has investigated creating 2D sensing structures using a homogeneous~\cite{Vallett2016a,Vallett2019a} or multi-material~\cite{ozbek2018novel} 1D filament. In work more closely related to ours, digital weft-knitting has been explored to construct textiles with only two connection points~\cite{Vallett2016a,Vallett2019a,mcdonald2020knitted}. We select this process since it uses continuous yarns in the production of textiles, which allows the construction of a fabric circuit through an electrically continuous trace. The conductive element in this work is carbon-coated nylon yarn, which becomes intertwined with regular, non-conductive yarns according to a programmatically-defined pattern of textile construction. The circuit design is compatible with the way yarns are routed during the machine knitting process, and does not require post-production assembly, besides attaching wires at the connection points. This process allows for different textile patterns to be produced according to application specifications, as well as similar form factors of different scales.  

\subsection{Fabric-based Capacitive Touch Sensing}
Capacitive touch sensing is a widely utilized method of interacting with electronic devices. Capacitive touch sensors measure the presence or proximity of a conductor, such as human skin, through its electromagnetic properties. These sensors can discern nuanced touch and gesture and often require no mechanical deflection of a sensing medium. This mode of interaction is advantageous for use in touch-sensitive textiles, which need a dynamic circuit structure. Alternative methods of distributed touch detection have been pursued in textiles, such as \emph{inter-yarn contact sensing}~\cite{Inaba1996a, karrer2011pinstripe} and \emph{resistive sensing}~\cite{sundholm2014smart, Parzer2018a, pointner2020knitted, honnet2020polysense, freire2017second}.

Textile-based capacitive touch sensing circuits can be categorized as either \emph{discrete} or \emph{continuous}. Discrete touch sensing relies on numerous electrically isolated conductors distributed across a sensitive surface~\cite{Post2000a,Poupyrev2016a,wu2020capacitivo}. The bulk magnitude of capacitance is measured on each conductor, and touch is localized using knowledge of each conductor’s location. Improving the accuracy and resolution of touch localization relies on increasing the density of conductors within a given area. This is often done at the cost of increasing the circuit’s structural complexity and increasing the number of connections from the circuit to sensing hardware. However, the sensing electronics required to scan the discrete electrodes are simpler than what is required in continuous sensing. Textile manufacturing techniques like weaving (Figure~\ref{fig:Capacitive_Sensing_Matrix}) are often used to create capacitive sensing matrices, while standalone buttons can be achieved using textile embellishment methods like embroidery (Figure~\ref{fig:Embroidered_Sensor}).

\begin{figure*}[ht]
    \centering
    \subfigure[]
    {\includegraphics[width=.49\textwidth]{src/figures/Differential_Capacitive_Sensing_Circuit_Diagram.pdf}\label{fig:DCS_Circuit_Diagram}}
    \hfill
    \subfigure[]
    {\includegraphics[width=.49\textwidth]{src/figures/Differential_Capacitive_Sensing_Waveform_Output.pdf}\label{fig:DCS_Waveform_Output}}
    \hfill
    \caption{Illustration of the differential capacitive touch sensing circuit and example waveform output. \emph{(a)}: Circuit diagram visualizing the current differential between the measurement points (A and B) and the location of touch. \emph{(b)}: Example waveform measurements showing the attenuation in voltage (gain A, B) based on the touch location and capacitance magnitude shown in \emph{(a)}. The gains are found as the amplitude ratio of the input and output waveforms using \emph{Bode analysis}.}
    \label{fig:Differential_Capacitive_Sensing}
    
    \Description{The two figures, (a) and (b) show a circuit with two electrode connections, and a graph with three sinusoidal lines. The circuit in figure (a) shows current path on each electrode of the circuit A, in red, and B, in blue. An illustrated hand demonstrates a touch event, upon which the capacitor is shown being discharged in purple. In (b), we see a graph whose X axis shows time (0-200 microseconds), while Y axis shows the voltage values. There are three lines, which represent the input waveform, the output waveform measured at electrode A, and the output waveform measured at electrode B. The gain is measured as the amplitude ratio between the input and measured outputs. Corresponding to the touch location in (a), the waveform from electrode A has a higher value at the peaks compared to electrode B.}
\end{figure*}


In contrast, continuous touch sensing infers the location of touch across a single electrically continuous conductor. Continuous touch sensing exploits the properties of the conductor to measure changes in capacitance along its length, which are localized using knowledge of the conductor’s overall path. The geometry of the conductor may facilitate localization, such as creating textile-based transmission lines for use with Time- and Frequency-domain Reflectometry (TDR and FDR)~\cite{Wimmer2011a,Hughes2014a,Hughes2017a}, as illustrated in Figure~\ref{fig:Reflectometry}, or other more recent methods~\cite{ku2020threadsense}. Other sensing methods like Electric Field Tomography (EFT) measure capacitance along a pattern-less conductive surface using measurements along the periphery~\cite{Zhang2017a}. These methods can localize touch using relatively simple circuits; however, the accuracy and resolution of touch location relies on complex measurement hardware to infer accurate location from recorded data. 

 Previous work has used \emph{differential capacitive sensing}, a continuous capacitive sensing strategy, in conjunction with a weft-knitted circuit composed of a single conductive yarn to localize capacitive touch~\cite{Vallett2016a,Vallett2019a,mcdonald2020knitted}. The circuit design of the sensor in this work is based on that same method. As shown in Figure~\ref{fig:DCS_Circuit_Diagram}, touch location is inferred along a linear conductive pathway via a current differential measured at each endpoint. The measured voltage is visualized in Figure~\ref{fig:DCS_Waveform_Output}, which illustrates the attenuation of voltage based on the location and magnitude of capacitive touch. The waveform gain is determined via \emph{Bode analysis}~\cite{Bode1940a}, which determines the amplitude ratio of the input and output waveforms using Fast Fourier Transforms. Initial applications focused on creating user interfaces capable of measuring simple tap location. Our work extends the structure and capabilities of the sensing circuit by using a planar conductive area to detect continuous gestures.  

\subsection{Machine Learning Applications for Fabric-based Touch Sensing}
While textile touch sensors can provide basic touch location, machine learning algorithms have been used to enable higher-level, accurate input recognition. \emph{Project Jacquard}~\cite{Poupyrev2016a} introduces a platform to produce interactive woven sensors. In its user evaluation study, gestures such as \emph{swipe left}, \emph{swipe right}, and \emph {hold} were recognized under three different condition: \emph{sitting}, \emph{standing}, and \emph{walking}. The work from Hughes et al.~\cite{Hughes2014a,Hughes2017a} describes the construction of an RF-based e-Textile and its ability to distinguish between gestures of \emph{tap}, \emph{up swipe}, and \emph{down swipe} based on a CNN model. \emph{SmartSleeve}~\cite{Parzer2017a}, a deformable, pressure-sensing textile sensor recognizes in real time both surface and deformation gestures. It relies on an algorithms which, after converting the data in image form, uses an SVM-trained model together with heuristics to detect the gesture type. \emph{GestureSleeve}~\cite{Schneegass2016a} is a touch-sensing textile sleeve, able to detect stroke-based gestures or taps through the \$P algorithm~\cite{vatavu2012gestures}. \emph{Smart-mat}~\cite{sundholm2014smart} uses a kNN classifier on features built using time series statistical information. \emph{RESi}~\cite{Parzer2018a}, a resistive, touch sensing interface uses SVM classification to detect basic gestures such as single, double and triple taps, as well as swipes in the four directions. More recent work~\cite{olwal2020textile} furthers cord-based interfaces by enabling continuous control as well as recognizing casual discrete gestures.

Our work focuses on detecting gestures that are relatively complex compared to most of these examples. Additionally, all the sensors described above, mainly focused on gesture recognition, differ from our work in construction and sensing strategy. In many of these cases and other similar sensors, much of the complexity lies on the hardware design. In this work however, the construction is kept simple and streamlined, while shifting the burden of complexity to computational models, similarly to McDonald et al.~\cite{mcdonald2020knitted}. In that work, an LSTM neural network with features constructed on statistical information from voltage gain values was used to extract location of touch from a single-yarn and two-connection capacitive knitted sensor. Extending the functionality of that work, we introduce a capacitive knitted sensor, which is also composed of a single yarn, but uses four external connections to recognize gestures. Below, we describe the machine learning concepts upon which our model is built. 

\subsection{CNNs and LSTMs}
To recognize gestures performed on the knitted sensor, we use a neural network architecture which combines Convolutional Neural Networks (CNN)~\cite{fukushima1980neocognitron,lecun1998gradient} with Long Short-Term Memory networks (LSTM)~\cite{gers1999learning}, a stable version of Recurrent Neural Networks (RNNs)~\cite{rumelhart1988learning}. The time-series data, acquired through the four connections in the sensor we design, needs to be processed to map to particular gestures. Since this is time series data, the classification algorithm should take advantage of the continuity of sampled frequencies over time, in order to properly model the necessary connections. Moreover, due to variability in the data, such as different users, conditions, finger placement and more, the same gestures should be represented in terms of essential patterns within them. To those ends, we utilize CNNs and LSTMs, which have found application for many classification tasks. 

CNNs are relatively sparse, regularized neural networks, which have translation-invariant characteristics. They aim to capture hierarchical patterns in data, composing higher-level features from lower-level ones. CNNs have been successfully used, among other areas, in image and video recognition tasks, natural language processing, and recommender systems. They capture the spatial relationships between samples. RNNs allow previous outputs to be utilized as inputs, in addition to hidden states. This flexibility makes them ideal for classification of inputs of any length, while maintaining limits on model size, taking into account historical information, and sharing weights across time. A known drawback of RNN models is their difficulty in capturing long term dependencies due to multiplicative gradients. The error of such models can be exponentially decreasing/increasing with respect to the number of layers. The Gated Recurrent Unit (GRU)~\cite{cho2014learning} and their generalizations, such as Long Short-Term Memory units (LSTM)~\cite{gers1999learning} address the vanishing gradient problem encountered by traditional RNNs. We use LSTMs to model the temporal dependencies in our data. More information is provided regarding our network architecture in the section below, as part of the gesture recognition system.  



\section{System Design}
\label{sec:system_design}
In this section, we describe our gesture recognition system, composed of a few main components: the knitted fabric containing conductive and non-conductive yarn, the measurement hardware and circuit, the algorithmic components which produce a trained machine learning model, and ultimately the NVIDIA\textregistered~Jetson Xavier\TM~NX, a powerful embedded system-on-module (SoM) on which the model is deployed. Currently, signal generation, acquisition, and processing occur offline through the use of an arbitrary waveform generator and digital oscilloscope. After the data from multiple users is collected, we train a classification model to distinguish between 12 different gestures. Subsequently, we deploy the model on the embedded CPU and test its ability to recognize a gesture event in real-time. In the future, we plan to perform signal generation, acquisition, and pre-processing on a standalone embedded micro-processor capable of communicating with the embedded CPU, which would allow this system to be fully portable, while providing real-time interaction. 

The selected gestures are the alphanumeric characters: \emph{'3'}, \emph{'5'}, \emph{'I'}, \emph{'J'}, \emph{'L'}, \emph{'M'}, \emph{'O'}, \emph{'S'}, \emph{'V'}, \emph{'W'}, \emph{'Z'}, with the addition of character \emph{'?'}. All of these gestures can be performed in one motion and have a distinct onset and offset. Some are more basic, such as \emph{'I'}, others more complex, such as \emph{'?'}, and some gestures are more similar to each other, such as \emph{'5'} and \emph{'S'}, which could cause incorrect classification. This character subset contains the basic motions upon which other alphanumeric characters can be built. Our choice of several alphanumeric characters serves to strike a balance between prototyping this system, and the complexity it has the potential to offer. Instead of testing its recognition ability with more basic gestures, such as taps and directional swipes, we opted for a more comprehensive set, including gestures with curvatures, to explore the limits of its capabilities. On the other hand, including the whole alphanumeric character set of the English language would increase its complexity even more, requiring more training data. It is worth noting that the focus of this work is not the specific application, but the gesture-enabling technology illustrated through some examples. Ultimately, we would like to build a recognition model capable of detecting numbers and the letters of the alphabet, which could serve as a communication system. Our system is a first step in that direction, but more generally, it unlocks the potential of gesture recognition on knitted sensors with many possible applications. Below we describe its components in more detail.

\begin{figure*}[ht]
    \centering
    \hfill
    \subfigure[]
    {\includegraphics[width=0.4\textwidth]{src/figures/Single-wire_Touchpad_Diagram.pdf}\label{fig:Single-wire_Touchpad_Diagram}}
    \hfill
    \subfigure[]
    {\includegraphics[width=0.4\textwidth]{src/figures/Planar_Touchpad_Diagram.pdf}\label{fig:Planar_Touchpad_Diagram}}
    \hfill
    \caption{Diagrams of the knitted touchpad designs. \emph{(a)}: A single-wire touchpad created using conductive and non-conductive yarns that serpentine across the textile surface. The points \emph{A} and \emph{B} connect to external sensing hardware. \emph{(b)}: A touchpad created as a planar conductive area with four connections points, \emph{A, B, C} and \emph{D}.}
    \label{fig:Touchpad_Designs}
    \Description{The two figures, (a) and (b) illustrate how weft-knitted sensors are constructed. In both cases, there is an illustrated hand touching somewhere on the sensor. In (a), the sensing structure is a linear serpentine, and there are two electrode points A and B. An additional view of the construction is provided with intertwined gray and black lines which represent conductive and non-conductive yarns combined to form the fabric. Figure (b) shows a similar sensor; however, the sensing area is a black rectangle in the middle, and there are four connection points A, B, C, and D. The additional view of the construction shows black lines intertwined to form the fabric, representing conductive yarn.}
\end{figure*}

\subsection{Knitted Sensor}
\label{sec:knitted_sensor}

The knitted sensor design we introduce is a continuous conductive rectangular sensing area, constructed using one conductive yarn and four electrode connections points, illustrated in Figure~\ref{fig:Planar_Touchpad_Diagram}. The yarn used is outwardly conductive, since it is composed of carbon-suffused nylon, and forms a resistive mesh when knitted. The inter-row yarn loop connections form an approximate uniform resistance gradient across the planar surface. This property is beneficial to measuring touch gesture as the resistance gradient is uniform in all directions. Touch location and capacitance magnitude are inferred through voltage measurements recorded at the corner locations. Capacitive touch draws current from the current sources attached at the corners, which affect the differential charge of the circuit (Figure~\ref{fig:Planar_CTS_Touch_Sensing_Circuit}).

Patterns developed in previous work~\cite{Vallett2016a,Vallett2019a,mcdonald2020knitted}, as seen in Figure~\ref{fig:Single-wire_Touchpad_Diagram}, utilized a single linear conductive pathway that covered a planar area while following the serpentine pathway of weft knitting. In those sensors, location was inferred as a linear distance along the pathway mapped along a 2-dimensional surface. Their circuit required two connections at the endpoints of the pathway (A and B). This design strategy reduces the dimension of the measured voltage outputs to what is required to decouple linear touch location and capacitance, and improves the simplicity of connecting the textile to measurement hardware. This strategy is acceptable for inferring static touch location or tap gestures, but it is non-ideal for inferring complex gesture pathways across the surface of the sensor. The measured output response is non-uniform between geometrically adjacent touch-points which complicates precise localization. Additionally, touch-points are spaced apart into a sparse sensitive area to prevent shunting when two or more touch-points are contacted. This spacing induces loss of contact when touch is transferred to another location, thus breaking the gesture’s continuity.

\begin{figure}[ht]
    \centering
    \hfill
    \subfigure[]
    {\includegraphics[width=0.3\textwidth]{src/figures/Planar_CTS_Touch_Sensing_Circuit.pdf}\label{fig:Planar_CTS_Touch_Sensing_Circuit}}
    \hfill
    \subfigure[]
    {\includegraphics[width=0.6\textwidth]{src/figures/Planar_CTS_Gesture_Data_Example.pdf}\label{fig:Planar_CTS_Gesture_Data_Example}}
    \caption{Illustrations of the measurement circuit and example recorded gesture data. \emph{(a)}: Illustration of the measurement circuit and resistor-capacitor network circuit formed during touch. \emph{(b)}: Plot of example measured data showing the changes in gain measurement during onset, gesture, and offset.}
    \Description{This figure illustrates the measurement circuit in (a), and the signal during different stages of a gesture event. In (a), an illustrated waveform generator, a digital oscilloscope, and a knitted sensor with four connection points are shown. Additionally, there is a 5-point star circuit illustrated, which includes the four sensor connections points A, B, C, D, and a user’s point of touch on the sensor. In (b), we see a graph on top with time ranging from 0 to 1 seconds on the X axis, and voltage gain values on the Y axis. There are four lines in the graph, each representing the signal measured from one of the electrodes. The plot is split into three areas in the time axis: before the gesture event, the gesture event, and after the gesture event. During the first and third phases, the behavior of the four lines is very similar, and their voltage gain values are high. However, during the gesture event, we notice a drop in the voltage gain value for all these values, and the differences among the individual lines becoming more pronounced. Below the plot, there are three illustrations of the four-connection knitted sensor with a hand touching the sensor. In the first illustration, we see the onset of touch, in the second one, the gesture being performed, and in the third, the offset.}
  \label{fig:Planar_CTS_Gesture_Data_Example}
\end{figure}

\subsection{Measurement Circuit}
\label{sec:measurement_circuit}
The capacitive touch circuit formed by the planar conductive area is modeled as a mesh resistor-capacitor ladder network. Figure~\ref{fig:Planar_CTS_Touch_Sensing_Circuit} illustrates the circuit diagrams of the resistances formed between the location of touch and the sensing points ($A, B, C, D$), represented as a star graph. The values of the resistances vary depending on touch location. The value $C_{t}$ represents the magnitude of capacitance induced by touch, which is used as a pseudo-pressure. The voltage measurement locations ($A, B, C, D$) have associated parasitic capacitances ($C_{p_{A}}$, $C_{p_{B}}$, $C_{p_{C}}$, $C_{p_{D}}$) which affect the voltages at the measurement locations. The excitation waveform passes through current-limiting resistors ($R_{A}$, $R_{B}$, $R_{C}$, $R_{D}$) that reduce the current entering and exiting the fabric, which allows voltage measurements to be discerned.

An example gesture and processed measurement are shown in Figure~\ref{fig:Planar_CTS_Gesture_Data_Example}. Figure~\ref{fig:Measured_Data_Processing_Pipeline} further explains the waveform processing steps. All measured gestures consist of a distinct touch onset, gesture action, and touch offset. During onset, the measured waveform gains decrease due to the increase in touch capacitance. The gain at each measurement point attenuates as a function of touch location, where attenuation increases as touch proximity increases. This phenomenon can be seen during the gesture action, where the motion trajectory and gain attenuation pf each sensing point correlate. Once the gesture is complete, the offset action returns the gain measurements to a baseline.

Waveform generation and acquisition are performed using a Keysight 33622A arbitrary waveform generator and a Keysight MSO-X-3024T oscilloscope. The waveform generator produces a 2 MHz sine wave used as input to the circuit (output 1) and 250 Hz square wave used to trigger oscilloscope sampling (output 2). The sine wave passes through current-limiting resistors attached to the four corners of the fabric circuit. The resistance values are approximately equal to the touchpad resistance of 4 kOhm/sq. A parasitic capacitance of approximately 60 pF is observed at each measurement point. Gesture data is collected over the span of 1 second at a rate of 250 frames per second. Each frame window consists of approximately 4000 samples collected at 625 MSamples/s. The frames are processed using Bode analysis to return the magnitude ratio (gain) of the input and measured waveforms. The output data is of dimension $250\times 4$ corresponding to the 250 measurement frames and 4 measurement channels.

\subsection{Signal Filtering}
\label{sec:wavelet_filtering}

After the data is captured by the four electrodes as changes of gain values over time, we subtract a baseline event, captured while no gesture was being performed on the knitted touchpad. The purpose of baseline subtraction is to remove any elements of the representation that stay the same across different measurements, therefore highlighting the differences between gestures, which makes classification easier. 

\begin{figure}
    \centering
    \includegraphics[width=0.95\textwidth]{src/figures/Measured_Data_Processing_Pipeline.pdf}
    \caption{Diagram of the data processing pipeline: The waveform data is sampled in segmented record windows over the duration of the gesture. The FFT of the window is processed and the magnitude of the bin at the input frequency is used to determine the window gain. The measured baseline gain is subtracted from the gesture gain data and a wavelet transform is applied to smooth high-frequency changes.}
    \label{fig:Measured_Data_Processing_Pipeline}
    \Description{This figure illustrates the signal processing steps after its acquisition. There are three major steps: the first one is recording the event window, while the input and output waveforms are continuously sampled. The second is converting the recorded time-domain window into frequency domain via Fast Fourier Transform. Subsequently, the baseline value is subtracted, and then the signal is filtered through wavelet transform.}
\end{figure}


Subsequently, in order to further reduce noise from the signal, we filter it using \emph{wavelet transform}, a powerful and flexible analytic tool which allows us to obtain signal information from both its time and frequency domains. It does not only indicate which frequencies are present in the signal, like Fourier Transform, but also when those frequencies occur in time. Wavelet Transform achieves this by calculated compromises: by low frequencies having a high resolution in the frequency domain, and low resolution in the time domain; and high frequencies having a low resolution in the frequency domain, and high resolution in the time domain. Wavelet Transform analyses the signal in different scales: first, working with larger windows of the signal for elements stretched in time, like low frequencies. Then, after that information is acquired, we can progressively make the window of interest smaller to look for information in those scales. As we shrink the window, we lose the ability to capture low frequencies, however, that information should have been obtained in the previous step. Ultimately, we are able to use the collected information from different scales to reconstruct the signal. 

In order to filter the signal, we first deconstruct the signal through a \emph{Symlet} mother wavelet with 4 vanishing moments. After having obtained all the elements necessary for reconstruction through analysis at different levels, we reconstruct the signal using only a subset of them. Typically, the higher frequencies correspond to noise, so they are removed. For our analysis, we keep only the first 4 levels of components. These settings were selected such that noise is removed, but the filtered signal follows the original closely. Figure~\ref{fig:Measured_Data_Processing_Pipeline} illustrates an example of the original signal after the baseline has been subtracted, as explained above, together with its filtered version through wavelet reconstruction.  

\begin{figure}[h]
    \centering
    \includegraphics[width=\textwidth]{src/figures/CNN-LSTM_Network_Architecture_2.pdf}
    \caption{CNN-LSTM neural network architecture: the four-signal representation of a gesture event is used as an input after filtering, and the output is a predicted gesture category. The CNN component captures spatial relationships in the signal pattern, while the LSTM component models time-dependency.}
    \label{fig:CNN-LSTM_Network_Architecture}
    \Description{This figure shows the neural network architecture. The input to the network is the gesture performed on the sensor after it has been processed. The output is a prediction about the gesture performed, out of 12 different classes of gestures. The input first goes through the convolutional neural network, which includes functions such as convolution, max-pooling, batch normalization and dropout. Then the output of the convolutional network becomes the input to a long-short term memory network with three hidden layers of 100 nodes each. The final layer is a fully connected layer, which then maps the values to the 12-node output layer through a Log SoftMax function.}
\end{figure}

\subsection{Neural Network Architecture}
\label{sec:classification_network}
The neural network architecture used to build the recognition model relies on a combination of CNNs and LSTMs. We chose deep learning over traditional machine learning techniques based on the strengths that both CNNs and LSTMs have shown in representing time-series signal data. Additionally, these networks automatically capture the important aspects of the data without needing feature construction, typically required by most non-neural network algorithms to represent a time-series signal.

The neural network takes as an input the data after being processed as described above. The architecture, illustrated in Figure~\ref{fig:CNN-LSTM_Network_Architecture}, starts with a one-dimensional convolution layer with a rectified linear unit (ReLU) activation function, followed by a batch normalization layer and a dropout layer for regularization purposes. It continues with another one-dimensional convolution layer, again with an ReLU activation function, followed by a max-pooling layer. The function of convolution is capturing local features at different scales in the signal. Subsequently, three layers of LSTM cells follow, each with a hidden size of 100. LSTMs have been widely used for sequential data analysis, including time-series signal data, since they capture the signal's temporal dependencies. Among recurrent layer variants, they are chosen since they are capable of handling long-term dependencies in sequential data~\cite{graves2013speech,sutskever2014sequence}.

Each input to the system consists of one gesture event as captured by the four electrodes connected to the sensor, and after being transformed to filtered frequency gain values. The sequence is 250 time steps long, and there are 4 values per time step, corresponding to an input layer of size 4 to the neural network. The temporally-related frequency gain values are transformed to one of 12 possible gesture classes through a 12-output linear layer at the end of the network. In order for the network to predict a correct gesture, it needs to be trained with user data, which we discuss in Sections~\ref{sec:experiments} and~\ref{sec:results}, together with the training hyper-parameter choices.


\subsection{Model Deployment}
\label{sec:model_deployment}
For an interactive system, after the above architecture is used for training, the resulting model needs to be deployed to lightweight hardware. We use a NVIDIA® Jetson Xavier™ NX Developer Kit  (Figure~\ref{fig:nvidia_picture}), running Ubuntu 18.04 with 8GB of RAM, a 6-core NVIDIA Carmel ARM CPU, and a NVIDIA Volta GPU for gesture classification (Table~\ref{tab:nvidia_specs}). The NVIDIA board's ability to do fast parallel math is important, as it allows running pre-trained models. In addition, its form factor is relatively compact, an important feature for many of the potential applications of the sensor.

Currently, the data used as the model input is a \emph{.csv} file containing one user-captured touchpad gesture. First, we subtract the baseline readings and perform waveform filtering to prepare the data for evaluation. Subsequently, the resulting waveform is input to the model for classification. The model then returns its prediction as an output, which is one of the 12 gesture classes. 

In the future, we plan to adapt the model deployment to handle continuous or real-time signal classification, as opposed to only pre-recorded discrete signals. A barrier to real-time signal classification is the practical lack of hardware that can perform the signal generation, acquisition, and filtering at the speeds necessary to continuously supply the model with new inputs. Custom hardware, which we aim to design, would be able to stream the acquired signal to the NVIDIA Jetson board for classification. Once the Jetson board would acquire the signal, it would need to parse it to isolate individual gesture windows, since the model would have been trained on inputs containing only a single gesture. An additional model may be necessary to accurately distinguish gestures. The main gesture classification model could then take gestures as inputs as they are returned from the parser.


\vspace{4mm}
\noindent
\begin{minipage}{\textwidth}
  \begin{minipage}[b]{0.49\textwidth}
      \centering
      \includegraphics[width=0.7\linewidth]{src/figures/jetson_nvidia.jpg}
      \captionof{figure}{NVIDIA Jetson Xavier NX Developer Kit~\cite{jetsonxaviernx}}
      \label{fig:nvidia_picture}
  \end{minipage}
  \hfill
  \begin{minipage}[b]{0.49\textwidth}
    \centering
    \captionof{table}{Hardware Specifications}
    \label{tab:nvidia_specs}
    \begin{tabular}[b]{c|p{0.7\textwidth}}
        \small{GPU} & NVIDIA Volta Architecture with 384 CUDA cores and 48 Tensor cores \\
        \hline
        \small{CPU}  &  6-core NVIDIA Carmel ARM Processor \\
        \hline
        \small{RAM} & 8GB LPDDR4x \\
        \hline
        \small{Size} & 70 mm x 45 mm\\
    \end{tabular}
    
  \end{minipage}
\end{minipage}
\vspace{4mm}





We present in section~\ref{ssec:faces} an application of PnP-HVAE on face images, using a pretrained state-of-the-art hierarchical VAE. 
Next, we study the application of our framework to natural images. To that end, we introduce  in section~\ref{ssec:patchVDVAE}  a patch hierachical VAE architecture, that is able to model natural images of different resolutions. In section~\ref{ssec:app_nat}, we provide deblurring, super-resolution and inpainting experiments to demonstrate the relevance of the proposed method.

Additional results are presented in Appendix~\ref{app:add}. All experiments can be reproduced using the code available at \url{https://github.com/jprost76/PnP-HVAE}.



\subsection{Face Image restoration (FFHQ)}\label{ssec:faces}
We first demonstrate the effectiveness of PnP-HVAE on highly structured data, by performing face image restoration.
Latent variable generative models can accurately model structured images such as face images \cite{karras2019style,vahdat2020nvae,child2021very,kingma2018glow}, and then be used to produce high quality restoration of such data. 
In our experiments, we use the VDVAE model of~\cite{child2021very}, pre-trained on the FFHQ dataset~\cite{karras2019style}, as our hierarchical VAE prior.
VDVAE has $L=66$ latent variable groups in its hierarchy and generates images at resolution $256\times256$.

We compare PnP-HVAE with the intermediate layer optimization algorithm (ILO)~\cite{daras2021intermediate} that is based on a different class of generative models than HVAE. ILO is a GAN inversion method which optimizes the image latent code along with the intermediate layer representation of a StyleGAN to generate an image consistent with a degraded observation.
We use the official implementation of ILO, along with a StyleGAN2 model~\cite{karras2020analyzing, stylegan2pytorch}, that was trained for 550k iterations on images of resolution $256\times256$ from FFHQ.  
As VDVAE and StyleGAN models are not trained on the same train-test split of FFHQ, we chose to evaluate the methods on a subset of 100 images from the CelebA dataset~\cite{liu2018large}. 
For super-resolution, the degradation model corresponds to the application of a gaussian low-pass filter followed by a $\times 4$ sub-sampling, and the addition of a gaussian white noise with $\sigma=3$.
For the deblurring, we considered motion blur and  gaussian kernels, both with a noise level $\sigma=8$. %

We provide quantitative comparisons in table~\ref{table:comp_ILO}, along with a visual comparison of the results in figure~\ref{fig:face_restoration}.
PnP-HVAE has the best  PSNR and SSIM results for all the considered restoration tasks, while ILO provides better results  for the perceptual distance.
By jointly optimizing the image and its latent variable, PnP-HVAE provides  results that are both realistic and consistent with the degraded observation.
On the other hand,  ILO  only optimizes on an extended latent space. This method generates  sharp and realistic images with better LPIPS scores,   
but the results lack  of consistency with respect to the observation, which explains the overall lower PSNR performance. 






\subsection{PatchVDVAE: a HVAE for natural images}\label{ssec:patchVDVAE}
Available generative models in the literature operate on images of  fixed resolutions and
are either restrained to datasets of limited diversity, or even to registered face images~\cite{kingma2018glow,child2021very, vahdat2020nvae, karras2019style}, or requiring additional class information~\cite{brock2018large, dhariwal2021diffusion, song2020score, luhman2022optimizing}.
Fitting an unconditional model on natural images appears to be a more difficult task, as their resolution can change, and their content is highly diverse.
The complexity of the problem can be reduced by learning a prior model on patches of reduced dimension. 
For image restoration problems, the patch model can be reused on images of higher dimensions~\cite{zoran2011learning,prost2021learning,altekruger2022patchnr}. When the model is a full CNN, the prior on the set of the  patches can  be computed efficiently by applying the network on the full image~\cite{prost2021learning}.

We thus introduce  patchVDVAE, a fully convolutional hierarchical VAE.
Contrary to existing HVAE models whose resolution is constrained by the constant tensor at the input of the top-down block, patchVDVAE can generate images of different resolutions by controlling the dimension of the input latent. 
This amounts to defining a prior on patches whose dimension corresponds to the receptive field of the VAE. A similar model is used for image denoising in~\cite{prakash2021interpretable}.

 
For PatchVDVAE architecture, we use the same bottom-up and top-down blocks as VDVAE~\cite{child2021very}, and replace the constant trainable input in the first top-down block by a latent variable, to make the model fully convolutional (details on the  architecture are given in Appendix~\ref{app:details}). 
The training dataset is composed of $128\times 128$ patches extracted from a combination of DIV2K~\cite{agustsson2017ntire} and Flickr2K~\cite{Lim_2017_CVPR_workshops} datasets.
We perform data augmentation by extracting  patches at $3$ resolutions: HR-images and $\times 2$ and $\times 4$ downscaled images. 
The model is trained for $7.10^5$ iterations with a batch size of $64$. Following the recommendation of~\cite{hazami2022efficient}, we use Adamax optimizer with an exponential moving average and gradient smoothing of the variance.
We set the decoder model to be a gaussian with diagonal covariance, as in~\cite{luhman2022optimizing}.
PatchVDVAE is fully convolutional and can generate images of dimension that are multiples of $64$ as illustrated by
figure~\ref{fig:vdvae}.

\newlength{\patchwidth}
\setlength{\patchwidth}{0.135\columnwidth}
\begin{figure}[!ht]
    \centering
    \begin{subfigure}[t]{.34\columnwidth}\hspace{0.1cm}
        \setlength{\tabcolsep}{0.02pt}
\renewcommand{\arraystretch}{0}
        \begin{tabular}{*{2}{p{1.03\patchwidth}}}
            \includegraphics[width=\patchwidth]{figures_arxiv/patchVDVAE/samples/generated/64x64/setup-5-image-0018.png} &
            \includegraphics[width=\patchwidth]{figures_arxiv/patchVDVAE/samples/generated/64x64/setup-5-image-0016.png} \\
            \includegraphics[width=\patchwidth]{figures_arxiv/patchVDVAE/samples/generated/64x64/setup-5-image-0008.png} &
            \includegraphics[width=\patchwidth]{figures_arxiv/patchVDVAE/samples/generated/64x64/setup-5-image-0019.png}   
        \end{tabular}
    \end{subfigure}\hspace{-0.15cm}
    \begin{subfigure}[t]{.64\columnwidth}
\begin{tabular}{cc}\vspace{-0.1cm}
\includegraphics[width=2\patchwidth]{figures_arxiv/patchVDVAE/samples/generated/256x256/setup-2-image-0009.png}&
        \includegraphics[width=2\patchwidth]{figures_arxiv/patchVDVAE/samples/generated/256x256/setup-2-image-0002.png}\end{tabular}

    \end{subfigure}
    \caption{\label{fig:vdvae} Left: $64\times64$ patches samples from our patchVDVAE model trained on patches from natural images.
    Right: PatchVDVAE is fully convolutional and it can generate images of higher resolution (here: $128\times128$).\vspace{-0.2cm}}
\end{figure}

\subsection{Natural images restoration}\label{ssec:app_nat}
We  evaluate PnP-HVAE on natural image restoration.
For each task, we report the average value of the PSNR, the SSIM, and the LPIPS metrics on $20$ images from the test set of the BSD dataset~\cite{MartinFTM01}.\\


\noindent
{\bf Image deblurring.}
In the experiments, we consider $2$ gaussian kernels and $2$ motion blur kernels from~\cite{levin2009understanding}, with $3$ different noise levels 
$\sigma \in \{2.55, 7.65, 12.75\}$.
As a baseline we consider  EPLL~\cite{zoran2011learning}, which learns a prior on image patches with a gaussian mixture model.
We also compare PnP-HVAE  with PnP-MMO and GS-PnP, $2$ competing convergent Plug-and-Play methods based on CNN denoisers.
PnP-MMO~\cite{pesquet2021learning} restricts the denoiser to be contraction in order to guarantee the convergence of the PnP forward-backard algorithm. GS-PnP~\cite{hurault2022gradient} considers a gradient step denoiser and reaches state-of-the-art performances of non converging methods~\cite{zhang2021plug}.
We set the temperature $\tau$  in our method as $0.95$, $0.8$ and $0.6$ for noise levels $2.55$, $7.65$ and $12.75$ respectively, and we let it run for a maximum of $50$ iterations. 
For the three compared methods we use the official implementations and pre-trained models provided by the respective authors. 
Details on the choice of hyperparameters for the concurrent methods are provided in the Appendix~\ref{app:details}
Figure~\ref{fig:deblurring_bsd} illustrates that our method provides correct deblurring results. 

According to table~\ref{tab:deb}, the performance of PnP-HVAE is between those of EPLL and GS-PnP and it outperforms PnP-MMO for large noise levels.\\

\begin{table}
\begin{center}\footnotesize
    \begin{tabular}{>{\centering}m{.3cm}*{5}{c}}
    $\sigma$ &Method & PSNR$\uparrow$ & SSIM$\uparrow$ & LPIPS$\downarrow$  \\ 
    \hline
    \multirow{4}{*}{\vcell{$2.55$}}
    & PnP-HVAE & $27.75$ & $0.79$ & $0.31$\\
    & GS-PNP \cite{hurault2022gradient} & $\mathbf{29.59}$ & $\mathbf{0.84}$ & $\mathbf{0.22}$\\
    & EPLL \cite{zoran2011learning} & $26.49$ & $0.71$ & $0.36$\\ 
    & PnP-MMO \cite{pesquet2021learning} & $\underbar{29.50}$ & $\underbar{0.83}$ & $\underbar{0.20}$ \\ \hline
    \multirow{4}{*}{\vcell{$7.65$}}
    & PnP-HVAE & $\underbar{26.36}$ & $\underbar{0.72}$ & $\underbar{0.40}$\\
    & GS-PNP \cite{hurault2022gradient} & $\mathbf{27.33}$ & $\mathbf{0.77}$ & $\mathbf{0.31}$\\
    & EPLL \cite{zoran2011learning} & $24.04$ & $0.66$ & $0.45$ \\ 
    & PnP-MMO \cite{pesquet2021learning} & $25.34$ & $0.69$ & $0.34$\\
    \hline
    \multirow{4}{*}{\vcell{$12.75$}}
    & PnP-HVAE & $\underbar{25.12}$ & $\mathbf{0.73}$ & $\underbar{0.47}$\\
    & GS-PNP \cite{hurault2022gradient} & $\mathbf{26.32}$ & $\mathbf{0.73}$ & $\mathbf{0.37}$\\
    & EPLL \cite{zoran2011learning} & $23.28$ & $0.61$ & $0.51$ \\ 
    & PnP-MMO \cite{pesquet2021learning} & $22.42$ & $0.53$& $0.54$ \\
    \hline
    &\vspace*{-.3cm}\\
            \multicolumn{2}{c}{Blur and motion kernels}& \multicolumn{3}{c}{
        \includegraphics*[scale=1]{figures_arxiv/kernels/4.png}\;\includegraphics*[scale=1]{figures_arxiv/kernels/7.png}\;\includegraphics*[scale=1]{figures_arxiv/kernels/9.png}\;\includegraphics*[scale=1]{figures_arxiv/kernels/11.png}} 
    \end{tabular}
        \caption{\label{tab:deb}Comparison  of PnP-HVAE  and other restoration methods on deblurring. Results are averaged on $4$ kernels.\vspace{-0.2cm}}% on image deblurring.}
    \end{center}
\end{table}

\begin{figure}
    
    \begin{subfigure}[h]{\linewidth}
        \centering
        \includegraphics*[width=\columnwidth]{figures_arxiv/deb_s255_k7.pdf}\vspace{-0.1cm}
        \caption{Gaussian blur, $\sigma=2.55$}
    \end{subfigure}
    \begin{subfigure}[h]{\linewidth}
        \centering
        \includegraphics*[width=\columnwidth]{figures_arxiv/deb_s765_k11.pdf}\vspace{-0.1cm}
        \caption{Motion blur, $\sigma=7.65$}
    \end{subfigure}\vspace*{-0.1cm}
    \caption{\label{fig:deblurring_bsd} Natural image deblurring\vspace{-0.1cm}}
\end{figure}

\noindent {\bf Effect of the temperature.}
PnP-HVAE gives control on the temperature of the prior over the latent space.
In figure~\ref{fig:temp_effect}, we illustrate that reducing the temperature increases the strength of the regularization prior. In this example the tuning $\tau=0.7$ produces the best performance.\\
\begin{figure}[!ht]
   
    \includegraphics[width=\columnwidth]{figures_arxiv/demo_temp.pdf}\vspace{-0.15cm}
    \caption{ \label{fig:temp_effect} Effect of the temperature in PnP-VAE on a deblurring problem, with $\sigma=7.65$.\vspace{-0.15cm}}
\end{figure}


\noindent
{\bf Image inpainting.}
Next we consider the task of noisy image inpainting. 
We compose a test-set of 10 images from the validation set of BSD~\cite{MartinFTM01} and we create masks
  by occluding diverse objects of small size in the images. 
A gaussian white noise with $\sigma=3$ is added to the images.
As a comparaison, we still consider GS-PnP and EPLL.
For PnP-HVAE, the temperature is set to $\tau=0.6$, and the algorithm is run for a maximum of $200$ iterations, unless the residual $||\x_{k+1}-\x_k||$ is on a plateau.
We provide on Table~\ref{tab:inpainting_bsd} the distortion metrics with the ground truth, as well as a visual
\begin{table}



\begin{center}
    \begin{tabular}{cccc}
        & PSNR$\uparrow$ & SSIM$\uparrow$ &LPIPS$\downarrow$ \\\hline
        PnP-HVAE  & $\mathbf{29.54}$ & $\mathbf{0.93}$ & $\mathbf{0.06}$\\
        GS-PNP & $28.52$ & $\mathbf{0.93}$ & $0.09$\\
        EPLL & $\underline{29.16}$ & $\mathbf{0.93}$ & $\mathbf{0.06}$\\
    \end{tabular}
    \caption{\label{tab:inpainting_bsd}Quantitative evaluation for inpainting on BSD.}
    \end{center}
\end{table}
comparison on figure~\ref{fig:inpainting_bsd}. 
With its hierarchical structure,  PnP-HVAE outperforms the compared methods. \vspace{0.05cm}



\begin{figure}[!h]
    \includegraphics[width=\columnwidth]{figures_arxiv/demo_inp_bsd2.pdf}\vspace{-0.1cm}
    \caption{\label{fig:inpainting_bsd}Natural image inpainting\vspace{-0.3cm}}
\end{figure}











\section{Results}
\label{results}

\begin{figure*}[ht]
    \centering
    \includegraphics[scale=0.15,trim={0 2.5cm 0 5cm},clip]{images/aoi-single_burst}
    \caption{The time average peak Age of Information with burst and \gls{soa} loss values against the dynamic reliability logic for different network topologies.}
    \label{fig:aoi_burst}\vspace{-0.4cm}
\end{figure*}


This paper focuses on both transport layer and application layer metrics to determine the feasibility of dynamic reliability. For this, we have selected the session packet volume, as transmitted, retransmitted, lost and backlogged packets as \glspl{kpi} for the transport layer; while focusing on the \gls{aoi} for the application layer. The \gls{aoi} was chosen as a crucial indicator for the freshness of packets in real-time applications. More specifically, this work adopts the time average peak \gls{aoi} equation \cite{aoi_equation} depicted in Eq. \ref{aoi}, where $\Delta(r_{i+1})$ is the $i$th update at the time it was received at the server, for a session time period of $\tau$.

\begin{equation}
    \label{aoi}
    \gls{aoi}_\tau = \frac{1}{n-1}\sum_{i=1}^{n-1} \Delta(r_{i+1})
\end{equation}

We include a comparison between the vanilla QUIC implementation which does not enjoy the dynamic reliability extension, with a number of dynamic reliability policies. The tests were run a number of times for statistical significance, with the mean value of vanilla implementation used as a baseline for comparison. The topology utilised both random loss and bursty loss to explore the bounds of dynamic reliability. The \gls{soa} loss in the figures correspond to the loss values presented in Table. \ref{tab:path_char}, for ease of comparison between bursty and random loss scenarios.

\subsection{Transport-Layer KPIs}

To analyse the performance gain at the transport layer due to dynamic reliability, the volume of transmitted and backlogged packets is examined. The figures are in the form of boxplots, which take the vanilla implementation as a benchmark, depicted as the red dashed line.

As seen in Fig. \ref{fig:sent_burst}, the loss plays a crucial role in the performance of the reliability policies. The policies under random loss did incredibly well for the networks with a larger capacity, namely \gls{mmwave} and Sub-6~GHz, whereas for burst loss, the lower network capacities had a larger packet reduction. With the increase in burst loss, the behaviour of the set split reliable policies became unpredictable, if a reliable assignment happened to coincide with a burst loss, the number of transmitted packets increases, and vice versa. On the other hand, in smarter policies, such as Loss-Aware, the performance lightly matched the vanilla baseline, as the reliable assignment dominated the session to compensate for a higher burst loss. Not only that but, the burst loss also impacted the variance of the transmitted packets for the policies.

Unsurprisingly, the unreliable focused policy, 80-20 split, outperformed other policies for all topologies in random and bursty loss scenarios, with an approximate reduction of 80\%. That being said, the majority of the policies reduced the transmitted packets on the link by approximately 70\% for random loss, while the reduction started at $\approx 15\%$ and decreased as the loss increased for the burst loss scenario.

The retransmitted and lost packets, not shown due to space limitations, followed the same trend as the transmitted packets for the random loss scenarios. However, for the burst loss scenarios, the larger capacity networks had a lower reduction in the retransmitted and lost packets. This can be seen as a favorable outcome since the lower capacity networks are scarce on resources. It is important to note that the Loss-Aware policy mimicked the vanilla approach as the burst loss increased, signifying the overwhelming appointment of reliable packets in adapting to the harsh burst loss conditions.
 
Alternatively, Fig. \ref{fig:backlog_burst} clearly shows a stark comparison between the policies and loss scenario in the reduction of the backlogged packets. The Loss-Aware policy for random loss scenario reduced the backlogged packets by up to 50\%, beating all other policies by approximately 30\%. Furthermore, it is clear that the unreliability focused policies resulted in the lowest backlog for the session. In comparison, we notice that the burst loss and the backlogged frequency have a positive correlation, where the maximum reduction of the backlogged packets for the policies is at most 20\%. Much like the transmitted packets, the probability of a burst loss occurrence plays a vital role in the number of retransmissions sent and by extension the number of backlogged packets. Thus, we can conclude that the stress placed on the buffer is a result of the reliable packets which is tightly coupled with the congestion on the session. Whereas, unreliable focused policies did not encounter such a phenomenon regardless if it was experiencing a burst loss.


\subsection{Application-Layer KPIs}

The feasibility of dynamic reliability for real-time applications can be determined by the \gls{aoi}, with comparison across different topologies and policies. If we take a strict approach and consider anything below $10$~ms is real-time \cite{real-time}, then all the reliability policies passed that requirement, which is attractive for real-time applications, as shown in Fig. \ref{fig:aoi_burst}. Utilising the median as an estimate of the runs, the policies in the WLAN and Sub-6~GHz topology with random loss floated around $4-5$~ms with negligible difference, while the \gls{aoi} for \gls{mmwave} was $\approx 2-3$~ms. It is clear that the \gls{aoi} and the network capacity have a negative correlation, as the network capacity decreases, the \gls{aoi} increases. The same correlation is extended to the bursty loss scenarios, where \gls{mmwave} dominated the other topologies. That being said, it is crucial to note that the \gls{aoi} for the reliability policies is often slightly better than or equal to the \gls{aoi} of the vanilla implementation, proving that dynamic reliability reduces the congestion of the session at no cost to the \gls{aoi}.

\section{Real-World Applicability}
The experimental results in Section~\ref{sec:results} are encouraging for the feasibility of our proposed technology, however they were performed in a controlled laboratory setting. Specifically, for all those experiments the sensor pinned in a static position on the table during data acquisition. In order to expand the experimental set up of CTS gesture pads and evaluate their functionality and accurately in a setup resembling the  real world, there are many many other aspects to be considered. In this section, we begin discussing and investigating some of these possibilities, even though further explorations are necessary. First, we conduct a new study, with subjects wearing the sensor while inputting gesture data. Next, we evaluate the effect that washing and drying have on the resistance of the sensor. Finally, we discuss the application potential and integration of this system.  

\subsection{Model Performance While the Sensor is Being Worn}
\label{sec:wearing_sensor}
One of the main ways the proposed sensor is expected to be used in the real world is via clothing integration, which is facilitated by the compact sensing area. The sensor in proximity to the human body induces capacitance, which is the principle upon which its circuit design relies. When worn near the skin, there is an additional capacitance induced, greater than the baseline parasitic capacitance. Ideally, when worn, the skin should not contact the sensor. Shielding the sensor from underside contact is currently achieved by knitting a back layer on the sensor and routing conductive yarn only on the top layer. 

\begin{figure*}[ht]
    \centering
    \subfigure[]{\raisebox{5.5mm}
    {\includegraphics[height=0.285\textwidth]{src/figures/wearing_sensor_connected-min.pdf}\label{fig:sensor_wearing_forearm}}}
    \hfill
    \subfigure[]
    {\includegraphics[width=0.495\textwidth]{src/figures/final_wearing_heatmap.pdf}\label{fig:wearing_confusion}}
    \hfill
    \caption{User study to test model performance while knitted sensor is being worn. The setup in \emph{(a)} shows the knitted sensor fixed on a removable Velcro-strapped pad to be worn on the forearm. Four electrodes are connected to the corners of the sensing area, similarly to Figure~\ref{fig:capturing_gesture} for data collection. The heatmap generated in \emph{(b)} shows the classification results for each gesture pathway. The matrix rows denote the true row categories of the gestures, while the columns show the ones predicted during evaluation. There is a clear difference between this figure and the heatmaps in Figure~\ref{fig:Confusion_Matrices}. However the diagonal in the middle is still distinguishable from the rest of the values.}
    \label{fig:wearing_sensor}
    \Description{This figure shows a user touching on a knitted sensor while wearing it on their forearm in (a). Figure (b) shows the gesture classification matrix, with the diagonal values being higher than the rest of the values, but less pronounced than the diagonals of Figure 10.}
\end{figure*}

\subsubsection{Methods and Results}
In order to further examine the robustness of the gesture recognizing model introduced above and its usefulness in the real world, we conducted a new user study with 3 subjects. The study is similar in design and analysis to the cross-validation and evaluation studies described in Sections~\ref{sec:experiments} and~\ref{sec:results}, but in this case, the subjects were wearing the sensor on their forearm while performing the same 12 gestures as in the studies above. The sensor was pinned on another Velcro-strapped pad as shown in Figure~\ref{fig:sensor_wearing_forearm}. Subjects were not moving while collecting the data, but some motion of their arms would be expected. Each subject performed each of the 12 gestures 20 times, with a total of 720 samples, or 60 per class collected. This dataset was tested against the already trained models from the cross-validation study. No additional training was performed to account for this new condition.

\begin{table}[h]
  \centering
  \caption{Classification results for gestures performed while the sensor was being worn.}~\label{tab:wearing_results}
  \vspace{0.5cm}
  \begin{tabular}{l|cccc}
    & {\small \textit{Accuracy}} & {\small \textit{F1-Score}} & {\small \textit{Precision}} & {\small \textit{Recall}}\\
    \midrule
    \small{CNN + LSTM} & 58.0\% & 58.2\% & 61.2\% & 58.0\% \\
    \Description{This table shows the evaluation performance measures of the sensor while it is being worn using trained CNN-LSTM model. The performance measures are accuracy, precision, recall, and F1-score. The evaluation accuracy is 58.0\%.}
  \end{tabular}
\end{table}

Table~\ref{tab:wearing_results} demonstrates the results computed for the same performance measures as the studies above, including: \emph{accuracy}, \emph{precision}, \emph{recall}, and \emph{F1-score}. Figure~\ref{fig:wearing_confusion} illustrates the classification matrix of the real and predicted gestures. The computed accuracy is $58.0\%$, which is lower than both the cross-validation and testing accuracy, however well above chance accuracy (8.3\%). We would expect increased accuracy with training that includes samples collected while the sensor is being worn. For a complete system implementation, training should be performed under a large variety of conditions to ensure robustness to different real-world circumstances.

\subsection{Effect of Washing and Drying on Sensor Resistance}
\label{sec:washing_drying}
Another aspect in evaluating the robustness for use of a knitted sensor is the effect of washing and drying on its conductivity. 
%First, it should be noted that some distortion is to be expected to occur after washing and drying the fabric component for the first time, due to the fabric shrinking in size to some extent. 
The resistance of the sensor is an important property in representing the resulting signal, and subsequently building a gesture-recognizing learning model~\cite{Vallett2016a,Vallett2019a,mcdonald2020knitted}. Minor changes in resistance from activities like folding and stretching are anticipated and can be accounted for within the sensing and signal processing pipeline. Furthermore, laundering is an essential post-processing step in the manufacturing process that permanently sets physical yarn properties, such as expanding heat-bulking Nylon fibers, which in turn alter the baseline electrical conductivity. The sample tested has been washed and dried before these experiments, in addition to being steamed, as part of its manufacturing process. In this experiment, we first measure the baseline resistance across every pair of connection points illustrated in Figure~\ref{fig:sensor_sketch}. Then, we wash and dry the sensor for five cycles, measuring the resistance across the same pairs of points after each cycle.

Planar conductivity involving multiple connection points is described using the symmetric matrix shown in (\ref{eqn:Conductivity_Matrix}), which is an extrapolation of \emph{Kirchhoff's Current Law} stating that the sum of the currents entering a node is equivalent to the sum of the currents exiting it. In this application, conductivity, $G$, is directly proportional to current, $I$, and inversely proportional to resistance, $R$, such that $I \simeq G=R^{-1}$. The inverse of the resistance values indicated in Figure~\ref{fig:sensor_sketch} comprise the non-diagonal elements of the conductivity matrix. The change in conductivity is assumed to be scalar in that the values of the conductivity matrix will change proportionally. In practice, these values vary due to local changes in conductivity. Therefore, the average change in values is used to describe the cumulative change in conductivity.

\small
\begin{equation}
    G_{ABCD} = 
    \begin{bmatrix}
        -\left(G_{AB} + G_{AC} + G_{AD}\right) && G_{AB} && G_{AC} && G_{AD} \\
        G_{AB} && -\left(G_{AB} + G_{BC} + G_{BD}\right) && G_{BC} && G_{BD} \\
        G_{AC} && G_{BC} && -\left(G_{BC} + G_{AC} + G_{CD}\right) && G_{CD} \\
         G_{AD} && G_{BD} && G_{CD} && -\left(G_{AD} + G_{BD} + G_{CD}\right)
    \end{bmatrix}
    \label{eqn:Conductivity_Matrix}
\end{equation}

\normalsize
\begin{wrapfigure}{l}{0.35\linewidth}
  \centering
    \includegraphics[width=0.8\linewidth]{src/figures/Planar_Touchpad_Resistance_Diagram.pdf}
    \captionof{figure}{Annotated sketch of the knitted sensor, showing points along which resistance was measured.}~\label{fig:sensor_sketch}
    \Description{This figure shows the sketch of the knitted sensor, illustrating four electrode connection points, A, B, C, D, each in one corner of the rectangular area. The edges and diagonals of the figure are annotated to show a resistance between each each pair of points.}
\end{wrapfigure}

\subsubsection{Procedure and Results}
The sensor was washed according to the American Association of Textile Chemists and Colorists (AATCC) Laboratory Procedure 1-2018 Home Laundering: Machine Washing protocol~\cite{AATCC}. This protocol specifies a 35 minute wash duration with $1.8$ kg of laundry and $66 \pm 1$ g of detergent, and a standard tumble drying protocol with a temperature of $68 \pm 6^\circ C$. This protocol was chosen as appropriate for everyday laundering of clothing.

For the proposed sensor, we measured 6 resistance values for each of the two conditions: \emph{baseline ($b$)}, and \emph{washing and drying ($d_n$)}, where $n =$ 1 to 5 indicates the cycle number. The values were measured between every pair of corners in the sensor, annotated as $A, B, C$ and $D$ in Figure~\ref{fig:sensor_sketch}, which represent the connection points to the measurement hardware. To fully characterize the resistance across the conductive area of the sensor the resistance between each pair of connection points is necessary. For each resistance measurement, 100 samples were captured using a Keysight 34465A digital multi-meter, which were then averaged to represent the value of that measurement. The results are included in Table~\ref{tab:wash_dry_resistance}. For each test, the percent change in resistance between the baseline measurements and measurements after each washing and drying cycle was calculated as $\%\Delta R_{(b,d_n)} = (R_{d_n} - R_b) / R_b$. In this case, $R_{d_n}$ stands for the resistance value between the two points after the $n^{-th}$ washing and drying cycle $(d_n)$, and $R_b$ for the baseline resistance value between those same points, before any of the washing and drying cycles recorded. 
%In order to provide an overall view of the resistance change across the sensor, the absolute values of resistance changes across all point pairs were averaged. 
The cumulative change in resistance is calculated from the cumulative average of the element-wise matrix division of the baseline and drying cycle conductivity matrices formed using the relation in (\ref{eqn:Conductivity_Matrix}), where $\%\Delta R_{\left(b,d_{n}\right)} = avg\left(G_{b} / \left(G_{d_{n}} - G_{b}\right)\right)$.

\vspace{0.2cm}
\begin{table*}[ht]
  \caption{The percent change in resistance ($\%\Delta R$) between the \emph{baseline ($b$)} and each of the experiments after \emph{washing and drying ($d$)} is reported. The resistance values are measured between all pairs of the sensor's connection points. The cumulative change in resistance is also reported.}~\label{tab:wash_dry_resistance}
  \begin{tabular}{c|cccccc|c}
    & {\small $[A, B]$} & {\small $[A, C]$} & {\small $[A, D]$} & {\small $[B, C]$} & {\small $[B, D]$} & {\small $[C, D]$} & {\small $Cumulative$}\\
    
    \midrule
    
    \small \textit{$\%\Delta R_{(b, d1)}$} & \DeltaRBDryOneAB & \DeltaRBDryOneAC & \DeltaRBDryOneAD & \DeltaRBDryOneBC & \DeltaRBDryOneBD & \DeltaRBDryOneCD & $8.10\%$ \\
    \small \textit{$\%\Delta R_{(b, d2)}$} & \DeltaRBDryTwoAB & \DeltaRBDryTwoAC & \DeltaRBDryTwoAD & \DeltaRBDryTwoBC & \DeltaRBDryTwoBD & \DeltaRBDryTwoCD & $22.16\%$ \\
    \small \textit{$\%\Delta R_{(b, d3)}$} & \DeltaRBDryThreeAB & \DeltaRBDryThreeAC & \DeltaRBDryThreeAD & \DeltaRBDryThreeBC & \DeltaRBDryThreeBD & \DeltaRBDryThreeCD & $9.69\%$ \\
    \small \textit{$\%\Delta R_{(b, d4)}$} & \DeltaRBDryFourAB & \DeltaRBDryFourAC & \DeltaRBDryFourAD & \DeltaRBDryFourBC & \DeltaRBDryFourBD & \DeltaRBDryFourCD & $11.23\%$ \\
    \small \textit{$\%\Delta R_{(b, d5)}$} & \DeltaRBDryFiveAB & \DeltaRBDryFiveAC & \DeltaRBDryFiveAD & \DeltaRBDryFiveBC & \DeltaRBDryFiveBD & \DeltaRBDryFiveCD & $3.19\%$ \\
    % \midrule
    % \small \textit{$\%\Delta R_{(b, w1)}$} & $0\%$ & $0\%$ & $0\%$ & $0\%$ & $0\%$ & $0\%$ & $0\%$ \\
    % \small \textit{$\%\Delta R_{(b, w5)}$} & $0\%$ & $0\%$ & $0\%$ & $0\%$ & $0\%$ & $0\%$ & $0\%$ \\
    \Description{This table shows the effect that washing and drying have on the resistance. The rows show the change in resistance between the baseline, and values measured after each washing and drying cycle 1-5. The columns show the point pairs along which the resistance is measured: [A,B], [A,C], [A,D], [B,C], [B,D], [C,D], and additionally the cumulative resistance change value. Most of the cumulative results values are between 3\% and 11\%, while in the rest of the table, values range from 1\% to 31\%.}
  \end{tabular}
\end{table*}

The results in Table~\ref{tab:wash_dry_resistance} show that there is stability in the resistance measurements of the sensor after washing and drying it. However, there is a change in the cumulative resistance across washing and drying trials. The fact that the change in overall resistance does not substantially increase from trial to trial is promising. However, the change in resistance varies from approximately from 1\% to 31\% for individual connection point pairs. In addition, values in the second washing and drying trial $d_2$ seem higher compared to previous and subsequent trials, possibly due to a measurement error. Further testing is necessary to investigate the effects of washing and drying more comprehensively, and this observed variability needs to be incorporated into the model design, so that gestures recognition is stable for any applications depending on it. 


\subsection{Building Interactive Applications with Gesture-Recognizing Knitted Sensors}
\label{sec:application_potential}

The system components described in Section~\ref{sec:system_design} create the foundation for building an interactive system that uses a knitted sensing area to accept gesture input and a machine learning model to determine the gesture performed by the user. In a real-time system, gestures can subsequently be interpreted by the specific application to trigger different events. This technology enables many kinds of applications, and additionally offers extensibility. Figure~\ref{fig:interactive_system} shows two related views of the system. On the left, it illustrates how data can be collected from users to train and evaluate a machine learning model. Gestures can be determined by the application needs, and the experiments and results in this work show that it is possible to recognize even relatively complex gestures, such as letters and numbers, with high accuracy. This allows great flexibility in the choice of gesture sets for training, even on a small-sized sensing area. Training happens offline on a server and after the cross-validation results achieve the required accuracy, the trained model is further evaluated, and then deployed on a smaller system, such as the NVIDIA Jetson computer. 

\begin{figure}[h]
    \centering
    \includegraphics[width=\textwidth]{src/figures/Interactive_System_Model_Deployment_Workflow.pdf}
    \caption{Processes and component interactions that describe the model creation and the working of an interactive gesture recognition system. Data collection, training and evaluation happen off-line and typically require more time and computing power. Once a model is trained to high accuracy, it can be deployed on lightweight hardware to recognize gestures in real time, supporting different interactive applications.}
    \label{fig:interactive_system}
    \Description{This figure illustrates two related process with their respective components: model training and evaluation on the left, and building an interactive system on the right. In the center, there are two components, a knitted sensor and embedded micro-controller, which are part of both processes. On the left, gesture sets for training and evaluation are depicted, gathered through user interaction with the knitted sensor. The sensor passes that information to the embedded micro-controller, and the data from there is used for training in a server computer. The server outputs a trained model, which is then deployed to the NVIDIA Jetson system-on-module, found on the right side of the figure, in the interactive system part. That process starts with the user performing a gesture on the knitted sensor, which is connected to the embedded micro-controller. The latter transmits the signal to the NVIDIA Jetson board, which, through its trained model, returns a gesture type. This gesture is then interpreted by an application, which gives a custom response to the user.}
\end{figure}

On the right side of Figure~\ref{fig:interactive_system}, the implementation of an interactive system based on this technology is illustrated. A user enters a gesture input on the knitted sensing area, connected to an embedded system for signal acquisition and pre-processing. This system would communicate in real time with the NVIDIA Jetson board hosting the trained gesture recognition model, capable of interpreting the signal as the gesture the user intended. The application relying on this technology would then respond to the user based on the meaning assigned to the particular gesture. %Signal acquisition and processing in Figure~\ref{fig:interactive_system} is illustrated through a micro-controller, for both the interactive system, and general data collection. In this work, data was collected using an oscilloscope as described in the setup in Figure~\ref{fig:Planar_CTS_Gesture_Data_Example}, but previous work~\cite{mcdonald2020knitted} has 

This system configuration is extensible in the types of gestures recognized, since the model can be re-trained offline and re-deployed on existing hardware, allowing for large-scale production. Previous work~\cite{Vallett2019a,mcdonald2020knitted} has explored the potential of similarly-constructed sensors, and has introduced prototypes to illustrate their possible functionalities. The developments introduced in this work also hold promise for creating innovative applications. Some potential examples are broadly described below:

\begin{itemize}
    \item[-] \emph{Character Recognition:} The gesture examples used for recognition in this work are a subset of the character set of the English language. They were used to explore the feasibility of constructing a character recognition system. Having a system trained on the whole character set would allow written messages captured through fabric sensors to be transmitted. Since letters already have meaning embedded in them by convention, such an application would be intuitive, easy-to-use, and have great expressive power. 
    
    \item[-] \emph{Controllers:} Another category of applications that can be built using this sensor is that of controllers. This sensor allows the emulation of existing controller functionality, but with the added flexibility of fabric. For example, users can control their phone functionalities such as accepting or rejecting a phone call, changing the music, and more, through knitted interactive areas in their clothes, capable of recognizing gestures. Home automation is another potential application area, with such sensors integrated into furniture, pillows, or blankets, giving controllers a softer, more tactile-friendly quality. Gaming could benefit from application where knitted sensors are used as portable, foldable, and lightweight alternatives to hard-electronic controllers. Additionally, pressure sensitivity, an aspect of these sensors only explored in a limited way so far~\cite{Vallett2016a,Vallett2019a,mcdonald2020knitted}, could offer new interactive modalities for gaming and other applications.
    
    \item[-] \emph{Gestures in 3D Space:} Gestures do not need to be confined into a 2D plane. Knitted fabric can flex, fold, stretch, and move dynamically. Instead of only considering those fundamental aspects of fabric behaviour as qualities to design out, in order to maintain stability of touch or gesture representation, for certain applications we can also choose to design with these qualities at the center. For example, stretching could be given a specific meaning in a interactive system, and so can folding the fabric, twisting, or pinching it. An important aspect that needs to be considered while designing for such use cases however, is that capacitive sensing, the sensing strategy on which these sensors are designed, requires the presence of two conductors in proximity. In the typical cases, also explored in this work, the two conductors are the conductive yarn and human skin. 

    \item[-] \emph{Virtual and Augmented Reality (VR/AR) Applications:} Knitted sensors with gesture recognition technology integrated into them can be useful for VR/AR environments. It is easy to imagine objects that can be designed somewhat generically using knitted fabric with interactive gesture-sensing areas, with their functionality depending on the specific VR/AR application. This sensor's construction process allows for scalable design, resulting in interactive shapes capable of being built in different sizes. Therefore a whole environment could be composed of soft, knitted interactive objects, which take a different meaning and functionality, depending on the application and visuals overlayed on them. Additionally, such applications could be especially useful for kids, since they offer more safety than potential hard-electronic equivalents.
\end{itemize}

\section{Limitations and Future Work}  

The results from experiments with our sensing and recognition system demonstrate progress towards developing interactive textile gesture recognition systems. However, there are still several areas that warrant further investigation, some of which we discuss below. 

\subsection{User Studies for Increasing Model Capacity and Usability}
Gesture recognition accuracy still needs to be increased and more gestures need to be included. Even though each application might require its custom-defined gestures, in order to extend this system, it is necessary to ensure its feasibility and robustness with more subjects and more gesture types. Data collected using different fingers for performing gestures, different finger-placements on the sensing pad, different orientations of gesture trajectory, as well as subjects of different ages will need to be explored, due to possible changes in physiology which affect conductivity. Usability studies need to be performed to explore the potential of gesture-recognizing knitted sensors to be incorporated into end-users' everyday lives. Some potential applications were discussed in Section~\ref{sec:application_potential}. Future work will focus on further investigating possible uses of this technology, as well as building specific applications, testing their performance and usability in real-world scenarios, and getting user feedback about design aspects. 
\subsection{Resistance to Real-World Conditions}
The physical durability of the sensor should be more closely examined, since such sensors need to be able to withstand exposure to different weather and environmental conditions in everyday life. Experiments are needed to test the ability of the carbon-suffused nylon yarn to resists material aging and abrasion. Methods of surface enhancement and preservation, such as coating and lamination, should be investigated as potential solutions, and their effectiveness needs to be quantified. Additionally, the robustness of the trained models needs to be evaluated under conditions of possible distortion. Prior work~\cite{mcdonald2020knitted} investigated model stability for touch location identification under conditions of stretching and exposure to electromagnetic radiation for knitted sensors constructed using the same manufacturing process and carbon-coated yarn. This work investigated the effects of washing and drying on the sensor in Section~\ref{sec:washing_drying}. Further tests are necessary since those studies were limited in the number of people, as well as types of conditions. 

Another aspect to be considered is sensor performance while being worn, integrated into clothing, even though our limited study in Section~\ref{sec:wearing_sensor} demonstrated its robustness through encouraging results. Moving while wearing this sensor is expected to produce little to no distortion, and the study above included some light motion, as gestures were collected while the sensor was being worn. Despite this, more studies are necessary to explore the effect of intense physical activity, especially since such sensors are expected to find applications in athletics. Additionally, sweat, could possibly interfere with conductivity, since it contains electrolytes. If it seeps through the sensor, the overall resistance of the sensor will change, which is expected to affect the quality of acquired gesture data. Moreover, if a user is grounding himself or herself while touching a location on the sensor, his/her conductivity is increased, which will again affect the sensor response. False positives could be induced if a conductor came into contact with the sensing areas of the knitted component. Further studies are needed to fully investigate the extent of the effect of such conditions. 


\subsection{Unexplored Interaction Modalities}
Other modalities of interaction enabled by the sensor introduced above, such as pressure sensitivity should be studied and implemented. For example, soft touch, which would cause a weak applied capacitance, could potentially pose a problem towards accurate gesture recognition. On the other hand, applications can be developed that use pressure differentiation as a feature. Additionally, current sensors do not recognize multi-touch, so only one gesture can be performed at a time on the knitted sensor. From experiments conducted with sensors built using a single conductive yarn and two connections, we know that if two locations are touched simultaneously, the generated signal appears to be coming from a point in between the two contact points~\cite{Vallett2019a}. In addition to these areas, we also plan to experiment with conductive yarns of different properties. The resistance of the yarn affects conductivity, and future models should also account for that quality.  

\section{Conclusion}\label{sec:conclusion}
In this work, we focus on addressing the fundamental challenge of OOD detection tasks, which is how to fully understand the semantic discrepancy between the ID/OOD samples. We reveal that the key to success in the realistic SCOOD task is to allocate as many ID samples in the unlabeled set correctly as possible. To this end, we propose a novel uncertainty-aware optimal transport scheme that introduces class-specific energy scores as guidance for effective label assignment. Experimental results show that our method achieves better performance than previous state-of-the-art methods on SCOOD benchmarks.

\textbf{Limitations.} In addition to temperature scaling, other techniques such as feature clipping applied in ReAct~\cite{sun2021react} also enhance the performance of energy score, so how to obtain an OOD score that best fits the SCOOD task can be further explored. Moreover, a setting highly related to SCOOD has been proposed in \cite{katz2022training} and formulated as a constrained optimization problem. We will also theoretically analyze these practical OOD settings in our feature work.

% \section*{Acknowledgments}
\textbf{Acknowledgments.} 
This work is supported by National Key R\&D Program of China under Grant 2020AAA0105701, National Natural Science Foundation of China (NSFC) under Grants 61872327, Major Special Science and Technology Project of Anhui, National Natural Science Foundation of China (62033012) and Ant Group through Ant Research Intern Program.


%%
%% The acknowledgments section is defined using the "acks" environment
%% (and NOT an unnumbered section). This ensures the proper
%% identification of the section in the article metadata, and the
%% consistent spelling of the heading.
\begin{acks}
We thank the members of Drexel University’s Pennsylvania Fabric Discovery Center at the Center for Functional Fabrics for their invaluable digital knitting expertise and supervision of student volunteers. This research is supported in part by the Pennsylvania Fabric Discovery Center and the US Army Manufacturing Technology Program (US Army DEVCOM) under Agreement number W15QKN-16-3-0001.
\end{acks}

%%
%% The next two lines define the bibliography style to be used, and
%% the bibliography file.
\bibliographystyle{ACM-Reference-Format}
\bibliography{textiles_ref,methods}

%%
%% If your work has an appendix, this is the place to put it.
\appendix


\end{document}
\endinput
%%
%% End of file `src/main.tex'.



 


 