\section{Conclusion}

In this work, we create the foundations for building an interactive gesture recognition system using an easily-manufactured capacitive weft-knitted sensor, and a classification model that recognizes 12 English language characters as gestures. Additionally, we deploy the model on NVIDIA\textregistered~Jetson Xavier\TM~NX, a small, lightweight, and powerful embedded system-on-module.  

Previous work has shown several designs of similarly-produced sensors, which use weft knitting, an industry-standard manufacturing technique, for easy integration into clothes and other textile surfaces. Sensing areas of different shapes can be designed through this process, allowing for application-specific form factors. The sensor introduced in this work uses a single carbon-coated conductive yarn to create a continuous sensing area. That yarn is combined in the knitting process with other regular yarns to produce the final fabric substrate. Four electrodes are attached to the corners of the rectangular conductive area to measure voltage across it. Different gestures produce different voltage responses from the four connection points. 

In order to capture those identifying differences, we train and evaluate a CNN-LSTM neural network model with data from a total of 8 subjects. Leave-one-out cross-validation was used for training on 5 subjects, achieving a subject-independent accuracy of 78.6\%, and the data from the 3 other subjects was used for further evaluation of the model's generizability, during which, the model achieved a subject-independent accuracy of 89.8\%. Finally, we tested the response time of the trained model, to gesture data, after the model had been deployed on the Jetson system. The average response time was 30 ms, which is very encouraging to building a real-time application. These technical contributions help us advance toward feasible interactive embedded systems, capable of recognizing gestures on knitted senors.

Subsequently, we ran a user study to explore the model's accuracy response with gesture data collected while the sensor was being worn. Another experiment consisted of measuring the sensor's resistance across all connection points, after washing and drying it. These considerations, together with our discussion of the components and processes necessary to build real-time applications upon this technology help advance the capabilities of knitted textiles toward their use in real-world environments. 

