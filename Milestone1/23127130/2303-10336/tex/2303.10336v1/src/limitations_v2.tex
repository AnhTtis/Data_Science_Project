\section{Limitations and Future Work}  

The results from experiments with our sensing and recognition system demonstrate progress towards developing interactive textile gesture recognition systems. However, there are still several areas that warrant further investigation, some of which we discuss below. 

\subsection{User Studies for Increasing Model Capacity and Usability}
Gesture recognition accuracy still needs to be increased and more gestures need to be included. Even though each application might require its custom-defined gestures, in order to extend this system, it is necessary to ensure its feasibility and robustness with more subjects and more gesture types. Data collected using different fingers for performing gestures, different finger-placements on the sensing pad, different orientations of gesture trajectory, as well as subjects of different ages will need to be explored, due to possible changes in physiology which affect conductivity. Usability studies need to be performed to explore the potential of gesture-recognizing knitted sensors to be incorporated into end-users' everyday lives. Some potential applications were discussed in Section~\ref{sec:application_potential}. Future work will focus on further investigating possible uses of this technology, as well as building specific applications, testing their performance and usability in real-world scenarios, and getting user feedback about design aspects. 
\subsection{Resistance to Real-World Conditions}
The physical durability of the sensor should be more closely examined, since such sensors need to be able to withstand exposure to different weather and environmental conditions in everyday life. Experiments are needed to test the ability of the carbon-suffused nylon yarn to resists material aging and abrasion. Methods of surface enhancement and preservation, such as coating and lamination, should be investigated as potential solutions, and their effectiveness needs to be quantified. Additionally, the robustness of the trained models needs to be evaluated under conditions of possible distortion. Prior work~\cite{mcdonald2020knitted} investigated model stability for touch location identification under conditions of stretching and exposure to electromagnetic radiation for knitted sensors constructed using the same manufacturing process and carbon-coated yarn. This work investigated the effects of washing and drying on the sensor in Section~\ref{sec:washing_drying}. Further tests are necessary since those studies were limited in the number of people, as well as types of conditions. 

Another aspect to be considered is sensor performance while being worn, integrated into clothing, even though our limited study in Section~\ref{sec:wearing_sensor} demonstrated its robustness through encouraging results. Moving while wearing this sensor is expected to produce little to no distortion, and the study above included some light motion, as gestures were collected while the sensor was being worn. Despite this, more studies are necessary to explore the effect of intense physical activity, especially since such sensors are expected to find applications in athletics. Additionally, sweat, could possibly interfere with conductivity, since it contains electrolytes. If it seeps through the sensor, the overall resistance of the sensor will change, which is expected to affect the quality of acquired gesture data. Moreover, if a user is grounding himself or herself while touching a location on the sensor, his/her conductivity is increased, which will again affect the sensor response. False positives could be induced if a conductor came into contact with the sensing areas of the knitted component. Further studies are needed to fully investigate the extent of the effect of such conditions. 


\subsection{Unexplored Interaction Modalities}
Other modalities of interaction enabled by the sensor introduced above, such as pressure sensitivity should be studied and implemented. For example, soft touch, which would cause a weak applied capacitance, could potentially pose a problem towards accurate gesture recognition. On the other hand, applications can be developed that use pressure differentiation as a feature. Additionally, current sensors do not recognize multi-touch, so only one gesture can be performed at a time on the knitted sensor. From experiments conducted with sensors built using a single conductive yarn and two connections, we know that if two locations are touched simultaneously, the generated signal appears to be coming from a point in between the two contact points~\cite{Vallett2019a}. In addition to these areas, we also plan to experiment with conductive yarns of different properties. The resistance of the yarn affects conductivity, and future models should also account for that quality.  
