\section{Related Work}
This section starts by providing some context regarding fabric touch sensors produced using different fabric construction or embellishment techniques, as well as different sensing methods, focusing on capacitive sensing upon which this system is built. Then, the discussion transitions to work that has focused on gesture recognition on fabric sensors, which have been produced through various techniques. Subsequently, a discussion of the neural network architectures upon which our model relies, provides an overview of their working and the reasons for our choice. 


\subsection{Smart Textile Construction Processes}

\begin{figure*}[h]
    \centering
    \subfigure[]
    {\includegraphics[width=0.18\textwidth]{src/figures/Embroidered_Sensor.pdf}\label{fig:Embroidered_Sensor}}
    \hfill
    \subfigure[]
    {\includegraphics[width=0.18\textwidth]{src/figures/Capacitive_Sensing_Matrix.pdf}\label{fig:Capacitive_Sensing_Matrix}}
    \hfill
    \subfigure[]
    {\includegraphics[width=0.18\textwidth]{src/figures/Reflectometry.pdf}\label{fig:Reflectometry}}
    \hfill
    \subfigure[]
    {\includegraphics[width=0.18\textwidth]{src/figures/Knitted_Sensor.pdf}\label{fig:Knitted_Sensor}}
    \hfill
    \caption{Illustrations of sensing techniques and electrode layouts used in textile touch sensors. \emph{(a)}: Embroidered capacitive buttons. \emph{(b)}: Woven or embroidered capacitive sensing matrix. \emph{(c)}: \emph{Reflectometry} measured using two parallel conductors. \emph{(d)}: Single weft knitted conductor measured using \emph{differential capacitive sensing}.}
    \label{fig:Designs}
    
    \Description{The four photos, (a), (b), (c), and (d) show different sensing strategies used for fabric sensors. On each figure, there is an illustrated hand touching on an area of the illustrated fabric sensor. In (a), there are nine buttons shown, from 1-9, with all of them colored black, except for number 2, which is red, illustrating the button being touched. Figure (b) shows a matrix of vertical and horizontal lines. There is a vertical blue line and red horizontal line intersecting on the point of touch. In (c), there are two parallel lines forming a shape together. One of the lines is red up to the point of touch on the fabric, and black after it; the other line is completely black. Figure (d) shows a serpentine line, where the top part of the line, which is before the point of touch, is blue, while the bottom one, after the point of touch is red.}
\end{figure*}

Textile assembly or embellishing methods, such as embroidery, weaving, and knitting have been used to create touch-sensitive textile devices using conductive yarn. \emph{Embroidery} is a textile embellishing method that inserts thread into a fabric substrate to create arbitrary patterns (Figure~\ref{fig:Embroidered_Sensor}), and has been used to create several interactive textile devices~\cite{Post2000a,Gilliland2010a,Hamdan2016a,Hamdan2018a,aigner2020embroidered,mlakar2020design}. Embroidered threads are not part of the original textile, therefore they can be overlaid to form a variety of sensing shapes on any textile, if conductive yarn in used. Large-area distributed touch sensing can also be achieved through integrating conductive yarns into the textile structure during formation, eliminating post processing. {\em Weaving} and {\em knitting} are textile manufacturing processes that can combine conductive and non-conductive yarns to form fabric circuits.

In {\em weaving}, a textile is formed by interlacing perpendicular yarns~(Figure~\ref{fig:Capacitive_Sensing_Matrix}). It is a widely-used method to produce distributed textile touch sensors~\cite{DeRossi2002a,Hasegawa2007a,Takamatsu2011a,Agcayazi2016a,Poupyrev2016a,wu2020zebrasense} due to its similarity to standard capacitive and resistive sensing matrices. Woven textiles use the perpendicular alignment of numerous horizontal and vertical yarns to form a sensing grid and detect touch at yarn intersection points, supporting a high-resolution sensing area. However, connecting a matrix to external sensing electronics requires an equal number of fabric-to-wire connections---a difficult and time-consuming post production process. These connections often exist at millimeter-scales and utilize techniques like soldering, which create brittle joints that reduce the textile's durability, and potentially real-world practical use. 

\emph{Knitting}, the technique used to construct our sensor, produces fabric by employing a continuous yarn or set of yarns to form a series of interlocking loops. Knitting has been recently explored in producing interactive textiles, such as force-sensing knitted structures~\cite{Pointner2020KnitedRESi}, knitted electronic musical controllers~\cite{Wicaksono2020KnittedKeyboard}, and knitted dynamic displays~\cite{devendorf2016don}. Other research has investigated creating 2D sensing structures using a homogeneous~\cite{Vallett2016a,Vallett2019a} or multi-material~\cite{ozbek2018novel} 1D filament. In work more closely related to ours, digital weft-knitting has been explored to construct textiles with only two connection points~\cite{Vallett2016a,Vallett2019a,mcdonald2020knitted}. We select this process since it uses continuous yarns in the production of textiles, which allows the construction of a fabric circuit through an electrically continuous trace. The conductive element in this work is carbon-coated nylon yarn, which becomes intertwined with regular, non-conductive yarns according to a programmatically-defined pattern of textile construction. The circuit design is compatible with the way yarns are routed during the machine knitting process, and does not require post-production assembly, besides attaching wires at the connection points. This process allows for different textile patterns to be produced according to application specifications, as well as similar form factors of different scales.  

\subsection{Fabric-based Capacitive Touch Sensing}
Capacitive touch sensing is a widely utilized method of interacting with electronic devices. Capacitive touch sensors measure the presence or proximity of a conductor, such as human skin, through its electromagnetic properties. These sensors can discern nuanced touch and gesture and often require no mechanical deflection of a sensing medium. This mode of interaction is advantageous for use in touch-sensitive textiles, which need a dynamic circuit structure. Alternative methods of distributed touch detection have been pursued in textiles, such as \emph{inter-yarn contact sensing}~\cite{Inaba1996a, karrer2011pinstripe} and \emph{resistive sensing}~\cite{sundholm2014smart, Parzer2018a, pointner2020knitted, honnet2020polysense, freire2017second}.

Textile-based capacitive touch sensing circuits can be categorized as either \emph{discrete} or \emph{continuous}. Discrete touch sensing relies on numerous electrically isolated conductors distributed across a sensitive surface~\cite{Post2000a,Poupyrev2016a,wu2020capacitivo}. The bulk magnitude of capacitance is measured on each conductor, and touch is localized using knowledge of each conductor’s location. Improving the accuracy and resolution of touch localization relies on increasing the density of conductors within a given area. This is often done at the cost of increasing the circuit’s structural complexity and increasing the number of connections from the circuit to sensing hardware. However, the sensing electronics required to scan the discrete electrodes are simpler than what is required in continuous sensing. Textile manufacturing techniques like weaving (Figure~\ref{fig:Capacitive_Sensing_Matrix}) are often used to create capacitive sensing matrices, while standalone buttons can be achieved using textile embellishment methods like embroidery (Figure~\ref{fig:Embroidered_Sensor}).

\begin{figure*}[ht]
    \centering
    \subfigure[]
    {\includegraphics[width=.49\textwidth]{src/figures/Differential_Capacitive_Sensing_Circuit_Diagram.pdf}\label{fig:DCS_Circuit_Diagram}}
    \hfill
    \subfigure[]
    {\includegraphics[width=.49\textwidth]{src/figures/Differential_Capacitive_Sensing_Waveform_Output.pdf}\label{fig:DCS_Waveform_Output}}
    \hfill
    \caption{Illustration of the differential capacitive touch sensing circuit and example waveform output. \emph{(a)}: Circuit diagram visualizing the current differential between the measurement points (A and B) and the location of touch. \emph{(b)}: Example waveform measurements showing the attenuation in voltage (gain A, B) based on the touch location and capacitance magnitude shown in \emph{(a)}. The gains are found as the amplitude ratio of the input and output waveforms using \emph{Bode analysis}.}
    \label{fig:Differential_Capacitive_Sensing}
    
    \Description{The two figures, (a) and (b) show a circuit with two electrode connections, and a graph with three sinusoidal lines. The circuit in figure (a) shows current path on each electrode of the circuit A, in red, and B, in blue. An illustrated hand demonstrates a touch event, upon which the capacitor is shown being discharged in purple. In (b), we see a graph whose X axis shows time (0-200 microseconds), while Y axis shows the voltage values. There are three lines, which represent the input waveform, the output waveform measured at electrode A, and the output waveform measured at electrode B. The gain is measured as the amplitude ratio between the input and measured outputs. Corresponding to the touch location in (a), the waveform from electrode A has a higher value at the peaks compared to electrode B.}
\end{figure*}


In contrast, continuous touch sensing infers the location of touch across a single electrically continuous conductor. Continuous touch sensing exploits the properties of the conductor to measure changes in capacitance along its length, which are localized using knowledge of the conductor’s overall path. The geometry of the conductor may facilitate localization, such as creating textile-based transmission lines for use with Time- and Frequency-domain Reflectometry (TDR and FDR)~\cite{Wimmer2011a,Hughes2014a,Hughes2017a}, as illustrated in Figure~\ref{fig:Reflectometry}, or other more recent methods~\cite{ku2020threadsense}. Other sensing methods like Electric Field Tomography (EFT) measure capacitance along a pattern-less conductive surface using measurements along the periphery~\cite{Zhang2017a}. These methods can localize touch using relatively simple circuits; however, the accuracy and resolution of touch location relies on complex measurement hardware to infer accurate location from recorded data. 

 Previous work has used \emph{differential capacitive sensing}, a continuous capacitive sensing strategy, in conjunction with a weft-knitted circuit composed of a single conductive yarn to localize capacitive touch~\cite{Vallett2016a,Vallett2019a,mcdonald2020knitted}. The circuit design of the sensor in this work is based on that same method. As shown in Figure~\ref{fig:DCS_Circuit_Diagram}, touch location is inferred along a linear conductive pathway via a current differential measured at each endpoint. The measured voltage is visualized in Figure~\ref{fig:DCS_Waveform_Output}, which illustrates the attenuation of voltage based on the location and magnitude of capacitive touch. The waveform gain is determined via \emph{Bode analysis}~\cite{Bode1940a}, which determines the amplitude ratio of the input and output waveforms using Fast Fourier Transforms. Initial applications focused on creating user interfaces capable of measuring simple tap location. Our work extends the structure and capabilities of the sensing circuit by using a planar conductive area to detect continuous gestures.  

\subsection{Machine Learning Applications for Fabric-based Touch Sensing}
While textile touch sensors can provide basic touch location, machine learning algorithms have been used to enable higher-level, accurate input recognition. \emph{Project Jacquard}~\cite{Poupyrev2016a} introduces a platform to produce interactive woven sensors. In its user evaluation study, gestures such as \emph{swipe left}, \emph{swipe right}, and \emph {hold} were recognized under three different condition: \emph{sitting}, \emph{standing}, and \emph{walking}. The work from Hughes et al.~\cite{Hughes2014a,Hughes2017a} describes the construction of an RF-based e-Textile and its ability to distinguish between gestures of \emph{tap}, \emph{up swipe}, and \emph{down swipe} based on a CNN model. \emph{SmartSleeve}~\cite{Parzer2017a}, a deformable, pressure-sensing textile sensor recognizes in real time both surface and deformation gestures. It relies on an algorithms which, after converting the data in image form, uses an SVM-trained model together with heuristics to detect the gesture type. \emph{GestureSleeve}~\cite{Schneegass2016a} is a touch-sensing textile sleeve, able to detect stroke-based gestures or taps through the \$P algorithm~\cite{vatavu2012gestures}. \emph{Smart-mat}~\cite{sundholm2014smart} uses a kNN classifier on features built using time series statistical information. \emph{RESi}~\cite{Parzer2018a}, a resistive, touch sensing interface uses SVM classification to detect basic gestures such as single, double and triple taps, as well as swipes in the four directions. More recent work~\cite{olwal2020textile} furthers cord-based interfaces by enabling continuous control as well as recognizing casual discrete gestures.

Our work focuses on detecting gestures that are relatively complex compared to most of these examples. Additionally, all the sensors described above, mainly focused on gesture recognition, differ from our work in construction and sensing strategy. In many of these cases and other similar sensors, much of the complexity lies on the hardware design. In this work however, the construction is kept simple and streamlined, while shifting the burden of complexity to computational models, similarly to McDonald et al.~\cite{mcdonald2020knitted}. In that work, an LSTM neural network with features constructed on statistical information from voltage gain values was used to extract location of touch from a single-yarn and two-connection capacitive knitted sensor. Extending the functionality of that work, we introduce a capacitive knitted sensor, which is also composed of a single yarn, but uses four external connections to recognize gestures. Below, we describe the machine learning concepts upon which our model is built. 

\subsection{CNNs and LSTMs}
To recognize gestures performed on the knitted sensor, we use a neural network architecture which combines Convolutional Neural Networks (CNN)~\cite{fukushima1980neocognitron,lecun1998gradient} with Long Short-Term Memory networks (LSTM)~\cite{gers1999learning}, a stable version of Recurrent Neural Networks (RNNs)~\cite{rumelhart1988learning}. The time-series data, acquired through the four connections in the sensor we design, needs to be processed to map to particular gestures. Since this is time series data, the classification algorithm should take advantage of the continuity of sampled frequencies over time, in order to properly model the necessary connections. Moreover, due to variability in the data, such as different users, conditions, finger placement and more, the same gestures should be represented in terms of essential patterns within them. To those ends, we utilize CNNs and LSTMs, which have found application for many classification tasks. 

CNNs are relatively sparse, regularized neural networks, which have translation-invariant characteristics. They aim to capture hierarchical patterns in data, composing higher-level features from lower-level ones. CNNs have been successfully used, among other areas, in image and video recognition tasks, natural language processing, and recommender systems. They capture the spatial relationships between samples. RNNs allow previous outputs to be utilized as inputs, in addition to hidden states. This flexibility makes them ideal for classification of inputs of any length, while maintaining limits on model size, taking into account historical information, and sharing weights across time. A known drawback of RNN models is their difficulty in capturing long term dependencies due to multiplicative gradients. The error of such models can be exponentially decreasing/increasing with respect to the number of layers. The Gated Recurrent Unit (GRU)~\cite{cho2014learning} and their generalizations, such as Long Short-Term Memory units (LSTM)~\cite{gers1999learning} address the vanishing gradient problem encountered by traditional RNNs. We use LSTMs to model the temporal dependencies in our data. More information is provided regarding our network architecture in the section below, as part of the gesture recognition system.  

