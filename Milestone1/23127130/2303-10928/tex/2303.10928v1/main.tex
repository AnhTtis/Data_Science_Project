\documentclass[twocolumn,prl,showpacs,preprintnumbers]{revtex4-2}
\usepackage{graphicx,latexsym,amssymb,amsmath,color,multirow,mathrsfs,booktabs,ifpdf,epsfig,dcolumn,bm,longtable,multirow}
\usepackage{changes}
\usepackage{hyperref}
\usepackage{soul} 

\hypersetup{colorlinks=true,linkcolor=blue,filecolor=magenta,urlcolor=cyan}
\date{\today}

\begin{document}

\title{Octupole deformation instability in atomic nuclei}

 	\author{Bui Minh Loc}
	\email{buiminhloc@ibs.re.kr}
	\affiliation{Center for Exotic Nuclear Studies, Institute for Basic Science (IBS), Daejeon 34126, Korea.}
       \author{Nguyen Le Anh}
	\email{anhnl@hcmue.edu.vn}
	\affiliation{Department of Physics, Ho Chi Minh City University of Education, 280 An Duong Vuong, District 5, Ho Chi Minh City, Vietnam.}
        \affiliation{Department of Theoretical Physics, Faculty of Physics and Engineering Physics, University of Science, Ho Chi Minh City, Vietnam}
	\affiliation{Vietnam National University, Ho Chi Minh City, Vietnam.}
         \author{Panagiota Papakonstantinou}
	\email{ppapakon@ibs.re.kr}
	\affiliation{Rare Isotope Science Project, Institute for Basic Science, Daejeon 34047, Korea}
 	\author{Naftali Auerbach}
	\email{auerbach@tauex.tau.ac.il}
	\affiliation{School of Physics and Astronomy, Tel Aviv University, Tel Aviv 69978, Israel}

\begin{abstract}
Recent high-energy heavy-ion collision experiments have revealed that some atomic nuclei exhibit unusual softness and significant shape fluctuations. In this Letter, we use the fully self-consistent mean-field theory to identify all even-even nuclei that are unstable or soft against octupole deformation. All exceptional cases of enhanced octupole transition strengths in stable even-even nuclei throughout the nuclide chart are resolved. These results represent a significant advance in our understanding of the underlying mechanisms of nuclear octupole deformation and have implications for further experimental and theoretical studies.
\end{abstract}

\keywords{Octupole excitation, Random Phase Approximation, Skyrme force, pairing, single-particle level}

\maketitle

\textbf{Introduction.}
The impact of nuclear deformation on the elliptic flow measured in relativistic heavy-ion collisions was emphasized and formalized in Ref.~\cite{Giacalone2021}. Subsequently, the STAR Collaboration reported in Ref.~\cite{Zhang2022} the first direct evidence of strong octupole correlation in the ground-state $^{96}$Zr in heavy-ion collisions at very high energy. The question is whether the deformation values in low-energy literature are consistent with these new data at high energy. 

The existence of low-lying octupole ($3^-$) vibrations in atomic nuclei, at excitation energies of only a few MeV, was recognized early on~\cite{LANE196039}. 
They are now among the best-established collective states in nuclear physics and have been observed in almost all even-even stable nuclei throughout the nuclide chart. 
A review of octupole collectivity can be found in Ref.~\cite{Butler2016}, while Ref.~\cite{Butler1996} offers a review of nuclear reflection asymmetry in general. 
Experimental information on the first $3^-$ states of even-even nuclides is compiled in Refs.~\cite{Spear1989, Kibedi2002}. Maximal strength values for the transition from the ground state to the first $3^-$ state are observed at neutron numbers $N = 34, 56, 88$, and $134$, and atomic (proton) numbers $Z = 30, 40, 62$, and $88$ \cite{Spear1990}. The term ``octupole-magic" number has been used to refer to them~\cite{Butler1996}.

As already indicated by the theoretical interpretation in Ref.~\cite{LANE196039}, low-lying $3^-$ collectivity in atomic nuclei is driven by the presence of pairs of single-particle states with opposite parity, whose angular momenta differ by 3, in the vicinity and on both sides of the Fermi energy. The presence of such pairs with energy differences much lower than the characteristic shell energy of 1$\hbar\omega$ becomes possible thanks to spin-orbit splitting. Enhanced collectivity at lower energies can be expected when the energy difference between such states becomes especially small as a result of the interplay of the spin-orbit coupling strength and the other nuclear interaction terms. Such is the case, for example, of the $2d_{5/2}-1h_{11/2}$ neutron particle-hole pair in $^{96}$Zr. Octupole-magic numbers can be explained in such a way. 

Global theoretical analyses of these excitations in even-even nuclei were given in Refs.~\cite{Robledo2011, Agbemava2016}.
It has been reported that to have a reliable way of understanding the character and making predictions about octupole states, one has to go beyond the mean-field approach \cite{Yao2015, Agbemava2016, Barnard2016, Li2016, Nomura2021, Nomura2021_2}. 
However, several questions remain open even then. 
Over time, the exceptional characteristic of some octupole transitions has been revealed experimentally without a satisfactory explanation. The nucleus $^{96}$Zr with $N = 56$ is quite irregular and the various theoretical attempts to study its octupole excitation are inconsistent with the experiment and with one another, as was recently discussed in Ref.~\cite{ISKRA2019396}. 
Very recently, Ref.~\cite{Spieker2022} reported the enhancement in the light atomic nucleus $^{72}$Se with $Z=34$. 

The irregular behavior of the energy of the low-lying $3^-$ in $^{96}$Zr was first found and discussed in Ref.~\cite{Abbas1981}. 
Examining the single-particle spectrum, one observed that the $2d_{5/2}$-neutron state is fully occupied and there is close opposite parity $1h_{11/2}$ state unoccupied. The excitation of each neutron from $2d_{5/2}$ to $1h_{11/2}$, therefore, may form the low-lying $3^-$ state. In contrast, $^{96}$Ru has only $2$ neutrons at the $2d_{5/2}$ and therefore fewer particle-hole states to form the low-lying $3^-$ state, resulting in less collectivity compared to $^{96}$Zr. The presence of this particle-hole pair enhances collectivity and pushes the $3^-$ state to lower energy.

As a collective vibration, the octupole vibration can be studied within the self-consistent mean-field approach, specifically, the self-consistent random-phase approximation (RPA), which is derived as the linearized limit of the time-dependent Hartree-Fock method~\cite{Co2023}. 
This was the approach followed early on in Ref.~\cite{Blaizot1976}, where the importance of self-consistency was emphasized. 
Self-consistent RPA is a unique tool not only for accurately describing collective vibrations and connecting their properties to an underlying energy density functional or effective interaction,  
but also for diagnosing instabilities and broken symmetries: For example, if the self-consistent RPA equations are solved for nuclei assuming a spherical ground state, 
the presence of imaginary solutions  in, {\em e.g.}, the quadrupole or octupole channel will indicate that the ground state is, in fact, quadrupole- or octupole-deformed, respectively~\cite{Thouless1960, Thouless1961, Rowe1968}. 
The spherical ground state is said to ``collapse." 
Similarly, softness towards shape fluctuations can be expected when there are collective vibrations at very low energy.

In this Letter, the instability or softness of certain even-even nuclei against the variations in the octupole collective variable is indicated throughout the nuclide chart using the self-consistent mean-field framework. The enhanced collectivity of octupole excitation around octupole-magic nuclei is explained and reproduced.

\textbf{Method.}
The self-consistent mean-field theory is a powerful tool for studying the properties of atomic nuclei, such as their shapes, energies, and excitation spectra~\cite{Bender2003, RS80, Co2023}. The theoretical foundation of the mean field is provided by the Hartree-Fock (HF) theory using the Skyrme interaction, which is one of the most commonly used types of effective interaction. 
RPA is used within the framework of the self-consistent mean-field theory to describe collective motion in atomic nuclei and especially harmonic vibrations.
Therefore, this framework is highly suitable for the study of the first $3^-$ octupole state which is low-lying and strongly collective. Self-consistency is ensured when the RPA particle-hole interaction is derived from the same effective interaction used for obtaining the HF ground state. 

Computational codes exist for solving the HF and RPA equations and they may include a pairing interaction for describing open-shell nuclei. 
The publically available computational code described in Ref.~\cite{Colo2013}, allows both theorists and experimentalists to perform computations on a wide range of nuclear excitations using fully self-consistent Skyrme HF-RPA, assuming spherical symmetry. 
All relevant terms of the residual particle-hole interaction are incorporated into the calculation, including the Coulomb and spin-orbit terms, the latter being especially relevant for the octupole vibration, as already discussed. 
With pairing correlation included, HF is extended to HF-BCS and RPA to Quasiparticle RPA (QRPA) accordingly \cite{Colo2021} with the self-consistency remaining. 
We use the above  approaches in the present study.
The pairing correlation is included in our calculation. The selected Skyrme forces in the present work are SkM* \cite{BARTEL198279}, SLy4, and SLy5 \cite{CHABANAT1998231} 
and, for comparison, the SIII, which was one of the first parameterizations.

The reduced transition strength from the correlated RPA ground state $| \tilde{0} \rangle$ to the first octupole state $3^-_1$ is 
\begin{equation}
B(E3) = |\langle 3^-_1|| \hat{F}_3 || \tilde{0} \rangle|^2,
\end{equation}
where the isoscalar octupole operator is
\begin{equation}
    \hat{F}_{3,0}(\bm r) = \sum_i^A r_i^3 Y_{3,0} (\hat{r}_i).
\end{equation}
All $B(E3)$ are presented in Weisskopf units (W.u.) with $1~\text{W.u.} = 0.0594~A^2 e^2 \text{fm}^6$.

The degree of stability or softness of the atomic nucleus against variations is expressed via the polarizability $\alpha$, which is obtained from the inverse energy-weighted sum rule of the response function~\cite{RS80, Abbas1981}.
In the RPA, it is calculated from the $m_{-1}$ moment
\begin{eqnarray}
    \alpha &=& 2m_{-1}/A, \\
    m_{-1} &=& \sum_n |\langle 3^-_n || \hat{F}_3|| \tilde{0} \rangle |^2 E_n^{-1}.
\end{eqnarray}
The larger $\alpha$ (W.u./MeV) is, the less stable the system is against deformations. For example, it is around $1$ W.u./MeV for $^{208}$Pb, which is not considered especially soft. 

We have performed calculations for all stable even-even nuclei and we will discuss a selection of them below.  
We will compare our results with experimental values from the NuDat database (\href{https://www.nndc.bnl.gov/nudat/}{https://www.nndc.bnl.gov/nudat/}) for the energies and from Table VII of Ref.~\cite{Kibedi2002} for $B(E3)$.

\textbf{Results and Discussions.}
\begin{table}[t]
    \centering
    \setlength{\tabcolsep}{6pt}
    \renewcommand{\arraystretch}{1.2}
    \caption{The HFBCS-QRPA results for $^{96}$Zr are irregular in contrast to the results of $^{96}$Ru. The values of $E(3_1^-)$, $B(E3)$, and $\alpha$ are in MeV, W.u., and W.u./MeV, respectively.}
    \begin{tabular}{|l|rrr|rrr|}
    \hline \hline 
\multirow{2}{*}{Force}	&	\multicolumn{3}{c|}{$^{96}$Zr} &	\multicolumn{3}{c|}{$^{96}$Ru}	\\ \cline{2-7}
                         & $E$($3^-_1$) & $B(E3)$ & $\alpha$ & $E$($3^-_1$) & $B(E3)$ &  $\alpha$ \\
\hline
SIII	&	\multicolumn{3}{c|}{collapse} &	$2.437$		&	$22.4$		&	$1.2$	\\ 
SkM*	&	\multicolumn{3}{c|}{collapse} & 	$3.065$	&		$19.7$		&	$0.9$	\\
SLy4	&	$0.961$	    &	$96.4$ 	& $12.6$	& $3.875$	&		$23.1$	&		$0.8$	\\
SLy5	&	$1.698$	    & $51.9$ &	$4.0$	&	$3.737$	&		$22.3$		&	$0.8$	\\
\hline
Exp.	&	$1.897$	& 	$53$ &	&	$2.650$		&	-		&		\\
\hline \hline 
    \end{tabular}
    \label{tab:I}
\end{table}
First, we discuss the difference between $^{96}$Zr and $^{96}$Ru with respect to the octupole excitation.
The recent STAR measurement showed a significant difference between $^{96}$Zr$+$$^{96}$Zr and $^{96}$Ru$+$$^{96}$Ru collisions that were explained  using a transport simulation as an octupole deformation of $^{96}$Zr in its intrinsic reference frame and large quadrupole deformation of $^{96}$Ru~\cite{Zhang2022}. 
Our results for $^{96}$Zr and $^{96}$Ru with different Skyrme forces are presented in Table \ref{tab:I}. 
The results of $^{96}$Ru are not dramatically different from the experiment and from each other and can be considered well-understood.  
The $^{96}$Ru nucleus has a small octupole polarizability, $\alpha \approx 1$ as we find, making it hard to be octupole deformed in, for example, isobaric heavy-ion collisions. 
In contrast, the results for $^{96}$Zr are irregular.
In Ref.~\cite{Abbas1981}, the energy of the $3^-_1$ state of $^{96}$Zr was found to depend very strongly on the single-particle spectrum that is obtained self-consistently in the framework. 
When the gap between occupied and unoccupied single-particle energies of opposite parity is abnormally small, the excitation energy of the $3^-_1$ state gets dramatically low. 
For example, the gap between $2d_{5/2}$ and $1h_{11/2}$ given by the calculation with SIII force \cite{Beiner1975} is so small that the $3^-_1$ state ``collapses."

When the energy of the first $3^-$ is imaginary in PRA, we indicate in Table \ref{tab:I} that there is ``collapse." 
As we assumed a spherically symmetric ground state, the interpretation is that the respective effective interactions predict that the ground state is octupole-deformed. In reality, $^{96}$Zr has a $3^-_1$ state at $1.9$ MeV but our results with the different Skyrme interactions suggest that this nucleus is soft against octupole deformation and a small change in the calculation input can even lead to octupole instability.

The calculation with SLy5 force reproduces both the excitation energy and transition strength correctly. 
The octupole collectivity of $^{96}$Zr is $52$ W.u., and the total octupole polarizability $\alpha$ is $4$ W.u./MeV. 
The large enhancement of octupole collectivity of $^{96}$Zr observed in reactions is understood and reproduced by our calculation. The condition for softness is not unique to $^{96}$Zr as there are other nuclei with similar characteristics. Our HFBCS-QRPA calculation identifies such nuclei in the nuclide chart and well reproduces the experimental data.

In the case of $^{96}$Zr, the pair with strong octupole coupling is $2d_{5/2}-1h_{11/2}$. From the single-particle spectrum, the other pairs are $2p_{3/2}-1g_{9/2}$, $2f_{7/2}-1i_{13/2}$, and $2g_{9/2}-1j_{15/2}$. They are corresponding to the octupole-magic nuclei which are around $34, 56, 88$, and $134$ (for the neutron). In addition, the pair $2s_{1/2}-1f_{7/2}$ is also valid. Therefore, we suggest that the lightest octupole-magic number is actually $16$. Table \ref{tab:II} shows the results of selected nuclei of interest.
\begin{table}
    \centering
    \setlength{\tabcolsep}{2pt}
    \renewcommand{\arraystretch}{1.2}
    \caption{The results for selected nuclei with the number of neutrons or protons are around $16, 34, 56, 88$, and $134$ (neutron only). Enhanced octupole transitions are found from light to heavy octupole-magic nuclei, while ``collapse" may be obtained for heavy nuclei.}
    \begin{tabular}{|cl|rrr|rrr|}
    \hline\hline
	 Nuclei & Force & 	$E$($3^-_1$)  &	$B(E3)$ & $\alpha$ & $E$($3^-_1$) & $B(E3)$ & $\alpha$ \\
        &  & \multicolumn{3}{c|}{RPA} & \multicolumn{3}{c|}{QRPA} \\ \hline 
        $^{32}_{16}$S$_{16}$ & SkM*	&	$5.692$	& $15.6$ & $1.0$ &  $5.686$ & $15.7$ & $1.0$ \\
        & SLy4	&	$6.255$ & $	18.3$ & $1.0$ &  $6.248$ &	$18.4$ & $1.0$ \\
        & SLy5	&	$6.382$  & $18.8$ & $1.0$ &   $6.147$ & $	20.4$ & $1.0$ \\
        & Exp.	&	 &	&	&  $5.006$ & $30$ &	\\ \hline
        $^{64}_{30}$Zn$_{34}$ & SkM*	&	$1.959$	& $5.8$ & $1.4$ & $3.315$ & $18.5$ & $1.4$ \\
        & SLy4	&	$3.381$ & $13.3$ & $1.2$  & $4.243$ &	$24.7$ & $1.2$ \\
        & SLy5	& $3.431$	 & $13.6$ & $1.2$ & $4.265$ & $24.8$ & $1.2$  \\
        & Exp.	&	 & & 	& $2.999$ & $20$ &	
            \\ \hline
        $^{72}_{34}$Se$_{38}$ & SkM*	&	$1.068$ & $54.8$ & $6.5$ &  $1.135$  & $75.3$ & $8.4$ \\
        & SLy4	&	$2.069$ & $	44.8$ & $2.9$  & $2.406$ &	$49.3$ & $2.8$ \\
        & SLy5	& $1.931$ & $44.2$ & $3.1$ &  $2.412$ & $	47.1$  & $2.7$\\
        & Exp.	&	 & &	&  $2.406$ &	$32$	& \\ \hline
        $^{90}_{34}$Se$_{56}$ & SkM*	&	$1.542$	& $38.6$ & $4.7$ & $1.882$ &	$33.6$ & $3.5$ \\
        & SLy4	&	$1.947$	& $37.1$ & $3.6$ & $2.749$ &	$26.8$ & $2.1$ \\
        & SLy5	&	$1.879$ & $	37.7$ & $3.8$ &   $2.574$ & $29.6$ & $2.4$\\
        & Exp.	&	 &	&	& $-$ & $-$ &\\ \hline
        $^{98}_{40}$Zr$_{58}$ & SkM*	&	\multicolumn{3}{c|}{collapse}  & \multicolumn{3}{c|}{collapse}  \\
        & SLy4	&	\multicolumn{3}{c|}{collapse} & $0.435$ & $215.6$ & $64.4$ \\
        & SLy5	&	\multicolumn{3}{c|}{collapse} & $1.332$ & $65.8$ & $6.6$ \\
        & Exp.	&	& &	&  $1.806$ & $-$  &\\ \hline
        $^{146}_{\phantom{1}56}$Ba$_{90}$ & SkM*	&	\multicolumn{3}{c|}{collapse} & \multicolumn{3}{c|}{collapse}  \\
        & SLy4	&	\multicolumn{3}{c|}{collapse} & $1.695$ & $39.7$ & $2.9$ \\
        & SLy5	&	\multicolumn{3}{c|}{collapse} & $1.541$ & $45.0$ & $3.4$ \\
        & Exp.	&	&	&	& $0.821$  & $-$ & \\ \hline
        $^{226}_{\phantom{1}88}$Ra$_{138}$ & SkM*	& \multicolumn{3}{c|}{collapse} & \multicolumn{3}{c|}{collapse} \\
        & SLy4	& \multicolumn{3}{c|}{collapse} &  $1.158$ & $59.5$ & $10.3$ \\
        & SLy5	& \multicolumn{3}{c|}{collapse} & $1.325$ & $51.4$ & $17.7$ \\
        & Exp.	& &	 &	& $0.322$ & $54$ & \\ \hline \hline
    \end{tabular}
    \label{tab:II}
\end{table}

The nucleus $^{32}$S is a double-octupole magic nucleus ($N = Z = 16$). The fully occupied state $2s_{1/2}$ is strongly coupled with the unoccupied state $1f_{7/2}$. The value of $E(3^-_1)$ is not negative for such light nuclei, but there is the enhancement of octupole collectivity. Indeed, the experimental value of $B(E3)$ of $^{32}$S is $30$ W.u. making it the strongest known $B(E3)/A$ value. Reminding that it is $34$ W.u. for $^{208}$Pb. Our result is $20$ W.u with SLy5 force. 

Reference~\cite{Spieker2022} reported a recent experiment that showed a notably higher octupole strength of approximately $32$ W.u. for $^{72}$Se, in contrast to $^{74,76}$Kr which had only exhibited the strength of $15$ W.u.. The origin of enhanced octupole collectivity in $^{72}$Se is well-explained in our discussion. The calculation for $^{72}$Se with the SLy5 force gave $E_x(3^-_1) = 2.41$ MeV close to the experimental value of $2.40$ MeV, and $B(E3) = 47$ W.u.. The neutron-rich nucleus $^{90}$Se that can be produced in a rare-isotope facility nowadays is also presented in Table \ref{tab:II}. In this region, the problem describing the $B(E3)$ in $^{64}$Zn mentioned in Ref.~\cite{Robledo2011} is also resolved as $N = 34$ which is an octupole magic number. It is $24$ W.u. in our result which is not far from the experimental value of $20$ W.u. (see Table \ref{tab:II}).

Of course, the nucleus $^{98}$Zr is similar to $^{96}$Zr which exhibits a distinct behavior compared to $^{90}$Zr or $^{96}$Ru mentioned earlier. Ba and Ra isotopes have been the subject of numerous studies in the literature (see \cite{Butler2016} for a review). The results for $^{146}$Ba and $^{226}$Ra are presented for examples. The details of research on lanthanides and actinides within the QRPA framework will be addressed in future work.

In Table \ref{tab:II}, we also show RPA results without pairing. Pairing plays a role as it keeps the nucleus less deformed and is essential to reproduce experimental data, but the presence of pairing does not alter the basic physics we discuss. 
It should be noted that previous research on the $3^-_1$ state using the Green's function RPA framework was conducted in Refs.\cite{Bertsch1975, Abbas1981}, but without full self-consistency.

The results for octupole-magic nuclei are extremely sensitive to the choice of Skyrme force. In Tables \ref{tab:I} and \ref{tab:II}, the calculations with SLy4 force for $^{96,98}$Zr overestimate the transition strengths $B(E3)$ while the results with SLy5 force are found reasonable, in both energy and strength. The difference between SLy5 and SLy4 is from the terms which depend on the spin-orbit densities \cite{CHABANAT1998231}. It is well-known that the spin-orbit interaction plays a key point in the single-particle spectrum which, as we saw, largely determines the octupole magic numbers. A small change in the spin-orbit component makes a significant change in the result.

Finally, we remark that in our calculations with the SLy5 force, the ``collapse" occurs in $^{152,154}$Sm, $^{154-160}$Gd, $^{156-170}$Dy, $^{162-170}$Er. These atomic nuclei are also octupole-magic nuclei with $Z$ and $N$ around $56$ and $88$, respectively. The research to find reliable forces or create a new one for the study of the octupole state and instability of octupole deformation across the nuclide chart will be discussed in the future.

\textbf{Conclusions and outlooks.}
The fully-self consistent mean-field framework reveals every soft-octupole deformed atomic nucleus. The octupole instability mode may lead to an octupole shape in the intrinsic frame. 
Such nuclei can enhance some unusual phenomena, such as violation of fundamental symmetries (invariance with respect to coordinate inversion and time reversal) \cite{Auerbach1996}. 
An interesting question is whether the evolution of shell structure towards the drip line leads to  new octupole magic numbers and reflection asymmetry in exotic nuclei. 
We plan to investigate the question of how well the QRPA framework would perform in a global context and the interaction of the octupole with the quadrupole degree of freedom in the future.

\textbf{Acknowledgment.} B. M. L. was supported by the Institute for Basic Science (IBS-R031-D1).
P.P. was supported by the Rare Isotope Science
Project of the Institute for Basic Science funded
by the Ministry of Science, ICT and Future Planning
and the National Research Foundation (NRF) of Korea
(2013M7A1A1075764).

\bibliography{refs}
\end{document}
%%%%%%%%%%%%%%%%%%%%%%%%%%%%%%%%%%%%%%%%%%%%%%%%%%%%%%%%%%%%%%%%%%%%