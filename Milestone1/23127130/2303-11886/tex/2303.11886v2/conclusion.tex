%\section{Conclusion}

% \Eitan{If we want to shorten, this entire conclusion would be ok to cut, and just end the previous section with future work as a high note; we might just want to zoom out from the very specific technical future work to some more ambitious or broader vision about the general agenda of interactive rate secondary dynamics and/or/in virtual worlds.}
% Alec: Agreed

We have presented a novel subspace for deformation that is well suited for augmenting real-time rig animations with secondary, complementary motion.
%
Our computation is well-balanced between fast small iterations on the CPU and memory-efficient, standard-pipeline vertex shaders on the GPU.
%
In future work, we are interested in not just adding features to our elasticity effects, but also considering complementary dynamics more broadly into domains such as fluid simulation, electrodynamics, and crowds.
%
We hope that our work also serves as general recipe for translating 
complementary dynamics to real-time scenarios.


%in real-time. We show that this subspace is closed under rotations and prove that this is a necessary and sufficient condition on any subspace simulation if it is to maintain rotation equivariance. Apart from this, our subspace has many advantages over other state of the art subspaces: it can represent rotational motion, it is material-sensitive, it can accommodate homogeneous equality constraints. It also allows us to implement projecting to the full space as an efficient low-memory operation in the vertex shader, further optimizing the draw-calls used to update our visualization. 
%We supplement this subspace with a novel local-global solver that makes use of clustering in order to approximate co-rotational elastic energy, without ever requiring any full-space operations. 
%We show that our resulting method can be used to breath life into digital characters in real-time, and can be plugged into a wide range of existing character controllers, such as mixamo animations, inverse kinematics, rigid body simulations, or face and pose trackers.

% \alec{keep this? Move it somewhere?}
% \alec{referring to our skinning subspace as a rig here might be confusing. I think it would be better to discuss this in the experiments implementation when we talk about vertex shaders.}
% We can compactly rewrite Linear Blend Skinning as a matrix multiplication,
% \begin{align}
%     \boldsymbol{u} &\approx f_{\text{lbs}}(\boldsymbol{T}; \boldsymbol{W}) = \boldsymbol{B}_{lbs} \boldsymbol{z} \ .\nonumber 
% \end{align}
% Here $\boldsymbol{z} = vec(\boldsymbol{T})$ are the flattened skinning parameters and $\boldsymbol{\Blbs} \in \mathbb{R}^{n(d) \times m}$ is our skinning Jacobian, mapping the contributions of each of these flattened parameters to high dimensional displacements given by \Sid{Incomplete sentence}
% We have two distinct ``rig''-like subspaces in our pipeline:
% \begin{itemize}
%     \item  The \textit{primary} rig allows the user to control and manipulate the pose of any given character. These are then mapped to rig displacements $\boldsymbol{u}^r$, and given as \emph{input} to our fast Complementary Dynamics.
%     \item  The \textit{secondary} Linear Blend Skinning subspace is \emph{not} exposed to the user, rather is employed by our fast Complementary Dynamics optimization to generate secondary motion.
% \end{itemize}
