\section{But What About Rotations?}
\label{sec:rotations}
%
Interesting elastodynamic effects exhibit rotations: both global, where the entire shape rotates in space, and local, where part or parts of the shape rotate relative to the rest/each other.
%
Rotations are notoriously difficult for previous linear subspaces.
%
For example, it is well known that displacement modes ($\Bdisp$ defined in \refeq{gevp-displacement-modes-derivation}) 
struggle to represent local rotational motion (see \reffig{local-rotation-experiment} and \reffig{rotation-fitting} (Left)) \cite{BarbicJames:RealTimeSTVK,Barbic:2011:RealTimeLargeDefoSubstructuring, RScoords, ModalWarping}). 

%
Our use case reveals that issues with rotations go beyond this and can be more insidious.
%
In the following discussion, we assume that the elastic potential $E$ is rotation invariant. That is, $E(\u + \x_0) = E(\repR (\u + \x_0))$, where multiplying by $\repR \in \mathbb{R}^{n(d) \times n(d)}$ applies the same rotation $\R \in SO(d)$ to all vertices.

\subsection{Rotation Spanning vs. Closure Under Rotations}
%
Rotational problems may be categorized into two separate issues.

First, does a given subspace span rotations?
%
By global rotation spanning, we mean there always exists some subspace parameters to reproduce any rotational displacement. If $\x_0$ are the rest positions then 
\begin{align}
\exists \ \z  \text{ such that } \repR \x_0 - \x_0 = \B \z \ \forall \R \in SO(d).
\end{align}
%
For free-flight objects, failing to span global rotations means the subspace will unnaturally deform in an attempt to minimize $E$ rather than rotate.
%
%
% Given a subspace $\B$ that does not span global rotations, it is trivial to augment it with additional columns $[\I_d \otimes \X]$ that span all global affine displacements (including rotations).
%
Unfortunately, this problem also occurs locally, too. For example, if the arms of a character bend in opposite directions, failure to span these local rotations will disturb (by introducing local shears and stretches to attempt to minimize $E$) or prevent (by the minimization of $E$ detesting such scales and shears) the desired deformation.
%
For example, \citet{BarbicJames:RealTimeSTVK} emphasize how displacement modes $\Bdisp$ induce scaling and shearing artifacts when approximating rotations and bending deformations.

Second, does a given subspace induce a rotationally equivariant simulation?
%
Treat the simulation as a map from problem specification parameters (e.g., forces, rig displacements, rest positions) to optimal (full-space) displacements.
%
Rotation equivariance means that any rotated version of the problem results in a correspondingly rotated solution:
%
\begin{align}
\forall \boldsymbol{R} \in \mathcal{SO}(d)\, ,  & \quad 
\forall \boldsymbol{x} \in \mathbb{R}^{n(d)}\, , \nonumber\\
     \boldsymbol{B} \argmin_{\boldsymbol{z} }{\boldsymbol{E}(\boldsymbol{B}\boldsymbol{z} + \repR \boldsymbol{x}}) &= 
    \repR \boldsymbol{B} \argmin_{\boldsymbol{z} }{\boldsymbol{E}(\boldsymbol{B}\boldsymbol{z} + \boldsymbol{x}}) \ ,
    \label{eq:rotation-equivariance-of-sim}
\end{align}
where --- without loss of generality --- we lump problem specification parameters into the vector $\x \in \mathbb{R}^{n(d)}$.

A subspace simulation lacking rotation equivariance may experience unpredictably different deformations under rotations.
%
This is especially problematic in a complementary dynamics setting where the entire object or large subpart may rotate due to the user rig. Users will expect analagously rotated secondary effects and be surprised by behavior that depends on global or local rotations coming from the rig.
%

 \begin{wrapfigure}[13]{r}{3.5cm}
\includegraphics[width=3.8cm,keepaspectratio]{images/bar-field-vis.pdf}\timestamp{\tsBarFieldVis}
\end{wrapfigure}
\reffig{subspace-comparisons} shows how this shortcoming expresses itself as  overly energetic deformation, while \reffig{random-init-linear-subspace} showcases some of the kinky local minima that can easily arise under simple rotations. 
%
The root of this problem is shown didactically in the inset: a single displacement mode describes a completely different type of motion if its underlying shape rotates.

%
Building on this intuition, we prove (see App.~\ref{sec:appendix-proof-sim-rotation-equivariance}) that a linear subspace simulation is rotation equivariant if and only if the subspace basis is \emph{closed under rotations}:
\begin{align}
    \forall \, \boldsymbol{R}\in \mathcal{SO}(d) \text{ and } \boldsymbol{z}\in\mathbb{R}^m \ \exists \, \boldsymbol{w}\in\mathbb{R}^m, \text{ such that }   \repR \boldsymbol{Bz} =  \boldsymbol{B} \boldsymbol{w}.
    \label{eq:rotation-equivariance-requirement} 
\end{align} 


%
%Before introducing our novel skinning subspace, we set the stage by reviewing the most common subspace method.
%%
%\subsubsection{Displacement Modes}
%Displacement modes form an $m$-dimensional subspace as the first $m$-eigenvectors of the elastic energy Hessian $\boldsymbol{H} \in \mathbb{R}^{n(d) \times n(d)}$:
%\begin{align}
%     \boldsymbol{H} \boldsymbol{B}_{disp}  &= \boldsymbol{M} \boldsymbol{B}_{disp} \boldsymbol{\Lambda} \label{eq:gevp-displacement-modes}
%\end{align}

%Because the columns of $\boldsymbol{B}$ can be viewed as displacement fields over the mesh, and our linear function is simply weighing each of these displacement fields by a scalar, we refer to this type of subspace basis as \textbf{\emph{displacement modes}}.




\subsection{Displacement Modes Simulations Are Fragile Under Rotations}
%
Displacement modes ($\Bdisp$ defined in \refeq{gevp-displacement-modes-derivation}) \cite{PentlandWilliams1989} and many of their 
improvements (e.g., \cite{BarbicJames:RealTimeSTVK}) are neither rotation spanning nor closed under rotations.
%
Rotations are a full-spectrum displacement, so any (reasonable) truncated elastic eigenspace will fail to span arbitrary global rotations (see \reffig{rotation-fitting} (Left)).
%
While --- as discussed above --- global rotation spanning has an easy fix, much effort has been made to improve local rotation spanning such as data-driven PCA bases \cite{EigenSkinKry2022}, modal derivatives
\cite{BarbicJames:RealTimeSTVK}, sub-structuring \cite{Barbic:2011:RealTimeLargeDefoSubstructuring}, or splitting the simulation into rigid and deformable components \cite{Terzopoulos1988}.

% %
% \edit{A common fix to this specific problem for free-flying objects is to embed our simulation in a rotating frame. The elastodynamics are computed in a rest frame, while the rotation is tracked explicitly (e.g. via a rigid body simulator \cite{Terzopoulos1988} or a user controlled rotating frame \cite{DyRT}) with some coupling forces between the two. However, extending this explicit solution to work with complex rigs and to harmonize with the rig-complementarity constraint remains non-trivial. }
% %
% \edit{Instead, we propose to bake the solution to this problem directly into the construction of our subspace basis $\boldsymbol{B}$}.
%
Nevertheless, large local rotations may still be problematic (see \reffig{local-rotation-experiment}).
%
Displacement modes --- except if truncated to just null modes or completely non-truncated --- are not closed under rotations (see counterexamples in \reffig{rotation-fitting} (Right) and \reffig{random-init-linear-subspace}).
%
% The column-space ``expansion'' of \citet{Tycowicz2013} converts a linear subspace (such as displacement modes) into one that is rotation spanning and closed under rotations. 
% %
% It is not clear how to adapt this filtering process to implicitly enforce constraints such as our rig complementarity.
% %
% In terms of ease of adoption into real-time animation pipelines, the method of \citet{Tycowicz2013} is tantamount to multi-weight skinning \cite{skinningcourse:2014}, which is obscure compared to the ubiquity of linear blend skinning.

% Displacement modes in general are \emph{not} closed under rotations (see \reffig{rotation-fitting} (Right)). 
% Of course, some configurations of displacement modes can satisfy this constraint, such as a set of $d$ orthogonal translations, or a \emph{full} set of displacement modes, but these configurations are too low-dimensional, computationally expensive, or tedious to find. \Sid{Yes, let's somehow (tersely) reflect this observation in the figure captions as well.}
% \Otman{Definitely! commenting out rn before submitting first draft for anonymity}

\begin{figure}
\includegraphics[width=\linewidth,keepaspectratio]{images/cthulu_local_minimum.pdf}\timestamp{\tsCthuluLocalMinimum}
\caption{
We compute the subspace at rest (top-left). A user rotates the mesh and perturbs the system with a random initial deformation.  Using displacement modes creates jarring local minimum artefacts in a rotated frame. Our skinning modes find the global minimum effortlessly, \edit{obtaining the same rest state than if we had embedded the simulation in a rigid frame. \label{fig:random-init-linear-subspace}}}
\end{figure}

\subsection{Skinning Eigenmodes Are Robust to Rotations}
\label{sec:skinning-modes}
In contrast, skinning eigenmodes are both rotation spanning and closed under rotations (see \reffig{rotation-fitting}).
%
When the complementarity constraint is absent, the first skinning eigenmode will be a constant function thus spanning all affine motions including rotations.
%
When used for fast complementary dynamics, the rig typically contains global rotations so we explicitly (and purposefully) avoid global rotation spanning.
%
We do still want and indeed observe local rotation spanning (see \reffig{local-rotation-experiment}).
%
%
%
%\alec{skinning modes are trivially rotation spanning (when not doing CD) because the first mode is constant. and exhibit good local rotation spanning see giraffe, spoon, hand1, twist, alien. skinnings modes are trivially rotation equivariant as well, see ref hand2}
%
%% We propose a linear blend skinning subspace which represents common low energy deformations, spans local rotational motion, and \emph{guarantees} rotation equivariance in its resulting simulation.
%Motivated by the fact that displacement modes cannot represent rotations, nor do they guarantee simulation rotation equivariance, we identify a linear subspace that possesses both of these qualities: linear blend skinning. For each vertex $i$, a linear blend skinning subspace is parameterized by $\boldsymbol{W}$ and can be written as:
%\begin{equation}
%    \boldsymbol{u}_i \approx f_{lbs}(\boldsymbol{T}; \boldsymbol{W}) = \sum_b^{m} w_{ib} \boldsymbol{T}_b \begin{bmatrix} \boldsymbol{x}_{0i} \\ \boldsymbol{1} \end{bmatrix} \nonumber .
%\end{equation}
We can easily show that the linear blend skinning --- and thus also skinning subspaces --- are closed under rotations. 
%
Given some rotation $\boldsymbol{R} \in \mathcal{SO}(d)$, rotating linear blend skinning's output is equivalent to rotating all of the input transformations:
\begin{equation}
\R \sum_{b=1}^m w_{ib} \T_b \X_i = 
\sum_{b=1}^m w_{ib} \R \T_b \X_i.
%\boldsymbol{R} \, f_{lbs}(\boldsymbol{T}; \boldsymbol{W}) =  \sum_b^{m} w_{ib} \boldsymbol{R}\boldsymbol{T}_b \begin{bmatrix} \boldsymbol{x}_{0i} \\ \boldsymbol{1} \end{bmatrix}   = f_{lbs}(\boldsymbol{RT}; \boldsymbol{W}).
    \label{eq:linear-blend-skinning}
\end{equation}
%
Any rotation of its output is producible by its input, as required for a rotation equivariant subspace simulation.
%
This fact was similarly utilized in previous works albeit in different settings \cite{Wang:2015:LinearSubspaceDesign,JacobsonBPS11,LangerS08}.
%
% \edit{Many prior methods propose subspaces for deformation that guarantee rotation equivariance \cite{Faure2011, 1Gilles2011, Wang:2015:LinearSubspaceDesign, Tycowicz2013}. To the best of our knowledge, we are the first to motivate our choice of subspace with whether or not it is closed under rotations. 
\edit{
There are many prior  subspaces \cite{Faure2011, 1Gilles2011, Wang:2015:LinearSubspaceDesign, Tycowicz2013}  that do not explicitly mention rotation equivariance as a criterion for the subspace simulation. In hindsight, leveraging the machinery of Appendix \ref{appenix-eq:closed-under-rotations}, we can see that since these prior subspaces are closed under rotations, those methods also maintain a rotation equivariant simulation.
}

\begin{figure}
\includegraphics[width=\linewidth,keepaspectratio]{images/bingby_local_rotation_light.pdf}\timestamp[-0.125cm]{\tsBingbyLocalRotation}
\caption{No matter how hard a user tries, the eyes of this reduced elastodynamic alien will never bend when using a small displacement mode subspace (middle). Our skinning subspace (right) enables the rotational motion. Both results use 60 degrees of freedom. 
% \Sid{Pedantry: it will presumably bend if you use enough modes, approximately the full set. We should clarify that this observation applies to bases that are relatively small, for some suitable definition of small.} 
\label{fig:local-rotation-experiment}} 
\end{figure}

\begin{figure*}
\includegraphics[width=\linewidth,keepaspectratio]{images/rotation_fitting.pdf}
\caption{ \textbf{Rotation Spanning vs. Closure Under Rotations.} Given some initial shape, we optimize for optimal displacements that minimize the squared distance between each vertex position and its rotated target. (Left) Our skinning modes are \textbf{rotation spanning}, as modulated by the skinning weights. With a single constant skinning weight, we can perfectly reconstruct (by least squares projection) any rotation of the rest shape. Displacement modes do not span rotational motion, even with excessively abundant modes. (Right-red) Our skinning modes are \textbf{closed under rotations}, so any deformation in the span of those can be reconstructed (by least squares projection) under the same set of modes even if the mesh is arbitrarily rotated by a user. \edit{The same cannot be said for even an impractically large number of displacement modes even if augmented with affine degrees of freedom  (right-purple).}
 \label{fig:rotation-fitting}}
\end{figure*} % great figure eris! And great caption :)

% \begin{figure}
% \includegraphics[width=\linewidth,keepaspectratio]{images/hand_eq_fitting.pdf}
% \caption{ 
%  \label{fig:equivariance-fitting}  Our subspace is closed under rotations, so any deformation in it can be reconstructed (by least squares projection) even if the mesh is arbitrarily rotated by a user. The same cannot be said for even an impractically large number of displacement modes.}
%  \end{figure}