\section{Background: Complementary Dynamics}
Complementary dynamics provides a methodology for augmenting rigged animations with the detailed  elastodynamics \cite{Zhang:CompDynamics:2020}.
%
They split the total displacement field $\boldsymbol{u} \in \mathbb{R}^{n(d)}$ for a mesh with $n$ vertices in $d$-dimensional space into an artist prescribed component $\boldsymbol{u}^r$, and a physical component $\boldsymbol{u}^c$:
\begin{align}
    \boldsymbol{u} = \boldsymbol{u}^r + \boldsymbol{u}^c.
\end{align}

The artist-prescribed component is obtained from a rig, which takes as input low dimensional rig parameters $\boldsymbol{p} \in \mathbb{R}^\dimp$
that are exposed to the user for interactive manipulation, and maps them to high dimensional rig displacement $\boldsymbol{u}^r$:
\begin{align}
    \boldsymbol{u}^r = f_{\text{rig}}(\boldsymbol{p}).
\end{align}

% A very popular type of rig is Linear Blend Skinning, whereby the user interacts with a set of affine matrices $\boldsymbol{p} = \mathrm{vec}(\boldsymbol{T})$ ( often representing "bones" or "handles"). These affine matrices are then blended with a set of \emph{skinning} weights defined over the mesh, and applied to the vertices to obtain the final motion for each vertex:
% \begin{align}
% \boldsymbol{u}^r_i =  \sum_b^\mathcal{B}\boldsymbol{w}_{ib} \boldsymbol{T}_b
% \begin{bmatrix}aaaa
% \boldsymbol{x}_{0i} \\
% 1
% \end{bmatrix} - \boldsymbol{x}_{0i}
% \label{eq:original-linear-blend-skinning}
% \end{align}
The physical motion on the other hand is obtained from a physics simulation, which can be formulated as an energy minimization problem:
 \begin{align}
 \label{eq:cd}
     \boldsymbol{u}^c = \argmin_{\boldsymbol{u}^c} E(\boldsymbol{u}^c + \boldsymbol{u}^r + \boldsymbol{x}_0) \quad \text{s.t.}\ 
    \boldsymbol{J}^T \boldsymbol{u}^c = \boldsymbol{0},
 \end{align}
We introduce the rig Jacobian  $\boldsymbol{J} = \frac{\partial f_{\text{rig}}(\boldsymbol{p}) }{\partial \boldsymbol{p}} \in \mathbb{R}^{n(d)\times \dimp }$.
%
Note that complementary dynamics is completely agnostic to the elasto-dynamic energy used $E(\cdot)$. Where it differs from a regular elasto-dynamic energy minimization is the specification of the complementarity constraint $\boldsymbol{J}^T \boldsymbol{u}^c = \boldsymbol{0}$. This constraint enforces that the physical motion must not be in the space of motions producible by the rig, $\mathrm{Col}(\boldsymbol{J})$. 
\citet{Zhang:CompDynamics:2020} enforce the complementarity constraint through Lagrange multipliers and the resulting energy is minimized using Newton's method, requiring the frequent solve of the following KKT system:
\begin{align}
    \begin{bmatrix}
    \boldsymbol{H} & \boldsymbol{J} \\
    \boldsymbol{J}^T & \boldsymbol{0}
    \end{bmatrix}
    \begin{bmatrix}
    d\boldsymbol{u}^c \\
    \boldsymbol{\lambda} 
    \end{bmatrix} = 
    \begin{bmatrix}
    -\boldsymbol{g} \\
    \boldsymbol{0} 
    \end{bmatrix},
\end{align}
where $\boldsymbol{H} \in \mathbb{R}^{n(d) \times n(d) }$ 
and
$\boldsymbol{g} \in \mathbb{R}^{n(d)}$ 
are
the elasto-dynamic energy Hessian \& gradient, $d\boldsymbol{u}^c \in \mathbb{R}^{n(d)}$ is the Newton search direction, and $\boldsymbol{\lambda} \in \mathbb{R}^\dimp$ collects the Lagrange multipliers enforcing the complementarity constraint.

Iteratively 
solving this system is too expensive in real-time applications for two main reasons:
\begin{enumerate}
  %\item \textbf{The constraints  $\boldsymbol{J}^T \boldsymbol{u}^c = \boldsymbol{0}$  are hard to satisfy}:  $\boldsymbol{J}$ scales in size with the control rig and the mesh resolution, is commonly dense, and makes the system indefinite. 
  %\item \textbf{The constraints  $\boldsymbol{J}^T \boldsymbol{u}^c = \boldsymbol{0}$  are dense even if $\boldsymbol{H}$ is sparse}:  
  %$\boldsymbol{J}$ scales in size with the control rig and the mesh resolution, is commonly dense, and makes the system indefinite. 
  %\item \textbf{ The system scales in size with mesh resolution}, making it costly both to construct and solve for every iteration. 
  \item the system scales with mesh resolution, and
  \item the constraints $\boldsymbol{J}^T \boldsymbol{u}^c = \boldsymbol{0}$ are typically dense, even if $\boldsymbol{H}$ is sparse.
\end{enumerate}

