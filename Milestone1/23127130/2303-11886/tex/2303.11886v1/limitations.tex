\section{Limitations and Future Work}
%
Compared to the original complementary dynamics \cite{Zhang:CompDynamics:2020}, we assume that the user's rig is linear.
%
While non-linear rigs are less common in real-time settings, it would still be interesting to accommodate them (e.g., \cite{Kavan-08-SDQ}, perhaps via low-rank updates to the reduced system matrices in Alg.~\ref{alg:simulationStep} \cite{RSPHahn}. Our reduction techniques are in many ways agnostic to the choice of elastic potential. In future work, we would like to explore beyond co-rotational models. Our skinning eigenmodes easily facilitate rig-orthogonality constraints for complementary dynamics; their other good qualities suggest they could be useful beyond our target application, applied to more general elasticity problems (e.g., structural analysis or inverse design).

We leave comprehensive collision handling as future work.
%
While rigid-body augmentation provides a drop-in heuristic for collision effects for relatively stiff or fast moving objects, it would be interesting to explore more accurate methods, perhaps combining our contributions with cubature techniques \cite{HarmonSubspaceLocalDeformation2013}.
%
For real-time VR avatar, applications we found the bottleneck for quality lies in the tracking and mapping of user movements to primary rig controls. Our secondary effects would immediately inherit any improvements in these active research areas.

We derive skinning eigenmodes by considering distributions of translations. This sidesteps the issue of determining an origin and scale associated with each mode. , It would be interesting to treat these as variables (to achieve theoretical optimality over distributions of general affine transformations). This appears non-trivial and irreducible to a generalized eigenvalue problem

% - other elastic energies
% - assume linear rigs (no DQS)
% - optimal weights with respect to translations, optimie over choice of origin
% - beyond CD
% - improve face and body tracking
% - dealing with collisions
%
%While we believe our fast Complementary Dynamics pipeline is generalizeable to many different elastic materials, we have only evaluated it on volumetric ARAP and co-rotational materials.
%To maintain many of the pre-computed matrices in Algorithm \ref{alg:simulationStep}, we assume $\boldsymbol{J}$, our primary rig Jacobian, does not change. This limits us to using linear rigs. It would be useful to allow for a changing rig Jacobian by performing low rank updates to these matrix products as the rig changes, like 
%\citet{RSPHahn}. This would let us accommodate more non-linear primary rigs such as Dual Quaternion Skinning \cite{Kavan-08-SDQ}.

% Unlike a traditional displacement mode subspace, the columns of our subspace matrix   $\boldsymbol{B}_{lbs}$ are \emph{not} constrained to be orthogonal to each other. If too many skinning modes are requested, there may very well occur redundancies in our subspace. In such cases, the redundant columns should be detected and removed. Indeed, if we requested $m=n$ skinning modes, then our subspace matrix $\boldsymbol{B}$ would have $12n$ columns in 3D, which has as many as $9n$ redundant columns. In practice, none of the examples on this paper have required this detection, and so long as the requested number of skinning modes is much less than the total number of vertices, this shouldn't be a problem. 

%We derive our secondary skinning weights to be optimal with respect to translations, and show that scaling and shearing both form bad subspaces for deformation.  While this may be true, scaling and shearing can form rotations, which do provide low-energy deformations. Figuring out a way to make our skinning weights optimal with respect to rotational motion could allow us to find efficient modes that would otherwise be missed by prioritizing translations.

%We've demonstrated many examples with which we use our skinning subspace for general deformation (not just complementary dynamics). More than this, we believe that the properties of our subspace can be used for more applications involving deformation, such as structural analysis, or inverse design.

%We provide methods of augmenting input rig motions with secondary dynamics in \emph{real-time}, and a cutting-edge application we have demonstrated is in its use for Virtual Reality applications on digital avatars. An interesting path for future work lies in investigating different mechanisms with which users can control various character rigs, especially via video stream input. In practice, we've found making use of out of the box video stream based pose trackers to be challenging due to noise present in the pose capture, and difficulty detecting depth. 

%More broadly speaking, many games place articulated characters in rigid body simulation environments. In such cases, the Complementarity constraint can clash with collision constraints imposed by the rigid body simulation. It would be useful to investigate when the rigid body simulation should override the complementarity constraint and vice versa.  

%In a similar vein, many computational models for fluid simulations remain difficult to control. Identifying what makes a good control mechanism for such phenomena remains a challenging but exciting open research area, and we are hopeful that similar rig-based control mechanisms and  physical augmentation methods could be employed. 
