\section{Introduction}

% \textwidth 7.06233in
% \linewidth 3.46889in
% Main text font 9pt LinuxLibertineT-TLF
% Caption text font 8pt LinuxBiolinumT-TLF

%\begin{tabular}{ll}
%\texttt{\textbackslash{textwidth}} & \printinunitsof{in}\prntlen{\textwidth} \\
%\texttt{\textbackslash{linewidth}} & \printinunitsof{in}\prntlen{\linewidth} \\
%Main text font &  \makeatletter \f@size pt \f@family \makeatother \\
%\sffamily \small Caption text font &  \sffamily \small \makeatletter \f@size pt \f@family \makeatother \\
%\end{tabular}

% Virtual environments req
% Realistic motion breathes life into digital characters. Artists, designers and consumers interacting with virtual environments require demand interactive interaction 



% narrow in on topic
Virtual reality, video games, and digital art increasingly make use of controllable animated characters. Such characters should provide \emph{interactive} responses to user input in order to communicate the action and emotion of the moment. On the other hand, their motion must be rich with \emph{realistic} visual details, which is what brings these characters to life. 

% Now more than ever, there is huge demand for animated characters in artificial and virtual reality, video games, and digital art. Such character deformations must be interactively animated to communicate the action and emotion of the moment. On the other hand, they must be rich with details that breath life into the characters.

The most widely used construct for authoring animations are deformation \emph{rigs} mapping low-dimensional parameters (e.g., skeleton or cage positions) to static geometric deformations. Managing these rigs can quickly overwhelm an artist. Simple rigs are easy to animate, but their deformations lack detail; complex rigs provide rich, fine-grained detail, but they are daunting to manipulate.
% dig a hole
\citet{Zhang:CompDynamics:2020} offer a way out of this dilemma; they employ a physics simulation whose role is to supplement rig motions with secondary dynamics that are orthogonal --- in the algebraic sense --- to the motion of the rig. The resulting \emph{complementary dynamics} are exactly those motions not producible by the rig itself.

 % This results in a clear-cut, well-posed division of responsibility between rig and secondary physics. Any rig type may be accommodated, so long as we can describe the space of motions it provides algebraically.

Unfortunately, complementary dynamics is poorly suited to interactive applications for two reasons. First, enforcing complementarity via linear equality constraints introduces significant computational overhead. Second, the runtime of the method grows with the mesh resolution.  Adding secondary motion even on a modestly sized mesh requires computation that lags far from real-time rates. As shown in \reffig{full_vs_reduced_cdl}, a modest animation using \cite{Zhang:CompDynamics:2020} on a mesh with 10,000 vertices, 42,000 tetrahedra, and two linear blend-skinning bones requires 3 seconds for each frame. The first thing to try accelerating this approach is to embedding the displayed mesh in a coarse simulation (see \reffig{coarsening-meshes}), but this comes at the cost of visible cage-embedding artifacts details. 


\begin{figure}
\includegraphics[width=\linewidth,keepaspectratio]{images/coarse_cage_vs_fast_cd-1.png}\timestamp[-0.125cm]{\tsCoarseningMeshes}
\caption{A coarse embedded simulation groups motion across vertices that are close in Euclidean space. Instead, our subspace groups motion across vertices that share elastic energy properties (cage constructed via \cite{breaking-good}).
 \label{fig:coarsening-meshes}}
\end{figure}

To accelerate any high-dimensional optimization problem, a popular approach is to solve the problem in a low-dimensional, representative \emph{subspace}. The \emph{de facto} subspace for elasto-dynamics in graphics is the one spanned by the first few eigenvectors of the elastic energy Hessian \cite{PentlandWilliams1989}. We call these eigenvectors \emph{\textbf{displacement modes}} because---for elasticity---they correspond to the set of least energy-incurring infinitesimal displacements about the rest state. 

Our use case exposes the pitfalls of this classical subspace. First, it is well known that this subspace does not represent rotational deformations, leading to warping or shearing artifacts \cite{ModalWarping}. Second and (as we will show) \textit{distinctly} this subspace induces simulations lacking \emph{rotation equivariance}. 
In a nutshell, a rotation equivariant optimization produces a rotated version of the same minimizer when the problem geometry is rotated. The absence of this important property leads to jarring frame-dependant artifacts. This is particularly noticeable in our application, where local rotations form a primary degree of freedom of the interactive rig. As the user interacts with the rig, the secondary physics exhibit non-physical dampened motion, and the mesh gets stuck in local minima as shown in \reffig{dead-tree-local-rotation-equivariance} and \reffig{random-init-linear-subspace}. 

To address these challenges, we propose \textbf{\textit{skinning eigenmodes}} for reduced simulation. Inspired by Linear Blend Skinning, the subspace spanned by our skinning modes yields rotation equivariant secondary elastodynamics. Indeed, we \textit{prove} that it meets the necessary and sufficient conditions for doing so. Our subspace is fully parameterized by a compact set of \emph{skinning weights}, which we derive through a physically motivated, material aware and simple to implement generalized eigenvalue problem. 

The formulation of our skinning modes as the solution to an eigenvalue problem has many advantages. A user can easily explore the cost versus richness tradeoff of the resulting dynamics by simply truncating the eigenspace through using fewer skinning modes. We also benefit from a large array of work that aims at promoting different qualities from eigen problems, such as enforcing locality and sparsity in our modes \cite{Brandt2017CompressedVibrationModesofElasticBodies, Nasikun2018, compressedModesADMM}, or enforcing homogeneous linear equality constraints \cite{golub1973}. 
In particular we benefit from the latter to make our skinning subspace orthogonal to the input rig-space. This allows our modes to more efficiently capture the space of secondary motions available to the geometry given an input rig and provide a mesh resolution-independent deformation subspace that is artifact free.

However, skinning modes alone are insufficient for achieving realistic-looking real-time dynamics. This is because secondary elastodynamics look best with non-linear elastic models. Simulating such materials, even with a deformation subspace,  \emph{still} requires the computation of per-tetrahedron quantities \cite{ModalWarping}. This once again ties our runtime complexity to the resolution of the mesh. To accommodate this, we adopt clustering \cite{Jacobson-12-FAST}, which approximates these per-tet quantities to per-cluster ones. We also extend Jacobson et al.\ \shortcite{Jacobson-12-FAST}'s local-global solver to approximate a co-rotational elastic potential.
%
The result is an iterative solver that harmonizes well with our subspace and allows rich non-linear secondary motions in a simulation step that is \emph{entirely} decoupled from the mesh resolution. Additionally, by virtue of being based on \emph{linear} blend skinning, projecting from our low dimensional subspace to the full space can be done entirely in the vertex shader. 

%

% The price we pay is that our subspace and clusters are computed to be optimal with respect to a rest-state prior we have on our rigged-character. Also, unlike the original complementary dynamics, we only assume \emph{linear} control rigs, as this allows us to precompute many matrices that let us perform \emph{all} simulation steps within our small subspace parameters, clustered non-linear quantities, or rig-parameters.  



% Our method provides a set of user parameters that intuitively control the speed/accuracy tradeoff for our animation and allow a user to tailor it to their needs.



% %Give a broad over view of our method%

% \Otman{Maybe give a high level highlight that we develop a suitable subspace capable of representing deformations AND enforcing complementarity. Say we handle non-linear materials like ARAP/corotational elasticity by making use of clustering. Just high level statements that overviews our method.}

%  Control over the pose and orientation of complex geometry is often achieved through rigs, a general framework that maps easy-to-use low-dimensional parameters to static geometric deformations. To control simulations, most other methods take a reductive interpretation of these rig interfaces; they constrain the vertices bound by some geometric notion of the rig to move along with the rig, like the rig is just a set of rigid steel bones embedded in the geometry.


% tease how we do it
%CD is slow for two reasons: 1) its complexity depends on the mesh resolution and 2) linear equality constraints add additional performance overhead.
%
%Modal analysis in the space constrained by a linear rig, so that CD constraints can be eliminated at runtime.
%
%Secondary elastodynamics look best with non-linear elastic models.
%
%We introduce a clustering method to approximate a co-rotational elasticity elastic potential that harmonize with our modal subspace.
%
%The result is a non-linear elasticity integrator for complementary dynamics that is fully decoupled from the mesh resolution.

% How will we provide evidence of success?
\reffig{teaser-figure} demonstrates our reduced elasto-dynamics augmenting a large scene with secondary motion in \emph{real}-time. We further demonstrate the success of our method through a variety of comparisons to the original (offline) complementary dynamics and against alternative real-time acceleration methods.
%
We additionally highlight how our skinning modes can be used for a wide range of standard deformation tasks (not just complementary dynamics).
% \alec{are we living up to this previous claim?} \Otman{Are our fitting to differente extreme deformations not enough?} \Eris{We also have examples e.g. binby, bar bending, bar volume preserving that are not controlled by a primary rig so unconstrained.}
%
Finally, we show successful application of our real-time method to scenarios with rapidly changing rig-input, such as rigid-body enrichment, VR puppetry, and rigged character secondary effects. 

\begin{figure}
\centering\includegraphics{images/cd_fish_full_vs_reduced-1.png}\timestamp[-0.125cm]{\tsFullvsReduced}
    \caption{
    \label{fig:full_vs_reduced_cdl} Our reduced and clustered complementary dynamics model can reproduce rich visual details at a fraction of the cost of the original method (10,000 vertices, 42,205 tets).}
\end{figure}

% %Now more than ever, humans are immersed in digital worlds. tacky%
% Digital worlds are becoming increasingly enticing for artistic creation, engineering design and entertainment applications; they allow unlimited opportunities for creative exploration, unbounded by real-world physical limitations. Interacting with such 3D environments, especially ones containing physically-based motion, begets a difficult set of graphics and simulation challenges. Specifically, providing user-control while also trying to exhibit physical motion forms a contradiction; user control is inherently a non-physical effect. As a result, a user usually has to carefully select geometry and assign them to either be controllable, or exhibit physical motion, never both.

%  Control over the pose and orientation of complex geometry is often achieved through rigs, a general framework that maps easy-to-use low-dimensional parameters to static geometric deformations. To control simulations, most other methods take a reductive interpretation of these rig interfaces; they constrain the vertices bound by some geometric notion of the rig to move along with the rig, like the rig is just a set of rigid steel bones embedded in the geometry.
 
%  Recently, a new method of controlling elastic simulations was proposed by \citet{Zhang:CompDynamics:2020}.
% This method encourages the use of control rigs without needing to artificially attribute a geometric interpretation to them. Specifically it defines rig motion as a primary motion that must not be undone. For secondary physical motion, these are found through normal simulation techniques with an additional constraint that ensures that the secondary motion must not be be motion producible by the rig. This results in an intuitive simulation control interface that allows for realistic physical motion, that is entirely controllable by a broad set of rigs.  

% Unfortunately, enforcing this constraint can add significant computational overhead to the simulation, barring the use of this method for real-time applications. Now more than ever, there is huge demand for interactive control over simulations in AR/VR applications, video games and digital art. t


% \Otman{Would like to flesh out more why handling this at interactive rates is important. What make sit so important that this can manipulate and control characters in real time, and have them exhibit secondary motion. (Can't just say people want to use it, need to say why!!) Secondary motion, physics based motion, is one of the first things that can bring a scene to life, realistic 3D environments must replicate this secondary motion. Other than that faster algorithm means faster iterating in the design loop. }

% We propose a method that extends the use of Complementary Dynamics for such real-time applications. The key to our method is to decouple the run-time complexity from both the mesh resolution, \textit{and} the rig complexity. Our method provides a set of user parameters that intuitively control the speed/accuracy tradeoff for our animation and allow a user to tailor it to their needs. 
% %Give a broad over view of our method%

% \Otman{Maybe give a high level highlight that we develop a suitable subspace capable of representing deformations AND enforcing complementarity. Say we handle non-linear materials like ARAP/corotational elasticity by making use of clustering. Just high level statements that overviews our method.}

