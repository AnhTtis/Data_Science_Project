% \documentclass[acmtog,anonymous,review, nonacm, balance=false, prologue, table]{acmart}
\documentclass[acmtog, nonacm, balance=false, prologue, table]{acmart}
\acmSubmissionID{502}

\usepackage{amsmath}
\usepackage{booktabs} % For formal tables
%\usepackage{amssymb}
\usepackage{float}
\usepackage{layouts}
\usepackage{wrapfig}
\usepackage[ruled,vlined]{algorithm2e}

\usepackage{bbm}

\setlength{\columnsep}{1.0em}
\setlength{\intextsep}{0em}

% TOG prefers author-name bib system with square brackets
\citestyle{acmauthoryear}
%\setcitestyle{nosort,square} % nosort to allow for manual chronological ordering

% \usepackage[ruled]{algorithm2e} % For algorithms
% \renewcommand{\algorithmcfname}{ALGORITHM}
\usepackage{appendix}
% \SetAlFnt{\small}
% \SetAlCapFnt{\small}
% \SetAlCapNameFnt{\small}
% \SetAlCapHSkip{0pt}

\definecolor{forestgreen}{rgb}{0.13, 0.55, 0.13}
\definecolor{lava}{rgb}{0.81, 0.06, 0.13}
\definecolor{magenta}{rgb}{0.7, 0.0, 1.0}
\newcommand{\alec}[1]{\textbf{\textcolor{lava}{[Alec: #1]}}}
\newcommand{\Otman}[1]{\textbf{\textcolor{forestgreen}{[Otman: #1]}}}
% \newcommand{\edit}[1]{\textbf{\textcolor{forestgreen}{ #1}}}
\newcommand{\edit}[1]{ #1}
\newcommand{\Eitan}[1]{\textbf{\textcolor{blue}{[Eitan: #1]}}}
\newcommand{\Yi}[1]{\textbf{\textcolor{orange}{[Yi: #1]}}}
\newcommand{\yi}[1]{\textbf{\textcolor{orange}{[Yi: #1]}}}
\newcommand{\Sid}[1]{\textbf{\textcolor{purple}{[Sid: #1]}}}
\newcommand{\Eris}[1]
{\textbf{\textcolor{magenta}{[Eris: #1]}}}
\newcommand{\eris}[1]
{\textbf{\textcolor{magenta}{[Eris: #1]}}}

% https://davidmathlogic.com/colorblind/#%23D81B60-%231E88E5-%23FFC107-%23004D40
\definecolor{staticColor}{HTML}{005AB5}
\newcommand{\staticp}[1]{\textcolor{staticColor}{#1}}
\def\staticColorName{blue}
\definecolor{dynamicColor}{HTML}{DC3220}
\newcommand{\dynamicp}[1]{\textcolor{dynamicColor}{#1}}
\def\dynamicColorName{red}

\newcommand{\reffig}[1]{Fig.~\ref{fig:#1}}
\newcommand{\refeq}[1]{Eq.~(\ref{eq:#1})}
\newcommand{\refsec}[1]{Sec.~\ref{sec:#1}}

% https://tex.stackexchange.com/a/258341/13600
\newcommand*{\img}[1]{%
    \raisebox{-0\baselineskip}{%
        \includegraphics[
        keepaspectratio,
        ]{#1}%
    }%
}

\newcommand*{\timestamp}[2][-0.5cm]{%
   \sffamily \small%
  \vspace{#1}%
  \begin{flushright}%
  \protect\img{images/video-symbol.pdf}#2%
  \end{flushright}%
}
\def\tsAquarium{0m40s}
\def\tsCoarseningMeshes{3m20s}
\def\tsFullvsReduced{0m16s}
\def\tsMorayModalDerivs{4m57s}
\def\tsSpoonModalFitting{3m33s}
\def\tsBingbyLocalRotation{3m34s}
\def\tsSubspaceComparison{1m45s}
\def\tsCthuluLocalMinimum{1m27s}
\def\tsBarFieldVis{1m21s}
\def\tsWeightVis{1m56s}
\def\tsAffineMotions{2m08s}
\def\tsHeterogeneousMaterialExperiment{4m01s}
\def\tsBarModalWarpingComparison{3m52s}
\def\tsGiraffeRSComparison{3m44s}
% \def\tsCorotSubspaceComparison{0m00s}
\def\tsConstrainedVsUnconstrained{4m27s}
\def\tsCrocodileCorotVsARAP{4m38s}
\def\tsRigidBodyGame{5m42s}
\def\tsMixamoRigSwitchingApp{5m04s}
\def\tsFastCDIK{5m26s}
\def\tsMediaPipeFace{6m02s}
\def\tsMediaPipePose{6m11s}
\usepackage{amsmath}
\DeclareMathOperator*{\argmax}{argmax}
\DeclareMathOperator*{\argmin}{argmin}
\usepackage{mfirstuc}
\MFUnocap{et}
\MFUnocap{al}
%\newcommand{\repR}{[\boldsymbol{R}]}
%\newcommand{\repR}{\left(\boldsymbol{I} \otimes \boldsymbol{R}\right)}

\newcommand{\A}{\boldsymbol{A}}
\newcommand{\Bdisp}{\boldsymbol{B}_\text{disp}}
\newcommand{\Blbs}{\boldsymbol{B}_\text{lbs}}
\newcommand{\B}{\boldsymbol{B}}
\newcommand{\I}{\boldsymbol{I}}
\newcommand{\J}{\boldsymbol{J}}
\newcommand{\M}{\boldsymbol{M}}
\newcommand{\R}{\boldsymbol{R}}
\newcommand{\T}{\boldsymbol{T}}
\newcommand{\W}{\boldsymbol{W}}
\newcommand{\X}{\boldsymbol{X}}
\newcommand{\Y}{\boldsymbol{Y}}
\newcommand{\Z}{\boldsymbol{Z}}
\newcommand{\g}{\boldsymbol{g}}
\newcommand{\p}{\boldsymbol{p}}
% \newcommand{\dim}{d}
\newcommand{\dimension}{d}
\newcommand{\da}{d(d+1)}
\newcommand{\dimp}{p}
\newcommand{\nummodes}{m}
\newcommand{\numdofs}{k}
\newcommand{\numtets}{t}
\newcommand{\numclusters}{r}
\newcommand{\x}{\boldsymbol{x}}
\newcommand{\z}{\boldsymbol{z}}

% providecommand ensures there is no error if
% the command doesn't already exist 
\providecommand{\C}{}
\renewcommand{\C}{\boldsymbol{C}}
\providecommand{\H}{}
\renewcommand{\H}{\boldsymbol{H}}
\providecommand{\M}{}
\renewcommand{\M}{\boldsymbol{M}}
\providecommand{\P}{}
\renewcommand{\P}{\boldsymbol{P}}
\providecommand{\u}{}
\renewcommand{\u}{\boldsymbol{u}}

\newcommand{\rep}[1]{\left(#1 \otimes \boldsymbol{I}\right)}
\newcommand{\repR}{\rep{\R}}

% Metadata Information
\acmJournal{TOG}
%\acmVolume{38}
%\acmNumber{4}
%\acmArticle{39}
%\acmYear{2019}
%\acmMonth{7}

% Copyright
%\setcopyright{acmcopyright}
%\setcopyright{acmlicensed}
%\setcopyright{rightsretained}
%\setcopyright{usgov}
%\setcopyright{usgovmixed}
%\setcopyright{cagov}
%\setcopyright{cagovmixed}

% DOI
%\acmDOI{0000001.0000001_2}

% Paper history
%\received{February 2007}
%\received{March 2009}
%\received[final version]{June 2009}
%\received[accepted]{July 2009}
% Document starts
\begin{document}
% Title portion
% \title{Fast Complementary Dynamics}
% \title{Fast Complementary Dynamics via Skinning Subspaces}
\title{Fast Complementary Dynamics via Skinning Eigenmodes}
% DO NOT ENTER AUTHOR INFORMATION FOR ANONYMOUS TECHNICAL PAPER SUBMISSIONS TO SIGGRAPH 2019!


\author{Otman Benchekroun}
\affiliation{
\institution{University of Toronto}
 \country{Canada}
}
\author{Jiayi Eris Zhang}
\affiliation{
 \institution{Stanford University}
  \country{USA}
}
\author{Siddhartha Chaudhuri}
\affiliation{
 \institution{IIT Bombay and Adobe Research}
    \country{India}
}
\author{Eitan Grinspun}
\affiliation{
  \institution{University of Toronto}
   \country{Canada}
}
 \author{Yi Zhou}
\affiliation{
\institution{Adobe Research}
 \country{USA}
}
\author{Alec Jacobson}
\affiliation{
 \institution{University of Toronto and Adobe Research}
 \country{Canada}
}
%\renewcommand\shortauthors{Zhou, G. et al}

\begin{abstract}
    We propose a reduced-space elasto-dynamic solver that is well suited for augmenting rigged character animations with secondary motion. At the core of our method is a novel deformation subspace based on Linear Blend Skinning that overcomes many of the shortcomings prior subspace methods face. Our skinning subspace is parameterized entirely by a set of scalar weights, which we can obtain through a small, material-aware and rig-sensitive generalized eigenvalue problem. The resulting subspace can easily capture rotational motion and guarantees that the resulting simulation is rotation equivariant.  We further propose a simple local-global solver for linear co-rotational elasticity and propose a clustering method to aggregate per-tetrahedra non-linear energetic quantities. The result is a compact simulation that is fully decoupled from the complexity of the mesh. 
\end{abstract}


%
% The code below should be generated by the tool at
% http://dl.acm.org/ccs.cfm
% Please copy and paste the code instead of the example below.
%
%\begin{CCSXML}


%
% End generated code
%


\keywords{Linear Blend Skinning, Secondary Motion, Complementary Dynamics}

% uncomment for using teaser
\begin{teaserfigure}
   \centering%
  \includegraphics[width=\textwidth]{images/aquarium-1.png}\timestamp{\tsAquarium}%
  %
  \caption{%
  The simple rig motions of 26 underwater sea creatures are augmented with our \emph{real-time} secondary dynamics.
  %
  The full scene of 330,563 mesh vertices and 1,293,625 tetrahedra runs at over 60 fps.
  %
  Throughout our paper, the \protect\img{images/video-symbol-1.png} indicates a corresponding clip in the supplemental video.
  \label{fig:teaser-figure}
  }
\end{teaserfigure}
\maketitle




% \Otman{This is Otman's color}
% \Eris{this is Eris' color}
% \alec{this is Alec's color}
% \Eitan{this is Eitan's color}
% \Sid{this is Sid's Color}
% \Yi{This is Yi's Color}
\section{Introduction}

The increasing complexity of source code poses a key challenge to the reliability of large-scale software systems. Software bugs in these systems can lead to safety issues~\cite{bug_safety} for users around the world as well as cause non-negligible financial losses~\cite{bug_loss}. As such, developers have to spend a large amount of time and effort on bug fixing. Consequently, \aprfull (\apr), designed to automatically generate patches to fix software bugs, has attracted wide attention from both academia and industry~\cite{long2016prophet, legoues2012genprog, long2015spr, lou2020can, tufano2018empstudy}. 


To achieve \apr, one popular approach is known as Generate-and-Validate (G\&V)~\cite{qi2015gv, ghanbari2019prapr, lou2020can, le2016hdrepair, legoues2012genprog, wen2018capgen, hua2018sketchfix, martinez2016astor, koyuncu2020fixminder, liu2019tbar, liu2019avatar}, which is typically based on the following pipeline: First, fault localization techniques~\cite{wong2016fl, abreu2007ochiai, zhang2013injecting, papadakis2015metallaxis, li2019deepfl, li2017transforming} are applied to determine the suspicious locations in programs where bugs are likely to exist. Then, the buggy locations are used by the \apr tools to generate a list of patches that replace buggy lines with correct lines. Afterward, each patch is validated against the original test suite to identify any \emph{plausible patches} (i.e., passing all tests in the test suite). Finally, to determine the \emph{correct patches}, developers examine the list of plausible patches to see if any of them can correctly fix the bug. 

Traditional \apr tools can mainly be categorized into heuristic-based~\cite{legoues2012genprog, le2016hdrepair, wen2018capgen}, constraint-based~\cite{mechtaev2016angelix, le2017s3, demacro2014nopol, long2015spr} and \template~\cite{ghanbari2019prapr, hua2018sketchfix, martinez2016astor, liu2019tbar, liu2019avatar}. Among these traditional tools, \template \apr tools~\cite{ghanbari2019prapr, liu2019tbar, benton2020effectiveness} have been able to achieve state-of-the-art results. \Template \apr tools typically leverage pre-defined templates (e.g., adding a nullness check) for bug fixing. However, since these fix templates are typically handcrafted, the number and types of bugs they are able to fix can be limited. 



To address the limitations of traditional \apr, researchers have proposed various \learning \apr tools~\cite{li2020dlfix, chen2018sequencer, jiang2021cure, lutellier2020coconut, zhu2021recoder, ye2022rewardrepair} based on the \nmtfull (\nmt) architecture~\cite{sutskever2014mt} where the input is the buggy code snippets and the goal is to translate the buggy code snippets into a fixed version. To accomplish this, \learning \apr tools require supervised training datasets with pairs of both buggy and fixed code snippets in order to learn how to perform this translation step. These training data are usually obtained by mining historical bug fixes using heuristics/keywords~\cite{dallmeier2007benchmark}, which can be imprecise for identifying bug-fixing commits; even the actual bug-fixing commits can include irrelevant code changes, leading to further pollution in the dataset~\cite{xia2022alpharepair}.
% 
Moreover, it can be hard for such \apr tools to generalize and fix bug types unseen during training. 



To better leverage recent advances in \plmfull{s} (\plm{s}), researchers~\cite{xia2022alpharepair, xia2023repairstudy, kolak2022patch, prenner2021codexws} have directly applied \plm{s} to generate patches without bug-fixing datasets. These \llm-based \apr tools work by either directly generating a complete code function~\cite{prenner2021codexws, xia2023repairstudy} or predict/infill the correct code snippet given its surrounding context~\cite{xia2022alpharepair, xia2023repairstudy}. By directly using \llm{s} that are pre-trained on billions of open-source code snippets, \llm-based \apr tools can achieve state-of-the-art performance on many repair datasets~\cite{xia2022alpharepair}. 


% 
%
%

Traditional \apr tools have long used the insight of the \emph{plastic surgery hypothesis}~\cite{barr2014plastic} where it states that the code ingredients to fix a bug already exist within the same project. Traditional \apr tools have manually designed pattern-~\cite{ghanbari2019prapr, saha2017elixir} or heuristic-based~\cite{jiang2018simfix, legoues2012genprog} approaches to finding and using such relevant code ingredients to generate fixes for bugs. However, the plastic surgery hypothesis has been largely ignored in \llm-based \apr. In fact, \llm provides a unique opportunity to fully automate the plastic surgery hypothesis idea via fine-tuning (learning project-specific information via model updates from the buggy project) and prompting (directly providing relevant code ingredients to the model), and make it directly applicable to different languages (since the \llm{s} are typically multi-lingual).%
Moreover, despite the intensive manual efforts involved, traditional \apr tools still cannot fully leverage project-specific information due to large search space for leveraging/composing existing code ingredients. In contrast, the project-specific information can effectively leveraged by \llm{s} due to their power in code understanding/vectorization, e.g., even partial/imprecise information may still guide \llm{s} in correct patch generation!
 To this end, we ask the question: \emph{How useful is the plastic surgery hypothesis in the era of \plm{s}}?








\mypara{Our Work.} To answer the question, we present \ourtech{\xspace} -- a \llm-based approach that automatically utilizes the plastic surgery hypothesis by systematically combining multiple fine-tuning and prompting strategies for \apr. \ourtech fine-tunes \plm{s} using two novel domain-specific training strategies: \textbf{\epfinetune} -- we fine-tune using the original buggy project by aggressively masking out a high percentage of tokens, which allows \plm to learn project-specific code tokens and programming styles; and \textbf{\rofinetune} -- which only masks out a single continuous code sequence per training sample, allowing the model to get used to the final \csapr task of predicting a single continuous code sequence. Furthermore, we directly leverage the ability for \plm{s} to understand natural language instructions and introduce a novel prompting strategy, \textbf{\idprompting}, which uses information retrieval and static analysis to obtain a list of relevant identifiers for the buggy lines. While such relevant identifiers are critical for fixing some difficult bugs, they may not be seen by the \llm during inference due to limited context window size. Through the use of prompting, we directly tell the model to use these extracted identifiers (relevant code ingredients) to generate the correct code. Finally, to perform repair, we combine all four model variants (including the base model, both fine-tuned models and the base model with prompting) for the final repair.





While our insight of leveraging the plastic surgery hypothesis for \llm-based \apr is generalizable across different types of \plm{s}, to implement \ourtech, we choose a recent \plm{\xspace}, \ctfive~\cite{wang2021codet5}, which is pre-trained on millions of open-source code snippets. \ctfive is an encoder-decoder model trained using \mspfull (\msp) objective where a percentage of tokens are masked out and each continuous masked token sequence is referred to as a masked span. Also, although we only extract relevant identifiers from the current buggy project (since this paper focuses on the plastic surgery hypothesis), our work can be easily extended to obtain other code information (such as relevant statements or functions) from other sources, such as  the massive pre-training corpora~\cite{husain2020codesearchnet} or historical bug-fixing datasets~\cite{jiang2019infer}, which can provide more coding knowledge for \llm{s}. Besides, although we mainly focus on using traditional string comparison algorithms for information retrieval in this paper, these techniques can be easily replaced by other frequency-based retrieval~\cite{robertson2009probabilistic} and neural search (or embedding-based search)~\cite{reimers2019sentence}.
  In summary, this paper makes the following contributions:


%


\begin{itemize}[noitemsep, leftmargin=*, topsep=0pt]
    \item \textbf{Dimension.} This paper is the first to revisit the important plastic surgery hypothesis in the era of \llm{s}. It opens up a new dimension for \llm-based \apr to incorporate previously neglected information from the buggy project itself to boost \apr performance. Furthermore, it demonstrates the promising future of retrieval-based prompting for modern \llm-based \apr.
    \item \textbf{Implementation.} We implement \ourtech based on the recent \ctfive model. We augment the model using two novel fine-tuning strategies: \epfinetune and \rofinetune, along with a novel prompting strategy based on information retrieval and static analysis: \idprompting. We combine the patches generated by all four models together and perform patch ranking to speed up \apr.% 
    \item \textbf{Evaluation Study.} We conduct an extensive evaluation against state-of-the-art \apr tools. On the widely studied \dfj 1.2 and 2.0 datasets~\cite{just2014dfj}, \ourtech is able to achieve the new state-of-the-art results of 89 and 44 correct bug fixes (15 and 8 more than best baseline) respectively.  Furthermore, we perform a broad ablation study to justify our design. \ourtech demonstrates for the first time that the plastic surgery hypothesis can substantially boost \llm-based \apr and advance state-of-the-art \apr, while being fully automated and general. Moreover, even partial/imprecise code ingredients may still effectively guide \llm{s} for \apr!
\end{itemize}



\section{Related work}
\noindent \textbf{Video foundation models.}
With sufficient computational power and an abundant source of data, there have been attempts to build a single large-scale foundation model that can be adapted to diverse downstream tasks.
Along with the success of foundations models in the natural language processing domain~\cite{brown2020language,chen2021evaluating,devlin2019bert} and in computer vision~\cite{bertasius2021space,jia2021scaling,radford2021learning}, video data has become another data type of interest, as it has grown in scale due to numerous internet video-sharing platforms.
Accordingly, several methods to train a video foundation model have been proposed.
Due to the innate multi-modality of video data, \textit{i.e.}, a combination of visual $\cdot$ vocal $\cdot$ textual context, most works have centered around the variations of the cross-modal attention mechanism \cite{akbari2021vatt,bertasius2021space,gabeur2020multi,luo2020univl,neimark2021video,tan2021look,wei2020multi,yang2021taco}.
In addition, as most video data lack proper labels or descriptions, contrastive learning methods were studied to learn meaningful feature representations or enhance video-text alignment in a self-supervised manner \cite{akbari2021vatt,kuang2021video,luo2020univl,yang2021taco}.

More specifically, MERLOT \cite{zellers2021merlot} proposed a multi-modal representation learning method for visual commonsense reasoning, which also performed well in twelve video reasoning tasks.
VATT \cite{akbari2021vatt} introduced a multi-modal learning method via contrastive learning. 
The pre-trained model performed well in a variety of vision tasks from image classification to video action recognition and zero-shot video retrieval.
Another representative work, UniVL \cite{luo2020univl} proposed a straightforward pre-training method with auxiliary loss functions. 
After fine-tuning on a specific task, the pre-trained model performed outstandingly in a wide range of tasks of text-to-video retrieval, action segmentation, action step localization, video sentiment analysis, and video captioning.
Other foundation models for multiple video tasks include \cite{li2020hero,sun2019learning,sun2019videobert,zhu2020actbert,fu2021violet,wang2022all}. 

\noindent \textbf{Auxiliary learning.}
In order to enhance the performance of one or a multitude of primary tasks, auxiliary learning methods can be incorporated.
\cite{ruder2017overview} introduced Multi-task learning (MTL) to the deep neural networks by training a single model with multiple task losses to assist learning on the main task.
Such a method is generally adapted to pre-train the foundation models in the self-supervised manner~\cite{li2020hero,sun2019learning,sun2019videobert,zhu2020actbert,fu2021violet,wang2022all}.
However, these various pretext task losses used in the pre-training phase are ignored in the fine-tuning phase, and only the primary task loss is minimized.

Recently, meta-learning methods have been introduced for auxiliary learning.
\cite{liu2019self,navon2020auxiliary,shu2019meta} proposed a meta-learning method in which the model learns auxiliary tasks to generalize well to unseen data. 
In these settings, a separate subset of data is held out as the primary task, while the others are used as auxiliary tasks that aid the primary task's performance.
Similar methods were adopted for computer vision tasks such as semantic segmentation \cite{xu2021leveraging}.
Other domain applications include navigation tasks with reinforcement learning \cite{ye2021auxiliary}, or self-supervised learning methods on graph data \cite{hwang2020self}.

\section{Background on Network Calculus}
\label{sec: background}


\begin{figure*}[tbh]
\centering
\begin{subfigure}[b]{0.3\textwidth}
    \centering
    \includegraphics[width=\linewidth]{images/in-out.png}
    \caption{Arrival and departure data and their relation with delay $d(t)$ and backlog $b(t)$. For a FIFO system, the delay is the horizontal distance between $R(t)$ and $R^*(t)$ but some other multiplexing techniques may shift the data to a later priority, causing a longer delay.}
    \label{fig: data in-out}
\end{subfigure}
\hfill
\begin{subfigure}[b]{0.35\textwidth}
    \centering
    \includegraphics[width=\linewidth]{images/arrival-service.png}
    \caption{Characteristics of an arrival curve and a service curve. From any point of observation, the arriving data never exceeds its arrival curve; the departure data is also never less than the service curve with respect to the data arrival.}
    \label{fig: arrival-service curves}
\end{subfigure}
\hfill
\begin{subfigure}[b]{0.33\textwidth}
    \centering
    \includegraphics[width=\linewidth]{images/bound.png}
    \caption{Delay and backlog bounds of a system. Backlog is the maximum vertical distance between $\alpha(t)$ and $\beta(t)$; FIFO delay is their maximum horizontal distance; but for arbitrary multiplexing, the delay guarantee is when the system clears its buffer, thus it's the intersection of $\alpha(t)$ and $\beta(t)$.}
    \label{fig: system bounds}
\end{subfigure}
\caption{Network calculus framework. We let $R(t)$ and $R^*(t)$ be the arrival and departure data flow of a system; $\alpha(t)$ be the piecewise linear concave arrival curve and $\beta(t)$ be the piecewise linear convex service curve of a system.}
% \hossein{Better to show piece-wise linear concave arrival curve and piece-wise linear convex service curve instead of token-bucket and rate-latency.}}
\end{figure*}

We recall some of the network calculus essentials for a better understanding of the framework used in Saihu. In the following context, we use the following notation: $\mbb{R}^+$ is the set of non-negative real numbers; $[x]_+$ denotes $\max(0, x)$

The data flow is by convention modeled as a left-continuous wide-sense increasing function $R(t): \mbb{R}^+ \mapsto \mbb{R}^+$ with respect to time $t$~\cite{ncbook2001leboudec}. 

A system $\mcal{S}$ receives arrival data described as a cumulative function $R(t)$ and delivers departure data as another cumulative function $R^*(t)$. Figure~\ref{fig: data in-out} illustrates such a system $\mcal{S}$. The benefit of representing a system like this is that we can observe system backlog and delay with such a model. 

\begin{definition}[Backlog and Delay~\cite{ncbook2001leboudec}]
    The backlog of a system at time~$t$ is
    \begin{equation}
        b(t) = R(t) - R^*(t)
    \end{equation}
    
    The virtual delay of a FIFO system at time $t$ is
    \begin{equation}
        d_{FIFO}(t) = \inf \lbp \tau \geq 0 : R(t) \leq R^*(t+\tau) \rbp
    \end{equation}
\end{definition}



The backlog of a system can be viewed as the vertical distance between $R$ and $R^*$. The FIFO (\textit{First-in First-out}) delay is the horizontal distance between $R$ and $R^*$. One may obtain other delay values if the multiplexing technique is not FIFO.

% \begin{figure}
%     \centering
%     \includegraphics[width=0.9\linewidth]{images/in-out.png}
%     \caption{In/out data flow; delay and backlog}
%     \label{fig: data in-out}
% \end{figure}

Since we are interested in the system guarantee instead of a single instance of data flow, we would like to have general bounds to the arrival and departure data flows. Therefore, we define \textit{arrival curve} and \textit{service curve} as the bounds of arrival and departure data flows.

\begin{definition}[Arrival Curve~\cite{ncbook2001leboudec}]
    Given a wide-sense increasing function $\alpha: \mbb{R}^+ \mapsto \mbb{R}^+$, we say that a flow $R(t)$ is $\alpha$-constrained if and only if for all $s \leq t$:
    \begin{equation}
        R(t) - R(s) \leq \alpha(t-s)
    \end{equation}
    We say $R(t)$ has $\alpha$ as an arrival curve.
\end{definition}

\begin{definition}[Service Curve~\cite{ncbook2001leboudec}]
    Given a wide-sense increasing function $\beta: \mbb{R}^+ \mapsto \mbb{R}^+$ and $\beta(0) = 0$. A system $\mcal{S}$ having $R(t)$ and $R^*(t)$ as its arrival and departure flows. We say $\mcal{S}$ offers a service curve $\beta$ if and only if
    \begin{equation}
        R^*(t) \geq (R \otimes \beta)(t) =: \inf_{s \leq t} \lbp R(s) + \beta(t-s) \rbp
    \end{equation}
    where $\otimes$ denotes the min-plus convolution
\end{definition}

Figure~\ref{fig: arrival-service curves} illustrates the arrival and service curves. Any segment of arrival flow $R(t)$ is constrained by arrival curve $\alpha$ and the output curve $R^*(t)$ is always no less than the curve $R\otimes\beta$. As a result, an arrival curve upper bounds the incoming traffic, and a service curve lower bounds the outgoing traffic.

% \begin{figure}
%     \centering
%     \includegraphics[width=\linewidth]{images/arrival-service.png}
%     \caption{Arrival/Service curve}
%     \label{fig: arrival-service curves}
% \end{figure}

We consider 2 special types of curves throughout this paper, \textit{token-bucket} (or sometimes called \textit{leaky-bucket}) curve and \textit{rate-Latency} curve.

\begin{definition}[Token-bucket and Rate-latency~\cite{ncbook2001leboudec}]
    A token-bucket curve $\gamma_{r,b}$ with arrival rate $r$ and burst $b$ is defined as
    \begin{equation}
        \gamma_{r,b}(t) = b + rt
    \end{equation}

    A rate-latency curve $\beta_{R,T}$ with service rate $R$ and latency $T$ is defined as
    \begin{equation}
        \beta_{R,T}(t) = R \lb t - T \rb_+
    \end{equation}
\end{definition}

A token-bucket curve is determined by a burst $b$ and an arrival rate~$r$. Burst represents the maximum possible data volume that can arrive simultaneously, and arrival rate represents the maximum long-term data rate~\cite{bouillard2022tradeoff}.
A rate-latency curve is determined by a latency~$T$ and a service rate~$R$. Latency represents the time a server needs before starting to process the incoming data, and service rate represents the minimum rate to process data after the initial latency.

With the help of arrival and service curves, we can derive delay and backlog bounds for a system $\mcal{S}$ illustrated in Figure~\ref{fig: system bounds}. Suppose a system $\mcal{S}$ has arrival curve $\alpha$ and service curve~$\beta$, its worst-case backlog $b^*$ is the maximum vertical distance between~$\alpha$ and~$\beta$. Similarly, depending on the multiplexing technique applied to the system, its worst-case delay bound $d^*$ is the maximum horizontal distance between $\alpha$ and $\beta$ if $\mcal{S}$ is a FIFO system. If we don't have any information about its multiplexing technique, referred to as arbitrary multiplexing, the best we can say is that when $\alpha$ and $\beta$ intersect each other, where all data has been delivered out of the system. Consequently, the worst-case delay bound for arbitrary multiplexing is the time required for $\mcal{S}$ to clear its buffer.

% \begin{figure}
%     \centering
%     \includegraphics[width=\linewidth]{images/bound.png}
%     \caption{System delay/backlog bounds}
%     \label{fig: system bounds}
% \end{figure}

While a service curve captures the slowest possible output speed of a system, a link's transmission capacity limits the speed as well. Hence, we model this phenomenon using a \textit{greedy shaper} with a sub-additive function $\sigma: \mbb{R}^+ \mapsto \mbb{R}^+$ concatenated with a server. We consider a concatenation as shown in Figure \ref{fig: system}. By convention we assume $\sigma(0) = 0$ and $\beta(t) \leq \sigma(t), \forall t \in \mbb{R}^+$, meaning that the buffer is cleared at the beginning and the service never exceed its physical limitation. With the above definition, such greedy shaper conserves the service provided by the system due to theorem \ref{thm: shaping}.

\begin{figure}[thb]
    \centering
    \includegraphics[width=0.7\linewidth]{images/system.png}
    \caption{Shaping of departure data. A flow that has an arrival curve $\alpha$ feeds into a server with an arrival data flow $R(t)$. The server having service curve $\beta$ takes $R(t)$ and gives a departure data flow $R^*(t)$ to a shaper with shaping function $\sigma$. The shaper takes $R^*(t)$ and shape the data flow as another departure $D(t)$.}
    \label{fig: system}
\end{figure}


\begin{theorem}[Shaping conserves service \cite{ncbook2001leboudec}]
\label{thm: shaping}
Following the system shown in Figure \ref{fig: system}, we have
\begin{equation}
     D = R^* \otimes \sigma \geq \lp R \otimes \beta \rp \otimes \sigma = R \otimes \lp \beta \otimes \sigma \rp = R \otimes \beta
\end{equation}
\end{theorem}

In the following context, we model the shaping function $\sigma$ as a token-bucket curve $\gamma_{C,L}$ with transmission capacity $C$ and the packet size $L$ to capture the link capacity and packetization~\cite{bouillard2022tradeoff}.

\section{Skinning Eigenmodes}
\label{sec:skinning-eigenmodes}

Our goal is to derive a suitable linear subspace so that 
full-space complementary displacements $\u^c \in \mathbb{R}^{n(d)}$ may be approximated with a smaller $m$-dimensional linear subspace:
\begin{align}
\u^c \approx  \B \z,
\label{eq:subspace-approx}
\end{align}
where the columns of $\B \in \mathbb{R}^{n(d) \times m}$ form the subspace basis, and the vector $\z \in \mathbb{R}^{m}$ are the reduced degrees of freedom optimized at run-time.

To apply subspace reduction to the complementary dynamics problem (\refeq{cd}), we would like our subspace $\B$ to simultaneously: deal well with large (global and local) rotations, well-approximate the space of low-energy displacements, and accommodate the rig-complementarity constraints.
%
\edit{We make use of a linear blend skinning subspace basis for deformation $\Blbs$ \cite{1Gilles2011} and demonstrate in the following sections how we meet these three desirable criteria.}

Linear blend skinning represents displacements as a weighted summation of $m$ affine transformations applied to a shape's rest positions. The $i$th vertex on the shape is displaced via
\begin{align}
\u_i = \sum_{b=1}^m  w_{ib} \T_b \X_i,
\end{align}
where $w_{ib} \in \mathbb{R}$ is the weight of the $b$th transformation at vertex $i$, $\T_b \in \mathbb{R}^{d \times (d+1)}$ is the $b$th transformation, and $\X_i \in \mathbb{R}^{d+1}$ is the $i$th vertex's rest position in homogeneous coordinates.
%
This equation is linear in $\T$ and so it follows that it may be rearranged so that the degrees of freedom in $\T$ are collected in a single vector $\z = \text{vec}(\T) \in \mathbb{R}^k$ with $k=d(d+1)m$ and the 
weights $w$ and rest positions $\X$ form the columns of a matrix $\Blbs \in \mathbb{R}^{n(d) \times k}$:
\begin{align}
\Blbs &= \I_{d} \otimes (( \boldsymbol{1}_m^T \otimes \X ) \odot ( \W \otimes \boldsymbol{1}_{d+1}^T) ) \ ,
    \label{eq:linear-blend-skinning-matmul}
\end{align}
where $\W \in \mathbb{R}^{n \times m}$ is a matrix with columns collecting each transformation's weights and 
$\X \in \mathbb{R}^{n \times (d+1)}$ collects homogeneous rest positions in rows.
%
Since the rest positions are generally given, the only variables in our subspace design are the weights $\W$. We now propose a method for choosing $\W$ to ensure that weights span low-energy motions \emph{and} satisfy the complementarity constraints by construction. We defer discussion of how our choice of linear blend skinning subspace directly ensures good rotational properties (see \refsec{rotations}).


%This subspace may be leveraged to reduce optimization problem over $\u$ to $\z$:
%\begin{align}
%     \argmin_{\u} E(\u)   \approx
%    \B \argmin_{\z} E(\boldsymbol{Bz})  \label{eq:linear-model-reduction} 
%\end{align}

%\alec{We need to write somehwere that we assume $f_\text{rig}$ is linear. Something like, the most common real-time rigs are linear. In games they don't even use blend shapes, \emph{everything} is linear blend skinning.}
%To achieve interactive simulation speeds, we fully decouple the simulation optimization step from the constraint complexity \alec{what is ``constraint complexity''?} and the mesh resolution \alec{This feels repetitive. Have we said this multiple times already?}.
%% What about deleting the previous sentence and just starting here:
%To this end, we propose a fast complementary dynamics pipeline composed of three main building blocks:
%\begin{enumerate}
%    \item \textbf{A skinning subspace} for elasticity that guarantees rotation equivariant simulations, can represent rotational motion \alec{I don't understand the distinction being made.}, and is material \emph{and} rig-aware, low memory, and fast to compute. (Section \ref{sec:skinning-modes})
%    \item \textbf{A clustering scheme} for fast approximation of per-tet non-linearities.(Section \ref{sec:clustering})
%    \item \textbf{A fast local-global solver} that leverages both our subspace and our clusters for a simulation step that never requires \emph{any} full space operations. (Section \ref{sec:local-global-solver})
%\end{enumerate}
%\subsection{Skinning Subspace}
%\label{sec:skinning-subspace}
%\subsubsection{Deriving Skinning Weights for Secondary Motion}
%\label{sec:deriving-skinning-weights-for-secondary-motion}
%Because we aim to use linear blend skinning as a subspace for deformation, we need to derive a set of weights $\W \in \mathbb{R}^{n \times m}$ that parameterize our subspace and provide low energy deformations.
%
%We show that these weights can be derived in a very similar fashion to traditional displacement modes using a generalized eigenvalue problem (GEVP). 

Our first step follows the process for standard modal subspaces.
%
We approximate our elastodynamic energy with a Taylor expansion about the rest state truncated to second order terms,
\begin{align}
    E(\u + \x_0) &=  E_0 + \u^T \g_0 + \frac{1}{2} \u^T \H_0 \u + \mathcal{O}(\left\|\u\right\|^3) \, ,\nonumber \\
    E(\u + \x_0) &\approx  \frac{1}{2} \u^T \H \u \, ,
\end{align}
where we have dropped the subscript for the elastic energy Hessian in the second line for readability.
Without loss of generality, we have assumed zero elastic energy and vanishing elastic forces at rest ($E_0 = 0, \, \g_0 = \boldsymbol{0}$). 

We arrive at a standard modal subspace by
adding a (mass-) orthogonality constraint  $\B^T \M \B = \I$; substituting the subspace $\u = \B \z$; assuming $\z \sim \mathcal{D}$ are sampled from an as-of-yet arbitrary distribution, and minimize the expected value of the energy over $\B$:
\edit{
\begin{align}
   \Bdisp =  &\argmin_{\B^T \M \B = \I}  \mathbb{E}_ {\z\sim \mathcal{D}} [\z^T \B^T  \H \B \z] \\ 
    =  & \argmin_{\B^T \M \B = \I} \mathrm{tr}(\B^T \H \B \mathbb{E}_ {\z \sim \mathcal{D} }[\boldsymbol{zz}^T]) \  . 
\end{align}
}
We further assume that $\z$ are independent and identically distributed (i.i.d.) samples of a normal distribution, then  $\mathbb{E}_ {\z  \overset{\text{i.i.d.}}{\sim} \mathcal{N}( \boldsymbol{0}, \boldsymbol{1})} [\boldsymbol{zz}^T] = \I$, and
    \begin{align}
    \Bdisp=  & \argmin_{\B^T \M \B = \I}  \mathrm{tr}(\B^T \H \B). \label{eq:gevp-displacement-modes-derivation}
\end{align}
%
The optimal $\Bdisp$ may be found relatively efficiently with a generalized eigenvalue solver that supports large sparse matrices.
%
The columns of $\Bdisp$ can be directly interpreted as minimal-energy \emph{displacement} eigenmodes.

Our \emph{skinning} eigenmodes follows a similar derivation but we replace the optimization over $\B$ with an optimization over the weights $\W$. While \refeq{linear-blend-skinning-matmul} may appear to define $\Blbs$ as a complicated function of $\W$, it is \emph{linear} in and separable over the columns of $\W$. Thus, we may rewrite it as
 \begin{align}
     \Blbs &= \begin{bmatrix} \A_{i, j} & \dots & \A_{d, (d+1)} \end{bmatrix} (\I_{d(d+1)} \otimes \W),
     \label{eq:weight-space-skinning-jacobian}
\end{align}
where we introduce $\A_{i, j} \in \mathbb{R}^{n (d) \times n}$, our weight-space skinning Jacobians. These map contributions of each weight for all $d(d+1)$ affine parameters to the final skinning Jacobian. 

%
We derive $\A_{i, j}$ for the $d=2$ and $d=3$
in Appendix \ref{appendix-sec:weight-space-skinning-jacobians}, but for clarity show the result for $d=3$ here:
\begin{align}
\begin{matrix}
\A_{1, 1} = \P_x \bar{\X}  &\A_{1, 2} = \P_x \bar{\Y} &\A_{1, 3} = \P_x \bar{\Z} &
\A_{1, 4} = \P_x \\
\A_{2, 1} = \P_y \bar{\X} &\A_{2, 2} = \P_y  \bar{\Y} &\A_{2, 3} = \P_y \bar{\Z} &
\A_{2, 4} = \P_y \\
\A_{3, 1} = \P_z \bar{\X}  &\A_{3, 2} = \P_z \bar{\Y} &\A_{3, 3} = \P_z \bar{\Z} &
\A_{3, 4} = \P_z
\end{matrix}
\label{eq:weight-space-skinning-jacobian}
\end{align}
where the $\P_* \in \mathbb{R}^{3n \times n}$ selection matrices concatenate to form the identity matrix $\I_{3n} = [\P_x\, \P_y\, \P_z]$ and 
$\bar{\X}, \bar{\Y}, \bar{\Z} \in \mathbb{R}^{n\times n}$ are diagonal matrices containing the the $x$, $y$ and $z$ rest position values.

 % \begin{align}
 % \A_{i, j} = \P_i \mathbb{V}_j \W 
 % \end{align}
 % Where $\P_i \in \mathbb{R}^{3n\times n}$ is a selection matrix selection out the entries corresponding to the $i$-th dimension and $\mathbb{V}_j$ is a diagonal matrix composed of


% Where $\z$ is our set of reduced space coefficients, which can now also be interpreted as a flattened vector of affine rig parameters. 

% We wish to construct a linear blend skinning subspace $\B_{\mathrm{lbs}}$ that is comprised of a set of low energy \emph{affine transformations}. We leverage that $\B_{\mathrm{lbs}}$ is \emph{fully} parameterized linearly by a set of weights $\W$. 
% \begin{align}
%     \B_{\mathrm{lbs}} = \A (\I_{d(d+1)} \otimes \W)
% \end{align}

% Where $\A$ is derived in \Otman{Appendix} and describes the use of the blend weights $\W$ for every $d(d+1)$ affine rig parameters.
Following a similar procedure as before, we add a weight space orthogonality constraint $\W^T \M_{w} \W = \I$ and assume a generic distribution $\mathcal{D}$ on our sampling of $\z \sim \mathcal{D}$ to obtain

\edit{
 \begin{align}
     \W = &\argmin_{\W^T \M_{w} \W = \I} \mathrm{tr}(  \B_{\mathrm{lbs}}^T \H \B_{\mathrm{lbs}}\mathbb{E}_{\z \sim \mathcal{D}} [\z\z^T]) . 
 \end{align}}
 %
 We now need to make assumptions on the distribution of $\z$, as these now correspond to flattened affine matrix parameters and so have some structure to their distribution.
 %
 Specifically we assume that parameters belonging to different affine matrices are i.i.d. with respect to each other, but generally allow for intradependence between parameters belonging to the same affine matrix, as measured by the covariance matrix  $\C \in \mathbb{R}^{d(d+1) \times d(d+1) }$.
 \begin{align}
    \W = &\argmin_{\W^T \M_{w} \W = \I} \mathrm{tr}\left( (\I_{d(d+1)} \otimes \W)^T \A^T \H \A (\I_{d(d+1)} \otimes \W) (\I_m \otimes \C)\right).
    \nonumber
\end{align}
Expanding out all the Kronecker products and leveraging that the trace is just a sum of diagonal entries :
 \begin{align}
     \W = &\argmin_{\W^T \M_{w} \W = \I} \mathrm{tr}\left( \W^T \left(\sum_i^{d(d+1)} \sum_j^{d(d+1)} (\A^T_i \H \A_j)  c_{ij} \right) \W  \right).
     \end{align}
 Leading to the weight-space optimization:
 \begin{align}
    \W = &\argmin_{\W^T \M_{w} \W = \I} \mathrm{tr}\left( \W^T \H_{w} \W \right)
     \label{eq:gevp-skinning-modes-derivation}
 \end{align}
where $\M_{w} \in \mathbb{R}^{n \times n}$ is the weight-space mass matrix (we use the diagonal lumped mass matrix) and 
where we call $\H_{w} \in \mathbb{R}^{n \times n}$ the \emph{weight-space} elastic energy Hessian.




\begin{figure*}
\includegraphics[width=\textwidth, keepaspectratio]{images/secondary_weights.pdf}\timestamp[-0.25cm]{\tsWeightVis}
\caption{
We generate a linear blend skinning subspace for secondary motion parameterized by a set of skinning weights. Each weight $i$ shown is independently normalized to lie between $[-1, 1]\text{abs}(\boldsymbol{W}_i)$ and centered around 0. (Top) Weights generated by solving the unconstrained generalized eigenvalue problem on a weight-space elasticity Hessian. (Bottom) Secondary skinning weights that satisfy the weight-space complementarity constraint and are orthogonal to our rig space. These are naturally rig-aware, leading to higher frequency motion. \label{fig:skinning-weights-for-secondary-motion}}
\end{figure*}

\begin{figure}
\includegraphics[width=\linewidth,keepaspectratio]{images/mode_of_first_weight.pdf}\timestamp{\tsAffineMotions}
\caption{
One secondary linear blend skinning weight could produce 12 different motions, corresponding to 12 d.o.f.s of an affine matrix. We showcase this by flexing those associated with weight \#3. \label{fig:motions-producible-by-skinning-modes}}
\end{figure}

\begin{figure}
\includegraphics[width=\linewidth,keepaspectratio]{images/transformation_optimal_weights.pdf}
\caption{ Prioritizing scaling and shearing (middle left and middle right) provides weights that are unnaturally centered around the origin. For this reason, we prioritize translations (right). \label{fig:prioritizing-affine-parameters-as-subspace-for-def}}
\end{figure}

\begin{figure}
\includegraphics[width=\linewidth,keepaspectratio]{images/clusters.pdf}
\caption{
We generate clusters to accelerate the computation of per-tet energetic non-linearities. Our clusters inherit the rig-sensitivity of our skinnning weights. \label{fig:cluster-vis}}
\end{figure}

It is important to note this is overly determined for $\W$; The same set of weights are used to specify $d(d+1)$ different types of affine motions: scales, shears and translations. As a result, the set of weights that leads to optimal translations may not be the same set of weights that lead to optimal scales or shears. We can change which of these parameters we prioritize by modifying our affine parameter covariance matrix $\C$.

We choose to prioritize translations, neglecting shears and scales entirely, which are poorly suited for deformation subspaces.
%
The logic is that shears and scales are origin-dependent.
%
This leads the optimization in \refeq{gevp-skinning-modes-derivation} to see vertices far from the origin as \emph{stiffer} than vertices that are close to it, resulting in weights that are unnaturally concentrated around the origin, and decay far away from it as shown in \reffig{prioritizing-affine-parameters-as-subspace-for-def}. 

For $d=3$, taking i.i.d. samples from the standard normal distribution of each of the three translation parameters, while neglecting shears and scales leads to a covariance matrix of the form:
\begin{align}
    \C = \I_{3} \otimes 
    \begin{bmatrix} 
    1 & 0 & 0 & 0 \\
    0 & 0 & 0 & 0 \\
    0 & 0 & 0 & 0 \\
    0 & 0 & 0 & 0 \\
    \end{bmatrix}
\end{align}
which very conveniently leads to a simplified weight space Hessian:
\begin{align}
    \H_w = \P_x^T \H \P_x +  \P_y^T \H \P_y + \P_z^T \H \P_z.
    \label{eq:weight-space-hessian}
\end{align}

\begin{wrapfigure}{r}{5.0cm}
\includegraphics[width=\linewidth,keepaspectratio]{images/H_matrix_sum.pdf}
%\caption{ A single displacement mode (top) corresponds to a low energy deformation in a rest frame. The deformation described by the mode completely changes as its underlying geometry rotates (bottom). \label{fig:didactic-rotation-representation}}
\end{wrapfigure}
The inset, unburdened by notation, more clearly shows the simplicity of deriving this final weight-space Hessian; just take the diagonal blocks for each dimension of the Hessian and sum them up.
%
For co-rotational elasticity with homogeneous materials, $\H_w$ is proportional to the mesh's cotangent Laplacian matrix.
%
Whereas heterogeneous materials distributions affect $\H_w$ non-trivially and thus also the our optimal weights $\W$.
%

With these matrices defined, our optimal skinning eigenmodes are solutions to
\refeq{gevp-skinning-modes-derivation}, found efficiently via a genearlized eigenvalue solver.
%
%Finally, we can enforce the orthogonality constraint via Lagrange multipliers and solve for the weights by solving a weight-space GEVP:
%%
%\begin{align}
%    \boxed{\H_{w} \W =\W \M_{w} \boldsymbol{\Lambda}.}
%    \label{eq:gevp-skinning-modes-unconstrained}
%\end{align}
%
Each individual skinning eigenmode --- as a linear blend skinning weight --- corresponds to $d(d+1)$ degrees of freedom and may be used to generate  $d(d + 1)$ different motions, as shown in \reffig{motions-producible-by-skinning-modes} for $d=3$.
%
% Alec: why is this important to note? Why would someone do that?
%It is important to note that we can not arbitrarily select subsets of these degrees of freedom as doing so would destroy our subspace's closure under rotations as well as its ability to capture rotations. 
%Figure \ref{fig:motions-producible-by-skinning-modes} visualizes the space of 12 affine motions described by a single weight for $d=3$. 

\subsubsection{Weight Space Complementarity Constraint}
%
At run-time, our secondary-effect displacements should satisfy $\J^T \u^c = \boldsymbol{0}$, where recall $\J \in \mathbb{R}^{3n \times \dimp}$ is the current rig Jacobian.
%
In our subspace, this becomes $\J^T \Blbs \z = \boldsymbol{0}$.
%
Without knowledge of $\J$ \emph{a priori}, our optimized skinning eigenmodes will, in general, not admit non-trivial solutions. Even if they did, enforcing this constraint at run-time leads to a more difficult constrained optimization problem.
%
Fortunately, our formulation above as a generalized eigenvalue problem allows us to 
easily add constraints to our skinning weights, thus ensuring that our modes admit non-trivial solutions but also implicitly satisfy the constraint allowing us to remove it entirely at run-time.

\citet{Zhang:CompDynamics:2020} define $f_\text{rig}(\p)$ generically. For real-time  applications, we will assume that $f_\text{rig}$ is linear (single affine handle, linear blend skinning, blendshapes, etc.) and thus has a constant rig Jacobian $\J$.
%
Given $\J$, the constraint we need to add is
%
\begin{align}
    \J^T \Blbs = \boldsymbol{0} 
\end{align}

To express this in terms of $\W$, we can again make use of our weight-space skinning Jacobian matrices from \refeq{weight-space-skinning-jacobian} (not to be confused with $\J$) and expand the constraint to act on each weight.
%
This leads to a series of constraints that our weights need to satisfy: 
%
\begin{align}
\begin{matrix}
\J^T \A_{i, j} \W = \boldsymbol{0}
\end{matrix}  \in \mathbb{R}^{\dimp \times \nummodes}
\quad \forall i \in \{1, ...,d\}, \, j  \in \{1, ..., d+1\}
\end{align}
We can stack all our constraint matrices $\J^T \A_{i, j}$:
\begin{align}
\begin{bmatrix}
\J^T\A_{1, 1} \\
\vdots \\
\J^T \A_{d, d+1}\\
\end{bmatrix} 
\W = \J_w \W = \boldsymbol{0} \in \mathbb{R}^{p(d)(d+1) \times m}
\label{eq:weight-space-complementary-constraint}
\end{align}
where we call $\J_w \in \mathbb{R}^{ \dimp (d)(d+1) \times n}$ our weight-space complementarity constraint matrix.

We can incorporate this constraint in a standard generalized eigenvalue problem by solving instead
%
\begin{align}
    \begin{bmatrix}
    \H_{w}  & \J_{w}^T \\
    \J_{w} & \boldsymbol{0}
    \end{bmatrix}
    \begin{bmatrix}
    \W \\
    \boldsymbol{\mu}
    \end{bmatrix}
    =
     \begin{bmatrix}
    \M_{w}  & \boldsymbol{0} \\
    \boldsymbol{0} & \boldsymbol{0}
    \end{bmatrix} 
        \begin{bmatrix}
    \W \\
    \boldsymbol{\mu}
    \end{bmatrix}
    \boldsymbol{\Lambda}.  
    \label{eq:gevp-skinning-modes-constrained}
\end{align}

\reffig{skinning-weights-for-secondary-motion} shows how our derived skinning weights change to accommodate the rig-complementarity constraint.


%\subsubsection{Localized Modes}
Given a locality length scale $r$ and a center of locality $\boldsymbol{c}_i$, we aim to find a set of modes that are entirely located within a specific region of the mesh $\boldsymbol{S}^T (\boldsymbol{c}_i; r) \boldsymbol{b}_i= \boldsymbol{0}$, where $\boldsymbol{S} \in \mathbb{R}^{n \times m}$ is a selection matrix that selects the m-vertices that are \emph{not} within our region of locality.

We treat the locality length scale $r$ as an input user parameter, and the centers of locality as degrees of freedom in our GEVP energy minimization. Enforcing the constraint above into our GEVP:

\begin{align}
\argmin_{\boldsymbol{b}_i, \boldsymbol{c}_i}\,&  \boldsymbol{b}_i^T \boldsymbol{H} \boldsymbol{b}_i \\  \text{s.t.} \quad \boldsymbol{S}(\boldsymbol{c}_i; \boldsymbol{r}) \boldsymbol{b}_i = \boldsymbol{0} \quad  & \boldsymbol{b}_i^T \boldsymbol{M} \boldsymbol{b}_i = 1 \quad  \boldsymbol{b}_i^T \boldsymbol{M} \boldsymbol{b}_j = 0  \quad  \forall j \neq i 
\label{eq:gevp-with-locality-minimization}
\end{align}



We minimize the above energy using block coordinate descent.
\paragraph{Global Step}

\begin{align}
\argmin_{\boldsymbol{b}_i}\,&  \boldsymbol{b}_i^T \boldsymbol{H} \boldsymbol{b}_i \nonumber  \\  \text{s.t.} \quad \boldsymbol{b}_i^T {\boldsymbol{S}^{n-1}}  = \boldsymbol{0} \quad  & \boldsymbol{b}_i^T \boldsymbol{M} \boldsymbol{b}_i = 1 \quad  \boldsymbol{b}_i^T \boldsymbol{M} \boldsymbol{B} = 0  \nonumber
\end{align}

We can introduce $\boldsymbol{u}_i = \boldsymbol{C}^T\boldsymbol{b}_i$
\begin{align}
\argmin_{\boldsymbol{u}_i}\,&  \boldsymbol{u}_i^T \boldsymbol{C}^T \boldsymbol{H} \boldsymbol{C} \boldsymbol{u}_i \nonumber  \\  \text{s.t.} \quad   & \boldsymbol{u}_i^T \boldsymbol{C}^T \boldsymbol{M}\boldsymbol{C} \boldsymbol{u}_i = 1 \quad  \boldsymbol{u}_i^T \boldsymbol{C}^T \boldsymbol{M} \boldsymbol{B} = \boldsymbol{0}  \nonumber
\end{align}

Enforcing the linear equality constraint via lagrange multipliers:

\begin{align}
\argmin_{\boldsymbol{u}_i, \boldsymbol{\mu}_i}\,&  \boldsymbol{u}_i^T \boldsymbol{C}^T \boldsymbol{H} \boldsymbol{C} \boldsymbol{u}_i + \boldsymbol{u}_i^T \boldsymbol{C}^T \boldsymbol{M} \boldsymbol{B} \boldsymbol{\mu}_i \nonumber  \\  \text{s.t.} \quad   & \boldsymbol{u}_i^T \boldsymbol{C}^T \boldsymbol{M}\boldsymbol{C} \boldsymbol{u}_i = 1 \nonumber
\end{align}

Enforcing the quadratic constraint via Lagrange multipliers:

\begin{align}
\argmin_{\boldsymbol{u}_i, \boldsymbol{\mu}_i, \lambda_i},  \,&  \boldsymbol{u}_i^T \boldsymbol{C}^T \boldsymbol{H} \boldsymbol{C} \boldsymbol{u}_i + \boldsymbol{u}_i^T \boldsymbol{C}^T \boldsymbol{M} \boldsymbol{B} \boldsymbol{\mu}_i + \boldsymbol{u}_i^T (\boldsymbol{C}^T \boldsymbol{M}\boldsymbol{C} \boldsymbol{u}_i - 1) \lambda_i \nonumber 
\end{align}

Deriving the KKT optimality conditions:

\begin{align}
2 \boldsymbol{C}^T \boldsymbol{H} \boldsymbol{C} \boldsymbol{u}_i +  \boldsymbol{C}^T \boldsymbol{M} \boldsymbol{B} \boldsymbol{\mu}_i +  2\boldsymbol{C}^T \boldsymbol{M}\boldsymbol{C} \boldsymbol{u}_i \lambda_i 
 = \boldsymbol{0} \nonumber \\
 \boldsymbol{C}^T\boldsymbol{M}   \boldsymbol{B}  \boldsymbol{u}_i = \boldsymbol{0} \\
\boldsymbol{u}_i^T\boldsymbol{C}^T \boldsymbol{M}\boldsymbol{C} \boldsymbol{u}_i = 1
 \nonumber
 \end{align}
 
Which we can rewrite as the constrained GEVP:

\begin{align}
\begin{bmatrix}
\boldsymbol{C^THC} & \boldsymbol{C}^T\boldsymbol{M} \boldsymbol{B} \\
\boldsymbol{B}^T \boldsymbol{M} \boldsymbol{C} & \boldsymbol{0}
\end{bmatrix}
\begin{bmatrix}
\boldsymbol{u}_i  \\
\boldsymbol{\mu}_i  
\end{bmatrix} =
\lambda_i
\begin{bmatrix}
\boldsymbol{C}^T \boldsymbol{M} \boldsymbol{C} &
\boldsymbol{0} \\ \boldsymbol{0} & \boldsymbol{0}
\end{bmatrix}
\begin{bmatrix}
\boldsymbol{u}_i  \\
\boldsymbol{\mu}_i  
\end{bmatrix}
\end{align}

\paragraph{Local Step}

Starting from  Equation \ref{eq:gevp-with-locality-minimization}, we can enforce our locality constraint through Lagrange multipliers:
\begin{align}
\argmin_{ \boldsymbol{c}_i, \boldsymbol{\gamma}_i}\,&  \boldsymbol{b}_i^T \boldsymbol{H} \boldsymbol{b}_i + \boldsymbol{\gamma}^T_i \boldsymbol{S}(\boldsymbol{c}_i; \boldsymbol{r}) \boldsymbol{b}_i \\  \text{s.t.}  \quad  & \boldsymbol{b}_i^T \boldsymbol{M} \boldsymbol{b}_i = 1 \quad  \boldsymbol{b}_i^T \boldsymbol{M} \boldsymbol{b}_j = 0  \quad  \forall j \neq i 
\end{align}
We then omit all terms that do not depend on our center of locality $\boldsymbol{c}_i$
\begin{align}
\argmin_{ \boldsymbol{c}_i, \boldsymbol{\gamma}_i}\,&  \boldsymbol{\gamma}_i^T \boldsymbol{S}(\boldsymbol{c}_i; \boldsymbol{r}) \boldsymbol{b}_i 
\end{align}
The above optimization problem penalizes non-zero terms that exist outside of our region of locality $\boldsymbol{S}(\boldsymbol{c}_i, r)\boldsymbol{b}_i$.  This is equivalent to maximizing the number of  non-zero terms inside our region of locality.
\begin{align}
\argmax_{ \boldsymbol{c}_i}\,&  \boldsymbol{C}(\boldsymbol{c}_i; \boldsymbol{r}) \boldsymbol{b}_i 
\end{align}
We can leverage the fact that the only possible number of $\boldsymbol{c}_i$ is finite because our problem is discrete. We can find $\boldsymbol{c}_i$ in O(n) time with the right data structure.  \Otman{Get back to this, maybe will omit this section entirely based on the decision we come to}


% \begin{center}
%   \includegraphics[width=\textwidth, keepaspectratio]{images/weights_saturdated.pdf}
%     \captionof{figure}{Skinning Weights for Secondary Motion Under Different Rigs \label{fig:skinning-weights-for-secondary-motion}}
% \end{center}




\begin{figure*}
\includegraphics[width=\linewidth,keepaspectratio]{images/hetergeneous-worm-propeller.pdf} \timestamp[-0.25cm]{\tsHeterogeneousMaterialExperiment}
\caption{A material-aware subspace more efficiently captures the space of motions available to our simulation. This directly leads to richer dynamics.\label{fig:heterogeneous-skinning-modes}}
\end{figure*}

\section{But What About Rotations?}
\label{sec:rotations}
%
Interesting elastodynamic effects exhibit rotations: both global, where the entire shape rotates in space, and local, where part or parts of the shape rotate relative to the rest/each other.
%
Rotations are notoriously difficult for previous linear subspaces.
%
For example, it is well known that displacement modes ($\Bdisp$ defined in \refeq{gevp-displacement-modes-derivation}) 
struggle to represent local rotational motion (see \reffig{local-rotation-experiment} and \reffig{rotation-fitting} (Left)) \cite{BarbicJames:RealTimeSTVK,Barbic:2011:RealTimeLargeDefoSubstructuring, RScoords, ModalWarping}). 

%
Our use case reveals that issues with rotations go beyond this and can be more insidious.
%
In the following discussion, we assume that the elastic potential $E$ is rotation invariant. That is, $E(\u + \x_0) = E(\repR (\u + \x_0))$, where multiplying by $\repR \in \mathbb{R}^{n(d) \times n(d)}$ applies the same rotation $\R \in SO(d)$ to all vertices.

\subsection{Rotation Spanning vs. Closure Under Rotations}
%
Rotational problems may be categorized into two separate issues.

First, does a given subspace span rotations?
%
By global rotation spanning, we mean there always exists some subspace parameters to reproduce any rotational displacement. If $\x_0$ are the rest positions then 
\begin{align}
\exists \ \z  \text{ such that } \repR \x_0 - \x_0 = \B \z \ \forall \R \in SO(d).
\end{align}
%
For free-flight objects, failing to span global rotations means the subspace will unnaturally deform in an attempt to minimize $E$ rather than rotate.
%
%
% Given a subspace $\B$ that does not span global rotations, it is trivial to augment it with additional columns $[\I_d \otimes \X]$ that span all global affine displacements (including rotations).
%
Unfortunately, this problem also occurs locally, too. For example, if the arms of a character bend in opposite directions, failure to span these local rotations will disturb (by introducing local shears and stretches to attempt to minimize $E$) or prevent (by the minimization of $E$ detesting such scales and shears) the desired deformation.
%
For example, \citet{BarbicJames:RealTimeSTVK} emphasize how displacement modes $\Bdisp$ induce scaling and shearing artifacts when approximating rotations and bending deformations.

Second, does a given subspace induce a rotationally equivariant simulation?
%
Treat the simulation as a map from problem specification parameters (e.g., forces, rig displacements, rest positions) to optimal (full-space) displacements.
%
Rotation equivariance means that any rotated version of the problem results in a correspondingly rotated solution:
%
\begin{align}
\forall \boldsymbol{R} \in \mathcal{SO}(d)\, ,  & \quad 
\forall \boldsymbol{x} \in \mathbb{R}^{n(d)}\, , \nonumber\\
     \boldsymbol{B} \argmin_{\boldsymbol{z} }{\boldsymbol{E}(\boldsymbol{B}\boldsymbol{z} + \repR \boldsymbol{x}}) &= 
    \repR \boldsymbol{B} \argmin_{\boldsymbol{z} }{\boldsymbol{E}(\boldsymbol{B}\boldsymbol{z} + \boldsymbol{x}}) \ ,
    \label{eq:rotation-equivariance-of-sim}
\end{align}
where --- without loss of generality --- we lump problem specification parameters into the vector $\x \in \mathbb{R}^{n(d)}$.

A subspace simulation lacking rotation equivariance may experience unpredictably different deformations under rotations.
%
This is especially problematic in a complementary dynamics setting where the entire object or large subpart may rotate due to the user rig. Users will expect analagously rotated secondary effects and be surprised by behavior that depends on global or local rotations coming from the rig.
%

 \begin{wrapfigure}[13]{r}{3.5cm}
\includegraphics[width=3.8cm,keepaspectratio]{images/bar-field-vis.pdf}\timestamp{\tsBarFieldVis}
\end{wrapfigure}
\reffig{subspace-comparisons} shows how this shortcoming expresses itself as  overly energetic deformation, while \reffig{random-init-linear-subspace} showcases some of the kinky local minima that can easily arise under simple rotations. 
%
The root of this problem is shown didactically in the inset: a single displacement mode describes a completely different type of motion if its underlying shape rotates.

%
Building on this intuition, we prove (see App.~\ref{sec:appendix-proof-sim-rotation-equivariance}) that a linear subspace simulation is rotation equivariant if and only if the subspace basis is \emph{closed under rotations}:
\begin{align}
    \forall \, \boldsymbol{R}\in \mathcal{SO}(d) \text{ and } \boldsymbol{z}\in\mathbb{R}^m \ \exists \, \boldsymbol{w}\in\mathbb{R}^m, \text{ such that }   \repR \boldsymbol{Bz} =  \boldsymbol{B} \boldsymbol{w}.
    \label{eq:rotation-equivariance-requirement} 
\end{align} 


%
%Before introducing our novel skinning subspace, we set the stage by reviewing the most common subspace method.
%%
%\subsubsection{Displacement Modes}
%Displacement modes form an $m$-dimensional subspace as the first $m$-eigenvectors of the elastic energy Hessian $\boldsymbol{H} \in \mathbb{R}^{n(d) \times n(d)}$:
%\begin{align}
%     \boldsymbol{H} \boldsymbol{B}_{disp}  &= \boldsymbol{M} \boldsymbol{B}_{disp} \boldsymbol{\Lambda} \label{eq:gevp-displacement-modes}
%\end{align}

%Because the columns of $\boldsymbol{B}$ can be viewed as displacement fields over the mesh, and our linear function is simply weighing each of these displacement fields by a scalar, we refer to this type of subspace basis as \textbf{\emph{displacement modes}}.




\subsection{Displacement Modes Simulations Are Fragile Under Rotations}
%
Displacement modes ($\Bdisp$ defined in \refeq{gevp-displacement-modes-derivation}) \cite{PentlandWilliams1989} and many of their 
improvements (e.g., \cite{BarbicJames:RealTimeSTVK}) are neither rotation spanning nor closed under rotations.
%
Rotations are a full-spectrum displacement, so any (reasonable) truncated elastic eigenspace will fail to span arbitrary global rotations (see \reffig{rotation-fitting} (Left)).
%
While --- as discussed above --- global rotation spanning has an easy fix, much effort has been made to improve local rotation spanning such as data-driven PCA bases \cite{EigenSkinKry2022}, modal derivatives
\cite{BarbicJames:RealTimeSTVK}, sub-structuring \cite{Barbic:2011:RealTimeLargeDefoSubstructuring}, or splitting the simulation into rigid and deformable components \cite{Terzopoulos1988}.

% %
% \edit{A common fix to this specific problem for free-flying objects is to embed our simulation in a rotating frame. The elastodynamics are computed in a rest frame, while the rotation is tracked explicitly (e.g. via a rigid body simulator \cite{Terzopoulos1988} or a user controlled rotating frame \cite{DyRT}) with some coupling forces between the two. However, extending this explicit solution to work with complex rigs and to harmonize with the rig-complementarity constraint remains non-trivial. }
% %
% \edit{Instead, we propose to bake the solution to this problem directly into the construction of our subspace basis $\boldsymbol{B}$}.
%
Nevertheless, large local rotations may still be problematic (see \reffig{local-rotation-experiment}).
%
Displacement modes --- except if truncated to just null modes or completely non-truncated --- are not closed under rotations (see counterexamples in \reffig{rotation-fitting} (Right) and \reffig{random-init-linear-subspace}).
%
% The column-space ``expansion'' of \citet{Tycowicz2013} converts a linear subspace (such as displacement modes) into one that is rotation spanning and closed under rotations. 
% %
% It is not clear how to adapt this filtering process to implicitly enforce constraints such as our rig complementarity.
% %
% In terms of ease of adoption into real-time animation pipelines, the method of \citet{Tycowicz2013} is tantamount to multi-weight skinning \cite{skinningcourse:2014}, which is obscure compared to the ubiquity of linear blend skinning.

% Displacement modes in general are \emph{not} closed under rotations (see \reffig{rotation-fitting} (Right)). 
% Of course, some configurations of displacement modes can satisfy this constraint, such as a set of $d$ orthogonal translations, or a \emph{full} set of displacement modes, but these configurations are too low-dimensional, computationally expensive, or tedious to find. \Sid{Yes, let's somehow (tersely) reflect this observation in the figure captions as well.}
% \Otman{Definitely! commenting out rn before submitting first draft for anonymity}

\begin{figure}
\includegraphics[width=\linewidth,keepaspectratio]{images/cthulu_local_minimum.pdf}\timestamp{\tsCthuluLocalMinimum}
\caption{
We compute the subspace at rest (top-left). A user rotates the mesh and perturbs the system with a random initial deformation.  Using displacement modes creates jarring local minimum artefacts in a rotated frame. Our skinning modes find the global minimum effortlessly, \edit{obtaining the same rest state than if we had embedded the simulation in a rigid frame. \label{fig:random-init-linear-subspace}}}
\end{figure}

\subsection{Skinning Eigenmodes Are Robust to Rotations}
\label{sec:skinning-modes}
In contrast, skinning eigenmodes are both rotation spanning and closed under rotations (see \reffig{rotation-fitting}).
%
When the complementarity constraint is absent, the first skinning eigenmode will be a constant function thus spanning all affine motions including rotations.
%
When used for fast complementary dynamics, the rig typically contains global rotations so we explicitly (and purposefully) avoid global rotation spanning.
%
We do still want and indeed observe local rotation spanning (see \reffig{local-rotation-experiment}).
%
%
%
%\alec{skinning modes are trivially rotation spanning (when not doing CD) because the first mode is constant. and exhibit good local rotation spanning see giraffe, spoon, hand1, twist, alien. skinnings modes are trivially rotation equivariant as well, see ref hand2}
%
%% We propose a linear blend skinning subspace which represents common low energy deformations, spans local rotational motion, and \emph{guarantees} rotation equivariance in its resulting simulation.
%Motivated by the fact that displacement modes cannot represent rotations, nor do they guarantee simulation rotation equivariance, we identify a linear subspace that possesses both of these qualities: linear blend skinning. For each vertex $i$, a linear blend skinning subspace is parameterized by $\boldsymbol{W}$ and can be written as:
%\begin{equation}
%    \boldsymbol{u}_i \approx f_{lbs}(\boldsymbol{T}; \boldsymbol{W}) = \sum_b^{m} w_{ib} \boldsymbol{T}_b \begin{bmatrix} \boldsymbol{x}_{0i} \\ \boldsymbol{1} \end{bmatrix} \nonumber .
%\end{equation}
We can easily show that the linear blend skinning --- and thus also skinning subspaces --- are closed under rotations. 
%
Given some rotation $\boldsymbol{R} \in \mathcal{SO}(d)$, rotating linear blend skinning's output is equivalent to rotating all of the input transformations:
\begin{equation}
\R \sum_{b=1}^m w_{ib} \T_b \X_i = 
\sum_{b=1}^m w_{ib} \R \T_b \X_i.
%\boldsymbol{R} \, f_{lbs}(\boldsymbol{T}; \boldsymbol{W}) =  \sum_b^{m} w_{ib} \boldsymbol{R}\boldsymbol{T}_b \begin{bmatrix} \boldsymbol{x}_{0i} \\ \boldsymbol{1} \end{bmatrix}   = f_{lbs}(\boldsymbol{RT}; \boldsymbol{W}).
    \label{eq:linear-blend-skinning}
\end{equation}
%
Any rotation of its output is producible by its input, as required for a rotation equivariant subspace simulation.
%
This fact was similarly utilized in previous works albeit in different settings \cite{Wang:2015:LinearSubspaceDesign,JacobsonBPS11,LangerS08}.
%
% \edit{Many prior methods propose subspaces for deformation that guarantee rotation equivariance \cite{Faure2011, 1Gilles2011, Wang:2015:LinearSubspaceDesign, Tycowicz2013}. To the best of our knowledge, we are the first to motivate our choice of subspace with whether or not it is closed under rotations. 
\edit{
There are many prior  subspaces \cite{Faure2011, 1Gilles2011, Wang:2015:LinearSubspaceDesign, Tycowicz2013}  that do not explicitly mention rotation equivariance as a criterion for the subspace simulation. In hindsight, leveraging the machinery of Appendix \ref{appenix-eq:closed-under-rotations}, we can see that since these prior subspaces are closed under rotations, those methods also maintain a rotation equivariant simulation.
}

\begin{figure}
\includegraphics[width=\linewidth,keepaspectratio]{images/bingby_local_rotation_light.pdf}\timestamp[-0.125cm]{\tsBingbyLocalRotation}
\caption{No matter how hard a user tries, the eyes of this reduced elastodynamic alien will never bend when using a small displacement mode subspace (middle). Our skinning subspace (right) enables the rotational motion. Both results use 60 degrees of freedom. 
% \Sid{Pedantry: it will presumably bend if you use enough modes, approximately the full set. We should clarify that this observation applies to bases that are relatively small, for some suitable definition of small.} 
\label{fig:local-rotation-experiment}} 
\end{figure}

\begin{figure*}
\includegraphics[width=\linewidth,keepaspectratio]{images/rotation_fitting.pdf}
\caption{ \textbf{Rotation Spanning vs. Closure Under Rotations.} Given some initial shape, we optimize for optimal displacements that minimize the squared distance between each vertex position and its rotated target. (Left) Our skinning modes are \textbf{rotation spanning}, as modulated by the skinning weights. With a single constant skinning weight, we can perfectly reconstruct (by least squares projection) any rotation of the rest shape. Displacement modes do not span rotational motion, even with excessively abundant modes. (Right-red) Our skinning modes are \textbf{closed under rotations}, so any deformation in the span of those can be reconstructed (by least squares projection) under the same set of modes even if the mesh is arbitrarily rotated by a user. \edit{The same cannot be said for even an impractically large number of displacement modes even if augmented with affine degrees of freedom  (right-purple).}
 \label{fig:rotation-fitting}}
\end{figure*} % great figure eris! And great caption :)

% \begin{figure}
% \includegraphics[width=\linewidth,keepaspectratio]{images/hand_eq_fitting.pdf}
% \caption{ 
%  \label{fig:equivariance-fitting}  Our subspace is closed under rotations, so any deformation in it can be reconstructed (by least squares projection) even if the mesh is arbitrarily rotated by a user. The same cannot be said for even an impractically large number of displacement modes.}
%  \end{figure}
\section{Method}
\label{sec:method}

% \ml{``Inconsistent'' to ``large variation''}

% In this section, we propose our methods based on the observations in Section \ref{sec:motivation}.
In this section, we propose two techniques to further enhance the strong baseline to capture the variation of activation distributions better.
We first introduce spatial re-scaling to adapt the network to pixel-to-pixel variation.
We then propose channel-wise shifting and re-scaling to better capture the channel-to-channel variation.
Meanwhile, as both of the two methods are image-dependent, the image-to-image variation can be captured naturally.
By combining the two methods with our strong baseline, we build our enhanced BNN for SR, named EBSR.

% Because the activation distributions among pixels, channels and images have large variations \red{**are highly inconsistent} in SR networks, we introduce spatial re-scaling to adapt to pixel-wise variations and channel shift and re-scaling to adapt to channel-wise variations. And both of them are image-dependent to adapt to image-wise variations, which means during inference our network re-scales and shifts the distributions of activations flexibly for different input images. Based on these methods, we build an enhanced binary neural network for image super-resolution (EBSR).

% According to [3], the difference of activation magnitudes indicates different scaling factors are needed for each pixel.

\subsection{Spatial Re-scaling}
% It is better to use different scaling factors for different pixels to reduce the quantization error and retain more detailed information for image super-resolution. 

% \ml{In the main method, we do not need to introduce the previous works but can focus on introducing our own method. Channel rescaling in Real-to-binary Net is not relevant in this context.}

% Re-scaling the output of binary convolutions was proposed at the birth of BNN in XNOR-Net \cite{rastegari2016xnor} to reduce quantization error and improve accuracy for image classification tasks.
% It is computed as below:
% \begin{equation}
% \mathcal{A} * \mathcal{W} \approx(\operatorname{sign}(\mathcal{A}) \circledast \operatorname{sign}(\mathcal{W})) \odot \mathcal{K} \alpha
% \label{eq:xnor-net rescale}
% \end{equation}
% where $\circledast$ denotes the binary convolution and $\odot$ denotes the element-wise multiplication.
% $\mathcal{A}$, $\mathcal{W}$, $\alpha$, and $\mathcal{K}$ denote the activation, weight, weight scaling factor, and activation scaling factor, respectively.
%  Later in XNOR-Net++ \cite{bulat2019xnor}, Bulat et al. fuse the activation and weight scaling factors into a single one that is learned end-to-end based on gradients and this improves the classification accuracy on ImageNet dataset.

% % It is computed as Eq.~\ref{eq:xnor-net rescale}, where $\circledast$ denotes 
% %  the binary convolution and $\odot$ denotes the element-wise multiplication. The binary convolution of $\mathcal{A}$ and $\mathcal{W}$ is rescaled by the weight scaling factor $\alpha$ and the activation scaling factor $\mathcal{K}$, both of which are calculated analytically.


% \zc{Similarly, you should explain the meaning of A, W and the operators $\circledast$ in the formula}
% Then in Real-to-binary Net \cite{martinez2020training}, Martinez et al. used a data-driven channel re-scaling module that takes the pre-convolution activations as input to predict the activation scaling factor. Unlike that in XNOR-Net++ \cite{bulat2019xnor}, these scaling factors are not fixed during inference but rather inferred from data. By doing this, they further improved the classification accuracy on ImageNet over XNOR-Net++. 
As is shown in Figure \ref{fig:pixel}, activation distributions have large pixel-to-pixel variation in SR networks
and the difference of activation magnitudes indicates different scaling factors are preferred for different pixels.
Inspired by \cite{martinez2020training}, we propose spatial re-scaling to better adapt the network to the spatial variation
of activation distributions in SR networks.
% fit the various pixel-wise distributions in SR networks.
We take the real-valued activations $A$ before convolution as input and predict pixel-wise scaling factors $S(A)$, which re-scale the binary convolution output. Spatial re-scaling process can be formulated as follows:
\begin{equation}
A * W \approx(\operatorname{sign}(A) \circledast \operatorname{sign}(W)) \odot \alpha \odot S(A)
\label{eq:spatial rescale}
\end{equation}
where $\circledast$ denotes 
the binary convolution and $\odot$ denotes the element-wise multiplication. $A$, $W$, $\alpha$, and $S\left(A\right)$ denote real-valued activations, weights, the scaling factor of weights, and the spatial-wise scaling factor of activations respectively. $S\left(A\right) \in \mathbb{R}^{1\times H\times W}$ can be calculated with a convolution and a sigmoid function.
% as $\sigma\left( CONV\left(A\right)\right)$. 
As shown in Figure \ref{fig:method}(a), real-valued activations first go through a convolution layer,
which has an input channel of $C$ and an output channel of 1, 
and then pass through a sigmoid function to produce the scaling factors $S(A)$ along the spatial dimension.
During inference, the scaling factor will change dynamically according to different input feature maps.
By re-scaling binary convolution output using $S(A)$, we can reduce the quantization error and the original pixel-wise information in FP activation
will be preserved much better.
Spatial re-scaling leads to a large PSNR improvement of 0.24 dB (from 30.30 dB to 31.54 dB) on Set5 and 0.22 dB (from 25.09 dB to 25.31 dB)
on Urban100 compared with our strong baseline. 

\subsection{Channel-wise Shifting and Re-scaling}

\begin{table}[!tb]
\centering
\caption{Comparison between whether to fuse channel-wise shifting and re-scaling or not based on our baseline with spatial re-scaling. }
\label{tab:fusing}

\scalebox{0.65}{
\begin{tabular}{c|cc|cc|cc}
\hline
\multirow{2}{*}{Method}     & \multirow{2}{*}{OPs} & \multirow{2}{*}{Params} & \multicolumn{2}{c|}{Set5} & \multicolumn{2}{c}{Urban100} \\ \cline{4-7} 
                            &                      &                         & PSNR        & SSIM        & PSNR          & SSIM         \\ \hline
Baseline + spatial re-scale & 2.16G                & 0.05M                   & 31.54       & 0.883       & 25.31         & 0.759        \\
+ channel-wise shift and re-scale             & 2.34G                & 0.09M                   & 31.61       & 0.885       & 25.35         & 0.761        \\
+ w/ fusing                   & 2.27G                & 0.08M                   & \textbf{31.64}       & \textbf{0.885}       & \textbf{25.36}         & \textbf{0.761}        \\ \hline
\end{tabular}
}
\end{table}

In SR networks, activation distributions exhibit larger channel-to-channel variation (Figure \ref{fig:chl}).
Both the mean and magnitude of the activation distributions vary significantly across channels.
% Thus we use channel-wise shifting and re-scaling to adapt to various channel-wise distributions. 
\cite{martinez2020training} has proposed the data-driven channel re-scaling, 
but our method differs from them in further introducing data-driven thresholds to handle the channel-wise variation of both mean and magnitude.
Since the blocks to generate the scaling factors and thresholds are very similar, we further propose to fuse them into one module.
% and fusing channel-wise shifting and re-scaling into one module.
We evaluate the effect of fusing the two blocks in Table \ref{tab:fusing}.
With channel-wise shifting and re-scaling fused, our models have fewer operations and parameters overhead and slightly higher performance.

For the specific process, we take the real-valued activations as input and predict different thresholds and scaling factors for each channel. They are also image dependent, e.g., $\beta_{i}$ in Eq.\ref{eq:act_binarize} is no longer fixed during inference but generated according to different input feature maps. Channel-wise shifting and re-scaling can be formulated as follows:
\begin{equation}
A * W \approx(\operatorname{sign}(A-C_s(A)) \circledast \operatorname{sign}(W)) \odot \alpha \odot C_r(A)
\label{eq:channel-wise_shift_and_rescale}
\end{equation}
where $\circledast$ denotes 
the binary convolution and $\odot$ denotes the element-wise multiplication. $C_s(A), C_r(A) \in \mathbb{R}^{C\times1\times1}$ denote the channel-wise threshold and scaling factor, respectively. 
We show the block diagram in Figure \ref{fig:method}(b).
The real-valued input feature map is first squeezed to a ${C\times1\times1}$ vector by a global average pooling (GAP) layer.
The subsequent fully connected layers and ReLU learn the channel-wise information and output a ${2C\times1\times1}$ vector.
Then the ${2C\times1\times1}$ vector is split into two ${C\times1\times1}$ vectors.
We use the first $C$ channels as the channel-wise bias and pass the last $C$ channels through a sigmoid layer 
as the channel-wise scaling factor, which are used to shift the real-valued activations and re-scale the binary convolution output, respectively. 


% \ml{We can mention previously, channel-wise re-scale has been proposed. We propose to fuse them. Add the comparison between fuse v.s. no fuse.}

\begin{figure}[!tbp]%
  \centering
    \includegraphics[width=0.4\textwidth]{fig/methods.png}
  
% \subfloat[channel-wise shifting\&re-scale]{
%     \label{subfig:channel-wise shifting and re-scale}
%     \includegraphics[width=0.2\textwidth]{fig/chl shift and rescale.png}
%   }

  \caption{Block diagram for spatial re-scaling, and channel-wise shifting and re-scaling.} 
  % Input A is the real-valued activation tensor and C, H, and W denote its dimension. GAP stands for global average pooling. The reduction ratio r is set to 16 for a better trade-off between the performance and the number of operations and parameters.}
  \label{fig:method}
\end{figure}


\subsection{Network Structure}

Combining the spatial re-scaling and the channel-wise shifting and re-scaling methods, we construct the enhanced convolution layer (E-Conv).
Then we build our EBSR model based on E-Conv.
In Figure \ref{fig:E-conv}, we compare the binary convolution layer used in the baseline network and our proposed E-Conv.
We use spatial and channel-wise scaling factors to re-scale the binary convolution output,
and use channel-wise shifting to learn appropriate thresholds for each channel before binarization.
The scaling factors and threshold used in E-Conv are learnable and depend on the real-valued input activations.
In this way, our proposed EBSR can adapt to pixel-to-pixel, channel-to-channel, and image-to-image variations
to reduce the large binarization error and preserve more details.
% In this way, our proposed E-Conv reduces the large quantization error caused by binarization and keeps the original information of input feature maps to a large extent.


\begin{figure}[!tb]%
  \centering

    \includegraphics[width=0.5\textwidth]{fig/E-conv.png}

  \caption{Comparison of (a) the binary convolution layer with a skip connection used in our baseline network and (b) the proposed E-Conv.}
  \label{fig:E-conv}
\end{figure}


Figure \ref{fig:network} shows the basic block based on the E-Conv and our EBSR composed of the basic blocks. Following existing works, the convolution layers in the head and tail modules are not binarized. We choose the lightweight EDSR which has 16 basic blocks and 64 channels, and EDSR which has 32 basic blocks and 256 channels as our backbones, which correspond to EBSR-light and EBSR, respectively.

\begin{figure}[!tb]%
  \centering
  {
    \includegraphics[width=0.35\textwidth]{fig/network.png}
  }
  
  \caption{The structure of our proposed EBSR.  Convolution layers in purple are real-valued vanilla 3x3 convolutions.}
  \label{fig:network}
\end{figure}
\section{Implementation}
\label{sec:impl}

At \company, we have deployed \sysname in our internal clusters to serve daily DL workloads.
The internal clusters consist of heterogeneous GPUs, including NVIDIA T4 GPU and NVIDIA A10 GPU.
Integrated with Kubernetes~\cite{k8s}, \sysname manages thousands of GPUs in each cluster and more than 20,000 GPUs in all.

\parabf{Service manager.}
For online workloads, we use the existing service manager at \company which deploys containers, discovers service, and autoscales horizontal pods.

\parabf{Global manager.}
We modify the Kubernetes scheduler to schedule offline workloads.
The workload profiler takes several dedicated GPUs, whose number is negligible to the total number of GPUs.
When a new offline workload comes, the workload profiler performs a few dry runs of the workload and utilizes the NVIDIA Data Center GPU Manager (DCGM) tools~\cite{dcgm} and NVIDIA Management Library (NVML)~\cite{nvml} libraries to collect GPU metrics.
We collect about 2,000 data for each GPU type to train the speed predictor.
The MLPs of the speed predictor have four layers with hidden size $64\times 64$.
The MLPs are trained with momentum SGD optimizer~\cite{ruder2016overview} in PyTorch v1.8.0~\cite{paszke2019pytorch} until they converge.
\sysname invokes the scheduler periodically to schedule all offline workloads.
When moving workloads, we record checkpoints of offline workloads and restart the workloads after transmitting the models and checkpoints.
As the datasets are usually colossal, we store the datasets in a remote file system and fetch data during the execution.
We implement the scheduler as a third-party plugin to the Kubernetes scheduler.


\parabf{Local executor.}
Each local executor executes online workloads according to the service manager and offline workloads according to the global manager.
DL workloads are executed in Docker containers with our customized components.
We add Best-Effort GPU DevicePlugin in Kubernetes and relevant control paths with Kubelet and \sysprobe for offline workloads.
To control SM percentage, we leverage the environment variable $CUDA\_MPS\_ACTIVE\_THREAD\_PERCENTAGE$ provided by MPS.
The GPU monitor collects resource metrics through DCGM~\cite{dcgm} and NVML~\cite{nvml} for NVIDIA GPU.
The \sysprobe updates the state machine with the collected resource metrics and empirically-set thresholds.
When the state is unhealthy, the \sysprobe will ask the NodeManager in Kubernetes to evict offline workloads.
\bytecuda intercepts nearly 800 CUDA driver APIs for GPU memory allocation and kernel launch.
The GPU memory quota of offline workloads is fixed to $40\%$ as Figure~\ref{fig:motiv_gpu_resource} reports that most online workloads use less than $60\%$ GPU memory.
We adopt the cpuset of Cgroup for CPU isolation.
For memory, \sysname will evict offline workloads if memory usage is higher than a threshold or the kernel swap daemon is busy for a long time.
The parameters to calculate GPU load in Equation~\ref{equ:gpu_load}$\&$\ref{equ:clock_factor} are empirically selected through trial-and-error.

We present in section~\ref{ssec:faces} an application of PnP-HVAE on face images, using a pretrained state-of-the-art hierarchical VAE. 
Next, we study the application of our framework to natural images. To that end, we introduce  in section~\ref{ssec:patchVDVAE}  a patch hierachical VAE architecture, that is able to model natural images of different resolutions. In section~\ref{ssec:app_nat}, we provide deblurring, super-resolution and inpainting experiments to demonstrate the relevance of the proposed method.

Additional results are presented in Appendix~\ref{app:add}. All experiments can be reproduced using the code available at \url{https://github.com/jprost76/PnP-HVAE}.



\subsection{Face Image restoration (FFHQ)}\label{ssec:faces}
We first demonstrate the effectiveness of PnP-HVAE on highly structured data, by performing face image restoration.
Latent variable generative models can accurately model structured images such as face images \cite{karras2019style,vahdat2020nvae,child2021very,kingma2018glow}, and then be used to produce high quality restoration of such data. 
In our experiments, we use the VDVAE model of~\cite{child2021very}, pre-trained on the FFHQ dataset~\cite{karras2019style}, as our hierarchical VAE prior.
VDVAE has $L=66$ latent variable groups in its hierarchy and generates images at resolution $256\times256$.

We compare PnP-HVAE with the intermediate layer optimization algorithm (ILO)~\cite{daras2021intermediate} that is based on a different class of generative models than HVAE. ILO is a GAN inversion method which optimizes the image latent code along with the intermediate layer representation of a StyleGAN to generate an image consistent with a degraded observation.
We use the official implementation of ILO, along with a StyleGAN2 model~\cite{karras2020analyzing, stylegan2pytorch}, that was trained for 550k iterations on images of resolution $256\times256$ from FFHQ.  
As VDVAE and StyleGAN models are not trained on the same train-test split of FFHQ, we chose to evaluate the methods on a subset of 100 images from the CelebA dataset~\cite{liu2018large}. 
For super-resolution, the degradation model corresponds to the application of a gaussian low-pass filter followed by a $\times 4$ sub-sampling, and the addition of a gaussian white noise with $\sigma=3$.
For the deblurring, we considered motion blur and  gaussian kernels, both with a noise level $\sigma=8$. %

We provide quantitative comparisons in table~\ref{table:comp_ILO}, along with a visual comparison of the results in figure~\ref{fig:face_restoration}.
PnP-HVAE has the best  PSNR and SSIM results for all the considered restoration tasks, while ILO provides better results  for the perceptual distance.
By jointly optimizing the image and its latent variable, PnP-HVAE provides  results that are both realistic and consistent with the degraded observation.
On the other hand,  ILO  only optimizes on an extended latent space. This method generates  sharp and realistic images with better LPIPS scores,   
but the results lack  of consistency with respect to the observation, which explains the overall lower PSNR performance. 






\subsection{PatchVDVAE: a HVAE for natural images}\label{ssec:patchVDVAE}
Available generative models in the literature operate on images of  fixed resolutions and
are either restrained to datasets of limited diversity, or even to registered face images~\cite{kingma2018glow,child2021very, vahdat2020nvae, karras2019style}, or requiring additional class information~\cite{brock2018large, dhariwal2021diffusion, song2020score, luhman2022optimizing}.
Fitting an unconditional model on natural images appears to be a more difficult task, as their resolution can change, and their content is highly diverse.
The complexity of the problem can be reduced by learning a prior model on patches of reduced dimension. 
For image restoration problems, the patch model can be reused on images of higher dimensions~\cite{zoran2011learning,prost2021learning,altekruger2022patchnr}. When the model is a full CNN, the prior on the set of the  patches can  be computed efficiently by applying the network on the full image~\cite{prost2021learning}.

We thus introduce  patchVDVAE, a fully convolutional hierarchical VAE.
Contrary to existing HVAE models whose resolution is constrained by the constant tensor at the input of the top-down block, patchVDVAE can generate images of different resolutions by controlling the dimension of the input latent. 
This amounts to defining a prior on patches whose dimension corresponds to the receptive field of the VAE. A similar model is used for image denoising in~\cite{prakash2021interpretable}.

 
For PatchVDVAE architecture, we use the same bottom-up and top-down blocks as VDVAE~\cite{child2021very}, and replace the constant trainable input in the first top-down block by a latent variable, to make the model fully convolutional (details on the  architecture are given in Appendix~\ref{app:details}). 
The training dataset is composed of $128\times 128$ patches extracted from a combination of DIV2K~\cite{agustsson2017ntire} and Flickr2K~\cite{Lim_2017_CVPR_workshops} datasets.
We perform data augmentation by extracting  patches at $3$ resolutions: HR-images and $\times 2$ and $\times 4$ downscaled images. 
The model is trained for $7.10^5$ iterations with a batch size of $64$. Following the recommendation of~\cite{hazami2022efficient}, we use Adamax optimizer with an exponential moving average and gradient smoothing of the variance.
We set the decoder model to be a gaussian with diagonal covariance, as in~\cite{luhman2022optimizing}.
PatchVDVAE is fully convolutional and can generate images of dimension that are multiples of $64$ as illustrated by
figure~\ref{fig:vdvae}.

\newlength{\patchwidth}
\setlength{\patchwidth}{0.135\columnwidth}
\begin{figure}[!ht]
    \centering
    \begin{subfigure}[t]{.34\columnwidth}\hspace{0.1cm}
        \setlength{\tabcolsep}{0.02pt}
\renewcommand{\arraystretch}{0}
        \begin{tabular}{*{2}{p{1.03\patchwidth}}}
            \includegraphics[width=\patchwidth]{figures_arxiv/patchVDVAE/samples/generated/64x64/setup-5-image-0018.png} &
            \includegraphics[width=\patchwidth]{figures_arxiv/patchVDVAE/samples/generated/64x64/setup-5-image-0016.png} \\
            \includegraphics[width=\patchwidth]{figures_arxiv/patchVDVAE/samples/generated/64x64/setup-5-image-0008.png} &
            \includegraphics[width=\patchwidth]{figures_arxiv/patchVDVAE/samples/generated/64x64/setup-5-image-0019.png}   
        \end{tabular}
    \end{subfigure}\hspace{-0.15cm}
    \begin{subfigure}[t]{.64\columnwidth}
\begin{tabular}{cc}\vspace{-0.1cm}
\includegraphics[width=2\patchwidth]{figures_arxiv/patchVDVAE/samples/generated/256x256/setup-2-image-0009.png}&
        \includegraphics[width=2\patchwidth]{figures_arxiv/patchVDVAE/samples/generated/256x256/setup-2-image-0002.png}\end{tabular}

    \end{subfigure}
    \caption{\label{fig:vdvae} Left: $64\times64$ patches samples from our patchVDVAE model trained on patches from natural images.
    Right: PatchVDVAE is fully convolutional and it can generate images of higher resolution (here: $128\times128$).\vspace{-0.2cm}}
\end{figure}

\subsection{Natural images restoration}\label{ssec:app_nat}
We  evaluate PnP-HVAE on natural image restoration.
For each task, we report the average value of the PSNR, the SSIM, and the LPIPS metrics on $20$ images from the test set of the BSD dataset~\cite{MartinFTM01}.\\


\noindent
{\bf Image deblurring.}
In the experiments, we consider $2$ gaussian kernels and $2$ motion blur kernels from~\cite{levin2009understanding}, with $3$ different noise levels 
$\sigma \in \{2.55, 7.65, 12.75\}$.
As a baseline we consider  EPLL~\cite{zoran2011learning}, which learns a prior on image patches with a gaussian mixture model.
We also compare PnP-HVAE  with PnP-MMO and GS-PnP, $2$ competing convergent Plug-and-Play methods based on CNN denoisers.
PnP-MMO~\cite{pesquet2021learning} restricts the denoiser to be contraction in order to guarantee the convergence of the PnP forward-backard algorithm. GS-PnP~\cite{hurault2022gradient} considers a gradient step denoiser and reaches state-of-the-art performances of non converging methods~\cite{zhang2021plug}.
We set the temperature $\tau$  in our method as $0.95$, $0.8$ and $0.6$ for noise levels $2.55$, $7.65$ and $12.75$ respectively, and we let it run for a maximum of $50$ iterations. 
For the three compared methods we use the official implementations and pre-trained models provided by the respective authors. 
Details on the choice of hyperparameters for the concurrent methods are provided in the Appendix~\ref{app:details}
Figure~\ref{fig:deblurring_bsd} illustrates that our method provides correct deblurring results. 

According to table~\ref{tab:deb}, the performance of PnP-HVAE is between those of EPLL and GS-PnP and it outperforms PnP-MMO for large noise levels.\\

\begin{table}
\begin{center}\footnotesize
    \begin{tabular}{>{\centering}m{.3cm}*{5}{c}}
    $\sigma$ &Method & PSNR$\uparrow$ & SSIM$\uparrow$ & LPIPS$\downarrow$  \\ 
    \hline
    \multirow{4}{*}{\vcell{$2.55$}}
    & PnP-HVAE & $27.75$ & $0.79$ & $0.31$\\
    & GS-PNP \cite{hurault2022gradient} & $\mathbf{29.59}$ & $\mathbf{0.84}$ & $\mathbf{0.22}$\\
    & EPLL \cite{zoran2011learning} & $26.49$ & $0.71$ & $0.36$\\ 
    & PnP-MMO \cite{pesquet2021learning} & $\underbar{29.50}$ & $\underbar{0.83}$ & $\underbar{0.20}$ \\ \hline
    \multirow{4}{*}{\vcell{$7.65$}}
    & PnP-HVAE & $\underbar{26.36}$ & $\underbar{0.72}$ & $\underbar{0.40}$\\
    & GS-PNP \cite{hurault2022gradient} & $\mathbf{27.33}$ & $\mathbf{0.77}$ & $\mathbf{0.31}$\\
    & EPLL \cite{zoran2011learning} & $24.04$ & $0.66$ & $0.45$ \\ 
    & PnP-MMO \cite{pesquet2021learning} & $25.34$ & $0.69$ & $0.34$\\
    \hline
    \multirow{4}{*}{\vcell{$12.75$}}
    & PnP-HVAE & $\underbar{25.12}$ & $\mathbf{0.73}$ & $\underbar{0.47}$\\
    & GS-PNP \cite{hurault2022gradient} & $\mathbf{26.32}$ & $\mathbf{0.73}$ & $\mathbf{0.37}$\\
    & EPLL \cite{zoran2011learning} & $23.28$ & $0.61$ & $0.51$ \\ 
    & PnP-MMO \cite{pesquet2021learning} & $22.42$ & $0.53$& $0.54$ \\
    \hline
    &\vspace*{-.3cm}\\
            \multicolumn{2}{c}{Blur and motion kernels}& \multicolumn{3}{c}{
        \includegraphics*[scale=1]{figures_arxiv/kernels/4.png}\;\includegraphics*[scale=1]{figures_arxiv/kernels/7.png}\;\includegraphics*[scale=1]{figures_arxiv/kernels/9.png}\;\includegraphics*[scale=1]{figures_arxiv/kernels/11.png}} 
    \end{tabular}
        \caption{\label{tab:deb}Comparison  of PnP-HVAE  and other restoration methods on deblurring. Results are averaged on $4$ kernels.\vspace{-0.2cm}}% on image deblurring.}
    \end{center}
\end{table}

\begin{figure}
    
    \begin{subfigure}[h]{\linewidth}
        \centering
        \includegraphics*[width=\columnwidth]{figures_arxiv/deb_s255_k7.pdf}\vspace{-0.1cm}
        \caption{Gaussian blur, $\sigma=2.55$}
    \end{subfigure}
    \begin{subfigure}[h]{\linewidth}
        \centering
        \includegraphics*[width=\columnwidth]{figures_arxiv/deb_s765_k11.pdf}\vspace{-0.1cm}
        \caption{Motion blur, $\sigma=7.65$}
    \end{subfigure}\vspace*{-0.1cm}
    \caption{\label{fig:deblurring_bsd} Natural image deblurring\vspace{-0.1cm}}
\end{figure}

\noindent {\bf Effect of the temperature.}
PnP-HVAE gives control on the temperature of the prior over the latent space.
In figure~\ref{fig:temp_effect}, we illustrate that reducing the temperature increases the strength of the regularization prior. In this example the tuning $\tau=0.7$ produces the best performance.\\
\begin{figure}[!ht]
   
    \includegraphics[width=\columnwidth]{figures_arxiv/demo_temp.pdf}\vspace{-0.15cm}
    \caption{ \label{fig:temp_effect} Effect of the temperature in PnP-VAE on a deblurring problem, with $\sigma=7.65$.\vspace{-0.15cm}}
\end{figure}


\noindent
{\bf Image inpainting.}
Next we consider the task of noisy image inpainting. 
We compose a test-set of 10 images from the validation set of BSD~\cite{MartinFTM01} and we create masks
  by occluding diverse objects of small size in the images. 
A gaussian white noise with $\sigma=3$ is added to the images.
As a comparaison, we still consider GS-PnP and EPLL.
For PnP-HVAE, the temperature is set to $\tau=0.6$, and the algorithm is run for a maximum of $200$ iterations, unless the residual $||\x_{k+1}-\x_k||$ is on a plateau.
We provide on Table~\ref{tab:inpainting_bsd} the distortion metrics with the ground truth, as well as a visual
\begin{table}



\begin{center}
    \begin{tabular}{cccc}
        & PSNR$\uparrow$ & SSIM$\uparrow$ &LPIPS$\downarrow$ \\\hline
        PnP-HVAE  & $\mathbf{29.54}$ & $\mathbf{0.93}$ & $\mathbf{0.06}$\\
        GS-PNP & $28.52$ & $\mathbf{0.93}$ & $0.09$\\
        EPLL & $\underline{29.16}$ & $\mathbf{0.93}$ & $\mathbf{0.06}$\\
    \end{tabular}
    \caption{\label{tab:inpainting_bsd}Quantitative evaluation for inpainting on BSD.}
    \end{center}
\end{table}
comparison on figure~\ref{fig:inpainting_bsd}. 
With its hierarchical structure,  PnP-HVAE outperforms the compared methods. \vspace{0.05cm}



\begin{figure}[!h]
    \includegraphics[width=\columnwidth]{figures_arxiv/demo_inp_bsd2.pdf}\vspace{-0.1cm}
    \caption{\label{fig:inpainting_bsd}Natural image inpainting\vspace{-0.3cm}}
\end{figure}











\section{Results}
\label{results}

\begin{figure*}[ht]
    \centering
    \includegraphics[scale=0.15,trim={0 2.5cm 0 5cm},clip]{images/aoi-single_burst}
    \caption{The time average peak Age of Information with burst and \gls{soa} loss values against the dynamic reliability logic for different network topologies.}
    \label{fig:aoi_burst}\vspace{-0.4cm}
\end{figure*}


This paper focuses on both transport layer and application layer metrics to determine the feasibility of dynamic reliability. For this, we have selected the session packet volume, as transmitted, retransmitted, lost and backlogged packets as \glspl{kpi} for the transport layer; while focusing on the \gls{aoi} for the application layer. The \gls{aoi} was chosen as a crucial indicator for the freshness of packets in real-time applications. More specifically, this work adopts the time average peak \gls{aoi} equation \cite{aoi_equation} depicted in Eq. \ref{aoi}, where $\Delta(r_{i+1})$ is the $i$th update at the time it was received at the server, for a session time period of $\tau$.

\begin{equation}
    \label{aoi}
    \gls{aoi}_\tau = \frac{1}{n-1}\sum_{i=1}^{n-1} \Delta(r_{i+1})
\end{equation}

We include a comparison between the vanilla QUIC implementation which does not enjoy the dynamic reliability extension, with a number of dynamic reliability policies. The tests were run a number of times for statistical significance, with the mean value of vanilla implementation used as a baseline for comparison. The topology utilised both random loss and bursty loss to explore the bounds of dynamic reliability. The \gls{soa} loss in the figures correspond to the loss values presented in Table. \ref{tab:path_char}, for ease of comparison between bursty and random loss scenarios.

\subsection{Transport-Layer KPIs}

To analyse the performance gain at the transport layer due to dynamic reliability, the volume of transmitted and backlogged packets is examined. The figures are in the form of boxplots, which take the vanilla implementation as a benchmark, depicted as the red dashed line.

As seen in Fig. \ref{fig:sent_burst}, the loss plays a crucial role in the performance of the reliability policies. The policies under random loss did incredibly well for the networks with a larger capacity, namely \gls{mmwave} and Sub-6~GHz, whereas for burst loss, the lower network capacities had a larger packet reduction. With the increase in burst loss, the behaviour of the set split reliable policies became unpredictable, if a reliable assignment happened to coincide with a burst loss, the number of transmitted packets increases, and vice versa. On the other hand, in smarter policies, such as Loss-Aware, the performance lightly matched the vanilla baseline, as the reliable assignment dominated the session to compensate for a higher burst loss. Not only that but, the burst loss also impacted the variance of the transmitted packets for the policies.

Unsurprisingly, the unreliable focused policy, 80-20 split, outperformed other policies for all topologies in random and bursty loss scenarios, with an approximate reduction of 80\%. That being said, the majority of the policies reduced the transmitted packets on the link by approximately 70\% for random loss, while the reduction started at $\approx 15\%$ and decreased as the loss increased for the burst loss scenario.

The retransmitted and lost packets, not shown due to space limitations, followed the same trend as the transmitted packets for the random loss scenarios. However, for the burst loss scenarios, the larger capacity networks had a lower reduction in the retransmitted and lost packets. This can be seen as a favorable outcome since the lower capacity networks are scarce on resources. It is important to note that the Loss-Aware policy mimicked the vanilla approach as the burst loss increased, signifying the overwhelming appointment of reliable packets in adapting to the harsh burst loss conditions.
 
Alternatively, Fig. \ref{fig:backlog_burst} clearly shows a stark comparison between the policies and loss scenario in the reduction of the backlogged packets. The Loss-Aware policy for random loss scenario reduced the backlogged packets by up to 50\%, beating all other policies by approximately 30\%. Furthermore, it is clear that the unreliability focused policies resulted in the lowest backlog for the session. In comparison, we notice that the burst loss and the backlogged frequency have a positive correlation, where the maximum reduction of the backlogged packets for the policies is at most 20\%. Much like the transmitted packets, the probability of a burst loss occurrence plays a vital role in the number of retransmissions sent and by extension the number of backlogged packets. Thus, we can conclude that the stress placed on the buffer is a result of the reliable packets which is tightly coupled with the congestion on the session. Whereas, unreliable focused policies did not encounter such a phenomenon regardless if it was experiencing a burst loss.


\subsection{Application-Layer KPIs}

The feasibility of dynamic reliability for real-time applications can be determined by the \gls{aoi}, with comparison across different topologies and policies. If we take a strict approach and consider anything below $10$~ms is real-time \cite{real-time}, then all the reliability policies passed that requirement, which is attractive for real-time applications, as shown in Fig. \ref{fig:aoi_burst}. Utilising the median as an estimate of the runs, the policies in the WLAN and Sub-6~GHz topology with random loss floated around $4-5$~ms with negligible difference, while the \gls{aoi} for \gls{mmwave} was $\approx 2-3$~ms. It is clear that the \gls{aoi} and the network capacity have a negative correlation, as the network capacity decreases, the \gls{aoi} increases. The same correlation is extended to the bursty loss scenarios, where \gls{mmwave} dominated the other topologies. That being said, it is crucial to note that the \gls{aoi} for the reliability policies is often slightly better than or equal to the \gls{aoi} of the vanilla implementation, proving that dynamic reliability reduces the congestion of the session at no cost to the \gls{aoi}.

\section{Limitations and Future Work}

We summarize the limitations we have identified for our method and propose
future research directions.

\textbf{Parallel implementation:} 
With a focus on accuracy and algorithms, our implementation for this work is
serial. Some of the most time-consuming routines in our method can easily
benefit from a parallel implementation, while the same is not obvious for the
SAP solver and the Schur complement computation. Leveraging the power of
parallelization on modern hardware for these computations is an interesting area
for future investigation.

\textbf{Rotational invariance:} 
As with all other linear constitutive models, our linearized model with lagged
rotational component is not rotationally invariant. Thus it is not suitable for
simulation of extreme deformations using large time steps. For those scenarios,
we fall back to traditional nonlinear models with Hessian positive definite
corrections proposed in \cite{bib:teran2005robust}.

\textbf{Self-contact:} 
We do not consider self-contact at the moment due to the lack of support by our
geometry engine. Self-contact can be incorporated into our method by updating the
geometry engine to augment the set of contacts reported.

\textbf{Tunneling at high speeds:} Though our method has a lower computational
cost, it could benefit from continuous collision detection strategies
\cite{bib:li2020ipc} to provide constraints before contact is established. This
would allow to mitigate issues such as objects tunneling past each other at high
speeds. Efficient solution to mitigate this issue is a topic of active research
for the authors.

\textbf{Redundant constraints:} Our geometry engine often introduces a large
number of constraints to resolve contact. Similarly, welding a large number of
deformable mesh vertices to a rigid body (as done in Section
\ref{sec:bubble_gripper}) introduces many constraints. Even though our SAP
solver \cite{bib:castro2022unconstrained} provides existence and uniqueness
guarantees, a large number of constraints hurts performance as can be observed
in the \emph{Soft-bubble} example. We are currently investigating strategies to
significantly reduce the number of constraints without sacrificing accuracy.

\section{Conclusion}\label{sec:conclusion}
In this work, we focus on addressing the fundamental challenge of OOD detection tasks, which is how to fully understand the semantic discrepancy between the ID/OOD samples. We reveal that the key to success in the realistic SCOOD task is to allocate as many ID samples in the unlabeled set correctly as possible. To this end, we propose a novel uncertainty-aware optimal transport scheme that introduces class-specific energy scores as guidance for effective label assignment. Experimental results show that our method achieves better performance than previous state-of-the-art methods on SCOOD benchmarks.

\textbf{Limitations.} In addition to temperature scaling, other techniques such as feature clipping applied in ReAct~\cite{sun2021react} also enhance the performance of energy score, so how to obtain an OOD score that best fits the SCOOD task can be further explored. Moreover, a setting highly related to SCOOD has been proposed in \cite{katz2022training} and formulated as a constrained optimization problem. We will also theoretically analyze these practical OOD settings in our feature work.

% \section*{Acknowledgments}
\textbf{Acknowledgments.} 
This work is supported by National Key R\&D Program of China under Grant 2020AAA0105701, National Natural Science Foundation of China (NSFC) under Grants 61872327, Major Special Science and Technology Project of Anhui, National Natural Science Foundation of China (62033012) and Ant Group through Ant Research Intern Program.



\section{Appendix for Proofs}

\paragraph{Proof of Theorem \ref{thm:main}.}

\begin{proof}
\label{proof:main}
Our proof has two steps. In Step 1, we will show that SimCLR is equivalent to minimizing the cross entropy loss defined in Eqn.~(\ref{eqn:cross-entropy}). 
In Step 2, we will show  that minimizing the cross-entropy loss 
is equivalent to spectral clustering on $\bfpi$. 
Combining the two steps together, we have proved our theorem. 

\textbf{Step 1: } SimCLR is equivalent to minimizing the cross entropy loss.

The cross-entropy loss takes expectation over 
$\bfW_\bfX\sim \mathbb{P}(\cdot ; \bfpi)$, 
which means $\bfW_\bfX$ has exactly one non-zero entry in each row $i$. By Lemma~\ref{lem:multinomial}, we know every row $i$ of $\bfW_\bfX$ is independent of other rows. Moreover, 
$\bfW_{\bfX,i}\sim \mathcal{M}(1, \bfpi_i/\sum_j \bfpi_{i,j})=\mathcal{M}(1, \bfpi_i)$, because $\bfpi_i$ itself is a probability distribution.
Similarly, we know $\bfW_\bfZ$ also has the row-independent property by sampling over $\mathbb{P}(\cdot;\bfK_\bfZ)$.
Therefore, by Lemma~\ref{lem:cross_split}, we know Eqn.~(\ref{eqn:cross-entropy}) is equivalent to:
\[
 -\sum_{i=1}^n \mathbb{E}_{\bfW_{\bfX,i}}[\log \mathbb{P}(\bfW_{\bfZ,i}=\bfW_{\bfX,i};\bfK_\bfZ)],
\]

This expression takes expectation over $\bfW_{\bfX,i}$ for the given row $i$. Notice that 
$\bfW_{\bfX,i}$ has exactly one non-zero entry, which equals $1$ (same for $\bfW_{\bfZ,i}$). 
As a result
we expand the above expression to be:
\begin{equation}
 -\sum_{i=1}^n \sum_{j\neq i} \Pr(\bfW_{\bfX,i,j}=1)\log \Pr(\bfW_{\bfZ,i,j}=1).
\label{eqn:detailed-expansion}    
\end{equation}


By Lemma~\ref{lem:multinomial}, $\Pr(\bfW_{\bfZ,i,j}=1)=\bfK_{\bfZ,i,j}/\|\bfK_{\bfZ,i}\|_1$ for $j\neq i$. Recall that $\bfK_\bfZ=(k(\bfZ_i-\bfZ_j))_{(i,j)\in[n]^2}$, which means 
$\bfK_{\bfZ,i,j}/\|\bfK_{\bfZ,i}\|_1=\frac{\exp(-\|\bfZ_i-\bfZ_j\|^2/{2\tau})}{\sum_{k\neq i}
\exp(-\|\bfZ_i-\bfZ_k\|^2/{2\tau})
}$ for $j\neq i$, when $k$ is the Gaussian kernel with variance $\tau$. 

Notice that $\bfZ_i=f(\bfX_i)$, so we know
\begin{equation}
-\log \Pr(\bfW_{\bfZ,i,j}=1)=
-\log \frac{\exp(-\|f(\bfX_i)-f(\bfX_j)\|^2/{2\tau})}{\sum_{k\neq i}
\exp(-\|f(\bfX_i)-f(\bfX_k)\|^2/{2\tau}),
}
\label{eqn:infonce-equivalence}    
\end{equation}


The right hand side is exactly the InfoNCE loss defined in Eqn.~(\ref{eqn:infonce}).
Inserting Eqn.~(\ref{eqn:infonce-equivalence}) into Eqn.~(\ref{eqn:detailed-expansion}), we get the SimCLR algorithm, which first samples augmentation pairs $(i,j)$ with $\Pr(\bfW_{\bfX,i,j}=1)$ for each row $i$, and then optimize the InfoNCE loss. 

\textbf{Step 2: } minimizing the cross entropy loss 
is equivalent to spectral clustering on $\bfpi$.


By Lemma~\ref{lem:convert_to_spectral}, we may further convert the loss to 
\begin{equation}
\label{eqn:main-theorem-repul-attr}
\min_{\bfZ}
-\sum_{(i,j)\in [n]^2} \mathbf{P}_{i,j}
\log k (\bfZ_i-\bfZ_j)+\log \mathbf{R}(\bfZ).
\end{equation}
Since $k$ is the Gaussian kernel, this reduces to \[
\min_\bfZ \mathrm{tr}(\bfZ^\top \mathbf{L}(\bfpi) \bfZ)
+\log \mathbf{R}(\bfZ),
\]

where we use the fact that $\mathbb{E}_{\bfW_\bfX\sim \mathbb{P}(\cdot; \bfpi)}[\mathbf{L}(\bfW_\bfX)]
=\mathbf{L}(\bfpi)
$, because the Laplacian operator is linear and $
\mathbb{E}_{\bfW_\bfX\sim \mathbb{P}(\cdot; \bfpi)}(\bfW_\bfX)=\bfpi
$.
\end{proof}

\paragraph{Proof of Theorem \ref{thm:clip}.}
\begin{proof}
Since $\bfW_\bfX\sim \mathbb{P}(\cdot;\bfpi_{\mathbf{A}, \mathbf{B}})$, we know 
$\bfW_\bfX$ has exactly one non-zero entry in each row, denoting the pair that got sampled. 
A notable difference compared to the previous proof is we now have $n_\mathcal{A}+n_\mathcal{B}$ objects in our graph. CLIP deals with this by taking a mini-batch of size $2N$, 
such that $n_\mathcal{A}=n_\mathcal{B}=N$, and adding the $2N$ InfoNCE losses together. We label the objects in $\mathcal{A}$ as $[n_\mathcal{A}]$, and the objects in $\mathcal{B}$ as $\{n_\mathcal{A}+1, \cdots, n_\mathcal{A}+n_\mathcal{B}\}$. 

Notice that $\bfpi_{\mathbf{A}, \mathbf{B}}$ is a bipartite graph, so the edges of objects in $\mathcal{A}$ will only connect to object in $\mathcal{B}$ and vice versa. We can define the similarity matrix in $\cZ$ as $\bfK_\bfZ$, 
where $\bfK_\bfZ(i, j+n_\mathcal{A})=\bfK_\bfZ(j+n_\mathcal{A},i)= k(\bfZ_i-\bfZ_j)$ for $i\in [n_\mathcal{A}], j\in [n_\mathcal{B}]$, and otherwise we set $\bfK_\bfZ(i,j)=0$. 
The rest is same as the previous proof. 
\end{proof}

\paragraph{Proof of Theorem \ref{thm:exponential}.}

\begin{proof}
\label{proof:exponential}
Since the objective function consists of a linear term combined with an entropy regularization, which is a strongly concave function, the maximization problem is a convex optimization problem. Owing to the implicit constraints provided by the entropy function, the problem is equivalent to having only the equality constraint. We then introduce the Lagrangian multiplier $\lambda$ and obtain the following relaxed problem:

$$
\widetilde{E}(\boldsymbol{\alpha})=\psi_{1}-\sum_{i=1}^n \alpha_{i} \psi_{i}+\tau \sum_{i=1}^n \alpha_{i}\log \alpha_{i}+\lambda\left(\boldsymbol{\alpha}^{\top} \mathbf{1}_n-1\right).
$$

As the relaxed problem is unconstrained, taking the derivative with respect to $\alpha_{i}$ yields

$$
\frac{\partial \widetilde{E}(\boldsymbol{\alpha})}{\partial \alpha_{i}}=-\psi_{i}+\tau\left(\log \alpha_{i}+\alpha_{i} \frac{1}{\alpha_{i}}\right)+\lambda=0.
$$

Solving the above equation implies that $\alpha_{i}$ takes the form
$
\alpha_{i}=\exp \left(\frac{1}{\tau} \psi_{i}\right) \exp \left(\frac{-\lambda}{\tau}-1\right).
$ Since $\alpha_{i}$ lies on the probability simplex, the optimal $\alpha_{i}$ is explicitly given by
$
\alpha^{*}_{i}=\frac{\exp \left(\frac{1}{\tau} \psi_{i}\right)}{\sum_{i^{\prime}=1}^n \exp \left(\frac{1}{\tau} \psi_{i^{\prime}}\right)} .
$ Substituting the optimal point into the objective function, we obtain
$$
\begin{aligned}
E\left(\boldsymbol{\alpha}^*\right)  &=\psi_1-\sum_{i=1}^n \frac{\exp \left(\frac{1}{\tau} \psi_{i}\right)}{\sum_{i^{\prime}=1}^n \exp \left(\frac{1}{\tau} \psi_{i^{\prime}}\right)} \psi_{i}+\tau \sum_{i=1}^n \frac{\exp \left(\frac{1}{\tau} \psi_{i}\right)}{\sum_{i^{\prime}=1}^n \exp \left(\frac{1}{\tau} \psi_{i^{\prime}}\right)}\log \frac{\exp \left(\frac{1}{\tau} \psi_{i}\right)}{\sum_{i^{\prime}=1}^n \exp \left(\frac{1}{\tau} \psi_{i^{\prime}}\right)} \\
& =\psi_1 - \tau \log \left(\sum_{i=1}^n \exp \left(\frac{1}{\tau} \psi_{i}\right)\right).
\end{aligned}
$$
Thus, the Lagrangian dual function is given by
\begin{equation*}
-E\left(\boldsymbol{\alpha}^*\right)= -\tau \log \frac{\exp \left(\frac{1}{\tau} \psi_{1}\right)}{\sum_{i=1}^n \exp \left(\frac{1}{\tau} \psi_{i}\right)}.\qedhere
\end{equation*}
\end{proof}



\section{More on Experiments} \label{section: experiment_details}

\paragraph{CIFAR-10 and CIFAR-100} CIFAR-10 ~\citep{krizhevsky2009learning} and CIFAR-100 ~\citep{krizhevsky2009learning} are well-known classic image classification datasets. Both CIFAR-10 and CIFAR-100 contain a total of 60k $32 \times 32$ labeled images of different classes, with 50k for training and 10k for testing. CIFAR-10 is similar to CIFAR-100, except there are 10 different classes in CIFAR-10 and 100 classes in CIFAR-100.

\paragraph{TinyImageNet} TinyImageNet ~\citep{le2015tiny} is a subset of ImageNet ~\citep{deng2009imagenet}. There are 200 different object classes in TinyImageNet, with 500 training images, 50 validation images, and 50 test images for each class. All the images in TinyImageNet are colored and labeled with a size of $64 \times 64$.

\textbf{Pseudo-code.} Algorithm \ref{alg:Training Procedure} presents the pseudo-code for our empirical training procedure.

\begin{algorithm}[!htbp]
\caption{Training Procedure}
\label{alg:Training Procedure}
\begin{algorithmic}[1]
\REQUIRE trainable encoder network $f$, batch size $N$, augmentation strategy \textit{aug}, loss function $L$ with hyperparameters \textit{args}
\FOR {sampled minibatch ${x_i}_{i=1}^N$}
\FORALL{$i \in { 1, ..., N }$}
\STATE draw two augmentations $t_i = \textit{aug}\left(x_i\right) $, $t_i' = \textit{aug}\left(x_i\right) $
\STATE $z_i = f\left(t_i\right)$, $z_i' = f\left(t_i'\right)$
\ENDFOR
\STATE compute loss $\mathcal{L} = L(N, z, z', \textit{args})$
\STATE update encoder network $f$ to minimize $\mathcal{L}$
\ENDFOR
\STATE \textbf{Return} encoder network $f$
\end{algorithmic}
\end{algorithm}

We also provide the pseudo-code for our core loss function used in the training procedure in Algorithm \ref{alg:Core loss}. The pseudo-code is almost identical to SimCLR's loss function, with the exception of an extra parameter $\gamma$.

\begin{algorithm}[!htbp]
\caption{Core loss function $\mathcal{C}$}
\label{alg:Core loss}
\begin{algorithmic}[1]
\REQUIRE batch size $N$, two encoded minibatches $z_1, z_2$, $\gamma$, temperature $\tau$
\STATE $z = \textit{concat}\left(z_1, z_2\right)$
\FOR {$i \in {1, ..., 2N }, j \in {1, ..., 2N}$ }
\STATE $s_{i,j} = \Vert z_i - z_j \Vert_2^{\gamma}$
\ENDFOR
\STATE \textbf{define} $l(i, j)$ \textbf{as} $l(i, j) = - \log \frac{exp\left(s_{i,j}/\tau \right)}{\sum_{k=1}^{2N} \mathbf{1}{[k \ne i]} exp\left(s{i, j} / \tau \right)} $
\STATE \textbf{Return} $\frac{1}{2N} \sum_{k=1}^N\left[l(i, i+N) + l(i+N, i)\right]$
\end{algorithmic}
\end{algorithm}

Utilizing the core loss function $\mathcal{C}$, we can define all kernel loss functions used in our experiments in Table \ref{table: loss definition}. For all $z_i \in z$ with even dimensions $n$, we define $z_{L_i} = z_i\left[0:n/2\right]$ and $z_{R_i} = z_i\left[n/2:n\right]$.

\begin{table}[ht]
\centering
\begin{tabular}{{@{}l|l@{}}}
Kernel  &  Loss function \\ \midrule
Laplacian & $\mathcal{C}\left(N, z, z', \gamma=1, \tau\right)$\\ \midrule
Sum       & $\lambda * \mathcal{C}\left(N, z, z', \gamma=1, \tau_1\right) + (1-\lambda) * \mathcal{C}\left(N, z, z', \gamma=2, \tau_2\right)$  \\ \midrule
Concatenation Sum&$\lambda * \mathcal{C}\left(N, z_L, z'_L, \gamma=1, \tau_1\right) + (1-\lambda) * \mathcal{C}\left(N, z_R, z'_R, \gamma=2, \tau_2\right)$\\ \midrule
$\gamma = 0.5$ & $\mathcal{C}\left(N, z, z', \gamma=0.5, \tau\right)$          \\ 

\end{tabular}

\caption{Definition of kernel loss functions in our experiments}
\label {table: loss definition}
\end{table}

\textbf{Baselines.} We reproduce the SimCLR algorithm using PyTorch Lightning~\citep{PytorchLightning}.

\textbf{Encoder details.}
The encoder $f$ consists of a backbone network and a projection network. We employ ResNet50~\citep{ResNet} as the backbone and a 2-layer MLP (connected by a batch normalization~\citep{ioffe2015batch} layer and a ReLU \cite{nair2010rectified} layer) with hidden dimensions 2048 and output dimensions 128 (or 256 in the concatenation kernel case).

\textbf{Encoder hyperparameter tuning.}
For each encoder training case, we randomly sample 500 hyperparameter groups (sample details are shown in Table \ref{table: Hyperparameter sample}) and train these samples simultaneously using Ray Tune ~\citep{RayTune}, with the ASHA scheduler~\citep{li2018massively}. Ultimately, the hyperparameter group that maximizes the online validation accuracy (integrated in PyTorch Lightning) within 5000 validation steps is chosen for the given encoder training case.

\begin{table}[ht]
\centering

\begin{tabular}{@{}l|l|l@{}}
\midrule
Hyperparameter  & Sample Range & Sample Strategy \\ \midrule
start learning rate & $\left[10^{-2}, 10\right]$ & log uniform \\ \midrule
$\lambda$       & $\left[0, 1\right]$ & uniform \\ \midrule
$\tau$, $\tau_1$, $\tau_2$ & $\left[0, 1\right]$ & log uniform \\ \midrule
\end{tabular}

\caption{Hyperparameters sample strategy}
\label {table: Hyperparameter sample}
\end{table}

\textbf{Encoder training.} 
We train each encoder using the LARS optimizer~\citep{LARSOptimizer}, LambdaLR Scheduler in PyTorch, momentum 0.9, weight decay $10^{-6}$, batch size 256, and the aforementioned hyperparameters for 400 epochs on a single A-100 GPU.

\textbf{Image transformation.} The image transformation strategy, including augmentation, is identical to the default transformation strategy provided by PyTorch Lightning.

\textbf{Linear evaluation.}
The linear head is trained using the SGD optimizer with a cosine learning rate scheduler, batch size 64, and weight decay $10^{-6}$ for 100 epochs. The learning rate starts at $0.3$ and ends at $0$.

\textbf{Moco Experiments.} We also tested our method based on MoCo~\citep{he2019moco}. The results are summarized in Table \ref{tab:results-moco}. Here we choose ResNet18~\citep{ResNet} as the backbone and set a temperature of $0.1$ as default. For our simple sum kernel, we set $\lambda=0.8$. The results show that our method outperforms the original MoCo method.

\begin{table}[thb]
\centering
\caption{MoCo Experiment Results on CIFAR-10 and CIFAR-100.}
\label{tab:results-moco}
\resizebox{\textwidth}{!}{%
\begin{tabular}{@{}c|ccc|ccc@{}}
\toprule
\multirow{3}{*}{Method} & \multicolumn{3}{c|}{CIFAR-10} & \multicolumn{3}{c}{CIFAR-100} \\ \cmidrule(lr){2-4} \cmidrule(lr){5-7} 
                        & 200 epochs & 400 epochs    & 1000 epochs   & 200 epochs & 400 epochs & 1000 epochs         \\ \midrule
MoCo (repro.)         & $76.41 \pm 0.12$    & $80.01 \pm 0.15$          & $84.45 \pm 0.08$    & $\mathbf{47.02 \pm 0.11}$ & $52.50 \pm 0.07$ & $57.62 \pm 0.15$            \\
\midrule
Laplacian Kernel        & ${78.09 \pm 0.10}$    & $\mathbf{83.85 \pm 0.09}$          & $\mathbf{88.34 \pm 0.16}$    & $46.12 \pm 0.22$   & $53.44 \pm 0.17$ & $59.10 \pm 0.14$        \\
Simple Sum Kernel & $\mathbf{78.12 \pm 0.15}$   & $83.23 \pm 0.18$ & $87.50 \pm 0.20$ & $46.65 \pm 0.06$ & $\mathbf{53.62 \pm 0.19}$ & $\mathbf{59.83 \pm 0.12}$\\
\bottomrule
\end{tabular}
}
\end{table}



\section{More Experiments on Synthetic Data}


Consider a scenario with $n$ clusters, each containing $k$ vertices. Let the probability of vertices $u$ and $v$ from the same cluster belonging to $\bfpi$ be $p$. Conversely, for vertices $u$ and $v$ from different clusters, let the probability of belonging to $\pi$ be $q$. We generate the graph $\bfpi$ randomly, based on $p$ and $q$. We experiment with values of $k=100$ and $n=6$ for ease of visualization, embedding all points in a two-dimensional space. Each vertex's initial position originates from a normal distribution. In each iteration, we sample a subgraph of $\bfpi$ uniformly, ensuring each vertex has an out-degree of $1$. We then optimize the corresponding vectors using InfoNCE loss with an SGD optimizer and iterate until convergence. Our experimental setup consists of an SGD learning rate of $1$, an InfoNCE loss temperature of $0.5$, and a batch size of $50$. We evaluate two scenarios with different $p$ and $q$ values: $p=1$, $q=0$, and $p=0.75$, $q=0.2$. The results of these experiments are visualized in Figure \ref{fig:vis-spectral-cluster}. The obtained embeddings exhibit the hallmark pattern of spectral clustering of graph $\bfpi$.

\begin{figure}[!tb]
\centering
\subfigure{
\includegraphics[width=1\textwidth]{Figures/cluster_pi.png}
\label{fig:vis-cluster}
}
\subfigure{
\includegraphics[width=1\textwidth]{Figures/noised_cluster_pi.png}
\label{fig:vis-noised-cluster}
}
\caption{Visualizations of the optimization process using InfoNCE Loss on the vectors corresponding to $\bfpi$. Points of identical color belong to the same cluster within $\bfpi$. To showcase the internal structure of $\bfpi$, we randomly select 10 vertices from each cluster to display the edge distribution of $\bfpi$.}
\label{fig:vis-spectral-cluster}
\end{figure}


\bibliographystyle{ACM-Reference-Format}
\bibliography{sample}





\end{document}