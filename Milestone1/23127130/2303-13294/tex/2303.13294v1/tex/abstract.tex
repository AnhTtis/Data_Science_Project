\begin{abstract}
Quality assessment algorithms can be used to estimate the utility of a biometric sample for the purpose of biometric recognition.
``Error versus Discard Characteristic'' (EDC) plots,
and ``partial Area Under Curve'' (pAUC) values of curves therein,
are generally used by researchers to evaluate the predictive performance of such quality assessment algorithms.
An EDC curve depends on 
an error type such as the ``False Non Match Rate'' (FNMR),
a quality assessment algorithm,
a biometric recognition system,
a set of comparisons each corresponding to a biometric sample pair,
and a comparison score threshold corresponding to a starting error.
To compute an EDC curve, comparisons are progressively discarded based on the associated samples' lowest quality scores, and the error is computed for the remaining comparisons.
Additionally, a discard fraction limit or range must be selected to compute pAUC values, which can then be used to quantitatively rank quality assessment algorithms.

This paper discusses and analyses various details for this kind of quality assessment algorithm evaluation,
including general EDC properties,
interpretability improvements for pAUC values based on a hard lower error limit and a soft upper error limit,
the use of relative instead of discrete rankings,
stepwise \vs{} linear curve interpolation,
and normalisation of quality scores to a $[0, 100]$ integer range.
We also analyse the stability of quantitative quality assessment algorithm rankings based on pAUC values across varying pAUC discard fraction limits and starting errors,
concluding that higher pAUC discard fraction limits should be preferred.
The analyses are conducted both with synthetic data and with real data for a face image quality assessment scenario,
with a focus on general modality-independent conclusions for EDC evaluations.
\end{abstract}
    