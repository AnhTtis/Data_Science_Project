\section{Quality score normalisation}
\label{sec:normalisation}

The ``raw'' QSs produced by different QA algorithms can be numbers that lie in various ranges with different granularity.
For example, the QSs produced by the real FIQA algorithms described in \autoref{sec:real-setup} are floating point numbers in various ranges.
The presence of different QS ranges per QA algorithm is unproblematic for the computation of EDC curves in a plot, provided the QSs from different QA algorithms aren't mixed for a curve, since the EDC curves depend only on the (discard) order of the (pairwise) QSs relative to each other.
Different QA algorithm output ranges and QS distributions are however relevant if the QSs from different algorithms should be made similarly interpretable (\eg{} to apply the same QS threshold to discard samples across multiple QA algorithms), or to fuse QSs from different QA algorithms \cite{Schlett-FIQA-Fusion-BIOSIG-2022}, or because the ``raw'' QS output isn't usable for a certain data format.

A concrete instance of the latter scenario can be found in ISO/IEC 29794-1:2016 \cite{ISO-IEC-29794-1-QualityFramework-160915},
which prescribes a $[0,100]$ integer range for QSs as a mandatory requirement of the standardised data interchange format \cite{ISO-IEC-39794-1-G3-Framework-191223}\cite{ISO-IEC-39794-5-G3-FaceImage-191015}.
QSs can be normalised to such integer ranges, and this can also be done before an EDC evaluation is carried out.
This section examines the effect of this normalisation on the EDC curves across different normalisation configurations using the real \autoref{sec:real-setup} FIQA data.

To normalise the ``raw'' QSs to $[0,100]$ integers (\ie{} 101 bins), 100 QS boundaries are calibrated based on a certain calibration function and a set of raw QSs.
The set of raw QSs is used as input for the calibration function.
Three different raw calibration QS set variants are part of the analysis:
\begin{itemize}
\item ``Same'': The same raw QSs used for the unnormalised EDC curves are used for calibration. \Ie{} the raw QSs for one QA algorithm on one dataset are used both to compute the EDC curve without normalisation, and to create the normalised variant thereof. This can therefore be considered as a best-case scenario in which the calibration data is equivalent to the evaluation data.
\item ``Other'': Raw QSs from the ``other'' of the two datasets (LFW and TinyFace) are used for calibration. \Ie{} to compute the EDC curve on LFW for one QA algorithm with normalisation, raw QSs for the same QA algorithm from TinyFace are used for the calibration (and vice versa). The QS distributions between these two datasets can differ substantially, as visible in the following results, so this represents a calibration scenario in which the calibration data is suboptimal.
\item ``Combined'': Here the raw QS sets from ``Same'' and ``Other'' (\ie{} from both datasets) are merged for calibration.
This represents a scenario wherein a broader coverage of calibration data across datasets is used, and where the optimal set of QSs happens to be included in the calibration data among other QSs.
\end{itemize}

\begin{figure}
\centering
\begin{tabular}{c}
	MinMax \\
	\includegraphics[width=0.97\linewidth]{img/normalisation/LFW/Dist--Same--MinMax--CR-FIQA-L} \\
	Proportional \\
	\includegraphics[width=0.97\linewidth]{img/normalisation/LFW/Dist--Same--Proportional--CR-FIQA-L} \\
\end{tabular}
\caption{\label{fig:normalisation-dist-same-minmax-vs-proportional} Comparison between ``MinMax'' and ``Proportional'' calibration of the $[0, 100]$ QS normalisation boundaries, based on the same QS scores from CR-FIQA(L) on the LFW dataset for example. The Y-axis shows QS histogram bin counts, with the share of the total QS count in parentheses.}
\vspace{-1em}
\end{figure}

\begin{figure}
\centering
\begin{tabular}{c}
	LFW \\
	\includegraphics[width=0.97\linewidth]{img/normalisation/LFW/Dist--Other--MinMax--CR-FIQA-L} \\
	TinyFace \\
	\includegraphics[width=0.97\linewidth]{img/normalisation/TinyFace/Dist--Other--MinMax--CR-FIQA-L} \\
\end{tabular}
\caption{\label{fig:normalisation-dist-other-lfw-vs-tinyface} $[0, 100]$ QS normalisation boundaries obtained by ``MinMax'' calibration on QSs for the other dataset (LFW for TinyFace, TinyFace for LFW), using CR-FIQA(L) for example. The Y-axis shows QS histogram bin counts, with the share of the total QS count in parentheses.}
\vspace{-1.5em}
\end{figure}

The shown analysis further includes two different calibration functions:
\begin{itemize}
\item ``MinMax'': The range between the minimum and the maximum of the raw QSs used for calibration is equally subdivided. All other calibration QSs simply have no effect on the calibration.
\item ``Proportional'': The normalisation QS boundaries are fitted to the distribution of the raw calibration QSs so that (approximately) the same number of calibration QSs is mapped to each of the 101 normalised QSs. \Ie{} the density of the calibration QSs is reflected in the resulting normalisation QS boundaries.
\end{itemize}

\begin{figure}
	\centering
	\begin{tabular}{c}
		Same \\
		\includegraphics[width=\linewidth]{img/normalisation/LFW/EDC--Same--MinMax} \\
		Other \\
		\includegraphics[width=\linewidth]{img/normalisation/LFW/EDC--Other--MinMax} \\
		\includegraphics[width=0.87\linewidth]{img/normalisation/EDC--Legend} \\
	\end{tabular}
	\caption{\label{fig:normalisation-edc-lfw-same-vs-other} Two EDC plot examples on LFW with curves for raw QSs and QSs normalised based on ``MinMax'' calibration, one using calibration QSs from the same dataset (LFW) and one from another dataset (TinyFace).}
	\vspace{-1.5em}
\end{figure}

\begin{table*}
	\centering
	\caption{\label{tab:normalisation} The divergence of EDC curves based on normalised QSs from the EDC curves based on the corresponding raw QSs, expressed as the area between the curves divided by the raw QS curve pAUC value, in percent (\ie{} always zero or positive, lower is better).}
	\scriptsize
	\begin{tabular}{cccc|c|ccccc}
		pAUC discard range & Calibration data &    Quantizer & EDC data &  Mean & CR-FIQA(L) & CR-FIQA(S) & MagFace & PCNet & SER-FIQ \\
		\hline
		$[0.00, 0.20]$ &             Same &       MinMax &      LFW &  2.76\% &       4.49\% &       2.22\% &    3.53\% &  2.02\% &    1.55\% \\
		$[0.00, 0.20]$ &             Same &       MinMax & TinyFace &  0.93\% &       0.81\% &       1.27\% &    1.29\% &  0.46\% &    0.83\% \\
		$[0.00, 0.20]$ &             Same & Proportional &      LFW &  1.85\% &       2.53\% &       1.98\% &    2.64\% &  1.49\% &    0.60\% \\
		$[0.00, 0.20]$ &             Same & Proportional & TinyFace &  2.64\% &       3.96\% &       1.39\% &    2.36\% &  1.01\% &    4.48\% \\
		\hline
		$[0.00, 0.20]$ &            Other &       MinMax &      LFW &  2.93\% &       4.40\% &       3.61\% &    2.53\% &  1.30\% &    2.80\% \\
		$[0.00, 0.20]$ &            Other &       MinMax & TinyFace & 11.04\% &      26.99\% &      10.48\% &    1.75\% &  8.86\% &    7.09\% \\
		$[0.00, 0.20]$ &            Other & Proportional &      LFW & 16.47\% &      38.80\% &       4.41\% &   35.95\% &  2.32\% &    0.86\% \\
		$[0.00, 0.20]$ &            Other & Proportional & TinyFace & 22.57\% &      26.99\% &      10.48\% &   26.40\% &  8.86\% &   40.14\% \\
		\hline
		$[0.00, 0.20]$ &         Combined &       MinMax &      LFW &  3.68\% &       4.93\% &       4.12\% &    3.88\% &  2.42\% &    3.04\% \\
		$[0.00, 0.20]$ &         Combined &       MinMax & TinyFace &  1.23\% &       1.06\% &       1.46\% &    1.97\% &  0.80\% &    0.88\% \\
		$[0.00, 0.20]$ &         Combined & Proportional &      LFW &  2.28\% &       3.42\% &       2.65\% &    3.22\% &  1.57\% &    0.56\% \\
		$[0.00, 0.20]$ &         Combined & Proportional & TinyFace &  7.19\% &       9.30\% &       3.60\% &    6.47\% &  2.87\% &   13.74\% \\
		\hline
		$[0.00, 0.30]$ &             Same &       MinMax &      LFW &  3.12\% &       4.96\% &       2.59\% &    4.44\% &  2.09\% &    1.52\% \\
		$[0.00, 0.30]$ &             Same &       MinMax & TinyFace &  1.21\% &       0.87\% &       1.55\% &    1.86\% &  0.73\% &    1.04\% \\
		$[0.00, 0.30]$ &             Same & Proportional &      LFW &  1.75\% &       2.46\% &       1.75\% &    2.57\% &  1.43\% &    0.56\% \\
		$[0.00, 0.30]$ &             Same & Proportional & TinyFace &  2.34\% &       3.49\% &       1.22\% &    2.26\% &  1.09\% &    3.65\% \\
		\hline
		$[0.00, 0.30]$ &            Other &       MinMax &      LFW &  4.44\% &       5.22\% &       4.10\% &    8.36\% &  1.49\% &    3.02\% \\
		$[0.00, 0.30]$ &            Other &       MinMax & TinyFace & 13.73\% &      29.60\% &      16.73\% &    2.28\% & 14.65\% &    5.38\% \\
		$[0.00, 0.30]$ &            Other & Proportional &      LFW & 28.03\% &      61.28\% &       8.04\% &   61.55\% &  8.44\% &    0.84\% \\
		$[0.00, 0.30]$ &            Other & Proportional & TinyFace & 32.72\% &      41.76\% &      16.73\% &   38.47\% & 14.65\% &   52.00\% \\
		\hline
		$[0.00, 0.30]$ &         Combined &       MinMax &      LFW &  4.15\% &       6.22\% &       4.51\% &    4.63\% &  2.59\% &    2.78\% \\
		$[0.00, 0.30]$ &         Combined &       MinMax & TinyFace &  1.62\% &       1.18\% &       1.57\% &    3.39\% &  0.93\% &    1.05\% \\
		$[0.00, 0.30]$ &         Combined & Proportional &      LFW &  2.21\% &       3.26\% &       2.25\% &    3.31\% &  1.67\% &    0.56\% \\
		$[0.00, 0.30]$ &         Combined & Proportional & TinyFace &  6.29\% &       8.42\% &       3.14\% &    5.95\% &  2.88\% &   11.07\% \\
	\end{tabular}
\end{table*}

\autoref{fig:normalisation-dist-same-minmax-vs-proportional} illustrates the difference between the ``MinMax'' and ``Proportional'' calibration, for the QA algorithm CR-FIQA(L) on the LFW dataset (\ie{} using the ``Same'' calibration data variant).
The plots show that the ``Proportional'' calibration neglects detail for a relatively large range of lower QSs,
whereas the ``MinMax'' calibration doesn't.
Be aware that these plots include the entire QS distribution, meaning that the X-axis shows the entire range from the QS minimum to the QS maximum, but the minimum data point is not easily visible due to a low Y-axis value.

\autoref{fig:normalisation-dist-other-lfw-vs-tinyface} provides an example for the ``Other'' data calibration variant, again for the QA algorithm CR-FIQA(L).
It shows that the QS distributions differ substantially for the two datasets, leading to a QS normalisation boundary calibration that doesn't fit the QSs on the other dataset well.
The mismatch would be even more severe if the ``Proportional'' calibration were used instead of the ``MinMax'' calibration,
since the densest QS concentrations differ (besides just the minima and maxima).

Actual EDC curve examples for the raw and the normalised QSs are depicted in \autoref{fig:normalisation-edc-lfw-same-vs-other}, for LFW with ``MinMax'' calibration on the ``Same'' and ``Other'' calibration data variants.
For the ``Same'' calibration data variant, the normalised QS curves exhibit a reasonably close fit to the raw QS curves.
However, normalised QS curves under the ``Other'' calibration data variant display substantial deficiencies towards higher discard fractions.
In this concrete example with LFW, this happens because the QSs on LFW tend to be higher than on TinyFace, as previously seen in \autoref{fig:normalisation-dist-other-lfw-vs-tinyface} for the QA algorithm CR-FIQA(L).

While the number of experiment configurations is too large to include all individual plots, more compact but complete results based on pAUC values for two discard ranges are listed in \autoref{tab:normalisation}.
From left to right, the columns show the pAUC discard range (0\% to either 20\% or 30\%), the calibration data variant (Same/Other/Combined), the calibration function (MinMax/Proportional), the dataset for the EDC evaluation (LFW/TinyFace),
the mean result of the QA algorithm result values (\ie{} across the row),
and then the individual results for the five QA algorithm.
Per configuration (table row), each QA algorithm (rightmost column set) has one raw/normalised QS EDC curve pair.
The corresponding result value is the area between these two curves divided by the pAUC value for the raw QS EDC curve, times 100.
This means that the result values are the percentages of the raw QS pAUC deviation caused by the normalisation, lower being better.

The results in \autoref{tab:normalisation} show that the configurations using the ``Other'' calibration data variant are substantially worse than the ones using the ``Same'' or ``Combined'' calibration data variants.
The differences between the ``Same'' and ``Combined'' variants are comparatively minor,
especially with the ``MinMax'' calibration function,
which demonstrates that a more general calibration (``Combined'') can be competitive with a best-case calibration (``Same'').
The ``Proportional'' calibration function does not appear to have a clear advantage over the simpler ``MinMax'' calibration function for ``Same''/``Combined''.
But it does have a clear disadvantage for ``Other'', since it is by design strongly affected by the distribution of the raw calibration QSs,
which becomes more apparent when higher pAUC discard limits are examined (here 30\% instead of 20\%).

In conclusion,
it may be preferable to evaluate different QA algorithms without normalisation first, before evaluating and selecting the best normalisation setup in a separate step,
since integer range normalisation is a strict reduction in precision that can substantially alter EDC curves depending on the approach.
A simple ``MinMax'' calibration can provide better results than a more QS-distribution-aware calibration such as ``Proportional''.
While a more expansive investigation of QS normalisation approaches is outside the scope of this EDC-centric paper, it could serve as a suitable topic for future work.
