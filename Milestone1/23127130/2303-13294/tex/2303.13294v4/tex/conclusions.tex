\section{Conclusions}
\label{sec:conclusions}

We discussed various general aspects of EDC QA algorithm evaluation,
as well as aspects of quantitative rankings based on pAUC values in particular,
with examples focusing on FNM-EDC configurations.
The primary findings can be summarised as follows:

\begin{enumerate}
\item Stepwise curve interpolation should be preferred for both for graphical plots and for pAUC value computation, to reflect the actual behaviour of the error with respect to the discard steps.
\item Relative rankings based on pAUC values can be used to show how close each QA algorithm is to the best/worst performing one.
\item If pAUC values are examined directly (instead of \eg{} the relative rankings), the interpretability can be improved by normalising them relative to the hard lower error limit defined by the ``area under theoretical best'', $max(0, Error-DiscardFraction)$, and the soft upper error limit defined by the starting error constant.
\item Normalising quality scores to an integer range such as $[0, 100]$ does naturally affect the EDC curves, and there are different normalisation approaches that also depend on the selected calibration data. It may therefore be preferable to evaluate different QA algorithms without normalisation, before evaluating and selecting the best normalisation setup in a separate step.
\item Despite the simplicity, simple min-max integer range normalisation of quality scores can be effective, while a normalisation proportional to calibration quality scores can allocate too much detail to higher quality scores, which is counterproductive since the differentiation between lower quality levels is more important.
\item Quantitative rankings based on pAUC values were shown to be less reliable for low pAUC discard limits, so higher operationally relevant discard limits should be preferred.
The starting error choice appeared to be less important at these higher discard limits.
\end{enumerate}

We are recommending that researchers and people involved in standardisation take the considerations of this work into account.
