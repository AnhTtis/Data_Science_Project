\section{Curve interpolation}
\label{sec:interpolation}

\begin{figure}
\begin{tabular}{c}
Stepwise interpolation \\
\includegraphics[width=\linewidth]{img/edc-properties/EDC-Example--Interpolation-Step} \\
Linear interpolation \\
\includegraphics[width=\linewidth]{img/edc-properties/EDC-Example--Interpolation-Linear} \\
\includegraphics[width=0.96\linewidth]{img/real-data/Real-FIQA-legend} \\
\end{tabular}
\caption{\label{fig:edc-interpolation}Stepwise \vs{} linear EDC curve interpolation example with real data.} %
\vspace{-2em}
\end{figure}

\begin{figure}
\begin{tabular}{c}
a) 10 steps at starting error 0.50 \\
\includegraphics[width=0.97\linewidth]{img/edc-properties/EDC-Example--Theoretical-Best-Steps--1} \\
b) 50 steps at starting error 0.50 \\
\includegraphics[width=0.97\linewidth]{img/edc-properties/EDC-Example--Theoretical-Best-Steps--2} \\
c) 50 steps at starting error 0.05 \\
\includegraphics[width=0.97\linewidth]{img/edc-properties/EDC-Example--Theoretical-Best-Steps--3} \\
\includegraphics[width=0.65\linewidth]{img/edc-properties/EDC-Example--Theoretical-Best-Steps--Legend} \\
\end{tabular}
\caption{\label{fig:edc-theoretical-best-steps} Examples for best possible EDC curves (coloured) in comparison to the dashed lower grey ``theoretical best'' line.}
\vspace{-2em}
\end{figure}

As described in \autoref{sec:edc},
every data point in an EDC curve corresponds to a discrete number of discarded comparisons.
This means that the fraction of discarded comparisons that is plotted on the X-axis technically is the fractional upper limit of discarded comparisons,
which maps to a discrete number of actually discarded comparisons.
To reflect this property,
we recommend using ``stepwise'' curve interpolation for EDC plots,
meaning that the curve's error (Y-axis) value only changes at each concrete data point.
Using linear interpolation instead may be misleading, which is demonstrated by the interpolation-dependent curve intersection points in \autoref{fig:edc-interpolation}.
Note that the curve interpolation choice also affects the pAUC computation, and thus QA algorithm rankings based on the pAUC values.

\autoref{fig:edc-theoretical-best-steps} further shows how a best possible EDC curve with stepwise interpolation in comparison to the ``theoretical best'' line may look.
For illustrative purposes the \autoref{fig:edc-theoretical-best-steps} a) plot shows curves with only ten equally sized ``steps'', corresponding to an equal number of comparisons for ten discard steps.
The Y-axis error values (FNMR) for the blue curves are identical to their corresponding X-axis discard fraction value,
which means that the error is computed by dividing the number of error cases (CSs below the threshold for FNMR) by the constant total number of comparisons,
thus as labelled the error is computed ``With discarded'' comparisons included. 
For the orange curves the error is instead computed by dividing the number of error cases by the number of remaining comparisons, \ie{} ``Without discarded'' comparisons.
The latter ``Without discarded'' error computation is used within this paper and should generally be preferred, since it corresponds to a scenario in which samples are discarded before they would be involved in comparisons.
In contrast, the ``With discarded'' computation would be unable to show increasing errors due to the discarding of non-error (true positive) comparisons, the denominator being a constant and the numerator being the error count that cannot increase.
As visible in the exaggerated case in the \autoref{fig:edc-theoretical-best-steps} a) plot,
the Y-axis error values for a best case curve ``Without discarded'' can technically deviate more substantially from the lower dashed theoretical best line than the best case curve ``With discarded'', despite the same X-axis discard steps.
This is because the denominator used to compute the FNMR, \ie{} the number of remaining comparisons, decreases with increasing discard fraction.
An increased number of discard steps cannot eliminate the difference of the ``Without discarded'' curve to the theoretical best line since this difference depends on the starting error, as the \autoref{fig:edc-theoretical-best-steps} b) plot exemplifies.
A lower starting error can however reduce the difference, as the \autoref{fig:edc-theoretical-best-steps} c) plot shows.
This detail of the best possible real (``Without discarded'') curve behaviour may thus not be important to consider in practice when lower starting errors and larger numbers of comparisons are used,
since these allow for a better approximation of the theoretical best line.
But note that real EDC curves as shown \eg{} in \autoref{fig:edc-interpolation} may also involve differently sized discard steps, \ie{} the discarding of different numbers of comparisons, which can happen in different orders depending on the (pairwise) QSs.
