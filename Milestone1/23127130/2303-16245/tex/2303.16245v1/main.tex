\documentclass[conference]{IEEEtran}
\IEEEoverridecommandlockouts
% The preceding line is only needed to identify funding in the first footnote. If that is unneeded, please comment it out.
\usepackage{cite}
\usepackage{amsmath,amssymb,amsfonts}
\usepackage{algorithmic}
\usepackage{graphicx}
\usepackage{textcomp}
\usepackage{xcolor}
\usepackage{subcaption}

\def\BibTeX{{\rm B\kern-.05em{\sc i\kern-.025em b}\kern-.08em
    T\kern-.1667em\lower.7ex\hbox{E}\kern-.125emX}}

% correct bad hyphenation here
\hyphenation{op-tical net-works semi-conduc-tor}
\makeatletter
\newcommand{\linebreakand}{%
  \end{@IEEEauthorhalign}
  \hfill\mbox{}\par
  \mbox{}\hfill\begin{@IEEEauthorhalign}
}
\makeatother

\newcommand\xingfu[1]{{\color{black}#1}}  
\newcommand{\PB}[1]{\textcolor{red}{PB: #1}}
\newcommand{\JK}[1]{\textcolor{green}{JK: #1}}
\begin{document}
%

%
\title{ytopt: Autotuning Scientific Applications for Energy Efficiency at Large Scales}
%
% author names and affiliations
% use a multiple column layout for up to three different
% affiliations

%\if 0

\author{\IEEEauthorblockN{Xingfu Wu, Prasanna Balaprakash, Michael Kruse, Jaehoon Koo, Brice Videau, Paul Hovland, Valerie Taylor}
\IEEEauthorblockA{Argonne National Laboratory, Lemont, IL 60439 \\
Email: \{xingfu.wu,pbalapra,michael.kruse,jkoo,bvideau,hovland,vtaylor\}@anl.gov}
\and
\linebreakand 
\IEEEauthorblockN{Brad Geltz, Siddhartha Jana}
\IEEEauthorblockA{Intel Corporation, Hillsboro, OR 97124  \\
            Email: \{brad.geltz,siddhartha.jana\}@intel.com} 
\and
\IEEEauthorblockN{Mary Hall}
\IEEEauthorblockA{University of Utah, Salt Lake City, UT 84103 \\ 
Email: mhall@cs.utah.edu}
}

%\fi

\maketitle

\thispagestyle{plain}
\pagestyle{plain}

\begin{abstract}
 As we enter the exascale computing era, efficiently utilizing power and optimizing the performance of scientific applications under power and energy constraints has become critical and challenging. We propose a low-overhead autotuning framework to autotune performance and energy for various hybrid MPI/OpenMP scientific applications at large scales and to explore the tradeoffs between application runtime and power/energy for energy efficient application execution, then use this framework to autotune four ECP proxy applications---XSBench, AMG, SWFFT, and SW4lite. Our approach uses Bayesian optimization with a Random Forest surrogate model to effectively search parameter spaces with up to 6 million different configurations on two large-scale production systems, Theta at Argonne National Laboratory and Summit at Oak Ridge National Laboratory. The experimental results show that our autotuning framework at large scales has low overhead and achieves good scalability.
Using the proposed autotuning framework to identify the best configurations, 
we achieve up to 91.59\% performance improvement, up to 21.2\% energy savings, and up to 37.84\% EDP improvement on up to 4,096 nodes.

\end{abstract}

%\begin{IEEEkeywords}
%autotuning, energy efficiency, performance optimization, hardware-software co-design
%\end{IEEEkeywords}

\section{Introduction}
\label{sec:introduction}
% \begin{itemize}
%     % Diffusion of FL
%     \item {\st{Diffusion of FL}}
%     % Security threats to FL
%     \item {\st{Security threats to FL with particular focus on model poisoning}}
%     % Limitations of existing countermeasures
%     \item {\st{Current countermeasures (e.g., KRUM) and their limitations}}
%     % Proposed method and its advantages
%     \item {\st{Intuitive description of the proposed method and its difference (i.e., advantages) w.r.t. state of the art}}
%     % Main contributions
%     \item {\st{Summary of the main contributions of this work}}
%     % Paper's structure and organization
%     \item {\st{Paper's structure and organization}}
% \end{itemize}

% Diffusion of FL
Recently, {\em federated learning} (FL) has emerged as the leading paradigm for training distributed, large-scale, and privacy-preserving machine learning (ML) systems~\cite{mcmahan2017googleai,mcmahan2017aistats}. 
The core idea of FL is to allow multiple edge clients to collaboratively train a shared, global model without disclosing their local private training data.
%Specifically, an FL system consists of a central server and many edge clients; 
A typical FL round involves the following steps: {\em(i)} the server randomly picks some clients and sends them the current, global model; {\em(ii)} each selected client locally trains its model with its own private data; then, it sends the resulting local model to the server;\footnote{Whenever we refer to global/local model, we mean global/local model {\em parameters}.} {\em(iii)} the server updates the global model by computing an \emph{aggregation function}, usually the average (FedAvg), on the local models received from clients.
% \begin{enumerate}
%     \item[{\em(i)}] the server sends the current, global model to the clients and appoints some of them for training;
%     \item[{\em(ii)}] each selected client locally trains its copy of the global model with its own private data; then, it sends the resulting local model back to the server;\footnote{Whenever we refer to global/local model, we mean global/local model {\em parameters}.}
%     \item[{\em(iii)}] the server updates the global model by computing an \emph{aggregation function} on the local models received from clients (by default, the average, also referred to as FedAvg~\cite{mcmahan2017aistats}).
% \end{enumerate}
This process goes on until the global model converges. %(e.g., after a certain number of rounds or other similar stopping criteria).
%\\
% The advantages of FL over the traditional, centralized learning paradigm are undoubtedly clear in terms of flexibility/scalability (clients can join/disconnect from the FL network dynamically), network communications (only model weights\footnote{We will use \textit{parameters} and \textit{weights} interchangeably.} are exchanged between clients and server), and privacy (each client's private training data is kept local at the client's end and not uploaded to the server).
\\
% Security threats to FL
%However, the growing adoption of FL also raises security concerns~\cite{costa2022covert}, particularly about its confidentiality, integrity, and availability.
Although its advantages over standard ML, FL also raises security concerns~\cite{costa2022covert}. %, particularly about its confidentiality, integrity, and availability~\cite{costa2022covert}.
% OLD, LONG VERSION
% Indeed, some work deals with privacy leakage that may expose the local data of some clients~\cite{melis2019sp}. 
% A large body of work, instead, investigates attacks that usually aim to detriment the predictive accuracy of the learned global model. For instance, \emph{data poisoning} attacks achieve this goal by letting an adversary pollute the training set of some corrupt FL clients with maliciously crafted examples~\cite{jagielski2018sp}.
% Similarly, in \emph{model poisoning} the attacker attempts to tweak the global model weights~\cite{bhagoji2019pmlr} by directly perturbing the local model's weights of some infected FL clients before these are sent to the central server for aggregation, usually via so-called Byzantine attacks. 
% It turns out that Byzantine model poisoning attacks severely impact standard FedAvg; therefore, more robust aggregation functions must be designed to make FL systems secure.
Here, we focus on \emph{untargeted model poisoning} attacks~\cite{bhagoji2019pmlr}, where an adversary attempts to tweak the global model weights %\footnote{We will use the terms \textit{parameters} and \textit{weights} interchangeably.} 
by directly perturbing the local model's parameters of some infected clients before these are sent to the central server for aggregation.
In doing so, the adversary aims to jeopardize the global model \textit{indiscriminately} at inference time.
Such model poisoning attacks severely impact standard FedAvg; therefore, more robust aggregation functions must be designed to secure FL systems.
\\
% In this paper, we focus on designing a novel robust aggregation scheme at the server's end to contrast the effect of Byzantine model poisoning attacks.
%
% Current countermeasures and their limitations
%Several countermeasures have been proposed in the literature to combat model poisoning attacks on FL systems.
% Some methods use simple statistics more robust than plain average to smooth the impact of malicious updates (e.g., Trimmed Mean and FedMedian~\cite{yin2018icml}). 
% Other defenses implement outlier detection techniques to discard malicious updates from the aggregation performed at the server's end. Those are either based on heuristics (e.g., Krum/Multi-Krum~\cite{blanchard2017nips} and Bulyan~\cite{mhamdi2018pmlr}) or data-driven approaches (e.g., K-means clustering~\cite{shen2016acm} or DnC via spectral analysis~\cite{shejwalkar2021ndss}). 
% Finally, some strategies rely on a centralized ``source of trust'' to spot potential malicious updates (e.g., FLTrust~\cite{cao2020fltrust}).
% Several countermeasures have been proposed in the literature to combat model poisoning attacks on FL systems, i.e., to discard possible malicious local updates from the aggregation performed at the server's end. 
% These techniques range from simple statistics more robust than plain average (e.g., Trimmed Mean and FedMedian~\cite{yin2018icml}) to outlier detection heuristics (e.g., Krum/Multi-Krum~\cite{blanchard2017nips} and Bulyan~\cite{mhamdi2018pmlr}) or data-driven approaches (e.g., spectral analysis via K-means clustering~\cite{shen2016acm} or spectral analysis), or methods based on ``source of trust'' (e.g., FLTrust~\cite{cao2020fltrust}).
% OLD, LONG VERSION
%Several countermeasures have been proposed in the literature to combat Byzantine model poisoning attacks on FL systems.
% Descriptive statistics
% For example, Trimmed Mean and FedMedian aggregate local model updates using more robust statistics than standard average~\cite{yin2018icml}.
%
% % Heuristics for outlier detection
% Many existing Byzantine-resilient strategies implement some outlier detection heuristics to discard the model updates sent by potentially malicious clients from the input of the aggregation function.
% One of the most popular heuristics is Krum~\cite{blanchard2017nips}.
% This strategy tries to mitigate the impact of Byzantine attacks by selecting as a global model the local model with the smallest sum of Euclidean distances to {\em all} the other local models.
% Although powerful, Krum requires the server to know (or, at least, estimate) the number of malicious FL clients upfront, which is generally impossible in a realistic attack scenario. %
% Moreover, Krum may become ineffective for complex, high-dimensional model parameter spaces due to the curse of dimensionality.
% Bulyan~\cite{mhamdi2018pmlr} tries to overcome this issue by combining Krum with a variant of Trimmed Mean.
% % Data-driven outlier detection
% Other strategies use data-driven outlier detection techniques -- e.g., via K-means clustering~\cite{shen2016acm} -- to spot potential malicious local model updates. 
% %For instance, Shen et al. propose to cluster local model updates with K-means and thus identify outliers.
%
% % Other techniques
% As far as the server is concerned, any local model received can be from a potential malicious client. 
% FLTrust~\cite{cao2020fltrust} assumes the server acts as a client, i.e., trains a local model on an additional {\em trustworthy} dataset at the server's end and compares it against all the local models from other clients. 
% This way, the server can rely on some ``source of trust'' when discarding potentially malicious clients.
%\\
% Limitations of existing Byzantine-resilient strategies
Unfortunately, existing defense mechanisms either rely on simple heuristics (e.g., Trimmed Mean and FedMedian by~\cite{yin2018icml}) or need strong and unrealistic assumptions to work effectively (e.g., foreknowledge or estimation of the number of malicious clients in the FL system, as for Krum/Multi-Krum~\cite{blanchard2017nips} and Bulyan~\cite{mhamdi2018pmlr}, which, however, cannot exceed a fixed threshold).
Furthermore, outlier detection methods using K-means clustering~\cite{shen2016acm} or spectral analysis like DnC~\cite{shejwalkar2021ndss} do not directly consider the temporal evolution of local model updates received.
Finally, strategies like FLTrust~\cite{cao2020fltrust} require the server to collect its own dataset and act as a proper client, thereby altering the standard FL protocol.
\\
% OLD, LONG VERSION
% Overall, existing Byzantine-resilient strategies are either simple heuristics (e.g., FedMedian) or, if they are more complex, they rely on strong and unrealistic assumptions to work effectively (e.g., knowing the number of malicious clients in the FL system in advance, as for Krum and alike).
% Furthermore, data-driven outlier detection methods do not consider the temporary evolution of local model updates received (e.g., K-means clustering). 
% Finally, strategies like FLTrust requires the server to collect its own dataset and act as a proper client, thereby altering the standard FL protocol.
%
% Description of the proposed method
This work introduces a novel pre-aggregation \textit{filter} robust to untargeted model poisoning attacks. Notably, this filter $(i)$ operates without requiring prior knowledge or constraints on the number of malicious clients and $(ii)$ inherently integrates temporal dependencies. 
The FL server can employ this filter as a preprocessing step before applying \textit{any} aggregation function, be it standard like FedAvg or robust like Krum or Bulyan.
Specifically, we formulate the problem of identifying corrupted updates as a multidimensional (i.e., matrix-valued) time series anomaly detection task. 
The key idea is that legitimate local updates, resulting from well-calibrated iterative procedures like stochastic gradient descent (SGD) with an appropriate learning rate, show \textit{higher predictability} compared to malicious updates. This hypothesis stems from the fact that the sequence of gradients (thus, model parameters) observed during legitimate training exhibit regular patterns, as validated in Section~\ref{subsec:intuition}. %until convergence. 
%This regularity may be more pronounced for smooth convex loss functions, but it can still be captured within an appropriate time window, even for more complex and convoluted loss surfaces. 
%We provide evidence of this claim in Appendix~B, where we show that the average mutual information (i.e., ``predictability''), calculated over pairs of legitimate model updates sent at different FL rounds, is significantly higher than the corresponding computation for a malicious client.
\\
Inspired by the matrix autoregressive (MAR) framework for multidimensional time series forecasting~\cite{chen2021je}, we propose the FLANDERS ({\em \textbf{F}ederated \textbf{L}earning meets \textbf{AN}omaly \textbf{DE}tection for a \textbf{R}obust and \textbf{S}ecure}) filter.
The main advantages of FLANDERS over existing strategies like FLDetector~\cite{zhao2020multivariate} are its resilience to large-scale attacks, where $50\%$ or more FL participants are hostile, and the capability of working under realistic non-iid scenarios.
We attribute such a capability to two key factors: $(i)$ FLANDERS works without knowing a priori the ratio of corrupted clients, and $(ii)$ it embodies temporal dependencies between intra- and inter-client updates, quickly recognizing local model drifts caused by evil players. Below, we summarize our main contributions:

\begin{itemize}
\item[{\em(i)}]
We provide empirical evidence that the sequence of models sent by legitimate clients is more predictable than those of malicious participants performing untargeted model poisoning attacks.
\\
\item[{\em(ii)}] 
We introduce FLANDERS, the first pre-aggregation filter for FL robust to untargeted model poisoning based on multidimensional time series anomaly detection.
\\
\item[{\em(iii)}] 
We integrate FLANDERS into Flower,\footnote{\scriptsize{\url{https://flower.dev/}}} a popular FL simulation framework for reproducibility.
\\
\item[{\em(iv)}] 
We show that FLANDERS improves the robustness of the existing aggregation methods under multiple settings: different datasets, client's data distribution (non-iid), models, and attack scenarios.
\\
\item[{\em(v)}] 
We publicly release all the implementation code of FLANDERS along with our experiments.\footnote{\scriptsize{\url{https://anonymous.4open.science/r/flanders_exp-7EEB}}}
\end{itemize}

% Paper's structure and organization
The remainder of the paper is structured as follows. %some related work and the current state-of-the-art solutions to security issues that FL entails. 
Section~\ref{sec:background} covers background and preliminaries. 
In Section~\ref{sec:related}, we discuss related work.
Section~\ref{sec:problem} and Section~\ref{sec:method} describe the problem formulation and the method proposed. % to tackle it. 
Section~\ref{sec:experiments} gathers experimental results. %, and Section~\ref{sec:limitations} discusses some limitations of this work.
Finally, we conclude in Section~\ref{sec:conclusion}.
 %discusses the limitations of this work and draws future research directions.
%reports conclusions and draws perspectives for future research directions.

%%%%%%% OLD %%%%%%%
%to overcome the resilience of Byzantine failures in distributed Stochastic Gradient Descent computations. 
% The strength of Krum is its time complexity, which is linear in the gradient dimension. 
% However, the robustness of the approach is guaranteed for gradient-based learning applications only when the majority of the clients are not compromised. 
% Besides, the aggregation mechanism of Krum, as well as that of similar methods, is robust from a coarse-grained perspective and does not provide solutions to errors and perturbations that may occur at inference time.
%A related approach to~\cite{blanchard2017nips} is the work of Su et al.~\cite{su2016dc}. Here, the authors propose an iterated approximate agreement to tackle a multi-layer scenario attacked by Byzantine agents. 
%However, the method works efficiently on the sole discrete context and it is inapplicable to continuous state environments.
%\gabri{Maybe, we should just talk about the main limitations of existing countermeasures without digging into their details (or, we can just mention Krum as this is the most popular one). I will move the description of all these methods to the Related Work section.}
\section{Background on Network Calculus}
\label{sec: background}


\begin{figure*}[tbh]
\centering
\begin{subfigure}[b]{0.3\textwidth}
    \centering
    \includegraphics[width=\linewidth]{images/in-out.png}
    \caption{Arrival and departure data and their relation with delay $d(t)$ and backlog $b(t)$. For a FIFO system, the delay is the horizontal distance between $R(t)$ and $R^*(t)$ but some other multiplexing techniques may shift the data to a later priority, causing a longer delay.}
    \label{fig: data in-out}
\end{subfigure}
\hfill
\begin{subfigure}[b]{0.35\textwidth}
    \centering
    \includegraphics[width=\linewidth]{images/arrival-service.png}
    \caption{Characteristics of an arrival curve and a service curve. From any point of observation, the arriving data never exceeds its arrival curve; the departure data is also never less than the service curve with respect to the data arrival.}
    \label{fig: arrival-service curves}
\end{subfigure}
\hfill
\begin{subfigure}[b]{0.33\textwidth}
    \centering
    \includegraphics[width=\linewidth]{images/bound.png}
    \caption{Delay and backlog bounds of a system. Backlog is the maximum vertical distance between $\alpha(t)$ and $\beta(t)$; FIFO delay is their maximum horizontal distance; but for arbitrary multiplexing, the delay guarantee is when the system clears its buffer, thus it's the intersection of $\alpha(t)$ and $\beta(t)$.}
    \label{fig: system bounds}
\end{subfigure}
\caption{Network calculus framework. We let $R(t)$ and $R^*(t)$ be the arrival and departure data flow of a system; $\alpha(t)$ be the piecewise linear concave arrival curve and $\beta(t)$ be the piecewise linear convex service curve of a system.}
% \hossein{Better to show piece-wise linear concave arrival curve and piece-wise linear convex service curve instead of token-bucket and rate-latency.}}
\end{figure*}

We recall some of the network calculus essentials for a better understanding of the framework used in Saihu. In the following context, we use the following notation: $\mbb{R}^+$ is the set of non-negative real numbers; $[x]_+$ denotes $\max(0, x)$

The data flow is by convention modeled as a left-continuous wide-sense increasing function $R(t): \mbb{R}^+ \mapsto \mbb{R}^+$ with respect to time $t$~\cite{ncbook2001leboudec}. 

A system $\mcal{S}$ receives arrival data described as a cumulative function $R(t)$ and delivers departure data as another cumulative function $R^*(t)$. Figure~\ref{fig: data in-out} illustrates such a system $\mcal{S}$. The benefit of representing a system like this is that we can observe system backlog and delay with such a model. 

\begin{definition}[Backlog and Delay~\cite{ncbook2001leboudec}]
    The backlog of a system at time~$t$ is
    \begin{equation}
        b(t) = R(t) - R^*(t)
    \end{equation}
    
    The virtual delay of a FIFO system at time $t$ is
    \begin{equation}
        d_{FIFO}(t) = \inf \lbp \tau \geq 0 : R(t) \leq R^*(t+\tau) \rbp
    \end{equation}
\end{definition}



The backlog of a system can be viewed as the vertical distance between $R$ and $R^*$. The FIFO (\textit{First-in First-out}) delay is the horizontal distance between $R$ and $R^*$. One may obtain other delay values if the multiplexing technique is not FIFO.

% \begin{figure}
%     \centering
%     \includegraphics[width=0.9\linewidth]{images/in-out.png}
%     \caption{In/out data flow; delay and backlog}
%     \label{fig: data in-out}
% \end{figure}

Since we are interested in the system guarantee instead of a single instance of data flow, we would like to have general bounds to the arrival and departure data flows. Therefore, we define \textit{arrival curve} and \textit{service curve} as the bounds of arrival and departure data flows.

\begin{definition}[Arrival Curve~\cite{ncbook2001leboudec}]
    Given a wide-sense increasing function $\alpha: \mbb{R}^+ \mapsto \mbb{R}^+$, we say that a flow $R(t)$ is $\alpha$-constrained if and only if for all $s \leq t$:
    \begin{equation}
        R(t) - R(s) \leq \alpha(t-s)
    \end{equation}
    We say $R(t)$ has $\alpha$ as an arrival curve.
\end{definition}

\begin{definition}[Service Curve~\cite{ncbook2001leboudec}]
    Given a wide-sense increasing function $\beta: \mbb{R}^+ \mapsto \mbb{R}^+$ and $\beta(0) = 0$. A system $\mcal{S}$ having $R(t)$ and $R^*(t)$ as its arrival and departure flows. We say $\mcal{S}$ offers a service curve $\beta$ if and only if
    \begin{equation}
        R^*(t) \geq (R \otimes \beta)(t) =: \inf_{s \leq t} \lbp R(s) + \beta(t-s) \rbp
    \end{equation}
    where $\otimes$ denotes the min-plus convolution
\end{definition}

Figure~\ref{fig: arrival-service curves} illustrates the arrival and service curves. Any segment of arrival flow $R(t)$ is constrained by arrival curve $\alpha$ and the output curve $R^*(t)$ is always no less than the curve $R\otimes\beta$. As a result, an arrival curve upper bounds the incoming traffic, and a service curve lower bounds the outgoing traffic.

% \begin{figure}
%     \centering
%     \includegraphics[width=\linewidth]{images/arrival-service.png}
%     \caption{Arrival/Service curve}
%     \label{fig: arrival-service curves}
% \end{figure}

We consider 2 special types of curves throughout this paper, \textit{token-bucket} (or sometimes called \textit{leaky-bucket}) curve and \textit{rate-Latency} curve.

\begin{definition}[Token-bucket and Rate-latency~\cite{ncbook2001leboudec}]
    A token-bucket curve $\gamma_{r,b}$ with arrival rate $r$ and burst $b$ is defined as
    \begin{equation}
        \gamma_{r,b}(t) = b + rt
    \end{equation}

    A rate-latency curve $\beta_{R,T}$ with service rate $R$ and latency $T$ is defined as
    \begin{equation}
        \beta_{R,T}(t) = R \lb t - T \rb_+
    \end{equation}
\end{definition}

A token-bucket curve is determined by a burst $b$ and an arrival rate~$r$. Burst represents the maximum possible data volume that can arrive simultaneously, and arrival rate represents the maximum long-term data rate~\cite{bouillard2022tradeoff}.
A rate-latency curve is determined by a latency~$T$ and a service rate~$R$. Latency represents the time a server needs before starting to process the incoming data, and service rate represents the minimum rate to process data after the initial latency.

With the help of arrival and service curves, we can derive delay and backlog bounds for a system $\mcal{S}$ illustrated in Figure~\ref{fig: system bounds}. Suppose a system $\mcal{S}$ has arrival curve $\alpha$ and service curve~$\beta$, its worst-case backlog $b^*$ is the maximum vertical distance between~$\alpha$ and~$\beta$. Similarly, depending on the multiplexing technique applied to the system, its worst-case delay bound $d^*$ is the maximum horizontal distance between $\alpha$ and $\beta$ if $\mcal{S}$ is a FIFO system. If we don't have any information about its multiplexing technique, referred to as arbitrary multiplexing, the best we can say is that when $\alpha$ and $\beta$ intersect each other, where all data has been delivered out of the system. Consequently, the worst-case delay bound for arbitrary multiplexing is the time required for $\mcal{S}$ to clear its buffer.

% \begin{figure}
%     \centering
%     \includegraphics[width=\linewidth]{images/bound.png}
%     \caption{System delay/backlog bounds}
%     \label{fig: system bounds}
% \end{figure}

While a service curve captures the slowest possible output speed of a system, a link's transmission capacity limits the speed as well. Hence, we model this phenomenon using a \textit{greedy shaper} with a sub-additive function $\sigma: \mbb{R}^+ \mapsto \mbb{R}^+$ concatenated with a server. We consider a concatenation as shown in Figure \ref{fig: system}. By convention we assume $\sigma(0) = 0$ and $\beta(t) \leq \sigma(t), \forall t \in \mbb{R}^+$, meaning that the buffer is cleared at the beginning and the service never exceed its physical limitation. With the above definition, such greedy shaper conserves the service provided by the system due to theorem \ref{thm: shaping}.

\begin{figure}[thb]
    \centering
    \includegraphics[width=0.7\linewidth]{images/system.png}
    \caption{Shaping of departure data. A flow that has an arrival curve $\alpha$ feeds into a server with an arrival data flow $R(t)$. The server having service curve $\beta$ takes $R(t)$ and gives a departure data flow $R^*(t)$ to a shaper with shaping function $\sigma$. The shaper takes $R^*(t)$ and shape the data flow as another departure $D(t)$.}
    \label{fig: system}
\end{figure}


\begin{theorem}[Shaping conserves service \cite{ncbook2001leboudec}]
\label{thm: shaping}
Following the system shown in Figure \ref{fig: system}, we have
\begin{equation}
     D = R^* \otimes \sigma \geq \lp R \otimes \beta \rp \otimes \sigma = R \otimes \lp \beta \otimes \sigma \rp = R \otimes \beta
\end{equation}
\end{theorem}

In the following context, we model the shaping function $\sigma$ as a token-bucket curve $\gamma_{C,L}$ with transmission capacity $C$ and the packet size $L$ to capture the link capacity and packetization~\cite{bouillard2022tradeoff}.


\section{Systems and ECP Proxy Applications}
In this section we discuss the HPC system platforms and four ECP proxy applications \cite{ECP} used in our experiments.

We conduct our experiments on the Cray XC40 Theta \cite{THETA} of approximately 12 petaflops peak performance at Argonne National Laboratory and the IBM Power9 heterogeneous system Summit \cite{SUMMIT} of approximately 200 petaflops peak performance at Oak Ridge National Laboratory. In this section, we briefly describe their specifications shown in Table~\ref{tab:sys}. 

\textbf{Theta:}
Theta has 4,392 Cray XC40 nodes. Each node has 64 compute cores (one Intel Xeon Phi Knights Landing (KNL) 7230 with the thermal design power (TDP) of 215 W), shared L2 cache of 32 MB (1 MB L2 cache shared by two cores), 16 GB of high-bandwidth in-package memory Multi-Channel DRAM (MCDRAM), 192 GB of DDR4 RAM, and a 128 GB SSD. MCDRAM can be configured as a shared last level cache L3 (cache mode) or as a distinct NUMA node memory (flat mode) in or somewhere in between. The default memory mode is the cache mode. The Cray XC40 system uses the Cray Aries dragonfly network with user access to a Lustre parallel file system with 10 PB of capacity and 210 GB/s bandwidth. 

In this work, we use GEOPM \cite{ES17} to measure node energy consumption on Theta. The power sampling rate used is approximately 2 samples per second (default). We conduct all autotuning experiments in performance and energy with the cache mode. The compilers on Theta are CrayPE 2.6.5 (default) and clang 14 installed \cite{SOLL}. The aprun command is used to specify to ALPS (Application Level Placement Scheduler) the resources and placement parameters needed for the application at application launch on Theta.  

\textbf{Summit:}
Summit has 4,608 IBM Power System AC922 nodes. Each node contains two IBM POWER9 processors with 42 cores and six NVIDIA Volta V100 accelerators. Each node has 512 GB of DDR4 memory for use by the POWER9 processors and 96 GB of high-bandwidth memory (HBM2) for use by the accelerators. Additionally, each node has 1.6 TB of nonvolatile memory that can be used as a burst buffer. Summit is connected to an IBM Spectrum Scale filesystem providing 250 PB of storage capacity with a peak write speed of 2.5 TB/s. For each Summit node, the TDP of each Volta GPU is 300 W, and the TDP of each Power9 is 190 W. The power consumption of each Summit node is 2,200 W. Although we use the NVIDIA System Management Interface (nvidia-smi) \cite{NSMI} to measure power consumption for each GPU, the power measurement for IBM Power9 is not available to the public. Therefore, we autotune only performance of HPC applications on Summit. The compilers on Summit are gcc 9.1.0 (default) and nvhpc 21.3. The jsrun command is used for managing an allocation that is provided by an external resource manager within IBM Job Step Manager (JSM) software package on Summit. 

\begin{table}
\center
\caption{System Platform Specifications and Tools}
\begin{tabular}{c}
  \includegraphics[width=.80\linewidth]{figs/theta-summit.png}
  \end{tabular}
\label{tab:sys}       
\end{table}  

\subsection{ECP Proxy Applications}
%In high-performance computing, ECP proxy applications \cite{ECP} are small, simplified codes that allow application developers to share important features of large applications without forcing collaborators to assimilate large and complex code bases. 
In this section we discuss four hybrid MPI/OpenMP ECP proxy applications for our experiments: XSBench \cite{XSB}, SWFFT \cite{SWF}, AMG \cite{AMG}, and SW4lite \cite{SW4L}.

\subsubsection{Weak-Scaling Applications}
We discuss the three weak-scaling ECP proxy applications XSBench, SWFFT, and AMG.

XSBench \cite{XSB} is a mini-app representing a key computational kernel of the Monte Carlo neutron transport algorithm and represents the continuous energy macroscopic neutron cross section lookup kernel. It serves as a lightweight stand-in for full neutron transport applications like OpenMC \cite{OMC}. This code provides a much simpler and more transparent platform for determining performance benefits resulting from a given hardware feature or software optimization.  XSBench provides an MPI mode which runs the same code on all MPI ranks simultaneously with no decomposition across ranks of any kind, and all ranks accomplish the same work. It is an embarrassingly parallel implementation. It supports history-based transport (default):  parallelism is expressed over independent particle histories, with each particle being simulated in a serial fashion from birth to death; and event-based transport: parallelism is instead expressed over different collision (or "event") types.  XSBench is the hybrid MPI/OpenMP code written in C and supports OpenMP offload. The OpenMP offload implementation only supports the event-based transport. The problem size is large as default. 

SWFFT \cite{SWF} is to test the Hardware Accelerated Cosmology Code (HACC) 3D distributed memory discrete fast Fourier transform (FFT) with one forward FFT and one backward FFT. It assumes that global grid will originally be distributed between MPI ranks using a 3D Cartesian communicator. That data needs to be re-distributed to three 2D pencil distributions in turn in order to compute the double-precision FFTs along each dimension. SWFFT is the hybrid MPI/OpenMP code written in C++ and C and requires the cubic number of MPI ranks and FFTW3 (double precision, OpenMP version) installed. We configure it as weak scaling case. The problem size is 4096x4096x4096 for 4096 MPI ranks. We also set the number of run tests 2.

AMG \cite{AMG} a parallel algebraic multigrid solver for linear systems arising from problems on unstructured grids and builds linear systems for various 3-dimensional problems. Parallelism is achieved by data decomposition. AMG achieves this decomposition by simply subdividing the grid into logical X x Y x Z (in 3D) chunks of equal size. It is the hybrid MPI/OpenMP code written in C. The problem size is the 3D Laplace problem "-laplace -n 100 100 100 -P X Y Z".
This will generate a problem with 1,000,000 grid points per MPI process with a domain of the size 100*X x 100*Y x 100*Z. 

\subsubsection{Strong-Scaling Application}

SW4lite \cite{SW4L} is a bare bone version of SW4 \cite{SP12, SB18} (Seismic Waves, 4th order accuracy) intended for testing performance in a few important numerical kernels of SW4.  SW4 implements substantial capabilities for 3-D seismic modeling with a free surface condition on the top boundary, absorbing super-grid conditions on the far-field boundaries, and an arbitrary number of point force and/or point moment tensor source terms. It uses a fourth order in space and time finite-difference discretization of the elastic wave equations in displacement formulation. The large problem LOH.1-h50 is from the SCEC (Southern California Earthquake Center) test suite \cite{Day01}. It sets up a grid with a spacing h (=50) over a domain (X x Y x Z) 30000 x 30000 x 17000. It will run from time t=0 to t=9. The material properties are given by the block commands. They describe a layer on top of a half-space in the z-direction. A single moment point source is used with the time dependency being the Gaussian function. SW4lite is the hybrid MPI/OpenMP code written in C++ and Fortran90. In \cite{WU21}, performance and energy of SW4lite were optimized for the improved version. We use the improved version to define the parameter space for autotuning.

\subsubsection{Compiling Time for Each Application}

Table~\ref{tab:cp} shows the average compiling time (in seconds) for each ECP proxy application on Theta and Summit. We measured the compiling time for each application five times to get the average compiling time. We observe that the compiling time for SW4lite is 162.066 s on Theta and 58 s on Summit. This really impacts the autotuning wall-clock time in Step 4 shown in Figure \ref{fig:pf}. Because of loading the NVidia nvhpc module to compile the XSBench OpenMP offload version for using GPUs on Summit, it takes 4.645 s, which is much larger than that on Theta.

\begin{table}[ht]
\center
\caption{Compiling time (s) on Theta and Summit}
\begin{tabular}{|r|c|c|c|c|}
\hline
System & XSBench & SWFFT & AMG & SW4lite  \\
\hline
Theta &  2.021 & 3.494  & 2.825   &   162.066 \\
\hline
Summit & 4.645  & 3.781  & 2.757 &  58.000 \\
\hline
\end{tabular}
\label{tab:cp}
\end{table}


\begin{algorithm}[h]
   \caption{GCRL with planning + \highlight{\ALGname}}
   \label{alg:framework}
\begin{algorithmic}
\State {\bfseries Input}: Number of training episodes $M$, horizon $H$
\State Initialize replay buffer $\mathcal{B} \leftarrow \varnothing$.
\State Initialize the parameters of goal-conditioned policy $\pi_{\theta}$.
\State Initialize the parameters of action-value function $Q_{\phi}$.
\For{$m=1, 2, 3, \ldots M$}
\State Reset the environment.
\State Sample a target goal $g$ and an initial state $s_{0}$.
    \For{$t=1, 2, 3, \ldots H$}
    \State Build a graph $\mathcal{H} = (\mathcal{V}, \mathcal{E}, d)$ using $\mathcal{B}$.
    \State Find the shortest subgoal-path $\tau_{g}$ from $s_{t}$ to $g$.
    \State \highlight{Find a desired subgoal $l^{*}$ via Algorithm~\ref{alg:skip}.}
    \State Collect a transition $(s_{t}, a_{t}, r_{t})$ using $\pi_{\theta} (s_{t}, l^{*})$.
    \State Store the transition and the planned path $\tau_{g}$ in $\mathcal{B}$.
    \EndFor
\State Update $Q_{\phi}$ using $\mathcal{L}_{\mathtt{critic}} (\phi)$ of Equation~\ref{eq:ddpg_critic}
\State Update $\pi_{\theta}$ using $\mathcal{L}_{\mathtt{actor}} (\theta) + \highlight{\lambda \mathcal{L}_{\mathtt{\ALGname}} (\theta)} $ of Equation~\ref{eq:total_loss}
\EndFor
\end{algorithmic}
\end{algorithm}
\section{Single Trajectory Model}
\label{sec:single trajectory model}
This section outlines our method to learn the distribution of aircraft trajectories relative to their flight procedures and to generate synthetic traffic scenarios. 
The interactions between multiple aircraft are not considered in this section. Our approach follows the following steps, which are described in detail in this section. First, we segment our aircraft trajectories to \textbf{construct input vectors} for our Gaussian mixture models. Next, we use these input vectors to \textbf{train a Gaussian Mixture Model} for each flight stage. To avoid overfitting and manage noise within our GMMs, we \textbf{use low-rank approximations of the GMM covariance matrices}. Finally, we use these approximated matrices to \textbf{generate synthetic trajectories}. \Cref{fig:flowchart} illustrates this process. %We note that our use of Gaussian Mixture Models assumes that our data points are generated from multiple Gaussian distributions.

\begin{figure}
    \centering
    \tikzstyle{process} = [rectangle, rounded corners, minimum width=3cm, minimum height=1cm, minimum width=5.5cm, text centered, draw=black]
    \tikzstyle{arrow} = [thick, ->, >=stealth]
    \begin{tikzpicture}[node distance=2cm]
    
        \node (step1) [process] {Segment aircraft trajectories};
        \node (step2) [process, below of=step1] {Fit GMMs};
        \node (step3) [process, below of=step2] {Compute low-rank approximation};
        \node (step4) [process, below of=step3] {Generate synthetic trajectories};
    
        \draw [arrow] (step1.south) -- (step2.north) node[midway,right] {Trajectory segments};
        \draw [arrow] (step2.south) -- (step3.north) node[midway,right] {GMMs};
        \draw [arrow] (step3.south) -- (step4.north) node[midway,right] {Low-rank covariance matrices};
    
    \end{tikzpicture}
        \caption{Flowchart overview of single trajectory model\label{fig:flowchart}}
\end{figure}

\subsection{Construction of Input Vector}
\label{subsec:input vector construction}
Before training a separate Gaussian mixture model for each flight stage, we need to construct the input vector sets in the proper format. 
First, each aircraft trajectory is divided into radar vector and final approach segments based on the distance to its IAP.
Then, the radar vector trajectory is assigned to one of the radar vector procedures using dynamic time warping (DTW) \cite{Muller2007}.

DTW is an algorithm for measuring similarity between two temporal sequences, which may vary in length.
It uses a dynamic programming approach to find the shortest distance between these sequences.
Given a pair of vectors ${x_1=[x_1^{(1)},\ldots, x_1^{(m)}] \in \mathbb{R}^m}$ and ${x_2=[x_2^{(1)}, \ldots, x_2^{(n)}] \in \mathbb{R}^n}$, the DTW distance between them is computed as
\begin{equation}
    DTW(x_1, x_2) = D(m, n)
\end{equation}
where for all ${i \in \{1,\ldots,m\},\ j \in \{1,\ldots,n\}}$,
\begin{equation}
    D(i, j) = \lVert x_1^{(i)} - x_2^{(j)} \rVert_2 + \min \begin{cases}
        D(i, j-1) \\ D(i-1, j) \\ D(i-1, j-1).
    \end{cases}
\end{equation}
After measuring the DTW distances between the radar vector trajectory and each of the radar vector procedures, we label the trajectory as the procedure to which it is closest, as measured by DTW distance.

Now with the two sets of aircraft trajectory-procedure pairs, one for each flight stage, we would like to train a GMM that takes as input the sequence of deviations between aircraft positions and the procedure points.
One challenge for training a GMM is that all the input vectors are required to be of the same length.
To deal with the issue of varying lengths of trajectories, we separately interpolate each dimension in an aircraft trajectory as a polynomial function of time and then re-sample a fixed number of points.
We also generate the procedural trajectories by interpolating the waypoints of the procedures and re-sampling the same number of points.
To integrate time into the spatial procedural trajectories of IAPs, we extract aircraft trajectories that pass very close to all the waypoints and take their mean. This allows us to estimate how a trajectory projected on a nominal path would proceed over time. %\todo{why do we do this?} 
The radar vector procedures already involve temporal factors because the procedures are defined as the nominal paths extracted from aircraft trajectories.
% \soyeonstrike{For the interpolation, we use piecewise cubic Hermite interpolating polynomials (PCHIP).
% PCHIP constructs a piecewise function where each piece ${p_i(x)}$ is a cubic polynomial for the observed data points with the specified values and first derivatives (slopes) at the interpolation points.}

\added{As done in prior work \cite{kochenderfer2010airspace}}, we use the piecewise cubic Hermite interpolation method. \added{We select this method because it has a number of desirable characteristics. First, it ensures that the interpolated function is continuous by fitting a cubic polynomial for each piece of the function and requiring first derivative continuity between pieces. Second, it preserves derivative information such as monotonicity; i.e., where the data is monotonic, the interpolated function will be monotonic. Finally, due to its use of low-degree polynomials, it generally avoids oscillation (Runge's phenomenon) that can be common in interpolation methods using high-degree polynomials. As a result of these properties, it is suitable for interpolating trajectory data and in particular procedural trajectories that should not oscillate between sample points.} \deleted{This method fits a cubic polynomial for each piece of the given function and imposes the continuity of the first derivative.
It preserves monotonicity and avoids oscillation in the intervals where the data is monotonic.
This property makes the piecewise cubic Hermite interpolation method appropriate for interpolating trajectory data, especially the procedural trajectories, which should not oscillate between the sample points.}

% For each sub-interval $x_i \leq x \leq x_{i+1}$,
% \begin{align}
%     p_i(x) = a_ix^3 + b_ix^2 + c_ix + d_i
% \end{align}
% with constraints:
% \begin{align}
%     \begin{split}
%         p_i(x_i) &= y_i\\
%         p_i(x_{i+1}) &= y_{i+1}\\
%         p'_i(x_i) &= y'_i = \frac{y_{i+1}-y_{i-1}}{x_{i+1}-x_{i-1}}\\
%         p'_i(x_{i+1}) &= y'_{i+1} = \frac{y_{i+2}-y_i}{x_{i+2}-x_i}
%     \end{split}
% \end{align} 

Finally, we have two sets of training input vectors. An individual input ${\tau \in \mathbb{R}^{3T+2}}$ is defined as
\begin{align}
\begin{split}
\tau = &[t, d, (x_1-x_1^p), (y_1-y_1^p), (z_1-z_1^p),\\
&\ldots,(x_T-x_T^p), (y_T-y_T^p), (z_T-z_T^p)]
\end{split}
\label{eq:single_model_training_data}
\end{align}
where $t$ and $d$ are the transit time and the total distance of the trajectory,
$[x_{1:T}, y_{1:T}, z_{1:T}]$ are the ENU coordinates of the aircraft trajectory at each timestep from 1 to $T$, and $[x_{1:T}^p, y_{1:T}^p, z_{1:T}^p]$ are the ENU coordinates of the corresponding procedural trajectory. 
We keep the transit time and total distance information of each trajectory so that we can later generate synthetic trajectories with reasonable airspeed.

%%%%%%%%%%%%%%%%%%%%%%%%%%%%%%%%%%%%%%%%%%%%%%%%%%%%%%%%%%%%%%%%%%
\subsection{Gaussian Mixture Model}
\label{subsec:GMM}

The Gaussian mixture model (GMM) is a probabilistic generative model that assumes the data points are generated from a mixture of Gaussian distributions.
% \soyeonstrike{
% It provides a convenient analytical form for representing data distributions that are potentially multimodal.
% Due to its flexibility of capturing an arbitrary level of complexity, GMM is widely used for many tasks including density estimation, statistical inference, and clustering.}
This model is capable of representing the multimodality and the uncertainty of complex data distributions \cite{kochenderfer2015decision}.
Additionally, GMMs are commonly used across broad domains to generate samples that capture representative characteristics of the training data \cite{liu2019improving, chokwitthaya2020applying, li2021gaussian}. %\todo{do  we need to cite these statements about GMMs, or okay to treat as common sense?}
These advantages align with our objectives of modeling the aircraft behavior and generating realistic trajectories. %\todo{not sure what this is missing that the reviewer was hoping for regarding justifying GMM}

For each flight stage, we construct a GMM with a set of training input vectors defined as (\ref{eq:single_model_training_data}).
If a single vector $\tau$ is sampled from $K$ mixture components, the marginal probability distribution of $\tau$ is 
% Consider a training set ${\mathbf{\tau}=[\tau^{(1)}, \ldots, \tau^{(m)}] \in \mathbb{R}^{m \times n}}$ where each data point ${\tau^{(i)} \in \mathbb{R}^{n}}$ is generated from a mixture of $K$ Gaussian distributions. The density function of $\tau^{(i)}$ is
\begin{align}
    \label{eq:gmm}
    p(\tau) = \sum_{j=1}^K \pi_j \mathcal{N}(\tau \mid \mu_j, \Sigma_j)
\end{align}
where ${\pi_j}$ are mixing coefficients that must satisfy ${ {\textstyle\sum}_{j=1}^{K} \pi_j=1}$ and ${\pi_j \geq 0}$ for all ${j \in \{1,\ldots,K\}}$.
Each Gaussian density ${\mathcal{N}(\tau  \mid \mu_j, \Sigma_j)}$ is called a component of the mixture.
The maximum likelihood estimates of the parameters ${\{\pi_j, \mu_j, \Sigma_j\}}$ for all $j$ given a dataset of the observations are obtained using the expectation-maximization (EM) algorithm \cite{dempster1977maximum}.

Our model learns the sequence of deviations (i.e., relative positions) of an aircraft from the corresponding points of the procedure. Fig. \ref{fig:gmm_deviations} shows an example sequence of deviations. 
The aircraft positions and the procedure points are indicated by black and blue crosses respectively, and the deviations are indicated by red dotted lines.
\begin{figure}[tb!]
\centering
\setlength\figureheight{4.8cm}
\setlength\figurewidth{6.3cm}
% This file was created by matplotlib2tikz v0.6.18.
\begin{tikzpicture}
\node[inner sep=0pt] (plot) at (0,0)
    {\includegraphics[height=4.8cm,width=6.3cm]{figures/gmm_dev.png}};

\node[text=blue, font=\fontsize{7pt}{9}\selectfont] at (-3.65,0.14) {\begin{math} ( x^p_6, y^p_6 ) \end{math}};
\node[text=blue, font=\fontsize{7pt}{9}\selectfont] at (-3.45,-0.7) {\begin{math} ( x^p_5, y^p_5 ) \end{math}};
\node[text=blue, font=\fontsize{7pt}{9}\selectfont] at (-1.55,-0.68) {\begin{math} ( x^p_4, y^p_4 ) \end{math}};
\node[text=blue, font=\fontsize{7pt}{9}\selectfont] at (-0.45,-0.68) {\begin{math} ( x^p_3, y^p_3 ) \end{math}};
\node[text=blue, font=\fontsize{7pt}{9}\selectfont] at (0.8,-0.6) {\begin{math} ( x^p_2, y^p_2 ) \end{math}};
\node[text=blue, font=\fontsize{7pt}{9}\selectfont] at (2.1,-0.28) {\begin{math} ( x^p_1, y^p_1 ) \end{math}};

\node[text=black, font=\fontsize{7pt}{9}\selectfont] at (-2.1,0.15) {\begin{math} ( x_6, y_6 ) \end{math}};
\node[text=black, font=\fontsize{7pt}{9}\selectfont] at (-1.96,-0.3) {\begin{math} ( x_5, y_5 ) \end{math}};
\node[text=black, font=\fontsize{7pt}{9}\selectfont] at (-2.46,-1.35) {\begin{math} ( x_4, y_4 ) \end{math}};
\node[text=black, font=\fontsize{7pt}{9}\selectfont] at (-1.6,-2.1) {\begin{math} ( x_3, y_3 ) \end{math}};
\node[text=black, font=\fontsize{8pt}{9}\selectfont] at (-0.19,-2.4) {\begin{math} ( x_2, y_2 ) \end{math}};
\node[text=black, font=\fontsize{8pt}{9}\selectfont] at (1.3,-2.5) {\begin{math} ( x_1, y_1 ) \end{math}};

\end{tikzpicture}
\caption{Example sequence of deviations between aircraft and procedural trajectory.}
\label{fig:gmm_deviations}
\end{figure}


%%%%%%%%%%%%%%%%%%%%%%%%%%%%%%%%%%%%%%%%%%%%%%%%%%%%%%%%%%%%%%%%%%%%
\subsection{Low-rank Approximation of Covariance Matrices}
\label{subsec:low-rank approximation}
Derived from the aircraft and procedural trajectories as in (\ref{eq:single_model_training_data}), the input data matrix of our model is likely to be high-dimensional and contain redundant features.
Also, the model can overfit the noise in the training set. %sparsity
To eliminate redundant features and reduce overfitting, we perform a low-rank approximation for each covariance matrix of our GMM in Section \ref{subsec:GMM} using eigenvalue decomposition. 

Consider the covariance of the $j$th Gaussian component ${\Sigma_j \in \mathbb{R}^{n \times n}}$.
The eigenvalue decomposition of ${\Sigma_j}$ is
\begin{align}
    \begin{split}
    \Sigma_j = Q\Lambda Q^{-1} &= Q\Lambda Q^T
    = \sum_{i=1}^n \lambda_i q_i q_i^T
    \end{split}
\end{align}
where ${\Lambda = \text{diag}(\lambda_1, \ldots, \lambda_n) \in \mathbb{R}^{n \times n}}$ is a diagonal matrix with eigenvalues in decreasing order, and ${Q \in \mathbb{R}^{n \times n}}$ is a matrix of the associated eigenvectors.

The best rank-${k}$ (${k \leq n}$) approximation of ${\Sigma_j}$ is obtained by 
\begin{align}
    \widehat{\Sigma}_j = Q_k \Lambda_k Q_k^{-1}
\end{align}
where ${\Lambda_k \in \mathbb{R}^{k \times k}}$ is a diagonal matrix with the largest $k$ eigenvalues, and the columns of ${Q_k \in \mathbb{R}^{n \times k}}$ are the first $k$ eigenvectors.
This is closely related to principal component analysis (PCA) where the $k$ principal axes, a set of orthonormal axes onto which the projection of the original data maximizes variance, are given by the first $k$ eigenvectors.
We can obtain equivalent results from performing a singular value decomposition of $\tau_j$, a set of data vectors assigned to the $j$th Gaussian component.

While the eigenvalue decomposition or PCA provides an analytical solution, both require the rank $k$ to be specified. 
To determine the optimal rank $k^*$ from the observed data rather than setting a specific value for $k$, we adopt the probabilistic principal component analysis (PPCA), a probabilistic formulation of PCA based on a latent variable model \cite{tipping1999mixtures}.

Consider a dataset ${x = \{x_i\}_{i=1}^m \in \mathbb{R}^{m \times n}}$ of $m$ observations.
PPCA assumes that each observation ${x_i \in \mathbb{R}^n}$ is generated from a low-dimensional latent variable ${z_i \in \mathbb{R}^k}$ (${k<n}$) via the following model. For ${i \in \{1,\ldots ,m\}}$, 
\begin{align}
    \label{eq:ppca}
    x_i &= W z_i+\mu+\varepsilon_i
\end{align}
where ${z_i \sim \mathcal{N}(0,I)}$ is a Gaussian latent variable with unit variance, 
${W \in \mathbb{R}^{n \times k}}$ is the weight matrix explaining the dependencies between latent and observed variables,
${\mu \in \mathbb{R}^n}$ is the location parameter that shifts the data,
and ${\varepsilon_i \sim \mathcal{N}(0, \sigma^2 I)}$ is an isotropic Gaussian noise unique to each observed variable.
From (\ref{eq:ppca}), we can compute the following conditional and marginal distributions:
\begin{align}
    \begin{split}
    x_i \mid z_i &\sim \mathcal{N}(W z_i+\mu, \sigma^2 I)\\
    x_i &\sim \mathcal{N}(\mu, WW^T + \sigma^2 I).
    \end{split}
\end{align}

The maximum likelihood estimate (MLE) of the parameters ${\{W, \sigma^2 \}}$ can be solved in closed form \cite{tipping1999mixtures} or using the EM algorithm \cite{roweis1998algorithms}.
The columns of the estimated $W$ define the principal subspace of standard PCA.

The optimal rank $k^*$ can be determined by choosing the latent dimension ${\mathbb{R}^{k}}$ that maximizes the marginal likelihood of the model.

% \begin{align}
%      \max_\theta{\log p(x;\theta)} &= \max_\theta{\sum_{i=1}^m \log p(x_i;\theta)}.
%     %  &= \max_\theta{\sum_{i=1}^m \log  {\int_{z_i} p(x_i \mid z_i;\theta) p(z_i;\theta) dz_i}}
% \end{align}
% \begin{align}
%     z_i \mid x_i &\sim \mathcal{N}(M^{-1} W^T (x_i-\mu), \sigma^2 M^{-1})\\
%     &\text{ where } M=W^T W + \sigma^2 I
% \end{align}

%%%%%%%%%%%%%%%%%%%%%%%%%%%%%%%%%%%%%%%%%%%%%%%%%%%%%%%%%%%%%%%%%%%%%%%%%%%%%%
\subsection{Trajectory Generation}
\label{subsec:trajs_generation}
Once we have the GMM parameters for each segment, we can generate synthetic trajectories of aircraft positions based on the trained models and test procedures.
To test the model on a set of procedures not used for the training, we need the procedural trajectories with the relative frequencies of procedures in each segment.

We start generating an aircraft trajectory with its radar vector segment.
First, we randomly select one of the test radar vector procedures with probability proportional to their relative frequencies.
Then, we sample a sequence of deviations ${\tau^v}$ from the radar vector GMM. Next, we construct a trajectory of aircraft positions using the sampled deviations and the test procedural trajectory.
While the total distance an aircraft travels varies with the procedure it follows, the reconstructed trajectory always has the same number of points.
To generate a trajectory with reasonable airspeed, we first compute the adjusted total transit time as ${t'^v = (\tau^v_1 / \tau^v_2) \times d'^v}$
% \begin{align}
%     t'^v = \frac{\tau^v_1}{\tau^v_2} \times d'^v
% \end{align}
where ${\tau^v_1, \tau^v_2}$ are the transit time and total distance of the sample, and ${d'^v}$ is the total distance of the given test procedure.
Then we align the trajectory with a vector of evenly spaced numbers over the interval ${[0, t'^v]}$.

For a smooth transition between two separately modeled segments of our generated trajectory,
we take the final $n$ positions of the reconstructed radar vector trajectory to compute the first $n$ deviations from the test final approach procedure.
Then, we form a conditional distribution of the final approach GMM to sample the remainder of the final approach segment given the first $n$ measurements.
% https://stats.stackexchange.com/questions/348941/general-conditional-distributions-for-multivariate-gaussian-mixtures

Suppose the input vector and the Gaussian components in (\ref{eq:gmm}) are partitioned as
\begin{align}
    % p(\tau) &= \sum_{j=1}^K \pi_j \mathcal{N}(\tau \mid \mu_j, \Sigma_j)\\
    p\left(
    \begin{bmatrix} 
        \tau_a \\ \tau_b 
    \end{bmatrix}\right) 
    &= \sum_{j=1}^K \pi_j \mathcal{N}\left(
    \begin{bmatrix} 
        \tau_a \\ \tau_b 
    \end{bmatrix} \;\middle|\;
    \begin{bmatrix}
        \mu_{j,a} \\ \mu_{j,b} 
    \end{bmatrix}, 
    \begin{bmatrix}
        \Sigma_{j,aa} \ \Sigma_{j,ab} \\
        \Sigma_{j,ba} \ \Sigma_{j,bb} 
    \end{bmatrix}\right).
\end{align}

Then, the conditional distribution of ${\tau_b}$ given ${\tau_a}$ is
\begin{align}
    \begin{split}  
    p(\tau_b \mid \tau_a) &=
    \sum_{j=1}^K \pi_{j,b \mid a}
    \mathcal{N}\left(\tau_b \mid \mu_{j, b \mid a}, \Sigma_{j, b \mid a}
    \right)\\
    \text{where } \ \pi_{j,b \mid a} &= \frac{\pi_j \mathcal{N}\left( \tau_a \mid \mu_{j,a}, \Sigma_{j,aa} \right)}{\sum_{k=1}^K \pi_k \mathcal{N}\left( \tau_a \mid \mu_{k,a}, \Sigma_{k,aa} \right)} \\
    \mu_{j, b \mid a} &= \mu_{j,b} + \Sigma_{j,ba}\Sigma_{j,aa}^{-1}(\tau_a - \mu_{j,a}) \\
    \Sigma_{j,b \mid a} &= \Sigma_{j,bb} - \Sigma_{j,ba} \Sigma_{j,aa}^{-1} \Sigma_{j,ba}^T.
    \end{split}
    \label{eq:conditional_density_GMM}
\end{align}

We form this conditional distribution for the final approach GMM by partitioning each vector $\tau^f$ as illustrated in Fig. \ref{fig:traj_generate_conditional}.
In our case, $\tau_a$ is defined as the first $n$ 3D coordinates of deviations and $\tau_b$ corresponds to the remainder of the vector.
In the figure, the blue crosses indicate the procedural points along the IAP. 
The set of dotted lines are the sequence of deviations $\tau^f$ partitioned into $\tau_a$ and $\tau_b$, and
the dots are the reconstructed trajectory of aircraft positions. 
Those marked red correspond to the conditioned part.

\begin{figure}[tb!]
\centering
% This file was created by matplotlib2tikz v0.6.18.
\begin{tikzpicture}

\node[inner sep=0pt] (plot) at (0,0)
    {\includegraphics[width=5.5cm, height=5.2cm]{figures/traj_generate_background.pdf}};
    
\node[mark size=3pt,color=gray] at (-1.1,-1.43) {\pgfuseplotmark{*}};
\node[mark size=3pt,color=gray] at (-0.72,-1.85) {\pgfuseplotmark{*}};

\node[mark size=3pt,color=red] at (-1.33,-0.96) {\pgfuseplotmark{*}};
\node[mark size=3pt,color=red] at (-1.42,-0.48) {\pgfuseplotmark{*}};

\node[mark size=3pt,color=black] at (-1.45,-0.02) {\pgfuseplotmark{*}};
\node[mark size=3pt,color=black] at (-1.35,0.4) {\pgfuseplotmark{*}};
\node[mark size=3pt,color=black] at (-1.2,0.75) {\pgfuseplotmark{*}};

\node[text=black, font=\fontsize{14pt}{9}\selectfont] at (-0.4,0.12) {\begin{math} \tau_b = \end{math}};

\node[text=red, font=\fontsize{14pt}{9}\selectfont] at (-0.4,-0.62) {\begin{math} \tau_a = \end{math}};

\node[text=black, font=\fontsize{12pt}{9}\selectfont] at (0.9,0.22) {\begin{math} \tau^f_{1:2,3n+3:} \end{math}};

\node[text=red, font=\fontsize{12pt}{9}\selectfont] at (0.75,-0.52) {\begin{math}  \tau^f_{3:3n+2} \end{math}};

\node[text=black, font=\fontsize{14pt}{9}\selectfont] at (2.6,-0.15) {\begin{math} \tau^f \end{math}};

\node[text=black, font=\fontsize{12pt}{9}\selectfont] at (-2.7,0.1) {\begin{math} n \end{math}};

\node[text=gray, align=right, font=\fontsize{9pt}{9}\selectfont, ] at (-2.35,-1.7) {\textbf{radar vector} \\ \textbf{segment}};

\node[text=blue, font=\fontsize{9pt}{9}\selectfont, ] at (-1.6,1.6) {\textbf{IAP}};

\end{tikzpicture}
\caption{Partition of the sequence of deviations for the final approach segment to form a conditional distribution.}
\label{fig:traj_generate_conditional}
\end{figure}

After we sample the remaining final approach segment of deviations from the conditional, we reconstruct an aircraft position trajectory as we need for the radar vector segment.
We also repeat the process for integrating time into the trajectory using $\tau^f_1$ and $\tau^f_2$.
Finally, the whole synthetic trajectory is obtained by combining the aircraft trajectories of both segments.

% For a smooth transition between two separately modeled segments of our generated trajectory, 
% we condition each Gaussian distribution component of the final approach GMM on partial observations computed with the final $n$ positions of the reconstructed radar vector trajectory.
% A sequence of deviations for the final approach segment $\tau^f$ and its associated Gaussian distribution can be partitioned as 
% \begin{align}
%     \begin{bmatrix} \tau_{3:3n+2}^f \\ \tau_{3n+3:}^f \end{bmatrix} =
%     \begin{bmatrix} \tau_{a} \\ \tau_{b} \end{bmatrix} &\sim
%     \mathcal{N}\left(\begin{bmatrix}
%                 \mu_a \\ \mu_b 
%             \end{bmatrix}, \begin{bmatrix}
%                     \Sigma_{aa} & \Sigma_{ab} \\
%                     \Sigma_{ba} & \Sigma_{bb} 
%                 \end{bmatrix}\right)
% \end{align}
% where $\tau_a$ corresponds to the first $n$ coordinates of deviations and $\tau_b$ is the remainder, as illustrated in Fig. \ref{fig:traj_generate_conditional}.
% Then, we can form the posterior distribution 
% \begin{align}
%     \begin{split}
%     p(\tau_{3n+3:}^f \mid \tau_{3:3n+2}^f) &= \mathcal{N}\left(\mu_{b \mid a}, \Sigma_{b \mid a}
%     \right)\\
%     \text{where } \ \mu_{b \mid a} &= \mu_b + \Sigma_{ba}\Sigma_{aa}^{-1}(\tau_a - \mu_a) \\
%     \Sigma_{b \mid a} &= \Sigma_{bb} - \Sigma_{ba} \Sigma_{aa}^{-1} \Sigma_{ba}^T.
%     \end{split}
%     \label{eq:conditional_density_GMM}
% \end{align}
% to sample the remainder of the final approach segment given the first $n$ measurements, obtained by computing the deviations between the last $n$ positions of the reconstructed radar vector trajectory and the first $n$ points of the final approach procedure.
% Then, the aircraft position trajectory is constructed using the sample and the remainder of the final approach procedure.

% Finally, the whole synthetic trajectory is obtained by combining the aircraft trajectories of both segments.


\begin{table}[]
    \centering
        \caption{Zero-shot task performance of \texttt{base/large} models after parameter-efficient training.$LwA$/$DA$ indicates adapter types, corresponding to (rows h/f in Table \ref{tab:ablations}). }
    \label{tab:performance}
    \begin{adjustbox}{max width=\textwidth}
        \begin{tabular}{lccccccccc}
% \hline & \multicolumn{4}{c}{ Pre-training Task } & & \multicolumn{2}{c}{ Zero-Shot Performance } \\
% \cline { 2 - 5 } \cline { 6 - 7 } 
\toprule
% \multicolumn{4}{c}{Components} &  \multicolumn{5}{c}{Flickr30k True 0-shot (1k test set)} \\
\multicolumn{4}{c}{Model (591k Training Pairs)} &  & \multicolumn{2}{c}{Flickr} & & \multicolumn{2}{c}{ImageNet V2} \\
\cmidrule(l{0.5em}r{0.5em}){2-4}  \cmidrule(l{0.5em}r{0.5em}){5-8} \cmidrule(l{0.5em}r{0.5em}){9-10}
& Configuration & \# Trainable & \% Trained &  TR@1 & IR@1 & TR@5 & IR@5 & Acc-1 & Acc-5 \\
\midrule
 % (a) &   LilT-tiny & 736.45 K & 7.37\% & 15.7 & 12.4 & 37.4 & 31.56 & 5.61 & 14.54  \\
 % (c) &   LilT-small & 5.19 M & 10.28\% & 37.6 & 27.38 & 66.9 & 54.96 & 10.92 & 23.99  \\
 (a) &   LilT$_{DA}$-base & 14.65 M & 7.51\% & 47.6 & 34.46 & 74.1 & 64.92 & 12.94 & 28.39  \\
 (b) &   LilT$_{DA}$-large & 25.92 M & 4.06\% & 57.6 & 42.18 & 82.2 & 72.38 & 13.97 & 30.89  \\ \midrule
  (c) &   LilT$_{LwA}$-base & 14.67 M & 7.01\% & 56.8 & 41.7 & 81.1 & 70.74 & 12.18 & 27.78  \\
   (d) &   LilT$_{LwA}$-large & 51.18 M & 7.43\% & 63.5 & 50.7 & 88.5 & 79.14 & 14.05 & 31.31 \\
 \midrule
 % (b) & LiT-tiny & 4.45 M & 44.57\% & 25.0 & 16.62 & 49.2 & 40.06 & 10.06 & 22.15  \\
 % (g) & LiT-small & 28.73 M & 56.98\% & 37.3 & 26.46 & 66.6 & 54.78 & \textbf{15.14} & 29.16  \\
 (e) & LiT-base & 109.28 M & 56.01\% & 44.1 & 29.64 & 72.1 & 59.94 & 15.0 & 29.44 \\
 % (d) & CLIP-tiny & 9.99 M & 100.0\% & 34.6 & 25.1 & 62.0 & 51.8 & 7.46 & 18.43  \\
 % (h) & CLIP-small & 50.42 M & 100.0\% & 50.4 & 38.34 & 76.8 & 66.56 & 10.64 & 24.42  \\
 (f) & CLIP-base & 195.13 M & 100.0\% & 56.1 & 44.3 & 81.7 & 71.98 & 12.29 & 28.44  \\
\bottomrule
% (e) & $\checkmark$ & & $\checkmark$ & & & $-$ & $-$ & & & &\\
% (f) & $\checkmark$ & & & $\checkmark$ & & $-$ & $-$ & & & &\\
% (g) & & $\checkmark$ & $\checkmark$ & & & $-$ & $-$ & & & &\\
% \hline (h) & $\checkmark$ & & $\checkmark$ & $\checkmark$ & & $-$ & $-$ & & & &\\
% (i) & $\checkmark$ & $\checkmark$ & & $\checkmark$ & & $-$ & $-$ & & & &\\
% (j) & $\checkmark$ & $\checkmark$ & $\checkmark$ & & & $-$ & $-$ & & & &\\
% (k) & & $\checkmark$ & $\checkmark$ & $\checkmark$ & & $-$ & $-$ & & & &\\
% (k) & & $\checkmark$ & $\checkmark$ & $\checkmark$ & & $-$ & $-$ & & & &\\
% \hline (l) & $\checkmark$ & $\checkmark$ & $\checkmark$ & $\checkmark$ & & $-$ & $-$ & & & &\\
% \hline
\end{tabular}
    \end{adjustbox}
\end{table}

\section{Autotuning Energy at Large Scales}

In this section we apply the proposed energy autotuning framework in Figure~\ref{fig:en} to autotune the energy and EDP of four ECP proxy applications---XSBench, AMG, SWFFT, and SW4lite---on up to 4,096 nodes on Theta. Because energy consumption captures the tradeoff between the application runtime and power consumption and EDP captures the tradeoff between the application runtime and energy consumption, we use the autotuning framework to explore these tradeoffs for energy efficient application execution.

For measuring the baseline energy for each application with a given problem size, we set the number of threads to 64 on Theta and use GEOPM to run the application under the default system configuration five times. Then we use the smallest energy as the baseline for the application. 

For autotuning energy or EDP, after the evaluation of a configuration GEOPM generates the summary report gm.report, which records the package energy and DRAM energy for each node; we accumulate these as the node energy. When ytopt receives the report from GEOPM, it 
calculates an average node energy and uses that average energy as the primary metric for autotuning. Similarly, the average EDP is calculated.

Figure~\ref{fig:w3} shows using our energy framework to autotune the energy of the four ECP proxy applications at large scales on Theta. Figure~\ref{fig:x4} presents autotuning the energy of XSBench on 4,096 nodes, where the red line stands for the baseline node energy of 2494.905J. Using the framework achieves the lowest energy of 2280.806J. This is an 8.58\% energy savings. Figure~\ref{fig:s3} shows autotuning the energy of SWFFT on 4,096 nodes. The baseline node energy for SWFFT is 3185.027J.  Using the framework achieves the lowest node energy of 3118.604J. This is a 2.09\% energy savings. 

\begin{figure}[ht]
    \centering
    \begin{subfigure}[t]{0.24\textwidth}
        \centering
        \includegraphics[width=\textwidth]{figs/xsbench4096-g.png}
        \subcaption{XSBench on 4096 nodes}
        \label{fig:x4}
    \end{subfigure}
    \begin{subfigure}[t]{0.24\textwidth}
        \centering
        \includegraphics[width=\textwidth]{figs/swfft4096-g.png}
        \subcaption{SWFFT on 4096 nodes}
        \label{fig:s3}
    \end{subfigure}
    \begin{subfigure}[t]{0.24\textwidth}
        \centering
        \includegraphics[width=\textwidth]{figs/amg-energy.png}
        \subcaption{AMG on 4096 nodes}
        \label{fig:w3a}
    \end{subfigure}
    \begin{subfigure}[t]{0.24\textwidth}
        \centering
        \includegraphics[width=\textwidth]{figs/sw4lite1024-g.png}
        \subcaption{SW4lite on 1024 nodes}
        \label{fig:w3b}
    \end{subfigure}
    \setlength{\belowcaptionskip}{-8pt}
    \caption{Autotuning Energy at Large Scales on Theta}
    \label{fig:w3}
\end{figure}

Figure~\ref{fig:w3a} presents autotuning the energy of AMG on 4,096 nodes. The baseline node energy for AMG is 5642.568J. Using the framework achieves the lowest node energy of 4566.747J. This is a 20.88\% energy saving. Figure~\ref{fig:w3b} presents autotuning the energy of SW4lite on 1,024 nodes. The baseline node energy for SW4lite is 8384.034J. Using the framework achieves the lowest node energy of 6606.233J. This is a 21.20\% energy saving. Compared with Figure~\ref{fig:w1a} for SW4lite, we identified the best configuration (32, 'sockets' ,'spread' ,'static', ' ', ' ' ,'\#pragma omp for nowait' ,' ' ) which resulted in 91.59\% performance improvement. The same configuration also resulted in the 21.20\% energy saving because the large performance improvement led to the energy saving. As we discussed before, the application runtime for SW4lite on 1024 nodes was dominated by the low power communication for the baseline, this was why the energy saving percentage is much less than the performance improvement percentage. Based on our observation, this is the case for other applications. 

Figure~\ref{fig:w4} shows using our energy framework to autotune the EDP of the four ECP proxy applications at large scales on Theta. Figure~\ref{fig:x5} presents autotuning the energy of XSBench on 4,096 nodes, where the red line stands for the baseline node EDP. Using the framework achieves the lowest EDP with 37.84\% improvement. Figure~\ref{fig:s4} shows autotuning the energy of SWFFT on 4,096 nodes. Using the framework achieves the lowest EDP with 5.24\% improvement.
Figure~\ref{fig:w4a} presents autotuning the energy of AMG on 4,096 nodes. Using the framework achieves the lowest EDP with 24.13\% improvement. Figure~\ref{fig:w4b} presents autotuning the energy of SW4lite on 1,024 nodes. Using the framework achieves the lowest EDP with 23.70\% improvement.
Because EDP is the product of energy and application runtime, the EDP improvement is better than the energy improvement shown in Table \ref{tab:es}. The best configuration for using EDP as the metric is similar to that for using energy as metric.

\begin{figure}[ht]
    \centering
    \begin{subfigure}[t]{0.24\textwidth}
        \centering
        \includegraphics[width=\textwidth]{figs/xsbench-edp.png}
        \subcaption{XSBench on 4096 nodes}
        \label{fig:x5}
    \end{subfigure}
    \begin{subfigure}[t]{0.24\textwidth}
        \centering
        \includegraphics[width=\textwidth]{figs/swfft-edp.png}
        \subcaption{SWFFT on 4096 nodes}
        \label{fig:s4}
    \end{subfigure}
    \begin{subfigure}[t]{0.24\textwidth}
        \centering
        \includegraphics[width=\textwidth]{figs/amg-edp.png}
        \subcaption{AMG on 4096 nodes}
        \label{fig:w4a}
    \end{subfigure}
    \begin{subfigure}[t]{0.24\textwidth}
        \centering
        \includegraphics[width=\textwidth]{figs/sw4lite-edp.png}
        \subcaption{SW4lite on 1024 nodes}
        \label{fig:w4b}
    \end{subfigure}
    \setlength{\belowcaptionskip}{-8pt}
    \caption{Autotuning Energy Delay Product at Large Scales on Theta}
    \label{fig:w4}
\end{figure}

Overall, using our energy autotuning framework to identify the best configurations for the four ECP proxy applications results in up to 21.2\% energy savings and up to 37.84\% improvement in EDP on up to 4,096 nodes shown in Table \ref{tab:es}. This aids us in exploring the tradeoffs between application runtime and power/energy for energy efficient application execution.

\begin{table}[ht]
\center
\caption{Improvement percentage (\%) for each application on Theta}
\begin{tabular}{|r|c|c|c|c|}
\hline
Theta & XSBench & SWFFT & AMG & SW4lite  \\
\hline
Energy &  8.58 & 2.09  & 20.88   &   21.20\\
\hline
EDP &  37.84 & 5.24  & 24.13   &   23.70\\
\hline
\end{tabular}
\label{tab:es}
\end{table}


\section{Conclusions}
We consider the phase-extraction problem, and we showed that, given a unitary $U = e^{i\pi H}$ and its inverse $U^{\dag}$, we could implement a block-encoding of $\phi(H)$ for some smooth function $\phi(x)$. The word `smooth' here means existence and continuity of the derivatives: the higher the number of continuous derivatives that a function has, the faster its Fourier sum (and thus the Laurent polynomial on the eigenphases) uniformly converges to that function. We are confident this can have many more applications beyond what is shown in this work. It is also worth remarking that Jackson showed that the convergence rate of a Fourier series is almost-optimal, in the sense that no trigonometric (or, equivalently, complex exponential) series can approximate the desired function faster, up to that $\log d$ factor~\cite[p.\ 21]{jacksonTheoryApproximation1930a}. Also remember that `smoothing' a function, i.e., replacing its derivative with a continuous function, does not give faster convergence for free in general, as its derivative will become steep in the points where we smooth out discontinuities, and this translates to a high Lipschitz constant: a~clear example is given by Eq.~\ref{eq:lipschitz-constant-recurrence-solution}, but in that case, fortunately, nothing depends on the size of the input $N$, and thus does not influence the asymptotic query complexity of Algorithm~\ref{alg:prop-sampling-qsp}, although the constant factor can become large even for $p = 20$. From a theoretical point of view, this work shows that, for any $\eta > 0$, there is an algorithm with query complexity 
$$\Tilde{\bigO}\left(\frac{1}{\bar{c}^{\frac{1}{2} + \eta}} \frac{1}{\epsilon^\eta} \right)$$
solving the proportional-sampling problem. This statement seems to suggest there exists an algorithm which directly solves the problem with $\eta = 0$, and an open question would be to find such algorithm.


It is also interesting to remark that Theorems~\ref{thm:haah-construction},~\ref{thm:haah-completion} indeed allow the construction for any $\phi$, even complex-valued, provided that its absolute value is reciprocal.

One could think that, in Section~\ref{sec:prop-sampling}, instead of using the linear function in the phase-extraction subroutine, we could approximate the square root and then apply the transformation directly on $e^{i \pi c(x)}$. However, in the case of proportional sampling this would be inconvenient, as the derivative of the square root function has a discontinuity with an infinite jump around 0, and we could not choose a constant $\delta$ if we had values of the oracle that are too close to $0$.

%\if 0
\section*{Acknowledgments}
This work was supported in part by DOE ECP PROTEAS-TUNE, in part by DOE ASCR RAPIDS2, and in part by NSF grant CCF-2119203.  We acknowledge the Argonne Leadership Computing Facility (ALCF) for use of Cray XC40 Theta under ALCF projects EE-ECP and Intel, and the Oak Ridge Leadership Computing Facility for use of Summit under the projects CSC383, MED106 and AST136. We also acknowledge Adrian Pope at ALCF for providing the SWFFT problem sizes. This material is based upon work supported by the U.S. Department of Energy, Office of Science, under contract number DE-AC02-06CH11357. Development of the GEOPM software package has been partially funded through contract B609815 with Argonne National Laboratory.
%\fi

%\IEEEtriggeratref{14}
%\newpage
\bibliographystyle{IEEEtran}
\bibliography{bibs}

%GAIL - don't  forget  the Govt license
\newpage
%\section{Appendix for Proofs}

\paragraph{Proof of Theorem \ref{thm:main}.}

\begin{proof}
\label{proof:main}
Our proof has two steps. In Step 1, we will show that SimCLR is equivalent to minimizing the cross entropy loss defined in Eqn.~(\ref{eqn:cross-entropy}). 
In Step 2, we will show  that minimizing the cross-entropy loss 
is equivalent to spectral clustering on $\bfpi$. 
Combining the two steps together, we have proved our theorem. 

\textbf{Step 1: } SimCLR is equivalent to minimizing the cross entropy loss.

The cross-entropy loss takes expectation over 
$\bfW_\bfX\sim \mathbb{P}(\cdot ; \bfpi)$, 
which means $\bfW_\bfX$ has exactly one non-zero entry in each row $i$. By Lemma~\ref{lem:multinomial}, we know every row $i$ of $\bfW_\bfX$ is independent of other rows. Moreover, 
$\bfW_{\bfX,i}\sim \mathcal{M}(1, \bfpi_i/\sum_j \bfpi_{i,j})=\mathcal{M}(1, \bfpi_i)$, because $\bfpi_i$ itself is a probability distribution.
Similarly, we know $\bfW_\bfZ$ also has the row-independent property by sampling over $\mathbb{P}(\cdot;\bfK_\bfZ)$.
Therefore, by Lemma~\ref{lem:cross_split}, we know Eqn.~(\ref{eqn:cross-entropy}) is equivalent to:
\[
 -\sum_{i=1}^n \mathbb{E}_{\bfW_{\bfX,i}}[\log \mathbb{P}(\bfW_{\bfZ,i}=\bfW_{\bfX,i};\bfK_\bfZ)],
\]

This expression takes expectation over $\bfW_{\bfX,i}$ for the given row $i$. Notice that 
$\bfW_{\bfX,i}$ has exactly one non-zero entry, which equals $1$ (same for $\bfW_{\bfZ,i}$). 
As a result
we expand the above expression to be:
\begin{equation}
 -\sum_{i=1}^n \sum_{j\neq i} \Pr(\bfW_{\bfX,i,j}=1)\log \Pr(\bfW_{\bfZ,i,j}=1).
\label{eqn:detailed-expansion}    
\end{equation}


By Lemma~\ref{lem:multinomial}, $\Pr(\bfW_{\bfZ,i,j}=1)=\bfK_{\bfZ,i,j}/\|\bfK_{\bfZ,i}\|_1$ for $j\neq i$. Recall that $\bfK_\bfZ=(k(\bfZ_i-\bfZ_j))_{(i,j)\in[n]^2}$, which means 
$\bfK_{\bfZ,i,j}/\|\bfK_{\bfZ,i}\|_1=\frac{\exp(-\|\bfZ_i-\bfZ_j\|^2/{2\tau})}{\sum_{k\neq i}
\exp(-\|\bfZ_i-\bfZ_k\|^2/{2\tau})
}$ for $j\neq i$, when $k$ is the Gaussian kernel with variance $\tau$. 

Notice that $\bfZ_i=f(\bfX_i)$, so we know
\begin{equation}
-\log \Pr(\bfW_{\bfZ,i,j}=1)=
-\log \frac{\exp(-\|f(\bfX_i)-f(\bfX_j)\|^2/{2\tau})}{\sum_{k\neq i}
\exp(-\|f(\bfX_i)-f(\bfX_k)\|^2/{2\tau}),
}
\label{eqn:infonce-equivalence}    
\end{equation}


The right hand side is exactly the InfoNCE loss defined in Eqn.~(\ref{eqn:infonce}).
Inserting Eqn.~(\ref{eqn:infonce-equivalence}) into Eqn.~(\ref{eqn:detailed-expansion}), we get the SimCLR algorithm, which first samples augmentation pairs $(i,j)$ with $\Pr(\bfW_{\bfX,i,j}=1)$ for each row $i$, and then optimize the InfoNCE loss. 

\textbf{Step 2: } minimizing the cross entropy loss 
is equivalent to spectral clustering on $\bfpi$.


By Lemma~\ref{lem:convert_to_spectral}, we may further convert the loss to 
\begin{equation}
\label{eqn:main-theorem-repul-attr}
\min_{\bfZ}
-\sum_{(i,j)\in [n]^2} \mathbf{P}_{i,j}
\log k (\bfZ_i-\bfZ_j)+\log \mathbf{R}(\bfZ).
\end{equation}
Since $k$ is the Gaussian kernel, this reduces to \[
\min_\bfZ \mathrm{tr}(\bfZ^\top \mathbf{L}(\bfpi) \bfZ)
+\log \mathbf{R}(\bfZ),
\]

where we use the fact that $\mathbb{E}_{\bfW_\bfX\sim \mathbb{P}(\cdot; \bfpi)}[\mathbf{L}(\bfW_\bfX)]
=\mathbf{L}(\bfpi)
$, because the Laplacian operator is linear and $
\mathbb{E}_{\bfW_\bfX\sim \mathbb{P}(\cdot; \bfpi)}(\bfW_\bfX)=\bfpi
$.
\end{proof}

\paragraph{Proof of Theorem \ref{thm:clip}.}
\begin{proof}
Since $\bfW_\bfX\sim \mathbb{P}(\cdot;\bfpi_{\mathbf{A}, \mathbf{B}})$, we know 
$\bfW_\bfX$ has exactly one non-zero entry in each row, denoting the pair that got sampled. 
A notable difference compared to the previous proof is we now have $n_\mathcal{A}+n_\mathcal{B}$ objects in our graph. CLIP deals with this by taking a mini-batch of size $2N$, 
such that $n_\mathcal{A}=n_\mathcal{B}=N$, and adding the $2N$ InfoNCE losses together. We label the objects in $\mathcal{A}$ as $[n_\mathcal{A}]$, and the objects in $\mathcal{B}$ as $\{n_\mathcal{A}+1, \cdots, n_\mathcal{A}+n_\mathcal{B}\}$. 

Notice that $\bfpi_{\mathbf{A}, \mathbf{B}}$ is a bipartite graph, so the edges of objects in $\mathcal{A}$ will only connect to object in $\mathcal{B}$ and vice versa. We can define the similarity matrix in $\cZ$ as $\bfK_\bfZ$, 
where $\bfK_\bfZ(i, j+n_\mathcal{A})=\bfK_\bfZ(j+n_\mathcal{A},i)= k(\bfZ_i-\bfZ_j)$ for $i\in [n_\mathcal{A}], j\in [n_\mathcal{B}]$, and otherwise we set $\bfK_\bfZ(i,j)=0$. 
The rest is same as the previous proof. 
\end{proof}

\paragraph{Proof of Theorem \ref{thm:exponential}.}

\begin{proof}
\label{proof:exponential}
Since the objective function consists of a linear term combined with an entropy regularization, which is a strongly concave function, the maximization problem is a convex optimization problem. Owing to the implicit constraints provided by the entropy function, the problem is equivalent to having only the equality constraint. We then introduce the Lagrangian multiplier $\lambda$ and obtain the following relaxed problem:

$$
\widetilde{E}(\boldsymbol{\alpha})=\psi_{1}-\sum_{i=1}^n \alpha_{i} \psi_{i}+\tau \sum_{i=1}^n \alpha_{i}\log \alpha_{i}+\lambda\left(\boldsymbol{\alpha}^{\top} \mathbf{1}_n-1\right).
$$

As the relaxed problem is unconstrained, taking the derivative with respect to $\alpha_{i}$ yields

$$
\frac{\partial \widetilde{E}(\boldsymbol{\alpha})}{\partial \alpha_{i}}=-\psi_{i}+\tau\left(\log \alpha_{i}+\alpha_{i} \frac{1}{\alpha_{i}}\right)+\lambda=0.
$$

Solving the above equation implies that $\alpha_{i}$ takes the form
$
\alpha_{i}=\exp \left(\frac{1}{\tau} \psi_{i}\right) \exp \left(\frac{-\lambda}{\tau}-1\right).
$ Since $\alpha_{i}$ lies on the probability simplex, the optimal $\alpha_{i}$ is explicitly given by
$
\alpha^{*}_{i}=\frac{\exp \left(\frac{1}{\tau} \psi_{i}\right)}{\sum_{i^{\prime}=1}^n \exp \left(\frac{1}{\tau} \psi_{i^{\prime}}\right)} .
$ Substituting the optimal point into the objective function, we obtain
$$
\begin{aligned}
E\left(\boldsymbol{\alpha}^*\right)  &=\psi_1-\sum_{i=1}^n \frac{\exp \left(\frac{1}{\tau} \psi_{i}\right)}{\sum_{i^{\prime}=1}^n \exp \left(\frac{1}{\tau} \psi_{i^{\prime}}\right)} \psi_{i}+\tau \sum_{i=1}^n \frac{\exp \left(\frac{1}{\tau} \psi_{i}\right)}{\sum_{i^{\prime}=1}^n \exp \left(\frac{1}{\tau} \psi_{i^{\prime}}\right)}\log \frac{\exp \left(\frac{1}{\tau} \psi_{i}\right)}{\sum_{i^{\prime}=1}^n \exp \left(\frac{1}{\tau} \psi_{i^{\prime}}\right)} \\
& =\psi_1 - \tau \log \left(\sum_{i=1}^n \exp \left(\frac{1}{\tau} \psi_{i}\right)\right).
\end{aligned}
$$
Thus, the Lagrangian dual function is given by
\begin{equation*}
-E\left(\boldsymbol{\alpha}^*\right)= -\tau \log \frac{\exp \left(\frac{1}{\tau} \psi_{1}\right)}{\sum_{i=1}^n \exp \left(\frac{1}{\tau} \psi_{i}\right)}.\qedhere
\end{equation*}
\end{proof}



\section{More on Experiments} \label{section: experiment_details}

\paragraph{CIFAR-10 and CIFAR-100} CIFAR-10 ~\citep{krizhevsky2009learning} and CIFAR-100 ~\citep{krizhevsky2009learning} are well-known classic image classification datasets. Both CIFAR-10 and CIFAR-100 contain a total of 60k $32 \times 32$ labeled images of different classes, with 50k for training and 10k for testing. CIFAR-10 is similar to CIFAR-100, except there are 10 different classes in CIFAR-10 and 100 classes in CIFAR-100.

\paragraph{TinyImageNet} TinyImageNet ~\citep{le2015tiny} is a subset of ImageNet ~\citep{deng2009imagenet}. There are 200 different object classes in TinyImageNet, with 500 training images, 50 validation images, and 50 test images for each class. All the images in TinyImageNet are colored and labeled with a size of $64 \times 64$.

\textbf{Pseudo-code.} Algorithm \ref{alg:Training Procedure} presents the pseudo-code for our empirical training procedure.

\begin{algorithm}[!htbp]
\caption{Training Procedure}
\label{alg:Training Procedure}
\begin{algorithmic}[1]
\REQUIRE trainable encoder network $f$, batch size $N$, augmentation strategy \textit{aug}, loss function $L$ with hyperparameters \textit{args}
\FOR {sampled minibatch ${x_i}_{i=1}^N$}
\FORALL{$i \in { 1, ..., N }$}
\STATE draw two augmentations $t_i = \textit{aug}\left(x_i\right) $, $t_i' = \textit{aug}\left(x_i\right) $
\STATE $z_i = f\left(t_i\right)$, $z_i' = f\left(t_i'\right)$
\ENDFOR
\STATE compute loss $\mathcal{L} = L(N, z, z', \textit{args})$
\STATE update encoder network $f$ to minimize $\mathcal{L}$
\ENDFOR
\STATE \textbf{Return} encoder network $f$
\end{algorithmic}
\end{algorithm}

We also provide the pseudo-code for our core loss function used in the training procedure in Algorithm \ref{alg:Core loss}. The pseudo-code is almost identical to SimCLR's loss function, with the exception of an extra parameter $\gamma$.

\begin{algorithm}[!htbp]
\caption{Core loss function $\mathcal{C}$}
\label{alg:Core loss}
\begin{algorithmic}[1]
\REQUIRE batch size $N$, two encoded minibatches $z_1, z_2$, $\gamma$, temperature $\tau$
\STATE $z = \textit{concat}\left(z_1, z_2\right)$
\FOR {$i \in {1, ..., 2N }, j \in {1, ..., 2N}$ }
\STATE $s_{i,j} = \Vert z_i - z_j \Vert_2^{\gamma}$
\ENDFOR
\STATE \textbf{define} $l(i, j)$ \textbf{as} $l(i, j) = - \log \frac{exp\left(s_{i,j}/\tau \right)}{\sum_{k=1}^{2N} \mathbf{1}{[k \ne i]} exp\left(s{i, j} / \tau \right)} $
\STATE \textbf{Return} $\frac{1}{2N} \sum_{k=1}^N\left[l(i, i+N) + l(i+N, i)\right]$
\end{algorithmic}
\end{algorithm}

Utilizing the core loss function $\mathcal{C}$, we can define all kernel loss functions used in our experiments in Table \ref{table: loss definition}. For all $z_i \in z$ with even dimensions $n$, we define $z_{L_i} = z_i\left[0:n/2\right]$ and $z_{R_i} = z_i\left[n/2:n\right]$.

\begin{table}[ht]
\centering
\begin{tabular}{{@{}l|l@{}}}
Kernel  &  Loss function \\ \midrule
Laplacian & $\mathcal{C}\left(N, z, z', \gamma=1, \tau\right)$\\ \midrule
Sum       & $\lambda * \mathcal{C}\left(N, z, z', \gamma=1, \tau_1\right) + (1-\lambda) * \mathcal{C}\left(N, z, z', \gamma=2, \tau_2\right)$  \\ \midrule
Concatenation Sum&$\lambda * \mathcal{C}\left(N, z_L, z'_L, \gamma=1, \tau_1\right) + (1-\lambda) * \mathcal{C}\left(N, z_R, z'_R, \gamma=2, \tau_2\right)$\\ \midrule
$\gamma = 0.5$ & $\mathcal{C}\left(N, z, z', \gamma=0.5, \tau\right)$          \\ 

\end{tabular}

\caption{Definition of kernel loss functions in our experiments}
\label {table: loss definition}
\end{table}

\textbf{Baselines.} We reproduce the SimCLR algorithm using PyTorch Lightning~\citep{PytorchLightning}.

\textbf{Encoder details.}
The encoder $f$ consists of a backbone network and a projection network. We employ ResNet50~\citep{ResNet} as the backbone and a 2-layer MLP (connected by a batch normalization~\citep{ioffe2015batch} layer and a ReLU \cite{nair2010rectified} layer) with hidden dimensions 2048 and output dimensions 128 (or 256 in the concatenation kernel case).

\textbf{Encoder hyperparameter tuning.}
For each encoder training case, we randomly sample 500 hyperparameter groups (sample details are shown in Table \ref{table: Hyperparameter sample}) and train these samples simultaneously using Ray Tune ~\citep{RayTune}, with the ASHA scheduler~\citep{li2018massively}. Ultimately, the hyperparameter group that maximizes the online validation accuracy (integrated in PyTorch Lightning) within 5000 validation steps is chosen for the given encoder training case.

\begin{table}[ht]
\centering

\begin{tabular}{@{}l|l|l@{}}
\midrule
Hyperparameter  & Sample Range & Sample Strategy \\ \midrule
start learning rate & $\left[10^{-2}, 10\right]$ & log uniform \\ \midrule
$\lambda$       & $\left[0, 1\right]$ & uniform \\ \midrule
$\tau$, $\tau_1$, $\tau_2$ & $\left[0, 1\right]$ & log uniform \\ \midrule
\end{tabular}

\caption{Hyperparameters sample strategy}
\label {table: Hyperparameter sample}
\end{table}

\textbf{Encoder training.} 
We train each encoder using the LARS optimizer~\citep{LARSOptimizer}, LambdaLR Scheduler in PyTorch, momentum 0.9, weight decay $10^{-6}$, batch size 256, and the aforementioned hyperparameters for 400 epochs on a single A-100 GPU.

\textbf{Image transformation.} The image transformation strategy, including augmentation, is identical to the default transformation strategy provided by PyTorch Lightning.

\textbf{Linear evaluation.}
The linear head is trained using the SGD optimizer with a cosine learning rate scheduler, batch size 64, and weight decay $10^{-6}$ for 100 epochs. The learning rate starts at $0.3$ and ends at $0$.

\textbf{Moco Experiments.} We also tested our method based on MoCo~\citep{he2019moco}. The results are summarized in Table \ref{tab:results-moco}. Here we choose ResNet18~\citep{ResNet} as the backbone and set a temperature of $0.1$ as default. For our simple sum kernel, we set $\lambda=0.8$. The results show that our method outperforms the original MoCo method.

\begin{table}[thb]
\centering
\caption{MoCo Experiment Results on CIFAR-10 and CIFAR-100.}
\label{tab:results-moco}
\resizebox{\textwidth}{!}{%
\begin{tabular}{@{}c|ccc|ccc@{}}
\toprule
\multirow{3}{*}{Method} & \multicolumn{3}{c|}{CIFAR-10} & \multicolumn{3}{c}{CIFAR-100} \\ \cmidrule(lr){2-4} \cmidrule(lr){5-7} 
                        & 200 epochs & 400 epochs    & 1000 epochs   & 200 epochs & 400 epochs & 1000 epochs         \\ \midrule
MoCo (repro.)         & $76.41 \pm 0.12$    & $80.01 \pm 0.15$          & $84.45 \pm 0.08$    & $\mathbf{47.02 \pm 0.11}$ & $52.50 \pm 0.07$ & $57.62 \pm 0.15$            \\
\midrule
Laplacian Kernel        & ${78.09 \pm 0.10}$    & $\mathbf{83.85 \pm 0.09}$          & $\mathbf{88.34 \pm 0.16}$    & $46.12 \pm 0.22$   & $53.44 \pm 0.17$ & $59.10 \pm 0.14$        \\
Simple Sum Kernel & $\mathbf{78.12 \pm 0.15}$   & $83.23 \pm 0.18$ & $87.50 \pm 0.20$ & $46.65 \pm 0.06$ & $\mathbf{53.62 \pm 0.19}$ & $\mathbf{59.83 \pm 0.12}$\\
\bottomrule
\end{tabular}
}
\end{table}



\section{More Experiments on Synthetic Data}


Consider a scenario with $n$ clusters, each containing $k$ vertices. Let the probability of vertices $u$ and $v$ from the same cluster belonging to $\bfpi$ be $p$. Conversely, for vertices $u$ and $v$ from different clusters, let the probability of belonging to $\pi$ be $q$. We generate the graph $\bfpi$ randomly, based on $p$ and $q$. We experiment with values of $k=100$ and $n=6$ for ease of visualization, embedding all points in a two-dimensional space. Each vertex's initial position originates from a normal distribution. In each iteration, we sample a subgraph of $\bfpi$ uniformly, ensuring each vertex has an out-degree of $1$. We then optimize the corresponding vectors using InfoNCE loss with an SGD optimizer and iterate until convergence. Our experimental setup consists of an SGD learning rate of $1$, an InfoNCE loss temperature of $0.5$, and a batch size of $50$. We evaluate two scenarios with different $p$ and $q$ values: $p=1$, $q=0$, and $p=0.75$, $q=0.2$. The results of these experiments are visualized in Figure \ref{fig:vis-spectral-cluster}. The obtained embeddings exhibit the hallmark pattern of spectral clustering of graph $\bfpi$.

\begin{figure}[!tb]
\centering
\subfigure{
\includegraphics[width=1\textwidth]{Figures/cluster_pi.png}
\label{fig:vis-cluster}
}
\subfigure{
\includegraphics[width=1\textwidth]{Figures/noised_cluster_pi.png}
\label{fig:vis-noised-cluster}
}
\caption{Visualizations of the optimization process using InfoNCE Loss on the vectors corresponding to $\bfpi$. Points of identical color belong to the same cluster within $\bfpi$. To showcase the internal structure of $\bfpi$, we randomly select 10 vertices from each cluster to display the edge distribution of $\bfpi$.}
\label{fig:vis-spectral-cluster}
\end{figure}



\end{document}
