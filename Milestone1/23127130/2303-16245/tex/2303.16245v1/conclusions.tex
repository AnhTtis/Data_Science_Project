
\section{Conclusions}

%Efficiently utilizing power and optimizing the performance of scientific applications under power and energy constraints are challenging tasks in scientific applications. To address these issues, 

In this paper, we proposed the low-overhead autotuning frameworks to autotune four hybrid MPI/OpenMP ECP proxy applications---XSBench, AMG, SWFFT, and SW4lite---at large scales and explored the tradeoffs between application runtime and power/energy for energy efficient application execution. We used Bayesian optimization with a Random Forest surrogate model to effectively search the parameter spaces with up to 6 million different configurations on Theta and Summit. We used the autotuning framework to explore the tradeoffs between application runtime and power/energy for energy efficient application execution. The experimental results showed that our autotuning framework had low overhead and good scalability. By using the autotuning framework to identify the best configuration, we achieved up to 91.59\% performance improvement, up to 21.2\% energy savings, and up to 37.84\% EDP improvement on up to 4,096 nodes. The ytopt autotuning framework is open source and available to download from the link in \cite{YTO}. 

For future work, we will improve the framework overhead by reducing the application compiling time with pre-compiling the unchanged code files and setting a proper evaluation timeout to evaluate more good configurations.
Our current autotuning framework uses Ray \cite{RAY} to do one evaluation each time; this affected the effectiveness of identifying the promising search regions at the beginning of the autotuning. We plan to extend the framework to do multiple evaluations in parallel using libensemble \cite{libem} to improve the initial effectiveness. 
We also plan to add transfer learning \xingfu{and online tuning} to the framework so that it can transfer what it learns from the applications at a small scale in problem sizes and system sizes to guide and/or predict \xingfu{the best configurations for} autotuning at large scales.
