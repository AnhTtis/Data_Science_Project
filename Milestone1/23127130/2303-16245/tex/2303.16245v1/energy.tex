
\section{Autotuning Energy at Large Scales}

In this section we apply the proposed energy autotuning framework in Figure~\ref{fig:en} to autotune the energy and EDP of four ECP proxy applications---XSBench, AMG, SWFFT, and SW4lite---on up to 4,096 nodes on Theta. Because energy consumption captures the tradeoff between the application runtime and power consumption and EDP captures the tradeoff between the application runtime and energy consumption, we use the autotuning framework to explore these tradeoffs for energy efficient application execution.

For measuring the baseline energy for each application with a given problem size, we set the number of threads to 64 on Theta and use GEOPM to run the application under the default system configuration five times. Then we use the smallest energy as the baseline for the application. 

For autotuning energy or EDP, after the evaluation of a configuration GEOPM generates the summary report gm.report, which records the package energy and DRAM energy for each node; we accumulate these as the node energy. When ytopt receives the report from GEOPM, it 
calculates an average node energy and uses that average energy as the primary metric for autotuning. Similarly, the average EDP is calculated.

Figure~\ref{fig:w3} shows using our energy framework to autotune the energy of the four ECP proxy applications at large scales on Theta. Figure~\ref{fig:x4} presents autotuning the energy of XSBench on 4,096 nodes, where the red line stands for the baseline node energy of 2494.905J. Using the framework achieves the lowest energy of 2280.806J. This is an 8.58\% energy savings. Figure~\ref{fig:s3} shows autotuning the energy of SWFFT on 4,096 nodes. The baseline node energy for SWFFT is 3185.027J.  Using the framework achieves the lowest node energy of 3118.604J. This is a 2.09\% energy savings. 

\begin{figure}[ht]
    \centering
    \begin{subfigure}[t]{0.24\textwidth}
        \centering
        \includegraphics[width=\textwidth]{figs/xsbench4096-g.png}
        \subcaption{XSBench on 4096 nodes}
        \label{fig:x4}
    \end{subfigure}
    \begin{subfigure}[t]{0.24\textwidth}
        \centering
        \includegraphics[width=\textwidth]{figs/swfft4096-g.png}
        \subcaption{SWFFT on 4096 nodes}
        \label{fig:s3}
    \end{subfigure}
    \begin{subfigure}[t]{0.24\textwidth}
        \centering
        \includegraphics[width=\textwidth]{figs/amg-energy.png}
        \subcaption{AMG on 4096 nodes}
        \label{fig:w3a}
    \end{subfigure}
    \begin{subfigure}[t]{0.24\textwidth}
        \centering
        \includegraphics[width=\textwidth]{figs/sw4lite1024-g.png}
        \subcaption{SW4lite on 1024 nodes}
        \label{fig:w3b}
    \end{subfigure}
    \setlength{\belowcaptionskip}{-8pt}
    \caption{Autotuning Energy at Large Scales on Theta}
    \label{fig:w3}
\end{figure}

Figure~\ref{fig:w3a} presents autotuning the energy of AMG on 4,096 nodes. The baseline node energy for AMG is 5642.568J. Using the framework achieves the lowest node energy of 4566.747J. This is a 20.88\% energy saving. Figure~\ref{fig:w3b} presents autotuning the energy of SW4lite on 1,024 nodes. The baseline node energy for SW4lite is 8384.034J. Using the framework achieves the lowest node energy of 6606.233J. This is a 21.20\% energy saving. Compared with Figure~\ref{fig:w1a} for SW4lite, we identified the best configuration (32, 'sockets' ,'spread' ,'static', ' ', ' ' ,'\#pragma omp for nowait' ,' ' ) which resulted in 91.59\% performance improvement. The same configuration also resulted in the 21.20\% energy saving because the large performance improvement led to the energy saving. As we discussed before, the application runtime for SW4lite on 1024 nodes was dominated by the low power communication for the baseline, this was why the energy saving percentage is much less than the performance improvement percentage. Based on our observation, this is the case for other applications. 

Figure~\ref{fig:w4} shows using our energy framework to autotune the EDP of the four ECP proxy applications at large scales on Theta. Figure~\ref{fig:x5} presents autotuning the energy of XSBench on 4,096 nodes, where the red line stands for the baseline node EDP. Using the framework achieves the lowest EDP with 37.84\% improvement. Figure~\ref{fig:s4} shows autotuning the energy of SWFFT on 4,096 nodes. Using the framework achieves the lowest EDP with 5.24\% improvement.
Figure~\ref{fig:w4a} presents autotuning the energy of AMG on 4,096 nodes. Using the framework achieves the lowest EDP with 24.13\% improvement. Figure~\ref{fig:w4b} presents autotuning the energy of SW4lite on 1,024 nodes. Using the framework achieves the lowest EDP with 23.70\% improvement.
Because EDP is the product of energy and application runtime, the EDP improvement is better than the energy improvement shown in Table \ref{tab:es}. The best configuration for using EDP as the metric is similar to that for using energy as metric.

\begin{figure}[ht]
    \centering
    \begin{subfigure}[t]{0.24\textwidth}
        \centering
        \includegraphics[width=\textwidth]{figs/xsbench-edp.png}
        \subcaption{XSBench on 4096 nodes}
        \label{fig:x5}
    \end{subfigure}
    \begin{subfigure}[t]{0.24\textwidth}
        \centering
        \includegraphics[width=\textwidth]{figs/swfft-edp.png}
        \subcaption{SWFFT on 4096 nodes}
        \label{fig:s4}
    \end{subfigure}
    \begin{subfigure}[t]{0.24\textwidth}
        \centering
        \includegraphics[width=\textwidth]{figs/amg-edp.png}
        \subcaption{AMG on 4096 nodes}
        \label{fig:w4a}
    \end{subfigure}
    \begin{subfigure}[t]{0.24\textwidth}
        \centering
        \includegraphics[width=\textwidth]{figs/sw4lite-edp.png}
        \subcaption{SW4lite on 1024 nodes}
        \label{fig:w4b}
    \end{subfigure}
    \setlength{\belowcaptionskip}{-8pt}
    \caption{Autotuning Energy Delay Product at Large Scales on Theta}
    \label{fig:w4}
\end{figure}

Overall, using our energy autotuning framework to identify the best configurations for the four ECP proxy applications results in up to 21.2\% energy savings and up to 37.84\% improvement in EDP on up to 4,096 nodes shown in Table \ref{tab:es}. This aids us in exploring the tradeoffs between application runtime and power/energy for energy efficient application execution.

\begin{table}[ht]
\center
\caption{Improvement percentage (\%) for each application on Theta}
\begin{tabular}{|r|c|c|c|c|}
\hline
Theta & XSBench & SWFFT & AMG & SW4lite  \\
\hline
Energy &  8.58 & 2.09  & 20.88   &   21.20\\
\hline
EDP &  37.84 & 5.24  & 24.13   &   23.70\\
\hline
\end{tabular}
\label{tab:es}
\end{table}

