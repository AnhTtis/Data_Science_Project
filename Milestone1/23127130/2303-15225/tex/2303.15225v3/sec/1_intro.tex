\section{Introduction}
\label{intro}
%
Recent years have seen a growing need for the conversion of real-world objects to computerized models \cite{xiao2021estimating, fernandes2021point} across several domains, such as digital preservation of cultural heritage \cite{PIERACCINI200163} and manufacturing of mechanical parts for industry \cite{LI200253}. This need has given rise to a range of modern data acquisition techniques such as laser scanning, which densely samples the surface of a 3D object, thereby generating millions of significantly redundant data points. 3D models can be obtained from this \textit{point cloud} by constructing a polygonal mesh using techniques such as the \textit{ball-pivoting algorithm} and \textit{Poisson surface reconstruction} \cite{bernardini1999ball, kazhdan2006poisson, berger2014state}. However, the sheer size of these dense point clouds makes this task computationally expensive in terms of both memory and time. Furthermore, the size of such generated meshes impedes further processing efforts, and necessitates the use of costly mesh simplification strategies \cite{garland1997surface, hoppe1993mesh, cignoni1998comparison} for size reduction. This makes efficient simplification of the underlying point cloud, prior to any surface reconstruction, an important and impactful problem which if addressed, has the potential to significantly improve the scalability of several computer vision applications. 

\begin{figure}
     \centering \vspace{-0.8cm}
        \includegraphics[scale=0.47, trim=0.4cm 13cm 10cm 1cm, clip]{page1.pdf}
         \caption{Point cloud simplification methods typically fail to strike a balance between preserving sharp features and maintaining the overall structure of the original cloud. Our approach circumvents this trade-off by achieving both targets, as is evident from the simplified versions of the Stanford Bunny \cite{levoy2005stanford} 
         obtained using the proposed technique and two pre-existing methods; Hierarchical Clustering (HC) and Weighted Locally
Optimal Projection (WLOP).}
\vspace{-0.5cm}
         \label{fig:figure1}
\end{figure}
%

The inherent dependency of surface reconstruction methods on surface normals, makes the visual perceptual quality of a point cloud an indirect yet important aspect of any mesh processing pipeline \cite{ cignoni1998comparison}. Although it is difficult to quantify this visual degradation in the case of point cloud simplification methods, one can say that the more enhanced the characteristic features of an object (such as sharp edges and high curvature regions) are in the simplified cloud, the higher is its human perceptual quality \cite{lee2005mesh}. Therefore, an optimal point cloud simplification technique should preserve both the global structural appearance, and the salient features of the point cloud in question. Some of these methods will be discussed in detail in the upcoming section.

Given that the point cloud representing an object exists on a Riemannian manifold in 3D space, Euclidean distance fails to measure the intrinsic distance between any two points on its surface. Recently, techniques which extend existing machine learning methods to model functions defined on manifolds have gained popularity.
For instance, \textit{Gaussian
processes} (GPs), a widely used class of non-parametric statistical
models, which often use Euclidean distance-based
covariance functions, have been made compatible for functions
whose domains are compact Riemannian manifolds
using ideas from harmonic analysis \cite{borovitskiy2020matern}.

In this work, we propose a novel, one-shot, feature-preserving simplification method using GPs with kernels defined on Riemannian manifolds. Using a greedy algorithm for GP sparsification, we iteratively construct a simplified representation of a point cloud without the need for any prior surface reconstruction or training on large point cloud datasets. We experiment on several point clouds, compare with several techniques and demonstrate competitive results both empirically and in terms of computational efficiency. Qualitatively, as shown in Fig. \ref{fig:figure1}, our method effectively preserves visual features whilst providing a sufficiently dense coverage of the domain of the original cloud.

\textit{Outline of the paper:} Section \ref{sec:rw} briefly reviews a number of existing point cloud simplification techniques which are relevant to our work. Section \ref{sec:background} provides background details regarding the computation of surface variation, GPs with kernels defined on non-Euclidean domains and a greedy subset-of-data scheme for GP inference. Section \ref{sec:methodology} outlines the proposed GP-based point cloud simplification algorithm. Section \ref{sec:experiments}, in combination with the supplementary material, includes an empirical evaluation of our method on various benchmark and self-acquired point clouds, with comparisons to competing simplification techniques, along with applications to some downstream tasks. Finally, Section \ref{sec:conc} summarises our contributions and provides a brief discussion of the scope for future work.
