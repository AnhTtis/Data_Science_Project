\clearpage
\setcounter{page}{1}
\maketitlesupplementary

\section{Additional experimental details}
Our approach practically has no user-specified parameters, as it generates a simplified cloud based on the desired simplification ratio ($\alpha$). However, we supply to our algorithm (\ref{alg:simpli}) some fixed default values which work well for most point clouds. We set $k_{\text{opt}} = 200$ for all experiments as even with a small subset of the original point cloud, the GP typically converges on an optimal set of hyperparameters within 100 iterations. $k_{\text{init}}$ was chosen to be $1/3$ of the target number of points in the simplified cloud, a ratio which empirically works well across all point clouds tested. Finally, $k_{\text{add}}$ is determined adaptively based on $k_{\text{init}}$, $N$ and $M$. Our algorithm is implemented in the PyTorch framework \cite{paszke2019pytorch}, and whilst the runtimes reported in Table \ref{tab:timings} were achieved with GPU acceleration using an NVIDIA A100 with 80GB of RAM, our algorithm can also be run purely on a CPU. All baselines were run on an Intel i7-11800H CPU with 32GB RAM.
\section{Additional visualisations}
On the following pages we present additional visualisations to accompany the results presented in the main text. Figure \ref{fig:lucy_all} shows the surface reconstruction results on \textit{Lucy} for all evaluated methods. Figure \ref{fig:armadillo_all} shows the simplified clouds and reconstructed meshes for all techniques on the \textit{Armadillo} cloud. Finally, Figure \ref{fig:regis} is a qualitative comparison of the HC and GP-based approaches to performing point cloud registration on the \textit{Dragon} cloud.
\begin{figure*}
     \centering
        \includegraphics[trim=0cm 3cm 0cm 3cm, clip, width=\linewidth]{images/lucy.pdf}
         \caption{Surface reconstruction results of the simplified version of the Lucy point cloud for simplification ratio $\alpha=0.002$ for all evaluated simplification techniques except PC-Simp.}
         \vspace{-0.3cm}
         \label{fig:lucy_all}
%
\end{figure*}
\begin{figure*}
     \centering
        \includegraphics[trim=0cm 0.5cm 0cm 0cm, clip,width=\linewidth]{images/armadillo.pdf}
         \caption{Simplified representations of the Armadillo point cloud for simplification ratio $\alpha=0.05$ (top row) and associated reconstructed meshes (bottom row) for all evaluated simplification techniques.}
         \vspace{-0.3cm}\label{fig:armadillo_all}
\end{figure*}
%
% trim= l d r u
%
\begin{figure*}
     \centering
        \includegraphics[scale=0.5, trim=0.2cm 0cm 14cm 0cm, clip]{images/regis.pdf}
         \caption{Global and ICP registration results shown for the original, HC and GP-simplified versions (simplification ratio $\alpha=0.03$) of the Stanford dragon.}
         \vspace{-0.4cm}
         \label{fig:regis}
\end{figure*}