\section{Related work}
\label{sec:rw}
In this section we will introduce a number of existing point cloud simplification techniques, with a particular focus on works which have a feature-preserving element to their approach. Some of the earliest curvature-sensitive simplification techniques were proposed by Pauly \textit{et al.} \cite{pauly2002efficient} and Moenning \textit{et al.} \cite{moenning2003new}. The former method, termed \textit{Hierarchical Clustering} (HC), recursively divides the original point cloud into two sets, until each child set attains a size smaller than a threshold \textit{size parameter}. Moreover, a \textit{variation parameter} plays an important role in sparsifying regions of low curvature by selective splitting. The perceptual quality and the size of the simplified cloud depend entirely on these two parameters, which must be carefully and manually tuned, making HC unsuitable for automated applications. Additionally, the surface reconstructions obtained from HC-simplified point clouds are often poor for clouds with complex surfaces, as will be seen in Section \ref{sec:experiments}. This is because it is challenging to tune the parameters of HC in such a way that preservation of sharp features is achieved whilst still ensuring dense coverage of the original cloud.

Another widely-used technique is \textit{Weighted Locally Optimal Projection} (WLOP) proposed by Huang \textit{et al.} \cite{huang2009consolidation}. In this work, the authors modified the existing parameterization-free denoising simplification scheme termed \textit{Locally Optimal Projection} (LOP) \cite{lipman2007parameterization}, which is  unsuitable for non-uniformly distributed point clouds. WLOP overcomes this limitation by incorporating locally adaptive density weights into LOP. Although WLOP results in an evenly distributed simplified cloud, it still lacks sensitivity towards salient geometric features which will also become apparent in Section \ref{sec:experiments}. Recently, Potamias \textit{et al.} \cite{potamias2022revisiting} have proposed a graph neural network-based learnable simplification technique which uses a modified variant of Chamfer distance in order to backpropagate errors. Their method can simplify point clouds in real-time but involves a computationally intensive training process using large point cloud datasets such as TOSCA \cite{bronstein2008numerical} and ModelNet10 \cite{wu20153d}. Moreover, their model's efficiency is limited to simplifying point clouds which are structurally similar to the learned data, as inherently neural networks struggle to generalize outside of the domain of the training data. 

\textit{Approximate Intrinsic Voxel Structure for Point Cloud Simplification} (AIVS), introduced by Lv \textit{et al.} \cite{lv2021approximate}, combines global voxel structure and local farthest point sampling to generate simplification demand-specific clouds which can be either isotropic, curvature-sensitive or have sharp edge preservation. As with HC however, AIVS requires manual tuning of user-specified parameters in order to obtain optimal results. Additionally, even in parallel computation mode, AIVS is quite costly in terms of computational runtime. Potamias \textit{et al.} and  Lv \textit{et al.} do not provide open-source implementations of their curvature-sensitive simplification techniques, which poses a challenge for reproducibility and benchmarking. Qi \textit{et al.} \cite{qi2019feature} introduced \textit{PC-Simp}, a method which aims to produce uniformly-dense and feature-sensitive simplified clouds, leveraging ideas from graph signal processing. This uniformity depends on a \textit{weight parameter} which as with HC and AIVS, is user-specified. Alongside simplification, they also apply their technique to point cloud registration. However, in practice PC-Simp is unreliable for complex-surfaced point clouds as it fails to provide a high degree of feature-preservation, regardless of the weight parameter chosen. Additionally, as discussed later in Section \ref{sec:rw}, the runtime of this technique is considerably longer than any other method tested.

Finally, it has been observed that most of the aforementioned works on feature-preserving point cloud simplification schemes experiment on structurally simple point clouds. Furthermore, surface reconstruction results are rarely presented and discussed. Hence, to underline the efficiency of our method, we experiment on point clouds generated from complex-surfaced objects and provide the corresponding reconstruction results.