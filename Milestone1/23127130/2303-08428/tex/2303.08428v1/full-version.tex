
%% bare_jrnl.tex
%% V1.4
%% 2012/12/27
%% by Michael Shell
%% see http://www.michaelshell.org/
%% for current contact information.
%%
%% This is a skeleton file demonstrating the use of IEEEtran.cls
%% (requires IEEEtran.cls version 1.8 or later) with an IEEE journal paper.
%%
%% Support sites:
%% http://www.michaelshell.org/tex/ieeetran/
%% http://www.ctan.org/tex-archive/macros/latex/contrib/IEEEtran/
%% and
%% http://www.ieee.org/



% *** Authors should verify (and, if needed, correct) their LaTeX system  ***
% *** with the testflow diagnostic prior to trusting their LaTeX platform ***
% *** with production work. IEEE's font choices can trigger bugs that do  ***
% *** not appear when using other class files.                            ***
% The testflow support page is at:
% http://www.michaelshell.org/tex/testflow/


%%*************************************************************************
%% Legal Notice:
%% This code is offered as-is without any warranty either expressed or
%% implied; without even the implied warranty of MERCHANTABILITY or
%% FITNESS FOR A PARTICULAR PURPOSE!
%% User assumes all risk.
%% In no event shall IEEE or any contributor to this code be liable for
%% any damages or losses, including, but not limited to, incidental,
%% consequential, or any other damages, resulting from the use or misuse
%% of any information contained here.
%%
%% All comments are the opinions of their respective authors and are not
%% necessarily endorsed by the IEEE.
%%
%% This work is distributed under the LaTeX Project Public License (LPPL)
%% ( http://www.latex-project.org/ ) version 1.3, and may be freely used,
%% distributed and modified. A copy of the LPPL, version 1.3, is included
%% in the base LaTeX documentation of all distributions of LaTeX released
%% 2003/12/01 or later.
%% Retain all contribution notices and credits.
%% ** Modified files should be clearly indicated as such, including  **
%% ** renaming them and changing author support contact information. **
%%
%% File list of work: IEEEtran.cls, IEEEtran_HOWTO.pdf, bare_adv.tex,
%%                    bare_conf.tex, bare_jrnl.tex, bare_jrnl_compsoc.tex,
%%                    bare_jrnl_transmag.tex
%%*************************************************************************

% Note that the a4paper option is mainly intended so that authors in
% countries using A4 can easily print to A4 and see how their papers will
% look in print - the typesetting of the document will not typically be
% affected with changes in paper size (but the bottom and side margins will).
% Use the testflow package mentioned above to verify correct handling of
% both paper sizes by the user's LaTeX system.
%
% Also note that the "draftcls" or "draftclsnofoot", not "draft", option
% should be used if it is desired that the figures are to be displayed in
% draft mode.
%
\documentclass[journal]{IEEEtran}
\usepackage{amsmath}
\usepackage{}
\usepackage{amssymb}
\usepackage{graphicx}
\usepackage{colortbl}

% pseudo code
\usepackage{algorithm}
\usepackage{algorithmic}


\usepackage{amsfonts}
\usepackage{latexsym,amsmath,amssymb}
\usepackage{graphicx,psfrag}
\renewcommand\arraystretch{1}
\newcommand{\Up}{\Upsilon}
\newcommand{\La}{\Lambda}
\newcommand{\la}{\lambda}
\newcommand{\no}{\nonumber}
\newcommand{\be}{\begin{equation}}
\newcommand{\ee}{\end{equation}}
\def\bee#1\eee{\begin{align}#1\end{align}}
\def\beee#1\eeee{\begin{align*}#1\end{align*}}
\def\le{\left}
\def\ri{\right}
\def\no{\nonumber}
\def\bee#1\eee{\begin{align}#1\end{align}}
\def\beee#1\eeee{\begin{align*}#1\end{align*}}
\newcommand{\bea}{\begin{eqnarray}}
\newcommand{\eea}{\end{eqnarray}}
\def\ba{\begin{array}}
\def\ea{\end{array}}
\def\dis{\displaystyle}
\def\eq{\stackrel\triangle=}
\def\ok{$\blacksquare$}

\newcommand{\bse}{\begin{subequations}}
\newcommand{\ese}{\end{subequations}}
\newtheorem{theorem}{Theorem}%[section]
\newtheorem{assumption}{Assumption}%[section]
\newtheorem{lemma}{Lemma}%[section]
\newtheorem{remark}{Remark}%[section]
\newtheorem{definition}{Definition}%[section]
\newtheorem{corollary}{Corollary}%[section]
\newtheorem{proposition}{Proposition}
\newtheorem{problem}{Problem}
\newtheorem{example}{Example}
%
% If IEEEtran.cls has not been installed into the LaTeX system files,
% manually specify the path to it like:
% \documentclass[journal]{../sty/IEEEtran}





% Some very useful LaTeX packages include:
% (uncomment the ones you want to load)


% *** MISC UTILITY PACKAGES ***
%
%\usepackage{ifpdf}
% Heiko Oberdiek's ifpdf.sty is very useful if you need conditional
% compilation based on whether the output is pdf or dvi.
% usage:
% \ifpdf
%   % pdf code
% \else
%   % dvi code
% \fi
% The latest version of ifpdf.sty can be obtained from:
% http://www.ctan.org/tex-archive/macros/latex/contrib/oberdiek/
% Also, note that IEEEtran.cls V1.7 and later provides a builtin
% \ifCLASSINFOpdf conditional that works the same way.
% When switching from latex to pdflatex and vice-versa, the compiler may
% have to be run twice to clear warning/error messages.






% *** CITATION PACKAGES ***
%
%\usepackage{cite}
% cite.sty was written by Donald Arseneau
% V1.6 and later of IEEEtran pre-defines the format of the cite.sty package
% \cite{} output to follow that of IEEE. Loading the cite package will
% result in citation numbers being automatically sorted and properly
% "compressed/ranged". e.g., [1], [9], [2], [7], [5], [6] without using
% cite.sty will become [1], [2], [5]--[7], [9] using cite.sty. cite.sty's
% \cite will automatically add leading space, if needed. Use cite.sty's
% noadjust option (cite.sty V3.8 and later) if you want to turn this off
% such as if a citation ever needs to be enclosed in parenthesis.
% cite.sty is already installed on most LaTeX systems. Be sure and use
% version 4.0 (2003-05-27) and later if using hyperref.sty. cite.sty does
% not currently provide for hyperlinked citations.
% The latest version can be obtained at:
% http://www.ctan.org/tex-archive/macros/latex/contrib/cite/
% The documentation is contained in the cite.sty file itself.


% *** GRAPHICS RELATED PACKAGES ***
%
\ifCLASSINFOpdf
  % \usepackage[pdftex]{graphicx}
  % declare the path(s) where your graphic files are
  % \graphicspath{{../pdf/}{../jpeg/}}
  % and their extensions so you won't have to specify these with
  % every instance of \includegraphics
  % \DeclareGraphicsExtensions{.pdf,.jpeg,.png}
\else
  % or other class option (dvipsone, dvipdf, if not using dvips). graphicx
  % will default to the driver specified in the system graphics.cfg if no
  % driver is specified.
  % \usepackage[dvips]{graphicx}
  % declare the path(s) where your graphic files are
  % \graphicspath{{../eps/}}
  % and their extensions so you won't have to specify these with
  % every instance of \includegraphics
  % \DeclareGraphicsExtensions{.eps}
\fi
% graphicx was written by David Carlisle and Sebastian Rahtz. It is
% required if you want graphics, photos, etc. graphicx.sty is already
% installed on most LaTeX systems. The latest version and documentation
% can be obtained at:
% http://www.ctan.org/tex-archive/macros/latex/required/graphics/
% Another good source of documentation is "Using Imported Graphics in
% LaTeX2e" by Keith Reckdahl which can be found at:
% http://www.ctan.org/tex-archive/info/epslatex/
%
% latex, and pdflatex in dvi mode, support graphics in encapsulated
% postscript (.eps) format. pdflatex in pdf mode supports graphics
% in .pdf, .jpeg, .png and .mps (metapost) formats. Users should ensure
% that all non-photo figures use a vector format (.eps, .pdf, .mps) and
% not a bitmapped formats (.jpeg, .png). IEEE frowns on bitmapped formats
% which can result in "jaggedy"/blurry rendering of lines and letters as
% well as large increases in file sizes.
%
% You can find documentation about the pdfTeX application at:
% http://www.tug.org/applications/pdftex


% *** MATH PACKAGES ***
%
%\usepackage[cmex10]{amsmath}
% A popular package from the American Mathematical Society that provides
% many useful and powerful commands for dealing with mathematics. If using
% it, be sure to load this package with the cmex10 option to ensure that
% only type 1 fonts will utilized at all point sizes. Without this option,
% it is possible that some math symbols, particularly those within
% footnotes, will be rendered in bitmap form which will result in a
% document that can not be IEEE Xplore compliant!
%
% Also, note that the amsmath package sets \interdisplaylinepenalty to 10000
% thus preventing page breaks from occurring within multiline equations. Use:
%\interdisplaylinepenalty=2500
% after loading amsmath to restore such page breaks as IEEEtran.cls normally
% does. amsmath.sty is already installed on most LaTeX systems. The latest
% version and documentation can be obtained at:
% http://www.ctan.org/tex-archive/macros/latex/required/amslatex/math/





% *** SPECIALIZED LIST PACKAGES ***
%
%\usepackage{algorithmic}
% algorithmic.sty was written by Peter Williams and Rogerio Brito.
% This package provides an algorithmic environment fo describing algorithms.
% You can use the algorithmic environment in-text or within a figure
% environment to provide for a floating algorithm. Do NOT use the algorithm
% floating environment provided by algorithm.sty (by the same authors) or
% algorithm2e.sty (by Christophe Fiorio) as IEEE does not use dedicated
% algorithm float types and packages that provide these will not provide
% correct IEEE style captions. The latest version and documentation of
% algorithmic.sty can be obtained at:
% http://www.ctan.org/tex-archive/macros/latex/contrib/algorithms/
% There is also a support site at:
% http://algorithms.berlios.de/index.html
% Also of interest may be the (relatively newer and more customizable)
% algorithmicx.sty package by Szasz Janos:
% http://www.ctan.org/tex-archive/macros/latex/contrib/algorithmicx/




% *** ALIGNMENT PACKAGES ***
%
%\usepackage{array}
% Frank Mittelbach's and David Carlisle's array.sty patches and improves
% the standard LaTeX2e array and tabular environments to provide better
% appearance and additional user controls. As the default LaTeX2e table
% generation code is lacking to the point of almost being broken with
% respect to the quality of the end results, all users are strongly
% advised to use an enhanced (at the very least that provided by array.sty)
% set of table tools. array.sty is already installed on most systems. The
% latest version and documentation can be obtained at:
% http://www.ctan.org/tex-archive/macros/latex/required/tools/


% IEEEtran contains the IEEEeqnarray family of commands that can be used to
% generate multiline equations as well as matrices, tables, etc., of high
% quality.




% *** SUBFIGURE PACKAGES ***
%\ifCLASSOPTIONcompsoc
%  \usepackage[caption=false,font=normalsize,labelfont=sf,textfont=sf]{subfig}
%\else
%  \usepackage[caption=false,font=footnotesize]{subfig}
%\fi
% subfig.sty, written by Steven Douglas Cochran, is the modern replacement
% for subfigure.sty, the latter of which is no longer maintained and is
% incompatible with some LaTeX packages including fixltx2e. However,
% subfig.sty requires and automatically loads Axel Sommerfeldt's caption.sty
% which will override IEEEtran.cls' handling of captions and this will result
% in non-IEEE style figure/table captions. To prevent this problem, be sure
% and invoke subfig.sty's "caption=false" package option (available since
% subfig.sty version 1.3, 2005/06/28) as this is will preserve IEEEtran.cls
% handling of captions.
% Note that the Computer Society format requires a larger sans serif font
% than the serif footnote size font used in traditional IEEE formatting
% and thus the need to invoke different subfig.sty package options depending
% on whether compsoc mode has been enabled.
%
% The latest version and documentation of subfig.sty can be obtained at:
% http://www.ctan.org/tex-archive/macros/latex/contrib/subfig/




% *** FLOAT PACKAGES ***
%
%\usepackage{fixltx2e}
% fixltx2e, the successor to the earlier fix2col.sty, was written by
% Frank Mittelbach and David Carlisle. This package corrects a few problems
% in the LaTeX2e kernel, the most notable of which is that in current
% LaTeX2e releases, the ordering of single and double column floats is not
% guaranteed to be preserved. Thus, an unpatched LaTeX2e can allow a
% single column figure to be placed prior to an earlier double column
% figure. The latest version and documentation can be found at:
% http://www.ctan.org/tex-archive/macros/latex/base/


%\usepackage{stfloats}
% stfloats.sty was written by Sigitas Tolusis. This package gives LaTeX2e
% the ability to do double column floats at the bottom of the page as well
% as the top. (e.g., "\begin{figure*}[!b]" is not normally possible in
% LaTeX2e). It also provides a command:
%\fnbelowfloat
% to enable the placement of footnotes below bottom floats (the standard
% LaTeX2e kernel puts them above bottom floats). This is an invasive package
% which rewrites many portions of the LaTeX2e float routines. It may not work
% with other packages that modify the LaTeX2e float routines. The latest
% version and documentation can be obtained at:
% http://www.ctan.org/tex-archive/macros/latex/contrib/sttools/
% Do not use the stfloats baselinefloat ability as IEEE does not allow
% \baselineskip to stretch. Authors submitting work to the IEEE should note
% that IEEE rarely uses double column equations and that authors should try
% to avoid such use. Do not be tempted to use the cuted.sty or midfloat.sty
% packages (also by Sigitas Tolusis) as IEEE does not format its papers in
% such ways.
% Do not attempt to use stfloats with fixltx2e as they are incompatible.
% Instead, use Morten Hogholm'a dblfloatfix which combines the features
% of both fixltx2e and stfloats:
%
% \usepackage{dblfloatfix}
% The latest version can be found at:
% http://www.ctan.org/tex-archive/macros/latex/contrib/dblfloatfix/




%\ifCLASSOPTIONcaptionsoff
%  \usepackage[nomarkers]{endfloat}
% \let\MYoriglatexcaption\caption
% \renewcommand{\caption}[2][\relax]{\MYoriglatexcaption[#2]{#2}}
%\fi
% endfloat.sty was written by James Darrell McCauley, Jeff Goldberg and
% Axel Sommerfeldt. This package may be useful when used in conjunction with
% IEEEtran.cls'  captionsoff option. Some IEEE journals/societies require that
% submissions have lists of figures/tables at the end of the paper and that
% figures/tables without any captions are placed on a page by themselves at
% the end of the document. If needed, the draftcls IEEEtran class option or
% \CLASSINPUTbaselinestretch interface can be used to increase the line
% spacing as well. Be sure and use the nomarkers option of endfloat to
% prevent endfloat from "marking" where the figures would have been placed
% in the text. The two hack lines of code above are a slight modification of
% that suggested by in the endfloat docs (section 8.4.1) to ensure that
% the full captions always appear in the list of figures/tables - even if
% the user used the short optional argument of \caption[]{}.
% IEEE papers do not typically make use of \caption[]'s optional argument,
% so this should not be an issue. A similar trick can be used to disable
% captions of packages such as subfig.sty that lack options to turn off
% the subcaptions:
% For subfig.sty:
% \let\MYorigsubfloat\subfloat
% \renewcommand{\subfloat}[2][\relax]{\MYorigsubfloat[]{#2}}
% However, the above trick will not work if both optional arguments of
% the \subfloat command are used. Furthermore, there needs to be a
% description of each subfigure *somewhere* and endfloat does not add
% subfigure captions to its list of figures. Thus, the best approach is to
% avoid the use of subfigure captions (many IEEE journals avoid them anyway)
% and instead reference/explain all the subfigures within the main caption.
% The latest version of endfloat.sty and its documentation can obtained at:
% http://www.ctan.org/tex-archive/macros/latex/contrib/endfloat/
%
% The IEEEtran \ifCLASSOPTIONcaptionsoff conditional can also be used
% later in the document, say, to conditionally put the References on a
% page by themselves.




% *** PDF, URL AND HYPERLINK PACKAGES ***
%
%\usepackage{url}
% url.sty was written by Donald Arseneau. It provides better support for
% handling and breaking URLs. url.sty is already installed on most LaTeX
% systems. The latest version and documentation can be obtained at:
% http://www.ctan.org/tex-archive/macros/latex/contrib/url/
% Basically, \url{my_url_here}.




% *** Do not adjust lengths that control margins, column widths, etc. ***
% *** Do not use packages that alter fonts (such as pslatex).         ***
% There should be no need to do such things with IEEEtran.cls V1.6 and later.
% (Unless specifically asked to do so by the journal or conference you plan
% to submit to, of course. )


% correct bad hyphenation here
\hyphenation{op-tical net-works semi-conduc-tor}


\begin{document}
% paper title
% can use linebreaks \\ within to get better formatting as desired
% Do not put math or special symbols in the title.
\title{Criteria for stabilizing a multi-delay stochastic system with multiplicative control-dependent noises}
% author names and IEEE memberships
% note positions of commas and nonbreaking spaces ( ~ ) LaTeX will not break
% a structure at a ~ so this keeps an author's name from being broken across
% two lines.
% use \thanks{} to gain access to the first footnote area
% a separate \thanks must be used for each paragraph as LaTeX2e's \thanks
% was not built to handle multiple paragraphs


\author{
Cheng Tan \IEEEmembership{Member,~IEEE},
%Haoting Sui,
%Yuzhe Li,
Zhengqiang Zhang \IEEEmembership{Senior Member,~IEEE},
Haoting Sui,\\
Wing Shing Wong  \IEEEmembership{Life Fellow,~IEEE}
           % <-this % stops a space
\thanks{This work was supported in part by the National Natural Science Foundation of China under Grants 62173206;
the Natural Science Foundation of Shandong Province under Grant ZR2021ZD13;
China Postdoctoral Science Foundation under Grant 2021M691849.}

\thanks{C. Tan, Z. Zhang and H. Sui are with the School of Engineering, QuFu Normal University, Rizhao, Shandong 276800, China,
 (e-mail: tancheng1987love@163.com, qufuzzq@126.com).}

%\thanks{Y. Li is with the State Key Laboratory of Synthetical Automation for Process Industries, Northeastern University, Shenyang 110004, China.
%%the College of Electrical Engineering and Automation Shandong University of Science and Technology, Qingdao 266590, China (e-mail: hszhang@sdu.edu.cn).
%}

\thanks{
W. S. Wong is with the Department of Information Engineering, The Chinese University of Hong Kong, Shatin, N. T., Hong Kong (e-mail: wswong@ie.cuhk.edu.hk).}
}



%\thanks{
%H. Zhang is with the School of Control Science and Engineering, Shandong University, Jinan, Shandong 250061, China (\small  e-mail: hszhang@sdu.edu.cn). }

%\thanks{W. S. Wong is with the Department of Information Engineering, The Chinese University of Hong Kong, Shatin, N. T., Hong Kong (\small  e-mail: wswong@ie.cuhk.edu.hk).}



% that ends a line with a % and do not let a space in before the next \thanks.
% Spaces after \IEEEmembership other than the last one are OK (and needed) as
% you are supposed to have spaces between the names. For what it is worth,
% this is a minor point as most people would not even notice if the said evil
% space somehow managed to creep in.


% The paper headers
%\markboth{Journal of \LaTeX\ Class Files,~Vol.~11, No.~4, December~2012}%
%{Shell \MakeLowercase{\textit{et al.}}: Bare Demo of IEEEtran.cls for Journals}
% The only time the second header will appear is for the odd numbered pages
% after the title page when using the twoside option.
%
% *** Note that you probably will NOT want to include the author's ***
% *** name in the headers of peer review papers.                   ***
% You can use \ifCLASSOPTIONpeerreview for conditional compilation here if
% you desire.


% If you want to put a publisher's ID mark on the page you can do it like
% this:
%\IEEEpubid{0000--0000/00\$00.00~\copyright~2012 IEEE}
% Remember, if you use this you must call \IEEEpubidadjcol in the second
% column for its text to clear the IEEEpubid mark.


% use for special paper notices
%\IEEEspecialpapernotice{(Invited Paper)}


% make the title area
\maketitle

% As a general rule, do not put math, special symbols or citations
% in the abstract or keywords.
\begin{abstract}
In this paper, we investigate the mean-square stabilization for discrete-time stochastic systems that endure both multiple input delays and multiplicative control-dependent noises.
For such multi-delay stochastic systems, we for the first time put forward two stabilization criteria: Riccati type and Lyapunov type.
On the one hand, we adopt a reduction method to reformulate the original multi-delay stochastic system to a delay-free auxiliary system and present their equivalent proposition for stabilization.
Then, by introducing a delay-dependent algebraic Riccati equation (DDARE), we prove that the system under consideration is stabilizable if and only if the developed DDARE has a unique positive definite solution.
On the other hand, we characterize the delay-dependent Lyapunov equation (DDLE)-based criterion, which can be verified by linear matrix inequality (LMI) feasibility test.
Besides, under some restricted structure, we propose an existence theorem of delay margin and more importantly, derive an explicit formula for computing its exact value.
\end{abstract}

% Note that keywords are not normally used for peerreview papers.
\begin{IEEEkeywords}
Delay-dependent algebraic Riccati equation, %(DDARE),
delay margin,
multiple input delay, stabilization, stochastic system
\end{IEEEkeywords}


% For peer review papers, you can put extra information on the cover
% page as needed:
% \ifCLASSOPTIONpeerreview
% \begin{center} \bfseries EDICS Category: 3-BBND \end{center}
% \fi
%
% For peerreview papers, this IEEEtran command inserts a page break and
% creates the second title. It will be ignored for other modes.
\IEEEpeerreviewmaketitle


\section{Introduction}

In the last decades, the stability/stabilization issues for stochastic systems have attracted considerable interest because of their extensive applications
in economics as well as in engineering fields; See \cite{A2012}--\cite{deng2022} and the references therein. %for a partial list of references.
For linear time invariant systems, a large body of excellent results, including necessary and sufficient conditions, have been developed.
In particular, the Lyapunov-type stabilization criteria was derived in terms of the feasibility of a certain LMI in \cite{zhouxy2000}, while the Riccati-type result was developed via a unique positive definite solution satisfying generalized algebraic Riccati equation (GARE) in \cite{huang2008}.
As a supplement, the mean-square stabilization was first characterized by the spectral locations of coefficient matrices in \cite{zhangwh2004}.

The aforementioned studies were exclusively concerned with delay-free stochastic models.
Recently, as the product of the integration of control technology and network communication technology, networked control systems (NCSs) have been actively researched due to their widespread applications.
Because of the limitation of network bandwidth power, the signal transmitted in unreliable channel may encounter various network-induced uncertainties including data packet loss (multiplicative noise) and transmission delay.
As a result, some studies have concentrated on the scenarios where time delay and random noises occur simultaneously \cite{yued2005}-\cite{tan2019}.
The most popular methods, such as Lyapunov-Krasivskii functional approach, are mainly suited for exploiting sufficient stabilization conditions by virtue of LMIs.
Specifically, for single-delay stochastic model, the necessary and sufficient stabilization condition was first presented in terms of coupled algebraic Riccati equations (CAREs) in \cite{lilin2015}.
However, it is difficult to calculate the positive definite solutions because CAREs are nonlinear and their positive solutions are coupled.
How to best utilize available information to design a stabilizing control policy as well as the search for stabilization criteria for more general multi-delay stochastic dynamics remain open and challenging questions.


On the other hand, delay margin, as a fundamental measure of robust stabilization against uncertain delays, has also received extensively attention; See \cite{margin01}--\cite{margin03}.
There are two natural questions for multi-delay stochastic systems: What is the largest range of delay so that there exists an admissible control law that can stabilize the target system within that entire range? How to calculate this delay margin?
Unfortunately, to the best of our knowledge, there does not exist work on characterizing the delay margin for multi-delay stochastic model, which motivates us to undertake an in-depth study.

Facilitated by stochastic control techniques, we seek to provide a series of stabilization criteria for stochastic system with multiple input delays and multiplicative noises.
Different from the previous work, one significant contribution is that our control law is designed as the feedback of an extended state that contains the recent available state information and part values from previous control inputs.
It is remarkable that the developed criteria are necessary and sufficient, which are first obtained within the framework of multi-delay stochastic system with multiplicative noises in control variables.
These criteria run in parallel to the classical results in \cite{zhouxy2000,huang2008}.

Our research methodology is described as follows.
First, motivated by predictive control methodology proposed in \cite{art1982}, we adopt a reduction strategy to transform the original multi-delay stochastic system into an auxiliary delay-free model and demonstrate their equivalent proposition for stabilization.
Then, we present the Riccati-type stabilization criterion and the design procedure of the stabilizing control law.
The expression is that system is stabilizable if and only if the predefined DDARE has a unique positive definite solution.
Of equal importance, we characterize the Lyapunov-type stabilization criterion by means of DDLE.
Utilizing Schur complement decomposition and matrix transformation technique, this criterion can be expressed by LMI-based condition.
Specifically, as an application, we apply our theory to study the delay margin problem.
On the basis of the stabilization criteria and operator theory, we show that there exists a unique delay margin for some restricted single-delay stochastic model.
More importantly, some explicit formula for computing the delay margin is derived for uncoupled system.
At last, we present two simple examples to confirm
%show the effectiveness of
our theoretic results.

Notation: For any integer $i< j$, we define $\mathbb{N}_{[i,j]}\triangleq \{i,\cdots,j\}$.
$Z\geq0~(>0)$ means that $Z$ is a positive semidefinite (positive definite) matrix, and $Z_1\geq Z_2~(>Z_2)$ means that $Z_1-Z_2\geq0~(>0)$.
$\{\omega_t,t\in\mathbb{N}\}$ denotes a sequence of real random variables defined on the complete filtered probability space $(\Omega,\mathcal{F},\mathcal{P};\mathcal{F}_t)$ with $\mathcal{F}_t=\sigma\{\omega_s,~s\in\mathbb{N}_{[0,t]} \}$.
Define $\hat{x}_{s|t}={\bf E}(x_s|\mathcal{F}_t)$ which signifies the conditional expectation of $x_s$ w.r.t. $\mathcal{F}_t$.

\section{Problem formulation}
In this paper, we consider the following discrete-time stochastic system with both multiple delays and multiplicative control-dependent noises
\be
x_{t+1}=Ax_t+\sum_{\tau=0}^D \le(B_\tau +\omega^\tau_t C_\tau \ri)u_{t-\tau},\label{system01}
\ee
where $x_t\in\mathbb{R}^n$ is the state, and $u_t\in\mathbb{R}^m$ is the control input executed at time $t\geq 0$.
System matrices
%$A$, $B_\tau$, and $C_\tau,~\tau\in\mathbb{N}_{[0,D]}$, are constant of compatible dimensions, where $\mathbb{N}_{[i,j]}\triangleq \{i,\cdots,j\}$.
Assume that the initial conditions $x_0$, $u_t$, $t\in \mathbb{N}_{[-d,-1]}$, are given a priori.
The multiplicative noises are assumed to be random sequences with independent and identically distributed realizations
\be
{\bf E}\le(\omega_t^\tau \ri)=0,
~{\bf E}\le(\omega_t^\tau \omega_s^\tau\ri)=\sigma_\tau^2\delta_{ts}, ~\forall\tau\in\mathbb{N}_{[0,D]}
\ee
where $\delta_{ts}$ is a Kronecker function.
%and $\mathbb{N}_{[i,j]}\triangleq \{i,\cdots,j\}$.
We further denote
\be
\omega_t\triangleq \left(\omega_t^0~\omega_t^1\cdots \omega_t^D \right)',~t\geq0.
\ee
which defines the $\sigma$-algebra as $\mathcal{F}_t=\sigma\{\omega_s,~s\in\mathbb{N}_{[0,t]} \}$.
%To facilitate the narrative, define $\hat{x}_{s|t}={\bf E}(x_s|\mathcal{F}_t)$ which signifies the conditional expectation of $x_s$ w.r.t. $\mathcal{F}_t$.
%For notational convenience, we describe system (\ref{system01}) as $\left(A,B,C,T\right)$.
Before proceeding further, we impose the following definition of asymptotical mean-square stabilization.
\begin{definition}
\label{def01}
System \eqref{system01} is said to be asymptotically mean-square stabilizable, if there exists a feedback control input $u_t$
such that the corresponding closed-loop system  is asymptotically mean-square stable, i.e. for any initial values $x_0$ and $u_t,~t\in\mathbb{N}_{[-D,-1]}$, the state $x_{t}$ in (\ref{system01}) satisfies $\lim_{t\rightarrow \infty} {\bf E}(x_{t}'x_t)=0$.
\end{definition}


This paper is concerned with the asymptotical mean-square stabilization problem for multi-delay stochastic system \eqref{system01} in which the control input should be $\mathcal{F}_{t-1}$-measurable.
It is remarkable that if the control law is designed to be a feedback of the state, owing to information gap, it is difficult to obtain the necessary and sufficient condition \cite{lilin2015,tan2019}.
%Until now, to the best of our knowledge, there have been no firm results to this problem.
To tackle this problem, motivated by Smith predictor in \cite{Ito1981}, we introduce a viable controller construction by utilizing an extended state vector that contains the recent available state and previous control inputs in finite horizon.
In this paper, the set of admissible control input is given as
\be
\mathcal {U}_{ad}\triangleq \le\{u_t\in L^2_{\mathcal{F}},u_{t}=K_0 x_t+\sum_{\tau=1}^D K_\tau u_{t-\tau} \ri\},\label{ad-input-set}
\ee
where
\bee
L^2_{\mathcal{F}}\triangleq \{ u(t)~is~\mathcal{F}_{t}-measurable,\sum_{t=0}^{\infty} {\bf E}(u_t'u_t)<\infty \}. \label{ad-input-set00}
\eee

The problems to be solved are formulated as follows.

\begin{itemize}
  \item Explore some control strategy $u_t\in\mathcal {U}_{ad}$ to stabilize system (\ref{system01}), while develop the necessary and sufficient stabilization conditions.
  \item Explore exact delay margin to guarantee stabilization.
\end{itemize}
%
%\noindent{\bf Problem 1} Explore some control strategy $u_t\in\mathcal {U}_{ad}$ to stabilize system (\ref{system01}), while develop the necessary and sufficient stabilization conditions.
%
%\noindent{\bf Problem 2} Explore exact delay margin to guarantee stabilization.

\begin{figure}[thpb]
      \centering
      \includegraphics[scale=0.33]{NCS.jpg}
      \centering\caption{\scriptsize{WNCS over multi-path routing network}}
\end{figure}

\begin{remark}
Note that the considered stochastic system has a wide application in wireless NCS (WNCS).
Notice that the considered WNCS with both transmission delay and packet loss in \cite{tan2015} is a special case of system \eqref{system01}.
More generally, as a flexible driving architecture in wireless sensor and Ad-hoc networks, the multi-path routing approach has gained popularity for various network management purposes \cite{multi01}--\cite{multi03}.
Fig. 1 shows a typical structure of WNCS over unreliable multi-path routing network, which is responsible for the transmission of control signals, where the delivered control input over the $i$-th path suffers both constant delay $d_i\geq 0$ and data packet loss.
Here, the arrival process of the control signal is modeled via a binary random variable $\gamma^{i}_t$ defined as
\be
\gamma^{i}_t=
\le\{
\begin{aligned}
& 1, ~\text{if $u_{t-d_i}$ has been delivered to the actuator},\\
& 0 , ~\text{otherwise}.
\end{aligned}
\ri.
\ee
Therefore, the dynamics of the overall WNCS follows
\be
x_{t+1}=Ax_t+\sum_{i=1}^N \gamma^{i}_t Bu_{t-d_i}.\label{NCS-original}
\ee
%where $0\leq d_1\leq d_2\leq\cdots\leq d_N$.
Assume that $\gamma^{i}_t$ follows an identical distributed realization
\be
\mathcal{P}(\gamma^{i}_t=0)=p_i,
~\mathcal{P}(\gamma^{i}_t=1)=1-p_i,
~p_i\in[0,1].
\ee
When we define $\omega_t^i=\gamma^{i}_t-(1-p_i)$, WNCS in \eqref{NCS-original} can be viewed as a special case of multi-delay stochastic system \eqref{system01}.
\end{remark}

\begin{remark}
For the multi-delay stochastic systems, \cite{lilin2016,lilin2021} addresses the LQ optimization problem utilizing the stochastic maximum principle. While dealing with the infinite horizon optimization, a necessary and sufficient stabilization condition is derived in terms of CAREs, where $D+1$ positive solutions are to be determined. Generally, it is impossible to verify whether the given result is valid.
Therefore, our purpose is to seeking more concise criteria for stabilizing system (\ref{system01}).
\end{remark}
\section{Main results}

\subsection{Riccati-type Stabilization Criterion}
Motivated by the predictive control technique in \cite{art1982}, we first adopt a feasible reduction strategy to transform system (\ref{system01}) into a delay-free equivalent form.
Based on that, we then develop the Riccati-type stabilization criterion.
To achieve this goal, we need a basic assumption that \emph{$A$ is invertible}.

Define an auxiliary state
\be
\eta_t\hspace{-0.8mm}=\hspace{-0.8mm}x_t
\hspace{-0.8mm}+\hspace{-0.8mm}
\sum_{\tau=1}^D \le[\sum_{j=\tau}^D A^{\tau-j-1} \left(B_j+\omega^j_{t+j-\tau}C_j \right)  \ri]u_{t-\tau}.\label{system02}
\ee
It follows from (\ref{system01}) that
\bee
\eta_{t+1}
%=&Ax_t+\sum_{\tau=0}^D \le(B_\tau +\omega^\tau_t C_\tau \ri)u_{t-\tau} \nonumber\\
%&+\sum_{\tau=1}^D \le[\sum_{j=\tau}^D A^{\tau-j-1} \left(B_j+\omega^j_{t+1+j-\tau}C_j \right)  \ri]u_{t+1-\tau} \no\\
=&Ax_t+\sum_{\tau=1}^D \le[\sum_{j=\tau}^D A^{\tau-j} \left(B_j+\omega^j_{t+j-\tau}C_j \right)  \ri]u_{t-\tau} \no\\
&+\sum_{j=0}^D A^{-j}\left(B_j+\omega_{t+j}^j C_j \right) u_t,
\eee
which leads to the following delay-free expression
\be
\eta_{t+1}=A\eta_t +\sum_{j=0}^D A^{-j}\left(B_j+\omega_{t+j}^j C_j \right) u_t.  \label{system03}
% H u_t+ \sum_{j=0}^D \omega_{t+j}^j A^{-j} C_j  u_t.
\ee
%with
%\be
%H=\sum_{j=0}^D A^{-j}B_j.\label{define-H}
%\ee
For notational convenience, we define
\bse\bee
\bar{\omega}_t\triangleq &\left(\omega_t^0~\omega_{t+1}^1\cdots \omega_{t+D}^D \right)',~t\geq0\\
\bar{\omega}_{-1}\triangleq &\left(\omega_1^0~\omega_{2}^0~\omega_{2}^1\cdots \omega_{D}^{0}\cdots\omega_{D}^{D-1} \right)',
%\mathcal{G}_t=&\sigma\{\bar{\omega}_s,s=-1,0,\cdots,t\}.
\eee\ese
which defines the $\sigma$-algebra as $\mathcal{G}_t=\sigma\{\bar{\omega}_s,s\in\mathbb{N}_{[-1,t]} \}$.
Different from the fact that $x_t\in\mathcal{F}_{t-1}$, the auxiliary state $\eta_t$ is $\mathcal{G}_{t-1}$-measurable, and the pre-defined $\sigma$-algebras satisfy
\be
\mathcal{F}_t \subset \mathcal{G}_t \subset \mathcal{F}_{t+D}.
\ee
Denote $\hat{\eta}_{t|s}={\bf E}\le[\eta_t|\mathcal{F}_{s} \ri]$.
It is of interest to point out that the stabilization of system (\ref{system01}) is equivalent to that of the delay-free system (\ref{system03}) where the control law is designed to be the feedback of $\hat{\eta}_{t|t-1}$ and $u_{t-\tau}$.


\begin{lemma}\label{lemma01}
System (\ref{system01}) is asymptotically mean-square stabilizable if and only if there exists a feedback control law
\be
u_{t}=L_0 \hat{\eta}_{t|t-1} +\sum_{\tau=1}^D L_\tau u_{t-\tau} \in \mathcal {U}_{ad},\label{input01}
\ee
such that the closed-loop auxiliary system in (\ref{system03}) is asymptotically mean-square stable.
\end{lemma}

\textbf{Proof.}
{\em Sufficiency.}
%Suppose there exists a control policy $u_t$ in \eqref{input01} such that system (\ref{system03}) is stable in the mean-square sense, i.e.
%\be
%\lim_{t\rightarrow \infty} {\bf E}\le(u_{t}'u_{t}\ri)=0, \lim_{t\rightarrow \infty} {\bf E}\le(\eta_{t}'\eta_{t}\ri)=0.
%\ee
First, taking the condition expectation on both sides of (\ref{system02}) w.r.t. $\mathcal{F}_{t-1}$ yields
\bee
\hat{\eta}_{t|t-1}=x_t+\sum_{\tau=1}^D \sum_{j=\tau}^D A^{\tau-j-1} B_j u_{t-\tau}.\label{system04}
\eee
The controller in (\ref{input01}) is equivalently expressed by
\beee
u_{t}=L_0 x_t+\sum_{\tau=1}^D \le(\sum_{j=\tau}^D  L_0A^{\tau-j-1} B_j + L_\tau \ri)u_{t-\tau}, \label{input02}
%=&~K_0 x_t+\sum_{\tau=1}^T K_\tau u_{t-\tau}. \no
\eeee
that belongs to the admissible control set $\mathcal {U}_{ad}$ in (\ref{ad-input-set}) with $K_0=L_0$ and $K_\tau=\sum_{j=\tau}^D L_0A^{\tau-j-1} B_j + L_\tau$.

In what follows, let us define
$\tilde{\eta}_{t|t-1}=\eta_t-\hat{\eta}_{t|t-1}.$
In this case, the orthogonality of $\hat{\eta}_{t|t-1}$ and $\tilde{\eta}_{t|t-1}$ can be obtained,
that is
\beee
{\bf E} \Big(\hat{\eta}_{t|t-1}'\tilde{\eta}_{t|t-1} \Big) =& {\bf E} \le[ {\bf E} \le(\hat{\eta}_{t|t-1}' (\eta_t-\hat{\eta}_{t|t-1}) | \mathcal{F}_{t-1} \ri) \ri]=0
%=& {\bf E} \Big(\hat{\eta}_{t|t-1}'\hat{\eta}_{t|t-1}\Big)- {\bf E} \Big(\hat{\eta}_{t|t-1}'\hat{\eta}_{t|t-1} \Big)  =0
%=&0,
\eeee
which leads to
%By utilizing, we obtain
\be
{\bf E}(\eta'_t \eta_t)= {\bf E} (\hat{\eta}_{t|t-1}'\hat{\eta}_{t|t-1})+{\bf E}(\tilde{\eta}_{t|t-1}' \tilde{\eta}_{t|t-1}).
\ee
Hence,
%Define $\bar{M}_{\tau}=\sum_{j=\tau}^T A^{\tau-j-1}B_j$ and $\bar{N}_\tau=\sum_{j=\tau}^T \omega^j_{t+j-\tau}\\A^{\tau-j-1}C_j$.
by (\ref{system04}), one obtains that
%\be
%x_t=\eta_t-\sum_{\tau=1}^D \sum_{j=\tau}^D A^{\tau-j-1} B_j u_{t-\tau}
%%\sum_{\tau=1}^T \le[ \sum_{j=\tau}^T A^{\tau-j-1} \left(B_j+\omega^j_{t+j-\tau}C_j \right) \ri]u_{t-\tau},
%%\sum_{\tau=1}^T \le(\bar{M}_{\tau}+ \bar{N}_\tau \ri) u_{t-\tau},
%\ee
%and
\bee\label{inequ}
&{\bf E}\le(x_{t}'x_{t}\ri) \leq
{\bf E}
\Bigg\{
\le(\eta_{t}'\eta_{t}\ri)^{\frac{1}{2}}
+\sum_{\tau=1}^D \Big[
u_{t-\tau}' \le( \sum_{j=\tau}^D A^{\tau-j-1} B_j \ri)' \no\\
&\times
\le( \sum_{j=\tau}^D A^{\tau-j-1} B_j\ri) u_{t-\tau} \Big]^{\frac{1}{2}} \Bigg\} ^2 \no\\
&\leq (D+1)\Bigg\{ {\bf E}\le(\eta_{t}'\eta_{t}\ri)
+\sum_{\tau=1}^D {\bf E} \Big[u_{t-\tau}' \le( \sum_{j=\tau}^D A^{\tau-j-1}B_j \ri)'
\no\\
&\times
\le( \sum_{j=\tau}^D A^{\tau-j-1}B_j \ri) u_{t-\tau}\Big] \Bigg\}.
%&+\sum_{\tau=1}^T {\bf E} \Big[u_{t-\tau}' \left(\sum_{j=\tau}^T \sigma_j^2 C_j'(A')^{\tau-j-1} A^{\tau-j-1}C_j \right) u_{t-\tau}  \Big]
\eee
%\bee
%&{\bf E}\le(x_{t}'x_{t}\ri) \no\\
%\leq &
%{\bf E}
%\Bigg\{
%\le(\eta_{t}'\eta_{t}\ri)^{\frac{1}{2}}
%+\sum_{\tau=1}^T \Big[
%u_{t-\tau}' \le( \sum_{j=\tau}^T A^{\tau-j-1} \left(B_j+\omega^j_{t+j-\tau}C_j \right) \ri)'  \no\\
%&\times  \le( \sum_{j=\tau}^T A^{\tau-j-1} \left(B_j+\omega^j_{t+j-\tau}C_j \right) \ri) u_{t-\tau} \Big]^{\frac{1}{2}} \Bigg\} ^2 \no\\
%\leq & (T+1)\Bigg\{ {\bf E}\le(\eta_{t}'\eta_{t}\ri)
%+\sum_{\tau=1}^T {\bf E} \Big[u_{t-\tau}' \le( \sum_{j=\tau}^T A^{\tau-j-1}B_j \ri)' \no\\
%&\times \le( \sum_{j=\tau}^T A^{\tau-j-1}B_j \ri) u_{t-\tau}\Big]  \no\\
%&+\sum_{\tau=1}^T {\bf E} \Big[u_{t-\tau}' \left(\sum_{j=\tau}^T \sigma_j^2 C_j'(A')^{\tau-j-1} A^{\tau-j-1}C_j \right) u_{t-\tau}  \Big] \Bigg\}.
%\eee
Since $\le( \sum_{j=\tau}^D A^{\tau-j-1}B_j \ri)'\le( \sum_{j=\tau}^D A^{\tau-j-1}B_j \ri)\geq 0$,
%and $C_j'(A')^{\tau-j-1} A^{\tau-j-1}C_j\geq 0$,
there exists a positive scalar $\gamma>0$ such that
\bee
{\bf E}\le(x_{t}'x_{t}\ri) \leq (D+1)\Bigg\{ {\bf E}\le(\eta_{t}'\eta_{t}\ri)
+\gamma\sum_{\tau=1}^D {\bf E} \left(u_{t-\tau}'u_{t-\tau}\right) \Bigg\}, \no
\eee
which implies that $\lim_{t\rightarrow \infty} {\bf E}\le(x_{t}'x_{t}\ri)=0$.

{\em Necessity.} This part can be derived from \eqref{inequ}.
\ok

Based on Lemma \ref{lemma01}, our objective in Problem 1 can be reformulated as seeking an admissible control law to stabilize auxiliary system \eqref{system03}.
Below, we propose the Riccati-type stabilization criterion.

\begin{theorem}\label{riccati-theorem}
System (\ref{system01}) is asymptotically mean-square stabilizable if and only if for any $Q>0$ and $R>0$, there exists a unique positive definite solution $Z>0$ satisfying the following nonlinear DDARE
\be
-Z+A'{Z}A+{Q}-A'ZL\Psi^{-1}L'{Z}A=0,\label{DARE-trans}
\ee
where
\bse
\bee
\Psi=& L'{Z}L+\sum_{\tau=0}^D \sigma_\tau^2 C_\tau' (A')^D {Z} A^D C_\tau+U_{{R},{Q}},  \label{DARE-trans-parameter-P} \\
L=&\sum_{j=0}^D A^{D-j}B_j, \label{DARE-trans-parameter-L}\\
U_{{R},{Q}}=&{R}+\sum_{\tau=1}^D \sum_{h=1}^\tau\sigma_\tau^2 C'_\tau (A')^{D-h} {Q} A^{D-h} C_\tau. \label{DARE-trans-parameter-V}
\eee
\ese
Moreover, the stabilizing control policy is given as
\bee
\hspace{-2.5mm}
u^*_t
%=&-\Psi^{-1}L'ZA^{D+1} \hat{\eta}_{t|t-1} \no\\
=-\Psi^{-1}L'ZA^{D} \left( \hspace{-1mm} Ax_t+\sum_{\tau=1}^D \sum_{j=\tau}^D A^{\tau-j} B_j u^*_{t-\tau}\right).
\eee
\end{theorem}

\textbf{Proof.}
See Appendix  \ref{app02}.
\ok

\begin{remark}
In Theorem 1, we have proposed the DDARE-based stabilization condition, which is necessary and sufficient.
However, since the developed Riccati equation is nonlinear, how to calculate the unique positive solution is challenging and to be solved.
\end{remark}

\subsection{Lyapunov-type Stabilization condition}
In this subsection, we propose a Lyapunov-type necessary and sufficient stabilization condition, which can be verified availably by the LMI feasibility test.
We reveal an interesting fact that the stabilization of system \eqref{system01} is equivalent to stabilizing another delay-free model of same dimensions
\be
\beta_{t+1}=A\beta_t+Lv_t+
\sum_{\tau=0}^D \omega^\tau_t A^DC_\tau v_{t},\label{system05}
\ee
where $L$ is defined in \eqref{DARE-trans-parameter-L}.
%the input control $v_t$ is designed to be the feedback of the state information, i.e..
We remark that system \eqref{system05} is said to be asymptotically mean-square stabilizable, if there exists a state feedback law $v_t=K\beta_t$ such that the following closed-loop system is stable \cite{zhouxy2000},
\be
\beta_{t+1}=(A+LK)\beta_t+
\sum_{\tau=0}^D \omega^\tau_t A^DC_\tau K\beta_t.\label{system05-close}
\ee

\begin{theorem}\label{Lyapunov-theorem}
The following statements are equivalent.\\
$a)$ System \eqref{system01} is asymptotically mean-square stabilizable. \\
$b)$ System \eqref{system05} is asymptotically mean-square stabilizable with $v_t=K\beta_t$. \\
$c)$ For any $Q>0$, there exist matrices $K$ and $P>0$ satisfying the following delay-dependent Lyapunov equation
\beee
P=Q+(A+LK)'P(A+LK)
+\sum_{\tau=0}^D \sigma^2_\tau K'C_\tau' (A')^D PA^DC_\tau K.
\eeee
$d)$ There exist matrices $K$ and $P>0$ satisfying the following delay-dependent Lyapunov inequality
\beee
P>& (A+LK)'P(A+LK)
+\sum_{\tau=0}^D \sigma^2_\tau K'C_\tau' (A')^D PA^DC_\tau K.\label{inequality}
\eeee
\end{theorem}

\textbf{Proof.}
Similar to Theorem 1 in \cite{zhouxy2000}, we have that $b) \Leftrightarrow c)$ and $c) \Leftrightarrow d)$.
Based on Theorem \ref{riccati-theorem}, we only need to prove that the stabilization of system \eqref{system05} is equivalent to DDARE \eqref{DARE-trans} has a unique positive solution.

$a)\Rightarrow b)$
Suppose system \eqref{system01} is stabilizable.
Let us define
\bee
V_t(\beta)= {\bf E} \le(\beta_t'Z\beta_t \ri),
\eee
where $Z>0$ is the unique positive solution of DDARE \eqref{DARE-trans}.
It follows that
\bee
&V_t(\beta)-V_{t+1}(\beta) ={\bf E} \Big\{\beta_t' Q \beta_t+v_t' U_{R,Q} v_t \no\\
& -\big( v_t+\Psi^{-1}L'PA {\beta}_{t} \big)'\Lambda\big( v_t+\Psi^{-1}L'PA {\beta}_{t}  \big) \Big\}.
\eee
%where
%\bse
%\bee
%Q_1=&\hat{Q}\succ0, \\
%R_1=&\sum_{\tau=1}^D \sum_{h=1}^\tau\sigma_\tau^2 C_\tau (A')^{D-h} \hat{Q} A^{D-h} C_\tau+\hat{R}\succ0.
%\eee
%\ese
Utilizing the control law $v_t=-\Psi^{-1}L'PA {\beta}_{t}$ yields that $V_t(\beta)-V_{t+1}(\beta)>0$, which indicates that system \eqref{system05} is stabilizable.

$b)\Rightarrow a)$
%To begin with,
Define the following finite horizon index function
\bee
J_T(\beta_0,v_t)
=\sum_{t=0}^T {\bf E}\le[\beta_t'Q\beta_t+v_t'U_{R,Q}v_t\ri].
\eee
By utilizing the matrix version of the maximum principle in \cite{zhouxy2002}, one obtains that the finite horizon LQ problem admits a unique optimal control
\be
v_t=-\Psi_{t+1}^{-1}L'P^T_{t+1}A\beta_t,
\ee
where the unique solution $P^T_t$ solves the following generalized difference Riccati equation (GDRE)
\bse
\bee
P^T_t=&A'P^T_{t+1} A+Q-A'P^T_{t+1} L\Psi_{t+1}^{-1}L'P^T_{t+1}A,\label{GDRE} \no\\
\Psi_{t+1}=& L'P^T_{t+1}L+\sum_{\tau=0}^D \sigma_\tau^2 C_\tau' (A')^DP^T_{t+1} A^D C_\tau+U_{R,Q}, \no
\eee
\ese
with the terminal condition $P^T_{T+1}=0$.
Similar to Theorem 1 in \cite{zhangwh2008}, one obtains that the stabilization of system \eqref{system05} guarantee the existence of the limit $P=\lim_{T\rightarrow\infty}P^T_t$.
Moreover, by taking the limit on both sides of \eqref{GDRE}, one obtains that $P>0$ is the unique positive solution satisfying the following GARE
\bee
P=A'P A+Q-A'P L\Psi^{-1}L'PA,
\eee
which is equivalent to DDARE \eqref{DARE-trans} with $Z=P$.
\ok


Utilizing Schur complement decomposition and matrix transformation technique, we give the following LMI-based criterion directly, which can be verified by some LMI solvers.

\begin{corollary}
System \eqref{system01} is stabilizable if and only if there exist matrices $Y$ and $S<0$ satisfying
\bee\label{lmi}
\left[
  \begin{array}{ccccc}
    S & * & * & \cdots & * \\
    AS+LY & S & * & \cdots & * \\
    \sigma_0 A^DC_0 Y & 0 & S & \cdots & * \\
    \vdots & \vdots & \vdots & \ddots & \vdots \\
    \sigma_D A^DC_D Y & 0 & 0 & \cdots & S \\
  \end{array}
\right]< 0,
\eee
where $Y=KP^{-1}$, $S=-P^{-1}<0$ and
$*$ represents the corresponding transpose part.
\end{corollary}


\begin{remark}
Specifically, let us set $B_i=C_i=0$, $i\in\mathbb{N}_{[0,D-1]}$. System \eqref{system01} is reduced to a single-delay stochastic system with control-dependent noise
\be
x_{t+1}=Ax_t+\le(B_D+\omega_t C_D \ri)u_{t-D}.\label{system01re}
\ee
Based on Theorem \ref{Lyapunov-theorem}, system \eqref{system01re} is stabilizable if and only if there exists a unique ${P}>0$ satisfying
\bee
P=Q+(A+LK)'P(A+LK)
+\sigma^2_D K'C_D' (A')^D PA^DC_D K,
\eee
which corroborates with the developed Lyapunov-type criterion in Theorem 2 in \cite{tan2019}.
\end{remark}


\section{Application}

This section focuses on the delay margin in the following restricted structure
\be
x_{t+1}=Ax_t+\omega_t^0 C_0 u_{t}+ \le(\bar{B} +\omega_t^D\bar{C} \ri)u_{t-D}, \label{system01single}
\ee
which will be described as $[A,C_0;\bar{B},\bar{C}]_D$ hereinafter.
The problem with our concern is to determine the largest delay range within which $[A,C_0;\bar{B},\bar{C}]_D$ is stabilizable.

Since for a stable system, an arbitrarily large delay margin can be achieved by zero input.
To proceed we state the following assumptions.

\noindent{\bf Assumption 1} $A$ is invertible and unstable. %and $\bar{B}$ has full-column rank.

\noindent{\bf Assumption 2} System $[A,C_0;\bar{B},\bar{C}]_0$ is stabilizable and the following delay-free system $(A,C_0)$ is destabilizable \cite{tan2022},
\be
x_{t+1}=Ax_t+\omega_t^0 C_0 u_{t}.
\ee
It is remarkable that Assumption 2 indicates that system $[A,C_0;\bar{B},\bar{C}]_D$ is stabilizable in delay-free case and it can not be stabilized when $D$ tends to infinity; See Fig. 2.

\begin{figure}[thpb]
      \centering
      \includegraphics[scale=0.38]{d-margin.jpg}
      \centering\caption{\scriptsize{Stabilization classification for system $[A,C_0;\bar{B},\bar{C}]_D$}}
\end{figure}

For the sake of discussion, we introduce the following nonlinear Riccati operator $\mathcal {R}^R_{D}(\cdot)$ as
\be
\mathcal {R}^R_{D}(Z)\triangleq  A'{Z}A+{Q}-A'Z\bar{B}\Psi^R_{D}(Z)^{-1}\bar{B}'{Z}A,\label{DARE-trans-operator}
\ee
where
\beee
\Psi^R_{D}(Z) =&\bar{B}'{Z}\bar{B}+\sigma_0^2 C_0' (A')^D {Z} A^D C_0+\sigma_D^2 \bar{C}' (A')^D {Z} A^D \bar{C}\no\\
&+\sum_{h=1}^D\sigma_D^2 \bar{C}' (A')^{D-h} {Q} A^{D-h} \bar{C}+R. \label{DARE-trans-parameter-P-up}
\eeee
Moreover, define an auxiliary Lyapunov operator $\mathcal {J}_K(\cdot)$
\beee
&\mathcal {J}_K(Z) \triangleq
(A+LK)'Z(A+LK) +\sum_{\tau=0}^D \sigma^2_\tau  K'C_\tau'(A')^D Z\no\\
&\times A^DC_\tau K+Q+K'(R+\sum_{h=1}^D\sigma_D^2 \bar{C}' (A')^{D-h} {Q} A^{D-h})K . %\label{lyapunov-J}
\eeee
Specifically, denote $K_Z=-\Psi^R_{D}(Z)^{-1}\bar{B}'{Z}A$, and the corresponding DDARE for $[A,C;\bar{B},\bar{C}]_D$ can be expressed as
\be
Z=\mathcal {R}^R_{D}(Z)=  \mathcal {J}_{K_Z}(Z). \label{lyapunov-J-2}
\ee
%and the stabilizing control policy is given as
%\bee\label{stable-input-02}
%u^*_t
%=-\Psi^{-1}\bar{B}'Z\left( A^{D+1}x_t+\sum_{\tau=1}^D A^{\tau} \bar{B} u^*_{t-\tau}\right).
%\eee
In addition, for any given $K$, using the completing square approach yields that
\bee
\mathcal {J}_{K}(Z)=&A'{Z}A+{Q}-A'ZL\Psi^{-1}L'{Z}A \no\\
&+(K-K_Z)'\Psi (K-K_Z)   \no\\
&\geq \mathcal {J}_{K_Z}(Z)=\mathcal {R}^R_{D}(Z).\label{lyapunov-J-3}
\eee


\begin{theorem}\label{existence-theorem}
Under Assumptions 1-2, there exists a unique delay margin $D_{\max}\geq 0$ such that $[A,C_0;\bar{B},\bar{C}]_D$ is asymptotically mean-square stabilizable for $D\in \mathbb{N}_{[0,D_{\max}]}$, and it can not be stabilized for each $D>D_{\max}$.
\end{theorem}

\textbf{Proof.}
To proceed we first show that the stabilization of system $[A,C_0;\bar{B},\bar{C}]_{D+1}$ can guarantee that of $[A,C_0;\bar{B},\bar{C}]_{D}$.
Select some positive definite matrices $R$ and $Q$ such that
\be
\hat{R}\triangleq R-\sigma_0^2 C_0' (A')^D Q A^D C_0>0.
\ee
For such $R>0$ and $Q>0$, there exists a unique positive solution $Z$ satisfying $Z=\mathcal {R}^R_{D+1}(Z)$.
It follows that
$Z\leq A'ZA+{Q}$,
which implies that
\bee
\sigma_0^2 C_0' (A')^D Z\ A^D C_0 -\sigma_0^2 C_0' (A')^D Q A^D C_0 \no\\
\leq
\sigma_0^2 C_0' (A')^{D+1} Z\ A^{D+1} C_0, \\
\sigma_D^2 \bar{C}' (A')^D Z\ A^D \bar{C} -\sigma_D^2 \bar{C}' (A')^D Q A^D \bar{C} \no\\
\leq
\sigma_D^2 \bar{C}' (A')^{D+1} Z\ A^{D+1}\bar{C}.
\eee
We further derive that $\Psi^R_{D+1}(Z)\geq \Psi^{\hat{R}}_{D}(Z).$
%\bee
%&\Psi^R_{D+1}(Z)  \no\\
%=& \bar{B}'Z\bar{B}+ \sigma_0^2 C_0' (A')^{D+1} Z A^{D+1} C_0+\sigma_D^2 \bar{C}' (A')^{D+1} Z \no\\
%&\times A^{D+1} \bar{C}+R+ \sum_{h=1}^{D+1}\sigma_D^2 \bar{C}' (A')^{D+1-h} {Q} A^{D+1-h} \bar{C} \no\\
%\geq &
%\bar{B}'Z\bar{B}+ \sigma_0^2 C_0' (A')^D Z A^D C_0 +\sigma_D^2 \bar{C}' (A')^D Z A^D \bar{C}+\hat{R}\no\\
%& +\sum_{h=1}^D\sigma_D^2 \bar{C}' (A')^{D-h} {Q} A^{D-h} \bar{C}
%=\Psi^{\hat{R}}_{D}(Z).
%\eee
Then, we can get that
$Z=\mathcal {R}^R_{D+1}(Z)\geq \mathcal {R}^{\hat{R}}_{D}(Z)$.
If we define $Z_{k+1}=\mathcal {R}^{\hat{R}}_D(Z_k)$ with $Z_0=Z$, it follows that
\be
Z_0\geq\mathcal {R}^{\hat{R}}_D(Z_0)=Z_1.
\ee
With the help of the pre-defined operator $\mathcal {J}_K(\cdot)$ and $K_{Z_k}=-\Psi^{\hat{R}}_{D}(Z_k)^{-1}\bar{B}'{Z_k}A$, we obtain that
\be
Z_{k+1}=\mathcal {R}^{\hat{R}}_{D}(Z_k)=\mathcal {J}_{K_{Z_k}}(Z_k)\geq Q>0.\label{iterative}
\ee
Then, it follows from \eqref{lyapunov-J-3} that
\bee
Z_1=\mathcal {J}_{K_{Z_0}}(Z_0)\geq\mathcal {J}_{K_{Z_0}}(Z_1)\geq\mathcal {J}_{K_{Z_1}}(Z_1)=Z_2,
\eee
%\bee
%Z_1=\mathcal {R}^{\hat{R}}_D(Z_0)=\mathcal {J}_{K_{Z_0}}(Z_0)
%\geq\mathcal {J}_{K_{Z_0}}(Z_1).
%\eee
%\bee
%Z_1\geq\mathcal {J}_{K_{Z_0}}(Z_1)\geq\mathcal {J}_{K_{Z_1}}(Z_1)=\mathcal {R}^{\hat{R}}_D(Z_1)=Z_2,
%\eee
which results in $Z_k\geq Z_{k+1}$ by induction.
i.e., $\{Z_k\}$ is a monotone decreasing and bounded sequence.
Therefore, there exists a $\hat{Z}=\lim_{k\rightarrow\infty}Z_k$ satisfying
\be
\hat{Z}=\lim_{k\rightarrow\infty}\mathcal {R}^R_{D}(Z_k)=\mathcal {R}^R_{D}(\hat{Z})
=\mathcal {J}_{K_{\hat{Z}}}(\hat{Z})>0,
\ee
which means that system $[A,C_0;\bar{B},\bar{C}]_{D}$ is stabilizable.
%based on Theorem \ref{riccati-theorem}.
% Corollary \ref{riccati-corollary}.

When $D=0$, system $[A,C_0;\bar{B},\bar{C}]_{0}$ is stabilizable on account of Assumption 2.
And when $D$ tends to $+\infty$, system $[A,C_0;\bar{B},\bar{C}]_{D}$ is equivalent to delay-free system $(A,C_0)$, that cannot be stabilized.
To sum up, the existence and uniqueness of delay margin $D_{\max}\geq 0$ is proposed.
\ok

%\begin{proof}
%See Appendix \ref{app03}.
%\end{proof}

\begin{remark}\label{remark-c}
The proof in Theorem \ref{existence-theorem} produces a numerically iterative algorithm for computing the definite positive solution to DDARE \eqref{lyapunov-J-2}.
Suppose system $[A,C_0;\bar{B},\bar{C}]_D$ is stabilizable.
For any initial value $Z_0\geq0$ satisfying $Z_0\geq\mathcal {R}^{{R}}_D(Z_0)$ with $Q>0$ and $R>0$, the solution satisfying the recurrence formula $Z_{k+1}=\mathcal {R}^{{R}}_D(Z_k)$ converges to the unique positive solution in \eqref{lyapunov-J-2}.
However, for general nonlinear DDARE \eqref{DARE-trans}, how to compute the value of the unique positive solution is
still unknown, which defines a challenging work direction.
\end{remark}

Below we study an uncoupled system, where
\bee
A=diag\{a_1,\cdots,a_n\},~C_0=diag\{c_1,\cdots,c_n\}, \no\\
\bar{B}=diag\{\bar{b}_1,\cdots,\bar{b}_n\},~\bar{C}=diag\{\bar{c}_1,\cdots,\bar{c}_n\}. \no
\eee
Hence, system $[A,C_0;\bar{B},\bar{C}]_{D}$ is stabilizable if and only if each scalar sub-system $[a_i,c_i;\bar{b}_i,\bar{c}_i]_{D}$ is stabilizable, and $D_{\max}$ satisfies
\be
D_{\max}=\min_{1\leq i\leq n}D^i_{\max},
\ee
where $D^i_{\max}$ is the delay margin for $[a_i,c_i;\bar{b}_i,\bar{c}_i]_{D}$.

Note that when $a_i^2<1$, the scalar sub-system $[a_i,c_i;\bar{b}_i,\bar{c}_i]_{D}$ is stabilizable for any bounded $D>0$.
To obtain a bounded delay margin, we assume that $A$ has at least one unstable eigenvalue $a^2_i\geq 1$ and $\bar{b}_j\neq 0$, $j\in\mathbb{N}_{[1,n]}$.
Denote $U_A$ to be the unstable eigenvalue set of the diagonal matrix $A$, i.e.,
\be
U_A\triangleq \{a_i:a^2_i\geq 1,i\in\mathbb{N}_{[1,n]}\}.
\ee
We are in a position to derive the exact delay margin.
%in terms of the system matrices and the probability statistics of multiplicative noises.

\begin{theorem}\label{delay-margin-theorem}
System $[A,C_0;\bar{B},\bar{C}]_{D}$ is stabilizable if and only if $0\leq D<D_{\max}$, where $D_{\max}$ is the  delay margin and satisfies the following conditions:
\\
a) If for each $a_i\in U_A$, $a_i^2=1$, then $D_{\max}=+\infty$.
\\
b) If for each $a_i\in U_A$, $a_i^2<\frac{\bar{b}^2_i}{\sigma_0^2 c^2_i+\sigma_D^2 \bar{c}^2_i}+1$,
and there exists a $a_j\in U_A$ such that $a_j^2>1$,
%there exists a scalar $a_j\in U(A)$ such that $1<a^2_j<{\rm SNR}_j+1$,
then
\beee
D_{\max}
=\min_{a_i\in U_A,a_i^2>1}\frac{\ln(\bar{b}^2_i)-\ln(\sigma_0^2 c^2_i+\sigma_D^2 \bar{c}^2_i)-\ln(a_i^2-1)}{\ln(a_i^2)}.
\eeee
c) If there exists a $a_i\in U_A$ satisfying $a_i^2\geq \frac{\bar{b}^2_i}{\sigma_0^2 c^2_i+\sigma_D^2 \bar{c}^2_i}+1$, then
$D_{\max}=0$.
\end{theorem}

\textbf{Proof.}
See Appendix  \ref{app03}.
\ok

\section{Simulation}

In this section, we give two examples to demonstrate our theoretical analysis.

\begin{example}
For simplicity, we consider the following system \eqref{system01} with $D=2$ and
\beee
&A=\left[ {\begin{array}{*{20}{c}}
1& 2&3\\
0&2 &2 \\
0&0&1
\end{array}} \right],
B_0=\left[ {\begin{array}{*{20}{c}}
2 &3  \\
1 &1  \\
1 &4
\end{array}} \right], \\
&B_1=\left[ {\begin{array}{*{20}{c}}
2 &4 \\
 2 &5 \\
 2 &2
\end{array}} \right],
B_2=\left[ {\begin{array}{*{20}{c}}
3 &4 \\
1 &3 \\
3 &5
\end{array}} \right], \\
&C_0=\left[ {\begin{array}{*{20}{c}}
 5     &5      \\
 40    &-3    \\
  3     &2
\end{array}} \right],
C_1=\left[ {\begin{array}{*{20}{c}}
2     &5      \\
2     &0      \\
4     &4
\end{array}} \right],
C_2=\left[ {\begin{array}{*{20}{c}}
0     &3 \\
2     &0 \\
1     &2 \\
\end{array}} \right],
\eeee
Assume $\sigma_\tau^2={\bf E}\le(\omega_t^\tau \omega_t^\tau\ri)=1,~\tau=0,1,2$,
and the initial condition is $x_0=[5~-3~10]'$ and $u_{-i}=[0~0]',~i=1,2$.
Utilizing the developed LMI in Corollary 1 yields the stabilizing feedback gain is
\beee
K=\left[ {\begin{array}{*{20}{c}}
   -0.0001 &   -0.0028  & -0.0105 \\
    0.0004  &  0.0132  &  0.0571
\end{array}} \right].
\eeee
In this case, the control policy $u_t=K\hat{\eta}_{t|t-1}$ can stabilize system \eqref{system01} as shown in Fig. 3.\ok

\begin{figure}[thpb]
      \centering
      \includegraphics[scale=0.3]{simu-1.jpg}
      \centering\caption{\scriptsize{State response of system (1) in Example 1}}
\end{figure}

\end{example}

\begin{example}
Consider system $[A,C_0;\bar{B},\bar{C}]_2$ with the following parameters
\beee
A=\left[ {\begin{array}{*{20}{c}}
1.1& 0\\
0&1.2
\end{array}} \right],
C_0=\left[ {\begin{array}{*{20}{c}}
1& -0.5\\
0& 1.25
\end{array}} \right], \\
\bar{B}=\left[ {\begin{array}{*{20}{c}}
2& 1\\
1&-1
\end{array}} \right],
\bar{C}=\left[ {\begin{array}{*{20}{c}}
2& 0\\
0&3
\end{array}} \right].
\eeee
Assume $\sigma_\tau^2={\bf E}\le(\omega_t^\tau \omega_t^\tau\ri)=1,~\tau=0,2$,
and the initial condition is $x_0=[1~1]'$ and $u_{-i}=[0~0]',~i=1,2$.
Utilizing the iterative algorithm proposed in Remark \ref{remark-c} yields that the unique positive solution to DDARE \eqref{lyapunov-J-2} is
\beee
Z=
\left[ {\begin{array}{*{20}{c}}
82.7362& -257.7524\\
 -257.7524 & 859.1263
\end{array}} \right]>0.
\eeee
In this case, the stabilizing control policy in Theorem \ref{riccati-theorem} is
%can be reduced to
\beee
u^*_t
=-\Psi^{-1}\bar{B}'Z\left( A^{D+1}x_t+\sum_{\tau=1}^D A^{\tau} \bar{B} u^*_{t-\tau}\right),
\eeee
which stabilizes the considered system as shown in Fig. 4.\ok
\begin{figure}[thpb]
      \centering
      \includegraphics[scale=0.3]{simu-2.jpg}
      \centering\caption{\scriptsize{State response of system $[A,C_0;\bar{B},\bar{C}]_2$ in Example 2}}
\end{figure}

\end{example}

\section{Conclusion}
In this paper, we are concerned with the stabilization problem of multi-delay stochastic system with multiplicative noises in control variables.
By utilizing a novel reduction method, we derive a set of necessary and sufficient conditions for stabilizing such systems in terms of DDARE and DDLE.
Under some basic assumptions, we prove that the unique delay margin exists, and further propose its explicit computation formula for uncoupled system.
%A promising future direction is applying the developed method to deal with continuous-time multi-delay dynamics.


%%%%%%%%%%%%%%%%%%%%%%%%%%%%%%%%%%%%%%%%%%%%%%%%%%%%%%%%%%%%%%%%%%%%%%%%%%%%%%%%%%%%%%%%%%%%%%%%%%%%%%%%%%%%%%%%%%%%%%%%%%%%%%%%%%%%%%%%%%%

%\section*{Appendix}
\appendices
%\section{Proof of Lemma \ref{lemma01}}\label{app01}

\section{Proof of Theorem \ref{riccati-theorem}}\label{app02}
\textbf{Proof.}
First, we give an equivalent form of DDARE \eqref{DARE-trans}.
Define $P=(A')^{D}ZA^{D}$ and $\hat{Q}=(A')^D{Q}A^D$.
Pre-multiplying $(A')^{D}$ and post-multiplying $A^{D}$ on both sides of DDARE \eqref{DARE-trans}, we obtain that
\bee
0=&(A')^{D}(-Z+A'{Z}A+Q)A^{D}-A'(A')^{D}ZA^{D} \no\\
&\times\left(\sum_{j=0}^D A^{-j}B_j\right)\Psi^{-1} \left(\sum_{j=0}^D A^{-j}B_j\right)(A')^{D}ZA^{D}A \no\\
=&-P+A'PA+\hat{Q}-A'PH\Lambda^{-1}H'PA, \label{DARE}
\eee
where
\bse
\bee
\Lambda=& H'PH+\sum_{\tau=0}^D \sigma_\tau^2 C_\tau'PC_\tau+W_{\hat{R},\hat{Q}}, \label{dare-parameter-L} \\
H=&\sum_{j=0}^D A^{-j}B_j, \label{dare-parameter-H} \\
W_{R,\hat{Q}}=&R+\sum_{\tau=1}^D \sum_{h=1}^\tau\sigma_\tau^2 C_\tau (A')^{-h}\hat{Q}A^{-h} C_\tau.  \label{dare-parameter-W}
\eee
\ese
Below, we prove that system \eqref{system03} is stabilizable if and only if DDARE \eqref{DARE} has a unique positive definite solution.

{\em Sufficiency.}
Due to the equivalent proposition between DDAREs \eqref{DARE-trans} and \eqref{DARE}, we are in a position to construct a Lyapunov function to guarantee $\lim_{t\rightarrow \infty} {\bf E}\le(\eta_{t}'\eta_{t}\ri)=0$
with $u_t=-\Lambda^{-1}H'PA \hat{\eta}_{t|t-1}$.

Step 1:
Based on DDARE \eqref{DARE}, define the following delay-dependent Lyapunov function
\bee
&V_t(\eta)={\bf E} \bigg\{\eta_t' \Big[ (A')^D PA^D+\sum_{\tau=0}^{D-1} (A')^\tau \hat{Q}A^\tau \Big] \eta_t \no\\
&-\sum_{\tau=0}^{D-1} \hat{\eta}_{t|t+\tau-1}' (A')^{\tau+1}PH \Lambda^{-1} H'PA^{\tau+1} \hat{\eta}_{t|t+\tau-1}
\bigg\}.
\eee
The orthogonality of $\hat{\eta}_{t|t+\tau-1}$ and $\tilde{\eta}_{t|t+\tau-1}$ yields that
\beee
&V_t(\eta)
%=& {\bf E} \bigg\{\eta_t' \Big[ (A')^D PA^D+\sum_{\tau=0}^{D-1} ( (A')^\tau \hat{Q}A^\tau- (A')^{\tau+1} \no\\
%&\times PH \Lambda^{-1} H'PA^{\tau+1} )\Big] \eta_t+\sum_{\tau=0}^{D-1} \tilde{\eta}_{t|t+\tau-1}' \no\\
%& \times(A')^{\tau+1}PH \Lambda^{-1} H'PA^{\tau+1} \tilde{\eta}_{t|t+\tau-1}
%\bigg\} \no\\
= {\bf E} \bigg\{\eta_t' \Big[ (A')^D PA^D+\sum_{\tau=0}^{D-1} ( (A')^\tau \hat{Q}A^\tau + (A')^{\tau}
\no\\
&\times
\le(P-A'PA-\hat{Q} \ri)  A^{\tau} )\Big] \eta_t \no \\
&+\sum_{\tau=0}^{D-1} \tilde{\eta}_{t|t+\tau-1}'  (A')^{\tau+1}PH \Lambda^{-1} H'PA^{\tau+1} \tilde{\eta}_{t|t+\tau-1}
\bigg\} \no\\
=& {\bf E} \bigg\{\eta_t'P\eta_t+\sum_{\tau=0}^{D-1} \tilde{\eta}_{t|t+\tau-1}' (A')^{\tau+1}PH \Lambda^{-1} H'PA^{\tau+1} \tilde{\eta}_{t|t+\tau-1}
\bigg\}.
\eeee
Because of $P>0$ and $(A')^{\tau+1}PH \Lambda^{-1} H'PA^{\tau+1}\geq 0$, it follows that $V_t(\eta)>0$ for any $\eta_t\neq 0$.

Step 2: Now move to calculate the difference of $V_{t}(\eta)$ and $V_{t+1}(\eta)$ as follows
\beee\label{lyapunov02}
&V_{t}(\eta)-V_{t+1}(\eta) \no\\
=& {\bf E} \Bigg\{ \eta_t' \bigg[ (A')^D PA^D+\sum_{\tau=0}^{D-1} (A')^\tau \hat{Q}A^\tau \bigg] \eta_t
\no\\
&
-\sum_{\tau=0}^{D-1} \hat{\eta}_{t|t+\tau-1}' (A')^{\tau+1}PH \Lambda^{-1} H'PA^{\tau+1} \hat{\eta}_{t|t+\tau-1} \no\\
&+\sum_{\tau=0}^{D-1} \hat{\eta}_{t+1|t+\tau}' (A')^{\tau+1}PH \Lambda^{-1} H'PA^{\tau+1} \hat{\eta}_{t+1|t+\tau}
\no\\
&-\eta_{t+1}' \bigg[ (A')^DPA^D+\sum_{\tau=0}^{D-1} (A')^\tau \hat{Q}A^\tau \bigg] \eta_{t+1} \Bigg\}.
\eeee
Since $\eta_t$ is $\mathcal{G}_{t-1}$-measurable and $\mathcal{G}_{t-1} \subset \mathcal{F}_{t+D-1}$, we have
$\hat{\eta}_{t|t+D-1}={\bf E} [\eta_t|\mathcal{F}_{t+D-1}]=\eta_t.$
%Substituting (\ref{system02}) into (\ref{lyapunov02}) yields that
%\beee
%&V_{t}(\eta)-V_{t+1}(\eta) \no\\
%=& {\bf E} \Bigg\{  \eta_t' \bigg[ (A')^DPA^D+\sum_{\tau=0}^{D-1} (A')^\tau \hat{Q} A^\tau-(A')^{D+1}PA^{D+1} \no\\
%&- \sum_{\tau=0}^{D-1} (A')^{\tau+1} \hat{Q}A^{\tau+1} \bigg]   \eta_t
%-u_t'  \bigg[ H'(A')^D P A^DH   \no\\
%&+\sum_{\tau=0}^{D-1}H'(A')^{\tau}\hat{Q}A^{\tau}H
%+\sum_{j=0}^{D}\sigma_j^2C_j' \Big( (A')^{D-j}PA^{D-j} \no\\
%&+\sum_{\tau=0}^{D-1}(A')^{\tau-j}\hat{Q}A^{\tau-j} \Big)C_j \no\\
%&+\sum_{\tau=0}^{D-1} \Big( H'(A')^{\tau+1}PH \Lambda^{-1} H'PA^{\tau+1}H \no\\
%&+\sum_{j=0}^\tau \sigma_j^2C_j(A')^{\tau-j+1}PH\Lambda^{-1}H'PA^{\tau-j+1}C_j
% \Big) \bigg]u_t \no\\
%&+\sum_{\tau=0}^{D-1} \hat{\eta}_{t|t+\tau}' (A')^{\tau+2}PH\Lambda^{-1}H'PA^{\tau+2}\hat{\eta}_{t|t+\tau} \no\\
%&-\sum_{\tau=0}^{D-1} \hat{\eta}_{t|t+\tau-1}' (A')^{\tau+2}PH\Lambda^{-1}H'PA^{\tau+2} \hat{\eta}_{t|t+\tau-1} \no\\
%&+\sum_{\tau=0}^{D-1} 2\hat{\eta}_{t|t+\tau}' (A')^{\tau+2}PH\Lambda^{-1}H'PA^{\tau+1}Hu_{t} \no\\
%&-\sum_{\tau=0}^{D-1}2\eta_t A\Big( (A')^{D}PA^{D}+\sum_{\tau=0}^{D-1} (A')^{\tau}\hat{Q}A^{\tau} \Big) H u_t
%\Bigg\}.
%\eeee
%Consider the part of ${\bf E}(\eta_t'\Omega_1\eta_t)$
Utilizing the fact that $A'PH\Lambda^{-1}H'PA=-P+A'PA+\hat{Q}$ with (\ref{system02}) yields that
\bee
&V_{t}(\eta)-V_{t+1}(\eta) \no\\
=& {\bf E} \Bigg\{  \eta_t' \bigg[ (A')^DPA^D+\sum_{\tau=0}^{D-1} (A')^\tau \hat{Q}A^\tau-(A')^{D+1}PA^{D+1}
\no\\
&- \sum_{\tau=0}^{D-1} (A')^{\tau+1} \hat{Q}A^{\tau+1}
+(A')^D( -P+A'PA+\hat{Q} ) A^D  \bigg]   \eta_t
\no\\
&-u_t'  \bigg[ H'(A')^DPA^DH +\sum_{\tau=0}^{D-1}H'(A')^{\tau}\hat{Q}A^{\tau}H \no\\
&+\sum_{\tau=0}^{D-1}  H'(A')^{\tau}(-P+A'PA+\hat{Q})A^{\tau}H
\no\\
&
+\sum_{j=0}^{D}\sigma_j^2C_j' \Big( (A')^{D-j}PA^{D-j}+\sum_{\tau=0}^{D-1}(A')^{\tau-j}\hat{Q}A^{\tau-j} \Big)C_j \no\\
&+\sum_{\tau=0}^{D-1}\sum_{j=0}^\tau \sigma_j^2C_j(A')^{\tau-j}(-P+A'PA+\hat{Q})A^{\tau-j}C_j\bigg]u_t
\no\\
&- \hat{\eta}_{t|t-1}' A'PH\Lambda^{-1}H'PA \hat{\eta}_{t|t-1} \Bigg\} \no\\
&+ {\bf E} \Bigg\{ {\bf E} \bigg[
\sum_{\tau=0}^{D-1} 2\hat{\eta}_{t|t+\tau}' (A')^{\tau+1} (-P+A'PA+\hat{Q}) A^{\tau}Hu_{t} \no\\
&-\sum_{\tau=0}^{D-1}2\eta_t A\Big( (A')^{D}PA^{D}+\sum_{\tau=0}^{D-1} (A')^{\tau}\hat{Q}A^{\tau} \Big) H u_t| \mathcal{F}_{t-1}\bigg] \Bigg\} \no\\
=&{\bf E} \Bigg\{\eta_t'\hat{Q}\eta_t-u_t' \bigg[H'PH+\sum_{\tau=0}^D \sigma_\tau^2 C_\tau'PC_\tau
\no\\
& +\sum_{\tau=1}^D \sigma_\tau^2 C_\tau\left(\sum_{j=-\tau}^{-1} (A')^{j}\hat{Q}A^{j} \right)C_\tau \bigg] u_t \no\\
&-\hat{\eta}_{t|t-1}' A'PH\Lambda^{-1}H'PA\hat{\eta}_{t|t-1}-2\hat{\eta}_{t|t-1}' A'PHu_{t} \Bigg\} \no\\
%=&{\bf E} \bigg\{\eta_t'\hat{Q}\eta_t+u_t'R u_t
%-u_t'\Lambda u_t -2\hat{\eta}_{t|t-1}' A'PHu_{t} \no\\
%&
%-\hat{\eta}_{t|t-1}' A'PH\Lambda^{-1}H'PA\hat{\eta}_{t|t-1} \bigg\} \no\\
=&{\bf E} \bigg\{\eta_t'\hat{Q}\eta_t+u_t'R u_t-\Big( u_t+\Lambda^{-1}H'PA \hat{\eta}_{t|t-1} \Big)'\Lambda \no\\
&\times
\Big( u_t+\Lambda^{-1}H'PA \hat{\eta}_{t|t-1} \Big) \bigg\}.
\eee
Applying
$u_t^*=-\Lambda^{-1}H'PA \hat{\eta}_{t|t-1}$
leads to
\be
V_{t}(\eta)-V_{t+1}(\eta) = {\bf E} \Big[\eta_t'\hat{Q}\eta_t+u_t'R u_t\Big] > 0.
\ee
By means of Lyapunov stability theory and Lemma \ref{lemma01}, we obtain that system \eqref{system03} is mean-square stabilizable while the stabilizing control input is
\beee
&u_t^*=-\Psi^{-1}L'ZA^{D+1} \hat{\eta}_{t|t-1} \\
=&-\Psi^{-1}L'ZA^{D+1} x_t-\sum_{\tau=1}^D \sum_{j=\tau}^D \Psi^{-1}L'ZA^{D+\tau-j} B_j u^*_{t-\tau}.
\eeee


{\em Necessity.}
Suppose system (\ref{system01}) is stabilizable. When $\hat{Q}>0$, the following deterministic system $(A;\hat{Q}^{\frac{1}{2}})$
\bee
x_{t+1}=Ax_t,~y_t=\hat{Q}^{\frac{1}{2}}x_t\label{observe-system}
\eee
is evidently observable. It follows from Theorem 4.2 in \cite{lilin2016}, there exist unique solutions $P_i,~i\in\mathbb{N}_{[1,D+1]}$ satisfying the following coupled algebraic Riccati equations
\bse\label{couple-riccati}
\bee
P_1=& A'P_1A+A'P_{D+1}A+\hat{Q}, \label{couple-riccati-1} \\
P_i=&-(A')^{i-2} M'\Upsilon M A^{i-2},~i\in \mathbb{N}_{[2,D+1]} \label{couple-riccati-i}\\
M=&\sum_{i=1}^{D+1}H'P_iA, \label{couple-riccati-m} \\
\Upsilon =& R+\sum_{i=1}^{D+1}H'P_iH+\sum_{j=0}^{D}\sigma^2_j H_j' P_1 H_j+\sum_{i=2}^{D+1}\sum_{j=0}^{i-2} \sigma^2_j H_j' P_i H_j  \label{couple-riccati-U}\\
H_j=&A^{-j}C_j, ~j\in \mathbb{N}_{[0,D]}
\eee\ese
where $\sum_{i=1}^{D+1}P_i>0$.
In what follows, we simplify the coupled algebraic Riccati equations in \eqref{couple-riccati} to DDARE \eqref{DARE}.
Let us define $P=\sum_{i=1}^{D+1}P_i>0,$
which implies that $M=H'PA$.
From \eqref{couple-riccati-i}, we have
\be
P_2=-M'\Upsilon M,~P_i=A'P_{i-1} A,~i\in \mathbb{N}_{[3,D+1]}. \label{couple-riccati-i+1}
\ee
Taking the sum from $P_1$ to $P_{D+1}$ on both sides of \eqref{couple-riccati-1} and \eqref{couple-riccati-i+1} yields
\bee
P=&A'(P_1+P_2+\cdots+P_{D+1})A+\hat{Q}+P_2  \no\\
=&A'PA+\hat{Q}-M'\Upsilon M.\label{reduce-riccati-P}
\eee
In this case, we have
\beee
P_1=&P-\sum_{i=2}^{D+1}(A')^{i-2} (P-A'PA-\hat{Q}) A^{i-2} \\
=&(A')^DPA^D+\sum_{i=0}^{D-1}(A')^{i}\hat{Q} A^{i},\\
P_i=&(A')^{i-2} (P-A'PA-\hat{Q}) A^{i-2},~i\in \mathbb{N}_{[2,D+1]},
\eeee
%and
%\bee
%P_1=&P-\sum_{i=2}^{D+1}(A')^{i-2} (P-A'PA-\hat{Q}) A^{i-2} \no\\
%=&(A')^DPA^D+\sum_{i=0}^{D-1}(A')^{i}\hat{Q} A^{i}.
%\eee
The parameter $\Upsilon$ satisfies
\bee
\Upsilon
%=& R+H'PH+\sum_{j=0}^{D}\sigma^2_j C_j' (A')^{D-j}PA^{D-j}C_j\no\\
%&+\sum_{j=0}^{D}\sum_{i=0}^{D-1}\sigma^2_j C_j' (A')^{i-j}\hat{Q} A^{i-j} C_j \no\\
%&+\sum_{i=2}^{D+1}\sum_{j=0}^{i-2} \sigma^2_j C_j'(A')^{i-j-2} (P-A'PA-\hat{Q}) A^{i-j-2} C_j \no\\
=&R+H'PH+\sum_{j=0}^{D}\sigma^2_j C_j' (A')^{D-j}PA^{D-j}C_j\no\\
&+\sum_{j=0}^{D-1}\sum_{i=0}^{D-1-j} \sigma^2_j C_j'(A')^{i} (P-A'PA) A^{i} C_j \no\\
&+\sum_{j=0}^{D}\sum_{i=0}^{D-1}\sigma^2_j C_j' (A')^{i-j}\hat{Q} A^{i-j} C_j \no\\
&-\sum_{j=0}^{D-1}\sum_{i=0}^{D-1-j} \sigma^2_j C_j'(A')^{i}\hat{Q} A^{i} C_j \no\\
=& W_{R,\hat{Q}}+H'PH+\sum_{j=0}^D \sigma_j^2 C_j'PC_j,
%+\sum_{j=1}^D \sum_{i=1}^j\sigma_j^2 C_j (A')^{-i}\hat{Q}A^{-i} C_j,
\eee
which is $\Lambda$ in \eqref{dare-parameter-L}.
This proof is complete.
\ok


\section{Proof of Theorem \ref{delay-margin-theorem}}\label{app03}

\textbf{Proof.}
%The proof is similar to that of Theorem 2 in \cite{tan2015} and we only provide a sketch.
System $[A,C_0;\bar{B},\bar{C}]_{D}$ is stabilizable if and only if there exists a unique positive solution $z_i>0$ satisfying
\be
z_i=\mathcal {R}^{r_i}_{D}(z_i)=a^2_i z_i+{q_i}-a^2_i\bar{b}_i^2 \Psi^{-1}_i z^2_i,\label{DARE-scalar}
\ee
where $q_i>0,$ $r_i>0$, and
%the parameter $\Psi_i$ satisfies
\bee
\Psi_i =&\le(\bar{b}^2_i+ \sigma_0^2 c^2_i a^{2D}_i +\sigma_D^2 \bar{c}^2_i a_i^{2D} \ri) z_i
\no\\
&
+ \le(r_i+ \sum_{h=1}^D\sigma_D^2 \bar{c}_i^2 a_i^{2(D-h)}q_i\ri)>0.
\eee
%where
%\bee
%\phi_i=&\bar{b}^2_i+ \sigma_0^2 c^2_i a^{2D}_i +\sigma_D^2 \bar{c}^2_i a_i^{2D}>0, \\
%\varphi_i=&r_i+ \sum_{h=1}^D\sigma_D^2 \bar{c}_i^2 a_i^{2(D-h)}q_i\geq r_i>0.
%\eee
Then,
DDARE \eqref{DARE-scalar} can be rewritten as a quadratic form
\be
c_{i2}z_i^2+c_{i1}z_i+c_{i0}=0, \label{quad}
\ee
where the parameters are
\bee
c_{i2}=&(a_i^2-1)c_i-a_i^2\bar{b}_i^2,
c_{i1}=(a_i^2-1)c_{i0} +c_i, \no\\
c_{i0}=&r_i+ \sum_{h=1}^D\sigma_D^2 \bar{c}_i^2 a_i^{2(D-h)}q_i, \no\\
c_i=&\bar{b}^2_i+ \sigma_0^2 c^2_i a^{2D}_i +\sigma_D^2 \bar{c}^2_i a_i^{2D} \no.
\eee
Similar to the proof of Theorem 2 in \cite{tan2015}, the above quadratic equation \eqref{quad} has a unique positive solution if and only if $c_{i2}<0$, i.e.,
%the following algebraic inequality holds,
\be
(a_i^2-1)\le( \bar{b}^2_i+ \sigma_0^2 c^2_i a^{2D}_i +\sigma_D^2 \bar{c}^2_i a_i^{2D} \ri)
-a_i^2\bar{b}_i^2<0,
\ee
which is equivalent to
\be
a^{2D}_i (\sigma_0^2 c^2_i+\sigma_D^2 \bar{c}^2_i)(a_i^2-1)<\bar{b}^2_i.\label{algebraic-inq}
\ee
When $a_i^2\leq 1$, inequality \eqref{algebraic-inq} holds for any bounded delay $D_i>0$, and hence, sub-system $[a_i,c_i;\bar{b}_i,\bar{c}_i]_{D}$ is stabilizable for $D<D^i_{\max}=+\infty$.
When $a_i^2> 1$, it follows that
% from \eqref{algebraic-inq} that
\be
a_i^2-1<a^{2D}_i (a_i^2-1)< \frac{\bar{b}^2_i}{\sigma_0^2 c^2_i+\sigma_D^2 \bar{c}^2_i}=h_i.
\ee
The delay margin of $[a_i,c_i;\bar{b}_i,\bar{c}_i]_{D}$ is derived as
\beee
D^i_{\max}=\left\{
\begin{array}{lr}
\frac{\ln(h_i)-\ln(a_i^2-1)}{2\ln(a_i)},&{\rm if}~1<a_i^2<h_i+1, \\
0,&{\rm if}~a_i^2\geq h_i+1.
\end{array}
\right.
\eeee
%On the other hand,
%the delay margin $D_{\max}$ of system $[A,C_0;\bar{B},\bar{C}]_{D}$ satisfies
%\be
%D_{\max}=\min_{1\leq i\leq n}D^i_{\max}.
%\ee
To sum up, if $a_i^2=1$ holds for each $a_i\in U_A$, system $[A,C_0;\bar{B},\bar{C}]_{D}$ is stabilizable for $D\geq 0$,
and if $a_i^2 \geq h_i+1$ for some $a_i\in U_A$, system $[A,C_0;\bar{B},\bar{C}]_{D}$ cannot be stabilized for any $D> 0$.
Otherwise, we obtain that
\be
D_{\max}=\min_{a^2_i>1}\frac{\ln(h_i)-\ln(a_i^2-1)}{\ln(a_i^2)},~a_i\in U_A,
\ee
which completes the proof.
\ok
%%%%%%%%%%%%%%%%%%%%%%%%%%%%%%%%%%%%%%%%%%%%%%%%%%%%%%%%%%%%%%%%%%%%%%%%%%%%%%%%%%%%%%%%%%%%%%%%%%%%%%%%%%%%%%%%%%%%%%%%%%%%%%%%%%%%%%%%%%%%%%%%%%


\begin{thebibliography}{}

\bibitem{A2012}
Saberi. A., Stoorvogel. A., Sannuti. P., Internal and External Stabilization of Linear Systems with Constraints. Springer, 2012.

\bibitem{zhouxy2000}
Ait Rami. M., Zhou. X.Y., Linear matrix inequalities, Riccati equations and indefinite stochastic linear quadratic controls, IEEE Transactions on Automatic Control, 2000, 45(6): 1131--1143.

\bibitem{zhangwh2008}
Zhang. W., Zhang. H., Chen B.C., Generalized
Lyapunov equation approach to state-dependent stochastic stabilization/detectability criterion, IEEE Transactions on Automatic Control, 2008, 53(7): 1630--1642.

\bibitem{huang2008}
Huang. Y., Zhang. W., Zhang. H., Infinite horizon linear quadratic optimal control for discrete-time stochastic systems, Asian Journal of Control, 2008, 10(5): 608--615.

\bibitem{zhangwh2004}
Zhang. W., Chen. B., On stabilizability and exact observability of stochastic systems with their applications, Automatica, 2004, 40(1): 87--94.

\bibitem{wang2018}
Wang. B., Zhu. Q., Stability analysis of semi-Markov switched stochastic systems, Automatica, 2018, 94: 72-80.

\bibitem{wang2021}
Wang. P., Wang. S., Su. H., Stochastic input-to-state stability of impulsive stochastic nonlinear systems in infinite dimensions, SIAM Journal on Control and Optimization, 2021, 59(4): 2774-2797.

\bibitem{qiqy2018}
Zhang. H., Qi. Q., Fu. M., Optimal stabilization control for discrete-time mean-field stochastic systems, IEEE Transactions on Automatic Control, 2018, 64(3): 1125--1136.

\bibitem{tan2022}
Tan. C., Di. J., Xiang. M., Chen. Z., Zhu B., Stochastic stabilization for discrete-time system with input delay and multiplicative noise in control variable, Processes, 2022, 10(5): 989.

\bibitem{deng2022}
Luo. S., Deng. F., Yu. X., Unified stability analysis for Ito stochastic systems: from almost surely asymptotic to finite-time convergence, IEEE Transactions on Automatic Control, 2022, 67(1): 406--412.
%%%%%%%%%%%%%%%%%%%%%%%%%%%%%%%%%%%%%%%%%%%%%
\bibitem{yued2005}
Yue. D., Han. Q. L., Delay-dependent exponential stability of stochastic systems with time-varying delay, nonlinearity, and Markovian switching, IEEE Transactions on Automatic Control, 2005, 50(2): 217--222.

\bibitem{mao2009}
Huang. L., Mao. X., Robust delayed-state-feedback stabilization of uncertain stochastic systems, Automatica, 2009, 45(5): 1332--1339.

\bibitem{lilin2015}
Zhang. H., Li. L., Xu. J., Fu. M., Linear quadratic regulation and stabilization of discrete-time systems with delay and multiplicative noise, IEEE Transactions of Automatic Control, 2015, 60(10): 2599--2613.

\bibitem{lilin2016}
Li. L., Zhang. H., Stabilization of discrete-time systems with multiplicative noise and multiple delays in the control variable, SIAM Journal on Control and Optimization, 2016, 54(2): 894--917.

\bibitem{lilin2021}
Li. L., Zhang. H., Wang. Y., Stabilization and optimal control of discrete-time systems with multiplicative noise and multiple input delays, Systems $\&$ Control Letters, 2021, 147: 104833.

\bibitem{tan2015}
Tan. C., Li. L., Zhang. H., Stabilization of networked control systems with network-induced delay and packet dropout, Automatica, 2015, 59: 194--199.

\bibitem{qinglong2019}
Zhang. X. M., Han. Q. L., Ge. X., Ding. D., Ding. L., Yue. D., Peng. C., Networked control systems: A survey of trends and techniques, IEEE/CAA Journal of Automatica Sinica, 2019, 7(1): 1-17.

\bibitem{tan2019}
Tan. C., Yang. L., Zhang. F., Zhang. Z., Wong. W.S., Stabilization of discrete time stochastic system with input delay and control dependent noise. Systems $\&$ Control Letters, 2019, 123: 62--68.

\bibitem{margin01}
Middleton. R. H., Miller. D. E., On the achievable delay margin using LTI control for unstable plants, IEEE Transactions on Automatic Control, 2007, 52(7): 1194--1207.

\bibitem{margin02}
Ma. D., Chen. J., Delay margin of low-order systems achievable by PID controllers, IEEE Transactions on Automatic Control, 2018, 64(5): 1958--1973.

\bibitem{margin03}
Zong. X., Li. T., Yin. G., Zhang. J., Delay tolerance for stable stochastic systems and extensions, IEEE Transactions on Automatic Control, 2020, 66(6): 2604--2619.

\bibitem{art1982}
Artstein. Z., Linear systems with delayed controls: A reduction, IEEE Transactions of Automatic Control, 1982, 27(4): 869--879.

%\bibitem{mpc2016}
%Mesbah. A., Stochastic model predictive control: An overview and perspectives for future research, IEEE Control System Magazine, 2016, 36(6): 30--44.

\bibitem{Ito1981}
Watanabe, K., Ito, M., A process-model control for linear systems with delay, IEEE Transactions on Automatic Control, 1981, 26(6), 1261--1269.

\bibitem{zhouxy2002}
Ait Rami. M., Chen. X., Zhou. X.Y., Discrete-time indefinite LQ control with state and control dependent noises, Journal of Global Optimization, 2002, 23: 245--265.

\bibitem{multi01}
Sakai. K., Sun. M. T., Ku. W. S., Wu. J., Lai. T. H., Secure data communications in wireless networks using multi-path avoidance routing, IEEE Transactions on Wireless Communications, 2019, 18(10), 4753-4767.

\bibitem{multi02}
Tan. C, Wong. W. S., Zhang. H., Zhang. Z., Integrated stabilisation policy over multipath routing-enabled network, IET Control Theory and Applications, 2020, 14(19): 3312-3319.

\bibitem{multi03}
Bagaa. M., Dutra. D. L. C., Taleb. T.,  Samdanis. K., On SDN-driven network optimization and qos aware routing using multiple paths, IEEE Transactions on Wireless Communications, 2020, 19: 4700-4714.


%\bibitem{athans1968}
%Athans. M.,  The matrix minimum principle, Information and Control, 1968, 11: 592--606.

%\bibitem{boyd1994}
%Boyd. S., El Ghaoui. L., Feron. E., Balakrishnan. V., Linear Matrix Inequality in Systems and Control Theory. Philadelphia, PA, USA: SIAM, 1994.

%\bibitem{houting2011}
%Hou. T., Zhang. W., Chen. B., Study on general stability and stabilization of linear discrete-time stochastic systems, Asian Journal of Control, 2011, 13(6): 977--987.

\end{thebibliography}

\end{document}
