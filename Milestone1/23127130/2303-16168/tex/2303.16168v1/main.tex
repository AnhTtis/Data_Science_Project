%                                                                 aa.dem
% AA vers. 9.1, LaTeX class for Astronomy & Astrophysics
% demonstration file
%                                                       (c) EDP Sciences
%-----------------------------------------------------------------------
%
%\documentclass[referee]{aa} % for a referee version
%\documentclass[onecolumn]{aa} % for a paper on 1 column  
%\documentclass[longauth]{aa} % for the long lists of affiliations 
%\documentclass[letter]{aa} % for the letters 
%\documentclass[bibyear]{aa} % if the references are not structured 
%                              according to the author-year natbib style

%

% \documentclass{aa}  
% \usepackage{longtable}
% \usepackage{lipsum} 

% %
% \usepackage{graphicx}
% %%%%%%%%%%%%%%%%%%%%%%%%%%%%%%%%%%%%%%%%
% \usepackage{txfonts}

%[referee]
\documentclass{aa}
%
\usepackage{longtable}
\usepackage{lipsum} 

\usepackage{graphicx}
%%%%%%%%%%%%%%%%%%%%%%%%%%%%%%%%%%%%%%%%
\usepackage{txfonts}
\usepackage{multirow}
\usepackage{hyperref}
\usepackage{comment}
\usepackage{microtype}

%%%%%%%%%%%%%%%%%%%%%%%%%%%%%%%%%%%%%%%%
%\usepackage[options]{hyperref}
% To add links in your PDF file, use the package "hyperref"
% with options according to your LaTeX or PDFLaTeX drivers.
%

\renewcommand{\arraystretch}{1.4} %% increase table row spacing
\renewcommand{\tabcolsep}{2mm}   %% increase table column spacing

\makeatletter
\renewcommand*\aa@pageof{, page \thepage{} of \pageref*{LastPage}}

\begin{document} 


   \title{XRBcats: Galactic Low Mass X-ray Binary Catalogue\thanks{A web-version is publicly accessible at \url{http://astro.uni-tuebingen.de/~xrbcat/} and at CDS via anonymous ftp to cdsarc.u-strasbg.fr (130.79.128.5) or via \url{http://cdsarc.u-strasbg.fr/viz-bin/cat/J/A+A/vol/pag}}}

%   \subtitle{}

   \author{A. Avakyan\inst{1}\thanks{E-mail: artur.avakyan@astro.uni-tuebingen.de},
    M. Neumann\inst{1},
   A. Zainab\inst{2},
          V. Doroshenko\inst{1},
          J. Wilms\inst{2}
        \and 
        A. Santangelo\inst{1}.
          }
\authorrunning{author}
          
   \institute{\inst{1}Universit{\"a}t T{\"u}bingen, Institut f{\"u}r Astronomie und Astrophysik T{\"u}bingen, Sand 1, 72076 T{\"u}bingen, Germany\\
            \inst{2}Dr. Karl-Remeis Sternwarte and Erlangen Centre for Astroparticle Physics, Friedrich-Alexander Universit{\"a}t Erlangen-N{\"u}rnberg, Sternwartstr. 7, 96049 Bamberg, Germany
            }

   \date{Received ; accepted }

  \abstract{
    We present a new catalogue of low-mass X-ray binaries (LMXBs) in
    the Galaxy. The catalogue contains source names, coordinates,
    source types, fluxes, distances, system parameters, and other
    characteristic properties of 348 LMXBs, including LMXBs that were newly discovered or re-classified since the latest releases of the catalogues by \citet{Liu07} and \citet{RitterKolb04}. 
    The aim of this catalogue is to provide a list of all currently known Galactic objects identified as LMXBs with some basic information on each system (including X-ray and optical/IR properties where possible). Literature published before March 2023 has, as far as possible, been taken into account when compiling this information. References for all reported properties as well as object finding charts in several energy bands are provided as part of the catalogue. We plan to update the catalogue regularly, in particular to reflect new objects discovered in the ongoing large scale surveys such as \textit{Gaia} and \textit{eROSITA}.}

  


   \keywords{catalogues -- binaries: close --
                stars: late-type  -- X-rays:
                binaries
               }
   \maketitle
%
%-------------------------------------------------------------------

\section{Introduction}

X-ray binaries (XRBs) are among the brightest objects in the sky in the X-rays. They represent endpoints of stellar evolution of massive stars. Understanding physical processes defining the observational appearance of individual objects and properties of the XRB population as a whole is thus essential for understanding massive star evolution and evolution of the Galaxy in general. It is important, therefore, to keep the up-to date census of known XRBs and their properties, especially in the era of current generation large-scale surveys like the extended ROentgen Survey Imaging Telescope Array \citep[\textit{eROSITA},][]{Merloni12, eROSITA} onboard SRG, \textit{Gaia}~\citep{Gaia_miss, Gaiadr3} and Wide-field Infrared Survey Explorer \citep[\textit{WISE},][]{WISE, WISE_cat} which allow for the possibility to discover ever fainter objects and provide a wealth of observational information for the entire sky. Identification of XRBs among millions of other objects requires, however, knowledge of properties and locations of already known objects. The compilation of such an updated database is the primary goal of this work although many other uses can of course also be envisaged.

The main source of high-energy radiation in XRBs is accretion of matter onto either a neutron star (NS) or a black hole (BH) from a companion star. The observed properties of XRBs are largely defined by the mass transfer mechanism powering the accretion, which can occur either directly from wind of a secondary or via Roche lobe overflow (RLOF), which primarily depends on the mass ratio of the compact object and the optical counterpart~($M_{\rm x}/M_{\rm opt}$). Based on this ratio, XRBs are subdivided into two large groups: high and low mass X-ray binaries (HMXBs and LMXBs, respectively). For objects with $M_{\rm x}>M_{\rm opt}; M_{\rm opt}\lesssim1M_\odot$ the mass is transferred to the compact object via RLOF. Such objects are classified as LMXBs. Those with $M_{\rm x} < M_{\rm opt}; M_{\rm opt}\gtrsim 5 M_\odot$ typically accrete directly from the stellar wind and are classified as HMXBs. The rather dramatic differences in the observed properties of LMXBs and HMXBs warrant treatment of each sub-class separately. Here, we focus  on the properties of LMXBs. An updated catalogue of HMXBs is presented by \citet{Marvin23}. 

As follows from the definition of an LMXB, the donor is (in most cases) a late spectral type star filling its Roche lobe. A-type stars, F-G-type subgiants, or even white dwarfs (WDs), however, can also act as donors in LMXBs.
The optical properties of LMXBs can also be affected by emission from the accretion disc around the compact object, where the disc can be heated by itself or illuminated by X-ray emission from the compact object. Nevertheless, LMXBs are generally intrinsically faint objects in the optical and IR bands. The main way to analyse features of known LMXBs in detail or detect new ones is to observe them during outbursts \citep{Las16}.
Thus, new sources continue to be discovered and the population of known LMXBs is ever growing.


The most recent catalogue of Galactic LMXBs was published 15 years ago by \citet{Liu07}. Since this time many new transient and persistent objects have been discovered. Some LMXB candidates have been reclassified while other sources have gained LMXB status. More importantly, a wealth of additional observational data have been collected with new facilities such as the X-ray Multi-Mirror Mission~\citep[\textit{XMM-Newton},][]{XMM_mission}, the INTErnational Gamma-Ray Astrophysics Laboratory \citep[\textit{INTEGRAL},][]{Integral1} or \textit{Gaia}.
Here we present an updated catalogue of the Galactic LMXBs including this multi-wavelength information as well as adding new sources discovered since the publications by \cite{Liu07} and  \citealt{RitterKolb04} (Liu07 and RK03 hereinafter). Our final catalogue contains 348 LMXBs and candidates, of which 204 were contained in the two catalogues mentioned above (the content of Liu07 and RK03 overlaps but neither includes the other completely). That is, the new catalogue presented here represents a more than 70\% increase in volume with respect to the two most extensive catalogues of the past. We made an effort to collect all relevant and up-to date multi-wavelength information including distances, optical magnitudes, variability in soft and hard X-rays, and more so that the catalogue can be useful in identification of new LMXBs in ongoing and future X-ray surveys, and population studies.  


The structure of the paper is as follows: in Sect.~\ref{sec:sample} we describe how the LMXBs sample was compiled and list data sources used to provide extra information on known objects. In the same section we list relevant X-ray and optical properties, and characterize methods used for identification of known and candidate optical/IR counterparts in large-scale surveys. In Sect.~\ref{sec:cross}, the description of the table fields and finding charts is provided, as well as the comparison of our catalogue with works of RK03 and Liu07. Finally, we summarize our results in Sect.~\ref{conc}. One can also find the list of identified archive optical/IR counterparts for a bunch of LMXBs, based on literature in Appendix~\ref{append1}. The description of the columns of the catalogue table is presented in Appendix~\ref{append2}.


\section{Definition of the sample and data sources}\label{sec:sample}

The bulk of the objects in the catalogue consists of LMXBs reported by the catalogues of RK03 and Liu07, so we started our compilation of the sample by merging these two catalogues. In this process a number of duplicate objects both within and across the two catalogues has been identified and removed from the sample. In addition, several sources in RK03 and Liu07 have been found to be extra-galactic (in some cases this possibility was already indicated in the original catalogues and sometimes reported in the literature afterwards). 
Such objects were also excluded from our final list. The same applies to the
two known LMXBs that are located in nearby galaxies, i.e., in LMC~X$-$2 and RX\,J0532.7$-$6926 which are located in LMC and were included in both RK03 and Liu07. All objects which were removed and otherwise problematic cases are listed and described in detail in Sect.~\ref{sec:except}. 

In addition to Liu07 and RK03, we also considered all Galactic objects classified as possible LMXBs in the \textit{SIMBAD}\footnote{\url{http://simbad.cds.unistra.fr/simbad/}} and \textit{VizieR}\footnote{\url{https://vizier.cds.unistra.fr/viz-bin/VizieR}} databases hosted by the Centre de Donn\'{e}es astronomiques de Strasbourg (CDS) databases. We include such objects in the catalogue giving appropriate references for classification. We also systematically searched the literature  (including Astronomoner's Telegrams\footnote{\url{https://astronomerstelegram.org/}}) for reports of new LMXB discoveries. Most of the new LMXBs reported in the literature between 2006 and 2020 have been, in fact, discovered by the \textit{INTEGRAL} \citep{2003A&A...411L...1W}, \textit{MAXI} \citep{MAXI}, \textit{Swift} \citep{SWIFT} or \textit{Gaia} \citep{2016A&A...595A...1G} missions as summarised by \citet{Bahramian_lmxb}. The latter publication can, therefore, also be considered as another major data source used in this work. It should be noted that although efforts have been made to find all LMXBs identified to date, it is still possible that we may have missed some objects or candidates, so we encourage the reader to report such omissions to the corresponding author so that those can be included in updated versions of the catalogue.

We note that systems in which a WD acts as an accretor are not generally considered as XRBs for historical reasons (despite the fact that they also emit in the X-ray band), so those systems (namely, cataclysmic variables or CVs) are not included in the current catalogue. 
We also do not include rotation-powered pulsars even if they are members of binary systems and emit in X-rays, unless the observed emission is accretion powered. That is, ``spider''-type pulsars including Black Widows~(BW) and Redbacks~\citep[RB; see, e.g.,][]{BW_Chen, BW_Roberts} are omitted unless they are also classified as transitional milliseconds pulsars \citep[tMSP; see, e.g.,][]{2009Sci...324.1411A, 2013Natur.501..517P, 2023MNRAS.520.3416K}. All tMSPs and their candidates included in the catalogue are listed in  Table~\ref{tabl:tMSPcands} in Sect.~\ref{sec:except}.

The masses quoted above for LMXBs/HMXBs are of course approximate, and there are some boundary cases such as Her~X$-$1 where accretion to a NS occurs via the RLOF of a B3 type $\sim2.2 M_{\rm \odot}$ donor \citep{Her, Her1}. This system is most often classified in the literature as an intermediate mass X-ray binary~(IMXB) rather than as an LMXB because the NS appears as an X-ray pulsar and thus is strongly magnetized, which is not typical for NS LMXBs. On the other hand, another system 4U\,1543$-$47 with an intermediate mass ($2.5 M_{\rm \odot}$) A2V type donor with accretion occurring via RLOF to a black hole ($M_{\rm x} = 9.4 M_{\rm \odot}$), is most often classified LMXB~\citep{A2V, Orosz98, Orosz2002}, so the lines are vague here. 
Since the total number of IMXBs is relatively small, and on top of that, although they 'bridge' the HMXB and LMXB classes, they are evolutionarily closer to LMXBs as binary systems, so we opted to include them as part of the current LMXB catalogue. This decision is also clearly communicated in the publication describing the HMXB catalogue released in parallel to the current work \citep{Marvin23} to avoid any confusion and sources duplication, except one specific case (Cir~X$-$1) described in Sect.~\ref{sec:except}.  

\subsection{Main properties and features}\label{sec:xtypes}
In order to reflect the huge variety of LMXB properties, we include available information on properties of the donor star and compact object, as well as
variability and other relevant information. Besides the quantitative properties (such as positions, fluxes in various bands, orbital or spin periods and so on), some of these properties or features can be incorporated as a set of flags defined to characterize several LMXB sub-classes, as was already done by Liu07 and RK03. Here we extend the classification adopted by these authors to account for extra features added to the current catalogue. To give a broad overview for the sample of known LMXBs in the Galaxy, the frequency of occurrence of individual flags is provided along with flag description below (in brackets) and also presented in graphical form in Fig.~\ref{fig:hist}. Note that the flags below are not meant as a basis for a systematic classification but rather simply reflect properties of individual systems to simplify selection of objects exhibiting certain observational features. Each of the sources in the catalogue might have, therefore, one or several of the flags listed below in alphabetical order: 
\begin{enumerate}\label{flags}
\item AS: atoll X-ray binary hosting a weakly magnetized neutron star as a main component, the spectrum is soft and no significant pulsations are present. The majority of the points on the color-color diagrams (``hard'' versus ``soft'') from atoll sources usually form a band at constant hard color.
    
\item BH: system contains a black hole candidate.
    
\item BO: X-ray burster (see XB definition below) with coherent burst oscillations at the neutron star spin period.

\item DC: system with an accretion disc corona.
    
\item DS: ``dipping'' source. System shows periodic, irregular dips in the X-ray intensity, which are generally connected with the partial obscuration of the neutron star by a thickened region of the accretion disc.
    
\item  EB: eclipsing or partially eclipsing binary system.
    
\item GC: source within a globular cluster.
    
\item MQ: microquasar, a source with reported evidence for relativistic jets.
   
\item NSH: negative (nodal) superhumps are present in the system. These negative superhumps are a periodic optical signals whose periods are smaller than the orbital period of the binary. 

  
\item QNS: quiescent neutron star.
   
\item RP: compact object acts as a radio pulsar. 
   
\item PSH: positive superhumps (permanent or transient) present in the system. The same as the NS type (see above), but the time period of the signal is larger than the orbital one. 
   
\item XB: X-ray burst source. This type collects X-ray binary systems with a neutron star, which are showing periodic increases in luminosity.

\item XP: compact object in the binary acts as an X-ray pulsar. 

\item  XT: transient X-ray source. System shows dramatic changes in luminosity (mostly in the X-ray band). Energy is produced by means of a non-stationary accretion process.

\item US: ultra-soft X-ray spectrum. In some catalogues this type is denoted as super soft. These binaries also include some ``extreme ultra-soft'' (EUS) sources.

\item ZS: Z-type source. This type of the X-ray binary is similar to atoll one (also contains weakly magnetized neutron star), but the plot on the color-color diagram is different (Z-letter shape). 
\end{enumerate}



\subsection{X-ray properties}

Considering that X-ray emission dominates the bolometric luminosity of XRBs, one of their main characteristics is their X-ray flux and spectrum. In the catalogue, we tried, therefore, to compile in a uniform way information on X-ray fluxes in soft and hard X-ray bands for all LMXBs. For the soft X-ray band we report fluxes observed by \textit{XMM-Newton}\footnote{\url{https://heasarc.gsfc.nasa.gov/W3Browse/xmm-newton/xmmssc.html}}\citep{2022arXiv221210995S,XMM}, the
Chandra X-ray Observatory\footnote{\url{https://vizier.cds.unistra.fr/viz-bin/VizieR-3?-source=IX/57/csc2master}}\citep[\textit{CXO},][]{CXO_miss, CXO}, and \textit{Swift}/XRT\footnote{\url{https://vizier.cds.unistra.fr/viz-bin/VizieR-3?-source=IX/58/2sxps}}\citep{2005SSRv..120..165B,XRT} throughout their lifetime. In particular, we report the flux in the 0.2--12\,keV energy band (corresponding to EPIC\_8 band of \textit{XMM-Newton} catalogues) and similar
energy bands of \textit{CXO} and \textit{Swift}/XRT, i.e., the 0.3--10\,keV observed flux for \textit{Swift}/XRT, whereas for \textit{CXO}, either the broad ACIS band (0.5--7.0\,keV) or the wide HRC band (0.1--10.0\,keV) were used.
Note that although the difference in energy bands for various instruments might result in a factor of ${\sim}1.5$ difference for a given same source, here we only report the range of fluxes detected over a long period, and this range is dominated by intrinsic variability and differences in source spectra, so for our purposes the difference in energy ranges of various instruments is not really relevant.
In the hard X-ray band fluxes from \textit{INTEGRAL}\footnote{\url{https://vizier.cds.unistra.fr/viz-bin/VizieR-3?-source=J/A\%2bA/545/A27}}\citep{INTEGRAL} and \textit{Swift}/BAT\footnote{\url{https://heasarc.gsfc.nasa.gov/W3Browse/swift/swbat105m.html}}\citep{BAT1, BAT} are reported. Here we report fluxes in energy range of 14-145\,keV for \textit{Swift}/BAT and an energy range of 17--60\,keV for \textit{INTEGRAL}. Note that these bands are largely equivalent as the flux above 60\,keV is comparatively small for most sources, and the variability argument still applies. 

Another key property provided by X-ray observations is the localization essential to identify or assess reliability of identified optical counterparts. We also include, therefore, information on the most accurate X-ray position available either from the literature or directly from soft (\textit{Chandra}, \textit{XMM-Newton} or \textit{Swift}/XRT) and hard (\textit{INTEGRAL}, \textit{Swift}/BAT) catalogues as described in the next section. 

    \begin{figure}
      \centering
    \includegraphics[width=\columnwidth]{Types_hist.pdf}
    \caption{Population of LMXBs of each type in the Galaxy.} 
    \label{fig:hist}
    \end{figure}

\subsection{Astrometry and identification of the optical counterparts}\label{sec:counter}

Considering that properties of XRBs are largely defined by properties of the donor star, the identification of the optical counterparts and characterization of their multi-wavelength properties is essential. Some of the LMXBs in the sample already have robustly identified optical counterparts, however, optical positions (as well as other optical data) reported in Liu07 and RK03 are in many cases outdated and not really accurate by modern standards. We therefore made an effort to update those to current astrometry using \textit{Gaia}~DR2, eDR3, DR3 catalogues~\citep{Gaiadr2, Gaiadr2dist, Gaiaedr3, Gaiaedr3dist, Gaiadr3}\footnote{\url{https://vizier.cds.unistra.fr/viz-bin/VizieR-3?-source=I/352}} and mid-IR \textit{CatWISE}2020 catalogue\footnote{\url{https://vizier.cds.unistra.fr/viz-bin/VizieR?-source=II/365}}\citep{CW2020}. 
To identify counterparts in these catalogues, we first checked whether the \textit{Simbad} database already contained identified \textit{Gaia}~DR3\footnote{\url{https://vizier.cds.unistra.fr/viz-bin/VizieR-3?-source=I/355}}(eDR3, DR2) and/or Two Micron All Sky Survey\footnote{\url{https://vizier.cds.unistra.fr/viz-bin/VizieR?-source=II/246}}\citep[2MASS,][]{2MASS_miss, 2MASS, 2MASS_1} counterparts. If this was not the case, we matched positions obtained from literature to \textit{Gaia}~DR3 (eDR3, DR2) using a match radius of 10$^{\prime\prime}$ and making sure that positions and magnitudes of the identified matches reported in the literature and observed by \textit{Gaia} are consistent within uncertainties.  In most cases this corresponded to the \textit{Gaia} source closest to the literature position with a V~band magnitude within 2\,mag (obtained from the G~magnitude reported by \textit{Gaia}). We note that most of the LMXBs are variable to some extent also in the optical band so an exact comparison would be impracticable. The threshold of 2\,mag was determined empirically by inspecting the difference between the same band for sources whose \textit{Gaia} counterpart was already known.

As mentioned earlier, some LMXBs have been classified as such only based on their X-ray properties, and have no uniquely identified optical counterparts in the literature. In most cases, these are relatively poorly studied sources (sometimes only detected once with X-ray monitors during an outburst). However, in some cases accurate X-ray positions obtained at a later time are available. For those objects we used \textit{CXO}, \textit{XMM-Newton}, or \textit{Swift}/XRT positions (in order of preference). The search radius was set to match the full positional uncertainty (i.e., statistical plus systematic) reported in the respective X-ray catalogue. The final identification of plausible optical counterparts was done manually by inspection of the finding charts also released as part of the current catalogues and researching properties of tentative counterparts in the literature. In all cases where it was possible to identify a plausible counterpart we list is as such, however, the optical position is only assigned as primary if a unique counterpart is identified. All sources where this procedure was applied are flagged as tentative identification in the catalogue (1 in the tentative column).

\subsection{Additional information}

For all optical counterparts identified as described above, we include information on IR and/or optical magnitudes wherever it is possible. In particular we include \textit{Gaia} G-band or V-band magnitudes in the optical band, near-IR photometry either from \textit{2MASS} or Visible and Infrared Survey Telescope for Astronomy \citep[\textit{VISTA},][]{Vista, Vista1} Variables in the Via Lactea (\textit{VVV}) DR4 catalogue \citep[VVV\footnote{\url{http://vvvsurvey.org}},][]{VVV} surveys, or from the literature.  Finally, we include also magnitudes in the W1- and W2-bands reported in the \textit{CatWISE}2020, \textit{AllWISE} or \textit{WISE} (in that order of preference) catalogues~\citep{CW2020, WISE_cat, WISE}. In all cases we give the references to the source where the V-band, G-band, JHK, and \textit{WISE} magnitudes are taken from  the `Vmag\_Ref', `Gmag\_Ref', `JHK\_Ref' and `WISE\_Ref' columns, respectively. It should be noted that we mainly tried to include the optical/IR magnitudes during the quiescent state of the corresponding binary (to represent LMXB population in the state where it stays most of the time). However, in some cases ``quiescent'' magnitudes were not available. In this case we list magnitudes during the outburst, if any.


Besides more accurate positions and photometry, we have also included some additional information unavailable at the time of Liu07 and RK03 publications. 
For all LMXBs for which a \textit{Gaia} counterpart was ultimately identified, we provide distance estimates from \citet{Gaiadr3, Gaiaedr3dist, StarHorse21, Gaiadr2dist, StarHorse19}\footnote{\url{https://vizier.cds.unistra.fr/viz-bin/VizieR-3?-source=I/354/starhorse2021}}. 
If multiple distance estimates were available, the mean distance $\langle d\rangle$ was calculated by using the arithmetic mean on all available distance estimations linked to the given source. The range of possible distances being the lowest and highest estimations of all distance estimates for a given system (both from the literature and \textit{Gaia}) is also reported and represents a conservative estimate of distance uncertainty. We note that the exact distance value estimated by each of the catalogues mentioned above can be obtained through cross-correlation of our catalogue with the respective database. 


The spectral type of the counterpart is also of interest, so we attempted to include this information using the following sources: \textit{Gaia}~DR3, 
\textit{SIMBAD}, and the general literature. We note that the list of X-ray pulsars maintained by Mauro Orlandini\footnote{\url{http://www.iasfbo.inaf.it/~mauro/pulsar_list.html}} was found to be useful source in this regard (and also for reported spin and orbital periods) and would like to acknowledge it here. For sources where the reported spectral type and corresponding reference were published before 2007, the spectral information in RK03 or Liu07 were used. 

Finally, for convenience we also provide the X-ray absorption column density ($N_{\rm H}$) estimates, which  might be relevant to estimate some additional LXMB properties such as X-ray to optical luminosity ratios in order to facilitate searches for new LMXBs with similar properties. 
We report values based either on \textit{Swift}/XRT data \cite{XRT}, or from those reported in the literature (all references are listed). In the absence of an X-ray column density estimate, we estimated them based on extinction in the optical band (using G-band absorption $A_{\rm G}$ from \textit{Gaia}~DR3, \textit{StarHorse} and literature) and for X-ray fluxes (using $N_{\rm H}$, see below for details). In cases where one of the two ($N_{\rm H}$ or $A_{\rm G}$) was unknown, we calculated the other one through the $N_{\rm H}(A_{\rm G})$ relation reported by \citet{Ag_NH}. 


\begin{figure*}
    \includegraphics[width=\textwidth]{GRO_J1744-28.pdf}
    \caption{Finding Chart of GRO\,J1744$-$28. The finding Charts are overlaid by markers and error circles to indicate observation of different instruments. A cyan cross indicates the position of the 2MASS-observation, an orange X the CatWISE-position, and a purple square the observation in \textit{Gaia} DR3. The soft X-ray observations are only using the error circles to prevent overcrowding, the yellow circle is for \textit{Chandra}, and the green one for \textit{Swift}/XRT. A deep-pink pentagram together with an deep-pink error circle is used for \textit{Swift}/BAT, and the lime-green triangle together with the corresponding circle indicates the INTEGRAL-observation. The red cross and the blue diamond are indicating the position reported by \textit{SIMBAD} and in the Literature, respectively. An orange star indicates the position of the LMXB, which is used in this catalogue.}
    \label{fig:finding}
  \end{figure*}

\section{Catalogue content and quality assurance}\label{sec:cross}
\subsection{Description of the fields}\label{sec:fields}

The catalogue contains 348 entries, each of which corresponds to a single Galactic source. The objects are sorted by right ascension (the second column) in the increasing order. There are a total of 61 columns listing various parameters and corresponding references. The content and format of individual columns is described in detail in the Table~\ref{tabl:cols} in Appendix~\ref{append2}.

\subsection{Finding charts and problematic cases}\label{sec:except}

For all sources we provide finding charts which consist of up to 6 different images ranging from near infra-red to hard X-rays as part of the catalogue. For the majority of the finding charts, we used the Hierarchical progressive surveys (HiPS, \citealt{2015A&A...578A.114F}); only in the case of \textit{Swift}/XRT did we use the SkyView Query. Both HiPS and SkyView offers the possibility to query their data automatically with Python. In case of HiPS, we used astroquery.hips2fits\footnote{\url{https://astroquery.readthedocs.io/en/latest/hips2fits/hips2fits.html}} package, and to access SkyView we used the astroquery.skyview\footnote{\url{https://astroquery.readthedocs.io/en/latest/skyview/skyview.html}} package. In the following, we will mention the used surveys and with their corresponding links as footnotes. The finding charts consist of up to 6 different surveys which we mention below, with their corresponding links as footnotes. 
In the top row we included, the image of the Visible and Infrared Survey Telescope for Astronomy (VISTA) Variables in the the Via Lactea (VVV\footnote{\url{http://alasky.cds.unistra.fr/VISTA/VVV_DR4/VISTA-VVV-DR4-J/}}) DR4 catalogue \citep{VVV} or 2MASS\footnote{\url{http://alasky.cds.unistra.fr/2MASS/J/}} (in order of preference, left), an image of \textit{unWISE}\footnote{\url{http://alasky.cds.unistra.fr/unWISE/W1/}} \citep{2019ApJS..240...30S} in the middle and the RGB-image of either \textit{Chandra}\footnote{\url{https://cdaftp.cfa.harvard.edu/cxc-hips/}}, \textit{XMM-Newton}\footnote{\url{http://skies.esac.esa.int/XMM-Newton/EPIC-RGB/}} or the Roentgensatellit (\textit{ROSAT}\footnote{\url{http://alasky.cds.unistra.fr/RASS/}}, \citealt{ROSAT_miss, ROSAT, ROSAT99}) (in order of preference, right). In the bottom row, there is the soft X-ray image of \textit{Swift}/XRT (SwiftXRTInt in astroquery.skyview) at the left corner, for hard X-rays \textit{Swift}/BAT\footnote{\url{http://cade.irap.omp.eu/documents/Ancillary/4Aladin/BAT_14_20/}} images in the middle and \textit{INTEGRAL}\footnote{\url{http://cade.irap.omp.eu/documents/Ancillary/4Aladin/INTEGRAL_17_60/}} images in the right corner. A 1 arcmin field of view was used for creation of the VVV, 2MASS, \textit{unWISE}, \textit{XMM-Newton}, and \textit{Chandra} images. In case of \textit{Swift}/XRT and \textit{ROSAT}, we used a field of view of 5 arcmin and 15 arcmin, respectively. The field of view in case \textit{Swift}/BAT and \textit{INTEGRAL}  was chosen to be $10^\circ$. In each case, the size of the region was chosen considering field of view and angular resolution of a given instrument. Every panel also shows coordinates and uncertainties of all detected sources within respective regions (position uncertainties are represented by error circles). A red cross indicates the position which is mentioned in \textit{SIMBAD} for the Source, and a dodger-blue diamond for the coordinates which are described in the literature. The position of the source which is used in this catalogue is indicated with an orange star. In case of the  soft X-ray instruments, the position of the observations is indicated by the error circles to prevent overcrowding, a golden circle indicates the \textit{Chandra} position, a red circle \textit{XMM-Newton}, a green circle \textit{Swift}/XRT, and a navy-blue circle \textit{ROSAT}. \textit{Swift}/BAT and \textit{INTEGRAL} are indicated as a deep-pink pentagon and lime-green triangle, respectively. In the optical band, we indicate \textit{CatWISE} with an orange X, 2MASS with a cyan +, and \textit{Gaia}~DR3 data with purple square.
As an example, Figure \ref{fig:finding} shows the finding chart of GRO\,J1744$-$28. We note also that imaging in more bands with a catalogue overlay is available via the \textit{Aladin lite} interface \citep{2022ASPC..532....7B} on the website hosting the catalogue\footnote{\url{http://astro.uni-tuebingen.de/~xrbcat}}. 

Based on the visual inspection of the finding charts and literature research, several problematic cases have been identified (grouped by cause of the problem):

\begin{table}
\caption{\label{tabl:tMSPcands}List of tMSP candidates presented in our catalogue along with known tMSPs: XSS~J12270$-$4859, IGR~J18245$-$2452 and PSR~J1023$+$0038.}
\begin{tabular}{c c}
\hline\hline 
Source Name & References$^\star$ 
 \\ \hline

1FGL~J1417.7$-$4407 (J1417$-$4402 in RK03) & [1] \\ 
CXOGlb~J183544.5$-$325939 (NGC~6652B) & [2] \\ 

4FGL~J0427.8$-$6704 & [3], [4] \\ 

4FGL~J0407.7$-$5702 & [5] \\ 

CXOGlb~J174804.5$-$244641 (Terzan~5~CX1) & [6] \\

CXOU~J110926.4$-$650224 & [7], [8] \\

3FGL~J1544.6$-$1125 & [9], [10], [11] \\
 
4FGL~J0540.0$-$7552 & [12] \\

RX~J1739.4$-$2942 (GRS~1736$-$297)  & [13] \\

\hline
\\
\end{tabular}
%\end{center}
$^\star$ 
[1] \citet{2018ApJ...866...83S}, 
[2] \citet{2021MNRAS.506.4107P},  
[3] \citet{2020MNRAS.494.3912K}, 
[4] \citet{2020ApJ...895...89L},
[5] \citet{2020ApJ...904...49M},
[6] \citet{2018ApJ...864...28B},
[7] \citet{2019A&A...622A.211C},
[8] \citet{2021A&A...655A..52C},
[9] \citet{2015ApJ...803L..27B},
[10] \citet{2017ApJ...849...21B},
[11] \citet{2021ApJ...923....3J},
[12] \citet{2021ApJ...917...69S}, 
[12] \citet{2016ATel.8744....1T}, 
\end{table} 



\begin{itemize}

    \item The X-ray source SAX~J0840.7$+$2248, which is classified as a LMXB in Liu07, has been excluded from our catalogue. This transient is more likely to be the X-ray rich gamma ray burst GRB980429 rather than an XRB~\citep{2007ATel.1089....1S}.

    \item SWIFT~J0732.6$-$1330, which is listed as a LMXB in \textit{SIMBAD}, also is not presented in our catalogue, due to the fact that the source is an intermediate polar~\citep{2007A&A...475L..29B}. 

    \item According to \citet{2021ApJ...916...80P}, the same scenario of polar nature applies to the LMXB candidate OGLE~BLG511.6~25872. Thus, we do not include it in our catalogue.

    \item The X-ray transient Swift~J061223.0$+$701243(.9), listed as LMXB in Liu07, could also be an intermediate polar~\citep{2011A&A...526A..77B}. In this case the authors suggest that an LMXB nature ``cannot be ruled out;', so the source can still be found in our catalogue but we urge caution when  considering the classification of this source.

    \item V*~V934~Cen, another \textit{SIMBAD} LMXB, seems to possess no XRB nature~(the corresponding link in \textit{SIMBAD} for the suggested LMXB type does not provide any data regarding the source), so it is excluded from our catalogue as well. 

    \item The same also goes for the binary TWA 22 (TWA 22AB), which is considered to be a LMXB in  
    \textit{SIMBAD} database. \textit{SIMBAD} refers to \citet{2009A&A...506..799B} as the source of the corresponding classification. However, authors in  \citet{2009A&A...506..799B} estimated the total mass 
    of the binary to be about a $220 M_{\rm Jup}$, which is way too low for the XRB. Although the system 
    emits X-rays via coronal activity~\citep{2022A&A...661A..44S} , it cannot be classified as an XRB.

    \item 4U~1745$-$203 (H~1745$-$203), is a source that is was considered as a separate LMXB, but it seems like this is just a counterpart of SAX~J1748.9$-$2021 (NGC 6440~CX1), which has already been noted in Liu07.

    \item The same goes for bursting source MXB~1742$-$29. Following \citet{2007A&A...462.1065M} we consider it to be the counterpart of LMXB 1A~1742$-$294.

    \item AX~J1620.1$-$5002 is also an X-ray transient which for a long time was considered to be separate XRB, but it is now strongly associated with bursting transient LMXB MAXI~J1621$-$501~\citep{2018ATel11272....1C, 2018ATel11317....1G}. 

    \item 4U~0614$+$09 (4U~0614$+$091) is an ultra-compact LMXB with uncertain orbital period, which is only constrained to be greater that 1\,h \citep{2014A&A...572A..99B}. However, following Liu07 and RK03 in the catalogue, we provide a value of 51.3\,min, based on \citet{2008PASP..120..848S}.

    \item The nature of IGR~J17404$-$3655 was not well constrained, since it was unclear rather the source is LMXB or HMXB, potentially BeXRB~\citep{2013A&A...560A.108C}. However, \citet{2018A&A...618A.150F} used K-band spectroscopy of the source, and revealed that the object is more likely to be a CV with a K3–5V donor star. Consequently, we made the decision to not include it.

    \item CXOGBS~J174623.5$-$310550 is an accreting binary, which is presented in our catalogue. However, its exact classification as either LMXB or CV is still unknown~\citep{2019MNRAS.487.2296T}. 

    \item 1RXS~J180431.1$-$273932 was, at first, considered as a
      symbiotic LMXB system~\citep{2007A&A...474L...1N} hosting a
      neutron star with a M5~III donor star. However, based on optical
      observations, \citet{2012A&A...544A.114M} excluded possible
      symbiotic LMXB nature and identified the source as a magnetic
      CV. Thus, despite the fact that 1RXS~J180431.1$-$273932 is
      listed as symbiotic X-ray binary in \citet{2015AstL...41..114K},
      we do not include it.

    \item Based on IR observations of 4U~1556$-$60 made by \citet{2013AstL...39..523R}, we associated for this source an optical counterpart \textit{Gaia}~DR3 583304229928809907~(see Table~\ref{tabl:opt}), which lies well inside the position error circle of 0\farcs3. However, as it goes from provided parallax, the distance to this source should not exceed 1kpc~\citep{Gaiaedr3dist}, which is 3--4 times less than distance estimations to 4U~1556$-$60~\citep{1997ApJS..109..177C, 2002A&A...391..923G}. We decided to leave this association with \textit{Gaia}~DR3 source, but keep in mind this discrepancy. 

    \item For the source IGR~J17597$-$2201 \citet{2006A&A...453..133W} proposed 2MASS~J17594556$-$2201435 as a potential counterpart. However, based on Very Large Telescope (VLT) observations conducted by~\citet{2018A&A...618A.150F}, we can exclude this possibility.

    \item The persistent X-ray bright LMXB 4U~1624$-$49, seems to have incorrect coordinates and optical/IR associations in the \textit{SIMBAD} database. 
    The source was localized by Chandra with 0\farcs6 uncertainty at the position
    $\alpha_{\rm J2000.0} = 16^{\rm h}28^{\rm m}02\fs825$, $\delta_{\rm J2000.0} = -49\degr11\arcmin54\farcs61$ \citep{2005ApJ...621..393W}, which corresponds to the known XMM source 2XMM~J162802.8$-$491154, located  $\approx 53\farcs$ away from the position of 4U~1624$-$49 in \textit{SIMBAD}. In our table, we provide the Chandra coordinates and uncertainty, as well as the K-band data obtained by \citet{2005ApJ...621..393W}. In the corresponding Chandra error circle there are no \textit{Gaia}, \textit{2MASS} or \textit{CatWISE}2020 sources. We also label 2XMM~J162802.8$-$491154 as an alternative name for 4U~1624$-$49, due to the fact that this name is not listed as identifier for this source in any other found database.

    \item The same scenario of incorrect \textit{SIMBAD} associations with optical/IR archives applies to the LMXB GX~5$-$1(4U~1758$-$25). At the position provided by \textit{SIMBAD} there is a bright star, located $\approx 200$ pc away in the Gaia survey \citep{Gaiadr3}, a distance which is totally inconsistent with another distance estimation ~\citep[4.2--5.6 kpc, according to][]{2018ApJ...852..121C}. The study \citet{2000MNRAS.315L..57J} found an IR counterpart of the X-ray source at the position 
    $\alpha_{\rm J2000.0} = 18^{\rm h}01^{\rm m}08\fs222$, $\delta_{\rm J2000.0} = -25\degr04\arcmin42\farcs46$ with 0\farcs35 uncertainty, leading to it being located  $\approx 20\farcs$ away from the position of GX~5$-$1 in \textit{SIMBAD}. As in the case of 4U~1624$-$49 there are no \textit{Gaia}, \textit{2MASS} or \textit{CatWISE}2020  counterparts. However, in the error circle we found a VVV source, from which the JHK magnitudes are provided in the table.


    \item We also decided to add the object named AX~J1659.0$-$4208 as a
    LMXB in our catalogue. The source was detected during Advanced
    Satellite for Cosmology and Astrophysics (ASCA) Galactic plane
    survey~\citep{2001ApJS..134...77S}. At the time, source's X-ray
    flux was at the level of $6 \times 10^{-12}$ erg/s/cm$^{2}$
    ($0.7-10$\,keV), but it's coordinates were not well constrained.
    Object's accurate position was then derived by Chandra, what led
    to identification of NIR counterpart by
    \citet{2013AstL...39..523R}. Authors suggested that the source is
    likely to be a symbiotic LMXB or a CV. However, based on high
    extinction towards this area, and distance estimations about 5-10
    kpc~\citep{2013AstL...39..523R}, unabsorbed X-ray luminosity can
    be estimated (at very least) to be $10^{34}$ (most likely even
    $10^{35}$) $\mathrm{erg}\,\mathrm{s}^{-1}$, i.e., significantly
    greater than expected from a typical CV. Hence, we suggest that
    the source is very likely a LMXB and it is added to our catalogue
    table.


    \item As for LMXB 4U~1728$-$34, it was supposed that the source has an orbital period of $\sim 11$
    min~\citep{2010ApJ...724..417G}, and this value is associated for the binary in RK03. However, analysis of other X-ray observations could not find further evidence for a periodic signal at 11 min. In the end, based on analysis of its type I X-ray burst, \citet{2020MNRAS.495L..37V} stated that the orbital period should be much higher, probably $\sim 66$ min or even $\gtrsim 2$ hr.  In the catalogue the value of 66 min is quoted, but keep in mind other possibilities.

    \item The X-ray source J2039$-$5617 (listed as [SMD2015]~3 in \textit{SIMBAD}) is presented in the RK03 catalogue as a LMXB. However, due to a likely association with the known RB source 1FGL~J2039.4$-$5621~\citep{2015ApJ...814...88S}, it is excluded from our catalogue.

    \item XB~1832$-$330 -- a galactic cluster LMXB (in NGC 6652) was tentatively identified to have 43.6\,min orbital period~\citep{2000ApJ...530L..21D}. But in \citet{2012ApJ...747..119E}, the authors argue that the 43.6\,minute candidate period is probably spurious and suggest a longer period (by a factor 3) of about 2.15\,hr (this value is quoted in our catalogue).

    \item There is uncertainty about the orbital period of the tMSP candidate 4FGL~J0540.0$-$7552. \citet{2021ApJ...917...69S} identified a consistent periodic signal at 2.7\,h in their photometry. However, this signal could represent either the orbital period, half the orbital period (if due to ellipsoidal variations), or could even be spurious. We decided to keep the 2.7\,h value of orbital period in our catalogue table for this source, but keep in mind other possibilities.
    
 \end{itemize}
As was mentioned in the introduction, we decided to add all 3 known tMSP: XSS~J12270$-$4859, IGR~J18245$-$2452 and PSR~J1023$+$0038 (all three are also listed in RK03 by the names J1227$-$4853, J1824$-$2452 and AY~Sex, respectively). In addition, we include some tMSP candidates identified as such in the literature, which can be found in the Table~\ref{tabl:tMSPcands} with following references. In the catalogue table itself, one can find a corresponding flag column named `tMSP\_Flag'~(see Table~\ref{tabl:cols}).

Several objects originally listed in the HMXB catalogue of \citet{Liu_HMXB} were now moved to the current LMXB catalogue as their preferred classification changed:
\begin{itemize}
\item \citet{Karasev08} proposed that XTE~J1901$+$014 could be the first low-mass Fast X-ray transient, but the nature of the system is not completely clear until today. However, according to \cite{Sato19} it appears to be more similar to a LMXB, hence it is included in our LMXB catalogue.

\item \citet{Chaty08} classified IGR~J16358$-$4726 as an HMXB, however, this classification was revoked by \citet{Nespoli10} and now the object is considered to be a Symbiotic X-ray binary which we consider to be a LMXB sub-class based on the typical donor mass.

\item 
The situation is similar for IGR~J17407$-$2808. The source was first classified as part of a Supergiant Fast X-ray Transients~\citep[SFXT;]{2006ApJ...646..452S}, which are all HMXBs. However, subsequent works~\citep{2009ApJ...701.1627H, 2011ATel.3688....1G} ruled out the presence of a supergiant companion, and showed that a LMXB with F-type dwarf is more likely to be the case. \citet{2011ATel.3695....1K} also found that a few days after the outburst, the IR counterpart was one magnitude brighter. Such a behavior is more expected during the LMXB outburst due to the exposure of the optical component to the X-rays emitted by the compact object.

\item With the de-reddened magnitudes of their observations in the JHK-bands and under black-body spectrum assumption, \citet{Kaur09} estimated the distance of SAX~J1452.8$-$5949. This ruled out the possibility of a HMXB due to the fact that the system would be an extra-galactic source in this case. They concluded that the binary system must have a low-mass companion and therefore is either a LMXB or an Intermediate Polar (accreting magnetized white dwarf, IP). Therefore, we consider the source as an LMXB candidate.

\item 
It has been unclear for some time whether 4U~1807$-$10 is a HMXB or a LMXB. However, based on the ratio of spin and orbital periods \citep{Blay08}, as well as the fact that the system shows type I X-ray bursts \citep{2017AstL...43..781C}, we conclude that the LMXB origin is more likely to be the case, despite the fact that 4U~1807$-$10 is marked as a HMXB-candidate in \citet{Blay08}, and thus we add 4U~1807$-$10 to our LMXB catalogue.

\end{itemize}


Along with that, there is one source that does present in both of
our catalogues, in this one, and in the one dedicated to
HMXB~\citep{Marvin23}. This happened due to the presence of
conflicting results and absence of any final decision on the origin
nature of the source:

\begin{itemize}

\item Since its discovery, Cir~X$-$1 was often referred to as a LMXB, until \citet{2013ApJ...779..171H} could determine the age of the system to be about 4500\,years. In addition, according to \citet{2007MNRAS.374..999J}, the system contains an A0 to B5 type supergiant companion. However, Cir~X$-$1 has also shown type~I X-ray bursts~\citep{1986MNRAS.221P..27T}, which indicates that the source is a LMXB. Possible LMXB origin nature is also supported by the fact that companion star itself in Cir~X$-$1 still cannot be unambiguously detected at optical wavelengths.
\citet{2016MNRAS.456..347J} concluded that the donor can be: 1) an unevolved  low mass star, or 2) a giant star that  is still recovering from the impact of the supernova blast that happened less than 5000\,years ago. Taking into account all this obscurity around Cir~X$-$1's nature, we decided to add it to both LMXB and HMXB catalogues. 
\end{itemize}
Several objects do actually have tentative or known optical counterparts in the literature but could not be found automatically through the procedure outlined above. 
For those objects, the literature was searched manually. In some cases optical, IR, or radio precise coordinates were available and we used them to find corresponding counterparts in \textit{Gaia}~DR3, 2MASS, etc. All solid cross-matches and their coordinates were added to the catalogue. Additionally, identifications by \citet{Arnason21} proved to be useful so these shall be considered as the main references for the cases marked in the Table~\ref{tabl:opt} in Appendix~\ref{append1}, which lists all sources with optical/IR counterparts found, except those listed in \textit{SIMBAD}.

As mentioned previously, several objects included in RK03 and Liu07 have been removed from our catalogue. Sometimes this was necessary because objects reported as independent in RK03 and Liu07 represented the same source (in either of the catalogues). This was mainly due to large positional uncertainties at the time of the catalogue compilation, or in some cases because the objects are extragalactic. In the case of the former, we include only the element from the pair in our LMXB list with the most used name, omitting the other, and summing up all known data. In the extragalactic origin scenario, the corresponding source is removed as we only consider Galactic LMXBs. Our initial source list was, therefore, modified as following:

\begin{itemize}
    \item  The relatively bright source 2A~0521$-$720, which is more known as LMC~X$-$2, is located (as it could be suggested from the name) in the LMC. Since we consider only Galactic objects, we exclude it from our catalogue.
    \item  This goes for RX~J0532.7$-$6926, i.e. it is also not included in the current catalogue.
    \item  The first objects AX~J1745.6$-$2901 and 1A~1742$-$289 reported by Liu07 actually correspond to the same source, and since the name ``AX~J1745.6$-$2901'' is more common to use, we combine information regarding the two from Liu07 under this name.
        
    \item The second similar case is 1E~1743.1$-$2852 and SAX~J1747.0$-$2853, so we did the same giving the most popular name to the source, SAX~J1747.0$-$2853.
    
 \end{itemize}

 
 \subsection{Summary of other differences with respect to the Liu07 and RK03 catalogues}

In addition to the problematic cases described above, and overall expansion of the sample, several other changes have been made compared to Liu07 and RK03:
\begin{itemize}
    \item  We homogenised the energy ranges for reported X-ray fluxes. Liu07 used an energy range between 2--10\,keV for most sources, but in some cases the actual energy range was different and it was not trivial to identify such cases. We now quote soft and hard X-ray fluxes in well defined energy ranges separately. 
    \item Furthermore, for most cases Liu07 only reported the maximal
      value of X-ray flux in units of Jy, now we report flux ranges in
      two energy bands in more commonly used cgs
      $\mathrm{erg}\,\mathrm{s}^{-1}\,\mathrm{cm}^{-2}$ units
      (equivalent to $\mathrm{mW}\,\mathrm{m}^{=2}$ preferred by the
      \textit{VizieR} database), which makes more sense for X-ray
      sources.
    \item We increased the number of flags used to characterize source properties from 11 in Liu07 to a total of 17 in our case to better reflect various phenomenological features reported in the literature.
    \item Hydrogen column density, all IR data (WISE and 2MASS), BP-RP colour, optical Gaia data, and G-band extinction are new fields introduced in this version of the catalogue.
    \item We include up to six finding charts as part of the catalogue (from near-infrared images to hard X-ray bands), instead of a reference for a finding chart.
    \item Finally, the total number of objects included in the catalogue increased from 206 (Liu07 and RK03 combined) to 348, which represents a 72\% increase in catalogue size.
\end{itemize}

\section{Summary and conclusions}\label{conc}
In this paper, we present a new LMXB catalogue which provides an up-to-date census of known LMXBs in the Galaxy with a variety of corresponding multi-wavelength information extracted from current optical/near-IR surveys. In total, the catalogue contains 348 sources and has 61 columns. A description of the columns is presented in Table~\ref{tabl:cols}. The full version of the catalogue can be obtained through \textit{VizieR}, as well as on a dedicated website\footnote{\url{http://astro.uni-tuebingen.de/~xrbcat}}. We ask authors who use our catalogue in their research, to refer to us by citing this article, as well as by referring to the website link in either the acknowledgments or a footnote.

We would like to emphasize here that the origin of some of the sources listed in the catalogue has not yet been fully certain. Some sources have only been tentatively classified as LMXBs due to the similarity of their X-ray properties to those of the systems identified, with no counterparts found in other bands. Such objects should be treated with caution in view of all the further work and are clearly marked in the catalogue (field 6).

Along with this article, a new similar catalogue of HMXBs is being published \citet{Marvin23}. We hope that both catalogues will be useful tools for future studies which include (but are not limited to) population studies of various types of X-ray emitting objects (especially XRBs), search for new XRBs in ongoing and upcoming surveys, classification of unidentified X-ray sources, and as training data-sets for machine learning classification algorithms~\citep{Yang2021, Yang2022}. In particular, this can be relevant in the context of the ongoing analysis of \textit{eRosita} survey data which is expected to substantially increase the number of known XRBs \citep{2014A&A...567A...7D}. We plan, therefore, to update our catalogues to the best of our ability to keep the ever-increasing census of the Galactic LMXBs up to date and accessible to the community.


\begin{acknowledgements}
  This research has made use of the \textit{SIMBAD} data base and
  \textit{VizieR} catalogue access tool operated at CDS, Strasbourg,
  France, and NASA’s Astrophysics Data System (ADS). This research has
  made use of "Aladin sky atlas" developed at CDS, Strasbourg
  Observatory, France. This work has made use of data from the
  European Space Agency (ESA) mission \textit{Gaia}
  (\url{https://www.cosmos.esa.int/gaia}), processed by the
  \textit{Gaia} Data Processing and Analysis Consortium (DPAC,
  \url{https://www.cosmos.esa.int/web/gaia/dpac/consortium}). Funding
  for the DPAC has been provided by national institutions, in
  particular the institutions participating in the \textit{Gaia}
  Multilateral Agreement. We acknowledge the public data from
  \textit{XMM-Newton}, \textit{CXO}, \textit{ROSAT}, \textit{Swift},
  \textit{INTEGRAL}, \textit{WISE}, \textit{2MASS} and \textit{VISTA}.
  AA thanks Deutsche Forschungsgemeinschaft (DFG) for support through
  the eRO-STEP research unit project 414059771 (DO 2307/2-1 and WI
  1860/19-1).
\end{acknowledgements}



\bibliographystyle{aa}
\bibliography{main.bib}

\newpage

\appendix


\onecolumn

\section{optical/IR counterparts}\label{append1}

\begin{longtable}
{p{4.9cm}p{4.0cm}p{4.0cm}p{4.0cm}}
\caption{\label{tabl:opt} Identified archive Optical/IR counterparts of LMXBs.}\\
\hline\hline

Source name & \textit{Gaia}~DR3 & \textit{2MASS} & \textit{CatWISE}2020 \\
\hline
\\
\multicolumn{4}{c}{Based on optical/IR/radio follow-ups:}
\\
\\
\endfirsthead
\multicolumn{4}{c}%
{{\bfseries \tablename\ \thetable{} -- continued}} \\
\hline\hline
Source name & \textit{Gaia}~DR3 & \textit{2MASS} & \textit{CatWISE}2020\\ 
\hline
\endhead

\hline
\endlastfoot
MAXI~J0556$-$332 & 2890346074897001216 &  &  \\

4FGL~J0427.8$-$6704 &
4656677385699742208 
&
J04274958$-$6704350 
&
J042749.64$-$670435.0 \\


IGR~J17494$-$3030 &
4056028099172996352 &  &  \\

1FGL J1417.7-4407 &
6096705840454620800 
&
J14173057$-$4402574 
&
J141730.57$-$440257.5 \\


XTE~J1637$-$498 &
5940421734427982336 
&
J16370267$-$4951401 
& \\


MAXI~J1807$+$132
&
4497207964419829632 &  &  \\


3A~1837$+$049 &
4283919201304278912 
&
J18395759$+$0502113 
&
J183957.57$+$050210.5 \\


MAXI~J1828$-$249 &
4076998775244833280 &  &  \\


XTE~J1810$-$189 &
4095486509152850176 
&
J18102033$-$1904136 
&  \\

1RXS~J180408.9$-$342058  &
4042163562572848384 &  &  \\


IGR~J17091-3624 &
5976921951382731520 
& 
& 
J170907.75$-$362425.0 \\


IGR~J17585$-$3057 &
4044163406621956992 
& 
&
J175829.80$-$305702.3 \\

4U~1543$-$624 &
5826501373348972288  &  &  \\

Swift~J1858.6$-$0814 &
4203799300867262720 &  &  \\


MAXI~J1305$-$704 &
5843823766718594560 
&  
& 
J130655.74$-$702704.2 \\


IGR~J17329$-$2731  &
4061336747511224704 
&
J17325067$-$2730015
&
J173250.54$-$273003.4 \\

IGR~J16358$-$4726 
&
&
J16355369$-$4725398
&
J163553.75$-$472541.0 \\


XMMU~J174445.5$-$295044 
&
&  
J17444541$-$2950446
&
J174445.43$-$295044.5 \\


IGR~J17454$-$2919 &
& 
J17452768$-$2919534
&  \\


4U~1705$-$250  &
4112450294268643456  &  &  \\


4FGL~J0540.0$-$7552 &
4648562676357022208 
&
&  
J054001.80$-$755419.7 \\

3FGL~J1544.6$-$1125  &
6268529198286308224 
&
&  
J154439.38$-$112804.5 \\


CXOU~J110926.4$-$650224  &
5240167590731178624
&
& 
J110926.21$-$650227.1 \\


4FGL~J0407.7$-$5702 &
4682464743003293312 
&
&  
J040731.65$-$570025.1 \\


1RXH~J173523.7$-$354013 
&
5974787971132982144
&
&  
 \\
  
IGR~J16287$-$5021
&
5934583950467938304
&
&  
 \\


4U~1556$-$60
&
5833042299288099072
&
&  
 \\

EXO~1846$-$031
&
4258794192383316864
&
J18491710$-$0303559
&  
J184917.08$-$030355.3
 \\


AX~J1735.8$-$3207
&
4055019091071763712
&
J17354627$-$3207099
&  
 \\


Swift~J2037.2$+$4151
&
&
J20370560+4150051
&  
J203705.60+415005.2 
\\




\\ 
\multicolumn{4}{c}{Based on \citet{Arnason21}:}
\\
\\

GRO~J1655$-$40 & 
5969790961312131456 
& 
J16540014$-$3950447 
& 
J165400.13$-$395044.7 \\


SWIFT~J061223.0$+$701243 & 1107229825742589696 
&  
& J061222.63$+$701243.1 \\


1A~1246$-$588 & 
6059778089610749440 
&
&  
J124939.17$-$590516.3 \\

MXB~1659$-$29 & 
6029391608332996224  &  &  \\

SAX~J1711.6$-$3808 & 
5973177495780065664 
& 
J17113714$-$3807073 
& 
J171137.14$-$380707.4 \\



4U~1724$-$307 & 
4058208396397618688 
 & 
J17273315$-$3048076 
 & 
J172733.28$-$304809.0 \\

EXO~1747$-$214 & 
4118590585673834624  &  &  \\

4U~1755$-$33 & 
4042473487415175168  &  &  \\

HETE~J1900.1$-$2455 & 
4074363039644919936 &  &  \\

4U~1915$-$05 & 
4211396994895217152  &  &  \\

XTE~J1901$+$014 & 
4268294763113217152 &  &  \\
\end{longtable}


\newpage


\section{catalogue format}\label{append2}
%\small

\begin{longtable}{p{0.2cm}p{2.2cm}p{1.9cm}p{12.5cm}}
\caption{\label{tabl:cols} Definition of columns in the LMXB catalogue. In total the catalogue contains 61 columns.}\\
\hline\hline
№ & Column name & Unit & Description \\
\hline
\endfirsthead
\multicolumn{4}{c}%
{{\bfseries \tablename\ \thetable{} -- continued}} \\
\hline\hline
№ & Column name & Unit & Description \\ 
\hline
\endhead

\hline
\endlastfoot
 \documentclass[lettersize,journal]{IEEEtran}
\usepackage{amsmath,amsfonts}
\usepackage{algorithmic}
\usepackage{algorithm}
\usepackage{array}
%\usepackage[caption=false,font=normalsize,labelfont=sf,textfont=sf]{subfig}
\usepackage{textcomp}
\usepackage{stfloats}
\usepackage{url}
\usepackage{verbatim}
\usepackage{graphicx}
\usepackage{cite}
\usepackage{color}
\hyphenation{op-tical net-works semi-conduc-tor IEEE-Xplore}
% updated with editorial comments 8/9/2021
\usepackage{microtype}
\usepackage{graphicx}
\usepackage{booktabs}
\usepackage{hyperref}
\usepackage{tcolorbox}
%\numberwithin{equation}{section}
\usepackage{amssymb}
\usepackage{mathtools}
\usepackage{amsthm}
\usepackage{caption}
\usepackage{pifont}
\usepackage{subcaption}
\usepackage[capitalize,noabbrev]{cleveref}
\newcommand{\Ren}[1]{\textcolor{blue}{Ren: #1}}
%%%%%%%%%%%%%%%%%%%%%%%%%%%%%%%%
% THEOREMS
%%%%%%%%%%%%%%%%%%%%%%%%%%%%%%%%
\theoremstyle{plain}
\newtheorem{theorem}{Theorem}
\newtheorem{proposition}[theorem]{Proposition}
\newtheorem{lemma}{Lemma}
\newtheorem{corollary}[theorem]{Corollary}
\theoremstyle{definition}
\newtheorem{definition}[theorem]{Definition}
\newtheorem{assumption}[theorem]{Assumption}
\theoremstyle{remark}
\newtheorem{remark}[theorem]{Remark}
\newtheorem{mydef}{\bf{Definition}}

\DeclareMathOperator*{\minimize}{\text{minimize}}
\DeclareMathOperator*{\maximize}{\text{maximize}}

\DeclareMathOperator*{\st}{\text{subject to}}
\DeclareMathAlphabet\mathbfcal{OMS}{cmsy}{b}{n}
\newcommand{\Def}[0]{\mathrel{\mathop:}=}

\newcommand{\din}{\mathcal D}
%\DeclareMathOperator*{\minimize}{\text{minimize}}
%\DeclareMathOperator*{\maximize}{\text{maximize}}
\DeclareMathOperator*{\argmax}{arg\,max}
\DeclareMathOperator*{\argmin}{arg\,min}
%\DeclareMathOperator*{\st}{\text{subject to}}
\newcommand{\C}{\mathbb{C}}
\newcommand{\R}{\mathbb{R}}
\usepackage[textsize=tiny]{todonotes}

 



\begin{document}


\title{Bridging Models to Defend: A Population-Based Strategy for Robust Adversarial Defense}


\author{Ren~Wang,~\IEEEmembership{Member,~IEEE,}  Yuxuan~Li,~\IEEEmembership{Student Member,~IEEE,}
Can~Chen,~\IEEEmembership{Member,~IEEE,}        Dakuo~Wang,~\IEEEmembership{Senior~Member,~IEEE,} Jinjun~Xiong,~\IEEEmembership{Fellow,~IEEE,} Pin-Yu~Chen,~\IEEEmembership{Fellow,~IEEE,}        Sijia~Liu,~\IEEEmembership{Senior~Member,~IEEE,} Mohammad~Shahidehpour,~\IEEEmembership{Life~Fellow,~IEEE,}       and~Alfred~Hero,~\IEEEmembership{Life~Fellow,~IEEE}% <-this % stops a space
\IEEEcompsocitemizethanks{\IEEEcompsocthanksitem Ren Wang is with the Department
of Electrical and Computer Engineering, Illinois Institute of Technology, Chicago,
IL 60616.%\protect\\
% note need leading \protect in front of \\ to get a newline within \thanks as
% \\ is fragile and will error, could use \hfil\break instead.
%E-mail: rwang74@iit.edu
\IEEEcompsocthanksitem Yuxuan Li is a graduate research intern in the Department
of Electrical and Computer Engineering, Illinois Institute of Technology, Chicago,
IL 60616.
\IEEEcompsocthanksitem Can Chen is with the School of Data Science and Society, University of North Carolina at Chapel Hill, Chapel Hill, NC 27599.
\IEEEcompsocthanksitem Dakuo Wang is with the Khoury College of Computer Sciences and the College of Arts, Media and Design, Northeastern University, Boston, MA 02115.
\IEEEcompsocthanksitem Jinjun Xiong is with the Department of Computer Science and Engineering, University at Buffalo,
Buffalo, NY 14260.
\IEEEcompsocthanksitem Pin-Yu Chen is with the IBM Thomas J. Watson Research Center, NY 10598.
\IEEEcompsocthanksitem Sijia Liu is with the Department of Computer Science and Engineering,
Michigan State University, East Lansing, MI 48824.
\IEEEcompsocthanksitem Mohammad Shahidehpour is with the Department
of Electrical and Computer Engineering, Illinois Institute of Technology, Chicago,
IL 60616.
\IEEEcompsocthanksitem Alfred Hero is with the Electrical Engineering and Computer Science Department,
University of Michigan, Ann Arbor, MI 48109.}% <-this % stops an unwanted space
\thanks{The first two authors contributed equally to this paper.}
\thanks{Corresponding author: Ren Wang. E-mail: rwang74@iit.edu}
\thanks{Early versions of this work partially appeared in the conference proceedings  \cite{wang2023exploring} and \cite{wang2024deep}. %These works only present a small part of our results and only include a limited set of experiments.
}
\thanks{This work was supported in part by the National Science Foundation under grants CCF-2450414,  IIS-2246157, FMitF-2319243, by the Department of Energy under grant DE-CR0000042, and by the US Army Research Office under grant W911NF2310343.}
%\thanks{Under review at IEEE TPAMI.}
%\thanks{Manuscript received April 19, 2005; revised August 26, 2015.}
}






% The paper headers
\markboth{Journal of \LaTeX\ Class Files,~Vol.~14, No.~8, August~2021}%
{Shell \MakeLowercase{\textit{et al.}}: A Sample Article Using IEEEtran.cls for IEEE Journals}

% \IEEEpubid{0000--0000/00\$00.00~\copyright~2021 IEEE}
% Remember, if you use this you must call \IEEEpubidadjcol in the second
% column for its text to clear the IEEEpubid mark.

\maketitle

\begin{abstract}
Adversarial robustness is a critical measure of a neural network's ability to withstand adversarial attacks at inference time. While robust training techniques have improved defenses against individual $\ell_p$-norm attacks (e.g., $\ell_2$ or $\ell_\infty$), models remain vulnerable to diversified $\ell_p$ perturbations. To address this challenge, we propose a novel Robust Mode Connectivity (RMC)-oriented adversarial defense framework comprising two population-based learning phases. In Phase I, RMC searches the parameter space between two pre-trained models to construct a continuous path containing models with high robustness against multiple $\ell_p$ attacks. To improve efficiency, we introduce a Self-Robust Mode Connectivity (SRMC) module that accelerates endpoint generation in RMC. Building on RMC, Phase II presents RMC-based optimization, where RMC modules are composed to further enhance diversified robustness. To increase Phase II efficiency, we propose Efficient Robust Mode Connectivity (ERMC), which leverages $\ell_1$- and $\ell_\infty$-adversarially trained models to achieve robustness across a broad range of $p$-norms. An ensemble strategy is employed to further boost ERMC’s performance. Extensive experiments across diverse datasets and architectures demonstrate that our methods significantly improve robustness against $\ell_\infty$, $\ell_2$, $\ell_1$, and hybrid attacks. Code is available at \url{https://github.com/wangren09/MCGR}.
\end{abstract}

\begin{IEEEkeywords}
Robustness, deep learning, neural network, robust mode connectivity, adversarial training, population-based optimization.
\end{IEEEkeywords}

\section{Introduction}\label{sec:introduction}
The past decade has witnessed rapid advances in deep learning, leading to widespread adoption in high-stakes domains such as medical imaging \cite{sarvamangala2022convolutional}, defect detection \cite{jiwei2019bottom}, and power systems \cite{li2023physics}, where security is critical. Neural networks (NNs), the core of modern deep learning, learn complex mappings from data but remain highly sensitive to small, often imperceptible, input perturbations known as adversarial examples \cite{goodfellow2014explaining,wang2022ask}. Although nearly imperceptible to humans, these perturbations can cause severe model failures, raising serious concerns about the trustworthiness of NNs in safety-critical applications \cite{madry2018towards,carlini2017towards}. 


\begin{figure}[h]
  \centering
  \includegraphics[trim=0 0 0 0,clip,width=.49\textwidth]{Figures/RMC.png}
  \caption{{Overview of the Robust Mode Connectivity (RMC)-Oriented Adversarial Defense Framework}. The upper level of the panel at the top shows Phase I, illustrating that a robust path (robust to adversary types 1 and 2) in the parameter space can be found by connecting one model robust to adversary type 1 and the second model robust to adversary type 2. Selecting optimal points from the path and implementing the RMC process again can further improve robustness, as illustrated in the lower level of the panel at the top. Phase II suggests that more adversary types can be considered by using RMC as the basic unit. The right side panel at the bottom shows the efficient robust mode connectivity (ERMC), which interlaces $\ell_1$ and $\ell_\infty$ robustness into mode connectivity's structure and extends protection to perturbations from $\ell_p \in [1,\infty)$ norms. The left side panel at the bottom illustrates an ensembling method that can further boost the performance of the defense.} 
  \label{fig: framework}
\end{figure}

To address this vulnerability, adversarial training (AT) and its variants have become the most prominent defenses \cite{madry2018towards,zhang2019theoretically,shafahi2019adversarial,wang2020fast}. AT updates model parameters using adversarial examples generated on-the-fly from clean data, enabling the network to learn from adversarial distributions and become more robust during inference. However, most AT methods are designed for a single $\ell_p$ norm constraint (e.g., $\ell_\infty$), and their robustness often degrades sharply under perturbations from other norms \cite{tramer2019adversarial}. While recent works attempt to address this by training on multiple $\ell_p$ norms \cite{croce2019provable,stutz2020confidence,tramer2019adversarial,maini2020adversarial,croce2022adversarial,wang2021adversarial}, they often fall short due to the inherent limitations of single-point optimization in the model parameter space. These approaches can get trapped in local minima or saddle points when optimizing for multiple robustness objectives simultaneously.

In contrast, population-based optimization maintains a diverse set of candidate solutions, enabling broader exploration of the parameter space and better handling of complex, multi-objective tasks such as diversified $\ell_p$ robustness \cite{eiben2015evolutionary,diaz2016review,mirjalili2019evolutionary}. One particularly promising avenue is mode connectivity, which reveals that low-loss, high-accuracy paths often exist between independently trained models \cite{ren2025revisiting,garipov2018loss,freeman2017topology}. This property offers an accelerated population-based strategy for generating many viable models. However, naively applying mode connectivity is insufficient in adversarial settings.


In this work, we aim to improve a model’s robustness against perturbations constrained by different $\ell_p$ norms, with experimental focus on $p = 1, 2, \infty$. Motivated by the limitations of traditional approaches and the promise of mode connectivity, we propose a robust mode connectivity-oriented adversarial defense framework built on population-based optimization.

In Phase I, we introduce Robust Mode Connectivity (RMC), which finds high-robustness paths between adversarially trained models using a multi-steepest descent (MSD) algorithm \cite{maini2020adversarial}. To improve efficiency, we incorporate a Self-Robust Mode Connectivity (SRMC) module, which accelerates the creation of path endpoints. In Phase II, we construct RMC-based optimization, a broader framework that composes RMC modules to generate a population of candidate models and select those with the highest diversified robustness. Further, motivated by theoretical insights that affine classifiers robust to both $\ell_1$ and $\ell_\infty$ attacks can generalize to a wide range of $\ell_p$ threats, we propose Efficient Robust Mode Connectivity (ERMC). This method combines $\ell_1$- and $\ell_\infty$-robust models using a mode connectivity path and ensemble aggregation, boosting efficiency and robustness across norms (Fig.~\ref{fig: framework}).

\noindent\textit{Contributions.}
We summarize our main contributions as follows:

\begin{itemize}
    \item Robust Mode Connectivity (RMC): We propose RMC to construct paths between adversarially trained models, yielding intermediate models with high robustness to diversified $\ell_p$ perturbations. We further introduce Self-Robust Mode Connectivity (SRMC) to accelerate endpoint generation, improving the training efficiency of RMC.
(See Figures~\ref{fig: adv_mc}, \ref{fig: rmc_vgg16_cifar100}, \ref{fig: adv_self_rmc})
    \item RMC-Based Optimization: We extend RMC to a multi-stage population-based optimization framework that further improves robustness by selecting optimal models across multiple RMC units.
(Figures~\ref{fig: adv_mc_opt1}, \ref{fig: adv_mc_opt12}, \ref{fig: adv_mc_opt2})
    \item Efficient Robust Mode Connectivity (ERMC): We propose ERMC, a theoretically grounded method combining $\ell_1$ and $\ell_\infty$ robustness via mode connectivity and ensemble learning, to enhance the efficiency of the RMC-Based Optimization.
(Figure~\ref{fig: adv_self_rmc2})
    \item Comprehensive Evaluation: We conduct extensive experiments demonstrating that RMC, RMC-based optimization, and ERMC significantly outperform existing methods in achieving diversified $\ell_p$ robustness.
(Table~\ref{tab: main})
\end{itemize}

\noindent The rest of this article is organized as follows. Section~\ref{sec: related_work} introduces related works on defenses against diversified $\ell_p$ norm perturbations and population-based neural network learning. In Section~\ref{sec: pre}, we provide the definition of diversified $\ell_p$ robustness, and give introductions to adversarial attack, adversarial training, and mode connectivity. Sections~\ref{sec: rmc} and \ref{sec: opt} introduce the two phases of the proposed mode connectivity-oriented adversarial defense. The RMC method is presented in Section~\ref{sec: rmc}. The RMC-based optimization is proposed in Section~\ref{sec: opt}, and is enhanced by the ERMC method introduced in Section~\ref{sec: ermc} to improve its efficiency. Section~\ref{sec: exp} shows the experimental results. Section~\ref{sec: conclusion} concludes the article.



\section{Related Work}\label{sec: related_work}

\subsection{Adversarial Attacks}
Techniques such as the Fast Gradient Sign Method \cite{goodfellow2014explaining} and Projected Gradient Descent (PGD) \cite{madry2018towards} exploit the local gradient details of the target model to craft attacks. Building on PGD, output diversified sampling \cite{tashiro2020diversity} utilizes an enhanced initialization approach to create varied initial positions. However, these methods often provide inaccurate robustness measurements due to incorrect hyper-parameter tuning and gradient masking. To address this, Auto Attack (AA) \cite{croce2020reliable} combines four attack techniques with adjusted step sizes. To evaluate robustness under diversified $\ell_p$ norm perturbations simultaneously, Multi Steepest Descent (MSD) \cite{maini2020adversarial} incorporates various perturbation models within each step of the projected steepest descent, producing an adversary with a comprehensive understanding of the perturbation region. In this work, we consider PGD, AA, and MSD attacks to generate diversified $\ell_p$ norm perturbations.


\subsection{Adversarial Training-Based Defense}
The defense approach known as Adversarial Training (AT) \cite{madry2018towards} pioneered the use of min-max optimization for adversarial defense and has subsequently given rise to a plethora of other effective defense strategies. This includes the TRADES which delves into the trade-off between robustness and accuracy \cite{zhang2019theoretically}, dynamic adversarial training \cite{wang2019convergence}, and semi-supervised robust training approaches \cite{stanforth2019labels}. Furthermore, recent works, such as those by \cite{shafahi2019adversarial,wang2020fast,Wong2020Fast,zhang2019you}, have sought to develop faster, albeit approximate, AT algorithms. However, a common challenge across many of these methods is their concentration on a singular type of $\ell_p$ norm perturbation during AT. This specificity often culminates in a substantial decline in robustness when models are exposed to inputs with perturbations differing from the training set \cite{tramer2019adversarial}.

\subsection{Defenses on Diversified $\ell_p$ Norm Perturbations} 

Among all the works, \cite{croce2019provable} is the only one that provides a provable defense, and \cite{stutz2020confidence} considers withholding specific inputs to improve model resistance to stronger attacks. \cite{tramer2019adversarial} designs the inner loss by either selecting the type of perturbation that provides the maximum loss or averaging the loss across all types of perturbations. Extreme Norm Adversarial Training (E-AT) \cite{croce2022adversarial} leverages a fine-tuning strategy to improve robustness, while Multi Steepest Descent (MSD) Defense \cite{maini2020adversarial} incorporates various perturbation models within each step of the projected steepest descent to achieve diversified $\ell_p$ robustness. Nevertheless, despite their efforts, all the aforementioned works still depend on optimizing a single set of parameters, and the challenge of addressing the deficiency in diversified $\ell_p$ robustness remains unresolved. This work solves the challenge from a population-based optimization perspective.


\subsection{Population-Based Neural Network Learning} 
Optimizing a population of neural networks instead of a single network can prevent getting stuck at local minimums and lead to improved results. In one approach, \cite{jaderberg2017population} trained multiple instances of a model in parallel and selected the best performing instances to breed new ones. \cite{cui2018evolutionary} proposed an evolutionary stochastic gradient descent method that improved upon existing population-based methods. However, such methods typically have low learning speed and neglect adversarial robustness. Inspired by the human immune system, researchers have mimicked the key principles of the immune system in the inference phase to increase the robustness and not affect the learning speed in the training phase \cite{wang2022rails}. Mode connectivity can be treated as a faster population-based learning with two ancestor models that enhances the learning efficiency in the training phase \cite{garipov2018loss}. Researchers also analyze the mode connectivity when
networks are tampered with backdoor or error-injection attacks or under the attack of a single type of perturbation \cite{zhaobridging}. Our research extends beyond the scope of \cite{zhaobridging} by developing a novel robust, population-based optimization method for identifying candidate models with diversified $\ell_p$ robustness, and by exploring the phenomena of robust mode connectivity among different types of $\ell_p$ perturbations. Our unique approach not only facilitates the discovery of robust paths between two adversarially trained models but also generates candidates with enhanced robustness, thereby achieving state-of-the-art results in Diversified $\ell_p$ robustness. \textit{Our prior works laid the foundation for this study}: The workshop paper \cite{wang2023exploring} introduced Phase I robust path discovery but offered only limited experiments and lacked theoretical support, while the subsequent Deep Adversarial Defense \cite{wang2024deep} presented ERMC with preliminary validation. In this paper, we unify and extend these ideas into a full two-phase RMC framework, supported by SRMC, RMC-Based Optimization, refined ERMC, theoretical guarantees, and extensive experiments across datasets and architectures.



\section{Preliminaries}\label{sec: pre}


\subsection{Adversarial Attack with Different Input Perturbation Generators}
 
Recent studies indicate that conventional learning methods struggle with perturbed datasets ($\mathcal D_1,\mathcal D_2,\cdots, \mathcal D_S$) generated by
\begin{align}\label{eq: adv_atk}
    \displaystyle \arg\max {\mathcal L(\boldsymbol \theta; {\bf x}^\prime, y)}, ~~ s.t.~~ \text{Dist}_i({\bf x}^\prime, {\bf x}) \leq \delta_i, i \in [S]
\end{align}
for $\forall {\bf x} \in \mathcal D_0$, where $\mathcal D_0$ denotes the benign dataset, and $\delta_i$s are predefined bounds of perturbations corresponding to $\mathcal D_i$s with $i \in [S]$ (where $[S]=\{1, 2, \cdots, S\}$). In this paper, we restrict distance measures $\text{Dist}_i$s to be $\ell_p, p=1,2,\infty$ norms in our experiments. \eqref{eq: adv_atk} is typically termed an adversarial attack \cite{madry2018towards}. A practical approach to solving \eqref{eq: adv_atk} involves applying gradient descent and projection $P_{\delta_i}$ that maps the perturbation $\boldsymbol \epsilon_i=\bf x^\prime - \bf x$ to a feasible set, commonly referred to as a PGD attack. We use $\ell_p$-PGD to denote the PGD attack using the $\ell_p$ norm.

\subsection{Adversarial Training (AT)} 
The min-max optimization-based adversarial training (AT) is known as one of the most powerful defense methods to obtain a robust model against adversarial attacks \cite{madry2018towards}. We summarize AT below: 
\begin{align}
\begin{array}{ll}
    \displaystyle \min_{\boldsymbol \theta} \mathbb E_{(\mathbf x, y) \in \mathcal D_0} [ \displaystyle \max_{\text{Dist}_i({\bf x}^\prime, {\bf x}) \leq  \delta_i} 
 \mathcal L( \boldsymbol \theta; \mathbf x^\prime,  y ) ],
 \end{array}
 \label{eq: at}
\end{align}
Although AT can achieve relatively high robustness on $\mathcal D_i$, it does not generalize to other $\mathcal D_j, j\not=i$. Moreover, training on all $\mathcal D_i, i \in [S]$ is not scalable and will not provide robustness to all types of perturbations \cite{tramer2019adversarial}. We will use $\ell_p$-AT to denote the AT with the $\ell_p$ norm.



\subsection{Metric Definition: Diversified $\ell_p$ Robustness}

We hope that models can be robust to every $\ell_p$ adversarial type in the adversarial set of concerns, and we need to give a metric to measure such robustness. Diversified $\ell_p$ Robustness (DLR) is defined as its capacity to sustain the worst type of perturbation confined by a specific level of attack power:
\begin{mydef}\label{def_gr}

For a loss function $\mathcal L$, an input-output mapping function $f(\cdot)$, and a benign dataset $\hat{\mathcal D}_0$, the Diversified $\ell_p$ Robustness of a set of neural network parameters $\boldsymbol \theta$ is 
\begin{equation}\label{general_robustness_def}
\min_{i \in [S]} \frac{\sum_{({\bf x}^\prime, y)\in \hat{\mathcal D}_i} \boldsymbol{1}_{f({\bf x}^\prime, \boldsymbol \theta)= y}}{|\hat{\mathcal D}_i|},
\end{equation}
\end{mydef}
\noindent where $\hat{\mathcal D}_i$ represents the data generated by \eqref{eq: adv_atk} from $\hat{\mathcal D}_0$ with $\textup{Dist}_i$ as one of $\ell_p, p=1,2,\infty$ norms. We remark that there are other ways to define DLR. For example, the definition can be based on the worst-case sample-wise instead of worst-case data set-wise. Despite the difference, they essentially measure the same quantity, i.e., how well the model performs under various types of $\ell_p$ perturbations.


\subsection{Nonlinear Mode Connectivity}

\begin{figure}[h]
  \centering
  \includegraphics[trim=0 0 0 0,clip,width=.3\textwidth]{Figures/MC_Path1.png}
  \caption{{The path with near-constant loss found by mode connectivity in the parameter space}. The endpoints are two pre-trained models, and any point on the path represents a model.} 
  \label{fig: mc_path}
\end{figure}

Mode connectivity is a neural network's property that local minimums found by gradient descent methods are connected by simple paths belonging to the parameter space \cite{freeman2017topology,garipov2018loss}. Everywhere on the paths achieves a similar cost as the endpoints. The endpoints are two sets of neural network parameters $\boldsymbol \theta_1, \boldsymbol \theta_2 \in \mathbb{R}^{d}$ with the same structure and trained by minimizing the given loss $\mathcal L$. The smooth parameter curve is represented using $\phi(t; \boldsymbol \theta) \in \mathbb{R}^{d}, t \in [0,1]$, such that $\phi(0; \boldsymbol \theta)=\boldsymbol \theta_1, \phi(1; \boldsymbol \theta)=\boldsymbol \theta_2$. To find a desired low-loss path between $\boldsymbol \theta_1$ and $\boldsymbol \theta_2$, it is proposed to find parameters that minimize the following expectation over a uniform distribution on the curve:
\begin{equation}
\min_{\boldsymbol \theta} \mathbb E_{t \sim q(t; \boldsymbol \theta)}  \displaystyle \mathbb E_{({\bf x},y) \sim \mathcal D_0}  \mathcal L( \phi(t; \boldsymbol \theta); ({\bf x},y)),
\label{eq: mc_original}
\end{equation}
where $q(t; \boldsymbol \theta)$ represents the distribution for sampling the parameters on the path. Note that \eqref{eq: mc_original} is generally intractable. A computationally tractable surrogate is proposed as follows 
\begin{equation}
\min_{\boldsymbol \theta} \mathbb E_{t \sim U(0,1)}  \displaystyle \mathbb E_{({\bf x},y) \sim \mathcal D_0}  \mathcal L( \phi(t; \boldsymbol \theta); ({\bf x},y)),
\label{eq: mc}
\end{equation}
where $U(0,1)$ denotes the uniform distribution on $[0,1]$. Two common choices of $\phi(t; \boldsymbol \theta)$ in nonlinear mode connectivity are the Bezier curve \cite{farouki2012bernstein} and Polygonal chain \cite{gomes2012computer}. As an example, a quadratic Bezier curve is defined as $\phi(t; \boldsymbol \theta) = (1-t)^2 \boldsymbol \theta_1 + 2t(1-t)\boldsymbol \theta + t^2 \boldsymbol \theta_2$. Training neural networks on these curves provides many similar-performing models on low-loss paths. As shown in Fig.~\ref{fig: mc_path}, a quadratic Bezier curve obtained from \eqref{eq: mc} connects the upper two models along a path of near-constant loss.






\section{Two-Phase Robust Mode Connectivity}

\subsection{Phase I: Robust Path Search Via Robust Mode Connectivity}\label{sec: rmc}



\subsubsection{A Pilot Exploration}
Mode connectivity and adversarial training seem to be two excellent ideas for achieving high DLR that has been defined in Definition~\ref{def_gr}. If we set $\phi(0; \boldsymbol \theta)$ and $\phi(1; \boldsymbol \theta)$ to be two adversarially-trained neural networks under different types of perturbations, applying \eqref{eq: mc} may result in a path with points having high robustness for all these perturbations. Thus we ask:
\begin{tcolorbox}[before skip=2.0mm, after skip=2.0mm, boxsep=0.0cm, middle=0.1cm, top=0.1cm, bottom=0.1cm]
\noindent \textit{\textbf{(Q1)} Can simply combining adversarial training with mode connectivity provide high DLR?}
\end{tcolorbox}

Here we aim to see if implementing vanilla mode connectivity can bring us high DLR. We combine two PreResNet110 models \cite{he2016identity}, one trained with $\ell_\infty$-AT ($\delta=8/255$, 150 epochs) and the other trained with $\ell_2$-AT ($\delta=1$, 150 epochs), to find the desired path using the vanilla mode connectivity \eqref{eq: mc}. $\phi(0; \boldsymbol \theta)$ and $\phi(1; \boldsymbol \theta)$ are models trained by $\ell_\infty$-AT and $\ell_2$-AT, respectively. The mode connectivity curve is obtained with additional 50 epochs of training. The results are shown in Fig.~\ref{fig: std_mc}. The left (right) endpoint represents the model trained with $\ell_\infty$-AT ($\ell_2$-AT). One can see that the path has high loss and low robust accuracies on both types of attacks, indicating that vanilla mode connectivity fails to find the path that enjoys high DLR.

\begin{figure}[h]
  \centering
  \includegraphics[trim=0 0 0 0,clip,width=.46\textwidth]{Figures/vanilla_mc.png}
  \caption{{The vanilla mode connectivity \eqref{eq: mc} with models trained by $\ell_\infty$-AT and $\ell_2$-AT as two endpoints fails to find the path with high DLR}. $\phi(0; \boldsymbol \theta)$ and $\phi(1; \boldsymbol \theta)$ are $\ell_\infty$-AT ($\delta=8/255$, 150 epochs) and $\ell_2$-AT ($\delta=1$, 150 epochs). } 
  \label{fig: std_mc}
\end{figure}

\subsubsection{Embedding Adversarial Robustness to Mode Connectivity}

Although the vanilla mode connectivity aims to provide insight into the loss landscape geometry, it searches space following the original data distribution. Therefore it cannot provide high DLR by simply using two adversarially-trained models as two endpoints. Instead, we ask:
\begin{tcolorbox}[before skip=2.0mm, after skip=2.0mm, boxsep=0.0cm, middle=0.1cm, top=0.1cm, bottom=0.1cm]
\noindent \textit{\textbf{(Q2)} Can we develop a new method to embed adversarial robustness to mode connectivity by searching the adversarial input space?}
\end{tcolorbox}

To answer (Q2), we connect mode connectivity \eqref{eq: mc} with adversarial training under diversified $\ell_p$ adversarial perturbations. In other words, we modify the objective \eqref{eq: mc} to adjust to our high DLR purpose. An adversarial generator is added as an inner maximization loop. We adopt different types of perturbations in the generator. This is because a single type of perturbation may result in robustness bias. Formally, we obtain a model path $\phi(t; \boldsymbol \theta), t \in [0,1]$ parameterized by $\boldsymbol \theta$.  
\begin{equation}
\min_{\boldsymbol \theta} \mathbb E_{t \sim U(0,1)}  \displaystyle \mathbb E_{({\bf x},y) \sim \mathcal D_0 } \sum_{i \in I} \max_{\text{Dist}_i({\bf x}^\prime, {\bf x}) \leq  \delta_i} \mathcal L( \phi(t; \boldsymbol \theta); ({\bf x}^\prime,y)),
\label{eq: mc_adv}
\end{equation}
where $\phi(0; \boldsymbol \theta)$ and $\phi(1; \boldsymbol \theta)$ are two models trained by \eqref{eq: at}, probably under different types of perturbations. Throughout this paper, we fix the curve as a quadratic Bezier curve. Thus a model at the point $t$ can be represented by $\phi(t; \boldsymbol \theta) = (1-t)^2 \boldsymbol \theta_1 + 2t(1-t)\boldsymbol \theta + t^2 \boldsymbol \theta_2$. Similar to the nonlinear mode connectivity, \eqref{eq: mc_adv} is a computationally tractable relaxation by directly sampling $t$ from the uniform distribution $U(0,1)$ during the optimization. Data points $({\bf x}^\prime,y)$ are generated from a union of adversarial strategies $i \in I$, where $I$ is a subset of $\{1,2, 3, \cdots, S\}$. For example, data can be generated by using $\ell_2$ or $\ell_\infty$ norm distance measure, which is commonly used in adversarial attacks and adversarial training. Formulation in \eqref{eq: mc_adv} ensures that the identified path adapts to the targeted adversarial perturbations. 

We call \eqref{eq: mc_adv} the Robust Mode Connectivity (RMC), which serves as the first defense phase for robust path search. We remark that RMC itself is a defense method as we can select the model with the highest DLR in the path. One can see that a group of models (all points in the path) are generated from two initial models. Therefore RMC is a population-based optimization. 

\subsubsection{On the Benefits of Population-Based Framework}
Training a single model presents a fundamental challenge: different data points, and even the same point under varying adversarial norms, do not achieve peak robustness simultaneously. We therefore ask: 
\begin{tcolorbox}[before skip=2.0mm, after skip=2.0mm, boxsep=0.0cm, middle=0.1cm, top=0.1cm, bottom=0.1cm]
\noindent \textit{\textbf{(Q3)} Can a population-based framework address this issue by leveraging a large number of models?}
\end{tcolorbox}
Here, we define ``simultaneous'' robustness at the epoch-level, i.e., data points are considered to peak concurrently if they do so within the same training epoch. We posit that with a sufficiently large population of models, it is possible for $N$ data points to achieve peak robustness simultaneously under $S$ distinct adversarial norms.

\begin{theorem}
Let $T\geq 2$ be the number of training epochs. For each model $k$, each data-point/robustness-type pair $(i,s) \in \{1,\cdots, N\}\times \{1,\cdots, S\}$ has a random variable $X_{i,s}^{(k)} \in \{1,\cdots, T\}$ equal to the epoch at which that pair attains its highest $\ell_p$ robustness in model $k$. Assume the arrays $\{X_{i,s}^{(k)}\}_{i,s}$ are i.i.d. over $k$, the $X_{i,s}^{(k)}$ are mutuallly independent across $(i,s)$, and $\text{Pr}[X_{i,s}^{(k)}=t]=\frac{1}{T}$ for every $t \in \{1,\cdots, T\}$ for all $(i,s)$, then:

\noindent For any target confidence $1-\gamma \in (0,1)$, the minimum number of models that achieves $Pr$(at least one alignment among the $K$ models)$\geq 1-\gamma$ is $K=\lceil{\frac{\ln{\gamma}}{\ln{1-T^{-(NS-1)}}}\rceil}$.
\end{theorem}
The alignment event in each model has the probability $T^{-(NS-1)}$, so the probability of no alignment in the $K$ models is $1-T^{-(NS-1)}$. The proof is done by requiring $(1-T^{-(NS-1)})^K\le \gamma$. Note that the independence assumption may not hold in practice. Nevertheless, our objective here is solely to demonstrate the inherent advantages of the population-based method over a single model solution.

With the problem formulation and the theoretical support, the next step is to find out how to solve the RMC \eqref{eq: mc_adv}.


\begin{algorithm}[h]
\caption{Robust Mode Connectivity}
\label{alg: RMC}
\begin{algorithmic}[1]
\REQUIRE $\phi(0; \boldsymbol \theta)$, $\phi(1; \boldsymbol \theta)$ - two selected models with the same structure (potentially trained with different strategies, e.g., AT under different perturbation types); initial model $\boldsymbol \theta^0$; the perturbation types $i \in I$ and the corresponding projections $P_{\delta_i}$; training set $\mathcal D_0$; inner loop iteration number $J$; batch size $B$; initial perturbation $\epsilon^{(0)}=\mathbf 0$.
\STATE{$\boldsymbol \theta = \boldsymbol \theta^0$}
\FOR{each data batch $\mathcal D_b \in \mathcal D_0$ in each epoch $e \in E$}
\STATE{Uniformly select $t \sim U(0,1)$}
\FOR{$\forall \bf x \in \mathcal D_b$}
\FOR{$j = 1, \cdots, J$}
\FOR{$i \in I$}
\STATE{\hspace{-1mm}$\boldsymbol{\epsilon}_i^{(j)} \leftarrow P_{\delta_i}\big(\boldsymbol{\epsilon}^{(j-1)}-{\nabla_{\boldsymbol{\epsilon}}\mathcal L(\phi(t; \boldsymbol{\theta}); {\bf{x}}+{\boldsymbol{\epsilon}}^{(j-1)},y)}\big)$}
\ENDFOR
\STATE{$\boldsymbol \epsilon^{(j)} \leftarrow \arg\max_{\boldsymbol \epsilon_i^{(j)}, i \in I} {\mathcal L(\phi(t; \boldsymbol \theta); {\bf{x}} + \boldsymbol \epsilon_i^{(j)}, y)}$}
\ENDFOR
\ENDFOR
\STATE{$\boldsymbol \theta \leftarrow \boldsymbol \theta - \nabla_{\boldsymbol \theta} \sum_{\bf x \in \mathcal D_b} \mathcal L(\phi(t; \boldsymbol \theta); {\bf x} + \boldsymbol \epsilon^{(j-1)}, y)$}
\ENDFOR
\RETURN $\boldsymbol \theta$, $\phi(t; \boldsymbol \theta), \forall t \in [0,1]$
\end{algorithmic}
\end{algorithm}


\begin{figure*}[ht]
  \centering
             \subfloat[Epoch=50]{\includegraphics[trim=90 5 90 100,clip,width=0.32\textwidth]{Figures/pgd-pgd2-50.png}}
             \subfloat[Epoch=100]{\includegraphics[trim=90 5 90 100,clip,width=0.32\textwidth]{Figures/pgd-pgd2-100.png}}
             \subfloat[Epoch=150]{\includegraphics[trim=90 5 90 100,clip,width=0.32\textwidth]{Figures/pgd-pgd2-150.png}}
  \caption{{The RMC \eqref{eq: mc_adv} with models trained by $\ell_\infty$-AT and $\ell_2$-AT as two endpoints can find the path with high DLR}. MSD \cite{maini2020adversarial} with perturbations generated by $\ell_2$ and $\ell_\infty$ norm distance measures is leveraged as the inner solver. Solving \eqref{eq: mc_adv} uses 50/100/150 epochs in panel (a)/(b)/(c).} 
  \label{fig: adv_mc}
\end{figure*}


\subsubsection{Solving Robust Mode connectivity Via Multi Steepest Descent}

Solving \eqref{eq: mc_adv} is difficult as it contains multi-type perturbations. The simplest ways are using `MAX' or `AVG' strategy proposed in \cite{tramer2019adversarial}, where the inner loss is obtained by selecting the type of perturbation that provides the maximum loss or averaging the loss on all types of perturbations. However, both strategies consider perturbations independently. We leverage a Multi Steepest Descent (MSD) approach that includes the various perturbation models within each step of the projected steepest descent in order to produce a PGD adversary with complete knowledge of the perturbation region \cite{maini2020adversarial}. The key idea is to simultaneously maximize the worst-case loss overall perturbation models at each step. Algorithm~\ref{alg: RMC} shows the details, where all the perturbation types are considered in each iteration. The complexity order remains consistent when we juxtapose the Algorithm with the conventional AT. This is primarily because the number of perturbations $I$ is essentially constant in our scenarios (specifically, $I=2$ or $3$). Next we test the effectiveness of the proposed RMC algorithm.

We again use models trained with $\ell_\infty$-AT ($\delta=8/255$, 150 epochs) and $\ell_2$-AT ($\delta=1$, 150 epochs) as two endpoints $\phi(0; \boldsymbol \theta)$ and $\phi(1; \boldsymbol \theta)$. The RMC \eqref{eq: mc_adv} with MSD as the inner solver is applied to obtain the path. Fig.~\ref{fig: adv_mc} shows the results of training an additional 50/100/150 epochs with $D_{i}$s generated by $\ell_2$ and $\ell_\infty$ norm distance measures. One can find that unlike Fig.~\ref{fig: std_mc}, the paths contain points with high accuracy and high robustness against both $\ell_\infty$-PGD and $\ell_2$-PGD attacks. Although the left endpoint has low $\ell_2$ robustness and the right endpoint has relatively low $\ell_\infty$ robustness, they can achieve high DLR in the connection, where the highest DLR is $48.19\%$ in panel (a). One can also notice that when the epoch number for solving \eqref{eq: mc_adv} increases, the path becomes smoother. One can see that the robust paths also function as effective mode connectivity paths, where both the clean accuracy and loss (indicated by red lines) maintain consistent levels between the two endpoints $t=0$ and $t=1$ in panels (c). We also observe that the optimal points in panels (a), (b), and (c) yield similar DLR. The experiments indicate that RMC can find a path with high DLR. If the goal is to select an optimal model from the path, it is enough to only conduct the training with a small number of epochs.

\subsubsection{Improving Learning Efficiency of the RMC With a Self-Robust Mode Connectivity Module}\label{sec: srmc}

One drawback of the previous scheme is that it needs to initially pre-train two neural networks, which could be slow when the computational resources are limited. We ask:
\begin{tcolorbox}[before skip=2.0mm, after skip=2.0mm, boxsep=0.0cm, middle=0.1cm, top=0.1cm, bottom=0.1cm]
\noindent \textit{\textbf{(Q4)} How can we accelerate the learning process of the RMC?}
\end{tcolorbox}

Here we propose to replace RMC with a self-robust mode connectivity (SRMC) module in the learning process. SRMC can accelerate the endpoints training in the path search process and thus speed up RMC. We start with one set of neural network parameters $\phi(0; \boldsymbol \theta)=\boldsymbol \theta_1$ that is trained by \eqref{eq: at} with a fixed $\text{Dist}_i$. After the model achieves high robustness on $\mathcal D_i$, we retrain the model for a few epochs using \eqref{eq: at} with $\text{Dist}_j$. The new model we obtained will be placed at the endpoint $\phi(1; \boldsymbol \theta)=\boldsymbol \theta_2$. Now the low-loss high-robustness path can be found by optimizing \eqref{eq: mc_adv}. By leveraging the SRMC module, our proposed framework yields both high DLR and learning efficiency.











\subsection{Phase II: Robust Model Selection Via Robust Mode Connectivity-Based Optimization}\label{sec: opt}

\begin{algorithm}[h]
\caption{Robust Mode Connectivity-Based Optimization ($\ell_1, \ell_2, \ell_\infty$ perturbations)}
\label{alg: RMC_opt}
\begin{algorithmic}[1]
%\REQUIRE .
\STATE{Train three models for $T$ epochs using $\ell_1, \ell_2, \ell_\infty$ perturbations, respectively. (Training can be accelerated using the SRMC module proposed in Section~\ref{sec: srmc})}
\STATE{Apply Algorithm~\ref{alg: RMC} with $\ell_2, \ell_1$-AT trained models ($I$ includes $\ell_2, \ell_1$) and $\ell_\infty, \ell_1$-AT trained models ($I$ includes $\ell_\infty, \ell_1$) as $\phi(0; \boldsymbol \theta)$, $\phi(1; \boldsymbol \theta)$, and return model trajectories $\phi_{\boldsymbol \theta-\ell_\infty}(t), \phi_{\boldsymbol \theta-\ell_2}(t), \forall t \in [0,1]$. (pairs of perturbations can be selected in different ways)}
\STATE{Randomly pick points $t_{-\ell_\infty}$, $t_{-\ell_2}$ from  optimal regions for each model trajectory.}
\STATE{Train models for $T$ epochs using $\ell_\infty, \ell_2$ perturbations starting from $\phi_{\boldsymbol \theta-\ell_\infty}(t_{-\ell_\infty}), \phi_{\boldsymbol \theta-\ell_2}(t_{-\ell_2})$, respectively.}
\STATE{Apply Algorithm~\ref{alg: RMC} with the two models as $\phi(0; \boldsymbol \theta)$, $\phi(1; \boldsymbol \theta)$ with $I$ including $\ell_1, \ell_2, \ell_\infty$ perturbations.}
\STATE{Find the optimal point $t_{\text{opt}}$ from the model trajectory.}
\RETURN $\phi_{\boldsymbol \theta}(t_{\text{opt}})$
\end{algorithmic}
\end{algorithm}

\subsubsection{From RMC to RMC-Based Optimization}

Suppose we have neural networks that share the same structure but are trained with different settings, e.g., different types of perturbations, perturbation magnitudes, learning rates, batch size, etc. In that case, we can use RMC to search for candidates potentially leading us to better solutions or even global optimums. The intuition behind the claim is that low-loss \& high-DLR paths connect all the minimums, and thus it becomes easier for search algorithms to jump out of the sub-local minimums. We have seen the exciting property of the proposed RMC, which indicates that a larger population of NNs can result in higher DLR. Notice that RMC can serve as a component in a larger population-based optimization to select robust models with higher DLR. A natural question to ask is:
\begin{tcolorbox}[before skip=2.0mm, after skip=2.0mm, boxsep=0.0cm, middle=0.1cm, top=0.1cm, bottom=0.1cm]
\noindent \textit{\textbf{(Q5)} Can we develop a general population-based optimization method built on RMC modules to further improve the DLR of a single RMC?}
\end{tcolorbox}


The RMC-based optimization we developed below includes an evolving process of RMC units for multiple generations. As a starting point, we generate an initial population by training neural networks with data points augmented using different $\text{Dist}_i$ in \eqref{eq: at}. 
We use gradient descent to train these networks but pause the training when specific stop criteria have been met, e.g., the number of epochs or accuracy achieving the preset threshold. The initial population then varies, and the system selects candidates with the best performances. The two operations in our approach are unified through the RMC that connects two adversarially-trained neural network models on their loss landscape using a high-accuracy \& high-DLR path characterized by a simple curve. Candidates for the next generation are selected among the high DLR points on the curve. The process can be repeated and an optimal solution that enjoys the highest DLR is selected among the final candidates. 

Algorithm~\ref{alg: RMC_opt} shows the pipeline using an example of three types of perturbations. We first train three models with $\ell_\infty$-AT, $\ell_2$-AT, and $\ell_1$-AT for $T$ epochs. We then connect the $\ell_2$-AT model with the $\ell_1$-PGD model and connect the $\ell_\infty$-AT model with the $\ell_1$-AT model using the RMC for some additional epochs. The two model trajectories are denoted by $\phi_{\boldsymbol \theta-\ell_\infty}(t)$ and $\phi_{\boldsymbol \theta-\ell_2}(t), \forall t \in [0,1]$. Notice that the curves will not be perfectly flat. But there exist some regions containing points with high DLR. We will randomly pick a model from a small optimal region in each curve. In practice, we will find the point with the highest DLR and randomly pick a point around the optimal point. The rationale behind this is to increase diversity during the training. After obtaining the models $\phi_{\boldsymbol \theta-\ell_\infty}(t_{-\ell_\infty})$ and $ \phi_{\boldsymbol \theta-\ell_2}(t_{-\ell_2})$ from both trajectories, the two new endpoints are obtained by training each model $T$ epochs using the $\ell_p$-AT that is different from the types used in the previous RMC. In this specific case, we use $\ell_\infty$-AT and $\ell_2$-AT. Finally, we connect the two new endpoints with RMC for some additional epochs and find the new optimum $\phi_{\boldsymbol \theta}(t_{\text{opt}})$ at $t_{\text{opt}}$. In the case of two types of perturbations, one can start to train two models from a single optimal point or train one model from each of the two optimal points. We refer readers to Section~\ref{subsec: rmc_opt} for more details. It's pertinent to note that parameter curves derived from distinct models can be concurrently computed. This inherent parallelizability means that when we leverage parallel computing for generating independent parameter curves, the execution time is equivalent to the time of generating one parameter curve. In a more general scenario where there are $S$ types of perturbations, the process is the same, except that it contains more RMC units, as illustrated in Fig.~\ref{fig: framework}. We learn optimal points from pairs of models trained by AT under different perturbations and finally find an optimal point with the highest DLR.



\section{Enhancing Phase II efficiency: ERMC With Model Ensemble}\label{sec: ermc}

From the insights of the previous Phase II, it becomes apparent that to enhance robustness against diversified $\ell_p$ perturbations, multiple RMC procedures might be necessary. We pose the question:
\begin{tcolorbox}[before skip=2.0mm, after skip=2.0mm, boxsep=0.0cm, middle=0.1cm, top=0.1cm, bottom=0.1cm]
\noindent \textit{\textbf{(Q6)} Can enhanced robustness against diversified $\ell_p$ perturbations be attained within a single RMC process?}
\end{tcolorbox}

In the literature \cite{croce2019provable}, it is discussed that affine classifiers, including CNNs with ReLU activations, can withstand various $\ell_p$ norm attacks if they are already fortified against $\ell_1$ and $\ell_\infty$ perturbations. Theorem 3.1 in ~\cite{croce2019provable} posits that the convex hull of the union ball of the $\ell_1$ and $\ell_\infty$ provides satisfactory robustness to $\ell_p, 1 \le p\le \infty$ perturbationsTheorem 3.1 in ~\cite{croce2019provable} posits that the convex hull of the union ball of the $\ell_1$ and $\ell_\infty$ provides satisfactory robustness to $\ell_p$ perturbations, where $1 \le p\le \infty$. 

\begin{theorem}
    \cite{croce2019provable} Suppose that the classifier is affine. Let $C$ be the convex hull of the union ball of the $\ell_1$ and $\ell_\infty$. If the input dimension $d_{\mathbf x}$ is larger than or equal to two and $\delta_1 \in (\delta_\infty, d_{\mathbf x}\delta_\infty)$, then
    \begin{equation}
  \min_{\mathbb{R}^{d_{\mathbf x}}\backslash C} \|\mathbf x^\prime - \mathbf x\|_p = \frac{\boldsymbol \delta_1}{(\boldsymbol \delta_1/\boldsymbol \delta_\infty - \beta + \beta^q)^{1/q}}
    \end{equation}
where $\beta=\frac{\boldsymbol \delta_1}{\boldsymbol \delta_\infty} - \lfloor \frac{\boldsymbol \delta_1}{\boldsymbol \delta_\infty}\rfloor$ and $\frac{1}{p}+\frac{1}{q}=1$.
\end{theorem}

A recent approach, E-AT \cite{croce2022adversarial}, proposes using fine-tuning as an efficient transition from $\ell_\infty$-adversarial training (AT) to $\ell_1$-AT, aiming to improve robustness against a broader range of $\ell_p$ attacks. However, this method faces two key limitations: \ding{182} the fine-tuning process may compromise the model's original robustness to $\ell_\infty$ attacks; and \ding{183} a single model often lacks sufficient capacity to maintain strong robustness against both $\ell_\infty$ and $\ell_1$ perturbations simultaneously.

Our proposed RMC can naturally overcome these issues thanks to the power of population-based strategies. Here we propose an efficient robust mode connectivity (ERMC) strategy, leveraging SRMC to fine-tune endpoint $\phi(1; \boldsymbol \theta)$ with $\ell_1$-AT from $\phi(0; \boldsymbol \theta)$ obtained by $\ell_\infty$-AT. We then optimize the following objective to maintain robustness against both $\ell_1$ and $\ell_\infty$ attacks, effectively expanding the defense boundary and improving overall model resilience:  
\begin{equation}
\begin{aligned}
\min_{\boldsymbol \theta} &\mathbb E_{t \sim U(0,1)}   \displaystyle \mathbb E_{({\bf x},y) \sim \mathcal D_0 } \{\\&\sum_{\text{Dist}_i \in \{\|\cdot\|_1,\|\cdot\|_\infty\}} \max_{\text{Dist}_i({\bf x}^\prime, {\bf x}) \leq  \delta_i} \mathcal L( \phi(t; \boldsymbol \theta); ({\bf x}^\prime,y))\},
\label{eq: ermc_adv}
\end{aligned}
\end{equation} 
which results in a larger union ball, thereby enhancing the model's resilience against a broader range of perturbations. The detailed algorithm can be found in Algorithm~\ref{alg: ERMC}. ERMC is efficient as it only needs to conduct the connection once.

\begin{algorithm}[h]
\caption{Efficient Robust Mode Connectivity}
\label{alg: ERMC}
\begin{algorithmic}[1]
\REQUIRE A model $\phi(0; \boldsymbol \theta)$ trained with $\ell_\infty$-AT; initial model $\boldsymbol \theta^0$; the corresponding projections $P_{\boldsymbol{\delta}_1}$ and $P_{\boldsymbol{\delta}_\infty}$; training set $\mathcal D_0$; iteration number $J$; batch size $B$; initial perturbation $\boldsymbol{\delta}^{(0)}=\mathbf 0$.
\STATE{Create a copy of $\phi(0; \boldsymbol \theta)$ and retrain it with AT-$\ell_1$ for 10 epochs to obtain a model $\phi(1; \boldsymbol \theta)$.}
\STATE{$\boldsymbol \theta = \boldsymbol \theta^0$}
\FOR{each data batch $\mathcal D_b \in \mathcal D$ in each epoch $e \in E$}
\STATE{Uniformly select $t \sim U(0,1)$}
\FOR{$\forall \bf x \in \mathcal D_b$}
\FOR{$j = 1, \cdots, J$}
\STATE{\hspace{-1mm}$\boldsymbol{\delta}_1^{(j)} \leftarrow P_{\boldsymbol \epsilon_1}\big(\boldsymbol{\delta}^{(j-1)}-{\nabla_{\boldsymbol{\delta}}\mathcal L( \phi(t; \boldsymbol \theta); {\bf{x}}+{\boldsymbol{\delta}}^{(j-1)},y)}\big)$}
\STATE{\hspace{-1mm}$\boldsymbol{\delta}_\infty^{(j)} \leftarrow P_{\boldsymbol \epsilon_\infty}\big(\boldsymbol{\delta}^{(j-1)}-{\nabla_{\boldsymbol{\delta}}\mathcal L( \phi(t; \boldsymbol \theta); {\bf{x}}+{\boldsymbol{\delta}}^{(j-1)},y)}\big)$}
\ENDFOR
\STATE{$\boldsymbol \delta^{(j)} \leftarrow \arg\max_{\boldsymbol \delta_i^{(j)}, i \in \{1,\infty\}} {\mathcal L( \phi(t; \boldsymbol \theta); {\bf{x}} + \boldsymbol \delta_i^{(j)}, y)}$}
\ENDFOR
\STATE{$\boldsymbol \theta \leftarrow \boldsymbol \theta - \nabla_{\boldsymbol \theta} \sum_{\bf x \in \mathcal D_b} \mathcal L(\phi(t; \boldsymbol \theta); {\bf x} + \boldsymbol \delta^{(j)}, y)$}
\ENDFOR
\RETURN $\boldsymbol \theta$, $\phi(t; \boldsymbol \theta), \forall t \in [0,1]$
\end{algorithmic}
\end{algorithm}

Acknowledging the presence of numerous models along the trajectory that demonstrate significant resistance to $\ell_\infty$ and $\ell_1$ attacks, adopting a model ensemble technique seems a logical step to enhance robustness. By doing so, we create an aggregated model with a collective defense against both $\ell_\infty$ and $\ell_1$ disruptions. The process for selecting the ensemble involves identifying a segment $t\in [a,b]$ on the path $\phi(t; \boldsymbol \theta)$ where each point on the segment has robust accuracies surpassing two prefixed model selection thresholds $\alpha_\infty, \alpha_1$ under $\ell_\infty$ and $\ell_1$ attacks, respectively. From this segment, we select $n > 1$ models situated at intervals defined by $t = a + \frac{b - a}{n - 1}i$, with $i$ varying from $0$ to $n - 1$. In scenarios where there are several non-adjacent intervals that fulfill the selection criteria, the models are proportionally allocated based on the length of these intervals. This approach, with $n$ models chosen, is referred to as ERMC-$n$. We then calculate the class probability prediction by averaging the outputs from the final layers of these $n$ models.



\section{Experimental Results}\label{sec: exp}


Figures~\ref{fig: std_mc}, \ref{fig: adv_mc} show that using the proposed RMC can find a path with points in it enjoying high robustness on diversified $\ell_p$ perturbations. In this section, we conduct more comprehensive experiments on the Robust Mode Connectivity-Oriented Adversarial Defense.


\subsection{Settings}

We evaluate our proposed methods using the CIFAR-10, CIFAR-100 \cite{krizhevsky2009learning}, and ImageNet-100 \cite{russakovsky2015imagenet} datasets across the PreResNet110, WideResNet-28-10, and Vision Transformer-base architectures. By default, we conduct experiments on CIFAR-10 and PreResNet110. For our experiments, the considered perturbation types, denoted as $\textup{Dist}_i$s, are based on $\ell_\infty$, $\ell_2$, and $\ell_1$ norms, constrained by perturbations of $\delta = 8/255, 1$, and $12$, respectively. To obtain the endpoints' models, we employ AT. Our methods are benchmarked against the standard $\ell_\infty$-AT baseline, RMC on two randomly initialized models (RMC-RI), the Extreme norm Adversarial Training (E-AT) \cite{croce2022adversarial}, and the MSD Defense \cite{maini2020adversarial}. The evaluation methods encompass basic PGD adversarial attacks and Auto-Attack (AA) \cite{croce2020reliable} under $\ell_\infty$, $\ell_2$, $\ell_1$ norm perturbations and the MSD attack. The evaluation metrics include: (1) Standard accuracy on clean test data; (2) Robust accuracies under $\ell_\infty$, $\ell_2$, $\ell_1$-PGD adversarial attacks, MSD attack, and $\ell_\infty/\ell_2/\ell_1$ AA; (3) Accuracy on worst-case sample-wise (Union) using all three basic PGD adversarial attacks; and (4) DLR on $\ell_\infty$, $\ell_2$, $\ell_1$-PGD adversarial attacks for three types of perturbations and DLR on $\ell_\infty$, $\ell_2$ for two types of perturbations. All the following experiments are conducted on two NVIDIA RTX A100 GPUs.

\begin{figure*}[h]
  \centering
\subfloat[CIFAR100]{\includegraphics[trim=0 0 0 0,clip,width=0.25\textwidth]{Figures/cifar100.png}}
\subfloat[ImageNet-100]{\includegraphics[trim=0 0 0 0,clip,width=0.25\textwidth]{Figures/imagenet100.png}}
\subfloat[WideResNet-28-10]{\includegraphics[trim=0 0 0 0,clip,width=0.25\textwidth]{Figures/wide.png}}
\subfloat[Vision Transformer-base]{\includegraphics[trim=0 0 0 0,clip,width=0.25\textwidth]{Figures/vit.png}}
  \caption{{RMC is capable of identifying paths with points that have high DLR across various datasets and model architectures}. Figures (a) and (b), as well as (c) and (d), demonstrate that RMC performs effectively on the CIFAR-100 and ImageNet-100 datasets, as well as the WideResNet-28-10 and Vision Transformer-base architectures.}  
  \label{fig: rmc_vgg16_cifar100}
\end{figure*}

\subsection{A More Comprehensive Study of the Robust Mode Connectivity}



In this subsection, we aim to study the effectiveness of the proposed method \eqref{eq: mc_adv} on different models, architectures, and datasets. We will consider models trained under various settings. By default, we train endpoints' models 50/150 epochs and the paths are obtained by training an additional 50 epochs.


Here we evaluate the effectiveness of RMC on the CIFAR-100 and ImageNet-100 datasets, as well as the WideResNet-28-10 and Vision Transformer-base model architectures. We consider two types of perturbations that are generated from $\ell_\infty$ and $\ell_2$-PGD attacks. It can be seen from Fig.~\ref{fig: rmc_vgg16_cifar100} (a) and (b) that paths with high DLR points are obtained when CIFAR-10 is replaced with CIFAR-100 and ImageNet-100. Similarly, Fig.~\ref{fig: rmc_vgg16_cifar100} (c) and (d) demonstrate that paths with high DLR points are obtained when PreResNet110 is replaced with WideResNet-28-10 and Vision Transformer-base. These results underscore that RMC is versatile and can be effectively applied to various datasets and architectures.


\subsection{RMC with SRMC Modules}

We then tested the proposed SRMC modules to expedite the RMC. Starting with a $\ell_\infty$-AT model, we trained an additional $\ell_2$-AT model and a $\ell_1$-AT model over 5 epochs. Subsequently, we connected each of these child models with the original $\ell_\infty$-AT model. The results, depicted in Fig.~\ref{fig: adv_self_rmc}, demonstrate the presence of paths with regions of high robustness under both connections. This indicates that we don't need to train all models from scratch to obtain the desired paths. For our Phase I experiments using PreResNet110 on CIFAR-10 with a single GPU, the process of learning RMC, which involved training two endpoint AT models for 150 epochs and the parameter curve for 50 epochs, took an average of $2750$ minutes. Learning SRMC under identical settings took $1780$ minutes. In the Vision Transformer setup with one GPU, learning RMC averaged $5785$ minutes, whereas learning SRMC in the same conditions required $3592$ minutes.

\begin{figure}[h]
  \centering
  \includegraphics[trim=0 0 0 0,clip,width=.34\textwidth]{Figures/selfrmc_infty_l2.png}
  \caption{{A single SRMC module can also find paths with high DRL by connecting a $\ell_\infty$ model and a $\ell_2$ model}. } 
  \label{fig: adv_self_rmc}
\end{figure}




\begin{figure}[h]
  \centering
  \includegraphics[trim=0 0 0 0,clip,width=.46\textwidth]{Figures/rmc_2pert_opt_1p.png}
  \caption{{RMC-based optimization considering two types of perturbations (one single mid-optimal point) can result in paths with smoother and higher DLR than the path in Fig.~\ref{fig: adv_mc} left panel}. The left (right) endpoint is an $\ell_\infty$-AT ($\ell_2$-AT) trained model starting from a single optimal point of a path connected between two models, which are trained by $\ell_\infty$-AT and $\ell_2$-AT for 50 epochs.} 
  \label{fig: adv_mc_opt1}
\end{figure}

\subsection{Robust Mode Connectivity-Based Optimization}\label{subsec: rmc_opt}

As introduced in Section~\ref{sec: opt}, Phase II is an enhanced optimization process based on units of RMC (Phase I). We show the effectiveness of RMC-based optimization (Phase II) below. Training epochs for all the experiments below are 200 (allow parallel computing).



\noindent\textbf{Optimization on two types of perturbations.} We first consider $\ell_\infty$ and $\ell_2$ norm perturbations. We train two models for 50 epochs under these two types of perturbations, then leverage RMC to find a path between the two models. Initializing from a single optimal point (randomly select from $t\in [0.77,0.83]$) on the curve, we train two models parallelly with $\ell_\infty$-AT and $\ell_2$-AT for 50 epochs. Finally, we plot the mode connectivity curve based on the two AT-trained models, as shown in Fig.~\ref{fig: adv_mc_opt1}. We obtain a smoother path with higher DLR than the path in Fig.~\ref{fig: adv_mc} left panel. The optimal point achieved is  $48.8\%$ at $t=0.72$.



Now instead of selecting a single optimal point, we randomly pick two optimal points in the ranges of $t\in [0.27,0.33]$ and $t\in [0.77,0.83]$, respectively. We train two models with $\ell_\infty$-AT and $\ell_2$-AT for 50 epochs starting from each initial point. We then plot the RMC curve based on the two AT-trained models, as shown in Fig.~\ref{fig: adv_mc_opt12}. The drop in accuracy observed at approximately $t=0.5$ is attributed to the small number of epochs (50) used in RMC. Increasing the number of epochs would result in smoother curves. The optimal point achieved in this optimization process is 48.89\% (DLR) at $t=0.71$, indicating that higher robustness can be improved by using a larger population with higher diversity.

\begin{figure}[h]
  \centering
  \includegraphics[trim=0 0 0 0,clip,width=.46\textwidth]{Figures/rmc_2pert_opt.png}
  \caption{{RMC-based optimization considering two types of perturbations with two mid-optimal points is able to achieve higher robustness compared with only selecting a single mid-optimal point}. The left (right) endpoint is an $\ell_\infty$-AT ($\ell_2$-AT) trained model starting from two optimal points of a path connected between two models, which are trained by $\ell_\infty$-AT and $\ell_2$-AT for 50 epochs.} 
  \label{fig: adv_mc_opt12}
\end{figure}


\noindent\textbf{Optimization on three types of perturbations.} We take one more step by considering the $\ell_1$ norm perturbation. The process is shown in Algorithm~\ref{alg: RMC_opt}. $T=50$ and we use 50 additional epochs to learn RMC. The results of the final connection are shown in Fig.~\ref{fig: adv_mc_opt2}. The trend of the $\ell_1$-PGD curve is increasing from left to right and the trend of the $\ell_2$-PGD curve from $t=0.7$ to $t=1$ is decreasing. There exists an optimal point with DLR$=46.21\%$ at $t=0.93$. RMC-based optimization in the case of three types of perturbations can further boost models' DLR against $\ell_\infty$, $\ell_1$, $\ell_2$ adversarial attacks. Additionally, one can select multiple models from the curve and use ensemble methods to further improve performance. 

\begin{figure}[h]
  \centering
  \includegraphics[trim=0 0 0 0,clip,width=.34\textwidth]{Figures/rmc_3pert_opt.png}
  \caption{{RMC-based optimization considering three types of perturbations can further boost models' DLR against more types of attacks}. The two endpoints are trained by $\ell_\infty$-AT and $\ell_2$-AT for 50 epochs starting from the optimal points selected from two RMC curves.} 
  \label{fig: adv_mc_opt2}
\end{figure}





\begin{figure}[ht]

\begin{minipage}[b]{.48\linewidth}
  \centering
  \centerline{\includegraphics[width=5.0cm]{Figures/prs-150-10-50.png}}
  \centerline{(a) CIFAR-10}\medskip
\end{minipage}
\begin{minipage}[b]{.48\linewidth}
  \centering
  \centerline{\includegraphics[width=5.0cm]{Figures/prs100-150-10-50.png}}
  \centerline{(b) CIFAR-100}\medskip
\end{minipage}
\begin{minipage}[b]{.48\linewidth}
  \centering
  \centerline{\includegraphics[width=5.0cm]{Figures/wr-150-10-50.png}}
  \centerline{(c) WideResNet-28-10}\medskip
\end{minipage}
\hfill
\begin{minipage}[b]{0.48\linewidth}
  \centering
  \centerline{\includegraphics[width=5.0cm]{Figures/vit_pgd_pgd1.png}}
  \centerline{(d) ViT-base}\medskip
\end{minipage}
\caption{ERMC can find paths with high robustness against $\ell_\infty/\ell_2/\ell_1$ attacks by connecting a $\ell_\infty$ model and a $\ell_1$ model. The effectiveness of ERMC is validated on different datasets and model architectures.}
\label{fig: adv_self_rmc2}
\end{figure}


\subsection{Results on ERMC}

In ERMC, the models situated at the left and right endpoints undergo different training processes. The left endpoint model receives training with $\ell_\infty$-AT, whereas the right endpoint model is subsequently refined with AT fine-tuning focused on $\ell_1$-AT. These procedures' outcomes are illustrated in Fig.~\ref{fig: adv_self_rmc2}. The observations from this process are twofold: Firstly, ERMC demonstrates commendable performance across all tested datasets and architectures. Secondly, the process of fine-tuning has a noticeable impact on the models' inherent robustness. Models at each endpoint exhibit a high degree of robustness to the type of perturbations they were trained against (i.e., $\ell_\infty$ for the left endpoint and $\ell_1$ for the right) yet they show a relative vulnerability to the alternate type of perturbations (i.e., $\ell_1$ for the left endpoint and $\ell_\infty$ for the right).




\begin{table*}[!t]
\caption{Our Methods Achieve State-of-the-Art DLR Levels Under Various Perturbations. Furthermore, our methods consistently achieve the highest accuracy under Union, $\ell_\infty/\ell_2/\ell_1$ AA \cite{croce2020reliable} (with the lowest accuracies being denoted by braces), and MSD. For methods utilizing two types of perturbations, we compare them using DLR only under $\ell_\infty$-PGD and $\ell_2$-PGD attacks, marking the DLR (representing the lowest accuracy) with an underline. For those employing three types of perturbations, we assess them across all metrics, marking the DLR (the lowest accuracy) under the three basic $\ell_p$ attacks with an overline.
}
\label{tab: main}
\begin{center}

\resizebox{0.99\textwidth}{!}{
\begin{tabular}{l||c|c|c|c|c|c|c|c}
\hline
\hline
& Standard Accuracy & $\ell_\infty$-PGD ($\delta=8/255$) & $\ell_2$-PGD ($\delta=1$) & $\ell_1$-PGD ($\delta=12$) & DLR & Union & AA ($\ell_\infty/\ell_2/\ell_1$) \cite{croce2020reliable} & MSD \\
\hline
\begin{tabular}[c]{@{}c@{}} $\ell_\infty$-AT \cite{madry2018towards}   \end{tabular}  &   85.00\% & 49.03\% & 29.66\%  & 16.61\% &  / & 21.85\%  & 46.02\%/20.86\%/\{10.45\%\} & 15.27\% \\
\hline
\begin{tabular}[c]{@{}c@{}} MSD - Defense \\ (\textbf{two} types of pert)  \end{tabular}  &  81.61\% & 48.57\% & \underline{45.92\%} & 35.64\% & 45.92\% & 34.37\% & 46.6\%/42.13\%/\{31.55\%\} & 45.72\%  \\
\hline
\begin{tabular}[c]{@{}c@{}} RMC-RI \\ (\textbf{two} types of pert)  \end{tabular}  &  63.08\% & \underline{37.4\%} & 38.44\% & 30.47\% & 30.47\% & 29.22\% & 36.85\%/37.17\%/\{28.33\%\} & 36.9\%  \\
\hline
\begin{tabular}[c]{@{}c@{}} RMC \\ (ours, \textbf{two} types of pert)  \end{tabular}  &  80.90\%  & \underline{48.19\%} & 48.63\% & 38.05\% & 48.19\%  & 36.3\% & 46.74\%/45.16\%/\{34.4\%\} & 46.52\%  \\
\hline
\begin{tabular}[c]{@{}c@{}} RMC-based optimization \\ (ours, \textbf{two} types of pert)  \end{tabular}  &  81.36\%  & \underline{48.89\%} & 49.03\% & 38.83\% & 48.89\%  & 36.86\% & 47.66\%/45.73\%/\{35.18\%\} & 47.18\%  \\
\hline
\begin{tabular}[c]{@{}c@{}} MSD - Defense \cite{maini2020adversarial} \\ (\textbf{three} types of pert)  \end{tabular}  &  81.35\% & $\overline{40.14\%}$ & 48.58\% & 47.50\% & 40.14\% & 38.35\% & \{37.87\%\}/45.9\%/45.27\% & 38.20\%  \\
\hline
\begin{tabular}[c]{@{}c@{}} E-AT  \cite{croce2022adversarial} \\ (\textbf{three} types of pert)  \end{tabular}  &  79.3\% & $\overline{44.07\%}$ & 49.12\% & 49.82\% & 44.07\% & 41.08\% & \{41.41\%\}/46.5\%/47.82\% & 42.67\%  \\
\hline
\begin{tabular}[c]{@{}c@{}} RMC-based optimization \\ (ours, \textbf{three} types of pert)  \end{tabular}  &  81.76\% & $\overline{46.21\%}$ & 51.86\% & 46.23\% & 46.21\% & 41.47\% & 44.58\%/49.35\%/\{43.42\%\} & 44.75\%   \\
\hline
\begin{tabular}[c]{@{}c@{}} RMC-based optimization \\ 5-model ensemble \\ (ours, \textbf{three} types of pert)  \end{tabular}  &  78.35\% & 55.91\% & 56.78\% & $\overline{51.05\%}$ & 51.05\% & 49.39\% & 50.15\%/49.85\%/{\{\bf{49.83\%}\}} & 48.79\%   \\
\hline
\begin{tabular}[c]{@{}c@{}} RMC-based optimization \\ with SRMC modules \\ (ours, \textbf{three} types of pert)  \end{tabular}  &  80.39\%  & $\overline{46.10\%}$ & 48.92\% & 46.39\% & 46.10\%  & 42.03\% & 44.95\%/46.66\%/\{43.91\%\} & 45.07\%   \\
\hline
\begin{tabular}[c]{@{}c@{}} ERMC-1 \\ (ours, \textbf{three} types of pert) \end{tabular}  &  82.66\%  &   $\overline{46.54\%}$ & 48.76\% & 47.06\% & 46.54\% &  41.94\% & 44.88\%/45.88\%/\{43.97\%\} & 44.88\%   \\
\hline
\begin{tabular}[c]{@{}c@{}} ERMC-3 \\ (ours, \textbf{three} types of pert)  \end{tabular}  &  79.61\%  & 49.29\% & 51.32\% & $\overline{48.49\%}$ & 48.49\% &  45.27\% & \{42.88\%\}/44.57\%/47.37\% & 43.31\%   \\
\hline
\begin{tabular}[c]{@{}c@{}} ERMC-5 \\ (ours, \textbf{three} types of pert) \end{tabular}  &  79.41\%  & 55.46\% & 57.28\% & $\overline{53.97\%}$ & \bf{53.97\%} &  \bf{51.41\%} & {\{49.33\%}\}/50.55\%/52.41\% & {\bf{49.78\%}}   \\
\hline
\hline
\end{tabular}}
\end{center}
\end{table*}


\subsection{A Comprehensive Comparison}


For MSD Defense, RMC-RI, RMC, and RMC-based optimization (when considering only two types of perturbations), we evaluated them using the $\ell_\infty$-PGD and $\ell_2$-PGD attacks, given that the $\ell_1$-PGD attack was not considered during training. The DLR (representing the lowest robust accuracy) for these attacks is indicated with an underline. 

For methods that employ three types of perturbations, we assessed them under the $\ell_\infty$-PGD, $\ell_2$-PGD, $\ell_1$-PGD, AA, MSD attacks, and the union metric. In RMC-based optimization, a 5-model ensemble means that we select five models from the curve shown in Fig.~\ref{fig: adv_mc_opt2} and ensemble them. For the model ensemble, the model selection thresholds are set at $\alpha_\infty=37\%$ for $\ell_\infty$ robustness and $\alpha_1=43\%$ for $\ell_1$ robustness. We determine the lowest accuracy among the $\ell_\infty$, $\ell_2$, and $\ell_1$ norms within the AA framework and mark it with braces. The DLRs (lowest robust accuracies) for the $\ell_\infty$-PGD, $\ell_2$-PGD, and $\ell_1$-PGD attacks are denoted with an overline. Additionally, we emphasized the highest accuracy values in the union, AA, and MSD columns. 


From Table~\ref{tab: main}, the following observations can be made: (1) RMC with two types of perturbations outperforms MSD with two types of perturbations on DLR by $2.27\%$ and also surpasses RMC-RI by $10.79\%$.; (2) The RMC-based optimization with two types of perturbations yields a slightly higher DLR than RMC and excels over RMC in all other metrics; (3) When considering three types of perturbations, RMC-based optimization surpasses both MSD Defense and E-AT in DLR by $6.07\%$ and $2.14\%$. Moreover, it exhibits accuracy improvements of $3.12\%$, $5.55\%$, and $6.55\%$ ($0.39\%$, $2.01\%$, and $2.08\%$) over MSD Defense (and E-AT) under the Union, AA, and MSD Attack metrics, respectively; (4) The RMC-based optimization method shows a trade-off between clean accuracy and DLR. However, its clean accuracy drop of $3.24\%$, when benchmarked against $\ell_\infty$-AT, is less severe than that observed in other defense methods like MSD Defense and E-AT. Currently, the model selection process in RMC prioritizes robustness (DLR) over clean accuracy, but this could be adjusted in future implementations to achieve a better balance between the two; (5) The RMC-based optimization with SRMC modules can achieve similar (slightly lower) DLR performance compared to the RMC-based optimization with three types of perturbations, and even has slightly higher accuracy under the AA and Union metric; (6) Using a multi-model ensemble method can further enhance the performance of RMC-based optimization; (7) ERMC-1 reaches similar performance as RMC-based optimizataion. As the number of models $n$ increases, the performance of ERMC correspondingly improves. When $n$ reaches 5, ERMC-5 outperforms all other methods in terms of accuracy improvements under DLR, Union, and MSD.

In conclusion, our proposed Robust Mode Connectivity-Oriented Adversarial Defense shows remarkable performance across a variety of metrics. The RMC-based optimization (Phase II) delivers a higher DLR compared to RMC (Phase I) alone. ERMC can achieve high robustness by only conducting one RMC process. On the whole, the Robust Mode Connectivity-Oriented Adversarial Defense introduces a novel defense paradigm rooted in population-based optimization, effectively enhancing the robustness of Neural Networks (NNs).





\section{Conclusion}\label{sec: conclusion}
In this work, we introduced a Robust Mode Connectivity (RMC)-oriented adversarial defense framework that leverages population-based optimization to strengthen neural networks against diversified $\ell_p$ attacks. Our two-phase design enables the discovery of robust paths (Phase I) and systematic selection of high-performing models through RMC-based optimization (Phase II). To improve efficiency, we further proposed the Efficient Robust Mode Connectivity (ERMC), which combine theoretical guarantees with practical scalability. Extensive experiments across CIFAR-10, CIFAR-100, and ImageNet-100, as well as multiple architectures, demonstrated that our approach consistently outperforms existing methods, achieving superior diversified $\ell_p$ robustness while maintaining competitive accuracy. Overall, this work establishes population-based mode connectivity as a powerful and generalizable principle for adversarial defense. Future directions include extending RMC to large-scale foundation models, integrating it with certified robustness techniques, and exploring applications in safety-critical domains such as healthcare and power systems.

\bibliographystyle{ieeetr}
\bibliography{ref_adv,ref_new}

\newpage
\twocolumn
\pagestyle{empty}


\vfill
\end{document}



\end{longtable}
\normalsize
\twocolumn


%%%%%%%%%%%%%%%%%%%%%%%%%%%%%%%%%%%%%%%%%%%%%%%%%%


% Don't change these lines
% typesetting comment
\label{lastpage}
% % \appendix
% \begin{appendix} %First appendix
% \section{catalogue format}



% \end{appendix}
%

\end{document}