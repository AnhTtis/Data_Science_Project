\pdfoutput=1

\usepackage[T1]{fontenc}
\usepackage{lmodern}
\usepackage{calc}
\usepackage{graphicx}
\usepackage{booktabs}
\usepackage{textcomp}
\usepackage{xspace}
\usepackage{relsize}
\usepackage{amssymb}
\usepackage{amsmath}
\usepackage{listings}
\usepackage{microtype}
\usepackage{multirow}
\usepackage{tabularx}
\usepackage{array}
\usepackage{placeins}
\usepackage{cuted}
\usepackage{soul} % only for \st; delete if this causes you problems.
\usepackage{fixltx2e}
\usepackage{slashed}
\usepackage{bm}
\usepackage[numbers,sort&compress]{natbib}
\usepackage[labelfont=bf,font=small]{caption}
\usepackage[skip=-2pt]{subcaption}
\usepackage[clockwise,figuresright]{rotating}
\usepackage{tikz}
\usepackage[normalem]{ulem}
\usepackage[utf8]{inputenc}
%\usepackage{bbding}


\usepackage{etoolbox}
\AfterEndEnvironment{strip}{\leavevmode}

\allowdisplaybreaks

\newcolumntype{L}{>{\raggedright\let\newline\\\arraybackslash\hspace{0pt}}X}
\newcolumntype{R}{>{\raggedleft\let\newline\\\arraybackslash\hspace{0pt}}X}
\newcolumntype{C}{>{\centering\let\newline\\\arraybackslash\hspace{0pt}}X}

\setlength{\rotFPtop}{0pt plus 1fil}
\setcounter{tocdepth}{3}

%%%%%% Author institutes %%%%%%%
\newcommand{\gambitinstitute}[1]{\expandafter\csname #1\endcsname\label{#1}}
\newcommand{\gi}[1]{\gambitinstitute{#1}\and}
\newcommand{\last}[1]{\gambitinstitute{#1}}
\newcommand{\aachen}{Institute for Theoretical Particle Physics and Cosmology (TTK), RWTH Aachen University, D-52056 Aachen, Germany}
\newcommand{\queens}{Department of Physics, Engineering Physics and Astronomy, Queen's University, Kingston ON K7L 3N6, Canada}
\newcommand{\mcdonald}{Arthur B. McDonald Canadian Astroparticle Physics Research Institute, Kingston ON K7L 3N6, Canada}
\newcommand{\perimeter}{Perimeter Institute for Theoretical Physics, Waterloo ON N2L 2Y5, Canada}
\newcommand{\imperial}{Department of Physics, Imperial College London, Blackett Laboratory, Prince Consort Road, London SW7 2AZ, UK}
\newcommand{\cambridge}{Cavendish Laboratory, University of Cambridge, JJ Thomson Avenue, Cambridge, CB3 0HE, UK}
\newcommand{\nordita}{NORDITA, Roslagstullsbacken 23, SE-10691 Stockholm, Sweden}
\newcommand{\oslo}{Department of Physics, University of Oslo, N-0316 Oslo, Norway}
\newcommand{\adelaide}{ARC Centre of Excellence for Dark Matter Particle Physics \& CSSM, Department of Physics, University of Adelaide, Adelaide, SA 5005}
\newcommand{\glasgow}{SUPA, School of Physics and Astronomy, University of Glasgow, Glasgow, G12 8QQ, UK}
\newcommand{\monash}{School of Physics and Astronomy, Monash University, Melbourne, VIC 3800, Australia}
\newcommand{\nanjing}{Department of Physics and Institute of Theoretical Physics, Nanjing Normal University, Nanjing, Jiangsu 210023, China}
\newcommand{\coepp}{Australian Research Council Centre of Excellence for Particle Physics at the Tera-scale}
\newcommand{\okc}{Oskar Klein Centre for Cosmoparticle Physics, AlbaNova University Centre, SE-10691 Stockholm, Sweden}
\newcommand{\su}{Department of Physics, Stockholm University, SE-10691 Stockholm, Sweden}
\newcommand{\mcgill}{Department of Physics, McGill University, 3600 rue University, Montr\'eal, Qu\'ebec H3A 2T8, Canada}
\newcommand{\ucla}{Physics and Astronomy Department, University of California, Los Angeles, CA 90095, USA}
\newcommand{\annecy}{LAPTh, Universit\'e de Savoie, CNRS, 9 chemin de Bellevue B.P.110, F-74941 Annecy-le-Vieux, France}
\newcommand{\harvard}{Department of Physics, Harvard University, Cambridge, MA 02138, USA}
\newcommand{\grappa}{GRAPPA, Institute of Physics, University of Amsterdam, Science Park 904, 1098 XH Amsterdam, Netherlands}
\newcommand{\sydney}{Centre for Translational Data Science, Faculty of Engineering and Information Technologies, School of Physics, The University of Sydney, NSW 2006, Australia}
\newcommand{\fermilab}{Fermi National Accelerator Laboratory, Batavia, IL 60510, USA}
\newcommand{\desy}{DESY, Notkestra\ss e 85, D-22607 Hamburg, Germany}
\newcommand{\wsu}{Department of Physics, Weber State University, 1415 Edvalson St., Dept. 2508, Ogden, UT 84408, USA}
\newcommand{\tum}{Physik-Department T30d, Technische Universit\"at M\"unchen, James-Franck-Stra\ss e 1, D-85748 Garching, Germany}
\newcommand{\cernth}{Theoretical Physics Department, CERN, CH-1211 Geneva 23, Switzerland}
\newcommand{\lyon}{Univ Lyon, Univ Lyon 1, CNRS/IN2P3, Institut de Physique des 2 Infinis de Lyon, UMR 5822, 69622 Villeurbanne, France}
\newcommand{\iuf}{Institut Universitaire de France, 103 boulevard Saint-Michel, 75005 , France}
\newcommand{\iap}{Institut d'Astrophysique de Paris, 98~bis boulevard Arago, 75014 Paris, France}
\newcommand{\zurich}{Physik-Institut, Universit\"at Z\"urich, Winterthurerstrasse 190, 8057 Z\"urich, Switzerland}
\newcommand{\cern}{Theoretical Physics Department, CERN, CH-1211 Geneva 23, Switzerland}
\newcommand{\krakow}{H.~Niewodnicza\'nski Institute of Nuclear Physics, Polish Academy of Sciences, 31-342  Krak\'ow, Poland}
\newcommand{\bonn}{Physikalisches Institut der Rheinischen Friedrich-Wilhelms-Universit\"at Bonn, 53115 Bonn, Germany}
\newcommand{\valencia}{Instituto de F\'isica Corpuscular, IFIC-UV/CSIC, Valencia, Spain}
\newcommand{\uq}{School of Mathematics and Physics, The University of Queensland, St.\ Lucia, Brisbane, QLD 4072, Australia}
\newcommand{\gottingen}{Institut f\"ur Astrophysik, Georg-August-Universit\"at G\"ottingen, Friedrich-Hund-Platz~1, 37077 G\"ottingen, Germany}
\newcommand{\ioa}{Institute for Astronomy, University of Cambridge, Madingley Road, Cambridge, CB3 0HA, UK}
\newcommand{\kicc}{Kavli Institute for Cosmology, University of Cambridge, Madingley Road, Cambridge, CB3 0HA, UK}
\newcommand{\caius}{Gonville \& Caius College, Trinity Street, Cambridge, CB2 1TA, UK}
\newcommand{\louvain}{Center for Cosmology, Particle Physics and Phenomenology, Universit\'{e} catholique de Louvain, B-1348 Louvain-la-Neuve, Belgium}
\newcommand{\kings}{Theoretical Particle Physics and Cosmology (TPPC), Department of Physics, King’s College London, Strand, London, WC2R 2LS, UK}
\newcommand{\cincy}{Department of Physics, University of Cincinnati, Cincinnati, Ohio 45221, USA}
\newcommand{\infn}{Istituto Nazionale di Fisica Nucleare, Sezione di Torino, via P. Giuria 1, I–10125 Torino, Italy}
\newcommand{\bom}{Bureau of Meteorology, Melbourne, VIC 3001, Australia}
\newcommand{\uppsala}{Department of Physics and Astronomy, Uppsala University, Box 516, SE-751 20 Uppsala, Sweden}
\newcommand{\kitIAP}{Institute for Astroparticle Physics (IAP), Karlsruhe Institute of Technology (KIT), Hermann-von-Helmholtz-Platz 1, D-76344 Eggenstein-Leopoldshafen, Germany}
\newcommand{\kitTTP}{Institute for Theoretical Particle Physics (TTP), Karlsruhe Institute of Technology (KIT), 76128 Karlsruhe, Germany}
\newcommand{\ucl}{Department of Physics and Astronomy, University College London, Gower St., London, WC1E 6BT, UK}
\newcommand{\qb}{Quantum Brilliance Pty Ltd, The Australian National University, Daley Road, Acton ACT 2601, Australia}
\newcommand{\sfu}{Department of Physics, Simon Fraser University, Burnaby BC, V5A 1S6, Canada}
\newcommand{\casbejing}{CAS Key Laboratory of Theoretical Physics, Institute of Theoretical Physics, Chinese Academy of Sciences, Beijing 100190, China}
\newcommand{\zhengzhou}{School of Physics, Zhengzhou University, Zhengzhou 450000, China}
\newcommand{\xjtlu}{Department of Physics, School of Mathematics and Physics, Xi'an Jiaotong-Liverpool University, 111 Ren'ai Road, Suzhou Dushu Lake, Science and Education Innovation District, Suzhou Industrial Park, Suzhou 215123, P.R.~China}
\newcommand{\telenor}{Telenor Research, N-1360 Fornebu, Norway}

%%%%%%% Acknowledgements %%%%%%%%
\newcommand{\gambitacknos    }{We warmly thank the Casa Matem\'aticas Oaxaca, affiliated with the Banff International Research Station, for hospitality whilst part of this work was completed, and the staff at Cyfronet, for their always helpful supercomputing support.  \GB has been supported by STFC (UK; ST/K00414X/1, ST/P000762/1), the Royal Society (UK; UF110191), Glasgow University (UK; Leadership Fellowship), the Research Council of Norway (FRIPRO 230546/F20), NOTUR (Norway; NN9284K), the Knut and Alice Wallenberg Foundation (Sweden; Wallenberg Academy Fellowship), the Swedish Research Council (621-2014-5772), the Australian Research Council (CE110001004, FT130100018, FT140100244, FT160100274), The University of Sydney (Australia; IRCA-G162448), PLGrid Infrastructure (Poland), Polish National Science Center (Sonata UMO-2015/17/D/ST2/03532), the Swiss National Science Foundation (PP00P2-144674), the European Commission Horizon 2020 Marie Sk\l{}odowska-Curie actions (H2020-MSCA-RISE-2015-691164), the ERA-CAN+ Twinning Program (EU \& Canada), the Netherlands Organisation for Scientific Research (NWO-Vidi 680-47-532), the National Science Foundation (USA; DGE-1339067), the FRQNT (Qu\'ebec) and NSERC/The Canadian Tri-Agencies Research Councils (BPDF-424460-2012).}

\newcommand{\gambitacknospmare}{We warmly thank the Casa Matem\'aticas Oaxaca, affiliated with the Banff International Research Station, for hospitality whilst part of this work was completed, and the staff at Cyfronet, for their always helpful supercomputing support.  \GB has been supported by STFC (UK; ST/K00414X/1, ST/P000762/1), the Royal Society (UK; UF110191), Glasgow University (UK; Leadership Fellowship), the Research Council of Norway (FRIPRO 230546/F20), NOTUR (Norway; NN9284K), the Knut and Alice Wallenberg Foundation (Sweden; Wallenberg Academy Fellowship), the Swedish Research Council (621-2014-5772), the Australian Research Council (CE110001004, FT130100018, FT140100244, FT160100274), The University of Sydney (Australia; IRCA-G162448), PLGrid Infrastructure (Poland), Red Espa\~nola de Supercomputaci\'on (Spain; FI-2016-1-0021), Polish National Science Center (Sonata UMO-2015/17/D/ST2/03532), the Swiss National Science Foundation (PP00P2-144674), the European Commission Horizon 2020 Marie Sk\l{}odowska-Curie actions (H2020-MSCA-RISE-2015-691164), the ERA-CAN+ Twinning Program (EU \& Canada), the Netherlands Organisation for Scientific Research (NWO-Vidi 680-47-532), the National Science Foundation (USA; DGE-1339067), the FRQNT (Qu\'ebec) and NSERC/The Canadian Tri-Agencies Research Councils (BPDF-424460-2012).}

\newcommand{\gambitacknosplus}{We warmly thank the Casa Matem\'aticas Oaxaca, affiliated with the Banff International Research Station, for hospitality whilst part of this work was completed, and the staff at Cyfronet, for their always helpful supercomputing support.  \GB has been supported by STFC (UK; ST/K00414X/1, ST/P000762/1), the Royal Society (UK; UF110191), Glasgow University (UK; Leadership Fellowship), the Research Council of Norway (FRIPRO 230546/F20), NOTUR (Norway; NN9284K), the Knut and Alice Wallenberg Foundation (Sweden; Wallenberg Academy Fellowship), the Swedish Research Council (621-2014-5772), the Australian Research Council (CE110001004, FT130100018, FT140100244, FT160100274), The University of Sydney (Australia; IRCA-G162448), PLGrid Infrastructure (Poland), Polish National Science Center (Sonata UMO-2015/17/D/ST2/03532), the Swiss National Science Foundation (PP00P2-144674), European Commission Horizon 2020 (Marie Sk\l{}odowska-Curie actions H2020-MSCA-RISE-2015-691164, European Research Council Starting Grant ERC-2014-STG-638528), the ERA-CAN+ Twinning Program (EU \& Canada), the Netherlands Organisation for Scientific Research (NWO-Vidi 016.149.331), the National Science Foundation (USA; DGE-1339067), the FRQNT (Qu\'ebec) and NSERC/The Canadian Tri-Agencies Research Councils (BPDF-424460-2012).}


\makeatletter

\newcommand{\preprintnumber}[1]{\gdef\@preprintnumber{\begin{flushright}{#1}\end{flushright}}}

% \DeclareRobustCommand{\kbd}[1]{{\texttt{#1}}}
% \DeclareRobustCommand{\code}[1]{\kbd{#1}\xspace}
% \DeclareRobustCommand{\To}{\ensuremath{\Rightarrow}\xspace}
\g@addto@macro\bfseries{\boldmath}
\makeatother

\bibliographystyle{JHEP_pat}
\sloppy

\let\underscore\_
\renewcommand{\_}{\discretionary{\underscore}{}{\underscore}}

\makeatletter
\let\orgdescriptionlabel\descriptionlabel
\renewcommand*{\descriptionlabel}[1]{%
  \let\orglabel\label
  \let\label\@gobble
  \phantomsection
  \protected@edef\@currentlabel{#1}%
  %\protected@edef\@currentlabelname{#1}
  \let\label\orglabel
  \orgdescriptionlabel{#1}%
}
\makeatother

%Journal definitions
\input{jdefs.tex}

% Fancy breaking with arrows
%\lstset{prebreak=\raisebox{0ex}[0ex][0ex]
%        {\ensuremath{\rhookswarrow}}}
\lstset{breaklines=true, breakatwhitespace=true}
\lstset{breakautoindent=false} % don't want lines offset based on existing indent
\lstset{breakindent=5pt}

% Allow breaking at forward slashes
%\lstset{literate={/}{/}{1\discretionary{}{}{}}} %doesn't seem to affect lstinline

%\newsavebox{\spacebox}
%\begin{lrbox}{\spacebox}
%\verb*! !
%\end{lrbox}
%\newcommand{\aspace}{\usebox{\spacebox}}%
%
%\lstset{prebreak={\aspace}}
\newcommand\postnewlinemarker{\hbox{\ensuremath{\hookrightarrow}}}
\lstset{postbreak=\postnewlinemarker} % This only seems to work at the whitespace breaks, not the 'literate' breaks for some reason...

\newcommand\cpp[1]{{\lstinline!#1!}}  % Apparently curly braces are only "experimental"
\newcommand\cpppragma[1]{{\CPPcommentstyle#1}}
\newcommand\yaml[1]{{\lstset{style=yaml}\lstinline!#1!\lstset{style=cpp}}}
\newcommand\yamlvalue[1]{{\YAMLvaluestyle\ttfamily#1}}
\newcommand\yamlcomment[1]{{\YAMLcommentstyle\ttfamily#1}}
\newcommand\pyid[1]{{\pyidentifierstyle#1}}
\newcommand\term[1]{{\lstset{style=terminal}\lstinline!#1!\lstset{style=cpp}}}
\newcommand\termalt[1]{{\lstset{style=terminalalt}\lstinline!#1!\lstset{style=cpp}}}
\newcommand\fortran[1]{{\lstset{style=fortran}\lstinline!#1!\lstset{style=cpp}}}
\newcommand\py[1]{{\lstset{style=python}\lstinline!#1!\lstset{style=cpp}}}
\newcommand\customtilde{{\raisebox{0.2ex}{\scalebox{0.6}{\boldmath$\sim$}}}}
\newcommand\mathematica[1]{{\lstset{style=Mathematica}\lstinline!#1!\lstset{style=cpp}}}
\newcommand\guminline[1]{{{\lstset{style=gum}\lstinline!#1!}}}
\newcommand\textinline[1]{{{\lstset{style=text}\lstinline!#1!}}}

\def\be{\begin{equation}}
\def\ee{\end{equation}}
\def\ba{\begin{eqnarray}}
\def\ea{\end{eqnarray}}
\newcommand{\bea}{\begin{eqnarray}}
\newcommand{\eea}{\end{eqnarray}}


\lstnewenvironment{lstlistingyaml}{\lstset{style=yaml}}{\lstset{style=cpp}}
\lstnewenvironment{lstlistingterm}{\lstset{style=terminal}}{\lstset{style=cpp}}
\lstnewenvironment{lstlistingfortran}{\lstset{style=fortran}}{\lstset{style=cpp}}
\lstnewenvironment{lstcpp}{\lstset{style=cpp}}{\lstset{style=cpp}}
\lstnewenvironment{lstcppalt}{\lstset{style=cppalt}}{\lstset{style=cpp}}
\lstnewenvironment{lstcppnum}{\lstset{style=cppnum}}{\lstset{style=cpp}}
\lstnewenvironment{lstyaml}{\lstset{style=yaml}}{\lstset{style=cpp}}
\lstnewenvironment{lstgum}{\lstset{style=gum}}{\lstset{style=cpp}}
\lstnewenvironment{lstterm}{\lstset{style=terminal}}{\lstset{style=cpp}}
\lstnewenvironment{lsttermalt}{\lstset{style=terminalalt}}{\lstset{style=cpp}}
\lstnewenvironment{lsttext}{\lstset{style=text}}{\lstset{style=cpp}}
\lstnewenvironment{lstfortran}{\lstset{style=fortran}}{\lstset{style=cpp}}
\lstnewenvironment{lstpy}{\lstset{style=python}}{\lstset{style=cpp}}
\lstnewenvironment{lstmathematica}{\lstset{style=mathematica}}{\lstset{style=cpp}}

% As cpp, but allows for adding a caption and label (with custom caption-label, e.g. "Algorithm 1")
\newcommand{\tmpname}{}
\newcommand{\tmplistingname}{}
\makeatletter
\newif\ifATOlabelname
\lst@Key{labelname}{Listing}{\def\ATOlabelname{#1}\global\ATOlabelnametrue}
\makeatother
\lstnewenvironment{lstcpplabel}[1][]{
  \lstset{style=cpp,#1} % #1 allows to add new options with [] same as for normal lstlistings environment
  \ifATOlabelname
    \renewcommand{\tmpname}{\lstlistingname}
    \renewcommand{\tmplistingname}{\lstlistlistingname}
    \renewcommand{\lstlistingname}{\ATOlabelname}% Listing -> labelname
    \renewcommand{\lstlistlistingname}{List of \lstlistingname s}% List of Listings -> List of labelname
    % I think this needs expanding though, since probably it will use the same counter no matter what label-type is given
  \fi
}{
  % restore defaults
  \renewcommand{\lstlistingname}{\tmpname}
  \renewcommand{\lstlistlistingname}{\tmplistingname}
  \lstset{style=cpp}
}
%C++ syntax highlighting, direct from http://marcusmo.co.uk/blog/latex-syntax-highlighting/
% Solarized colour scheme for listings
\definecolor{solarized@base03}{HTML}{002B36}
\definecolor{solarized@base02}{HTML}{073642}
\definecolor{solarized@base01}{HTML}{586e75}
\definecolor{solarized@base00}{HTML}{657b83}
\definecolor{solarized@base0}{HTML}{839496}
\definecolor{solarized@base1}{HTML}{93a1a1}
\definecolor{solarized@base2}{HTML}{EEE8D5}
\definecolor{solarized@base3}{HTML}{FDF6E3}
\definecolor{solarized@yellow}{HTML}{B58900}
\definecolor{solarized@orange}{HTML}{CB4B16}
\definecolor{solarized@red}{HTML}{DC322F}
\definecolor{solarized@magenta}{HTML}{D33682}
\definecolor{solarized@violet}{HTML}{6C71C4}
\definecolor{solarized@blue}{HTML}{268BD2}
\definecolor{solarized@cyan}{HTML}{2AA198}
\definecolor{solarized@green}{HTML}{859900}
\definecolor{darkred}{HTML}{550003}
\definecolor{darkgreen}{HTML}{00AA00}
\definecolor{orchid}{HTML}{AF06F5}
\newcommand\YAMLcolonstyle{\footnotesize\color{solarized@red}\mdseries}
\newcommand\YAMLstringstyle{\footnotesize\color{solarized@green}\mdseries}
\newcommand\YAMLkeystyle{\footnotesize\color{solarized@blue}\ttfamily}
\newcommand\YAMLvaluestyle{\footnotesize\color{blue}\mdseries}
\newcommand\ProcessThreeDashes{\llap{\color{cyan}\mdseries-{-}-}}
% Define C++ syntax highlighting colour scheme
\newcommand\CPPplainstyle{\footnotesize\ttfamily}
\newcommand\CPPkeywordstyle{\color{solarized@orange}\footnotesize\ttfamily}
\newcommand\CPPidentifierstyle{\color{solarized@blue}\footnotesize\ttfamily}
\newcommand\CPPcommentstyle{\color{solarized@violet}\footnotesize\ttfamily}
\newcommand\CPPdirectivestyle{\color{solarized@magenta}\footnotesize\ttfamily}
% Define terminal syntax highlighting colour scheme (move more here as needed)
\newcommand\termplainstyle{\footnotesize\ttfamily}
\newcommand\pyidentifierstyle{\footnotesize\ttfamily\color{darkred}}
% Define YAML syntax highlighting colour scheme
\newcommand\YAMLcommentstyle{\color{solarized@orange}\ttfamily}


%\newcommand\processCppLineContinuation
%{
%  \lst@CalcLostSpaceAndOutput{test}%
%  \lst@modetrue%
%  \lst@Lmodetrue%
%  \CPPcommentstyle%
%}
\newcommand\processLongMacroDelimiter
{%
%\\lst@CalcLostSpaceAndOutput%
\CPPdirectivestyle%
\#define%
}

\newcommand\processCPPCOMMENT
{%
\CPPcommentstyle%
{//}%
}

\newcommand\processCPPTRIPCOMMENT
{%
\CPPcommentstyle%
{///}%
}

\lstdefinestyle{cpp}
{
  language=C++,
  basicstyle=\footnotesize\ttfamily,
  basewidth={0.53em,0.44em}, %Ben: experimenting a bit with the fixed-width width (first argument); feels a bit more readable to me with the slightly smaller width (was 0.6em by default)
  numbers=none,
  tabsize=2,
  breaklines=true,
  escapeinside={@}{@},
  showstringspaces=false,
  numberstyle=\tiny\color{solarized@base01},
  keywordstyle=\color{solarized@orange},
  stringstyle=\color{solarized@red}\ttfamily,
  identifierstyle=\color{solarized@blue},
  commentstyle=\CPPcommentstyle,
  directivestyle=\CPPdirectivestyle,
  emphstyle=\color{solarized@green},
  frame=single,
  rulecolor=\color{solarized@base2},
  rulesepcolor=\color{solarized@base2},
  literate={~} {\customtilde}1,
  moredelim=*[directive]\ \ \#,
  moredelim=*[directive]\ \ \ \ \#
}

% C++ style with different escape character (so I can use @'s in strings)
% Also allows for correct multi-line macro highlighting)
\lstdefinestyle{cppalt}
{
  language=C++,
  basicstyle=\footnotesize\ttfamily,
  basewidth={0.53em,0.44em}, %Ben: experimenting a bit with the fixed-width width (first argument); feels a bit more readable to me with the slightly smaller width (was 0.6em by default)
  numbers=none,
  tabsize=2,
  breaklines=true,
  escapeinside={*@}{@*},
  showstringspaces=false,
  numberstyle=\tiny\color{solarized@base01},
  keywordstyle=\color{solarized@orange},
  stringstyle=\color{solarized@red}\ttfamily,
  identifierstyle=\color{solarized@blue},
  commentstyle=\CPPcommentstyle,
  directivestyle=\CPPdirectivestyle,
  emphstyle=\color{solarized@green},
  frame=single,
  rulecolor=\color{solarized@base2},
  rulesepcolor=\color{solarized@base2},
  literate={~}{\customtilde}1,
  %literate={/}{/}{1\discretionary{}{\hbox{\ensuremath{\hookrightarrow}}}{}} {//}{CPPCOMMENT}{2} {///}{CPPTRIPCOMMENT}{3}, %allow breaking at single forward slash without breaking comments
  %moredelim=[il][\processCPPTRIPCOMMENT]{CPPTRIPCOMMENT},
  %moredelim=[il][\processCPPCOMMENT]{CPPCOMMENT},
  moredelim=**[is][\processLongMacroDelimiter]{BeginLongMacro}{EndLongMacro} %special delimiter for long macros that go over several lines
  %moredelim=*[directive]\ \ \#,
  %moredelim=*[directive]\ \ \ \ \#
}

% C++ style with line numbers (try to keep everything else matching the 'cpp' style)
\lstdefinestyle{cppnum}
{
  language=C++,
  basicstyle=\footnotesize\ttfamily,
  basewidth={0.53em,0.44em}, %Ben: experimenting a bit with the fixed-width width (first argument); feels a bit more readable to me with the slightly smaller width (was 0.6em by default)
  numbers=none,
  tabsize=2,
  breaklines=true,
  escapeinside={@}{@},
  numberstyle=\tiny\color{solarized@base01},
  showstringspaces=false,
  keywordstyle=\color{solarized@orange},
  stringstyle=\color{solarized@red}\ttfamily,
  identifierstyle=\color{solarized@blue},
  commentstyle=\CPPcommentstyle,
  directivestyle=\CPPdirectivestyle,
  emphstyle=\color{solarized@green},
  frame=single,
  rulecolor=\color{solarized@base2},
  rulesepcolor=\color{solarized@base2},
  literate={~} {\customtilde}1,
  moredelim=*[directive]\ \ \#,
  moredelim=*[directive]\ \ \ \ \#
}

% Define python syntax highlighting colour scheme
\lstdefinestyle{python}
{
  language=Python,
  basicstyle=\footnotesize\ttfamily,
  basewidth={0.53em,0.44em},
  numbers=none,
  tabsize=2,
  breaklines=true,
  escapeinside={@}{@},
  showstringspaces=false,
  numberstyle=\tiny\color{solarized@base01},
  keywordstyle=\color{blue},
  stringstyle=\color{orange}\ttfamily,
  identifierstyle=\color{darkred},
  commentstyle=\color{purple},
  emphstyle=\color{green},
  frame=single,
  rulecolor=\color{solarized@base2},
  rulesepcolor=\color{solarized@base2},
  literate = {~}{\customtilde}1
             {\ as\ }{{\color{blue}\ as\ \color{black}}}3
             {.set}{{\color{black}.}{\color{darkred}set}}4
}

% Define fortran syntax highlighting colour scheme
\lstdefinestyle{fortran}
{
  language=Fortran,
  basicstyle=\footnotesize\ttfamily,
  basewidth={0.53em,0.44em},
  numbers=none,
  tabsize=2,
  breaklines=true,
  escapeinside={@}{@},
  showstringspaces=false,
  numberstyle=\tiny\color{solarized@base01},
  keywordstyle=\color{blue},
  stringstyle=\color{orange}\ttfamily,
  identifierstyle=\color{Periwinkle},
  commentstyle=\color{purple},
  emphstyle=\color{green},
  morekeywords={and, or, true, false},
  frame=single,
  rulecolor=\color{solarized@base2},
  rulesepcolor=\color{solarized@base2},
  literate={~}{\customtilde}1
}

% Define shell syntax highlighting colour scheme
% Ben: I cannot get the damn comment highlighting to work for the 'bash' language. No idea what the problem is, the internet seems to think that it should just work.
% Pat: I asked the internet why it thinks this.  It said something about cats.
\lstdefinestyle{terminal}
{
  language=bash,
  basicstyle=\termplainstyle,
  numbers=none,
  tabsize=2,
  breaklines=true,
  escapeinside={@}{@},
  frame=single,
  showstringspaces=false,
  numberstyle=\tiny\color{solarized@base01},
  keywordstyle=\color{solarized@orange},
  stringstyle=\color{solarized@red}\ttfamily,
  identifierstyle=\color{black},
  commentstyle=\color{solarized@violet},
  emphstyle=\color{solarized@green},
  frame=single,
  rulecolor=\color{solarized@base2},
  rulesepcolor=\color{solarized@base2},
  morekeywords={gambit, cmake, make, mkdir, gum, python, wget, tar, cp, pippi, mpirun},
  deletekeywords={test},
  literate = {/gambit}{{/}{\color{black}}gambit}6
             {gambit/}{{\color{black}}gambit{/}}6
             {gum/}{{\color{black}}gum{/}}4
             {/include}{{/}{\color{black}}include}8
             {cmake/}{{\color{black}}cmake/}6
             {.cmake}{{.}{\color{black}}cmake}6
             {.gum}{{.}{\color{black}}gum}6
             {.tar}{{.}{\color{black}}tar}4
             {source/}{{\color{black}}source{/}}7
             { type}{{\color{black}}{}type}5
             {~}{\customtilde}1
             {math}{{\color{solarized@orange}}math}4
}

% Terminal style with alternate escape character
\lstdefinestyle{terminalalt}
{
  language=bash,
  basicstyle=\footnotesize\ttfamily,
  numbers=none,
  tabsize=2,
  breaklines=true,
  escapeinside={*@}{@*},
  frame=single,
  showstringspaces=false,
  numberstyle=\tiny\color{solarized@base01},
  keywordstyle=\color{solarized@orange},
  stringstyle=\color{solarized@red}\ttfamily,
  identifierstyle=\color{black},
  commentstyle=\color{solarized@violet},
  emphstyle=\color{solarized@green},
  frame=single,
  rulecolor=\color{solarized@base2},
  rulesepcolor=\color{solarized@base2},
  morekeywords={gambit, cmake, make, mkdir},
  deletekeywords={test},
  literate = {\ gambit}{{\ }{\color{black}}gambit}7
             {/gambit}{{/}{\color{black}}gambit}6
             {gambit/}{{\color{black}}gambit{/}}6
             {/include}{{/}{\color{black}}include}8
             {cmake/}{{\color{black}}cmake/}6
             {.cmake}{{.}{\color{black}}cmake}6
             {~}{\customtilde}1
}

% Terminal style with alternate escape character
\lstdefinestyle{text}
{
  language={},
  basicstyle=\footnotesize\ttfamily,
  identifierstyle=\color{black},
  numbers=none,
  tabsize=2,
  breaklines=true,
  escapeinside={*@}{@*},
  showstringspaces=false,
  frame=single,
  rulecolor=\color{solarized@base2},
  rulesepcolor=\color{solarized@base2},
  literate={~}{\customtilde}1
}

% Define yaml syntax highlighting colour scheme
\lstdefinestyle{yaml}
{
  language=bash,
  escapeinside={@}{@},
  keywords={true,false,null},
  otherkeywords={},
  keywordstyle=\color{solarized@base0}\bfseries,
  basicstyle=\footnotesize\color{black}\ttfamily,
  identifierstyle=\YAMLkeystyle,
  sensitive=false,
  commentstyle=\YAMLcommentstyle,
  morecomment=[l]{\#},
  morecomment=[s]{/*}{*/},
  stringstyle=\YAMLstringstyle\ttfamily,
  moredelim=**[s][\YAMLkeystyle]{,}{:},   % switch to value style at : but back to key style at,
  moredelim=**[l][\YAMLvaluestyle]{:},    % switch to value style at :
  morestring=[b]',
  morestring=[b]",
  literate =    {---}{{\ProcessThreeDashes}}3
                {>}{{\textcolor{solarized@red}\textgreater}}1
                {gtr}{\textgreater}1
                {grt}{\textgreater}1
                {|}{{\textcolor{solarized@red}\textbar}}1
                {\ -\ }{{\mdseries\color{black}\ -\ \negmedspace}}3
                {\}}{{{\color{black} \}}}}1
                {\{}{{{\color{black} \{}}}1
                {[}{{{\color{black} [}}}1
                {]}{{{\color{black} ]}}}1
                {~}{\customtilde}1,
  breakindent=0pt,
  breakatwhitespace,
  columns=fullflexible
}

% Define gum syntax highlighting colour scheme
\lstdefinestyle{gum}
{
  language=bash,
  escapeinside={@}{@},
  keywords={true,false,null,all},
  otherkeywords={},
  keywordstyle=\color{solarized@base02}\bfseries,
  basicstyle=\footnotesize\color{black}\ttfamily,
  identifierstyle=\color{solarized@magenta},
  sensitive=false,
  commentstyle=\color{solarized@cyan}\ttfamily,
  morecomment=[l]{\#},
  morecomment=[s]{/*}{*/},
  stringstyle=\footnotesize\color{solarized@base01}\mdseries\ttfamily,
  moredelim=**[l][\footnotesize\color{solarized@base02}\mdseries]{:},    % switch to value style at :
  morestring=[b]',
  morestring=[b]",
  literate =    {---}{{\ProcessThreeDashes}}3
                {grt}{{\textcolor{solarized@magenta}\textgreater}}1
                {gtr}{{\textcolor{solarized@base02}\textgreater}}1
                {/>}{{\textcolor{solarized@magenta}\textgreater}}1
                {/<}{{\textcolor{solarized@magenta}\textless}}1
                {lss}{{\textcolor{solarized@base02}\textless}}1
                {pls}{{\textcolor{solarized@magenta}+}}1
                {mns}{{\textcolor{solarized@magenta}-}}1
                {|}{{\textcolor{solarized@base02}\textbar}}1
                {\ -\ }{{\mdseries\color{solarized@base02}\ -\ \negmedspace}}3
                {\}}{{{\color{solarized@base02} \}}}}1
                {\{}{{{\color{solarized@base02} \{}}}1
                {[}{{{\color{solarized@base02} [}}}1
                {]}{{{\color{solarized@base02} ]}}}1
                {~}{\customtilde}1,
  breakindent=0pt,
  breakatwhitespace,
  columns=fullflexible
}

% Define Mathematica syntax highlighting colour scheme
\lstdefinestyle{mathematica}
{
  language={Mathematica},
  basicstyle=\footnotesize\ttfamily,
  basewidth={0.53em,0.44em},
  numbers=none,
  tabsize=2,
  breaklines=true,
  postbreak=,
  escapeinside={@}{@},
  numberstyle=\tiny\color{black},
  showstringspaces=false,
  numberstyle=\tiny\color{solarized@base01},
  keywordstyle=\color{solarized@orange},
  stringstyle=\color{solarized@red}\ttfamily,
  identifierstyle=\color{solarized@orange}\ttfamily,
  commentstyle=\color{solarized@gray}\ttfamily,
  directivestyle=\color{solarized@orange}\ttfamily,
  emphstyle=\color{solarized@green},
  frame=single,
  rulecolor=\color{solarized@base2},
  rulesepcolor=\color{solarized@base2},
  literate={~} {\customtilde}1,
  moredelim=*[directive]\ \ \#,
  moredelim=*[directive]\ \ \ \ \#,
  mathescape=false
}

% Start with C++ style on
\lstset{style=cpp}

% Glossary commands
\newcommand{\cross}[1]{\ref{#1}}
\newcommand{\doublecross}[2]{\hyperref[#2]{\textbf{#1}}}
\newcommand{\doublecrosssf}[2]{\hyperref[#2]{\textbf{\textsf{#1}}}}
\newcommand{\gitem}[1]{\item[\textbf{#1}\label{#1}]}
\newcommand{\gsfitem}[1]{\item[\textbf{\textsf{#1}}\label{#1}]}
\newcommand{\gsfitemc}[1]{\item[\textbf{\textsf{#1}}\label{#1}:]}
\newcommand{\gentry}[2]{\gitem{#1}\input{"glossary/#2.glossentry"}}
\newcommand{\gsfentry}[2]{\gsfitem{#1}\input{"glossary/#2.glossentry"}}
\newcommand{\startglossary}{\section{Glossary}\label{glossary}Here we explain some terms that have specific technical definitions in \GB.\begin{description}}
\newcommand{\finishglossary}{\end{description}}

% Code commands
\newcommand{\bcode}{\begin{lstlisting}}
\newcommand{\ecode}{\end{lstlisting}}
\newcommand{\metavarf}[1]{\textit{\color{darkgreen}\footnotesize\textrm{#1}}}
\newcommand{\metavars}[1]{\textit{\color{darkgreen}\scriptsize\textrm{#1}}}
\newcommand{\metavar}{\metavarf}

% For sign(mu), etc.
\DeclareMathOperator{\sign}{sign}
\DeclareMathOperator\erf{erf}

% Physics units
\newcommand{\eV}{\ensuremath{\text{e}\mspace{-0.8mu}\text{V}}\xspace}
\newcommand{\keV}{\text{k\eV}\xspace}
\newcommand{\MeV}{\text{M\eV}\xspace}
\newcommand{\GeV}{\text{G\eV}\xspace}
\newcommand{\TeV}{\text{T\eV}\xspace}
\newcommand{\pb}{\text{pb}\xspace}
\newcommand{\fb}{\text{fb}\xspace}
\newcommand{\invpb}{\ensuremath{\pb^{-1}}\xspace}
\newcommand{\invfb}{\ensuremath{\fb^{-1}}\xspace}

% Physical quantities
\newcommand{\pt}{\ensuremath{p_\mathrm{T}}\xspace}
\newcommand{\et}{\ensuremath{E_\mathrm{T}}\xspace}
\newcommand{\etmiss}{\ensuremath{E_\mathrm{T}^\mathrm{\mspace{1.5mu}miss}}\xspace}
\newcommand{\etmissx}[1]{\ensuremath{E_\mathrm{T}^\mathrm{\mspace{1.5mu}miss,{#1}}}\xspace}
\newcommand{\hT}{\ensuremath{H_\mathrm{T}}\xspace}
\newcommand{\dphi}{\ensuremath{\Delta\phi}\xspace}
\newcommand{\sss}{\scriptscriptstyle}
\newcommand{\ms}{m_{\sss S}}
\newcommand{\msms}{\overline{m}_{\sss S}}
\newcommand{\lhs}{\lambda_{h\sss S}}
\newcommand{\ls}{\lambda_{\sss S}}
\newcommand{\lh}{\lambda_{h}}
\newcommand{\mh}{m_h}
\newcommand{\msmh}{\overline{m}_h}
\newcommand{\DR}{$\overline{DR}$\xspace}
\newcommand{\DRbar}{\DR}
\newcommand{\MSbar}{$\MSBar$\xspace}
\newcommand{\MSBar}{\overline{MS}}
\newcommand{\CL}{\text{CL}\xspace}
\newcommand{\CLs}{\ensuremath{\CL_{s}}\xspace}
\newcommand{\CLsb}{\ensuremath{\CL_{s+b}}\xspace}
\newcommand{\BR}{\ensuremath{\mathrm{BR}}\xspace}
\newcommand{\ztwo}{$\mathbb{Z}_2$\xspace}
\newcommand{\zthree}{$\mathbb{Z}_3$\xspace}

% Textual shortcuts
\newcommand{\ie}{i.e.\ }
\newcommand{\eg}{e.g.\ }
\newcommand{\atlas}{ATLAS\xspace}
\newcommand{\cms}{CMS\xspace}
\newcommand{\gambit}{\textsf{GAMBIT}\xspace}
\newcommand{\gambitversion}{1.0.0}
\newcommand{\gambitVer}{\gambit \textsf{\gambitversion}\xspace}
\newcommand{\darkbit}{\textsf{DarkBit}\xspace}
\newcommand{\cosmobit}{\textsf{CosmoBit}\xspace}
\newcommand{\colliderbit}{\textsf{ColliderBit}\xspace}
\newcommand{\flavbit}{\textsf{FlavBit}\xspace}
\newcommand{\specbit}{\textsf{SpecBit}\xspace}
\newcommand{\decaybit}{\textsf{DecayBit}\xspace}
\newcommand{\precisionbit}{\textsf{PrecisionBit}\xspace}
\newcommand{\scannerbit}{\textsf{ScannerBit}\xspace}
\newcommand{\examplebita}{\textsf{ExampleBit\_A}\xspace}
\newcommand{\examplebitb}{\textsf{ExampleBit\_B}\xspace}
\newcommand{\neutrinobit}{\textsf{NeutrinoBit}\xspace}
\newcommand{\BOSS}{\textsf{BOSS}\xspace}
\newcommand{\GB}{\gambit}
\newcommand{\DB}{\darkbit}
\newcommand{\omp}{\textsf{OpenMP}\xspace}
%\newcommand{\openmpi}{\textsf{OpenMPI}\xspace} % Shouldn't be talking about OpenMPI! Just refer to MPI.
\newcommand{\mpi}{\textsf{MPI}\xspace}
\newcommand{\posix}{\textsf{POSIX}\xspace}
\newcommand{\buckfast}{\textsf{BuckFast}\xspace}
\newcommand{\delphes}{\textsf{Delphes}\xspace}
\newcommand{\EOS}{\textsf{EOS}\xspace}
\newcommand{\flavio}{\textsf{Flavio}\xspace}
\newcommand{\pythia}{\textsf{Pythia}\xspace}
\newcommand{\pythiaeight}{\textsf{Pythia\,8}\xspace}
\newcommand{\PythiaEM}{\textsf{PythiaEM}\xspace}
\newcommand{\prospino}{\textsf{Prospino}\xspace}
\newcommand{\nllfast}{\textsf{NLL-fast}\xspace}
\newcommand{\madgraph}{\textsf{MadGraph}\xspace}
\newcommand{\MGaMCNLO}{\textsf{MadGraph5\_aMC@NLO}\xspace}
\newcommand{\fastjet}{\textsf{FastJet}\xspace}
\newcommand{\smodels}{\textsf{SModelS}\xspace}
\newcommand{\fastlim}{\textsf{FastLim}\xspace}
\newcommand{\checkmate}{\textsf{CheckMATE}\xspace}
\newcommand{\higgsbounds}{\textsf{HiggsBounds}\xspace}
\newcommand{\susypope}{\textsf{SUSYPope}\xspace}
\newcommand{\higgssignals}{\textsf{HiggsSignals}\xspace}
\newcommand{\ds}{\textsf{DarkSUSY}\xspace}
\newcommand{\darksusy}{\ds}
\newcommand{\wimpsim}{\textsf{WimpSim}\xspace}
\newcommand{\threebit}{\textsf{3-BIT-HIT}\xspace}
\newcommand{\pppc}{\textsf{PPPC4DMID}\xspace}
\newcommand{\mo}{\micromegas}
\newcommand{\micromegas}{\textsf{micrOMEGAs}\xspace}
\newcommand{\rivet}{\textsf{Rivet}\xspace}
\newcommand{\yoda}{\textsf{YODA}\xspace}
\newcommand{\contur}{\textsf{Contur}\xspace}
\newcommand{\feynrules}{\textsf{Feynrules}\xspace}
\newcommand{\feynhiggs}{\textsf{FeynHiggs}\xspace}
\newcommand{\FH}{\feynhiggs}
\newcommand{\eos}{\textsf{EOS}\xspace}
\newcommand{\flavkit}{\textsf{FlavorKit}\xspace}
\newcommand\FS{\FlexibleSUSY}
\newcommand\flexiblesusy{\FlexibleSUSY}
\newcommand\FlexibleSUSY{\textsf{FlexibleSUSY}\xspace}
\newcommand\FlexibleEFTHiggs{\textsf{FlexibleEFTHiggs}\xspace}
\newcommand\SOFTSUSY{\textsf{SOFTSUSY}\xspace}
\newcommand\SUSPECT{\textsf{SuSpect}\xspace}
\newcommand\NMSSMCalc{\textsf{NMSSMCALC}\xspace}
\newcommand\NMSSMTools{\textsf{NMSSMTools}\xspace}
\newcommand\NMSPEC{\textsf{NMSPEC}\xspace}
\newcommand\NMHDECAY{\textsf{NMHDECAY}\xspace}
\newcommand\HDECAY{\textsf{HDECAY}\xspace}
\newcommand\prophecy{\textsf{PROPHECY4F}\xspace}
\newcommand\SDECAY{\textsf{SDECAY}\xspace}
\newcommand\SUSYHIT{\textsf{SUSY-HIT}\xspace}
\newcommand\susyhd{\textsf{SUSYHD}\xspace}
\newcommand\HSSUSY{\textsf{HSSUSY}\xspace}
\newcommand\susyhit{\SUSYHIT}
\newcommand\gmtwocalc{\textsf{GM2Calc}\xspace}
\newcommand\SARAH{\textsf{SARAH}\xspace}
\newcommand\SPheno{\textsf{SPheno}\xspace}
\newcommand\superiso{\textsf{SuperIso}\xspace}
\newcommand\heplike{\textsf{HEPLike}\xspace}
\newcommand\SFOLD{\textsf{SFOLD}\xspace}
\newcommand\HFOLD{\textsf{HFOLD}\xspace}
\newcommand\FeynHiggs{\textsf{FeynHiggs}\xspace}
\newcommand\Mathematica{\textsf{Mathematica}\xspace}
\newcommand\Kernel{\textsf{Kernel}\xspace}
\newcommand\WSTP{\textsf{WSTP}\xspace}
\newcommand\lilith{\textsf{Lilith}\xspace}
\newcommand\nulike{\textsf{nulike}\xspace}
\newcommand\gamLike{\textsf{gamLike}\xspace}
\newcommand\gamlike{\gamLike}
\newcommand\daFunk{\textsf{daFunk}\xspace}
\newcommand\pippi{\textsf{pippi}\xspace}
\newcommand\MultiNest{\textsf{MultiNest}\xspace}
\newcommand\multinest{\MultiNest}
\newcommand\Polychord{\textsf{PolyChord}\xspace}
\newcommand\polychord{\Polychord}
\newcommand\great{\textsf{GreAT}\xspace}
\newcommand\twalk{\textsf{T-Walk}\xspace}
\newcommand\diver{\textsf{Diver}\xspace}
\newcommand\ddcalc{\textsf{DDCalc}\xspace}
\newcommand\capgen{\textsf{Capt'n General}\xspace}
\newcommand{\gum}{\textsf{GUM}\xspace}
\newcommand{\dgum}{\!\!\term{.gum}\!\xspace}
\newcommand{\fr}{\textsf{FeynRules}\xspace}
\newcommand{\sarah}{\textsf{SARAH}\xspace}
\newcommand{\CH}{\textsf{CalcHEP}\xspace}
\newcommand{\MG}{\textsf{MadGraph}\xspace}
\newcommand{\mdm}{\textsf{MadDM}\xspace}
\newcommand{\ufo}{\textsf{UFO}\xspace}
\newcommand{\veva}{\textsf{Vevacious}\xspace}
\newcommand{\spheno}{\textsf{SPheno}\xspace}
\newcommand{\FA}{\textsf{FeynArts}\xspace}
\newcommand{\ddm}{\textsf{DirectDM}\xspace}
\newcommand\tpcmc{\textsf{TPCMC}\xspace}
\newcommand\nest{\textsf{NEST}\xspace}
\newcommand\luxcalc{\textsf{LUXCalc}\xspace}
\newcommand\xx{\raisebox{0.2ex}{\smaller ++}\xspace}
\newcommand\Cpp{\textsf{C\xx}\xspace}
\newcommand\Cppeleven{\textsf{C\raisebox{0.2ex}{\smaller ++}11}\xspace}
\newcommand\plainC{\textsf{C}\xspace}
\newcommand\Python{\textsf{Python}\xspace}
\newcommand\python{\Python}
\newcommand\Cython{\textsf{Cython}\xspace}
\newcommand\cython{\Cython}
\newcommand\Fortran{\textsf{Fortran}\xspace}
\newcommand\YAML{\textsf{YAML}\xspace}
\newcommand\Yaml{\YAML}
\newcommand\cmake{\textsf{CMake}\xspace}
\newcommand\gpp{\textsf{g++}\xspace}
\newcommand\gfortran{\textsf{gfortran}\xspace}
\newcommand\icpc{\textsf{icpc}\xspace}
\newcommand\ifort{\textsf{ifort}\xspace}
\newcommand\Zenodo{\textsf{Zenodo}\xspace}

\newcommand\beq{\begin{equation}}
\newcommand\eeq{\end{equation}}

\newcommand{\mail}[1]{\href{mailto:#1}{#1}}
%\renewcommand{\url}[1]{\href{#1}{#1}}

% Author comments
\newcommand{\TODO}[1]{\textbf{\textcolor{red}{#1}}}
\newcommand{\tb}[1]{{\color{green}\textbf{[TB: #1]}}}
\newcommand{\je}[1]{{\color{pink}\textbf{[JE: #1]}}}
\newcommand{\ps}[1]{\Pat{#1}}
\newcommand{\cw}[1]{{\color{red}Christoph: #1}}
\newcommand{\gm}[1]{{\color{violet}\textbf{Greg: #1}}}
\newcommand{\Ben}[1]{\textbf{\color{magenta}Ben #1}}
\newcommand{\Csaba}[1]{\textrm{\color{blue!60!green}#1}}
\newcommand{\Pat}[1]{\textbf{\color{blue}Pat: #1}}
\newcommand{\Peter}[1]{\textbf{\color{purple}Peter #1}}
\newcommand{\Anders}[1]{\textbf{\color{brown}Anders: #1}}
\newcommand{\Martin}[1]{\textbf{\color{yellow}Martin: #1}}
\newcommand{\James}[1]{\textbf{\color{teal}James #1}}
\newcommand{\Abram}[1]{\textbf{\color{black!20!gray!60!magenta}Abram: #1}}
\newcommand{\Andy}[1]{\textbf{\color{orange!60!black}Andy: #1}}
\newcommand{\Marcin}[1]{\textbf{\color{darkgreen}Marcin #1}}
\newcommand{\nazila}{\textsc{{\color{red}{Nazila}}}\xspace}
\newcommand{\sw}[1]{\textbf{\color{red}Sebastian: #1}}
\newcommand{\fk}[1]{\textbf{\color{magenta}[FK: #1]}}
\newcommand{\ar}[1]{\textbf{\color{pink}Are: #1}}
\newcommand{\Tomas}[1]{\textbf{\color{black!20!gray!60!red}Tomas: #1}}
\newcommand{\pst}[1]{\textbf{\color{orange!30!black!70!green}[Patrick: #1]}}
\newcommand{\jr}[1]{\textbf{\color{orchid}[JR]:} {\color{orchid}#1}}
\newcommand{\jredit}[1]{{\color{orchid}#1}}
\newcommand{\SB}[1]{\textbf{\color{orange}Sanjay:} {\color{orange}#1}}
\newcommand{\EC}[1]{\textbf{\color{Blue!90!black}Eliel: #1}}
\newcommand{\will}[1]{\textbf{\color{darkgreen}Will:} {\color{darkgreen}#1}}
\newcommand{\seb}[1]{\textbf{\color{teal}{[Seb: #1]}}\xspace}
\newcommand{\selim}[1]{{\textbf{\color{purple!50!pink}{[{\bf Selim}: #1]}}}}
\newcommand{\cullan}[1]{\textbf{\color{cyan}Cullan:} {\color{cyan}#1}}
\newcommand{\cchang}[1]{\textbf{\color{Violet}ChrisC:} {\color{Violet}#1}}


\def\at{\alpha_t}
\def\ab{\alpha_b}
\def\as{\alpha_s}
\def\atau{\alpha_{\tau}}
\def\oat{\mathcal{O}(\at)}
\def\oab{\mathcal{O}(\ab)}
\def\oatau{\mathcal{O}(\atau)}
\def\oatab{\mathcal{O}(\at\ab)}
\def\oatas{\mathcal{O}(\at\as)}
\def\oabas{\mathcal{O}(\ab\as)}
\def\oatababq{\mathcal{O}(\at\ab + \ab^2)}
\def\oatqatababq{\mathcal{O}(\at^2 + \at\ab + \ab^2)}
\def\oatasatq{\mathcal{O}(\at\as + \at^2)}
\def\oatasabas{\mathcal{O}(\at\as +\ab\as)}
\def\oatasabasatq{\mathcal{O}(\at\as + \at^2 +\ab\as)}
\def\oatq{\mathcal{O}(\at^2)}
\def\oabq{\mathcal{O}(\ab^2)}
\def\oatauq{\mathcal{O}(\atau^2)}
\def\oabatau{\mathcal{O}(\ab \atau)}
\def\oatplusabsq{\mathcal{O}((\at+\ab)^2)}
\def\oas{\mathcal{O}(\as)}
\def\oatauqatab{\mathcal{O}(\atau^2 +\ab \atau )}

% Custom \chapter-like command  (svjour3 document class does not define \part or \chapter)
\newcommand{\segment}[1]{
 {\clearpage\noindent\phantomsection\huge\it#1\par}
 {\addcontentsline{toc}{section}{\it#1}}
}
