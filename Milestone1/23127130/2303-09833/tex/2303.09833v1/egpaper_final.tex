\documentclass[10pt,twocolumn,letterpaper]{article}

\usepackage{iccv}
\usepackage{times}
\usepackage{epsfig}
\usepackage{graphicx}
\usepackage{amsmath}
\usepackage{amssymb}
\usepackage{algorithm}
\usepackage{algorithmicx}
\usepackage[noend]{algpseudocode}

\usepackage{booktabs}

\usepackage{caption}
\usepackage{subcaption}
\usepackage[dvipsnames]{xcolor}

\usepackage{pifont}
\usepackage{enumitem}
\usepackage{xfrac} 
\usepackage{makecell}
\setlist[itemize]{leftmargin=*}

\usepackage{multirow}
\usepackage{color}
\usepackage{xcolor}
\usepackage{colortbl}
\definecolor{color3}{rgb}{0.95,0.95,0.95}

\usepackage{tipa}

\usepackage[title]{appendix}
\usepackage[accsupp]{axessibility}

% Include other packages here, before hyperref.

% If you comment hyperref and then uncomment it, you should delete
% egpaper.aux before re-running latex.  (Or just hit 'q' on the first latex
% run, let it finish, and you should be clear).
\usepackage[breaklinks=true,bookmarks=false]{hyperref}

\iccvfinalcopy % *** Uncomment this line for the final submission

\def\iccvPaperID{6537} % *** Enter the ICCV Paper ID here
\def\httilde{\mbox{\tt\raisebox{-.5ex}{\symbol{126}}}}

% Pages are numbered in submission mode, and unnumbered in camera-ready
\ificcvfinal\pagestyle{empty}\fi

\newcommand{\chen}[1]{{{\color{red}[\textbf{Chen}: \emph{#1}]}}}

\begin{document}

%%%%%%%%% TITLE
\title{FreeDoM: Training-Free Energy-Guided Conditional Diffusion Model}


\author{Jiwen Yu$^1$\qquad\quad Yinhuai Wang$^1$\qquad\quad Chen Zhao$^2$ \quad\quad Bernard Ghanem$^2$ \quad\quad  Jian Zhang$^{1}$\\
$^1$ Peking University Shenzhen Graduate School \qquad $^2$ KAUST\\
% {\tt\small \{yujiwen, yinhuaiwang\}@stu.pku.edu.cn} \qquad\quad {\tt\small \{jianzhang\}@pku.edu.cn}\\
\url{https://github.com/vvictoryuki/FreeDoM}
}

% \maketitle
% Remove page # from the first page of camera-ready.
\ificcvfinal\thispagestyle{empty}\fi

% \begin{figure*}[!ht]
%   \centering
%   \vspace{-0.4cm}
%   \includegraphics[width=1\linewidth]{figure/figure1.pdf}
%   \vspace{-0.4cm}
%   \caption{Overview of applications.}
%   \vspace{-0.4cm}
% \label{fig:overview} 
% \end{figure*}

\twocolumn[{
\renewcommand\twocolumn[1][]{#1}
\maketitle
\centering
\vspace{-0.4cm}
\includegraphics[width=\textwidth]{figure/figure1_v3.pdf}
\vspace{-0.5cm}
\captionsetup{type=figure}
\caption{\textbf{FreeDoM controls the generation process of diffusion models in a training-free way.} Here, we demonstrate some 
results of the applications FreeDoM supports. Part (a)-(c) show various face editing applications with training-free guidance.
(a) We use the segmentation map, sketch, landmarks, and face ID as conditions to guide the generation process of an unconditional diffusion model; 
(b) We use CLIP~\cite{radford2021learning} based text guidance to control image synthesis and editing. For editing, we use the segmentation masks to limit the editing areas (see Fig.~\ref{fig:trick_segmentation} for details);
(c) We combine different conditions to control the generation process.
Part (d)-(f) show that training-free guidance can work with other training-required conditional diffusion models, like Stable Diffusion~\cite{rombach2022high} and ControlNet~\cite{zhang2023adding}, to achieve a more sophisticated control mechanism. 
The conditions of scribbles in (d), human poses in (e), and prompt texts in (f) are controlled by the training-required interfaces provided by ControlNet and Stable Diffusion. Training-free energy functions control the conditions of face IDs from the reference images in (e) and style images in (d) and (f).
\textbf{Zoom in for best view.} 
}
\label{fig:overview}

\vspace{0.9cm}
}]



%%%%%%%%% ABSTRACT
\begin{abstract}
% Recently, conditional diffusion models have been studied and used in many applications because of their fantastic generation ability. 
Recently, conditional diffusion models have gained popularity in numerous applications due to their exceptional generation ability. 
However, many existing methods are training-required. They need to train a time-dependent classifier or a condition-dependent score estimator, which increases the cost of constructing conditional diffusion models and is inconvenient to transfer across different conditions.
% Some current works try to overcome this limitation by proposing training-free solutions, but most can only apply to a single category of tasks rather than to more general conditions. 
Some current works aim to overcome this limitation by proposing training-free solutions, but most can only be applied to a specific category of tasks and not to more general conditions.
In this work, we propose a training-\textbf{Free} conditional \textbf{D}iffusi\textbf{o}n \textbf{M}odel (\textbf{Free}\textbf{Do}\textbf{M}) used for various conditions. 
% Specifically, we use off-the-shelf pre-trained networks, e.g., a face detection model, to build time-independent energy functions, which achieve training-free guidance of the generation process.
Specifically, we leverage off-the-shelf pre-trained networks, such as a face detection model, to construct time-independent energy functions, which guide the generation process without requiring training.
Furthermore, because the construction of the energy function is very flexible and adaptable to various conditions, our proposed FreeDoM has a broader range of applications than existing training-free methods. 
FreeDoM is advantageous in its simplicity, effectiveness, and low cost. Experiments demonstrate that FreeDoM is effective for various conditions and suitable for diffusion models of diverse data domains, including image and latent code domains.
% \footnote{For reproducible research, the complete source codes of our proposed method will be made publicly available.}
\end{abstract}
\vspace{-1cm}
%%%%%%%%% BODY TEXT


\section{Introduction}
Recently, diffusion models have been demonstrated to outperform previous state-of-the-art generative models~\cite{dhariwal2021diffusion}, such as GANs \cite{gan,mirza2014conditional,brocklarge}. The impressive generative power of diffusion models~\cite{ho2020denoising, song2019generative, song2021scorebased} has motivated researchers to apply diffusion models to various downstream tasks. Conditional generation is one of the most popular focus areas. Conditional diffusion models (CDMs) with diverse conditions have been proposed, such as text~\cite{Kim_2022_CVPR, avrahami2022blended, gu2021vector, liu2019more, rombach2022high, sheynin2023knndiffusion, ramesh2022hierarchical, saharia2022photorealistic, liu2022compositional, nichol2021glide}, class labels~\cite{dhariwal2021diffusion}, degraded images~\cite{Choi_2021_ICCV, chung2023diffusion, Chung_2022_CVPR, chung2022improving, kawar2022denoising, Lugmayr_2022_CVPR, song2023pseudoinverseguided, song2021solving, wang2023zeroshot, saharia2022palette, saharia2022image, whang2022deblurring}, segmentation maps~\cite{mou2023adapter, zhang2023adding}, landmarks~\cite{mou2023adapter, zhang2023adding}, hand-drawn sketches~\cite{mou2023adapter, zhang2023adding}, style images~\cite{mou2023adapter, zhang2023adding}, etc. 
These CDMs can be roughly divided into two categories: training-required or training-free.

A typical type of training-required CDMs trains a time-dependent classifier to guide the noisy image $\mathbf{x}_t$ toward the given condition $\mathbf{c}$ ~\cite{dhariwal2021diffusion, nichol2021glide, zhao2022egsde, liu2019more}. Another branch of training-required CDMs directly trains a new score estimator $s(\mathbf{x}_t, t, \mathbf{c})$ conditioned on $\mathbf{c}$~\cite{mou2023adapter, ho2022classifier, rombach2022high, saharia2022palette, saharia2022image, whang2022deblurring, zhang2023adding, Kim_2022_CVPR, avrahami2022blended, gu2021vector, ramesh2022hierarchical, nichol2021glide}. 
% It is the approximation of the conditional score function, that is, $\nabla_{\mathbf{x}_t}\log p(\mathbf{x}_t|\mathbf{c})\approx s_{\theta}(\mathbf{x}_t, t, \mathbf{c})$. 
These methods yield impressive performance but are not flexible. Once a new target condition is needed for generation, they have to retrain or finetune the models, which is inconvenient and expensive.

In contrast, training-free CDMs try to solve the same problems without extra training. \cite{ron2022nulltext, feng2023trainingfree, hertz2022prompt} attempt to use the cross-attention control to realize the conditional generation; 
\cite{Choi_2021_ICCV, chung2023diffusion, Chung_2022_CVPR, chung2022improving, kawar2022denoising, Lugmayr_2022_CVPR, song2023pseudoinverseguided, song2021solving, wang2023zeroshot,
wang2023unlimited} directly modify the intermediate results to achieve zero-shot image restoration; \cite{meng2022sdedit} realizes image translation by adjusting the initial noisy images. While these methods are effective in a single application, they are difficult to generalize to a wider range of conditions, e.g., style, face ID, and segmentation masks.

% 我们的思路:
% (1)用能量函数来引导控制,由于能量函数的构造非常灵活,这将允许我们适用各种条件 (energy-guided general condition)
% (2)用off-the-shelf预训练的模型来构造能量函数,回避训练(training-free)
In order to make CDMs support a wide range of conditions 
% In this paper, we aim to realize CDMs controlled by a wide range of conditions 
in a training-free manner, this paper  proposes a training-\textbf{Free} conditional \textbf{D}iffusi\textbf{o}n \textbf{M}odel (\textbf{Free}\textbf{Do}\textbf{M}) with the following two key points. 
% Our solution can be summarized into two key points: 
\textit{Firstly}, to emphasize generalization, 
we propose a sampling process guided by the energy function~\cite{zhao2022egsde, lecun2006tutorial}, which is very flexible to construct and can be applied to various conditions.
% we choose to use the energy function~\cite{zhao2022egsde} to guide the sampling process because the construction of the energy function is very flexible and can be applied to various conditions. 
\textit{Secondly}, to make the proposed method training-free, we use off-the-shelf pre-trained time-independent models, which are easily accessible online, to construct the energy function.


% 我们的贡献:
Our FreeDoM has the following advantages: (1) \textbf{Simple and effective.} We only insert a derivative step 
% \chen{What do you mean by `automatic'?}
of the energy function gradient into the unconditional sampling process of the original diffusion models. Extensive experiments show its 
% {\color{purple} excellent}
effective 
controlling capability. (2) \textbf{Low cost and efficient.} The energy functions we construct are time-independent and do not need to be retrained. The diffusion models we choose do not need to be trained on the desired conditions. Thanks to the efficient time-travel strategy we use for large data domains, the number of sampling steps we use is quite small, which speeds up the sampling process while ensuring good generated results. (3) \textbf{Amenable to a wide range of applications.} The conditions our method supports include, but are not limited to, text, segmentation maps, sketches, landmarks, face IDs, style images, etc. In addition, various complex but interesting applications can be realized by combining multiple conditions. (4) \textbf{Supports different types of diffusion models.} Regardless of the considered data domain, such as human face images, images in ImageNet, or latent codes extracted from an image encoder, extensive experiments demonstrate that our method does well on all of them. 
% \chen{We should be more specific about `support'. We mean our method can do well with different models, not only that it is compatible with those models.}

\section{Related Work}
\label{sec:related_work}
% In this section, we will select some representative works in CDMs to analyze their contributions and limitations.

\subsection{Training-Required Methods}
The training-required methods can obtain strong control generation ability thanks to supervised learning with data pairs. One of the most prominent applications of these methods is the text-to-image task. The most widely used text-to-image model, Stable Diffusion~\cite{rombach2022high}, generates high-quality images that conform to the text description by inputting a prompt text. Recent works, such as ControlNet~\cite{zhang2023adding} and T2I-Adapter~\cite{mou2023adapter}, have introduced more training-required conditional interfaces to Stable Diffusion, such as edge maps, segmentation maps, depth maps, etc.

Although these training-required methods can achieve satisfactory control results under trained conditions, the cost of training is still a factor to be considered, especially for the scenario that requires more complex control with multiple conditions. The training-required method is not the cheapest or most convenient solution in practical applications.

\subsection{Training-Free Methods}
The training-free methods develop various interesting technologies to realize the training-free condition generation on some tasks exploiting the unique nature of the diffusion model, namely, the iterative denoising process. 
\cite{hertz2022prompt} proposes to inject the target cross attention maps to the source cross attention maps to solve the prompt-to-prompt task without training. The limitation of this method is that a text prompt is needed to anchor the content of the image to be edited in advance.
DDNM~\cite{wang2023zeroshot} proposes to use the Range-Null Space Decomposition to modify the intermediate results to solve the image restoration in a training-free way. It is based on the degradation operators of image restoration tasks and is hard to be adopted in other applications.
SDEdit~\cite{meng2022sdedit} proposes to adjust the initial noisy images to control the generation process, which is useful in stroke-based image synthesis and editing. Its limitation is that the guidance of stroke is not precise and versatile. 

According to the limitations mentioned above, the training-free CDMs for a broad range of applications need to be studied urgently. We have noticed some recent efforts~\cite{parmar2023zero, bansal2023universal} in this area.
Our FreeDoM has a faster generation speed and applies to a broader range of applications.
% The supplementary materials will discuss the relationship between us in detail.


% \subsection{Zero-Shot Image Restoration}
% Image restoration can be seen as a conditional generation task with degraded images as the conditions. Many existing methods~\cite{Choi_2021_ICCV, chung2023diffusion, Chung_2022_CVPR, chung2022improving, kawar2022denoising, Lugmayr_2022_CVPR, song2023pseudoinverseguided, song2021solving, wang2023zeroshot} try to zero-shot restore images with diffusion models by modifying the intermediate generated results. For example, \cite{Lugmayr_2022_CVPR} paste the unmasked region to the intermediate results to solve the image inpainting task; \cite{kawar2022denoising} use Singular Value Decomposition to force the intermediate results constraint to the data consistency; \cite{wang2023zeroshot} performs the Range-Null Space Decomposition to do the same thing. These methods are designed for only image restoration tasks and fail in other type of conditions.

% \subsection{Image Translation from Noisy Source}

% \subsection{Cross Attention Control for Prompt-to-Prompt}

\section{Preliminaries}
\subsection{Score-based Diffusion Models}
Score-based Diffusion Models (SBDMs)~\cite{song2019generative, song2021scorebased} are a kind of diffusion model based on score theory, which reveals that the essence of diffusion models is to estimate the score function $\nabla_{\mathbf{x}_{t}}\log p(\mathbf{x}_{t})$, where $\mathbf{x}_t$ is noisy data. 
During the sampling process, SBDMs predict $\mathbf{x}_{t-1}$ from $\mathbf{x}_t$ using the estimated score step by step. In our work, we resort to discrete SBDMs with the setting of DDPM~\cite{ho2020denoising} and its sampling formula is:
% In order to derive the following methods of CDMs, the score theory is necessary to introduce.
% The SDE-based discrete sampling process of SBDMs is:
% \begin{equation}
%     \mathbf{x}_{t-1}=\mathbf{x}_{t}-f(\mathbf{x}_{t},t)+g^2(t)\nabla_{\mathbf{x}_{t}}\log p(\mathbf{x}_{t})+g(t)\epsilon,
%     \label{eq:sde_discrete_sampling}
% \end{equation}
% where $f(\cdot,t):\mathbb{R}^{D}\to\mathbb{R}^{D}$ represents the drift coefficient, $g(t):\mathbb{R}\to\mathbb{R}$ denotes the diffusion coefficient, and $\epsilon\sim\mathcal{N}(\mathbf{0}, \mathbf{I})$ is a randomly sampled Gaussian noise. Eq.~\ref{eq:sde_discrete_sampling} can be speicified as Dinoising Diffusion Probabilistic Models (DDPMs)~\cite{ho2020denoising} when the coefficients are defined as $f(\mathbf{x},t)=-\frac{1}{2}\beta_t\mathbf{x}$, $g(t)=\sqrt{\beta_t}$, where $\beta_t\in\mathbb{R}$ is the pre-defined parameter. Then we can write the sampling process of SDE-based discrete DDPMs:
\begin{equation}
    \mathbf{x}_{t-1}=(1+\frac{1}{2}\beta_t)\mathbf{x}_t+\beta_t\nabla_{\mathbf{x}_{t}}\log p(\mathbf{x}_{t})+\sqrt{\beta_t}\boldsymbol{\epsilon},
    \label{eq:ddpm_sampling}
\end{equation}
where $\boldsymbol{\epsilon}\sim\mathcal{N}(\mathbf{0}, \mathbf{I})$ is randomly sampled Gaussian noise and $\beta_t\in\mathbb{R}$ is a pre-defined parameter.
In actual implementation, the score function will be estimated using a score estimator $s(\mathbf{x}_t, t)$, that is, $s(\mathbf{x}_t, t)\approx\nabla_{\mathbf{x}_{t}}\log p(\mathbf{x}_{t})$.
However, the original diffusion models can only serve as an unconditional generator with randomly synthesized results.

\subsection{Conditional Score Function}
In order to adapt the generative power of the diffusion models to different downstream tasks, conditional diffusion models (CDMs) are needed. SDE~\cite{song2021scorebased} proposed to control the generated results with a given condition $\mathbf{c}$ by modifying the score function as $\nabla_{\mathbf{x}_{t}}\log p(\mathbf{x}_{t}|\mathbf{c})$. Using the Bayesian formula $p(\mathbf{x}_t|\mathbf{c})=\frac{p(\mathbf{c}|\mathbf{x}_t)p(\mathbf{x}_t)}{p(\mathbf{c})}$, we can rewrite the conditional score function as two terms:
\begin{equation}
    \nabla_{\mathbf{x}_t}\log p(\mathbf{x}_t|\mathbf{c}) = \nabla_{\mathbf{x}_t}\log p(\mathbf{x}_t) + \nabla_{\mathbf{x}_t}\log p(\mathbf{c}|\mathbf{x}_t),
    \label{eq:condition_score_theory}
\end{equation}
where the first term $\nabla_{\mathbf{x}_t}\log p(\mathbf{x}_t)$ can be estimated using the pre-trained unconditional score estimator $s(\cdot, t)$ and the second term $\nabla_{\mathbf{x}_t}\log p(\mathbf{c}|\mathbf{x}_t)$ is the critical part of constructing conditional diffusion models. 
We can interpret the second term  $\nabla_{\mathbf{x}_t}\log p(\mathbf{c}|\mathbf{x}_t)$ as a correction gradient, pointing $\mathbf{x}_t$ to a hyperplane in the data space, where all data are compatible with the given condition $\mathbf{c}$.
Classifier-based methods~\cite{dhariwal2021diffusion, nichol2021glide, zhao2022egsde, liu2019more} train a time-dependent classifier to compute this correction gradient for conditional guidance.

\subsection{Energy Diffusion Guidance}
Modeling the correction gradient $\nabla_{\mathbf{x}_t}\log p(\mathbf{c}|\mathbf{x}_t)$ remains an open question. A flexible and straightforward way is resorting to the energy function~\cite{zhao2022egsde, lecun2006tutorial} as follows: 
% We can choose an energy function $\mathcal{E}(\mathbf{c}, \mathbf{x}_t)$ which satisfies:
\begin{equation}
    p(\mathbf{c}|\mathbf{x}_t)=\frac{\exp\{-\lambda\mathcal{E}(\mathbf{c}, \mathbf{x}_t)\}}{\mathnormal{Z}},
\end{equation}
where $\lambda$ denotes the positive temperature coefficient and $\mathnormal{Z}>0$ denotes a normalizing constant, computed as $\mathnormal{Z}=\int_{\mathbf{c}\in\mathcal{C}}\exp\{-\lambda\mathcal{E}(\mathbf{c}, \mathbf{x}_t)\}$ where $\mathcal{C}$ denotes the domain of the given conditions. 
% \chen{$\mathcal{C}$ should be a set, not a distribution.}
$\mathcal{E}(\mathbf{c}, \mathbf{x}_t)$ is an energy function that measures the compatibility between the condition $\mathbf{c}$ and the noisy image $\mathbf{x}_t$ --- its value will be smaller when $\mathbf{x}_t$ is more compatible with $\mathbf{c}$. If $\mathbf{x}_t$ satisfies the constraint of $\mathbf{c}$ perfectly, the energy value should be zero. Any function satisfying the above property 
% measuring the compatibility between $\mathbf{x}_t$ and $\mathbf{c}$ 
can serve as a feasible energy function, with which we just need to adjust the coefficient $\lambda$ to obtain $p(\mathbf{c}|\mathbf{x}_t)$. 

Therefore, the correction gradient $\nabla_{\mathbf{x}_t}\log p(\mathbf{c}|\mathbf{x}_t)$ can be implemented with the following:
% of CDMs, we can replace  as:
\begin{equation}
    \nabla_{\mathbf{x}_t}\log p(\mathbf{c}|\mathbf{x}_t)\propto -\nabla_{\mathbf{x}_t}\mathcal{E}(\mathbf{c}, \mathbf{x}_t),
    \label{eq:score_approx_energy}
\end{equation}
which is referred to as energy guidance. 
With Eq.~(\ref{eq:ddpm_sampling}), Eq.~(\ref{eq:condition_score_theory}), and Eq.~(\ref{eq:score_approx_energy}), we get the conditional sampling:
% \begin{align}
% \mathbf{x}_{t-1}&=(1+\frac{1}{2}\beta_t)\mathbf{x}_t \notag \\
% &+\beta_t\nabla_{\mathbf{x}_{t}}\log p(\mathbf{x}_{t})-\rho_t\nabla_{\mathbf{x}_t}\mathcal{E}(\mathbf{c}, \mathbf{x}_t)+\sqrt{\beta_t}\epsilon,
%     \label{eq:energy_guided_sampling}
% \end{align}
\begin{equation}
    \mathbf{x}_{t-1}=\mathbf{m}_t - \rho_t\nabla_{\mathbf{x}_t}\mathcal{E}(\mathbf{c}, \mathbf{x}_t),
    \label{eq:energy_guided_sampling}
\end{equation}
where $\mathbf{m}_t = (1+\frac{1}{2}\beta_t)\mathbf{x}_t+\beta_t\nabla_{\mathbf{x}_{t}}\log p(\mathbf{x}_{t})+\sqrt{\beta_t}\boldsymbol{\epsilon}$, and $\rho_t$ is a scale factor, which  can be seen as the learning rate of the correction term.
% gradient.
Eq.~(\ref{eq:energy_guided_sampling}) is a generic formulation of conditional diffusion models, which enables the use of different  energy functions. 
% {\color{purple} Eq.~\ref{eq:energy_guided_sampling} implements the conditional diffusion models flexibly because we have great freedom to choose the appropriate energy function.} 


\section{The Proposed FreeDoM Method}
In Sec.~\ref{subsec:approximation}, we approximate the time-dependent energy function using time-independent distance measuring functions, making our method training-free and flexible  for various conditions. 
In Sec.~\ref{subsec:ttt}, we first analyze the reason why the energy guidance fails in a large data domain and then propose an efficient version of the time-travel strategy~\cite{Lugmayr_2022_CVPR, wang2023zeroshot}.
In Sec.~\ref{subsec:construct_energy_functions}, we describe the details of how to construct the energy functions.
In Sec.~\ref{subsec:specific_examples}, we provide specific examples of supported conditions.


\subsection{Approximate Time-Dependent Energy}
\label{subsec:approximation}

% {\color{purple}In order to support a wide range of conditions, we choose the energy function to guide the generation because the construction of the energy function is very flexible.}\chen{?}
% In order to support a wide range of conditions,
We use the energy function to guide the generation due to its flexibility to  construct and suitability to various conditions. 
% The details about how to build a training-free energy function are introduced in Sec.~\ref{subsec:construct_energy_functions}.
Existing 
classifier-based 
% \chen{We didn't mention classifier-based training-free models before. What do you refer to here?}
methods~\cite{dhariwal2021diffusion, nichol2021glide, zhao2022egsde, liu2019more}  choose time-dependent distance measuring functions $\mathcal{D}_{\boldsymbol{\phi}}(\mathbf{c}, \mathbf{x}_t, t)$ to approximate the energy functions as follows: 
\begin{equation}
    \mathcal{E}(\mathbf{c}, \mathbf{x}_t)\approx  \mathcal{D}_{\boldsymbol{\phi}}(\mathbf{c}, \mathbf{x}_t, t),
    \label{eq:e_st}
\end{equation}
where $\boldsymbol{\phi}$ defines the pre-trained parameters. $\mathcal{D}_{\boldsymbol{\phi}}(\mathbf{c}, \mathbf{x}_t, t)$ computes the distance
% \chen{similarity and distance are opposite. Which exactly?} 
between the given condition $\mathbf{c}$ and noisy intermediate results $\mathbf{x}_t$. 
% Similarity measuring functions for noisy data $\mathbf{x}_t$ is hard to con and we have to train or finetune one for one specific type of conditions. 
The distance measuring functions for noisy data $\mathbf{x}_t$ cannot be directly constructed because it is difficult to find an existing pre-trained network for noisy images. In this case, we have to train (or finetune) a time-dependent network for each type of condition.

% , which makes the time-dependent energy-guided methods not training-freely and inconvenient to generalize between different conditions.

Compared with time-dependent networks, time-independent distance measuring functions for clean data $\mathbf{x}_{0}$ are widely available. Many off-the-shelf pre-trained networks such as classification networks, segmentation networks, and face ID encoding networks are open-source and work well on clean images. We denote these distance measuring networks for clean data as $\mathcal{D}_{\boldsymbol{\theta}}(\mathbf{c}, \mathbf{x}_0)$, where $\boldsymbol{\theta}$ denotes their pre-trained parameters.
% To propose the training-free energy-guided conditional diffusion models,
To use these networks for the energy functions, a straightforward way is to approximate $\mathcal{D}_{\boldsymbol{\phi}}(\mathbf{c}, \mathbf{x}_t, t)$ using $\mathcal{D}_{\boldsymbol{\theta}}(\mathbf{c}, \mathbf{x}_0)$, formulated as:
\begin{equation}
    \mathcal{D}_{\boldsymbol{\phi}}(\mathbf{c}, \mathbf{x}_t, t)\approx\mathbb{E}_{p(\mathbf{x}_0|\mathbf{x}_t)}[\mathcal{D}_{\boldsymbol{\theta}}(\mathbf{c}, \mathbf{x}_0)]. 
    \label{eq:estimate_noisy_s_using_clean_x0t}
\end{equation}
Eq.~(\ref{eq:estimate_noisy_s_using_clean_x0t}) is reasonable because if the distance between the noise image $\mathbf{x}_t$ and the condition $\mathbf{c}$ is small, the clean image $\mathbf{x}_{0}$ corresponding to the noise image $\mathbf{x}_t$ should also have a small distance with the condition $\mathbf{c}$, especially during the late stage of the sampling process when the noise level of $\mathbf{x}_t$ is relatively small.
However, during the sampling process, it is infeasible to get the clean image $\mathbf{x}_0$ corresponding to an intermediate noisy result $\mathbf{x}_t$, so we need to approximate  $\mathbf{x}_0$.
Considering the expectation of  $p(\mathbf{x}_0|\mathbf{x}_t)$ \cite{chung2023diffusion}: 
\begin{equation}
    \mathbf{x}_{0|t}:=\mathbb{E}[\mathbf{x}_0|\mathbf{x}_t]=\frac{1}{\sqrt{\bar{\alpha}_t}}(\mathbf{x}_t+(1-\bar{\alpha}_t)s(\mathbf{x}_t, t)),
    \label{eq:x0t}
\end{equation}
where $\bar{\alpha}_t = \prod_{i=1}^{t} (1 - \beta_i)$ and $s(\cdot, t)$ is the pre-trained score estimator. According to Eq.~(\ref{eq:x0t}), from $\mathbf{x}_t$, we can estimate the clean image 
% $\mathbf{x}_{0}$  and 
denoted as $\mathbf{x}_{0|t}$.
% Using the estimated $\mathbf{x}_{0|t}$ which is a function of $\mathbf{x}_t$ 
Then with Eq.~(\ref{eq:e_st}) and Eq.~(\ref{eq:estimate_noisy_s_using_clean_x0t}), we can approximate the time-dependent energy function of noisy data $\mathbf{x}_t$: 
\begin{equation}
    \mathcal{E}(\mathbf{c}, \mathbf{x}_t)\approx \mathcal{D}_{\boldsymbol{\theta}}(\mathbf{c}, \mathbf{x}_{0|t}).
    \label{eq:energy_approx_tid_smf}
\end{equation}
According to Eq.~(\ref{eq:energy_guided_sampling}) and Eq.~(\ref{eq:energy_approx_tid_smf}),
the approximated sampling process can be written as:
% \begin{align}
% &\mathbf{x}_{t-1}=(1+\frac{1}{2}\beta_t)\mathbf{x}_t \notag \\
% &+\beta_t\nabla_{\mathbf{x}_{t}}\log p(\mathbf{x}_{t})-\rho_t\nabla_{\mathbf{x}_t}\mathcal{S}_{\theta}(\mathbf{c}, \mathbf{x}_{0|t}(\mathbf{x}_t))+\sqrt{\beta_t}\epsilon,
%     \label{eq:approximated_energy_guided_sampling}
% \end{align}
\begin{equation}
    \mathbf{x}_{t-1} = \mathbf{m}_t - \rho_t\nabla_{\mathbf{x}_t}\mathcal{D}_{\boldsymbol{\theta}}(\mathbf{c}, \mathbf{x}_{0|t}(\mathbf{x}_t)),
    \label{eq:approximated_energy_guided_sampling}
\end{equation}
and the detailed algorithm is shown in \textbf{Algo.}~\ref{alg:aio}.

\begin{algorithm}[t]
\vspace{-0.1cm}
\scriptsize
\caption{Sampling Process of our proposed FreeDoM}
\label{alg:aio}
\begin{algorithmic}[1] %[1] enables line numbers
    \Require condition $\mathbf{c}$, unconditional score estimator $s(\cdot, t)$, time-independent distance measuring function $\mathcal{D}_{\boldsymbol{\theta}}(\mathbf{c}, \cdot)$, pre-defined parameters $\beta_t$, $\bar{\alpha}_t$ and learning rate $\rho_t$.
    \State $\mathbf{x}_{T}\sim\mathcal{N}(\mathbf{0},\mathbf{I})$
    \For{$t = T, ..., 1$}
        \State $\boldsymbol{\epsilon}\sim\mathcal{N}(\mathbf{0},\mathbf{I})$ if $t>1$, else $\boldsymbol{\epsilon}=\mathbf{0}$.
        \State $\mathbf{x}_{t-1} = (1+\frac{1}{2}\beta_t)\mathbf{x}_t+\beta_t s(\mathbf{x}_t, t)+\sqrt{\beta_t}\boldsymbol{\epsilon}$
        \State $\mathbf{x}_{0|t}=\frac{1}{\sqrt{\bar{\alpha}_t}}(\mathbf{x}_t+(1-\bar{\alpha}_t)s(\mathbf{x}_t, t))$
        \State $\boldsymbol{g}_t = \nabla_{\mathbf{x}_t}\mathcal{D}_{\boldsymbol{\theta}}(\mathbf{c}, \mathbf{x}_{0|t}(\mathbf{x}_t)))$
        \State $\mathbf{x}_{t-1} = \mathbf{x}_{t-1} - \rho_t \boldsymbol{g}_t$
    \EndFor
    \State \textbf{return} $\mathbf{x}_{0}$
\end{algorithmic}
\end{algorithm}

\subsection{Efficient Time-Travel Strategy}
\label{subsec:ttt}

% \begin{figure}[!tbp]
%   \centering
%   \vspace{-0.1cm}
%   \includegraphics[width=1\linewidth]{figure/time_trick_1.pdf}
%   \vspace{-0.4cm}
%   \caption{\textbf{Explanation of badly generated results on a large data domain.} The \textcolor[RGB]{236,89,60}{red} arrow denotes the randomly generated direction, the \textcolor[RGB]{162,215,105}{green} arrow indicates the energy-guided direction, and the \textcolor[RGB]{112,194,232}{blue} arrow represents the corrected direction. In a large data domain, the random direction provided by the unconditional diffusion model (\textcolor[RGB]{236,89,60}{red} arrow) has more freedom and deviates more from the target direction indicated by the energy function (\textcolor[RGB]{162,215,105}{green} arrow), that is, \textcolor[RGB]{255,187,73}{$\alpha_2 > \alpha_1$}. Therefore, considering the red and green arrows of the same length in the different data domains, the corrected direction (\textcolor[RGB]{112,194,232}{blue} arrow) in the larger data domain will be shorter and deviate more from the target direction (\textcolor[RGB]{112,48,160}{$\beta_2 > \beta_1$}), which results in poorer guidance. 
%   % \chen{The `green' colors seem to be different greens in caption and in the figure.} 
%   % \chen{Why is the freedom of the energy function the same for different domains?}
%   }
%   \vspace{-0.4cm}
% \label{fig:ttt1} 
% \end{figure}

 \begin{figure}[!tbp]
  \centering
  \vspace{-0.1cm}
  \includegraphics[width=1\linewidth]{figure/tt_effective.pdf}
  \vspace{-0.7cm}
  \caption{\textbf{Comparison of results generated before and after using the time-travel strategy.} The prompt is ``orange''. We can see that the results in (a) do not match the given conditions. After using the time-travel strategy, we get better results in (b).}
  \vspace{-0.1cm}
\label{fig:tt_effective} 
\end{figure}

\begin{figure*}[ht]
  \centering
%   \vspace{-0.1cm}
  \includegraphics[width=1\linewidth]{figure/time_trick_2.pdf}
  \vspace{-0.6cm}
  \caption{\textbf{Demonstration of the importance of different sampling stages.} Most of the semantic content is generated during the semantic stage, so we only employ the time-travel strategy in this stage to achieve an efficient version of FreeDoM. The shown images are $\mathbf{x}_{0|t}$ generated by diffusion models pre-trained on the ImageNet data domain.}
  \vspace{-0.1cm}
\label{fig:ttt2} 
\end{figure*}

In the process of applying \textbf{Algo.}~\ref{alg:aio}, we find that the performance varies significantly on different data domains.  For small data domains such as human faces, \textbf{Algo.}~\ref{alg:aio} can effectively produce results that satisfy the given conditions within 100 DDIM~\cite{song2021denoising} sampling steps. However, for large data domains such as ImageNet, we often get results that are not closely related to the given conditions or even randomly generated results (shown in Fig.~\ref{fig:tt_effective}(a)). 
% \textcolor{red}{[Jiwen: need a figure to demonstrate the phenomenon before and after using time travel.]} 
 We attribute the failure of  \textbf{Algo.}~\ref{alg:aio} on large data domains to poor guidance. 
 The reason for poor guidance is that the direction of unconditional score generated by diffusion models in large data domains has more freedom, making it easier to deviate from the direction of conditional control.
 % and provide an intuitive explanation of this phenomenon in Fig.~\ref{fig:ttt1}. 
To solve this problem, we adopt the time-travel strategy~\cite{Lugmayr_2022_CVPR, wang2023zeroshot}, which has been empirically shown to inhibit the generation of disharmonious results when solving hard generation tasks.

The time-travel strategy is a technique that takes the current intermediate result $\mathbf{x}_t$ back by $j$ steps to $\mathbf{x}_{t+j}$ and resamples it to the $t$-th timestep again. This strategy inserts more sampling steps into the sampling process and refines the generated results. In our experiments specifically, we go back by $j=1$ step each time and resample. We repeat this resampling process $r_t$ times at the $t$-th timestep. Our experiments demonstrate that the time travel strategy is effective in solving the poor guidance problem (shown in Fig.~\ref{fig:tt_effective}(b)). 
% \chen{Better point out which part in the experiment section.}  
However, the time cost is also expensive because the number of sampling steps is larger, especially considering that each timestep will include the cost of calculating the gradient of the energy function.

% Fortunately, we found the {\color{purple} effectiveness} \chen{What do you mean by `effectivenss'? You mentioned the trick is effective, but then why is it small? I guess you want to say this trick doesn't affect the time cost much?} of the time-travel trick is relatively small during most timesteps. 
Fortunately, we find that the time-travel strategy does not have the same effect in each time step. In fact, using this technique in most time steps will not significantly modify the final result, which means we can use this strategy only in a small portion of  the timesteps, thus significantly reducing the number of additional iteration steps.
In Fig.~\ref{fig:ttt2}, we try to analyze this phenomenon by dividing the sampling process into three stages. In the early stage, i.e., the chaotic stage, the generated result $\mathbf{x}_{0|t}$ is extremely blurred, and the energy guidance is hard to make anything reasonable, so we do not need to employ the time-travel strategy. During the late stage, i.e., the refinement stage, the change in the generated results is minor, so the time-travel strategy is useless.
% as it is too late to take complete control. 
During the intermediate stage, i.e., the semantic stage, the change in the generated result is significant, so this stage is critical for conditional generation. Based on this observation, we only apply the time-travel strategy in the semantic stage to implement efficient sampling while solving the problem of poor guidance. 
The range of the semantic stage is an experimental choice depending on the specific diffusion models we choose.
The detailed algorithm of our proposed FreeDoM with the efficient time-travel strategy is shown in \textbf{Algo.}~\ref{alg:aio_ttt}, where $r_t=1$ means we do not apply the time-travel strategy in the $t$-th timestep.

\begin{algorithm}[t]
\vspace{-0.1cm}
\scriptsize
\caption{FreeDoM + Efficient Time-Travel Strategy}
\label{alg:aio_ttt}
\begin{algorithmic}[1] %[1] enables line numbers
    \Require condition $\mathbf{c}$, unconditional score estimator $s(\cdot, t)$, time-independent distance measuring function $\mathcal{D}_{\boldsymbol{\theta}}(\mathbf{c}, \cdot)$, pre-defined parameters $\beta_t$, $\bar{\alpha}_t$, learning rate $\rho_t$, and the repeat times of time travel of each step $\{r_1, \cdots, r_T\}$.
    \State $\mathbf{x}_{T}\sim\mathcal{N}(\mathbf{0},\mathbf{I})$
    \For{$t = T, ..., 1$}
        \For{$i = r_t, ..., 1$}
            \State $\boldsymbol{\epsilon}_1\sim\mathcal{N}(\mathbf{0},\mathbf{I})$ if $t>1$, else $\boldsymbol{\epsilon}_1=\mathbf{0}$.
            \State $\mathbf{x}_{t-1} = (1+\frac{1}{2}\beta_t)\mathbf{x}_t+\beta_t s(\mathbf{x}_t, t)+\sqrt{\beta_t}\boldsymbol{\epsilon}_1$
            \State $\mathbf{x}_{0|t}=\frac{1}{\sqrt{\bar{\alpha}_t}}(\mathbf{x}_t+(1-\bar{\alpha}_t)s(\mathbf{x}_t, t))$
            \State $\boldsymbol{g}_t = \nabla_{\mathbf{x}_t}\mathcal{D}_{\boldsymbol{\theta}}(\mathbf{c}, \mathbf{x}_{0|t}(\mathbf{x}_t)))$
            \State $\mathbf{x}_{t-1} = \mathbf{x}_{t-1} - \rho_t \boldsymbol{g}_t$
            \If{$i > 1$}
                \State $\boldsymbol{\epsilon}_2\sim\mathcal{N}(\mathbf{0},\mathbf{I})$
                \State $\mathbf{x}_{t} = \sqrt{1 - \beta_t}\mathbf{x}_{t-1} + \sqrt{\beta_t}\boldsymbol{\epsilon}_2$
            \EndIf
        \EndFor
    \EndFor
    \State \textbf{return} $\mathbf{x}_{0}$
\end{algorithmic}
\end{algorithm}


\subsection{Construction of the Energy Function}
\label{subsec:construct_energy_functions}
\vspace{3pt}
\noindent \ding{113}~\textbf{Single Condition Guidance.}
% \paragraph{Single Condition Guidance.}
To incorporate in specific applications, we use the distance measuring function conforming to the following structure to construct the energy function:
\begin{equation}
    \mathcal{E}(\mathbf{c}, \mathbf{x}_t)\approx\mathcal{D}_{\boldsymbol{\theta}}(\mathbf{c}, \mathbf{x}_{0|t})=Dist(\mathcal{P}_{\boldsymbol{\theta}_1}(\mathbf{c}), \mathcal{P}_{\boldsymbol{\theta}_2}(\mathbf{x}_{0|t})),
    \label{eq:single_condition}
\end{equation}
where $Dist(\cdot)$ denotes the distance measuring methods like Euclidean distance, and $\boldsymbol{\theta} = \{\boldsymbol{\theta}_1, \boldsymbol{\theta}_2\}$. $\mathcal{P}_{\boldsymbol{\theta}_1}(\cdot)$ and $\mathcal{P}_{\boldsymbol{\theta}_2}(\cdot)$ project the condition and image to the same space for distance measurement. These projection networks can be off-the-shelf pre-trained classification networks, segmentation networks, etc. In most cases, we  only need one network to project the clean image $\mathbf{x}_{0|t}$ to the condition space. In the cases with reference images $\mathbf{x}_{ref}$, we also  only need one feature encoder to project the reference image $\mathbf{x}_{ref}$ and $\mathbf{x}_{0|t}$ to the same feature space. 

\vspace{3pt}
\noindent \ding{113}~\textbf{Multi Condition Guidance.}
% \paragraph{Multi Condition Guidance.}
In some more involved applications, multiple conditions can be available to provide control over the generated results. Take the image style transfer task as an example. Here, we have two conditions: the structure information from the source image and the style information from the style image. In these multi-condition cases, assume that the given conditions are denoted as $\{\mathbf{c}_1, \cdots, \mathbf{c}_n\}$, we can approximately construct the energy function as :
\begin{align}
    \mathcal{E}&(\{\mathbf{c}_1, \cdots, \mathbf{c}_n\}, \mathbf{x}_t) \notag \\
    &\approx\eta_1\mathcal{D}_{\boldsymbol{\theta}_1}(\mathbf{c}_1, \mathbf{x}_{0|t})+\cdots+\eta_n\mathcal{D}_{\boldsymbol{\theta}_n}(\mathbf{c}_n, \mathbf{x}_{0|t}),
    % \\&=\eta_1\mathcal{D}(\mathcal{C}_1, \mathcal{A}_{\theta_1}(\mathbf{x}_{0|t})) + \cdots + \eta_n\mathcal{D}(\mathcal{C}_1, \mathcal{A}_{\theta_1}(\mathbf{x}_{0|t})),
    \label{eq:multi_condition}
\end{align}
where $\eta_i$ is the weighting factor. We use different distance measuring functions $\{\mathcal{D}_{\boldsymbol{\theta}_1}(\cdot, \cdot), \cdots, \mathcal{D}_{\boldsymbol{\theta}_n}(\cdot, \cdot)\}$ for specific conditions and sum the whole for gradient computation.

\vspace{3pt}
\noindent \ding{113}~\textbf{Guidance for Latent Diffusion.}
% \paragraph{Guidance for Latent Diffusions.}
Our method applies not only to image diffusions but also to latent diffusions, such as Stable Diffusion~\cite{rombach2022high}. 
In this case, the intermediate results $\mathbf{x}_t$ are latent codes rather than images. We can use the latent decoder to project the generated latent codes to images and then use the same algorithm in the image domain.
% Assume that we have a score estimator $s(\mathbf{z}_t, t)$ for noisy latent code $\mathbf{z}_t$ and the given condition is $\mathbf{c}$, we can construct $\mathcal{E}(\mathbf{c}, \mathbf{z}_t)$ as:
% \begin{align}
%     \mathcal{E}(\mathbf{c}, \mathbf{z}_t) &\approx \mathcal{S}_{\theta}(\mathbf{c}, \mathbf{z}_{0|t})\notag\\
%     &=Dist(\mathcal{P}_{\theta_1}(\mathbf{c}), \mathcal{P}_{\theta_2}(Decoder(\mathbf{z}_{0|t}))),
% \end{align}
% where $Decoder(\cdot)$ projects the clean latent code $\mathbf{z}_{0|t}$ to the clean image domain and other notations are the same as Eq.~\ref{eq:single_condition}.
% Similar to the algorithm of image diffusion, we use $\nabla_{\mathbf{z}_t}\mathcal{S}_{\theta}(\mathbf{c}, \mathbf{z}_{0|t}(\mathbf{z}_t))$ to update the intermediate generated latent code for training-free guidance.

\subsection{Specific Examples of Supported Conditions}
\label{subsec:specific_examples}
\vspace{3pt}
\noindent \ding{113}~\textbf{Text.}
% \paragraph{Texts.} 
For given text prompts, we construct the distance measuring function based on CLIP~\cite{radford2021learning}. Specifically, we take the CLIP image encoder (as $\mathcal{P}_{\boldsymbol{\theta}_2}(\cdot)$) and the CLIP text encoder (as $\mathcal{P}_{\boldsymbol{\theta}_1}(\cdot)$) to project the image $\mathbf{x}_{0|t}$ and given text in the same CLIP feature space. 
% The similarity distance measuring method we choose is $\ell_2$ Euclidean distance.
Compared with the commonly used cosine distance measurement and for simplicity, we choose the $\ell_2$ Euclidean distance measurement, since the sampling quality in our experiments is not significantly different.

\vspace{3pt}
\noindent\ding{113}~\textbf{Segmentation Maps.}
% \paragraph{Segmentation Maps.} 
For segmentation maps, we choose a face parsing network based on the real-time semantic segmentation network BiSeNet~\cite{yu2018bisenet} to generate the parsing map of an input human face and directly compute the $\ell_2$ Euclidean distance between the given parsing map and the parsing results of $\mathbf{x}_{0|t}$. An interesting usage of the face parsing network is to constrain the gradient update region so that we can edit the target semantic region without changing other regions (shown in Fig.~\ref{fig:trick_segmentation}).

\begin{figure}[!tbp]
  \centering
  \vspace{-0.1cm}
  \includegraphics[width=1\linewidth]{figure/edit_face_area.pdf}
  \vspace{-0.7cm}
  \caption{\textbf{Practical usage of face parsing maps.} We can limit the gradient of the energy function to update the image only in the target semantic region indicated by the mask so that other regions remain unchanged while editing.}
  \vspace{-0.4cm}
\label{fig:trick_segmentation} 
\end{figure}

\vspace{3pt}
\noindent\ding{113}~\textbf{Sketches.}
% \paragraph{Sketches.} 
We choose an open-source pre-trained network~\cite{xiang2022adversarial} that transfers a given anime image to the style of hand-drawn sketches. Experiments prove that the network is still effective for real-world images. We use the $\ell_2$ Euclidean distance to compare the given sketches with transferred sketch-style results of $\mathbf{x}_{0|t}$. 

\vspace{3pt}
\noindent\ding{113}~\textbf{Landmarks.}
% \paragraph{Landmarks.} 
We use an open-source pre-trained human face landmark detection network~\cite{PFL} for this application. The detection network has two stages: the first stage finds the position of the center of a face and the second stage marks the landmarks of this detected face. We compute the $\ell_2$ Euclidean distance between predicted face landmarks of $\mathbf{x}_{0|t}$ and the given landmarks condition, and only use the gradient in the face area detected in the first stage to update the intermediate results. 

\vspace{3pt}
\noindent\ding{113}~\textbf{Face IDs.}
% \paragraph{Face IDs.} 
We use ArcFace~\cite{deng2019arcface}, an open-source pre-trained human face recognition network, to extract the target features of reference faces to represent face IDs and compute the $\ell_2$ Euclidean distance between the extracted ID features of $\mathbf{x}_{0|t}$ and those of the reference image.

\vspace{3pt}
\noindent\ding{113}~\textbf{Style Images.}
% \paragraph{Style Images.} 
The style image is denoted as $\mathbf{x}_{style}$. We use the following equation to compute the distance of the style information between $\mathbf{x}_{style}$ and $\mathbf{x}_{0|t}$:
\begin{equation}
    Dist(\mathbf{x}_{style}, \mathbf{x}_{0|t}) = ||G(\mathbf{x}_{style})_j - G(\mathbf{x}_{0|t})_j||_F^2,
\end{equation}
where $G(\cdot)_j$ denotes the Gram matrix~\cite{johnson2016perceptual} of the $j$-th layer feature map of an image encoder. In our experiments, we choose the features from the third layer of the CLIP image encoder to generate  satisfactory results.

\vspace{3pt}
\noindent\ding{113}~\textbf{Low-pass Filters.}
% \paragraph{Low-pass Filters.} 
For the image transferring task, we need an energy function to constrain the generated results conforming to the structure information of the source image $\mathbf{x}_{source}$. Similar to EGSDE~\cite{zhao2022egsde} and ILVR~\cite{Choi_2021_ICCV}, we choose a low-pass filter $\mathcal{K}(\cdot)$ in this setup. The distance between the source image $\mathbf{x}_{source}$ and $\mathbf{x}_{0|t}$ is computed as:
\begin{equation}
    Dist(\mathbf{x}_{source}, \mathbf{x}_{0|t}) = ||\mathcal{K}(\mathbf{x}_{source}) - \mathcal{K}(\mathbf{x}_{0|t})||^2_2.
\end{equation}

% \paragraph{Multi Conditions.} The examples demonstrated for multi conditions include but not least: face driven (face IDs $+$ landmarks), face editing (texts $+$ segmentation maps) and style transfer (style images $+$ low-pass filters).
%% general comment about other applications


\section{Experiments}

\begin{figure*}[t]
  \centering
%   \vspace{-0.1cm}
  \includegraphics[width=1\linewidth]{figure/single_face_v3.pdf}
  \vspace{-0.6cm}
  \caption{\textbf{Qualitative results of using a single condition for human face images.} The included conditions are: (a) text; (b) face parsing maps; (c) sketches; (d) face landmarks; (e) IDs of reference images. \textbf{Zoom in for best view.}}
  \vspace{-0.5cm}
\label{fig:single_energy_face} 
\end{figure*}

\begin{figure}[t]
  \centering
  \vspace{-0.1cm}
  \includegraphics[width=1\linewidth]{figure/single_others_v2.pdf}
  \vspace{-0.6cm}
  \caption{\textbf{Qualitative results of using a single condition for ImageNet images.} Pre-trained diffusion models are: (a) unconditional ImageNet diffusion model; (b) classifier-based ImageNet diffusion model. \textbf{Zoom in for best view.}}
  \vspace{-0.3cm}
\label{fig:single_energy_imagenet} 
\end{figure}

\subsection{Implementation Details}
% For the pre-trained unconditional diffusion models, we choose the open-source parameters provided by \cite{dhariwal2021diffusion} (pretrained on ImageNet) and \cite{meng2022sdedit} (pretrained on CelebA-HQ~\cite{karras2017progressive}). 
% For the basic sampling algorithm, we choose DDIM~\cite{song2021denoising} with 100 sampling steps for all experiments.
% The resolution of sampled images is $256 \times 256$. Considering the generation ability of diffusion models in image space (not in latent space like \cite{rombach2022high})
Our proposed method applies to many open-source pre-trained diffusion models (DMs). In our experiment, we have tried the following models and conditions:

\noindent\ding{226}~\textbf{Unconditional Human Face Diffusion Model~\cite{meng2022sdedit}.}
The supported image resolution of this model is $256 \times 256$, and the pre-trained dataset is aligned human faces from CelebA-HQ~\cite{karras2017progressive}. We experiment with conditions that include text, parsing maps, sketches, landmarks, and face IDs. 

\noindent\ding{226}~\textbf{Unconditional ImageNet Diffusion Model~\cite{dhariwal2021diffusion}.}
The supported image resolution of this model is $256 \times 256$ and the pre-trained dataset is ImageNet. We experiment with conditions that include text and style images.  

\noindent\ding{226}~\textbf{Classifier-based ImageNet Diffusion Model~\cite{dhariwal2021diffusion}.}
The supported image resolution of this model is $256 \times 256$, and the pre-trained dataset is ImageNet. This model also has a time-dependent classifier to guide its generation process. We experiment with the condition of style images.

\noindent\ding{226}~\textbf{Stable Diffusion~\cite{rombach2022high}.} Stable Diffusion is a widely used Latent Diffusion Model. The standard resolution of its output images is $512 \times 512$, but it supports  higher resolutions. In our work, we use its pre-trained text-to-image model. We experiment with the condition of style images.

\noindent\ding{226}~\textbf{ControlNet~\cite{zhang2023adding}.} ControlNet is a Stable Diffusion based model supporting extra conditions input with the original text input. In our work, we use the pre-trained pose-to-image and scribble-to-image models. We experiment with conditions that include face IDs and style images.

We choose DDIM~\cite{song2021denoising} with 100 steps as the sampling strategy of all experiments, and other more detailed configurations will be provided in the supplementary material.

\subsection{Qualitative Results}

\noindent\ding{113}~\textbf{Single Condition.}
We present the single-condition-guided results of human face images in Fig.~\ref{fig:single_energy_face}. We can see that the generated results meet the requirements of the given conditions and have rich diversity and good quality. In Fig.~\ref{fig:single_energy_imagenet}, we show the single-condition-guided results of the ImageNet domain. The diversity of the generated results is still high. In order to ensure that the generated results can better meet the control of the given conditions, we use the proposed efficient time-travel strategy.

\begin{figure}[t]
  \centering
%   \vspace{-0.1cm}
  \includegraphics[width=1\linewidth]{./figure/multi_condition_v3.pdf}
  \vspace{-0.6cm}
  \caption{\textbf{Qualitative results of using multiple conditions.} Pre-trained models are: (a) and (b): unconditional human face diffusion model; (c) and (d): unconditional ImageNet diffusion model. \textbf{Zoom in for best view.}}
  \vspace{-0.5cm}
\label{fig:multi_energy} 
\end{figure}

\noindent\ding{113}~\textbf{Multiple Conditions.}
Fig.~\ref{fig:multi_energy} shows the synthesized results guided by multiple conditions in the domain of human faces and ImageNet. In the human face domain (a small data domain), we produce good results with rich diversity and high consistency with the conditions. We use the efficient time-travel strategy in the ImageNet domain (a large data domain) to produce acceptable results.

\noindent\ding{113}~\textbf{Training-free Guidance for Latent Domain.} It should be pointed out that FreeDoM supports diffusion models in both image and latent domains. In our work, we experiment with two latent diffusion models: Stable Diffusion~\cite{rombach2022high} and ControlNet~\cite{zhang2023adding}. 
We try to add the training-free conditional interfaces based on their energy functions to work with the existing training-required conditional interfaces, leading to satisfactory results  shown in Fig.~\ref{fig:overview}(d)-(f). As such, we can see great application potential for mixing training-free and training-required conditional interfaces in various practical applications.

\subsection{Further Studies}

\begin{figure*}[t]
  \centering
  % \vspace{-0.1cm}
  \includegraphics[width=1\linewidth]{./figure/comp_tedigan.pdf}
  \vspace{-0.6cm}
  \caption{\textbf{Comparison between FreeDoM and TediGAN~\cite{xia2021tedigan} in three conditional image synthesis tasks}: (a) segmentation maps to human faces; (b) sketches to human faces; (c) text prompts to human faces. \textbf{Zoom in for best view.}}
  \vspace{-0.5cm}
\label{fig:comp_tedigan} 
\end{figure*}

\begin{table}[t]
    \centering
    \scriptsize
    % \vspace{-0.5cm}
    \tabcolsep=0.09cm
    % \renewcommand\arraystretch{2}
    \begin{tabular}{c | cc | cc | cc}
        \toprule[1.2pt]
          \multirow{2}{*}{Methods} & \multicolumn{2}{c}{Segmentation maps} \vline & \multicolumn{2}{c}{Sketches}\vline & \multicolumn{2}{c}{Texts}\\
        \rule{0pt}{8pt}
            & Distance↓ & FID↓ & Distance↓ & FID↓ & Distance↓ & FID↓\\
        \toprule[1.2pt]
        % \hline
        % \rule{0pt}{10pt}
TediGAN~\cite{xia2021tedigan} & 2037.2 & \textbf{52.77} & 48.61 & 91.11 & 12.31 & 71.71\\
\rule{0pt}{10pt}
        FreeDoM (ours) & \textbf{1696.1} & 53.08 & \textbf{33.29} & \textbf{70.97} & \textbf{10.83} & \textbf{55.91} \\
        \bottomrule[1.2pt]
    \end{tabular}
    \caption{We compare FreeDoM with the training-required method TediGAN~\cite{xia2021tedigan} in three image conditional synthesis tasks. We compute the distance with given conditions and FID to judge the performance. The comparison shows that FreeDoM generates images matching given conditions better and having a comparable or better image quality.
    }
    \vspace{-0.5cm}
    \label{tb:comp_tedigan}
\end{table}

\noindent\ding{113}~\textbf{Comparison between FreeDoM and TediGAN~\cite{xia2021tedigan}.}
We compare FreeDoM with the training-required conditional human face generation method TediGAN under three conditions: segmentation maps, sketches, and text. 
A qualitative comparison is shown in Fig.~\ref{fig:comp_tedigan}, and quantitative comparison results are reported in Tab.~\ref{tb:comp_tedigan}.
For the comparison, we choose 1000 segmentation maps, 1000 sketches, and 1000 text prompts to generate 1000 results, respectively. Then we compute FID and the average distance with given conditions using the methods introduced in Sec.~\ref{subsec:specific_examples} to judge the performance. The comparison shows that the images generated by FreeDoM match the given conditions better and have a comparable or better image quality.

\begin{figure}[t]
  \centering
  % \vspace{-0.1cm}
  \includegraphics[width=1\linewidth]{figure/comp_ugd.pdf}
  \vspace{-0.6cm}
  \caption{\textbf{Comparison between FreeDoM and UGD~\cite{bansal2023universal} in style-guided generation.} 
  The UGD results are taken from the original paper. 
  % The results of FreeDoM are generated with random seeds, not selected. 
  The number in the lower right corner of each image represents its distance with the provided style image (smaller is better), which is calculated using the method described in Sec.~\ref{subsec:specific_examples}.
  FreeDoM offers obvious advantages in image quality and in the degree of statisfaction of the conditions.
  \textbf{Zoom in for best view.}
  % \chen{Can we have a gap between the two lines of images? Numbers on images are not clear.}
  }
  \vspace{-0.3cm}
\label{fig:comp_ugd} 
\end{figure}

\noindent\ding{113}~\textbf{Comparison between FreeDoM and UGD~\cite{bansal2023universal}.}
We compare FreeDoM with Universal Guidance Diffusion (UGD)~\cite{bansal2023universal} in style-guided generations. From Fig.~\ref{fig:comp_ugd}, we find that FreeDoM has significant advantages over UGD in the degree of alignment with the conditioned style image. Regarding the inference speed, UGD runs in about 40 minutes (using the open-source code) on a GeForce RTX 3090 GPU to synthesize one image with a resolution of $512 \times 512$, while we only need about 84 seconds (nearly 30$\times$ faster). 

\begin{figure}[t]
  \centering
%   \vspace{-0.1cm}
  \includegraphics[width=1\linewidth]{figure/scalable.pdf}
  \vspace{-0.6cm}
  \caption{\textbf{Demonstration of the effect of different learning rates from small scale to large scale.} (a): unconditional ImageNet diffusion models with prompt ``orange''; (b): unconditional human face diffusion models with a face ID from the reference image. \textbf{Zoom in for best view.}}
  \vspace{-0.5cm}
\label{fig:scalable} 
\end{figure}

\noindent\ding{113}~\textbf{Effect of different learning rates.}
We studied the effect of different learning rates on the results.
Fig.~\ref{fig:scalable} shows the results while increasing the energy function's learning rate ($\rho_t$ in Eq.~(\ref{eq:approximated_energy_guided_sampling})) from $0$. We can see that FreeDoM is scalable in terms of its control ability, which means that users can adjust the intensity of control as needed. 

% \section{Limitations}
% The limitations of FreeDoM: (1) The time cost of the sampling is still higher than the training-required methods because each iteration adds a derivative process of the energy function, and the time-travel strategy introduces more sampling steps. (2) 
% It is difficult to use the energy function to control the fine-grained structure features in the large data domain.
% For example, using the canny edge maps as the conditions may result in poor guidance, even if we use the time-travel strategy. In this case, the training-required methods will be a better choice. (3) Eq.~\ref{eq:multi_condition}, which deals with multi-condition control, assumes that the multiple conditions provided are independent, which is not always true in practice.
% When conditions conflict with each other, FreeDoM will produce bad results.

\section{Conclusions \& Limitations}
We propose a training-free energy-guided conditional diffusion model, FreeDoM, to address a wide range of conditional generation tasks without training. Our method uses off-the-shelf pre-trained time-independent networks to approximate the time-dependent energy functions. Then, we use the gradient of the approximated energy to guide the generation process. Our method supports different diffusion models, including image  and latent diffusion models. It is worth emphasizing that the applications presented in this paper are only a subset of the applications FreeDoM supports and should not be limited to these. In future work, we aim to explore even more energy functions for a broader range of tasks.

Despite its merits, our FreeDoM method has some limitations: (1) The time cost of the sampling is still higher than the training-required methods because each iteration adds a derivative operation for the energy function, and the time-travel strategy introduces more sampling steps. (2)  It is difficult to use the energy function to control the fine-grained structure features in the large data domain.
For example, using the Canny edge maps as the conditions may result in poor guidance, even if we use the time-travel strategy. In this case, the training-required methods will provide a better alternative. (3) Eq.~\ref{eq:multi_condition} deals with multi-condition control and assumes that the provided conditions  are independent, which is not necessarily true in practice. When conditions conflict with each other, FreeDoM may produce subpar generation results.


{\small
\bibliographystyle{ieee_fullname}
\bibliography{egbib}
}

\appendix
\onecolumn{
\begin{appendices}

This appendix is organized as follows:
\begin{itemize}
    \item Section~\ref{sec:more_results}: More results to show the performance of FreeDoM.
    \item Section~\ref{sec:zir}: The relationship between FreeDoM and zero-shot image restoration methods.
\end{itemize}

% \section{Overall Experimental Configurations}
\section{More Results}
\label{sec:more_results}
In this section, we provide more generated results to demonstrate the effects of FreeDoM under various conditions and the applications FreeDoM support with training-required latent diffusion models. 

We show the results of various conditions in Fig.~\ref{fig:text2face} (text-to-image), Fig.~\ref{fig:seg2face} (segmentation-to-image), Fig.~\ref{fig:sketch2face} (sketch-to-image), Fig.~\ref{fig:land2face} (landmark-to-image), and Fig.~\ref{fig:id2face} (id-to-image).

We show the results with latent diffusion models in Fig.~\ref{fig:sd_style} (style guidance $+$ Stable Diffusion~\cite{rombach2022high}), Fig.~\ref{fig:cn_style} (style guidance $+$ Scribble ControlNet~\cite{zhang2023adding}) and Fig.~\ref{fig:cn_id} (face ID guidance $+$ Human-pose ControlNet~\cite{zhang2023adding}).
In order to further illustrate the implementation process of the application with the Human-pose ControlNet demonstrated in Fig.~\ref{fig:cn_id}, we provide Fig.~\ref{fig:cn_id_process}.

\begin{figure*}[ht]
  \centering
%   \vspace{-0.1cm}
\includegraphics[width=1\linewidth]{./figure/more_res_text2face.pdf}
  % \vspace{-0.6cm}
  \caption{Generated human faces for the text-to-image task. We choose four short and four long prompts to demonstrate the performance of FreeDoM. The characteristics described by these short prompts are experientially seldom seen in the training set. These results are consistent with the given conditions and have good diversity.}
  % \vspace{-0.6cm}
\label{fig:text2face} 
\end{figure*}
\clearpage

\begin{figure*}[ht]
  \centering
%   \vspace{-0.1cm}
\includegraphics[width=1\linewidth]{./figure/more_res_seg2face.pdf}
  % \vspace{-0.6cm}
  \caption{Generated human faces for the segmentation-to-image task. We choose four parsing maps to guide the generation process and output the parsing maps of the generated results to check the matching degree with given conditions. We can see that these results are consistent with the given conditions and have good diversity.}
  % \vspace{-0.6cm}
\label{fig:seg2face} 
\end{figure*}

\begin{figure*}[ht]
  \centering
%   \vspace{-0.1cm}
\includegraphics[width=1\linewidth]{./figure/more_res_sketch2face.pdf}
  % \vspace{-0.6cm}
  \caption{Generated human faces for the sketch-to-image task. We choose four sketches to guide the generation process and output the sketches of the generated results to check the matching degree with the given conditions. These results are consistent with the given conditions and have good diversity.}
  % \vspace{-0.6cm}
\label{fig:sketch2face} 
\end{figure*}
\clearpage

\begin{figure*}[ht]
  \centering
%   \vspace{-0.1cm}
\includegraphics[width=1\linewidth]{./figure/more_res_land2face.pdf}
  % \vspace{-0.6cm}
  \caption{Generated human faces for the landmark-to-image task. We selected landmarks of four faces from different angles to guide the generation process and output the landmarks of the generated results to check the matching degree with given conditions. These results are consistent with the given conditions and have good diversity.}
  % \vspace{-0.6cm}
\label{fig:land2face} 
\end{figure*}

\begin{figure*}[ht]
  \centering
%   \vspace{-0.1cm}
\includegraphics[width=1\linewidth]{./figure/more_res_id2face.pdf}
  % \vspace{-0.6cm}
  \caption{Generated human faces for ID-to-image task. We choose the face IDs of six celebrities as the reference to guide the generation process. These results are consistent with the given conditions and have good diversity.}
  % \vspace{-0.6cm}
\label{fig:id2face} 
\end{figure*}

\begin{figure*}[ht]
  \centering
%   \vspace{-0.1cm}
\includegraphics[width=1\linewidth]{./figure/more_res_SD_style.pdf}
  % \vspace{-0.6cm}
  \caption{Generation results of training-free style guidance with text-to-image Stable Diffusion~\cite{rombach2022high}. We choose five style images to guide the style of the results generated by Stable Diffusion. These generated results well match the provided style. \textbf{Zoom in for best view.}}
  % \vspace{-0.6cm}
\label{fig:sd_style} 
\end{figure*}

\begin{figure*}[ht]
  \centering
%   \vspace{-0.1cm}
\includegraphics[width=1\linewidth]{./figure/more_res_CN_style.pdf}
  % \vspace{-0.6cm}
  \caption{Generated results of training-free style guidance with Scribble ControlNet~\cite{zhang2023adding}. We choose four style images to guide the style of results generated by ControlNet. These generated results well match the provided style. \textbf{Zoom in for best view.}}
  % \vspace{-0.6cm}
\label{fig:cn_style} 
\end{figure*}

\begin{figure*}[ht]
  \centering
%   \vspace{-0.1cm}
\includegraphics[width=1\linewidth]{./figure/more_res_CN_id.pdf}
  % \vspace{-0.6cm}
  \caption{Generated Results of face ID guidance with Human-pose ControlNet~\cite{zhang2023adding}. By fixing random seeds, we can see the effects before and after introducing the ID guidance. These ID-guided results well match the given IDs in the face area. \textbf{Zoom in for best view.}}
  % \vspace{-0.6cm}
\label{fig:cn_id} 
\end{figure*}

\begin{figure*}[ht]
  \centering
%   \vspace{-0.1cm}
\includegraphics[width=1\linewidth]{./figure/more_res_CN_id_process.pdf}
  % \vspace{-0.6cm}
  \caption{Visualization of the whole training-free face ID guidance process using FreeDoM in Fig.~\ref{fig:cn_id}. We first decode the clean latent code $\mathbf{x}_{0|t}$ into the image domain. Then we detect the position of the human face and the corresponding landmarks. After getting the landmarks, we compute the affine parameters, which are used to perform an affine transformation to extract the aligned face area from the original decoded image. Finally, we compute the ID-based energy function between the aligned and reference faces. The gradient of the energy function to $\mathbf{x}_t$ will be used to update $\mathbf{x}_{t-1}$. Note that the computation of the Decoder and affine transformation is all differentiable, so the energy gradient to $\mathbf{x}_t$ is computable. \textbf{Zoom in for best view.}}
  % \vspace{-0.6cm}
\label{fig:cn_id_process} 
\end{figure*}
\clearpage

\section{Relationship between FreeDoM and Zero-Shot Image Restoration Methods}
\label{sec:zir}


The proposed FreeDoM is a framework that can support various conditions, including the degraded images in the image restoration tasks. Many existing zero-shot image restoration methods~\cite{Choi_2021_ICCV, chung2023diffusion, Chung_2022_CVPR, chung2022improving, kawar2022denoising, Lugmayr_2022_CVPR, song2023pseudoinverseguided, song2021solving, wang2023zeroshot,
wang2023unlimited} can be considered special cases of FreeDoM. Their idea can be summarized as updating the clean intermediate result $\mathbf{x}_{0|t}$ to meet the data consistency constraint, $\mathbf{y}=\mathcal{A}(\mathbf{x}_{0|t})$, where $\mathbf{y}$ is a degraded image and $\mathcal{A}(\cdot)$ is a linear or non-linear degradation operator. When dealing with linear degradation, the degradation operator $\mathcal{A}(\cdot)$ can be written into a matrix $\mathbf{A}$.

Since the image restoration tasks can also be seen as particular conditional generation tasks, these zero-shot image restoration methods can also be explained using the framework of FreeDoM. Take two typical examples: DPS~\cite{chung2023diffusion} uses $-\nabla_{\mathbf{x}_t}||\mathbf{y}-\mathcal{A}(\mathbf{x}_{0|t})||_2^2$ to update the intermediate results, which can be interpreted as a distance measurement function without learning parameters to improve the matching degree between the restored image $\mathbf{x}_{0|t}$ and the degraded image $\mathbf{y}$ in the measurement space; DDNM~\cite{wang2023zeroshot} obtains that the update direction for linear noiseless tasks is $-\mathbf{A}^{\dagger}(\mathbf{A}\mathbf{x}_{0|t}-\mathbf{y})$ through the derivation of Range-Null Space Decomposition, which can also be interpreted as an approximated analytical solution of the gradient of the distance measurement function in DPS on linear cases.

% We are grateful for the contribution and inspiration of these works, and our FreeDoM explores broader applications, not only image restoration tasks and energy functions based on degradation operators.

\end{appendices}
}

\end{document}
