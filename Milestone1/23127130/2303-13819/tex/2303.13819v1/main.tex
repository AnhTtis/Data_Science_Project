%%
%% This is file `sample-lualatex.tex',
%% generated with the docstrip utility.
%%
%% The original source files were:
%%
%% samples.dtx  (with options: `sigconf')
%% 
%% IMPORTANT NOTICE:
%% 
%% For the copyright see the source file.
%% 
%% Any modified versions of this file must be renamed
%% with new filenames distinct from sample-sigconf.tex.
%% 
%% For distribution of the original source see the terms
%% for copying and modification in the file samples.dtx.
%% 
%% This generated file may be distributed as long as the
%% original source files, as listed above, are part of the
%% same distribution. (The sources need not necessarily be
%% in the same archive or directory.)
%%
%% Commands for TeXCount
%TC:macro \cite [option:text,text]
%TC:macro \citep [option:text,text]
%TC:macro \citet [option:text,text]
%TC:envir table 0 1
%TC:envir table* 0 1
%TC:envir tabular [ignore] word
%TC:envir displaymath 0 word
%TC:envir math 0 word
%TC:envir comment 0 0
%%
%%
%% The first command in your LaTeX source must be the \documentclass command.
\documentclass[sigconf]{acmart}
\usepackage{graphicx} % remove [demo] in your file
\usepackage{subfigure} % for subfigures
\usepackage{caption}
\usepackage{soul}
\usepackage{color}
\setcopyright{none}
\settopmatter{printacmref=false} % Removes citation information below abstract
\renewcommand\footnotetextcopyrightpermission[1]{} % removes footnote with conference information in first column
\pagestyle{plain}
\def\pan#1{{\color{green}Pan: #1}}
\def\red#1{{\color{red}  #1}}
%% NOTE that a single column version may be required for 
%% submission and peer review. This can be done by changing
%% the \doucmentclass[...]{acmart} in this template to 
%% \documentclass[manuscript,screen]{acmart}
%% 
%% To ensure 100% compatibility, please check the white list of
%% approved LaTeX packages to be used with the Master Article Template at
%% https://www.acm.org/publications/taps/whitelist-of-latex-packages 
%% before creating your document. The white list page provides 
%% information on how to submit additional LaTeX packages for 
%% review and adoption.
%% Fonts used in the template cannot be substituted; margin 
%% adjustments are not allowed.
%%
%%
%% \BibTeX command to typeset BibTeX logo in the docs
\AtBeginDocument{%
  \providecommand\BibTeX{{%
    \normalfont B\kern-0.5em{\scshape i\kern-0.25em b}\kern-0.8em\TeX}}}

%% Rights management information.  This information is sent to you
%% when you complete the rights form.  These commands have SAMPLE
%% values in them; it is your responsibility as an author to replace
%% the commands and values with those provided to you when you
%% complete the rights form.
% \setcopyright{acmcopyright}
% \copyrightyear{2018}
% \acmYear{2018}
% \acmDOI{XXXXXXX.XXXXXXX}

% %% These commands are for a PROCEEDINGS abstract or paper.
% \acmConference[Conference acronym 'XX]{Make sure to enter the correct
%   conference title from your rights confirmation emai}{June 03--05,
%   2018}{Woodstock, NY}
% %
% %  Uncomment \acmBooktitle if th title of the proceedings is different
% %  from ``Proceedings of ...''!
% %
% %\acmBooktitle{Woodstock '18: ACM Symposium on Neural Gaze Detection,
% %  June 03--05, 2018, Woodstock, NY} 
% \acmPrice{15.00}
% \acmISBN{978-1-4503-XXXX-X/18/06}


%%
%% Submission ID.
%% Use this when submitting an article to a sponsored event. You'll
%% receive a unique submission ID from the organizers
%% of the event, and this ID should be used as the parameter to this command.
%%\acmSubmissionID{123-A56-BU3}

%%
%% For managing citations, it is recommended to use bibliography
%% files in BibTeX format.
%%
%% You can then either use BibTeX with the ACM-Reference-Format style,
%% or BibLaTeX with the acmnumeric or acmauthoryear sytles, that include
%% support for advanced citation of software artefact from the
%% biblatex-software package, also separately available on CTAN.
%%
%% Look at the sample-*-biblatex.tex files for templates showcasing
%% the biblatex styles.
%%

%%
%% The majority of ACM publications use numbered citations and
%% references.  The command \citestyle{authoryear} switches to the
%% "author year" style.
%%
%% If you are preparing content for an event
%% sponsored by ACM SIGGRAPH, you must use the "author year" style of
%% citations and references.
%% Uncommenting
%% the next command will enable that style.
%%\citestyle{acmauthoryear}

%%
%% end of the preamble, start of the body of the document source.
\begin{document}

%%
%% The "title" command has an optional parameter,
%% allowing the author to define a "short title" to be used in page headers.
\title{Verification of $\mathcal{L}_1$ Adaptive Control using Verse Library: A Case Study of Quadrotors}

%%
%% The "author" command and its associated commands are used to define
%% the authors and their affiliations.
%% Of note is the shared affiliation of the first two authors, and the
%% "authornote" and "authornotemark" commands
%% used to denote shared contribution to the research.
% \author{Lin Song, Naira Hovakimyan, Pan Zhao, Sayan Mitra, Sheng Cheng, Yangge Li}
\author{Lin Song, Yangge Li, Sheng Cheng, Pan Zhao, Sayan Mitra, Naira Hovakimyan}
% \authornote{Both authors contributed equally to this research.}

% \authornotemark[1]
\affiliation{University of Illinois Urbana-Champaign, United States
  \state{}
  \country{}}
% \email{linsong2@illinois.edu}




% \author{Naira Hovakimyan}
% % \authornote{Both authors contributed equally to this research.}

% % \authornotemark[1]
% \affiliation{%
%   \institution{University of Illinois Urbana-Champaign}
%   \city{Champaign}
%   \state{Illinois}
%   \country{USA}}
% \email{nhovakim@illinois.edu}



% \author{Pan Zhao}
% % \authornote{Both authors contributed equally to this research.}

% % \authornotemark[1]
% \affiliation{%
%   \institution{University of Illinois Urbana-Champaign}
%   \city{Champaign}
%   \state{Illinois}
%   \country{USA}}
%   \email{panzhao2@illinois.edu}

% \author{Sayan Mitra}
% % \authornote{Both authors contributed equally to this research.}

% % \authornotemark[1]
% \affiliation{%
%   \institution{University of Illinois Urbana-Champaign}
%   \city{Champaign}
%   \state{Illinois}
%   \country{USA}}
% \email{mitras@illinois.edu}


% \author{Sheng Cheng}
% % \authornote{Both authors contributed equally to this research.}

% % \authornotemark[1]
% \affiliation{%
%   \institution{University of Illinois Urbana-Champaign}
%   \city{Champaign}
%   \state{Illinois}
%   \country{USA}}
% \email{chengs@illinois.edu}



% % \email{aaa@illinois.edu}

% \author{Yangge Li}
% % \authornote{Both authors contributed equally to this research.}

% % \authornotemark[1]
% \affiliation{%
%   \institution{University of Illinois Urbana-Champaign}
%   \city{Champaign}
%   \state{Illinois}
%   \country{USA}}
% \email{li213@illinois.edu}




%%
%% By default, the full list of authors will be used in the page
%% headers. Often, this list is too long, and will overlap
%% other information printed in the page headers. This command allows
%% the author to define a more concise list
%% of authors' names for this purpose.
\renewcommand{\shortauthors}{Song and Li, et al.}
% \bibliographystyle{unsrt}
%%
%% The abstract is a short summary of the work to be presented in the
%% article.
\begin{abstract}
  $\mathcal{L}_1$ adaptive control ($\mathcal{L}_1$AC) is a control design technique that can  handle a broad class of system uncertainties and provide transient performance guarantees. 
In this work-in-progress abstract, we discuss how existing formal verification tools can be applied to check the performance of $\mathcal{L}_1$AC systems.
We show that the theoretical transient performance and robustness guarantees 
of an $\mathcal{L}_1$ adaptive controller for an 18-dimensional quadrotor system can be verified using the recently developed Verse reachability analysis tool. We will further consider the performance verification of $\mathcal{L}_1$AC on systems with learning-enabled components.
% \textcolor{blue}{sm: Statement about future directions that this work suggests.}.
  % \textcolor{red}{sc: when the system suffers uncertainties and disturbances?}.
% OLD Version
  % Safety-critical Cyber-Physical Systems (CPSs) rely on  control techniques that ensure safety  in the presence of uncertainties and/or disturbances. 
  % $\mathcal{L}_1$ adaptive control ($\mathcal{L}_1$AC) is characterized by its capability of handling large system uncertainties or changes, as well as providing transient performance and robustness guarantees. 
  % % On the other hand, system verification is also essential to further enhance the reliability level 
  % % \textcolor{red}{sc: maybe remove this sentence? It seems unnecessary}. 
  % In safety-critical systems, a verification and validation (V \& V) program needs to be designed and to ensure that no combination of inputs will result in an undesirable output. However, difficulties in the formal verification techniques for the  performance of adaptive control architectures -- under all operating conditions -- are far from being completely addressed. Our work-in-progress aims to verify the $\mathcal{L}_1$AC for its transient and robust performance guarantees  in a variety of application scenarios.
  % % \textcolor{red}{sc: instead of ``more'', maybe use ``a variety of''?}  
  % Our in-progress results show that the theoretical transient performance and robustness guarantees for $\mathcal{L}_1$AC are verifiable using the newly developed Verse Library, and the verification results are illustrated in the form of reachtubes.
  % % \textcolor{red}{sc: when the system suffers uncertainties and disturbances?}.
\end{abstract}
\keywords{Adaptive Control Verification, Safe Autonomous Systems}
\maketitle

\footnotetext{This work is funded by NASA ULI grant 80NSSC22M0070 and AFOSR. }

%%
%% The code below is generated by the tool at http://dl.acm.org/ccs.cfm.
%% Please copy and paste the code instead of the example below.
%%
% \begin{CCSXML}
% <ccs2012>
%  <concept>
%   <concept_id>10010520.10010553.10010562</concept_id>
%   <concept_desc>Computer systems organization~Embedded systems</concept_desc>
%   <concept_significance>500</concept_significance>
%  </concept>
%  <concept>
%   <concept_id>10010520.10010575.10010755</concept_id>
%   <concept_desc>Computer systems organization~Redundancy</concept_desc>
%   <concept_significance>300</concept_significance>
%  </concept>
%  <concept>
%   <concept_id>10010520.10010553.10010554</concept_id>
%   <concept_desc>Computer systems organization~Robotics</concept_desc>
%   <concept_significance>100</concept_significance>
%  </concept>
%  <concept>
%   <concept_id>10003033.10003083.10003095</concept_id>
%   <concept_desc>Networks~Network reliability</concept_desc>
%   <concept_significance>100</concept_significance>
%  </concept>
% </ccs2012>
% \end{CCSXML}

% \ccsdesc[500]{Computer systems organization~Embedded systems}
% \ccsdesc[300]{Computer systems organization~Redundancy}
% \ccsdesc{Computer systems organization~Robotics}
% \ccsdesc[100]{Networks~Network reliability}

%%
%% Keywords. The author(s) should pick words that accurately describe
%% the work being presented. Separate the keywords with commas.



% \received{20 February 2007}
% \received[revised]{12 March 2009}
% \received[accepted]{5 June 2009}

%%
%% This command processes the author and affiliation and title
%% information and builds the first part of the formatted document.
\section{Introduction}
% \textcolor{red}{(LS: The document is still a bit overlength, I found the text in blue might be removable, and then it is exactly 2-page, please help to check and make sure the document without the blue texts is still readable and logically coherent. Please let me know if there is anything missing/unclear.)}

% (LOCATE THE PROBLEM TO SOLVE IN FEW SENTENCES, NOT MUCH SURVEY OF METHODS, EMPHASIZE OBJECTIVE) 
%  technical merit and innovation as well as their potential to stimulate interesting discussions and exchanges of ideas at the conference
Advanced air mobility (AAM) aims to build a reliable and efficient aviation transportation system using highly automated vehicles, e.g., vertical take-off and landing (VTOL) aircraft. A verification and validation (V\&V) framework is a critical element for AAM to ensure  enhanced level of reliability via formal verification of the controller performance. 
% For safety-critical scenarios, according to~\cite{schumann2002toward}, any program or mission with more than \$100M budget has a need for independent verification and validation (IV\&V). 
% \textcolor{blue}{sm: The previous sentences are redundant and tangential for this audience and paper;  citation does not seem right. (3) Use L1AC instead of $\mathcal{L}_1$AC? It will index better.}.

$\mathcal{L}_1$ adaptive controller ($\mathcal{L}_1$AC) distinguishes itself by its capability of compensating for a broad class of model uncertainties with fast adaptation while providing transient and steady-state performance guarantees~\cite{hovakimyan2010l1}. $\mathcal{L}_1$ adaptive control has been successfully deployed and validated on NASA's AirStar 5.5\% subscale generic transport aircraft model~\cite{gregory2009l1}, Calspan's Learjet~\cite{ackerman2017evaluation}, and unmanned aerial vehicles~\cite{wu2023L1QuadFull}. However, there is no prior research  on the formal verification of the transient bounds and robust performance of $\mathcal{L}_1$AC. For complex but deterministic control systems, verification tools, like DryVR~\cite{fan2017dryvr} and the recently-developed Verse~\cite{li2023verse} which can handle black-box components, can be  promising.   
% The ideal reliability level is achieved when every possible operating condition is tested and no problematic behavior  occurs. However, it is neither affordable nor realistic to conduct such exhaustive testing. 
In this paper, we explore applying the Verse Library to formally verify   $\mathcal{L}_1$AC's  robustness and transient performance. From the verification results, we observe that $\mathcal{L}_1$AC achieves verifiable robust transient   performance, and is capable of fast adaptation in systems with time-varying uncertainties. Moreover, we verify that $\mathcal{L}_1$AC guarantees a delay margin (bounded away from zero) when control inputs are subject to time delays, and the tracking performance provided by $\mathcal{L}_1$AC  degrades gracefully as the injected input delay increases.
%
% \textcolor{blue}{sm: Summarize the main findings in 2-3 sentences.}.
% \textcolor{blue}{The benefits of verifying an adaptive control scheme lie in two aspects. First, verification tools, e.g., reachability analysis tools allow users to test on a certain model in simulation, which significantly reduces the development and testing cost since some of those combinations even possibly can lead to catastrophic behaviors in real experiments. Second, approximate algorithms are often applied in the reachability analysis tool development, such that multiple scenarios can be verified in a single test, which significantly improves the verification efficiency and makes it possible to implement a more comprehensive verification.}\textcolor{red}{sc: I think it's fine to drop this blue section per the page limit.}
%In real-world applications, systems can experience various types of uncertainties, while the $\mathcal{L}_1$ adaptive control ($\mathcal{L}_1$AC) approach is capable of compensating for uncertainties and providing guarantees for the transient performance and robustness. However, study on the formal verification of adaptive control performance is still limited. Therefore, it is meaningful to establish a verification framework for adaptive control scheme. Towards this end, we start with exploring the verifiable performance of geometric tracking control with $\mathcal{L}_1$ augmentation on quadrotor systems. Our experiments are designed to imitate different application-motivated scenarios and then establish verifiable robustness performance for $\mathcal{L}_1$AC scheme in the presence of uncertainties.


\section{Problem Formulation}
% {\textcolor{red}{[TO BE COMPLETE]}}
\paragraph{Scenarios}
We consider the performance verification of $\mathcal{L}_1$AC on an 18-dimensional quadrotor system. 
% The goal for the controller is to generate adequate inputs for the plant such that the quadrotor is capable of tracking a prescribed reference trajectory. 
We use a geometric tracking controller as the baseline controller. For comparison, we implement verification on the drone in two scenarios with different controllers --- baseline geometric control with and without $\mathcal{L}_1$AC. 
% In this section, we introduce the quadrotor dynamical model used for adaptive control verification as well as the safety specifications associated with the reachability-based verification.
\paragraph{Plant --- Quadrotor Dynamics}
% We use an inertial frame and a body-fixed frame spanned by unit vectors $\{i_1,i_2,i_3\}$ and $\{b_1,b_2,b_3\}$ in north-east-down directions respectively to depict the quadrotor motion. We assume the origin of the body-fixed frame is at the center of mass (COM) of the quadrotor. The rotation matrix $R$ satisfies the properties as an element from the special orthogonal group $SO(3)$, i.e., $R \in \mathbb{R}^{3 \times 3}, R^\top R = I, \det(R)=1$. \textcolor{red}{sc: I'm not sure if they will appreciate the detailed explanations of the terms here. Maybe just show the equations of motion and refer to our previous paper for details.}
% According to the rotation matrix $R$, the direction of the $i$-th body-fixed axis $b_i$ is given by $Re_i$ in the inertial frame, where $e_i$ is the unit vector with the $i$-th element being 1 and $j$-th element being 0 for $i,j \in \{1,2,3\}$ and $i \ne j$.

The equations of motions (EOMs) of a quadrotor~\cite{wu20221,lee2010geometric} are given by
\begin{equation}\label{eq:eom}
    \dot p = v, 
    % \dot v = ge_3 - \frac{f}{m}Re_3,\\
    \dot v = ge_3 - fRe_3/m,
    \dot R = R\Omega^\wedge, 
    \dot \Omega = J^{-1}(M-\Omega \times  J\Omega),
\end{equation}
 where $p, v \in \mathbb{R}^3$ are the position and velocity of the quadrotor's center of mass (COM) in the inertial frame, $g$ is the gravitational acceleration, $m$ is the vehicle mass, $\Omega \in \mathbb{R}^3$ is the angular velocity in the body-fixed frame, $J \in \mathbb{R}^{3 \times 3}$ is the moment of inertia matrix, and $R \in \{ \mathbb{R}^{3 \times 3}|R^\top R = I, \det(R) = 1\}$ is the rotation matrix. The wedge operator $(\cdot)^{\wedge}: \mathbb{R}^3 \to \mathfrak{so}(3)$ denotes the mapping to the space of skew-symmetric matrices. The control inputs include the collective thrust $f $ and the moment $M \in \mathbb{R}^3$ in the body-fixed frame. 
 %\pan{the explanation of $R$ is missing}
% \begin{align}
%     \dot p &= v,\\
%     \dot v &= ge_3 - \frac{f}{m}Re_3,\\
%     \dot R &= R\Omega^\wedge,\\
%     \dot \Omega &= J^{-1}(M-\Omega \times  J\Omega),
% \end{align}
% \vspace{-1.6pc}





% \begin{figure*}[h]
% \centering
% \subfloat[Performance of geometric control]{\label{fig:1_mdleft}{\includegraphics[height=2.5cm]{nodelay_nol1_tv_mass_zt_verifytube.png}}}
% \subfloat[Performance of geometric control w/ $\mathcal{L}_1$AC]{\label{fig:1_mdmiddle}{\includegraphics[height=2.5cm]{nodelay_l1_tv_mass_zt_verifytube.png}}}
% \subfloat[Time-varying mass profile]{\label{fig:1_mdright}{\includegraphics[height=2.5cm]{mass_profile_fig1.png}}}
% \caption{Transient performance verification of $\mathcal{L}_1$AC subject to time-varying system parameters}
% \label{fig:fig1_big}
% \end{figure*}

\paragraph{Controller --- Geometric Control with $\mathcal{L}_1$ Augmentation}\label{sec2_ctrl}
The geometric controller~\cite{lee2010geometric} can ensure exponential stability for quadrotor's nominal dynamics while tracking a prescribed trajectory $p_d(t) \in \mathbb{R}^3$ and yaw angle $\psi_d(t) \in \mathbb{R}$ for $t \in [0,t_f]$.  In this paper, we compute the baseline geometric control law for the nominal quadrotor dynamics~\eqref{eq:eom}. The desired thrust is  $f = -F_d \cdot (Re_3)$, where $F_d = -K_pe_p - K_v e_v - mge_3 + m\ddot{p}_d$ denotes the desired force vector, $e_p, e_v$ are the position and velocity error vectors. The desired moment given by the geometric controller is $
     M = -K_R e_R - K_\Omega e_\Omega + \Omega \times J\Omega \nonumber 
     - J(\Omega^\wedge R^\top R_d \Omega_d - R^\top R_d \dot{\Omega}_d)$, where $e_R,e_\Omega, R_d$ are the attitude error, angular velocity error, and desired rotation matrix (see~\cite{wu20221,lee2010geometric} for the computation of $e_R, e_\Omega,R_d,\Omega_d,\dot{\Omega}_d$). 
     % Nevertheless, when the system is subject to uncertainties or disturbances, the baseline geometric control for the nominal quadrotor dynamics is insufficient to guarantee desired trajectory-tracking behavior or even stability. 
     To establish verification in the presence of uncertainties, we use the geometric control with $\mathcal{L}_1$ augmentation introduced in~\cite{wu20221} to compensate for the uncertainties. $\mathcal{L}_1$AC includes a state predictor, an adaptation law, and a low-pass filter. (Interested readers can refer to~\cite{wu2023L1QuadFull} for a detailed discussion on the theoretical   bounds guaranteed by the $\mathcal{L}_1$AC.) 
\paragraph{Uncertainty \& Input Delays}
We create a model mismatch setting, i.e., the mass applied for controller design ($m_0$) is different from the actual mass ($m'$) for verification, where $m'$ is also a time-varying uncertain value with known time-dependent bounds. This setting is designed to verify the robust performance of $\mathcal{L}_1$AC and its capability of fast adaptation in systems with time-varying parametric uncertainties. In the use of a verification tool (e.g., Verse~\cite{li2023verse}), we introduce the uncertain mass as an augmented state, such that the time-dependency and uncertain initial masses can be captured in the verification. Time delays on control inputs widely exist on the hardware. In addition to the uncertain model mass, we also inject input delays to verify the robustness margins achieved by $\mathcal{L}_1$AC. 

% One property for $\mathcal{L}_1$AC is the graceful degradation as the input delay increases and $\mathcal{L}_1$AC can preserve a positive delay margin, and we aim to verify the claim using formal verfication tools.  

% \begin{figure}[ht]
%   \centering
%   \includegraphics[height=2cm]{mass_profile_fig1.png}
%   \caption{Time-varying mass profile.}
%   \label{fig:4_mass}

% \end{figure}
% \pan{better not to include delay as part of uncertainties. Delay is for testing the robustness margin}
%Uncertainties (e.g., uncertain model parameters, mode transition, or external disturbances) can lead the system to enter non-deterministic states.


\section{Verification Solution}



\begin{figure}[h]
  \centering
  % \vspace{-0.3cm}
  \includegraphics[height=2.7cm]{method_arch.png}
  \caption{Verification architecture of geometric tracking control with $\mathcal{L}_1$ augmentation using the Verse Library~\cite{li2023verse}.}
  \label{fig:arch}

\end{figure}

\paragraph{Solution of ordinary differential equation (ODE)}
An ordinary differential equation (ODE) describes the instantaneous rule governing the change of states of a physical system.  The solution of the ODE, called trajectory,  models how the system state changes over time. ODE solutions can describe the system states  reachable within a given time horizon, which can also be interpreted from the perspective of trajectory evolution. The complete formulation of the quadrotor's ODE dynamics~\eqref{eq:eom} with geometric controller and $\mathcal{L}_1$AC is omitted, and we refer the interested readers to~\cite{wu2023L1QuadFull} for more details.
\paragraph{Reachability Analysis}
\textcolor{black}{For an ODE system with a set of initial states $X_0$ or parameter values, the {\em reachable set\/}, denoted by $Reach(X_0, t_f)$, is the  set of states that the solutions of the system can hit from {\em any\/} initial state $X_0$ within time $t_f$. 
%
Given an unsafe set $U$, checking that $Reach(X_0, t_f) \cap U =\emptyset$ is a standard way for checking bounded safety. There are several tools for over-approximating $Reach(X_0, t_f)$ for the above check (see~\cite{ReachSurveyChenNASA22,MitraCPSBook2021} for recent survey).
}

% \textcolor{blue}{sm: We can drop the rest of this section. (2) Reachtube is a general concept where the timing information is recorded along with the state information. It has nothing to do with 2D. (3) I dont understand the point you are making with $t_1, t_2$. (3) We can merge the next section on Verse with this section.}
%  For example, in a 2D space, the set of reachable states are in the shape of tubes, i.e., reachtubes. For the safety verification purpose, the set of reachable states are typically over-approximated such that the computation can cover all the possible reachable states in the concerned time horizon, so that no over-optimistic verification result is returned. 
% \textcolor{black}{On the time interval $[0,t_f]$, we use $\mathcal{Y}([0,t_f])$ to describe the \textcolor{black}{set of reachable state} for a system. Then by definition, $\mathcal{Y}([0,t_1]) \subseteq \mathcal{Y}([0,t_2])$ for $0 \le t_1 \le t_2$. We denote the safe region in state space by $\mathcal{S}$ and the logical negation by $\neg$. Once the reachable set is computed, the safety verification problem reduces to a trivial task for satisfiability modulo theories (SMT) solvers and is achieveble by checking $\mathcal{Y} \cap \neg \mathcal{S} = \phi$.}

\paragraph{Reachability analysis with Verse}

\textcolor{black}{Verse~\cite{li2023verse} is a Python library for modeling and reachability analysis of hybrid, multi-agent scenarios. It allows for the dynamics of the individual agents to be described by black-box simulators (written in any language), uses the probabilistic algorithm of DryVR~\cite{fan2017dryvr} for computing sensitivity functions with probably approximately correct (PAC) guarantees, and then uses a simulation-based algorithm for over-approximating the reachable states.} 
% \textcolor{blue}{sm: Again, we could drop the rest of this para.}
% In Verse, the simulator code can be written in any language and is treated as a black-box. However, when the system dynamics are given or known as in our scenario, a model-based reachability analysis is implemented within Verse. Furthermore, Verse was primarily developed to address the verification problem for multi-agent scenarios with different dynamics and different decision logics through reachability analysis, and built upon existing reachability analysis algorithm initiated in DryVR~\cite{fan2017dryvr} where a black-box simulator and white-box transition graph are combined.      
We present the verification architecture for the quadrotor system with 
 $\mathcal L_1$AC in Fig~\ref{fig:arch}. The simulator code of the closed-loop quadrotor system is taken as the Verse library input and is intepreted as a black-box simulator. The inputs for verification also include some hyperparameters, e.g., time horizon and step size. 
 % In verification, the quadrotor is designed to experience different uncertainties, including but not limited to, mode transitions, time-varying system parameters, and uncertain initial states.
% \subsection{Guarantees on the computed reachset by DryVR}
 






% \begin{figure}[h]
%   \centering
%   \includegraphics[width=0.8\linewidth]{10s_singlemode_nol1_xy.png}
%   \caption{Verification without $\mathcal{L}_1$AC.}
%   \Description{A woman and a girl in white dresses sit in an open car.}
% \end{figure}
% \vspace{-1.6pc}




\section{Experiments and Results}
% In the experiment, we model the quadrotor mass in dynamics~\eqref{eq:eom} as an uncertain parameter to test the controller robustness performance. This verification experiment is motivated by real-world applications where accurate identification/measurement of system parameters is not achievable and some model-based controller design techniques may fail. We consider the actual quadrotor mass as an arbitrary value ranging from $0.5m_0$ to $3.5m_0$, where $m_0$ is the nominal quadrotor mass used for controller design. The quadrotor is designed to track a circular trajectory under parametric uncertainties, and we present the verification results in Fig~\ref{fig:result}. We observe the set of quadrotor's reachable coordinates $(x,y)$ is much smaller with the $\mathcal{L}_1$ augmentation when the model parameter is uncertain. 

\begin{figure}
\subfigure{\includegraphics[width=4cm]{nodelay_nol1_tv_mass_zt_verifytube_new.png}}
\subfigure{\includegraphics[width=4cm]{nodelay_l1_tv_mass_zt_verifytube_new.png}}
\caption{Transient performance verification of $\mathcal{L}_1$AC subject to time-varying system parameters: performance of geometric control (left) and geometric control w/ $\mathcal{L}_1$AC (right)}
\label{fig:fig1_big}
\end{figure}

\begin{figure}
\vspace{-1pc}
\subfigure{\includegraphics[width=4cm]{delay60ms_l1_tv_mass_zt_verifytube.png}}
\subfigure{\includegraphics[width=4cm]{delay120ms_l1_tv_mass_zt_verifytube.png}}
\caption{Robust performance verification of $\mathcal{L}_1$AC in the presence of input delay and time-varying system parameters:  with input delay of 60ms (left) and 120ms (right)}
\label{fig:subfigures}
\end{figure}


% \begin{figure}[ht]
% \vspace{-0.5cm}
% \centering
% \subfloat[input delay = 60ms]{\label{fig:2_mdleft}
% {\includegraphics[height=2cm]{delay60ms_l1_tv_mass_zt_verifytube.png}}}
% \subfloat[input delay = 120ms]{\label{fig:2_mdright}{\includegraphics[height=2cm]{delay120ms_l1_tv_mass_zt_verifytube.png}}}
% \caption{Performance verification of $\mathcal{L}_1$AC in the presence of input delay and time-varying system parameters}
% \label{fig:subfigures}
% \end{figure}


In the experiments, we evaluate the robust performance of $\mathcal{L}_1$AC by observing the computed reachable states on uncertain systems using Verse. 
% This experiment is motivated by real-world applications with time-varying system parameters, where some model-based controller design techniques may fail. However, 
 We first verify the transient performance guarantees of $\mathcal{L}_1$AC by modeling the quadrotor mass in~\eqref{eq:eom} as an uncertain and rapidly changing value with known bounds. The control goal for the quadrotor is to track a given reference trajectory. From the computed z-axis reachtube in the cases of with and without $\mathcal{L}_1$AC shown in~Fig.~\ref{fig:fig1_big}, we see that $\mathcal{L}_1$AC achieves consistent and better tracking performance even with time-varying system parameters. $\mathcal{L}_1$AC is capable of handling rapidly changing parametric uncertainties and achieving consistent performance with fast adaptation. Furthermore, we verify the robustness of $\mathcal{L}_1$AC against input delay. We still consider an uncertain time-varying quadrotor mass and gradually increase the injected time delay on the control input. The verification results in the form of \textcolor{black}{reachtube} in the presence of different input delays are shown in Fig.~\ref{fig:subfigures}. %Fig.~\ref{fig:2_mdleft} and Fig.~\ref{fig:2_mdright}.  
 It is observed that $\mathcal{L}_1$AC delivers reasonably good performance in the presence of input delay up to 60~ms. The verification results also show that the tracking performance degrades gracefully as the injected time delay increases. 
% Our result-in-progress shows that the performance of $\mathcal{L}_1$ adaptive control is formally verifiable via verification tools. The verification results are illustrated in the form of reachtubes and 
For future work, we will consider the verification of $\mathcal{L}_1$ adaptive controllers with learning-enabled components.

% \begin{figure}[h]%
% \centering
% \subfigure{%
% \label{fig:first}%
% \includegraphics[height=1.3in]{10s_singlemode_WITHOUTl1_xy.png}}%
% \subfigure{%
% \label{fig:second}%
% \includegraphics[height=1.3in]{10s_singlemode_WITHl1_xy.png}}%
% \caption{Verification results of an uncertain quadrotor w/ geometric control without (Left) v/s with (Right) $\mathcal{L}_1$AC.}
% \label{fig:result}
% \end{figure}
% \section{Summary and Future Work}
% In this work, we propose a verification framework for $\mathcal{L}_1$ adaptive control. 
% \begin{figure}
%   \centering
%   \includegraphics[width=0.8\linewidth]{10s_singlemode_l1_xy.png}
%   \caption{Verification with $\mathcal{L}_1$AC.}
%   \Description{A woman and a girl in white dresses sit in an open car.}
% \end{figure}









































%%
%% The acknowledgments section is defined using the "acks" environment
%% (and NOT an unnumbered section). This ensures the proper
%% identification of the section in the article metadata, and the
%% consistent spelling of the heading.
% \begin{acks}

% \end{acks}

%%
%% The next two lines define the bibliography style to be used, and
%% the bibliography file.
\bibliographystyle{abbrv}
\bibliography{main-ref}

%%
%% If your work has an appendix, this is the place to put it.




\end{document}
\endinput
%%
%% End of file `sample-lualatex.tex'.
