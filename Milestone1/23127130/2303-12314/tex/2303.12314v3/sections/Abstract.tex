\begin{abstract}
Prompt tuning is a parameter-efficient method, which learns soft prompts and conditions frozen language models to perform specific downstream tasks. Though effective, prompt tuning under few-shot settings on the one hand heavily relies on a good initialization of soft prompts. On the other hand, it can easily overfit to few-shot training samples, thereby undermining generalizability. Existing works leverage pre-training or supervised meta-learning to initialize soft prompts but they fail to data-efficiently generalize to unseen downstream tasks. To address the above problems, this paper proposes a novel \textbf{\underline{S}}elf-s\textbf{\underline{U}}pervised meta-\textbf{\underline{P}}rompt learning framework with \textbf{\underline{ME}}ta-gradient \textbf{\underline{R}}egularization for few-shot generalization ~(\textbf{SUPMER}). SUPMER leverages self-supervised meta-learning with a diverse set of well-designed meta-tasks to learn a universal prompt initialization for efficient adaptation using only unlabeled data. Additionally, it jointly meta-learns a gradient regularization function to transform raw gradients into a domain-generalizable direction, thus alleviating the problem of overfitting. Extensive experiments show that SUPMER achieves better performance for different few-shot downstream tasks, and also exhibits a stronger domain generalization ability. 

\end{abstract}