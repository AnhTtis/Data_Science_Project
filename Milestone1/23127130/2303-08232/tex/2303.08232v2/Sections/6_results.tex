\section{RESULTS}

\subsection{Framework Timing}

The teleoperation framework divides computation among three threads: a kinematics-statics thread that computes Eq. \ref{eq:IKQP}, a VR thread and a transition feasibility thread. The VR application is run on a Valve Index \cite{Index} at a refresh rate of 90\si{\hertz}. We use Java Monkey Engine \cite{JME} for scene graphics. The execution times of various tasks in the framework were measured and are reported in Tab. \ref{tab:timing}. Each reported time is the average over a window of 100 updates. The evaluation was performed on a desktop with a 10th gen 10-core i9 processor (3.70 \si{\giga\hertz}). This was measured under a nominal operating condition while standing with 8 contact anchors, two taskspace posture anchors, two joint position anchors and a CoM anchor.

\renewcommand{\arraystretch}{1.0}
\begin{table}[h!]
\centering
\captionof{table}{Timing Evaluation} \label{tab:timing} 
\begin{tabular}{ | c | c | c | } 
 \hline
 Task & Description & Time (\si{\milli\second}) \\
 \hline
 \hline
 Kinematics-Statics & \textbf{Total} & \textbf{4.63} \\
        Solver    & CoM Region & 4.05 \\
                           & Kinematics Solver & 0.58 \\
        \hline 
VR Interface & \textbf{Total} & \textbf{5.81} \\
        & VR API & 2.00 \\
        & Kinematic Task Processing & 1.06 \\
        & Collision Check & 0.06 \\
        & Contact Removal & 2.43 \\
        & Gravity Compensation Torques & 0.26 \\
        \hline
Transition Feasibility & \textbf{Total} & \textbf{77.3} \\
        \hline
\end{tabular}
\end{table}

\subsection{Experiment 1: Crouch to Stand, Simulated}

We tested\footnote[2]{A video of the teleoperation is available at \url{https://www.youtube.com/watch?v=AJlSsU4Tvkk}} our teleoperation framework in a simulated scenario where Valkyrie starts in a crouched position on the ground next to two flat boxes, to the front and left of the robot. Tab. \ref{tab:script_stats} shows the teleoperation statistics, including anchor count and achieved contact modes. As shown on the bottom rows, significant use of line contacts was made in addition to point and plane contacts. Fig. \ref{fig:results_combo} highlights important keyframes and contact modes. In keyframe 14, the left forearm is supported through a line contact between the elbow and hand. Also in keyframe 14, the right foot forms a line contact on the inside edge of the foot. In keyframe 21, the front edge of the left foot forms a line contact while placing the foot down. The trajectory was executed using a whole-body impedance controller at 2s per keyframe with an average joint tracking $(0.0043\pm 0.002)$\si{\radian}.

\renewcommand{\arraystretch}{1.0}
\begin{table*}[h!]
\begin{center}
\captionof{table}{Experiment 1 Statistics Simulated Crouch to Stand. Operation Time: 16m18s.} \label{tab:script_stats} 
\begin{tabular}{| c | c | c | c | c | c | c | c | c | c | c | c | c | c | c | c | c | c | c | c | c | c | c | c | c | c | c | c | c | c | c | c | c |}
\hline
 & Keyframe & 1 & 2 & 3 & 4 & 5 & 6 & 7 & 8 & 9 & 10 & 11 & 12 & 13 & 14 & 15 & 16 & 17 & 18 & 19 & 20 & 21 & 22 & 23 & 24 & 25 & 26 & 27 & 28 & 29 & 30 & 31 \\
 \hline
\hline
 & Event & \multicolumn{5}{c|}{\text{Place R. Hand}} & \multicolumn{5}{c|}{\text{Place L. Hand/Elbow}} & \multicolumn{4}{c|}{\text{Place R. Foot}} & \multicolumn{7}{c|}{\text{Lift L. Knee, Place L. Foot}} & \multicolumn{3}{c|}{\text{Lift L. Elbow}} & \multicolumn{3}{c|}{\text{Lift R. Hand}} & \multicolumn{4}{c|}{\text{Lift L. Hand}} \\
 \hline
\rotatebox[origin=c]{90}{Taskspace} & \makecell{Total \\ Modified \\ Add/Rem.} & \makecell{0\\  - \\ - }  & \makecell{1\\  - \\+1}  & \makecell{1\\ 1\\ - }  & \makecell{0\\  - \\-1}  & \makecell{0\\  - \\ - }  & \makecell{1\\  - \\+1}  & \makecell{1\\ 1\\ - }  & \makecell{1\\ 1\\ - }  & \makecell{0\\  - \\-1}  & \makecell{0\\  - \\ - }  & \makecell{1\\  - \\+1}  & \makecell{0\\  - \\-1}  & \makecell{0\\  - \\ - }  & \makecell{0\\  - \\ - }  & \makecell{1\\  - \\+1}  & \makecell{1\\ 1\\ - }  & \makecell{0\\  - \\-1}  & \makecell{0\\  - \\ - }  & \makecell{0\\  - \\ - }  & \makecell{0\\  - \\ - }  & \makecell{0\\  - \\ - }  & \makecell{1\\  - \\+1}  & \makecell{1\\  - \\ - }  & \makecell{0\\-\\-1}  & \makecell{0\\  - \\ - }  & \makecell{0\\  - \\ - }  & \makecell{1\\  - \\+1}  & \makecell{1\\ 1\\ - }  & \makecell{1\\ 1\\ - }  & \makecell{2\\  - \\+1}  & \makecell{2\\ 2\\ - }  \\
 \hline
\rotatebox[origin=c]{90}{Joint} & \makecell{Total \\ Modified \\ Add/Rem.} & \makecell{4\\  - \\ - }  & \makecell{5\\  - \\+1}  & \makecell{5\\ 1\\ - }  & \makecell{5\\  - \\ - }  & \makecell{5\\  - \\ - }  & \makecell{6\\  - \\+1}  & \makecell{5\\  - \\-1}  & \makecell{5\\  - \\ - }  & \makecell{6\\  - \\+1}  & \makecell{6\\ 1\\ - }  & \makecell{6\\  - \\ - }  & \makecell{9\\ 1\\+3}  & \makecell{10\\ 3\\+1}  & \makecell{4\\  - \\-6}  & \makecell{4\\  - \\ - }  & \makecell{4\\  - \\ - }  & \makecell{7\\ 1\\+3}  & \makecell{6\\ 2\\-1}  & \makecell{6\\ 3\\ - }  & \makecell{7\\ 4\\+1}  & \makecell{2\\ 1\\-5}  & \makecell{2\\  - \\ - }  & \makecell{2\\  - \\ - }  & \makecell{3\\-\\+1}  & \makecell{3\\  - \\ - }  & \makecell{3\\  - \\ - }  & \makecell{3\\  - \\ - }  & \makecell{4\\ 1\\+1}  & \makecell{4\\  - \\ - }  & \makecell{4\\  - \\ - }  & \makecell{4\\  - \\ - }  \\
 \hline
\rotatebox[origin=c]{90}{CoM} & \makecell{Enabled \\ Modified } & \makecell{\checkmark\\ \checkmark}  & \makecell{\checkmark\\ \checkmark}  & \makecell{\checkmark\\ \checkmark}  & \makecell{\checkmark\\ \checkmark}  & \makecell{\checkmark\\ \checkmark}  & \makecell{\checkmark\\ \checkmark}  & \makecell{\checkmark\\ \checkmark}  & \makecell{\checkmark\\ \checkmark}  & \makecell{\checkmark\\ \checkmark}  & \makecell{\checkmark\\ \checkmark}  & \makecell{\checkmark\\ \checkmark}  & \makecell{\checkmark\\ \checkmark}  & \makecell{\checkmark\\ \checkmark}  & \makecell{\checkmark\\ \checkmark}  & \makecell{ - \\ \checkmark}  & \makecell{\checkmark\\ \checkmark}  & \makecell{\checkmark\\ \checkmark}  & \makecell{\checkmark\\ \checkmark}  & \makecell{\checkmark\\ \checkmark}  & \makecell{\checkmark\\ \checkmark}  & \makecell{\checkmark\\ \checkmark}  & \makecell{\checkmark\\ \checkmark}  & \makecell{\checkmark\\ \checkmark}  & \makecell{\checkmark\\ \checkmark}  & \makecell{\checkmark\\ \checkmark}  & \makecell{\checkmark\\  - }  & \makecell{\checkmark\\  - }  & \makecell{\checkmark\\ \checkmark}  & \makecell{\checkmark\\ \checkmark}  & \makecell{\checkmark\\ \checkmark}  & \makecell{\checkmark\\ \checkmark}  \\
 \hline
\rotatebox[origin=c]{90}{Contact} & \makecell{Total \\ Modified \\ Add/Rem.} & \makecell{4\\  - \\ - }  & \makecell{3\\  - \\-1}  & \makecell{3\\  - \\ - }  & \makecell{4\\  - \\+1}  & \makecell{4\\  - \\ - }  & \makecell{3\\  - \\-1}  & \makecell{3\\  - \\ - }  & \makecell{3\\  - \\ - }  & \makecell{4\\  - \\+1}  & \makecell{5\\  1 \\+1}  & \makecell{4\\  - \\-1}  & \makecell{4\\  - \\ - }  & \makecell{4\\  - \\ - }  & \makecell{6\\ 2\\+2}  & \makecell{5\\  - \\-1}  & \makecell{6\\  - \\+1}  & \makecell{6\\  - \\ - }  & \makecell{6\\  - \\ - }  & \makecell{6\\  - \\ - }  & \makecell{6\\  - \\ - }  & \makecell{7\\  - \\+2,-1}  & \makecell{6\\  - \\-1}  & \makecell{6\\  - \\ - }  & \makecell{7\\2\\+1}  & \makecell{7\\ - \\ - }  & \makecell{7\\ 1\\ - }  & \makecell{6\\  - \\-1}  & \makecell{7\\  - \\+1}  & \makecell{7\\  - \\ - }  & \makecell{6\\  - \\-1}  & \makecell{6\\  - \\ - }  \\
 \hline
 \hline
 \rotatebox[origin=c]{90}{C. Mode} & \makecell{Point \\ Line \\ Plane} & \makecell{4\\  - \\ - }  & \makecell{3\\  - \\ - }  & \makecell{3\\  - \\ - }  & \makecell{4\\  - \\-}  & \makecell{4\\  - \\ - }  & \makecell{3\\  - \\-}  & \makecell{3\\  - \\ - }  & \makecell{3\\  - \\ - }  & \makecell{4\\  - \\-}  & \makecell{3\\  1 \\-}  & \makecell{2\\  1 \\-}  & \makecell{2\\  1 \\-}  & \makecell{2\\  1 \\-}  & \makecell{2\\ 2\\-}  & \makecell{1\\ 2\\-}  & \makecell{1\\ 1\\1}  & \makecell{1\\ 1\\1}  & \makecell{1\\ 1\\1}  & \makecell{1\\ 1\\1}  & \makecell{1\\ 1\\1}  & \makecell{1\\ 2\\1}  & \makecell{2\\ 1\\1}  & \makecell{2\\ 1 \\ 1 }  & \makecell{2\\ 1 \\ 1 }  & \makecell{2\\ 1 \\ 1 }  & \makecell{2\\ 1\\ 1 }  & \makecell{1\\  1 \\1}  & \makecell{1\\  - \\2}  & \makecell{1\\  - \\ 2 }  & \makecell{-\\  - \\2}  & \makecell{-\\  - \\ 2 }  \\
 \hline
\end{tabular}
\end{center}
\end{table*}

\subsection{Experiment 2: Crouch to Kneel, Simulated with Hardware Validation}

We deployed a modified version of the keyframe sequence in Experiment 1 to a physical Valkyrie robot. The sequence starts in the same configuration but only has a front block, as forearm contact on the left block was not possible due to exposed wiring. The same three initial events are present as in Experiment 1: place right hand, place left hand and place right foot. As shown in Fig. \ref{fig:results_combo}, triple support between two knees and a single hand contact were achieved while placing a hand on the blocks (keyframe 8). Additionally, line contact on the right foot was made while placing the foot down (keyframe 19). The trajectory is tracked at 4s per keyframe using an impedance controller which modulates gains based on limb loading (see Appendix) and achieved an average joint tracking of $(0.023\pm 0.009)$\si{\radian}. This experiment was performed by first teleoperating the robot in simulation, then deploying the trajectory to Valkyrie. Deploying to hardware serves as empirical verification of the presented feasibility checks as well as simulation accuracy. Using this framework for live hardware teleoperation will be pursued in future work and was not performed due to hardware safety concerns.

% Why :
%% Safety - for crouching manuever something is very likely to break
%% Motor overheating - torque specs may break down as motors heat up.
%% Harness - for crouching harness isn't great. Limit robot testing with harness as-is.

\subsection{Experiment 3: Bracing Against a Wall, Simulated}

For experiment 3, the robot stands on a narrow platform with a wall on the side and 34cm tall obstacle blocking the path. The robot braces against the wall with the left forearm while swinging the left foot over the obstacle (Fig. \ref{fig:results_combo}). The use of the left elbow contact reduces load on the elbow joint from 40-55Nm (without elbow contact) to 15-25Nm (with elbow contact). In this scenario, CoM feasibility plays a crucial role as the robot is swinging its foot (Fig. \ref{fig:exp3_com}). During the motion, the feasible CoM region changes on an order of 10cm as the posture, namely the supporting limb Jacobians, are changing. This highlights a case where CoM feasibility is not intuitive and the operator relies on the automated feasibility checks during teleoperation. The trajectory was executed using a whole-body inverse dynamics controller at 2s per keyframe with an average joint tracking of $(0.036\pm 0.008)$\si{\radian}.

\begin{figure}
    \centering
    \includegraphics[width=\columnwidth]{Figures/exp3_com_cropped.png}
    \caption{The trajectory for experiment 3 consists of 4 keyframes as the robot is braced against a wall. This top-down view shows the CoM trajectory along with the set of keyframe CoM constraint regions. The regions with blue, orange, green and red outlines correspond to the feasible CoM regions in keyframes 1-4 respectively.}
    \label{fig:exp3_com}
\end{figure}