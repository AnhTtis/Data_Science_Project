\textit{Feasible CoM Region}. Following the approach from Orsolino et al. \cite{orsolino2020feasible}, the extreme feasible planar CoM position $\mathbf{c}_{xy} \in \mathbb{R}^2$ along a query direction $\mathbf{a}\in \mathbb{R}^2$ is computed as:

\begin{equation} \label{eq:feasible_region}
\begin{aligned}
    \max_{\mathbf{c}_{xy},\mathbf{f}} \, \mathbf{a}^T \mathbf{c}_{xy} \;\;\; \textrm{s.t.}   \quad &  \mathbf{A}_f \mathbf{f} + \mathbf{A}_c \mathbf{c}_{xy} = \mathbf{b}_f \\
                                            \quad &   \mathbf{C}_f \mathbf{f} \leq \mathbf{d}_f \\ 
\end{aligned}
\end{equation}

Where $\mathbf{A}_f, \mathbf{A}_c,\mathbf{b}_f,\mathbf{C}_f,\mathbf{d}_f$ are the constraints in Eq. \ref{eq:StaticFeasibility}. The equality constraint enforces static equilibrium and the inequality constraint unilateral contact forces, friction constraints, and actuation limits. When solving Eq. \ref{eq:feasible_region} for a given direction $\mathbf{a}_i$, the optimized CoM value $\mathbf{c}_{xy}^*$ lies on the boundary of the feasible CoM region. By querying in a set of directions, an approximation of the feasible CoM region is determined.

\textit{Static Force Distribution} An optimal static force distribution is computed similarly to Eq. \ref{eq:feasible_region} but with $\mathbf{c}_{xy}$ constrained to the keyframe's current position. 
\begin{equation} \label{eq:ForceDistribution}
\begin{aligned}
    \min_{\mathbf{f}} \; \lVert \mathbf{f} \rVert_2^2 \;\; \textrm{s.t.} \quad &  \mathbf{A}_{f} \mathbf{f} = \mathbf{b}_{f} - \mathbf{A}_c\mathbf{c}_{xy}  \\
                        \quad &  \mathbf{C}_{f} \mathbf{f} \leq \mathbf{d}_f
\end{aligned}
\end{equation}

\textit{Valkyrie Impedance Gains} Tab. \ref{tab:impedance_gains} is the set of loaded joint stiffness and damping values used for Experiment 2. A limb is considered loaded when any link contains a contact anchor. Unloaded gains are computed by reducing the loaded gain by a factor of 0.65.

\renewcommand{\arraystretch}{1.0}
\begin{table}[h!]
\centering
\captionof{table}{Hardware Loaded Joint Impedance Gains} \label{tab:impedance_gains} 
\begin{tabular}{ c c c } 
 \hline
  Joint & Stiffness & Damping \\ 
     &  [\si{\newton\meter\per\radian}] & [\si{\newton\meter\second\per\radian}] \\ 
 \hline
Sh. Pitch & 1600 & 32  \\
Sh. Roll & 1750 & 50  \\
Sh. Yaw & 550 & 12  \\
El. Pitch & 900 & 30  \\
Hip Yaw & 1600 &  32   \\
Hip Roll & 2000 & 55  \\
Hip Pitch & 2000 & 55  \\
Knee Pitch & 2000 & 60 \\
Ankle Roll & 800 &  10  \\
Ankle Pitch & 800  & 10  \\
Spine Yaw & 1750 & 65  \\
Spine Pitch & 1500 & 45  \\
Spine Roll & 1250  & 55  \\
 \hline
\end{tabular}
\end{table}