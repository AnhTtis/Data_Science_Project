%%%%%%%%%%%%%%%%%%%%%%%%%%%%%%%%%%%%%%%%%%%%%%%%%%%%%%%%%%%%%%%%%%%%%%%%%%%%%%%%
%2345678901234567890123456789012345678901234567890123456789012345678901234567890
%        1         2         3         4         5         6         7         8

\documentclass[letterpaper, 10 pt, conference]{ieeeconf}  % Comment this line out if you need a4paper

%\documentclass[a4paper, 10pt, conference]{ieeeconf}      % Use this line for a4 paper

\IEEEoverridecommandlockouts                              % This command is only needed if 
                                                          % you want to use the \thanks command

\overrideIEEEmargins                                      % Needed to meet printer requirements.

%In case you encounter the following error:
%Error 1010 The PDF file may be corrupt (unable to open PDF file) OR
%Error 1000 An error occurred while parsing a contents stream. Unable to analyze the PDF file.
%This is a known problem with pdfLaTeX conversion filter. The file cannot be opened with acrobat reader
%Please use one of the alternatives below to circumvent this error by uncommenting one or the other
%\pdfobjcompresslevel=0
%\pdfminorversion=4

% See the \addtolength command later in the file to balance the column lengths
% on the last page of the document

% The following packages can be found on http:\\www.ctan.org
\usepackage{graphics} % for pdf, bitmapped graphics files
\usepackage{epsfig} % for postscript graphics files
\usepackage{mathptmx} % assumes new font selection scheme installed
\usepackage{times} % assumes new font selection scheme installed
\usepackage{amsmath} % assumes amsmath package installed
\usepackage{amssymb}  % assumes amsmath package installed
\usepackage{graphicx}
\usepackage{caption}
\usepackage{mathrsfs}
\usepackage[ruled,vlined]{algorithm2e}
\usepackage[font={small}]{caption}
\usepackage[mathscr]{euscript}

\usepackage{array}
\newcommand{\PreserveBackslash}[1]{\let\temp=\\#1\let\\=\temp}
\newcolumntype{C}[1]{>{\PreserveBackslash\centering}p{#1}}
\newcolumntype{R}[1]{>{\PreserveBackslash\raggedleft}p{#1}}
\newcolumntype{L}[1]{>{\PreserveBackslash\raggedright}p{#1}}

% \usepackage{tabularx}

\usepackage[hidelinks]{hyperref}
\hypersetup{
    colorlinks=true,
    linkcolor=blue,
    urlcolor=blue,
    citecolor=blue
}

\usepackage{balance}

\usepackage[utf8]{inputenc}
\usepackage[english]{babel}
\usepackage[backend=biber,sorting=none,citestyle=numeric-comp]{biblatex}
% \usepackage[comma,numbers]{natbib}

\addbibresource{bibliography.bib}
\renewcommand*{\bibfont}{\footnotesize}

% Title
\title{\LARGE \bf

% A Direct Teleoperation Method for Generating Humanoid Multi-Contact Trajectories
% Intuitive Generation of Deployable Humanoid Multi-Contact Trajectories

A Virtual-Reality Driven Approach for Generating Humanoid Multi-Contact Trajectories

% An Intuitive Framework for Generating Humanoid Multi-Contact Trajectories

}

\author{Stephen McCrory$^{1,2}$, Sylvain Bertrand$^{1\,*}$, Duncan Calvert$^{1,2}$, Jerry Pratt$^{3}$, Robert Griffin$^{1,2}$ % <-this % stops a space
\thanks{$^{1}$Author is with the Institute of Human and Machine Cognition (IHMC),
        40 S Alcaniz St, Pensacola, FL 32502, USA
        {\tt\small author@ihmc.org}}% <-this % stops a space
\thanks{$^{2}$Author is with the University of West Florida (UWF),
        11000 University Pkwy, Pensacola, FL 32514, USA
        {\tt\small author@uwf.edu}}%
\thanks{$^{3}$ Author is with Figure AI, Inc., Sunnyvale, CA}%
\thanks{$^{*}$ The author initiated this project.}
\thanks{This work was supported through ONR Grant No. N00014-19-1-2023 and NASA Grant No. 80NSSC20M0197.}%
}

\begin{document}
\maketitle
\thispagestyle{empty}
\pagestyle{empty}

%%%%%%%%%%%%%%%%%%%%%%%%%%%%%%%%%%%%%%%%%%%%%%%%%%%%%%%%%%%%%%%%%%%%%%%%%%%%%%%%
\begin{abstract}

We present a virtual reality (VR) framework designed to intuitively generate humanoid multi-contact maneuvers for use in unstructured environments. Our framework allows the operator to directly manipulate the inverse kinematics objectives which parameterize a trajectory. Kinematic objectives consisting of spatial poses, center-of-mass position and joint positions are used in an optimization based inverse kinematics solver to compute whole-body configurations while enforcing static contact stability. Virtual ``anchors'' allow the operator to freely drag and constrain the robot as well as modify objective weights and constraint sets. The interface's design novelty is a generalized use of anchors which enables arbitrary posture and contact modes. The operator is aided by visual cues of actuation feasibility and tools for rapid anchor placement. We demonstrate our approach in simulation and hardware on a NASA Valkyrie humanoid, focusing on multi-contact trajectories which are challenging to generate autonomously or through alternative teleoperation approaches.

\end{abstract}
%%%%%%%%%%%%%%%%%%%%%%%%%%%%%%%%%%%%%%%%%%%%%%%%%%%%%%%%%%%%%%%%%%%%%%%%%%%%%%%%
% \begin{figure}[t]
%     % \begin{subfigure}{1\linewidth}
%     %   \centering
%     % %   \includegraphics[width=1\linewidth]{figs/fig_1_moti_textattn.pdf}  
%     % %   \includegraphics[width=1\linewidth]{figs/fig_1_moti_textattn_v2.pdf}  
%     %   \includegraphics[width=1\linewidth]{figs/fig_1_moti_textattn_v5.pdf}  
%     %   \vspace{-0.5cm}
%     %     \caption{Amount of attention added to each video clip from the source video and query text in the self-attention layers of Moment-DETR encoder.}
%     %     % \caption{Distribution of attention for source and query in Moment-DETR encoder}
%     %     % Visualization of video clip's self-attention score in Moment-DETR encoder.
%     %   \label{fig:fig1_text_attn_ex}
%     % \end{subfigure}%\hfill% or  or \hspace{0.3\textwidth}
%     \vspace{0.2cm}
%     % \begin{subfigure}{1\linewidth}
%       \centering
%     %   \includegraphics[width=1\linewidth]{figs/fig1_moti_negattn.pdf}  
%       \includegraphics[width=1\linewidth]{figs/fig1_moti_negattn_v3.pdf}  
%       \vspace{-0.4cm}
%     %   \caption{Correspondence of saliency scores on the relevance between video clips and the text query.}
%     % \caption{Predicted saliency scores against the video relevant positive query and video irrelevant negative query}
%       \label{fig:fig1_neg_attn_ex}
%     % \end{subfigure}%\hfill% or  or \hspace{0.3\textwidth}
%     \caption{
%     % 원준 원본
%     % (a) Comparison between attention scores of source and query for each video clip~(We sum the attention scores from video and text). 
%     % We observe that the attention scores are dominated by other clips in the source video. 
%     % Text queries do not account for much attention regardless of the relevance to the video clips.
%     % \textbf{(a)} Inspection of the query dependency in Moment-DETR encoder.
%     % % We visualize the attention score of video tokens in the transformer encoder and observe that text query accounts for only a low portion of attention.
%     % % This tendency occurs regardless of the relevance between the text query and video clips. 
%     % We visualize the attention score of video tokens in the transformer encoder and observe 1) text query only accounts for a low portion of attention, and 2) relevance between video-query pair does not affect the attention scores ratio of text.
%     \textbf{(b)} Comparison of highlight-ness when relevant and non-relevant queries are input.
%     As observed in , existing work only uses queries to play an insignificant role, thereby may not be capable of detecting false queries and considering the video-query relevance even when the problem in (a) is resolved. 
%     % \SE{} % 이 부분이 "not capable of" 란 용어가 세다는 피드백이 있는 듯 합니다. 이러한 능력이 없다는 것은 굉장히 강한 어조인거 같기는 하고, 이러한 경우들이 종종 있다거나 좀 약화시킬 필요가 있어보이긴 하네요.
%     On the other hand, our QD-DETR yields a query-dependent representation that the relevance between the source video and query text is updated in the saliency scores.
%     There is a large gap between positive and negative saliency scores, and scores are consistent since the clips are all highly correlated to others.
%     }
%     \label{fig:motivation_ex}
%     % \captionsetup{belowskip=13pt}
%     % \setlength{\belowcaptionskip}{-10pt}
% \end{figure}
\begin{figure}
    \centering
    \includegraphics[width=1\linewidth]{figs/fig1_moti_negattn_1111.pdf}
    % \includegraphics[width=1\linewidth]{figs/fig1_moti_negattn_1109.pdf}
    % \includegraphics[width=1\linewidth]{figs/fig1_moti_negattn_stat.pdf}
    \vspace{-0.6cm}
    \caption{
        % \SE{} % 수정 필요
        Comparison of highlight-ness~(saliency score) when relevant and non-relevant queries are given.
        We found that the existing work only uses queries to play an insignificant role, thereby may not be capable of detecting negative queries and video-query relevance; saliency scores for clips in ground-truth~(GT) moments are low and equivalent for positive and negative queries.
        % This also results in mispredicted moments when ground-truth~(GT) moment is dominated by clips unrelated to GT since their prediction is highly focused on the video.
        % \SE{} % 여기 한번 더 보면 좋을 듯 합니다. GT moment에 unrelated한 clip이 많으면? label이 틀렷을 경우를 말씀하시는건지?
        % As observed in saliency graph, existing work only uses queries to play an insignificant role, thereby may not be capable of detecting false queries and considering the video-query relevance.
        On the other hand, query-dependent representations of QD-DETR result in corresponding saliency scores to the video-query relevance and precisely localized moments.
        % On the other hand, our QD-DETR yields a query-dependent representation that the
        % saliency scores are in accordance with the relevance between the video and query.
        % text is in accordance with the saliency scores.
        % There is a large gap between positive and negative saliency scores, and scores are consistent since the clips are all highly correlated to others.
}
    \label{fig:motivation_ex}
\end{figure}


\section{Introduction}
% 원준 원본
% Along with the advance of digital devices and platforms, video is now one of the most desired data type for consumers. However, although the large information capacity of videos may be beneficial in many aspects, e.g., informative and entertaining, on the contrary perspective, videos are time-consuming, and hard to search for desirable moments. 
% This has led many creators to use extra manpower to crop and edit the video to generate highlight clips to gain the consumer’s attention.
Along with the advance of digital devices and platforms, video is now one of the most desired data types for consumers~\cite{apostolidis2021video,wu2017deep}.
% SE: Video aware deep learning application & survey papers?
Although the large information capacity of videos might be beneficial in many aspects, e.g., informative and entertaining, inspecting the videos is time-consuming, so that it is hard to capture the desired moments~\cite{anne2017localizing,apostolidis2021video}. 
% This has led many creators to use extra manpower to crop and edit the video to generate highlight clips to gain the consumer’s attention.


% On the other side, 
Indeed, the need to retrieve user-requested or highlight moments within videos is greatly raised.
Numerous research efforts were put into the search for the requested moments in the video~\cite{anne2017localizing, gao2017tall, liu2015multi, escorcia2019temporal} and summarizing the video highlights~\cite{zhang2016video, mahasseni2017unsupervised, badamdorj2022contrastive, wei2022learning}.
% Numerous research efforts were put into the search for the requested moments in the video~\cite{anne2017localizing, gao2017tall, liu2015multi, escorcia2019temporal}, summarizing the video to generate highlights was another popular topic~\cite{zhang2016video, mahasseni2017unsupervised, badamdorj2022contrastive, wei2022learning}.
Recently, Moment-DETR~\cite{momentdetr} further spotlighted the topic by proposing a QVHighlights dataset that enables the model to perform both tasks, retrieving the moments with their highlight-ness, simultaneously.

% 원준 원본
% To detect the desired moments, previous works employed transformer encoder-decoder architectural designs to fuse the text query into the video representations. Moment-DETR~\cite{mDETR} modified detection transformer to process capture the moment as a set, and UMT~\cite{umt} implemented transformer decoder as to output clip-wise saliency. 
% Yet to their outstanding breakthroughs in the literature of moment retrieval with the seminal architectures, their limitation is that the role of the given text query is insignificant in representing the query-conditioned video representation; the attention mechanism of moment DETR is not explicitly conditioned on the text query, and the text query is conditioned on multi-modal clips where the differences between the clips are smoothed after encoding process in UMT.



% \begin{figure}[t]
% \centering
%     \begin{subfigure}[l]{0.37\linewidth}
%       \centering
%       \vspace{0.20cm}
%     %   \includegraphics[width=1\linewidth]{figs/fig_1_moti_textattn.pdf}  
%     %   \includegraphics[width=1\linewidth]{figs/fig_1_moti_textattn_v2.pdf}  
%       \includegraphics[width=1\linewidth]{figs/fig1_moti_violin_a.pdf}  
%       \vspace{-0.60cm}
%     %   \caption{text attention}
%         \caption{Importance of queries in video representation}
%       \label{fig:fig1_text_attn}
%     \end{subfigure}%\hfill% or  or \hspace{0.3\textwidth}
%     \vspace{0.2cm}
%     \begin{subfigure}[r]{0.61\linewidth}
%       \centering
%     %   \includegraphics[width=1\linewidth]{figs/fig1_moti_negattn.pdf}  
%       \includegraphics[width=1\linewidth]{figs/fig1_moti_violin_b.pdf}  
%     %   \caption{neg attention}
%         % \caption{Relation between the highlight-ness and the relevance between videos and query texts.}
%         \caption{Highlight-ness~(saliency) histogram of positive and negative video-query pairs\SE{}}
%       \label{fig:fig1_neg_attn}
%     \end{subfigure}%\hfill% or  or \hspace{0.3\textwidth}
%     % \vspace{-0.2cm}
%     \caption{Overall statistics for attention scores in Fig.~\ref{fig:motivation_ex} in QVHighlights dataset. 
%     (a) For the attention scores that measure how much the text query is generally involved in video representation, we use violin plots to show the probability density. We plot the score for each layer in the encoder.
%     % (b) Using the histogram, we compare how the baseline and QD-DETR yield different salient scores given the positive and negative video-text pairs.
%     (b) Saliency histogram shows the distributional gap between positive and negative video-text query pairs of baseline~(Moment-DETR) and proposed QD-DETR.\SE{}
%     }
%     \label{fig:motivation}
%     % \captionsetup{belowskip=13pt}
%     % \setlength{\belowcaptionskip}{-10pt}
% \end{figure}

% \begin{figure}[t]
% \centering

%     \begin{subfigure}[r]{1\linewidth}
%       \centering
%       \hspace{-0.2cm}
%     %   \includegraphics[width=1\linewidth]{figs/fig1_moti_negattn.pdf}  
%       \includegraphics[width=1.1\linewidth]{figs/fig1_moti_violin_a_v2.pdf}  
%     %   \caption{neg attention}
%         % \caption{Relation between the highlight-ness and the relevance between videos and query texts.}
%         \vspace{-0.5cm}
%         % \caption{Saliency histogram of positive and negative video-query pairs}
%         \caption{We plot the histograms and its average value~(dotted line) to compare saliency scores when true and false text queries are given for each method. (left) Since the video representations do not include much textual information, both the true and false queries yield similar saliency scores. (Middle) Even when the video representation is enforced to be updated with the textual information, the issue is not much resolved. (Right) By extracting discriminative features in the text query, distributions are differentiated.
%         % \SE{} % R1@0.5 설명
%         Also, R1@0.5 indicates evaluation metric, Recall at 1 with IoU 0.5 threshold on QVhighlight \textit{val} set.
%         }
%       \label{fig:fig1_neg_attn}
%     \end{subfigure}%\hfill% or  or \hspace{0.3\textwidth}
%     \\
%     \begin{tabular}{cc}
%     \hspace{-0.2cm}
%         \begin{minipage}{.4\linewidth}
%             \begin{subfigure}[l]{1\linewidth}
%               \centering
%             %   \vspace{0.20cm}
%             %   \includegraphics[width=1\linewidth]{figs/fig_1_moti_textattn.pdf}  
%             %   \includegraphics[width=1\linewidth]{figs/fig_1_moti_textattn_v2.pdf}  
%               \includegraphics[width=1\linewidth]{figs/fig1_moti_violin_a.pdf}  
%               \vspace{-0.60cm}
%             %   \caption{text attention}
%                 \caption{Importance of queries in video representation}
%               \label{fig:fig1_text_attn}
%             \end{subfigure}%\hfill% or  or \hspace{0.3\textwidth}
%         \end{minipage}
        
%         \begin{minipage}{.6\linewidth}
%             \vspace{-0.2cm}
%             \caption{Overall statistics of Fig.~\ref{fig:motivation_ex} in QVHighlights dataset. 
%             (a) Saliency histogram shows the distributional gap between positive and negative video-text query pairs.
%             % (a) For the attention scores that measure how much the text query is generally involved in video representation, we use violin plots to show the probability density. We plot the score for each layer in the encoder.
%             % (b) Using the histogram, we compare how the baseline and QD-DETR yield different salient scores given the positive and negative video-text pairs.
%             % (b) Text ratio in self-attention layer to  of Moment-DETR
%             % (b) Ratio of text when representing video tokens in self-attention of Moment-DETR.
%             % (b) Magnitude of attention text query involved.
%             % (b) Attention score of video tokens
%             % (b) Magnitude of text query to refine the video tokens in self-attention layer of Moment-DETR.
%             (b) Probability density depicting the weight of the text query in attention score for video clips. Scores are from the self-attention layers in Moment-DETR encoder.
%             % (b) The text query ratio in attention score of video clips (Self-attention layer in Moment-DETR encoder). We use violin plots to show probability density.
%             % 텍스트 쿼리가, 비디오 피쳐에 얼만큼 attend 하는지
%             }
%         \end{minipage}
    
%     \end{tabular}
%     \vspace{-0.5cm}
%     \label{fig:moti}
%     % \captionsetup{belowskip=13pt}
%     % \setlength{\belowcaptionskip}{-10pt}
% \end{figure}


% \begin{figure}
%     \centering
%     % \includegraphics[width=1\linewidth]{figs/fig1_moti_negattn_1109.pdf}
%     \includegraphics[width=1\linewidth]{figs/fig1_moti_negattn_stat_v2.pdf}
%     \vspace{-0.8cm}
%     \caption{
%         Histogram of saliency when the positive and negative queries are given. We plot the histograms and its average value~(dotted line) to compare saliency scores when relevant~(positive) and irrelevant~(negative) text queries are given for each method. (Left) Since the video representations do not properly reflect textual information, both the positive and negative queries yield similar saliency scores. 
%         % (Middle) Even when the video representation is enforced to be updated with the textual information, the issue is not much resolved. 
%         (Right) By representing video clips in query-dependent manner, distributions are differentiated.
%     }
%     \vspace{-0.6cm}
%     \label{fig:motivation}
% \end{figure}


% One of the demanding task is moment retrieval task, which is detecting the desired moments from the given query, typically the text query.
When describing the moment, one of the most favored types of query is the natural language sentence~(text)\cite{anne2017localizing}. 
While early methods utilized convolution networks~\cite{zhang2020learning, gao2021fast, wang2020temporally}, recent approaches have shown that deploying the attention mechanism of transformer architecture is more effective to fuse the text query into the video representation.
% To handle these modalities, previous works simply employed the attention mechanism of transformer architecture to fuse the text query into the video representation.
For example, Moment-DETR~\cite{momentdetr} introduced the transformer architecture which processes both text and video tokens as input by modifying the detection transformer~(DETR), and UMT~\cite{umt} proposed transformer architectures to take multi-modal sources, e.g., video and audio. 
Also, they utilized the text queries in the transformer decoder.
Although they brought breakthroughs in the field of MR/HD with seminal architectures, they overlooked the role of the text query.
To validate our claim, we investigate the Moment-DETR~\cite{momentdetr} in terms of the impact of text query in MR/HD~(Fig.\ref{fig:motivation_ex}).
Given the video clips with a relevant positive query and an irrelevant negative query, we observe that the baseline often neglects the given text query when estimating the query-relevance scores, i.e., saliency scores, for each video clip.
% the output saliency score, i.e. query-relevance scores.
% Based on the observation, we traced the actual saliency prediction of the model against both the video-relevant query and the irrelevant dummy one where we find that the baseline often neglects the given text query when estimating the query-relevance scores of video clips.
% For example, in Fig.~\ref{fig:motivation_ex}, saliency scores are not affected even when the query is substituted with the dummy.
% % General statistics for Fig.~\ref{fig:motivation_ex} is shown in Fig.~\ref{fig:motivation}. 
% General statistics corresponding to Fig.~\ref{fig:motivation_ex} are also shown in Fig.~\ref{fig:motivation}.



% The limitation of the concrete baseline~\cite{momentdetr} is inspected in two different aspects; 1) Utilization of text-query in the encoding process and 2) the output saliency score, i.e. query-relevance scores.
% Firstly, we visualize the attention score when video clips are given as a query in self-attention. 
% We observe that the text queries have relatively small impacts compared to other video features, as shown in Fig.~\ref{fig:fig1_text_attn_ex}.
% That is, the text does not account for much in representing every video clip, although the goal of MR/HD is to detect query-relevant moments.
% Based on the observation, we traced the actual saliency prediction of the model against both the video-relevant query and the irrelevant dummy one where we find that the baseline often neglects the given text query when estimating the query-relevance scores of video clips.
% For example, in Fig.~\ref{fig:motivation_ex}, saliency scores are not affected even when the query is substituted with the dummy.
% % General statistics for Fig.~\ref{fig:motivation_ex} is shown in Fig.~\ref{fig:motivation}. 
% General statistics are also shown in Fig.~\ref{fig:motivation}.

% Consequently, in Fig.~\ref{fig:fig1_neg_attn_ex}~(b), we found that the baseline often neglects the given text query when estimating the query-relevance scores of video clips; 
% For example, 


% We validate the previous work sometimes neglects the given query when estimating the saliency of video clips.
% For example, there is an example that the saliency scores from positive and negative queries cannot be distinguishable, as shown in Fig.~\ref{fig:fig1_neg_attn_ex}.
% % 우리는 추가로 text attention을 추가도 해봤지만, 효과가 있긴 했으나, still 이슈가 있는 것을 확인하였다?
% % Still, we observe that assuring the high attendance of text queries does not resolve the overlap which motivates us to question the quality of the naive use of task-agnostic text representation~\cite{momentdetr, umt}.
% We found that introducing the text-attention for ensuring the high attendance of text queries relieve the overlap, but there still be a severe overlap.


% To validate their limitations, we inspect the impacts of text queries in the concrete baseline~\cite{momentdetr} with the two different aspects, 1) tendency of attention in self-attention layer and 2) saliency score, i.e. query-relevance scores. \SE{} % attention 이 갑자기 등장하는가?
% Firstly, we visualize the attention score when video clips are given as a query in self-attention. We observe the text queries have relatively low attention scores compared to the video features, as shown in Fig.~\ref{fig:fig1_text_attn_ex}.
% That is, the text does not account for much in representing every video clip, although the goal of MR/HD is to detect query-relevant moments.
% Based on this observation, we trace the actual saliency prediction of the model against both positive and negative text queries.
% We validate the previous work sometimes neglects the given query when estimating the saliency of video clips.
% For example, there is an example that the saliency scores from positive and negative queries cannot be distinguishable, as shown in Fig.~\ref{fig:fig1_neg_attn_ex}.
% % 우리는 추가로 text attention을 추가도 해봤지만, 효과가 있긴 했으나, still 이슈가 있는 것을 확인하였다?
% % Still, we observe that assuring the high attendance of text queries does not resolve the overlap which motivates us to question the quality of the naive use of task-agnostic text representation~\cite{momentdetr, umt}.
% We found that introducing the text-attention for ensuring the high attendance of text queries relieve the overlap, but there still be a severe overlap.



% Thus, we 
% query dependency를 높이기 위해 
% Cross-attention? text-attention? detailed explanation on text-attention should be needed?
% By handling these two issues, we find that more precise retrieval can be achieved.
% 
% 
%
% By projecting video-discriminative text features with high text attendance to source video, we f 
% We also find the need to improve the quality of query features since assuring high text attendance also results in...
% pairs are not finetuned to be discriminative that even the similarity within the pairs does not reflect the relevance between the query and the video clips.
% General statistics for Fig.~\ref{fig:motivation_ex} is shown in Fig.~\ref{fig:motivation}. 
% \SE{} % 이거 ??로 뜨는데, 위처럼 figure 그리면 label이 안되는걸까요
% \SE{}
% 형님 아래 사항 생각 좀 해보는게 좋을 거 같아요.
% fig 1. (a) 그림만 봤을 때 모든 clip에 대해 text attention이 일정이상 존재하긴 하니까, 뭔가 not assured to be conditioned가 와닿지 않는거 같아요.
% + 왜 text가 항상 attend 해야하나?
% not assured to be conditioned --> text shows relatively low affects compared to video 같이 실제 나타난 현상까지 같이 적으면 어떨까 싶어요.
% fig 1. (b) 덜 반영한다?

% \SU{}
% 일단 text가 attend 잘 되어야 한다는 것에 좀 궁금점이 생깁니다. 결국에는 text와 관련있는 frame들을 attend해서 higlight를 찾아야 하는게 아닐까요? 그리고, 현제 저희의 모델 구조상 text query가 Key와 Value로 거의 활용되고 있는데 그렇다면 결국에는 해당 모델은 text에 대한 attention이 전혀 없다고 봐도 무방하지 않을까요? 그런 면에서 text attention을 강조하는게 좀 걸리긴 합니다.

% Specifically, the text query is not assured to be explicitly conditioned on every clip of the video, and as the query texts are evenly treated, discriminative keywords may not be spotlighted.
% attention mechanism of Moment-DETR is not explicitly conditioned on the text query as shown in Fig~\ref{}(d), and in UMT, the text are only used for conditioning the queries while the video representation are refined itself by self-attention.

% \begin{figure}[t]
%     \begin{subfigure}{1\linewidth}
%       \centering
%     %   \includegraphics[width=1\linewidth]{figs/fig_1_moti_textattn.pdf}  
%     %   \includegraphics[width=1\linewidth]{figs/fig_1_moti_textattn_v2.pdf}  
%       \includegraphics[width=1\linewidth]{figs/fig_1_moti_textattn_v4.pdf}  
%       \vspace{-0.5cm}
%     %   \caption{text attention}
%         \caption{Distribution of attention scores in Moment-DETR encoder}
%       \label{fig:fig1_text_attn}
%     \end{subfigure}%\hfill% or  or \hspace{0.3\textwidth}
%     \vspace{0.2cm}
%     \begin{subfigure}{1\linewidth}
%       \centering
%     %   \includegraphics[width=1\linewidth]{figs/fig1_moti_negattn.pdf}  
%       \includegraphics[width=1\linewidth]{figs/fig1_moti_negattn_v2.pdf}  
%       \vspace{-0.5cm}
%     %   \caption{neg attention}
%         \caption{Saliency score against positive and negative text queries}
%       \label{fig:fig1_neg_attn}
%     \end{subfigure}%\hfill% or  or \hspace{0.3\textwidth}
%     \vspace{0.2cm}
%     \begin{subfigure}{1\linewidth}
%       \centering
%     %   \includegraphics[width=1\linewidth]{figs/fig1_moti_violin.pdf}  
%       \includegraphics[width=1\linewidth]{figs/fig1_moti_violin_v2.pdf}  
%       \vspace{-0.5cm}
%       \caption{violin}
%       \label{fig:fig1_violin}
%     \end{subfigure}%\hfill% or  or \hspace{0.3\textwidth}
%     \vspace{-0.2cm}
%     \caption{(a) 1. portion of text attention vs. video attention 2. relation with text query and content (e.g. fg, bg) of clip seems not to affect the attention score
%     (b) 1. high variability even though entire clips are highly correlated with the given text query 2. positive and negative query makes overlaps on saliency score distribution
%     (3) actual distribution on validation dataset.}
%     \label{fig:motivation}
%     % \captionsetup{belowskip=13pt}
%     % \setlength{\belowcaptionskip}{-10pt}
% \end{figure}

To this end, we propose Query-Dependent DETR~(QD-DETR) that produces query-dependent video representation.
% Our key focus is to ensure each clip in predicted moments is explicitly conditioned by the query, particularly on the video-descriptive portion of the text query.
% Our key focus is to ensure that query-relevant clips are predicted by enforcing each clip to be explicitly conditioned by the query.
%Our key focus is to ensure that the model prediction for each clip is highly relevant to the query.
Our key focus is to ensure that the model's prediction for each clip is highly dependent on the query.
% by enforcing each clip to be explicitly conditioned by the query. :)
% hmm...
% \SE {} % "query-relevant clips are predicted" 이 문장이 좀 애매한거 같습니다. relevant 클립을 놓지지 않고 찾는 것을 보장한다? 이런 느낌인지 아니면 높은 saliency 를 주는게 목적이다? model prediction이 query-relevance를 반영하는 것을 보장한다?
% Our key focus is to ensure that the model prediction reflects query-relevance of clips by enforcing each clip to be explicitly conditioned by the query.
First, to fully utilize the contextual information in the query, we revise the transformer encoder to be equipped with cross-attention layers at the very first layers.
% 상익's thought :  single video - query간의 관계만 고려 - 같은 word가 더 많이 쓰이는 것을 보고 
% 교수님's thought : neg pair 를 쓰면 쿼리를 보지 않고서는 video clip간만 고려하는 것이 사라짐. 왜냐면 0으로 내보내야 하기 때문. --> SE: relative difference 만 고려하다가, 
By inserting a video as the query and a text as the key and value of the cross-attention layers, our encoder enforces the engagement of the text query in extracting video representation.
% 원준 교수님 코멘트 반영해서 다시
Then, in order to not only inject a lot of textual information into the video feature but also make it fully exploited, we leverage the negative video-query pairs generated by mixing the original pairs.
Specifically, the model is learned to suppress the saliency scores of such  negative~(irrelevant) pairs.
Our expectation is the increased contribution of the text query in prediction since the videos will be sometimes required to yield high saliency scores and sometimes low ones depending on whether the text query is relevant or not.
% \SE{}
% learns to?
% By suppressing the saliency scores of the irrelevant video-query pairs, the model learns to spotlight only the video-specific discriminative words in the query.
% % \SE{} % ====================== 상익 수정 ========================
% However, this architectural design still lacks the capability of identifying the video-descriptive keywords in the query.
% % However, this architectural design still lacks in identifying proper query relevance.
% This is because the current training scheme only focuses on the interactions of video and clips within a single video while neglecting information shared throughout the entire video.
% % We argue the problem of the current training scheme that only focuses on distinguishing the clips in a single video while neglecting information shared throughout the entire video.
% Therefore, we leverage the negative video-query relationships to enhance the capability of identifying the contextual similarity of query and video clips.
% 
% 원준 원본 
% However, this architectural design heavily relies on the quality of the text query.
% Therefore, we leverage the negative video-query relationships to enable the model to emphasize key corresponding query features.
% By suppressing the saliency scores of the irrelevant video-query pairs, the model learns to spotlight only the video-specific discriminative words in the query.
% =========================================================
Lastly, to apply the dynamic criterion to mark highlights for each instance, we deploy a saliency token to represent the entire video and utilize it as an input-adaptive saliency criterion. 
With all components combined, our QD-DETR produces query-dependent video representation by integrating source and query modalities.
This further allows the use of positional queries~\cite{dabdetr} in the transformer decoder.
% Furthermore, we can exploit the advanced DETR decoder architectures using the positional information, e.g., DAB-DETR, since our encoded tokens consist of identical position representations from a single modality.
% \SE{} % ====================== 상익 수정 ========================
% Furthermore, we can exploit the advanced DETR decoder architectures using the positional information, e.g., DAB-DETR, since our video clip tokens consist of identical position representations from a single modality.
% 원준 원본
% It also enables the use of advanced DETR decoder architectures, e.g., DAB-DETR, for the first time, as these works exploit the position information within a single modality.
% =========================================================
Overall, our superior performances over the existing approaches validate the significance of the role of text query for MR/HD.
% Our extensive experiments on QVHighlights, TVSum, and Charades-STA datasets validate the significance of considering the role and the quality of text query.

% All components combined with dynamic anchor moments for the query of decoder, our FOQUE fosters the query-dependent video representation, thereby making the 
% All components combined, our modified transformer encoding process fosters the query-dependent video representation thereby achieving the state-of-the-art results on various benchmarks of moment-retrieval and highlight detection.
	
% -	Video Platform & Streamer & Consumer의 증가. 
% Video는 다른 데이터 타입보다 정보가 많아 유용하지만, 이는 다른 말로 해석하면 video를 보는 것은 time-consuming 하고, 원하는 것을 찾아보기에는 힘들 수 있음.
% 따라서, 많은 매체에서는 사람들의 더 많은 이목을 끌기 위해 highlight 비디오라는 것을 편집하여 공유도 함.
% 하지만, highlight video를 만들기 위해 사람의 노력이 필요한 현 시점에서, This spotlights the need to retrieve the user-requested / Highlight moments in the video.

% -	이전에도 이러한 문제를 해결하기 위해 (asdfasdf) for moment retrieval, (asdfasdf) for highlight detection 등이 제안 되었지만, 이들은 비디오의 특정 영역을 찾는다는 공통된 목적을 가지고 있으면서도, 데이터 셋의 한계로 인해 따로 연구되었음. 이를 문제 삼으며, 최근에는 두 task를 동시에 학습할 수 있는 dataset이 소개 되었는데, 컴퓨터비전에서 최근 각광을 받고 있는 Transformer 모델 도입과 함께 큰 발전을 거듭하고 있음.

% -	구체적으로, 이 두가지 task를 수행하기 위해서는 transformer를 두가지 방법으로 이용할 수 있는데, moment-DETR 처럼 moment 를 clip의 set 단위로 예측할 수 있고, UMT 처럼 clip-wise prediction을 할 수 있음. 하지만, 이들은 query를 condition이 아닌 video와 동등한 레벨로 취급하거나 [mDETR], 매 클립이 self-attention으로 mixing 된 후에 condition을 걸어주어 clip간의 차이를 확실하지 이용하지 못하였고, 또한, 확실하게 condition으로 주지 못하였고, video와 query 사이의 관계를 한정적으로만 이용하였다.

% -	따라서, we explore three different ways to fully exploit query information. First, we design one-way cross-attention layer to condition every clip with the query features. Then, we utilized the negative video-text pairs to better model the relationships between the video and the text embeddings. Lastly, we define the saliency token to be the video-query dependent saliency estimator.


















% ===================== neg pair 부분 ===========================
% Nevertheless, the current training scheme, only considering the given video-query pair, still disturbs the model from identifying proper query-relevance prediction.
% In detail, the model focus on learning the fine-grained discrepancy between video clips, while neglecting the information they share, which contains significant clues to understand the context of video.
% Therefore, we leverage the negative video-query relationships to enhance the capability of identifying the contextual similarity of query and video clips.
% Therefore, we leverage the negative video-query relationships by suppressing those pairs, so that enhance the capability of identifying the contextual similarity of query and video clips.
% We hypothsize the diversity in query-video pairs are insufficient to learn the general relationship between text query and video.
% Therefore, we leverage the negative video-query relationships by suppressing the saliency scores of the irrelevant video-query pairs.
% However, this architectural design still lacks in identifying proper query relevance.
% We argue that the current training scheme only focuses on learning the fine-grained discrepancy between clips in a single video, while neglecting the information they share, which contains significant clues to understand the context of the video.
% Therefore, we leverage the negative video-query relationships to enhance the capability of identifying the contextual similarity of query and video clips.
% However, this architectural design still lacks in identifying proper query relevance.
% We argue the problem of the current training scheme that only focuses on learning the fine-grained discrepancy between clips in a single video.
% That is, the current design neglects the information shared throughout the video, although it contains significant clues to understand the context of the video.
\section{Related Work}
\label{sec:related_work}
\subsection{Co-Speech Gesture Synthesis}
The early approaches for generating co-speech gestures often involve creating linguistic rules to translate speech input into a sequence of pre-collected gesture segments, which are typically referred to as rule-based methods \cite{cassell1994rulefullbody,cassell2001beat,kipp2004gesture,kopp2006bml}. \citet{wagner2014rulereview} provide a comprehensive review of these methods. Rule-based methods produce interpretable and controllable results, but creating gesture datasets and rules requires significant effort. To alleviate the manual effort of designing rules in rule-based methods, data-driven approaches have gradually become predominant in this field. \citet{nyatsanga2023data_driven_gesture_survey} offer a thorough survey of these methods. Early data-driven approaches aim to directly learn mapping rules from data through statistical models \cite{neff2008videogesture,levine2009prosodygesture,levine2010gesturecontroller} and combine them with predefined gesture units for gesture generation. Later, the powerful modeling capability of deep neural networks makes it possible to train complex end-to-end models using raw speech-gesture data directly. One option is deterministic models, such as MLP \cite{kucherenko2020gesticulator}, CNN \cite{habibie2021videogesture}, RNN \cite{yoon2019robot,yoon2020trimodalgesture,bhattacharya2021affectivegesture,liu2022hierarchicalgesture}, and Transformer \cite{bhattacharya2021text2gestures}. Another choice is generative models, including flow-based models \cite{alexanderson2020stylegesture,ye2022styleflowgesture}, VAEs \cite{li2021audio2gesture,ghorbani2022zeroeggs}, and VQ-VAE \cite{yi2022talkshow,yazdian2022gesture2vec,liu2022vqgesturevideo}. Due to the inherent many-to-many relationship between speech and gesture, end-to-end models can generate natural-looking gestures but face challenges in ensuring content matching between speech and generated gestures \cite{yoon2022genea}. To address this issue, some neural systems aim to explicitly model both rhythm and semantics from the perspective of model structure \cite{kucherenko2021speech2properties2gestures,ao2022rhythmicgesticulator,liu2022disco} or training supervision strategy \cite{liang2022seeg}. Furthermore, hybrid systems, such as the combination of deep features and motion graphs \cite{zhou2022gesturemaster}, have been proposed to harness the advantages of different approaches. Recently, diffusion models \cite{sohldickstein2015diffusion,song2020improvedscore,ho2020ddpm} have demonstrated impressive results in image synthesis \cite{ramesh2022dalle2} and human motion generation \cite{tevet2022humanmotiondiffusion, zhang2022motiondiffuse}. Inspired by these works, our system adapts the latent diffusion model \cite{rombach2022latentdiffusion} for the co-speech gesture generation task and achieves appealing results.

\subsection{Style Control for Human Motion}
A typical approach to style control for human motion involves specifying a motion clip as a reference and transferring the reference clip's style to the source motion. This task is also known as \emph{style transfer}. Early works in motion style transfer integrate traditional machine learning techniques with manually defined features to infer motion styles \cite{hsu2005motion_style_translation,ma2010motion_style_transfer,xia2015realtime_motion_style_transfer,yumer2016spectral_motion_style_transfer}. Recently, deep learning-based methods have significantly enhanced motion quality. \citet{holden2016deepmotion} first propose a learning framework enabling motion style control through optimization in the motion manifold space. \citet{du2019stylemotioncvae} improve transfer efficiency by training a conditional VAE. \citet{mason2018few-shot_motion_style_transfer} use few-shot learning to generate stylized locomotion. \citet{aberman2020adain} employ a temporally invariant adaptive instance normalization (AdaIN) layer for target style injection, eliminating the need for paired data during training. \citet{wen2021stylemotionflow} achieve unsupervised style transfer using a flow model. \citet{jang2022motionpuzzle} introduce a method capable of controlling styles for individual body parts.

Previous co-speech gesture synthesis systems with style control can be categorized based on whether or not they require style labels. For methods needing labeled data, early works can only learn an individual style for one generator \cite{levine2010gesturecontroller,neff2008videogesture,ginosar2019stylegesture}. \citet{ahuja2022lowresource} propose a strategy that efficiently adapts the source generator to another speaker style using low-resource data. Some works learn a speaker style embedding space with labeled speaker-motion data, enabling gesture style control by sampling from this space \cite{ahuja2020stylegesture,yoon2020trimodalgesture,bhattacharya2021affectivegesture}. \citet{alexanderson2020stylegesture} aimat controlling fine-grained styles, such as gesturing speed and spatial scope, using preprocessed control signal-motion data. Their later work \cite{alexanderson2022diffusiongesture} utilizes a diffusion model for audio-driven motion synthesis, achieving label-based style control by training the model on labeled data. For methods not requiring style labels, \citet{habibie2022motionmatching} propose a motion matching framework to achieve flexible style control. Other studies achieve arbitrary style control by imitating an example given as a video \cite{liu2022hierarchicalgesture} or a motion clip \cite{ghorbani2022zeroeggs,ye2022styleflowgesture,kuriyama2022tokenizedgestures}.  In this work, we utilize a CLIP-based encoder to extract a style embedding from an arbitrary text prompt and incorporate it into the generator via an AdaIN layer, guiding the synthesis of stylized gestures. Our system supports fine-grained multimodal style prompts as opposed to label-based style control. It employs a self-supervised learning scheme and eliminates the need for labeled data. Additionally, we use an autoregressive model rather than a parallel model, making it potentially suitable for real-time applications.
\section{ARCHITECTURE}

\begin{figure}
    \centering
    \includegraphics[width=0.92\columnwidth]{Figures/Pipeline.png}
    \caption{High-level architecture of the motion generation framework. Solid arrows designate ROS 2 messages which are sent between modules. Hollow arrows designate saving and loading to file. Blue dashed lines represent a message-level layer of abstraction.}
    \label{fig:script_pipeline}
\end{figure}

The diagram in Fig. \ref{fig:script_pipeline} shows the control flow of our framework. At a high level, the operator creates whole-body trajectories by incrementally generating statically stable, collision-free key frames. These key frames can either be created offline using a simulation or online with hardware and are logged for future editing and reuse. At a software level, there are three processes involved in generating trajectories. The first is the user-interface (UI) which enables the operator to specify kinematic objectives and validate key frames. Both a VR app in Java Monkey Engine \cite{JME} and a desktop app in JavaFX \cite{JavaFX} are implemented with similar functionality. In practice, the VR UI is preferred for generating trajectories from scratch and the desktop for editing existing trajectories. The VR app is developed for the Valve Index \cite{Index} and also has bindings for the HTC Vive \cite{Vive}. The second process is an optimization-based inverse kinematics solver (Sec. \ref{sec:ik}) which solves for whole-body configurations given a set of kinematic objectives. The third process is a robot controller which executes the commanded motion in either a physics simulator or on hardware. Both the UI and robot controller have message-layer abstraction to allow any permutation of active modules (Fig. \ref{fig:script_pipeline}). All inter-process communication is done through ROS 2 messages \cite{ros2}.

% The architecture of the framework is shown in Fig. 2
% It consists of four main components. 
%   - The first is an optimization-based inverse kinematics solver. This solver accepts kinematic tasks and outputs whole-body configurations, next section has deets
%   - The second component is a GUI for the operator to interface with the kinematics solver. Both a VR and desktop app are implemented with identical functionality. The idea is to create trajectories from scratch in VR and have the desktop app for offline editing. 
%   - The third component represents the robot's onboard controller, which executes motions commanded by the operator
%   - The fourth is the sequence of keyframes being assembled

% We develop trajectories by prototyping in simulation, assembling a library of trajectories which can be dispatched online.

% Messaging is all done in ROS 2. 
\section{Kinematics Solver} \label{sec:ik}

An optimization-based IK solver is used to compute quasi-statically stable whole-body configurations given a set of kinematic tasks. We solve the IK problem using Sequential Quadratic Programming (SQP) due to its generality and success on humanoids \cite{brossette2018multicontact, rouxel2022multicontact, beeson2015trac}. At every solve step, a desired velocity $\mathbf{v}_d\in\mathbb{R}^{n+6}$ is computed to drive the model towards a configuration that achieves the desired task objectives, where $n$ is the number of actuated degrees of freedom in the robot. Note that $\mathbf{v}_d$ represents the velocity of the solver model and is independent of the controller's velocity. The IK iteratively solves the following Quadratic Program (QP): \begin{equation}
\begin{aligned}
 \label{eq:IKQP}
    \min_{\mathbf{v}_d} \quad   & c_{\mathrm{nom}} + c_{\mathbf{J}} + c_{\mathbf{v}_d}    \\
    \textrm{s.t.} \quad         & \mathbf{v}_{min} \leq \mathbf{v}_d \leq \mathbf{v}_{max}      % \\
    % \quad                       & \mathbf{C}\mathbf{v}_d \leq \mathbf{d}
\end{aligned}
\end{equation}

The objective function terms are given by: %\vspace{1mm}

\renewcommand{\arraystretch}{1.35}
\setlength{\tabcolsep}{2.5pt}

\begin{tabular}{ll}
 Nominal Objective: & $         c_{\mathrm{nom}}    = (\mathbf{v}_d - \mathbf{v}_{\mathrm{nom}})^T  \mathbf{C}_{\mathrm{nom}}    (\mathbf{v}_d - \mathbf{v}_{\mathrm{nom}}) $ \\
 Kinematic Tasks: & $         c_{\mathbf{J}}      = (\mathbf{J}\mathbf{v}_d - \mathbf{p})^T       \mathbf{C}_{\mathbf{J}}      (\mathbf{J}\mathbf{v}_d - \mathbf{p}) $ \\
 Velocity Cost: & $   c_{\mathbf{v}_d}      = \mathbf{v}_d^T \mathbf{C}_{\mathbf{v}_d} \mathbf{v}_d$,
\end{tabular}

where the terms are given by:

\begin{itemize}
    \item $\mathbf{v}_{\mathrm{nom}}$ drives the robot to a nominal whole-body configuration, which by default is the controller's current configuration. The user can set the nominal configuration as IK's current solution as by selecting ``snap ghost'' (Fig. \ref{fig:teleop_combo}(d)), which is generally used for larger motions where the controller and IK differ significantly.
    \item $\mathbf{J} = [\mathbf{J}^T_1 \ldots \mathbf{J}^T_k]^T$ and $\mathbf{p} = [\mathbf{p}^T_1 \ldots \mathbf{p}^T_k]^T$ are the stacked Jacobian matrices and motion objectives computed as feedback terms from the kinematic tasks and $\mathbf{C_J} = \mathrm{diag}(\mathbf{w}_0, \ldots, \mathbf{w}_k)$ is a block-diagonal weight matrix, which is detailed below.
    \item $\mathbf{v}_{min}$ and $\mathbf{v}_{max}$ bound the joint velocity so the joint remains within its bounds within the update period $\Delta T$.
    % \item $\mathbf{A}$ is the linear centroidal momentum matrix \cite{orin2008centroidal} and $\mathbf{h}=\mathbf{A}\mathbf{v}_d$ is the linear momentum. $\mathbf{H}$ constrains the centroidal momentum to such that the CoM remains inside the support region.
    % \item $\mathbf{C}, \mathbf{d}$ constrain the CoM to remain in the support region during the solver tick, detailed in Sec. \ref{sec:com_constraint}.  
    \item $\mathbf{C}_{\mathrm{nom}} = 0.5\,\mathbf{I}_{n+6}$ and $\mathbf{C}_{\mathbf{v}_d} = 0.1\,\mathbf{I}_{n+6}$ are constant weight matrices.
\end{itemize}

With each solve iteration, the candidate keyframe configuration $\mathbf{q}_d$ is updated by integrating the computed velocities (Eq. \ref{ikupdate}), where $\Delta T$ is the solver update period. \begin{equation} \label{ikupdate}
    \mathbf{q}_d \leftarrow \mathbf{q}_d + \mathbf{v}_d \Delta T
\end{equation}

For kinematic task $i$, a task Jacobian $\mathbf{J}_i$ and feedback motion $\mathbf{p}_i$ are used to compute a feedback term:

\begin{itemize}
    \item Joint Position: $\mathbf{J}_i$ is a selection matrix for the joint and $\mathbf{p}_i$ is a velocity proportional to the position error.
    \item Center of Mass: $\mathbf{J}_i$ is the linear centroidal momentum matrix $\mathbf{A}$ \cite{orin2008centroidal} and the objective $\mathbf{p}_i$ is a momentum proportional to the linear position error.
    \item Taskspace Posture and Contact Point: $\mathbf{J}_i$ is the geometric Jacobian \cite{spong2020robot} of the control frame $F_p$ (Sec. \ref{sec:task_generation}). A proportional feedback law on the relative transform between $F_p$ and $F_d$ is used to compute $\mathbf{p}_i$ \cite{bullo1995proportional}.
\end{itemize}

Kinematic tasks are each assigned a weight matrix $\mathbf{w}_i = w_i \mathbf{I}_{N_i}$, where $w_i$ is computed from the task's priority given by Tab. \ref{tab:weights} and $N_i$ is the tasks's dimensionality. Note that the CoM task weight is scaled down by the robot mass $m$ so feedback is only dependent on kinematic quantities. For Taskspace Posture, Center of Mass and Contact Points, a selection matrix $\mathbf{S}_i$ is used to only provide feedback for the constrained axes. This is done by premultiplying both the Jacobian and objective by $\mathbf{S}_i$.

\renewcommand{\arraystretch}{1.2}
\begin{table}[t!]
\centering
\captionof{table}{Kinematic Task Weights} \label{tab:weights} 
\begin{tabular}{ c c c c } 
 \hline
 & Soft & Mid & Hard \\ 
 \hline
Taskspace Posture & 0.1 & 1.0 & 10.0 \\ 
CoM & $0.01/m$ & $0.1/m$ & $1.0/m$ \\ 
Joint Position & 0.1 & 1.0 & 10.0 \\ 
Contact Point & 50.0 & 200.0 & 500.0 \\ 
 \hline
\end{tabular}
\end{table}

\begin{figure*}
    \centering
    \includegraphics[width=\textwidth]{Figures/FeasibilityComboFigure2.PNG}
    \caption{(a) Contact anchor color indicates whether a contact is removable. The operator can select a contact and preview the CoM region in the absence of the contact. The CoM region shown is a preview with the right arm contact removed, which is currently infeasible. The preview robot turns red if a keyframe is invalid (b) or yellow if the keyframe transition is invalid. In (c), although the keyframe is valid, the keyframe transition requires the CoM to leave the CoM feasible region before placing the left hand. (d) Actuation feasibility can be visualized by a force polytope at contact points and by indicating joint torque saturation. In this figure the right knee and elbow are a darker shade, indicating they are close to torque saturation.}
    \label{fig:feasibility}
\end{figure*}

% Following the approaches by Bretl et al. \cite{bretl2008testing} and Orsolino et al. \cite{orsolino2020feasible}, we compute a feasible 

% The vector $\boldsymbol{\rho} = [\rho_{00} \ldots \rho_{N_cm}]^T$ is the stacked set of basis vector multipliers of the $N_c$ contact points. The static equilibrium feasibility condition is given by

% \begin{equation} \label{eq:StaticFeasibility}
% \begin{aligned}
%     \exists \boldsymbol{\rho}   \quad   &    \\
%     \textrm{s.t.}               \quad   &  \boldsymbol{\rho} > 0  \\
%                                 \quad   &  
% \end{aligned}
% \end{equation}



% The CoM constraint region used to compute $\mathbf{H}$ is computed by imposing friction \cite{bretl2008testing} and actuation \cite{orsolino2020feasible} constraints. The region is approximated by solving a series of Linear Programs in which friction and actuation limits are imposed as inequality constraints on contact forces. To model the actuation constraints for a contact point $i$, a force polytope $\mathscr{P}_i$ is computed which represents the set of actuation-consistent forces at the contact point \cite{chiacchio1997force}:

% \begin{equation}
%     \mathscr{P}_i = \{ \mathbf{f}_i \in \mathbb{R}^3 \; | \; \underline{\mathbf{\tau}} \leq \mathbf{J}^T \mathbf{f}_i \leq \bar{\mathbf{\tau}} \}
% \end{equation}

% Where $\underline{\mathbf{\tau}}$ and $\bar{\mathbf{\tau}}$ are the lower and upper torque bounds along the contacting limb and $\mathbf{J}$ is the Jacobian of the contact point. We compute the vertices of the force polytope by projecting each actuation limit $\mathbf{\tau}_{lim}$ using the pseudoinverse of the contact Jacobian to get the corresponding force $\mathbf{f}_{lim}$.

% \begin{equation}
%     \mathbf{f}_{lim} = (\mathbf{J}^T)^{\dagger} \mathbf{\tau}_{lim}
% \end{equation}

% For a limb with $n$ joints, the force polytope is backed out by iterating over each permutation of the $2^{n}$ values of $\tau_{lim}$.

\section{VIRTUAL REALITY USER INTERFACE} \label{sec:vrui}

\begin{figure*}
    \centering
    \includegraphics[width=\textwidth]{Figures/ComboFigure6.PNG}
    \caption{(a) Operator view while using the VR interface to pose Valkyrie in a squatting position. The robots and anchors are fixed in the scene and the operator can walk or teleport to move around. (b) A spatial pose anchor and its corresponding menu panel. (c) A joint position anchor and its corresponding menu panel. (d) The main menu is used to export or load scripts, toggle visibility of the controller robot and joint position anchors, configure the IK solver behavior, and change anchor behavior  by toggling contact mode and dragging. (e) Valkyrie braced against a wall with the left arm and reaching with the right arm. Visual cues for actuation feasibility include force polytopes, actuation and friction consistent support region, and alerts for joints at risk of torque saturation.}
    \label{fig:vr_combo}
\end{figure*}

The VR user-interface is designed to enable an operator to pose the robot by creating and modifying constraints on-the-fly. To do this, we implemented virtual interactable ``anchors'' which correspond to kinematic tasks. Fig. \ref{fig:vr_combo}a shows the robot in a squatting configuration which has anchors on the chest, legs and arms. The interface displays two robots: a transparent puppet robot which the operator manipulates and (optionally) an opaque controller robot.

\subsection{Anchor Configuration}
 
The interface contains three types of motion control anchors: spatial pose, Center of Mass, and joint position. Spatial pose anchors are created by simply clicking on the desired rigid body and dragging it to the desired pose. Individual axes can also be activated, cleared or modified. A ``ring-and-arrows'' \cite{leeper2012strategies} indicates the anchor setpoint, with actively constrained axes highlighted (Fig. \ref{fig:vr_combo}b). Spatial pose anchors can be designated as a ``contact anchor'', shown as smaller, blue anchors, which increases the weight and can be used to construct the support region (see \ref{sec:contact_mode}). The CoM anchor is represented by only arrows since there is not an orientation setpoint but is modified in the same way as the spatial pose. Joint position anchors are a single arrow centered at the joint. When dragging the joint, vertical motion in world frame is mapped to a desired joint position.

Each anchor has a menu with additional customization options. Figures \ref{fig:vr_combo}(b) and \ref{fig:vr_combo}(c) show the spatial pose and joint position menus, respectively. Each menu allows switching between low, medium and high weight for the corresponding kinematic task, which corresponds to entries of $\mathbf{C_J}$ in the IK. Each anchor can optionally be set to match the controller robot's pose. Two options are provided: a tracking option which continuously updates to the controller's pose and a snapping option which performs the update once. The tracking feature is beneficial for any contact that does not need precise placement in world frame but needs to remain stationary. Each anchor can also be flagged as ``persistent,'' such that the user can quickly clear all anchors which are not persistent. We find that often the operator would keep certain anchors static while iterating through configurations of other anchors and a notion of persistence aids the operator in this process. The joint position anchor has a ``mirroring'' option in which dragging or activating a joint will perform the same action on the opposite-side joint, if one exists.

Spatial pose contact anchors are created in a specific way to ensure the robot and environment mesh polytopes are in contact and tangent, shown in Fig. \ref{fig:contact_creation}. The operator first enables ``contact mode'', in which the controller is then projected to the surface of the nearest robot mesh as the operator moves (Fig. \ref{fig:contact_creation}(a)). The surface normal of the robot mesh is also shown during this step. When the operator chooses a contact point, the controller then switches to projecting to the environment mesh, with that surface normal also being shown. When the operator chooses a setpoint, the contact point then snaps to this position. We note that the orientation is only constrained in the two orientations required to keep the polytopes tangent. This is achieved by first having the $\mathbf{z}$ axis of the task frame aligned with the robot mesh surface normal. The $\mathbf{x}$ and $\mathbf{y}$ orientation setpoints are then computed to align the two surface normals.

\begin{figure}
    \centering
    \includegraphics[width=0.7\columnwidth]{Figures/ContactCreationPipeline_scaled.PNG}
    \caption{Contact points are projected to the surface of the collision mesh while creating and placing a contact anchor. The arrow indicates the surface normal of the robot or environment mesh.}
    \label{fig:contact_creation}
\end{figure}

% This is done by ensuring the anchor is on the surface of the robot's collision mesh and setpoint on the surface of the enviroment's collision mesh. Fig. \ref{fig:contact_creation} shows this process from an operator perspective. First the anchor is created (Fig. \ref{fig:contact_creation}a), and the closest robot collision mesh is queried and the anchor is projected to the surface of the mesh. The anchor position follows the VR controller and is continuously recomputed until the operator confirms the anchor position or switches modes. The operator can then place the contact anchor onto an environment mesh, in which the anchor setpoint is moved to the closest environment mesh.

\subsection{Actuation Feasibility}\label{sec:contact_mode}

The interface is intended for multi-contact scenarios which may saturate the robot's actuation limits. Visual cues are included to assist the operator in maintaining actuation feasibility, shown in Fig. \ref{fig:vr_combo}(e). The force polytope \cite{chiacchio1997force} of any contact point can be visualized and is updated as the robot's posture is adjusted. This can be useful when the ``major axis'' of the polytope has an intuitive orientation, such as vertical when contacting the ground or horizontal when bracing against a wall. A multi-contact support polygon can also be visualized for cases where the feasibility of the CoM position needs to be checked. This region is a convex set of feasible CoM XY positions based on the contact friction constraints \cite{bretl2008testing} and actuation constraints \cite{orsolino2020feasible} of the robot. The operator can toggle between this generalized support region and the nominal convex hull of contact points. The active support polygon is also sent to the inverse kinematics module to compute $\mathbf{H}$, to constrain the robot's CoM. A visualization of the feasibility of individual joint torques for the target configuration is also provided. Given the robot's contact state, we compute the torques required for static equilibrium through inverse dynamics. The color of each joint anchor from this to provide a visual cue for joints which are near saturation. For example, the elbow joint in Fig. \ref{fig:vr_combo}(e) is red, indicating it is near a torque limit for this configuration.

\subsection{Additional Operator Tools}\label{sec:op_tools}

% Additional tools are included to facilitate the operator's workflow and control of the kinematic solver's behavior. 
An option to ``snap anchors to puppet'' is provided which replaces each anchor's setpoint with the achieved value of the IK solver. For tasks with numerous anchors this is a tool for ensuring the operator is requesting a feasible set of objectives. The operator can also ``snap puppet'' when a preferred configuration is reached, which sets the kinematic solver's nominal configuration (Sec. \ref{sec:ik}) to the current solver solution. This was found to be helpful during manipulation tasks where the arm's nullspace may not be sufficiently constrained.

\subsection{Interface Workflow}

The trajectory design workflow has three phases: prototype, edit and deploy. We primarily prototype in simulation but the framework does allow for online trajectory creation (Fig. \ref{fig:script_pipeline}). When prototyping, the puppet robot is initialized to the controller robot's configuration and operator begins placing anchors. When the desired pose is reached, the operator dispatches the pose to the controller which executes the motion. The key frame is also stored which includes the controller robot's pose, the puppet robot's pose and all anchor data. If the operator is not satisfied with the motion or the robot falls down, an ``undo'' button is available which rewinds the simulator to the start of the motion and removes the stored key frame. This process is iteratively performed until the trajectory is complete, which usually is comprised of 5-20 key frames. The list of key frames as well as environment model is logged as a JSON file. Trajectory scripts can be edited and replayed in both the VR or desktop applications. Often it is useful to prototype a ``template'' script from which other scripts can be created. For example, a push-up template can be copied and push-ups of varied heights can be achieved by only modifying the $\mathbf{z}$ setpoint of the CoM objective. In the following section we present results from deploying trajectories in simulation and on hardware.

% At a high level, the goal is designing a library of trajectories which can be saved.
% We have two main uses for these trajectories:
%   one is to prototype specific scenarios in simulation which could in turn inform robot specs
%   the other is operational, in which a trajectory could be re-played on hardware
% The workflow has three main phases: prototyping, editting and deploying.
%   the prototyping phase involves the setup in fig., in which the simulation is used as the robot. during this phase the operator generates a keyframe, executes it, and checks for stability. the option to undo execution is provided which removes the keyframe, etc.
%   when the operator is satisfied with the script, the desktop editor can be used to copy or modify scripts. for example a pushup position can be duplicated at various heights by simply modifying the z setpoint on the CoM
%   finally the deployment step as mentioned is either in simulation or on hardware

% \subsection{Script Sequence Creation}

% We use the VR interface to build trajectories by incrementally generating keyframes. Each keyframe corresponds to a static wholebody configuration which is the IK solver's solution. When a keyframe is finalized by the operator, it is simultaneously saved and dispatched to a simulation which executes the motion as an additional validation step. If the keyframe causes the simulated robot to fall or become unstable, the operator can press a reset button which rewinds the simulation to before the latest keyframe and removes the saved keyframe. The operator can then make adjustments to the anchors and try dispatching the keyfrarme again. This process continues incrementally until the desired trajectory is complete, which usually consists of 5-20 keyframes depending on the motion's complexity. We use this workflow to build trajectory libraries so that motions can be saved for future use. Figure \ref{fig:script_pipeline} shows a simple schematic of the control flow for building a trajectory. This was implemented using a distributed setup in which the IK, VR and simulation were running on separate processes and coordinated using ROS 2 messages \cite{ros2}.

\section{POST-PROCESSING AND CONTROL}

In order to execute a given motion, the key frames are converted to a continuous jointspace trajectory $\mathbf{q}(t)$ which can be tracked by a controller. This trajectory is constrained by the list of key frame configurations $\mathbf{q}_0 \ldots \mathbf{q}_n$ and key frame times $t_0 \ldots t_n$, such that $\mathbf{q}(t_i) = \mathbf{q}_i$. We model the trajectory as a sequence of third-order polynominal splines between key frames and enforce position and velocity continuity between splines. This results in $n - 2$ free variables for the velocity at each key frame, given the start and end velocities are constrained to the controller robot's current velocity and zero, respectively. The following optimization is solved to compute the unconstrained key frame velocities $\mathbf{\dot{q}}_u$, where $q_i(t)$ is the scalar trajectory of joint $i$:

\begin{equation}
    \min_{\mathbf{\dot{q}}_u} \,\, \sum_i \,\, \int \| \ddot{q}_i(t) \|^2 dt
\end{equation}

The trajectory $\mathbf{q}(t)$ is tracked using an impedance controller consisting of a PD tracking law with an optional gravity compensation term. For trajectories where the contact state is trusted the gravity compensation is enabled. However for highly contact-rich motion such as scenarios (c) and (e) in Table \ref{table:simulation_summary}, gravity compensation is disabled.
\section{RESULTS}

\begin{table*}[t]
\caption{Trajectories validated on the Valkyrie humanoid in simulation (all) and hardware (e, f).}

% \caption{Trajectories validated on the Valkyrie humanoid. All trajectories are validated in a physics simulation, (e) and (f) are additionally validated on hardware.}
\centering
\begin{tabular}{C{2em} C{15em} C{6em} C{4em} C{8em} C{7em} C{7em} C{7em}} 
 \hline
  & Description & Number of key frames & Duration (s) & Number of unique non-contact 6-dof anchors & Number of unique contact 6-dof anchors & Number of unique 1-dof anchors  & Number of unique CoM anchors\\ [0.5ex] 
 \hline
 (a) & Standing up from lying down on flat ground & 22 & 57.5 & 22 & 18 & 53 & 16  \\ 
 (b) & Stepping over a 45cm tall barrier with handholds & 18 & 59 & 14 & 38 & 38 & 17 \\
 (c) & Climbing up and standing on an 80cm tall ledge & 24 & 61.5 & 16 & 27 & 32 & 20 \\
 (d) & Rolling over from facing down to facing up & 6 & 19.5 & 3 & 7 & 45 & 0 \\
 (e) & Reaching forward and bracing against a wall to extend range of motion & 5 & 25 & 6 & 9 & 1 & 3 \\ 
 (f) & Crawling to kneeling with flat handholds & 21 & 84 & 24 & 27 & 47 & 18  \\ 
 \hline
\end{tabular}
\label{table:simulation_summary}
\end{table*}

\begin{figure*}
    \centering
    \includegraphics[width=0.95\textwidth]{Figures/DemoFigures3.PNG}
    \caption{Operator view while generating trajectories.}
    \label{fig:demo_trajectories}
\end{figure*}

We tested our framework by generating motions for a variety of multi-contact scenarios. Table \ref{table:simulation_summary} contains some of these scenarios along with key frame statistics. All trajectories were validated in simulation and trajectories (e) and (f) were validated on hardware. Key frame transitions have a default value of 2s for simulation and 4s for hardware but the operator can override this value to both shorten or extend transitions. We selected contact-rich scenarios to generate motions that are difficult to plan autonomously or through standard teleoperation. Many of the robot's limbs are used for contact, including feet (all), arms (all), knees (a, c, d, f), and chest (c, d). Additionally, many contact geometries are included. Planar contact occurs when the foot is in full contact with the environment. Point contacts are also common when the arms or knees are in contact with the environment, since these are modelled using curves meshes. Line contacts are also used when one edge of the foot is in contact, such as both feet in Fig. \ref{fig:demo_trajectories}(c). We find that the operator's use of anchors varies significantly based on the scenario. For example, when rolling over on flat ground (scenario d) the robot is often in a position where the CoM cannot be directly controlled and was not used as an anchor. In contrast, climbing onto a ledge (scenario c) requires careful positioning of the CoM while climbing and therefore is used in almost every key frame.

\subsection{CoM Constraint Regions}

To validate effectiveness of the CoM constraint region (Sec. \ref{sec:contact_mode}) for the generated motions, we compare it to a baseline flat-ground constraint. Figure \ref{fig:com_prox} shows this comparison performed for scenarios (e) and (f). The ``flat ground'' model computes the constraint region as the convex hull of the robot's contact points.  The ``multi-contact'' constraint region is computed using the friction- and actuation-aware model. The plotted quantity is the distance of the CoM to the nearest constraint edge of both regions. Both scenarios have key frames with substantial (multiple centimeter) difference in stability margin. Although the multi-contact constraint region is generally more restrictive, scenario (e) key frame 2 demonstrates this is not always the case. This key frame corresponds to a braced reaching motion, shown in Figure \ref{fig:hardware_demo} (left). In this situation, using the multi-contact constraint region enables a higher range of motion than would be possible if using the flat-ground model. Conversely, the multi-contact region is very restrictive for configurations in scenario (f) that require support from the arms. In scenario (f) key frame 7 (Fig. \ref{fig:hardware_demo}), the robot places significant weight on the right arm while lifting the left arm. The actuation limits of the right arm are reflected by the multi-contact constraint being 5cm higher than the flat ground model.

\begin{figure}
    \centering
    \includegraphics[width=0.8\columnwidth]{Figures/CoMStabilityMargins.png}
    \caption{CoM stability margins for scenarios (e) and (f).}
    \label{fig:com_prox}
\end{figure}

\begin{figure}
    \centering
    \includegraphics[width=\columnwidth]{Figures/hardware_demo.PNG}
    \caption{Valkyrie executing two multi-contact motions: (left) bracing against a wall with the right arm and reaching forward with the left arm and (right) placing the arms on cinder blocks while maneuvering to a kneel.}
    \label{fig:hardware_demo}
\end{figure}


% Timing and frequency of ''undo'' button
% other features that were useful. highlight cases where the actuation-feasible region is useful

\subsection{User Interface Operation}

We find there are two primary reasons for a generated key frame to be infeasible: controller failures and inverse kinematics failures. Controller failures often occur because the CoM trajectory is unstable or there is an unexpected collision while moving to a key frame. Our approach assumes that key frames are sufficiently close such that validating subsequent key frames serves as a validation of the trajectory between them. However in practice this does not always hold, particularly when a limb is moving near the environment such as the foot moving over the barrier in Fig. \ref{fig:demo_trajectories}(b). This could be addressed by incorporating a motion preview similar such as \cite{johnson2017team, marion2018director}. Inverse kinematics failures occur for two reasons: getting ``stuck'' and going unstable. Since our IK solver is based on local optimization, it is susceptible to getting stuck in local minima. Often the operator can guide the robot out of the minimum when aware that the problem is occurring. Solver instability can occur when inconsistent objectives are requested with high weight, such as contacts, collisions and CoM positioning. Such cases require halting the IK and reverting to the last key frame. For this reason, the operator may prefer disabling CoM and collision constraints in the solver and using visual cues as indication of feasibility, which can mitigate solver instability.

% Discuss different strategies, i.e. no com anchor for rolling on the ground, etc

% Emphasize how often the operator would press the abort button
% etc.

% Discuss the time taken for each one. ~5min per key frame.
% Discuss when and why the undo button is used.


\section{Conclusion}
\label{sec:conclusion}

We consider top-down attention by explaining from an Analysis-by-Synthesis (AbS) view of vision. Starting from previous work on the functional equivalence between visual attention and sparse reconstruction, we show that AbS optimizes a similar sparse reconstruction objective but modulates it with a goal-directed top-down modulation, thus simulating top-down attention. We propose \model, a top-down modulated ViT model that variationally approximates AbS. We show that \model achieves controllable top-down attention and improves over baselines on V\&L tasks as well as image classification and robustness.

\printbibliography
% \bibliography{bibliography}
\balance

% - Is "Direct" the right word?
% - Should this be pitched as a teleoperation framework? Or a trajectory generation framework, or maybe guided motion planning
% - 

\end{document}
