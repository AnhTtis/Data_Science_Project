\section{POST-PROCESSING AND CONTROL}

In order to execute a given motion, the key frames are converted to a continuous jointspace trajectory $\mathbf{q}(t)$ which can be tracked by a controller. This trajectory is constrained by the list of key frame configurations $\mathbf{q}_0 \ldots \mathbf{q}_n$ and key frame times $t_0 \ldots t_n$, such that $\mathbf{q}(t_i) = \mathbf{q}_i$. We model the trajectory as a sequence of third-order polynominal splines between key frames and enforce position and velocity continuity between splines. This results in $n - 2$ free variables for the velocity at each key frame, given the start and end velocities are constrained to the controller robot's current velocity and zero, respectively. The following optimization is solved to compute the unconstrained key frame velocities $\mathbf{\dot{q}}_u$, where $q_i(t)$ is the scalar trajectory of joint $i$:

\begin{equation}
    \min_{\mathbf{\dot{q}}_u} \,\, \sum_i \,\, \int \| \ddot{q}_i(t) \|^2 dt
\end{equation}

The trajectory $\mathbf{q}(t)$ is tracked using an impedance controller consisting of a PD tracking law with an optional gravity compensation term. For trajectories where the contact state is trusted the gravity compensation is enabled. However for highly contact-rich motion such as scenarios (c) and (e) in Table \ref{table:simulation_summary}, gravity compensation is disabled.