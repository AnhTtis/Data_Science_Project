\section{RELATED WORK}

% Overview of retargeting
Translating an operator's intent to robot motion is a challenging task, particularly when commanding coordinated motions to legged or dexterous robots. This challenge has given rise to various retargeting frameworks for efficiently mapping operator input to robot motion. Methods for retargeting depend on factors such as available human measurements, task application and control scheme \cite{darvish2023teleoperation}.

% Simple retargeting - manipulating control/IK setpoints
One approach is to use a simple map to give the operator direct control of the mapped task. In Leeper et al. \cite{leeper2012strategies} this is implemented through an interactable ``rings-and-arrows'' widget that directly maps to the controller's end-effector setpoint. Similar strategies of direct setpoint control were widely used during the DARPA Robotics Challenge for teleoperating humanoids \cite{zucker2015general, johnson2017team, marion2018director}. In these frameworks, the operator directly interfaced with an inverse kinematics solver's setpoint through similar interactable elements. These approaches, in particular \cite{marion2018director}, also had additional flexibility such as selecting hand, chest, pelvis and Center of Mass setpoints as well as the taskspace constraint set and base of the kinematic chain. Recently this approach has been adapted to VR for use in humanoid teleoperation \cite{allspaw2021implementing, wonsick2021human}. In Wonsick et al. \cite{wonsick2021human} the operator places hand waypoints in the world which can be freely adjusted before planning an arm trajectory.

% Partial/Whole-body retargeting
The use of VR and motion capture enables more sophisticated operator mapping such as partial- or whole-body kinematic retargeting. Early pioneering work by Pollard et al. \cite{pollard2002adapting} used motion capture data from a human actor to mimic dancing motions by mapping to the torso and upper body of a fixed-base humanoid. The operator-to-robot mapping accounted for joint limits, velocity limits and singularities. Extensions of this work include online reference motion for enabling upper-body dancing with active balance \cite{dariush2009online} and simulated stepping and Tai-chi maneuvers \cite{yamane2010control}. Recently, there have been integrated architectures in which the operator commands reference postures through a retargeting scheme and walking behavior through a joystick or treadmill \cite{penco2019multimode, elobaid2020telexistence, jorgensen2022cockpit}. These architectures have had success in tasks requiring both walking and manipulation such as door opening, wall bracing and debris removal. In Otani et al. \cite{otani2017adaptive} a motion capture based retargeting scheme is applied to multi-contact manipulation scenarios. Operator motion is mapped to support, free and manipulation tracking sets and is demonstrated in simulation by bracing against a table while manipulating an object.