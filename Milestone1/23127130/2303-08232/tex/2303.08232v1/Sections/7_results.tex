\section{RESULTS}

\begin{table*}[t]
\caption{Trajectories validated on the Valkyrie humanoid in simulation (all) and hardware (e, f).}

% \caption{Trajectories validated on the Valkyrie humanoid. All trajectories are validated in a physics simulation, (e) and (f) are additionally validated on hardware.}
\centering
\begin{tabular}{C{2em} C{15em} C{6em} C{4em} C{8em} C{7em} C{7em} C{7em}} 
 \hline
  & Description & Number of key frames & Duration (s) & Number of unique non-contact 6-dof anchors & Number of unique contact 6-dof anchors & Number of unique 1-dof anchors  & Number of unique CoM anchors\\ [0.5ex] 
 \hline
 (a) & Standing up from lying down on flat ground & 22 & 57.5 & 22 & 18 & 53 & 16  \\ 
 (b) & Stepping over a 45cm tall barrier with handholds & 18 & 59 & 14 & 38 & 38 & 17 \\
 (c) & Climbing up and standing on an 80cm tall ledge & 24 & 61.5 & 16 & 27 & 32 & 20 \\
 (d) & Rolling over from facing down to facing up & 6 & 19.5 & 3 & 7 & 45 & 0 \\
 (e) & Reaching forward and bracing against a wall to extend range of motion & 5 & 25 & 6 & 9 & 1 & 3 \\ 
 (f) & Crawling to kneeling with flat handholds & 21 & 84 & 24 & 27 & 47 & 18  \\ 
 \hline
\end{tabular}
\label{table:simulation_summary}
\end{table*}

\begin{figure*}
    \centering
    \includegraphics[width=0.95\textwidth]{Figures/DemoFigures3.PNG}
    \caption{Operator view while generating trajectories.}
    \label{fig:demo_trajectories}
\end{figure*}

We tested our framework by generating motions for a variety of multi-contact scenarios. Table \ref{table:simulation_summary} contains some of these scenarios along with key frame statistics. All trajectories were validated in simulation and trajectories (e) and (f) were validated on hardware. Key frame transitions have a default value of 2s for simulation and 4s for hardware but the operator can override this value to both shorten or extend transitions. We selected contact-rich scenarios to generate motions that are difficult to plan autonomously or through standard teleoperation. Many of the robot's limbs are used for contact, including feet (all), arms (all), knees (a, c, d, f), and chest (c, d). Additionally, many contact geometries are included. Planar contact occurs when the foot is in full contact with the environment. Point contacts are also common when the arms or knees are in contact with the environment, since these are modelled using curves meshes. Line contacts are also used when one edge of the foot is in contact, such as both feet in Fig. \ref{fig:demo_trajectories}(c). We find that the operator's use of anchors varies significantly based on the scenario. For example, when rolling over on flat ground (scenario d) the robot is often in a position where the CoM cannot be directly controlled and was not used as an anchor. In contrast, climbing onto a ledge (scenario c) requires careful positioning of the CoM while climbing and therefore is used in almost every key frame.

\subsection{CoM Constraint Regions}

To validate effectiveness of the CoM constraint region (Sec. \ref{sec:contact_mode}) for the generated motions, we compare it to a baseline flat-ground constraint. Figure \ref{fig:com_prox} shows this comparison performed for scenarios (e) and (f). The ``flat ground'' model computes the constraint region as the convex hull of the robot's contact points.  The ``multi-contact'' constraint region is computed using the friction- and actuation-aware model. The plotted quantity is the distance of the CoM to the nearest constraint edge of both regions. Both scenarios have key frames with substantial (multiple centimeter) difference in stability margin. Although the multi-contact constraint region is generally more restrictive, scenario (e) key frame 2 demonstrates this is not always the case. This key frame corresponds to a braced reaching motion, shown in Figure \ref{fig:hardware_demo} (left). In this situation, using the multi-contact constraint region enables a higher range of motion than would be possible if using the flat-ground model. Conversely, the multi-contact region is very restrictive for configurations in scenario (f) that require support from the arms. In scenario (f) key frame 7 (Fig. \ref{fig:hardware_demo}), the robot places significant weight on the right arm while lifting the left arm. The actuation limits of the right arm are reflected by the multi-contact constraint being 5cm higher than the flat ground model.

\begin{figure}
    \centering
    \includegraphics[width=0.8\columnwidth]{Figures/CoMStabilityMargins.png}
    \caption{CoM stability margins for scenarios (e) and (f).}
    \label{fig:com_prox}
\end{figure}

\begin{figure}
    \centering
    \includegraphics[width=\columnwidth]{Figures/hardware_demo.PNG}
    \caption{Valkyrie executing two multi-contact motions: (left) bracing against a wall with the right arm and reaching forward with the left arm and (right) placing the arms on cinder blocks while maneuvering to a kneel.}
    \label{fig:hardware_demo}
\end{figure}


% Timing and frequency of ''undo'' button
% other features that were useful. highlight cases where the actuation-feasible region is useful

\subsection{User Interface Operation}

We find there are two primary reasons for a generated key frame to be infeasible: controller failures and inverse kinematics failures. Controller failures often occur because the CoM trajectory is unstable or there is an unexpected collision while moving to a key frame. Our approach assumes that key frames are sufficiently close such that validating subsequent key frames serves as a validation of the trajectory between them. However in practice this does not always hold, particularly when a limb is moving near the environment such as the foot moving over the barrier in Fig. \ref{fig:demo_trajectories}(b). This could be addressed by incorporating a motion preview similar such as \cite{johnson2017team, marion2018director}. Inverse kinematics failures occur for two reasons: getting ``stuck'' and going unstable. Since our IK solver is based on local optimization, it is susceptible to getting stuck in local minima. Often the operator can guide the robot out of the minimum when aware that the problem is occurring. Solver instability can occur when inconsistent objectives are requested with high weight, such as contacts, collisions and CoM positioning. Such cases require halting the IK and reverting to the last key frame. For this reason, the operator may prefer disabling CoM and collision constraints in the solver and using visual cues as indication of feasibility, which can mitigate solver instability.

% Discuss different strategies, i.e. no com anchor for rolling on the ground, etc

% Emphasize how often the operator would press the abort button
% etc.

% Discuss the time taken for each one. ~5min per key frame.
% Discuss when and why the undo button is used.

