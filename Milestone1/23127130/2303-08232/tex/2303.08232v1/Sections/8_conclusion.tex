\section{DISCUSSION AND FUTURE WORK}

Using the presented VR-based framework, we were able to generate a wide variety of feasible multi-contact humanoid trajectories. The ability to generate complex trajectories from scratch has applications to many areas of robotics, such as informing both mechanical design and controller design through simulations and deployment during teleoperation. Our main design choice was a user interface that makes minimal assumptions of the motion and relies on an operator for all kinematic objectives. We believe this high level of operator burden can be mitigated in future work which could in turn reduce motion generation time. One approach could be to predict the subsequent key frame by performing a single planning iteration so that the operator only needs to modify the predictions. This could leverage planning approaches such as \cite{lin2018humanoid} which use terrain traversability to switch between between locomotion modes such as walking with or without hand supports. Another method to reduce operator burden could be to initialize a new key frame to the operator's posture using a retargeting framework \cite{di2016multi, penco2019multimode}. However while this applies well to standard postures it may be difficult for abnormal postures such as Fig \ref{fig:demo_trajectories}(c,d).

Future work will also focus on controllers more capable of executing the motions generated by this framework. For this work we relied on a simple impedance controller for trajectory validation. Recent advancements have been made in applying model-predictive control (MPC) for multi-contact motions on legged robots \cite{mastalli2020crocoddyl, kim2019highly}. We plan to investigate a method for extracting contact sequences and posture profiles from our trajectories which can be consumed by an MPC or similar controller.

% Integration with teleop. Could be seen as a manual alternative to retargeting

% The use of key frames and offline motion generation draws similarities to previous works that apply animation techniques to robotics \cite{Yamane2003, fender2015creature}.

% TODO. Possible improvements (parts which can be automated). Integration with online teleop.

% Discuss tools that could further speed up process:
    % Suggesting hand-holds or posture
    % Integrating path planning for end-effector trajectory
    % In general, pushing more functionality to a planner
    % Switching to Robot Data eXplorer to incorporate VR and desktop more seamlessly


\subsection{Acknowledgements}

Thanks to Mark Paterson, Alex Sowell and the rest of the NASA JSC team for supporting this effort and in particular enabling joint impedance commands. Thanks to Joseph Godwin for his internship and fabricating test structures.