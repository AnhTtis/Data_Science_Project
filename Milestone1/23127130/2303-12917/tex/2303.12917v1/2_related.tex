\section{Related Works} \label{sec:related_work}

\textbf{Point Cloud Attribute Compression (PCAC).}


{\it Rules-based solutions} mostly utilize transforms to exploit attribute correlations across neighborhood points. For example, Graph Fourier Transform (GFT) and its variants were examined in~\cite{zhang2014point,cohen2016attribute,shao2017attribute,Xu2021PredictiveGG}. Similarly, Gaussian Process Transforms (GPTs) were also studied in~\cite{de2017transform}. 
Queiroz~\textit{et al.}~\cite{de2016compression} proposed a Region-Adaptive Hierarchical Transform (RAHT), a hierarchical sub-band transform that resembles an adaptive variation of a Haar wavelet, which was adopted in the MPEG G-PCC. The latest G-PCC version (TMC13v14), notably improved the original RAHT with state-of-the-art PCAC efficiency. G-PCC also adopted the Predicting and Lifting Transforms to compress the attribute according to the organization of the Level of Details.


{\it Learning-based approaches} have attracted extensive attention recently.
Quach~\textit{et al.}~\cite{quach2020folding} folded 3D point cloud attributes onto the 2D grids and then directly applied the conventional 2D image codec for compression.  
Fang~\textit{et al.}~\cite{fang20223dac} designed an MLP-based entropy model to approximate the probability of RAHT coefficients. 
Alternatively, end-to-end PCAC was also studied.  
Deep-PCAC~\cite{sheng2021deep} applied a point-based network to compress point cloud attribute, while Alexiou~\textit{et al.}~\cite{alexiou2020towards} directly used 3D dense convolutions for compression. 
Recently, SparsePCAC~\cite{Wang2022SparsePCAC} was developed to process the sparse tensor under the variational autoencoder structure for efficient attribute representation.
Unfortunately, despite the technical progress provided by these learning-based PCAC solutions, the lossy compression efficiency is still inferior to the latest G-PCC, not to mention that some of them are exceptionally complex for practical application~\cite{quach2020folding,sheng2021deep,alexiou2020towards}.


\textbf{Multiscale Sparse Representation based PCGC.} Recently, the compression of point cloud geometry has been significantly improved by applying learning-based solutions to effectively exploit correlations.  Among them, multiscale sparse representation-based solutions have demonstrated leading compression performance in both lossy and lossless modes for a variety of point clouds~\cite{Wang2021SparseTM}. This work extends the multiscale structure to support PCAC by exhaustively exploiting cross-scale, cross-group, and cross-color (if applicable) correlations.
