\documentclass[twocolumn]{aastex63}

%\newcommand{\vdag}{(v)^\dagger}
%\newcommand\aastex{AAS\TeX}
%\newcommand\latex{La\TeX}
\newcommand{\vx}{\vec{x}}

\received{June 1, xxxx}
\revised{January 10, xxxx}

\submitjournal{ApJ}

\shorttitle{}
\shortauthors{Wang et al.}

\usepackage{amsmath}
\usepackage{subfigure}
\usepackage{graphicx}
\usepackage{lineno}
\usepackage{orcidlink}
\begin{document}
% \linenumbers
\title{Towards Optimal Reconstruction of Shear Field with PDF-Folding}
\correspondingauthor{JUN ZHANG}
\email{betajzhang@sjtu.edu.cn}

\author{Haoran Wang~\orcidlink{0000-0001-5945-8399}}
\affiliation{Department of Astronomy, Shanghai Jiao Tong University, Shanghai 200240, China}

\author{Jun Zhang~\orcidlink{0000-0003-0002-630X}}
\affiliation{Department of Astronomy, Shanghai Jiao Tong University, Shanghai 200240, China}
\affiliation{Shanghai Key Laboratory for Particle Physics and Cosmology, Shanghai 200240, China}

\author{Hekun Li~\orcidlink{0000-0002-0610-2361}}
\affiliation{Department of Astronomy, Shanghai Jiao Tong University, Shanghai 200240, China}

\author{Cong Liu~\orcidlink{0009-0006-2694-6752}}
\affiliation{Department of Astronomy, Shanghai Jiao Tong University, Shanghai 200240, China}

\begin{abstract}
Weak lensing provides a direct way of mapping the density distribution in the universe. To reconstruct the density field from the shear catalog, an important step is to build the shear field from the shear catalog, which can be quite nontrivial due to the inhomogeneity of the background galaxy distribution and the shape noise. We propose the PDF-Folding method as a statistically optimal way of reconstructing the shear field. It is an extention of the PDF-SYM method, which is previously designed for optimizing the stacked shear signal as well as the shear-shear correlation for the Fourier\_Quad shear estimators. PDF-Folding does not require smoothing kernels as in traditional methods, therefore it suffers less information loss on small scales, and avoids possible biases due to the spatial variation of shear on the scale of the kernel. We show with analytic reasoning as well as numerical examples that the new method can reach the optimal signal-to-noise ratio on the reconstructed shear map under general observing conditions, i.e., with inhomogeneous background densities or masks. We also show the performance of the new method on real data around foreground galaxy clusters.

\end{abstract}

\keywords{gravitational lensing: weak --- large-scale structure of universe --- methods: data analysis}

\section{Introduction} \label{sec:intro}
Cosmic shear refers to the coherent shape distortion of the background galaxies due to the gravitational lensing effect by the foreground large scale density fluctuation \citep{Kaiser1992,Bacon2000,Kaiser2000,Wittman2000,Bartelmann2001,Hoekstra2008,Mandelbaum2018}. It has been routinely measured in galaxy surveys for the purpose of not only constraining the cosmological parameters %\citep{Bacon2000,Planck2014,Hildebrandt2017,Heymans2021,Asgari2021,Doux2022}
\citep{Schrabback2010,Heymans2013,Kilbinger2013,Fu2014,Hikage2019,Heymans2021,Asgari2021,Doux2022}
, but also reconstructing density profiles of the dark matter halos, voids, filaments \citep{Mandelbaum2006,Melchior2014,Clampitt2016,Sanchez2016,Luo2018,Dong2019,Schrabback2021,Xu2021,Fong2022,Wang2022}. 

To our knowledge, lensing is so far the only direct way of mapping out the matter (surface) density field, making it a particularly valuable tool in modern cosmology. The generated mass/density map can provide details about the small scale structure of the universe and the interaction between galaxies, clusters, and the cosmic web. It holds the information about the integrated density fluctuation along the line of sight. Compared with shear two-point correlations, popular applications on the convergence map, such as N-point statistics \citep{Secco2022}, 
%higher order moments \citep{Cooray_2001}, 
peak statistics \citep{Fan_2007,Dietrich2010,Fan2010,Liu_2015,Zorrilla2016,Liu_2016,Shan_2017,Chen_2020,Zhang_2022,Liu_2022,Liu2023}, and Minkowski functionals \citep{Petri2013,Fang2017,Liu2022}, can provide complementary information about non-Gaussian density field at late times generated by non-linear gravitational collapse on small scales. 
It is usually convenient to implement these methods to the density field directly and get the constraint to cosmological parameters and models.
The mass maps can also be intrinsically useful. For example, using the DES Science Verification mass map, \cite{Clerkin2017} shows that the one-point distribution of the density field is more consistent with log-normal than Gaussian. Combining mass maps with the spatial distributions of stellar mass or galaxy clusters enable us to study the relation between the visible baryonic matter and invisible dark matter. Using mass maps to constrain galaxy bias \citep{Chang2016}, the relation between the distribution of galaxies and matter, can in turn aid cosmological probes other than weak lensing. It also enables simple tests for systematic errors in the galaxy shape catalogs.


The surface density field can be constructed from the background shear field through a linear transformation \citep{KS93}. This process is often refined by adopting prior knowledge on the density field through the maximum likelihood method, including Wiener filtering, Sparse priors \citep{Leonard2014,Starck2015,Li2021,Price2021}, DeepMass \citep{Jeffrey2020}. 
However, all the methods focus on deriving the $\kappa$ map from the shear map, but much less on how to generate a shear map from the shear catalog, which is perhaps an equally important problem. Currently, the shear field is typically made by taking the weighted sum of the shear estimators within a given smoothing kernel. The choice of the kernel size can be quite nontrivial: it should be neither too small for keeping enough source galaxies, nor too large for the sake of preserving a reasonably good spatial resolution. A large kernel may also introduce systematic errors in the shear field due to the coupling between the inhomogeneity of the shear field and that of the source density on the kernel scale. In this paper, we aim at solving these problems by proposing a new way to extract the shear map from a shear catalog. Not only that we try to avoid the systematic errors aforementioned, we also hope to reach the optimal statistical uncertainty in the reconstructed shear map. We call this new method PDF-Folding (called PF method hereafter), as it is based on symmetrizing the probability distribution function (PDF) of the shear estimators, similar to the PDF-SYM method  previously proposed in \cite{Zhang2017}.

The rest of the paper is organized as follows.
In \S\ref{method}, we introduce the PF algorithm. In \S\ref{numerical}, we test the accuracy of this method using the mass map computed from the Illutris simulation \citep{Nelson2015}, and demonstrate the accuracy of the PF method in the presence of inhomogeneous source distribution. As a check of the actual performance of PF, we apply the method on the real shear catalog around a few massive foreground galaxy clusters in \S\ref{real}. We give a brief conclusion in \S\ref{conclusion}, and discuss some standing issues in the PF method that the users should be careful with.  


\section{Method} 
\label{method}

\subsection{The PDF-SYM Method for Shear Recovery}
PDF-SYM is introduced in \cite{Zhang2017} as a new statistical approach of estimating the shear signal from the shear estimators. It aims at achieving the minimum statistical error (the Cramer-Rao Bound) without introducing systematic biases. Although it is developed based on the Fourier\_Quad shear estimator, the idea is in principle applicable to shear estimators of any form. 

To demonstrate the idea, let us assume that the shear estimator is simply the galaxy ellipticity $e$ of an ideal form, i.e., the measured $e$ is related to the intrinsic ellipticity $e_I$ and the shear signal $g$ via $e=e_I+g$. Note that realistic shear estimators typically require additional corrections, or even take some unconventional forms \citep{Zhang2011a,Sheldon2017}. These changes however does not affect our discussion below, and can be easily incorporated into the PDF-based algorithms, as shown in \cite{Zhang2017}. For simplicity, we have also neglected the sub-indices of the ellipticity and shear. The basic idea of PDF-SYM is to find the value of a pseudo signal $\hat{g}$, which can best symmetrize the PDF of the corrected shear estimator $\hat{e}(=e-\hat{g})$. The new PDF of $\hat{e}$ is related to the PDF of the intrinsic ellipticity $e_I$ via $P(\hat{e})d\hat{e}=P_I(e_I)de_I$, and we have:
\begin{equation}
\label{pi}
P(\hat{e})=P_I(\hat{e}+\hat{g}-g).
\end{equation}
Since $P_I$ is a symmetric function, one can see from eq.(\ref{pi}) that $P(\hat{e})$ is best symmetrized when $\hat{g}$ is equal to the true shear value $g$. The symmetry level of the PDF can be quantified by comparing the galaxy number counts within bins that are symmetrically placed on the two sides of zero. 

The above algorithm is only well defined for recovering a single shear signal. In the following sections, we try to extend it in a way to reconstruct a shear field, which is the main purpose of this work.


\subsection{Reconstruction of a shear field}
\label{rsf}

It is helpful to start with a simple example. Let us assume that the shear field is one dimensional, i.e. $g(x)=ax+b$, in which $x$ is the coordinate in the range of [-1, 1]. For further simplicity, we can assume that the source galaxies are evenly distributed along $x$. A naive idea of applying PDF-SYM would be to adopt the same form of spatial distribution for the pseudo shear signal, i.e., $\hat{g}(x)=\hat{a}x+\hat{b}$, and try to find the best values of $\hat{a}$ and $\hat{b}$ to symmetrize the overall PDF of the galaxies. This is indeed a natural choice for modelling the local distribution of the shear field if the region is small, and is exactly what we do at the early stage of this project. It turns out that in this case, the parameter $\hat{b}$ can be constrained by symmetrizing the PDF, but not the parameter $\hat{a}$. The change of $\hat{a}$ has opposite effects on the ellipticity distributions for galaxies located at $x<0$ and $x>0$, the overall symmetry of the whole PDF is therefore not affected by $\hat{a}$.

This is illustrated in fig.\ref{fig:ex2}, in which we show the impact of $\hat{a}$ and $\hat{b}$ on the PDF. For clarity, in the figure, we split the PDF for the galaxies with $x<0$ and $x\ge 0$, shown with the blue and red colors respectively. 
As one can see, the parameter $\hat{b}$ moves the blue and red populations along the same direction, therefore can change the symmetry of the PDF. The parameter $\hat{a}$, on the other hand, moves the two groups of galaxies towards opposite directions, thus does not affect the overall symmetry of the whole PDF.


A possible remedy for this problem is to invert the sign of the corrected shear estimators $\hat{e}$ for galaxies of $x<0$. The symmetry of the resulting new PDF becomes sensitive to $\hat{a}$. 
Its value can therefore be found by symmetrizing the folded PDF. This is why our new method is called PDF-Folding. More generally, for a shear field parameterized by a set of orthogonal functions, we find that one can recover the coefficients one-by-one, and each time by symmetrizing the PDF that is folded at the places where the corresponding basis function changes its sign. 
%The implication here is that for recovering different parameters of a shear field, we need to construct PDFs of different forms, i.e., by folding the PDF in appropriate places. This is why our new method is called PDF\_Folding.


To realize this idea, let us consider a general shear field $g(\vec{x})$ in 2D. It can be expanded with a set of orthogonal functions: $g(\vec{x})=\sum_k{a_kf_k(\vec{x})}$.
The question becomes how to estimate each mode's magnitude $a_k$. To characterize the PDF of the corrected shear estimator $\hat{e}$, we define $u_i$ ($i=0, \pm 1,...,\pm l$) as the boundaries of the bins placed symmetrically on the two sides of zero, i.e., $u_i=-u_{-i}$. In total there are $2(l+1)$ bins. Note that the outer boundaries of the two outmost bins are at infinite. Assuming we want to recover $a_w$, the estimated shear field is then written as $\hat{g}(\vec{x})=\hat{a}_wf_w(\vec{x})$, in which we caution that the summation sign is not present. $\hat{a}_w$ is the presumed value of $a_w$. The number of galaxies $N_i$ of the $i^{th}$ bin in our new PDF-Folding scheme is defined as:
\begin{eqnarray}
\label{ni}
N_i&=&\bar{n}\int_{i\cdot f_w(\vec{x})\ge 0} d^2\vec{x} \vert f_w(\vec{x}) \vert\int_{u_{\vert i\vert-1}}^{u_{\vert i\vert}}d\hat{e} P(\hat{e})\\ \nonumber
&+&\bar{n}\int_{i\cdot f_w(\vec{x})<0} d^2\vec{x} \vert f_w(\vec{x}) \vert\int_{-u_{\vert i \vert}}^{-u_{\vert i\vert-1}}d\hat{e} P(\hat{e})
\end{eqnarray}

For now, we simply assume that the galaxy number density $\bar{n}$ is a constant. $P$ is the normalized PDF, which we assume does not change with position. The form in eq.(\ref{ni}) is quite different from the usual definition of $N_i$, which is simply $\bar{n}\int d^2\vec{x} \int_{u_{i-1}}^{u_{i}}d\hat{e} P(\hat{e}) $ (for $i>0$). The idea of folding is manifested by mixing the galaxies on the two sides of zero in PDF according to the sign of $f_w(\vec{x})$ (which is known) in eq.(\ref{ni}). The factor $\vert f_w(\vec{x}) \vert$ is an additional weight for filtering out the contamination from other orthogonal modes, as we show next.

To find out how the value of $\hat{a}_w$ can change the symmetry of the PDF, let us calculate $N_i-N_{-i}$, which is:
\begin{eqnarray}
\label{ni3}
&&N_{i (>0)}-N_{-i} \\ \nonumber
&=&\bar{n}\int d^2\vec{x} f_w(\vec{x})\left[\int_{u_{i-1}}^{u_{i}}-\int_{-u_i}^{-u_{i-1}}\right]d\hat{e} P(\hat{e})
\end{eqnarray}
Note that the integration without lower and upper bounds means integrating over the whole available range of the variable. Using eq.(\ref{pi}), we can relate the function $P$ with the intrinsic PDF $P_I$ (which is symmetric) as: 
\begin{eqnarray}
\label{pi2}
&&P(\hat{e})=P_I(\hat{e}+\hat{g}-g) \\ \nonumber
&=&P_I\left[\hat{e}+\hat{a}_wf_w(\vec{x})-\sum_k{a_kf_k(\vec{x})}\right]\\ \nonumber
&\approx&P_I(\hat{e})+\left[\hat{a}_wf_w(\vec{x})-\sum_k{a_kf_k(\vec{x})}\right]P_I'(\hat{e})
\end{eqnarray}
where $P_I'(e)=dP_I(e)/de$.
The last step above is from Taylor's expansion to the first order, as the shear signal is assumed to be small. Using the results in eq.(\ref{pi2}), we can get:
\begin{eqnarray}
\label{ni4}
&&\left[\int_{u_{i-1}}^{u_{i}}-\int_{-u_i}^{-u_{i-1}}\right]d\hat{e} P(\hat{e}) \\ \nonumber
&\approx&2\left[\hat{a}_wf_w(\vx)-\sum_k{a_kf_k(\vx)}\right]\left[P_I(u_i)-P_I(u_{i-1})\right]
\end{eqnarray}
Using the result of eq.(\ref{ni4}) in eq.(\ref{ni3}), and the orthogonality of $f_k(\vx)$, we get:
\begin{eqnarray}
\label{ni5}
&&N_{i (>0)}-N_{-i} \\ \nonumber
&\approx&2\bar{n}(\hat{a}_w-a_w)\left[P_I(u_i)-P_I(u_{i-1})\right]\int d^2\vec{x} f_w^2(\vec{x})
\end{eqnarray}
It is clear that the folded PDF is best symmetrized when $\hat{a}_w=a_w$. This is true for all the bin pairs. Therefore, to estimate $\hat{a}_w$, we just need to minimize the following $\chi^2$\footnote{The definition of $\chi^2$ follows the form given in \cite{Zhang2017}. In our case, the quantity $N_{i}-N_{-i}$ is a number to quantify the symmetry level of the PDF, as required by our method. The statistical mean of $N_{i}-N_{-i}$ is zero when the PDF is fully symmetric with respect to zero. The denominator in our $\chi^2$ definition is the variance of $N_{i}-N_{-i}$.  The minimum of $\chi^2$ indicates that the PDF reaches a best symmetrized state.}:
% \footnote{Here the defination of $\chi^2$ is only demonstrate the ansymmetryhas few modifications in denominator, which is justifiable because galaxy number in real case do not follows possion distribution. We need to note that our formalism is just approaching the Cramer-Rao bound.}:
\begin{equation}
\chi^2=\frac{1}{2}\sum_{i(>0)} \frac{(N_i-N_{-i})^2}{\langle(N_i-N_{-i})^2\rangle}
\end{equation}
The above procedure is repeated for each $\hat{a}_w$ to recover the whole shear field. This is the basic implementation of the PF method. 

For inhomogeneous galaxy distribution, it is not hard to check that a similar conclusion can be reached if we simply replace the weighting of the galaxy number in eq.(\ref{ni}) from $\vert f_w(\vec{x}) \vert$ to $\vert f_w(\vec{x}) \vert/n(\vec{x})$, where $n(\vec{x})$ is the galaxy number density. The form of eq.(\ref{ni5}) remains the same, but without the factor $\bar{n}$. This, however, turns out not to be the end of the story. The formulation can be further improved to achieve two more goals: 1. proper treatment of masks; 2. minimum statistical noise. This is what we discuss next.

\begin{figure}
    \centering
    \includegraphics[width=\linewidth,height=0.5\linewidth]{simple_example_2.pdf}
    \caption{The behavior of the ellipticity PDF of galaxies located at x$<$0 (in blue color) or x$\geq$0 (in red color) when the parameter a (left panel) or b (right panel) changes. The dotted and solid lines correspond to the PDFs before and after the change of the parameter respectively.}
    \label{fig:ex2}
\end{figure}


\subsection{Optimal Reconstruction Method}
\label{orm}

If we chose the weight to be $\vert f_w(\vec{x}) \vert/n(\vec{x})$, one can immediately identify a problem: what if there are masked areas? A simple option would be to skip the masked areas. However, in this case, the orthogonality of the functions $f_k(\vec{x})$ would not lead to eq.(\ref{ni5}) because of the incomplete domain of integration. Another option would be to use a set of orthogonal functions that are defined in the domain excluding the masked areas. This would be a fine choice, except that the form of $f_k(\vec{x})$ could be quite complicated.

In this work, we take another route: we fill out the masked area with galaxies of a certain number density, e.g., the average number density of the whole field. Each such galaxy is given a random shear estimator/ellipticity, i.e., the shear field is assumed to be zero in the masked areas. This procedure guarantees the completeness of the domain of integration, and therefore the validity of the PF method. 

More generally, we are interested in finding the optimal way of extracting the shear field information from the shear catalog in the case of inhomogeneous galaxy distribution. To do so, we find that the following generalized form of series expansion is useful:
\begin{equation}
\label{formula_alpha}
    g(\vec{x})n^{\alpha}(\vec{x})=\sum_k a_kf_k(\vec{x})
\end{equation}
For recovering the coefficient $a_w$, the galaxy weight and therefore the definitions of the PDF should be updated accordingly as:
\begin{eqnarray}
\label{ni6}
N_i&=&\int_{i*f_w(\vec{x})\ge 0} d^2\vec{x} \vert f_w(\vec{x}) \vert n^{\alpha}(\vec{x})\int_{u_{\vert i\vert-1}}^{u_{\vert i\vert}}d\hat{e} P(\hat{e})\\ \nonumber
&+&\int_{i*f_w(\vec{x})<0} d^2\vec{x} \vert f_w(\vec{x}) \vert n^{\alpha}(\vec{x})\int_{-u_{\vert i\vert}}^{-u_{\vert i\vert-1}}d\hat{e} P(\hat{e})
\end{eqnarray}
In this case, one can show that the difference between the opposite bins is given by:
\begin{eqnarray}
\label{ni50}
&&N_{i (>0)}-N_{-i} \\ \nonumber
&\approx&2(\hat{a}_w-a_w)\left[P_I(u_i)-P_I(u_{i-1})\right]\int d^2\vec{x} f_w^2(\vec{x})
\end{eqnarray}

Note that the galaxy weight is $\vert f_w(\vec{x}) \vert n^{\alpha-1}(\vec{x})$ at $\vec{x}$. To recover $a_w$, e.g., we assume that $\hat{g}(\vec{x})$ is given by $\hat{g}(\vec{x})=n^{-\alpha}(\vec{x})\hat{a}_wf_w(\vec{x})$. Following similar calculations as in \S\ref{rsf}, one can show that eq.(\ref{ni5}) still holds (without the factor $\bar{n}$). Apparently, we now have the exponent $\alpha$ as an additional degree of freedom in our formalism. It turns out that when $\alpha=1/2$, the statistical uncertainty of $a_w$ reaches its minimum. It is therefore our best choice for the shear field reconstruction. We show the details of our proof in Appendix A. 

The above discussion is all about reconstructing the shear field at a given background redshift. More generally, we should consider the redshift distribution of the source galaxies. For the single thin lens case (which is what we consider in this work), the shear fields at different redshifts are related through a simple rescaling with the corresponding critical surface densities. The problem is therefore still two dimensional, and we only need to choose a certain background redshift as a reference. The optimization of the PF method in this case are given in Appendix B.  

In the rest of the paper, we demonstrate some advantages of the PF method using numerical examples, and show some shear/density field reconstruction examples from the real data. 



\section{Numerical Examples}\label{numerical}

In this section, we demonstrate several advantages of the PF method with data from the Illutris-1-Dark simulation \citep{Nelson2015}. The side length of the simulation box is $75 {\rm Mpc/h}$ (comoving). We use the snapshot with ID 120 corresponding to $z=0.2$. We select a $6\times6 ({\rm Mpc/h})^2$ (comoving) part containing cluster-like structure, 
and calculate the projected surface density by integrating over the whole box size. The shear field is then deduced from the mass distribution by assuming the source galaxies are all at $z=0.50$, and placed on a $120\times 120$ grid as shown in fig.\ref{f_field_ill} for the $g_1$ component.
This shear field is applied onto a large number of background galaxies whose intrinsic ellipticities are generated according to \cite{Miller2013}. The source galaxies are randomly placed inside the foreground area. To reduce the shape noise, we group every $5\times5$ grid area to form a $24\times 24$ shear map. 

\begin{figure}[htbp]
\centering
\includegraphics[scale=0.5]{field.pdf}
\caption{Shear field of $g_1$ from the density map of the Illutris-1-Dark simulation within a $6\times6 ({\rm Mpc/h})^2$ comoving area at redshift 0.2. The source redshift is assumed to be at 0.5.}
\label{f_field_ill}
\end{figure}

As our first test, we adopt three strategies to reconstruct the shear map from the shear catalog: 1. by taking the local averages of the shear estimators (called Local\_AVE hereafter); 2. by using the PDF-SYM method, which is designed for optimizing the stacked shear signal on the local ensemble of the shear estimators (called Local\_PDF hereafter); 3. by the PF method. In the last method, we expand the shear field with the Fourier series.

In our first test, the shear field is applied onto $10^{7}$ background galaxies whose ellipticities follow the form of disk-dominated galaxies. We set the parameter $e_{max}$ in \cite{Miller2013} to be $1.0$.
The recovered shear maps ($g_1$) with the methods of Local\_AVE, Local\_PDF, and PF are shown in the upper panel of fig.\ref{f_field}. The corresponding residuals are shown in the lower panels accordingly. One can see that the PDF-based methods generally generate somewhat less noise than the Local\_AVE method. The advantage would be more obvious if the PDF of the ellipticities has a more significant extension to large values. 

Note that although in this simple example, the Local\_PDF method seems to work similarly well as PF, its performance is actually more sensitive to the background galaxy number density, i.e., there should be enough galaxies in each grid cell, as we show in the real data examples in \S\ref{real}. The PF method avoids this issue by using the background galaxies in the field altogether to form the PDF. It therefore does not require a large smoothing kernel or grid size, and can keep information on smaller scales in principle. 

\begin{figure}[htbp]
\centering
\includegraphics[scale=0.7]{compare_HomoPDF0116_0.0-1.0.pdf}
\caption{From the left to right, we show the recovered shear maps of $g_1$ from the Local\_AVE, Local\_PDF, and PF methods respectively. The upper panels show the shear fields, and the lower panels show the deviations of the results from the input shear map accordingly. }
\label{f_field}
\end{figure}



%\begin{figure}[htbp]
\begin{figure}
\centering
\includegraphics[scale=0.6]{diff_sinmap_compare_30.pdf}
\caption{Upper left: the central region of the input shear field from Illutris; upper right: the distribution of the galaxy number density field; lower left: the difference between the input shear field and the recovered one from the Local\_PDF method; lower right: the difference between the input shear field and the recovered one from the PF method.}
\label{f_diff_sinmap_compare}
\end{figure}

%\begin{figure}[htbp]
\begin{figure}
\centering
\includegraphics[scale=0.6]{mask_patch_compare_30.pdf}
\caption{The organization of the panels is the same as that of fig.\ref{f_diff_sinmap_compare}. It demonstrates the mask effect on the recovery of the shear field.}
\label{f_mask_patch_compare}
\end{figure}


\subsection{Effect of Inhomogeneous Galaxy Distribution}
\label{inhomo}
The distribution of the background galaxies are often inhomogeneous (as shown in some examples of \S\ref{real}). In traditional methods, if both the shear signal and the background density varies significantly within the smoothing kernel, we expect the coupling of these two types of fluctuations to induce systematic errors in the recovered shear field. In the new PF method, we find that such a bias can be largely avoided. 

To study this effect, we let the background galaxy density vary with the vertical axis y as '$\sin(ky) +const$'  where $k=2\pi/5$, whose period equals the grid size of the recovered shear field. To enhance the signal-to-noise ratio of the systematic bias, we use $5\times10^7$ galaxies with ellipticities set in the range of $[0.05, 0.1]$, 90\% of which follow the ellipticity distribution of the disk-dominated galaxies, and the other 10\% follow that of the bulge-dominated ones. In fig.\ref{f_diff_sinmap_compare}, we show the original shear field in the upper-left panel, the background galaxy distribution in the upper-right panel (the unit of which is arbitrary). The lower two panels show the differences between the recovered shear field and the original one for the Local\_PDF method (left) and the PF method (right). One can see that the shear residuals of the Local\_PDF method are quite significant, and correlated with the signal. In contrast, the PF method performs quite well in the presence of the background inhomogeneity. Note that in the current study, the smoothing kernel for the Local\_PDF is simply the top-hat function within the grid size (of the recovered shear field). If a larger kernel is chosen, the effect is more obvious. For the PF method, we use the expansion defined in eq.(\ref{formula_alpha}) with $\alpha=1/2$. The local number density around each galaxy is calculated by averaging the galaxy numbers within its neighboring $3\times 3$ finer grid cells (the grid on the original $120\times 120$ map). 

%In traditional way, to extract shear field from shear catalog, it is usual to calculate each grid's shear value with averaging shear signals from all the galaxies in this grid, which is named local-average. If the galaxy number is tiny enough to calculate the shear value, it requires us to enlarge the local-average region, which is consumed to bring systematic bias especially when galaxies distributing inhomogenously or existing mask effect. To prove the influence by inhomogenously distributing galaxies, we recover the traditional shear field with local average $d=5$. 

%The results are shown in fig.\ref{f_diff_sinmap_compare}. The upper right panel of fig.\ref{f_diff_sinmap_compare} presents that galaxes distributes as the form of $\vert\sin24 y \vert +const$, whose period equals local average size $d$, to magnify the bias due to local average. The lower left and lower right panel in fig.\ref{f_diff_sinmap_compare} are residual results of local-average and PF. It obvious strong systematical bias is left in traditional recovery, which is caused by the inhomogenous galaxy distribution coupling with shear field.

% we place same size sample but inhomogenously as fig.\ref{f_diff_densitymap_compare}a, which is a real case copied from DECALS catalogue. The low galaxy number density region is not zero but have few. 
\subsection{Effect of Mask}

Masks are just special cases of background inhomogeneity. For similar reasons, the Local\_PDF operation can also generate systematic biases near masks. In this test, we assign a masked area in the central region of the simulated shear map from Illutris, as shown in the upper-left panel of fig.\ref{f_mask_patch_compare}. It can also be seen from the upper-right panel of the same figure, in which we show the distribution of the background galaxy density (the unit is again arbitrary). We apply the shear field to $10^{7}$ background galaxies in total, the ellipticities of which are generated in the same way as those in \S\ref{inhomo}. The background galaxies are evenly distributed in the whole area except the masked region. In the lower panels of fig.\ref{f_mask_patch_compare}, we again show the performances of the Local\_PDF method (left) and the PF method (right) in terms of the differences between their recovered shear fields and the original one. We can see that there are significant residuals near the masked area in the Local\_PDF method. As a comparison, the shear field from the PF method shows no such residuals, demonstrating a consistently better performance. We note that to use PF in this case, one needs to add fake galaxies of random (or zero) ellipticities in the masked area. The number density of the fake galaxies inside the mask is chosen to be comparable to the mean of the whole field. 

%\textbf{This shear field is applied onto $1\times 10^{7}$ background galaxies whose ellipticities follow the recipe described in \cite{Miller2013}, i.e., 90\% of them are disk-dominated galaxies, and 10\% of them bulge-dominated. The ellipticites are set to be in range of [0.05-0.1].}
%The upper right one shows the galaxy distribution. We can discover that there is no galaxy distributing in mask region. In local average case, shown in the lower left panel, we can see there are residuals nearing the mask area with the local average size $d=5$. But PF way gives a systematic-bias-free recovery as the lower right panel shows.
% But the residuals disappear in SF way shown in subfig.b. Subfig.c and d shows the residuals of local-average and SF after patching the mask region with $<e_i>=0$ galaxies.
 
% Each galaxies was added with a random intrinsic ellipticity following a normal gaussian distribution($\mu=0$, $\sigma=0.1$) and the shear value in its positon. We do the recovery with local-average method and new proposed method as well. The result shown in fig.\ref{f_diff_densitymap_compare}b,c indicates that the tradition local average gives a low resolution reconstrution in low density region. We need to note that in this region, the recovered shear value can be influenced by the nearing galaxies and get a systematical bias. 

% \begin{figure}[htbp]
% \centering
% \label{f_abell1835_sdss_pdf}
% \includegraphics[scale=0.6]{abell1835_sdss_pdf}
% \caption{The aimed cluster A(ra:210.247, dec:2.869, z=0.26) in X-ray band.}
% \end{figure}
% \begin{figure}[htbp]
% \centering
% \label{f_part4_cor_analyse_densitymap}
% \includegraphics[scale=0.6]{part4_cor_analyse_densitymap}
% \caption{The galaxy number density around cluster A.}
% \end{figure}


\subsection{Step-by-step Procedure of the PF Method}
\label{SBS}
% \textbf{Here for simplicity, we still use the ellipticity $e$ to represent the shear estimator, and neglect its sub-index. The application of the PF method for reconstructing the shear field follows the following steps :}

% \textbf{\quad 1. Set up a rectangular grid for the shear field; }

% \textbf{\quad 2. Calculate the galaxy number density $n(\vec{x})$ in each grid;}

% \textbf{\quad 3. Fill up the masked area with galaxies of the average number density and random ellipticities/shear estimators;}

% \textbf{\quad 4. Determine a set of orthogonal and complete functions to parameterize the shear field: $g(\vec{x},z_{s0})\sqrt{n(\vec{x})}=\sum_k{a_kf_k(\vec{x})}$, where $z_{s0}$ is the reference background redshift. The rest of the steps is about how to determine the coefficients $a_k$. This is done one-by-one; }

% \textbf{\quad 5. Each $a_w$ is determined by symmetrizing the PDF of the whole background galaxy sample. The pseudo shear signal for a galaxy of redshift $z$ is given by $\hat{g}(\vec{x},z)=a_wf_w(\vec{x})/\sqrt{n(\vec{x})}\cdot\Sigma_c(z_l,z_{s0})/\Sigma_c(z_l,z)$ (without summing over all possible values of $w$). $z_l$ is the redshift of the lens. For each galaxy, its counting weight is $\vert f_w(\vec{x}) \vert /\sqrt{n(\vec{x})}\cdot\Sigma_c(z_l,z)/\Sigma_c(z_l,z_{s0})$. The sign of $\hat{e}(=e-\hat{g})$ needs to be inverted wherever $f_w(\vec{x})<0$;}




% \textbf{\quad 6. Repeat step 5 to find the values of all $a_w$. Calculate the resulting shear field (at the reference redshift $z_{s0}$) with $g(\vec{x},z_{s0})=\sum_k{a_kf_k(\vec{x})}/\sqrt{n(\vec{x})}$. The resulting shear field can be converted to the $\kappa$ field through, e.g., the K-S inversion algorithm.} 

% A:comparation version

Here for simplicity, we still use the ellipticity $e$ to represent the shear estimator, and neglect its sub-index. The application of the PF method for reconstructing the shear field follows the following steps :

\quad 1. Set up a rectangular grid for the shear field; 

\quad 2. Count the galaxy number density $n(\vec{x})$ in each grid;

\quad 3. Fill up the masked area with galaxies of the average number density and random ellipticities/shear estimators;

% There are two cases, one is for constructing shear map directly, the other is constructing shear map with given redshift plane. The earlier one follows:

\quad 4. Determine a set of orthogonal and complete functions, e.g., Fourier series $f_w(x)$,  to parameterize the shear field as: $g(\vec{x})\sqrt{n(\vec{x})}=\sum_w{a_wf_w(\vec{x})}$, where $w$ is the index of each Fourier mode.  
The rest of the steps are about determining the coefficients $a_w$ one-by-one.

\quad 5. Each $a_w$ is determined by changing its assumed value $\hat{a}_w$, so that the PDF of the whole background galaxy sample is best symmetrized. The pseudo shear signal for a galaxy is given by $\hat{g}(\vec{x})=\hat{a}_wf_w(\vec{x})/\sqrt{n(\vec{x})}$ (without summing over all possible values of $w$). For each galaxy, its weight is $\vert f_w(\vec{x}) \vert /\sqrt{n(\vec{x})}$ as defined in eq.\ref{ni6} with $\alpha=1/2$. The sign of $\hat{e}(=e-\hat{g})$ needs to be inverted wherever $f_w(\vec{x})<0$;

\quad 6. Repeat step 5 to find the values of all $a_w$. Calculate the resulting shear field with $g(\vec{x})=\sum_w{a_wf_w(\vec{x})}/\sqrt{n(\vec{x})}$. The resulting shear field can be converted to the $\kappa$ field through, e.g., the K-S inversion algorithm.




Note that in the PF method, the choice of the pixel size is rather arbitrary. Large pixel size is fine for those who are only interested in the large scale behavior of the shear field. On the other hand, if the pixel size is small, say each pixel only containing a handful of source galaxies on average, the method still works fine. In this case, although the recovered shear field in each pixel would become more noisy, the signal-to-noise ratio of the large scale modes would not be affected. 

\section{Application in real data}
\label{real}

In this section, we demonstrate the performance of the PF method with real data. We adopt the shear estimator of the form defined in the Fourier\_Quad shear measurement method \citep{Zhang2015,Zhang2010}, which has been shown to work well with the PDF-SYM method \citep{Zhang2017}. Note that although the Fourier\_Quad shear estimator takes an unconventional form, it works equally well as the ideal shear estimators discussed in \S\ref{method}, as shown in \cite{Zhang2017}. In other words, the whole discussion in \S\ref{method} can be almost trivially replicated for the Fourier\_Quad shear estimator. 

Our shear catalog comes from the HSC galaxy survey Data Release 2 \citep{Aihara2019}. The images are processed by the Fourier\_Quad pipeline in five bands (grizy) individually. The Fourier\_Quad pipeline is previously used to process the imaging data of CFHTLenS \citep{Zhang2019} and DECaLS \citep{Zhang_2022}, and the results of which have all passed the accuracy test using the field-distortion effect. We find that the same pipeline also performs quite well for the HSC data. The details of the measurement are presented in a separate paper (Liu et al., in preparation). In this work, we use the shear catalog of the 'rizy' four bands. Note that in Fourier\_Quad, shear estimators from different exposures or bands are treated as independent ones even if they are from the same galaxy. 

The foreground lens in our examples are three massive galaxy clusters chosen from the SDSS group catalog \citep{Yang2007sdss,Yang2008sdss,Yang2009sdss}. Their information is shown in Table \ref{cluster3}, including their angular positions, redshift ($z_{lens}$), and mass (from the catalog). To reconstruct the surface density map, we first build up the background shear field at a reference redshift ($z_{lens}+0.3$) following the more general version of the PF method (taking into account the background redshift distribution) in Appendix B, in which a modified version of the step-by-step procedures given in \S\ref{SBS} is also provided. The convergence/density map is then converted from the shear map through the standard procedures given in \cite{KS93}. One complication is that what we actually measure from the background galaxy shapes are not the shear, but the reduced shear \citep{Zhang2011}. To improve the accuracy of the recovered density map, one therefore needs to modify the shear estimators with the factor $1-\kappa$, where $\kappa$ is the convergence just derived from the shear map. Usually with several iterations, the shear map and $\kappa$ map can both achieve stable solutions \citep{Liu2014}.
To avoid possible contamination from the cluster members, we only use the background galaxies with redshifts larger than $z_{lens}+0.1$.  

%Cluster A is about $10^{15.13}$ $M_{s}$, locating at ra:197.873 dec:-1.348, and redshift:0.189. Cluster B is about $10^{14.21}$ $M_{s}$, locating at ra:355.276 dec:-0.003, and redshift:0.190. Cluster C is about $10^{14.47}$ $M_{s}$, locating at ra:210.847 dec:0.128, and redshift:0.184. 

\begin{table}
\centering
\begin{tabular}{|c|c|c|c|} 
\hline
Cluster ID & (RA, Dec) & redshift & mass ($M_{\odot}$)\\
\hline\hline
Cluster A & (197.873,-1.348) & 0.189 & $10^{15.13}$ \\ 
\hline
Cluster B & (355.276,-0.003) & 0.190 & $10^{14.21}$ \\ 
\hline
Cluster C & (210.847,0.128) & 0.184 & $10^{14.47}$ \\ 
\hline
\end{tabular}
\caption{The information of the three clusters used as foreground lenses in \S\ref{real}.}
\label{cluster3}
\end{table}

%To make the reconstructed mass map In the usual cases like real data, we also apply our method on a few large mass, high richness and low redshift clusters from SDSS group catalog\citep{Yang2007sdss,Yang2008sdss,Yang2009sdss}. 

For comparison, the Local\_PDF method, Local\_AVE method and the PF method are implemented. The results are shown in fig.\ref{f_355},\ref{f_197},\ref{f_210} for cluster A, B, and C respectively. All the panels in the figures have the dimension of $1^{\circ}\times 1^{\circ}$. In the grey area of each figure, we show the results of Local\_PDF, Local\_AVE and PF in the first three rows respectively, with grid size of $2.5'\times 2.5'$.

We note that in the Local\_AVE method, we use the ensemble average of the local Fourier\_Quad shear estimators to evaluate the local shear field $g_1$ and $g_2$, as shown in eq.(28) of \cite{Zhang2017}. This approach is usually not encouraged to use for the Fouier\_Quad shear estimators because it uses unnormalized quadruple moments of the galaxy, and therefore variations in galaxy luminosity/size can lead to large scatters in shear recovery (this fact however does not cause any problem in PDF-based methods like Local\_PDF and PF here). Here for the purpose of comparison, we still present the results of Local\_AVE, but excluding 8\% highest and 8\% lowest shear estimators to reduce the noise. We would not recommend doing so in practice, as the selection would inevitably introduce local shear bias.

%\textbf{It is noted that when applying Local\_AVE, we take the local average of part of shear estimators. We do not include 8\% highest and 8\% lowest shear estimators in our calculation.}

The two left columns show the recovered shear fields $g_1$ and $g_2$ of the two methods. The third and fourth columns show the surface density maps from the E ($\Sigma_E$) and B ($\Sigma_B$) modes of the convergence fields respectively.
The rightmost panel (out of the grey region) shows the background galaxy image density used in the same area of the sky. It is clear from the figures that the results of the PF method are generally less noisy than those of Local\_PDF. Some of the extreme values on the shear maps of Local\_PDF are caused by two reasons: 1. lack of background galaxies; 2. outliers in shear catalog. These two factors fortunately do not affect the performance of the PF method. We consider this feature a significant advantage of PF. It allows us to further increase the spatial resolution of the density map. For example, in the bottom rows of fig.\ref{f_355}, \ref{f_197}, and \ref{f_210}, we show the reconstructed density maps using PF with a much finer grid size of $0.5'\times 0.5'$. The final results are smoothed with a Gaussian filter of $\sigma=1.65'$ to increase the visibility of the density structure. In this case, one can see that the central overdensities of the clusters stand out more clearly. In contrast, we find that it is not feasible to further increase the spatial resolution in the method of Local\_PDF. 

%\textbf{The step-by-step procedure of PF method is given as:\\
%\quad 1. Setting spatial bins of source galaxies and filling out the masked area with galaxies of a certain number density.\\
%\quad 2. Calculating the galaxy number density $n(x)$ in each grid.\\
%\quad 3. For each galaxy, its corresponding counting weight is $\vert f_w(\vec{x}) \vert /\sqrt{n(\vec{x})}*\Sigma_c(z_l,z)/\Sigma_c(z_l,z_{s0})$. \\
%\quad 4. Changing the sign of $e$ where $f_w(x)<0$.\\
%\quad 5. By searching for the modes coefficient $\hat{a_k}$ best symmetrizing the PDF of $e*\Sigma_c(z_l,z_{s0})/\Sigma_c(z_l,z)- (a_k-\hat{a_k})\vert f_k(x)/\sqrt{n(x)}\vert$ around 0 with counting weight $\vert f_k(x)\vert/\sqrt{n(x)}*\Sigma_c(z_l,z)/\Sigma_c(z_l,z_{s0})$.\\
%\quad 6. Calculating $\hat{a_k}f_k(x)$, which is the best estimation of $g(\vec{x})\sqrt{n(\vec{x})}$.\\
%\quad 6. Deviding the reconstructed field $g(\vec{x})\sqrt{n(\vec{x})}$ by $\sqrt{n(\vec{x})}$.\\
%\quad 7. Acquring the stable $\kappa$ field by K-S invertion algorithm.\\ }


%All those three are from SDSS group catalog. Cluster A is about $10^{15.13}$ $M_{s}$, locating at ra:197.873 dec:-1.348, and redshift:0.189. Cluster B is about $10^{14.21}$ $M_{s}$, locating at ra:355.276 dec:-0.003, and redshift:0.190. Cluster C is about $10^{14.47}$ $M_{s}$, locating at ra:210.847 dec:0.128, and redshift:0.184. 
% The image from X-ray is shown in fig.\ref{f_abell1835_sdss_pdf}.

%The background galaxies are selected from HSC DR2 shear catalog including 'rizy' four bands, which is generated by Fourier\_Quad data processing pipeline\citep{Zhang2019,2021ApJ...911...10W}. We cut the sources' redshifts $z_s > z_l +0.1$ to avoid photo-z dilution. 

\begin{figure*}
    \centering
    \includegraphics[scale=0.45]{Cluster_355.27638159_-0.0025478_0.189714.pdf}
    \caption{The recovered shear and density maps around Cluster A listed in Table \ref{cluster3}. 
    The first two columns show the $g_1$ and $g_2$ fields for the Local\_PDF method(the first row), Local\_AVE method(the second row) and the PF method (the third row)}. The third and fourth columns show the corresponding surface density fields of E ($\Sigma_E$) and B ($\Sigma_B$) modes respectively. The rightmost column shows the distribution of the background galaxy number density in the same area of the sky. The fourth row shows the density fields recovered also by the PF method, but with a higher spatial resolution.
    \label{f_355}
\end{figure*}
\begin{figure*}
    \centering
    \includegraphics[scale=0.45]{Cluster_197.88264806_-1.3137863_0.1833561.pdf}
    \caption{Same as fig.\ref{f_355}, but for Cluster B listed in Table \ref{cluster3}. }
    \label{f_197}
\end{figure*}
\begin{figure*}
    \centering
    \includegraphics[scale=0.45]{Cluster_dss145hscrizy_210.84751834_0.12812977_0.1837546_14.4698..pdf}
    \caption{Same as fig.\ref{f_355}, but for Cluster C listed in Table \ref{cluster3}.}
    \label{f_210}
\end{figure*}

%The source galaxies are from different redshifts. To regularize the shear value, we suppose to move sources to the $z_{lens}+0.3$ plane by multiplying a factor $\frac{\Sigma_{cr}(z_{lens},z_s)}{\Sigma_{cr}(z_{lens},z_{lens}+0.3)}$. 

%It is noted results by traditional way show some extreme shear values. There are two possible reasons. One is the galaxy number is extremely tiny in that region, the other is some wrongly measured data exist in shear catalog. The new proposed approach presents a relatively lower statistical error and rational shear value. 

%We want to address that PF method is a forward use of PDF-SYM method, which is naturally designed for Fourier\_Quad shear estimators with proved advantages\citep{Zhang2015}. So the different performances about real clusters include both influence of PF itself and using PF together with Fourier\_Quad estimators.




%With measurements of the distortion of background galaxy shapes caused by weak gravitational lensing, we can learn about the mass distribution in the foreground with few physical assumptions and almost not relying on phenomenological models.  Following the convertion in \ref{AppendixC}, we can easily acquire mass surface density to estimate lens mass, which is free of recovery plane or shear ratio.
%\begin{equation}
%\Sigma(r)=\Sigma_{cr}(z_{l},z_{s})\kappa(r)
%\end{equation}


% \begin{figure*}[h]
% \centering
% \includegraphics[scale=1]{part4_massmap_sfcase2.pdf}
% \caption{mass surface density in lens plane and the corresponding B mode of cluster A recoverd by PF method.}
% \label{f_part4_case2}
% \end{figure*}
% \begin{figure*}[h]
% \centering
% \includegraphics[scale=1]{part4_massmap_sfcase4.pdf}
% \caption{mass surface density in lens plane and the corresponding B mode of cluster A recoverd by PF method.}
% \label{f_part4_case4}
% \end{figure*}
% \begin{figure*}[h]
% \centering
% \includegraphics[scale=1]{part4_massmap_sfcase3.pdf}
% \caption{mass surface density in lens plane and the corresponding B mode of cluster A recoverd by PF method.}
% \label{f_part4_case3}
% \end{figure*}

% \begin{figure}[htbp]
% \centering
% \includegraphics[width=\linewidth,height=0.82\linewidth]{part4_massmap_ks17_sf.pdf}
% \caption{Recovered suface density $\Sigma$(left column) and its B mode $\Sigma_{B}$(right column) field of cluster A with traditional way(upper row, local average radius = 4.25 arcmin) and new way(lower row, for fair comparison, convolved with the same filter).}
% \label{f_part4_combine3}
% \end{figure}

%here $r$ is the projected comoving radius from halo center. The critical surface mass density $\Sigma_{cr}$ for a given system of lens cluster and source at redshift $z_l$ and $z_s$, respectively, is given as
%\begin{equation}
%\Sigma_{cr}(z_l,z_s)=\frac{c^2}{4\pi G}\frac{D_{A}(z_{s})}{D_{A}(z_{l})D_{A}(z_l,z_{s})(1+z_l)^2}
%\end{equation}
%Where $D_{A}(z)$ is the angular diameter distance and the factor $(1+z_l)^2$ is from our use of the comoving scale. To eliminate statistical noise, a gauss filter with $\sigma=1.65\ arcmin$ has been convolved in the last step. 

% The green symbols in the $\Sigma$ map are the member galaxies of the cluster, of which the most massive nine are marked with blue symbols. The number rank from 1 to 9 represents the mass rank from large to small. Therefore, it means the BCG center is located in position '1'. The red, black and grey symbols represents the luminosity center, geometric center of all member galaxies and geometric center of most massive nine member galaxies.

% We make this figure to show departure mass center from these centers. $z, N$ and $M$ in the footprint means redshift of the cluster, average galaxy number density, and cluster mass in log unit.



% \begin{figure}[htbp]
% \centering
% \includegraphics[scale=0.5]{decals_densitymap}
% \caption{The galaxy number density of DECALS catalogue around aimed cluster B.}
% \label{f_decals_densitymap}
% \end{figure}

% \begin{figure}[htbp]
% \centering
% \includegraphics[scale=0.65]{cluster_decals}
% \caption{Cluster B's recovery results of KS method(a,b) and SF method(c,d) from DECALS catalogue.}
% \label{f_cluster_decals}
% \end{figure}

% \begin{figure}[htbp]
% \centering
% \includegraphics[scale=0.5]{hsc_densitymap}
% \caption{The galaxy number density of HSC catalogue around aimed cluster B.}
% \label{f_hsc_densitymap}
% \end{figure}

% \begin{figure}[htbp]
% \centering
% \includegraphics[scale=0.65]{cluster_hsc}
% \caption{Cluster B's recovery results of KS method(a,b) and SF method(c,d) from HSC catalogue.}
% \label{f_cluster_hsc}
% \end{figure}

% Cluster B is used to point out that SF method have advantages in keeping the value of recovery in the same level when facing different catalogue. Cluster B weight $15.13\times 10^{10}M_{s}$ and locates in (ra:197.873,dec:-1.348,z=0.189), which can be covered both by HSC and DECALS. The galaxy number density around this cluster are fig.\ref{f_decals_densitymap} and fig.\ref{f_hsc_densitymap} for DECALS and HSC. KS method and SF are implemented for two catalogues desperately. Results produced by DECALS catalogue are in fig.\ref{f_cluster_decals}, where a,b are recovered by KS method and c,d are by SF method. Results produced by HSC catalogue are in fig.\ref{f_cluster_decals}, subfigures sre same with DECALS one. By reconstructing the one cluster, we find traditional way provides a dirty map when processing the low galaxy number DECALS survey. Alos, new way gives a significant peak in both cases.

%\newpage
\section{Conclusion and Discussions}
\label{conclusion}

Weak lensing provides a direct way of reconstructing the foreground density distribution. So far, The existing algorithms mostly focus on refining the conversion from the shear field to the density map. The construction of the shear field from the shear catalog however receives much less attention. Currently, it is done by locally averaging the shear estimators within some smoothing kernel. The choice of the kernel size can be quite nontrivial: on one hand, it should be large enough to guarantee sufficient background galaxies; on the other hand, it is desirable to keep the kernel size small for a good spatial resolution. The conventional method may also induce systematic errors in the shear field due to the coupling between its fluctuation and the inhomogeneity of the background source galaxies. 

In this work, based on the method of PDF-SYM in \cite{Zhang2017}, we propose the PDF-Folding method as an alternative way of recovering the shear field from the shear catalog. The details of the new method is given in \S\ref{method}. It is significantly different from the conventional weighted-sum methods. The main differences include: 

1. Instead of recovering the local shear value, PDF-Folding reconstruct the coefficients of the orthogonal modes that are used to decompose the shear field as a whole. It therefore does not require any smoothing kernels; 

2. Inhomogeneous distribution of the source galaxies (as well as masks) are automatically taken into account in the method without incurring systematic biases; 

3. The statistical errors on the coefficients of the orthogonal modes can reach the theoretical minimum in general observing conditions, i.e., with inhomogeneous source distribution and realistic shape noise.  

The above points are demonstrated with analytical reasoning (see \S\ref{method} and the Appendixes) as well as numerical examples using the Illutris-1-Dark simulation data (see \S\ref{numerical}). As real world examples, in \S\ref{real}, we apply the PDF-Folding method on the shear catalog of the HSC survey (constructed from the Fourier\_Quad pipeline) to reconstruct the surface density maps around three massive galaxy clusters identified in SDSS. Comparing to the local reconstruction of the shear field, we show that PDF-Folding shows a consistently better performance. It is mainly because the new method is much less susceptive to the local 
inadequacy of source galaxies or outliers in shape noise.       

%In this work, we constructed weak lensing convergence maps with newly proposed PF method. We apply the PF method on Illutris-1-Dark simulation data and HSC catalog. The performance was compared with the direct inversion of the shear field, also known as K-S method. For one hand, the recovery quality of traditional way can be influenced by the intrinsic shape noise. For the other, we note that a smoothing of small angular scale is needed in K-S method, which can lead to a systematic bias caused by inhomogenous galaxy distribution. The general existed mask effect is the special case of such inhomogeneity.
%To mitigate mask effect, we propose a solution to patch mask region with random orientation galaxies, which is proved to obtain a unbiased reconstruction by using Illutris-1-Dark data.
%We also test its performance around given lens with HSC catalog generated by Fourier\_Quad and make a comparasion between K-S method's.  

The main caveat in PDF-Folding is in the calculation of $N_{i (>0)}-N_{-i}$ in \S\ref{method}, in which we expand the PDF of the shear estimators to the first order in shear. It implies that the shear signal is small enough comparing to the dispersion of the shape noise. This condition is not necessarily well satisfied in the neighbourhood of massive galaxy clusters, and therefore there can be small amount of systematic errors. This is though not a serious issue, as the very central regions of galaxy clusters can be simply masked if necessary. Such an expansion in PDF may also fail when the PDF itself has singularities. This has been shown as an advantage in the PDF-SYM method discussed in \cite{Zhang2017}, because it can in principle lead to almost infinite signal-to-noise ratio in shear recovery. In PDF-Folding, however, we find that such singularities in the PDF lead to significant systematic errors in the recovered shear field. It is because the failure of the expansion invalidates the cancellation of the couplings between different orthogonal modes. Nevertheless, this problem is neither a serious concern in practice, as realistic PDF of the shear estimators rarely contains singularities. 

Finally, we have not considered the impact of the shear measurement errors on the recovered shear field. One cause of the error is the uncertainty of the point spread function (PSF), which could depend on the local density of the stars, and therefore induce a variation of the shear field. This topic will be studied in a future work.

%Moreover, PSF errors, which is a systematic bias in shear measurement, would bias the recovery of shear map by providing an extra multiplicative bias(m) and addititive bias(c) on shear catalog.}



\acknowledgments
We thank the referee for very useful suggestions, which significantly improve the writing of the paper. We also thank Jiaqi Wang, Haojie Xu, Ji Yao, Xiaokai Chen, Zhiwei Shao and Zhi Shen for useful discussions on many technical details. This work is supported by the National Key Basic Research and Development Program of China (No.2018YFA0404504), the NSFC grants (11621303, 11890691, 12073017), and the science research grants from China Manned Space Project (No. CMS-CSST-2021-A01). The computations in this paper were run on the $\pi$ 2.0 cluster supported by the Center for High Performance Computing at Shanghai Jiao Tong University. 


\bibliography{SF.bib}{}
\bibliographystyle{aasjournal}

\appendix

\section{The Optimal Choice of $\alpha$}

Using the maximum likelyhood method, let us derive the Cramer-Rao Bound on the statistical uncertainties of the coefficients of the shear field. The likelyhood function is simply the PDF of the observed galaxy ellipticities written as:
\begin{align}
P(e)=P_I(e-g(\vx))=P_I(e-\sum_ka_kf_k(\vx)n^{-\alpha}(\vx))
\end{align}
According to the maximum likelihood estimation (MLE), the uncertainty of $a_w$ ($\sigma_{a_w}$) can be written out as:
\begin{eqnarray}
\sigma_{a_w}^{-2}(MLE)&=&-\sum_i\frac{\partial^{2}\ln P(e_i)}{\partial a_w^{2}} = \int d^2\vx\cdot n(\vec{x})\int de P_I^{-1}(e)\left[f_w(\vx)n^{-\alpha}(\vx) P_I'(e)\right]^{2}\\ \nonumber
&=& \int d^2\vx f_w^{2}(\vx)n^{-2\alpha+1}(\vx)\int deP_I^{-1}(e){P_I'}^2(e)
\end{eqnarray}


In the PF method, the value of $a_w$ is estimated by minimizing the $\chi^2$ defined as:
\begin{equation}
\chi^2=\frac{1}{2}\sum_{i(>0)} \frac{(N_i-N_{-i})^2}{\langle(N_i-N_{-i})^2\rangle}
\end{equation}
The denominator in the above formula can be worked out in the following way:
\begin{equation}
\langle(N_{i(>0)}-N_{-i})^2\rangle\approx \langle(\delta N_{i(>0)}-\delta N_{-i})^2\rangle\approx 2\langle(\delta N_{i(>0)})^2\rangle
\end{equation}
in which the $\delta$ sign refers to the deviation from the statistical mean. In this calculation, we have also neglected the influence of the lensing effect, which yields a slightly non-zero mean for $N_{i(>0)}-N_{-i}$. According to eq.(\ref{ni6}), 
\begin{equation}
\label{ni60}
N_{i(>0)}\approx\int d^2\vec{x} \vert f_w(\vec{x}) \vert n^{\alpha}(\vec{x})\int_{u_{i-1}}^{u_i}d\hat{e} P_I(\hat{e})=\int d^2\vec{x} W(\vec{x}) n(\vec{x})\int_{u_{i-1}}^{u_i}d\hat{e} P_I(\hat{e})=\sum_jW(\vec{x}_j)\Delta N(\vec{x}_j)
\end{equation}
in which $W(\vec{x})=\vert f_w(\vec{x}) \vert n^{\alpha-1}(\vec{x})$ is the weight at $\vec{x}$. The last step in the above equation is a discrete version of the integretion.
Assuming that the galaxy number obeys Poisson distribution, we get:
\begin{equation}
\label{ni61}
\langle(\delta N_{i(>0)})^2\rangle=\sum_jW^2(\vec{x}_j)\langle[\delta\Delta N(\vec{x}_j)]^2\rangle=\sum_jW^2(\vec{x}_j)\Delta N(\vec{x}_j)=\int d^2\vec{x} W^2(\vec{x}) n(\vec{x})\int_{u_{i-1}}^{u_i}d\hat{e} P_I(\hat{e})
\end{equation}
Therefore, we have:
\begin{equation}
\langle(N_{i(>0)}-N_{-i})^2\rangle\approx 2\int d^2\vx\cdot f_w^{2}(\vx)n^{2\alpha-1}(\vx)\int_{u_{i-1}}^{u_{i}}d\hat{e}P_I(\hat{e})
\end{equation}
Combining with eq.(\ref{ni50}), we get:
\begin{equation}
\chi^2=(\hat{a}_w-a_w)^2\frac{\left[\int d^2\vx\cdot f_w^{2}(\vx)\right]^2}{\int d^2\vx\cdot f_w^{2}(\vx)n^{2\alpha-1}(\vx)}\sum_{i(>0)} \frac{\left[P_I(u_i)-P_I(u_{i-1})\right]^2}{\int_{u_{i-1}}^{u_{i}}d\hat{e}P_I(\hat{e})}
\end{equation}
In the limit of very small bin size, we can turn the summation in the above formula into an integral form as:
\begin{equation}
\chi^2=\frac{(\hat{a}_w-a_w)^2}{2}\frac{\left[\int d^2\vx\cdot f_w^{2}(\vx)\right]^2}{\int d^2\vx\cdot f_w^{2}(\vx)n^{2\alpha-1}(\vx)}\int deP_I^{-1}(e){P_I'}^2(e) 
\end{equation}
which implies that the error on the parameter $\hat{a}_w$ is:
\begin{equation}
\sigma_{\hat{a}_w}^{-2}(PF)=\frac{\left[\int d^2\vx\cdot f_w^{2}(\vx)\right]^2}{\int d^2\vx\cdot f_w^{2}(\vx)n^{2\alpha-1}(\vx)}\int deP_I^{-1}(e){P_I'}^2(e) 
\end{equation}
According to the Cauchy-Schwarz Inequality, we have:
\begin{equation}
\left[\int d^2\vx\cdot f_w^{2}(\vx)n^{2\alpha-1}(\vx)\right]\cdot\left[\int d^2\vx\cdot f_w^{2}(\vx)n^{1-2\alpha}(\vx)\right]\ge\left[\int d^2\vx\cdot f_w^{2}(\vx)\right]^2,
\end{equation}
therefore, $\sigma_{\hat{a}_w}(PF)\ge\sigma_{a_w}(MLE)$, and the equality is achieved when $\alpha=1/2$. This completes our proof of our statement in \S\ref{orm}.

\section{Reconstruction of the Foreground Surface Density Field}

Let us discuss about reconstructing the surface density distribution $\Sigma(\vec{x},z_l)$ around the foreground lens at redshift $z_l$. The background shear signal is related to the foreground density field via $\kappa(\vec{x},z_s)=\Sigma(\vec{x},z_l)/\Sigma_c(z_l,z_s)$, where the critical surface density $\Sigma_c$ is defined as:
\begin{equation}
\Sigma_c(z_l,z_s)=\frac{c^2}{4\pi G}\frac{D_{A}(z_{s})}{D_{A}(z_{l})D_{A}(z_l,z_{s})(1+z_l)^2}
\end{equation}
Where $D_{A}(z)$ is the angular diameter distance, and the factor $(1+z_l)^2$ is from our use of the comoving scale. To use all the background galaxies to reconstruct the foreground density field $\Sigma(\vec{x},z_l)$, we need to take into account the fact that the background sources are distributed in a range of redshifts. In terms of the shear field, we therefore need to specify at which background redshift we are reconstructing it. Fortunately, under the thin lens assumption, the background shear signal scales with their redshift $z_s$ in a simple enough way through the critical surface density, i.e,  we have:
$g(\vec{x},z_{s1})\Sigma_c(z_l,z_{s1})=g(\vec{x},z_{s2})\Sigma_c(z_l,z_{s2})$. Note that for now we neglect the differences between the shear and the reduced shear. If we choose $z_{s0}$ as our reference background redshift for the reconstructed shear field $g(\vec{x},z_{s0})$, all the background shear at $z_s$ can then be expressed as: $g(\vec{x},z_{s})=g(\vec{x},z_{s0})\Sigma_c(z_l,z_{s0})\Sigma_c^{-1}(z_l,z_{s})$. Following the idea of the PF method, we can expand the reference shear field as $g(\vec{x},z_{s0})\sqrt{n(\vx)}=\sum{a_kf_k(\vx)}$, where $n(\vx)=\int dz\cdot \phi(\vx,z)$, and $\phi(\vx,z)$ is the galaxy number density in the redshift space at angular position $\vx$. To reconstruct $\hat{a}_w$, we should assume that $\hat{g}(\vec{x},z_{s0})\sqrt{n(\vx)}=\hat{a}_wf_w(\vx)$. For any background galaxy at redshift $z_s$, we therefore have: $\hat{g}(\vec{x},z_{s})=\hat{g}(\vec{x},z_{s0})\Sigma_c(z_l,z_{s0})\Sigma_c^{-1}(z_l,z_{s})$, The PDF for the PF method can be formulated as:
\begin{eqnarray}
\label{ni16}
N_i&=&\int_{i*f_w(\vec{x})\ge 0} d^2\vec{x} \int dz\cdot\phi(\vx,z)\vert f_w(\vec{x}) \vert /\sqrt{n(\vec{x})}\frac{\Sigma_c(z_l,z)}{\Sigma_c(z_l,z_{s0})}\int_{u_{\vert i\vert-1}}^{u_{\vert i\vert}}d\hat{e} P(\hat{e})\\ \nonumber
&+&\int_{i*f_w(\vec{x})<0} d^2\vec{x} \int dz\cdot\phi(\vx,z)  \vert f_w(\vec{x}) \vert/\sqrt{n(\vec{x})}\frac{\Sigma_c(z_l,z)}{\Sigma_c(z_l,z_{s0})}\int_{-u_{\vert i\vert}}^{-u_{\vert i\vert-1}}d\hat{e} P(\hat{e}),
\end{eqnarray}
Note that for each background galaxy, we have given it a weight: $\vert f_w(\vec{x}) \vert /\sqrt{n(\vec{x})}*[\Sigma_c(z_l,z)/\Sigma_c(z_l,z_{s0})]$. The lensed PDF can be expanded as:
\begin{eqnarray}
\label{pi12}
&&P(\hat{e})=P_I(\hat{e}+\hat{g}-g)=P_I\left\{\hat{e}+\left[\hat{a}_wf_w(\vec{x})-\sum_k{a_kf_k(\vec{x})}\right]\frac{1}{\sqrt{n(\vx)}}\frac{\Sigma_c(z_l,z_{s0})}{\Sigma_c(z_l,z)}\right\}\\ \nonumber
&\approx&P_I(\hat{e})+\frac{1}{\sqrt{n(\vx)}}\frac{\Sigma_c(z_l,z_{s0})}{\Sigma_c(z_l,z)}\left[\hat{a}_wf_w(\vec{x})-\sum_k{a_kf_k(\vec{x})}\right]P_I'(\hat{e})
\end{eqnarray}
Note that here we assume that the form of $P_I$ does not depend on the redshift. As a result, we have: 
\begin{equation}
\label{ni15}
N_{i (>0)}-N_{-i}\approx2(\hat{a}_w-a_w)\left[P_I(u_i)-P_I(u_{i-1})\right]\int d^2\vec{x} f_w^2(\vec{x})
\end{equation}
The PF method in this definition should therefore yield an unbiased estimate of the parameter $\hat{a}_w$. Following the procedures similar to those in Appendix A, we can get the error on the parameter $\hat{a}_w$ in the limit of small bin size as:
\begin{equation}
\label{err1}
\sigma_{\hat{a}_w}^{-2}(PF)=\frac{\left[\int d^2\vx\cdot f_w^{2}(\vx)\right]^2}{\int d^2\vx\cdot f_w^{2}(\vx)n^{-1}(\vec{x})\int dz\cdot\phi(\vx,z) \left[\Sigma_c(z_l,z)/\Sigma_c(z_l,z_{s0})\right]^2}\int deP_I^{-1}(e){P_I'}^2(e)
\end{equation}
We can further ask whether this form is statistically optimal by finding out the Cramer-Rao bound in this case as:
\begin{eqnarray}
\label{err2}
\sigma_{a_w}^{-2}(MLE)&=&-\sum_i\frac{\partial^{2}\ln P(e_i)}{\partial a_w^{2}} = \int d^2\vx\int dz\cdot\phi(\vx,z)\int de P_I^{-1}(e)\left[f_w(\vx)\frac{1}{\sqrt{n(\vx)}}\frac{\Sigma_c(z_l,z_{s0})}{\Sigma_c(z_l,z)} P_I'(e)\right]^{2}\\ \nonumber
&=& \int d^2\vx f_w^{2}(\vx)n^{-1}(\vx)\int dz\cdot\phi(\vx,z)\left[\frac{\Sigma_c(z_l,z_{s0})}{\Sigma_c(z_l,z)}\right]^2\int deP_I^{-1}(e){P_I'}^2(e)
\end{eqnarray}
From the results of eq.(\ref{err1}) and (\ref{err2}), and using the Cauchy-Schwarz Inequality again, we can show that $\sigma_{\hat{a}_w}(PF)\ge\sigma_{a_w}(MLE)$. Indeed, in general, the equality cannot be achieved without modifying the weighting scheme. 

To further improve our formalism to approach the Cramer-Rao bound, we can change our definition of $n(\vx)$ as: 
\begin{equation}
\label{nxx}
n(\vx)=\int dz\cdot \phi(\vx,z)\Sigma_c^2(z_l,z_{s0})/\Sigma_c^2(z_l,z).
\end{equation}
The weighting in the definition of $N_i$ should be modified accordingly as:
\begin{eqnarray}
\label{ni17}
N_i&=&\int_{i*f_w(\vec{x})\ge 0} d^2\vec{x} \int dz\cdot\phi(\vx,z)\vert f_w(\vec{x}) \vert /\sqrt{n(\vec{x})}\frac{\Sigma_c(z_l,z_{s0})}{\Sigma_c(z_l,z)}\int_{u_{\vert i\vert-1}}^{u_{\vert i\vert}}d\hat{e} P(\hat{e})\\ \nonumber
&+&\int_{i*f_w(\vec{x})<0} d^2\vec{x} \int dz\cdot\phi(\vx,z)  \vert f_w(\vec{x}) \vert/\sqrt{n(\vec{x})}\frac{\Sigma_c(z_l,z_{s0})}{\Sigma_c(z_l,z)}\int_{-u_{\vert i\vert}}^{-u_{\vert i\vert-1}}d\hat{e} P(\hat{e}),
\end{eqnarray}
Note that subtle difference between eq.(\ref{ni17}) and eq.(\ref{ni16}) regarding the positions of the critical surface densities. Once we adopt this formalism, it is straightforward to show that
\begin{equation}
\sigma_{\hat{a}_w}^{-2}(PF)=\sigma_{a_w}^{-2}(MLE)=\int d^2\vx f_w^{2}(\vx)\int de P_I^{-1}(e){P_I'}^2(e).
\end{equation}

In practice, however, we find that the two types of weighting schemes defined in eq.(\ref{ni17}) and eq.(\ref{ni16}) do not lead to significant differences in the final results. For this reason, our results in \S\ref{real} are produced with the PF version defined in eq.(\ref{ni16}), which takes a more straightforward definition of $n(\vx)$. The step-by-step description of shear field reconstruction is a modified version of \S\ref{SBS}:

\quad 1. Set up a rectangular grid for the shear field, and determine a reference background redshift $z_{s0}$, at which the shear field is reconstructed;

\quad 2. Count the galaxy number density $n(\vec{x})$ in each grid;

\quad 3. Fill up the masked area with galaxies of the average number density and random ellipticities/shear estimators, and assign the reference redshift $z_{s0}$ to all of them;

\quad 4. Determine a set of orthogonal and complete functions, e.g., Fourier series $f_w(x)$,  to parameterize the shear field as: $g(\vec{x},z_{s0})\sqrt{n(\vec{x})}=\sum_w{a_wf_w(\vec{x})}$, where $w$ is the index of each Fourier mode.  
The rest of the steps are about determining the coefficients $a_w$ one-by-one. 

\quad 5. Each $a_w$ is determined by changing its assumed value $\hat{a}_w$, so that the PDF of the whole background galaxy sample is best symmetrized. The pseudo shear signal for a galaxy of redshift $z$ is given by $\hat{g}(\vec{x},z)=\hat{a}_wf_w(\vec{x})/\sqrt{n(\vec{x})}\cdot\Sigma_c(z_l,z_{s0})/\Sigma_c(z_l,z)$ (without summing over all possible values of $w$). $z_l$ is the redshift of the lens. For each galaxy, its weight is $\vert f_w(\vec{x}) \vert /\sqrt{n(\vec{x})}\cdot\Sigma_c(z_l,z)/\Sigma_c(z_l,z_{s0})$. The sign of $\hat{e}(=e-\hat{g})$ needs to be inverted wherever $f_w(\vec{x})<0$;

\quad 6. Repeat step 5 to find the values of all $a_w$. Calculate the resulting shear field (at the reference redshift $z_{s0}$) with $g(\vec{x},z_{s0})=\sum_w{a_wf_w(\vec{x})}/\sqrt{n(\vec{x})}$. The resulting shear field can be converted to the $\kappa$ field through, e.g., the K-S inversion algorithm, and the $\kappa$ field can be further converted to the surface density distribution. 


A point to note is about our assumption that the form of the PDF $P_I$ does not depend on the redshift. This is not true in practice. This fact would require us to further change the weightings defined in eq.(\ref{nxx}) and (\ref{ni17}) to take into account the redshift-dependent shape noise, for the purpose of achieving the optimal statistical uncertainty. We may study this issue in a future work. Our current formalism of PF is at least close to the optimal form. It is also not hard to show, e.g., with eq.(\ref{ni15}), that even if $P_I$ depends on redshift, the PF method would not generate systematic biases at the first order of shear.  



% \section{an extension:eliminating the influence of sources' redshift distribution}
% \label{B3}
% After defining the galaxy number density as $n(\Vec{x})=\int dz\cdot \phi(\Vec{x},z)\Sigma_{C}^{2}(z_l,z_{s0})/\Sigma_{C}^{2}(z_l,z_s)$, where $\phi(\Vec{x},z)$ is the galaxy number density in three-dimension space, we suppose $\Delta(\Vec{x})\sqrt{n(\Vec{x})}=\sum_{i=1}^{N}a_{i}f_{i}(\Vec{x})$, which $f_{i}(\Vec{x})$ are orthogonal functions. Here, $\Delta(\Vec{x})$ actually equals $g(\Vec{x})\Sigma_{C}(z_l,z_s)/\Sigma_{C}(z_l,z_{s0})$. If we do not consider the background galaxy redshift distribution, $\Delta(\Vec{x})$ is $g(\Vec{x})$ exactly. Supposing we want to recover $a_{k}$ for a specific value of $k$, we need weighting the galaxy number with $\vert f_{k}(\Vec{x})\vert$ at the location of the galaxy. 

% \begin{align}
% N_{i} 
% %=&\{\int_{f_{k}(x)>0}dx\cdot \int_{u_{i-1}}^{u_{i}}d\widehat{G}+\int_{f_{k}(x)<0}dx\cdot \int_{-u_{i}}^{-u_{i-1}}d\widehat{G}\}\\
% %&\int dz\cdot\phi(x,z)\cdot \Sigma_{C}^{-1}(z)\cdot \vert f_{k}(x)\vert/\sqrt{N(x)}\int dB\cdot \\
% %&P[\widehat{G}+(\widehat{a_k}f_{k}(x)/\sqrt{N(x)}-\Delta(x))\cdot\Sigma_{C}^{-1}(z)\cdot B,B] \\
% &=\int_{f_{k}(\Vec{x})>0}dx\int dz\cdot\phi(\Vec{x},z) \frac{\Sigma_{C}(z_l,z_{s0})}{\Sigma_{C}(z_l,z_s)} \vert f_{k}(\Vec{x})\vert/\sqrt{n(\Vec{x})}\int_{u_{i-1}}^{u_{i}}d\widehat{e} P[\widehat{e}+(\widehat{a_k}f_{k}(\Vec{x})/\sqrt{n(\Vec{x})}-\Delta(\Vec{x}))\frac{\Sigma_{C}(z_l,z_{s0})}{\Sigma_{C}(z_l,z_s)}]\\ \nonumber
% &+\int_{f_{k}(\Vec{x})<0}dx\int dz\cdot\phi(\Vec{x},z) \frac{\Sigma_{C}(z_l,z_{s0})}{\Sigma_{C}(z_l,z_s)}\vert f_{k}(\Vec{x})\vert/\sqrt{n(\Vec{x})}\int_{u_{i-1}}^{u_{i}}d\widehat{e}P[\widehat{e}-(\widehat{a_k}f_{k}(\Vec{x})/\sqrt{n(\Vec{x})}-\Delta(\Vec{x}))\frac{\Sigma_{C}(z_l,z_{s0})}{\Sigma_{C}(z_l,z_s)}]
% \end{align}
% then,
% \begin{align}
% N_{i}-N_{-i}
% %&=2\{\int_{f_{k}(x)>0}dx\vert f_{k}(x)\vert\/\sqrt{N(x)}[\widehat{a}_{k}f_{k}(x)\/\sqrt{N(x)}-\\
% %&\Delta (x)]N(x)-\int_{f_{k}(x)<0}dx\vert f_{k}(x)\vert/\sqrt{N(x)}[\widehat{a}_{k}f_{k}(x)/\\
% %&\sqrt{N(x)}-\Delta (x)]N(x)\}\int dBB[P(u_{i},B)-P(u_{i-1},B)]\\ \nonumber
% =2(\widehat{a}_{k}-a_{k})\int dx\vert f_{k}(\Vec{x})\vert^{2}[P(u_{i})-P(u_{i-1})]
% \label{num_dif2}
% \end{align}
% Where $P(u_i)$ is the normalized PDF of the galaxy distribution function. Smilar as Appendix B,
% \begin{align}
% N_i=\int dx\int dz\cdot \phi(\Vec{x},z) \frac{\Sigma_{C}(z_l,z_{s0})}{\Sigma_{C}(z_l,z_s)}\vert f_{k}(\Vec{x})\vert/\sqrt{n(\Vec{x})}\int_{u_{i-1}}^{u_i}d\widehat{e}P(\widehat{e})
% \end{align}
% %\begin{equation}
% %\delta n_i = \sum_{j}\Sigma_{C}^{-1}(z_j)\vert f_{k}(x_{j})\vert /\sqrt{N(x_j)}\delta L^{i}(j)
% %\end{equation}
% %\begin{align}
% %<(\delta n_i - \delta n_{-i})^2> = \sum_{j}\Sigma_{C}^{-2}(z_j)f_{k}^{2}(x_j)/N(x_j)\{<[\delta L^i (j)]^2>+<[\delta L^i (j)]^2>\}
% %\end{align}
% Assuming it obeys Poisson distribution,
% \begin{align}
% %&<(\delta n_i - \delta n_{-i})^2> 
% N_i + N_{-i}
% %= 2\int dx \int dz\cdot \phi(x,z)\cdot \Sigma_{C}^{-2}(z)\cdot f_{k}^{2}(x)\\
% %&/N(x)\int_{u_{i-1}}^{u_{i}}d\widehat{G}\int dB\cdot P(\widehat{G},B)\\
% =2\int dx f_{k}^{2}(\Vec{x})\int_{u_{i-1}}^{u_{i}}d\widehat{e}P(\widehat{e})
% \end{align}
% Thus, in the limit of very small bin size, we have:
% \begin{equation}
% \chi^{2}=\frac{1}{2}(\widehat{a}_{k}-a_k)^{2}[\int dx f_{k}^{2}(\Vec{x})]\int de\frac{[P^{'}(e)]^{2}}{P(e)}
% \end{equation}




% the PDF of $\widehat{e}$ can be defined through the intrinsic estimator $e^{S}$ as:
% \begin{align}
% P(\widehat{e})&=\int de^{S} P(e^{S})\delta_{D}[\widehat{e}-e^{S}-(g-\widehat{g})] \\ \nonumber
% &=P(\widehat{e}+(\widehat{g}-g))
% \end{align}
% with $P(\widehat{e})$, we can count the $i$th bin's galaxy number by integrating $\widehat{e}$ over $u_{i-1}$ to $u_i$ and $g$ by multiplying $f(g)$, the distribution of $g$:
% \begin{align}
% 	n_{i}&=\int dg\cdot f(g)\cdot\int_{u_{i-1}}^{u_{i}}d\widehat{e}P(\widehat{e}) \\ \nonumber
% &= \int dg\cdot f(g)\cdot\int_{u_{i-1}}^{u_{i}}d\widehat{e}P[\widehat{e}+(\widehat{g}-g)]
% \end{align}

% furthermore,
% \begin{align}
% n_{i}-n_{-i}
% %&=\int dg\cdot f(g)\cdot \int dB[\int_{u_{i-1}}^{u_{i}}d\widehat{G}P[\widehat{G}+(\widehat{g}-g)B,B]\\ \nonumber
% %&-\int_{u_{-i}}^{u_{-i+1}}d\widehat{G}P[\widehat{G}+(\widehat{g}-g)B,B]]\\\nonumber
% \approx 2\int dg\cdot f(g)\cdot (\widehat{g}-g)[P(u_{i})-P(u_{i-1})] \nonumber
% \end{align}
% To estimate the value of $\widehat{g}$ that can maximally symmetrize the distribution of $x_i-\widehat{g}$ with respect to zero, we form the $\chi^{2}$ as follows:
% \begin{equation}
% \chi^2=\frac{1}{2}\sum_{j>0}\frac{(n_j-n_{-j})^2}{n_j+n_{-j}}
% \end{equation}
% Where we assume that the fluctuation of $n_j$ obeys Poisson statistics, so that $<(n_j-n_{-j})^2>\approx n_j + n_{-j}$. $\widehat{g}$ is estimated by minimizing $\chi^{2}$.
% Consequently, we have:
% \begin{align}
% 	\chi^{2}=2\sum_{i>0}\frac{\{\int dg\cdot f(g)\cdot (\widehat{g}-g)[P(u_{i})-P(u_{i-1})] \}^{2}}{\int_{u_{i-1}}^{u_{i}}de\cdot P(e)} 
% \end{align}
% If the shear distribution $f(g)$ does not correlate with the intrinsic galaxy morphology distribution, we have:
% \begin{align}
% 	\chi^{2}=2[\int dg&\cdot f(g)\cdot(\widehat{g}-g)]^{2}\sum_{i>0}\frac{\{[P(u_{i})-P(u_{i-1})]\}^{2}}{\int_{u_{i-1}}^{u_{i}}de\cdot P(e)}
% \end{align}
% In which we find that when $\widehat{g}=[\int dg\cdot g\cdot f(g)]/[\int dg\cdot f(g)]$, $\chi^{2}$ reaches its minimum.

% \section{recovering shear signal with PDF-SYM in Fourier\_Quad shear catalog}

% Fourier\_Quad proposed in \citep{2008MNRAS.383..113Z,2015JCAP...01..024Z} is a promising shear measurement method that is independent on model assumptions. In Fourier\_Quad, shear estimators($G_1,G_2,N,U,V$) are defined in Fourier space. They can be calculated from the galaxy image.  We establish a convergence reconstrution method in frame of Fourier$\_$Quad, but initially, we need to define some estimators in Fourier space of galaxy image.
% \begin{equation}
% 	N=\int{d^{2}\vec{k}(k^{2}-\frac{\beta^{2}}{2}k^{4})T(\vec{k})M(\vec{k})}
% \end{equation}
% \begin{equation}
% 	T(\vec{k})=\vert\widetilde{W}_{\beta}(\vec{k})\vert^{2}/\vert\widetilde{W}_{P}(\vec{k})\vert^{2}
% 	\label{1}
% \end{equation}

% \begin{equation}
% 	U=-\frac{\beta^{2}}{2}\int d^{2}\vec{k}(k_{x}^{4}-6k_{x}^{2}k_{y}^{2}+k_{y}^{4})T(\vec{k})M(\vec{k})
% \end{equation}

% \begin{equation}
% 	G_{1}=-\frac{1}{2}\int d^{2}\vec{k}k_{x}k_{y}T(\vec{k})M(\vec{k})
% \end{equation}

% \begin{equation}
% 	G_{2}=-\int d^{2}\vec{k}k_{x}k_{y}T(\vec{k})M(\vec(k))
% \end{equation}

% $T(\vec{k})$ is used to convert the original PSF $W_P$ to the Gaussian PSF $W_{\beta}$, so the effect of PSF can be corrected rigorously and model-independently.
% $\beta$ is scale factor which should be a little larger than scale of $W_{P}(x)$ to avoid the singularities in the factor $T(k)$. $M(\vec(k))$ is the galaxy power spectrum, $S$ means source and $B$ means background:
% After considering the background and the Poisson noise, M(k) becomes:

% \begin{equation}
% 	M(\vec{k})=\vert\widetilde{f}^{S}(\vec{k})\vert^{2}-F^{S}-\vert\widetilde{f}^{B}(\vec{k})\vert^{2}+F^{B},
% \end{equation}

% $\widetilde{f}^{S}(\vec{k})$ is the galaxy image in Fourier space, and $\widetilde{f}^{B}(\vec{k})$ is the Fourier transformation of the galaxy's neighboring background noise. $F^{S}$ and $F^{B}$ give a correction because of the Poisson noise power spectra on the source and background images.$U$ is an estimator taking into account the parity properties of the shear estimators
% In frame of Fourier\_Quad, the shear signal $\hat{g}_{1,2}$ can be recovered by shear estimators in two ways.
% One is to take the ensemble averages of the shear estimators directly:
% \begin{equation}
% \frac{G_{1}}{N}=g_{1}+O(g_{1,2}^{3}),\frac{G_{2}}{N}=g_{2}+O(g_{1,2}^{3}).
% \end{equation}
% Another is by constructing the PDF of $\widehat{G}_{1,2}=G_{1,2}-\hat{g}_{1,2}(N+U)$ and adjust the value of $\hat{g}_{1,2}$ until the PDF become symmetrization with zero, which is called PDF-SYM. Here $\hat{g}_{1,2}$ is the pseudo shear signal, which is the assumed value to correct the anisotropic part cause by shear in $G_{1,2}$.  It has been proved in \citep{2017ApJ...834....8Z} that PDF-SYM method recovers the shear signal without introducing systematic biases. With the promising shear recovery method, we propose a convergence recovery method called shear-flipping method in this paper. For convience, hereafter we drop the subindex for the shear component and define $B=N+U$. 

%\section{A introduction to K-S convertion}
%\label{AppendixC}
%The lensing convergence is related by the foreground inhomogeneous density distribution. Integrating over the comoving distance $\omega$, the weak lensing convergence $\kappa$ can be defined as:
%\begin{equation}
%\kappa(\vec{\phi})=\frac{3H_{0}^{2}\Omega_{m}}{2}\int_{0}^{\infty}[\int_{0}^{\omega}d\omega^{'}\frac{\omega^{'}(\omega-\omega^{'})}{\omega}\frac{\delta(\vec{\phi},\omega^{'})}{a(\omega^{'})}]n(\omega)d\omega
%\end{equation}

%Where $n$ is a given distribution of source galaxies in radial comoving distance $\omega$. $\vec{\phi}$ is the position vector, $H_{0}$ is the present value of the Hubble parameter,  $a$ is the cosmological scale factor, $\Omega_{m}$ is the matter density parameter, and $\delta$ is the overdensity. In matrix notation, the observed shear $\gamma$ can be related by $\kappa$ as a linear model:
%\begin{equation}
%\label{matrixnota}
%\gamma = \textbf{A}\kappa + \textbf{n}
%\end{equation}

%Where $\textbf{n}$ is a noise vector and $\textbf{A}$ is described in \cite{2018MNRAS.479.2871J}. It shows a consistent form of eq.\ref{matrixnota} as:
%\begin{equation}
%\label{basicrelation}
%\gamma(\vec{\phi})=\int d^{2}\phi^{'}D(\vec{\phi}-\vec{\phi^{'}})\kappa(\vec{\phi})
%\end{equation}
%Where $D(\vec{\phi})=-\frac{1}{\pi}\frac{1}{(\phi_{1}-i\phi_{2})^{2}}$ is used to calculate $\gamma_{t}$, which is the tangential part of \textbf{$\gamma$}. $\textbf{A}$ is a discretized version of $D(\vec{\phi})$, which is the kernel function of popular K-S method.

%We take $\kappa(\vec{\phi})$ into 2-D fourier space and get the $\kappa(k)$:
%\begin{equation}
%\kappa(k)=\int d^{2}\phi\kappa(\vec{\phi})\exp(i\vec{\phi}\cdot k)
%\end{equation}
%then eq.\ref{basicrelation}, the basic convergence-to-shear relationship, can be inducted in fourier space version as:
%\begin{equation}
%\label{ktg}
%\gamma(k)=\pi^{-1}D(k)\kappa(k)
%\end{equation}
%where the kernel $D(k)$ is the Fourier transformation of $D(\vec{\phi})$. It shows as:
%\begin{equation}
%D(k)=\pi\frac{k_{1}^{2}-k_{2}^{2}+2ik_{1}k_{2}}{\vert k\vert^{2}}
%\end{equation}
%here $k_{1}$ and $k_{2}$ are the components of $k$. From equation.\ref{ktg}, after transformation we can get the shear-to-convergence relationship as:
%\begin{equation}
%\label{gtk}
%\kappa(k)=\pi^{-1}\gamma(k)D^{\dagger}(k)
%\end{equation} 
%Where $DD^{\dagger}=\pi^{2}$. Specially noted, here requiring $k\neq 0$ due to mass sheet degeneracy. The inverse Fourier transform of $\kappa(k)$ then returns the convergence reconstruction in configuration space, which has the real and imaginary parts representing the $E$ and $B$ modes. In standard cosmology, the $\kappa$ field is totally sourced by inhomogenous density field, which should be a pure $E$ mode. It is the presence of noise part or other systematic errors lead to the B mode of $\kappa$ field.
% By the way, iteration technical is referred as \citep{2014ApJ...784...31L}.
\end{document}

\subsection{details of reconstructing of a continuous shear field}
\label{B1}
From the defination, we know the galaxy number in the $i$th bin is:
\begin{align}
\label{B11}
n_{i}=\int_{f_k(x)>0}dx\vert f_k(x) \vert \int_{u_{i-1}}^{u_{i}}d\widehat{e} P[\widehat{e}+(\widehat{a}_{k}f_{k}(x)-g(x)),x] \\ \nonumber 
+ \int_{f_k(x)<0}dx \vert f_{k}(x)\vert \int_{-u_{i}}^{-u_{i-1}}d\widehat{e}P[\widehat{e}+(\widehat{a}_{k}f_{k}(x)-g(x)),x]
\end{align}
Where we use: 
\begin{align}
\label{B12}
\int_{f_k(x)<0}dx \vert f_{k}(x)\vert \int_{-u_{i}}^{-u_{i-1}}d\widehat{e}P[\widehat{e}+(\widehat{a}_{k}f_{k}(x)-g(x)),x]= \\ \nonumber 
\int_{f_k(x)<0}dx \vert f_{k}(x)\vert \int_{u_{i-1}}^{u_{i}}d\widehat{e}P[\widehat{e}-(\widehat{a}_{k}f_{k}(x)-g(x)),x]
\end{align}
and consequently get:
\begin{align}
n_i-n_{-i}=2\{\int_{f_{k}(x)>0}dx\vert f_k(x) \vert[\widehat{a}_{k}f_{k}(x)-g(x)]-\int_{f_{k}(x)>0}dx\vert f_k(x) \vert[\widehat{a}_{k}f_{k}(x)-g(x)]\} [P(u_i)-P(u_{i-1})]
\end{align}
following the defination of $\chi^{2}$, we can get the final result:
\begin{align}
\chi^{2}=2[\int dx (\widehat{a}_k-a_k)\vert f_k(x)\vert^{2}-(\int_{f_k(x)>0}dx-\int_{f_k(x)>0}dx)\vert f_{k}(x)\vert(\sum_{i\neq k}a_{i}f_{i}(x))]^2\sum_{i>0}\frac{\{[P(u_i)-P(u_{i-1})]\}^{2}}{\int_{u_{i-1}}^{u_i}d\widehat{e} P(\widehat{e})}
\label{A4}
\end{align}
where the second term equals to zero.
Where $P(u_i)$ is the normalized PDF of the galaxy distribution function.
\begin{align}
n_i=\int dx\vert f_{k}(x)\vert/\sqrt{N(x)}\int_{u_{i-1}}^{u_i}d\widehat{e} P(\widehat{e},x) =\sum_{j}\vert f_{k}(x_j)\vert/\sqrt{N(x_j)}L^{i}(x_{j})
\end{align}
where $L^{i}(x_{j})=\Delta_x \cdot \int_{u_{i-1}}^{u_i}d\widehat{e}P(\widehat{e},x_j)$. We can get:
%\begin{equation}
%\delta n_i = \sum_{j}\vert f_{k}(x_{j})\vert /\sqrt{N(x_j)}\delta L^{i}(j)
%\end{equation}
%and 
%\begin{align}
%\langle(\delta n_i - \delta n_{-i})^2\rangle = \sum_{j}f_{k}^{2}(x_j)/N(x_j)\{\langle[\delta L^i (x_j)]^2+[\delta L^i (x_j)]^2\rangle\}
%\end{align}
Assuming it obeys Poisson distribution, so that $\langle(n_j-n_{-j})^{2}\rangle\approx n_j+n_{-j}$. We have:
\begin{align}
%&\langle(\delta n_i - \delta n_{-i})^2\rangle \\ \nonumber
& n_i + n_{-i}=\sum_{j}f_{k}^{2}(x_j)/N(x)\{L^{i}(x_j)+L^{-i}(x_j)\}=2\int dx\cdot f_{k}^{2}(x)\int_{u_{i-1}}^{u_{i}}d\widehat{e}P(\widehat{e})
\end{align}
Thus, in the limit of very small bin size, we have:
\begin{align}
\chi^{2}=(\widehat{a}_{k}-a_k)^{2}[\int dx\cdot f_{k}^{2}(x)] \sum_{i>0}\frac{ [\widehat{P}(u_i)-\widehat{P}(u_{i-1})]^{2}}{\int_{u_{i-1}}^{u_i}d\widehat{e}P(\widehat{e})}
\end{align}
The following calculation shows that SF method can approach the C-R bound by symmetrizing the PDF of the data in the limit of small bin size. We know that when $\hat{a}=a$, $\chi^{2}$ reaches its minimum, meaning that the best fit value of $\hat{a}$ is an unbiased estimator of $a$. $\chi^{2}$ can be rewritten as:
\begin{equation}
\chi^{2}=\frac{(\hat{a}-a)^2}{2\sigma^{2}_{\hat{a}}}
\end{equation}
where 
\begin{equation}
\label{sig1}
\sigma^{-2}_{\hat{a}}=\int dx\cdot f_{k}^{2}(x)\sum_{i>0}\frac{[\widehat{P}(u_i)-\widehat{P}(u_{i-1})]^{2}}{\int_{u_{i-1}}^{u_i}d\widehat{e}P(\widehat{e})}
\end{equation}
 Note that in this case, we have $-\sum_{j}\frac{\partial^{2}\ln P(e^{j},x^j)}{\partial a_{i}\partial a_k}=0$.



%Photons leaving distant galaxies travel through the universe, distorted by overdensities and underdensities along the line of sight, which include both visible matter and non-visible dark matter. The ongoing and upcoming weak gratitional lensing surveys such as DES\citep{troxel18}, KIDS\citep{Hildebrandt2016}, HSC\citep{Hikage2019}, Euclid\citep{Laureijs2011}, and LSST\citep{Abell2009} are aimed at recovering the weak lensing signal from these minor distortions which is called shear. One scientific goal of the the ongoing and upcoming is to study the large structure of the matter field by measuring this weak distortion on the galactic images. 

%Mass maps can also enable simple tests for systematic error in the galaxy shape catalogs. Therefore, mass maps have been a standard product of large weak lensing surveys.\citep{2018MNRAS.475.3165C,2018}
%But, from galaxies to the useful mass map, the processes are not obviously direct. Three steps are needed: 1.from CCD image to shear catalogs, 2.from shear catalogs to shear map, 3.from shear map to mass map. We discuss these three respectively.

%The first step is always called shear measurement. By precisely measuring shear of each galaxy, we collect it and generate shear catalogs. Many shear measurement methods has been proposed, such as KSB+\citep{1998ApJ...504..636H}, Metacalibration\citep{Sheldon2017}, Lensfit\citep{Miller2013}, and Fourier\_Quad\citep{Zhang2015}... Some of them focusing on model fitting or machine learning, some of them focusing on quadrupole\citep{Mandelbaum2015}.

%The last step is also researched a lot because 

%Many methods have been proposed to reconstruct the mass map, including Wiener filtering, Sparse priors\citep{2014MNRAS.440.1281L,2021ApJ...916...67L,2021MNRAS.506.3678P}, DeepMass\citep{2020MNRAS.492.5023J} and the widely used Kaiser\&Squires methods\citep{1995ApJ...449..460K}. 

%Constructing The second step, is definitely an important step as for the quality of shear map directly determines the quality of mass map, meaning the cosmological information about small scale structure trustworthy or not. Traditionally, shear map was calculated by shear value average over nearing galaxies. 



%K-S method is a classic algorithm that get shear map by using an smooth filter firstly and then convert shear map into convergence map directly.  The main difficulties associated with K-S method are the treatment of the mask effects and shape noise, which will be presented in details later. Generally, regions where the shear values is unable to precisely estimated are masked out, which causes a biased estimation nearing the mask regions. 

% We also demonstrate its performance with real shear catalog around foreground clusters. by using HSC presented by Fourier\_Quad pipeline(the reference paper about pipeline is in preparation).

%Section 3 and 4, we present the main reconstrution results on simulation data and real cases with comparison between the traditional way, showing the influence of galaxy number distribution and existence of mask. We give the disscusion and conclusion in the end.

%Supposing the given sources are with the same shear signal $g$, and the galaxy population has an intrinsic ellipticity distribution. With each galaxy, we can estimate a shear estimator .which directly cause the shear estimator $G$ with a data dependent probability distribution function $P(\widehat{G}_{1,2})$. 

%The observable $G$ is related to $G$ with $G=G-\widehat{g}B$, where $\widehat{g}$ is the assumed shear value. From the defination, we know $\widehat{G}=G-gB$ and the PDF $P(\widehat{G},B)=P(-\widehat{G},B)$. We need to claim the differences among $G, \widehat{G}$ and $G$. That is, $G$ representing the real value of the shear component, whcih is unobservable. $\widehat{G}$ is the unobservable intrinsic value of shear component, which is free of shear. $G$ is the observable value. Since we can relate $\widehat{G}$ to $G$ as:
%
%\begin{equation}
%	\widehat{G}=G+(\widehat{g}-g)B
%\end{equation}

%We can count the $i$th bin's galaxy number by integrating $\widehat{e}$ over $u_{i-1}$ to $u_i$:
%\begin{align}
%n_{i}= \int_{u_{i-1}}^{u_{i}}d\widehat{e%} P[\widehat{e}+(\widehat{g}-g)]
%\label{eq2}
%\end{align}

%furthermore,
%\begin{align}
%&n_{i}-n_{-i}\\ \nonumber
%%&=\int dg\cdot f(g)\cdot \int dB[\int_{u_{i-1}}^{u_{i}}d\widehat{G}P[\widehat{G}+(\widehat{g}-g)B,B]\\ \nonumber
%%&-\int_{u_{-i}}^{u_{-i+1}}d\widehat{G}P[\widehat{G}+(\widehat{g}-g)B,B]]\\\nonumber
%&\approx 2\int dg\cdot f(g)\cdot (\widehat{g}-g)\int dBB[P(u_{i},B)-P(u_{i-1},B)] \nonumber
%\end{align}
%\cite{2017ApJ...834....8Z} shows that to estimate the value of $\widehat{g}$ we can maximally symmetrize the full probability distribution function of the shear corrected estimators $\widehat{e}$ with respect to zero by minimizing $\chi^{2}$, which is defined as:
%\begin{equation}
%\chi^2=\frac{1}{2}\sum_{j>0}\frac{(n_j-n_{-j})^2}{<(n_j-n_{-j})^2>}=\frac{1}{2}\sum_{j>0}\frac{(n_j-n_{-j})^2}{n_j+n_{-j}}
%\end{equation}
%Where we assume that the fluctuation of $n_j$ obeys Poisson statistics, so that $<(n_j-n_{-j})^2>\approx n_j + n_{-j}$. Given different value of $\widehat{g}$ corresponds with different value of $\chi^2$. The best estimated $\widehat{g}$ exactly leads to the minimum value of $\chi^{2}$.
% Consequently, if the shear distribution $f(g)$ does not correlate with the intrinsic galaxy morphology distribution, we have:
% \begin{align}
% \label{chi2no1}
% &\chi^{2}= \\ \nonumber
% & 2[\int dgf(g)(\widehat{g}-g)]^{2}\sum_{i>0}\frac{\{\int dBB[P(u_{i},B)-P(u_{i-1},B)]\}^{2}}{\int dB\int_{u_{i-1}}^{u_{i}}dA\cdot P(A,B)}
% \end{align}
% We find that when $\widehat{g}=[\int dg\cdot g\cdot f(g)]/[\int dg\cdot f(g)]$, $\chi^{2}$ reaches its minimum.
%PDF-SYM proposed in \cite{2017ApJ...834....8Z} has been applied on calculating single shear value, shear-shear correlation, and galaxy-galaxy lensing. But in terms of shear field, it seems still lack a useful way now. 

%Their PDF perform as the dotted line in fig.\ref{fig:ex2}. To recover the shear field, the question becomes if $a,b$ can be estimated from shear catalog? $b$ can be estimated with symmetrizing $P(\widehat{e})=P(e_S+ax+b-\widehat{b})$, but after $b$ fixed, any given $a$ returns a symmetrized $P(\widehat{e})$. This cause the problem as the left panel in fig.\ref{fig:ex2} shows, any given $a$ only causes $P(\widehat{e})$ not symmetrizing with -0.6 and 0.6, but the total PDF, blue part plus red part, is still symmetrizing with 0. We notice any coefficients of odd symmetric terms are not successfully estimated. The const term $b$ can be estimated by PDF-SYM as the right panel in fig.\ref{fig:ex2} shows. It turns out that to get $a$ we need another way. The main difficulty is that the wrong estimate also gives a symmetric total PDF. It is not hard to imagine fliping one-side-of-zero PDF to another side to let PDF-SYM work. It is explicity to know which galaxies' ellipticities need to be flipped to another side. We point out after galaxies with $x<0$, the blue one, are flipped, we get parameter $a$ in a generally form. Only the truely estimated $a$ give a symmetrized PDF of all galaxies. We name this way PDF-Folding(PF).

 
%Consequently, we can furthur get $\chi^{2}$, refer as Appendix \ref{B1}. The weight term $\vert f_k(x)\vert$ ensures the second term in \ref{A4} being zero for $\vert f_k(x)\vert$ and $\sum_{i\neq k}a_{i}f_{i}(x)$ orthogonal with each other when integrating with $x$. Following this way, every $\widehat{a}_k$ is estimated free of bias. Meanwhile, the fitted process is approaching C-R bound, which lead to a optimal estimated $g(x)$.

% \begin{figure}
%     \centering
%     \includegraphics[width=\linewidth,height=0.5\linewidth]{simple_example_1.pdf}
%     \caption{Left panel:hundreds of galaxies randomly distributed in range of [-1.0,1.0], galaxies number distribution in x$<$0 are drawn in blue line, in x$\geq$0 are in red line. Right panel: The PDF of galaxies' ellipticity, blue or red line corresponding with galaxies in x$<$0 or x$\geq$0.}
%     \label{fig:ex1}
% \end{figure}

% \begin{figure}[htbp]
% \centering
% \includegraphics[scale=0.5]{G1_hist}
% \caption{PDF of the shear estimator for given galaxy sample.}
% \label{f_G1_hist}
% \end{figure}

% \begin{figure}[htbp]
% \centering
% \includegraphics[scale=0.5]{diff_pdf_hist}
% \caption{From top to bottom, the PDF of residuals are generated by local-average, local-pdf-sym(using pdf-sym to seek the shear value of the given set of galaxies), and SF respectively.}
% \label{f_diff_pdf_hist}
% \end{figure}



%Now we consider extracting $g(\vec{x})$ from shear catalog. We have proved PF can give a bias-free recovery of shear field $g(\Vec{x})$. The fact is almost all shear catalogs give an inhomogenous galaxy distribution, which cause $g(\vec{x})$ coupling with galaxy number density field $n({x})$. Therefore, we consider reconstructing $g(\vec{x})\cdot N(\vec{x})$ instead of $g(\vec{x})$. After this, we just devide $g(\vec{x})\cdot N(\vec{x})$ by $N(\vec{x})$ to get $g(\vec{x})$. $N(\vec{x})$ can be easily calculated by counting galaxy number in each grid. But, we may ask a question, the so got shear field is the optimal one in statistics? Through calculation, we find reconstructing $g(\vec{x})\cdot \sqrt{N(\vec{x})}$ is the optimal way whose statistical bias is corresponding with the minimum value estimated by the C-R bound. We need to note that mask region is the special case where $N(\vec{x})=0$. After importing $N(\vec{x})$  weight term, the PF method need to be revised. The galaxy number in the $i$th bin becomes:
%\begin{align}
%n_{i}=\int_{f_k(x)>0}dx \vert f_{k}(x)\vert/\sqrt{N(x)} \int_{u_{i-1}}^{u_{i}}d\widehat{e} P[\widehat{e}+ \\ \nonumber
%(\widehat{a}_{k}f_{k}(x)/\sqrt{N(x)}-g(x)),x]   \\ \nonumber
%+ \int_{f_k(x)<0}dx \vert f_{k}(x)\vert/\sqrt{N(x)} \int_{-u_{i}}^{-u_{i-1}}d\widehat{e}P[\widehat{e}+ \\ \nonumber
%(\widehat{a}_{k}f_{k}(x)/\sqrt{N(x)}-g(x)),x]
%\end{align}

% Although we have talked a lot about the continuous shear field reconstruction, in reality, the shear field always goes discontinuous since the existence of masked regions caused by image defects, bright diffraction spikes, data loss, or etc.. In this situation, we propose a revision in the previous calculation to achieve a bias-free recovery of mask-existence shear field. 

% It is noticed that the mask region in shear field naturally reflects the inhomogenous distribution of galaxy number. To include property of galaxy number density, we propose to recover $g(x)\cdot N(x)^{\gamma}$, where $\gamma$ is the power law index and $N(x)$ is the galaxy number density field. We prove it in \ref{B2} that when $\gamma=\frac{1}{2}$, statistical error of recovered shear map reaches its minimum.
% With expanding $g(x)\sqrt{N(x)}$ with orthogonal series, the weighting term also changes from $\vert f_{k}(x)\vert$ to $\vert f_{k}(x)\vert/\sqrt{N(x)}$. 
% hence, the galaxy number in the $i$th bin shows in \ref{A21}.
 

%Similarly\ref{B2}, 
%\begin{align}
%\label{num_dif}
%&n_{i}-n_{-i}\\ \nonumber
%%&=2\{\int_{f_{k}(x)>0}dx\vert f_{k}(x)\vert\/\sqrt{N(x)}[\widehat{a}_{k}f_{k}(x)\/\sqrt{N(x)}-\\
%%&g(x)]N(x)-\int_{f_{k}(x)<0}dx\vert f_{k}(x)\vert/\sqrt{N(x)}[\widehat{a}_{k}f_{k}(x)/\\
%%&\sqrt{N(x)}-g(x)]N(x)\}\int dBB[P(u_{i},B)-P(u_{i-1},B)]\\ \nonumber
%&=2(\widehat{a}_k-a_k)\int dx\vert f_k(x)\vert^{2}\int dBB[P(u_i,B)-P(u_{i-1},B)]
%\end{align}
%
%Where $P(u_i,B)$ is the normalized PDF of the galaxy distribution function.
%\begin{align}
%n_i=\sum_{j}\vert f_{k}(x_j)\vert/\sqrt{N(x_j)}L^{i}(x_{j})
%\end{align}
%where $L^{i}(x_{j})=\Delta_x \cdot \int_{u_{i-1}}^{u_i}d\widehat{G}\int dB\cdot P(\widehat{G},B,x_j)$. We can get:
%
%Assuming it obeys Poisson distribution, so that $\langle(n_j-n_{-j})^{2}\rangle\approx n_j+n_{-j}$. We have:
%\begin{align}
%%&\langle(\delta n_i - \delta n_{-i})^2\rangle \\ \nonumber
%n_i + n_{-i}=2\int dx\cdot f_{k}^{2}(x)\int_{u_{i-1}}^{u_{i}}d\widehat{G}\int dB\cdot P(\widehat{G},B)
%\end{align}
%in the limit of very small bin size, we have:
%\begin{align}
%&\chi^{2}= \\ \nonumber
%&(\widehat{a}_{k}-a_k)^{2}&[\int dx f_{k}^{2}(x)]\sum_{i>0}\frac{\{\int dBB [\widehat{P}(u_i,B)-\widehat{P}(u_{i-1},B)]\}^{2}}{\int_{u_{i-1}}gg^{u_i}d\widehat{G} \int dB P(\widehat{G},B)}
%\end{align}
%which is not only free of bias, and also approaching the C-R bound.

%We also consider the influence of source redshift distribution on shear field reconstruction. The details are beyond the scope of the main text and disscussed in the \ref{B3}.

%To test the validation of shear-flipping method, we reconstruct the $\kappa$ field of a simulated  Navarro-Frenk-White\citep{}(hereafter NFW) halo profile, which is parametrized by two parameters as,
%\begin{equation}
%\rho_{NFW}(r)=\frac{\delta_{c}\rho_{cr}(z)}{(r/r_{s})(1+r/r_{s})^{2}}
%\end{equation}
%where $\rho_{cr}(z)=3H^{2}(z)/8\pi G$ is the critical density of the Universe at the redshift of the halo. Hence, $\rho_{cr}(z)$ immediately yields the mass of the halos, $M=200\rho_{cr}(z)4\pi r_{200}^{3}/3$, where $r_200$ is the virial radius and the radius inside which the mean mass density of the halo equals $200\rho_{cr}(z)$. (The ratio of the virial radius $r_{200}$ and the scale radius $r_s$ is called the concentration parameter $c=r_{200}/r_s$. From the definition of $r_{200}$, the parameter $\delta_c$ can be related to the concentration parameter through)
%\begin{equation}
%\delta_c = \frac{200}{3}\frac{c^3}{\ln(1+c)-c/(1+c)}
%\end{equation}
%For a NFW profile, we can derive an analytical expression for the lensing convergence and shear profiles\citep{}, where $\Sigma_{NFW}$ is an intergral of $\rho_{NFW}$ over redshift:

%In our simulation, we set the halo mass at $10^{12} M_{sun}/h$ and the concentration $c=4$ under the condition $\Omega_m=0.3$ and $\Omega_\Lambda=0.7$. The redshift of the foreground lens $z_l$ is $0.3$ and the source galaxies are located at $z_s=0.6$. These galaxies are generated with the Galsim toolkit\citep{Rowe2015} with exponential profile.
%\begin{equation}
%I(r,z)=\frac{I_{0}}{2h_s}\frac{1}{\cosh(\frac{z}{h_s})}\exp{(-\frac{r}{r_s})}
%\end{equation}
%Here the parameters $r_s$ and $h_s$ give the scale distances in the 3D distribution. Since $h_s$ is correlated with $r_s$, we give the ratio $h_s/r_s$ as an independent value. We limit the radius $r$ of the galaxy in the range of [0.2, 0.4](arcsec) and the $h_s/r_s$ in range of [0.1,0.2](arcsec). 
%
%The flux of these galaxies is around $10^5 AU / arcsec^2$. Each Galsim galaxy is also convolved with the moffat-type PSF, with the FWHM of 0.6 arcsec and truncated at 1.8 arcsec.
%
%To better demonstrate the lensing field around the halo, we randomly place around $1.6\times10^7$ simulated galaxy imagies nearing the halo center within $[0.6,10]$ (arcsec). We do shear measurement with the Fourier\_Quad method\citep{Zhang2008,Zhang2015,Zhang2016,Zhang2019} for each galaxy and store the measurement result in a catalogue. To avoid alising, we add an extra margin for each halo to generate a $24\times24 arcsec^2$ final field. The geometric center of the field is (0,0)(arcsec) and we set the halo center at (2,2) (arcsec). Fig.\ref{input} shows the cases that a pure NFW halo, where the pixel scale is 0.2 arcsec.

% \begin{figure}[htbp]
% \centering
% \includegraphics[scale=0.41]{diff_lesshomodensity_hist}
% \caption{PDF of local-pdf-sym(top) and SF(bottom) recovery results when encountering low galaxy number density galaxies.}
% \label{f_diff_lesshomodensity_hist}
% \end{figure}

% This shear field is applied onto $1\times 10^{7}$ background galaxies whose ellipticities follow the recipe described in \cite{Miller2013}, i.e., 90\% of them are disk-dominated galaxies, and 10\% of them bulge-dominated. 

%Moreover, shear fields recovered by PF method are independent of local-average step, which is regarded to bring extra systematic bias if the galaxies distributing inhomogenously or existing mask effect.  

%We show the recovered results about different cases in the following text, including the performance of shear-flipping on simulated and real data, and comparisons between traditional method. 

%We note that in PF method, the shear field can be expanded as any orthogonal series. But here, the reconstructed area is small enough to regard the curved surface as flat surface, we expand the reduced shear field with Fourier series.

% as eq.\ref{2DFS}.
% \begin{align}
% \label{2DFS}
% g_{1/2}(x,y)&=\sum_{\mu=0}^{n}\sum_{\nu=0}^{n}a_{\mu\nu}\cos(2\pi\mu x/l_x + 2\pi\nu y/l_y)\\ \nonumber
% &+\sum_{\mu=0}^{n}\sum_{\nu=0}^{n}b_{\mu\nu}\sin(2\pi\mu x/l_x + 2\pi\nu y/l_y)\\ \nonumber
% \end{align}
% where $a_{\mu\nu}$and $b_{\mu\nu}$ are magnitudes of Fourier series that needing estimating.  




% \begin{figure}[htbp]
% \centering
% % \includegraphics[scale=0.6]{diff_homodensitymap_compare.pdf}
% \includegraphics[scale=0.6]{compare_HomoPDF0306.pdf}
% \caption{Upper left:recovered shear map by the traditional local-average way. Upper right:recovered shear map by the PF way. Lower left:residual of the traditional local-average way. Lower right:residual of the PF way. }
% \label{f_field}
% \end{figure}

%Performance of PF method on simulated shear field is launched firstly. In this paper we use the widely used Illutris-1-Dark\citep{2015A&C....13...12N}, which has 136 snapshots available. 

% Over all snapshots, have about
% $4.5\times 10^{8}$ FoF groups, $5.1\times 10^{8}$ Subfind groups, and $8.2\times 10^{11}$ particles. 


%project it into $z=0.20$ plane with resolution of $120\times 120\  grid^2$ to generate mass density.
 

% Different ellipticity PDF are adopted for these two types. For the disk one:
% \begin{equation}
% P(e)=\frac{Ae[1-\exp(\frac{e-e_{max}}{a})]}{(1+e)(e^{2}+e_{0}^{2})^{1/2}}
% \end{equation}
% with $e_{max}=0.804$, $e_{0}=0.0256$, $a=0.2539$ and $A=2.4318$, which is fitted by 66762 disk-dominated galaxies from DR7\citep{2009ApJS..182..543A}. For bulge-dominated ones, we follow:
% \begin{equation}
% P(e)=Ae\exp{-\alpha e-\beta e^2}
% \end{equation}
% where $\alpha=2.368$,$\beta=6.691$ and $A=27.8366$. With setting $e$ in range of $0.01\sim0.8$, we get the distribution of $G_1$ in fig.\ref{f_G1_hist}. Here we use a simple model to regard $G_1$ as $e+g$. We use traditional local-average method and SF method to recover the shear field. Corresponding recovery results by two methods of galaxies distribution in fig.\ref{f_G1_hist} are shown in fig.\ref{f_diff_pdf_hist}. 
% Obviously, residuals recovered by SF method highly concentrate arounds zero, meaning the lowest statistical error. It can be explained by \ref{sig1}, which proves the pdf-sym procession theoretically approaching C-R bound.
% With adequate number of homogenously distributed galaxies, if the PDF of intrinsic ellipticities do not show long tail, the traditional way and new way both can provide relatively presant recovery. However, Now we consider the case with low galaxy number density by eliminating the total galaxy number from $1\times 10^{7}$ to $5\times 10^{5}$. Under this circumstance, new way gives a lower FWHM PDF of residuals, shown in fig.\ref{f_diff_lesshomodensity_hist}.
