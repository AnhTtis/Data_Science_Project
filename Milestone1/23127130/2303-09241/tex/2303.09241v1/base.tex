\documentclass[12pt]{article}
\usepackage{article_base}
\usepackage{datetime2}
\usepackage[misc]{ifsym}
\title{A Hodge-Type Filtration on Rigid Cohomology}
\date{}
\author{
Bruno Chiarellotto\footnote{\Letter \, chiarbru@math.unipd.it}, Yukihide Nakada\footnote{\Letter \, yukihide.nakada@unipd.it \newline \indent $\ast, \dag$ Dipartimento di Matematica ``Tullio Levi-Civita", Universit\'a degli Studi di Padova, Via Trieste 63, 35121 Padova, Italy
\newline
\newline 
}
}
%\date{Compiled \DTMnow}
\begin{document}
\maketitle
\begin{abstract}
    Given a scheme $X$ over a complete discrete valuation ring $\O_K$ of mixed characteristic, Gros, in his study of syntomic cohomology in the smooth and proper case, introduced a filtration on the rigid cohomology of the special fiber $X_k$.
Gros claimed that this construction was independent of the choice of immersions involved, citing an unpublished paper of Berthelot.
    However this paper and Berthelot's argument, in preparation at the time, was never published and no manuscripts are known exist.
    Here, we provide a complete proof of this result.
\end{abstract}

\textbf{Keywords}\, Rigid Cohomology $\cdot$ $p$-adic cohomology

\vspace{12pt}

\noindent \textbf{Mathematics Subject Classification}\, 14F30; Secondary 14G22, 11G25
\section{Introduction}
Fix a prime number $p$.
Let $K/\Q_p$ be a finite extension with ring of integers $\O_K$ and residue field $k$.
In \cite{Gros1994}, Gros studies the syntomic cohomology of a scheme defined over $\O_K$.
This cohomology theory has several, overlapping definitions (see as a sampler \cite[Definition 5.3.2]{ChiarellottoCiccioniMazzari2013}, \cite[Definition 2.1]{Gros1994}, and \cite[Definition 8.4]{Besser2000}) but the idea is that for a scheme $X \to \op{Spec}(\O_K)$, one searches for a filtration that entwines the Hodge filtration on the cohomology of the generic fiber $H^*_{\op{dR}}(X_K)$ with the Frobenius action (and its filtration) on the $p$-adic cohomology of the special fiber.
From a broader point of view, it is the $p$-adic analogue of Deligne-Beilinson cohomology.

Instead of a traditional Hodge filtration, Gros's definition of syntomic cohomology in the proper and smooth case used a distinct filtration \cite[(3.2)]{Gros1994} on the rigid cohomology arising from the characteristic $p$ special fiber.
This filtration has also been contextualized by Besser \cite[\S 9]{Besser2000}, who showed that it is related to his definition of syntomic cohomology.
This filtration turns out to coincide with the na\"ive filtration of the dagger de Rham complex in the smooth and affine case (hence in the  Monsky-Washnitzer cohomology), but there are potential links to prismatic cohomology and the Nygaard filtration.

Just as the construction of rigid cohomology involves the choice of a compactification $X_k \subseteq Y_k$ followed by a closed immersion $Y_k \subseteq P$ into a formal $\O_K$-scheme smooth around $X_k$, the definition of this filtration involves some choices.
An essential element of the construction of rigid cohomology is that the resulting complex $R\Gamma_{\op{rig}}(X_k)$ is independent of the choices.
The same independence for the filtration has been suggested as proven by both Gros and Besser but no such proof in fact exists, leaving a hole in the literature: our goal in this paper is to prove that the filtration is independent of the choice of frame.

To be more precise, we recall that for any algebraic $k$-variety $X_k$,  Berthelot \cite{Berthelot1986} defines the rigid cohomology groups $H^n_{\op{rig}}(X_k)$ by embedding $X_k$ within a $K$-frame 
\[
X_k \hookrightarrow Y_k \hookrightarrow \fP
\]
where $X_k \hookrightarrow Y_k$ is an open immersion into a proper $k$-scheme $Y_k$ and $Y_k \hookrightarrow \fP$ is a closed immersion of $Y_k$ into a formal $\O_K$-scheme $\fP$, smooth in a neighborhood of $X_k$.
In this context he constructs the complex of overconvergent differential forms $\jdag_{X_k} \Omega^{\bullet}_{]Y[_\fP}$ and defines rigid cohomology as
\[
    H^n_{\op{rig}}(X_k) := H^n(\tube{Y_k}{\fP},\jdag_{X_k}\Omega^{\bullet}_{]Y_k[_\fP}).
\]
This construction is shown to be independent of the choice of frame.

Suppose now that there exists a pair of morphisms
\[
X_k \to Y \hookrightarrow P
\]
where $Y$ and $P$ are $\O_K$-schemes, $X_k \hookrightarrow Y_k$ is an open immersion into the special fiber $Y_k$ of $Y$, and $Y \hookrightarrow P$ is a closed immersion. 
Let $\fP := \widehat{P}$ denote the formal completion of $P$. 
In this setting Gros introduces in \cite[\S3]{Gros1994} a decreasing filtration on $\jdag_{X_k}\Omega^\bullet_{\tube{Y_k}{\fP}}$ of complexes of $\jdag_{X_k}\O_{\tube{Y_k}{\fP}}$-modules, which we denote by $\op{Fil}^s \subseteq \jdag_{X_k}\Omega^\bullet_{\tube{Y}{\fP}}$.
We call it the \emph{Gros filtration}, and it induces a filtration
\[
    F^sH^m_{\op{rig}}(X_k) := \op{Im}(H^n(]Y_k[_P, \op{Fil}^s) \to H^m_{\op{rig}}(X_k)) \subseteq H^m_{\op{rig}}(X_k)
\]
on rigid cohomology.

This filtration is simply rigid cohomology in degree $s = 0$, so it is natural to ask whether the entire filtration is independent on the choice of immersions $(X_k \subseteq Y \subseteq P)$, in a sense that we will make more precise in Section \ref{gpl_section_gros_filtration}.
Gros credits a positive answer to this question to Berthelot as \cite[Proposition 3.3, Proposition 3.5]{Gros1994} but the paper, which was in preparation at the time, remained unpublished with no known existent partial proofs. 
Our main result in this paper is Theorem \ref{gros_independence}, which provides a complete proof.

\section{The Gros Filtration}\label{gpl_section_gros_filtration}
Let $X_k$ be a $k$-scheme. 
We give a name to the lifts of frames over $\O_K$ that we need to define the filtration:
\begin{definition}
    An \emph{algebraic $\O_K$-frame for $X_k$} is a sequence of embeddings
    \[
    X_k \hookrightarrow Y \hookrightarrow P
    \]
    where $Y$ and $P$ are $\O_K$-schemes, $X_k \hookrightarrow Y_k$ is an open immersion, and $Y$ is closed in $P$. 
    We denote an algebraic $\O_K$-frame by $(X_k \subseteq Y \subseteq P)$.
    Morphisms of algebraic $\O_K$-frames are defined in the natural way. 
\end{definition}

% \textbf{Or rather, an algebraic $\O_K$-frame has nothing to do with $X_k$, and the filtration is defined when $X_k$ can be embedded in an algebraic $\O_K$-frame.}

\begin{remark}
    We note the distinction between the notion of an algebraic $\O_K$-frame and that of a $S$-frame when $S$ is a \emph{formal} $\O_K$-scheme, for example $S = \op{Spf}(\O_K)$. 
    This is typically defined (see \cite[Definition 3.1.6]{LeStum2007}) to be a $K$-frame over the trivial frame 
    \[
    S_k = S_k \hookrightarrow S    
    \]
    associated to $S$.
    They are linked by the fact that any algebraic $\O_K$-frame $(X_k \subseteq Y \subseteq P)$ induces a $\op{Spf}(\O_K)$-frame $(X_k \subseteq Y_k \subseteq \widehat{P})$.
\end{remark}

We define properties of algebraic $\O_K$-frames analogously to their standard counterparts:
\begin{definition}\label{definition_morphisms_of_frames}
    Let
    \begin{center}
        \begin{tikzcd}
            X_k \ar[r,hook] \ar[d] & Y \ar[r, hook] \ar[d, "f"] & P \ar[d,"u"]\\
            C_k \ar[r,hook] & D \ar[r,hook] & Q
        \end{tikzcd}
    \end{center}
    be a morphism of algebraic $\O_K$-frames.
    We say that it is
    \begin{enumerate}
        \item \emph{open} when all vertical maps are open immersions, \emph{mixed} when the first two are closed immersions and the last is an open immersion, \emph{Cartesian} if the left-hand square is Cartesian,
        % [\textbf{I need to check if this is the right definition for our new definition of algebraic $\O_K$-frame with $X_k$ instead of $X$}]
        and \emph{strictly Cartesian} or \emph{strict} when both are;
    
        \item \emph{smooth} (resp.\,\emph{\'etale}) if $u$ is smooth (resp.\,\'etale) in a neighborhood of $X$; 

        \item \emph{projective} (resp.\,\emph{proper}) if $f$ is projective (resp.\,proper).
    \end{enumerate}

    An algebraic $\O_K$-frame $(X_k \subseteq Y \subseteq P)$ is called \emph{smooth} (resp.\,\emph{proper}, \emph{projective}, \emph{\'etale}) if the morphism 
    \begin{center}
        \begin{tikzcd}
            X_k \ar[r,hook] \ar[d] & Y \ar[r, hook] \ar[d] & P \ar[d]\\
            \op{Spec}k \ar[r,hook] & \op{Spec}\O_K \ar[r, equal] & \op{Spec}\O_K
        \end{tikzcd}
    \end{center}
    is smooth (resp. proper, projective, \'etale). 
\end{definition}
It is easy to check that if $\mathcal{P}$ is one of the properties of a morphism of algebraic $\O_K$-frames defined above then $\mathcal{P}$ holds for the induced morphism of frames, in the sense of \cite[Definitions 3.1.11, 3.3.5, 3.3.10]{LeStum2007}.

% Fix an algebraic $k$-variety $X_k$. Locally we may embed it inside a $K$-frame $X_k \to Y_k \to \widehat{P}$ arising from an \emph{algebraic $\O_K$-frame}:

For brevity, given an algebraic $\O_K$-frame $(X_k \subseteq Y \subseteq P)$, we'll write $\jdag_X := \jdag_{X_k}$, $]X[_P := ]X_k[_{\widehat{P}}$ and $]Y[_P := ]Y_k[_{\widehat{P}}$ when there's no risk of confusion
Let $I_{X,Y,P}$, or simply $I$ when the context is clear, be the ideal defining the closed subspace $\widehat{Y}_K$ in $]Y[_P$, that is, the kernel 
\[
0 \to I_{X,Y,P} \to \O_{\tube{Y}{P}} \to i_*\O_{\widehat{Y}_K} \to 0.
\]

Recall that the complex $\jdag_X \Omega^\bullet_{\tube{Y}{P}}$ computes rigid cohomology.
Our goal in this paper is to study the following filtration on $\jdag_X\Omega^\bullet_{\tube{Y}{P}}$ and the corresponding filtration in cohomology:
\begin{definition}
    \begin{enumerate}
        \item The \emph{Gros filtration} on $\jdag_X \Omega_{\tube{Y}{P}}^\bullet$ is given in degree $s \in \bN^{\ge 0}$ by 
        \begin{align*}
            \op{Fil}^s = \op{Fil}^s_{X,Y,P} &:= \jdag_X(I^s \to I^{s-1}\Omega^1_{\tube{Y}{P}} \to I^{s - 2}\Omega^2_{\tube{Y}{P}} \to \dots)\\
            &= \jdag_X(I^{s - \bullet} \otimes \Omega^\bullet_{\tube{Y}{P}}).
        \end{align*}
        \item The induced filtration
    \[
        F^sH^n_{\op{rig}}(X_k) := \op{Im}(H^n(\tube{Y}{P}, \op{Fil}^s) \to H^n_{\op{rig}}(X_k)) \subseteq H^n_{\op{rig}}(X_k)    
    \]
    we call the \emph{Hodge-type filtration} on rigid cohomology.
    \end{enumerate}
\end{definition}

The term Hodge-type filtration on rigid cohomology is justified in part by the following special cases.
\begin{example}
Suppose $X_k$ admits a frame of the form $(X_k \subseteq Y_k \subseteq \widehat{Y})$ induced by an algebraic $\O_K$-frame $(X_k \subseteq Y \subseteq Y)$ where $Y$ is proper. 
In this case the filtration collapses into the usual ``naive'' filtration.
Indeed, if such a frame exists then with respect to this frame we have 
\[
    \tube{Y}{P} \cong \widehat{Y}_K
\]
so that $I_{X,Y,P} = 0$.
Then
\begin{align*}
    (\op{Fil}^s_{X,Y,P})^i &= \jdag_X(I^{s-i} \otimes \Omega_{\tube{Y}{P}}^i)\\
    &= 
    \begin{cases}
        0 & \text{ when } i \le s\\
        \jdag_X \Omega^i_{\tube{Y}{P}} &\text{ when }i > s
    \end{cases}
\end{align*}
which is the filtration inducing the ``naive'' filtration.

This condition is satisfied, for instance, when $X_k$ is affine smooth or when it has a proper and smooth lifting $X$ over $\O_K$.
If $X_k \cong \op{Spec}(k[x_1,\dots,x_n]/\mathfrak{a})$ is smooth and affine, in which case rigid cohomology coincides with Monsky-Washnitzer cohomology, $X_k$ has a smooth lifting
\[
X \cong \op{Spec}(\O_K[x_1,\dots,x_n]/\tilde{\mathfrak{a}}).
\]
If $Y = \overline{X} \subseteq \P^n_{\O_K}$ denotes the closure of $X$ in $\P^n_{\O_K}$, then 
\[
    X_k \hookrightarrow Y \hookrightarrow Y
\]
is such an $\O_K$-frame for $X_k$. 
If $X$ has a proper and smooth lifting $X$ over $\O_K$, the trivial $\O_K$-frame 
\[
    X_k \hookrightarrow X \hookrightarrow X
\]
is such an $\O_K$-frame for $X_k$. 
% In this setting, we may write this filtration as 
% \[
%     \op{Fil}^s \cong \Omega^{\ge s}_{A^{\dagger}} \otimes K
% \]
\end{example}

Berthelot shows that $H^n_{\op{rig}}(X_k)$ is independent of the proper smooth frame $(X_k \subseteq Y_k \subseteq \fP)$ used to compute it.
In order to use the Gros filtration to induce a well-defined filtration on rigid cohomology, we need to show that this filtration is also independent of our choices in the derived category.
More precisely, our goal is to show the following:
\begin{theorem}[({\cite[Proposition 3.3, 3.5]{Gros1994}})]\label{gros_independence}
    Suppose 
    \begin{center}
        \begin{tikzcd}
        & Y' \ar[dd,"g"] \ar[r, closed, hook] & P' \ar[dd, "u"]\\
        X_k \ar[ur, open, hook] \ar[dr, open, hook]& \\
        & Y \ar[r, closed, hook]& P
        \end{tikzcd}
    \end{center}
    is a proper smooth morphism of smooth algebraic $\O_K$-frames with $Y$ and $Y'$ reduced.
    Let $u_K: \tube{Y'}{P'} \to \tube{Y}{P}$ denote the induced map of tubes. 
    Then the filtration induced by these frames are isomorphic; that is, the natural base change map
    \[
    \op{Fil}^s_{X,Y,P} \cong Ru_{K*}\op{Fil}^s_{X,Y', P'}   
    \]
    is an isomorphism.
\end{theorem}

This result should be thought of as a variant of the independence of rigid cohomology proven by Berthelot and detailed by Le Stum:
\begin{proposition}[(Berthelot, Le Stum {{\cite[Proposition 6.5.3]{LeStum2007}}})]\label{gros_Berthelot_result}
    Let $S$ be a formal $\O_K$-scheme and let
    \begin{center}
        \begin{tikzcd}
        & Y_k' \ar[dd,"g"] \ar[r, closed, hook] & \fP' \ar[dd, "u"]\\
        X_k \ar[ur, open, hook] \ar[dr, open, hook]& \\
        & Y_k \ar[r, closed, hook]& \mathfrak{P}
        \end{tikzcd}
    \end{center}
    be a proper smooth morphism of smooth $S$-frames.
    Let $E$ be a coherent $\jdag_{X_k}\O_{\tube{Y_k}{\fP}}$-module with an overconvergent integrable conection over $S_K$.
    Then the base change map is an isomorphism 
    \[
    u*: E \otimes_{\tube{Y_k}{\fP}} \Omega^\bullet_{\tube{Y_k}{\fP}/S_K} \cong Ru_{K*} u^{\dagger}E \otimes_{\O_{\tube{Y_k'}{\fP'}}} \Omega_{\tube{Y_k'}{\fP'}/S_K}^\bullet.
    \]
\end{proposition}
% \textbf{[Are there any conditions on $S$?]}

\begin{notation}
    The notation $\jdag_X$ is ambiguous since $\jdag_X := \jdag_{\tube{X}{P}}$ depends on our choice of algebraic $\O_K$-frame $(X_k \subseteq Y \subseteq P)$.
    When there's ambiguity in the choice of embedding we'll write $\jdag_P := \jdag_{\tube{X}{P}}$ for the dagger functor corresponding to an algebraic $\O_K$-frame $(X_k \subseteq Y \subseteq P)$.
\end{notation}

Berthelot's proof of Proposition \ref{gros_Berthelot_result} is a series of reductions concluding in an explicit computation of the filtration; namely, one reduces the general result to the case where (1) the morphism is proper \'etale, and (2) $g = \op{id}_Y$.
Our argument for the independence of the filtration follows this argument closely. 
We begin the proof in \S \ref{gpl_section_basics_and_general_case} by laying out the basic formalism for working with the Gros filtration.
We then prove Theorem \ref{gros_independence} by assuming the following special cases, analogous to the cases to which Berthelot's proof reduced the independence of rigid cohomology:
% Likewise, two special cases to which our result is reduced are the following:

\begin{manuallemma}{\ref{gpl_proper_etale} \textnormal{(Proper Etale (\emph{PE}))}}
    Suppose 
    \begin{center}
        \begin{tikzcd}
        & Y' \ar[dd,"g"] \ar[r, closed, hook] & P' \ar[dd,"u"]\\
        X_k \ar[ur, open, hook] \ar[dr, open, hook]& \\
        & Y \ar[r, closed, hook]& P
        \end{tikzcd}
    \end{center}
    is a proper \'etale morphism of smooth algebraic $\O_K$-frames with $Y$ and $Y'$ reduced.
    Then
    \[
    \op{Fil}^s_{X,Y,P} \xrightarrow{\sim} Ru_{K*}\op{Fil}^s_{X,Y', P'}.
    \]
\end{manuallemma}

\begin{manualtheorem}{\ref{gpl} \textnormal{(Global Filtered Poincar\'e Lemma (\emph{GFPL})})}
    Let
    \begin{center}
        \begin{tikzcd}
            & & P' \ar[dd, "u"]\\
            X_k \ar[r, hook, open] & Y \ar[ur, hook, closed] \ar[dr, hook, closed]\\
            & & P
        \end{tikzcd}
    \end{center}
    be a smooth morphism of smooth algebraic $\O_K$-frames.
    % and let $u_K$ denote the the induced map of tubes $u_K: \tube{Y}{P'} \to \tube{Y}{P}$.
    Then there is a quasi-isomorphism 
    \[
       \jdag_P I_{X,Y,P}^\ell \xrightarrow{\sim} Ru_{K*}\jdag_{P'} (I_{X,Y,P'}^{\ell - \bullet} \otimes_{\O_{\tube{Y}{P'}}} \Omega_{\tube{Y}{P'}/\tube{Y}{P}}^\bullet).
    \]
    Moreover, $\op{Fil}^s_{X,Y,P} \cong Ru_{K*}\op{Fil}^s_{X,Y,P'}$.
\end{manualtheorem}
We prove these special cases in \S \ref{gpl_section_proper_etale} and \S \ref{gpl_section_gpl}, respectively.

\section{Proof of Main Theorem by Reduction to Special Cases}\label{gpl_section_basics_and_general_case}
We begin by providing some basic manipulations for the Gros filtration, analogous to results proved in \cite{LeStum2007} for working with ordinary (relative) rigid cohomology.
% In order to make the parallels more transparent, we define a filtered version of relative rigid cohomology:
% \begin{definition}[(c.f. {{\cite[Definition 6.2.1]{LeStum2007}}})]\label{gpl_filtered_rigid_cohomology}
    % Let 
    % \begin{center}
    %     \begin{tikzcd}
    %         X_k \ar[r,hook] \ar[d] & Y \ar[r, hook] \ar[d] & P \ar[d,"u"]\\
    %         C_k \ar[r,hook] & D \ar[r,hook] & Q
    %     \end{tikzcd}
    % \end{center}
    % be a morphism of algebraic $\O_K$-frames.
    % If $E$ is a $\jdag_X \O_{\tube{Y}{P}}$-module with an integrable connection over $K$, we define
    % \[
    % Ru_{\op{rig-fil}, s}E := Ru_{K*}(E \otimes I^{s-\bullet}_{X,Y,P}\Omega^\bullet_{\tube{Y}{P}})    
    % \]
    % with $u_K: \tube{Y}{P} \to \tube{D}{Q}$.
% \end{definition}
First of all, our arguments require that $\op{Fil}^s_{X,Y,P}$ and $Ru_{K*}\op{Fil}^s_{X,Y',P'}$ be local in some sense.
A prerequisite is that the ideal $I_{X,Y,P}$ is appropriately local:
\begin{lemma}\label{gpl_ideal_restriction}
    % Let
    % \begin{center}
    %     \begin{tikzcd}
    %         X' \ar[r, open, hook] \ar[d, hook] & Y' \ar[r, closed, hook] \ar[d, hook, open] & P' \ar[open, d]\\
    %         X \ar[r, open] & Y \ar[r, closed] & P
    %     \end{tikzcd}
    % \end{center}
    % be an open immersion of algebraic frames.
    % Then we have an isomorphism 
    % \[
    % (\jdag_X I_{X,Y,P})|_{\tube{Y'}{P'}} \cong \jdag_{X'}I_{X',Y',P'}
    % \]
    Let $(X_k \subseteq Y \subseteq P)$ be an algebraic $\O_K$-frame.
    Let $U \subseteq P$ be an open subscheme and consider the induced morphism of algebraic frames 
    \begin{center}
        \begin{tikzcd}
            X_k \cap U_k \ar[r,hook,open] \ar[d, hook, open] & Y \cap U \ar[r,hook,closed] \ar[d,hook,open] & U \ar[d, hook, open]\\
            X_k \ar[r, hook, open] & Y \ar[r,hook, closed] & P.
        \end{tikzcd}
    \end{center}
    Then there is a canonical isomorphism 
    \[
    I_{X,Y,P}|_{\tube{Y \cap U}{U}} \cong I_{X \cap U, Y \cap U, U}.
    \]
\end{lemma}

\begin{proof}
    By the functoriality of the specialization map, we have an identification 
    \[
    \tube{Y \cap U}{U} = \tube{Y}{P} \cap \widehat{U}_K
    \]
    where we implicitly identify everything with its image in $\widehat{P}_K$.
    We also have an isomorphism $\widehat{Y \cap U}_K = \widehat{Y}_K \cap \widehat{U}_K$. 
    It follows that the ideal $I_{X \cap U, Y \cap U}$ is the ideal defining the bottom closed immersion in the restriction 
    \begin{center}
        \begin{tikzcd}
            \widehat{Y}_K \ar[r,hook,closed] & \tube{Y}{P}\\
            \widehat{Y}_K \cap \widehat{U}_K \ar[r, hook, closed] \ar[u,hook,open]& \tube{Y}{P} \cap \widehat{U}_K\ar[u,hook, open].
        \end{tikzcd}
    \end{center}
    But the formation of ideals of definition commutes with restriction, which immediately gives the claim.  
\end{proof}

By an overconvergent $\O_{\tube{Y}{P}}$-module we mean an $\jdag_{X}\O_{\tube{Y}{P}}$-module (c.f. \cite[Proposition 5.3.1]{LeStum2007}). 
In particular, it has nothing to do with the separate concept of the overconvergence of a connection.

Like rigid cohomology, $Ru_{K*}\op{Fil}^s_{X,Y,P}$ is well-behaved with respect to restrictions to strict neighborhoods:
\begin{lemma}[(c.f. {{\cite[Proposition 6.2.2]{LeStum2007}}})]\label{gpl_res_strict_nbhd_lemma} 
    Let   
    \begin{center}
        \begin{tikzcd}
            X_k \ar[r,hook] \ar[d] & Y \ar[r, hook] \ar[d] & P \ar[d,"u"]\\
            C_k \ar[r,hook] & D \ar[r,hook] & Q
        \end{tikzcd}
    \end{center}
    be a morphism of algebraic $\O_K$-frames
    % and $E$ a $\jdag_X\O_{\tube{Y}{P}}$-module with an integrable connection over $K$.
    If $W$ is a strict neighborhood of $\tube{C}{Q}$ in $\tube{D}{Q}$, $V$ is a strict neighborhood of $\tube{X}{P}$ in $u_K^{-1}(W) \cap \tube{Y}{P}$, and $u_K: V \to W$ is the map induced by $u$, then
    % $Ru_{\op{rig-fil},s}E$ is overconvergent for all $s \ge 0$ and 
    we isomorphisms 
    \[
    (Ru_{K*}\op{Fil}^s_{X,Y,P})|_W = Ru_{K*}(\op{Fil}^s_{X,Y,P})|_V
    \]
    and
    \[
    (Ru_{K*}(\jdag_X I^{s-\bullet}_{X,Y,P} \Omega^\bullet_{\tube{Y}{P}/\tube{D}{Q}}))|_W \cong Ru_{K*}(\jdag_X I^{s-\bullet}_{X,Y,P}\Omega^\bullet_{\tube{Y}{P}/\tube{D}{Q}})|_V
    \]
    for all $s \ge 0$.
\end{lemma}
\begin{proof}
    We may assume as in \cite[Proposition 5.1.17]{LeStum2007} that $W = \tube{D}{Q}$.
    If instead we let $u_K: \tube{Y}{P} \to \tube{D}{Q}$ be the map induced by $u$, and we let $j: V \hookrightarrow \tube{Y}{P}$ denote the inclusion map, the first isomorphism is reduced to proving that 
    \[
    Ru_{K*}(\jdag_X I_{X,Y,P}^{s-\bullet} \Omega^\bullet_{\tube{Y}{P}})= R(u_K \circ j)_*j^{-1}(\jdag_X I^{s-\bullet}_{X,Y,P}\Omega^\bullet_{\tube{Y}{P}}).
    \]
    To show this, it suffices to show that for any $\jdag_X\O_{\tube{Y}{P}}$-module $E$ we have
    \[
    Rj_*j^{-1}E \cong E
    \]
    and this was proven in \cite[Proposition 5.3.7]{LeStum2007}.
    The second isomorphism is proven in exactly the same way.
    % \textbf{I need to understand the rest of the argument, but it should be the same. In particular, why do we have}
    % \[
    % \jdag_X \Omega^\bullet_{\tube{Y}{P}} \cong Rj_*j^{-1} \jdag_X \Omega^\bullet_{\tube{Y}{P}} 
    % \]
    % \textbf{?}
\end{proof}

We also have an analogue to the fact that rigid cohomology is `local downstairs'.

\begin{lemma}[(c.f. {{\cite[Proposition 6.2.9]{LeStum2007}}})]\label{gpl_local_downstairs}
    Let
    \begin{center}
        \begin{tikzcd}
            X_k \ar[r,hook] \ar[d] & Y \ar[r, hook] \ar[d] & P \ar[d,"u"]\\
            C_k \ar[r,hook] & D \ar[r,hook] & Q
        \end{tikzcd}
    \end{center}
    be a morphism of algebraic $\O_K$-frames,
    \begin{center}
        \begin{tikzcd}
            C'_k \ar[r,hook] \ar[d,hook] & D' \ar[r, hook] \ar[d,hook] & Q' \ar[d,hook]\\
            C_k \ar[r,hook] & D \ar[r,hook] & Q
        \end{tikzcd}
    \end{center}
    be an open or mixed immersion of frames which is cartesian, and
    \begin{center}
        \begin{tikzcd}
            X'_k \ar[r,hook] \ar[d] & Y' \ar[r, hook] \ar[d] & P' \ar[d,"u'"]\\
            C'_k \ar[r,hook] & D' \ar[r,hook] & Q'
        \end{tikzcd}
    \end{center}
    be the pullback of the first morphism of frames along this immersion.
    
    Then we have isomorphisms
    \[
    (Ru_{K*}\op{Fil}^s_{X,Y,P})|_{\tube{D'}{Q'}} \cong Ru'_{K*}\op{Fil}^s_{X',Y',P'}
    \]
    % \[
    % \op{Fil}^s_{X,Y,P}|_{\tube{D'}{Q'}} \cong Ru'_{K*}(\jdag_{X'}I^{s-\bullet}_{X',Y',P'}\Omega^\bullet_{\tube{Y'}{P'}})  
    % \]
    % \[
        % (Ru_{\op{rig-fil},s}E)|_{\tube{D'}{Q'}} \cong Ru_{\op{rig-fil}, s}E|_{\tube{Y'}{P'}}
    % \]
    and 
    \[
        (Ru_{K*}(\jdag_X I^{s-\bullet}_{X,Y,P}\Omega^\bullet_{\tube{Y}{P}/\tube{D}{Q}}))|_{\tube{D'}{Q'}} \cong Ru'_{K*}(\jdag_{X'} I_{X',Y',P'}^{s-\bullet} \Omega^\bullet_{\tube{Y'}{P'}/\tube{D'}{Q'}})
    \]
    for all $s \ge 0$.
\end{lemma}

\begin{proof}
%    \textbf{Review this proof, do I really get it? And change all the de Rham complexes from relative to absolute, and track what changes!} 
   As noted before, the morphism of frames associated to a cartesian morphism of algebraic frames is cartesian, since base change commutes with pullbacks.
   Similarly, the morphism of frames associated to an open (resp. mixed) immersion of algebraic frames is open (resp. mixed).
   Therefore we may simply follow the proof of \cite[Proposition 6.2.9]{LeStum2007} and similarly exploit the fact that higher direct images commute with open immersions to obtain 
   \begin{align*}
   (Ru_{K*}\op{Fil}^s_{X,Y,P})|_{\tube{D'}{Q'}} &= (Ru_{K*}(\jdag_X I^{s-\bullet}_{X,Y,P}\Omega^\bullet_{\tube{Y}{P}}))|_{\tube{D'}{Q'}}\\
   &\cong Ru'_{K*}(\jdag_X I^{s-\bullet}_{X,Y,P}\Omega^\bullet_{\tube{Y}{P}})|_{\tube{Y'}{P'}}\\
   &\cong Ru'_{K*}(\jdag_{X'} I_{X,Y,P}^{s-\bullet}|_{\tube{Y'}{P'}} \Omega^\bullet_{\tube{Y}{P}}|_{\tube{Y'}{P'}})\\
   &\cong Ru'_{K*}(\jdag_{X'} I_{X',Y',P'}^{s-\bullet}\Omega^\bullet_{\tube{Y'}{P'}})\\
   &= Ru'_{K*}\op{Fil}^s_{X',Y',P'}
   \end{align*}
    where the third line follows from the fact that restriction commutes with tensor products, and the last line follows from Lemma \ref{gpl_ideal_restriction}.
   The restriction of the relative de Rham complex is similarly computed as
   \begin{align*}
   (Ru_{K*}(\jdag_X I^{s-\bullet}_{X,Y,P}\Omega^\bullet_{\tube{Y}{P}/\tube{D}{Q}}))|_{\tube{D'}{Q'}} &\cong Ru'_{K*}(\jdag_X I^{s-\bullet}_{X,Y,P}\Omega^\bullet_{\tube{Y}{P}/\tube{D}{Q}})|_{\tube{Y'}{P'}}\\
%    &\cong Ru'_{K*}(E|_{\tube{Y'}{P'}} \otimes I_{X,Y,P}^{s-\bullet}|_{\tube{Y'}{P'}} \Omega^\bullet_{\tube{Y}{P}/\tube{D}{Q}}|_{\tube{Y'}{P'}})\\
   &\cong Ru'_{K*}(\jdag_{X'} I_{X',Y',P'}^{s-\bullet}\Omega^\bullet_{\tube{Y'}{P'}/\tube{D'}{Q'}})
   \end{align*}
   using the fact that $\Omega^\bullet_{\tube{Y}{P}/\tube{D}{Q}}|_{\tube{Y'}{P'}} \cong \Omega^\bullet_{\tube{Y'}{P'}/\tube{D'}{Q'}}$.
\end{proof}

In particular,
\begin{corollary}\label{gros_lemma_is_local}
    Theorem \ref{gros_independence} is local on $X_k$ and $P$.
\end{corollary}

\begin{proof}
    The theorem is local on $X_k$ by \cite[Proposition 5.2.8]{LeStum2007}, and Lemma \ref{gpl_local_downstairs} says precisely that we can localize the Gros filtration on the base.
\end{proof}

% Finally, in the reduction we will need the flexibility of an \emph{a priori} stronger but in fact equivalent version of Theorem \ref{gros_independence}:
% \begin{lemma}\label{gros_proposition_locally_free_equivalent}
%     Suppose 
%     \begin{center}
%         \begin{tikzcd}
%         & Y' \ar[dd,"g"] \ar[r, closed, hook] & P' \ar[dd, "u"]\\
%         X_k \ar[ur, open, hook] \ar[dr, open, hook]& \\
%         & Y \ar[r, closed, hook]& P
%         \end{tikzcd}
%     \end{center}
%     is a proper smooth morphism of proper smooth algebraic $\O_K$-frames (see Definition \ref{definition_morphisms_of_frames}).
%     Then the following are equivalent:
%     \begin{enumerate}
%         \item The natural base change map 
%         \[
%             \op{Fil}^s_{X,Y,P} \cong Ru_{K*}\op{Fil}^s_{X,Y', P'}   
%         \]
%         is an isomorphism;

%         \item For any (finite) locally free $\O_{\tube{Y}{P}}$-module $\mathcal{M}$, we have 
%         \[
%             \mathcal{M} \otimes \op{Fil}^s_{X,Y,P} \cong Ru_{K*}(u_K^*\mathcal{M} \otimes \op{Fil}^s_{X,Y',P'})    
%         \]
%     \end{enumerate}
% \end{lemma}

% \begin{proof}
%     This is an immediate consequence of the following projection formula in the derived category \cite[\href{https://stacks.math.columbia.edu/tag/0B54}{Lemma 0B54}]{stacks-project}. Let $f: X \to Y$ is a morphism of ringed spaces and let $E \in D(\O_X)$ and $F \in D(\O_Y)$. If $F$ is perfect (see  \cite[\href{https://stacks.math.columbia.edu/tag/08CL}{Section 08CL}]{stacks-project}), then 
%     \[
%     Rf_*E \otimes_{\O_Y}^{\mathbb{L}} F = Rf_*(E \otimes_{\O_X}^{\mathbb{L}} Lf^*F).  
%     \]
%     Note that the functor $u_K^*$ is exact and locally free sheaves are both perfect complexes and their own projective resolutions.
%     Using \cite[Proposition 5.3.2]{LeStum2007} we can infer for any $\mathcal{M}$ that
%     \[
%         Ru_{K*}(u_K^*\mathcal{M} \otimes \op{Fil}^s_{X,Y',P'}) \cong \mathcal{M} \otimes Ru_{K*}\op{Fil}^s_{X,Y',P'}.
%     \]
%     We can thus get (2) by tensoring (1) with $\mathcal{M}$. 
% \end{proof}
We now prove Theorem \ref{gros_independence} given Lemma \hyperref[gpl_proper_etale]{PE} and Theorem \hyperref[gpl]{GFPL}. 

\begin{proof}[Proof of Theorem \ref{gros_independence}]
    By Corollary \ref{gros_lemma_is_local} we can localize on $X_k$ and $P$.
    In particular, we may assume that $X_k$ is quasi-projective. 

    By Chow's lemma \cite[Corollary 5.7.14]{GrusonRaynaud1971} we may blow up a closed subscheme of $Y'$ outside $X_k$ in $P'$ and obtain a diagram
    \begin{center}
        \begin{tikzcd}
            & \widetilde{Y} \ar[r,hook] \ar[d, "f"] & \widetilde{P} \ar[d, "v"]\\
            X_k \ar[ur, hook] \ar[dr, hook] \ar[r,hook] & Y' \ar[r, hook] \ar[d,"g"] & P' \ar[d,"u"]\\
            & Y \ar[r,hook] & P
        \end{tikzcd}
    \end{center}
    where the upper morphism of frames is strict with $v$ a blow-up and where $g \circ f$ is projective.

    Because $g$ is separated (being proper), it follows from  \cite[\href{https://stacks.math.columbia.edu/tag/0C4Q}{Lemma 0C4Q}]{stacks-project} that $f$ is projective.
    Furthermore, since $v$ is a blow-up outside of $X_k$ it follows that $v$ is an isomorphism (and hence smooth) in a neighborhood of $X_k$.
    Thus the morphism of frames corresponding to $v$ and to $u \circ v$ are both projective smooth, and this reduces the theorem to the projective smooth case. 
    Indeed, if the theorem holds in this special case then we can infer isomorphisms
    \[
    \op{Fil}_{X,Y,P}^s \cong R(u \circ v)_{K*}\op{Fil}^s_{X,\widetilde{Y}, \widetilde{P}} \quad \text{ and } \quad \op{Fil}^s_{X,\widetilde{Y}, \widetilde{P}} \cong Rv_{K*}\op{Fil}^s_{X,Y', P'}
    \]
    for their respective frames.
    % \textbf{It's either this or using the fact that a blow-up induces an isomorphism on strict neighborhoods a la \cite[Proposition 3.1.13]{LeStum2007}. In this case we might need to convert our algebraic frame to a usual frame before blowing-up.}
    The result for our original morphism of frames then follows immediately by composing the latter with $Ru_{K*}$.

    Thus we may assume that $g$ is projective.
    Since we can assume that $P$ is affine, there exists an integer $n \ge 0$ such that 
    \begin{center}
        \begin{tikzcd}
        & Y' \ar[dd,"g"] \ar[r, closed, hook] & \P_P^n \ar[dd, "\pi"]\\
        X_k \ar[ur, open, hook] \ar[dr, open, hook]& \\
        & Y \ar[r, closed, hook]& P
        \end{tikzcd}
    \end{center}
    commutes, where $\pi$ is the canonical projection. 
    In this situation, we may further find a closed subscheme $Y'' \subseteq Y'$ and $P'' \subseteq \P^n_P$ such that the composition of frames 
    \begin{center}
        \begin{tikzcd}
            &   Y'' \ar[r] \ar[d, closed, hook, "\iota"] & P'' \ar[d]\\
        X_k \ar[ur, open] \ar[r, open] \ar[dr, open] &  Y' \ar[d, "g"] \ar[r] &   \P_P^n \ar[d, "\pi"]\\
            &   Y   \ar[r,hook, closed] & P
        \end{tikzcd}
    \end{center}
    is proper \'etale.
    Indeed, note that, localizing on $X_k$ if necessary, the closed immersion $X \hookrightarrow \pi^{-1}(X_k) = \P_{X_k}^n$ is a section of the canonical projection which is smooth.
    It follows that there exists a covering of $\P^n_{X_k}$ by open sets $U$ such that $X_k$ is defined in $U$ by a regular sequence $(\tilde{t}_1,\dots,\tilde{t}_d)$, induced by sections $t_1,\dots,t_d \in \Gamma(\P_P^n,\O(m))$.
    We may assume that $U = D^+(s) \cap \pi^{-1}(Y)$ for some $s \in \Gamma(\P_P^n, \O(r))$ for some $r$.
    It then suffices to take $P'' := V(t_1,\dots,t_d)$ and $Y'' := Y' \cap P''$, since by construction $P''$ is the identity (and hence \'etale) in a neighborhood of $X_k$.

    Let us take this embedding $Y'' \xrightarrow{\iota} Y'$ and consider its composition with the original morphism of frames:
    \begin{center}
        \begin{tikzcd}
            & Y'' \ar[d, hook, closed, "\iota"] \ar[dr, hook, closed]\\
            X_k \ar[r, hook, open] \ar[ur, hook, open] \ar[dr, hook, open] & Y' \ar[d, "g"] \ar[r, closed, hook] & P' \ar[d, "u"]\\
            & Y \ar[r, hook, closed] & P 
        \end{tikzcd}
    \end{center}
    The theorem holds for the top morphism of frames by Lemma \ref{gpl_closed_immersion} (to be proven later), so to prove it for the bottom morphism it suffices to prove it for the diagram corresponding to $g \circ \iota$.
    Since $\iota$ is a closed immersion, $g \circ \iota$ remains projective.
    Thus we can assume not only that $g$ is projective, but that there exists a morphism of frames 
    \begin{center}
        \begin{tikzcd}
        & Y' \ar[dd,"g"] \ar[r, closed, hook] & P'' \ar[dd, "w"]\\
        X_k \ar[ur, open, hook] \ar[dr, open, hook]& \\
        & Y \ar[r, closed, hook]& P
        \end{tikzcd}
    \end{center}
    with $w$ \'etale in a neighborhood of $X_k$.
    The theorem holds for this morphism of frames by virtue of Lemma \hyperref[gpl_proper_etale]{PE}.

    Now consider the diagonal embedding $Y' \to P' \times_P P''$ with its respective projections $p$ and $q$.
    These morphisms fit into the commutative diagram 
    \begin{center}
        \begin{tikzcd}
            & & & P' \times_P P'' \ar[dr, swap, "q"] \ar[dl, "p"]\\
        & Y' \ar[r, hook, closed] \ar[rrr, crossing over, bend right=20] \ar[urr, hook, closed] \ar[dd, "g"]&  P' \ar[dd, "v"] & & P'' \ar[ddll, "w"]\\
        X_k \ar[ur,hook,open] \ar[dr, hook, open]\\
        & Y \ar[r,hook, closed] & P
        \end{tikzcd}
    \end{center}
    The projections $p$ and $q$ are both smooth in a neighborhood of $X_k$.
    Thanks to Theorem \hyperref[gpl]{GFPL} and Lemma \ref{gpl_locally_free_sheaf_equivalent}, we thus have 
    \begin{align*}
        \op{Fil}^s_{X,Y',P''} &\cong Rq_{K*}\op{Fil}^s_{X,Y',P' \times P''}\\
        \op{Fil}^s_{X,Y',P'} &\cong Rp_{K*}\op{Fil}^s_{X,Y',P' \times P''}
    \end{align*}
    Since $v \circ p = w \circ q$, we have 
    \begin{align*}
        \op{Fil}^s_{X,Y,P} &\cong Rw_{K*}\op{Fil}^s_{X,Y',P''}\\
        &\cong Rw_{K*}(Rq_{K*}\op{Fil}^s_{X,Y',P' \times P''})\\
        &\cong R(w \circ q)_{K*}(\op{Fil}^s_{X,Y',P' \times P''})\\
        &= R(v \circ p)_{K*}(\op{Fil}^s_{X,Y',P' \times P''})\\
        &\cong Rv_{K*}(Rp_{K*}\op{Fil}^s_{X,Y',P' \times P''})\\
        &\cong Rv_{K*}\op{Fil}^s_{X,Y',P'}
    \end{align*}
    which was what we wanted to show. 
\end{proof}
\section{The Proper \'Etale Case}\label{gpl_section_proper_etale}
    In this section we prove Lemma \hyperref[gpl_proper_etale]{PE}, which validates the isomorphism for proper \'etale morphisms of frames:
\begin{lemma}[(Proper \'Etale (\emph{PE}))]\label{gpl_proper_etale}
    Suppose 
    \begin{center}
        \begin{tikzcd}
        & Y' \ar[dd,"g"] \ar[r, closed, hook] & P' \ar[dd,"u"]\\
        X_k \ar[ur, open, hook] \ar[dr, open, hook]& \\
        & Y \ar[r, closed, hook]& P
        \end{tikzcd}
    \end{center}
    is a proper \'etale morphism of smooth algebraic $\O_K$-frames with $Y$ and $Y'$ reduced. 
    Then
    \[
    \op{Fil}^s_{X,Y,P} \xrightarrow{\sim} Ru_{K*}\op{Fil}^s_{X,Y', P'}.    
    \]
\end{lemma}

\begin{remark}
        The hypothesis that $Y$ and $Y'$ are reduced is necessary for the argument.
        However, the frame $(X_k \subseteq Y_k \subseteq \mathfrak{P})$ used to compute rigid cohomology, which depends only on the tubes of $X_k$ and $Y_k$ in $\mathfrak{P}$, can be assumed to have $X_k$ and $Y_k$ reduced. 
        Indeed, if $Y_k \hookrightarrow \mathfrak{P}$ is a closed immersion of a $k$-scheme $Y_k$ into a formal $\O_K$-scheme $\mathfrak{P}$ then $\tube{Y_k}{\mathfrak{P}} = \tube{Y_k^{\op{red}}}{\mathfrak{P}}$ where $Y_k^{\op{red}}$ is $Y_k$ with the reduced induced closed subscheme structure.
        This is because the tube is insensitive, locally, to the degree of the elements defining $Y$ in $\mathfrak{P}_k$; see \cite[Proposition 2.2.13]{LeStum2007}.
        As such, insofar as we're trying to define a filtration on rigid cohomology, there is no loss of generality in assuming that $Y$ and $Y'$ are reduced. 
\end{remark}

We first prove a special case:
\begin{lemma}\label{gpl_closed_immersion}
    Given a morphism of smooth algebraic $\O_K$-frames 
    \begin{center}
        \begin{tikzcd}
        X_k \ar[r] \ar[dr] & Y' \ar[d,hook, closed,"g"] \ar[r] & P\\
        & Y \ar[ur]
        \end{tikzcd}
    \end{center}
    where $g$ is a closed immersion and $Y$ is reduced, the natural open immersion of tubes $\tube{Y'}{P} \hookrightarrow \tube{Y}{P}$ induces an isomorphism $\jdag_X\O_{\tube{Y'}{P}} \cong \jdag_X \O_{\tube{Y}{P}}|_{\tube{Y'}{P}}$.
    Moreover, it gives an isomorphism of complexes 
    \[
    \op{Fil}^s_{X,Y,P}|_{\tube{Y'}{P}} \xrightarrow{\sim} \op{Fil}^s_{X,Y',P}.
    \]
\end{lemma}

\begin{proof}
    The claim is local on $P$ and $X_k$, so we may suppose that $P$ is affine and $X_k$ is the complement of a hypersurface of $Y_k$ defined by a function $f$ on $Y_k$. 
    Of course since $X_k \subseteq Y'_k$ we have $Y_k = Y'_k \cup (Y_k \setminus X_k) = Y_k' \cup V(f)$.
    Take a lifting $\tilde{f}$ of $f$.
    We then have that
    \[
    \widehat{Y}_K = \widehat{Y}'_K \cup V(\tilde{f})
    \] 
    as a closed analytic subspace of the tube $\tube{Y}{P}$.

    On the other hand, consider the system of admissible open subsets
    \[
        V^\lambda := \tube{Y}{P} \setminus \{|\tilde{f}| < \lambda\}    
    \]
    for $\lambda < 1$, shown to be strict neighborhoods of $\tube{X}{P}$ in $\tube{Y}{P}$ in \cite[Proposition 3.3.1]{LeStum2007}.
    By \cite[Propositions 3.1.9, 3.1.10]{LeStum2007}, we have that $\{V^\lambda \cap \tube{Y'}{P}\}_\lambda$ are strict neighborhoods of $\tube{X}{P}$ in $\tube{Y'}{P}$ as well as in $\tube{Y}{P}$.

    Since $|\tilde{f}| \ge \lambda > 0$ on $V^\lambda$, we have that $V(\tilde{f}) \cap V^\lambda = \varnothing$ and hence 
    \[
    \widehat{Y}_K \cap V^\lambda = \widehat{Y}'_K \cap V^\lambda    
    \]
    for all $\lambda$.
    Intersecting with $\tube{Y'}{P}$, we have 
    \[
    \widehat{Y}_K \cap (V^\lambda \cap \tube{Y'}{P}) = \widehat{Y}_K' \cap   (V^\lambda \cap \tube{Y'}{P}) \hookrightarrow V^\lambda \cap \tube{Y'}{P}.
    \]
    But the ideals of the immersions
    \begin{align*}
        \widehat{Y}_K \cap (V^\lambda \cap \tube{Y'}{P}) &\hookrightarrow V^\lambda \cap \tube{Y'}{P}\\
        \widehat{Y}'_K \cap (V^\lambda \cap \tube{Y'}{P}) &\hookrightarrow V^\lambda \cap \tube{Y'}{P}
    \end{align*}
    are respectively the restrictions of $I_{X,Y,P}$ and $I_{X,Y',P}$ to the strict neighborhood $V^\lambda \cap \tube{Y'}{P}$.
    That these two immersions are identical means that 
    \[
        {I_{X,Y,P}}_{|(V^\lambda \cap \tube{Y'}{P})} = {I_{X,Y',P}}_{|(V^\lambda \cap \tube{Y'}{P})}.
    \]

    By \cite[Proposition 5.1.12]{LeStum2007}, the result follows if we can show that, for any quasi-compact admissible open $W \subseteq \tube{Y'}{P}$, the subsets $W \cap (V^\lambda \cap \tube{Y'}{P})$ are cofinal in the set of all $W \cap V'$ where $V'$ ranges over all strict neighborhoods of $\tube{X}{P}$ in $\tube{Y'}{P}$.  

    To see this, fix such a $V'$.
    Since $g$ is a closed immersion, $\tube{Y'}{P}$ is an admissible open subset of $\tube{Y}{P}$, so $W$ is an admissible open subset of $\tube{Y}{P}$ as well.
    Furthermore, by \cite[Proposition 3.1.10]{LeStum2007}, $V'$ is a strict neighborhood of $\tube{X}{P}$ in $\tube{Y}{P}$.
    So by \cite[Proposition 3.3.2]{LeStum2007} there exists $\lambda < 1$ such that 
    \[
    W \cap V^\lambda \subseteq V'.
    \]
    It follows that
    \[
    (W \cap V^\lambda) \cap \tube{Y'}{P} \subseteq W \cap V'
    \]
    and we're done.
\end{proof}

In addition, we provide for completion a proof of a fact that is well-known:
\begin{fact}\label{fact_overconvergent_restriction}
        Let $(X \subseteq Y \subseteq P)$ be a frame and let $f: \cE_1^\bullet \to \cE_2^\bullet$ be a morphism of complexes of overconvergent sheaves in $D(\tube{Y}{P})$.
        If there exists a strict neighborhood $V$ of $\tube{X}{P}$ in $\tube{Y}{P}$ such that $f|_V$ is a quasi-isomorphism, then $f$ is a quasi-isomorphism. 
    \end{fact}

    \begin{proof}
        The category of overconvergent modules has enough injectives and the construction of injective resolutions is functorial, so there exist injective resolutions $\cE_1^\bullet \xrightarrow{\sim} \cI_1^\bullet$ and $\cE_2^\bullet \xrightarrow{\sim} \cI_2^\bullet$ and a morphism $g: \cI_1^\bullet \to \cI_2^\bullet$ such that
        \begin{center}
            \begin{tikzcd}
                \cE_1^\bullet \ar[r,"f"] \ar[d, "\cong"] & \cE_2^\bullet \ar[d,"\cong"]\\
                \cI_1^\bullet \ar[r, "g"] & \cI_2^\bullet
            \end{tikzcd}
        \end{center}
        commutes.

        If a sheaf $\mathcal{F}$ is overconvergent then it trivially satisfies the universal property of $j^\dagger_X \mathcal{F}$ so $\mathcal{F} = \jdag_X \mathcal{F}$.
        Hence we have $\cI_1^\bullet = \jdag_X \cI_1^\bullet$ and similarly for $\cI_2^\bullet$.
        Furthermore, restriction to $V$ is functorial and we can check that a morphism $\vp$ is an isomorphism in the derived category by checking that $R\Gamma(W,\vp)$ is an isomorphism in the derived category for all affinoid open subsets of $\tube{Y}{P}$.

        Putting this together, let $W$ be an affinoid open, which in particular is a quasi-compact admissible open subset of $\tube{Y}{P}$.
        We have 
        \begin{center}
            \begin{tikzcd}
                R\Gamma(W, \cE_1^\bullet) \ar[r,"R\Gamma(f)"] \ar[d, "\cong"] & R\Gamma(W,\cE_2^\bullet) \ar[d,"\cong"]\\
                \Gamma(W, \cI_1^\bullet) \ar[d, "="] \ar[r, "R\Gamma(g)"] & \Gamma(W, \cI_2^\bullet) \ar[d,"="]\\
                \Gamma(W, \jdag_X \cI_1^\bullet) \ar[r] \ar[d, "="] & \Gamma(W, \jdag_X \cI_2^\bullet) \ar[d,"="] \\
                \underset{V' \subset \tube{Y}{P}}{\varinjlim}\, \Gamma(W \cap V', \cI_1^\bullet) \ar[r] \ar[d,"="]& \underset{V'\subset \tube{Y}{P}}{\varinjlim}\,\Gamma(W \cap V', \cI_2^\bullet) \ar[d,"="]\\
                \underset{V' \subset V}{\varinjlim}\, \Gamma(W \cap V', {\cI_1^\bullet}_{|V}) \ar[r, "\cong"] & \underset{V'\subset V}{\varinjlim}\,\Gamma(W \cap V', {\cI_2^\bullet}_{|V})
            \end{tikzcd}
        \end{center}
        where the last morphism (equal to $R\Gamma(W, g|_V)$) is a quasi-isomorphism since $f|_V$ is a quasi-isomorphism by assumption.
        This gives the result.
    \end{proof}
\begin{proof}[Proof of Lemma {\hyperref[gpl_proper_etale]{PE}}]
        We first reduce Lemma \hyperref[gpl_proper_etale]{PE} to the following: there exist a strict neighborhood $V'$ of $\tube{X}{P'}$ in $\tube{Y'}{P'}$ and a strict neighborhood $V$ of $\tube{X}{P}$ in $\tube{Y}{P}$ such that $u_K$ restricts to a morphism $u_K: V' \to V$ and the vertical morphisms in the diagram 
        \begin{center}
            \begin{tikzcd}
                \widehat{Y}_K' \cap V' \ar[r,hook,closed] \ar[d,"u_K"] & V' \ar[d,"u_K"] \\
                \widehat{Y}_K \cap V \ar[r,hook,closed] & V
            \end{tikzcd}
        \end{center}
        are isomorphisms.
        To see that this is sufficient, note the top (resp. bottom) closed immersion is the restriction to $V'$ (resp. $V$) of the closed immersion
        \begin{align*}
            \widehat{Y}'_K &\hookrightarrow \tube{Y'}{P'}\\
            (\text{resp. } \widehat{Y}_K &\hookrightarrow \tube{Y}{P})
        \end{align*}
        whose ideals of definition is $I_{P'} := I_{X,Y',P'}$ (resp. $I_{P'} := I_{X,Y',P'}$).
        Lemma \ref{gpl_res_strict_nbhd_lemma} tells us that
        \[
            (Ru_{K*}\op{Fil}^s_{X,Y',P'})|_V = Ru_{K*}(\jdag_{P'}I^{s-\bullet}_{P'} \otimes \Omega^\bullet_{\tube{Y'}{P'}})|_{V'}.
        \]
        But $u_K$ is an isomorphism so in particular its own derived functor, and restiction commutes with dagger functors (\cite[Proposition 5.1.5]{LeStum2007}) so
        \[
        (Ru_{K*}\op{Fil}^s_{X,Y',P'})|_V = u_{K*}(\jdag_{P'}(I_{P'}^{s - \bullet}|_{V'}) \otimes \Omega_{V'}^\bullet).
        \]
        But the fact that the above vertical maps are isomorphisms means precisely that $I_P^{s-\bullet}|_V \cong u_{K*}(I_{P'}^{s - \bullet}|_{V'})$, since the closed immersions are the restrictions to $V$ and $V'$ of the closed immersions defining the ideals $I_P$ and $I_{P'}$, respectively.

        This implies that 
        \begin{align*}
        \op{Fil}^s_{X,Y,P}|_V &= \jdag_P(I_P^{s-\bullet}|_V) \otimes \Omega_V^\bullet \\
        &\cong u_{K*}(\jdag_{P'}(I_{P'}^{s - \bullet}|_{V'}) \otimes \Omega_{V'}^\bullet)\\
        &\cong (Ru_{K*}\op{Fil}^s_{X,Y',P'})|_V
        \end{align*}
        and by Fact \ref{fact_overconvergent_restriction} this is enough to conclude the lemma.

        Having reduced Lemma \hyperref[gpl_proper_etale]{PE} to this problem, we now find such strict neighborhoods $V$ and $V'$. 
        Proper maps are in particular closed, so we have a factorization
        \begin{center}
            \begin{tikzcd}
                & Y' \ar[r, hook, closed] \ar[d, two heads, "g"] & P' \ar[d,"u"]\\
            X_k \ar[r, hook, open] \ar[ur, hook, open] \ar[dr, hook, open] & \op{Im}(g) \ar[r, hook, closed] \ar[d, hook, closed] & P\\
                & Y \ar[ur, hook, closed]
            \end{tikzcd}
        \end{center}
        where $\op{Im}(g)$ is the schematic image of $g$. 
        We know from Lemma \ref{gpl_closed_immersion} that 
        \[
        \op{Fil}^\bullet_{X,Y,P} = \op{Fil}^\bullet_{X,\op{Im}(g), P}    
        \]
        so we may replace $Y$ by $\op{Im}(g)$ and thus suppose that $g$ is proper surjective.

        We know from the proof of \cite[Proposition 3.4.12]{LeStum2007} and \cite[Proposition 3.3.11]{LeStum2007} that given a proper \'etale morphism of frames we may find isomorphic strict neighborhoods $V' \xrightarrow{u_K} V$ as follows.
        \cite[Proposition 3.3.11]{LeStum2007} tells us that for a fixed sequence $\eta_n \xrightarrow{<} 1$ we can find $\delta_n \xrightarrow{<} 1$ and $\lambda_n \xrightarrow{<} 1$ such that, for each $n \in \mathbb{N}$, the morphism $u_K$ induces an isomorphism
        \[
            V_n' := u_K^{-1}([Y]_{P\eta_n}) \cap V^{'\lambda_n}_{\eta_n} = [Y']_{P'\delta_n} \cap u_K^{-1}(V^{\lambda_n}_{\eta_n}) \cong V^{\lambda_n}_{\eta_n}
        \]
        and that the induced morphism $u_K: V' := \cup_n V_n' \to \cup_n V^{\lambda_n}_{\eta_n} =: V$ is an isomorphism of strict neighborhoods.
        We show that these strict neighborhoods suffice.

        Since $g: Y' \to Y$ is proper surjective, it follows from base change that $g_K: \widehat{Y}'_K \to \widehat{Y}_K$ is surjective.
        We may identify $\widehat{Y}'_K$ and $\widehat{Y}_K$ with their images in $\tube{Y'}{P'}$ and $\tube{Y}{P}$, respectively, and by commutativity this identification is compatible with $u_K$.
        Hence for each $n$ we have a surjection 
        \[
            u_K: \widehat{Y}'_K \cap u_K^{-1}(V^{\lambda_n}_{\eta_n}) \twoheadrightarrow  \widehat{Y}_K \cap V^{\lambda_n}_{\eta_n}.
        \]
        Since $\widehat{Y}_K' \subset [Y']_{P'\delta}$ for all $\delta$, we have
        \[
        \widehat{Y}'_K \cap [Y']_{P'\delta_n} \cap u_K^{-1}(V^{\lambda_n}_{\eta_n}) = 
        \widehat{Y}'_K \cap u_K^{-1}(V^{\lambda_n}_{\eta_n})
        \]
        so this is identical to the map 
        \[
        u_K: \widehat{Y}'_K \cap [Y']_{P'\delta_n} \cap u_K^{-1}(V^{\lambda_n}_{\eta_n}) \twoheadrightarrow \widehat{Y}_K \cap V^{\lambda_n}_{\eta_n}. 
        \]
        Taking the union over all $n$, we find that 
        \[
        u_K: \widehat{Y}'_K \cap V' \twoheadrightarrow \widehat{Y}_K \cap V    
        \]
        is surjective.

        To show that this is an isomorphism, we first show that it is a closed immersion.
        Identifying $V'$ with $V$ via the isomorphism $u_K$, we have a diagram of rigid-analytic spaces 
        \begin{center}
            \begin{tikzcd}
                \widehat{Y}_K' \cap V'  \ar[dr, hook, closed, "\iota'"] \ar[dd, swap, "u_K"]\\
                & V\\
                \widehat{Y}_K \cap V \ar[ur, swap, hook, closed, "\iota"]
            \end{tikzcd}
        \end{center}
        Let $V = \bigcup V_i$ be an admissible affinoid covering with $V_i \cong \op{Sp}A_i$. 
        Then by \cite[Proposition 9.5.3.2]{BGR}, $\widehat{Y}_K' \cap V' = \bigcup \iota'^{-1}(V_i)$ and $\widehat{Y}_K \cap V = \bigcup \iota^{-1}(V_i)$ are admissible affinoid coverings as well, say $\iota'^{-1}(V_i) \cong \op{Sp}B_i$ and $\iota^{-1}(V_i) \cong \op{Sp}C_i$, and the corresponding homomorphisms $A_i \to B_i$ and $A_i \to C_i$ are surjective.
        Thus the restriction of the above diagram to $V_i$ is a diagram of affinoid spaces, and the corresponding diagram of rings is 
        \begin{center}
            \begin{tikzcd}
                C_i\\
                & A_i \ar[dl, twoheadrightarrow] \ar[ul, twoheadrightarrow]\\
                B_i \ar[uu]
            \end{tikzcd}
        \end{center}
        It follows that each homomorphism $B_i \to C_i$ must also be surjective.
        Thus the covering $\widehat{Y}_K \cap V = \bigcup \iota^{-1}(V_i) \cong \bigcup \op{Sp}B_i$ is an admissible affinoid covering with the property that for each $i$ the preimage $u_K^{-1}(B_i) = \iota'^{-1}(V_i) = \op{Sp}C_i$ is affinoid and the corresponding homomorphism $B_i \to C_i$ is surjective, i.e., $u_K$ is a closed immersion by definition (\cite[\S 9.5.3]{BGR}).

        Being a surjective closed immersion does not guarantee in itself that $u_K$ is an isomorphism.
        It does follow, however, that the ideal corresponding to this closed immersion is locally nilpotent.
        Indeed, locally it is a morphism of the form
        \[
            u_K: \op{Sp}(B/I) \twoheadrightarrow \op{Sp}(B)
        \]
        for $B$ an affinoid $K$-algebra and $I$ an ideal.
        Algebraically this means that every maximal ideal of $B$ contains $I$, so that $I$ is contained in the intersection of all maximal ideals of $B$.
        But all affinoid algebras are Jacobson, so $I$ is nilpotent.

        But we have the additional fact that $Y$ is a reduced scheme - the schematic image of a closed map with reduced source is the topological image endowed with the reduced induced closed subscheme structure.
        It would suffice, then, to show that the rigid-analytic variety $\widehat{Y}_K$ associated to the reduced scheme $Y$ is itself reduced: if $\widehat{Y}_K$ is reduced then the admissible open subset $\widehat{Y}_K \cap V$ is reduced as well, and it would follow that the locally nilpotent ideal sheaf corresponding to the closed immersion $u_K$ is in fact zero.

        To see this, note first that since the question is local on $Y$, and $Y$ is of finite type over $\O_K$, we can assume that $Y$ is affine with an embedding $Y \subseteq \mathbb{A}^d_{\O_K}$.
        The generic fiber $Y_K$ is open in $Y$ and hence reduced, so that $Y_K^{\op{an}}$ is reduced as well
        % [\textbf{do I need to justify this?}]
        But we have 
        \[
        \widehat{Y}_K = Y_K^{\op{an}} \cap B^d(0,1^+)
        \]
        by \cite[Corollary 2.2.12]{LeStum2007} so $\widehat{Y}_K$ must also be reduced, as desired.
    \end{proof}

\section{The Global Filtered Poincar\'e Lemma}\label{gpl_section_gpl}
%     In this situation we know from \cite[Theorem 3.4.12]{LeStum2007} that $u_K$ induces an isomorphism between strict neighborhoods of $\tube{X}{\widetilde{P}}$ in $\tube{\widetilde{Y}}{\widetilde{P}}$ and strict neighborhoods of $\tube{X}{P}$ in $\tube{Y}{P}$. 
    Here we prove the remaining case, which is a variant of the Global Poincar\'e Lemma for rigid cohomology \cite[Lemma 6.5.5]{LeStum2007}:
\begin{theorem}\label{gpl}
    Let
    \begin{center}
        \begin{tikzcd}
            & & P' \ar[dd, "u"]\\
            X_k \ar[r, hook, open] & Y \ar[ur, hook, closed] \ar[dr, hook, closed]\\
            & & P
        \end{tikzcd}
    \end{center}
    be a smooth morphism of smooth algebraic $\O_K$-frames.
    % and let $u_K$ denote the the induced map of tubes $u_K: \tube{Y}{P'} \to \tube{Y}{P}$.
    Then there is a quasi-isomorphism 
    \[
       \jdag_P I_{X,Y,P}^\ell \xrightarrow{\sim} Ru_{K*}\jdag_{P'} (I_{X,Y,P'}^{\ell - \bullet} \otimes_{\O_{\tube{Y}{P'}}} \Omega_{\tube{Y}{P'}/\tube{Y}{P}}^\bullet).
    \]
    Moreover, $\op{Fil}^s_{X,Y,P} \cong Ru_{K*}\op{Fil}^s_{X,Y,P'}$.
    % Let
    % \begin{center}
    %     \begin{tikzcd}
    %         & & P' \ar[dd, "u"]\\
    %         X \ar[r, hook, open] & Y \ar[ur, hook, closed] \ar[dr, hook, closed]\\
    %         & & P
    %     \end{tikzcd}
    % \end{center}
    % be a smooth morphism of frames, and let $u_K$ denote the the induced map of tubes $u_K: \tube{Y}{P'} \to \tube{Y}{P}$.
    % Then there is a quasi-isomorphism 
    % \[
    %    \jdag I_{X,Y,P}^\ell \xrightarrow{\sim} Ru_{K*}\jdag_X (I_{X,Y,P'}^{\ell - \bullet} \otimes_{\O_{\tube{Y}{P'}}} \Omega_{\tube{Y}{P'}/\tube{Y}{P}}^\bullet) =: Ru_{\op{fil-rig, s}}(\jdag_X \O_{\tube{Y}{P'}})
    % \]
    % Moreover, $\op{Fil}^s_{X,Y,P} \cong Ru_{K*}\op{Fil}^s_{X,Y,P'}$.
\end{theorem}

To prove this theorem we will need to localize it not only on $X$ and $P$ as in Corollary \ref{gros_lemma_is_local} but also on $P'$.
The same reduction has been used in \cite[Lemma 6.5.5]{LeStum2007}, but it is not enough to be able to find open affine refinements of $P$ and $P'$ as in \cite[Lemma 2.3.14]{LeStum2007}.
Indeed, to apply \cite[Proposition 6.2.9]{LeStum2007} for example, we need an affine refinement $P = \bigcup P_i$ such that $u^{-1}(P_i)$ is affine, and this is not guaranteed by \cite[Lemma 2.3.14]{LeStum2007}.

\begin{lemma}\label{gpl_local_on_P_and_P'}
   Theorem \hyperref[gpl]{GFPL} is local on $X$, $P$, and $P'$, and we can assume that both $P$ and $P'$ are affine.
\end{lemma}

\begin{proof}
    As before, it is local on $X$ and $P$ by \cite[Proposition 5.2.8]{LeStum2007} and Lemma \ref{gpl_local_downstairs}, respectively. 
    It suffices to prove the following: there exists a collection of affine open subsets $(V_\alpha)_\alpha$ of $P$ such that $\bigcup_\alpha Y \cap V_\alpha = Y$ and, for each $\alpha$, in the pullback of the morphism of frames by the natural inclusion induced by $V_\alpha \hookrightarrow P$
    % \begin{center}
    %     \begin{tikzcd}
    %         & & & U_\alpha \ar[d, hook, open]\\
    %         & & & P' \ar[dd, "u"]\\
    %         & & P' \cap u^{-1}(V_\alpha)\\
    %     & Y \ar[rr, closed] \ar[uurr, closed] & & P\\
    %     Y \cap V_\alpha \ar[ur] \ar[rr] & & V_\alpha \ar[ur]
    %     \end{tikzcd}
    % \end{center}
    \begin{center}
        \begin{tikzcd}
            & P' \cap u^{-1}(V_\alpha) \ar[d, "u"]\\
        Y \cap V_\alpha \ar[r, hook, closed] \ar[ur, hook, closed] & V_\alpha
        \end{tikzcd}
    \end{center}
    there is an affine open $U_\alpha \subseteq P' \cap u^{-1}(V_\alpha)$ admitting a factorization 
    \begin{center}
        \begin{tikzcd}
            & U_\alpha \ar[d, hook, open]\\
            & P' \cap u^{-1}(V_\alpha) \ar[d, "u"]\\
        Y \cap V_\alpha \ar[uur, hook, dotted] \ar[r, hook, closed] \ar[ur, hook, closed] & V_\alpha.
        \end{tikzcd}
    \end{center}
    where the dotted morphism $Y \cap V_\alpha \hookrightarrow U_\alpha$ is a closed immersion. 
    Indeed, Lemma \ref{gpl_local_downstairs} implies that
    % [\textbf{This isn't $Ru_{\op{fil-rig}}$ any more, but the full notation is too long. What to do?}]
    \[
    (Ru_{K*}\jdag_{P'} (I_{X,Y,P'}^{\ell - \bullet} \otimes_{\O_{\tube{Y}{P'}}} \Omega_{\tube{Y}{P'}/\tube{Y}{P}}^\bullet))|_{\tube{Y \cap V_\alpha}{V_\alpha}}
    \]
    coincides with
    \[
    Ru_{K*}\jdag_{P'} (I_{X \cap V_\alpha,Y \cap V_\alpha,U_\alpha}^{\ell - \bullet} \otimes_{\O_{\tube{Y \cap V_\alpha}{U_\alpha}}} \Omega_{\tube{Y \cap V_\alpha}{U_\alpha}/\tube{Y \cap V_\alpha}{V_\alpha}}^\bullet)
    \]
    % \begin{align*}
        % Ru_{K*}\jdag_X (I_{X,Y,P'}^{\ell - \bullet} \otimes_{\O_{\tube{Y}{P'}}} \Omega_{\tube{Y}{P'}/\tube{Y}{P}}^\bullet)|_{\tube{Y \cap V_\alpha}{V_\alpha}} &= Ru_{K*}(\jdag_X (I_{X,Y,P'}^{\ell - \bullet} \otimes_{\O_{\tube{Y}{P'}}} \Omega_{\tube{Y}{P'}/\tube{Y}{P}}^\bullet)|_{\tube{Y \cap V_\alpha}{P' \cap u^{-1}(V_\alpha)}}
        % (Ru_{\op{fil-rig}, s}\O_{\tube{Y}{P'}})|_{\tube{Y \cap V_\alpha}{V_\alpha}} &\cong Ru_{\op{fil-rig},s}\O_{\tube{Y \cap V_\alpha}{P' \cap u^{-1}(V_\alpha)}}\\
    %     % &= Ru_{\op{fil-rig},s}\O_{\tube{Y \cap V_\alpha}{U_\alpha}}
    % \end{align*}
    for each $\alpha$, where the last equality follows from the fact that for a general formal embedding $X \hookrightarrow P$ the tube $\tube{X}{P}$ depends only on an open formal subscheme of $P$ containing $X$.
    Since $\tube{Y}{P} = \bigcup_\alpha \tube{Y \cap V_\alpha}{P} = \bigcup_\alpha \tube{Y \cap V_\alpha}{V_\alpha}$ is an admissible open covering, this shows that we can prove Theorem \hyperref[gpl]{GFPL} by proving it for each morphism of frames
    \begin{center}
        \begin{tikzcd}
            & U_\alpha \ar[d, "u"]\\
            Y \cap V_\alpha \ar[ur, hook, closed] \ar[r, hook, closed] & V_\alpha.
        \end{tikzcd}
    \end{center}

    We now construct such a family $(V_\alpha)$.
    Since we already know that the theorem is local on $P$, let $P = \op{Spec}A$ be affine.
    Closed subschemes of affine schemes are affine, so $Y \cong \op{Spec}(A/I)$ for some ideal $I$.
    To begin, let $P' = \bigcup_i U_i$ be an affine open covering, say $U_i \cong \op{Spec}B_i$ for each $i$, and fix an index $i$.

    Let $\op{Spec}(A/I) \cap U_i = \bigcup_{j} D(\overline{f_{ij}})$ be a covering of the open subscheme $\op{Spec}(A/I) \cap U_i$ of $\op{Spec}(A/I)$ by principal open subsets, where $\overline{f_{ij}} \in A/I$ is the reduction of an element $f_{ij} \in A$.
    Fix an index $j$.
    The pullback of the diagram 
    \begin{center}
        \begin{tikzcd}
            & & U_i \ar[d, hook, open]\\
            & & P' \ar[d, "u"]\\
        \op{Spec}(A/I) \cap U_i \ar[r, hook, open] \ar[uurr, hook, closed] & \op{Spec}(A/I) \ar[ur, hook, closed] \ar[r, hook, closed] & \op{Spec}A
        \end{tikzcd}
    \end{center}
    by the open immersion on the base induced the inclusion $D(f_{ij}) \hookrightarrow P$ is the diagram 
    \begin{center}
        \begin{tikzcd}
            & & U_i \cap u^{-1}(D(f_{ij})) \ar[d, hook, open]\\
            & & P' \cap u^{-1}(D(f_{ij})) \ar[d, "u"]\\
        D(\overline{f_{ij}}) \ar[r, "="] \ar[uurr, hook, closed] & D(\overline{f_{ij}}) \ar[r, hook, closed] \ar[ur, hook, closed] & D(f_{ij}).
        \end{tikzcd}
    \end{center}
    But if we denote by $\vp: A \to B_i$ the morphism of rings corresponding to the composition $v: \op{Spec}B_i \cong U_i \hookrightarrow P' \to \op{Spec}A$, we see that 
    \begin{align*}
      U_i \cap u^{-1}(D(f_{ij})) &= v^{-1}(D(f_{ij})) = D(\vp(f_{ij}))
    \end{align*}
    is affine, and we're done.
\end{proof}

As a technicality, we also need the following lemma:
\begin{lemma}\label{gpl_locally_free_sheaf_equivalent}
    In the setting of Theorem \hyperref[gpl]{GFPL}, suppose the base change map
        \[
            \jdag_P I_{X,Y,P}^\ell \xrightarrow{\sim} Ru_{K*}\jdag_{P'} (I_{X,Y,P'}^{\ell - \bullet} \otimes_{\O_{\tube{Y}{P'}}} \Omega_{\tube{Y}{P'}/\tube{Y}{P}}^\bullet)
        \]
        is an isomorphism.
        Then for any finite locally free $\O_{\tube{Y}{P}}$-module $\mathcal{M}$, the base change map 
        \begin{align*}
               \jdag_P I_{X,Y,P}^\ell \otimes \mathcal{M} &\xrightarrow{\sim} Ru_{K*}\jdag_{P'} (I_{X,Y,P'}^{\ell - \bullet} \otimes_{\O_{\tube{Y}{P'}}}u_K^*\mathcal{M} \otimes_{\O_{\tube{Y}{P'}}} \Omega_{\tube{Y}{P'}/\tube{Y}{P}}^\bullet)
        \end{align*}
        is an isomorphism.
\end{lemma}

\begin{proof}
    % The proof follows the proof of Lemma \ref{gros_proposition_locally_free_equivalent} verbatim. 
    This follows immediately from the following projection formula: let $f: X \to Y$ be a morphism of rigid-analytic spaces, let $\cF^\bullet$ be a complex of $\O_X$-modules with differential operators of order 1, and let $\cE$ be a finite locally free $\O_Y$-module.
    Then there is a functorial isomorphism 
    \[
    Rf_*\blt{\cF} \otimes_{\O_Y} \cE \xrightarrow{\sim} Rf_*(\blt{\cF} \otimes_{\O_X} f^*\cE).
    \]
    
    To see this, note that there is a morphism of spectral sequences
    \begin{center}
        \begin{tikzcd}
        E_1^{p,q} = \cE \otimes_{\O_X} R^pf_*(\cF^q) \ar[r,Rightarrow] \ar[d] & \cE \otimes_{\O_X} R^{p+q}\blt{\cF} \ar[d]\\
        E_1^{p,q} = R^pf_*(f^*\cE \otimes_{\O_Y} \cF^q) \ar[r, Rightarrow] & R^{p+q}f_*(f^*\cE \otimes_{\O_Y} \blt{\cF}).
        \end{tikzcd}
    \end{center}
    where the morphisms are defined as in \cite[\href{https://stacks.math.columbia.edu/tag/01E6}{Section 01E6}]{stacks-project} and \cite[Proposition 5.6]{Hartshorne1966}; here the assumption that $\cE$ is finite locally free is crucial.
    But the morphism at the $E_1$-level is an isomorphism by \cite[\href{https://stacks.math.columbia.edu/tag/01E8}{Lemma 01E8}]{stacks-project}, so it follows that the induced morphism on the limit is an isomorphism as well.
    Since $\cE$ is locally free and hence flat, this implies the claimed projection formula. 
    % \begin{align*}
    %     E_1^{p,q} \  \&Rightarrow
    % \end{align*}
\end{proof}

% \begin{proof}
%     This is an immediate consequence of the following projection formula in the derived category \cite[\href{https://stacks.math.columbia.edu/tag/0B54}{Lemma 0B54}]{stacks-project}. Let $f: X \to Y$ is a morphism of ringed spaces and let $E \in D(\O_X)$ and $K \in D(\O_Y)$. If $K$ is perfect (see  \cite[\href{https://stacks.math.columbia.edu/tag/08CL}{Section 08CL}]{stacks-project}), then 
%     \[
%     Rf_*E \otimes_{\O_Y}^{\mathbb{L}} K = Rf_*(E \otimes^{\mathbb{L}} Lf^*K).  
%     \]
%     Note that the functor $u_K^*$ is exact and locally free sheaves are both perfect complexes and their own projective resolutions.
%     Using \cite[Proposition 5.3.2]{LeStum2007} we can infer that 
%     \begin{align*}
%         Ru_{\op{fil-rig},s}(\jdag_Xu_K^*\mathcal{M}) &= Ru_{K*}\jdag_X (I_{X,Y,P'}^{\ell - \bullet} \otimes_{\O_{\tube{Y}{P'}}}u_K^*\mathcal{M} \otimes_{\O_{\tube{Y}{P'}}} \Omega_{\tube{Y}{P'}/\tube{Y}{P}}^\bullet)\\
%         &\cong Ru_{K*}\jdag_X (I_{X,Y,P'}^{\ell - \bullet} \otimes_{\O_{\tube{Y}{P'}}} \Omega_{\tube{Y}{P'}/\tube{Y}{P}}^\bullet) \otimes \mathcal{M}\\
%         &= Ru_{\op{fil-rig},s}(\jdag_X \O_{\tube{Y}{P'}}) \otimes \mathcal{M}
%     \end{align*}
%     so we get (2) from (1) by tensoring each side with $\mathcal{M}$.
% \end{proof}

For brevity, let $A = \tube{Y}{P'}$, $B = \tube{Y}{P}$, and write $I_A := I_{X,Y,P'}$ and $I_B := I_{X,Y,P}$.
As a prerequisite for proving the Global Filtered Poincar\'e Lemma, we extend the Gauss-Manin construction \cite[pp.82]{LeStum2007} to our context.
Since $u_K$ is smooth, there is a short exact sequence of locally free sheaves 
\[
    0 \to u_K^*\Omega_B^1 \to \Omega_A^1 \to \Omega^1_{A/B} \to 0.
\]
It is described in detail in \cite[pp.82]{LeStum2007} and \cite[\href{https://stacks.math.columbia.edu/tag/0FMK}{Section 0FMK}]{stacks-project} how this naturally provides a filtration by locally free subsheaves 
\[
\op{Fil}^\ell\Omega^r_A := \op{Im}(u_K^*\Omega_B^\ell \otimes \Omega^{r-\ell}_A \to \Omega^r_A)    
\]
with graded parts 
\[
\op{Gr}^\ell \Omega^r_A = \Omega^{r-\ell}_{A/B} \otimes u_K^*\Omega^\ell_B.
\]
Varying over all $r$, we obtain a filtration $\{\op{Fil}^\ell \Omega_A^\bullet\}_{\ell \ge 0}$ on the complex $\Omega^\bullet_A$ with graded parts
\[
\op{Gr}^\ell \Omega^\bullet_A = \Omega_{A/B}^\bullet[-\ell] \otimes u_K^*\Omega^\ell_B.
\]
More generally, if $\cE$ is an $\O_A$-module with an overconvergent integrable connection, then on $M^\bullet = \cE \otimes \Omega^\bullet_A$ we have a filtration 
\[
   \op{Fil}^\ell M^\bullet = \op{Im}(M^\bullet[\ell] \otimes u_K^*\Omega_B^\ell \to M^\bullet)
\]
with graded parts
\[
    \op{Gr}^\ell M^\bullet = (\cE \otimes \Omega^\bullet_{A/B})[-\ell] \otimes u_K^*\Omega_B^\ell.
\]

For any $r,s \ge 0$ the above filtration on $\Omega^r_A$ canonically defines a filtration on the submodule $\jdag_A I_A^{s-r}\Omega^r_A$ in the natural way (detailed in \cite[\href{https://stacks.math.columbia.edu/tag/0120}{Section 0120}]{stacks-project}), with graded parts 
\[
\op{Gr}^\ell(I_A^{s-r}\Omega^r_A) = I_A^{s-r}(\Omega^{r-\ell}_{A/B} \otimes_A u_K^*\Omega_B^\ell),
\]
which induces a filtration on $I_A^{s-\bullet}\Omega_A^r$ with graded parts 
\[
\op{Gr}^\ell(I_A^{s-\bullet}\Omega_A^\bullet) = I_A^{s-\bullet}(\Omega_{A/B}^\bullet[-\ell] \otimes_A u_K^*\Omega_B^\ell).
\]
We can then obtain a filtration on $M^\bullet = \cE \otimes_A I_A^{s-\bullet}\Omega^\bullet_A$ as above; indeed, our situation of an overconvergent module equipped with a submodule of the trivial connection is a special case of a module with an overconvergent integrable connection.
We omit the details for brevity. 

The following connects the Gros filtration to the main quasi-isomorphism of Theorem \hyperref[gpl]{GFPL}:
\begin{lemma}\label{gpl_gros_to_total_complex_lemma}
    For fixed $s \ge 0$, let 
    \[
        C_u^{a,b}(s) := \jdag_A(I_A^{s-a-b} \otimes_A \Omega_{A/B}^a \otimes_A u_K^*\Omega_B^b)
    \]
    defined for $a,b \in \mathbb{Z}^{\ge 0}$, where the vertical (resp. horizontal) differential is induced by that of $\Omega_B^\bullet$ (resp. $\Omega^\bullet_{A/B}$).
    Then there is a natural isomorphism 
    \[
    \op{Tot}(C_u^{\bullet,\bullet}(s)) \cong \op{Fil}^s_{X,Y,P'}    
    \]
\end{lemma}

\begin{proof}
    We prove this by comparing the the natural filtrations on these two complexes. 
    On the former we assign the row filtration 
    \[
    F^i\op{Tot}(C_u^{\bullet,\bullet}(s)) = \op{Tot}C_i^{\bullet,\bullet}    
    \]
    where 
    \[
    C_i^{a,b} = 
    \begin{cases}
        C_u^{a,b} &\text{ if } b \ge i\\
        0 &\text{ if }b < i
    \end{cases}    
    \]
    % [graphically it's $C_u^{i,j}$ with rows 0 through $i-1$ replaced with zeroes.]
    The $\ell^{\op{th}}$ graded part $\op{Gr}^\ell_F(\op{Tot}(C_u^{\bullet,\bullet}(s)))$ is the $\ell^{\op{th}}$-row of $C_u^{\bullet,\bullet}$; more precisely, 
    \begin{align*}
    \op{Gr}^\ell_F(\op{Tot}(C_u^{\bullet,\bullet}(s))) &= C_u^{\bullet - \ell, \ell}\\
    &= \jdag_A(I_A^{s-\bullet} \otimes_A \Omega^{\bullet}_{A/B}[-\ell]\otimes_A u_K^*\Omega_B^\ell).
    \end{align*}

    On the other hand, our discussion of the Gauss-Manin construction says that $\op{Fil}^s_{X,Y,P'}$ has a natural filtration $F$ which also has graded parts 
    \[
        \op{Gr}^\ell \op{Fil}^s_{X,Y,P'} = \jdag_A(I_A^{s-\bullet} \otimes_A \Omega^{\bullet}_{A/B}[-\ell]\otimes_A u_K^*\Omega_B^\ell).
        % \jdag_A(I_A^{s-\bullet} \otimes_A u_K^*\Omega_B^\ell \otimes_A \Omega^\bullet_{A/B}[-\ell])    
    \]
    Thus, since both complexes are bounded and finite, to prove the claim it suffices to find a filtration-preserving morphism of complexes 
    \[
    \op{Tot}(C_u^{\bullet,\bullet}(s)) \to \op{Fil}^s_{X,Y,P'}.
    \]
    For this the natural morphism suffices.
    Namely we choose, for each $i \ge 0$, the map
    \[
        \op{Tot}(C_u^{\bullet,\bullet}(s))^i \twoheadrightarrow \jdag_A(I_A^{s-i} \otimes_A u_K^*\Omega^i_B) \hookrightarrow \jdag_AI_A^{s-i} \otimes \Omega_A^i = (\op{Fil}^s_{X,Y,P'})^i.
    \]
    This map is filtration-preserving since for $\ell \le i$ it restricts to the inclusion 
    \begin{align*}
        F^\ell\op{Tot}(C_u^{\bullet,\bullet}(s))^i &\twoheadrightarrow \jdag_A(I_A^{s-i} \otimes_A u_K^*\Omega^i_B)\\
        &\hookrightarrow \op{Im}(\jdag_A I_A^{s-i} \otimes_A \Omega_A^{i-\ell} \otimes_A u_K^*\Omega_B^\ell  \to \jdag_A I_A^{s-i} \Omega_A^i)\\
        &= F^\ell (\op{Fil}^s_{X,Y,P'})^i.
    \end{align*}
\end{proof}
By virtue of the previous lemma, to prove that $Ru_{K*}\op{Fil}^s_{X,Y,P'} \cong \op{Fil}^s_{X,Y,P}$, it suffices to prove that $Ru_{K*}\op{Tot}(C^{\bullet,\bullet}(s)) \cong \op{Fil}^s_{X,Y,P}$.
To accomplish this we may show that the direct image $Ru_{K*}C^{k,\bullet}(s)$ of each column of the total complex is isomorphic to $(\op{Fil}^s_{X,Y,P})^k$, i.e.,
    \begin{equation}\label{gpl_double_complex_column}
    \jdag_B(I_B^{s-k}\otimes_B \Omega_B^k) \cong Ru_{K*}\jdag_A(I_A^{s-k-\bullet} \otimes u^*\Omega_B^k \otimes \Omega_{A/B}^\bullet).
    \end{equation}
By Lemma \ref{gpl_locally_free_sheaf_equivalent}, it suffices to prove that
    \[
    \jdag_B I_B^{s-k} \cong Ru_{K*}\jdag_A(I_A^{s-k-\bullet} \otimes \Omega^\bullet_{A/B}),
    \]
which justifies our claim in the statement of Theorem \hyperref[gpl]{GFPL}.

In the course of reducing this quasi-isomorphism to an explicitly computable special case, we'll need the following fact:
\begin{lemma}\label{gpl_composition}
    Consider a diagram of smooth morphisms of smooth algebraic $\O_K$-frames 
    \begin{center}
        \begin{tikzcd}
            &   & P' \ar[d,"u"]\\
        X_k \ar[r,hook, open] & Y \ar[ur, hook, closed] \ar[r, hook, closed] \ar[dr, hook, closed] & P \ar[d,"v"]\\
            &   & P''.
        \end{tikzcd}
    \end{center}
    If isomorphism (\ref{gpl_double_complex_column}) holds for the frames corresponding to $v$ and $u$, respectively, then it holds for the frame corresponding to $v \circ u$.
\end{lemma}

\begin{proof}
    Fix $s \ge 0$.
    We keep the notation of the previous proof and further write $C = \tube{Y}{P''}$ and $I_C := I_{X,Y,P''}$ for the ideal corresponding to the algebraic frame $(X \subseteq Y \subseteq P'')$.

    Consider $u$ as a smooth morphism of smooth $C$-frames, via the structure morphisms 
    \begin{center}
        \begin{tikzcd}
            A \ar[rr, "u"] \ar[dr, swap, "v \circ u"] && B \ar[dl, "v"]\\
            & C
        \end{tikzcd}
    \end{center}
    The Gauss-Manin construction with respect to this morphism provides a filtration on $I_A^{s-\bullet}\Omega^\bullet_{A/C}$ with graded parts 
    \[
    \op{Gr}^\ell(I_A^{s-\bullet}\Omega^\bullet_{A/C}) = I_A^{s-\bullet}(\Omega^\bullet_{A/B}[-\ell] \otimes u_K^*\Omega^\ell_{B/C})    
    \]
    for $\ell \ge 0$.

    With this in mind, fix $k \ge s$ and define 
    \[
        K^\bullet := \jdag_A(I_A^{s-k-\bullet}  \otimes \Omega^\bullet_{A/C}\otimes (v \circ u)^*_K\Omega_C^k)
    \]
    which we can rewrite this as 
    \[
    K^\bullet = \jdag_A((v \circ u)_K^*\Omega_C^k) \otimes I_A^{s-k-\bullet}\Omega^\bullet_{A/C}.
    \]
    The Gauss-Manin construction then provides a filtration on $K^\bullet$ with graded parts 
    \begin{align*}
    \op{Gr}^\ell K^\bullet &= \jdag_A ((v \circ u)_K^*\Omega_C^k) \otimes I_A^{s-k-\bullet}(\Omega_{A/B}^{\bullet}[-\ell] \otimes u_K^*\Omega^\ell_{B/C})\\
    &= \jdag_A(I_A^{s-k-\ell-\bullet} \otimes \Omega^\bullet_{A/B} \otimes (v \circ u)_K^*\Omega_C^k )[-\ell] \otimes u_K^*\Omega^\ell_{B/C}.
    \end{align*}
    Applying the functor $Ru_{K*}$ to both sides and using the projection formula, we obtain an isomorphism
    \begin{equation}\label{gros_gauss_manin_RuK}
        Ru_{K*}\op{Gr}^\ell K^\bullet \cong Ru_{K*}\jdag_A(I_A^{s-k-\ell-\bullet}  \otimes \Omega^\bullet_{A/B}\otimes (v \circ u)_K^*\Omega_C^k)[-\ell] \otimes \Omega^\ell_{B/C}.
    \end{equation}
    But by assumption we have an isomorphism 
    \[
    Ru_{K*}\jdag_A(I_A^{s-k-\ell-\bullet} \otimes \Omega^\bullet_{A/B}) \cong \jdag_B I_B^{s-k-\ell} 
    \]
    which provides, in combination with Lemma \ref{gpl_locally_free_sheaf_equivalent}, an isomorphism 
    \begin{align*}
        Ru_{K*}\jdag_A(I_A^{s-k-\ell-\bullet} \otimes \Omega^\bullet_{A/B}\otimes (v \circ u)_K^*\Omega_C^k) &\cong \jdag_B I_B^{s-k-\ell} \otimes v_K^*\Omega_C^k.
    \end{align*}
    Plugging this into Equation (\ref{gros_gauss_manin_RuK}) and using the fact that $Ru_{K*}\op{Gr}^\ell K^\bullet = \op{Gr}^\ell Ru_{K*}K^\bullet$, we finally obtain an isomorphism 
    \[
    \op{Gr}^\ell Ru_{K*}K^\bullet \cong (\jdag_B I_B^{s-k-\ell} \otimes v_K^*\Omega_C^k)[-\ell] \otimes \Omega^\ell_{B/C}.
    \]
    This isomorphism implies that the filtration on $Ru_{K*}K^\bullet$ is canonical in the sense of \cite[pp.79]{FaisceauxPervers}.
    It follows by \cite[Proposition 3.1.6]{FaisceauxPervers} that 
    \[
    Ru_{K*}K^\bullet \cong \jdag_B(I_B^{s-k-\bullet} \otimes \Omega_{B/C}^\bullet \otimes v_K^*\Omega_C^k)
    \]
    and since the desired isomorphism holds for the morphism corresponding to $v$ also by assumption, we see that
    \begin{align*}
        R(v \circ u)_{K*}K^\bullet &= Rv_{K*}(Ru_{K*}K^\bullet)\\
        &= Rv_{K*}(\jdag_B(I_B^{s-k-\bullet} \otimes \Omega_{B/C}^\bullet \otimes v_K^*\Omega_C^k))\\
        &\cong \jdag_C I_C^{s-k} \otimes \Omega_C^k
    \end{align*}
    which was what we wanted to show.
\end{proof}
        


\begin{proof}[Proof of the Global Filtered Poincar\'e Lemma]
    By Corollary \ref{gpl_local_on_P_and_P'} we can assume that $P$ and $P'$ are both affine. 
    Further, since the question is local on $X_k$, we can assume that the complement of $X_k$ in $Y_k$ is a hyperplane.
    The morphism in Lemma \hyperref[gpl]{GFPL} is not a strict Cartesian morphism of frames in general, but note that both morphisms in the composition 
    \[
    Y \to Y \times_P Y \to Y \times_P P'    
    \]
    are closed immersions, the first since $Y \hookrightarrow P$ is closed and hence separated, and second because both $Y \xrightarrow{id} Y$ and $Y \hookrightarrow P'$ are closed immersions.
    This inserts into a morphism of frames 
    \begin{center}
        \begin{tikzcd}
            X_k \ar[r,hook,open] \ar[dr, hook, open] & Y \ar[r,hook, closed] \ar[d,hook, closed]& P'\\
            & Y \times_P P' \ar[ur, hook, closed]
        \end{tikzcd}
    \end{center}
    Lemma \ref{gpl_closed_immersion} says that the theorem is independent of closed immersions, so it is equivalent to prove the theorem for the morphism 
    \begin{center}
        \begin{tikzcd}
            & & P' \ar[dd, "u"]\\
            X_k \ar[r, hook, open] & Y \times_P P' \ar[ur, hook, closed] \ar[dr, hook, closed]\\
            & & P
        \end{tikzcd}
    \end{center}
    In other words, we may even assume that the morphism of algebraic frames is strictly Cartesian. 

    For any $d \ge 1$, let $\mathbb{A}^d_P := \mathbb{A}^d \times_{\Z} P$ denote the affine space of dimension $d$ over $P$.
    By \cite[Proposition 3.3.13]{LeStum2007} we may assume, locally, that there is an \'etale morphism of affine frames 
    \begin{center}
        \begin{tikzcd}
            & & P' \ar[dd,"u'"]\\
            X_k \ar[r,hook, open] & Y \ar[ur, hook, closed] \ar[dr, hook, closed]\\
            & & \mathbb{A}_P^d
        \end{tikzcd}
    \end{center}
    where $Y$ is embedded into $\widehat{\mathbb{A}}^d_P$ using the zero section.
    We thus have a factorization 
    \begin{center}
        \begin{tikzcd}
            &   & P' \ar[d,"u'"] \ar[dd, bend left=50, "u"]\\
        X_k \ar[r,hook, open] & Y \ar[ur, hook, closed] \ar[r, hook, closed] \ar[dr, hook, closed] & \mathbb{A}_P^d \ar[d,"\pi_P"]\\
            &   & P
        \end{tikzcd}
    \end{center}
    As we already have the result for the upper morphism of frames by Lemma \hyperref[gpl_proper_etale]{PE},
    Lemma \ref{gpl_composition} reduces the theorem to the case
    \begin{center}
        \begin{tikzcd}
            & & \mathbb{A}_P^d\ar[dd,"u"]\\
            X_k \ar[r,hook, open] & Y \ar[ur, hook, closed] \ar[dr, hook, closed]\\
            & & P
        \end{tikzcd}
    \end{center}
    with $u$ the canonical projection.
    Chaining together multiple instances of Lemma \ref{gpl_composition}, we may further reduce ourself to the case $d = 1$.
    Since we may prove the result by checking it on $R\Gamma(W,-)$ for all affinoid subsets $W \subseteq \tube{Y}{P}$, it remains to prove the following variation of the Local Poincar\'e Lemma \cite[Lemma 6.5.7]{LeStum2007}.
\end{proof}

\begin{lemma}[(Local Filtered Poincar\'e Lemma)]
    Let $(X_k \subseteq Y \subseteq P)$ be an affine algebraic $\O_K$-frame where the complement of $X_k$ in $Y_k$ is a hypersurface, and let
    \[
        p: \mathbb{A}_P^1 \to P
    \]
    be the projection.
    Consider the morphism of affine frames
    \begin{center}
        \begin{tikzcd}
                &   &  \mathbb{A}^1_P \ar[dd, "p"]\\
            X_k \ar[r, hook, open] & Y \ar[ur, hook, closed] \ar[dr, hook, closed]\\
                &   &   P 
        \end{tikzcd}
    \end{center}
    Let $I_{\mathbb{A}^1}$ denote the ideal corresponding to the frame $(X_k \subseteq Y \subseteq \mathbb{A}^1_P)$ and $I_P$ the ideal corresponding to the frame $(X_k \subseteq Y \subseteq P)$.  
    If $W$ is an affinoid open subset of $\tube{Y}{P}$, there is a canonical isomorphism 
    \[
    \Gamma(W, \jdag_X I_P^\ell) \cong R\Gamma(W \times \mathbb{D}(0,1^{-}), \jdag_X I_{\mathbb{A}^1}^\ell \xrightarrow{\partial/\partial t} \jdag_X I_{\mathbb{A}^1}^{\ell - 1}).
    \]
\end{lemma}

We start by giving an explicit description of $I_{\mathbb{A}^1}^\ell$.

\begin{lemma}\label{lpl_explicit}
    Let $W = \op{Sp}(A) \subseteq \tube{Y}{P}$ be an affinoid open subset and let $0 < \eta < 1$.
    We have 
    \[
    \Gamma(W \times \mathbb{D}(0,\eta^+),I^n_{\mathbb{A}^1}) = \left\{\sum_{i \ge 0} a_it^i \in A\{t\}_{\eta}\,:\, a_i \in \Gamma(W,I_P^{n-i})\right\}
    \]
    for all $n \ge 0$, where
    \[
        A\{t\}_{\eta} = \left\{\sum_{i \ge 0}a_it^i \,:\, a_i \in A, |a_i|\eta^i \to 0\right\}.
    \]
    % Furthermore, the map $I_{\mathbb{A}^1}^n(W) \xrightarrow{\partial/\partial t} I_{\mathbb{A}^1}^{n-1}(W)$ is surjective for all $n \ge 1$.
\end{lemma}
\begin{proof}
    Let $\widehat{P}_K = \op{Sp}(R)$.
    The diagram 
    \begin{center}
        \begin{tikzcd}
            & \tube{Y}{\mathbb{A}^1_P} \ar[dd,"u"]\\
        \widehat{Y}_K \ar[ur, hook, closed, "\iota_{\mathbb{A}^1}"] \ar[dr, hook, closed, swap, "\iota_P"]\\
            & \tube{Y}{P}
        \end{tikzcd}
    \end{center}
    corresponding to the ideals $I_P$ and $I_{\mathbb{A}^1}$ induces on sheaves of sections the diagram 
    \begin{center}
        \begin{tikzcd}
            & \O_{\tube{Y}{\mathbb{A}^1}}(W \times D(0,1^{-})) \cong A \otimes_R R\{t\} \cong A\{t\} \ar[dl, swap, "\iota_{\mathbb{A}^1}^\flat(W)"]\\ 
        \iota_{P*}\O_{\widehat{Y}_K}(W)\\
            & \O_{\tube{Y}{P}}(W) = A \ar[uu,hook] \ar[ul, "\iota_P^\flat(W)"]
        \end{tikzcd}
    \end{center}
    where the vertical arrow is the natural inclusion. Since $Y$ is embedded into $\mathbb{A}^1_P$ by the zero section, $\iota_{\mathbb{A}^1}^\flat(W)$ is the map $t \mapsto 0$.
    By restriction, we get a similar diagram with the uppermost group being replaced by $\O_{\tube{Y}{\mathbb{A}^1}}(W \times D(0,\eta^+)) \cong A\{t\}_{\eta}$; as an abuse of notation we'll use the same notation for any $\eta > 0$.

    Hence if $\Gamma(W,I_P) = \op{ker}(\iota_P^\flat(W))$ is generated as an $A$-module by sections $(h_1,\dots,h_s)$ (such a finite set of generators exists since $I_P$ is coherent), then $\Gamma(W \times D(0,\eta^+), I_{\mathbb{A}^1})$ is generated as an $A\{t\}_{\eta}$-module by $(h_1,\dots,h_s,t)$.

    With this description it is easy to see that 
    \[
    \Gamma(W \times D(0,\eta^+), I_{\mathbb{A}^1}) = \left\{\sum_{i \ge 0}a_it^i \in A\{t\}_\eta \,:\, a_0 \in \Gamma(W, I_P)\right\}
    \]
    which is precisely the lemma when $n = 1$.
    We'll prove the general case by induction.

    For the rest of the proof, we write $I_P^n$ and $I_{\mathbb{A}^1}^n$ for $\Gamma(W,I_P^n)$ and $\Gamma(W \times \mathbb{D}(0,\eta^+), I_{\mathbb{A}^1}^n)$, respectively, with the implicit assumption that we're always working with sections.
    
    Let
    \[
    S^n := \left\{\sum_{i \ge 0} a_it^i \in A\{t\}_{\eta}\,:\, a_i \in I_P^{n-i}\right\}.
    \]
    First note that this is an ideal. Indeed, if $\sum_{j \ge 0}b_jt^j \in A\{t\}_{\eta}$ is arbitrary and $\sum_{i \ge 0}a_it^i \in S^n$ then 
    \[
        \left(\sum_{j \ge 0} b_jt^j\right)\left(\sum_{i \ge 0} a_it^i\right) = \sum_{k \ge 0}\left(\sum_{i + j = k}a_ib_j\right)t^k,
    \]
    but $a_ib_j \in I_P^{n-i} \subseteq I_P^{n-k}$ for all such $i$ and $j$, so this product is in $S^n$.

    As our inductive hypothesis suppose $I^{n-1}_{\mathbb{A}^1} = S^{n-1}$.
\begin{itemize}
        \item $S^n \subseteq I_{\mathbb{A}_P^1}^n$: Let $\sum a_it^i \in S^n$ and view this sum as
        \begin{align*}
            \sum_{i \ge 0}a_it^i &= a_0 + t\sum_{i \ge 0}a_{i+1}t^i
        \end{align*}
        Since $a_{i+1} \in I_P^{n - i - 1}$ for all $i$ by assumption, we have by the inductive hypothesis that $\sum_{i \ge 0}a_{i+1}t^i \in I^{n-1}_{\mathbb{A}_P^1}$, and hence that $t\sum_{i \ge 0}a_{i+1}t^i \in I^n_{\mathbb{A}_P^1}$. Since also
        \[
        a_0 \in I_P^n \subseteq I^n_{\mathbb{A}_P^1}    
        \]
        we have that each summand is in $I^n_{\mathbb{A}_P^1}$, which is all we need.

        \item $I^n_{\mathbb{A}_P^1} \subseteq S^n$: By definition $I^n_{\mathbb{A}_P^1} = I^{n-1}_{\mathbb{A}_P^1}\cdot I_{\mathbb{A}_P^1}$. Applying the inductive hypothesis we see that this ideal consists of sums of elements of the form
        \[
            \left(\sum_{i \ge 0}a_it^i\right) \left(\sum_{j \ge 0}b_jt^j\right)    
        \]
        where $a_i \in I_P^{n - i - 1}$ for all $i$ and $b_0 \in I_P$. We rearrange this as 
        \begin{align*}
            \left(\sum_{i \ge 0}a_it^i\right) \left(\sum_{j \ge 0}b_jt^j\right) &= \left(\sum_{i \ge 0}a_it^i\right)\left(b_0 + t\left(\sum_{j \ge 1}b_jt^{j-1}\right)\right)\\
            &=
            \left(\sum_{i \ge 0}b_0a_it^i\right) + t\left(\sum_{i \ge 0}a_it^i\right)\left(\sum_{j \ge 1}b_jt^{j-1}\right)\\
            &= \left(\sum_{i \ge 0}b_0a_it^i\right) + \left(\sum_{i \ge 1}a_{i-1}t^i\right)\left(\sum_{j \ge 1}b_jt^{j-1}\right)
        \end{align*}
        We have $b_0a_i \in I_P^{n-i}$ for all $i$ so the first summand is in $S^n$. Likewise, $a_{i-1} \in I_P^{n-i}$ for all $i$ so $\sum_{i \ge 1} a_{i-1}t^i \in S^n$, and since $S^n$ is an ideal the second summand is in $S^n$ also.
    \end{itemize}
\end{proof}

\begin{proof}[Proof (of local filtered Poincar\'e lemma)]
    Fix a complement $Z = V(f)$ for $X$ in $Y$.
    Let $\{V^\lambda\}_{\lambda < 1}$ be the usual system of strict neighborhoods
    \[
    V^\lambda = \tube{Y}{P} \setminus \tube{Z}{P\lambda}
    \]
    of $\tube{X}{P}$ in $\tube{Y}{P}$ (see, e.g., \cite[Proposition 3.3.1]{LeStum2007}).
    If $M(\ell) := \Gamma(W, \jdag_P I_P^\ell)$, it follows from \cite[Proposition 5.1.12]{LeStum2007} that 
    \[
        M(\ell) = \varinjlim_\lambda M^\lambda(\ell)
    \]
    where $M^\lambda(\ell) := \Gamma(W \cap V^\lambda, I_P^\ell)$.

    Choose a sequence $\eta_k \xrightarrow{<} 1$ and define 
    \[
        M_k(\ell) := \Gamma(W \times \mathbb{D}(0,\eta_k^+),\jdag_{\mathbb{A}^1} I^\ell_{\mathbb{A}^1}).
    \]
    In order to apply \cite[Proposition 5.1.12]{LeStum2007} to this space of sections, we explicitly describe the $V^\lambda$ construction with respect to the algebraic frame $(X_k \subseteq Y \subseteq \bbA_P^1)$, which we denote by $V^\lambda_{\bbA^1_P}$
    If the complement of $X$ in $Y$ is the hypersurface $V(f)$ then, since $Y$ is defined in $\bbA_P^1$ by the zero section $t \mapsto 0$ by assumption, the complement as a closed subset of $\bbA_P^1$ is given by
    \[
    Z_{\bbA_P^1} = V(f,t) \cap (\bbA_P^1)_k \subseteq \bbA_P^1. 
    \]
    Fix an $\eta_k$.
    Given $\lambda > 0$, we have by \cite[Lemma 3.2.2]{LeStum2007} that
    \[
    V^\lambda_{\bbA_P^1} = (V^\lambda \times \mathbb{D}(0,1^+)) \cup (\tube{Y}{P} \times \mathbb{A}(0,\lambda^+,1^+))
    \]
    where $\bbA(0,\lambda^+,1^+) = \mathbb{D}(0,1^+) \setminus \mathbb{D}(0,\lambda^-)$ is the closed annulus of inner radius $\lambda$ and outer radius 1.

    Since in \cite[Propoition 5.1.12]{LeStum2007} we care only about the limit $\lambda \to 1$, we may suppose that $\lambda > \eta_k$.
    In this case, we have
    \[
    (W \times \mathbb{D}(0,\eta_k^+)) \cap V^\lambda_{\bbA_P^1} = (W \cap V^\lambda) \times \mathbb{D}(0,\eta_k^+)    
    \]
    since $\mathbb{D}(0,\eta_k^+) \cap \bbA(0,\lambda^+,1^+) = \varnothing$.
    Thus by \cite[Proposition 5.1.12]{LeStum2007},
    \[
    % \Gamma(W \times \mathbb{D}(0,\eta_k^+), \jdag_{\bbA^1}I^\ell_{\bbA^1})
    M_k(\ell) \cong \varinjlim_{\lambda} M_k^\lambda(\ell)
    \]
    where
    \[ 
    M_k^\lambda(\ell) := \Gamma((W \cap V^\lambda) \times \mathbb{D}(0,\eta_k^+), I^\ell_{\bbA^1})
    \]

    Lemma \ref{lpl_explicit} does not directly describe $M_k^\lambda(\ell)$ since $W \cap V^\lambda$ is not necessarily affinoid.
    However, $W \cap V^\lambda$ is quasi-Stein in the sense of \cite[pp.\,146]{Chiarellotto1990}.
    Indeed, suppose $Y = V(\overline{g_1},\dots,\overline{g_r}) \cap P_k \subseteq P_k$ and $Y_k \setminus X_k = V(\overline{f}) \cap Y_k \subseteq P_k$.
    If $g_1,\dots,g_r,f \in P$ are liftings of $\overline{g_1},\dots,\overline{g_r},\overline{f}$, respectively, then by \cite[Proposition 3.2.15]{LeStum2007} we have
    \[
    W \cap V^\lambda = W \cap \widetilde{V}^\lambda    
    \]
    where 
    \begin{align*}
    \widetilde{V}^\lambda &:= \{x \in \tube{Y}{P} \,:\, |f(x)| \ge \lambda\}\\
    &= \{x \in P_K \,:\, |g_1(x)|,\dots,|g_r(x)| < 1, |f(x)| \ge \lambda\}
    \end{align*}
    using \cite[Lemma 2.2.10]{LeStum2007} to explicitly describe $\tube{Y}{P}$. 
    This is easily seen to be an increasing limit of Laurent, and hence affinoid, subdomains satisfying the density condition for being quasi-Stein.
    In addition, since the intersection of an affinoid subdomain with a quasi-Stein subspace is again quasi-Stein, we see that $W \cap \widetilde{V}^\lambda = W \cap V^\lambda$ is quasi-Stein, as needed.

    Thus $W \cap V^\lambda$ is also an increasing limit of affinoid subdomains, say 
    \[
    W \cap V^\lambda = \bigcup_n \op{Sp}A^\lambda_n.    
    \]
    We thus have
    \begin{align*}
        \Gamma((W \cap V^\lambda) \times \bbD(0,\eta_k^+), I^\ell_{\bbA^1}) &= \Gamma\left(\bigcup_n (\op{Sp}A_n^\lambda \times \bbD(0,\eta_k^+)), I^\ell_{\bbA^1}\right)\\
        &= \varprojlim_n \Gamma(\op{Sp}A_n^\lambda \times \bbD(0,\eta_k^+), I^\ell_{\bbA^1})
    \end{align*}
    which, using the explicit description from Lemma \ref{lpl_explicit}, can be written as
    \begin{align*}
        \varprojlim_n \left\{\sum_{i \ge 0} a_{n,i}t^i \in A_n^\lambda\{t\}_{\eta}\,:\, a_{n,i} \in \Gamma(\op{Sp}A_n^\lambda,I_P^{n-i})\right\}.
    \end{align*}
    Passing to rings with the inverse limit topology, we can write this as 
    \[
        \left\{\sum_{i \ge 0}a_it^i \in \varprojlim_n A_n^\lambda\{t\}_\eta \,:\, a_n \in \varprojlim_n \Gamma(\op{Sp}A_n^\lambda, I_P^{n-i})\right\}.
    \]
    % \begin{align*}
    %     \Gamma((W \cap V^\lambda) \times \bbD(0,\eta_k^+), I^\ell_{\bbA^1}) &= \Gamma\left(\bigcup_n (\op{Sp}A_n^\lambda \times \bbD(0,\eta_k^+)), I^\ell_{\bbA^1}\right)\\
    %     &= \varprojlim_n \Gamma(\op{Sp}A_n^\lambda \times \bbD(0,\eta_k^+), I^\ell_{\bbA^1})\\
    %     &\cong \varprojlim_n \left\{\sum_{i \ge 0} a_{n,i}t^i \in A_n^\lambda\{t\}_{\eta}\,:\, a_{n,i} \in \Gamma(\op{Sp}A_n^\lambda,I_P^{n-i})\right\}\\
    %     &= \left\{\sum_{i \ge 0}a_it^i \in \varprojlim_n A_n^\lambda\{t\}_\eta \,:\, a_n \in \varprojlim_n \Gamma(\op{Sp}A_n^\lambda, I_P^{n-i})\right\}
    % \end{align*}
    Explicitly, we impose on $\varprojlim_n A_n^\lambda$ the inverse limit topology
    \[
    \varprojlim_n A_n^\lambda\{t\}_\eta = \left\{\sum_{i \ge 0}(a_{n,i})t^i \,:\, |a_{n,i}|_n \eta^i \to 0 \, \forall n  \right\}    
    \]
    where $(a_{n,i}) \in \varprojlim A_n^\lambda$ is a compatible system and $|\cdot|_n$ is the Banach norm on $A_n^\lambda$. 
    This computation allows us to work explicitly with sections of $\Gamma((W \cap V^\lambda) \times \bbD(0,\eta_k^+), I^\ell_{\bbA^1})$ as convergent power series over a ring. 

    Since we assumed that $X_k$ is the complement of a hypersurface in $Y_k$ it follows from \cite[Proposition 5.4.14]{LeStum2007} that the affinoid covering
    \[
    W \times \mathbb{D}(0,1^{-}) = \bigcup_k (W \times \mathbb{D}(0,\eta_k^+))    
    \]
    is acyclic for coherent modules.
    Hence we are in the setting of \cite[Lemma 6.5.10]{LeStum2007}, which tells us that the cohomology 
    \[
    R\Gamma(W \times \mathbb{D}(0,1^{-}), \jdag I_{\mathbb{A}^1}^\ell \xrightarrow{\partial/\partial t} \jdag I_{\mathbb{A}^1}^{\ell - 1})    
    \]
    can be computed as the total complex of the double complex
    \begin{center}
        \begin{tikzcd}
            \prod_k M_k(\ell) \ar[r,"\partial"] \ar[d, "d"]&\prod_k M_k(\ell - 1) \ar[d,"d"]\\
            \prod_k M_k(\ell) \ar[r,"\partial"] &\prod_k M_k(\ell - 1)\\
        \end{tikzcd}
    \end{center}
    where $d(s_k) = (s_{k+1}|_{D(0,\eta_k^+)}  - s_k)$ and $\partial(s_k) = (\partial/\partial t (s_k))$.

    Let $i: M^\lambda(\ell) \hookrightarrow M_k^\lambda(\ell)$ denote the inclusion as the constant coefficient (see Lemma \ref{lpl_explicit}).
    We'll also make use of the integration map 
    \begin{align}
        M_k^\lambda(\ell) &\xrightarrow{I} M_{k-1}^\lambda(\ell + 1)\\
        t^m &\mapsto \frac{t^{m+1}}{m+1}.
    \end{align}
    This is well-defined: $|\frac{a_{i-1}}{m}|\eta^m_{k-1} \to 0$ if $|a_m|\eta_k^m \to 0$ given that $\eta_k \xrightarrow{<} 1$ is strictly increasing, and if $\sum a_mt^m \in M^\lambda_k(\ell)$ then 
    \[
    \sum_{m \ge 1}\frac{a_{m-1}}{m}t^m \in M^\lambda_{k-1}(\ell+1)
    \]
    since $a_{m-1} \in I_P^{\ell+1-m}$ for all $m$.

    The maps $i$ and $I$ are compatible with the transition maps, so taking limits, we can extend $i$ and $I$ to maps $i: M(\ell) \hookrightarrow M_k(\ell)$ and $I: M_k(\ell) \to M_{k-1}(\ell+1)$, respectively, and taking products we obtain maps
    \begin{align}
        i: M(\ell)^{\mathbb{N}} &\hookrightarrow \prod_k M_k(\ell)\\
        I: \prod_k M_k(\ell) &\to \prod_k M_k(\ell+1).
    \end{align}
    Finally, let $\delta: M(\ell) \to M(\ell)^{\mathbb{N}}$ be the diagonal embedding.

    With these notations, the lemma is equivalent to the following claim:
    \begin{claim}
        The sequence 
        \begin{center}
            \footnotesize{
        \[
        0 \to M(\ell) \xrightarrow{i \circ \delta} \prod_k M_k(\ell) \xrightarrow{(d,\partial)} \prod_k M_k(\ell) \oplus \prod_k M_k(\ell-1) \xrightarrow{\partial - d} \prod_k M_k(\ell-1) \to 0
        \]
            }
        \end{center}
        is exact.
    \end{claim}
    \begin{enumerate}
        \item It's immediate that this is a complex and that the first map is injective.
        \item To see that the last map is surjective, note that for all $s = (s_k) \in M_k(\ell-1)$ we have 
        \[
        (s_k)_k \overset{I}{\mapsto} (I(s_{k+1}))_k \overset{\partial}{\mapsto} (s_{k+1})_k.    
        \]
        Hence 
        \[
        (s_k)_k = (s_{k+1})_k - (s_{k+1} - s_k)_k = \partial(I(s_k)) - d(s_k)  
        \]
        that is, $(\partial - d)(I(s), s) = s$.

        \item Next we show exactness in the second term.
        Each $s_k \in M_k(\ell) = \varinjlim_\lambda M_k^\lambda(\ell)$ can be represented as an equivalence class $\overline{(\sum_i a_{\lambda,k,i}t^i)_\lambda} \in \varinjlim_\lambda M_k^\lambda(\ell) = \prod_\lambda M_k^\lambda(\ell)/\sim$ in the usual way.
        It is easy to see that all of our operations are compatible with the equivalence relation, so as an abuse of notation we treat each $s_k$ as a sequence $(\sum_i a_{\lambda,k,i}t^i)_\lambda$.
        % Let $(s_k) \in \prod_k M_k(\ell)$ and suppose that $d(s_k) = \partial(s_k) = 0$.

        Let $(s_k) \in \prod_k M_k(\ell)$ and suppose that $d(s_k) = \partial(s_k) = 0$.
        The fact that $\partial(s_k) = 0$ means that $(\partial/\partial t(\sum_i a_{\lambda,k,i}t^i))_\lambda = 0$ for all $k$. 
        The transition functions are simply restrictions of coefficients of power series so this means, as expected, that all of these power series are constant, i.e. $a_{\lambda,k,i} = 0$ for $i \ge 1$.
        On the other hand, $d(s_k) = 0$ means that, as systems of power series, $s_i = s_j$ for all $i,j \ge 0$.
        Thus 
        \[
            (s_k)_k = (s_0)_k = ((a_{\lambda})_\lambda)_k = (i \circ \delta)((a_\lambda)_\lambda)    
        \]
        as needed.

        \item Finally we show exactness in the third term.
        For brevity, we extend the abuse of notation from the previous step and imagine $(s_k) \in \prod_k M_k(\ell)$ to consist of power series $s_k = \sum a_{k,i}t^i$, with the implicit understanding that, while $s_k$ in equality is represented by a sequence $(\sum_i a_{\lambda,k,i}t^i)_\lambda$, all of our operations are defined componentwise and hence there is no harm in implicitly focusing our calculation on a single component. 

        Let $s = (s_k) = (\sum a_{k,i}t^i) \in \prod_k M_k(\ell)$ and $s' = (s_k') = (\sum b_{k,j}t^j) \in \prod_k M_k(\ell-1)$ with $\partial(s) = d(s')$.
        We're looking for $s'' \in \prod_k M_k(\ell)$ such that $d(s'') = s$ and $\partial(s'') = s'$.
        We have 
        \begin{align*}
            I(s_k')_k &= \left(\sum_{j \ge 1} \frac{b_{j-1, k+1}}{i}t^i\right)_k\\
            \Rightarrow d(I(s_k')) &= \left(\sum_{j \ge 1}\frac{b_{j-1,k+2} - b_{j-1,k+1}}{i}t^i\right)_k\\
            &= I(s_{k+1}' - s_k')_k\\
            &= I(d(s'))\\
            &= I(\partial(s))\\
            &= (s_{k+1} - a_{0,k+1})_k.
        \end{align*} 
        Hence 
        \begin{align*}
            d(I(s') - s + (a_{0,k+1})) &= (s_{k+1} - a_{0,k+1})_k - (s_{k+1} - s_k)_k - (a_{0,k+1})_k\\
            &= s
        \end{align*}
        and 
        \begin{align*}
            \partial (I(s') - s + (a_{0,k+1})_k) &= \partial(I(s') - s)\\
            &= s' + d(s') - \partial(s)\\
            &= s' + d(s') - d(s')\\
            &= s'
        \end{align*}
        (see the computation in (2)), so $s'' = I(s') - s + (a_{0,k+1})_k$ does the trick.
    \end{enumerate}
\end{proof}

{\bf Acknowledgments.} We thank Nicola Mazzari fo some useful suggestions.  Chiarellotto and Nakada were supported by grant MIUR-PRIN2017 ``Geometric, Algebraic, and Analytic Methods in Arithmetic". Nakada was also supported by INdAM grant INdAM-DP-COFUND-2015.

    \bibliographystyle{halpha-abbrv}
    \bibliography{Algebraic_Geometry}
\end{document}