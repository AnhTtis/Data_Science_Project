\section{Conclusion and Future Work}
We propose NeTO, a novel neural rendering based method for transparent object reconstruction, which adopts implicit signed distance function as surface representation and leverage volume rendering to enforce refraction-tracing consistency.
With our proposed self-occlusion checking strategy, the reconstructed geometries of self-occluded parts are further improved.
Our method significantly outperforms the state-of-the-art methods qualitatively and quantitatively by a large margin.

Although our method achieves high-quality reconstruction of transparent objects, the objects should be solid. 
This is because we adopt the ray-location correspondences, which assumes that most of the camera rays only refract on the object surfaces exactly twice.
In the future, we would like to explore how to reconstruct hollow transparent objects, where refraction is more complex and
most of the camera rays will refract on the surfaces more than twice.

	
% \textbf{Limitations.} We have demonstrated that our method can reconstruct accurate and detailed models of transparent objects. However, some limitations remain. This work only considers one position in the background contributing to a single image pixel in the data acquisition. Multiple background positions may contribute to a single image pixel in reality. Besides, it takes much time to train the model of a transparent object. In the future, we would like to speed up the training process and extend our approach to more general objects whose color, refractive index, or transmission properties may not be uniform.