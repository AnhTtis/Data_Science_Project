


\section{EXPERIMENTS}

\subsection{Experimental Setup}
\textbf{Datasets.} We conduct evaluations on the DRT~\cite{lyu2020differentiable} dataset. The dataset contains eight transparent objects. Each transparent object contains 72 views with corresponding masks, ray-pixel correspondences, and extrinsic and intrinsic camera parameters. 
The view resolution is $960\times1280$ or $1080\times1920$.
Ground truth 3D models are also provided for the transparent models. 


\textbf{Implementation Details.} The geometry function $g$ is modeled by an MLP, which consists of 8 hidden layers with a hidden size of 256. We use PyTorch ~\cite{paszke2017automatic} to implement our approach and use the Adam optimizer with a global learning rate $5e^{-4}$ for the network training. Our network architecture and initialization scheme are similar to those of prior works~\cite{wang2021neus,wang2022neuralroom}. We sample $512$ rays per batch and train our model for 300k iterations on a single NVIDIA RTX 2080Ti GPU. We extract explicit mesh from the learned SDF field via a marching cube algorithm~\cite{lorensen1987marching}.


A hierarchical sampling strategy is used to sample points along a ray in a coarse-to-fine manner for volume rendering. We first uniformly sample 64 points along the ray, and then iteratively conduct importance sampling~\cite{wang2021neus} to sample more points on top of coarse probability estimation for $4$ times. The positional encoding is applied to the spatial location with 5 frequencies. 
The hyper-parameters used in the experiments are set as $\omega_1=0.0001, \omega_2=0.1, \omega_3=0.1$. 
Following the prior work~\cite{wu2018full}, the IOR (index of refraction) of air is set to 1.0003 and the IOR of transparent material (glass) is set to 1.4723. 
