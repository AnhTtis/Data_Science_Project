\section{Conclusion and Future Work}
We have presented NeTO, a novel neural rendering-based method for transparent object reconstruction, which adopts implicit signed distance function as surface representation and leverage volume rendering to enforce refraction-tracing consistency.
With our proposed self-occlusion checking strategy, the reconstructed geometries of self-occluded parts are further improved.
Our method significantly outperforms the state-of-the-art methods qualitatively and quantitatively by a large margin.

Although our method achieves high-quality reconstruction of transparent objects, the objects should be solid. 
This is because we adopt the ray-location correspondences, which assumes that most of the camera rays only refract on the object surfaces exactly twice.
In the future, we would like to explore how to reconstruct hollow transparent objects, where refraction is more complex and
most of the camera rays will refract on the surfaces more than twice.

\begin{figure}[t]
% \vspace{-5mm}
\centering
\begin{overpic}
% [width=1.0\linewidth]{figure/ICCV/ablation/vol_surface.pdf}
[width=1.0\linewidth]{figure/ICCV/ablation/surface_vs_volume.pdf}
\put(8.5, -2.5){Surface rendering}
\put(60.5,-2.5){Volume rendering}
% \put(72,-1.5){\small \small DRT\cite{lyu2020differentiable}}
\end{overpic}
% \vspace{-1pt}
\caption{Reconstruction with volume or surface rendering. The F-score values (from left to right) are 0.472 and 0.884, respectively.}
\label{rendering}
% \vspace{-5mm}
\end{figure}
% \vspace{-1mm}
\section{Acknowledgments}
This work is partially supported by NSFC (No.61972298).
% , Bingtuan Science and Technology Program (No.2019BC008), and CAAI-Huawei MindSpore Open Fund.