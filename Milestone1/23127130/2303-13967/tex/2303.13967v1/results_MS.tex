
\section{Implementation of \mech}
%In the previous section, we argued that \mech\ is a demanding and attractive solution concept for the dynamic combinatorial assignment problem. We now turn to the issue of implementation. 
In this section, we introduce an \ocamlong\ (\ocam)    that produces a \mech\ with high probability when the market is sufficiently large and gives agents a strong incentive for truthtelling. To do so, we  define a new equilibrium  concept in which markets clear in expectation, which we call \emph{expected} \ceei\ (\eceei). In Section~\ref{sec:offline} we  formally define the \eceei\   and give a constructive proof for its existence. The construction uses the concept of a \emph{pseudoequilibrium} \citep{milgrom2009substitute} in an economy where agents' utilities depend on the quantity of tokens they have. As the tokens do not affect agents' preferences, a competitive equilibrium need not exist. The key idea in our construction is to approximate the ordinal preference ranking with cardinal utilities that are affected by money when the cost of the bundle is within $\epsilon$ of the budget.
The \ocam\ consists of two phases. In the first phase (which we call the \emph{sample} phase), we allocate bundles to arriving agents using a serial dictatorship. We then compute a set of prices and budgets for the sample agents that support a  competitive equilibrium in which markets clear in expectation. %; we call , which we call \emph{expected} \ceei\ (\eceei). % and prove its existence in Section~\ref{sec:offline}. 
In the second phase, we use the prices and budgets from the first phase   to allocate bundles to  the remaining agents in an online fashion. We describe the mechanism in detail in Section~\ref{sec:online}; in Section~\ref{sec:prop} we describe the properties of the mechanism and state our main result. 

\subsection{\eceei\ }\label{sec:offline}
%In this section, we introduce a new equilibrium concept for static combinatorial assignment problems which will play a key role in the online mechanism. The distinguishing features of this equilibrium concept are that agents' budgets, while still close to 1, will be random and that markets will clear exactly in expectation.
Consider a discrete random variable $\randombudget{}$ which we call a \emph{random budget}.
We say that a random budget $\randombudget{}$ is \emph{$\epsilon$-close-to-1} if all realizations of $\randombudget{}$ are between $1-\epsilon$ and $1$. %\sv{can we have rename this $\epsilon$-good or something like that?}
For any agent $\agent$,  price vector $\prices$ and random budget $\randombudget{\agent}$, define 
\begin{equation}\label{eq:xhat}
\optbundle{\agent}(\prices,\randombudget{\agent}) := \left\{\max_{\succ_{\agent}} \{\gallocv{} \in \choices_{\agent} \text{ and } \prices\cdot \gallocv{} \leq \budget{\agent} \} \text{ with probability } \budget{\agent} \sim \randombudget{\agent}\right\}.
\end{equation}
 We call $\optbundle{\agent}(\prices,\randombudget{\agent})$ a \emph{random optimal bundle} for agent $\agent$. The realizations of $\optbundle{\agent}$ are the   optimal bundles for agent $\agent$ at prices $\prices$ when $\agent$'s budget is drawn from the distribution $\randombudget{\agent}$. 
Denote agent $i$'s \emph{expected optimal bundle} by $\expect{\optbundle{\agent}(\prices,\randombudget{\agent})}$, where the expectation is over $\randombudget{\agent}$.  %In particular, 
%We are now ready to state our equilibrium concept.
Using these definitions, we state our equilibrium concept.

\begin{definition}Fix an economy $(N,\pi,M,\supplyvector{}, (\choices_\agent)_{\agent\in\agentset}, \succ)$. Given an $\epsilon>0$,  random $\epsilon$-close-to-$1$ budgets $(\optrandombudget{1},\ldots,\optrandombudget{\nagents})$,  prices $\optprices = (\optprice{1},\ldots,\optprice{\ngoods})$ and the random  allocation $\optbundlev = (\optbundle{1},\ldots,\optbundle{\nagents})$ comprise an \emph{$\epsilon$-expected competitive equilibrium from equal incomes} ($\epsilon$-\eceei) %is defined by a price vector $\prices$ and a random budget $\randombudget{\agent}$  for each agent $\agent$ such that
if for every good $\good$, 
\begin{enumerate}[label=(\alph*)]
\item $\sum_{\agent \in \agentset} \expect{\optbundle{\agent}(\optprices,\optrandombudget{\agent})}_\good \le \supply{\good}$,
\item if $\price{\good}>0$, then  $\sum_{\agent\in \agentset} \expect{\optbundle{\agent}(\optprices,\optrandombudget{\agent})}_\good= \supply{\good}$.
 \end{enumerate}  
\end{definition}

The definition of the \eceei\ is intuitive: if each agent consumes their expected optimal bundle given their random budgets then the markets for goods clear exactly. Before proving the existence of 
an $\epsilon$-\eceei\ for any $\epsilon >0$, we require some more definitions. 
%Notice that in our construction all random budgets are distributions on finite support.



For each agent $\agent$ and a feasible bundle $\bundle$, let  $r_\agent(\bundle)$ be some \emph{numerical utility}  that is consistent with the preference order $\succ_\agent$. % Since $\emptybundle \in \choices_\agent$ for all $\agent \in \agentset$, we set
We also set  $r_\agent(\emptybundle)=0$ and $r_\agent(\bundle)=-\infty$  for any infeasible bundle $\bundle$.
For any  $\epsilon>0$, define the following  \emph{auxiliary utility} function for each agent $\agent$.\footnote{Thus,
the auxiliary utility of a feasible bundle 
is  $-\infty$ if its cost is at least 1, while for  an infeasible bundle, the utility is always $-\infty$ regardless of price.}

\begin{equation}\label{eq:auxiliary}
  \util^{\epsilon}_\agent(\bundle,\prices)=\begin{cases}
    r_\agent(\bundle)+\min\{0,\log\frac{1-\prices\cdot \bundle}{\epsilon}\} , & \text{if $\prices\cdot \bundle<1$}.\\
    -\infty , & \text{otherwise}.
  \end{cases}
\end{equation}

%\sv{Something feels wrong here: it either holds that $\prices\cdot \bundle<1$ or $\bundle \notin \choices_\agent$? This seems like a weird way to phrase things. Otherwise, we are overloading the word ``feasible''.}

%\tn{not at all, if $\bundle \notin \choices_\agent$, then the utility is always $-\infty$, independent of prices. A bundle can be feasible, but when it is too expensive, it becomes infeasible.}

%\sv{I understand what is happening, I just mean the phrasing is currently a bit ambiguous; I was trying to convey my feeling of what might be understood.}
%\tn{Added a footnote.}

\begin{figure}[tbp]
    \centering
    \includegraphics[width=3.5in]{dceei.pdf}
    \caption{Auxiliary utility  \label{fig:aux}}
      
\end{figure}
The auxiliary utility is illustrated in Figure~\ref{fig:aux}.
Note that the auxiliary utility only differs from 
$r_{\agent}(\bundle)$ when the cost of the bundle $\prices \cdot \bundle$ is at least $1-\epsilon$. 
We now define a pseudoequilibrium  for agents with auxiliary utilities. In an auxiliary economy, ordinal preferences of agents are simply replaced by auxiliary utilities $\util^{\epsilon}_{\agent}(\bundle,\prices)$, and budgets are no longer relevant. For a price vector $\prices$, define $Ch_{\agent}(\prices)=\arg\,\max_\bundle\{\util^{\epsilon}_{\agent}(\bundle,\prices)\}$ and denote by $conv(A)$ the convex hull of set $A$.

\begin{definition} 
Fix an auxiliary economy. The allocation $\optallocs=(\optgallocv{1}, \ldots, \optgallocv{\nagents})$ and prices $\optprices=(\optprice{1}, \ldots, \optprice{\ngoods})$ constitute a \emph{pseudoequilibrium}
      if $\optgallocv{\agent}\in conv(Ch_{\agent}(\prices))$ for all $\agent \in \agentset$
and $\sum_{\agent \in \agentset}\optgallocv{\agent} \leq \supplyvector $ with equality for every good $\good \in \goodset$ with $\price{\good} \neq 0$.
\end{definition}
In words, a pseudoequilibrium is a relaxation of competitive equilibrium in settings with indivisible goods where all agents' preferences are convexified by replacing the choice correspondence $Ch_{\agent}(\prices)$ of each agent $\agent$ with its convex hull $conv(Ch_{\agent}(\prices))$. We are now ready to state our main result for this section.


\begin{theorem}\label{theo:ECEEI}
%A pseudoequilibrium exists with auxiliary utilities $\{\util^{\epsilon}_{\agent}(\bundle,\prices)\}_{\agent \in [n]}$ for  every $\epsilon>0$.
For every  $\epsilon>0$, there exists a pseudoequilibrium  with auxiliary utilities $\{\util^{\epsilon}_{\agent}(\bundle,\prices)\}_{\agent \in \agentset}$.
Furthermore, suppose that allocations $\optallocs$ and prices $\optprices$ form such a pseudoequilibrium. Then one can construct  $\epsilon$-close-to-1 random budgets 
$(\optrandombudget{1},\ldots,\optrandombudget{\nagents})$, such that these budgets, random allocations $\{\optbundle{\agent}(\optprices,\optrandombudget{\agent})\}_{\agent  \in \agentset}$
and prices $\optprices$ form an $\epsilon$-\eceei. Moreover, in the pseudoequilibrium agents of the same type have the same allocation and, as a result, in the constructed $\epsilon$-\eceei, agents of the same type have the same random budget.
\end{theorem}


%he key reason  that a pseudoequilibrium exists in this context is that when prices are too high, agents consume bundle $\emptybundle$. Armed with existence, all that remains to show is that such a pseudoequilibrium corresponds to an \eceei\ with the original ordinal preferences. 

%Theorem~\ref{theo:ECEEI} provides a construction for an \eceei~for any $\epsilon>0$, which we will use to build the online mechanism; its proof is given in Appendix~\ref{app:eceei}.


\subsection{The \ocamlong\  (\OCAM)}\label{sec:online}
In this section, we describe our mechanism; the pseudocode appears in the online appendix. The initial input to the \ocam\ is a set $\goodsset$ of goods  with capacity $\supplyvector{}$ % an algorithm $\A$  for computing an  $\epsb$-\eceei\ (as in Theorem~\ref{theo:ECEEI}),
and three error terms, $\epsilons$ (as in Definition~\ref{def:aceeb}). 
%At a high level, \ocam\ first elicits the preferences of a small sample of agents. These preferences are used to compute an \eceei\ for these agents, which is used to determine an allocation rule that will be used to allocate bundles to future arrivals.   %We  distinguish between the first $\realsample$ agents, which we call the \emph{initial sample}, and the remaining agents. At a high level,  \ocam\ uses the agents in the initial sample to calculate an \eceei\ for these agents and then uses the prices from their  \eceei\ in order to allocate bundles to future arrivals. The main subtlety is setting appropriate budgets for all the agents outside the initial sample.
%Let us now give a more detailed description of \ocam. 
We  set the sample size  to be $\realsample$, where $\epss = \frac{\epsf \epsn}{4}$. Recall that $\epsf$  determines the deviation from exact market-clearing and $\epsn$ is the fraction of agents for whom we do not impose any conditions. In this way, the \ocam\ enables a parameterized tradeoff between these two guarantees. The
\ocam\ is a direct online mechanism: it elicits agents' types and allocates each agent a bundle immediately upon arrival. 
When a sample agent arrives, the \ocam\ allocates the agent a bundle using serial dictatorship with the capacities scaled down proportionally to the initial sample size. That is, the \ocam\ allocates the agent their favorite bundle such that for any good $\good$, the total amount allocated of good $\good$ is at most $\epss \supply{\good}$. The \ocam\ also allocates the agent an arbitrary budget from $[1-\epsb, 1]$ (this budget is never used, but it is technically required for the definition of \mech). 

Once all of the sample agents have arrived, we compute an  $\epsb$-\eceei~ for the sample. An $\epsb$-\eceei\ is defined by random allocations, budgets, and prices, but we only require the  budgets $\randombudget{1}, \ldots, \randombudget{\realsample}$ and prices $\prices$. Note that these are \emph{not} the budgets that the \ocam\ actually assigned to the sample agents; they  will only be used for future assignments. We now define a function that maps agent types to random budgets. For any agent type $\ttype$, if there was an agent $\agent \in \{1, \ldots, \realsample\}$ of type $\ttype$, set $\budgetfunc(\ttype) = \randombudget{\agent}$. In other words, when an agent arrives, if their (declared) type was in the sample, set their budget to be identical to the (perfunctory) budget of an agent of the same type. %Otherwise set $\budgetfunc(\ttype)=1-\budgeteps$.  
Otherwise, set their budget arbitrarily between  $1-\budgeteps$ and $1$.
Define the allocation rule for the \mech\ as follows: for agent $\agent  = \realsample+1, \ldots, \nagents$, draw $\budget{\agent} \sim \budgetfunc(\ttype_\agent)$, where $\ttype_{\agent}$ is agent $\agent$'s type and set $\allocv{\agent} = \max_{\succ_{\agent}}\{\bundle{}: \bundle{} \in \choices_{\agent} \text{ and } \prices\cdot \bundle{} \leq \budget{\agent}\}$ (i.e., allocate the agent their favorite bundle given the drawn budget).
Using this allocation rule, allocate bundles and budgets to agents $\realsample+1, \ldots, \nagents$ upon arrival. %The pseudo-code for this mechanism is given as Mechanism~\ref{alg:one}.



\ignore{
 
\begin{theorem}\label{thm1}
For any $\epsf, \sampleeps,\budgeteps,\ngoods, \nagents, \numtypes>0$, if Assumption~\ref{ass1} holds, there exists an algorithm that outputs an allocation $(\allocv{1}, \ldots, \allocv{\nagents})$, budgets $(\budget{1},\ldots, \budget{\nagents})$ and prices $(\price{1}, \ldots, \price{\ngoods})$ that constitute a dynamically  $(\sampleeps,\budgeteps,\epsf)-$approximate competitive equilibrium with equal budget with probability at least $1- \frac{1}{\nagents}$.
\end{theorem}

}

\subsection{Properties of the \ocam}\label{sec:prop}


In this section, we demonstrate three desirable properties of the \ocam: efficiency, fairness, and strategyproofness. The first two properties stem from the fact that the \ocam\ implements \mech\ with high probability (Theorem~\ref{thm1}).

There is a variety of existing definitions of strategyproofness for random mechanisms that use only ordinal preference information. One such definition demands that truthtelling be a dominant strategy for every realized outcome of the mechanism (we refer to it as `ex post' strategyproofness). In our setting, the only efficient mechanisms that satisfy ex post strategyproofness are serial dictatorships \citep{klaus2002strategy}. Weaker (or `interim') notions of strategyproofness only require non-manipulability whenever agents' preferences over lotteries are represented by von Neumann-Morgenstern utility functions or by the stochastic dominance partial order \citep{bogomolnaia2001new}.\footnote{To see that ex post strategyproofness is a stronger non-manipulability condition, note that it requires that truthtelling be a dominant strategy for \emph{any} utility function representation of preferences over lotteries while interim strategyproofness notions restricts which functions can represent preferences.}

%Prior work has offered several notions of strategyproofness for random mechanisms which use only ordinal preference information. For example, one might demand that truthtelling is a dominant strategy  for \emph{every} realized outcome  of random mechanisms (`ex post' strategyproofness); in this case the only efficient mechanism is a randomized serial dictatorship in our setting. 
%Alternatively, one could merely require non-manipulability whenever agents preferences are represented by von Neumann-Morgenstern utility functions or by the stochastic dominance partial order \citep{bogomolnaia2001new} (`interim' strategyproofness).


Here, we use a notion of ex post strategyproofness and require that an agent cannot significantly improve \emph{any} realization of the random allocation by misrepresenting their type. We avoid the incompatibility of ex post strategyproofness with Pareto efficiency and non-dictatorship by only requiring that manipulations be sufficiently profitable and efficiency be approximate. Furthermore, %because our mechanism allocates bundles according to prices computed during the sample phase, 
our mechanism disincentivizes any {group}  of  agents from misrepresenting their type.

%\at{are we talking about the unseen agents who end up with a ``low'' budget?}
%, 


We will need some additional notation. Given an economy and a random mechanism $\mechanism$, let $\mechanism_i(\succ)$ denote the mapping of the agents' preference profile to agent $i$'s random allocation, and let $R(\mechanism_i(\succ))$ denote the set of all bundles with positive probability in the random allocation $\mechanism_i(\succ)$. Using this notation, we can now define
 group-strategyproofness up to one object.

\begin{definition}\label{defn:gsp}
A mechanism $\mechanism$ is (weakly) \emph{group-strategyproof up to one object}   if for any group of agents $\gsset$ and any two preference profiles of these agents $\succ_\gsset$ and  $\succ'_\gsset$ at least one of the following conditions is true.
\begin{enumerate}[label=(\roman*)]
    \item there exists an agent $\agent \in G$
such that
for any $\gallocv{\agent}\in R(\mechanism_i(\succ_\gsset ,\succ_{-\gsset}))$  and  $\gallocv{\agent}'\in R(\mechanism_i(\succ'_\gsset ,\succ_{-\gsset}))$, it holds that $\gallocv{\agent} \succeq_\agent \gallocv{\agent}'$,
\item for all $\agent \in G$,  $\gallocv{\agent}\in R(\mechanism_i(\succ_\gsset ,\succ_{-\gsset}))$, and   $\gallocv{\agent}'\in R(\mechanism_i(\succ'_\gsset ,\succ_{-\gsset}))$, then there exists  $\good \in \goodsset$ such that  $\gallocv{\agent}\succeq_\agent (\gallocv{\agent}'-\mathbf{e}^\good)^+$.
\end{enumerate}
\end{definition}

Definition~\ref{defn:gsp} says that if a group of agents $G$ misreports their types, then either there exists an agent  $\agent \in G$ that is weakly worse off, or for \emph{every agent} $\agent \in G$, any realized allocation that agent  $i$ obtains by misreporting their  type can be made weakly worse than any realized allocation under truthtelling by removing some object.

%\sv{in case I missed any pronouns, let's use only `they'.}
%It is worth emphasizing that under Definition~\ref{defn:gsp}, every agent (as opposed to just a single agent) of the misreporting coalition must be significantly (i.e., by more than the value of any individual object) better off than under truthtelling.

%Maybe this???

%It is worth emphasizing that under Definition~\ref{defn:gsp}, the second condition requires that under truthtelling every agent (as opposed to just a single agent) are not significantly worse off (i.e., by more than the value of any individual object) than by misreporting.







%there is an agent in the group, such that any realized allocation he obtains by misreporting their  type can be made worse than any realized allocation he obtains under truthtelling by removing some object.

%




Next, we introduce two assumptions---on the arrival order and on market size---that allow us to show the existence of a mechanism that implements a DACEEI.
First, we require that the agents arrive at random.
\begin{assumption}\label{ass0}
    Agents arrive according to  permutation selected uniformly at random  from the set of  all permutations on $[n]$.
\end{assumption}
Note that we make no assumption on the distribution of types; our results hold for  `arbitrarily bad'  distributions.  
Second, we require that the capacities of the goods be sufficiently large.
\begin{assumption}\label{ass1}
%$\min_{\good} \{\supply{\good}\} \geq \max\left\{ \frac{3 ( \log{(\ngoods/\epsf)} + \numtypes \log{(\sampleeps \nagents)})}{\epsf^2 \sampleeps},  \frac{ \sqrt{\left(\nagents/\log{\epsf}\right)} }{2\epsf^2 \sampleeps} \right\}$.
$\min_{\good} \{\supply{\good}\} \geq  \dfrac{70\left(\numtypes\log{(\realsample)} + \log{(\ngoods)}+\sqrt{\nagents}\log{\nagents}\right)}{\epss \epsf^2}$, where $\epss = \frac{\epsn\epsf}{4}$.
\end{assumption}

We do not attempt to optimize the constant in Assumption~\ref{ass1}, preferring clarity in the proofs to a better constant. The $\sqrt{\nagents}$ term in Assumption~\ref{ass1} is only required to ensure that the allocation is approximately fair and efficient along the entire sequence of agents' arrival. If one is willing to relax this assumption and only require that these desiderata hold for the final allocation, we can discard the $\sqrt{\nagents}$ term (alternatively, if the bound on $\supplyvector$ holds only for the other terms in the numerator, the result still holds for the final allocation). For comparison, the capacity assumption of~\citet{AgrawalOnline} %and~\citet{MolinaroOnline} are 
is $\min_{\good} \{\supply{\good}\} \geq  \Omega\left(\frac{\ngoods \log{(\nagents/\epsilon)}}{\epsilon^3}\right)$.
When  $\numtypes$ is sufficiently small, as is the case for all of the applications described in the introduction, our bound is only worse than theirs by a factor of $\epsilon$. We also note that the assumption is independent of $\epsb$, hence we can make $\epsb$ arbitrarily small. 
 Our main result on  the properties of the \ocam\ is the following.
\begin{theorem}\label{thm1}
{For any economy $(N,\pi,M,\supplyvector{}, (\choices_\agent)_{\agent\in\agentset}, \succ)$,  $\epsf, \sampleeps>0$, and $0<\budgeteps< \frac{1}{m}$,} the \ocam\,~is: 
   \begin{enumerate}[label=(\alph*)]
   \item group-strategyproof up to one object,\label{thm1a}
   \item envy-free up to one object for a $(1-\epss)$ fraction of the agents, where $\epss = \frac{\epsn\epsf}{4}$. \label{thm1b}
   \end{enumerate}
Furthermore,   when   Assumptions~\ref{ass0} and \ref{ass1} hold, the \ocam\ outputs a tuple $(\optallocs,\optbudgets,\optprices)$   %allocation $(\allocv{1}, \ldots, \allocv{\nagents})$, budgets $(\budget{1},\ldots, \budget{\nagents})$ and prices $(\price{1}, \ldots, \price{\ngoods})$ 
that constitutes an $\daisyeps-$\mech\ with probability at least $1- \frac{1}{\nagents}$. 
\end{theorem}

Note that neither the fairness nor the strategic properties depend on the order of arrival or on market size. Indeed, Assumptions~\ref{ass0} and~\ref{ass1} are only needed to prove that approximate market-clearing  holds in almost every period as required by the \mech.

\ignore{
\proof{Proof outline of.  Theorem~\ref{thm1}} 




First, we show that our mechanism is group-strategyproof up to one object. This is true because, for the first $\realsample$ agents (the sample), the mechanism is serial dictatorship and group-strategyproof. Thus, if the `non-truthful' group of agents contains these agents, at least one of them will be weakly worse off by misreporting their types. 

Now, consider a `non-truthful' group that consists of only agents arriving after the sample. Notice any agent arriving after $\realsample$ picks the best bundle according to their budget. If the agent misreports, they might get a  larger budget. Because changes in one agent's preferences do not affect prices nor allocation of any other agents, any improvement the agent achieves would come from the gain in the budget, which is at most $\epsb<\frac{1}{m}$. In Appendix~\ref{app:envy}, we show that this implies that the improvement is bounded by at most one object.
Thus, overall our mechanism \ocam\ is group-strategyproof up to one object for all agents.

Next, we give an informal outline of the proof that the allocation is almost market-clearing at every step. The complete formal proof is given in Appendix~\ref{app:proofthm}. We first prove that with high probability (where the sample space is the permutations of the agents), the randomized allocation that is generated by Algorithm $A$ is almost market-clearing in expectation: it does not violate the capacity of any good, and any good $\good$ whose price is not zero is almost completely allocated. This allocation includes the hypothetical random bundles that the algorithm would allocate to the sample, but not the real bundles allocated to the sample. We will address this discrepancy shortly. %This is shown in Lemma~\ref{lemma:rnd} in Appendix~\ref{app:proofthm}.
We then show the total realized allocation is close to the expected allocations; that is, once all agents have been allocated a bundle, the total amount allocated of each good is close to its expectation. Once again, this holds with high probability, but this time, the sample space is the realized budgets (with respect to the random budgets $\randombudget{1}, \ldots, \randombudget{\nagents}$).    Finally, we show that if the total realized allocation is approximately market-clearing, then all of the intermediary allocations are approximately market-clearing as well, with high probability. Once again, the sample space is the permutations of the agents.   % The last two steps are proven in Lemma~\ref{lemma:det} in Appendix~\ref{app:proofthm}. 
Finally, we adjust our bounds to accommodate the sample agents, whose true allocations do not necessarily match those generated by Algorithm $A$.

\proofen



}


