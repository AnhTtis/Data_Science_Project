

\ignore{
Daniela Saban\\
Sid Banerjee\\
Daniel Freund\\
Eduardo Azevedo\\
Eric Budish\\
}




\section{Introduction}



In many complex allocation  problems, such as daycare assignments, refugee resettlement and airport slot allocation, the market designer needs to assign combinations of objects to agents, but cannot use money. In such \emph{combinatorial assignment} settings, the only efficient and strategyproof mechanisms are dictatorships which can be grossly unfair (e.g., \citealt{klaus2002strategy}). One approach that circumvents these stark tradeoffs is to use \emph{pseudomarkets}: agents are allocated (equal) budgets of tokens and can use them to trade the resources. An allocation in a \emph{competitive equilibrium from equal incomes} (CEEI) in this setting is efficient and eliminates envy \citep{varian1974equity}. Moreover, in large markets truthfully revealing preferences becomes a dominant strategy in a mechanism that implements a CEEI \citep{azevedo2019strategy}. 

Unfortunately, when resources are indivisible and/or highly complementary, a CEEI might not necessarily exist \citep{hylland1979efficient}.
A seminal paper by \citet{budish2011combinatorial} offered an alternative to the (exact) CEEI called \emph{approximate} CEEI (\aceei). In an \aceei, agents' budgets are slightly perturbed and the market for each resource clears only approximately. The approximation bound on the excess demand reflects the number of objects that have been left unallocated or need to be added in order to establish an exact CEEI. Thus, an \aceei\ remains nearly envy-free and efficient when the excess demand is small relative to the total supply. 


This paper considers pseudomarkets in the context of \emph{dynamic} (online)  
combinatorial assignment problems without monetary transfers. %\footnote{Similar problems include multi-flight revenue management for customers arriving online but in that setting, preferences are not ordinal and money can be used.} 
The following  examples illustrate our setting.

\begin{enumerate}
    \item Refugee resettlement~\citep[e.g.,][]{ahani2021dynamic}: refugees arrive  over the course of some time. As soon as a refugee family arrives, it needs to be matched to a location. Refugees have preferences over different possible locations for initial resettlement. Refugee families are of different sizes and might require a different amount of services offered at different locations (e.g., school places). %Locations prefer to fill their capacities gradually. % over the course of the year.
    \item Daycare allocation~\citep[e.g.,][]{kennes2014day}: 
    families with children demand childcare. %As soon as the parents/carers go back to work, the children need to be matched to a daycare center. 
    Families have preferences over different daycare centers (or over combinations of daycare centers and days of the week). %Families have different numbers of children that potentially need to be placed in daycare on different days.
    Daycare centers cannot easily add extra staff or rooms so their capacity constraints need to be respected throughout the year.
    
    %\item Online professional course allocation: students need to take courses throughout the year. As soon as a student joins an online platform, they are assigned to a number of courses. Students have preferences over the different course combinations. Even though students can join different courses, courses are capacity constrained at any point in time (e.g., by the number of TAs).
    
    \item Assignment of airport take-off and landing slots~\citep[e.g.,][]{schummer2013assignment}: 
    airlines demand landing slots, gates and taking-off slots for  their flights. %\footnote{There have also been proposals to auction off slots using combinatorial auctions \citep{rassenti1982combinatorial}. However, these auctions are for longer-term property rights at airports rather than a minute-by-minute assignment of slots.}
    The assignment frequently needs to be negotiated hour-by-hour, often when aircraft are still in the air, because of varying weather conditions. Airlines might prefer to have take-off and landing slots that are close to each other to avoid idling the aircraft at the gate. Airports have strict capacities at different gates and for the number of aircraft that can land  or take-off from a runway in any time period which must be respected for safety reasons.
\end{enumerate}

%\at{I'm not sure if this the next two paragraphs are the best motivation. One thing says there is an additional challenge of preference elicitation. The second challenge was already mentioned above and it's really a problem with a possible solution. Maybe if we swap them and say we could use ACCEI repeatedly, but that's a bad idea, plus there is an issue with having the preference data? Let me know if you like that proposal and I'll implement.} 

There are two significant hurdles that need to be overcome in order to extend the \aceei\ to dynamic settings. The first  involves making irreversible allocation decisions as agents arrive. As a result, naive repeated use of ACEEI becomes increasingly inefficient because the bound on excess demand scales up linearly with each iteration, making the inefficiency directly proportional to the number of agents in online settings. The second hurdle is the lack of available preference data before agents arrive, which requires the development of an online allocation mechanism that can function without any prior assumptions about the preference distribution.  

%, making it crucial to devise alternative approaches to overcome these challenge.}



 

\ignore{
arises when dealing with indivisible goods, as CEEI may not exist. One possible solution could be to repeatedly use \aceei. However, this approach may lead to poor results, since the bound on excess demand scales up at the rate of the number of times \aceei\ is used, which can increase linearly with the number of agents in online settings. 


To address this, we utilize the  random arrival framework \citep[e.g.,][]{Freeman83}, in which agents arrive in a random order  from an unknown distribution. An efficient mechanism in this model will need to learn preferences over time while allocating resources to them. 
}








We propose a pseudomarket mechanism for dynamic combinatorial assignment problems such as those described above.
Despite the above challenges, the allocation produced by our mechanism satisfies a novel and strong solution concept called \emph{dynamic} \aceei\ (\mech)\footnote{Pronounced `Daisy' for convenience.}, in which markets clear approximately \emph{in every period} (i.e., whenever a new agent or a batch of agents appear), except for a small number of initial periods.  We show that the \mech\ inherits a number of desirable properties of its static counterpart: it is envy-free up to one object (Proposition~\ref{prop:EF}) and ex-post Pareto efficient with respect to realized supply  (Proposition~\ref{prop:PE}). We then describe an online mechanism that implements a \mech\ with high probability while giving almost all agents a limited incentive to misreport their preferences. Importantly, we leverage the random arrival framework \citep[e.g.,][]{Freeman83}, which allows agents to arrive randomly from an \emph{unknown} distribution. This framework necessitates our mechanism to learn preferences over time while effectively allocating resources. We obtain a efficient dynamic combinatorial assignment mechanism in two key steps.

First, in order to control the compounding error in excess demand that would arise from the repeated use of \aceei, we introduce a  new equilibrium concept called \emph{expected} CEEI (\eceei) for static combinatorial assignment settings. 
Given any $\epsilon>0$, an \eceei\  is characterized by  equilibrium prices and distributions over perturbed agent incomes  such that the agents' average consumption at the equilibrium prices satisfies the \emph{exact} market-clearing condition, and the incomes are all in the interval $[1-\epsilon, 1]$. Theorem~\ref{theo:ECEEI} shows that an \eceei\ exists for any $\epsilon>0$. 
%The construction is as follows. First, we approximate agents' ordinal preferences by cardinal preferences that depend on token holdings. These preferences have the following structure: when expenditure on resources is low, the utility is constant in tokens, but as the expenditure approaches the budget, the utility falls to negative infinity. Second, adapting the arguments from \citet{nguyen2021delta}, we use the cardinal utility functions to prove the existence of a \emph{pseudoequilibrium}  \citep{milgrom2009substitute}, in which agents' choices correspondences are convexified by replacing them with their convex hulls.\footnote{Pseudoequilibrium has origins in concepts such as `convex preference equilibrium' \citep{starr1969quasi}.} Finally, we show that the pseudoequilibrium corresponds to an \eceei\ with perturbed budgets. We note that the existence of \eceei\ implies the existence of \aceei\ with its usual bounds and that the \eceei\ retains the desirable properties of \aceei\ while allowing for exact market clearing in expectation. 

Second, we show how to adapt  \eceei\ to dynamic settings to ensure that markets clear approximately in every period, after an initial sample. Specifically, our \ocamlong\ (\ocam) %computes an \eceei\ for a subset of the agents who arrive in the initial periods and uses the equilibrium prices to allocate objects to the remaining agents. 
%\sv{\ocam makes no assumption on the type distribution of the agents but assumes that they arrive in random order.} \tn{\ocam\ is a mechanism, it does not need any assumption. I think put that assumption to the property of the mechanism. } 
%\ocam\ 
has two phases:
\begin{enumerate}
\item For an initial set of arriving agents (which we refer to as the \emph{sample}), allocate the objects according to a (random) serial dictatorship (SD). %Since the serial dictatorship is strategyproof,
We use the  reported preferences of these agents to calculate an \eceei.  
\item For the remaining agents, allocate each agent their favorite bundle using prices from the sample \eceei\ conditional on the \emph{random} budget determined by their type (i.e., their preference ordering over objects). 
\end{enumerate}


Theorem~\ref{thm1} shows that the \ocam \ is  group-strategyproof up to one object,
envy-free up to one object for all agents outside the sample,  efficient (i.e., market-clearing) with high probability when capacities are sufficiently large and agents arrive in random order, and
 uses up the capacity of each good at a constant rate.
The high-level idea behind the proof is the following: the SD guarantees that the reports of the agents are truthful, and the budgets of the agents outside the  sample are perturbed based on their reported type to maintain incentives for truthtelling.
%\sv{should we leave this in? It's pushing the truth, but gives intuition about the truthtelling, which is valuable I think} \tn{ YES leave it}. 
The \eceei\ ensures that market-clearing conditions (for the sample) are not violated in expectation. Suitable  concentration inequalities are then used to extend the guarantees on the expectation for the sample to bounds on the deviations of the realized allocations in almost every period. 


%While the envy-free and strategyproof properties of our mechanism do not depend on the assumptions about market size or random order assumption,the efficiency of the mechanism relies on both of these assumptions.\footnote{These assumptions allow us to show that the initial sample of agents is a good presentation of the overall population. 
%In an extreme case, suppose that the number of types is equal to the number of agents. Then, the \eceei\ calculated from the initial sample gives no guidance about how to allocate efficiently to agents outside the initial sample.} 


