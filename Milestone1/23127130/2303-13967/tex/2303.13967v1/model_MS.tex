\section{Model}
There is a finite set  $\goodsset$ of \emph{goods}, $|\goodsset|=\ngoods$  and a finite set $\agentset$ of agents, $|\agentset| = \nagents$. Each good $\good$ has a finite integer capacity $\supply{\good}\in \mathbb{N}$. %Occasionally, we refer to a unit of a good as an \emph{object}.
 There is a permutation $\pi: \agentset \rightarrow \agentset$ on the agents; the permutation defines the order of arrival of the agents. Without loss of generality, agents are indexed by their order of arrival. We allow agents to arrive in batches or one by one. %We also use index $k$ for agents.
 
 A \emph{bundle} $\bundle \in \{0,1\}^m$ contains at most one unit of each good.\footnote{As \citet{budish2011combinatorial} points out, this is without loss of generality as identical units can be simply relabelled as different objects.} We use $(\x-\e^j)^+$  to denote the  bundle $\x$ with %the unit of 
 object $\good$ removed. % from $\x$. 
The bundle consumed by agent $\agent$ is denoted by~$\bundle_{\agent}$. There might be further constraints on consumption (for example, taking off and landing slots that are too close to one another). Let $\choices_\agent \subseteq \{0,1\}^m $ denote the set of \emph{feasible} bundles for agent $\agent$. We assume free disposal, i.e., that $\ooo\in \choices_\agent$ for all $\agent$.
An \emph{allocation} $\allocs=(\gallocv{1}, \ldots, \gallocv{\nagents})$ is a list of feasible bundles, one for each agent. Allocation $\allocs$ is \emph{feasible with respect to capacities} $\supplyvector$ if for all $\good \in \goodset$, we have that $\sum_{\agent \in \agentset} \alloc{\good}{\agent} \leq \supply{\good}$.  



Each agent $\agent$ has a strict preference relation $\succ_\agent$ over the set $\choices_\agent$. % of feasible bundles. 
We assume that $\ooo$ is the least preferred bundle for all agents. 
We denote the weak relation of $\succ_\agent$ by $\succeq_\agent$; i.e.,  $\x\succeq_\agent \y$ means  either $\x\succ_\agent \y$ or $\x=\y$.
Denote the preference profile of  all agents by $\succ:= (\succ_\agent)_{\agent \in \agentset}$.  The preference profile of a group of agents $\gsset$ is denoted by $\succ_\gsset := (\succ_\agent)_{\agent \in \gsset}$, and the preference profile of all agents outside $\gsset$ is denoted by $\succ_{-\gsset} := (\succ_\agent)_{\agent \notin \gsset}$. 
%\at{Do we ever use  $\succ_{-i}$ because we might want to flag slight abuse of notation}.
The \emph{type}  of agent $\agent$ is their preference relation $\succ_\agent$; we will sometimes denote the type of agent $\agent$ by $\ttype_{\agent}$ for notational clarity. Since the sets of feasible bundles and preference orderings are finite, the set of types is finite. Let $\typeset$ denote the set of agents' types (i.e., set of all strict preference orderings over bundles), and let $\numtypes = |\typeset|$ denote the number of agent types.

%\sv{The following definition was wrong, and I rewrote it. Please check} 
The \emph{economy} is a tuple $(N,\pi,M,\supplyvector{}, (\choices_\agent)_{\agent\in\agentset}, \succ)$. A (direct) \emph{mechanism} $\mechanism$ maps every preference profile  to a (random) allocation. %Consider two permutations $\pi$ and $\pi'$ which are identical for agents $1, \ldots, k$. An \emph{online mechanism} is a mechanism that for any two economies  that are identical except for their permutations, $\pi$ and $\pi'$, produces the same (random) allocation for agents $1, \ldots, k$ in both economies.
Consider two economies $E$ and $E'$ that differ only in their permutations, denoted $\pi$ and $\pi'$ respectively, where $\pi$ and $\pi'$ are identical for agents $1, \ldots, k$. An \emph{online mechanism} is a mechanism that produces the same (random) allocation for agents $1, \ldots, k$ in both economies.




\section{\mech}

In a pseudomarket, agents are endowed with a \emph{budget} of tokens. They can spend these tokens to `buy' bundles at prevailing prices for the goods. Let  $\budgets = (\budget{1}, \ldots, \budget{\nagents})$ denote the vector of token budgets of the agents. In a \ceei, agents are allocated equal budgets of tokens \citep{varian1974equity}. However, in our setting, a \ceei\ might not exist.
We will therefore follow the spirit of \aceei\ introduced by \citet{budish2011combinatorial} and allow a slight perturbation of budgets as well as only approximate market-clearing in order to guarantee equilibrium existence.
%Our solution concept for dynamic pseudomarkets is the following.

\begin{definition}\label{def:aceeb} Fix an economy $(N,\pi,M,\supplyvector{}, (\choices_\agent)_{\agent\in\agentset}, \succ)$. The allocation $\optallocs=(\optgallocv{1}, \ldots, \optgallocv{\nagents})$, budgets $\optbudgets=(\optbudget{1},\ldots, \optbudget{\nagents})$ and prices $\optprices=(\optprice{1}, \ldots, \optprice{\ngoods})$ constitute an $\daisyeps-$\emph{dynamic approximate competitive equilibrium from equal incomes} (\mech)  if the following hold:
\begin{enumerate}[label=(\roman*)]
\item for all $\agentdef \in [\epsnn,\nagents]$, $\optgallocv{\agentdef} = \max_{\succ_\agentdef} \{\gallocv{\agentdef} \in \choices_\agentdef \text{ and } \optprices\cdot \gallocv{\agentdef} \leq \optbudget{\agentdef}\}$,\label{ACEEI1} 
\item  for all $\agentdef \in [\epsnn,\nagents]$, $1-\budgeteps \leq \optbudget{\agentdef} \leq 1$,\label{ACEEI2} 
\item for all $\agentdef \in [\epsnn,\nagents]$, $\good \in \goodset$,  $\sum_{\iter = 1}^{\agentdef} \optgalloc{\good}{\iter} \leq (1+\approxeps)\dfrac{\knum\supply{\good}}{\nagents}$, \label{ACEEI3}
\item for all $\agentdef \in [\epsnn,\nagents]$, $\good \in \goodset$,  if $\price{\good}>0$, then $\sum_{\iter = 1}^{\agentdef} \optgalloc{\good}{\iter}  \geq (1-\approxeps)\dfrac{\knum\supply{\good}}{\nagents} $.\label{ACEEI4}
\end{enumerate}
\end{definition}

\mech\ is parameterized by three error terms. The first error term, $\budgeteps$, determines the worst deviation from the equal budget. The second, $\sampleeps$, determines the fraction of agents for whom we do not impose any optimality, budget perturbation, or market-clearing conditions. Finally, $\approxeps$ controls the deviations from exact market-clearing. 
%The definition of \mech\ is rather demanding. 
The first two parts in the definition of \mech\ are similar to \aceei: Part~\ref{ACEEI1} says that agents are allocated their most preferred bundle at the prevailing prices given their perturbed budgets; Part~\ref{ACEEI2} says that the agents' budget might be relaxed by at most $\budgeteps$.\footnote{This condition is slightly different from Definition 1.iii. in \citet{budish2011combinatorial} which bounds the increases in budgets. None of the results would be affected if we used his definition; our definition allows for a somewhat simpler exposition.} The final two parts of the definition extend \aceei~to the dynamic setting. Parts~\ref{ACEEI3} and~\ref{ACEEI4} say the markets for all objects clearly approximately \emph{in every period} except (possibly) for the first $\epsnn$ time periods. 
Note that 
$\frac{\agentdef\supply{\good}}{\nagents}$
denotes the proportional scaling of the capacity for each object when a $\frac{\agentdef}{\nagents}$--fraction of the agents has arrived. Therefore, the capacity of all goods is being used up at the rate of agent arrival which ensures that agents arriving earlier and later are treated fairly.

\subsection{Properties of \mech}
We first prove several results that highlight the fairness and efficiency properties of \mech.
%\mech\ inherits that it inherits from \aceei. Both properties only hold for agents outside the initial sample since \mech\ does not specify anything about the allocation of the agents in the initial sample. 
%The first property concerns fairness. 
While an allocation in  a \mech\ (or indeed in an \aceei)  might not be envy-free, we can show that whenever agent $i$ envies agent $i'$, it is possible to remove one object from $i'$'s bundle in such a way that $i$ no longer envies $i'$. 
\begin{definition} (\citealp{budish2011combinatorial}).
An allocation $\allocs$ is \emph{envy-free up to one object} (EF1) for a set of agents $\gsset$  if, for any $\agent, \agent' \in \gsset$, either  (i) $\gallocv{\agent}\succeq_\agent \gallocv{\agent'}$  or (ii) there exists some object $\good \in \goodsset$ such that $\gallocv{\agent}\succeq_\agent (\gallocv{\agent'}-\e^j)^+$.


\end{definition}

\begin{proposition}\label{prop:EF}
If $(\optallocs,\optbudgets,\optprices)$ is an $\daisyeps-$\mech\ with $\budgeteps< \frac{1}{m}$, then $\optallocs$ is envy-free  up to one object for the agents in $[\epsn \nagents,\nagents]$.
\end{proposition}
The proof of Proposition~\ref{prop:EF}  closely follows Theorem~3 in \citet{budish2011combinatorial} while adjusting for small differences in our equilibrium definitions.


 Our efficiency result is a version of the First Fundamental Theorem of Welfare which states that competitive equilibrium allocations are Pareto-efficient. Since capacities can be perturbed in our model, we first introduce a definition of Pareto efficiency that specifies the capacities of goods as well as the agents.



\begin{definition}
An allocation $\allocs$ that is feasible with respect to capacities $\supplyvector$ is (ex-post) \emph{Pareto-efficient for agents in $G$} if there is no other allocation which is feasible with respect to  $\supplyvector$ and is weakly preferred by all agents in $G$, with at least one strict preference.
\end{definition}

%\sout{We consider the realized consumption of the goods once all the agents have arrived. We then perturb the capacities to match the realized consumption  exactly and look at the efficiency of the allocations constrained by these capacities for agents outside the initial sample\sv{in  $[\epsn \nagents,\nagents]$}.} 
We consider the realized consumption of the goods once all the agents have arrived. We then perturb the capacities to match the realized consumption of the agents $\epsn \nagents,\ldots, \nagents$  and look at the efficiency of the allocations with respect to  these capacities.
\begin{proposition}\label{prop:PE}
Let $(\optallocs,\optbudgets,\optprices)$  be an $\daisyeps-$\mech\ of the economy and let $\optsupply{j}:=\sum_{\iter = \epsn \nagents}^{\nagents} \optgalloc{\good}{\iter}$  for all $\good \in \goodset$.
Then the allocation $\optallocs$ which is feasible with respect to capacities $\optsupplyvector$ is Pareto-efficient for agents in $[\epsn \nagents,\nagents]$.    
\end{proposition}

The proof of Proposition~\ref{prop:PE} is the standard revealed preference argument used for the First Welfare Theorem and is omitted.

