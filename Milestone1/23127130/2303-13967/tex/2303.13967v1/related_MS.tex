

\subsection{Related literature}

%\sv{Still need to add the citations we talked about. Should we also add a short comparison to papers on  refugees are mention that a huge advantage of our approach is that we don't need a prior?}




Our paper is related to several strands of the literature. 
The first strand is the analysis of equilibrium in static pseudomarkets.
\citet{varian1974equity} offered a striking analysis of the fairness and efficiency properties of the CEEI in a divisible object setting.
\cite{hylland1979efficient} first proposed using pseudomarkets (via trading probability shares) for the allocation of indivisible objects. 
\citet{budish2011combinatorial} dealt with the non-existence of equilibrium in pseudomarkets for combinatorial assignment by introducing the \aceei. Our solution concept, the \mech\, directly extends Budish's \aceei\ to the dynamic setting. 
Our paper is also related to \cite{nguyen2021delta}, which shows the existence of a pseudoequilibrium in an economy with money. 
In this paper, we take our results one step further by demonstrating how to leverage a pseudoequilibrium to construct an \eceei\ using a lottery over random budgets. This approach is a crucial element in extending the combinatorial assignment to dynamic settings.

%explicitly allows for pseudomarkets (utility must change with money), we can map ordinal preferences in our pseudomarket to cardinal preferences in the model. Most of their results, therefore, become crucial in the proof of the existence of \eceei.

Second, our paper is related to dynamic pseudomarkets and dynamic fairness problems. Several papers~\citep[e.g.][]{balseiro2019multiagent,gorokh2021monetary} analyze scrip systems; \cite{combe2021dynamic} considers dynamic assignment via spot mechanisms. The key difference in this paper is that we consider fairness and incentive compatibility for each new agent that arrives in every period, whereas the previous models focused on objects arriving over time and long-lived agents.

Another stream of papers  studies the trade-off between efficiency and fairness in the dynamic setting \citep[e.g.][]{Zeng2020,manshadi2021fair}. These papers typically focus on fair division of either divisible or indivisible goods, where the agents have cardinal valuations for the goods, in contrast to our setting where the agents' preferences over bundles of objects are the key feature in our mechanism. 



Finally, from a methodological perspective, our paper builds on recent literature on algorithms for online linear programming. Specifically, our paper is most related to \citet{AgrawalOnline} and \citet{MolinaroOnline}, which study a related dynamic setting where agents are associated with the variables of a packing linear program. %the value of each variable has to be irrevocably set on arrival.  
They sample a constant fraction of the agents, solve a dual problem constructed from this sample, and use these prices as the basis of a dynamic binary allocation rule; they use the geometric structure of the solution space to bound the number of possible distinct solutions.
Our setting differs from theirs in several aspects: our input is ordinal, while theirs is cardinal, and their decisions are binary---to allocate or not---while we must allocate a bundle to every agent. Further, our allocations are randomized; our proofs make use of the two different dependence structures of our randomness sources. 
 %\citet{VardiOnline} extend the result of \citet{AgrawalOnline} to a convex setting; their bound is similar to that of \citet{AgrawalOnline}, and while their allocation rule is not binary, its range is also quantized, which again allows them to use a geometric property of the solution space to bound the number of possible solutions. In contrast, our solution space is much larger, and we instead bound the number of possible (randomized) solutions with the number of possible agent types. %Similarly to~\citep{VardiOnline}, and unlike~\citep{AgrawalOnline,MolinaroOnline}, we use two distinct parameters for the error bound: one defining the sample size
More recently, \citet{Vera} study a related model in which each agent is single-minded, and the decision-maker has to decide whether or not to allocate the bundle of interest to the agent.

%\tn{maybe we want to say our result for all periods, which is different from the literature...}
%In their setting, unlike in our setting, the decision maker knows the type distribution a-priori. 
%We note that similarly to our paper, \citet{AgrawalOnline} and \citet{MolinaroOnline}  make no assumption on the distribution of the columns of the linear program (analogous to the agents' preferences in our model), but only that the arrival order is random. %Stronger bounds can be proven when the columns are drawn i.i.d. from some distributions~\citep[e.g.,][]{Nikhil}. 
 %Similarly to our model, the agents have a finite number of types, but in their model, 
%Unlike or paper,  We further note that in our model, the decision maker does not necessarily learn the type distribution after the sampling phase (in fact, if there are several rare types, the decision maker will not learn the distribution with high probability).

