
\section{Proof of Theorem~\ref{theo:ECEEI}: Existence of \eceei}\label{app:eceei}


\cite{nguyen2021delta} establish the following result on the  existence of a pseudoequilibrium in a more general setting.
%, that \sv{can we say ``a more general setting''?} settings that including ours.

 \begin{lemma}[\citealp{nguyen2021delta}]\label{lemma:pseudo}
Let $\choices_{\agent}$ denote the finite set of bundles that agent $j\in N$ can feasibly consume, $\emptybundle \in \choices_{\agent}$. Each agent $\agent$'s utility function $\util_{\agent}(\bundle,\prices)$ satisfies
\begin{itemize}
    \item $\util_{\agent}(\emptybundle,\prices)=0$,
    \item $\util_{\agent}(\bundle,\prices)=-\infty$ for $\bundle\notin \choices_{\agent}$
    \item  $\util_{\agent}(\bundle,\prices)$ is continuous  in $\prices \in \mathbb{R}^m$ for each  $\bundle\in \choices_{\agent}$ and
    \item there exists some constant  $C>0$ such that if $\prices\cdot\bundle \ge C$, then, $\util_{\agent}(\bundle,\prices)<0$.
    %\sv{not sure what to replace $B$ by}
\end{itemize}
Then, there exists a pseudoequilibrium.
\end{lemma}

We use this result to prove Theorem~\ref{theo:ECEEI}.
\proof{Proof of Theorem~\ref{theo:ECEEI}. }
The utilities $\util^{\epsilon}_{\agent}(.,.)$ in the theorem statement  satisfy Lemma~\ref{lemma:pseudo}, therefore a pseudoequilibrium exists.  We use this pseudoequilibrium to construct an $\epsilon$-\eceei, as follows.

%$\optallocs=(\optgallocv{\agent}, \ldots, \optgallocv{\nagents})$

Let  $\optprices$ and $\{\optgallocv{\agent}\in conv(Ch_{\agent}(\optprices))\}$ be a pseudoequilibrium under the utilities $\util^{\epsilon}_{\agent}(.,.)$. For  $\y\in Ch_{\agent}(\optprices)$, define a budget $b_\y$ as follows:

\begin{equation}\label{eq:budget}
  b_\y=\begin{cases}
    1-\epsilon  & \text{if $\optprices\cdot \y\le 1-\epsilon$}.\\
    \optprices\cdot \y  & \text{otherwise}.
  \end{cases}
\end{equation}
Notice  that $\optprices\cdot \y<1$ because  $\y\in Ch_{\agent}(\optprices)$, $\util^{\epsilon}_{\agent}(\y,\optprices)\ge 0$. Thus, for every $\y\in Ch_{\agent}(\optprices)$, $1-\epsilon \le b_\y<1$.
Next we show that $\y\in Ch_{\agent}(\optprices)$ implies that $\y$ is the best bundle among all the bundles with a cost at most $b_\y$. That is:
\begin{equation}\label{eq:optimal}
\y=\max_{\succ_{\agent}}\{\gallocv{} \in \choices_{\agent} \text{ and } \optprices\cdot \gallocv{} \leq b_\y\}.
\end{equation}
This is true because in the case that $b_\y= 1-\epsilon$,  the auxiliary utility of bundles with a cost at most $b_\y$ is equal to the  numerical value $r_{\agent}(.)$ associated with its ordinal ranking. $\y\in Ch_{\agent}(\optprices)$ implies that $r_{\agent}(\y)$ is the highest among these bundles.
Now, if $b_\y=\optprices\cdot \y >1-\epsilon$, then  for any bundle $\bundle$ such that  $r_{\agent}(\bundle)>r_{\agent}(\y)$ we have $\optprices\cdot \bundle>\optprices\cdot \y$, otherwise the  
auxiliary utility of bundle $\bundle$ is strictly higher than that of  $\y$,
contradicting the fact that $\y\in Ch_{\agent}(\optprices)$.



Since $\optgallocv{\agent}\in conv(Ch_{\agent}(\optprices))$,  $\optgallocv{\agent}$ can be expressed as the  expectation  of a lottery $\mathcal Z$ over the bundles in $Ch_{\agent}(\optprices)$.  Let  
$\optrandombudget{\agent}$ be the corresponding random budget defined as in \eqref{eq:budget} for $\y$ drawn from $\mathcal Z$.

In this construction, every realization of $\optrandombudget{\agent}$ is $b_\y$ for $\y\in Ch_{\agent}(\optprices)$, thus it is between $1-\epsilon$ and 1.
Moreover, because of \eqref{eq:optimal}, the random optimal bundle with respect to budget $\optrandombudget{\agent}$ is the lottery $\mathcal Z$ over $Ch_{\agent}(\optprices)$ with the average equal to $\optgallocv{\agent}$. The market-clearing condition of the pseudoequilibrium implies the market clear condition of the \eceei.
Finally, agents of the same type have the same auxiliary utility, and we can select a pseudoequilibrium such that they have the same fractional allocation, which implies that agents of the same type have the  same random budgets in the $\epsilon$-\eceei.  
\proofen


\section{\eceei\ implies  \aceei}

In this section, we show that 
 the existence of \eceei~implies the existence of an approximate competitive equilibrium from equal incomes with the same bound on the {excess} demand as in \cite{budish2011combinatorial}. In particular we show the following result.

 \begin{proposition}~\label{prop:budish}
     There is a realisation of  \eceei\ which has the excess demand bounded in $\ell_2$--norm at most  $\sqrt{\sigma m/2}$, where $\sigma$ is the size of the maximum bundle consumed by any agent.
 \end{proposition}
 

To show this, we use the following improved bound for the Shapley--Folkman theorem \citep{budish2020improved} in order to obtain an  \aceei. 
\begin{lemma}[Theorem 3.1 in \citealp{budish2020improved}] \label{theo:bush} 
If $S_1,..,S_n$ are compact subsets of $\mathbf{R}^m$, if ${\bf c}\in conv(S_1+..+S_n)$, and $D$ is the maximum diameter of $S_i$, then there exists $\x_i\in S_i$ such that 
    $$
    ||{\bf c}-\sum_i \x_i||_{\ell_2} \le D\sqrt{m}/2.
    $$
\end{lemma}

\proof{Proof of Proposition~\ref{prop:budish}. }
Let $\prices, \randombudget{}$ be an $\epsilon$-\eceei.  Apply Lemma~\ref{theo:bush}, where $S_i=\optbundle{\agent}(\prices,\randombudget{\agent})$ and ${\bf c}$ is the capacity vector.  If all agents' feasible bundles are of size at most $\sigma$, then the diameter of $S_i$ is at most $\sqrt{2\sigma}$. We obtain the existence of 
$\x_i\in S_i$ such that $ ||{\bf c}-\sum_i \x_i||_2 \le \sqrt{\sigma m/2}.$ For each $\x_i\in \optbundle{\agent}(\prices,\randombudget{\agent}) $, there is a realization $b_i$ of $\randombudget{\agent}$ such that 
$\x_i$ is the optimal choice of agent $i$ under budget $b_i$. Thus the allocation $\{\x_1,..,\x_n\}$ corresponds to an approximate equilibrium, where the budget of each agent is perturbed by at most $\epsilon$.
\proofen




\section{Concentration Inequalities}
We use the following concentration inequalities.

\begin{theorem} [Chernoff Bound]\label{Chernoff}
Let $X_1, X_2, \ldots, X_n$ be independent random variables with  $X_i = 1$ with probability $p_i$ and $X_i = 0$ with probability $1-p_i$. Define $X = \sum_{i=1}^{N} X_i$, and  $\mu = \expect{X} = \sum_{i=1}^{N} p_i$. Then for $\epschernoff   \in (0,1)$,
\begin{enumerate}
\item $\Pr[X-\mu \geq \epschernoff \mu ] \leq \exp\left(-\mu\epschernoff ^2/3\right)$,
\item $\Pr[\mu-X \geq \delta \mu ] \leq \exp\left(-\mu\epschernoff ^2/2\right)$.
\end{enumerate}
\end{theorem}

The following is adapted from~\cite{bardenet2015concentration}. They only prove a single-sided bound (i.e., only the first inequality), but it is not too difficult to see that the proof holds for the complementary inequality.\footnote{Their proof hedges on showing that $Z^*_k = \frac{1}{N-k}\sum_{t=1}^{k}(X_t-\mu)$ is a martingale: $\expect{Z^*_k|Z^*_{k-1}, \ldots,Z^*_1} = Z^*_{k-1}$. It is straightforward to adapt their proof to show that $Z_k = \frac{1}{N-k}\sum_{t=1}^{k}(\mu-X_t)$ is also a martingale, and the result follows from this. } A similar (two-sided) bound can be derived from~\cite{Serfling}. We note that the bound of~\cite{bardenet2015concentration} is tighter than the one below, but for our values of $\nagents$  and $\nsample$ (which represent the number of agents and sample size respectively), the improvement is negligible.

\begin{theorem} [Hoeffding-Serfling Inequality] \label{Hoeffding}
 Let  $\hoefset = \{\hoef_1, \ldots, \hoef_\nagents\}$ be a finite set of $\nagents$ elements, where for $\agent \in \agentset$, $\hoef_\agent \in [0,1]$.  Let $\hoefrand_1, \ldots, \hoefrand_\nsample$ be a random sample drawn without replacement from $\hoefset$. Define $\avgpref = \frac{1}{\nagents}\sum_{\agent = 1}^{\nagents} \hoef_\agent$.  Then for $\epsHoef>0$, 
 \begin{enumerate}
 \item $\Pr\left[\max_{ \nsample \leq \kn \leq \nagents} \sum_{\agent = 1}^{\kn} \hoefrand_\agent - \kn \avgpref > \kn \epsHoef  \right] \leq \exp{\left(-2\nsample \epsHoef^2\right)},$
\item $\Pr\left[\max_{ \nsample \leq \kn \leq \nagents} \sum_{\agent = 1}^{\kn} \hoefrand_\agent - \kn \avgpref < \kn \epsHoef  \right] \leq \exp{\left(-2\nsample \epsHoef^2\right)}.$
\end{enumerate}
\end{theorem}
The following bound is tighter than the Hoeffding bound when the variance of $\hoefset$ is bounded. We use a version that appears in~\citep[e.g.,][]{aw}.

\begin{theorem}[Hoeffding-Bernstein Inequality]\label{thm:Bern} Let  $\hoefset = \{\hoef_1, \ldots, \hoef_\nagents\}$ be a finite set of $\nagents$ elements, where for $\agent \in \agentset$, $\hoef_\agent \in [0,1]$.  Let $\hoefrand_1, \ldots, \hoefrand_\nsample$ be a random sample drawn without replacement from $\hoefset$. Define $\avgpref = \frac{1}{\nagents}\sum_{\agent = 1}^{\nagents} \hoef_\agent$.  Then  for $\epsBern>0$,
	$$\Pr\left[ \left|\sum_{\agent=1}^{\nsample} \hoefrand_\agent -  \nsample \avgpref \right| \geq \epsBern\right]\leq 2\exp\left( \frac{-\epsBern^2}{2\nsample \sigma^2_\nagents+\epsBern} \right),  $$
	where $\sigma^2_\nagents = \frac{1}{\nagents}\sum_{\agent=1}^\nagents (\hoef_\agent-\avgpref)^2$.
\end{theorem}






\section{Proof of Theorem~\ref{thm1}} \label{app:proofthm}

We first prove Parts~\ref{thm1a} and~\ref{thm1b} of Theorem~\ref{thm1}, rephrased as the following lemma.
\begin{lemma}\label{lemmathm1}
For any economy $(N,\pi,M,\supplyvector{}, (\choices_\agent)_{\agent\in\agentset}, \succ)$,  $\epsf, \sampleeps>0$ and $0<\budgeteps< \frac{1}{m}$, 
 the \ocam\ is: 
   \begin{enumerate}[label=(\alph*)]
   \item group-strategyproof up to one object,\label{asdasd1a}
   \item envy-free up to one object for a $(1-\epss)$ fraction of the agents, where $\epss = \frac{\epsn\epsf}{4}$. \label{asasd1b}
   \end{enumerate}
\end{lemma}


\proof{Proof of Lemma~\ref{lemmathm1}.} For the first $\realsample$ agents (the sample), the mechanism is a serial dictatorship and therefore group-strategyproof. It remains to show that when $\epsb<\frac{1}{m}$, the allocations of the agents arriving after the sample are  envy-free, and group-strategyproof up to one object.


First observe that, in the OCAM, the allocations of agents arriving after the sample  phase
do not depend on the reported type of others. For these agents, 
when misreporting their type from $t$ to $t'$, they will receive an allocation that  agents of type $t'$ receive.  
Thus, ex-post envy-freeness up to one object implies group-strategyproofness up to one object.


It remains to show that the allocation is ex-post envy-free up to one object for the agents arriving after the  sample phase. Let $i,i'$ be two such agents, and let $(\x_i, b_i)$ and   $(\x_{i'}, b_{i'})$ be the ex-post allocation and budget of agent $i$ and $i'$, respectively. 


Assume the contrary: that taking any object out of $\x_{i'}$, agent $i$ still prefers it to their current bundle $\x_i$, then it must be that the cost of that bundle ($\x_{i'}$ with the object removed) is higher than $b_i$. Thus,  we have
$$b_i< \prices \cdot (\x_{i'}-\e^j) \text{ for every  good $j$ contained in bundle $\x_{i'}$}. $$
Summing up these inequalities for all such  $j$, we obtain  
$$
\sigma\cdot  b_i< (\sigma -1)\cdot  \prices \cdot \x_{i'}, \text{ where $\sigma$ is the size of bundle } \x_{i'}.
$$
On the other hand, because $\x_{i'}$ is the bundle consumed by agent $i'$, we have $\prices \cdot \x_{i'}\le b_{i'}$. Thus, $\sigma\cdot  b_i< (\sigma -1)\cdot b_{i'}$, which implies
$$
\frac{b_i}{b_{i'}}<\frac{\sigma-1}{\sigma}\le \frac{m-1}{m}.
$$
This is a contradiction because both $b_i,b_{i'} \in (1-\frac{1}{m},1)$.
\proofen


 To complete the proof of Theorem~\ref{thm1}, we need to show that   when   Assumptions~\ref{ass0} and \ref{ass1} hold, the mechanism outputs a tuple $(\optallocs,\optbudgets,\optprices)$   %allocation $(\allocv{1}, \ldots, \allocv{\nagents})$, budgets $(\budget{1},\ldots, \budget{\nagents})$ and prices $(\price{1}, \ldots, \price{\ngoods})$ 
that constitutes an $\daisyeps-$\mech\ with probability at least $1- \frac{1}{\nagents}$. 
We first require some additional notation. 

For any subset of size $\realsample$ of the agents, we define a triple $\triple = (\outputtypeset, \budgetfunc, \prices)$ where $\outputtypeset$ is an unordered set (with repetition) of the types of the agents in the subset, $\budgetfunc$ is a function mapping agent types  to 
 a random budget, and $\prices$ is a price vector for $\ngoods$ objects ($\budgetfunc$ and $\outputtypeset$ are as defined in the pseudocode of Mechanism~\ref{alg:one}.) The triple $\triple$ will be used to define allocations for all of the agents. However, the allocations defined by $\triple$ and those generated by the mechanism are not (necessarily) the same:   the allocations will be identical  for agents $\realsample+1, \ldots, \nagents$,  but  for agents $1, \ldots, \realsample$, they  may not be. For any agent $\agent$ and triple $\triple = (\outputtypeset, \budgetfunc, \prices)$, let  $\ballocv{\agent}(\triple)\in [0,1]^{2^\ngoods}$ denote the randomized allocation of agent $\agent$ that is generated by $\triple$, where $\balloc{\agent}{\bundle}$ denotes the probability that agent $\agent$ is allocated bundle $\bundle$. % (we never explicitly compute   $\ballocv{\agent}(\triple)$; we only use it for clarity in the proofs).  
  The value of $\ballocv{\agent}$ is  uniquely determined by $(\outputtypeset, \budgetfunc, \prices)$.  Let $\rallocv{\agent}(\triple)\in [0,1]^{\ngoods}$ be  such that $\ralloc{\good}{\agent}(\triple)$ denotes the  probability that agent $\agent$ is allocated object $\good$ (that is, $\ralloc{\good}{\agent}(\triple) = \sum_{\bundle: \good \in \bundle}\balloc{\agent}{\bundle}(\triple)$); let $\rvrallocv{\agent}$ be the corresponding random variable (with the randomness over the choice of $\triple$).
Finally, let $\rvallocv{\agent} (\triple)$ be the random variable denoting the realization of $\ballocv{\agent}(\triple)$; that is, $\rvalloc{\good}{\agent}(\triple) \in \{0,1\}$, where $\rvalloc{\good}{\agent}(\triple) = 1$ with probability $\ralloc{\good}{\agent}(\triple)$. %$\allocv{\agent} $ is a random variable whose value is determined by the value of $\triple$ as well as the realization of $\budget{\agent}$.
Note that the events $\rvalloc{\good}{\agent}(\triple) = 1 $ and $\rvalloc{\agent}{\good'}(\triple)= 1$ are dependent,  but for any $\agent \neq \agent'$, $\good, \good'$, $\rvalloc{\good}{\agent}(\triple)= 1 $ and $\rvalloc{\agent'}{\good'}(\triple)= 1 $ are independent.  Overloading the notation, let $\rvallocv{\agent}$ denote the random variable for the allocation of agent $\agent$, where the sample space includes randomness from both the random arrival order and the budget selection.
In Sections~\ref{d1} and~\ref{d2}, we bound the complete randomized and deterministic allocations defined by the triple $\triple$ (where by `complete' we mean with respect to all of the agents' allocations). In Sections~\ref{d3} and~\ref{d4} we use these results to prove bounds on the allocations generated by~\ocam. 




\subsection{Step 1: Bounding the  randomized allocation generated by the triple \texorpdfstring{$\triple$}{q}. }\label{d1}

For the first step, we fix an arbitrary triple $\triple = (\outputtypeset, \budgetfunc, \prices)$ and define a sample to be bad with respect to this triple as follows:






\begin{definition}
We say that a sample $\sample$ is \emph{bad} for triple $\triple = (\outputtypeset, \budgetfunc, \prices)$ and object $\good$ if  either 
\begin{enumerate}[label=(\roman*)]
\item $\sum_{\agent \in \sample}\ralloc{\good}{\agent}(\triple) \leq \epss \supply{\good}$ and $\sum_{\agent \in \agentset}\ralloc{\good}{\agent}(\triple) >  \left(1+\epsfa\right) \supply{\good}$ or
\item $\sum_{\agent \in \sample}\ralloc{\good}{\agent}(\triple) \geq  \epss \supply{\good}$ and $\sum_{\agent \in \agentset}\ralloc{\good}{\agent}(\triple) <  \left(1-\epsfa\right)\supply{\good}$.
 %$X^j(\sample) \geq (1-2\epsf)c s_j$ and $S^j(\sample) < s_j(1-\epsf)$. ?? both sides??
 \end{enumerate}
\end{definition}




\begin{lemma}\label{lemma:rnd}
For any $\epsilons$, if Assumption~\ref{ass1} holds, then the randomized allocation $\rvrallocv{}$ and prices $\prices$ generated  by the triple $\triple$ satisfy the following: 
\begin{enumerate}
\item The probability that there exists some object $\good \in \goodset$ such that $\sum_{\agent \in \agentset}\rvralloc{\good}{\agent} >\left(1+\epsfa\right) \supply{\good}$ is at most $\repsc$, and 
\item The probability that there exists some object $\good \in \goodset$ such that both $\price{\good}>0$ and $\sum_{\agent \in \agentset}\rvralloc{\good}{\agent} < \left(1-\epsfa\right) \supply{\good}$  is at most $\repsc$,
\end{enumerate}
where the randomness is over the random arrival permutation.
\end{lemma}

\proofst
Fix a triple $\triple = (\outputtypeset, \budgetfunc, \prices)$ and object $\good$. We first show that the probability that a sample $\sample$  is bad with respect to this triple and object is low. The sample is bad if $\sum_{\agent \in \sample}\ralloc{\good}{\agent}(\triple) \leq  \epss \supply{\good} \wedge \sum_{\agent \in \agentset}\ralloc{\good}{\agent}(\triple) > \left(1+\epsfa\right) \supply{\good}$. We bound the probability that this event happens; the probability is over the choice of the sample. 
For any triple $\triple$, set $\tempy{\agent} = \dfrac{\supply{\good}\ralloc{\good}{\agent}(\triple)}{\sum_{\agent \in \agentset}\ralloc{\good}{\agent}(\triple)}$.

\begin{align}
& \Pr \left[\sum_{\agent \in \sample}\ralloc{\good}{\agent}(\triple) \leq \epss \supply{\good} \wedge \sum_{\agent \in \agentset}\ralloc{\good}{\agent}(\triple) > \left(1+\epsfa\right) \supply{\good}\right] \notag \\
&\leq \Pr \left[\sum_{\agent \in \sample}\tempy{\agent} \leq  \epss \supply{\good} \wedge \sum_{\agent \in \agentset}\tempy{\agent} = \left(1+\epsfa\right) \supply{\good}\right]\notag\\
&\leq \Pr \left[\sum_{\agent \in \sample}\tempy{\agent} \leq \epss \supply{\good} \bigg|  \sum_{\agent \in \agentset}\tempy{\agent} = \left(1+\epsfa\right) \supply{\good}\right]\notag\\
&\leq  \Pr \left[  \left|\sum_{\agent \in \sample}\tempy{\agent}  - \epss \expect{\sum_{\agent \in \agentset}\tempy{\agent} } \right| \geq \epsfa \epss \supply{\good}\right] \notag\\
&\leq 2 \exp \left(-\frac{(\epsf \epss \supply{\good})^2}{64 \epss \supply{\good} +\epsf \epss \supply{\good} } \right) \label{feelTheBern}\\
&\leq 2 \exp \left(-\frac{\epsf^2 \epss \supply{\good}}{65} \right), \notag\\
& \leq \frac{1}{6\ngoods \nagents \cdot (\epss \nagents)^{\numtypes}}. \notag
\end{align}
Inequality~\eqref{feelTheBern} is obtained using Theorem~\ref{thm:Bern}, noting that  $\sum_{\agent=1}^\nagents (\tempy{\agent}-\frac{1}{\nagents}\sum_{\altagent = 1}^{\nagents}{\tempy{\altagent}})^2 \leq 2\supply{\good}$; hence $\sigma^2_\nagents \leq \frac{\supply{\good}}{\nagents}$. The last inequality is due to Assumption~\ref{ass1}.

We now take a union bound over all distinct triples and objects. Each triple $\triple = (\outputtypeset, \budgetfunc, \prices)$ is uniquely determined by $\outputtypeset$. Therefore it suffices to bound the possible values of $\outputtypeset$. As the sample size is $\epss \nagents$, each type can appear in the sample at most $\epss \nagents$ times; hence there  are at most $(\epss \nagents)^{\numtypes}$ possible values of $\outputtypeset$, and therefore of $\triple$.  While this is a loose upper bound, it is asymptotically tight if there are no restrictions on the type space. Taking a union bound over the objects and possible values of $\triple$ gives the first result. The proof for the second result is similar, only we have to bound the probability that $\sum_{\agent \in \sample}\ralloc{\good}{\agent}(\triple) \geq  \epss \supply{\good} \wedge \sum_{\agent \in \agentset}\ralloc{\good}{\agent}(\triple) < \left(1-\epsfa\right) \supply{\good}$ for objects $\good$ such that $\price{\good} > 0$. As the bound of  Theorem~\ref{thm:Bern} is symmetrical, the proof is virtually identical and omitted.  
\proofen


\subsection{Step 2: Bounding the deterministic allocation generated by the triple \texorpdfstring{$\triple$}{q}.}

\label{d2}
\begin{lemma}\label{lemma:det}
For any $\epsilons$, if Assumption~\ref{ass1} holds, then the (deterministic) allocation  and prices generated  by the triple $\triple$ satisfy the following: 
\begin{enumerate}
\item The probability that there exists some object $\good \in \goodset$ and $\knum \in [\epsn\nagents,\nagents]$ such that  $\sum_{\agent = 1}^{\knum} \rvalloc{\good}{\agent} > \left(1+\epsfc\right)\dfrac{\knum\supply{\good}}{\nagents}$  is at most $\repsa$ and 
\item  The probability that there exists some object $\good \in \goodset$ and $\knum \in [\epsn\nagents,\nagents]$ such that both $\price{\good}>0$ and  $\sum_{\agent = \realsample}^{\knum} \rvalloc{\good}{\agent} < (1-\epsfd)\dfrac{\knum\supply{\good}}{\nagents}$  is at most $\repsa$,
\end{enumerate}
where the probability is taken over the  random arrival permutation and the budget realization.
\end{lemma}


\proofst
For  any triple $\triple = (\outputtypeset, \budgetfunc, \prices)$ and object $\good$, denote the event  $\sum_{\agent \in \agentset}\ralloc{\good}{\agent}(\triple) >  \left(1+\epsfa\right)\supply{\good}$ by $\eventy$, %Recall that  $\rvalloc{\good}{\agent}(\triple)$ is the  random variable for the deterministic allocation of object $\good$ to agent $\agent$ (where the randomness is over the choice of budget). 
 the event $\sum_{\agent = 1}^{\nagents} \rvalloc{\good}{\agent}(\triple) > (1+\epsfb) \supply{\good}$ by $\eventtwo$, and  the event  that there exists some $\knum \in [\epsn\nagents,\nagents]$ such that $\sum_{\agent = 1}^{\knum} \rvalloc{\good}{\agent}(\triple) > \left(1+\epsfc\right)\frac{\knum\supply{\good}}{\nagents}$ by $\eventone$. We first bound the probability of $\eventtwo$. 
\begin{align}
\Pr\left[\eventtwo\right] &\leq \Pr\left[\eventtwo |\neg \eventy \right] \Pr\left[ \neg \eventy \right] + \Pr\left[  \eventy \right] \notag\\ 
%\Pr\left[\sum_{\agent \in \agentset}\rvalloc{\good}{\agent}(\triple) > (1+\epsf)\supply{\good}\right] 
& \leq \Pr\left[\sum_{\agent \in \agentset}\rvalloc{\good}{\agent}(\triple) - \sum_{\agent \in \agentset}\ralloc{\good}{\agent}(\triple)> \epsft\supply{\good}\right]\label{reason1} + \repsc\\
&= \Pr\left[\sum_{\agent \in \agentset}\rvalloc{\good}{\agent}(\triple) - \expect{\sum_{\agent \in \agentset}\rvalloc{\good}{\agent}(\triple)}> \epsft\supply{\good}\right] +\repsc\notag \\%\label{reason2}\\
&\leq  \exp\left(-\frac{\epsf^2\supply{\good}}{32}\right) +\repsc \label{reason3}\\
&\leq \repsb,  \notag
\end{align}
%where~\eqref{reason1} is because $\sum_{\agent = 1}^{\Nn} \ralloc{\good}{\agent}(\triple) \leq \supply{\good}$, 
where~\eqref{reason3} is from Theorem~\ref{Chernoff}, using the  fact that conditioned on $ \neg \eventy$, $\expect{\sum_{\agent = 1}^{\Nn} \rvalloc{\good}{\agent}} \leq (1+\epsfb)\supply{\good}$, and the last inequality is due to Assumption~\ref{ass1}. 


We can now bound the probability of $\eventone$ using the bound on $\eventtwo$ and Theorem~\ref{Hoeffding}:

\begin{align}\Pr\left[\eventone\right] &\leq \Pr\left[\eventone |\neg \eventtwo \right] \Pr\left[ \neg \eventtwo \right] + \Pr\left[  \eventtwo \right] \notag\\ 
& \leq  \Pr\left[\max_{\knum \in [\epsn\nagents,\nagents]}\sum_{\agent = 1}^{\knum} \rvalloc{\good}{\agent}(\triple) > \left(1+\epsfc\right)\dfrac{\knum\supply{\good}}{\nagents}\bigg| \sum_{\agent = 1}^{\nagents} \rvalloc{\good}{\agent}(\triple) \leq  \left(1+\epsfb\right)\supply{\good}\right ] +\repsb\notag\\
& \leq  \Pr\left[\max_{\knum \in [\epsn\nagents,\nagents]} \left\{\sum_{\agent = 1}^{\knum} \rvalloc{\good}{\agent}(\triple) - \frac{\knum}{\nagents}\sum_{\agent = 1}^{\nagents} \rvalloc{\good}{\agent}(\triple) \right\}>\epsfa\dfrac{\knum\supply{\good}}{\nagents}\right ]+\repsb\notag\\ 
&\leq \exp{\left(\frac{- 2\realsample\epsf^2{\supply{\good}}^2}{16\nagents^2}\right)}+\repsb\label{eq:serfling}\\
&\leq \repsa,\notag
\end{align}
where Inequality~\eqref{eq:serfling} is due to Theorem~\ref{Hoeffding} and the last inequality is due to Assumption~\ref{ass1}.

The proof that conditioned on $\price{\good}>0$, $\Pr\left[\sum_{\agent = \realsample}^{\nagents}\rvalloc{\good}{\agent}<(1-\epsfd)\supply{\good}\right] \leq \repsa$ is similar and omitted.
\proofen









\subsection{Step 3: Bounding the deterministic allocation of the~\ocam}

\label{d3}


\begin{lemma}\label{lemma:detmech}
For any $\epsilons$, if Assumption~\ref{ass1} holds, the output of the~\ocam,  $(\optallocs,\optbudgets,\optprices)$ satisfies the following: 
\begin{enumerate}
\item The probability that there exists some object $\good \in \goodset$ and $\knum \in [\epsn\nagents,\nagents]$ such that  $\sum_{\iter = 1}^{\agentdef} \optgalloc{\good}{\iter} > (1+\epsfd)\dfrac{\knum\supply{\good}}{\nagents}$  is at most $\repsa$ and 
\item  The probability that there exists some object $\good \in \goodset$ and $\knum \in [\epsn\nagents,\nagents]$ such that both $\price{\good}>0$ and  $\sum_{\iter = 1}^{\agentdef} \optgalloc{\good}{\iter} < (1-\epsfd)\dfrac{\knum\supply{\good}}{\nagents}$  is at most $\repsa$,
\end{enumerate}
where the probability is taken over the  random arrival permutation and the budget realization.
\end{lemma}


\proofst
 The probability that $\exists \agentdef \in [\realsample,\nagents], \good \in \goodset$ such that $\sum_{\iter = 1}^{\agentdef} \optgalloc{\good}{\iter}  \geq  (1+\approxeps)\dfrac{\agentdef\supply{\good}}{\nagents}$  can be bounded as follows:
 
 \begin{align*}   \Pr\left[ \sum_{\iter = 1}^{\agentdef} \optgalloc{\good}{\iter}  >(1+\approxeps)\dfrac{\agentdef\supply{\good}}{\nagents}\right]  
 \leq   \Pr\left[ \sum_{\iter = 1}^{\agentdef} \rvalloc{\good}{\agent}
> \left(1+\epsfc\right)\dfrac{\agentdef\supply{\good}}{\nagents}\right]
 \leq \repsa,
\end{align*}

where the first inequality is because 
\begin{enumerate}
\item $\sum_{\iter = 1}^{\knum} \optgalloc{\good}{\iter} \leq  \sum_{\iter = 1}^{\realsample} \optgalloc{\good}{\iter}+  \sum_{\iter = 1}^{\agentdef} \rvalloc{\good}{\agent}$, and 
\item $\sum_{\iter = 1}^{\realsample} \optgalloc{\good}{\iter}  \leq \epss\supply{\good} \leq   \frac{\epsf\epsn}{4}\supply{\good} \leq \epsfa\frac{\agentdef\supply{\good}}{\nagents},$ 
\end{enumerate}
and the second inequality is from Lemma~\ref{lemma:det}.
Similarly, if $\price{\good}>0$, we have 

 \begin{align*}   \Pr\left[ \sum_{\iter = 1}^{\agentdef} \optgalloc{\good}{\iter}  <  (1-\approxeps)\dfrac{\agentdef\supply{\good}}{\nagents}\right]  
 \leq   \Pr\left[ \sum_{\iter = \realsample}^{\agentdef} \rvalloc{\good}{\agent}
<  \left(1-\epsfd\right)\dfrac{\agentdef\supply{\good}}{\nagents}\right]
 \leq \repsa,
\end{align*}
where the first inequality is because $\sum_{\iter = 1}^{\knum} \optgalloc{\good}{\iter} = \sum_{\iter = 1}^{\realsample} \optgalloc{\good}{\iter}+  \sum_{\iter = \realsample}^{\agentdef} \rvalloc{\good}{\agent}$, and the second inequality is from Lemma~\ref{lemma:det}.
\proofen




\subsection{Step 4: Putting it all together}
\label{d4}

\proof{Proof of Theorem~\ref{thm1}.}
Parts~\ref{thm1a} and~\ref{thm1b}  of Theorem~\ref{thm1} are proven in  Lemma~\ref{lemmathm1}. From the construction of the allocations in the \ocam, the output $(\optallocs,\optbudgets,\optprices)$ trivially satisfies Parts~\ref{ACEEI1}  and~\ref{ACEEI2}  of Definition~\ref{def:aceeb} (for all $k \in [1,n]$).   Lemma~\ref{lemma:detmech} shows that  $(\optallocs,\optbudgets,\optprices)$ satisfies Parts~\ref{ACEEI3} and~\ref{ACEEI4} of Definition~\ref{def:aceeb} with probability at least $\frac{1}{n}$.
\proofen







\clearpage 


\section{Pseudocode for the OCAM}


\begin{algorithm}[htpb]

\medskip

	\SetAlgoNoLine
	\KwIn{A set $\goodsset$ of goods  with capacity  $\supplyvector{}$, $\epsilons$, $\nagents$ agents arrive online.}
	\KwOut{Allocation $\allocv{}$, budgets $\budgets$ and prices $\prices$.}
    Set $\epss = \frac{\epsf \epsn}{4}$\;
    \For{each  agent $\agent \in [\realsample]$}
        {Agent $\agent$ reports their type $\ttype_{\agent}$\;   
    
        Set $\allocv{\agent} = \max_{\succ_{\agent}}\{\bundle{}: \bundle{} \in \choices_{\agent} \text{ and } \sum_{\ell=1}^{\agent} \allocv{\ell} \leq \epss \supplyvector\}$\;
    
         Arbitrarily set $\budget{\agent} \in [1-\epsb,1]$\;
        }
 
    Define the economy $\econ'=([\realsample],\iden,M, \epss\supplyvector{}, (\choices_\agent)_{\agent\in[\realsample]}, (\succ_\agent)_{\agent \in [\realsample]})$, where $\iden$ is the identity permutation\;
    
    Compute an $\epsb$-\eceei\  for  $\econ'$:  $\bundlev, \randombudget{1}, \ldots, \randombudget{\realsample},  \prices$ \; 
    

    Set $ \sampletypeset = \cup_{\agent \in \{1, \ldots, \realsample\}} \ttype_{\agent}$\; 
        
    Define $\budgetfunc: \sampletypeset \rightarrow \{\randombudget{1}, \ldots, \randombudget{\realsample}\}$ as follows:  $\budgetfunc(\ttype) = \randombudget{\agent}$, where %$\agent \in \{1, \ldots, \sampleeps \nagents\} $ and 
    $\ttype_{\agent} = \ttype $ for some $\agent \in [\realsample]$\; %If  $\ttype \notin \sampletypeset$, set $\budgetfunc(\ttype) = 1-\budgeteps$.

    \For{each  agent $\agent \in \{ \realsample+1, \ldots, \nagents\}$} {Agent $\agent$ reports their type $\ttype_{\agent}$\;
    \eIf{$\ttype_\agent \in \sampletypeset$}{Draw $\budget{\agent} \sim \budgetfunc(\ttype_\agent)$\;}
    {Arbitrarily set $\budget{\agent} \in [1-\epsb,1]$\;} %, where $\ttype_{\agent}$ is agent $\agent$'s type and 
    Set $\allocv{\agent} = \max_{\succ_{\agent}}\{\bundle{}: \bundle{} \in \choices_{\agent} \text{ and } \prices\cdot \bundle{} \leq \budget{\agent}\}$\;} % (i.e., allocate the agent their favorite bundle conditioned on their budget).
    %\item Using the above allocation rule, allocate bundles to agents  $1, \ldots, \sampleeps \nagents$, and thereafter to every agent upon arrival. 
    

	\caption{\ocam}
 %$(\sampleeps,\budgeteps,\approxeps)-$\mech\ Mechanism}
	\label{alg:one}
\end{algorithm}
 













