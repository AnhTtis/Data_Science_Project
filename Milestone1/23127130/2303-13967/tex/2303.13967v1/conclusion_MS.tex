\section{Conclusion}
The \ocam\  is a promising mechanism for complex, dynamic allocation problems without money. It finds an attractive compromise among fairness, efficiency and incentives for truthtelling and uses up the capacity of each good at a constant rate.

There are several possible fruitful directions for further work. One direction is theoretical: one might wish to relax the assumption that the arrival order is random (Assumption~\ref{ass0}); however, this would require a significant adaptation of our techniques. 
Theorem~\ref{thm1} suggests that if the initial sample represents the population well, then equilibrium prices computed from initial sample will be  a good guide to an efficient allocation for the remaining agents. In richer environments with a non-stationary arrival order, one could, for example, consider using adaptive sampling to ensure that the samples are representative.
Another direction would be to test the \ocam\ empirically or 
even implement it in practice. %Of course, further complexities, such as preference elicitation, might arise in real-world market design. % \citep{budish2017course}.
%\sv{The last sentence ends on a negative note. Remove it or move it up?}
