% This is samplepaper.tex, a sample chapter demonstrating the
% LLNCS macro package for Springer Computer Science proceedings;
% Version 2.20 of 2017/10/04
%
\documentclass[runningheads,a4paper]{llncs}
%
\usepackage{amsmath, amssymb}
\usepackage{graphicx}
\usepackage{amsfonts}
\usepackage{xcolor}
\usepackage{caption}
\usepackage{subcaption}
\usepackage{algorithm}
\usepackage{algpseudocode}
\usepackage[inline]{enumitem}
\usepackage{bm}
\usepackage{booktabs}
\usepackage{hyperref} 

% Used for displaying a sample figure. If possible, figure files should
% be included in EPS format.
%
% If you use the hyperref package, please uncomment the following line
% to display URLs in blue roman font according to Springer's eBook style:
% \renewcommand\UrlFont{\color{blue}\rmfamily}

\begin{document}
%
\title{Bayesian Optimization for Function Compositions with Applications to Dynamic Pricing}
% \title{Pricing/Demand Curve Optimisation using Novel Bayesian Optimisation Techniques}
%

\titlerunning{BO for Function Compositions}

\author{Kunal Jain\inst{1}, Prabuchandran K. J.\inst{2} \and Tejas Bodas\inst{1}}

\authorrunning{K. Jain et al.}

\institute{International Institute of Information Technology, Hyderabad \\ \email{\{kunal.jain@research.,tejas.bodas@\}iiit.ac.in} \and Indian Institute of Technology, Dharwad \\ \email{prabukj@iitdh.ac.in}}

\footnotetext[2]{Prabuchandran K.J. was supported by the Science and Engineering Board (SERB), Department of Science and Technology, Government of India for the startup research grant ‘SRG/2021/000048’.}
\maketitle              % typeset the header of the contribution
%
\begin{abstract}
Bayesian Optimization (BO) is used to find the global optima of black box functions. In this work, we propose a practical BO method of function compositions where the form of the composition is known but the constituent functions are expensive to evaluate. By assuming an independent Gaussian process (GP) model for each of the constituent black-box function, we propose Expected Improvement (EI) and Upper Confidence Bound (UCB) based BO algorithms and demonstrate their ability to outperform not just vanilla BO but also the current state-of-art algorithms. We demonstrate a novel application of the proposed methods to dynamic pricing in revenue management when the underlying demand function is expensive to evaluate.

\keywords{Bayesian optimization \and revenue maximization \and function composition \and dynamic pricing and learning}
\end{abstract}
%
%
%

\section{Introduction}
\label{sec:introduction}
% \begin{itemize}
%     % Diffusion of FL
%     \item {\st{Diffusion of FL}}
%     % Security threats to FL
%     \item {\st{Security threats to FL with particular focus on model poisoning}}
%     % Limitations of existing countermeasures
%     \item {\st{Current countermeasures (e.g., KRUM) and their limitations}}
%     % Proposed method and its advantages
%     \item {\st{Intuitive description of the proposed method and its difference (i.e., advantages) w.r.t. state of the art}}
%     % Main contributions
%     \item {\st{Summary of the main contributions of this work}}
%     % Paper's structure and organization
%     \item {\st{Paper's structure and organization}}
% \end{itemize}

% Diffusion of FL
Recently, {\em federated learning} (FL) has emerged as the leading paradigm for training distributed, large-scale, and privacy-preserving machine learning (ML) systems~\cite{mcmahan2017googleai,mcmahan2017aistats}. 
The core idea of FL is to allow multiple edge clients to collaboratively train a shared, global model without disclosing their local private training data.
%Specifically, an FL system consists of a central server and many edge clients; 
A typical FL round involves the following steps: {\em(i)} the server randomly picks some clients and sends them the current, global model; {\em(ii)} each selected client locally trains its model with its own private data; then, it sends the resulting local model to the server;\footnote{Whenever we refer to global/local model, we mean global/local model {\em parameters}.} {\em(iii)} the server updates the global model by computing an \emph{aggregation function}, usually the average (FedAvg), on the local models received from clients.
% \begin{enumerate}
%     \item[{\em(i)}] the server sends the current, global model to the clients and appoints some of them for training;
%     \item[{\em(ii)}] each selected client locally trains its copy of the global model with its own private data; then, it sends the resulting local model back to the server;\footnote{Whenever we refer to global/local model, we mean global/local model {\em parameters}.}
%     \item[{\em(iii)}] the server updates the global model by computing an \emph{aggregation function} on the local models received from clients (by default, the average, also referred to as FedAvg~\cite{mcmahan2017aistats}).
% \end{enumerate}
This process goes on until the global model converges. %(e.g., after a certain number of rounds or other similar stopping criteria).
%\\
% The advantages of FL over the traditional, centralized learning paradigm are undoubtedly clear in terms of flexibility/scalability (clients can join/disconnect from the FL network dynamically), network communications (only model weights\footnote{We will use \textit{parameters} and \textit{weights} interchangeably.} are exchanged between clients and server), and privacy (each client's private training data is kept local at the client's end and not uploaded to the server).
\\
% Security threats to FL
%However, the growing adoption of FL also raises security concerns~\cite{costa2022covert}, particularly about its confidentiality, integrity, and availability.
Although its advantages over standard ML, FL also raises security concerns~\cite{costa2022covert}. %, particularly about its confidentiality, integrity, and availability~\cite{costa2022covert}.
% OLD, LONG VERSION
% Indeed, some work deals with privacy leakage that may expose the local data of some clients~\cite{melis2019sp}. 
% A large body of work, instead, investigates attacks that usually aim to detriment the predictive accuracy of the learned global model. For instance, \emph{data poisoning} attacks achieve this goal by letting an adversary pollute the training set of some corrupt FL clients with maliciously crafted examples~\cite{jagielski2018sp}.
% Similarly, in \emph{model poisoning} the attacker attempts to tweak the global model weights~\cite{bhagoji2019pmlr} by directly perturbing the local model's weights of some infected FL clients before these are sent to the central server for aggregation, usually via so-called Byzantine attacks. 
% It turns out that Byzantine model poisoning attacks severely impact standard FedAvg; therefore, more robust aggregation functions must be designed to make FL systems secure.
Here, we focus on \emph{untargeted model poisoning} attacks~\cite{bhagoji2019pmlr}, where an adversary attempts to tweak the global model weights %\footnote{We will use the terms \textit{parameters} and \textit{weights} interchangeably.} 
by directly perturbing the local model's parameters of some infected clients before these are sent to the central server for aggregation.
In doing so, the adversary aims to jeopardize the global model \textit{indiscriminately} at inference time.
Such model poisoning attacks severely impact standard FedAvg; therefore, more robust aggregation functions must be designed to secure FL systems.
\\
% In this paper, we focus on designing a novel robust aggregation scheme at the server's end to contrast the effect of Byzantine model poisoning attacks.
%
% Current countermeasures and their limitations
%Several countermeasures have been proposed in the literature to combat model poisoning attacks on FL systems.
% Some methods use simple statistics more robust than plain average to smooth the impact of malicious updates (e.g., Trimmed Mean and FedMedian~\cite{yin2018icml}). 
% Other defenses implement outlier detection techniques to discard malicious updates from the aggregation performed at the server's end. Those are either based on heuristics (e.g., Krum/Multi-Krum~\cite{blanchard2017nips} and Bulyan~\cite{mhamdi2018pmlr}) or data-driven approaches (e.g., K-means clustering~\cite{shen2016acm} or DnC via spectral analysis~\cite{shejwalkar2021ndss}). 
% Finally, some strategies rely on a centralized ``source of trust'' to spot potential malicious updates (e.g., FLTrust~\cite{cao2020fltrust}).
% Several countermeasures have been proposed in the literature to combat model poisoning attacks on FL systems, i.e., to discard possible malicious local updates from the aggregation performed at the server's end. 
% These techniques range from simple statistics more robust than plain average (e.g., Trimmed Mean and FedMedian~\cite{yin2018icml}) to outlier detection heuristics (e.g., Krum/Multi-Krum~\cite{blanchard2017nips} and Bulyan~\cite{mhamdi2018pmlr}) or data-driven approaches (e.g., spectral analysis via K-means clustering~\cite{shen2016acm} or spectral analysis), or methods based on ``source of trust'' (e.g., FLTrust~\cite{cao2020fltrust}).
% OLD, LONG VERSION
%Several countermeasures have been proposed in the literature to combat Byzantine model poisoning attacks on FL systems.
% Descriptive statistics
% For example, Trimmed Mean and FedMedian aggregate local model updates using more robust statistics than standard average~\cite{yin2018icml}.
%
% % Heuristics for outlier detection
% Many existing Byzantine-resilient strategies implement some outlier detection heuristics to discard the model updates sent by potentially malicious clients from the input of the aggregation function.
% One of the most popular heuristics is Krum~\cite{blanchard2017nips}.
% This strategy tries to mitigate the impact of Byzantine attacks by selecting as a global model the local model with the smallest sum of Euclidean distances to {\em all} the other local models.
% Although powerful, Krum requires the server to know (or, at least, estimate) the number of malicious FL clients upfront, which is generally impossible in a realistic attack scenario. %
% Moreover, Krum may become ineffective for complex, high-dimensional model parameter spaces due to the curse of dimensionality.
% Bulyan~\cite{mhamdi2018pmlr} tries to overcome this issue by combining Krum with a variant of Trimmed Mean.
% % Data-driven outlier detection
% Other strategies use data-driven outlier detection techniques -- e.g., via K-means clustering~\cite{shen2016acm} -- to spot potential malicious local model updates. 
% %For instance, Shen et al. propose to cluster local model updates with K-means and thus identify outliers.
%
% % Other techniques
% As far as the server is concerned, any local model received can be from a potential malicious client. 
% FLTrust~\cite{cao2020fltrust} assumes the server acts as a client, i.e., trains a local model on an additional {\em trustworthy} dataset at the server's end and compares it against all the local models from other clients. 
% This way, the server can rely on some ``source of trust'' when discarding potentially malicious clients.
%\\
% Limitations of existing Byzantine-resilient strategies
Unfortunately, existing defense mechanisms either rely on simple heuristics (e.g., Trimmed Mean and FedMedian by~\cite{yin2018icml}) or need strong and unrealistic assumptions to work effectively (e.g., foreknowledge or estimation of the number of malicious clients in the FL system, as for Krum/Multi-Krum~\cite{blanchard2017nips} and Bulyan~\cite{mhamdi2018pmlr}, which, however, cannot exceed a fixed threshold).
Furthermore, outlier detection methods using K-means clustering~\cite{shen2016acm} or spectral analysis like DnC~\cite{shejwalkar2021ndss} do not directly consider the temporal evolution of local model updates received.
Finally, strategies like FLTrust~\cite{cao2020fltrust} require the server to collect its own dataset and act as a proper client, thereby altering the standard FL protocol.
\\
% OLD, LONG VERSION
% Overall, existing Byzantine-resilient strategies are either simple heuristics (e.g., FedMedian) or, if they are more complex, they rely on strong and unrealistic assumptions to work effectively (e.g., knowing the number of malicious clients in the FL system in advance, as for Krum and alike).
% Furthermore, data-driven outlier detection methods do not consider the temporary evolution of local model updates received (e.g., K-means clustering). 
% Finally, strategies like FLTrust requires the server to collect its own dataset and act as a proper client, thereby altering the standard FL protocol.
%
% Description of the proposed method
This work introduces a novel pre-aggregation \textit{filter} robust to untargeted model poisoning attacks. Notably, this filter $(i)$ operates without requiring prior knowledge or constraints on the number of malicious clients and $(ii)$ inherently integrates temporal dependencies. 
The FL server can employ this filter as a preprocessing step before applying \textit{any} aggregation function, be it standard like FedAvg or robust like Krum or Bulyan.
Specifically, we formulate the problem of identifying corrupted updates as a multidimensional (i.e., matrix-valued) time series anomaly detection task. 
The key idea is that legitimate local updates, resulting from well-calibrated iterative procedures like stochastic gradient descent (SGD) with an appropriate learning rate, show \textit{higher predictability} compared to malicious updates. This hypothesis stems from the fact that the sequence of gradients (thus, model parameters) observed during legitimate training exhibit regular patterns, as validated in Section~\ref{subsec:intuition}. %until convergence. 
%This regularity may be more pronounced for smooth convex loss functions, but it can still be captured within an appropriate time window, even for more complex and convoluted loss surfaces. 
%We provide evidence of this claim in Appendix~B, where we show that the average mutual information (i.e., ``predictability''), calculated over pairs of legitimate model updates sent at different FL rounds, is significantly higher than the corresponding computation for a malicious client.
\\
Inspired by the matrix autoregressive (MAR) framework for multidimensional time series forecasting~\cite{chen2021je}, we propose the FLANDERS ({\em \textbf{F}ederated \textbf{L}earning meets \textbf{AN}omaly \textbf{DE}tection for a \textbf{R}obust and \textbf{S}ecure}) filter.
The main advantages of FLANDERS over existing strategies like FLDetector~\cite{zhao2020multivariate} are its resilience to large-scale attacks, where $50\%$ or more FL participants are hostile, and the capability of working under realistic non-iid scenarios.
We attribute such a capability to two key factors: $(i)$ FLANDERS works without knowing a priori the ratio of corrupted clients, and $(ii)$ it embodies temporal dependencies between intra- and inter-client updates, quickly recognizing local model drifts caused by evil players. Below, we summarize our main contributions:

\begin{itemize}
\item[{\em(i)}]
We provide empirical evidence that the sequence of models sent by legitimate clients is more predictable than those of malicious participants performing untargeted model poisoning attacks.
\\
\item[{\em(ii)}] 
We introduce FLANDERS, the first pre-aggregation filter for FL robust to untargeted model poisoning based on multidimensional time series anomaly detection.
\\
\item[{\em(iii)}] 
We integrate FLANDERS into Flower,\footnote{\scriptsize{\url{https://flower.dev/}}} a popular FL simulation framework for reproducibility.
\\
\item[{\em(iv)}] 
We show that FLANDERS improves the robustness of the existing aggregation methods under multiple settings: different datasets, client's data distribution (non-iid), models, and attack scenarios.
\\
\item[{\em(v)}] 
We publicly release all the implementation code of FLANDERS along with our experiments.\footnote{\scriptsize{\url{https://anonymous.4open.science/r/flanders_exp-7EEB}}}
\end{itemize}

% Paper's structure and organization
The remainder of the paper is structured as follows. %some related work and the current state-of-the-art solutions to security issues that FL entails. 
Section~\ref{sec:background} covers background and preliminaries. 
In Section~\ref{sec:related}, we discuss related work.
Section~\ref{sec:problem} and Section~\ref{sec:method} describe the problem formulation and the method proposed. % to tackle it. 
Section~\ref{sec:experiments} gathers experimental results. %, and Section~\ref{sec:limitations} discusses some limitations of this work.
Finally, we conclude in Section~\ref{sec:conclusion}.
 %discusses the limitations of this work and draws future research directions.
%reports conclusions and draws perspectives for future research directions.

%%%%%%% OLD %%%%%%%
%to overcome the resilience of Byzantine failures in distributed Stochastic Gradient Descent computations. 
% The strength of Krum is its time complexity, which is linear in the gradient dimension. 
% However, the robustness of the approach is guaranteed for gradient-based learning applications only when the majority of the clients are not compromised. 
% Besides, the aggregation mechanism of Krum, as well as that of similar methods, is robust from a coarse-grained perspective and does not provide solutions to errors and perturbations that may occur at inference time.
%A related approach to~\cite{blanchard2017nips} is the work of Su et al.~\cite{su2016dc}. Here, the authors propose an iterated approximate agreement to tackle a multi-layer scenario attacked by Byzantine agents. 
%However, the method works efficiently on the sole discrete context and it is inapplicable to continuous state environments.
%\gabri{Maybe, we should just talk about the main limitations of existing countermeasures without digging into their details (or, we can just mention Krum as this is the most popular one). I will move the description of all these methods to the Related Work section.}
\section{Problem Description}\label{sec:problem_description}
We begin by describing the problem of BO for composite functions in subsection \ref{sec:desc_BO}. In subsection \ref{sec:demand_model} we describe the dynamic pricing problem and model the revenue function as a function composition to which BO methods for composite functions can be applied. 
\subsection{BO for Function Composition}\label{sec:desc_BO}
We consider the problem of optimizing $g(\textbf{x}) = h(f_1(\textbf{x}), f_2(\textbf{x}), \ldots , f_M(\textbf{x}))$ where $g : \mathcal{X} \rightarrow \mathrm{R}$, $f_i : \mathcal{X} \rightarrow \mathrm{R}$, $h:\mathrm{R}^{M} \rightarrow \mathrm{R}$ and $\mathcal{X} \subseteq \mathrm{R}^d$. We assume each $f_i$ is a black-box expensive-to-evaluate continuous function and $h$ is known and cheap to evaluate. The optimization problem that we consider is 
\begin{equation}\label{eq:problemdesc}
  \max_{\textbf{x} \in \mathcal{X}} ~h(f_1(\textbf{x}), f_2(\textbf{x}), \ldots, f_M(\textbf{x})). 
\end{equation}
We want to solve Problem~\ref{eq:problemdesc} in an iterative manner where in the $n^{\text{th}}$ iteration, we can use the previous observations $\{\textbf{x}_i, f_1(\textbf{x}_i), \ldots, f_M(\textbf{x}_i\})\}^{n-1}_{i=1}$ to request a new observation $\{\textbf{x}_n, f_1(\textbf{x}_1), \ldots, f_M(\textbf{x}_n)\}$.

A vanilla BO algorithm applied to this problem would first assume a prior GP model on $g$, denoted by $\mathcal{GP}(\mu(\cdot), K(\cdot, \cdot))$ where $\mu$ and $K$ denote the mean and covariance function of the prior model. Given some function evaluations, an updated posterior GP model is obtained. A suitable acquisition function, such as EI or PI can be used, to identify the next query point. For example, in the $n+1^{\text{th}}$ update round, one would first use the $n$ available observations $(g(\textbf{x}_1), g(\textbf{x}_2), \ldots, g(\textbf{x}_n))$ to update the GP model to  $\mathcal{GP}(\mu^{(n)}(\cdot), K^{(n)}(\cdot, \cdot))$   where $\mu^{(n)}(\cdot)$ is the posterior mean function and $K^{(n)}(\cdot, \cdot)$ is the posterior covariance function, see \cite{Rasmussen2005} for more details. The acquisition function then uses this posterior model to identify the next query location $\textbf{x}_{n+1}$. In doing so, vanilla BO ignores the values of the member functions in the composition $h$.
%while building a posterior.

BO for composite function, on the other hand, takes advantage of the available information about $h$, and its easy-to-compute nature. Astudillo and Frazier~\cite{https://doi.org/10.48550/arxiv.1906.01537} model the constituent functions of the composition by a single multi-output function $\textbf{f}(\textbf{x}) = (f_1(\textbf{x}), \ldots, f_M(\textbf{x}))$ and then model the uncertainty in $\textbf{f}(\textbf{x})$ using a multi-output Gaussian process to optimize $h({\textbf{f}(\textbf{x})})$. Since the prior over $f$ is modelled as a MOGP, the proposed method tries to capture the correlations between different components of the multi-output function $\textbf{f}(\textbf{x})$. Note that the proposed EI and PI-based acquisition functions are required to be computed using Monte Carlo sampling. Furthermore, a sample from the posterior distribution is obtained by first sampling an $n$ variate normal distribution, then scaling it by the lower Cholesky factor and then centering it with the mean of the posterior GP. Two problems arise due to this: \begin{enumerate*} 
    \item Such simulation based averaging approach increases the time complexity of the procedure  linearly with the number of samples taken for averaging and
    \item calculation of the lower Cholesky factor increases the function's time complexity cubically with the number of data points. 
\end{enumerate*}
These factors render the algorithm unsuitable, particularly for problems with large number of member functions or for problems with large dimensions.


% \textbf{Expand on this, why sampling and inversion is needed, describe their acquisiton in 1-2 lines, enumerate drawbacks of astuldo work, 1) computationally expensive 2) inversion 3) 1&2 make it slow and not suitable for large dimension problems  }


To alleviate these problems, in this work, we model the constituent functions using independent GPs. This modelling approach allows us to train GPs for each output independently and hence the posterior GP update can be parallelized. We propose two acquisition functions, cEI which is based on the EI algorithm and cUCB, which is based on the GP-UCB algorithm \cite{Srinivas_2012}. Our cEI acquisition function is similar in spirit to the EI-CF acquisition function of \cite{https://doi.org/10.48550/arxiv.1906.01537} but is less computationally intensive owing to the independent GP model. Since we have independent one dimensional GP model for each constituent function, sampling points from the posterior GP does not require computing the Cholesky factor (and hence matrix inversion), something that is needed in the case of high-dimensional GP's of \cite{https://doi.org/10.48550/arxiv.1906.01537}.  
This greatly reduces the complexity of the MC sampling steps of our algorithm (see section \ref{sec:our_approach} for more details). However, the cEI acquisition function still suffers from the drawback of requiring Monte Carlo averaging. 
To alleviate this problem, we propose a UCB based acquisition function that uses the current mean plus scaled variance of the posterior GP at a point as a surrogate for the constituent function at that point. As shown by Srinivas et al.~\cite{Srinivas_2012}, while the mean term in the surrogate guides exploitation, it is the variance of the posterior GP at a point that allows for suitable exploration. The scaling of the variance term is controlled in such a way that it balances the trade off between exploration and exploitation. In Section~\ref{sec:experiments}, we illustrate the utility of our method, first for standard test functions and then as an application to dynamic pricing problem. Our algorithms, especially the cUCB one, outperforms not only vanilla BO but also those proposed in Astudillo and Frazier~\cite{https://doi.org/10.48550/arxiv.1906.01537}.

%and does not require matrix inversion, resulting in the complexity for computing the variance for each function reducing from cubic to constant time. Our UCB variant helps us eliminate Monte Carlo sampling while computing the acquisition function. 
% \textbf{describe 1-2 lines what are the main advantages of doing this over atuldo}. In this work, we also propose a a UCB-style algorithm which helps us reduce the computational complexity of the process.
% \textbf{Refer to GP UCB paper here, again say in detail how the UCB offers benefits, other than not requiring matrix inversion are there more benefits ? The sigma term in UCB allows for better exploration?} Our method does not require the computation of an inverse matrix. 

% \textbf{Vanilla BO}
%  One possible approach to solve this is using a vanilla Bayesian Optimization of $g(x)$, we assume $g$ to be drawn from a GP prior probability distribution. It would use an acquisition function like EI to optimize the GP prior over $f$ and ignore the observations on $f_i$ in the process.

% \textbf{Astuldo}
% We want to solve Problem~\ref{eq:problemdesc} in an iterative manner. In the $k^{th}$ iteration, we can use the previous observations $\{x_i, f_1(x_i), \ldots, f_N(x_N)\}^{k-1}_{i=1}$ to request a new observation $\{x_k, f_1(x_k), \ldots, f_N(x_k)\}$.

% \textbf{Drawbacks of Astuldo}
% Multi output GP, computationally intensive, acquisition functions require matrix inversion etc 

% \textbf{Motivation to our algorithms }
% Independent GP, UCB type acquisition function that does not need inversion. Performance gains we see in terms of run time.
% more details section 3

% \textbf{Move this para to vani9lla BO, add mathematical details, see prabu paper or fobo paper}
% While traditional BO would model a GP over the function $g$ as described in Section~\ref{sec:problem_description} and ignore the values of the member functions in the composition of $h$ while building a prior, our idea is to take advantage of the information we have about the cheap to evaluate function, $h$. Traditional BO would then look to use standard acquisition functions such as Expected Improvement (EI) or Probability of Improvement (PI) to find the point at which the model predicts the best results should be. That point would be evaluated next and the result would be used to update the GP and suggest the next point to evaluate. This method works well with standard functions but would ignore the domain information we have about $h$ when optimizing a composite function.


\subsection{Bayesian Optimization for dynamic pricing}\label{sec:demand_model}
We consider Bayesian Optimization for two types of revenue optimization problems. The first problem optimizes the revenue per customer where customers are characterized by their willingness-to-pay distribution (which is unknown). In the second problem, we assume a parametric demand model (the functional form is assumed to be unknown) and optimizes the associated revenue. 

In the first model, we assume that an arriving customer has an associated random variable, $V$, with complimentary cumulative distribution function $\bar{F}$, indicating its maximum willingness to pay for the item sold. For an item on offer at a price $p$, an arriving customer purchases it with the probability 
\begin{equation}\label{eq:fp}
    d(p):= \bar{F}(p) = \text{Pr}\{V \geq p\}  .
\end{equation}
In this case, when the product is on offer at a price $p$, the revenue per customer $r(p)$ is given by $r(p) = p \bar{F}(p)$. The revenue function is a composition of the price and demand or purchase probability and we assume that the distribution of the purchase probability i.e., $F$ is not known and also expensive to estimate. One could perform a vanilla BO algorithm by having a GP model on $r(p)$ itself. However to exploit the known nature of the revenue function, we will apply our function composition method by instead having a GP on  $\bar{F}(p)$ and demonstrate its superiority over vanilla BO.  

In the second model, we assume that the true demand $d(p)$ for a commodity at price $p$ has a functional form. This forms the ground truth model that governs the demand, but we assume that the functional form for this demand is not known to the manager optimizing the revenue. In our experiments, we assume linear, logit, Booth and Matyas functional forms for the demand (see section \ref{sec:experiments} for more details.)
Along similar lines, one could build more sophisticated demand models to account for external factors (such as supply chain issues, customer demographics or inventory variables), something that we leave for future explorations. 
%in the model by modelling demand as a parametric function of price
%\begin{equation}
 %   d(p, \boldsymbol{z}):= \bar{F}_{\boldsymbol{z}}(p) = \text{Pr}\{V \geq p\}
%\end{equation}
%where $z$ is the function parameter over $\bar{F}$ which can evolve over time with changing conditions.

Note that we make some simplifying assumptions about the retail environment in these two models and our experiments. We assume a non-competitive monopolistic market with an unlimited supply of the product and no marginal cost of production. However, these assumptions can easily be relaxed by changing the ground truth demand model appropriately, which are used in the experiments to reflect these aspects. The fact that we use a GP model as a surrogate for the unknown demand model offers it the ability to model a diverse class of demand functions under diverse problem settings. We do not discuss these aspects further but focus on the following simple yet meaningful experimental examples that one typically encounters in revenue management problems.  
%We also assume that the price for a product can be adjusted instantaneously and that customer feedback is received in real time. Moreover, we also assume that it is feasible to quote different prices to the customer simultaneously as well and expose them to customers with identical natures. 
%We now describe the experimental setups to which we apply our composite BO algorithms. 
In the following, $\textbf{p}$ denotes the price vector:
\begin{enumerate}
    \item \textbf{Independent demand model:} A retailer supplies its product to two different regions whose customer markets behave independently from each other. Thus, the same product has independent and different demand functions (and hence different optimal prices) in different geographical regions and under such black-box demand models for the two regions $(d_1, d_2)$. The retailer is interested in finding the optimal prices, leading to the optimization of the following function: $g(\textbf{p}) = p_1d_1(p_1) + p_2d_2(p_2)$. 
    \item \textbf{Correlated demand model:} Assume that a retailer supplies two products at prices $p_1$ and $p_2$ and the demand for the two products is correlated and influenced by the price for the other product. 
    %product to two regions, but their customer market does not behave independently from each other, leading to the objective function 
    Such a scenario can be modelled by a revenue function of the form $g(\textbf{p}) = p_1d_1(\textbf{p}) + p_2d_2(\textbf{p})$. Consider the  example where the prices of business and economy class tickets can influence the demand in each segment. Similarly, the demand for a particular dish in a fast food chain might be influenced by the prices for other dishes.
    \item \textbf{Identical price model: } In this case, the retailer is compliant with having a uniform price across locations. However, the demand function across different locations could be independent at each of these locations, leading to the following objective function: $g(p) = pd_1(p) + pd_2(p)$. This scenario can be used to model different demand functions for different population segments in their age, gender, socio-economic background, etc. 
\end{enumerate}
%Note that the first scenario is a special case of the second scenario where function $d_i$ is only dependent on $p_i$.

\subsection{Approach}
This paper describes three concepts with the aim 
of a robust, energy-efficient robot control. 
While these concepts are rather straightforward and intuitive, they 
are not yet utilised in mainstream manipulator control.
It is not argued that all of these principles 
need to be used, but if the (sub)task allows it, using any 
of these principles can have a positive impact on the energy-efficiency.

\vspace{-4mm}
\subsubsection{Contextual prior knowledge:}
When humans perform a transportation task, they do  
not perform strict PTP motions such as traditional industrial 
robots. Instead, movements with a certain tolerance on the position 
are performed. 
This allows the natural dynamics of the system to be exploited, 
as will be explained in the following subsection. Typically, the 
tolerances come from 
knowledge about both the environment and the task context. For example, 
the spatial constraints, 
fragility of the payload, 
if a certain part of the task requires a higher precision, 
etc. 
It is clear that this knowledge precedes the task execution and 
determines how the human will perform the task.
The execution is generally done in multiple states, e.g., picking up 
the payload, moving and placing near the target position, making small 
adjustments when necessary.

This knowledge is used to split up the task in multiple 
subtasks and identify the different requirements. Robust 
controllers and monitors are then developed to perform and coordinate 
between these subtasks. 
Examples of such requirements are crane like operations such as:
 lifting the load to a certain 
height, transporting it without colliding, and lowering the load 
until contact is made.

The task also does not require high control precision throughout, 
but only for the initial grasping and final placement. 
In addition, this does not need to come only from the 
control.
Geometric constraints such as the environment or a previously placed
payload can be used to achieve this accuracy by sliding against them. 
This is further explained in section \ref{sec:discrete_control}.

By using this knowledge, lower-cost (and often also lower-weight) 
hardware can be used, so that a more robust, 
energy efficient execution can be developed. 
Thus, for a repetitive task, the cost of 
designing and implementing a task-specific controller is not 
necessarily higher than a generic, less energy-efficient controller.\\

\vspace{-8mm}
\subsubsection{Exploiting natural dynamics}
In this work, the natural dynamics of the system are used to inject 
as little energy as possible, resulting in energy-efficient 
motions.
However, precise control of the timing is lost when the system freely
follows its natural dynamics.

Due to the layout of the used cable robot (Fig.1), when the end effector is
in a fully constrained position, releasing the power of one (or more) 
of the motors, will result in a pendulum-like swing around the 
cables that are still powered, or braked. 
This swing is used in the control strategy to cover the horizontal 
distance while consuming a minimal amount of energy. 

\vspace{-4mm}
\subsubsection{Active use of brakes}
Based on the context, certain subtasks may occur where a joint 
does not need to move. Instead of producing a constant standstill
torque, it can also be opted to brake the joint.
Another case occurs when the demanded motion is in line with 
external forces such as gravity. In case of a continuous brake,
the brake force can be directly controlled to achieve a certain 
resulting force. 
With a discrete brake, a tolerance region can be determined between  
which the brake switches on-and-off to achieve a similar effect.
Section \ref{sec:continuous_control} utilises this concept to drop the
payload without driving the motor. The brakes can also be used to stop 
the natural dynamics, if necessary.
\section{Experimental Results}
\label{sec:experimental_results}

This section describes the experimental validations on the effectiveness and reliability of \ourmodel. First, we describe the model setup in Sec.~\ref{sec:experiment_setups}. Sec.~\ref{sec:single_attr_diagnosis} and Sec.~\ref{sec:validation_diagnosis} visualize and validate the model diagnosis results for the single-attribute setting. In Sec.~\ref{sec:multiple_attr_diagnosis}, we show results on synthesized multiple-attribute counterfactual images and apply them to counterfactual training.

\subsection{Model Setup}
\label{sec:experiment_setups}
{\bf Pre-trained models:} We used Stylegan2-ADA \cite{Karras2020ada} pretrained on FFHQ \cite{2019stylegan} and AFHQ \cite{choi2020starganv2} as our base generative networks, and the pre-trained CLIP model \cite{CLIP}  which is parameterized by ViT-B/32. We followed StyleCLIP \cite{2021StyleCLIP} setups to compute the channel relevance matrices $\mathcal{M}$.

{\bf Target models:} Our classifier models are ResNet50 with single fully-connected head initialized by TorchVision\footnote{https://pytorch.org/blog/how-to-train-state-of-the-art-models-using-torchvision-latest-primitives/}. In training the binary classifiers, we use the Adam optimizer with learning rate 0.001 and batch size 128. We train binary classifiers for \textit{Eyeglasses, Perceived Gender, Mustache, Perceived Age} attributes on CelebA and for \textit{cat/dog} classification on AFHQ. For the 98-keypoint detectors, we used the HR-Net architecture~\cite{WangSCJDZLMTWLX19} on WFLW~\cite{wayne2018lab}. %Unless explicitly mentioned, our approach samples 1000 images from StyleGAN for each diagnosis by histogram.

\subsection{Visual Model Diagnosis: Single-Attribute}
\label{sec:single_attr_diagnosis}
Understanding where deep learning model fails is
an essential step towards building trustworthy models. Our proposed work allows us to generate counterfactual images (Sec.~\ref{sec:Counterfactual_Synthesis}) and provide insights on sensitivities of the target model (Sec.~\ref{sec:Attribute_Sensitivity_Analysis}). This section visualizes the counterfactual images in which only one attribute is modified at a time. 

Fig. \ref{fig:age_classifier_single} shows the single-attribute counterfactual images. Interestingly (but not unexpectedly), 
we can see that reducing the hair length or joyfulness causes the age classifier more likely to label the face to an older person. Note that our approach is extendable to multiple domains, as we change the cat-like pupil to dog-like, the dog-cat classification tends towards the dog. 
Using the counterfactual images, we can conduct model diagnosis with the method mentioned in Sec.~\ref{sec:Attribute_Sensitivity_Analysis}, on which attributes the model is sensitive to. In the histogram generated in model diagnosis, a higher bar means the model is more sensitive toward the corresponding attribute.

Fig.~\ref{fig:histograms_vanilla} shows the model diagnosis histograms on regularly-trained classifiers. For instance, the cat/dog classifier histogram shows outstanding sensitivity to green eyes and vertical pupil.
The analysis is intuitive since these are cat-biased traits rarely observed in dog photos. Moreover, the histogram of eyeglasses classifier shows that the mutation on bushy eyebrows is more influential for flipping the model prediction. 
It potentially reveals the positional correlation between eyeglasses and bushy eyebrows. The advantage of single-attribute model diagnosis is that the score of each attribute in the histogram are independent from other attributes, enabling unambiguous understanding of exact semantics that make the model fail. Diagnosis results for additional target models can be found in Appendix B.

\subsection{Validation of Visual Model Diagnosis} 
\label{sec:validation_diagnosis}
Evaluating whether our zero-shot sensitivity histograms (Fig.~\ref{fig:histograms}) explain the true vulnerability is a difficult task, since we do not have access to a sufficiently large and balanced test set fully annotated in an open-vocabulary setting. To verify the performance, we create synthetically imbalanced cases where the model bias is known. We then compare our results with a supervised diagnosis setting~\cite{sia}. In addition, we will validate the decoupling of the attributes by CLIP. 

\vspace{-2mm}
\subsubsection{Creating imbalanced classifiers}
\label{sec:creating_imbalance_classifiers}
\vspace{-1mm}
In order to evaluate whether our sensitivity histogram is correct, we train classifiers that are highly imbalanced towards a known attribute and see whether \ourmodel can detect the sensitivity w.r.t the attribute. For instance, when training the perceived-age classifier (binarized as Young in CelebA), we created a dataset on which the trained classifier is strongly sensitive to Bangs (hair over forehead). The custom dataset is a CelebA training subset that consists of $20,200$ images. More specifically, there are $10,000$ images that have both young people that have bangs, represented as $(1,1)$, 
and $10,000$ images of people that are not young and have no bangs, represented as $(0,0)$. The remaining combinations of $(1,0)$ and $(0,1)$ have only 100 images.
With this imbalanced dataset, bangs is the attribute that dominantly correlates with whether the person is young, and hence the perceived-age classifier would be highly sensitive towards bangs.
% will learn that bangs is the most sensitive attribute to predict age. 
See Fig.~\ref{fig:histogram_attgan} (the right histograms) for an illustration of the sensitivity histogram computed by our method for the case of an age classifier with bangs (top) and lipstick (bottom) being imbalanced. 
\begin{figure}[t]
    \begin{subfigure}[b]{\linewidth}
        \label{fig:histogram_attgan_1}
         \centering
         \includegraphics[width=\linewidth]{images/histograms/attgan_histogram_1.pdf}\\
    \end{subfigure}
    \begin{subfigure}[b]{\linewidth}
    \label{fig:histogram_attgan_2}
         \includegraphics[width=\linewidth]{images/histograms/attgan_histogram_2.pdf}
    \end{subfigure}
        \vspace{-6mm}
         \caption{ The sensitivity of the age classifier is evaluated with \ourmodel (right) and AttGAN (left), achieving comparable results. }
         \label{fig:histogram_attgan}
         \vspace{-1mm}
    %  \end{subfigure}
\end{figure}

 We trained multiple imbalanced classifiers with this methodology,  and visualize the model diagnosis histograms of these imbalanced classifiers in Fig.~\ref{fig:histograms_unbalanced}. We can observe that the \ourmodel histograms successfully detect the synthetically-made bias, which are shown as the highest bars in histograms. See the caption for more information. 

\begin{figure}[t]
    \begin{subfigure}[b]{0.49\linewidth}
        \centering
        \includegraphics[width=\linewidth]{images/matrix/confusion-matrix-Mustache.pdf}
        \caption{Mustache classifier}
        \label{fig:matrix_CLIP_Score_a}
    \end{subfigure}
    \begin{subfigure}[b]{0.49\linewidth}
        \centering
        \includegraphics[width=\linewidth]{images/matrix/confusion-matrix-Young.pdf}
        \caption{Perceived age classifier}
        \label{fig:matrix_CLIP_Score_b}
    \end{subfigure}
    \vspace{-2mm}
    \caption{Confusion matrix of CLIP score variation (vertical axis) when perturbing attributes (horizontal axis). This shows that attributes in \ourmodel are highly decoupled. }
    \label{fig:matrix_CLIP_Score}
    \vspace{-3mm}
\end{figure}

\begin{figure*}[ht]
    \centering
    \includegraphics[width=\linewidth]{images/multi_attr_human.pdf}
    \caption{Multi-attribute counterfactual in faces. The model probability is documented in the upper right corner of each image.}
    \label{fig:human_classifier_multiattr}
    \vspace{-4mm}
\end{figure*}

\vspace{-2mm}
\subsubsection{Comparison with supervised diagnosis}
\vspace{-1mm}
We also validated our histogram by comparing it with the case in which we have access to a generative model that has been explicitly trained to disentangle attributes.  We follow the work on~\cite{sia} and used AttGAN~\cite{attGAN} trained on the CelebA training set over $15$ attributes\footnote{\textit{Bald, Bangs, Black\_Hair, Blond\_Hair, Brown\_Hair, Bushy\_Eyebrows, Eyeglasses, Male, Mouth\_Slightly\_Open, Mustache, No\_Beard, Pale\_Skin, Young, Smiling, Wearing\_Lipstick.}}.
After the training converged, we performed adversarial learning in the attribute space of AttGAN and create a sensitivity histogram using the same approach as Sec.~\ref{sec:Attribute_Sensitivity_Analysis}. Fig.~\ref{fig:histogram_attgan} shows the result of this method on the perceived-age classifier which is made biased towards bangs.  As anticipated, the AttGAN histogram (left) corroborates the histogram derived from our method (right). Interestingly, unlike \ourmodel, AttGAN show less sensitivity to remaining attributes. This is likely because AttGAN has a latent space learned in a supervised manner and hence attributes are better disentangled than with StyleGAN. Note that AttGAN is trained with a fixed set of attributes; if a new attribute of interest is introduced, the dataset needs to be re-labeled and AttGAN retrained. ZOOM, however, merely calls for the addition of a new text prompt.  More results in Appendix B.

\vspace{-2mm}
\subsubsection{Measuring disentanglement of attributes}
\vspace{-1mm}
Previous works demonstrated that the StyleGAN's latent space can be entangled~\cite{interfacegan, EditinginStyle}, adding undesired dependencies when searching single-attribute counterfactuals. This section verifies that our framework can disentangle the attributes and mostly edit the desirable attributes.

We use CLIP as a super annotator to measure attribute changes during single-attribute modifications. For $1,000$ images, we record the attribute change after performing adversarial learning in each attribute, and average the attribute score change. Fig.~\ref{fig:matrix_CLIP_Score} shows the confusion matrix during single-attribute counterfactual synthesis. The horizontal axis is the attribute being edited during the optimization, and the vertical axis represents the CLIP prediction changed by the process. For instance, the first column of Fig.~\ref{fig:matrix_CLIP_Score_a} is generated when we optimize over bangs for the mustache classifier. We record the CLIP prediction variation. It clearly shows that bangs is the dominant attribute changing during the optimization. From the main diagonal of matrices, it is evident that the \ourmodel mostly perturbs the attribute of interest. The results indicate reasonable disentanglement among attributes.



\subsection{Visual Model Diagnosis: Multi-Attributes}
\label{sec:multiple_attr_diagnosis}
In the previous sections, we have visualized and validated single-attribute model diagnosis histograms and counterfactual images. 
In this section, we will assess \ourmodel's ability to produce counterfactual images by concurrently exploring multiple attributes within $\mathcal{A}$, the domain of user-defined attributes.  The approach conducts multi-attribute counterfactual searches across various edit directions, identifying distinct semantic combinations that result in the target model's failure. By doing so, we can effectively create more powerful counterfactuals images (see Fig.~\ref{fig:multiple_attribute_is_more_powerful}).


\begin{figure}[t]
    \centering
    \includegraphics[width=\linewidth]{images/multi_attr_dog_cut.pdf}
    \caption{Multi-attribute counterfactual on Cat/Dog classifier. The number in each image is the predicted probability of being a dog.}
    \label{fig:dog_classifier_multiattr}
    \vspace{-2mm}
\end{figure}

\begin{figure}[t]
    \centering
    \includegraphics[width=\linewidth]{images/multi_attr_is_more_powerful/multi_attr_is_more_powerful_2.pdf}\\
    \vspace{-1mm}
    \includegraphics[width=\linewidth]{images/multi_attr_is_more_powerful/multi_attr_is_more_powerful_1.pdf}
    \vspace{-8mm}
    \caption{ Multiple-Attribute Counterfactual (MAC, Sec.~\ref{sec:multiple_attr_diagnosis}) compared with Single-Attribute Counterfactual (SAC, Sec.~\ref{sec:single_attr_diagnosis}). We can see that optimization along multiple directions enable the generation of more powerful counterfactuals.}
    \label{fig:multiple_attribute_is_more_powerful}
    \vspace{-4mm}
\end{figure}

Fig.~\ref{fig:human_classifier_multiattr} and Fig.~\ref{fig:dog_classifier_multiattr} show examples of multi-attribute counterfactual
images generated by \ourmodel, against human and animal face classifiers. 
It can be observed that multiple face attributes such as lipsticks or hair color are edited in Fig.~\ref{fig:human_classifier_multiattr}, and various cat/dog attributes like nose pinkness, eye shape, and fur patterns are edited in Fig.~\ref{fig:dog_classifier_multiattr}. 
These attribute edits are blended to affect the target model prediction. Appendix B further illustrates \ourmodel counterfactual images for semantic segmentation, multi-class classification, and a church classifier. By mutating semantic representations, \ourmodel reveals semantic combinations as outliers where the target model underfits.


In the following sections, we 
will use the Flip Rate (the percentage of counterfactuals that flipped the model prediction) and Flip Resistance (the percentage of counterfactuals for which the model successfully withheld its prediction) to evaluate the multi-attribute setting. 
\begin{figure}[t]
    \centering
    \begin{subfigure}[b]{\linewidth}
    \includegraphics[width=0.495\linewidth]{images/histograms/multi_attr_eyeglasses.pdf}
    \includegraphics[width=0.495\linewidth]{images/histograms/multi_attr_age_biased_beard.pdf}
    \caption{Sensitivity histograms generated by \ourmodel on attribute combinations.}
    \label{fig:histograms_combination}
    \end{subfigure}\\
    \begin{subfigure}[b]{\linewidth}
    \includegraphics[width=\linewidth]{images/histograms/grand_histogram.pdf}
    \caption{Model diagnosis by \ourmodel over $19$ attributes. Our framework is generalizable to analyze facial attributes of various domains.}
    \label{fig:histograms_grand}
    \end{subfigure}
    \vspace{-6mm}
    \caption{Customizing attribute space for \ourmodel.}
    \label{fig:multiple_attribute_histogram}
    \vspace{-4mm}
\end{figure}
\vspace{-3mm}
\subsubsection{Customizing attribute space}
\vspace{-2mm}
\looseness=-1

In some circumstances,  users may finish one round of model diagnosis and proceed to another round by adding new attributes, or trying a new attribute space.
The linear nature of attribute editing (Eq.~\ref{eq:total_edit}) in \ourmodel makes it possible to easily add or remove attributes. 
Table~\ref{tab:model_flip_rate} shows the flip rates results when adding new attributes into $\mathcal{A}$ for perceived age classifier and big lips classifier.  We can observe that a different attribute space will results in different effectiveness of counterfactual images. Also, increasing the search iteration will make counterfactual more effective (see last row). 
 Note that neither re-training the StyleGAN nor user-collection/labeling of data is required at any point in this procedure.  Moreover, Fig.~\ref{fig:histograms_combination} shows the model diagnosis histograms generated with combinations of two attributes. Fig.~\ref{fig:histograms_grand} demonstrates the capability of \ourmodel in a rich vocabulary setting where we can analyze attributes that are not labeled in existing datasets~\cite{liu2015celeba,MAAD}.
 
\vspace{-4mm}
\subsubsection{Counterfactual training results}
\label{sec:ct_result}
\vspace{-1mm}


This section evaluates regular classifiers trained on CelebA~\cite{liu2015celeba} and counterfactually-trained (CT) classifiers on a mix of CelebA data and counterfactual images as described in Sec.~\ref{sec:ct}. Table \ref{tab:ct_acc_table} presents accuracy and flip resistance (FR) results. CT outperforms the regular classifier. FR is assessed over 10,000 counterfactual images, with FR-25 and FR-100 denoting Flip Resistance after 25 and 100 optimization iterations, respectively. Both use $\eta=0.2$ and $\epsilon=30$. We can observe that the classifiers after CT are way less likely to be flipped by counterfactual images while maintaining a decent accuracy on the CalebA testset. Our approach robustifies the model by increasing the tolerance toward counterfactuals. Note that CT slightly improves the CelebA classifier when trained on a mixture of CelebA images (original images) and the counterfactual images generated with a generative model  trained in the FFHQ~\cite{2019stylegan} images (different domain).  


\begin{table}[t]
  \centering
  \footnotesize
  \begin{tabular}{@{}lccc@{}}
     \toprule
     Method & \makecell{AC Flip Rate (\%)} & \makecell{BC Flip Rate (\%)} \\
     \midrule
     Initialize \ourmodel by $\mathcal{A}$                        & 61.95 &  83.47\\
     + Attribute: Beard                                           &  72.08 & 90.07\\
     + Attribute: Smiling                                        &  87.47 &  \textbf{96.27}\\
     + Attribute: Lipstick                                         &  90.96 &  94.07\\
     + Iterations increased to 200                                &  \textbf{92.91} &  94.87\\
     \bottomrule
  \end{tabular}
  \caption{\label{tab:model_flip_rate} Model flip rate study. The initial attribute space $\mathcal{A} =$ \{Bangs, Blond Hair, Bushy Eyebrows, Pale Skin, Pointy Nose\}. AC is the perceived age classifier and BC is the big lips classifier.} 
  \vspace{-3mm}
\end{table}


\begin{table}[t]
    \centering
    \footnotesize
    \begin{tabular}{ccccc}
        \toprule
         Attribute & \makecell{Metric} & \makecell{Regular (\%)} & \makecell{CT (Ours) (\%)} \\

\midrule
        \multirow{3}{*}{Perceived Age} & CelebA Accuracy   & 86.10 & \textbf{86.29}   \\
        & \ourmodel FR-25  & 19.54 & \textbf{97.36}  \\
        & \ourmodel FR-100  & 9.04 & \textbf{95.65}  \\
        \midrule
        \multirow{3}{*}{Big Lips} & CelebA Accuracy   & 74.36 & \textbf{75.39}    \\
        & \ourmodel FR-25  & 14.12 & \textbf{99.19}  \\
        & \ourmodel FR-100  & 5.93 & \textbf{88.91}  \\
        \bottomrule
    \end{tabular}
    \caption{\label{tab:ct_acc_table} Results of network inference on CelebA original images and \ourmodel-generated counterfactual. The CT classifier is significantly less prone to be flipped by counterfactual images, while test accuracy on CelebA remains performant.}
    \vspace{-6mm}
\end{table}

\vspace{-2mm}

\section{Conclusion and Discussion} \label{conclusion_and_future}
\looseness=-1
\vspace{-2mm}

In this paper, we present \ourmodel, a zero-shot model diagnosis framework that generates sensitivity histograms based on 
user's input of natural language attributes. 
\ourmodel assesses failures and generates corresponding sensitivity histograms for each attribute.  A significant advantage
of our technique is its ability to analyze the failures of a target deep model without the need for laborious collection and annotation of test sets. \ourmodel effectively visualizes the correlation between attributes and model outputs, elucidating model behaviors and intrinsic biases.

Our work has three primary limitations. First, users should possess domain knowledge as their input (text of attributes of interest) should be relevant to the target domain.  Recall that it is a small price to pay for model evaluation without an annotated test set. Second, StyleGAN2-ADA struggles with generating out-of-domain samples. Nevertheless, our adversarial learning framework can be adapted to other generative models (e.g., stable diffusion), and the generator can be improved by training on more images. We have rigorously tested our generator with various user inputs, confirming its effectiveness for regular diagnosis requests. Currently, we are exploring stable diffusion models to generate a broader range of classes while maintaining the core concept. Finally, we rely on a pre-trained model like CLIP which we presume to be free of bias and capable of encompassing all relevant attributes.

{\bf Acknowledgements: }We would like to thank George Cazenavette, Tianyuan Zhang, Yinong Wang, Hanzhe Hu, Bharath Raj for suggestions in the presentation and experiments. We sincerely thank Ken Ziyu Liu, Jiashun Wang, Bowen Li, and Ce Zheng for revisions to improve this work.

\section{Conclusion}\label{sec:future_works}
In this work, we have proposed EI and UCB based BO algorithms, namely cEI and cUCB for optimizing functions with a composite nature. We further apply our algorithms to the revenue maximization problem and test our methods on different market scenarios. We show that our algorithms, particularly cUCB, outperforms vanilla BO as well as the current state of the art BO-CF algorithm. Our algorithms are  computationally superior because they do not require multiple Cholesky decompositions as required in the BO-CF algorithm.

As part of future work, we would like to provide  theoretical bounds on cumulative regret for the proposed algorithms. We would also like to see the applicability of the proposed algorithms in hyper-parameter tuning for optimizing F1 score. It would also be interesting to propose BO algorithms for an extended model wherein the member functions can be probed independently from each other at different costs.

% \textcolor{blue}{Conclude by saying we outperform classic acquistion functions in regret and eicf/hogp in runtime. Thus, we propose a good balance. Think of some future works maybe or change the section title.}

% \textbf{expand on the future work part .... say theoretical guarantees on regret is one future work. Second future work is to study the problem of precision and recall. The other is to study the case when one has to evaluate different functions independently and when there is a different cost for evaluating different functions, then one has to trade off different function evaluations.}


% \textcolor{red}{Loss in revenue due to regret in comparison to eicf can also be accounted for in savings of hardware requirements for our methodology maybe?}

% \textcolor{blue}{Future work: look at applications of causal graph BO to assist in supply chain optimization. this lays the groundwork for that.}

% \textcolor{red}{fix references, they are not consistent in the information they contain.... somewhere you have doi, reference 5 is going to 5 lines .... reference 1 and 2 have DOI given twoice ,,, arxiv link given twice .. this is enough ground for rejection by virtue of being shabby}


%
% ---- Bibliography ----
%
% BibTeX users should specify bibliography style 'splncs04'.
% References will then be sorted and formatted in the correct style.
%
\bibliographystyle{splncs04}
\bibliography{myrefs}
\end{document}
