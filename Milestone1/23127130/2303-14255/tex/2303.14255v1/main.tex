\documentclass[journal]{vgtc}                     % final (journal style)
%\documentclass[journal,hideappendix]{vgtc}        % final (journal style) without appendices
%\documentclass[review,journal]{vgtc}              % review (journal style)
%\documentclass[review,journal,hideappendix]{vgtc} % review (journal style)
%\documentclass[widereview]{vgtc}                  % wide-spaced review
%\documentclass[preprint,journal]{vgtc}            % preprint (journal style)


%% Uncomment one of the lines above depending on where your paper is
%% in the conference process. ``review'' and ``widereview'' are for review
%% submission, ``preprint'' is for pre-publication in an open access repository,
%% and the final version doesn't use a specific qualifier.

%% If you are submitting a paper to a conference for review with a double
%% blind reviewing process, please use one of the ``review'' options and replace the value ``0'' below with your
%% OnlineID. Otherwise, you may safely leave it at ``0''.
\onlineid{1480}

%% In preprint mode you may define your own headline. If not, the default IEEE copyright message will appear in preprint mode.
%\preprinttext{To appear in IEEE Transactions on Visualization and Computer Graphics.}

%% In preprint mode, this adds a link to the version of the paper on IEEEXplore
%% Uncomment this line when you produce a preprint version of the article 
%% after the article receives a DOI for the paper from IEEE
\ieeedoi{10.1109/TVCG.2023.3247054}

%% declare the category of your paper, only shown in review mode
\vgtccategory{Research}
%% please declare the paper type of your paper to help reviewers, only shown in review mode
%% choices:
%% * algorithm/technique
%% * application/design study
%% * evaluation
%% * system
%% * theory/model
\vgtcpapertype{algorithm/technique}

%% Paper title.
\title{PACE: Data-Driven Virtual Agent Interaction in Dense and Cluttered Environments}

%% This is how authors are specified in the journal style
%% indicate IEEE Member or Student Member in form indicated below

%% Author ORCID IDs should be specified using \authororcid like below inside
%% of the \author command. ORCID IDs can be registered at https://orcid.org/.
%% Include only the 16-digit dashed ID.
\author{\authororcid{James F. Mullen Jr}{0000-0002-4117-1741} and \authororcid{Dinesh Manocha}{0000-0001-7047-9801}}
\affiliation{University of Maryland College Park}

\authorfooter{
  %% insert punctuation at end of each item
  \item
  	James Mullen E-mail: mullenj@umd.edu
  \item
  	Dinesh Manocha E-mail: dmanocha@umd.edu
}

%other entries to be set up for journal
\shortauthortitle{Mullen \MakeLowercase{\textit{et al.}}: Virtual Agent Interaction}

%% Abstract section.
\abstract{We present PACE, a novel method for modifying motion-captured virtual agents to interact with and move throughout dense, cluttered 3D scenes. Our approach changes a given motion sequence of a virtual agent as needed to adjust to the obstacles and objects in the environment. We first take the individual frames of the motion sequence most important for modeling interactions with the scene and pair them with the relevant scene geometry, obstacles, and semantics such that interactions in the agents motion match the affordances of the scene (e.g., standing on a floor or sitting in a chair). We then optimize the motion of the human by directly altering the high-DOF pose at each frame in the motion to better account for the unique geometric constraints of the scene. Our formulation uses novel loss functions that maintain a realistic flow and natural-looking motion. We compare our method with prior motion generating techniques and highlight the benefits of our method with a perceptual study and physical plausibility metrics. Human raters preferred our method over the prior approaches. Specifically, they preferred our method 57.1\% of the time versus the state-of-the-art method using existing motions, and 81.0\% of the time versus a state-of-the-art motion synthesis method. Additionally, our method performs significantly higher on established physical plausibility and interaction metrics. Specifically, we outperform competing methods by over 1.2\% in terms of the non-collision metric and by over 18\% in terms of the contact metric. We have integrated our interactive system with Microsoft HoloLens and demonstrate its benefits in real-world indoor scenes. Our project website is available at \url{https://gamma.umd.edu/pace/}
} % end of abstract

%% Keywords that describe your work. Will show as 'Index Terms' in journal
%% please capitalize first letter and insert punctuation after last keyword
\keywords{Embodied agents, Virtual humans, Human Factors}

%% ACM Computing Classification System (CCS). 
%% See <http://www.acm.org/class/1998/> for details.
%% The ``\CCScat'' command takes four arguments.

%\CCScatlist{ % not used in journal version
% \CCScat{K.6.1}{Management of Computing and Information Systems}%
%{Project and People Management}{Life Cycle};
% \CCScat{K.7.m}{The Computing Profession}{Miscellaneous}{Ethics}
%}

%% A teaser figure can be included as follows
\teaser{
  \centering
  \includegraphics[width=\linewidth]{media/front.pdf}
  \caption{We present an approach to create moving virtual agents that interact with a dense or cluttered 3D scene, either virtual or real, based in motion-captured human data. We have integrated our system with Microsoft HoloLens and highlight its benefit in terms of moving virtual agents in real-world scenes. As key functionality of our method, we adjust or tailor the motion-captured agents motion to the obstacles in the scene like the chairs and tables in the scene above. Our optimization-based approach improves the performance in terms of well-known interaction metrics corresponding to non-collision and contact. In this figure, which we captured using our Microsoft HoloLens, we observe a virtual agent navigating around the obstacles in the environment.}
  \label{fig:front}
}

%% Uncomment below to disable the manuscript note
%\renewcommand{\manuscriptnotetxt}{}

%% Copyright space is enabled by default as required by guidelines.
%% It is disabled by the 'review' option or via the following command:
%\nocopyrightspace


%%%%%%%%%%%%%%%%%%%%%%%%%%%%%%%%%%%%%%%%%%%%%%%%%%%%%%%%%%%%%%%%
%%%%%%%%%%%%%%%%%%%%%% LOAD PACKAGES %%%%%%%%%%%%%%%%%%%%%%%%%%%
%%%%%%%%%%%%%%%%%%%%%%%%%%%%%%%%%%%%%%%%%%%%%%%%%%%%%%%%%%%%%%%%

%% Tell graphicx where to find files for figures when calling \includegraphics.
%% Note that due to the \DeclareGraphicsExtensions{} call it is no longer necessary
%% to provide the the path and extension of a graphics file:
%% \includegraphics{diamondrule} is completely sufficient.
\graphicspath{{figs/}{figures/}{pictures/}{images/}{./}} % where to search for the images

%% Only used in the template examples. You can remove these lines.
\usepackage{tabu}                      % only used for the table example
\usepackage{booktabs}                  % only used for the table example
\usepackage{lipsum}                    % used to generate placeholder text
\usepackage{mwe}                       % used to generate placeholder figures

%% We encourage the use of mathptmx for consistent usage of times font
%% throughout the proceedings. However, if you encounter conflicts
%% with other math-related packages, you may want to disable it.
\usepackage{mathptmx}                  % use matching math font

\usepackage{amsmath}

%%%%%%%%%%%%%%%%%%%%%%%%%%%%%%%%%%%%%%%%%%%%%%%%%%%%%%%%%%%%%%%%
%%%%%%%%%%%%%%%%%%%%%% START OF THE PAPER %%%%%%%%%%%%%%%%%%%%%%
%%%%%%%%%%%%%%%%%%%%%%%%%%%%%%%%%%%%%%%%%%%%%%%%%%%%%%%%%%%%%%%%%

\begin{document}

\section{Introduction}
\IEEEPARstart{T}{he} method Neural Radiance Fields (NeRF)~\cite{mildenhall2020nerf} is proposed for photorealistic novel view synthesis. Given many views of the scene, it creates implicit multi-view geometry and learns for view synthesis. However, it has poor generalizations to new scenes and requires retraining or fine-tuning on each scene. 
 
 Recent work~\cite{Yu_2021_CVPR,Trevithick_2021_ICCV} has explored the ways of using a single image to train NeRF. They introduce a convolutional feature encoder to learn the image representation which gives it some limited generalization abilities to unseen scenes.  But, without fine-tuning, these methods produce many floats and artifacts in rendering novel views. 
 
  Multi-Plane Images (MPI) representation that learns multiple RGB images from a single image is also used in \cite{Wu_2021_ICCV,Tucker_2020_CVPR,wu2022remote} for  novel view synthesis. However, MPI heavily relies on the qualities of the planar images and needs plenty of image planes to avoid blurs. There is no strong 3D geometry constraint and it fails in many complex scenes.
  
  MINE~\cite{Li_2021_ICCV2} introduces the volume rendering of NeRF into the MPI. It runs faster and produces better depth rendering quality compared with single-view NeRFs~\cite{Yu_2021_CVPR,Trevithick_2021_ICCV}. However, the rendering quality heavily relies on the number of image planes. It needs high-resolution 4D volumes to store the 4-channel  (RGB and volume density) image planes that cost a large amount of GPU memory in both training and 
 prediction.  
 

 
 \begin{figure}[t]
\setlength{\abovecaptionskip}{7pt}
\setlength{\belowcaptionskip}{0pt}
	\centering
% 	\subfigure[MINE (PSNR:14.9)]{  % for AAAI
	\subfloat[MINE (PSNR:14.9)]{
%			\centering
			\includegraphics[width=0.23\textwidth]{figure/intro/DJI_20200223_163206_598_0_MINE.png}
%			\label{subfig:pixelnerf}
	}\subfloat[MINE (depth)]{
%			\centering
			\includegraphics[width=0.23\textwidth]{figure/intro/MINE_disp.png}
%			\label{subfig:mpi}
	}
	\\[-3mm]
	\subfloat[Ours (PSNR:17.0)]{
%			\centering
			\includegraphics[width=0.23\textwidth]{figure/intro/DJI_20200223_163206_598_0_ours.png} 
	}\subfloat[Ours (depth)]{
%			\centering
			\includegraphics[width=0.23\textwidth]{figure/intro/ours_disp.png}
	}
	\caption{Comparison with state-of-the-art methods. (a-b) RGB and depth rendering results of  \cite{Li_2021_ICCV2}. It produces many blurs and floats in the occluded regions and at the object/depth edges. 
	(c-d) Our method employs a joint rendering mechanism that preserves more image details and predicts sharp depth edges.}
	\label{fig:performance_illustration}
\end{figure}
 
 In this paper, we propose a joint rendering mechanism that takes the MPI strategy for coarse sampling proposals and the MLP\&volume-based rendering~\cite{mildenhall2020nerf} for fine sampling and rendering. Then, both the coarse point samples and the fine samples are combined according to their geometry distribution to realize a more accurate joint rendering. More importantly, we introduce a depth teacher net that serves as the guidance for the joint rendering. The monocular depth teacher predicts dense pseudo depth maps that assist the consistent 3D geometry learning between the MPI, the fine volume, and the joint rendering. It also boosts the multi-view geometry consistency between the source view and the target novel views that 
helps handle the occlusions, reduce the blurs and floats, and render accurate depths. 
 
In the experiments,  we verify the effectiveness of our method on three challenging real-scene datasets (RealEstate10K~\cite{zhou2018stereo}, NYU~\cite{silberman2012indoor} and  NeRF-LLFF~\cite{mildenhall2020nerf}) for novel view synthesis or depth estimation. Given a single image as input, our method is shown able to produce higher qualities in both the RGB image rendering and depth map prediction. It far outperforms state-of-the-art methods~\cite{Li_2021_ICCV2,Yu_2021_CVPR} with improvements of 5$\sim$20\% in PSNR and SSIM for the RGB rendering and reduces 20$\sim$50\% of the errors for the depth prediction.
\vspace{-4mm}
\section{Related Works}
\noindent\textbf{Sign Language Recognition.} Sign language recognition (SLR) is a fundamental task in the field of sign language understanding.
Feature extraction plays a key role in an SLR model.
% 
Most recent SLR works \cite{jiang2021sign, jiang2021skeleton, hu2021signbert, hu2021hand, li2020transferring, li2020word, joze2019ms, hu2021global, stmc, zuo22_interspeech, vac} adopt CNN-based architectures, \eg, I3D \cite{I3D} and R3D \cite{qiu2017learning}, to extract vision features from RGB videos.
In this work, we adopt S3D \cite{xie2018rethinking} as the backbone of our VKNet due to its excellent accuracy-speed trade-off.

However, RGB-based SLR models may suffer from the large variation of video backgrounds. 
As a complement, some SLR works \cite{jiang2021skeleton, jiang2021sign, hu2021hand, hu2021signbert, chentwo} explore to jointly model RGB videos and keypoints.
For example, SAM-SLR \cite{jiang2021skeleton} uses graph convolutional networks (GCNs) to model pre-extracted keypoints.
HMA \cite{hu2021hand} and SignBERT \cite{hu2021signbert} propose to decode 3D hand keypoints from RGB videos.
A common deficiency of these works is that they need a dedicated network to model keypoints.
In this work, we represent keypoints as a sequence of heatmaps~\cite{duan2022revisiting, chentwo} so that the keypoint encoder of our VKNet can share the identical architecture with the video encoder.

To enable mini-batch training, previous works \cite{jiang2021sign, jiang2021skeleton, hu2021signbert, hu2021hand, li2020transferring, li2020word} crop fixed-length clips from raw videos as model inputs.
However, the model may overfit to the training videos of fixed temporal receptive fields.
In contrast, our VKNet is trained on videos with varied temporal receptive fields to improve its generalization capability.



\noindent\textbf{Word Representation Learning.}
Word2vec \cite{word2vec} and GloVe \cite{glove} are two classical word representation learning frameworks in the field of NLP.
Based on word2vec, fastText \cite{mikolov2018advances} improves word representations with several modifications including the use of sub-word information \cite{bojanowski2017enriching} and position independent features \cite{mnih2013learning}.
Although some advanced language models, \eg, BERT \cite{kenton2019bert}, can also be used to extract word representations, they are computationally intensive and are not dedicated to word representation learning.
In this paper, we adopt the lightweight but effective fastText, which is also used in a recent sign language translation work \cite{yin2021simulslt}, to pre-compute gloss (word) representations.


\noindent\textbf{Vision-Language Models.}
Recently, a majority of vision-language models \cite{clip, align, yao2022filip, gu2022wukong} learn visual representations on large-scale image-text pairs.
Among them, CLIP \cite{clip} is the pioneer to jointly optimize an image encoder and a text encoder through a contrastive loss. 
% 
Besides, the pre-trained CLIP can be generalized to various downstream tasks, \eg, semantic segmentation \cite{xu2022groupvit, li2021language, xu2021simple}, object detection \cite{du2022learning, rao2022denseclip}, image classification~\cite{zhou2022learning,huang2022unsupervised}, and style transfer \cite{patashnik2021styleclip, kwon2022clipstyler}.
In this work, we exploit the implicit knowledge included in glosses (sign labels), which is distinct from previous works on vision-language modeling.


\noindent\textbf{Multi-label Classification.} Real-world objects may have multiple semantic meanings, which motivates research on multi-label classification \cite{ridnik2021asymmetric, ke2022hyperspherical, zhang2013review, rajeswar2022multi, kim2022large} requiring models to map inputs to multiple possible labels.
Although the VISigns may be associated with the multi-label classification problem, most widely-adopted SLR datasets \cite{li2020word, joze2019ms, hu2021global} are singly labeled.
In this work, we deal with the VISigns by incorporating language information included in glosses.

\vspace{-0.3em}
\section{Method}
\vspace{-0.3em}

Our sensitivity-aware visual parameter-efficient fine-tuning consists of two stages. In the first stage, SPT measures the task-specific sensitivity for the pre-trained parameters (Section~\ref{subsec:sensitivity}). Based on the parameter sensitivity and a given parameter budget, SPT then adaptively allocates trainable parameters to task-specific important positions (Section~\ref{subsec:SPT}).

\vspace{-0.3em}
\subsection{Task-specific Parameter Sensitivity}
\label{subsec:sensitivity}
\vspace{-0.3em}

Recent research has observed that pre-trained backbone parameters exhibit varying feature patterns~\cite{raghu2021vision,naseer2021intriguing} and criticality~\cite{zhang2019all,chatterji2019intriguing} at distinct positions. 
Moreover, when transferred to downstream tasks, their efficacy varies depending on how much pre-trained features are reused and how well they adapt to the specific domain gap~\cite{yosinski2014transferable,kumar2022finetuning,neyshabur2020being}. Motivated by these observations, we argue that not all parameters contribute equally to the performance across different tasks in PEFT and propose a new criterion to measure the sensitivity of the parameters in the pre-trained backbone for a given task.

Specifically, given the training dataset $\gD_t$ for the $t$-th task and the pre-trained model weights $\vw=\left\{w_1, w_2, \ldots, w_N\right\}\in \sR^N$ where $N$ is the total number of parameters, the objective for the task is to minimize the empirical risk: $\min_{\vw} E(\gD_t, \vw)$.
We denote the parameter sensitivity \bohan{set} as $\gS=\{s_1, \ldots, s_N\}$ and the sensitivity $s_n$ for parameter $w_n$ is measured by the empirical risk difference when tuning it:
\begin{equation}
\vspace{-0.3em}
    \begin{aligned}
        s_n = E(\gD_t, \vw)-E(\gD_t, \vw\mid w_n=w_n^*),
    \end{aligned}
\label{eq:sensitivity}
\end{equation}
where $w_n^*=\underset{w_n}{\rm argmin}(E(\gD_t, \vw))$. We can reparameterize the tuned parameters as  $w_n^*=w_n+\Delta_{w_n}$, where $\Delta_{w_n}$ denotes the update for $w_n$ after tuning. Here we individually measure the sensitivity of each parameter, which is reasonable given that most of the parameters are frozen during fine-tuning in PEFT. However, it is still computationally intensive to compute Eq.~(\ref{eq:sensitivity}) for two reasons. Firstly, getting the empirical risk for $N$ parameters requires forwarding the entire network $N$ times, which is time-consuming. Secondly, it is challenging to derive $\Delta_{w_n}$, as we have to tune each individual $w_n$ until convergence.

{\begin{algorithm}[t!]
\caption{\label{alg:tps} Computing task-specific parameter sensitivities}
\begin{algorithmic}
    \STATE \textbf{Input:} Pre-trained model with network parameters $\vw$, training set $\gD_t$ for the $t$-th task, and number of training samples $C$ used to calculate the parameter sensitivities
    \STATE \textbf{Output:} Sensitivity set $\gS=\{s_1, \ldots, s_N\}$
    \STATE Initialize $\gS=\{0\}^N$
    \FOR{$i\in\{1,\ldots,C\}$}
        \STATE Get the $i$-th training sample of $\gD_t$
	    \STATE Compute loss $E$
		\STATE Compute gradients $\vg$
		\FOR{$n\in\{1,\ldots,N\}$}
                \STATE Update sensitivity for the $n$-th parameter: $s_{n} = s_{n} + g_n^2$
		    \ENDFOR
    \ENDFOR
\end{algorithmic}
\end{algorithm}}


\begin{figure*}[t]
\begin{center}
    \includegraphics[width=\linewidth]{main_figure.pdf}
\end{center}\vspace{-2em}
\caption{Overview of our trainable parameter allocation strategy. With the parameter sensitivity \bohan{set} $\gS$, we first get the top-$\tau$ sensitive parameters. Instead of directly tuning these sensitive parameters, we also boost the representational capability by replacing unstructured tuning with structured tuning at sensitive weight matrices that have a large number of sensitive parameters, which can be implemented by an existing structured tuning method, \eg, LoRA~\cite{hu2022lora} and Adapter~\cite{houlsby2019parameter}. Red lines and blocks represent trainable parameters and modules, while blue lines represent frozen parameters.}
\label{fig:main}
\vspace{-1.5em}
\end{figure*}


To overcome the first barrier, we simplify the empirical loss by approximating $s_n$ in the vicinity of $\vw$ by its first-order Taylor expansion
\vspace{-0.3em}
\begin{equation}
\vspace{-0.5em}
    \begin{aligned}
        s_n^{(1)} = -g_n\Delta_{w_n},
    \end{aligned}
\label{eq:first-order}
\end{equation}
where the gradients $\vg=\partial E/\partial\vw$, and $g_n$ is the gradient of the $n$-th element of $\vg$. 
To address the second barrier, following~\cite{liu2018darts,cai2018proxylessnas}, we take the one-step unrolled weight as the surrogate for $w_n^*$ and approximate $\Delta_{w_n}$ in Eq.~(\ref{eq:first-order}) with a single step of gradient descent. We can accordingly get $s_n^{(1)} \approx g_n^2\epsilon$,
where $\epsilon$ is the learning rate. Since $\epsilon$ is the same for all parameters, we can eliminate it when comparing the sensitivity with the other parameters and finally get 
\vspace{-0.5em}
\begin{equation}
\vspace{-0.3em}
    \begin{aligned}
        s_n^{(1)} \approx g_n^2.
    \end{aligned}
\label{eq:first-order-simp}
\end{equation}
Therefore, the sensitivity of a parameter can be efficiently measured by its potential to reduce the loss on the target domain. Note that although our criterion draws inspiration from pruning work~\cite{molchanov2019importance}, it is distinct from it. \cite{molchanov2019importance} measures the parameter importance by the squared change in loss when removing them, \ie, $\left( E(\gD_t, \vw)-E(\gD_t, \vw\mid w_n=0) \right)^2$ and finally derives the parameter importance by $\left( g_n w_n \right)^2$, which is different from our formulations in Eqs.~(\ref{eq:sensitivity}) and~(\ref{eq:first-order-simp}).

In practice, we accumulate $\gS$ from a total number of $C$ training samples ahead of fine-tuning to generate accurate sensitivity as shown in Algorithm~\ref{alg:tps}, where $C$ is a pre-defined hyper-parameter. In Section~\ref{subsec:abl}, we show that employing only 400 training samples is sufficient for getting reasonable parameter sensitivity, which requires only 5.5 seconds with a single GPU for any VTAB-1k dataset with ViT-B/16 backbone~\cite{vit}.

\vspace{-0.3em}
\subsection{Adaptive Trainable Parameters Allocation}
\label{subsec:SPT}
\vspace{-0.2em}

Our next step is to allocate trainable parameters based on the obtained parameter sensitivity set $\gS$ and a desired parameter budget $\tau$. A straightforward solution is to directly tune the top-$\tau$ most sensitive unstructured connections (parameters) \rev{while keeping the rest frozen}, which we name unstructured tuning. Specifically, we select the top-$\tau$ most sensitive weight connections in $\gS$ to form the sensitive weight connection set $\gT$. Then, for \rev{a} weight matrix $\mW\in \sR^{d_{\rm in}\times d_{\rm out}}$, we can get a binary mask $\mM\in \sR^{d_{\rm in}\times d_{\rm out}}$ computed by
\vspace{-0.5em}
\begin{equation}
\vspace{-0.5em}
    {\begin{array}{ll}
    \small
    \begin{aligned}
    \mM^j =
    \left\{\begin{array}{ll} 
    1 ~~~~~ \mW^j \in \gT \\
    0 ~~~~~ \mW^j \notin \gT
    \end{array}\right.
    \end{aligned},
    \small
    \end{array}}
\label{eq:mask}
\end{equation}
where $\mW^j$ and $\mM^j$ are the $j$-th element in $\mW$ and $\mM$, respectively. Accordingly, we can train the sensitive parameters by gradient descent and the updated weight matrix can be formulated as $\mW'\leftarrow \mW - \epsilon\vg_{\mW}\odot\mM$, where $\vg_{\mW}$ is the gradient for $\mW$.

However, considering PEFT approaches generally limit the proportion of trainable parameters to less than 1\%, tuning only a small number of unstructured weight connections might not have enough representational capability to handle the downstream datasets with large domain gaps from the source pre-training data. Therefore, to improve the representational capability, we propose to replace unstructured tuning with structured tuning at the sensitive weight matrices that have a high number of sensitive parameters. To preserve the parameter budget, we can implement structured tuning with an existing efficient structured tuning PEFT method~\cite{hu2022lora,chen2022adaptformer,houlsby2019parameter,jie2022convolutional} that learns to directly adjust \rev{all hidden dimensions at once}. We depict an overview of our trainable parameter allocation strategy in Figure~\ref{fig:main}. For example, we can employ the low-rank reparameterization trick LoRA~\cite{hu2022lora} to the sensitive weight matrices \rev{and the one-step update for $\mW$ can be formulated as}
\vspace{-0.4em}
\begin{equation}
\vspace{-0.4em}
    {\begin{array}{ll}
    \small
    \begin{aligned}
    \mW' = \left\{\begin{array}{ll} 
    \mW + \mW_{\rm down}\mW_{\rm up} & ~~ \text { if } ~~ \sum_{j=0}^{d_{\rm in}\times d_{\rm out}} \mM^j \geq \sigma_{\rm opt} \\
    \mW - \epsilon\vg_{\mW}\odot\mM & ~~ {\rm otherwise}
    \end{array}\right.
    \end{aligned},
    \small
    \end{array}}
\label{eq:weight_updat}
\end{equation}
where $\mW_{\rm down}\in \sR^{d_{\rm in}\times r}$ and $\mW_{\rm up}\in \sR^{r\times d_{\rm out}}$ are two learnable low-rank matrices to approximate the update of $\mW$ and rank $r$ is a hyper-parameter where $r \ll {\rm min}(d_{\rm in},d_{\rm out})$. In this way, we perform structured tuning on $\mW$ when its number of sensitive parameters exceeds $\sigma_{\rm opt}$, whose value depends on the pre-defined type of structured tuning method. For example, since implementing structured tuning with LoRA requires $2\times d_{\rm in} \times d_{\rm out} \times r$ trainable parameters for each sensitive weight matrix, we set $\sigma_{\rm LoRA} \leftarrow 2\times d_{\rm in} \times d_{\rm out} \times r$ to ensure that the number of trainable parameters introduced by structured tuning is always equal to or lower than the number of sensitive parameters.

In this way, our SPT adaptively incorporates both structured and unstructured tuning granularities to enable higher flexibility and stronger representational power, simultaneously. In Section~\ref{subsec:abl}, we show that structured tuning is important for the downstream tasks with larger domain gaps and both unstructured and structured tuning contribute clearly to the superior performance of our SPT.
\begin{table*}[t!]
\begin{minipage}{0.175\linewidth}
\centering
% \hspace{1.8mm}
\captionof{table}{\small Datasets statistics \label{graph_datasets}}
\begin{tiny}
\begin{tabular}{c||c|c}
      {\bf Graph} & {\bf \#nodes} & {\bf \#edges} \\ \hline
      {\em FL} & 80\,513     & 5\,899\,882 \\
      {\em YT} & 1\,138\,499 & 2\,990\,443 \\
      {\em LJ} & 2\,238\,731 &14\,608\,137 \\
      {\em OR} & 3\,072\,441 & 117\,185\,083 \\
      {\em TW} & 41\,652\,230 & 1\,468\,365\,182 \\
\end{tabular}
\end{tiny}
\end{minipage}%
 \quad
 \begin{minipage}{.265\linewidth}
\centering
\tabcolsep=0.05cm

\captionof{table}{\small Avg. memory footprint (GB) of {\sf DistGER} and {\sf KnightKing} on each machine, where $\sigma$ is the standard deviation}
\label{Memory_usage}
\begin{tiny}
\newcommand{\tabincell}[2]{\begin{tabular}{@{}#1@{}}#2\end{tabular}}
  % \caption{\small {\color{blue} Avg. memory footprint (GB) of {\sf DistGER} and {\sf KnightKing} on each machine, where $\sigma$ is the standard deviation.}}
  \begin{tabular}{c|cc|cc}
    %\hline
    { }&\multicolumn{2}{c|}{\bfseries{ Sampling}}&\multicolumn{2}{c}{\bfseries{Training}}\\
    \hline
    {\bf{Graph}} &{\sf KnightKing} &{\sf DistGER} &{\sf KnightKing} &{\sf DistGER} \\
    \hline
     {\em FL} & 0.66($\pm$0.06)	&{\bf 0.41($\pm$0.02)}	&1.31($\pm$0.17) 	&{\bf 0.86($\pm$0.06)} 	\\

     {\em YT} &4.11($\pm$0.55)	&{\bf 1.36($\pm$0.23)} 	&4.73($\pm$0.72) 	&{\bf 4.26($\pm$0.63)} \\

     {\em LJ} & 7.65($\pm$0.82)	&{\bf 1.95($\pm$0.16)}	&6.38($\pm$0.97) 	&{\bf 5.49($\pm$0.85)} 	\\

     {\em CO} &10.98($\pm$1.03)	&{\bf 3.27($\pm$0.79)} 	&8.52($\pm$1.01) 	&{\bf 6.86($\pm$0.69)} 	\\

     {\em TW} & out-of-memory	&{\bf 20.18($\pm$3.62)} 	&out-of-memory 	& {\bf 67.16($\pm$5.18)} 	\\
  %\hline
\end{tabular}
\end{tiny}

\end{minipage}
\quad
\begin{minipage}{.25\linewidth}
    \centering
    \includegraphics[width= 1.85in, height = 1.2in]{./Figures/Dist_total_time_partition.eps}%
    \captionof{figure}
      {\small Efficiency: {\sf PBG} \cite{PBG_2019}, {\sf DistDGL} \cite{DistDGL_2020}, {\sf KnightKing} \cite{KnighKing_2019}, {\sf HuGE-D} (baseline), {\sf DistGER} (ours)
        \label{overall_performance}
      }
\end{minipage}%\hfill
\quad
\begin{minipage}{.25\linewidth}
    \centering
    \includegraphics[width= 1.85in, height = 1.2in]{./Figures/Dist_scalability_partition.eps}%
    \captionof{figure}
      {\small Scalability: {\sf PBG} \cite{PBG_2019}, {\sf DistDGL} \cite{DistDGL_2020}, {\sf KnightKing} \cite{KnighKing_2019}, {\sf HuGE-D} (baseline), {\sf DistGER} (ours)
        \label{Dist_scalability}
      }
\end{minipage}
\end{table*}


\section{Experimental Results}
\label{sec:experiments}
We evaluate the efficiency (\S \ref{sec:overall}) and scalability (\S \ref{sec:scalability}) of our proposed method, {\sf DistGER}
by comparing with {\sf HuGE-D} (baseline),
{\sf KnightKing} \cite{KnighKing_2019}, {\sf PyTorch-BigGraph} ({\sf PBG}) \cite{PBG_2019}, and {\sf Distributed DGL}
({\sf DistDGL}) \cite{DistDGL_2020}. We also compare the effectiveness (\S \ref{sec:effectiveness}) of generated embeddings
on link prediction.
% and multi-label classification tasks. 
Finally, we analyze efficiency due to individual
parts of {\sf DistGER} (\S \ref{sec:individual})
and the generality of {\sf DistGER} for other random walk-based embeddings (\S \ref{sec:generality}).
Our codes and datasets are at \cite{code}.
%
\subsection{Experimental Setup}
\label{sec:setup}
%
\spara{Environment.} We conduct experiments on a cluster of 8 machines with 2.60GHz Intel $^\circledR$ Xeon $^\circledR$ Gold 6240 CPU with 72 cores (hyper-threading)
in a dual-socket system, and each machine is equipped with 192GB DDR4 memory and connected by a 100Gbps network.
The machines run Ubuntu 16.04 with Linux kernel 4.15.0. We use GCC v9.4.0 for compiling {\sf DistGER}, {\sf KnightKing}, and {\sf HuGE-D},
and use Python v3.6.15 and torch v1.10.2 as the backend deep learning framework for {\sf Pytorch-BigGraph} and {\sf DistDGL}.

\spara{Datasets. } We employ five widely-used, real-world graphs
(Table~\ref{graph_datasets}): {\em Flickr} (FL) \cite{Flickr_Youtube_Graph},
{\em Youtube} (YT) \cite{Flickr_Youtube_Graph},
{\em LiveJournal} (LJ) \cite{BlogCatalog_Twitter_LiveJournal_Graph},
{\em Com-Orkut} (OR) \cite{com-orkut_2012}, and {\em Twitter} (TW) \cite{twitter_2010}.
The first two graphs are selected for multi-label node classification with distinct number of node labels 195 and 47, respectively, %in {\em Flickr} and {\em Youtube},
where labels in {\em Flickr} represent interest groups of users, and {\em Youtube}'s labels represent groups of viewers that enjoy common video genres. The last four graphs are used in link prediction. We also use synthetic graphs \cite{RMAT_2004} (up to 1 billion nodes, 10 billion edges) and a real-world {\em UK graph} \cite{BSVLTAG} (100M nodes, 3.7B edges) to assess the scalability of {\sf DistGER}.
Considering the default settings of popular random walk-based methods (e.g., Deepwalk, node2vec, HuGE), we use their undirected version.

\spara{Competitors.} We compare {\sf DistGER} against three state-of-the-art distributed graph embedding frameworks: the distributed random walk engine, {\sf KnightKing} {\scriptsize\url{https://github.com/KnightKingWalk/KnightKing}}
\cite{KnighKing_2019}; the distributed multi-relations based graph embedding system, {\sf PyTorch-BigGraph} ({\sf PBG})
{\scriptsize\url{https://github.com/facebookresearch/PyTorch-BigGraph}} \cite{PBG_2019} -- designed by Facebook; and
the distributed graph neural networks-based system, {\sf DistDGL} {\scriptsize\url{https://github.com/dmlc/dgl}}
\cite{DistDGL_2020} -- recently proposed by Amazon. We also implement {\sf HuGE-D}, a distributed version of
information-centric random walk-based graph embedding ({\sf HuGE} \cite{HuGE_2021}), on top of {\sf KnightKing},
served as our baseline. Since {\sf KnightKing} and {\sf HuGE-D} provide distributed support only for
random walk without that for embedding learning, we generate their node embeddings using
{\sf Pword2vec} {\scriptsize\url{https://github.com/IntelLabs/pWord2Vec}} \cite{Pword2vec_2019},
the most popular distributed {\sf Skip-Gram} system released by Intel.
%We find that {\sf pSGNScc} \cite{pSGNSCC_2017} (\S \ref{sec:learning})
%only provides a single-machine implementation, thus we do not include it in our distributed experiments.

\spara{Parameters.} For {\sf DistGER} and {\sf HuGE-D} random walks, we set
parameters $\mu$=0.995, $\delta$=0.001 based on information measurements (\S \ref{sec:preliminaries}),
while {\sf KnightKing} uses $L$=80 and $r$=10 that are routine configurations in the traditional
random walk-based graph embedding \cite{node2vec_2016, DeepWalk_2014, KnighKing_2019}. For {\sf DistGER}, {\sf KnightKing}, and {\sf HuGE-D} training,
we set the sliding window size $w$=10, number of negative samples $K$=5, and synchronization period=0.1 sec \cite{Pword2vec_2019},
and additionally, multi-windows number=2, $\gamma$=2 for {\sf DisrGER}.
%For {\sf Pytorch-BigGraph} ({\sf PBG}), we set the number of partitions to 16 following \cite{PBG_2019}, that is, using $2m$ partitions for the number of machines $m$ = 8
%in our case. %For {\sf DistDGL}, the deployed {\sf GaphSAGE} model uses three graph convolutional layers.
For fair comparison across all systems, %the efficiency performance of all systems involved in the experiments,
we set the embedding dimension $d$=128 that is commonly used \cite{HuGE_2021,node2vec_2016,DeepWalk_2014,Line_2015,Verse_2018,ProNE_2019},
and report the average running time for each epoch. For task effectiveness evaluations,
we find the best results from a grid search over learning rates from 0.001-0.1, \# epochs from 1-30,
and \# dimensions from 128-512.


%
\eat{
\begin{table}
\newcommand{\tabincell}[2]{\begin{tabular}{@{}#1@{}}#2\end{tabular}}
  \caption{\small Avg. memory footprint (GB) of {\sf DistGER} and {\sf KnightKing} on each machine, where $\sigma$ is the standard deviation.}
  \label{Memory_usage}
  \begin{center}
   \footnotesize
  \begin{tabular}{c|cc|cc}
    %\hline
    { }&\multicolumn{2}{c|}{\bfseries{ Sampling}}&\multicolumn{2}{c}{\bfseries{Training}}\\
    \hline
    {\bf{Graph}} &{\sf KnightKing} &{\sf DistGER} &{\sf KnightKing} &{\sf DistGER} \\
    \hline
     {\em Flickr} & 0.66($\pm$0.06)	&{\bf 0.41($\pm$0.02)}	&1.31($\pm$0.17) 	&{\bf 0.86($\pm$0.06)} 	\\

     {\em Youtube} &4.11($\pm$0.55)	&{\bf 1.36($\pm$0.23)} 	&4.73($\pm$0.72) 	&{\bf 4.26($\pm$0.63)} \\

     {\em LiveJournal} & 7.65($\pm$0.82)	&{\bf 1.95($\pm$0.16)}	&6.387($\pm$0.97) 	&{\bf 5.49($\pm$0.85)} 	\\

     {\em Com-Orkut} &10.98($\pm$1.03)	&{\bf 3.27($\pm$0.79)} 	&8.52($\pm$1.01) 	&{\bf 6.86($\pm$0.69)} 	\\

     {\em Twitter} & out-of-memory	&{\bf 37.1($\pm$5.28)} 	&out-of-memory 	& {\bf 79.5($\pm$7.27)} 	\\
  %\hline
\end{tabular}
\end{center}
\end{table}
%
}
%


%
\subsection{Efficiency and Memory Use w.r.t. Competitors}
\label{sec:overall}
%\begin{figure}
%  \centering
%  \includegraphics[width= 3 in]{Dist_total_time.eps}
%  \caption{\small Overall performance of PBG, DistDGL, KnightKing, HuGE-D and DistGER for generating embeddings on different read-word graphs, {\color{blue}for Twitter graph, DistDGL cannot finish in one day, and KnightKing fails to perform due to memory issue, where the y axis is in log-scale.}}
%  \label{overall_performance}
%\end{figure}
%
We report the end-to-end running times of {\sf PBG}, {\sf DistDGL}, {\sf KnightKing}, {\sf HuGE-D}, and {\sf DistGER}
on five real-world graphs with the cluster of 8 machines in Figure~\ref{overall_performance}.
The reported end-to-end time includes the running time of partitioning, random walks (for random walk-based frameworks), and training procedures.
%{\color{blue} Noted that the reported end-to-end time in our experiments excludes the partition time for all evaluated frameworks due to the all used partition schemes are executed as a preprocessing component, and we separately evaluate the partition efficiency in Section 6.5, thus the end-to-end time refers to the running time of random walk (only for random walk-based framework) and training procedure.}
{\sf DistGER} significantly outperforms the competitors
on all these graphs, achieving a speedup ranging from 2.33$\times$ to 129$\times$. %, by an average acceleration of $39.78 \times$.
Recall that {\sf DistGER} is a similar type of system as {\sf KnightKing} and {\sf HuGE-D},
and our key improvements are discussed in \S \ref{sec:DistGER} and in \S \ref{sec:learning}.
Analogously, Figure~\ref{overall_performance} exhibits that our system, %designs are more effective (see more evaluation details in \S 6.3),
{\sf DistGER} achieves an average speedup of 9.25$\times$ and 6.56$\times$ compared with {\sf KnightKing} and {\sf HuGE-D}.
Notice that we fail to run {\sf KnightKing} on the largest {\em Twitter} dataset
because its routine random walk strategy requires more main memory space.
%Although Huge-D achieves comparable performance,
The advantage of information-centric random walk in {\sf HuGE} is almost wiped out in {\sf HuGE-D}
due to on-the-fly information measurements and the higher communication costs in a distributed setting.
The multi-relation-based {\sf PBG} leverages a parameter server to synchronize embeddings between clients,
resulting in more load on the communication network. As a result, {\sf PBG} is on average
26.22$\times$ slower than {\sf DistGER}. For graph neural network-based system {\sf DistDGL},
due to the long running time of graph sampling (e.g., taking 80\% of the overhead for the {\sf GraphSAGE}),
it is highly inefficient than other systems. For the billion-edge {\em Twitter} graph, it does not terminate in 1 day.
%
%Considering the resource consumption that affects scalability,
{Table ~\ref{Memory_usage}} shows {\sf DistGER}'s average memory footprint on each machine of the 8-machine cluster. %from a cluster of 8 machines.
%and the standard deviation %($\sigma$) %of the results
%in 
%Table ~\ref{Memory_usage}. 
Compared
to %other methods, %with the 
same type of system
{\sf KnightKing}, 
% that is of the same system type, 
{\sf DistGER} requires less memory for sampling and training.


\subsection{Scalability w.r.t. Competitors}
\label{sec:scalability}
%
%\begin{figure}
%  \centering
%  \includegraphics[width= 3.2 in]{Dist_scalability.eps}
%  \caption{\small Scalability comparison on LiveJournal graph, where the y axis is in log-scale.}
% \label{Dist_scalability}
%\end{figure}
%
Figure~\ref{Dist_scalability} shows end-to-end running times of all competing
systems on the {\em LiveJournal} graph, as we increase \# machines
from 1 to 8 to evaluate scalability. {\sf DistGER} achieves better scalability than the other
four distributed systems.
%Due to space limitation, we omit results on other graph datasets,
%which exhibit similar trends.
{\sf PBG} leverages a parameter server and a shared network filesystem
to synchronize the parameters in the distributed model. %The edges are partitioned into $m^2$ buckets
%and training can be performed in parallel using up to $m/2$ machines. After one bucket completes
%the training, it needs to communicate with the parameter server.
When the number of machines increases, {\sf PBG} puts more load
on the communications network, resulting in poor scalability. Likewise, {\sf DistDGL}
is bounded by the synchronization overhead for gradient updates,
limiting its scalability.
%Since {\sf DistDGL} uses mini-batches for sampling, %features %for GraphSAGE,
%if the mini-batch samples cannot be generated on time, the trainer will be delayed on the forward pass, and all other
%machines need to wait before starting the backward pass. Thus, increasing the number of
%machines also affects the efficiency of backward pass. %Being the random walk-based distributed systems,
Both {\sf KnightKing} and {\sf HuGE-D} suffer from higher communication costs during random walks,
due to their only workload-balancing partitioning scheme (\S \ref{sec:dRand}, \S \ref{sec:individual}).
%Their scalability is relatively poor as the number of machines increases.
%{\sf KnightKing} partitions the graph by a workload-balancing scheme, inevitably introducing higher
%cross-machine communications due to the randomness inherent in the random walking procedure (\S \ref{sec:dRand}, \S \ref{sec:individual}).
%With more machines, the inefficiency of the partitioning scheme is further magnified.
Since {\sf HuGE-D} is implemented on top of {\sf KnigtKing},
it exhibits worse scalability due to high communication costs and on-the-fly information measurements in a distributed setting (\S \ref{sec:HUGED}).
%In contrast, its performance is much better than all the competitors in a single machine.
In comparison, {\sf DistGER} incorporates multi-proximity-aware streaming graph partitioning and incremental computations
to reduce both communication and computation costs, it also employs hotness-block based parameters synchronization
during training to dramatically reduce the pressure on network bandwidth. Hence, {\sf DistGER} achieves better scalability than other systems.
Due to space limitations, we omit {\sf DistGER}'s scalability results on other graphs, which exhibit similar trends. On {\em Twitter}, the end-to-end running times {\sf DistGER} on 1, 2, 4, and 8 machines are 3090s, 1739s, 1197s, and 746s, respectively,
while on {\em Com-Orkut}, the results are 304s, 204s, 149s, and 89s, respectively. 
The results show a good linear relationship.
% The results demonstrate a desired scalability with the increase of the machines.

\begin{table}
\quad
\begin{minipage}{0.46\linewidth}
    \centering
    \includegraphics[width= 1.6 in]{./Figures/Dist_scalability_datasize.eps}%
    \captionof{figure}
      {\small {Scalability of {\sf DistGER} on synthetic graphs, where Y-axis is in log-scale}}
      %The lines depict the running time required for random walk (blue line) and training (red line), respectively. Pentagrams show the time cost of six real-world graphs,
        \label{Dist_scalability_data}
      
\end{minipage}\hfill
\quad
\begin{minipage}{.46\linewidth}
    \centering
    \includegraphics[width= 1.6 in]{./Figures/Dist_time_auc.eps}%
    \captionof{figure}
      {\small {The influence of running time on embedding quality for {\sf DistGER} and competitors}}
        \label{Dist_time_auc}
\end{minipage}
\end{table}


To further assess the scalability of {\sf DistGER}, we generate synthetic graphs \cite{RMAT_2004} with a fixed node degree of 10 and the number of nodes from $10^5$ to $10^9$. Figure~\ref{Dist_scalability_data} presents the running times for random walks and training on these synthetic graphs using a cluster of 8 machines, suggesting that the running time increases linearly with the size of a graph, and {\sf DistGER} has the capability to handle even billion-node graphs. Moreover, the running times for six real-world graphs (including the {\em UK graph} with $|E|=3.7B$, $|V|=100M$, for which the competing systems do not terminate in 1 day or crash due to hardware and memory limitation) are inserted into the plot, which is consistent with the trend on synthetic data.

%
%
\subsection{Effectiveness w.r.t. Competitors}
\label{sec:effectiveness}
%
\spara{Link prediction.} To perform link prediction on a given graph $G$, following \cite{HuGE_2021,node2vec_2016,Verse_2018,NRP_2020},
we first uniformly at random remove 50\% edges as positive test edges, and the rest are used as positive training edges.
We also provide negative training and test edges by considering those node pairs between which no edge exists in $G$.
We ensure that the positive and negative set sizes are similar. %For a pair of nodes $(u, v)$, let $\varphi(u)$ and
%$\varphi(v)$ be the vectors learned by embedding methods.
The link prediction is conducted as a classification task
based on the similarity of $u$ and $v$, i.e., $\varphi(u)\cdot\varphi(v)$.
The effectiveness of link prediction is measured via the $AUC$ (Area Under Curve) score \cite{AUC_kdd} -- the higher the better.
We repeat this procedure 50 times to offset the randomness of edge removal and report the average $AUC$ in
Table~\ref{AUC_results}.
%shows $AUC$ for all the methods on five real-world graphs.
%, respectively, where a ``$-$'' indicates that the method fails due to the limitation of computing resources or because its running time exceeds 1 day.
{\sf DistGER} outperforms all competitors on these graphs, except for {\sf PBG} on {\em Com-Orkut}, where {\sf DistGER} ranks second.
On average, {\sf DistGER} has an 11.7\% higher $AUC$ score compared with the other three systems, thanks to our
information-centric random walks. {\sf PBG} is the best on {\em Com-Orkut} because this graph is much denser
and is friendly to the multi-relationship-based model in {\sf PBG}.
Figure~\ref{Dist_time_auc} exhibits accuracy-efficiency tradeoffs of {\sf DistGER} and competitors, i.e., their $AUC$ convergence curves w.r.t. increasing running times of random walks and training, over {\em LiveJournal}, further indicating
that {\sf DistGER} has better efficiency and effectiveness than the competitors.
%As a system of the same type, DistGER achieves better accuracies on all graphs than KnightKing which leverages the routine random walk configuration, thanks to its information-centric random walk strategies. We do not report the effectiveness of HuGE-D here because it uses the same random walk model as DistGER.
%
\begin{table}[h!]
\newcommand{\tabincell}[2]{\begin{tabular}{@{}#1@{}}#2\end{tabular}}
  \caption{\small $AUC$ scores of {\sf DistGER} and competitors for link prediction}
  \label{AUC_results}
  \begin{center}
  \footnotesize
  \begin{tabular}{cccccc}
%    \hline
    {Method}&\tabincell{c}{Youtube}&{LiveJournal}&\tabincell{c}{Com-Orkut}&{ Twitter}\\
    \hline
    {\sf PBG}        & 0.753           &0.882            &\bfseries{0.955} &0.912\\

    {\sf DistDGL}    &0.894            &0.718            &0.815            & running time $>$ 1 day \\

    {\sf KnightKing} &0.904            &0.963            & $0.918$         & out-of-memory\\

    {\sf DistGER}    &\bfseries{0.966} &\bfseries{0.976} &0.921            &\bfseries{0.919}\\
%  \hline
\end{tabular}
\end{center}
\end{table}

% \eat{
\spara{Multi-label node classification.}
This task predicts one or more labels for each graph node and has applications in %modern applications ranging from
text categorization \cite{zhang2006multilabel} and bioinformatics \cite{zhang2018ontological}.
We use embedding vectors and a one-vs-rest logistic regression classifier
with L2 regularization \cite{MLC_LIBLINEAR_2008}, %(using the LIBLINEAR library),
then evaluate the effectiveness by micro-averaged F1 ($Micro-F1$) and macro-averaged F1 ($Macro-F1$) \cite{WangC016}
scores, where $Micro-F1$ gives equal weight to each test instance and $Macro-F1$ assigns equal weight to each label category \cite{keikha2018community}.
%To train a classifier, nodes are uniformly at random split into training and test sets.
Following \cite{HuGE_2021,node2vec_2016,DeepWalk_2014,Line_2015,Verse_2018},
we select 10\% to 90\% training data ratio on {\em Flickr}, and 1\% to 9\% training ratio on {\em Youtube}.
%and the remaining nodes for testing.
We report the averaged $Macro-F1$ and $Micro-F1$ scores from 50 trials in Figure~\ref{Dist_MLC_mac_mic_F1}.
% shows the $Macro-F1$ and $Micro-F1$ scores achieved by each system as a
%function of the training ratio variation, respectively.
We find that {\sf DistGER} has better $Macro-F1$ and $Micro-F1$ scores
than existing frameworks, %on these graphs, %. In particular, compared with the KnightKing,
%DistGER consistently outperforms the other random walk-based systems on all graphs in $Macro-F1$ and $Micro-F1$ scores,
gaining 9.2\% and 3.3\% average improvements, respectively, due to its more effective information-centric random walks.
%Definition of $Macro-F1$ and $Micro-F1$ are as the following:

\begin{figure}[h!]
  \centering
  \includegraphics[width= 3.45 in]{./Figures/Dist_MLC_mac_mic_F1_1.eps}
  \caption{\small $Macro-F1$ (a1, b1) and $Micro-F1$ (a2, b2) scores for multi-label node classification. $X$-axis: training data ratio}
  \label{Dist_MLC_mac_mic_F1}
\end{figure}

% }
%\begin{equation}
%Precision = \frac{\sum\nolimits_{i}^{K}TP(i)}{\sum\nolimits_{i}^{K}(TP(i)+FP(i))}
%\end{equation}
%
%\begin{equation}
%Recall = \frac{\sum\nolimits_{i}^{K}TP(i)}{\sum\nolimits_{i}^{K}(TP(i)+FN(i))}
%\end{equation}
%
%\begin{equation}
%Micro-F1 = \frac{2\times Precision\times Recall}{Precision+Recall}
%\end{equation}
%
%\begin{equation}
%Macro-F1 = \frac{\sum\nolimits_{i}^{K}Micro-F1(i)}{|K|}
%\end{equation}
%
%where $TP(i)$, $FP(i)$ and $FN(i)$ are the number of true positives, false positives and false negatives in the instances which are predicted as $i$, respectively. Suppose $K$ is the overall label set, $Micro-F1$($i$) and $Macro-F1$ are the measure of $Micro-F1$ and $Macro-F1$ for the label $i$, respectively.
%\begin{table}
%\setlength{\abovecaptionskip}{0.cm}
%\setlength{\belowcaptionskip}{-0.cm}
%\newcommand{\tabincell}[2]{\begin{tabular}{@{}#1@{}}#2\end{tabular}}
%  \caption{$Macro-F1$ and $Micro-F1$ for multi-label classification on Flickr and Youtube graph, where train ratio is 0.5.}
%  \label{cluster_results}
%  \begin{center}
%  \small
%  \begin{tabular}{ccccc}
%    \hline
%    { }&\multicolumn{2}{c}{\bfseries{ \scriptsize Flickr}}&\multicolumn{2}{c}{\bfseries{\scriptsize Youtube}}\\
%    \hline
%    { }&Macro-F1 &Micro-F1&Macro-F1 &Micro-F1\\
%
%    \hline
%    \small PBG & 0.225	&0.387 	&0.295 	&0.406 	\\
%
%    \small DistDGL &0.205 	&0.378 	&0.283 	&0.403 	\\
%
%    \small KnightKing &0.239    &0.386 &0.285 	&0.402 	\\
%
%    \small DistGER  &\bfseries{0.277} &\bfseries{0.409}&\bfseries{0.298} &\bfseries{0.417}\\
%
%  \hline
%\end{tabular}
%\end{center}
%\end{table}
%

\begin{figure}
  \centering
  \includegraphics[width= 3.2 in]{./Figures/Dist_sampling_training_Mpad_efficiency.eps}
  \caption{\small {(a) Random walk efficiency, (b) training efficiency, (c) \# cross-machine messages, (d) random walk efficiency for {\sf MPGP} (ours) and workload-balancing scheme ({\sf KnightKing})}}
  \label{Dist_efficiency_sampling_training_MPGP}
\end{figure}

\begin{table*}[t!]
\begin{minipage}{0.275\linewidth}
\centering
\renewcommand\arraystretch{1.2}
\captionof{table}{\small Performance evaluation of partitioning for {\sf DistGER} and Competitors } %{\sf PBG} and {\sf DistDGL}
\label{Partition_sechme_overhead}
\begin{scriptsize}
\begin{tabular}{ccccc}
    \multicolumn{5}{c}{{\bfseries (a) Partitioning time for {\sf DistGER} and competitors }} \\
    \hline
    {\bf graph} & {\sf PBG} & {\sf DistDGL} & {\sf DistGER}\\
                &           & ({\sf METIS}) &  ({\sf MPGP}) \\
    \hline
    {\sf FL} & 383.28 s & 127.72 s & \bfseries{15.96 s} \\
    {\sf YT} & 349.15 s & 116.30 s & \bfseries{13.56 s} \\
    {\sf LJ} & 458.52 s & 425.19 s & \bfseries{36.42 s} \\
    {\sf OR} & 2662.62 s & 2761.25 s &\bfseries{294.68 s}\\
    {\sf TW} & 22 hour s & $>$ 1 day &\bfseries{9 hours}\\
    \hline
%    \multicolumn{5}{c}{} \\
    \multicolumn{5}{c}{{\bfseries (b) Evaluation of {\sf Parallel MPGP} }} \\
    \hline
    {\bf graph} & {\sf Streaming} & {\sf Partitioning} & {\sf Walking}\\
    \hline
  %  {\sf MPGP}   &DFS+deg  & 9 hours & \bfseries{575.22 s} \\
    \multirow{2}{*}{\sf LJ} &DFS+deg & 21.86 s & \bfseries{23.78 s} \\
           & BFS+deg & \bfseries{21.25 s} & 24.79 s \\
    \multirow{2}{*}{\sf OR} &DFS+deg & \bfseries{151.29 s} & 77.12 s \\
           & BFS+deg & 156.37 s & \bfseries{46.55 s} \\
    \multirow{2}{*}{\sf TW} &DFS+deg & \bfseries{1940.65} s & 683.81 s \\
           & BFS+deg & 2034.21 s & \bfseries{590.36 s}
\end{tabular}
\end{scriptsize}
\end{minipage}%\hfill
\quad
\begin{minipage}{.3\linewidth}
    \centering
    \includegraphics[width= 2.5in, height = 1.45 in]{./Figures/Dist_Mpad_streaming_vertex_time.eps}%
    \captionof{figure}
      {\small The distribution of local computations and cross-machine communications for different streaming orders on {\em LiveJournal}. The top table reports their running times for partitioning and random walks
        \label{Dist_MPaD_streaming}
      }
\end{minipage}%\hfill
\qquad
\begin{minipage}{.37\linewidth}
    \centering
    \includegraphics[width= 2.5 in, height = 1.45 in]{./Figures/Dist_generality_table_HuGE+.eps}%
    \captionof{figure}
      {\small Generality of {\sf DistGER} vs. {\sf KnightKing}. The bars show random walk efficiency ($-R$) and training efficiency ($-T$) for {\sf Deepwalk} ({\sf DW}), {\sf node2vec} ({\sf n2v}) and {\sf HuGE+}. The top table shows the ratio $\frac{\text{{\em AUC} for {\sf DistGER}}}{\text{{\em AUC} of {\sf KnightKing}}}$, with {\sf DW} and {\sf n2v}, task: link prediction
        \label{Dist_generality}
      }
\end{minipage}
\end{table*}


\subsection{Efficiency due to Individual Parts of DistGER}
\label{sec:individual}
\spara{Random walk and training efficiency.}
To evaluate the system design of {\sf DistGER} (\S \ref{sec:DistGER}, \S \ref{sec:learning}),
we first compare the efficiency of random walks and training with those of {\sf KnighKing} and {\sf HuGE-D}.
%For fair comparison, the running times that we reported for {\sf KnightKing} and {\sf HuGE-D} exclude the
%time of vocabulary table construction, since it is a serial process in {\sf Pwode2vec}, while {\sf DistGER}
%pipelines the construction during random walks.
For random walks (Figure~\ref{Dist_efficiency_sampling_training_MPGP}(a)),
{\sf DistGER} significantly outperforms {\sf KnightKing} and {\sf HuGE-D} on all our graph
datasets, achieving an average speedup of $3.32\times$ and $3.88\times$, respectively.
Although {\sf HuGE-D} implements information-oriented random walks on {\sf KnightKing},
due to additional computation and communication overheads during on-the-fly information
measurements (\S \ref{sec:HUGED}), its efficiency can be lower than that of {\sf KnightKing}.
We also notice that the random walk lengths ($L$) and the number of random walks ($r$) reduce (on average)
63.2\% and 18\%, respectively, in our information-oriented random walks, compared to {\sf KnightKing}'s
routine random walk configuration.
%which supports the traditional
%random walk methods. %To provide a straightforward adaptation for the information-oriented approach, DistGER leverages the incremental information-centric computation mechanism to mitigate the redundant computation and high communication cost in HuGE-D, then it achieves an average speedup of $3.32\times$ and $3.88\times$ in random walk procedure compared to KnightKing and HuGE-D.

Another benefit of information-centric random walks is that it generates concise and effective corpus to improve 
training efficiency. Compared to {\sf KnightKing}, {\sf DistGER} achieves $17.37\times$-$27.95\times$ acceleration
in training over all our graphs. Next, considering the same corpus size, we compare the training efficiency of {\sf Pword2vec} and {\sf DSGL}
(trainer in {\sf DistGER}). Figure~\ref{Dist_efficiency_sampling_training_MPGP}(b) shows that {\sf DSGL} achieves $4.31\times$ average speedup
compared to {\sf Pword2vec}. We also notice that the average throughput (number of nodes processed per second) for {\sf DSGL} is up to 49.5 million/s,
while that of {\sf Pword2vec} is only up to 16.1 million/s. These results indicate that our distributed {\sf Skip-Gram} learning model (\S \ref{sec:learning})
is more efficient than {\sf Pword2vec}.
%
%\begin{figure}
%  \centering
%  \includegraphics[width= 2.5 in]{Dist_Mpad_efficiency.eps}
%  \caption{\small (a) exhibits the number of cross-machine computation for DistGER on workload-balancing and MPGP partition scheme, respectively, and (b) shows the random walk time of DistGER on the two schemes, where y axis is in log-scale.}
%  \label{Dist_efficiency_MPaD}
%\end{figure}

\spara{Partitioning efficiency.} Considering the %large number cross-machine computing introduced by the
randomness inherent in random walks, the partitioning scheme is critical to overall efficiency. %of the distributed framework.
%To validate the efficiency of our multi-proximity-aware streaming graph partitioning (MPGP),
%we deploy the workload balancing scheme used in KnightKing and MPGP on DistGER,
%respectively,
%and report the number of cross-machine computations during the random walk procedure for the two schemes.
%We also present the efficiency performance of MPGP compared with the workload-balancing scheme.
For {\sf DistGER},
Figure~\ref{Dist_efficiency_sampling_training_MPGP}(c) exhibits that our multi-proximity-aware streaming graph partitioning ({\sf MPGP})
significantly reduces (avg. reduction $45\%$) the number of cross-machine messages than the workload-balancing partition of {\sf KnightKing}
on five graphs. Moreover, it improves the efficiency by 38.9\% for the random walking procedure
(Figure~\ref{Dist_efficiency_sampling_training_MPGP}(d)) over the same set of walks.
We report in Table~\ref{Partition_sechme_overhead}(a) the time required for graph partitioning in competing systems,
where {\sf DistDGL} uses the {\sf METIS} algorithm \cite{METIS_1998} for partitioning.
The results show that {\sf MPGP} performs partitioning with very little overhead in most cases, and
the partitioning efficiency is on average $25.1\times$ faster than competitors.
In Figure~\ref{Dist_MPaD_streaming}, we exhibit the distribution of local computations and cross-machine communications
on four machines for different streaming orders, and the top table reports their running times for partitioning and random walks.
For sequential {\sf MPGP}, we find that the {\sf DFS+degree}-based streaming order (\S \ref{sec:partition}) is more efficient than other streaming orders,
and it also strikes the best balance between cross-machine communications reduction and workload balancing.
Table~\ref{Partition_sechme_overhead}(b) exhibits the performance evaluation of {\sf parallel MPGP} on the small- ({\em LiveJournal}), medium- ({\em Com-Orkut}) and large-scale ({\em Twitter}) graphs. The results show that {\sf DFS+Degree} in {\sf parallel MPGP} is still the best or comparable in terms of partition time, due to the same reason as stated in our third optimization scheme (\S \ref{sec:partition}). On the other hand, {\sf BFS+Degree} in {\sf parallel MPGP} works the best in terms of random walk time due to preserving the locality of the graph structure (our fourth optimization scheme in \S \ref{sec:partition}).
%as using its streaming order to parallel partitioning can reduce the influence of relevance between each segment.
We ultimately recommend {\sf BFS+Degree} for {\sf parallel MPGP}, since it reduces the partition time greatly, while the random walk time is comparable to that obtained from sequential {\sf MPGP}.
%
%\begin{figure}
%  \centering
%  \includegraphics[width= 3 in]{Dist_Mpad_streaming_vertex_time.eps}
%  \caption{\small The distribution of local computations and cross-machine communications for different streaming orders on {\em LiveJournal}. The top table reports their running times for partitioning and random walks.}
%  \label{Dist_MPaD_streaming}
%\end{figure}
%
%\begin{table}
%\setlength{\abovecaptionskip}{0.cm}
%\setlength{\belowcaptionskip}{-0.cm}
%\newcommand{\tabincell}[2]{\begin{tabular}{@{}#1@{}}#2\end{tabular}}
%  \caption{\small Time execution time (seconds) of the partition scheme in PBG, DistDGL, and DistGER, ``$-$'' means the scheme fails under constrains of computation resource.}
%  \label{Partition_sechme_overhead}
%  \begin{center}
%  \small
%  \begin{tabular}{ccccc}
%    \hline
%    {Graph}&{PBG}&{DistDGL(METIS)}&{DistGER}\\
%    \hline
%    Flickr& 383.28 &127.72 &\bfseries{15.96} \\
%
%    Youtube& 349.15 &116.30&\bfseries{13.56} \\
%
%    LiveJournal& 458.52 &425.19 &\bfseries{36.42} \\
%
%    Com-Orkut& 2662.62 &2761.25 &\bfseries{294.68}\\
%
%    Twitter&78986.85 &$-$&\bfseries{35500.41}\\
    %\hline
%    \multicolumn{5}{l}{* HuGE+ generates the smallest corpus size for training among all methods tested.} \\
%
%  \hline
%\end{tabular}
%\end{center}
%\end{table}
%


\subsection{Generality of DistGER}
\label{sec:generality}
%\begin{figure}
%  \centering
%  \includegraphics[width= 3 in, height= 1.65 in]{Dist_generality_table.eps}
%  \caption{\small Generality comparison for DistGER and KnightKing, %on real-word graphs,
%  The bars display random walk (denoted as $-R$) and training efficiency (denoted as $-T$) for {\sf Deepwalk} (DW) and {\sf node2vec} (n2v), respectively.%, and the y axis is in log-scale.
%  Top table shows the ratio $\frac{\text{{\em AUC} for {\sf DistGER}}}{\text{{\em AUC} of {\sf KnightKing}}}$, both with Deepwalk and node2vec, respectively, considering link prediction.}
%  \label{Dist_generality}
%\end{figure}
%
%Since our proposed information-oriented random walk framework DistGER aims to address the redundant computations and high communication cost introduced by the effectiveness measurement of the generated walking information in distributed setting, it provides a good systematic support for the information-centric approach HuGE as shown by the previous experimental results. A natural question arises: can DistGER also support the traditional random-walk-based methods?
To demonstrate the generality of {\sf DistGER}, we deploy {\sf Deepwalk} \cite{DeepWalk_2014}, {\sf node2vec} \cite{node2vec_2016} and {
\sf HuGE+} \cite{HuGE+_2022}
on {\sf DistGER}. While the original {\sf Deepwalk} and {\sf node2vec} follow
traditional random walks, in {\sf DistGER} the walk length and the number of walks are decided via information-centric measurements.
Next, we also deploy both {\sf Deepwalk} and {\sf node2vec} on {\sf KnightKing} which supports the routine configuration random walk.
Figure~\ref{Dist_generality} illustrates that {\sf DistGER} reduces the random walks time by 41.1\% and 51.6\% on average for
{\sf Deepwalk} and {\sf node2vec}, respectively. For training, {\sf DistGER} is on average $17.7\times$ and $21.3\times$ faster than {\sf KnightKing}+{\sf Pword2vec}
for {\sf Deepwalk} and {\sf node2vec}, respectively.
Moreover, we also show the {\em AUC} ratio of {\sf DistGER} and {\sf KnightKing}, considering {\sf Deepwalk} and {\sf node2vec}, for link prediction.
% tasks, where performing multi-label classification on Flickr graph  and link prediction on other graphs, the accuracy metric for the two task are $Miro-F1$ and $AUC$ score, respectively, it can be found from
Our results depict that {\sf DistGER} has comparable (in most cases, higher) {\em AUC} scores, while it improves the efficiency significantly
even for traditional random walk-based graph embedding methods.
{\sf HuGE+} is an extension of {\sf HuGE}, and it uses the same {\sf HuGE} information-centric method to determine the walk length and the number of walks per node. Figure~\ref{Dist_generality} exhibits the compatibility of {\sf HuGE+} on {\sf DistGER} via its general API.
\section{ADLs Dataset with Visual Privacy Features}
\label{sec:dataset}

This section describes the dataset we used to explore the effect of image resolution on humans' and machines' performance on activity recognition and visual privacy awareness tasks.

\subsection{Constructing the ADLs Dataset}

In order to evaluate the model in realistic environments, we used the publicly-available PA-HMDB51 dataset for privacy-preserving activity recognition~\cite{wu2019framework}. This dataset consists of about 355 minutes and 51 types of human activity videos collected from realistic environments with various visual privacy features annotated. 

In this paper, we mainly focus on activities of daily living (ADLs) in a smart home scenario. Therefore, three of our authors selected the qualified videos from the PA-HMDB51 dataset together with the following requirements. 1) The video represents a home environment. 2) All authors agreed that the main character conducted the same kind of activities. 3) All authors felt comfortable to publish the video online. For instance, due to the internet policy, we only chose men's or kids' topless videos in this study. Then, we divided the human activities in the PA-HMDB51 dataset into five basic kinds of activities of daily living (ADLs) including \textit{functional mobility}, \textit{feeding}, \textit{intimacy}, \textit{entertainment}, and \textit{personal hygiene}. Finally, we obtained 46, 30, 22, 37, and 16 minutes of videos for functional mobility, feeding, intimacy, entertainment, and personal hygiene, respectively.

We randomly split the PA-HMDB51 dataset into a training dataset, a validation dataset, and an evaluation dataset, which accounts for 90\%, 5\%, and 5\%, respectively. Considering the difference of the video duration in the PA-HMDB51 dataset, we divided all the videos into 2-second clips for later training and evaluation without affecting the judgment of the video content. Therefore, there are 226 clips of the videos in the evaluation dataset, with 69, 45, 33, 55, and 24 clips for functional mobility, feeding, intimacy, entertainment, and personal hygiene, respectively.

\subsection{Labeling the Privacy Features}

Based on the user study results presented in section~\ref{sec:study1}, we annotated each frame and each clip in our dataset with privacy features including \textit{nudity}, \textit{identifiable face}, \textit{valuable property}, and \textit{relationship}. 
Since privacy features may vary during the video clip, for example, even in the same video clip, the visibility of a person's face may be different, we provided both \textit{frame-level} and \textit{clip-level} labels of for each video in our dataset. First of all, we annotated all of the privacy attributes on each frame of different clips. Then, we annotated each clip according to the frames in the clip for later user studies and machine experiments. The detailed description of both frame-level and clip-level labels are listed below.
\begin{figure*}[!ht]
    \centering
    \begin{subfigure}{0.475\textwidth}
        \centering
        \includegraphics[width=1\textwidth]{figures/source/annotation_1.pdf}
    \end{subfigure}
    \begin{subfigure}{0.475\textwidth}
        \centering
        \includegraphics[width=1\textwidth]{figures/source/annotation_2.pdf}
    \end{subfigure} 
    \begin{subfigure}{0.475\textwidth}
        \centering
        \includegraphics[width=1\textwidth]{figures/source/annotation_3.pdf}
    \end{subfigure}
    \begin{subfigure}{0.475\textwidth}
        \centering
        \includegraphics[width=1\textwidth]{figures/source/annotation_4.pdf}
    \end{subfigure} 
    \caption{Examples of the annotated frames in our dataset.}
    \label{fig:annotation_example}
    \Description{Examples of the annotated frames in our dataset. The four samples are shown in the upper left, upper right, lower left, and lower right of the figure, respectively. Each example is shown with a frame on the left, and labels of types of ADLs, nudity, identifiable face, valuable property, and relationship on the right.}
\end{figure*}
\begin{itemize}
    \item \textbf{Nudity}. The nudity label of each frame included three types that are \textit{naked or semi-naked (topless or bottomless)}, \textit{fully clothed}, and \textit{no person}. A clip is labeled as \textit{naked or semi-naked (topless or bottomless)} if at least one frame of the clip is labeled as \textit{naked or semi-naked (topless or bottomless)}. Otherwise, the clip is labeled as \textit{fully clothed} in a similar way. If every frame is labeled as \textit{no person}, we will finally label the clip as \textit{no person}.
    \item \textbf{Identifiable face}. If more than 70\% of a human face is visible, we consider the frame to contain an identifiable face. Therefore, each frame is labeled as \textit{yes}, \textit{no}, and \textit{no person}. A clip with more than one frame labeled as \textit{yes} is labeled as \textit{yes}, otherwise \textit{no}. A clip with every frame labeled as \textit{no person} is then labeled as \textit{no person}. 
    \item \textbf{Valuable property}. We only consider safe box, jewelry, watch, ring, and cash as valuable properties. Each frame is labeled as \textit{yes}, \textit{no}, and \textit{no person}. We label clips with at least one frame labeled \textit{yes} as \textit{yes}, otherwise \textit{no}. Clips with no person on any frame are labeled as \textit{no person}.
    \item \textbf{Relationship}. We consider the relationship of all the people presented in the video. There are four types of labels for each frame: \textit{intimate relationship}, \textit{non-intimate relationship}, \textit{only one person}, and \textit{no person}. A video clip is labeled as \textit{intimate relationship} if at least one frame of the clip is labeled as \textit{intimate relationship} and the frames labeled as \textit{intimate relationship} are no less than those labeled as \textit{non-intimate relationship}. Otherwise, a clip is labeled as \textit{non-intimate relationship} in a similar way. A clip with only one person presented is labeled as \textit{only one person} and labeled as \textit{no person} if there is no person existing in the clip.
\end{itemize}

Examples of the annotated frames in the dataset are demonstrated in Figure~\ref{fig:annotation_example}. Each frame was annotated by at least three of our authors and then cross-checked.
\section{Conclusion and Future Work}
\label{sec:conclusion}

We have presented a novel neural network that successively learns shape sketch and extrusion without any expensive annotations of shape segmentation and labels as the supervision.
%Without the guidance of sketch labels, 
Our approach is able to learn smooth sketches, followed by the differentiable extrusion to reconstruct CAD models that are close to the ground truth. 
We evaluate SECAD-Net using diverse CAD datasets and demonstrate the advantages of our approach by ablation studies and comparing it to the state-of-the-art methods. 
We further demonstrate our method’s applicability in single-image CAD reconstruction. 
Additionally, the CAD shapes generated by our approach can be directly fed into off-the-shelf CAD software for sketch-level or cylinder primitive-level editing. 

% We tested SE-Net on ABC dataset and Fusion 360 dataset. Quantitative results demonstrate that SE-Net can efficiently reconstruct 3D CAD shapes. Qualitative results show that our model can learn fine 2d sketches without any associated ground-truth.


% We propose SE-Net, a network that successively learns shape sketch and extrusion in an unsupervised manner. The CAD shapes generated by the network can be directly sent to off-the-shelf CAD software for sketch-level or cylinder primitive-level editing. SE-Net can be reconstructed to generate smooth sketches and the reconstruction effect is due to the current state-of-the-art, including supervised methods. Additionally, our method is the first to learn sketches from raw shapes without the guidance of sketch labels.

In future work, we plan to extend our approach to learn more CAD-related operations such as \emph{revolve, bevel, and sweep}. %using neural methods. 
Besides, we find that current deep learning models perform poorly on datasets with large differences in shape geometry and structure. %structural and topological variations
Therefore, another promising direction is to explore how to improve the generalization of neural networks and enhance the realism of the generated shapes by learning structural and topological information.

%% if specified like this the section will be committed in review mode
%\acknowledgments{.}

%\bibliographystyle{abbrv}
\bibliographystyle{abbrv-doi}
%\bibliographystyle{abbrv-doi-narrow}
%\bibliographystyle{abbrv-doi-hyperref}
%\bibliographystyle{abbrv-doi-hyperref-narrow}

\bibliography{main}
\end{document}

%   - Provide details on the implementation of PACE, especially the optimizer (R1)
%   - Clarify which parts of the technical approach are novel vs implementation of prior work, highlighted in a system diagram (R1)
%   - Clearly call out the 15-sec limitation and discuss potentials and challenges to improve it (R1)
%   - Discuss how the complexity and density of the environment can affect the approach (R1, R2)
%   - Add examples from prior work explaining scenarios when pre-captured motion sequences are important (R2)
%   - Fix typo and graph visibility (R2, R4)

%% These few lines make a distinction between latex and pdflatex calls and they
%% bring in essential packages for graphics and font handling.
%% Note that due to the \DeclareGraphicsExtensions{} call it is no longer necessary
%% to provide the the path and extension of a graphics file:
%% \includegraphics{diamondrule} is completely sufficient.
%%
\ifpdf%                                % if we use pdflatex
  \pdfoutput=1\relax                   % create PDFs from pdfLaTeX
  \pdfcompresslevel=9                  % PDF Compression
  \pdfoptionpdfminorversion=7          % create PDF 1.7
  \ExecuteOptions{pdftex}
  \usepackage{graphicx}                % allow us to embed graphics files
  \DeclareGraphicsExtensions{.pdf,.png,.jpg,.jpeg} % for pdflatex we expect .pdf, .png, or .jpg files
\else%                                 % else we use pure latex
  \ExecuteOptions{dvips}
  \usepackage{graphicx}                % allow us to embed graphics files
  \DeclareGraphicsExtensions{.eps}     % for pure latex we expect eps files
\fi%

%% it is recomended to use ``\autoref{sec:bla}'' instead of ``Fig.~\ref{sec:bla}''
\graphicspath{{figures/}{pictures/}{images/}{./}} % where to search for the images

\usepackage{microtype}                 % use micro-typography (slightly more compact, better to read)
\PassOptionsToPackage{warn}{textcomp}  % to address font issues with \textrightarrow
\usepackage{textcomp}                  % use better special symbols
\usepackage{mathptmx}                  % use matching math font
\usepackage{times}                     % we use Times as the main font
\renewcommand*\ttdefault{txtt}         % a nicer typewriter font
\usepackage{cite}                      % needed to automatically sort the references
\usepackage{tabu}                      % only used for the table example
\usepackage{booktabs}                  % only used for the table example
%% We encourage the use of mathptmx for consistent usage of times font
%% throughout the proceedings. However, if you encounter conflicts
%% with other math-related packages, you may want to disable it.
\usepackage{amsmath}
\usepackage{hyperref}
