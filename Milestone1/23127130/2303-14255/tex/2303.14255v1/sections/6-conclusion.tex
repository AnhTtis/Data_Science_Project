\section{Conclusions, Limitations, and Future Work}

In this paper, we propose PACE, a novel method for generating placements for virtual agents into a dense cluttered 3D scene with accurately modeled human-scene interactions and tailored motion. Our key insight is that tailoring the motion of the virtual agents is essential to creating agent-scene pairings that are natural-looking, especially in dense or cluttered environments. Human raters preferred PACE agent placements over existing methods including PAAK and POSA. Additionally, PACE agent-scene pairs are quantitatively rated as more physically plausible than existing methods.

\textbf{Limitations and Future Work.} Note that PACE does not \textit{always} create natural placements. There are still cases where a given motion is simply too long or large to fit in a given environment. Additionally, if there is no semantic match the placement can look unnatural. For example, a sitting motion will not look right in a scene without any chairs. PACE is currently limited to static 3D scenes, although future work can explore the use of dynamic scenes. Our optimization method can sometimes miss good placements as it relies on a grid of initial placements before optimizing the best ones. For example, initial placements with heavily penalized penetrations could become the best available with further optimization to the motion sequence. We limited this due to processing time but more compute could enable improved agent-scene pairings from PACE. This leads us to the largest limitation of PACE, the amount of processing time it requires. Future work can look into placement proposal to limit the number of initial placements PACE must attempt to optimize. This would work similar to the way a region proposal network works for object detection models. An additional interesting future direction would extend PACE to model human-human interactions when placing multiple animations into a scene. More testing with real-world AR and VR headsets is needed to determine how users respond to the virtual agents added by PACE.

\textbf{Acknowledgements.} This material is based upon work supported by the National Science Foundation Graduate Research Fellowship Program under Grant No. DGE 1840340. It is also supported by ARO Grant W911NF2110026  and U.S. Army Cooperative Agreement W911NF2120076. Any opinions, findings, and conclusions or recommendations expressed in this material are those of the author(s) and do not necessarily reflect the views of these funding agencies.