% In this section, we evaluate the performance of our cable routing and assembly pipeline. Specifically, we investigate the following three questions: 1) what are the benefits of tactile sensing in cable manipulation tasks, 2) how reliable are the individual tactile primitives, and 3) can the long-horizon cable manipulation task be accomplished reliably and when will our method fail?

In this section, we evaluate the performance of our cable routing and assembly pipeline. Specifically, we investigate the benefits of tactile sensing in cable manipulation tasks and the reliability of our designed tactile primitives.

% With tactile reading, we can complete the long-sequence routing task in most cases as we are less susceptible to accumulated drift due to sensor noise. 

% In this section, we evaluate the performance of our cable routing pipeline. Specifically, we show that:
% %describe our experimental setup and show that:
% \begin{enumerate}
%     \item Tactile feedback is essential for completing the cable manipulation task. With tactile reading, we can complete the long-sequence routing task in most cases.
%     % Tactile-guided motion primitives perform much better than not using any tactile signals, highlighting the usefulness of incorporating tactile sensors for low-level motion control for manipulating deformable objects. 
%     \item Our pipeline generalizes well to different board configurations.
% \end{enumerate}

\subsection{Robot setup}

Our full setup is shown in Fig.~\ref{fig:setup}. We use a UR5e 6-DoF robot arm  switching between position control and hybrid force-position control modes. %for the manipulation. operating in the position control mode for the cable following, peg pivoting, slot weaving primitives and in hybrid force-position mode for the connector insertion primitive.
To grasp the cable, we use a Weiss Robotics WSG-50 gripper, which operates in an indirect force control. Mounted on the gripper is a GelSight R1.5 sensor along with a custom designed and 3D printed finger which has design features to prevent accidental cable drops. These design features include an inner surface with an elastomer pad with adequate friction, a shape which bulges slightly towards the tip, and a beak on one side which helps picking up thin cables from the flat surface and holds it from falling down. We also use a Microsoft Azure Kinect camera to capture the entire taskboard and acquire RGBD data for task parsing. 
% We fixed one end of the cable similarly as in the NIST Assembly Challenge, and a USB2.0 type A connector on the free end. 

% \helen{do we need implementation details here? or in appendix? like parameters used, etc}\wenzhen{You don't need too much details about implementation but definitely need to explain something clear}

\begin{table}[t]
\centering
\resizebox{\linewidth}{!}{ 
\begin{tabular}{lcccc}
\toprule \addlinespace[0.15cm]
%  & \multicolumn{4}{c}{Primitive}\\
%  \cmidrule(lr){2-5}
Primitive & \begin{tabular}[c]{@{}c@{}}Cable\\Following\end{tabular} & \begin{tabular}[c]{@{}c@{}}Peg\\Pivoting\end{tabular} & \begin{tabular}[c]{@{}c@{}}Slot\\Weaving\end{tabular} & \begin{tabular}[c]{@{}c@{}}Connector\\Insertion\end{tabular} \\
\midrule
w/o Tactile (Baseline) & 43/50 & 28/50 & 32/50 & 0/50 \\
w/ Tactile (Ours) & \textbf{50/50} & \textbf{50/50} & \textbf{50/50} & \textbf{48/50}\\

\bottomrule
\end{tabular}
}

{\small}
\caption{\textbf{Success rate of individual primitives.} For each primitive, we calculate the success rate with and without tactile sensing. 
%We find that tactile sensing becomes vital for complex primitives such as connector insertion.}
\label{tab:successrate}} 
\end{table}

%preventing failure modes such as connector misalignment and cable dropping/entanglement
\subsection{Motion primitives with and without tactile sensing}

We demonstrate the robustness of the tactile-guided motion primitives by comparing the success rate against a baseline without tactile sensing. The baseline applies the same trajectory which is generated analytically and does not use tactile sensing for adapting it.

We run 50 trials for each primitive (see Fig.~\ref{fig:primitives}) with varying fixture positions, and present the results in Table~\ref{tab:successrate}.A trial is considered success if the robot is able to properly perform the corresponding primitive on that fixture. For all four primitives, our tactile-guided approach significantly outperforms the baseline without tactile sensing. The connector insertion primitive shows this the most clearly: the baseline has a success rate of 0\% because the robot cannot align the orientation of the USB head with the connector, whereas with the tactile sensing, the controller can reorient the connector pose to align with the socket. The other baseline primitives fail due to the inability to estimate contacts and maintain tension in the cable.  




\begin{table}[t]
\centering

\begin{tabular}{@{}lcc@{}}
\toprule \addlinespace[0.15cm]
          & Success Rate & Failure Modes     \\ \midrule
GT + Tactile (Ceiling) & 53/60 & A(3), B(2), C(2), D(0) \\
%GT + No Tactile (Baseline) & 0/25 & A(0), B(0), C(0), D(25) \\ 
GT + No Tactile (Baseline) & 0/60 & A(0), B(2), C(10), D(48) \\
Vision + Tactile (Ours)    &   \textbf{51/60}   & A(4), B(3), C(2), D(0) \\ 
\bottomrule
\end{tabular}
{\small
\caption{\textbf{Success rate of entire pipeline.} We summarize the success rate of the whole task across 6 different board configurations with 10 trials each. Failure Modes: A (Connector insertion failed), B (Cable got stuck), C (Cable escaped from previous fixture), D (Accumulated error in motion)}\label{tab:full_traj}}

\end{table}
\begin{figure}[t]
    \centering

    \includegraphics[width=\linewidth]{figures/board_cfg2.png}
    \caption{\textbf{Taskboard configurations.} We evaluate cable routing and assembly on each of these 6 boards 10 times.}
    \label{fig:boardconfigs}
\end{figure}

\subsection{End-to-end cable routing and assembly}
% \wenzhao{Adds a failure mode analysis discussion, possibly with a table}
% Generalization over various board configurations
% \begin{itemize}
%     \item Accuracy of configuration estimation
%     \item Success rate of trajectory generation
% \end{itemize}
We analyze the reliability of our full pipeline over various board configurations. First to evaluate the performance of our vision pipeline described in Section~\ref{sec:visual_perception}, we compare our approach (Vision + Tactile) against an oracle baseline that uses the ground truth (i.e. human-specified) board configuration (GT + Tactile). We consider this to be an upper bound on performance.
To evaluate the effectiveness of using tactile sensing, we experimented with a baseline without tactile sensing but using the ground truth board configuration (GT + No Tactile).
% and automatically recovered (Vision + No Tactile).

We compare the success rate and failure modes by evaluating each approach across 6 different board configurations (Fig.~\ref{fig:boardconfigs}), with 10 trials each, summarized in Table~\ref{tab:full_traj}. 
We label the trial as a success if the entire task was completed.
The oracle which uses ground truth board configurations (53/60) performs only slightly better than our Vision + Tactile pipeline (51/60), suggesting that our vision pipeline can effectively parse the board task configuration. 
We find that the baseline without tactile sensing fails catastrophically due to the accumulated motion error, typically after the second or third fixture.  While individual primitives could be completed without tactile sensing, although with limited success, as shown in Table~\ref{tab:successrate},  we find that chaining the primitives reliably is difficult. 
This happens due to the lack of tactile feedback which corrects the trajectory while following the parsed path. The robot may also experience failures beyond its direct control, such as cable entanglement in fixtures or cable detachment from previously completed fixtures. 
%The reason is that each primitive assumes a small feasible set of preconditions, which are often no longer met after the execution of the previous primitive without tactile sensing, e.g., cable has too much slack after pivoting or weaving.
%In contrast, with tactile sensing, each primitive effectively has a larger feasible set of preconditions as it continuously estimates the cable state and re-orients the cable accordingly. 
During complex operations such as pivoting, weaving, and insertion, tactile signals are crucial for inferring the cable-environment interaction state to online adapt the trajectory. As summarized in the failure modes, our method occasionally fails because of the cable getting stuck or slack, and connector insertion remains the most challenging operation. We postulate such ``global'' state changes cannot be sufficiently captured by ``local'' tactile signals. Thus we plan to fuse the global visual sensing with local tactile sensing in the future to further improve the reliability.
% \wenzhen{I think you can be a little more specific here, something like ``they typically fail at the second or third fixtures''}


% We run each experiment 25 total trials, with 5 different board configurations, 5 trials each. 
% Given multiple goal board configurations, we start with using our vision pipeline . 


% \begin{center}
% \footnotesize
% \begin{tabular}{ |l|l| }
% \hline
% \textbf{Primitive} & \textbf{Failure Mode}  \\ 
% \hline
% \multirow{1}{*}{Cable Following} & N/A  \\
%   \hline
% \multirow{2}{*}{Pivoting around peg}  & Cable Drop \\
%  & Cable Entanglement  \\
%  \hline
% \multirow{1}{*}{Slot Insertion} & Cable Entanglement   \\ 
% \hline
% \multirow{1}{*}{Connector Insertion} & Connector Misalignment   \\
% \hline
% \end{tabular}
% \break
% \break
% \caption{Table:2 Success rate of Individual Motion Primitives.}
% \label{table:2}
% \end{center}
% \normalsize




