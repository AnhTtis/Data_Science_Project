% \wenzhao{2 sections: tactile sensing, deformable linear object manipulation and assembly}


\begin{figure}[t]
    \centering
    %\smallskip
    \includegraphics[width=\linewidth]{figures/fullsetup.png}
    \caption{\textbf{System setup and task board configuration.} The task board contains 4 types of fixtures, labeled 1-4 in the figure (1 = start, 2 = pivot, 3 = slot, 4 = USB connector). The cable begins at the start and ends being inserted into the USB connector. In between, the cable may be wrapped around the yellow pivots or woven through red slots. A UR5e robot with a GelSight sensor attached to the gripper is used to manipulate the cable. We also have a Azure Kinect camera overlooking the board to capture the board configuration.}
    \label{fig:setup}
\end{figure}
\begin{figure}[t]
    \centering
    
%\smallskip
    \includegraphics[width=\linewidth]{figures/reference_primitives.pdf}
    \caption{\textbf{Task board operations.} Our task is split into 4 operations: 1) cable following (gripping and following along the cable), 2) peg pivoting (wrapping the cable around the vertical peg), 3) slot weaving (threading the cable through the horizontal slot), and 4) connector insertion (inserting the USB head at the end of the cable into the connector). }
    \label{fig:ref_prim}
\end{figure}


\subsection{Tactile sensing for robotics}
The sense of touch, one of the main reasons that imparts dexterous and fine manipulation skills to human hands, has been inspiring robotics researchers since early days. A multitude of tactile sensing technologies \cite{s18040948} have been developed to aid robot manipulation. This includes measurement of features such as shape, texture, and forces. A recent and increasingly common mode of tactile sensing is optical tactile sensors such as GelSight \cite{yuan2017gelsight} \cite{9811832}. 

GelSight has been used for estimating contact shape and pose of objects~\cite{8794298} and manipulating them~\cite{6943123}. The shear and slip forces on a grasped object can also be estimated \cite{8202149}, which can be used to adjust the grasp \cite{8593528}. \cite{9196976} developed manipulation primitives that exploited tactile sensing for reactive manipulation of tabletop objects. \cite{9341006} used tactile exploration to learn objects' physical properties for dynamic manipulation. \cite{kim2022active} uses tactile sensing to estimate contact locations between a grasped object and the external environment for peg-in-hole insertion. End-to-end learning methods using tactile inputs have been developed for manipulation of small objects \cite{tian2019manipulation}. However, all of the above works focus on rigid object state estimation, while we investigate tasks that require continuous state estimation of deformable objects.
% Model-free Reinforcement Learning methods have been developed in \cite{https://doi.org/10.48550/arxiv.2104.01167} to align objects with the help of environment and tactile based peg-in-hole insertion.

% Control systems to use high resolution tactile sensors for tactile servoing has been developed in 
Another line of relevant work is tactile servoing. A neural network based edge detector taking tactile images as input was learned in \cite{8641397}, and was used for contour following. Reinforcement learning approaches for tactile contour following and closing a ziploc bag was developed in \cite{8039205}.
In the spectrum of classical control methods, a tactile servoing controller was designed to track an object edge as well as a cable lying flat on a surface~\cite{li2013control}. In \cite{9353219}, a tactile-based closed-loop velocity controller was developed to regulate the sliding behavior of a grasped object. 
% Tactile data from a tactile matrix sensor pad is used to estimate the cable pose which is then used for switchgear assembly in \cite{pirozzi2018tactile}.
One of the closest work to ours is \cite{she2021cable}, where a GelSight gripper is utilized to enable a closed-loop controller for cable following in free space. In contrast, we tackle a more realistic scenario where the cables rest naturally on a task board, and a more complex task that involves routing and weaving in a constrained workspace.
% In contrast to our goal of deforming the cable to a goal configuration, they focus on following along the cable. The unrestricted free space in which they operate gives them the flexibility to constrain the cable between the horizontal surfaces of the gripper, thereby eliminating the forces of gravity and the risk of cable falling down.

% The most similar work to ours is by She \cite{doi:10.1177/02783649211027233} who designed a GelSight gripper and developed methods for cable manipulation. In contrast to our goal of deforming the cable to a goal configuration, they focus on following along the cable. The unrestricted free space in which they operate gives them the flexibility to constrain the cable between the horizontal surfaces of the gripper, thereby eliminating the forces of gravity and the risk of cable falling down.

%\noindent{\bf Deformable Linear Object Manipulation \& Assembly}

\begin{figure*}[t]
    \centering
    \includegraphics[width=\linewidth]{figures/full_pipeline.pdf}
    \caption{\textbf{Pipeline overview.} We propose an end-to-end framework for cable routing and assembly. 
    \textbf{(a)} Our vision module takes in RGBD images of the goal taskboard with and without the cable to output the task specification. We use color filtering to determine the fixtures' positions and types, Principal Component Analysis (PCA) to determine the orientations and shapes, and Coherent Point Drift (CPD) to detect the cable thus determining the fixture order. \textbf{(b)} The parsed task description (fixture order, type, position and orientation) is mapped to a sequence of parameterized tactile primitives, meanwhile generating an initial reference robot trajectory. \textbf{(c, d, e)} Sequentially, the robot executes each primitive as an individual state machine, where the state transitions are governed by the sensed tactile data. In each state, a parametric trajectory generator is activated to generate the trajectory online, for example, with lines and splines.}

    \label{fig:pipeline}
\end{figure*}

% \wenzhao{In Full Pipeline: replace vision pipeline, robot pipeline to task parsing, trajectory generation, and tactile-guided execution; Input: Goal RGBD image; Add cable to the parsed point cloud; In the parsed fixtures file screenshot, spell out position/orientation, replace numbers with ...; I would suggest layout is something like P112 in the Google slides, open to discussion.}

\subsection{Deformable linear object manipulation and assembly}

Manipulating deformable linear objects such as cables and ropes with robots is generally challenging due to the objects' infinite degrees of freedom. Previous works have attempted to design methods ranging from state estimation, representation learning, motion planning, to end-to-end learning.
% Most previous works involves representation, state estimation, simulating the dynamics or motion planning of deformable linear objects. \helen{cite more papers here}

In an early work~\cite{619320}, the cable deformation due to external forces is estimated using stereo vision, and manipulation techniques are developed to straighten the cable for through-hole insertion. \cite{jin2022robotic} proposes a novel spatial representation between the cable and environment objects for motion planning. In recent years, end-to-end approaches were proposed to learn the deformation model from simulation and manipulate cables to a target shape\cite{Wang_2022}. These works commonly assume the system to be quasistatic and achieves manipulation using repetitive pick and place actions \cite{yan2020self} \cite{nair2017combining}. Few approaches leverage environment contacts when manipulating cables, e.g., leveraging force-torque sensing \cite{feeltension} or visual perception \cite{8851170}. \cite{cableplanning} proposes a task-space planner, which builds a roadmap from predefined tasks and employs a replanning strategy based on a genetic algorithm which executes cable routing using a dual-arm robot. \cite{wireterminalinsertion}  developed a system for insertion of wire-terminal insertion via visuo-tactile methods. \cite{symbolic_state} developed Bayesian state estimation methods to predict symbolic states with predicate classifiers for connector insertion. In this work, we aim to build an entire cable routing and connector insertion system using visual sensing for initial plan generation and then tactile perception to monitor the cable-environment contact state, and adjust the control policy accordingly.



% \noindent{\bf Vision-based state estimation}

% RGB vision based techniques have been used in \cite{9732654} to detect configuration of the cable


% \noindent{\bf Force and Tactile based state estimation} 

% Estimating pose of the grasped cable and the friction forces using GelSight optical tactile sensors are developed in \cite{doi:10.1177/02783649211027233}. Usage of Force-torque sensors to estimate environment contacts are explored in \cite{blindmanipft}

% \noindent{\bf Deformable object manipulation} 

% Manipulating cables in free space using optical tactile sensing is achieved in \cite{doi:10.1177/02783649211027233}. End to end approaches to learn the deformation model and manipulate cables in \cite{Wang_2022}  cloth manipulation, etc


% \noindent{\bf Connector Insertion } 
%  Closed loop motion primitives which use force and vision sensing for inserting connectors have been developed in \cite{525977}. The task of inserting a flexible wire into a hole, by modelling and estimating the plastic deformation on the cable is achieved in  \cite{619320}


