% Manipulating flexible objects such as cables with robotis are largely an open problem still now.

% We propose a framework to estimate the configuration of a cable assembly setup, generate a trajectory and use tactile conditioned motion primitives for completing the task.
% In this paper, we propose a novel end-to-end framework 


Robotic manipulation and planning has seen substantial progress in recent years. However, manipulating flexible objects such as cables and wires remains an open problem \cite{grannen2021untangling, pirozzi2018tactile, jin2022robotic, switchgear}. In particular, tasks such as cable assembly require robot systems to be able to effectively track narrow cables and route them in between obstacles. Because the object of interest is thin, malleable, and subject to severe occlusions, this often requires advanced sensing, planning, and control.

While previous efforts in cord manipulation have considered knot tying \cite{grannen2021untangling}, wire insertion \cite{pirozzi2018tactile}, and (most relevant to us) cable routing \cite{jin2022robotic}, they are limited in task complexity and generalizability. Specifically, simplistic operations such as picking, moving, and placing prevent the robot from handling more complex tasks such as weaving parts through slots~\cite{nistcomp}. In addition, the above mentioned methods often cannot recover from failure due to the open-loop nature of the execution~\cite{jin2022robotic}. The main difficulty lies in how to continuously estimate the cable state, particularly when vision alone suffers from severe gripper/object occlusions, and how to generate motion commands accordingly.

Tactile-guided manipulation is a promising approach, but has only been demonstrated in a simple cable following task \cite{she2021cable}. In this paper, we consider a full task of cable routing and assembly inspired by the NIST Assembly Task Board 3~\cite{nistcomp}. We design a reconfigurable setup with fixtures that require 4 operations covering common tethered object manipulation, as illustrated in Fig.~\ref{fig:teaser}. These tasks have longer horizons, and require reliable generalization across configurations.

\begin{figure}[t]
    \centering
    \includegraphics[width=\linewidth]{figures/teaser.pdf}
    \caption{\textbf{Goal-conditioned cable manipulation using visual-tactile perception.} Given an RGBD image of a goal configuration, our system parses the task, generates reference trajectories, and applies a sequence of tactile-guided motion primitives to accomplish a cable routing and assembly task.}
    \label{fig:teaser}
\end{figure}

To solve this problem, we propose to use visual perception for task understanding, i.e., a task description file is automatically extracted from a goal configuration image. The parsed task file is then mapped to a sequence of tactile-guided motion primitives. Notably, we leverage tactile sensing to design a library of primitives to continuously estimate the cable state and manipulate it in a closed-loop manner. Concretely, four primitives (cable following, pivoting, weaving, and insertion) are defined, each as a state machine, where the state transitions are governed by tactile perception. State-dependent controllers are then activated sequentially to realize the desired primitive behavior. Because of this modular design where a task is parsed via visual perception into primitives and each tactile-guided primitive is task context independent, our approach is able to generalize across different task board configurations with zero adaptation effort. Experiment results demonstrate the effectiveness and generalizability of the tactile-guided motion primitives, as well as our fully integrated pipeline.

Our contributions are summarized as follows. First, we propose a novel integrated solution for cable routing and assembly, where visual perception enables automatic high-level task parsing. Second, we design a library of tactile-guided motion primitives for low-level motion control to accomplish complex cable operations. Third, we provide baselines with and without tactile sensing on a reconfigurable cable routing and assembly task for future research.


% \begin{figure}[t]
%     \centering
%     \includegraphics[width=\linewidth]{figures/teaser_2.png}
%     \caption{\textbf{Goal-conditioned cable manipulation using visual-tactile feedback.} From an RGBD image of a goal configuration, We use vision and tactile-driven motion primitives to manipulate the cable around various fixtures on the taskboard to replicate the human demonstration. The cables are configured around 4 types of fixtures, labeled 1-4 in the figure (1 = start, 2 = peg, 3 = slot, 4 = USB connector). The cable begins at the start and ends inserted into the USB connector. In between, the cable may be wrapped around the yellow pegs or woven through red slots.\wenzhen{I think it's better to remove the labels to make this figure clean. You can add a few pictures of GelSight reading though to emphasize the idea of 'using tactile feedback'}}
%     \label{fig:teaser}
% \end{figure}



% \wenzhao{
% Outline: problem motivation, problem definition, method outline, contribution.\\
% Describing cable manipulation tasks are prevalent, but hard: Shiyu's paper, NIST challenge (https://www.nist.gov/el/intelligent-systems-division-73500/robotic-grasping-and-manipulation-assembly/assembly). List a few previous efforts (e.g., references from Shiyu's paper, Ken Goldberg's recent paper on knots, the paper Wenzhen shared on switchgear assembly). Just mention but no need to elaborate as that would be done in related work. Then summarize they are limited in task complexity and reliable generalization (elaborate a bit), e.g., only routing, cannot tackle or recover from failure for a different cable/task configuration. The main difficulty lies in how to continuously estimate the cable state and generate motion commands accordingly.\\
% Tactile-guided manipulation is a promising approach, but only demonstrated in a simple cable following task (Yu She's paper). In this paper, a full task is considered inspired by NIST task board 3. We designed fixtures that require 4 operations covering common tethered object manipulation, and reconfigurable tasks. The tasks have longer horizon, and require reliable generalization.\\
% To solve this problem, we propose to use visual perception for task understanding and transfer. The parsed tasks are processed by a trajectory generator to generate reference trajectories for the end-effector. Most importantly, multiple tactile-guided motion primitives are designed to continuously estimate the state and follow the trajectory. These can be elaborated with the core components.\\
% To summarize, our contribution is summarized as follows..
% Novel integrated solution for cable manipulation and assembly, where visual perception is used for high-level automatic task parsing and specification, and tactile sensing is used for low-level closed-loop motion control. We also provided a baseline for cable manipulation assembly tasks.}

