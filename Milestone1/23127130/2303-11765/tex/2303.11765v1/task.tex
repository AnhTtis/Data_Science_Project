% \wenzhao{subsection 1, describe the task, task board, possibly reconfigurable; subsection 2, 4 operations required}

% \begin{figure}
%     \centering
%     \includegraphics[width=\linewidth]{figures/taskboard.png}
%     \caption{\textbf{Example task configuration with fixtures.} Here we show an example of a cable configuration for this taskboard. The cables are configured around 4 types of fixtures, starting at the start fixture and ending with the USB connector fixture. In between, the cable may be wrapped around vertical peg fixtures or woven through horizontal slot fixtures. \wenzhao{I think this is redundant to the teaser figure. Maybe add small index numbers in Figure 1 for reference when describing the task.}}
%     \label{fig:taskboard}
% \end{figure}





\subsection{Task}


%  We use vision and tactile-driven motion primitives to manipulate the cable around various fixtures on the taskboard to replicate the human demonstration. 
% \begin{figure}[t]
%     \centering
%     \includegraphics[width=\linewidth]{figures/teaser_2.png}
%     \caption{\textbf{Example task configuration with fixtures.} We use vision and tactile-driven motion primitives to manipulate the cable around various fixtures on the taskboard to replicate the human demonstration. The cables are configured around 4 types of fixtures, labeled 1-4 in the figure (1 = start, 2 = peg/pivot, 3 = slot, 4 = USB connector). The cable begins at the start and ends inserted into the USB connector. In between, the cable may be wrapped around the yellow pegs or woven through red slots.\wenzhen{Since you are putting this figure here, you can remove the label ``GelSight'' and probably ``fixtures''}}
%     \label{fig:taskboard}
% \end{figure}

% \helen{i rewrote this section and shortened it a lot, maybe can re-add more details if necessary}

We aim to solve the cable manipulation problem inspired by the NIST Assembly Task Board 3\cite{nistcomp}. We consider a taskboard that consists of fixtures such as pegs and channels to route and manipulate the cable to a pre-specified configuration. 
% More specifically, given an RGBD image of a taskboard with a goal cable configuration, our algorithm generates the robot plan to replicate the desired cable configuration. 
The taskboard consists of 4 different types of fixtures in various configurations: 1) Start fixture, 2) Vertical pegs / pivots, 3) Horizontal slots, and 4) USB connector.
% \begin{enumerate}
%     \item Start Fixture
%     \item Vertical Pegs / Pivots
%     \item Horizontal Slots
%     \item USB Connector 
% \end{enumerate} 
An example of the taskboard with fixtures with its goal configuration is shown in Fig.~\ref{fig:setup}.


% The cable manipulation problem is inspired and formulated from the NIST taskboard for the flexible cable assembly performance evaluation. The task evaluates competencies in tracking, placing, weaving and manipulation of cables and inserting the ends into various connectors.The taskboard consists of a cable, which is fixed at one end and loose at the other end. The taskboard has fixtures such as pegs and channels which should be used to route to and constrain the cable to the desired  configuration. 

\subsection{Operations} 

To move the cable around the fixtures on the board, we divide the entire task into the following subtasks, as illustrated in Fig.~\ref{fig:ref_prim}.

\subsubsection{Cable following} The robot holds the cable and follows along without dropping it.

\subsubsection{Pivoting around vertical pegs} The robot pivots the cable's heading direction by rotating it around the pegs.

\subsubsection{Weaving through horizontal slots} The robot weaves or threads the cable through the horizontal channel slots that ``lock'' the cable inside. 

\subsubsection{Connector insertion} The robot inserts the USB head at the cable's end into the connector fixture.

% \subsubsection{Pivoting around vertical pegs} The goal of this subtask is to pivot the direction of cables by rotating it around vertical pegs. The pegs have a standoff at a clearance above the taskboard. The cable should be tucked in and constrained under the standoff of the vertical peg.

% \subsubsection{Weaving through slots or channels} This subtask involves weaving or threading the cable through narrow parallel channels/slots which function to constrain the cable. 
% %not sure if to put here or not
% \subsubsection{Connector Insertion} The final subtask is inserting the connector at the end of cable into the peg with the corresponding socket.

% \subsubsection{Cable Following} Grasping a cable and then following along the length of cable while bending it into desired shapes is one of the most basic capabilities a robot must have to successfully manipulate cables. This also involves maintaining the perfect tension in the cable, without which the slack or over-tension in the cable would lead to task failure. We should also ensure that the robot gripper does not drop the cable anytime during the task execution.


% \subsubsection{Pivoting around vertical pegs} The goal of this subtask is to pivot the direction of cables by rotating it around vertical pegs. The pegs have a standoff at a clearance above the taskboard. The cable should be tucked in and constrained under the standoff of the vertical peg.

% \subsubsection{Weaving through slots or channels} This subtask involves weaving or threading the cable through narrow parallel channels/slots which function to constrain the cable. 
% %not sure if to put here or not
% \subsubsection{Connector Insertion} The final subtask is inserting the connector at the end of cable into the peg with the corresponding socket.

% \subsection{Estimating Board Configuration}
% \wenzhao{This can be rephrased and moved to Method section}
% The task board is occupied by four different types of fixtures: the start position,vertical pillars, horizontal slots and the connectors. The system should be able to find the pose of the fixtures and the goal configuration of the cable in a completed taskboard
% \helen{i think defininng the task board makes more sense here than in method?} 
