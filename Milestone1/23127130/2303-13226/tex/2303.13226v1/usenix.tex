% TEMPLATE for Usenix papers, specifically to meet requirements of
%  USENIX '05
% originally a template for producing IEEE-format articles using LaTeX.
%   written by Matthew Ward, CS Department, Worcester Polytechnic Institute.
% adapted by David Beazley for his excellent SWIG paper in Proceedings,
%   Tcl 96
% turned into a smartass generic template by De Clarke, with thanks to
%   both the above pioneers
% use at your own risk.  Complaints to /dev/null.
% make it two column with no page numbering, default is 10 point

% Munged by Fred Douglis <douglis@research.att.com> 10/97 to separate
% the .sty file from the LaTeX source template, so that people can
% more easily include the .sty file into an existing document.  Also
% changed to more closely follow the style guidelines as represented
% by the Word sample file. 

% Note that since 2010, USENIX does not require endnotes. If you want
% foot of page notes, don't include the endnotes package in the 
% usepackage command, below.

\documentclass[letterpaper,twocolumn,10pt]{article}
\usepackage{usenix,epsfig,endnotes}
\usepackage{graphicx}
\usepackage{url}
\usepackage{float}
\usepackage{caption}
\usepackage{subfigure}
\usepackage{stfloats}
% \usepackage{subfigure}
\usepackage{pifont}
% \usepackage{showframe}
\usepackage[export]{adjustbox}
% \usepackage{natbib}
% \usepackage{url}

\usepackage{usenix}

% to be able to draw some self-contained figs
\usepackage{tikz} 
\usepackage{amsmath}

% inlined bib file
\usepackage{filecontents}

\usepackage{etoolbox}
\makeatletter
% \patchcmd{\maketitle}
% 	{\@maketitle}
% 	{\@maketitle\vspace{-7em}}% change the value as needed
% 	{}
% 	{}
% \makeatother

%-------------------------------------------------------------------------------
\begin{filecontents}{\jobname.bib}
% \begin{filecontents}{sample.bib}
%-------------------------------------------------------------------------------
@Book{arpachiDusseau18:osbook,
  author =       {Arpaci-Dusseau, Remzi H. and Arpaci-Dusseau Andrea C.},
  title =        {Operating Systems: Three Easy Pieces},
  publisher =    {Arpaci-Dusseau Books, LLC},
  year =         2015,
  edition =      {1.00},
  note =         {\url{http://pages.cs.wisc.edu/~remzi/OSTEP/}}
}
@InProceedings{waldspurger02,
  author =       {Waldspurger, Carl A.},
  title =        {Memory resource management in {VMware ESX} server},
  booktitle =    {USENIX Symposium on Operating System Design and
                  Implementation (OSDI)},
  year =         2002,
  pages =        {181--194},
  note =         {\url{https://www.usenix.org/legacy/event/osdi02/tech/waldspurger/waldspurger.pdf}}}
\end{filecontents}

%-------------------------------------------------------------------------------
\begin{document}
%-------------------------------------------------------------------------------

%don't want date printed
\date{}

% make title bold and 14 pt font (Latex default is non-bold, 16 pt)
\title{\Large \bf LearnedFTL: A Learning-based Page-level FTL for Improving Random Reads in Flash-based SSDs}

%for single author (just remove % characters)
\author{
{\rm Shengzhe Wang$^{1}$, Zihang Lin$^{1}$, Suzhen Wu$^{1}$, Hong Jiang$^{2}$, Jie Zhang$^{3}$, Bo Mao$^{1}$}\\
$^{1}$Xiamen University, $^{2}$University of Texas at Arlington, $^{3}$Peking University
% \and
% {\rm }\\
% Xiamen Univeristy
% copy the following lines to add more authors
% \and
% {\rm Name}\\
%Name Institution
} % end author


\maketitle

\begin{abstract}
{Guided depth super-resolution aims at using a low-resolution depth map and an associated high-resolution RGB image to recover a high-resolution depth map. However, restoring precise and sharp edges near depth discontinuities and fine structures is still challenging for state-of-the-art methods. To alleviate this issue, we propose a novel multi-stage depth super-resolution network, which progressively reconstructs HR depth maps from explicit and implicit high-frequency information. We introduce an efficient transformer to obtain explicit high-frequency information. The shape bias and global context of the transformer allow our model to focus on high-frequency details between objects, i.e., depth discontinuities, rather than texture within objects. 
Furthermore, we project the input color images into the frequency domain for additional implicit high-frequency cues extraction. Finally, to incorporate the structural details, we develop a fusion strategy that combines depth features and high-frequency information in the multi-stage-scale framework. Exhaustive experiments on the main benchmarks show that our approach establishes a new state-of-the-art.}\\

\textit{This paper is under consideration at Computer Vision and Image Understanding.}


\end{abstract}


\begin{keyword}
Guided depth super-resolution \sep CNN \sep Transformer \sep Multi-scale \sep High-frequency information
\end{keyword}

The advance of Pre-trained Language Models (PLMs) like GPT-3 \cite{brown2020language} and LLaMA \cite{DBLP:journals/corr/abs-2302-13971} has substantially improved the performance of deep neural networks across a variety of Natural Language Processing (NLP) tasks. Various language models, based on the Transformer \cite{vaswani2017attention} architecture,  have been proposed, leading to state-of-the-art (SOTA) performance on the fundamental discrimination tasks. These models are first trained with self-supervised training objectives (e.g., predicting masked tokens according to surrounding tokens) on massive unlabeled text data, then fine-tuned on annotated data to adapt to downstream tasks of interest.  However, annotated data is usually limited to a wide range of downstream tasks, which results in overfitting and a lack of generalization to unseen data.

One straightforward way to deal with this data scarcity problem is data augmentation , and incorporating generative models to perform data augmentation has been widely adopted recently . Despite its popularity, the generated text can easily deviate from the real data distribution without exploiting any of the signals passed back from the discrimination task. In previous studies, generative data augmentation and discrimination have been well studied as separate problems, but it is less clear how these two can be leveraged in one framework and how their performances can be improved simultaneously. \looseness=-1

Generative Adversarial Networks (GANs) \cite{https://doi.org/10.48550/arxiv.1406.2661} are good attempts to couple generative and discriminative models in an adversarial manner, where a two-player minimax game between learners is carefully crafted. GANs have achieved tremendous success in domains such as image generation , and related studies have also shown their effectiveness in semi-supervised learning. However,  in the text field, GANs are difficult to train, most training objectives work well for only one model, either the discriminator or the generator, so rarely both learners can be optimal at the same time. This essentially arises from the adversarial nature of GANs, that during the process, optimizing one learner can easily destroy the learning ability of the other, making GANs fail to converge.

Another limitation of simultaneously optimizing the generator and the discriminator comes from the discrete nature of text in NLP, as no gradient propagation can be done from discriminators to generators. One theoretically sound attempt is to use reinforcement learning (RL), but the sparsity and the high variance of the rewards in NLP make the training particularly unstable \cite{caccia2019language}. 

To address these shortcomings, we novelly introduce a self-consistent learning framework based on one generator and one discriminator: the generator and the discriminator are alternately trained by way of cooperation instead of competition, and the selected samples are used as the medium to pass the feedback signal from the discriminator. Specifically, in each round of training, the samples generated by the generator are synthetically labeled by the discriminator, and then only part of them would be selected based on dynamic thresholds and used for the training of the discriminator and the generator in the next round. Several benefits can be discovered from this cooperative training process. First, a closed-loop form of cooperation can be established so that we can get the optimal generator and discriminator at the same time. Second, this framework helps improve the generation quality while ensuring the domain specificity of generator, which in turn contributes to training. Third, a steady stream of diverse synthetic samples can be added to the training in each round and lead to continuous improvement of the performance of all learners. Finally, we can start the training with only domain-related corpus and obtain strong results, while these data can be easily sampled with little cost or supervision. Also, the performance on labeled datasets can be further boosted based on the strong baselines. As an example to demonstrate the effectiveness of our framework in the text field, we examine it on four downstream text generation benchmarks, including AFQMC, CHIP-STS, QQP, and MRPC. The experiments show that our method significantly improves over standalone state-of-the-art discriminative models on zero-shot and full-data settings.

Our contributions are summarized as follows,

$\bullet$ We propose a self-consistent learning framework in the text field that incorporates the generator and the discriminator, in which both achieve remarkable performance gains simultaneously.

$\bullet$ We propose a dynamic selection mechanism such that cooperation between the generator and the discriminator drives the convergence to reach their scoring consensus.

$\bullet$ Experimental results show that the generator in our framework can continuously adjust its generation samples based on the performance of downstream tasks, while the discriminator can outperform the strong baselines.



\section{Background and Motivation}
\label{background}


\subsection{Demand-based Page-Level FTLs}
\label{two-one}


Address translation is a vital function in FTL, which searches the physical addresses of flash memory for incoming requests. There exist several mapping schemes such as page-level mapping, block-level mapping, and hybrid mapping~\cite{lee2008last,lee2006fast,bez2003introduction}.
% Address Translation is a vital function in FTL since all requests accessing the flash memory need to undergo address translation.
Since the flash page is the basic unit for read/write operations, page-level mapping can handle requests flexibly and performs well. However, this fine-grained mapping scheme requires a huge DRAM memory to accommodate its mapping table. For example, suppose an SSD has a 10TB capacity with a 4KB page size, and each entry of the LPN-PPN mapping is 8B, the SSD requires 20GB of DRAM memory to store the LPN-PPN mapping of 2.5 billion entries. This huge DRAM memory consumption, unfortunately, is impractical for enterprise SSDs.

Block-level mapping~\cite{kang2006superblock} and hybrid mapping~\cite{lee2006fast,lee2008last} addressed the space issue by compressing the mapping table. In these schemes, the address mappings are organized at the granularity of a flash block, leading to a significantly lower mapping space overhead. However, block-level mapping has a limitation, that is, data stored in a block must have contiguous LPNs. Flash pages can only be written to a fixed location in the flash block. Consequently, these mapping schemes exhibit poor writing performance. 


To strike a good balance between write performance and DRAM memory, demand-based page-level FTLs (DFTL)~\cite{gupta2009dftl} is proposed. Specifically, DFTL uses a selective cache solution to only buffer frequently accessed mappings into SSD memory to exploit workloads' temporal locality, thus reducing memory usage without compromising performance. Figure~\ref{three_schemes}(a) illustrates the general structure of DFTL. It stores the whole mapping table in multiple flash pages, called \textbf{translation pages}. DFTL contains two data structures in SSD memory. \textbf{Cached Mapping Table (CMT)} stores mapping information for frequently accessed flash pages while \textbf{Global Translation Directory (GTD)} records the physical location of translation pages in flash memory. For requests that miss from CMT, DFTL imposes a high miss penalty. In particular, the SSD controller needs an additional flash read to fetch the missing mapping from the translation page. A read request may generate two flash reads for data and metadata, which is called \textbf{double reads}.

Several demand-based page-level FTLs have been proposed to address double reads by exploiting workload locality characteristics. Examples include TPFTL~\cite{zhou2015efficient}, HCFTL~\cite{chen2019hcftl}, and ZFTL~\cite{wang2012zftl}. Among them, TPFTL is a well-known FTL that utilizes both temporal and spatial locality. It proposes a workload-adaptive loading policy to prefetch mappings to CMT based on the request length. This approach improves the hit ratio of CMT and significantly alleviates double reads in workloads with a strong locality. 


\subsection{Performance Impact of Double Reads}
Despite the effectiveness of demand-based page-level FTLs in reducing memory usage, their efficiency is limited to workloads with high locality. This becomes problematic in some modern applications with random accesses. Therefore, the ability of FTL to solve double reads under random workloads is of great importance. 

\begin{figure}[t]
    % \setlength{\abovecaptionskip}{0pt}
    \subfigure[Read throughput]{
        \begin{minipage}[t]{0.46\linewidth}
            \centering
            \includegraphics[scale=0.82]{photos/seq_rand.pdf}
            \label{seqrand}
        \end{minipage}%
    }
    \subfigure[CMT hit ratio]{
        \begin{minipage}[t]{0.46\linewidth}
            \centering
            \includegraphics[scale=0.80]{photos/motivation/hit_ratio.pdf}
            \label{breakdown}
        \end{minipage}
    }
    \caption{The performance of a FEMU-emulated SSD under sequential reads and random reads.} 
    \label{first-test}
    % \vspace{-15pt}
\end{figure}

To investigate whether demand-based FTLs can handle random workloads, we evaluate the sequential and random read performance of TPFTL driven by FIO~\cite{FIOmisc} stress testing tool. As shown in Figure~\ref{seqrand}, regardless of the variation in the number of threads, the performance of random reads consistently falls short compared to sequential reads (i.e., up to 60\% degradation). Figure~\ref{breakdown} shows the CMT hit ratio under different threads. Under random reads, although TPFTL adopts a prefetching strategy, it can only predictively prefetch PPNs near one PPN. However, the two following requests in random reads may be far apart. As a result, the prefetching strategy becomes ineffective, incurring a very low CMT hit ratio.


Increasing the size of the CMT is a straightforward solution to improve the random read performance. However, this approach remains ineffective due to cache contention. Figure~\ref{diff_cmt_space} illustrates the changes in the CMT hit ratio of TPFTL when increasing CMT space. Even when the CMT space expands to 50\% of the total page mappings, the hit ratio only improves slightly to 25.9\%. It is clear that contention for the CMT will exist unless the CMT can accommodate the majority of the mappings. Consequently, regardless of the practical capacity of the CMT, the prefetched mappings will be frequently replaced, leading to a low CMT hit ratio. 

\begin{figure}[t]
    \centering
    % \setlength{\abovecaptionskip}{0em}
    \includegraphics[scale=0.84]{photos/motivation/different_ratio.pdf}  
    \caption{The hit ratio of TPFTL under different CMT space.}
    \label{diff_cmt_space}
    % \vspace{-10pt}
\end{figure}

The above experiments and analysis demonstrate that the selective cache solution of demand-based FTL cannot handle random reads. Since a random read request may access any LPN in the entire address space, an efficient solution is to place as many mappings as possible in the small capacity of SSD memory. The mapping table compression scheme mentioned earlier is the only approach that meets these criteria. However, this approach adversely impacts SSD write performance. Thus, to improve the random read performance, we need a new solution to compress the mapping table without degrading SSD write performance. 


\subsection{Learned Index and LeaFTL}
\label{two-four}
\noindent \textbf{Learned Index.}
Recent studies on \emph{learned index}~\cite{kraska2018case,ALEX,ferragina2020pgm,li2021finedex,Tsunami,ROLEX} have demonstrated its potential for compressing the mapping table since it has a high compression rate and low write limit. The learned index only requires the LPN-PPN mappings to be ordered and builds lightweight models for key-position mappings. A model with several parameters can calculate hundreds of data locations, thus reducing memory consumption.
Figure~\ref{fig-principle} illustrates the workflow of the learned index. Building a learned index model only requires two steps: training an approximate model (usually linear models) over the key-position mappings and identifying the maximum error between the fitted model and the actual values. With the simple model and maximum error, the needed value can be found in the error interval [$y-error$, $y+error$], where $y$ is the predicted position by the approximate model. 

\begin{figure}[t]
    \centering
    % \setlength{\abovecaptionskip}{0pt}
    \includegraphics[scale=0.9]{photos/motivation/WorkFlow.pdf}
    \caption{The workflow of the learned index.}
    \label{fig-principle}
    % \vspace{-20pt}
\end{figure}

\noindent \textbf{LeaFTL~\cite{sun2023leaftl}.}
Ideally, if the learned index could completely replace the existing mapping table structure and index all mappings in memory, the double-read problem could be solved. Recently, LeaFTL has taken this approach. The primary motivation behind LeaFTL is to utilize the learned indexes to replace the current mapping table, thereby reducing the DRAM memory overhead to store the mapping table.

Figure~\ref{three_schemes}(b) illustrates the structure of LeaFTL. LeaFTL uses a learned segment design for learned indexes, and each learned segment has four parameters \emph{\textbf{[S, K, L, I]}}, expressed as a model $PPN = LPN * K + I, LPN\in[S, S+L]$. In LeaFTL's configuration, one learned segment can index up to 256 mappings. For learned segments that are not 100\% accurate (denoted as approximate segments), LeaFTL conceals the error interval in the Out Of Band (OOB) area of each flash page. When the model predicts a wrong PPN, LeaFTL reads the error interval from the OOB of the mispredicted flash page and finds the correct PPN, then LeaFTL can read the correct PPN to access data. With this approach, each misprediction requires 2 flash reads.

Since the learned index cannot be updated unless retrained, LeaFTL adopts the idea of a Log-Structured Merge-tree to ensure the timeliness of the learned segments. LeaFTL allocates a small area in SSD internal memory, called \textbf{data buffer}, to buffer newly written data (up to 2048 pages). When the data buffer is full, LeaFTL sorts all data by their LPNs and then writes them to flash pages of continuous PPNs. After that, LeaFTL groups these mappings according to the translation pages they belong, and each group trains a newly learned segment. Then all the learned segments are flushed to the corresponding translation pages. In each translation page, the learned segments are organized in a log-structured mapping table (LSMT). The newly created segment is inserted into the top layer. If one layer has overlapped segment, LeaFTL will migrate the old segment to the next layer. 
Since the log-structured design brings space amplification (In our evaluations, LSMT can only reduce the space to 10\%-15\% of the original mapping table, which is still too large to be fully stored in memory), LeaFTL continues to use the idea of CMT and only caches the most frequently used learned segments into memory. 

\subsection{Challenges in Learned Indexes/LeaFTL}
While LeaFTL can reduce the size of the mapping table by several times, it, unfortunately, fails in improving the read performance. We take an in-depth analysis and observe multiple key challenges in Learned Index/LeaFTL.

\begin{figure}[t]
    \centering
    % \setlength{\abovecaptionskip}{0pt}
    % \setlength{\abovecaptionskip}{5pt}
    \includegraphics[scale=0.9]{photos/motivation/triple_read.pdf}
    \caption{The workflow of triple reads in LeaFTL~\cite{sun2023leaftl}.}
    \label{fig-ptriple}
    % \vspace{-15pt}
\end{figure}

\noindent \textbf{Challenge \#1: Accuracy of learned indexes.} The accuracy of learned indexes directly determines the efficiency of address translation. LeaFTL is a purely learned index based address translation scheme and replaces the mapping cache of DFTL/TPFTL with a model cache. Thus, mispredictions of learned indexes will bring \textbf{double reads} (one for error interval in OOB and one for data) in LeaFTL. LeaFTL uses a linear regression model~\cite{kraska2018case,li2021finedex,APEX} that can only express \emph{PPN=LPN*K+B}. As a result, if the LPN-PPNs in the model buffer are not linear, part of the requests may experience double reads.

Moreover, LeaFTL even causes \textbf{triple reads} owing to its model cache design. Figure~\ref{fig-ptriple} illustrates the workflow of triple reads in LeaFTL. When an LPN fails to hit any model in the model cache, it initiates a translation read to find the corresponding model from NAND flash. However, as the model in LeaFTL is an approximate one, the predicted PPN may be wrong. After sending a second flash read to access the wrong flash page, this request has to find the correct PPN via the error interval stored in OOB. Finally, this request reads the correct PPN to access data with a third flash read. The workflow of triple reads indicates that the miss penalty in LeaFTL is much higher than double reads in DFTL.

Considering the fact that the accuracy of learned indexes cannot reach 100\%, the problems of double reads and triple reads will have a significant impact on the performance of LeaFTL. Figure~\ref{leaf_rand} illustrates the normalized throughput of TPFTL and LeaFTL under FIO~\cite{FIOmisc} random reads. LeaFTL exhibits a 29\% lower throughput compared to TPFTL. Figure~\ref{stat} shows the fraction of double reads and triple reads during random reads. Triple reads and double reads account for 43\% and 52\%, respectively. These results demonstrate that the double reads and triple reads make LeaFTL completely unable to handle random reads.


\begin{figure}[t]
    % \setlength{\abovecaptionskip}{0pt}
    \subfigure[Random read]{
        \begin{minipage}[t]{0.46\linewidth}
            \centering
            \includegraphics[scale=0.75]{photos/motivation/rand_leaf.pdf}
            \label{leaf_rand}
        \end{minipage}%
    }
    \subfigure[Multi-read count statistics]{
        \begin{minipage}[t]{0.46\linewidth}
            \centering
            \includegraphics[scale=0.7]{photos/motivation/rand_stat.pdf}
            \label{stat}
        \end{minipage}
    }    
    \caption{The performance results under random reads.}    
    \label{second-test}
    % \vspace{-15pt}
\end{figure}


Besides affecting random workload, double and triple reads also have a negative effect on workloads with high locality. Figure~\ref{leaf_filebench} shows the performance comparison between TPFTL and LeaFTL under three Filebench~\cite{Filebench} workloads. The performance of LeaFTL is equal to or even worse than that of TPFTL. Figure~\ref{hit_ratio_filebench} shows the cache and model hit ratio under \textbf{webserver} workload (read-intensive). In this context, the cache hit ratio of LeaFTL simply indicates that the model cache contains the corresponding model for the queried LPN, and it does not mean that the correct PPN has been calculated. Due to the space efficiency of learned indexes, the corresponding model of an LPN can be easily found in the model cache, making the model cache hit ratio high. However, there are instances where models experience mispredictions, leading to a significant number of requests requiring double reads. Consequently, the proportion of LPNs that are successfully hit in the model cache and accurately predicted by the model is significantly lower than the LPNs hit in the CMT of TPFTL.
%the number of double reads induced by mispredictions is greater than the number of double reads induced by CMT misses in TPFTL. 
Therefore, TPFTL performs much better than LeaFTL under workloads with high locality. This experiment indicates that when dealing with locality-based workloads, using direct mapping in the cache is more reliable and efficient than using models.


The above analysis and experiments all indicate that handling mispredictions caused by incompletely accurate learned index models directly affects SSD performance.

 
\noindent \textbf{Challenge \#2: Conflict between the linear model and access parallelism.} 
A key requirement for training the linear model in learned indexes is sorted LPN-PPN mappings. In LeaFTL, after the model buffer is sorted with LPNs, LeaFTL needs to allocate contiguous PPNs for these LPNs. However, modern SSDs are highly dependent on internal parallelism so that multiple flash blocks can be accessed simultaneously across separate flash chips~\cite{hu2011performance,ParaFS}. To be specific, when a set of LPNs needs to be written to an SSD, these LPNs are written to different parallel units (channels, chips, dies, and planes). Since the parallel units belong to a high hierarchical structure, the PPNs in different parallel units may be far apart. Therefore, assigning contiguous PPNs for sorted LPNs is hard in the parallel writing strategy. 


\begin{figure}[t]
    % \setlength{\abovecaptionskip}{0pt}
    \subfigure[Throughput]{
        \begin{minipage}[t]{0.55\linewidth}
            \centering
            \includegraphics[scale=0.8]{photos/motivation/filebench_leaf_tp.pdf}
            \label{leaf_filebench}
        \end{minipage}%
    }
    \subfigure[Hit ratio in webserver]{
        \begin{minipage}[t]{0.38\linewidth}
            \centering
            \includegraphics[scale=0.8]{photos/motivation/file_hit.pdf}
            \label{hit_ratio_filebench}
        \end{minipage}
    }    
    \caption{The performance of TPFTL and LeaFTL under workloads with high locality.}    
    \label{first-filebench-test}
    % \vspace{-10pt}
\end{figure}


\noindent \textbf{Challenge \#3: High training overhead.} 
In LeaFTL, model training is performed on the critical write path. It brings two overheads: (1) Performance overhead: The model training includes sorting, parameters fitting, and compaction, which is time-consuming. These operations performed on the critical write path will directly affect write performance. (2) Space overhead: The space overhead happens in random writes. The LPNs of adjacent write requests are dramatically separated. It is difficult for the model buffer to gather these LPNs in LeaFTL. In the worst case, each LPN-PPN mapping in the model buffer becomes an individual learned segment, leading to a huge space overhead.


To sum up, recent advances in the learned index have shown that it can achieve significantly faster lookup speed and index space savings. Motivated by the urgent need to resolve the double-read problem caused by random reads in flash-based SSDs, along with the challenges learned from learned indexes/LeaFTL, we propose LearnedFTL, which utilizes lightweight learned index models in the existing on-demand page-level FTL (TPFTL) to enhance the random-read performance of flash-based SSDs. 
\section{Design}\label{design}
\subsection{LearnedFTL Overview}

The main idea of LearnedFTL is combining the learned index with demand-based FTL, where the demand-based mapping scheme handles locality-based access patterns and learned indexes handle random access patterns. This design allows LearnedFTL to serve all types of workloads efficiently. Figure~\ref{three_schemes}(c) illustrates the system overview of LearnedFTL. 


In LearnedFTL, each request first checks the CMT. If the CMT fails, LearnedFTL queries the corresponding GTD entry and uses the learned index model to predict the PPN. If the prediction is correct, LearnedFTL accesses the predicted PPN directly, thus eliminating the flash double-read operation. If the prediction is inaccurate, LearnedFTL accesses the data by using the original flash double-read method in TPFTL.

Each model in the GTD is called an \textbf{in-place-update linear model}. Each in-place-update model is a piece-wise linear model, and each linear model has adjustable parameters. To guarantee the accuracy of the model predictions (\textbf{Challenge \#1}), each in-place-update linear model is equipped with a bitmap filter, which indicates whether the prediction of a certain LPN is accurate, thus reducing the cost of inaccurate predictions. To obtain contiguous PPNs for sorted LPNs (\textbf{Challenge \#2}), LearnedFTL proposes the virtual PPN (VPPN) representation to convert PPNs from different parallel units into sequential ones. To reduce the space overhead and performance overhead of model training under random writes (\textbf{Challenge \#3}), LearnedFTL proposes a group-based allocation to bring LPNs belonging to the same GTD entry together and proposes two model training strategies, including a computation-free sequential initialization and a model training via GC/rewrite strategy.


\begin{figure}[t]
    \centering
    % \setlength{\abovecaptionskip}{5pt}
    \includegraphics[scale=0.62]{photos/design/inplace_model1.pdf}
    \caption{The structure of an in-place-update model in GTD.}
    \label{fig-model-layer}
    % \vspace{-10pt}
\end{figure}

\subsection{In-Place-Update Linear Model}
\label{in-place-model}

The model layer in GTD is the most critical component in LearnedFTL, as it determines the efficiency of the entire address mapping process. Figure~\ref{fig-model-layer} illustrates the structure of the \textbf{in-place-update linear model} used in LearnedFTL. Since each model is attached to a GTD entry, each model is only used to predict the mappings of the LPN range represented by its attached GTD entry. An in-place update linear model is a piece-wise linear regression model (PLR model), and it consists of two parts: a \textbf{parameter array} $<$k,b, off$>$[N] and a \textbf{bitmap filter}.

In the parameter array, 
Each $<$k,b,off$>$ represents a linear model, including intercept (\textbf{$\mathrm{b}$}), slope (\textbf{$\mathrm{k}$}), and the offset (\textbf{$\mathrm{off}$}) from this PLR model's starting LPN. Given a certain LPN, the offset (\textbf{$\mathrm{off_x}$}) from the starting LPN is calculated first, and then LearnedFTL queries the corresponding linear model \textbf{$\mathrm{<k_n, b_n, off_n>}$} based on the \textbf{$\mathrm{off_x}$}. The PPN can be predicted using the $y=k_n\times(LPN-LPN_{start}) + b_n$.

A bitmap filter is a bitmap, and each bit in the bitmap is associated with an LPN, representing whether an LPN can be accurately predicted (\emph{1} means accurate, \emph{0} means inaccurate). The bitmap is updated during model training (detailed in Section~\ref{three-three}).
With the bitmap filter, the in-place-update linear model offers two significant benefits over the traditional learned indexes: 

(1) \textbf{Accurate predictions}. The bitmap filter can mark which LPNs can make accurate predictions, assisting models to make only accurate predictions.
Figure~\ref{fig-model-predict} illustrates the two different instances of the bitmap filter. For a request with an $LPN_{req1}$ that needs to use the model to predict the PPN, LearnedFTL first checks the corresponding bit in the bitmap and finds the bit is \emph{1}. Then LearnedFTL will perform model prediction to generate the true $PPN_{req1}$. Since this prediction is marked as accurate, LearnedFTL directly uses this $PPN_{req1}$ to access data. For another request with $LPN_{req2}$ whose corresponding bit is \emph{0}, LearnedFTL will perform a double read for this LPN and not use the model to make predictions. With the bitmap filter, LearnedFTL can make only the correct model predictions and avoid miss penalty caused by wrong model predictions. 

\begin{figure}[t]
    \centering
    % \setlength{\abovecaptionskip}{5pt}
    \includegraphics[scale=0.85]{photos/design/predictor.pdf}
    \caption{The workflow of bitmap filter.}
    \label{fig-model-predict}
    % \vspace{-10pt}
\end{figure}

% In addition to filtering wrong predictions, the bitmap filter can assist LearnedFTL to in-place update learned indexes, maintaining the timeliness of learned indexes without incurring additional overhead.

%Figure~\ref{fig-inplace-model} shows the workflow of model in-place update.

\begin{figure}
   \centering
   \setlength{\abovecaptionskip}{5pt}
   \includegraphics[scale=0.65]{photos/design/inplace_update_process.pdf}
   \caption{The workflow of model in-place update.}
   \label{fig-inplace-model}\vspace{-10pt}
\end{figure}

(2) \textbf{The model parameters can be updated as needed}. The bitmap filter offers the ability to control each LPN, making in-place update of the model possible. Figure~\ref{fig-inplace-model} shows the workflow of model in-place update, when we retrain the LPN-PPN mapping for $model_1$ (with $k_1$ and $b_1$), we can directly update the original model in-place. The model in-place update first modifies the slope $k_1$ and then intercepts $b_1$ to the newly calculated value $k_1'$ and $b_1'$. Since the range of new $model_1'$ and the range of $model_2$ have conflict, the $off_2$ of $model_2$ should be increased until it does not conflict with the new $model_1'$. Finally, the bitmap is updated based on the accuracy of the newly generated model. With the in-place-update ability, an in-place update linear model can always maintain a fixed space overhead, avoiding the need for space compaction like the LSMT in LeaFTL.

The data consistency of the in-place-update linear model is guaranteed upon each update. Specifically, for each write request with an LPN, LearnedFTL first checks if the corresponding bit of this LPN in the bitmap is \emph{\textbf{1}}. If so, LearnedFTL will set this bit to \emph{\textbf{0}} to prevent the model from making wrong predictions. 


Since persisting the models to flash upon each update will bring additional writing overhead,
the models are saved to flash follows the GTD saving procedure as the TPFTL and DFTL handle. During a normal power-off, the models are saved in a flash area alongside GTD. This allows us to easily retrieve and use the stored models when the device reboots.
In the event of a power failure, GTD is rebuilt by scanning all translation pages. Models can also be reconstructed from the mapping information within these translation pages, similar to TPFTL and DFTL. The reconstruction won't take much time since the time overhead for model training is minimal, as shown in Figure~\ref{fig-compute-simulation}.


\subsection{Virtual PPN Representation}

We propose virtual PPN representation to address the problem of non-contiguous PPNs caused by SSD internal parallelism. During model training of learned indexes, it is important to allocate contiguous PPNs for contiguous LPNs. However, the pages with consecutive LPNs may be written back to different flash chips, leading to non-contiguous PPNs.
To tackle this problem, LearnedFTL uses a VPPN representation to transform the non-contiguous PPNs scattered across different chips into contiguous ones. Figure~\ref{fig-virtual} shows the translation principle from PPN to virtual PPN. Since the total number of physical flash pages is fixed in an SSD, the PPN is formed in such a way that it represents the hierarchical tree structure of an SSD by the concatenation of address fields representing different levels of the hierarchy from the highest (channel) to the lowest (page) granularity. Because of the commutative law of multiplication, the order of these address fields in PPN can be changed to obey the allocation order. Thus, each physical page retains its unique number, and the new page number will become contiguous according to the allocation order.
\begin{figure}[t]
    \centering
    % \setlength{\abovecaptionskip}{5pt}
    \includegraphics[scale=0.85]{photos/design/virtual_address_principle.pdf}
    \caption{The principle of virtual PPN translation.}
    \label{fig-virtual}
    % \vspace{-10pt}
\end{figure}


Figure~\ref{fig-virtual-example} gives an example of the PPN-to-VPPN translation. In LearnedFTL, the allocation order is \emph{channel, chip, plane, page, and block}, which is the fastest allocation order based on the previous study~\cite{hu2011performance}. For requests with LPNs \emph{1001, 1002, 1003} that are already written to flash-based SSD, their PPNs are \emph{5013631, 6062207, 7110783}, which are not contiguous. However, after the PPN-to-VPPN translation by changing the order of the fields in the address appropriately, LearnedFTL obtains contiguous VPPNs \emph{2105388, 2015389, 2105390} for these LPNs.



The virtual PPN representation allows LearnedFTL to generate contiguous VPPNs for model training when valid pages are written to the flash-based SSDs concurrently. Since the training model is built based on LPN-VPPN mappings, the predicted VPPN needs to be translated back to PPN to obtain the physical flash page.


\begin{figure}
    \centering
    % \setlength{\abovecaptionskip}{5pt}
    \includegraphics[scale=0.85]{photos/design/virtual_address_sample.pdf}
    \caption{An example of PPN-to-VPPN translation.}
    \label{fig-virtual-example}
    % \vspace{-10pt}
\end{figure}




\subsection{Group-based Allocation Strategy}

Since random writes generate requests of non-contiguous LPNs, it's non-trivial to group them together and create a learned index during writes. Fortunately, garbage collection provides the opportunity to rearrange PPNs. Specifically, during GC, LearnedFTL can rearrange one GTD entry's PPNs to consecutive PPNs and then train models over these newly arranged PPNs.


However, the current dynamic allocation strategy used by LeaFTL and TPFTL makes PPN rearrangement difficult. This is because when allocating a flash page for a PPN, dynamic allocation will select the least busy flash chip to allocate pages for optimal parallelism and write efficiency. As a result, the PPNs of a GTD entry will be scattered across various locations. When building a learned model over this GTD entry via GC, LearnedFTL needs to collect the valid pages across multiple flash blocks, and these blocks also may contain PPNs belonging to other GTD entries. As a result, the GC process generates frequent data movement, which significantly increases the complexity and overhead of the model training process. 


To address this PPN rearrangement issue, we propose a \textbf{group-based allocation strategy}. The basic idea is to divide GTD into groups of consecutive entries, referred to as \emph{GTD entry group}. Each group is allocated an exact number of contiguous flash blocks to accommodate all the LPNs of the group. When the flash blocks allocated to a GTD entry group are full, these used flash blocks are replaced by the same number of contiguous empty flash blocks. When there are no empty flash blocks or the cumulative number of flash blocks allocated to this GTD entry group reaches a threshold, GC is performed on the GTD entry group with the most invalid data pages. During GC, LearnedFTL reclaims data blocks by relocating the valid data pages and retrains the learned models for all GTD entries in this group. 


Figure~\ref{fig-group-allocation} illustrates an example of group-based allocation. In this instance, for the convenience of presentation, each GTD entry group contains two entries and needs two contiguous flash blocks to accommodate all its LPNs. Therefore, LPNs \emph{0-1023} belong to group 0 and LPNs \emph{4096-5119} belong to group 4. When a request for data with LPN belonging to group 0 arrives, two contiguous blocks, \emph{blk1} and \emph{blk2}, are allocated to group 0. When a request for data with LPN belonging to group 4 arrives, another two contiguous blocks, \emph{blk110} and \emph{blk111}, are allocated to group 4 to accommodate the required data pages. When group 0 has no free physical pages, another two contiguous blocks, \emph{blk3} and \emph{blk4}, are allocated to this group. If group 0 is selected for garbage collection, all four blocks are collected directly.
 
\begin{figure}[t]
    \centering
    % \setlength{\abovecaptionskip}{0pt}
    \includegraphics[scale=0.85]{photos/design/group_allocation.pdf}
    \caption{An example of group-based allocation.}
    \label{fig-group-allocation}
    % \vspace{-20pt}
\end{figure}

A serious write-amplification concern arises with this group-based allocation strategy: when all GTD entry groups have been written at least once, a few hot GTD entry groups have been written frequently, which causes huge write amplification. To solve the problem, a global counter is associated with each GTD entry group to identify the hot groups by counting available free pages in the group. 


To address the situation where hot GTD entry groups have limited or no free pages, LearnedFTL employs an \emph{opportunistic cross-group allocation} strategy. This approach allows these hot groups to utilize the available free-page spaces within flash blocks belonging to "cold" GTD entry groups that have an abundance of free pages and untrained models. By encroaching into the free-page spaces of the cold groups, LearnedFTL effectively avoids or delays the need for GC operations. Once the amount of encroachment reaches a specific threshold, GC is triggered for both the encroaching (hot) group and the encroached (cold) group. Subsequently, their respective models undergo retraining and training processes. Consequently, this opportunistic cross-group allocation approach not only reduces the frequency of GC and the write amplification caused by GC operations in hot groups but also ensures the early training of models in cold groups.



\subsection{Model Training}
\label{three-three}

To ensure the timeliness of the in-place-update model, LearnedFTL uses two model training strategies. One is sequential initialization, which is used to initialize the model through sequential write requests during data writing. The other is model training via GC, which is used to train the model during garbage collection to achieve higher accuracy.


\subsubsection{Sequential Initialization}
\label{sequential_init}

The main idea of sequential initialization is to update the learned index model in place based on sequential write requests. In many workloads, the I/O size of each request may range from several to tens of flash pages. When assigning contiguous PPNs for each I/O request, these LPN-PPN mappings can be seen as a \textbf{\emph{y=x}} model. Therefore, we can use these y=x models to update the corresponding in-place-update linear model. For each write request, there are four steps in sequential initialization:

\ding{172} \textbf{Obtaining contiguous PPNs.} LearnedFTL first writes the data of this request to the flash memory and obtains contiguous PPNs. After obtaining contiguous PPN, each LPN must check whether the corresponding bit in the bitmap is `1'. If it is, LearnedFTL updates it to `0'. 

\ding{173} \textbf{Generate the linear model.} LearnedFTL builds a \emph{\textbf{y=x}} model on these LPN-PPN mappings. Then LearnedFTL obtains the model's starting LPN ($\mathrm{LPN_{start}}$), ending LPN ($\mathrm{LPN_{end}}$), and length ($\mathrm{L}$). 

\ding{174} \textbf{Check corresponding model.} LearnedFTL locates the corresponding model with $\mathrm{<k, b, off>}$ in GTD by $\mathrm{LPN_{start}}$ and $\mathrm{LPN_{end}}$. After that, LearnedFTL calculates the length $\mathrm{L_{old}}$ of the existing model through the corresponding bitmap. 

\ding{175} \textbf{Update the model.} If $\mathrm{L_{old}}$ is smaller than $\mathrm{L}$, LearnedFTL performs in-place-update to replace the existing linear model with the newly generated linear model.


\subsubsection{Model Training via GC}
\label{model-training}

Since only long write requests will perform sequential initialization, LearnedFTL also proposes model training via GC to obtain a more comprehensive and accurate model.
With the help of group-based allocation, when LearnedFTL performs garbage collection, all valid pages of one GTD entry group can be collected and trained. 
When a GTD entry group needs to perform GC, the whole model training process via GC is divided into four steps:

\textbf{\ding {172} Regulate valid mappings.} First, LearnedFTL reads all the translation pages of this GTD entry group and only keeps the valid translations in memory. Then LearnedFTL sorts the valid translations by their LPNs to make them ordered. 

\textbf{\ding {173} Write valid pages back and obtain PPNs.} After regulating the valid LPNs, LearnedFTL allocates another group of flash blocks to this GTD entry group, then writes the valid pages back to the newly allocated flash blocks to get contiguous PPNs. 

\textbf{\ding {174} Train the learned model.} In this step, each GTD entry in this group will train its in-place-update linear model. For each GTD entry, calculate the offset of PPNs/LPNs from this GTD entry’s starting PPN/LPN. Then, perform greedy linear regression to fit to get the $<$k, b, off$>$ parameters array. 

\textbf{\ding {175} Evaluate the model.} After training the models, Learned FTL will evaluate the model and update the bitmap filter. During this process, each LPN will be inputted into the model. If the predicted PPN is accurate, the corresponding bit will be set to `1'.


\subsubsection{Model Training via Rewrite}

For some scenarios where GC rarely happens, the model training can be integrated into the SSD rewrite process~\cite{cai2015data,maneas2022operational}. The rewrite is a widely used reliability mechanism to reduce retention errors in modern SSDs by periodically reading, correcting, and reprogramming the flash memory. Rewrite happens frequently and is the most significant factor for write amplification~\cite{maneas2022operational}. During SSD rewrite, LPNs of flash pages can be sorted in order so that these pages are written back in contiguous PPNs, which then enables a model to be built and trained on them by LearnedFTL.



\subsection{Cost Analysis}

\label{three-cost}
Though LearnedFTL introduces multiple new components to apply the learned index in the FTL, it only introduces minor computational overhead, illustrated as follows.

\textbf{(1) Write}: For each write request, LearnedFTL incurs two additional operations, one is the \textbf{bitmap check} operation (Section~\ref{in-place-model}) to maintain the consistency of the model, the other is the sequential initialization (Section~\ref{sequential_init}). Both operations are performed in memory, and there are no calculation operations. Thus, the overhead can be ignored.

\textbf{(2) Read}: For each read request with an LPN, LearnedFTL incurs two additional operations when this LPN cannot hit in the CMT. The first operation is a \textbf{bitmap check} to check if this LPN can predict a real PPN. The second operation is a \textbf{model prediction} when the bitmap check is true. For these LPNs, LearnedFTL will use the model to predict the real PPN instead of an extra flash read. The model prediction includes calculating the VPPN with the \emph{y=kx+b} model and translating the predicted VPPN to PPN. 

\textbf{(3) GC}: The model training incurs two computational overheads during the GC period. The first one is \textbf{sorting} all the LPNs within each GTD entry (Step \ding {172} in Section~\ref{model-training}). The second one is \textbf{training} each GTD entry's model (Step \ding{174} in Section~\ref{model-training}).

Our experiments in Section~\ref{overhead-analysis} have detailed evaluations to quantitatively analyze these overheads.


 \section{Performance Evaluation}
%  \vspace{-5pt}
 \label{performance}
%In this section, we first describe the prototype implementation and experiment setup, followed by the extensive evaluation driven by different benchmarks and workloads. Finally, we discuss the overhead in LearnedFTL.

\begin{figure}[t]
    \setlength{\abovecaptionskip}{0em}
    \setlength{\abovecaptionskip}{-0em}
    \subfigure[Memory consumption]{
        \begin{minipage}[t]{0.45\linewidth}
            \centering
            \includegraphics[scale=0.8]{photos/evaluation/space_occupation.pdf}
            \label{fig-space-overhead}
        \end{minipage}
    }
    \subfigure[Computing cost]{
        \begin{minipage}[t]{0.48\linewidth}
            \centering
            \includegraphics[scale=0.8]{photos/evaluation/compute_power.pdf}
            \label{fig-compute-simulation}
        \end{minipage}
    }
    \caption{The memory consumption and computing cost of the additional operations between ARM and X86 processors.}
    
    \label{fig-simu}
\end{figure}

\subsection{Implementation and Experiment Setup}

\textbf{Experiment Setup}: The experiments are conducted on FEMU~\cite{li2018case}. FEMU is a QEMU-based and DRAM-backend SSD emulator that is widely used in recent studies~\cite{li2021loda, zhou2021remap, han2021zns+}. It runs in a machine with two Intel(R) Xeon(R) Gold 5318Y 2.10GHz CPUs and 128GB DRAM. The operating system is Linux with kernel version 5.4.0. The emulated SSD is configured with 32GB logical capacity plus 2GB over-provisioning space and has 64 parallel chips (8 channels and 8 ways per channel). Each flash chip has 256 flash blocks and each flash block has 512 flash pages. The size of a flash page is set to 4KB. The latency of NVMe SSD is 40 $\mu$s for NAND read, 200$\mu$s for NAND write, and 2ms for NAND erase, which are the default settings in FEMU and widely used in the recent flash-based studies~\cite{li2018case,li2021loda,han2021zns+}. Since the SSD rewrite for retention errors is not implemented in FEMU~\cite{maneas2022operational}, we only train models in GC.

LearnedFTL is compared against two representative page-level FTL designs, DFTL~\cite{gupta2009dftl} and TPFTL~\cite{zhou2015efficient}. We also use full-page mapping as a control (denoted as \emph{\textbf{ideal}}, which is considered a performance upper bound). In the experiments, we use both FIO benchmark~\cite{FIO} and real-world applications/traces to evaluate different FTL designs.

\textbf{Prototype implementation}: We implement LearnedFTL by modifying the blackbox mode of the FEMU based on the TPFTL scheme. The CMT has 8192 slots which are about 0.1\% of the total flash pages. According to the allocation strategy and internal parallelism of the SSDs, we group each 64 consecutive GTD entries as a \emph{GTD entry group}. Since the size of a flash page is 4KB and each translation page has 512 LPN-PPN mappings, the GTD has 16384 entries. Each GTD entry group is allocated 64 flash blocks at a time, one for each of the 64 translation pages. For parameter setting in the piecewise linear model, 8 pieces are set by default. 



\begin{figure*}[t]
    \setlength{\abovecaptionskip}{0em}
    \setlength{\abovecaptionskip}{-0em}
    \subfigure[Throughput]{
        \begin{minipage}[t]{0.35\linewidth}
            \centering
            \includegraphics[scale=0.76]{photos/evaluation/ReadAfterRandom1.pdf}
            \label{fig-read-after-read}
        \end{minipage}%
    }
    \subfigure[Model and CMT hit ratio]{
        \begin{minipage}[t]{0.37\linewidth}
            \centering
            \includegraphics[scale=0.76]{photos/evaluation/ReadRandHit.pdf}
            \label{fig-read-rand-hit}
        \end{minipage}
        \vspace{5pt}
    }
    \subfigure[Write amplification]{
        \begin{minipage}[t]{0.23\linewidth}
            \centering
            \includegraphics[scale=0.76]{photos/evaluation/WA.pdf}
            \label{fig-wa}
        \end{minipage}
    }
    
    \caption{The performance results of FIO benchmark under 64 threads (D: DFTL, TP: TPFTL, LD: LearnedFTL, I: ideal FTL).}
    
    \label{double-read}
\end{figure*}

Since the previous demand-based FTLs, such as DFTL and TPFTL, are implemented on trace-driven simulators, such as SSDsim~\cite{hu2011performance} and Flashsim~\cite{kim2009flashsim}, we incorporate them into the FEMU emulator according to their designs in the papers. For their allocation strategy, we use FEMU's default greedy dynamic allocation strategy. We added and modified about 4,000 LoC to implement these baselines and the LearnedFTL in FEMU. The source code of these prototype implementations, along with our LearnedFTL on the FEMU platform, will be released upon the publication of the paper.

\textbf{Memory consumption}: LearnedFTL adds a model layer in GTD on the basis of TPFTL, which brings additional memory space overhead. In LearnedFTL, each model has three parameters, \emph{start\_PPN}, \emph{<k,b,off>[N]} and \emph{bitmap}. For \emph{start\_PPN}, LearnedFTL stores it as a \emph{uint64\_t} integer. For \emph{<k,b,off>[N]}, each integer is a type of \emph{uint8\_t} to save space. For \emph{bitmap}, each slot is a bit, which makes the bitmap the biggest memory space overhead. To sum up, a model in a GTD entry requires 96 Bytes. For a fair comparison, we increase the CMT size in DFTL and TPFTL to ensure the same memory space overhead as that of LearnedFTL. Figure~\ref{fig-space-overhead} illustrates the total memory consumption of DFTL, TPFTL, and LearnedFTL.


\textbf{Controller computing}: Since LearnedFTL adds some additional computing operations mentioned in Section~\ref{three-cost}, it is necessary to correctly simulate the computing power of the SSD controller. The mainstream SSD controller CPUs are ARM's Cortex A series and Cortex R series. we compared the time consumption of executing the additional operations on the FEMU simulated CPU (X86) and a low-end embedded processor (ARM Cortex-A72) and each operation is at the maximum complexity. Figure~\ref{fig-compute-simulation} shows that the ARM A72 processor even performs better than the X86, which shows that we can use the X86 FEMU simulator to simulate LearnedFTL's computing power.



\subsection{FIO Benchmark}
\label{fio-sec}

We first use the FIO benchmark~\cite{FIO} to evaluate the performance of sequential writes, random writes, sequential reads, and random reads for different FTL designs.  


\emph{\textbf{(1) Read}}: 
For random-read and sequential-read evaluations, we first perform 10 minutes of FIO random writes to warm up the whole SSD, then we perform a corresponding FIO read benchmark for 5 minutes. All the above benchmarks use 4KB I/O size and \emph{psync} I/O engine with 64 threads.

Figure~\ref{fig-read-after-read} illustrates the throughput results for different FTL designs under different access patterns. For random read, LearnedFTL outperforms DFTL and TPFTL by 1.5$\times$ and 1.4$\times$, respectively. For sequential read, LearnedFTL outperforms DFTL and TPFTL by 1.1$\times$ and 1.1$\times$, respectively. Moreover, the performance of LearnedFTL is very close to that of the ideal FTL, achieving about 89.2\% and 96.8\% of the performance of the ideal FTL under random reads and sequential reads, respectively.

% Under random reads, there is no locality to be exploited by DFTL and TPFTL. As a result, the random read incurs two flash reads, one for the LPN-PPN translation and another for the data page. By contrast, LearnedFTL uses the learned models built based on the data location. Some requests can get the needed PPNs by model predictions. Those requests that can get correct PPNs through learned models can directly get the data pages based on the predicted PPNs, thus reducing the LPN-PPN translation-induced flash reads.

To explore the behind reasons, we also recorded the percentage of requests that hit the CMT and the learned models during random and sequential reads. The ideal FTL is used as a control which can be considered as an upper bound since its CMT has a hit ratio of 100\% and infinite space.

Figure~\ref{fig-read-rand-hit} shows that the CMT hit ratios of DFTL and TPFTL designs are almost 0 under random reads. The reason is that random reads show no locality, which makes the cache replacement policy fails to capture the access pattern. By contrast, LearnedFTL utilizes the low-cost learned models to index most of the mappings, achieving 55.5\% hit ratios in the models, which reduceing 55.5\% extra flash translation reads. As a result, LearnedFTL significantly improves the random-read performance over DFTL and TPFTL. 

Under sequential reads, LearnedFTL still outperforms DFTL and TPFTL. Although DFTL and TPFTL respectively achieve 61\% and 80\% hit ratios on CMT, their CMT hit ratios are quite different in a single-threaded environment shown in Figure~\ref{hit_ratio}. The reason is that the throughput of DFTL and TPFTL can be affected by parallel threads which in turn affect the locality of sequential reads (i.e., increasing temporal locality but decreasing spatial locality). Driven by parallel threads, the space contention in the CMT may cause some misses, resulting in a lower CMT hit ratio. 


By contrast, LearnedFTL can resolve contentions effectively. Compared to the mapping entries in the CMT, the learned models have a much longer life cycle because they are only rebuilt during GC periods and will not be evicted as the mapping entries in the CMT. As a result, LearnedFTL achieves a combined CMT-Model hit ratio of up to 90\%, eliminating 90\% of the LPN-PPN double reads. Thus, LearnedFTL achieves the best performance among all FTLs and approaches that of the ideal FTL, which is the upper bound.


% \emph{\textbf{(2) Model accuracy}}:
% Another important metric is the model accuracy under sequential reads and random reads. Figure~\ref{fig-diff-model} shows that LearnedFTL achieves a model accuracy of 55.6\% and 80.0\% in random reads and sequential reads, respectively. The reason for the higher accuracy of sequential reads is that when a read request fails to hit in both the CMT and the learned model, the corresponding LPN-PPN translation flash page will be fetched to the CMT, effectively prefetching LPN-PPN mappings of the subsequent requests in the sequential reads. This reduces the model prediction failures and makes the model more accurate for sequential reads than for random reads.

% During experiments, we also find that LearnedFTL has a much lower CMT hit ratio than DFTL and TPFTL under sequential reads. The reason is that when an LPN does not hit in the CMT but hits in the learned model prediction, LearnedFTL does not insert the corresponding PPN to the CMT for space efficiency consideration because a relatively very small CMT space is allocated to LearnedFTL to compensate for its very large learned-model space. As a result, subsequent requests for this LPN will not hit in the CMT but will hit in the learned models in LearnedFTL.



\emph{\textbf{(3) Write}}:
For the random-write and sequential-write evaluations, we perform a corresponding write benchmark in FIO for 10 minutes from the empty state of SSD, and all the evaluations use 4KB I/O size and \emph{psync} I/O engine with 64 threads. 

Figure~\ref{fig-read-after-read} shows that under random writes, LearnedFTL outperforms DFTL and TPFTL by 1.4$\times$ and 1.2$\times$, respectively, because of LearnedFTL's group-based allocation strategy. Since LearnedFTL selects a GTD entry group for each GC, only the translation pages of this GTD entry group need to be updated. That is, a maximum of 64 translation pages are updated per GC. However, for the dynamic allocation strategies used in DFTL and TPFTL, when the same number of data blocks are collected, the LPN range of flash pages written back may be more than 64 translation pages, which causes additional write amplification. 

Owing to the spatial locality of sequential writes, LearnedFTL performs almost the same as DFTL and TPFTL, by less than 2\%. Unlike the dynamic allocation strategy which selects the blocks with the fewest valid pages in each GC, the group-based allocation strategy performs GC on a group-by-group basis, which may result in more valid pages being written back. But the result shows that this does not have a large impact on sequential write performance.

% Fortunately, the opportunistic cross-group allocation allows the hot GTD entry group in sequential writes to use free pages of cold GTD entry groups, reducing the number of valid pages being written back. As a result, the sequential write performance of LearnedFTL is the same as that of DFTL and TPFTL. 


% \emph{\textbf{(4) Effect of queue depth}}:
% To evaluate the read performance under different queue depths in open-loop workloads, we conduct the performance evaluations under the asynchronous engine \emph{libaio} with different I/O depths. 

% Figure~\ref{fig-iodepth} shows the average read latency of the four FTLs under different I/O depths. We can see that LearnedFTL achieves much lower read latency than both DFTL and TPFTL. As the queue depth increases, LearnedFTL outperforms DFTL and TPFTL by up to 41.5\% and approaches that of the ideal FTL at the I/O depth of 64. This indicates that LearnedFTL performs much better under I/O-intensive workloads and approaches the ideal FTL’s performance. 


% \emph{\textbf{(3) Model accuracy}}:
% The model accuracy is the most important metric that affects the performance of LearnedFTL. During experiments (1), we find that whether the requests are random or not, they significantly impact the accuracy of the learned model. In this experiment, we will further explore the factors that affect the accuracy of the learned models.

% \textbf{Load factor of SSD}: Load factor denotes how much valid data is in SSD. We build the learned model over the LPN-PPN translations of a range represented by a GTD entry. A high load factor means more contiguous segments in this range of LPN-PPN mappings. In random writes of the FIO benchmark, the load factor is influenced by the execution time of random writes. To explore the effects of the load factor, we record the hit ratios of the learned models after random writes at different times. Figure~\ref{fig-exec-accu} shows that as the loading factor in the SSD gets higher and higher, the accuracy of the piecewise function becomes much more accurate.


% \begin{figure}[t]
% \centering
% \setlength{\abovecaptionskip}{5pt}
% \includegraphics[scale=0.82]{photos/evaluation/accu_growth.pdf}
% \caption{The model accuracy and accuracy's growth rate under different number of pieces in piecewise linear model.}
% \label{fig-piece}\vspace{-10pt}
% \end{figure}


% \textbf{Number of pieces in piecewise linear model}: Another factor that affects the model accuracy is the number of pieces of the piecewise linear model. The more pieces in the piecewise function, the higher the accuracy with the higher the space overhead. In order to better measure the accuracy and the space occupation under different numbers of pieces in the piecewise linear model, we record the model accuracy and the growth rate of model accuracy as the number of pieces increases. Figure~\ref{fig-piece} shows that the accuracy of the model increases rapidly at the beginning and then gradually flattens. It is also reflected by the growth rate of model accuracy. The growth rate peaks at 7 pieces and then gradually declines. These results validate that selecting 7 pieces for each piecewise linear regression is the most appropriate.


% \emph{\textbf{(4) Training overhead}}: 
% LearnedFTL integrates model training into the GC process. Therefore, it is important to investigate whether the model training process incurs additional overhead to GC. As mentioned in Section~\ref{three-three}, the model training process mainly adds two steps to GC: (1) sort the valid LPNs of the GTD entry group and (2) train the dynamic piece-wise linear regression models. To obtain the time overhead of these two steps, we record the proportion of the time used in these two steps relative to the total runtime consumed by GC at regular intervals. Figure~\ref{fig-time-overhead} illustrates that the proportion of the total time for sorting and model calculations is very low and does not exceed 2\% of total GC time as the run time grows. As a result, the training impact on the GC is minimal.

\begin{figure}[t]
\centering
\setlength{\abovecaptionskip}{1pt}
% \setlength{\abovecaptionskip}{-0em}
\includegraphics[scale=0.9]{photos/evaluation/gc_frequency.pdf}
\caption{The GC frequency of all FTL designs under FIO random and sequential write benchmarks.}
\label{fig-gc-frequency}
\end{figure}


% \vspace{-4pt}
\subsection{Overhead Analysis}

\label{overhead-analysis}
LearnedFTL added additional operations to SSD. In this subsection, we evaluate the overhead induced by these operations.

\textbf{(1) GC frequency and write amplification}: In LearnedFTL, model training happens in GC, and LearnedFTL proposes group-based allocation to assist model training. Therefore, the GC frequency and write amplification are critical indicators. Figure~\ref{fig-gc-frequency} illustrates the GC frequency of various FTLs in the FIO write evaluations. Although the GC frequency of LearnedFTL fluctuates, the total number of GCs triggered under random writes and sequential writes of LearnedFTL (4188 and 4285) are less than DFTL (4335 and 4572) and TPFTL (4335 and 4304).
Figure~\ref{fig-wa} also shows that the write amplifications of DFTL and TPFTL are larger than LearneFTL in random writes because the group-based allocation requires fewer translation page writes. For sequential writes, although the group-based allocation may write more valid pages, the write amplification of LearnedFTL is comparable to DFTL and TPFTL.
Overall, our group-based allocation can effectively assist the learning models without causing additional GC and write amplification.

\begin{figure}[t]
\centering
\setlength{\abovecaptionskip}{2pt}
    
\includegraphics[scale=0.85]{photos/evaluation/gc_overhead.pdf}
\caption{The time overhead of sorting and training under different running times of FIO random writes (MAX means almost all pages are valid during GC).}
\label{fig-gc-overhead}
\end{figure}

\begin{figure}[t]
    \setlength{\abovecaptionskip}{0em}
    \setlength{\abovecaptionskip}{-0em}
    \subfigure[Write and GC]{
        \begin{minipage}[t]{0.52\linewidth}
            \centering
            \includegraphics[scale=0.8]{photos/evaluation/write_overhead.pdf}
            \label{fig-write-overhead}
        \end{minipage}%
    }
    \subfigure[Read]{
        \begin{minipage}[t]{0.43\linewidth}
            \centering
            \includegraphics[scale=0.8]{photos/evaluation/read_overhead.pdf}
            \label{fig-read-overhead}
        \end{minipage}
    }
    
    \caption{Performance of LearnedFTL with and without additional computing operations (LD: LearnedFTL, ideal LD: ideal LearnedFTL  }
    \label{overhead-two}\vspace{-12pt}
\end{figure}


\textbf{(2) Overhead of training and sorting}: The \textbf{model training} (denoted as training) and \textbf{PPNs-sorting} (denoted as sorting) are two additional operations added to GC. In our implementation, we group 64 GTD entries into one group. In one GC, a maximum of 64 PPN-sorting and model training operations will be triggered for one GTD entry group. As mentioned in Figure~\ref{fig-compute-simulation}, each GTD entry needs about 50$\mu$s for sorting and training in ARM Cortex-A72. The maximum additional overhead incurred by sorting and training are equivalent to about 80 SSD reads ($\mu$s per read), which is negligible since GC for one GTD entry group will incur tens of thousands of SSD reads and writes. Figure~\ref{fig-gc-overhead} show that the time consumption of training and sorting only accounts for up to 3.2\% of the GC execution time.


% Compared to TPFTL and DFTL, as mentioned in Section~\ref{three-cost}, for normal write operations, there are three additional operations in LearnedFTL: \textbf{bitmap check} in SSD write, \textbf{sorting} and \textbf{training} in GC.
GC and write requests do not block each other. To further explore whether they will introduce additional latency, we compare FIO random write performance of LearnedFTL with and without these additional operations. Figure~\ref{fig-write-overhead} shows that their performance difference is nearly negligible (less than 0.7\%), further verifying that the computing overhead of training and sorting is minimal in LearnedFTL.


\textbf{(3) Overhead in read operations}: As mentioned in Section~\ref{three-cost}, only LPNs that can be correctly predicted will perform \textbf{model prediction} (0.65$\mu$s in Figure~\ref{fig-compute-simulation}). This means there is no miss penalty in model predictions. Although there is no miss penalty, if the model prediction takes too long, it will reduce the advantage of reducing double reads. We implement ideal LearnedFTL which put all mappings in memory. For ideal LearnedFTL, each time the bitmap check is yes, it can directly get the PPN through mapping table without model prediction. Figure~\ref{fig-read-overhead} shows that the FIO read performance gap between LearnedFTL and ideal LearnedFTL does not exceed 1\%, demonstrate that the model predictions are lightweight.

% LearnedFTL with model prediction outperforms the one without model by 1.4$\times$ and 1.1$\times$ in random reads and sequential reads, respectively. This result shows that it is worthwhile to use a model prediction instead of an additional translation read.



\subsection{Real-World Applications}
We use Filebench~\cite{Filebench} and RocksDB~\cite{RocksDB} as two real-world applications to evaluate the efficacy of different FTL designs. 
%Before each experiment, we first warm up the SSD with FIO's random write to reach a stable performance state.

\begin{figure}[t]
    \setlength{\abovecaptionskip}{0em}
    \setlength{\abovecaptionskip}{-0em}
    \subfigure[Normalized throughput]{
        \begin{minipage}[t]{0.47\linewidth}
            \centering
            \includegraphics[scale=0.77]{photos/evaluation/rocksdb_all_per.pdf}
            \label{rocksdb_throughput}
        \end{minipage}%
    }
    \subfigure[CMT and model hit ratio]{
        \begin{minipage}[t]{0.48\linewidth}
            \centering
            \includegraphics[scale=0.77]{photos/evaluation/rocksdb_hit.pdf}
            \label{rocksdb_hit_ratio}
        \end{minipage}
    }
    % \vspace{-10pt}
    \caption{Performance results of RocksDB with one thread (D: DFTL, TP: TPFTL, LD: LearnedFTL, I: ideal FTL). }
    
    \label{rocksdb_performance}\vspace{-15pt}
\end{figure}



\textbf{RocksDB}: RocksDB~\cite{RocksDB} is a widely used LSM-Tree-based KV store designed to exploit the parallelism of flash-based SSDs. As we mentioned before, LSM-Trees can merge random writes into sequential ones, but at the cost of relatively poor services to random reads. We deploy RocksDB with EXT4 file system on top of each FTL design and use the \emph{db\_bench} tool of RocksDB with one thread, which is consistent with the previous studies~\cite{yao2020matrixkv, kannan2018redesigning, raju2017pebblesdb}. To evaluate the read performance, we first use the \emph{fillseq} and \emph{overwrite} in $db\_bench$ to write the DB to 80\% full, then we perform \emph{readrandom} and \emph{readseq} in $db\_bench$ to evaluate the read performance in RocksDB. 

In terms of throughput, Figure~\ref{rocksdb_throughput} illustrates that LearnedFTL outperforms DFTL and TPFTL by 1.3$\times$ and 1.3$\times$ in random reads. LearnedFTL also outperforms DFTL and TPFTL by 1.7$\times$ and 1.02$\times$ in sequential reads.


To better understand these results, Figure~\ref{rocksdb_hit_ratio} shows the model and CMT hit ratios recorded in these evaluations. In a single-threaded environment, DFTL does not exploit and thus fails to benefit from the spatial locality, so its CMT hit ratio is zero. TPFTL can achieve an 81\% CMT hit ratio by exploiting the spatial locality. By contrast, since LearnedFTL exploits both the spatial locality and the learned model, it achieves 0.3\% and 46\% CMT hit ratio, 55\% and 41\% model hit ratio in random reads and sequential reads, respectively. 


\begin{table}[!t]   
\small
\begin{center}   
\setlength{\abovecaptionskip}{0pt}
\caption{Filebench configurations.} 
\label{table_filebench_configuration} 
\begin{tabular}{|c|c|c|c|}   
\hline   \textbf{Name} & \textbf{Fileset} & \textbf{Feature} & \textbf{Threads} \\   
\hline   fileserver & 22,500 $\times$ 128KB & write heavy & 50 \\
\hline   webserver & 82,500 $\times$ 16KB & read heavy & 64 \\ 
\hline   randomread & 24,000 $\times$ 1MB & all read & 64  \\
\hline
\end{tabular}   
\end{center}  
\end{table}


\begin{figure}[t]
\centering
\setlength{\abovecaptionskip}{5pt}
\includegraphics[scale=0.9]{photos/evaluation/filebench.pdf}
\caption{The normalized throughput of Filebench.}
\label{fig-filebench}\vspace{-15pt}
\end{figure}

\textbf{Filebench}: \emph{Filebench}~\cite{Filebench} is a highly flexible storage benchmark. We select three workloads that are most widely used in previous studies~\cite{zhou2021remap, han2021zns+, bjorling2021zns}: 
\emph{fileserver} (write heavy), \emph{webserver} (read heavy, less random write), and \emph{randomread} (all random reads). Their configurations, consistent with previous studies~\cite{han2021zns+, zhou2021remap}, are summarized in Table~\ref{table_filebench_configuration}.

Figure~\ref{fig-filebench} shows that LearnedFTL outperforms DFTL by 2.4$\times$, 1.5$\times$, and 1.2$\times$ under the three workloads, respectively. LearnedFTL outperforms TPFTL by 1.05$\times$, 1.1$\times$, and 1.2$\times$ under the three workloads, respectively. 
%Moreover, LearnedFTL's performance reaches 97.3\%, 93.4\%, and 91.3\% of the ideal FTL under the three workloads, respectively. 





\begin{table}[t]   
\small
\begin{center}   
\setlength{\abovecaptionskip}{0em}
\setlength{\abovecaptionskip}{-0em}
\caption{Workload characteristics of 4 traces.}
\label{table_trace_configuration} 
\begin{tabular}{|c|c|c|c|}   
\hline   \textbf{Traces} & \textbf{\# of I/O} & \textbf{avg. I/O size} & \textbf{Read ratio} \\   
\hline   WebSearch1 & 1,055,235 & 15.5KB & 100\%  \\
\hline   Websearch2 & 1,200,964 & 15.3KB & 99.98\%  \\ 
\hline   Websearch3 & 793,073 & 15.7KB & 99.96\%  \\
\hline   Systor17 & 1,253,423 & 10.25KB & 61.6\% \\
\hline
\end{tabular}   
\end{center}\vspace{-7pt}    
\end{table}


\begin{figure}[t]
\centering
\setlength{\abovecaptionskip}{5pt}
\includegraphics[scale=1]{photos/evaluation/tail_latency.pdf}
\caption{The tail latency results under 3 WebSearch traces (WS\# demotes WebSearch\#).}
\label{fig-tail}\vspace{-15pt}
\end{figure}


\subsection{Real-world Traces}
\vspace{-3pt}

We select four traces (Three WebSearch traces and one Systor trace) to evaluate the efficacy of different FTL designs. The three WebSearch traces are read-intensive workloads that are generated from a popular search engine~\cite{Oltp}. The Systor trace is the enterprise storage traffic on modern commercial office VDI for 28 days~\cite{snia-trace-block-io-4928,Lee2017Understanding}. The four traces all have strong locality. For these traces, we pick the busiest periods (20 minutes to 2 hours). Since the WebSearch traces is relatively old, we re-rated 64 times more intense to reflect modern SSD workloads~\cite{li2021loda}. The workload characteristics of the four traces are summarized in Table~\ref{table_trace_configuration}. Before we replay the four traces, we warm up the whole SSD to a steady state with FIO random writes which is consistent with the previous works~\cite{li2021loda}. Since TPFTL exploits both spatial locality and temporal locality, we choose TPFTL as the baseline for the tail latency evaluation.

% We select three WebSearch traces with the strong temporal and spatial locality to evaluate the efficacy of different FTL designs. WebSearch1, WebSearch2, and WebSearch3 are read-intensive workloads generated from a popular search engine~\cite{Oltp}. We use the 1-hour busiest period of the three traces and scale them up to reflect modern SSD workloads, particularly at the cache layer in datacenters~\cite{yadgar2021ssd}. The workload characteristics of the three traces are summarized in Table~\ref{table_trace_configuration}. Before we replay the three traces, we warm up the whole SSD to a steady state with FIO random writes, which is consistent with the previous study~\cite{li2021loda}. Since TPFTL exploits both spatial locality and temporal locality, we choose TPFTL as the baseline for the tail latency evaluation.


Figure~\ref{fig-tail} shows the \emph{P99}, \emph{P99.9} tail latency of TPFTL, LearnedFTL, and ideal FTL driven by the four traces. Under the four traces, compared to TPFTL, LearnedFTL reduces the \emph{P99} tail latency by 5.3$\times$, 7.4$\times$, 6.5$\times$, and 3.0$\times$ respectively, with an average of 4.8$\times$. Moreover, LearnedFTL also significantly reduces the P99.9 tail latency of TPFTL, by up to 13.9$\times$. LearnedFTL’s tail latency in WebSearch2 and WebSearch3 is extremely close to that of the ideal FTL. Although TPFTL can maintain high CMT hit ratios on workloads with strong locality, sporadic double reads still induce high tail latency. By contrast, LearnedFTL's learned model can further reduce these sporadic double reads by accurate PPN prediction, thus reducing tail latency.

%% 首先FTL中的地址映射研究工作,归纳总结三种并说明在此基础上都是基于局部性进行优化改进;
%% 分析局部性的研究对于随机负载下的读没有作用,必然引起二次读降低性能,而随机读又是闪存固态盘中非常重要的一类访问负载;
%% 说明Learned Index的研究进展,包括其在LSM树等存储领域的应用,突出其效应,从而引出是否可以用在闪存的FTL中,也就是本文的工作,是第一个尝试这么做的。我们受到的启发是Learned Index可以有效,希望能带动更多的人来做这个方向和尝试

\section{Related Work}
\label{related}

\vspace{-5pt}

\subsection{Mapping Schemes in FTL Design}\vspace{-5pt}

Existing address mapping schemes in flash-based SSDs can be classified into three categories: (1) page-level mapping, (2) block-level mapping, and (3) hybrid mapping. A survey paper~\cite{FTLSurvey} provides a broad overview of some typical address translation technologies for flash memories~\cite{gupta2009dftl,qin2011two,jiang2011s,wang2012zftl}. Among them, page-level mapping shows the best performance due to fine-grained mapping, but it requires a large mapping table~\cite{zhou2015efficient,chen2019hcftl,zhou2018correlation,mativenga2019rftl,ni2017hash,han2019wal,chen2019beyond,GFTL}. The existing mapping schemes share a common design principle: exploiting the workload locality to selectively cache a small part of mapping entries in DRAM while storing the whole mapping table on flash. As a result, they work well under workloads with strong access locality. However, if the working set is too large or accesses are random, the cache hit ratio will be very low and the performance will degrade significantly due to frequent swapping operations and double reads~\cite{zhou2015efficient}.


%Existing address mapping schemes in flash-based SSDs can be classified into three categories: (1) page-level mapping, (2) block-level mapping, and (3) hybrid mapping. 

%Among all mapping granularities, page-level mapping shows the best performance due to fine-grained mapping, but it requires a large mapping table. Some demand-based page-level FTL schemes were proposed to speed up the address translation with a limited size of DRAM~\cite{gupta2009dftl,qin2011two,thontirawong2014scftl,jiang2011s,zhou2015efficient,chen2019hcftl,wang2012zftl,zhou2018correlation,mativenga2019rftl,GFTL,zhang2019reinforcement,ni2017hash,chen2019beyond}. DFTL~\cite{gupta2009dftl}, the first demand-based page-level FTL, shows a great improvement over hybrid FTL designs. Its subsequent variants, such as TPFTL and HCFTL~\cite{chen2019hcftl}, focus on the optimization of cache replacement and management to further improve the hit ratio of mapping cache and the performance of flash-based SSDs. 

Recent studies also utilize machine learning (ML) techniques to improve the performance of flash-based SSDs~\cite{yang2019reducing,akgun2021machine,kang2017reinforcement,zhang2019reinforcement,LearnedSSD}. For example, Q-FTL~\cite{zhang2019reinforcement} uses a reinforcement learning-driven cache replacement algorithm to adapt and respond to ever-changing I/O streams, but only for specific workloads, i.e., remote sensing datasets. LearnedSSD~\cite{LearnedSSD} accelerates the development of new SSD devices by automating the hardware parameter configurations by utilizing both supervised and unsupervised ML techniques. Yang et al.~\cite{yang2019reducing} reduced GC overhead in SSD by introducing ML techniques to predict the future temperature of data. However, none of them have considered ML-based methods on the address mapping optimizations within flash-based SSDs.


\subsection{Learned Index}
% \vspace{-3pt}

Learned index builds an explicit model of the underlying data to provide effective indexing. It was first introduced in~\cite{kraska2018case} and many learned indexes have been proposed based on the idea, such as PGM index~\cite{ferragina2020pgm}, FITing tree~\cite{FITing}, ALEX~\cite{ALEX}, Flood~\cite{Flood}, Tsunami~\cite{Tsunami}, and FINEdex~\cite{li2021finedex}. They all investigate how to support update capability, provide better worst-case guarantees, and/or efficient index construction for different workloads.

Recently, BOURBON~\cite{Bourbon} shows how to integrate learned index structure for an LSM-based KV store. BOURBON employs greedy piecewise linear regression to learn key distributions and applies a cost-benefit strategy to decide when learning will be worthwhile, thus enabling fast lookup with minimal computation. APEX~\cite{APEX} is a PM-optimized learned index based on ALEX~\cite{ALEX}. APEX retains the benefits of learned indexes while guaranteeing crash consistency on PM and supporting instant recovery and scalable concurrency. Abu-Libdeh et al.~\cite{BigTable} demonstrate how a learned index can be integrated into a distributed and disk-based database system, Google’s Bigtable. Their results also validate that integrating the learned index can significantly improve the end-to-end read latency and throughput for Bigtable.

Inspired by these advancements in the learned index techniques, LearnedFTL is the first study on leveraging the learned index to improve the page-level FTL designs of flash-based SSDs to accelerate the address translation for workloads with random accesses. Different from the aforementioned learned index techniques, LearedFTL exploits the GC characteristics of flash devices to relocate physical flash pages and a virtual PPN representation to satisfy the training requirements of the learned index. Moreover, it eliminates the error interval associated with the traditional learned index with the bitmap prediction filter to reduce read amplification in flash storage.

\section{Conclusion}

% In this work, we propose PGKD to distill the knowledge from high-accuracy GNNs to low-latency MLPs.
% The distillation process is edge-free and the learned MLP students are structure-aware.
% Firstly, we analyze the impact of graph structure~(graph edges) on GNNs.
% Specifically, we categorize the graph edges into Intra-class edges and Inter-class edges and study their impact, respectively.
% Based on the analysis, we design two corresponding losses via class prototypes to transfer the graph structural knowledge from GNNs to MLPs.
% Experiments on popular benchmarks demonstrate the effectiveness of our proposed PGKD.
% Further analysis indicate that PGKD is robust to noisy node features and performs well in different training settings.

% For future work, we would consider to apply PGKD to other graph tasks other than node classification.
% Moreover, generating the prototypes basing on the node representations rather than the class labels would be another interesting topic.

A novel PGKD scheme has been proposed to distill the knowledge from high-accuracy GNNs to low-latency MLPs, wherein the distillation process is edge-free and the learned MLP students are structure-aware. 
Specifically, we analyze the impact of graph structure~(graph edges) on GNNs and categorize them into intra-class and inter-class edges. 
Two corresponding losses via class prototypes are designed to transfer the graph structural knowledge from GNNs to MLPs.
Experiments on popular benchmarks demonstrate the effectiveness of PGKD.
Additionally, we show PGKD is robust to noisy node features, and performs well under different training settings.

For our future work, PGKD will be generalized to other graph tasks beyond node classification. 
Another interesting direction will be to generate prototypes utilizing node representations rather than class labels.


\clearpage
\section*{Limitations}
In PGKD, we adopt the class prototypes to capture graph structural information for MLPs in an edge-free setting.
Subsequently, PGKD requires slightly more computing cost compared to the baseline GLNN.
Meanwhile, the gap between the MLP learned by PGKD and its teacher GNN under the inductive setting is larger than that under the transductive setting, especially on Cora and Penn94 datasets.
More effort to improve the performance under the inductive setting is required underway.


\newpage{
\bibliographystyle{plain}
\bibliography{sample-base}
}

% \theendnotes
\end{document}







