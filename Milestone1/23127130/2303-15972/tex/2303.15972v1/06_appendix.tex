% \section{Margin Checking Algorithm}
% \label{app:margin}
% \begin{algorithm}[H]
% 	\caption{Margin Between Low Confidence (LC) Regions}
%  \label{alg:margin}
% 	\begin{algorithmic}[1]
% 	    \Function {sufficient\_margins}{$\cdot$}
%             \State $i_p, t_p \leftarrow \varnothing,\varnothing$ \Comment{no previous yet}
% 	        \For {$t_n \in [0, T_{\textrm{total}} ]$}
%                 \State $\textrm{start}\_\textrm{LC} \leftarrow !\min(\bconf(t_n)) \And \min(\bconf(t_{n-1}))$
%                 \State $\textrm{end}\_\textrm{LC} \leftarrow \min(\bconf(t_n)) \And !\min(\bconf(t_{n-1}))$
%                 \If{$\textrm{start}\_\textrm{LC}$}
%                     \State $i_c \leftarrow \argmin(\bconf(t))$
%                     \State $T_c \leftarrow t$

%                     \If{$i_p$}
%                         \State $ok \leftarrow \Call{check\_margin}{i_p,T_p,i_c,T_c,\cdot}$
%                         \If{$!ok$}
%                             \State \Return False
%                         \EndIf
%                     \EndIf
%                 \EndIf
%                 \If{$\textrm{end}\_\textrm{LC}$}
%                     \State $i_p \leftarrow \argmin(\bconf(t-1))$
%                     \State $T_p \leftarrow t-1$
%                 \EndIf
% 	        \EndFor
% 	        \State \Return True
% 	    \EndFunction
% 	    \Function{check\_margin}{$i_p,T_p,i_c,T_c, \cdot$}
%             \If{$i_p = i_c$} \Comment{same robot, no issue}
%                 \State \Return True
%             \EndIf
%       \State $done \leftarrow $False
%       \State $t_p, t_c \leftarrow \tau$ \Comment{only times when both robots are running}
%       \While{$! done$}
%           \If{$\conf_{i_c}(t_c)=0 \OR \conf_{i_p}(t_p)=0$}
%           \If{$!\conf_{i_{c}}(t)$} \Comment{worst case is faster}
%             \State $t_c \leftarrow t_c + (\dot{\psi}_{i_c}^{F}(t_c))^{-1}\Delta t_s$
%             \State $t_p \leftarrow t_p + \delta t^{\textrm{resp}}_{i_p}(t_p-t_c, t_p)$
%           \Else \Comment{worst case is slower}
%             \State $t_p \leftarrow t_p + (\dot{\psi}_{i_p}^{S}(t_p))^{-1}\Delta t_s$
%             \State $t_c \leftarrow t_c + \delta t^{\textrm{resp}}_{i_c}(t_c - t_p, t_c)$
%           \EndIf
%           \Else
%           \If{$(t_c-t_p) >0$} \Comment{previous is behind}
%             \State $\delta t_c, \delta t_p \leftarrow \Call{high-conf}{t_c,t_p,i_c,i_p,\cdot}$
%           \Else \Comment{current is behind or the two are equal}
%           \State $\delta t_p, \delta t_c \leftarrow \Call{high-conf}{t_p,t_c,i_p,i_c,\cdot}$
            
%           \EndIf
%           \State $t_p \leftarrow t_p + \delta t_p$
%             \State $t_c \leftarrow t_c + \delta t_c$
%           \EndIf
%           \If{$t_c \ge T_c$}
%             \State \Return False \Comment{possible overlap}
%           \EndIf
%         \If{$t_p \ge T_p$}
%             \State $done \leftarrow \textrm{True}$ \Comment{no overlap}
%           \EndIf
%           \EndWhile
%           \State \Return True
% 	\EndFunction
% 	\end{algorithmic} 
% \end{algorithm}

\section{Evaluation Parameters}
\label{app:parameters}
In this appendix, we provide the values of parameters used for all parts of the method and evaluation. For simplicity, when choosing weights to align demonstrations in the time warping, we used only the Cartesian variables with equal weights (i.e., $x$, $y$, and $z$). The augmented robot state consisted of 11 state variables including the position, orientation (represented as a quaternion), force, valve state, and time warp gradient variable. While the four entries of the quaternion are coupled, our changes in orientation were sufficiently small such that an interpolation and re-normalization procedure was sufficient. The weights for the principal component analysis were based on setting the expected range, $r$, of each state variable. The weight was then calculated as the squared inverse (i.e., $w = \frac{1}{r^2}$) to match the squared units of Equation \ref{eq:variancefromdemos}.

\begin{center}
\begin{table}[H]
\centering
\caption{Parameters used in Method}
\label{table:parameters}
\begin{tabular}{cccc} \toprule
    \multicolumn{4}{c}{\textbf{Corrections from Variance}}\\ \midrule
    \textit{Parameter} & \textit{Value} & \textit{Parameter} & \textit{Value}\\ \midrule
    $r_x$ & $2$ & $r_y$ & $2$  \\
    $r_z$ & $2$ & $r_{q_x}$ & $1$  \\
    $r_{q_y}$ & $1$ & $r_{q_z}$ & $1$  \\
    $r_{q_w}$ & $1$ & $r_{f_x}$ & $30$  \\
    $r_{f_y}$ & $30$ & $r_{f_z}$ & $30$  \\
    $r_{valve}$ & $1$ & $r_{\dot{\psi}_\mathcal{D}}$ & $1$  \\ \midrule
    \multicolumn{4}{c}{\textbf{Bayesian Inference}}\\ \midrule
    \textit{Parameter} & \textit{Value} & \textit{Parameter} & \textit{Value}\\ \midrule
    $\gamma_p$  & $20$ & $\sigma^{2}_{\textrm{MAX}}$ & $0.1$  \\ 
    $\mu_c$ & $0.5$ & $\gamma_c$ & $0.9$   \\
    $\epsilon$ & $1e-6$ \\ \bottomrule
\end{tabular}
\end{table}
\end{center}

In Table \ref{table:parameters}, we report both the ranges used for each state variable in principal component analysis of the variability (i.e., corrections) and the parameters used for the Bayesian Inference. The parameters chosen for the Bayesian Inference directly impact the required number of executions to achieve high confidence. Other values may be more appropriate depending on the application and criticality. The sample rate, $\Delta t_s$, was 0.2 seconds for all demonstrations and executions. All of the parameters were fixed prior to collecting the final demonstrations used in the evaluation.

