\section{Discussion}
\label{sec:discussion}
Our preliminary findings suggest that our proposed method enables users to efficiently share autonomy with multiple robots. We demonstrated how the scheduling algorithm reduces the time to complete a sanding task as new high-confidence times are identified. As the high-confidence samples approached the ground-truth percentage, our example found a schedule with alternating sanding passes. This observation highlights the value of our optimization-based scheduling. The system will always find solutions that respect the optimization constraints (e.g., no overlapping, low-confidence actions) and reduce the total time by either speeding up high-confidence actions or overlapping executions when possible.

\subsection{Limitations \& Future Work}
Our method inherits limitations from LfD and time warping, which means that the demonstrations need to be performed in a similar way. To avoid these assumptions, we are interested in exploring a reward-based structure that can better generalize to heterogeneous demonstrations. We are also interested in scaling our technical methods from two robots to larger fleets and also exploring the practical considerations of controlling larger numbers of agents \cite{chen2014human}. Finally, our preliminary evaluation presented only one example sanding task to highlight the benefits of the proposed method. In the future, we will conduct user studies and explore a range of tasks, including different tasks for each robot.

