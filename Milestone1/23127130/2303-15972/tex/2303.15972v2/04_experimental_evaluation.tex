\section{Preliminary Evaluation}
\label{sec:experimentalevaluation}

\begin{figure}[]
\centering
\includegraphics[width=3.4in]{./figures/experimental_setup_comp.pdf}
\caption{Instrumented LfD tool, experiment task, and prototype system.}
\label{fig:expsetup}
\vspace{-15pt}
\end{figure}

As a preliminary assessment of our method, we designed an experimental task, collected demonstrations, and ran a simulated experiment to assess how our scheduling solution adapts with use of the shared autonomy system. The experimental task was inspired by industrial sanding applications (e.g., aircraft and automotive manufacturing) where portions of the task have variable sanding needs. The task consisted of four sanding passes of gray spray paint on a piece of white acrylic, as shown in Figure \ref{fig:expsetup} (top-right). The first three passes were variable, where 50\% of the piece was painted in randomly chosen sections. The fourth pass was consistently painted (i.e., did not require corrections). During demonstrations, the operator applied more force and moved the tool more slowly over the painted sections and applied very little force and moved quickly when there was no paint. As a result, the learned nominal program \textit{lightly} sanded the first three passes and required corrections by the operator to adjust the abrasiveness. During each demonstration, the tool was moved slowly between passes as a simple way to ensure the task was amenable to a highly-parallel execution (i.e., one robot sanding while the other moved between passes). In practice, other high-confidence actions between passes could enable parallel scheduling, such as changing sanding discs. The authors collected a total of six sanding demonstrations using the instrumented sanding tool shown in Figure \ref{fig:expsetup} (top-left). We collected the pose of the tool, the applied force, and the state of the tool (i.e., on/off) during each demonstration.

The simulated experiment consisted of five trials with $300$ simulated executions (in pairs of two) of the system with operator corrections. Consistent with the task, corrections were given $50\%$ of the time on samples during the first three passes. While in the future, we would like to estimate human error rates during corrections to set appropriate confidence bounds, the simulation used a fixed one percent error rate (i.e., the human failed to provide needed corrections $1\%$ of the time and provided unneeded corrections $1\%$ percent of the time). After each pair of executions, the confidence was updated and the schedule was resampled with $5000$ iterations. For our task, the schedule sampling took $3$ minutes ($i5$-$10400F$, $10$ threads). The results, shown in Figure \ref{fig:exp1results}, demonstrate that as the robots receive corrections, the system can refine and leverage its confidence to optimize the task scheduling.

We also developed a prototype system (with an early version of the real-time adaptation), shown in Figure \ref{fig:expsetup}, to conduct a pilot study involving ten participants (3M, 5F, 2 non-binary), aged 18--24 ($M=20.5$, $SD=1.8$), recruited from the UW--Madison campus, under an approved protocol from the university Institutional Review Board (IRB). Participants interacted with the robots serially (one at a time) as a baseline condition and with the final parallel solution found by the scheduler. Primary measures included total task time and paint removal performance, calculated through automated imaging. Results indicate that participants could complete the task with the parallel robots with shorter time on task ($t(9)=25.75,\; p<0.001$) and increased idle time ($t(9)=7.30,\; p<0.001$) without significant impact on paint removal performance ($F(1,9)=2.18,\; p=0.17$). These findings provide a preliminary demonstration of the promise of our approach and a basis for a comprehensive future evaluation.


% We conducted a two-part evaluation designed to demonstrate and assess the contributions of the proposed method. The first part focused on assessing the scheduling algorithm and Bayesian inference. The second part focused on assessing the multi-robot shared autonomy and real-time adaptation. Both parts of the evaluation used the experimental sanding task described in the following subsection. The parameters of the method used during the evaluation appear in the supplement.


% \subsection{Task}
% The experimental task was inspired by industrial sanding applications (e.g., fabrication, refinishing, and repair applications from aircraft and automotive manufacturing and maintenance industries)  where portions of the task have variable sanding needs. The task consisted of four sanding passes of gray spray paint on a piece of white acrylic, as shown in Figure \ref{fig:expsetup}. The first three passes were variable, where 50\% of the piece was painted in randomly chosen sections. The fourth pass was consistently painted (i.e., did not require corrections). During demonstrations, the operator applied more force and moved the tool more slowly over the painted sections and applied very little force and moved quickly when there was no paint. As a result, the learned nominal program \textit{lightly} sands the first three passes and required corrections by the operator to adjust the abrasiveness (i.e., more sanding over paint, less sanding when there is no paint to save time and extend tool life).

% The authors collected a total of six sanding demonstrations using the instrumented sanding tool shown in Figure \ref{fig:expsetup} (top-left).
% %Four of the demonstrations had random patterns on the first three passes, which were balanced so that each area was painted 50\% of the time.
% To limit the number of required demonstrations in our experiment, we included a demonstration for the least abrasive (i.e., zero paint on the first three passes) and a demonstration for the most abrasive (i.e., fully painted passes) sanding required for the task. In practice, the number of demonstrations would likely be much larger (e.g., to obtain statistical guarantees) and would expose the variability bounds.

% During each demonstration, the tool was moved slowly between passes as a simple way to ensure the task was amenable to a highly-parallel execution (i.e., one robot sanding while the other moved between passes). In practice, other high-confidence actions between passes could enable parallel scheduling, such as changing sanding discs.

% We collected the pose of the tool, the applied force, and the state of the tool (i.e., on/off) during each of the demonstrations. The pose of the tool was tracked using an Optitrack motion capture. The forces were measured using an ATI Axia80 Force-Torque Sensor. The state of the tool was controlled and tracked using a solenoid valve.

% \subsection{Simulation \& Results}
% To highlight how our scheduling solution adapts with use of the shared autonomy system, we conducted a simple simulated experiment on the demonstration data. We ran a total of five trials. Each trial simulated 300 executions (in pairs of two) of the system with operator corrections. The time-steps where the paint was variable were manually labeled (i.e., the first three sanding passes). Consistent with the task, corrections were given 50\% of the time during these samples. We chose to include a human error rate of one percent in our simulation (i.e., the human failed to provide corrections when needed one percent of the time and provided corrections when unnecessary one percent of the time). After each pair of executions, the confidence was updated and the schedule was resampled with 5000 iterations. We recorded both the percentage of high-confidence samples after each set of executions and the task length.
% The results, shown in Figure \ref{fig:exp1results}, demonstrate that as the robots receive corrections, the system can refine and leverage its confidence to optimize the task scheduling.

\begin{figure}[]
\centering
\includegraphics[width=3.4in]{./figures/exp1_type3.pdf}
\vspace{-12pt}
\caption{Scheduling solution evolution. \textit{Top:} The task length decreases over repeated interactions. The dark blue line shows the mean task length (with shading for the standard deviation). For each iteration, we show an example schedule where blue means the first robot is low confidence, red means the second robot is low confidence, and gray means neither robot is low confidence. Initially, the robots are mostly low confidence and execute the task one at a time. Eventually the robots alternate executing low-confidence actions. \textit{Bottom:} Percent of high-confidence samples. The gray dotted line is the ground-truth percentage.}
\label{fig:exp1results}
\vspace{-18pt}
\end{figure}
