\section{Introduction}
\label{sec:introduction}

\begin{figure}[t]
\centering
\includegraphics[width=3.4in]{./figures/teaser_placeholder_v3.pdf}
\vspace{-5pt}
\caption{Our method enables \textit{multi-robot shared autonomy} by leveraging expert demonstrations and coordinating robots around times where they may need assistance from the operator. \textit{Top:} An operator (task expert) provides a set of demonstrations to inform the robot task model, task variability, and flexibility in scheduling. \textit{Bottom:} The operator provides corrections across two robots completing the same task. The robots are scheduled such that only one robot (\textit{e.g.}, the robot with the red indicator) could require corrections at any given time (\textit{i.e.}, the other robot has high confidence in its action). The confidence is estimated from demonstration variability and empirical correction data.}
\label{fig:teaser}
\vspace{-20pt}
\end{figure}

\IEEEPARstart{H}{uman}-robot teaming is a promising alternative for tasks where robust automation is infeasible due to high complexity and variability.
% \textit{Human-robot teaming} promises to improve the efficiency and ergonomics of tasks where robust automation is infeasible due to high complexity and variability. 
In many cases, human-robot teaming is achieved through \textit{shared autonomy}, where a human operator and a robot policy share command over the physical robot platform and leverage their respective strengths \cite{selvaggio2021autonomy,losey2018review}. For example, the expert operator can offload a task's physical burden to the robot and use their own task knowledge and superior sensing to make corrections to the robot's behavior. However, many tasks do not require input from the operator during the entire execution but only during \emph{regions of task variability}. For example, during a fastener insertion task, the robot may only need help from the human to fix alignment error when inserting fasteners but not while fetching or prepping the fasteners. When the regions of variability make up a small part of the overall task the worker is poorly utilized. While the worker may perform secondary tasks during idle time, there are advantages to the operator working with multiple robots engaging in the same task, such as reducing context switching. In this work, we propose a method for multi-robot shared autonomy that sequences the execution of two robots based on regions where they may require assistance.

Previous work includes investigations of elements of multi-robot shared autonomy, such as interfaces for operator attention management \cite{gao2012teamwork}, supervisor allocation across a fleet of agents (e.g., mobile robots) \cite{crandall2010computing,rosenfeld2017intelligent,dahiya2022scalable,swamy2020scaled,hoque2022fleet,ji2022traversing}, and scheduling of agent subtasks and supervision \cite{yan2013survey,gombolay2013fast,cai2022scheduling,zanlongo2021scheduling}. However, the allocation methods focus on enabling an operator to temporarily teleoperate an agent needing assistance and in scheduling, little to no work focuses on coordinating agents around operator intervention. As illustrated in Figure \ref{fig:teaser}, we are interested in \textit{coordinated shared autonomy} where the robots operate at a high level of autonomy. In this paradigm, the robots sequence their behaviors such that only one robot could require assistance at any point, and the operator provides targeted corrections to the low-confidence robot if its action is incorrect. The major challenges in coordinated shared autonomy are determining \emph{what} interventions the operator may want to provide and \emph{when} they may occur during the task. Addressing these challenges requires models of the task, the types of corrections the operator may desire, and when corrections might be needed. Scheduling coordinated shared autonomy also introduces new challenges addressed in this work, such as how to compensate for changes to the timing of the robot's execution that result from operator corrections.

In this paper, we propose a method for multi-robot shared autonomy based on operator corrections \cite{hagenow2021corrective}, multi-agent scheduling, and Learning from Demonstration (LfD) \cite{ravichandar2020recent}. Our method schedules two robots such that one operator can provide real-time corrections at times when they are needed by both robots. The task model, corrections an operator can make, and sequencing of agents are inferred from expert demonstrations and past shared autonomy executions. The main contributions of this paper include (1) an optimization-based method to schedule corrections-based shared autonomy of multiple agents by leveraging variability in demonstrations; (2) a technique for inferring when corrections are needed based on task variability and Bayesian inference; and (3) a real-time adaptation strategy to update the timing of robots when operator corrections cause deviations from the schedule.