\section{Discussion}
\label{sec:discussion}
Our preliminary findings suggest that our proposed method can enable users to efficiently share autonomy with multiple robots, though future studies with the prototype system are needed to confirm the benefits. We demonstrated how the scheduling algorithm reduces the time to complete a sanding task as new high-confidence times are identified. As the high-confidence samples approached the ground-truth percentage, our example found a schedule with alternating sanding passes. Generally, our scheduler finds solutions that respect the optimization constraints (e.g., no overlapping, low-confidence actions) and reduces the total time by either speeding up high-confidence actions or overlapping executions when possible.

\subsection{Limitations \& Future Work}
Our method inherits limitations from LfD and time warping, which requires similar demonstrations. To avoid these assumptions, we are interested in exploring a reward-based structure that can better generalize to heterogeneous demonstrations. We also desire to scale our technical methods to larger robot fleets, which requires a modified approach to the margin-checking algorithm presented in Algorithm \ref{alg:margin}, and explore the practical considerations of shared autonomy with larger numbers of agents \cite{chen2014human}. Finally, our preliminary evaluation presented only one example sanding task to highlight the benefits of the proposed method. In the future, we will conduct user studies to assess the empirical human error correction rates and the performance (i.e., accuracy and generalization) of our method across a range of tasks, including different tasks for each robot.

