\documentclass[sigconf]{acmart}
\usepackage{subcaption}

\usepackage{pdfrender}
\newcommand*{\boldcheckmark}{%
  \textpdfrender{
    TextRenderingMode=FillStroke,
    LineWidth=.5pt,
  }{\checkmark}
}

\AtBeginDocument{%
  \providecommand\BibTeX{{%
    \normalfont B\kern-0.5em{\scshape i\kern-0.25em b}\kern-0.8em\TeX}}}


\setcopyright{acmcopyright}
\copyrightyear{2018}
\acmYear{2018}
\acmDOI{XXXXXXX.XXXXXXX}

\acmConference[Conference acronym 'XX]{Make sure to enter the correct
  conference title from your rights confirmation emai}{June 03--05,
  2018}{Woodstock, NY}

\acmPrice{15.00}
\acmISBN{978-1-4503-XXXX-X/18/06}


\begin{document}

\title[STIXnet: A Novel and Modular Solution for Extracting All STIX Objects in CTI Reports]{STIXnet: A Novel and Modular Solution\\for Extracting All STIX Objects in CTI Reports}


\author{Francesco Marchiori}
\email{francesco.marchiori@math.unipd.it}
\affiliation{%
 \institution{University of Padova}
 \streetaddress{Via Trieste, 63}
 \city{Padua}
 \country{Italy}}

\author{Mauro Conti}
\email{mauro.conti@unipd.it}
\affiliation{%
 \institution{University of Padova}
 \streetaddress{Via Trieste, 63}
 \city{Padua}
 \country{Italy}}


\author{Nino Vincenzo Verde}
\email{nino.verde@leonardo.com}
\affiliation{%
 \institution{Leonardo S.p.A.}
 \streetaddress{Via Laurentina, 760}
 \city{Rome}
 \country{Italy}}


\begin{abstract}
The automatic extraction of information from Cyber Threat Intelligence (CTI) reports is crucial in risk management. The increased frequency of the publications of these reports has led researchers to develop new systems for automatically recovering different types of entities and relations from textual data. Most state-of-the-art models leverage Natural Language Processing (NLP) techniques, which perform greatly in extracting a few types of entities at a time but cannot detect heterogeneous data or their relations. Furthermore, several paradigms, such as STIX, have become de facto standards in the CTI community and dictate a formal categorization of different entities and relations to enable organizations to share data consistently. 
\par
This paper presents STIXnet, the first solution for the automated extraction of all STIX entities and relationships in CTI reports. Through the use of NLP techniques and an interactive Knowledge Base (KB) of entities, our approach obtains F1 scores comparable to state-of-the-art models for entity extraction (0.916) and relation extraction (0.724) while considering significantly more types of entities and relations. Moreover, STIXnet constitutes a modular and extensible framework that manages and coordinates different modules to merge their contributions uniquely and exhaustively. With our approach, researchers and organizations can extend their Information Extraction (IE) capabilities by integrating the efforts of several techniques without needing to develop new tools from scratch.
\end{abstract}


\begin{CCSXML}
<ccs2012>
<concept>
<concept_id>10002978</concept_id>
<concept_desc>Security and privacy</concept_desc>
<concept_significance>500</concept_significance>
</concept>
<concept>
<concept_id>10002951.10003317.10003347.10003352</concept_id>
<concept_desc>Information systems~Information extraction</concept_desc>
<concept_significance>300</concept_significance>
</concept>
</ccs2012>
\end{CCSXML}

\ccsdesc[500]{Security and privacy}
\ccsdesc[300]{Information systems~Information extraction}

\keywords{Cyber Threat Intelligence, Natural Language Processing, Information Extraction, STIX}

\maketitle

%\vspace{-3mm}
\section{Introduction}
\label{sec:introduction}


\begin{comment}
    Points to be made in the intro
        - High value GEMM workloads have complex dependencies []
        - Traditional pipelining does not respond well to these dependencies
        -Keeping operands on-chip is key to get a good performance.
                - Caches - Implicit and Workload agnostic + hardware overheads
                - Spads - Explicitly managed (large map-space and expensive search and buffer allocation)
                - CHORD: A hybrid implicit and explicit management, which is cycle-level implicit and coarse-grained explicit, so the cost lowers enough to manage the SRAM online..
        - SCORE: Fast scheduler for low intensity ops that deals with more complex dependencies, and also provides metadata to CHORD..
        - Skewed GEMMs can have surprisingly low arithmetic intensity??
        
    


    
\end{comment}





As Deep Learning (DL) emerged as a high-value workload, the computer architecture community responded by proposing custom accelerators for DNNs~\cite{eyeriss2016isca,kwon2018maeri,tpu-isca,nvdla}. 
The most common fundamental operation for these DNNs was matrix multiplication, often expanded from convolutional layers that were common in DNNs at that time. These GEMMs offered good reuse opportunities because of their large dimensions and relatively cubic aspect ratios, allowing early DNN accelerators to successfully schedule each layer independently and achieve maximum utilization for the networks they targeted at design-time. Simultaneously, these schedules could be efficiently implemented using \emph{scratchpad} buffers that were explicitly controlled by the schedule to the stage intermediate \emph{tiles} of data according to the traversal order within each GEMM, without concern for inter-layer efficiency.

%Furthermore, prior works have also looked at the accelerators for sparse tensor workloads~\cite{sigma,eie,extensor,eyeriss2} which eliminate irrelevant multiplications and memory accesses.}% Unfortunately, this approach lowers intensity even further, and straightforwardly applying these to CG on a GEMM-by-GEMM operation basis results in compute under-utilization due to limited memory bandwidth.


Recently, work like the Tensor Algebra COmplier (TACO) \cite{XXXTaco} has spurred research interest in generalized tensor applications, including sparse tensor algebra. %The computer architecture community has responded with recent proposals that generalize custom accelerators to efficiently target a superset of DNNs \cite{extensor, tensaurus, dave2020hardware, tensorlib, flextensor} namely, applications represented as arbitrary DAGs of \emph{tensor operations} and non-linearities for DL activation functions.
However, this generalization brings new challenges: tensor-algebra applications have diverse shapes, sparsity ratios, and dependency graphs. %Worse, this is coupled with a trend in DNNs away from large, monolithic GEMMs and towards numerous small independent matrix multiplications\footnote{Often called \emph{batched} GEMMs, which should not be confused with DNN batch size.} with less cubic aspect ratios.
This is significant because of an underappreciated fundamental property: namely, \textit{not all dense GEMMs with large number of multiplications are compute-bound even in the best case}. As shown in \autoref{fig:ai1}, a \emph{skewed} aspect ratio inherently decreases \emph{arithmetic intensity} (AI), thus making the individual GEMM memory-bound and leaving datapath resources idle. For example, to saturate its datapaths the TPU v3 and v4 architectures require an AI of approximately 150 and 250 respectively~\cite{tpuv4}.  %Sparse tensor algebra introduces similar challenges, as removing multiplication-by-zero decreases the AI numerator, while transferring \emph{metadata} for compression formats increases the denominator relatively.}

\insertFigureScaled{ai1}{Degradation of arithmetic intensity on two GEMMs with the same number of multiplications due to aspect ratio skew.\vspace{-3mm}}{.75}



%\insertFigure{no-fusion}{Pipelining cannot simply be applied to complex DAGs due to - 1) Delayed downstream dependency, 2) varying shapes, 3) consumers at multiple reuse distances 4) need to preserve layout across conusmers.\vspace{-3mm}}


%The reason for this is low arithmetic intensity for a skewed GEMM and it can be as low below 1 op/byte. Currently, most scheduling strategies optimize matrix multiplications independently and compose them to execute in an op-by-op manner, which leads to low performance in applications with skewed GEMMs.}

\begin{comment}
A real-world quantification of this phenomenon is captured by HPCG benchmark~\cite{dongarra2015hpcg,hpcg2021}, which runs Conjugate Gradient (CG)---a widely used HPC solver application represented as a DAG of tensor operations.


\begin{scriptsize}
    
\begin{table}[t!]
\begin{scriptsize}

\begin{center}

    
    %\vspace{-2mm}
  \center
  \caption{Performance of CG compared to Linpack (HPL) on Top5 supercomputers. Adapted from HPCG~\cite{hpcg2021}.
} 
    \label{tables:hpcg}
  %\Rav{There is space for remarks too if needed}}
  %\TK{@Raveesh - by x do you mean you can do either s or t for that dimension? So does that mean each row is actually multiple datapoints? Thats confusing.}\Rav{It means that the datapoint that we choose for evaluation of that dataflow can have varaible tile sizes for dimension marked by X. S and T on the other hand mean that its NECESSARILY spatial or temporal}}
  \begin{tabular}{|l|l|l|l|l|}
    \hline
    \textbf{Supercomputer} & \textbf{HPL} &\textbf{HPCG} & \textbf{HPCG flops} &\textbf{HPCG:}  \\
     & \textbf{Pflops/s} &\textbf{Pflops/s} & \textbf{as \% of HPL } &\textbf{\% of peak}  \\
    \hline
1. Frontier & 1206 & 14.05 & 1.16\% & 0.8\% \\ \hline
2. Aurora & 1012 & 5.61 & 0.55 & 0.3\\\hline
3. Eagle & 561.2 & \multicolumn{3}{l|}{\revision{Not available}} \\\hline
4. Fugaku & 442.01 & 16 & 3.62\% & 3\% \\ \hline
5. Lumi & 379.7 & 4.587 & 1.2\% & 0.87\% \\ \hline
%6. Leornardo & 238.7 & 3.114 & 1.3\% & 1\% \\ \hline
%7. Summit & 148.6 & 2.93 & 2\% & 1.5\% \\ \hline
  \end{tabular}
%\vspace{-5mm}
\end{center}
\end{scriptsize}

\end{table}
\end{scriptsize}



%\footnote{HPCG is a benchmark for supercomputers that runs CG. \MP{Footnote 1 adds no real information. Cut. }}
As \autoref{tables:hpcg} shows, CG achieves only 1-3\% of peak performance on top 7 supercomputers. 
Therefore, intra-operation reuse alone is not sufficient.
The overall throughput can be increased by improving the arithmetic intensity (i.e., on-chip data reuse), which can be done by seeking inter-operation reuse.
\end{comment}

Thus to achieve full utilization we must consider inter-GEMM operation scheduling. Prior works have also shown that simple pipelining of adjacent operation~\cite{tileflow,isca-pip,yan2020hygcn} can improve things, but these solutions exclude significant potential sources of reuse, including delayed downstream consumers, and multiple consumers with varying reuse distances or traversal orders as~\autoref{fig:no-fusion} shows. To make matters worse, these extra options explode the scheduling space for finding good scratchpad configurations, as the total combinations and proportions of allocations explodes with operation DAG depth and the number of tensors involved. This means that exhaustive schedule-space exploration techniques become unacceptably slow, and also that heuristic solutions are more likely to miss the true optimal.

%It is also challenging to simply apply traditional inter-operation pipelining in cases of these dependencies, because of the delayed downstream dependencies, varying shapes of skewed GEMMs and multiple of these downstream consumers as~\autoref{fig:no-fusion} shows. %\TK{Fig 4 getting referenced before Fig 3, i suggest bring it before Fig 4 then} 
%~\autoref{fig:dfg} shows an actual complex cascade of tensor operations found in CG, a high-value HPC workload rich in these complexities.
%Thus, we need mechanisms to cache these intermediate tensors within SRAMs, since there are situations where simple pipelining cannot be applied.

%Prior works like buffets~\cite{buffets} make use of \emph{explicit decoupled data orchectration} to supplement scratchpad RAM with simple pointer and credit management scoreboarding. This is called explicit because data placement and RAM replacement are schedule-controlled, and decoupled because it operates on bulk-synchronous fills. %These work well for executing one layer at a time, when all dimensions of matrices have sufficient reuse.
%While explicit orchestration works well for scheduling single matrix multiplication at a time, considering data placement, for inter-operation reuse statically in general tensor algebra is a diabolically hard problem. 
% In order to reuse such operands, specially where intra-operation reuse is simply not enough, storing these operands in the on-chip buffer is essential, and multiple of these operands contend for space inside the buffer. We show in~\autoref{sec:arguments} that the design-space to accommodate multiple tensors inside a buffer space explodes in complexity. ~\autoref{sec:arguments} also discusses why running statically known DAGs does not necessarily imply that scratchpads are not burdensome and don't have design-time overheads.
%Overall, the co-dependence on schedule, available scratchpad capacity and different downstream consumers make the explicit scheduling problem too challenging.

\insertFigure{no-fusion}{Pipelining cannot simply be applied to complex DAGs due to - 1) Delayed downstream dependency, 2) varying shapes, 3) consumers at multiple reuse distances 4) need to preserve layout across conusmers. We discuss this in detail in~\autoref{sec:nofusion}\vspace{-3mm}}

Of course, scratchpads can be supplemented with pointer and credit management logic to become queues or buffets~\cite{buffets}, but these do not solve the schedule explosion problem as their staging decisions are still \emph{explicitly} controlled. Alternatively, caches are widespread on-chip storage structures that use \emph{implicit} data orchestration. Ideally, the presence of the cache is not architecturally exposed and the hardware itself does the best job possible of reusing the data. (In practice, best optimization is often made by cache-aware scheduling policies, blurring the line between implicit and explicit.) However, the area and energy overhead for tag matching make it less appealing for custom accelerators, as well as the possibility of increased misses due to conflicts. Overheads aside, cache policies typically operate at line granularity of unified address streams, rather than having higher-level knowledge of tensors, blocks, or intended reuse distance. This results in cache policies often rejecting the data that may have high reuse frequency (by virtue of reuse of the whole block) but might not be the immediate vicinity (i.e., high reuse distance).

In order to effectively exploit all available sources of reuse, we propose a unique approach: co-designing the buffer storage idiom with the scheduling algorithm. Our goal is that it becomes tractable to find schedules that obtain a sufficient level of inter-operation reuse to reach the compute bound, even in the face of complex DAGs of dependencies. To achieve this, we propose ~\SpadNamenospace\footnote{\SpadNameexp}, a novel \emph{explicit+implicit hybrid} buffering scheme that aims to combine the ``best of both worlds'' by placing coarse-grained decisions under the explicit control of the scheduler for efficiency, while making low-level fine-grained decisions implicitly like a cache, thus significantly reducing the size of the schedule space.
%~\SpadName is workload-aware, as it uses high-level metadata like starting and ending global address of a tensor, tensor-level reuse frequency and distance from the workload (the explicit component), but implicitly controls placement of elements of tensors without being burdensome to the programmer. ~%\TK{it might be worth adding a footnote saying you use scratchpad and buffer interchangeably in this paper} 
%\TK{based on what the footnote says its unclear whether chord is a scratchpad or cache }
Notably, this approach also significantly reduces the area and energy overheads of traditional line-level caches. %Specifically,~\SpadName uses per-tensor replacement policies that can be configured by the schedule at coarse granularity.
%for operands with downstream consumers that uses implicit replacement at a tensor granularity rather than a line granularity. 
%In this work we propose two such policies- \\(1) \PolicyA - \SpadName is filled in the queue order and once the buffer is full, the spilling data is sent straight to the DRAM.\\
%(2) \PolicyBnospace\footnote{\PolicyBexp} - If the current tensor has a higher reuse frequency and lower reuse distance (in case of same frequency) than the previously written tensor, the current tensor starts replacing the previous tensor by the tail.
%Because of its extremely coarsened granularity, ~\SpadName retains the benefit of a scratchpad over a cache (i.e., with minimal (<1\% of that of cache) tag matching overhead) 
%Hence it gets rid of caches' tag matching overhead and 
%while
%our proposed implicit tensor replacement policies~\PolicyA and \PolicyB remove the challenge of statically coming up with the data placement and replacement strategies statically for operands in multiple operations.

To schedule accelerators that use this structure, we propose ~\DataflowNamenospace\footnote{\DataflowNameexp} a downstream dependency aware novel scheduling strategy. 
\DataflowName identifies the delayed downstream dependencies that require writeback from those where pipelining would work, and steers the tensors with delayed writeback dependencies to~\SpadNamenospace. In order to exploit the reuse on delayed downstream consumers, it is important to make sure that the order in which the elements are produced as same as the order in which they are consumed, otherwise, layout transformation is required. Unlike prior mappers which search for tile-sizes for fine-grained buffer allocation, \DataflowNamenospace's involvement in buffer allocation is coarse-grained at an operand granularity rather than element-wise granularity, and low-level fine-grained decisions are made implicitly by~\SpadNamenospace's policies. 

All in all, we show that \SpadName and \DataflowName are applicable to diverse tensor applications and achieve 2.51x geomean speedup and 4x improvement in energy efficiency across a broad range of workloads (\autoref{sec:eval}).

%\TK{minimizing layout transformation is an important feature. Lets highlight it more explcitly. As in mention that the dataflow of the current op and downstream op may be different requiring expensive layout transformation, and SCORE tries to minimize that.}%It consists of the following steps - \\ 
%\TK{there seems to be some missing text here?}
%(1) Marking all the edges in the DAG of tensor operations with pipelining opportunities. This is useful in situations where tensors have delayed downstream consumers which can be visualized as a long edge in the DAG.\\
%(2) Assigning loop orders and tiling strategies to ensure that pipelining is actually exploited and layout transformation overhead of a tensor is minimized.\\
%\DataflowName reduces the contending tensors and also ensures minimum layout transformation (aka swizzling) which is also the motivation behind~\PolicyA policy.
%The main reason is the low arithmetic intensity achieved by the execution of Conjugate Gradient on CPUs/GPUs. 
%The main reason for this that the tensor multiplications---which we term \emph{operation} for this paper---%\MP{This definition is also too important to be in a footnote
 %used in CG have extremely unbalanced aspect ratios, which significantly lowers arithmetic intensity and data reuse.} %\MP{I feel like this point should come earlier, perhaps before the previous paragraph.}
%We use the term \emph{skewed} GEMMs to refer to such SpMMs/GEMMs---though notably, in the limit they can devolve to matrix-vector multiplication, i.e. if the shape of the matrix is 100,000:8.}%The main reason for this is costly data movement and communication between compute clusters/pods\MP{This explanation doesn't make sense either. Needs a reuse/intensity component. Even if this made sense, does GOGETA actually fix this problem?}.




%\TK{@Raveesh - I think we can move this para to para 3. Basically start with current para 2 on GEMMs, reuse and spatial accelerators. Then introduce CG. So intersperse this para with current para 3}


%making inter-operation reuse essential.

%This work explores inter-operation reuse opportunities for kernels like CG to enhance its arithmetic intensity.

%However, the~\GEMM in CG have orders of magnitude lower arithmetic intensity compared to SpMMs/GEMMs in DNNs and offer less reuse opportunities within one operation as~\autoref{fig:ai}~(\autoref{sec:ai}) discusses later in the paper. 
%Thus, executing CG at the granularity of a GEMM, reading tensors from DRAM and writing the output of each GEMM to DRAM, makes it highly memory bound leading to low compute utilization as~\autoref{\:hpcg} shows. 
\begin{comment}
Fusing \textit{adjacent} low AI matrix multiplication operators has recently been leveraged for applications like Graph Convolutional Networks (GCN)~\cite{garg2021understanding,yan2020hygcn,liang2020engn} and Transformers~\cite{flat,flashattention} 
wherein tiles of the intermediate tensor are manifested and consumed within the 
on-chip memory hierarchy in a \textit{pipelined} manner, reducing  intermediate output data movement to and from DRAM. We call this \emph{adjacent-op pipelining} in this work.
Other recent works have looked at enumerating the design-space for such adjacent-op pipelining~\cite{garg2021understanding} and 
%\TK{is this what the isca paper does?}\RG{Yep}
identifying optimal pipelined dataflow choices~\cite{isca-pip}, thereby enhancing better compute utilization when running memory-bound tensors.

While traditional adjacent-op pipelining is promising for DAGs with linear chains of operators (i.e., most DNNs today and GCNs), we show in this work that it is insufficient to capture nuances in more complex DAGs, such as those used in HPC kernels like CG.
Specifically, delayed downstream consumption of fanout of tensors, implies that traditional pipelining which overwrites the previously consumed tile cannot be applied directly due to a later dependency.%~\autoref{fig:no-fusion} shows an example, where tensor S has a delayed consumer.
 Moreover nuances such as data layouts of one tensor consumed in various operations is not captured by adjacent pipelining.
%that prior works cannot capture, as we discuss later. \autoref{table:related} enumerates this.

In this work, we coin the term \emph{generalized inter-operation reuse} to widen the scope of inter-operation reuse beyond adjacent operators to include additional reuse opportunities (\autoref{table:related}).
%as a wider generalization of traditional \emph{adjacent-operator pipelining}, introducing three additional reuse opportunities (\autoref{table:related}). 
We also propose \DataflowName (\TitleExpansion), which is a systematic strategy for mapping DAG of tensor operations exploiting the generalized inter-operation reuse opportunities, with the goal of enhancing the AI of the overall application.
%\DataflowName is applicable to any DAG of operations. 
We demonstrate that while \DataflowName is essential for extracting reuse in complex DAGs like CG, it can also be applied to simpler DAGs like GCNs and DNNs, thus preserving generalizability.

%\insertFigurePartnnnn{no-fusion}{A part of the CG tensor dependency graph where a node represents the equation in~\autoref{alg:cg_einsum} and edge represents the output of the source node equation. Please refer to ~\autoref{fig:dfg} for complete graph of CG.\vspace{-3mm}}%\RG{contraction heavy}}
We summarize our contributions below:
%\TK{I think the contribitions can be listed more succinctly. I would suggest each contribution bullet pointing to a specific section of paper. }

\squishlist
%\item We characterize the challenge of skewed GEMMs/SpMMs in tensor-algebra applications and the challenges of generalizing traditional inter-operation pipelining~\footnote{Not to be confused with inter-operation "reuse" since pipelining is a narrow aspect of inter-operation reuse}. (\autoref{sec:ai}).
%We also observe that these patterns are frequent across other HPC workloads which we discuss in ~\autoref{sec:background}.
%\item We propose a systematic methodology to formulate the data reuse opportunities in an arbitrary DAG of tensor operations and based on the reuse opportunities, we derive the loop orders and tile sizes.(\autoref{sec:dataflows}).
%\item We co-optimize the loop orders and tiling strategies for the GEMM operations in order to leverage reuse from both traditional inter-operation pipelining and distance based inter-operation reuse.

\item We propose~\DataflowName (\autoref{sec:score}), a scheduler which identifies the operands with delayed dependencies that require writeback (and hence \SpadNamenospace), and proposes a schedule that maximizes inter-operation reuse and minimizes the layout transformation of the operands across different consumers.
\item We propose~\SpadName(\autoref{sec:chorus}), a buffer structure for operands with downstream consumers that uses tensor-operand level replacement, that reduces tag match overhead and considers a more wholistic view of the object rather than a cache line. Cycle-level implicit tensor level replacement also eases the burden of solving tensor allocation involving multiple tensors statically which is a hard problem. 
%\item %Prior works on pipelining~\cite{flat,tangram,garg2021understanding,yan2020hygcn} divide the whole compute region into contiguous chunks and an operation is mapped on to a chunk as~\autoref{fig:spacetime} (top sub-figure) shows.\TK{seems odd to cite Fig 12 in intro}
%However, there is not much reuse within a single operation and the whole tensor needs to be communicated between the chunks. 
%\TK{this seems like a very specific optimization - and confusing here about what is lower BW vs higher BW NoC .. in fact i dont recall our arch section / Table IV talking about two NoCs with diff BWs? cant you state this contribution more generally about a communication BW optimized scalable inter-operation tiling strategy?}
%We propose a scalable inter-operation tiling strategy that reduces inter-cluster communication(~\autoref{sec:tiling}).  %a mapping that reduces memory accesses and communication at the tensor dependency graph level.
%\end{itemize}
%\item We propose buffer management schemes that reuse initial tiles of the output and can also replace the tensor based on future reuse distance and frequency~(\autoref{sec:tornado}). %the notion of Tensor-operand level reuse distance for such patterns and a tensor organization strategy inside the buffer hierarchy (\autoref{sec:tornado}).

\item We demonstrate the limitations of caches, namely, area overhead and line-level policies and the limitations of scratchpad, namely complexities in static buffer allocation of multiple delayed downstream consumers (\autoref{sec:arguments}).

\item We show that \SpadName and \DataflowName are applicable to diverse tensor applications and achieve 2.51x geomean speedup and 4x improvement in energy efficiency across a broad range of workloads (\autoref{sec:eval}).

%\item \reviewme{ } 
%\item We propose a novel data orchestration technique \DataflowName which allows for efficient replacement, placement and prefetching of data for applications with variable reuse distance patterns. Since reuse distance is a generalization, \DataflowName can be used for applications with only intra-operation reuse and inter-operation pipelining opportunities as well.
%\item We propose a novel buffer idiom \TODO{name} which improves performance and energy over caches and scratchpads by \TODO{xx} and \TODO{xx}
\squishend
\end{comment}


\begin{comment}
\emph{Inter-operation pipelining (aka fusion)} has been shown to be beneficial for accelerators for applications like Graph Neural Networks~\cite{garg2021understanding,yan2020hygcn,liang2020engn} where the intermediate tensor is reused between an SpMM and a GEMM,  reducing  intermediate output data movement to and from DRAM.
It has been shown that this approach is more challenging~\cite{dnnfusion,flat} and has a larger design-space~\cite{garg2021understanding} than straightforward element-wise fusion done by ML compilers today, for example, matrix-multiplication and ReLU fusion.
%Unfortunately, \textit{traditional inter-operation pipelining} often consumes the intermediate data and does not keep it in the memory. 
%Traditionally, \textit{inter-operation pipelining} has been exploited in prior works~\cite{tangram,garg2021understanding,yan2020hygcn,flat} %across various application domains 
%to reduce the data movement to DRAM, by consuming the portion of the tensor as it is produced 
\TK{I think we need to transition to saying that inter-operation pipelining does not capture all inter-operation reuse opportunities as this work identifies.}
Unfortunately, the additional complexity of inter-operation dependency graphs in certain applications introduces additional challenges which confounds the attempts at \emph{traditional inter-operation pipelining.} Furthermore, just capturing reuse in adjacent operations misses the overall opportunity to consider future instances of reuse of tensors in the entire program. \reviewme{We use CG as an example to discuss these challenges and missed opportunities:}

(1) Operations can have a delayed downstream consumer. Therefore, the data must remain resident in the memory hierarchy. However, traditional pipelining overwrites tiles that are consumed by the adjacent operation.
%~\autoref{fig:dfg} shows CG's dependency graph. Output of operation 1 is required in a delayed downstream consumer (op4) in addition to the adjacent consumer (op2). %The intermediate matrix cannot be overwritten or discarded, since its required in a future computation. 


%This complex dependency graph can often complicate the optimization of loop orders and tiling strategies to determine efficient intra-operation and inter-operation dataflows.


(2) CG has some tensor operations with contracted rank being much larger than the other ranks. This diminishes the benefit of pipelining because significant computation is required to produce any given final sum; therefore, making it a rate limiting step. 
%This prevents pipelining the entire graph.
%\TK{the following line seems out of place - since its one of many techniques we propose. Should maybe remove it}
%We propose \textit{pipelining with writeback}, which involves traditional pipelining and writing back the intermediate data to the buffer, hence incurring only write accesses and avoiding the read accesses for the intermediate matrix.

\insertFigure{no-fusion}{A part of the CG tensor dependency graph where a node represents the equation in~\autoref{alg:cg_einsum} and edge represents the output of the source node equation. The data cannot be consumed and be shielded from memory hierarchy since its reused again in another tensor. Please refer to ~\autoref{fig:dfg} for complete graph of CG.\vspace{-3mm}}%\RG{contraction heavy}}

(3) As the DAG becomes more complex, it is important to make sure that the loop order choice minimizes the data layout transformation (we use the term swizzling for it) of these tensors across various consumer operations.

(4) The downstream consumers also result in multiple \textit{tensor operand reuse distances}.

In this work, we identify various \textit{inter-operation} reuse opportunities to enhance the arithmetic intensity of such applications.
Our proposed \emph{generalized inter-operation reuse} is a wider generalization of \emph{traditional inter-operation pipelining}. We propose \DataflowName (\TitleExpansion), a strategy for mapping DAG of tensor operations which exploits reuse opportunities between operations.

\DataflowName is applicable to any DAG of operations. So while it helps extract reuse in the complex DAGs like CG, these patterns can also apply to simpler DAGs like GCNs, thus preserving generalizability.

%\reviewme{\DataflowName consists of the following contributions.

 %First, we identify the inter-operation reuse patterns in an arbitrary DAG of tensor operations. We propose a methodology to mark the edge of the DAG with the appropriate reuse pattern. This addresses the DAG complexity problems.

% Second, we propose loop orders that try to take the maximum advantage of the reuse patterns, given that the ability to extract reuse also depends upon the order of loops in these operations. The loop order also minimize swizzling

 %Third, prior works on pipelining~\cite{flat,tangram,garg2021understanding,yan2020hygcn}, divide the whole compute region into contiguous chunks and an operation is mapped on to a chunk as in~\autoref{fig:spacetime} (top sub-figure) shows. However, there is not much reuse within a single operation and the whole tensor needs to be communicated between the chunks. Instead, we pipeline within the cluster and split dominant ranks across cluster (\autoref{fig:spacetime} bottom).


% Fourth, given that we minimize swizzling, we propose a buffer management scheme that writes the data into the SRAM in the same order and skips the SRAM once its full. Then it reads the portion that's already in the SRAM first. We show that this can improve the op-by-op baseline considerably. We also propose buffer management strategies to prioritize operations with low reuse distances and high reuse frequencies.

% }

%In this work, we identify various \textit{inter-operation} reuse opportunities which are not limited to \textit{inter-operation pipelining}.
%Please note that \textit{inter-operation reuse} is a wider generalization of \textit{traditional inter-operation pipelining} here, since inter-operation pipelining only exploits reuse between consecutive operations. \autoref{fig:venn} shows the scope of our work on inter-operation dataflows compared to prior works on acceleration.
%One example is \textit{pipelining with writeback}

%Moreover, it is important to focus on leveraging the knowledge of future occurrence of the tensor beyond the immediate tensor. In CG, these tensor operands have variable \textit{tensor operand reuse distances}. This pattern is frequent in various scientific applications as~\autoref{sec:background} shows. This complex dependency graph can often complicate the determination of loop orders and tiling strategies to determine efficient intra-operation and inter-operation dataflows.

%Also, given that the graph of tensors becomes more complex, it is important to make sure that the loop order choice minimzes the transformation of layouts of these tensors across various operations where that tensor is used. We use the term \textit{swizzling} for the layout transformation. Thus we need to minimize swizzling.

%\RG{Can we cut this para and make contributions slightly longer ?}
%Based on the insights from the tensor dependency graphs of various applications, we propose a systematic methodology to identify, classify and exploit reuse opportunities in an arbitrary graph of tensor operations targeting spatial accelerators. This also involves deriving the loop orders and tile sizes for individual operations since the ability to exploit the inter-operation reuse can also depend on the individual operation's dataflow.
%Some other applications which have individual \GEMM of low arithmetic intensity include Graph Neural Network, Transformers etc. where our methodology is applicable, however, we often refer to Conjugate Gradient for demonstration purposes in this paper since its tensor dependency graph already has the characteristics of low intensity Graph and ML workloads but also has additional unique characteristics providing opportunity to demonstrate different kinds of inter-operation reuse.
%We also propose a scalable tiling strategy that involves splitting a memory bound GEMM by the dominating rank into sub-tensors and reusing the data in a fine-grained manner between the sub-tensors from different operations within a compute node as the bottom half of~\autoref{fig:spacetime} shows.
% (shown later in~\autoref{fig:spacetime})\TK{not sure if we need to cite a figure that'll come this late. Might be better to point to the section that will discuss this}. 
%
% We propose \DataflowName (\TitleExpansion), a strategy for mapping DAG of tensor operations which exploits reuse opportunities between operations. It comprises of identifying inter-operation reuse patterns from the DAG structure, loop-reordering and tiling based on those patterns and custom scratchpad management strategies. 
Our key contributions are as follows:
%\vspace{-5mm}

%\begin{itemize}
\squishlist
\item We characterize the challenge of skewed GEMMs/SpMMs in tensor-algebra applications and systematically formulate \emph{generalized inter-operation reuse opportunities} (beyond pipelining) in a DAG of operations (\autoref{sec:dataflows}).
%We also observe that these patterns are frequent across other HPC workloads which we discuss in ~\autoref{sec:background}.
%\item We propose a systematic methodology to formulate the data reuse opportunities in an arbitrary DAG of tensor operations and based on the reuse opportunities, we derive the loop orders and tile sizes.(\autoref{sec:dataflows}).
%\item We co-optimize the loop orders and tiling strategies for the GEMM operations in order to leverage reuse from both traditional inter-operation pipelining and distance based inter-operation reuse.
\item Based on the reuse opportunities, we derive loop orders for the operations that allow consumers to maximally reuse the data, and require minimum changes to memory layout~(\autoref{sec:loop}).
\item Prior works on pipelining~\cite{flat,tangram,garg2021understanding,yan2020hygcn} divide the whole compute region into contiguous chunks and an operation is mapped on to a chunk as~\autoref{fig:spacetime} (top sub-figure) shows. However, there is not much reuse within a single operation and the whole tensor needs to be communicated between the chunks. We propose a scalable inter-operation tiling strategy which splits the dominant rank across compute clusters and pipelines the operations within a cluster as~\autoref{fig:spacetime} (bottom) shows~(\autoref{sec:tiling}).  %a mapping that reduces memory accesses and communication at the tensor dependency graph level.
%\end{itemize}
\item We propose a buffer management scheme that writes the data into the SRAM in the same order and skips the SRAM once its full. Then it reads the portion that's already in the SRAM taking advantage of swizzle minimzation. We also propose buffer management strategies that prioritize operations with low reuse distances and high reuse frequencies~(\autoref{sec:tornado}). %the notion of Tensor-operand level reuse distance for such patterns and a tensor organization strategy inside the buffer hierarchy (\autoref{sec:tornado}).
\item \DataflowName achieves \reviewme {geomean 5.97x (ranging from 1.34x to 23x)} improvement in the arithmetic intensity over operation by operation execution~(\autoref{sec:eval}).

\item \reviewme{\DataflowName is generally applicable to diverse tensor applications as it applies to all DAG structures. We also evaluate it on GCNs and ResNets and obtain 2.71x and xx geomean improvement in arithemetic intensity over op-by-op baseline~(\autoref{sec:eval}).} 
%\item We propose a novel data orchestration technique \DataflowName which allows for efficient replacement, placement and prefetching of data for applications with variable reuse distance patterns. Since reuse distance is a generalization, \DataflowName can be used for applications with only intra-operation reuse and inter-operation pipelining opportunities as well.
%\item We propose a novel buffer idiom \TODO{name} which improves performance and energy over caches and scratchpads by \TODO{xx} and \TODO{xx}
\squishend
\end{comment}

\begin{comment}

\begin{enumerate}
    \item Identifying inter-operation reuse patterns in an arbitrary DAG of operations
    \item Deriving loop orders to make sure that they are amenable to inter-operation reuse and minimize swizzling
    \item Proposing tiling strategy that does pipelining within the compute node and splits the large rank across nodes (bottom half of~\autoref{fig:spacetime}).
    \item Proposing scratchpad management strategies like reuse distance and frequency based tensor replacement and proposing strategies to extract reuse from the portion of the tensor that does fit in SRAM.
    
\end{enumerate}
\vspace{-1mm}
Our methodology is also applicable to %(and evaluated on) 
other applications with low intensity GEMMs like GNNs and transformers. %where our methodology is applicable,
However, we often refer to Conjugate Gradient as the application for demonstration purposes since its tensor dependency graph has the characteristics of low intensity Graph and ML workloads and also has additional unique characteristics providing opportunity to demonstrate different kinds of inter-operation reuse.
\autoref{fig:venn} shows the scope of our work on inter-operation dataflows compared to prior works on accelerator dataflow.
%Our proposed \textit{inter-operation reuse} is a wider generalization of \textit{traditional inter-operation pipelining} here, since inter-operation pipelining only exploits reuse between consecutive operations.
%\autoref{fig:venn} shows the scope of our work on inter-operation dataflows compared to prior works on acceleration.

%Therefore data orchestration which accounts for such reuse is critical for HPC workloads. Cache replacement policies often have a global view of an individual line rather than a tensor as a whole, which limits the applicability for such algorithms. Moreover, LRU replacement does not see the future reuse of the tensor and replaces it if it has not been used recently. Belady's optimal replacement policy requires knowledge of future accesses for each line and often requires additional structures and meta data accesses to replace one line which is costly to implement~\cite{popt-hpca21}. Scratchpads, on the other hand provide ability to the programmer to have control over data orchestration. However, these scratchpads are often over-provisioned for the worst case applications.

%To this end, we propose \DataflowName data orchestration that analyzes the tensor operand-level reuse patterns in the algorithm and the dataflows of the individual GEMMs to manage the data in the memory hierarchy. \DataflowName also enables efficient prefetching for these HPC algorithms due to the knowledge of the future tensor reuse pattern and dataflows.

 %Some other applications with low arithmetic intensity include Graph Neural Network, Transformers etc., however, we often refer to Conjugate Gradient for demonstration purposes in this paper since its DAG already has the characteristics of Graph and ML workloads but also has additional unique characteristics providing opportunity to demonstrate different kinds of inter-operation reuse.


\noindent  \textbf{\textit{The key contributions of this paper are:}}
\end{comment}

%Our methodology is also applicable to %(and evaluated on) 
%other applications with low intensity GEMMs like GNNs and transformers. %where our methodology is applicable,
%However, we often refer to Conjugate Gradient as the application for demonstration purposes since its tensor dependency graph has the characteristics of low intensity Graph and ML workloads and also has additional unique characteristics providing opportunity to demonstrate different kinds of inter-operation reuse.
%\autoref{fig:venn} shows the scope of our work on inter-operation dataflows compared to prior works on accelerator dataflow.
%Our proposed \textit{inter-operation reuse} is a wider generalization of \textit{traditional inter-operation pipelining} here, since inter-operation pipelining only exploits reuse between consecutive operations.
%\autoref{fig:venn} shows the scope of our work on inter-operation dataflows compared to prior works on acceleration.


































\begin{comment}
\section{Problem and Motivation}
\label{sec:introduction}

Sparse and Dense matrix multiplications are prime operations for a variety of applications spanning Graph Analytics~\cite{kipf2017semisupervised,hamilton2017inductive}, High-Performance Computing~\cite{cools2017communication,cerebras} and Artificial Intelligence~\cite{resnet,nlp}. The GEMM operations used in DNNs offer vast reuse opportunities~\cite{kwon2019understanding,interstellar,timeloop,eyeriss2016isca} owing to their high arithmetic intensity. This has led to a plethora of spatial accelerators for DNN applications~\cite{eyeriss2016isca,kwon2018maeri,tpu-isca,nvdla,shi}. Prior works have also looked at the acceleration for Sparse workloads~\cite{sigma,eie,extensor,eyeriss2} by eliminating redundant matrix multiplications and memory accesses.

However, certain PDE solver algorithms used scientific applications like the Conjugate Gradient~\cite{hestenes1952methods} have SpMM/SpMV and GEMM/GEMV operations with low arithmetic intensity~\cite{cerebras}. The best supercomputers ranked on High-Performance Linpack benchmark are able to achieve only upto 3\% of the performance on the HPCG benchmark~\cite{cerebras,osti_1089988}.
~\autoref{alg:cg_einsum} shows the Block Conjugate Gradient Algorithm. ~\autoref{fig:results}a) and b) show maximum achievable arithmetic intensity of individual operations. Please note that instead of counting multiplications and additions separately, we count MAC as a unit of computation. We also consider matrix addition of, for example, $X^{k-1}$ with $P^{k-1}.\alpha^{k-1}$ in the equation $X^{k}=X^{k-1}+P^{k-1}\alpha^{k-1}$ to be done immediately after the MAC and we absorb the addition in same operation. Therefore, effectively we count number of multiplications per memory access of an element (independent of the bit precision). Please note, here number of RHS is a parameter with values 1,8,16,32 and 64. 

\insertFigure{cg}{Block CG Algorithm.}


~\autoref{fig:results}a) shows the maximum achievable arithmetic intensity for individual SpMM operations for the suitesparse CG matrices. We note that the SpMMs, especially the highly sparse ones like barth4 have extremely low arithmetic intensity even for RHS=64.
~\autoref{fig:results}b) shows the maximum achievable arithmetic intensity of dense GEMMs in an individual operation in terms of number of multiplications per memory access. We notice that the arithmetic intensity of the DenseGEMMs is severely limited by the number of RHS (ie. the number of problems solved simultaneously.) Therefore traditional DNN accelerators~\cite{eyeriss2016isca,kwon2018maeri,tpu-isca} do not help accelerate these workloads due to fundamentally low reuse.

~\autoref{fig:results}c) shows the arithmetic intensity for one iteration of the CG loop. Please note that for matrix inverse, we consider LU factorization followed by Triangular Solve. Also, we conservatively count accesses to $P$ and $P^T$ separately owing to different layout in the memory. We note that the amount of reuse that can be extracted between the operations is high and this motivates us to design a new dataflow for HPCG workloads which tries to maximize inter-operation rather than optimizing for inter-operation reuse.

\end{comment}
\section{Background and Related Work}
\vspace{-1mm}
\label{sec:background}
\subsection{Einsums, Dataflows and Mappings}
\vspace{-1mm}
Tensor operations like convolutions and matrix multiplications can be concisely and precisely expressed using Einsum (Einstein summation) notation. Einsums are well-supported state-of-the-art tools like Python's numpy and the tensor algebra compiler TACO \cite{XXXTaco}. Compared to traditional mathematical matrix contraction notation, they have the advantage of explicitly describing the volume of data being operated on. For example, the equations below describe GEMM and CONV:

\vspace{-3.5mm}

\begin{equation}
    Z_{m,n}=\sum_{k}A_{m,k}*B_{k,n}
\end{equation}

\vspace{-3.5mm}

\begin{equation}
    O_{n,h,w,k}=\sum_{c,r,s}I_{n,h+r,w+s,c}*W_{r,s,c,k}
\end{equation}

In equation (1), $A$, $B$ and $Z$ are the tensors and $m$,$n$ and $k$ are dimensions/ranks. $k$ is a \textit{contracted rank} and $m$ while $n$ are \textit{uncontracted ranks}. For brevity the summation can be omitted as the contracted ranks do not appear in the output tensor.

Einsums can be straightforwardly implemented using loop nests, for example:
%\begin{center}
\vspace{-1mm}
\begin{small}
  \begin{verbatim}
1 for m in range(M):       for m in range(M):
2  for n in range(N):       for k in range(K):
3   pfor k in range(K):      pfor n in range(N):   
4    Z(m,n)+=A(m,k)*B(k,n)    Z(m,n)+=A(m,k)*B(k,n)
\end{verbatim}  
\end{small}
\vspace{-1mm}
%\end{center}

\textit{Dataflow} refers to the loop transformations for staging the operations in compute and memory. A dataflow can affect compute utilization and the locality of tensors inside the memory hierarchy. Within the Einsum, an \textit{intra-operation} dataflow is determined by the loop order and the parallelism strategy. The code sequences above represent two different loop orders: the left is $MNK$, whereas the right is $MKN$. The \texttt{pfor} indicates that the rank is parallelized. Thus, the left sequence is \emph{K-parallel} and the right one is \emph{N-parallel}. The \textit{inter-operation} dataflow for multiple chained Einsums is one of the main contributions of this work, as discussed in detail in~\autoref{sec:dataflows} and~\autoref{sec:gogeta}.

Another result of the loop order is the concept of stationarity~\cite{eyeriss2016isca}. An \emph{A-stationary}  dataflow signifies that $A$ is the tensor whose operands change slowest--therefore the $N$ rank is the fastest to change (as it does not index $A$). In GEMMs, there are two possible loop orders for \emph{A-stationary} dataflows, $MKN$ and $KMN$. Similarly, for \emph{Output-stationary} dataflows, the $K$ rank is the fastest to change, hence $MNK$ and $NMK$ are the two possible loop orders.

Tiling refers to slicing the tensors, in order for sub-tensors to fit in local memory buffers to extract reuse. Note that typically partitioning any given rank can affect multiple tensors. An example code sequence for tiling is as follows:
\vspace{-1mm}
\begin{small}
  \begin{verbatim}
1 M1=M/M0; K1=K/K0; N1=N/N0
2 for m1 in range(M1):    
3  for k1 in range(K1):  
4   for n1 in range(N1):
5    for m0 in range(M0):
      m = m1 * M0 + m0
6     for k0 in range(K0):
       k = k1*K0+k0
7      pfor n0 in range(N0):
        n = n1*N0+n0
8       Z(m,n) += A(m,k) * B(k,n)  
\end{verbatim}  
\end{small}

Note the interaction of parallelism and tiling: $N0$ is \texttt{pfor} in line 7. Thus in a tile, the $n0$ indices of $N$ are spatially mapped, resulting in $N1$ temporal tiles of size $N0$. 

The combination of dataflow and tiling is called a \emph{mapping}: a schedule of the exact execution of the workload on a specific hardware accelerator. Mapping directly affects data movement, buffer utilization, memory bandwidth utilization, and compute utilization.

\subsection{HPC Applications: Chains of Einsums}
\label{sec:apps}

Single Einsums are kernels, whereas the main loops of scientific applications consist of a chain of Einsums where tensors produced by earlier equations are consumed by later ones. This results in a \emph{tensor dependency graph} dictating the high-level production/consumption of data throughout the HPC region of code. Throughout this section we use Conjugate Gradient as a running example because its tensor dependency graph exhibits multiple kinds of reuse opportunities and challenges. We briefly discuss other scientific applications with similar patterns where our work is applicable in this section.

\subsubsection{Block Conjugate Gradient}


Iterative linear solvers solve the system of linear equations-

\vspace{-2.5mm}
\begin{equation}
    A_{m, k} * X_{k} = B_{m}
\end{equation}

While traditional conjugate gradient considers b and x as vectors, block conjugate gradient works on multiple initial guesses simultaneously for faster convergence, thus making it a matrix multiplication problem:

\vspace{-1mm}
\begin{equation}
    A_{m, k} * X_{k, n} = B_{m, n}
\end{equation}
\vspace{-1mm}
%%\vspace{-2mm}
\LinesNotNumbered

\begin{algorithm}
\begin{small}

%\label{algo:inter-op}
\caption{Block Conjugate Gradient. 'prev' and 'cur' stand for previous and current. Only lines with tensor operations inside the loop are numbered.}\label{alg:cg}
$R_0=B-AX_0$\\
$P_0=R_0$\\
$prev=0$\\
\For{iteration=1,2,3...}
{
\nl $S_{prev}=A.P_{prev}$\\
\nl $\Delta=P^{T}_{prev}.S_{prev}$ and $\Lambda=\Delta^{-1}.\Gamma_{prev}$\\
\nl $X_{cur}=X_{prev}+P_{prev}.\Lambda$\\
\nl $R_{cur}=R_{prev}-S_{prev}.\Lambda$\\
\nl $\Gamma_{cur}=R^{T}_{cur}.R_{cur}$\\
\If{$all(diag(\Gamma_{cur})\leq \in)$}
{
    break
}
\nl $\Phi=\Gamma^{-1}_{prev}.\Gamma_{cur}$\\
\nl $P_{cur}=R_{cur}+P_{prev}.\Phi$ \\

$prev=cur$\\
$cur++$\\

}
\textbf{Return} $X_{cur}$
\end{small}

\end{algorithm}

\begin{comment}

  \nl   $S'(m,n)=A(m,k)*P'(k,n)$     \\
 \nl   $\Delta (n,n)=P'^{T}(n,k)*S'(k,n)$ and $\Lambda (n,n)=\Delta ^{-1}(n,k)*\Gamma (k,n)$ \\
 \nl   $X(m,n)=X'(m,n)+P'(m,k)*\Lambda (n,n)$ \\
\nl    $R(m,n)=R'(m,n)-S'(m,k)*\Lambda (n,n)$ \\
 \nl   $\Gamma (n,n)=R^{T}(n,k)*R(k,n)    $  \\
\If{$all(diag(\Gamma)\leq \in)$}
{
    break
}
\nl    $\Phi(n,n)=\Gamma ^{-1}(n,k)*\Gamma (k,n)$ \\
\nl $P(m,n)=R(m,n)+P'(m,k)*\Phi (n,n)$\\

\end{comment}


\vspace{-5mm}
\LinesNotNumbered

\begin{algorithm}
\begin{small}

%\label{algo:inter-op}
\caption{Chain of Einsums for Conjugate Gradient. $k$ represents contractions of size $M$ and $j$ represents contractions of size $N$ and $N'$. Line numbers represent significant computation steps referred to throughout the text.}
\label{alg:cg_einsum}
\KwIn{$A_{m,m}$, $B_{m,n}$, $X_{m,n}$}
$R_{m,n} = B_{m,n} - A_{m,k} * X_{k,n}$ \\
$\Gamma_{n,n}=R_{k,n}*R{k,n}$\\
$P_{m, n} = R_{m, n}$ \\
\While{not converged}
{
  \nl  $S_{m,n} = A_{m,k} * P_{k,n}$     \\
 \nl   $\Delta_{n',n} = P_{k,n'} * S_{k,n}$ and $\Lambda_{n',n} = \Delta^{-1}_{n',j} * \Gamma_{j,n}$ \textcolor{gray}{~~~// $\Delta=P^TS$} \\
 \nl   $X_{m,n} = X_{m,n} + P_{m,j} * \Lambda_{j,n}$ \\
\nl    $R_{m,n} = R_{m,n} - S_{m,j} * \Lambda_{j,n}$ \\
$\Gamma\_prev_{n,n}=\Gamma_{n,n}$\\
 \nl   $\Gamma_{n',n} = R_{k,n'} * R_{k,n}$ \textcolor{gray}{~~~~~~~~// $\Gamma=R^TR$} \\
\If{$all(diag(\Gamma)\leq \epsilon)$} 
{
    \textbf{break}
}
\nl    $\Phi_{n',n} = \Gamma\_{prev}^{-1}_{n',j} * \Gamma_{j,n}$ \\
\nl    $P_{m,n} = R_{m,n} + P_{m,j} * \Phi_{j,n}$
}
\textbf{return} $X$
\end{small}
\end{algorithm}


\begin{comment}

  \nl   $S_{prev}(m,n)=A(m,k)*P_{prev}(k,n)$     \\
 \nl   $\Delta (n,n)=P^{T}_{prev}(n,k)*S_{prev}(k,n)$ and $\Lambda (n,n)=\Delta ^{-1}(n,k)*\Gamma (k,n)$ \\
 \nl   $X_{cur}(m,n)=X_{prev}(m,n)+P_{prev}(m,k)*\Lambda (n,n)$ \\
\nl    $R_{cur}(m,n)=R_{prev}(m,n)-S_{prev}(m,k)*\Lambda (n,n)$ \\
 \nl   $\Gamma _{cur}(n,n)=R^{T}_{cur}(n,k)*R_{cur}(k,n)    $  \\
\If{$all(diag(\Gamma_{cur})\leq \in)$}
{
    break
}
\nl    $\Phi(n,n)=\Gamma ^{-1}(n,k)*\Gamma _{cur}(k,n)$ \\
\nl $P_{cur}(m,n)=R_{cur}(m,n)+P_{prev}(m,k)*\Phi (n,n)$\\
\end{comment}




%\insertFigurePart{cg}{Block Conjugate Gradient Algorithm. 'prev' and 'cur' stand for previous and current variables.}


%~\autoref{alg:cg} shows Conjugate Gradient and
~\autoref{alg:cg_einsum} shows the Einsums in the Conjugate Gradient Algorithm. Intuitively, we start with an initial guess $X$ and we update $R$ which at any iteration is equal to $B-AX$. If $R$ is sufficiently small, then we have reached the solution. $P$ represents the search direction for the next iteration of the loop. We have validated the functional correctness of our Einsum representation against Python's \texttt{scipy.sparse.linalg.cg}.

%\vspace{-3.5mm}


From the perspective of tensors, $P$, $R$, $S$ and $X$ (named using English-letter variables except A) are highly skewed, for example $1000000\times 8$. In contrast, tensors like $\Delta$, $\Lambda$, $\Phi$ and $\Gamma$ (named using Greek letters) are small tensors, for example, of size $8\times8$ Also, $A$ is the only sparse tensor in CG with a maximum shape of, for example, $1000000\times1000000$ but with occupancy of 1-100 non-zeros per row. In~\autoref{alg:cg_einsum}, $M$ represents the large rank while $N$ represents the small rank. $N'$  is equivalent to $N$ but is used to differentiate between the dimensions of square matrices in an Einsum without accidentally contracting them. As a result, line 1 is a sparse SpMM operation while all the other matrix multiplication operations are dense. The inverse operations (lines 2 and 6) are insignificant in terms of the magnitude of computation but they do affect the dependency graph since they require the complete tensor to be produced for execution. As we discuss in ~\autoref{sec:ai}, these matrix operations have low arithmetic intensity. The only inputs from the application side are $A$, $B$ and the initial guess, which is the initial $X$, the final output is $X$ at convergence: all other tensors are intermediates not observable by the invoking context.

Another peculiar feature about the dense matrix multiplications in Conjugate Gradient is that one matrix dimension is large.
%In this work, we represent each matrix multiplication as $(M\times K)$.$(K\times N) = M\times N$, thus $K$ is the contracted rank while $M$ and $N$ are uncontracted.
In operations denoted by lines 1, 3, 4 and 7, an uncontracted rank is the dominating rank (assuming $A$ has been compressed using a standard format like CSR). In matrix multiplication operations denoted by line 2 ($\Delta=P^TS$) and 5, a contracted rank is the dominating rank resulting in small outputs of the order 1$\times$1 to 16$\times$16. Contracted rank being dominating significantly diminishes the benefits of pipelining the entire CG application efficiently since most of the compute is spent in just large amounts of reduction to generate an output a usable output for the operation after it thus not exploiting the staging opportunity and affecting the overall utilization. 

~\autoref{fig:dfg} shows the dependency graph between Einsum operations in CG. We observe that most tensor operands are not only reused in the immediate operation but also in some later operation (i.e., transitive edge) therefore having multiple reuse distances. This is unlike dependency graphs of applications like DNNs and GNNs where the output is mostly used immediately in the next operation, making fusion straightforward. One exception in Deep Learning is the skip connections in applications like ResNet.


\insertFigure{dfg}{Tensor dependency graph of intermediates in Conjugate Gradient across first two iterations of the CG loop where a node's number corresponds to the line in~\autoref{alg:cg_einsum}.}

\subsubsection{Other Applications with similar patterns}

The pattern of variable reuse distance is commonly observed in Machine Learning models like Resnet~\cite{resnet} with skip connections, although ResNet has a high arithmetic intensity per Einsum. Its also observed in other solver methods like GMRES~\cite{gmres} and BiCGStab~\cite{van1992bi}. The problem of low arithmetic intensity individual \GEMM is common in workloads like Graph Neural Networks~\cite{kipf2017semisupervised}.

For example, a layer of GCN (Graph Convolution Network) has the following Einsums (Only A is sparse).

Variables: $A_{m,m}, ~~X0_{m,n}, ~~Z_{m,n}, ~~W_{n,o}, ~~X1_{m,o}$

$Z_{m,n}=\sum_{k}A_{m,k}*X0_{k,n}$  and  $X1_{m,o}=\sum_{j}Z_{m,j}*W_{j,o}$

\subsection{Related Work}

\textbf{Conjugate Gradient Acceleration:} Cerebras~\cite{cerebras} proposes mapping BiCGStab on the wafer-scale engine specifically for stencil application where the matrix $A$ is structured. Plasticine~\cite{plasticine} has inherent support for Vector Parallelism and Pipelined Parallelism. ALRESCHA~\cite{asgari2020alrescha} proposes an accelerator for Preconditioned Conjugate Gradient (PCG) and optimizes the locality of the SpMM and the SymGS kernels, however, even at maximum reuse, single kernels have low arithmetic intensity. None of these works have identified and exploited \textit{inter-operation} reuse.

\textbf{Dataflows and Mappers:} MAESTRO~\cite{kwon2019understanding}, Timeloop~\cite{timeloop}, Interstellar~\cite{interstellar}, GAMMA (Genetic Algorithm Mapper)~\cite{kao2020gamma}, CoSA~\cite{cosa} propose a mapping optimization search space, cost model or a mapping search algorithm for a single tensor operation at a time. Prior works like FLAT~\cite{flat} and TANGRAM~\cite{tangram} have proposed a new dataflow for pipelining between exactly two adjacent Einsums. Garg et al.~\cite{garg2021understanding} formulate the design-space for pipelined mappings for exactly two Einsums and propose a cost model OMEGA to evaluate those mappings. We identify reuse opportunities beyond pipelining between tensors and propose a systematic methodology to determine the dataflow of the whole dependency graph of tensor operations (including Einsums and tensor additions).

%\subsubsection{Specialized Data Orchestration} Various spatial accelerators for Deep Learning~\cite{tpu-isca,nvdla,eie,kwon2018maeri,sigma,extensor,eie} use custom buffers tied to their compute. Buffets~\cite{buffets} is a storage idiom designed for Deep Learning Algorithms. It uses Explicit Decoupled Data Orchestration. Prior works have also proposed domain-specific cache replacement policies, for example P-OPT~\cite{popt-hpca21} proposes a replacement policy for Graph algorithms.



\section{Related Works}
\label{sec:relatedworks}

The need for the automatic processing of CTI reports and retrieving entities and relations has pushed researchers to adopt Information Extraction methods in the field. However, this is not an easy task due to many reasons. First, information in raw text reports can be conveyed differently through semantics, and bulletin styles can differ from vendor to vendor. Furthermore, reports might (and frequently do) include new entities, making the usage of a static and non-interactive database of entity templates inefficient. Moreover, ever-changing Indicators of Compromise (IOCs, e.g., IP addresses, hashes, URLs, Bitcoin addresses) must be recognized and linked to their respective actor or malware. For these reasons, many different models have been proposed to push research on constructing a Knowledge Graph in Cybersecurity~\cite{9480953}. One of the aspects that can be noticed in the current literature on IE techniques applied to CTI reports is that many of the proposed models focus on one or a few types of entities/relations at a time while neglecting the others. This allows researchers to obtain impressive results in a narrow domain that does not always reflect the whole needs of the CTI community. Moreover, while being very efficient, using Machine Learning and Deep Learning models in Information Extraction has a few drawbacks. First, it is not easily scalable on wider domains to include more entity types and thus requires fully retraining the model while making changes to the architecture. Secondly, the training dataset might be annotated with just a few entity types, and to include other STIX types, analysts should perform the procedure all over again. This is quite time-consuming and can become very costly for companies and organizations.

In You et al.~\cite{You2022}, researchers have developed a model that can retrieve Tactics, Techniques, and Procedures (TTPs) with an accuracy of 0.941, which constitutes the state-of-the-art for this task. However, not only do TTPs not reflect the overall spectrum of Cybersecurity entities that should be extracted in a report, but the number of these TTPs is just 6. At the same time, the MITRE ATT\&CK Knowledge Base indicates at least 14 tactics and 191 techniques (just in enterprise environments and thus neglecting mobile and ICS attacks).

Similar work has been previously done by Legoy et al. with rcATT, a Python tool used to predict MITRE ATT\&CK tactics and techniques from cyber threat reports~\cite{https://doi.org/10.48550/arxiv.2004.14322}. It has a maximum precision of around 0.75 in both tactics and techniques. These reduced performance levels are justified by an increased number of labels, comprising 215 MITRE ATT\&CK techniques and 12 tactics. However, this work was published in 2020, and the MITRE ATT\&CK framework has changed with new techniques and tactics, so careful reparametrization is needed, and (as stated in the future work's sections of the paper) retraining it on a bigger dataset might improve its performance.

A more general approach that can tackle a broader domain of entities is the work of Gasmi et al.~\cite{app9193945}, which includes both entity extraction and relation extraction on data from the National Vulnerability Database (NVD)\footnote{\url{https://nvd.nist.gov/}}. Researchers addressed seven common entity types and six relationship types. This database contains CVE items, i.e., disclosed cybersecurity vulnerabilities specifically formatted to be more easily cataloged, evaluated, and shared among the community. The tool reaches a precision value of 89\% on the entity extraction task and 92\% on the relation extraction task. However, the data used for training and testing consists of vulnerability descriptions that, while written in natural language, present a common structure and thus do not reflect the variance that might be present in more common CTI reports or bulletins.

While some proposed works leverage Machine Learning models, which perform particularly well in narrow domains, other string tagging techniques can tackle the entity extraction task.

\begin{itemize}
    \item \textbf{Name-Matching Strings}: if a Knowledge Base containing the entity names is already in place inside the platform, it can match words inside a text. Even though the construction of the KB can be time-consuming, there are a lot of public sources from which to retrieve information on these entities, and it is also possible to provide aliases for each of them, thus being able to recognize pseudonyms and still link them to the correct entity.
    \item \textbf{RE-Matching Strings}: Indicators of Compromises can be found by their particular structure constant across the same type of IOC. For example, IP(v4) addresses are always be written in the format \texttt{XXX.XXX.XXX.XXX}, i.e., four sets of numbers from 0 to 255 separated by a dot character. Different rules apply to different types of entities. Still, in the domain of words with distinct character structures, it is possible to use regular expression tools to identify and extract them.
    \item \textbf{Verb-Related String}: this technique is used when the other two fail in extracting entities like companies and new malware names, which do not present any particular character structure and might not be present inside the Knowledge Base. These entities can be retrieved by analyzing each sentence through POS Tagging and Dependency Parsing. After that, we can retrieve the verb to match it with a predefined set of words that might indicate the presence of an entity.
\end{itemize}

STIXnet expands these results and further generalizes the entities and relationships that can be extracted in a document to comprise all STIX entities and relationships, which most closely represent the information an analyst should extract from CTI reports.
\section{Methodology}
\label{sec:methodology}

In this section, we present STIXnet in more detail on how it works and which algorithms and models are used to perform Information Extraction on unstructured Cyber Threat Intelligence reports. We first show its pipeline and briefly overview the various modules (Section~\ref{subsec:pipeline}). Then we give more details on its main components: the text extraction module (Section~\ref{subsec:textextraction}), the entity extraction module (Section~\ref{subsec:entityextraction_methodology}) and the relation extraction module (Section~\ref{subsec:relationextraction_methodology}).

\subsection{Pipeline}
\label{subsec:pipeline}

STIXnet performs a highly complex Information Extraction task on many different types of entities and relations. There are indeed 18 different types of entities compliant with the STIX standard and more than 100 types of relations. To accomplish this, STIXnet uses different modules for each one of the different tasks that it must achieve: textual extraction, entity extraction, and relation extraction. A graphic overview of the STIXnet platform is shown in Figure~\ref{fig:pipeline}.

\begin{figure*}[!htpb]
  \centering
  \includegraphics[width=\linewidth]{Figures/04-Pipeline3.pdf}
  \caption{Pipeline of the STIXnet platform.}
  \label{fig:pipeline}
\end{figure*}

\begin{itemize}
    \item \textbf{Text Extraction}: the first module converts the program's input into raw text. While doing this, artifacts are inevitably be introduced in the text, and therefore they must be handled to obtain a single string of characters in a common encoding.
    \item \textbf{Entity Extraction}: this module handles the extraction of the different entities in the text. It uses four different sub-modules to accomplish that:
    \begin{itemize}
        \item \textbf{IOC Finder}: we extract Indicators Of Compromise by using different regular expression rules and looking at text patterns.
        \item \textbf{Knowledge Base Entities Extraction}: using the entities in a Knowledge Base, we use an efficient algorithm for string search in a text to retrieve names and aliases. We mitigate errors and false positives caused by this approach using NLP techniques.
        \item \textbf{Novel Entities Extraction}: using NLP libraries, we extract entities not present in the Knowledge Base, which can then be integrated into the KB.
        \item \textbf{TTPs Extraction}: techniques and tactics are not always represented in a text by their name but can be implicit and not explicitly expressed. We use a Machine Learning model trained on MITRE ATT\&CK tactics and techniques to recognize them.
    \end{itemize}
    \item \textbf{Relation Extraction}: from the extracted entities, we now retrieve the relations. We use two sub-modules for this task:
    \begin{itemize}
        \item \textbf{Rule-Based Approach}: using NLP techniques, we perform Dependency Parsing and compute the shortest paths between entities. Comparing the verb inside the path with the ones in the STIX relationships, we estimate the most similar one with a degree of confidence.
        \item \textbf{Deep Learning Based Approach}: to adjust the results of the previous approach, we also compute embeddings from the sentences with a Deep Learning model. We then determine the similarity between these embeddings and those computed from the list of relationship labels.
    \end{itemize}
    \item \textbf{Output}: we create a JSON file from the extracted entities and relations, which is processed by the graphical interface of the platform to be interactive and dynamic.
\end{itemize}

Eventual interactions with the Knowledge Base or other platform components are disclosed in the individual sections of the different modules. Indeed, we show that by interacting with the database, it is possible to improve performance over time, particularly if an analyst decides to validate the results of the STIXnet output. 

Furthermore, the structure of the STIXnet pipeline allows for easy and immediate management of the different modules. Formally defining the interactions between modules and submodules allows results from different pipeline components to be compared and merged in a unique output. Thus, researchers and organizations can implement the STIXnet framework in different IE scenarios and add or remove components depending on their needs.

\subsection{Text Extraction}
\label{subsec:textextraction}

One of the platform's most relevant and sensible aspects is its input. As mentioned, STIXnet can take in input various reports and bulletins, which can come from various vendors or sources. For this reason, reports have some stylistic and linguistic differences. However, data must be converted univocally before processing the raw text to have consistent processing between different inputs and a common ground for evaluating the various modules. This means considering many different aspects that can change from input to input. First of all, we must be able to parse text from files with different formats, and thus we use Apache Tika\footnote{\url{https://tika.apache.org/}} for \texttt{pdf} and \texttt{doc} files and ConvertAPI\footnote{\url{https://www.convertapi.com/}} to extract text from the HTML data of web reports. However, since text can be formatted in many different ways, we must process it to remove artifacts, fix line breaks, and remove eventual sanifications on IP and other addresses. This phase is crucial to ensure that the following modules are presented with a clean input; otherwise, artifacts are propagated in the pipeline and compromise the results.

\subsection{Entity Extraction}
\label{subsec:entityextraction_methodology}

This section tackles the different submodules used for entity extraction: IOC Finder (Section~\ref{subsub:iocfinder}), a rule-based entity extractor (Section~\ref{subsub:kbentext}), a novel entity extractor (Section~\ref{subsub:novelentityextraction}), and a TTPs extractor (Section~\ref{subsub:ttpext}). Finally, we clarify how the submodules interact with one another and merge their results (Section~\ref{subsub:subint}).

\subsubsection{IOC Finder}
\label{subsub:iocfinder}

The particular structure of Indicators Of Compromise allows us to use regular expression (Regex) rules to find them in the text. Moreover, in some of the reports distributed by CTI vendors, the end of the document is often presented with a table containing the IOCs of interest for that particular topic. While different, all these indicators share a common structure across each type, which can be recognized with Regex rules without applying NLP techniques. Some examples of the types of IOCs supported by IOC Finder can be found in Table~\ref{tab:iocfinder}. During the execution of STIXnet and after processing the report as raw text, the first step is to run the IOC Finder submodule on it, which returns a dictionary containing the entities found.

We implement this module by forking an open-source project by Floyd Hightower\footnote{\url{https://github.com/fhightower/ioc-finder}}. To adapt it for our pipeline, we contributed to the main project by updating some libraries to a newer version and adding the ability to track the position of the found IOCs. The code of the forked project can be found in our repository.

\begin{table}[!htpb]
  \centering
  \caption{Examples of IOCs and their structure.}
  \label{tab:iocfinder}
  \begin{tabular}{ll}
    \toprule
    \textbf{IOC Type} & \textbf{IOC Structure}\\
    \midrule
    ATT\&CK Techniques & \texttt{T1518} or \texttt{T1518.001}\\
    CVEs & \texttt{CVE-2021-44228} \\
    Email Addresses & \texttt{example@mail.com} \\
    File Paths & \texttt{/path/to/file} \\
    MD5s & \texttt{e802c6b77dd5842906ed96ab1674c525} \\
    \bottomrule
\end{tabular}
\end{table}

\subsubsection{Knowledge Base Entity Extraction}
\label{subsub:kbentext}

After finding IOCs, all other entities of interest do not share a common structure and thus cannot be found through regular expression rules. Thus, we leverage a rich Knowledge Base integrated with multiple OSINT that explicitly indicate which names represent important entities and allow us to link them to their correct entity type. For this reason, a rule-based algorithm can search specific words in the text, retrieve their position and thus highlight them as extracted entities. The different sources for the intelligence are:

\begin{itemize}
    \item \textbf{Knowledge Base}: Leonardo S.p.A., an Italian multinational company that collaborated in this research, provided a rich database of STIX entities. Cyber threat intelligence analysts built this database over the years at their Security Operation Center (SOC), which read and manually annotated entities from many reports.
    \item \textbf{MITRE ATT\&CK}: the ATT\&CK framework can be used as a source of intelligence for different entities such as techniques, tactics, groups, and software (of which the conversion to the STIX standard has been disclosed in Table~\ref{tab:mitre2stix}). To retrieve this data, we use the Trusted Automated Exchange of Intelligence Information (TAXII) application protocol~\cite{connolly2014trusted}, which allows for exchanging threat intelligence over HTTPS and defines a RESTful API that can be used to provide or collect data.
    \item \textbf{Locations}: to retrieve the names of countries and continents, we used a \texttt{csv} file in which each nation is associated with its nationality. In this way, we are able to identify locations even when used as an attribute to another entity (e.g., "a \textit{Russian} malware").
\end{itemize}

After retrieving the entities, a quick pre-processing is performed to unify their formats and add the possibility of aliases for each one. Through aliases, it is possible to recognize an entity in a text and map it to the correct one, avoiding duplicates and fixing the issue of multiple names for a single Advanced Persistent Threat. We then use the Aho-Corasick algorithm to find terms of this thesaurus of words in the report~\cite{10.1145/360825.360855}. To mitigate false positives, we process each sentence with NLP techniques, in particular, Part-Of-Speech Tagging (POS tagging), allowing us to assign part-of-speech tags to each word (e.g., noun, verb). After defining a table of entity types and their possible POS tags, we use it to compare the entities found in the report with their extracted POS tag. For example, "us" can be used as a pronoun or can be used as a noun to reference the United States, constituting an entity of type "location".

\subsubsection{Novel Entities Extraction}
\label{subsub:novelentityextraction}

Some CTI reports are published to spread awareness of newly discovered actors, malwares, or techniques. These entities are thus named by CTI researchers and analysts and, for this reason, are most likely not present in the Knowledge Base. To find these new entities, we can leverage the previous execution of POS Tagging to extend its results and create a dependency graph from the tokens found in each sentence. In this way, we identify specific patterns used in the text to express a new entity. To create such a graph, we leverage both the POS tags and the dependencies between the tokens, as shown in Figure~\ref{fig:spacygraph}. To perform this processing, we use Spacy\footnote{\url{https://spacy.io/}}, a free, open-source library for advanced NLP in Python. By looking at numerous reports and bulletins from different vendors and sources, we can identify a limited number of ways a new entity can be introduced, allowing us to write pattern rules that the NLP processing can recognize.

\begin{figure*}[!htpb]
  \centering
  \includegraphics[width=\linewidth]{Figures/04-Spacy.pdf}
  \caption{Example of dependency graph generated from a sentence.}
  \label{fig:spacygraph}
\end{figure*}

\subsubsection{TTPs Extraction}
\label{subsub:ttpext}

While malwares and threat actors are often explicitly mentioned, some other entities are not and can be referenced without categorically stating their names. It is the case of tactics and techniques, which constitute the TTPs mapped into the STIX objects "x-mitre-tactic" and "attack pattern". Rule-based methods cannot be used to retrieve these entities since the variance that characterizes the expression of these concepts is too broad and is hardly definable through a set of rules. For this reason, a multi-label text classification model must be deployed and trained on the MITRE ATT\&CK Knowledge Base of tactics and techniques. We already presented a tool named rcATT~\cite{https://doi.org/10.48550/arxiv.2004.14322} that suffered from its age since it was published in 2020, and the MITRE ATT\&CK framework has had several changes and renovations ever since. The source code for rcATT is publicly available in their GitHub repository\footnote{\url{https://github.com/vlegoy/rcATT}}, and thus we retrained it from scratch with new techniques and tactics. Also, to address one of the limitations and future works of the paper presenting rcATT, we expanded the training dataset with more data from MITRE ATT\&CK descriptions and external sources for each tactic and technique. The new dataset, the pre-trained models, and the updated code for its training can be found in our repository.

\subsubsection{Submodules Interaction}
\label{subsub:subint}

As stated in Section~\ref{subsec:pipeline}, the pipeline structure of our framework allows us to differentiate the tasks in many modules. For entity extraction, in particular, we identified four different submodules independent from one another. Their input is always the same and is constituted by the textual data of the report. Once each submodule produces an output, a final check is performed to ensure no overlapping occurs during the processing. This process includes cross-checking the found entities in the Knowledge Base to maximize their confidence in their extracted type and the addition of the novel entities. More details are disclosed in Section~\ref{subsec:entityextraction_results}. Finally, all the entities are merged into one data structure, constituting the following module's input. Since this procedure is automatic, submodules can be added and removed in the pipeline according to the user's needs, and no conflicts occur during the process.

\subsection{Relation Extraction}
\label{subsec:relationextraction_methodology}

This module can retrieve relations between the found entities while processing the sentences in the raw text. However, one of Spacy's limitations is its inability to grasp relations between distant entities in a text. To address this, we propose two different approaches for relation extraction. In the first approach, we leverage POS Tagging and Dependency Parsing to compute a graph of each sentence and retrieve relations by looking at the shortest paths between entities (Section~\ref{subsub:rulebased}). In the second approach, we use a Transformer model to compute embeddings of the sentences and compute their similarity (Section~\ref{subsub:dlbased}). Finally, we clarify how the submodules interact with one another and merge their results (Section~\ref{subsub:relsubint}).

\subsubsection{Rule Based Approach}
\label{subsub:rulebased}

The main idea of the rule-based approach is to leverage the dependency graphs already computed by Spacy for novel entity extraction (Section~\ref{subsub:novelentityextraction}) and use graph theory functions to grasp the relation between two entities inside a sentence. In particular, we can process the dependency graph and retrieve the relations between any couple of nodes by discovering the Shortest Dependency Path (SDP), i.e., the shortest path between two nodes in the graph. It has been observed in other studies that the nodes in the SDPs usually contain the necessary information to identify a relationship between two entities while also being dependent on the structure and semantic complexity of the sentence~\cite{hua2016shortest, xu-etal-2015-classifying}.

After retrieving the shortest path between each couple of entities in the sentence, we extract their STIX type and focus on the verbs in the path. For example, considering the sentence in Figure~\ref{fig:spacygraph}, the extracted paths (between the three entities \texttt{"APT29"}, \texttt{"7-Zip"}, and \texttt{"Raindrop"}) are:
\begin{itemize}
    \item \texttt{[APT29, \underline{used}, 7-Zip]};
    \item \texttt{[APT29, \underline{used}, \underline{decode}, malware, Raindrop]};
    \item \texttt{[7-Zip, \underline{used}, \underline{decode}, malware, Raindrop]}.
\end{itemize}
By looking at the entity types and the root form of the verbs (underlined in the previous example), it is possible to compare them with the ones found in the list of STIX relationships and thus label the path as a specific STIX Relationship Object (SRO). However, since the verbs used to describe the relationship in the sentence might not be the same as the associated SRO, we use a similarity function to determine their likeliness. If it surpasses a certain threshold, we can consider them synonyms. To accomplish this, we use the Wu \& Palmer similarity function~\cite{https://doi.org/10.48550/arxiv.cmp-lg/9406033}, which, given two words and their synsets (i.e., groupings of similar words that express the same concept), outputs a value in the range $\left[0,1\right]$, where $1$ means maximum similarity. For this reason, this value can be used as a confidence measure of the relation. The taxonomy used for this task is the WordNet taxonomy, an extensive lexical database of English words developed by Princeton University~\cite{Fellbaum2010}.

For example, considering the SDP \texttt{[APT29, used, 7-Zip]}, the entity extraction module identified \texttt{"APT29"} as an \texttt{intrusion-set} and \texttt{"7-Zip"} as a \texttt{tool}. In the list of SROs, there is only one entry containing both an \texttt{intrusion-set} and a \texttt{tool}, which is "intrusion-set uses tool": since the root form of the verb in the SDP is equal to the one in the SRO, we label it accordingly with maximum confidence. Instead, considering another SDP \texttt{[APT29, attacks, the, US]} (where \texttt{"US"} is identified as a \texttt{location} entity), there are two SROs including both an \texttt{intrusion-set} and a \texttt{location}: "intrusion-set originates-from location" and "intrusion-set targets location". In this case, none of the root forms of SROs verbs ("originate" and "target") coincide with the one in the SDP ("attack"), and thus we compute the similarity between them:
\begin{equation*}
    wup("attack", "originate") = 0.4,
\end{equation*}
\begin{equation*}
    wup("attack", "target") = 0.5.
\end{equation*}
While keeping the confidence threshold at 0.5, we can label the SDP as a relationship of type "intrusion-set targets location".

The confidence value in the extracted relations becomes particularly important when dealing with sentences containing multiple entities. Indeed, the number of relations increases exponentially with the number of entities found, and some of the extracted relations could not exist. To include a "non-relation" label in this task, we set a threshold for the confidence, under which we discard the extracted relations.

\subsubsection{Deep Learning Based Approach}
\label{subsub:dlbased}

The rule-based approach is particularly efficient when dealing with simple phrases or when entities are close in the graph. However, many elements might be introduced in the SDP whenever two entities are far from each other. The verb with the highest similarity could be linked to different tokens in the text and might not reach the confidence threshold. To address this problem, we use embeddings, fixed-size vectors that can also be generated from textual data by Deep Learning models such as Transformers~\cite{https://doi.org/10.48550/arxiv.1706.03762}.

In this specific case of relation extraction, we are interested in computing the similarity between each sentence's embeddings and the STIX relationships' embeddings. To perform these embeddings, the best tool at our disposal is Sentence BERT (SBERT)\footnote{\url{https://www.sbert.net/}}, a variation of the BERT model (Bidirectional Encoder Representations from Transformers) developed by Google AI language~\cite{https://doi.org/10.48550/arxiv.1905.05950}. While BERT constitutes the state-of-the-art in many NLP applications, it becomes inefficient when dealing with a large corpus of sentence processing. SBERT addresses this problem using siamese and triplet network structures, drastically reducing processing time~\cite{reimers-2019-sentence-bert}.

For its STIXnet implementation, we compute the embeddings of the different STIX relationships and the sentences extracted from the report. For each sentence, we perform a pre-processing procedure for each contained entity couple by substituting their tokens with their extracted STIX type. Then, we compute the cosine similarity between these embeddings and normalize it to use it as a confidence value. We also use a threshold for the confidence of the relation to discriminate false positives (0.5).

\subsubsection{Submodules Interaction}
\label{subsub:relsubint}

As with the entity extraction module, the relation extraction task is divided into different submodules that work independently. While their input is always the same (i.e., textual data from the report and the previously identified entities), the two submodules perform virtually the same task this time. Thus we expect a degree of overlap in their results. However, their coexistence is necessary to extract relations from both simple and complex scenarios. To handle conflicts in the possible output, we use the confidence values generated during the processing and keep the relations between entities with maximum confidence, which must also be over the acceptance threshold. Therefore, users can add their desired submodules for the relation extraction task, and STIXnet will automatically merge their results.
\section{Evaluation}
\label{sec:results}

We formally evaluate the proposed entity and relation extraction modules in STIXnet. We use three metrics to evaluate the modules: precision, recall, and F1-score. We define True Positives (TP), False Positives (FP), and False Negatives (FN) as follows.

\begin{itemize}
    \item \textbf{True Positives}: entities or relations correctly classified by the model.
    \item \textbf{False Positives}: entities and relations found by the model but misclassified or that do not constitute an entity or a relation.
    \item \textbf{False Negatives}: entities and relations not found by the model but present in the report.
\end{itemize}

The metrics for the evaluation are Precision, Recall, and F1 Score and are defined as follows.

\begin{equation}
    Precision = \frac{TP}{TP + FP},
\end{equation}

\begin{equation}
    Recall = \frac{TP}{TP + FN},
\end{equation}

\begin{equation}
    F1 = 2\frac{Precision \cdot Recall}{Precision + Recall}.
\end{equation}

Given the lack of available annotated reports in the literature, we generated our own dataset of CTI reports to evaluate our model. Each report treats a group or threat actor from the MITRE ATT\&CK framework. All their related data has been extracted from their official descriptions and the external sources listed by the ATT\&CK APIs. We then manually label each report with LabelStudio, a free and open-source software for data labeling. Both the dataset and the annotations are free to use and accessible in our repository. Annotations are exported in a \texttt{JSON} file and can be graphically visualized through the LabelStudio software.

To show the effectiveness of both entity and relation extraction, we tackle the evaluation of the modules separately, respectively, in Section~\ref{subsec:entityextraction_results} and Section~\ref{subsub:relationextraction_results}.

\subsection{Entity Extraction}
\label{subsec:entityextraction_results}

While evaluating the entity extraction module for STIXnet, we must consider the different sub-modules separately since they perform different independent operations.

To have a baseline for comparing the results, we first evaluate the completeness of the deployed Knowledge Base in a fixed position in time, i.e., not being enhanced by adding new entities when found. We take the whole dataset and run the rule-based entity extraction sub-module on each report. We then extract precision, recall, and F1 scores and compute the mean of these scores over the number of reports processed. We compare this approach with other rule-based algorithms for entity extraction found in the literature. Since rule-based approaches are domain-dependent and language-dependent, in Table~\ref{tab:kbeval} we compare our approach with other works in literature that operate in specific domains to more accurately resemble our task (for~\cite{10.1093/jamia/ocz109}, we extracted the evaluation of the hybrid model since it more accurately resembles our rule-based algorithm). While the compared models are not designed for CTI data extraction, to the best of our knowledge, no other tools in the literature perform such an evaluation on rule-based approaches for STIX entities. As a note, the ground truth constituted by the manually annotated reports also contains novel entities and TTPs (not extractable with just this submodule), thus highlighting the contribution of the sub-module in the whole entity extraction system.

\begin{table*}[!htpb]
  \centering
  \caption{Baseline evaluations for the Knowledge Base entity extraction sub-module and comparisons.}
  \label{tab:kbeval}
  \begin{tabular}{llllll}
    \toprule
    \textbf{Model} & \textbf{Domain} & \textbf{Entities} & \textbf{Precision} & \textbf{Recall} & \textbf{F1 Score}\\
    \midrule
    Godény~\cite{6406529} & Consumer electronics & Product names & N/A & N/A & 0.221 \\
    Quimbaya et al.~\cite{QUIMBAYA201655} & Electronic health records & Diagnosis, treatment & 0.630 & 0.573 & 0.600 \\
    Chen et al.~\cite{10.1093/jamia/ocz109} & Clinical trial cohort selection & Patient data and conditions & N/A & N/A & 0.845 \\
    \textbf{STIXnet Rule-Based Algorithm} & \textbf{CTI Reports} & \textbf{All STIX entity types} & \textbf{0.835} & \textbf{0.869} & \textbf{0.846}\\
    \bottomrule
\end{tabular}
\end{table*}

We then evaluate the novel entity extraction sub-module separately. We created a new dataset of sentences containing non-existent entities with made-up names for its evaluation. We do this since, in the original dataset, some reports do not contain any new entities and would thus bias the results of this evaluation. By creating new sentences instead, we ensure a constant number of novel entities that the sub-module can extract. In such a scenario, the isolated sub-module reaches a precision of 0.927, a recall of 0.854, and an F1 score of 0.889. Given the system's modularity and the submodules' specific task, a comparison with other models for Information Extraction in Cyber Threat Intelligence is given in Table~\ref{tab:noveleval} (the results of STIXnet extraction are obtained with the dynamically augmented Knowledge Base). Results are obtained after combining the contribution of the different approaches and evaluating them in the same dataset used for the baseline evaluation. Also, while focusing on CTI applications, we highlight the number of entity types the compared models can extract.

\begin{table*}[!htpb]
  \centering
  \caption{Comparison of evaluations for the entity extraction task in the CTI domain.}
  \label{tab:noveleval}
  \begin{tabular}{lllll}
    \toprule
    \textbf{Model} & \textbf{Number of Entity Types} & \textbf{Precision} & \textbf{Recall} & \textbf{F1 Score}\\
    \midrule
    Weerawardhana et al.~\cite{10.1007/978-3-319-17040-4_24} & 4 & 0.730 & 0.820 & 0.720 \\
    Li et al.~\cite{9023758} & 4 & 0.839 & 0.789 & 0.813 \\
    Zhou (2022) et al.~\cite{Zhou2022} & 5 & 0.785 & 0.697 & 0.739 \\
    Zhou (2023) et al.~\cite{zhou2023cdtier} & 5 & 0.768 & 0.792 & 0.795 \\
    Ranade et al.~\cite{9671824} & 6 & 0.879 & 0.874 & 0.883 \\
    Wang et al.~\cite{wang2022cyber} & 8 & 0.859 & 0.863  & 0.861 \\
    \textbf{STIXnet Entity Extraction} & \textbf{18} & \textbf{0.903} & \textbf{0.935} & \textbf{0.916} \\
    \bottomrule
\end{tabular}
\end{table*}

We also show the class-specific results for the most frequent entity types in Table~\ref{tab:classpecific}.
As we can see, the \texttt{location} class has the highest performance, given the limited number of entities and their unambiguous nature. Among the best-performing entities, there are also \texttt{intrusion set} and \texttt{malware}, thanks to their peculiar names that are easily identifiable in the text. Classes \texttt{tool} and \texttt{campaign}, however, can cause many false negatives, which is reflected by their recall values. Indeed, these concepts are fairly easy to identify once introduced in the Knowledge Base, but their novel identification might be more difficult. For example, campaign concepts might be hard to recognize in a text, and new tools might be referenced without their introduction (or can be confused with malwares).

\begin{table}[!htpb]
  \centering
  \caption{Entity extraction results for the most frequent STIX entity types.}
  \label{tab:classpecific}
  \begin{tabular}{llll}
    \toprule
    {\bfseries Entity Type} & {\bfseries Precision} & {\bfseries Recall} & {\bfseries F1 Score}\\
    \midrule
    Attack Pattern & 0.702 & 0.861 & 0.771 \\
    Campaign & 0.615 & 0.322 & 0.409 \\
    Identity & 0.719 & 0.878 & 0.790 \\
    Intrusion Set & 0.948 & 0.936 & 0.941 \\
    Location & 0.962 & 0.913 & 0.936 \\
    Malware & 0.888 & 0.790 & 0.835 \\
    Tool & 0.949 & 0.560 & 0.698 \\
    \bottomrule
\end{tabular}
\end{table}

\subsubsection{Temporal Evolution}
\label{subsub:temporal}

The interaction of the entity extraction module with the Knowledge Base allows STIXnet to easily and quickly extract previously recognized entities and constantly update the contents of the database to guarantee high-performance values over time. To ensure that the reports' processing order does not influence the evaluation results, we randomly shuffled the dataset many times and evaluated each shuffle. We obtained a standard deviation between results close to 0 for all the evaluation metrics.

To highlight the capability of STIXnet to maintain its performance over time, we must simulate the execution of the module on subsequent reports following a temporal evolution. To do this, we evaluate its performance according to the following steps:

\begin{enumerate}
    \item We divided the dataset into batches of 5 reports each.
    \item For one of the batches, we run the entire module on each report and evaluate its performance.
    \item Entities found by the novel entity extraction sub-modules are added to the Knowledge Base after a quick manual validation.
    \item Repeat steps (2) and (3) for all the other batches.
\end{enumerate}

We think this evaluation is fair on the premises of the possible implementations of the tool. Indeed, in a real-world application, several reports are be published each day and thus need to be processed. Given the deep relationship with the Knowledge Base, we need to avoid introducing bad entities that could degrade the system's performance in the long term. Thus, the newly extracted entities can be manually validated by an analyst, who should ensure the integrity of the data instead of fully annotating the report. The results of this evaluation are shown in Figure~\ref{fig:metricevol}, where the evolution of the metrics is given in function of the processed batch of reports. Furthermore, to highlight the advantage of this approach concerning a static deployment of the Knowledge Base, we performed a different evaluation by repeating the same steps but skipping the third and thus freezing the state of the Knowledge Base in time. As shown, adding entities in the Knowledge Base and their quick validation greatly improve overall performances for all three evaluation metrics and ensure the system's updated status on the latest reports.

\begin{figure*}[!htpb]
  \centering
  \begin{subfigure}{0.32\textwidth}
     \centering
     \includegraphics[width=\textwidth]{Figures/05-Prec.pdf}
     \caption{Precision.}
     \label{fig:entextprec}
  \end{subfigure}
  \begin{subfigure}{0.32\textwidth}
     \centering
     \includegraphics[width=\textwidth]{Figures/05-Rec.pdf}
     \caption{Recall.}
     \label{fig:entextrec}
  \end{subfigure}
  \begin{subfigure}{0.32\textwidth}
     \centering
     \includegraphics[width=\textwidth]{Figures/05-F1.pdf}
     \caption{F1 Score.}
     \label{fig:entextf1}
  \end{subfigure}
  \caption{Evolution of the metrics for evaluating Entity Extraction in function of reports batch.}
  \label{fig:metricevol}
\end{figure*}

\subsection{Relation Extraction}
\label{subsub:relationextraction_results}

For evaluating the relation extraction module of STIXnet, we use the final results derived from the combined use of the rule-based approach and the deep learning based approach for relation extraction. As mentioned, relations are extracted by both submodules with a confidence value which is used to compare the results and eventually impose a threshold of confidence under which relations are discarded. In this way, we include the possibility of a non-relation between two entities. To compare the confidence values for both submodules, we normalized the cosine similarity of the embeddings given by the deep learning based approach in the $[0,1]$ range. We found that a value of 0.5 for the threshold provides a fair tradeoff between false positives and false negatives.

Evaluation has been performed by comparing the relations extracted manually from the reports with the ones that the relation extraction module of STIXnet has found. This module input is constituted by the entities extracted by the entity extraction module and works on them and the sentences in the text to find possible relations. However, the modularity enforced by the STIXnet pipeline introduces the error propagation problem from entity extraction to relation extraction. Indeed, whenever an entity is misclassified by one of the sub-modules of entity extraction, it is passed as input in the relation extraction module, which inevitably produces an error since that entity constitutes a false positive. For this reason, Table~\ref{tab:relext} shows the evaluation for both scenarios. In the first one, we execute the relation extraction module after the results from the entity extraction module have been generated. In the second one, we do not consider relationships between entities where at least one was misclassified to remove the error propagation between the modules. While the F1 scores of the two scenarios are similar, the precision when removing the error propagation effect increases by around 0.1. This type of scenario, however, affects the recall value since the overall number of true positives is decreased, but the number of false negatives is not.

\begin{table}[!htpb]
  \centering
  \caption{Relation Extraction Evaluation.}
  \label{tab:relext}
  \begin{tabular}{llll}
    \toprule
    \textbf{Scenario} & \textbf{Precision} & \textbf{Recall} & \textbf{F1 Score}\\
    \midrule
    Standard & 0.721 & 0.753 & 0.724\\
    No Error Propagation & 0.828 & 0.692 & 0.733 \\
    \bottomrule
\end{tabular}
\end{table}

While the performances of the relation extraction module in the regular scenario have a lower value with respect to the ones of the entity extraction module, we must keep in mind that with each extracted entity, the overall number of possible relationships in the text exponentially increases. Indeed, to the best of our knowledge, ours is the only model that tackles both tasks subsequently by considering each of the entity types of the STIX standard and each of the STIX relationship objects.
\section{Conclusions}
\label{sec:conclusions}

Extracting entities and relations from CTI reports becomes more challenging as the number of classes increases. This paper presents STIXnet, the first solution for automatically extracting all STIX entities and relationships in unstructured Cyber Threat Intelligence reports. Our contribution uses rule-based, NLP, and DL techniques to retrieve all STIX entities and relationships in a report automatically. We also resort to regular expression rules and an extensible Knowledge Base to effectively create an infrastructure for cyber threat analysts to consult and gather all their data and resources. The proposed pipeline enforces a modular approach to be more flexible on the user and their demands, thus separating the different tasks into different modules. The formal definition of the interaction between the sub-modules allows researchers and organizations to use our framework by adding, swapping, and removing sub-modules at will. This is particularly useful in specific scenarios where threat analysts might want to focus on specific entity classes, thus fine-tuning the framework to their needs.

In future works, we would like to explore the text extraction module of STIXnet further to enhance its results. Indeed, being the first module in the pipeline, accurate extraction and efficient artifact removal could improve the precision of the subsequent modules. Furthermore, STIX Domain Objects have many fields that vary depending on the entity types we are considering. All these fields can be filled with information that can be extracted from the sentences. Thus, it could be possible to extend the scope of our processing and provide additional intelligence when present.

\balance
\bibliographystyle{ACM-Reference-Format}
\bibliography{references}


\end{document}