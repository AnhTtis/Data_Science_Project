\section{Background}
\label{sec:background}

We now give a more thorough background on the techniques that we use in the methodology (Section~\ref{subsec:nlp}) and further expose the challenges of Cyber Threat Intelligence (Section~\ref{subsec:cti}).

\subsection{Natural Language Processing}
\label{subsec:nlp}

Given the fluency and convenience of natural language for human interactions, the field of Natural Language Processing is born to develop algorithms and models that can comprehend and analyze this type of language. These algorithms can also include Machine Learning or Deep Learning techniques, which create many opportunities for its applications. With ML and DL techniques, many documents can be automatically processed to extract named entities, detect their attributes, and retrieve their existing relations. For this reason, NLP has become particularly suitable for tasks involving the analysis of multiple reports and extracting a predefined set of data in a text~\cite{Chowdhary2020}.

NLP can be applied in every domain in which human language is the main vector through which information is conveyed, e.g., speech recognition (i.e., speech-to-text, the act of translating voice data into text data), Part-Of-Speech tagging (i.e., grammatical tagging or POS tagging, the act of determining the part of speech of a particular word based on its context) and Named Entity Recognition (i.e., NER, the act of identifying specific words in a text as a specific type of entity)~\cite{8629225}. In particular, this last technique is heavily used for Information Extraction tasks where a large number of text data is involved~\cite{Piskorski2013}, such as medical applications~\cite{Weegar2021}, scientific research~\cite{10.1007/978-3-030-00671-6_8} and cybersecurity intelligence~\cite{You2022}.

Information Extraction also comprises Relation Extraction, i.e., retrieving and classifying the semantic relationships between two (or more) tokens inside a text~\cite{10.1007/978-3-319-12580-0_2}. This task can become particularly important in sequence with the Entity Extraction task. In this way, the information in textual data becomes intertwined, and a knowledge graph of the processed text can be created~\cite{https://doi.org/10.48550/arxiv.2106.00459}.

\subsection{Cyber Threat Intelligence}
\label{subsec:cti}

According to one of the definitions of the Computer Security Research Center at NIST\footnote{\url{https://csrc.nist.gov/glossary/term/cyber\_threat}}, a cyber threat can be defined \textit{"any circumstance or event with the potential to adversely impact organizational operations [...] via unauthorized access, destruction, disclosure, modification of information, and/or denial of service"}. CTI is the field that studies these threats and analyzes the intentions of the threat actors, the techniques they use, and the tools and malwares they deploy. In this way, it is possible to profile the activities of the malicious actors and thus design more effective cyber defense strategies~\cite{WAGNER2019101589}.

Several types of intelligence are treated by CTI, which can be more or less technical depending on the target audience that must digest it. Also, each piece of intel follows a specific life cycle from planning and direction to its dissemination and integration~\cite{porkorny2018phases}.

To efficiently share and distribute the collected information, a common protocol must be in place to avoid misunderstandings between the teams that consume it. To address these problems, the STIX (Structured Threat Information eXpression) standard has been created~\cite{barnum2012standardizing}. This standardized language has been created by MITRE\footnote{\url{https://www.mitre.org/about/corporate-overview}} and is driven by the collaboration of many individuals who keep it up-to-date. Including different types of entities and relationships makes it possible to accurately represent the information in a cyber security report through a STIX file. The latest version of the software is STIX 2.1\footnote{\url{https://docs.oasis-open.org/cti/stix/v2.1/csprd01/stix-v2.1-csprd01.html}}, which includes 18 STIX Domain Objects (SDOs) and more than 100 possible relations among them.

While STIX provides a standardized language for disseminating CTI, it does not provide intelligence. One of the most popular frameworks for Cyber Threat Intelligence is the MITRE ATT\&CK framework, a publicly accessible Knowledge Base of TTPs extracted from real-world CTI reports that can be used as a foundation for building a personalized database of threat intelligence~\cite{strom2018mitre}. Entities in the ATT\&CK Knowledge Base are categorized with different labels, of which the ones of interest for the scope of this paper are \textit{tactics}, \textit{techniques}, \textit{groups}, and \textit{software}. However, since these entities do not have an official one-to-one correspondence with the STIX entity types, a mapping is needed for its referencing, which can be found in Table~\ref{tab:mitre2stix}.

\begin{table}[!htpb]
  \centering
  \caption{Conversion of ATT\&CK entity types to STIX Objects.}
  \label{tab:mitre2stix}
  \begin{tabular}{ll}
    \toprule
    \textbf{ATT\&CK Entity Type} & \textbf{STIX Object}\\
    \midrule
    Tactic & \texttt{x-mitre-tactic}\\
    Technique & Attack Pattern\\
    Mitigation & Course of Action \\
    Group & Intrusion Set \\
    Software & Malware/Tool \\
    \bottomrule
\end{tabular}
\end{table}