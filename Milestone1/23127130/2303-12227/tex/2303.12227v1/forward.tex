\section{Characterizing forward paths}\label{sec:forward}

We consider when $T$ is connected to $T'$ by a sequence of forward moves,
i.e.\ pairing exchanges which increase $c(T)$ by 1.
As proved in Lemma~\ref{lem:connect}, there is a forward path from 
$T$ to $\tp{n}$,
and dually one backward down to $\bt{n}$.
Since pairing exchange alters $c(T)$ by exactly 1, 
with $c(\bt{n}) = 1$ and $c(\tp{n}) = n$,
then the former has length $n - c(T)$, and the latter $c(T) - 1$.

Call two trees, like $\tp{n}$ and $\bt{n}$,
\emph{complementary} if there is only one simply aligned subset.
Such trees will be considered more generally in 
Section~\ref{sec:geodesic}.
Now we show there is a forward path from $T$ to $T'$
exactly when there is a tree partition 
which splits them into pairs of complementary ``star'' subtrees.

For $S \subseteq T$, let $c(S)$ denote the number of odd edges.
If $S$ is connected, then
$S \subtr T$ is isomorphic to $S' \in \mathcal{T}_{|S|}$
under an order-preserving bijection on its indices.
We distinguish whether edge parity is preserved or reversed, 
denoted $S \fiso S'$ or $S \biso S'$ respectively.
When preserved, $c(S) = c(S')$.
If reversed, $c(S) = |S| - c(S')$.

\begin{definition} 
The tree $T$ has a \emph{minmax decomposition} with $T'$,
denoted $\mmx{T}{T'}$, if there exists a
tree partition $\mathcal{P}$ such that,
for every $P \in \mathcal{P}$ with $|P| = p$,
$\trept{T}{P} \fiso \bt{p}$, $\trept{T'}{P} \fiso \tp{p}$ or 
$\trept{T}{P} \biso \tp{p}$, $\trept{T'}{P} \biso \bt{p}$.
\end{definition}

In other words, if the odd (and even) indices of $T$ and $T'$ partition
so that the induced subtrees are isomorphic to the star tree, with opposite
choices of root determined by the edge parity, 
then they have a minmax decomposition.
Note that $\mmx{T}{T}$ under the singleton
partition and $\tp{1} = \bt{1} = \{(1,2)\}$.
Dually, $\mmx{\bt{n}}{\tp{n}}$ under the trivial partition.
If $\mmx{T}{T'}$,
then the induced subtree of 
$T'$ has $p-1$ more odd edges than the one in $T$. 
Hence, $c(T) \leq c(T')$.

Call $(i,j) \in T \cap T'$ a \emph{common} edge.
Equivalently, $\{(i,j)\}$ is a simply aligned subset,
or the induced subtree for a singleton part of $\mcp{T}{T'}$.

\begin{theorem}\label{th:minmax}
There is a forward path from $T$ to $T'$ in \Gn\ if and only if $\mmx{T}{T'}$.
\end{theorem}

\begin{proof}

Let $T = T_0$, $T_1$, \ldots, $T_{k-1}$, $T_{k} = T'$ be a forward
path in \Gn.
At each step, either an odd parent/even child are
converted into two odd siblings, or two even siblings are
changed into an even parent/odd child.

Let $\mathcal{P}_1$ be the partition which is all singletons,
except for a doubleton $P$ which consists of the odd indices 
involved in the pairing exchange on $T$.
If the exchange started with parent/child,
then $\trept{T}{P} \fiso \bt{2}$ and $\trept{T_1}{P} \fiso \tp{2}$.
Otherwise, 
$\trept{T}{P} \biso \tp{2}$ and $\trept{T_1}{P} \biso \bt{2}$.

Assume after $m$ steps that $T$ has a minmax decomposition with $T_m$
for tree partition $\mathcal{P}_m$.
Then an induced subtree in $T_m$ 
contains either no even edges or exactly one even parent.
Hence the next forward pairing exchange necessarily involves 
edges associated with distinct parts.
Let $T_{m+1} = \mu_{E}(T_m)$ for $E = \{(i,j), (i',j')\}$.
Then 
$(i,j) \in \trept{T_m}{P}$ and $(i',j') \in \trept{T_m}{P'}$  
for $P, P' \in \mathcal{P}_m$ with $P \neq P'$, $|P| = p$ and $|P'| = p'$.

Suppose $(i,j), (i',j') \in T_m$ are even siblings 
with $i < j < i' < j'$.
Then $j \in P$ and $(i,j)$ is the parent in $\trept{T_m}{P} \biso \bt{p}$, 
and similarly for $(i',j')$ and $P'$.
After the pairing exchange, we have
$T_{m+1} = (T_m \setminus \{(i,j), (i',j')\}) \cup \{(i,j'), (j,i')\}$.
The even edge $(i,j')$ is the parent of $(j,i')$ as well as 
of all the odd children of $(i,j)$ in $\trept{T_m}{P}$
and of $(i',j')$ in $\trept{T_m}{P'}$. 
Hence $\trept{T_{m+1}}{P \cup P'}$ is a subtree of $T_{m+1}$ and
by construction $\biso \bt{p + p'}$.

Consider the partition 
$\mathcal{P}_{m+1} = (\mathcal{P}_m \setminus \{P, P'\}) \cup \{P \cup P'\}$.
If the even siblings comprising $\trept{T}{P}$ and $\trept{T}{P'}$ 
have the same parent, then $\trept{T}{P \cup P'} \subtr T$.
By construction the subtree is aligned with $\trept{T_{m+1}}{P \cup P'}$ 
and $\biso U_{p + p'}$.

Otherwise, let $(k,l) \in T$ be the odd parent of $\trept{T}{P} \biso \tp{p}$.
By the alignment criteria, $P \cup \even{T}{P} \subseteq [i,j]$
since $(i,j) \in \trept{T_m}{P}$ is the even parent. 
Hence, $1 \leq k < i < j < l \leq 2n$.
Without loss of generality, $(k,l)$ is not an ancestor of $\trept{T}{P'}$.
But then, since $i < i'$ by assumption,
$l \in [j+1, i'-1]$.

Let $Q \in \mathcal{P}_m$ such that 
$(k,l) \in \trept{T}{Q}$.
By induction, an induced subtree of $T$ 
has either zero or one odd edge.
Hence, $\trept{T}{Q} \fiso \bt{|Q|}$ and so $\trept{T_m}{Q} \fiso \tp{|Q|}$.
Let $K = [k, i-1]$ and $L = [j+1, l]$.
Suppose there is $(k',l') \in \trept{T_m}{Q}$ with $k' \in K$, $l' \in L$.
But this obstructs $(i,j)$ from $(i',j')$.
Hence there are an even number of indices from $\trept{T_m}{Q}$ in $K$ and 
in $L$.
But then by counting there is a child $(k',l') \in \trept{T}{Q}$ with 
$k' \in K$ and $l' \in L$, contradicting the choice of $(k,l)$.
Thus, if the pairing exchange on $T_m$ began with two even siblings, 
then $\mmx{T}{T_{m+1}}$.

Suppose instead $(i,j)$ is the odd parent of $(i',j')$ in $T_m$. 
Then $i < i' < j' < j$ and 
$T_{m+1} = (T_m \setminus \{(i,j), (i',j')\}) \cup \{(i,i'), (j',j)\}$.
Before the pairing exchange, 
although $\trept{T_m}{P'} \biso \bt{p'}$ as before,
$\trept{T_m}{P}$ may be either $\fiso \tp{p}$ or $\biso \bt{p}$. 
In either case, the new edges in $T_{m+1}$ are odd siblings, 
along with the former odd siblings of $(i,j)$ and odd children of $(i',j')$.
Hence, $\trept{T_{m+1}}{P \cup P'} \biso \bt{p + p'}$ or 
$\fiso \tp{p + p'}$ according to whether the
even parent of $(i,j)$ is in $\trept{T_m}{P}$ or not.

If the edges of $\trept{T}{P \cup P'}$ are not connected in $T$,
then by the same type of argument as above, we arrive at a contradiction.
Since $\trept{T}{P \cup P'} \subtr T$, then 
either $\biso \tp{p + p'}$ or $\fiso \bt{p + p'}$ respectively. 
Since all other parts of the partition were unchanged, $\mathcal{P}_{m+1}$
is a tree partition yielding a minmax decomposition for $T$ 
with $T_{m+1}$.

Conversely, suppose $\mmx{T}{T'}$ with tree partition $\mathcal{P}$.
Let $S = \trept{T}{P}$ and $S' = \trept{T}{P'}$
for $P \in \mathcal{P}$ with $|P| = p \geq 2$. 
Suppose $S \fiso \bt{p}$ and $S' \fiso \tp{p}$.
There is a forward path of length $p-1$ 
from $\bt{p}$ to $\tp{p}$ in $\mathcal{G}_p$.
Operating on the corresponding edges in $S$, while 
keeping $T \setminus S$ fixed, there is a forward
path from $T$ to $T'' = (T \setminus S) \cup S'$ in \Gn.
Dually, the backward path in $\mathcal{G}_p$ becomes a forward one
when $S \biso \tp{p}$ and $S' \biso \bt{p}$.
Then $T''$ has $p$ more common edges with $T'$ than $T$ does.
Inductively, the other pairs of induced subtrees are unchanged.
Hence $\mmx{T''}{T}$ for the tree partition $\mathcal{Q} = 
(\mathcal{P} \setminus \{P\}) \cup \{\{q\} \mid q \in P\}$.
\end{proof}

Suppose $\mmx{T}{T'}$. 
Then the forward path's branching skew is
$\max \{ \abs{c(T) - \frac{n+1}{2}}, \abs{c(T') - \frac{n+1}{2}} \}$
depending on whether $c(T) - 1 < n - c(T')$.
Hence, it has the least possible barrier height given 
the start and end points.
In this case, by construction, there is a bijection between 
parts of $\mcp{T}{T'}$ and simply aligned subsets.
So the path's length is $\sum |P| - 1 = n - k$,
when there are $k$ parts $P \in \mcp{T}{T'}$.
This is the shortest possible, and is generalized to geodesics between
all trees in Section~\ref{sec:geodesic}.
However, bounding the branching skew when $\notmmx{T}{T'}$ is more
challenging, and we consider several different types of paths.
