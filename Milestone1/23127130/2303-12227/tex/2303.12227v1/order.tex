\section{Connection with noncrossing partitions}\label{sec:order}

The relation $\mmx{T}{T'}$ is a partial order on \Tn\ 
with \Gn\ as its Hasse diagram and $c(T)$ as a rank function.
In other words,
$T'$ covers $T$ if $T' = \mu_{E}(T)$ where $E$ is
either an odd parent and even child or two even siblings,
i.e.\ the pairing exchange is a forward move and $c(T') = c(T) + 1$.
When viewed as a poset,
$\lub{T}{T'}$ and $\glb{T}{T'}$ are the least upper bound
and greatest lower bound, respectively.
It is worth noting the symmetry in their construction.

We show that this partial order is isomorphic to
the well-known lattice of noncrossing partitions, \NCn.
A partition of $[1,n]$ is \emph{noncrossing} if there does not exists
$1 \leq a < b < c < d \leq n$ such that $a,c$ are in one part and
$b,d$ in another.
Noncrossing partitions are still ordered under refinement.
The greatest lower bound remains the largest refinement.
However, the least upper bound is the smallest enlargement that
is also noncrossing.

\begin{theorem}
There is an order preserving bijection from \Tn\ under $\mmx{T}{T'}$
to \NCn.
\end{theorem}

\begin{proof}
Let $\nc{T}$ be the partition of $[1,n]$ obtained by
projecting $\mcp{\bt{n}}{T}$ down under $\theta: 2i-1 \rightarrow i$.
Let $P, P' \in \mcp{\bt{n}}{T}$.
Recall that $\trept{T}{P}$ consist
of an even parent and all its odd children, or all the orphan edges.

Suppose $\nc{T}$ has a crossing $1 \leq a < b < c < d \leq n$ 
with $a,c \in \theta(P)$, $b,d \in \theta(P')$, and $a, b$ least possible.
Let $i' = \min{P'}$, $j' = \max{P'}$.
Then $(i', j')$ is the even parent in $\trept{T}{P'} \biso \bt{|P'|}$
with $2a - 1 < i' = 2b - 2$ and $2d - 1 \leq j' \leq 2n-1$.
But then $(i',j')$ obstructs the edge with index $2c-1$ from 
the one with $2a-1$ in $\trept{T}{P} \subtr T$.
Contradiction.
Hence $\nc{T} \in \NCn$.

Suppose now $\mathcal{N} \in \NCn$.
Let $N \in \mathcal{N}$ and consider
$I_N = \{ 2i-1, 2i-2 \pmod{2n} \mid i \in N\}$.
Then $\min I_N$ is odd exactly when $1 \in N$.
Define the pairings $\lambda(N)$ on the ordered indices in $I_N$ 
as in the proof of Theorem~\ref{th:lub}, and let 
$T_{\mathcal{N}} = \bigcup_{N \in \mathcal{N}} \lambda(N)$.
We claim that $T_{\mathcal{N}} \in \Tn$.
If so, then $\mcp{\bt{n}}{T_{\mathcal{N}}}$ 
projects down to $\mathcal{N}$ by construction
since $\lambda(N) \fiso \tp{|N|}$ when $1 \in N$
and $\biso \bt{|N|}$ otherwise. 

Suppose there is $(i,j) \in \lambda(N)$, $(k,l) \in \lambda(N')$
with $1 \leq i < k < j < l \leq 2n$.
For $x \in \{i,j,k,l\}$, let $x'$ be $\frac{x+1}{2}$ if $x$ is odd,
and $\frac{x+2}{2}\pmod{n}$ otherwise.
Then $i', j' \in N$, $k',l' \in N'$ and either
$1 \leq i' < k' < j' < l' \leq n$ if $l \neq 2n$
or $1 = l' < i' < k' < j' \leq n$ otherwise.
Contradiction.
Thus $T_{\mathcal{N}} \in \Tn$ is the unique pre-image of $\mathcal{N}$.

Let $E$ be two unobstructed edges in $T$ and $T' = \mu_{E}(T)$.
Given how $\mcp{\bt{n}}{T}$ splits $T$,
this is a forward move if and only if distinct 
$P, P' \in \mcp{\bt{n}}{T}$ are involved.
As in the proof of Theorem~\ref{th:minmax}, 
$\mcp{\bt{n}}{T'} \setminus \mcp{\bt{n}}{T} = \{P \cup P'\}$ as a result.
But then $\mathcal{N}_{T'}$ covers $\mathcal{N}_{T}$ in \NCn,
and the bijection is order-preserving.
\end{proof}

An immediate consequence is that 
plane trees with $k$ odd edges are equinumerous with 
noncrossing partitions with $n - k + 1$ parts,
which are counted by the Narayana number
$N(n,k) = \frac{1}{n} \binom{n}{k} \binom{n}{k-1}$.  
This partition of \Tn\ differs from the common 
one according to $k$ leaves~\cite{dershowitz-zaks-80},
yielded by the classic bijections~\cite{dershowitz-zaks-86, prodinger-83}  
with \NCn.
However, a more recent enumerative result~\cite{liu-wang-li-14}
gives a bijection via vertices of odd distance from the root, 
and hence a Narayana decomposition with the same sets.

The correspondence has three related bijections:\
taking the minimal tree partition with $\tp{n}$
and/or using the even indices to partition $T \in \Tn$.
Moreover,
the connection between $\mcp{\bt{n}}{T}$ and $\mcp{T}{\tp{n}}$ 
yields insight into counting orbits in \NCn\ under Kreweras 
complementation~\cite{kcomp,kreweras-72}.

