\section{Existence of geodesics with bounded skew}\label{sec:geodesic}

Finally, we consider shortest paths, also called geodesics, in \Gn.
We show their length is determined by the 
number of simply aligned subsets, which ranges from $n$ (when $T = T'$)
down to $1$.

When there is only one, the two trees are called \emph{complementary},
consistent with lattice terminology.
The simplest example is $\tp{n}$ and $\bt{n}$, and
all other complementary pairs likewise~\cite{fpsac} have $c(T) + c(T') = n + 1$.
Moreover~\cite{fpsac},
the diameter of \Gn\ is $n - 1$, and is achieved by complementary trees.
Their V-paths necessarily pass through $\tp{n}$ and $\bt{n}$, so 
we are interested in alternative geodesics with bounded skew.

Note that removing the edge $(i,j)$ splits $T$ into two subtrees 
--- its descendents and the rest of $T$.
Denote the former as $\desc{T}{i}{j}$ and latter as 
$\notdesc{T}{i}{j}$.
One may be empty;
vacuously $\emptyset \subtr T$.

\begin{lemma}\label{lem:indstep}
Let $(i,j) \in T \setminus T'$.
Suppose either $\desc{T}{i}{j} \subtr T'$ or 
$\notdesc{T}{i}{j} \subtr T'$.
Then the edges in $T'$ with indices $i$ and $j$ are unobstructed.
\end{lemma}

\begin{proof}
Suppose $i$ pairs with $k$ and $j$ with $l$ in $T'$.
If $\notdesc{T}{i}{j} \subtr T'$, then $i < k < l < j$.
An obstructing edge must have one index in $[k+1,l-1]$ and the 
other in $[1, i-1] \cup [j+1, 2n]$.
But the latter is not possible, since any edges in $T'$ with 
an index $< i$ or $> j$ agrees with $T$ by assumption. 
When the ordering of indices is considered circularly, the case
$\desc{T}{i}{j} \subtr T'$ is symmetric.
\end{proof}

\begin{lemma}
If $T$ and $T'$ have $k$ simply aligned subsets,
then their geodesic has length $n - k$.
\end{lemma}

\begin{proof}
Construct a path
$T = T_0$, $T_1$, \ldots, $T_{n-k} = T'$ 
inductively by considering $(i,j) \in T' \setminus T_m$ with minimal 
$j - i \geq 1$.
Then $\desc{T'}{i}{j} \subtr T_m$. 
Let $E \subset T_m$ have $i, j \in \ind{E}$.
Hence $(i,j) \in T_{m+1} = \mu_E(T_m)$.
Since the number of common edges
increases monotonically to $n$,
the path is a geodesic with length $\sum_{S} (|S| - 1) = n - k$ where 
$S \subseteq T$ are the $k$ original simply aligned subsets with $T'$.
\end{proof}

Call an edge $(k,k+1)$ a \emph{stem}.  
Also define $(1,2n)$ to be one.
Let $e$ be a pairing between index $1 \leq i \leq 2n$ and $j = i+1 \pmod{2n}$.
If $e \notin T$, then Lemma~\ref{lem:indstep} applies.
Call $e$ a \emph{forward stem} if $i$ is odd, since the move 
is a forward one.
Dually, $e$ and the move are both \emph{backward} if $i$ is even.
Note that $(1,2n)$ is an odd edge but a backward stem.

For technical reasons, $T = \{(1,2)\}$ is considered
both a forward and backward stem.
Since every unrooted tree has at least two leaves,
$T \in \Tn$ has two or more stems when $n > 1$.

\begin{lemma}\label{lem:stem}
If $c(T) \geq \frac{n+1}{2}$, then $T$ has at least one forward stem.
Dually, $T$ has a backward stem when $c(T) \leq \frac{n+1}{2}$.
\end{lemma}

\begin{proof}
Let $n \geq 2$. 
Suppose $c(T) \geq \frac{n+1}{2}$, and consider 
$T' = \{(1,2k)\} \cup \desc{T}{1}{2k}$.
Assume $T'' = T \setminus T' \neq \emptyset$.
If $c(T) \geq \frac{n}{2} + 1$, then the result holds by induction on
either $c(T') \geq \frac{k+1}{2}$ or $c(T'') \geq \frac{n-k+1}{2}$.
If $c(T) = \frac{n+1}{2}$, then $n$ is odd and, we may assume, so is $k$.
Then $c(T') < \frac{k+1}{2}$ and $c(T'') < \frac{n-k+1}{2}$ implies
$c(T') \leq \frac{k-1}{2}$ and $c(T'') \leq \frac{n-k}{2}$, a contradiction.
When $k = n$, the result holds by induction on 
$\desc{T}{1}{2n} \biso T'' \in \mathcal{T}_{n-1}$ 
with $c(T'') = n - c(T) \leq \frac{n-1}{2}$.
The dual result follows from the mapping $i \rightarrow i + 1 \pmod{2n}$
on the half-edge indices,
which is a bijection on \Tn.
The image has $n - c(T) + 1$ odd edges, and the forward/backward 
orientation of stems reversed.
\end{proof}

\begin{lemma}\label{lem:zigzag}
If $T, T'$ are complementary with $c(T) = c(T')$, 
then they have a bounded geodesic.
\end{lemma}

\begin{proof}
Since $c(T) + c(T') = n+1$, 
consider odd $n \geq 3$.
By Lemma~\ref{lem:stem},
$T'$ has both a forward and backward stem.
The corresponding moves on 
$E \subset T$ and $F \subset \mu_{E}(T)$
yield $T'' = \mu_{F}(\mu_{E}(T))$ with $c(T'') = \frac{n+1}{2}$.
Then $\mcp{T''}{T'}$ has three parts, two of which are singletons.
Let $P$ be the non-singleton.
Then $\trept{T''}{P}$ and $\trept{T'}{P}$ are complementary 
with $\frac{n-1}{2}$ odd edges.
Inductively, their images in $\mathcal{G}_{n-2}$ have a bounded geodesic.
Keeping common edges fixed, so do $T$ and $T'$.
\end{proof}

\noindent
In other words, it is possible to ``zigzag'' between complementary 
$T$ and $T'$ when $c(T) = c(T')$.
Since the two moves can be made in either order,
when $c(T) < c(T')$, they can be sequenced 
not to exceed the original skew.

\begin{lemma}\label{lem:nonzigzag}
If $T, T'$ are complementary with $c(T) \neq c(T')$, 
they have a tightly bounded geodesic.
\end{lemma}

\begin{proof}
Suppose $n \geq 4$
and $c(T) < \frac{n+1}{2} < c(T')$.
There is a forward move on $E \subset T$ corresponding to $(i,i+1) \in T'$.
Let $S = \mu_{E}(T) \setminus \{(i,i+1)\}$
and $S' = T' \setminus \{(i,i+1)\}$ be the resulting complementary
subtrees.
If $c(T) = c(S) < \frac{n}{2} < c(S') = c(T') - 1$, 
the result holds inductively.
Else, $\mu_{E}(T) = \frac{n}{2} + 1 = c(T')$,
and Lemma~\ref{lem:zigzag} applies to $S$ and $S'$.
By applying the backward move first to $\mu_{E}(T)$,
the geodesic from $T$ to $T'$ will be tightly bounded.
\end{proof}

These results extend directly when there is a bijection between 
simply aligned subsets and parts of the minimal tree partition. 
Call simply aligned $S \subseteq T$, $S' \subseteq T'$ 
a \emph{block} if $S \subtr T$ and $S' \subtr T'$.
Say $T$ and $T'$ have a \emph{block decomposition} when the induced 
subtrees from $\mcp{T}{T'}$ are simply aligned,
i.e.\ complementary.
Call a pairing exchange a \emph{geodesic} move if it maintains 
a block decomposition while increasing the number of common edges. 

\begin{lemma}
Suppose $T$ and $T'$ have a block decomposition.
If $c(T') = c(T)$, then there is a bounded geodesic from $T$ to $T'$.
Otherwise, there is a tightly bounded one.
\end{lemma}

\begin{proof}
Let $S_i \subseteq T$, $S'_i \subseteq T'$ be the simply aligned pairs.
Since $S_i \subtr T$ and $S'_i \subtr T'$,
each pair can be treated independently.

Suppose $c(T) = c(T')$.
If $c(S_i) \neq c(S'_i)$, then Lemma~\ref{lem:nonzigzag} applies.
Since $\sum c(S_i) = \sum c(S'_i)$, alternate a geodesic forward move for $T$
on $S_i$ where $c(S_i) < c(S'_i)$ with a backward one on $S_j$ where 
$c(S_j) > c(S'_j)$ 
until Lemma~\ref{lem:zigzag} applies to all pairs.

If $c(T) = c(T') - 1$, again alternate moves 
until Lemma~\ref{lem:zigzag} applies 
to all pairs but $c(S_i) = c(S'_i) - 1$.
Then, as in the proof of Lemma~\ref{lem:nonzigzag}, the geodesic moves
on $S_i$ and the other pairs 
can be sequenced so that the path is tightly bounded.

Suppose $c(T') - c(T) > 1$.
Then either $c(S_i) + 2 \leq c(S'_i)$ or  
$c(S_i) + 1 \leq c(S'_i)$ and $c(S_j) + 1 \leq c(S'_j)$.
But then there is a geodesic forward move on $T$ and a backward one 
on $T'$ which decreases $c(T') - c(T)$ by 2.
Keeping common edges fixed, 
inductively applying any of these cases will not increase the skew beyond
the original bounds.
\end{proof}

\noindent
The case when $c(T') -  c(T) = 1$ differs from Lemma~\ref{lem:bind} 
because the moves for Lemma~\ref{lem:dp} are ordered, 
unlike Lemma~\ref{lem:zigzag}.
When $T$ and $T'$ do not have a block decomposition, 
moves on simply aligned subsets are not necessarily independent. 
Hence, the sequencing becomes more complicated,
and such bounds may not hold in general.         

