
The branching of an RNA molecule is an important 
structural characteristic yet difficult to predict correctly, 
especially for longer sequences.
Using plane trees as a combinatorial model for RNA folding,
we consider the thermodynamic cost, known as the barrier height,
of transitioning between branching configurations.
Using branching skew as a coarse energy approximation, we characterize
various types of paths in the discrete configuration landscape.
In particular,
we give sufficient conditions for a path to have both minimal length 
and minimal branching skew.
The proofs offer some biological insights, notably the potential 
importance of both hairpin stability and domain architecture
to higher resolution RNA barrier height analyses. 
