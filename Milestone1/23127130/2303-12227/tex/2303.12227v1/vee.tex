\section{Characterizing minimal skew V-paths}\label{sec:vee}

Even if $\notmmx{T}{T'}$, they are still connected by  
a forward V-path of length $2n - c(T) - c(T')$ through $\tp{n}$, 
respectively backward of length $c(T) + c(T') - 2$ through $\bt{n}$.
These paths have the maximum possible skew 
of $\frac{n-1}{2}$,
and hence represent the highest barrier in branching thermodynamics.
However, this can be reduced in many cases
by restricting the rearrangements to suitable subtrees.

We beging by introducing some additional notation and terminology.
Let $\mathcal{P}$ be a tree partition of $\mathcal{S} \subseteq \Tn$.
For $P \in \mathcal{P}$, let $\min{P}$ be the least index in 
$P \cup \even{T}{P}$ and $\max{P}$ the greatest.
By the alignment criteria, these are well-defined.
Note they have opposite parity.
Call $P$ \emph{odd} if $\min{P}$ is, and \emph{even} otherwise.
Let $(i,j), (i', j') \in T$.
Call $(i',j')$ the \emph{first} child of $(i,j)$ if $i' = i+1$,
respectively \emph{last} if $j' + 1 = j$.
Say $(i', j')$ is the \emph{next} sibling of $(i,j)$ if $i' = j+1$,
or \emph{previous} if $j' + 1 = i$.

\begin{theorem}\label{th:above}
Suppose $\mmx{T}{T'}$.  
Then $\mcp{\bt{n}}{T'}$ is a tree partition of $T$ and $T'$.
\end{theorem}

\begin{proof}
Let $\mathcal{P} = \mcp{T}{T'}$, $\mathcal{Q} = \mcp{\bt{n}}{T'}$,
and $Q \in \mathcal{Q}$.
We show that $\mathcal{Q}$ is an enlargement of $\mathcal{P}$,
which implies $\even{T}{Q} = \even{T'}{Q}$,
and that $\trept{T}{Q} \subtr T$.

To start, we characterize how $\mathcal{Q}$ splits $T'$.
Let $S' = \trept{T'}{Q}$.
Since $\mmx{\bt{n}}{T'}$, then $S' \subtr T'$ consists 
of some odd siblings or an even parent with some odd children. 
We claim that an odd edge is in the same induced subtree as 
all its sibling along with its even parent (if it has one).

Let $1 \in Q$.  
Then $S' \fiso \tp{|Q|}$.
Consider orphan $(i,j) \in T' \setminus S'$ with least $i > 1$. 
But then its previous sibling $(k, i-1) \in S'$ which contradicts
alignment of $\mathcal{Q}$ since $(i-1, i) \in \bt{n}$.
Hence $S'$ consists of all orphans in $T'$.
By a similar argument, if $1 \notin Q$, then $S' \biso \bt{|Q|}$ 
consists of an even parent and all its odd children.

Suppose $(i,j) \in T'$ is an even edge with $j \in P \cap Q$
for $P \in \mathcal{P}$. 
But then $\trept{T'}{P} \biso \bt{|P|}$
since $\mmx{T}{T'}$ by assumption.
Hence $(i,j)$ is the parent in $\trept{T'}{P} \subseteq S'$,
so $P \subseteq Q$.
If $P = Q$, then 
$\trept{T}{Q} \subtr T$ also, and 
$\even{T}{Q} = \even{T'}{Q}$.

Otherwise, let $(i',j') \in S' \setminus \trept{T'}{P}$
with least $i' > i$, and consider $P' \in \mathcal{P}$ with $i' \in P'$.
Since $(i',j')$ is an odd child of $(i,j) \in T'$, then
$\trept{T'}{P'} \fiso \tp{|P'|}$.
Thus, $P' \subset Q$.  
Moreover, $\trept{T'}{P \cup P'} \subtr T'$ 
and $\biso \bt{|P| + |P'|}$ by construction.

We claim that $\trept{T}{P \cup P'} \subtr T$ also.
Note $\min{P'} = i'$ odd.  
Let $\max{P'} = j''$ for $1 < i < i' < j' \leq j'' < j < 2n$.
Since $\trept{T}{P'} \fiso \bt{|P'|}$, then $(i', j'') \in T$ 
is the odd parent.
By choice of $i'$, there exists $(i'-1, k) \in \trept{T}{P} \biso \tp{|P|}$ 
with $j'' < k \leq j$.
Hence, $(i', j'')$ is the first child of $(i-1, k)$, 
and $\trept{T}{P'}$ is connected to $\trept{T}{P}$.

Inductively, $Q$ is the union of $P \in \mathcal{P}$.
Since $\even{T}{P} = \even{T'}{P}$, then $\trept{T}{Q}$ is aligned
with $\trept{T'}{Q}$.
Connectivity of $\trept{T}{Q}$ follows by building the enlargement
in order of the missing odd children of $(i,j) \in T'$.
Such a child belongs to $\trept{T'}{P} \fiso \tp{|P|}$ for 
odd $P \in \mathcal{P}$.
But then $\trept{T}{P} \fiso \bt{|P|}$.
By choice of $P$,
the odd parent in each additional $\trept{T}{P}$ 
must be the first child of some even child,
or the next sibling of an odd parent, 
already in the growing induced subtree.
The case when $1 \in Q$ proceeds along similar lines beginning with $P \ni 1$.
\end{proof}

\begin{corollary}\label{th:upper}
Let $S \in \Tn$.
Then $\mmx{T}{S}$ and $\mmx{T'}{S}$ if and only if
there exists a tree partition $\mathcal{P}$ of $T$, $T'$, and $S$
such that,
for $P \in \mathcal{P}$,
$\trept{S}{P} \fiso \tp{|P|}$ for $P$ odd
and $\biso \bt{|P|}$ otherwise. 
\end{corollary}

\begin{proof}
Suppose $\mmx{T, T'}{S}$.
Let $\mathcal{P} = \mcp{\bt{n}}{S}$.
Then by the proof of Theorem~\ref{th:above}, $\mathcal{P}$ splits $S$
as desired.
Also $\trept{T}{P} \subtr T$, $\trept{T'}{P} \subtr T'$, and
$\even{T}{P} = \even{S}{P} = \even{T'}{P'}$.
For the converse,
it suffices to observe that
pairing exchanges on distinct aligned subtrees are independent.
\end{proof}

\noindent
Note that $\mcp{\bt{n}}{S}$ is an enlargement of both 
$\mcp{\bt{n}}{T}$ and $\mcp{\bt{n}}{T'}$.  
However, it is not necessarily the least enlargement in $\polat$.
For example, consider again
$T = \{(1,8), (2,7), (3,6), (4,5)\}$ and
$T' = \{(1,4), (2,3), (5, 8), (6,7)\}$.
Then $\mcp{\bt{n}}{T} = \{\{1\}, \{3,7\}, \{5\}\}$ whereas
$\mcp{\bt{n}}{T'} = \{\{1, 5\}, \{3\}, \{7\}\}$.
Their only forward V-path is through $\tp{n}$.

\begin{theorem}\label{th:lub}
There is a unique $S \in \Tn$ with $c(S)$ minimal 
such that $\mmx{T}{S}$ and $\mmx{T'}{S}$.
\end{theorem}

\begin{proof}
Let $\mathcal{P} = \mcp{T}{T'}$.
For $P \in \mathcal{P}$ with $|P| = p$,
let $P = \{i_1,\ldots, i_p\}$ and $\even{T}{P} = \{j_1, \ldots, j_p\}$ 
in increasing order.
If $P$ odd, then $i_1 < j_1 < i_2 < \ldots i_p < j_p$.
Otherwise, $j_1 < \ldots < i_p$.
Define $\lambda(P) = \{(i_k, j_k) \mid 1 \leq k \leq p\}$ for $P$ odd,
else $\{(j_1, i_p)\} \cup \{(i_k, j_{k+1}) \mid 1 \leq k < p \}$.
Let $S = \bigcup_{P \in \mathcal{P}} \lambda(P)$.

We claim $S \in \Tn$.
As constructed, each index from $[1,2n]$
appears in exactly one ordered pair, 
and $\lambda(P)$ contains no crossing.
Suppose there are $(i,j), (i',j') \in S$ with
$1 \leq i < i' < j < j' \leq 2n$
for distinct $P, P' \in \mathcal{P}$ with
$i, j \in P \cup \even{T}{P}$, $i', j' \in P' \cup \even{T}{P'}$.

Let $J = [i+1,j-1]$.
Consider $(k,l) \in \trept{T}{P'}$.
Suppose either $k \in J$, $l \in [j+1, 2n]$ or 
$k \in [1, i-1]$, $l \in J$.
However, such an edge obstructs the edge in $\trept{T}{P}$ with 
index $i$ from the one with $j$.
Hence an edge from $\trept{T}{P'}$ has both or neither indices
in $J$ which implies that $\trept{T}{P'\cap J} \subtr T$.
The same reasoning holds for $T'$, contradicting 
minimality of $P'$.

By construction, $\mathcal{P}$ is a tree partition of $S$ as well as 
$T$ and $T'$.
Moreover, $\trept{S}{P} \fiso \tp{p}$ for $P$ odd, and 
$\biso \bt{p}$ otherwise. 
Hence $\mmx{T, T'}{S}$.
Furthermore, $S$ is the only tree 
which meets the isomorphism requirements in Corollary~\ref{th:upper} 
using $\mathcal{P}$ as the tree partition.

Let $k$ be the number of even $P \in \mathcal{P}$.
Then $c(S) = \sum_{P\ \text{odd}} p + \sum_{P\ \text{even}} (p-1) = n - k$.
We claim this is least possible.

Suppose $\mmx{T, T'}{S'}$ for $S' \neq S$.
Let $\mathcal{Q}$ be a tree partition satisfying Corollary~\ref{th:upper}
for $S'$.
Then $\mathcal{Q}$ must be a strict enlargement of $\mathcal{P}$ 
in $(O_n, \cap)$.
Also $c(S') = n - k'$ for $\mathcal{Q}$ with $k'$ even parts.  
Let $Q = P \cup P'$ for $Q \in \mathcal{Q}$, $P, P' \in \mathcal{P}$.
If $P$ and $P'$ are both odd, then $\trept{S'}{Q} = \lambda(P) \cup \lambda(P)$.
Hence, by choice of $S'$, $k' < k$.
\end{proof}

Exchanging backward moves for forward, and the roles of the star trees,
we have the following dual versions of these results.

\begin{corollary}\label{th:below}
Suppose $\mmx{T}{T'}$.  
Then $\mcp{T}{\tp{n}}$ is a tree partition of $T$ and $T'$.
\end{corollary}

\begin{proof}
Subtrees in $T$ induced by $\mcp{T}{\tp{n}}$ consist 
of an odd parent and all its even children.
A similar argument to Theorem~\ref{th:above} shows that 
$\mcp{T}{\tp{n}}$ is an enlargement of $\mcp{T}{T'}$, 
and that the corresponding induced subsets of $T'$ are subtrees 
aligned with those in $T$.
\end{proof}

\begin{corollary}\label{th:lower}
Let $S \in \Tn$.
Then $\mmx{S}{T}$ and $\mmx{S}{T'}$ if and only if
there exists a tree partition $\mathcal{P}$ of $T$, $T'$, and $S$
such that,
for $P \in \mathcal{P}$,
$\trept{S}{P} \fiso \bt{|P|}$ for $P$ odd
and $\biso \tp{|P|}$ otherwise.
\end{corollary}

\begin{corollary}\label{th:glb}
There is a unique $S \in \Tn$ with $c(S)$ maximal 
such that $\mmx{S}{T}$ and $\mmx{S}{T'}$.
\end{corollary}

\begin{proof}
Let $\mathcal{P} = \mcp{T}{T'}$ and define $\gamma(P) = 
\{(i_1, j_p)\} \cup \{(j_k, i_{k+1}) \mid 1 \leq k < p \}$ for odd 
$P \in \mathcal{P}$,
and $\{(j_k, i_k) \mid 1 \leq k \leq p\}$ otherwise.
A similar argument to Theorem~\ref{th:lub} 
for $S = \bigcup_{P \in \mathcal{P}} \gamma(P)$ holds 
with $c(S) = \sum_{P\ \text{odd}} 1 + \sum_{P\ \text{even}} 0 = k$
where $k$ is now the number of odd $P$.
\end{proof}

Let $\lub{T}{T'}$ denote the tree from Theorem~\ref{th:lub}
and $\glb{T}{T'}$ the one from Corollary~\ref{th:glb}.
Then these are the ``apex'' of 
the forward and backward V-paths with the lowest branching barrier.
If the apex of a V-path is $S$, then its branching skew is 
$\abs{c(S) - \frac{n+1}{2}}$
and length is $\abs{c(T) + c(T') - 2 \cdot c(S)}$.
Hence, the minimal skew one is a function of the number of even
and odd parts in $\mcp{T}{T'}$, and so is the length.
At least one of the V-paths through $\lub{T}{T'}$ or $\glb{T}{T'}$
has length at most $n-1$, 
although the other orientation (i.e.\ forward/backward) could be longer.


