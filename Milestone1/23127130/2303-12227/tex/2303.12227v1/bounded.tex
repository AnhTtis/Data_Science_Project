\section{Existence of bounded skew paths}\label{sec:bounded}

Although V-paths are well-characterized mathematically, 
their branching skew is biologically unfavorable.
Hence, 
we now show there are paths in \Gn, other than forward ones,
having the minimum possible branching skew.

Call a path from $T$ to $T'$ through $T_m$ \emph{bounded} if  
$c(T) - 1 \leq c(T_m) \leq c(T') + 1$
and \emph{tightly bounded} if $c(T) \leq c(T_m) \leq c(T')$.
The distinction accounts for $c(T') - c(T) \leq 1$.
In other words, such a path is bounded away from high branching 
skew trees to the extent possible given its start and end.

A \emph{planted} plane tree $T$ has a monovalent root vertex so $(1,2n) \in T$.
Call $T$ \emph{doubly planted} if $(2, 2n-1) \in T$ also.

\begin{lemma}\label{lem:dp}
If $1 < c(T) < n$, then there is a bounded path from $T$ to a doubly 
planted $T'$ with $c(T') = c(T)$.
\end{lemma}

\begin{proof}
Suppose $(1,2n) \in T$.
If $(2,2n-1) \notin T$, then a forward move on  
$(2,i), (j,2n-1) \in T$ yields a doubly planted $T''$ with $c(T'') = c(T) + 1$. 
There is a backward move on $T''$ to yield a suitable $T'$ unless 
$T'' \setminus \{(1,2n), (2,2n-1)\} \fiso \bt{n-2}$.
But then $T = \bt{n}$ and $c(T) = 1$.

Otherwise, a backward move on $(1,i), (j, 2n) \in T$ 
yields a planted $T''$ with $c(T'') = c(T) - 1$. 
If $T''$ is doubly planted, there is a suitable forward move unless
$T'' \setminus \{(1,2n), (2,2n-1)\} \fiso \tp{n-2}$.
But then $T = \tp{n}$ and $c(T) = n$.
Else, $2 < i < j < 2n-1$ and a forward move on the first and last
children of $(1,2n) \in  T''$ yields $T'$.
\end{proof}

\begin{lemma}\label{lem:bind}
If $c(T') - c(T) \leq 1$, then there is a bounded path from $T$ to $T'$.
Otherwise, there is a tightly bounded one.
\end{lemma}

\begin{proof}
Any forward path is tightly bounded.
Hence, consider $\notmmx{T}{T'}$ where 
$c(\bt{n}) = 1 < c(T) \leq c(T') < n = c(\tp{n})$.
The result holds for $n = 3, 4$ since 
there is either a bounded forward V-path through $\tp{n}$ or 
backward one through $\bt{n}$.

Suppose $c(T') - c(T) > 1$.  
Then there are $S, S' \in \Tn$ with $\mmx{T}{S}$, $\mmx{S'}{T'}$,
and $c(T) < c(S) = c(S') < c(T')$.
The existence of a bounded path from $S$ to $S'$ implies a tightly 
bounded one from $T$ to $T'$.
By the previous lemma, we may assume $S$ and $S'$ are doubly planted.
Keeping $(1,2n)$ and $(2,2n-1)$ fixed, inductively there is a 
bounded path from $\mathcal{G}_{n-2}$ connecting
$S$ and $S'$ in \Gn.
If $c(T') =  c(T) + 1$, then the same reasoning holds as for
$c(S) = c(S')$.
\end{proof}

While the skew of these paths is well-characterized, the length 
is not straightforward since the recursion has various dependencies.
A forward path not involving 
$\bt{n}$ or $\tp{n}$ has length at most $n-3$.
Let $g_n$ be the maximum length of a bounded path, or tightly bounded 
if possible, for $\notmmx{T}{T'}$ in \Gn.
Then $g_3 = 2$, $g_4 = 3$, and $g_{n} \leq (n-3) + 4 + g_{n-2}$. 
