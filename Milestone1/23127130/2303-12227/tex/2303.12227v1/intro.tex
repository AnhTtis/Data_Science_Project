\section{Introduction}

An RNA sequence is said to fold into a secondary structure via the
formation of (noncrossing, canonical) base pairings.
There are many possible secondary structures for a given
sequence, but the most biologically relevant typically
have a low free energy approximation under the nearest neighbor
thermodynamic model (NNTM).
The barrier height problem~\cite{morgan-higgs-98} then 
considers the thermodynamic cost of transitioning between low-energy
configurations.
Progress has typically 
focused on steps consisting of adding/removing a base pair, 
c.f.~\cite{dotu-etal-10, li-zhang-12, takizawa-etal-20} and 
related work discussed therein. 
Here, we take a complementary approach, focusing on larger 
structural rearrangements by using plane trees as a 
combinatorial model of RNA branching configurations.

A plane tree is a rooted tree whose subtrees are linearly 
ordered~\cite{stanley-99}.
Also know as ordered or linear trees, they are one of the 
many combinatorial families enumerated by the Catalan numbers.
Depending on the question of interest,
there are different ways\footnote{See~\cite{insights_chapt}
for an overview of the combinatorics of RNA secondary structures
and more comprehensive references.}
of associating RNA secondary structures
with trees in general, e.g.~\cite{gan-pasquali-schlick-03},
and plane trees in particular, e.g.~\cite{schmitt-waterman-94}.
As done in other branching  
analyses~\cite{ldpmath, ldpbio, dna8, parametric},
we take a low-resolution approach,
associating helices to edges and loops to vertices
with the external loop as the distinguished root vertex.

A plane tree is thus an abstract representation of 
an arbitrary RNA secondary structure.
By focusing on the overall arrangement of edges/helices
and vertices/loops, 
mathematical results have provided insight into  
the challenge of designing RNA sequences with 
a particular branching structure~\cite{dna8}, 
configurations which minimize loop energy 
costs~\cite{ldpmath, ldpbio},
and a parametric analysis of the branching 
entropy approximation~\cite{parametric}.
This work has lead both to better understanding of 
RNA prediction accuracy~\cite{regions, robust, bnb} 
as well as some new combinatorics~\cite{fpsac}.

Here, we extend this theoretical branching analysis to consider 
folding pathways between plane trees.
We move from one tree to another under a ``pairing exchange'' 
operation inspired by the challenge of encoding a particular
branched structure in a sequence~\cite{insights_chapt, dna8}.
Under a coarse approximation to the thermodynamics (branching skew),
combinatorial analysis of different types of transition paths is possible
in this model of RNA folding.
The proofs offer some biological insights.

First, there is a direct path between any two trees, i.e.\ one where 
each step increases the number of edges from the final tree by at
least one.
The edges incident on a leaf are the crucial first steps in 
such a path, and indeed the stability of RNA hairpins is a 
critical component of biological function~\cite{bevilacqua-blose-08}
and modeling accuracy~\cite{sulc-20}.
Hence, a suitable model for hairpin rearrangements~\cite{xu-chen-12} may be 
an important component of a higher resolution barrier height analysis.

Second, the branching skew of a direct path is provably bounded
when the edges of the two trees decompose into consistent blocks
that can rearrange from initial to final configurations independent
of each other.
This suggests that modeling the domain architecture of RNA 
secondary structures,
which emerges in the folding of longer 
sequences~\cite{huston-etal-21,petrov-etal-13,quinn-etal-14}, 
may be critical to the analysis of optimal folding pathways. 

