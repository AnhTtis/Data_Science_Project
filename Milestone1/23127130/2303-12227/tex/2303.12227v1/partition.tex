\section{Introducing tree partitions}\label{sec:partition}

Since a plane tree $T$ is specified as a collection of paired half-edges 
$\{(i,j) \mid 1 \leq i < j \leq 2n\}$, we distinguish when a 
subset $S \subseteq T$ is a subtree, denoted $S \subtr T$.
When $S$ is connected,
by a generalization of Lemma~\ref{lem:exdef}, 
an edge in $T \setminus S$ has either both or neither indices 
in an interval between the ordered indices of $S$.
Hence, pairing exchanges on $S$ and its
subsequent images are independent of all other edges in $T$.

Because each edge has exactly one odd index,
subsets of $T \in \Tn$ are in bijection with subsets of 
$O_n = \{1, 3, \ldots, 2n - 1\}$.
For $P \subseteq O_n$, let 
$\trept{T}{P} = \{ (i,j) \in T \mid i \in P \mbox{ or } j \in P\}$.
Let $\mathcal{P}$ be a (set) partition of $O_n$. 
We distinguish when the parts of $\mathcal{P}$ decompose $T$ into subtrees.

\begin{definition}
Say $\mathcal{P}$ \emph{splits} $T$ 
if $\trept{T}{P} \subtr T$ for every $P \in \mathcal{P}$.
\end{definition}

\noindent
The trees $\tp{n}$ and $\bt{n}$ are split by any $\mathcal{P}$,
since all edges are incident on a common vertex.
However, suppose there exists $P \in \mathcal{P}$ and 
odd integers $i < j < k$ (circularly ordered) such that
$i, k \in P$ and $j \notin P$.
Let $T = \mu_E(\tp{n})$ for $E = \{(j,j+1), (k,k+1)\}$.
Then $(j, k+1)$ obstructs $(j+1, k)$ from $(i, i+1)$ in $T$, 
and so $\mathcal{P}$ does not split $T$.
Hence, there are exactly two partitions which split every $T \in \Tn$:
$\{O_n\}$ and $\{ \{2i - 1\} \mid 1 \leq i \leq n\}$.
The latter will be referred to as the singleton partition, 
and the former as the trivial one.

To characterize paths in \Gn\ from $T$ to $T'$, 
we consider partitions of $O_n$ which split both trees.
However, we must insure that the even indices also 
partition in the same way.
For $S \subseteq T$, let $\ind{S} = \bigcup_{(i,j) \in S} \{i, j\}$
be the collection of indices.
Denote the odd ones by $\indodd{S}$, respectively 
$\indeven{S}$ for the even.

\begin{definition}
The subsets $S \subseteq T$, $S' \subseteq T'$ are \emph{aligned} if 
$\ind{S} = \ind{S'}$.
The alignment is \emph{simple} if no proper subsets of $S$ and $S'$
are also aligned.
\end{definition}

\noindent
The simply aligned subsets correspond to connected components in 
the graph with vertices in $[1,2n]$ and edges in $T \cup T'$.
It is known~\cite{fpsac} that
a pairing exchange either splits a connected component into two 
or joins two disjoint ones.
Hence, the number of simply aligned subsets changes by exactly 1
across each edge $\{T, T'\} \in \Gn$.

Observe that if $T$ and $T'$ decompose into $k$ pairs of aligned subtrees
$S_m \subtr T$, $S'_m \subtr T'$, then there is a path in \Gn\ from 
$T = T_0$ to $T' = T_k$ through $T_{m} = (T_{m-1} \setminus S_m) \cup S'_m$.
In other words, pairing exchanges on distinct aligned subtrees are 
independent.

Since $\trept{T}{P}$ and $\trept{T'}{P}$ have the same odd indices by 
definition, let $\even{T}{P} = \indeven{\trept{T}{P}}$. 
Then the induced subtrees are aligned exactly when 
$\even{T}{P} = \even{T'}{P}$.

\begin{definition}
Let $\mathcal{P}$ be a partition of $O_n$ and $\mathcal{S} \subseteq \Tn$.
Suppose $\mathcal{P}$ splits every $T \in \mathcal{S}$.
Suppose further that $\even{T}{P} = \even{T'}{P}$ for every 
$T, T' \in \mathcal{S}$ and $P \in \mathcal{P}$.
Then $\mathcal{P}$ is a \emph{tree partition of $\mathcal{S}$}.
\end{definition}

\noindent
While the trivial partition meets the alignment criteria for any 
$T$ and $T'$, the singleton one fails unless $T = T'$.

Recall that set partitions form a lattice, partially ordered under 
refinement.
Here we take the singleton partition of $O_n$ as the minimum element
since it induces subtrees with the fewest number of edges.
The trivial partition, which is a tree partition for any  
$\mathcal{S} \subseteq \Tn$, is then the maximum.
This lattice will be denoted \polat. 

\begin{lemma}
For $\mathcal{S} \subseteq \Tn$, 
there is a unique tree partition of $\mathcal{S}$,
denoted $\pi(\mathcal{S})$,
minimal in \polat.
\end{lemma}

\begin{proof}
Suppose $\mathcal{Q} \neq \mathcal{Q'}$ are both minimal
and let $\mathcal{P}$ be their greatest lower bound under refinement.
Let $P \in \mathcal{P}$.
Then $P = Q \cap Q'$ for some $Q \in \mathcal{Q}$, $Q' \in \mathcal{Q'}$.

Let $T \in \mathcal{S}$, and 
suppose $(k,l) \in T$ lies on the path
between $(i,j), (i',j') \in \trept{T}{P}$.
Since $\trept{T}{X} \subtr T$ for $X = Q, Q'$,
then $(k,l) \in \trept{T}{P}$.
Since the edges in $T$ with odd endpoints in $P$ are connected,
$\mathcal{P}$ splits $T$.

Let $i \in \even{T}{P}$.  
Let $T' \in \mathcal{S}$, and suppose $i$ pairs with $j$ in $T'$.
Since $\even{T}{Q} = \even{T'}{Q}$, then $j \in Q$.  
Likewise for $Q'$.  
Hence $j \in P$, and $i \in \even{T'}{P}$.
Since the induced subtrees are aligned,
$\mathcal{P}$ is a tree partition of $\mathcal{S}$.
Contradiction.
\end{proof}

When $\mathcal{S} = \{T, T'\}$, write $\mcp{T}{T'}$.
To produce $\mcp{T}{T'}$, 
we can start with the simply aligned subsets.  
For example, consider 
$T = \{(1,8), (2,7), (3,6), (4,5)\}$ and 
$T' = \{(1,4), (2,3), (5, 8), (6,7)\}$.
The aligned subsets have $\ind{S} = \{1, 4, 5, 8\}$,
$\ind{S'} = \{2, 3, 6, 7\}$ and induce the partition of $O_n$ with 
parts $\indodd{S}$, $\indodd{S'}$.
However, $\trept{T}{\{1,5\}} \not\subtr T$ 
and $\trept{T'}{\{3,7\}} \not\subtr T'$.
To obtain a partition which also splits these trees, it suffices to take
the union of $\indodd{S}$ and $\indodd{S'}$.

More generally, let $\{S_i\}$ be the simply aligned subsets 
for $\mathcal{S} \subseteq \Tn$, 
i.e.\ connected components in the graph on $[1,2n]$
with all edges from $T \in \mathcal{S}$.
Then $\mathcal{P} = \{\indodd{S_i} \}$ 
satisfies the tree partition alignment condition by definition.
Moreover, any enlargement of $\mathcal{P}$ in \polat\ is still aligned.
If $\trept{T}{P}$ is not connected for some $T \in \mathcal{S}$,
$P \in \mathcal{P}$, then there is an edge $(k,l) \notin \trept{T}{P}$
on the path in $T$ between some $(i,j), (i',j') \in \trept{T}{P}$.
But this can be addressed by enlarging $P$ to include 
$\indodd{S_i}$ where $(k,l) \in \trept{T}{\indodd{S_i}}$.
Inductively, 
$\pi(\mathcal{S})$ is the unique least enlargement of $\mathcal{P}$
where the induced subtrees are all connected.
Note also that $\pi(S)$ is an enlargement of $\pi(S')$ for every 
$S' \subset S$.

