\section{Pairing exchanges and branching skew}\label{sec:prelim}

Let \Tn\ denote the set of plane trees with $n$ edges.
Then $|\Tn| = \frac{1}{n+1}\binom{2n}{n}$, the $n$-th Catalan number.
Motivated by RNA secondary structures, we consider $T \in \Tn$ to be a set 
of paired half-edges.
For $i, j \in \N$, let $[i,j] = \{k \in \N \mid i \leq k \leq j\}$.
Label the boundary of $T$ counter-clockwise from the root
with $[1,2n]$ in increasing order.
Let $(i,j)$ denote the edge in $T$ which has $i$ as the label on its
left side and $j$ on the right for $1 \leq i < j \leq 2n$.

\begin{lemma}
A set $I = \{(i,j) \mid 1 \leq i < j \leq 2n\}$ is a plane tree
when each index appears in exactly one ordered pair and there do
not exist $(i,j), (i',j') \in I$ with $i < i' < j < j'$.
\end{lemma}

\begin{proof}
Consider $(1, k) \in I$. 
If $k$ is odd, then either there exists an $(i,j) \in I$ with $1 < i < k < j$
or an index in $[2,k-1]$ that is unpaired or in more than one pairing.
Since $k$ must be even, induct on the pairings with indices in 
$[2,k-1]$ and $[k+1,2n]$.
\end{proof}

In other words, there is a simple bijection between noncrossing perfect
matchings on $2n$ endpoints and plane trees with $n$ edges.
Previous work~\cite{fpsac} considered the comparable operation on 
matchings to the pairing exchange defined below with the goal of 
better understanding meanders 
(interpreted as pairs of noncrossing perfect matchings which form
a single closed loop). 

Here we consider plane trees as a low-resolution model
of RNA secondary structures,
and analyze (very approximately) the thermodynamic cost of 
moving around this branching configuration landscape.
Inspired by the challenge of minimizing alternative lower-energy 
configurations when designing RNA secondary structures
(c.f.\ Fig.~1 in \cite{dna8}),
we transition from one tree to the next by breaking apart and ``repairing'' 
two edges. 

We start by applying to edges in $T$ 
the common familial terminology for vertices in rooted trees, 
i.e.\ parent/child, siblings, ancestor/descendent, etc.
Additionally, an edge incident on the root vertex is called an orphan.
Two edges in $T$ are \emph{unobstructed} if they are
incident on the same vertex, in which case they are either parent/child 
or siblings.

We define a \emph{pairing exchange} on unobstructed edges 
$E = \{(i,j), (i',j')\} \subseteq T$ as 
\[
\mu_E(T) = 
(T \setminus E) \cup 
\left \{
\begin{array}{cc} 
(i, i') \mbox{ and } (j', j) & \mbox{ if } i < i' < j' < j \\
(i, j') \mbox{ and } (j,i')) & \mbox{ if } i < j < i' < j' 
\end{array} 
\right \} 
\]
and claim that converting a parent/child into siblings, 
or vice versa, introduces no crossings.

\begin{figure}
\centering
\includegraphics[width = .5\textwidth]{pairingexch}
\caption{A pairing exchange on unobstructed edges with 
indices $1 \leq a < b < c < d \leq 2n$
converts siblings $(a,b)$, $(c,d)$ into parent/child 
$(a,d)$, $(b,c)$, and vice versa.
Note how the incident vertices split and merge.
However, edges in the four subtrees with indices exclusively in 
$A = [1,a-1] \cup [d+1,2n]$,  
$B = [a+1,b-1]$, $C = [b+1,c-1]$, or $D = [c+1,d-1]$ 
are unaltered.
\label{fig:pairingexch}}
\end{figure}

\begin{lemma}\label{lem:exdef}
The pairing exchange operation is well-defined.
\end{lemma}

\begin{proof}
Let $1 \leq a < b < c < d \leq 2n$ be the indices of two edges in $T$.
Let $A = [1,a-1] \cup [d+1,2n]$, 
$B = [a+1,b-1]$, $C = [b+1,c-1]$, and $D = [c+1,d-1]$.
Observe that if the edges are $(a,d)$ and $(b,c)$, then all other edges
must have indices in either $A$ or in $B \cup D$ or in $C$ exclusively.
However, if $(a,d)$ and $(b,c)$ are parent and child, then there
cannot be an edge $(k,l)$ with $k \in B$ and $l \in D$.
A similar argument holds if $(a,b)$ and $(c,d)$ are siblings.
\end{proof}

As illustrated in Figure~\ref{fig:pairingexch},
pairing exchanges are reversible operations.  
Let \Gn\ be the (undirected) graph with vertex set \Tn\ and edges 
which connect two plane trees that differ by a single pairing exchange.

Before proving that \Gn\ is connected,
we distinguish two trees which have the maximum degree of 
$\binom{n}{2}$ in \Gn.
Let $\tp{n} = \{(2i-1,2i) \mid 1 \leq i \leq n\}$
and $\bt{n} = \{(1,2n)\} \cup \{(2i,2i+1) \mid 1 \leq i \leq n-1\}$.
Then $\tp{1} = \bt{1}$, and $\mathcal{G}_1$ consists of a single vertex.
For $n \geq 2$,
$\tp{n}$ and $\bt{n}$ differ in the choice of root; 
both are ``star'' trees with $n$ unobstructed edges.

\begin{figure}
\centering
\includegraphics[width = .5\textwidth]{g3-top}
\caption{The graph $\mathcal{G}_3$ with $\bt{3}$ on left, 
$\tp{3}$ on right, and the three plane trees $T \in \mathcal{T}_3$
with $c(T) = 2$ in the middle.
Dashed lines are pairing exchanges.
The number of odd edges increases by 1 moving left to right.
\label{fig:g3top}}
\end{figure}

\begin{lemma}\label{lem:connect}
The graph \Gn\ is connected.
\end{lemma}

\begin{proof}
We claim there is a path in \Gn\ from $T$ to $\tp{n}$.
If $(1,2) \in T$, then inductively the subtree with  
indices in $[3,2n]$ is connected by pairing exchanges to $\tp{n-1}$.
Else the edges $(1,k), (2,l) \in T$ are unobstructed, and there is an
edge in \Gn\ between $T$ and 
$[T \setminus \{(1,k), (2,l)\}] \cup \{(1,2), (k,l)\}$.
\end{proof}

Dually, $T$ is connected to $\bt{n}$ by
first considering the edge $(1,2n)$, and then successive $(2i, 2i+1)$.
Thus, any two trees are connected by a path through $\tp{n}$ or 
one through $\bt{n}$.
Unless a star tree is one of the endpoints, the path is indirect
since it effectively erases most pairing information in the first
tree before replacing it with second.
Moreover, although mathematically simple, 
these are the highest possible barrier paths in terms of branching
thermodynamics.

Previous results~\cite{ldpbio, insights_chapt, parametric} demonstrated
that branching is locally favorable but globally balanced by 
increasing the number of leaves ---
since hairpins are the most energetically expensive type of 
loop structures~\cite{turner-mathews-10}.
Hence, a low barrier path in \Gn\ passes through trees with 
a low degree of branching.
Since tracking changes in branching degree
under pairing exchanges is complicated, 
we instead consider ``branching skew.''

We start by defining the parity of an edge.
Since $(i,j) \in T$ has $\frac{j-i-1}{2}$ descendents,
exactly one of $i$ and $j$ is odd.
Call the edge \emph{odd} if $i$ is, and \emph{even} otherwise.
Let $c(T)$ denote the number of odd edges in $T$.
Then $1 \leq c(T) \leq n$,  
with $c(T) = 1$ exactly when $T = \bt{n}$ and $n$ only if $\tp{n}$.
Moreover, a pairing exchange alters the number of odd edges by exactly one.   
An example is seen in Figure~\ref{fig:g3top}.

Note that edge parity along a path in $T$ from the root vertex 
to a leaf must alternate, and all orphan edges are odd.
Hence, if $c(T) = k$, then the maximum possible vertex degree
in $T$ is either $k$ or $n - k + 1$.
A tree which achieves this is $k$ orphans
with one having $n - k$ even children.
Thus, $c(T)$ is well-behaved under pairing exchanges, and
yields an upper bound, seldom tight, on vertex degree.

More precisely, we are interested in minimizing the maximum possible
vertex degree over paths in \Gn.
Define the \emph{skew} of $T$ to be 
$\abs{c(T) - \frac{n+1}{2}}$.
This is maximal at both $\tp{n}$ and $\bt{n}$, and decreases to 
a minimum of $0$ or $1/2$ when 
the number of odd and even edges are most evenly balanced.
The skew of a path $T = T_0$, $T_1$, \ldots, $T_k = T'$
is $\max_{0 \leq m \leq k} \abs{c(T_m) - \frac{n+1}{2}}$.

Call a pairing exchange a \emph{forward} move if $c(T)$ increases,
and \emph{backward} otherwise.
In Section~\ref{sec:forward}, we characterize when there is a 
path from $T$ to $T'$ consisting only of forward moves.
Call this a forward path, and the reverse a backward one.
Such a path has the least increase in skew possible given 
the start and end.

Lemma~\ref{lem:connect} showed that,
even if there is not a forward
path from $T$ to $T'$, they are still connected
by a pair of forward paths through $\tp{n}$.
Dually, backward through $\bt{n}$.
More generally, we call a path from $T$ to $T'$ 
a \emph{forward V-path} (respectively \emph{backward}) 
if there exists $S \in \Tn$ with $S \neq T, T'$
such that there is a forward (resp. backward) path from $T$ to $S$ 
and also from $T'$ to $S$.
In Section~\ref{sec:vee}, we characterize the minimal skew V-paths,
and make explicit in Section~\ref{sec:order} 
the connection with the well-studied lattice of noncrossing partitions.

In Section~\ref{sec:bounded}, we show that when forward and backward 
moves are interspersed, it is possible to have paths in \Gn\ whose skew
exceeds the start and end by at most 1.
We conclude in Section~\ref{sec:geodesic} by characterizing shortest
paths, and proving that their skew is similarly bounded
under certain conditions.

