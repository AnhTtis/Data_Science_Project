\documentclass[aps,epsfig,prl,twocolumn]{revtex4-1}
%\documentclass[aps,epsfig,prl,preprint]{revtex4-1}
%\bibliographystyle{apsrev4-1}

\usepackage{amsmath}
\usepackage{graphicx}

\def\nn{\nonumber}
\def\be{\mbox{\bf e}}
\newcommand{\bsigma}{\mbox{\boldmath $\sigma$}}
\newcommand{\bepsilon}{\mbox{\boldmath $\epsilon$}}
\newcommand{\bpi}{\mbox{\boldmath $\pi$}}
\newcommand{\bv}{\mbox{\boldmath $v$}}
\newcommand{\bu}{\mbox{\boldmath $u$}}
\newcommand{\bq}{\mbox{\boldmath $q$}}
\newcommand{\bk}{\mbox{\boldmath $k$}}

\begin{document}

\title{Stationary Two-State System in Optics using Layered Materials}

\author{Ken-ichi Sasaki}
\email{ke.sasaki@ntt.com}
\affiliation{NTT Research Center for Theoretical Quantum Physics and NTT Basic Research Laboratories, NTT Corporation,
3-1 Morinosato Wakamiya, Atsugi, Kanagawa 243-0198, Japan}

%\setlength{\baselineskip}{26pt}

\date{\today}

\begin{abstract}
 When electrodynamics is quantized in a situation where
 the electrons exist only at a flat surface such as graphene, 
 one of the Maxwell equations appears as a local part of the Hamiltonian. 
 As a consequence of gauge invariance,
 any physical state has to be a zero-energy state of the local Hamiltonian.
 We construct two stationary quantum states;
 one reproduces scattering and absorption of light, which is familiar in classical optics
 and the other is more fundamentally related to photon creation.
 These two states are inseparable by the Hamiltonian and forming a two-state system, 
 but there is a special number of surfaces for which two states are decoupled.
 The number is $2/\pi \alpha$ where $\pi \alpha$ is the absorption probability of single surface.
\end{abstract}

%\pacs{73.20.-r, 73.40.-c, 72.80.Vp}
\maketitle



Two-state systems are the simplest case that exhibit the essential features of quantum physics.
Spin-$1/2$, two base states for the ammonia molecule, 
clockwise and counterclockwise persistent currents of superconducting qubit, 
and the hydrogen molecular ion are well-known examples of two-state quantum system.~\cite{Feynman1989,Sakurai2017}
In optical science, two-state polarization of radiation field has the fundamental role in 
discovering the law of physics, such as the non-cloning theorem,~\cite{Wootters1982} 
and as the source of quantum entanglement,~\cite{Kocher1967,Aspect1982,Ou1988,Takesue2004} 
and in modern optical transmission technology 
to double the amount of information transmitted simultaneously.
Recently, two-state coherent phase of a pulsed field has been related to an artificial spin.~\cite{Wang2013,Honjo2021}
There can be more two-state systems other than these in optics. 
Even if the system seems to have many degrees of freedom, 
we sometimes notice that from a certain point of view
we can extract only two important states that play an essential role in the mechanism of the phenomenon
under investigation.~\cite{Bohm1951,Pines1952,Sasaki2006c,Sasaki2008e}
In this paper, we show from the viewpoint of quantum electrodynamics~\cite{sakurai67}
that there are two stationary quantum states in optics with layered materials.
While each state is known, 
what was missing in the past study is that they are a pair of degenerate quantum states, 
like up and down spin states.
Our formulation reveals that the emergence of the two states is related to the gauge invariance
and that the origin of the two-state degree of freedom is inherent in the 
chiral nature of photon, namely, right- and left-going components.


The space is divided into two regions, $x<0$ (left) and $x>0$ (right),
by an absorbing surface located at $x=0$.
Some fraction of a continuum incident light coming from $x=-\infty$
is absorbed only by the surface where the electronic current ${\bf j}$ exists, 
and the remaining is either reflected back to $x=-\infty$ or transmitted forward to $x=+\infty$.
In a classical theory of optics, 
the reflection and transmission coefficients are obtained from
Amp\'ere's circuital law, 
$c^2 \nabla \times {\bf B} = \frac{\bf j}{\epsilon_0} + \dot{\bf E}$.
By integrating it with respect to an infinitesimal element around the surface,
the magnetic field is discontinuous at the surface,
$\epsilon_0 c^2 \{ B_y(0_+)-B_y(0_-) \}= \int_{0_-}^{0_+} j_z dx$,
due to the localization of the current 
$j_z = J_z \delta(x)$ about the surface.~\cite{Sasaki2020a,Sasaki2020b}
Since the magnetic field is expressed in terms of the radiation gauge field 
by $B_y=-\partial_x A_z$, we obtain the boundary condition
\begin{align}
 \partial_x A_z|_{x=0_+} - \partial_x A_z|_{x=0_-} + \frac{J_z}{\epsilon_0 c^2}=0,
 \label{eq:bc}
\end{align}
showing that the first derivative of $A_z(x)$ is discontinuous at the surface,
meanwhile, $A_z(x)$ must be continuous.
Now, in quantum field theory,
$A_z(x)$ as well as $J_z$ is a field operator consisting of particle creation and annihilation operators.


It is possible to show that the boundary condition appears as a part of Hamiltonian
and is related to the gauge invariance in the fundamental level.
For the photon's Hamiltonian of
\begin{align}
 H_{ph} = \frac{\epsilon_0}{2} \int_{-\infty}^{+\infty} dx \left\{ \dot{A}_z(x)^2 + c^2 ( \partial_x A_z(x))^2 \right\},
\end{align}
the integral may be divided into a left region, right region,
and the surface $[0_-,0_+]$.
Then, the Hamiltonian of the radiation field at the surface is given by 
$\epsilon_0 c^2 A_z(x) \partial_x A_z(x) |_{0_-}^{0_+}$.
By combining it with the gauge coupling of the interaction Hamiltonian, $H_{int} = A_z(0) J_z$,
we obtain the Hamiltonian at the surface $H_{ph}|_{x=0}+H_{int} (\equiv H_{local})$ as
\begin{align}
 \epsilon_0 c^2 A_z(0) 
 \left\{ \partial_x A_z(x)|_{x=0_+} - \partial_x A_z(x)|_{x=0_-} + \frac{J_z}{\epsilon_0 c^2} \right\}.
\end{align}
The Hamiltonian has to be invariant with respect to a residual gauge transformation
$A_z(0) \to A_z(0) + \partial_z \lambda(x,z)|_{x=0}$, where $\lambda(x,z) = C z$ satisfies
$\partial^2_z \lambda(x,z)=0$.
Thus, rather than saying that Eq.~(\ref{eq:bc}) is satisfied as the expectation value (Ehrenfest's theorem),
any physical state must appear as a zero-energy state of the local Hamiltonian because
it annihilates physical states.





We will show that the Hilbert space holds two quantum states
denoted by $|\Phi_a \rangle$ and $|\Phi_b \rangle$.
Each state is not single quantum but a specific combination of light and matter quanta.
The former reproduces scattering and absorption of light by matter 
which is consistent with the classical description of optics,
and the latter is more fundamentally related to the photon creation or light emission.
To simplify the notation, we use ${\cal B}$ to denote the operator, 
\begin{align}
 {\cal B} \equiv
 \partial_x A_z(x)|_{x=0_+} - \partial_x A_z(x)|_{x=0_-} + \frac{J_z}{\epsilon_0 c^2}.
\end{align}
We use the Schr\"odinger picture (representation),~\cite{Sakurai2017,sakurai67} 
and the operators are time-independent.
Moreover, the state vectors we construct are stationary states with zero-energy of the local Hamiltonian,
the variable of time is concealed. 
The gauge field operator is a sum of right- and left-moving components
as $A_z(x) = A_R(x) + A_L(x)$.
The electric field operator is denoted by $E_z(x)$
which forms canonical equal-time commutation relations with $A_z(x)$, 
such as $\left[ A_z(x),E_z(x') \right] = -i\frac{\hbar c}{\epsilon_0} \delta(x-x')$, 
$\left[ A_R(x),E_R(x')  \right] = -i\frac{\hbar c}{2\epsilon_0} \delta(x-x')$, and 
$\left[ A^{+}(x),E^{-}(x')  \right] = -i\frac{\hbar c}{2\epsilon_0} \delta(x-x')$, 
where $-$ and $+$ of the superscript stands for 
the annihilation and creation part of the field operators, respectively.
Note that $A_z(x)$ commutes with $B_y(x')$ and so with ${\cal B}$.
However, each component satisfies
$[\partial_x A_R(x),A_R(x')]= +i\frac{\hbar}{2\epsilon_0} \delta(x-x')$ and 
$[\partial_x A_L(x),A_L(x')]= -i\frac{\hbar}{2\epsilon_0} \delta(x-x')$,
which are canceled in
$[B_y(x),A_z(x')]= 0$.







First, we define an entangled state 
that is a superposition of right- and left-going photons
counter-propagating along the $x$-axis.
The spatially uniformity is perturbed by the absorbing surface as
\begin{align}
 |a \rangle = N_a \int_{-\infty}^{+\infty} dx 
 & \left[ \theta(-x) + t \theta(x) \right] e^{i\frac{\omega}{c}x} A_R(x)|0\rangle \nn \\
 &+ r \theta(-x) e^{-i\frac{\omega}{c}x}A_L(x) |0 \rangle,
\end{align}
where $t$ ($r$) is the transmission (reflection) coefficient, 
$N_a$ is a normalization constant (space is taken to be infinity),
and $\theta(x)$ is the step function; $\theta(x)=1$ for $x>0$ and vanishes otherwise.
The following quantum state reproduces the result obtained in classical optics,~\cite{Shen2005}
\begin{align}
 |\Phi_a \rangle \equiv |a \rangle \otimes |vac\rangle 
 + \frac{1}{g}\langle 0| E_z(0) |a\rangle |0\rangle \otimes J_z|vac \rangle.
\end{align}
Since the electric field is continuous at the surface (as we see below), 
we can also take either $x=0_+$ or $0_-$ for the argument of $E_z$.
The ground state of the photon (matter) is expressed by $|0\rangle$ ($|vac\rangle$),
and $|\Phi_a \rangle$ is a time-independent state vector.
The classical physics is reproduced by replacing $|0\rangle$ with a coherent state.


To specify $|\Phi_a \rangle$ uniquely, 
we use two physical conditions.
First condition is the continuity of the electric field 
$\langle 0 | E_z(0_+) |a \rangle = \langle 0 | E_z(0_-) |a \rangle$
giving $1 + r = t$.
Second is the necessary condition with respect to a physical state,
\begin{align}
 \langle 0|\otimes \langle vac| {\cal B} |\Phi_a \rangle = 0.
 \label{eq:physc}
\end{align}
The meaning of this condition may become more clear 
by taking the complex conjugate of the above,
namely, the physical state should not have an overlap to a state 
generated by acting the vacuum with ${\cal B}$. 
We note that 
$\langle vac|J_zJ_z|vac\rangle = g \sigma$, 
where $\sigma$ is dynamical conductivity (or vacuum polarization)
and $g$ is introduced for dimensional consistency (whose value is not of particular importance here).
In the case of graphene, the dynamical conductivity is given by $\sigma=\pi \alpha \epsilon_0 c$,
where $\alpha$ is the fine-structure constant
(the absorption probability is $\pi \alpha \sim$ 2.3 percent).~\cite{Shon1998,Nair2008,ando02-dc}
Using commutation relations, we obtain from Eq.~(\ref{eq:physc})
\begin{align}
 i \frac{\hbar}{2\epsilon_0} t -i \frac{\hbar}{2\epsilon_0} (1-r) 
 + \frac{\sigma g}{\epsilon_0 c^2}i \frac{1}{g} \frac{\hbar c}{2\epsilon_0} t= 0.
\end{align}
Thus, these two conditions lead to
$t = \frac{2}{2+\pi \alpha}$ and $r = \frac{-\pi \alpha}{2+\pi \alpha}$.
Moreover, we obtain $\langle \Phi_a| H_{local}  |\Phi_a \rangle = 0$ which 
also reproduces the energy (or probability) conservation given by Poynting vector, 
$1 - r^2 - t^2 = \pi \alpha t^2 (\equiv A_{abs})$.~\cite{Sasaki2020a,Sasaki2020b}
To calculate $\langle a |A_z(0)$, 
we take care that two-point correlation function 
such as $\langle 0 | [A_z(x+\varepsilon),A_z(x)] |0 \rangle$
has an anomalous commutator part when $\varepsilon$ is tiny (the length scale of wave function spreading).
It is actually defined by $\lim_{\varepsilon \to 0}\langle 0 | [(A_R(x+\varepsilon)+A_L(x-\varepsilon)),A_z(x)] |0 \rangle$
which gives 
$\lim_{\varepsilon \to 0} \varepsilon \langle 0 | [\partial_x A_R(x)-\partial_x A_L(x),A_z(x)] |0 \rangle 
= \lim_{\varepsilon \to 0} \varepsilon \frac{i\hbar}{\epsilon_0}\delta(0)$.



%%%%%%%%%%%%%%%%%%%%%%%%%%%%
\begin{figure}[htbp]
 \begin{center}
  \includegraphics[scale=0.5]{QFT_graphene.eps}
 \end{center}
 \caption{Schematic of the two-state system.
 The arrow from left to right expresses light ray, which is reflected or transmitted by a thin film (graphene).
 The dashed line is zero photon state.
 Only the off-diagonal matrix element of $H_{local}$, $\langle \Phi_b | H_{local} | \Phi_a \rangle$, 
 is non-zero and proportional to $tb_R + rb_L$.
 }
 \label{fig:1}
\end{figure}
%%%%%%%%%%%%%%%%%%%%%%%%%%%%

Next we construct the other state vector,
\begin{align}
 |\Phi_b \rangle 
 \equiv |b \rangle \otimes |vac\rangle 
 + \frac{1}{g} \langle 0| E_z(0) |a\rangle |0\rangle \otimes |exc\rangle,
\end{align}
where $|b\rangle =N_a \int_{-\infty}^{+\infty} dx
\left\{ b_{R} \theta(x) A_R(x) + b_{L} \theta(-x) A_L(x)\right\}|0 \rangle$
represents light emission from the surface.
The presence of $\langle 0| E_z(0) |a\rangle$ on the last term 
shows that the emission is the stimulated one.
The state vector of the matter $|exc\rangle$ denotes the excited state.
This state is specified primarily by $J_z^{-}|exc \rangle =g F |vac \rangle$.
Since $J_z^{+}J_z^{-}$ contains the number operator of electron-hole pair $N_{eh}$, 
we have $g^2 |F|^2= g\sigma \langle exc|N_{eh}|exc \rangle = g^2 \sigma^2 A_{abs}$.
The magnitude of $F$ is given by $|F|=\sigma \sqrt{A_{abs}}$, 
while the sign of $F$ depends on the matter system.~\cite{Durrant1998,Cray1998,Sasaki2022a}
In the same way as $|\Phi_a \rangle$,
the continuity of the electric field with respect to $|b \rangle$, 
$\langle 0 | E_z(0_+) |b \rangle = \langle 0 | E_z(0_-) |b \rangle$,
and the physical state condition $\langle 0|\otimes \langle vac| {\cal B} |\Phi_b \rangle = 0$
are sufficient to obtain that $b_R = b_L (\equiv b)$ and 
\begin{align}
 b =  -\frac{F}{2\epsilon_0 c} t.
\end{align}
This state automatically satisfies $\langle \Phi_b| H_{local} |\Phi_b \rangle = 0$.
Thus, we found that absorption means the two state-vectors.




Though the zero-energy conditions $\langle \Phi_a |H_{local}|\Phi_a \rangle=0$
and $\langle \Phi_b |H_{local}|\Phi_b \rangle=0$ are both satisfied,
the off-diagonal element $\langle \Phi_b |H_{local}|\Phi_a \rangle$ is generally non-zero as
\begin{align}
 \langle \Phi_b | H_{local} |\Phi_a \rangle 
 = \frac{\epsilon_0 g}{\sigma t^2} \varepsilon \delta(0) (t+r)b^*,
\end{align}
and therefore the two states evolve inseparably by the Hamiltonian.
We shall introduce two states written as 
$|\Phi_\theta^\pm \rangle = |\Phi_a \rangle \pm e^{i\theta}|\Phi_b \rangle$,
where the relative phase $\theta$ is determined so that 
$ \langle \Phi^+_\theta | H_{local}  |\Phi^-_\theta \rangle = 0$ 
or ${\rm Im}[e^{i\theta}(t+r)^* b]=0$
(i.e., $e^{i\theta}b$ is a real number when $t+r$ is a real number).
Then, their energy expectation values become non-zero as 
$\langle \Phi_\theta^\pm |H_{local}| \Phi_\theta^\pm \rangle = \pm E_\theta$. 
An analogy to the electron spin is useful in understanding the physics.
Suppose we have up and down spin states, $|\uparrow \rangle $ and $|\downarrow \rangle$,
which are the eigenstates of $\sigma_z$.
Let us assume that each spin state is in a magnetic field along the $y$-axis.
The energy expectation value vanishes.
Now, $|\Phi_\theta^\pm \rangle$ corresponds to 
$|\pm \rangle \equiv (|\uparrow \rangle \pm e^{i\theta} |\downarrow \rangle)/\sqrt{2}$.
When $\theta$ is determined so that $\langle + |\sigma_y | - \rangle = 0$,
we have $\langle \pm |\sigma_y | \pm \rangle = \pm 1$.
Note that the general expression of $H_{local}$ in terms of the spin is 
$H_{spin}=a_+ \sigma_+ + a_- \sigma_-$, where $\sigma_\pm = \sigma_x \pm i \sigma_y$ is the ladder operators
and $a_\pm$ is the amplitudes.
It satisfies $\langle \uparrow|H_{spin}|\uparrow\rangle =0$ and $\langle \downarrow|H_{spin}|\downarrow\rangle =0$.



Because the reflection (and transmission) is calculated by 
$\langle 0 |E_L(-\infty)|\Phi^\pm_\theta \rangle$ ($\langle 0 |E_R(+\infty)|\Phi^\pm_\theta \rangle$),
$|\Phi_b \rangle$ appears as a correction to the reflection (and transmission) coefficient, 
and there are two possibilities of the correct reflection coefficient as
\begin{align}
 r \pm e^{i\theta} b.
\end{align}
Since the lower energy state is realized,
the actual reflection is given by either sign.
The other perturbations not included in this analysis such as Raman effect and stacking order
contribute to the energy and should affect the choice. 
This can be verified by experiments.~\cite{Sasaki2022a}
In the above spin analogy, these are perturbations proportional to $\sigma_x$ (that cause a change in $\theta$)
and $\sigma_z$ (that mixes $| \pm \rangle$).
Note in particular that $|\Phi_b \rangle$ appears
not as a correction to the dynamical conductivity. 
Meanwhile, the (spontaneous) emission is generally treated as a loss, 
and it is often included as the phenomenological relaxation constant in the dynamical conductivity. 
%For example, in Ref~\onlinecite{El-Sayed2021} the authors introduce relaxation constants 
%for the Drude-Lorentz oscillator model to interpret the measured optical constants for graphene. 
%However, too large relaxation constant (or very short lifetime) for the Drude term (0.6 fs) 
%already indicates the possibility that the inclusion of such a relaxation parameter is an unnatural interpretation. 
%Our model is consistent also with a theoretical result that 
%the dynamical conductivity is free from such a correction 
%when graphene is undoped (i.e., charge neutrality condition is satisfied).~\cite{ando02-dc}



Because the two states slightly overlap, i.e.,
$\langle \Phi_b | \Phi_a \rangle = F^*/\sigma \ne 0$, 
a simple interpretation of a two-state system is difficult to be applied.
Apparently, from the reflected (or transmitted) light we will observe, 
the signal is inseparable between $|\Phi_a \rangle$ and $|\Phi_b \rangle$.
If the two states were truly indistinguishable from each other, 
the situation is something like an electron spin in the absence of a magnetic field, and 
we need to face the issue of control.
However, at least when $t+r=0$ is satisfied, $\langle \Phi_b |H_{local}|\Phi_a \rangle$ vanishes.
Then, we define new base states,
\begin{align}
 & |\Phi_A \rangle = |\Phi_a \rangle, \\
 & |\Phi_B \rangle = |\Phi_a \rangle -\frac{\sigma}{F} |\Phi_b \rangle,
\end{align}
which become orthogonal and are not mixed by the local Hamiltonian.
Such an interesting situation is indeed possible to realize 
for multilayer graphene.~\cite{Sasaki2020a}
Let the layer number is $N$.
The corresponding transmission and reflection coefficients 
are given by the replacement $\pi \alpha \to N \pi \alpha$ in $t$ and $r$.
Thus, $t+r=0$ when $N=2/\pi \alpha$.
Quite recently, we found also that one can access $|\Phi_b \rangle$
by suppressing the contribution of the other state to the reflection ($r \sim 0$)
using a multilayer graphene (about 20 layers) and 
destructive interface effect of 
a SiO$_2$/Si substrate.~\cite{Sasaki2022a}




The application of the above analysis to multilayer graphene is straightforward
when the electronic current flowing between layers is negligible.
When the electrons can hop between adjacent layers, 
the gauge coupling $A_x(x)J_x(x)$ is not negligible.
This seems to be an another important issue relating to the stacking effect.
A detailed analysis of it will be shown elsewhere.
It is also straightforward to include the degrees of freedom of photon polarization.
Namely, we will have additional two states for $A_y$ as well as $A_z$.
Thus, the system holds four states in total.
Because the characteristics of the matter system enters through the current operator only,
our formulation is applicable to various layered systems besides graphene 
by taking an appropriate $\sigma$ and $F$ for the corresponding current operator $J$.
A fundamental issue is that $A_y$ is just a copy of $A_z$, or 
there is some sort of correlation between them.
In the case of pristine graphene, the current operators $J_y$ and $J_z$
excite different electron-hole pairs and therefore
such a correlation is suppressed by optical selection rule.~\cite{Sasaki2011}
%Because $A_x(x)$ is related to $A_z(x)$ through the Coulomb gauge condition $\partial_x A_x + \partial_z A_z=0$.
%The coupling may be related to the residual gauge transformations $\lambda(x,z)=C(x)z$, 
%where the coefficient has now $x$-dependence $C(x)$.






To conclude, we formulated a two-state system
that originates from the different states of the matter in layered materials
$J_z|vac\rangle$ and $|exc\rangle$.
%Each state expresses the state of the system as a whole,
Since the photon states are the direct consequence of the local Hamiltonian 
with the zero-energy constraint caused by gauge invariance,
we can consider that the two states of the matter 
are actually represented by two different configurations of light field
(annihilation and creation of a photon).
$|\Phi_a \rangle$ is rather asymmetrical with respect to $A_R$ and $A_L$ (or the origin $x=0$)
meanwhile $|\Phi_b \rangle$ is symmetrical as $b_R = b_L$.
An interesting asymmetrical case would be multilayer graphene with $N \simeq 2/\pi \alpha$
giving that $|\Phi_a \rangle$ and $|\Phi_b \rangle$ are decoupled by $H_{local}$ because $t=1/2$ and $r=-1/2$.
Then, it is reasonably understood that 
the detection of a left-going or right-going mode does not tell us 
which state of $|\Phi_a \rangle$ and $|\Phi_b \rangle$ is realized.


\section*{Acknowledgments}

The author thank T. Matsui (Anritsu Corporation), M. Kamada (Anritsu Corporation),
and K. Hitachi for helpful discussions.

%\appendix


%\bibliographystyle{apsrev4-1}
%\bibliographystyle{apsrev}
% \bibliography{/Users/Sasaki/tex/bib/library}
%\bibliography{/Users/sasakikenichi/bib/sasaki,/Users/sasakikenichi/bib/library.bib}


%merlin.mbs apsrev4-1.bst 2010-07-25 4.21a (PWD, AO, DPC) hacked
%Control: key (0)
%Control: author (72) initials jnrlst
%Control: editor formatted (1) identically to author
%Control: production of article title (-1) disabled
%Control: page (0) single
%Control: year (1) truncated
%Control: production of eprint (0) enabled
\begin{thebibliography}{24}%
\makeatletter
\providecommand \@ifxundefined [1]{%
 \@ifx{#1\undefined}
}%
\providecommand \@ifnum [1]{%
 \ifnum #1\expandafter \@firstoftwo
 \else \expandafter \@secondoftwo
 \fi
}%
\providecommand \@ifx [1]{%
 \ifx #1\expandafter \@firstoftwo
 \else \expandafter \@secondoftwo
 \fi
}%
\providecommand \natexlab [1]{#1}%
\providecommand \enquote  [1]{``#1''}%
\providecommand \bibnamefont  [1]{#1}%
\providecommand \bibfnamefont [1]{#1}%
\providecommand \citenamefont [1]{#1}%
\providecommand \href@noop [0]{\@secondoftwo}%
\providecommand \href [0]{\begingroup \@sanitize@url \@href}%
\providecommand \@href[1]{\@@startlink{#1}\@@href}%
\providecommand \@@href[1]{\endgroup#1\@@endlink}%
\providecommand \@sanitize@url [0]{\catcode `\\12\catcode `\$12\catcode
  `\&12\catcode `\#12\catcode `\^12\catcode `\_12\catcode `\%12\relax}%
\providecommand \@@startlink[1]{}%
\providecommand \@@endlink[0]{}%
\providecommand \url  [0]{\begingroup\@sanitize@url \@url }%
\providecommand \@url [1]{\endgroup\@href {#1}{\urlprefix }}%
\providecommand \urlprefix  [0]{URL }%
\providecommand \Eprint [0]{\href }%
\providecommand \doibase [0]{http://dx.doi.org/}%
\providecommand \selectlanguage [0]{\@gobble}%
\providecommand \bibinfo  [0]{\@secondoftwo}%
\providecommand \bibfield  [0]{\@secondoftwo}%
\providecommand \translation [1]{[#1]}%
\providecommand \BibitemOpen [0]{}%
\providecommand \bibitemStop [0]{}%
\providecommand \bibitemNoStop [0]{.\EOS\space}%
\providecommand \EOS [0]{\spacefactor3000\relax}%
\providecommand \BibitemShut  [1]{\csname bibitem#1\endcsname}%
\let\auto@bib@innerbib\@empty
%</preamble>
\bibitem [{\citenamefont {Feynman}\ \emph {et~al.}(1989)\citenamefont
  {Feynman}, \citenamefont {Leighton},\ and\ \citenamefont
  {Sands}}]{Feynman1989}%
  \BibitemOpen
  \bibfield  {author} {\bibinfo {author} {\bibfnamefont {R.~P.}\ \bibnamefont
  {Feynman}}, \bibinfo {author} {\bibfnamefont {R.~B.}\ \bibnamefont
  {Leighton}}, \ and\ \bibinfo {author} {\bibfnamefont {M.}~\bibnamefont
  {Sands}},\ }\href@noop {} {\emph {\bibinfo {title} {The Feynman Lectures on
  Physics}}}\ (\bibinfo  {publisher} {Addison Wesley},\ \bibinfo {year}
  {1989})\BibitemShut {NoStop}%
\bibitem [{\citenamefont {Sakurai}\ and\ \citenamefont
  {Napolitano}(2017)}]{Sakurai2017}%
  \BibitemOpen
  \bibfield  {author} {\bibinfo {author} {\bibfnamefont {J.~J.}\ \bibnamefont
  {Sakurai}}\ and\ \bibinfo {author} {\bibfnamefont {J.}~\bibnamefont
  {Napolitano}},\ }\href {\doibase 10.1017/9781108499996} {\emph {\bibinfo
  {title} {{Modern Quantum Mechanics}}}}\ (\bibinfo  {publisher} {Cambridge
  University Press},\ \bibinfo {year} {2017})\BibitemShut {NoStop}%
\bibitem [{\citenamefont {Wootters}\ and\ \citenamefont
  {Zurek}(1982)}]{Wootters1982}%
  \BibitemOpen
  \bibfield  {author} {\bibinfo {author} {\bibfnamefont {W.~K.}\ \bibnamefont
  {Wootters}}\ and\ \bibinfo {author} {\bibfnamefont {W.~H.}\ \bibnamefont
  {Zurek}},\ }\href {\doibase 10.1038/299802a0} {\bibfield  {journal} {\bibinfo
   {journal} {Nature}\ }\textbf {\bibinfo {volume} {299}},\ \bibinfo {pages}
  {802} (\bibinfo {year} {1982})}\BibitemShut {NoStop}%
\bibitem [{\citenamefont {Kocher}\ and\ \citenamefont
  {Commins}(1967)}]{Kocher1967}%
  \BibitemOpen
  \bibfield  {author} {\bibinfo {author} {\bibfnamefont {C.~A.}\ \bibnamefont
  {Kocher}}\ and\ \bibinfo {author} {\bibfnamefont {E.~D.}\ \bibnamefont
  {Commins}},\ }\href {\doibase 10.1103/PhysRevLett.18.575} {\bibfield
  {journal} {\bibinfo  {journal} {Physical Review Letters}\ }\textbf {\bibinfo
  {volume} {18}},\ \bibinfo {pages} {575} (\bibinfo {year} {1967})}\BibitemShut
  {NoStop}%
\bibitem [{\citenamefont {Aspect}\ \emph {et~al.}(1982)\citenamefont {Aspect},
  \citenamefont {Grangier},\ and\ \citenamefont {Roger}}]{Aspect1982}%
  \BibitemOpen
  \bibfield  {author} {\bibinfo {author} {\bibfnamefont {A.}~\bibnamefont
  {Aspect}}, \bibinfo {author} {\bibfnamefont {P.}~\bibnamefont {Grangier}}, \
  and\ \bibinfo {author} {\bibfnamefont {G.}~\bibnamefont {Roger}},\ }\href
  {\doibase 10.1103/PhysRevLett.49.91} {\bibfield  {journal} {\bibinfo
  {journal} {Physical Review Letters}\ }\textbf {\bibinfo {volume} {49}},\
  \bibinfo {pages} {91} (\bibinfo {year} {1982})}\BibitemShut {NoStop}%
\bibitem [{\citenamefont {Ou}\ and\ \citenamefont {Mandel}(1988)}]{Ou1988}%
  \BibitemOpen
  \bibfield  {author} {\bibinfo {author} {\bibfnamefont {Z.~Y.}\ \bibnamefont
  {Ou}}\ and\ \bibinfo {author} {\bibfnamefont {L.}~\bibnamefont {Mandel}},\
  }\href {\doibase 10.1103/PhysRevLett.61.50} {\bibfield  {journal} {\bibinfo
  {journal} {Physical Review Letters}\ }\textbf {\bibinfo {volume} {61}},\
  \bibinfo {pages} {50} (\bibinfo {year} {1988})}\BibitemShut {NoStop}%
\bibitem [{\citenamefont {Takesue}\ and\ \citenamefont
  {Inoue}(2004)}]{Takesue2004}%
  \BibitemOpen
  \bibfield  {author} {\bibinfo {author} {\bibfnamefont {H.}~\bibnamefont
  {Takesue}}\ and\ \bibinfo {author} {\bibfnamefont {K.}~\bibnamefont
  {Inoue}},\ }\href {\doibase 10.1103/PHYSREVA.70.031802/FIGURES/3/MEDIUM}
  {\bibfield  {journal} {\bibinfo  {journal} {Physical Review A - Atomic,
  Molecular, and Optical Physics}\ }\textbf {\bibinfo {volume} {70}},\ \bibinfo
  {pages} {031802} (\bibinfo {year} {2004})},\ \Eprint
  {http://arxiv.org/abs/0408032} {arXiv:0408032 [quant-ph]} \BibitemShut
  {NoStop}%
\bibitem [{\citenamefont {Wang}\ \emph {et~al.}(2013)\citenamefont {Wang},
  \citenamefont {Marandi}, \citenamefont {Wen}, \citenamefont {Byer},\ and\
  \citenamefont {Yamamoto}}]{Wang2013}%
  \BibitemOpen
  \bibfield  {author} {\bibinfo {author} {\bibfnamefont {Z.}~\bibnamefont
  {Wang}}, \bibinfo {author} {\bibfnamefont {A.}~\bibnamefont {Marandi}},
  \bibinfo {author} {\bibfnamefont {K.}~\bibnamefont {Wen}}, \bibinfo {author}
  {\bibfnamefont {R.~L.}\ \bibnamefont {Byer}}, \ and\ \bibinfo {author}
  {\bibfnamefont {Y.}~\bibnamefont {Yamamoto}},\ }\href {\doibase
  10.1103/PhysRevA.88.063853} {\bibfield  {journal} {\bibinfo  {journal}
  {Physical Review A}\ }\textbf {\bibinfo {volume} {88}},\ \bibinfo {pages}
  {063853} (\bibinfo {year} {2013})}\BibitemShut {NoStop}%
\bibitem [{\citenamefont {Honjo}\ \emph {et~al.}(2021)\citenamefont {Honjo},
  \citenamefont {Sonobe}, \citenamefont {Inaba}, \citenamefont {Inagaki},
  \citenamefont {Ikuta}, \citenamefont {Yamada}, \citenamefont {Kazama},
  \citenamefont {Enbutsu}, \citenamefont {Umeki}, \citenamefont {Kasahara},
  \citenamefont {Kawarabayashi},\ and\ \citenamefont {Takesue}}]{Honjo2021}%
  \BibitemOpen
  \bibfield  {author} {\bibinfo {author} {\bibfnamefont {T.}~\bibnamefont
  {Honjo}}, \bibinfo {author} {\bibfnamefont {T.}~\bibnamefont {Sonobe}},
  \bibinfo {author} {\bibfnamefont {K.}~\bibnamefont {Inaba}}, \bibinfo
  {author} {\bibfnamefont {T.}~\bibnamefont {Inagaki}}, \bibinfo {author}
  {\bibfnamefont {T.}~\bibnamefont {Ikuta}}, \bibinfo {author} {\bibfnamefont
  {Y.}~\bibnamefont {Yamada}}, \bibinfo {author} {\bibfnamefont
  {T.}~\bibnamefont {Kazama}}, \bibinfo {author} {\bibfnamefont
  {K.}~\bibnamefont {Enbutsu}}, \bibinfo {author} {\bibfnamefont
  {T.}~\bibnamefont {Umeki}}, \bibinfo {author} {\bibfnamefont
  {R.}~\bibnamefont {Kasahara}}, \bibinfo {author} {\bibfnamefont {K.~I.}\
  \bibnamefont {Kawarabayashi}}, \ and\ \bibinfo {author} {\bibfnamefont
  {H.}~\bibnamefont {Takesue}},\ }\href {\doibase
  10.1126/SCIADV.ABH0952/SUPPL_FILE/SCIADV.ABH0952_SM.PDF} {\bibfield
  {journal} {\bibinfo  {journal} {Science Advances}\ }\textbf {\bibinfo
  {volume} {7}},\ \bibinfo {pages} {952} (\bibinfo {year} {2021})}\BibitemShut
  {NoStop}%
\bibitem [{\citenamefont {Bohm}\ and\ \citenamefont {Pines}(1951)}]{Bohm1951}%
  \BibitemOpen
  \bibfield  {author} {\bibinfo {author} {\bibfnamefont {D.}~\bibnamefont
  {Bohm}}\ and\ \bibinfo {author} {\bibfnamefont {D.}~\bibnamefont {Pines}},\
  }\href {\doibase 10.1103/PhysRev.82.625} {\bibfield  {journal} {\bibinfo
  {journal} {Physical Review}\ }\textbf {\bibinfo {volume} {82}},\ \bibinfo
  {pages} {625} (\bibinfo {year} {1951})}\BibitemShut {NoStop}%
\bibitem [{\citenamefont {Pines}\ and\ \citenamefont {Bohm}(1952)}]{Pines1952}%
  \BibitemOpen
  \bibfield  {author} {\bibinfo {author} {\bibfnamefont {D.}~\bibnamefont
  {Pines}}\ and\ \bibinfo {author} {\bibfnamefont {D.}~\bibnamefont {Bohm}},\
  }\href {\doibase 10.1103/PhysRev.85.338} {\bibfield  {journal} {\bibinfo
  {journal} {Physical Review}\ }\textbf {\bibinfo {volume} {85}},\ \bibinfo
  {pages} {338} (\bibinfo {year} {1952})}\BibitemShut {NoStop}%
\bibitem [{\citenamefont {Sasaki}\ \emph {et~al.}(2006)\citenamefont {Sasaki},
  \citenamefont {Murakami},\ and\ \citenamefont {Saito}}]{Sasaki2006c}%
  \BibitemOpen
  \bibfield  {author} {\bibinfo {author} {\bibfnamefont {K.-i.}\ \bibnamefont
  {Sasaki}}, \bibinfo {author} {\bibfnamefont {S.}~\bibnamefont {Murakami}}, \
  and\ \bibinfo {author} {\bibfnamefont {R.}~\bibnamefont {Saito}},\ }\href
  {\doibase 10.1143/JPSJ.75.074713} {\bibfield  {journal} {\bibinfo  {journal}
  {Journal of the Physical Society of Japan}\ }\textbf {\bibinfo {volume}
  {75}},\ \bibinfo {pages} {074713} (\bibinfo {year} {2006})}\BibitemShut
  {NoStop}%
\bibitem [{\citenamefont {Sasaki}\ and\ \citenamefont
  {Saito}(2008)}]{Sasaki2008e}%
  \BibitemOpen
  \bibfield  {author} {\bibinfo {author} {\bibfnamefont {K.-i.}\ \bibnamefont
  {Sasaki}}\ and\ \bibinfo {author} {\bibfnamefont {R.}~\bibnamefont {Saito}},\
  }\href {\doibase 10.1143/PTPS.176.253} {\bibfield  {journal} {\bibinfo
  {journal} {Progress of Theoretical Physics Supplement}\ }\textbf {\bibinfo
  {volume} {176}},\ \bibinfo {pages} {253} (\bibinfo {year}
  {2008})}\BibitemShut {NoStop}%
\bibitem [{\citenamefont {Sakurai}(1967)}]{sakurai67}%
  \BibitemOpen
  \bibfield  {author} {\bibinfo {author} {\bibfnamefont {J.~J.}\ \bibnamefont
  {Sakurai}},\ }\href@noop {} {\emph {\bibinfo {title} {{Advanced Quantum
  Mechanics}}}}\ (\bibinfo  {publisher} {Addison-Wesley},\ \bibinfo {address}
  {Canada},\ \bibinfo {year} {1967})\BibitemShut {NoStop}%
\bibitem [{\citenamefont {Sasaki}\ and\ \citenamefont
  {Hitachi}(2020)}]{Sasaki2020a}%
  \BibitemOpen
  \bibfield  {author} {\bibinfo {author} {\bibfnamefont {K.}~\bibnamefont
  {Sasaki}}\ and\ \bibinfo {author} {\bibfnamefont {K.}~\bibnamefont
  {Hitachi}},\ }\href {\doibase 10.1038/s42005-020-0354-y} {\bibfield
  {journal} {\bibinfo  {journal} {Communications Physics}\ }\textbf {\bibinfo
  {volume} {3}},\ \bibinfo {pages} {90} (\bibinfo {year} {2020})}\BibitemShut
  {NoStop}%
\bibitem [{\citenamefont {Sasaki}(2020)}]{Sasaki2020b}%
  \BibitemOpen
  \bibfield  {author} {\bibinfo {author} {\bibfnamefont {K.-i.}\ \bibnamefont
  {Sasaki}},\ }\href {\doibase 10.7566/JPSJ.89.094706} {\bibfield  {journal}
  {\bibinfo  {journal} {Journal of the Physical Society of Japan}\ }\textbf
  {\bibinfo {volume} {89}},\ \bibinfo {pages} {094706} (\bibinfo {year}
  {2020})}\BibitemShut {NoStop}%
\bibitem [{\citenamefont {Shen}\ and\ \citenamefont {Fan}(2005)}]{Shen2005}%
  \BibitemOpen
  \bibfield  {author} {\bibinfo {author} {\bibfnamefont {J.~T.}\ \bibnamefont
  {Shen}}\ and\ \bibinfo {author} {\bibfnamefont {S.}~\bibnamefont {Fan}},\
  }\href {\doibase 10.1364/OL.30.002001} {\bibfield  {journal} {\bibinfo
  {journal} {Optics Letters}\ }\textbf {\bibinfo {volume} {30}},\ \bibinfo
  {pages} {2001} (\bibinfo {year} {2005})}\BibitemShut {NoStop}%
\bibitem [{\citenamefont {Shon}\ and\ \citenamefont {Ando}(1998)}]{Shon1998}%
  \BibitemOpen
  \bibfield  {author} {\bibinfo {author} {\bibfnamefont {N.~H.}\ \bibnamefont
  {Shon}}\ and\ \bibinfo {author} {\bibfnamefont {T.}~\bibnamefont {Ando}},\
  }\href {\doibase 10.1143/JPSJ.67.2421} {\bibfield  {journal} {\bibinfo
  {journal} {Journal of the Physics Society Japan}\ }\textbf {\bibinfo {volume}
  {67}},\ \bibinfo {pages} {2421} (\bibinfo {year} {1998})}\BibitemShut
  {NoStop}%
\bibitem [{\citenamefont {Nair}\ \emph {et~al.}(2008)\citenamefont {Nair},
  \citenamefont {Blake}, \citenamefont {Grigorenko}, \citenamefont {Novoselov},
  \citenamefont {Booth}, \citenamefont {Stauber}, \citenamefont {Peres},\ and\
  \citenamefont {Geim}}]{Nair2008}%
  \BibitemOpen
  \bibfield  {author} {\bibinfo {author} {\bibfnamefont {R.~R.}\ \bibnamefont
  {Nair}}, \bibinfo {author} {\bibfnamefont {P.}~\bibnamefont {Blake}},
  \bibinfo {author} {\bibfnamefont {A.~N.}\ \bibnamefont {Grigorenko}},
  \bibinfo {author} {\bibfnamefont {K.~S.}\ \bibnamefont {Novoselov}}, \bibinfo
  {author} {\bibfnamefont {T.~J.}\ \bibnamefont {Booth}}, \bibinfo {author}
  {\bibfnamefont {T.}~\bibnamefont {Stauber}}, \bibinfo {author} {\bibfnamefont
  {N.~M.~R.}\ \bibnamefont {Peres}}, \ and\ \bibinfo {author} {\bibfnamefont
  {A.~K.}\ \bibnamefont {Geim}},\ }\href {\doibase 10.1126/science.1156965}
  {\bibfield  {journal} {\bibinfo  {journal} {Science}\ }\textbf {\bibinfo
  {volume} {320}},\ \bibinfo {pages} {1308} (\bibinfo {year}
  {2008})}\BibitemShut {NoStop}%
\bibitem [{\citenamefont {Ando}\ \emph {et~al.}(2002)\citenamefont {Ando},
  \citenamefont {Zheng},\ and\ \citenamefont {Suzuura}}]{ando02-dc}%
  \BibitemOpen
  \bibfield  {author} {\bibinfo {author} {\bibfnamefont {T.}~\bibnamefont
  {Ando}}, \bibinfo {author} {\bibfnamefont {Y.}~\bibnamefont {Zheng}}, \ and\
  \bibinfo {author} {\bibfnamefont {H.}~\bibnamefont {Suzuura}},\ }\href
  {\doibase 10.1143/JPSJ.71.1318} {\bibfield  {journal} {\bibinfo  {journal}
  {Journal of the Physical Society of Japan}\ }\textbf {\bibinfo {volume}
  {71}},\ \bibinfo {pages} {1318} (\bibinfo {year} {2002})}\BibitemShut
  {NoStop}%
\bibitem [{\citenamefont {Durrant}(1998)}]{Durrant1998}%
  \BibitemOpen
  \bibfield  {author} {\bibinfo {author} {\bibfnamefont {A.~V.}\ \bibnamefont
  {Durrant}},\ }\href {\doibase 10.1119/1.10322} {\bibfield  {journal}
  {\bibinfo  {journal} {American Journal of Physics}\ }\textbf {\bibinfo
  {volume} {44}},\ \bibinfo {pages} {630} (\bibinfo {year} {1998})}\BibitemShut
  {NoStop}%
\bibitem [{\citenamefont {Cray}\ \emph {et~al.}(1998)\citenamefont {Cray},
  \citenamefont {Shih},\ and\ \citenamefont {Milonni}}]{Cray1998}%
  \BibitemOpen
  \bibfield  {author} {\bibinfo {author} {\bibfnamefont {M.}~\bibnamefont
  {Cray}}, \bibinfo {author} {\bibfnamefont {M.}~\bibnamefont {Shih}}, \ and\
  \bibinfo {author} {\bibfnamefont {P.~W.}\ \bibnamefont {Milonni}},\ }\href
  {\doibase 10.1119/1.12956} {\bibfield  {journal} {\bibinfo  {journal}
  {American Journal of Physics}\ }\textbf {\bibinfo {volume} {50}},\ \bibinfo
  {pages} {1016} (\bibinfo {year} {1998})}\BibitemShut {NoStop}%
\bibitem [{\citenamefont {Sasaki}\ \emph {et~al.}(2022)\citenamefont {Sasaki},
  \citenamefont {Hitachi}, \citenamefont {Kamada}, \citenamefont {Yokosawa},
  \citenamefont {Ochi},\ and\ \citenamefont {Matsui}}]{Sasaki2022a}%
  \BibitemOpen
  \bibfield  {author} {\bibinfo {author} {\bibfnamefont {K.-i.}\ \bibnamefont
  {Sasaki}}, \bibinfo {author} {\bibfnamefont {K.}~\bibnamefont {Hitachi}},
  \bibinfo {author} {\bibfnamefont {M.}~\bibnamefont {Kamada}}, \bibinfo
  {author} {\bibfnamefont {T.}~\bibnamefont {Yokosawa}}, \bibinfo {author}
  {\bibfnamefont {T.}~\bibnamefont {Ochi}}, \ and\ \bibinfo {author}
  {\bibfnamefont {T.}~\bibnamefont {Matsui}},\ }\href {\doibase
  10.48550/arxiv.2208.01311} {\  (\bibinfo {year} {2022}),\
  10.48550/arxiv.2208.01311},\ \Eprint {http://arxiv.org/abs/2208.01311}
  {arXiv:2208.01311} \BibitemShut {NoStop}%
\bibitem [{\citenamefont {Sasaki}\ \emph {et~al.}(2011)\citenamefont {Sasaki},
  \citenamefont {Kato}, \citenamefont {Tokura}, \citenamefont {Oguri},\ and\
  \citenamefont {Sogawa}}]{Sasaki2011}%
  \BibitemOpen
  \bibfield  {author} {\bibinfo {author} {\bibfnamefont {K.-i.}\ \bibnamefont
  {Sasaki}}, \bibinfo {author} {\bibfnamefont {K.}~\bibnamefont {Kato}},
  \bibinfo {author} {\bibfnamefont {Y.}~\bibnamefont {Tokura}}, \bibinfo
  {author} {\bibfnamefont {K.}~\bibnamefont {Oguri}}, \ and\ \bibinfo {author}
  {\bibfnamefont {T.}~\bibnamefont {Sogawa}},\ }\href {\doibase
  10.1103/PhysRevB.84.085458} {\bibfield  {journal} {\bibinfo  {journal}
  {Physical Review B}\ }\textbf {\bibinfo {volume} {84}},\ \bibinfo {pages}
  {085458} (\bibinfo {year} {2011})}\BibitemShut {NoStop}%
\end{thebibliography}%




\end{document}
