% Ego4d的实验结果
\begin{table}[!tbp]
  \centering\small
  \caption{%
    Performance on the val set of Ego4d dataset, under Ego4d-NLQ and Ego4d-Video-NLQ settings. Noted that 2D-TAN and CONE are sliding window-based methods while VSLNet is downsampling-based method.
  }
  \vspace{-3pt}
  % \renewcommand{\arraystretch}{0.9}
  \setlength{\tabcolsep}{9pt}{
    \begin{tabular}{lcccc}
    \toprule
    \multirow{2}{*}{Model} & \multicolumn{2}{c}{\textbf{IoU = 0.3}} & \multicolumn{2}{c}{\textbf{IoU = 0.5}} \\
          & R@1    & R@5    & R@1    & R@5 \\
    \midrule
    \multicolumn{5}{c}{Ego4d-NLQ (\textit{avg. 8.25 min / video})} \\
    \midrule
    2D-TAN~\cite{zhang2020learning}  & 5.04  & 12.89 & 2.02  & 5.88 \\
    VSLNet~\cite{zhang2020span}  & 5.45  & 10.74 & 3.12  & 6.63 \\
    CONE\footnotemark ~\cite{hou2022cone}   & 10.40  & 22.74 & 5.03  & 11.87 \\
    % 2D-TAN$^{*}$    & 3.34  & 8.41 & 1.73  & 4.50 \\
    % VSLNet$^{*}$   & 4.83  & 9.91 & 2.81  & 6.12 \\ 
    % \midrule
    \method(Ours) & 8.00      & 22.40    & 3.76     & 11.09  \\
    \midrule
    \multicolumn{5}{c}{Ego4d-Video-NLQ (\textit{avg. 25.7 min / video})} \\
    \midrule
    2D-TAN~\cite{zhang2020learning}   & 1.70  & 4.59 & 0.82  & 2.77 \\
    VSLNet~\cite{zhang2020span}   & 1.57  & 4.44 & 0.75  & 2.22 \\
    % \midrule
    \textbf{\method(Ours)} &  \textbf{3.90}      &  \textbf{10.71}     & \textbf{1.80}      & \textbf{5.09}     \\
    \bottomrule
    \end{tabular}
  }
  \vspace{-5mm}
  \label{tab:ego4d}
\end{table}
\footnotetext{As for CONE, its code has not been released,  so for fair comparison we only report its result on Ego4d-NLQ provided by its original paper.}