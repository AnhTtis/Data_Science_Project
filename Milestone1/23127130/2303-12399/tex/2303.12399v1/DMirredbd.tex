\documentclass[20pt, oneside]{article}   	% use "amsart" instead of "article" for AMSLaTeX format
\usepackage{geometry}                		% See geometry.pdf to learn the layout options. There are 
\usepackage{amsmath}
\usepackage{amsthm}	
\usepackage{amssymb}
\usepackage{tikz}
\usepackage[bottom]{footmisc}
\usepackage[affil-it]{authblk}
\usepackage[toc,page]{appendix}
\newtheorem{thm}{Theorem}[section]
\newtheorem{defi}[thm]{Definition}
\newtheorem{rem}[thm]{Remark}
\newtheorem{prop}[thm]{Proposition}
\newtheorem{note}[thm]{Note}
\newtheorem{lem}[thm]{Lemma}
\newtheorem{cor}[thm]{Corollary}
\newtheorem*{claim*}{Claim}
\newtheorem{lemma}{Lemma}
\newtheorem*{thm*}{Theorem}
\newtheorem*{corr}{Corollary}
\newtheorem*{theorem}{Theorem}
\newtheorem*{thmm}{Main Theorem}
\newtheorem*{ques}{Question}
\newtheorem{ex}[thm]{Example}
\newtheorem*{man}{Main Result}

\newcommand{\F}{\mathbb{F}}
\newcommand{\fl}{\mathfrak{l}}
\newcommand{\fq}{\mathfrak{q}}
\newcommand{\fa}{\mathfrak{a}}
\newcommand{\fp}{\mathfrak{p}}
\newcommand{\Gal}{\mathrm{Gal}}
\newcommand{\Aut}{\mathrm{Aut}}
\newcommand{\sep}{\mathrm{sep}}
\newcommand{\alg}{\mathrm{alg}}
\newcommand{\GL}{\mathrm{GL}}



\newcommand{\hookuparrow}{\mathrel{\rotatebox[origin=c]{90}{$\hookrightarrow$}}}
\newcommand{\hookdownarrow}{\mathrel{\rotatebox[origin=c]{-90}{$\hookrightarrow$}}}
\newcommand{\xdownarrow}[1]{%
  {\left\downarrow\vbox to #1{}\right.\kern-\nulldelimiterspace}
}










\title{ Masser-W\"ustholz bound for reducibility of Galois representations for Drinfeld modules of arbitrary rank}
					% Activate to display a given date or no date
\author{Chien-Hua Chen \thanks{Electronic address: \texttt{danny30814@ncts.ntu.edu.tw}; ORCID: \texttt{0000-0003-3267-5603} ; Corresponding author}}
\affil{Mathematics Division,\\ National Center for Theoretical Sciences,\\ Taipei, Taiwan}

\begin{document}

%\date{}		
\maketitle
\abstract
In this paper, we give an explicit bound on the irreducibility of mod-$\fl$ Galois representation for Drinfeld modules of arbitrary rank without complex multiplication. This is a function field analogue of Masser-Wustholz bound on irreducibility of mod-$\ell$ Galois representation for elliptic curves over number field.

\section {Introduction}
In 1993, Masser and W\"ustholz \cite{MW93-2} proved a famous result on existence of isogeny, with degree bounded by an explicit formula,  between two isogenous Elliptic curves. Later on, they \cite{MW93} applied such an isogeny estimation to give an explicit bound on the irreducibility of mod-$\ell$ Galois representation associated to elliptic curves over number field without complex multiplication (CM).  This bound is then used to deduce a bound on the surjectivity of mod-$\ell$ Galois representation for elliptic curves over number field without CM. 

As a function field analogue of the theory in elliptic curves, David and Dennis \cite{DD99} gave an isogeny estimation for Anderson $t$-modules. In particular, they deduced an isogeny estimation for Drinfeld $\F_q[T]$-modules over a global function field, see Theorem \ref{dd} for more details. Thus it is natural to ask whether one can apply the same strategy as Messer-W\"ustholz to deduce a bound on irreducibility of mod-$\fl$ Galois representation for rank-$r$ Drinfeld modules without CM. However, the Messer-W\"ustholz strategy can not be applied directly to the context of Drinfeld modules. The main resean is when one computes the degree of isogeny between Drinfeld modules, the degree is always a power of $q$, which is not a prime number. Thus the computational trick in Lemma 3.1 of \cite{MW93} does not work for Drinfeld modules.

However, the idea from Messer-W\"ustholz inspired us to produce a similar method. Combining with the estimation on heights between isogenous Drinfeld modules given by Breuer, Pazuki, and Razafinjatovo (see Theorem \ref{bp}), we can deduce our main result for an explicit bound on irreducibility of mod-$\fl$ Galois representation for Drinfeld modules of arbitrary rank and without CM:





\begin{thm}\label{main}
Let $q=p^e$ be a prime power, $A:=\F_q[T]$, and $K$ be a finite extension of $F:=\F_q(T)$ of degree $d$. Let $\phi$ be a rank-$r$ Drinfeld $A$-module over $K$ of generic characteristic and assume that ${\rm End}_{\bar{K}}(\phi)=A$. Let $\fl=(\ell)$ be a prime ideal of $A$, consider the mod-$\fl$ Galois representation
$$\bar{\rho}_{\phi,\fl}:\Gal(\bar{K}/K)\rightarrow {\rm Aut}(\phi[\fl])\cong \GL_r(A/\fl).$$
If $\bar{\rho}_{\phi,\fl}$ is reducible, then either
\begin{equation}
\deg_T\ell-10(d+1)^7\log \deg_T\ell   \leqslant \log c_2+ 10(d+1)^7\left\{ \log d +r+\log[h_G(\phi)+1+\frac{q}{q-1}-\frac{q^r}{q^r-1}]               \right\}
\end{equation}

or
\begin{equation}
\deg_T\ell\leqslant \log c_2+10(d+1)^7[\log d\cdot h(\phi)]
\end{equation}

\end{thm}

As a corollary of Theorem \ref{main}, we deduce a lowerbound for $\deg_T\ell$ such that the mod-$\fl$ Galois representation $\bar{\rho}_{\phi,\fl}$ is irreducible. See corollary \ref{cormain} for details.


For the special case ``rank-$2$ Drinfeld modules over $\F_q(T)$'', there is actually a finer estimation on irreducibility of mod-$\fl$ Galois representation made by Chen and Lee \cite{CL19}. However, their strategy uses the fact that a power of $1$-dimensional group representation is again a group representation, see proof of Proposition 7.1 in \cite{CL19}. In the rank-2 case, the reducibility of mod-$\fl$ Galois representation always contributes a $1$-dimensional subrepresentation. But this is not true for higher rank Drinfeld modules.

On the other hand, Chen and Lee \cite{CL19} gave an explicit bound on surjectivity of mod-$\fl$ Galois representations for rank-$2$ Drinfeld modules over $\F_q(T)$ without CM. Such an explicit bound is still unknown for higher rank Drinfeld modules. The main difficulty is the classification of maximal subgroups (up to conjugacy classes) in $\GL_r$ over finite field is much more complicated comparing to the $\GL_2$ case, where one only need to take care of the Borel and Cartan cases.


\section{Preliminaries}

Let $A=\F_q[T]$ be the polynomial ring over finite field with $q=p^e$ an odd prime power, $F=\F_q(T)$ be the fractional field of $A$, and  $K$ be a finite extension over $F$. Set $K\{\tau\}$ to be the twisted polynomial ring with the multiplication rule $\tau\alpha=\alpha^q\tau \text{ for any } \alpha\in K$. Throughout this paper, ``$\log$'' refers to the logarithm with base $q$.

\subsection{Drinfeld modules}
We view $K$ as an $A$-field, which is a field equipped with a homomorphism $\gamma: A\rightarrow K$. The {\bf{$A$-characteristic}} of $K$ is defined to be the kernel of $\gamma$.

\begin{note}
Throughout this paper, we take $\gamma:A\rightarrow K$ to be the natural embedding from $A$ to $K$. i.e. the $A$-characteristic of $K$ is always equal to zero.
\end{note}


\begin{defi}
A Drinfeld $A$-module of rank $r$ over $K$ of generic characteristic is a ring homomorphism

$$\phi: A\rightarrow K\{\tau\}={\rm End}_{\F_q}(\mathbb{G}_{a,K})$$
such that
\begin{enumerate}
\item $\phi(a):=\phi_a$ satisfies ${\rm deg}_{\tau}\phi_a=r\cdot {\rm deg}_Ta$
\item Denote $\partial: F\{\tau\}\rightarrow F$ by $\partial(\sum a_i\tau^i)=a_0$, then $\phi$ satsfies $\gamma=\partial\circ\phi$.
\end{enumerate}
\end{defi}

From the definition of Drinfeld $A$-module, we can characterize a Drinfeld module $\phi$ by writing down $$\phi_T=T+g_1\tau+\cdots+g_{r-1}\tau^{r-1}+g_r\tau^r, \textrm{ where } g_i\in K \textrm{ and }g_r\in K^*.$$


\begin{prop}
There is an isomorphism between the twisted polynomial ring $K\{\tau\}$ and the ring of $q$-polynomials $(K<x>,+,\circ)$ where $K<x>:=\left\{\sum_{i=0}^{n}c_ix^{q^i}\mid c_i\in K\right\}$ and the mutiplication of $K<x>$ is defined to be composition of $q$-polynomials.
\end{prop}

\begin{proof}
Consider the map sending $\sum_{i=0}^nc_i\tau^i$ to $\sum_{i=0}^nc_ix^{q^i}$. This defines an isomorphism between $K\{\tau\}$ and $K<x>$.

\end{proof}

Fix a Drinfeld module $\phi$ over $K$. From the above proposition, the image $\phi_a$ of Drinfeld module $\phi$ at $a\in A$ corresponds to a $q$-polynomial $\phi_a(x)$. Hence for an ideal $\mathfrak{a}=<a>$ of $A$, we may define the $\mathfrak{a}$-torsion of the Drinfeld module $\phi$ over $K$.


\begin{defi}
The $\mathfrak{a}$-torsion of a Drinfeld module $\phi$ over $K$ is defined to be
$$\phi[\mathfrak{a}]:=\left\{\textrm{ zeros of }\phi_a(x) \textrm{ in } \bar{K} \right\}\subset \bar{K}.$$
\end{defi}


Now we define the $A$-module structure on $\bar{K}$. For any elements $b\in A$ and $\alpha\in \bar{K}$. We define the $A$-action of $b$ on $\alpha$ via
$$b\cdot\alpha:=\phi_b(\alpha).$$
This gives $\bar{K}$ an $A$-module structure. And the $A$-module structure inherits to $\phi[\mathfrak{a}]$. As our Drinfeld module $\phi$ over $K$ has generic characteristic, we have the following proposition

\begin{prop}
$\phi[\mathfrak{a}] $ is a free $A/\mathfrak{a}$-module of rank $r$.
\end{prop}
\begin{proof}
See \cite{G96}, Proposition 4.5.3.
\end{proof}

Let $\fl$ be a prime ideal of $A$, then the $\fl$-torsion $\phi[\fl]$ of the Drinfeld module $\phi$ is an $r$-dimensional $A/\fl$-vector space. Applying the action of absolute Galois group $\Gal(\bar{K}/K)$ on $\phi[\fl]$, we obtain the so-called mod-$\fl$ Galois representation
$$\bar{\rho}_{\phi,\fl}:\Gal(\bar{K}/K)\rightarrow {\rm Aut}(\phi[\fl])\cong \GL_r(A/\fl)$$
 for the Drinfeld module $\phi$ over $K$.


Now we define some heights of Drinfeld modules. We denote by $M_K$ the set of all places of $K$ including places above $\infty$. For each place $\nu\in M_K$, define $n_\nu:=[K_\nu:F_\nu]$ to be the degree of local field extension $K_\nu/F_\nu$. And define $|\cdot|_\nu$ to be a normalized valuation of $K_\nu$. 

\begin{defi}
Let $\phi$ be a rank-$r$ Drinfeld module over $K$ characterized by
$$\phi_T=T+g_1\tau+\cdots+g_{r-1}\tau^{r-1}+g_r\tau^r, \textrm{ where } g_i\in K \textrm{ and }g_r\in K^*.$$

\begin{enumerate}
\item The naive height of $\phi$ is defined to be 
$$h(\phi):={\rm max}\{h(g_1),\cdots,h(g_r)\},$$
where $h(g_i):=\frac{1}{[K:F]}\sum_{\nu\in M_K}n_\nu\cdot {\rm log}|g_i|_\nu$.


\item The graded height of $\phi$ is defined to be
$$h_G(\phi):=\frac{1}{[K:F]}\sum_{\nu\in M_K}n_\nu\cdot {\rm log}\ {\rm max}\{|g_i|_\nu^{1/(q^i-1)}\mid 1\leqslant i\leqslant r\}$$

\end{enumerate}
\end{defi}

\begin{cor}\label{heightineq}
One can observe from the definition of naive height and graded height that
$$h(\phi)\leqslant(q^r-1)\cdot h_G(\phi).$$

\end{cor}










\subsection{Isogenies}
\begin{defi}
Let $\phi$ and $\psi$ be two rank-$r$ Drinfeld $A$-modules over $K$. A {\bf{morphism}} $u:\phi\rightarrow \psi$ over $K$ is a twisted polynomial $u\in K\{\tau\}$ such that
$$u\phi_a=\psi_a u \text{ \rm for all } a\in A.$$ 
A non-zero morphism $u:\phi\rightarrow \psi$ is called an isogeny. A morphism $u:\phi\rightarrow \psi$ is called an {\bf{isomorphism}} if its inverse exists. 
\end{defi}

Set ${\rm Hom}_K(\phi,\psi)$ to be the group of all morphisms $u:\phi\rightarrow \psi$ over $K$. We denote ${\rm End}_K(\phi)={\rm Hom}_K(\phi,\phi)$. For any field extension $L/K$, we define
$${\rm Hom}_L(\phi,\psi)=\{u\in L\{\tau\} \mid u\phi_a=\psi_a u  \text{ \rm for all } a\in A \}.$$
For $L=\bar{K}$, we omit subscripts and write

$${\rm Hom}(\phi,\psi):={\rm Hom}_{\bar{K}}(\phi,\psi) \text{ \rm and } {\rm End}(\phi):={\rm End}_{\bar{K}}(\phi)$$




\begin{defi}
The composition of morphisms makes ${\rm End}_L(\phi)$ into a subring of $L\{\tau\}$, called the {\bf{endomorphism ring}} of $\phi$ over $L$. For any rank-$r$ Drinfeld module $\phi$ over $K$ with ${\rm End}(\phi)=A$, we say that $\phi$ does not have complex multiplication.

\end{defi}

\begin{defi}
Let $f:\phi\rightarrow \psi$ be an isogeny of Drinfeld modules over $K$ of rank $r$, we define the degree of $f$ to be
$$\deg f:=\# {\rm{ker}}(f).$$

\end{defi}

\begin{prop}\label{dual}
Let $f:\phi\rightarrow \psi$ be an isogeny of Drinfeld modules over $K$ of rank $r$. There exists a dual isogeny $\hat{f}:\psi\rightarrow \phi$ such that $$f\circ\hat{f}=\psi_a \textrm{ and } \hat{f}\circ f=\phi_a.$$
Here  $0\neq a\in A$ is an element of minimal $T$-degree such that ${\rm ker}(f)\subset \phi[a]$. 

\end{prop}
\begin{proof}
See \cite{G96} Proposition 4.7.13 and Corollary 4.7.14.

\end{proof}

The following corollary is immediate by counting cardinalities.  
\begin{cor}\label{countdeg}
As in the setting of Proposition \ref{dual}, we have 
$$q^{r\cdot\deg_T(a)}=(\deg f)\cdot(\deg \hat{f}).$$

\end{cor}


Now we can state the key tools to derive our main result: 




\begin{thm}[\cite{DD99} Theorem 1.3]\label{dd}
 Let $K$ be a finite extension over $F$ with $[K:F]:=d$. Suppose that there are two $\bar{K}$-isogenous Drinfeld modules $\phi$ and $\psi$ defined over $K$. Then there is an isogeny $f: \phi\rightarrow \psi$ such that
 $$\deg f\leqslant c_2\cdot (dh(\phi))^{10(r+1)^7}.$$

Here $c_2=c_2(r,q)$ is a effectively computable constant depends only on $d$ and $q$. 


\end{thm}



\begin{thm}[\cite{BPR21} Theorem 3.1]\label{bp}
Let $f:\phi\rightarrow \psi$ be an isogeny of rank-$r$ Drinfeld modules over $\bar{K}$ and suppose that ${\rm ker}(f)\subset \phi[N]$ for some $0\neq N\in A$. Then we have

$$|h_G(\psi)-h_G(\phi)|\leqslant \deg_T(N)+\left( \frac{q}{q-1}-\frac{q^r}{q^r-1}\right).$$





\end{thm}




\section{Proof of Theorem \ref{main}}
We are given a rank-$r$ Drinfeld module $\phi$ defined over $K$ with ${\rm End}(\phi)=A$. And suppose the image of  mod-$\fl$ Galois representation ${\rm{Im}}\bar{\rho}_{\phi,\fl}$ acting on $\phi[\fl]$ has an invariant $A/\fl$-subspace of dimension $1\leqslant k\leqslant r-1$. Denote such an invariant subspace by $H$. 


From Proposition 4.7.11 and Remark 4.7.12 of \cite{G96},  there is an isogeny $$f:\phi\rightarrow \phi/H$$ with ${\rm ker}(f)=H$. Since $\phi$ and $f$ both are defined over $K$, one can see that the Drinfeld module $\phi/H$ is a rank-$r$ Drinfeld module defined over $K$ as well. In addition, we have $$\deg f=\# H=q^{k\cdot \deg_T\fl}.$$

Take a dual isogeny $\hat{f}:\phi/H\rightarrow \phi$ of $f$. The degree of $\hat{f}$ can be computed using Corollary \ref{countdeg}. We get
$$\deg \hat{f}=q^{(r-k)\cdot\deg_T\fl}.$$


Besides, we can find two isogenies between $\phi$ and $\phi/H$ with bounded degree from Theorem \ref{dd}:
\begin{enumerate}
\item[$\bullet$]  $u: \phi\rightarrow \phi/H$ is such an isogeny defined over $\bar{K}$ with $\deg u\leqslant c_2\cdot(dh(\phi))^{10(d+1)^7}$

\item[$\bullet$]$u': \phi/H\rightarrow \phi$ is such an isogeny defined over $\bar{K}$ with $\deg u'\leqslant c_2\cdot(dh(\phi/H))^{10(d+1)^7}$\end{enumerate}

Since ${\rm End}(\phi)=A$, we have $u'\circ u=\phi_b$ for some $b\in A$. 


Now we consider the composition of isogenies
$$u'\circ f\circ \hat{f}\circ u:\phi\rightarrow \phi/H\rightarrow \phi\rightarrow \phi/H\rightarrow \phi.$$

Since ${\rm End}(\phi)=A$, we can find  $N_1$ and $N_2$ in $A$ such that 
$$u'\circ f=\phi_{N_1}, \textrm{ and }\hat{f}\circ u=\phi_{N_2}.$$
Thus we have 
$$u'\circ f\circ \hat{f}\circ u=(u'\circ f)\circ (\hat{f}\circ u)=\phi_{N_1N_2}.$$
On the other hand, we compute in different order and get
$$u'\circ f\circ \hat{f}\circ u=u'\circ (f\circ \hat{f})\circ u=u'\circ(\phi/H)_\ell\circ u=\phi_\ell\circ(u'\circ u)=\phi_{\ell b}.$$
Thus we get the equality $\ell b=N_1N_2$.  As $\ell$ is prime, we have either case (1): $\ell | N_1$ or case (2): $\ell | N_2$.

\begin{enumerate}
\item[case (1):] $\ell | N_1$.

Then we may write $N_1=\ell\cdot \beta$ for some $0\neq\beta\in A$. From the equality $u'\circ f=\phi_{N_1}$, we have

$$\log \deg u'+k\cdot \deg_T\ell=r(\deg_T\ell+\deg_T\beta).$$

Hence we get $\log \deg u'= (r-k)\deg_T\ell+r\deg_T\beta$. Combining with the bound $\deg u'\leqslant c_2\cdot(dh(\phi/H))^{10(d+1)^7}$, we obtain the inequality
$$(r-k)\deg_T\ell\leqslant \log c_2 +10(d+1)^7\log[dh(\phi/H)]-r\deg_T\beta\leqslant \log c_2 +10(d+1)^7\log[dh(\phi/H)]$$

Thus we have
\begin{equation}
\deg_T\ell\leqslant\frac{1}{r-k}\cdot \left(  \log c_2 +10(d+1)^7\log[dh(\phi/H)]  \right)\leqslant \log c_2 +10(d+1)^7\log[dh(\phi/H)].\tag{$\star$}
\end{equation}

Now from Corollary \ref{heightineq} and Theorem \ref{bp}, we have
$$h(\phi/H)\leqslant (q^r-1)h_G(\phi/H)\leqslant (q^r-1)\cdot\left[h_G(\phi)+\deg_T\ell+(\frac{q}{q-1}-\frac{q^r}{q^r-1})       \right].$$
Deduce from the above inequality, we get
$$\begin{array}{ccc}
\log h(\phi/H)&\leqslant& \log (q^r-1)+\log \left( h_G(\phi)+\deg_T\ell+(\frac{q}{q-1}-\frac{q^r}{q^r-1})                    \right)\\
\ \\
&\leqslant&r+\log \deg_T\ell+\log\left(      h_G(\phi)+1+(\frac{q}{q-1}-\frac{q^r}{q^r-1})               \right)
\end{array}$$

Combining with the inequality ($\star$), we have the desired inequality (1)
$$
\deg_T\ell-10(d+1)^7\log \deg_T\ell   \leqslant \log c_2+ 10(d+1)^7\left\{ \log d +r+\log[h_G(\phi)+1+\frac{q}{q-1}-\frac{q^r}{q^r-1}]               \right\}
$$



\item[case (2):] $\ell | N_2$.

Then we may write $N_2=\ell\cdot \beta$ for some $0\neq\beta\in A$. From the equality $\hat{f}\circ u=\phi_{N_2}$, we have
$$(r-k)\deg_T\ell+\log \deg u=r(\deg_T\ell+\deg_T\beta).$$

Thus we get $\log \deg u=k\deg_T\ell+r\deg_T \beta$. Together with the bound $\deg u\leqslant c_2\cdot(dh(\phi))^{10(d+1)^7}$, we achieve that
$$k\deg_T\ell\leqslant \log c_2+10(d+1)^7\log [dh(\phi)]-r\deg_T\beta\leqslant c_2\cdot(dh(\phi))^{10(d+1)^7}.$$
Hence we have the inequality (2)
$$
\deg_T\ell\leqslant \frac{1}{k}\cdot \left(\log c_2+10(d+1)^7[\log d\cdot h(\phi)]\right)\leqslant \log c_2+10(d+1)^7[\log d\cdot h(\phi)]
$$



\end{enumerate}
This complete the proof of Theorem \ref{main}. 
\section{Lower bound on irreducibility of  $\bar{\rho}_{\phi,\fl}$}

Under the setting of Theorem \ref{main}, one may further solve the inequality (1) for $\deg_T\ell$. Set
$$\Omega_\phi:={\rm max}\left\{\log c_2+ 10(d+1)^7\left( \log d +r+\log[h_G(\phi)+1+\frac{q}{q-1}-\frac{q^r}{q^r-1}]\right), \log c_2+10(d+1)^7[\log d\cdot h(\phi)] \right\},$$
and $N_d:=10(d+1)^7$. Theorem \ref{main} implies that  the mod-$\fl$ Galois representation is irreducible when $$\textrm{(1'): } \frac{q^{\deg_T\ell}}{\deg_T\ell^{N_d}}>q^{\Omega_\phi}  \textrm{ and (2): } \deg_T\ell>\Omega_\phi$$

Since when we fix a finite extension $K/F$ and a Drinfeld module $\phi$, the numbers $N_d$ and $\Omega_\phi$ are fixed. Elementary Calculus can tell us that the fraction $\frac{q^{\deg_T\ell}}{\deg_T\ell^{N_d}}$ tends to infinity as $\deg_T\ell$ goes to infinity. Thus we can always find a real number $C_{\phi,d}$ such that $\deg_T\ell>C_{\phi.d}$ implies $\frac{q^{\deg_T\ell}}{\deg_T\ell^{N_d}}>q^{\Omega_\phi}$. Therefore, we can conclude the following corollary:

\begin{cor} \label{cormain}
Let $q=p^e$ be a prime power, $A:=\F_q[T]$, and $K$ be a finite extension of $F:=\F_q(T)$ of degree $d$. Let $\phi$ be a rank-$r$ Drinfeld $A$-module over $K$ of generic characteristic and assume that ${\rm End}_{\bar{K}}(\phi)=A$. Let $\fl=(\ell)$ be a prime ideal of $A$, consider the mod-$\fl$ Galois representation
$$\bar{\rho}_{\phi,\fl}:\Gal(\bar{K}/K)\rightarrow {\rm Aut}(\phi[\fl])\cong \GL_r(A/\fl).$$

If $\deg_T\ell>{\rm max}\{C_{\phi,d}, \Omega_\phi\}$, then $\bar{\rho}_{\phi,\fl}$ is irreducible. Here

$$\Omega_\phi:={\rm max}\left\{\log c_2+ 10(d+1)^7\left( \log d +r+\log[h_G(\phi)+1+\frac{q}{q-1}-\frac{q^r}{q^r-1}]\right), \log c_2+10(d+1)^7[\log d\cdot h(\phi)] \right\},$$

and $C_{\phi,d}$ is a real solution of the inequality $\frac{q^{\deg_T\ell}}{\deg_T\ell^{10(d+1)^7}}>q^{\Omega_\phi}$ for $\deg_T\ell$.

\end{cor}

\begin{rem}
Unfortunately, the inequality $$\frac{q^{\deg_T\ell}}{\deg_T\ell^{10(d+1)^7}}>q^{\Omega_\phi}$$ does not have a closed-form solution. However, one can solve this inequality using numerical method and the Lambert W-function, i.e. the inverse function of the complex function $$f(z)=ze^z.$$ 

\end{rem}
\section*{Acknowledgement}

The author would like to thank Sophie Marcques for inspiring discussions.






\bibliographystyle{alpha}
\bibliography{DMirredbd.bib}

\end{document}