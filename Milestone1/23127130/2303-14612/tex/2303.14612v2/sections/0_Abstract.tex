\begin{abstract}
The metaverse has gained significant attention from various industries due to its potential to create a fully immersive and interactive virtual world. However, the integration of deepfakes in the metaverse brings serious security implications, particularly with regard to impersonation. This paper examines the security implications of deepfakes in the metaverse, specifically in the context of gaming, online meetings, and virtual offices. The paper discusses how deepfakes can be used to impersonate in gaming scenarios, how online meetings in the metaverse open the door for impersonation, and how virtual offices in the metaverse lack physical authentication, making it easier for attackers to impersonate someone. The implications of these security concerns are discussed in relation to the confidentiality, integrity, and availability (CIA) triad. The paper further explores related issues such as the darkverse, and digital cloning, as well as regulatory and privacy concerns associated with addressing security threats in the virtual world.
\end{abstract}