\section{Introduction}

\begin{figure}
    \centering
\includegraphics[clip, trim=26.5pt 30.5pt 27pt 30pt, width=1\linewidth]{figures/Metaverse.pdf}
    \caption{The three most commonly publicized scenarios in the metaverse: virtual gaming, virtual meetings, and virtual offices. These applications highlight the potential for immersive virtual experiences, but also raise concerns about security and privacy in this emerging technology.}
    \label{fig:scenarios}
\end{figure}

% ----Online gaming

% - Essential technologies: VR/AR tech for motion control. Collectable items with real-world value (\$). Use of real time data. 3D world

% - Popular metaverse games.

% Axis Infinity https://axieinfinity.com/.

% Sandbox https://www.sandbox.game/en/.

% Minecraft https://www.minecraft.net/en-us

% - Security and privacy concerns

% --(Privacy) 20 minutes VR session could generate 2 million unique data points which could reveal a lot about the individual 

% https://wirewheel.io/blog/privacy-ai/ 

% --(Scam) Non-fungible tokens (NFTs) scams could be much easier in the metaverse gaming. It could be introduced in 10 different ways. 

% https://us.norton.com/blog/online-scams/nft-scams 

 % The metaverse, a virtual world where people can interact with each other and objects in a fully immersive environment, has become a hot topic among technology companies. Meta and other companies have advertised three different scenarios where the metaverse can be utilized: gaming, online meetings, and virtual offices. While these scenarios have the potential to revolutionize the way we interact with each other, the integration of deepfakes in the metaverse brings serious security implications, particularly with regards to impersonation.

 The emergence of the metaverse~\cite{Metaverse_Survey} has captured the attention of the technology community, giving rise to widespread anticipation and debate. Prominent companies such as Meta (formerly Facebook), Microsoft, and Nvidia have expressed interest in the concept of a fully immersive virtual world, where individuals can interact with one another and their surroundings. To this end, various companies are releasing their own metaverse experiences, including Meta's Horizon Worlds~\cite{Facebook_metaverse,Meta_Horizon_Worlds}, Roblox's gaming metaverse~\cite{Roblox_metaverse}, Microsoft's Mesh~\cite{Microsoft_Mesh}, and Nvidia's Omniverse~\cite{NVIDIA_Omniverse}. While the potential applications of the metaverse are vast, with possibilities ranging from gaming to virtual meetings and offices, each method has its own benefits and limitations.

However, the applicability of deepfakes~\cite{Deepfake_survey,le2023deepfake} in the metaverse presents significant security implications, particularly with regard to impersonation. Deepfakes are computer-generated images or videos that can be manipulated to look like real people or events. The ability to generate such content has significantly increased with advances in machine learning and artificial intelligence. Recently, deepfakes in the metaverse have become a topic of discussion on different forums and media articles~\cite{Media_DeepfakesInMetaverse_1,Media_DeepfakesInMetaverse_2,Media_DeepfakesInMetaverse_3,Media_DeepfakesInMetaverse_4}.

The benefits of metaverse technology are numerous, including the ability to interact with others and objects in a fully immersive virtual environment. However, the potential for deepfake technology to be utilized in malicious ways, such as impersonation, highlights the limitations of the technology and the need for robust security measures.

In this paper, we explore the security implications of deepfakes in the metaverse. We will start by defining the concept of the metaverse and how it is expected to be used in three scenarios: (i) gaming, (ii) online meetings, and (iii) virtual offices (see Fig.~\ref{fig:scenarios}). We will also discuss the potential dangers of deepfake in each of the three scenarios by exploring the potential consequences of deepfake misuse in the metaverse, such as the ability to impersonate others, manipulate meetings, and disrupt virtual work environments. We will also discuss potential solutions to mitigate these risks and ensure the safety and security of metaverse users. We also explore the security implications of deepfakes in the metaverse to fake digital identity, the CIA triad, legal and regulatory challenges, privacy issues, and darkverse. Through this work, we aim to explore the potential security implications of deepfakes in the metaverse and to raise awareness of the risks and challenges posed by this  technology. 



