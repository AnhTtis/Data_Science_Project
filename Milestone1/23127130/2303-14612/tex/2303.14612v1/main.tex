%%
%% This is file `sample-sigconf.tex',
%% generated with the docstrip utility.
%%
%% The original source files were:
%%
%% samples.dtx  (with options: `sigconf')
%% 
%% IMPORTANT NOTICE:
%% 
%% For the copyright see the source file.
%% 
%% Any modified versions of this file must be renamed
%% with new filenames distinct from sample-sigconf.tex.
%% 
%% For distribution of the original source see the terms
%% for copying and modification in the file samples.dtx.
%% 
%% This generated file may be distributed as long as the
%% original source files, as listed above, are part of the
%% same distribution. (The sources need not necessarily be
%% in the same archive or directory.)
%%
%% Commands for TeXCount
%TC:macro \cite [option:text,text]
%TC:macro \citep [option:text,text]
%TC:macro \citet [option:text,text]
%TC:envir table 0 1
%TC:envir table* 0 1
%TC:envir tabular [ignore] word
%TC:envir displaymath 0 word
%TC:envir math 0 word
%TC:envir comment 0 0
%%
%%
%% The first command in your LaTeX source must be the \documentclass command.
%\usepackage[table,xcdraw]{xcolor}
% \PassOptionsToPackage{hyphens}{url}
\documentclass[sigconf,nonacm]{acmart}
% \usepackage[colorinlistoftodos]{todonotes}
% \usepackage{soul}
% \usepackage{xr}
% \usepackage{epsfig}
% \usepackage{graphicx,comment}
% \usepackage{amsmath}
% \usepackage{indentfirst}
% \usepackage{algorithm}
% \usepackage{algpseudocode} 
% % \usepackage{amssymb}
% \usepackage{verbatim}
% \usepackage{multirow}
% \usepackage{booktabs}
% \usepackage{makecell}
% \usepackage{longtable}
% \usepackage{soul}
% % \usepackage{caption}
% \usepackage{arydshln}
% \usepackage{enumitem}
% \usepackage{xcolor}
% \usepackage[dvipsnames]{xcolor}


% \usepackage{xspace}
% \usepackage{bm}



% %%% START: COLOR for the table
% \usepackage{color, colortbl}
% \definecolor{codegreen}{rgb}{0,0.6,0}
% \definecolor{codegray}{rgb}{0.5,0.5,0.5}
% \definecolor{codepurple}{rgb}{0.58,0,0.82}
% \definecolor{backcolour}{rgb}{0.95,0.95,0.92}
% \definecolor{LightCyan}{rgb}{0.88,1,1}
% \definecolor{LightRed}{RGB}{255, 204, 203}
% %%% END: COLOR for the table

% %%% HELPER CODE FOR ARGMIN, ARGMAX
% \usepackage{amsmath}
% \DeclareMathOperator*{\argmax}{arg\,max}
% \DeclareMathOperator*{\argmin}{arg\,min}
% % \DeclareMathOperator*{\argmaxA}{arg\,max} % Jan Hlavacek
% %%% END HELPER CODER FOR ARGMIN ARGMSAX
% \newcommand\Tstrut{\rule{0pt}{2.6ex}}       % "top" strut
% \newcommand\Bstrut{\rule[-1.1ex]{0pt}{0pt}} % "bottom" strut
% \newcommand{\TBstrut}{\Tstrut\Bstrut} % top&bottom struts
% %%% END: HELPER CODE FOR  TOP and BOTTOM CELL SPACE
% %%% START VERTICLE LINE IN TABLE
% \newcommand\VRule[1][\arrayrulewidth]{\vrule width #1}
% %%% END

%%% START: CODE FOR SET FIGURE EXACTLY ON TOP OF PAGE
\makeatletter
\setlength{\@fptop}{0pt}
\makeatother
%%% END: CODE FOR SET FIGURE EXACTLY ON TOP OF PAGE

% %%% HELPER CODE FOR DASH LINES IN TABLES

% \makeatletter
% \def\adl@drawiv#1#2#3{%
%         \hskip.5\tabcolsep
%         \xleaders#3{#2.5\@tempdimb #1{1}#2.5\@tempdimb}%
%                 #2\z@ plus1fil minus1fil\relax
%         \hskip.5\tabcolsep}
% \newcommand{\cdashlinelr}[1]{%
%   \noalign{\vskip\aboverulesep
%            \global\let\@dashdrawstore\adl@draw
%            \global\let\adl@draw\adl@drawiv}
%   \cdashline{#1}
%   \noalign{\global\let\adl@draw\@dashdrawstore
%            \vskip\belowrulesep}}
% \makeatother

% %%% END: HELPER CODE FOR DASH LINES IN TABLES

% %%% HELPER CODE FOR "CHECK"
% \usepackage{pifont}% http://ctan.org/pkg/pifont
% \newcommand{\cmark}{\ding{51}}%
% \newcommand{\xmark}{\ding{55}}%


% \usepackage{minibox}
% \newcounter{observcntr}
% \newcommand*{\observ}[1]{%
%     \stepcounter{observcntr}%
%     \begin{center}
%     \vspace{2pt}
%     \minibox[frame, rule=1pt,pad=3pt]{
%         \begin{minipage}[t]{0.95\columnwidth}
%         \textbf{Answer RQ~\arabic{observcntr}:} \textit{#1}.
%         \end{minipage}
%     }
%     \vspace{2pt}
%     \end{center}
% }

%% NOTE that a single column version may be required for 
%% submission and peer review. This can be done by changing
%% the \doucmentclass[...]{acmart} in this template to 
%% \documentclass[manuscript,screen]{acmart}
%% 
%% To ensure 100% compatibility, please check the white list of
%% approved LaTeX packages to be used with the Master Article Template at
%% https://www.acm.org/publications/taps/whitelist-of-latex-packages 
%% before creating your document. The white list page provides 
%% information on how to submit additional LaTeX packages for 
%% review and adoption.
%% Fonts used in the template cannot be substituted; margin 
%% adjustments are not allowed.
%%
%%
%% \BibTeX command to typeset BibTeX logo in the docs
\AtBeginDocument{%
  \providecommand\BibTeX{{%
    \normalfont B\kern-0.5em{\scshape i\kern-0.25em b}\kern-0.8em\TeX}}}

%% Rights management information.  This information is sent to you
%% when you complete the rights form.  These commands have SAMPLE
%% values in them; it is your responsibility as an author to replace
%% the commands and values with those provided to you when you
%% complete the rights form.
\setcopyright{acmcopyright}
\copyrightyear{2018}
\acmYear{2018}
\acmDOI{XXXXXXX.XXXXXXX}

%% These commands are for a PROCEEDINGS abstract or paper.
\acmConference[Conference acronym 'XX]{Make sure to enter the correct
  conference title from your rights confirmation emai}{June 03--05,
  2018}{Woodstock, NY}
%
%  Uncomment \acmBooktitle if th title of the proceedings is different
%  from ``Proceedings of ...''!
%
%\acmBooktitle{Woodstock '18: ACM Symposium on Neural Gaze Detection,
%  June 03--05, 2018, Woodstock, NY} 
\acmPrice{15.00}
\acmISBN{978-1-4503-XXXX-X/18/06}

\newcommand{\subheading}[1]{\noindent{\textbf{#1}}}
\newcommand{\sharif}[1]{\textsf{\color{blue}{[{#1 -- Sharif}]}}}
\newcommand{\binh}[1]{\textsf{\color{green}{[{#1 -- Binh}]}}}
\newcommand{\sh}[1]{\textsf{\color{magenta}{[{#1 -- Shahroz}]}}}
\newcommand{\kristen}[1]{\textsf{\color{violet}{[{#1 -- Kristen}]}}}
%%
%% Submission ID.
%% Use this when submitting an article to a sponsored event. You'll
%% receive a unique submission ID from the organizers
%% of the event, and this ID should be used as the parameter to this command.
%%\acmSubmissionID{123-A56-BU3}

%%
%% For managing citations, it is recommended to use bibliography
%% files in BibTeX format.
%%
%% You can then either use BibTeX with the ACM-Reference-Format style,
%% or BibLaTeX with the acmnumeric or acmauthoryear sytles, that include
%% support for advanced citation of software artefact from the
%% biblatex-software package, also separately available on CTAN.
%%
%% Look at the sample-*-biblatex.tex files for templates showcasing
%% the biblatex styles.
%%

%%
%% The majority of ACM publications use numbered citations and
%% references.  The command \citestyle{authoryear} switches to the
%% "author year" style.
%%
%% If you are preparing content for an event
%% sponsored by ACM SIGGRAPH, you must use the "author year" style of
%% citations and references.
%% Uncommenting
%% the next command will enable that style.
%%\citestyle{acmauthoryear}

%%
%% end of the preamble, start of the body of the document source.
\begin{document}

%%
%% The "title" command has an optional parameter,
%% allowing the author to define a "short title" to be used in page headers.
% \title{Security Implications of Three Prominent Use Cases in Metaverse}
\title{Deepfake in the Metaverse: Security Implications for Virtual Gaming, Meetings, and Offices}

\author{Shahroz Tariq}
\affiliation{%
  \institution{CSIRO's Data61, Australia}
  \country{}}
\email{shahroz.tariq@data61.csiro.au}

\author{Alsharif Abuadbba}
\affiliation{%
  \institution{CSIRO's Data61, Australia}
  \country{}}
\email{sharif.abuadbba@data61.csiro.au}

\author{Kristen Moore}
\affiliation{%
  \institution{CSIRO's Data61, Australia}
  \country{}}
\email{kristen.moore@data61.csiro.au}
%%
%% The "author" command and its associated commands are used to define
%% the authors and their affiliations.
%% Of note is the shared affiliation of the first two authors, and the
%% "authornote" and "authornotemark" commands
%% used to denote shared contribution to the research.
% \author{Ben Trovato}
% \authornote{Both authors contributed equally to this research.}
% \email{trovato@corporation.com}
% \orcid{1234-5678-9012}
% \author{G.K.M. Tobin}
% \authornotemark[1]
% \email{webmaster@marysville-ohio.com}
% \affiliation{%
%   \institution{Institute for Clarity in Documentation}
%   \streetaddress{P.O. Box 1212}
%   \city{Dublin}
%   \state{Ohio}
%   \country{USA}
%   \postcode{43017-6221}
% }

% \author{Lars Th{\o}rv{\"a}ld}
% \affiliation{%
%   \institution{The Th{\o}rv{\"a}ld Group}
%   \streetaddress{1 Th{\o}rv{\"a}ld Circle}
%   \city{Hekla}
%   \country{Iceland}}
% \email{larst@affiliation.org}

% \author{Valerie B\'eranger}
% \affiliation{%
%   \institution{Inria Paris-Rocquencourt}
%   \city{Rocquencourt}
%   \country{France}
% }

% \author{Aparna Patel}
% \affiliation{%
%  \institution{Rajiv Gandhi University}
%  \streetaddress{Rono-Hills}
%  \city{Doimukh}
%  \state{Arunachal Pradesh}
%  \country{India}}

% \author{Huifen Chan}
% \affiliation{%
%   \institution{Tsinghua University}
%   \streetaddress{30 Shuangqing Rd}
%   \city{Haidian Qu}
%   \state{Beijing Shi}
%   \country{China}}

% \author{Charles Palmer}
% \affiliation{%
%   \institution{Palmer Research Laboratories}
%   \streetaddress{8600 Datapoint Drive}
%   \city{San Antonio}
%   \state{Texas}
%   \country{USA}
%   \postcode{78229}}
% \email{cpalmer@prl.com}

% \author{John Smith}
% \affiliation{%
%   \institution{The Th{\o}rv{\"a}ld Group}
%   \streetaddress{1 Th{\o}rv{\"a}ld Circle}
%   \city{Hekla}
%   \country{Iceland}}
% \email{jsmith@affiliation.org}

% \author{Julius P. Kumquat}
% \affiliation{%
%   \institution{The Kumquat Consortium}
%   \city{New York}
%   \country{USA}}
% \email{jpkumquat@consortium.net}

%%
%% By default, the full list of authors will be used in the page
%% headers. Often, this list is too long, and will overlap
%% other information printed in the page headers. This command allows
%% the author to define a more concise list
%% of authors' names for this purpose.
\renewcommand{\shortauthors}{Tariq, et al.}

%%
%% The abstract is a short summary of the work to be presented in the
%% article.


%%
%% The code below is generated by the tool at http://dl.acm.org/ccs.cfm.
%% Please copy and paste the code instead of the example below.
%%
% \begin{CCSXML}
% <ccs2012>
%  <concept>
%   <concept_id>10010520.10010553.10010562</concept_id>
%   <concept_desc>Computer systems organization~Embedded systems</concept_desc>
%   <concept_significance>500</concept_significance>
%  </concept>
%  <concept>
%   <concept_id>10010520.10010575.10010755</concept_id>
%   <concept_desc>Computer systems organization~Redundancy</concept_desc>
%   <concept_significance>300</concept_significance>
%  </concept>
%  <concept>
%   <concept_id>10010520.10010553.10010554</concept_id>
%   <concept_desc>Computer systems organization~Robotics</concept_desc>
%   <concept_significance>100</concept_significance>
%  </concept>
%  <concept>
%   <concept_id>10003033.10003083.10003095</concept_id>
%   <concept_desc>Networks~Network reliability</concept_desc>
%   <concept_significance>100</concept_significance>
%  </concept>
% </ccs2012>
% \end{CCSXML}

% \ccsdesc[500]{Computer systems organization~Embedded systems}
% \ccsdesc[300]{Computer systems organization~Redundancy}
% \ccsdesc{Computer systems organization~Robotics}
% \ccsdesc[100]{Networks~Network reliability}

%%
%% Keywords. The author(s) should pick words that accurately describe
%% the work being presented. Separate the keywords with commas.
\keywords{metaverse, deepfake, security, impersonation, gaming, online meetings, virtual offices}


\begin{abstract}
The metaverse has gained significant attention from various industries due to its potential to create a fully immersive and interactive virtual world. However, the integration of deepfakes in the metaverse brings serious security implications, particularly with regard to impersonation. This paper examines the security implications of deepfakes in the metaverse, specifically in the context of gaming, online meetings, and virtual offices. The paper discusses how deepfakes can be used to impersonate in gaming scenarios, how online meetings in the metaverse open the door for impersonation, and how virtual offices in the metaverse lack physical authentication, making it easier for attackers to impersonate someone. The implications of these security concerns are discussed in relation to the confidentiality, integrity, and availability (CIA) triad. The paper further explores related issues such as the darkverse, and digital cloning, as well as regulatory and privacy concerns associated with addressing security threats in the virtual world.
\end{abstract}
\maketitle
\vspace{-10pt}
%%%%%%%%% BODY TEXT

\section{Introduction}
\label{section:introduction}
%% 1. why should someone care?

%The advent of advanced interactive computer vision systems~\cite{hololens} and recent progress in vision-language and multi-modal models~\cite{} opens doors for such next generation of assistive agents. 
% We envision that the future assistive agents would build up on these visual and language reasoning capabilities of today and empower users to achieve goals in their everyday lives. In particular, such agents would be able to reason about \emph{unseen} human goals... 
% We posit that such agents would require the ability to understand user goals described in natural language at high-level i.e., without complete details about as well as unseen user goals. 

%Recent progress in augmented reality systems~\cite{hololens, magicleap}, as well as vision-language and multi-modal models~\cite{}, opens doors for the next generation of assistive agents. 
Inspired by recent progress in visual systems~\cite{MagicLeap, ungureanu2020hololens}, we consider an assistive egocentric agent capable of reasoning about daily activities. When invoked via natural language commands, for e.g., while baking a cake, the agent understands the steps involved in baking, tracks progress through the various stages of the task, detects and proactively prevents mistakes by making suggestions. Such an agent would empower users to learn new skills and accomplish tasks efficiently.
% One could envision invoking such an agent merely through natural language descriptions of tasks similar to how present day assistants such as Alexa, Siri etc.~\cite{voice_assistants} are invoked. 
%We envision such agents to empower users in daily life by  invoking them naturally through 

%% 2. Why is it challenging? 
%While recent progress in vision-language and multi-modal models~\cite{} opens doors for such next generation of assistive agents, various challenges remain in making such agents a reality. 
%To make such agents a reality, 

Developing such an egocentric agent capable of tracking and verifying everyday tasks based on their natural language specification is challenging for multiple reasons. First, such an agent must reason about various ways of doing a \emph{multi-step} task specified in natural language. This entails decomposing the task into relevant actions, state changes, object interactions as well as any necessary causal and temporal relationships between these entities. Secondly, the agent must ground these entities in egocentric observations to track progress and detect mistakes. Lastly, to truly be useful, such an agent must support tracking and verification for a combination of tasks and, ideally, even unseen tasks. These three challenges -- causal and temporal reasoning about task structure from natural language, visual grounding of sub-tasks, and compositional generalization -- form the core goals of our work.

% %% 3. What are we doing? What is our approach?
% \aks{I think this is a matter of preference, but I personally don't like related work in intro. I would make this paragraph be about EgoTV and NSG. Starting with something like - "To this end, we propose...", ie, your next paragraph.}
% \nk{+1, we should move parts of this para to lit review and delete the rest.}
% Recent research on language modeling enables decomposing tasks into multiple steps from natural language descriptions~\cite{llm_zero_shot_planning,proscript}. However, such \emph{task decompositions} cannot directly be leveraged for task tracking in egocentric agents because of lack of grounding into the visual observations or context. In parallel, the computer vision community has advanced action recognition~\cite{}, object detection and tracking~\cite{}, hand object interaction and object state change detection~\cite{ego_4d,change_it,}, step classification in procedural tasks~\cite{}, and even vision language reasoning~\cite{nsvqa,nscl,star_situated_reasoning,clevrer}, which may help with the grounding challenge. However, majority of current research on identifying actions, objects, steps, or state changes does not account for the overall task structure. Likewise, predominant research on vision language understanding~\cite{} and multi-modal grounding~\cite{} does not consider the temporal and causal constraints that emerge in task tracking and verification. We therefore focus on the order-aware visual grounding problem in our work, with an eye towards compositional generalization to scale usability of these agents. In particular, we aim to achieve visual grounding of the actions and objects corresponding to each step or sub-task obtained from the task description decomposition in an order-aware manner.

%% 4. What are our results/contributions?
As our first contribution, we propose a benchmark -- \emph{\textbf{Ego}centric \textbf{T}ask \textbf{V}erification} (\etv \inlineimg{figures/TV}) -- and a corresponding dataset in the AI2-THOR~\cite{ai2thor} simulator. % \emoji{tv}
Given a natural language (NL) task description and a corresponding egocentric video of an agent, the goal of \etv is to verify whether the task was successfully completed in the video or not.
\etv contains multi-step tasks with \emph{ordering} constraints on the steps and \emph{abstracted} NL task descriptions with omitted low-level task details inspired by the needs of real-world assistants. We also provide splits of the dataset focused on different generalization aspects, e.g., unseen visual contexts, compositions of steps, and tasks (see Figure~\ref{figure:dataset}).
% Next, we create splits of the dataset focused on different aspects of generalization, ranging from generalization to unseen visual context to unseen compositions of steps and tasks. Figure~\ref{figure:dataset} shows an example task and overview of generalization splits from \etv. Succeeding at \etv tasks requires decomposing tasks into partially-ordered steps from the NL description and order-aware visual grounding of these steps into the video. 

Our second contribution is a novel approach for order-aware visual grounding~--~\emph{\textbf{N}euro-\textbf{S}ymbolic \textbf{G}rounding} (NSG), capable of compositional reasoning and generalizing to unseen tasks owing to its ability to leverage abstract NL descriptions and compositional structure of tasks (task decomposition, ordering).~In contrast, state-of-the-art vision-language models~\cite{coca,clip,videoclip,clip_hitchiker} struggle to ground NL descriptions in egocentric videos, and do not generalize to unseen tasks.~NSG outperforms these models by~$\mathbf{33.8}\%$~on compositional generalization and~$\mathbf{32.8}\%$~on abstractly described task verification. Finally, to evaluate \nsg on real-world data, we instantiate \etv on the CrossTask~\cite{cross_task} instructional video dataset. %Specifically, we synthetically create videos with mistakes in CrossTask. 
We find that it also outperforms state-of-the-art models at task verification on CrossTask. We hope that the \etv~benchmark and dataset will enable future research on egocentric agents capable of aiding in everyday tasks.

% We experiment with many for the \etv tasks. We find that while these models generalize well to unseen visual context, they struggle to perform grounding from abstracted task descriptions and to generalize to new compositions of tasks. To deal with these challenges, we take inspiration from recent research on and develop . ~\rd{unclear why neurosymbolic models would do well on abstraction.} 

% To summarize, our main contributions are:~1)~\etv: a benchmark and synthetic dataset to systematically study egocentric task verification.
% 2)~\nsg: a novel neuro-symbolic approach to enable the core reasoning capability for \etv -- order-aware visual grounding. We demonstrate \nsg's capability on our synthetic \etv dataset as well as a real-world dataset derived from CrossTask. We will release both of these datasets and our models for future research on egocentric task tracking and verification. 


% Assistive agents require the ability to track actions and state changes from an egocentric perspective for effective assistance in day-to-day tasks. For example, an agent helping a user prepare a recipe would need to both generate the steps of the recipe (\textit{plan generation}) and track the user's actions to ensure the plan is executed correctly (\textit{plan verification}). We formulate this as a Video Entailment task~\cite{violin_dataset,9710490} \rd{should we call our task video-based goal entailment?}, wherein, given an egocentric video of an agent (or human) performing a task (\textit{premise}) and a NL task description (\textit{hypothesis}), the objective is to learn a model to track whether the given task was successfully executed in the video. 
% An ideal model should also be able to seamlessly generalize to novel compositions (of actions and objects) unseen during training. \rd{add a line about what we mean by abstraction and why is it important.} To this end, we generate a novel Vision-Language dataset on the AI2-THOR simulator~\cite{ai2thor} to study compositional and abstraction-based generalization. Our dataset provides effective evaluation measures in a controlled setting, while closely reflecting the diversity of real-world events. We implement and train a variety of end-to-end models based on existing state-of-the-art approaches. We empirically demonstrate that neural models suffer from overfitting and cannot effectively generalize to novel compositions of actions, objects, and scenes. 
% To address this problem, we propose an end-to-end Neuro-Symbolic (NeSy) framework that performs plan generation and verification. At the heart of our approach is the hypothesis that symbolic reasoning models are good at generalization and capturing compositional substructure, while neural models are good at learning representations from sensory data~\cite{10.5555/3326943.3327039,nscl,clevrer}. \rd{summarize contributions in a bulleted list.} \rd{also add a line about the main result e.g., x\% improvement as compared to end-to-end models}. 

% \rd{we also evaluate NeSy with real-world data: add briefly about CrossTask experiments.}

% % \fbox{\begin{minipage}{\linewidth}
% % \textbf{Problem Statement}

% % Given:
% % (i) Premise: Egocentric video of an agent performing a task.
% % (ii) Hypothesis: NL description of the task.

% % Learn: A model to track whether the premise entails the hypothesis. The output of the model is True if the given task is executed successfully in the video.
% % \end{minipage}}

% \textbf{Contributions:} 
% \begin{itemize}
%     \item We generate a benchmark video-language dataset to study compositional and abstraction-based generalization.
%     \item We evaluate the performance of a variety of state-of-the-art models and show that these (baseline) models cannot effectively generalize to novel compositions of actions.
%     \item We propose a novel end-to-end NeSy approach that significantly outperforms the baselines on some compositional generalization splits while performing on par with them on the rest.
%     \item We also evaluate our NeSy approach with real-world data showing similar performance improvements.
% \end{itemize}

\section{What is the Metaverse?}
The metaverse is a collective virtual shared space that offers an immersive and interactive 3D environment, facilitating real-time engagement with digital content and other individuals through advanced technologies such as virtual reality (VR) and augmented reality (AR). While originally a concept in science fiction, such as in Neal Stephenson's novel "Snow Crash" and the movie "The Matrix," recent technological advances have made the metaverse increasingly feasible. As a new form of social and economic infrastructure, the metaverse provides opportunities for people to work, play, socialize, learn, and consume content within a shared virtual space. While different visions of the metaverse exist among companies, organizations, and individuals, ranging from a fully autonomous world to a combination of different virtual platforms and experiences, the metaverse is considered a transformative technology that has the potential to  impact various aspects of our lives.

\section{Gaming in the Metaverse}
Gaming in the metaverse encompasses playing video games in a virtual world shared by millions globally, including massively multiplayer online games (MMOs), social games, and casual games such as puzzle and card games. These games offer interactive and detailed virtual worlds, providing players with opportunities to explore and interact with their surroundings.

The use of deepfakes in the context of gaming in the metaverse raises significant security concerns. Potential issues include identity theft, cyberbullying, distribution of malware, non-fungible token (NFT) scams, and intellectual property theft. As a significant demographic within metaverse gaming~\cite{Kids_metaverse_gaming}, minors are particularly vulnerable to these threats due to their limited experience and knowledge of online safety, which can result in sexual exploitation, social engineering, online grooming, and exposure to misinformation.

To address these risks, gaming companies must invest in advanced security measures, including identity verification systems, content monitoring and moderation tools, and anti-malware software. Additionally, players should be educated about the dangers of deepfakes and encouraged to report any suspicious activity encountered in the metaverse. Parents and guardians should take an active role in educating minors on the risks of deepfakes, monitoring their online activities, and encouraging them to report any questionable behavior. Online safety education, privacy settings, and parental controls can also help safeguard minors from the potential harms of deepfakes. 
% Gaming in the metaverse refers to the experience of playing video games in a virtual world that is created and shared by millions of people around the globe. Gaming in the metaverse can take on many forms, such as massively multiplayer online games (MMOs), social games, and even casual games like puzzle games and card games. These games are typically designed to be played within a virtual world that is rich in detail and interactivity, allowing players to explore, discover, and interact with their surroundings. 

% In the context of gaming in the metaverse, deepfakes could be used to create fake game items, fake avatars, or even fake interactions between players. This could lead to a variety of security concerns, including:

% Identity theft: Deepfakes could be used to create fake identities in the metaverse, which could be used to commit fraud or other types of cybercrime.

% Cyberbullying: Deepfakes could be used to create fake interactions between players, such as fake chat logs or fake in-game messages, which could be used to harass or bully other players.

% Malware: Deepfakes could be used to distribute malware or other types of malicious software through the metaverse.

% Intellectual property theft: Deepfakes could be used to create fake game items or other virtual goods, which could be sold or traded for real-world money.

% metaverse gaming is a popular activity among minors~\cite{Kids_metaverse_gaming}. At the same time, minors are more vulnerable to the threat of deepfakes due to their lack of experience and knowledge about online safety. They can be an easy target for sexual exploitation, social engineering, online grooming, and misinformation.

% To mitigate these risks, gaming companies will need to invest in advanced security measures, such as advanced identity verification systems, content moderation tools, and anti-malware software. Additionally, players should be educated on the risks of deepfakes and encouraged to report any suspicious activity they encounter within the metaverse. Especially, it is important for parents and guardians to educate minors about the risks of deepfakes, monitor their online activity, and encourage them to report any suspicious activity they encounter online. Online safety education, privacy settings, and parental controls can also help protect minors from the potential harms of deepfakes.




\section{Online Meetings in the Metaverse}
Online meetings in the metaverse offer a virtual space for individuals to communicate and collaborate within a shared immersive digital environment, spanning from basic text-based chat rooms to fully immersive 3D environments, and utilizing voice chat, instant messaging, or other means of communication. These virtual meetings are versatile and can serve various purposes, such as team collaboration, networking, socializing, or attending virtual events, such as conferences. 

However, the potential risks to privacy and reputation are significant due to the metaverse's ability to create an opportunity for attackers to impersonate others. Deepfake technology can be utilized to deceive others through impersonation, leading to potential fraud or espionage, as demonstrated in a recent example of Elon Musk's deepfake zoom-bombing online meetings~\cite{Elon_ZoomBomb}. In addition, the authenticity and trust of participants may be compromised by the creation of convincing deepfakes, which could undermine trust and collaboration.

To address these issues, it is essential to implement measures such as identity verification, digital signatures, or other security measures to ensure the authenticity of participants. Moreover, the development of tools and technologies that can detect various forms of deepfakes in the metaverse and prevent their use during online meetings may be necessary. Overall, while deepfake technology poses a disruptive risk to online meetings in the metaverse, it is also possible to mitigate these risks effectively by utilizing appropriate security measures and technology.

% Online meetings in the metaverse are virtual gatherings that take place in a shared, immersive digital environment. Online meetings in the metaverse provide a platform for people to communicate and collaborate in a virtual environment. Virtual meetings in the metaverse can take on a variety of formats, from simple text-based chat rooms to fully immersive 3D environments. Participants can communicate with each other through voice chat, instant messaging, or other means. These meetings can be used for a variety of purposes, including team collaboration, networking, socializing, or even attending virtual events such as conferences. 

% However, the metaverse also creates an opportunity for attackers to impersonate someone else, leading to privacy violations and reputational damage. One potential use of deepfake technology in online meetings is for impersonation or deception. For example, someone could create a deepfake video or audio of a person they want to impersonate and use it during a meeting to deceive other participants. This could potentially be used for nefarious purposes, such as fraud or espionage. We have seen a practical example of this recently when fake Elon Musk zoom-bomb online meetings. Another potential impact of deepfake technology on online meetings in the metaverse is on trust and authenticity. If it becomes easy to create convincing deepfakes in the metaverse, participants may become more skeptical of the authenticity of the people they are interacting with, which could undermine trust and collaboration. 

% To address these concerns, it may be necessary to implement measures such as identity verification, digital signatures, or other security measures to ensure that participants are who they say they are. Additionally, it may be necessary to develop tools or technologies that can detect different forms of deepfakes in the metaverse and prevent them from being used during online meetings. Overall, while deepfake technology has the potential to be disruptive in online meetings in the metaverse, it is also possible that it can be effectively managed and mitigated through the use of appropriate security measures and technology.

\section{Virtual Offices in the Metaverse}

The emergence of virtual offices or workplaces in the metaverse is a recent but promising development that has the potential to revolutionize collaboration and work practices. By leveraging the metaverse's virtual environment, colleagues can work together in a customizable, shared digital workspace that offers numerous benefits, such as reduced overhead cost, increased flexibility, and access to a global talent pool. In addition, virtual offices in the metaverse can enable more dynamic and immersive meetings, greater collaboration, and increased creativity.

However, the virtual nature of the metaverse presents a security challenge, as attackers can impersonate team members through the use of deepfakes, leading to data breaches and financial loss. For instance, deepfakes can be employed to create fake identities or to impersonate colleagues, which could lead to trust issues and confusion within the team. For example, an employee could use a deepfake to create a fake version of their boss or co-worker to make it appear as if they are giving instructions. Furthermore, deepfakes can be used to spread false information or propaganda, potentially impacting important decisions.

To address the potential threats posed by deepfakes in virtual offices or workplaces in the metaverse, clear guidelines and protocols are required to verify team members' identities and the authenticity of content shared in the virtual environment. This can be critical in establishing one's innocence in the case of a crime. It is also important to remain abreast of the latest deepfake technology developments and to leverage tools and software that can aid in identifying and detecting deepfakes.
% The concept of virtual offices or workplaces in the metaverse is still relatively new. Still, it is an exciting development that has the potential to revolutionize the way we work and collaborate.
% metaverse offers a unique opportunity to create a completely digital workplace where colleagues can work together in a shared virtual environment. Imagine logging into your virtual office and being able to collaborate with your colleagues in a shared virtual workspace that can be customized to your needs. Virtual offices in the metaverse could offer a range of benefits, such as reduced overhead cost, increased flexibility, and access to a global talent pool. Moreover, it could facilitate new ways of working, such as more dynamic and immersive meetings, greater collaboration, and increased creativity.

% The metaverse provides a platform for virtual offices, where employees can work from anywhere in the world. However, the lack of physical authentication in the virtual environment makes it easier for attackers to impersonate someone else, leading to data breaches and financial loss. Deepfakes could be used to create fake identities or impersonate colleagues, which could lead to trust issues and confusion within the team. For example, an employee could use a deepfake to create a fake version of their boss or co-worker to make it appear as if they are giving instructions or orders. Furthermore, deepfakes could be used to spread false information or propaganda within the virtual workspace. This could have serious consequences if important decisions are made based on this misinformation.

% To mitigate the potential impact of deepfakes in virtual offices or workplaces in the metaverse, it is essential to establish clear guidelines and protocols for verifying the identity of team members and the authenticity of the content shared in the virtual environment, which could be critical in proving one's innocence in case of a crime.  It is also important to stay informed about the latest developments in deepfake technology and to use tools and software that can help identify and detect deepfakes.



\section{Discussion}

% \sh{add a few introductory sentences}

\noindent
\textbf{Fake Digital Identity and Cloning in the digital world. } 
One of the central ideas underlying the concept of the metaverse is the ability for individuals to create digital replicas of themselves, known as avatars, in the virtual world. These avatars are designed to mimic the physical appearance and behavior of their real-life counterparts, allowing individuals to interact with one another in the digital realm. However, the ability to clone oneself in the metaverse also raises concerns about the potential for impersonation. Unlike the physical world, where impersonating someone convincingly is challenging, it is much easier to create a convincing digital clone of a person in the metaverse due to the abundance of personal information available on the internet that can be used to create deepfakes. The possibility of an attacker using deepfakes to impersonate someone in the metaverse is a significant concern, as it could be used to commit various illicit activities. % As such, the question arises as to who will be responsible for preventing impersonation in the metaverse. 
One potential solution to this problem is the implementation of digital identity verification systems. Such systems could use biometric data, such as facial recognition, to verify an individual's identity before allowing them to create a digital avatar. By doing so, attackers would be prevented from creating digital clones of other people without their consent, thereby ensuring a higher level of security in the metaverse.
% One of the central ideas behind the metaverse is the ability to clone oneself in the digital world. This means that a person can create a digital avatar that looks and acts like them in the physical world. This avatar can interact with other avatars in the virtual world, just as the person interacts with other people in the physical world. However, cloning oneself in the metaverse raises concerns about the potential for impersonation. In the physical world, it is difficult for someone to impersonate another person convincingly. However, in the metaverse, it is much easier to create a convincing digital clone of a person. This is because there is already a vast amount of personal information available on the internet that can be used to create deepfakes. The concern is that an attacker could use deepfakes to impersonate someone in the metaverse. For example, an attacker could create a digital clone of a person and use it to commit crimes or engage in other illicit activities. This raises questions about who will be responsible for preventing impersonation in the metaverse. One possible solution to this problem is to use digital identity verification systems in the metaverse. These systems would use biometric data, such as facial recognition, to verify a person's identity before allowing them to create a digital avatar. This would prevent attackers from creating digital clones of other people without their consent.

% The idea of the metaverse is built on the concept of cloning oneself in the digital world. However, this raises the question of who will stop impersonators from cloning you in the digital world. All the information that is already available about you on the internet can be used to create deepfakes, making it easier for an attacker to impersonate you.

\noindent
\textbf{Deepfakes impact on CIA Triad in the Metaverse. } The CIA Triad, established by the National Institute of Standards and Technology (NIST), comprises confidentiality, integrity, and availability, which are the three primary objectives of information security. Confidentiality safeguards sensitive information by ensuring that only authorized parties can access it. Integrity ensures that information remains accurate and unaltered, while availability guarantees that authorized parties can access information when necessary. In the context of the metaverse, deepfakes pose a potential threat to the CIA Triad's objectives of confidentiality, integrity, and availability. Specifically, deepfakes have the potential to compromise confidentiality by enabling the impersonation of authorized individuals, thereby permitting unauthorized access to sensitive areas. Furthermore, deepfakes can undermine the integrity of information, images, or videos, by spreading false or misleading information about individuals or organizations, causing reputational harm. Finally, deepfakes can also disrupt availability by disseminating propaganda or fake news, leading to confusion and chaos. 
% The CIA Triad refers to the three primary goals of information security: confidentiality, integrity, and availability. Confidentiality ensures that sensitive information is only accessible to authorized parties. Integrity ensures that information remains unaltered and accurate. Availability ensures that information is accessible to authorized parties when needed. The CIA Triad is a widely recognized model for information security developed by the National Institute of Standards and Technology (NIST).
% Since deepfakes in the metaverse are digital manipulations that can be used to alter, fabricate, or impersonate someone's identity. These deepfakes can be a threat to the CIA Triad's goals of confidentiality, integrity, and availability in several ways.

% Confidentiality: Deepfakes can be used to steal sensitive information or access restricted areas by impersonating authorized individuals, thereby compromising confidentiality. For example, if someone creates a deepfake of a security officer and uses it to gain access to a secure area, it can be a significant security threat.

% Integrity: Deepfakes can be used to alter information, images, or videos, thereby compromising integrity. For instance, a deepfake can be created to spread false information about a company or individual, leading to reputational harm.

% Availability: Deepfakes can also be used to deny access to information or systems, leading to availability issues. For example, a deepfake can be used to create fake news or propaganda, leading to confusion and chaos.


\noindent
\textbf{Legal and Regulatory Challenges. } The lack of regulations regarding the application of laws from the physical world in the metaverse presents a significant challenge. For instance, identifying and prosecuting an offender who has committed a crime in the virtual world using deepfake impersonation can be difficult. Additionally, jurisdictional issues arise due to the existence of varying laws in different countries. In the event that an attacker is located in a country where there are no legal consequences for their actions in the metaverse, holding them accountable can become problematic. Consequently, a universal set of rules and regulations for the metaverse becomes difficult to establish, given that different countries may have different interpretations of what constitutes criminal behavior. To address security concerns in the metaverse, a coordinated effort between governments, regulatory bodies, and technology companies is necessary~\cite{Deepfakes_In_Metaverse_Legal_Issues}. This entails the development of universally applicable standards and regulations that can transcend geographical and jurisdictional barriers. Also, continuous efforts toward the development of new technologies capable of preventing and detecting criminal activities in the metaverse are also required. 
% The lack of regulations on how laws from the physical world will be applied in the metaverse. For example, if someone commits a crime in the virtual world using deepfake impersonation, it can be challenging to identify and prosecute the offender.

% Moreover, different countries have different laws, which can create jurisdictional issues. If an attacker is located in a country with no legal consequences for their actions in the metaverse, it can be challenging to hold them accountable for their actions. This can make it difficult to establish a universal set of rules and regulations for the metaverse, as different countries may have different interpretations of what constitutes criminal behavior.

% Another issue is the potential for cyberattacks and data breaches. As the metaverse becomes more popular, it will likely become a target for cybercriminals seeking to exploit vulnerabilities in the system. This could lead to the theft of personal information, financial fraud, and other types of cybercrime.

% Overall, addressing security-related scenarios in the metaverse will require a coordinated effort between governments, regulatory bodies, and technology companies. This will involve developing a set of standards and regulations that can be applied universally, regardless of location or jurisdiction. It will also require ongoing efforts to develop new technologies that can help to prevent and detect criminal activities in the metaverse.

% Finally, the paper examines the legal and regulatory challenges in addressing security-related scenarios in the metaverse. There are currently no regulations on how laws from the physical world will be applied if someone commits a crime in the virtual world. The paper also discusses how different countries have different laws, which can make it difficult to prosecute an attacker located in a country with no legal consequences for their actions in the metaverse.

% \sharif{Found 3 interesting threats}

\noindent
\textbf{Privacy issues. } Although digital identity verification systems can serve as a potential measure for mitigating deepfake-based impersonation in the metaverse, the utilization of these systems raises concerns. The apprehension stems from the possibility of digital identity verification systems being utilized to track and monitor individuals' virtual activities, consequently, potentially infringing on their privacy and freedom. It is argued that any digital identity verification system deployed in the metaverse must maintain a balance between security and the need for privacy and freedom.

% As the metaverse expands, privacy concerns are bound to arise. Meta space publishers will have complete control over all aspects of their virtual worlds, including collecting vast amounts of user data and monetizing it. Even open-source metaverse worlds hosted by users will still allow publishers to collect and monetize user data.

% though digital identity verification systems can help prevent deepfake based impersonation in the metaverse. there are also concerns about the use of digital identity verification systems in the metaverse. Some people argue that these systems could be used to track and monitor people's activities in the virtual world, potentially infringing on their privacy and freedom. Therefore, any digital identity verification system used in the metaverse would need to balance the need for security with the need for privacy and freedom.

\noindent
\textbf{Darkverse. }
The rise of metaverse technology has created new opportunities for both legitimate users and malicious actors. One of the primary concerns is the creation of private spaces that enable illegal activities and communication among criminals, which Trend Micro refers to as the \textit{darkverse}~\cite{darkverse}. This space operates similarly to the dark web, but it exists within the metaverse and is unindexed, making it challenging to locate via standard search engines. The darkverse's pseudo-physical user presence makes it more dangerous than the dark web, as criminals can use proximity-based messaging or other methods to conceal their communications, rendering them difficult for law enforcement agencies to intercept. Darkverse could be used to facilitate illegal activities such as deepfake-based revenge pornography and misinformation campaigns. %Authentication tokens are used for access to underground marketplaces, which further complicates law enforcement's efforts.
Despite the possibility of the darkverse being a space for free speech, the primary objective of these spaces is to facilitate illegal activities, and it may become a safe haven for criminals seeking to engage in such activities with minimal risk of detection.
% The rise of metaverse technology is opening up new avenues for malicious actors, as well as legitimate users. One of the major concerns is the creation of private spaces that facilitate illegal activities or communication among criminals. Trend Micro defines these areas as the darkverse. It is a space that mimics clandestine physical meetings and is similar to the dark web. However, it exists inside the metaverse and is unindexed, making it difficult to search for by standard search engines. darkverse could be used to facilitate illegal activities, revenge porn, and misinformation campaigns. The pseudo-physical presence of users in the darkverse makes it more dangerous than the dark web. Criminals can use proximity-based messaging or other methods to hide their communications, making it difficult for law enforcement agencies to intercept them. The use of authentication tokens for access to underground marketplaces further complicates the situation for law enforcement. The darkverse could become a safe haven for criminals who wish to conduct illegal activities without fear of being caught. However, the darkverse could also be used for free speech against oppressive entities or governments. It could be a space where individuals can voice their opinions without fear of censorship or retribution. While this is a positive aspect of the darkverse, it is important to remember that the primary purpose of these spaces is to facilitate illegal activities.

% The darkverse is a sinister corner of the metaverse that is akin to the infamous dark web. Unlike the open discussion threads on the dark web, the darkverse allows users to simulate clandestine physical meetings in a pseudo-physical environment, making it even more dangerous. Moreover, it resides within the deepverse, which is unindexed, just like the deep web.

% \noindent
% \textbf{Financial fraud. } With the vast volume of e-commerce transactions that will take place in the metaverse, it is likely to attract criminals and criminal groups seeking to exploit unsuspecting users, steal their money, and seize their digital assets.






\section{Conclusion}
In conclusion, deepfakes in the metaverse present significant security implications, particularly around impersonation. The three scenarios of gaming, online meetings, and virtual offices serve as examples of how these security implications can play out in practice. The lack of physical authentication in the metaverse makes it easier for attackers to impersonate others and commit crimes without being held accountable. Mitigating these security implications will require a combination of technological solutions and legal frameworks that balance security and privacy concerns. As the metaverse continues to evolve, it is important to address these issues proactively to ensure a safe and secure virtual environment for all users.

\bibliographystyle{ACM-Reference-Format}
\bibliography{ref}
\end{document}
\endinput
%%
%% End of file `sample-sigconf.tex'.
