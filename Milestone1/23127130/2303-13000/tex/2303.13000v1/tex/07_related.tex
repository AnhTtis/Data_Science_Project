\section{Related Works}
This section describes the existing literature on intermittent computing, scheduling battery-free systems, and distributed sensor systems.

\parlabel{Intermittent Computing}
Due to the stochastic nature and the lack of adequate power from the energy sources, batter-free systems experience frequent power failures. These failures result in the re-execution of instructions and inconsistency in non-volatile memory. For reliable forward execution of instructions despite power failure, previous works use software check-pointing~\cite{ransford2012mementos, maeng2018adaptive, hicks2017clank, lucia2015simpler, van2016intermittent, jayakumar2014quickrecall, mirhoseini2013automated, bhatti2016efficient}, hardware interruption~\cite{balsamo2015hibernus, balsamo2016hibernus++, mirhoseini2013idetic}, atomic task-based model~\cite{maeng2017alpaca, colin2016chain, colin2018termination}, and non-volatile processors(NVP)~\cite{ma2017incidental, ma2015architecture}. Some recent works have explicitly focused on inferring deep neural networks in batteryless systems by combining atomic task-based models with loop continuation~\cite{gobieskiintermittent, gobieski2019intelligence} or introducing imprecise computing~\cite{islam2020zygarde}. We utilize the SONIC~\cite{gobieskiintermittent, gobieski2019intelligence} task-based execution model to implement deep neural network inference. Recent research on intermittent computing focuses on formally verifying the systems~\cite{bohrer2022batfly} and designing developer-friendly platforms~\cite{bakar2022protean}. However, none of these works focus on validating whether the task can be finished or have considered multiple intermittent nodes. 

\parlabel{Scheduling in Intermittent Computing Systems}
Prior works have explored optimum voltage calculation~\cite{buettner2011dewdrop}, execution rate adaptation~\cite{sorber2007eon, dudani2002energy}, and stale data discarding to increase task completion likelihood~\cite{hester2017timely}. Some recent works have proposed scheduling algorithms for intermittent computing tasks to address the missing part of deadline-aware execution~\cite{islam2020zygarde, islam2020scheduling, yildirim2018ink}. None of these works consider the potential of scheduling multiple nodes to utilize their collaborative potential fully. Recently some works for intermittent sensor systems use reinforcement learning to increase the performance of intermittent computing nodes~\cite{fraternali2020aces, smarton}. None of these works consider the potential of using multiple nodes. 

\parlabel{Distributed Sensor Systems}
Though literature exists on distributed wireless sensors~\cite{zheng2013survey, zhu2019broadcast}, these algorithms cannot be directly applied for collaborative intermittently-powered systems as they are assumed to have communication among them. Communication is the most energy-expensive operation, and such frequent communication lowers the utilization of the system. 

Though recent works~\cite{torrisi2020zero, geissdoerfer2021bootstrapping} have shown the possibility of using zero-power backscatter communication, they can only communicate with one node at a time, thus not sufficient for emulating always-on scenarios. Recent work~\cite{geissdoerfer2021bootstrapping} uses external catalysts like light flickering, which is not generalizable. 
Moreover, active radio requires the transmitting and receiving battery-free node to turn on the radio of both devices at the same time. Recent works~\cite{geissdoerfer2022learning} have proposed wake-up schedules and neighbor discovery for effective communication in an intermittent computing system. However, they only support communication with one node and a known energy pattern. For a swarm of identical nodes to synchronize, multiple-node support is necessary, which involves the recovery of conflicting packets. 
Our closest related work, CIS~\cite{majid2020continuous}, enables continuous sensing with distributed battery-free devices. Unlike \Sys, CIS only considers the scenario where the energy available to every node is the same.

