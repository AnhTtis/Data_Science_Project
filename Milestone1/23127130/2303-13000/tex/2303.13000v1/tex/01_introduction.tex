\section{Introduction}

Harvesting ambient energy (solar, kinetic, thermal, and radio frequency) is considered a sustainable alternative to batteries~\cite{sudevalayam2010energy, merrett2016energy} for powering the next trillion Internet of Things (IoT)~\cite{sparks2017route}. However, ambient energy is often insufficient to power these devices directly due to lower energy availability than energy requirements. Hence, a battery-free system accumulates adequate energy in an energy buffer (e.g., super-capacitor) before operating. As a result, these devices go through frequent power-on and power-off periods and experience intermittency during execution. Thus, these systems are called \textit{intermittent systems}.

The intermittent system's advancement has occurred chiefly in programming language and system architecture, where the focus lies on the progress of instruction execution~\cite{balsamo2015hibernus, gobieski2019intelligence, yildirim2018ink}, memory consistency~\cite{hardin2018application}, and the development of emulators and debuggers~\cite{colin2016energy, colin2017energy}. Recent works have concentrated on intermittent systems' timing aspect to track time~\cite{dereliable, hester2016persistent, rahmati2012tardis}, finish data processing within a deadline~\cite{islam2020scheduling, islam2020zygarde}, discard stale data~\cite{hester2017timely, yildirim2018ink}, and increase energy buffer~\cite{colin2018reconfigurable}. Despite these commendable efforts, current intermittent nodes fail to sense or process events during a power failure, limiting their potential in continuous monitoring and fault-intolerant application domains. Another body of work~\cite{bhatti2014sensors} focuses on efficient energy harvesting and distribution techniques that provide continuous energy to wireless nodes using beamforming from a wall-powered source. However, such methods only apply when we have complete control over the energy source, which is often false. 

Three primary factors make event sensing and timely execution in a battery-free node challenging -- (1) \textit{dynamic available energy}, (2) \textit{asynchronous energy and event source}, and (3) \textit{sporadic occurrence of rapidly varying events}. 
Recent works have focused on formulating predictability~\cite{islam2020zygarde}, determining the optimal duty cycle for maximum wake-up time for a single node~\cite{fraternali2020aces}, and learning the distribution of an event's occurrence~\cite{smarton,luo2019spoton} to reduce the uncertainty of dynamic energy. However, these works have yet to have solutions against missing events during long power-off periods, where events refer to the occurrence of a specific phenomenon, e.g., bell ringing.

When the energy source and event are the same (e.g., solar-powered UV ray monitor), events only occur when power is available, guaranteeing the system's availability to capture even sporadic events. These are passive event sensing, where the presence of a source is the external trigger. However, the same can only be said for independent energy and target event source when they are aligned in time (e.g., solar-powered acoustic event detector)~\cite{bashima2019ondevice,lee2019intermittent}. These are active sensing systems that sense and compute through pooling. This asynchronous wake-up and event occurrence causes failure in sporadic events observation.

The practical significance of this loss of observability is dependent on what the sensors intend to sense. While a low sampling rate (1 sample/minute) is sufficient for indoor temperature monitoring, monitoring stochastic event-based data or rapidly varying environmental stimuli (e.g., sound event detection) requires more observation. As both dynamic energy and sporadic event are uncontrollable, despite all the efforts, battery-free nodes fail to sense and compute all samples to determine the events when the energy and event sources are uncorrelated.

Our key idea to decrease missing samples and unfinished computation is straightforward. Imagine a scenario where we aim to determine an alarm or bell ringing or what appliance is active in a room. The system uses a microphone to capture audio clips and computes on-device to determine the sound source with classification. Depending on the source type, it sends a notification.
A battery-powered system is always on and thus can capture data and identify when the bell rang at time unit 2 (Figure~\ref{fig:teaser}a). However, an intermittent system that activates once every three seconds misses capturing the event (Figure~\ref{fig:teaser}b). Even if a battery-free node captures the sample, it may fail to finish processing it in time, resulting in an undetected event. Multiple identical nodes are also redundant because they will all be active simultaneously (Figure~\ref{fig:teaser}c).

The opportunity lies in that only one of the many intermittent systems must capture and process the event successfully to identify if the bell rang. Thus, scheduling the wake-up and sleep cycle can amalgamate three identical intermittent systems with only one active node at any time. As a result, an amalgamated intermittent system captures and processes all the samples and can identify the bell ring at time unit 2 in Figure~\ref{fig:teaser}d. In summary, with sufficient intermittent systems working as an amalgamated system, we can emulate a battery-powered node when the power-off period does not exceed the energy buffer self-discharging time.

Though scheduling a swarm of nodes to work collaboratively is not a new problem~\cite{zheng2013survey, zhu2019broadcast}, these algorithms rely on active communication among the nodes. Frequent communication among multiple nodes is infeasible in intermittent computing systems, as communication is often the most power-hungry operation~\cite{gobieski2019intelligence}. Though zero-power passive communication~\cite{torrisi2020zero, geissdoerfer2021bootstrapping} is promising, these methods can communicate with only one node at a time, increasing communication time among all the nodes in a swarm. Moreover, while passive backscatter communication depends on an active receiver~\cite{torrisi2020zero} or transceivers~\cite{geissdoerfer2021bootstrapping}, they require precise (milliseconds or even nanoseconds) time synchronization to allow the sender and receiver to turn on the radio simultaneously. Bonito~\cite{geissdoerfer2022learning} proposes a learned wake-up schedule for effective communication, and Flync~\cite{geissdoerfer2021bootstrapping} provides an effective solution for neighbor discovery despite intermittency. However, these works only support one device and often require prior knowledge about the energy pattern. Moreover, required packet conflict recovery makes them insufficient for frequent communication across a swarm of systems. 

We propose \Sys, a framework that amalgamates a swarm of intermittent nodes for prolonging the collective power-on period without inter-node communication. \Sys allows each intermittent system to decide whether to wake up or go to sleep by predicting other nodes' behavior without communication. To increase successful capture and computing and maximize energy utilization, \Sys reduces the number of simultaneously active nodes when the power-off period is less than the self-discharging time.

Coalesced Intermittent Sensor (CIS)~\cite{majid2020continuous} is our closest related work which randomly wakes up each intermittent system to avoid concurrent active systems. However, it only considers the scenario where the same energy is available to every node of the swarm. Despite being an important scenario, it fails to provide a solution for a broad spectrum of stochastic scenarios that can happen in real-world. \Sys primarily focuses on the dynamic scenario where the available energy to each node is different. This scenario is more common as different swarm nodes will have access to variable amounts of energy from the same source due to their placement, distances, and other environmental factors, e.g., occlusion. 

Through \Sys, we make three technical contributions:

\begin{itemize} 
    \item We propose an algorithm (Prime-Co-Prime) that determines tailored sleep and wake-up duty cycles for each intermittent node using the relative energy distribution when the energy available to each system is static over time. However, each system has access to a different amount of energy. We also prove that the proposed algorithm requires a minimum number of swarm nodes.

    \item When the available energy varies over time and is also different at each node, we provide an online tailored duty cycle selection method by formulating a Decentralized Partially Observable Markov Decision Process (Dec-POMDP). Due to the high computation overhead to solve a Dec-POMDP, we extend the Prime-Co-Prime algorithm to define duty cycles as states and provide energy-efficient heuristics to allow offline transition among these states based on the measured energy at each node. These states provide the following action by choosing the correct and diligent duty cycles.  
\end{itemize}

To understand the feasibility of our approach, we develop solar and RF-powered custom prototypes with an MSP430FR5994~\cite{msp430} microcontroller and controllable cascading capacitor array. We deploy a deep learning-based acoustic event detector that can determine the occurrence of events and classify them. We evaluate with real-world and simulation-based experiments and compare \Sys with an oracle, a duty cycle approach, a reinforcement learning-based approach (ACES~\cite{fraternali2020aces}), and a greedy approach. \Sys, on average, achieves 54.40\%, and 35.73\% higher capture and process success rate than a swarm of nodes with greedy, and ACES, respectively. In addition, \Sys achieves a 41.17\% higher capture and process success rate and spends 69.7\% less time with multiple active nodes than a swarm of greedy battery-free nodes in real-world experiments.