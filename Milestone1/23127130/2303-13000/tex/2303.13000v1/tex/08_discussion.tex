\section{Discussion}

\parlabel{Handling Different Events}
Though this paper assumes that all nodes capture all events equally (global), many real-world events are sensed differently at each node (local). For example, all swarm nodes in a building can hear a fire alarm, which is a global event. On the contrary, only some of the sensors around a building can sense when a person is breathing irregularly. Note that only global events are in the scope of this paper. 

When events are local, a subset of intermittent nodes can sense the target event. This subset selection can be formulated as a wireless sensor network formation problem, such as a topology-based network with cluster-based formation. Here the cluster formation is done based on the predetermined subset of the nodes for the event type. A single node can belong to multiple clusters, and after formulating the clusters, our proposed Falinks algorithms apply to each cluster. 

It is more beneficial to select the nodes belonging to the maximum number of clusters to select the minimum number of nodes while having at least one node from each cluster. In summary, along with the Prime-Co-Prime algorithm, which uses the node with the highest energy harvesting rate, using the node that belongs to most clusters more frequently is effective.

\noindent\textbf{Why not compare with Coalesced Intermittent Sensor (\textbf{CIS})?}
CIS~\cite{majid2020continuous} only works for scenarios where the target energy is variable but equal for every node. Here, each node keeps track of the energy at the other nodes and then calculates the number of active nodes to randomly chooses whether to sleep or wake up. We do not compare \Sys with CIS, as it fails to precisely count active nodes when the available energy at each node varies, and \Sys exclusively focuses on scenarios where available energy is different at each node. 

\parlabel{Dependency on Super-capacitor Discharging}
Most battery-free nodes use supercapacitors as energy buffers. Unlike batteries, supercapacitors have high charging and discharging rate. Therefore, even if a node does not utilize its energy greedily, it may still lose unused energy due to self-discharging. However, as other nodes are already active while a node is preserving energy, the swarm's performance remains unaffected. However, due to the sub-optimal heuristics, this energy loss might lower the performance when the available energy source varies across time and at every node. Therefore, a lower discharging rate is beneficial for our proposed algorithm.

\parlabel{Proof of Optimality of the Oracle Algorithm}
The following proof shows that -- "\textit{when the available energy at all nodes is known, waking up only the node with the maximum total harvestable and stored energy is optimal}."

\itparlabel{Proof} 
To prove this theorem using contradiction, we assume that an intermittent node $I_i$ ($i = 1, 2, ..., N$) has total energy $E_{i} = E_{H_i} + E_{C_i}$, where $E_{H_i}$ and $E_{C_i}$ are the available energy to harvest and energy stored for the $i^{th}$ intermittent node. Let $I_m$ be the node with maximum total energy, $E_m = max(E_i)$, for $\forall i \in N$. Let us assume that waking $I_m$ is not optimal. Thus, another node $I_p$ exists with $E_{min} < E_p < E_m$, and waking $I_p$ is optimal. Here, $E_{min}$ is the minimum operational energy. 

$E_m > E_p$ can happen if $E_{H_m} > E_{H_p}$ or $E_{C_m} > E_{C_p}$. If $E_{H_m} > E_{H_p}$ and we decide to wake only $I_p$ up while keeping all the other nodes asleep, ${E_p}' = E_p - E_{min} + E_{H_p}$ and ${E_m}' = E_m + E_{H_m}$. As  $E_{H_m} > E_{H_p}$ and ${E_m}' > {E_p}'$, $E_m$ will keep increasing and reach the maximum energy storage capacity. Then, $I_m$ will fail to harvest available energy, wasting potential energy. Similarly, when $E_{C_m} > E_{C_p}$, the energy storage of $I_m$ will fill up more quickly, wasting available energy. As wastage of potential energy is never optimal, waking up $I_p$ is not optimal, which contradicts our assumption. Therefore, we prove that waking $I_m$ up is optimal. \proved

\parlabel{Position of the Intermittent Nodes}
The placement of intermittent nodes is essential to avoid missing an event. A sensor swarm can miss sensing an event if (1) all of them are inactive or (2) the event is out of all nodes' sensing range. Thus the placement of intermittent nodes depends on three factors -- energy source, event source, and physical constraints. This paper focuses on finding the appropriate duty cycle of pre-positioned intermittent nodes instead of looking at placing the nodes. However, the PrimeCoPrime algorithm can be used to determine the placement with respect to the energy sources. The placement concerning the event source can be formed as the art gallery problem~\cite{abrahamsen2018art}, a well-studied visibility problem in computational geometry. Combining all three factors for determining a node's placement is an unsolved problem that we will address in future works.