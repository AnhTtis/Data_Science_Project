\section{Amalgamated Intermittent Computing Systems}
An amalgamated intermittent computing system (\Sys) is a swarm of $N$ intermittent nodes where $N>1$. Each node has the same group of sensors to capture data and performs the same computational steps to process them. Intermittent computing nodes go through wake-up and sleep cycles due to the unavailability of continuous and sufficient energy. However, the availability of energy guides these on-off cycles. In other words, whenever a node harvests sufficient energy to operate, it wakes up. A few previous works~\cite{islam2020scheduling} have shown the benefit of waking up only when there is sufficient energy. Instead of waking a node up whenever there is energy, \Sys provides duty cycles to each node to keep the other nodes asleep when one node is active. This way, one of the other nodes can be active after the active node consumes all its energy and increases the active time. For example, suppose a node has 1 second operating time with the available energy and requires two additional seconds to accumulate sufficient energy to wake up. In that case, the node has a duty cycle of 3 seconds. If two other nodes can be active during those two seconds, the active time of the system gets a three times increase. A fixed duty cycle is sufficient if all nodes have access to the same available energy (like CIS). However, in most real-world scenarios, available energy at nodes of a swarm varies, and a fixed duty cycle is insufficient. \Sys utilizes two tailored duty cycle selection methods described in Section~\ref{sec:algo}, and allows the nodes to amalgamate and increase their collective active time and processing capability. 

These duty cycles can be pre-defined during the compile time or can be selected from a pre-defined list of duty cycles during runtime. We found that whether compile time selection is sufficient or a runtime selection is needed depends on the available energy's characteristics. Thus, before exploring the proposed duty cycle selection method, we describe the energy characteristics and the assumptions we make in this paper. 

% \begin{figure*}[!htb]
% \begin{minipage}{0.45\textwidth}
% \centering
%     \includegraphics[width=\linewidth]{fig/scenario.pdf}
%     \caption{Different scenarios based on the energy availability}
%     \label{fig:scenarios}
% \end{minipage}
% \hspace{0.7em}
% \begin{minipage}{0.45\textwidth}
% \centering
%     \includegraphics[width=\linewidth]{fig/pcp.pdf}
%     \caption{Prime-Co-Prime (PCP) with the lowest allowed duty cycle of 3 and hyper-period 15.}
%     \label{fig:primecoprime}
% \end{minipage}
% % \vspace{-2em}
% \end{figure*}

\begin{figure}
    \centering
    \includegraphics[width=0.45\textwidth]{fig/scenario.pdf}
    \caption{Different scenarios based on the energy availability}
    \label{fig:scenarios}
\end{figure}

\subsection{Energy Characteristics}
\parlabel{Available Energy}
To simplify the analysis, we quantize the available energy harvest into discrete energy levels by dividing the total energy at each time slot by unit energy. The constant time slot is empirically determined using a task's shortest execution time and the lowest energy consumption.

\parlabel{Energy Availability}
Depending on the stochasticity of the energy source, its availability varies over time. When the available energy is constant in time, a node gets unchanging energy, making it easier for the node to make future decisions. In Figure~\ref{fig:scenarios} (left), we show that when a node is harvesting from the RF transmitter mounted on the roof and there is no changing interference, the available energy to a node is unchanged. On the contrary, Figure~\ref{fig:scenarios} (right) shows a mobile RF transmitter mounted on a robot where the available energy at each node varies depending on the changing distance between the node and transmitter. 

The ubiquity of the available energy to all the nodes in a swarm can also be constant or variable. Figure 1 shows the two scenarios where each node has access to a diverse energy. This paper focuses explicitly on these scenarios. A swarm of RF-powered nodes at equidistant and line-in-sight for an RF transmitter have access to the same energy in time. This is a more straightforward case than the previous one because, unlike the previous case, each node can easily calculate the energy status of the other nodes. Previous works~\cite{majid2020continuous} have focused on the second scenario, and thus we focus on the more explored scenario where the available energy at each node varies. 




\subsection{Assumptions}
We make the following assumptions in the paper. 
\begin{itemize}
    \item All the nodes are identical and have equal harvesting efficiency and energy consumption rate. Every node has the same components and executes the same procedure on the captured sample. This paper considers the worst-case scenario (lowest harvesting efficiency and highest consumption rate across all nodes) to ensure reliability. This assumption may result in suboptimal performance. However, our experiments found that the variation among five nodes with 100 iterations at each node is negligible. 

    \item The relative energy distribution among all the nodes is static and known as a priori. In other words, the nodes are not moving. 

    \item The scope of tasks is limited to those where all the nodes can capture any event. To illustrate, in the audio event detection in a room example in Section 1, all the nodes in various room positions can hear the same sound. On the contrary, tasks where the same data is not accessible to all nodes, e.g., vibration measurement at different walls, will be out of the scope of this paper. 

    \item We consider that the clocks of all nodes have the same start time. Over time these clocks can drift. Existing battery-free clocks have only $\pm$1 ms to -1 s of an error on 2.2 -- 2200 nF capacitors. The implementation of the proposed algorithm considers this drift and randomly adds counter-drifts to handle this. 
\end{itemize}








