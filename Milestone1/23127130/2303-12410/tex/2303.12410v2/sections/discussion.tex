%================================================================================
\section{Discussion}
%================================================================================

Embodied agents often solve tasks that are structured through the spatial, temporal, or permutation symmetries of our 3D world. Taking this structure into account in the design of planning algorithms can improve sample efficiency and generalization---notorious weaknesses of RL algorithms.

We introduced \eqd, an equivariant planning algorithm that operates as conditional sampling in a generative model.
The main innovation is a new diffusion model that is equivariant with respect to the symmetry group $\fullgroup$ of spatial, temporal, and object permutation symmetries.
Beyond this concrete architecture, our work presents a general blueprint for the construction of networks that are equivariant with respect to a product group and support multiple representations in the data.
Integrating this equivariant diffusion model into a planning algorithm allows us to model an invariant base density, but still solve non-invariant tasks through task-specific soft symmetry breaking. We demonstrated the performance, sample efficiency, and robustness of \eqd on object manipulation and navigation tasks.

While our work shows encouraging results, training and planning are currently expensive. Progress on this issue can come both from more efficient layers in the architecture of the denoising model as well as from switching to recent continuous-time diffusion methods with accelerated sampling.

%--------------------------------------------------------------------------------
\subsection*{Acknowledgements}
%--------------------------------------------------------------------------------

We would like to thank Gabriele Cesa, Daniel Dijkman, and Pietro Mazzaglia for helpful discussions.
