
\documentclass{article}
%\usepackage[english,russian]{babel}
\usepackage{hyperref}       % hyperlinks
\usepackage{url}            % simple URL typesetting
\usepackage{booktabs}       % professional-quality tables
\usepackage{amsfonts}       % blackboard math symbols
\usepackage{tablefootnote}
\usepackage{verbatim}
\usepackage{nicefrac}       % compact symbols for 1/2, etc.
\usepackage{microtype}      % microtypography
\usepackage{lipsum}
\usepackage{mathtools}

\usepackage{amsmath,amssymb}
\usepackage{algorithm,algorithmic}
\usepackage{pifont}
\usepackage{cases}
\usepackage{subcaption,graphicx}
\usepackage{stackengine}    % circled symbols
\usepackage{wrapfig}
\usepackage{enumitem}

%\newtheorem{theorem}{Theorem}[section]
%\newtheorem{corollary}[theorem]{Corollary}
%\newtheorem{lemma}[theorem]{Lemma}
\newtheorem{assumption}[theorem]{Assumption}
%\newtheorem{definition}[theorem]{Definition}
%\newtheorem{remark}[theorem]{Remark}
%\newtheorem{proposition}[theorem]{Proposition}

\newcommand*{\LargerCdot}{\raisebox{-0.25ex}{\scalebox{2.4}{$\cdot$}}}
\newcommand{\Sum}{\displaystyle\sum\limits}
\newcommand{\Max}{\max\limits}
\newcommand{\Min}{\min\limits}
\newcommand{\fromto}[3]{{#1}=\overline{{#2},\,{#3}}}
\newcommand{\floor}[1]{\left\lfloor{#1}\right\rfloor}
\newcommand{\ceil}[1]{\left\lceil{#1}\right\rceil}

\newcommand{\tild}{\widetilde}
\newcommand{\eps}{\varepsilon}
\newcommand{\lam}{\lambda}
\newcommand{\ol}{\overline}
\newcommand{\one}{\mathbf{1}}
\newcommand{\cset}{\mathcal{C}}
%\newcommand{\Breg}{\mathcal{D}_{h}}
%\newcommand{\PBreg}{\mathbb{D}_{h}}

%\newcommand{\EndProof}{\begin{flushright}$\square$\end{flushright}}

\newcommand{\circledOne}{\text{\ding{172}}}
\newcommand{\circledTwo}{\text{\ding{173}}}
\newcommand{\circledThree}{\text{\ding{174}}}
\newcommand{\circledFour}{\text{\ding{175}}}
\newcommand{\circledFive}{\text{\ding{176}}}
\newcommand{\circledSix}{\text{\ding{177}}}
\newcommand{\circledSeven}{\text{\ding{178}}}
\newcommand{\circledEight}{\text{\ding{179}}}
\newcommand{\circledNine}{\text{\ding{180}}}
\newcommand{\circledTen}{\text{\ding{181}}}
\newcommand{\balashstar}{\stackMath\mathbin{\stackinset{c}{0ex}{c}{0ex}{\text{\ding{83}}}{\bigcirc}}}
\renewcommand\balashstar{\stackMath\mathbin{\stackinset{c}{0ex}{c}{0ex}{\ast}{\bigcirc}}}


\renewcommand{\le}{\leqslant}
\renewcommand{\ge}{\geqslant}
\renewcommand{\hat}{\widehat}

\newcommand{\numberthis}{\addtocounter{equation}{1}\tag{\theequation}}


\DeclareMathOperator*{\argmin}{arg\,min}
\DeclareMathOperator*{\argmax}{arg\,max}
\DeclareMathOperator*{\Argmin}{Arg\,min}
\DeclareMathOperator*{\Argmax}{Arg\,max}
\DeclareMathOperator{\spn}{span}
\DeclareMathOperator{\kernel}{Ker}
\DeclareMathOperator{\image}{Im}
\DeclareMathOperator{\prox}{prox}
\DeclareMathOperator{\proj}{Proj}
\DeclareMathOperator{\col}{col}
\DeclareMathOperator{\diag}{diag}

\newcommand{\N}{\mathbb{N}}
\newcommand{\R}{\mathbb{R}}
\newcommand{\Z}{\mathbb{Z}}
\newcommand{\V}{\mathbb{V}}
\newcommand{\E}{\mathbb{E}}
%\newcommand{\P}{\mathbb{P}}
\newcommand{\I}{\mathbb{I}}
\newcommand{\F}{\mathbb{F}}

\newcommand{\mA}{{\bf A}}
\newcommand{\mB}{{\bf B}}
\newcommand{\mC}{{\bf C}}
\newcommand{\mD}{{\bf D}}
\newcommand{\mE}{{\bf E}}
\newcommand{\mF}{{\bf F}}
\newcommand{\mG}{{\bf G}}
\newcommand{\mH}{{\bf H}}
\newcommand{\mI}{{\bf I}}
\newcommand{\mJ}{{\bf J}}
\newcommand{\mK}{{\bf K}}
\newcommand{\mL}{{\bf L}}
\newcommand{\mM}{{\bf M}}
\newcommand{\mN}{{\bf N}}
\newcommand{\mO}{{\bf O}}
\newcommand{\mP}{{\bf P}}
\newcommand{\mQ}{{\bf Q}}
\newcommand{\mR}{{\bf R}}
\newcommand{\mS}{{\bf S}}
\newcommand{\mT}{{\bf T}}
\newcommand{\mU}{{\bf U}}
\newcommand{\mV}{{\bf V}}
\newcommand{\mW}{{\bf W}}
\newcommand{\mX}{{\bf X}}
\newcommand{\mY}{{\bf Y}}
\newcommand{\mZ}{{\bf Z}}

\newcommand{\cA}{{\mathcal{A}}}
\newcommand{\cB}{{\mathcal{B}}}
\newcommand{\cC}{{\mathcal{C}}}
\newcommand{\cD}{{\mathcal{D}}}
\newcommand{\cE}{{\mathcal{E}}}
\newcommand{\cF}{{\mathcal{F}}}
\newcommand{\cG}{{\mathcal{G}}}
\newcommand{\cH}{{\mathcal{H}}}
\newcommand{\cI}{{\mathcal{I}}}
\newcommand{\cJ}{{\mathcal{J}}}
\newcommand{\cK}{{\mathcal{K}}}
\newcommand{\cL}{{\mathcal{L}}}
\newcommand{\cM}{{\mathcal{M}}}
\newcommand{\cN}{{\mathcal{N}}}
\newcommand{\cO}{{\mathcal{O}}}
\newcommand{\cP}{{\mathcal{P}}}
\newcommand{\cQ}{{\mathcal{Q}}}
\newcommand{\cR}{{\mathcal{R}}}
\newcommand{\cS}{{\mathcal{S}}}
\newcommand{\cT}{{\mathcal{T}}}
\newcommand{\cU}{{\mathcal{U}}}
\newcommand{\cV}{{\mathcal{V}}}
\newcommand{\cW}{{\mathcal{W}}}
\newcommand{\cX}{{\mathcal{X}}}
\newcommand{\cY}{{\mathcal{Y}}}
\newcommand{\cZ}{{\mathcal{Z}}}

\newcommand{\ba}{{\bf a}}
\newcommand{\bb}{{\bf b}}
\newcommand{\bc}{{\bf c}}
\newcommand{\bd}{{\bf d}}
\newcommand{\be}{{\bf e}}
%\newcommand{\bf}{{\bf f}}
\newcommand{\bg}{{\bf g}}
\newcommand{\bh}{{\bf h}}
\newcommand{\bi}{{\bf i}}
\newcommand{\bj}{{\bf j}}
\newcommand{\bk}{{\bf k}}
\newcommand{\bl}{{\bf l}}
\newcommand{\bm}{{\bf m}}
\newcommand{\bn}{{\bf n}}
\newcommand{\bo}{{\bf o}}
\newcommand{\bp}{{\bf p}}
\newcommand{\bq}{{\bf q}}
\newcommand{\br}{{\bf r}}
\newcommand{\bs}{{\bf s}}
\newcommand{\bt}{{\bf t}}
\newcommand{\bu}{{\bf u}}
\newcommand{\bv}{{\bf v}}
\newcommand{\bw}{{\bf w}}
\newcommand{\bx}{{\bf x}}
\newcommand{\by}{{\bf y}}
\newcommand{\bz}{{\bf z}}

\newcommand{\ds}{\displaystyle}
\newcommand{\norm}[1]{\left\| #1 \right\|}
\newcommand{\normtwo}[1]{\left\| #1 \right\|_2}
\newcommand{\sqn}[1]{\norm{#1}_2^2}
\newcommand{\angles}[1]{\left\langle #1 \right\rangle}
\newcommand{\cbraces}[1]{\left( #1 \right)}
\newcommand{\sbraces}[1]{\left[ #1 \right]}
\newcommand{\braces}[1]{\left\{ #1 \right\}}
\def\<#1,#2>{\langle #1,#2\rangle}

\newcommand{\sigmamax}{\sigma_{\max}(\cA)}
\newcommand{\sigmamaxsqr}{\sigma_{\max}^2(\cA)}
\newcommand{\sigmaminplus}{\sigma_{\min}^+(\cA)}
\newcommand{\sigmaminplussqr}{(\sigma_{\min}^+(\cA))^2}

\usepackage[colorinlistoftodos,bordercolor=blue,backgroundcolor=blue!20,linecolor=blue,textsize=scriptsize]{todonotes}
\newcommand{\arogozin}[1]{\todo[inline]{{\textbf{Alexander R.:} \emph{#1}}}}
\newcommand{\schezhegov}[1]{\todo[inline]{{\textbf{Savelii C.:} \emph{#1}}}}


\title{\eqd: Equivariant Diffusion for Planning with Embodied Agents}

\author{Johann Brehmer\thanks{Equal contribution, order determined through a game of table tennis}\\
Qualcomm AI Research\thanks{Qualcomm AI Research is an initiative of Qualcomm Technologies, Inc.} \\
\texttt{jbrehmer@qti.qualcomm.com}
\And
Joey Bose$^*$\,\thanks{Work done during an internship at Qualcomm AI Research}\\
Qualcomm AI Research \\
Mila, McGill University
\And
Pim de Haan\\
Qualcomm AI Research\\
\And
Taco Cohen\\
Qualcomm AI Research
}

\iclrfinalcopy
\begin{document}

\maketitle

\begin{abstract}
Embodied agents operate in a structured world, often solving tasks with spatial, temporal, and permutation symmetries. Most algorithms for planning and model-based reinforcement learning (MBRL) do not take this rich geometric structure into account, leading to sample inefficiency and poor generalization.
We introduce the Equivariant Diffuser for Generating Interactions (\eqd), an algorithm for MBRL and planning that is equivariant with respect to the product of the spatial symmetry group $\sethree$, the discrete-time translation group $\timetranslation$, and the object permutation group $\permutation$.
\eqd follows the Diffuser framework \citep{janner2022planning} in treating both learning a world model and planning in it as a conditional generative modeling problem, training a diffusion model on an offline trajectory dataset.
We introduce a new $\fullgroup$-equivariant diffusion model that supports  multiple representations. We integrate this model in a planning loop, where conditioning and classifier-based guidance allow us to softly break the symmetry for specific tasks as needed.
On navigation and object manipulation tasks, \eqd improves sample efficiency and generalization.
\end{abstract}

%================================================================================

\section{Introduction}
\label{sec:introduction}
% \begin{itemize}
%     % Diffusion of FL
%     \item {\st{Diffusion of FL}}
%     % Security threats to FL
%     \item {\st{Security threats to FL with particular focus on model poisoning}}
%     % Limitations of existing countermeasures
%     \item {\st{Current countermeasures (e.g., KRUM) and their limitations}}
%     % Proposed method and its advantages
%     \item {\st{Intuitive description of the proposed method and its difference (i.e., advantages) w.r.t. state of the art}}
%     % Main contributions
%     \item {\st{Summary of the main contributions of this work}}
%     % Paper's structure and organization
%     \item {\st{Paper's structure and organization}}
% \end{itemize}

% Diffusion of FL
Recently, {\em federated learning} (FL) has emerged as the leading paradigm for training distributed, large-scale, and privacy-preserving machine learning (ML) systems~\cite{mcmahan2017googleai,mcmahan2017aistats}. 
The core idea of FL is to allow multiple edge clients to collaboratively train a shared, global model without disclosing their local private training data.
%Specifically, an FL system consists of a central server and many edge clients; 
A typical FL round involves the following steps: {\em(i)} the server randomly picks some clients and sends them the current, global model; {\em(ii)} each selected client locally trains its model with its own private data; then, it sends the resulting local model to the server;\footnote{Whenever we refer to global/local model, we mean global/local model {\em parameters}.} {\em(iii)} the server updates the global model by computing an \emph{aggregation function}, usually the average (FedAvg), on the local models received from clients.
% \begin{enumerate}
%     \item[{\em(i)}] the server sends the current, global model to the clients and appoints some of them for training;
%     \item[{\em(ii)}] each selected client locally trains its copy of the global model with its own private data; then, it sends the resulting local model back to the server;\footnote{Whenever we refer to global/local model, we mean global/local model {\em parameters}.}
%     \item[{\em(iii)}] the server updates the global model by computing an \emph{aggregation function} on the local models received from clients (by default, the average, also referred to as FedAvg~\cite{mcmahan2017aistats}).
% \end{enumerate}
This process goes on until the global model converges. %(e.g., after a certain number of rounds or other similar stopping criteria).
%\\
% The advantages of FL over the traditional, centralized learning paradigm are undoubtedly clear in terms of flexibility/scalability (clients can join/disconnect from the FL network dynamically), network communications (only model weights\footnote{We will use \textit{parameters} and \textit{weights} interchangeably.} are exchanged between clients and server), and privacy (each client's private training data is kept local at the client's end and not uploaded to the server).
\\
% Security threats to FL
%However, the growing adoption of FL also raises security concerns~\cite{costa2022covert}, particularly about its confidentiality, integrity, and availability.
Although its advantages over standard ML, FL also raises security concerns~\cite{costa2022covert}. %, particularly about its confidentiality, integrity, and availability~\cite{costa2022covert}.
% OLD, LONG VERSION
% Indeed, some work deals with privacy leakage that may expose the local data of some clients~\cite{melis2019sp}. 
% A large body of work, instead, investigates attacks that usually aim to detriment the predictive accuracy of the learned global model. For instance, \emph{data poisoning} attacks achieve this goal by letting an adversary pollute the training set of some corrupt FL clients with maliciously crafted examples~\cite{jagielski2018sp}.
% Similarly, in \emph{model poisoning} the attacker attempts to tweak the global model weights~\cite{bhagoji2019pmlr} by directly perturbing the local model's weights of some infected FL clients before these are sent to the central server for aggregation, usually via so-called Byzantine attacks. 
% It turns out that Byzantine model poisoning attacks severely impact standard FedAvg; therefore, more robust aggregation functions must be designed to make FL systems secure.
Here, we focus on \emph{untargeted model poisoning} attacks~\cite{bhagoji2019pmlr}, where an adversary attempts to tweak the global model weights %\footnote{We will use the terms \textit{parameters} and \textit{weights} interchangeably.} 
by directly perturbing the local model's parameters of some infected clients before these are sent to the central server for aggregation.
In doing so, the adversary aims to jeopardize the global model \textit{indiscriminately} at inference time.
Such model poisoning attacks severely impact standard FedAvg; therefore, more robust aggregation functions must be designed to secure FL systems.
\\
% In this paper, we focus on designing a novel robust aggregation scheme at the server's end to contrast the effect of Byzantine model poisoning attacks.
%
% Current countermeasures and their limitations
%Several countermeasures have been proposed in the literature to combat model poisoning attacks on FL systems.
% Some methods use simple statistics more robust than plain average to smooth the impact of malicious updates (e.g., Trimmed Mean and FedMedian~\cite{yin2018icml}). 
% Other defenses implement outlier detection techniques to discard malicious updates from the aggregation performed at the server's end. Those are either based on heuristics (e.g., Krum/Multi-Krum~\cite{blanchard2017nips} and Bulyan~\cite{mhamdi2018pmlr}) or data-driven approaches (e.g., K-means clustering~\cite{shen2016acm} or DnC via spectral analysis~\cite{shejwalkar2021ndss}). 
% Finally, some strategies rely on a centralized ``source of trust'' to spot potential malicious updates (e.g., FLTrust~\cite{cao2020fltrust}).
% Several countermeasures have been proposed in the literature to combat model poisoning attacks on FL systems, i.e., to discard possible malicious local updates from the aggregation performed at the server's end. 
% These techniques range from simple statistics more robust than plain average (e.g., Trimmed Mean and FedMedian~\cite{yin2018icml}) to outlier detection heuristics (e.g., Krum/Multi-Krum~\cite{blanchard2017nips} and Bulyan~\cite{mhamdi2018pmlr}) or data-driven approaches (e.g., spectral analysis via K-means clustering~\cite{shen2016acm} or spectral analysis), or methods based on ``source of trust'' (e.g., FLTrust~\cite{cao2020fltrust}).
% OLD, LONG VERSION
%Several countermeasures have been proposed in the literature to combat Byzantine model poisoning attacks on FL systems.
% Descriptive statistics
% For example, Trimmed Mean and FedMedian aggregate local model updates using more robust statistics than standard average~\cite{yin2018icml}.
%
% % Heuristics for outlier detection
% Many existing Byzantine-resilient strategies implement some outlier detection heuristics to discard the model updates sent by potentially malicious clients from the input of the aggregation function.
% One of the most popular heuristics is Krum~\cite{blanchard2017nips}.
% This strategy tries to mitigate the impact of Byzantine attacks by selecting as a global model the local model with the smallest sum of Euclidean distances to {\em all} the other local models.
% Although powerful, Krum requires the server to know (or, at least, estimate) the number of malicious FL clients upfront, which is generally impossible in a realistic attack scenario. %
% Moreover, Krum may become ineffective for complex, high-dimensional model parameter spaces due to the curse of dimensionality.
% Bulyan~\cite{mhamdi2018pmlr} tries to overcome this issue by combining Krum with a variant of Trimmed Mean.
% % Data-driven outlier detection
% Other strategies use data-driven outlier detection techniques -- e.g., via K-means clustering~\cite{shen2016acm} -- to spot potential malicious local model updates. 
% %For instance, Shen et al. propose to cluster local model updates with K-means and thus identify outliers.
%
% % Other techniques
% As far as the server is concerned, any local model received can be from a potential malicious client. 
% FLTrust~\cite{cao2020fltrust} assumes the server acts as a client, i.e., trains a local model on an additional {\em trustworthy} dataset at the server's end and compares it against all the local models from other clients. 
% This way, the server can rely on some ``source of trust'' when discarding potentially malicious clients.
%\\
% Limitations of existing Byzantine-resilient strategies
Unfortunately, existing defense mechanisms either rely on simple heuristics (e.g., Trimmed Mean and FedMedian by~\cite{yin2018icml}) or need strong and unrealistic assumptions to work effectively (e.g., foreknowledge or estimation of the number of malicious clients in the FL system, as for Krum/Multi-Krum~\cite{blanchard2017nips} and Bulyan~\cite{mhamdi2018pmlr}, which, however, cannot exceed a fixed threshold).
Furthermore, outlier detection methods using K-means clustering~\cite{shen2016acm} or spectral analysis like DnC~\cite{shejwalkar2021ndss} do not directly consider the temporal evolution of local model updates received.
Finally, strategies like FLTrust~\cite{cao2020fltrust} require the server to collect its own dataset and act as a proper client, thereby altering the standard FL protocol.
\\
% OLD, LONG VERSION
% Overall, existing Byzantine-resilient strategies are either simple heuristics (e.g., FedMedian) or, if they are more complex, they rely on strong and unrealistic assumptions to work effectively (e.g., knowing the number of malicious clients in the FL system in advance, as for Krum and alike).
% Furthermore, data-driven outlier detection methods do not consider the temporary evolution of local model updates received (e.g., K-means clustering). 
% Finally, strategies like FLTrust requires the server to collect its own dataset and act as a proper client, thereby altering the standard FL protocol.
%
% Description of the proposed method
This work introduces a novel pre-aggregation \textit{filter} robust to untargeted model poisoning attacks. Notably, this filter $(i)$ operates without requiring prior knowledge or constraints on the number of malicious clients and $(ii)$ inherently integrates temporal dependencies. 
The FL server can employ this filter as a preprocessing step before applying \textit{any} aggregation function, be it standard like FedAvg or robust like Krum or Bulyan.
Specifically, we formulate the problem of identifying corrupted updates as a multidimensional (i.e., matrix-valued) time series anomaly detection task. 
The key idea is that legitimate local updates, resulting from well-calibrated iterative procedures like stochastic gradient descent (SGD) with an appropriate learning rate, show \textit{higher predictability} compared to malicious updates. This hypothesis stems from the fact that the sequence of gradients (thus, model parameters) observed during legitimate training exhibit regular patterns, as validated in Section~\ref{subsec:intuition}. %until convergence. 
%This regularity may be more pronounced for smooth convex loss functions, but it can still be captured within an appropriate time window, even for more complex and convoluted loss surfaces. 
%We provide evidence of this claim in Appendix~B, where we show that the average mutual information (i.e., ``predictability''), calculated over pairs of legitimate model updates sent at different FL rounds, is significantly higher than the corresponding computation for a malicious client.
\\
Inspired by the matrix autoregressive (MAR) framework for multidimensional time series forecasting~\cite{chen2021je}, we propose the FLANDERS ({\em \textbf{F}ederated \textbf{L}earning meets \textbf{AN}omaly \textbf{DE}tection for a \textbf{R}obust and \textbf{S}ecure}) filter.
The main advantages of FLANDERS over existing strategies like FLDetector~\cite{zhao2020multivariate} are its resilience to large-scale attacks, where $50\%$ or more FL participants are hostile, and the capability of working under realistic non-iid scenarios.
We attribute such a capability to two key factors: $(i)$ FLANDERS works without knowing a priori the ratio of corrupted clients, and $(ii)$ it embodies temporal dependencies between intra- and inter-client updates, quickly recognizing local model drifts caused by evil players. Below, we summarize our main contributions:

\begin{itemize}
\item[{\em(i)}]
We provide empirical evidence that the sequence of models sent by legitimate clients is more predictable than those of malicious participants performing untargeted model poisoning attacks.
\\
\item[{\em(ii)}] 
We introduce FLANDERS, the first pre-aggregation filter for FL robust to untargeted model poisoning based on multidimensional time series anomaly detection.
\\
\item[{\em(iii)}] 
We integrate FLANDERS into Flower,\footnote{\scriptsize{\url{https://flower.dev/}}} a popular FL simulation framework for reproducibility.
\\
\item[{\em(iv)}] 
We show that FLANDERS improves the robustness of the existing aggregation methods under multiple settings: different datasets, client's data distribution (non-iid), models, and attack scenarios.
\\
\item[{\em(v)}] 
We publicly release all the implementation code of FLANDERS along with our experiments.\footnote{\scriptsize{\url{https://anonymous.4open.science/r/flanders_exp-7EEB}}}
\end{itemize}

% Paper's structure and organization
The remainder of the paper is structured as follows. %some related work and the current state-of-the-art solutions to security issues that FL entails. 
Section~\ref{sec:background} covers background and preliminaries. 
In Section~\ref{sec:related}, we discuss related work.
Section~\ref{sec:problem} and Section~\ref{sec:method} describe the problem formulation and the method proposed. % to tackle it. 
Section~\ref{sec:experiments} gathers experimental results. %, and Section~\ref{sec:limitations} discusses some limitations of this work.
Finally, we conclude in Section~\ref{sec:conclusion}.
 %discusses the limitations of this work and draws future research directions.
%reports conclusions and draws perspectives for future research directions.

%%%%%%% OLD %%%%%%%
%to overcome the resilience of Byzantine failures in distributed Stochastic Gradient Descent computations. 
% The strength of Krum is its time complexity, which is linear in the gradient dimension. 
% However, the robustness of the approach is guaranteed for gradient-based learning applications only when the majority of the clients are not compromised. 
% Besides, the aggregation mechanism of Krum, as well as that of similar methods, is robust from a coarse-grained perspective and does not provide solutions to errors and perturbations that may occur at inference time.
%A related approach to~\cite{blanchard2017nips} is the work of Su et al.~\cite{su2016dc}. Here, the authors propose an iterated approximate agreement to tackle a multi-layer scenario attacked by Byzantine agents. 
%However, the method works efficiently on the sole discrete context and it is inapplicable to continuous state environments.
%\gabri{Maybe, we should just talk about the main limitations of existing countermeasures without digging into their details (or, we can just mention Krum as this is the most popular one). I will move the description of all these methods to the Related Work section.}
\section{Background on Network Calculus}
\label{sec: background}


\begin{figure*}[tbh]
\centering
\begin{subfigure}[b]{0.3\textwidth}
    \centering
    \includegraphics[width=\linewidth]{images/in-out.png}
    \caption{Arrival and departure data and their relation with delay $d(t)$ and backlog $b(t)$. For a FIFO system, the delay is the horizontal distance between $R(t)$ and $R^*(t)$ but some other multiplexing techniques may shift the data to a later priority, causing a longer delay.}
    \label{fig: data in-out}
\end{subfigure}
\hfill
\begin{subfigure}[b]{0.35\textwidth}
    \centering
    \includegraphics[width=\linewidth]{images/arrival-service.png}
    \caption{Characteristics of an arrival curve and a service curve. From any point of observation, the arriving data never exceeds its arrival curve; the departure data is also never less than the service curve with respect to the data arrival.}
    \label{fig: arrival-service curves}
\end{subfigure}
\hfill
\begin{subfigure}[b]{0.33\textwidth}
    \centering
    \includegraphics[width=\linewidth]{images/bound.png}
    \caption{Delay and backlog bounds of a system. Backlog is the maximum vertical distance between $\alpha(t)$ and $\beta(t)$; FIFO delay is their maximum horizontal distance; but for arbitrary multiplexing, the delay guarantee is when the system clears its buffer, thus it's the intersection of $\alpha(t)$ and $\beta(t)$.}
    \label{fig: system bounds}
\end{subfigure}
\caption{Network calculus framework. We let $R(t)$ and $R^*(t)$ be the arrival and departure data flow of a system; $\alpha(t)$ be the piecewise linear concave arrival curve and $\beta(t)$ be the piecewise linear convex service curve of a system.}
% \hossein{Better to show piece-wise linear concave arrival curve and piece-wise linear convex service curve instead of token-bucket and rate-latency.}}
\end{figure*}

We recall some of the network calculus essentials for a better understanding of the framework used in Saihu. In the following context, we use the following notation: $\mbb{R}^+$ is the set of non-negative real numbers; $[x]_+$ denotes $\max(0, x)$

The data flow is by convention modeled as a left-continuous wide-sense increasing function $R(t): \mbb{R}^+ \mapsto \mbb{R}^+$ with respect to time $t$~\cite{ncbook2001leboudec}. 

A system $\mcal{S}$ receives arrival data described as a cumulative function $R(t)$ and delivers departure data as another cumulative function $R^*(t)$. Figure~\ref{fig: data in-out} illustrates such a system $\mcal{S}$. The benefit of representing a system like this is that we can observe system backlog and delay with such a model. 

\begin{definition}[Backlog and Delay~\cite{ncbook2001leboudec}]
    The backlog of a system at time~$t$ is
    \begin{equation}
        b(t) = R(t) - R^*(t)
    \end{equation}
    
    The virtual delay of a FIFO system at time $t$ is
    \begin{equation}
        d_{FIFO}(t) = \inf \lbp \tau \geq 0 : R(t) \leq R^*(t+\tau) \rbp
    \end{equation}
\end{definition}



The backlog of a system can be viewed as the vertical distance between $R$ and $R^*$. The FIFO (\textit{First-in First-out}) delay is the horizontal distance between $R$ and $R^*$. One may obtain other delay values if the multiplexing technique is not FIFO.

% \begin{figure}
%     \centering
%     \includegraphics[width=0.9\linewidth]{images/in-out.png}
%     \caption{In/out data flow; delay and backlog}
%     \label{fig: data in-out}
% \end{figure}

Since we are interested in the system guarantee instead of a single instance of data flow, we would like to have general bounds to the arrival and departure data flows. Therefore, we define \textit{arrival curve} and \textit{service curve} as the bounds of arrival and departure data flows.

\begin{definition}[Arrival Curve~\cite{ncbook2001leboudec}]
    Given a wide-sense increasing function $\alpha: \mbb{R}^+ \mapsto \mbb{R}^+$, we say that a flow $R(t)$ is $\alpha$-constrained if and only if for all $s \leq t$:
    \begin{equation}
        R(t) - R(s) \leq \alpha(t-s)
    \end{equation}
    We say $R(t)$ has $\alpha$ as an arrival curve.
\end{definition}

\begin{definition}[Service Curve~\cite{ncbook2001leboudec}]
    Given a wide-sense increasing function $\beta: \mbb{R}^+ \mapsto \mbb{R}^+$ and $\beta(0) = 0$. A system $\mcal{S}$ having $R(t)$ and $R^*(t)$ as its arrival and departure flows. We say $\mcal{S}$ offers a service curve $\beta$ if and only if
    \begin{equation}
        R^*(t) \geq (R \otimes \beta)(t) =: \inf_{s \leq t} \lbp R(s) + \beta(t-s) \rbp
    \end{equation}
    where $\otimes$ denotes the min-plus convolution
\end{definition}

Figure~\ref{fig: arrival-service curves} illustrates the arrival and service curves. Any segment of arrival flow $R(t)$ is constrained by arrival curve $\alpha$ and the output curve $R^*(t)$ is always no less than the curve $R\otimes\beta$. As a result, an arrival curve upper bounds the incoming traffic, and a service curve lower bounds the outgoing traffic.

% \begin{figure}
%     \centering
%     \includegraphics[width=\linewidth]{images/arrival-service.png}
%     \caption{Arrival/Service curve}
%     \label{fig: arrival-service curves}
% \end{figure}

We consider 2 special types of curves throughout this paper, \textit{token-bucket} (or sometimes called \textit{leaky-bucket}) curve and \textit{rate-Latency} curve.

\begin{definition}[Token-bucket and Rate-latency~\cite{ncbook2001leboudec}]
    A token-bucket curve $\gamma_{r,b}$ with arrival rate $r$ and burst $b$ is defined as
    \begin{equation}
        \gamma_{r,b}(t) = b + rt
    \end{equation}

    A rate-latency curve $\beta_{R,T}$ with service rate $R$ and latency $T$ is defined as
    \begin{equation}
        \beta_{R,T}(t) = R \lb t - T \rb_+
    \end{equation}
\end{definition}

A token-bucket curve is determined by a burst $b$ and an arrival rate~$r$. Burst represents the maximum possible data volume that can arrive simultaneously, and arrival rate represents the maximum long-term data rate~\cite{bouillard2022tradeoff}.
A rate-latency curve is determined by a latency~$T$ and a service rate~$R$. Latency represents the time a server needs before starting to process the incoming data, and service rate represents the minimum rate to process data after the initial latency.

With the help of arrival and service curves, we can derive delay and backlog bounds for a system $\mcal{S}$ illustrated in Figure~\ref{fig: system bounds}. Suppose a system $\mcal{S}$ has arrival curve $\alpha$ and service curve~$\beta$, its worst-case backlog $b^*$ is the maximum vertical distance between~$\alpha$ and~$\beta$. Similarly, depending on the multiplexing technique applied to the system, its worst-case delay bound $d^*$ is the maximum horizontal distance between $\alpha$ and $\beta$ if $\mcal{S}$ is a FIFO system. If we don't have any information about its multiplexing technique, referred to as arbitrary multiplexing, the best we can say is that when $\alpha$ and $\beta$ intersect each other, where all data has been delivered out of the system. Consequently, the worst-case delay bound for arbitrary multiplexing is the time required for $\mcal{S}$ to clear its buffer.

% \begin{figure}
%     \centering
%     \includegraphics[width=\linewidth]{images/bound.png}
%     \caption{System delay/backlog bounds}
%     \label{fig: system bounds}
% \end{figure}

While a service curve captures the slowest possible output speed of a system, a link's transmission capacity limits the speed as well. Hence, we model this phenomenon using a \textit{greedy shaper} with a sub-additive function $\sigma: \mbb{R}^+ \mapsto \mbb{R}^+$ concatenated with a server. We consider a concatenation as shown in Figure \ref{fig: system}. By convention we assume $\sigma(0) = 0$ and $\beta(t) \leq \sigma(t), \forall t \in \mbb{R}^+$, meaning that the buffer is cleared at the beginning and the service never exceed its physical limitation. With the above definition, such greedy shaper conserves the service provided by the system due to theorem \ref{thm: shaping}.

\begin{figure}[thb]
    \centering
    \includegraphics[width=0.7\linewidth]{images/system.png}
    \caption{Shaping of departure data. A flow that has an arrival curve $\alpha$ feeds into a server with an arrival data flow $R(t)$. The server having service curve $\beta$ takes $R(t)$ and gives a departure data flow $R^*(t)$ to a shaper with shaping function $\sigma$. The shaper takes $R^*(t)$ and shape the data flow as another departure $D(t)$.}
    \label{fig: system}
\end{figure}


\begin{theorem}[Shaping conserves service \cite{ncbook2001leboudec}]
\label{thm: shaping}
Following the system shown in Figure \ref{fig: system}, we have
\begin{equation}
     D = R^* \otimes \sigma \geq \lp R \otimes \beta \rp \otimes \sigma = R \otimes \lp \beta \otimes \sigma \rp = R \otimes \beta
\end{equation}
\end{theorem}

In the following context, we model the shaping function $\sigma$ as a token-bucket curve $\gamma_{C,L}$ with transmission capacity $C$ and the packet size $L$ to capture the link capacity and packetization~\cite{bouillard2022tradeoff}.

%================================================================================
\section{Equivariant diffuser for generating interactions (\eqd)}
\vspace{-5pt}
\label{sec:equivariant_diffuser}


We now describe our \eqd method.
We begin by discussing the symmetry group $\fullgroup$ and common representations in robotic problems.
In Sec.~\ref{sec:eqd_diffusion} we introduce our key novelty, an $\fullgroup$-equivariant diffusion model for state-action trajectories $\tau$.
Finally, we show how a diffusion model trained on offline trajectory data can be used for planning in Sec.~\ref{sec:eqd_planning}.


%================================================================================
\subsection{Symmetry and representations}
%================================================================================

%--------------------------------------------------------------------------------
\xhdr{Symmetry group}
We consider the symmetry group $\fullgroup$, which is a product of three distinct groups: 1.\ the group of spatial translations and rotations $\sethree$, 2.\ the discrete time translation symmetry $\timetranslation$, and 3.\  the permutation group over $n$ objects $\permutation$. It is important to note, however, that this symmetry group may be softly broken in an environment. For instance, the direction of gravity usually breaks the spatial symmetry group $\sethree$ to the smaller group $\setwo$, and distinguishable objects in a scene may break permutation invariance. We follow the philosophy of modeling invariance with respect to the larger group and including any symmetry-breaking effects as inputs to the networks.

We require that spatial positions are always expressed relative to a reference point, for example, the robot base or center of mass. This guarantees equivariance with respect to spatial translations: to achieve $\sethree$ equivariance, we only need to design an $\sothree$-equivariant architecture.

%--------------------------------------------------------------------------------
\xhdr{Data representations}
We consider 3D environments that contain an embodied agent as well as $n$ other objects. We parameterize their degrees of freedom with two $\sothree$ representations, namely the scalar representation $\rho_0$ and the vector representation $\rho_1$. Any $\sethree$ pose can be transformed to these two representations, see Appendix~\ref{app:algo} for details.
We assume that all trajectories transform under the regular representation of the time translation group $\timetranslation$ (similar to how images transform under spatial translations).
Under $\permutation$, object properties permute, while robot properties or global properties of the state remain invariant. Each feature is thus either in the trivial or the standard representation of $\permutation$.

Overall, we thus expect that data in environments experienced by our embodied agent to be categorized into four representations of the symmetry group $\fullgroup$: scalar object properties, vector object properties, scalar robotic degrees of freedom (or other global properties of the system), and vector robotic degrees of freedom (again including other global properties of the system).

%================================================================================
\subsection{Equivariant diffusion model}
\label{sec:eqd_diffusion}
%================================================================================

%--------------------------------------------------------------------------------
\begin{figure*}[t]
    \centering
    \includegraphics[width=1.0\linewidth]{figures/equi_diffuser_architecture.pdf}
    \vskip-4pt
    \caption{Architecture of our $\fullgroup$-equivariant denoising network. Input trajectories (top left), which consist of features in different representations of the symmetry group, are first transformed into a single internal representation (green block). The data are then processed with equivariant blocks (blue), which consist of  convolutional layers along the time dimension, attention over objects, normalization layers, and geometric layers, which mix scalar and vector components of the internal representations. These blocks are combined into a U-net architecture.
    For simplicity, we leave out many details, including residual connections, downsampling, and upsampling layers; see Appendix~\ref{app:algo}.}
    \label{fig:architecture}
    \vspace{-12pt}
\end{figure*}
%--------------------------------------------------------------------------------

\looseness=-1
Our main contribution is a novel $\fullgroup$-equivariant diffusion model which leads to an \emph{invariant} distribution over trajectories. Specifically, given an invariant base density with respect to our chosen symmetry group---a Gaussian satisfies this property for $\fullgroup$---and an equivariant denoising model with respect to the same group we arrive at a diffusion model that is $\fullgroup$-invariant \citep{kohler2020equivariant, papamakarios2021normalizing}. Under mild assumptions, such an equivariant map that pushes forward the base density always exists~\citep{bose2021equivariant}.

We design a novel equivariant architecture for the denoising model $f$. Implemented as a neural network, it maps noisy input trajectories $\tau$ and a diffusion time step $i$ to an estimate $\hat{\epsilon}$ of the noise vector that generated the input.
Our architecture does this in three steps. First, the input trajectory consisting of various representations is transformed into an internal representation of the symmetry group. Second, in this representation the data are processed with an equivariant network. Finally, the outputs are transformed from the internal representation into the original data representations present in the trajectory.
We illustrate the architecture of our \eqd model in Fig.~\ref{fig:architecture}.

%--------------------------------------------------------------------------------
\looseness=-1
\xhdr{Step 1: Representation mixer}
The input noisy trajectory consists of features in different representations of the symmetry group (see above). While it is possible to mirror these input representations for the hidden states of the neural network, the design of equivariant architectures is substantially simplified if all inputs and outputs transform under a single representation. Hence, we decouple the data representation from the representation used internally for the computation---in a similar fashion to graph neural networks that decouple the data and computation graphs.

%--------------------------------------------------------------------------------
\xhdr{Internal representation}
We define a single internal representation that for each trajectory time step $t \in [H]$, each object $o \in [n]$, each channel $c \in [n_c]$ consists of an\footnote{Pairing up just one scalar and one vector is a design choice; for systems in which scalar or vectorial quantities play a larger role, it may be beneficial to use multiple copies of either representation here.} $\sothree$ scalar $s_{toc}$ and an $\sothree$ vector $v_{toc}$. We write $w_{toc} = (s_{toc}, v_{toc}) \in \R^4$.
Under spatial rotations $g \in \sothree$, these features thus transform as the direct sum of the scalar and vector representations $\rho_0 \oplus \rho_1$:
%
\begin{equation}
    w_{toc} = \m{s_{toc} \\ v_{toc}} \to w'_{toc} = \m{\rho_0(g)s_{toc} \\ \rho_1(g) v_{toc}} \,.
\end{equation}

These internal features transform in the regular representation under time shift and in the standard representation under permutations $\mathbb{P}$ as $w_{toc} \to w_{to'c} = \sum_o \mathbb{P}_{o'o} \, w_{toc}$. There are thus no global (not object-specific) properties in our internal representations.

%--------------------------------------------------------------------------------
\xhdr{Transforming input representations into internal representations}
The first layer in our network transforms the input $\tau$, which consists of features in different representations of $\fullgroup$, into the internal representation. On the one hand, we pair up $\sothree$ scalars and $\sothree$ vectors into $\rho_0 \oplus \rho_1$ features. On the other hand, we get rid of global features\,--\,those unassigned to one of the $n$ objects in the scene\,--\,by including them in the representation of each of the $n$ objects. 

Concretely, for each object $o \in [n]$, each trajectory step $t \in [H]$, and each channel $c =[n_c]$, we define the input in the internal representation as $w_{toc} \in \R^4$ as follows:
%
\begin{equation}
    w_{toc} =
    \m{
        \sum_{c'} \rmW^1_{occ'} s_{toc'} \\
        \sum_{c'} \rmW^2_{occ'} v_{toc'}
    }
    +
    \m{
        \sum_{c'} \rmW^3_{occ'} s_{t\emptyset c'} \\
        \sum_{c'} \rmW^4_{occ'} v_{t\emptyset c'}
    }
    \,.
    \label{eq:input_trf}
\end{equation}
%
\looseness=-1
The matrices $\rmW^{1, 2, 3, 4}$ are learnable and of dimension $n \times n_c \times n_s^\text{object}$, $n \times n_c \times n_v^\text{object}$, $n \times n_c \times n_s^\text{global}$, or $n \times n_c \times n_v^\text{global}$, respectively. Here $n_s^\text{object}$ is the number of $\sothree$ scalar quantities associated with each object in the trajectory, $n_v^\text{object}$ is the number of $\sothree$ vector quantities associated with each object, $n_s^\text{global}$ is the number of scalar quantities associated with the robot or global properties of the system, and $n_v^\text{global}$ is the number of vectors of that nature.
The number of input channels $n_c$ is a hyperparameter.
We initialize the matrices $\rmW^i$ such that Eq.~\eqref{eq:input_trf} corresponds to a concatenation of all object-specific and global features along the channel axis at the beginning of training.

%--------------------------------------------------------------------------------
\xhdr{Step 2: $\fullgroup$-equivariant U-net}
We then process the data with a $\modgroup$-equivariant denoising network.
Its key components are three alternating types of layers. Each type acts on the representation dimension of one of the three symmetry groups while leaving the other two invariant---\ie they do not mix internal representation types of the other two layers:
%
\begin{itemize}[itemsep=1pt, topsep=1pt, parsep=1pt]
    \item \emph{Temporal layers}: Time-translation-equivariant convolutions along the temporal direction (i.\,e.\ along trajectory steps), organized in a U-Net architecture.
    \item \emph{Object layers}: Permutation-equivariant self-attention layers over the object dimension.
    \item \emph{Geometric layers}: $\sothree$-equivariant interaction between the scalar and vector features.
\end{itemize}
%
\looseness=-1
In addition, we use residual connections, a new type of normalization layer that does not break equivariance, and context blocks that process conditioning information and embed it in the internal representation (see Appendix \ref{app:algo} for more details). These layers are combined into an equivariant block consisting of one instance of each layer, and the equivariant blocks are arranged in a U-net, as depicted in Fig.~\ref{fig:architecture}. Between the levels of the U-net, we downsample (upsample) along the trajectory time dimension by factors of two, increasing (decreasing) the number of channels correspondingly.

%--------------------------------------------------------------------------------
\looseness=-1
\xhdr{Temporal layers}
Temporal layers consist of $1$D convolutions along the trajectory time dimension. To preserve $\sothree$ equivariance, these convolutions do not add any bias and there is no mixing of features associated with different objects nor the four geometric features of the internal $\sothree$ representation.

%--------------------------------------------------------------------------------
\xhdr{Permutation layers}
Permutation layers enable features associated with different objects to interact via an equivariant multi-head self-attention layer. Here, there is no mixing between features associated with different time steps, nor between the four geometric features of the internal $\sothree$ representation. This is $\sothree$-equivariant, as the attention weights compute invariant $\sothree$ norms.

%--------------------------------------------------------------------------------
\looseness=-1
\xhdr{Geometric layers}
Geometric layers enable mixing between the scalar and vector quantities that are combined in the internal representation but do not mix between different objects or across the time dimension. We construct an expressive equivariant map between scalar and vector inputs and outputs following \citet{villar2021scalars}: We first separate the inputs into $\sothree$ scalar and vector components, $w_{toc} = (s_{toc}, v_{toc})^T$. We then construct a complete set of $\sothree$ invariants by combining the scalars and pairwise inner products between the vectors, $S_{to} = \{ s_{toc} \}_c \cup \{ v_{toc} \cdot v_{toc'} \}_{c, c'}$.
These are then used as inputs to two MLPs $\phi$ and $\psi$, and finally we get output scalars and vectors, $w'_{toc} = \left( \phi(S_{to})_c , \sum_{c'} \psi(S_{to})_{cc'} v_{toc'} \right)$.
\Citet{villar2021scalars} show that this approach can approximate any equivariant map between $\sothree$ scalars and vectors under mild assumptions. In its original form, however, it can become prohibitively expensive, as the number of $\sothree$ invariants $S_{to}$ scales quadratically with the number of channels. Thus, we first linearly map the input vectors into a smaller number of vectors, apply this transformation, and increase the number of channels again with another linear map.

%--------------------------------------------------------------------------------
\xhdr{Step 3: Representation unmixer}
The equivariant network outputs internal representations $w_{toc}$ that are transformed back to data representations using linear maps, in analogy to Eq.~\eqref{eq:input_trf}.
Global properties, \eg robotic degrees of freedom, are aggregated from the object-specific internal representations by taking the mean, minimum, and maximum across the objects. These three aggregates are then concatenated along the channel dimension. We find it beneficial to apply an additional geometric layer to these aggregated global features before separating them into the original representations.

%--------------------------------------------------------------------------------
\looseness=-1
\xhdr{Training}
We train \eqd by optimizing for a simplified variational lower bound \citep{ho2020denoising} on offline trajectories without any reward information.

%================================================================================
\subsection{Planning with equivariant diffusion}
\label{sec:eqd_planning}
%================================================================================

%--------------------------------------------------------------------------------
\looseness=-1
\xhdr{Planning as diffusion}
A diffusion model trained on offline trajectory data jointly learns a world model and a policy. Following \citet{janner2022planning}, we use it to solve planning problems by choosing a sequence of actions to maximize the expected task rewards.

To do this, we use three features of diffusion models. The first is the ability to sample from them by drawing noisy trajectory data from the base distribution and iteratively denoising them with the learned network yielding trajectories similar to those in the training set.
For such sampled trajectories to be useful for planning, they need to begin in the current state of the environment. We achieve this by conditioning the sampling process such that the initial state of the generated trajectories matches the current state, in analogy to inpainting.
Finally, we can guide this sampling procedure toward solving concrete tasks specified at test time using classifier-based guidance where a regression model is trained offline to map trajectories to task rewards.

%--------------------------------------------------------------------------------
\xhdr{Symmetry breaking}
By construction, our equivariant diffusion model learns a $\fullgroup$-invariant density over trajectories. Unconditional samples will reflect this symmetry property---it will be equally likely to sample a trajectory and its rotated or permuted counterpart. However, concrete tasks will often break this invariance, for instance by requiring that a robot or object is brought into a particular location.
Our \eqd approach allows us to elegantly break the symmetry at test time for concrete tasks.
Such a soft symmetry breaking can happen through conditioning, for instance by specifying the initial or final state of the sampled trajectories, or through a non-invariant reward model used for guidance during sampling.

We present in section~\ref{ssec:faces} an application of PnP-HVAE on face images, using a pretrained state-of-the-art hierarchical VAE. 
Next, we study the application of our framework to natural images. To that end, we introduce  in section~\ref{ssec:patchVDVAE}  a patch hierachical VAE architecture, that is able to model natural images of different resolutions. In section~\ref{ssec:app_nat}, we provide deblurring, super-resolution and inpainting experiments to demonstrate the relevance of the proposed method.

Additional results are presented in Appendix~\ref{app:add}. All experiments can be reproduced using the code available at \url{https://github.com/jprost76/PnP-HVAE}.



\subsection{Face Image restoration (FFHQ)}\label{ssec:faces}
We first demonstrate the effectiveness of PnP-HVAE on highly structured data, by performing face image restoration.
Latent variable generative models can accurately model structured images such as face images \cite{karras2019style,vahdat2020nvae,child2021very,kingma2018glow}, and then be used to produce high quality restoration of such data. 
In our experiments, we use the VDVAE model of~\cite{child2021very}, pre-trained on the FFHQ dataset~\cite{karras2019style}, as our hierarchical VAE prior.
VDVAE has $L=66$ latent variable groups in its hierarchy and generates images at resolution $256\times256$.

We compare PnP-HVAE with the intermediate layer optimization algorithm (ILO)~\cite{daras2021intermediate} that is based on a different class of generative models than HVAE. ILO is a GAN inversion method which optimizes the image latent code along with the intermediate layer representation of a StyleGAN to generate an image consistent with a degraded observation.
We use the official implementation of ILO, along with a StyleGAN2 model~\cite{karras2020analyzing, stylegan2pytorch}, that was trained for 550k iterations on images of resolution $256\times256$ from FFHQ.  
As VDVAE and StyleGAN models are not trained on the same train-test split of FFHQ, we chose to evaluate the methods on a subset of 100 images from the CelebA dataset~\cite{liu2018large}. 
For super-resolution, the degradation model corresponds to the application of a gaussian low-pass filter followed by a $\times 4$ sub-sampling, and the addition of a gaussian white noise with $\sigma=3$.
For the deblurring, we considered motion blur and  gaussian kernels, both with a noise level $\sigma=8$. %

We provide quantitative comparisons in table~\ref{table:comp_ILO}, along with a visual comparison of the results in figure~\ref{fig:face_restoration}.
PnP-HVAE has the best  PSNR and SSIM results for all the considered restoration tasks, while ILO provides better results  for the perceptual distance.
By jointly optimizing the image and its latent variable, PnP-HVAE provides  results that are both realistic and consistent with the degraded observation.
On the other hand,  ILO  only optimizes on an extended latent space. This method generates  sharp and realistic images with better LPIPS scores,   
but the results lack  of consistency with respect to the observation, which explains the overall lower PSNR performance. 






\subsection{PatchVDVAE: a HVAE for natural images}\label{ssec:patchVDVAE}
Available generative models in the literature operate on images of  fixed resolutions and
are either restrained to datasets of limited diversity, or even to registered face images~\cite{kingma2018glow,child2021very, vahdat2020nvae, karras2019style}, or requiring additional class information~\cite{brock2018large, dhariwal2021diffusion, song2020score, luhman2022optimizing}.
Fitting an unconditional model on natural images appears to be a more difficult task, as their resolution can change, and their content is highly diverse.
The complexity of the problem can be reduced by learning a prior model on patches of reduced dimension. 
For image restoration problems, the patch model can be reused on images of higher dimensions~\cite{zoran2011learning,prost2021learning,altekruger2022patchnr}. When the model is a full CNN, the prior on the set of the  patches can  be computed efficiently by applying the network on the full image~\cite{prost2021learning}.

We thus introduce  patchVDVAE, a fully convolutional hierarchical VAE.
Contrary to existing HVAE models whose resolution is constrained by the constant tensor at the input of the top-down block, patchVDVAE can generate images of different resolutions by controlling the dimension of the input latent. 
This amounts to defining a prior on patches whose dimension corresponds to the receptive field of the VAE. A similar model is used for image denoising in~\cite{prakash2021interpretable}.

 
For PatchVDVAE architecture, we use the same bottom-up and top-down blocks as VDVAE~\cite{child2021very}, and replace the constant trainable input in the first top-down block by a latent variable, to make the model fully convolutional (details on the  architecture are given in Appendix~\ref{app:details}). 
The training dataset is composed of $128\times 128$ patches extracted from a combination of DIV2K~\cite{agustsson2017ntire} and Flickr2K~\cite{Lim_2017_CVPR_workshops} datasets.
We perform data augmentation by extracting  patches at $3$ resolutions: HR-images and $\times 2$ and $\times 4$ downscaled images. 
The model is trained for $7.10^5$ iterations with a batch size of $64$. Following the recommendation of~\cite{hazami2022efficient}, we use Adamax optimizer with an exponential moving average and gradient smoothing of the variance.
We set the decoder model to be a gaussian with diagonal covariance, as in~\cite{luhman2022optimizing}.
PatchVDVAE is fully convolutional and can generate images of dimension that are multiples of $64$ as illustrated by
figure~\ref{fig:vdvae}.

\newlength{\patchwidth}
\setlength{\patchwidth}{0.135\columnwidth}
\begin{figure}[!ht]
    \centering
    \begin{subfigure}[t]{.34\columnwidth}\hspace{0.1cm}
        \setlength{\tabcolsep}{0.02pt}
\renewcommand{\arraystretch}{0}
        \begin{tabular}{*{2}{p{1.03\patchwidth}}}
            \includegraphics[width=\patchwidth]{figures_arxiv/patchVDVAE/samples/generated/64x64/setup-5-image-0018.png} &
            \includegraphics[width=\patchwidth]{figures_arxiv/patchVDVAE/samples/generated/64x64/setup-5-image-0016.png} \\
            \includegraphics[width=\patchwidth]{figures_arxiv/patchVDVAE/samples/generated/64x64/setup-5-image-0008.png} &
            \includegraphics[width=\patchwidth]{figures_arxiv/patchVDVAE/samples/generated/64x64/setup-5-image-0019.png}   
        \end{tabular}
    \end{subfigure}\hspace{-0.15cm}
    \begin{subfigure}[t]{.64\columnwidth}
\begin{tabular}{cc}\vspace{-0.1cm}
\includegraphics[width=2\patchwidth]{figures_arxiv/patchVDVAE/samples/generated/256x256/setup-2-image-0009.png}&
        \includegraphics[width=2\patchwidth]{figures_arxiv/patchVDVAE/samples/generated/256x256/setup-2-image-0002.png}\end{tabular}

    \end{subfigure}
    \caption{\label{fig:vdvae} Left: $64\times64$ patches samples from our patchVDVAE model trained on patches from natural images.
    Right: PatchVDVAE is fully convolutional and it can generate images of higher resolution (here: $128\times128$).\vspace{-0.2cm}}
\end{figure}

\subsection{Natural images restoration}\label{ssec:app_nat}
We  evaluate PnP-HVAE on natural image restoration.
For each task, we report the average value of the PSNR, the SSIM, and the LPIPS metrics on $20$ images from the test set of the BSD dataset~\cite{MartinFTM01}.\\


\noindent
{\bf Image deblurring.}
In the experiments, we consider $2$ gaussian kernels and $2$ motion blur kernels from~\cite{levin2009understanding}, with $3$ different noise levels 
$\sigma \in \{2.55, 7.65, 12.75\}$.
As a baseline we consider  EPLL~\cite{zoran2011learning}, which learns a prior on image patches with a gaussian mixture model.
We also compare PnP-HVAE  with PnP-MMO and GS-PnP, $2$ competing convergent Plug-and-Play methods based on CNN denoisers.
PnP-MMO~\cite{pesquet2021learning} restricts the denoiser to be contraction in order to guarantee the convergence of the PnP forward-backard algorithm. GS-PnP~\cite{hurault2022gradient} considers a gradient step denoiser and reaches state-of-the-art performances of non converging methods~\cite{zhang2021plug}.
We set the temperature $\tau$  in our method as $0.95$, $0.8$ and $0.6$ for noise levels $2.55$, $7.65$ and $12.75$ respectively, and we let it run for a maximum of $50$ iterations. 
For the three compared methods we use the official implementations and pre-trained models provided by the respective authors. 
Details on the choice of hyperparameters for the concurrent methods are provided in the Appendix~\ref{app:details}
Figure~\ref{fig:deblurring_bsd} illustrates that our method provides correct deblurring results. 

According to table~\ref{tab:deb}, the performance of PnP-HVAE is between those of EPLL and GS-PnP and it outperforms PnP-MMO for large noise levels.\\

\begin{table}
\begin{center}\footnotesize
    \begin{tabular}{>{\centering}m{.3cm}*{5}{c}}
    $\sigma$ &Method & PSNR$\uparrow$ & SSIM$\uparrow$ & LPIPS$\downarrow$  \\ 
    \hline
    \multirow{4}{*}{\vcell{$2.55$}}
    & PnP-HVAE & $27.75$ & $0.79$ & $0.31$\\
    & GS-PNP \cite{hurault2022gradient} & $\mathbf{29.59}$ & $\mathbf{0.84}$ & $\mathbf{0.22}$\\
    & EPLL \cite{zoran2011learning} & $26.49$ & $0.71$ & $0.36$\\ 
    & PnP-MMO \cite{pesquet2021learning} & $\underbar{29.50}$ & $\underbar{0.83}$ & $\underbar{0.20}$ \\ \hline
    \multirow{4}{*}{\vcell{$7.65$}}
    & PnP-HVAE & $\underbar{26.36}$ & $\underbar{0.72}$ & $\underbar{0.40}$\\
    & GS-PNP \cite{hurault2022gradient} & $\mathbf{27.33}$ & $\mathbf{0.77}$ & $\mathbf{0.31}$\\
    & EPLL \cite{zoran2011learning} & $24.04$ & $0.66$ & $0.45$ \\ 
    & PnP-MMO \cite{pesquet2021learning} & $25.34$ & $0.69$ & $0.34$\\
    \hline
    \multirow{4}{*}{\vcell{$12.75$}}
    & PnP-HVAE & $\underbar{25.12}$ & $\mathbf{0.73}$ & $\underbar{0.47}$\\
    & GS-PNP \cite{hurault2022gradient} & $\mathbf{26.32}$ & $\mathbf{0.73}$ & $\mathbf{0.37}$\\
    & EPLL \cite{zoran2011learning} & $23.28$ & $0.61$ & $0.51$ \\ 
    & PnP-MMO \cite{pesquet2021learning} & $22.42$ & $0.53$& $0.54$ \\
    \hline
    &\vspace*{-.3cm}\\
            \multicolumn{2}{c}{Blur and motion kernels}& \multicolumn{3}{c}{
        \includegraphics*[scale=1]{figures_arxiv/kernels/4.png}\;\includegraphics*[scale=1]{figures_arxiv/kernels/7.png}\;\includegraphics*[scale=1]{figures_arxiv/kernels/9.png}\;\includegraphics*[scale=1]{figures_arxiv/kernels/11.png}} 
    \end{tabular}
        \caption{\label{tab:deb}Comparison  of PnP-HVAE  and other restoration methods on deblurring. Results are averaged on $4$ kernels.\vspace{-0.2cm}}% on image deblurring.}
    \end{center}
\end{table}

\begin{figure}
    
    \begin{subfigure}[h]{\linewidth}
        \centering
        \includegraphics*[width=\columnwidth]{figures_arxiv/deb_s255_k7.pdf}\vspace{-0.1cm}
        \caption{Gaussian blur, $\sigma=2.55$}
    \end{subfigure}
    \begin{subfigure}[h]{\linewidth}
        \centering
        \includegraphics*[width=\columnwidth]{figures_arxiv/deb_s765_k11.pdf}\vspace{-0.1cm}
        \caption{Motion blur, $\sigma=7.65$}
    \end{subfigure}\vspace*{-0.1cm}
    \caption{\label{fig:deblurring_bsd} Natural image deblurring\vspace{-0.1cm}}
\end{figure}

\noindent {\bf Effect of the temperature.}
PnP-HVAE gives control on the temperature of the prior over the latent space.
In figure~\ref{fig:temp_effect}, we illustrate that reducing the temperature increases the strength of the regularization prior. In this example the tuning $\tau=0.7$ produces the best performance.\\
\begin{figure}[!ht]
   
    \includegraphics[width=\columnwidth]{figures_arxiv/demo_temp.pdf}\vspace{-0.15cm}
    \caption{ \label{fig:temp_effect} Effect of the temperature in PnP-VAE on a deblurring problem, with $\sigma=7.65$.\vspace{-0.15cm}}
\end{figure}


\noindent
{\bf Image inpainting.}
Next we consider the task of noisy image inpainting. 
We compose a test-set of 10 images from the validation set of BSD~\cite{MartinFTM01} and we create masks
  by occluding diverse objects of small size in the images. 
A gaussian white noise with $\sigma=3$ is added to the images.
As a comparaison, we still consider GS-PnP and EPLL.
For PnP-HVAE, the temperature is set to $\tau=0.6$, and the algorithm is run for a maximum of $200$ iterations, unless the residual $||\x_{k+1}-\x_k||$ is on a plateau.
We provide on Table~\ref{tab:inpainting_bsd} the distortion metrics with the ground truth, as well as a visual
\begin{table}



\begin{center}
    \begin{tabular}{cccc}
        & PSNR$\uparrow$ & SSIM$\uparrow$ &LPIPS$\downarrow$ \\\hline
        PnP-HVAE  & $\mathbf{29.54}$ & $\mathbf{0.93}$ & $\mathbf{0.06}$\\
        GS-PNP & $28.52$ & $\mathbf{0.93}$ & $0.09$\\
        EPLL & $\underline{29.16}$ & $\mathbf{0.93}$ & $\mathbf{0.06}$\\
    \end{tabular}
    \caption{\label{tab:inpainting_bsd}Quantitative evaluation for inpainting on BSD.}
    \end{center}
\end{table}
comparison on figure~\ref{fig:inpainting_bsd}. 
With its hierarchical structure,  PnP-HVAE outperforms the compared methods. \vspace{0.05cm}



\begin{figure}[!h]
    \includegraphics[width=\columnwidth]{figures_arxiv/demo_inp_bsd2.pdf}\vspace{-0.1cm}
    \caption{\label{fig:inpainting_bsd}Natural image inpainting\vspace{-0.3cm}}
\end{figure}











\section{Related work}
% There is extensive recent work on speeding up analytical queries due to the need for consistent execution times in the face of the explosive growth in the volume of available data.
% In this section, we divide existing work into two categories: maintaining data freshness in MVs (\Cref{sec:server_side}) and optimizations for minimizing ad-hoc query latency (\Cref{sec:client_side}).

% \subsection{Maintaining Data Freshness in MVs}
% \label{sec:server_side}
% There exists a variety of data warehousing applications aimed at supporting low-latency analytical queries on fresh data.
% In particular, these applications require efficiency in the propagation of newly ingested data into downstream MVs.
 
\mypara{Efficient MV Refresh}
Incremental view maintenance (IVM) aims to update MVs to reflect newly ingested data, taking advantage of already computed results to perform the update in a manner more efficient than computing from scratch (full refresh)
~\cite{ahmad2012dbtoaster,mcsherry2013differential,armbrust2013generalized,zeng2016iolap, palpanas2002incremental, griffin1995incremental, agiwal2021napa, braun2015analytics}. 
There is an abundance of work in IVM, including incremental updates on duplicate values~\cite{griffin1995incremental}, non-distributive aggregate functions~\cite{palpanas2002incremental}, higher-order views~\cite{ahmad2012dbtoaster}, and sliding windows~\cite{braun2015analytics}. 
More recent works also investigate the scalability aspect of IVM, proposing scale-independent updates~\cite{armbrust2013generalized} and sampled views~\cite{zeng2016iolap}. Since \system is applicable to arbitrary SQL statements, \system is orthogonal to and is fully compatible with existing IVM techniques.

\mypara{MV Refresh Scheduling}
There exist works on scheduling the refresh of a MV set focusing on resolving cyclic dependencies~\cite{folkert2005optimizing}, minimizing weighted average staleness~\cite{golab2009scheduling}, and prioritizing MVs with the highest speedups on predicted future queries~\cite{ahmed2020automated}.
\system's scheduling to speed up the end-to-end refresh of the MV set is not addressed in existing works.

\mypara{DAG Workflow Scheduling}
The execution of workloads consisting of individual jobs with acyclic dependencies is a well-studied topic~\cite{apacheoozie,sparkdag,marchal2018parallel,bathie2020revisiting,baruah2022ilp}; many of these techniques can be applied to MV refresh runs studied in this paper.
Existing workflow scheduling systems such as Apache Oozie~\cite{apacheoozie}, Apache Airflow~\cite{airflow}, and Spark DAG scheduler~\cite{sparkdag} automate the execution of user-defined workflows following a topological order.
There are recent works aimed at finding more optimal execution orders in terms of peak memory usage~\cite{marchal2018parallel, bathie2020revisiting} and execution time on parallel platforms~\cite{baruah2022ilp}.
While \system is designed for use with MV refresh runs/workloads, our technique on joint scheduling and optimization can be reasonably applied to general workloads as a possible future direction.

% \paragraph{Incremental MV indexing}
% Update-optimized indices such as the log-structured merge-trees (LSM)~\cite{o1996log} are used for indexing MVs due to frequent updates induced by data ingestion~\cite{gupta2016mesa,agiwal2021napa}.
% \system is orthogonal to indexing: \system is capable of efficiently performing MV refresh runs regardless of whether the individual MVs are indexed or not.

% \subsection{Ad-hoc Query Latency Reduction}
% \label{sec:client_side}

% The minimization of ad-hoc analytical query response times is a well-studied topic due to latency being negatively correlated with the productivity of a data analyst during a data analysis session~\cite{liu2014effects}.
% Sessions are commonly conducted within visualization systems that contain a variety of optimization techniques to ensure that query response times fall within a certain latency tolerance.

% \mypara{Data prefetching}
% Data is often loaded into memory on a by-need basis in visualization systems to minimize interference with user-issued query computations~\cite{mani2017effective,xin2021enhancing,galakatos2017revisiting, yan2020auto, battle2016dynamic, crotty2016case, jalaparti2018netco}. 
% Query-time data retrieval can be significantly expedited by anticipating the data usage of the user in future queries and pre-loading the data into memory during the downtime between user queries (`think time'). SMART~\cite{mani2017effective} prefetches data for modified versions of current user-issued queries with different filters and dimensions. A-WARE~\cite{crotty2016case} maintains a sample store constantly refined through ingesting data based on speculations of future plots.
% ForeCache~\cite{battle2016dynamic} uses an SVM to predict the user's current analysis phase and accordingly prefetches data tiles partitioned based on different numerical values. NetCo predicts future queries via log analysis, and solves an ILP formulation to prefetch data to maximize the number of SLO-meeting queries~\cite{jalaparti2018netco}.
% In the case of MV refresh workloads, `think time' is nonexistent as individual MVs are refreshed back-to-back, rendering data prefetching techniques non-applicable.

\mypara{Intermediate Data Caching}
Some existing data visualization systems cache user-defined variables to support the typical incremental construction of data visualizations~\cite{zgraggen2016progressive, eichmann2020idebench} during data analysis sessions~\cite{jupyter, rstudio, colab}. 
Recent work proposes a management scheme for these cached variables under a memory constraint that greedily keeps variables with the highest estimated time savings based on predicted future user behavior via neural networks~\cite{xin2021enhancing}.
While useful for data visualization, a greedy approach to memory management fails to achieve satisfactory results compared to \system.

\mypara{Intermediate Result Reuse}

There exist works on storing intermediate results from computations to speedup future computations~\cite{yang2018intermediate, dursun2017revisiting, nagel2013recycling, michiardi2019memory, galakatos2017revisiting}.
Studied topics include the identification of reuse opportunities by finding overlaps in computation graphs of successive jobs~\cite{yang2018intermediate, michiardi2019memory},
selective storage under a space constraint with heuristics such as reuse probability~\cite{dursun2017revisiting}, expected savings~\cite{yang2018intermediate}, and recompute-storage cost difference~\cite{nagel2013recycling},
and rewriting incoming jobs to take advantage of stored intermediates~\cite{galakatos2017revisiting}.
These works share similarity with \system in their selection of items to store under a memory constraint, however, \system's problem setting requires it to uniquely consider the joint (re)ordering of job executions along with the selection of items.

% work that considers both job execution (re)order as well as intermediate result caching with a bounded amount of memory. but notably lack the joint aspect of \system and cannot be used to achieve immediate speedup on an incoming MV refresh run if no intermediates are stored beforehand. 

\mypara{Incremental Query Processing} Incremental processing (IQP) is useful for cases where not all data required for a query is immediately available. Similar to online aggregation~\cite{hellerstein1997online}, initial results of a query are computed on a subset of required data and progressively refined as the rest of the required data arrives in a predictable pattern~\cite{tang2019intermittent,wangtempura}. Tang et al. propose a dynamic programming formulation to pick intermediate states to store in memory given a limited memory budget~\cite{tang2019intermittent}. Tempura rewrites the query plan for more efficient execution based on predicted data arrival patterns~\cite{wangtempura}. While similarities exist between the problem setting of IQP and \system, such as management of bounded memory, \system notably includes additional joint optimization for the order of MV updates.

% \paragraph{Sampling}
% Sampling has seen wide use in visualization systems for reducing the computation time of ad-hoc queries by computing an approximate result over a subset of data as exact results are not always required by the user~\cite{crotty2016case, mani2017effective, zgraggen2014panoramicdata, kraska2021northstar, galakatos2017revisiting, kandula2016quickr}. 
% Commonly studied topics in sampling for ad-hoc queries include complex query sampling~\cite{kandula2016quickr}, rare event aggregation~\cite{kraska2021northstar, galakatos2017revisiting}, and maintaining consistency between related sampled visualizations~\cite{zgraggen2014panoramicdata}.
% Sampling server-side at the MV level compromises the assumptions of downstream applications and is thus not considered in \system.

% \paragraph{Progressive visualization}
% The latency tolerance for time-consuming queries can be circumvented by presenting a partially-computed visualization to the user within the tolerance, which is then incrementally refined until it is fully accurate~\cite{rahman2017ve, zgraggen2016progressive, crotty2015vizdom, kraska2021northstar, kamat2017infiniviz}.
% Example plots which benefit from progressive visualization include bar charts~\cite{kamat2017infiniviz} and heatmaps~\cite{rahman2017ve}.
% Similar to sampling, study on this topic is orthogonal to \system as pushing out partially-updated MVs compromises downstream assumptions.
We provide some comments on the growth conditions which constituted the majority of our analysis in sections \ref{sec:Hmixing} and \ref{sec:Hsigma}. In the simplest cases of Lemma \ref{lemma:unstableGrowth}, growth was established in an analogous fashion to the old one-step expansion condition (\ref{eq:oldOneStepExpansion}), finding the relevant Jacobians $M_j$ and checking that their expansion factors $K(M_j)$ satisfy
\begin{equation}
    \label{eq:discussionOneStep}
    \sum_j \frac{1}{K(M_j)} <1.
\end{equation}
For the more complicated cases, the inductive method used to establish growth near the accumulation points in Lemma \ref{lemma:unstableGrowth} and the weakened one-step expansion condition (\ref{eq:oneStep}) both address the same fundamental issue: the splitting of unstable curves by singularities into an unbounded number of small components. They circumvent this obstacle in rather different ways, however. While (\ref{eq:oneStep}) generalises (\ref{eq:discussionOneStep}) to ensure an growth of unstable curves `on average' (see \cite{chernov_statistical_2009} for a precise statement), our inductive method is a more direct adaptation of (\ref{eq:discussionOneStep}), using it to generate contradictory geometric conditions which a hypothetical non-growing unstable curve must satisfy. It may be possible to prove Theorem \ref{sec:Hmixing} using (\ref{eq:oneStep}) as the basis for growth. Since we required (\ref{eq:oneStep}) anyway for proving Theorem \ref{thm:HsigmaExp}, this could potentially condense our analysis, but only to a minor extent. A convenience of the method used in section \ref{sec:Hmixing} is that, by way of the `simple intersection' property, it naturally gives geometric information on the images of manifolds, useful for proving the property \textbf{(M)} of Theorem \ref{thm:katok-strelcyn}.

We expect that essentially analogous analysis can be applied to establish mixing properties in a wide class of piecewise linear non-uniformly hyperbolic maps, including those (like the OTM) which sit on the boundary of ergodicity and beyond. While we have relied on the precise partition structure of $H_\sigma$, its fundamental feature (self-similar sequences of elements $A^k$, sharing boundaries with its neighbours $A^{k-1},A^{k+1}$ and accumulating onto some point $p$) is quite typical to return map systems. See, for example, those of various stadium billiards \cite{chernov_chaotic_2006,chernov_improved_2008,chernov_statistical_2009} and LTMs \cite{springham_polynomial_2014}. Indeed, the same method can be used to prove the Bernoulli property for non-monotonic LTMs \cite{myers_hill_mixing_2022}, where monotonicity of the manifold images cannot be assumed and the classical argument \cite{sturman_mathematical_2006} fails. The OTM is the pointwise limit of these maps as the boundary shrinks to null measure. It further has utility in proving growth conditions for maps which are uniformly hyperbolic but possess regions $A_j$ where the hyperbolicity is very weak, signified by $K(M_j) \approx 1$, so that (\ref{eq:discussionOneStep}) fails. Typically this leads to suboptimal bounds on mixing windows, see e.g. \cite{wojtkowski_model_1981,przytycki_ergodicity_1983,myers_hill_family_2022}. The map $H_{(\eta,\eta)}$ for $\eta \approx 1/2$ is another example, possessing weak hyperbolicity over $A_2, A_3$. Letting $\varepsilon = |\eta-1/2|>0$, there is an upper bound $N = N(\varepsilon)$ on escape times from the intersections $A_2\cap \sigma, A_3 \cap \sigma$. The growth lemma then follows by applying the inductive step roughly $N$ times and can be established for arbitrarily small $\varepsilon$, opening the door to establishing optimal mixing windows.

The above gives two examples of piecewise linear perturbations to $H$ where mixing with respect to Lebesgue is preserved and our methods can be applied. Nonlinear perturbations to the shear profiles complicate the analysis in several ways. Firstly as the map's Jacobians takes on a broader range of values, cone invariance becomes an increasingly harder condition to establish. Cones must be widened, giving looser bounds on expansion factors, which may already be weak due to new regions of weaker stretching. This, together with the change from polygonal to curvilinear return time partition elements and nonlinear local manifolds, adds some complexity to showing growth conditions. This does not rule out certain (small) nonlinear perturbations however. There is some leeway in the inequalities which govern cone invariance and growth of local manifolds, the latter of which is not too dissimilar from the piecewise linear setting (see Lemmas \ref{lemma:piecewiseApprox}, \ref{lemma:componentLength}). Certain small perturbations would not alter the \emph{topological} structure of the return time partition, i.e. which elements share boundaries, the key information needed for setting up the induction. Finally while the partition elements would no longer be polygonal, only coarse geometric information is required for verifying each inductive step. Following the above, a potential perturbation could be to replace the linear portions of each shear by a cubic, perturbing the tent profile
\[  f(t) = \begin{cases} 2t & 0 \leq t \leq 1/2, \\ 2(1-t) & 1/2 \leq t \leq 1 ,\end{cases} \]
of the OTM shears to
\[  f_a(t) = \begin{cases} \frac{1}{8} t \left(16 - a + 6at - 8at^{2} \right) & 0 \leq t \leq 1/2, \\ \frac{1}{8}\left(1-t\right)\left( 16 - a + 6a\left(1-t\right) - 8a\left(1-t\right)^{2}\right)  & 1/2 \leq t \leq 1, \end{cases}   \]
for $a>0$. For small enough $a$ the gradient range $f'(t)$ is restricted to small neighbourhoods of $\{ 2, -2\}$ and the escape time partition retains a similar structure. We illustrate this in Figure \ref{fig:perturbations}, showing escapes from the square $S_3$ under the map $G \circ F$, equivalent to escapes from the perturbed $A_3$ under the $G \circ F$, but with a cleaner geometry for comparison. When $a$ is too large the analogy to the OTM breaks down. At $a=16$ the map is twice differentiable everywhere and features a new source of slowed mixing, the Jacobian is the identity at the corner points $x,y \in \{  0, 1/2 \}$ giving locally parabolic behaviour (visible in the escape time partition). 

\begin{figure}
    \centering
    \includegraphics[width=0.24 \linewidth]{0.png}
    \includegraphics[width=0.24 \linewidth]{4.png}
    \includegraphics[width=0.24 \linewidth]{8.png}
    \includegraphics[width=0.24 \linewidth]{16.png}
    \caption{Partition of escape times from $S_3$ under the mapping $F \circ G$ for $a= 0,4,8,16$. }
    \label{fig:perturbations}
\end{figure}

%================================================================================

\bibliography{references}
\bibliographystyle{abbrvnat}

\section{Appendix for Proofs}

\paragraph{Proof of Theorem \ref{thm:main}.}

\begin{proof}
\label{proof:main}
Our proof has two steps. In Step 1, we will show that SimCLR is equivalent to minimizing the cross entropy loss defined in Eqn.~(\ref{eqn:cross-entropy}). 
In Step 2, we will show  that minimizing the cross-entropy loss 
is equivalent to spectral clustering on $\bfpi$. 
Combining the two steps together, we have proved our theorem. 

\textbf{Step 1: } SimCLR is equivalent to minimizing the cross entropy loss.

The cross-entropy loss takes expectation over 
$\bfW_\bfX\sim \mathbb{P}(\cdot ; \bfpi)$, 
which means $\bfW_\bfX$ has exactly one non-zero entry in each row $i$. By Lemma~\ref{lem:multinomial}, we know every row $i$ of $\bfW_\bfX$ is independent of other rows. Moreover, 
$\bfW_{\bfX,i}\sim \mathcal{M}(1, \bfpi_i/\sum_j \bfpi_{i,j})=\mathcal{M}(1, \bfpi_i)$, because $\bfpi_i$ itself is a probability distribution.
Similarly, we know $\bfW_\bfZ$ also has the row-independent property by sampling over $\mathbb{P}(\cdot;\bfK_\bfZ)$.
Therefore, by Lemma~\ref{lem:cross_split}, we know Eqn.~(\ref{eqn:cross-entropy}) is equivalent to:
\[
 -\sum_{i=1}^n \mathbb{E}_{\bfW_{\bfX,i}}[\log \mathbb{P}(\bfW_{\bfZ,i}=\bfW_{\bfX,i};\bfK_\bfZ)],
\]

This expression takes expectation over $\bfW_{\bfX,i}$ for the given row $i$. Notice that 
$\bfW_{\bfX,i}$ has exactly one non-zero entry, which equals $1$ (same for $\bfW_{\bfZ,i}$). 
As a result
we expand the above expression to be:
\begin{equation}
 -\sum_{i=1}^n \sum_{j\neq i} \Pr(\bfW_{\bfX,i,j}=1)\log \Pr(\bfW_{\bfZ,i,j}=1).
\label{eqn:detailed-expansion}    
\end{equation}


By Lemma~\ref{lem:multinomial}, $\Pr(\bfW_{\bfZ,i,j}=1)=\bfK_{\bfZ,i,j}/\|\bfK_{\bfZ,i}\|_1$ for $j\neq i$. Recall that $\bfK_\bfZ=(k(\bfZ_i-\bfZ_j))_{(i,j)\in[n]^2}$, which means 
$\bfK_{\bfZ,i,j}/\|\bfK_{\bfZ,i}\|_1=\frac{\exp(-\|\bfZ_i-\bfZ_j\|^2/{2\tau})}{\sum_{k\neq i}
\exp(-\|\bfZ_i-\bfZ_k\|^2/{2\tau})
}$ for $j\neq i$, when $k$ is the Gaussian kernel with variance $\tau$. 

Notice that $\bfZ_i=f(\bfX_i)$, so we know
\begin{equation}
-\log \Pr(\bfW_{\bfZ,i,j}=1)=
-\log \frac{\exp(-\|f(\bfX_i)-f(\bfX_j)\|^2/{2\tau})}{\sum_{k\neq i}
\exp(-\|f(\bfX_i)-f(\bfX_k)\|^2/{2\tau}),
}
\label{eqn:infonce-equivalence}    
\end{equation}


The right hand side is exactly the InfoNCE loss defined in Eqn.~(\ref{eqn:infonce}).
Inserting Eqn.~(\ref{eqn:infonce-equivalence}) into Eqn.~(\ref{eqn:detailed-expansion}), we get the SimCLR algorithm, which first samples augmentation pairs $(i,j)$ with $\Pr(\bfW_{\bfX,i,j}=1)$ for each row $i$, and then optimize the InfoNCE loss. 

\textbf{Step 2: } minimizing the cross entropy loss 
is equivalent to spectral clustering on $\bfpi$.


By Lemma~\ref{lem:convert_to_spectral}, we may further convert the loss to 
\begin{equation}
\label{eqn:main-theorem-repul-attr}
\min_{\bfZ}
-\sum_{(i,j)\in [n]^2} \mathbf{P}_{i,j}
\log k (\bfZ_i-\bfZ_j)+\log \mathbf{R}(\bfZ).
\end{equation}
Since $k$ is the Gaussian kernel, this reduces to \[
\min_\bfZ \mathrm{tr}(\bfZ^\top \mathbf{L}(\bfpi) \bfZ)
+\log \mathbf{R}(\bfZ),
\]

where we use the fact that $\mathbb{E}_{\bfW_\bfX\sim \mathbb{P}(\cdot; \bfpi)}[\mathbf{L}(\bfW_\bfX)]
=\mathbf{L}(\bfpi)
$, because the Laplacian operator is linear and $
\mathbb{E}_{\bfW_\bfX\sim \mathbb{P}(\cdot; \bfpi)}(\bfW_\bfX)=\bfpi
$.
\end{proof}

\paragraph{Proof of Theorem \ref{thm:clip}.}
\begin{proof}
Since $\bfW_\bfX\sim \mathbb{P}(\cdot;\bfpi_{\mathbf{A}, \mathbf{B}})$, we know 
$\bfW_\bfX$ has exactly one non-zero entry in each row, denoting the pair that got sampled. 
A notable difference compared to the previous proof is we now have $n_\mathcal{A}+n_\mathcal{B}$ objects in our graph. CLIP deals with this by taking a mini-batch of size $2N$, 
such that $n_\mathcal{A}=n_\mathcal{B}=N$, and adding the $2N$ InfoNCE losses together. We label the objects in $\mathcal{A}$ as $[n_\mathcal{A}]$, and the objects in $\mathcal{B}$ as $\{n_\mathcal{A}+1, \cdots, n_\mathcal{A}+n_\mathcal{B}\}$. 

Notice that $\bfpi_{\mathbf{A}, \mathbf{B}}$ is a bipartite graph, so the edges of objects in $\mathcal{A}$ will only connect to object in $\mathcal{B}$ and vice versa. We can define the similarity matrix in $\cZ$ as $\bfK_\bfZ$, 
where $\bfK_\bfZ(i, j+n_\mathcal{A})=\bfK_\bfZ(j+n_\mathcal{A},i)= k(\bfZ_i-\bfZ_j)$ for $i\in [n_\mathcal{A}], j\in [n_\mathcal{B}]$, and otherwise we set $\bfK_\bfZ(i,j)=0$. 
The rest is same as the previous proof. 
\end{proof}

\paragraph{Proof of Theorem \ref{thm:exponential}.}

\begin{proof}
\label{proof:exponential}
Since the objective function consists of a linear term combined with an entropy regularization, which is a strongly concave function, the maximization problem is a convex optimization problem. Owing to the implicit constraints provided by the entropy function, the problem is equivalent to having only the equality constraint. We then introduce the Lagrangian multiplier $\lambda$ and obtain the following relaxed problem:

$$
\widetilde{E}(\boldsymbol{\alpha})=\psi_{1}-\sum_{i=1}^n \alpha_{i} \psi_{i}+\tau \sum_{i=1}^n \alpha_{i}\log \alpha_{i}+\lambda\left(\boldsymbol{\alpha}^{\top} \mathbf{1}_n-1\right).
$$

As the relaxed problem is unconstrained, taking the derivative with respect to $\alpha_{i}$ yields

$$
\frac{\partial \widetilde{E}(\boldsymbol{\alpha})}{\partial \alpha_{i}}=-\psi_{i}+\tau\left(\log \alpha_{i}+\alpha_{i} \frac{1}{\alpha_{i}}\right)+\lambda=0.
$$

Solving the above equation implies that $\alpha_{i}$ takes the form
$
\alpha_{i}=\exp \left(\frac{1}{\tau} \psi_{i}\right) \exp \left(\frac{-\lambda}{\tau}-1\right).
$ Since $\alpha_{i}$ lies on the probability simplex, the optimal $\alpha_{i}$ is explicitly given by
$
\alpha^{*}_{i}=\frac{\exp \left(\frac{1}{\tau} \psi_{i}\right)}{\sum_{i^{\prime}=1}^n \exp \left(\frac{1}{\tau} \psi_{i^{\prime}}\right)} .
$ Substituting the optimal point into the objective function, we obtain
$$
\begin{aligned}
E\left(\boldsymbol{\alpha}^*\right)  &=\psi_1-\sum_{i=1}^n \frac{\exp \left(\frac{1}{\tau} \psi_{i}\right)}{\sum_{i^{\prime}=1}^n \exp \left(\frac{1}{\tau} \psi_{i^{\prime}}\right)} \psi_{i}+\tau \sum_{i=1}^n \frac{\exp \left(\frac{1}{\tau} \psi_{i}\right)}{\sum_{i^{\prime}=1}^n \exp \left(\frac{1}{\tau} \psi_{i^{\prime}}\right)}\log \frac{\exp \left(\frac{1}{\tau} \psi_{i}\right)}{\sum_{i^{\prime}=1}^n \exp \left(\frac{1}{\tau} \psi_{i^{\prime}}\right)} \\
& =\psi_1 - \tau \log \left(\sum_{i=1}^n \exp \left(\frac{1}{\tau} \psi_{i}\right)\right).
\end{aligned}
$$
Thus, the Lagrangian dual function is given by
\begin{equation*}
-E\left(\boldsymbol{\alpha}^*\right)= -\tau \log \frac{\exp \left(\frac{1}{\tau} \psi_{1}\right)}{\sum_{i=1}^n \exp \left(\frac{1}{\tau} \psi_{i}\right)}.\qedhere
\end{equation*}
\end{proof}



\section{More on Experiments} \label{section: experiment_details}

\paragraph{CIFAR-10 and CIFAR-100} CIFAR-10 ~\citep{krizhevsky2009learning} and CIFAR-100 ~\citep{krizhevsky2009learning} are well-known classic image classification datasets. Both CIFAR-10 and CIFAR-100 contain a total of 60k $32 \times 32$ labeled images of different classes, with 50k for training and 10k for testing. CIFAR-10 is similar to CIFAR-100, except there are 10 different classes in CIFAR-10 and 100 classes in CIFAR-100.

\paragraph{TinyImageNet} TinyImageNet ~\citep{le2015tiny} is a subset of ImageNet ~\citep{deng2009imagenet}. There are 200 different object classes in TinyImageNet, with 500 training images, 50 validation images, and 50 test images for each class. All the images in TinyImageNet are colored and labeled with a size of $64 \times 64$.

\textbf{Pseudo-code.} Algorithm \ref{alg:Training Procedure} presents the pseudo-code for our empirical training procedure.

\begin{algorithm}[!htbp]
\caption{Training Procedure}
\label{alg:Training Procedure}
\begin{algorithmic}[1]
\REQUIRE trainable encoder network $f$, batch size $N$, augmentation strategy \textit{aug}, loss function $L$ with hyperparameters \textit{args}
\FOR {sampled minibatch ${x_i}_{i=1}^N$}
\FORALL{$i \in { 1, ..., N }$}
\STATE draw two augmentations $t_i = \textit{aug}\left(x_i\right) $, $t_i' = \textit{aug}\left(x_i\right) $
\STATE $z_i = f\left(t_i\right)$, $z_i' = f\left(t_i'\right)$
\ENDFOR
\STATE compute loss $\mathcal{L} = L(N, z, z', \textit{args})$
\STATE update encoder network $f$ to minimize $\mathcal{L}$
\ENDFOR
\STATE \textbf{Return} encoder network $f$
\end{algorithmic}
\end{algorithm}

We also provide the pseudo-code for our core loss function used in the training procedure in Algorithm \ref{alg:Core loss}. The pseudo-code is almost identical to SimCLR's loss function, with the exception of an extra parameter $\gamma$.

\begin{algorithm}[!htbp]
\caption{Core loss function $\mathcal{C}$}
\label{alg:Core loss}
\begin{algorithmic}[1]
\REQUIRE batch size $N$, two encoded minibatches $z_1, z_2$, $\gamma$, temperature $\tau$
\STATE $z = \textit{concat}\left(z_1, z_2\right)$
\FOR {$i \in {1, ..., 2N }, j \in {1, ..., 2N}$ }
\STATE $s_{i,j} = \Vert z_i - z_j \Vert_2^{\gamma}$
\ENDFOR
\STATE \textbf{define} $l(i, j)$ \textbf{as} $l(i, j) = - \log \frac{exp\left(s_{i,j}/\tau \right)}{\sum_{k=1}^{2N} \mathbf{1}{[k \ne i]} exp\left(s{i, j} / \tau \right)} $
\STATE \textbf{Return} $\frac{1}{2N} \sum_{k=1}^N\left[l(i, i+N) + l(i+N, i)\right]$
\end{algorithmic}
\end{algorithm}

Utilizing the core loss function $\mathcal{C}$, we can define all kernel loss functions used in our experiments in Table \ref{table: loss definition}. For all $z_i \in z$ with even dimensions $n$, we define $z_{L_i} = z_i\left[0:n/2\right]$ and $z_{R_i} = z_i\left[n/2:n\right]$.

\begin{table}[ht]
\centering
\begin{tabular}{{@{}l|l@{}}}
Kernel  &  Loss function \\ \midrule
Laplacian & $\mathcal{C}\left(N, z, z', \gamma=1, \tau\right)$\\ \midrule
Sum       & $\lambda * \mathcal{C}\left(N, z, z', \gamma=1, \tau_1\right) + (1-\lambda) * \mathcal{C}\left(N, z, z', \gamma=2, \tau_2\right)$  \\ \midrule
Concatenation Sum&$\lambda * \mathcal{C}\left(N, z_L, z'_L, \gamma=1, \tau_1\right) + (1-\lambda) * \mathcal{C}\left(N, z_R, z'_R, \gamma=2, \tau_2\right)$\\ \midrule
$\gamma = 0.5$ & $\mathcal{C}\left(N, z, z', \gamma=0.5, \tau\right)$          \\ 

\end{tabular}

\caption{Definition of kernel loss functions in our experiments}
\label {table: loss definition}
\end{table}

\textbf{Baselines.} We reproduce the SimCLR algorithm using PyTorch Lightning~\citep{PytorchLightning}.

\textbf{Encoder details.}
The encoder $f$ consists of a backbone network and a projection network. We employ ResNet50~\citep{ResNet} as the backbone and a 2-layer MLP (connected by a batch normalization~\citep{ioffe2015batch} layer and a ReLU \cite{nair2010rectified} layer) with hidden dimensions 2048 and output dimensions 128 (or 256 in the concatenation kernel case).

\textbf{Encoder hyperparameter tuning.}
For each encoder training case, we randomly sample 500 hyperparameter groups (sample details are shown in Table \ref{table: Hyperparameter sample}) and train these samples simultaneously using Ray Tune ~\citep{RayTune}, with the ASHA scheduler~\citep{li2018massively}. Ultimately, the hyperparameter group that maximizes the online validation accuracy (integrated in PyTorch Lightning) within 5000 validation steps is chosen for the given encoder training case.

\begin{table}[ht]
\centering

\begin{tabular}{@{}l|l|l@{}}
\midrule
Hyperparameter  & Sample Range & Sample Strategy \\ \midrule
start learning rate & $\left[10^{-2}, 10\right]$ & log uniform \\ \midrule
$\lambda$       & $\left[0, 1\right]$ & uniform \\ \midrule
$\tau$, $\tau_1$, $\tau_2$ & $\left[0, 1\right]$ & log uniform \\ \midrule
\end{tabular}

\caption{Hyperparameters sample strategy}
\label {table: Hyperparameter sample}
\end{table}

\textbf{Encoder training.} 
We train each encoder using the LARS optimizer~\citep{LARSOptimizer}, LambdaLR Scheduler in PyTorch, momentum 0.9, weight decay $10^{-6}$, batch size 256, and the aforementioned hyperparameters for 400 epochs on a single A-100 GPU.

\textbf{Image transformation.} The image transformation strategy, including augmentation, is identical to the default transformation strategy provided by PyTorch Lightning.

\textbf{Linear evaluation.}
The linear head is trained using the SGD optimizer with a cosine learning rate scheduler, batch size 64, and weight decay $10^{-6}$ for 100 epochs. The learning rate starts at $0.3$ and ends at $0$.

\textbf{Moco Experiments.} We also tested our method based on MoCo~\citep{he2019moco}. The results are summarized in Table \ref{tab:results-moco}. Here we choose ResNet18~\citep{ResNet} as the backbone and set a temperature of $0.1$ as default. For our simple sum kernel, we set $\lambda=0.8$. The results show that our method outperforms the original MoCo method.

\begin{table}[thb]
\centering
\caption{MoCo Experiment Results on CIFAR-10 and CIFAR-100.}
\label{tab:results-moco}
\resizebox{\textwidth}{!}{%
\begin{tabular}{@{}c|ccc|ccc@{}}
\toprule
\multirow{3}{*}{Method} & \multicolumn{3}{c|}{CIFAR-10} & \multicolumn{3}{c}{CIFAR-100} \\ \cmidrule(lr){2-4} \cmidrule(lr){5-7} 
                        & 200 epochs & 400 epochs    & 1000 epochs   & 200 epochs & 400 epochs & 1000 epochs         \\ \midrule
MoCo (repro.)         & $76.41 \pm 0.12$    & $80.01 \pm 0.15$          & $84.45 \pm 0.08$    & $\mathbf{47.02 \pm 0.11}$ & $52.50 \pm 0.07$ & $57.62 \pm 0.15$            \\
\midrule
Laplacian Kernel        & ${78.09 \pm 0.10}$    & $\mathbf{83.85 \pm 0.09}$          & $\mathbf{88.34 \pm 0.16}$    & $46.12 \pm 0.22$   & $53.44 \pm 0.17$ & $59.10 \pm 0.14$        \\
Simple Sum Kernel & $\mathbf{78.12 \pm 0.15}$   & $83.23 \pm 0.18$ & $87.50 \pm 0.20$ & $46.65 \pm 0.06$ & $\mathbf{53.62 \pm 0.19}$ & $\mathbf{59.83 \pm 0.12}$\\
\bottomrule
\end{tabular}
}
\end{table}



\section{More Experiments on Synthetic Data}


Consider a scenario with $n$ clusters, each containing $k$ vertices. Let the probability of vertices $u$ and $v$ from the same cluster belonging to $\bfpi$ be $p$. Conversely, for vertices $u$ and $v$ from different clusters, let the probability of belonging to $\pi$ be $q$. We generate the graph $\bfpi$ randomly, based on $p$ and $q$. We experiment with values of $k=100$ and $n=6$ for ease of visualization, embedding all points in a two-dimensional space. Each vertex's initial position originates from a normal distribution. In each iteration, we sample a subgraph of $\bfpi$ uniformly, ensuring each vertex has an out-degree of $1$. We then optimize the corresponding vectors using InfoNCE loss with an SGD optimizer and iterate until convergence. Our experimental setup consists of an SGD learning rate of $1$, an InfoNCE loss temperature of $0.5$, and a batch size of $50$. We evaluate two scenarios with different $p$ and $q$ values: $p=1$, $q=0$, and $p=0.75$, $q=0.2$. The results of these experiments are visualized in Figure \ref{fig:vis-spectral-cluster}. The obtained embeddings exhibit the hallmark pattern of spectral clustering of graph $\bfpi$.

\begin{figure}[!tb]
\centering
\subfigure{
\includegraphics[width=1\textwidth]{Figures/cluster_pi.png}
\label{fig:vis-cluster}
}
\subfigure{
\includegraphics[width=1\textwidth]{Figures/noised_cluster_pi.png}
\label{fig:vis-noised-cluster}
}
\caption{Visualizations of the optimization process using InfoNCE Loss on the vectors corresponding to $\bfpi$. Points of identical color belong to the same cluster within $\bfpi$. To showcase the internal structure of $\bfpi$, we randomly select 10 vertices from each cluster to display the edge distribution of $\bfpi$.}
\label{fig:vis-spectral-cluster}
\end{figure}



\end{document}
