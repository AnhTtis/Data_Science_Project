\begin{table*}[t]
    \centering
    % MP-SSL共使用4个不同器官的数据集进行实验,其中BraTS2020数据集为多模态数据,Spleen数据集只需要分割脾脏,其他数据集均为多目标分割数据集。剪裁大小为训练时输入的大小,PT-decoding size 为预训练时选择的局部重构区域大小。样本大小为数据集的规模。增强为训练时所用的数据增强方式。
    \caption{HybridMIM used a total of 4 datasets with different organs for the experiments. the BraTS2020 dataset is multimodal data, the Spleen dataset only needs to segment the spleen, and all other datasets are multi-target segmentation datasets. The crop size is the size of the input during training. Sample size is the size of the dataset. Augmentation is the data augmentation method used during training.}
    % \vspace{-3mm}
    \label{tab:data_intro}
    \renewcommand\arraystretch{1.3}
    \setlength\tabcolsep{10pt}%调列距
    \resizebox{\textwidth}{!}{
    \begin{tabular}{c c c c c c | c}
    
    \hline
    Dataset & Modality & Object number & Crop size & Sample size & Augmentation \\
    \hline
    BraTS2020 & 4 & Brain tumor (3) & (128,128,128) & 369 & \multirow{4}{*}{\makecell{random flip \\ random rotate \\ random scale \\ random shift}} \\
    % \hline

    BTCV & 1 & Multi-organ (13) & (96,96,96)  & 30 &  \\
    % \hline

    Liver & 1 & Liver and tumor (2) & (96,96,96) &  201 &  \\
    % \hline

    Spleen & 1 & Spleen (1) & (96,96,96) & 61 &  \\
    \hline
    \end{tabular}
    }
% \vspace{-5mm}
\end{table*}

