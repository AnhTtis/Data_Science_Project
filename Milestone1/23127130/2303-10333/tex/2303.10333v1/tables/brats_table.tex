\begin{table}[t]
    \centering
    % BraTS2020数据集包含四个模态,三个分割目标。我们选择UNet和SwinTransformer作为backbone,分别于有监督学习方法跟自监督学习方法对比,结果展示了UniLearn对不同架构的有效性。
    \caption{Quantitative comparison on BraTS 2020 dataset, which contains four modalities and three segmentation targets. }
    % \vspace{-3mm}
    \label{tab:brats_segmentation}
    \renewcommand\arraystretch{1.3}
    \setlength\tabcolsep{10pt}%调列距
    \resizebox{0.48\textwidth}{!}{
    \begin{tabular}{c | c c c c}
    \hline
    Methods & WT & TC & ET & Avg\\
    \hline
    SegresNet & 90.04 & 85.08 & 78.81 & 84.64 \\
    
    UNETR & 89.92 & 84.79 & 79.51 & 84.74\\
    SwinUNETR & 90.08 & 85.19 & 80.01 & 85.09\\
    \hline
    ModelGen & 90.60 & 86.59 & 79.95 & 85.71\\
    TransVW & 90.96 & 86.26 & 80.20 & 85.80 \\
    UNetFormer* & 90.93 & 86.17 & 79.97 & 85.69\\
    UNetFormer & 90.71 & 86.22 & 80.19 & 85.71\\
    \hline
    HybridMIM*(Swin) & \textbf{91.48} & {86.88} & \textbf{80.81} & \textbf{86.39} \\
    HybridMIM*(UNet) & 90.62 & 86.28 & 80.17 & 85.69\\
    \hline
    HybridMIM(Swin) & 90.95 & \textbf{87.34} & 80.71 & 86.33\\
    HybridMIM(UNet) & 90.41 & 86.49 & 80.61 & 85.83 \\
    \hline
    \end{tabular}
    }
    \vspace{-2mm}
\end{table}

