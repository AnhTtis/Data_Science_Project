\section{EXPERIMENTS}
\subsection{Datasets}
% 为了充分的验证我们提出的MP-SSL方法的有效性,我们一共在4个3D医学图像分割数据集上进行实验,分别是BraTS2020,BTCV,MSD Liver,MSD Spleen。每个数据集的参数信息被展示在表1中。
%In order to fully validate the effectiveness of our proposed HybridMIM method, we conduct experiments on a total of four 3D medical image segmentation datasets, BraTS2020, BTCV, MSD Liver, and MSD Spleen, respectively. The parameter information of each dataset is presented in Table \ref{tab:data_intro}.

\textbf{Pre-training Dataset:}
We collected a total of 1897 CT images to construct our pre-training dataset. They come from 4 public CT image datasets, including the ATM22~\cite{atm22} (150 cases of chest), luna16~\cite{setio2017validation} (888 cases of lung), covid-19~\cite{roth2022rapid} (448 cases of lung) and FLARE21~\cite{MedIA-FLARE21} (411 cases of abdomen) datasets. 
%We split 20\% of each dataset for validation in the pre-training stage. 
% 80\% of the 1897 CT images collected for pre-training and 20\% for validation

%% BraTS2020数据集共包含369个病例的脑部图像,每个病例图像有4个模态分别是T1、T1Gd、T2、T2-FLAIR和3个分割目标分别为whole tumor (WT)、enhancing tumor (ET), and tumor core (TC)。由于输入图像尺寸较大,因此对每个病例我们会裁剪一块(128,128,128)的补丁在预训练和训练过程中。在预训练过程中,局部重构区域大小为(96,96,96)。
\textbf{BraTS2020 dataset}: BraTS2020 dataset~\cite{menze2014multimodal,bakas2017advancing} contains a total of 369 brain MRI images, and each case image has 4 modalities (namely T1, T1Gd, T2, T2-FLAIR) and 3 segmentation targets (WT: whole tumor, ET: enhancing tumor, TC: tumor core). All the data have been resampled to the same spacing (1.0, 1.0, 1.0). Due to the large size of the input image, we crop the training sub-volume of a size of (128,128,128). 

% BTCV数据集共包含30个病例的3D腹部多器官图像,每个病例图像有1个模态和13个器官分割目标。对每个病例我们会裁剪一块(96,96,96)的补丁在预训练和训练过程中。在预训练过程中,局部重构区域大小为(64,64,64)。
\textbf{BTCV dataset}: BTCV~\cite{BTCV} contains a total of 30 cases of 3D abdominal multi-organ CT images, with 1 modality and 13 organ segmentation targets per case. All cases are resampled to the same spacing (1.5, 1.5, 2.0). We crop the training sub-volumes of a size of (96,96,96).

% MSD Liver数据共包含201个病例的3D肺部图像,每个病例存在1个模态和2个分割目标(肺部和肺部肿瘤),MSD Spleen数据集包含了61个病例的3D脾脏图像,每个病例有1个模态和1个分割目标(脾脏)。对每个病例我们会裁剪一块(96,96,96)的补丁在预训练和训练过程中。在预训练过程中,局部重构区域大小为(64,64,64)。
We further use two task datasets from MSD dataset~\cite{antonelli2022medical}. \textbf{MSD Liver} contains a total of 201 cases of 3D liver CT images, with 1 modality and 2 segmentation targets (liver, liver tumor) per case. The \textbf{MSD Spleen} contains 61 cases of 3D spleen CT images with 1 modality and 1 segmentation target (spleen) per case. For these two datasets, we resample all the cases to the same spacing (1.5, 1.5, 2.0) and crop the training sub-volumes of a size of (96,96,96).

\subsection{Evaluation Modes:} 
% 为了避免数据泄漏,预训练与训练数据保持一致。所有数据均切分80%作为预训练/训练集,20%作为验证集。
To make a comprehensive comparison, we adopt two evaluation modes to do pre-training and finetuning:
%on the above four downstream segmentation tasks.
%, including BraTS2020, BTCV, MSD-Spleen and MSD-Liver.
\begin{itemize}
    \item Generic pre-training mode: A generic model is pre-trained on our collected pre-training dataset, and finetuned on the four segmentation tasks.
    \item Task-specific pre-training mode: A task-specific model is pre-trained and finetuned on each segmentation task dataset. To avoid data leakage, we do self-supervised pre-training on the training set of the downstream task datasets, which are all split 80\% for training.
\end{itemize}
In finetuning stage for segmentation, we utilize the pre-trained encoder and decoder by replacing the last layer with a new random $c \times 1 \times 1 \times 1$ convolution layer, where $c$ is the number of segmentation targets.


\subsection{Implementation Details}
%\textbf{Basic setup for pre-training.}
% 我们通过多组对照实验选出 最优的一级区域,二级区域与重建的局部区域的大小。看 V. Result, section A for details. 
The default options for the components of HybridMIM are: the two-level masking strategy with a mask ratio of 0.4, the first-level sub-volume size of $32\times 32 \times 32$, the second-level patch size of $16\times 16 \times 16$; the partial region prediction with the reconstructed region size of $96\times 96 \times 96$ for BraTS2020, and $64\times 64 \times 64$ for BTCV, MSD Liver and MSD Spleen. 
%%
These settings are determined via multiple control experiment; See Figure~\ref{fig:pretraining_setting} for details. 
%%
% 对于 MSD 肝脏、MSD 脾脏和 BTCV 实验,我们使用分辨率为 96 × 96 × 96 的随机裁剪图像,并且预训练实验使用每个 GPU 4 个批量大小(使用 $96\times96\times96$ 补丁和 $64\times64 \times64$ 解码大小)。对于 BraTS2020 数据集,我们使用128作为随机裁剪大小,并且每个GPU使用2个批量大小。
For the pre-training dataset, BTCV, MSD liver and MSD spleen, we use randomly cropped images with a resolution of 96 × 96 × 96 and a batch size of 4 per GPU. For BraTS2020 dataset, we use $128\times128\times128$ as the random crop size and a batch size of 2 per GPU. As each image case in BraTS2020 contains 4 modalities, we concatenate each modality in channel dimension at the input of the network.

% 我们使用Pytorch1.12.1-cuda11.3与Monai1.0.0作为基本框架。使用4 个NVIDIA A100 Tensor Core GPU和2个NVIDIA Tesla V100 GPU作为运行环境。
Our model is implemented in Pytorch 1.12.1-cuda11.3 and Monai 1.0.0. 
%所有实验均应用随机翻转、旋转、强度缩放和移位的数据增强变换。优化器使用AdamW以及 1e-4 的初始学习率、 1e-5 的衰减和50个epochs的warmup。
In both pre-training and finetuning, we use an AdamW\cite{loshchilov2017decoupled} optimizer along with a $cosine$ learning rate scheduler (an initial learning rate of 1e-4, a decay of 1e-5, and a warmup\cite{he2016deep} of 50 epochs).
% 对于BraTS2020数据集,我们共运行300个epoch,对于BTCV数据集,我们共运行2000个epoch,对于MSD Liver与Spleen数据集,运行600个epoch。
We run a totoal of 10w step for all pre-training experiments, and
in finetuning, we run 300 epochs for the BraTS2020 dataset, 2000 epochs for the BTCV dataset, and 600 epochs for the MSD Liver and Spleen datasets.
%%
No data augmentation is applied at the pre-training stage, while a light strategy is used in the finetuning: random flip, rotation, intensity scaling and shifts with probabilities of 0.2, 0.2, 0.1, and 0.1, respectively.
%with a probability of 0.2 in each dimension, random rotations 90 degree with a probability of 0.2, intensity scaling and shifts with a probability of 0.1.  
%%
All experiments are conducted on a cloud computing platform with four NVIDIA A100 Tensor Core GPUs and two NVIDIA Tesla V100 GPUs.

% 此外,由于对比学习需要正负样本,批量大小大于1并且网络中必须添加dropout层。
%In addition, since comparison learning requires positive and negative samples, the batch size is larger than 1 and a dropout layer must be added to the network.

\begin{table*}[t]
    \vspace{-3mm}
    \caption{Quantitative comparison on BTCV muti-organ sementation dataset. Note: Spl: spleen, RKid: right kidney, LKid: left kidney, Gall: gallbladder, Eso: esophagus,Liv: liver, Sto: stomach, Aor: aorta, IVC: inferior vena cava, PSV: portal and splenic veins, Pan: pancreas, Rag: right adrenal glands, Lag: left adrenal glands. The task-specific pre-trained models are marked with *.}
    \centering 
    \label{tab:btcv_segmentation} %\footnotesize
    \renewcommand\arraystretch{1.3}
    \setlength\tabcolsep{4pt}%调列距
    \resizebox{0.9\textwidth}{!}{
    \begin{tabular}{c | c | c c c c c c c c c c c c c }
    % \toprule[1pt]
    % Dataset & \multicolumn{11}{c}{BraTS2020} \\
    % \hline
    % \multirow{2}{*}{Methods} & \multirow{2}{*}{\makecell{Param\\(M)}} & \multirow{2}{*}{\makecell{FLOPs\\(G)}} &  & \multicolumn{2}{c}{WT} &  & \multicolumn{2}{c}{TC} & &  \multicolumn{2}{c}{ET} &  & & \multicolumn{2}{c}{Ave} \\
    % \cline{5-6} \cline{8-9} \cline{11-12} \cline{15-16} 
    \hline
    Methods & Avg & Spl & RKid & LKid & Gall & Eso & Liv & Sto & Aor & IVC & PSV & Pan & Rag & Lag \\
    \hline
    Segresnet & 81.29 & 94.55 & 93.35 & 93.41 & {\color{green}75.59} & 73.44 & 95.96 & 80.89 & 89.00 & 84.24 & 71.48 & 79.12 & 65.51 & 60.07 \\
    
    UNETR & 81.33 & 94.66 & {94.27} & 94.09 & 65.23 & 74.20 & {\color{red}96.90} & 80.06 & 89.16 & 84.04 & {73.46} & 80.32 & 64.30 & {\color{green}66.65} \\
    SwinUNETR &81.81 & 94.72 & 94.23 & 93.89 & 66.60 & 74.54 & 96.63 & 78.77 & 89.79 & 83.64 & {\color{blue}74.69} & {\color{red}82.19} & {67.76} & {66.18} \\
    \hline
     ModelGen & 81.45 & 91.99 & 93.52 & 91.81 & 65.11 & {\color{red}76.14} & 95.98 & {\color{red}86.88} & 89.29 & 83.59 & 71.79 & {81.62} & {\color{green}67.97} & 63.18 \\
    TransVW & {82.27} & {95.56} & 94.20 & {\color{red}94.59} & 70.42 & 73.25 & 96.51 & {\color{blue}85.65} & {\color{blue}90.44} & {\color{blue}85.80} & 73.19 & {\color{blue}81.91} & 66.17 & 61.62\\
    UNetFormer* & 82.18 & 94.27 & 94.15 & 93.80 & {\color{blue}75.86} & {75.05} & {96.72} & 81.74 & 90.13 & 83.32 & 72.41 & 79.90 & 67.06 & 63.95 \\
    UNetFormer & 82.44 & {\color{green}95.90} & {\color{red}94.61} & 94.28 & 71.51 & 75.08 & 96.51 & 81.46 & 90.06 & 85.84 & {\color{red}75.34} & 80.72 & {\color{blue}68.50} & 61.78\\
    
    \hline
    HybridMIM*(Swin) & {82.41} & {\color{red}95.95} & {\color{blue}94.60} & {\color{green}94.36} & 65.79 & {\color{green}75.46}  & {\color{green}96.74} & 82.54 & {90.14} & {84.80} & {\color{green}74.02} & 80.36 & {67.87} & {\color{red}68.47}\\
    HybridMIM*(UNet) & {\color{green}82.62} & {95.27} & {94.25} & {94.15} & {\color{red}78.67} & 74.24 & {96.68} & {\color{green}83.31} & {\color{green}90.25} & {\color{red}85.82} & 73.07 & 80.24 & 65.02 & 62.94 \\
    \hline
    HybridMIM(Swin) & {\color{blue}82.63} & {\color{blue}95.95} & 94.25 & 94.26 & 72.56 & 74.14 & {\color{blue}96.78} & 79.25 & 90.17 & 85.10 & 73.64 & {\color{green}81.83} & {\color{red}69.21} & {\color{blue}66.88}\\
    HybridMIM(UNet) & {\color{red}83.00} & 95.68 & {\color{green}94.43} & {\color{blue}94.40} & 74.33 & {\color{blue}75.84} & 96.72 & 82.92 & {\color{red}90.86} & {\color{red}86.43} & 72.96 & 81.16 & 66.98 & 66.25\\
    \hline
    \end{tabular}
    }
\end{table*}



% \subsection{Evaluation Metrics}
% $Dice$ score and 95\% Hausdorff Distance ($HD 95$) are adopted for quantitative comparison. $HD 95$ is based on the calculation of the $95^{th}$ percentile of the distances between boundary points in $X$ and $Y$. 
% \begin{equation}
% Dice=\frac{2|A \cap B|}{|A|+|B|},
% \end{equation}

% \begin{equation}
% HD=\max \left\{\sup _{x \in X} \inf_{y \in Y} d(x, y), \sup _{y \in Y} \inf_{x \in X} d(y, x)\right\},
% \end{equation}

% where A and B denote the ground truth and prediction of
% voxel values. X and Y denote ground truth and prediction surface point sets. $\sup$ represents the supremum, $\inf$ the infimum.

\subsection{Benchmarking}

% 为了使实验结果更有说服力,MP-SSL共与其他六种方法进行对比。这些方法中包含了不同的网络架构,同时也包含了SOTA的监督学习方法与自监督学习方法。
For a thorough evaluation, HybridMIM is compared with SOTA SSL methods as well as fully supervised learning methods, which both cover  CNN and transformer architectures. 

%%
\textbf{Self-supervised methods:} We compare HybridMIM with Models Genesis~\cite{zhou2021models}and TransVW~\cite{haghighi2021transferable}, which are the most recent multi-task SSL methods for 3D medical imaging. 
%%
We also examine an masked image modeling-based SSL method, UNetFormer~\cite{wang2022unetformer}, which is built upon the SwinTransformer architecture. 
% 对于TransVW和ModelGenesis方法,我们使用的是官方开源的权重。对于UNetFormer方法,我们复现了两种预训练模型,UNetFormer*表示在我们收集的1897个CT图像上预训练的通用模型,而UNetFormer表示任务特定的预训练模型,预训练数据与下游任务一致。
As ModelGenesis and TransVW officially release their pre-trained model weights, we skip the pre-training step and conduct finetuning on the four downstream segmentation tasks. 
%%
%Note that since these two methods both utilize a pre-training dataset of 5050 CT images, larger than ours, it is reasonable to start with their models for finetuning.
%%
For UNetFormer, we use its public codes and experiment with it in both evaluation modes. 

%replicate two pre-trained models. UNetFormer* denotes the generic model pre-trained on our collection of 1897 CT images, while UNetFormer denotes the task-specific pre-trained model with pre-trained data consistent with the downstream task.

%%
% 对于监督学习方法,我们选择了SegresNet,UNETR和SwinUNETR作为对比方法。其中SegresNet是一个基于卷积神经网络且性能良好的架构。UNETR与SwinUNETR分别采用ViT transformer与SwinTransformer结构作为编码器,这可以更好的建模全局特征。他们均是最近表现良好的3D医学图像分割方法。
\textbf{Supervised methods:}
We also make comparison with state-of-the-art supervised segmentation methods in medical imaging. 
%%
SegresNet~\cite{myronenko20183d} is a CNN-based architecture with good performance. 
%%
UNETR~\cite{hatamizadeh2022unetr} and SwinUNETR~\cite{hatamizadeh2022swin} are the most recent transformer-based methods for 3D medical image segmentation, which use vision transformer and SwinTransformer structures as encoders, respectively. Here, we use their public codes in the experiments.
%%
% 此外,HybridMIM适用于不同的网络架构。因此我们分别使用UNet和SwinUNETR作为基础架构,并且分别预训练通用模型与任务特定模型,来验证HybridMIM方法对其性能的提升。
%Moreover, HybridMIM applies to different network architectures. Therefore, we use UNet and SwinUNETR as the infrastructure and pre-train the generic and task-specific models, respectively, to verify the performance improvement of the HybridMIM approach.


