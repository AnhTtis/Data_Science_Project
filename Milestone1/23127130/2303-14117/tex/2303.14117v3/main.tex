\documentclass[prd,preprint,linenumbers,amssymb,amsmath,aps,nofootinbib,superscriptaddress]{revtex4}
\usepackage{graphicx}% Include figure files
%\usepackage{dcolumn}% Align table columns on decimal point
\usepackage{bm}% bold math
\usepackage{subfigure}
\usepackage{amssymb}
\usepackage{color,soul}
\usepackage{amsmath}

\begin{document}


\title{SKA sensitivity for possible radio emission from  dark matter in Omega Centauri}


\author{Guan-Sen Wang}
\affiliation{Key Laboratory of Dark Matter and Space Astronomy, Purple Mountain Observatory, Chinese Academy of Sciences, Nanjing 210023, China}
\affiliation{School of Astronomy and Space Science, University of Science and Technology of China, Hefei, Anhui 230026, China}


\author{Zhan-Fang Chen}
\affiliation{Key Laboratory of Dark Matter and Space Astronomy, Purple Mountain Observatory, Chinese Academy of Sciences, Nanjing 210023, China}
\affiliation{School of Astronomy and Space Science, University of Science and Technology of China, Hefei, Anhui 230026, China}




\author{Lei Zu \footnote{Corresponding author: zulei@pmo.ac.cn}}
\affiliation{Key Laboratory of Dark Matter and Space Astronomy, Purple Mountain Observatory, Chinese Academy of Sciences, Nanjing 210023, China}
\affiliation{School of Astronomy and Space Science, University of Science and Technology of China, Hefei, Anhui 230026, China}


\author{Hao Gong}
\affiliation{School of Mathematics and Physics, Bengbu University, Bengbu, Anhui 233030, China}




\author{Lei Feng \footnote{Corresponding author: fenglei@pmo.ac.cn}} 
\affiliation{Key Laboratory of Dark Matter and Space Astronomy, Purple Mountain Observatory, Chinese Academy of Sciences, Nanjing 210023, China}
\affiliation{School of Astronomy and Space Science, University of Science and Technology of China, Hefei, Anhui 230026, China}
\affiliation{Joint Center for Particle, Nuclear Physics and Cosmology,  Nanjing University -- Purple Mountain Observatory,  Nanjing  210093, China}


%\author{Qiang Yuan \footnote{Corresponding author: yuanq@pmo.ac.cn}}
%\affiliation{Key Laboratory of Dark Matter and Space Astronomy, Purple Mountain Observatory, Chinese Academy of Sciences, Nanjing 210023, China}
%\affiliation{School of Astronomy and Space Science, University of Science and Technology of China, Hefei, Anhui 230026, China}




\author{Yi-Zhong Fan \footnote{Corresponding author: yzfan@pmo.ac.cn}}
\affiliation{Key Laboratory of Dark Matter and Space Astronomy, Purple Mountain Observatory, Chinese Academy of Sciences, Nanjing 210023, China}
\affiliation{School of Astronomy and Space Science, University of Science and Technology of China, Hefei, Anhui 230026, China}


%\date{December 2022}



\begin{abstract}

Omega Centauri, the largest known globular cluster in the Milky Way, is believed to be the remains of a dwarf galaxy's core. Giving its potential abundance of dark matter (DM), it is an attractive target for investigating the nature of this elusive substance in our local environment. Our study demonstrates that by observing Omega Centauri with the SKA for 1000 hours, we can detect synchrotron radio or Inverse Compton (IC) emissions from the DM annihilation products. It enables us to constrain the cross-section of DM annihilation down to  $\sim {\rm 10^{-30}~cm^3~s^{-1}}$ for DM mass from several $\rm{GeV}$ to $\rm{100~GeV}$, which is much stronger compared with other observations. Additionally, we explore the axion, another well-motivated DM candidate, and provide stimulated decay calculations. It turns out that the sensitivity can reach  $g_{\rm{a\gamma\gamma}} \sim 10^{-10} ~\rm{GeV^{-1}}$ for $2\times 10^{-7} ~\rm{eV} < m_a < 2\times 10^{-4} ~\rm{eV}$.
\end{abstract}

\maketitle

\section{Introduction}
Modern astrophysics is still grappling with the enigmatic nature of DM. Various theories have been put forward to explain its characteristics, including Weakly-Interacting Massive Particles (WIMPs) and axions. WIMPs are a particularly well-motivated model, as they can account for the correct relic abundance of DM. As a result of WIMP annihilation. Particles such as electrons, positrons, and antiprotons can be detected by space-based telescopes~\cite{q1,q2,q3,q4}. Furthermore, these secondary cosmic ray particles would interact with their surroundings, generating radio emissions through synchrotron and IC effects~\cite{Chen_2021,Arpan_2020,Cembranos_2020,huang}.

Moreover, axions have gained popularity as a potential solution to the strong CP problem and are considered as a viable candidate for DM~\cite{Peccei_1977,Weinberg_1978,new1,new2,new3}. Axions, as well as axion-like particles, exhibit a wide range of masses, spanning from $\rm{\mu}$eV to radio frequency scales. As a result of the coupling between axions and photons, axion decay produces line radio emissions~\cite{Andrea_2018,Andrea2_2018}. Additionally, background photons can stimulate this axion decay~\cite{Andrea_2018,Andrea2_2018}. The radio signal is raised by both the synchrotron and IC of WIMPs and the decay of axions, indicating that radio observation is a potent technique for detecting DM particles.

The Square Kilometre Array (SKA) is a highly promising instrument in the field of radio astronomy~\cite{ska}, renowned for its exceptional resolution and sensitivity. It is an indispensable tool for detecting radio emissions from the annihilation products of DM particles. Extensive research has been conducted on various sources, such as the galactic center of the Milky Way, Draco (a dwarf galaxy), A2199 (a galaxy cluster), and Dragonfly 44 (a DM-rich galaxy)~\cite{Andrea_2018,Andrea2_2018,Cembranos_2020,Chen_2021,Wang_2021}. Also, DM in globular clusters has been widely discussed~\cite{1,2,3,4,5,6,7,8,9}. Omega Centauri, the largest globular
cluster in the Milky Way, is probably a remnant core of a dwarf galaxy with a considerable amount of
DM. Some previous work have revealed Omega Centauri with $J_{\rm factor} \sim 10^{22} ~\rm{GeV^2~cm^{-5}}$~\cite{Addy_2021, Javier_2021}. In addition, the gamma-ray emission from Omega Centauri has also been widely discussed as a potential outcome of WIMP annihilation \cite{Javier_2021,Profumo2017,Conrad2017}. To confirm whether the gamma-ray emission is indeed due to WIMP annihilation is also one of the motivations of this article.

In this study, we computed the adjoint radio emissions of DM in Omega Centauri, encompassing the continuum secondary synchrotron and IC radiation of Weakly Interacting Massive Particles (WIMPs) or the decay line of axions, with the sensitivity required for detection using SKA within 1000 hours.


This work is organized as follows: In Section II, we present some background information, introduce the properties of Omega Centauri, and calculate the SKA sensitivity for both the continuum and line spectrum. Then, we analyze the radio emission from the secondary synchrotron and IC of WIMPs in Sections III, IV, and V before discussing the axion decay model in Section VI. Lastly, we summarize our findings and discuss the results in the final section.

\section{Omega Centauri and SKA sensitivity}

Omega Centauri is one of the most massive and brightest globular clusters in the Milky Way, located $\rm 5.4~kpc$ away from Earth with celestial coordinates
($201.7^{\circ}$, $-47.56^{\circ}$). Its high mass core and half-light radius ($\rm 7~pc$) as reported in Ref.~\cite{Baumgardt_2017,Harris_1996}, have led scientists to speculate that it is a remnant core of a captured dwarf galaxy~\cite{Bekki_2003}. Observations of the stellar kinematics in Omega Centauri reveal a component of DM estimated to be around $10^{6}~M_{\bigodot}$, which supports the assumption that it is a captured dwarf galaxy remnant~\cite{Addy_2021,Wang_2021}. Furthermore, there is evidence suggesting that the gamma-ray emission in Omega Centauri can be explained by DM annihilation~\cite{Javier_2021,Brown_2019}.

For DM density in Omega Centauri, we use the NFW profile following \cite{Wang_2021}:

\begin{eqnarray}
\rho=\frac{\rho_{\rm s}}{\frac{r}{r_{\rm s}}(1+\frac{r}{r_{\rm s}})^2}.
\label{eqnfw}
\end{eqnarray}
While the total mass of DM in Omega Centauri is consider with a high degree of uncertainty significant uncertainty. Previous studies have attempted to measure its total DM mass through velocity dispersion, different groups have gotten the results with about one order deviation, $M_{\rm{DM}} = 2.154\times10^6~M_{\bigodot}$ in~\cite{Watkins2015} and $2\times10^5~M_{\bigodot}$ in~\cite{Brown_2019}. In our work, we choose a moderate estimation for the total mass of DM, which leads us to estimate that the total DM mass in Omega Centauri is approximately $10^{6}~M_{\bigodot}$~\cite{Addy_2021,Wang_2021}. Also, we have chosen the Omega Centauri's parameters as $\rho_{\rm s} = 7650.59~ M_{\bigodot}~\rm {pc^{-3}}$ and $r_s = 1.63~\rm pc$, which is consistent with the observed velocity dispersion of stars~\cite{Wang_2021,Brown_2019}. To further explore the impact of the density profile, we consider two additional parameter sets: $\rho_{\rm s} = 27860.5~ M_{\bigodot}~\rm {pc^{-3}}$, $r_s = 1.0~\rm pc$, and $\rho_{\rm s} = 4391.65~ M_{\bigodot}~\rm {pc^{-3}}$, $\rm r_s = 2.0~pc$ \cite{Wang_2021}. The velocity dispersion of DM is expected to be around $30~\rm km/s$~\cite{Wang_2021}, which is much lower than that of the Milky Way. Also, Omega Centauri is much closer to us than the dwarf galaxy. This feature makes Omega Centauri an ideal target to probe DM properties. Furthermore, the low gas density nature of the globular cluster ensures a negligible background in the radio frequency range. Despite no prior reports of radio emissions from Omega Centauri, we anticipate that the Square Kilometre Array (SKA), a highly sensitive radio telescope array with advanced energy resolution capabilities, may detect the potential radio signals.


SKA is a radio telescope under construction with high sensitivity and energy resolution~\cite{ska}. SKA1-LOW and SKA1-MID cover the frequency range from 50 MHz to 50 GHz~\cite{ska}. Previous radio telescopes, such as the Murchison Widefield Array, have achieved $\rm \sim O(mJy)$ for null-detection~\cite{Arpan_2020}. While SKA, in the near future, will be able to detect signals with $\rm {O(\mu Jy)}$ sensitivity.

The sensitivity of the SKA for specific observations can be described by the following \cite{ska}:
\begin{eqnarray}
S_{\rm {min}}=\frac{2 k_b S_D T_{\rm sys}}{\eta_s A_e (\eta_{\rm pol} t \Delta \nu)^{0.5}}.
\label{smin}
\end{eqnarray}
Here, $k_b$ represents the Boltzmann constant, and $S_D$ is a degradation factor with a value of 2 for continuum images and 2.5 for line images. The system efficiency, represented by $\eta_{\rm{s}} = 0.9$, the number of polarization states, represented by $\eta_{\rm{pol}} = 2$, the total observation time $t$, and the channel width $\Delta \nu$ are also included in the equation. The minimum detectable flux is determined by the ratio of the effective collecting area $A_e$ and the total system noise temperature $T_{\rm sys}$, which can be found in \cite{ska} for both SKA1 and SKA2 phases.

To calculate the minimum detectable flux for our analysis, we utilize the longest integration time of 1000 hours, a channel width of $\Delta \nu =300 ~\rm{MHz}$ for the continuum image, and a channel width of $\Delta \nu =10^{-4} \nu$ for the line image~\cite{ska}. These values are chosen based on the velocity dispersion in Omega Centauri, which is about $\sim 30~\rm{km/s}$ \cite{Cembranos_2020, Wang_2021}.


\section{Calculation formulas}
The synchrotron and IC radiation of charged products from DM annihilation have been widely discussed. Several tools have been developed to calculate this processes, such as RX-DMFIT \cite{rx-dmfit}. In this draft, we adopt the RX-DMFIT package to calculate the flux of synchrotron and IC emission.

The equilibrium spectrum of CR $e^{+}/e^{-}$  can be calculated by following equation:
\begin{eqnarray}
\frac{\partial}{\partial t}\frac{\partial n_e}{\partial E}={\bf \bigtriangledown}[D(E,{\bf r}){\bf \bigtriangledown} \frac{\partial n_e}{\partial E}]+\frac{\partial}{\partial E}[b(E,{\bf r})\frac{\partial n_e}{\partial E}]+Q(E,{\bf r}),
\end{eqnarray}
where $n$ is the equilibrium electron density, $Q(E,{\bf r})$ is the injected electron source term, $D(E,{\bf r})$ is the diffusion coefficient and $b(E,{\bf r})$ is the energy loss term. For the steady-state solution, the left side of the above equation is equal to zero, and it can be solved numerically using the Greens function method \cite{green1,green2} in the RX-DMFIT package.

The source term here is the contribution of WIMP annihilation described by the following formula:
\begin{eqnarray}
Q(E,r)=\frac{<\sigma v>\rho_{\chi}^2(r)}{2M_{\chi}^2}\frac{dN}{dE_{\rm inj}},
\end{eqnarray}
where $dN/dE_{\rm inj}$ is the $e^{+}/e^{-}$ injection spectrum per WIMP annihilation event, and it is derived from DarkSUSY package for a given DM mass and annihilation channel. For the diffusion coefficient, we choose a simplified power law form as follows:
\begin{eqnarray}
D(E)=D_0E^\gamma,
\end{eqnarray}
where $D_{0}$ is the diffusion coefficient in the range of $\rm 10^{27} \sim 10^{29}~cm^2~s^{-1}$ \cite{d01,d02} and we choose $\gamma=1/3$ in this work though the real situation may be more complicated \cite{DAMPE2022BC}.

The energy loss term contains several contributions: synchrotron, IC, Coulomb, and bremsstrahlung losses which are described as follows
\begin{eqnarray}
b(E,{\bf r})&=&b_{\rm IC}(E)+b_{\rm Synch}(E,{\bf r})+b_{Coul}(E)+b_{\rm Brem}(E) \nonumber \\
&=&b_{\rm IC}^0E^2+b_{\rm Synch}^0B(r)^2E^2+b_{\rm Coul}^0n_e(1+log(\frac{E/m_e}{n_e})/75)\\
&+&b_{Brem}^0n_e(log(\frac{E/m_e}{n_e})+0.36), \nonumber
\end{eqnarray}
where $n_e$ is the mean number density of thermal electrons and it is about $\rm 0.05~cm^{-3}$ for GCs~\cite{47ne}. For magnetic field strength $B$, we conservatively choose $\sim 1\rm{\mu G}$ here. The energy loss coefficients are taken to be $b_{\rm syn}^0 \simeq 0.0254,~b_{\rm IC}^0 \simeq 0.25,~b_{\rm brem}^0 \simeq 1.51 $ and $ b_{\rm Coul}^0 \simeq 6.13$, all in units of $\rm 10^{-16}~GeV/s$~\cite{bloss}.

Then the integrated flux density spectrum of the synchrotron and IC can be easily calculated through the following equation
\begin{eqnarray}
S(\nu)=\int_{\Omega}d\Omega \int_{los} (j_{\rm syn}(\nu,l)+j_{\rm IC}(E_{\gamma },l))dl,
\end{eqnarray}
where
\begin{eqnarray}
j_{\rm syn}(\nu,r)=2\int_{m_e}^{M_{\chi}}dE \frac{dn_e}{dE}(E,r)P_{\rm syn}(\nu,E,r),
\end{eqnarray}
\begin{eqnarray}
j_{\rm IC}(E_{\gamma },r)=2\int_{m_e}^{M_{\chi}}dE \frac{dn_e}{dE}(E,r)P_{\rm IC}(E,E_{\gamma }).
\end{eqnarray}
Here $P_{\rm syn}$ and $P_{\rm IC}$ represent the standard synchrotron power and IC power, which refer to the number of photons of a specific frequency that an electron with a certain energy can produce. Further details about these components can be found in~\cite{syn}.

Here, we combine synchrotron and IC radiation, which provides a more accurate depiction of the scenario compared to the approach that calculates the impact of each component separately. This is particularly important when the two components have equal strengths, as their combined effect can be twice as large. These differences can result in variations in the constraints on DM.

The IC radiation has two components: IC with Cosmic Microwave Background (CMB) photons and IC with starlight. However, since the frequency of starlight is already higher than the radio band, the resulting frequency after IC is too high to be detected by SKA. Consequently, the impact of starlight photons on the measurement is negligible.



\section {THE ENERGY SPECTRUM DISTRIBUTION}



\begin{figure}[htb]
\includegraphics[width=12cm]{SED_differentrho.pdf}
\caption{The SED of synchrotron and IC emission for $m_\chi = 10~\rm{GeV}$, $D_0 = 3\times10^{28}    ~\rm {cm^{2}~s^{-1}}$, $<\sigma v>=10^{-30} ~\rm{cm^{3}~s^{-1}}$, $B=1~\rm {\mu G}$ and various choice of NFW profile parameters: $\rho_{\rm s} = 7650.59~M_{\bigodot}~\rm {pc^{-3}} $, $r_s $=1.63 pc (red line), $\rho_{\rm s} = 27860.5~M_{\bigodot}~\rm {pc^{-3}} $, $r_s $=1.0 pc (blue line) and $\rho_{\rm s} = 4391.65~ M_{\bigodot}~\rm {pc^{-3}} $, $r_s $=2.0 pc (yellow line) . The cyan (green) region represents the sensitivity of SKA phase1 (SKA phase2) with 1000 hours of observation, similarly hereinafter.}
\label{SED_differentrho}
\end{figure}


To account for the uncertainty of the DM profile in Omega Centauri, we investigated the effects of different NFW profile parameters as shown in Fig.~\ref{SED_differentrho}. 
Our analysis indicates that there are no significant differences between the different choices of $r_s$ and $\rho_s$. Therefore, we adopt a typical profile with $\rho_{\rm s}=7650.59~M_{\bigodot}~\rm{pc^{-3}}$ and $r_s=1.63$~pc for the following analysis.

\begin{figure}[htb]
\includegraphics[width=10cm]{SED_differentmass.pdf}
\caption{The SED of synchrotron and IC emission for $D_0 = 3\times10^{28}    ~\rm {cm^{2}~s^{-1}}$, $<\sigma v>=10^{-30} ~\rm{cm^{3}~s^{-1}}$, $B=1~\rm {\mu G}$ and various choice of DM mass: 1 MeV (blue line), 1 GeV (red line), and 1 TeV (yellow line).} %The cyan (green) band represents the sensitivity of SKA phase1 (SKA phase2) with 1000 hours of observation. }
\label{SED_differentmass}
\end{figure}

\begin{figure}[htb]
\subfigure[$B=B_{0}$]{\includegraphics[width=8.1cm]{SED_differentB0_bb1.pdf}}
\subfigure[$B=B_{0}\times e^{-(r/r_{c})}$]{\includegraphics[width=8.1cm]{SED_differentB0_bb0.pdf}}
\caption{The SED of synchrotron and IC emission for $m_\chi =10~\rm {GeV}$, $D_0 = 3\times10^{28}~\rm {cm^{2}~s^{-1}}$, $<\sigma v>=10^{-30} ~\rm{cm^{3}~s^{-1}}$ and various choice of magnetic field strength in constant magnetic field model (a) and exponentially decaying magnetic field model (b). }%The cyan (green) band represents the sensitivity of SKA phase1 (SKA phase2) with 1000 hours of observation.}
\label{SED_differentB}
\end{figure}



The spectral energy distribution (SED) of radio emissions relies heavily on the mass of the DM particle. That is because the energy of the electron products and the corresponding photon flux is determined by the DM mass. In Fig.~\ref{SED_differentmass}, we present the results of SEDs for various $m_\chi$ with fixed diffusion coefficient, magnetic field strength and cross-section. 
%The SKA radio frequency test can detect WIMP masses ranging from MeV to TeV. 
For MeV DM, IC emission is the primary contributor, while synchrotron emission dominates for heavier DM particles.

The diffusion coefficient and magnetic field strength also significantly affect the final radio flux, as they impact both the charged particle propagation and photon emission processes. Their influences on SED are presented in Fig.~\ref{SED_differentB} and Fig.~\ref{SED_differentD0}. The synchrotron emission is strongly dependent on the magnetic field since the synchrotron process is produced by the electron propagation in the magnetic field, while the IC emission is not affected. To account for the uncertainty of magnetic field measurements, we explored several different magnetic field strengths as illustrated in Fig.\ref{SED_differentB}. The radio emission flux increases by nearly an order of magnitude as the magnetic field strength rises from 0.5 $\rm{\mu G}$ to 2 $\rm{\mu G}$. The influence of the diffusion coefficient is shown in Fig.~\ref{SED_differentD0}. Larger $D_0$ means that electrons are more likely to interact with gas or other materials, resulting in more significant energy losses before escaping the diffusion zone, which in turn leads to a lower flux.

\begin{figure}[htb]
\includegraphics[width=11cm]{SED_differentD0.pdf}
\caption{The SED of synchrotron and IC emission for $m_\chi =10~\rm{GeV}$, $B=1~\rm {\mu G}$, $<\sigma v>=10^{-30} ~\rm{cm^{3}~s^{-1}}$ and various choice of diffusion coefficient.}% The cyan (green) band represents the sensitivity of SKA phase1 (SKA phase2) with 1000 hours of observation.}
\label{SED_differentD0}
\end{figure}

\section{THE constraints on the WIMP case}

By comparing the results of SED with SKA's sensitivity, we establish the upper limits of the DM annihilation cross-section. For each parameter set, we calculate the minimum cross-section required for the photons to be just detectable, i.e. the radio emission equals the SKA's sensitivity. Thus SKA offers a distinctive way of determining whether the gamma-ray emission of Omega Centauri comes from DM annihilation. The sensitivity of SKA we adopted here is $8\times 10^{-8}~\rm{Jy}$ at $\nu =1~\rm{GHz}$~\cite{ska}. The results are presented in Fig.~\ref{sigmav_withmu}. The exclusion limits of DM velocity-averaged annihilation cross-section is about $\rm 10^{-16}~cm^3~s^{-1}$ ($\rm 10^{-18}~cm^3~s^{-1}$) at sub-GeV and $\rm 10^{-32}~cm^3~s^{-1}$ ($\rm 10^{-31}~cm^3~s^{-1}$) at several GeV for electron-positron ($\mu^+\mu^-$) channel. For $b\Bar{b}$ channel, due to the large mass of b quark, we only have the upper limits from GeV to TeV, where radio photons are mainly produced by synchrotron emission, and the limits can reach $\rm 10^{-30}~cm^3~s^{-1}$ at 10-100 GeV range. Here, we compare our limits with other previous results from different observations: The limits from M31 by the Westerbork Synthesis Radio Telescope (WSRT)~\cite{42,43}, Fermi LAT~\cite{44,45}, Ophiuchus by VLA ~\cite{46,47}, from 23 dwarf galaxies (TGSS)~\cite{48}, CMB data~\cite{49,50,51} and Voyager plus AMS02 data~\cite{49,50,51}. It is interesting to note that our constraints are generally stronger in the GeV-TeV range but weaker in the Sub-GeV range compared with other observations. The best-fit parameter corresponding to the gamma-ray emission of Omega Centauri~\cite{Javier_2021} is also present in Fig.~\ref{sigmav_withmu}. Here we have already calibrated the DM cross-section at $J_{\rm{factor}}=10^{22}~\rm{GeV^2~cm^{-5}}$ with the result of $J_{\rm{factor}}<\sigma v>$ in Ref.~\cite{Javier_2021}. It is easy to see that parameters corresponding to the gamma-ray emission can be accurately tested by SKA. 


\begin{figure}[htb]
\subfigure[~$e^+~e^-$~channel]
{\includegraphics[width=11cm]{sigmav_e.pdf}}
\subfigure[~$\mu^+~\mu^-$~channel]{\includegraphics[width=8.15cm]{sigmav_mu.pdf}}
\subfigure[~$b\Bar{b}$~channel]
{\includegraphics[width=8.15cm]{sigmav_bb.pdf}}
\caption{The exclusion limits of DM velocity-averaged annihilation cross-section for $D_0 = 3\times10^{28}~\rm {cm^{2}~s^{-1}}$ and $B=\rm{1~\mu G}$. Comparing with other limits from different sources based on (a) ~$e^+~e^-$~channel, (b)~$\mu^+~\mu^-$~channel and $b\Bar{b}$ channel (c). The blue star representing the case where WIMPs annihilate into $\rm{\mu^+ \mu^-}$ to explain gamma-rays in Omega Centauri~\cite{Javier_2021}.}

\label{sigmav_withmu}

\end{figure}

The constraint of $b\Bar{b}$ channel is presented in  Fig.~\ref{sigmav_withmu}~(c). Unlike the $e^+e^-$ and $\mu^+\mu^-$ channel, the limit of $b\Bar{b}$ final stats are mainly from synchrotron contribution because the decay products of the b quark  have so high energy that the photons generated by the following IC process are not in the radio band. The 68\% and 95\% credible regions (shaded regions) ) by fitting to the antiproton data~\cite{52,53} and the 99.7\%, 68\% and 95\% credible regions (black line) by fitting to the Galactic center GeV excess are also presented in the figure~\cite{54}. The result shows that the SKA has sufficient capability in detecting these favored parameter regions.



Next, we need to discuss the impact of other parameters on the limits, here we take the $e^+ e^-$ annihilation channel as an example. As discussed earlier, the magnetic field solely affects the synchrotron radiation intensity. For high-mass WIMPs, the radio emission is predominantly from the synchrotron process, with the upper limit becoming stronger as the magnetic field increases. However, for low-mass WIMPs, the radio emission is mainly produced by IC radiation, so the upper limit remains unchanged, as depicted in Fig.~\ref{sigmav_differentB}. In contrast, the diffusion coefficient considerably impacts the upper limits of the cross-section in most cases. As previously discussed, lower diffusion coefficients induce tight upper limits of DM cross-section as shown in Fig.~\ref{sigmav_differentD0}. With a high magnetic field or low diffusion coefficient, the upper limit can even reach $10^{-33}~\rm{cm^3~s^{-1}}$ for a WIMP mass of approximately 10 GeV. Furthermore, the diffusion coefficient only affects the constraint strength, while the peak and valley position of the DM cross-section remains the same as shown in Fig.~\ref{sigmav_differentD0}.

\begin{figure}[htb]
\subfigure[$B=B_{0}$]{\includegraphics[width=8.15cm]{sigmav_differentB0_bb1.pdf}}
\subfigure[$B=B_{0}\times e^{-(r/r_{c})}$]{\includegraphics[width=8.15cm]{sigmav_differentB0_bb0.pdf}}
\caption{The exclusion limits of DM velocity-averaged annihilation cross-section on $e^+ e^-$ channel for $D_0 = 3\times10^{28}~\rm {cm^{2}~s^{-1}}$ and different magnetic field model, including constant magnetic field (a) and exponentially decaying magnetic field (b).}
\label{sigmav_differentB}
\end{figure}

\begin{figure}[htb]
\includegraphics[width=11cm]{sigmav_differentD0.pdf}

\caption{The exclusion limits of DM velocity-averaged annihilation cross-section on $e^+ e^-$ channel for $B=\rm{1~\mu G}$ and various choice of diffusion coefficients.}

\label{sigmav_differentD0}

\end{figure}

Generally, even if accounting for uncertainties in the diffusion coefficient and magnetic field, all parameters interpreting the gamma-ray signals are expected to produce detectable radio signals for SKA.

\section{Constraints in the case of stimulated decay of axion}

\begin{figure}[htb]
\includegraphics[width=14cm]{axionskanew.pdf}
\caption{Projected sensitivities of stimulated decay of axion for Omega Centauri with 1000 hours observation in both SKA1 and SKA2 The pink regions are excluded by CAST experiments~\cite{cast}. The gray regions are excluded by experiments \cite{Hagmann_1998,Stephen_2001,Du_2018,Zhong_2018}}
\label{axion}
\end{figure}

Axions, which have a mass with the order of $\mu \rm{eV}$, are also potential DM candidates. Unlike WIMPs, they do not produce continuous synchrotron or IC radio emissions, but line structure emissions at a frequency of $\nu = \frac{m_a}{4\pi}$ through the decay of axions. The lifetime of axions in a vacuum environment is given by:
\begin{equation}
\tau_a=\frac{64\pi}{m_a^3 g_{a\gamma\gamma}^2},
\end{equation}
where $g_{a\gamma\gamma}$ is the axion-to-two-photon coupling parameter.
 
The background radio radiation at frequency $\nu = \frac{m_a}{4\pi}$ enhances the photon production rate due to the indistinguishability of photons and Bose-Einstein statistics \cite{Andrea_2018,Andrea2_2018}. The effective stimulated emission is given by:
\begin{equation}
\Gamma_{s}=\Gamma_{a}(1+2f_{\gamma}),
\end{equation}
where $\Gamma_{a}$ is the reciprocal of $\tau_{a}$, $\Gamma_{s}$ is the stimulated emission rate and $f_{\gamma}$ is the background photon occupation number. In this work, we choose a conservative estimate by considering only the CMB and extragalactic backgrounds and ignore the Milky way background. Additionally, since the Milky way background photons are mainly gathered in the galactic center and the photon energy density decreases rapidly as the radius increases~\cite{Andrea2_2018}, the galactic background is not expected to significantly impact Omega Centauri. The CMB and extragalactic backgrounds can be described by the following equation~\cite{Andrea2_2018},
\begin{eqnarray}
f_{\gamma}=f_{\rm {\gamma ,CMB}}+f_{\rm {\gamma ,exb-bkg}}.
\end{eqnarray}
The photon occupation from a blackbody spectrum is given by:
\begin{eqnarray}
f_{\rm {\gamma , bb}}=\frac{1}{e^{\frac{E_{\gamma}}{k_b T}}-1}, 
\end{eqnarray}
where $k_b$ is Boltzmann constant and T is the blackbody temperature. T=2.725 K for CMB and $T_{\rm {ext-bkg}}(\nu) \sim 1.19~(\rm{GHz}/\nu)^{2.62}~{\rm K}$ for extragalactic background. 

The flux for the line radio signal is~\cite{Andrea2_2018}:
\begin{eqnarray}
S=\frac{\Gamma_a}{4\pi \Delta \nu} \int d \Omega dl \rho_a(l,\Omega) (1+2f_\gamma),
\label{axion_sig}
\end{eqnarray}
where $\Delta \nu = (v_{disp}/c) \nu$ is the width of axion line. In this work, we use $v_{disp}=30~\rm{km/s}$ \cite{Wang_2021}.
With $D_{\rm {factor}}=\int d \Omega dl \rho_a(l,\Omega)$, we rewrite Eq.\ref{axion_sig} as:
\begin{eqnarray}
S=\frac{\Gamma_a}{4\pi \Delta \nu}  D_{\rm {factor}}(1+2f_\gamma).
\end{eqnarray}
For point-like target, $D_{\rm {factor}}$ is approximated as \cite{Lisanti_2018}:
\begin{eqnarray}
D_{\rm {factor}} \approx \frac{1}{d^{2}} \int dV \rho = \frac{M_{\rm {DM}}}{d^2},
\end{eqnarray}
where $d$ is the distance between the target and the Earth.
Thus, we can obtain the limit of the axion-photon coupling parameter with the following equation:
\begin{eqnarray}
g_{a\gamma \gamma}~>~\left[\frac{256\pi ^{2} \Delta \nu d^{2}S_{min}}{m_a^3M_{\rm DM}(1+2f_{\gamma})}\right]^{1/2}.
\end{eqnarray}

The results presented in Fig.~\ref{axion} demonstrate a continuous limit between $10^{-7}~\rm{eV} \sim 10^{-4}~\rm{eV}$, providing a significant improvement over previous studies \cite{Hagmann_1998,Stephen_2001,Du_2018,Zhong_2018}. Unlike the calculation of radio signals from axion-photon resonant conversion in the magnetospheres of magnetic white dwarf stars or neutron stars, which heavily depends on the compact stars model and magnetic field strength, our results are independent of these parameters. Furthermore, using SKA1, we obtained an upper limit of approximately $10^{-10}~\rm{GeV}^{-1}$, and with SKA2, the limit can be improved to $10^{-11}~\rm{GeV}^{-1}$, surpassing the current result from CAST. Therefore, SKA provides an excellent opportunity for the detection of low-mass axions.



\section{Conclusion}
Omega Centauri is proposed as a DM-rich globular cluster near the Milky Way. Here, we calculated the  radio emission from the DM annihilation/decay, including WIMPs and axions, and  discussed the detection capability of SKA.

%sensitivity of SKA to detect radio emission from the dark matter components, including WIMPs and axions.

The synchrotron and IC emissions from cosmic ray $e^+/e^-$ produced by WIMP annihilation provide a complementary method for detecting DM particles. We calculate the synchrotron and IC emission of WIMP annihilation in Omega Centauri
%given the WIMP mass, cross-section, and density profile, 
and evaluate the measurement prospects of SKA. Our calculations show that, for the parameters of DM corresponding to the gamma-ray emission of Omega Centauri, the radio emissions are detectable compared to the sensitivity of the SKA. By setting the radio emission flux from DM annihilation to be less than SKA's sensitivity at $\nu=1~\rm{GHz}$, we obtain the upper limits of DM cross-section for DM masses from MeV to TeV. The upper limits can reach $10^{-33}~\rm{cm^3s^{-1}}$ at 10 GeV, which is much stronger compared with other observations. In general, the measurement of SKA can be used to test the parameter space proposed to explain the tentative gamma-ray radiation of Omega Centauri or that involved in interpreting the Galactic GeV excess, antiproton excess and the W-boson mass anomaly \cite{Fan2022,Zhu2022} in the near future.

Next, we calculate the line emission in the radio band from the stimulated decay of axions. The projected sensitivity of SKA is shown in Fig.~\ref{axion}, and it is effective over a broad mass range of $10^{-7}-10^{-4}~\rm{eV}$. Generally speaking, SKA can probe some of the unexplored axion-photon coupling regions.% in this mass range.

In summary, the SKA's high sensitivity allows us to investigate the untapped parameter space of DM annihilation or decay. 
%We can also test possible photon emissions from DM with observations at different frequencies. 
With multi-frequency measurements, 
%we have a new window of opportunity to 
some DM candidates suggested in the literature can be effectively tested.
%the nature of DM in the near future.


{\bf Acknowledgments} 
This work is supported by the National Key R\&D Program of China (Grants No. 2022YFF0503304), the National Natural Science Foundation of China (12220101003), the National Natural Science Foundation of China (Grants No. 11773075) and the Youth Innovation Promotion Association of Chinese Academy of Sciences (Grant No. 2016288). Hao Gong is supported by the Youth Talent Project of Anhui Education Department (Grant No. gxyq2019105).


\begin{thebibliography}{}

\bibitem{q1}
M.~Cirelli,
%``Indirect Searches for Dark Matter: a status review,''
Pramana \textbf{79} (2012), 1021-1043
doi:10.1007/s12043-012-0419-x
[arXiv:1202.1454 [hep-ph]].

\bibitem{q2}
R.~K.~Leane,
%``Indirect Detection of Dark Matter in the Galaxy,''
[arXiv:2006.00513 [hep-ph]].

\bibitem{q3}
L.~Bergstrom,
%``Dark Matter Evidence, Particle Physics Candidates and Detection Methods,''
Annalen Phys. \textbf{524} (2012), 479-496
doi:10.1002/andp.201200116
[arXiv:1205.4882 [astro-ph.HE]].

\bibitem{q4}
X.~J.~Bi, P.~F.~Yin and Q.~Yuan,
%``Status of Dark Matter Detection,''
Front. Phys. (Beijing) \textbf{8} (2013), 794-827
doi:10.1007/s11467-013-0330-z
[arXiv:1409.4590 [hep-ph]].

\bibitem{Arpan_2020} 
%\cite{Kar:2020coz}
%\bibitem{Kar:2020coz}
A.~Kar, B.~Mukhopadhyaya, S.~Tingay, B.~McKinley, M.~Haverkorn, S.~McSweeney, N.~Hurley-Walker, S.~Mitra and T.~R.~Choudhury,
%``Dark matter annihilation in $\omega$ Centauri: Astrophysical implications derived from the MWA radio data,''
Phys. Dark Univ. \textbf{30}, 100689 (2020)
doi:10.1016/j.dark.2020.100689
[arXiv:2005.11962 [astro-ph.HE]].

%Arpan Kar, Biswarup Mukhopadhyaya, Steven Tingay, Ben McKinley, Marijke Haverkorn, Sam McSweeney, Natasha Hurley-Walker, Sourav Mitra and Tirthankar Roy Choudhury. Phys.Dark Univ. {\bf 30}, 100689 (2020)

\bibitem{Chen_2021}
Z.~Chen, Y.~L.~S.~Tsai and Q.~Yuan,
%``Sensitivity of SKA to dark matter induced radio emission,''
JCAP \textbf{09} (2021), 025
doi:10.1088/1475-7516/2021/09/025
[arXiv:2105.00776 [astro-ph.HE]].

\bibitem{Cembranos_2020} J.~A.~R.~Cembranos, \'A.~De La Cruz-Dombriz, V.~Gammaldi and M.~M\'endez-Isla,
%``SKA-Phase 1 sensitivity to synchrotron radio emission from multi-TeV Dark Matter candidates,''
Phys. Dark Univ. \textbf{27} (2020), 100448
doi:10.1016/j.dark.2019.100448
[arXiv:1905.11154 [hep-ph]].

\bibitem{huang}
W.~Q.~Guo, Y.~Li, X.~Huang, Y.~Z.~Ma, G.~Beck, Y.~Chandola and F.~Huang,
%``Constraints on dark matter annihilation from the FAST observation of the Coma Berenices dwarf galaxy,''
[arXiv:2209.15590 [astro-ph.HE]].

\bibitem{Peccei_1977} R.~D.~Peccei and H.~R.~Quinn,
%``CP Conservation in the Presence of Instantons,''
Phys. Rev. Lett. \textbf{38} (1977), 1440-1443
doi:10.1103/PhysRevLett.38.1440

\bibitem{Weinberg_1978} S.~Weinberg,
%``A New Light Boson?,''
Phys. Rev. Lett. \textbf{40} (1978), 223-226
doi:10.1103/PhysRevLett.40.223

\bibitem{new1}
J.~Preskill, M.~B.~Wise and F.~Wilczek,
%``Cosmology of the Invisible Axion,''
Phys. Lett. B \textbf{120} (1983), 127-132
doi:10.1016/0370-2693(83)90637-8

\bibitem{new2}
L.~F.~Abbott and P.~Sikivie,
%``A Cosmological Bound on the Invisible Axion,''
Phys. Lett. B \textbf{120} (1983), 133-136
doi:10.1016/0370-2693(83)90638-X

\bibitem{new3}
M.~Dine and W.~Fischler,
%``The Not So Harmless Axion,''
Phys. Lett. B \textbf{120} (1983), 137-141
doi:10.1016/0370-2693(83)90639-1


\bibitem{Andrea_2018} A.~Caputo, C.~P.~Garay and S.~J.~Witte,
%``Looking for Axion Dark Matter in Dwarf Spheroidals,''
Phys. Rev. D \textbf{98} (2018) no.8, 083024
[erratum: Phys. Rev. D \textbf{99} (2019) no.8, 089901]
doi:10.1103/PhysRevD.98.083024
[arXiv:1805.08780 [astro-ph.CO]].

\bibitem{Andrea2_2018}
A.~Caputo, M.~Regis, M.~Taoso and S.~J.~Witte,
%``Detecting the Stimulated Decay of Axions at RadioFrequencies,''
JCAP \textbf{03} (2019), 027
doi:10.1088/1475-7516/2019/03/027
[arXiv:1811.08436 [hep-ph]].

\bibitem{ska} R.~Braun, A.~Bonaldi, T.~Bourke, E.~Keane and J.~Wagg,
%``Anticipated Performance of the Square Kilometre Array -- Phase 1 (SKA1),''
SKA-TEL-SKO-0000818, (2017)
[arXiv:1912.12699 [astro-ph.IM]].

\bibitem{Wang_2021} 
J.~W.~Wang, X.~J.~Bi and P.~F.~Yin,
%``Detecting axion dark matter through the radio signal from Omega Centauri,''
Phys. Rev. D \textbf{104} (2021) no.10, 103015
doi:10.1103/PhysRevD.104.103015
[arXiv:2109.00877 [astro-ph.HE]].

\bibitem{1}
L.~Feng, Q.~Yuan, P.~F.~Yin, X.~J.~Bi and M.~Li,
%``Search for dark matter signals with Fermi-LAT observation of globular clusters NGC 6388 and M 15,''
JCAP \textbf{04} (2012), 030
doi:10.1088/1475-7516/2012/04/030
[arXiv:1112.2438 [astro-ph.HE]].

\bibitem{2}
J.~M.~Diego, M.~Pascale, B.~Frye, A.~Zitrin, T.~Broadhurst, G.~Mahler, G.~B.~Caminha, M.~Jauzac, M.~G.~Lee and J.~H.~Bae, \textit{et al.}
%``On the correlation between dark matter, intracluster light and globular cluster distribution in SMACS0723,''
[arXiv:2301.03629 [astro-ph.CO]].

\bibitem{3}
G.~Ogiya, F.~C.~van den Bosch, A.~Burkert and X.~Kang,
%``Testing the Galaxy-collision-induced Formation Scenario for the Trail of Dark-matter-deficient Galaxies with the Susceptibility of Globular Clusters to the Tidal Force,''
Astrophys. J. Lett. \textbf{940} (2022) no.2, L46
doi:10.3847/2041-8213/aca2a7
[arXiv:2211.05993 [astro-ph.GA]].

\bibitem{4}
M.~J.~Dolan, F.~J.~Hiskens and R.~R.~Volkas,
%``Advancing globular cluster constraints on the axion-photon coupling,''
JCAP \textbf{10} (2022), 096
doi:10.1088/1475-7516/2022/10/096
[arXiv:2207.03102 [hep-ph]].

\bibitem{5}
L.~Staveley-Smith, E.~Bond, K.~Bekki and T.~Westmeier,
%``A search for annihilating dark matter in 47 Tucanae and Omega Centauri,''
Publ. Astron. Soc. Austral. \textbf{39} (2022), e026
doi:10.1017/pasa.2022.23
[arXiv:2205.01270 [astro-ph.HE]].

\bibitem{6}
R.~G.~Carlberg and C.~J.~Grillmair,
%``Testing for Dark Matter in the Outskirts of Globular Clusters,''
Astrophys. J. \textbf{922} (2021) no.2, 104
doi:10.3847/1538-4357/ac289f
[arXiv:2106.00751 [astro-ph.GA]].

\bibitem{7}
J.~E.~Doppel, L.~V.~Sales, J.~F.~Navarro, M.~G.~Abadi, E.~W.~Peng, E.~Toloba and F.~Ramos-Almendares,
%``Globular clusters as tracers of the dark matter content of dwarfs in galaxy clusters,''
Mon. Not. Roy. Astron. Soc. \textbf{502} (2021) no.2, 1661-1677
doi:10.1093/mnras/staa3915
[arXiv:2010.06590 [astro-ph.GA]].

\bibitem{8}
E.~Ardi and H.~Baumgardt,
%``Depletion of dark matter within globular clusters,''
J. Phys. Conf. Ser. \textbf{1503} (2020) no.1, 012023
doi:10.1088/1742-6596/1503/1/012023

\bibitem{9}
D.~E.~McLaughlin and R.~P.~van der Marel,
%``Resolved Massive Star Clusters in the Milky Way and its Satellites: Brightness Profiles and a Catalogue of Fundamental Parameters,''
Astrophys. J. Suppl. \textbf{161} (2005), 304
doi:10.1086/497429
[arXiv:astro-ph/0605132 [astro-ph]].

\bibitem{Addy_2021} A.~J.~Evans, L.~E.~Strigari and P.~Zivick,
%``Dark and luminous mass components of Omega Centauri from stellar kinematics,''
Mon. Not. Roy. Astron. Soc. \textbf{511} (2022) no.3, 4251-4264
doi:10.1093/mnras/stac261
[arXiv:2109.10998 [astro-ph.GA]].

\bibitem{Javier_2021} J.~Reynoso-Cordova, O.~Burgue\~no, A.~Geringer-Sameth, A.~X.~Gonzalez-Morales, S.~Profumo and O.~Valenzuela,
%``On the origin of the gamma-ray emission from Omega Centauri: millisecond pulsars and dark matter annihilation,''
JCAP \textbf{02} (2021), 010
doi:10.1088/1475-7516/2021/02/010
[arXiv:1907.06682 [astro-ph.HE]].

\bibitem{Profumo2017}
S.~Profumo,
%``An Introduction to Particle Dark Matter,''
World Scientific, 2017,
ISBN 978-1-78634-000-9, 978-1-78634-001-6, 978-1-78634-001-6
doi:10.1142/q0001

\bibitem{Conrad2017}
J.~Conrad and O.~Reimer,
%``Indirect dark matter searches in gamma and cosmic rays,''
Nature Phys. \textbf{13} (2017) no.3, 224-231
doi:10.1038/nphys4049
[arXiv:1705.11165 [astro-ph.HE]].

\bibitem{Baumgardt_2017} Baumgardt, H. MNRAS, {\bf 464}, 2174 2017

\bibitem{Harris_1996} W.~E.~Harris,
%``A Catalog of Parameters for Globular Clusters in the Milky Way,''
Astron. J. \textbf{112} (1996), 1487
doi:10.1086/118116

\bibitem{Bekki_2003} K.~Bekki and K.~C.~Freeman,
%``Formation of omega Centauri from an ancient nucleated dwarf galaxy in the young Galactic disc,''
Mon. Not. Roy. Astron. Soc. \textbf{346} (2003), L11
doi:10.1046/j.1365-2966.2003.07275.x
[arXiv:astro-ph/0310348 [astro-ph]].

\bibitem{Brown_2019} A.~M.~Brown, R.~Massey, T.~Lacroix, L.~E.~Strigari, A.~Fattahi and C.~B\oe{}hm,
%``The glow of annihilating dark matter in Omega Centauri,''
[arXiv:1907.08564 [astro-ph.HE]].

\bibitem{Watkins2015}
Laura~L. Watkins, Roeland~P. van~der Marel, Andrea Bellini, and Jay Anderson. Hubble space telescope proper motion (hstpromo) catalogs of galactic globular clusters. iii. dynamical distances and mass-to-light ratios. {\em The Astrophysical Journal}, \textbf{812} (2015)

\bibitem{rx-dmfit}A.~McDaniel, T.~Jeltema, S.~Profumo and E.~Storm,
%``Multiwavelength Analysis of Dark Matter Annihilation and RX-DMFIT,''
JCAP \textbf{09} (2017), 027
doi:10.1088/1475-7516/2017/09/027
[arXiv:1705.09384 [astro-ph.HE]].

\bibitem{green1} S.~Colafrancesco, S.~Profumo and P.~Ullio,
%``Multi-frequency analysis of neutralino dark matter annihilations in the Coma cluster,''
Astron. Astrophys. \textbf{455} (2006), 21
doi:10.1051/0004-6361:20053887
[arXiv:astro-ph/0507575 [astro-ph]].

\bibitem{green2} V. L. Ginzburg and S. I. Syrovatskii, The Origin of Cosmic Rays. (1964)

\bibitem{d01} E.~A.~Baltz and J.~Edsjo,
%``Positron propagation and fluxes from neutralino annihilation in the halo,''
Phys. Rev. D \textbf{59} (1998), 023511
doi:10.1103/PhysRevD.59.023511
[arXiv:astro-ph/9808243 [astro-ph]].

\bibitem{d02} J.~Buch, M.~Cirelli, G.~Giesen and M.~Taoso,
%``PPPC 4 DM secondary: A Poor Particle Physicist Cookbook for secondary radiation from Dark Matter,''
JCAP \textbf{09} (2015), 037
doi:10.1088/1475-7516/2015/9/037
[arXiv:1505.01049 [hep-ph]].

\bibitem{DAMPE2022BC} 
%\cite{Collaboration:2022vwu}
%\bibitem{Collaboration:2022vwu}
Alemanno, F., et al. (DAMPE~Collaboration),
%``Detection of spectral hardenings in cosmic-ray boron-to-carbon and boron-to-oxygen flux ratios with DAMPE,''
Sci. Bull. \textbf{67}, 2162-2166 (2022)
doi:10.1016/j.scib.2022.10.002
[arXiv:2210.08833 [astro-ph.HE]].
%8 citations counted in INSPIRE as of 19 Mar 2023

\bibitem{47ne} P.~C.~C.~Freire, M.~Kramer, A.~G.~Lyne, F.~Camilo, R.~N.~Manchester and N.~D'Amico,
%``Detection of ionized gas in the globular cluster 47 tucanae,''
Astrophys. J. Lett. \textbf{557} (2001), L105-L108
doi:10.1086/323248
[arXiv:astro-ph/0107206 [astro-ph]].

\bibitem{bloss} S.~Colafrancesco, S.~Profumo and P.~Ullio,
%``Multi-frequency analysis of neutralino dark matter annihilations in the Coma cluster,''
Astron. Astrophys. \textbf{455} (2006), 21
doi:10.1051/0004-6361:20053887
[arXiv:astro-ph/0507575 [astro-ph]].

\bibitem{syn} E.~Storm, T.~E.~Jeltema, M.~Splettstoesser and S.~Profumo,
%``Synchrotron Emission from Dark Matter Annihilation: Predictions for Constraints from Non-detections of Galaxy Clusters with New Radio Surveys,''
Astrophys. J. \textbf{839} (2017) no.1, 33
doi:10.3847/1538-4357/aa6748
[arXiv:1607.01049 [astro-ph.CO]].

\bibitem{42}
R.~Gie\ss{}\"ubel, G.~Heald and R.~Beck,
%``Polarized synchrotron radiation from the Andromeda Galaxy M31 and background sources at 350 MHz,''
Astron. Astrophys. \textbf{559} (2013), A27
doi:10.1051/0004-6361/201321765
[arXiv:1309.2539 [astro-ph.CO]].

\bibitem{43}
M.~H.~Chan,
%``Revisiting the constraints on annihilating dark matter by the radio observational data of M31,''
Phys. Rev. D \textbf{94} (2016) no.2, 023507
doi:10.1103/PhysRevD.94.023507
[arXiv:1606.08537 [astro-ph.HE]].

\bibitem{44}
M.~G.~Baring, T.~Ghosh, F.~S.~Queiroz and K.~Sinha,
%``New Limits on the Dark Matter Lifetime from Dwarf Spheroidal Galaxies using Fermi-LAT,''
Phys. Rev. D \textbf{93} (2016) no.10, 103009
doi:10.1103/PhysRevD.93.103009
[arXiv:1510.00389 [hep-ph]].

\bibitem{45}
A.~Albert \textit{et al.} [Fermi-LAT and DES],
%``Searching for Dark Matter Annihilation in Recently Discovered Milky Way Satellites with Fermi-LAT,''
Astrophys. J. \textbf{834} (2017) no.2, 110
doi:10.3847/1538-4357/834/2/110
[arXiv:1611.03184 [astro-ph.HE]].

\bibitem{46}
M.~Murgia, D.~Eckert, F.~Govoni, C.~Ferrari, M.~Pandey-Pommier, J.~Nevalainen and S.~Paltani,
%``GMRT observations of the Ophiuchus galaxy cluster,''
Astron. Astrophys. \textbf{514} (2010), A76
doi:10.1051/0004-6361/201014126
[arXiv:1002.2332 [astro-ph.CO]].

\bibitem{47}
Z.~S.~Yuan and J.~L.~Han,
%``Dynamical state for 964 galaxy clusters from Chandra X-ray images,''
Mon. Not. Roy. Astron. Soc. \textbf{497} (2020) no.4, 5485-5497
doi:10.1093/mnras/staa2363
[arXiv:2008.01299 [astro-ph.GA]].

\bibitem{48}
A.~Basu, N.~Roy, S.~Choudhuri, K.~K.~Datta and D.~Sarkar,
%``Stringent constraint on the radio signal from dark matter annihilation in dwarf spheroidal galaxies using the TGSS,''
Mon. Not. Roy. Astron. Soc. \textbf{502} (2021) no.2, 1605-1611
doi:10.1093/mnras/stab120
[arXiv:2101.04925 [astro-ph.HE]].

\bibitem{49}
J.~Cang, Y.~Gao and Y.~Z.~Ma,
%``Probing dark matter with future CMB measurements,''
Phys. Rev. D \textbf{102} (2020) no.10, 103005
doi:10.1103/PhysRevD.102.103005
[arXiv:2002.03380 [astro-ph.CO]].

\bibitem{50}
T.~R.~Slatyer and C.~L.~Wu,
%``General Constraints on Dark Matter Decay from the Cosmic Microwave Background,''
Phys. Rev. D \textbf{95} (2017) no.2, 023010
doi:10.1103/PhysRevD.95.023010
[arXiv:1610.06933 [astro-ph.CO]].

\bibitem{51}
M.~Boudaud, J.~Lavalle and P.~Salati,
%``Novel cosmic-ray electron and positron constraints on MeV dark matter particles,''
Phys. Rev. Lett. \textbf{119} (2017) no.2, 021103
doi:10.1103/PhysRevLett.119.021103
[arXiv:1612.07698 [astro-ph.HE]].

\bibitem{52}
M.~Y.~Cui, Q.~Yuan, Y.~L.~S.~Tsai and Y.~Z.~Fan,
%``Possible dark matter annihilation signal in the AMS-02 antiproton data,''
Phys. Rev. Lett. \textbf{118} (2017) no.19, 191101
doi:10.1103/PhysRevLett.118.191101
[arXiv:1610.03840 [astro-ph.HE]].

\bibitem{53}
M.~di Mauro, F.~Donato, A.~Goudelis and P.~D.~Serpico,
%``New evaluation of the antiproton production cross section for cosmic ray studies,''
Phys. Rev. D \textbf{90} (2014) no.8, 085017
[erratum: Phys. Rev. D \textbf{98} (2018) no.4, 049901]
doi:10.1103/PhysRevD.90.085017
[arXiv:1408.0288 [hep-ph]].

\bibitem{54}
F.~Calore, I.~Cholis and C.~Weniger,
%``Background Model Systematics for the Fermi GeV Excess,''
JCAP \textbf{03} (2015), 038
doi:10.1088/1475-7516/2015/03/038
[arXiv:1409.0042 [astro-ph.CO]].

\bibitem{cast} V.~Anastassopoulos \textit{et al.} [CAST],
%``New CAST Limit on the Axion-Photon Interaction,''
Nature Phys. \textbf{13} (2017), 584-590
doi:10.1038/nphys4109
[arXiv:1705.02290 [hep-ex]].

\bibitem{Hagmann_1998} C.~Hagmann \textit{et al.} [ADMX],
%``Results from a high sensitivity search for cosmic axions,''
Phys. Rev. Lett. \textbf{80} (1998), 2043-2046
doi:10.1103/PhysRevLett.80.2043
[arXiv:astro-ph/9801286 [astro-ph]].

\bibitem{Stephen_2001} S.~J.~Asztalos \textit{et al.} [ADMX],
%``Large scale microwave cavity search for dark matter axions,''
Phys. Rev. D \textbf{64} (2001), 092003
doi:10.1103/PhysRevD.64.092003

\bibitem{Du_2018} N.~Du \textit{et al.} [ADMX],
%``A Search for Invisible Axion Dark Matter with the Axion Dark Matter Experiment,''
Phys. Rev. Lett. \textbf{120} (2018) no.15, 151301
doi:10.1103/PhysRevLett.120.151301
[arXiv:1804.05750 [hep-ex]].

\bibitem{Zhong_2018} L.~Zhong \textit{et al.} [HAYSTAC],
%``Results from phase 1 of the HAYSTAC microwave cavity axion experiment,''
Phys. Rev. D \textbf{97} (2018) no.9, 092001
doi:10.1103/PhysRevD.97.092001
[arXiv:1803.03690 [hep-ex]].

\bibitem{Lisanti_2018} 
M.~Lisanti, S.~Mishra-Sharma, N.~L.~Rodd, B.~R.~Safdi and R.~H.~Wechsler,
%``Mapping Extragalactic Dark Matter Annihilation with Galaxy Surveys: A Systematic Study of Stacked Group Searches,''
Phys. Rev. D \textbf{97} (2018) no.6, 063005
doi:10.1103/PhysRevD.97.063005
[arXiv:1709.00416 [astro-ph.CO]].

%\cite{Fan:2022dck}
\bibitem{Fan2022}
Y.~Z.~Fan, T.~P.~Tang, Y.~L.~S.~Tsai and L.~Wu,
%``Inert Higgs Dark Matter for CDF II W-Boson Mass and Detection Prospects,''
Phys. Rev. Lett. \textbf{129}, no.9, 091802 (2022)
doi:10.1103/PhysRevLett.129.091802
[arXiv:2204.03693 [hep-ph]].
%95 citations counted in INSPIRE as of 19 Mar 2023

%\cite{Zhu:2022tpr}
\bibitem{Zhu2022}
C.~R.~Zhu, M.~Y.~Cui, Z.~Q.~Xia, Z.~H.~Yu, X.~Huang, Q.~Yuan and Y.~Z.~Fan,
%``Explaining the GeV Antiproton Excess, GeV \ensuremath{\gamma}-Ray Excess, and W-Boson Mass Anomaly in an Inert Two Higgs Doublet Model,''
Phys. Rev. Lett. \textbf{129}, no.23, 23 (2022)
doi:10.1103/PhysRevLett.129.231101
[arXiv:2204.03767 [astro-ph.HE]].
%56 citations counted in INSPIRE as of 19 Mar 2023




\end{thebibliography}
\end{document}
