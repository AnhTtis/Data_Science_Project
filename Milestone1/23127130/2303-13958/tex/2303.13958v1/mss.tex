
\documentclass[sn-mathphys]{sn-jnl}
\newcommand{\MR}{\texttt{Measure and Resend}}
\newcommand{\M}{\texttt{Measure}}
\newcommand{\Resend}{\texttt{Resend}}
\newcommand{\Perm}{\texttt{Permute}}
\newcommand{\Refresh}{\texttt{Refresh}}
\usepackage{physics}
\usepackage{caption}
\usepackage{tablefootnote}

\usepackage{multicol}
\usepackage{amsmath}
\usepackage{multirow}
\usepackage{amsfonts,amssymb}
\usepackage{tabularx}
\usepackage{multirow}
\usepackage{bbold}
\jyear{2021}%
\theoremstyle{thmstyleone}%
\newtheorem{theorem}{Theorem}
\newtheorem{proposition}[theorem]{Proposition}% 
\theoremstyle{thmstyletwo}%
\newtheorem{example}{Example}%
\newtheorem{remark}{Remark}%
\usepackage{tabularx}
\usepackage{multirow}
\usepackage{tabularx,multirow,array}
\usepackage{float}
\theoremstyle{thmstylethree}%
\newtheorem{definition}{Definition}%

\raggedbottom

\begin{document}
\title[Boosted quantum and semi-quantum communication protocols]{Boosted quantum and semi-quantum communication protocols}
\author*[]{\fnm{Rajni} \sur{Bala}}\email{Rajni.Bala@physics.iitd.ac.in}
\author[]{\fnm{Sooryansh} \sur{Asthana}}\email{sooryansh.asthana@physics.iitd.ac.in}
\author[]{\fnm{ V.} \sur{Ravishankar}}\email{vravi@physics.iitd.ac.in}

\affil*[]{\orgdiv{Department of Physics}, \orgname{IIT Delhi}, \orgaddress{\street{Hauz Khas}, \city{New Delhi}, \postcode{110016}, \state{Delhi}, \country{India}}}

%-------------------------------------------------------------------------------
\abstract{ Secure quantum communication protocols based on prepare-and-measure scheme employ mutually unbiased bases. In these protocols, a large number of runs, in which different participants measure in different bases, simply go wasted.  In this paper, we show that it is possible to reduce the number of such runs by a suitable design of the key generation rule. This results in a  significant increase in the key generation rate (KGR). We illustrate this advantage by proposing   quantum key distribution protocols and semi-quantum key distribution protocols by employing effective qubits encoded in higher dimensional quantum systems. Since none of them demands  the preparation of entangled states as resources and a relatively large amount of information can be transferred, we believe that our proposals are worth pursuing experimentally. 
}
\keywords{quantum network, quantum cryptography, quantum secure direct communication}
%-----------------------------------------------------------------------------------------------

\maketitle

%----------------------------------------------------









\abstract{

}
\keywords{quantum network, semi--quantum cryptography, layered quantum cryptography}

\maketitle
\section{Introduction}
\label{introduction}
The proposal of the seminal BB84 quantum key distribution protocol (QKD) has resulted in the advent of the field of secure quantum communication \cite{Bennett84}. Since then,  the field has witnessed unprecedented growth \cite{gisin2002quantum,pirandola2020advances,chau2015quantum}. QKD protocols employing quantum resources such as contextuality, entanglement, and non-orthogonality of states have been proposed \cite{Ekert91, Bennett92, Bennett93,bala2021contextuality}.


Any secure quantum communication protocol is expected to have a high communication rate.  
To meet this requirement, several efforts have been made, resulting in proposals for many novel protocols such as,  
(i) QKD with high-dimensional states 
\cite{bradler2016finite},
(ii)  differential phase shift (DPS) QKD protocol,  which provides a higher KGR than BB$84$ protocol \cite{Inoue02}, (iii)  the protocol employing biased probabilities for the two bases used in BB$84$ protocol \cite{EfficientQKD_qubit}, and (iv) the protocol obtained by clubbing of two protocols by employing different degrees of freedom of single photons 
\cite{wang2010efficient}.

 However, all these protocols have one ubiquitous feature-- keys are generated only if measurements are performed in the same basis.  Consequently, the data of the rest of the rounds -- amounting to half of the total rounds or more -- is simply discarded,  thereby limiting the KGR. 


 \begin{table}[h!]
 \begin{center}
 \resizebox{10cm}{3.5cm} {
\begin{tabular}{ | c| c| c |c|}
\hline
\multirow{3}{*}{}&\multirow{3}{3cm}{ \textbf{ QKD with mutually unbiased bases \cite{cerf2002security}}}&\multirow{3}{4cm}{\bf QKD based on orthogonal state encoding \cite{shu2022quantum}}&\multirow{3}{2cm}{\bf bQKD} \\
 && &\\
&&&\\\hline
\multirow{3}{*}{ Quantum channel} &\multirow{3}{*}{ideal}&\multirow{3}{*}{ ideal}&\multirow{3}{*}{ ideal}\\
&&&\\
&&&\\\hline
\multirow{3}{2cm}{ Resource states} &\multirow{3}{2.5cm}{ qudits}&\multirow{3}{4cm}{ Two entangled and two separable states}&\multirow{3}{2cm}{pseudo-qubits}\\
&&&\\
&&&\\\hline
\multirow{3}{3cm}{ Yield of states }&\multirow{3}{3cm}{ $\sim kHz$}&\multirow{3}{*}{{ $\sim mHz$}}&\multirow{3}{3cm}{ $\sim kHz$}\\
 %&&&\\
 &&&\\
&&&\\ \hline
\multirow{3}{3cm}{ security against eavesdropping}& \multirow{3}{*}{ non-orthogonality of states}&\multirow{3}{3cm}{ \textit{additional} decoy states are required}& \multirow{3}{*}{ Non-orthogonality of states }\\
 &&&\\
 &&&\\
 \hline
 \multirow{3}{3cm}{ Quantum memory} &\multirow{3}{3cm}{not required} & \multirow{3}{3cm}{ Required} &\multirow{3}{3cm}{ not required} \\
&&&\\
&&&\\\hline
\multirow{3}{3cm}{ $\#$ minimum bases used}&\multirow{3}{*}{$2$}&\multirow{3}{*}{ $1$}&\multirow{3}{*}{ $3$}\\
 %&&&\\
&&&\\ %\hline
%\multirow{3}{3cm}{\centering $\#$ rounds discarded}&\multirow{3}{*}{\centering $50\%$}&\multirow{3}{*}{\centering $-$}&\multirow{3}{*}{\centering $\frac{2}{9}$ (for $d=4$)}\\
 %&&&\\
&&&\\ \hline
\multirow{3}{3cm}{{ Scope for generalisation with current technology }}&\multirow{3}{*}{ moderate*}&\multirow{3}{*}{{difficult}}&\multirow{3}{*}{{ easy}}\\
&&&\\
&&&\\\hline
 \end{tabular}}
\vspace{0.06cm}
       \end{center}
    \caption{Comparison of the bQKD protocol proposed in this paper with existing QKD protocols. Please note that if the one employs effective qubits encoded in ququart systems, $11.11\%$ of the total rounds are yet to be discarded. The symbol $*$ represents the decrement in fidelity with increasing dimensions.}
     \label{tab:Comparison_QKD}
\end{table}
%================================================================================================
\begin{table}[h!]
\begin{center}
 \resizebox{10 cm}{3.5cm}{
\begin{tabular}{ | c| c| c |c|}
\hline
\multirow{3}{*}{}&\multirow{3}{3cm}{ \textbf{Efficient mediated SQKD\cite{chen2021efficient}}}&\multirow{3}{2.5cm}{\bf Efficient SQKD \cite{pan2022semi} }&\multirow{3}{2.5cm}{\bf bSQKD}\\
 && &\\
&&&\\\hline
\multirow{2}{*}{ Quantum channel} &\multirow{2}{*}{ ideal}&\multirow{2}{*}{ ideal}&\multirow{2}{*}{ ideal}\\
%&&&\\
&&&\\\hline
\multirow{2}{3cm}{ Third party} &\multirow{2}{2.5cm}{required}&\multirow{2}{2cm}{ not required}&\multirow{2}{2cm}{ not required}\\
%&&&\\
&&&\\\hline
 \multirow{3}{3cm}{ \# Classical participants}& \multirow{3}{*}{ $2$}& \multirow{3}{*}{ $1$}&\multirow{3}{*}{ $1$}\\
 &&&\\
 &&&\\
 \hline
\multirow{3}{3cm}{ Resource states} &\multirow{3}{2.5cm}{ Separable states}&\multirow{3}{2cm}{ bipartite entanglement}&\multirow{3}{2cm}{ effective qubits}\\
&&&\\
&&&\\\hline
\multirow{3}{3cm}{Yield of states }&\multirow{3}{*}{ $\sim kHz$}&\multirow{3}{*}{{ $\sim mHz$}}&\multirow{3}{*}{ $\sim kHz$ }\\
 &&&\\
&&&\\ \hline
%--------------------------------------------------------------------------------------
\multirow{3}{3cm}{ Fidelity of states }&\multirow{3}{*}{$\sim 90\%$ \cite{wang2019demand}}&\multirow{3}{*}{$\sim 90\%$ \cite{wang2019demand}}&\multirow{3}{*}{ $\sim 99\%$}\\
  &&&\\
&&&\\ 
&&&\\ \hline
\multirow{3}{3cm}{ Scope for generalisation to higher-dimensions}&\multirow{3}{*}{ difficult}&\multirow{3}{*}{{ difficult}}&\multirow{3}{*}{easy}\\
 &&&\\
&&&\\ \hline 
\end{tabular}
}\end{center}
    \caption{Comparison of the bSQKD protocol with existing protocols.}
    \label{tab:comparison_SQKD}
\end{table}



On the other hand, the idea of semi-QKD (SQKD) -- that allows sharing a key with only one quantum participant\footnote{ A quantum participant (QP) can prepare a state and measure it in any basis whereas a {\it classical} participant (CP) can prepare a state and perform a measurement only in the computational basis.} -- has attracted a lot of  research attention \cite{iqbal2020semi,boyer2009semiquantum,xie2018semi,ye2018semi,pan2020semi,bala2022quantum,bala2022semi}. Since SQKD allows for sharing of secure keys with only one quantum participant, it utilizes the existing classical resources. This, in turn, reduces the burden of availability of quantum resources,  making them appropriate in the current noisy-intermediate scale quantum (NISQ) regime \cite{yan2019semi,li2020new}. However, analogous to QKD protocols, these protocols also discard data from many rounds. In fact, the number of rounds that do not contribute to key generation is even larger (growing up to $75\%$), putting a stringent bar on the key generation rate. 
 
 
   In this work, we show that the problem associated with KGR can be overcome, and that too in an experiment-friendly manner. For this, we put forth the idea of employing the bases in such a manner that measurements in different bases also contribute to the generation of key symbols. For experimental feasibility, we identify the states which are not only prepared easily but also with high fidelity. Though there is an advancement in the generation of high-dimensional states, their fidelities decrease with an increase in dimensions \cite{goel2022inverse}. However, states involving superposition of only two levels, i.e., effective qubits  have been generated with fidelities of $\sim 99\%$ \cite{ding2017high}. This experimental study suggests that QKD protocols involving qubits are experimentally more feasible.

Taking this into consideration, we propose a new protocol, which we shall designate as boosted QKD. We make use of the feature that the  effective qubits encoded in a high-dimensional space  may carry more than $1$ bit per transmission. This is in contrast to physical qubit systems belonging to a two-dimensional Hilbert space \cite{PhysRevA.92.062324}. This feature, together with ease in preparation of effective qubit states and appropriate choice of bases, allows us to use a large fraction of  data. This naturally results in an increase in KGR.   As illustrations, we present boosted QKD (bQKD) protocols,  employing effective qubits encoded in ququart systems (bQKD$_4$) and quhex systems (bQKD$_6$) (section (\ref{QKD})). We show 
the robustness of the protocol by analyzing it against various eavesdropping strategies (section (\ref{security_QKD})). We note that in bQKD$_4$ protocol, data generated in $2/9^{\rm th}$ rounds is discarded (discussed in detail in section (\ref{QKD})).  Thereafter, we generalize the protocol to qudit systems (section (\ref{generalisation})).

As the next step, noting the advantages offered by SQKD,
 we  study how the same idea (of employing effective qubits in high-dimensional systems) can be coupled with SQKD for increasing KGR.  For this, we present a boosted SQKD protocol, which for brevity, we designate as bSQKD. Along the lines of bQKD, we also study its robustness against various eavesdropping strategies (sections (\ref{boosted SQKD}) and (\ref{Robustness_SQKD})).  As a quick comparison in tables (\ref{tab:Comparison_QKD}) and (\ref{tab:comparison_SQKD}), we compare various features of our protocols with those of already existing ones. Section (\ref{conclusion}) concludes the paper.






%===========================================================================================================
\section{Boosted QKD (bQKD)}
\label{QKD}

In this section, we present a protocol that allows sharing of keys securely between two participants. We start with two examples- (i) effective qubits encoded in ququart systems and, (ii) effective qubits encoded in quhex systems. The two examples reflect how the key generation rate increases considerably with an increase in the dimensionality of the space.  After that, we generalize these protocols to qudit systems (with $d$ being even) and deduce the corresponding KGR.
\subsection{bQKD with ququarts (bQKD$_4$)}
\label{QKD_four}
Let Alice and Bob be the two participants who want to share a key by employing ququarts.\\


\noindent{\textit{\textbf{Aim:}}} distribution of a key with minimum discarding of data.\\
\noindent{\textit{\textbf{Resources:}}} Three bases sets $B_0, B_1, B_2$ as given below are employed,
\begin{align}
    B_0 &= \{\ket{0}, \ket{1}, \ket{2}, \ket{3}\};\nonumber\\
    B_1 &= \Big\{\frac{1}{\sqrt{2}}\Big(\ket{0}\pm\ket{1}\Big), \frac{1}{\sqrt{2}}\Big(\ket{2}\pm \ket{3}\Big)\Big\};\nonumber\\
B_2 &= \Big\{\frac{1}{\sqrt{2}}\Big(\ket{0}\pm \ket{3}\Big), \frac{1}{\sqrt{2}}\Big(\ket{1}\pm \ket{2}\Big)\Big\}.
    \label{basis_ququart}
\end{align}



\begin{center}
    \textbf{The protocol}
\end{center}

 
 \noindent The steps of the protocol are as follows:
 \begin{enumerate}
     \item  Alice prepares a state randomly from $B_0, B_1$ or $B_2$ with equal probability and sends it to Bob. 
\item Bob measures the received state in one of the bases $B_0$, $B_1$, or $B_2$ with  equal probability. 
    \item Steps (1-2) are repeated for a sufficiently large number of rounds. Thereafter, Alice and Bob  reveal the bases that have been employed in each round on an authenticated classical channel.
    \item {\it Detection of eavesdropping:}
     To check if there is eavesdropping,  Alice and Bob choose a subset of rounds and compare the corresponding data. In the absence of eavesdropping, there would be no errors. Otherwise,  the protocol is aborted.
     \item {\it Generation of key:}
      The rounds in which Alice and Bob have chosen bases either from the set $\{B_0, B_1\}$ or $\{B_0, B_2\}$ generate key letters. The data of the rest of the rounds (which are two in number) is discarded. 
 \end{enumerate}



 \noindent{\textbf{\textit{Key generation rule:}}}
The key generation rule for different rounds is given as follows:
\begin{enumerate}
    \item  Whenever  Alice and Bob choose the same bases, four key letters are generated, amounting to $2$ bits of information. 
    \item  In addition to this, when Alice chooses the basis $B_0$ and Bob measures either in the basis $B_1$ or $B_2$ or vice-versa, two key letters are generated as has been shown explicitly in the table (\ref{tab:distinct_sooryansh}).
\end{enumerate}
 



 
 

         
%----------------------------------------------------------------------------------------------------------------------

      
      \begin{table}[h!]
\centering
 \resizebox{10cm}{3cm}{
\begin{tabular}{|c|c|c|c|c|} 
 \hline
\multirow{3}{*}{{\bf Alice's basis}}& \multirow{3}{2cm}{{\bf Alice's state}}&\multirow{3}{*}{{\bf Bob's basis}}& \multirow{3}{2.5cm}{{\bf Post-measurement state of Bob}}&\multirow{3}{*}{{\bf Key letter}}\\
 &&&& \\
 &&&&\\ \hline\hline
\multirow{4}{*}{$B_0$} & \multirow{2}{*}{ $\ket{0}\slash\ket{1}$}&\multirow{4}{*}{$B_1$}&\multirow{2}{*}{$\frac{1}{\sqrt{2}}\big(\ket{0}\pm \ket{1}\big)$}&\multirow{2}{*}{$0$}\\
&&&&\\%\cline{2-2}\cline{4-5}
&\multirow{2}{*}{$\ket{2}\slash\ket{3}$}& &\multirow{2}{*}{$\frac{1}{\sqrt{2}}\big(\ket{2}\pm \ket{3}\big)$}&\multirow{2}{*}{$1$}\\
&&&&\\
  \hline
\multirow{4}{*}{$B_0$}  & \multirow{2}{*}{ $\ket{0}/\ket{3}$}&\multirow{4}{*}{$B_2$}&\multirow{2}{*}{$\frac{1}{\sqrt{2}}\big(\ket{0}\pm \ket{3}\big)$}&\multirow{2}{*}{$0$}\\
&&&&\\%\cline{2-2}\cline{4-5}
&\multirow{2}{*}{$\ket{1}/\ket{2}$}&&\multirow{2}{*}{$\frac{1}{\sqrt{2}}\big(\ket{1}\pm \ket{2}\big)$}&\multirow{2}{*}{$1$}\\
&&&&\\
  \hline
  \multirow{4}{*}{$B_1$}  & \multirow{2}{*}{$\frac{1}{\sqrt{2}}\big(\ket{0}\pm \ket{1}\big)$ }&\multirow{4}{*}{$B_0$}&\multirow{2}{*}{$\ket{0}/\ket{1}$}&\multirow{2}{*}{$0$}\\
&&&&\\%\cline{2-2}\cline{4-5}
&\multirow{2}{*}{$\frac{1}{\sqrt{2}}\big(\ket{2}\pm \ket{3}\big)$}&&\multirow{2}{*}{$\ket{2}/\ket{3}$}&\multirow{2}{*}{$1$}\\
&&&&\\
  \hline
\multirow{4}{*}{$B_2$}  & \multirow{2}{*}{$\frac{1}{\sqrt{2}}\big(\ket{0}\pm \ket{3}\big)$ }&\multirow{4}{*}{$B_0$}&\multirow{2}{*}{$\ket{0}/\ket{3}$}&\multirow{2}{*}{$0$}\\
&&&&\\%\cline{2-2}\cline{4-5}
&\multirow{2}{*}{$\frac{1}{\sqrt{2}}\big(\ket{1}\pm \ket{2}\big)$}&&\multirow{2}{*}{$\ket{1}/\ket{2}$}&\multirow{2}{*}{$1$}\\
&&&&\\
  \hline
\end{tabular}}
\vspace{0.05cm}
\caption{Key generation rule between Alice and Bob when both measure in different bases.}
    \label{tab:distinct_sooryansh}
\end{table}
The description of bQKD$_4$ protocol and details of bases choices corresponding to key generation are shown explicitly in figures (\ref{fig:EQKD}) and (\ref{EQKD_protocol}) respectively. 
%-----------------------------------------------------------------------
 \begin{figure}[h!]
 \centering
\includegraphics[width=0.9\textwidth]{EQKD.png}
\caption{Pictorial representation of bQKD$_4$ protocol. The inputs $x$ and $a$ at Alice's end represent choices of bases and states respectively. $x=0, 1, 2$ represent the bases $B_0, B_1$ and $B_2$ respectively. The input $y$ and the output $b$ at Bob's end represent choices of measurement bases and outcome respectively. For key generation, Alice and Bob represent $x$, $a$ and $y$, $b$ at base $2$ representation, i.e., $x\equiv (x_1x_0), a\equiv (a_1a_0), y\equiv (y_1y_0)$ and $b \equiv (b_1b_0)$. Three cases may arise: (i) If $(x_1x_0) = (y_1y_0)$, both the bits $a_1a_0$ and $b_1b_0$ contribute to key generation, (ii) if $x_1 = y_1$, $x_0 \neq y_0$, only the bit $a_1$ and $b_1$ contribute to key generation, and (iii) if $x_1 \neq y_1$, $x_0 = y_0$, only the bit $a_0$ and $b_0$ contribute to key generation.}
\label{fig:EQKD}
\end{figure}
 \begin{figure}[h!]
 \centering
\includegraphics[width=0.9\textwidth]{QKDPROTOCOL.png}
\caption{Pictorial representation of key generation for different bases choices of bQKD$_4$ protocol.}
\label{EQKD_protocol}
\end{figure}
     


%===============================================================================================================================================================================================================================================
      
     
\subsubsection{KGR}  
\label{KGR_QKD} 
In the protocol, both Alice and Bob choose one of three bases with equal probability of $1/3$. Additionally, Alice sends states with equal probability. With this information and following the key generation rule, $2$ bits of information are generated whenever Alice and Bob choose the same basis. However, in other cases, $1$ bit of information is generated. Thus, the KGR of the  bQKD$_4$ protocol is:
\begin{equation}
    r=\frac{1}{3^2}\times\big(2\times 3+1\times 4\big)=\frac{10}{9} {\rm ~bits~per~transmission.}
\end{equation}
Clearly, there is an increase in KGR as compared to the BB84 protocol with a ququart. The KGR of the latter is 1 bit per transmission.


  In the above analysis, we have considered unbiased probabilities at both Alice's and Bob's end in choosing a particular operation (states in the case of Alice and measurements in the case of Bob). However, one can also choose different bases with biased probabilities, introduced in \cite{ardehali1998efficient}, which will further enhance the KGR.



%======================================================================================================================================================

\subsection{Security of the bQKD$_4$ protocol}
\label{security_QKD}
In this section, we show the robustness of the protocol against various eavesdropping strategies. The aim of Eve is to obtain information about the key being shared without getting detected. However, Eve will not be able to do so, thanks to the non-orthogonality of states. We start with a subspace attack that Eve may employ. This is because  information on subspaces is employed to generate key letters. This attack can be implemented with current technology.
\subsubsection{Subspace attack} In the bQKD$_4$ protocol, key symbols are shared even if Alice and Bob choose different bases. Due to this, Eve may design a strategy that allows her to gain partial information with minimal errors, if not without errors. So, Eve may perform a subspace attack in which she measures an observable, say, $O$ defined as:
\begin{equation}    O\equiv\lambda_1\big(\ket{0}\bra{0}+\ket{1}\bra{1}\big)+\lambda_2\big(\ket{2}\bra{2}+\ket{3}\bra{3}\big),~~~~~~~\lambda_1\neq \lambda_2.
\end{equation}
In such an attack, Eve gets information of $1$ bit without introducing any errors, whenever Alice or Bob or both choose the bases $B_0$ or $B_1$. However, in the rounds in which both Alice and Bob choose the basis $B_2$, Eve introduces errors with a probability of $0.5$. So, in $l$ such rounds, Eve gets detected with a probability $(1-0.5^l)$.
%which approaches unity for a sufficiently large $l$. 
Let $1- \epsilon$ be the desired probability of Eve’s detection, then,
\begin{align*}
1-0.5^l=1-\epsilon \implies l =o\Big(\log \frac{1}{\epsilon}\Big),
\end{align*}
which is a weak dependence on $\epsilon^{-1}$.

%------------------------------------------------------------------------------------------------------------------

\subsubsection{Intercept-resend attack}
In such attacks, Eve measures intercepted states in the bases that are used for key generation. In the bQKD$_4$ protocol, since the three bases, {\it viz.}, $B_0, B_1, B_2$ are employed for key generation, Eve may employ all three bases to obtain information about a key or one of the bases. The three eavesdropping strategies are shown in figure (\ref{fig:intercept}). We discuss these possibilities in detail.\\

%%%%%%%%%%%%%%%%%%%%%%%%%%%%%%%%%%%%%%%%%%%%%%%%%%
\begin{figure}[h!]
\centering
\includegraphics[width=0.95\textwidth]{Eve2.png}
\caption{Pictorial representation of eavesdropping strategies in case of intercept-resend attack.}\label{fig:intercept}
\end{figure}
%%%%%%%%%%%%%%%%%%%%%%%%%%%%%%%%%%%%%%%%%%%%%%%%%%%%%%%%%%
\noindent\textbf{Strategy I - Eve measures in all the three bases randomly:}\\

\noindent This strategy of Eve corresponds to the standard case which is considered in all P\&M protocols. Since Eve measures in all three bases randomly, her measurement choice may or may not match that of Alice and Bob. 
\begin{itemize}
\item  If the choice of Eve's basis matches with that of Alice and Bob, she obtains full information ($2$ bits in this case) without introducing any errors.
\item The rounds in which Eve's choice of basis matches with either that of Alice or that of Bob, she does not introduce any errors. In addition, she may also gain information ($1$ bit), if the choice of bases follows from table $(\ref{tab:distinct_sooryansh})$.
\item The rounds in which her basis choice does not match with Alice and Bob, her interventions introduce errors. The detailed analysis of the cases has been given in table (\ref{tab:Eve_QKD}) which shows the probability of detecting Eve approaches unity for a sufficiently large $l$. %with a probability of $0.5$.  In $l$ such rounds, Eve gets detected with a probability of $(1-0.5^l)$.
\end{itemize}

      \begin{table}[h!]
\begin{center}
 \resizebox{10cm}{!} {
\begin{tabular}{|c|c|c|c|c|} 
 \hline
\multirow{4}{2cm}{ {\bf Alice's and Bob's basis}}& \multirow{4}{2cm}{{\bf Eve's basis}}&\multirow{4}{2cm}{{\bf Information gained by Eve} }&\multirow{4}{3cm}{{\bf Probability of detecting Eve in one round}}&\multirow{4}{3cm}{{\bf Probability of detecting Eve in $l$ rounds}}\\
  & & &&\\
  &&&&\\
  &&&&\\
     \hline\hline
$B_0(B_1)$  & $B_1(B_0)$&$1$ bit&$0.5$&$1-0.5^l$\\\hline
$B_0(B_2)$  & $B_2(B_0)$&$1$ bit&$0.5$&$1-0.5^l$\\\hline
$B_1$  & $B_2$&$0$ bit&$0.75$&$1-0.25^l$\\\hline
$B_2$  & $B_1$&$0$ bit&$0.75$&$1-0.25^l$\\
\hline
\end{tabular}}
\caption{Eve's information gain and probability of detection of Eve when her choice of bases does not match with that of Alice and Bob.}
    \label{tab:Eve_QKD}
    \end{center}
\end{table}

 

\noindent{\textbf{Strategy II - Eve measures only in basis $\boldsymbol{B_0}$:}}\\

\noindent In this case, whenever both Alice and Bob measure in the basis $B_0$, Eve gets complete information ($2$ bits) without introducing any errors.  However, in the rounds in which both Alice and Bob choose either bases $B_1$ or $B_2$, Eve gets partial information ($1$ bit) and also introduces errors with a probability of $0.5$. Thus, Eve's interventions, in $l$ such rounds, get detected with a $(1-0.5^l)$ probability which approaches unity for a sufficiently large $l$.
%The details of Eve's information gain and probability of detecting Eve for possible cases are given in table (\ref{tab:Eve2}).
    %  \begin{table}[h!]
%\begin{center}
 %\resizebox{10.5cm}{!} {
%\begin{tabular}{|c|c|c|c|} 
% \hline
%\multirow{3}{2cm}{ {\bf Alice and Bob's basis}} &\multirow{3}{2cm}{{\bf Information gained by Eve }}&\multirow{3}{2cm}{{\bf Probability of detecting Eve}}&\multirow{3}{2.5cm}{{\bf Probability of detecting Eve in $l$ rounds}}\\
 %&&&\\
 %&&&\\
  %   \hline\hline
%$B_1$ &$1$ bit&$0.5$&$1-0.5^l$\\
%\hline
%$B_2$ &$1$ bit&$0.5$&$1-0.5^l$\\
%\hline
%\end{tabular}}
%\vspace{0.05cm}
%\caption{Eve's information gain and probability of her detection.}
 %   \label{tab:Eve2}
  %  \end{center}
%\end{table}
\\



\noindent{\textbf{Strategy III - Eve measures in basis $\boldsymbol{B_1}$:}}\\

\noindent In such an attack, whenever either Alice or Bob or both measure in the basis $B_1$, Eve gets full information being shared without introducing any error. In the rest of the cases, Eve's interventions introduce errors and hence she gets detected.  This has been detailed in the table (\ref{tab:Eve23}).

    \begin{table}[h!]
\begin{center}
 \resizebox{!}{!} {
\begin{tabular}{|c|c|c|c|c|} 
 \hline
\multirow{4}{*}{\bf  Alice's basis} & \multirow{4}{*}{\bf Bob's basis}&\multirow{4}{2cm}{\bf Eve's information gain }&\multirow{4}{2cm}{\bf Probability of detecting Eve in one round}&\multirow{4}{2cm}{\bf Probability of detecting Eve in $l$ rounds}\\
&&& &\\
 &&& &\\
 &&&&\\
     \hline\hline
$B_0$&$B_0$  &$1$ bit  &$0.5$&$1-0.5^l$\\\hline
$B_0$&$B_2$  &$0$ bit &$0.5$&$1-0.5^l$\\\hline
$B_2$&$B_0$  &$0$ bit&$0.5$&$1-0.5^l$\\\hline
$B_2$&$B_2$ &$0$ bit&$0.75$&$1-0.25^l$
\\\hline
\end{tabular}}
\vspace{0.05cm}
\caption{Eve's information gain and probability of detection of Eve when her choice of basis does not match with that of Alice and Bob.}
    \label{tab:Eve23}
    \end{center}
\end{table}
A similar analysis holds if Eve measures in the basis $B_2$.

From the above discussion, it is clear that strategy II is more advantageous from Eve's point of view as it provides her with more information and introduces errors only in two rounds.

%------------------------------------------------------------------------------------------------------------------
\subsubsection{Entangle-and-measure attack}
In this attack, Eve entangles her ancilla with a traveling state. The effect of this operation can be expressed as:
\begin{align}    &\ket{i}_T\ket{0}_E\xrightarrow{U}\ket{ii}_{TE}, ~~~~~\forall i\in \{0,1,2,3\}.
\end{align}
 As is clear from the above equation, whenever Alice employs the basis $B_0$ Eve's actions do not introduce any errors and she gains information that is being shared.
However, this will not be the case whenever the rest of the two bases are employed. This can be understood by observing the effect of Eve's entangling operations on the states belonging to $B_1$ and $B_2$ as follows:
\begin{align}
    &B_1:\Big\{\frac{1}{\sqrt{2}}\Big(\ket{0}\pm\ket{1}\Big),~\frac{1}{\sqrt{2}}\Big(\ket{2}\pm\ket{3}\Big)\Big\}\xrightarrow{U}\Big\{\frac{1}{\sqrt{2}}\Big(\ket{00}\pm\ket{11}\Big),~\frac{1}{\sqrt{2}}\Big(\ket{22}\pm\ket{33}\Big)\Big\};\\
     &B_2:\Big\{\frac{1}{\sqrt{2}}\Big(\ket{0}\pm\ket{3}\Big),~\frac{1}{\sqrt{2}}\Big(\ket{1}\pm\ket{2}\Big)\Big\}\xrightarrow{U}\Big\{\frac{1}{\sqrt{2}}\Big(\ket{00}\pm\ket{33}\Big),~\frac{1}{\sqrt{2}}\Big(\ket{11}\pm\ket{22}\Big)\Big\}.
\end{align}
The above equations clearly indicate that Eve's interventions are detected with a probability of $0.5$ whenever Bob uses the same bases as those of Alice. In $l$ such rounds, Eve gets detected with a probability of $1-(0.5)^l$ which approaches unity for sufficiently large $l$.

We next present the robustness of the bQKD$_4$ protocol against a general entangling attack, in which Eve interacts with an arbitrary unitary operation.
%------------------------------------------------------------------------------------------------------------------------
\subsubsection{General entangling attack}
In this attack, Eve interacts her ancilla and traveling states with a unitary $U$ whose action in the computational basis can be expressed as:
\begin{align}    \ket{i}_T\ket{0}_E\xrightarrow{U}\sum_{j=0}^3\ket{j}_T\ket{E_{ij}}_E,
\end{align}
{\it where  Eve's ancilla states, viz., $\ket{E_{ij}}$ are not necessarily of the unit norm, or, for that matter, orthogonal}. Since Alice sends states from the three bases, we analyze these cases one by one. 


\subsubsection*{(I) Alice sends a state from basis $B_0$} 

Suppose Alice sends a state $\ket{0}$. If Bob measures in the basis $B_0$, Eve's actions will be detected with a probability $\big(1-\norm{\ket{E_{00}}}^2\big)$. However, if Bob measures in the basis $B_1$, Eve gets detected with a probability $p_+^{(1)}+ p_-^{(1)}$ where $p_{\pm}^{(1)}=\frac{1}{2}\big(\norm{\ket{E_{02}}\pm\ket{E_{03}}}^2\big)$.
\\
Similarly, if Bob measures in the basis $B_2$, Eve gets detected with a probability $p_+^{(2)}+p_-^{(2)}$ where $p_{\pm}^{(2)}=\frac{1}{2}\big(\norm{\ket{E_{01}}\pm\ket{E_{02}}}^2\big)$. 


\subsubsection*{(II) Alice sends  a state from basis $B_1$} 
The action of Eve's interaction on the states of basis $B_1$ can be expressed as:
\begin{align}
 \frac{1}{\sqrt{2}} \big(\ket{0}\pm \ket{1}\big)_T\ket{0}_E&\xrightarrow{U}\frac{1}{\sqrt{2}}\sum_{j=0}^3\ket{j}_T\big(\ket{E_{0j}}\pm\ket{E_{1j}}\big)_E,\nonumber\\  \frac{1}{\sqrt{2}} 
\big(\ket{2}\pm \ket{3}\big)_T\ket{0}_E&\xrightarrow{U}\frac{1}{\sqrt{2}}\sum_{j=0}^3\ket{j}_T\big(\ket{E_{2j}}\pm\ket{E_{3j}}\big)_E.
\end{align}
The rounds in which Alice sends states $\frac{1}{\sqrt{2}}\big(\ket{0}\pm\ket{1}\big)$ and  Bob measures in basis $B_0$, Eve gets detected with a probability $\frac{1}{2}\Big(\norm{\ket{E_{02}}\pm\ket{E_{12}}}^2+\norm{\ket{E_{03}}\pm\ket{E_{13}}}^2\Big)$. Similarly, when Alice sends states $\frac{1}{\sqrt{2}}\big(\ket{2}\pm\ket{3}\big)$ and  Bob measures in basis $B_0$, Eve is detected with a probability $\frac{1}{2}\Big(\norm{\ket{E_{20}}\pm\ket{E_{30}}}^2+\norm{\ket{E_{21}}\pm\ket{E_{31}}}^2\Big)$. 
\subsubsection*{(III) Alice sends a state from basis $B_2$}
The action of Eve's interaction on the states of basis $B_2$ can be expressed as:
\begin{align}
     & \frac{1}{\sqrt{2}}\big(\ket{0}\pm \ket{3}\big)_T\ket{0}_E\xrightarrow{U}\frac{1}{\sqrt{2}}\sum_{j=0}^3\ket{j}_T\big(\ket{E_{0j}}\pm\ket{E_{3j}}\big)_E, \nonumber\\   
&\frac{1}{\sqrt{2}}\big(\ket{1}\pm \ket{2}\big)_T\ket{0}_E\xrightarrow{U}\frac{1}{\sqrt{2}}\sum_{j=0}^3\ket{j}_T\big(\ket{E_{1j}}\pm\ket{E_{2j}}\big)_E.
\end{align}

Consider the round in which Alice sends a state $\frac{1}{\sqrt{2}}\big(\ket{0}+\ket{3}\big)$. If Bob measures in the basis $B_0$, he gets wrong result with probability $\frac{1}{2}\Big(\big(\norm{\ket{E_{01}}+\ket{E_{31}}}^2\big)+\big(\norm{\ket{E_{02}}+\ket{E_{32}}}^2\big)\Big)$. 
However, if Bob measures in the basis $B_2$ itself, he gets the correct result with a probability $p=\frac{1}{4}\big(\norm{\ket{E_{00}}+\ket{E_{30}}+\ket{E_{03}}+\ket{E_{33}}}^2\big)$. Thus, Eve's interventions get detected with a probability $(1-p).$

%----------------------------------------------------------------------------------------------------------------------------
\subsubsection{Photon-number-splitting (PNS) attack}
\label{PNS} 
Though theoretical proposals of QKD protocols employ single photons, in experiments, these are implemented with weak coherent pulses (WCP). Employment of WCP facilitates Eve with opportunities that are otherwise unavailable. PNS attack is one such strategy. In this attack, Eve can take one photon off a multi-photon pulse and block all the single-photon pulses. This technique will help Eve in obtaining all the shared information without introducing any errors. Alice and Bob can detect this attack by employing an additional decoy pulse with a lower mean photon number as has been discussed in \cite{acin2004coherent}. Since both pulses are sent randomly, Eve cannot distinguish between the two. As a result, her interventions will change the statistics revealing her presence.


 \subsubsection{Trojan-horse attack}\label{Trojan}
 
 This attack is another example in which Eve employs imperfections of the devices employed to implement protocols, which was first introduced in \cite{gisin2006trojan}. One attack in this category is the `invisible photon attack'. In this attack, Eve sends photons of  different wavelengths to determine  the measurement settings of Bob's measurement device. These attacks can be circumvented by using a filter of a specified wavelength. As a result, only signal photons can reach the detector, thus removing the possibility of such attacks.
 
 Another important attack in this category is the `delay photon attack'. Bob by counting the number of photons for a randomly chosen subset of the signal can detect this attack. This completes the robustness to eavesdropping attacks.


  This concludes our discussion of bQKD with ququart systems. In this protocol, data of the rounds in which Alice and Bob choose $B_1$ and $B_2$ respectively, and vice-versa are completely discarded. However, by employing effective qubits encoded in quhex  and, for that matter, in qudit ($d\geq 6)$ systems, even some of these runs can be employed for key generation. This further limits the data which is to be discarded and so using resources in their full form. We show it explicitly by presenting a bQKD protocol with quhex systems designated as bQKD$_6$ in what follows.
\subsection{bQKD with quhex systems (bQKD$_6$)}
\label{QKD_six}
In this section, we present a bQKD protocol with a quhex system. As before, let there be two participants, {\it viz.}, Alice and Bob who wish to share a secret key. 

\noindent{\textbf{\textit{Aim:}}} distribution of a key with minimal discarding of data.

\noindent{\textbf{\textit{Resources:}}} effective qubits in quhex systems with three bases ${\cal B}_0, {\cal B}_1, {\cal B}_2$
\begin{align}
    &{\cal B}_0\equiv\Big\{\ket{0},\ket{1},\ket{2},\ket{3},\ket{4},\ket{5}\Big\};\nonumber\\
    &{\cal B}_1\equiv\Big\{\frac{1}{\sqrt{2}}\Big(\ket{0}\pm\ket{1}\Big),\frac{1}{\sqrt{2}}\Big(\ket{2}\pm\ket{3}\Big),\frac{1}{\sqrt{2}}\Big(\ket{4}\pm\ket{5}\Big)\Big\};\nonumber\\
      &{\cal B}_2\equiv\Big\{\frac{1}{\sqrt{2}}\Big(\ket{1}\pm\ket{2}\Big),\frac{1}{\sqrt{2}}\Big(\ket{3}\pm\ket{4}\Big),\frac{1}{\sqrt{2}}\Big(\ket{5}\pm\ket{0}\Big)\Big\}.
\label{bases_six_sooryansh}
\end{align}
The steps of the protocol are the same as those of bQKD$_4$ protocol, given in section (\ref{QKD_four}). The subtle difference is that data corresponding to none of bases choices are discarded in contrast to bQKD$_4$ protocol. \\


\noindent{\textbf{Key generation rule:}} For different choices of bases by Alice and Bob, the key generation rule is as follows:
\begin{itemize}
    \item The rounds in which Alice and Bob measure in the same bases,  $6$ key letters are generated with equal probabilities, sharing $\log_2{6}$ bits of information.
    \item The rounds in which either Alice or Bob chooses the basis ${\cal B}_0$, $3$ key symbols are generated with equal probability as given in table (\ref{tab:distinct_1}).
    \item The third case is unique to this protocol. This corresponds to  the rounds in which Alice and Bob choose the bases ${\cal B}_1$ and ${\cal B}_2$ respectively or vice-versa. In contrast to bQKD$_4$ protocol, in bQKD$_6$ protocol, these rounds contribute to key generation. The key generation rule for the same is given in table (\ref{tab:B_1B_2_1}).
\end{itemize}
 
%=================================================================================================================================== 
\begin{enumerate}
    \item The rounds in which Alice sends the states $\frac{1}{\sqrt{2}}(\ket{2}\pm \ket{3})$ are simply discarded. 
    \item Thereafter, the rounds in which Bob gets $\frac{1}{\sqrt{2}}(\ket{5}\pm \ket{0})$ are discarded.
    \item This discarding leads to a perfect correlation between the subspace from which Alice has sent a state and the subspace to which Bob's post-measurement belongs. This perfect correlation is used to assign key letters as shown in table (\ref{tab:B_1B_2_1}).
\end{enumerate}
A similar key generation rule holds when Alice sends a state ${\cal B}_2$ and Bob measures in the basis ${\cal B}_1$ (shown in table (\ref{tab:B_1B_2_1})).


      \begin{table}[h]
\begin{center}
 \resizebox{!}{3.5cm} {
\begin{tabular}{|c|c|c|c|c|} 
 \hline
\multirow{3}{*}{\bf Alice's basis}& \multirow{3}{2cm}{{\bf Alice's state}}&\multirow{3}{2.7cm}{{\bf Bob's basis}}& \multirow{3}{3cm}{{\bf Post-measurement state of Bob}}&\multirow{3}{*}{\bf Key symbol}\\
 &   &&& \\
 &&&&\\
     \hline\hline
\multirow{6}{*}{${\cal B}_0$}  & \multirow{2}{*}{ $\ket{0}/\ket{1}$}&\multirow{6}{*}{${\cal B}_1$}&\multirow{2}{*}{$\frac{1}{\sqrt{2}}\big(\ket{0}\pm \ket{1}\big)$}&\multirow{2}{*}{$0$}\\
&&&&\\%\cline{2-2}\cline{4-5}
&\multirow{2}{*}{$\ket{2}/\ket{3}$}&&\multirow{2}{*}{$\frac{1}{\sqrt{2}}\big(\ket{2}\pm \ket{3}\big)$}&\multirow{2}{*}{$1$}\\
&&&&\\%\cline{2-2}\cline{4-5}
&\multirow{2}{*}{$\ket{4}/\ket{5}$}&&\multirow{2}{*}{$\frac{1}{\sqrt{2}}\big(\ket{4}\pm \ket{5}\big)$}&\multirow{2}{*}{$2$}\\
&&&&\\
  \hline
  %======================================================================================================
\multirow{6}{*}{${\cal B}_0$}  & \multirow{2}{*}{ $\ket{1}/\ket{2}$}&\multirow{6}{*}{${\cal B}_2$}&\multirow{2}{*}{$\frac{1}{\sqrt{2}}\big(\ket{1}\pm \ket{2}\big)$}&\multirow{2}{*}{$0$}\\
&&&&\\%\cline{2-2}\cline{4-5}
&\multirow{2}{*}{$\ket{3}/\ket{4}$}&&\multirow{2}{*}{$\frac{1}{\sqrt{2}}\big(\ket{3}\pm \ket{4}\big)$}&\multirow{2}{*}{$1$}\\
&&&&\\%\cline{2-2}\cline{4-5}
&\multirow{2}{*}{$\ket{5}/\ket{0}$}&&\multirow{2}{*}{$\frac{1}{\sqrt{2}}\big(\ket{5}\pm \ket{0}\big)$}&\multirow{2}{*}{$2$}\\
&&&&\\
  \hline
  %=========================================================================
  \multirow{6}{*}{${\cal B}_1$}  & \multirow{2}{*}{$\frac{1}{\sqrt{2}}\big(\ket{0}\pm \ket{1}\big)$ }&\multirow{6}{*}{${\cal B}_0$}&\multirow{2}{*}{$\ket{0}/\ket{1}$}&\multirow{2}{*}{$0$}\\
&&&&\\%\cline{2-2}\cline{4-5}
&\multirow{2}{*}{$\frac{1}{\sqrt{2}}\big(\ket{2}\pm \ket{3}\big)$}&&\multirow{2}{*}{$\ket{2}/\ket{3}$}&\multirow{2}{*}{$1$}\\
&&&&\\%\cline{2-2}\cline{4-5}
&\multirow{2}{*}{$\frac{1}{\sqrt{2}}\big(\ket{4}\pm \ket{5}\big)$}&&\multirow{2}{*}{$\ket{4}/\ket{5}$}&\multirow{2}{*}{$2$}\\
&&&&\\
  \hline
  %==============================================================================================
\multirow{6}{*}{${\cal B}_2$}  & \multirow{2}{*}{$\frac{1}{\sqrt{2}}\big(\ket{1}\pm \ket{2}\big)$ }&\multirow{6}{*}{${\cal B}_0$}&\multirow{2}{*}{$\ket{1}/\ket{2}$}&\multirow{2}{*}{$0$}\\
&&&&\\%\cline{2-2}\cline{4-5}
&\multirow{2}{*}{$\frac{1}{\sqrt{2}}\big(\ket{3}\pm \ket{4}\big)$}&&\multirow{2}{*}{$\ket{3}/\ket{4}$}&\multirow{2}{*}{$1$}\\
&&&&\\%\cline{2-2}\cline{4-5}
&\multirow{2}{*}{$\frac{1}{\sqrt{2}}\big(\ket{5}\pm \ket{0}\big)$}&&\multirow{2}{*}{$\ket{5}/\ket{0}$}&\multirow{2}{*}{$2$}\\
&&&&\\
  \hline
  %=====================================================================
\end{tabular}
}
\vspace{0.25cm}
\caption{Key generation rule between Alice and Bob when one of them employs the basis ${\cal B}_0$.}
    \label{tab:distinct_1}
    \end{center}
\end{table}
%----------------------------------------------------------------------------------------------------------------------------------
%\begin{table}[h]
%\begin{center}
% \resizebox{!}{2cm} {
%\begin{tabular}{|c|c|c|} 
% \hline
% \multirow{3}{3cm}{{\bf Alice's state from the basis ${\cal B}_1$}}& \multirow{3}{3cm}{{\bf Post-measurement state of Bob (in the basis ${\cal B}_2$)}}&\multirow{3}{*}{\bf Key letter}\\
 %  && \\
% &&\\
 %    \hline\hline
 %\multirow{4}{*}{ $\frac{1}{\sqrt{2}}\big(\ket{0}\pm\ket{1}\big)$}&\multirow{2}{*}{$\frac{1}{\sqrt{2}}\big(\ket{1}\pm \ket{2}\big)$}&\multirow{2}{*}{$0$}\\
 %&&\\
%&\multirow{2}{*}{$\frac{1}{\sqrt{2}}\big(\ket{5}\pm \ket{0}\big)$}&\multirow{2}{*}{$\perp$}\\
%&&\\
 % \hline
  %--------------------------------------
%\multirow{4}{*}{ $\frac{1}{\sqrt{2}}\big(\ket{2}\pm\ket{3}\big)$}&\multirow{2}{*}{$\frac{1}{\sqrt{2}}\big(\ket{1}\pm \ket{2}\big)$}&\multirow{4}{*}{$\perp$}\\
%&&\\
%&\multirow{2}{*}{$\frac{1}{\sqrt{2}}\big(\ket{3}\pm \ket{4}\big)$}&\\
%&&\\\hline
%------------------------------------------------------------------------
%\multirow{4}{*}{$\frac{1}{\sqrt{2}}\big(\ket{4}\pm\ket{5}\big)$}&\multirow{2}{*}{$\frac{1}{\sqrt{2}}\big(\ket{3}\pm \ket{4}\big)$}&\multirow{2}{*}{$1$}\\
%&&\\
%&\multirow{2}{*}{$\frac{1}{\sqrt{2}}\big(\ket{5}\pm \ket{0}\big)$}&\multirow{2}{*}{$\perp$}\\
%&&\\
 % \hline
%\end{tabular}
%}
%\vspace{0.25cm}
%\caption{Key generation rule in the rounds when Alice sends state from the basis ${\cal B}_1$ and Bob measures in the basis ${\cal B}_2$. The symbol $\perp$ corresponds to the rounds which are discarded. This key generation rule is motivated by what is being employed in B$92$ protocol.}
 %   \label{tab:B_1B_2_1}
  %  \end{center}
%\end{table}
%============================================================================================
\begin{table}[h]
\begin{center}
 \resizebox{!}{3.5cm} {
\begin{tabular}{|c|c|c|c|c|} 
 \hline
 \multirow{3}{3cm}{{ \bf Alice's  basis }}&\multirow{3}{3cm}{{\bf Alice's state}}&\multirow{3}{3cm}{{\bf Bob's  basis }}& \multirow{3}{3cm}{{\bf Post-measurement state of Bob (in the basis ${\cal B}_2$)}}&\multirow{3}{*}{\bf Key letter}\\
   &&&& \\
 &&&&\\
     \hline\hline
\multirow{6}{*}{\large ${\cal B}_1$}& \multirow{6}{*}{ \large$\frac{1}{\sqrt{2}}\big(\ket{0}\pm\ket{1}\big)$}&\multirow{6}{*}{\large ${\cal B}_2$}&\multirow{3}{*}{\large$\frac{1}{\sqrt{2}}\big(\ket{1}\pm \ket{2}\big)$}&\multirow{3}{*}{\large$0$}\\
&&&&\\
&&&&\\
&&&\multirow{3}{*}{\large$\frac{1}{\sqrt{2}}\big(\ket{5}\pm \ket{0}\big)$}&\multirow{3}{*}{\large$\perp$}\\
&&&&\\
&&&&\\
  \hline
  %--------------------------------------
\multirow{6}{*}{\large ${\cal B}_1$}&\multirow{6}{*}{\large $\frac{1}{\sqrt{2}}\big(\ket{2}\pm\ket{3}\big)$}&\multirow{6}{*}{\large ${\cal B}_2$}&\multirow{3}{*}{\large$\frac{1}{\sqrt{2}}\big(\ket{1}\pm \ket{2}\big)$}&\multirow{6}{*}{\large$\perp$}\\
&&&&\\
&&&&\\
&&&\multirow{3}{*}{\large$\frac{1}{\sqrt{2}}\big(\ket{3}\pm \ket{4}\big)$}&\\
&&&&\\
&&&&\\\hline
%------------------------------------------------------------------------
\multirow{6}{*}{\large ${\cal B}_1$}&\multirow{6}{*}{\large$\frac{1}{\sqrt{2}}\big(\ket{4}\pm\ket{5}\big)$}&\multirow{6}{*}{\large ${\cal B}_2$}&\multirow{3}{*}{\large$\frac{1}{\sqrt{2}}\big(\ket{3}\pm \ket{4}\big)$}&\multirow{3}{*}{\large$1$}\\
&&&&\\
&&&&\\
&&&\multirow{3}{*}{\large$\frac{1}{\sqrt{2}}\big(\ket{5}\pm \ket{0}\big)$}&\multirow{3}{*}{\large$\perp$}\\
&&&&\\
&&&&\\
  \hline\hline
  \multirow{6}{*}{\large ${\cal B}_2$} &\multirow{6}{*}{\large $\frac{1}{\sqrt{2}}\big(\ket{1}\pm\ket{2}\big)$}&\multirow{6}{*}{\large ${\cal B}_1$}&\multirow{3}{*}{\large$\frac{1}{\sqrt{2}}\big(\ket{0}\pm \ket{1}\big)$}&\multirow{3}{*}{\large$\perp$}\\
&&&&\\
&&&&\\
&&&\multirow{3}{*}{\large$\frac{1}{\sqrt{2}}(\ket{2}\pm \ket{3})$}&\multirow{3}{*}{\large$0$}\\
&&&&\\
&&&&\\
  \hline
 \multirow{6}{*}{\large ${\cal B}_2$}  &\multirow{6}{*}{ \large$\frac{1}{\sqrt{2}}\big(\ket{3}\pm\ket{4}\big)$}&\multirow{6}{*}{\large ${\cal B}_1$} &\multirow{3}{*}{\large$\frac{1}{\sqrt{2}}\big(\ket{2}\pm \ket{3}\big)$}&\multirow{6}{*}{\large$\perp$}\\
&&&&\\
&&&&\\
&&&\multirow{3}{*}{\large$\frac{1}{\sqrt{2}}(\ket{4}\pm \ket{5})$}&\\
&&&&\\
&&&&\\\hline

 \multirow{6}{*}{ \large${\cal B}_2$}&\multirow{6}{*}{\large $\frac{1}{\sqrt{2}}\big(\ket{5}\pm\ket{0}\big)$}& \multirow{6}{*}{\large ${\cal B}_1$}&\multirow{3}{*}{\large$\frac{1}{\sqrt{2}}\big(\ket{4}\pm \ket{5}\big)$}&\multirow{3}{*}{\large$1$}\\
&&&&\\
&&&&\\
&&&\multirow{3}{*}{\large$\frac{1}{\sqrt{2}}\big(\ket{0}\pm \ket{1}\big)$}&\multirow{3}{*}{\large$\perp$}\\
&&&&\\
&&&&\\
  \hline
\end{tabular}
}
\vspace{0.25cm}
\caption{Key generation rule in the rounds when Alice sends states from the basis ${\cal B}_1 ({\cal B}_2)$ and Bob measures in the basis ${\cal B}_2 ({\cal B}_1)$. The symbol $\perp$ corresponds to the rounds which are discarded. This key generation rule is motivated by what is being employed in B$92$ protocol. Discarding certain rounds is required because of unambiguous subspace discrimmination.}
    \label{tab:B_1B_2_1}
    \end{center}
\end{table}




%-----------------------------------------------------------------------------------------------------------------
\subsubsection{KGR}   
Alice and Bob have three bases which they choose with equal probability. From the key generation rule discussed above, for each basis choice, key letters are generated with equal probability. Therefore, the KGR of the protocol is,
\begin{equation}
    r_{{\rm bQKD}_6}=\frac{1}{3^2}\times\big(\log_2{6}\times 3+\log_2{3}\times 4+\frac{2}{3}\times \log_22\big)\approx  1.64 {\rm ~bits~per~transmission.}
\end{equation}
At this juncture, we wish to point out that there is an increase in KGR as compared to the BB84 protocol with a quhex system. The KGR of the latter is $\frac{\rm log_2 6}{2}\approx 1.29$ bit per transmission.





 In the subsequent section, we generalize the protocol to qudit systems, with $d$ being even.

%==============================================================================================================
\section{Generalisation to qudit systems}
\label{generalisation}
The generalization of the bQKD protocol to qudit systems is straightforward, with the only condition of $d$ being even. It is clear from the bQKD$_4$ and bQKD$_6$ protocols that a minimum of three bases are required for the secure distribution of keys. 
These bases are a must in order to detect subspace attacks as has been shown in figure (\ref{Subspace attack}).
\begin{figure}[h!]
\centering
\includegraphics[width=0.65\textwidth]{Subspaceattack}
\caption{Schematic representation of choice of bases for the bQKD$_6$ protocol involving a quhex system (six-dimensional quantum system). Since the states belonging to the three bases do not allow for a reducible representation, all the subspace attacks will be detected.}\label{Subspace attack}
\end{figure}
%==================================================================================================

For this reason,  the bQKD$_d$ protocol employs three bases. 
Similar to the illustrative protocols, we fix the basis ${\cal B}_0$ to be the computational basis. The rest of the two bases, {\it viz.}, ${\cal B}_1$ and ${\cal B}_2$ consist of states which are superpositions of two states belonging to the computational basis. The three bases are given explicitly as follows:
\begin{align}
    &{\cal B}_0\equiv \Big\{\ket{0},\ket{1},\cdots, \ket{d-1}\Big\},\nonumber\\
    &   {\cal  B}_1\equiv \Big\{\frac{1}{\sqrt{2}}\Big(\ket{0}\pm\ket{1}\Big), \frac{1}{\sqrt{2}}\Big(\ket{2}\pm\ket{3}\Big),\cdots, \frac{1}{\sqrt{2}}\Big(\ket{d-2}\pm \ket{d-1}\Big)\Big\},\nonumber\\
    &   {\cal  B}_2\equiv \Big\{\frac{1}{\sqrt{2}}\Big(\ket{1}\pm\ket{2}), \frac{1}{\sqrt{2}}\Big(\ket{3}\pm\ket{4}\Big),\cdots, \frac{1}{\sqrt{2}}\Big(\ket{d-1}\pm \ket{0}\Big)\Big\}.
\end{align}

The steps of the protocols are essentially the same as in section (\ref{QKD_four}). \\
The key generation rule is the same as employed in the previous two protocols which we describe here for the sake of completeness. 

\begin{enumerate}
\item The rounds in which both Alice and Bob choose the same bases, Bob can uniquely identify the state sent by Alice. So, he obtains the full information which, in this case, is $\log_2{d}$ bits.
\item The rounds in which Alice and Bob choose different bases from either sets $\{{\cal B}_0,{\cal B}_1\}$ and $\{{\cal B}_0,{\cal B}_2\}$, a key letter is generated by employing the rule given in table (\ref{tab:rule_d}).

      \begin{table}[h]
\begin{center}
 \resizebox{11cm}{3.5cm} {
\begin{tabular}{|c|c|c|c|c|} 
 \hline
\multirow{3}{*}{\bf Alice's basis}& \multirow{3}{2cm}{{\bf Alice's state}}&\multirow{3}{2.7cm}{{\bf Bob's basis}}& \multirow{3}{3cm}{{\bf Post-measurement state of Bob}}&\multirow{3}{*}{\bf Key symbol}\\
 &   &&& \\
 &&&&\\
     \hline\hline
\multirow{6}{*}{$\boldsymbol{{\cal B}_0}$}  & \multirow{2}{*}{ $\boldsymbol{\ket{0}/\ket{1}}$}&\multirow{6}{*}{$\boldsymbol{{\cal B}_1}$}&\multirow{2}{*}{$\boldsymbol{\frac{1}{\sqrt{2}}(\ket{0}\pm \ket{1}})$}&\multirow{2}{*}{$\boldsymbol{0}$}\\
&&&&\\
&\multirow{2}{*}{$\boldsymbol{\vdots}$}&&\multirow{2}{*}{$\boldsymbol{\vdots}$}&\multirow{2}{*}{$\boldsymbol{\vdots}$}\\
&&&&\\
&\multirow{2}{*}{$\boldsymbol{\ket{d-2}/\ket{d-1}}$}&&\multirow{2}{*}{$\boldsymbol{\frac{1}{\sqrt{2}}(\ket{d-2}\pm \ket{d-1}})$}&\multirow{2}{*}{$\boldsymbol{\dfrac{d}{2}-1}$}\\
&&&&\\
  \hline
  %======================================================================================================
\multirow{6}{*}{$\boldsymbol{{\cal B}_0}$}  & \multirow{2}{*}{ $\boldsymbol{\ket{1}/\ket{2}}$}&\multirow{6}{*}{$\boldsymbol{{\cal B}_2}$}&\multirow{2}{*}{$\boldsymbol{\frac{1}{\sqrt{2}}(\ket{1}\pm \ket{2}})$}&\multirow{2}{*}{$\boldsymbol{0}$}\\
&&&&\\%\cline{2-2}\cline{4-5}
&\multirow{2}{*}{$\boldsymbol{\vdots}$}&&\multirow{2}{*}{$\boldsymbol{\vdots}$}&\multirow{2}{*}{$\boldsymbol{\vdots}$}\\
&&&&\\
&\multirow{2}{*}{$\boldsymbol{\ket{d-1}/\ket{0}}$}&&\multirow{2}{*}{$\boldsymbol{\frac{1}{\sqrt{2}}(\ket{d-1}\pm \ket{0}})$}&\multirow{2}{*}{$\boldsymbol{\dfrac{d}{2}-1}$}\\
&&&&\\
  \hline
  %=========================================================================
  \multirow{6}{*}{$\boldsymbol{{\cal B}_1}$}  & \multirow{2}{*}{$\boldsymbol{\frac{1}{\sqrt{2}}(\ket{0}\pm \ket{1}})$ }&\multirow{6}{*}{$\boldsymbol{{\cal B}_0}$}&\multirow{2}{*}{$\boldsymbol{\ket{0}/\ket{1}}$}&\multirow{2}{*}{$\boldsymbol{0}$}\\
&&&&\\%\cline{2-2}\cline{4-5}
&\multirow{2}{*}{$\boldsymbol{\vdots}$}&&\multirow{2}{*}{$\boldsymbol{\vdots}$}&\multirow{2}{*}{$\boldsymbol{\vdots}$}\\
&&&&\\%\cline{2-2}\cline{4-5}
&\multirow{2}{*}{$\boldsymbol{\frac{1}{\sqrt{2}}(\ket{d-2}\pm \ket{d-1}})$}&&\multirow{2}{*}{$\boldsymbol{\ket{d-2}/\ket{d-1}}$}&\multirow{2}{*}{$\boldsymbol{\dfrac{d}{2}-1}$}\\
&&&&\\
  \hline
  %==============================================================================================
\multirow{6}{*}{$\boldsymbol{{\cal B}_2}$}  & \multirow{2}{*}{$\boldsymbol{\frac{1}{\sqrt{2}}(\ket{1}\pm \ket{2}})$ }&\multirow{6}{*}{$\boldsymbol{{\cal B}_0}$}&\multirow{2}{*}{$\boldsymbol{\ket{1}/\ket{2}}$}&\multirow{2}{*}{$\boldsymbol{0}$}\\
&&&&\\%\cline{2-2}\cline{4-5}
&\multirow{2}{*}{$\boldsymbol{\vdots}$}&&\multirow{2}{*}{$\boldsymbol{\vdots}$}&\multirow{2}{*}{$\boldsymbol{\vdots}$}\\
&&&&\\%\cline{2-2}\cline{4-5}
&\multirow{2}{*}{$\boldsymbol{\frac{1}{\sqrt{2}}(\ket{d-1}\pm \ket{0}})$}&&\multirow{2}{*}{$\boldsymbol{\ket{d-1}/\ket{0}}$}&\multirow{2}{*}{$\boldsymbol{\dfrac{d}{2}-1}$}\\
&&&&\\
  \hline
  %=====================================================================
\end{tabular}
}
\vspace{0.25cm}
\caption{Key generation rule between Alice and Bob when both measures in distinct bases. }
    \label{tab:rule_d}
    \end{center}
\end{table}

%-------------------------------------------------------For the case when d=4n-------

%=======================================================================================================
    \item  To generate key letters, Bob needs to determine uniquely the state sent by Alice. For this purpose, it is defined that only alternate states are used for key generation. The rounds in which Alice and Bob choose different bases from the set $\{{\cal B}_1,{\cal B}_2\}$, Bob - for a pair of outcomes - can determine uniquely the set from which Alice has sent states. As a result, $\frac{d}{4}$ key letters are generated. The details of the two cases are given as follows:
\begin{itemize}
   % \item Suppose that Alice sends a state from the basis ${\cal B}_1$ and Bob measures in the basis ${\cal B}_2$.   Whenever Bob gets outcomes corresponding to the post-measurement states $\ket{i}\pm\ket{j}\in {\cal B}_2$, Alice must have sent states from the set $\{\ket{i-1}\pm\ket{i},\ket{j}\pm\ket{j+1}\}\in {\cal B}_1$. 
   % \item Similarly, in the rounds in which Bob measures in the basis ${\cal B}_2$ and gets outcomes corresponding to post-measurement states $\ket{i}\pm\ket{j}\in {\cal B}_1$, Alice must have sent states from the set $\{\ket{i-1}\pm\ket{i},\ket{j}\pm\ket{j+1}\}\in {\cal B}_2$.
    \item For $d=4n ~(n \geq 2)$, in order to unambiguously discriminate the subspaces, Alice first discards the rounds in which she has sent the states $\frac{1}{\sqrt{2}}(\ket{2}\pm \ket{3}), \frac{1}{\sqrt{2}}(\ket{6}\pm \ket{7}), \cdots, \frac{1}{\sqrt{2}}(\ket{d-2}\pm \ket{d-1})$. This leads to unambiguous subspace discrimination, which, in turn, leads to the generation of key letters, as shown in the table (\ref{tab:B_1B_2_4}).
    \item For $d=4n+2~(n \geq 1)$, in order to unambiguously discriminate the subspaces, as in previous case,  Alice first discards the rounds in which alternate states are sent. In addition to this, Bob discards the rounds in which his post-measurement states are   $\frac{1}{\sqrt{2}}(\ket{d-1}\pm \ket{0})$. This leads to unambiguous subspace discrimination, which, in turn, leads to the generation of key letters, as shown in the table (\ref{tab:B_1B_2_7}).
\end{itemize}
\end{enumerate}



\subsection{KGR}
\label{KGR_qudit}
Employing the key generation rules described above, key letters are generated in most of the rounds.  As a result, the key generation rate increases considerably as can be seen below. 

\noindent For $d=4n$ where $ (n\geq 2)$
\begin{align}
 r_{{\rm bQKD}_d}=\frac{1}{3^2}\times\Big(\log_2{d}\times 3+\log_2{\frac{d}{2}}\times 4+\log_2{\frac{d}{4}}\Big){\rm ~bits~per~transmission},
\end{align}
For $d=4n+2$
\begin{align}
 r_{{\rm bQKD}_d} &=\frac{1}{3^2}\times\Big\{\log_2{d}\times 3+\log_2{\frac{d}{2}}\times 4+\frac{d-2}{d} h\Big(\frac{1}{2n},\frac{1}{2n},\frac{1}{n},\cdots,\frac{1}{n}\Big)\Big\}\nonumber\\ &~~~~{\rm ~bits~per~transmission}.
\end{align}
 The symbol $h(p_1, p_2, \cdots, p_N)$ represents Shannon entropy defined as $h(p_1, p_2, \cdots, p_N)=-\sum_{i=1}^Np_i\log_2p_i$.
 The KGR of the latter is given by $r=\frac{\log_2{d}}{2}$. The value of $r_{{\rm bQKD}_d}$  is much higher than KGR of $d$-dimensional BB$84$, as is also reflected from the plot shown in figure (\ref{Two_KGR}).% Figure (\ref{Two_KGR}) shows the KGR of the two protocols with the dimensionality of systems whereas figure (\ref{KGR_fractional}) shows the fractional increase in the KGR of bQKD protocol w.r.t KGR of high-dimensional BB$84$ protocol with dimensions.
\begin{figure}[h!]
\centering
\includegraphics[width=0.8\textwidth]{Ratios.png}
\caption{Plot of ratio of the key generation rates of the bQKD$_d$ (denoted  by $r_{bQKD_d}$) and that of BB84 protocol (denoted by $r$) with qudits involving only two bases ({\it viz.}, the computational basis and the Fourier basis).}
\label{Two_KGR}
\end{figure}
%\begin{figure}[h!]
%\centering
%\includegraphics[width=0.8\textwidth]{KGR_fractional.png}
%\caption{Plot of fractional difference in key generation rates of the bQKD$_N$ protocol (denoted by $r$) and that of BB84 protocol with qudits involving only two bases ({\it viz.}, the computational basis and the Fourier basis) (denoted by $r_{\rm MUB}$).}
%\label{KGR_fractional}
%\end{figure}


%------------------------------------------------------------------------------------------------------


%===================================================================================================
The security analysis given in section (\ref{security_QKD}) of the bQKD$_4$ protocol can be extended to bQKD$_d$ protocol straightforwardly. This ensures the robustness of the protocol against various eavesdropping strategies.







Since this idea provides an advantage in QKD protocols, it becomes worthwhile to see what advantage this idea offers when employed in SQKD protocols which we study in the subsequent section. Please note that for the purpose of an uncluttered discussion, we present a brief recap of the semi-quantum key distribution protocol proposed by Boyer \cite{Boyer07} in Appendix (\ref{SQKD_Boyer}). 
%-------------------------------------------

\section{Boosted semi-quantum key distribution (bSQKD)}
\label{boosted SQKD}
In this section, we propose a boosted semi-quantum key distribution (bSQKD$_4$) protocol with ququart systems.\\
\subsection{The bSQKD$_4$ protocol}
In this protocol, Alice has all the quantum capabilities, i.e., she can prepare a state and perform measurements on it in any  basis. However, as Bob is classical, he can perform a measurement in only one of the basis states, i.e.,  the computational basis. 
 Alice has three sets of four-dimensional basis states as given below (which is the same as given in equation (\ref{basis_ququart})): 
\begin{align}
    &{B}_0=\Big\{\ket{0},\ket{1},\ket{2},\ket{3}\Big\}\nonumber\\
    & {B}_1=\Big\{\frac{1}{\sqrt{2}}\Big(\ket{0}\pm\ket{1}\Big), \frac{1}{\sqrt{2}}\Big(\ket{2}\pm\ket{3}\Big)\Big\};~~\nonumber\\
   & {B}_2=\Big\{\frac{1}{\sqrt{2}}\Big(\ket{0}\pm\ket{3}\Big), \frac{1}{\sqrt{2}}\Big(\ket{1}\pm\ket{2}\Big)\Big\}.\nonumber
\end{align}
As Bob is classical, he can measure  only in the computational basis i.e., $B_0$.
 The steps for the protocol are as follows:
\begin{enumerate}
    \item Alice randomly prepares one of the states from any of the three basis sets and sends it to Bob.
    \item Bob has two choices with equal probability:
    \begin{itemize}
        \item He performs the measurement in the basis $B_0$ and sends the post-measurement state to Alice.
        \item He does not measure and simply sends the state back to Alice.
    \end{itemize}
    \item Alice measures the received state in the same basis in which she has prepared the state.
    \item This completes one round. The same process is repeated for many rounds.
    \item After a sufficient number of rounds, Bob reveals the rounds in which he performs the measurement.
    \item Alice uses data from other rounds to check the presence of an eavesdropper.
    \item In the absence of Eve, Alice reveals the bases employed in each round.
    \item The rounds in which Bob performs a measurement constitute a key.
\end{enumerate}


\begin{figure}[h!]
\centering
\includegraphics[width=0.95\textwidth]{ESQKD.png}
\caption{Pictorial representation of bSQKD$_4$ protocol. The inputs $x$ and $a$ at Alice's end represent choices of bases and states respectively. $x=0, 1, 2$ represent the bases $B_0, B_1$ and $B_2$ respectively. The input $y$ and the output $b$ at Bob's end represents choice of measurement and outcome respectively. For key generation, Alice and Bob represent $x$, $a$ and $y$, $b$ at base $2$ representation, i.e., $x\equiv (x_1x_0), a\equiv (a_1a_0), y\equiv (y_1y_0)$ and $b \equiv (b_1b_0)$. Three cases may arise: (i) If $x_1x_0 = 00$, both the bits $a_1a_0$ and $b_1b_0$ contribute to key generation, (ii) if $x_1 = 0$, $x_0 \neq 0$, only the bit $a_1$ and $b_1$ contribute to key generation, and (iii)if $x_1 \neq 0$, $x_0 = 0$, only the bit $a_0$ and $b_0$ contribute to key generation.}
\label{ESQKD_sooryansh}
\end{figure}
 \begin{figure}[h!]
 \centering
\includegraphics[width=0.9\textwidth]{SQKDPROTOCOL.png}
\caption{Pictorial representation of bSQKD$_4$ protocol showing the number of bits generated and rounds used for eavesdropping check (E.C.).}
\label{SQKD_PROTOCOL_rounds_sooryansh}
\end{figure}

The rounds in which Alice sends a state from the basis $B_0$, two bits of information is generated. However, if Alice sends states from basis $B_1$ or $B_2$, $1$ bit of information is generated following the last two rows of the table (\ref{tab:distinct_sooryansh}). This clearly indicates that the data of no rounds in which Bob measures is discarded in contrast to conventional SQKD protocols.

 The protocol has been schematically shown in figure (\ref{ESQKD_sooryansh}) reflecting how different bases leads to key generation. In addition, the number of bits generated in each round and rounds used for eavesdropping check have been schematically shown in figure (\ref{SQKD_PROTOCOL_rounds_sooryansh}).

%-------------------------------------------------------
\subsection{KGR}
From the above discussion, it is clear that a bit for the key is generated whenever Bob performs a measurement in the basis $B_0$, which happens with a probability of $1/2$. The probability for Alice to choose any one of the bases set is also equal and is $1/3$. Therefore, the KGR ($r_{{\rm bSQKD}_4}$) for this protocol is: 
\begin{align}
    r_{{\rm bSQKD}_4}&= \frac{1}{3}\times\frac{1}{2}\times 2 +\frac{1}{3}\times\frac{1}{2}\times 1+\frac{1}{3}\times\frac{1}{2}\times 1=\frac{2}{3}=0.66 ~\rm{bits}.
\end{align}
In the protocol, we have assumed that probability of Bob's two operations, {\it viz.}, reflect and measure, is the same. However, this can be tweaked such that Bob measure with probability $q>\frac{1}{2}$ which increases the KGR to  $r_{{\rm bSQKD}_4}=\frac{4}{3}q$.

We now move on to present the robustness of the protocol against various eavesdropping strategies.

%===========================================================================


\subsection{Robustness of the protocol}
\label{Robustness_SQKD}
In this section, we discuss the robustness of bSQKD$_4$ protocol.  Like other SQKD protocols, bSQKD$_4$ is also a two-way communication protocol. Exploiting this, Eve may eavesdrop either in both ways or only in one way. The security against a one-way eavesdropping attack follows directly from the security of the bQKD protocol. So, we will discuss the robustness of the protocol against two-way entangling attacks. However, for the sake of completeness, we explicitly discuss the security of the protocol against intercept-resend attacks. 
\subsubsection{Intercept-resend attack}
Since in SQKD, Bob can only measure in the computational basis, it is beneficial for Eve to measure in the computational basis. However, if she does so in each round, her presence will be detected as in some of the rounds, Bob  does not perform any measurements. In such rounds, Eve's measurement will introduce errors since the process is random. Thus, whenever Alice sends states from basis $B_1$ and $B_2$, her actions will be detected with a probability of $0.5$. Thus, in $l$ such rounds, the actions of Eve will be detected with a probability $(1-0.5^l)$ which approaches unity for a sufficiently large $l$.

\subsubsection{Two-way entangling attack}
In such an attack, Eve may try to interact with traveling states with her ancillae. Let $U_F$ and $U_B$ be the two transformations that Eve employs to interact with states traveling from Alice to Bob and Bob to Alice respectively. The actions of these transformations can be expressed as:
 \begin{align}
     U_F\ket{i}_T\ket{0}\to \sum_{j=0}^3 \ket{j}_T\ket{E_{ij}},~ U_B\ket{j}_T\ket{0}\to \sum_{k=0}^3\ket{k}_T\ket{F_{jk}}
 \end{align}
 In the protocol, since Bob is classical, he either measures in the basis $B_0$ or he simply sends the received states to Alice. We discuss the two cases one-by-one.
 \subsubsection*{A. Bob performs a measurement}
 \noindent{\bf (I) Alice sends states from the basis $B_0$:}\\
 
 \noindent Whenever Alice sends a state say $\ket{i}$ from  the basis $B_0$, Bob gets the correct with the probability, $\norm{\ket{E_{ii}}}^2$. Given Bob gets the state sent by Alice, Alice gets the same state with a probability $\norm{\ket{F_{ii}}}^2$. So, both Alice and Bob get the same state as initially sent by Alice with a probability $p=\norm{\ket{E_{ii}}}^2\norm{\ket{F_{ii}}}^2$. Thus, Eve's interventions are detected  with a probability $(1-p)$ when Alice and Bob compare a subset of their outcomes.\\
 
 
 \noindent{\bf (II) Alice sends states from the basis $B_1$:}\\
 
\noindent The effect of Eve's interaction on the states of the basis $B_1$ traveling from Alice to Bob can be described as follows: 
\begin{align}
    &  \frac{1}{\sqrt{2}}\big(\ket{0}\pm\ket{1}\big)\ket{0}\xrightarrow[]{U_F}\frac{1}{\sqrt{2}}\sum_{j=0}^3\ket{j}\Big(\ket{E_{0j}}\pm \ket{E_{1j}}\Big)\nonumber\\
      &\frac{1}{\sqrt{2}}\big(\ket{2}\pm\ket{3}\big)\ket{0}\xrightarrow[]{U_F}\frac{1}{\sqrt{2}}\sum_{j=0}^3\ket{j}\Big(\ket{E_{2j}}\pm \ket{E_{3j}}\Big).
\end{align}
So, if Alice sends states $\frac{1}{\sqrt{2}}(\ket{0}\pm\ket{1})$, Bob gets wrong result with a probability $\frac{1}{2}\big(\norm{\ket{E_{02}}\pm\ket{E_{12}}}^2+\norm{\ket{E_{03}}\pm\ket{E_{13}}}^2 \big)$ reflecting Eve's presence. Similarly, if Alice sends states  $\frac{1}{\sqrt{2}}(\ket{2}\pm\ket{3})$, Bob gets the wrong result revealing Eve's presence with a probability $\frac{1}{2}\big(\norm{\ket{E_{20}}\pm\ket{E_{30}}}^2+\norm{\ket{E_{21}}\pm\ket{E_{31}}}^2 \big)$.\\

\noindent{\bf (III) Alice sends states from the basis $B_2$:}\\

\noindent The action of Eve's interventions on the states from the basis $B_2$ as follows:
\begin{align}
     &\frac{1}{\sqrt{2}}\big(\ket{0}\pm\ket{3}\big)\ket{0}   \xrightarrow[]{U_F}\frac{1}{\sqrt{2}}\sum_{j=0}^3\ket{j}\Big(\ket{E_{0j}}\pm\ket{E_{3j}}\Big)\nonumber\\
&\frac{1}{\sqrt{2}}\big(\ket{1}\pm\ket{2}\big)\ket{0}\xrightarrow[]{U_F}\frac{1}{\sqrt{2}}\sum_{j=0}^3\ket{j}\Big(\ket{E_{1j}}\pm \ket{E_{2j}}\Big).
\end{align}
So, whenever Alice sends the states $\frac{1}{\sqrt{2}}(\ket{0}\pm\ket{3})$, Bob gets wrong results with a probability $\frac{1}{2}\big(\norm{\ket{E_{01}}\pm\ket{E_{31}}}^2+\norm{\ket{E_{02}}\pm\ket{E_{32}}}^2 \big)$. Similarly, when Alice sends states $\frac{1}{\sqrt{2}}(\ket{1}\pm\ket{2})$, Bob gets wrong result with a probability $\frac{1}{2}\big(\norm{\ket{E_{10}}\pm\ket{E_{20}}}^2+\norm{\ket{E_{13}}\pm\ket{E_{23}}}^2 \big)$. When Alice and Bob compare a subset of rounds, interventions of Eve get reflected.

Additionally, Alice can also detect the presence of Eve by comparing the states (I) that she had sent, and (ii) the post-measurement state after her measurement. 

\subsubsection*{B. Bob performs reflect operation}

In the absence of an eavesdropper, the rounds in which Bob performs the reflect  operation, Alice would get the same post-measurement state as she had sent. However, due to Eve's interventions, this will not be the case, reflecting her presence and thus making these protocols robust against eavesdropping attacks.


Since Bob performs reflect operation, the action of Eve's attack in such case can be expressed as:
\begin{align}
    &U_BU_F\ket{i}\ket{0}_E\ket{0}_B\rightarrow U_B\sum_{j=0}^3\ket{j}\ket{E_{ij}}\ket{0}_B\rightarrow \sum_{j,k=0}^3 \ket{k}\ket{E_{ij}}\ket{F_{jk}}
\end{align}
As is clear from the above equation, whenever Alice sends a state from the basis $B_0$, she  gets the correct result with probability $p=\norm{\sum_{j=0}^3\ket{E_{ij}}\ket{F_{ji}}}^2$. Thus, Eve's interventions get detected with a probability $(1-p)$.
% A similar analysis holds when Alice sends state from bases $B_1$ or $B_2$ and Bob performs reflect operation.

\noindent{\textbf{Alice sends state from the basis $B_1$:}}
Suppose that Alice sends the state $\frac{1}{\sqrt{2}}(\ket{0}+\ket{1})$. Since Bob performs a reflect operation, the effect of Eve's interactions on the state can be expressed as: 
\begin{align}
        U_BU_F\frac{1}{\sqrt{2}}\Big(\ket{0}+\ket{1}\Big)\ket{0}\ket{0}
    &=\frac{1}{\sqrt{2}}\Big(\sum_{i,j=0}^3\ket{j}\big(\ket{E_{0i}}+\ket{E_{1i}}\big)\ket{F_{ij}}\Big)
\end{align}
Quite clearly, this state is nonorthogonal to the states $\frac{1}{\sqrt{2}}(\ket{0}-\ket{1})$ and $\frac{1}{\sqrt{2}}(\ket{2}\pm\ket{3})$, which implies that Alice's outcome will not always correspond to the initial state that she had sent. This signals Eve's presence. 

\noindent{\textbf{Alice sends state from the basis $B_2$:}}
Suppose that Alice sends the state $\frac{1}{\sqrt{2}}(\ket{0}\pm\ket{3})$ belonging to the basis $B_2$, which undergoes unitary transformations $U_F$ and $U_B$ in the forward and backward paths respectively as follows:
\begin{align}
    U_BU_F\frac{1}{\sqrt{2}}\Big(\ket{0}+\ket{3}\Big)\ket{0}\ket{0}
    &=\frac{1}{\sqrt{2}}\Big(\sum_{i,j=0}^3\ket{j}\big(\ket{E_{0i}}+\ket{E_{3i}}\big)\ket{F_{ij}}\Big)
\end{align}
Quite clearly, this state is nonorthogonal to the states $\frac{1}{\sqrt{2}}\big(\ket{0}-\ket{3}\big)$ and $\frac{1}{\sqrt{2}}\big(\ket{1}\pm\ket{2}\big)$, which implies that Alice's outcome will not always correspond to the initial state that she had sent. This signals Eve's presence.

This concludes our discussion on the robustness of bSQKD$_4$ protocol against eavesdropping.



The generalization of bSQKD protocols to effective qubits encoded in high-dimensional systems is straightforward. The steps of the protocols are essentially the same as of bSQKD$_4$. The three bases are  the same as employed in bQKD$_d$ protocol (given in section (\ref{generalisation})). 


We now briefly discuss the experimental setup for the implementation of bQKD$_4$ and bSQKD$_4$ protocol. \\


%-----------------------------------
 \noindent {\it Experimental implementation:}\\
 
 \noindent The bQKD$_4$ and bSQKD$_4$ protocols (proposed in sections (\ref{QKD_four}) and (\ref{boosted SQKD})) may be implemented with OAM modes of light. For this, the states belonging to the computational basis may be identified with  the Laguerre-Gauss modes of light. These states can be generated by passing Gaussian modes of light through appropriate phase masks.  The states belonging to other two basis involve coherent superpositions of two Laguerre Gauss modes respectively which can be realized with spatial light modulators \cite{PhysRevApplied.11.064058}. One can also employ a digital micro-mirror device as has been done in \cite{mirhosseini2015high} for preparing the two bases. In fact, since the bQKD and bSQKD protocols do not employ entangled states, they may be implemented with faint coherent pulses. To prepare faint coherent pulses, an attenuator can be employed. In fact, two attenuators with different degrees of attenuation  can be employed to make the protocols resilient against photon-number-splitting (PNS) attacks (discussed in detail in section (\ref{PNS})). Measurements of Laguerre Gauss modes (and superposition thereof) can be realized by employing a spatial light modulator (SLM) and detectors or with the techniques  presented in \cite{mirhosseini2013efficient}.
%----------------------------------------------------------------
\section{Conclusion}
\label{conclusion} 
%In this paper, we have employed effective qubits encoded in high-dimensional systems

 In this paper, we have proposed bQKD and bSQKD protocols, with higher KGRs by employing effective qubits encoded in higher dimensional systems only. The KGR of bQKD protocols turns out to be higher than the sum of KGRs of two QKD protocols running in parallel.  All the protocols can be realized with linear optics only. Though we have laid down the protocols for bQKD and bSQKD, the procedure is equally applicable to other secure quantum communication protocols, such as quantum dialogue and quantum key agreement, etc. Another direction in which our protocols can be generalized is by increasing the number of participants (which is chosen to be two in this paper, for the sake of simplicity). In fact, employing this strategy in a quantum network constitutes an interesting study. Furthermore, the protocols can also be generalized to those quantum networks, in which different parties are divided into different subsets constituting different {\it layers} \cite{pivoluska2018layered}. The protocols can be made robust against side-channel attacks by extending them to the measurement-device-independent setting. 

%======================================================================

 %===========================================================================================================================================================
 
 



\begin{appendices}
\end{appendices}

\section*{Acknowledgements}
Rajni and Sooryansh thank UGC and CSIR (File no.: 09/086(1278)/ 2017-EMR-I) for funding their research in the initial stages of the work.
\section*{Data availability statement}
Data sharing is not applicable to this article as no datasets were generated or analyzed during the current study.
\section*{Disclosures}
The authors declare no conflicts of interest.
%\section*{Author contribution statement}
%======================================================================================================
\appendix
%-----------------------------------------------------------------------------------------

%================================================================================================


%=================================================================================
\section{Key generation rule for the rounds in which Alice chooses basis ${\cal B}_1$ and Bob measures in basis ${\cal B}_2$}
In this appendix, we present tables (\ref{tab:B_1B_2_4}) and (\ref{tab:B_1B_2_7}). In these tables, we compactly show the key generation rule when Alice chooses a state from the basis ${\cal B}_1$ and Bob makes a measurement in the basis ${\cal B}_2$ in bQKD$_d$ protocol (presented in section (\ref{generalisation})). 
%-------------------------------------------------------For the case when d=4n and Alice sends state from the basis $B_1$ and Bob measures in the basis $B_2$-------
\begin{table}[h]
\begin{center}
 \resizebox{!}{!} {
\begin{tabular}{|c|c|c|} 
 \hline
 \multirow{3}{3cm}{{\bf Alice's state from the basis ${\cal B}_1$}}& \multirow{3}{3cm}{{\bf Post-measurement state of Bob (in the basis ${\cal B}_2$)}}&\multirow{3}{*}{\bf Key letter}\\
   && \\
 &&\\
     \hline\hline
 \multirow{4}{*}{ $\frac{1}{\sqrt{2}}(\ket{0}\pm\ket{1})$}&\multirow{2}{*}{$\frac{1}{\sqrt{2}}(\ket{d-1}\pm \ket{0})$}&\multirow{4}{*}{$0$}\\
&&\\
&\multirow{2}{*}{$\frac{1}{\sqrt{2}}(\ket{1}\pm \ket{2})$}&\\
&&\\
  \hline
  %--------------------------------------
\multirow{4}{*}{ $\frac{1}{\sqrt{2}}(\ket{2}\pm\ket{3})$}&\multirow{2}{*}{$\frac{1}{\sqrt{2}}(\ket{1}\pm \ket{2})$}&\multirow{4}{*}{$\perp$}\\
&&\\
&\multirow{2}{*}{$\frac{1}{\sqrt{2}}(\ket{3}\pm \ket{4})$}&\\
&&\\\hline
%------------------------------------------------------------------------
\multirow{4}{*}{$\frac{1}{\sqrt{2}}(\ket{4}\pm\ket{5})$}&\multirow{2}{*}{$\frac{1}{\sqrt{2}}(\ket{3}\pm \ket{4})$}&\multirow{4}{*}{$1$}\\
&&\\
&\multirow{2}{*}{$\frac{1}{\sqrt{2}}(\ket{5}\pm \ket{6})$}&\\
&&\\
  \hline
  \multirow{4}{*}{$\frac{1}{\sqrt{2}}(\ket{6}\pm\ket{7})$}&\multirow{2}{*}{$\frac{1}{\sqrt{2}}(\ket{5}\pm \ket{6})$}&\multirow{4}{*}{$\perp$}\\
&&\\
&\multirow{2}{*}{$\frac{1}{\sqrt{2}}(\ket{7}\pm \ket{8})$}&\\
&&\\
  \hline
   \multirow{4}{*}{$\vdots$}&\multirow{4}{*}{$\vdots$}&\multirow{4}{*}{$\vdots$}\\
&&\\
&&\\
&&\\
  \hline
     \multirow{4}{*}{$\frac{1}{\sqrt{2}}(\ket{d-4}\pm\ket{d-3})$}&\multirow{2}{*}{$\frac{1}{\sqrt{2}}(\ket{d-5}\pm \ket{d-4})$}&\multirow{4}{*}{$\frac{d}{4}-1$}\\
&&\\
&\multirow{2}{*}{$\frac{1}{\sqrt{2}}(\ket{d-3}\pm \ket{d-2})$}&\\
&&\\
  \hline
   \multirow{4}{*}{$\frac{1}{\sqrt{2}}(\ket{d-2}\pm\ket{d-1})$}&\multirow{2}{*}{$\frac{1}{\sqrt{2}}(\ket{d-3}\pm \ket{d-2})$}&\multirow{4}{*}{$\perp$}\\
&&\\
&\multirow{2}{*}{$\frac{1}{\sqrt{2}}(\ket{d-1}\pm \ket{0})$}&\\
&&\\
  \hline
\end{tabular}
}
\vspace{0.25cm}
\caption{Key generation rule in the rounds when Alice sends state from the basis ${\cal B}_1$ and Bob measures in the basis ${\cal B}_2$. The symbol $\perp$ corresponds to the rounds which are discarded. The rounds in which $d$ is not multiple of $4$ but of $2$, Bob also has to discard the rounds in which he receives the outcome corresponding to the post-measurement states $\frac{1}{\sqrt{2}}\big(\ket{d-1}\pm\ket{0}\big)$. This key generation rule is motivated by what is employed in B$92$ protocol \cite{PhysRevLett.68.3121} for unambiguous state discrimination. }
    \label{tab:B_1B_2_4}
    \end{center}
\end{table}

\begin{table}[h]
\begin{center}
 \resizebox{!}{!} {
\begin{tabular}{|c|c|c|} 
 \hline
 \multirow{3}{3cm}{{\bf Alice's state from the basis ${\cal B}_1$}}& \multirow{3}{3cm}{{\bf Post-measurement state of Bob (in the basis ${\cal B}_2$)}}&\multirow{3}{*}{\bf Key letter}\\
   && \\
 &&\\
     \hline\hline
 \multirow{4}{*}{ $\frac{1}{\sqrt{2}}\Big(\ket{0}\pm\ket{1}\Big)$}&\multirow{2}{*}{$\frac{1}{\sqrt{2}}\Big(\ket{d-1}\pm \ket{0}\Big)$}&\multirow{2}{*}{$\perp$}\\
&&\\
&\multirow{2}{*}{$\frac{1}{\sqrt{2}}\Big(\ket{1}\pm \ket{2}\Big)$}&\multirow{2}{*}{$0$}\\
&&\\
  \hline
  %--------------------------------------
\multirow{4}{*}{ $\frac{1}{\sqrt{2}}\Big(\ket{2}\pm\ket{3}\Big)$}&\multirow{2}{*}{$\frac{1}{\sqrt{2}}\Big(\ket{1}\pm \ket{2}\Big)$}&\multirow{4}{*}{$\perp$}\\
&&\\
&\multirow{2}{*}{$\frac{1}{\sqrt{2}}\Big(\ket{3}\pm \ket{4}\Big)$}&\\
&&\\\hline
%------------------------------------------------------------------------
\multirow{4}{*}{$\frac{1}{\sqrt{2}}\Big(\ket{4}\pm\ket{5}$}&\multirow{2}{*}{$\frac{1}{\sqrt{2}}\Big(\ket{3}\pm \ket{4}\Big)$}&\multirow{4}{*}{$1$}\\
&&\\
&\multirow{2}{*}{$\frac{1}{\sqrt{2}}\Big(\ket{5}\pm \ket{6}\Big)$}&\\
&&\\
  \hline
  \multirow{4}{*}{$\frac{1}{\sqrt{2}}\Big(\ket{6}\pm\ket{7}\Big)$}&\multirow{2}{*}{$\frac{1}{\sqrt{2}}\Big(\ket{5}\pm \ket{6}\Big)$}&\multirow{4}{*}{$\perp$}\\
&&\\
&\multirow{2}{*}{$\frac{1}{\sqrt{2}}\Big(\ket{7}\pm \ket{8}\Big)$}&\\
&&\\
  \hline
   \multirow{4}{*}{$\vdots$}&\multirow{4}{*}{$\vdots$}&\multirow{4}{*}{$\vdots$}\\
&&\\
&&\\
&&\\
  \hline
     \multirow{4}{*}{$\frac{1}{\sqrt{2}}\Big(\ket{d-4}\pm\ket{d-3}\Big)$}&\multirow{2}{*}{$\frac{1}{\sqrt{2}}\Big(\ket{d-5}\pm \ket{d-4}\Big)$}&\multirow{4}{*}{$\perp$}\\
&&\\
&\multirow{2}{*}{$\frac{1}{\sqrt{2}}\Big(\ket{d-3}\pm \ket{d-2}\Big)$}&\\
&&\\
  \hline
   \multirow{4}{*}{$\frac{1}{\sqrt{2}}\Big(\ket{d-2}\pm\ket{d-1}\Big)$}&\multirow{2}{*}{$\frac{1}{\sqrt{2}}\Big(\ket{d-3}\pm \ket{d-2}\Big)$}&\multirow{2}{*}{$\frac{d}{4}-1$}\\
&&\\
&\multirow{2}{*}{$\frac{1}{\sqrt{2}}\Big(\ket{d-1}\pm \ket{0}\Big)$}&\multirow{2}{*}{$\perp$}\\
&&\\
  \hline
\end{tabular}
}
\vspace{0.25cm}
\caption{Key generation rule in the rounds when Alice sends state from the basis ${\cal B}_1$ and Bob measures in the basis ${\cal B}_2$. The symbol $\perp$ corresponds to the rounds which are discarded.  Key letters $0$ and $\frac{d}{4}-1$ are generated with probabilities $\frac{1}{2}$ in contrast to other key letters. }
    \label{tab:B_1B_2_7}
    \end{center}
\end{table}


%-------------------------------------------------------------------------------------------------------
\section{Semi-quantum key distribution (SQKD): a brief recap}
\label{SQKD_Boyer}
In this appendix, we briefly present the SQKD protocol proposed in \cite{Boyer07} (designated as SQKD07), to make the paper self-contained. The SQKD07 protocol  securely distributes a  key between a quantum participant (QP) ({\it viz.}, Alice) and a classical participant (CP) ({\it viz.}, Bob). A QP can prepare any state and perform measurements in any basis whereas a CP can prepare and measure only in the computational basis. Since the SQKD protocol has only one QP, it eases down the experimental implementation of the SQKD07 protocol \cite{massa2022experimental}. For reference, the steps of the SQKD07 protocol are indicated in the figure (\ref{SQKD_figure_1}) and explicitly enumerated as follows:
\begin{enumerate}
    \item Alice randomly prepares one of the states from the two bases $B_0\equiv \{\ket{0}, \ket{1}\}, B_1\equiv  \{\ket{+}, \ket{-}\}$ with  equal probability and sends it to Bob. 
    \item Bob, upon receiving the state, exercises one of the two options with equal probabilities: (i) he measures the incoming state in the computational basis. He prepares the same state as the post-measurement state afresh and sends it back to Alice, (ii) He sends the incoming state back to Alice.  
    \item Alice measures each incoming state in the same basis in which she has prepared it. This constitutes one round.
    \item After a sufficient number of rounds, Bob reveals the rounds in which he has performed measurements  and Alice reveals the rounds in which she has sent the states in the computational basis, i.e., $\{\ket{0}, \ket{1}\}$. 
    \item Alice analyses the data of those rounds in which Bob has not performed measurements to check for the presence of an eavesdropper.
    \item The outcomes of those rounds, in which Bob has performed measurements and Alice has measured in the computational basis, constitute a key.
\end{enumerate}
In this way, a key is shared between Alice and Bob. The protocol has been schematically shown in figure (\ref{SQKD_figure_1}).

\begin{figure}[h!]
\centering
\includegraphics[width=0.8\textwidth]{SQKD}
\caption{Pictorial representation of SQKD07 protocol.  The inputs $x$ and $a$ at Alice's end represent choices of bases and states respectively. $x=0, 1$ represent the bases $B_0$ and $B_1$ respectively. The input $y$ and the output $b$ at Bob's end represent choices of measurement bases and outcome respectively. }
\label{SQKD_figure_1}
\end{figure}

%=============================================================================================
%\bibliography{bibliography}
%% BioMed_Central_Bib_Style_v1.01

\begin{thebibliography}{39}
% BibTex style file: bmc-mathphys.bst (version 2.1), 2014-07-24
\ifx \bisbn   \undefined \def \bisbn  #1{ISBN #1}\fi
\ifx \binits  \undefined \def \binits#1{#1}\fi
\ifx \bauthor  \undefined \def \bauthor#1{#1}\fi
\ifx \batitle  \undefined \def \batitle#1{#1}\fi
\ifx \bjtitle  \undefined \def \bjtitle#1{#1}\fi
\ifx \bvolume  \undefined \def \bvolume#1{\textbf{#1}}\fi
\ifx \byear  \undefined \def \byear#1{#1}\fi
\ifx \bissue  \undefined \def \bissue#1{#1}\fi
\ifx \bfpage  \undefined \def \bfpage#1{#1}\fi
\ifx \blpage  \undefined \def \blpage #1{#1}\fi
\ifx \burl  \undefined \def \burl#1{\textsf{#1}}\fi
\ifx \doiurl  \undefined \def \doiurl#1{\url{https://doi.org/#1}}\fi
\ifx \betal  \undefined \def \betal{\textit{et al.}}\fi
\ifx \binstitute  \undefined \def \binstitute#1{#1}\fi
\ifx \binstitutionaled  \undefined \def \binstitutionaled#1{#1}\fi
\ifx \bctitle  \undefined \def \bctitle#1{#1}\fi
\ifx \beditor  \undefined \def \beditor#1{#1}\fi
\ifx \bpublisher  \undefined \def \bpublisher#1{#1}\fi
\ifx \bbtitle  \undefined \def \bbtitle#1{#1}\fi
\ifx \bedition  \undefined \def \bedition#1{#1}\fi
\ifx \bseriesno  \undefined \def \bseriesno#1{#1}\fi
\ifx \blocation  \undefined \def \blocation#1{#1}\fi
\ifx \bsertitle  \undefined \def \bsertitle#1{#1}\fi
\ifx \bsnm \undefined \def \bsnm#1{#1}\fi
\ifx \bsuffix \undefined \def \bsuffix#1{#1}\fi
\ifx \bparticle \undefined \def \bparticle#1{#1}\fi
\ifx \barticle \undefined \def \barticle#1{#1}\fi
\bibcommenthead
\ifx \bconfdate \undefined \def \bconfdate #1{#1}\fi
\ifx \botherref \undefined \def \botherref #1{#1}\fi
\ifx \url \undefined \def \url#1{\textsf{#1}}\fi
\ifx \bchapter \undefined \def \bchapter#1{#1}\fi
\ifx \bbook \undefined \def \bbook#1{#1}\fi
\ifx \bcomment \undefined \def \bcomment#1{#1}\fi
\ifx \oauthor \undefined \def \oauthor#1{#1}\fi
\ifx \citeauthoryear \undefined \def \citeauthoryear#1{#1}\fi
\ifx \endbibitem  \undefined \def \endbibitem {}\fi
\ifx \bconflocation  \undefined \def \bconflocation#1{#1}\fi
\ifx \arxivurl  \undefined \def \arxivurl#1{\textsf{#1}}\fi
\csname PreBibitemsHook\endcsname

%%% 1
\bibitem{Bennett84}
\begin{botherref}
\oauthor{\bsnm{Bennett}, \binits{C.}},
\oauthor{\bsnm{Brassard}, \binits{G.}}:
Quantum cryptography: public key distribution and coin tossing.
Proc. IEEE Int. Conf. on Comp. Sys. Signal Process (ICCSSP),
175
(1984).
\end{botherref}
\endbibitem

%%% 2
\bibitem{gisin2002quantum}
\begin{barticle}
\bauthor{\bsnm{Gisin}, \binits{N.}},
\bauthor{\bsnm{Ribordy}, \binits{G.}},
\bauthor{\bsnm{Tittel}, \binits{W.}},
\bauthor{\bsnm{Zbinden}, \binits{H.}}:
\batitle{Quantum cryptography}.
\bjtitle{Reviews of modern physics}
\bvolume{74}(\bissue{1}),
\bfpage{145}
(\byear{2002})
\end{barticle}
\endbibitem

%%% 3
\bibitem{pirandola2020advances}
\begin{barticle}
\bauthor{\bsnm{Pirandola}, \binits{S.}},
\bauthor{\bsnm{Andersen}, \binits{U.L.}},
\bauthor{\bsnm{Banchi}, \binits{L.}},
\bauthor{\bsnm{Berta}, \binits{M.}},
\bauthor{\bsnm{Bunandar}, \binits{D.}},
\bauthor{\bsnm{Colbeck}, \binits{R.}},
\bauthor{\bsnm{Englund}, \binits{D.}},
\bauthor{\bsnm{Gehring}, \binits{T.}},
\bauthor{\bsnm{Lupo}, \binits{C.}},
\bauthor{\bsnm{Ottaviani}, \binits{C.}}, \betal:
\batitle{Advances in quantum cryptography}.
\bjtitle{Advances in optics and photonics}
\bvolume{12}(\bissue{4}),
\bfpage{1012}--\blpage{1236}
(\byear{2020})
\end{barticle}
\endbibitem

%%% 4
\bibitem{chau2015quantum}
\begin{barticle}
\bauthor{\bsnm{Chau}, \binits{H.}}:
\batitle{Quantum key distribution using qudits that each encode one bit of raw
  key}.
\bjtitle{Physical Review A}
\bvolume{92}(\bissue{6}),
\bfpage{062324}
(\byear{2015})
\end{barticle}
\endbibitem

%%% 5
\bibitem{Ekert91}
\begin{barticle}
\bauthor{\bsnm{Ekert}, \binits{A.K.}}:
\batitle{Quantum cryptography based on bell's theorem}.
\bjtitle{Phys. Rev. Lett.}
\bvolume{67},
\bfpage{661}--\blpage{663}
(\byear{1991}).
\end{barticle}
\endbibitem

%%% 6
\bibitem{Bennett92}
\begin{barticle}
\bauthor{\bsnm{Bennett}, \binits{C.H.}},
\bauthor{\bsnm{Wiesner}, \binits{S.J.}}:
\batitle{Communication via one- and two-particle operators on
  einstein-podolsky-rosen states}.
\bjtitle{Phys. Rev. Lett.}
\bvolume{69},
\bfpage{2881}--\blpage{2884}
(\byear{1992}).
\end{barticle}
\endbibitem

%%% 7
\bibitem{Bennett93}
\begin{barticle}
\bauthor{\bsnm{Bennett}, \binits{C.H.}},
\bauthor{\bsnm{Brassard}, \binits{G.}},
\bauthor{\bsnm{Cr{\'e}peau}, \binits{C.}},
\bauthor{\bsnm{Jozsa}, \binits{R.}},
\bauthor{\bsnm{Peres}, \binits{A.}},
\bauthor{\bsnm{Wootters}, \binits{W.K.}}:
\batitle{Teleporting an unknown quantum state via dual classical and
  einstein-podolsky-rosen channels}.
\bjtitle{Physical review letters}
\bvolume{70}(\bissue{13}),
\bfpage{1895}
(\byear{1993})
\end{barticle}
\endbibitem

%%% 8
\bibitem{bala2021contextuality}
\begin{barticle}
\bauthor{\bsnm{Bala}, \binits{R.}},
\bauthor{\bsnm{Asthana}, \binits{S.}},
\bauthor{\bsnm{Ravishankar}, \binits{V.}}:
\batitle{Contextuality-based quantum conferencing}.
\bjtitle{Quantum Information Processing}
\bvolume{20}(\bissue{10}),
\bfpage{1}--\blpage{27}
(\byear{2021})
\end{barticle}
\endbibitem

%%% 9
\bibitem{bradler2016finite}
\begin{barticle}
\bauthor{\bsnm{Br{\'a}dler}, \binits{K.}},
\bauthor{\bsnm{Mirhosseini}, \binits{M.}},
\bauthor{\bsnm{Fickler}, \binits{R.}},
\bauthor{\bsnm{Broadbent}, \binits{A.}},
\bauthor{\bsnm{Boyd}, \binits{R.}}:
\batitle{Finite-key security analysis for multilevel quantum key distribution}.
\bjtitle{New Journal of Physics}
\bvolume{18}(\bissue{7}),
\bfpage{073030}
(\byear{2016})
\end{barticle}
\endbibitem

%%% 10
\bibitem{Inoue02}
\begin{barticle}
\bauthor{\bsnm{Inoue}, \binits{K.}},
\bauthor{\bsnm{Waks}, \binits{E.}},
\bauthor{\bsnm{Yamamoto}, \binits{Y.}}:
\batitle{Differential phase shift quantum key distribution}.
\bjtitle{Phys. Rev. Lett.}
\bvolume{89},
\bfpage{037902}
(\byear{2002}).
\end{barticle}
\endbibitem

%%% 11
\bibitem{EfficientQKD_qubit}
\begin{barticle}
\bauthor{\bsnm{Lo}, \binits{H.-K.}},
\bauthor{\bsnm{Chau}, \binits{H.F.}},
\bauthor{\bsnm{Ardehali}, \binits{M.}}:
\batitle{Efficient quantum key distribution scheme and a proof of its
  unconditional security}.
\bjtitle{Journal of Cryptology}
\bvolume{18}(\bissue{2}),
\bfpage{133}--\blpage{165}
(\byear{2005})
\end{barticle}
\endbibitem

%%% 12
\bibitem{wang2010efficient}
\begin{barticle}
\bauthor{\bsnm{Wang}, \binits{J.-D.}},
\bauthor{\bsnm{Wei}, \binits{Z.-J.}},
\bauthor{\bsnm{Zhang}, \binits{H.}},
\bauthor{\bsnm{Qin}, \binits{X.-J.}},
\bauthor{\bsnm{Liu}, \binits{X.-B.}},
\bauthor{\bsnm{Zhang}, \binits{Z.-M.}},
\bauthor{\bsnm{Liao}, \binits{C.-J.}},
\bauthor{\bsnm{Liu}, \binits{S.-H.}}:
\batitle{Efficient quantum key distribution via single-photon two-qubit
  states}.
\bjtitle{Journal of Physics B: Atomic, Molecular and Optical Physics}
\bvolume{43}(\bissue{9}),
\bfpage{095504}
(\byear{2010})
\end{barticle}
\endbibitem

%%% 13
\bibitem{cerf2002security}
\begin{barticle}
\bauthor{\bsnm{Cerf}, \binits{N.J.}},
\bauthor{\bsnm{Bourennane}, \binits{M.}},
\bauthor{\bsnm{Karlsson}, \binits{A.}},
\bauthor{\bsnm{Gisin}, \binits{N.}}:
\batitle{Security of quantum key distribution using d-level systems}.
\bjtitle{Physical review letters}
\bvolume{88}(\bissue{12}),
\bfpage{127902}
(\byear{2002})
\end{barticle}
\endbibitem

%%% 14
\bibitem{shu2022quantum}
\begin{barticle}
\bauthor{\bsnm{Shu}, \binits{H.}}:
\batitle{Quantum key distribution based on orthogonal state encoding}.
\bjtitle{International Journal of Theoretical Physics}
\bvolume{61}(\bissue{12}),
\bfpage{271}
(\byear{2022})
\end{barticle}
\endbibitem

%%% 15
\bibitem{chen2021efficient}
\begin{barticle}
\bauthor{\bsnm{Chen}, \binits{L.}},
\bauthor{\bsnm{Li}, \binits{Q.}},
\bauthor{\bsnm{Liu}, \binits{C.}},
\bauthor{\bsnm{Peng}, \binits{Y.}},
\bauthor{\bsnm{Yu}, \binits{F.}}:
\batitle{Efficient mediated semi-quantum key distribution}.
\bjtitle{Physica A: Statistical Mechanics and its Applications}
\bvolume{582},
\bfpage{126265}
(\byear{2021})
\end{barticle}
\endbibitem

%%% 16
\bibitem{pan2022semi}
\begin{barticle}
\bauthor{\bsnm{Pan}, \binits{X.}}:
\batitle{Semi-quantum key distribution protocol with logical qubits over the
  collective-rotation noise channel}.
\bjtitle{International Journal of Theoretical Physics}
\bvolume{61}(\bissue{3}),
\bfpage{77}
(\byear{2022})
\end{barticle}
\endbibitem

%%% 17
\bibitem{wang2019demand}
\begin{barticle}
\bauthor{\bsnm{Wang}, \binits{H.}},
\bauthor{\bsnm{Hu}, \binits{H.}},
\bauthor{\bsnm{Chung}, \binits{T.-H.}},
\bauthor{\bsnm{Qin}, \binits{J.}},
\bauthor{\bsnm{Yang}, \binits{X.}},
\bauthor{\bsnm{Li}, \binits{J.-P.}},
\bauthor{\bsnm{Liu}, \binits{R.-Z.}},
\bauthor{\bsnm{Zhong}, \binits{H.-S.}},
\bauthor{\bsnm{He}, \binits{Y.-M.}},
\bauthor{\bsnm{Ding}, \binits{X.}}, \betal:
\batitle{On-demand semiconductor source of entangled photons which
  simultaneously has high fidelity, efficiency, and indistinguishability}.
\bjtitle{Physical review letters}
\bvolume{122}(\bissue{11}),
\bfpage{113602}
(\byear{2019})
\end{barticle}
\endbibitem

%%% 18
\bibitem{iqbal2020semi}
\begin{barticle}
\bauthor{\bsnm{Iqbal}, \binits{H.}},
\bauthor{\bsnm{Krawec}, \binits{W.O.}}:
\batitle{Semi-quantum cryptography}.
\bjtitle{Quantum Information Processing}
\bvolume{19},
\bfpage{1}--\blpage{52}
(\byear{2020})
\end{barticle}
\endbibitem

%%% 19
\bibitem{boyer2009semiquantum}
\begin{barticle}
\bauthor{\bsnm{Boyer}, \binits{M.}},
\bauthor{\bsnm{Gelles}, \binits{R.}},
\bauthor{\bsnm{Kenigsberg}, \binits{D.}},
\bauthor{\bsnm{Mor}, \binits{T.}}:
\batitle{Semiquantum key distribution}.
\bjtitle{Physical Review A}
\bvolume{79}(\bissue{3}),
\bfpage{032341}
(\byear{2009})
\end{barticle}
\endbibitem

%%% 20
\bibitem{xie2018semi}
\begin{barticle}
\bauthor{\bsnm{Xie}, \binits{C.}},
\bauthor{\bsnm{Li}, \binits{L.}},
\bauthor{\bsnm{Situ}, \binits{H.}},
\bauthor{\bsnm{He}, \binits{J.}}:
\batitle{Semi-quantum secure direct communication scheme based on bell states}.
\bjtitle{International Journal of Theoretical Physics}
\bvolume{57}(\bissue{6}),
\bfpage{1881}--\blpage{1887}
(\byear{2018})
\end{barticle}
\endbibitem

%%% 21
\bibitem{ye2018semi}
\begin{barticle}
\bauthor{\bsnm{Ye}, \binits{T.-Y.}},
\bauthor{\bsnm{Ye}, \binits{C.-Q.}}:
\batitle{Semi-quantum dialogue based on single photons}.
\bjtitle{International Journal of Theoretical Physics}
\bvolume{57}(\bissue{5}),
\bfpage{1440}--\blpage{1454}
(\byear{2018})
\end{barticle}
\endbibitem

%%% 22
\bibitem{pan2020semi}
\begin{barticle}
\bauthor{\bsnm{Pan}, \binits{H.-M.}}:
\batitle{Semi-quantum dialogue with bell entangled states}.
\bjtitle{International Journal of Theoretical Physics}
\bvolume{59}(\bissue{5}),
\bfpage{1364}--\blpage{1371}
(\byear{2020})
\end{barticle}
\endbibitem

%%% 23
\bibitem{bala2022quantum}
\begin{botherref}
\oauthor{\bsnm{Bala}, \binits{R.}},
\oauthor{\bsnm{Asthana}, \binits{S.}},
\oauthor{\bsnm{Ravishankar}, \binits{V.}}:
Quantum and semi--quantum key distribution in networks.
arXiv preprint arXiv:2212.10464
(2022)
\end{botherref}
\endbibitem

%%% 24
\bibitem{bala2022semi}
\begin{bchapter}
\bauthor{\bsnm{Bala}, \binits{R.}},
\bauthor{\bsnm{Asthana}, \binits{S.}},
\bauthor{\bsnm{Ravishankar}, \binits{V.}}:
\bctitle{Semi-quantum key distribution in networks with oam states of light}.
In: \bbtitle{Frontiers in Optics},
pp. \bfpage{4}--\blpage{65}
(\byear{2022}).
\bcomment{Optica Publishing Group}
\end{bchapter}
\endbibitem

%%% 25
\bibitem{yan2019semi}
\begin{barticle}
\bauthor{\bsnm{Yan}, \binits{L.}},
\bauthor{\bsnm{Zhang}, \binits{S.}},
\bauthor{\bsnm{Chang}, \binits{Y.}},
\bauthor{\bsnm{Sheng}, \binits{Z.}},
\bauthor{\bsnm{Sun}, \binits{Y.}}:
\batitle{Semi-quantum key agreement and private comparison protocols using bell
  states}.
\bjtitle{International Journal of Theoretical Physics}
\bvolume{58},
\bfpage{3852}--\blpage{3862}
(\byear{2019})
\end{barticle}
\endbibitem

%%% 26
\bibitem{li2020new}
\begin{barticle}
\bauthor{\bsnm{Li}, \binits{H.-H.}},
\bauthor{\bsnm{Gong}, \binits{L.-H.}},
\bauthor{\bsnm{Zhou}, \binits{N.-R.}}:
\batitle{New semi-quantum key agreement protocol based on high-dimensional
  single-particle states}.
\bjtitle{Chinese Physics B}
\bvolume{29}(\bissue{11}),
\bfpage{110304}
(\byear{2020})
\end{barticle}
\endbibitem

%%% 27
\bibitem{goel2022inverse}
\begin{botherref}
\oauthor{\bsnm{Goel}, \binits{S.}},
\oauthor{\bsnm{Leedumrongwatthanakun}, \binits{S.}},
\oauthor{\bsnm{Valencia}, \binits{N.H.}},
\oauthor{\bsnm{McCutcheon}, \binits{W.}},
\oauthor{\bsnm{Conti}, \binits{C.}},
\oauthor{\bsnm{Pinkse}, \binits{P.W.}},
\oauthor{\bsnm{Malik}, \binits{M.}}:
Inverse-design of high-dimensional quantum optical circuits in a complex
  medium.
arXiv preprint arXiv:2204.00578
(2022)
\end{botherref}
\endbibitem

%%% 28
\bibitem{ding2017high}
\begin{barticle}
\bauthor{\bsnm{Ding}, \binits{Y.}},
\bauthor{\bsnm{Bacco}, \binits{D.}},
\bauthor{\bsnm{Dalgaard}, \binits{K.}},
\bauthor{\bsnm{Cai}, \binits{X.}},
\bauthor{\bsnm{Zhou}, \binits{X.}},
\bauthor{\bsnm{Rottwitt}, \binits{K.}},
\bauthor{\bsnm{Oxenl{\o}we}, \binits{L.K.}}:
\batitle{High-dimensional quantum key distribution based on multicore fiber
  using silicon photonic integrated circuits}.
\bjtitle{npj Quantum Information}
\bvolume{3}(\bissue{1}),
\bfpage{25}
(\byear{2017})
\end{barticle}
\endbibitem

%%% 29
\bibitem{PhysRevA.92.062324}
\begin{barticle}
\bauthor{\bsnm{Chau}, \binits{H.F.}}:
\batitle{Quantum key distribution using qudits that each encode one bit of raw
  key}.
\bjtitle{Phys. Rev. A}
\bvolume{92},
\bfpage{062324}
(\byear{2015}).
\end{barticle}
\endbibitem

%%% 30
\bibitem{ardehali1998efficient}
\begin{botherref}
\oauthor{\bsnm{Ardehali}, \binits{M.}},
\oauthor{\bsnm{Chau}, \binits{H.}},
\oauthor{\bsnm{Lo}, \binits{H.-K.}}:
Efficient quantum key distribution.
arXiv preprint quant-ph/9803007
(1998)
\end{botherref}
\endbibitem

%%% 31
\bibitem{acin2004coherent}
\begin{barticle}
\bauthor{\bsnm{Acin}, \binits{A.}},
\bauthor{\bsnm{Gisin}, \binits{N.}},
\bauthor{\bsnm{Scarani}, \binits{V.}}:
\batitle{Coherent-pulse implementations of quantum cryptography protocols
  resistant to photon-number-splitting attacks}.
\bjtitle{Physical Review A}
\bvolume{69}(\bissue{1}),
\bfpage{012309}
(\byear{2004})
\end{barticle}
\endbibitem

%%% 32
\bibitem{gisin2006trojan}
\begin{barticle}
\bauthor{\bsnm{Gisin}, \binits{N.}},
\bauthor{\bsnm{Fasel}, \binits{S.}},
\bauthor{\bsnm{Kraus}, \binits{B.}},
\bauthor{\bsnm{Zbinden}, \binits{H.}},
\bauthor{\bsnm{Ribordy}, \binits{G.}}:
\batitle{Trojan-horse attacks on quantum-key-distribution systems}.
\bjtitle{Physical Review A}
\bvolume{73}(\bissue{2}),
\bfpage{022320}
(\byear{2006})
\end{barticle}
\endbibitem

%%% 33
\bibitem{Boyer07}
\begin{barticle}
\bauthor{\bsnm{Boyer}, \binits{M.}},
\bauthor{\bsnm{Kenigsberg}, \binits{D.}},
\bauthor{\bsnm{Mor}, \binits{T.}}:
\batitle{Quantum key distribution with classical bob}.
\bjtitle{Phys. Rev. Lett.}
\bvolume{99},
\bfpage{140501}
(\byear{2007}).
\end{barticle}
\endbibitem

%%% 34
\bibitem{PhysRevApplied.11.064058}
\begin{barticle}
\bauthor{\bsnm{Cozzolino}, \binits{D.}},
\bauthor{\bsnm{Bacco}, \binits{D.}},
\bauthor{\bsnm{Da~Lio}, \binits{B.}},
\bauthor{\bsnm{Ingerslev}, \binits{K.}},
\bauthor{\bsnm{Ding}, \binits{Y.}},
\bauthor{\bsnm{Dalgaard}, \binits{K.}},
\bauthor{\bsnm{Kristensen}, \binits{P.}},
\bauthor{\bsnm{Galili}, \binits{M.}},
\bauthor{\bsnm{Rottwitt}, \binits{K.}},
\bauthor{\bsnm{Ramachandran}, \binits{S.}},
\bauthor{\bsnm{Oxenl\o{}we}, \binits{L.K.}}:
\batitle{Orbital angular momentum states enabling fiber-based high-dimensional
  quantum communication}.
\bjtitle{Phys. Rev. Applied}
\bvolume{11},
\bfpage{064058}
(\byear{2019}).
\end{barticle}
\endbibitem

%%% 35
\bibitem{mirhosseini2015high}
\begin{barticle}
\bauthor{\bsnm{Mirhosseini}, \binits{M.}},
\bauthor{\bsnm{Maga{\~n}a-Loaiza}, \binits{O.S.}},
\bauthor{\bsnm{O’Sullivan}, \binits{M.N.}},
\bauthor{\bsnm{Rodenburg}, \binits{B.}},
\bauthor{\bsnm{Malik}, \binits{M.}},
\bauthor{\bsnm{Lavery}, \binits{M.P.}},
\bauthor{\bsnm{Padgett}, \binits{M.J.}},
\bauthor{\bsnm{Gauthier}, \binits{D.J.}},
\bauthor{\bsnm{Boyd}, \binits{R.W.}}:
\batitle{High-dimensional quantum cryptography with twisted light}.
\bjtitle{New Journal of Physics}
\bvolume{17}(\bissue{3}),
\bfpage{033033}
(\byear{2015})
\end{barticle}
\endbibitem

%%% 36
\bibitem{mirhosseini2013efficient}
\begin{barticle}
\bauthor{\bsnm{Mirhosseini}, \binits{M.}},
\bauthor{\bsnm{Malik}, \binits{M.}},
\bauthor{\bsnm{Shi}, \binits{Z.}},
\bauthor{\bsnm{Boyd}, \binits{R.W.}}:
\batitle{Efficient separation of the orbital angular momentum eigenstates of
  light}.
\bjtitle{Nature communications}
\bvolume{4}(\bissue{1}),
\bfpage{1}--\blpage{6}
(\byear{2013})
\end{barticle}
\endbibitem

%%% 37
\bibitem{pivoluska2018layered}
\begin{barticle}
\bauthor{\bsnm{Pivoluska}, \binits{M.}},
\bauthor{\bsnm{Huber}, \binits{M.}},
\bauthor{\bsnm{Malik}, \binits{M.}}:
\batitle{Layered quantum key distribution}.
\bjtitle{Physical Review A}
\bvolume{97}(\bissue{3}),
\bfpage{032312}
(\byear{2018})
\end{barticle}
\endbibitem

%%% 38
\bibitem{PhysRevLett.68.3121}
\begin{barticle}
\bauthor{\bsnm{Bennett}, \binits{C.H.}}:
\batitle{Quantum cryptography using any two nonorthogonal states}.
\bjtitle{Phys. Rev. Lett.}
\bvolume{68},
\bfpage{3121}--\blpage{3124}
(\byear{1992}).
\end{barticle}
\endbibitem

%%% 39
\bibitem{massa2022experimental}
\begin{barticle}
\bauthor{\bsnm{Massa}, \binits{F.}},
\bauthor{\bsnm{Yadav}, \binits{P.}},
\bauthor{\bsnm{Moqanaki}, \binits{A.}},
\bauthor{\bsnm{Krawec}, \binits{W.O.}},
\bauthor{\bsnm{Mateus}, \binits{P.}},
\bauthor{\bsnm{Paunkovi{\'c}}, \binits{N.}},
\bauthor{\bsnm{Souto}, \binits{A.}},
\bauthor{\bsnm{Walther}, \binits{P.}}:
\batitle{Experimental semi-quantum key distribution with classical users}.
\bjtitle{Quantum}
\bvolume{6},
\bfpage{819}
(\byear{2022})
\end{barticle}
\endbibitem

\end{thebibliography}

\end{document}
