\documentclass[letterpaper,12pt,leqno]{article}
\usepackage{paper,math}
\pdfoutput=1
\newcommand{\bib}{paper.bib}
\newcommand{\pdf}{figures.pdf}
\hypersetup{pdftitle={Why Are Immigrants Always Accused of Stealing People's Jobs?}}
\available{https://pascalmichaillat.org/14/}
\begin{document}

\title{Why Are Immigrants Always Accused of Stealing People's Jobs?}
\author{Pascal Michaillat
\thanks{Pascal Michaillat: Brown University. I thank Varanya Chaubey, Ken Lee, Gary Richardson, Emmanuel Saez, and Finn Schuele for helpful comments and discussions.}}
\date{March 2023}
\begin{titlepage}\maketitle

Immigrants are always accused of stealing people's jobs. Yet, in a neoclassical model of the labor market, there are jobs for everybody and no jobs to steal. (There is no unemployment, so anybody who wants to work can work.) In standard matching models, there is some unemployment, but labor demand is perfectly elastic so new entrants into the labor force are absorbed without affecting jobseekers' prospects. Once again, no jobs are stolen when immigrants arrive. This paper shows that in a matching model with job rationing, in contrast, the entry of immigrants reduces the employment rate of native workers. Moreover, the reduction in employment rate is sharper when the labor market is depressed---because jobs are more scarce then. Because immigration reduces labor-market tightness, it makes it easier for firms to recruit and improves firm profits. The overall effect of immigration on native welfare depends on the state of the labor market. It is always negative when the labor market is inefficiently slack, but some immigration improves welfare when the labor market is inefficiently tight.

\end{titlepage}\section{Introduction}
 
Immigrants are always accused of stealing people's jobs. As \citet[p. 249]{C90} notes in his famous paper on the Mariel Boatlift, a three-day riot occurred in several Black neighborhoods following the arrival of the Cuban immigrants in Miami in 1980, killing 13 people. The government-sponsored committee that was set up to investigate the riot cited the labor market competition of Cuban refugees as an important factor.

Yet, standard models of the labor market cannot capture this phenomenon. In a neoclassical model of the labor market, there is no unemployment, so anybody who wants to work can work. This means that there are jobs for everybody and no jobs to steal. In standard matching models---either the textbook model from \citet{P00} or the version with rigid wage from \citet{H05}---there is some unemployment, but labor demand is perfectly elastic with respect to tightness \citep{M14}. Hence, new entrants into the labor force are absorbed without affecting other workers. Once again, no jobs are stolen when immigrants arrive. 

These standard models can therefore not capture the phenomenon that immigrant workers entering the labor market might take jobs away from native workers.\footnote{What these models can capture is the effect of immigrant workers on the wages of native workers \citep{OP12,A21}. Such effect does not appear much in the public discourse, however.} Because they rule out the phenomenon, these models cannot say under which circumstance it might occur and under which circumstances it might not. If job stealing occurs, they cannot predict when it is likely to be severe and when it is likely to be mild. And down the line, they cannot be used to design immigration policies because they rule out what is---at least in the public discourse---the main negative effect of immigration on the life of natives.

This paper therefore presents a model that allows for job stealing (section~\ref{s:model}). In the matching model with job rationing from \citet{M12}, the entry of migrants reduces the employment rate and increases the unemployment rate of native workers---who might naturally feel that immigrants steal their jobs. Moreover, the wage of natives is unaffected by immigration, so immigration hurts the number of natives who have a job but not their income on the job. These basic predictions of the model are consistent with the impact of immigration estimated from natural experiments in European countries \citep{H92,AK03,G12}.

The mechanism is simple (section~\ref{s:immigration}). In the model the number of jobs available is somewhat limited. When immigrants enter the labor force, it increases the number of jobseekers and reduces labor market tightness. This makes hiring more attractive to firms, so they hire some more workers---some native and some immigrants. While this additional hiring boosts labor market tightness, tightness always remains below its level before immigration because the number of jobs available is not sufficient to absorb all newcomers. Since labor market tightness is lower, the job-finding rate of all jobseekers is lower, the unemployment rate goes up, and the employment rate goes down. That is, fewer natives hold jobs---explaining their feeling that immigrants steal their jobs.

In bad times, the lack of jobs is more stringent (section~\ref{s:businesscycle}). This means that firms absorb fewer of the new entrants into the labor force. More of the new entrants therefore remain jobless, which increases the competition for jobs. As a result, natives are more negatively affected by the arrival of immigrants in bad times. Formally, the elasticity of the employment rate with respect to immigration is more negative in bad times. This might explain why the backlash against immigration seems stronger in bad times. 

Because immigration reduces labor-market tightness, however, it makes it easier for firms to recruit workers and improves firm profits. So while immigration always hurts workers, it always helps firms. The overall effect of immigration on native welfare---the sum of native labor income and firm profits---depends on the state of the labor market (section~\ref{s:welfare}). When the labor market is inefficiently slack, allowing immigration always reduces native welfare. But some immigration improves native welfare when the labor market is inefficiently tight. In that case, the native labor income lost from immigration is more than offset by the increase in firm profits. 

While the analysis focuses on immigration, it could also be applied to any other shocks to the size of the labor force (section~\ref{s:discussion}). The model shows that an increase in labor force reduces labor-market tightness in the short run---an outward shift of the labor supply along a downward-sloping labor demand in an employment–tightness plane. Conversely, a decrease in labor force increases labor-market tightness in the short run---an inward shift of the labor supply along a downward-sloping labor demand in an employment–tightness plane. Thus the model helps explain why US labor-market tightness was especially high during World War 2, the Korean War, and the Vietnam War \citep[figure 12B]{MS23}. A reason is that millions of workers were sent abroad on military duty instead of being the labor force at home. It also helps explain why labor-market tightness has been historically high in the recovery from the coronavirus pandemic \citep[figure 12B]{MS23}. A reason is that the pandemic induced a drop in labor-force participation rate by more than 3 percentage points.

\section{Model of the labor market}\label{s:model}

This section introduces the model of the labor market on which the analysis is based---which is a simplified version of the model in \citet{M12}. The model features both frictional and rationing unemployment.

\subsection{Assumptions}

The labor market is composed of a mass 1 of firms and a labor force of size $H$. The matching function between unemployment workers and vacant jobs is takes a Cobb-Douglas form: 
\begin{equation*}
m(U,V) = \o \cdot U^{\h} \cdot V^{1-\h},
\end{equation*} 
where $U$ is the number of unemployed workers, $V$ is the number of vacant jobs, and $\h\in (0,1)$ is the matching elasticity. 

All workers are paid a same real wage $w>0$. An implication is that wages in a given labor market do not respond to immigration, in line with US evidence that immigration has small and generally insignificant effects on the wages of native workers \citep{OP12}.\footnote{In the matching model wages are determined in a situation of bilateral monopoly so it is a wage norm and not an auctioneer that determines wages. The advantage is that the wage norm assumed can be shaped by evidence. Here I assume no response of wages to immigration because this is what evidence suggests. But if evidence evolves and it happens that wages systematically increase or decrease with immigration, the wage norm could be amended to capture this fact. Then the analysis can be repeated with the updated wage norm. Another advantage of the matching model is that the response of employment and wages to shocks are disconnected, because slack might also respond to shocks. For instance in a neoclassical model, if wages do not respond to immigration, the labor demand must be perfectly elastic, so that the employment rate must be invariant to immigration. Not so in the matching model. In this paper the wage does not respond to immigration but the unemployment rate and employment rate do. The matching model is therefore much more flexible and more able to describe the facts.}

Firms have a concave production function 
\begin{equation*}
y(N) = a \cdot N^{\a},
\end{equation*} 
where $a$ governs labor productivity, $N$ denotes the number of producers in the firm, and $\a \in (0,1)$ indicates diminishing marginal returns to labor. 

Firms also incur a recruiting cost of $r > 0$ recruiters per vacancy and face a job-destruction rate $s > 0$. The total number of recruiters in the firm is $R = rV$ and the total number of workers is $L = R + N$.

\subsection{Matching rates}

Workers match with firms at a rate $f(\t)$ given by
\begin{equation*}
f(\t) = \frac{m(u,V)}{U} =  m(1,\t) = \o \t^{1-\h}.
\end{equation*}
The elasticity of the job-finding rate with respect to tightness simply is $\e^f_{\t} = \oex{f}{\t} = (1-\h)$.

Vacancies are filled with workers at a rate $q(\t)$ given by
\begin{equation*}
q(\t)= \frac{m(u,V)}{V} = m(\t^{-1},1) = \o \t^{-\h}.
\end{equation*}
The elasticity of the vacancy-filling rate with respect to tightness simply is $\e^q_{\t} = \oex{q}{\t} = -\h$.

\subsection{Labor supply}

In the matching framework, the employment level is given by the following law of motion:
\begin{equation}
 \dot{L}(t) = f(\t) U(t) - s L(t).
\label{e:ldot}\end{equation} 
That is, the employment level increases over time ($\dot{L}>0$) if more jobseekers find jobs than employed workers lose their jobs ($f(\t) U > s L$). Conversely, employment decreases over time if more employed workers lose their jobs than jobseekers find jobs.

Since $U(t) = H - L(t)$, the law of motion \eqref{e:ldot} can be rewritten as the following differential equation:
\begin{equation*}
\dot{L}(t) = f(\t) H - \bs{s+f(\t)} L(t).
\end{equation*} 
The critical point of this differential equation is
\begin{equation}
L = \frac{f(\t)}{s+f(\t)} H.
\label{e:l}\end{equation}
This positive relationship between employment and tightness is the locus of unemployment and tightness such that the number of new employment relationships created at any point in time equals the number of relationships dissolved at any point in time. It is the locus of points such that the employment level remains constant over time and labor-market flows are balanced. It is also isomorphic to the Beveridge curve.

The employment levels given by equations~\eqref{e:ldot} and~\eqref{e:l} are indistinguishable (\citealt[pp.~398--399]{Ha05}; \citealt[p.~236]{Pi09}). In fact, \citet[p.~31]{MiS21} find that when $s$ and $f(\t)$ are calibrated to US data, the deviation between the two employment levels decays at an exponential rate of 62\% per month. This means that about 50\% of the deviation evaporates within a month, and about 90\% within a quarter. 

I therefore assume that at the time scale of the model, labor-market flows are always balanced, and the employment level is given by equation~\eqref{e:l} at all times. This assumption is akin to the assumption that people are always rational in macroeconomic models, while neglecting the learning period that it takes for people to converge to a rational behavior. It is also akin to the assumption that people always know Nash or other equilibria in game theory, while neglecting the learning or coordination period that it takes to reach such equilibria.

The employment level consistent with balanced flows is the labor supply:
\begin{equation}
L^s(\t,H) = \frac{f(\t)}{s+f(\t)}H.
\label{e:ls}\end{equation}
In the model, the labor supply holds at any point in time. The labor supply links employment to tightness and labor force. From the labor supply we can also relate the unemployment rate to tightness. The unemployment rate is $u = (H-L^s)/H = 1 - L^s/H$ so
\begin{equation}
u(\t) = \frac{s}{s+f(\t)}.
\label{e:u}\end{equation}

The elasticity of labor supply with respect to tightness is
\begin{equation}
\e^{s}_{\t} = \oe{L^s}{\t} = \e^{f}_{\t} - \frac{f}{s+f}\e^{f}_{\t} = (1-\h) - \frac{f(\t)}{s+f(\t)}(1-\h) = u(\t) (1-\h),
\label{e:est}\end{equation}
where $u(\t)$ is the unemployment rate implied by the labor supply, given by \eqref{e:u}.
The elasticity of labor supply with respect to the labor force is
\begin{equation*}
\e^{s}_H = \oe{L^s}{H} = 1.
\end{equation*}

Since the unemployment rate satisfies $u(\t) = 1 - L^s(\t)/H$, the elasticity of the unemployment rate with respect to tightness is
\begin{equation*}
\e^{u}_{\t} = \oe{u}{\t} = \frac{-L^s/H}{1-L^s/H}\cdot \e^s_{\t} = -\frac{1-u(\t)}{u(\t)}\cdot \e^s_{\t}.
\end{equation*}
Using the expression \eqref{e:est} for $\e^s_{\t}$ then yields
\begin{equation}
\e^{u}_{\t} =  - [1-u(\t)] (1-\h).
\label{e:eut}\end{equation}

\subsection{Recruiter-producer ratio}

Because it takes time to fill vacancies and each vacancy requires the attention of a recruiter, firms must allocate a share of their workforce to recruiting. And because the model is cast on a time scale where labor-market flows are balanced, firms post vacancies to maintain their firm at a given desirable size. That is, they post vacancies $V$ so the number of new hires $q(\t)V$ just replaces the number of workers who left the firm $sL$. 

Multiplying $sL = q(\t)V$ by the recruiting cost $r$, and using $L = R+N$ and $R = rV$, yields $rs(N+R) = q(\t) R$. Dividing both sides by $R$ then gives $rs(1+\tau(\t)^{-1}) = q(\t)$, where $\tau(\t) = R/N$ is the recruiter-producer ratio. The recruiter-producer ratio is therefore given by
\begin{equation*}
\tau(\t) = \frac{rs}{q(\t)-rs}.
\end{equation*}
This means that
\begin{equation*}
1+\tau(\t) = \frac{q(\t)}{q(\t)-rs}.
\end{equation*}
The recruiter-producer ratio $\tau(\t)$ is positive and increasing on $[0,\t_{\tau})$, where $\t_{\tau}$ is defined by $q(\t_{\tau}) = rs$; furthermore, $\tau(0)=0$ and $\lim_{\t\to \t_{\tau}} \tau(\t) = + \infty$.

The wedge $1+\tau(\t)$ plays an important role in the analysis because it determines the gap between numbers of employees and producers in the workforce: 
\begin{equation*}
L = N + R = [1+\tau(\t)] N. 
\label{e:lnr}\end{equation*}
The elasticity of $1+\tau$ with respect to $\t$ is
\begin{equation*}
\e^{1+\tau}_{\t} = \e^{q}_{\t} - \e^{q-rs}_{\t} = -\h - \frac{q(\t)}{q(\t)-rs} (-\h),
\end{equation*}
so
\begin{equation}
\e^{1+\tau}_{\t} = \h\tau(\t).
\label{e:etaut}\end{equation}

\subsection{Labor demand}

Firms operate within the balanced-flow paradigm. By choosing how many vacancies to post, they determine their workforce, which they in turn choose to maximize flow real profits: 
\begin{equation*}
y(N) - wL = y(N) - [1+\tau(\t)] \cdot w \cdot N. 
\end{equation*}
The first-order condition of the firm's maximization problem is $y'(N) - [1+\tau(\t)] \cdot w = 0$, or
\begin{equation*}
a \a N^{\a-1} =[1+\tau(\t)]w.
\end{equation*}

With \eqref{e:lnr}, the first-order condition becomes
\begin{equation*}
a \a [1+\tau(\t)]^{1-\a}{L}^{\a-1} = [1+\tau(\t)]w,
\end{equation*}
which then yields the firm's labor demand:
\begin{equation}
L^d(\t,a) = \bc{\frac{a\a}{w[1+\tau(\t)]^{\a}}}^{\frac{1}{1-\a}}.
\label{e:ld}\end{equation}
The labor demand gives the firm's employment level for any tightness and productivity. There is job rationing in the model because the labor demand is decreasing with tightness. The labor demand fluctuates in response to productivity shocks over the business cycle.

The elasticity of labor demand with respect to tightness is
\begin{equation}
\e^{d}_{\t}=\oe{L^d}{\t}= -\frac{\a}{1-\a}\e^{1+\tau}_{\t} = -\frac{\a}{1-\a}\h\tau(\t).
\label{e:edt}\end{equation}

\subsection{Solution of the model}

The model requires that firms maximize profits and employment is determined by the matching and separation process. This imposes that labor demand equals labor supply:
\begin{equation*}
L^d(\t,a) = L^s(\t,H).
\end{equation*}

The labor demand is strictly decreasing in $\t$ while the labor supply is strictly increasing in $\t$. For any given $H$, the equation pins down a unique $\t$: the unique solution of the model. Therefore, the supply-equals-demand condition implicitly defines $\t(H)$. Once we have $\t(H)$, plugging into either labor supply or labor demand defines the employment level as a function of $H$, $L(H)$. The unemployment rate is then $u(\t(H))=s/[s+f(\t(H))]$. The solution of the model is illustrated in figure~\ref{f:solution}.

\begin{figure}[t]
\includegraphics[scale=\sfig,page=1]{\pdf}
\caption{Solution of the model}
\note{The labor demand curve is given by \eqref{e:ld}. The labor supply is given by \eqref{e:ls}. The solution of the model is at the intersection of the labor demand and supply curves.}
\label{f:solution}\end{figure}

\section{Impact of immigration on the labor market}\label{s:immigration}

An inflow of immigration changes the labor force and therefore tightness, job-finding rate, and unemployment rate. Here we describe the impact of immigration on the labor market. 

\subsection{Modeling immigration}

A wave of immigration leads to a sudden increase in the number of workers in the labor force, so a sudden increase in $H$. Here the focus is on the short-term effects of immigration: only labor supply and not labor demand is affected. Assuming that labor demand is unaffected for instance rules out that firms adjust their capital stock when immigrants arrive. This is a standard assumption to describe the short run. In the long run, as firms adjust capital stock and production process, the impact of immigration on the labor market will vanish because the labor demand will scale up with labor supply.

\subsection{Impact of immigration on native workers}

I now determine the impact of immigration on the prospects of native jobseekers. The main step is to compute the elasticity of tightness $\t(H)$ with respect to the labor force $H$. From this, I will infer the elasticity of the job-finding rate $f(H)$ with respect to the labor force $H$. 

Consider a small change in the size of the labor force generated by a small wave of immigration, $d\ln H$. This small change generates a small change in tightness, $d\ln \t$. These changes generate small changes in labor supply and demand:
\begin{align*}
d \ln L^s &= \e^{s}_{\t} d\ln \t + \e^{s}_H d\ln H\\
d \ln L^{d} &= \e^{d}_{\t} d\ln \t.
\end{align*}

Since the supply-equals-demand condition must hold both before and after the wave of immigration, $d\ln L^s = d\ln L^d$. This means that
\begin{equation*}
\e^{s}_{\t} d\ln \t + \e^{s}_H d\ln H = \e^{d}_{\t}d\ln \t.
\end{equation*}
In other words, the elasticity of tightness with respect to labor force, $\e^{\t}_H  = \oex{\t}{H}$, is given by
\begin{equation}
\e^{\t}_H = \frac{-\e^{s}_H}{\e^{s}_{\t}-\e^{d}_{\t}} =  \frac{-1}{\e^{s}_{\t}-\e^{d}_{\t}}.
\label{e:eth}\end{equation}
Therefore, the elasticity of the job-finding rate with respect to the labor force, $\e^{f}_H  = \oex{f(\t)}{H} = \oex{f}{\t} \cdot \oex{\t}{H}$, is given by
\begin{equation}
\e^{f}_H = \frac{-(1-\h)}{\e^{s}_{\t}-\e^{d}_{\t}}.
\label{e:efh}\end{equation}
Finally, the elasticity of the unemployment rate with respect to the labor force is
\begin{equation*}
\e^{u}_H = \e^{u}_{\t} \cdot \e^{\t}_H = \frac{1-u(\t)}{u(\t)}\cdot \frac{\e^s_{\t}}{\e^{s}_{\t}-\e^{d}_{\t}}.
\end{equation*}
From this, I infer the semi-elasticity of the unemployment rate with respect to the labor force, which gives the percentage-point change in unemployment rate when the labor force changes by one percent:
\begin{equation}
\frac{du}{d\ln(H)} = u \cdot \e^{u}_{H} = \frac{1-u(\t)}{1-[\e^{d}_{\t}/\e^s_{\t}]}.
\label{e:suh}\end{equation}

A last useful statistic is the the elasticity of the employment rate $l = 1 - u$ with respect to labor force, which follows from \eqref{e:suh}:
\begin{equation}
\oe{l}{H} = \frac{1}{1-u}\cdot\frac{-du}{d\ln(H)} = \frac{-1}{1-[\e^{d}_{\t}/\e^s_{\t}]}.
\label{e:elh1}\end{equation}

Collecting these results yields the following proposition:
\begin{proposition}
A wave of immigration leads to a decrease in labor market tightness, which causes a decrease in the job-finding rate of native workers, an increase in their unemployment rate, and a decrease in their employment rate.
\label{p:labor}\end{proposition}

The proposition is a direct consequence from the facts that $\e^s_{\t}>0$ and $\e^d_{\t}<0$ and from \eqref{e:eth}, which shows that the elasticity of tightness with respect to the labor force is negative, from \eqref{e:efh}, which shows that the elasticity of the job-finding rate with respect to the labor force is negative, from\eqref{e:suh}, which shows that the semi-elasticity of unemployment rate with respect to the labor force is positive, and finally from \eqref{e:elh1}, which shows that elasticity of the employment rate with respect to the labor force is positive. 

The intuition for the proposition is simple. A wave of immigration increases the number of available workers, but not the number of jobs, so it raises labor supply and not labor demand. Such increase in supply relative to demand leads to a decrease in tightness (figure~\ref{f:immigration}). 

To understand why tightness has to fall after a wave of immigration, let's think about what would happen if tightness did not respond. Then firms would want to employ the same number of workers as before, since the labor demand had not changed. But since tightness has not changed, jobseekers' job-finding rate has not changed. And since the pool of jobseekers increased after immigrants joined the labor force, more jobseekers will end up with a job. (That is, labor supply is higher than labor demand at the current tightness.) Firms would respond to such excess of new hires by posting fewer vacancies, leading to a drop in tightness, until supply and demand are equalized.

\begin{figure}[t]
\includegraphics[scale=\sfig,page=2]{\pdf}
\caption{Impact of immigration on the labor market}
\note{The labor demand curve is given by \eqref{e:ld}. The labor supply is given by \eqref{e:ls}. This graph illustrates the results from propositions~\ref{p:labor} and \ref{p:firms}.}
\label{f:immigration}\end{figure}

The other results in the proposition directly follow from the drop in tightness. Lower tightness means lower job-finding rate, so a wave of immigration reduces the job-finding rate of the natives. A lower job-finding rate means a higher unemployment rate, so a wave of immigration increases the unemployment rate faced by the natives, and decreases their employment rate. The underlying reason is that after an influx of new workers into the labor force, there is the same number of jobs but more jobseekers, so it becomes harder to find a job. As a result, a larger fraction of workers remains unemployed, and a smaller fraction is employed. Of course, since the unemployment rate increases and the size of the labor force increases, the number of unemployed workers increases sharply after a wave of immigration---some of the unemployed are native workers and some are immigrant workers.

Given the comparative statics obtained in proposition~\ref{p:labor}, it is unsurprising that native workers feel that immigrants ``steal their jobs.'' First, the job-finding rate for native jobseekers falls when immigrants arrive. So it becomes harder for any one jobseeker to find a job. They have fewer jobs available to them because the number of available jobs did not scale up with the increase in labor-force participants. Second, the employment rate of native decreases when immigrants arrive. So native workers might feel that immigrants ``steal their jobs'': the fraction of native workers how have a job is indeed lower, and the fraction who remain unemployed is higher. And the reason is that immigrant workers are now employed in some of the available jobs, displacing native workers.

\subsection{Evidence of job stealing in natural experiments}

A popular perception is that immigrants steal people's jobs. There is also evidence of job stealing in natural experiments. The focus of economists studying the natural experiments is often on the effect of immigration on wages, but they also often report the effect of immigration on unemployment. 

The predictions in proposition~\ref{p:labor} are consistent with the findings by \citet{H92}, who examines the repatriation to France of Algerians of European origin following Algerian independence in 1962. \citet[p. 566]{H92} finds that a 1 percentage point increase in the labor force caused by the arrival of repatriates raised the unemployment rate of natives by 0.2 percentage points. This corresponds to a semi-elasticity of $0.2>0$. In several specification the wages of natives are unaffected by immigration, as the model assumes.

Proposition~\ref{p:labor}'s predictions are also consistent with the findings by \citet{AK03}, who look at immigration from former Yugoslavia into other European countries in the 1990s. With a basic OLS specification, \citet[p. F318]{AK03} finds that the entry of 100 immigrants in the labor force pushes 35 native workers into unemployment. With an IV specification, \citet[p. F322]{AK03} finds that the entry of 100 immigrants in the labor force pushes 83 native workers into unemployment. These findings imply quite a large, negative elasticity of the employment rate with respect to labor force. Such elasticity is measured by $\e^l_H = \oex{l}{H} = (H dl)/(l dH) = -(H du)/(l dH)$. The quantity $H du $ is the number of native workers who become unemployed as a result of immigration, while $dH = 100$ is the number of immigrants who enter the labor force. The OLS result is $H du = 35$ while the IV result is $H du = 83$. So the OLS elasticity is $\e^l_H = -0.35/(1-u)$ while the IV elasticity is $\e^l_H = -0.83/(1-u)$, where $u$ is the unemployment rate. With an unemployment rate around 10\% in Europe at the time, the resulting elasticities are $\e^l_H = -0.35/0.9 = -0.39<0$ from OLS, and $\e^l_H = -0.83/0.9 = -0.92<0$ from IV.

\citet{G12} studies the return of 2.8 million ethnic Germans to Germany in the 15 years following the fall of the Berlin Wall. The key finding is that the immigrants had no effect on relative wages, as our model assumes, but 3.1 native workers became unemployed for every 10 immigrants that find a job. This corresponds to an elasticity of the employment rate with respect to labor force about $-0.3<0$. Indeed, the elasticity is measured as $\e^l_H = -(H du)/(l dH)$. The quantity $H du = 3.1$ is the number of native workers who become unemployed as a result of immigration, while $l dH = 10$ is the number of immigrants who find a job.

Interestingly, the predictions in proposition~\ref{p:labor} are not inconsistent with the results from \citet{C90}, who studies the impact of the Cuban immigrants from the Mariel Boatlift on the Miami labor market in the 1980s. A first, well-known finding from the study is that ``the Mariel immigration had essentially no effect on the wages or employment outcomes of non-Cuban workers in the Miami labor market'' \citep[p. 255]{C90}. A second finding is that ``perhaps even more surprising, the Mariel immigration had no strong effect on the wages of other Cubans'' \citep[p. 255]{C90}. In a matching model, wages follow a wage norm. With a wage norm independent from the labor force, as assumed here, there is no reason that wages respond to a wage of immigration. 

A third, less-known finding is that the unemployment rate for Cuban workers increased drastically: ``Unlike the situation for whites and blacks, there was a sizable increase in Cuban unemployment rates in Miami following the Mariel immigration. Cuban unemployment rates were roughly 3 percentage points higher during 1980-81 than would have been expected on the basis of earlier (and later) patterns'' \citep[p. 251]{C90}. If labor markets are segregated by ethnicity, then that is what proposition~\ref{p:labor} would predict: a large increase in the labor force in the Cuban labor market would result in a sharp increase in the unemployment rate in that labor market. Whether unemployment affects particularly new Cuban immigrants or equally all Cuban workers depends on the specifics of the matching process. This model assumes completely random matching, in which case all workers would be equally affected. But with any form of ranking in the recruiting process, as in \citet{BD94}, it is not hard to imagine that the Mariel workers face a higher unemployment rate than incumbent Cuban workers. Either way, the fact that the Cuban workers face a higher unemployment rate is evidence that the labor market could not absorb all new arrivals, and that jobs are somewhat rationed.

\subsection{Impact of immigration on firms}

Immigration unambiguously hurts native workers in the short run. On the other hand, it unambiguously helps native firms:
\begin{proposition}
A wave of immigration leads to an increase in employment, a reduction in the recruiter-producer ratio, and an increase in real profits.
\label{p:firms}\end{proposition}

The beginning of the proposition follows from the facts that labor demand is decreasing with tightness, the recruiter-producer ratio is increasing with tightness, and tightness falls after an immigration wave.

The impact of immigration on real profits requires a little bit of algebra. Aggregate real profits are given by $\pi(\t) = y(N) - [1+\tau(\t)] w N$. Output can be rewritten as a function of the marginal product of labor: $y(N) = y'(N)N/\a$. Moreover, on the labor demand, the marginal product of labor is always related to the wage and recruiter-producer ratio: $y'(N) = [1+\tau(\t)] w$. Combining these results I express real profits as a function of the wage bill $wL$:
\begin{equation}
\pi(\t) =\bs{\frac{1}{\a}-1} [1+\tau(\t)] w N = \frac{1-\a}{\a} w L.
\label{e:pi}\end{equation}
A wave of immigration leads to higher employment so higher profits.

The number of jobs in the economy actually increases after the wave of immigration, as do firms' real profits. Firm owners are therefore always benefiting from immigration---unlike workers who are always suffering from it. If workers own share of firms, of course, the impact of immigration is murkier: immigration reduces their labor income but raise their capital income. If workers own very little capital, then it is clear that they are negatively affected by immigration.


\section{Immigration in good and bad times}\label{s:businesscycle}

Beyond the basic results from section~\ref{s:immigration}, I now contrast the effects of immigration on the labor market in good times---when tightness is high---and bad times---when tightness is low. This might help better understand popular perceptions of immigration. 

\subsection{Modeling good and bad times}

In the United States labor-market fluctuations are driven by labor-demand shocks, not labor-supply shocks \citep{MS15}. I therefore model good and bad times as period with high and low labor demand. In this simple model labor demand is governed by the wage-to-productivity ratio, $w/a$ (equation \eqref{e:ld}). Good times are periods when the wage-to-productivity ratio is low so hiring workers is particularly profitable and the labor demand is elevated (figure~\ref{f:goodtimes}). Bad times are periods when the wage-to-productivity ratio is high so hiring workers is not very profitable and the labor demand is depressed (figure~\ref{f:badtimes}). 

What causes changes in the wage-to-productivity ratio, $w/a$? The typical cause of these fluctuations are fluctuations in productivity $a$ under a fixed wage $w$. \citep{H05}. Another possibility are fluctuations in productivity $a$ under a partially rigid wage $w = \o \cdot a^{\g}$, where $\g<1$ \citep{BG10,M12,M14,LaMS18}. Assuming an elasticity of the wage with respect to productivity is below 1 is in line with evidence found by \citet{HSV13}, and is sufficient to generate fluctuations in the wage-to-productivity ratio.\footnote{In fact the elasticity of the wage with respect to productivity estimated by \citet{HSV13} is low enough to generate realistic business cycles \citep{M12}.}

\begin{figure}[t]
\subcaptionbox{Bad times: high wage-to-productivity ratio\label{f:badtimes}}{\includegraphics[scale=\sfig,page=3]{\pdf}}\hfill
\subcaptionbox{Good times: low wage-to-productivity ratio\label{f:goodtimes}}{\includegraphics[scale=\sfig,page=4]{\pdf}}
\caption{Good and bad times in the labor-market model}
\note{The labor supply is given by \eqref{e:ls}. A: The labor demand curve is given by \eqref{e:ld} with a high $w/a$. B: The labor demand curve is given by \eqref{e:ld} with a low $w/a$. This graph illustrates good and bad times in the model.}
\label{f:goodbadtimes}\end{figure}

In this basic model, all fluctuations are driven by productivity shocks. In reality, it is aggregate demand shocks and not technology shocks that drive fluctuations in labor demand. Aggregate-demand shocks affect labor demand because they influence the utilization rate of workers, which would show in the productivity parameter $a$. So in a macroeconomic version of the model, aggregate demand shocks would materialize just like the present productivity shocks \citep{MS15}. The analysis therefore carries over whether technology or aggregate demand generate labor-demand shocks.

\subsection{Labor market in good and bad times}

Figure~\ref{f:goodbadtimes} allows me to obtain a range of comparative statics describing the business cycle:

\begin{proposition}
Labor-market conditions deteriorate when the wage-to-productivity ratio ($w/a$) increases: the labor-market tightness falls; the job-finding rate falls; the unemployment rate increases; the employment rate decreases. In these conditions, however, recruiting becomes easier: the vacancy-filling rate increases and the recruiter-producer ratio falls.
\label{p:businesscycle}\end{proposition}

Figure~\ref{f:goodbadtimes} shows that tightness drops when the wage-to-productivity ratio increases and labor-demand curve falls. All the other results follow since all the other quantities are functions of tightness. The response of tightness could also be obtained by implicitly differentiating the equation $L^s(\t) = L^d(\t,w/a)$ with respect to $w/a$.

\subsection{Impact of immigration in good and bad times}

Proposition~\ref{p:welfare} shows that immigration always reduces welfare when the labor market is inefficiently slack, while some immigration improves welfare when the labor market is inefficiently tight. Here I examine how the amount of ``job stealing'' varies with the business cycle. 

An increase in immigration increases the unemployment rate, so it reduces the employment rate. Native workers therefore hold fewer jobs after a wave of immigration: the model produces the type of job stealing that native workers commonly complain about. In addition, such job stealing is worse in bad times:
\begin{proposition}
The elasticity of the employment rate with respect to the labor force is
\begin{equation}
\e^l_H(\t) = \oe{l}{H} = \frac{-1}{1+\frac{\a}{1-\a}\cdot \frac{\h}{1-\h}\cdot \frac{\tau(\t)}{u(\t)}}. 
\label{e:elh2}\end{equation}
The elasticity $\e^l_H(\t)$ is negative and increasing with tightness. As a result, when labor-market conditions deteriorate, the elasticity becomes more negative. The elasticity tends to $-1$ when tightness goes to $0$.
\label{p:stealing}\end{proposition}

\begin{proof} The elasticity \eqref{e:elh2} is obtained from \eqref{e:elh1}, \eqref{e:est}, and \eqref{e:edt}. Since $\tau(\t)$ is increasing with $\t$ while $u(\t)$ is decreasing with $\t$, it is clear that $\e^l_H(\t)$ is increasing with tightness.\end{proof}
	
The proposition shows that job stealing is more prevalent in bad times. Formally, the percentage reduction in employment rate due to a one-percent increase in the labor force is larger in bad times, when tightness is low. 

The worst case scenario occurs when the labor market is the slackest: then a one-percent increase in the labor force leads to a one-percent decrease in employment rate. The reason is that when the labor market is extremely slack, job rationing is the most stringent, so the number of jobs is almost fixed. With a fixed number of jobs, employment rate and labor force are related by $l \cdot H = \text{constant}$ so the elasticity of the employment rate with respect to the labor force is clearly $-1$.

The model therefore predicts that job stealing is more prevalent in bad times. Because the number of jobs is more limited in bad times, immigration will reduce the native employment rate more drastically in bad times. Native workers are therefore likely to be more opposed to immigration in bad times, because it hurts their labor-market prospects more.

The semi-elasticity of unemployment with respect to the labor force found by \citet{H92} translates into an elasticity of the employment rate with respect to the labor force of $-0.2$, since the employment rate is very close to 1 in France at the time.\footnote{The exact relation is $\e^l_H = -[du/dln(H)]/(1-u)$, but $1-u\approx 1$ in France at the time \citep[table 1]{H92}.} Since might seems low, but the labor market was also extremely tight in France at the time. When repatriation started in 1962, the unemployment rate in France was only $1\%$  \citep[table 1]{H92}! Proposition~\ref{p:stealing} predicts that the elasticity would be less negative when in good times. So the very tight labor market in France in the 1960s might explain why the effects of repatriation on the native employment rate were so muted.

\subsection{Relation with previous work}

Equation \eqref{e:elh1} shows that the effect of immigration on employment is determined by the ratio between the elasticities of labor supply and demand with respect to tightness, $\e^d_{\t}/\e^s_{\t}$. This ratio captures the relative slopes of supply and demand. This ratio also determines the size of the public-employment multiplier $\l$, as showed by \citet[equation (8)]{M14}: 
\begin{equation}
\l = 1- \frac{1}{1 -(\e^s_{\t}/\e^d_{\t})} = \frac{1}{1 -(\e^d_{\t}/\e^s_{\t})} = \e^l_H.
\label{e:lambda}\end{equation}

Equation \eqref{e:lambda} shows that the public-employment multiplier is actually exactly the same as the elasticity of the employment rate with respect to the labor force. It is for the same reasons that the public-employment multiplier is positive and larger in bad times that job stealing occurs and especially in bad times. Because the number of jobs in the private sectors is somewhat limited, creating public-sector jobs will increase employment. And because the number of jobs in the private sectors is somewhat limited, the arrival of immigrants will take some jobs away from natives and reduce the employment rate of natives. 

When the labor market is more depressed, the crowding out of private jobs by public jobs is less because private firms are not hurt much by public vacancies---this is because their are so many workers looking for jobs. At the same time, private firms will not benefit much from the presence of immigrant jobseekers---again because there are already so many jobseekers available. So private firms will not create many new jobs when immigrants arrive in bad times; as a consequence, immigrants end up taking jobs away from native workers.

The mechanism also explains why the macro effect of an increase in unemployment insurance on unemployment is less than its micro effect \citep{LMS18,LaMS18}. Just like the arrival of immigrants as larger effects on the employment rate of natives in bad times, when tightness is low, the gap between macro and micro effects of unemployment insurance is larger.

\subsection{Impact of immigration in standard models}

Standard matching models have constant marginal returns to labor instead of diminishing marginal returns ($\a=1$). With constant returns to labor, the labor demand is degenerate: firm's optimal employment choice solely determines tightness. Setting $\a=1$ in \eqref{e:ld} gives
\begin{equation}
[1+\tau(\t)] \frac{a}{w} = 1,
\label{e:ldstd}\end{equation}
which determines tightness in the model, independently from employment. This labor-demand relation holds irrespective of the wage-setting assumption. It holds for instance with rigid wages, as in \citet{H05}, or with the more traditional Nash bargaining, as in the textbook model \citep{P00}.

The labor demand does not involve labor force $H$, so the tightness is the same irrespective of the amount of immigration. This means that the employment rate is independent of labor force: there is no job stealing at all, neither in good times nor not in bad times. Figure \ref{f:standard} illustrates: an increase in the labor force from immigration is absorbed entirely by firms, leaving native workers unaffected.

\begin{figure}[t]
\includegraphics[scale=\sfig,page=5]{\pdf}
\caption{Impact of immigration on the labor market in a standard matching model}
\note{The labor demand curve is given by \eqref{e:ldstd}. The labor supply is given by \eqref{e:ls}.}
 % This graph illustrates the results from proposition~\ref{p:immigrationstd}.}
\label{f:standard}\end{figure}

This result is of course connected to the result that in standard matching models with constant returns to labor, the public-employment multiplier is zero \citep[p. 199]{M14}. It is also related to the result that in the same standard matching models, the macroelasticity of unemployment with respect to unemployment insurance is the same or larger than the microelasticity \citep{LMS18}.

\section{Impact of immigration on native welfare}\label{s:welfare}

In the short run immigration hurts native workers---especially in bad times---but helps native firms. To understand the overall impact of immigration on natives, I now assess the effect of immigration on native welfare---the welfare of native workers plus native firm owners.

\subsection{Computing welfare}

In the model there is only one consumption good, produced by firms, which goes to native workers through the labor income, to immigrant workers through their labor income, and to firm owners through profits. The goal is to assess how immigration impacts native labor income plus profits.

We denote the native labor force as $H$, and the total labor force as $\m H$, where $\m\geq 1$ captures the growth of the labor force due to immigration. Broadly, $\m - 1 \geq 0$ is the percentage change in the labor force cause by immigration.

From \eqref{e:pi}, I express profits as a function of the scale of immigration $\m$ and the employment rate $l = 1 - u$:
\begin{equation*}
\pi = \frac{1-\a}{\a} w L = \frac{1-\a}{\a} w l \m H.
\end{equation*}
The native labor income is just the real wage times native employment:
\begin{equation*}
w l H.
\end{equation*}
Adding both components gives native welfare:
\begin{equation*}
\Wc = w H l \bs{\frac{1-\a}{\a} \m + 1} = \frac{w H}{\a} l [(1-\a)\m + \a] .
\end{equation*}

\subsection{Elasticity of welfare with respect to immigration}

From this expression, I compute the elasticity of native welfare with respect to immigration:
\begin{equation*}
\e^{\Wc}_{\m} = \oe{\Wc}{\m} = \oe{l}{\m} + \frac{(1-\a)\m}{(1-\a)\m + \a}
\end{equation*}
Combining this expression with \eqref{e:elh1} then yields
\begin{equation}
\e^{\Wc}_{\m} = \oe{\Wc}{\m} = \frac{(1-\a)\m}{(1-\a)\m + \a} - \frac{1}{1-[\e^{d}_{\t}/\e^s_{\t}]}.
\label{e:ewm}\end{equation}

\subsection{Effect of an infinitesimal wave of immigration on welfare}

As first step, I assess whether any immigration might ever improve welfare. To do that, I determine whether the elasticity $\e^{\Wc}_{\m}$ might ever be positive at $\m = 1$. (Recall that $\m$ goes from $\m=1$ to $\m>1$ when immigration starts.) That is, I compute the effect of an infinitesimal wave of immigration on welfare.

Setting $\m=1$ and using \eqref{e:edt} and \eqref{e:est} yields:
\begin{equation}
\e^{\Wc}_{\m} = \oe{\Wc}{\m} = (1-\a) - \frac{1}{1+\frac{\a}{1-\a}\frac{\h}{1-\h}\frac{\tau(\t)}{u(\t)}}.
\label{e:ewm1}\end{equation}

When $\t \to \t_{\tau}$, $\tau \to \infty$, so $\e^{\Wc}_{\m} \to 1-\a>0$. Clearly, when the labor market is at its tightest (at which point all workers are recruiters), then immigration is desirable.

When $\t \to 0$, $\tau \to 0$, so $\e^{\Wc}_{\m} \to -\a<0$. Clearly again, when the labor market is at its slackest (at which point all workers are unemployed), then immigration is undesirable.

Given that $\tau/u$ is strictly increasing in $\t\in(0,\t_{\tau})$, $\e^{\Wc}_{\m}$ is continuous and strictly increasing in $\t$, and there exists a unique $\t_m\in(0,\t_{\tau})$ at which $\e^{\Wc}_{\m} = 0$, and some immigration improves welfare at any $\t>\t_m$ while any immigration reduces welfare at any $\t<\t_m$.

Solving for $\e^{\Wc}_{\m}(\t) = 0$ with \eqref{e:ewm1}, we obtain
\begin{equation}
\frac{\h}{1-\h}\cdot\frac{\tau(\t)}{u(\t)} = 1.
\label{e:efficiency1}\end{equation}
But this condition is just the efficiency condition in the model \citep[lemma 1]{MS19}. The tightness $\t_m$ is just the tightness that maximizes the number of producers and consumption for a given labor force---the efficient tightness $\t^*$. It is easy to see why. Consumption is determined by the number of producers, $N = L/[1+\tau(\t)] = [1-u(\t)] H/[1+\tau(\t)]$. Maximizing the number of producers is the same as maximizing $[1-u(\t)]/[1+\tau(\t)]$. The elasticity of $1-u(\t)$ with respect to tightness is $(1-\h))u(\t)$ (equation \eqref{e:eut}). The elasticity of $1+\tau(\t)$ with respect to tightness is $\h\tau(\t)$ (equation \eqref{e:etaut}). The number of producers is maximized for a given labor force when its elasticity with respect to tightness is zero, or $\h \tau(\t)= (1-\h) u(\t)$. 

To highlight the parameters that determine the efficient tightness, I can also re-express \eqref{e:efficiency1} as in \citet[equation (29)]{MS22}:
\begin{align*}
1 &= \frac{\h}{1-\h}\cdot\frac{rs}{q(\t)-rs}\cdot\frac{s+f(\t)}{s}\\
(1-\h)[q(\t)-rs] & = \h r [s+f(\t)]\\
(1-\h) q(\t) &= rs + \h r f(\t).
\end{align*}
Dividing both sides by $(1-\h) q(\t)$ and noting that $f(\t)/q(\t)=\t$, I finally get
\begin{equation}
1 = \frac{r}{1-\h} \bs{\frac{s}{q(\t)} + \h\t}.
\label{e:efficiency2}\end{equation}
This is just the efficiency condition in a standard DMP matching model in which the interest rate is 0 and the social value of unemployment is 0 \citep[equation (16)]{MiS21}. Such efficiency condition is obtained by combining the job-creation curve, given in \citet[equation (1.24)]{P00}, with the \citet{H90} condition.

The following proposition summarizes the results:
\begin{proposition}
In any labor market that is inefficiently slack ($\t<\t^*$), allowing some immigration reduces social welfare ($d\Wc/d\m<0$ at $\m=1$). In any labor market that is inefficiently tight ($\t>\t^*$), allowing some immigration improves social welfare ($d\Wc/d\m>0$ at $\m=1$). When the labor market is efficient ($\t=\t^*$), a small wave of immigration has no effect on social welfare ($d\Wc/d\m=0$ at $\m=1$).
\label{p:welfare}\end{proposition}

Whenever the labor market is inefficiently slack, a further decrease in tightness---keeping labor force constant---reduces welfare. From the perspective of native workers, a drop in tightness caused by immigration is equivalent to a drop in tightness keeping labor force constant, since their number does not change. The welfare generated by the increase in the labor force, keeping tightness constant, goes to immigrants and is not counted toward native welfare. Accordingly, immigration reduces native welfare whenever the labor market is inefficiently slack by further reducing tightness. Conversely, immigration improves native welfare whenever the labor market is inefficiently tight by bringing tightness toward efficiency.
% also profits but zero through envelope theorem when tightness is constant since then only through employment change, which is zero at the margin.
	

\subsection{Optimal immigration over the business cycle}\label{s:optimal}

Proposition~\ref{p:welfare} shows that some immigration improves welfare when the labor market is too tight. I now turn to the next question: what is the optimal amount of immigration when the labor market is initially too tight. The question boils down to finding the immigration factor $\m^*$ such that the elasticity $\e^{\Wc}_{\m} = 0$. At that immigration factor, native welfare is maximized ($d\Wc/d\mu =0$) so immigration is optimal.

Using \eqref{e:ewm}, the optimality condition $\e^{\Wc}_{\m} = 0$ becomes
\begin{align*}
\frac{(1-\a)\m}{(1-\a)\m + \a} &= \frac{1}{1-[\e^{d}_{\t}/\e^s_{\t}]}\\
\frac{(1-\a)\m + \a}{(1-\a)\m} &= 1-[\e^{d}_{\t}/\e^s_{\t}]\\
\frac{\a}{1-\a}\cdot \frac{1}{\m} &= -[\e^{d}_{\t}/\e^s_{\t}].
\end{align*}
Using the expressions \eqref{e:est} and \eqref{e:edt} for the elasticities of demand and supply with respect to tightness, I rewrite the condition as
\begin{align*}
\frac{\a}{1-\a}\cdot \frac{1}{\m} &= \frac{\a}{1-\a}\cdot \frac{\h}{1-\h} \cdot\frac{\tau(\t)}{u(\t)}\\
\frac{1}{\m} &=\frac{\h}{1-\h}\cdot \frac{\tau(\t)}{u(\t)}\\
\m &=\frac{1-\h}{\h}\cdot \frac{u(\t)}{\tau(\t)}.
\end{align*}

Since tightness is itself a function of the immigration factor $\m$, I express the optimal immigration as the solution to a fixed-point problem and obtain a first set of results:
\begin{proposition}
The optimal amount of immigration $\hat{\m}>1$ solves the following fixed-point equation:
\begin{equation}
\m =\frac{1-\h}{\h} \frac{u(\t(\m))}{\tau(\t(\m))}.
\label{e:fixedpoint}\end{equation}
At optimum immigration, both left-hand side and right-hand side are positive, so the labor market is inefficiently slack. That is, immigration improves welfare whenever the labor market is inefficiently tight ($\t > \t^*$), and the optimal amount of immigration brings the labor market to an inefficiently slack situation ($\hat{\t}<\t^*$). An implication is that additional immigration improves welfare whenever the labor market is  inefficiently tight ($\t > \t^*$)---even if some immigration has already been allowed.
\label{p:implicit}\end{proposition}

Equation \eqref{e:fixedpoint} implicitly defines the optimum immigration, just like sufficient-statistic formula do \citep{C09}. But it nevertheless offers interesting insights. The most important is that if immigration is a tool available to policymakers, and if policymakers aim to maximize native welfare, then the labor market should always be inefficiently slack. Otherwise immigration is suboptimal: more immigration is required to cool the labor market down. 

What is the intuition for the result that optimal immigration brings the labor market to a slack situation? At the efficient tightness, by definition, a drop in tightness reduces labor income, but this reduction is exactly offset by an increase in profits, so output (the sum of labor income and profits) is unaffected. As immigration increases, the share of labor income going to native workers shrinks, while all profits continue to go to native firm owners. So as immigration increases, the profit motive plays an increasingly large role in welfare. This means that at the efficient tightness $\t^*$, a decrease in tightness raises native profits more than it reduces labor income---which means that a drop in tightness raises welfare. Accordingly, it is optimal to allow some more immigration so as to reduce tightness further to the slack territory, $\hat{\t}<\t^*$.

Finally, the solution of the model $\t(\m,w/a)$ is strictly decreasing in the wage-to-productivity ratio $w/a$ but strictly decreasing in immigration $\m$, and $\t \mapsto u(\t)/\tau(\t)$ is strictly decreasing in $\t$. Implicit differentiation of \eqref{e:fixedpoint} therefore gives the following:
\begin{proposition}
When the labor market is initially hotter (lower $w/a$), then the optimal amount of immigration is larger (higher $\hat{\m}$). The resulting labor market, after immigration, is slacker (lower $\hat{\t}$).
\label{p:cyclicality}\end{proposition}

\section{Discussion}\label{s:discussion}

The model predicts the impact of immigration in the short run on two constituencies--- workers and firm owners---as well as the overall impact of immigration on native welfare. From this I conjecture the political support that immigration might receive under different circumstances, and discuss some policy implications. I also discuss other labor-supply shocks that the model could be applied to.

\subsection{Conditional support for immigration in egalitarian regimes} 

Let's first consider egalitarian regimes---in which workers have access to firm profits. These could be regimes in which workers own all firms and therefore receive their profits, or regimes in which profits are fully taxed and redistributed to workers. In such regimes, there are no tensions: policy would aim to maximize welfare. Such regime would therefore be favorable to some immigration in good times, when the labor market is inefficiently tight, and would be against it in bad times, when the labor market is inefficiently slack. 

In addition, the model predicts that such regimes would allow more immigration when the labor market is initially tighter, and less immigration when the labor market is initially closer to efficiency. 

\subsection{Opposition to immigration in unequal and populist regimes} 

Let's consider next unequal regimes---in which workers do not have access to firm profits. Let's focus further on populist regimes, that base their policy decisions solely on the welfare of workers. Since immigration always hurts the welfare of native workers, I would expect such regimes to always oppose immigration. 

Moreover, since the elasticity of the employment rate with respect to the labor force is more negative in bad times, when tightness is lower, I would expect opposition to immigration and accusations of job stealing to be louder in bad times.

\subsection{Support for immigration in unequal and capitalist regimes}

Let's finally look at another type of unequal regimes: capitalist regimes that base their policy decisions solely on the welfare of firms. Since immigration always improves the profits of firms, I would expect such regimes to always favor immigration.

\subsection{Immigration after the coronavirus pandemic}

In the United States the labor market has been incredibly tight in the recovery from the coronavirus pandemic \citep{MS23}. Labor market tightness reached a value of $2$ in 2022, a level it had not reached since the end of World War 2. The unemployment rate fell as much as $1.6$ percentage points below the efficient unemployment rate---again something that had not happened since the end of World War 2.

A natural response to such inefficiently tight labor market is to tighten monetary policy, which the Fed did in 2022Q2, one year after the labor market turned inefficiently tight. However, the labor market has been slow to cool in 2022, maybe because monetary policy percolates only slowly to the labor market. 

Given such delays in deciding to tighten monetary policy, and then delays for monetary policy to reach the labor market, allowing for some immigration between 2021Q2 and the end of 2022 would have rapidly cooled the labor market and improved the welfare of natives. 

\subsection{Other labor-supply shocks}

This paper uses the model to study one specific type of labor-supply shock: a wave of immigration that triggers an increase in the size of the labor force. But there is nothing special about immigration in the model. It is just a sudden change in the size of the labor force. Any other changes in the size of the labor force could be analyzed with the same model.

For instance, \citet{MiS21,MS23} find that labor market tightness is particularly elevated during World War 2, the Korean War, and the Vietnam War in the United States. This model explains why. Part of the reason, described in \citet{M14}, is that the government spends and hires a lot during wars, which boosts labor demand and increases tightness. Another part of the reason, which is the focus of this paper, is that millions of potential labor force participants were sent abroad on military duty. As this paper shows, such drastic reduction in labor force will lead to elevated labor market tightness and reduced unemployment rate among the workers who stayed in the United States.

Another example of a drastic drop in labor-force participation and subsequent elevated labor-market tightness is the recovery from the coronavirus pandemic in the United States. The pandemic triggered a sharp drop in labor-force participation, from 63.3\% in February 2020 down to 60.1\% in March 2020, which had not completely subsided in February 2023 \citep{CIVPART}. At the same time, labor-market tightness reached levels not seen since the end of World War 2 \citep{MS23}. Part of the reason might be the elevated aggregate demand generated by the fiscal stimulus during the pandemic. But another possible reason is that a reduction in labor force leads to higher labor market tightness, as described in this paper.


\bibliography{\bib}
\end{document}