\documentclass[letterpaper,11pt,leqno]{article}
\usepackage{paper,math}
\pdfoutput=1
\hypersetup{pdftitle={Modeling Migration-Induced Unemployment}}
\available{https://pascalmichaillat.org/14/}
\newcommand{\bib}{paper.bib}
\newcommand{\pdf}{figures.pdf}

\begin{document}

\begin{titlepage}
\title{Modeling Migration-Induced Unemployment}
\author{Pascal Michaillat
\thanks{University of California--Santa Cruz. I thank Avidit Acharya, Christoph Albert, Leah Boustan, Varanya Chaubey, Evgeniya Duzhak, Anthony Edo, Yagan Hazard, Claudio Labanca, Michael Lachanski, Carlo Medici, Joan Monras, Emmanuel Saez, Paola Sapienza, Stefanie Stantcheva, Marco Tabellini, Jonathan Vogel, Romain Wacziarg, and Josef Zweimueller for helpful comments.}}
\date{December 2024}
\maketitle

Immigration is often blamed for increasing unemployment among local workers. This sentiment is reflected in the rise of anti-immigration parties and policies in Western democracies. And in fact, numerous studies estimate that in the short run, the arrival of new workers in a labor market raises the unemployment rate of local workers. Yet, standard migration models, such as the Walrasian model and the Diamond-Mortensen-Pissarides model, inherently assume that immigrants are absorbed into the labor market without affecting local unemployment. This paper presents a more general model of migration that allows for the possibility that not only the wages but also the unemployment rate of local workers may be affected by the arrival of newcomers. This extension is essential to capture the full range of potential impacts of labor migration on labor markets. The model blends a matching framework with job rationing. In it, the arrival of new workers raises the unemployment rate among local workers, particularly in a depressed labor market where job opportunities are limited. On the positive side, in-migration helps firms fill vacancies more easily, boosting their profits. The overall impact of in-migration on local welfare varies with labor market conditions: in-migration reduces welfare when the labor market is inefficiently slack, but it enhances welfare when the labor market is inefficiently tight.

\end{titlepage}\section{Introduction}
 
People are always worried about immigrants taking their jobs \citep{J01,S10}. Not surprisingly, then, immigration is always one of people's chief policy concerns \citep{CDP12,LJ23}. In fact, since the 1970s, specifically anti-immigration political parties have appeared and started to grow in Western countries, now attracting 11\% of votes on average across these countries \citep[p.~22]{GMP22}. And far-right parties with anti-immigration platforms have become increasingly popular across European countries, all the more so in countries and at times in which the share of immigrants in the population is larger \citep[figure~1]{HWZ17}. 

People's worries might not be completely unfounded. A substantial number of studies estimate that in the short run, the arrival of new workers in a labor market raises the unemployment rate of local workers \citep{H92,AK03,BFK10,G12,DSS17,BM17,M17a,L20,ABC21}. On average across 11 migration experiments, the arrival of 100 new workers in a labor market pushes 37 local workers into unemployment.

Yet, by assumption, standard models of migration rule out the possibility that the unemployment rate of local workers may be affected by newcomers. When migration takes place in a Walrasian labor market, there is no unemployment at all: anybody who wants to work can work. In a \citet{D82}-\citet{M82}-\citet{P85} (DMP) labor market, there is some unemployment, but labor demand is perfectly elastic. Thus, newcomers are absorbed into the labor market without affecting other workers. Hence, the standard models cannot capture the phenomenon that newly arrived workers compete for jobs with local workers and make it more difficult for them to find employment. 

Because they rule out the possibility that the local unemployment rate may be affected by migration, standard models cannot describe and explain the short-run response of unemployment to migration. They cannot predict when such response is likely to be severe and when it is likely to be mild. And they cannot be used to design compelling migration policies because they rule out what is in the public's mind one the most deleterious effects of migration. In particular, they cannot say whether migration policies should respond to the business cycle or the state of the local economy because they assume that jobs are always plentiful.

This paper therefore presents a model that allows for migration-induced unemployment. In the labor-market model from \citet{M12}, which adds job rationing to the DMP model, the entry of migrants reduces labor-market tightness and therefore the job-finding rate of local workers. As a result, the arrival of migrants increases the unemployment rate of local workers---who might naturally feel that migrants take their jobs away. 

The mechanism is as follows. In normal times, the number of jobs available is somewhat limited. When new workers enter the labor force, the number of jobseekers increases, so labor-market tightness drops. This makes hiring more attractive to firms, so they hire some more workers---some local and some newly arrived. While this additional hiring boosts labor-market tightness, tightness always remains below its initial level because the number of jobs available is not sufficient to absorb all newcomers. Since labor-market tightness is lower, the prevailing job-finding rate is lower, the unemployment rate goes up, and the employment rate goes down. That is, fewer local workers hold jobs. This mechanism is consistent with evidence from Germany that the local unemployment rate increases because of a reduced inflow of local workers into employment, not increased outflow of local workers from employment \citep{DSS17}.

In bad times, the lack of jobs is more stringent. Thus, firms absorb fewer of the newcomers. More of the newcomers remain jobless, which increases the competition for jobs. As a result, local workers are more negatively affected by in-migration in bad times. Formally, the elasticity of employment with respect to in-migration is more negative in bad times. This might explain why the backlash against immigration is stronger when the unemployment rate is elevated \citep{HWZ17}. 

While in-migration hurts workers, it helps firms. This is because in-migration reduces labor-market tightness, so it makes it easier for firms to recruit workers. Easier recruiting in turn improves firm profits.

The overall effect of in-migration on local welfare---the sum of local labor income and local firm profits---depends on the health of the labor market. When the labor market is inefficiently slack, allowing in-migration always reduces local welfare. But in-migration improves local welfare when the labor market is inefficiently tight. In that case, the local labor income lost from in-migration is more than offset by the increase in firm profits. 

In the model calibrated to the US labor market, the amount of migration-induced unemployment is substantial and commensurate to the amount observed in migration experiments. For instance, 33 local workers become unemployed when 100 migrants arrive in their labor market and labor-market tightness is at its efficient level of 1. When labor-market tightness is 0.5, corresponding to a somewhat slack labor market, the number of local workers who become unemployed raises to 49. When labor-market tightness is 0.1, corresponding to an extremely slack labor market, the number of local workers who become unemployed raises further to 73. By contrast, when labor-market tightness is 2, corresponding to a very tight labor market, the number of local workers who become unemployed falls to 20. 

The model presented in this paper offers a potential way to bridge the current divide in the immigration literature. A first branch of the literature argues that the labor demand is downward-sloping so the number of jobs available in the labor market is somewhat limited \citep{B03}. Under this perspective, the arrival of immigrants reduces the opportunities available to local workers. A second branch of the literature argues that the arrival of immigrants does not reduce the wage of local workers \citep{C90}. Under this perspective, the arrival of immigrants does not affect local workers who are employed. This paper's model features a downward-sloping labor demand, so local jobs are somewhat limited, and rigid wages, so competition between local and immigrant workers does not necessarily reduce local wages. The novelty is that competition between local workers and immigrants is reflected not only in wages but also in labor-market tightness, which itself determines workers' job-finding rate and unemployment rate. So immigration may adversely affect local workers even when it does not affect local wages---because it raises the local unemployment rate and adversely affects local jobseekers. In this way, the model may help align insights from both the Borjasian and Cardian perspectives on immigration.

In the long run, immigration has significant positive effects for the economy at large, in particular through increased innovation \citep{HG10,MVW14,PSS15,JS18,SNQ20,BDJ22,TCB24}.\footnote{For a review of the literature on high-skill immigration and innovation, see \citet{K17}.} However, political backlash is typically generated by short-run and local effects. My paper explains how immigration can generate unemployment in the short run in local labor markets. The paper might therefore help to understand public anxieties about immigration, which often center on the idea that immigrants take jobs away from local workers. In that, the paper adds to existing explanations for anti-immigrant sentiments. One such explanation is that people hold misbeliefs about immigrants: they overestimate the number of immigrants already in their community, they exaggerate their cultural and religious differences with immigrants, and they underestimate immigrants' socioeconomic status \citep{AMS23}.

\section{Evidence of migration-induced unemployment}\label{s:evidence}

This section presents evidence of migration-induced unemployment from natural and quasi experiments. The experiments correspond to events in which newcomers arrive into local labor markets for exogenous reasons, often because they were forced to migrate by external events such as wars or political upheavals \citep{BF19}. The experiments occurred across different countries and historical periods, and they involved both international and domestic migrations, but they paint a consistent pattern: an influx of new workers into a local labor market raises the unemployment rate among local workers. The evidence is summarized in table~\ref{t:evidence}.

\begin{sidewaystable}[p]
\caption{Migration-induced unemployment across countries and periods}
\begin{tabular*}{\textwidth}{@{\extracolsep{\fill}}p{2.9cm}cccc}\toprule
Unemployed locals per 100 arrivals & Country & Period & Event & Reference\\
\midrule
\multicolumn{5}{c}{A. Domestic migration}\\
21 & United States & 1935--1940 & Great Depression & \citet{BFK10} \\
\midrule
\multicolumn{5}{c}{B. Return migration}\\
20 & France & 1962 & Algerian War of Independence & \citet{H92} \\
22 & Portugal & 1974--1977 & Independence of Mozambique \& Angola & \citet{M17a} \\
31 & Germany & 1996--2001 & Fall of Berlin Wall & \citet{G12} \\
\midrule
\multicolumn{5}{c}{C. International migration}\\
24--27 & Algeria $\to$ France & 1962 & Algerian War of Independence & \citet{BM17} \\
15 & Cuba  $\to$ United States & 1980--1981 & Mariel Boatlift & \citet{C90} \\
66--77  & Czech Republic $\to$ Germany & 1991--1993 & Fall of Berlin Wall & \citet{DSS17} \\
30--40 & Eastern Europe $\to$ West Germany & 1987--2001 & Fall of Berlin Wall & \citet{DOP10} \\
35--83 & Yugoslavia $\to$ Europe & 1983--1999 & Yugoslav Wars & \citet{AK03} \\
21--47 & Yugoslavia $\to$ Europe & 1991--2001 & Yugoslav Wars & \citet{BM17} \\
63--80 & North Africa $\to$ Italy & 2011 & Arab Spring & \citet{L20} \\
\bottomrule
\multicolumn{5}{c}{Summary statistics for the number of unemployed locals per 100 arrivals}\\
 Mean: 37 & Median: 31 & Minimum: 15 & Maximum: 72 & Standard deviation: 20 \\
\bottomrule\end{tabular*}
\note{The evidence described in the table is obtained in Section~\ref{s:evidence}. For studies that provide a range of estimates, the summary statistics use the midpoint of the range.}
\label{t:evidence}\end{sidewaystable}

\subsection{US Great Depression}

\citet{BFK10} studies the effect of internal migration in the United States during the Great Depression. Just like native workers protest immigration from abroad, locals in areas where the depression was less severe protested against the arrival of migrants from other less fortunate regions. Locals accused newcomers of taking jobs away from them. For example, Californians tried to scare possible migrants away from the state. A billboard in Oklahoma carried the message ``NO JOBS in California / If YOU are looking for work—KEEP OUT / 6 men for every job / No state relief available for non-residents'' \citep[p.~720]{BFK10}.

To assess the impact of internal migrants on existing residents, \citet{BFK10} use variation in the generosity of New Deal programs and extreme weather events to instrument for migrant flows to and from US cities. They find that in-migration had little effect on the hourly earnings of existing residents, but they prompted some residents to move away from their city and others to lose weeks of work or access to relief jobs. They estimate that for every 100 arrivals, 21 locals were displaced from relief jobs. An additional 19 local workers were forced to shift from full-time to part-time work. And a further 19 residents moved out to other cities. They also find that, just as in-migration diminished the work opportunities of local workers, out-migration improved local opportunities symmetrically.

\subsection{Algerian War of Independence}

\citet{H92} examines the repatriation to France of Algerians of European origin following the independence of Algeria in 1962. \citet[p.~566]{H92} finds that the arrival of 100 repatriates in the labor force pushed 20 natives into unemployment.

\citet{BM17} look at Algerian refugees who fled Algeria for France at the same time as European repatriates. They aim to estimate the impact of immigration in individual region-education cells of the French labor market. For these refugees, \citep[table~10]{BM17} find larger effects than for the repatriates: 24--27 natives were pushed into unemployment for every 100 refugees in any region-education cell.

\subsection{Independence of Mozambique and Angola}

\citet{CL96} and \citet{M17a} study the return of half a million workers from Mozambique and Angola to Portugal in 1974--1976, following the independence of Mozambique and Angola. This return led to a 15\% increase in the size of the Portuguese civilian labor force. \citet[table~2]{CL96} find that in 1981, returnees faced an unemployment rate of 14\%, much higher than the unemployment rate of 6\% faced by locals. This elevated unemployment rate indicates that it was difficult for returnees to get absorbed by the Portuguese labor market. Moreover, \citet[p.~248]{M17a} finds that in 1977, after the wave of migration subsided, the Portuguese unemployment rate was 3.3pp higher than in a counterfactual outcome. This means that for each 100 returnees into the labor force, $3.3/15 \times 100 = 22$ additional local Portuguese workers were facing unemployment in 1977.

\citet[p.~242]{M17a} also reports that Portuguese workers were concerned about the return of workers from the African colonies, especially because the Portuguese economy was slumping at that time. The returnees were accused of ``stealing housing and jobs'' from Portuguese residents.

\subsection{Mariel Boatlift}

Interestingly, the results from \citet{C90}'s famous study are not inconsistent with migration-induced unemployment. \citeauthor{C90} studies the impact of the Cuban immigrants from the Mariel Boatlift on the Miami labor market in the 1980s. A first, well-known finding from the study is that ``the Mariel immigration had essentially no effect on the wages or employment outcomes of non-Cuban workers in the Miami labor market'' \citep[p.~255]{C90}. A second finding is that ``perhaps even more surprising, the Mariel immigration had no strong effect on the wages of other Cubans'' \citep[p.~255]{C90}.

A third, less-known finding is that the unemployment rate for Cuban workers increased drastically: ``Unlike the situation for whites and blacks, there was a sizable increase in Cuban unemployment rates in Miami following the Mariel immigration. Cuban unemployment rates were roughly 3 percentage points higher during 1980-81 than would have been expected on the basis of earlier (and later) patterns'' \citep[p.~251]{C90}. The fact that the Cuban workers faced a higher unemployment rate is evidence that the labor market could not absorb all new arrivals, and that jobs were somewhat rationed. In fact, the boatlift raised the Cuban labor force in Miami by 20\% \citep[p.~246]{C90}. So for each 100 new arrivals into Miami's Cuban labor force, $3/20 \times 100= 15$ local Cubans were pushed into unemployment.

Relatedly, \citet[figure~5]{ABC21} compute the response of labor-market tightness in Miami upon the arrival of Cuban immigrants from the Mariel Boatlift in the 1980s. Relative to the tightness in comparable cities, they find that Miami tightness fell by 40\% after the Mariel Boatlift. This implies that it became much harder for local workers to find jobs after the boatlift.

\subsection{Fall of the Berlin Wall}

\citet{G12} studies the return of 2.8 million ethnic Germans to Germany following the fall of the Berlin Wall. He estimates that for every 100 immigrants that found a job, 31 locals became unemployed, with no effect on wages. \citet{CV08} finds a similar number in the same empirical context.

\citet[pp. 462, 475, 476]{DSS17} estimate the impact of Czech commuters who were allowed to work in German border towns in 1991--1993, just after the fall of the Iron Curtain. They find evidence of migration-induced unemployment again. For each 100 commuters that found a job, 66--77 German workers were relegated to unemployment. Some further 16--27 German workers relocated to towns further away from the border, where Czech commuters did not compete for jobs.

A natural question that arises is whether local employment falls because locals are fired at a higher rate, or because they are hired at a slower rate. \citet[p.~438]{DSS17} find that the increase in unemployment for German workers was caused by reduced inflows into employment---not increased outflows from employment. So German workers did not lose their jobs at a higher pace when Czech commuters arrived, but it became harder for German workers who did not have a job to find one.

Relatedly, \citet{DOP10} estimate how West German workers were impacted by the arrival of ethnic Germans, East Germans, and foreigners into Germany in 1987--2001. They find that the arrival of new immigrants did not affect natives much, but it negatively affected previous immigrants, who were already working in West Germany. \citet[p.~559]{DOP10} find that for 100 new immigrants who found a job, 30--40 previous immigrants were pushed into unemployment.

\subsection{Yugoslav Wars}

\citet{AK03} look at migration from the former Yugoslavia into other European countries in the 1990s, because of the Yugoslav Wars. With an OLS specification, \citet[p.~F318]{AK03} find that the entry of 100 refugees in the labor force pushed 35 native workers into unemployment. With an IV specification, \citet[p.~F322]{AK03} then find that the entry of 100 refugees in the labor force pushed 83 native workers into unemployment.

\citet{BM17} revisit the evidence using a region-education-cell strategy. They observe again migration-induced unemployment, although their results are weaker than the original results. With an OLS specification, \citet[table~13]{BM17} find that 21 natives were pushed into unemployment when 100 refugees entered a region-education cell. With an IV specification, the number of natives who were pushed into unemployment rises to 47.

\subsection{Arab Spring}

Finally, \citet{L20} examines the arrival of Arab Spring refugees into the Italian labor market in 2011. He finds that for every 100 refugees employed, 63--80 Italian workers became unemployed.

\subsection{Nonexperimental evidence}

Beside the experimental evidence described in this section, a few papers provide nonexperimental evidence that the arrival of immigrants on a labor market reduces the employment rate of local workers. 

\citet{C01} uses the 1990 US Census to study the effects of immigration on occupation-specific labor market outcomes. He finds that occupation-specific wages and employment rates are systematically lower in cities with higher relative supplies of workers in each occupation. Hence, immigration over the 1980s reduced wages and employment rates of low-skill natives in gateway cities like Miami and Los Angeles.

\citet{BGH07} use the 1960--2000 US Censuses to examine the correlation between immigration, black wages, and black employment rates. They find that when 100 immigrants arrived in the labor market for a particular skill, 35 black workers were pushed out of employment. They also find that the wage of employed black workers fell, and the incarceration rate of black men increased.

\citet{E15} uses microdata for France, 1990--2002, and finds that in the short run, immigrants increased the unemployment rate of native workers but did not reduce their wages.

\citet{L15} uses provincial panel data to examine the impact of permanent immigration on the unemployment rate in Canada. He finds that in the short run, immigration causes a raise in the unemployment rate. He also finds that this effect disappears in the long run. 

Finally, \citet{D23,D24} uses US state-level data to study the response of labor-market tightness to immigration. She finds that the slowdown in immigration between 2017 and 2021 raised the tightness of local labor markets. In 2022 and 2023, the rebound in immigration helped lower labor-market tightness (which was inefficiently high).

\subsection{Discussion}

There is ample evidence of migration-induced unemployment in natural and quasi experiments. Table~\ref{t:evidence} reports the number of local workers who became unemployed for each 100 migrant workers who entered their local labor market. Across the 11 migration experiments, 100 new arrivals in a labor market pushes on average 37 local workers into unemployment. There is substantial variation in the amount of migration-induced unemployment across experiments. The number of local workers pushed into unemployment per 100 arrivals ranges from 15 to 72, with a standard deviation of 20. Nevertheless, it is undeniable that in-migration raises local unemployment.

Yet, academic economists tend to reject the notion that in-migration negatively affects local workers \citep{RR07}. \citet[p.~302]{FHK06} observe for instance:
\begin{quote}
One of the central questions in the debate over immigration policy is whether immigrants adversely affect labor market outcomes for natives. Some Americans believe they do, worrying that immigrants take jobs away from native workers. Most of the empirical evidence produced by economists, however, does not support these concerns.
\end{quote}

The general perception by academic economists that immigration has no adverse effect on local workers might be an indirect consequence of not having models to think about the effect of migration on unemployment. If such phenomenon was present in standard models, researchers might have paid more attention to the empirical findings in existing studies. Indeed, one of the main functions of models is to guide and stimulate empirical work about the phenomena that they describe \citep{K57}. 

Because existing models only allow migration to affect wages, the empirical literature has focused on wages, which do not appear to respond much to immigration. This has created a puzzle: ``The depth of public concern over immigration is somewhat puzzling, given that most studies find only small economic impacts on the native population'' \citep[p.~78]{CDP12}. The adverse impact of immigration on unemployment is not emphasized at all in economic research---but it is often there, offering a resolution of the puzzle.

The absence of migration-induced unemployment in existing migration models severely limits the scope of empirical inquiry about migration. A wonderful example of such limitation appears in a seminal paper by \citet{SS01}. The paper documents people's perceptions about immigration. \citet[footnote~8]{SS01} candidly admit that they do not examine people's perception of job stealing because such perceptions would be inconsistent with the effects of immigration in standard models:
\begin{quote}
The 1992 National Election Studies survey asked other questions about immigration that we do not analyze. For example, respondents were asked whether they think Asians or Hispanics `take jobs away from people already here.' We do not focus on this question because its responses cannot clearly distinguish among our three competing economic models. All our models assume full employment, so no natives could have jobs `taken away' by immigrants.
\end{quote}

The response of respondents in the 1992 National Election Studies is presented in table~\ref{t:stealing}. More than 80\% of respondents are indeed worried that Hispanic and Asian immigrants take jobs away from them. It is quite possible that the existence of migration-induced unemployment has not received the attention it deserves just because standard models do not feature it.

\begin{table}[t]
\caption{Migration-induced unemployment in popular perceptions}
\begin{tabular*}{\textwidth}{@{\extracolsep{\fill}}p{6cm}*{4}{c}}
\toprule
 &   \multicolumn{4}{c}{How likely is it?}\\
\cmidrule{2-5}
The growing number of these immigrants takes jobs away from people already here. &  Extremely &  Very &  Somewhat & Not at all \\
\midrule
Hispanics 	& 20\% & 29\% & 38\% & 13\% \\
Asians	 	& 19\% & 30\% & 37\% & 13\% \\
\bottomrule\end{tabular*}
\note[Source]{1992 American National Election Studies survey. The responses to the question about Hispanic workers (question Q4c) can be accessed at \url{https://electionstudies.org/data-tools/anes-variable/variable.html?year=1992&variable=V926238}. The responses to the question about Asian workers (question Q5c) can be accessed at \url{https://electionstudies.org/data-tools/anes-variable/variable.html?year=1992&variable=V926241}.}
\label{t:stealing}\end{table}

Even to establish empirically that migration has zero effect on unemployment, it is useful to have a model in which migration may affect not only wages but also tightness and unemployment. The model describes the mechanism through which migration may affect unemployment in the short run. The model also predicts when the effect of migration on unemployment is likely to be severe and when it is likely to be mild. Regardless of the eventual findings, the model will be useful to empiricists who want to establish whether migration induces unemployment, because it explains through which channels and under which conditions migration may induce unemployment.

\section{Model of the labor market}\label{s:model}

To formulate a theory of migration that makes sense of the evidence presented in section~\ref{s:evidence}, I use a model in which not only wages but also labor-market tightness might respond to migration. The model, developed by \citet{M12}, inserts job rationing into the DMP matching model. This is done by assuming that the production function is concave instead of linear, and that wages are determined by a rigid wage norm instead of Nash bargaining.\footnote{Previous papers had modified the DMP model with a concave production function \citep{S99,CW01a,EM13,AH14} or a rigid wage norm \citep{S04,H05,BG10}, but only the combination of the two assumptions yields job rationing.} The modified model features not only frictional but also rationing unemployment. Introducing rationing unemployment is required to generate migration-induced unemployment. Indeed, if jobs were not rationed, all newcomers would simply be absorbed by firms.

\subsection{Assumptions}

The labor market is composed of a mass 1 of firms and a labor force of size $H$. The rate at which unemployment workers are hired by firms is given by a Cobb-Douglas matching function: 
\begin{equation}
h(U,V) = \m \cdot U^{\h} \cdot V^{1-\h},
\label{e:matching}\end{equation} 
where $U$ is the number of unemployed workers, $V$ is the number of vacant jobs, and $\h\in (0,1)$ is the elasticity of matching with respect to unemployment.

In the matching model wages are determined in a situation of bilateral monopoly so it is a wage norm and not an auctioneer that determines wages \citep{HM87,H05,MS15}. The advantage is that the assumed wage norm can be shaped by evidence. Immigration does not seem to generate quantitatively important effects on local wages or the wage distribution \citep{FH95,OP12,C12a,BK15}. Hence, I assume that all workers are paid a fixed real wage $w>0$, which in particular does not respond to migration.\footnote{The model describes the impact of migration when all workers have the same productivity and are paid the same wage. \citet{A21} studies an additional mechanism that operates when newcomers accept lower wages than locals. Firms post vacancies, collect applications, and pick the worker who accepts the lowest wage among all applicants. If a newcomer and a local both apply to the same job, the firm picks the cheaper newcomer over the more expensive local. Through such wage competition, the arrival of new workers on a labor market increases the unemployment rate of locals. \citeauthor{A21} finds that this mechanism operates with illegal immigrants, who accept much lower wages than local workers, but not with legal immigrants, because they are paid roughly the same as local workers.}

Another advantage of assuming a fixed wage is that it allows us to concentrate on the response of tightness to migration---which is the key innovation of this paper. Once the response of tightness is well understood, however, I will generalize the wage norm and allow it to respond to migration (equation \eqref{e:wage}). Then I will repeat the analysis with this generalized wage norm (section~\ref{s:wageresponse}). It turns out that the qualitative results are the same whether wages are completely fixed or partially rigid, as long as wages are not completely flexible in response to migration. Hence, all the results in sections \ref{s:firms}--\ref{s:politics} will be derived under general wage norm \eqref{e:wage}.

Next, firms have a concave production function 
\begin{equation}
y(P) = a P^{1-\a},
\label{e:production}\end{equation} 
where $a$ governs labor productivity, $P$ denotes the number of producers in the firm, and $\a \in (0,1)$ governs decreasing returns to labor. Why might the production function exhibit decreasing instead of constant returns to labor? A basic reason is that in the short run, capital and other factors of production are fixed. For instance, if in the long run production is given by a Cobb-Douglas function of labor and capital, $y(K,P) = b K^{\a} P^{1-\a}$, but capital $K$ is fixed in the short run, then in the short run the production function takes the form \eqref{e:production}.

Firms also incur a recruiting cost of $\k > 0$ recruiters per vacancy and face a job-destruction rate $\l > 0$. The total number of recruiters in the firm is $R = \k V$ and the total number of workers is $L = R + P$.

\subsection{Matching rates}

Workers match with firms at a rate $f(\t)$ given by
\begin{equation}
f(\t) = \frac{h(U, V)}{U} =  h(1,\t) = \m \t^{1-\h}.
\label{e:f}\end{equation}
Vacancies are filled with workers at a rate $q(\t)$ given by
\begin{equation}
q(\t)= \frac{h(U, V)}{V} = h(\t^{-1},1) = \m \t^{-\h}.
\label{e:q}\end{equation}
The job-finding rate $f(\t)$ is increasing in $\t$, and the vacancy-filling rate $q(\t)$ is decreasing in $\t$. Hence, when tightness is higher, it is easier for workers to find  jobs but harder for firms to find workers. Moreover, the elasticity of the job-finding rate with respect to tightness is $\e^f_{\t} = \oex{f}{\t} = (1-\h)$, and the elasticity of the vacancy-filling rate with respect to tightness is $\e^q_{\t} = \oex{q}{\t} = -\h$.

\subsection{Labor supply}

In the matching framework, the employment level is given by the following law of motion:
\begin{equation}
\dot{L}(t) = f(\t) U(t) - \l L(t).
\label{e:ldot}\end{equation} 
That is, the employment level increases over time ($\dot{L}>0$) if more jobseekers find jobs than employed workers lose their jobs ($f(\t) U > \l L$). Conversely, employment decreases over time if more employed workers lose their jobs than jobseekers find jobs.

Since $U(t) = H - L(t)$, the law of motion \eqref{e:ldot} can be rewritten as the following differential equation:
\begin{equation*}
\dot{L}(t) = f(\t) H - \bs{\l+f(\t)} L(t).
\end{equation*} 
The critical point of this differential equation is
\begin{equation}
L = \frac{f(\t)}{\l+f(\t)} H.
\label{e:l}\end{equation}
This positive relationship between employment and tightness is the locus of unemployment and tightness such that the number of new employment relationships created at any point in time equals the number of relationships dissolved at any point in time. It is the locus of points such that the employment level remains constant over time and labor-market flows are balanced. It is also isomorphic to the Beveridge curve.

The employment levels given by equations~\eqref{e:ldot} and~\eqref{e:l} are indistinguishable (\citealt[pp.~398--399]{H05a}; \citealt[p.~236]{P09a}). In fact, \citet[p.~31]{MS21b} find that when $\l$ and $f(\t)$ are calibrated to US data, the deviation between the two employment levels decays at an exponential rate of 62\% per month. This implies that about 50\% of the deviation evaporates within a month, and about 90\% within a quarter.

I therefore assume that at the time scale of the model, labor-market flows are always balanced, and the employment level is given by equation~\eqref{e:l} at all times. This assumption is akin to the assumption that people are always rational in macroeconomic models, while neglecting the learning period that it takes for people to converge to a rational behavior. It is also akin to the assumption that people always know Nash or other equilibria in game theory, while neglecting the learning or coordination period that it takes to reach such equilibria.

The employment level consistent with balanced flows is the labor supply:
\begin{equation}
L^s(\t,H) = \frac{f(\t)}{\l+f(\t)}H.
\label{e:ls}\end{equation}
In the model, the labor supply holds at any point in time. The labor supply links employment to tightness and labor force. From the labor supply I can also relate the unemployment rate to tightness. The unemployment rate is $u = (H-L^{s})/H = 1 - L^{s}/H$ so
\begin{equation}
u(\t) = \frac{\l}{\l+f(\t)}.
\label{e:u}\end{equation}

The elasticity of labor supply with respect to tightness satisfies
\begin{equation*}
\e^{s}_{\t} = \oe{L^s}{\t} = \e^{f}_{\t} - \frac{f(\t)}{\l+f(\t)} \e^{f}_{\t} = \bp{1-\h} - \frac{f(\t)}{\l+f(\t)}\bp{1-\h},
\end{equation*}
so it simplifies to
\begin{equation}
\e^{s}_{\t} = \bp{1-\h} u(\t),
\label{e:est}\end{equation}
where $u(\t)$ is the unemployment rate along the labor supply, given by \eqref{e:u}.
The elasticity of labor supply with respect to the labor force is
\begin{equation*}
\e^{s}_H = \oe{L^s}{H} = 1.
\end{equation*}

Since the unemployment rate satisfies $u(\t) = 1 - L^s(\t)/H$, the elasticity of the unemployment rate with respect to tightness is
\begin{equation*}
\e^{u}_{\t} = \oe{u}{\t} = \frac{-L^{s}/H}{1-L^{s}/H}\cdot \e^s_{\t} = -\frac{1-u(\t)}{u(\t)}\cdot \e^s_{\t}.
\end{equation*}
Using the expression \eqref{e:est} for $\e^s_{\t}$ then yields
\begin{equation}
\e^{u}_{\t} =  - \bp{1-\h}\bs{1-u(\t)}.
\label{e:eut}\end{equation}

\subsection{Recruiter-producer ratio}

Because it takes time to fill vacancies and each vacancy requires the attention of a recruiter, firms must allocate a share of their workforce to recruiting. And because the model is cast on a time scale where labor-market flows are balanced, firms post vacancies to maintain their firm at a given desirable size. That is, they post $V$ vacancies, so the number of new hires $q(\t)V$ just replaces the number of workers who left the firm $\l L$. 

Multiplying $\l  L = q(\t)V$ by the recruiting cost $\k$, and using $L = R+P$ and $R = \k V$, yields $\l \k (P+R) = q(\t) R$. Dividing both sides by $R$ then gives $\l \k (1+\tau(\t)^{-1}) = q(\t)$, where $\tau(\t) = R/P$ is the recruiter-producer ratio. The recruiter-producer ratio is therefore given by
\begin{equation*}
\tau(\t) = \frac{\l \k}{q(\t)-\l \k}.
\end{equation*}
This means that
\begin{equation*}
1+\tau(\t) = \frac{q(\t)}{q(\t)-\l \k}.
\end{equation*}
The recruiter-producer ratio $\tau(\t)$ is positive and increasing on $[0,\t_{\tau})$, where $\t_{\tau}$ is defined by $q(\t_{\tau}) = \l \k $; furthermore, $\tau(0)=0$ and $\lim_{\t\to \t_{\tau}} \tau(\t) = + \infty$.

The wedge $1+\tau(\t)$ plays an important role in the analysis because it determines the gap between numbers of employees and producers in the workforce: 
\begin{equation}
L = P + R = \bs{1+\tau(\t)} p.
\label{e:lnr}\end{equation}
The elasticity of $1+\tau$ with respect to $\t$ satisfies
\begin{equation*}
\e^{1+\tau}_{\t} = \e^{q}_{\t} - \e^{q-\l \k}_{\t} =-\h - \frac{q(\t)}{q(\t)-\l \k} (-\h),
\end{equation*}
so it simplifies to
\begin{equation}
\e^{1+\tau}_{\t} = \h\tau(\t).
\label{e:etaut}\end{equation}

\subsection{Labor demand}

Firms operate within the balanced-flow paradigm. By choosing how many vacancies to post, they determine their workforce, which they in turn choose to maximize flow real profits: 
\begin{equation*}
y(P) - wL = y(P) - \bs{1+\tau(\t)} w P. 
\end{equation*}
The first-order condition of the firm's maximization problem is $y'(P) - \bs{1+\tau(\t)} w = 0$, or
\begin{equation}
(1-\a) a P^{-\a} =\bs{1+\tau(\t)} w.
\label{e:foc}\end{equation}
This condition says at the optimum, the marginal product of labor equals the marginal cost of labor.

With \eqref{e:lnr}, the first-order condition \eqref{e:foc} becomes
\begin{equation*}
(1-\a) a \bs{1+\tau(\t)}^{\a}{L}^{-\a} = \bs{1+\tau(\t)} w,
\end{equation*}
which then yields the firm's labor demand:
\begin{equation}
L^d(\t,a) = \bc{\frac{(1-\a) a}{\bs{1+\tau(\t)}^{1-\a} w}}^{\frac{1}{\a}}.
\label{e:ld}\end{equation}
The labor demand gives the firm's employment level for any tightness and productivity. There is job rationing in the model because the labor demand is decreasing with tightness. The labor demand fluctuates in response to productivity shocks over the business cycle.

The elasticity of labor demand with respect to tightness, defined by $\e^{d}_{\t}=\oex{L^d}{\t}$, admits the following expression:
\begin{equation}
\e^{d}_{\t}= -\frac{1-\a}{\a}\e^{1+\tau}_{\t} = -\frac{1-\a}{\a}\h\tau(\t).
\label{e:edt}\end{equation}

\subsection{Solution of the model}

The model requires that firms maximize profits, and that employment is determined by the matching and separation processes. These requirements impose that labor-market tightness $\t$ equalizes labor demand and labor supply:
\begin{equation*}
L^d(\t,a) = L^s(\t,H).
\end{equation*}

The labor demand is strictly decreasing in tightness while the labor supply is strictly increasing in tightness. For any given $H$, the equation pins down a unique tightness: the unique solution of the model. Therefore, the supply-equals-demand condition implicitly defines tightness as a function $\t(H)$ of the labor force. Plugging  $\t(H)$ into either labor supply or labor demand defines the employment level as a function of $H$, $L(H)$. The unemployment rate is then $u(\t(H))=\l/[\l+f(\t(H))]$. The solution of the model is illustrated in figure~\ref{f:solution}.

\begin{figure}[t]
\subcaptionbox{Labor market in normal times \label{f:normaltimes}}{\includegraphics[scale=\scale,page=8]{\pdf}}\hfill
\subcaptionbox{Labor market in bad times \label{badtimes}}{\includegraphics[scale=\scale,page=9]{\pdf}}
\caption{Solution of the model}
\note{The labor demand curve is given by \eqref{e:ld}. The labor supply is given by \eqref{e:ls}. The solution of the model is at the intersection of the labor demand and supply curves.}
\label{f:solution}\end{figure}

\section{Impact of in-migration on the labor market}\label{s:migration}

Migration alters the size of the labor force and, consequently, labor-market tightness, the job-finding rate, and the unemployment rate. This section describes how a wave of in-migration affects the labor market.

\subsection{Modeling in-migration}

In-migration leads to a sudden increase in the number of workers in the labor force, so a sudden increase in $H$. By modeling a wave of in-migration as an increase in the labor force, I implicitly assume that local and migrant workers are the same: they have the same productivity, command the same wage, and are hired through the same channels by firms. This assumption is consistent with the evidence collected by \citet[p.~220]{MPV18} in Portuguese administrative data:
\begin{quote}
These estimates imply that, even for narrowly defined jobs, employers do not substitute natives with immigrants. On the contrary, immigrants and natives are jointly hired and fired. This is true for all types of jobs considered, including lower skilled jobs.
\end{quote}

Here the focus is on the short-term effects of in-migration, so the labor demand is assumed to remain unaffected by migration. Assuming that labor demand is unaffected for instance rules out that firms adjust their capital stock when migrants arrive. It also rules out that migrants bring innovations to the production process that improve labor productivity. These are standard assumptions to describe the short run. In the long run, as firms adjust capital stock and production process, the impact of in-migration on the labor market will vanish because the labor demand will scale up with labor supply \citep{C12a}.

Of course, the paper's results could also be applied to out-migration, such as the Irish emigration studied by \citet{BHO94}, or the Mexican emigration studied by \citet{H07}. Out-migration would be modeled as a reduction in labor force $H$, which would have the opposite effects of in-migration.

\subsection{Impact of in-migration on local workers}\label{s:inmigration}

I now determine the impact of in-migration on the prospects of local jobseekers. The main step is to compute the elasticity of tightness $\t(H)$ with respect to the labor force $H$. From this, I will infer the elasticity of the job-finding rate $f(H)$ with respect to the labor force $H$. 

Consider a small change in the size of the labor force generated by in-migration, $d\ln H$. This small change generates a small change in tightness, $d\ln \t$. These changes generate small changes in labor supply and demand:
\begin{align*}
d \ln L^s &= \e^{s}_{\t} d\ln \t + \e^{s}_H d\ln H\\
d \ln L^{d} &= \e^{d}_{\t} d\ln \t.
\end{align*}

Since the supply-equals-demand condition must hold both before and after in-migration, $d\ln L^s = d\ln L^d$. This means that
\begin{equation*}
\e^{s}_{\t} d\ln \t + \e^{s}_H d\ln H = \e^{d}_{\t}d\ln \t.
\end{equation*}
In other words, the elasticity of tightness with respect to labor force, $\e^{\t}_H  = \oex{\t}{H}$, is given by
\begin{equation}
\e^{\t}_H = \frac{-\e^{s}_H}{\e^{s}_{\t}-\e^{d}_{\t}} =  \frac{-1}{\e^{s}_{\t}-\e^{d}_{\t}}.
\label{e:eth}\end{equation}
Therefore, the elasticity of the job-finding rate with respect to the labor force, 
$\e^{f}_H  = \oex{f(\t)}{H} = [\oex{f}{\t}] \cdot [\oex{\t}{H}]$, is given by
\begin{equation}
\e^{f}_H = \frac{-\bp{1-\h}}{\e^{s}_{\t}-\e^{d}_{\t}}.
\label{e:efh}\end{equation}
Finally, the elasticity of the unemployment rate with respect to the labor force, which is defined by $\e^{u}_{H} = \oex{u}{H}$, is
\begin{equation*}
\e^{u}_H = \e^{u}_{\t} \cdot \e^{\t}_H = \frac{1-u(\t)}{u(\t)}\cdot \frac{\e^s_{\t}}{\e^{s}_{\t}-\e^{d}_{\t}}.
\end{equation*}
From this, I infer the semi-elasticity of the unemployment rate with respect to the labor force, which gives the percentage-point change in unemployment rate when the labor force changes by one percent:
\begin{equation}
\frac{du}{d\ln H} = u \cdot \e^{u}_{H} = \frac{1-u(\t)}{1-\e^{d}_{\t}/\e^s_{\t}}.
\label{e:suh}\end{equation}

A last useful statistic is the elasticity of the employment rate $l = 1 - u$ with respect to labor force, which is defined by $\e^{l}_{H} = \oex{l}{H}$, and which follows from \eqref{e:suh}:
\begin{equation}
\e^{l}_{H} = \frac{1}{1-u}\cdot\frac{-du}{d\ln H} = \frac{-1}{1-\e^{d}_{\t}/\e^s_{\t}}.
\label{e:elh1}\end{equation}

Collecting these results yields the following proposition:
\begin{proposition}
In-migration leads to a decrease in labor-market tightness, which causes a decrease in the job-finding rate of local workers, an increase in their unemployment rate, and a decrease in their employment rate.
\label{p:labor}\end{proposition}

The proposition is a direct consequence from the facts that $\e^s_{\t}>0$ and $\e^d_{\t}<0$ and from \eqref{e:eth}, which shows that the elasticity of tightness with respect to the labor force is negative, from \eqref{e:efh}, which shows that the elasticity of the job-finding rate with respect to the labor force is negative, from \eqref{e:suh}, which shows that the semi-elasticity of unemployment rate with respect to the labor force is positive, and finally from \eqref{e:elh1}, which shows that elasticity of the employment rate with respect to the labor force is positive. 

The intuition for the proposition is simple. In-migration increases the number of available workers, but not the number of jobs, so it raises labor supply and not labor demand. Such increase in supply relative to demand leads to a decrease in tightness (figure~\ref{f:migration}). 

To understand why tightness must fall after a wave of in-migration, let's think about what would happen if tightness did not respond. Then firms would want to employ the same number of workers as before, since the labor demand had not changed. But since tightness has not changed, jobseekers' job-finding rate has not changed. And since the pool of jobseekers increased after migrants joined the labor force, more jobseekers will end up with a job. (That is, labor supply is higher than labor demand at the current tightness.) Firms would respond to such excess of new hires by posting fewer vacancies, leading to a drop in tightness, until supply and demand are equalized.

\begin{figure}[t]
\subcaptionbox{Initial situation}{\includegraphics[scale=\scale,page=10]{\pdf}}\hfill
\subcaptionbox{After in-migration\label{f:tightnessadjustment}}{\includegraphics[scale=\scale,page=11]{\pdf}}
\caption{In-migration reduces the job-finding rate of local workers, and therefore increases their unemployment rate}
\note{The labor demand curve is given by \eqref{e:ld}. The labor supply is given by \eqref{e:ls}. This graph illustrates the results from proposition~\ref{p:labor}.}
\label{f:migration}\end{figure}

The other results in the proposition directly follow from the drop in tightness. Lower tightness means lower job-finding rate, so in-migration reduces the job-finding rate of the local workers. A lower job-finding rate means a higher unemployment rate, so in-migration increases the unemployment rate faced by local workers and decreases their employment rate. The underlying reason is that after an influx of new workers into the labor force, there is the same number of jobs but more jobseekers, so it becomes harder to find a job. As a result, a larger fraction of workers remains unemployed, and a smaller fraction is employed. Of course, since the unemployment rate increases and the size of the labor force increases, the number of unemployed workers increases sharply after in-migration---some of the unemployed are local workers and some are migrant workers.

In the model, the unemployment rate of local workers goes up after a wave of in-migration because it becomes harder for local jobseekers to find jobs, as they compete with newly arrived workers. The job-separation rate of local workers is unaffected, so local workers do not lose their jobs at a higher rate. But, once a local worker loses her job, she will remain unemployed longer since the job-finding rate is lower. This mechanism is consistent with the evidence provided by \citet{DSS17}. The mechanism also reconciles the evidence presented in section~\ref{s:evidence}, that in-migration induces local unemployment, and the evidence presented by \citet{WZ99}, that local workers are not displaced by immigrants in the sense that local workers are not laid off at a higher rate to be replaced by immigrant workers.

Public anxieties about migration often focus on the idea that migrant workers "steal jobs" from local workers. For instance, \citet[table~C3]{HWZ17} report that Austrian voters worry about the negative effects of immigration on the labor market: they think that immigrants threaten their labor market opportunities. Given the results from proposition~\ref{p:labor}, these anxieties are unsurprising. First, the job-finding rate for local jobseekers falls when immigrants arrive. So it becomes harder for any one jobseeker to find a job. They have fewer jobs available to them because the number of available jobs did not scale up with the increase in labor-force participants. Second, the employment rate of local decreases when immigrants arrive. So local workers might feel that immigrants steal their jobs: the fraction of local workers how have a job is indeed lower, and the fraction who remain unemployed is higher. And the reason is that immigrant workers are now employed in some of the available jobs, relegating local workers to unemployment.

\begin{figure}[t]
\subcaptionbox{Borjasian model in employment-wage diagram}{\includegraphics[scale=\scale,page=3]{\pdf}}\hfill
\subcaptionbox{Effect of in-migration in the Borjasian model\label{f:wageadjustment}}{\includegraphics[scale=\scale,page=4]{\pdf}}
\caption{Parallel with the Borjasian model of the labor market}
\label{f:borjas}\end{figure}

The model developed here shares a similarity with the Borjasian model of immigration: both have a downward-sloping labor demand curve (figure~\ref{f:borjas}). This means that the number of jobs available in the labor market is somewhat limited, which explains why newly arrived workers compete for jobs with local workers, who face reduced opportunities. The novelty is that the adjustment to migration does not happen through wages but through labor-market tightness (compare figure~\ref{f:tightnessadjustment} to figure~\ref{f:wageadjustment}). Tightness itself determines the local job-finding rate and unemployment rate. In fact, in this basic version of the model, wages do not respond to migration at all, so just as in the Cardian perspective, the arrival of migrants does not reduce the local wage, and thus does not affect local workers who are employed. Here in-migration only affects local workers who are unemployed. 

\subsection{Impact of in-migration when wages respond}\label{s:wageresponse}

In general, both tightness and wages respond to migration. This section extends the model so wages also respond to migration. To capture the effect of migration on wages, I assume that the real wage is not constant but is a function of the size of the labor force:
\begin{equation}
w = \o \cdot H^{-\b},
\label{e:wage}\end{equation}
where $\b\geq 0\in [0,\a)$ captures the flexibility of wages with respect to migration. If $\b=0$, wages do not respond to migration at all. If $\b>0$, wages fall when new workers enter the labor force, because of wage competition. Since $\b$ is the elasticity of real wages with respect to the labor force, it can be estimated empirically \citep{B03,BK15,B17}.

What is the impact of in-migration when wages respond to migration? Let's follow the same steps as in section~\ref{s:inmigration} but allow for a wage response. Consider a small change in the size of the labor force generated by in-migration, $d\ln H$. In addition to affecting the labor supply directly, this small change generates a small change in tightness, $d\ln \t$, and a small change in wages, $d\ln W$. Combined, these changes generate small changes in labor supply and demand:
\begin{align*}
d \ln L^s &= \e^{s}_{\t} d\ln \t + \e^{s}_H d\ln H\\
d \ln L^{d} &= \e^{d}_{\t} d\ln \t + \e^{d}_W d\ln W.
\end{align*}
Since the supply-equals-demand condition must hold both before and after in-migration, $d\ln L^s = d\ln L^d$, so
\begin{equation*}
\e^{s}_{\t} d\ln \t + \e^{s}_H d\ln H = \e^{d}_{\t} d\ln \t + \e^{d}_W d\ln W.
\end{equation*}

The small wage change is given by $d\ln W  = \e^W_H d\ln H$. Given the wage norm \eqref{e:wage}, the elasticity of wages with respect to labor force is $\e^W_H = -\b$. Thus, $d\ln W  = -\b d\ln H$, the elasticity of tightness with respect to labor force is given by
\begin{equation}
\e^{\t}_H = - \frac{\b \e^d_W + \e^{s}_H}{\e^{s}_{\t}-\e^{d}_{\t}} =  -\bp{1-\frac{\b}{\a}} \cdot \frac{1}{\e^{s}_{\t}-\e^{d}_{\t}},
\label{e:ethwage}\end{equation}
where the second equality comes from the fact that $\e^s_H = 1$ and $\e^d_W =-1/\a $, which can themselves be obtained from \eqref{e:ls} and \eqref{e:ld}. Then, the elasticity of the job-finding rate with respect to the labor force is given by
\begin{equation}
\e^{f}_H = -\bp{1-\frac{\b}{\a}} \cdot \frac{1-\h}{\e^{s}_{\t}-\e^{d}_{\t}},
\label{e:efhwage}\end{equation}
The semi-elasticity of the unemployment rate with respect to the labor force, which gives the percentage-point change in unemployment rate when the labor force changes by one percent, is
\begin{equation}
\frac{du}{d\ln H}=\bp{1-\frac{\b}{\a}} \cdot \frac{1-u(\t)}{1-\e^{d}_{\t}/\e^s_{\t}}.
\label{e:suhwage}\end{equation}
And the elasticity of the employment rate with respect to labor force is:
\begin{equation}
\e^{l}_{H} = -\bp{1-\frac{\b}{\a}}\frac{1}{1-\e^{d}_{\t}/\e^s_{\t}}.
\label{e:elh1wage}\end{equation}

Collecting these results yields the following proposition:
\begin{proposition}
Even when wages fall after a wage of in-migration, in-migration leads to a decrease in labor-market tightness, which causes a decrease in the job-finding rate of local workers, an increase in their unemployment rate, and a decrease in their employment rate. However, all the effects are weaker than when wages do not respond to in-migration. Quantitatively, the effects are muted by a factor $1-\b/\a \in (0,1)$, where $\b\in (0,\a)$ is the positive elasticity of real wages with respect to the labor force and $\a\in (0,1)$ is the positive elasticity of the marginal product of labor with respect to employment.
\label{p:laborwage}\end{proposition}

The proposition is obtained like proposition~\ref{p:labor}, but using equations \eqref{e:ethwage}, \eqref{e:efhwage}, \eqref{e:suhwage}, and \eqref{e:elh1wage}. The main intuition for the results remains the same, except that here, tightness and wages both respond to migration. It is the combined drops in tightness and in wages that absorb the influx of migrants (figure~\ref{f:combined}), whereas in section~\ref{s:inmigration} the influx of migrants was solely absorbed by a drop in tightness.

Here, both employed and unemployed local workers are hurt by in-migration. For unemployed workers, it is harder to find a job; for employed workers, labor income is reduced. However, the reduction in job-finding rate experienced by jobseekers is less severe than when wages do not respond. In that way, the drop in wages spreads the negative impact of in-migration across all workers---employed and unemployed.

\begin{figure}[t]
\subcaptionbox{Drop in tightness when wages do not fall\label{f:wagefixed}}{\includegraphics[scale=\scale,page=13]{\pdf}}\hfill
\subcaptionbox{Drop in tightness when wages fall\label{f:wagerigid}}{\includegraphics[scale=\scale,page=14]{\pdf}}
\caption{Joint response of labor-market tightness and wages to in-migration}
\note{The labor demand curve is given by \eqref{e:ld}. The labor supply is given by \eqref{e:ls}. In panel A, the real wage $w$ does not respond to $H$: $\b=0$ in \eqref{e:wage}. In panel B, the real wage $w$ falls when $H$ increases: $\b\in(0,\a)$ in \eqref{e:wage}. The figure illustrates the results from proposition~\ref{p:laborwage}.}
\label{f:combined}\end{figure}

\citet{E16} illustrates why different labor markets might see different wage response to immigration, and how tightness responds accordingly, just as predicted by the model. In France, certain contracts provide much stronger wage protection than others. As predicted by the model, workers whose wages are protected see a significant increase in unemployment rate when immigrants enter the labor market: the arrival of 100 immigrants pushes 30--50 workers into unemployment \citep[p.~2]{E16}. For workers whose wages are not protected, however, the impact of immigration is different: wages fall significantly but the unemployment rate is unaffected \citep[p.~3]{E16}. Because most workers have contracts that protect their wages, in aggregate it is tightness and not wages that responds to immigration.

\citet{BE21} provide another example of markets in which the response of wages, and accordingly the response of tightness, varies. In France, they find that the labor market for female workers exhibited no wage response but strong tightness response. The labor market for male workers operated the other way around: no tightness response but sharp wage response. 

\subsection{Incorporating the response of aggregate demand to migration}\label{s:ad}

The model focuses on the impact of migration on the labor supply. But of course migrants who enter a local economy also affect the aggregate demand, since migrants consume goods and services where they live. Under the standard assumption that the product market is Walrasian, the aggregate-demand response can be incorporated through the response of the real wage to migration. 

After the arrival of new workers into the local economy, the aggregate demand is boosted, which may raise the prices of locally provided goods and services \citep[p.~957]{GH84}. For instance, \citet{AS10} and \citet{MV14} find that prices of local goods went up when refugees from Burundi and Rwanda arrived in Tanzania in the 1990s.

The standard assumption in matching models of the labor market is that workers and firms bargain over real wages. In that case, changes in aggregate demand have no effect on labor demand. As the price charged by a firm for its product goes up, the nominal wage paid to its workers goes up in tandem to keep the real wage at the same level, so its demand for labor is unaffected. Price changes have no effect on labor demand, so the boost in aggregate demand does not translate into a boost in labor demand.

In the less standard but maybe more realistic case in which wages are nominally rigid, then changes in aggregate demand do affect labor demand. As aggregate demand increases and prices $P$ rise, nominally rigid wages $W$ will rise less than prices, leading to a drop in real wages $w = W/P$. Such drop in real wages stimulates labor demand. 

Since the model already allows real wages to fall with migration, as displayed in \eqref{e:wage}, the model can easily incorporate the response of aggregate demand to migration. It suffices to interpret and calibrate the elasticity $\b$ as the response of real wages to migration generated both by lower nominal wages due to higher labor supply and higher prices due to higher aggregate demand.

Another related observation is that, irrespective of the response of labor demand to aggregate demand, migrants do consume in the model. Output equals real income, so all real income is consumed. Firm owners, local workers, and migrant workers all consume their respective real income, which is produced by the firms in which they work.

\subsection{Impact of in-migration in DMP models}\label{s:dmp}

Standard matching models have constant instead of decreasing returns to labor ($\a=0$). With constant returns to labor, the labor demand is degenerate: firm's optimal employment choice solely determines tightness. Setting $\a=0$ in \eqref{e:ld} gives
\begin{equation}
\bs{1+\tau(\t)} w = a,
\label{e:ldstd}\end{equation}
which determines tightness in the model, independently from employment. This labor-demand relation holds irrespective of the wage-setting assumption. It holds with Nash bargained wages, as in the textbook model by \citet{P00}, or with fixed wages, as in \citet{H05}.

Here I analyze the DMP model in an employment-tightness diagram (figure~\ref{f:dmp}). It is more common to analyze the DMP model in an unemployment-vacancy diagram (figure~\ref{f:dmpuv}). However, the two presentations are equivalent. The labor demand curve in figure~\ref{f:dmp} and the job-creation curve in figure~\ref{f:dmpuv} both represent the tightness level defined by \eqref{e:ldstd}. The labor supply curve in figure~\ref{f:dmp} and the Beveridge curve in in figure~\ref{f:dmpuv} both represent the locus of points satisfying \eqref{e:ls}. Equation \eqref{e:ls} links tightness $\t$ to the employment level $L$. But since $\t = v/u$ and $L = (1-u) H$, and since $f(\t) = \m\t^{1-\h}$, it is easy to rewrite it as a Beveridge curve linking the vacancy rate $v$ to the unemployment rate $u$:
\begin{equation*}
v(u) = \bp{\frac{\l}{\m}\cdot\frac{1-u}{u^{\h}}}^{1/(1-\h)}.
\end{equation*}
The Beveridge curve is decreasing and convex \citep[appendix~A1]{MS21b}, as plotted in figure~\ref{f:dmpuv}. 

There are two advantages to studying the DMP model in an employment-tightness diagram. First, the diagram offers a more interesting description of migration. In the unemployment-vacancy diagram, migration would not appear since it affects neither the job-creation curve nor the Beveridge curve. In the employment-tightness diagram, by contrast, a wave of in-migration shifts the labor supply curve (figure~\ref{f:dmpmigration}). Second, in the employment-tightness diagram, the parallel between the DMP and Cardian models become clear (compare figure~\ref{f:standard} to figure~\ref{f:card}).

Let's turn to the effect of migration in the DMP model. The labor demand \eqref{e:ldstd} does not involve labor force $H$, so labor-market tightness is the same irrespective of the amount of migration. This means that the employment rate is independent of labor force: migration has no effect on the unemployment rate. Figure~\ref{f:dmpmigration} illustrates. An increase in the labor force from in-migration is absorbed entirely by firms, which post more vacancies and create more jobs. In fact, the vacancy rate is unaffected by migration, so the number of vacancies grows in proportion to the labor force. As a result, local workers are completely unaffected by migration.

\begin{figure}[p]
\subcaptionbox{DMP model in unemployment-vacancy diagram \label{f:dmpuv}}{\includegraphics[scale=\scale,page=5]{\pdf}}\hfill
\subcaptionbox{DMP model in employment-tightness diagram \label{f:dmp}}{\includegraphics[scale=\scale,page=6]{\pdf}}\fspace
\subcaptionbox{In-migration in DMP model if wages are constant \label{f:dmpmigration}}{\includegraphics[scale=\scale,page=7]{\pdf}}\hfill
\subcaptionbox{In-migration in DMP model if wages fall \label{f:dmpwage}}{\includegraphics[scale=\scale,page=21]{\pdf}}\hfill
\caption{Impact of in-migration in a DMP model}
\note{The job-creation curve and labor demand curve represent the same locus of points They are given by \eqref{e:ldstd}. The Beveridge curve and labor supply curve represent the same locus of points. They are given by \eqref{e:ls}.}
\label{f:standard}\end{figure}

If wages respond to in-migration, the DMP model even predicts that in-migration reduces the unemployment rate (figure~\ref{f:dmpwage}). In equation~\eqref{e:ldstd}, the wage $w$ drops with in-migration, so tightness goes up with in-migration. In the textbook DMP model, wages are determined by bargaining. A reason for the wage drop is that migrants have worse outside options than local workers and cannot command the same wages \citep{CP15,A21}. In that case, average wages fall, boosting labor demand. 

Just as in the DMP model presented in figure~\ref{f:dmp}, the Cardian model of the labor market has a horizontal labor demand curve (figure~\ref{f:card}). The difference is that in the DMP model the demand curve is horizontal in an employment-tightness diagram, because the labor market is organized around a matching function, while in the Cardian model the demand curve is horizontal in an employment-wage diagram, because the labor market is competitive. The labor demand curve is horizontal in the Cardian model of immigration to capture the fact that immigration has no effect on local wages.

\begin{figure}[t]
\subcaptionbox{Cardian model in employment-wage diagram}{\includegraphics[scale=\scale,page=1]{\pdf}}\hfill
\subcaptionbox{Effect of in-migration in the Cardian model}{\includegraphics[scale=\scale,page=2]{\pdf}}
\caption{Parallel between the DMP model and Cardian model of the labor market}
\label{f:card}\end{figure}

\section{Impact of in-migration on firms}\label{s:firms}

The model also reveals winners and losers from migration. Local workers, facing a higher unemployment rate and potentially lower wages, experience a decline in labor income from in-migration. On the other hand, firms benefit from in-migration due to easier recruitment and potentially lower labor costs. Thus, the benefits from in-migration accrue to firms while the costs are borne by local workers.

Indeed, in-migration unambiguously helps local firms:
\begin{proposition}
In-migration leads to an increase in employment, a reduction in the recruiter-producer ratio, and an increase in real profits. These results hold whether wages respond to migration or not.
\label{p:firms}\end{proposition}

The beginning of the proposition follows from the facts that labor demand is decreasing with tightness, the recruiter-producer ratio is increasing with tightness, and tightness falls after an in-migration wave. This is true whether wages respond to in-migration or not (see propositions~\ref{p:labor} and~\ref{p:laborwage}).

The impact of in-migration on real profits requires a little bit of algebra. Aggregate real profits are given by 
\begin{equation*}
\pi = a P^{1-\a} - [1+\tau(\t)] w P,
\end{equation*}
where the first term is firm's output and the second term is the real wage bill.

The first-order condition of the firm problem \eqref{e:foc} links the real wage $W$ to the number of producers $P$: 
\begin{equation*}
w = \frac{(1-\a) a P^{-\a}}{1+\tau(\t)}.
\end{equation*}
Therefore the real wage bill is
\begin{equation*}
[1+\tau(\t)] w P = (1-\a) a P^{1-\a}.
\end{equation*}
Combining these results yields real profits as a function of the number of producers: 
\begin{equation}
\pi = \a \cdot a \cdot P^{1-\a}.
\label{e:profitshare}\end{equation}
This equation simply says that the economy's profit share $\pi/y$ is $\a$. Profits are therefore proportional to output. Since output is divided between labor income and profits, the economy's labor share is $1-\a$.

After in-migration, employment $L$ goes up, and tightness $\t$ falls, so the recruiter-producer ratio $\tau(\t)$ falls. Hence, the number of producers $P = L/[1+\tau(\t)]$ goes up. Therefore, equation~\eqref{e:profitshare} shows that profits go up after a wave of in-migration. This result holds whether wages respond to migration or not.

The number of jobs in the economy increases after in-migration, as do firms' profits. Firm owners therefore always benefit from in-migration---unlike workers who always suffer from it. If workers own share of firms, of course, the impact of in-migration is murkier: in-migration reduces their labor income but raise their capital income. If workers own very little capital, then they are clearly negatively affected by in-migration.

\section{In-migration in good and bad times}\label{s:businesscycle}

I now contrast the effects of migration on the labor market in good times---when jobs are abundant and tightness is high---and bad times---when jobs are scarce and tightness is low. The model reveals that the impact of in-migration on unemployment is not uniform across all economic conditions, but it is more severe in bad times than in good times.

\subsection{Modeling good and bad times}

In the United States labor-market fluctuations are driven by labor-demand shocks, not labor-supply shocks \citep{MS15}. I therefore model good and bad times as periods with high and low labor demand. In this simple model labor demand is governed by the wage-to-productivity ratio, $w/a$ (equation~\eqref{e:ld}). Good times are periods when the wage-to-productivity ratio is low so hiring workers is particularly profitable and the labor demand is elevated (figure~\ref{f:lowwage}). Bad times are periods when the wage-to-productivity ratio is high so hiring workers is not very profitable and the labor demand is depressed (figure~\ref{f:highwage}). 

What causes changes in the wage-to-productivity ratio, $w/a$? The typical cause of these fluctuations are fluctuations in productivity $a$ under a fixed wage $w$ or under a partially rigid wage $w = \o \cdot a^{1-\g}$, where $\g>0$ measures the rigidity of real wages with respect to productivity \citep{H05,BG10,M12,M14}.

In this section, I assume a wage norm that is as general as possible:
\begin{equation}
w = \o \cdot a^{1-\g} \cdot H^{-\b},
\label{e:wageGeneral}\end{equation}
where $\g\in(0,1]$ governs the rigidity of real wages with respect to productivity and $\b\in[0,\a)$ governs the flexibility of real wages with respect to the labor force. Then the wage-to-productivity ratio becomes
\begin{equation*}
\frac{w}{a} = \o \cdot a^{-\g} \cdot H^{-\b},
\end{equation*}
which is strictly decreasing in productivity and enters the labor demand as follows:
\begin{equation}
L^d(\t,a) =\bc{\frac{\bp{1-\a} a^{\g} H^{\b}}{\bs{1+\tau(\t)}^{1-\a}\o}}^{\frac{1}{\a}}.
\label{e:lda}\end{equation}
The demand equation shows that labor demand is increasing in productivity $a$ (since $\g<1$). So good times are represented by a high labor productivity and bad times by a low labor productivity. I also allow real wages to decrease with in-migration ($\b\geq 0$) but just as in proposition~\ref{p:laborwage}, that will not affect the qualitative results at all.

\begin{figure}[t]
\subcaptionbox{Good times: high productivity\label{f:lowwage}}{\includegraphics[scale=\scale,page=8]{\pdf}}
\subcaptionbox{Bad times: low productivity\label{f:highwage}}{\includegraphics[scale=\scale,page=9]{\pdf}}\hfill
\caption{Good and bad times in the model}
\note{The labor supply is given by \eqref{e:ls}. In panel A, the labor demand curve is given by \eqref{e:lda} with a high $a$. In panel B, the labor demand curve is given by \eqref{e:lda} with a low $a$. The figure illustrates the results from proposition~\ref{p:businesscycle}.}
\label{f:goodbadtimes}\end{figure}

\subsection{Labor market in good and bad times}

Figure~\ref{f:goodbadtimes} allows me to obtain a range of comparative statics describing the business cycle:
\begin{proposition}
Labor-market conditions deteriorate when productivity decreases: the labor-market tightness falls; the job-finding rate falls; the unemployment rate increases; the employment rate decreases. In these conditions, however, recruiting becomes easier: the vacancy-filling rate increases and the recruiter-producer ratio falls.
\label{p:businesscycle}\end{proposition}

Figure~\ref{f:goodbadtimes} shows that tightness drops when productivity decreases and labor-demand curve falls. All the other results follow since all the other quantities are functions of tightness. 

The response of tightness to productivity could also be obtained by implicitly differentiating the equation $L^s(\t,H) = L^d(\t,a)$ with respect to $a$.  Consider a small change in productivity, $d\ln a$. In addition to affecting the labor demand directly, this small change generates a small change in tightness, $d\ln \t$. Combined, these changes generate small changes in labor supply and demand:
\begin{align*}
d \ln L^s &= \e^{s}_{\t} d\ln \t\\
d \ln L^{d} &= \e^{d}_{\t} d\ln \t + \e^{d}_a d\ln a.
\end{align*}
Since the supply-equals-demand condition must hold both before and after the productivity shock, $d\ln L^s = d\ln L^d$, so
\begin{equation*}
\e^{s}_{\t} d\ln \t = \e^{d}_{\t} d\ln \t + \e^{d}_a d\ln a.
\end{equation*}

Given the expression \eqref{e:lda}, the elasticity of the labor demand with respect to productivity is $\e^d_a = \g/\a$. Thus, the elasticity of tightness with respect to productivity is
\begin{equation*}
\e^{\t}_a =  \frac{\g}{\a \bp{\e^{s}_{\t}-\e^{d}_{\t}}}.
\end{equation*}
From this, I find that the elasticity of the job-finding rate with respect to productivity is given by
\begin{equation*}
\e^{f}_a = \frac{\bp{1-\h} \g}{\a \bp{\e^{s}_{\t}-\e^{d}_{\t}}}.
\end{equation*}
The semi-elasticity of the unemployment rate with respect to productivity, which gives the percentage-point change in unemployment rate when productivity changes by one percent, is
\begin{equation*}
\frac{du}{d\ln a}=-\frac{\g}{\a} \cdot \frac{1-u}{1-\e^{d}_{\t}/\e^s_{\t}}.
\end{equation*}
And the elasticity of the employment rate with respect to productivity is:
\begin{equation*}
\e^{l}_{a} = \frac{\g}{\a-\a \bp{\e^{d}_{\t}/\e^s_{\t}}}.
\end{equation*}

Equations \eqref{e:est} and \eqref{e:edt} show that $\e^s_{\t}>0$ while $\e^d_{\t}<0$, so $\e^s_{\t}-\e^d_{\t}>0$ and $1-\e^s_{\t}/\e^d_{\t}>0$. With wage rigidity, $\g>0$, so $\e^{\t}_a>0$, $\e^{l}_a>0$, and $du/d\ln a<0$. In other words, when productivity drops, tightness and employment fall while unemployment rises.

Conversely, the elasticity of the vacancy-filling rate with respect to productivity is
\begin{equation*}
\e^{q}_a = -\frac{\h \g}{\a \bp{\e^{s}_{\t}-\e^{d}_{\t}}}.
\end{equation*}
The elasticity of the recruiter-producer ratio with respect to productivity is
\begin{equation*}
\e^{\tau}_a = \frac{\h \g \bp{1+\tau}}{\a \bp{\e^{s}_{\t}-\e^{d}_{\t}}}.
\end{equation*}
For the same reasons as above, $\e^q_a<0$ and $\e^{\tau}_a>0$. When productivity drops, the recruiter-producer ratio drops, so it becomes easier and cheaper for firms to recruit workers.

\subsection{Impact of in-migration in good and bad times}

I now examine how the amount of unemployment induced by migration varies over the business cycle. A wave of in-migration increases the unemployment rate, so it reduces the local employment rate. As such, the model produces the type of job stealing that local workers commonly complain about. In addition, such job stealing is worse in bad times:
\begin{proposition}
The elasticity of the employment rate with respect to the labor force is
\begin{equation}
\e^l_H(\t) = - \bp{1-\frac{\b}{\a}} \frac{1}{1+\frac{1-\a}{\a} \cdot \frac{\h}{1-\h}\cdot \frac{\tau(\t)}{u(\t)}}. 
\label{e:elh2}\end{equation}
The elasticity is negative, but its amplitude is decreasing with tightness. As a result, when labor-market conditions deteriorate, the elasticity becomes more negative. The elasticity tends to $0$ when tightness tends to $\t_{\tau}$, it equals  $\b-\a<0$ when tightness is efficient, and it tends to $\b/\a - 1 < \b - \a$ when tightness goes to $0$. If wages do not respond to migration ($\b=0$), the elasticity is $-\a$ when tightness is efficient, and it falls to $-1$ when tightness goes to $0$.
\label{p:stealing}\end{proposition}

The elasticity \eqref{e:elh2} in the proposition is obtained from \eqref{e:elh1wage}, \eqref{e:est}, and \eqref{e:edt}. Since $\tau(\t)$ is increasing with $\t$ while $u(\t)$ is decreasing with $\t$, the amplitude of $\e^l_H(\t)$ is decreasing with tightness. When $\t\to 0$, $\tau(\t)\to 0$, so $\e^l_H \to -(1-\b/\a)$. When tightness is efficient, $\h\tau(\t) = (1-\h) u(\t)$, so $\e^l_H \to -(\a - \b)$. When $\t\to \t_{\tau}$, $\tau(\t)\to \infty$, so $\e^l_H \to 0$.
	
The proposition shows that job stealing is more prevalent in bad times. Formally, the percentage reduction in employment rate due to a one-percent increase in the labor force is larger in bad times, when tightness is low (figure~\ref{f:businesscycle}). 

\begin{figure}[t]
\subcaptionbox{Small employment drop in good times\label{f:goodtimes}}{\includegraphics[scale=\scale,page=22]{\pdf}}\hfill
\subcaptionbox{Larger employment drop in bad times\label{f:badtimes}}{\includegraphics[scale=\scale,page=13]{\pdf}}
\caption{A wave of in-migration reduces the local employment rate more in bad times than in good times}
\note{The labor supply is given by \eqref{e:ls}. The labor demand curve is given by \eqref{e:lda}, with a high $a$ in good times and a low $a$ in bad times, and with $\b=0$ for simplicity. This graph illustrates the results from proposition~\ref{p:stealing}.}
\label{f:businesscycle}\end{figure}

The worst-case scenario occurs when the labor market is the slackest. Then, if wages do not respond to migration, a one-percent increase in the labor force leads to a one-percent decrease in employment rate. The reason is that when the labor market is extremely slack, job rationing is the most stringent, so the number of jobs is almost fixed. With a fixed number of jobs, employment rate and labor force are related by $l \cdot H = \text{constant}$ so the elasticity of the employment rate with respect to the labor force is $-1$.

Because the number of jobs is more limited in bad times, in-migration reduces the local employment rate more drastically in bad times. The model therefore predicts that job stealing is more prevalent in bad times. In line with this prediction, \citet{EO23} find that immigration has a negative effect on employment in the short run, but this negative effect is weaker in regions that are experiencing good economic conditions than in depressed regions.

In bad times, local workers are likely to be more opposed to in-migration because it hurts their labor-market prospects more. In line with this prediction, \citet[table~5A]{HWZ17} find that the shift of Austrian voters toward the far-right, anti-immigration FPO party in response to immigration is stronger when the local unemployment rate is higher. The effect of immigration on voting patterns is about twice as large in local labor markets with unemployment rates in the top quartile compared to those with unemployment rates in the bottom quartile.


\section{Optimal migration policy}\label{s:welfare}

Lastly, the model has implications for the design of migration policy. In the short run in-migration hurts local workers---especially in bad times---but helps local firms. To understand the overall impact of in-migration, I assess the effect of in-migration on total local welfare---the welfare of local workers plus the welfare of local firm owners. The model suggests that a procyclical in-migration policy would be optimal. This would involve encouraging in-migration during periods when labor markets are inefficiently tight, and limiting in-migration when labor markets are inefficiently slack. Such a policy balances the benefits of in-migration---the increased labor supply that makes it easier for firms to fill vacant jobs---with the costs of in-migration---the diversion of labor income from local workers to migrant workers, particularly during times of economic hardship.

\subsection{Computing local welfare}

In the model there is only one consumption good, produced by firms, which goes to local workers through local labor income, to immigrant workers through their labor income, and to firm owners through local profits. The goal is to assess how in-migration impacts local welfare, which is determined by the sum of local labor income and local profits.

The local labor force is denoted by $H$, and the total labor force by $m \cdot H$, where $m\geq 1$ captures the growth of the labor force due to in-migration. Broadly, $m - 1 \geq 0$ is the percentage change in the labor force caused by in-migration.

Aggregate real profits are given by $\pi = y(P) - [1+\tau(\t)] w P$. Output can be rewritten as a function of the marginal product of labor: $y(P) = y'(P)P/(1-\a)$. Moreover, on the labor demand, the marginal product of labor is always related to the wage and recruiter-producer ratio: $y'(P) = [1+\tau(\t)] w$. Combining these results I express real profits as a function of the wage bill $w L$:
\begin{equation}
\pi(\t) =\bs{\frac{1}{1-\a}-1} [1+\tau(\t)] w P = \frac{\a}{1-\a} w L.
\label{e:pi}\end{equation}
From \eqref{e:pi}, I express local profits as a function of the scale of in-migration $m$ and the employment rate $l = 1 - u$:
\begin{equation*}
\pi = \frac{\a}{1-\a} w l m H.
\end{equation*}
The local labor income is just the real wage times local employment:
\begin{equation*}
w l H.
\end{equation*}

Here for simplicity, distributional considerations are left out of social welfare, so local welfare is completely determined by local income. Distributional considerations can be excluded by assuming that workers and firm owners have the same linear utility function. They can also be excluded by assuming that workers have access to firm profits, so everyone consumes the same. Workers have access to profits if workers own all firms and therefore receive their profits, or if profits are fully taxed and redistributed to workers. In any case, adding both types of income gives local welfare:
\begin{equation}
\Wc = w H l \bs{\frac{\a}{1-\a} m + 1} = \frac{H}{1-\a} \bs{\a m + 1- \a} w l.
\label{e:welfare}\end{equation}


\subsection{Elasticity of welfare with respect to in-migration}

From expression \eqref{e:welfare}, I compute the elasticity of local welfare with respect to in-migration, allowing not only employment but also wages to respond to migration:
\begin{equation*}
\e^{\Wc}_{m} = \oe{\Wc}{m} = \oe{w}{m} + \oe{l}{m} + \frac{\a m}{\a m + 1- \a}
\end{equation*}
Combining this expression with the elasticity \eqref{e:elh1wage} of employment with respect to the labor force, and the elasticity $\oex{w}{m} = -\b$ assumed in \eqref{e:wageGeneral}, I get
\begin{equation}
\e^{\Wc}_{m} = \oe{\Wc}{m} = \frac{\a m}{\a m + 1- \a}- \b - \frac{1-\b/\a}{1-\e^{d}_{\t}/\e^s_{\t}}.
\label{e:ewm}\end{equation}

\subsection{Effect of infinitesimal in-migration on welfare}

As first step, I assess whether any in-migration might ever improve welfare. To do that, I determine whether the elasticity $\e^{\Wc}_{m}$ might ever be positive at $m = 1$. (Recall that $m$ goes from $m=1$ to $m>1$ when in-migration starts.) That is, I compute the effect of an infinitesimal wave of in-migration on welfare.

Setting $m=1$ in \eqref{e:ewm} and using \eqref{e:edt} and \eqref{e:est}, I find:
\begin{equation}
\e^{\Wc}_{m} =\bp{\a - \b } \bs{1 - \frac{1}{\a+(1-\a)\frac{\h}{1-\h}\frac{\tau(\t)}{u(\t)}}}.
\label{e:ewm1}\end{equation}

When $\t \to \t_{\tau}$, $\tau(\t) \to \infty$, so $\e^{\Wc}_{m} \to\a-\b>0$. Thus, when the labor market is at its tightest (at which point all workers are recruiters), then in-migration is desirable.

When $\t \to 0$, $\tau(\t) \to 0$, so $\e^{\Wc}_{m} \to (\a-\b)(1-1/\a)<0$. Hence, when the labor market is at its slackest (at which point all workers are unemployed), then in-migration is undesirable.

Given that $\tau/u$ is strictly increasing in $\t\in(0,\t_{\tau})$, $\e^{\Wc}_{m}$ is continuous and strictly increasing in $\t$, and there exists a unique $\t_m\in(0,\t_{\tau})$ at which $\e^{\Wc}_{m} = 0$, and in-migration improves welfare at any $\t>\t_m$ while in-migration reduces welfare at any $\t<\t_m$.

Solving for $\e^{\Wc}_{m}(\t) = 0$ with \eqref{e:ewm1}, I obtain
\begin{equation}
\frac{\h}{1-\h}\cdot\frac{\tau(\t)}{u(\t)} = 1.
\label{e:efficiency1}\end{equation}
But this condition is just the efficiency condition in the model \citep[lemma 1]{MS19}. The tightness $\t_m$ is just the tightness that maximizes the number of producers and consumption for a given labor force---the efficient tightness $\t^*$. It is easy to see why. Consumption is determined by the number of producers, 
\begin{equation*}
P = \frac{L}{1+\tau(\t)} = \frac{1-u(\t)}{1+\tau(\t)} H. 
\end{equation*}
Maximizing the number of producers is the same as maximizing the share of labor that is productive,
\begin{equation*}
\frac{1-u(\t)}{1+\tau(\t)}.
\end{equation*}
The elasticity of $1-u(\t)$ with respect to tightness is $(1-\h)u(\t)$ (equation~\eqref{e:eut}). The elasticity of $1+\tau(\t)$ with respect to tightness is $\h\tau(\t)$ (equation~\eqref{e:etaut}). The number of producers is maximized for a given labor force when its elasticity with respect to tightness is 0, or 
\begin{equation*}
\h \tau(\t)= (1-\h) u(\t). 
\end{equation*}

To highlight the parameters that determine the efficient tightness, I can re-express \eqref{e:efficiency1} as in \citet[equation~(29)]{MS22}:
\begin{equation*}
1 = \frac{\h}{1-\h}\cdot\frac{\l\k}{q(\t)-\l\k}\cdot\frac{\l+f(\t)}{\l},
\end{equation*}
which, after reshuffling terms, gives
\begin{equation*}
\bp{1-\h} q(\t) = \l\k + \h \k f(\t).
\end{equation*}
Dividing both sides by $(1-\h) q(\t)$ and noting that $f(\t)/q(\t)=\t$, I get
\begin{equation}
1 = \frac{\k}{1-\h} \bs{\frac{\l}{q(\t)} + \h\t}.
\label{e:efficiency2}\end{equation}
This is just the efficiency condition in a standard DMP matching model in which the interest rate is 0 and the social value of unemployment is 0 \citep[equation~(16)]{MS21b}. Such efficiency condition is obtained by combining the job-creation curve, given in \citet[equation~(1.24)]{P00}, with the \citet{H90} condition.

The following proposition summarizes the results:
\begin{proposition}
In any labor market that is inefficiently slack, allowing in-migration reduces local welfare. In any labor market that is inefficiently tight, allowing in-migration improves local welfare. When the labor market is efficient, in-migration has no first-order effect on local welfare. These results hold whether wages respond to migration or not.
\label{p:welfare}\end{proposition}

When the labor market is inefficiently slack, a further decrease in tightness---keeping labor force constant---reduces welfare. From the perspective of local workers, a drop in tightness caused by in-migration is equivalent to a drop in tightness keeping labor force constant. The welfare generated by the labor-force increase, keeping tightness constant, goes to migrants and does not count toward local welfare. Hence, in-migration reduces local welfare when the labor market is inefficiently slack by further reducing tightness. Conversely, in-migration improves local welfare when the labor market is inefficiently tight by bringing tightness toward efficiency.

The result that immigration improves welfare when the labor market is excessively tight explains why governments have historically relied on immigration to address labor shortages in their economies. \citet[section~5.1]{HWZ17} provide an example of such policy. In the postwar boom of the 1950s and 1960s, growing labor shortages were hindering the Austrian economy. To alleviate those shortages, the Austrian government enacted policies designed to attract foreign workers from southern Europe. More immigrant workers were allowed in regions of Austria where unemployment was particularly low, as proposition~\ref{p:welfare} would recommend.

In the model, local welfare is the sum of local labor income plus local firm profits, so local welfare is the same as total local income. The results of proposition~\ref{p:welfare} can therefore be reframed in income terms:
\begin{corollary}
In-migration reduces total local income---local labor income plus local firm profits---when the labor market is inefficiently slack, but it raises total local income when the labor market is inefficiently tight. These results hold whether wages respond to migration or not.
\label{c:welfare}\end{corollary}

The results of the corollary help explain the findings by \citet{CCI24} that unauthorized immigration boosts the financial health of local governments when the local labor market is tight, but it damages their financial health when the local labor market is slack. \citeauthor{CCI24} find that in areas with tight labor markets, unauthorized immigration lowers US municipal bond yields. So in such situations, lenders believe that  immigration boosts the fiscal strength of the issuing jurisdiction. Since unauthorized immigrants do not pay local taxes, it must be that the taxes paid by local workers and firm owners go up, which occurs naturally if local income goes up with in-migration. 

Conversely, in slack labor markets, \citet{CCI24} find that unauthorized immigration raises US municipal bond yields, suggesting that immigration increases the fiscal strain imposed on the issuing jurisdiction. This occurs naturally if local income goes down with in-migration in slack labor markets, as corollary~\ref{c:welfare} predicts. An additional source of fiscal strain documented by \citeauthor{CCI24} is that unauthorized immigration requires more spending on public services, including education and law enforcement (this occurs whether the labor market is slack or tight).

\subsection{Optimal in-migration over the business cycle}\label{s:optimal}

Proposition~\ref{p:welfare} shows that in-migration improves welfare when the labor market is too tight. I now turn to the next question: what is the optimal amount of in-migration when the labor market is initially too tight? The question boils down to finding the migration factor $m^*$ such that the elasticity $\e^{\Wc}_{m} = 0$. At that migration factor, local welfare is maximized so migration is optimal.

Using \eqref{e:ewm}, the optimality condition $\e^{\Wc}_{m} = 0$ becomes
\begin{align*}
\frac{\a m}{\a m + 1- \a} - \b & = \frac{\a-\b}{\a-\a \e^{d}_{\t}/\e^s_{\t}}.
\end{align*}
Using the expressions \eqref{e:est} and \eqref{e:edt} for the elasticities of demand and supply with respect to tightness, I rewrite the condition as
\begin{align}
\frac{\a m}{\a m + 1- \a} - \b & = \frac{\a-\b}{\a+(1-\a) \frac{\h}{1-\h} \cdot\frac{\tau(\t)}{u(\t)}}.
\label{e:fixedpointwage}\end{align}

Since tightness is itself a function of the migration factor $m$, the optimal in-migration is the solution to a fixed-point problem. The optimal amount of in-migration $\hat{m}>1$ solves the fixed-point equation \eqref{e:fixedpointwage}. The left-hand side of the equation is an increasing function of $m$. At $m=1$, the left-hand side is $\a-\b$, so at optimum in-migration $\hat{m}>1$, the left-hand side is strictly greater than $\a-\b$. The implication is that at optimum in-migration, the right-hand side is also strictly greater than $\a-\b$. This requires the denominator of the right-hand side to be less than 1, and accordingly
\begin{equation*}
\frac{\h}{1-\h} \cdot\frac{\tau(\t)}{u(\t)}<1.
\end{equation*}
Hence, the labor-market tightness must be strictly less than the efficient tightness $\t^*$, which is defined by \eqref{e:efficiency1}. Thus, at optimum in-migration, the labor market is inefficiently slack. The proposition below follows:
\begin{proposition}
From an initial situation with no migration and an inefficiently tight labor market, the optimal amount of in-migration brings the labor market to an inefficiently slack situation. From an initial situation with no migration and an inefficiently slack labor market, the optimal amount of in-migration is zero.
\label{p:implicit}\end{proposition}

Equation \eqref{e:fixedpointwage} defines the optimum in-migration only implicitly, but it nevertheless offers interesting insights. The most important one is that if migration is the only tool available to policymakers, and if policymakers aim to maximize local welfare, then the labor market should always be inefficiently slack. Otherwise migration is suboptimal: more in-migration would improve welfare.

What is the intuition for the result that optimal migration brings the labor market to a slack situation? At the efficient tightness, a drop in tightness reduces labor income, but this reduction is entirely offset by an increase in local profits, so local welfare (the sum of labor income and profits) is unaffected. As in-migration increases, the share of labor income going to local workers shrinks, while all profits continue to go to local firm owners. So as in-migration increases, the profit motive plays an increasingly large role in welfare. This means that at the efficient tightness $\t^*$, if in-migration is already positive, a decrease in tightness raises local profits more than it reduces labor income---which means that a drop in tightness raises welfare. Accordingly, it is optimal to allow some more in-migration to reduce tightness further, into the slack territory.

\subsection{Feasibility of a cyclical migration policy}

Given the documented effects of migration on unemployment, a cyclical migration policy would be desirable. Such cyclical immigration policy might not be difficult to implement. In fact, in 2008 France enacted a policy that makes it easier to hire immigrant workers for firms facing a high labor-market tightness \citep{S24}. The policy was targeted to narrow skills, but it is conceivable that a similar policy could be implemented based on the national labor-market tightness.

In the United States, such cyclical policy could be implemented very much like unemployment insurance. The duration of unemployment insurance is automatically extended when the unemployment rate in a state reaches specific thresholds \citep{M17}. Typically, unemployment benefits last for 26 weeks. When the state unemployment rate reaches 6.5\%, the duration of benefits in that state is automatically extended by 13 weeks, to reach 39 weeks. And when the state unemployment rate further increases to 8\%, the duration of benefits is automatically extended by 20 weeks, to reach 46 weeks. 

A system similar to the unemployment insurance system could be implemented for immigration policies. The number of immigrants authorized in the country or in specific states, or the number of specific visas, could depend on the national unemployment rate or the unemployment rate in specific states. More immigrants would be allowed to enter when the labor market is inefficiently tight. This would alleviate recruiting difficulties for firms and improve social welfare. When the labor market is inefficiently slack, as it was for instance in the aftermath of the 2008 financial crisis, immigration could be curbed, which would protect local workers who are already struggling to find jobs. Such policy would again improve welfare.

One practical difficulty is that the duration of business cycles is virtually impossible to forecast. If the economy is hot and a lot of migrant workers are brought in to cool the labor market, but the economy suddenly turns around due to some unexpected headwinds, the policy might not improve welfare after all. One advantage of migration policy is that in large part it operates through visas that are valid for a fixed period time. In the postwar period, 1945--2020, the average US business cycle lasts about 6 years \citep{NBER23}. So multi-year work visas fit squarely within the business-cycle timeframe. In fact, longer work visas could be granted at the beginning of expansions, and shorter one once the economy starts cooling.

\section{Political support and opposition to immigration}\label{s:politics}

The previous sections looked at the positive and normative implications of the migration model. This section turns to the political implications of the model. The model predicts the impact of migration in the short run on two constituencies--- workers and firm owners---as well as the overall impact of migration on local welfare. From this, it is easy to infer the political support that immigration might receive from different groups under different circumstances.

\subsection{Opposition to immigration from pro-labor parties} 

Let's focus further on pro-labor parties, that design their political programs solely to improve the welfare of workers. Since immigration always hurts the welfare of local workers, such parties are expected to always oppose immigration. 

Moreover, since the elasticity of the employment rate with respect to the labor force is more negative in bad times, the model suggests that opposition to immigration from pro-labor parties, and accusations of job stealing, are louder in bad times.

For the same reasons, labor unions are expected to be opposed to immigration, since immigration reduces labor income. Evidence from the United States supports this prediction. The American Federation of Labor and other white working-class activists were  pushing for restrictions on immigration at the end of the nineteenth century and onset of the twentieth century \citep{F64,J21}. 

Lobbying around the Chinese Exclusion Act provides a good illustration. The Chinese Exclusion Act was passed in 1882 to block immigration from Chinese laborers into the United States. The Act initially lasted 10 years, was then extended for another 10 years in 1892, and then made permanent in 1902 \citep[p.~8]{LMQ24}. In 1881, before the Chinese Exclusion Act was enacted, the AFL adopted the following resolution at their convention ``Whereas the experience of the last thirty years in California and on the Pacific coast having proved conclusively that the presence of Chinese and their competition with free white labor is one of the greatest evils with which any country can be afflicted: Therefore be it resolved, that we use our best efforts to get rid of this monstrous evil'' \citep[p.~25]{AFL02}. In 1902, the leaders of the AFL told Congress that letting the Act lapse would result in ``hundreds of thousands of our citizens to be deprived of employment to make room for this Asiatic coolie'', which would lead to a drastic reduction in ``the standard of living of our entire laboring class'' \citep[p.~23]{AFL02}. They also wrote that before the Chinese Exclusion Act, white workers were losing jobs in large number: ``Soon after the negotiation of the Burlingame treaty in 1868 large
numbers of Chinese coolies were brought to this country under contract. Their numbers so increased that in 1878 the people of [California] made a practically unanimous demand for the restriction of immigration. Our white population suffered in every department of labor and trade, having in numerous instances been driven out of employment
by the competition of the Chinese'' \citep[p.~25]{AFL02}.\footnote{The argument against Chinese immigration was not only economic but also racist and xenophobic \citep{J21}. The \citet[p.~29]{AFL02} pleads with Congress to extend the Chinese Exclusion Act for several reasons, including ``the absolute necessity of keeping up the standard of population and not permitting it to deteriorate by contact with inferior and nonassimilative races.''}

And a final example supporting the model's prediction: After World War 2, German unions also opposed the entry of foreign workers by arguing that they competed for jobs with native workers \citep[p.~24]{AT24}.

\subsection{Support for immigration from pro-business parties}

Let's now turn to pro-business parties. These political parties design their political programs to improve the welfare of business and firm owners. Since immigration always improves the profits of firms, such parties are expected to favor immigration.

The debate around the Chinese Exclusion Act provides again a good illustration. While labor organizations supported the Act, business owners generally opposed it \citep{LMQ24}. These business owners were supported by pro-business Republicans, who welcomed Chinese workers \citep[p.~8]{J21}. One such businessman who testified in Congress on behalf of Chinese workers is Charles Crocker, who was president of the Southern Pacific Railroad, and who hired a large number of Chinese immigrants for the construction of the first transcontinental railroad. In Congress, he argued that ``without Chinese labor we
would be thrown back in all the branches of industry, farming, mining, reclaiming lands, and everything else'' \citep[p.~19]{Z19}.

Pro-business parties typically belong to what \citet{GMP22} call the ``merchant right.'' These parties, such as the Republican Party, now often have anti-immigration views and promote anti-immigration policies. This is inconsistent with the model's prediction. 

But \citet{GMP22} offer a possible reconciliation. While high-income and high-education voters typically supported the merchant right, the last decades have witnessed a drastic shift. Low-education voters have now joined the high-income voters to support these right-wing parties. Voters with low education, however, face much higher unemployment than voters with high education. For example, in the United States, the average unemployment rate is 2.7\% for workers with a college degree, whereas it is 9.3\% for workers who dropped out of high school \citep[table~6]{CC18}. The reason is that jobs are much less stable for workers with low education, so they cycle much more often through unemployment, and therefore spend much more time looking for jobs. This means that these workers are much more exposed to job competition from immigrants than highly educated workers. This would be especially true if immigrants are less educated than the average local worker. In this context, the merchant-right parties may have taken an anti-immigration stance to cater to their new constituency of less educated voters---at the expense of their business-minded voters.

\section{Numerical illustration}\label{s:illustration}

I now use the results of proposition~\ref{p:stealing} to illustrate what the elasticity of the employment rate with respect to the labor force might be at different stages of the business cycle, for different values of labor-market tightness.

\subsection{Calibration of the model}\label{s:calibration}

In theory, the matching model can capture all possible effects of immigration on the labor market: drop in local wages or not, and rise in local unemployment rate or not, occurring together or separately. In practice, the model should therefore be calibrated based on the relevant evidence. Here I calibrate the model to the US labor market.

\paragraph{Concavity of the production function} Before it started falling in the 1990s, the US labor share was hovering around 65\% \citep[figure~1]{ADK20}. The global labor share was also at 65\% in 1980 \citep[figure~1]{KN14}. Accordingly, I set the concavity parameter in the production function \eqref{e:production} to $\a=0.35$, which produces a labor share of $1-\a = 65\%$. Targeting a lower labor share would require a  higher $\a$, which would generate more migration-induced unemployment (as showed by proposition~\ref{p:stealing}).

\paragraph{Rigidity of real wages} I set the rigidity parameter in the wage function \eqref{e:wageGeneral} to $\g = 0.3$, as in \citet[table~1]{M12}. Such a calibration produces an elasticity of real wages with respect to productivity of $1-\g = 0.7$, which is in line with the findings by \citet[table~6]{HSV13} and is low enough to generate realistic business cycles \citep[section~4B]{M12}.


\begin{table}[t]
\caption{Parameter values in simulations}
\begin{tabular*}{\textwidth}{@{\extracolsep{\fill}}llp{7cm}}\toprule
Value & Description & Source or target\\
\midrule
\multicolumn{3}{c}{A. Calibration from direct measurement}\\
$\g = 0.3$ & Rigidity of real wages & \citet{M12} \\
$\b = 0$ & Migration elasticity of real wages &  \citet{OP12}, \citet{C12a}\\
$\k = 1$ & Recruiting cost & \citet{MS24} \\
$\l = 0.097$ & Quarterly job-separation rate & \citet{MS21b}\\
\midrule
\multicolumn{3}{c}{B. Calibration to match target}\\
$\a = 0.35$ & Decreasing returns to labor & Labor share $=65\%$ \citep{ADK20} \\
$\h = 0.48$& Unemployment elasticity of matching & Beveridge elasticity $=1$\citep{MS24}\\
$\m = 2.16$ & Quarterly matching efficacy & $u^* = 4.3\%$ \citep{MS24}\\
$\o = 0.64$ & Level of real wages & $u = u^*$ when $ a =1$ (normalization)\\
\bottomrule\end{tabular*}
\note{The parameter values described in the table are obtained in Section~\ref{s:calibration}.}
\label{t:calibration}\end{table}



\paragraph{Migration elasticity of real wages} The empirical response of wages to migration is captured through the elasticity $\b$ in the wage norm \eqref{e:wageGeneral}. Available evidence suggests that migration does not generally have significant effects on local wages \citep{FH95,OP12,C12a,BK15}. So as a baseline I simply set $\b=0$. I will also explore the behavior of the model for all values of $\b$ between 0 and 0.35 (the value of $\a$, corresponding to a flexible wage).

\paragraph{Recruiting cost}  I set the recruiting cost to $\k = 1$, following \citet[section~2C]{MS24}. This calibration is based on the evidence provided by \citet{GMV18} and \citet{MS21b}. It implies that it takes 1 full-time worker to service a job vacancy.

\paragraph{Job-separation rate} I set the quarterly job-separation rate to $\l = 0.097$, which is its average value in the United States between 1951 and 2019 \citep[appendix~B2]{MS21b}.

\paragraph{Matching elasticity} I set the elasticity parameter in the matching function \eqref{e:matching} so that the elasticity of the Beveridge curve is $1$, as it is in the United States \citet[section~2E]{MS24}. The matching elasticity $\h$ is related to the elasticity of the Beveridge curve $\e$ and unemployment rate $u$ by 
\begin{equation*}
\e = \frac{1}{1-\h}\bp{\h + \frac{u}{1-u}},
\end{equation*}
as showed by \citet[equation (12)]{MS21b}. Since $\h$ is fixed, the Beveridge curve is not exactly isoelastic, as $\e$ varies a little bit when $u$ varies. Hence, I target a Beveridge elasticity of $1$ when the unemployment rate is efficient. The above equation can be rewritten as
\begin{equation*}
\h = \frac{1}{1+\e} \bp{\e - \frac{u}{1 - u}}.
\end{equation*}
The average value of the efficient unemployment rate in the United States over 1951--2019 is $u^* = 4.3\%$ \citep[section 3A]{MS24}. I plug $u = 4.3\%$ and $\e=1$ in the above equation. I obtain $\h = 0.48$. This calibration is very close to the value $\h = 0.5$ which is standard in the macro-labor literature \citep{MP94,PP01,S04,HM08,M12}.

\paragraph{Matching efficacy} I calibrate the efficacy parameter in the matching function \eqref{e:matching} so that the location of the Beveridge curve is the same as in US data. The location of the Beveridge curve determines the efficient unemployment rate $u^*$, so the calibration ensures that the efficient unemployment rate is the same in the model as in the data. The average value of the efficient unemployment rate in the United States over 1951--2019 is $u^* = 4.3\%$, while the efficient tightness is $\t^* = 1$ \citep[section 3A]{MS24}. The efficient unemployment rate, efficient tightness, and matching efficacy are themselves related by
\begin{equation*}
\m = \frac{\l}{(\t^*)^{1-\h}} \cdot \frac{1 - u^*}{u^*}.
\end{equation*}
This equation is obtained by reshuffling \eqref{e:u}. Setting tightness and unemployment to their efficient levels, $\t^* = 1$ and $u^* = 4.3\%$, and setting $\l = 0.097$, this equation yields $\m = 2.16$.

\paragraph{Level of real wages} I calibrate the real-wage level so that the labor market is efficient for a productivity level of 1---which is just a normalization. Through the labor demand \eqref{e:ld}, the wage level is related to various parameters by
\begin{equation*}
\o =  \frac{(1 - \a) a^{\g}}{(1 - u^*)^\a} \cdot \bs{1 + \frac{\k\l}{\m \cdot (\t^*)^{-\h} - \k\l}}^{\a-1}.
\end{equation*}
With $\a = 0.35$, $a=1$, $u^* = 0.043$, $\k=1$, $\l = 0.097$, $\m = 2.16$, $\t^* =1$, the equation gives $\o = 0.64$.

\subsection{Simulation of the effects of in-migration}

Next, I simulate the calibrated model to quantify the effects of migration on unemployment at different stages of the business cycle (figure~\ref{f:illustration}). The simulations also illustrate how to model operates.


\begin{figure}[p]
\subcaptionbox{Source of business cycles\label{f:atheta}}{\includegraphics[scale=\scale,page=23]{\pdf}}\hfill
\subcaptionbox{Unemployment over the business cycle\label{f:thetau}}{\includegraphics[scale=\scale,page=24]{\pdf}}\fspace
\subcaptionbox{Beveridge curve\label{f:uv}}{\includegraphics[scale=\scale,page=25]{\pdf}}\hfill
\subcaptionbox{Migration elasticity of employment\label{f:thetaelh}}{\includegraphics[scale=\scale,page=26]{\pdf}}\fspace
\subcaptionbox{Migration-induced unemployment\label{f:thetastealing}}{\includegraphics[scale=\scale,page=28]{\pdf}}\hfill
\subcaptionbox{Migration elasticity of employment for different wage responses\label{f:elhwage}}{\includegraphics[scale=\scale,page=27]{\pdf}}
\caption{Effects of in-migration over the business cycle}
\note{The figure is obtained by simulating the labor-market model under the calibration presented in table~\ref{t:calibration}. The pink dots and lines represent the efficient location of the labor market. The set of lines in panel F represents the migration elasticity of employment for different migration elasticities of real wages, from $\b=0$ (unresponsive wages) to $\b = 0.35$ (fully responsive wages).}
\label{f:illustration}\end{figure}

In the model, business-cycle fluctuations are generated by productivity shocks (figure \ref{f:atheta}). When productivity is below 1, the labor market is inefficiently slack. In that situation too many workers are unemployed. When productivity is at 1, the labor market is efficient. And when productivity is above 1, the labor market is inefficiently tight. In that situation, too many workers are allocated to recruiting.

The response of tightness to productivity in the model is comparable to that observed in the United States, confirming that the model is not subject to the \citet{S05} puzzle. In the United States, the elasticity of tightness with respect to productivity is 8.6 \citep[p.~1741]{M12}. In the model, when productivity falls from 1 to 0.95, tightness drops from 1 to 0.5. So when productivity falls by 5\%, tightness drops by 50\%, which corresponds to an elasticity of $50\%/5\% = 10$. Hence, thanks to wage rigidity, the model produces realistic fluctuations in tightness in response to productivity shocks.

When tightness fluctuates, it becomes harder or easier for jobseekers to find jobs, so the unemployment rate fluctuates (figure~\ref{f:thetau}). When the labor market is efficient, tightness is 1 and the unemployment rate is 4.3\%. When tightness rises to 1.5, the unemployment rate falls to 3.5\%. When tightness falls to 0.5, the unemployment rate rises to 6.1\%. If tightness falls further to 0.25, unemployment rises further to 8.5\%.

Fluctuations in productivity move the labor market along the Beveridge curve (figure~\ref{f:uv}). When the unemployment rate falls, the vacancy rate rises; conversely, when the unemployment rate rises, the vacancy rate falls. By calibration, when the labor market is efficient,  the unemployment and vacancy rates are equal, and equal 4.3\%. When the labor market is inefficiently tight, the vacancy rate is above 4.3\% and the unemployment rate is below 4.3\%. When the labor market is inefficiently slack, the vacancy rate is below 4.3\% and the unemployment rate is above 4.3\%.

From proposition~\ref{p:stealing} and the calibration in table~\ref{t:calibration}, I infer that the elasticity of employment with respect to migration is $-1$ when tightness is $0$, it grows to $-\a = -0.35$ when tightness reaches 1, and it asymptotes to $0$ when tightness converges to $\t_{\tau}$ (which is a very large number). The model simulations are useful to display how rapidly the migration elasticity of employment falls when tightness drops (figure~\ref{f:thetaelh}). When tightness is 0.5, the migration elasticity of employment falls to $-0.52$, implying that a wave of in-migration amounting to 1\% of the labor force reduces the number of local workers who are employed by 0.52\%. By contrast, when tightness is 1.5, the migration elasticity of employment is only half of that, at $-0.26$, implying that the reduction in the number of employed local workers is only half of what it is with a tightness of 0.5. Hence, the response of employment to migration is strongly state-dependent.

Accordingly, the amount of migration-induced unemployment is substantial and larger when the labor market is slacker (figure~\ref{f:thetastealing}). When labor-market tightness is at its efficient level of 1, 33 local workers become unemployed when 100 migrants arrive in their labor market. When labor-market tightness is 0.5, corresponding to a somewhat slack labor market, the number of local workers who become unemployed raises to 49. When labor-market tightness is 0.1, corresponding to an extremely slack labor market, the number of local workers who become unemployed raises further to 73. By contrast, when labor-market tightness is 2, corresponding to a very tight labor market, the number of local workers who become unemployed falls to 20.\footnote{When tightness becomes extremely low, below 0.05, the amount of migration-induced unemployment starts falling and converges to 0 when tightness goes to 0. This is just a mechanical effect: when tightness goes to 0, all workers become unemployed, so it becomes physically impossible for migration to push more employed workers into unemployment---since there are no employed workers left. This mechanical effect explains why the statistic $du/d\ln H$, depicted in figure~\ref{f:thetastealing}, is not monotonic, whereas the elasticity $\oex{l}{H}$, depicted in figure~\ref{f:thetaelh}, is monotonic.}

The amount of migration-induced unemployment in the calibrated model is commensurate to the amount observed in migration experiments (table~\ref{t:evidence}). On average across experiments, 37 local workers become unemployed for 100 arrivals. In the model, when the labor market is efficient, 33 local workers become unemployed for 100 arrivals. In the United States, the labor market is generally inefficiently slack: tightness averages $0.73<1$ between 1930 and 2024 \citep[section~3D]{MS24}. At that tightness of 0.73, 40 local workers become unemployed for 100 arrivals. These simulated numbers are close to the experimental numbers. 

The range of results in the simulations is also comparable to the range of empirical results. \citet{H92} finds that the return of 100 repatriates from Algeria pushed 20 French workers into unemployment. This number might seem low, but the labor market was also extremely tight in France at the time. When repatriation started in 1962, the unemployment rate in France was only 1\% \citep[table~1]{H92}. At a tightness of 2, which is extremely high for the United States (since World War 2, it has only been reached during the pandemic), the calibrated model predicts that 20 local workers are pushed into unemployment for 100 arrivals. So the very tight labor market in France in the 1960s might explain why the repatriation only had muted effects on the local employment rate. 

The same observations apply to the results on the Portuguese returnees. \citet{M17a} finds that only 22 local workers become unemployed for 100 returns. Once again, this low number can be explained by the very tight labor market in Portugal at the time: the unemployment rate was only about 2\% in 1974 \citep[figure~4]{M17a}. Such tight market helped absorb the returnees into the workforce.

By contrast, the labor market was quite slack in Europe in the 1990s. In 1995, the unemployment rate in European OECD countries averaged 10.3\% \citep[table~1]{LS98}. Such slack might explain why  found that immigration from Yugoslavia affected local workers so adversely in the 1990s. For instance, the midpoint estimate found by \citet{AK03} is that 59 European workers were pushed to unemployment when 100 Yugoslavian workers entered their local labor market. This is comparable with the effect that in-migration would have in the model for a tightness of 0.3, or equivalently an unemployment rate of 7.8\%.

Finally, the elasticity of employment with respect to migration shrinks to zero as the response of wages to migration grows (figure~\ref{f:elhwage}). When real wages fall in response to in-migration, then in-migration stimulates not only labor supply but also labor demand, so tightness falls less when in-migration occurs, meaning that local employment falls less and the elasticity is closer to 0. When the elasticity of real wages with respect to migration is high enough to equal the concavity parameter in the production function ($\b = \a = 0.35$), wages are so responsive that tightness and employment are unaffected by migration, and the migration elasticity of employment is just 0.

\subsection{Simulation of the optimal migration policy}

Last, I simulate the calibrated model to quantify the optimal migration policy over the business cycle (figure~\ref{f:policy}).

\begin{figure}[p]
\subcaptionbox{Migration elasticity of welfare over the business cycle\label{f:ewh}}{\includegraphics[scale=\scale,page=29]{\pdf}}\hfill
\subcaptionbox{Migration elasticity of welfare for different wage responses\label{f:ewhwages}}{\includegraphics[scale=\scale,page=30]{\pdf}}\fspace
\subcaptionbox{Optimal in-migration over the business cycle\label{f:istar}}{\includegraphics[scale=\scale,page=31]{\pdf}}\hfill
\subcaptionbox{Optimal in-migration for different wage responses\label{f:istarwages}}{\includegraphics[scale=\scale,page=32]{\pdf}}\fspace
\subcaptionbox{Optimal tightness over the business cycle\label{f:thetastar}}{\includegraphics[scale=\scale,page=33]{\pdf}}\hfill
\subcaptionbox{Optimal tightness for different wage responses\label{f:thetastarwages}}{\includegraphics[scale=\scale,page=34]{\pdf}}
\caption{Optimal in-migration over the business cycle}
\note{The figure is obtained by simulating the labor-market model under the calibration presented in table~\ref{t:calibration}. The pink dots represent the efficient location of the labor market. The set of lines in panel B, D, and F describe how the results change for different migration elasticities of real wages, from $\b=0$ (unresponsive wages) to $\b = 0.15$ (somewhat responsive wages).}
\label{f:policy}\end{figure}

First, as predicted by proposition~\ref{p:welfare}, the elasticity of social welfare with respect to migration depends on the health of the labor market (figure~\ref{f:ewh}). This means that the marginal effect of migration on welfare depends on the health of the labor market. If the labor market is operating efficiently, migration has no marginal effect on welfare. If the labor market is inefficiently slack, a small wave of in-migration reduces welfare. By contrast, if the labor market is inefficiently tight, a small wave of in-migration improves welfare. These results hold whether in-migration depresses local wages or not, but the elasticity of welfare with respect to migration is attenuated when wages respond more to migration (figure~\ref{f:ewhwages}).

Because of how migration affects social welfare, the optimal in-migration policy is sharply procyclical (figure~\ref{f:istar}). When the labor market is inefficiently slack, in-migration reduces welfare, so it is optimal to disallow in-migration. When the labor market is inefficiently tight, in-migration improves welfare, so it is optimal to allow in-migration. For instance, if the initial tightness is 1.2 instead of 1, then optimal in-migration is 1\% of the labor force. For context, such optimal in-migration would amount to 1.7 million immigrants in the United States today \citep{CLF16OV}. This is about twice the yearly amount of net in-migration in the United States over 2010--2019, and about half the amount of net in-migration in the United States in 2023 \citep[p.~5]{CBO24}. Furthermore, more in-migration should be allowed when the labor market is initially tighter. If the initial tightness is 1.4 instead of 1.2, the optimal amount of in-migration doubles to 2\% of the labor force.

When the labor market is initially too hot and in-migration is optimal, labor-market tightness falls below 1, but not by much (figure~\ref{f:thetastar}). The optimal tightness, which prevails under optimal migration, is 0.99 when then initial tightness is 1.2, and 0.98 when the initial tightness is 1.4. So roughly speaking, optimal in-migration should be sufficient to bring tightness down to its efficient level of 1.

When wages fall in response to in-migration, the optimal amount of in-migration is larger (figure~\ref{f:istarwages}). This result might seem surprising, given that the migration elasticity of welfare is less positive when wages fall (figure~\ref{f:ewhwages}). But this is just a marginal result, which determines the sign but not the optimal amount of migration. Instead, we saw that optimal migration should bring tightness close to 1 (figure~\ref{f:thetastar}). That remains true when wages respond to migration (figure~\ref{f:thetastarwages}). At the same time, the effect of migration on tightness falls when wages respond more (figure~\ref{f:elhwage}). Accordingly, since migration is less potent when wages respond more, a larger amount of in-migration is required to bring tightness to the vicinity of its efficient level of 1.


\section{Reconciling the Cardian and Borjasian perspectives on immigration}\label{s:perspectives}

The model in this paper offers a broader theory of migration, which captures the effects of migration not only on local wages but also on the local unemployment rate. By accounting for the response of unemployment to migration, the model can bridge the gap between the two branches of the immigration literature: the Cardian branch and the Borjasian branch.

\subsection{The two branches of the immigration literature}

The immigration literature is sharply divided. A first branch of the literature argues that the labor demand is downward-sloping so the number of jobs available in the labor market is somewhat limited \citep{B03}. Under this perspective, the arrival of immigrants reduces the opportunities available to local workers. A second branch of the literature argues that the arrival of immigrants does not reduce the wage of local workers \citep{C90}. Under this perspective, the arrival of immigrants does not affect local workers who are employed. 

Because this model is based on the matching framework, it is more flexible than the Walrasian model, and it allows us to reconcile the Cardian and Borjasian perspectives on immigration. In a matching model, both tightness and wage may adjust to equilibrate the market---not just the wage, as in a Walrasian model \citep{MS15}. 

First, the labor demand curve is downward sloping in the model, just as in the Borjasian perspective (figure~\ref{f:normaltimes}). Hence, the number of jobs available in the labor market is somewhat limited, explaining why immigrants take jobs away from local workers.

The novelty is that the adjustment to immigration does not happen only through wages but also through labor-market tightness, which itself determines workers' job-finding rate and the unemployment rate (figure~\ref{f:wagerigid}). In fact, in the basic version of the model, wages do not respond to immigration, so just as in the Cardian perspective, the arrival of immigrants does not reduce the wage of local workers, and thus does not affect locals who are employed (figure~\ref{f:wagefixed}).

\subsection{A general theory of migration}

In fact, the model offers a general theory of migration, which captures the effects of migration not only on local wages but also on the local unemployment rate. The model can of course describe the effects typically appearing in a Walrasian labor market. In addition, it can describe effects that cannot be described in the Walrasian framework.

For instance,  with a linear production function, the labor demand is horizontal. If the wage is also fixed, then tightness is unaffected by migration. In this purely Cardian case, the arrival of new workers in a labor market would affect neither local wages nor the local unemployment (figure~\ref{f:purecard}).

With a concave production function, the labor demand becomes downward sloping. Then, if wages drop enough when immigrants arrive, the model can produce a purely Borjasian scenario, whereby local wages drop with immigration, but the unemployment rate is unaffected (figure~\ref{f:pureborjas}). In that case, just like in the Cardian and Borjasian traditions, all adjustment to immigration occurs through wages. There is no adjustment of tightness or unemployment rate.

\begin{figure}[p]
\subcaptionbox{Purely Cardian scenario ($\a = 0$, $\b = 0$): no response of wages or tightness \label{f:purecard}}{\includegraphics[scale=\scale,page=17]{\pdf}}\hfill
\subcaptionbox{Purely Borjasian scenario ($\a > 0$, $\b = \a$): drop in wages, no response of tightness \label{f:pureborjas}}{\includegraphics[scale=\scale,page=18]{\pdf}}\fspace
\subcaptionbox{Cardo-Borjasian scenario ($\a > 0$, $\b = 0$): no response of wages, drop in tightness \label{f:cardborjas}}{\includegraphics[scale=\scale,page=19]{\pdf}}\hfill
\subcaptionbox{General scenario ($\a > 0$, $\b \in (0,\a)$): drop in wages, drop in tightness \label{f:general}}{\includegraphics[scale=\scale,page=20]{\pdf}}
\caption{Possible effects of immigration in the model}
\note{The labor supply is given by \eqref{e:ls}. In The labor demand curve is given by \eqref{e:ld}. In panel A, the production function is linear and the real wage does not respond to migration: $\a = 0$ in \eqref{e:production} and $\b =0$ in \eqref{e:wage}. In panel B, the production function is concave and the real wage is flexible with respect to migration: $\a > 0$ in \eqref{e:production} and $\b = \a$ in \eqref{e:wage}. In panel C, the production function is concave and the real wage does not respond to migration: $\a > 0$ in \eqref{e:production} and $\b = 0$ in \eqref{e:wage}. In panel D, the production function is concave and the real wage is somewhat rigid with respect to migration: $\a > 0$ in \eqref{e:production} and $\b \in (0,\a)$ in \eqref{e:wage}.}
\label{f:scenarios}\end{figure}

In the Walrasian model, the Cardian and Borjasian views are incompatible, because the Borjasian view requires a downward-sloping labor demand (figure~\ref{f:borjas}) while the Cardian view requires a horizontal labor demand (figure~\ref{f:card}). In the matching model, however, they are compatible. A wave of immigration might increase the unemployment rate of local workers without affecting their wages (figure~\ref{f:cardborjas}).

In fact, the model flips the usual association between wage and employment responses that appears in the Walrasian world. In a Walrasian model, either neither local employment and wages respond to migration, or both local employment and wages respond to migration. Furthermore, the responses move in tandem, as determined by the slope of labor demand. If the labor demand curve is flat, both wage and employment responses are small; if the labor demand curve is steep, both wage and employment responses are large. In the matching model, this association breaks down. If wages fall not at all or little with in-migration, then labor demand is barely boosted by in-migration, so tightness falls significantly with in-migration, which means that local unemployment rises significantly and local employment declines significantly (figure~\ref{f:cardborjas}). On the contrary, if wages fall significantly with in-migration, then labor demand is boosted by in-migration, so tightness falls less, which implies that local unemployment rises little and local employment does not decline much (figure~\ref{f:general}). 

Because the model features unemployment, the effect of migration can be distributed on both employed and unemployed workers, and the response of wages determines the incidence of migration. If wages do not respond to migration, then migration only impacts unemployed workers (figure~\ref{f:cardborjas}). If wages decline a bit with in-migration, the impact is split between employed workers, who are paid less, and unemployed workers, who face a lower job-finding rate (figure~\ref{f:general}). If wages are perfectly flexible, then only employed workers are negatively impacted by in-migration (figure~\ref{f:borjas}). Through this mechanism, and unlike in the Walrasian model, the model predicts either a weak wage response and a strong employment response to migration, or a strong wage response with a weak employment response. This pattern seems consistent with a lot of evidence. \citet[p.~175]{G12} finds for instance ``a displacement effect of 3.1 unemployed workers for every 10 immigrants that find a job, but no effect on relative wages.'' Similarly, \citet[p.~435]{DSS17} find ``a moderate decline in local native wages and a sharp decline in local native employment.''

\section{Conclusion}\label{s:conclusion}

To conclude, I apply the paper's model and findings to the current immigration debate. I also connect the effects of migration on the labor market to the effects of public employment and unemployment insurance.

\subsection{Application to the recovery of the coronavirus pandemic}

In the United States the labor market has been incredibly tight in the recovery from the coronavirus pandemic. Labor-market tightness reached a value of 2 in 2022, a level it had not reached since the end of World War 2 \citep[figure~12]{MS24}.

A natural response to such inefficiently tight labor market is to tighten monetary policy, which the Fed did in 2022Q2, one year after the labor market turned inefficiently tight \citep{FEDFUNDS}. However, the labor market has been slow to cool, maybe because monetary policy percolates only slowly to the labor market \citep{C12}. In 2024Q2, the labor market is still inefficiently tight, albeit much closer to efficiency.

Given such delays in deciding to tighten monetary policy, and then delays for monetary policy to reach the labor market, allowing for some more immigration between 2021Q2 and 2024Q2 would have rapidly cooled the labor market and improved the welfare of US workers. Net immigration was substantial in 2022 and 2023 compared to previous years \citep[p.~5]{CBO24}, but not enough to bring tightness below 1, so immigration remained suboptimal.

\subsection{Connection to results on unemployment insurance and public employment}

Finally, I connect the effects of migration on the labor market to the effects of unemployment insurance and public employment. Migration, unemployment insurance, and public employment shift labor supply relative to labor demand (migration, unemployment insurance) or labor demand relative to labor supply (public employment), so their effects are connected. The optimal cyclicality of public employment, unemployment insurance, and migration policy are also all connected.

\paragraph{Public employment} Equation \eqref{e:elh1} shows that the effect of in-migration on employment is determined by the ratio between the elasticities of labor supply and demand with respect to tightness, $\e^d_{\t}/\e^s_{\t}$. This ratio captures the relative slopes of supply and demand. This ratio also determines the size of the public-employment multiplier $\l$, as showed by \citet[equation~(8)]{M14}: 
\begin{equation}
\l = 1- \frac{1}{1 - \e^s_{\t}/\e^d_{\t}} = \frac{1}{1 -\e^d_{\t}/\e^s_{\t}} = \e^l_H.
\label{e:lambda}\end{equation}

Equation \eqref{e:lambda} shows that the public-employment multiplier is the same as the elasticity of local employment with respect to the labor force when wages do not respond to migration. It is for the same reasons that the public-employment multiplier is positive and larger in bad times and that the elasticity of local employment with respect to in-migration is negative and lower in bad times. Because the number of jobs in the private sectors is somewhat limited, creating public-sector jobs will increase employment. And because the number of jobs in the private sectors is somewhat limited, the arrival of migrants will take some jobs away from local workers and reduce local employment. 

When the labor market is more depressed, the crowding out of private jobs by public jobs is less because private firms are not hurt much by public vacancies. This is because there are so many workers looking for jobs. At the same time, private firms will not benefit much from the presence of migrant jobseekers. Again, this is because there are already so many jobseekers available. So private firms will not create many jobs when migrants arrive in bad times; therefore, migrants end up taking jobs away from local workers.

\paragraph{Optimal public employment} In the United States, the unemployment gap---the gap between the actual and efficient unemployment rates---is sharply countercyclical \citep{MS21b,MS24}. As a result, if the public-employment multiplier is positive, the optimal public employment is countercyclical: higher in bad times than in good times \citep{MS19}. This optimal policy result parallels the one presented in proposition~\ref{p:implicit} that optimal in-migration is positive when the unemployment rate is inefficiently low but zero when the unemployment rate is inefficiently high. The logic behind both results is the same. To improve welfare, policy ought to bring labor-market tightness closer to its efficient level. By raising public employment, tightness can be increased; and by allowing in-migration, tightness can be lowered. This is why more public employment is desirable in bad times while more in-migration is desirable in good times.

\paragraph{Unemployment insurance} Because of job rationing, migration affects local workers. Via the same mechanism, the macro effect of unemployment insurance on unemployment is less than its micro effect \citep{LMS18a}. When a jobseeker receives unemployment insurance that is less generous, she searches more intensely, which improves the chances that she finds a job. If all jobseekers do that, however, competition for jobs becomes fiercer. There is a rat race for jobs, and the return on each unit of search effort falls. As a result, the additional number of jobseekers who find jobs after the reduction in unemployment insurance is not as much as what the increased search effort would suggest. While job-search effort goes up, tightness goes down, so the job-finding rate does not increase as much as search effort. Similarly, if many migrants enter the labor market, competition for jobs becomes fiercer, tightness and the job-finding rate fall, so it is more difficult for local jobseekers to find jobs. This induces locals to feel that their jobs are taken away.

Furthermore, just like the arrival of migrants has larger effects on the employment rate of local workers in bad times, when tightness is low, the gap between macro and micro effects of unemployment insurance is larger in bad times. This is because jobs are scarcer in bad times, so the rat race is stronger among jobseekers.

\paragraph{Optimal unemployment insurance}  Just like the unemployment gap is countercyclical, the tightness gap---the gap between the actual and efficient labor-market tightness---is sharply procyclical \citep{LMS18b,MS21b,MS24}. In a model with job rationing, an increase in unemployment insurance reduces job-search effort, which lessens competition for jobs and raises tightness. As a result, the optimal generosity of unemployment insurance is countercyclical \citep{LMS18a}. Again, having higher unemployment insurance in bad times raises tightness just when tightness is inefficiently low, which improves welfare. Through a parallel mechanism., having more in-migration in good times lowers tightness just when tightness is inefficiently high, which improves welfare. 

\paragraph{Discussion} The model therefore connects a range of policies that affects the labor market. If the world features job rationing, then all the following properties must be true:
\begin{enumerate}
\item Public employment does not completely crowd out private employment so the government multiplier is positive. The multiplier is especially large in bad times.
\item The increase in unemployment caused by an increase in unemployment insurance is not as large as what the reduction in job-search effort would suggest. The increase in unemployment is especially small in bad times. 
\item In-migration takes jobs away from local workers so local employment falls with in-migration. The fall is especially large in bad times.
\end{enumerate}
None of these properties can be true without the others. This means for instance that it would not be logical to believe jointly that government spending can stimulate employment and that local workers are unaffected by immigration. It would not be logical either to believe jointly that unemployment insurance has a substantial adverse effect on unemployment and that local workers are adversely affected by immigration.

The same holds for optimal policies. In a world with job rationing and procyclical tightness gap, all the following policy statements are true:
\begin{enumerate}
\item Optimal public employment is countercyclical, as public employment raises tightness.
\item Optimal unemployment insurance is countercyclical, unemployment insurance raises tightness.
\item Optimal in-migration is procyclical, as in-migration lower tightness.
\end{enumerate}

\bibliography{\bib}
\end{document}