\section{Introduction}

The direct current (DC) resistivity method is a geophysical technique used for
non-invasive investigations in a wide range of applications and is best suited
to detect subsurface saturation and pollution, as underground fluids and
dissolved ions strongly control subsurface resistivities \cite{Binley2005}.

Our target application is the geoelectrical monitoring of municipal solid waste
landfills (MSWLFs). In more detail, we want to explore the capability of the DC
method to characterize the high- or low-density polyethylene (HDPE or LDPE)
liner that is generally placed underneath the landfill to prevent leachate
leakage in the underground. The method is promising to discriminate between the
plastic liner, that is practically an electrical insulator, from leachate that
can have resistivity as low as 1 \sib{\ohm\meter}. The scientific literature reports several
case studies where geoelectrical methods have been applied to landfills with
different design, size, and confining materials, consisting of diverse types of
waste, and located either in urbanized or remote areas
\cite{Bernstone2000,MEJU2000115,Leroux2010,VAUDELET2011738,DeCarlo2013,Tsourlos2014,DIMAIO2018629}.
In many cases, the aim of the
studies was to delineate the extent of the buried waste body and DC measurements
have been conducted together with Induced Polarization ones
(\cite{Dahlin2010,DEDONNO2017302}). Some numerical and down-scaled laboratory
studies have been devoted to detecting and locating ruptures and holes in the
plastic liner with set ups with different levels of complexity
(\cite{Frangos1997,Binley1997,Binley2003,Ling2019,Aguzzoli2020}).
However, the geoelectrical method aimed to analyse the integrity of the
geomembrane at the field scale still lacks thorough validation and issues
related to the survey design, the investigation depth and to the spatial
resolution need to be addressed (\cite{aguzzoli2021inversion}).

In the simplest DC configuration, two electrodes inject current $I$ in the
investigated medium and two electrodes measure the potential difference $\Delta
V$.
The ratio $\Delta V / I$  multiplied by a factor $K$ related to the geometrical arrangement
of the electrodes yields the apparent electrical resistivity $\rho_a$, which is a
function of the resistivities of the investigated media and changes according to
$K$.

Up to date, geoelectrical equipment easily handles tens to hundreds of
electrodes for 2D, 3D and 4D (i.e., 3D-time lapse) surveys (\cite{Loke2019})
allowing for the tomographic reconstruction of the subsurface resistivities
(electrical resistivity tomography - ERT). ERT constitutes a non-linear and
ill-posed inverse problem in which the non-linearity is due to the equation
linking the subsurface resistivity values to the measured apparent resistivity
data, whereas the ill-posedness is mainly related to incomplete data coverage,
considerable number of unknown parameters to be estimated and noise
contamination in the data (\cite{Menke2018}).
Whether the inversion of geoelectrical
data is performed with a deterministic or probabilistic approach, a robust and
effective numerical code is needed to solve the forward problem. Though both
commercial (e.g., \cite{Loke2022}) and open-source forward geoelectrical codes
(\cite{RUCKER2017106,BLANCHY2020104423})
have been developed, in this work we
propose a mathematical model taking into account the peculiar aspects of the
geoelectrical monitoring of MSWLFs related to the different sizes of the
elements to be considered. The liner is a few millimetres thick and has lateral
extent of tens of thousands square meters; similarly, the ratio of the diameter
to the length of the electrodes is typically one to forty centimetres (
\cite{Ruecker2011}). It is clear that representing these items as three-dimensional
objects requires a computational grid that might be unnecessary refined, as
discussed in \cite{Berre2020a} for a similar context, with obvious computational cost.
Accordingly, we introduce new conceptual models to lighten the computational
cost of the forward problem while maintaining the accuracy. We consider a model
reduction technique to approximate the electrodes as one-dimensional objects of
the same length as their three-dimensional representation, and the liner as
two-dimensional object of the same extent. Example of model reduction to
one-dimensional objects can be found, for instance, in \cite{Peaceman1978,Peaceman1983}
and more
recently in \cite{Cerroni2019,Gjerde2019,Gjerde2020,Kuchta2021,Berrone2022}
while for two-dimensional
objects in
\cite{Martin2005,DAngelo2011,Fumagalli2016a,Scialo2017,Nordbotten2018}.
A new set of equations will be
derived for the electrodes and the liner along
with appropriate interface conditions for their coupling with the surrounding
domain. With the reduced model, a coarser grid can be used, or bigger problems
can be solved, which will reduce ill-posedness and computational cost.

The outcomes of the model-based code are then compared with results obtained
through down-scaled experiments in the laboratory and, whenever possible, with
theoretical results. In particular, we consider several settings by changing the
depth of the geomembrane and the presence or absence of holes.

To solve the model numerically two cell-centred finite volume schemes, multi-point flux approximation (MPFA) \cite{Aavatsmark2007,Aavatsmark2002} and two-point flux
approximation (TPFA) \cite{Aavatsmark2007a} methods, are exploited and the results are
compared with experimental data to identify the most suitable approach in terms
of computational costs and accuracy.

This article is organized as follows. In Section \ref{sec:model}, first we introduce the
mathematical model for a three-dimensional domain, and then the
mixed-dimensional model for the electrodes and the liner. Section \ref{sec:approximation} is devoted
to the spatial discretization of the equations by discussing appropriate
strategies to make their solution more efficient. In Section \ref{sec:experiments} we consider two
different settings to validate the proposed model: in the first one the liner is
flat and changes its depth, while in the second one the liner has a box shape
with the possible presence of a hole. Finally, we draw the conclusions in
Section \ref{sec:conclusions}.
