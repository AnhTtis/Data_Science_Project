\section{The mathematical model}\label{sec:model}

We
indicate
the domain where the geoelectrical equations will be applied with
$\Upsilon\subset \mathbb{R}^3$, with boundary $\partial \Upsilon$ and outward
unit normal $\bm{n}_\partial$. We consider the
following main variables: $\bm{E}:\Upsilon \rightarrow \mathbb{R}^3$, the electric
field in \sib{\volt\per\meter}, $\bm{J}:\Upsilon \rightarrow \mathbb{R}^3$,
the current density field in \sib{\ampere\per\square\meter}, and
$\varphi: \Upsilon \rightarrow \mathbb{R}$, the electric potential
in \sib{\volt}.

We present now the governing equations.
Ohm-Kirchhoff's law states that $\bm{J} = \sigma \bm{E}$,
with $\sigma:\Upsilon \rightarrow \mathbb{R}_{>0}$ the electric
conductivity of the medium in
\sib{\siemens\per\metre}. The latter is the inverse of the electric resistivity
$\rho$, measured in \sib{\ohm\metre}.
Moreover, as a consequence of the static Maxwell-Faraday equation, $\nabla \times
\bm{E} =\bm{0}$, and by the Helmholtz decomposition we get that the electric field
$\bm{E}$ can be related to the
gradient of the electric potential $\varphi$ as $\bm{E} = -\nabla \varphi$.
We can thus establish a constitutive relation between the current density field $\bm{J}$ and
the electric potential $\varphi$ as follow
\begin{gather}\label{eq:current_potential}
    \bm{J} + \sigma \nabla \varphi = \bm{0}.
\end{gather}
Gauss' law, or charge conservation equation for $\bm{J}$, can be expressed as
\begin{gather}\label{eq:continuity}
    \nabla \cdot \bm{J} = q,
\end{gather}
with $q:\Upsilon \rightarrow \mathbb{R}$ being the source of volumetric charge
density in \sib{\ampere\per\cubic\meter}.
For simplicity, we
consider an isolated body so at the boundary $\partial \Upsilon$ we set
$\bm{J}\cdot\bm{n}_\partial = 0$.
By combining \eqref{eq:current_potential} and \eqref{eq:continuity} the
problem can be written as:
\begin{problem}[The DC problem]\label{eq:the_system}
    The direct current problem is: find $(\bm{J}, \varphi)$ in $\Upsilon$ such that
    \begin{gather*}
        \begin{aligned}
            &\begin{aligned}
                &\bm{J} + \sigma \nabla \varphi = \bm{0}\\
                &\nabla \cdot \bm{J} = q
            \end{aligned}
            && \text{in } \Upsilon\\
            &\bm{J} \cdot \bm{n}_\partial = 0
            && \text{on } \partial \Upsilon
        \end{aligned}
    \end{gather*}
\end{problem}
In Subsection \ref{subsec:electrode}, we present a suitable model for the electrodes
represented as objects of dimension 1 immersed in $\Upsilon$ and in Subsection
\ref{subsec:liner} the bi-dimensional model for the liner.

\subsection{A model for the electrodes}\label{subsec:electrode}

Depending on the chosen electrode configuration, multiple electrodes are immersed in
$\Upsilon$
and can be used, in pairs, to inject electrical current and measure
the electric potential differences.

We want to introduce an effective and efficient mathematical model able to
describe their influence in Problem \ref{eq:the_system}.  For simplicity, we
consider a single electrode, being immediate the extension to multiple (separated)
electrodes, and set $q=0$.  Even if an electrode is a three-dimensional object,
typically
a cylinder of stainless steel, its radius is very small compared to its length
and domain size. Typically, the ratio between its radius and length is
in the order of 1/30 $\sim$ 1/50. Our idea is thus to approximate the electrode by an object of dimension
1, making the gridding process much easier and with fewer cells, essential
when dealing with real geometries.

Let us consider Figure \ref{fig:electrode} on the left where a three-dimensional electrode
$\Gamma$ is immersed in $\Upsilon$, with $ \mathring{\Gamma} \cap
\mathring{\Upsilon} =\emptyset$. We assume that one end of the electrode is
in contact with the top of $\partial \Upsilon$.
\begin{figure}[tb]
    \centering
    \resizebox{0.25\textwidth}{!}{\fontsize{0.75cm}{2cm}\selectfont\input{fig/electrode_full.pdf_tex}}%
    \hspace{0.1\textwidth}%
    \resizebox{0.25\textwidth}{!}{\fontsize{0.75cm}{2cm}\selectfont\input{fig/electrode_reduced.pdf_tex}}
    \caption{An electrode $\Gamma$ represented as three-dimensional object
    (right)
    and the reduced electrode $\gamma$ as one-dimensional object (left).}
    \label{fig:electrode}
\end{figure}
We suppose $\Gamma$ to be a cylinder with
radius $r$ and longitudinal axis $\gamma$ of length $l$, described as
\begin{gather*}
    \Gamma = D \times \gamma
    \quad \text{with} \quad
    D = \{ (\xi \cos \theta, \xi \sin
    \theta) \text{ with } \xi \in [0, r) \text{ and } \theta \in [0, 2\pi) \}.
\end{gather*}
We indicate with $\bm{v}_\Gamma$ the unit vector aligned with line
$\gamma$. The equations of
Problem \ref{eq:the_system} apply to $\Gamma$ with coefficients that are different
than those of the surrounding domain $\Upsilon$. We
indicate variables in $\Gamma$ with a subscript.
To couple the two systems, suitable conditions are needed
on the common boundary between $\Upsilon$ and $\Gamma$, we call
$\partial_i \Gamma = \overline{\Gamma} \cap \overline{\Upsilon}$
the internal part of $\partial \Gamma$. Since
the electrode is a cylinder, we further divide
$\partial_i \Gamma$ into two disjoint portions $\partial_l \Gamma$ and
$\partial_b \Gamma$, the former being the lateral surface and the latter the bottom
one. Finally, $\partial_t \Gamma$ is the top part of the boundary of
$\Gamma$, which is not in contact with the interior of $\Upsilon$.
Conservation of the current density through $\partial_i \Gamma$ and continuity
of the electric potential are required,
\begin{gather*}
    \begin{aligned}
        &\bm{J} \cdot \bm{n}_\Gamma = \bm{J}_\Gamma \cdot \bm{n}_\Gamma\\
        &\varphi = \varphi_\Gamma
    \end{aligned}
    \quad \text{on } \partial_i \Gamma
\end{gather*}
where $\bm{n}_\Gamma$ is the unit normal of $\partial_i \Gamma$ pointing from
$\Upsilon$ toward $\Gamma$.

To avoid excessive refinement of the computational grid to
capture the three-dimensional nature of the electrode, we employ
a model reduction strategy by approximating $\Gamma$ with its centre line
$\gamma$ while the surrounding domain, with an abuse of notation, will be
still indicated with $\Upsilon$, see Figure \ref{fig:electrode} on the right.
We will derive a new set of equations for $\gamma$ and coupling conditions
between $\gamma$ and $\Upsilon$, by following the
approach presented in
\cite{Peaceman1978,Peaceman1983,DAngelo2008,Gjerde2019,Gjerde2020,Kuchta2021}
derived for wells in porous media and here adapted
to the electrodes. We assume the following:
\begin{assumption}[Electrode radius]
    We assume that the radius $r$ of the electrode is much smaller than any
    other characteristic length in the domain.
\end{assumption}
\begin{assumption}[Lateral exchange]\label{ass:lateral_exchange}
    We assume that the current density is exchanged between the electrode
    and the surrounding media through the lateral boundary of the electrode $\partial_l
    \Gamma$, the
    exchange through $\partial_b \Gamma$ is assumed to be negligible.
\end{assumption}
The line current density exchanged between the electrode, now represented as $\gamma$, and the
surrounding media is denoted as
$j_\gamma$
in \sib{\ampere\per\metre} and approximated by the relation
\begin{gather}\label{eq:varsigma_electrde}
    j_\gamma + \sigma_\gamma (\varphi_\gamma - \varphi) = 0
    \quad \text{on } \gamma,
\end{gather}
where $\sigma_\gamma$ in \sib{\siemens\per\metre} is the conductivity of the interface between the electrode
and the surrounding media, whose meaning will be discussed in the following.
The reduced variables for the electrode $\varphi_\gamma$ in \sib{\volt}
and $\bm{J}_\gamma$ now in
\sib{\ampere} are defined as
\begin{gather*}
    \varphi_\gamma(z) = \frac{1}{\pi r^2}\int_D
    \varphi_\Gamma(\xi, \theta, z) d \xi d \theta
    \quad \text{and} \quad
    \bm{J}_\gamma(z) = \int_D (\bm{v}_\Gamma \otimes \bm{v}_\Gamma)
    \bm{J}_\Gamma (\xi, \theta, z) d \xi d \theta.
\end{gather*}
By following the model reduction procedure considered in the aforementioned
works, we can introduce a new model for the electrode represented as an object
$\gamma$ of dimension $n-2$
\begin{gather}\label{eq:electrode_dc}
    \begin{aligned}
        &\bm{J}_\gamma + \pi r^2 \sigma_\Gamma \nabla \varphi_\gamma = \bm{0}\\
        &\nabla \cdot \bm{J}_\gamma - j_\gamma = 0
    \end{aligned}
    \quad \text{in } \gamma,
\end{gather}
where the gradient and divergence are now computed along $\gamma$.
The second equation is a conservation equation which accounts for the
current along $\gamma$ and its exchange with
$\Upsilon$, represented as $j_\gamma$.
On the base of
assumption \ref{ass:lateral_exchange}, we set the so-called tip condition on the
bottom boundary of $\gamma$
\begin{gather}\label{eq:tip_condition}
    \bm{J}_\gamma \cdot \bm{v}_\Gamma = 0
    \quad \text{on } \partial_b \gamma.
\end{gather}
As mentioned, the electrodes inject a prescribed current density
$\overline{J}_\Gamma$ in \sib{\ampere\per\square\meter}, or
measure potential difference, thus $\overline{J}_\Gamma = 0$ in this latter
case. Accordingly, we employ one of the following conditions on
$\partial_t \gamma$:
\begin{gather*}
    \begin{aligned}
        &\bm{J}_\gamma \cdot \bm{n}_\partial = i_\gamma\\
        &\bm{J}_\gamma \cdot \bm{n}_\partial = 0
    \end{aligned}
    \quad \text{on } \partial_t \gamma,
\end{gather*}
for current and potential electrodes, respectively, with $i_\gamma = \pi r^2
\overline{J}_\Gamma$ being the current in \sib{\ampere}.

In $\Upsilon$, the electrode is seen as an immersed line acting as
source/sink of electric current. Introducing  the Dirac delta $\delta_\gamma$ distributed along $\gamma$ with measure
\sib{\per\square\metre}, the new set of equations for $\Upsilon$ reads
\begin{gather}\label{eq:cont_electr}
    \begin{aligned}
        &\bm{J} + \sigma \nabla \varphi = \bm{0}\\
        &\nabla \cdot \bm{J} + j_\gamma \delta_\gamma =0
    \end{aligned}
    \quad \text{in } \Upsilon
\end{gather}
Solutions of the previous equations exhibit a logarithmic
singularity in $\varphi$ with strong derivatives in the vicinity of $\gamma$.
Assuming a radial current density exiting from the electrode, following the
approach given \cite{Peaceman1978,Peaceman1983}, it is possible to write the coefficient in
\eqref{eq:varsigma_electrde} as
\begin{gather*}
    \sigma_\gamma = \frac{2\pi \sigma}{\ln(r_e/r) + S},
\end{gather*}
which is a semi-discrete parameter that is used to mimic the effect of the
singularity around $\gamma$, when a discretization in space is applied, without
the need of over-refining the grid.
The parameter $r_e$ in \sib{\metre} is the radius at which the electric potential in the medium
$\varphi$ is equal to the averaged
grid cell potential.
$r_e$ is normally taken equal to $r_e\approx 0.2 h$ with $h$, in
\sib{\metre}, the cell diameter.
$S$ in
\sib{\cdot} is the so-called skin-factor, which gives an additional electric
potential drop due to the actual features of the electrode and of the
surrounding zone.
$S$ can be used to model rust or other imperfections of the electrode, and also
to model the imperfect contact between the electrode and the ground.
 We are finally ready to introduce the new mixed-dimensional problem.
\begin{problem}[The mixed-dimensional electrode DC problem]\label{pb:electrode}
    Find $(\bm{J}, \varphi)$ in $\Upsilon$ and $(j_\gamma, \bm{J}_\gamma, \varphi_\gamma)$
    in $\gamma$ such that
    \begin{gather*}
        \begin{aligned}
            &\bm{J} + \sigma \nabla \varphi = \bm{0}\\
            &\nabla \cdot \bm{J} + j_\gamma \delta_\gamma =  0
        \end{aligned}
        \quad \text{in } \Upsilon,
        \qquad
        \begin{aligned}
            &\bm{J}_\gamma + \pi r^2 \sigma_\Gamma \nabla \varphi_\gamma = \bm{0}\\
            &\nabla \cdot \bm{J}_\gamma - j_\gamma = 0
        \end{aligned}
        \quad \text{in } \gamma,
    \end{gather*}
    coupled with the following interface condition
    \begin{gather*}
        j_\gamma + \sigma_\gamma (\varphi_\gamma - \varphi) = 0,
        \quad \text{on } \gamma
    \end{gather*}
    and with boundary conditions given by
    \begin{gather*}
        \begin{aligned}
            &\bm{J} \cdot \bm{n}_\partial = 0 && \text{on } \partial \Upsilon,\\
            &\bm{J}_\gamma \cdot \bm{n}_\partial = \pi r^2 \overline{J}_\Gamma
            && \text{on } \partial_t \gamma,\\
            &\bm{J}_\gamma \cdot \bm{v}_\Gamma = 0 &&  \text{on } \partial_b
            \gamma.
        \end{aligned}
    \end{gather*}
\end{problem}

\subsection{A model for the liner}\label{subsec:liner}

A liner is placed below the waste body
to avoid the infiltration of leachate in the underground, that
can be
dangerous for the groundwater.
For simplicity, we assume that the liner consists of a planar portion of a highly
resistive geomembrane whose thickness is much smaller than other lengths
in the domain. For real applications, the ratio between its thickness and lateral extension
is in the order of $\sim 10^{-5}$. To simplify gridding and avoid
excessive refinement in representing the liner as a three-dimensional object we
present a two-dimensional model of the liner with suitable coupling conditions
with the surrounding media.

In the case of a liner composed by several planar parts,
e.g. having a box shape, we will apply the new model to each part and use suitable
conditions to connect them. This aspect will be detailed later.

Let us consider Figure \ref{fig:liner} on the left where a
liner $\Lambda$  is immersed in $\Upsilon$.
\begin{figure}[tb]
    \centering
    \resizebox{0.25\textwidth}{!}{\fontsize{0.75cm}{2cm}\selectfont\input{fig/liner_full.pdf_tex}}%
    \hspace{0.1\textwidth}%
    \resizebox{0.25\textwidth}{!}{\fontsize{0.75cm}{2cm}\selectfont\input{fig/liner_reduced.pdf_tex}}
    \caption{The liner $\Lambda$ represented as three-dimensional
    object (right) and the reduced liner $\lambda$ as two dimensional
    object (left).}
    \label{fig:liner}
\end{figure}
It can be described as
a domain with
thickness $\varepsilon$ in $\sib{\metre}$ and central (portion of a) plane $\lambda$,
\begin{gather*}
    \Lambda = \lambda \times \left(-\frac{\varepsilon}{2},
    \frac{\varepsilon}{2}\right).
\end{gather*}
We indicate with
$\bm{n}_\Lambda$ the unit normal to the top boundary of $\Lambda$ pointing from
$\Upsilon$ toward $\Lambda$. Equations of
Problem \ref{eq:the_system}
apply also to $\Lambda$ with different conductivity than the surrounding $\Upsilon$,
therefore, as done for the electrode, we propose now a suitable reduced model
where the liner is approximated with its central plane $\lambda$, by
following the same approach as in
\cite{Martin2005,DAngelo2011,Nordbotten2018,Fumagalli2019a}.
First, we divide the boundary $\partial \Lambda$ of
$\Lambda$ into two parts: the lateral surface of thickness $\varepsilon$ is
named $\partial_l \Lambda$, and the bottom and top surfaces are identified as
$\partial_s \Lambda$.
If the liner ends inside $\Upsilon$, we can identify the portion of its
lateral boundary that is in contact with the internal part of $\Upsilon$.
We call this internal part of
the boundary as $\partial_i \Lambda \subset \partial_l \Lambda$.
Let us state the following assumptions.
\begin{assumption}[Liner thickness]
    We assume that the thickness $\varepsilon$ of the liner is much smaller than any
    other sizes in the domain.
\end{assumption}
\begin{assumption}[Lateral exchange]%\label{ass:lateral_exchange}
    We assume that current density is exchanged between the liner
    and the surrounding media through $\partial_s \Lambda$, the
    exchange though $\partial_i \Lambda$ is assumed to be negligible.
\end{assumption}
We substitute the equi-dimensional representation of the liner $\Lambda$ with
its lower-dimensional counterpart $\lambda$,
so that the mesh size is not constraned by its thickness
$\varepsilon$. We present now the mathematical model for $\lambda$. Note that
the liner is in contact with $\Upsilon$ with its top
and bottom surfaces.

For each side of $\lambda$, the current density exchanged between the liner and the surrounding media, following
\eqref{eq:varsigma_electrde}, is denoted as
$j_\lambda$ in \sib{\ampere\per\square\metre}
and approximated by
\begin{gather}\label{eq:varsigma}
    \varepsilon j_\lambda + \sigma_\lambda (\varphi_\lambda - \varphi) = 0
    \quad \text{on } \lambda.
\end{gather}
Being $\lambda$ of codimension 1, it does not introduce
singular solutions and thus we do not have the need to tune $\sigma_\lambda$
as we did for $\sigma_\gamma$.
Indeed, the  value of $\sigma_\lambda$
in \sib{\siemens\per\metre} simply
represents
the conductivity of the liner along its normal.
The reduced variables are $\varphi_\lambda$ in \sib{\volt}
and $\bm{J}_\lambda$ now in
\sib{\ampere\per\metre} and defined as
\begin{gather*}
    \varphi_\lambda(x, y) = \frac{1}{\varepsilon}\int_{-\frac{\varepsilon}{2}}^{\frac{\varepsilon}{2}}
    \varphi_\Lambda(x, y, z) d z
    \quad \text{and} \quad
    \bm{J}_\gamma(x, y) = \int_{-\frac{\varepsilon}{2}}^{\frac{\varepsilon}{2}}
    (I - \bm{n}_\Lambda \otimes \bm{n}_\Lambda)
    \bm{J}_\Lambda (x, y, z) d z,
\end{gather*}
with $I$ the identity matrix.
We can now introduce a model for the liner
represented as an object $\lambda$ of dimension $n-1$
\begin{gather*}
    \begin{aligned}
        &\bm{J}_\lambda + \varepsilon \sigma_\Lambda \nabla \varphi_\lambda = \bm{0}\\
        &\nabla \cdot \bm{J}_\lambda - j_\lambda = 0
    \end{aligned}
    \quad \text{in } \lambda.
\end{gather*}
The gradient and divergence are now computed over the plane $\lambda$.
To couple this model with Problem \ref{eq:the_system} in $\Upsilon$, we simply
require current conservation so that $\bm{J}\cdot \bm{n}_\Lambda =
j_\lambda$ on each side of $\lambda$.
In the experiments we will consider the boundary of $\lambda$ to be
in contact with $\partial \Upsilon$, and thus inheriting the same boundary conditions, or
immersed into $\Upsilon$ in which case we impose null current exhange, as in \eqref{eq:tip_condition}.

When the liner geometry is more complex but can still be split into multiple
planar polygons, we can follow the same strategy for each one of them. Since the liner is composed by a
homogeneous material, at the interface between two planes we
simply require that the current density is conserved and that the electric
potential is continuous.

The problem is thus formalized as follows.
\begin{problem}[The mixed-dimensional liner DC problem]\label{eq:liner}
    Find $(\bm{J}, \varphi)$ in $\Upsilon$ and $(j_\lambda, \bm{J}_\lambda,
    \varphi_\lambda)$
    in $\lambda$ such that
    \begin{gather*}
        \begin{aligned}
            &\bm{J} + \sigma \nabla \varphi = \bm{0}\\
            &\nabla \cdot \bm{J}=  0
        \end{aligned}
        \quad \text{in } \Upsilon
        \qquad
        \begin{aligned}
            &\bm{J}_\lambda + \varepsilon \sigma_\Lambda \nabla \varphi_\lambda = \bm{0}\\
            &\nabla \cdot \bm{J}_\lambda - j_\lambda = 0
        \end{aligned}
        \quad \text{in } \lambda
    \end{gather*}
    coupled with the following interface condition on both sides of $\lambda$
    \begin{gather*}
        \varepsilon j_\lambda + \sigma_\lambda ( \varphi_\lambda - \varphi) = 0
        \quad \text{on } \lambda
    \end{gather*}
    with boundary conditions given by
    \begin{gather*}
        \begin{aligned}
            &\bm{J} \cdot \bm{n}_\partial = 0 && \text{on } \partial \Upsilon\\
            &\bm{J}_\lambda \cdot \bm{n}_\partial = 0 &&  \text{on } \partial
            \lambda
        \end{aligned}
    \end{gather*}
\end{problem}

\subsection{The complete mixed-dimensional model}

For simplicity we assume that the electrodes and the liner are not intersecting
nor in contact, so
the global problem considers both contributions presented in Problem \ref{pb:electrode} and Problem
\ref{eq:liner} in a straightforward case. The mixed-dimensional model consider a set of $N$ electrodes,
indexed with $i=1, \ldots, N$, and a single liner, see Figure \ref{fig:all}.
\begin{figure}[tb]
    \centering
    \resizebox{0.25\textwidth}{!}{\fontsize{0.75cm}{2cm}\selectfont\input{fig/all_reduced.pdf_tex}}%
    \caption{Schematic representation of the mixed-dimensional objects: the
    three-dimensional domain $\Upsilon$, the two-dimensional liner $\lambda$,
    and the one-dimensional electrodes $\gamma_i$.}
    \label{fig:all}
\end{figure}
The equations are presented
as following.
\begin{problem}[The mixed-dimensional DC problem]\label{eq:model}
    Find $(\bm{J}, \varphi)$ in $\Upsilon$,
    $(j_{\gamma_i}, \bm{J}_{\gamma_i}, \varphi_{\gamma_i})$
    in $\gamma_i$ for $i=1, \ldots, N$
    and
    $(j_\lambda, \bm{J}_\lambda,
    \varphi_\lambda)$
    in $\lambda$ such that
    \begin{gather*}
        \begin{aligned}
            &\bm{J} + \sigma \nabla \varphi = \bm{0}\\
            &\nabla \cdot \bm{J} + {\textstyle\sum_{i}} j_{\gamma_i} \delta_{\gamma_i} =  0
        \end{aligned}
        \quad \text{in } \Upsilon
        \qquad
        \begin{aligned}
            &\bm{J}_{\gamma_i} + \pi r_i^2 \sigma_{\Gamma_i} \nabla
            \varphi_{\gamma_i} = \bm{0}\\
            &\nabla \cdot \bm{J}_{\gamma_i} - j_{\gamma_i} = 0
        \end{aligned}
        \quad \text{in } \gamma_i
        \qquad
        \begin{aligned}
            &\bm{J}_\lambda + \varepsilon \sigma_\Lambda \nabla \varphi_\lambda = \bm{0}\\
            &\nabla \cdot \bm{J}_\lambda - j_\lambda = 0
        \end{aligned}
        \quad \text{in } \lambda
    \end{gather*}
    coupled with the following interface condition
    \begin{gather*}
        j_{\gamma_i} + \sigma_{\gamma_i} (\varphi_{\gamma_i} - \varphi) = 0
        \quad \text{on } \gamma_i
        \qquad
        \varepsilon j_\lambda + \sigma_\lambda ( \varphi_\lambda - \varphi) = 0
        \quad \text{on } \lambda
    \end{gather*}
    with boundary conditions given by
    \begin{gather*}
        \begin{aligned}
            &\bm{J} \cdot \bm{n}_\partial = 0 && \text{on } \partial \Upsilon\\
            &\bm{J}_\lambda \cdot \bm{n}_\partial = 0 &&  \text{on } \partial
            \lambda
        \end{aligned}
        \qquad
        \begin{aligned}
            &\bm{J}_{\gamma_i} \cdot \bm{n}_\partial = \pi r_i^2
            \overline{J}_{\Gamma_i}
            && \text{on } \partial_t \gamma_i\\
            &\bm{J}_{\gamma_i} \cdot \bm{v}_{\Gamma_i} = 0 &&  \text{on } \partial_b
            \gamma_i\\
        \end{aligned}
    \end{gather*}
\end{problem}
By balancing the current density imposed on the boundary,
Problem \ref{eq:model} admits a unique solution for $(\bm{J}, \bm{J}_{\gamma_i},
j_{\gamma_i}, \bm{J}_\lambda,  j_\lambda)$, however the electric potentials
$(\varphi, \varphi_{\gamma_i}, \varphi_\lambda)$
are defined up to a constant $c \in \mathbb{R}$. To uniquely define this
constant
it is possible to consider different strategies like imposing the following
null average condition
\begin{gather*}
    \int_\Upsilon \varphi + \sum_{i=1}^{N} \int_{\gamma_i} \varphi_{\gamma_i} +
    \int_\lambda \varphi_\lambda = 0.
\end{gather*}
However, in our application, we are interested in potential differences, thus the
constant $c$ does not have any practical influence.
