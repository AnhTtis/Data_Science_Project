\section{Numerical experiments}\label{sec:experiments}

The purpose of this section is to validate the proposed approach against laboratory experiments and to study the distribution of current density and electric potential in presence of a highly-resistive liner.
We fill a $52 \sib{\centi\metre} \times34\sib{\centi\metre}\times40\sib{\centi\metre}$
plastic box with different volumes of tap water, whose
conductivity is estimated with a Crison MM40+ multimeter to be equal to $\sigma = 1/29$
\sib{\siemens\per\meter}.

We deploy a spread of 4 electrodes according to a
Wenner-$\alpha$ configuration at the centre of the box along the $x$-axis (Figure \ref{fig:wenner_config}).
The two outer electrodes $C_1$ and $C_2$ are used to inject the current
density
with value $\pm \overline{J}_\Gamma$, while the two
inner ones $P_1$ and $P_2$ are used to compute the potential difference
$\varphi_{\gamma_2} - \varphi_{\gamma_3}$, the latter being evaluated at the top of the
electrodes.
The penetration depth of the Wenner-$\alpha$ array for a homogeneous medium can be analytically estimated to be about $0.11L$ or $0.17L$, being $L$ the distance between the current electrodes, either one considers the peak or the median value of the one-dimensional sensitivity function, respectively \cite{roy1971depth, BARKER1989}. However, the penetration depth of a geoelectrical survey only estimates the depth at which the used array is maximally sensitive and may vary considerably if the investigated medium is heterogeneous.

We set $\Upsilon$ to be completely isolated with null current flow through all the boundaries (i.e., homogeneous Neumann boundary condition) as presented in Problem \ref{eq:model}. The domain is
discretized with an unstructured mesh with a characteristic length of 5
\sib{\centi\meter}, which is successively refined to 1 \sib{\centi\meter} near the electrodes. No significant improvements are observed in the numerical results considering smaller characteristic lengths.
\begin{figure}
    \centering
    \raisebox{0.05\textwidth}{\resizebox{0.45\textwidth}{!}{\fontsize{0.75cm}{2cm}\selectfont\input{fig/wenner_alpha.pdf_tex}}}%
    \resizebox{0.55\textwidth}{!}{\fontsize{2cm}{2cm}\selectfont\input{fig/geometry_with.pdf_tex}}%
    \caption{Wenner-$\alpha$ configuration with four electrodes with spacing $a$.}
    \label{fig:wenner_config}
\end{figure}

The
electrodes have radius equal to $r=1\sib{\milli\meter}$, length of $5
\sib{\milli\meter}$ and conductivity equal to $\sigma_\gamma = 1.45 \cdot 10^6
\sib{\siemens\per\meter}$. The ratio between the radius and the length of the electrodes is larger than what stated previously for convenience in the lab experiments. However, numerical tests with smaller ratios down to 1/100 do not show significant differences. The liner, which will have different shapes for each
example, has thickness equal to $\varepsilon = 1
\sib{\milli\meter}$ and conductivity $\sigma_\lambda =
10^{-9} \sib{\siemens\per\meter}$. As mentioned before, the liner is approximated as a
bi-dimensional object. Contrary to the electrodes, the computational grid is
conforming to the liner.

We consider two sets of experiments with increasing level of complexity: in Subsection \ref{subsec:itact_liner} the
liner is a horizontal plane, while in Subsection \ref{subsec:broken_liner} the
liner has a box shape. The latter mimics the shape of the liner that is placed underneath real landfills to completely isolate the waste body from the surrounding media. In such a case, it is interesting to analyse current and electrical potential distribution according to the position of the electrodes and to the presence of damages in the liner. We will compare the numerical and the experimental values
of the estimated
apparent resistivity $\rho_a$ in \sib{\ohm\metre}, defined for the
Wenner-$\alpha$ configuration as
\begin{gather*}
    \rho_{a} = 2\pi a \frac{\varphi_{\gamma_2} -
    \varphi_{\gamma_3}}{i_\gamma},
\end{gather*}
where the value of $i_\gamma$ is imposed and the difference
$\varphi_{\gamma_2} - \varphi_{\gamma_3}$ comes from measurements or modelling.

\subsection{A horizontal liner}\label{subsec:itact_liner}

In this first test, we want to evaluate the effect of the liner on $\rho_a$.
The liner is perfectly horizontal and touches all the lateral boundaries of $\Upsilon$. We further assume that $\lambda$ is intact, meaning that neither holes nor
defects are present. We consider two settings with different spacing
between the electrodes: $a=3\sib{\centi\meter}$ and
$a=6\sib{\centi\meter}$. Since the liner is a highly resistive
membrane, it can be approximated with a perfect insulator. In the laboratory test, instead of introducing the liner, we
simply decrease the height of the water layer from $h=17
\sib{\centi\metre}$ to $h=3
\sib{\centi\metre}$ to ease operations. Following \cite{Sheriff1990}, as reference value we
consider the apparent resistivity
$\widetilde{\rho}_a$ in \sib{\ohm\meter} analytically expressed as
\begin{gather}\label{eq:rho_teo}
    \widetilde{\rho}_a = \frac{\rho i_\gamma}{2 \pi}
    \left(
    \frac{1}{a}
    + 4 \sum_{n=1}^\infty
    \frac{1}{\sqrt{a^2 + 4n^2 h^2}}
    -
    \frac{1}{\sqrt{4a^2 + 4n^2 h^2}}
    \right),
\end{gather}
with $\rho = 29 \sib{\ohm\meter}$ being the resistivity of the water.
This formula was derived under the condition that $\Upsilon$ is indefinitely
laterally extended (i.e., 1D model) and the electrodes are represented as zero-dimensional
sources, and, in this work, further simplified since the resistivity of the liner is several orders of magnitude higher than the water resistivity.
The instrument used in the laboratory performs several measurements of $\rho_a$ for each depth of the liner, so that we can estimate the mean value and the standard deviation, which can be
used as proxy for the reliability of the measurements, as reported in Figure
\ref{fig:experiment1result_b_plot}. In addition, measurement errors were also checked with reciprocal measurements, i.e. switching current and potential electrodes, but values are rather small confirming that the measured $\rho_a$ values are reliable.
\begin{figure}
    \centering
    \includegraphics[scale=0.45]{fig/b_plot_error_3cm}%
    \hspace{0.05\textwidth}%
    \includegraphics[scale=0.45]{fig/b_plot_error_6cm}
    \caption{Percentages of standard deviations, centred at their mean value, for the measurements of $\rho_a$
    for the two experiments, for the example in Subsection \ref{subsec:itact_liner}.}
    \label{fig:experiment1result_b_plot}
\end{figure}

In the numerical simulations, we
adopt two strategies to account for the liner $\lambda$: i) we consider the entire domain $\Upsilon$ and place a planar surface with very low electric
conductivity at a certain depth; ii) since $\lambda$ is highly
resistive, we simply remove the part of $\Upsilon$ below the liner and impose a
null current density flux condition at the boundary, as schematically shown in Figure \ref{fig:domain_example1}.
Both TPFA and MPFA are considered as discretization schemes.
We can thus evaluate the effect of the liner and compare
the obtained results with the laboratory experiments.
\begin{figure}
    \centering
    \resizebox{0.35\textwidth}{!}{\fontsize{0.5cm}{2cm}\selectfont\input{fig/example1_with.pdf_tex}}%
    \hspace*{0.1\textwidth}%
    \resizebox{0.35\textwidth}{!}{\fontsize{0.5cm}{2cm}\selectfont\input{fig/example1_without.pdf_tex}}%
    \caption{Schematic representation of the domain of example in Subsection \ref{subsec:itact_liner}. The resistive interface of the liner is represented by an interface (left) or by the bottom of the domain (right).}
    \label{fig:domain_example1}
\end{figure}

In Figure \ref{fig:experiment1result} we show the apparent resistivities
obtained in the simulations, in the laboratory experiments, and with the
analytical relationship \eqref{eq:rho_teo}. The measured standard deviations (Figure \ref{fig:experiment1result_b_plot}) are also represented and now centred
in the actual values of the apparent resistivity. Since these standard
deviations are rather small they are hardly visible in the graph.
All the results show the same trend:
by decreasing the liner depth, the apparent resistivity increases because the liner is closer to the electrodes.

We note that the results of the simulations does not change if either $\lambda$ is included in the modelled domain or $\Upsilon$ is reduced accordingly. This means that the
laboratory setting is appropriate and can be compared with the simulations when
$\lambda$ is considered. Moreover, there is a clear difference between the
results obtained from the TPFA and MPFA schemes, having for the former, for
$a=3\sib{\centi\meter}$, a non-monotone behaviour with respect to the liner
depth.

By comparing the synthetic data with the measured ones, we observe that the numerical solutions computed with MPFA are in good agreement,
while the TPFA ones are less accurate. A possible explanation is that the latter
is not consistent when using simplicial grids and, therefore, the obtained results might not be reliable.

Finally, we observe that, for a spacing of $a=6\sib{\centi\meter}$, the measured and computed values of $\rho_a$ are higher than the actual one. This can be due to the fact that the considered model is not 1D, thus we could not apply relationship \eqref{eq:rho_teo}.
\begin{figure}
    \centering
    \includegraphics[scale=0.45]{fig/error_3cm}%
    \hspace{0.05\textwidth}%
    \includegraphics[scale=0.45]{fig/error_6cm}\\[0.25cm]
    \includegraphics[width=0.85\textwidth]{fig/legend1}
    \caption{Results for the example in Subsection \ref{subsec:itact_liner}. Apparent resistivities computed and measured for several liner depths and for electrode spacing $a=3\sib{\centi\meter}$ (left) and $a=6\sib{\centi\meter}$ (right).}
    \label{fig:experiment1result}
\end{figure}
Figure \ref{fig:experiment1resulta} shows a comparison between the numerical
results and the values obtained with  \eqref{eq:rho_teo} when the computational domain
$\Upsilon$ is laterally extended to $100\times100\times40$
\sib{\centi\metre\cubed}. The results are now in good agreement, confirming that
the discrepancy in Figure \ref{fig:experiment1result} for
$a=6\sib{\centi\meter}$ is mainly due to boundary effects. With respect to the values of penetration depth stated before, our tests show that the Wenner-$\alpha$ array becomes insensitive to the resistive liner at depths approximately greater than 3 times the electrode spacing.
\begin{figure}
    \centering
    \includegraphics[scale=0.45]{fig/large_domain}\\[0.25cm]
    \includegraphics[width=0.6\textwidth]{fig/legend2}
    \caption{Apparent resistivities computed with MPFA and estimated with relationship \eqref{eq:rho_teo} for different values of $a$ when the computational domain $\Upsilon$ is laterally extended to $100\times100\times40$ \sib{\centi\metre\cubed}.}
    \label{fig:experiment1resulta}
\end{figure}

Some of the numerical solutions computed with MPFA are illustrated in Figure
\ref{fig:experiment1potential} with the liner at different depths and for an
electrode spacing $a=3\sib{\centi\metre}$. The influence of the liner on the
electric potential is clearly visible, but becomes less evident when the
liner is farther from the electrodes. Below the liner the potential is essentially homogeneous. Figure \ref{fig:experiment1current} shows, for the same setting, the current lines with the associate current density, which is injected in $\Upsilon$ from $\gamma_1$ and is drawn from $\gamma_4$. Again, it is clearly visible how the liner practically confines current circulation in the upper part of the domain only.
\begin{figure}
    \centering
    \includegraphics[width=0.475\textwidth]{fig/potential_3cm}%
    \hspace*{0.05\textwidth}%
    \includegraphics[width=0.475\textwidth]{fig/potential_8cm}\\
    \includegraphics[width=0.475\textwidth]{fig/legend_potential_3cm}%
    \hspace*{0.05\textwidth}%
    \includegraphics[width=0.475\textwidth]{fig/legend_potential_8cm}%
    \caption{Results for the example in Subsection \ref{subsec:itact_liner}. Electric potential in the modelled domain with the liner at depth $h=3\sib{\centi\meter}$ (left) and $h=8\sib{\centi\meter}$ (right).}
    \label{fig:experiment1potential}
\end{figure}

\begin{figure}
    \centering
    \includegraphics[width=0.475\textwidth]{fig/3cm_v2}%
    \hspace*{0.05\textwidth}%
    \includegraphics[width=0.475\textwidth]{fig/8cm_v2}\\
    \includegraphics[width=0.475\textwidth]{fig/3cm_v1}%
    \hspace*{0.05\textwidth}%
    \includegraphics[width=0.475\textwidth]{fig/8cm_v1}\\
    \includegraphics[width=0.475\textwidth]{fig/legend_3cm}%
    \hspace*{0.05\textwidth}%
    \includegraphics[width=0.475\textwidth]{fig/legend_8cm}
    \caption{Results for the example in Subsection \ref{subsec:itact_liner}. Current lines and the associated current density in the modelled domain with the liner at depth $h=3\sib{\centi\meter}$ (left) and $h=8\sib{\centi\meter}$ (right).}
    \label{fig:experiment1current}
\end{figure}


\subsection{A box-shaped liner}\label{subsec:broken_liner}

In this section we mainly want to evaluate the effect of a hole in the liner and its
impact on the resulting apparent resistivity according to different electrode deployments. We consider the liner has a box shape, both with and without a hole $\eta$ in its bottom surface. The box has dimension $20 \times 14
 \times 7.2 \sib{\cubic\centi\meter}$
and minimum coordinates in $(20.5, 10.5, 7.2) \sib{\centi\meter}$, the hole has
diameter of
$0.5\sib{\centi\meter}$ and is centred in $(25.5,
18.5, 7.2) \sib{\centi\meter}$ (Figure \ref{fig:domain_box}).
We consider three deployments for the electrodes in a Wenner-$\alpha$ configuration:
\textit{case i}) all electrodes are within the box with a spacing of
$3\sib{\centi\meter}$; \textit{case ii}) just the current electrodes $C_1$ is placed outside the box; \textit{case iii}) both the current electrode $C_1$ and the potential electrode $P_1$ are placed outside the box. For these last two cases the electrode spacing is set to $6\sib{\centi\meter}$. Please note that, for this set of experiments, tap water conductivity is estimated in the lab to be equal to $\sigma = 1/24$ \sib{\siemens\per\meter} and the numerical simulations are performed only with the MPFA scheme. The other parameters of the problem are the same of the previous example. In addition, by taking into account possible uncertainties in the laboratory setup, we perform several simulations with the variation of the water level of $\pm 3 \sib{\milli\meter}$, the variation of the water resistivity of $\pm 2 \sib{\ohm\meter}$, the variation of the radius of $\eta$ of $\pm 0.15 \sib{\milli\meter}$, and the shift of the electrodes along $x$ and $y$, with respect to the box, of $\pm 6 \sib{\milli\meter}$.
 Figure \ref{fig:hist} shows that, for all the cases, the measured apparent resistivities fall into the range of the modelled values. We point out that the measured standard deviations are always lower than 1\% and that, for \textit{case ii}) and  \textit{case iii}), it was not possible to obtain reliable measurements when there is no hole in the liner.

\begin{figure}
    \centering
    \raisebox{0.\textwidth}{\resizebox{0.475\textwidth}{!}{\fontsize{0.75cm}{2cm}\selectfont\input{fig/domain_hole2.pdf_tex}}}%
    \hspace*{0.05\textwidth}%
    \raisebox{0.\textwidth}{\resizebox{0.475\textwidth}{!}{\fontsize{0.75cm}{2cm}\selectfont\input{fig/domain_hole1.pdf_tex}}}%
    \caption{
    Domains for the example in Subsection \ref{subsec:broken_liner}:
    electrode
    configurations for \textit{case ii} (left) and \textit{case iii} (right).
    Note the presence of the hole $\eta$ in the bottom surface of the liner. See text for details.}
    \label{fig:domain_box}
\end{figure}

%Table \ref{tab:values} lists the values of the apparent resistivity computed with the proposed approach against the values measured in the laboratory experiments.
%\begin{table}[htb]
 %   \centering
 %   \footnotesize\setlength{\tabcolsep}{4pt}
 %   \begin{tabular}{c|cc|cc|cc}
 %       & \multicolumn{2}{c|}{\textit{case i}}
 %       & \multicolumn{2}{c|}{\textit{case ii}}
 %       & \multicolumn{2}{c}{\textit{case iii}}\\
 %       % \hline
 %       & without $\eta$ & with $\eta$ & without $\eta$ & with $\eta$ & without $\eta$ & with $\eta$ \\
 %       \hline
 %       simulation & 31.8  \sib{\ohm\meter}& 31.71 \sib{\ohm\meter}& 26.88 \sib{\ohm\meter}& 39.99 \sib{\ohm\meter}& 5.48e10 \sib{\ohm\meter}& 2437.11 \sib{\ohm\meter} \\
 %       laboratory & 35.45 \sib{\ohm\meter}& 36.43 \sib{\ohm\meter}& -                     & 40.89\sib{\ohm\meter} & - & 2055.32 \sib{\ohm\meter} \\
 %       \hline
 %   \end{tabular}
 %   \caption{Values of the apparent resistivity obtained for the three
 %   cases described in Subsection
 %   \ref{subsec:broken_liner}. See text for details.}
 %   \label{tab:values}
%\end{table}

Regarding \textit{case i}, it is interesting to note that the presence of the hole $\eta$ does not significantly affect the value of the apparent resistivity $\rho_a$, which implies that this electrode configuration may not be helpful for detecting defects of the liner. Accordingly, the numerical solutions for this configuration do not show relevant differences either the hole is present or not (Figure \ref{fig:experiment2c}). The current circulates mostly inside the box-shaped liner, where we observe the variations of the electric potential.

\begin{figure}
    \centering
    \includegraphics[width=0.24\textwidth,height=0.225\textwidth]{fig/out_case1_hole.pdf}%
    \hspace*{0.01333\textwidth}%
    \includegraphics[width=0.24\textwidth,height=0.225\textwidth]{fig/out_case1_nohole.pdf}%
    \hspace*{0.01333\textwidth}%
    \includegraphics[width=0.24\textwidth,height=0.225\textwidth]{fig/out_case2_hole.pdf}%
    \hspace*{0.01333\textwidth}%
    \includegraphics[width=0.24\textwidth,height=0.225\textwidth]{fig/out_case3_hole.pdf}%
    \caption{Modelling results for different setups in Subsection 4.2. Histograms of the apparent resistivities modelled by taking into account uncertainties possibly affecting lab setup parameters. Red dashed lines are mean measured resistivities. See text for details.}
    \label{fig:hist}
\end{figure}

\begin{figure}
    \centering
    \includegraphics[width=0.475\textwidth]{fig/case_box_all_in_potential}%
    \hspace*{0.05\textwidth}%
    \includegraphics[width=0.475\textwidth]{fig/case_box_all_in_hole_potential}
    \\
    \includegraphics[width=0.475\textwidth]{fig/legend_case_box_all_in_potential}%
    %\hspace*{0.05\textwidth}%
    %\includegraphics[width=0.475\textwidth]{fig/legend_case_box_all_in_hole_potential}
    \\[0.75cm]
    \includegraphics[width=0.475\textwidth]{fig/4d0f_nb}%
    \hspace*{0.05\textwidth}%
    \includegraphics[width=0.475\textwidth]{fig/4d0f_b}
    \\
    \includegraphics[width=0.475\textwidth]{fig/legend_4d0f_nb}%
    %\hspace*{0.05\textwidth}%
    %\includegraphics[width=0.475\textwidth]{fig/legend_4d0f_b}
    \caption{Results for \textit{case i} described in Subsection \ref{subsec:broken_liner} without (left) and with (right) the hole in the liner. The top and the bottom images show the modelled electric potential and current lines with the associated current density, respectively.}
    \label{fig:experiment2c}
\end{figure}

The modelling results for \textit{case ii} and \textit{case iii} are similar (Figure \ref{fig:experiment2b} and \ref{fig:experiment2a}), except for the fact that the estimated apparent resitivity is approximately two orders of magnitude higher for the latter (Figure Figure \ref{fig:hist}). This is due to the presence of the high-resistivity liner between the voltage electrodes in \textit{case iii}. We generally observe two equi-potential regions, one inside and the other outside the box-shaped liner. When the liner is perforated, the variation of $\varphi$ across $\lambda$ is obviously lower and the main path of the current is from $C_1$ to $C_2$ through $\eta$. On the contrary, when there is no hole, there in no clear preferential direction for current circulation.
For sake of completeness, we report that, for \textit{case ii} and \textit{case iii} with no hole, the estimated apparent resitivities are $26.9 \sib{\ohm\meter}$ and $5.48e10 \sib{\ohm\meter}$, respectively. Such a large gap between those values is obviously related to the position of both current and voltage electrodes with respect to the liner.

\begin{figure}
    \centering
    \includegraphics[width=0.475\textwidth]{fig/case_box_3_in_1_out_potential}%
    \hspace*{0.05\textwidth}%
    \includegraphics[width=0.475\textwidth]{fig/case_box_3_in_1_out_hole_potential}\\
    \includegraphics[width=0.475\textwidth]{fig/legend_case_box_3_in_1_out_potential}%
    \hspace*{0.05\textwidth}%
    \includegraphics[width=0.475\textwidth]{fig/legend_case_box_3_in_1_out_hole_potential}\\[0.75cm]
    \includegraphics[width=0.475\textwidth]{fig/3d1f_nb}%
    \hspace*{0.05\textwidth}%
    \includegraphics[width=0.475\textwidth]{fig/3d1f_b}\\
    \includegraphics[width=0.475\textwidth]{fig/legend_3d1f_nb}%
    \hspace*{0.05\textwidth}%
    \includegraphics[width=0.475\textwidth]{fig/legend_3d1f_b}
    \caption{Results for \textit{case ii} described in Subsection \ref{subsec:broken_liner} without (left) and with (right) the hole in the liner. The top and the bottom images show the modelled electric potential and current lines with the associated current density, respectively. Please note the different colormap ranges for each image.}
    \label{fig:experiment2b}
\end{figure}

\begin{figure}
    \centering
    \includegraphics[width=0.475\textwidth]{fig/case_box_2_in_2_out_potential}%
    \hspace*{0.05\textwidth}%
    \includegraphics[width=0.475\textwidth]{fig/case_box_2_in_2_out_hole_potential}\\
    \includegraphics[width=0.475\textwidth]{fig/legend_case_box_2_in_2_out_potential}%
    \hspace*{0.05\textwidth}%
    \includegraphics[width=0.475\textwidth]{fig/legend_case_box_2_in_2_out_hole_potential}\\[0.75cm]
    \includegraphics[width=0.475\textwidth]{fig/2d2f_nb}%
    \hspace*{0.05\textwidth}%
    \includegraphics[width=0.475\textwidth]{fig/2d2f_b}\\
    \includegraphics[width=0.475\textwidth]{fig/legend_2d2f_nb}%
    \hspace*{0.05\textwidth}%
    \includegraphics[width=0.475\textwidth]{fig/legend_2d2f_b}
    \caption{Results for \textit{case iii} described in Subsection \ref{subsec:broken_liner} without (left) and with (right) the hole in the liner. The top and the bottom images show the modelled electric potential and current lines with the associated current density, respectively. Please note the different colormap ranges for each image.}
    \label{fig:experiment2a}
\end{figure}
