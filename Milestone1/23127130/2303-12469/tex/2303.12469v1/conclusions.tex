\section{Conclusions}\label{sec:conclusions}

In this work we have presented a mixed-dimensional mathematical model to obtain the electric potential and current density in direct current simulations. The model can handle multiple electrodes and a thin high-resistivity liner included in the domain. These objects have one or two of their dimensions that are very small and thus difficult to be represented exactly by equi-dimensional grids in the simulations. The mixed-dimensional approach used in this work approximates them as object with lower dimension: electrodes as one-dimensional and the liner as two-dimensional objects. New equations have been derived along with new interface conditions to couple the electrodes and the liner with the surrounding domain. To improve the efficiency of the simulation, the electrodes are placed in the background computational grid avoiding an excessive refinement around them. The numerical approximation relies on two cell-centred
finite-volume schemes: the two-point and multi-point flux approximations.
The mathematical model has been tested against laboratory experiments
giving reliable solutions, in particular when the multi-point flux approximation was considered. In the first set of laboratory experiments, we validated the code with an analytical relationship  and showed that the liner influences the measured apparent resistivity at depth much higher then the penetration depth of a Wenner-$\alpha$ array. In the second set of experiments, we analyzed the presence of a hole in the liner for different deployments of the electrodes and observed that some configurations are more favourable to detect possible defects.
We can conclude that the proposed mixed-dimensional model is a reliable tool for direct current simulations. The model can handle particular features that are very important when dealing with geoelectrical investigations of MSWLFs, where large variations of electrical resistivity can occur in very limited space.
The approach we developed could also be exported to other application fields, presenting similar governing equations and peculiarities.
