%%
%% Beginning of file 'sample62.tex'
%%
%% Modified 2018 January
%%
%% This is a sample manuscript marked up using the
%% AASTeX v6.2 LaTeX 2e macros.
%%
%% AASTeX is now based on Alexey Vikhlinin's emulateapj.cls 
%% (Copyright 2000-2015).  See the classfile for details.

%% AASTeX requires revtex4-1.cls (http://publish.aps.org/revtex4/) and
%% other external packages (latexsym, graphicx, amssymb, longtable, and epsf).
%% All of these external packages should already be present in the modern TeX 
%% distributions.  If not they can also be obtained at www.ctan.org.

%% The first piece of markup in an AASTeX v6.x document is the \documentclass
%% command. LaTeX will ignore any data that comes before this command. The 
%% documentclass can take an optional argument to modify the output style.
%% The command below calls the preprint style  which will produce a tightly 
%% typeset, one-column, single-spaced document.  It is the default and thus
%% does not need to be explicitly stated.
%%
%%
%% using aastex version 6.2
%\documentclass[twocolumn,linenumbers]{aastex62}
\documentclass[12pt,twocolumn]{aastex62}
\usepackage[left]{lineno}
%\linenumbers
%% The default is a single spaced, 10 point font, single spaced article.
%% There are 5 other style options available via an optional argument. They
%% can be envoked like this:
%%
%% \documentclass[argument]{aastex62}
%% 
%% where the layout options are:
%%
%%  twocolumn   : two text columns, 10 point font, single spaced article.
%%                This is the most compact and represent the final published
%%                derived PDF copy of the accepted manuscript from the publisher
%%  manuscript  : one text column, 12 point font, double spaced article.
%%  preprint    : one text column, 12 point font, single spaced article.  
%%  preprint2   : two text columns, 12 point font, single spaced article.
%%  modern      : a stylish, single text column, 12 point font, article with
%% 		  wider left and right margins. This uses the Daniel
%% 		  Foreman-Mackey and David Hogg design.
%%  RNAAS       : Preferred style for Research Notes which are by design 
%%                lacking an abstract and brief. DO NOT use \begin{abstract}
%%                and \end{abstract} with this style.
%%
%% Note that you can submit to the AAS Journals in any of these 6 styles.
%%
%% There are other optional arguments one can envoke to allow other stylistic
%% actions. The available options are:
%%
%%  astrosymb    : Loads Astrosymb font and define \astrocommands. 
%%  tighten      : Makes baselineskip slightly smaller, only works with 
%%                 the twocolumn substyle.
%%  times        : uses times font instead of the default
%%  linenumbers  : turn on lineno package.
%%  trackchanges : required to see the revision mark up and print its output
%%  longauthor   : Do not use the more compressed footnote style (default) for 
%%                 the author/collaboration/affiliations. Instead print all
%%                 affiliation information after each name. Creates a much
%%                 long author list but may be desirable for short author papers
%%
%% these can be used in any combination, e.g.
%%
%% \documentclass[twocolumn,linenumbers,trackchanges]{aastex62}
%%
%% AASTeX v6.* now includes \hyperref support. While we have built in specific
%% defaults into the classfile you can manually override them with the
%% \hypersetup command. For example,
%%
%%\hypersetup{linkcolor=red,citecolor=green,filecolor=cyan,urlcolor=magenta}
%%
%% will change the color of the internal links to red, the links to the
%% bibliography to green, the file links to cyan, and the external links to
%% magenta. Additional information on \hyperref options can be found here:
%% https://www.tug.org/applications/hyperref/manual.html#x1-40003
%%

\usepackage{hyperref}
\usepackage{amsmath,mathtools,mathrsfs,amssymb}
\usepackage{graphicx}% Include figure files
\usepackage{bm}% bold math
\usepackage[percent]{overpic}

%% If you want to create your own macros, you can do so
%% using \newcommand. Your macros should appear before
%% the \begin{document} command.
%%
\newcommand{\vdag}{(v)^\dagger}
\newcommand\aastex{AAS\TeX}
\newcommand\latex{La\TeX}

%\defcitealias{kadowaki_etal_2020}{KGM20}

%\newcommand{\bete}[1]{\textcolor{orange}{\texttt{#1}}}
\newcommand{\bete}[1]{\textcolor{red}{#1}}
\newcommand{\tania}[1]{\textcolor{red}{#1}}
\newcommand{\greg}[1]{\textcolor{cyan}{Greg: #1}}

\defcitealias{medinatorrejon_etal_2021}{MGK+21}
\defcitealias{kadowaki_etal_2021}{KGM+21}
%% Tells LaTeX to search for image files in the 
%% current directory as well as in the figures/ folder.
\graphicspath{{./}{figures/}}

%% Reintroduced the \received and \accepted commands from AASTeX v5.2
%\received{\today}
%\revised{January 7, 2018}
\accepted{May 16, 2023}
%% Command to document which AAS Journal the manuscript was submitted to.
%% Adds "Submitted to " the arguement.
\submitjournal{ApJ}

%% Mark up commands to limit the number of authors on the front page.
%% Note that in AASTeX v6.2 a \collaboration call (see below) counts as
%% an author in this case.
%
%\AuthorCollaborationLimit=3
%
%% Will only show Schwarz, Muench and "the AAS Journals Data Scientist 
%% collaboration" on the front page of this example manuscript.
%%
%% Note that all of the author will be shown in the published article.
%% This feature is meant to be used prior to acceptance to make the
%% front end of a long author article more manageable. Please do not use
%% this functionality for manuscripts with less than 20 authors. Conversely,
%% please do use this when the number of authors exceeds 40.
%%
%% Use \allauthors at the manuscript end to show the full author list.
%% This command should only be used with \AuthorCollaborationLimit is used.

%% The following command can be used to set the latex table counters.  It
%% is needed in this document because it uses a mix of latex tabular and
%% AASTeX deluxetables.  In general it should not be needed.
%\setcounter{table}{1}

%%%%%%%%%%%%%%%%%%%%%%%%%%%%%%%%%%%%%%%%%%%%%%%%%%%%%%%%%%%%%%%%%%%%%%%%%%%%%%%%
%%
%% The following section outlines numerous optional output that
%% can be displayed in the front matter or as running meta-data.
%%
%% If you wish, you may supply running head information, although
%% this information may be modified by the editorial offices.
\shorttitle{Particle acceleration by magnetic reconnection in relativistic jets}
\shortauthors{Medina-Torrejon, de Gouveia Dal Pino and  Kowal}
%%
%% You can add a light gray and diagonal water-mark to the first page 
%% with this command:
%\watermark{DRAFT}
%% where "text", e.g. DRAFT, is the text to appear.  If the text is 
%% long you can control the water-mark size with:
%  \setwatermarkfontsize{dimension}
%% where dimension is any recognized LaTeX dimension, e.g. pt, in, etc.
%%
%%%%%%%%%%%%%%%%%%%%%%%%%%%%%%%%%%%%%%%%%%%%%%%%%%%%%%%%%%%%%%%%%%%%%%%%%%%%%%%%

%% Other packages (lhsk)
\usepackage{multirow}

%% This is the end of the preamble.  Indicate the beginning of the
%% manuscript itself with \begin{document}.

%%%%%%%%%%%%%%%%%%%%%%%%%%%%%%%%%%%%%%%%%%%%%%%%%%%%%%%%%%%%%%%%%%%%%%%%%%%%%%%%%
\begin{document}
%%%%%%%%%%%%%%%%%%%%%%%%%%%%%%%%%%%%%%%%%%%%%%%%%%%%%%%%%%%%%%%%%%%%%%%%%%%%%%%%%
%\title{Particle acceleration by relativistic magnetic reconnection driven by kink instability turbulence in Poynting flux dominated jets by PIC-MHD PLUTO code}

\title{Particle acceleration by magnetic reconnection in relativistic jets: the transition from small to large scales}
%\title{Particle acceleration by magnetic reconnection driven by kink instability turbulence in relativistic MHD jets }

%% LaTeX will automatically break titles if they run longer than
%% one line. However, you may use \\ to force a line break if
%% you desire. In v6.2 you can include a footnote in the title.

%% A significant change from earlier AASTEX versions is in the structure for 
%% calling author and affilations. The change was necessary to implement 
%% autoindexing of affilations which prior was a manual process that could 
%% easily be tedious in large author manuscripts.
%%
%% The \author command is the same as before except it now takes an optional
%% arguement which is the 16 digit ORCID. The syntax is:
%\author[0000-0003-4666-1843]{Tania E. Medina-Torrejón}
%%
%% This will hyperlink the author name to the author's ORCID page. Note that
%% during compilation, LaTeX will do some limited checking of the format of
%% the ID to make sure it is valid.
%%
%% Use \affiliation for affiliation information. The old \affil is now aliased
%% to \affiliation. AASTeX v6.2 will automatically index these in the header.
%% When a duplicate is found its index will be the same as its previous entry.
%%
%% Note that \altaffilmark and \altaffiltext have been removed and thus 
%% can not be used to document secondary affiliations. If they are used latex
%% will issue a specific error message and quit. Please use multiple 
%% \affiliation calls for to document more than one affiliation.
%%
%% The new \altaffiliation can be used to indicate some secondary information
%% such as fellowships. This command produces a non-numeric footnote that is
%% set away from the numeric \affiliation footnotes.  NOTE that if an
%% \altaffiliation command is used it must come BEFORE the \affiliation call,
%% right after the \author command, in order to place the footnotes in
%% the proper location.
%%
%% Use \email to set provide email addresses. Each \email will appear on its
%% own line so you can put multiple email address in one \email call. A new
%% \correspondingauthor command is available in V6.2 to identify the
%% corresponding author of the manuscript. It is the author's responsibility
%% to make sure this name is also in the author list.
%%
%% While authors can be grouped inside the same \author and \affiliation
%% commands it is better to have a single author for each. This allows for
%% one to exploit all the new benefits and should make book-keeping easier.
%%
%% If done correctly the peer review system will be able to
%% automatically put the author and affiliation information from the manuscript
%% and save the corresponding author the trouble of entering it by hand.

\correspondingauthor{Tania E. Medina-Torrejon
}
\email{temttm@gmail.com}

\author[0000-0003-4666-1843]{Tania E. Medina-Torrej\'{o}n}           
\author[0000-0001-8058-4752]{Elisabete M. de Gouveia Dal Pino}
\affiliation{Universidade de S\~{a}o Paulo, Instituto de Astronomia, Geof\'{i}sica e Ci\^{e}ncias Atmosf\'{e}ricas, Departamento de Astronomia, 1226 Mat\~{a}o Street, S\~{a}o Paulo, 05508-090, Brasil} 

\author[0000-0002-0176-9909]{Grzegorz Kowal}
\affiliation{Escola de Artes, Ci\^encias e Humanidades - Universidade de S\~ao Paulo,
Av. Arlindo B\'ettio, 1000 -- Vila Guaraciaba, CEP: 03828-000, São Paulo - SP, Brazil}


%% Note that the \and command from previous versions of AASTeX is now
%% depreciated in this version as it is no longer necessary. AASTeX 
%% automatically takes care of all commas and "and"s between authors names.

%% AASTeX 6.2 has the new \collaboration and \nocollaboration commands to
%% provide the collaboration status of a group of authors. These commands 
%% can be used either before or after the list of corresponding authors. The
%% argument for \collaboration is the collaboration identifier. Authors are
%% encouraged to surround collaboration identifiers with ()s. The 
%% \nocollaboration command takes no argument and exists to indicate that
%% the nearby authors are not part of surrounding collaborations.

%% Mark off the abstract in the ``abstract'' environment. 


\begin{abstract}
%There  has been recent results, based on particle-in-cell (PIC) simulations, suggesting that particle acceleration by magnetic reconnection would not be sustained at large scales of turbulent flows. In contrast, several MHD works and, in particular, the recent one by Medina-Torrejon et al. ApJ (2021) based on MHD simulations of relativistic jets has demonstrated that reconnection acceleration driven by the  turbulence in the flow is sustained from the resistive  up to the large injection scales of the turbulence, with particles experiencing   Fermi-type acceleration in the reconnection layers due to the ideal electric fields of the background fluctuations ($V\times B$) up to ultra-high-energies. In this letter, we show results of PIC-MHD simulations that follow the early stages of the particle acceleration in the relativistic jets which confirm these previous results. This indicates that direct extrapolation from the resistive small scales probed by PIC simulations (wherein non-ideal accelerating electric fields prevail), to large MHD scales should be taken with caution.   


 Several MHD works and, in particular, the recent one by %\cite{medinatorrejon_etal_2021} 
 Medina-Torrejon et al. (2021) based on three-dimensional MHD simulations of relativistic jets, have evidenced that particle  acceleration by magnetic reconnection driven by the  turbulence in the flow occurs from the resistive  up to the large injection scale of the turbulence.  Particles experience   Fermi-type acceleration up to ultra-high-energies, predominantly  of the parallel velocity component to the local magnetic field, in the reconnection layers in all scales due to the ideal electric fields of the background fluctuations ($V\times B$, where $V$  and $B$ are the velocity and magnetic field of the  fluctuations, respectively). In this work, we show MHD-particle-in-cell (MHD-PIC) simulations following the early stages of the particle acceleration in the relativistic jet which confirm these previous results, demonstrating the strong potential of magnetic reconnection driven by turbulence to accelerate relativistic particles  to extreme energies in magnetically dominated flows. %This is in contrast with  recent results based on pure PIC simulations that suggest that particle acceleration by  reconnection would not be sustained at large scales.
 Our results also show that the dynamical time variations of the background magnetic fields do not influence the acceleration of the particles in this process.
 %This apparent inconsistency is essentially due to the difference in scales and indicates that direct extrapolation from the resistive small scales probed by PIC simulations (wherein non-ideal accelerating electric fields generally prevail), to the large MHD scales should be considered with caution.   

\end{abstract}



%% Keywords should appear after the \end{abstract} command. 
%% See the online documentation for the full list of available subject
%% keywords and the rules for their use.
\keywords{acceleration of particles - magnetic reconnection - magnetohydrodynamics (MHD) - particle-in-cell - methods: numerical}

%% From the front matter, we move on to the body of the paper.
%% Sections are demarcated by \section and \subsection, respectively.
%% Observe the use of the LaTeX \label
%% command after the \subsection to give a symbolic KEY to the
%% subsection for cross-referencing in a \ref command.
%% You can use LaTeX's \ref and \label commands to keep track of
%% cross-references to sections, equations, tables, and figures.
%% That way, if you change the order of any elements, LaTeX will
%% automatically renumber them.
%%
%% We recommend that authors also use the natbib \citep
%% and \citet commands to identify citations.  The citations are
%% tied to the reference list via symbolic KEYs. The KEY corresponds
%% to the KEY in the \bibitem in the reference list below. 

\section{Introduction} \label{sec:intro}

The role of magnetic reconnection  in the acceleration of energetic particles has lately gained tremendous importance in high energy astrophysics \citep{dalpino_lazarian_2005,giannios_etal_09,dgdp_etal_10,zhang_yan_11,hoshinoLyu2012,mckinney_2012,arons2013,kadowaki_etal_15, singh_etal_15,zhangli2015,zhang_etal_2018}. 
%(de Gouveia Dal Pino and Lazarian 2005; Giannios et al. 2009; de Gouveia Dal Pino et al. 2010; Zhang and Yan 2011; Arons 2012; Hoshino and Lyubarsky 2012; McKinney and Uzdensky 2012; \cite{kadowaki_etal_15, singh_etal_15,},  Zhang et al. 2015, 2018, ...). 
It is now regarded as a strong candidate for the production
of ultra-high energy cosmic rays (UHECRs)  \citep[e.g.][]{medinatorrejon_etal_2021} and very high energy (VHE) flares in the magnetically dominated regions of relativistic sources (i.e., where the magnetic energy is of the order or exceeds the rest mass energy of the particles) \citep[e.g.,][]{cerutti_etal_2013,yuan_etal_2016,lyutikov_etal_2018,petropoulou_etal_2016,christie_etal_19,mehlhaff_etal_2020,kadowaki_etal_2021}.

%It is general consensus that magnetic reconnection plays a crucial role in the acceleration of energetic particles in the Earth magnetotail (e.g., \cite{dahlin_2020,lazarian09}) and the solar corona (e.g., \cite{drake_etal_2006,drake09,gordovskyy10,gordovskyy11,li_etal_2017}), but lately  this mechanism has gained tremendous importance  in the framework of high energy astrophysical phenomena (de Gouveia dal Pino and Lazarian 2005; Giannios et al. 2009; de Gouveia Dal Pino et al. 2010; Zhang and Yan 2011; Arons 2012; Hoshino and Lyubarsky 2012; McKinney and Uzdensky 2012; \cite{kadowaki_etal_15, singh_etal_15},  Zhang et al. 2015, 2018, ...). It is now regarded as a strong candidate for the production of ultra-high energy cosmic rays (UHECRs) (e.g.  \cite{ medina-torrejon_etal_2021}) and very high energy flaring in magnetically dominated regions of relativistic sources (i.e., where the magnetic energy exceeds the rest mass energy of the particles) (e.g., Cerutti et al. 2013;  Yuan,  Nalewajko,  Zrake et al. 2016;  Lyutikov,  Komissarov,  Sironi, et al. 2018. Petropoulou,  Giannios, and  Sironi 2016, Christie, et al. 2019; Mehlhaff, Werner et al. 2020,  Kadowaki et al. 2021).

The comprehension of particle acceleration driven by magnetic reconnection has greatly improved  thanks to both  particle-in-cell (PIC) simulations (predominantly performed in two-dimensions - 2D) 
%that probe the resistive (kinetic) very small scales 
\citep[e.g.,][]{zenitani_H_2001,drake_etal_2006,zenitani_H_2007,zenitani_H_2008,lyubarsky_etal_2008,drake_etal_2010,clausen-brown_2012,cerutti_etal_2012,cerutti_etal_2014,li_etal_2015,werner_etal_2018,werner_etal_2019,lyutikov_etal_2017,sironi_spitkovsky_2014,guo_etal_2015,guo_etal_2016,guo_etal_2020,
%,werner_uzdensky2017,werner_etal_2016,kagan_etal_2013,melzani_etal_2014,li_etal_2017,jaroschek04,bessho12,liu_etal_2015,liu_etal_2011,nalewajko15,
sironi_etal_2015,
%sironi_etal_2016,kagan_etal_2016,rowan_etal_2017,kagan_etal_2018,
ball_etal_2018,kilian_etal_2020,sironi2022}), 
%\bete{[TANIA: acrescenta por favor as refs. de Guo et al. 2020; kilian 2020; ball etal 2018]},
and MHD  simulations (generally performed in 3D) 
%which probe the large scales of the process 
\citep[e.g.,][]{kowal_etal_2011,kowal_etal_2012,delvalle_etal_16,beresnyak_etal_2016,guo_etal_2019, medinatorrejon_etal_2021}.
%(e.g., Kowal et al. 2011, Kowal et al. 2012; del Valle et al. 2016; Beresnyak and Li 2016; Guo et al. 2019??; 
%\bete{Yang et al. 2020)}: somente simulacao MHD de reconexao turbulenta, sem partículas. 
They both have established reconnection as an efficient process of acceleration.



%It is  well accepted that magnetic reconnection plays a crucial role in the acceleration of energetic particles in the Earth magnetotail (e.g., \cite{dahlin_2020,lazarian09}) and the solar corona (e.g., \cite{drake_etal_2006,drake09,gordovskyy10,gordovskyy11,li_etal_2017}), but lately  this mechanism has gained  importance also in the framework of high energy astrophysical phenomena. It is now regarded as a strong candidate for the production of the most energetic particles coming from space, the so called ultra-high energy cosmic rays (UHECRs) (e.g.  \cite{dalpino_etal_2020, medina-torrejon_etal_2021}). 
%are ubiquitous in astrophysical systems and environments and their acceleration 
%With energies  above $\simeq 10^{17}$ eV, %to 
%the so called GZK limit   and  $\sim 10^{21}$ eV,  
%The UHECR spectrum is consistent with an extragalactic origin within  sources  that range from the birth of compact objects to flares of gamma-ray bursts (GRBs) and  active galaxies (AGNs),  specially blazars (the high luminous branch of AGNs) 
%(see e.g.   \cite{kotera11,deGouveiadalPinoKowal:2015,matthews_etal_2020} for reviews). 

%Closely related to this, and equally debated, is the  origin of the highly variable gamma-ray emission coming from these sources \cite{sol13}. For instance, ground based gamma-ray observatories 
%(HESS, VERITAS and MAGIC) have detected very short duration gamma-ray flares of minutes  at  the very high energy (VHE)  band from several blazars (e.g., 
%TeV band of the blazars PKS 2155-304 \cite{Aharonian2007a},  MRK501 \cite{albert_etal_07}, 3C 279 \cite{ackermann_etal_2016}, and 3C 54.3 \cite{britto_elal_2016}), which  imply extremely compact acceleration/emission regions
%($< R_S/c$, where $R_S$ is  the Schwartzschield radius) 
%with Lorentz factors much larger than the typical  bulk values  in the jet of these sources ($\Gamma_{bulk} \simeq$ 5--10). 
%The large Lorentz factors are necessary in order to prevent the re-absorption of the observed gamma-rays  within the source due to electron-positron pair creation  \cite{begelman_etal_08}. 
%The standard particle acceleration in shocks cannot account for this emission structure, particularly if it comes from the magnetically dominated regions of these sources. Indeed, the only  mechanism  that seems to be able to explain both, the high variability and
%compactness of the TeV emission in these sources is particle acceleration  in magnetic reconnection layers to PeV or larger energies \cite{giannios_etal_2009,giannios_2013,kushwaha_etal_17,medina-torrejon_etal_2021}.  A similar process has been proposed 
%also invoked 
%to explain the 
%transition from magnetically to kinetically dominated flow and the 
%prompt  emission in gamma-ray-bursts \cite{giannios_2008,zhang_yan_11}. Furthermore, the recent outstanding  simultaneous detection of gamma-rays and  high-energy neutrinos from the blazar TXS 0506$+$056
%\cite{aartsen_etal_2018}, has confirmed the potetial capability of AGN jets as UHECR accelerators, evidencing  for the first time the presence of high-energy protons interacting with ambient photons, producing  pions and  subsequent decaying in gamma-rays and neutrinos.  It has been speculated  that if these protons are produced in the magnetically dominated regions of the jet near the source core, then they are most probably accelerated by fast magnetic reconnection \cite{dalpino_etal_2018,dalpino_etal_2020}.




%Originally, particle acceleration in reconnection sites was essentially envisioned as a linear process due to the advective electric field (also referred as the reconnection electric field) that develops along the current sheet, in the normal direction to the reconnecting magnetic field ($\epsilon = V_r B/c$, where $V_r$ is the reconnection velocity) (e.g., \cite{speiser65,litvinenko96,zenitani01,giannios_10}).  While describing  a betatron-like orbit (also  called Speiser orbit) along this direction, particles are continuously accelerated by the electric field  with its energy increasing linearly with the distance (z)  travelled in the current sheet or reconnection layer ($E \sim  e V_R B z/c$) \cite{speiser65}.


Our understanding is that particles are predominantly accelerated in reconnection sites by a Fermi-type mechanism in ideal electric fields \citep{dalpino_lazarian_2005,drake_etal_2006,kowal_etal_2012,guo_etal_2019}.
%, henceforth GL05)   
%proposed a more efficient way to accelerate particles to relativistic velocities within the reconnection layer in analogy to  the first-order Fermi process that occurs in shocks.
%which is able to increase the energy exponentially.
They undergo multiple crossings in the two converging magnetic fluxes of opposite polarity moving to each other at the reconnection velocity ($V_{rec}$), thereby gaining energy from
%through a first-order Fermi 
head-on interactions with background magnetic irregularities \citep[see also][for reviews] {lazarian_etal_2012,dalpino_kowal_15,lazarian20}.
%Magnetic reconnection leads to  shrinking of  magnetic loops and this  induces  charged particles  in the magnetic loops to get accelerated and to interact  with the bulk magnetic mirrors of the flux (\cite{dalpino_lazarian_2005, kowal_etal_2011}). Following the first-order Fermi formalism, it is easy to show that  the particle energy gain after each round trip is $\Delta E/E \simeq V_{R}/c$ which, for constant $V_{R}$,  implies an exponential growth of the energy in time.  
In order to produce fast reconnection and hence, efficient particle acceleration, the  ubiquitous turbulence in astrophysical MHD flows is 
%always invoked as one of the main driving mechanisms. %According to Lazarian and Vishniac able to drive fast reconnection according to LV99, as stressed in the previous sections, and %\cite{dalpino_lazarian_2005} actually conceived their first-order Fermi picture having in mind the possible presence of turbulent reconnection. 
%The ubiquitous turbulence in astrophysical flows 
 acknowledged as one of the main driving mechanisms.   The wandering of the magnetic field lines in the turbulent flow allows for many simultaneous events of reconnection and the enlargement of the outflow regions, removing the reconnected flux more efficiently. These two factors result in the reconnection rate being a substantial fraction of the Alfvén speed and  independent of the microscopic magnetic resistivity  (i.e., independent of the Lundquist number and depending only on the parameters of the turbulence) \citep[][]{lazarian_vishiniac_99,kowal_etal_09,eyink2013,takamoto_etal_15,santoslima_etal_2010,santos-lima_etal20,lazarian20}. 
 The intrinsic 3D nature of  the turbulent reconnection %\citep{lazarian_vishiniac_99,lazarian20} 
 and the particle acceleration that it entails makes the process more efficient than the acceleration in the 2D shrinking plasmoids and X-points that are usually excited by tearing mode instability in  PIC \citep{hoshinoLyu2012,drake_etal_2006,sironi_spitkovsky_2014} and in resistive MHD  \citep[e.g.,][]{kowal_etal_2011,puzzoni_etal_2022} simulations. Moreover, 2D plasmoids are nothing but the cross section of 3D reconnecting magnetic flux tubes, and particle acceleration in nature cannot be confined to 2D plasmoids.
 %, being actually 3D. 
 This has been successfully verified in  3D MHD simulations considering  the injection of thousands of test particles in a current sheet with embedded forced turbulence 
 %driving fast reconnection 
 \citep{kowal_etal_2012,delvalle_etal_16}. In these simulations, the formation of a thick volume filled with large number of converging  reconnecting layers covering the entire inertial range of the turbulence, from the resistive to the injection scale, allows particle acceleration up to the very large scales of the system and to high energies. 
 %Kowal et al.  (2011) \cite{Kowal_etal:2011}, in particular, \bete{injecting thousands of test particles in multiple Harris current sheet layers,}    showed  that particle acceleration in 3D MHD reconnection behaves quite differently from the acceleration in 2D domains. These results have called for focusing on more realistic 3D geometries of reconnection. Kowal et al. (2012) \cite{Kowal_etal:2012}  and del Valle et al. (2015)  \cite{delvalle_etal_16}, for instance,  introduced turbulence within a 3D current sheet with initial Sweet-Parker configuration and followed the trajectories of test protons in this domain. 
%An important consequence of  fast reconnection driven by turbulent magnetic fields is the formation of a thick volume filled with large number of converging  current sheets,  \bete{from the resistive to the injection scale of the turbulence,} that produce the broadening of the acceleration region (see also Sec. \ref{mag_shear}).  Figure \ref{trajectory} shows a single test particle being scattered  by the irregularities in this domain.  Moreover, the reconnection speed, which in this case is independent of resistivity (LV) and determines the velocity at which the current sheets scatter particles, is naturally increased to a substantial fraction of the Alfv\'en speed, as stressed previously \cite{Kowal_etal:2009}. 
These are crucial  differences with regard to PIC simulations  which can probe only the kinetic small (resistive) scales of the acceleration process, dealing  with  large intrinsic  resistivity wherein particles are predominantly accelerated by  non-ideal electric fields and only up to a few thousand times their rest mass energy. 
Due to these differences one has to be very cautious
%These differences make it necessary to be very cautious 
%{lazarian_etal_2012,lazarian20}..
when extrapolating the results of particle acceleration from PIC simulations to the macroscopic scales of real systems \citep[see e.g. review in][]{lazarian_etal_2012}. %\bete{Lazarian et al. 2023)}.
%{lazarian20}.

%This is a  crucial   difference with regard to PIC simulations,  While the kinetic PIC simulations  deal with  large intrinsic   resistivity and can probe only the resistive small scales  of the acceleration process, the MHD simulations with test particles   probe the macroscopic scales, i.e.,  particle acceleration in the  reconnection sheets driven  in all scales of the turbulent MHD flow inertial range: from the minimum resistive scale (Ohmic scale) up to the injection large scale of the turbulence, which is of the order of the size of the real system. Therefore, only MHD simulations can really probe particle acceleration up to ultra-high energies (UHECRs). Furthermore, a natural consequence of the scale limitation is that one cannot extrapolate straightforwardly the results from PIC simulations to the macroscopic scales. In particular, the accelerating electric field is  distinct in both scales. These points will be further discussed in the next sections. 

%...Apart from the first-order Fermi acceleration in reconnection layers, the stochastic second-order Fermi acceleration is also possible in purely turbulent environments. This acceleration can be due to the resonance interactions or Transient-Time Damping (TTD) acceleration (Schlickeiser 2006). This is a classical acceleration process for a turbulent fluid. In addition, since  an MHD turbulent flow
%, even without any large scale magnetic field reversals, suffers reconnection, then further acceleration
%according to Brunetti \& Lazarian (2016), is also possible in  reconnection layers driven by the turbulence itself,} according to Brunetti \& Lazarian (2016) \cite{Bru_Laz2016} (see also \cite{kowal_etal_2012}. The stochasticity of the acceleration arises from the fact that in MHD turbulence the zones of reconnection in which the length of magnetic field lines shrink are necessarily accompanied with the regions of dynamo over which the length of magnetic field lines increases. In the latter regions, the particles slow down. The peculiarity of this process is that the acceleration in the regions of reconnection can increase the momentum  not by a small fraction $V/c p$, but by the increment of the momentum $\delta p$, that  can be of the order of $p$. Hence, this type of acceleration was termed in \cite{Bru_Laz2016} "the one and a half Fermi acceleration".

The MHD studies mentioned above \citep{kowal_etal_2012,delvalle_etal_16}  considered particle acceleration in non-relativistic domains of 3D reconnection. %due to ideal electric fields. 
%\greg{solely do to the ideal fields.} 
More recently, \citet{medinatorrejon_etal_2021} \citepalias[hereafter][]{medinatorrejon_etal_2021} and \citet{kadowaki_etal_2021} \citepalias[hereafter][]{kadowaki_etal_2021}, motivated by current debates related to the origin of cosmic ray acceleration and VHE variable emission in relativistic jets, and specially in blazars \citep[e.g.,][]{aharonian_etal_07,ackermann_etal_2016,britto_elal_2016,aartsen_etal_2018},
%\cite{medinatorrejon_etal_2021} and \cite{kadowaki_etal_2021}
%(e.g.  \citep{aharonian_etal_07}, 
% %\citep{albert_etal_07}
% \citep{ackermann_etal_2016}, \citep{britto_elal_2016}, \citep{aartsen_etal_2018}), 
% Medina-Torrejon et al. (2021) \cite{medinatorrejon_eta_2021} \bete{(hereafter MGK+21)}  and Kadowaki et al. (2021) \cite{kadowaki_etal_2021}
investigated particle acceleration 
in a 3D relativistic magnetically dominated  jet 
%with magnetization parameter $\sigma \sim 1 $,
%magnetization parameter
%$\sigma = B_0^2/ \gamma^2 \rho h =$ 0.6 (where $h$ is the specific enthalpy) at the jet axis. 
subject to  current driven kink  instability (CDKI), by means of relativistic MHD simulations \citep[using the \texttt{RAISHIN} code;][]{mizuno_etal_12, singh_etal_16}.
The instability drives  turbulence and fast magnetic reconnection in the jet flow. 
Its  growth and saturation causes the excitation of large amplitude wiggles along the jet and the disruption of the initial helical magnetic field configuration, leading to the
 formation of several sites of fast reconnection. The turbulence developed  follows approximately a Kolmogorov spectrum 
 \citepalias[][]{kadowaki_etal_2021}.
 %\citep{kadowaki_etal_2021}.
Test protons injected in the nearly stationary snapshots of the jet,  experience an exponential acceleration in time, predominantly its momentum component  parallel to the local field,
%its parallel momentum component to the local field, 
up to a maximum energy.  For a background magnetic field of $B \sim 0.1$ G, this saturation  energy is 
$\sim 10^{16}$ eV, while for $B \sim 10$ G it is  $\sim 10^{18}$ eV. 
There is a clear association of the accelerated particles with the regions of fast reconnection and largest current density.
The particles interact with magnetic fluctuations from the small dissipative 
%resistive
scales up to the injection scales of the turbulence, which is of the order of the size of the jet diameter. For this reason, the Larmor radius of the particles attaining the saturation energy, which gives the maximum size of the acceleration region, is also of the same order.
 Beyond the saturation value, the particles suffer further acceleration to energies up to 100 times larger, but at a slower rate, due to drift in the   largest scale non-reconnecting fields.
%In the early stages of the development of the non-linear growth of CDKI in the jet, when there are still no sites of fast reconnection, injected particles are also efficiently accelerated, but by magnetic curvature drift in the wiggling jet spine. However, in order to particles to be accelerated by this process, they have to be  injected  with an initial energy  much larger than that required for  particles to accelerate in reconnection sites.
%Finally, we have also obtained from the simulations an acceleration time due to reconnection with a weak dependence on the particles  energy $E$,$t_A \propto E^{0.1}$.
The energy  spectrum of the accelerated particles develops a  high energy tail with a power law index $p \sim$ -1.2 in the beginning of the acceleration, in agreement with earlier works \citepalias[][]{medinatorrejon_etal_2021}. 
%Our resultsprovide an appropriate multi-dimensional framework for exploring this process in real systems and explain their complex emission patterns, specially in the very high energy bands and the associated neutrino emission recently detected in some blazars.








%in a 3D relativistic magnetically dominated  jet 
%with magnetization parameter $\sigma \sim 1 $,
%magnetization parameter
%$\sigma = B_0^2/ \gamma^2 \rho h =$ 0.6 (where $h$ is the specific enthalpy) at the jet axis. subject to  current driven kink  instability (CDKI).The instability drives  turbulence and fast magnetic reconnection in the jet flow. 
%Its  growth and saturation causes the excitation of large amplitude wiggles along the jet and the disruption of the initial helical magnetic field configuration, leading to the formation of several sites of fast reconnection. The turbulence developed  follows approximately a Kolmogorov spectrum \cite{kadowaki_etal_2021}.

 %The analysis of the structure and distribution of these reconnection sites  has confirmed its turbulent nature, with the magnetic field distribution  following a Kolmogorov spectrum, and the reconnection velocities following a log-normal distribution  \cite{kadowaki_etal_2021}. 
 
 %The average reconnection rate obtained from these simulations, $V_r \simeq 0.05 V_A$, with maximum values around $0.4 V_A$,  is compatible with the results of both relativistic   \cite{takamoto_etal_15}  and non-relativistic 3D models of reconnection in current sheets with slab geometries \cite{Kowal_etal:2012,delvalle_etal_16}. 
 
 %As in  \cite{Kowal_etal:2012,delvalle_etal_16}, thousands of test protons injected in the nearly stationary turbulent jet (Fig. \ref{pos3DHisto}), experience an exponential acceleration in time, predominantly  in its parallel momentum component to the local field,  up to a maximum energy. For  a jet with initial  magnetic field  $B \sim 0.1 $ G ($B \sim 10 $ G), this energy  is $E_p\sim 10^{7} m_p c^2 \simeq  10^{16}$ eV ($\sim 10^{18}$ eV).
%The giro-radius  of the particles attaining this  energy corresponds to the size of the acceleration region, given by the diameter of the perturbed jet.  
%During this exponential acceleration, the parallel momentum   to the local magnetic field is the dominant .} 
%This regime of particle acceleration lasts for several hundred hours until the saturation energy. The simulations reveal a clear association of the accelerated particles with the regions of fast reconnection and largest current density), indicating its dominant role on the acceleration process. 
%During this time, the particles undergo interactions with magnetic fluctuations from the small resistive scales up to the injection scales of the turbulent structures} driven by the CDKI, which are  of the order to the jet diameter.
%Beyond the saturation energy, the particles suffer further acceleration at a  slower rate, due to drift in the non-reconnected fields (mostly of the perpendicular component of their momentum) up to energies  100 times larger. This contributes to extend the CR particle spectrum to UHEs up to $10^{20}$ eV, directly demonstrating for the first time, that the magnetically dominated regions of relativistic jets in astrophysical sources like AGNs  (in particular, blazars) and GRBs can produce UHECRs by turbulence driven magnetic reconnection acceleration. 

%Though we have not taken into account particle losses, such as non-thermal radiation, electron-positron pair production, particle back reaction into the jet plasma, or particle diffusion, the  energies achieved by the particles, $\sim 10^{16}$  eV or $\sim 10^{18}$ eV (or even larger), depending on the strength of the background magnetic fields $\sim 0.1$ G or $\sim 10$ G,  are more than sufficient to explain energetic particles and even ultra-high-energy-cosmic-rays (UHECRs) in these sources. Protons with these energies could explain observed very high energy (VHE) emission, as well as the production of  neutrinos, 

%\cite{medina-torrejon_etal_2021}  have also obtained from the simulations an acceleration rate due to reconnection with slightly weaker  dependence on the particles  energy than in the non-relativistic studies \cite{delvalle_etal_16},  $t_A^{-1} \propto E^{-0.1}$, implying faster rate, and compatible with exponential growth in time. This is consistent with the fact that the jet has relativistic Alfv\'en velocities and thus intrinsically higher reconnection speeds $V_{r} \simeq 0.05 V_A$ that  naturally make the process slightly more efficient.

%The energy  spectrum of the accelerated particles in the jet also develops a  high energy tail with a power law index $p \sim$ $-1.0,-1.2$ in the beginning of the acceleration (Fig. \ref{NE50}), in agreement with  non-relativistic works as well \cite{delvalle_etal_16}. We should note, however, that this hard spectrum  derivation does not account for particle losses, such as non-thermal radiation or particle back reaction into the jet plasma, so that it can be much steeper, as indicated by the observations.  

% \cite{medina-torrejon_etal_2021} also found that in  the early stages of the development of the non-linear growth of CDKI in the jet, before the development of  fast reconnection, injected particles could be efficiently accelerated by magnetic curvature drift in the wiggling jet spine. A similar process was also identified by  \cite{alves_etal_2018}   in their PIC simulations of relativistic jets. However, in order to the particles to be accelerated by this process, they have to be  injected  with an initial energy  much larger than that required for  particles to accelerate in the reconnection sites. Besides, this regime of acceleration can be sustained only until the development of fast reconnection driven by the unavoidable growth of the turbulence.

%These results provide an appropriate multi-dimensional framework for exploring relativistic reconnection acceleration  in real systems and explain their complex emission patterns, specially in the very high energy bands and even the associated neutrino emission recently detected in some blazars, as stressed.

%These results provide an appropriate multi-dimensional framework for exploring relativistic reconnection acceleration  in real systems and explain their complex emission patterns. Protons with the energies obtained above could explain not only  observed VHE emission, but also the production of the neutrinos recently detected in some blazars \cite{rodriguez-ramirezetal2022}, as stressed in the beginning of this Section.


In this work, we present results of 3D MHD-PIC simulations of relativistic jets \citep[using the \texttt{PLUTO} code;][]{mignone_etal_2018}, considering in most of the tests the same initial jet setup as in 
%\textcolor{red}{ \citet{medinatorrejon_etal_2021} 
\citetalias[][]{medinatorrejon_etal_2021} and %\citet{kadowaki_etal_2021}
\citetalias[][]{kadowaki_etal_2021}. Our main goals here are: (i) to test the early stages of the acceleration of the particles evolving at the same time that the jet develops the turbulence driven by the CDKI; (ii) to compare with these previous studies which were performed with test particles launched in the MHD jet after it achieved a nearly steady state regime of fully developed  turbulence; and (iii) to investigate potential effects of the background magnetic field dynamical time evolution on particle acceleration. We find that the results are very similar to the previous studies.  Particles are accelerated by the ideal electric field of the background fluctuations in the reconnection layers of the turbulent flow, from the small resistive scale up to the large injection scales of the turbulence. 
%\citepalias[as in][]{medinatorrejon_etal_2021}
Furthermore, the time evolution of the background fields does not affect their acceleration.
%in agreement with previous estimates 
%\citep[as estimated in][]{kowal_etal_2012}. 

%\bete{These results reinforce the importance of fast magnetic reconnection driven by turbulence as a powerful mechanism of particle acceleration in magnetically dominated flows in all scales.}

%These results are in contrast to recent studies based on 3D PIC simulations
%(e.g., Comisso and Sironi 2019, 2020; Sironi et al. 2021) 
%that suggest that reconnection acceleration would be  dominant only in the very early stages of  particle acceleration. This apparent inconsistency is essentially due to the intrinsic difference in scales and indicates that direct extrapolation from the resistive small scales probed by PIC simulations (wherein non-ideal accelerating electric fields generally prevail), to the large MHD scales should be taken with caution.
%(see also Guo et al. 2022).


%These results are in contrast to recent studies based on 3D PIC simulations (e.g., Comisso and Sironi 2019, 2020; Sironi et al. 2021) that suggest that reconnection acceleration would be  dominant only in the very early stages pf  particle energy injection. This apparent inconsistency is essentially due to the intrinsic difference in scales and indicates that direct extrapolation from the resistive small scales probed by PIC simulations (wherein non-ideal accelerating electric fields generally prevail), to the large MHD scales should be taken with caution (see also Guo et al. 2022).

The paper is organized as follows, in Section 2 we describe the numerical method and setup, in Section 3, the results we obtained from the numerical simulations, and in Section 4 we discuss the results and draw our conclusions. 
%~\\

\section{Numerical Method and Setup} \label{sec:numethod}

We performed 3D relativistic MHD-PIC  simulations of a jet using %a modified version of 
the \texttt{PLUTO} code with non explicit resistivity \citep{mignone_etal_2018}. 
%The evolution of the plasma flow follows the RMHD equations and resolves by the Godunov method. While, the particle evolution follows PIC methods. 
%\subsection{3D PIC-MHD jet simulation setup} \label{sec:PICMHDsetup}
We  employed the HLLD Riemann solver  to calculate the fluxes \citep*{mignone_u_b_2009}, a flux-interpolated constrained  transport to control the divergence $\nabla \cdot B = 0$ \citep{mignone_etal_2019}, and a second-order TVD Runge–Kutta scheme to advance the equations in time. 

We have used a similar setup as in \citetalias{medinatorrejon_etal_2021} and \citetalias{kadowaki_etal_2021},  considering a rotating relativistic jet with initial force-free helical magnetic field and  initial decreasing radial density profile \citepalias[for more details, see][]{medinatorrejon_etal_2021} and also \citep{mizuno_etal_12,singh_etal_16}. %(for more details, see Medina-Torrejon et al. 2021). 

The computational domain in  Cartesian coordinates $(x, y, z)$ has dimensions $10L \times 10L \times 6L$,  where $L$ is the length scale unit. The larger domain adopted in the x and y directions is  due to the fact that the jet structure exceeds the boundaries of the box in evolved times.  We have imposed outflow boundaries in the transverse directions x and y and  periodic boundaries in the z direction.  We have considered in most of the simulations  a   grid resolution with $256$ cells in each direction, (implying a cell size in the z direction of $\sim$0.02 L, and in the x and y directions of 0.04 L), but in order to test the convergence of the results we have also run a 
%$\sigma \sim 1$  jet 
model with  $426$ cells in the x and y directions and 256 in e the z direction, (implying a cell size of $\sim$0.02 L in all directions). 
%\bete{For simplicity, these two models are often referred to as 256 and 426 resolution models, respectively.}





%As we mention above, we use the same initial and boundaries conditions as Medina-torrejon et al. (2021) and  Kadowaki et al. (2021), in order to obtain a same magnetization $\sigma \sim 1$. 

The code unit (c.u.) for the velocity is the light speed  $c$, for time is $L/c$,  for  density  is $\rho_0 = $1, for  magnetic field 
%$B_0$ 
is  $\sqrt{4 \pi \rho_0 c^2}$,  and for  pressure is  $\rho_0 c^2$. 

We have considered two different initial values of the magnetization parameter  $\sigma_0 = B_0^2/\gamma^2 \rho h \sim 0.6$  and $10$ at the jet axis, corresponding to a magnetic field $B_0 = 0.7$ and density $\rho = 0.8$, and $B_0= 4.0$ and $\rho= 1.6$, respectively, where $\gamma$ is the Lorentz factor and $h$ is the specific enthalpy (with $\gamma \sim 1$ and $h \sim 1$ at the axis). Hereafter, We will refer to these models simply as the $\sigma \sim 1$ and $\sigma \sim 10$ models.

%[\bete{Tania: dizer qual a entalpia e o fator de Lorenz iniciais no eixo do jato.}]

%We consider two different values of the initial jet magnetization parameter in the jet spine $\sigma = B^2/ \rho \sim 1$  and $10$ \textbf{at the jet axis}, %where $B$ is the magnetic field amplitude and $\rho$ is the density) $\sigma \sim 1$  and 10. 
%corresponding to 
%In order to obtain $\sigma \sim 1$, we use the values of the magnetic field amplitude 
%$B_0 = 0.7$ and density $\rho_1 = 0.8 \rho_0$, 
%and to obtain $\sigma \sim 10$, we use and $B_0 = 4.0$ and $\rho_1 = 1.6 \rho_0$, respectively.


In order to drive turbulence in the  jet, we allow for the development of the current-driven-kink instability (CDKI) by  imposing an initial perturbation in the radial velocity profile as in \citetalias{medinatorrejon_etal_2021} \citep[equation 7; see also][]{mizuno_etal_12,singh_etal_16}.

In the MHD-PIC mode, the test particle trajectories are integrated in the time evolving plasma fields (velocity and magnetic ) using
%the particles evolve along with the flow through 
the Boris pusher method \citep{boris1970} which requires the definition of the 
 charge-to-mass ratio for the particles. We have adopted here  $e/mc = $ 20,000,  which implies a physical length scale relation in cgs units: 
 %
 \begin{equation}
  \left ( \frac{e}{mc} \right ) = \left ( \frac{e}{mc} \right )_{cgs} L_{cgs} \sqrt{\rho_{cgs}}
 \end{equation}
 %
Where $e$ and $m$ are the particle charge and mass, respectively. We have adopted $\rho_{cgs} = 1.67 \times 10^{-24} g$ $cm^{-3}$ (or  $n_{cgs} =1$  $cm^{-3}$), which results  a physical length scale  $L_{cgs} \sim 5.2 \times 10^{-7}$ pc. In most of the models, we integrated the trajectories of 10,000 - 50,000 protons with initial uniform space distribution inside the domain, and initial kinetic energies between $(\gamma_p -1) \sim$  1 and 200, where $\gamma_p$ is the particle Lorentz factor, with velocities randomly generated by a Gaussian distribution.
%of x \& y $\epsilon \, [-1,1]$L, and z $\epsilon \, [0,6]$L, 
%\tania{and uniform deviates in the directions of motion in the domain.} 
%randomly chosen initial positions and directions of motion in the domain. \bete{Tania, a random distribution vale tambem para a corrida MHD-PIC?} 
%\tania{We have assumed an initial random distribution for the particle velocities in the range of $-200c$ to $200c$.}
%~\\

%\subsection{Setup for Test Particle Acceleration}\label{sec:setuptp}

Besides employing the MHD-PIC mode of the \texttt{PLUTO} code to investigate particle acceleration, we have also considered a model where we injected  test particles after the full development of  turbulence  in the jet flow, as in \citetalias{medinatorrejon_etal_2021}. This test was performed with the \texttt{GACCEL} code \citep{kowal_etal_2012,medinatorrejon_etal_2021}. 

We further notice  that, in order to make direct comparisons of the MHD-PIC simulations with the previous work involving test particle injections in frozen-in-time MHD fields, 
%an MHD flow \greg{(in order to make direct comparison of particle acceleration in time varying vs frozen in time MHD fields, we did not account...)}, 
we did not account for the accelerated particles feedback on the background plasma, which will be considered in forthcoming work.



%\textcolor{red}{We use a similar parametrization as in Medina-torrejon et al. (2021) and  Kadowaki et al. (2021). We consider a rotating relativistic jet with initial force-free helical magnetic field $B$, that decreases as a funtion of the radius, and an initial decreasing radial density profile ($\rho = \rho_1 \sqrt{B^2/B^2_0}$, where $B_0$ and $\rho_1$ are the magnetic field and the density amplitud at the jet spine, respectively), for more details see Medina-Torrejon et al. (2021).}

%\textcolor{red}{The code unit (c.u.) for the velocity is the light speed  $c = 1$, the code unit for density is $\rho_0 = $ 1, the length scale unit is $L = 1$, the magnetic field $B$ is in units of $\sqrt{4 \pi \rho_0 c^2}$, the pressure is in units of $\rho_0 c^2$, and the time is in units of $L/c$. }




%\textcolor{red}{ We have adopted a HLLD Riemann solver (Mignone, Ugliano, and Bodo. 2009) to calculated the numerical fluxes, and flux-interpolated constranted transport method (Mignone, Mattia, Bodo, Del Zanna, 2019) is used to control divergence $\nabla \cdot B = 0$ condition, and a second-order TVD Runge–Kutta scheme to advance the equations in time.  The computational domain is $10L \times 10L \times 6L $ in a \textbf{Cartesian coordinate system} $(x, y, z)$. The larger domain adopted in the x and y directions is  due to the fact that the jet structure exceeds the boundaries of the box in evolved times. We consider two different grid resolutions with $256$ cells in each direction and other with $426$ in x \& y direction and $256$ in z direction. We impose outflow boundaries in the transverse directions x \& y, and  periodic boundaries in the z direction. }

%As we mention above, we use the same initial and boundaries conditions as Medina-torrejon et al. (2021) and  Kadowaki et al. (2021), in order to obtain a same magnetization $\sigma \sim 1$. 




%\textcolor{red}{Particles evolve along with the flow in the PIC-MHD module using the Boris pusher method (ref.? ), we use the Cosmic Rays particle module, which requires the definition of the charge-to-mass ratio for the particles in normalized units:}
 % \begin{equation} \left ( \frac{e}{mc} \right ) = \left ( \frac{e}{mc} \right )_{cgs} L_{cgs} \sqrt{\rho_{cgs}}  \end{equation}
% 
% \textcolor{red}{Where $e$ and $m$ are the CR charge and mass, respectively. We have adopted here $e/mc = $ 20,000 and $\rho = 1$ particle per $cm^{3}$, which implies a physical lentgh scale relation $L\sim 5.2 \times 10^{-7}$ pc. We integrate the trajectory for 10,000 - 50,000 protons  with randomly chosen initial positions and directions of motion in the domain. We assume an initial Maxwellian distribution for the particle velocities corresponding to a Lorentz factor of 200.}
%~\\

%\subsection{Setup for Test Particle Acceleration}\label{sec:setuptp}








~\\
\section{Results}\label{sec:results}

\begin{figure*}[h] %jet_points
\centering
   \includegraphics[scale=0.3]{partic3D_r256s01_t20.pdf}
   \includegraphics[scale=0.3]{r256s01_t20.pdf}
 %  \includegraphics[scale=0.3]{partic3D_r256s01_t25.pdf}
 %  \includegraphics[scale=0.3]{r256s01_t25.pdf}
   \includegraphics[scale=0.3]{partic3D_r256s01_t45.pdf}
   \includegraphics[scale=0.3]{r256s01_vrecgt0.05_t45.pdf}
\caption{Three dimensional view of the  $\sigma \sim 1$ jet evolved with the MHD-PIC mode at t = 20  (top), and 45 L/c (bottom). Left panels: the black lines represent the magnetic field, and the circles the 50,000 particles distribution. The color and size of the circles indicate the value of their kinetic energy normalized by the rest mass energy ($\gamma_p -1$). Right panels: the orange color represents  iso-surfaces of half of the maximum of the current density intensity $|J|$, the black lines the magnetic field, and the green squares  correspond to the positions of the fastest magnetic reconnection events,  with reconnection rate $\geq 0.05$. See text for more details.}
\label{jet_points}
\end{figure*}


Figure \ref{jet_points} shows  the  $\sigma \sim 1$ jet evolved with the MHD-PIC mode of the PLUTO code (with a resolution $256^3$) for two snapshots. A total of 50,000 particles were initially injected in the system. The dynamical  evolution of the jet is very similar to the one obtained in \citetalias{medinatorrejon_etal_2021} and \citetalias{kadowaki_etal_2021} with the \texttt{RAISHIN} MHD code.  With the growth of  the CDKI, the initial helical magnetic field structure 
 %($t=? L/c$) 
 starts to wiggle (see $t=20$ L/c) and then,  turbulence develops distorting entirely the field lines and driving fast magnetic reconnection sites, as we see in the right panel for $t=45 L/c$. We note that there are already a few particles being accelerated in the wiggling jet spine at $t=20 L/c$ (left top panel). This is due to curvature drift acceleration, as detected also in the PIC simulations by \citet{alves_etal_2018}, and  in \citetalias{medinatorrejon_etal_2021} with  test particles injected in a similar snapshot of the  background MHD jet (see their Figure 6). Nevertheless, massive particle acceleration takes place only later on, when turbulence and fast reconnection fully develops in the system, as indicated in the left bottom panel at $t=45 L/c$.  The correlation of the accelerated particles (represented by the red   circles with increasing diameter as the energy increases) 
 with the sites of high current density and fast reconnection  (right bottom panel) is evident. A very similar result was obtained for the $\sigma \sim$ 1 jet model run with  larger resolution  ($426^2 \times 256$). In the next paragraphs, we will further quantify these associations.


%As it develops, EM energy is converted into kinetic energy, and this is a striking feature revealed by Figure 2. Note that in this figure, the EM energy is presented in linear scale, while the kinetic energy is in log scale. The initial relaxation of the system to equilibrium leads to a hump in the kinetic and EM energy density curves (until ∼10 L/c).10 The kink instability comes into play only after the relaxation finishes. There is an initial linear growth of the EM energy between 10 and ∼30 L/c due to the increasing wiggling distortion of the magnetic field structure in the jet spine in the initial increase of the CDK instability (which, in log scale, would be harder to perceive; see, e.g., Mizuno et al. 2012). After a slower increase, the kinetic energy undergoes an exponential growth from ~ t 30 L/c to a maximum near ~ t 40 L/c, after which it approximately reaches a plateau while the magnetic energy decreases (see more details in Singh et al. 2016). We note that this plateau time also coincides with the one after which we have detected an increase in the turbulence and the number of fast reconnection sites in Figure 1. This plateau regime characterizes the achievement of nonlinear saturation of the CDK instability and a nearly steady-state turbulent regime in the system. A similar trend has also been detected in the evolution of the average magnetic reconnection speeds, which nearly achieves a plateau around the same epoch (see Figure 8 in KGM20).


Figure \ref{energy} shows the time evolution of the volume-averaged kinetic energy density transverse to the z-axis (upper panel), 
%within a cylinder of radius \bete{??} around the jet axis, 
and the volume-averaged total relativistic electromagnetic energy density ($E_m$) (bottom panel)  
for the $\sigma \sim 1$ jet, as the CDKI grows \citep[see also ][]{mizuno_etal_12,singh_etal_16,medinatorrejon_etal_2021}. 
For this jet model, these quantities are presented for two different resolutions, $256^3$ (solid red lines)  and $426^2 \times 256$ (dot-dashed black lines), and the results are both very similar.  
These curves are also compared with those  obtained by \citetalias{medinatorrejon_etal_2021} \citepalias[and][]{kadowaki_etal_2021} using the \texttt{RAISHIN} code for the same jet model
%, with similar resolution (with cell size of $\sim$0.03 L in all directions) 
(labeled as MGK+21 in  Figure \ref{energy}),
%\footnote{\bete{We note that the $\sigma \sim 1$ jet model run with the RAISHIN code by MGK+21 has a resolution slightly distinct, having a cell size  $\sim 0.03$ L in the three directions.}}), 
and  with the $\sigma \sim 10$ jet. 
Note that $E_m$ is presented in the linear scale, while the kinetic energy is in the log scale.
The results of both $\sigma \sim 1$ jet models are comparable. 
As the CDKI develops,  $E_m$ is converted into kinetic energy.  For the $\sigma \sim 1$ models, the initial relaxation of the system to equilibrium leads to a hump in the kinetic and $E_m$ curves.
%(until $\sim  15$ L/c). melhor nao colocar pois o tempo é distinto para os modelos sigma=1 deste trabalho e de MGK+21.
After this relaxation,   there is an initial   growth of  $E_m$  
%between 10 and $\sim 30$ L/c 
caused by the increasing wiggling distortion of the magnetic field structure in the jet spine due to the initial growth of the CDKI. The kinetic energy, after a slower increase,  undergoes an exponential growth which is a little more advanced in time in the \texttt{PLUTO} run, that starts  around  $\sim  25$ L/c, than in the \texttt{RAISHIN} run (MGK+21), that starts around    $\sim  30$ L/c.  
This causes   the  jet model in this work to achieve earlier a turbulent state than in the model of MGK+21, with a time delay  $\Delta t \sim  5$ L/c between them\footnote{We attribute this small delay to intrinsic numerical differences between the two codes and to the slight difference in the grid resolution. The  $\sigma \sim 1$ jet model run with the \texttt{RAISHIN} code by \citetalias{medinatorrejon_etal_2021} has a cell size  $\sim 0.03$ L in the three directions.}. 
%to a maximum near ~ t 40 L/c, 
After the exponential growth, the kinetic energy reaches  approximately a plateau while $E_m$ decreases. 
This  coincides with full increase of the turbulence  and of the number of fast reconnection events in Figure \ref{jet_points} (bottom right; see also Figure \ref{vrec}). In fact, this plateau characterizes the achievement of  saturation  of the CDKI and a nearly steady-state turbulent regime in the system (see Figure \ref{spectrum}). A similar behaviour has been identified in \citetalias{medinatorrejon_etal_2021} and \citetalias{kadowaki_etal_2021}. We also notice that there is a  difference of at most  $30\%$ in the amplitude  of $E_m$ between the two models.
%, which  
%is within the numerical dispersion
%have a  numerical dispersion of \tania{ $\pm0.0006$ (\texttt{RAISHIN} code) and $\pm 0.0007$}  (\texttt{PLUTO} code). 
 In the  $\sigma \sim 10$ jet, the CDKI clearly increases faster  achieving saturation much earlier, at about half of the time of the $\sigma \sim 1$ jet. 
 
 Since the two models with different resolution for the $\sigma \sim 1$ jet are so similar,  in the rest of the manuscript we consider only the $256^3$ resolution model.
 
 %Also plotted in Figure \ref{energy} are the volume-average energy densities for the 

%Both runs achieve near saturation of the turbulence growth at about the same time and with similar values. For the EM energy density, on the other hand, though the values vary very little, there is a maximum difference of about $30\% $. \textbf{This is probably due to small intrinsic differences between  the two codes. They are both Godunov based, but in  RAISHIN we used an HLLE Riemann solver (Einfeldt 1988; Komissarov, 1999), while in PLUTO we employed ...?. RAISHIN is slightly more dissipative and we find that PLUTO results   slightly larger distortion of the magnetic fields. The more magnetized model ($\simga = 10$) clearly develops the instability much faster in about half of the time of the $\sigma=1$ model.}    


%\textbf{TANIA: VC REPETIU AS CURVAS DE ENERGIA CONSIDERANDO UM RAIO MAIOR? VEJA O QUE DÁ?}

%We start by presenting the results of the 3D PIC-RMHD (Pluto code) simulations of the relativistic jet, and then we describe the results of the injection of test particles into this domain in comparison of test particles into 3D MHD (Raishin code) for different snapshots along the jet evolution.



\begin{figure} %
  \centering
 \includegraphics[scale=0.6]{energy_evol_c.pdf}
\caption{Top: time evolution of the  volume-averaged kinetic energy density transverse to the z-axis within a cylinder of radius $R \leq 3.0L$  for the $\sigma \sim 1$ jet (red solid line for the model with resolution $256^3$  and dashed-dotted black line   for the model with resolution $426^2-256$), and for the  $\sigma \sim 10$ jet (blue solid line). Bottom: volume-averaged relativistic electromagnetic energy density for the same models. For comparison, also plotted with dashed red lines are the results obtained in MGK+21 for the $\sigma \sim 1$. The kinetic energy is presented in log scale, while $E_m$ is in linear scale.
}
\label{energy}
\end{figure}



To  quantify the development  of the turbulence, 
%we  performed a Fourier analysis and 
we have evaluated the three-dimensional power spectra of the magnetic and kinetic energy densities in the  jet, considering averages in  
spherical or ellisoidal 
 shells between $k$ and $k + dk$ (where $k=\sqrt{k_x^2+k_y^2+k_z^2}$ in the Fourier space) \citepalias{kadowaki_etal_2021}. Figure \ref{spectrum} depicts 
 these power  spectra
 %power spectra for the magnetic (top diagram) and kinetic (bottom diagram) energy densities 
 for different 
 %evolved 
 times 
 %\bete{(from $t=?$ to $?$)}                 
 for both,   $\sigma \sim 1$ and $\sigma \sim 10$ jets. A $3D$-Kolmogorov spectrum slope  ($\propto k^{-11/3}$; red dotted line) was included for comparison. 
 The diagrams show inertial ranges 
 %with slopes ($k^\nu$) between $-?$ and $-?$ 
 both for the kinetic    $|\sqrt{\rho}\boldsymbol{v}(\boldsymbol{k})|^2$ 
 and 
 for the magnetic  $|\boldsymbol{B}(\boldsymbol{k})|^2$ energy density spectra
 between $0.2 \lesssim  k \lesssim 25$ (in units of 1/L) 
 %$\gtrsim $
 %(which  is the range of the red dotted line)},  
 in agreement with a Kolmogorov-like spectrum, after $t\simeq30$L/c for the $\sigma \sim 1$ jet and $t\simeq10$L/c for the $\sigma \sim 10$ jet. This indicates a turbulent energy cascade between an injection scale $\sim 5$L and a resistive small scale $\sim 0.11$L.  The magnetic energy spectrum shows a little steeper slope, probably due to the strong (guiding) magnetic  field of the background plasma \citep[see, e.g.,][]{kowal_etal_07,kadowaki_etal_2021}. As expected, the $\sigma \sim 10$ jet has maximum magnetic energy density 10 times larger than the $\sigma \sim 1$ jet.
 %At  the time that the CDKI achieves the saturation ($t\sim 40 L/c$; Figure \ref{energy}), a nearly Kolmogorov-like shape is established in the inertial range of the spectrum, though it is not yet as well defined as in later times ?. 
 The results are comparable to those obtained in 
 %the previous work 
\citetalias{kadowaki_etal_2021} for the $\sigma \sim 1$ jet, as shown in the left diagrams of the figure\footnote{We note that the  turbulent power spectra  of the kinetic and magnetic energy densities   of the $\sigma \sim 1$ jet presented in \citetalias{kadowaki_etal_2021} were produced with  a distinct  normalization from the one used in Figure \ref{spectrum}. For this reason, we have reproduced them again here for direct comparison with  the other spectra of  Figure \ref{spectrum}.}. 
 %\bete{Similar results were also obtained for the $\sigma \sim 10$ model, confirming the turbulent nature of the flow driven by the CDKI ??. }
 

\begin{figure*}[h] 
 \centering
  \includegraphics[scale=0.8]{raishin_spectrum_evol_113.pdf}
  \includegraphics[scale=0.8]{s01_spectrum_evol_113.pdf}
  \includegraphics[scale=0.8]{s10_spectrum_evol_113.pdf}
\caption{Power spectrum  of the magnetic (left) and kinetic (right) energy densities  %showing an inertial range, 
for  the $\sigma \sim 1$ jet model of \citetalias{kadowaki_etal_2021} (upper row), the $\sigma \sim 1$  (middle row) and    $\sigma \sim 10$ (bottom row) jet models of this work,
%by $B_0 \sim 0.1G$ and $\rho \sim$ 1 part/$cm^3$
for different times in unit of L/c.
%of the evolved turbulence. 
The red doted line corresponds to a $k^{-11/3}$ 3D-Kolmogorov spectrum and its extension gives the inertial range of the turbulence for evolved times
$>30$ L/c
%$\gtrsim 30$
for the $\sigma \sim 1$ models, and $>10$ L/c for the $\sigma \sim 10$ model. The wavenumber is in unit of L.}
\label{spectrum}
\end{figure*}

In order to identify fast magnetic reconnection sites in the turbulent flow  of the relativistic jet  and quantify their reconnection velocities, we have used the same  algorithm employed in \citetalias{kadowaki_etal_2021} wherein the method is described in detail  \citep[see also,][]{zhdankin_etal_13,kadowaki_etal_2018b}.
%As in the previous work, we select a sample of cells with a current density value ($\boldsymbol{J}=\boldsymbol{\nabla}\times \boldsymbol{B}$) \deleted{higher than} five times \added{higher than} the average one taken in the whole system ($|\boldsymbol{J}_{max}|> \epsilon \langle |\boldsymbol{J}| \rangle$, for $\epsilon=5$), and choose those cells where $|\boldsymbol{J}_{max}|$ is a local maximum within a subarray data cube of size $3 \times 3 \times 3$ cells
The time evolution of the  magnetic reconnection rate, ${V}_{rec}$, for all identified sites  and  the  time evolution of the average value, $\langle {V}_{rec}\rangle$ (blue line in the upper and middle panels), in units of the Alfvén velocity, are shown in Figure \ref{vrec}.
The evolution of $\langle{V}_{rec} \rangle$   changes more abruptly after  $t \sim25$ in the $\sigma \sim 1$  jet and $t \sim 10$ in the $\sigma \sim 10$  jet, when the CDKI starts to grow exponentially (Figure \ref{energy}). After that, as the CDKI tends to saturation, the average reconnection rate also attains a value $\langle{V}_{rec} \rangle \sim 0.03 \pm 0.02$ for the $\sigma \sim 1$ jet, in agreement with \citetalias{kadowaki_etal_2021} (see their reference model  m240ep0.5 and their Figure 8). For the $\sigma \sim 10$ jet, it is still growing to a plateau to a similar average value (middle diagram) $\langle{V}_{rec} \rangle \sim 0.02 \pm 0.02$.
%\bete{and $\langle{V}_{rec} \rangle \sim 0.05$ for the $\sigma \sim 10$ jet}. 
%\bete{It is interesting to see that the average value is nearly the same for both jet models, but 
A peak reconnection rate of the order $\sim 0.9$ (not shown in the figure)
%\tania{(Na verdade é 0.9, não sei porque chega a valores tão altos)} 
is obtained for the  $\sigma \sim 1$ jet, %\bete{\citepalias[in][the peak is 0.4]{kadowaki_etal_2021} }, 
while a peak value $\sim 0.6$ is attained for the $\sigma \sim 10$ jet. The bottom diagram compares directly the evolution of the average reconnection speed of both models including their respective variances which are similar\footnote{We note that the slightly smaller mean value of the reconnection rate for the larger $\sigma$  model is compatible with the fact that the necessary wandering of the field lines by the turbulence in order to drive fast reconnection is naturally more difficult the larger the strength of the magnetic field \citep{lazarian_vishiniac_99}.}.
 
 %All the models show few and slow events with a peak  around $t=32$, and a gap between $t=33$ and $34$. After this time, the models m240ep0.1 and m240ep0.05 show convergence, but with $\langle \tilde{V}_{rec} \rangle_s$ values larger than those obtained for the reference model m240ep0.5 \added{(at the limit of 1$\sigma$ uncertainty, indicated by the colored shades in Figure \ref{fig:vrec_comp})}, particularly after the CDK instability has achieved the saturation and  quasi-steady-state turbulence is settled in the system, beyond $t=40$. The reconnection regions in the models m240ep0.1 and m240ep0.05 are larger than in model m240ep0.5 since the asymptotic magnetic field and the velocity have been measured at locations where the current density decays to $0.1$ and $0.05$ of the maximum, respectively. Despite the convergence of the models m240ep0.1 and m240ep0.05, we adopted for our reference model the conservative criterion of $0.5|\boldsymbol{J}_{max}|$ (model m240ep0.5), also used in previous works \citep[see, e.g.,][]{kowal_etal_09,zhdankin_etal_13,kadowaki_etal_18}. \added{In fact, we do not expect a convergence of $\langle \tilde{V}_{rec} \rangle_s$ in such comparisons since we can extrapolate the size of a single identified region.}  

%Furthermore, We also notice that $\langle \tilde{V}_{rec} \rangle_s$ achieves a quasi-steady-state behavior with the fastest rates between $t=50$ and $66$ (see Figure \ref{fig:vrec_comp}), demonstrating the correlation between turbulence and fast magnetic reconnection events.


\begin{figure} 
\centering
    \includegraphics[scale=0.4]{vrec_jtot_nsub1ep5.0b0.5tol1.1_hist_s01p.pdf} 
    \includegraphics[scale=0.4]{vrec_jtot_nsub1ep5.0b0.5tol1.1_hist_s10p.pdf} 
    \includegraphics[scale=0.4]{vrec_jtot_nsub1ep5.0b0.5tol1.1_comp.pdf}
\caption{Histogram of the reconnection rate evolution for the   $\sigma \sim 1$  (top) and  $\sigma \sim 10$ jet (middle). The blue line gives the average reconnection rate evolution. Bottom diagram compares the average reconnection rate evolution of the two models and the colored shades correspond to the standard deviations of each model.  
%\bete{[Tania, e a avaliacao do erro do vrec médio? E Qual foi o Jmax adotado?  e o que acontece em torno de t=7 para o jato sigma=10, onde parece que a  velocidade de reconexao tem um pico? Poderia mostrar esses graficos com mais detalhe e com barra de erro? ou como fizemos em Kadowaki et al.?]]}
}\label{vrec}
\end{figure}






In \citetalias{medinatorrejon_etal_2021}, test particles were injected with an initial Mawellian distribution (with initial mean kinetic energy $\left < E_p \right > \sim 10^{-2} m_p c^2$) in the simulated  $\sigma \sim 1$  jet  with already fully developed turbulence (with the \texttt{RAISHIN} code), and accelerated by magnetic reconnection up to VHEs. 
Figure \ref{GACELL-PLUTO45} (upper panel) depicts the kinetic energy growth as a function of time for 1,000 particles injected (with the \texttt{GACCEL} code) in the snapshot $t=50$ L/c of their model \citepalias[see also  bottom panel of Figure 5 in][]{medinatorrejon_etal_2021}. 
The lower panel of Figure \ref{GACELL-PLUTO45}  shows  a similar  plot, but obtained  for particles injected (also with the \texttt{GACCEL} code) in the fully turbulent jet simulated in this work with the \texttt{PLUTO} code, at $t=45$ L/c. As remarked previously in Figure \ref{energy}, the model run here develops turbulence earlier, with an advance in time of $\Delta t \sim 5$ L/c and thus, in order to compare with \citetalias{medinatorrejon_etal_2021} results, we have considered the corresponding earlier snapshot.   The results are very similar, as expected. As in \citetalias{medinatorrejon_etal_2021},  particles are accelerated exponentially in the  magnetic reconnection sites in all scales of the turbulence driven by the CDKI up to $\sim 10^7 m c^2$, which corresponds to a Larmor radius comparable to the  diameter of the  jet and the size of the largest turbulent magnetic structures (see the plot in the inset). As we see in the figure, beyond this energy, particles suffer further acceleration at a smaller rate, which is attributed to drift in the large scale non-reconnected fields.  We also see that the parallel component of the velocity is predominantly accelerated in the exponential regime, as expected in a Fermi-type process, while in the drift regime, it is the perpendicular component that prevails (see \citetalias{medinatorrejon_etal_2021} for more details). 



%\subsection{Particle acceleration}

\begin{figure} %
  \centering
   \begin{overpic}[scale=0.44]{Raishint50_240_1000_oB-1_en_vpervpar.pdf}
    \put(14,50){\includegraphics[scale=0.23]{Raishint50_240_1000_oB-1_gy.pdf}}
  \end{overpic} 
  \begin{overpic}[scale=0.44]{Plutot45_256_1000_oB-1_en_vpervpar.pdf}
    \put(14,50){\includegraphics[scale=0.23]{Plutot45_256_1000_oB-1_gy.pdf}}
  \end{overpic}  
\caption{Kinetic energy evolution, normalized by the proton rest mass energy, for 1,000  particles injected into the fully turbulent snapshot $t =50$ L/c  of  the $\sigma \sim 1$ jet run by \citetalias{medinatorrejon_etal_2021} (top). The same for particles injected into the the snapshot $t =45$ L/c of the $\sigma \sim 1$ jet in this work (bottom). 
 The colors indicate which velocity component is being accelerated (red or blue for the parallel or perpendicular component to the local magnetic field, respectively). The insets in the upper left corner show the time evolution of the particles gyroradius. The color bars indicate the number of particles. The horizontal grey stripe  is bounded on the upper part by the jet diameter  ($4L$) and on lower part by the cell size of the simulated background jet. In these  particle simulations, the particle acceleration  time  is given in hours and the adopted physical size for $L$  is the same as in \citetalias{medinatorrejon_etal_2021} for comparison, $L = 3.5 \times 10^{- 5}$ pc.
}
\label{GACELL-PLUTO45}
\end{figure}




The figures described above  evidence the similarity of the results obtained with the two MHD codes and reinforce the results of \citetalias{medinatorrejon_etal_2021} and \citetalias{kadowaki_etal_2021}.


Figure \ref{pic1} shows the first stages of the kinetic energy evolution of the particles evolving together with the background  jet as obtained with the present 
%MHD-PIC
model (i.e., employing the MHD-PIC mode) both for the $\sigma \sim 1$ and $\sigma \sim 10$ jet.  In the very beginning, while the CDKI is still developing, particles only suffer drift in the background magnetic fields. Then, as the jet column  starts to wiggle  around $t \sim 20$ L/c in the $\sigma \sim 1$, and around $t \sim 7$ L/c in the  $\sigma \sim 10$ jet, due to the kink instability (Figure \ref{jet_points}),  the particles suffer curvature drift acceleration. Note that at these times, fast reconnection driven by turbulence is not developed yet (Figure \ref{vrec}). As stressed earlier, curvature drift  acceleration has been also detected in the $\sigma \sim 1$ jet by \citetalias{medinatorrejon_etal_2021}, for a similar resolution,  around similar jet dynamical time (more precisely, at $t\sim 25$  L/c, due to the time delay between the two runs; see their Figure 6), and by \citet{alves_etal_2018} in PIC simulations of the early stages  of the development of the kink instability. 

After $t \sim 30$ L/c in the $\sigma \sim 1$ jet  (and $t \sim 15$ L/c in the $\sigma \sim 10$ jet), which coincides with the nonlinear growth  and saturation  of the CDKI leading to fully developed turbulence in the jet (Figures \ref{energy} and \ref{spectrum}), the particles in Figure \ref{pic1} start exponential acceleration, 
%following the same trend of
as in Figure \ref{GACELL-PLUTO45}. The maximum achieved energy is about 10 times larger for the jet with  corresponding larger  $\sigma.$ 
%\sim 10$. 
%Note, however, that in both cases, the particles maximum energy and Larmor  radius (in the inset in the figure), are far below the saturation values  expected for  reconnection acceleration. 
The entire dynamical time of the system evolution is of only $60$ L/c for the   $\sigma \sim 1$ jet (and half this time for the $\sigma \sim 10$ jet). For the particles, the physical time elapsed is  only  $\sim 60 L/c \sim 1$ hr (and half-hour for the  $\sigma \sim 10$ jet, for the adopted $L=5.2 \times 10^{- 7}$  
pc in physical units), which is much smaller than the several hundred hours that particles can accelerate in the nearly steady state jet snapshot of Figure \ref{GACELL-PLUTO45}  where they can re-enter the system several times through the periodic boundaries of the jet in the z direction until they reach the saturation energy (see also \citetalias{medinatorrejon_etal_2021}).
%, where $L = ?$. 
This explains why particles do not achieve the maximum possible energy
by acceleration in the largest turbulent magnetic reconnection structures of the order of the jet diameter ($\sim  4L$), as we see in the inset in the figure, which depicts the particles Larmor radius distribution. 
For this value of the Larmor radius ($R_{max}\sim 4L$), the particles would achieve an energy $E_{sat} \sim e \, B \, R_{max} \sim 200,000$ $m_p c^2$ in the  $\sigma \sim 1$ jet, and $\sim 1,000,000$ $m_p c^2$  in the $\sigma \sim 10$ jet, if  the jet were allowed to evolve for a dynamical time about one hundred times larger (where $R_{max} \sim 4L = 2.1 \times 10^{-6}$ pc, and $B \sim 0.1$ G and $\sim 0.6 $ G for the $\sigma \sim 1$ and $\sigma \sim 10$ jets, respectively, considering the physical units employed in the MHD-PIC simulations).  %\sqrt{4 \pi \rho_0 c^2} \sim 0.13$ G is the physic units of the magnetic field, $R_{max}$ is the maximum gyroradius of the particles to be confined in the acceleration region, in this case it is the jet distorted diameter 4L.) }
Nonetheless, the results in these early stages of particle acceleration,  follow the same trend depicted in Figure \ref{GACELL-PLUTO45}, indicating that particles are accelerated exponentially by magnetic reconnection in the turbulent flow, from the small resistive scales up to the large scales of the turbulence in the ideal electric field of the magnetic reconnecting structures. These results also indicate that the time evolution of the background magnetic fields does not influence the acceleration of the particles since they enter the exponential regime of  acceleration in the same jet dynamical times in which turbulence becomes fully developed, as obtained  in the MHD simulations with test particles of Figure \ref{GACELL-PLUTO45}. At the more evolved dynamical times, particularly in the $\sigma \sim 10$  jet, we also identify particles having their perpendicular velocity component being accelerated suggesting the presence of drift acceleration too, as in the late stages of particle acceleration in  Figure \ref{GACELL-PLUTO45}.
%This  is also in agreement  with previous MHD studies (e.g. de Gouveia Dal Pino and Kowal 2015).  
%We further notice that in order for the particles to attain the saturation energy by magnetic reconnection they should reach a Larmor radius of the order of the jet diameter $\sim 4L$ which for the scales implied correspond to ? and ?, for the $\sigma \sim 1$ and $\sigma \sim 10$, respectively. 

\begin{figure} %
  \centering
  \begin{overpic}[scale=0.44]{PICMHD_s01_p50000emc020000c1v200.0_en_vpervpar.pdf}
    \put(14,52 ){\includegraphics[scale=0.19]{PICMHD_s01_p50000emc020000c1v200.0_gy.pdf}}
  \end{overpic} 
  \begin{overpic}[scale=0.44]{PICMHD_r256s10_p50000emc020000c1v200.0_en_vpervpar.pdf}
    \put(15,51){\includegraphics[scale=0.19]{PICMHD_s10_p50000emc020000c1v200.0_gy.pdf}}
  \end{overpic}  
\caption{Kinetic energy evolution for 50,000 particles evolved in the MHD-PIC simulation for the $\sigma \sim 1$ (top) and for the  $\sigma \sim 10$ (bottom) jet. Particles are initially injected with  energy $\left < E_p \right > \sim 1-200 m_p c^2$. %and a Maxwellian distribution. 
The colors indicate which velocity component of the particles is being accelerated (red or blue for the parallel or perpendicular component to the local magnetic field, respectively). The inset panels depict the evolution of the particles gyroradius, and the red horizontal lines correspond to the jet diameter ($4L$) (top)  and the cell size of the simulated jet (bottom).
}
\label{pic1}
\end{figure}




We have also run the MHD-PIC model for the $\sigma \sim 1$ and 10 jets with the larger resolution $426^2-256$,
%as in \citetalias{medinatorrejon_etal_2021}, 
and the results we obtained for particle acceleration evolution are very similar to those shown in Figure \ref{pic1}. 
%(top). 
The only difference is that less particles re-enter the system and thus the histogram has comparatively less accelerated particles. In particular, there are  almost no particles undergoing curvature drift in the very early times  (around $t\sim 20$ L/c), 
but the exponential  regime, with a dominance of  the acceleration of the parallel component of the velocity, is clearly detected, as in Figure \ref{pic1} (top)\footnote{The absence of accelerated particles by curvature drift in this case could be explained by the fact that this acceleration can be experienced only by particles with a Larmor radius large enough to $feel$ the curvature of the field (\citet{alves_etal_2018}, \citetalias{medinatorrejon_etal_2021}). When we increase the resolution  of the MHD domain (and thus decrease the cell size), particles with the same (still small) Larmor radius, at the same dynamical time step around $t\sim 20$ L/c  as in the lower resolution simulation (Figure \ref{pic1}), will see no field curvature when moving  from a smaller cell to the other and then, experience only linear drift, as in much earlier times.}.
%\footnote{\bete{The smaller number of accelerated particles in all the higher resoultion domain could be explained by the fact that, while increasing the resolution  of the MHD domain (and thus decreasing the cell size), particles with a given Larmor radius will feel magnetic fields of smaller cells which will be more uniform than in the smaller resolution MHD simulation and therefore, less particles will be captured in accelerating regions.}}
  

%Finally, 
In Figure \ref{picmhd-NE}  we show the  particle energy spectrum for the $\sigma \sim 1$ and $\sigma \sim 10$ jets, for different time steps in these early stages of the acceleration. 
The initial 
%Maxwellian 
distribution 
is represented by a red line.
As particles accelerate, they start to populate  the high energy tail in the distribution, which becomes flatter  as time evolves. In the $\sigma \sim 1$ jet, 
%the tail evolution is smoother. 
we note the formation of two slopes in more evolved times with a smooth transition between them which may be an indication of the two different regimes of acceleration specially coexisting at larger energies, the reconnection and later drift acceleration regimes we identified in Figure \ref{pic1}. Interestingly, the power-law tail of the flatter  part of the spectrum  for $t= 45$ L/c, when  the $\sigma \sim 1$ jet  develops a fully turbulent regime, is very   similar to   the slope obtained in the  snapshot $t=50$ L/c in \citetalias{medinatorrejon_etal_2021} which is in a similar dynamical state  of the background jet (see their Figure 11). 
For the $\sigma \sim 10$ jet, the transition is more abrupt and characterized by large  humps around 6000 and 10000 $E_p/m_p c^2$. Examining the particles energy evolution in Figure \ref{pic1}, these humps seem to concentrate a substantial number of particles with acceleration of the parallel component predominantly, 
but the two regimes of acceleration also seem to coexist in these large energies, as indicated by the presence of  particles also with  the perpendicular component dominating the acceleration.  Clearly, for this model the amount of particles accelerated in this short dynamical time is comparatively smaller.
%making the building of the particle spectrum  very irregular. 
Since the acceleration of the particles is still in very early stages and far from reaching the saturation energy by reconnection, the large energy tails of these spectra are clearly still under development.   
%Of course, in realistic systems, the presence of radiative losses and dynamical feedback of the accelerated particles into the plasma will result in steeper spectrum in these early  times too. 
%We should remember that in our numerical setup, particles are continuously re-injected into the system and therefore, they can continue to be accelerated. For this reason, the distribution shifts to larger  and larger energies. Furthermore, even after the particles attain the maximum (saturated)  energy at the end of the exponential acceleration regime due to reconnection (or magnetic curvature in the case of the snapshot $t= 25\,L/c$, top panel), they continue to accelerate at a smaller rate  due to normal drift (as remarked in Section \ref{sec:magrec}). 
% and  Figure \ref{t44o}). 
%This means that   in the absence of radiative losses or a escape from the acceleration region, the particles may gain energy continuously. 


%We should also remember that the maximum energy achievable by the stochastic  mechanism (at the saturation of the fast acceleration growth) occurs around $t \sim 10^3 $ hr for all the models depicted (see Figures \ref{t44o}, and \ref{t25} ), except in the top one, for which this occurs around $t \sim 10^{3.5} $ hr (see Figure \ref{t25}).
%Interestingly, we see that for this model ($ut25o$),  and also model $t30o$ (second panel from top), around these times, there is a double hump in the distribution (green dot-dashed curve), with an accumulation of particles at energies above $10^7\,m_p c^2$  for both, thus  highlighting the transition from the exponential to the linear drift acceleration regimes. In the other models depicted, this transition is more smooth.
%Also notable, is the double peak in the distribution that appears in model $t30o$ in the earlier times at $t\sim 10^2$ to  $10^{2.3}$ hr. This is possibly connected to the superposition of the two acceleration processes in this model, namely the magnetic curvature drift and the reconnection acceleration, as we discussed in section \ref{sec:t25}.

%In Figure \ref{NE50} we show the total number of particles as a function of energy for two different early  timesteps of the acceleration, for the jet snapshot $t=50$ $L/c$. The initial Maxwellian (normal) distribution is shown in gray dotted line. The earliest time step plotted corresponds to the approximate time when a high-energy power-law tail starts to form (i.e., when particles reach kinetic energies larger than $\sim 10^{-1}\,m_p c^2$, according to Figure \ref{t44o}, bottom panel); the second time corresponds to a little later time step. 
%In each time step two components can approximately be distinguished in the distributions: a normal (shown in solid gray line) and a power-law tail that fits the high energies. 
%We see that the  power-law index at the earlier time can be fitted by $p= -1.19$. The second power-law at later time is flatter due to the effects discussed above and therefore, it must be taken only as illustrative of the limitations of the method. 


\begin{figure} %picmhd-NE
\centering
    \includegraphics[scale=0.45]{PICMHD_s01_p50000emc020000c1v200.0_NE.pdf}
    \includegraphics[scale=0.45]{PICMHD_s10_p50000emc020000c1v200.0_NE.pdf}
\caption{Particle energy spectrum evolution as a function of the normalized kinetic energy  for the particles evolved in the MHD-PIC simulation for the $\sigma \sim 1$ (top)  and $\sigma \sim 10$  (bottom) jet. 
 The solid red line corresponds to the initial %Maxwellian 
 distribution. The high-energy tails in more evolved times of the system are fitted by  power laws.
% \bete{Tania, Por que 51 e nao 50?] Por que t=30 está abaixo de t=0?}
}
\label{picmhd-NE}
\end{figure}


\begin{figure} 
\centering
    \includegraphics[scale=0.7]{alpha3.pdf}
\caption{Power-law index $\alpha = \Delta (\log t)/\Delta (\log E_p)$ of the acceleration time as function of the particle kinetic energy normalized by the proton rest mass energy. The minimum in the curves, $\alpha \sim 0.1$, indicates the nearly exponential regime of particle acceleration. Depicted are the models with steady-state turbulent background of Figure \ref{GACELL-PLUTO45}, namely  the  $\sigma \sim 1$ jet at  t=50 L/c run by \citetalias{medinatorrejon_etal_2021}  
%(with the RAISHIN code) 
(black line) and the $\sigma \sim 1$ jet  at  t = 45 L/c run in this work 
%(with the Pluto Code) 
(blue line). Also shown is  $\alpha$  
for the nearly exponential regime  (between   $30 L/c<t<50 L/c$)
of the $\sigma \sim 1$ MHD-PIC model of the top of Figure \ref{pic1} where particles evolved with the background plasma (red curve).
%\citetalias{medinatorrejon_etal_2021} ($\sigma \sim 1$, t=50)  the  evolving with the system MHD-PIC ($\sigma \sim 1$, $30<t<50$) and in a steady-state snapshot in a given time in the evolving jet ($\sigma \sim 1$, t = 45) and in \citetalias{medinatorrejon_etal_2021} ($\sigma \sim 1$, t=50).} 
}\label{alpha}
\end{figure}

%\tania{In order to compare the acceleration time of the particles  evolving with the system MHD-PIC (Figure \ref{pic1}, top), and in the steady state snapshot (Figure \ref{GACELL-PLUTO45}, bottom) of the evolved $\sigma \sim 1$ jet with \citetalias{medinatorrejon_etal_2021}, we calculated in Figure \ref{alpha} the power-law index $\alpha = \Delta (\log t_{acc})/\Delta (\log E_p)$ of the acceleration time dependence with particle energy, $t_{acc} \propto E_p^ \alpha$. We quantified the nearly exponential growth, by evaluating the acceleration time directly from a diagram of particles’ kinetic energy vs. time (Figures \ref{GACELL-PLUTO45} and \ref{pic1}), in a similar way as we have done previously in \citetalias{medinatorrejon_etal_2021} and \citet*{delvalle_etal_16}. Specifically, we computed the slope of the logarithmic diagrams in Figure \ref{GACELL-PLUTO45}. We find that the slope $\alpha$ has essentially the same value during the nearly exponential regime of the acceleration of the particles in all models, i.e., $t_{acc} \propto E_p^{0.1}$ (see Figure \ref{alpha}).}


Finally, we can quantify and compare the particle acceleration, in particular, in the nearly exponential regime, by evaluating the acceleration time directly from the diagrams of particles kinetic energy versus  time, in a similar way as performed previously
in \citet*{delvalle_etal_16} and \citetalias{medinatorrejon_etal_2021}. Specifically, we  compute
the slope of the logarithmic diagrams in Figures \ref{GACELL-PLUTO45} and \ref{pic1} (top), $\alpha = \Delta (\log t)/\Delta (\log E_p)$, which gives the acceleration time dependence with particle energy, $t_{acc} \propto E_p^ \alpha$.  The result is shown in Figure \ref{alpha}. We find that the slope $\alpha$ has essentially the same minimum value in all models, which corresponds to the nearly exponential regime of the acceleration of the particles, i.e., $\alpha \sim 0.1$, implying an acceleration time $t_{acc} \propto E_p^{0.1}$, as found in \citetalias{medinatorrejon_etal_2021}, with very weak dependence on the energy, as expected in this regime. The increase in $\alpha$ (and thus  in the acceleration time) around $E_p/m_pc^2 \sim 10^3$ for the MHD-PIC model is due to the contribution of several  particles that are already experiencing drift and thus slower acceleration at this energy (see the blue points in Figure \ref{pic1}  that correspond to the perpendicular  momentum component, predominant in drift acceleration).




~\\
\section{Discussion and Conclusions}\label{sec:discut}

In this work, we have investigated the early stages of the acceleration of the particles in  3D Poynting flux dominated jets with  magnetization $\sigma \sim$  1 and 10,  subject to CDKI, using the MHD-PIC mode of the \texttt{PLUTO} code, in order to follow the evolution of the particles along with the flow. The CDKI drives turbulence and fast magnetic reconnection which we find to be the dominant mechanism of particle acceleration. 

Our results are very similar to those of \citetalias{medinatorrejon_etal_2021} which were carried out with test particles launched in the simulated MHD relativistic jet after it achieved a %nearly steady state 
regime of fully developed  turbulence.  Particles are accelerated by the ideal electric field ($V \times B$) of the background fluctuations, over the entire inertial range of the turbulence, starting in the small, resistive scales up to the large injection scales (Figure \ref{spectrum}). The connection of the accelerated particles with the magnetic reconnection layers is clear (Figure \ref{jet_points}). During this regime, the particles energy grow nearly exponentially and the parallel velocity component to the local magnetic field is the one that is preferentially accelerated, both expected in a Fermi-type process. In the test particle simulations of \citetalias{medinatorrejon_etal_2021} (see also Figure \ref{GACELL-PLUTO45}), particles re-enter the system several times through the periodic boundaries of the nearly steady state turbulent jet and are  accelerated in the reconnection sites up to the saturation energy that is achieved when their Larmor radius becomes of the order of the size of the acceleration region, or the jet diameter. This takes several hundred hours in the $\sigma \sim 1$ jet and the particles energy become as large $\sim 10^ 7$ $m_pc^2$. Beyond this energy, particles still experience further acceleration, but at smaller rate due to drift in the large scale non-reconnected fields.   In the MHD-PIC simulations, we can follow particle acceleration only during the dynamical time evolution of the MHD jet which  lasts $\sim 60$ L/c  and $\sim 35$ L/c for the $\sigma \sim 1$  and $\sigma \sim 10$ jet, respectively, and  corresponds to   only  $\sim 1 hr$ and half-hour, respectively, in physical units for the particles. During this time, the particles obviously do not reach the maximum possible (saturation) energy, 
%we see in Figure \ref{GACELL-PLUTO45}, 
but follow the same  exponential acceleration trend as in the test particle simulations
%with the \texttt{GACCEL} code 
(Figure \ref{pic1}). 

At later times, when turbulence is fully developed, the particle energy  spectrum develops a power law tail with  two slopes (better defined in  the $\sigma \sim 1$ jet), suggesting the presence of the two different regimes of acceleration, the reconnection and the drift regimes (Figure \ref{picmhd-NE}).
The slope of the power-law tail of the flatter  part of the spectrum  for $t= 45$ L/c in the $\sigma \sim 1$  is the same as obtained for particles accelerating in the  snapshot $t=50$ L/c in \citetalias{medinatorrejon_etal_2021}, which has a similar state  of the background jet (see their Figure 11). These slopes are also comparable to  previous studies of particle acceleration both in MHD flows \citep{kowal_etal_2012,delvalle_etal_16} and PIC simulations \citep[e.g.,][]{comisso18, werner_etal_2018}. However, we expect that in realistic systems, the presence of 
%physical particle escape from the acceleration zone, 
radiative losses and dynamical feedback of the accelerated particles into the plasma will lead to steepening of the spectra \citepalias[e.g.,][]{medinatorrejon_etal_2021}. 

Our results also indicate that the time evolution of the background magnetic field ($\partial  B/ \partial  t$) does not influence the acceleration of the particles. They enter the exponential regime of  acceleration in the same dynamical times of the jet in which turbulence becomes fully developed ($\sim 30$ L/c for the $\sigma \sim 1$  jet, and $\sim 15$ L/c for the $\sigma \sim 10$, respectively;  Figure \ref{pic1}), in agreement with the results of the MHD simulations with test particles injected in the nearly steady state turbulent jet in \citetalias{medinatorrejon_etal_2021} (see also Figure \ref{GACELL-PLUTO45}). The particles also undergo curvature drift acceleration in the initial stage of the CDKI when the jet column starts to wiggle in similar dynamical time both in the test particle $+$ MHD and in the MHD-PIC simulations. The background magnetic field time evolution effect, also known  as betatron acceleration, has been found to affect particle acceleration in pure turbulent flows only by a factor two in the acceleration rate \citep[e.g.,][]{dalpino_kowal_15}. Therefore, while it can be substantial in  very early times when particles are still undergoing linear drift acceleration, it is negligible in the more advanced times when exponential acceleration takes over. 

The increase of the jet magnetization by a factor 10, speeds up the growth of the CDKI which attains saturation in nearly half of the time (see Figure \ref{energy}) and  particles are accelerated to energies about 10 times larger, as also expected from PIC simulations \citep[e.g.][]{werner_etal_2018}. 


The  results  above indicate that particle acceleration by fast magnetic reconnection in a Fermi process can be dominant in magnetically dominated flows from the injection (large) to the resistive (small) scales of the turbulence. These results \citep[and those produced in earlier MHD works with test particles; e.g.][]{kowal_etal_2012,delvalle_etal_16,medinatorrejon_etal_2021} %(and those produced in earlier MHD works with test particles; e.g. Kowal et al. 2012; ...; del Valle et al. 2016; MGK+21)   
are in contrast with recent studies based on 3D PIC simulations that suggest that acceleration by reconnection would be  dominant only in the very early stages of particle energizing \citep[e.g.,][]{comisso19,sironi_etal_2021,sironi2022,comisso_sironi_2022}. %(e.g., Comisso and Sironi 2019, 2020; Sironi et al. 2021, Sironi 2022). 
%These results are in contrast to recent studies based on 3D PIC simulations
%(e.g., Comisso and Sironi 2019, 2020; Sironi et al. 2021) 
%that suggest that reconnection acceleration would be  dominant only in the very early stages of  particle acceleration. This apparent inconsistency is essentially due to the intrinsic difference in scales and indicates that direct extrapolation from the resistive small scales probed by PIC simulations (wherein non-ideal accelerating electric fields generally prevail), to the large MHD scales should be taken with caution.
%(see also Guo et al. 2022).
This apparent inconsistency is essentially due to the intrinsic difference in scales and in the accelerating electric fields that prevail in the two regimes. 
While in these PIC simulations, plasmoid-like reconnection acceleration occurs at the small kinetic, resistive scales and is dominated by the resistive electric field ($\eta J$, where $\eta$ is the resistivity  and $J$  the current density), in our collisional MHD turbulent flow simulations where  resistivity is naturally small (the ubiquitous Ohmic resistivity is mimicked by the numerical truncation error), the reconnection layers persist up to the large injection scales and particles are accelerated by the ideal electric fields (V$\times B$) of the fluctuations in these sites.  
Therefore, these intrinsic differences (inherent to scale and accelerating electric field),    indicate that direct extrapolation from
the resistive small scales probed by PIC simulations (wherein non-ideal accelerating electric fields
generally prevail), to the large MHD scales should be taken with caution \citep[see also][]{guo_etal_2019, guo_etal_2022}.
%, \bete{Guo et al. 2022)}.

The same applies to the recent study of \citet*{puzzoni_etal_2022} who examined the impact of resistive electric fields on particle acceleration in reconnection layers. The authors claimed  that their results are  in contradiction with  earlier MHD works \citep*{kowal_etal_2011,kowal_etal_2012,medinatorrejon_etal_2021}. However, they are clearly exploring a different regime of reconnection endowed with extremely high artificial resistivity, which is much larger than the Ohmic resistivity expected in most astrophysical  MHD flows and in particular, in turbulent ones. In other words, they are exploring the resisistive, kinetic  scales well below the inertial range of the turbulence that is explored in the works above and in the present one. %Their MHD flow is essentially in a laminar regime.
While in the present simulations and those of the previous works mentioned above, particles are predominantly accelerated by the ideal electric fields of the magnetic fluctuations in the reconnection layers, in \citep{puzzoni_etal_2022} simulations, the dominant component is the resistive electric field component 
%\bete{($\eta J$, where $\eta$ is the resistivity  and $J$  the current density)}
which prevails in the kinetic scales. Therefore, \textbf{there is no contradiction} with the MHD (non-resistive) works above.\footnote{One may still inquire how the results of the present study would change if we had included an explicit resistivity in the flow.  As remarked above, this would affect only the very small scales of the flow, of the order of a few grid cells size \citep[e.g.][]{santoslima_etal_2010}. In the integration of the particles equation of motion, we accounted only for the ideal electric fields of the magnetic fluctuations that persist in the entire range of the turbulence.  Still, the non-ideal term could be important for the small-scale topology of the velocity and magnetic fields, especially in the vicinity of the reconnection regions, indirectly affecting the particles' evolution before they reach a gyroradius of the order of a few cells size.
Therefore, if we had included an initial small explicit resistivity of the typical strength of Ohmic resistivity (as expected in astrophysical turbulent flows), the results for particle acceleration would be the same as in the present work.  On the other hand, if we had adopted an artificial much larger explicit resistivity, well above the Ohmic resistivity, this would kill all the turbulence in the range of scales smaller than this resistive scale and particle acceleration by turbulent reconnection would be possible only in a more limited inertial range of turbulent structures, from the injection scale down to the resistive scale.}


Future studies exploring in depth both regimes and scales, and also including particle feedback into the plasma are required. Our present study, combining PIC and MHD altogether in a relativistic jet with turbulence induced by an instability was a first attempt in this direction and the results in general confirm the predictions of previous MHD studies with test particles which show that  turbulent reconnection acceleration prevails in most of the scales of the system.  As stressed, e.g. in \citetalias{medinatorrejon_etal_2021}, the implications of these results for particle acceleration and the origin of VHE emission phenomena  in  Poynting flux dominated systems like the relativistic jets in microquasars, AGN and GRBs, is rather important.







%~\\
%--------------------------------------------------------------------------------------------

%% If you wish to include an acknowledgments section in your paper,
%% separate it off from the body of the text using the \acknowledgments
%% command.


\acknowledgments
 The authors acknowledge very useful discussions with L. Kadowaki.  TEMT and EMdGDP acknowledge support   from the Brazilian Funding Agency FAPESP (grant 13/10559-5),  EMdGDP also acknowledges support  from CNPq (grant 308643/2017-8),  and G.K. from  FAPESP (grants 2013/10559-5, 2019/03301-8, and 2021/06502-4).  The  simulations presented in this work were performed in the cluster of the Group of Plasmas and High-Energy Astrophysics (GAPAE), acquired with support from  FAPESP (grant 2013/10559-5), 
 %and Superintend\^{e}ncia de Tecnologia da Informa\c{c}\~{a}o da Universidade de S\~{a}o Paulo (USP), 
 and  the computing facilities of the Laboratory of Astroinformatics (IAG/USP, NAT/Unicsul), whose purchase was also made possible by FAPESP (grant 2009/54006-4) and the INCT-A. 
 %\textbf{The authors are also thankful to an anonymous referee whose comments have helped to improve the paper.}
%~\\
%% To help institutions obtain information on the effectiveness of their 
%% telescopes the AAS Journals has created a group of keywords for telescope 
%% facilities.
%
%% Following the acknowledgments section, use the following syntax and the
%% \facility{} or \facilities{} macros to list the keywords of facilities used 
%% in the research for the paper.  Each keyword is check against the master 
%% list during copy editing.  Individual instruments can be provided in 
%% parentheses, after the keyword, but they are not verified.

%\vspace{5mm}
%\facilities{HST(STIS), Swift(XRT and UVOT), AAVSO, CTIO:1.3m,
%CTIO:1.5m,CXO}

%% Similar to \facility{}, there is the optional \software command to allow 
%% authors a place to specify which programs were used during the creation of 
%% the manusscript. Authors should list each code and include either a
%% citation or url to the code inside ()s when available.

%\software{astropy \citep{2013A&A...558A..33A},  
%          Cloudy \citep{2013RMxAA..49..137F}, 
%          SExtractor \citep{1996A&AS..117..393B}
%          }

%% Appendix material should be preceded with a single \appendix command.
%% There should be a \section command for each appendix. Mark appendix
%% subsections with the same markup you use in the main body of the paper.

%% Each Appendix (indicated with \section) will be lettered A, B, C, etc.
%% The equation counter will reset when it encounters the \appendix
%% command and will number appendix equations (A1), (A2), etc. The
%% Figure and Table counter will not reset.

%\appendix

%\section{Appendix information}
%
%...
%
%\section{Author publication charges} \label{sec:pubcharge}
%
%...

%% The reference list follows the main body and any appendices.
%% Use LaTeX's thebibliography environment to mark up your reference list.
%% Note \begin{thebibliography} is followed by an empty set of
%% curly braces.  If you forget this, LaTeX will generate the error
%% "Perhaps a missing \item?".
%%
%% thebibliography produces citations in the text using \bibitem-\cite
%% cross-referencing. Each reference is preceded by a
%% \bibitem command that defines in curly braces the KEY that corresponds
%% to the KEY in the \cite commands (see the first section above).
%% Make sure that you provide a unique KEY for every \bibitem or else the
%% paper will not LaTeX. The square brackets should contain
%% the citation text that LaTeX will insert in
%% place of the \cite commands.

%% We have used macros to produce journal name abbreviations.
%% \aastex provides a number of these for the more frequently-cited journals.
%% See the Author Guide for a list of them.

%% Note that the style of the \bibitem labels (in []) is slightly
%% different from previous examples.  The natbib system solves a host
%% of citation expression problems, but it is necessary to clearly
%% delimit the year from the author name used in the citation.
%% See the natbib documentation for more details and options.


\bibliography{bibliography.bib}

%% This command is needed to show the entire author+affilation list when
%% the collaboration and author truncation commands are used.  It has to
%% go at the end of the manuscript.
%\allauthors

%% Include this line if you are using the \added, \replaced, \deleted
%% commands to see a summary list of all changes at the end of the article.
%\listofchanges

\end{document}

% End of file `sample62.tex'.



\bete{RASCUNHO:} 

The same applies to the recent study of Puzzoni, Mignone and Bodo (2022) that explored the  impact of resistive electric fields on particle acceleration in reconnection layers. The authors claimed  that their results are  in contradiction with  earlier works Kowal, dGDP, Lazarian 2011, 2012; Medina-Torrejon, dGDP, Kadowaki+ 2021, etc.). However, they are exploring a regime of reconnection endowed with extremily large artificial resistivity, which is much larger than the Ohmic resistivity expected in most astrophysical  MHD flows and in particular in turbulent ones. In other words, they are exploring the resisistive kinetic  scales well below the inertial scale of the turbulence that are explored in the works above. While in the present simulations and those of the previous works, particles are predominantly accelerated by the ideal electric fields of the magnetic fluctuations in the reconnection layers, in their simulations, the dominant component is the resistive electric field component which prevails in the kinetic scales. Therefore, there is no contradiction with the MHD (non-resisitive) works above. 

in the   On the contrary, they probe a different scales and a different regime of acceleration.
Their study deals with  strong  resistivity and thus, non-ideal resistive electric field is important for particle acceleration. However,  in the case of turbulent flows, this process can be relevant only at the resistive very  small scales, i.e., below the inertial range of a turbulent flow.  Therefore, they cannot be in contradiction with MGK21+ (see also Kowal et al. 2012, del Valle etal. 2016) or with the present results or the present ones. They are at most complementary.  of the acceleration process, while our study probes the macroscopic scales, i.e.,  particle acceleration by Fermi stochastic acceleration in the  reconnection sheets driven  in all scales of the turbulent MHD flow: from the small resistive scale (Ohmic scale) to the injection, large scale of the turbulence, which is of the order of the size of the system. Particles are accelerated by non-ideal resistive fields only if resistivity is large and this is important in turbulent flows  only in the small (resistive) scales. For the macroscopice MHD turbulent flows, it is not important.


Sironi:
You probe reconnection acceleration in the resistive, very small scales essentially. People doing resistive MHD are getting similar results to yours because with strong resistivity you naturally decrease the inertial scale of any turbulence in real systems and therefore, you focus on the injection, resistive scales, and thus on resistive reconnection dependent on the Landquist number! But, MHD turbulent flows, as you know, have low resistivity (ohmic),  and the diffusivity of the lines leads to fast (turbulence induced) reconnection (Lazarian-Vishniac theory; see e.g.,  Eiynk et al, Nature 2013; Kowal et al. 2009; Lazarian et al. 2012; 2020).

You see reconnection particle acceleration in plasmoids driven by tearing mode, but plasmoids are just the cross sections of 3D flux tubes and describe only the microscopic, very early stage of the process. In nature reconnection acceleration is 3D and occurs  up to the large macroscopic scales.

In the MHD scales, we see particle acceleration by Fermi stochastic acceleration in the  reconnection sheets driven  in all scales of the turbulent MHD flow: from the small resistive scale (Ohmic scale) to the injection large scale of the turbulence, of the order of the size of the system, and this size actually provides the Larmor scale of the maximum saturation energy of reconnection acceleration (Kowal et al. 2012; de Gouveia Dal Pino & Kowal 2015; Medina-Torrejon et al. 2021)! We see this directly in the simulations, particles are accelerated by the background (vxB) fluctuations in the reconnection layers in all scales (as in shock acceleration, Bell 2022).

See for instance, the turbulent power spectrum  of a kink unstable relativistic jet  that is subject to kink instability (Kadowaki, dGDP+, ApJ 2021). This is the same turbulent relativistic jet within which we injected test particles in Medina-Torrejon et al. ApJ 2021,  in order to probe the acceleration of the particles  by fast reconnection in relativistic Poynting  flux dominated jets. Particles are accelerated exponentially  from energies < mc^2  up to 10^ 20 eV in the case, and it is the parallel component that is predominantly accelerated in this regime; we also identify, secondary, slower rate acceleration by drift. There is no new physics  here. We have seen this for at least a decade (Kowal et al 2012 PRL).  Particles undergo Fermi process in the fast reconnection current sheets (de Gouveia Dal Pino & Lazarian 2005; Lazarian et al. 2012) driven by turbulence,  which have sizes spanning from the resistive, small scale, up to the large sizes of injection of the turbulence (of the order of  the diameter of the jet, in this example).

In summary, as we understand, the resistive term of the accelerating electric field is relevant only in the very early stages of the acceleration (as you and several other PIC studies have demonstrated), only in the resistive small scales of the flow (where you see tearing mode and plasmoids), (as we stressed in Kowal et al PRL 2012 with regard to non-ideal field).
Above this small scale,   3D MHD flows with turbulence, suffer fast reconnection (driven by turbulence), from the resistive to the injection scale of the turbulence (Lazarian & Vishniac theory). Therefore, the injected particles, as soon as they are captured in  the current sheets, starting from the small resistive scale (mimicked by numerical ohmic resistivity in our simulations) up to the injection scale of the turbulence, they suffer exponential Fermi acceleration in time (see Kowal et al. 2012, Figure 1 middle; and Medina-Torrejon et al. 2021, Figures 5, 8. 12, 13).

Also note how similar is the result that you get in Comisso & Sironi (2018, 2019) and what we get in Figure 1, bottom panel with pure turbulence (in Kowal et al. , PRL,2012); only that your scales are very much smaller,  and you interpret results  a little  distinctly. Maybe because turbulence in PIC is strongly resistive.  


Guo:
Determining the Dominant Acceleration Mechanism during Relativistic Magnetic Reconnection in Large-scale Systems Fan Guo1 , Xiaocan Li1, William Daughton1 , Patrick Kilian1 , Hui Li1 , Yi-Hsin Liu2, Wangcheng Yan3, and Dylan Ma1 1 Los Alamos National Laboratory, NM 87545, USA 2 Dartmouth College, Hanover, NH 03750, USA 3 The University of Tennessee, Knoxville, TN 37996, USA Received 2019 March 16; revised 2019 June 10; accepted 2019 June 14; published 2019 July 10 

Abstract While a growing body of research indicates that relativistic magnetic reconnection is a prodigious source of particle acceleration in high-energy astrophysical systems, the dominant acceleration mechanism remains controversial. Using a combination of fully kinetic simulations and theoretical analysis, we demonstrate that Fermi-type acceleration within the large-scale motional electric fields dominates over direct acceleration from non-ideal electric fields within small-scale diffusion regions. This result has profound implications for modeling particle acceleration in large-scale astrophysical problems, as it opens up the possibility of modeling the energetic spectra without resolving microscopic diffusion regions.


Sironi, Rowan & Narayan (ApJ 2021)
Reconnection driven acceleration in Relativistic Shear Flows
The trajectory of a representative high-energy electron is displayed in Figure 4.Thefirst stage of acceleration (ωpt ≈ 4750, vertical orange in (a)) is powered by · ˆ  = Eb E (compare solid black and red dashed lines). The dominance of the nonideal electric field EP in the process of particle acceleration is expected for reconnection-powered acceleration with a strong nonalternating component (Ball et al. 2019; Comisso & Sironi 2019), although other mechanisms may also play a role (Guo et al. 2019). During this injection stage, the electron is located within a reconnecting current sheet (b), where particle acceleration/heating occurs (c). At later times, while EP no longer contributes to acceleration, the electron energy still steadily grows—a similar two-stage acceleration process was reported for magnetically dominated plasma turbulence (Comisso & Sironi 2018, 2019).

Sironi et al. (2021), based on PIC simulations, identify reconnection acceleration by strong non-ideal (resistive) electric fields as important in the  initial stages of  particle acceleration (see also Sironi 2022)  mechamism at small scales (as also Comisso and Sironi 2019). In contrast, our MHD simulations with small ohmic resistivity (mimicked by numerical resistivity) evidence the importance of the ideal electric field acceleration by magnetic fluctuations in the turbulent flow starting in the resistive scale up to the large scales of the turbulence at injection scale.









Comisso-Sironi
By following a large sample of particles, we show that particle injection happens at reconnecting current sheets; the injected particles are then further accelerated by stochastic interactions with turbulent fluctuations. Our results have important implications for the origin of non-thermal particles in high-energy astrophysical sources.

Sironi (2022)
Magnetic reconnection in the relativistic regime [1–3], where the magnetic energy is larger than the particle rest-mass energy (equivalently, the mean magnetic energy per particle is ∼ σmc2 ≫ mc2, with σ the magnetization), has been invoked to explain the most dramatic flaring events in astrophysical high-energy sources [e.g. 4–11]. Our understanding of the physics of relativistic reconnection has greatly advanced thanks to fully-kinetic particle-in-cell (PIC) simulations, which have established reconnection as an efficient particle accelerator [e.g. 1216]. It is widely accepted that most of the energy gain of ultra-relativistic particles comes from ideal fields [e.g. 13, 17]. It was then argued that the spectrum of highenergy particles would remain unchanged, if non-ideal fields were to be ignored [17].


Non-resonant particle acceleration in strong turbulence: comparison to kinetic and MHD simulations
Virginia Bresci
Martin Lemoine
Laurent Gremillet
Luca Comisso and Lorenzo Sironi
Camilia Demidem

Collisionless, magnetized turbulence offers a promising framework for the generation of non- thermal high-energy particles in various astrophysical sites. Yet, the detailed mechanism that gov- erns particle acceleration has remained subject to debate. By means of 2D and 3D PIC, as well as 3D (incompressible) magnetohydrodynamic (MHD) simulations, we test here a recent model of non-resonant particle acceleration in strongly magnetized turbulence [1], which ascribes the ener- gization of particles to their continuous interaction with the random velocity flow of the turbulence, in the spirit of the original Fermi model. To do so, we compare, for a large number of particles that were tracked in the simulations, the predicted and the observed histories of particles momenta. The predicted history is that derived from the model, after extracting from the simulations, at each point along the particle trajectory, the three force terms that control acceleration: the acceleration of the field line velocity projected along the field line direction, its shear projected along the same direction, and its transverse compressive part. Overall, we find a clear correlation between the model predictions and the numerical experiments, indicating that this non-resonant model can successfully account for the bulk of particle energization through Fermi-type processes in strongly magnetized turbulence. We also observe that the parallel shear contribution tends to dominate the physics of energization in the PIC simulations, while in the MHD incompressible simulation, both the parallel shear and the transverse compressive term provide about equal contributions.


I'd like to emphasize  that I see no contradiction between the results you are getting and those we've got (for the injection energy scale).  PIC and MHD  probe different scales and hence, are complementary, and should be analysed as such.

You probe reconnection acceleration in the resistive, very small scales essentially. People doing resistive MHD are getting similar results to yours because with strong resistivity you naturally decrease the inertial scale of any turbulence in real systems and therefore, you focus on the injection, resistive scales, and thus on resistive reconnection dependent on the Landquist number! But, MHD turbulent flows, as you know, have low resistivity (ohmic),  and the diffusivity of the lines leads to fast (turbulence induced) reconnection (Lazarian-Vishniac theory; see e.g.,  Eiynk et al, Nature 2013; Kowal et al. 2009; Lazarian et al. 2012; 2020).

You see reconnection particle acceleration in plasmoids driven by tearing mode, but plasmoids are just the cross sections of 3D flux tubes and describe only the microscopic, very early stage of the process. In nature reconnection acceleration is 3D and occurs  up to the large macroscopic scales.

In the MHD scales, we see particle acceleration by Fermi stochastic acceleration in the  reconnection sheets driven  in all scales of the turbulent MHD flow: from the small resistive scale (Ohmic scale) to the injection large scale of the turbulence, of the order of the size of the system, and this size actually provides the Larmor scale of the maximum saturation energy of reconnection acceleration (Kowal et al. 2012; de Gouveia Dal Pino & Kowal 2015; Medina-Torrejon et al. 2021)! We see this directly in the simulations, particles are accelerated by the background (vxB) fluctuations in the reconnection layers in all scales (as in shock acceleration, Bell 2022).

See for instance, the turbulent power spectrum  of a kink unstable relativistic jet  that is subject to kink instability (Kadowaki, dGDP+, ApJ 2021). This is the same turbulent relativistic jet within which we injected test particles in Medina-Torrejon et al. ApJ 2021,  in order to probe the acceleration of the particles  by fast reconnection in relativistic Poynting  flux dominated jets. Particles are accelerated exponentially  from energies < mc^2  up to 10^ 20 eV in the case, and it is the parallel component that is predominantly accelerated in this regime; we also identify, secondary, slower rate acceleration by drift. There is no new physics  here. We have seen this for at least a decade (Kowal et al 2012 PRL).  Particles undergo Fermi process in the fast reconnection current sheets (de Gouveia Dal Pino & Lazarian 2005; Lazarian et al. 2012) driven by turbulence,  which have sizes spanning from the resistive, small scale, up to the large sizes of injection of the turbulence (of the order of  the diameter of the jet, in this example).

In summary, as we understand, the resistive term of the accelerating electric field is relevant only in the very early stages of the acceleration (as you and several other PIC studies have demonstrated), only in the resistive small scales of the flow (where you see tearing mode and plasmoids), (as we stressed in Kowal et al PRL 2012 with regard to non-ideal field).
Above this small scale,   3D MHD flows with turbulence, suffer fast reconnection (driven by turbulence), from the resistive to the injection scale of the turbulence (Lazarian & Vishniac theory). Therefore, the injected particles, as soon as they are captured in  the current sheets, starting from the small resistive scale (mimicked by numerical ohmic resistivity in our simulations) up to the injection scale of the turbulence, they suffer exponential Fermi acceleration in time (see Kowal et al. 2012, Figure 1 middle; and Medina-Torrejon et al. 2021, Figures 5, 8. 12, 13).

Also note how similar is the result that you get in Comisso & Sironi (2018, 2019) and what we get in Figure 1, bottom panel with pure turbulence (in Kowal et al. , PRL,2012); only that your scales are very much smaller,  and you interpret results  a little  distinctly. Maybe because turbulence in PIC is strongly resistive.  


Determining the Dominant Acceleration Mechanism during Relativistic Magnetic Reconnection in Large-scale Systems Fan Guo1 , Xiaocan Li1, William Daughton1 , Patrick Kilian1 , Hui Li1 , Yi-Hsin Liu2, Wangcheng Yan3, and Dylan Ma1 1 Los Alamos National Laboratory, NM 87545, USA 2 Dartmouth College, Hanover, NH 03750, USA 3 The University of Tennessee, Knoxville, TN 37996, USA Received 2019 March 16; revised 2019 June 10; accepted 2019 June 14; published 2019 July 10 

Abstract While a growing body of research indicates that relativistic magnetic reconnection is a prodigious source of particle acceleration in high-energy astrophysical systems, the dominant acceleration mechanism remains controversial. Using a combination of fully kinetic simulations and theoretical analysis, we demonstrate that Fermi-type acceleration within the large-scale motional electric fields dominates over direct acceleration from non-ideal electric fields within small-scale diffusion regions. This result has profound implications for modeling particle acceleration in large-scale astrophysical problems, as it opens up the possibility of modeling the energetic spectra without resolving microscopic diffusion regions.

---------

Reconnection-driven Particle Acceleration in Relativistic Shear Flows Lorenzo Sironi1 , Michael E. Rowan2 , and Ramesh Narayan3 1 Department of Astronomy and Columbia Astrophysics Laboratory, Columbia University, New York, NY 10027, USA; lsironi@astro.columbia.edu 2 Lawrence Berkeley National Laboratory, Berkeley, CA 94720, USA; mrowan@lbl.gov 3 Harvard-Smithsonian Center for Astrophysics, Cambridge, MA 02138, USA; rnarayan@cfa.harvard.edu Received 2020 December 17; revised 2021 January 6; accepted 2021 January 8; published 2021 February 3 Abstract Particle energization in shear flows is invoked to explain nonthermal emission from the boundaries of relativistic astrophysical jets. Yet the physics of particle injection, i.e., the mechanism that allows thermal particles to participate in shear-driven acceleration, remains unknown. With particle-in-cell simulations, we study the development of Kelvin–Helmholtz (KH) instabilities seeded by the velocity shear between a relativistic magnetically dominated electron–positron jet and a weakly magnetized electron–ion ambient plasma. We show that, in their nonlinear stages, KH vortices generate kinetic-scale reconnection layers, which efficiently energize the jet particles, thus providing a first-principles mechanism for particle injection into shear-driven acceleration. Our work lends support to spine-sheath models of jet emission—with a fast core/spine surrounded by a slower sheath —and can explain the origin of radio-emitting electrons at the boundaries of relativistic jets.


---------

The impact of resistive electric fields on particle acceleration in reconnection layers
Puzzoni, Mignoe, Bodo (2022)

I'd like to remark that your results are not in contradiction with ours (Kowal, dGDP, Lazarian 2011, 2012; Medina-Torrejon, dGDP, Kadowaki+ 2021, etc.) at all. On the contrary, they probe different scales and hence, are complementary,

Your study deals with  strong  resistivity and can probe only the resistive (kinetic)  small scales  of the acceleration process, while our study probes the macroscopic scales, i.e.,  particle acceleration by Fermi stochastic acceleration in the  reconnection sheets driven  in all scales of the turbulent MHD flow: from the small resistive scale (Ohmic scale) to the injection, large scale of the turbulence, which is of the order of the size of the system. Particles are accelerated by non-ideal resistive fields only if resistive is large and this is important in turbulent flows,  only in the small (resistive) scales. For the macroscopice MHD turbulent flows, it is not important.

See for instance, the turbulent power spectrum  of a kink unstable relativistic jet  that is subject to kink instability (Kadowaki, dGDP+, ApJ 2021). This is the same turbulent relativistic jet within which we injected test particles in Medina-Torrejon et al. ApJ 2021,  in order to probe the acceleration of the particles  by fast reconnection in relativistic Poynting  flux dominated jets,  from energies < mc^2  up to 10^ 20 eV in the case. There is no new physics  here. We have known this for at least a decade (Kowal et al 2012 PRL).  Particles undergo Fermi process in the fast reconnection current sheets (de Gouveia Dal Pino & Lazarian 2005) driven by turbulence,  which have sizes spanning from the resistive, small scale, up to the large sizes of injection of the turbulence (of the order of  the diameter of the jet, in this example). These results are compatible with fast reconnection theory in MHD flows driven by turbulence (Lazarian-Vishniac, see e.g.,  Eiynk et al, Nature 2013; Kowal et al. 2009; Lazarian et al. 2012; 2020).

In summary, as we understand, the resistive term of the accelerating electric field is relevant only in the very early stages of the acceleration (as you and several other PIC studies have demonstrated), only in the resistive small scales of the flow (where you see tearing mode and plasmoids), as we stressed in Kowal et al (PRL 2012).
Above this small scale,   3D MHD flows with turbulence, suffer fast reconnection (driven by turbulence), from the resistive to the injection scale of the turbulence (Lazarian & Vishniac theory). Therefore, the injected particles, as soon as they are captured in  the current sheets, starting from the small resistive scale (mimicked by numerical resistivity in our simulations) up to the injection scale of the turbulence, they suffer exponential Fermi acceleration in time (see Kowal et al. 2012, Figure 1 middle; and Medina-Torrejon et al. 2021, Figures 5, 8. 12, 13).

That is why we have been  claiming for the last decade or so, that PIC is unable to probe the macroscopic MHD scales of  reconnection particle acceleration in nature, only the microscopic scales (which coincide with the resistive scales of MHD flows!). This makes both studies complementary and essencial! Only now PIC people are doing 3D and even trying to reproduce turbulence (Comisso & Sironi) and they are getting similar results to ours (published a decade ago), only that their scales are very much smaller and they interpret results a little distinctly (without invoking Lazarian-Vishniac turbulent fast reconnection theory).  (Compare our Figure 1, bottom (Kowal et al. 2012)  with Figs. of (Comisso & Sironi 2019). They are very similar in spite of the difference in scales.

One note more: plasmoids, are only the 2D cross section of reconnecting flux tubes, but reconnection acceleration is 3D in nature.

--------

F. Guo, X. Li, W. Daughton, P. Kilian, H. Li, Y.-H. Liu, W. Yan, and D. Ma, The Astrophysical Journal 879, L23 (2019), 1901.08308.
Ideal accelerating fields can accelerate particles also at kinetic resistive scales (Sironi 2022 disagrees)


Sironi, Rowan & Narayan (ApJ 2021)
Reconnection driven acceleration in Relativistic Shear Flows
The trajectory of a representative high-energy electron is displayed in Figure 4.Thefirst stage of acceleration (ωpt ≈ 4750, vertical orange in (a)) is powered by · ˆ  = Eb E (compare solid black and red dashed lines). The dominance of the nonideal electric field EP in the process of particle acceleration is expected for reconnection-powered acceleration with a strong nonalternating component (Ball et al. 2019; Comisso & Sironi 2019), although other mechanisms may also play a role (Guo et al. 2019). During this injection stage, the electron is located within a reconnecting current sheet (b), where particle acceleration/heating occurs (c). At later times, while EP no longer contributes to acceleration, the electron energy still steadily grows—a similar two-stage acceleration process was reported for magnetically dominated plasma turbulence (Comisso & Sironi 2018, 2019).

Sironi et al. (2021), based on PIC simulations, identify reconnection acceleration by strong non-ideal (resistive) electric fields as important in the  initial stages of  particle acceleration (see also Sironi 2022)  mechamism at small scales (as also Comisso and Sironi 2019). In contrast, our MHD simulations with small ohmic resistivity (mimicked by numerical resistivity) evidence the importance of the ideal electric field acceleration by magnetic fluctuations in the turbulent flow starting in the resistive scale up to the large scales of the turbulence at injection scale.