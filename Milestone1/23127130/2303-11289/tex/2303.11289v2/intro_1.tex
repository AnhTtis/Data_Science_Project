\section{Introduction}
\label{sec: intro}%\todo{also in the abstract we should start from the more fundamental issue addressed}
While the derivation of nonlinear but uniformly parabolic equations from microscopic dynamics \cite{kipnis1989hydrodynamics}, fluctuations around these limits \cite{kipnis1990large,kipnis1998scaling,quastel1999large} and the corresponding canonical choice of a gradient flow structure \cite{dirr2016entropic,adams2011large,adams2013large} are now well-understood, much less\footnote{We refer to some recent works \cite{goncalves2009hydrodynamic,blondel2018convergence,goncalves2023exclusion}  and references therein; see the literature review in Section \ref{sec: Lit}.} is known for equations with either degenerate, or unbounded, diffusivity.  Indeed, for the model case of the \emph{porous medium equation} (PME)
\begin{equation}
	\label{eq: PME} \partial_t u_t=\frac12\Delta(u_t^\alpha),\qquad x\in \TTd, \qquad \alpha>1 
\end{equation} 
multiple gradient flow structures
have been known since the works of Br\'ezis and Otto \cite{brezis1971monotonicity,otto2001geometry}, but it is not known which, if any, are {\em thermodynamic}, in that they arise through the large deviations of a microscopic model. A necessary first step to a rigorous answer is to identify the dynamical large deviations of a suitable microscopic model whose hydrodynamic limit is \eqref{eq: PME}. 

The degeneracy and unboundedness of the diffusivity pose serious challenges on the probabilistic side. Indeed, the key property of rapid relaxation into local equilibrium, which
underlies the usual arguments for the so-called superexponential estimate, becomes problematic for two reasons: Firstly, the presence of degenerate diffusivity leads to degenerate mixing rates, for which the property of rapid local equilibration underlying the `one-block
estimate' may break down. {Secondly, the faster-than-linear growth of the local mobility may lead to the explosion of the total mobility\footnote{See the discussion on page 4 below for more details.} in finite time.} For these reasons the established methods for proving the superexponential replacement lemma \cite{kipnis1998scaling,benois1995large,kipnis1989hydrodynamics} do not apply. 
% ; let us remark that this does not appear to be a consequence of our model equation \eqref{eq: PME} or of the choice of particle system, but rather an intrinsic difficulty of any microscopic model whose scaling limit has the interesting analytic features we seek to capture.
A contribution of this work is to resolve these difficulties for a suitable model, leading to an LDP and to a thermodynamic gradient flow structure for \eqref{eq: PME}. 

In this work, in order to prove that the (formal) geometric picture we obtain is thermodynamic, we consider a rescaling of the zero-range process (ZRP) which converges to \eqref{eq: PME}. The first main contribution then is to resolve both the hydrodynamic limit and the large deviations for this rescaled ZRP and \eqref{eq: PME}, in the presence of both degenerate and unbounded diffusivity. To the best of the authors' knowledge, this represents the first time that large deviations around \eqref{eq: PME}, or any other equation with degenerate and unbounded diffusivity, have been identified for a particle system. A key technical tool developed in this work in order to overcome the difficulties described above, is a new approach to the superexponential estimate via {\em pathwise regularity estimates}.  


The second main contribution of this work is the rigorous identification of a gradient flow structure for \eqref{eq: PME} arising from the large deviation picture, which thus completes the programme described above. As a result of the degenerate and unbounded diffusivity, and in contrast to \cite{dirr2016entropic}, one cannot define a global geometric structure induced by the rescaled ZRP. Instead, we rigorously validate the generalisation to the case of degenerate  and unbounded  diffusivity of the connection \cite{dirr2016entropic,adams2011large,adams2013large,mielke2014relation} between dynamic large deviations and the {\em entropy-dissipation inequality} (EDI). Since in non-degenerate cases, the EDI is an equivalent formulation of the gradient flow \cite{ambrosio2005gradient,sandier2004gamma,serfaty2011gamma}, this expresses the PME \eqref{eq: PME} as the gradient flow of a thermodynamic entropy in a degenerate geometry induced by the ZRP. The geometric structure which we derive is thermodynamic, since it is derived from the ZRP, while being more universal, as it describes the fluctuations of any other particle system with the same dynamic large deviations. The proof of the identity of the large deviations rate function and the EDI without restriction on regular paths is novel even for the case of nondegenerate and bounded diffusivity.


%\dhedit{The second main contribution of this work is the rigorous identification of a gradient flow structure for \eqref{eq: PME} arising from microscopic dynamics. As is well-known, such dissipative equations often admit multiple gradient flow structures, and at least two have been known since the works of Br\'ezis and Otto \cite{brezis1971monotonicity, otto2001geometry}. For the particular choice \eqref{eq: PME}, it has been an open problem since these works to determine which gradient flow structures, if any, are \emph{thermodynamic} in the sense of arising as the free energy associated to a microscopic model. In recent years a number of works \cite{fathi2016gradient,dirr2016entropic,adams2011large,adams2013large,mielke2014relation} have uncovered a connection between between the dynamical large deviations of models with nondegenerate and bounded diffusivity, and the {\em entropy-dissipation inequality} (EDI), which is an equivalent characterisation of gradient flows in metric spaces \cite{ambrosio2005gradient}. \\ \\  In attempting to generalise from these examples to the model case of the PME, the reintroduction of degenerate and unbounded diffusivity creates significant difficulties for understanding a full geometric picture; attempting to define a local structure in the same way as \cite{otto2001geometry,dirr2016entropic}, the distance between two profiles may be infinite, as a result of degenerate diffusivity, or zero, as a result of unbounded diffusivity. As a result, characterisations of gradient flow depending on the global geometric picture, for example the convergence of the JKO-scheme \cite{jordan1998variational, adams2011large} are not available. In contrast, the EDI can still be defined, since it relies only on defining the lengths of curves rather than the distance between two points. Our approach therefore consists of rigorously validating the EDI via the connection to the rate function, extending the connection \cite{dirr2016entropic,adams2011large,adams2013large,mielke2014relation}, under the minimal possible assumptions on the curve, and in this way we resolve the problem of selecting a gradient flow corresponding to a microscopic model for the PME.} 

We first describe the rescaling of the ZRP with which we work. We start from the standard ZRP $\hat{\eta}^N(t, x)$, defined on the periodic lattice $\{0,1\dots,N-1\}^d$ with periodic boundary conditions and local jump rate $ g(k):= d k^\alpha$, see \cite{kipnis1989hydrodynamics,kipnis1990large}. In order to see macroscopic behaviour, as usual, we parabolically rescale time and space by $(N^{2}t, N^{}x)$. In addition, in order to make visible small masses, and, thus, degenerate diffusion, we rescale particle size and time by setting, for $x\in \TND=\{0, N^{-1},\dots 1-N^{-1}\}^d$ and $t\ge 0$, $\eta^N_t(x):=\chi_N\hat{\eta}^N(N^{2}\chi_N^{\alpha-1}t, Nx)$. In this way, $\chi_N$ plays the role of particle size, and at rate $$ d\eta^N_t(x)^\alpha N^2/\chi_N$$ the value at $x\in \TND$ is decreased by $\chi_N$, and the value at a neighbouring site is simultaneously increased by the same amount. 
 %\\  \dhedit{Version 2}  We first describe the rescaling of the ZRP with which we work. We start from the standard ZRP $\hat{\eta}^N(t, x)$, defined on the periodic lattice $\{0,1\dots,N-1\}^d$ with periodic boundary conditions and local jump rate $ g(k):= d k^\alpha$, see \cite{kipnis1989hydrodynamics,kipnis1990large}. Each site is occupied by an integer number of particles $\hat{\eta}^N(x)$, and an exponential clock of rate $g(\hat{\eta}^N(x))$ is instantiated at each site. When the first exponential clock rings, a particle is moved from the selected lattice site $x$ to a uniformly selected neighbour $y\sim x$, increasing $\hat{\eta}^N(y)\mapsto \hat{\eta}^N(y)+1$ and decreasing $\hat{\eta}^N(x)\mapsto \hat{\eta}^N(x)-1$. New exponential clocks are sampled at the modified sites, and the process repeats.  In contrast to the usual hydrodynamic limit, in order to introduce degenerate diffusion we must perform a double-rescaling in both space-time and value-time. We set, for $x\in \TND=\{0, N^{-1},\dots 1-N^{-1}\}^d$ and $t\ge 0$, $\eta^N_t(x):=\hat{\eta}^N(N^{-2}\chi_N^{\alpha-1}t, N^{-1}x)$. In this way, $\chi_N$ plays the role of particle size, and at rate $$ d\eta^N_t(x)^\alpha N^2/\chi_N$$ the value at $x\in \TND$ is decreased by $\chi_N$, and the value at a neighbouring site is simultaneously increased by the same amount. %Old, in case we want to revert. Let us briefly describe the ZRP under consideration\dhedit{, which is a further rescaling of the standard zero-range process \cite{kipnis1989hydrodynamics,kipnis1990large} with the superlinear and non-Lipschitz jump rate $g(k)=k^\alpha$ in both the time and the value.} Each site $x$ in the discrete lattice $\TND=\{0, N^{-1},\dots 1-N^{-1}\}^d$ is occupied by a discrete number of particles of size $\chi_N$, and we set $\eta^N_t(x)$ to be $\chi_N$ times the number of particles at $x$ at time $t$. At rate $$d\eta^N_t(x)^\alpha N^2/\chi_N$$  a particle jumps from $x$ and is reassigned to a neighbour $y$ chosen according to the nearest neighbours kernel $p^N(x,y)$ on $\TND$; at this jump, $\eta^N(x)$ is decreased by $\chi_N$ and $\eta^N(y)$ is increased by the same amount. 
 % \\ \dhedit{paralell versions end} \\
 We impose the scaling relation on the parameters \begin{equation}
	\label{eq: scaling hypothesis} N^2\chi^{\min(1,\alpha/2)}_N\text{ is bounded as }N\to \infty.
\end{equation}  We identify the particle configuration with a nonnegative integrable function $\eta^N \in L^1_{\ge 0}(\TTd)$ by assigning the value $\eta^N(x)$ at all points $z\in c^N_x$ in a the box of side length $N^{-1}$ centred at $x\in \TND$. We will always work on a fixed, finite time interval $[0,\tf]$. With these conventions fixed, we can now describe the main results, which correspond to the strategy described below \eqref{eq: PME}.\paragraph{\textbf{Hydrodynamic Limit.}} The first step is to justify the implicit claim that the hydrodynamic limits of $\eta^N_\bullet$ are indeed governed by \eqref{eq: PME}. The following theorem resolves this under the condition only that $u_0\in L^1_{\ge 0}(\TTd)$ has finite entropy, and mild conditions on the initial data $\eta^N_0$.

%\bigskip \\ A key role in the arguments will be played by the \emph{Boltzmann entropy} $\cH_\rho$, defined for $u, \rho\in L^1_{\ge 0}(\TTd)$  by \begin{equation} \cH_\rho(u)=\int_{\TTd} \left(\Psi\left(\frac{u(x)}{\rho(x)}\right)-\frac{u(x)}{\rho(x)}+1\right) \rho(x) dx \in [0,\infty]\end{equation} where $\Psi$ is the function $\Psi(x)=x\log x$. We allow the value $\infty$ if the set $\{u>0, \rho=0\}$ has positive Lebesgue measure or if the integral fails to converge; if $\rho$ is omitted, we take it to be the constant function $\rho\equiv 1$.

\begin{theorem}[Hydrodynamic Limit]\label{th: hydrodynamic limit} Assume (\ref{eq: scaling hypothesis}). Fix $u_0\in L^1_{\ge 0}(\TTd)$ with finite entropy $\cH(u_0)<\infty$, and let $u_\bullet=(u_t)_{t\ge 0}$ be the unique weak solution to the PME (\ref{eq: PME}) starting at $u_0$. Let $\PP$ be a probability measure, let $\eta^N_\bullet$ be the ZRP with initial data satisfying \begin{equation}
	\label{eq: entropy UI hyp} \limsup_{M\to \infty} \limsup_{N\to \infty} \PP\left(\cH(\eta^N_0)>M\right)=0;
\end{equation} and, for all $\e>0$,
\begin{equation}
	\label{eq: weak convergence hyp} \langle \varphi, \eta^N_0\rangle \to \langle \varphi, u_0\rangle \text{ in probability, for all }\varphi \in C(\TTd).
\end{equation}  Then \begin{equation}
	\eta^N_\bullet \to u_\bullet
\end{equation} in probability in the topology of the Skorokhod space $\mathbb{D}:=\mathbb{D}\left([0,\tf],(L^1_{\ge 0}(\TTd))_{\rm w}\right)$, where $L^1_{\ge 0}(\TTd)$ is equipped with the weak topology, and in the topology of $L^\alpha([0,\tf]\times\TTd)$.	
\end{theorem} This theorem treats much more general initial data than other models considered in the literature; the works of Ekehaus and Sepp\"al\"ainen \cite{ekhaus1995stochastic} and Feng, Iscoe and Sepp\"al\"ainen \cite{feng1997microscopic} require a more regular initial datum $u_0\in C^{2+\e}(\TTd), \e>0$, and the work of Gon\c{c}alves, Landim and Toninelli \cite{goncalves2009hydrodynamic} require that $\inf_{x\in \TTd}u_0(x)>0$, so that the initial data remains in the region where (\ref{eq: PME}) is nondegenerate parabolic. 

\paragraph{\textbf{Large Deviations.}} Having found the hydrodynamic limit, we consider the dynamical large deviations of the ZRP $\eta^N_\bullet$ on a time interval $[0,\tf]$. This completes the probabilistic elements of the strategy described below \eqref{eq: PME}, in that we have identified the large deviations of a particle system around this limit. If  the initial data $\eta^N_0$ are sampled from an equilibrium distribution or a slowly-varying local equilibrium $\Pi^N_\rho$, see (\ref{eq: eq}), then the large deviations rate functional associated to large deviations of the initial data is the rescaled relative entropy $\alpha \cH_\rho(u_0)$ relative to $\rho$.  The other component is a dynamic cost $\mathcal{J}$ through the {\em skeleton equation} which extends the definitions of Benois, Kipnis and Landim \cite[p.70]{benois1995large}, \cite[Equation (5.1)]{kipnis1998scaling}. The {skeleton equation} around (\ref{eq: PME})  is\begin{equation}\label{eq: Sk}
				\partial_t u_t = \frac12\Delta(u_t^\alpha)-\nabla\cdot(u^{\alpha/2}_t g_t), \qquad g\in {L^2_{t,x}}:= L^2([0, \tf]\times\TTd, \RRd)
			\end{equation} where, since we often work with the space $L^2([0, \tf]\times\TTd, \RRd)$, we introduce the notation ${L^2_{t,x}}$ and $\|\cdot\|_{L^2_{t,x}}$ for the associated norm.  The dynamic cost of a trajectory $u_\bullet=(u_t)_{0\le t\le \tf}$ is then given by \begin{equation}\label{eq: dynamic cost}
		\mathcal{J}(u_\bullet):=\frac{1}{2}\inf\left\{\|g\|^2_{L^2_{t,x}}: \h u_\bullet \text{ solves the skeleton equation (\ref{eq: Sk}) for }g\right\}
	\end{equation} where, as for (\ref{eq: PME}) above, we use the notion of weak solutions in \cite{fehrman2019large}, recalled in Definition \ref{def: solutions}. The total rate function $\cI_\rho(u_\bullet)$ for a path $u_\bullet$ relative to an initial profile $\rho\in C(\TTd, (0,\infty))$ is thus given by \begin{equation}\label{eq: rate function}
		\cI_\rho(u_\bullet):=\alpha\cH_\rho(u_0)+\mathcal{J}(u_\bullet).
	\end{equation} With this notation, the result is as follows. \begin{theorem}[Large Deviations Principle]\label{thrm: LDP} Let $\rho\in C(\TTd, (0,\infty))$, let $\PP$ be a probability measure, and for all $N$ let $\eta^N_\bullet$ be the rescaled ZRP with initial data sampled according to $\Law_{\PP}(\eta^N_0)=\Pi^N_\rho$. Then, with speed $\frac{N^d}{\chi_N}$, the processes $\eta^N_\bullet$ are exponentially tight and satisfy a large deviation principle in $\mathbb{D}$ with rate function $\mathcal{I}_\rho$; that is, for all $\cDD$-closed sets $\mathcal{E}$, \begin{equation}
	\label{eq: UB statement} \limsup_N \frac{\chi_N}{N^d}\log \PP\left(\eta^N_\bullet \in \mathcal{E}\right)\le -\inf\left\{\cI_\rho(u_\bullet): u_\bullet \in \mathcal{E}\right\}
\end{equation} and all $\cDD$-open sets $\cU$, \begin{equation}
	\label{eq: LB statement} \liminf_N \frac{\chi_N}{N^d}\log \PP\left(\eta^N_\bullet \in \cU\right)\ge -\inf\left\{\cI_\rho(u_\bullet): u_\bullet \in \cU\right\}.
\end{equation} Moreover, the function $\mathcal{I}_\rho$ is lower-semicontinuous with respect to $\mathbb{D}$ and has compact sublevel sets. \end{theorem} We emphasise that we obtain a \emph{full} large deviations principle in which the infimum in the lower bound runs over the whole set $\cU$, rather than being restricted to a class of `good' paths $\mathcal{Q}$ as in the works \cite{quastel1999large,landim1995large,landim1997hydrodynamic}. This issue will be discussed in detail in the literature review. \\\paragraph{\textbf{A regularity-based approach to the replacement lemma}} We now turn to the main technical tool which will be needed in the derivation of Theorems \ref{th: hydrodynamic limit} - \ref{thrm: LDP}. As is common in the hydrodynamic limits and large deviations of lattice gasses, we will need to take a limit of terms similar to $\int_0^t\langle (\Delta \varphi), (\eta^N_s)^\alpha\rangle ds$ which are nonlinear in the particle configuration $\eta^N_s$, and where convergence of $\eta^N_\bullet$ in the topology of $\mathbb{D}$ is not sufficient; this is usually called a \emph{replacement lemma} or \emph{superexponential estimate} (e.g. \cite[Theorem 3.1]{kipnis1998scaling}, Theorem 2 of Benois, Kipnis and Landim \cite{benois1995large}, or Theorem 2 in Kipnis, Olla and Varadhan \cite{kipnis1989hydrodynamics}). 

As already remarked, the usual approach to the superexponential estimate in the previously cited works cannot be applied for the rescaled ZRP, and the difficulties correspond exactly to the features of degenerate and unbounded diffusivity that we aim to study. Firstly, the presence of degenerate diffusivity leads to a vanishing minimum rate; if one attempts to carry out the usual argument leading to the one-block estimate, the control provided by the Dirichlet form degenerates as $N\to \infty$, and the requisite compactness is lost. {At the same time, the superlinear growth of the local jump rate $(\eta^N(x))^\alpha$ means that the functionals $\langle \Delta \varphi, (\eta^N_t)^\alpha\rangle$ could diverge in the limit $N\to \infty$ for a nontrivial set of times $t$. Since the local jump rate $(\eta^N(x))^\alpha$ corresponds to the mobility $u^\alpha$ in the limit, this is the possibility of the explosion of the total mobility described above \eqref{eq: PME}. In order to establish the superexponential estimate, we must establish a sufficiently strong integrability estimate to show that such divergences occur at most for a small set of times, and in fact the contributions to the error from any small space-time region are small, with superexponentially high probability. }
\\ The proof developed in this work is inspired by techniques in the theory of stochastic partial differential equations \cite{dirr2020conservative,fehrman2019large,fehrman2021well} and the Aubin-Lions-Simon lemma \cite{aubin1963theoreme,lions1969quelques,simon1986compact}. For the case of a lattice-based model, as we deal with here, the situation is complicated by the fact that the regularity is measured by different functionals $\mathcal{F}_{\alpha,N}$ for each $N$, whose $\Gamma$-convergence in the limit is not immediate. This issue further cannot be remedied by a convenient choice of $\mathcal{F}_{\alpha,N}$, since these are dictated by the fundamental regularity and cannot be freely chosen.\\ \\ Before stating the theorem, let us first introduce the entropy dissipation which is central to the argument.  It is a classical computation that, along solutions to (\ref{eq: PME}) $\partial_t \cH(u_t)=-\cD_\alpha(u_t) \le 0$ for the entropy dissipation given by \begin{equation}\label{eq: entropy dissipation}
			\cD_\alpha(u)=\frac2\alpha \int_{\TTd} |\nabla (u^{\alpha/2})|^2(x) dx. \end{equation} This estimate is also central to the theory of (\ref{eq: Sk}) in \cite{fehrman2019large}, where for some $c=c(d,\alpha)$ \begin{equation}\label{eq: basic entropy estimate} \begin{split} & \sup_{t\le \tf}\cH(u_t) + \int_0^\tf \cD_\alpha(u_t) dt  \le  \cH(u_0) + c\|g\|^2_{L^2_{t,x}}.\end{split} \end{equation}  In particular, solutions to (\ref{eq: Sk}) for any choice of $g$ always take values in the space $\cR \subset L^\alpha([0,\tf]\times \TTd)$ of paths which are continuous in $d$ with respect to time and where $\int_0^\tf \cD_\alpha(u_s)ds<\infty$, see (\ref{eq: energy class}). We will use a similar path-by-path estimate, with lattice discretisations $\cD_{\alpha,N}$ of (\ref{eq: entropy dissipation}), to exclude paths outside $\cR$ from the large deviations, and to obtain the convergence in the norm of $L^\alpha([0,\tf]\times\TTd)$. The first property will be a step towards obtaining a true large deviation principle, as advertised below Theorem \ref{thrm: LDP}, and the latter property plays the same role as the superexponential estimate in taking limits of the nonlinear terms.   \begin{theorem}[Large Deviations Weak-to-Strong Principle]\label{thrm: WtS}
	Suppose the scaling relation (\ref{eq: scaling hypothesis}) holds, and let $\mathbb{P}$ be a probability measure under which $\eta^N_\bullet$ is the rescaled ZRP with initial data satisfying  \begin{equation} \label{eq: entropy finite hyp}
	\limsup_N \frac{\chi_N}{N^d}\log\EE\left[\exp\left(\frac{N^d}{\chi_N}\gamma \cH(\eta^N_0)\right)\right]<\infty
\end{equation} for all $\gamma<\alpha$. Then the following hold. \begin{enumerate}[label=\roman*).]
		\item (Open Sets Version) For any $u_\bullet \in \mathbb{D}\cap \cR \subset L^\alpha([0,\tf]\times\TTd)$, any open neighbourhood $\cV$ of $u_\bullet$ in the (norm) topology of $L^\alpha([0,\tf]\times\TTd)$ and any $z<\infty$, there exists an open neighbourhood $\cU \ni u_\bullet$ in the topology of $\mathbb{D}$ such that \begin{equation}\label{eq: WtS good case}
		\limsup_N \frac{\chi_N}{N^d}\log \PP\left(\eta^N_\bullet \in \cU\setminus \cV\right) \le -z.
	\end{equation} If instead $u_\bullet \not \in \cR$, then for all $z<\infty$, there exists a $\cDD$-open set $\cU\ni u_\bullet$ such that \begin{equation} \label{eq: QY2}
		\limsup_N \frac{\chi_N}{N^d}\log \PP\left(\eta^N_\bullet \in \cU\right) \le -z.
	\end{equation} \item (Almost-Sure-Convergence Version) For $\PP$ as above, let $\mathbb{Q}$ be a probability measure with the bound \begin{equation}\label{eq: entropy bound hypothesis}
		\limsup_N \frac{\chi_N}{N^d}H\left(\Law_{\QQ}[\eta^N_\bullet]\right|\left.\Law_{\PP}[\eta^N_\bullet]\right)<\infty
	\end{equation} where $H(\cdot|\cdot)$ denotes the relative entropy of probability measures on $\mathbb{D}$. Suppose further that, $\QQ$-almost surely, on a subsequence $N_k\to \infty$, $\eta^{N_k}_\bullet$ converges to a random variable $\eta_\bullet$ in the topology of $\mathbb{D}$. Then there exists a further subsequence $N'_k$ such that $\eta^{N'_k}_\bullet \to \eta_\bullet$ almost surely in the topology of $L^\alpha([0,\tf]\times\TTd)$. 
	\end{enumerate}
 \end{theorem}
 
\paragraph{\textbf{From Large Deviations to Gradient Flows}} {%It is by now well-known that the dynamical large deviations of reversible systems are intimately related to gradient flows \cite{adams2013large,mielke2014relation} and in particular that the dynamical rate function $\mathcal{J}$ admits a geometric reformulation in terms of the EDI , which, in metric spaces, is an equivalent formulation of gradient flows \cite{ambrosio2005gradient,sandier2004gamma,serfaty2011gamma}. 
Having achieved the probabilistic parts of the strategy, we turn to the derivation of a gradient flow structure, hence resolving the problem described below \eqref{eq: PME}. The final theorem extends the correspondence between the dynamical large deviation cost and the EDI \cite{fathi2016gradient,dirr2016entropic,adams2011large} corresponding to a suitable formal Riemannian structure for \eqref{eq: PME}; see also the literature review below for further discussion.} {In addition to extending this identity to the case of degenerate and unbounded diffusion, we also prove, for the first time, that the two functionals appearing in Theorem \ref{thrm: gradient flow} below are equal whenever either is finite. Notably, this is a novel result even in the setting of non-degenerate and bounded diffusion.} 
%\todo{This bit I changed a little. The use of LDP$\leftrightarrow$EDI was introduced by Fathi, Adams etc. appear to be the first to do it with IPS. Strictly speaking the identity we prove isn't new, so we emphasise doing something for the first time without the risk of offending anyone.  } 
%\todo{B: Meaning the degenerate/unbounded setting this was proven for a resticted set of fluctuations before? At least this is how i understand the current formulation.}
%\todo{D: Our novelties are:
%-Move to degenerate / unbounded setting with the same identity
%-Prove the identity for all paths, not restricted set - previously the arguments are all formal (with no real proofs) and attempting to make it a rigorous proof relies on the path being nice enough. I have never even seen it formulated as a full identity without hypotheses even for the heat equation (although of course the proof for that case can be simplified a lot).}


We consider the space of absolutely continuous measures $\mathcal{M}_{ac, \lambda}(\mathbb{T}^d)$ on $\mathbb{T}^d$ with a prescribed total mass $\lambda>0$, and define the \emph{action} of a curve $u_\bullet \in \cDD$ to be \begin{equation}\label{eq: action 2}
		\cA(u_\bullet)=\frac12\inf\left\{\left\|\theta\right\|_{L^2_{t,x}}^2\right\}. 
	\end{equation} In this definition, the infimum runs over all $\theta \in {L^2_{t,x}}$ such that $u_\bullet$ solves the continuity equation \begin{equation} \label{eq: CE}
		\partial_t u_t+\nabla\cdot(\frac{1}{2}u^{\alpha/2}_t\theta_t)=0
	\end{equation} setting $\mathcal{A}(u_\bullet)=\infty$ if no such $\theta$ exists. With this defined, the result is as follows.
 \begin{theorem}\label{thrm: gradient flow}
	Let $u_\bullet \in \mathbb{D}$ with $\cH(u_0)<\infty$ and fix a constant $\rho>0$. Then we have the identity
	\begin{equation}\label{eq: conclusion of gf}
		\mathcal{J}(u_\bullet)=\frac12\left(\alpha\cH_\rho(u_\tf)-\alpha\cH_\rho(u_0)+\frac\alpha2\int_0^\tf \cD_\alpha(u_s)ds +\frac12 \cA(u_\bullet)\right)
	\end{equation} allowing both sides to be infinite. In particular, the functional on the right-hand side is nonnegative, and vanishes if and only if $u_\bullet$ is a solution to (\ref{eq: PME}). Moreover, if $u_\bullet$ is such that $\mathcal{J}(u_\bullet)<\infty$, then for almost all $0\le t\le \tf$,  $u_t^{\alpha/2}\in H^1(\TTd)$ and \begin{equation} \label{eq: tangent space useful def} \nabla u_t^{\alpha/2}\in {T}_{u_t}\mathcal{M}_{\text{ac}, \lambda}(\TTd):=\overline{\left\{u_t^{\alpha/2}\nabla \varphi: \varphi\in C^2(\TTd)\right\}}^{L^2(\TTd)}.\end{equation} We can also choose $g$ such that the skeleton equation (\ref{eq: Sk}) holds, attaining the minimum (\ref{eq: dynamic cost}), and such that $g_t\in T_{u_t}\cM_{\text{ac},\lambda}(\TTd)$ for almost all $t\le \tf$. Finally, in the special case where $u_\bullet$ is a solution to the PME (\ref{eq: PME}) and $\rho>0$, it holds for all $0\le t\le \tf$ that \begin{equation}
		\label{eq: entropy dissipation equality}\cH_\rho(u_t) +\int_0^t \cD_\alpha(u_s)ds=\cH_\rho(u_0).
	\end{equation}\end{theorem} Although the statement (\ref{eq: conclusion of gf}) involves only objects defined at the level of the limiting trajectories, the proof will be probabilistic and exploit Theorem \ref{thrm: LDP}.  
	\subsection{Plan of the Paper} The paper is structured as follows. Section \ref{sec: Lit} discusses related works in the literature and contextualises the results of the paper. Section \ref{sec: prelim} collects some preliminaries: We give formal definitions of all objects used in the paper and give an overview of the tools which we will use, but which are not novel in the present work, including summarising some results of \cite{fehrman2019large}  regarding (\ref{eq: Sk}) in Section \ref{sec: skg}. Since all results in this section are of very little novelty, proofs are ommited in favour of references to similar proofs in the literature. Section \ref{sec: equilibrium} gives some estimates and computations for static large deviations in slowly-varying local equilibria{, which differ from the standard large deviations in equilibrium due to the additional rescaling of particle size by $\chi_N$.} \medskip \\ The main technical novelties of the paper, culminating in Theorem \ref{thrm: WtS}, are in Sections \ref{sec: DTC} - \ref{sec: WTS}. The key ingredients are separated into the functional-analytic aspects in Section \ref{sec: DTC} and the key large deviations estimate in Section \ref{sec: a priori}; the proof of Theorem \ref{thrm: WtS} is assembled in Section \ref{sec: WTS}. \medskip \\ 
 \ifx\jrnl\undefined
undefed
\else
  \if\jrnl1
With these ingredients at hand, Theorems \ref{th: hydrodynamic limit} - \ref{thrm: LDP} follow: the proofs will closely follow those of the works \cite{kipnis1989hydrodynamics,kipnis1998scaling}, with the difference that we use Theorem \ref{thrm: WtS} to play the role of a superexponential estimate.  Theorem \ref{th: hydrodynamic limit} is proven in Section \ref{sec: hydrodynamic}. The proof of the large deviations upper bound (\ref{eq: UB statement}) is given in Section \ref{sec: UB}. The proof of the lower bound is very similar to the standard argument, although a priori complicated by the double-rescaling and the presence of unbounded diffusivity; in Section \ref{sec: LB} we give the details of these technical points, using Theorem \ref{thrm: WtS} for technical support. Finally, Theorem \ref{thrm: gradient flow} is deduced in Section \ref{sec: gradient flow}. 
  \else
With these ingredients at hand, Theorems \ref{th: hydrodynamic limit} - \ref{thrm: LDP} follow: the proofs will closely follow those of the works \cite{kipnis1989hydrodynamics,kipnis1998scaling}, with the difference that we use Theorem \ref{thrm: WtS} to play the role of a superexponential estimate.  Theorem \ref{th: hydrodynamic limit} is proven in Section \ref{sec: hydrodynamic}. The proof of the large deviations upper bound (\ref{eq: UB statement}) is given in Section \ref{sec: UB}, and the lower bound is proven in Section \ref{sec: LB}. Finally, Theorem \ref{thrm: gradient flow} is deduced in Section \ref{sec: gradient flow}. 
  \fi
\fi 
	
	\section{Discussion \& Literature Review}\label{sec: Lit}
\paragraph{\textbf{1. PME from Particle Systems}}  Suzuki and Ushiyama \cite{suzuki1993hydrodynamic} introduce a model, called `stick-breaking' in later works, where each site $x\in \TND$ is assigned  a `stick-length' in $[0,\infty)$, and show the convergence in probability to the PME with homogeneity $\alpha=2$. The same convergence was also proven under slightly different assumptions in \cite{ekhaus1995stochastic}, who also introduced a version of the model where the stick lengths take discrete values, and the model was generalised to relax the restriction to $\alpha=2$ in \cite{feng1997microscopic}. More recently, the work \cite{goncalves2009hydrodynamic} and Blondel, Canc{\`e}s, Sasada and Simon\footnote{At the time of writing, this preprint contains a disclaimer that the work is incomplete.} \cite{blondel2018convergence} considered the hydrodynamic limit of an exclusion-type particle system with kinetic constraints, which can be constructed to lead to (\ref{eq: PME}) for any choice of the exponent $\alpha\in \mathbb{N}$. In the special case $\alpha=1$, this model is the simple exclusion process, where the limit was proven in \cite{kipnis1989hydrodynamics}. The restriction to integer $\alpha$ for similar processes was lifted by \cite{goncalves2023exclusion}, who derive both fast and slow diffusion $\partial_t u= \Delta u^\alpha, \alpha\in (0,2]$ from an exclusion-type particle system. Let us also mention the works by Seo \cite{seo2017large} and Dembo, Shkolnikov, Varadhan and Zeitouni \cite{dembo2016large}, who proved matching upper and lower large deviation bounds for interacting particle systems driven by Brownian motions. The work \cite{seo2017large} proves a large deviation principle around a system of nonlinear, nondegenerate parabolic PDEs for the interaction of `colours' of particles, and \cite{dembo2016large} characterises the large deviations from a non-degenerate PME in dimension $d=1$.   \bigskip \\ The works \cite{ekhaus1995stochastic,feng1997microscopic} are based on an entropy method first introduced by Varadhan \cite{varadhan1991scaling}, while the works \cite{goncalves2009hydrodynamic,blondel2018convergence} on the exclusion-type model use a relative entropy method due to Yau \cite{yau1991relative}. All of these works deal exclusively with the hydrodynamic limit equivalent to Theorem \ref{th: hydrodynamic limit}, and the method cannot reach large deviations. Indeed, as discussed above Theorem \ref{thrm: WtS}, the key step is to replace the nonlinearity $(\eta^N(x))^\alpha$ by $(\overline{\eta}^{N,r}(x))^\alpha$, which is the same nonlinearity applied to the spatial averages $\overline{\eta}^{N,r}$ on a macroscopic spatial scale $r$, with only a small error. In the case of large deviations, the error probability must be superexponentially small, whereas the methods in the cited papers can at most show a $L^1(\PP)$-convergence of these errors, and are unable to say anything at the large deviation level. 
%To the best of the authors' knowledge, Theorem \ref{thrm: LDP} represents the first time that large deviations around (\ref{eq: PME}) have been identified for a particle system. 
\bigskip \\  The hypotheses (\ref{eq: entropy UI hyp} - \ref{eq: weak convergence hyp}) of Theorem \ref{th: hydrodynamic limit} allow significant freedom in choosing the initial data. For example, given $u_0\in C(\TTd,[0,\infty))$, we will see in Section \ref{sec: equilibrium} that both conditions hold with $\Law_{\PP}[\eta^N_0]=\Pi^N_{u_0}$ given by the slowly-varying local equilibrium (\ref{eq: eq}), which is a typical construction for hydrodynamic limits. On the other hand, we can also take $\eta^N_0$ to be deterministic, which is usually out of the reach of (relative) entropy methods.  We are also able to treat much more general initial data than previous works: the work \cite{goncalves2009hydrodynamic} required that the initial profile $u_0$ be bounded away from $0$ and $1$; in \cite{ekhaus1995stochastic,feng1997microscopic} some more regularity of the initial data $u_0\in C^{2+\e}(\TTd), \e>0$ is required. In contrast, we can treat (for example) deterministic $\eta^N_0$ converging to an irregular $u_0$ with support only on a subset of $\TTd$, provided only that $\cH(u_0)<\infty$.  Similarly to the works \cite{ekhaus1995stochastic,feng1997microscopic}, we will use some integrability estimates, i.e. estimates on $\|\eta^N_t\|_{L^\beta(\TND)}$-norms with $\beta>\alpha$, in a tightness argument. In the present context, we need estimates at the large-deviations level, which are rather more subtle than the estimates in expectation.\bigskip \\ Finally, we note that the large deviations are much more sensitive to the fine details of the particle model than the hydrodynamic limit, and different ideas would be needed to treat the large deviations of the other models mentioned above. For example, the equilibrium distribution of the `stick-breaking' model investigated by  %Eckhaus and Sepp\"al\"ainen \cite{ekhaus1995stochastic} and Feng, Eckhaus and Sepp\"al\"ainen \cite{feng1997microscopic}
\cite{ekhaus1995stochastic,feng1997microscopic} leads to large deviations with the same speed, but where the finite-rate large deviations may be measures singular with respect to the Lebesgue measure. Correspondingly, it appears to be much harder to establish a large deviations principle for this model. \bigskip \\  \paragraph{\textbf{2. Relation to the Zero-Range Process}   The limits of the zero-range process for general jump rates, assuming bounded diffusivity, as well as fluctuation theorems about these limits, have been widely studied. The hydrodynamic limit, see \cite{kipnis1998scaling}, is a nonlinear parabolic equation $\partial_t u=\Delta \Phi(u)$, with globally bounded and locally nondegenerate diffusivity $\sup_\rho \Phi'(\rho)<\infty, \inf_{\rho\le M}\Phi'(\rho)>0$. Let us cite the works by Menegaki \cite{menegaki2021quantitative} and Menegaki and Mouhot \cite{menegaki2022consistence} for a more recent hydrodynamic limit with an explicit rate of convergence. Equilibrium large deviations for a variant of the zero-range process have been studied by Bernardin, Gon{\c{c}}alves, Jim\'enez-Oviedo and Scotta in \cite{bernardin2022non}. In infinite volume, which we do not consider in the present work, the large deviations and hydrodynamic limit have been studied by Landim and Yau \cite{landim1995large} and Landim and Mourragui \cite{landim1997hydrodynamic}. Quastel, Rezakhanlou and Varadhan \cite{quastel1999large} found the analagous rate function for the simple symmetric exclusion process.  \bigskip \\As already mentioned below Theorem \ref{thrm: LDP}, an important difference to previous works is that we obtain a full large deviation principle, with matching upper and lower bounds and the infimum of (\ref{eq: LB statement}) running over the whole open set $\cU$, rather than being restricted to $\cU \cap \cQ$ for a class of `good' paths $\cQ$. This is achieved through two steps, namely the exclusion of paths outside $\cR$ by Theorem \ref{thrm: WtS}i), and the characterisation of $\cI_\rho|_\cR$ as the lower semicontinuous envelope of a restriction $\cI_\rho|_\cX$(Proposition \ref{prop: lsc envelope}), which we recall from \cite{fehrman2019large}. This property is known for relatively few of the models whose large deviations have been studied. In the case \cite{kipnis1989hydrodynamics}, the rate function is globally convex and such approximations can be obtained by convolution, and in works on reaction-diffusion systems \cite{jona-lasino1993large,bodineau2012,landim2018large,farfan2019}, the rate is a pertubation of a convex functional by a lower-order term. In \cite{quastel1999large,bertini2009non}, the required property is proven for exclusion-type processes, using {\em a priori} $L^\infty_{t,x}$ bounds and the boundedness and nondegeneracy of the diffusion.} In other works, for example Quastel and Yau \cite{quastel1998lattice}, the rate function is defined as a lower semicontinuous envelope, so that this property holds by definition, but lack a corresponding explicit characterisation. A counterexample in a different setting, where the na\"ive rate function does not coincide with the lower semicontinuous envelope of its restriction, has been found by the second author \cite{heydecker2021large}. \bigskip \paragraph{\textbf{3. Macroscopic Fluctuation Theory and Large Deviations of Conservative SPDE}} The same rate function $\mathcal{I}_\rho$ studied in this work appears naturally  as the informal large deviation rate function for the stochastic partial differential equation (SPDE) with conservative noise \begin{equation}
	\label{eq: SPDE} \partial_t u^\e_t = \frac12 \Delta \left((u^\epsilon_t)^\alpha\right)-\sqrt{\e} \nabla\cdot\left((u^\e_t)^{\alpha/2}\xi^K\right)
\end{equation} where $\xi^K$ is the convolution of a space-time white noise $\xi$ with a spatial mollifier on a scale $K^{-1}$, see the works of Dirr, Stamatakis and Zimmer \cite{dirr2016entropic} and Giacomin, Lebowitz and Presutti \cite{giacomin1999deterministic}. The well-posedness of SPDEs with a conservative noise term, including (\ref{eq: SPDE}) as a special case, was established by  \cite{fehrman2021well} as soon as $K<\infty$, so the noise has nontrivial spatial correlation. In general, the regularisation of the noise is necessary, since the case $\alpha=1$ is the Dean-Kawasaki equation, corresponding to non-interacting particles \cite{dean1996langevin,donev2014reversible,donev2014dynamic}, which was shown by Konarovskyi, Lehmann and von Renesse \cite{konarovskyi2019dean} to admit only trivial solutions if $K=\infty$. Large deviations for singular SPDEs have been studied in the works \cite{cerrai2011approximation,faris1982large,jona-lasinio1990large,hairer2015large}; in general, renormalisation constants may enter the rate function, as shown by Hairer and Weber \cite{hairer2015large}. In the work \cite{fehrman2019large}, a large deviation principle \cite[Theorem 6.8]{fehrman2019large} with rate function $\mathcal{J}$ for fixed initial data is proven under a scaling relation between $\e$ and $K$ which prevents renormalisation constants from appearing in the rate function. The cited theorem proves the large deviation principle for a much broader class of SPDEs, replacing $u^\alpha, u^{\alpha/2}$ by $\Phi(u), \Phi^{1/2}(u)$ for a class of $\Phi$ which includes the porous-medium nonlinearity considered here. Dirr, Fehrman and the first author \cite{dirr2020conservative} proved the analagous large deviation principle, now with $u^\alpha$ replaced by $\Phi(u)=u(1-u)$, for the symmetric exclusion process. This current work therefore extends the ideas of \cite{dirr2020conservative,fehrman2019large} to a particle setting, and Theorems \ref{th: hydrodynamic limit}-\ref{thrm: LDP}, together with \cite[Theorem 6.8]{fehrman2019large}, show that the SPDE (\ref{eq: SPDE}) has the same small-noise limit and large deviations as the ZRP we consider. We may therefore consider (\ref{eq: SPDE}) as a phenomenological model for the ZRP.  \bigskip \\ The large deviation rate functional is also closely related to macroscopic fluctuation theory (MFT) \cite{bertini2009non,bertini2015macroscopic,derrida2007non}, which postulates an Ansatz for the large-deviations rate function, for example in Bertini \cite[Equation 1.3]{bertini2015macroscopic}. We also refer to \cite{hohenberg1977theory,spohn2012large} for a rigorous justification. In this context, the time-reversal argument in Section \ref{sec: gradient flow} is due to Onsager \cite{onsager1931reciprocali,onsager1931reciprocalii}. \bigskip \\  \paragraph{\textbf{4. The scaling hypothesis (\ref{eq: scaling hypothesis}).}} We conclude with a discussion of the scaling hypothesis (\ref{eq: scaling hypothesis}). This hypothesis will play a key role in this paper, as it allows the path-by-path estimates on $\int_0^\tf \cD_{\alpha,N}(\eta^N_s)ds$ for a discrete entropy dissipation $\cD_{\alpha,N}$ defined in (\ref{eq: DNA}).  \bigskip \\ We observe that we could also take the limits of shrinking particle size $\chi\to 0$ and growing lattice $N\to \infty$ with decoupled parameters, and in either order, to obtain meaningful limits, which we summarise in Figure \ref{fig: all limits} below. The six relevant limits are described below: This paper is concerned with the curved arrow (e), and we leave the problems of making other limits rigorous to future work. \begin{figure}[htb]
\centering
\resizebox{12cm}{!}{ \begin{tikzpicture}[->,>=stealth',shorten >=1pt,node distance=2.5cm,auto,main node/.style={rectangle,rounded corners,draw,align=center}]

\node[main node] (1) {ZRP $\eta^N_t$};
\node[main node] (2) [below of=1] {Discretised PME on $\TND$};
\node[main node] (3) [right of=1, xshift = 6cm] {$\partial_t u^\chi_t = \Delta \Phi_\chi(u^\chi_t)$};
\node[main node] (4) [below of=3] {PME (\ref{eq: PME})};

\path
(1) edge node [swap] {(a): $\chi\to 0$, $N$ fixed} (2)

(2) edge node [swap] {(b): $\chi=0$, $N\to \infty$} (4)
(1) edge node [above] {(c): $\chi$ fixed, $N\to \infty$} (3)
(3) edge node [right] {(d): $\chi\to 0$, $N=\infty$} (4)

(1) edge [bend left=10] node  {(f)} (4)
(1) edge [bend right=13] node[swap]  {(e)} (4);

\end{tikzpicture}}  \caption{All possible double-limits. The horizontal arrows represent $N\to \infty$ with the hydrodynamic rescaling of space and time, and the vertical arrows represent $\chi\to 0$ with the value-time rescaling. The two curved arrows are the two double limits, depending on whether $\chi_N\to 0$ sufficiently quickly (e) or sufficiently slowly (f) with $N$. } \label{fig: all limits}\end{figure} 
 \begin{enumerate}[label=(\alph*)]
	\item {\bf $\chi\to 0$, $N$ fixed} gives a discrete version of the PME (\ref{eq: PME}). Fluctuations are described by the Freidlin-Wentzell theory in a $N$-dependent space. \item {\bf Subsequently taking $ N\to \infty$}  should recover the continuum PME (\ref{eq: PME}) from the discretised version, and $\Gamma$-convergence of the rate functionals. \item {\bf $\chi$ fixed, $N\to \infty$} is the usuals scaling limit of the zero-range process, in the spirit of \cite{kipnis1998scaling}. In this case, we expect to find the superexponential estimate through the same mecahnism as \cite{kipnis1989hydrodynamics} (rapid equilibration on macroscopic boxes), in contrast to the current work.  \item  {\bf Subsequently taking $\chi\to 0$} will recover the nonlinearity $\Phi(u)=u^\alpha$ from the nonlinearities $\Phi_\chi(u)$.  As in (b), a natural approach would be to seek the  $\Gamma$-convergence of the associated rate functions.  \item  {\bf Double-limit $N\to 0, \chi_N \to 0$ sufficiently fast}  is the content of the current paper, with the scaling relation imposed by (\ref{eq: scaling hypothesis}). In this case we find the superexponential estimate via regularity estimates, in contrast to (c). \item {\bf Double-limit $N\to 0, \chi_N \to 0$ sufficiently slowly} is the opposite type of relation from the ideas of this paper. As in (c) above, different ideas will be needed for the superexponential estimate. 

\end{enumerate} 


\paragraph{\textbf{5. PME as a gradient flow}} {As discussed in the introduction, the problem of identifying a thermodynamic gradient flow structure for \eqref{eq: PME} has stood open since the works of Br\'ezis and Otto \cite{brezis1971monotonicity,otto2001geometry}. Theorem \ref{thrm: gradient flow} is one resolution of this problem by extending, to the case of degenerate and unbounded diffusivity, the connection between large deviations and gradient flow \cite{adams2011large,mielke2014relation,fathi2016gradient}, and more specifically the connection via the EDI \cite{adams2013large,dirr2016entropic}.\\
We first start by recalling the state of the art in the nondegenerate and boundedly diffusive setting. In \cite{dirr2016entropic}, for nondegenerate and sublinear nonlinearity $\Phi(u)$ replacing $u^\alpha$ in \eqref{eq: PME}, it is shown how the identity analogous to \eqref{eq: conclusion of gf} arises from large deviations of the ZRP with Lipschitz jump rates, and how this identity corresponds to the EDI. The corresponding gradient flow is in a Riemannian structure determined by the underlying ZRP, which is defined by modifying the Otto calculus \cite{otto2001geometry} by replacing the weight $u$ by $\Phi(u)$ in the Riemannian tensor. This Riemannian structure leads to a global metric which may be defined in the same way as the Benamou-Brenier formulation for the Wasserstein distance: One defines the action $\mathcal{A}_\Phi$ of a curve in the same way as \eqref{eq: action 2}, with $\Phi(u)$ in place of $u^\alpha$, and defines the distance between two profiles $u_0, u_1$ as the infimal action $\mathcal{A}_\Phi(u_\bullet)$ of all curves starting at $u_0$ and ending at $u_1$. The modified Otto calculus also permits to identify the action as $\int_0^\tf g_{u_t}(\dot{u}_t, \dot{u}_t)dt$, for a Riemannian metric $g_u$ for which the tangent space may be identified in the same way as (\ref{eq: tangent space useful def}). For these cases, where a global metric structure may be defined, the identity corresponding to \eqref{eq: conclusion of gf} is the entropy dissipation identity, which gives an equivalent formulation of the gradient flow property \cite{ambrosio2005gradient,sandier2004gamma,serfaty2011gamma, fathi2016gradient}. In addition, in these cases further connections between the large deviations rate function and gradient flow structure are possible, for instance the convergence of a JKO scheme \cite{jordan1998variational, adams2011large,mielke2014relation}.} \\\\ {Let us now contrast this {nondegenerate and boundedly diffusive case} with the {case of degenerate and unbounded diffusivity of interest here.}   Replacing the nonlinearity $\Phi(u)$ in the arguments of Dirr et al. \cite{dirr2016entropic} by the porous medium nonlinearity $u^\alpha, \alpha>1$, introduces both degenerate diffusivity at $u=0$, and unbounded diffusivity as $u\to \infty$. As a result, the formal Riemann structure no longer gives a well-defined global geometry: The distance between two profiles may become zero, as a result of unbounded diffusivity, or infinite, as a result of degeneracy.  \ifx\jrnl\undefined
\else
  \if\jrnl1
  \else
{For completeness, the formal Riemannian metric and gradient flow structure are recalled in Appendix \ref{sec: formal GF}.}
  \fi
\fi As a result, characterisations of the gradient flow involving the metric, such as the JKO scheme, are no longer meaningful. \\ The approach in Theorem \ref{thrm: gradient flow} therefore consists of rigorously validating the EDI via the connection to the rate function, extending the connection \cite{dirr2016entropic,adams2011large,adams2013large,mielke2014relation}, and in this way we resolve the problem of selecting a gradient flow corresponding to a microscopic model for the PME. \\ 
The second part of Theorem \ref{thrm: gradient flow} shows that the gradient of the entropy in the formal Otto calculus is indeed an element of the tangent space according to (\ref{eq: tangent space useful def}). The final part of the theorem makes the formal computation above (\ref{eq: entropy dissipation}) precise for solutions to the PME (\ref{eq: PME}) no weaker than the solution concept of \cite{fehrman2019large}.\\
Notably, Theorem \ref{thrm: gradient flow} even extends what is known in the setting of nondegenerate and bounded diffusivity, since the formal calculations of \cite{adams2011large,dirr2016entropic} are implicitly restricted to fluctuations where each $u_t$ is regular enough to satisfy a `chain rule for the entropy', see the display above \cite[Equation (2.3)]{dirr2016entropic} and Erbar \cite[Proposition 4.1]{erbar2016gradient} for a similar calculation in a different context. %Informally, it may be that certain profiles $u$ lack sufficient regularity to take $\frac{\delta \cH}{\delta u}=\log u$as a test function against the evolution \eqref{eq: Sk} to derive an evolution for $\partial_t \cH(u_t)$. While one might hope that this only occurs for a small set of exceptional times along a trajectory $u_\bullet=(u_t)_{0\le t\le T}$, this is not immediate if one only assumes the finiteness of all objects in \eqref{eq: conclusion of gf}, and it is further possible that even a small set of times still contributes to the time evolution of the entropy $\cH(u_t)$. 
The argument in Section \ref{sec: gradient flow} circumvents these difficulties by using the large deviations principle and the machinery of the skeleton equation \eqref{eq: Sk} from \cite{fehrman2019large}}, thereby removing any additional assumptions on the path $u$ aside from the finiteness of the objects in \eqref{eq: conclusion of gf}.  \\\\   Gradient flow structures for the PME (\ref{eq: PME}) are well-known: Br\'ezis \cite{brezis1971monotonicity} formulated it as the gradient flow of the functional $\cE_\alpha(u)=\frac{1}{\alpha+1}\int u^{\alpha+1}$ with respect to a flat $H^{-1}$-metric, and Otto \cite{otto2001geometry} showed that is the gradient flow of $\cE'_\alpha(u)=\frac1{\alpha-1}\int u^\alpha$ with respect to a geometry inducing the Wasserstein$_2$ distance $\cW_2$. As already remarked, the microscopic model studied here is not the only model which leads to the PME (\ref{eq: PME}) as a macroscopic limit, and other choices of the underlying particle system may lead to different large deviation rate functions, and hence other gradient flow formulations. In particular, the energy functional $\alpha \cH$ and the geometry defined by $\cA$ should be thought of as intrinsic {\em to the class of particle systems whose large deviations are given by Theorem \ref{thrm: LDP}}, and not to the PME (\ref{eq: PME}). 
