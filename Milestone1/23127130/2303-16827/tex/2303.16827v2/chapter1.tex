\chapter{CFT entanglement}	\label{chapterCFT}

\abstract*{Entanglement entropy in a conformal field theory can be defined as the von Neumann entropy of the reduced density matrix. We review in this chapter the replica trick derivation of CFT entanglement, following the work of Cardy and Calabrese in section \ref{sectionreplicaCFT} and of Holzhey, Larsen and Wilczek in section \ref{sectionThermal}. This follows an introductory section \ref{sectintro} covering the pictorial notation of wave-functionals and density matrices in the path integral formalism, which is also heavily used in section \ref{sectTFD} on the thermofield double construction. }  


Entanglement entropy in a conformal field theory (CFT) can be defined as the von Neumann entropy of the reduced density matrix. We review in this chapter the replica trick derivation of CFT entanglement, following the work of Cardy and Calabrese \cite{Calabrese:2004eu,Calabrese:2009qy,Cardy:2008jc} in section \ref{sectionreplicaCFT} and of Holzhey, Larsen and Wilczek \cite{Holzhey:1994we} in section \ref{sectionThermal}. This follows an introductory section \ref{sectintro} covering the pictorial notation of wave-functionals and density matrices in the path integral formalism, which is also heavily used in section \ref{sectTFD} on the thermofield double construction. 


\section{Wave-functionals and density matrices} \label{sectintro}

 
We start with reviewing\footnote{For more details, see Appendix A of \cite{Polchinski:1998rq} and section 4 of \cite{Hartmanlectures}.} some basic pictorial notation for wave-functionals and density matrices that will be used throughout. 

In a quantum mechanical system with Hamiltonian $H$ and Euclidean action $I$, the transition amplitude from a position eigenstate at Euclidean time $t_E = -T$ to another position eigenstate at $t_E = 0$ can be written as a Euclidean path integral    
\ali{
	\langle q_f| e^{-H \, T}|q_i \rangle = \int_{q_i, -T}^{q_f,0} \mathcal D q e^{-I}.  
}
Inserting a complete set of energy eigenstates on the left hand side and taking the limit  
$T \ra \infty$ picks out the vacuum state contribution $|\psi\rangle$,   
such that the wavefunction $\psi(q_f)$ is given by path integral evolution from past Euclidean infinity (at a fixed and unimportant initial position $q_i$)    
\ali{
	\psi(q_f) = \int_{q_i, t_E = -\infty}^{q_f,t_E = 0} \mathcal Dq \, e^{-I} , 
}
where $q_f$-independent factors have been dropped. The ket $|\psi\rangle$ is the function  
\ali{
	|\psi\rangle = \int_{q_i, t_E = -\infty}^{\cdot,t_E = 0} \mathcal Dq \, e^{-I}   
}
that takes a position $q_f$ (at the dot $\cdot$) and gives back a complex number $\psi(q_f) = \langle q_f|\psi\rangle$.  
On slices of the path integral one recovers the Hilbert space of the theory. 

Similarly, in a quantum field theory with field content $\phi(\vec x,t)$, the (unnormalized) vacuum state is prepared by Euclidean evolution 
\ali{ 
	|\psi \rangle = \int_{\phi(t_E = -\infty) = \phi_i}^{\phi(t_E = 0) = \,  \cdot} \mathcal D\phi \, e^{-I} \quad = \quad  \adjincludegraphics[width=2cm,valign=c]{figs/figpsi.png}  \llap{
		\parbox[][-1cm][b]{2.1cm}{$t_E$}} \llap{
		\parbox[][1.5cm][b]{0.5cm}{$x$}},	\label{psiket}
}
where we have already for concreteness restricted the pictorial representation to field theories with one spatial direction $x$.  
The state can then further be evolved in Lorentzian time $t$, with the wave-functional   
\ali{
	\psi(\phi_f) = \int_{\phi(t_i) = \phi_i}^{\phi(t_f) = \phi_f} \mathcal D \phi \, e^{i \mathcal I}  = \int_{\phi(t_i) = \phi_i}^{\phi(t_f) = \phi_f} \mathcal D \phi \, e^{i \int_{t_i}^{t_f} L[\phi]}  
}
shown in \cite{Larsen:1994yt} to satisfy the Schr\"odinger evolution equation. Again, the ket for the vacuum state \eqref{psiket} is a functional  with open boundary condition (at the dot $\cdot$) where it can take on a given field configuration $\phi_f$ and give back 
\ali{
	\langle \phi_f | \psi \rangle = \int_{\phi(t_E = -\infty) = \phi_i}^{\phi(t_E = 0) = \phi_f} \mathcal D\phi \, e^{-I} = \adjincludegraphics[width=1.5cm,valign=c]{figs/figpsib.png}\llap{
		\parbox[][-1.1cm][b]{0.9cm}{$\phi_f$}}  ,  \label{psiphif}
}
obtained pictorially by gluing in the bra $\langle \phi_f |$. 

The corresponding bra $\langle \psi|$ and (unnormalized) density matrix $\rho = |\psi \rangle \langle \psi|$ for the pure vacuum state can then pictorially be presented as 
\ali{
	\langle \psi| = \adjincludegraphics[width=1.5cm,valign=c]{figs/figpsiket.png}, \qquad \rho = \adjincludegraphics[width=1.5cm,valign=c]{figs/figrho.png} . \label{rhofig}
}
An upper dashed line can take on a bra, a lower dashed line a ket, and the matrix $\rho$ both, with matrix elements 
\ali{
	\langle \phi| \rho | \phi' \rangle \equiv (\rho)_{\phi \phi'} = \adjincludegraphics[width=1.5cm,valign=c]{figs/figrhob.png}\llap{
		\parbox[][-.25cm][b]{0.85cm}{$\phi'$}} \llap{
		\parbox[][0.75cm][b]{0.85cm}{$\phi$}}  .  \label{rhophiphip}
}
Its trace $\tr \rho = \int \mathcal D \phi \langle \phi| \rho |\phi \rangle$, obtained by gluing along the previously dashed lines to identify and integrate out $\phi$,  
gives the partition function $Z$, 
\ali{
	Z = \tr \rho =  {\color{glueblue} \int \mathcal D \phi}   \adjincludegraphics[width=1.5cm,valign=c]{figs/figtrrho.png} \llap{
		\parbox[][-0.25cm][b]{1.2cm}{${\color{glueblue}\phi}$}} \llap{
		\parbox[][0.75cm][b]{1.2cm}{${\color{glueblue}\phi}$}}  = \adjincludegraphics[width=1.5cm,valign=c]{figs/figtrrhob.png} . \label{trrhoblue} 
}

In some figures of states \eqref{psiket} and density matrices \eqref{rhofig}  throughout in the text we will include field configurations in brackets to indicate how to read off the corresponding wave-functional \eqref{psiphif} and density matrix elements \eqref{rhophiphip}.  





\section{CFT entanglement from replica trick} \label{sectionreplicaCFT}


For a given statistical ensemble described by $\rho$, where $\rho$ is the probability distribution classically or density matrix quantum mechanically, the von Neumann entropy is defined in terms of the normalized density matrix $\hat \rho = \rho/\tr \rho = \rho/Z$ as 
\ali{
	S = - \tr (\hat \rho \log \hat \rho).   
}
It provides a fundamental, observer-dependent measure for the indeterminacy or lack of resolution of the system, e.g.~$S = k_B \log \Omega(E)$ in the microcanonical ensemble, for an observer in a closed system, in which every microstate is equally probable $\hat \rho = 1/\Omega(E)$, or $S = (1 - \beta \p_\beta) \log Z(\beta)$ for an open system observer in the canonical ensemble with Boltzmann probability distribution $\hat \rho = e^{-\beta H}/Z(\beta)$. 



The von Neumann entropy 
can be applied in the context of a conformal field theory to define the concept of `geometric entropy' or  entanglement entropy. 

The set-up we consider is a $(1+1)$-dimensional CFT with Euclidean path integral 
\ali{ 
	Z = \int \mathcal D \phi \, e^{-I[\phi]},  \qquad I[\phi] = \int_\mathbb{C} dx dt_E\mathcal L[\phi(x,t_E)],  
} 
prepared in a pure state $|\psi\rangle$ as pictured in \eqref{psiket}. 
The corresponding density matrix $\rho = |\psi\rangle \langle \psi|$ is  pictured in \eqref{rhofig}. We consider a constant time slice and geometrically bipartition the system, assuming the Hilbert space can be factorized, into a spatial region $A$ and its complement $\bar A$. 
An observer that only has access to region $A$ will measure a different density matrix, called the reduced density matrix 
\ali{
	\rho_A = \tr_{\bar A} \rho.  
}
It is obtained from $\rho$ by tracing out degrees of freedom in $\bar A$, and in general takes the form of a density matrix for a mixed state. 
The observer's lack of information about the full system can be quantified by the von Neumann entropy of the normalized reduced density matrix $\hat \rho_A = \rho_A/ Z$,  
\ali{
	S_A = - \tr (\hat \rho_A \log \hat \rho_A ) . \label{SAdef} 
}
This is by definition the \emph{geometric entropy} or \emph{entanglement entropy} associated with region $A$. 
It is a measure for the amount of missing information from the point of view of the observer in $A$, vanishing in the limit that the observer has access to the full system since the von Neumann entropy for a pure density matrix  is zero, and a measure for the amount of entanglement between degrees of freedom in $A$ and degrees of freedom in $\bar A$, vanishing when the pure state of the CFT is separable, $|\psi \rangle = |\psi_A \rangle|\psi_{\bar A} \rangle$ and $\rho_A= |\psi_A \rangle \langle \psi_A|$, and there is thus no such entanglement. 

From applying l'H\^{o}pital's rule, the definition for the entanglement entropy $S_A$ in \eqref{SAdef} can be rewritten as  
\ali{
	S_A &= (1 - n \p_n) \log Z(n)|_{n \ra 1},  \label{Sformula} \\
	Z(n) &= \tr \, (\rho_A^n) .  \label{eqZn} 
}
A positive integer $n$ factors of $\rho_A$ construct $Z(n)$, an $n$-fold replicated description of the system. Then $n$ needs to be analytically continued to non-integer values of $n$ to be able to take the derivative and limit in the definition \eqref{Sformula} of $S_A$. This is called the replica method. 


Now consider region $A$ to be an interval $x = x_1..x_2$ at $t_E=0$, with corresponding reduced density matrix $\rho_A$ and its square $\rho_A^2$ given by 
\ali{
	\rho_A = \adjincludegraphics[width=2cm,valign=c]{figs/figrhoA.png}, \qquad \rho_A^2 = {\color{glueblue} \int \mathcal D \phi}   \adjincludegraphics[width=4cm,valign=c]{figs/figrhoAsquared.png} \llap{
		\parbox[][-0.4cm][b]{3cm}{${\color{glueblue}\phi}$}}  
	\llap{
		\parbox[][-0.65cm][t]{1.1cm}{${\color{glueblue}\phi}$}}. 
}    
To calculate the corresponding entanglement $S_A$ in \eqref{Sformula}, we need to construct $Z(n) = \tr \rho_A^n$ for integer $n>1$. 
We can think of the gluing condition (referring  
to the identification and integrating out of the field configuration) in the matrix multiplication $\rho_A^2$ in two ways: 1) as continuing the \emph{coordinates} $(t_E,x)$ to live on a connected manifold consisting of two copies of the complex plane glued along the region $A$ slit,  
or 2) as a condition on the \emph{field content}, connecting the fields of two separate copies of the theory $\mathcal L(\phi_1)$ and $\mathcal L(\phi_2)$ along $A$. 
The first is a `worldsheet' and the second a `target space' perspective, with $(t_E,x)$ running over $\mathcal R_{n,A}$ and $\mathbb{C}$ respectively. 
Constructing $\tr \rho_A^n$ in the first way  
gives rise to the replicated worldsheet  
$\mathcal R_{n,A}$, path integral integration over which gives $Z(n)$ 
\ali{
	\tr \rho_A^n = Z(n) &= \int [\mathcal D\phi]_{_{\mathcal R_{n,A}}} \, e^{-\int_{\mathcal R_{n,A}} dx dt_E \mathcal L[\phi(x,t_E)]} . \label{WSZn} 
}
$\mathcal R_{n,A}$ is called the replica manifold and is pictured in the left figure of Fig \ref{figreplica}. In the $Z(1)$ manifold a rotation over $2\pi$ will bring you back to the same location, but in the $Z(n)$ manifold it takes a rotation of $2\pi n$ around the boundary points $\p A$ of region $A$ to get back to the same location. That is, there are branch points and conical singularities at $\p A$, with a conical excess of $2\pi n - 2\pi$.  


\begin{figure}
	\centering \includegraphics[scale=0.15]{figs/figreplica} \llap{
		\parbox[][-10cm][b]{4.3cm}{$t_E$}} \llap{
		\parbox[][-0.5cm][b]{4.1cm}{$x$}}  \qquad \includegraphics[scale=0.15]{figs/figreplicab} \llap{
		\parbox[][-2.9cm][b]{4.2cm}{$t_E$}} \llap{
		\parbox[][-0.5cm][b]{4.1cm}{$x$}}
	\caption{$Z(n) = \tr \rho_A^n$ from the WS perspective \eqref{WSZn} (left) and the TS perspective \eqref{TSZn} (right). It is the path integral of the theory $\mathcal L[\phi]$ on the $\mathbb Z_n$ symmetric replica manifold $\mathcal R_{n,A}$, or equivalently the path integral of the theory $\mathcal L^{(n)}[\phi_i]$ over the orbifold manifold $\mathbb C \equiv \mathcal R_{n,A}/\mathbb Z_n$ in the presence of twist fields. On the right, $Z(n) = \langle T_n(x_1) \tilde T_n(x_2)\rangle$. 
	} 
	\label{figreplica}
\end{figure}

In the second perspective we write (here $i = 1 \cdots n$) 
\ali{
	\tr \rho_A^n = Z(n) &= \int [\mathcal D\phi_i]_{_{\mathbb{C},\, bc}} \, e^{-\int_\mathbb{C} dx dt_E\mathcal L^{(n)}[\phi_i(x,t_E)]} \label{TSZn}
}
with 	
\ali{ 
	\mathcal L^{(n)}[\phi_1, ..., \phi_n] &=  \mathcal L[\phi_1(x,t_E)] + \cdots + \mathcal L[\phi_n(x,t_E)] \\  
	bc &= \left\{ \begin{array}{l} \phi_1(t_E = 0^+, x \in A) = \phi_2(t_E=0^-, x\in A) \\  \phi_2(t_E = 0^+, x \in A) = \phi_3(t_E=0^-, x\in A) \\ \cdots \\ \phi_n(t_E = 0^+, x \in A) = \phi_1(t_E=0^-, x\in A) \end{array} \right. . 
}	
The boundary conditions $bc$ express a global symmetry of the theory $\mathcal L^{(n)}[\phi_1, ..., \phi_n]$ under exchange of the fields $\phi_i \ra \phi_{i+1}$,  the $\mathbb{Z}_n$ permutation symmetry. 
The conditions can be implicitly implemented by placing twist fields at $\p A$. These have the property that when circling a twist field $T_n$ resp. anti twist field $\tilde T_n$, a field $\phi_{i \text{ mod } n}$ is transformed into $\phi_{i+1 \text{ mod } n}$, resp. to $\phi_{i-1 \text{ mod } n}$. Then,    
\ali{
	Z(n) &= \int [\mathcal D\phi_i]_{_{\mathbb C}} \,\, T_n(x_1) \, \tilde  T_n(x_2)  \, e^{-\int_\mathbb{C} dx dt_E\mathcal L^{(n)}[\phi_i(x,t_E)]} \nonumber  \\
	&= \langle  T_n(x_1) \, \tilde T_n(x_2) \rangle_{\mathcal L^{(n)},\mathbb C}  \label{TSZnbis}
}
where we used a condensed notation for the locations of the twist fields $T_n(x=x_1,t_E=0)$ and $\tilde  T_n(x=x_2,t_E=0)$,   
and in the second line we rewrite the $Z(n)$ partition function as a 2-point function of twist fields. This interpretation of $Z(n)$ is pictured on the right of Fig \ref{figreplica}.  




\subsection{Replica manifold} \label{subsectreplica}

We will proceed first with calculating $Z(n)$ in the replicated worldsheet point of view of Fig \ref{figreplica}a, following \cite{Cardy:2008jc}, and comment on the twist field correlator perspective later. 


For a theory with stress tensor defined via $\delta I = \frac{1}{4\pi} \int T_\mn \delta g^\mn \sqrt{g}\, d^d x$, 
the partition function satisfies $\delta \log Z = -\frac{1}{4\pi} \int \vev{T_\mn} \delta g^\mn \sqrt{g}\, d^d x$. This allows us to write down 
the behavior of $Z(n)$ under a coordinate transformation $x^\mu \ra x'^\mu = x^\mu + \delta x^\mu$ (which induces a metric transformation $\p_\mu \delta x_\nu + \p_\nu \delta x_\mu$)
\ali{
	\delta \log Z(n) = - \frac{1}{2\pi} \int \vev{T^\mu_{\phantom{\mu}\nu}} \frac{\p \delta x^\nu}{\p x^\mu} d^2 x . \label{deltalogZn}
} 
This is for the theory $\mathcal L[\phi(x,t_E)]$ on the replica manifold $\mathcal R_{n,A}$ spanned by $x^\mu = (x,t_E)$, which is everywhere flat except at the branch points $\p A = (x_1,x_2)$. 

\begin{figure}
	\centering \includegraphics[scale=0.16]{figs/figcontour} \llap{
		\parbox[][-3.6cm][b]{3.75cm}{$t_E$}} \llap{
		\parbox[][-1.5cm][b]{0.3cm}{$x$}} \llap{
		\parbox[][-1.15cm][b]{2.6cm}{$x_1$}} \llap{
		\parbox[][-1.15cm][b]{1.68cm}{$x_2$}}  \llap{
		\parbox[][-3cm][b]{3.1cm}{{\color{contourpink}$2\pi n$}}}
	\caption{Contour for calculating $\delta \log Z(n)$ in \eqref{toevaluate}.} \label{figContour} 
\end{figure}
\begin{figure}
	\centering \includegraphics[scale=0.16]{figs/figcoordtransf} \llap{
		\parbox[][-2.1cm][b]{2.6cm}{$2\pi n$}} \llap{
		\parbox[][-0.5cm][b]{2.4cm}{$x_1$}} \llap{
		\parbox[][-0.5cm][b]{1.4cm}{$x_2$}}  \qquad \quad  \includegraphics[scale=0.16]{figs/figcoordtransfb} \llap{
		\parbox[][-2.1cm][b]{2.3cm}{$2\pi n$}} \llap{
		\parbox[][-0.5cm][b]{1.85cm}{$0$}} \llap{
		\parbox[][-0.5cm][b]{0.76cm}{$\infty$}} \qquad \quad  \includegraphics[scale=0.16]{figs/figcoordtransfc} \llap{
		\parbox[][-1.8cm][b]{1cm}{$2\pi$}} \llap{
		\parbox[][-0.95cm][b]{1.9cm}{$0$}} \llap{
		\parbox[][-0.95cm][b]{0.76cm}{$\infty$}}
	\caption{Coordinate transformation \eqref{orbifoldcoord} from $\zeta$ on the replica manifold (left) to $\frac{\zeta-x_1}{x_2-\zeta}$ (middle) to $z$ on the orbifold, i.e.~the regular complex plane (right). The `pizza slice' representing the orbifold is drawn here for $n = 6$. 
	} \label{figPizza} 
\end{figure}


We can now consider a coordinate transformation that induces a change $\delta L$ of the length $L = x_2-x_1$ of the interval $A$, and then integrate $\delta \log Z(n)$ to find the dependence of $Z(n)$ on $x_1$ and $x_2$. We can choose for example the non-conformal transformation $x' = x + \theta(x-x_0)  \delta L$, where $x_1 < x_0 < x_2$. This indeed has the effect $L' = x_2'-x_1' = L + \delta L$. 
Then 
\ali{
	\delta \log Z(n) &= - \frac{\delta L}{2\pi} \int \vev{T_{xx}} \delta(x-x_0) dx dt_E = - \frac{\delta L}{2\pi} \int \vev{T_{xx}(x_0,t_E)} dt_E \\ 
	&= - \delta L \left( \oint_{x_1} \frac{d\zeta}{2\pi i} 
	\vev{T_{\zeta\zeta}(\zeta)} - \oint_{x_1} \frac{d\bar\zeta}{2\pi i} \vev{T_{\bar\zeta\bar\zeta}(\bar \zeta)} \right) \label{toevaluate}			  
} 
In the second line we moved to complex coordinates $\zeta = x + i t_E$, deforming the contour along the full infinite range of $t_E$ at $x=x_0$ to $\zeta$ encircling $x_1$ ($x_2$ is an equally good choice), see Fig \ref{figContour}. Now we still need to know $\vev{T_{\zeta\zeta}}$. On a regular complex plane with coordinate $z$, the stress tensor expectation value is zero because of rotational and translation invariance. The replica manifold can be conformally transformed into the complex plane, i.e.~uniformized, by the conformal transformation 
\ali{
	z = \left(\frac{\zeta-x_1}{x_2-\zeta}\right)^{1/n}  \label{orbifoldcoord} 
}
which consists of first mapping the branch points to $0$ and infinity, and then going to the $\mathbb{Z}_n$ orbifold, pictured as `pizza slice' in Fig \ref{figPizza}.     
Under this conformal transformation, the stress tensor of the CFT transforms anomalously, $T_{\zeta\zeta}(\zeta) = \left(\p z/\p \zeta\right)^2 T_{zz}(z) + \frac{c}{12} \{z,\zeta\}$, with $c$ the central charge of the CFT. It follows that $\vev{T_{\zeta\zeta}}$ is determined by the Schwarzian derivative of $z(\zeta)$ to be 
\ali{
	\vev{T_{\zeta\zeta}(\zeta)} = \frac{c}{24} \left(1 - \frac{1}{n^2}\right) \frac{(x_2-x_1)^2}{(\zeta-x_1)^2(\zeta-x_2)^2}. \label{Tzetazetavev} 
}  
In evaluating the complex integrals in \eqref{toevaluate}, the residue theorem picks up an extra $n$. This is because $\zeta$ is a complex coordinate with an argument of range $2\pi n$ around $x_1$ or $x_2$. 
It follows that 
\ali{
	\frac{\delta \log Z(n)}{\delta L} = - \frac{c}{6} \frac{n - 1/n}{L}   
}
or $\log Z(n) =  - \frac{c}{6} (n-1/n) \log L + ...$, where the dots are $L$-independent terms. 
We have obtained an expression for the $L$-dependence of $Z(n) = \tr \rho_A^\alpha|_{\alpha = n}$ for integer values of $\alpha=n\geq1$. 
The full partition function   
for complex $\alpha$ is then of the form $Z(\alpha) + \sin(\pi \alpha) g(\alpha)$ in the $\text{Re } \alpha > 1$ region with $g$ an analytic function. 
It can be shown based on the fact that $\hat \rho_A$ 
has eigenvalues $\lambda \in [0,1]$ and using Carlson's theorem \cite{Solodukhin:2011gn} %p.9 
that $g(\alpha) \equiv 0$ and therefore the obtained $Z(n)$ is valid beyond integer $n$. After substitution in \eqref{Sformula}, this leads to the famous formula for the vacuum entanglement of an interval $A$ in a 2-dimensional CFT  
\ali{
	S_A = \frac{c}{3} \log \frac{L}{\epsilon}. \label{Sresult}
}
Its scaling with $c$ shows the Weyl anomaly of the CFT crucially enters the derivation -- in \eqref{Tzetazetavev} in this case, as the Schwarzian derivative transformation rule is implied by the Weyl anomaly. 
The UV cutoff $\epsilon$ has to be introduced for dimensional reasons, and regulates the arbitrarily large contributions to the entropy from UV degrees of freedom arbitrarily close to $\p A$. 
 

For the choice of $\delta L$-inducing coordinate transformation we could have also followed \cite{Holzhey:1994we} and chosen a scale transformation.  


\subsection{Replicated target space} Let us also comment on the twist field correlator derivation of $\log Z(n)$, i.e.~the perspective of Fig \ref{figreplica}b. Details can be found in  \cite{Calabrese:2009qy,Calabrese:2004eu}. 

The stress tensor expectation value $\vev{T_{\zeta\zeta}(\zeta)}$ in \eqref{Tzetazetavev} corresponds to the insertion of one stress tensor operator on the replica manifold in Fig \ref{figreplica}a, let's say on the $i$-th sheet. From the point of view of Fig \ref{figreplica}b, it is the insertion of the stress tensor of the $i$-th copy of the theory in the presence of twist field operators and thus (normalizing left and right with the insertion of the unit operator) 
\ali{
	\left \langle T_{\zeta\zeta}(\zeta) \right \rangle_{{\mathcal L, \mathcal R_{n,A}}} \, = \, \frac{\left \langle T_{\zeta\zeta}^{(i)}(\zeta) \, T_n(x_1) \, \tilde T_n(x_2) \right \rangle_{\mathcal L^{(n)},\mathbb C}}{\left \langle T_n(x_1) \, \tilde T_n(x_2) \right \rangle_{\mathcal L^{(n)},\mathbb C}} , \label{condition}
}
where $x_1, x_2$ are short for the twist field locations $(\zeta_1,\bar \zeta_1), (\zeta_2,\bar \zeta_2)$ more generally, and  where we have included for clarity subscripts specifying the theory in which the expectation values are taken. 
The denominator of the right hand side is the $Z(n)$ we are to determine. For primary twist fields \cite{Cardy:2007mb} it takes the form $1/|x_1-x_2|^{2 d_n}$, with $d_n$ the unknown scaling dimension of $T_n$ and $\tilde T_n$.   
The numerator of the right hand side can then be rewritten in terms of the twist field two-point function,  
\ali{
	\left \langle T_{\zeta\zeta}^{(i)}(\zeta) \, T_n(x_1) \, \tilde T_n(x_2) \right \rangle = \frac{1}{n} \frac{d_n}{2} \frac{(x_1-x_2)^2}{(\zeta-x_1)^2 (x_2 - \zeta)^2} \left \langle T_n(x_1) \, \tilde T_n(x_2) \right \rangle, \label{fromWardid}
}
by applying the Ward identity for the $(\mathcal L^{(n)}, \mathbb{C})$ theory with stress tensor $T_{\zeta\zeta}^{(n)} = n T_{\zeta\zeta}^{(i)}$ (by extensivity of the Lagrangian and thus Hamiltonian density). 
Then \eqref{Tzetazetavev}, \eqref{condition} and \eqref{fromWardid} impose $d_n = \frac{c}{12}(n - 1/n)$, leading to the same $\log Z(n)$ as in the previous section, and the same result for $S_A$ in \eqref{Sresult}. 

One main advantage of this perspective is that the known transformation behavior of the %a
conformal two-point function \eqref{TSZnbis} under conformal transformations immediately gives us the formula for $\log Z(n)$ and thus the entanglement $S_A$ in a conformally related geometry, as we will now briefly discuss.  

In the notation $f = x + i t_E, \bar f=x-i t_E$ for our current set-up, with metric $ds^2 = df d\bar f = dt_E^2 + dx^2$ and state the vacuum state $|0\rangle_f$ as measured in $f$-coordinates, 
we have   
\ali{
	S_A = \frac{c}{3} \log \left| \frac{f_1 - f_2}{\epsilon_f} \right| \label{planeformula}
}
for 
\ali{
	\langle \, T_n(f_1,\bar f_1) \tilde T_n(f_2, \bar f_2) \, \rangle = |f_1-f_2|^{-\frac{c}{6}(n - 1/n)}.  
}
Under a conformal transformation 
$f = f(z)$, $\bar f = \bar f(\bar z)$, the metric transforms $ds^2 = \left|\p f/\p z \right|^2 dz d\bar z \equiv \Omega(z,\bar z) dz d\bar z$ with a Weyl factor $\Omega$, and the entanglement  
in $z$-coordinates becomes 
\ali{
	S_A = \frac{c}{6} \log  \frac{(f(z_1) - f(z_2))(\bar f(\bar z_1) - \bar f(\bar z_2))}{\sqrt{f'(z_1) f'(z_2) \bar f'(\bar z_1) \bar f'(\bar z_2)} \epsilon_z \epsilon_{\bar z}}. \label{Sconf}
}
This follows from 
\ali{
	\langle \, T_n(z_1,\bar z_1) \tilde T_n(z_2, \bar z_2) \,\rangle &= \left|\frac{f(z_1)-f(z_2)}{\sqrt{f'(z_1)f'(z_2)}}\right|^{-\frac{c}{6}(n - 1/n)} \\ 
	&= |f'(z_1) f'(z_2)|^{d_n}  \langle \, T_n(f_1,\bar f_1) \tilde T_n(f_2, \bar f_2) \, \rangle  
} 
where we used the transformation behavior of a primary field \\
$\mathcal O(z,\bar z) =\left(\p z'/\p z\right)^h \left(\p \bar z'/\p \bar z\right)^{\bar h} \mathcal O'(z',\bar z')$ 
for the twist fields of dimension $h = \bar h = d_n/2$ under an 
$f(z)$ transformation. 




As can be seen from comparison of \eqref{planeformula} and \eqref{Sconf}, the UV cutoffs as measured in $f$ or $z$ coordinates are related by \cite{Holzhey:1994we} 
\ali{
	\epsilon_f = f'(z) \epsilon_z   \label{cutoffs}
}
(and writing $\epsilon$ as $\sqrt{\epsilon_1 \epsilon_2}$). 
Indeed, from $f(z+\epsilon_z) \approx f(z) + \epsilon_z f'(z)$ with $\epsilon_z \equiv \delta z$, it follows that $\epsilon_f \equiv \delta f$ is given by the above.  

The formula \eqref{Sconf} can be applied immediately to the cases pictured in Fig \ref{figcylS} of intervals in the $z$-cylinder.  

\begin{figure}
	\qquad \qquad \qquad 
	\includegraphics[scale=0.115]{figs/figcylSa} \llap{
		\parbox[][-2.1cm][b]{1.75cm}{$u$}} \llap{
		\parbox[][-2.1cm][b]{1.1cm}{$v$}} \llap{
		\parbox[][-0.3cm][b]{0.25cm}{$\phi$}}
	\qquad \qquad \qquad 
	\includegraphics[scale=0.115]{figs/figcylSb} \llap{
		\parbox[][-1.7cm][b]{1.95cm}{$u$}} \llap{
		\parbox[][-1.7cm][b]{1.3cm}{$v$}} \llap{
		\parbox[][-0.3cm][b]{0.25cm}{$\tau$}}
	\caption{
		Left:  Finite size formula $S_A = \frac{c}{3} \log \left( \frac{\Sigma}{\pi \epsilon} \sin \frac{\pi (v-u)}{\Sigma} \right)$ for an interval $z_1 = \bar z_1 = u$, $z_2 = \bar z_2 = v$ on the cylinder of size $\Delta \phi = \Sigma$ follows 
		from \eqref{Sconf} with $f(z) = \exp{ \left( \frac{2\pi}{\Sigma} i z \right)}$ relating the $f$-coordinate of the plane to the $z$-coordinate of the cylinder with compact $\phi = \text{Re }z$.  
		Right:  Finite temperature formula $S_A = \frac{c}{3} \log \left( \frac{\beta}{\pi \epsilon} \sinh \frac{\pi (v-u)}{\beta} \right)$ for an interval $z_1 = \bar z_1 = u$, $z_2 = \bar z_2 = v$ on the cylinder of size $\Delta \tau = \beta$ follows from \eqref{Sconf} with $f(z) = \exp{ \left(\frac{2\pi}{\beta} z\right)}$ relating the $f$-coordinate of the plane to the $z$-coordinate of the cylinder with compact $\tau = \text{Im }z$.   
	}
	\label{figcylS}
\end{figure}







\section{Relation to thermal entropy} \label{sectionThermal} 


There is a `short-cut' for deriving the interval entanglement $S_A$ in \eqref{Sresult} which will also be important for the holographic interpretation. It consists of mapping the set-up of section \ref{sectionreplicaCFT} to a thermal system. This section follows \cite{Holzhey:1994we}.   

A conformal change of coordinates induces a change of basis among the operators of the theory and affects the density matrix 
through a unitary transformation, to which the trace in \eqref{SAdef} 
is insensitive. 
Moreover, the anomalous ($\sim c$) contribution to the transformed Hamiltonian only affects the unnormalized density matrix (and thus e.g.~$\log(Z)$), but not the normalized density matrix that appears in the von Neumann entropy \eqref{SAdef}. 
This is to say that the entanglement entropy is conformal invariant, and hence its calculation can be simplified by considering well-chosen conformal mappings.      


We consider the same theory $Z \equiv Z(1)$ of section \ref{sectionreplicaCFT}, in complex coordinates $\zeta = x + i t_E$, $\bar \zeta = x - i t_E$. First we translate the interval $A$ from $x = [x_1, x_2]$ to $x = [0,L]$. The vacuum state of the system is pictured in Fig \ref{figHolzheypsi}a.   Then we consider the conformal transformation to $w = (\zeta - x_1)/(x_2-\zeta)$ or 
\ali{
	w = \frac{\zeta}{L-\zeta},  
}
which maps the interval to the positive half-line. 
Keeping track of the UV cutoff, the more precise statement is that the $x = [\epsilon, L-\epsilon]$ interval is mapped to $w = \text{Re }w = [\frac{\epsilon}{L} ,\frac{L}{\epsilon}]$. Then, we further transform to 
\ali{ 
	z = \frac{1}{\kappa} \log w, 
} 
with $\kappa$ an arbitrary real number, that should therefore not affect  
any physics.  
It maps the positive half-line to $z = \text{Re }z = [\frac{1}{\kappa} \log \frac{\epsilon}{L} , \frac{1}{\kappa} \log \frac{L}{\epsilon}]$ and the negative half-line to $z = [\frac{1}{\kappa} \log \frac{\epsilon}{L} - \frac{i \pi}{\kappa}, \frac{1}{\kappa} \log \frac{L}{\epsilon} - \frac{i \pi}{\kappa}]$, so that $A$ is now one side of a strip of width $\pi/\kappa$. The partition function $Z(1) = \tr \rho$ 
is mapped to the $w$-\emph{annulus} and the $z$-\emph{cylinder} respectively. The imaginary part of $z$ 
\ali{
	\text{Im }z \equiv \tau 
}
brings forth a periodic coordinate; as the angle of the $w$-annulus with periodicity $\Delta \tau = 2\pi$, and as the compact direction of the $z$-cylinder with periodicity $\Delta \tau = 2\pi/\kappa$. 
A direct consequence of this is that the reduced density matrix $\rho_A$ in both pictures takes the form of, not just any mixed density matrix, but a \emph{thermal} density matrix 
\ali{ \rho_A= e^{-(\Delta \tau) H_\tau} = \rho_{thermal} \label{BWtheorem}} 
as demonstrated in Fig \ref{figHolzheyrhoA}.  
On the annulus, $H_\tau$ is the generator of rotation in the Euclidean plane, and thus the boost generator in Lorentzian signature. On the cylinder, $H_\tau$ is the actual Hamiltonian or generator of cylinder (Euclidean) time translation. 
The corresponding inverse temperature is denoted $\beta$ by choosing $\kappa = 2\pi/\beta$. It is a fictitious temperature measured by an observer that only has access to half of the $w$-plane or only one side of the $z$-strip. 
Fig \ref{figHolzheyrhoA}b gives us the path integral derivation of the Bisognano Wichmann theorem  \eqref{BWtheorem} \cite{Bisognano:1976za}  (e.g.~\cite{UnruhWeiss:1983ac,Headrick:2019eth}). 

\begin{figure}
	\centering \includegraphics[scale=0.18]{figs/figHolzheya}\llap{
		\parbox[b]{5cm}{a)\\\rule{0ex}{2cm}}} \llap{
		\parbox[][-4.1cm][b]{2.45cm}{$\epsilon$}} \llap{
		\parbox[][-4.1cm][b]{1.8cm}{$L-\epsilon$}} \llap{
		\parbox[][-2.3cm][b]{2.7cm}{{\color{taugreen} $\tau$}}} 
	\qquad \qquad \qquad \includegraphics[scale=0.18]{figs/figHolzheyb}\llap{
		\parbox[b]{5cm}{b)\\\rule{0ex}{2cm}}} \llap{
		\parbox[][-4.1cm][b]{2.1cm}{$\frac{\epsilon}{L}$}} \llap{
		\parbox[][-4.1cm][b]{0.9cm}{$\frac{L}{\epsilon}$}} \llap{
		\parbox[][-2.3cm][b]{2.4cm}{{\color{taugreen} $\tau$}}} \\ 
	\quad \includegraphics[scale=0.18]{figs/figHolzheyc-bisa}\llap{
		\parbox[b]{5.4cm}{c)\\\rule{0ex}{2.5cm}}}  \llap{
		\parbox[][-4.1cm][b]{4.6cm}{$\frac{1}{\kappa}\log\frac{\epsilon}{L}$}} \llap{
		\parbox[][-4.1cm][b]{0.9cm}{$\frac{1}{\kappa}\log\frac{L}{\epsilon}$}} \llap{
		\parbox[][-3.4cm][b]{5.1cm}{$0$}} \llap{
		\parbox[][-.9cm][b]{5.5cm}{$-\frac{i \pi}{\kappa}$}} \llap{
		\parbox[][-2.3cm][b]{2.5cm}{{\color{taugreen} $\text{Im } z = \tau$}}} 
	\qquad \qquad 
	\includegraphics[scale=0.18]{figs/figHolzheyc-bisb} \llap{
		\parbox[][-2.4cm][b]{5.1cm}{or}} \llap{
		\parbox[][-1.2cm][b]{4.2cm}{{\color{taugreen} $\tau$}}}
	\caption{ 
		Prepared state, with region $A$ in red and complementary region $\bar A$ in green, under conformal mappings between the $\zeta$, $w$- and $z$-geometry, in respectively a), b) and c). 
		In terms of metrics (for $\kappa=1$ choice), $d\tau^2 + \frac{dR^2}{R^2} = dz d\bar z$ (cylinder)  
		$=d(\log w) d(\log \bar w) = \frac{1}{w \bar w} dw d\bar w$ 
		$ \ra dw d\bar w$ (plane) $= R^2 d\tau^2 + dR^2$ with $w = R \exp(i \tau)$ and $z = x_z + i \tau = (\log R) + i \tau$. Under these mappings, a periodic coordinate $\tau$ around $\p A$ becomes the angle on the annulus or the Euclidean time coordinate of the cylinder. The small blue arrows point in the direction of the Euclidean evolution $t_E$.   
	} \label{figHolzheypsi}
\end{figure}

\begin{figure}
	\centering \includegraphics[scale=0.16]{figs/figHolzheyrhoa}\llap{
		\parbox[b]{5cm}{a)\\\rule{0ex}{3cm}}} \llap{
		\parbox[][-2.7cm][b]{2.6cm}{${\color{taugreen} \tau}$}}  \llap{
		\parbox[][-2cm][b]{1.9cm}{$(\phi)$}} \llap{
		\parbox[][-3.4cm][b]{2cm}{$(\phi')$}} \qquad \qquad \qquad \includegraphics[scale=0.16]{figs/figHolzheyrhob}\llap{
		\parbox[b]{5cm}{b)\\\rule{0ex}{3cm}}} \llap{
		\parbox[][-3cm][b]{2.9cm}{${\color{taugreen} \tau}$}} \llap{
		\parbox[][-2.5cm][b]{1.3cm}{$(\phi)$}} \llap{
		\parbox[][-3.7cm][b]{1.4cm}{$(\phi')$}} \\ 
	\quad \includegraphics[scale=0.16]{figs/figHolzheyrhoc-bisa}\llap{
		\parbox[b]{5.4cm}{c)\\\rule{0ex}{2.5cm}}} \llap{
		\parbox[][-2.3cm][b]{2.7cm}{${\color{taugreen} \tau}$}} \llap{
		\parbox[][-5.35cm][b]{3.3cm}{$(\phi)$}} \llap{
		\parbox[][0.35cm][b]{3.4cm}{$(\phi')$}} \llap{
		\parbox[][-2.3cm][b]{0.6cm}{$\frac{2\pi}{\kappa}$ or $\beta$}} 
	\quad \quad  
	\includegraphics[scale=0.17]{figs/figHolzheyrhoc-bisb} \llap{
		\parbox[][-2.4cm][b]{5.5cm}{or}} \llap{
		\parbox[][-1.2cm][b]{4.9cm}{{\color{taugreen} $\tau$}}} \llap{
		\parbox[][-4.65cm][b]{3.3cm}{$(\phi')$}} \llap{
		\parbox[][-3.1cm][b]{3.4cm}{$(\phi)$}} \llap{
		\parbox[][-1.2cm][b]{1.65cm}{$\beta$}} 
	\caption{
		Reduced density matrix $\rho_A$ for the state in Fig \ref{figHolzheypsi}, under conformal mappings between the $\zeta$, $w$- and $z$-geometry, in respectively a), b) and c). In brackets are the field configurations to read off the matrix elements $(\rho_A)_{\phi \phi'} = \langle \phi| \rho_A | \phi'\rangle$.    		
	} \label{figHolzheyrhoA} 
\end{figure}


In the cylinder picture, $S_A$ is now equal to the thermal entropy $S_{thermal}$ of the strip of width $\beta$, which is simply given by the thermodynamic formula $S_{thermal} = \beta E + \log Z(\beta)$  
in terms of the energy and free energy, or  
\ali{
	S_{thermal} = (1 - \beta \p_\beta) \log Z(\beta).  \label{eqSthermal}
}
The partition function of the CFT on the strip of width $\beta$ and length $L_z \gg \beta$ was determined in \cite{Bloete:1986qm,Affleck:1986bv},  
again by making use of the Weyl anomaly, to be (to leading order in $1/\beta$) 
\ali{
	\log Z(\beta) = \text{const } \beta \,  L_z + \frac{\pi c}{6 \beta} L_z \, .   
}
The constant in the first term is non-universal, i.e.~CFT-dependent, and drops out of the formula for $S_{thermal}$, which gives 
\ali{
	S_{thermal} = \frac{c \pi}{3 \beta} L_z = \frac{c}{3} \log\frac{L}{\epsilon} \, . 
}
$L_z$ is the full (divergent) length $A$ of the cylinder in Fig \ref{figHolzheypsi}c, which is given by $\frac{\beta}{\pi} \log \frac{L}{\epsilon} \equiv \frac{\beta}{\pi} \log \frac{L_\zeta}{\epsilon_\zeta}$, so that in the second equality we find, as we should, $S_A = \frac{c}{3} \log\frac{L}{\epsilon}$ in \eqref{Sresult}.  

As pointed out in \cite{Casini:2011kv}, there is an IR/UV quality to the relation $L_z \sim \log L/\epsilon$ between the cylinder regulator (large $L_z$) and the plane regulator (small $\epsilon$), which you typically find in a holographic setting\footnote{ 
	The difference is in the contractability of the $\tau$-circle (non-contractible on the cylinder, contractible on the plane), 
	which is also reflected in the holographic description (non-contractible on the cylinder, contractible in the bulk description, see Fig \ref{figBTZ}).  
	}.   
A last comment about the derivation is that the formula for the entanglement of a finite interval on the thermal cylinder in Fig \ref{figcylS}b should become the thermal entropy for the state in Fig \ref{figHolzheypsi}c in the limit of $L_z \gg \beta$. This can be easily checked: $\lim_{L_z \ra \infty} \log(\beta/(\pi \epsilon_z) \sinh (\pi L_z/\beta)) = \pi L_z /\beta + \log (\beta/(2\pi \epsilon_z))$, with the second term vanishing for $\epsilon_z = \beta/(2\pi) \ll L_z$ following from \eqref{cutoffs} with $z = \beta/(2\pi) \log (\zeta/(L-\zeta))$.  



\begin{figure}
	\centering \includegraphics[scale=0.12]{figs/figZ-copy} \llap{
		\parbox[][-0.4cm][b]{8cm}{$\beta$}} \llap{
		\parbox[][-0.4cm][b]{0.8cm}{${n \beta}$}}
	\caption{$Z(1)$ and $Z(n)$} 
	\label{figZncylCFT}
\end{figure}



In the above paragraph, $Z(\beta)$ is the $Z(1)$ on the thermal cylinder (where $\beta$ could also be taken to be $2\pi$ by choosing $\kappa=1$). 
The replicated manifold $\tr \rho_A^n$, obtained by gluing $n$ copies of 
the $Z(1)$ strip along $A$, is again a cylinder, $Z(n \beta) \equiv Z(n)$, only this time with periodicity $n \beta$ (Fig \ref{figZncylCFT}). By the conformal mapping to the cylinder, the conical singularities at $\p A$ have been removed from the replica manifold, and the analytic continuation to non-integer $n$ is immediate, since the periodicity of the cylinder can be varied continuously.    
The replica trick \eqref{Sformula} indeed 
gives $S_A = S_{thermal}$ \ali{
	S_{thermal} = (1 - n \p_n) \log Z(n)|_{n \ra 1} = \frac{c \pi}{3 \beta} L_z 
}
from the trivial $n$-dependence in $Z(n) \equiv Z(n \beta) = \frac{\pi c}{6 n \beta} L_z + \text{const } n \beta L_z$.  
The triviality of the thermal replica trick on the cylinder, combined with the argument that $S_A$ is conformal invariant, signals that the replica derivation of the interval entanglement $S_A$ and in particular the question of analytic continuation of $n$ should be well-defined (section \ref{sectionreplicaCFT}), as should its holographic interpretation (section \ref{sectholoholzhey}).  

The replica manifold in the annulus picture does have a conical 
singularity at the origin $\epsilon \ra 0$ with conical excess $2\pi (n-1)$. The length of the region $A$ can be varied by a scale transformation $x^\mu \ra x'^\mu = (1 - 2 \frac{\delta\epsilon}{\epsilon}) x^\mu$, such that $Z(n)$ in \eqref{deltalogZn} has to satisfy  
\ali{
	\delta \log Z(n) = \frac{\delta \log \epsilon}{\pi} \int \vev{T^\mu_{\phantom{\mu}\mu}} d^2 x . 
}
Here, the trace of the stress tensor can be obtained from applying the argument leading to \eqref{Tzetazetavev}, or by directly making use of the Weyl anomaly $\vev{T^\mu_{\phantom{\mu}\mu}} = \frac{c}{12} R$ 
which relates it to the curvature $R \sim (1-n) (\delta^{(2)}(x_1) + \delta^{(2)}(x_2))$ of the manifold and its boundaries (see also \cite{Ryu:2006ef} e.g.). This procedure for deriving \eqref{Sresult}, detailed in \cite{Holzhey:1994we}, gives an alternative to the replica trick derivation of subsection \ref{subsectreplica} that makes the coarse graining physics behind the geometric entropy $S_A$ apparent. 



\section{Thermofield double}  \label{sectTFD}



In the previous section we encountered the  
possibility of an observer restricted to a part of spacetime $A$ measuring a state $\rho_A$ that is thermal even though the full system is in the vacuum state. This fits in the general concept of the thermofield double 
\cite{Takahasi:1974zn,Martin:1959jp}, which will be discussed in this section.   



A thermofield double (TFD) is a particular vacuum state that is constructed to reproduce thermal physics of a given QFT. We will focus on a conformal QFT in particular, with a Hamiltonian $H_R$ and Hilbert space $\mathcal H_R$, and on $1+1$ dimensions. 
The Euclidean path integral for compactified Euclidean time with period $\Delta \tau = \beta$ gives the thermal partition function $Z(\beta)$ of the theory. In path integral visualization, 
$Z(\beta) = \tr \rho_{thermal}$ is a cylinder $\mathcal C(\beta)$ when the spacelike direction of the theory is non-compact, and thus $\rho_{thermal}$ is the cylinder before tracing, i.e.~with open cut  
\ali{
	\rho_{thermal} = \adjincludegraphics[width=2cm,valign=c]{figs/TFDrhothermal.png}.  
} 
The statement is that the same physics can be described by constructing a state $|\psi\rangle$, called the TFD state, as the state 
\ali{
	|\psi\rangle = \adjincludegraphics[width=2.5cm,valign=c]{figs/TFDpsi.png} \llap{
		\parbox[][1.3cm][b]{2.5cm}{{\scriptsize$\tau$}}} \llap{
		\parbox[][2.3cm][b]{2.15cm}{{\tiny $\beta/2$}}} \llap{
		\parbox[][-0.7cm][b]{2.1cm}{{\scriptsize$(\phi_R)$}}} \llap{
		\parbox[][-0.4cm][b]{0.8cm}{{\scriptsize$(\phi_L)$}}}   
	\label{TFDpicture}
}
living in the doubled Hilbert space $\mathcal H_L \times \mathcal H_R$, with total Hamiltonian $H = H_R-H_L$ (as time $\tau$ runs downwards in the left copy and upwards in the right copy). We refer to the left copy as the one where $\tau$ starts and the right copy where $\tau$ arrives. This state is constructed such that it satisfies the property that its density matrix 
\ali{
	\rho = |\psi\rangle \langle \psi| =  \adjincludegraphics[width=2cm,valign=c]{figs/TFDrho.png} \label{rhoTFD}
}
reduces to the thermal density matrix of the original `right' system when the left copy is traced out\footnote{
	In equation \eqref{rhoATFD} and in the rest of the section, the pictorial representation of tracing out fields is short for the notation introduced in \eqref{trrhoblue}, leaving out the explicit integration over the fields for conciseness. 
}, 
\ali{
	\rho_R \equiv \tr_L \rho = \adjincludegraphics[width=2cm,valign=c]{figs/TFDrhoL_a.png} =  \adjincludegraphics[width=2cm,valign=c]{figs/TFDrhothermal.png} = \rho_{thermal}. 
\label{rhoATFD}   
}

Said otherwise, the TFD state is the answer to the question \cite{Laflamme:1988wg}  ``is there a state $|\psi\rangle \equiv |0(\beta)\rangle$  
for which the QFT vacuum expectation value $\langle \psi|\mathcal O|\psi\rangle = \tr(\mathcal O \rho)$ 
reproduces the statistical average $\tr(\mathcal O \rho_{thermal})$ 
of an operator $\mathcal O$?", 
\ali{
	\langle	\psi|\mathcal O|\psi\rangle &= \tr(\mathcal O \rho_{thermal}) \label{TFDstatement1} \\
	\adjincludegraphics[width=2cm,valign=c]{figs/TFDnormO_a.png}\llap{
		\parbox[][1.6cm][b]{2.1cm}{$\mathcal O$}} &=  \adjincludegraphics[width=2cm,valign=c]{figs/TFDnormO_b.png}\llap{
		\parbox[][1.6cm][b]{2.1cm}{$\mathcal O$}}. 
}
Of course, for $\mathcal O$ the identity operator, this is just the statement that 
\ali{
	\langle \psi|\psi \rangle &= \tr(\rho_{thermal}) \label{} \\ 
	\adjincludegraphics[width=2cm,valign=c]{figs/TFDnorm_a.png} &=  \adjincludegraphics[width=2cm,valign=c]{figs/TFDnorm_b.png}.
}
For a compact space dimension,  
\ali{
	|\psi\rangle = \adjincludegraphics[width=2.5cm,valign=c]{figs/psiTorus.png}  \llap{
		\parbox[][1.2cm][b]{.7cm}{{\scriptsize$\tau$}}} \llap{
		\parbox[][2.cm][b]{1.35cm}{{\tiny $\beta/2$}}} \llap{
		\parbox[][-1.3cm][b]{2.2cm}{{\scriptsize$(\phi_L)$}}} \llap{
		\parbox[][-1.5cm][b]{0.8cm}{{\scriptsize$(\phi_R)$}}}   
	\label{TFDpicturecompact}
}
and 
\ali{ 
	\rho_L\equiv \tr_R\rho = \adjincludegraphics[width=2.5cm,valign=c]{figs/rhoLtorus.png} = \adjincludegraphics[width=2.5cm,valign=c]{figs/rhoThermalTorus.png} . 
}

The above statements are given in path integral language  
$|\psi \rangle = \int^\cdot \mathcal D \phi \, e^{-I[\phi]}$ 
or
\ali{
	\psi(\phi_L,\phi_R) = \int^{{\tiny{\begin{array}{ll} \phi(\tau_0)=\phi_L \\ \phi(\tau_0+\beta/2) = \phi_R \end{array}}}} \mathcal D \phi \, e^{-I[\phi]}, 
	\label{psiphiLphiR}
} 
as this gives an immediate visual derivation of \eqref{rhoATFD}-\eqref{TFDstatement1}. It is a simple exercise to derive \eqref{rhoATFD}-\eqref{TFDstatement1} from the explicit formula  
for the TFD state \eqref{TFDpicture} given by  
\ali{
	|\psi \rangle = \sum_n \sqrt{p_n} \,  |E_n\rangle_L |E_n\rangle_R, \qquad p_n = \frac{e^{-\beta E_n}}{Z(\beta)}  
}
with $|E_n\rangle$ the energy eigenstates of $H_L$ and $H_R$. 
It is a pure state per construction (using \eqref{rhoTFD}), and the construction is therefore referred to as purification. That is, $\tr(\rho^2) = \tr \rho$ and hence its von Neumann entropy is zero, while the von Neumann entropy of one copy $S_A = -\tr (\hat \rho_R \log \hat \rho_R)$ 
is the thermal entropy 
\ali{
	S_A = S_{thermal},  
}
where we used the notation $A$ for the region the observer of the `right' theory has access to. The TFD construction thus provides a set-up where the reduced density matrix is thermal \eqref{rhoATFD}, and the entanglement entropy $S_A$ is exactly given by a thermal entropy. 
As discussed in section \ref{sectionThermal}, this can be applied to the calculation of CFT entanglement of an interval $A$ by conformally mapping the interval to one copy of a TFD in Fig \ref{figHolzheypsi}. 

As is apparent from the path integral pictures above, a thermofield double  description arises for Euclidean manifolds that can be divided into two disconnected parts by the specification of \emph{two} values of a periodic coordinate $\tau$ (the values $\tau_0$ and $\tau_0 + \beta/2$ in \eqref{psiphiLphiR}), such as the cylinder or torus. 

Let us discuss two more examples. 
(For more details, see e.g.~\cite{Harlow:2014yka}.)     
The first is the Euclidean disk with metric $ds^2 = dR^2 + R^2 d\tau^2$. In polar coordinates with polar angle $\tau$, one can  
consider the TFD state 
\ali{
	|\psi\rangle = \adjincludegraphics[width=2.5cm,valign=c]{figs/halfdisk.png}  \llap{
		\parbox[][-.2cm][b]{.7cm}{{{\color{taugreen}\scriptsize$\tau$}}}}
}
with 
\ali{
	\tr_L\rho = \adjincludegraphics[width=2cm,valign=c]{figs/halfdiskRhoA.png} 
	= \adjincludegraphics[width=2cm,valign=c]{figs/halfdiskRhoB.png} 
	= \rho_{thermal}   
}
the thermal density matrix encountered in Fig \ref{figHolzheyrhoA}b, for $\beta = 2\pi$ the inverse Unruh temperature \cite{Unruh:1976db} detected by a Rindler observer in Lorentzian signature (Fig \ref{figPenrose}a). 
Euclidean evolution prepares the vacuum state $|\psi\rangle$  
on the Minkowski plane $ds^2 = -dt^2 + dx^2$, at the spacelike slice located at $t=0=t_E$, i.e.~the intersection of zero Lorentzian and Euclidean time $t_E=i t$. 
A Rindler observer is a boosted observer (with acceleration set to one below) who only has access to the positive half-line $x > 0$ at $t=0$, with domain of dependence the Rindler wedge 
\ali{
	ds^2_{Rindler} &= -dt^2 + dx^2 , \qquad (x \geq 0, \, |t|<x) \\ 
	&= dR^2 - R^2 dt_S^2, \qquad (R\geq 0, \, \text{all } t_S) \label{dsRindler}
}
which is covered by Rindler coordinates $(R, t_S)$ or $(x_z, t_S)$ with $x_z = \log R$. These are related to Minkowski coordinates by $x = R \cosh t_S$, $t = R \sinh t_S$. The half-line $A$ ($x >0$) is effectively separated from the half-line $\bar A$ ($x<0$) by the Rindler horizon $R=0$.  

A last example is the cigar manifold, which interpolates between the Euclidean disk geometry (at the tip of the cigar) and the Euclidean cylinder. It appears in the Wick-rotated metric of black hole backgrounds, e.g.~the Schwarzschild black hole in Fig \ref{figPenrose}b. 
The associated TFD state $|\psi\rangle$ provides the Hartle-Hawking state \cite{Hartle:1976tp} of quantum fields on the black hole background, defined by doing the path integral over half the Euclidean geometry, 
and $\tr_L |\psi\rangle \langle \psi| = \rho_{thermal}$ describes a thermal state at the Hawking temperature of the black hole. The prepared state at the intersection of zero Euclidean and Lorentzian time can then be further  evolved in Lorentzian time $t$.  


\paragraph{The holographic dual of the TFD} 

The extended $(2+1)$-dimensional AdS-Schwarzschild black hole or BTZ solution \cite{BTZ-Banados:1992wn} is shown in Fig \ref{figPenrose}c, with a Hartle-Hawking state prepared by evolution over the Poincar\'e disk $ds^2 = 4 dw d\bar w/(1 - w \bar w)^2$. 
This $(2+1)$-dimensional geometry has two asymptotic boundaries where two dual CFT copies live. It is the holographic dual of the TFD state of $(1+1)$-dimensional CFT \cite{Maldacena:2001kr}. Indeed, when the suppressed spacelike coordinate $x_z$ of the CFT is added back to the Euclidean half of the Penrose diagram in Fig \ref{figPenrose}c, the AdS TFD state at the conformal boundary becomes the CFT TFD state of Fig \eqref{TFDpicture} or \eqref{TFDpicturecompact}, depending on whether $x_z$ has an infinite or compact range. 


\begin{figure}
\quad	\includegraphics[scale=0.18]{figs/lorentzA.png} \llap{
		\parbox[][-7.6cm][b]{2.3cm}{$t$}} \llap{
		\parbox[][-3.6cm][b]{4.6cm}{$t_E$}} \llap{
		\parbox[][-4cm][b]{0.15cm}{$x$}}  \llap{
		\parbox[][-2.2cm][b]{3.5cm}{{\color{taugreen}$\tau$}}} \llap{
		\parbox[][-4.5cm][b]{3.1cm}{{\color{tSblue}$t_S$}}} \llap{
		\parbox[][-4.5cm][b]{1.84cm}{{\color{tSblue}$t_S$}}} 
	\includegraphics[scale=0.18]{figs/lorentzB.png} \quad	\includegraphics[scale=0.18]{figs/lorentzC.png}
	\caption{Penrose diagrams of a) Minkowski spacetime, b) the extended 
		Schwarzschild black hole, and c) the extended 
		AdS-Schwarzschild or BTZ black hole, covered by the full range of $(t,x)$ coordinates. In the case of the black holes, $(t,x)$ are the Kruskal coordinates. 
		Superimposed is the Euclidean preparation of the TFD state, which provides the Hartle-Hawking state.  
		The state can then be further evolved in Lorentzian time. 
		To an observer in resp.~the Rindler wedge, the (1-sided) Schwarzschild geometry and the (1-sided) BTZ geometry (all marked in light blue), the state appears thermal. 
		The (Lorentzian) time coordinate $t_S$ that covers these regions is Wick rotated, $t_S = i \tau$, to the periodic coordinate $\tau$ of the thermofield double construction.     
	}
	\label{figPenrose}
\end{figure}



\paragraph{An application of equation \eqref{Sconf}} 

To end this section, we focus on the Minkowski geometry of Fig \ref{figPenrose}a and determine the Lorentzian time $t_b$-dependence of the entanglement of the region $A$ pictured in Fig \ref{figlogcosh}, following Hartman and Maldacena  \cite{Hartman:2013qma}. 
The region consists of the half-line $x_z>0$ in each Rindler wedge, in cylinder coordinates $dx_z^2 - dt_S^2$. 


At time $t_b=0$, the Cauchy slice on which $A$ is defined is simply at $t=0$.  Then, point $P_1$ in the left Rindler wedge has Rindler time coordinate $\tau = 0$ and $x_z=0$, while point $P_2$ in the right Rindler wedge has coordinates $\tau = \beta/2$ and $x_z=0$ or $(z_2,\bar z_2) = (i \beta/2,-i \beta/2)$. Next, the Cauchy slice can be pushed upwards, with time coordinate $t_b$ equal to $t_S$ in the right Rindler wedge and equal to $-t_S$ in the left one. By analytic continuation, the locations of the interval endpoints $\p A$ are 
\ali{
	(z_1, \bar z_1) = (-t_b, t_b), \qquad (z_2, \bar z_2) = (t_b +  \frac{i\beta}{2}, -t_b - \frac{i\beta}{2}).  \label{HartmanMaldacenalocations}
} 
Applying equation \eqref{Sconf} with these values, and with $f(z) = \exp{(2\pi z/\beta)}$ the planar coordinate, one obtains 
\ali{
	S_A = \frac{c}{3} \log \left( \frac{\beta}{\pi \epsilon_z} \cosh(\frac{2\pi}{\beta} t_b) \right) \label{SAlogcosh}
}
for the time-dependence. At large times, the behavior is linear $S_A \sim t_b$. 

This is indeed $S_A = c/3  \log (\Delta x/\epsilon_{x})$ with the relation $x = e^{2\pi x_z/\beta} \cosh (2 \pi\, t_S /\beta)$ between planar and Rindler coordinates evaluated at $x_z=0$ at the interval endpoints, and 
$\epsilon_{x} = \frac{2\pi}{\beta} \epsilon_z$ by equation \eqref{cutoffs} with \eqref{HartmanMaldacenalocations}. 
The holographically dual calculation of $S_A$ is also performed in \cite{Hartman:2013qma}. 


\begin{figure}
    \centering
	\includegraphics[scale=0.18]{figs/figlogcosh.png} \llap{
		\parbox[][-3.6cm][b]{1.15cm}{$t_b$}} \llap{
		\parbox[][-3.6cm][b]{3.15cm}{$t_b$}}
	\caption{Region $A$ in red, with entanglement $S_A$ given in \eqref{SAlogcosh}. } 
	\label{figlogcosh}
\end{figure}
