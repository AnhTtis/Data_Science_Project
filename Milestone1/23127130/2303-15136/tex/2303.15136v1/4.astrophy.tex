\section{Dense Matter Equation of State}\label{sec:astro}
 In terrestrial laboratories, heavy-ion collisions compress nuclear matter to such high densities but inevitably involve high temperatures~\cite{Dexheimer:2020zzs, Fukushima:2020yzx}. In numerical calculations which serve as virtual laboratories, non-perturbative lattice QCD calculations can explore the finite-temperature region of the QCD phase diagram, yet leaves a long-standing challenge in finite-density part due to the inevitable sign-problem (see introductions in Sec.~\ref{sec:4:sign}). Nevertheless, studies dedicated to probe the cold dense nuclear matter properties have benefited from astronomical observations in past decades~\cite{Watts:2016uzu}. Neutron stars~(NSs) serve as cosmic laboratories for the study of neutron-rich nuclear matter, with densities far greater than the nuclear saturation density~($\rho_0\sim0.16~\text{fm}^{-3}$). NS structures~(e.g., mass and radius) are connected with their bulk properties~[e.g., equation of state~(EoS)]. Therefore, to understand QCD matter in such an ultra-high density, low temperature and large proton-neutron imbalance environment, one can infer the physical properties from NS observables, as an inverse problem (see, e.g., Refs.~\cite{Ozel:2016oaf, Baym:2017whm, Baiotti:2019sew, Kojo:2020krb, Lattimer:2021emm} for recent reviews).
 
 The EoS, a relationship of pressure ($p$) and energy density ($\varepsilon$), can be employed to deduce many-body interactions in nuclear matter or the presence of de-confined quarks at high densities~\cite{Fukushima:2013rx,Burgio:2021vgk}. For densities within the range of $n\simeq 1\text{--}2\, n_0$, one can utilize a combination of \textit{ab initio} techniques and the nuclear force derived from Chiral Effective Theory($\chi$EFT)~\cite{Drischler:2021kxf}. 
In the extremely high density region of $n\geq 50\,n_0$, perturbative QCD calculations provide a reliable understanding~\cite{Ghiglieri:2020dpq}. Accordingly, the neutron stars, with densities of up to a few times of the nuclear saturation density, covers the intermediate region($n\simeq 2\text{--}10\,n_0$). Now, rapidly cumulating neutron star observations have opened a new window for extracting the EoS. The conventional measurements like the Shapiro delay provide observations of \textit{massive pulsars}($M>2M_{\odot}$), extending the mass upper limit of the neutron star realized before~\cite{Ozel:2016oaf}. Quiescent low-mass X-ray binary systems(qLMXBs) and thermonuclear burst sources can determine the radius~\cite{Miller:2013tca,Miller:2016pom}. The latest Neutron Star Interior Composition Explorer (NICER; see~\cite{10.1117/12.2056811,gendreau2017searching}) collaboration provides more accurate X-ray spectral-timing measurements of pulsars' \textit{masses}($M$) and \textit{radii}($R$)~\cite{Yunes:2022ldq}. In addition, there are also \textit{compactness} ($M/R$), \textit{moment of inertia}($I$), \textit{quadrupole moment}($Q$) and \textit{tidal deformability}($\lambda$) can be measured. The last three observables can also be connected in $I$-Love-$Q$ formula without knowledge of the inner matter~\cite{Yagi:2013bca}. Observations of long-lived and colliding NSs are growing since the advent of gravitational waves (GWs) and multi-messenger astrophysics~\cite{Yunes:2022ldq}. The GW measurements from LIGO/VIRGO/KAGRA collaborations reveal an increasing number of events~\cite{LIGOScientific:2018mvr,LIGOScientific:2020ibl}, e.g., GW170817~\cite{LIGOScientific:2017vwq,LIGOScientific:2018cki}, GW190425~\cite{LIGOScientific:2020aai}, and GW200105/GW200115~\cite{LIGOScientific:2021qlt}. The former two have been analyzed sufficiently, which sets a tidal deformability boundary for undetermined EoSs~\cite{Yunes:2022ldq,Annala:2021gom}. One can expect that more well-analyzed events will constrain EoSs to a more narrow area with statistical approaches and machine learning techniques. Meanwhile, the astrophysics observations themselves are beneficial from the rapid development of machine learning, but it is not our main topic in this review (one can read related references in Ref.~\cite{2020WDMKD..10.1349F}). 

 
\subsection{Reconstructions from Neutron Star Mass-Radius}\label{subsec:mr}
As mentioned before, the mass-radius($M$-$R$) curve of NSs is strongly dependent on the EoS of its internal dense matter. Its underlying physical law comes from the Einstein equation~\cite{Baym:2017whm,Schaffner-Bielich:2020psc}, which relates the geometric structure of space-time with the distribution of matter within it. In a spherically symmetric body of isotropic matter, the Tolman--Oppenheimer--Volkov (TOV) equation~\cite{Tolman:1939jz,Oppenheimer:1939ne} gives a hydro-static description of the balance between pressure ($p$) and gravity (mass $m$) included in a neutron star with radius ($r$) from the center
\begin{align}
\begin{split}
\frac{\mathrm{d}p}{\mathrm{d}r} =\;&
   -G \frac{m(r) \varepsilon(r)}{r^2}\left(1+\frac{p(r)}{\varepsilon(r)}\right)\left(1+\frac{4 \pi r^3 p(r)}{m(r)}\right)\left(1-\frac{2 G m(r)}{r}\right)^{-1}
   \,, \\
\frac{\mathrm{d}m}{\mathrm{d}r} =\;&
    4\pi r^2 \varepsilon \,,
\end{split}
\label{eq:4:tov}
\end{align}
where $\varepsilon$ is the energy density. The observed mass and radius are set at the surface of neutron stars, as $M$ and $ R$ where the pressure $p(r=R)\simeq 0$. The similar equals mean the boundary of a neutron star will depend on how to define the vacuum pressure. To solve the TOV equations, one needs EoS in each shell of NSs. In principle, EoS depends on two independent variables, i.e., temperature and baryon chemical potential. However, compared with its Fermi temperature (>$10^{12}$K), the dense matter in the NSs is actually cold enough($10^8-10^{10}$K) to be treated as in zero-temperature environments~\cite{Yunes:2022ldq}. Then, there is only one independent thermodynamic degree of freedom, and one may represent the EoS as the relation, $\varepsilon = \varepsilon(p)$, between total energy density $\varepsilon$ and pressure $p$. 

%%%%%%%%%%%%%%%%%%%%%%%%%%%%%%%%%%%%%%%%%%%%%%%%%%%%%%%%%%%%%%%%%%%%%%%%%%%%%%%%%%%%
\begin{figure}[!htbp]
\begin{center}
\includegraphics[width=0.8\textwidth]{figures/fig_4-1-1_TOV_inverse.pdf}
\caption{A flow chart of TOV and its inverse mapping. The red circle represents a compact star, whereas the black arrows represent the gravity and the yellow arrows indicate the pressure inside it.}\label{fig:4:TOV}
\end{center}
\end{figure}
%%%%%%%%%%%%%%%%%%%%%%%%%%%%%%%%%%%%%%%%%%%%%%%%%%%%%%%%%%%%%%%%%%%%%%%%%%%%%%%%%%%%%

\subsubsection{Statistical Inference}\label{subsubsec:infer}
Before inferring the equation of state, it is important to specify the parameterization schemes and the necessary physical constraints. For a realistic EoS which can reflect properties of the dense matter, it should satisfy~\cite{Kojo:2020krb,Krastev:2021reh,Han:2021kjx},
\begin{enumerate}[label= \arabic*)]
	\item the microscopicly stable condition, i.e.,$(dp/d\varepsilon)\geq0$,
	\item the causality condition, i.e., the speed of sound $c_s$ obeys,
        \begin{equation}
            \frac{dp}{d\varepsilon} = \frac{c_s^2}{c^2}<1,
        \end{equation}
        where $c$ is the speed of light in vacuum.
	\item the experimental constraints, e.g., the massive star observations forcing the EoS to produce a neutron star with mass at least $M\sim2M_\odot$, nuclear physics experiments suggesting the low-density part of the EoS.
\end{enumerate}
The first two constraints should serve as a ``hard'' condition for reconstructed EoSs, while the fact such as the massive mass neutron stars(NSs) would be introduced as priors of the Bayesian Inference. In fact, observations of $M>2M_\odot$ NSs, such as measurement of PSR J1614-2230~\cite{Demorest:2010bx} effectively has ruled out too ``soft'' equation of states. A ``soft'' EoS, for which the pressure increases slowly as the energy density increases, leads to smaller maximum masses. In contrast, the pressure-energy density curve of the ``stiff'' EoS has a larger slope, making for a larger maximum mass of NSs. Although the cores of the massive mass NSs could be composed of quark-gluon matter, most non-baryonic EOS models(e.g., hyperons, kaons, pure quark stars, etc.~\cite{Baym:2017whm}) have been ruled out.


Proper parametrization of EoSs means introducing as small as possible bias that is irrelevant to the physical priors~\footnote{There is a fascinating work to explore correlations among different physics model-motivated EoSs  by means of dimensionality reduction algorithms~\cite{Lobato:2022ajs}.}. The common schemes, such as the spectral representation~\cite{Lindblom:2010bb, Lindblom:2022mkr} and the piece-wise polytropic expansion~\cite{Read:2008iy, Ozel:2009da, Steiner:2010fz, Steiner:2012xt,Raithel:2016bux}, have been proved to be useful in inferring EoS. In Ref~\cite{Raithel:2017ity}, Raithel et al. manifest the feasibility of inferring pressures at five density segments from mock NS masses and radii. Although the authors demonstrate the five-polytropic model can infer possible phase transitions within $30\%$ error, it is still limited by the coarse representation. Thus, non-parametric methods, e.g, the Gaussian process(GP)~\cite{Landry:2018prl} and neural networks~\cite{Han:2021kjx, Soma:2022qnv, Soma:2022vbb} have been proposed to avoid the biased outcome due to misspecification~\cite{Han:2022sxt}. In addition to parametric forms, the self-consistent introduction of physical prior knowledge can also improve reconstructions. For instance, the meta-modeling EoS intuitively continue the Taylor expansions beyond the saturation density of symmetric nuclear matter~\cite{Margueron:2017eqc,Margueron:2017lup}. The well-measured nuclear empirical parameters can be used as priors~\cite{Xie:2019sqb}. Besides, parametrizing the speed of sound, $c_s^2$, rather than $\varepsilon$ itself, is beneficial for detecting phase transitions or crossover in dense matter~\cite{Brandes:2022nxa,Altiparmak:2022bke,Ecker:2022xxj,Jiang:2022tps}. Inspired by a similar form in spectral approach~\cite{Lindblom:2010bb}, a useful expression defines an auxiliary variable $\phi = \mathrm{log}(c^2/c^2_s - 1)$, in which the stability and causality conditions can be naturally satisfied~\cite{Landry:2018prl}.


With the parametric EoS, $\varepsilon_\theta(p)$[or speed of sound $c_{s,\theta}(p)$, auxiliary variable $\phi_\theta(p)$], one can infer parameters $\{\theta\}$ from observations following the Bayesian approach described in Sec.~\ref{subsubsec:bi}. The posterior is $P(\theta \mid \text{data})$ which describes the probability of obtaining a particular parametric EoS from a data set. Using Eq.~\eqref{eq:bayesian}, one can rewrite it as,
\begin{equation}
    P(\theta\mid\text{data}) \propto P(\text{data}|\theta)\frac{P(\theta)}{P(\text{data})},
\end{equation}
where $P(\theta) $ and $ P(\text{data})$ are priors on the parameters and observations, respectively. Given a set of EoS parameters $\{\theta\}$, one can derive the likelihood from data sets of observables $O$ as,
\begin{equation}
    P(\text{data}\mid \theta) = \prod_{i=1}^N P(O_i\mid\theta),\label{eq:4:likelihood}
\end{equation}
where $N$ counts the number of observations and the observable $O$ can be total mass-radius observations $(M,R)$ or the observations from multi-messenger measurements. To combine uncorrelated observables from different sources, one can multiply the corresponding likelihood at the right-hand side of Eq.~\eqref{eq:4:likelihood}. Eventually the likelihood can be estimated through $\chi^2$ fitting~\cite{d2003bayesian}
\begin{equation}
    \chi^2(O \mid \theta) = \sum_{i=1}^N\frac{(O_i - \tilde{O}_i(\theta))^2}{\sigma_i^2},
\end{equation}
where $\sigma_i$ is the measurement uncertainty associated with the observable $O_i$, and the prediction $\tilde{O}_i$ is from calculations based on the corresponding $\epsilon_\theta(p)$. Serving as an efficient alternative of maximizing likelihood, minimizing $\chi^2$ can provide an approximation to the posterior $P(\text{data}|\theta)$~\cite{berkson1980minimum}. The MCMC algorithm introduced in Sec.~\ref{subsubsec:bi} is widely applied for optimizing parameters $\{\theta\}$.


In the early works, although observations were limited, the Bayesian Inference of parametric EoSs anchored our basic understanding of dense nuclear matter. In Ref.~\cite{Steiner:2010fz}, Steiner et al. estimated eight parameters of EoSs in four energy density regimes from six NS masses and radii. The determined EoS and symmetry energy around the saturation density are ``soft'', leading to $R_{1.4M_{\odot}}=11-12$ km. The predicted EoS is ``stiff'' at higher densities, leading to a maximum mass of about $1.9-2.2 M_{\odot}$, consistent with the massive pulsar observation announced later~\cite{Demorest:2010bx}. In the following work~\cite{Steiner:2012xt}, the authors extended the data set to 8 NSs, and attempted to combine more constraints from both HICs and quantum MC calculations at relatively low densities. In Figure~\ref{fig:4:mrobs}, \:Ozel and Freire summarized NS observations can be used in Bayesian techniques at that time~\cite{Ozel:2016oaf}.

%%%%%%%%%%%%%%%%%%%%%%%%%%%%%%%%%%%%%%%%%%%%%%%%%%%%%%%%%%%%%%%%%%%%%%%%%%%%%%%%%%%%
\begin{figure}[!htbp]
\begin{center}
\includegraphics[width=0.8\textwidth]{figures/fig_4-1-2_mr_obs.pdf}
\caption{The neutron-star mass and radius constraints (68\% Confidence Level), (left panel) low-mass X-ray binary neutron stars in a quiescent state, and (right panel) neutron stars exhibiting thermonuclear bursts. The mass relations corresponding to various EoSs are depicted using light gray lines. Figures are taken from Ref.~\cite{Ozel:2016oaf}. }\label{fig:4:mrobs}
\end{center}
\end{figure}
%%%%%%%%%%%%%%%%%%%%%%%%%%%%%%%%%%%%%%%%%%%%%%%%%%%%%%%%%%%%%%%%%%%%%%%%%%%%%%%%%%%%%

Since 2017, GW observations have been playing a profound role in the study of dense matter EoSs~\cite{LIGOScientific:2017vwq, LIGOScientific:2018cki, LIGOScientific:2018mvr, LIGOScientific:2020aai, LIGOScientific:2020ibl, LIGOScientific:2021qlt, Bogdanov:2022faf}. In Bayesian Inference, the measurements can be conveniently considered after marginalizing with multiple-source observations. They will be discussed in the next section. Moreover, the massive mass NSs (PSR J1614-2230~\cite{2010Natur.467.1081D}, PSR J0348+0432~\cite{2013Sci...340..448A}, and PSR J0740+6620~\cite{2020NatAs...4...72C}) set a baseline in inference when marginalizing over the mass measurement by taking into account the measurement uncertainty. The measurements of the NS moment of inertia, e.g., PSR J0737-3039~\cite{Landry:2018jyg, Miao:2021gmf} can also serve as a constraint in inference. Because mass and radius are jointly determined by the inner matter EoSs of NSs, more accurate $M$-$R$ measurements can provide more precise constraints in inference. The NASA X-ray timing mission (NICER) which is currently in operation, has produced $R$ and $M$ measurements of a few of the radio millisecond pulsars that produce thermal radiation (J0030+0451~\cite{Miller:2019cac, Riley:2019yda}, J0740+6620~\cite{Miller:2021qha, Riley:2021pdl}).

Besides these observations, for improving the inference, another recent development is the application of machine learning in representing EoSs. For instance, Gaussian process is a non-parametric method~\cite{Landry:2018prl,Essick:2019ldf} can be used to represent the auxiliary variable $\phi$ at each pressure $p$ as a multivariate normal distribution, $\phi\sim\mathcal{N}(\mu(p_i),K(p_i,p_j))$, where $K$ is a kernel function for approximating the covariance. In priors composed of seven well-established EoS models, the GP process is implemented to estimate the posterior $P(\text{EoS}|\text{data})$ using the Markov Chain Monte Carlo (MCMC) algorithm. The alternative method introduced in Ref.~\cite{Han:2021kjx} is a shallow neural network representation of EoS that can also handle uncertainties from observations with MCMC while preserving flexibility. To alleviate the difficulty of sampling in a high dimensional space of parameters, the authors develop a variational auto-encoder(VAE) assisted framework for reducing the number of parameters in representing EoS~\cite{Han:2022sxt}. 



\subsubsection{Deep Learning Inverse Mapping}\label{subsubsec:dlim}
As shown in Fig.~\ref{fig:4:TOV}, the mapping from the $M$-$R$ curve to the EoS relationship in the TOV equations is a one-to-one correspondence~\cite{1992ApJ...398..569L}, which sets up a well-defined inverse problem when the observations are sufficient, i.e., constructing the EoS from a continuous $M$-$R$ curve. The task becomes natural to build deep neural networks for constructing the complicated mapping from observed data to indeterminate physical variables~\cite{pang:2020deep,Boehnlein:2021eym}. 

%%%%%%%%%%%%%%%%%%%%%%%%%%%%%%%%%%%%%%%%%%%%%%%%%%%%%%%%
\begin{figure}[!htbp]
\begin{center}
\includegraphics[width=0.7\textwidth]{figures/fig_4-1-3_dl.pdf}
\caption{Schematic flowchart of data generation procedure for deep learning the inverse mapping, in which the step(3) introduces the data augmentation. Figures from Ref.~\cite{Fujimoto:2021zas} with permission.}\label{fig:data_gen_dl}
\end{center}
\end{figure}
%%%%%%%%%%%%%%%%%%%%%%%%%%%%%%%%%%%%%%%%%%%%%%%%%%%%%%%%%%%

Fujimoto et al.~\cite{Fujimoto:2017cdo,Fujimoto:2019hxv} develop a supervised learning method to constrain the nuclear matter EoS, where a piecewise polytropic expansion is used to represent the EoS. The authors take the squared sound speed $c^2_s$ at the corresponding pressure as the output of the network, and the mass, radius, and their variances as the input, ($M_i,R_i;\sigma_{M,i},\sigma_{R,i}$). Thus, one can use the NS observations to obtain the parameters of the muclear matter EoS via the trained network. After a mock observation validation in the preliminary proof-of-concept study~\cite{Fujimoto:2017cdo}, the authors then use the mapping to construct the EoS from $M$-$R$ distributions of 14 observed neutron stars~\cite{Fujimoto:2019hxv}. In a more comprehensive work~\cite{Fujimoto:2021zas}, the authors introduce data augmentation to improve the performance and compared the deep learning approach with polynomial regression, proving the robustness of their DNN method. The generation procedures of training data are shown in Figure~\ref{fig:data_gen_dl}, the new data augmentation strategy was introduced by assigning random uncertainty parameters($\sigma_{M,i},\sigma_{R,i}$) for sampled 14 points from solved $M$-$R$ curve. To build the uncertainty for the reconstruction, they further proposed the validation method and the ensemble method, but it is still a statistical estimation to the real uncertainties.

In Ref.~\cite{Morawski:2020izm}, Morawski and Bejger further use a 4 hidden layer deep neural network to reconstruct EoSs from masses, radii, and tidal deformabilities. This model can also construct EoS and produce a $M$-$R$ curve that closely matches the observations within a range of approximately $1$--$7$ times the nuclear saturation density. They demonstrate the effectiveness of this method by applying it to mock data generated from a randomly selected polytropic EoS, achieving reasonable accuracy with only 11 mock $M$-$R$ pairs, similar to the current number of actual observations. They also validate the approach on mock data containing realistic EoSs. Different from Ref.~\cite{Fujimoto:2019hxv}, the authors also study how the NS radii can be determined using only the GW observations of tidal deformability. In their recent work~\cite{Morawski:2022aud}, the problem of recognizing phase transitions is transformed into an anomaly detection task. The algorithm of normalizing flow, trained on samples of observations without phase transition signatures, interprets a phase transition sample as an anomaly. Although the results are inconclusive due to limited observations and large errors, it offers an alternative way to detect phase transitions may happen in dense matter.

In Ref.~\cite{Traversi:2020dho}, Traversi and Char compare two methods for estimating the quark matter EoS: Bayesian Inference and Deep Learning. These methods are applied to study the constant speed of sound EoS and the structure of quark stars within the two-family scenario. The observations include mass and radius estimates from various X-ray sources, and mass and tidal deformability measurements from gravitational wave events. The results from both methods are consistent, and the predicted speed of sound is in agreement with the conformal limit. 

All the works discussed in this subsubsection fall under the category of supervised learning, where the goal is to learn the inverse mapping from a large ensemble of training data. Well-trained deep neural networks can successfully approximate this mapping when the generated dataset is diverse and of sufficient size. This approach provides a new perspective for constructing EoS from mass-radius pairs, and can be extended to other observations as well. Meanwhile, it unavoidably requires preparing training data and faces difficulties in obtaining uncertainty from finite noisy observations. In the next section, a physics-driven approach will be introduced to address these challenges.

\subsubsection{Gradient-Based Inference}\label{subsubsec:ad}
%%%%%%%%%%%%%%%%%%%%%%%%%%%%%%%%%%%%%%%%%%%%%%%%%%%%%%%%
\begin{figure}[!htbp]
\begin{center}
\includegraphics[width=0.8\textwidth]{figures/fig_4-1-4_AD_flow_chart.pdf}
\caption{A flow chart of AD methods, with (a) the EoS represented by neural networks named as \texttt{EoS Network}. Note that in (b) the \texttt{TOV-Solver} is a well-trained and static network.}\label{fig:framework}
\end{center}
\end{figure}
%%%%%%%%%%%%%%%%%%%%%%%%%%%%%%%%%%%%%%%%%%%%%%%%%%%%%%%%%%%
In recent works~\cite{Soma:2022qnv,Soma:2022vbb}, Shriya et al. propose an automatic differentiation (AD) framework to reconstruct the EoS from finite observations. A sketch of the framework is shown in Fig.~\ref{fig:framework}. It consists of two differentiable modules: (a) the \texttt{EoS Network}, $p_{\theta}(\mathbf{\rho})$, an unbiased and flexible DNN parametrization of pressure as function of baryon number density; and (b) the \texttt{TOV Solver}, a DNN for translating any given EoS into its corresponding $M$-$R$ curve. The latter is an emulator for solving TOV equations. With a well-trained \texttt{TOV-Solver} network, the \texttt{EoS Network} can be optimized in an unsupervised manner. Given $N_{\text{obs}}$ number of NS observations, we train the \texttt{EoS Network} to fit the pairwise ($M$, $R$) predictions from the pipeline (\texttt{EoS Network} $+$ \texttt{TOV-Solver}) to the observations.

A gradient-based algorithm within the AD framework is deployed to minimize the loss function, $\mathcal{L}\equiv\chi^2$, which is expressed as
\begin{equation}
\chi^2 = \sum_{i=1}^{N_{\text{obs}}} \frac{(M_{i} - M_{\text{obs},i})^2}{\Delta M_i^2}
+    \frac{(R_{i} - R_{\text{obs},i})^2}{\Delta R_i^2}. \label{eq:chi2}
\end{equation}
Here ($M_{i},R_i$) represents the output of the \texttt{TOV-Solver}, and ($M_{\text{obs},i},R_{\text{obs},i}$) are observations which have an uncertainty ($\Delta M_{i},\Delta R_i$). With a static well-trained \texttt{TOV-Solver} network, the gradients of the loss with respect to parameters of the \texttt{EoS Network} are
\begin{equation}
\frac{\partial\chi^2}{\partial \theta} = \sum_{i=1}^{N_{\text{obs}}}\int
\bigg[\frac{\partial\chi^2}{\partial M_i} \frac{\delta M_i}{\delta p_{\theta}(\rho)}
+\frac{\partial\chi^2}{\partial R_i} \frac{\delta R_i}{\delta p_{\theta}(\rho)}\bigg]
\frac{\partial p_{\theta}(\rho)}{\partial \theta} \mathrm{d}\rho,
\label{eq:ad}
\end{equation}
where the \texttt{TOV-Solver} is a functional mapping $f: p_{\theta}(\rho) \rightarrow {(M_i, R_i)}$. The partial/functional derivatives ${\partial p_{\theta}}/{\partial \theta}$, ${\delta M_i}/{\delta p_{\theta}(\rho)}$, and ${\delta R_i}/{\delta p_{\theta}(\rho)}$ can be directly computed with a back-propagation algorithm~\cite{baydin2018automatic} within the AD framework. Optimizing the parameters of the \texttt{EoS Network} can obtain the best fit to the finite and noisy observational $M$-$R$ data from the well-prepared \texttt{TOV-Solver}~\cite{Soma:2022qnv}.



\textbf{Uncertainty estimation}. To evaluate the reconstruction uncertainty, one can adopt a Bayesian perspective and focus on the posterior distribution of EoSs given the astrophysical observations, $\text{Posterior}(\boldsymbol{\theta}_{\text{EoS}}|\text{data})$. In the computations, the authors first draw an ensemble of $M$-$R$ samples from the fitted Gaussian distribution for real observations (see details in Ref.~\cite{Soma:2022vbb}). From this ensemble, one can infer the corresponding EoS deterministically with maximum likelihood estimation. Given the ensemble of reconstructed EoSs, one can then apply the importance sampling technique to estimate the uncertainty associated with the desired posterior distribution, where an appropriate weight is assigned to each EoS. In general, a physical variable $\hat{O}$ can be estimated as
\begin{equation}
    \bar{O} = \langle \hat{O} \rangle = \sum_j^{N} w^{(j)} O^{(j)},
\end{equation}
with the standard deviation can also be estimated as $(\Delta O)^2 = \langle \hat{O}^2 \rangle - \bar{O}^2$. The weights are
\begin{equation}
   w^{(j)} = \frac{\text{Posterior}(\boldsymbol{\theta}^{(j)}_{\text{EoS}}|\text{data})}{\text{Proposal}(\boldsymbol{\theta}^{(j)}_{\text{EoS}})}  \propto \frac{P(\text{data}|\boldsymbol{\theta}^{(j)}_{\text{EoS}})\; \text{Prior}(\boldsymbol{\theta}^{(j)}_{\text{EoS}})}{P(\boldsymbol{\theta}^{(j)}_{\text{EoS}}|\text{samples}^{(j)})\; P(\text{samples}^{(j)}|\text{data})\; \text{Prior}(\text{data})},
\end{equation}
where $j$ indicates index of samples and $i$ indicates index of $M$-$R$ observables. $P(\boldsymbol{\theta}^{(j)}_{\text{EoS}}|\text{samples}^{(j)})=1$, because it is deterministic. $P(\text{data}|\boldsymbol{\theta}^{(j)}_{\text{EoS}})\propto \exp{(-\chi^2(M_{\boldsymbol{\theta}^{(j)}_{\text{EoS}}},R_{\boldsymbol{\theta}^{(j)}_{\text{EoS}}})/2})$ and $P(\text{samples}|\text{data})=\mathcal{N}(M_{\text{obs}},\Delta{M}^2)\mathcal{N}(R_{\text{obs}},\Delta{R}^2)$ calculated from the corresponding Gaussian distribution. In practical, weights should be normalized so that $1 = \sum_j w^{(j)}$ and cut off to avoid outliers in samples, which will also remove the prior terms.
%%%%%%%%%%%%%%%%%%%%%%%%%%%%%%%%%%%%%%%%%%%%%%%%%%%%%%%%%%%%%%%%%%%%%%%%%%%%%%%%%%%%
\begin{figure}[htbp!]
    \centering
    \begin{minipage}[t]{0.48\textwidth}
        \centering
        \includegraphics[width=8.cm]{figures/fig_4-1-5_EOS.pdf}
        \caption{The EoS reconstruted from observational data. The gray band locates the $\chi$EFT prediction. The red shaded area represents the automatic differentiaion results (AD~\cite{Soma:2022vbb}). Other results are derived from Bayesian methods (AJ.765,L5~\cite{Steiner:2012xt} and ARAA.54,401~\cite{Ozel:2016oaf}) and the direct inverse mapping (PRD.101,054016~\cite{Fujimoto:2019hxv}) are also included.}
        \label{fig:rec-eos}
    \end{minipage}
    \hspace{0.5cm}
    \begin{minipage}[t]{0.48\textwidth}
        \centering
        \includegraphics[width=8.1cm]{figures/fig_4-1-6_MR.pdf}
        \caption{$M$-$R$ contour corresponding to the reconstructed EoSs in the left panel. The dots with uncertainties are fitting observations summarized in Ref.~\cite{Soma:2022vbb}, in which black dots are from NICER observations.}
        \label{fig:rec-mr}
    \end{minipage}
\end{figure}
%%%%%%%%%%%%%%%%%%%%%%%%%%%%%%%%%%%%%%%%%%%%%%%%%%%%%%%%%%%%%%%%%%%%%%%%%%%%%%%%%%%%

\textbf{Reconstructions}. In Fig.~\ref{fig:rec-eos}, reconstructed EoSs from different works are plotted for comparison. The $\chi$EFT calculation combined with polytropic extrapolation and the two-solar-mass pulsar constraint is labeled as ``$\chi$EFT+Astro'', showing as a gray band. It is expected that any reasonable predictions should fall within this gray band, since this first-principle calculation sets a theoretical baseline. Additionally, the Bayesian analyses (noted as ``ARAA,54,401'' of \:Ozel et al.~\cite{Ozel:2016oaf} and ``AJ.765,L5'' of Steiner et al.~\cite{Steiner:2012xt}) and the supervised learning result (labelled as ``PRD.101,054016'' of Fujimoto et al.~\cite{Fujimoto:2019hxv}) are also shown in the figure. It is worth noting that \cite{Ozel:2016oaf} and \cite{Fujimoto:2019hxv} use the same data set including eight neutron stars in quiescent low-mass X-ray binaries (qLMXB) and six thermonuclear bursters, while \cite{Steiner:2012xt} uses a subset of the data, i.e., eight of X-ray sources. The AD results (labelled as ``AD'' of Shriya et al.~\cite{Soma:2022vbb}) are presented as red band, which includes additional four observations: 4U 1702-429~\cite{Nattila:2017wtj}, PSR J0437–4715~\cite{Bogdanov:2019ixe}, J0030+0451~\cite{Miller:2019cac,Riley:2019yda} and J0740+6620~\cite{Miller:2021qha,Riley:2021pdl}, the three pulsars are from the latest NICER analysis. Figure~\ref{fig:rec-mr} shows the corresponding $M$-$R$ curves derived from the EoSs in Fig.~\ref{fig:rec-eos}. The newest reconstructions certainly consistent with massive neutron stars, i.e., $M\geq2M_\odot$.


\subsection{Constraints from Multimessenger Observations}\label{subsec:gw}
Multimessenger measurements have provided new insights into ultrahigh-density matter, including gravitational waves (GWs) and electromagnetic (EM) signals. This new era began with the discovery of a binary neutron star merger by the Advanced LIGO and Virgo gravitational wave detectors in 2017~\cite{LIGOScientific:2017vwq}.  Because the properties of neutron stars are sensitive to the microscopic interactions that govern the EoS, gravitational waves from the mergers of neutron stars can carry information about the inner cold dense matter. The EoS of dense matter can further be strongly constrained by gravitational wave observations of neutron stars.

\textbf{Tidal deformability.} The finite size of the stars gives rise to tidal interactions that affect the evolution of the binary system during the late stages of the inspiral of coalescing NSs~\cite{Schaffner-Bielich:2020psc}. The tidal field produced by each binary component induces a quadrupole moment on the companion star, resulting in enhanced emission of gravitational radiation and a slight increase in their relative acceleration, accelerating the inspiral phase. The magnitude of the induced quadrupole moment is related to the internal structure of the NS. Larger stars are less compact and thus more easily deformable under the influence of an external field of a given amplitude. The tidal deformability $\lambda$ is a parameter for the quantification of the effects~\cite{Flanagan:2007ix,Hinderer:2007mb}. Dimensionless tidal deformability $\Lambda$ is defined as the ratio of the induced quadrupole moment to the external perturbing tidal field, $\Lambda = \lambda/M^5 = {2k_2}/{(3\beta^5)}$, where $k_2$ is tidal Love number which is determined as 
\begin{align}
     k_2\left(\beta, y_R\right) &=\frac{8}{5} \beta^5(1-2 \beta)^2\left[2-y_R+2 \beta\left(y_R-1\right)\right] \times\left\{2 \beta\left[6-3 y_R+3 \beta\left(5 y_R-8\right)\right]\right. \nonumber \\
     &+4 \beta^3\left[13-11 y_R+\beta\left(3 y_R-2\right)+2 \beta^2\left(1+y_R\right)\right.] + 3(1-2 \beta)^2\left[2-y_R+2 \beta\left(y_R-1\right)\right] \ln (1-2 \beta)\}^{-1},
\end{align}
where $\beta\equiv M/R$ is dimensionless compactness parameter. $y_R \equiv y(R)$ is the boundary value of $y(r)$, which follows the differential equation
\begin{equation}
    \frac{d y(r)}{d r}=-\frac{y(r)^2}{r}-\frac{y(r)}{r} F(r)-r Q(r),
\end{equation}
with
\begin{align}
    F(r) &=\left\{1-4 \pi r^2[\varepsilon(r)-p(r)]\right\}\left[1-\frac{2 m(r)}{r}\right]^{-1}, \\ 
    Q(r) &=4 \pi\left[5 \varepsilon(r)+9 p(r)+\frac{\varepsilon(r)+p(r)}{c_s^2(r)}-\frac{6}{r^2}\right]\left[1-\frac{2 m(r)}{r}\right]^{-1} - \frac{4 m^2(r)}{r^4}\left[1+\frac{4 \pi r^3 p(r)}{m(r)}\right]^2\left[1-\frac{2 m(r)}{r}\right]^{-2}.
\end{align}
The total tidal effect of two compact stars in an inspiraling binary system is given by the mass-weighted (dimensionless) tidal deformability,
\begin{equation}
    \tilde{\Lambda}=\frac{16}{13}\frac{(M_1 + 12M_2)M_1^4\Lambda_1+(M_2+12M_1)M_2^4\Lambda_2}{(M_1+M_2)^5},
\end{equation}
where the subscripts for $\Lambda$ and $M$ indicate different stars. For single neutron stars, one can calculate their $\Lambda$, predicting the tidal phase contribution for a given binary system from each EOS, for the universality of the nuclear matter EoS. The weighted (dimensionless) deformability $\tilde{\Lambda}$ is usually plotted as a function of the chirp mass $\mathcal{M}=(M_1M_2)^{3/5}/(M_1+M_2)^{1/5}$ for asymemetric mass binary systems. See a more systematic introduction in Ref.~\cite{Schaffner-Bielich:2020psc}.

%%%%%%%%%%%%%%%%%%%%%%%%%%%%%%%%%%%%%%%%%%%%%%%%%%%%%%%%
\begin{figure}[!htbp]
\begin{center}
\includegraphics[width=0.9\textwidth]{figures/fig_4-1-6_GP.pdf}
\caption{In Ref.~\cite{Landry:2020vaw}, the authors presented individual-event constraints from various observation classes. The 90\% credible intervals for the EoS are displayed in the pressure-density plane (left panel) and the mass-radius plane (right panel). The prior range is indicated by black lines, and the posterior using only heavy pulsar measurements is represented by turquoise lines. The posterior based on only GW data is depicted by green shaded regions, and the posterior using only NICER data is depicted by blue shaded regions. The left panel includes vertical lines marking multiples of the nuclear saturation density, while the right panel includes horizontal shaded regions showing the 68\% credible mass estimate for the two heaviest known pulsars. Figures are taken from Ref.~\cite{Landry:2020vaw}.}\label{fig:5:gp}
\end{center}
\end{figure}
%%%%%%%%%%%%%%%%%%%%%%%%%%%%%%%%%%%%%%%%%%%%%%%%%%%%%%%%%%%
In terms of Bayesian Inference (BI) methods, there are many works using or combining the current GW observations to reconstruct the dense matter EoS~\cite{Brandes:2022nxa,Ecker:2022xxj,Miao:2021nuq,Golomb:2021tll,Chimanski:2022wzi,Chimanski:2022wzi}. As a modern variant of BI methods, the Gaussian Process (GP) has been used to represent the EoS by inferring it from GW observations in the work of Landry et al.~\cite{Landry:2018prl}. The authors first build a mapping from binary observables $(M_1,M_2,\Lambda_1,\Lambda_2)$ to the GP-represented EoSs and then use MCMC to infer their corresponding posteriors directly. These synthetic EOSs keep a high flexibility that means covering a large enough interval of stiffnesses and core pressures while satisfying other characteristics introduced in Sec.~\ref{subsubsec:infer}. The authors validate the inference approach on simulated GW170817-like signals using real detector noise, with reasonable efficiency in recovering a known EOS. Eventually, the authors demonstrate the inferred EOSs and corresponding observables under posterior constraints from the single GW170817 event. In two prior set-ups, the deformability of $\tilde{\Lambda}=210^{+383}_{-113}(631^{+164}_{-122})$ and $\Lambda_{1.4}=160^{+448}_{-113}(556^{+163}_{-172})$, maximum mass of $M_{\text{max}}=2.09^{+0.37}_{-0.16}(2.04^{+0.22}_{-0.002}) M_{\odot}$ are consistent with previous analyses~\cite{LIGOScientific:2018cki,De:2018uhw}. In the following works, Essick et al.~\cite{Essick:2019ldf} extend the analysis, obtaining a more reliable conclusion that is the GW170817 favors a ``softer'' EoS. Through combining more information from other observations, i.e., maximum masses of NSs, more GW events shown in Table~\ref{tab:4:GW} and $M$-$R$ in the newest X-ray observation (J0030+0451). Landry et al.~\cite{Landry:2020vaw} further improve the inference by combing different observations. As a demonstration, individual constraints from different observations can be found in Fig.~\ref{fig:5:gp}. Through combining them, the authors find the radius of the $1.4 M_{\odot}$ NS is $R_{1.4}=12.32^{+1.09}_{-1.47}$~km.

%%%%%%%%%%%%%%%%%%%%%%%%%%%%%%%%%%%%%%%%%%%%%%%%%%%%%%%%%%%%%%%%%%%%%
\begin{table}[!hbpt]
\centering
\begin{tabular}{llll}
\hline\hline
    Events & $\mathcal{M}[M_\odot]$ & $\tilde{\Lambda}$ & $\Lambda_{1.4}$ \\
\hline
    GW170817 & 
    $1.186^{+0.001}_{-0.001}$~\cite{LIGOScientific:2017vwq}&
    $300^{+500}_{-190}$~\cite{LIGOScientific:2018hze}&
    $190^{+390}_{-120}$~\cite{LIGOScientific:2018cki}\\
    GW190425~\cite{LIGOScientific:2020aai} & 
    $1.44^{+0.02}_{-0.02}$&
    $\lesssim 600$&
    ---\\
\hline\hline
\end{tabular}
\caption{Summary of the GW events and their astrophysical observations. It 
 contains the median and uncertainties (90\% credible level) of  the chirp mass M, and the tidal parameters $\tilde{\Lambda}$ and its corresponding value $\Lambda_{1.4}$ at 1.4 solar mass.}
\label{tab:4:GW}
\end{table}

%%%%%%%%%%%%%%%%%%%%%%%%%%%%%%%%%%%%%%%%%%%%%%%%%%%%%%%%%%%%%%%%%%%%%

As introduced in Sec.~\ref{subsubsec:dlim}, Morawsk et al. implement deep learning to rebuild EoSs from GW observations~\cite{Morawski:2020izm, Morawski:2022aud}, in which the learnable inverse mapping consists of mass-radius and tidal deformability as inputs. In Ref~\cite{Han:2021kjx, Han:2022sxt}, the mass-tidal-deformability data of GW170817 event has also contributed to the inference of neural network represented EoSs. In the work of Ferreira et al.~\cite{Ferreira:2019bny}, the authors prepare training data from a set of metamodeling EoSs. The inverse mapping is learned to predict the parameters of the nuclear matter from the tidal deformability and the radius of the NS. In addition to DNNs, they adopt a classical machine learning method of support vector machine regression (SVM-R). Although the DNNs show a high level of accuracy than the SVM-R in their tests, it provides an alternative way to explore the possible inverse mapping. For classical machine learning algorithms, in Refs.~\cite{PhysRevD.100.103009, HernandezVivanco:2020cyp}, Hernandez et al. develop a random forest regressor algorithm to interpolate the marginalized likelihood for each gravitational-wave observation. It can be utilized in the hierarchical Bayesian Inference for constraining the EoS from GW170817 and GW190425 events directly. They provide a new constraint for the 1.4~$M_\odot$ neutron star with radius $R=11.6_{-0.9}^{+1.6}$~km. In addition, the AD framework introduced in Sec.~\ref{subsubsec:ad} can conceivably be implemented with the above differentiable formula to combine the observation of tidal deformability with the stellar structures for improving the reconstruction.


\textbf{Multi-messager measurements.} It is also feasible to explore the nuclear matter EoS from multi-messenger measurements directly with modern deep learning techniques. In Ref.~\cite{Goncalves:2022smd}, Goncalves et al. examine the Audio Spectrogram Transformer (AST) model for analyzing gravitational-wave data. The AST machine-learning model is a convolution-free classifier that captures long-range global dependencies through a purely attention-based mechanism. In this work, the model is applied to a simulated dataset of inspiral GW signals from binary neutron star coalescences built from five EoSs. It is shown that the model can correctly classify the EOS purely from the gravitational-wave spectra. Additionally, the generalization ability of the machine is investigated in testing data set. Overall, the results suggest that the well-trained attention-based model can infer the cold nuclear matter EOS directly from the GW signals using the simplified setup of noise-free waveforms. In work of Farrell et al.~\cite{Farrell:2022lfd}, the Transformer model is implemented to deduce EoSs directly from X-ray spectra of neutron stars. The approach can determine the EoS by analyzing the high-dimensional spectra of observed stars, bypassing the need for intermediate mass-radius calculations. They demonstrate the end-to-end machine slightly but consistently perform better than the two-step method using the intermediate $M$-$R$ information. The network used in this approach takes into account the sources of uncertainty for each star, enabling a comprehensive propagation of uncertainties to the EOS. These attempts enlighten us that in the future, using deep learning techniques to process more measurements~\cite{Bogdanov:2022faf}(e.g., short gamma-ray burst~\cite{DAvanzo:2015kdp} and subsequent kilonova~\cite{Cowperthwaite:2017dyu}) will help us to eventually obtain a concrete understanding of the dense matter EoS~\cite{Huang:2022mqp,Fujimoto:2022xhv}.


\subsection{Constraints from Nuclear Physics}

\subsubsection{Connections with Nuclear Symmetry Energy}
One important aspect of the nuclear matter EoS is the symmetry energy, $E_\text{sym}(\rho)$, which represents the energy cost of converting neutrons into protons or vice versa. This quantity can affect many properties of nuclei, including the stability of nuclei and nuclear processes~\cite{Baldo:2016jhp}. The nucleonic component of the nuclear matter EoS can be expressed in terms of the energy per nucleon $\rho$ as 
\begin{equation}
    E(\rho, A) =  E(\rho,0) + E_\text{sym}(\rho)A^2,
\end{equation}
where $ E(\rho,0)\equiv E_{SNM}(\rho)$ is the energy per nucleon of symmetric nuclear matter(SNM) and $A = (\rho_n - \rho_p)/\rho$ is the isospin asymmetry with $\rho_{n/p}$ being the neutron (proton) density. Thus, as a part of the EoS, the symmetry energy $E_\text{sym}(\rho)$ can also be constrained by studying neutron star observations~\cite{Xie:2019sqb,Li:2021thg}. Around saturation density $\rho_0$, $E_{SNM}(\rho)$ can be expanded as $E_0 + K_0 x^2/2 +  J_0 x^3/6$ with $x\equiv (\rho - \rho_0)/(3\rho_0)$. $E_0, K_0$ and $ J_0$ are the binding energy, incompressibility, and skewness of symmetric nuclear matter; the symmetry energy can be expanded as $E_\text{sym}(\rho) = S_0 + L x + K_\text{sym} x^2/2 +  J_\text{sym} x^3/6$, where $S_0, L, K_\text{sym}$ and $ J_\text{sym}$ are the magnitude, slope, curvature, and skewness of the symmetry energy at $\rho_0$. These physical parameters can be determined from NS observables~\cite{Zhang:2018vrx}. This information can be further used to improve our understanding of how nuclear matter behaves under extreme conditions and to test nuclear models~\cite{Li:2019xxz}. In Ref.~\cite{Anil:2020lch}, the authors apply a machine learning approach of support vector machines(SVMs) to predict binding energies of nuclei which are highly related to the outer crust of neutron stars in the density range $\lesssim 10^{-3} \rho_0$. In a series of works from Li et al.~\cite{Xie:2019sqb, Zhang:2018vrx, Xie:2020rwg}, they have developed the Bayesian Inference approach to infer, e.g., EoS, phase transitions, symmetry energy and empirical parameters, from observations of neutron stars together with nuclear experiments. See a recent review in Ref.~\cite{Li:2021thg}.

In Ref.~\cite{Krastev:2021reh}, Krastev trains a deep neural network(DNN) to determine the nuclear symmetry energy as a function of density directly from observational neutron star data. The author demonstrates that DNNs can accurately reconstruct the nuclear symmetry energy from a set of available observables of the neutron star, i.e., the masses and the tidal deformabilities. In Ref.~\cite{Ferreira:2022nwh}, Ferreira et al. also use DNNs to analyze the relationship between the cold $\beta$-equilibrium matter of NSs and the properties of nuclear matter, which is a non-trivial mapping~\cite{Mondal:2021vzt}. Using a Taylor expansion of the energy per particle of homogeneous nuclear matter, they generate a data set of different $\beta$-equilibrium neutron star matter scenarios and their corresponding nuclear matter properties. The neural network is trained and achieved high accuracy on the test set. In a real case scenario, they test the neural network on 33 nuclear models and are able to accurately recover the nuclear matter parameters with reasonable standard deviations for both the symmetry energy slope and the nuclear matter incompressibility at saturation. In the work of Thete et al.~\cite{Thete:2022eif}, a DNN is trained to predict EoSs from nuclear matter saturation parameters, then used as a static module to infer a set of seven nuclear parameters from astrophysical observations.



\subsubsection{Constraints from $\chi$EFT, HICs, and pQCD}

In addition to using observations from astrophysics, one can improve the performance of the reconstruction by combining more physical priors from other domains of nuclear physics, i.e., relativistic mean-field(RMF) calculations at low-density region~\cite{Serot:1984ey,Oertel:2016bki}, heavy-ion collisions (HICs) at mid-density region~\cite{Fukushima:2020yzx,Dexheimer:2020zzs,Huth:2021bsp,Sorensen:2023zkk} and perturbative calculations at nuclear matter and quark matter regions respectively~\cite{Drischler:2021kxf,Ghiglieri:2020dpq}. They have been treated as straightforward physics constraints for reducing the dense matter EoS ~\cite{Annala:2021gom,Demircik:2021zll,Shirke:2022tta}.

Chiral effective field theory ($\chi$EFT) is a widely used framework for studying the properties of nuclear matter at moderate densities (see recent reviews ~\cite{Sammarruca:2019ncy,Drischler:2019xuo,Drischler:2021kxf}). Recent advances in $\chi$EFT have led to a computationally efficient tool for determining the properties of nuclear matter at densities up to $\sim 2\,n_0$. In $\chi$EFT, the effective strong interaction \textit{degree of freedom}s are nucleons and pions (and delta isobars), presenting throughout most of the neutron star interior.  The effective Lagrangian is formed in an expansion in powers of the momenta of the hadrons and the quark masses, which are the small parameters in the theory. A new framework is introduced for quantifying the correlated uncertainties of the nuclear matter EoS that is derived from $\chi$EFT~\cite{Drischler:2020hwi,Drischler:2020yad}. This framework uses Gaussian processes with physics-based \textit{multitask} design to efficiently identify theoretical uncertainties from $\chi$EFT truncation errors to derived quantities, e.g., pressure, sound velocity, symmetry energy and its slope. This approach has been applied to the calculations with nucleon-nucleon and three-nucleon interactions up to fourth order in the expansion. At nuclear saturation density, the predicted symmetry energy and its slope are in agreement with various experimental constraints. This work has provided a statistically robust uncertainty estimate in the low density region, which can contribute as a reliable constraint.


The RMF approach uses a relativistic Lagrangian written in terms of baryon and meson fields with simplicity and flexibility~\cite{Oertel:2016bki}. The in-medium coupling constants are chosen to reproduce nuclear physics measurement around $n\simeq n_0$.
To determine the EOS based on a relativistic approach with minimal constraints, Malik et al.~\cite{Malik:2022zol} use the Bayesian approach on a set of models based on the RMF framework with density-dependent coupling parameters and no nonlinear mesonic terms. These models are constrained by the EoS derived from $\chi$EFT and four saturation properties of nuclear matter: density, binding energy per particle, incompressibility, and symmetry energy. The authors have verified that the derived posterior distribution of NS maximum masses, radii, and tidal deformabilities are consistent with recent NS observations. In related works from Li et al.~\cite{Zhu:2022ibs,Sun:2022yor}, the parametric interactions of RMF models are inferred from astrophysical observations. This is also a promising attempt to understand the nuclear matter from hadron degree of freedom.

Heavy-ion collisions involve the collision of heavy atomic nuclei at high energies, creating a state of matter known as the quark-gluon plasma. This state of matter exists in the cores of neutron stars. The results of heavy-ion collision experiments can be used to study the equation of state of dense matter under these extreme conditions~\cite{Huth:2021bsp,Lovato:2022vgq}. Bayesian methods have been employed to restrict the density dependence of the QCD EoS for dense nuclear matter using the mean transverse kinetic energy and elliptic flow of protons from HICs~\cite{OmanaKuttan:2022aml}, in the beam energy range $\sqrt{s_\text{NN}}=2$--$10$~GeV. The analysis results in tight constraints on EoS at density $n>4\,n_0$. The extracted EoS is found to be consistent with other observables measured in HIC experiments and constraints from astrophysical observations, which were not used in the analysis. 

At extremely high densities, perturbative QCD (pQCD) can vary the order of renormalization scales, which allows one to derive thermodynamical contributions of higher-order corrections~\cite{Kurkela:2009gj}. In recent works of Gorda et al.~\cite{Gorda:2022jvk,Gorda:2022lsk}, the authors demonstrate that the pQCD calculations provide significant and non-trivial information about the nuclear matter EoS, going beyond what can be obtained from current astrophysical observations. The EoS is extrapolated with a Gaussian process and conditioned with observations and the QCD input. They find that imposing the QCD input does not require extrapolation to a large-$n$ density, while providing strong additional constraints at high densities. In practice, the reliable interpolation between low- and high-density parts can produce an ensemble of EoSs that can be further constrained by astrophysical observations~\cite{Annala:2021gom}.



\subsection{Summary}
This chapter provides an overview of recent research on nuclear matter EoS inference using statistical learning algorithms (e.g. Bayesian Inference) and advanced deep learning methods (e.g. deep neural networks). The latest measurements, including pulsar masses and radii, are discussed. The neural network EoS and automatic differentiation framework for solving the inverse problem are also introduced. Multimessenger observations and nuclear constraints are discussed as ways to improve reconstruction. As more data is accumulated, such as from GW observations~\cite{LIGOScientific:2019lzm,LIGOScientific:2023vdi}, and as prior knowledge (e.g. physical constraints) is incorporated into deep models, the reconstruction performance will be improved continuously.
