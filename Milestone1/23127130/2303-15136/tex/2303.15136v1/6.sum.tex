\section{Summary and outlook}\label{sec:sum}

\textit{AI for science} has seen a surge in popularity and adoption in general~\cite{osti_1604756}. Especially, paradigms with modern machine- and deep-learning techniques are transforming across broad scientific domains -- those are overwhelmed by large-scale intensive computations and/or large amount of high-dimensional data, including high-energy nuclear physics (HENP)~\cite{Boehnlein:2021eym, He:2023zin,  Calafiura:2022ges}. The integration of machine learning techniques has led to remarkable advances and a host of results in the field, opening a new horizon for exploration and discovery. These ML-based methods have revolutionized the way of analyzing data, improved the ability to discover new phenomena and develop more efficient simulations. HENP is an extremely fruitful area in this sense, and many advances have been made in the past decade~\cite{Bzdak:2019pkr, Fukushima:2020yzx, Bogdanov:2022faf}. We are in the right and exciting era to work in this direction and to further our understanding of QCD matter in extreme conditions. 

This review aims to provide an overview of the current applications of ML in HENP theory studies, and to highlight some of the recent developments in this rapidly-evolving crossing field. Several aspects, revolving around ML for theoretical study of extreme QCD matter exploration, are discussed: from data analysis in high energy heavy-ion collisions(HICs) sector, to improving lattice QCD/QFT simulations, and to the inference of Neutron Star(NS) interior matter properties, covering the current efforts for strongly interacting nuclear matter study. Then a refined advanced development summary and discussion is presented in the last chapter to try to form a common ground from a methodology perspective in inspiring further exploration.

In terms of big \textbf{data analysis}, HENP indeed forms us a golden playground. Copious amounts of multifarious data can be collected from HIC experiment detectors, astrophysical observatories, and also lattice QCD simulations. Many of the well-established physics models or software/packages also makes it feasible to generate large-scale simulation data, to which the disentangled physics understanding and correlation analysis are yet daunting with conventional methods. Thus, as a modern numerical method to process complex data for hidden pattern decoding, ML and DL techniques have provided a powerful tool for exploring physics across the different disciplines. 

Besides being data-rich, a variety of areas in HENP are also \textbf{computation intensive}, with many of the accessible measurements such as billions of events from the detectors require understanding from theory simulations but not yet fully calculable in first principle's manner. For the purpose of whatever evolving or confirming our theory understanding, or exploring new physics for discoveries, the ability to perform efficient and prompt simulations is critical for many aspects of HENP. ML and DL have also made significant progresses in HENP computations, e.g., various algorithms have been developed to improve the simulation speed and accuracy~\cite{Shanahan:2022ifi}. Advanced calculations with faster and optimized models incorporate with ML methodologies allow for better prediction and comprehension in confronting data.

In addition to those head-on applications of ML/DL in HENP, there are still many challenges and questions need to be addressed further. Probably the first-line concern from physicists is the \textbf{interpretability} of the ML approaches utilized in HENP research~\cite{Neubauer:2022zpn}. Detailed understanding on the reason and condition for the used methods to work are desired in physics. More efforts need to be deployed in uncovering the somehow “Black Box” characteristic of ML algorithms, especially those are with huge amounts of parameters. Techniques to reveal the patterns and validate the computations with ML are called for as well. Then per purpose of physics exploration, those practical computation results from the ML methods need to point to or be transferred to physics knowledge or inspirations, which better to be conformed to the well-established language of physics in controlled way, such as with uncertainties properly given. The incorporation of physics priors into those ML paradigms to specific physics studies deserve further development in addressing this concern~\cite{2021NatRP...3..422K}.

Another challenge is how to reliably \textbf{connect experimental measurements and physical theory} using ML/DL techniques, with the associated \textbf{uncertainties} also properly evaluated. As emphasized, in HENP, to understand the nature of strongly interacting matter, experiments and observations from big scientific infrastructures play crucial role, which need to turn to physics knowledge as per mission. This however is with unprecedented scales and complexities, such as the high dimensionality and highly correlated data stream from measurement, the multi-scale and intricate physics simulations for the dynamical process involved of the measurement. Statistical learning methods such as Bayesian inference~\cite{Cranmer:2019eaq} and then alternative ML/DL approaches give series of great demonstrations to analyze data for pinning down physics knowledge. Principled uncertainty estimation for results inferred from naive ML/DL applications is not yet fully developed. These methods still have great potential to be unleashed to explore QCD matter. 

As future prospects, with ML/DL assisted, important physics properties and hopefully new physics phenomena are expected to be extracted from accumulating measurements in e.g., heavy ion collision experiments or astrophysics observations, to inform physics discovery. Be specific, ever \textbf{faster and more efficient analysis} to large amount of data could be anticipated to better connect experiment to physics theory; \textbf{smart, controllable and fast simulation} with ML methods such as generative modeling may largely reduce the demand in computational resource for the field, and also mitigate the enormous complexities in disentangling different physics involved for the simulation; in the course of neutron star, it'd be intriguing to explore the possible potential of ML in identifying e.g., presence of exotic phases in NS interiors from observations, and also develop more reliable ways in \textbf{combining evidence from multi-source data} such as those from HICs together those from multi-messenger astrophysics. Furthermore, in the lattice QCD sector, while efforts to \textbf{improve the efficiency and accuracy of lattice sampling and simulations} continue, it may also be possible to enhance our understanding of QCD from an ML algorithmic perspective. The intersection of HENP and ML is rapidly advancing, and with the continued progress and enthusiasm in ML, we can expect even more exciting developments and remarkable achievements in the near future.

	