\documentclass[10pt, reqno, a4paper]{amsart}
\usepackage{amssymb,latexsym,amsmath,amsfonts,bbm, amsthm}
%\documentclass{article}
%\usepackage{amsmath}
  \usepackage{paralist}
 % \usepackage{lineno}
 % \linenumbers
  %\usepackage[pagewise]{lineno}\linenumbers
  \usepackage{graphics} %% add this and next lines if pictures should be in esp format
  \usepackage{epsfig} %For pictures: screened artwork should be set up with an 85 or 100 line screen
  %\usepackage[notcite,notref,draft]{showkeys}
  %\usepackage[sort,nocompress]{cite}
\usepackage{graphicx}  \usepackage{epstopdf}%This is to transfer .eps figure to .pdf figure; please compile your paper using PDFLeTex or PDFTeXify.
\usepackage[linkcolor=red, citecolor=magenta]{hyperref}
\hypersetup{colorlinks=true, urlcolor=blue}
\usepackage[nameinlink]{cleveref}
%\usepackage{url}
\numberwithin{equation}{section}
% \usepackage[colorlinks=true]{hyperref}
 \usepackage{comment}
   % Warning: when you first run your tex file, some errors might occur,
   % please just press enter key to end the compilation process, then it will be fine if you run your tex file again.
   % Note that it is highly recommended by AIMS to use this package.
%\hypersetup{urlcolor=blue, citecolor=red}
%\usepackage{hyperref}
\voffset = -19pt
\hoffset = -51pt
\textwidth = 6.1in
\textheight = 9in

  %\textheight=8.9 true in
  % \textwidth=5.9 true in
    %\topmargin 30pt
     \setcounter{page}{1}
\numberwithin{equation}{section}
% The next 5 line will be entered by an editorial staff.
\def\currentvolume{X}
 \def\currentissue{X}
  \def\currentyear{200X}
   \def\currentmonth{XX}
    \def\ppages{X--XX}
     \def\DOI{10.3934/xx.xxxxxxx}

 % Please minimize the usage of "newtheorem", "newcommand", and use
 % equation numbers only situation when they provide essential convenience
 % Try to avoid defining your own macros
 \newtheorem{theorem}{Theorem}[section]
 \newtheorem{assertion}[theorem]{Assertion}
 \newtheorem{claim}[theorem]{Claim}
 \newtheorem{conjecture}[theorem]{Conjecture}
 \newtheorem{cor}[theorem]{Corollary}
 \newtheorem{defn}[theorem]{Definition}
 \newtheorem{eg}[theorem]{Example}
 \newtheorem{figger}[theorem]{Figure}
 \newtheorem{lem}[theorem]{Lemma}
 \newtheorem{prop}[theorem]{Proposition}
 \newtheorem{rem}[theorem]{Remark}
 \newcommand{\pa} {\partial}
 \newcommand{\R}{\mathbb{R}}
 \newcommand{\N}{\mathbb{N}}
 \newcommand{\ds}{\displaystyle}
 \renewcommand{\leq}{\leqslant}
 \newcommand{\norm}[1]{\left\Vert#1\right\Vert}
 \newcommand{\vect}[1]{\left[#1\right]^{\top}}
 \def\hmath$#1${\texorpdfstring{{\rmfamily\textit{#1}}}{#1}}
 \newcommand{\rd}{{\rm d}}
 \newcommand{\mc}{\mathcal}
 \def\textmatrix#1&#2\\#3&#4\\{\bigl({#1 \atop #3}\ {#2 \atop #4}\bigr)}
 \def\dispmatrix#1&#2\\#3&#4\\{\left({#1 \atop #3}\ {#2 \atop #4}\right)}
 \newcommand{\ip}[2]{\left<{#1},{#2}\right>}
 \newcommand{\rea}{\mathbb{R}}
 \newcommand{\abs}[1]{\left|#1\right|}
 \newcommand{\lt}{L^2(0,1)}
 \newcommand{\cplx}{\mathbb{C}}
 \newcommand{\z}{\mathbb{Z}}
 \newcommand{\tl}{\tilde}
 \theoremstyle{remark}
 \newsavebox{\savepar}		 
 \newenvironment{boxit}{\begin{center}
 		\def{\bphi}{\boldsymbol \phi}
 		\begin{lrbox}{\savepar}
 			\begin{minipage}[t]{13cm} }
 			{\end{minipage}\end{lrbox}\fbox{\usebox{\savepar}}\end{center}}
 \usepackage{xcolor}
% \usepackage{titlesec}
% \DeclareMathOperator{\spn}{span}
%\newcommand{\norm}[1]{\left\|#1\right\|}
%\newcommand{\abs}[1]{\left|#1\right|}
%\newtheorem{theorem}{Theorem}[section]
%\newtheorem{corollary}{Corollary}
%\newtheorem*{main}{Main Theorem}
%\newtheorem{lemma}[theorem]{Lemma}
%\newtheorem{proposition}{Proposition}
%\newtheorem{conjecture}{Conjecture}
%\newtheorem*{problem}{Problem}
%\theoremstyle{definition}
%\newtheorem{definition}[theorem]{Definition}
%\newtheorem{remark}{Remark}
%\newtheorem*{notation}{Notation}
%\newcommand{\ep}{\varepsilon}
%\newcommand{\eps}[1]{{#1}_{\varepsilon}}
%\newcommand{\tl}{\tilde}
%\newcommand{\lt}{L^2(0,1)}
%\newcommand{\lT}{L^2(0,T)}
%\newcommand{\ip}[2]{\left<{#1},{#2}\right>}
%\newcommand{\eps}{\varepsilon}
%\newcommand{\zetbar}{\overline{\zeta}}
%\newcommand{\zt}{\zeta}
%\newcommand{\zbar}{\overline{z}}
%\newcommand{\abar}{\overline{a}}
%\newcommand{\lbar}{\overline{\lambda}}
%\newcommand{\albar}{\overline{\alpha}}
\newcommand{\al}{\alpha}
\newcommand{\lm}{\lambda}
\newcommand{\tht}{\theta}
\newcommand{\zahl}{\mathbb{Z}}
\newcommand{\nat}{\mathbb{N}}
\newcommand{\fie}{\varphi}
\newcommand \del[1]{\delta_{#1}}
\newcommand{\dl}{\delta}
\newcommand{\om}{\omega}
\newcommand{\rank}{\operatorname{rank}}
%domains
\newcommand{\bdy}{\partial}
\newcommand{\CC}{\mathbb{C}^2}
%\newcommand{\cplx}{\mathbb{C}}
\newcommand{\RR}{\mathbb{R}^2}
%\newcommand{\rea}{\mathbb{R}}



\address{Department of Mathematics \& Statistics, Indian Institute of Technology Kanpur, Kanpur, UP- 208016}
\email{sakilahamed777@gmail.com}
\address{Department of Mathematics, Indian Institute of Technology Bombay, Powai, Mumbai- 400076}
\email{subratamajumdar634@gmail.com}


\date{\today}
\begin{document}
\title[Control of Compressible Navier-Stokes system with Maxwell's law]{Controllability and Stabilizability of the linearized compressible Navier-Stokes system with Maxwell's law}
\author[Sakil Ahamed, and Subrata Majumdar]{Sakil Ahamed$^*$, and Subrata Majumdar$^{\dagger}$}
\thanks{$^*$Department of Mathematics \& Statistics, Indian Institute of Technology Kanpur, Kanpur, UP- 208016, India; e-mail:  sakilahamed777@gmail.com}
\thanks{$^\dagger$ Department of Mathematics, Indian Institute of Technology Bombay, Powai, Mumbai- 400076, India; e-mail: subratamajumdar634@gmail.com}
\thanks{$^*$Corresponding author: Sakil Ahamed}
\keywords{Linearized compressible Navier-Stokes equation with Maxwell's law, Exact controllability, Boundary control, distributed control, Ingham inequality, Observability inequality, Feedback stabilization}
\subjclass[2020]{35Q35,35Q30, 93B05, 93B07, 93D15}

%\medskip
%{\footnotesize


\medskip



%The abstract of your paper
\begin{abstract}
In this paper, we study the control properties of the linearized compressible Navier-Stokes system with Maxwell's law around a constant steady state $(\rho_s, u_s, 0), \rho_s>0, u_s>0$ in the interval $(0, 2\pi)$ with periodic boundary data. We explore the exact controllability of the coupled system by means of a localized interior control acting in any of the equations when time is large enough. We also study the boundary exact controllability of the linearized system using a single control force when the time is sufficiently large. In both cases, we prove the exact controllability of the system in the space $L^2(0,2\pi)\times L^2(0, 2\pi)\times L^2(0, 2\pi)$. We establish the exact controllability results by proving an observability inequality with the help of an Ingham-type inequality. Next, we prove the small time lack of controllability of the concerned system for the case of localized interior control. 

Further, using a Gramian-based approach demonstrated by Urquiza, we prove the exponential stabilizability of the corresponding closed-loop system with an arbitrary prescribed decay rate using boundary feedback control law. 
\end{abstract}
	\maketitle
\tableofcontents

%The title of your section 1
\section{Introduction and Main results}
\subsection{Setting of the problem}
 The control and stabilization of fluid flow have been extensively studied due to their numerous practical applications such as weather prediction, water flow in a pipe, designing aircraft and cars. Many researchers have been interested in the subject of the controllability of fluid flows, more so for incompressible flow (see \cite{VB06}, \cite{Fur04}, \cite{Raymond}, \cite{Raymond1}, \cite{VK08}) than for compressible flow (see \cite{SC00}, \cite{SC02}). In our research, we focus on investigating the exact controllability and stabilizability of the linearized compressible Navier-Stokes system with Maxwell's law.

 Let us consider the one-dimensional compressible  Navier-Stokes system in the domain $(0, 2\pi)$:
\begin{equation}\label{eq1}
	\left.
	\begin{aligned}
		&\partial_t\hat{\rho}+\partial_x(\hat{\rho}\hat{u})=0 \qquad\qquad\qquad\qquad\text{ in }(0, T)\times (0, 2\pi),\\&\partial_t\left( {\hat{\rho}\hat u}\right) +\partial_x\left( {\hat{\rho}\hat u^2}\right) +\partial_xp =\partial_x\hat{S}\qquad\text{in }(0, T)\times (0, 2\pi),
		\end{aligned}
	\right\}
\end{equation} 
where, $\hat{\rho}$, $\hat{u}$, $p$ and  $\hat{S}$ represent the density, velocity, pressure and stress tensor of the fluid, respectively. Equation \eqref{eq1}$_1$ is the consequence of conservation of mass and equation \eqref{eq1}$_2$ is the consequence of conservation of momentum. We assume that the pressure $p$ satisfies the following constitutive law:
\begin{equation}\label{pr}
	p(\hat \rho)=a\hat\rho^{\gamma},\: \quad \gamma\geq 1,
\end{equation}
where $a$ is a positive constant. While, the stress tensor $\hat S$ is assumed to satisfy the Maxwell's law:
\begin{equation}\label{max law}
	\kappa\partial_t\hat{S}+\hat{S}=\mu\partial_x\hat{u}, \text{ in }(0, T)\times (0, 2\pi).
\end{equation} Here $\mu$ and $\kappa$ are positive constants, with $\mu$ representing the fluid viscosity and $\kappa$ denoting the relaxation time that characterizes the time delay in the response of the stress tensor to the velocity gradient. For more detailed information, refer to \cite{hu2019global} and references therein. Relation $\eqref{max law}$ is first proposed by Maxwell, in order to describe the relation of stress tensor and velocity gradient for a non-simple fluid.
 


In this paper, we study the control aspects of the one-dimensional compressible Navier-Stokes system with Maxwell's law and with periodic boundary conditions
linearized around a constant steady state $\left( \rho_s, u_s, 0\right) , \rho_s > 0, u_s > 0$ of \eqref{eq1}.
	 More precisely we consider the following system:
\begin{equation}\label{nmaxeq3}
	\left.
	\begin{aligned}
		&\partial_t\rho+u_s\partial_x\rho+\rho_s\partial_xu=\mathbbm{1}_{\mathcal{O}_{1}} f_1, \qquad&&\text{in }(0, T)\times (0, 2\pi),\\&\partial_tu+u_s\partial_xu+a\gamma{\rho_s}^{\gamma-2}\partial_x\rho-\frac{1}{\rho_s}\partial_xS=\mathbbm{1}_{\mathcal{O}_{2}}  f_2,\qquad&&\text{in }(0, T)\times (0, 2\pi),\\&\partial_tS+\frac{1}{\kappa}S-\frac{\mu}{\kappa}\partial_xu=\mathbbm{1}_{\mathcal{O}_{3}} f_3, \qquad&&\text{in }(0, T)\times (0, 2\pi),\\&\rho(t, 0)=\rho(t, 2\pi),\: u(t,0)=u(t,2\pi),\: S(t,0)=S(t,2\pi),\qquad&&  t\in (0, T),\\&\rho(0,x)=\rho_0(x),\quad u(0,x)=u_0(x),\quad S(0,x)=S_0(x),\qquad&&  x\in (0,2\pi),
	\end{aligned}
	\right\}
\end{equation}
where $\mathbbm{1_{\mathcal{O}_{j}}}$ is  the characteristic function of an open set ${\mathcal{O}}_j \subset (0,2\pi),$  $j=1,2,3$ and $f_i$, $i=1,2,3$ are the distributed controls.

To establish the controllability results, we have to consider the following restriction on the constant steady state $(\rho_s, u_s,0)$:
 \begin{equation}\label{nmaxcc}
  \left(a\gamma\rho_s^{\gamma-1}- u_s^2\right) \geq 0.
  \end{equation}
 \begin{defn}
	The system \eqref{nmaxeq3} is exactly controllable in $(L^2(0, 2\pi))^3$ at time $T>0$, if for any initial condition $(\rho_0, u_0, S_0)^{\top} \in (L^2(0, 2\pi))^3$ and any other $(\rho_1, u_1, S_1)^{\top}\in (L^2(0, 2\pi))^3$, there exist controls $f_i\in L^2\left( 0,T;L^2(\mathcal{O}_i) \right), i=1,2,3,$ such that the corresponding solution $(\rho, u, S)^{\top}$ of \eqref{nmaxeq3} satisfies
	\begin{equation}
		(\rho, u, S)^{\top}(T, x)=(\rho_1, u_1, S_1)^{\top}(x), \text{ for all } x \in (0, 2\pi).
	\end{equation}
\end{defn}
Let us first mention controllability results for some fluid models related to our system. The existence of the solution of the compressible Navier-Stokes system with Maxwell's law, along with the blow-up results has been studied, for example, in \cite{HuRacke, hu2019global, WangHu} and references therein. If $\kappa = 0$, then the Maxwell's law \eqref{max law} turns into the Newtonian law $\hat{S}=\mu\partial_x\hat{u}$ and the equation \eqref{eq1} becomes Navier-Stokes system of a viscous, compressible, isothermal barotropic fluid (density is function of pressure only), in a bounded domain $(0,2\pi)$. The compressible Navier-Stokes system linearized around a constant trajectory $(\rho_s, u_s)$ for $\rho_s>0, u_s>0$, yields a coupled transport-parabolic system with constant coefficients.
The controllability of such systems with constant coefficients in one dimension has been extensively studied in the literature. In \cite{CM15, chowdhury2014null}, the authors studied this system in the domain  $(0,2\pi)$ with periodic boundary conditions and localized interior control acting only in the parabolic equation. In \cite{CM15}, using the moment method, the authors proved the null controllability in $H^{s+1}_{\textnormal{m}}(0,2\pi) \times H^{s}(0,2\pi),$ $s > 6.5,$ at time $T > \frac{2\pi}{|u_{s}|},$ where $H^{s}_{\textnormal{m}}(0, 2\pi)$ is the space of periodic Sobolev space with mean zero. This result was improved in \cite{chowdhury2014null} by showing that the null controllability holds for any initial data in $H^{1}_{\textnormal{m}}(0,2\pi) \times L^{2}(0,2\pi).$ Moreover, the authors also proved that, the system in consideration is not null controllable in $H^{s}_{\textnormal{m}}(0,2\pi) \times L^{2}(0,2\pi),$ $0\leqslant s < 1,$ at any time $T > 0$ by $L^{2}$-control acting in the parabolic equation. Thus  $H^{1}_{\textnormal{m}}(0,2\pi) \times L^{2}(0,2\pi)$ is the largest space in which the system is  null controllable by a $L^{2}$-parabolic control.

%It is worth mentioning that, all the above works consider only the case where control is active on the parabolic equation only. The proofs are based on an explicit computation of the eigenvalues and eigenfunctions of the linear operator, and thus restricted to certain boundary conditions.

Recently, in \cite{Beauchard}, the above results have been extended to more general coupled transport-parabolic systems with constant coefficients. The authors considered coupling of several transport and parabolic equations in one-dimensional torus $\mathbb{T}$ and studied the null controllability with localized interior control in optimal time. Moreover, an algebraic necessary and sufficient condition on the coupling term was obtained when controls act only on the parabolic or transport components. For the extension of these results to the general coupling matrix, one can refer to the work \cite{PLissy}.

The local null controllability of the nonlinear compressible Navier-Stokes system around a trajectory with non-zero velocity at large time $T>0$ has been obtained in \cite{ervedoza1, ervedoza2, ervedoza2018local, MR-17, MRR-17, MN-19}. 

 

In \cite{ahamed}, the authors considered the one-dimensional compressible Navier-Stokes equations with Maxwell's law linearized around a constant steady state $(\rho_s, 0,0), \rho_s>0$ with Dirichlet boundary conditions and with interior controls in the interval $(0, \pi)$. They have proved that the system is not null controllable at any time
using localized controls in density and stress equations and even everywhere control in the velocity equation. Moreover, they have shown that the system is null controllable at any time $T>0$ in the space $(L^2(0, \pi))^3$, if the control acts everywhere in the density or stress equation. Approximate controllability at large time using localized controls has also been studied. 

In the following subsection, we state our main results regarding the interior controllability of the system \eqref{nmaxeq3}
\subsection{Interior controllability}\label{intc}
At first, we consider the case when $f_2=0=f_3$ in \eqref{nmaxeq3}. Performing integration by parts and using the boundary conditions, from the system \eqref{nmaxeq3}, we deduce%from the velocity and stress equations of \eqref{nmaxeq3},
%using the boundary conditions, we deduce 
\begin{equation*}
	\frac{d}{dt}\left( \int_{0}^{2\pi}u(t,x)\:\rd x\right)=0,\quad \frac{d}{dt}\left( \int_{0}^{2\pi}S(t,x)e^{\frac{t}{\kappa}}\:\rd x\right)=0,
\end{equation*}
and therefore, 
\begin{equation*}
	\int_{0}^{2\pi}u(T,x)\:\rd x=\int_{0}^{2\pi}u_0(x)\:\rd x,\quad \int_{0}^{2\pi}S(T,x)e^{\frac{T}{\kappa}}\:\rd x=\int_{0}^{2\pi}S_0(x)\:\rd x.
\end{equation*}
Thus if the system \eqref{nmaxeq3} with $f_2=0=f_3$ is exactly controllable at time $T>0$ then necessarily
\begin{equation}\label{zm}
\int_{0}^{2\pi}u_0(x)\:\rd x=0=\int_{0}^{2\pi}S_0(x)\:\rd x.
\end{equation}
Let us define the space
\begin{equation*}
	L^2_m(0,2\pi)=\left\lbrace f\in L^2(0, 2\pi) : \int_{0}^{2\pi}f(x)\, \rd x=0\right\rbrace. 
\end{equation*}
\begin{theorem}\label{nmaxthm_pos} Let $f_2=0=f_3$ in \eqref{nmaxeq3} and $\mathcal{O}_1\subseteq (0,2\pi)$. Let us assume \eqref{nmaxcc}. Then there exists a $T_0>0$ such that the system \eqref{nmaxeq3} is exactly controllable in $L^{2}(0,2\pi) \times L^{2}_m(0,2\pi) \times L_m^{2}(0,2\pi)$ at time $T>T_0$, by a interior
	control $f_1\in L^2\left( 0,T;L^2(\mathcal{O}_1) \right) $ for the density.
\end{theorem}
\begin{rem}\label{nmaxremnull} In fact, if the control is used in one of the equations, then the following exact controllability results can be obtained:
	\begin{enumerate}
		\item Let $f_1=0=f_3$ in \eqref{nmaxeq3} and $\mathcal{O}_2\subseteq (0,2\pi)$. Let us assume \eqref{nmaxcc}. Then the system \eqref{nmaxeq3} is exactly controllable in $L_m^{2}(0,2\pi) \times L^{2}(0,2\pi) \times L_m^{2}(0,2\pi)$ at time $T>T_0$, by a interior
		control $f_2\in L^2\left( 0,T;L^2(\mathcal{O}_2) \right) $ for the velocity.% if $T>T_0$.
		
		\item Let $f_1=0=f_2$ in \eqref{nmaxeq3} and $\mathcal{O}_3\subseteq (0,2\pi)$. Let us assume \eqref{nmaxcc}. Then the system \eqref{nmaxeq3} is exactly controllable in $L_m^{2}(0,2\pi) \times L_m^{2}(0,2\pi) \times L^{2}(0,2\pi)$ at time $T>T_0$, by a interior control $f_3\in L^2\left( 0,T;L^2(\mathcal{O}_3) \right) $ for the stress.
	\end{enumerate}
\end{rem}
\begin{rem}\label{optimal time}
	Note that the characteristics equations associated to our system \eqref{nmaxeq3} can be written as:
	\begin{equation*}
		x+\beta_1 t=c_1,\quad x+\beta_2 t=c_2,\quad x+\beta_3 t=c_3,
	\end{equation*}
	where $c_1,c_2,c_3$ are constants and $\beta_1, \beta_2, \beta_3$ are the roots (distinct and nonzero) of the equation
	\begin{equation}\label{nmaxpolybeta intro}
		r^3+2u_sr^2+\left( u_s^2-b\rho_s-\frac{\mu}{\kappa\rho_s}\right)r-\frac{\mu u_s}{\kappa\rho_s}=0.
	\end{equation} 
	In the above theorem, the waiting time $T_0$  is of the form $$T_0={2\pi}{\left(\frac{1}{|\beta_1|}+\frac{1}{|\beta_2|}+\frac{1}{|\beta_3|}\right)}.$$ This result is a consequence of the particular construction on the biorthogonal family in \Cref{biorthothm}. In \Cref{nmaxthm_pos}, the exact controllability result is proved for any time $T>T_0$. We can not claim that $T_0$ is the minimal time to have the exact controllability of the system. Determine the minimal time $T_{min}>0$, such that the system is exactly controllable at $T\geq T_{min}$ and the system is not exactly controllable at $T<T_{min}$ is a challenging open problem. In particular, notice that the minimal control time should depend on the support of the localized control.

\end{rem}
\begin{rem}
	Assumption \eqref{nmaxcc} may not be a necessary condition for the controllability of the system \eqref{nmaxeq3}. In this paper, the proof of the controllability results relies on the spectrum of the linear operator associated with the system. We have used an Ingham-type inequality to prove these results. To use this inequality, we need a uniform spectral gap(see \Cref{gap}) and condition \eqref{nmaxcc} ensures the same. Thus assuming condition \eqref{nmaxcc} is a limitation of our method.
\end{rem}

%Let us consider the following boundary control system:
Next, we will study the controllability properties of the linearized compressible Navier–Stokes system with Maxwell's law when the control acts only in the boundary. 



\subsection{Boundary controllability}
%This paper also studies the boundary exact controllability issues for t
Let us consider the following system:
\begin{equation}\label{nmaxeq3 bd}
	\left.
	\begin{aligned}
		&\partial_t\rho+u_s\partial_x\rho+\rho_s\partial_xu=0, \qquad&&\text{in }(0, T)\times (0, 2\pi),\\&\partial_tu+u_s\partial_xu+a\gamma{\rho_s}^{\gamma-2}\partial_x\rho-\frac{1}{\rho_s}\partial_xS=0,\qquad&&\text{in }(0, T)\times (0, 2\pi),\\&\partial_tS+\frac{1}{\kappa}S-\frac{\mu}{\kappa}\partial_xu=0, \qquad&&\text{in }(0, T)\times (0, 2\pi),\\&\rho(0,x)=\rho_0(x),\quad u(0,x)=u_0(x),\quad S(0,x)=S_0(x),\qquad&&  x\in (0,2\pi),
	\end{aligned}
	\right\}
\end{equation}
 along with one of the following boundary conditions in the time interval $(0, T):$
\begin{align}
	\label{density}&\textit{Control in density: } \rho(t, 0)=\rho(t, 2\pi)+q(t),\: u(t,0)=u(t,2\pi),\: S(t,0)=S(t,2\pi),\\ 
	\label{velocity}&\textit{Control in velocity: } \rho(t, 0)=\rho(t, 2\pi),\: u(t,0)=u(t,2\pi)+r(t),\: S(t,0)=S(t,2\pi),\\
	\label{stress}&\textit{Control in stress: } \rho(t, 0)=\rho(t, 2\pi),\: u(t,0)=u(t,2\pi),\: S(t,0)=S(t,2\pi)+p(t),
	%\label{both end control} &\textit{Dirichlet control in both end points: },
\end{align}
where $q, r, p$, are the controls. We recall the definition of boundary exact controllability of the system \eqref{nmaxeq3 bd}-\eqref{density}.	
\begin{defn}
	The system \eqref{nmaxeq3 bd}-\eqref{density} is boundary exactly controllable in $(L^2(0, 2\pi))^3$ at time $T>0$, if for any initial condition $(\rho_0, u_0, S_0)^{\top} \in (L^2(0, 2\pi))^3$ and any other $(\rho_1, u_1, S_1)^{\top}\in (L^2(0, 2\pi))^3$, there exists control $q\in L^2\left( 0,T \right)$ such that the corresponding solution $(\rho, u, S)^{\top}$ of \eqref{nmaxeq3 bd}-\eqref{density} satisfies
	\begin{equation*}
		(\rho, u, S)^{\top}(T, x)=(\rho_1, u_1, S_1)^{\top}(x), \text{ for all } x \in (0, 2\pi).
	\end{equation*}
\end{defn}
We now mention some boundary controllability results of the linearized compressible Navier-Stokes system. In \cite{CM15}, the authors studied null controllability of the concerned system in $H^{s+1}_{\textnormal{m}}(0,2\pi) \times H^{s}(0,2\pi),$ $s > 4.5,$ using boundary control acting at the parabolic component at time $T > \frac{2\pi}{|u_{s}|}$ with periodic boundary data. Recently, in the work \cite{jiten2}, the above result has been improved. The author established the boundary null controllability of the system in the space $L^2(0, 2\pi)\times L^2(0, 2\pi)$ using density control and in the space $H^1_m(0, 2\pi)\times L^2(0, 2\pi)$ using velocity control in time $T > \frac{2\pi}{|u_{s}|}$. This paper extends these two results for the linearized compressible Navier-Stokes equation with non-barotropic fluids. In the context of Dirichlet boundary, the paper \cite{jiten} dealt with the boundary null controllability of this system with a density control in the space $H^s(0,1)\times L^2(0,1), s>\frac{1}{2}$ at time $T>1$. The authors have also explored the approximate controllability of the system in the space $L^2(0,1)\times L^2(0,1)$ at time $T>1$. Null controllability ``up to a finite-dimensional space" in the space $H^s(0,1)\times L^2(0,1), s>\frac{1}{2}$ has been shown by a velocity control in time $T>1$ with periodic-Dirichlet boundary set up.


%%%%%%%%%%%%%%%%%%%%%%%%%%%%%%%%%%%%%%%%%%%%%%%%%%%
Next, we will state our boundary controllability results for the system \eqref{nmaxeq3 bd}. 
 Observe that, if the system \eqref{nmaxeq3 bd}-\eqref{density} is boundary exactly controllable then necessarily 
\begin{align*}
	\int_{0}^{2\pi}\rho_0(x)\,\rd x+&u_s\int_{0}^{T}q(t)\rd t=0,\quad	\int_{0}^{2\pi}u_0(x)\,\rd x+b\int_{0}^{T}q(t)\, \rd t=0,\quad \int_{0}^{2\pi}S_0(x)\,\rd x=0,
\end{align*}
which imply
\begin{equation*}
	b\int_{0}^{2\pi}\rho_0(x)\, \rd x=u_s \int_{0}^{2\pi}u_0(x)\, \rd x, \quad \int_{0}^{2\pi}S_0(x)\,\rd x=0.
\end{equation*}
For the sake of simplicity, we take $\int_{0}^{2\pi}\rho_0(x)\, \rd x= \int_{0}^{2\pi}u_0(x)\, \rd x=0.$ Hence we consider the following space for our study: $$L_m^2(0, 2\pi)\times L_m^2(0, 2\pi)\times L_m^2(0, 2\pi).$$
%Now we are in a position to state our boundary exact controllability results. 
\begin{theorem}\label{nmaxthm_pos bd} Let us assume \eqref{nmaxcc}. Then the system \eqref{nmaxeq3 bd}-\eqref{density} is exactly controllable in $L_m^{2}(0,2\pi) \times L_m^{2}(0,2\pi) \times L_m^{2}(0,2\pi)$ at time $T>T_0$ (same as in \Cref{optimal time}), by a boundary 
	control $q\in L^2( 0,T) $ for the density.
\end{theorem}
\begin{rem}\label{nmaxremnul}If the control acts in one of the boundary, then we get the exact controllability of the system \eqref{nmaxeq3 bd}:
	\begin{enumerate}
		\item  Let us assume \eqref{nmaxcc}. Then the control system \eqref{nmaxeq3 bd}-\eqref{velocity} is exactly controllable in $L_m^{2}(0,2\pi) \times L_m^{2}(0,2\pi) \times L_m^{2}(0,2\pi)$ at time $T>T_0$, by a boundary
		control $r \in L^2( 0,T) $ for the velocity.
		
		\item Let us assume \eqref{nmaxcc}. Then the control system \eqref{nmaxeq3 bd}-\eqref{stress} is exactly controllable in $L_m^{2}(0,2\pi) \times L_m^{2}(0,2\pi) \times L_m^{2}(0,2\pi)$ at time $T>T_0$, by a boundary
		control $p \in L^2( 0,T) $ for the stress.
	\end{enumerate}
\end{rem}
The proof of the exact controllability result relies on an observability inequality and the spectral analysis of the linearized operator. The spectrum of the linear operator consists of three sequences of complex eigenvalues whose real parts converge to three distinct finite numbers, and the imaginary parts behave as $n$ for $|n|\rightarrow \infty$ (see, \Cref{nmaxsecspec}). Therefore the system behaves like a hyperbolic system. Hence the geometric control condition is needed for the controllability of the system using localized control. Moreover, the eigenfunctions of the linearized operator and its adjoint form Riesz bases on $L^2(0,2\pi)\times L^2(0,2\pi)\times L^2(0,2\pi)$. Then using the series representation of the solution of the adjoint problem and a hyperbolic type Ingham inequality, we can prove \Cref{nmaxthm_pos}. The proof of the Ingham-type inequality relies on the construction of a family biorthogonal to the family of exponentials $\{e^{-{\lambda_k^l}t}, k\in \z^*, l=1,2,3\}$, see \Cref{biorthogonal}. The technique we have adapted is inspired from \cite{LR14} and \cite{biccari2019null}. The entire study relies on the theory of logarithmic integral and complex analysis. One can refer to the book \cite{RY} for more details. Moreover, it is worthwhile to mention that a modification (for example \cite[Proposition 3.1]{chowdhury2014null} or \cite[Proposition 8.1]{ahamed}) to the general hyperbolic Ingham inequality (see \cite{ingham1936some}, \cite[Theorem 4.1]{micu2004introduction}) is not enough to prove the required Ingham-type inequality for our case.

This paper also deals with the lack of controllability of the system \eqref{nmaxeq3}. For the influence of the transport component of the system, it is relevant to guess that the system \eqref{nmaxeq3} cannot be steered to zero (as exact and null controllability are equivalent for our system) at small time (see \Cref{lack theorem}). More precisely, in \Cref{lack sec}, we have established that, the system is not exactly controllable at small time by means of localized interior control. We have utilized the work \cite[Section 3]{Beauchard} to establish this result. The same result can be proved for boundary case also (for example, see \cite{Shirshendu}, \cite{jiten2}).
\subsection{Stabilizability}
In this paper, we also study the boundary feedback stabilization of linearized compressible Navier-Stokes system with Maxwell's law \eqref{nmaxeq3 bd} with the control acting only in the any boundary of density \eqref{density}, velocity \eqref{velocity} and stress \eqref{stress}. More precisely, we construct a feedback control law which forces the solution of \eqref{nmaxeq3 bd}-\eqref{density} to decay exponentially towards the origin with any order. In finite-dimensional cases, the notions of controllability and complete stabilizability (stabilization with any decay rate) are equivalent. Whereas for infinite-dimension cases, this does not hold in general. For bounded control operator, it is known that controllability immediately implies stabilizability, see Theorem 3.3 in \cite{zabczyk2008mathematical}, \cite{DR71}. Thus for the case of interior control, our linearized Maxwell's system is exponentially stabilizable with any decay rate. However, for the boundary control cases, the corresponding control operator is unbounded. Thus our work adds one more example where controllability implies stabilizability for the unbounded control operator, like \cite{CE09}, \cite{Shirshendu}. 



In our case, we prove the exponential stabilization result by an explicit feedback law following a method demonstrated by J.M. Urquiza \cite{UR05} which relies on the Gramian approach and the Riccati equations, see \cite{AV}, \cite{KV1} for more details. This method addresses the exponential stabilization issue for an exactly controllable system with an operator that generates an infinitesimal group of continuous operator. This approach has been successfully utilized to study rapid exponential stabilization of KdV equation \cite{CE09}, Boussinesq system of KdV-KdV type in \cite{Filho2021RapidES}, linearized compressible Navier-Stokes equation in the case of creeping flow in \cite{Shirshendu}. Our work regarding exponential stabilization is inspired from \cite{CE09}, \cite{Filho2021RapidES} and \cite{Shirshendu}. Recently, classical moment method has been extensively utilized to establish the exact controllability and stabilizability of some dispersive system, see \cite{leal2021control}, \cite{leal2021simple}. In these works, utilizing the group structure of the corresponding system, the authors conclude the stabilizability from the controllability in some periodic Sobolev spaces.

Let us state the main stabilization result for the closed-loop system \eqref{nmaxeq3 bd}-\eqref{density}.
\begin{theorem}\label{nmaxthm_pos st 1}Let us assume \eqref{nmaxcc}. Then the control system \eqref{nmaxeq3 bd}-\eqref{density} is completely stabilizable in $L_m^{2}(0,2\pi) \times L_m^{2}(0,2\pi) \times L_m^{2}(0,2\pi)$ by boundary feedback control $q\in L^2\left( 0,\infty\right) $ for the density. That means, for any initial data $(\rho_0, u_0, S_0) \in L_m^2(0, 2\pi)\times L_m^{2}(0,2\pi) \times L_m^{2}(0,2\pi)$ and any $\nu>0$, there exists a continuous linear map $\Pi$ such that the system \eqref{nmaxeq3 bd}-\eqref{density} with $q(t)= \Pi(\left(\rho(t,\cdot), u(t,\cdot), S(t,\cdot)\right))$ satisfies the following
	{\small
	\begin{align*}
		\norm{\rho(t, \cdot )}_{L^2(0,2\pi)}+\norm{u(t, \cdot )}_{L^2(0,2\pi)}+\norm{S(t, \cdot )}_{L^2(0,2\pi)}& \leq C e^{-\nu t} \bigg[ \norm{\rho_0}_{L^2(0,2\pi)}+\norm{u_0}_{L^2(0,2\pi)}+\norm{S_0}_{L^2(0,2\pi)} \bigg], 
	\end{align*}}
	for all  $t>0$,	where $C=C(T)$ is a positive constant, independent of $\rho_0, u_0, S_0, q$ and $t$.
\end{theorem}
%the when control operator is unbounded. This notion is also known as complete stabilizability of the closed-loop system.
\subsection{Organization of the paper}
The plan of the paper is as follows. In \Cref{nmaxwell-posedness}, we study the well-posedness of the system \eqref{nmaxeq3} using semigroup theory, and we determine the adjoint of the linear operator associated to the coupled system. Then in \Cref{nmaxsecspec}, we analyze the behavior of the spectrum of the linearized operator associated to the system \eqref{nmaxeq3}. After that we show that the eigenfunctions of the linear operator form a Riesz basis in $(L^2(0, 2\pi))^3$.  The exact controllability result \Cref{nmaxthm_pos} is proved in \Cref{nmaxsecnullcont}. The proof of boundary exact controllability result \Cref{nmaxthm_pos bd} is given in \Cref{nmax-bd}. In \Cref{lack sec}, we will present a further discussion on the lack of controllability of the system \eqref{nmaxeq3} in small time. After that, we study the rapid exponential stabilization (\Cref{nmaxthm_pos st 1}) in \Cref{stab}. \Cref{biorthogonal} is devoted to the proof of the Ingham-type inequality \ref{ingham2}.  %Finally, the Appendix contains useful results related to the spectrum of the corresponding linearized operator.


\section{Linearized operator}\label{nmaxwell-posedness}
Let us define $\mathcal {{Z}}=L^2(0, 2\pi)\times L^2(0, 2\pi)\times L^2(0, 2\pi)$ and the positive constant
\begin{equation}\label{nmaxequ-constant}
	b=a\gamma{\rho_s}^{\gamma-2}.
\end{equation} 
Let $\mathcal {{Z}}$ be endowed with the inner product
\begin{equation}\label{nmaxinnerproduct}
	\left\langle \begin{pmatrix}
		\rho\\ u\\S
	\end{pmatrix},\begin{pmatrix}
		\sigma\\ v\\\tilde{S}
	\end{pmatrix} \right\rangle_{\mathcal {{Z}}}=b\int_{0}^{2\pi}\rho\bar{\sigma} \,\rd x+\rho_s\int_{0}^{2\pi}u\bar{v}\, \rd x+\frac{\kappa}{\mu}\int_{0}^{2\pi}S\bar{\tilde{S}}\, \rd x. 
\end{equation}
We now define the unbounded operator $\left( \mathcal A, \mathcal D(\mathcal A;\mathcal {{Z}})\right) $ in $\mathcal {{Z}}$ by{\small$$\mathcal D(\mathcal A;\mathcal{{Z}})=\left\lbrace \begin{pmatrix}
		\rho\\u\\S
	\end{pmatrix}\in \mathcal {{Z}}\: :\:\left(\rho,u,S \right)^\top \in H^1(0,2\pi)\times H^1(0,2\pi)\times H^1(0,2\pi),\:\begin{aligned}
		&\rho(0)=\rho(2\pi),\\ &u(0)=u(2\pi),\\&S(0)=S(2\pi).
	\end{aligned}
	\right\rbrace
	$$}and
\begin{equation}\label{op}
	\mathcal A=\begin{bmatrix}
		-u_s\frac{d}{dx}\hspace{1mm}&-\rho_s\frac{d}{dx}\hspace{1mm}&0\vspace{2mm}\\-b\frac{d}{dx}&-u_s\frac{d}{dx}&\frac{1}{\rho_s}\frac{d}{dx}\vspace{2mm}\\0&\frac{\mu}{\kappa}\frac{d}{dx}&-\frac{1}{\kappa}
	\end{bmatrix}.
\end{equation}
The control operator $\mathcal{B} \in \mathcal{L}(\mathcal {{Z}};\mathcal {{Z}})$ is defined by 
\begin{equation}\label{nmaxeqcontr}
	\mathcal{B} f  = \left( \mathbbm{1}_{\mathcal{O}_{1}} f_1, \mathbbm{1}_{\mathcal{O}_{2}} f_2, \mathbbm{1}_{\mathcal{O}_{3}} f_3\right)^{\top}, \qquad f=(f_1, f_2, f_3)^{\top} \in \mathcal{{ Z}}.
\end{equation}
With the above introduced notations, the system  \eqref{nmaxeq3} can be rewritten as 
\begin{equation} \label{nmaxop-eqn}
	\dot{z}(t) = \mathcal{A} z(t) + \mathcal{B} f(t), \quad t\in (0,T), \qquad z(0) = z_{0},
\end{equation}
where, we have set $z(t) = (\rho(t,\cdot), u(t, \cdot), S(t,\cdot))^{\top},$ $z_{0} =(\rho_{0}, u_{0}, S_0)^{\top}, \text{ and } f(t)=(f_1(t,\cdot), f_2(t,\cdot), f_3(t,\cdot))^{\top} .$

Next, we will prove the well-posedness of the system \eqref{nmaxop-eqn}.

\begin{prop} \label{nmaxpr:semigroup-z} 
	The operator $(\mathcal{A}, \mathcal{D}(\mathcal{A}; \mathcal {{Z}}))$ is the infinitesimal generator of a strongly continuous semigroup $\left\lbrace \mathbb{T}_{t} \right\rbrace_{t\geq 0} $ on $\mathcal {{Z}}.$
	Further, for any $f\in L^2(0,T; \mathcal{{ Z}})$ and for any $z_0\in \mathcal {{Z}}$, \eqref{nmaxop-eqn} admits a unique solution $(\rho, u, S)\in C([0,T]; \mathcal {{Z}})$ with 
	\begin{equation*}
		\|(\rho, u, S)\|_{C([0,T];\mathcal {{Z}})} \leqslant  C \Big(\|z_0\|_{\mathcal {{Z}}}+ \|f\|_{L^2(0,T;\mathcal{{ Z}})}\Big).
	\end{equation*}
\end{prop}
\begin{proof}
	We rewrite $\mathcal{A}:= \mathcal{A}_{1} + \mathcal{A}_{2},$ with 
	\begin{equation*}
		\mathcal{A}_{1} = \begin{bmatrix}
			-u_s\frac{d}{dx}\hspace{1mm}&-\rho_s\frac{d}{dx}\hspace{1mm}&0\vspace{2mm}\\-b\frac{d}{dx}&-u_s\frac{d}{dx}&\frac{1}{\rho_s}\frac{d}{dx}\vspace{2mm}\\0&\frac{\mu}{\kappa}\frac{d}{dx}&0
		\end{bmatrix}, \quad 
		\mathcal{A}_{2} = \begin{bmatrix}
			0&0&0\\0&0&0\\0&0&-\frac{1}{\kappa}
		\end{bmatrix}.
	\end{equation*}
	Note that 
	$$ \norm{\mathcal{A}_{2}\begin{pmatrix} \rho \\ u \\ S \end{pmatrix}}_{\mathcal {{Z}}}\leqslant \frac{1}{{\kappa}} \norm{\begin{pmatrix} \rho \\ u \\ S  \end{pmatrix} }_{\mathcal {{Z}}} \quad \mbox{ for all }
	\begin{pmatrix} \rho \\ u \\ S \end{pmatrix}  \in \mathcal {{Z}}.$$
	Thus $\mathcal{A}_{2}$ is a bounded perturbation of the operator $\mathcal{A}_{1}$ on $\mathcal {{Z}}$.
	Therefore, it is sufficient to show that $\mathcal{A}_{1}$ generates a $C^{0}$-semigroup on $\mathcal {{Z}},$ (for details, see \cite[Theorem 2.11.2]{TW09}).
	
	The adjoint $\mathcal{A}^*_1$ of $\mathcal{A}_1$ with the domain 
	$\mathcal D(\mathcal A_1^*;\mathcal {{Z}})=\mathcal D(\mathcal A;\mathcal {{Z}})$ is given by
	\begin{equation*}
		\mathcal A^*_1=\begin{bmatrix}
			u_s\frac{d}{dx}\hspace{1mm}&\rho_s\frac{d}{dx}\hspace{1mm}&0\vspace{2mm}\\b\frac{d}{dx}&u_s\frac{d}{dx}&-\frac{1}{\rho_s}\frac{d}{dx}\vspace{2mm}\\0&-\frac{\mu}{\kappa}\frac{d}{dx}&0
		\end{bmatrix}=-\mathcal A_1.
	\end{equation*}
	Thus $\mathcal{A}_1$ is a skew-adjoint operator. Then using Stone Theorem (\cite[Theorem 3.8.6]{TW09}), we get that $\mathcal{A}_1$  generates a unitary group on $\mathcal {{Z}}$.
	\end{proof}
Thus, we have the following well-posedness result for the system \eqref{nmaxeq3}.
\begin{theorem}
For any $(\rho_0, u_0, S_0)^{\top}\in \mathcal{{ Z}}$ and $f_i\in L^2(0, T; L^2(0, 2\pi)), \, i=1,2,3,$ the system \eqref{nmaxeq3} admits a unique solution $(\rho, u, S)^{\top}\in C\left( [0, T];\mathcal{{ Z}}\right)$ with
\begin{equation*}
	\|(\rho, u, S)\|_{C([0,T];\mathcal {{Z}})} \leqslant  C \Big(\|(\rho_0, u_0, S_0)\|_{\mathcal {{Z}}}+ \|(f_1, f_2, f_3)\|_{L^2(0,T;\mathcal{{ Z}})}\Big).
\end{equation*}
\end{theorem}
Note that, the exact controllability of the pair $(\mc A, \mc B)$ is equivalent to the observability of the pair $(\mc A^*, \mc B^*),$ where $\mc A^*$ and $B^*$ are the adjoint operators of $\mc A \text{ and } \mc B$ respectively (see \cite[Chapter 2.3]{Co07} for details). Thus it is important to derive the adjoint of the operator $\mc A:$ 
\begin{prop} \label{nmaxadj-op}
	The adjoint of  $(\mathcal{A}, \mathcal{D}(\mathcal{A} ; \mathcal {{Z}}))$ in $\mathcal {{Z}}$ is defined by 
	\begin{equation} \label{nmaxdom-A*}
		\mathcal{D}(\mathcal{A}^*;\mathcal {{Z}}) =\mathcal{D}(\mathcal{A};\mathcal {{Z}}),
	\end{equation}
	and 
	\begin{equation} \label{nmaxop-A*}
		\mathcal{A}^* = \begin{bmatrix}
			u_s\frac{d}{dx}\hspace{1mm}&\rho_s\frac{d}{dx}\hspace{1mm}&0\vspace{2mm}\\b\frac{d}{dx}&u_s\frac{d}{dx}&-\frac{1}{\rho_s}\frac{d}{dx}\vspace{2mm}\\0&-\frac{\mu}{\kappa}\frac{d}{dx}&-\frac{1}{\kappa}
		\end{bmatrix}.
	\end{equation}
	Moreover, $(\mathcal{A}^*, \mathcal{D}(\mathcal{A}^*; \mathcal {{Z}}))$ is the infinitesimal generator of a strongly continuous semigroup $\left\lbrace \mathbb{T}^*_{t} \right\rbrace_{t\geq 0} $ on $\mathcal {{Z}}.$
\end{prop}
As mentioned in \Cref{intc}, the controllability results for \eqref{nmaxeq3} is expected in the subspaces of $\mathcal{Z}$ satisfying \eqref{zm}. Thus, we need to have that the restriction of the operator $\mathcal{A}$ on those subspaces of  $\mathcal{Z}$ also generates a semigroup. 
From \cite[Proposition 2.4.4, Section 2.4, Chapter 2]{TW09}, the following result holds. 
\begin{prop}
	Let us define the spaces
	\begin{align}
		{\mathcal Z_m}=& L^2(0, \pi)\times L_m^2(0, \pi)\times L_m^2(0, \pi),\\
		{\mathcal Z_{m,m}}=& L_m^2(0, \pi)\times L_m^2(0, \pi)\times L_m^2(0, \pi)\label{defn_Z_mm}.
	\end{align}
	where the spaces ${\mathcal Z_m}, {\mathcal Z_{m,m}}$ are equipped with the inner product \eqref{nmaxinnerproduct}.
	
	Then $\mathcal Z_m$ is invariant under the semigroups $\mathbb{T}$ and $\mathbb{T}^*$. The restriction of $\mathbb{T}$ to $\mathcal Z_m$ is a strongly continuous
	semigroup in $\mathcal Z_m$ generated by $(\mathcal{A}, \mathcal{D}(\mathcal{A};\mathcal{Z}_m))$, where $\mathcal{D}(\mathcal{A};\mathcal{Z}_m)=\mathcal{D}(\mathcal{A}, \mathcal{Z})\cap \mathcal{Z}_m$. Also, the restriction of $\mathbb{T}^*$ to $\mathcal Z_m$ is a strongly continuous semigroup in $\mathcal Z_m$  generated by $(\mathcal{A}^*, \mathcal{D}(\mathcal{A}^*;\mathcal{Z}_{m}))$, where $\mathcal{D}(\mathcal{A}^*;\mathcal{Z}_{m})=\mathcal{D}(\mathcal{A}^*, \mathcal{Z})\cap \mathcal{Z}_m$. 
	
	Similarly, $\mathcal{Z}_{m,m}$ is invariant under the semigroups $\mathbb{T}$ and $\mathbb{T}^*$. 
	The restriction of $\mathbb{T}$ to $\mathcal Z_{m,m}$ is a strongly continuous
	semigroup in $\mathcal Z_{m,m}$ generated by $(\mathcal{A}, \mathcal{D}(\mathcal{A};\mathcal{Z}_{m,m}))$, where $\mathcal{D}(\mathcal{A};\mathcal{Z}_{m,m})=\mathcal{D}(\mathcal{A}, \mathcal{Z})\cap \mathcal{Z}_{m,m}$. Also, the restriction of $\mathbb{T}^*$ to $\mathcal Z_{m,m}$ is a strongly continuous semigroup in $\mathcal Z_{m,m}$  generated by $(\mathcal{A}^*, \mathcal{D}(\mathcal{A}^*;\mathcal{Z}_{m,m}))$, where $\mathcal{D}(\mathcal{A}^*;\mathcal{Z}_{m,m})=\mathcal{D}(\mathcal{A}^*, \mathcal{Z})\cap \mathcal{Z}_{m,m}$. 
\end{prop}
\subsection{Spectral analysis of the linearized operator}\label{nmaxsecspec}
We define a Fourier basis $\left\lbrace \phi_{0,1},\:\phi_{n,l},\: l =1,2,3\right\rbrace_{n\in\mathbb{Z}^*}$ $\left(\: \mathbb{Z}^* = \mathbb{Z}\setminus \{0\}\right) $ in $\mathcal {{Z}}_m$ as follows :
\begin{align}\label{nmaxfbasis}
	&\phi_{0,1}(x)=\frac{1}{\sqrt{2\pi b}}\begin{pmatrix}
		1\\0\\0
	\end{pmatrix},\:\phi_{n,1}(x)=\frac{1}{\sqrt{2\pi b}}\begin{pmatrix}
		e^{inx}\\0\\0
	\end{pmatrix},\notag\\
	&\phi_{n,2}(x)=\frac{1}{\sqrt{2\pi \rho_s}}\begin{pmatrix}
		0\\e^{inx}\\0
	\end{pmatrix},\:\phi_{n,3}(x)=\sqrt{\frac{\mu}{2\pi\kappa}}\begin{pmatrix}
		0\\0\\e^{inx}
	\end{pmatrix},\quad n\in\mathbb{Z}^*.
	%	&\Phi^{(2)}_{n}(x)=\frac{1}{\sqrt{2\pi b}}\begin{pmatrix}
		%		e^{-inx}\\0\\0
		%	\end{pmatrix},\:\Phi^{(4)}_{n}(x)=\frac{1}{\sqrt{2\pi Q_0}}\begin{pmatrix}
		%		0\\e^{-inx}\\0
		%	\end{pmatrix},\:\Phi^{(6)}_{n}(x)=\sqrt{\frac{\mu}{2\pi\kappa}}\begin{pmatrix}
		%		0\\0\\e^{-inx}
		%	\end{pmatrix},
	%	 n\in\mathbb{N}.
\end{align}
Let us define the following finite dimensional spaces
$$\mathbf{V}_{0}=\text{span} \left\lbrace \phi_{0,1}\right\rbrace , \:\mathbf{V}_n=\text{span} \left\lbrace \phi_{n,l},\: l=1,2,3\right\rbrace , n\in\mathbb{Z}^*,$$where 'span' stands for the vector space generated by those functions. One can verify that $\mathcal {{Z}}_m=\oplus_{n\in \mathbb{Z}}\mathbf{V}_n.$ 
\begin{lem}\label{nmaxinv}
	For all $n\in\mathbb{Z}^*$, $\mathbf{V}_n$ is invariant under $\mathcal A$ and $$\mathcal A_n=\mathcal A|_{\mathbf{V}_n}\in\mathcal{L}(\mathbf{V}_n),$$has the following matrix representation$$\begin{pmatrix}
		-inu_s&-in\sqrt{b\rho_s}&0\\
		-in\sqrt{b\rho_s}&-inu_s&in\sqrt{\frac{\mu}{\kappa\rho_s}}\\
		0&in\sqrt{\frac{\mu}{\kappa\rho_s}}&-\frac{1}{\kappa}
	\end{pmatrix}$$in the basis $\left\lbrace \phi_{n,l},\: l=1,2,3\right\rbrace_{n\in\mathbb{Z}^*} $ of $\mathbf{V}_n$.
\end{lem}
\begin{prop}\label{sp}
	The spectrum of the operator $(\mathcal{A}, \mathcal{D}(\mathcal{A}; \mathcal {{Z}}_m))$ consists of $0$ and three sequences $\lambda_n^1$, $\lambda_n^2$ and $\lambda_n^3$ of eigenvalues. Furthermore, we have the following asymptotic behaviors:
	\begin{equation}\label{nmaxasmlambda}
		\left.
		\begin{aligned}
			\lambda_n^1=&-\omega_1+i \beta_1n + O\left( \frac{1}{|n|}\right),\\
			\lambda_n^2=&-\omega_2+i \beta_2n + O\left( \frac{1}{|n|}\right),\\
			\lambda_n^3=&-\omega_3+i \beta_3n + O\left( \frac{1}{|n|}\right),
		\end{aligned}
		\right\}
	\end{equation}
	for $|n|$ large enough, where $\beta_j, j=1,2,3$ are the distinct real roots of the equation
	\begin{equation}\label{nmaxpolybeta}
		r^3+2u_sr^2+\left( u_s^2-b\rho_s-\frac{\mu}{\kappa\rho_s}\right)r-\frac{\mu u_s}{\kappa\rho_s}=0,
	\end{equation}
	and 
	\begin{equation}
		\omega_j=\frac{ \beta_j^2+2u_s \beta_j+u_s^2-b\rho_s }{\kappa\left( 3\beta_j^2+4u_s\beta_j+u_s^2-b\rho_s-\mu/\kappa\rho_s\right) }\neq 0,\quad j=1,2,3.
	\end{equation}
\end{prop}
	\begin{proof}
		We see that $\mathcal{A}\phi_{0,1}=0\cdot\phi_{0,1}$ and $ \phi_{0,1}\neq \mathbf 0$. Thus $0$ is an eigenvalue of $\mathcal{A}$.
		
		From \Cref{nmaxinv}, the characteristic equation is given by:
		\begin{align}\label{nmaxchar.poly}
			\mathcal{F}_n(\lambda) : = \lambda^3+\left( \frac{1}{\kappa}+2inu_s\right)\lambda^2+&\left(-u_s^2n^2+b\rho_sn^2+\frac{\mu n^2}{\kappa\rho_s}+\frac{2iu_sn}{\kappa}
			\right)\lambda\notag\\
			+&\left( -\frac{u_s^2n^2}{\kappa}+\frac{b\rho_sn^2}{\kappa}+\frac{i\mu u_sn^3}{\kappa\rho_s}\right) =0,\quad n\in\mathbb{Z}^*.
		\end{align}
%		
%		
%		
		Let $\lambda_n^l=\eta_n^l+i\tau_n^l$ with $\eta_n^l\in\mathbb{R}$, $\tau_n^l\in \mathbb{R}$, and $l=1,2,3$ are the roots of \eqref{nmaxchar.poly}. Using \cite[Theorem 3.2]{ZZ}, we can show that all the roots of \eqref{nmaxchar.poly} have negative real parts, i.e., 
		\begin{equation}\label{negrealpart}
			\eta_n^l <0 , \text{ for } l=1,2,3 \text{  and } n\in\mathbb{Z}^*.
		\end{equation}
		
From the relation between roots and coefficients of the equation \eqref{nmaxchar.poly}, we get
		\begin{align}
			&\eta_n^1+\eta_n^2+\eta_n^3=-\frac{1}{\kappa},\label{nmaxrc1}\\
			&\tau_n^1+\tau_n^2+\tau_n^3=-2u_s n,\label{nmaxrc2}\\
			&\left(\eta_n^1\eta_n^2+\eta_n^1\eta_n^3+\eta_n^2\eta_n^3 \right) -\left(\tau_n^1\tau_n^2+\tau_n^1\tau_n^3+\tau_n^2\tau_n^3 \right) =\left(b\rho_s+\frac{\mu}{\kappa\rho_s}-u^2_s \right)n^2,\label{nmaxrc3}\\ 
			&\eta_n^1\left(\tau_n^2+\tau_n^3 \right) +\eta_n^2\left(\tau_n^1+\tau_n^3 \right)+\eta_n^3\left(\tau_n^1+\tau_n^2 \right)=\frac{2u_s}{\kappa}n,\label{nmaxrc4}\\
			&\eta_n^1\eta_n^2\eta_n^3-\eta_n^1\tau_n^2\tau_n^3-\eta_n^2\tau_n^1\tau_n^3-\eta_n^3\tau_n^1\tau_n^2=\frac{u^2_s-b\rho_s}{\kappa}n^2,\label{nmaxrc5}\\
			&\tau_n^1\tau_n^2\tau_n^3-\tau_n^1\eta_n^2\eta_n^3-\eta_n^1\tau_n^2\eta_n^3-\eta_n^1\eta_n^2\tau_n^3=\frac{\mu u_s}{\kappa\rho_s}n^3.\label{nmaxrc6}
		\end{align}
		We will prove \eqref{nmaxasmlambda} in the following steps.
		
		\textit{Step 1: Asymptotic behavior of real part of eigenvalues: }
		
		Since $\eta_n^l <0 $, for $l=1,2,3$ and $n\in\mathbb{Z}^*$, from \eqref{nmaxrc1}, we observe that the sequence $\left\lbrace \eta_n^l\right\rbrace_{n\in\mathbb{Z}^*} $ is bounded for $l=1,2,3.$ Thus we get $\frac{\eta_n^l}{n}\rightarrow 0$ as $|n|\rightarrow \infty$, for $l=1,2,3.$
		
		\vspace{2mm}
		
		\textit{Step 2: Asymptotic behavior of imaginary part of  eigenvalues: }	
		
		Thanks to \eqref{nmaxrc1}, \eqref{nmaxrc2} and \eqref{nmaxrc3}, we have
		\begin{align}\label{nmaxrc61}
			\left(\left( \eta_n^1\right)^2+\left( \eta_n^2\right)^2+\left( \eta_n^3\right)^2 \right) - \left(\left( \tau_n^1\right)^2+\left( \tau_n^2\right)^2+\left( \tau_n^3\right)^2 \right)= \frac{1}{\kappa^2}- 2\left( u_s^2+b\rho_s+\frac{\mu}{\kappa\rho_s}\right)n^2.
		\end{align} 
		Set $\tilde{\tau}_n^l=\frac{\tau_n^l}{n}$, $l=1,2,3$. Then using Step 1 and \eqref{nmaxrc61}, it follows that
		\begin{equation}
			\lim\limits_{|n|\rightarrow \infty}\left(\left( \tilde\tau_n^1\right)^2+\left( \tilde\tau_n^2\right)^2+\left( \tilde\tau_n^3\right)^2 \right)= 2\left( u_s^2+b\rho_s+\frac{\mu}{\kappa\rho_s}\right).
		\end{equation}
		Thus the sequence $\left\lbrace \tilde\tau_n^l\right\rbrace_{n\in\mathbb{Z}^*} $ is bounded for $l=1,2,3.$	
		
		Also from \eqref{nmaxrc2}, \eqref{nmaxrc3} and \eqref{nmaxrc6},  we deduce that 
		\begin{equation}\label{nmaxrc7}
			\left.
			\begin{aligned}
				\tilde\tau_n^1+\tilde\tau_n^2+\tilde\tau_n^3 =& -2u_s,\\
				\lim\limits_{|n|\rightarrow \infty}	\left( \tilde\tau_n^1\tilde\tau_n^2+\tilde\tau_n^1\tilde\tau_n^3+\tilde\tau_n^2\tilde\tau_n^3\right) =& u_s^2-b\rho_s-\frac{\mu}{\kappa\rho_s},\\
				\lim\limits_{|n|\rightarrow \infty}	 \tilde\tau_n^1\tilde\tau_n^2\tilde\tau_n^3 =& \frac{\mu  u_s}{\kappa\rho_s}.
			\end{aligned}
			\right\}
		\end{equation}
	Thus from \eqref{nmaxrc7}, it follows that
	\begin{equation}\label{nmaxrc8}
		\left.
		\begin{aligned}
			\tilde\tau_n^1+\tilde\tau_n^2+\tilde\tau_n^3 =& -2u_s,\\
			\tilde\tau_n^1\tilde\tau_n^2+\tilde\tau_n^1\tilde\tau_n^3+\tilde\tau_n^2\tilde\tau_n^3 =& u_s^2-b\rho_s-\frac{\mu}{\kappa\rho_s}+\delta^1_n,\\
			\tilde\tau_n^1\tilde\tau_n^2\tilde\tau_n^3 =& \frac{\mu  u_s}{\kappa\rho_s}+\delta^2_n,
		\end{aligned}
		\right\}
	\end{equation}
	where $\left| \delta^l_n\right| \rightarrow 0$, as $|n|\rightarrow \infty$, for $l=1,2$. Therefore, $\tilde\tau_n^l,  l=1,2,3 $ satisfies the following equation:
	\begin{equation}\label{p_nx}
		\mathcal{P}_n\left(r \right) : = r^3+2u_sr^2+\left( u_s^2-b\rho_s-\frac{\mu}{\kappa\rho_s}+\delta^1_n\right)r-\left( \frac{\mu u_s}{\kappa\rho_s}+\delta^2_n\right) =0,\quad n\in\mathcal{Z}^*.
	\end{equation}
	Set
	\begin{equation}\label{nmaxrimag}
		\mathcal{P}\left(r \right) : = r^3+2u_sr^2+\left( u_s^2-b\rho_s-\frac{\mu}{\kappa\rho_s}\right)r-\frac{\mu u_s}{\kappa\rho_s}=0.
	\end{equation}
	Let us denote the roots of \eqref{nmaxrimag} by $\beta_l$, $ l=1,2,3$. Observe that the coefficients of $\mathcal{P}_n$ converge to the coefficients of $\mathcal{P}$, as $|n|\rightarrow \infty$. Then using the fact that the roots of a polynomial continuously depend on the coefficients of the polynomial (see \cite{RCDC}), we get that the roots of the polynomial $\mathcal{P}_n$ converge to the roots of the polynomial $\mathcal{P}$, as $|n|\rightarrow \infty$, i.e., $\frac{\tau_n^l}{n}\rightarrow \beta_l$ as $|n|\rightarrow \infty$, for $l=1,2,3.$
	
	\vspace{2mm}
	
	\textit{Step 3: Behavior of roots of the polynomial $\mathcal{P}$: }
	
	The discriminant of the cubic polynomial $\mathcal{P}$ is given by
	{\small
		\begin{align}\label{nmaxdiscriminant}
			D= \frac{9\mu^2u_s^2}{\kappa^2\rho_s^2}+\frac{36\mu u_s^2}{\kappa\rho_s}\left(b\rho_s-u_s^2 \right)+\frac{32\mu u_s^4}{\kappa\rho_s}
			+  4u_s^2\left( u_s^2-b\rho_s-\frac{\mu}{\kappa\rho_s}\right)^2+4\left( b\rho_s-u_s^2 +\frac{\mu}{\kappa\rho_s}\right)^3. 
		\end{align}	
	}
	Using \eqref{nmaxcc} and \eqref{nmaxequ-constant}, we get $D>0$, i.e., the roots of the polynomial $\mathcal{P}$ are real and distinct, which imply	
	\begin{equation}\label{nmaxnodoubleroot}
		\mathcal{P}'\left( \beta_l\right)  = 3\beta_l^2+4u_s\beta_l +u_s^2-b\rho_s -\frac{\mu}{\kappa\rho_s}\neq 0,\text{ for }l=1,2,3.	
	\end{equation}
	
	
	%%%%%%%%%%%%%%%%%%%%%%%%%%%%%%%%%%%%%%%%%%%%%%%%%%%%%%%%%%%%%%%%%%%%%%%%%%%%%%%%%%%%%%%%%%%%%%
	%		\begin{equation}\label{nmaxrimag}
		%			\mathcal{P}\left(\beta \right) : = \beta^3+2u_s\beta^2+\left( u_s^2-b\rho_s-\frac{\mu}{\kappa\rho_s}\right)\beta-\frac{\mu u_s}{\kappa\rho_s}=0.
		%		\end{equation}
	%		{\color{red}Why $\lim\limits_{|n|\rightarrow \infty}\tilde\tau_n^l, l=1,2,3$ exists?}
	
	%				{\color{red} $\Big($Why $\lim\limits_{|n|\rightarrow\infty}\tilde\tau_n^l$ exists for $l=1,2,3 $ ?$\Big)$ }
	%We can observe that all the roots of \eqref{nmaxrimag} are real as $\tilde\tau_n^l\in\mathbb{R}$ for $l=1,2,3$, and for all $n\in\mathbb{Z}^*$. 
	
	\textit{Step 4: Asymptotic behavior of the eigenvalues: }
	
	Using Step 1 and Step 2, we can rewrite the roots of \eqref{nmaxchar.poly} as 
	\begin{equation}\label{nmaxsetlambda}
		\lambda_n = i\beta n +\epsilon_n,
	\end{equation}
	where $\beta$ is a root of the equation \eqref{nmaxrimag} and
	\begin{equation}\label{expepsilonn}
		\epsilon_n= n\theta_n^1+i n \theta_n^2,\quad \theta_n^l\rightarrow 0 \text{ as } \left|n \right|\rightarrow \infty, \text{ for }l=1,2. 
	\end{equation}
	From \eqref{expepsilonn}, it follows that
	\begin{equation}\label{cgsepsilon_n}
		\frac{\epsilon_n}{n}\rightarrow 0 \text{ as } \left|n \right|\rightarrow \infty.
	\end{equation}
	Now, we will see the asymptotic behavior of $\epsilon_n$. From \eqref{nmaxchar.poly} and \eqref{nmaxsetlambda}, we get that $\epsilon_n$ satisfy the following equation:
	\begin{align*}
		\epsilon_n^3+\left(\frac{1}{\kappa}+\left( 3\beta +2u_s\right) in \right)\epsilon_n^2+\left(\left( -3\beta^2-4u_s\beta -u_s^2+b\rho_s +\frac{\mu}{\kappa\rho_s}\right) n^2+2\frac{\beta+u_s}{\kappa}in \right)\epsilon_n&\notag\\
		+ \frac{b\rho_s-u_s^2-2u_s\beta-\beta^2}{\kappa}n^2=0.&
	\end{align*}
	Using \eqref{nmaxnodoubleroot} and the fact that $\beta$ is root of \eqref{nmaxrimag}, we can rewrite the above equation in the following form
	\begin{align}\label{nmaxepsilonn}
		\epsilon_n=  -\frac{\frac{\epsilon_n^2}{\kappa n^2} -\frac{b\rho_s-u_s^2-2u_s\beta-\beta^2}{\kappa}}{\left(\frac{2u_s+3\beta}{2}+\frac{\epsilon_n}{in} \right)^2+\mathcal{P}'(\beta)-\left(\frac{\left(2u_s+3\beta \right)^2 }{4}+2i\frac{u_s+\beta}{\kappa n}\right)}.
	\end{align}
	Denoting numerator of \eqref{nmaxepsilonn} by $c_n$ and denominator of \eqref{nmaxepsilonn} by $d_n$, then using \eqref{nmaxnodoubleroot} and \eqref{cgsepsilon_n}, we get
	\begin{align}
		&c_n\rightarrow \frac{b\rho_s-u_s^2-2u_s\beta-\beta^2}{\kappa} \text{ as } \left|n \right|\rightarrow \infty,\label{cgsepsilon1}\\
		&d_n\rightarrow \mathcal{P}'(\beta)\neq 0 \text{ as } \left|n \right|\rightarrow \infty.\label{cgsepsilon2}
	\end{align}
	Therefore \eqref{nmaxepsilonn} is well-defined for $\left| n\right|$ large enough. Thus from \eqref{nmaxepsilonn}, \eqref{cgsepsilon1} and \eqref{cgsepsilon2}, it follows that
	\begin{equation}\label{cgsepsilon4}
		\epsilon_n\rightarrow -\frac{ \beta^2+2u_s\beta+u_s^2-b\rho_s}{\kappa \mathcal{P}'(\beta)} : = -\omega (\text{say}), \text{ as } \left|n \right|\rightarrow \infty.
	\end{equation}
	%				where $\mathcal{P}$ is defined in \eqref{nmaxrimag}.
	%Note that $3\beta^2+4u_s\beta +u_s^2-b\rho_s -\frac{\mu}{\kappa\rho_s}\neq 0$ as $\beta$ is not a double root of \eqref{nmaxrimag}.	
Next, we will find the explicit asymptotic of the eigenvalues. % we are going to see that in which order of $\frac{1}{n}$,  the convergence \eqref{cgsepsilon4} holds.
	For that, let us define
	\begin{equation}\label{nmaximplicitfunc}
		F\left( \zeta,\eta\right)=\zeta+ \frac{-\frac{1}{\kappa }\zeta^2\eta^2-\frac{b\rho_s-u_s^2-2u_s\beta-\beta^2}{\kappa} }{\left(\frac{2u_s+3\beta}{2}-{\zeta \eta} \right)^2+\mathcal{P}'(\beta)-\left(\frac{\left(2u_s+3\beta \right)^2 }{4}+2\frac{u_s+\beta}{\kappa }\eta\right)},\quad \zeta,\eta\in \mathbb{C}.
	\end{equation}
	{ Using \eqref{nmaxnodoubleroot}, we can see that $F$ is well defined and analytic in a neighborhood of $\left( -\omega,0\right) $. }Moreover
	\begin{equation*}
		F\left( -\omega,0\right)= \lim\limits_{\eta\rightarrow 0}F\left( -\omega,\eta\right)=0,
	\end{equation*}
	and
	\begin{align*}
		\frac{\partial F}{\partial \zeta}\left( -\omega,0\right)= \lim\limits_{\eta\rightarrow 0}\frac{\partial F}{\partial \zeta}\left( -\omega,\eta\right)=1\neq  0 .
	\end{align*}
	Then using the Implicit Function Theorem, there exist neighborhoods $N\left( -\omega\right) $ and $N\left(0 \right) $ such that the equation $F\left( \zeta,\eta\right)=0$ has a unique root $\zeta=G\left( \eta\right) $ in $N\left( -\omega\right) $ for any given $\eta\in N\left(0 \right) $. Moreover, the function $G\left( \eta\right) $ is single-valued and analytic in $N\left(0 \right)$, and satisfies the condition $G\left( 0\right)=-\omega$. Therefore we obtain
	\begin{align}\label{nmaximplicit2}
		\zeta=G\left( \eta\right)=&G\left( 0\right)+G'\left( 0\right)\eta+\dots\notag\\
		=& -\omega + O\left( \eta\right)\quad\text{ in }N\left(0 \right).
	\end{align}
	From \eqref{nmaxepsilonn}, \eqref{cgsepsilon4}, \eqref{nmaximplicitfunc} and \eqref{nmaximplicit2}, it follows that, for $|n|$ large enough, $\epsilon_n\in N\left( -\omega\right)$, $\frac{i}{n}\in N(0)$, $F\left( \epsilon_n,\frac{i}{n}\right)=0$ and
	\begin{equation*}
		\epsilon_n=-\omega + O\left( \frac{1}{|n|}\right).
	\end{equation*}
	Thus from \eqref{nmaxsetlambda}, we get
	\begin{equation}\label{nmaxlambdaasm}
		\lambda_n = -\omega + i\beta n + O\left( \frac{1}{|n|}\right),
	\end{equation}
	for $|n|$ large enough.
	
	Since the equation \eqref{nmaxrimag} has three distinct real roots, we get three distinct real values of $\beta$ and for each $\beta$, from \eqref{cgsepsilon4}, we get a real $\omega.$ 
	Furthermore, we can show that $\omega_1\neq\omega_2\neq\omega_3\neq 0$. Indeed, 
	suppose that $\omega_j=0$, which implies
	\begin{equation}\label{nembj}
		\beta_j^2+2u_s \beta_j+u_s^2-b\rho_s=0.
	\end{equation}
	Since $\beta_j$ is a root of the equation \eqref{nmaxrimag}, we get	
	\begin{align*}
		&\beta_j^3+2u_s\beta_j^2+\left( u_s^2-b\rho_s-\frac{\mu}{\kappa\rho_s}\right)\beta_j-\frac{\mu u_s}{\kappa\rho_s}=0,
	\end{align*}
which along with \eqref{nembj} implies $\beta_j=-u_s$. 
	This is a contradiction, since $-u_s$ is not a root of the equation \eqref{nmaxrimag} $\left( \text{ as }\mathcal{ P}\left( -u_s\right)=b\rho_s u_s >0  \right) $.
	
	
Moreover, from \eqref{nmaxnodoubleroot} and \eqref{cgsepsilon4}, we have
		\begin{equation}\label{wj-wl}
		\omega_j-\omega_l=\frac{\beta_j-\beta_l}{\kappa \mathcal{P}'(\beta_j)\mathcal{P}'(\beta_l)}\left[ -\frac{2\mu u_s^2}{\kappa \rho_s\beta_p}-\beta_p\left(2b\rho_s-2u_s^2-\frac{\mu}{\kappa\rho_s}\right)-2u_s(b\rho_s-u_s^2)\right],
		\end{equation}
		where $ j,l,p\in \{1,2,3\}, j\neq l\neq p.$ Using the fact that $\beta_j$ are distinct and using \eqref{nmaxcc}, from 
		\eqref{wj-wl}, one can show that $\omega_j\neq \omega_l$. Thus \Cref{sp} follows.
\end{proof}

\begin{rem}\label{rem multi}
	From the asymptotic behaviors \eqref{nmaxasmlambda} of the eigenvalues, it is clear that all eigenvalues are simple at least for large $n$. Thus if there exist multiple eigenvalues  depending on the system parameters $\rho_s, u_s, b, \kappa, \mu$, they are finite in numbers.
	\end{rem}

%Since the asymptotic behavior of the coefficient of the eigen vectors is needed for our later analysis, we avoid the explicit computation of the generalized eigenfunctions.
 
\begin{figure}[ht!]
	\includegraphics[width=.8\textwidth]{k=1}
	
	\caption{Eigenvalues of $\mathcal A$ in the complex plane for $|n|$ varies from $1$ to $30$ when $\mu=\rho_s=u_s=b=1$ and k=1. }     
\end{figure}
For the case of multiple eigenvalues the following results follow by introducing generalized eigenfunctions suitably (for more details see \cite[Section 7]{ahamed}). From the previous remark, we see that the multiple eigenvalues of $\mc A$, if at all they occur, are only finitely many. Since the asymptotic behaviors of the coefficients of the eigen vectors are needed for our later analysis, we avoid the explicit computation of the generalized eigenfunctions. 

Thus, now onwards, we assume the following :
\begin{equation}\label{nmaxsimpleeigen}
	\text{The spectrum of $\mathcal{A}$ has simple eigenvalues on $\mathcal {{Z}}_m$.}\tag{${{\mc H}}$}
\end{equation}


We define the family $\left\lbrace \xi_0\right\rbrace \cup \left\lbrace \xi_{n,l}\: |\: 1\leq l\leq 3, n\in\mathbb{Z}^*\right\rbrace$ as follows : 

\noindent
We choose a normalized eigenfunctions of $\mathcal{A}$ for the eigenvalue $\lambda_0=0$ defined by
\begin{equation*}
	\xi_0=\frac{1}{\sqrt{2b\pi}}\begin{pmatrix}
		1\\0\\0
	\end{pmatrix},
	%\quad \xi_1=\sqrt{\frac{\mu}{2\pi \kappa}}\begin{pmatrix}
		%	0\\0\\1
		%\end{pmatrix}
	\end{equation*}
	and for the eigenvalue $\lambda_n^l$, the normalized eigenfunction defined by
	\begin{equation}\label{nmaxequ-xi_n}
		\xi_{n,l}=\frac{1}{\theta_{n,l}}\begin{pmatrix}
			-1\vspace{1mm}\\\frac{\lambda_n^l+inu_s}{{in\rho_s}}\vspace{1mm}\\\frac{\mu\left( \lambda_n^l+inu_s\right) }{\rho_s\left( 1+\kappa\lambda_n^l\right) }
		\end{pmatrix}e^{inx},\: l\in \left\lbrace 1,2,3\right\rbrace,\: n\in\mathbb{Z}^*,
	\end{equation}
	where 
	\begin{equation}\label{nmaxequ-theta_nl}
		\theta_{n,l}=\sqrt{2{\pi}\left(b+\frac{\left| \lambda_n^l+inu_s\right|^2 }{\rho_sn^2} + \frac{\kappa\mu\left| \lambda_n^l+inu_s\right|^2 }{\rho_s^2\left| 1+\kappa\lambda_n^l\right|^2}\right) }.
	\end{equation} $\left| 1+\kappa\lambda_n^l\right|\neq 0  \left( l=1,2,3\right) $, as $-1/\kappa$ is not a root of the characteristic polynomial \eqref{nmaxchar.poly}.\\
	Similarly, we choose $\left\lbrace \xi^*_0\right\rbrace \cup \left\lbrace \xi^*_{n,l}\: |\: 1\leq l\leq 3, n\in\mathbb{Z}^*\right\rbrace$ as follows : 
	
	$\xi^*_0$ is an eigenfunction of $\mathcal{A}^*$ with the eigenvalue $\lambda_0=0$ defined by
	\begin{equation}\label{nmaxequ-xi_0^*}
		\xi^*_0=\frac{1}{\sqrt{2b\pi}}\begin{pmatrix}
			1\\0\\0
		\end{pmatrix},
	\end{equation}
and for the eigenvalue $\overline{\lambda_n^l}$, the eigenfunction defined by
	\begin{equation}\label{nmaxequ-xi_n^*}
		\xi^*_{n,l}=\frac{1}{\psi_{n,l}}\begin{pmatrix}
			1\vspace{1mm}\\\frac{\overline{\lambda_n^l}-inu_s}{{in\rho_s}}\vspace{1mm}\\-\frac{\mu\left( \overline{\lambda_n^l}-inu_s\right) }{\rho_s\left( 1+\kappa\overline{\lambda_n^l}\right) }
		\end{pmatrix}e^{inx},\: l\in \left\lbrace 1,2,3\right\rbrace,\: n\in\mathbb{Z}^*,
	\end{equation}
	where
	\begin{equation}\label{nmaxequ-psi_nl}
		\psi_{n,l}=\frac{\sqrt{{2\pi}}\left(-b+\frac{\left(  \overline{\lambda_n^l+inu_s}\right)^2 }{\rho_sn^2} - \frac{\mu\kappa\left(  \overline{\lambda_n^l+inu_s}\right)^2 }{\rho_s^2\left(  1+\kappa\overline{\lambda_n^l}\right)^2}\right)  }{\sqrt{\left(b+\frac{\left| \lambda_n^l+inu_s\right|^2 }{\rho_sn^2} + \frac{\kappa\mu\left| \lambda_n^l+inu_s\right|^2 }{\rho_s^2\left| 1+\kappa\lambda_n^l\right|^2}\right) }}.
	\end{equation} 
	Here for all $n\in\mathbb{Z}^*$, $\psi_{n,l}\neq 0$ for $l=1,2,3 $, because of the assumption \eqref{nmaxsimpleeigen}.		
	
	The choice of this family of eigenfunctions of $\mathcal{A}$ and $\mathcal{A}^*$ ensures the following lemma.
	
	\begin{lem}\label{nmaxbiorthonormality}
		Under the assumption \eqref{nmaxsimpleeigen}, the families $\left\lbrace \xi_0\right\rbrace \cup \left\lbrace \xi_{n,l}\: |\: 1\leq l\leq 3, n\in\mathbb{Z}^*\right\rbrace$ and $\left\lbrace \xi^*_0\right\rbrace \cup \left\lbrace \xi^*_{n,l}\: |\: 1\leq l\leq 3, n\in\mathbb{Z}^*\right\rbrace$ satisfy the following bi-orthonormality relations :
		\begin{align}
			&\left\langle \xi_0, \xi^*_{k,p}\right\rangle_{\mathcal{Z}}=0,\quad \left\langle \xi_{n,l}, \xi^*_{k,p}\right\rangle_{\mathcal{Z}}=\delta_k^n\delta_p^l,\: l,p\in \left\lbrace 1,2,3\right\rbrace,\: k,n\in\mathbb{Z}^*,\label{nmaxbiortho}\\
			& \left\langle \xi_0, \xi^*_{0}\right\rangle_{\mathcal{Z}}=1,\quad \left\langle \xi_{n,l}, \xi^*_{0}\right\rangle_{\mathcal{Z}}=0,\: l\in \left\lbrace 1,2,3\right\rbrace,\: n\in\mathbb{Z}^*,
		\end{align}
		where 
		$$
		\delta_n^l=\begin{cases}
			0, \: n\neq l\\
			1,\: n=l.
		\end{cases}
		$$
		
		Moreover, the asymptotic behaviors of $\theta_{n,l}$ and $\psi_{n,l}$ are as follows :
		\begin{equation}\label{nmaxcgs-theta_nl}
			\begin{aligned}
				\left|\theta_{n,l} \right|, \left|\psi_{n,l} \right|\rightarrow  \sqrt{2{\pi}\left( b+\frac{\left( \beta_l+u_s\right)^2 }{\rho_s}+\frac{\mu\left( \beta_l+u_s\right)^2 }{\kappa\rho_s^2 \beta_l^2}\right)   }, \text{ for }l=1,2,3\text{ as }|n|\rightarrow \infty.\\
			\end{aligned}
		\end{equation}
		Also if we write the eigenfunctions of $\mathcal{A}^*$ in the following way
		\begin{equation}\label{nmaxcoeffofxi}
			\xi_{n,l}^{*}=\frac{1}{\psi_{n,l}}\begin{pmatrix}
				\alpha_{n,l}^1\\\alpha_{n,l}^2\\\alpha_{n,l}^3
			\end{pmatrix} e^{inx},\quad n\in \mathbb{Z}^*, l=1,2,3, 
		\end{equation}
		then we have the following asymptotic behaviors
		\begin{align}
			&\alpha_{n,l}^1= 1, \text{ for }l=1,2,3, \text{ for all  }n\in\mathbb{Z}^*,\label{nmaxcgsalpha1}\\
			&\left|\alpha_{n,l}^2 \right| \rightarrow \left| \frac{ \beta_l+u_s }{ \rho_s}\right|, \text{ for }l=1,2,3, \text{ as }|n|\rightarrow \infty,\label{nmaxcgsalpha2}\\
			&\left|\alpha_{n,l}^3 \right| \rightarrow \left| \frac{\mu\left( \beta_l+u_s\right) }{\kappa \rho_s \beta_l}\right| , \text{ for }l=1,2,3, \text{ as }|n|\rightarrow \infty.\label{nmaxcgsalpha3}
		\end{align} 
		
	\end{lem}	
		Now, we show that the eigenfunctions of $\mathcal{A}$ form a Riesz basis in $\mathcal{Z}_m$.
	Recall from \eqref{nmaxfbasis} that $\left\lbrace \phi_{0,1}, \phi_{n,1},\phi_{n,2}, \phi_{n,3} | n\in\mathbb{Z}^*\right\rbrace $  forms an orthonormal basis in $\mathcal{Z}_m$.
	
	
	For the definition of Riesz basis in $\mathcal{Z}_m$, we refer \cite[Definition 2.5.1, Chapter 2, Section
	2.5]{TW09}. In other words, the sequence $\left\lbrace \xi_0\right\rbrace \cup \left\lbrace \xi_{n,l}\: |\: 1\leq l\leq 3\right\rbrace_{ n\in\mathbb{Z}^*}$ forms a Riesz basis in $\mathcal{Z}_m$ if there exists an
	invertible operator $Q\in\mathcal{L}\left( \mathcal{Z}_m\right) $ such that
	\begin{equation*}
		Q\phi_{0,1}=\xi_0,\: Q\phi_{n,1}=\xi_{n,1},\:Q\phi_{n,2}=\xi_{n,2},\:Q\phi_{n,3}=\xi_{n,3},\:\forall n\in\mathbb{Z}^*.
	\end{equation*}
	
	\begin{lem}\label{nmaxlem-bais_representation}
		Assume \eqref{nmaxsimpleeigen}. Let us set $\mathcal{B}_n^1=\left\lbrace \phi_{n,l}\: | l=1,2,3\right\rbrace $
		and $\mathcal{B}_n^2=\left\lbrace \xi_{n,l}\: | l=1,2,3\right\rbrace $ for all $n\in\mathbb{Z}^*$, where $\left\lbrace \phi_{n,l}\right\rbrace_{1\leq l\leq 3} $ and $\left\lbrace \xi_{n,l}\right\rbrace_{1\leq l\leq 3} $ are defined as in \eqref{nmaxfbasis} and \eqref{nmaxequ-xi_n}. Let $z_n\in \mathbf{V}_n$ be expressed in the basis $\mathcal{B}_n^1$ and $\mathcal{B}_n^2$ as follows :
		\begin{equation}\label{nmaxequ-z_n}
			z_n=\sum\limits_{p=1}^{3}c_{n,p}\phi_{n,p}=\sum\limits_{l=1}^{3}d_{n,l}\xi_{n,l},
		\end{equation}
		and let the operator $\Gamma_n\in \mathbb{C}^{3\times 3}$ be defined by
		\begin{equation}\label{nmaxwedge}
			\Gamma_n \begin{pmatrix}
				c_{n,1}\\c_{n,2}\\c_{n,3}
			\end{pmatrix}= \begin{pmatrix}
				d_{n,1}\\d_{n,2}\\d_{n,3}
			\end{pmatrix}.
		\end{equation}
		Then there exist positive constants $l_1$ and $l_2$ independent of $n$ such that
		\begin{equation*}
			\left\| \Gamma_n\right\| < l_1, \quad  \left\| \Gamma_n^{-1}\right\| < l_2,\text{ for all }n\in \mathbb{Z}^*.
		\end{equation*}
	\end{lem}
	\begin{proof}
		Let $n\in\mathbb{Z}^*$ and $z_n$ as in \eqref{nmaxequ-z_n}. Taking inner product of $z_n$ with each $\xi_{n,l}^*$ for $1\leq l\leq 3$, and using the bi-orthonormality relation \eqref{nmaxbiortho}, we get
		\begin{align*}
			d_{n,l}=&\left\langle z_n,\xi_{n,l}^*\right\rangle_{\mathcal {{Z}}}=\sum\limits_{p=1}^{3}c_{n,p}\left\langle \phi_{n,p},\xi_{n,l}^*\right\rangle_{\mathcal {{Z}}}\\
			=&\sqrt{2\pi b}\frac{\overline{\alpha_{n,l}^1}}{\overline{\psi_{n,l}}}c_{n,1}+\sqrt{2\pi \rho_s}\frac{\overline{\alpha_{n,l}^2}}{\overline{\psi_{n,l}}}c_{n,2}+\sqrt{\frac{2\pi \kappa}{\mu}}\frac{\overline{\alpha_{n,l}^3}}{\overline{\psi_{n,l}}}c_{n,3}.
		\end{align*}
		Then the matrix $\Gamma_n$ in \eqref{nmaxwedge} is given by
		\begin{equation}\label{nmaxGamma}
			\Gamma_n=\begin{pmatrix}
				\sqrt{2\pi b}\frac{\overline{\alpha_{n,1}^1}}{\overline{\psi_{n,1}}}&\sqrt{2\pi \rho_s}\frac{\overline{\alpha_{n,1}^2}}{\overline{\psi_{n,1}}}&\sqrt{\frac{2\pi \kappa}{\mu}}\frac{\overline{\alpha_{n,1}^3}}{\overline{\psi_{n,1}}}\\
				\sqrt{2\pi b}\frac{\overline{\alpha_{n,2}^1}}{\overline{\psi_{n,2}}}&\sqrt{2\pi \rho_s}\frac{\overline{\alpha_{n,2}^2}}{\overline{\psi_{n,2}}}&\sqrt{\frac{2\pi \kappa}{\mu}}\frac{\overline{\alpha_{n,2}^3}}{\overline{\psi_{n,2}}}\\
				\sqrt{2\pi b}\frac{\overline{\alpha_{n,3}^1}}{\overline{\psi_{n,3}}}&\sqrt{2\pi \rho_s}\frac{\overline{\alpha_{n,3}^2}}{\overline{\psi_{n,3}}}&\sqrt{\frac{2\pi \kappa}{\mu}}\frac{\overline{\alpha_{n,3}^3}}{\overline{\psi_{n,3}}}
			\end{pmatrix}.
		\end{equation}
		Then using \eqref{nmaxsimpleeigen}, we obtain
		\begin{align}\label{nmaxdetwedge}
			\det \left( \Gamma_n\right) = & \frac{2\pi\sqrt{2\pi b\rho_s\kappa\mu}}{in\rho_s^2\overline{\psi_{n,1}}\overline{\psi_{n,2}}\overline{\psi_{n,3}}}\begin{vmatrix}
				1&\lambda_n^1+inu_s&\frac{\lambda_n^1+inu_s}{1+\kappa\lambda_n^1}\\
				1&\lambda_n^2+inu_s&\frac{\lambda_n^2+inu_s}{1+\kappa\lambda_n^2}\\
				1&\lambda_n^3+inu_s&\frac{\lambda_n^3+inu_s}{1+\kappa\lambda_n^3}
			\end{vmatrix}\notag\\
			= & \frac{2\pi\kappa\sqrt{2\pi b\rho_s\kappa\mu}}{in\rho_s^2\overline{\psi_{n,1}}\overline{\psi_{n,2}}\overline{\psi_{n,3}}} \frac{\left( \lambda_n^1-\lambda_n^2\right) \left( \lambda_n^1-\lambda_n^3\right)\left( \lambda_n^2-\lambda_n^3\right)\left(1-\kappa inu_s \right)  }{\left( 1+\kappa\lambda_n^1\right) \left(1+\kappa\lambda_n^2 \right)\left( 1+\kappa\lambda_n^3\right) } \neq 0. 
		\end{align}
		Using \eqref{nmaxasmlambda} and \eqref{nmaxcgs-theta_nl}, we get
		\begin{equation*}
			\left| \det \left( \Gamma_n\right)\right|\rightarrow C\text{ as }|n|\rightarrow \infty,
		\end{equation*}
		where $C$ is a positive constant.
		Thus, for each $n\in\mathbb{Z}^*, \Gamma_n$ is invertible with the inverse $\Gamma_n^{-1}$.
		
		
		From the matrix $\Gamma_n$, using  \eqref{nmaxcgs-theta_nl}, \eqref{nmaxcgsalpha1}, \eqref{nmaxcgsalpha2} and \eqref{nmaxcgsalpha3}, we get a positive constant $l_1$ independent of $n$ such that
		\begin{equation*}
			\left\| \Gamma_n\right\| < l_1, \:\forall\: n\in\mathbb{Z}^*.
		\end{equation*}
		Similarly, we can get a positive constant $l_2$ independent of $n$ such that
		\begin{equation*}
			\left\| \Gamma_n^{-1}\right\| < l_2, \:\forall\: n\in\mathbb{Z}^*.
		\end{equation*}
	\end{proof}
Using \Cref{nmaxlem-bais_representation} and from the second part of  \cite[Proposition 2.5.3, Chapter 2, Section 2.5]{TW09}, we can see that the eigenfunctions of $\mathcal{A}$, $\big\lbrace\xi_0 , \xi_{n,l}\: | 1\leq l\leq 3, \: n\in \mathbb{Z}^*\big\rbrace $ forms a Riesz basis in $\mathcal{Z}_m$. Also, using \cite[Proposition 2.8.6, Chapter 2, Section 2.8]{TW09}, \cite[Definition 2.6.1, Chapter 2, Section 2.6]{TW09} and  \Cref{nmaxbiorthonormality}, we obtain
$\left\lbrace\xi_0^* , \xi^*_{n,l}\: | 1\leq l\leq 3, \: n\in \mathbb{Z}^*\right\rbrace $ forms a Riesz basis in $\mathcal{Z}_m$.
	\begin{prop}\label{nmaxbasis_representation}
		Assume \eqref{nmaxsimpleeigen} holds. Then any $z\in{\mathcal {{Z}}_m}$ can be uniquely represented as
		\begin{equation}\label{nmaxexpansion_z}
			z=\sum\limits_{n\in\mathbb{Z}^*}\sum\limits_{l=1}^{3}\left\langle z, \xi^*_{n,l} \right\rangle_{\mathcal {{Z}}} \xi_{n,l}+\left\langle z, \xi^*_{0} \right\rangle_{\mathcal {{Z}}} \xi_{0}. 
		\end{equation}
		There exist positive numbers $C_1$ and $C_2$ such that
		\begin{equation}\label{nmaxequ-est-z_0}
			C_1\left( \left| \left\langle z, \xi^*_{0} \right\rangle\right|^2+\sum_{n\in\mathbb{Z}^*}\sum_{l=1}^{3}\left| \left\langle z, \xi^*_{n,l} \right\rangle\right|^2 \right) \leq \left\|z \right\|^2_{{\mathcal {{Z}}}}\leq C_2\left(\left| \left\langle z, \xi^*_{0} \right\rangle\right|^2+ \sum_{n\in\mathbb{Z}^*}\sum_{l=1}^{3}\left| \left\langle z, \xi^*_{n,l} \right\rangle\right|^2 \right). 
		\end{equation}
		
		
		Similarly, any $z\in\mathcal {{Z}}_m$ can be uniquely represented in the basis $\left\lbrace\xi_0^* \right\rbrace \cup \{ \xi^*_{n,l}\: | 1\leq l\leq 3, \: n\in \mathbb{Z}^*\} $ by
		\begin{equation}\label{nmaxexpansion_z-adj}
			z=\sum\limits_{n\in\mathbb{Z}^*}\sum\limits_{l=1}^{3}\left\langle z, \xi_{n,l} \right\rangle_{\mathcal {{Z}}} \xi^*_{n,l}+\left\langle z, \xi_{0} \right\rangle_{\mathcal {{Z}}} \xi^*_{0}, 
		\end{equation}
		and
		\begin{equation}\label{nmaxequ-est-adj-z_0}
			\frac{1}{C_2}\left( \left| \left\langle z, \xi_{0} \right\rangle\right|^2+\sum_{n\in\mathbb{Z}^*}\sum_{l=1}^{3}\left| \left\langle z, \xi_{n,l} \right\rangle\right|^2 \right) \leq \left\|z \right\|^2_{{\mathcal {{Z}}}}\leq \frac{1}{C_1}\left(\left| \left\langle z, \xi_{0} \right\rangle\right|^2+ \sum_{n\in\mathbb{Z}^*}^{\infty}\sum_{l=1}^{3}\left| \left\langle z, \xi_{n,l} \right\rangle\right|^2 \right). 
		\end{equation}
	\end{prop}
	
	As a consequence of the above proposition, from \cite[Remark 2.6.4, Chapter 2, Section
2.6]{TW09}, we get the following result of spectrum of $\mathcal{A}$ and $\mathcal{A}^*$.
	\begin{theorem}
		Assume \eqref{nmaxsimpleeigen} holds. Then the spectrum of the operator $(\mathcal{A}, \mathcal{D}(\mathcal{A}; \mathcal {{Z}}_m))$ and $(\mathcal{A}^*, \mathcal{D}(\mathcal{A}^*; \mathcal {{Z}}_m))$ are given by 
		\begin{align*}
			\sigma(\mathcal{A}) & =\left\lbrace \lambda_0=0,  \lambda_n^j,\: n\in\mathbb{Z}^*,\: j=1,2,3\right\rbrace,\\
			\sigma(\mathcal{A}^*) & =\left\lbrace \lambda_0,  \overline{\lambda_n^j},\: n\in\mathbb{Z}^*,\: j=1,2,3\right\rbrace.
		\end{align*}
	\end{theorem}
\section{Interior controllability}\label{nmaxsecnullcont}
This section is devoted to the exact controllability of the system \eqref{nmaxeq3} by means of localized interior control acting in any of the equations. Our proof relies on the duality between the controllability of the system  and the observability of the corresponding adjoint system. Here we only prove the exact controllability result when the control acts only in the density equation. For the other two cases, the proofs follow in a similar fashion.

 We recall the final state observability of  $(\mathcal{A}^{*}, \mathcal{B}^{*}):$
For $(\mathcal{A}^*, \mathcal{D}(\mathcal{A}^*; \mathcal {{Z}}_m))$ defined in \eqref{nmaxdom-A*}-\eqref{nmaxop-A*} and $(\sigma_{T}, v_{T}, \tilde{S}_T )^{\top} \in \mathcal {{Z}}_m,$ we set 
\begin{equation*}
	(\sigma(t), v(t), \tilde{S}(t))^{\top} = \mathbb{T}_{T-t}^{*} (\sigma_{T}, v_{T}, \tilde{S}_T)^{\top} \qquad (t\in [0,T]),
\end{equation*}
where $\mathbb{T}^*$ is the $C^{0}$-semigroup generated by $(\mathcal{A}^*, \mathcal{D}(\mathcal{A}^*; \mathcal {{Z}}_m))$ on $\mathcal {{Z}}_m$. 
In view of \Cref{nmaxadj-op}, $(\sigma, v, \tilde{S})^{\top}$ belongs to $C([0,T];\mathcal {{Z}}_m)$ and satisfies:
\begin{equation}\label{nmaxeqadj}
	\begin{cases}
		\partial_t\sigma +u_s\pa_x\sigma+{\rho_s}\pa_x v =0 & \mbox{ in } (0,T) \times (0, 2\pi), \\
		\partial_t v+b \partial_{x} \sigma+u_s\pa_x v - \frac{1}{\rho_s} \partial_x\tilde{S}= 0 &  \mbox{ in } (0,T) \times (0, 2\pi),\\
		\partial_t\tilde{S}-\frac{1}{\kappa}\tilde{S} - \frac{\mu}{\kappa} \partial_x v= 0 &  \mbox{ in } (0,T) \times (0, 2\pi),\\
		\sigma(t,0) = \sigma(t, 2\pi), v(t,0) = v(t, 2\pi),\: \tilde S(t,0)=\tilde S(t,2\pi),&  \mbox{ in } (0,T), \\
		\sigma(T,x)=\sigma_{T}(x), \quad v(T,x)=v_T(x), \quad \tilde{S}(T,x)=\tilde{S}_T(x) & \mbox{ in } (0,2\pi).
	\end{cases}
\end{equation} 
 Here, at first we discuss a standard approach to deduce the observability inequality for the adjoint system \eqref{nmaxeqadj}  which essentially gives the exact controllability of the main system \eqref{nmaxeq3}. Let us consider the linear map $\mathcal{F}_T: L^2(0,T, L^2(0,2\pi)\to (L^2(0,2\pi))^3$ by $\mathcal{F}_T(f_1)= \left(\rho(T,\cdot),u(T, \cdot), S(T, \cdot)\right)^{\top}$ where $(\rho, u, S)^{\top}$ is the solution of the system \eqref{nmaxeq3} with $(\rho_0, u_0, S_0)^{\top}=(0,0,0)^{\top}$ and  $f_2=f_3=0$. It is clear that exact controllability of \eqref{nmaxeq3} is equivalent to the surjectivity of the linear map $\mathcal{F}_T$. Note that the map $\mathcal{F}_T$ is surjective if and only if there exists a constant $C>0$ such that the following inequality holds:
\begin{align}\label{eq:fstar}
	\norm{\mathcal{F}_T^*  {z}}_{L^2(0,T, L^2(0,2\pi)}\geq C\norm{ z}_{(L^2(0,2\pi))^3}, \, \text{ for all }  z \in (L^2(0,2\pi))^3.
\end{align} A direct computation shows that
$\mathcal{F}_T^*((\sigma_{T}, v_{T}, \tilde{S}_T)^{\top} )= \mathbbm{1}_{\mathcal{O}_{1}} \sigma$,  where $(\sigma,v, \tl S)$ is the solution of the adjoint system \eqref{nmaxeqadj} with the terminal data $(\sigma_{T}, v_{T}, \tilde{S}_T)$. Thus the exact controllability of the system \eqref{nmaxeq3} is equivalent to the following observability inequality: 

\begin{prop}\label{nmaxthobs1}
	Let $T>0$. 
	%	\begin{enumerate}
		%\item 
		Assume $f_2=0=f_3$. 	
		Then the system \eqref{nmaxeq3} is exactly controllable in $\mathcal {{Z}}_m$ at time $T>0$ using a control $f_1$ in $L^2(0,T;L^2(0,2\pi))$ with support in $\mathcal{O}_1$ acting only in the density equation, if and only if, there exists a positive constant $C_T>0$  such that for any $(\sigma_{T}, v_{T}, \tilde{S}_T)^{\top}\in \mathcal {{Z}}_m$, 
		$(\sigma, v, \tilde{S})^{\top}$, the solution of \eqref{nmaxeqadj}, satisfies the following observability inequality:
		\begin{multline}\label{nmaxobsden}
			\int_{0}^{2\pi} |\sigma_T(x)|^{2} \ \rd x \; + \; \int_{0}^{2\pi} |v_T(x)|^{2} \ \rd x + \; \int_{0}^{2\pi} |\tilde{S}_T(x)|^{2} \ \rd x 
			\leqslant C_T  \int_0^T\int_{\mathcal{O}_1}|\sigma(t,x)|^2\, \rd x\,\rd t. 
		\end{multline}
		\end{prop}
Next, we prove \Cref{nmaxthm_pos} by showing the observability inequality \eqref{nmaxobsden} using the following Ingham-type inequality.
\begin{prop}\label{propI-4}
Let $ T> {2\pi}\left(\frac{1}{|\beta_1|}+\frac{1}{|\beta_2|}+\frac{1}{|\beta_3|}\right)$ and $M\in \mathbb{N}$.
	Then there exist positive constants $C$ and $C_1$ depending on $T$ such that for
	$ g(t)= \displaystyle\sum_{|n|\ge M}\sum_{l=1}^{3}{a_n^l e^{\overline{\lambda_n^l} (T-t)}}$ with
	$\displaystyle{\sum_{|n|\ge M}}\sum_{l=1}^{3}|a_n^l|^2<\infty$, the following inequality holds:
	\begin{equation}\label{I-4}
		C\sum_{|n|\ge M}\sum_{l=1}^{3}|a_n^l|^2\leq \int_0^T|g(t)|^2 \, \rd t\leq C_1\sum_{|n|\ge M}\sum_{l=1}^{3}|a_n^l|^2.
	\end{equation}
\end{prop}
\begin{proof}
	The proof follows trivially from \eqref{ingham2} of  \Cref{biorthogonal} by suitable choices of $\{a_n^l\}$ and a change of variable.
\end{proof}
\subsection{Proof of \Cref{nmaxthm_pos}:}
%	\begin{proof}
	%	To prove this theorem, from the \Cref{nmaxthobs1} it is enough to prove the observability inequality \eqref{nmaxobsden}.
	Let $(\sigma_{T}, v_{T}, \tilde{S}_T )^\top \in \mathcal {{Z}}_m.$ From \Cref{nmaxbasis_representation}, we have
	\begin{equation}\label{nmaxest_alphanl}
		\begin{pmatrix}
			\sigma_{T}\\v_{T}\\\tilde{S}_{T}
		\end{pmatrix}=\sum\limits_{n\in\mathbb{Z}^*}\sum\limits_{l=1}^{3}c_{n,l} \xi^*_{n,l} + c_0\xi_0^*, \quad \mathrm{with} \quad 
		\sum\limits_{n\in\mathbb{Z}^*}\sum\limits_{l=1}^{3}\left| c_{n,l}\right|^2 + \left|c_0 \right|^2  < \infty .
	\end{equation}

	Then the corresponding solution $(\sigma(t), v(t), \tilde{S}(t))^\top$ of the adjoint problem \eqref{nmaxeqadj} can be written as
	\begin{equation}\label{nmaxrepressoln}
		\begin{pmatrix}
			\sigma\\v\\\tilde{S}
		\end{pmatrix}(t,x)=\sum\limits_{n\in\mathbb{Z}^*}\sum\limits_{l=1}^{3}c_{n,l}e^{\overline{\lambda_n^l}(T-t)} \xi^*_{n,l}(x) + c_0e^{\overline{\lambda_0}(T- t)}\xi_0^*(x).
	\end{equation}
	In particular, using the expression of $\xi^*_0$ and $\xi^*_{n,l}$ ( eq. \eqref{nmaxequ-xi_0^*}, \eqref{nmaxequ-xi_n^*} and \eqref{nmaxcoeffofxi}), from \eqref{nmaxrepressoln} we get
	\begin{align}
		&\sigma(t,x)=\sum\limits_{n\in\mathbb{Z}^*}^{\infty}\sum\limits_{l=1}^{3}\frac{c_{n,l}}{\psi_{n,l}}e^{\overline{\lambda_n^l}(T-t)}e^{inx} + \frac{c_0}{\sqrt{2b\pi}}, \quad \forall\, t\in (0,T), \quad x\in (0,2\pi),\label{nmaxrepsigma}\\
		&v(t,x)=\sum\limits_{n\in\mathbb{Z}^*}^{\infty}\sum\limits_{l=1}^{3}\frac{c_{n,l}}{\psi_{n,l}}\alpha_{n,l}^2e^{\overline{\lambda_n^l}(T-t)}e^{inx}, \quad \forall\, t\in (0,T), \quad x\in (0,2\pi),\\
		&\tilde S(t,x)=\sum\limits_{n\in\mathbb{Z}^*}^{\infty}\sum\limits_{l=1}^{3}\frac{c_{n,l}}{\psi_{n,l}}\alpha_{n,l}^3 e^{\overline{\lambda_n^l}(T-t)}e^{inx}, \quad \forall\, t\in (0,T), \quad x\in (0,2\pi).
	\end{align} 
	
	Since $\left\lbrace e^{inx}, x\in (0,2\pi)\right\rbrace_{n\in\mathbb{Z}} $ is an orthonormal basis in $L^2\left( 0,2\pi\right) $ and $\{ e^{inx},$  $ x\in (0,2\pi)\}_{n\in\mathbb{Z}^*} $ is an orthonormal basis in $L_m^2\left( 0,2\pi\right) $, then using Parseval's identity, we get
	\begin{align}\label{nmaxobsleft1}
		&\int_{0}^{2\pi} |\sigma^T(x)|^{2} \ \rd x \; + \; \int_{0}^{2\pi} |v^T(x)|^{2} \ \rd x + \; \int_{0}^{2\pi} |\tilde{S}^T(x)|^{2} \ \rd x\notag\\
		=&\sum\limits_{n\in\mathbb{Z}^*}^{\infty}\left(\left| \sum\limits_{l=1}^{3}\frac{c_{n,l}}{\psi_{n,l}}+ \frac{c_0}{\sqrt{2b\pi}}\right|^2+\left| \sum\limits_{l=1}^{3}\frac{c_{n,l}}{\psi_{n,l}}\alpha_{n,l}^2\right|^2+\left| \sum\limits_{l=1}^{3}\frac{c_{n,l}}{\psi_{n,l}}\alpha_{n,l}^3 \right|^2 \right). 
	\end{align} 
	Now using \eqref{nmaxcgsalpha2} and \eqref{nmaxcgsalpha3}, we get a positive constant $C$, independent of $n$ and a large $N\in \N$, such that
	\begin{align}\label{nmaxobsleft2}
		&\sum\limits_{|n|>N}\left(\left| \sum\limits_{l=1}^{3}\frac{c_{n,l}}{\psi_{n,l}}+ \frac{c_0}{\sqrt{2b\pi}}\right|^2+\left| \sum\limits_{l=1}^{3}\frac{c_{n,l}}{\psi_{n,l}}\alpha_{n,l}^2\right|^2+\left| \sum\limits_{l=1}^{3}\frac{c_{n,l}}{\psi_{n,l}}\alpha_{n,l}^3\right|^2 \right)\notag\\
		\leq & C \left( \sum\limits_{|n|>N}\sum\limits_{l=1}^{3}\left|\frac{c_{n,l}}{\psi_{n,l}} \right|^2 + \left|\frac{c_0}{\sqrt{2b\pi}} \right|^2\right) .
	\end{align}
Let us define 
\begin{equation*}
	a_n^l(x)=\frac{c_{n,l}}{\psi_{n,l}} e^{inx}, \text{ for } |n|>N, l=1,2,3,\text{ and } x\in (0, 2\pi).
\end{equation*}
Using \eqref{nmaxcgs-theta_nl} and \eqref{nmaxest_alphanl}, we have $\displaystyle{\sum_{|n|\ge N}}\sum_{l=1}^{3}|a_n^l(x)|^2<\infty$.  Thanks to \Cref{propI-4}, for $T> {2\pi}\left(\frac{1}{|\beta_1|}+\frac{1}{|\beta_2|}+\frac{1}{|\beta_3|}\right)$, we have 
\begin{align*}
	\sum\limits_{|n|> N} \sum\limits_{l=1}^{3}\left| \frac{c_{n,l}}{\psi_{n,l}} e^{inx}\right|^2 \leq C \int_{0}^{T}\left|\sum\limits_{\left| n\right| > N}\sum_{l=1}^{3}\frac{c_{n,l}}{\psi_{n,l}}e^{inx}e^{\overline{\lambda_n^l} (T-t)}\right|^2 \, \rd t .
\end{align*}

	\noindent
	Integrating both sides over $\mathcal{O}_1$, we get 
	\begin{align}\label{nmaxobsin1}
		\sum\limits_{|n|> N} \sum\limits_{l=1}^{3}\left| \frac{c_{n,l}}{\psi_{n,l}}\right|^2\leq C \int_{0}^{T}\int_{\mathcal{O}_1}\left|\sum\limits_{\left| n\right| > N}\sum_{l=1}^{3}\frac{c_{n,l}}{\psi_{n,l}}e^{inx}e^{\overline{\lambda_n^l} (T-t)} \right|^2 \, \rd x \, \rd t.  
	\end{align}
	Therefore only finitely many terms are leaving. Since by \ref{nmaxsimpleeigen}, all the eigenvalues are distinct, then the
	missing finitely many exponential ( for $|n| \leq N$) can be added one by one in the inequality \eqref{nmaxobsin1} as the required
	gap condition for the eigenvalues holds. For similar details one can see \cite[Chapter 4,
	Theorem 4.3 ]{micu2004introduction}.

	Thus for $T> {2\pi}\left(\frac{1}{|\beta_1|}+\frac{1}{|\beta_2|}+\frac{1}{|\beta_3|}\right)$, we get
	\begin{align}\label{nmaxobsin2}
		\sum\limits_{n\in\mathbb{Z}^*} \sum\limits_{l=1}^{3}\left| \frac{c_{n,l}}{\psi_{n,l}}\right|^2+ \left| \frac{c_0}{\sqrt{2b\pi}}\right|^2&\leq C \int_{0}^{T}\int_{\mathcal{O}_1}\left|\sum\limits_{n\in\mathbb{Z}^*}\sum_{l=1}^{3}\frac{c_{n,l}}{\psi_{n,l}}e^{inx}e^{\overline{\lambda_n^l} (T-t)}+\frac{c_0}{\sqrt{2b\pi}} \right|^2 \rd x \, \rd t\notag\\
		& = C \int_{0}^{T}\int_{\mathcal{O}_1}\left|\sigma(t,x)  \right|^2 \rd x \,\rd t.\: (\text{ using }\eqref{nmaxrepsigma})
	\end{align}
	Therefore using \eqref{nmaxobsleft1}, \eqref{nmaxobsleft2} and \eqref{nmaxobsin2}, we obtain \eqref{nmaxobsden}. Then by \Cref{nmaxthobs1}, we conclude \Cref{nmaxthm_pos}.\qed
	\begin{rem}
		In the proof of \Cref{nmaxthm_pos}, i.e, in the proof of the observability inequality \eqref{nmaxobsin2}, we assume that all the eigenvalues of $\mc A$ are simple to get the Ingham-type inequality for full class of exponentials $\{e^{-\overline{\lambda_n^l}t}: n\in \z^*, l\in\{1,2,3\}\}.$ In \Cref{rem multi}, we have mentioned that the spectrum of $\mc A$ may have finite number of multiple eigenvalues. In this case also, one can prove the observability inequality with a slight modification of the technique for simple eigenvalue case. Details analysis can be found in \cite[Section 4.1]{chowdhury2014null}. One can also refer to the book \cite[Remarks, Page 178]{KV} for a version of Ingham inequality with repeated eigenvalues. 
	\end{rem}
	\section{Boundary controllability}\label{nmax-bd}
	In this section, we will discuss the boundary controllability  of the compressible Navier–Stokes system with Maxwell's law \eqref{nmaxeq3 bd} with any of the boundary conditions \eqref{density}, \eqref{velocity} and \eqref{stress}. At first, we study the boundary controllability of the system \eqref{nmaxeq3 bd}, when the control acts only in density.
	\subsection{Control in density}
	Let us consider the system \eqref{nmaxeq3 bd} with \eqref{density}. For the reader's convenience, we recall the system:
	\begin{equation}\label{nmaxeq3 bd den}
		\left.
		\begin{aligned}
			&\partial_t\rho+u_s\partial_x\rho+\rho_s\partial_xu=0, \qquad&&\text{in }(0, T)\times (0, 2\pi),\\&\partial_tu+u_s\partial_xu+a\gamma{\rho_s}^{\gamma-2}\partial_x\rho-\frac{1}{\rho_s}\partial_xS=0,\qquad&&\text{in }(0, T)\times (0, 2\pi),\\&\partial_tS+\frac{1}{\kappa}S-\frac{\mu}{\kappa}\partial_xu=0, \qquad&&\text{in }(0, T)\times (0, 2\pi),\\&\rho(t, 0)=\rho(t, 2\pi)+q(t),\: u(t,0)=u(t,2\pi),\: S(t,0)=S(t,2\pi),\qquad&&  t\in (0, T),\\&\rho(0,x)=\rho_0(x),\quad u(0,x)=u_0(x),\quad S(0,x)=S_0(x),\qquad&&  x\in (0,2\pi).
		\end{aligned}
		\right\}
	\end{equation}
	\subsubsection{Well-posedness}
	%	Let us first denote $\mathcal {\bar{Z}}=L_m^2(0, 2\pi)\times L_m^2(0, 2\pi)\times L_m^2(0, 2\pi)$.
	Like interior controllability cases, at first let us write the system \eqref{nmaxeq3 bd den} in the following abstract form:
	  \begin{equation} \label{nmaxop-eqnbd}
		\dot{z}(t) = \mathcal{A} z(t) + \mathcal{B}_{\rho} q(t), \quad t\in (0,T), \qquad z(0) = z_{0},
	\end{equation}
	where we have set $z(t) = (\rho(t,\cdot), u(t, \cdot), S(t,\cdot))^{\top},$ $z_{0} =(\rho_{0}, u_{0}, S_0)^{\top}$.
	One can identify the operator $\mc A, \mc B$ through its adjoint by taking the scalar product of \eqref{nmaxop-eqnbd} with a vector of smooth functions $(\phi, \psi, \xi)$ and comparing it with \eqref{nmaxeq3 bd den}.
	$\mc A$ is the unbounded operator given by \eqref{op} with its domain {\small$$\mathcal D(\mathcal A; \mathcal Z_{m,m})=\left\lbrace \begin{pmatrix}
			\rho\\u\\S
		\end{pmatrix}\in \mathcal Z_{m,m}\: :\:\left(\rho,u,S \right)^\top \in H^1(0,2\pi)\times H^1(0,2\pi)\times H^1(0,2\pi),\:\begin{aligned}
			&\rho(0)=\rho(2\pi),\\ &u(0)=u(2\pi),\\&S(0)=S(2\pi).
		\end{aligned}
		\right\rbrace.
		$$}
The control operator $\mathcal{B}_{\rho} \in \mathcal{L}(\mathbb{C};\mc D (\mc A^*)')$ defined by 
	\begin{equation*}
		\ip{\mc B_{\rho} q(t)}{(\phi, \psi, \xi)}_{\mc D (\mc A^*)',\mc D (\mc A^*)}=q(t)\left(bu_s\overline{\phi(2\pi)}+b\rho_s\overline{\psi(2\pi)}\right).
	\end{equation*}
	Clearly $\mc B_{\rho}$ is well-defined as $\mc B_{\rho}q$ is continuous on $H^1(0,2\pi)\times H^1(0,2\pi)\times H^1(0,2\pi)$ (by the embedding theorem $H^1(0, 2\pi)\hookrightarrow C[0,2\pi]$). Its adjoint $\mc B_{\rho}^* \in \mathcal{L}(\mc D (\mc A^*); \mathbb{C})$ is
	\begin{equation}\label{nmaxeqcontr bd}
		\mathcal{B}_{\rho}^*(\phi, \psi, \xi)=\mathbbm{1}_{\{x=2\pi\}}\begin{pmatrix}
			bu_s\\b\rho_s\\0
		\end{pmatrix}(\phi, \psi, \xi) =bu_s {\phi(2\pi)}+b\rho_s{\psi(2\pi)}.
	\end{equation}
	One can prove that the operator $\mc B_{{\rho}}$ satisfies the following so-called admissibility condition
	\begin{equation}\label{adm}
		\int_{0}^{T}|\mc B_{\rho}^* \mathbb{T}_{T-t}^*\Phi|^2\, \rd t\leq C \norm{\Phi}_{{\mc Z}}, \forall \, \Phi \in \mc D(\mc A^*),
	\end{equation}
	where $C$ is some positive constant. The proof relies on the explicit expression of the solution of the adjoint \eqref{nmaxeqadj} and the right hand side inequality of \eqref{I-4}.
	Since $\mc A$ generates a contraction semigroup and $\mc B_{\rho}$ is an admissible operator, we can prove the following well-posedness result (see Theorem $2.37$ in page 53 of \cite{Co07}, for more details):
	\begin{prop}\label{prop wel}
		Let $T>0$, $(\rho_0,u_0, S_0)^{\top} \in L_m^2(0, 2\pi)\times L_m^2(0, 2\pi)\times L_m^2(0, 2\pi)$ and $q \in L^2(0,T)$. Then the system \eqref{nmaxeq3 bd den} has a unique solution $(\rho, u, S)^{\top}\in C\left([0,T]; (L^2(0, 2\pi)^3)\right)$ and the solution satisfies 
		\begin{align*}
				\|(\rho, u, S)\|_{C([0,T];\mathcal {{Z}})} \leqslant  C \Big(\|(\rho_0, u_0, S_0)\|_{\mathcal {{Z}}}+ \|q\|_{L^2(0,T)}\Big), 
		\end{align*}
	where $C=C(T)$ is a positive constant, independent of $\rho_0, u_0, S_0, q$ and $t$.
	\end{prop}

	\subsubsection{Exact Controllability}
	In this section, we prove the exact controllability of the system \eqref{nmaxeq3 bd den}. As interior control case, here at first we discuss the classical approach to deduce the observability inequality for the adjoint system \eqref{nmaxeqadj}  which essentially gives the exact controllability of the main system \eqref{nmaxeq3 bd den}. Let us consider the linear map $\mathcal{F}_T: L^2(0,T)\to (L^2(0,2\pi))^3$ by $\mathcal{F}_T(q)= \left(\rho(T,\cdot),u(T, \cdot), S(T, \cdot)\right)^{\top}$, where $(\rho, u, S)$ is the solution of the system \eqref{nmaxeq3 bd den} with $(\rho_0, u_0, S_0)=(0,0,0)$. It is clear that exact controllability for \eqref{nmaxeq3 bd den} is equivalent to the surjectivity of the map $\mathcal{F}_T$. Note that the map $\mathcal{F}_T$ is surjective if and only if there exists a constant $C>0$ such that the following inequality holds:
\begin{align}\label{eq:fstar bd}
	\norm{\mathcal{F}_T^* \mathbf {z}}_{L^2(0,T)}\geq C\norm{\mathbf z}_{(L^2(0,2\pi))^3}, \, \text{ for all } \mathbf z \in (L^2(0,2\pi))^3.
\end{align} A direct computation shows that
$\mathcal{F}_T^*((\sigma_{T}, v_{T}, \tilde{S}_T) )=b u_s \sigma(t,2\pi)+b\rho_s  v(t,2\pi)$, where $(\sigma,v, \tl S)$ is the solution of the adjoint system \eqref{nmaxeqadj} with the terminal data $(\sigma_{T}, v_{T}, \tilde{S}_T)$. Thus we get the following proposition.
\begin{prop}\label{nmaxthobs1 bd} 
	%	\begin{enumerate}
		%\item 
		The system \eqref{nmaxeq3 bd den} is exactly controllable in $\mathcal Z_{m,m}$ at time $T>0$ using a control $q$ in $L^2(0,T)$ acting only in the density, if and only if, there exists a positive constant $C_T>0$  such that for any $(\sigma_T, v_T, \tilde{S}_T)\in \mathcal Z_{m,m}$, 
		$(\sigma, v, \tilde{S})$, the solution of \eqref{nmaxeqadj}, satisfies the following observability inequality:
		\begin{align}\label{nmaxobsden bd}
			\int_{0}^{2\pi} |\sigma_T(x)|^{2} \, \rd x+\int_{0}^{2\pi} |v_T(x)|^{2}\, \rd x+\int_{0}^{2\pi} |\tilde{S}_T(x)|^{2} \, \rd x\leq C_T\int_{0}^{T}|bu_s \sigma(t,2\pi)+b\rho_s v(t,2\pi)|^2 \,\rd t.
		\end{align}
	\end{prop} 
Before giving the proof of the above observability inequality, we mention the following lemma which ensures that the observation term is nonzero.
	%\subsubsection{Nonzero observation.}
	\begin{lem}\label{non zero ob}
		Let us recall the eigen functions $\mc E(\mc A^*)=  \left\lbrace \xi^*_{n,l}\: |\: 1\leq l\leq 3, n\in\mathbb{Z}^*\right\rbrace$ of the unbounded operator $\mc A^*.$ Then we have the following result:
		$$\mc B_{\rho}^* \xi\neq 0, \forall \, \xi \in \mc E(\mc A^*), $$
		where $\mc B_{\rho}^*$ is the observation operator associated to the system \eqref{nmaxeq3 bd den}, defined by \eqref{nmaxeqcontr bd}.
	\end{lem}
	\begin{proof}
		Observe that, $\mc B_{\rho}^* \xi^*_{n,l}=\frac{b}{{\psi}_{n}^l}\left({ u_s \alpha^1_{n,l}+\rho_s \alpha^2_{n,l}}\right).$ 
		From the first equation of the eigen equations $\mathcal{A}^*\xi^*_{n,l}=\overline{\lambda_n^l}\xi^*_{n,l}$, we obtain
		\begin{equation}\label{nmaxeigenequ 1}
			\frac{b}{{\psi}_{n}^l}\left({ u_s \alpha^1_{n,l}+\rho_s \alpha^2_{n,l}}\right)=\frac{b\alpha_{n,l}^1\overline{\lambda_n^l}}{{\psi}_{n}^l in}=\frac{\overline{\lambda_n^l}}{{\psi}_{n}^l in}\neq 0 \, ( \text{ using } \eqref{negrealpart},\, \eqref{nmaxequ-psi_nl}) .
		\end{equation}	
Thus	$\mc B_{\rho}^* \xi^*_{n,l}\neq0.$
	\end{proof}
		%%%%%%%%%%%%%%%%%%%%%%%%%%%%%%%%%%%%%%%%%%%%%%%%%%%%%%%%%%%%%%%%%%%%%%%%%%%%%%%%%%%%%%%%%%%%%%%%%%%%%%%%%%%%%%%%%%%%%%%%%%%%%%%%%%%%%%%%%%%%%%%%%%%%%%%%%%%%%%%%%%%%%%%%%%%%%
\subsubsection{\textbf{Proof of \Cref{nmaxthm_pos bd}}}
\begin{proof}	Let $(\sigma_T, v_T, \tilde{S}_T )^\top \in \mathcal Z_{m,m}.$ From \Cref{nmaxbasis_representation}, we have
	\begin{equation}\label{nmaxest_alphanl bd}
		\begin{pmatrix}
			\sigma_T\\v_T\\\tilde{S}_T
		\end{pmatrix}=\sum\limits_{n\in\mathbb{Z}^*}\sum\limits_{l=1}^{3}c_{n,l} \, \xi^*_{n,l} \quad \mathrm{with} \quad 
		\sum\limits_{n\in\mathbb{Z}^*}\sum\limits_{l=1}^{3}\left| c_{n,l}\right|^2   < \infty .
	\end{equation}
	Then the corresponding solution $(\sigma(t), v(t), \tilde{S}(t))^\top$ of the adjoint problem \eqref{nmaxeqadj} can be written as
	\begin{equation}\label{nmaxrepressoln bd}
		\begin{pmatrix}
			\sigma\\v\\\tilde{S}
		\end{pmatrix}(t,x)=\sum\limits_{n\in\mathbb{Z}^*}\sum\limits_{l=1}^{3}c_{n,l}e^{\overline{\lambda_n^l}(T-t)} \xi^*_{n,l}(x) 
	\end{equation}
	In particular, using the expression of $\xi^*_{n,l}$  \eqref{nmaxrepressoln bd}, we get
	\begin{align}
		&\sigma(t,x)=\sum\limits_{n\in\mathbb{Z}^*}^{\infty}\sum\limits_{l=1}^{3}\frac{c_{n,l}}{\psi_{n,l}}e^{\overline{\lambda_n^l}(T-t)}e^{inx} , \quad \forall\, t\in (0,T), \quad x\in (0,2\pi),\label{nmaxrepsigma bd}\\
		&v(t,x)=\sum\limits_{n\in\mathbb{Z}^*}^{\infty}\sum\limits_{l=1}^{3}\frac{c_{n,l}}{\psi_{n,l}}\alpha_{n,l}^2e^{\overline{\lambda_n^l}(T-t)}e^{inx}, \quad \forall\, t\in (0,T), \quad x\in (0,2\pi),\label{nmaxrepv bd}\\
\label{st obs}		&\tilde S(t,x)=\sum\limits_{n\in\mathbb{Z}^*}^{\infty}\sum\limits_{l=1}^{3}\frac{c_{n,l}}{\psi_{n,l}}\alpha_{n,l}^3 e^{\overline{\lambda_n^l}(T-t)}e^{inx}, \quad \forall\, t\in (0,T), \quad x\in (0,2\pi).
	\end{align} 
then, in a similar technique used in the proof of \Cref{nmaxthm_pos}, we get a positive constant $C$, independent of $n$, and a large $N\in \mathbb{N}$, such that
\begin{align}\label{nmaxobsleft2 bd}
	&\sum\limits_{|n|> N}\left(\left| \sum\limits_{l=1}^{3}\frac{c_{n,l}}{\psi_{n,l}}\right|^2+\left| \sum\limits_{l=1}^{3}\frac{c_{n,l}}{\psi_{n,l}}\alpha_{n,l}^2\right|^2+\left| \sum\limits_{l=1}^{3}\frac{c_{n,l}}{\psi_{n,l}}\alpha_{n,l}^3\right|^2 \right)\notag\\
	\leq & C  \sum\limits_{|n|> N}\sum\limits_{l=1}^{3}\left|\frac{c_{n,l}}{\psi_{n,l}} \right|^2  .
\end{align}
Putting $x=2\pi$ in \eqref{nmaxrepsigma bd} and \eqref{nmaxrepv bd}, we have
\begin{align}
	&\sigma(t,2\pi)=\sum\limits_{n\in\mathbb{Z}^*}^{\infty}\sum\limits_{l=1}^{3}\frac{c_{n,l}}{\psi_{n,l}}e^{\overline{\lambda_n^l}(T-t)}, \quad \forall\, t\in (0,T),\label{nmaxrepsigmatwopi bd}\\
	&v(t,2\pi)=\sum\limits_{n\in\mathbb{Z}^*}^{\infty}\sum\limits_{l=1}^{3}\frac{c_{n,l}}{\psi_{n,l}}\alpha_{n,l}^2 \, e^{\overline{\lambda_n^l}(T-t)}, \quad \forall\, t\in (0,T),\label{nmaxrepvtwopi bd}\\
\label{nmaxrepstress}	&S(t,2\pi)=\sum\limits_{n\in\mathbb{Z}^*}^{\infty}\sum\limits_{l=1}^{3}\frac{c_{n,l}}{\psi_{n,l}}\alpha_{n,l}^3 \, e^{\overline{\lambda_n^l}(T-t)}, \quad \forall\, t\in (0,T).
\end{align} 
Now the observation term becomes
\begin{align}\label{obs term bd}
	|bu_s \sigma(t,2\pi)+b\rho_s v(t,2\pi)|=\left|\sum\limits_{n\in\mathbb{Z}^*}^{\infty}\sum\limits_{l=1}^{3}\frac{c_{n,l}}{\psi_{n,l}}(bu_s+\alpha_{n,l}^2 b\rho_s)e^{\overline{\lambda_n^l}(T-t)}\right|.
\end{align}
%Note that $(u_s+\alpha_{n,l}^2 b)\neq 0, \forall n\in \mathbb{Z}^* , l=1,2,3.$
Using \Cref{propI-4}, along with \eqref{nmaxobsleft2 bd} and \eqref{obs term bd}, for $T>{2\pi}\left(\frac{1}{|\beta_1|}+\frac{1}{|\beta_2|}+\frac{1}{|\beta_3|}\right)$, we have the following inequality:
\begin{align}\label{nmax ob}
	&\sum\limits_{|n|> N}\left(\left| \sum\limits_{l=1}^{3}\frac{c_{n,l}}{\psi_{n,l}}\right|^2+\left| \sum\limits_{l=1}^{3}\frac{c_{n,l}}{\psi_{n,l}}\alpha_{n,l}^2\right|^2+\left| \sum\limits_{l=1}^{3}\frac{c_{n,l}}{\psi_{n,l}}\alpha_{n,l}^3\right|^2 \right)\notag\\
	& \quad \quad\leq  C \int_{0}^{T} \left|\sum\limits_{|n|> N}\sum\limits_{l=1}^{3}\frac{c_{n,l}}{\psi_{n,l}}(u_s+\alpha_{n,l}^2 b)e^{\overline{\lambda_n^l}(T-t)}\right|^2 \, \rd t.
\end{align}
Then by \ref{nmaxsimpleeigen} and \Cref{non zero ob}, the missing finitely many exponential ( for $|n| \leq N$) can be added one by one in the inequality \eqref{nmax ob} as the required
gap condition for the eigenvalues holds. 
Thus finally we deduce the following observability inequality
\begin{align}\label{obs 21}
	&\int_{0}^{2\pi} |\sigma_T(x)|^{2} \ \rd x \; + \; \int_{0}^{2\pi} |v_T(x)|^{2} \ \rd x + \; \int_{0}^{2\pi} |\tilde{S}_T(x)|^{2} \ \rd x\notag\\
	&\quad \quad \quad \quad \leq C \int_{0}^{T} \left|u_s \sigma(t,2\pi)+b v(t,2\pi)\right|^2 \, \rd t.
\end{align}
%This completes the proof of the inequality \eqref{nmaxobsden bd},
 provided $T>{2\pi}\left(\frac{1}{|\beta_1|}+\frac{1}{|\beta_2|}+\frac{1}{|\beta_3|}\right)$. Hence \Cref{nmaxthm_pos bd} is proved.
\end{proof}


	
In a similar approach as density control case, we can also prove the exact controllability result (see \Cref{nmaxremnul}) of the system \eqref{nmaxeq3 bd}, when control acts only on the velocity or stress. In the following sections, we just indicate the changes according to the system \eqref{nmaxeq3 bd}-\eqref{velocity} and \eqref{nmaxeq3 bd}-\eqref{stress}.
\subsection{Control in velocity}
Here the control operator $\mathcal{B}_{u} \in \mathcal{L}(\mathbb{C};\mc D (\mc A^*)')$ is defined by 
\begin{equation*}
	\ip{\mc B_{u} r(t)}{(\phi, \psi, \xi)}_{\mc D (\mc A^*)',\mc D (\mc A^*)}=r\left(b\rho_s\overline{\phi(2\pi)}+\rho_s u_s\overline{\psi(2\pi)}-\overline{\xi(2\pi)}\right).
\end{equation*}
Clearly $\mc B_{u} $ is well-defined as $\mc B_{u} r$ is continuous on $H^1(0,2\pi)\times H^1(0,2\pi)\times H^1(0,2\pi)$ (by the embedding theorem $H^1(0, 2\pi)\hookrightarrow C^0[0,2\pi]$). Its adjoint $\mc B_{u} ^* \in \mathcal{L}(\mc D (\mc A^*); \mathbb{C})$ is given by
\begin{equation}\label{nmaxeqcontr bd_vel}
	\mathcal{B}_{u} ^*(\phi, \psi, \xi)=\mathbbm{1}_{\{x=2\pi\}}\begin{pmatrix}
		b\rho_s\\\rho_s u_s\\-1
	\end{pmatrix}(\phi, \psi, \xi) =b\rho_s{\phi(2\pi)}+\rho_s u_s{\psi(2\pi)}-{\xi(2\pi)}.
\end{equation}
As in the density control case, here also one can prove that the control operator satisfies the admissibility condition \eqref{adm}.
%Thus since $\mc A$ generates a contraction semigroup and $\mc B$ is an admissible operator we prove the following well-posedness result (see Theorem $2.37$ in page 53 of \cite{Co07}, for more details.):
Thus for any $(\rho_0,u_0, S_0)^{\top} \in \mathcal Z_{m,m}$ and $r \in L^2(0,T)$, the system \eqref{nmaxeq3 bd}-\eqref{velocity} has a unique solution and the solution satisfies: 
	\begin{align*}
		\|(\rho, u, S)\|_{C([0,T]; (L^2(0, 2\pi)^3)} \leqslant  C \Big(\|(\rho_0,u_0, S_0)\|_{\mathcal Z}+ \|r\|_{L^2(0,T)}\Big), 
	\end{align*}
	where $C=C(T)$ is a positive constant, independent of $\rho_0, u_0, S_0, r$ and $t$.
%\subsubsection{Exact Controllability}
\begin{prop}\label{nmaxthobs1 bd_vel} 
	%	\begin{enumerate}
		%\item 
		The system \eqref{nmaxeq3 bd}-\eqref{velocity} is exactly controllable in $\mathcal {{ Z}}_{m,m}$ at time $T>0$ using a  boundary control $r$ in $L^2(0,T)$ acting only in the velocity, if and only if, there exists a positive constant $C_T>0$  such that for any $(\sigma_T, v_T, \tilde{S}_T)^{\top}\in \mathcal Z_{m,m}$, 
		$(\sigma, v, \tilde{S})$, the solution of \eqref{nmaxeqadj}, satisfies the following observability inequality:
		\begin{multline}\label{nmaxobsden bd_vel}
			\int_{0}^{2\pi} |\sigma_T(x)|^{2} \, \rd x+\int_{0}^{2\pi} |v_T(x)|^{2} \, \rd x+\int_{0}^{2\pi} |\tilde{S}_T(x)|^{2}\, \rd x\leq C_T\int_{0}^{T}|b\rho_s \sigma(t,2\pi)\\+\rho_s u_s v(t,2\pi)-\tilde{S}(t,2\pi)|^2\, \rd t.
		\end{multline}
	\end{prop}

\begin{lem}
	Let us recall the eigen functions $\mc E(\mc A^*)= \left\lbrace \xi^*_{n,l}\: |\: 1\leq l\leq 3, n\in\mathbb{Z}^*\right\rbrace$ of the unbounded operator $\mc A^*.$ Then we have the following result:
	$$\mc B_{u} ^* \xi\neq 0, \forall \, \xi \in \mc E(\mc A^*). $$
\end{lem}
\begin{proof}
	From \eqref{nmaxeqcontr bd_vel} and \eqref{nmaxcoeffofxi}, we have
	\begin{equation}\label{nmaxb*xi}
		\mathcal{B}_{u} ^*\xi^*_{n,l}=\frac{1}{\psi_{n,l}}(b\rho_s+\alpha_{n,l}^2 \rho_s u_s-\alpha_{n,l}^3).
	\end{equation}
	The second equation of the eigen-equations $\mathcal{A}^*\xi^*_{n,l}=\overline{\lambda_n^l}\xi^*_{n,l}$ gives
	\begin{equation}\label{nmaxeigenequ}
		b\rho_s+\alpha_{n,l}^2 \rho_s u_s-\alpha_{n,l}^3=\frac{\rho_s\alpha_{n,l}^2\overline{\lambda_n^l}}{in}.
	\end{equation}	
	Thus from \eqref{nmaxb*xi}, \eqref{nmaxeigenequ} and \eqref{nmaxequ-xi_n^*}, we have
	\begin{equation*}
		\mathcal{B}_{u} ^*\xi^*_{n,l}=-\frac{\rho_s\overline{\lambda_n^l}(\overline{\lambda_n^l}-inu_s)}{n^2\psi_{n,l}\rho_s}\neq 0,
	\end{equation*}
	since all the eigenvalues have negative real part.
\end{proof}




\subsection{Control in stress}
The control operator $\mathcal{B}_{S}  \in \mathcal{L}(\mathbb{C};\mc D (\mc A^*)')$ defined by 
\begin{equation*}
\ip{\mc B_{ S} p(t)}{(\phi, \psi, \xi)}_{\mc D (\mc A^*)',\mc D (\mc A^*)}=-p\psi(2\pi).
\end{equation*}
Clearly $\mc B_{S} $ is well-defined as $\mc B_{S} p$ is continuous on $H^1(0,2\pi)\times H^1(0,2\pi)\times H^1(0,2\pi)$ (by the embedding theorem $H^1(0, 2\pi)\hookrightarrow C^0[0,2\pi]$). Its adjoint $\mc B_{S}^* \in \mathcal{L}(\mc D (\mc A^*); \mathbb{C})$ is
\begin{equation}\label{nmaxeqcontr bd st}
\mathcal{B}_{S} ^*(\phi, \psi, \xi)=\mathbbm{1}_{\{x=2\pi\}}\begin{pmatrix}
	0\\-1\\0
\end{pmatrix}(\phi, \psi, \xi) =-{\psi(2\pi)},
\end{equation}
and also $\mc B_{S} ^* \xi\neq 0, \forall \,\xi \in \mc E(\mc A^*). $

\section{Small time lack of controllability}\label{lack sec}
In this section, we study the lack of exact controllability of the system \eqref{nmaxeq3} when the localized interior control is acting on the density equation. Similar result holds true for the remaining two cases (control in velocity or control in stress). We establish the result by violating the observability inequality on an appropriate solution of the corresponding adjoint system. Our proof is in the same spirit of \cite[Section 3]{Beauchard}. We first find a candidate function as a terminal data of the transport equation and exploit the lack of null controllability of the same in small time.



Without loss of generality, we consider the control domain $\mathcal{O}_1=(l_1,l_2), 0\leq l_1<l_2<2\pi.$	
\begin{theorem}\label{lack theorem}
	Let us take $|\beta_1|T<\max\{l_1, 2\pi-l_2\}$. Then the system \eqref{nmaxeq3} is not exactly controllable in time $T$ by means of a control $f_1\in L^2\left( 0,T;L^2(\mathcal{O}_1)\right)$ acting on density equation.
\end{theorem}
\begin{proof}
	Since $|\beta_1|T<\max\{l_1, 2\pi-l_2\}$, there exists a nontrivial function $\hat \sigma_T \in C_c^{\infty}(0,2\pi)$ 
	such that the solution of the following equation \begin{equation}
		\begin{cases}
			\hat{\sigma}_t-\beta_1\hat{\sigma}_x-\omega_1\hat \sigma=0, &  \mbox{ in } (0,T) \times (0, 2\pi),\\
			\hat \sigma(t,0)=\hat \sigma(t, 2\pi), &  \mbox{ in } (0,T),\\
			\hat \sigma(T,x)=\hat \sigma_T(x), &  \mbox{ in } (0, 2\pi),
			%\tilde{\sigma}(t,x)=\tilde{\sigma}_T(x-\beta_1 t)
		\end{cases}
	\end{equation} 
	satisfies that $\text{supp}(\hat{\sigma}) \cap [0,T]\times \mathcal{O}_1 =\emptyset $ (see also \cite[Section 3]{Beauchard}). We will use the asymptotic behavior of the spectrum of the linearized operator to show the lack of controllability.  At first, we construct a high-frequency version of the candidate function $\hat \sigma_T$. Let $N>0$ be a fixed integer. Next, we define the polynomial
	\begin{equation*}
		P^N(X):=\prod_{j=-N}^{N}(X-j),
	\end{equation*}
	and the function
	\begin{equation*}
		\hat{\sigma}_T^N:=P^N\left(-i\frac{d}{dx}\right)\hat{\sigma}_T.
	\end{equation*}
	Since $\hat{\sigma}_T^N$ is the image of $\hat{\sigma}_T$ by a differential operator, we have $\text{supp}({\hat{\sigma}_T^N})\subset \text{supp}(\hat{\sigma}_T)$. Let us write
	$$
	\hat{\sigma}_T=\sum_{n \in \mathbb{Z}} a_n e^{ in   x} \text {, where } a_n=\int_{0}^{2\pi} e^{ in   x} \hat{\sigma}_T(x) d x, \forall n \in \mathbb{Z} .
	$$
	Then using the definition of $P^N$ and $\hat\sigma_T^N$, we can write $\hat\sigma_T^N$ in the following:
	\begin{equation*}
		\hat{\sigma}_T^N(x)=\sum_{n\in\mathbb{Z}}a_n\prod_{j=-N}^{N}\left(-i\frac{d}{dx}-j\right)e^{inx}=\sum_{n\in\mathbb{Z}}a_n\prod_{j=-N}^{N}\left(n-j\right)e^{inx}=\sum_{n\in\mathbb{Z}}a_nP^N(n)e^{inx},
	\end{equation*}
	for $(t,x)\in (0,T)\times(0,2\pi)$. Note that $P^N(n)=0$ for all $|{n}|\leq N$ and therefore
	\begin{equation}\label{term}
		\hat{\sigma}_T^N(x)=\sum_{{|n|}\geq N+1}a_nP^N(n)e^{inx}.
	\end{equation}
	Next, we denote the function $\hat{\sigma}^N$ is the solution of the following equation
	\begin{equation}\label{tr}
		\begin{cases}
			\hat{\sigma}^N_t-\beta_1\hat{\sigma}^N_x-\omega_1\hat \sigma^N=0, &  \mbox{ in } (0,T) \times (0, 2\pi),\\
			\hat \sigma^N(t,0)=\hat \sigma^N(t, 2\pi), &  \mbox{ in } (0,T),\\
			\hat{\sigma}^N(T, x)=\hat{\sigma}_T^N(x), &  \mbox{ in } (0, 2\pi).
			%\hat{\sigma}(t,x)=\hat{\sigma}_T(x-\beta_1 t)
		\end{cases}
	\end{equation} 
	Let $(\sigma^N, v^N, \tl S^N)$ be the solution of the following adjoint system:
	\begin{equation}\label{nmaxeqadj lack}
		\begin{cases}
			\partial_t \sigma^N +u_s\pa_x\sigma^N+{\rho_s}\pa_x v^N =0, & \mbox{ in } (0,T) \times (0, 2\pi), \\
			\partial_t  v^N+b \partial_{x}  \sigma^N+u_s\pa_x  v^N - \frac{1}{\rho_s} \partial_x\tilde{S}^N= 0, &  \mbox{ in } (0,T) \times (0, 2\pi),\\
			\partial_t\tilde{S}^N-\frac{1}{\kappa}\tilde {S}^N - \frac{\mu}{\kappa} \partial_x  v^N= 0, &  \mbox{ in } (0,T) \times (0, 2\pi),\\
			\sigma^N(t,0) = \sigma^N(t, 2\pi),  v^N(t,0) = v^N(t, 2\pi),\:  \tilde S^N(t,0)=\tilde S^N(t,2\pi),&  \mbox{ in } (0,T), \\
			\sigma^N(T,x)=\hat \sigma^N_{T}(x), \quad  v^N(T,x)= \hat v^N_T(x), \quad \tilde{S}^N(T,x)=\hat{S}^N_T(x), & \mbox{ in } (0,2\pi),
		\end{cases}
	\end{equation}
	where $\displaystyle (\hat\sigma^N_{T}, \hat v^N_T, \hat S^N_T)=\sum_{|n|\geq N+1}c^1_n  \,\xi^*_{n,1},  \text{ where } \frac{c^1_n}{\psi_{n,1}} =a_n^N=a_n P^N(n)$ and $\psi_{n,1}, \,\xi^*_{n,1}$ are defined in \eqref{nmaxcoeffofxi}.
	We write the solutions of the systems \eqref{tr} and \eqref{nmaxeqadj lack} respectively as
	\begin{align}
		\hat{\sigma}^N(t,x)&=\sum_{|{n}|\geq N+1}a^N_n \, e^{(-\beta_1in-\omega_1)(T-t)}e^{inx},\\
		\sigma^N(t,x)&=\sum_{|{n}|\geq N+1}a^N_n \, e^{\overline{\lambda_n^1}(T-t)}e^{inx}.
	\end{align}
	Now, we prove that the solution component $\sigma^N$ of \eqref{nmaxeqadj lack}  approximates the solution $\hat{\sigma}^N$ of \eqref{tr}. Indeed,
	\begin{align*}
		&\norm{\sigma^N(t,\cdot)-\hat{\sigma}^N(t,\cdot)}_{L^2(0,2\pi)}^2\\
		&\leq C\sum_{|{n}|\geq N+1}|{a^N_n}|^2\abs{e^{\overline{\lambda_n^1}(T-t)}-e^{\left({-\omega_1-i \beta_1n }\right)(T-t)}}^2\\
		%&\leq\sum_{\mod{n}\geq N+1}\mod{a_n}^2\mod{P^N(n)}^2\norm{e^{\bar{u}in(T-t)}e^{-\frac{\mu_0n}{2}(n-\sqrt{n^2-\frac{4b\bar{\rho}}{\mu_0^2}})(T-t)}-e^{(\bar{u}in-\frac{b\bar{\rho}}{\mu_0})(T-t)}}_{L^2(0,T)}^2\\
		%&\leq\sum_{|{n}|\geq N+1}|{a^N_n}|^2\norm{e^{-\frac{\mu_0n}{2}\left(n-\sqrt{n^2-\frac{4b\bar{\rho}}{\mu_0^2}}\right)(T-t)}-e^{-\frac{b\bar{\rho}}{\mu_0}(T-t)}}_{L^2(0,T)}^2\\
		&\leq C\sum_{|{n}|\geq N+1}\frac{1}{|{n}|^2}|{a^N_n}|^2,
	\end{align*}
	for all $t\in[0,T]$ and therefore performing integration over $[0,T]$ we have
	\begin{equation*}
		\norm{\sigma^N-\hat{\sigma}^N}_{L^2(0,T, L^2(0,\pi))}^2\leq\frac{C}{|{N}|^2}\sum_{|{n}|\geq N+1}|{a^N_n}|^2.
	\end{equation*}
	Using triangle inequality, we deduce
	\begin{align*}
		\norm{\sigma^N}_{L^2(0,T, L^2(\mathcal{O}_1))}^2&\leq 	\norm{ \sigma^N-\hat{\sigma}^N}_{L^2(0,T, L^2(\mathcal{O}_1)}^2+	\norm{\hat \sigma^N}_{L^2(0,T, L^2(\mathcal{O}_1))}^2.
	\end{align*}
	Since the support of $\hat \sigma_N$ does not intersect the domain $[0,T]\times \mathcal{O}_1$, the second term of the right hand sides vanishes. Thus we get
	\begin{equation}\label{lack 1}
		\norm{ \sigma^N}_{L^2(0,T, L^2(\mathcal{O}_1))}^2\leq \frac{C}{|{N}|^2}\sum_{|{n}|\geq N+1}|{a^N_n}|^2. 
	\end{equation}
	Next, if possible we assume that the observability for the system \eqref{nmaxeqadj lack} holds. Therefore we have the following:
	\begin{equation}\label{lack 2}
		\int_{0}^{2\pi} |\hat\sigma^N_T(x)|^{2} \ \rd x \; + \; \int_{0}^{2\pi} | \hat v^N_T(x)|^{2} \ \rd x + \; \int_{0}^{2\pi} |\hat{S}^N_T(x)|^{2} \ \rd x 
		\leqslant C_T  \int_0^T\int_{\mathcal{O}_1}|\sigma^N(t,x)|^2\, \rd x\,\rd t.
	\end{equation}
	Using \eqref{term}, \eqref{lack 1} \eqref{lack 2}, we have
	\begin{equation}\label{lack 3}
		\int_{0}^{2\pi} |\hat\sigma^N_T(x)|^{2} \ \rd x\leq	\norm{\hat \sigma^N}_{L^2(0,T, L^2(\mathcal{O}_1))}^2\leq \frac{C}{|{N}|^2}\sum_{|{n}|\geq N+1}|{a^N_n}|^2=\frac{C}{|{N}|^2}\int_{0}^{2\pi} |\hat \sigma^N_T(x)|^{2} \ \rd x,
	\end{equation}
	which gives a contradiction.
\end{proof}
\section{Rapid exponential stabilization}\label{stab} Let us recall the control system \eqref{nmaxeq3 bd}. The goal of this section is to study the boundary stabilization issues for \eqref{nmaxeq3 bd} with single boundary control force. More precisely, we construct a stationary feedback law $q(t)$, $r(t)$ or $p(t)$) of the form $\Pi\left(\rho(t,\cdot), u(t,\cdot), S(t,\cdot)\right)$ such that the solution of the closed loop system \eqref{nmaxeq3 bd} decays exponentially to zero at any prescribed decay rate. At first we describe Urquiza's approach \cite{UR05} which is the key argument of this section.
\subsection{Urquiza's method}
Let us consider an abstract control system
\begin{equation} \label{eq:abs}
\begin{cases}
	\dot{y}(t)=\mc Ay(t)+ \mc B q(t) , \quad t\in (0,T),\\
	y(0)=y_0,
\end{cases}
\end{equation}
where $y(t)\in \mc H, y_0\in \mc H, q\in L^2(0,T)$, $\mc B$ is an unbounded operator from $\cplx$ to $\mc H$. $\mc A:D(\mc A)\subset \mc H\to \mc H$ is an unbounded operator and $\mc D(\mc A)$ is dense in $\mc H.$ To employ the method of Urquiza, one needs to take the following assumptions on the operator $\mc A$ and $\mc B$
\begin{itemize}
\item[(H1)] $\mc A$ is the infinitesimal generator of a strongly continuous group $\{e^{t \mc A}\}_{t\in \rea}$ on $\mc H$.
\item[(H2)] $\mc B:\cplx\to \mc D(\mc A^*)'$ is linear and continuous.
\item[(H3)] \textit{Regularity property.} For all $T>0$ there exists $C(T)>0$ such that
\begin{equation*}
	\int_{0}^{T}\abs{\mc B^*e^{-t\mc A^*} y_T}^2\leq C \norm{y_T}^2_{\mc H}, \quad \forall \,y_T \in \mc D(\mc A^*).
\end{equation*}
\item[(H4)] \textit{Controllability property.} There are two constants $T>0$ and $c(T)>0$ such that
\begin{equation*}
	\int_{0}^{T}\abs{\mc B^*e^{-t\mc A^*} y_T}^2\geq c \norm{y_T}^2_{\mc H}, \quad \forall \,  y_T \in \mc D(\mc A^*).	
\end{equation*}
\end{itemize}
These hypotheses lead to the the following stabilization result. Its proof relies on algebraic Riccati equation associated with the linear quadratic regulator problem \cite{IL88}. Let us introduce the growth bound of the semigroup $\{e^{\mc A t}\}_{t\geq 0}$ of continuous linear operator as follows:
\begin{equation*}
g(\mc A)=\inf_{t> 0}\frac{1}{t}\log\norm{e^{t\mc A}}_{\mathcal{L}(\mc H)}.
\end{equation*}
\begin{theorem}\label{thm}[Urquiza \cite{UR05}, Theorem $2.1$]
Consider operators $\mc A$ and $\mc B$ under assumptions (H1)-(H4). For any $\omega>\max\left(g(-\mc A),0\right)$, we have
\begin{itemize}
	\item[(i)] The symmetric positive operator $\Lambda_{\omega}$ defined by
	\begin{equation*}
		\ip{\Lambda_{\omega}x}{z}_{\mc H}=\int_{0}^{\infty}\ip{\mc B^*e^{-\tau(\mc A+\omega I)^*}x}{\mc B^*e^{-\tau(\mc A+\omega I)^*}z}_{\cplx}d\tau, \, \forall \, x, z\in \mc H
	\end{equation*}
	is coercive and is an isomorphism on $\mc H$.
	\item[(ii)] Let $\Pi_{\omega}:=-\mc B^*\Lambda_{\omega}^{-1}$. The operator $\mc A+\mc B\Pi_{\omega}$ with $\mc D(\mc A+\mc B\Pi_{\omega})=\Lambda_{\omega}(\mc D(\mc A^*))$ is the infinitesimal generator of a strongly continuous semigroup on $\mc H$.
	\item[(iii)] The closed-loop system \eqref{eq:abs} with $q=\Pi_{\omega}(y)$ is exponentially stable that is,
	\begin{equation*}
		\norm{e^{t(\mc A+\mc B \Pi_{\omega})}x}_{\mc H}\leq C e^{\left(-2\omega+g(-\mc A)\right) t} \norm{x}_{\mc H}, \forall x \in \mc H,\end{equation*} where $C$ is a positive constant.
\end{itemize}
\end{theorem}
We utilize \Cref{thm} to prove the complete exponential stabilization of the linearized compressible Navier-Stokes equation with Maxwell's law \eqref{nmaxeq3 bd}-\eqref{density} with boundary feedback law.
\subsection{Control in density}
To apply the method mentioned above we need to show that all the assumptions (H1)-(H4) hold true for our system \eqref{nmaxeq3 bd den}. 
Let us recall the corresponding spatial operator $(\mc A, \mc D(\mc A; \mathcal Z_{m,m})$, where $\mathcal Z_{m,m}=L_m^2(0, 2\pi)\times L_m^2(0, 2\pi)\times L_m^2(0, 2\pi)$.

%We also recall that $\mc A={\mc A}_1+{\mc A}_2,$ where ${\mc A}_1$ is skew-adjoint and ${\mc A}_2$ is bounded.
It can be check that the operator $(\mc A, \mc D(\mc A; \mathcal Z_{m,m})$ generates a strongly continuous group $\{\mathbb{T}_t\}_{t\in \rea}$ of continuous linear operator. Hence (H1) holds. Also (H2) is true, see \eqref{nmaxeqcontr bd}. We get (H3) and (H4) by Ingham inequality.
\subsection{Feedback control law and proof of \Cref{nmaxthm_pos st 1}}
In this section we employ Urquiza's method to construct the feedback law for our system \eqref{nmaxeq3 bd den}. Let us take $ U^1_0=(\sigma^1_0, v^1_0, \tilde{S}^1_0), U^2_0=(\sigma^1_0, v^1_0, \tilde{S}^1_0) \in \mathcal Z_{m,m}.$ We consider the bilinear form
\begin{equation*}
a_w(U^1_0, U^2_0)=\int_{0}^{\infty}e^{-2\omega t}\left(b u_s {\sigma^1(t, 2\pi)}+b\rho_s{v^1(t,2\pi)}\right) \overline{\left(b u_s {\sigma^2(t, 2\pi)}+b\rho_s{v^2(t,2\pi)}\right) }\, \rd t,
\end{equation*}
where $U^1=(\sigma^1, v^1, \tilde{S}^1)$ and $U^2=(\sigma^2, v^2, \tilde{S}^2)$ are the solutions of the following
systems respectively
\begin{equation}\label{eq:adj periodic1}
\begin{cases}
	\partial_t\sigma^1 +u_s\pa_x\sigma^1+{\rho_s}\pa_x v^1 =0 & \mbox{ in } (0,T) \times (0, 2\pi), \\
	\partial_t v^1+b \partial_{x} \sigma^1+u_s\pa_x v^1 - \frac{1}{\rho_s} \partial_x\tilde{S}^1= 0 &  \mbox{ in } (0,T) \times (0, 2\pi),\\
	\partial_t\tilde{S}^1-\frac{1}{\kappa}\tilde{S}^1 - \frac{\mu}{\kappa} \partial_x v^1= 0 &  \mbox{ in } (0,T) \times (0, 2\pi),\\
	\sigma^1(t,0) = \sigma^1(t, 2\pi), v^1(t,0) = v^1(t, 2\pi),\: \tilde S^1(t,0)=\tilde S^1(t,2\pi),&  \mbox{ in } (0,T), \\
	%\sigma^1(T,x)=\sigma^1_T(x), \quad v(T,x)=v_T(x), \quad \tilde{S}(T,x)=\tilde{S}_T(x) & \mbox{ in } (0,2\pi)\\.
	\left(\sigma^1(T, x), v^1(T,x), \tilde{S}^1(T,x)\right) = \mathbb{T}_{-T} \left(\sigma^1_0(x), v^1_0(x), \tilde{S}^1_0(x)\right) & x \in (0, 2\pi).
\end{cases}
\end{equation} and\begin{equation}\label{eq:adj periodic2}
\begin{cases}
	\partial_t\sigma^2 +u_s\pa_x\sigma^2+{\rho_s}\pa_x v^2 =0 & \mbox{ in } (0,T) \times (0, 2\pi), \\
	\partial_t v^2+b \partial_{x} \sigma^2+u_s\pa_x v^2 - \frac{1}{\rho_s} \partial_x\tilde{S}^2= 0 &  \mbox{ in } (0,T) \times (0, 2\pi),\\
	\partial_t\tilde{S}^2-\frac{1}{\kappa}\tilde{S}^2 - \frac{\mu}{\kappa} \partial_x v^2= 0 &  \mbox{ in } (0,T) \times (0, 2\pi),\\
	\sigma^2(t,0) = \sigma^2(t, 2\pi), v^2(t,0) = v^2(t, 2\pi),\: \tilde S^2(t,0)=\tilde S^2(t,2\pi),&  \mbox{ in } (0,T), \\
	%\sigma^1(T,x)=\sigma^1_T(x), \quad v(T,x)=v_T(x), \quad \tilde{S}(T,x)=\tilde{S}_T(x) & \mbox{ in } (0,2\pi)\\.
	\left(\sigma^2(T, x), v^2(T,x), \tilde{S}^2(T,x)\right) = \mathbb{T}_{-T} \left(\sigma^2_0(x), v^2_0(x), \tilde{S}^2_0(x)\right) & x \in (0, 2\pi).
\end{cases}
\end{equation}
Let us define the operator $\Lambda_{\omega}:\mathcal Z_{m,m}\to \mathcal Z_{m,m}$ satisfying the following
\begin{equation}\label{lambda}
\ip{\Lambda_{\omega} U^1_0 } {U^2_0}_{\mathcal Z_{m,m}}=a_{\omega}(U^1_0 ,U^2_0), \,  U^1_0,U^2_0 \in \mathcal Z_{m,m}.
\end{equation}
Next we see that \begin{align*}
\nonumber a_w(U^1_0 ,U^2_0)=&\int_{0}^{\infty}e^{-2\omega t}\left(b u_s {\sigma^1(t, 2\pi)}+b \rho_s{v^1(t,2\pi)}\right)\,\, \overline{\left(b u_s {\sigma^2(t, 2\pi)}+b \rho_s{v^2(t,2\pi)}\right)} \,\rd t\\
\nonumber	=&\int_{0}^{\infty}e^{-2\omega t}{\mc B^*_{\rho}} U^1(t) \,\,  \overline{{\mc B^*_{\rho}} U^2(t)} \, \rd t\\
=&\int_{0}^{\infty}e^{-2\omega t}\left({\mc B^*_{\rho}} \mathbb{T}^*_{T-t}\mathbb{T}^*_{-T}U^1_0\right) \overline{\left({\mc B^*_{\rho}}\mathbb{T}^*_{T-t}\mathbb{T}^*_{-T}U^2_0\right)}\, \rd t\\
=&\int_{0}^{\infty}e^{-2\omega t}\left({\mc B^*_{\rho}}\mathbb{T}^*_{-t}U^1_0\right) \overline{\left({\mc B^*_{\rho}}\mathbb{T}^*_{-t} U^2_0\right)}\, \rd t.
\end{align*}
Therefore from \eqref{lambda}, we have
\begin{equation}\label{lambda 1}
\ip{\Lambda_{\omega} U^1_0 } {U^2_0}_{\mathcal Z_{m,m}}=\int_{0}^{\infty}e^{-2\omega t}\ip{{\mc B^*_{\rho}}\mathbb{T}^*_{-t}U^1_0}{\, {\mc B^*_{\rho}}\mathbb{T}^*_{-t} U^2_0}_{\cplx} \, \rd t.
\end{equation}
Thanks to Theorem \ref{thm}, the operator $\Lambda_{\omega}$ defined by \eqref{lambda} is coercive and isomorphism. Finally, let us define the operator ${\Pi}_{\omega}: \mathcal Z_{m,m} \to \mathbb{C}$ by
\begin{align*}
{\Pi}_{\omega}(\mathbf{z})=-\left(b u_s {\sigma^1_0( 2\pi)}+b\rho_s{v^1_0(2\pi)}\right) ,	\end{align*} where $U^1_0=(\sigma^1_0, v^1_0, \tilde{S}^1_0)$ is the solution of the following Lax-Milgram  problem
\begin{equation*}
a_{\omega}(U^1_0, U^2_0)=\ip{\mathbf{z}}{U^2_0}, \, \forall \, U^2_0 \in \mathcal Z_{m,m}.
\end{equation*}
Hence we obtain $\ip{\Lambda_{\omega} U^1_0 } {U^2_0}_{\mathcal Z_{m,m}}=\ip{\mathbf{z}}{U^2_0}, \forall \, U^2_0 \in \mathcal Z_{m,m}.$ This gives $\Lambda_{\omega}U^1_0=\mathbf{z}.$ It follows that $U^1_0=\Lambda_{\omega}^{-1} \mathbf{z}.$ Thus we have ${\Pi}_{\omega}=-{\mc B}^*\Lambda_{\omega}^{-1}.$
Thanks to Theorem \ref{thm}, rapid exponential stabilization for the system \eqref{nmaxeq3 bd den} is established by means of the feedback law $q(t)={\Pi}_{\omega}(\rho(t,\cdot), v(t,\cdot), S(t,\cdot))$. More precisely, we get a positive constant $C$ such that the solution of \eqref{nmaxeq3 bd den} satisfies the following estimate
{\small
\begin{align}\label{rho}
\norm{\rho(t, \cdot )}_{L^2(0,2\pi)}+\norm{u(t, \cdot )}_{L^2(0,2\pi)}+\norm{S(t, \cdot )}_{L^2(0,2\pi)}& \leq C e^{-\nu t} \bigg[ \norm{\rho_0}_{L^2(0,2\pi)}+\norm{u_0}_{L^2(0,2\pi)}+\norm{S_0}_{L^2(0,2\pi)} \bigg].
\end{align}
}
%where $\omega> \frac{4\pi^2}{1+4\pi^2}.$
%\subsection{Control in velocity/ stress}
A similar process will give the complete stabilization for the control systems \eqref{nmaxeq3 bd}-\eqref{velocity} and \eqref{nmaxeq3 bd}-\eqref{stress}.

\section{Construction of the biorthogonal family and Ingham-type inequality}\label{biorthogonal}
This section is devoted to the proof of a suitable Ingham-type inequality which essentially helps to derive the required observability inequality. The proof of the Ingham-type inequality relies on the construction of a biorthogonal family of $\{e^{-{\lambda_k^l}t}\}_{k\in \mathbb{Z}^*, l\in \{1,2,3\}}$.  All the constants used in this section while finding the relevant estimates are generic, which may vary from line to line. 


Let us assume \eqref{nmaxsimpleeigen}.  We recall the expressions of the eigenvalues of $\mc A$ in the following manner.
\begin{equation}\label{nmaxasmlambdanew}
	\left.
	\begin{aligned}
		\lambda_n^1=&-\omega_1+i \beta_1n + O\left( \frac{1}{|n|}\right), & n\in \mathbbm{Z}^*,\\
		\lambda_n^2=&-\omega_2+i \beta_2n + O\left( \frac{1}{|n|}\right),  & n\in \mathbbm{Z}^*,\\
		\lambda_n^3=&-\omega_3+i \beta_3n + O\left( \frac{1}{|n|}\right)  & n\in \mathbbm{Z}^*.
	\end{aligned}
	\right\}
\end{equation}
We further denote the following notation:
\begin{equation*}
	\Sigma=\{(n,j): n\in \z^*, 1\leq j\leq 3\}.
\end{equation*}
\begin{lem}[Gap properties of the spectrum]\label{gap}
	For any $(n,j), (k,l)\in \Sigma$, there exists $\hat{C}>0$ depending on $\omega_1, \omega_2, \omega_3, \beta_1, \beta_2, \beta_3$, such that 
	\begin{equation}\label{gap con}
		\abs{\lambda_n^j-\lambda_k^l}\geq \hat{C}.
	\end{equation}
\end{lem}  
\begin{proof}
	Note that $\omega_1, \omega_2, \omega_3$ are distinct and $\beta_1, \beta_2, \beta_3$ are also distinct. Thanks to the asymptotic behavior of the eigenvalues \eqref{nmaxasmlambdanew}, we have a large $N\in \N$ such that for $(n,j), (k,j)\in \Sigma, \text{ with } n\neq k$ and $|n|, |k|>N$,
	\begin{equation*}
		\abs{\lambda_n^j-\lambda_k^j}\geq \abs{\Im({\lambda_n^j-\lambda_k^j})}\geq C \abs{\beta_j(n-k)}\geq C |\beta_j|.
	\end{equation*}
	Analogously, 
	for $(n,j), (k,l)\in \Sigma, j\neq l$, $|n|, |k|>N$, we have
	\begin{equation*}
		\abs{\lambda_n^j-\lambda_k^l}\geq \abs{\Re({\lambda_n^j-\lambda_k^l})}\geq C\abs{\omega_j-\omega_l}.
	\end{equation*}
	As the eigenvalues are simple, we also find the gap condition \eqref{gap con} for finite number of the lower frequencies. 
\end{proof}
Let us first define the exponential type and sine type functions.
\begin{defn}[Entire functions of exponential type]
	An entire function $f$ is said to be of \textbf{exponential type $A$} if there exist positive constants $A, B$ such that
	\begin{align*}
		|f(z)|\leq B e^{A|z|}, \, z \in \cplx,
	\end{align*}
	and of \textbf{exponential type at most $A$}, if for any $\epsilon>0$, there exits $B_\epsilon >0$ such that
	\begin{align*}
		|f(z)|\leq B_\epsilon\, e^{(A+\epsilon)|z|}, \, z \in \cplx.
	\end{align*}
\end{defn}
\begin{defn}(Sine type function)
	An entire function $f$ of exponential type $\pi$ is said to be of sine type if
	\begin{itemize}
		
		\item the zeros of $f(z)$, say {$\mu_k$} satisfy the gap condition, i.e., there exists $\delta>0$ such that $|\mu_k-\mu_l|>\delta$ for $k \neq l$, and
		\item there exist positive constants $C_1$,$C_2$ and $C_3$ such that
		$$C_1e^{\pi|y|}\leq |f(x+iy)| \leq C_2e^{\pi|y|},\,\forall \,x,y \in \rea \text{ with }|y|\geq C_3. $$
	\end{itemize}
\end{defn}
The following proposition states some important properties of sine-type functions:
\begin{prop}
	Let $f$ be a sine type function, and let $\{\mu_k\}_{k\in \mathcal{I}}$ with $\mathcal{I}\subset\z$ be its sequence of zeros. Then, we have:
	\begin{itemize}
		\item for any $\epsilon>0$, there exist constants $K_{\epsilon},\tl K_{\epsilon}>0$ such that
		$$K_{\epsilon}e^{\pi|y|}\leq |f(x+iy)|\leq \tl K_{\epsilon}e^{\pi|y|}, \text{ if dist$\left(x+iy,\{\mu_k\}\right)>\epsilon$},$$
		\item there exist some constants $K_1,K_2>0$ such that
		$$K_1<|f'(\mu_k)|<K_2, \quad \forall k\in\mathcal{I}.$$
	\end{itemize}
\end{prop} 
The main goal of this section is to find a class of entire functions $\mathcal{G}=\{\Psi_k^j\}_{(k,j)\in \Sigma }$ with the following properties:
\begin{itemize}
	\item[1.]	The family $\mathcal{G}$ contains entire functions of exponential type $\frac{T}{2}$. That means there exists a positive constant $C$ such that
	\begin{align}\label{exp typ}
		%\abs{\Psi_0(z)}\leq C e^{\frac{T}{2}\abs{z}} \text{ and } 	
		\abs{\Psi_k^j(z)}\leq C e^{\frac{T}{2}\abs{z}}, \quad \forall z\in \cplx, \,\, (k,j)\in \Sigma. 
		%\quad \forall k\in\zahl\setminus\{-1,0\}.
	\end{align}
	\item[2.] All the members of $\mathcal{G}$ are square-integrable on the real line, i.e.,
	\begin{align}\label{sq intg1}
		\int_{\rea}\abs{\Psi_k^j(x)}^2\, \rd x<\infty. %\int_{\rea}\abs{\Psi_0(x)}^2 dx< \infty.
	\end{align}
\end{itemize}

Let us now state the celebrated Paley-Wiener theorem, from which we can conclude about the desired biorthogonal family using the family $\mathcal{G}$.
\begin{theorem}[\textbf{Paley-Wiener}]\label{Paley-Wiener}
	Let $f$ be an entire function of exponential type $A$ and suppose
	\begin{align*}\int_{-\infty}^{\infty}|f(x)|^2 \, \rd x < \infty.
	\end{align*}
	Then there exists a function $\phi \in L^2(-A, A)$ with the following representation
	\begin{align*}
		f(z)=\int_{-A}^{A}e^{izt}\phi(t) \, \rd t,\, z\in \cplx.
	\end{align*}
\end{theorem}

Thus if we have the existence of such family $\mathcal{G}$ of entire functions satisfying \eqref{exp typ}-\eqref{sq intg1}, then  applying the Paley-Wiener theorem for the same, one can get a family of functions $\mathcal{J}=\{\Theta_k^j,\}_{(k,j)\in \Sigma}$ supported in $[-\frac{T}{2}, \frac{T}{2}]$, such that the following representation holds
\begin{align}\label{representation p}
	\Psi_k^j(z)=\int_{-\frac{T}{2}}^{\frac{T}{2}}e^{izt}\Theta_k^j(t) dt,\, z\in \cplx.
\end{align}
Clearly, $\Theta_k^j$ are the Fourier transformations of $\Psi_k^j$ respectively, for $ (k,j)\in \Sigma.$
Also, by Plancharel's Theorem, we have 
\begin{align*}
	\int_{\rea}\abs{\Psi_k^j(x)}^2=2\pi\int_{-\frac{T}{2}}^{\frac{T}{2}}\abs{\Theta_k^j(t)}^2\, \rd t.
\end{align*}
Now we are in the position of formulating the construction of the family $\mathcal{G}$ satisfying \eqref{exp typ}-\eqref{sq intg1}. 
%Existence of $\mathcal{E}$ will ensure the existence of the biorthogonal family $\mathcal{B}.$
Let us first introduce the following entire function, which has simple zeros exactly at $i\overline{\lambda_k^j}:$
\begin{align}\label{eq:P}
	P(z)=z^3\prod_{(k,j)\in \Sigma}\left(1-\frac{z}{i\overline{\lambda_k^j}}\right).
\end{align}
\begin{prop}
	Let $P$ be the canonical product defined in \eqref{eq:P}. Then $P$ is an entire function of exponential type $\pi\left(\frac{1}{|\beta_1|}+\frac{1}{|\beta_2|}+\frac{1}{|\beta_3|}\right)$, which satisfies the following properties:
	\begin{itemize}
		\item There exists a positive constant $C>0$ such that
		\begin{equation}\label{bound for p}
			|P(x)|\leq C, \quad \forall x \in \rea.
		\end{equation}
		\item  There exists  constant $C_2>0$ such that 
		\begin{equation}\label{p prime}
			\left|P'\left(i\overline{\lambda_k^j}\right)\right|\geq C_2, \forall (k,j) \in \Sigma.
		\end{equation}
	\end{itemize}
\end{prop}
\begin{proof} Let us denote $\nu_n^j=\frac{1}{\beta_j}\lambda_n^j.$ thus, we have:
	\begin{equation}\label{nmaxasmlambda 1}
		\begin{cases}
			\nu_n^1=-\frac{\omega_1}{\beta_1}+i n + O\left( \frac{1}{|n|}\right),\\
			\nu_n^2=-\frac{\omega_2}{\beta_2}+i n + O\left( \frac{1}{|n|}\right),\\
			\nu_n^3=-\frac{\omega_3}{\beta_3}+in + O\left( \frac{1}{|n|}\right).
		\end{cases}
	\end{equation}
	Let us denote the products $P_j, j\in \{1,2,3\}$ by
	\begin{equation}\label{pj}
		P_j(z)=z\prod_{k\in \z^*}\left(1-\frac{z}{i\overline{\nu_k^j}}\right).
	\end{equation}
	Thus the canonical product $P$ can be written in the following form 
	\begin{equation}\label{p pj}
		P(z)=\prod_{j\in \{1,2,3\}}\beta_j P_j\left(\frac{z}{\beta_j}\right).
	\end{equation}
	\begin{lem}[Young\cite{RY}, Rosier \cite{LR14}]\label{Rosier}
		Let $\Lambda_k=k+d_k$, where $d_k=d+ O(k^{-1}),$
		as $\abs{k} \to \infty$ for some constant $d \in \cplx$, and that $\Lambda_k\neq\Lambda_l$ for $k \neq l.$ Then $f(z)=z\prod_{k\in \z^*} \left(1-\frac{z}{ \Lambda_k}\right)$ is an entire function of sine type.
	\end{lem}
Note that, $i\overline{\nu_k^j}=-i\frac{\omega_j}{\beta_j}+ n + O\left( \frac{1}{|n|}\right), j\in \{1,2,3\}$. Also $i\overline{\nu_k^j}\neq i\overline{\nu_l^j},$ for $ l\neq k$.  Thanks to \Cref{Rosier}, each $P_j$ is a sine type function. Moreover, each of these functions is an entire function of exponential type $\pi$. Hence, $P$ is an entire function of the exponential type $\pi\left(\frac{1}{|\beta_1|}+\frac{1}{|\beta_2|}+\frac{1}{|\beta_3|}\right)$.
	
	
	\noindent
	Thanks to the definition of sine-type function, for any $\epsilon>0$ there exist positive constants $C_1, C_2, C_3, C_4,$ $ C_5$, where $C_2$ and $C_3$ depend on $\epsilon$, such that
	\begin{align}
		\label{bound}
		\abs{P_j(z)}\leq C_1^je^{\pi\abs{z}},&\:\: z \in \mathbb{C},\quad j \in \{1,2,3\},\\
		\label{bound1}C_2^je^{\pi\abs{y}}\leq \abs{P_j(x+iy)} \leq C_3^je^{\pi \abs{y}},\:\:& \text{if } dist\left(x+iy,\{i\overline{\nu_j^k}\}\right) > \epsilon, \\
		\label{bound2}C_4^j<\abs{P_j'\left(i\overline{\nu_k^j}\right)}<C_5^j,&\:\: \forall (k,j) \in \Sigma.
	\end{align}
	Now using the above inequality \eqref{bound1} and continuity of $P_j$, we get a positive constant $C^j$ such that
	\begin{equation}\label{eq:p1real}
		\abs{P_j(x)}\leq C^j, \quad \forall\, x\in\rea.
	\end{equation}
	Therefore the estimate \eqref{bound for p} holds.
	
	Using \eqref{p pj} and \eqref{bound2} we have:
	\begin{align}\label{p prime es} 
		\abs{P'\left(i\overline{\lambda_n^j}\right)}=\abs{P_j'\left(i\overline{\nu_n^j}\right)\prod_{\substack{l\in \{1,2,3\} \\ l\neq j}} \beta_l P_l\left(\frac{i\overline{\lambda_n^j}}{\beta_l}\right) }\geq \min_{1\leq j\leq 3} \{C_4^j\} \prod_{\substack{l\in \{1,2,3\} \\ l\neq j}}
		\abs{\beta_l P_l\left(\frac{i\overline{\lambda_n^j}}{\beta_l}\right)}.
		\end{align}
			Therefore to prove the estimate \eqref{p prime}, we need to compute the estimate of the quantities $P_l\left(\frac{i\overline{\lambda_n^j}}{\beta_l}\right), l\neq j.$ Thanks to \Cref{gap}, we have 
			\begin{align*}
				\abs{\frac{i\overline{\lambda_n^j}}{\beta_l}-i\overline{\nu_k^l}}=\frac{1}{|\beta_l|}\abs{{\overline{\lambda_n^j}}-\beta_l\overline{\nu_k^l}}=\frac{1}{|\beta_l|}\abs{{\overline{\lambda_n^j}}-\overline{\lambda_k^l}}> \hat{C}.
			\end{align*}
			 Therefore using the estimate \eqref{bound1}, we finally have the estimate
			\begin{equation}\label{p prime est1}
				\abs{P_l\left(\frac{i\overline{\lambda_n^j}}{\beta_l}\right)}\geq C_2^l e^{\pi\abs{\frac{\omega_j}{\beta_l}}}.
			\end{equation}
		Hence \eqref{p prime es} along with \eqref{p prime est1} provide the estimate \eqref{p prime}.
		\end{proof}
		\begin{theorem}\label{biorthothm}
			Let $T> 2\pi\left(\frac{1}{|\beta_1|}+\frac{1}{|\beta_2|}+\frac{1}{|\beta_3|}\right).$ Then there exists a family $\{\Theta_n^j\}_{(n,j)\in \Sigma}$ which is biorthogonal to the family of exponentials $\{e^{-{\lambda_k^l}t}\}_{(k,l)\in \Sigma}$ in $L^2\left(-\frac{T}{2}, \frac{T}{2}\right)$, i.e.,
			\begin{equation}\label{bior}
				\int_{-\frac{T}{2}}^{\frac{T}{2}}   \Theta_n^j(t) e^{-\overline{\lambda_k^l}t}\, \rd t= \delta_{nk}\delta_{jl}.
			\end{equation}
			Moreover, there exists a positive constant $C>0$ such that the following estimate holds:
			\begin{equation}\label{ingham1}
				\norm{\sum_{(n,j)\in \Sigma} a_n^j \Theta_n^j}_{L^2\left(-\frac{T}{2}, \frac{T}{2}\right)}^2 \leq C \sum_{(n,j)\in \Sigma} |a_n^j|^2 ,
			\end{equation}       for any finite sequence of complex numbers $\{ a_n^j\}_{(n,j)\in \Sigma}$.
		\end{theorem}
		\begin{proof}
			Let us first define the entire function 
			\begin{equation*}
				\Psi_n^j(z):=\frac{P(z)}{(z-i\overline{\lambda_n^j}) P'\left(i\overline{\lambda_n^j}\right)}.
			\end{equation*}
			Clearly, $\Psi_n^j$ is a collection of entire functions of exponential type at most $\pi\left(\frac{1}{|\beta_1|}+\frac{1}{|\beta_2|}+\frac{1}{|\beta_3|}\right).$ Moreover, it is easy to check that 
			\begin{align}\label{sq intg}
				\int_{\rea}\abs{\Psi_n^j(x)}^2 \, \rd x< C. %\int_{\rea}\abs{\Psi_0(x)}^2 dx< \infty.
			\end{align}
%			Also, we can deduce, 
%			\begin{equation*}\Psi_n^j(i\overline{\lambda_k^l})=\delta_{nk} \delta_{jl}, \,\, (n,j), (k,l)\in \Sigma.
%			\end{equation*}
			Therefore by Paley-Wiener Theorem (\Cref{Paley-Wiener}), there exists a collection of functions $\{\Theta_n^j\}_{(n,j)\in \Sigma}$ supported in $[-\frac{T'}{2}, \frac{T'}{2}]$, with $T'=2\pi\left(\frac{1}{|\beta_1|}+\frac{1}{|\beta_2|}+\frac{1}{|\beta_3|}\right)$ such that the following relation holds 
			\begin{align}\label{representation}
				\Psi_n^j(z)=\int_{-\frac{T'}{2}}^{\frac{T'}{2}}e^{izt}\Theta_n^j(t) \, \rd t,\, z\in \cplx.
			\end{align}
		Also, we can deduce
					\begin{equation*}\Psi_n^j(i\overline{\lambda_k^l})=\delta_{nk} \delta_{jl}, \,\, (n,j), (k,l)\in \Sigma.
					\end{equation*}
			Thus we have the family $\{\Theta_n^j\}_{(n,j)\in \Sigma}$ which is biorthogonal to the family of exponentials $\{e^{-{\lambda_k^l}t}\}_{(k,l)\in \Sigma}$ in $L^2\left(-\frac{T'}{2}, \frac{T'}{2}\right)$ and by Plancharel's Theorem and \eqref{sq intg}, we have 
			\begin{equation*}
				2\pi\int_{-\frac{T'}{2}}^{\frac{T'}{2}}\abs{\Theta_k^j(t)}^2\, \rd t=\int_{\rea}\abs{\Psi_k^j(x)}^2 \, \rd x\leq C.
			\end{equation*}
			Then using \cite[Proposition 8.3.9]{TW09}, we have for any $T>T'$, there exists a biorthogonal family  $\{\Theta_n^j\}_{(n,j)\in \Sigma}$ of $\{e^{-{\lambda_k^l}t}\}_{(k,l)\in \Sigma}$ in $L^2\left(-\frac{T}{2}, \frac{T}{2}\right)$ such that the estimate \eqref{ingham1} holds. 
		\end{proof}
		The following corollary is an immediate consequence of the above theorem which will essentially give us the required Ingham-type inequality.
		\begin{cor}
			Let us assume $T> 2\pi\left(\frac{1}{|\beta_1|}+\frac{1}{|\beta_2|}+\frac{1}{|\beta_3|}\right).$ Then for any finite sequence of scalars $\{ a_n^j\}_{(n,j)\in \Sigma}$, there exist two positive constants $C_1, C_2$ such that the following inequality holds:
			\begin{equation}\label{ingham2}
				C_1\sum_{(n,j)\in \Sigma} |a_n^j|^2\leq \norm{\sum_{(n,j)\in \Sigma} a_n^j e^{-\overline{\lambda_n^j}t}}_{L^2\left(-\frac{T}{2}, \frac{T}{2}\right)}^2\leq C_2 \sum_{(n,j)\in \Sigma} |a_n^j|^2.
			\end{equation}
		\end{cor}       
		\begin{proof}
			Using \eqref{bior}, we can deduce
			\begin{align}\label{ing 11}
				\nonumber
				\sum_{(n,j)\in \Sigma} |a_n^j|^2 =&\int_{-\frac{T}{2}}^{\frac{T}{2}}{\left(\sum_{(n,j)\in \Sigma} a_n^j e^{-\overline{\lambda_n^j}t}\right)}\left(\sum_{(k,l)\in \Sigma} \overline{a_k^l} \Theta_k^l(t) \right)\, \rd t \\
				\leq & \norm{\sum_{(n,j)\in \Sigma} a_n^j e^{-\overline{\lambda_n^j}t}}_{L^2\left(-\frac{T}{2}, \frac{T}{2}\right)}\norm{\sum_{(n,j)\in \Sigma} \overline{a_k^l} \Theta_k^l}_{L^2\left(-\frac{T}{2}, \frac{T}{2}\right)}.
			\end{align}
			The inequality \eqref{ing 11} along with \eqref{ingham1} establish the left inequality of \eqref{ingham2}.
			
			Straightforward estimate provides the right hand inequality. Indeed,
			\begin{align*}
				\norm{\sum_{(n,j)\in \Sigma} a_n^j e^{-\overline{\lambda_n^j}t}}_{L^2\left(-\frac{T}{2}, \frac{T}{2}\right)}^2\leq C_2\sum_{j=1}^{3}\norm{\sum_{n\in \z^*} a_n^j e^{-\overline{\lambda_n^j}t}}_{L^2\left(-\frac{T}{2}, \frac{T}{2}\right)}^2.
			\end{align*}
			Next, utilizing the Ingham-type inequality \cite[Proposition 8.1]{ahamed}, one can get the right hand inequality.\end{proof}


\section*{Acknowledgments}The authors would like to thank  Dr. Shirshendu Chowdhury and Dr. Debanjana Mitra for fruitful discussions. Sakil Ahamed expresses his gratitude to the IISER Kolkata's Department of Mathematics \& Statistics for their hospitality and assistance during his visit.



\section*{Data availability statement}
This article describes entirely theoretical research. Thus, data sharing is not applicable to this article
as no datasets were generated or analysed during the current study.



%	\bibliographystyle{siam}
%\bibliography{Thesis}
\begin{thebibliography}{10}
	
	\bibitem{ahamed}
	{\sc S.~Ahamed and D.~Mitra}, {\em Some controllability results for linearized
		compressible {N}avier-{S}tokes system with {M}axwell's law}, {Submitted,
		(2022) }.
	
	\bibitem{VB06}
	{\sc V.~Barbu, I.~Lasiecka, and R.~Triggiani}, {\em Tangential boundary
		stabilization of {N}avier-{S}tokes equations}, Mem. Amer. Math. Soc., 181
	(2006), pp.~x+128.
	
	\bibitem{Beauchard}
	{\sc K.~Beauchard, A.~Koenig, and K.~Le~Balc'h}, {\em Null-controllability of
		linear parabolic transport systems}, J. \'{E}c. polytech. Math., 7 (2020),
	pp.~743--802.
	
	\bibitem{jiten}
	{\sc K.~Bhandari, S.~Chowdhury, R.~Dutta, and J.~Kumbhakar}, {\em Boundary
		null-controllability of 1d linearized compressible navier-stokes system by
		one control force}, 2022.
	
	\bibitem{biccari2019null}
	{\sc U.~Biccari and S.~Micu}, {\em Null-controllability properties of the wave
		equation with a second order memory term}, Journal of Differential Equations,
	267 (2019), pp.~1376--1422.
	
	\bibitem{Filho2021RapidES}
	{\sc R.~d.~A. Capistrano-Filho, E.~Cerpa, and F.~A.~Gallego}, {\em Rapid
		exponential stabilization of a boussinesq system of kdv-kdv type},
	Communications in Contemporary Mathematics,  (2021).
	
	\bibitem{CE09}
	{\sc E.~Cerpa and E.~Cr\'{e}peau}, {\em Rapid exponential stabilization for a
		linear {K}orteweg-de {V}ries equation}, Discrete Contin. Dyn. Syst. Ser. B,
	11 (2009), pp.~655--668.
	
	\bibitem{Shirshendu}
	{\sc S.~Chowdhury, R.~Dutta, and S.~Majumdar}, {\em Boundary controllability
		and stabilizability of a coupled first-order hyperbolic-elliptic system},
	Evol. Equ. Control Theory, 12 (2023), pp.~907--943.
	
	\bibitem{CM15}
	{\sc S.~Chowdhury and D.~Mitra}, {\em Null controllability of the linearized
		compressible {N}avier-{S}tokes equations using moment method}, J. Evol. Equ.,
	15 (2015), pp.~331--360.
	
	\bibitem{chowdhury2014null}
	{\sc S.~Chowdhury, D.~Mitra, M.~Ramaswamy, and M.~Renardy}, {\em Null
		controllability of the linearized compressible {N}avier {S}tokes system in
		one dimension}, J. Differential Equations, 257 (2014), pp.~3813--3849.
	
	\bibitem{SC00}
	{\sc S.~S. Collis, K.~Ghayour, M.~Heinkenschloss, M.~Ulbrich, and S.~Ulbrich},
	{\em Numerical solution of optimal control problems governed by the
		compressible {N}avier-{S}tokes equations}, in Optimal control of complex
	structures ({O}berwolfach, 2000), vol.~139 of Internat. Ser. Numer. Math.,
	Birkh\"{a}user, Basel, 2002, pp.~43--55.
	
	\bibitem{SC02}
	\leavevmode\vrule height 2pt depth -1.6pt width 23pt, {\em Optimal control of
		unsteady compressible viscous flows}, Internat. J. Numer. Methods Fluids, 40
	(2002), pp.~1401--1429.
	
	\bibitem{Co07}
	{\sc J.-M. Coron}, {\em Control and nonlinearity}, vol.~136 of Mathematical
	Surveys and Monographs, American Mathematical Society, Providence, RI, 2007.
	
	\bibitem{DR71}
	{\sc R.~Datko}, {\em A linear control problem in an abstract {H}ilbert space},
	J. Differential Equations, 9 (1971), pp.~346--359.
	
	\bibitem{ervedoza2}
	{\sc S.~Ervedoza, O.~Glass, and S.~Guerrero}, {\em Local exact controllability
		for the two- and three-dimensional compressible {N}avier-{S}tokes equations},
	Comm. Partial Differential Equations, 41 (2016), pp.~1660--1691.
	
	\bibitem{ervedoza1}
	{\sc S.~Ervedoza, O.~Glass, S.~Guerrero, and J.-P. Puel}, {\em Local exact
		controllability for the one-dimensional compressible {N}avier-{S}tokes
		equation}, Arch. Ration. Mech. Anal., 206 (2012), pp.~189--238.
	
	\bibitem{ervedoza2018local}
	{\sc S.~Ervedoza and M.~Savel}, {\em Local boundary controllability to
		trajectories for the 1{D} compressible {N}avier {S}tokes equations}, ESAIM
	Control Optim. Calc. Var., 24 (2018), pp.~211--235.
	
	\bibitem{IL88}
	{\sc F.~Flandoli, I.~Lasiecka, and R.~Triggiani}, {\em Algebraic {R}iccati
		equations with nonsmoothing observation arising in hyperbolic and
		{E}uler-{B}ernoulli boundary control problems}, Ann. Mat. Pura Appl. (4), 153
	(1988), pp.~307--382 (1989).
	
	\bibitem{Fur04}
	{\sc A.~V. Fursikov}, {\em Stabilization for the 3{D} {N}avier-{S}tokes system
		by feedback boundary control}, vol.~10, 2004, pp.~289--314.
	\newblock Partial differential equations and applications.
	
	\bibitem{RCDC}
	{\sc G.~Harris and C.~Martin}, {\em The roots of a polynomial vary continuously
		as a function of the coefficients}, Proc. Amer. Math. Soc., 100 (1987),
	pp.~390--392.
	
	\bibitem{HuRacke}
	{\sc Y.~Hu and R.~Racke}, {\em Compressible navier-stokes equations with
		revised maxwell's law}, J. Math. Fluid Mech., 19 (2017), p.~77–90.
	
	\bibitem{hu2019global}
	{\sc Y.~Hu and N.~Wang}, {\em Global existence versus blow-up results for one
		dimensional compressible navier--stokes equations with maxwell's law},
	Mathematische Nachrichten, 292 (2019), pp.~826--840.
	
	\bibitem{ingham1936some}
	{\sc A.~E. Ingham}, {\em Some trigonometrical inequalities with applications to
		the theory of series}, Math. Z., 41 (1936), pp.~367--379.
	
	\bibitem{PLissy}
	{\sc A.~Koenig and P.~Lissy}, {\em Null-controllability of underactuated linear
		parabolic-transport systems with constant coefficients}, 2023.
	
	\bibitem{KV1}
	{\sc V.~Komornik}, {\em Exact controllability and stabilization}, RAM: Research
	in Applied Mathematics, Masson, Paris; John Wiley \& Sons, Ltd., Chichester,
	1994.
	\newblock The multiplier method.
	
	\bibitem{KV}
	{\sc V.~Komornik and P.~Loreti}, {\em Fourier series in control theory},
	Springer Monographs in Mathematics, Springer-Verlag, New York, 2005.
	
	\bibitem{jiten2}
	{\sc J.~Kumbhakar}, {\em Null controllability of one-dimensional linearized
		compressible navier-stokes system in periodic setup using one boundary
		control}, 2023.
	
	\bibitem{LR14}
	{\sc P.~Martin, L.~Rosier, and P.~Rouchon}, {\em Null controllability of the
		structurally damped wave equation with moving control}, SIAM J. Control
	Optim., 51 (2013), pp.~660--684.
	
	\bibitem{micu2004introduction}
	{\sc S.~Micu and E.~Zuazua}, {\em An introduction to the controllability of
		partial differential equations}, Quelques questions de th{\'e}orie du
	contr{\^o}le. Sari, T., ed., Collection Travaux en Cours Hermann, to appear,
	(2004).
	
	\bibitem{MRR-17}
	{\sc D.~Mitra, M.~Ramaswamy, and M.~Renardy}, {\em Interior local null
		controllability of one-dimensional compressible flow near a constant steady
		state}, Math. Methods Appl. Sci., 40 (2017), pp.~3445--3478.
	
	\bibitem{MR-17}
	{\sc D.~Mitra and M.~Renardy}, {\em Interior local null controllability for
		multi-dimensional compressible flow near a constant state}, Nonlinear Anal.
	Real World Appl., 37 (2017), pp.~94--136.
	
	\bibitem{MN-19}
	{\sc N.~Molina}, {\em Local exact boundary controllability for the compressible
		{N}avier-{S}tokes equations}, SIAM J. Control Optim., 57 (2019),
	pp.~2152--2184.
	
	\bibitem{Raymond}
	{\sc J.-P. Raymond}, {\em Feedback boundary stabilization of the
		two-dimensional {N}avier-{S}tokes equations}, SIAM J. Control Optim., 45
	(2006), pp.~790--828.
	
	\bibitem{Raymond1}
	{\sc J.-P. Raymond}, {\em Feedback boundary stabilization of the
		three-dimensional incompressible {N}avier-{S}tokes equations}, J. Math. Pures
	Appl. (9), 87 (2007), pp.~627--669.
	
	\bibitem{TW09}
	{\sc M.~Tucsnak and G.~Weiss}, {\em Observation and control for operator
		semigroups}, Birkh\"{a}user Advanced Texts: Basler Lehrb\"{u}cher.
	[Birkh\"{a}user Advanced Texts: Basel Textbooks], Birkh\"{a}user Verlag,
	Basel, 2009.
	
	\bibitem{UR05}
	{\sc J.~M. Urquiza}, {\em Rapid exponential feedback stabilization with
		unbounded control operators}, SIAM J. Control Optim., 43 (2005),
	pp.~2233--2244.
	
	\bibitem{VK08}
	{\sc R.~Vazquez and M.~Krstic}, {\em Control of turbulent and
		magnetohydrodynamic channel flows}, Systems \& Control: Foundations \&
	Applications, Birkh\"{a}user Boston, Inc., Boston, MA, 2008.
	\newblock Boundary stabilization and state estimation.
	
	\bibitem{AV}
	{\sc A.~Vest}, {\em Rapid stabilization in a semigroup framework}, SIAM J.
	Control Optim., 51 (2013), pp.~4169--4188.
	
	\bibitem{leal2021control}
	{\sc F.~J. Vielma~Leal and A.~Pastor}, {\em Control and stabilization for the
		dispersion generalized {B}enjamin equation on the circle}, ESAIM Control
	Optim. Calc. Var., 28 (2022), pp.~Paper No. 54, 42.
	
	\bibitem{leal2021simple}
	\leavevmode\vrule height 2pt depth -1.6pt width 23pt, {\em Two simple criterion
		to obtain exact controllability and stabilization of a linear family of
		dispersive {PDE}'s on a periodic domain}, Evol. Equ. Control Theory, 11
	(2022), pp.~1745--1773.
	
	\bibitem{WangHu}
	{\sc N.~Wang and Y.~Hu}, {\em Blowup of solutions for compressible
		navier-stokes equations with revised maxwell's law}, Appl. Math. Lett., 103
	(2020), pp.~106221, 6 pp.
	
	\bibitem{RY}
	{\sc R.~M. Young}, {\em An introduction to nonharmonic {F}ourier series},
	vol.~93 of Pure and Applied Mathematics, Academic Press, Inc. [Harcourt Brace
	Jovanovich, Publishers], New York-London, 1980.
	
	\bibitem{zabczyk2008mathematical}
	{\sc J.~Zabczyk}, {\em Mathematical control theory, modern birkh{\"a}user
		classics}, 2008.
	
	\bibitem{ZZ}
	{\sc Z.~Zahreddine and E.~F. Elshehawey}, {\em On the stability of a system of
		differential equations with complex coefficients}, Indian J. Pure Appl.
	Math., 19 (1988), pp.~963--972.
	
\end{thebibliography}


\end{document}
