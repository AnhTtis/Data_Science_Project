\section{Related Work} \label{sec:related}

\noindent \textbf{Passive NLOS.}~
Previous passive NLOS imaging techniques mainly focus on reconstructing static information of the invisible scene. Some works leverage pinholes or pinspecks as ``accidental cameras" \cite{cohen1982anti, torralba2012accidental} while others rely on occluders (\eg, blocking objects or corners). These works make use of shadows and penumbrae cast on the visible wall or floor to extract useful information about the hidden scene \cite{baradad2018inferring, saunders2019computational, yedidia2019using, wang2021accurate, geng2022passive, seidel2019corner, Seidel2021TwoDim}. 

As for the dynamic NLOS scenario, it was first shown by Bouman \etal \cite{bouman2017turning} that obstructions with edges can be exploited as ``corner cameras". They reveal the number and trajectories of people moving in an occluded scene with recovered 1-D spatio-temporal videos.
Sharma \etal \cite{PrafullSharma2021WhatYC} presented a deep learning method that reveals the number or activity of people in an unknown room by observing a blank relay wall.
Wang \etal \cite{Wang2022event} first proposed a novel method for NLOS moving target reconstruction, which uses an event camera to extract rich dynamic information of the speckle movement. 

Compared to existing methods, our technique doesn't introduce any additional structures or special devices. With only a visible blank wall and a conventional RGB camera, we can extract both motion and static information on the frame level and perform tracking in real time.

\noindent \textbf{Active NLOS localization and tracking.}~ 
Some previous works accomplished NLOS tracking directly through locating the hidden object or person frame-by-frame~\cite{klein2016transient, chan2017fast, Chan:17, tancik2018flash, brooks2019single, metzler2020keyhole, cao2022computational}, whereas other methods consider object motion to assist tracking~\cite{gariepy2016detection, klein2016tracking, smith2018tracking}. All methods mentioned above rely on active illumination and most of them rely on time-resolved detection techniques. Methods employing lasers typically take advantage of the high flatness of optical experimental platforms. In contrast, our method removes the need for any additional illumination, equipment, and special detector. 
\looseness=-1

\noindent \textbf{NLOS datasets.}~ 
Large-scale, labeled and readily accessible datasets are vital to technique development. However, only few NLOS works provide datasets, and most of them focus on active NLOS imaging.
Jarabo \etal \cite{jarabo2014framework} proposed an effective framework for rendering in transient state, which has been exploited by several following works for data generation \cite{klein2018quantitative, liu2019phasor, zhu2022fast}.
Klein \etal \cite{klein2018quantitative} released a synthetic data foundation with a few scenes and the first reconstruction benchmark platform for a variety of NLOS imaging tasks, along with task-specific quality metrics. To further expand the data scale and facilitate data-driven methods, a new benchmark dataset for time-resolved NLOS imaging, Z-NLOS, is proposed by Galindo \etal \cite{galindo2019dataset}.
% \looseness=-1

As for passive datasets, Chen \etal have presented the first large-scale static passive NLOS dataset\cite{geng2022passive}. Wang \etal \cite{Wang2022event} created the first event-based NLOS imaging dataset, which explores a novel modal in dynamic NLOS imaging. 
In addition to the fact that only a proportion of datasets are available, the lack of realistic dynamic passive NLOS datasets also remains an obstacle to exploring passive NLOS methods. To address this issue, we propose a new dataset, NLOS-Track, which contains both synthetic data and real-shot data.

