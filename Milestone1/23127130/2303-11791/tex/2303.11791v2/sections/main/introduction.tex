\vspace{-5mm}
\section{Introduction}
\label{sec:intro}

In contrast to conventional imaging within the direct line-of-sight (LOS), non-line-of-sight (NLOS) imaging aims to tackle an inverse problem, \ie, using indirect signal (\eg, reflection from a visible relay wall) to recover information of invisible areas. To specify, NLOS tracking manages to reconstruct a continuous trajectory in real time when an object or a person is moving in an invisible region, which is sketched in \cref{fig:teaser}.
The ability to track moving objects outside the LOS would enable promising applications, such as autonomous driving, robotic vision, security, medical imaging, post-disaster searching, and rescue operations, \etc \cite{Borges2012Pedestrian, maeda2019recent, faccio2020non, geng2021recent}, thus receiving increasing attention in recent years.

\begin{figure}[t]
    \centering
    % \fbox{\rule{0pt}{2in} \rule{0.9\linewidth}{0pt}}
    \includegraphics[width=\linewidth]{images/teaser.pdf}
    \caption{\textbf{A schematic of the passive NLOS tracking.} The character is walking in the hidden scene and we can perform real-time tracking by observing and analyzing the relay wall from outside the room with a RGB camera, without any additional equipment.
    \looseness=-1}
    \label{fig:teaser}
    \vspace{-10pt}
\end{figure}

Existing NLOS tracking techniques mostly rely on active illumination from the detection side \cite{gariepy2016detection, klein2016tracking, klein2016transient, chan2017fast, Chan:17, smith2018tracking, tancik2018flash, brooks2019single, metzler2020keyhole, cao2022computational}. 
Although introducing denser and finer information, active illumination typically requires expensive equipment (\eg, ultra-fast pulsed laser) and elaborate experimental conditions \cite{saunders2019computational}. 
These defects cause a gap between active techniques and practical applications. Besides, the oversimplified setting in previous works even expands the gap. Unlike active methods, passive NLOS techniques \cite{baradad2018inferring, saunders2019computational, yedidia2019using, wang2021accurate, geng2022passive, seidel2019corner, Seidel2021TwoDim, bouman2017turning, PrafullSharma2021WhatYC, he2022non, cao2022computational} only depend on the feeble diffuse reflection of the hidden region, getting rid of requirements of expensive equipment. So this paper focuses on the low-cost passive NLOS tracking task in realistic scenarios. 
\looseness=-1

We find that most existing NLOS tracking works merely locate the object in each frame independently \cite{klein2016transient, chan2017fast, Chan:17, tancik2018flash, brooks2019single, metzler2020keyhole, cao2022computational, he2022non}, without considering the position relationship between adjoining moments. This practice directly causes jitters of trajectory, thus resulting in inaccurate tracking (see \cref{sec:c-net} for more details). In this paper, we consider the significance of making use of motion information and taking advantage of motion continuity prior, which helps achieve more coherent and accurate tracking results.

Furthermore, passive NLOS techniques face the dilemma that the signal-to-noise ratio (SNR) is extremely low \cite{Chan:17, caramazza_2018}. To address this problem, some previous works conduct background estimation with the video's temporal mean and apply background subtraction to every frame \cite{bouman2017turning, PrafullSharma2021WhatYC, he2022non}. In this way, the difference between frames could be amplified, thus increasing the SNR.
However, temporal-mean subtraction inevitably mixes up information from early period. Consequently, it reintroduces extra noise into originally low-SNR signals, which is still a hazard to excavating faint differences between frames.


To address the aforementioned problems, we first introduce \textit{difference frame} to describe motion information.
Compared to background estimation and subtraction, a difference frame can be readily obtained by subtracting the previous frame from the current frame.
In this way, a difference frame can represent the immediate motion information, and will not introduce noise from other periods. Our experiments show that difference frames do convey essential dynamic messages (see \cref{sec:p-net} for more details).
Additionally, we propose a novel network named PAC-Net (\textbf{P}ropagation \textbf{A}nd \textbf{C}alibration \textbf{Net}work), which integrates motion continuity prior into the algorithm. Consisting of two dual modules, Propagation-Cell and Calibration-Cell, PAC-Net maintains a good continuity of trajectory via propagating with difference frames and then alternately calibrating with raw frames. 
Our experimental results demonstrate that PAC-Net can achieve centimeter-level precision when tracking a walking person in real time.

We also build NLOS-Track, the first public-accessible video dataset for passive NLOS tracking. It contains realistic scenes to support the proposed task and method, and we expect NLOS-Track to facilitate more NLOS works. In contrast to oversimplified settings in existing NLOS tracking works, NLOS-Track dataset manages to simulate realistic scenarios with humans walking in unknown scenes. The dataset consists of 500 real-shot videos and more than 1,000 synthetic videos, each recording the relay wall when a character walks along the randomly generated trajectory. Paired trajectory ground truth of each video clip is also provided. 

Our contributions are mainly in three folds:
\begin{itemize}
    \item We propose and formulate the purely passive NLOS tracking task, which avoids the use of expensive equipment. Development on this task will allow promising and valuable applications in many fields, such as robotic vision, medical imaging, \etc.
    \item We propose a passive NLOS tracking network, PAC-Net, which is capable of utilizing both dynamic and static messages on a frame level. As for dynamic messages, we specially introduce difference frames as clear carriers of motion information, which gets rid of introducing extra noise from other periods.
    \item We establish the first passive NLOS trajectory tracking dataset, NLOS-Track, which contains thousands of video clips with a variety of scene settings.
\end{itemize}
