\vspace{-5mm}
\section{Introduction}
% \vspace{-2mm}
\label{sec:intro}

% intro to NLOS & value of NLOS
In contrast to conventional imaging within the direct line-of-sight (LOS), non-line-of-sight (NLOS) imaging aims to tackle an inverse problem, \ie, using indirect signal (\eg, reflection from a visible relay wall) to recover information of invisible areas. To specify, NLOS tracking manages to reconstruct a continuous trajectory in real time when an object or a person is moving in an invisible region, which is sketched in \cref{fig:teaser}.
The ability to track moving objects outside the LOS would enable promising applications, such as autonomous driving, robotic vision, security, medical imaging, post-disaster searching, and rescue operations, \etc \cite{Borges2012Pedestrian, maeda2019recent, faccio2020non, geng2021recent}, thus receiving increasing attention in recent years.
% In this paper, we propose a purely passive data-driven method for NLOS tracking, along with a synthetic dataset. By ``purely passive", we imply not using any active illumination and using merely a conventional camera as a detector.

\begin{figure}[t]
    \centering
    % \fbox{\rule{0pt}{2in} \rule{0.9\linewidth}{0pt}}
    \includegraphics[width=\linewidth]{images/teaser.pdf}
    \caption{\textbf{A schematic of the passive NLOS tracking.} The character is walking in the hidden scene and we can perform real-time tracking by observing and analyzing the relay wall from outside the room with a RGB camera, without any additional equipment.
    \looseness=-1}
    \label{fig:teaser}
    \vspace{-10pt}
\end{figure}

% problems of NLOS tracking using active illumination
Existing NLOS tracking techniques mostly rely on active illumination from the detection side \cite{gariepy2016detection, klein2016tracking, klein2016transient, chan2017fast, Chan:17, smith2018tracking, tancik2018flash, brooks2019single, metzler2020keyhole, cao2022computational}. 
Although introducing denser and finer information, active illumination typically requires expensive equipment (\eg, ultra-fast pulsed laser) and elaborate experimental conditions \cite{saunders2019computational}. 
These defects cause a gap between active techniques and practical applications. Besides, the oversimplified setting in previous works even expands the gap. Unlike active methods, passive NLOS techniques \cite{baradad2018inferring, saunders2019computational, yedidia2019using, wang2021accurate, geng2022passive, seidel2019corner, Seidel2021TwoDim, bouman2017turning, PrafullSharma2021WhatYC, he2022non, cao2022computational} only depend on the feeble diffuse reflection of the hidden region, getting rid of requirements of expensive equipment. So this paper focuses on the low-cost passive NLOS tracking task in realistic scenarios. 
\looseness=-1

% 1.1 没有运动信息
% 1.2 差分图,clean
% 2.1 运动信息和运动连续性
% 2.2 提出PAC-Net

We find that most existing NLOS tracking works merely locate the object in each frame independently \cite{klein2016transient, chan2017fast, Chan:17, tancik2018flash, brooks2019single, metzler2020keyhole, cao2022computational, he2022non}, without considering the position relationship between adjoining moments. This practice directly causes jitters of trajectory, thus resulting in inaccurate tracking (see \cref{sec:c-net} for more details). In this paper, we consider the significance of making use of motion information and taking advantage of motion continuity prior, which helps achieve more coherent and accurate tracking results.

% challenges of passive NLOS tracking
% challenge 1: low SNR
Furthermore, passive NLOS techniques face the dilemma that the signal-to-noise ratio (SNR) is extremely low \cite{Chan:17, caramazza_2018}. To address this problem, some previous works conduct background estimation with video's temporal mean and apply background subtraction to every frame \cite{bouman2017turning, PrafullSharma2021WhatYC, he2022non}. In this way, the difference between frames could be amplified, thus increasing the SNR.
% He \etal \cite{he2022non} 
% Some works directly use a picture of empty scene as a \textit{true} background frame. Since such empty background image is not always readily available,
% Bouman \etal \cite{bouman2017turning} and Sharma \etal \cite{PrafullSharma2021WhatYC} 
% other works propose to estimate the background frame with video's temporal mean. However, on the one hand, this practice assumes a roughly uniform distribution of the subject's motion over the video clip, which is not always true. On the other hand,
However, temporal-mean subtraction inevitably mixes up information from early period. Consequently, it reintroduces extra noise into originally low-SNR signals, which is still a hazard to excavating faint differences between frames.
% \looseness=-1


%Although these works achieve certain successes, there are still two problems remained. 

% challenge 1: making use of motion information and motion continuity prior

To address the aforementioned problems, we first introduce \textit{difference frame} to describe motion information.
Compared to background estimation and subtraction, a difference frame can be readily obtained by subtracting the previous frame from the current frame.
In this way, a difference frame can represent the immediate motion information, and will not introduce noise from other periods. Our experiments show that difference frames do convey essential dynamic messages (see \cref{sec:p-net} for more details).
Additionally, we propose a novel network named PAC-Net (\textbf{P}ropagation \textbf{A}nd \textbf{C}alibration \textbf{Net}work), which integrates motion continuity prior into the algorithm. Consisting of two dual modules, Propagation-Cell and Calibration-Cell, PAC-Net maintains a good continuity of trajectory via propagating with difference frames and then calibrating with raw frames in an alternate manner. 
Our experimental results demonstrate that PAC-Net can achieve centimeter-level precision when tracking a walking person in real time.



% However, it is the insight of Bouman \etal \cite{bouman2017turning} and Sharma \etal \cite{PrafullSharma2021WhatYC} that object motion can mitigate the ill-pose dilemma of NLOS imaging, making it more addressable. They have demonstrated that motion information can be exploited to identify the motion of objects around the corner \cite{bouman2017turning} and determine the number and activity of people in the hidden scene \cite{PrafullSharma2021WhatYC}. However, the temporal motion information extracted from these works either stays at the phenomenon level, needs further interpretation by human beings, or is integrated into the overall information. Making use of temporal motion information from the frame level granularity is still an unexplored challenge. This is a prerequisite for real-time passive NLOS tracking.
%这段读下来的感觉是previous的工作很重要,已经解决了80%的问题了,然后我们做了一个很incremental的改进。
%我的思路是,Considering the aforementioned drawbacks of active NLOS works, this paper focuses on passive NLOS tracking which relies only on the feeble diffuse reflection from the hidden region. (承上启下,引出我们的scope)
% While it is a particular dilemma of passive NLOS imaging that the signal-to-noise ratio (SNR) is extremely low. (递进展开下一个问题, 问题要选准,这一路下来抛出的问题太多了,找不到重点了。选一两个对应我们方法的问题来阐述,其他再重要也一笔带过或干脆不提。)
% This is because natural lights drasticcally attenuate during propogation. (如果后文与the loss of angular information无关,这点可以不要了,比较过于专业,影响读者对本文贡献的把握。This nature makes the first-principles modeling of passive NLOS imaging difficult.这句实际意义也不大)
%On the top of low SNR, it could be very challenging to accurately locate the moving object in real time and maintain temporal continuity of trajectories.
%Fortunately, xxx work (表明不是NLOS tracking) Bouman \etal \cite{bouman2017turning} and Sharma \etal \cite{PrafullSharma2021WhatYC} prove that object motion can mitigate the ill-pose dilemma of NLOS imaging. (他们具体做了什么最好不要再详细展开了)
% Inspired by these works, we xxxx (重点突出我们做了什么,这块需要再讨论下,如果重点是the use of temporal motion information from the frame level,那么前面那些问题和对应方法的介绍就离题太遥远了)@王之港



% how to solve problems and challenges mentioned above
% 这段第一句有点前不着村后不着店的感觉(手动笑哭)
% We aim to develop a purely passive method for real-time NLOS tracking task. To this end, we first introduce difference images as efficient dynamic information carriers in NLOS imaging. The sparse nature of difference images is essential for real-time inference.
%这里该丰富介绍一下difference images,结果反而没了(再次手动笑哭)。需要概括性说明difference images是什么,它的目的,它为什么能达到这样的目的 @王之港
% As suggested by previous works, it could improve the performance of NLOS imaging by integrating motion continuity priors into the algorithm and designing novel network architecture based on the characteristics of NLOS tasks . 
% We also propose a novel network architecture named PAC-Net (\textbf{P}ropagate \textbf{A}nd \textbf{C}alibration network). The recurrent structure allow the model to be benefit from the motion continuity prior \cite{metzler2020keyhole, geng2021recent}, thus overcomes the trajectory continuity challenge. % 这个trajectory continuity challenge是从哪来的@王之港
% By processing difference and raw frames in an alternate manner, PAC-Net is capable of exploit both dynamic and static information efficiently. %为什么需要exploit both dynamic and static information?
% This process could be an analogy to the kinetic integral over time. %这句又是什么意思?
% Our experimental results demonstrate that the working pipeline of calibration after propagation %这两个propagation和calibration应该提前说明,并且各自的作用要介绍。
% enable PAC-Net to achieve centimeter-level precision when tracking a walking person in real time.
%这一段的逻辑连贯性有待提高@王之港

We also build NLOS-Track, the first public-accessible video dataset for passive NLOS tracking. It contains realistic scenes to support the proposed task and method, and we expect NLOS-Track to facilitate more NLOS works. In contrast to oversimplified settings in existing NLOS tracking works, NLOS-Track dataset manages to simulate realistic scenarios with humans walking in unknown scenes. The dataset consists of 500 real-shot videos and more than 1,000 synthetic videos, each recording the relay wall when a character walks along the randomly generated trajectory. Paired trajectory ground truth of each video clip is also provided. 
%这段可以从之前work的简陋setting讲起,然后为了验证我们的方法,也为了更贴近实际,建了个新数据集。然后介绍新数据集是什么样的。概括性突出它比之前的更realistic

% contribution应该是:
% 1)第一个提出这个任务;
% 2)设计了什么方法来克服challenge;
% 3)数据集?
Our contributions are mainly in three folds:
\begin{itemize}
    \item We propose and formulate the purely passive NLOS tracking task, which avoids the use of expensive equipment. Development on this task will allow promising and valuable applications in many fields, such as robotic vision, medical imaging, \etc.
    \item We propose a passive NLOS tracking network, PAC-Net, which is capable of utilizing both dynamic and static messages on a frame level. As for dynamic messages, we specially introduce difference frames as clear carriers of motion information, which gets rid of introducing extra noise from other periods.
    \item We establish the first passive NLOS trajectory tracking dataset, NLOS-Track, which contains thousands of video clips with a variety of scene settings.
\end{itemize}
% difference frame 可以作为一个contribution,放到第二点 @王之港
