\section{More Analysis}

\subsection{Light sources}

We set all three light sources on the ceiling (please refer to Fig.1 in the main paper) so the useful information on the wall mainly comes from the diffuse reflection rather than obvious shadows (please refer to Sec.3 in the main paper). Since the tracked person is always moving, the position of lights relative to the person is not fixed.

In the synthetic dataset we vary the \textit{type(point, spot, and area), position, rotation}, and \textit{power} of three light sources. With so many factors influencing the light condition, it is a ponderous task to conduct a fair and comprehensive investigation. In \cref{fig:light_analysis}, we preliminarily show how metrics vary with different light type combinations. We can see that with 2 area lights, we have relatively good tracking results on average because more area light makes the room brighter, thus can provide sufficient diffuse signal cast on the relay wall for NLOS tracking. We found there is a coarsely positive correlation between light source area and performance. \\

\begin{figure}[t]
    \vspace{-15pt}
    \centering
    \includegraphics[width=\linewidth]{images/supp/light_source_metric.pdf}
    \vspace{-18pt}
    \caption{\textbf{How light sources affect metrics.} Light type of $(1,0,2)$ means 1 point light, 0 spot light and 2 area lights are in the room.}
    \label{fig:light_analysis}
    \vspace{-10pt}
\end{figure}

\subsection{Warm-up}

When facing a variety of room settings in synthetic data, the Warm-up stage plays an important role in ``adapting" to the current room (please refer to the grey dashed lines in Fig.4 in the main paper). In contrast, due to the relatively limited real-shot scenes, the potential of the Warm-up stage cannot be fully demonstrated.
We conduct an extra comparison test on a small synthetic dataset with only two room sizes. And we observe only minor differences between w/ and w/o Warm-up (\cref{tab:warm-up}). Thus we conclude that Warm-up requires a diverse dataset to work.

\begin{table}[ht]
    \centering
    \vspace{-5pt}
    \scalebox{0.8}{
    \begin{tabular}{cccccc}
        \toprule
        Model & RMS$_x$ & RMS$_v$\small{$(\times 10^{-3})$} & Area & DTW & PCM  \\
        \hline
        w/o Warm-up & \textbf{0.1103} & 2.50 & 0.0776 & 1.698 & \textbf{2.613} \\
        w/ Warm-up & 0.1105 & \textbf{2.03} & \textbf{0.0715} & \textbf{1.589} & 3.429 \\
        \bottomrule
    \end{tabular}
    }
    \vspace{-5pt}
    \caption{\textbf{Comparison test on warm-up.}}
    \label{tab:warm-up}
    \vspace{-10pt}
\end{table}