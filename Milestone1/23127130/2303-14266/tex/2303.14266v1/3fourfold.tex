
Consider $X=\CC^4$ with a $(\CC^4)^*$-action by scaling coordinates
\[(t_1,t_2,t_3,t_4)\cdot (x_1,x_2,x_3,x_4) = (t_1x_1,t_2x_3,t_3x_3,t_4x_4).\]
Let $\T_0=\{(t_1,t_2,t_3,t_4):t_1t_2t_3t_4=1\}\seq (\CC^4)^*$ be the subtorus which preserves the usual volume form on $X$, making $X$ a smooth quasi-projective toric Calabi-Yau 4-fold. As in the surface case, we include two additional tori
\[\T_1=(\CC^*)^N,\indent\T_2=(\CC^*)^{r+s}.\]
where $\T_1$ acts naturally on $\CC^N$, and $\T_2$ acts naturally on $\CC^{r}\times \CC^s$. Set $\T=\T_0\times\T_1\times \T_2$, and consider $E=\oplus_{i=1}^N\O_X\<y_i\>, \a=[\oplus_{i=1}^r\O_X\<v_i\>]-[\oplus_{i=r+1}^{r+s}\O_X\<v_i\>]$. Write
\equa{K_\T(\pt)&=\ZZ[t_1^{\pm1},t_2^{\pm1},t_3^{\pm1},t_4^{\pm1};y_1^{\pm1},\dots ,y_N^{\pm1};v_1^{\pm1},\dots ,v_{r+s}^{\pm1}]/(t_1t_2t_3t_4-1),\\
H^*_\T(\pt)&=\CC[\lambda_1,\lambda_2,\lambda_3,\lambda_4;m_1,\dots ,m_N;w_1,\dots ,w_{r+s}]/(\lam_1+\lam_2+\lam_3+\lam_4).}

By \cite[Theorem~4.1]{HT} the truncated Atiyah class of the universal subsheaf $\mathcal{I}$ defines an obstruction theory
\[\mathbf{R}\shom_{p}(\mathcal{I},\mathcal{I})_0^\vee[-1]\->L_{\quot_X(E,n)}^\bullet\]
where $\mathbf{R}\shom_{q}=\mathbf{R}q_*\circ \mathbf{R}\shom$, $(\cdot)_0$ denotes the trace free part. Note that the obstruction theory is $\T$-equivariant by \cite[Theorem B]{Ri}. The virtual tangent bundle is then
\[T^\vir=-\mathbf{R}\shom_{p}(\mathcal{I},\mathcal{I})_0\in K_\T(\quot_X(E,n)).\]



\subsection{Cohomological virtual invariants}\label{sec:vir-inv4d}
When $X$ is a projective Calabi-Yau 4-fold, the virtual fundamental class involves a choice of orientation on $\quot_X(E,n)$. Let $\mathcal{L}=\text{det}\mathbf{R}\shom_{q}(\mathcal{I},\mathcal{I})$ be the determinant line bundle. An \emph{orientation} $o(\mathcal{L})$ is a choice of square root of the isomorphism
\[Q:\mathcal{L}\otimes \mathcal{L}\->\mathcal{O}_{\quot_X(E,n)}\]
induced by Serre duality. A virtual class $[\quot_X(E,n)]^\vir_{o(\mathcal{L})}\in H_{2nN}(\quot_X(E,n),\ZZ)$ was constructed in \cite[§2.1]{Bojko2} for $X$ a strict Calabi-Yau 4-fold and $E$ a simple rigid locally-free sheaf. For $\gamma\in H^{2nN}(\quot_X(E,n))$, the holomorphic Donaldson invariants are defined to be
\[\mathcal{Q}(\gamma)=\int_{[\quot_X(E,n)]^\vir_{o(\mathcal{L})}}\gamma.\]

For the non-compact $X=\CC^4$, similar to the surface case, we have that the $\T$-fixed locus of $\hilb^n(X)$ consists of only finitely many reduced points \cite[Lemma~3.6]{CK1}, so we can define these invariants equivariantly using Oh-Thomas' localization formula \cite[Theorem~7.1]{Oh:2020rnj}. For a fixed orientation and any $SO(2k)$-bundle $B$, one can define its Edidin-Graham square root Euler class $\sqrt{e}(B)$ \cite[§3]{Oh:2020rnj}. Consider the self-dual resolution (\ref{eqn:res}) of $T^\vir$
\[\phi: B^\bullet\to L^\bullet_{\quot_{X}(E,n)}\]
where $B^\bullet = (T\-> B\->T^*)$ and $B$ a $SO(6n^2)$-bundle. Recall that $T^\vir$ has no fixed parts. Set \cite[(115)]{Oh:2020rnj}
\[\sqrt{e_\T}(T^\vir):=\sqrt{e_\T}(B^\bullet):=\frac{e_\T(T)}{\sqrt{e_\T}(B)}.\]
For computational purposes, we consider a square root $\sqrt{T^\vir|_Z}\in K_\T(\pt)$ for each fixed point $Z$ such that
\[T^\vir|_Z=\sqrt{T^\vir|_Z}+\overline{\sqrt{T^\vir|_Z}},\]
where $\overline{(\cdot)}$ denotes the involution $t_i\mapsto t_i^{-1}$. This allows us to compute the square root Euler class at the cost of a sign dependent on the choice of orientation $o(\mathcal{L})$
\[\sqrt{e}(T^\vir|_Z)=\pm e\left(\sqrt{T^\vir|_Z}\right).\]
As each fixed point is reduced, Kool-Rennemo \cite{KR-draft} show that their virtual fundamental classes are given by further signs determined by some induced orientation on $B^\bullet$.
We will denote the product of the two signs at each $Z\in\quot_X(E,n)^\T$ by 
$(-1)^{o(\mathcal{L})|_Z}$.
\begin{definition} For $n>0$, $\gamma\in H^{*}_\T(\quot_X(E,n))$, the \emph{holomorphic Donaldson invariants} are
\[
\mathcal{Q}(X,E,n,\gamma):=\sum_{Z\in\quot_X(E,n)^\T}(-1)^{o(\mathcal{L})|_Z}\frac{\gamma|_Z}{ e(\sqrt{T^\vir|_Z})}.\]
\end{definition}



The authors of \cite{KR-draft} further gave an explicit description of the  moduli space $\quot_X(E,n)$ when $X=\CC^4$ as a vanishing locus of an isotropic section. This allowed them to derive the signs $(-1)^{o(\mathcal{L})|_Z}$ by knowing $\L$ and $T^\vir$. For the purpose of motivating the Verlinde invariants from the introduction, we recall their approach here. 

 Consider the quiver 
 \begin{figure}[h]
     \centering
 \includegraphics[scale = 0.5]{framed_4_loop_quiver.png}
     \caption{Framed quiver with four loops at one node.}
     %\label{fig:4loop}
 \end{figure}
with four loops and $N$ framings. After imposing the relations
\begin{equation}
\label{eq:relation}
[x_i,x_j] = 0\,,
\end{equation}
its representations with dimension vector $(1,n)$ consist of a one-dimensional complex vector space $\CC$ and an $n$-dimensional vector space $V$ together with $N$ morphisms $f_i: \CC\to V$ for $i=1,\ldots,N$ and 4 morphisms $x_i: V\to V$ satisfying \eqref{eq:relation}.

The space of all representations without requiring the relations is 
$$R = \End(V)^{\oplus 4}\oplus\hom(E,V)$$
where we used the suggestive notation $E=\CC^N$. It carries an $SO(6n^2)$ vector bundle $B$ trivial with fiber  
\[
\Lambda^2\CC^4\otimes \End(V)
\]
and the pairing $$
b:  \Lambda^2\CC^4\otimes \End(V) \otimes  \Lambda^2\CC^4\otimes \End(V)\xrightarrow{(-\wedge -)\otimes \tr\big(-\circ-\big)} \Lambda^4\CC^4\otimes \End(V) =\End(V)\,.
$$
The existence of an isotropic section 
$$
s: R\to B\,,\qquad (\vec{x},\vec{f})\mapsto \sum_{i\neq j}(e_i\wedge e_j) \otimes x_i\circ x_j\,.
$$
connects it to the local toy model of \cite[(1)]{Oh:2020rnj}.


To construct the Quot scheme, we need to take a quotient by the $GL(V)$-action on $R$  defined for each $g\in GL(V)$ by $\big(\vec{x},\vec{f}\big)\mapsto \big(g\circ \vec{x}\circ g^{-1}, g\circ \vec{f}\big)$. One can also extend it a natural way to an action on $B$. In particular, after defining $R^0\subset R$  as the open subscheme of representations satisfying
$$
\CC[x_1,\ldots, x_4] \cdot \CC\big\langle f_1(1),\ldots, f_N(1)\big\rangle = V\,,
$$
the vector bundle $B$ restricts and then descends to the non-commutative Quot scheme
$$
A = \text{NC}\quot_{X}(E,n) = \big[R^0/GL(V)\big]\,.
$$
The usual Quot scheme is identified with the zero locus
\begin{equation}
\label{eq:quotinncquot}
\quot_{X}(E,n) = s^{-1}(0) \subset A\,.
\end{equation}
We will always use the letter $B$ to denote the vector bundle on $R$, $R^0$, its descent to $\text{NC}\quot_{X}(E,n)$ and restriction to $\quot_{X}(E,n)$ without distinguishing the four cases.

To make it all equivariant, one uses the action of $\T_0\times \T_1$ on $R$ by 
$$
(\vec t;\vec y)\cdot ( \vec x;\vec f)  =  ( t_1\cdot x_1,t_2\cdot x_2,t_3\cdot x_3,t_4\cdot x_4;y_1\cdot f_1,\ldots,y_N\cdot f_N)
$$
which commutes with the action of $\gl(V)$ so it descends to one on $\quot_{X}(E,n)$. If we use $T$ to denote the tangent bundle of $A$, which at each point is the cokernel of some injective map 
\begin{equation}
\label{eq:T}
\End(V)\hookrightarrow R
\end{equation} obtained by differentiating the action of $\gl(V)$ on $R$,
then there is a $\T_0\times \T_1$-equivariant resolution of $\mathbf{R}\shom_{p}(\mathcal{I},\mathcal{I})^\vee_0[-1]$ given by 
$$
B^\bullet = (T\xrightarrow{ds^*} B\xrightarrow{ds}T^*)
$$
which gives the natural $\T_0\times \T_1$-equivariant obstruction theory 
\equa{\label{eqn:res}
\phi: B^\bullet\to L^\bullet_{\quot_{X}(E,n)}\,.
}
Note that the first term in \eqref{eq:T} has trivial weights.

The choice of orientations $o(\L)$ was shown in \cite[Prop. 4.2]{Oh:2020rnj} to be equivalent in this setting to choosing a positive isotropic subbundle $\sI$ of $B$. This is done explicitly in \cite{KR-draft} by constructing
$
\sI
$ as a trivial bundle with the fibre 
\begin{equation}
\label{eq:I}
\langle v\rangle \wedge \langle v\rangle^{\perp}\otimes \End(V)\end{equation} for some non-zero vector $v\in \CC^4$\footnote{In fact, the particular choice of signs compatible with the existing literature corresponds to setting $v=e_4$ for the fourth vector of the canonical basis of $\CC^4$.}. We will not go further into recalling the explicit derivation of $o(\L)|_Z$ done in \cite{KR-draft}, as we only do computations up to a fixed order $n$ to formulate conjectures the proof of which we leave for the future. 

 
\begin{definition}Let $X=\CC^4$, $\alpha=[\oplus_{i=1}^r\O_X\<v_i\>]-[\oplus_{i=r+1}^{r+s}\O_X\<v_i\>]\in K_\T(X)$, and $E=\oplus_{i=1}^N\O_X\<y_i\>$. The \emph{equivariant Segre and Chern series} for a choice of signs $o(\mathcal{L})$ are respectively
\equa{{\mathcal{S}}_X(E,\a;q):=&\sum_{n=0}^\infty q^n\sum_{Z\in\quot_X(E,n)^\T}(-1)^{o(\mathcal{L})|_Z}\frac{s(\alpha^{[n]}|_Z)}{ e\left(\sqrt{T^\vir|_Z}\right)}\\
&\in\frac{\CC(\lambda_1,\lambda_2,\lambda_3,\lambda_4;m_1,\dots ,m_N;w_1,\dots ,w_{r+s})}{(\lambda_1+\lambda_2+\lambda_3+\lambda_4)}[q],\\
{\mathcal{C}}_X(E,\a;q):=&\sum_{n=0}^\infty q^n\sum_{Z\in\quot_X(E,n)^\T}(-1)^{o(\mathcal{L})|_Z}\frac{c(\alpha^{[n]}|_Z)}{ e\left(\sqrt{T^\vir|_Z}\right)}.
}
\end{definition}
\subsection{K-theoretic virtual invariants}
\label{sec:CY4ktheory}
In the setting where the moduli space $M$ is a zero locus of an isotropic section of an $SO(2m)$ bundle $E$ on some ambient space $A$, just like in \eqref{eq:quotinncquot} above, \cite{Oh:2020rnj} give a simpler construction of $\hat{\O}^\vir$ relying on their equivariant \textit{localized K-theoretic square root Euler class} $$\sqrt{\mathfrak{e}_{\T}}(E,s): K_0\big(A,\ZZ\big)\to K_0\big(A,\ZZ[2^{-1}]\big)$$ defined after choosing orientations on $M$.

When $E$ admits an isotropic subbundle $\sI$ compatible with the choice of orientation, then the pushforward under the inclusion 
$$
\iota:M=s^{-1}(0)\hookrightarrow A
$$
becomes just tensoring with the equivariant \textit{K-theoretic square root Euler class} 
$$
\iota_*\big(\sqrt{\mathfrak{e}_{\T}}(E,s)\big) = \otimes \sqrt{\mathfrak{e}_{\T}}(E) = (-1)^{ m}\mathfrak{e}_{\T}(\Lambda^*)\sqrt{\det}\big(\sI^*\big)
$$
where $\mathfrak{e}_{\T}(\sI^*) = \Lambda_{-1}\sI^*$. In fact, $\sqrt{\mathfrak{e}_{\T}}(E,s)$ can be also written as the product
$$
\sqrt{\mathfrak{e}_{\T}}(E,s) = (-1)^m\mathfrak{e}_{\T}(I^*,s) \sqrt{\det}\big(\sI^*\big)\,,
$$
where $\mathfrak{e}_{\T}(I^*,s)$ is some localization of $\mathfrak{e}_{\T}(I^*)$ to $Z(s)$ constructed using Kiem-Li's cosection localization in K-theory \cite{KLcosectionKtheory}. Their equivariant \textit{twisted virtual structure sheaf} $\hat{\O}^{\vir}$ is then constructed as 
\begin{align*}
\hat{\O}^{\vir} &= \sqrt{\mathfrak{e}_T}(E,s)\big)[\![ \O_A]\!]\big) \sqrt{\det}(T^*)\\
&= (-1)^m\mathfrak{e}_{\T}(I^*,s)\sqrt{\det}\big(T^*+\sI^* \big)\,.
\end{align*}
For our case of $M=\quot_{\CC^4}(E,n)$,  we can use \eqref{eq:T} and $\eqref{eq:I}$ to show that $$\det(T) = \det\left(\End(V)\right)^{\otimes 4}\prod_{i=1}^4t_i^{n^2}\det\left(\hom(E,V)\right) =\det\left(\hom(E,V)\right)$$ and 
$\det(\sI) = t_4^{2n^2}$. This implies that the only term in the construction of $\hat{\O}^\vir$ that does not admit a square root is
\begin{align*}
{\det}^{-1}\big(\hom(E,V)\big) &= {\det}^{-1}\big((E^\vee)^{[n]}\big) \\
&= (y_1\ldots y_N)^{-n}{\det}^{-N}(V)\,.
\end{align*}
This motivates the definition of Verlinde numbers for Calabi-Yau fourfolds as we want them to be integers. This gives further motivation for the definition of the \emph{untwisted virtual structure sheaf}
\equanum{\label{def:untwist}\mathcal{O}^\vir :=\hat{\mathcal{O}}^\vir\otimes \mathsf{E}^{\frac{1}{2}}\,,\quad \mathsf{E} = \det((E^\vee)^{[n]}).}
that appeared in \cite[Def. 5.10]{boj}, \cite[§1.4]{Bojko2}. From the above discussion it is clear that the following holds.
\begin{proposition}
\label{prop:integral}
The untwisted virtual structure sheaf is an integral class:
$$
\mathcal{O}^\vir \in K_0\big(\quot_{\CC^4}(E,n),\ZZ\big)\,.
$$
\end{proposition}

Using $\hat{\O}^\vir$ and $\O^\vir$, we define the following \emph{twisted} and \emph{untwisted Euler characteristics}
\[\hat\chi^\vir(\quot_Y(E,n),-) = \chi\big(\quot_Y(E,n),\hat\O^\vir\otimes (-)\big),\]
$$\chi^\vir(\quot_Y(E,n),-) = \chi\big(\quot_Y(E,n),\O^\vir\otimes (-)\big).$$
For compact $X$ and $\alpha\in K^0(X)$, the \emph{Verlinde series} \cite[§1.3]{Bojko2} is then defined by
\equa{{\mathcal{V}}_X(E,\alpha;q):=&\sum_{n=0}^\infty q^n\chi^{\vir}\left(\quot_X(E,n),\text{det}(\alpha^{[n]})\right)\\
=&\sum_{n=0}^\infty q^n\hat\chi^\vir\left(\quot_X(E,n),\text{det}(\alpha^{[n]})\otimes \sqrt{\det}((E^\vee)^{[n]})\right).
}

Using the virtual Riemann-Roch formula and equivariant localization of Oh-Thomas \cite[Theorem 6.1, Theorem 7.3]{Oh:2020rnj}, we have the following \emph{twisted} equivariant virtual Euler characteristic for $X=\CC^4$:
\[\hat\chi^\vir_\T(\quot_X(E,n),\alpha):=\sum_{Z\in\quot_X(E,n)^\T}(-1)^{o(\mathcal{L})|_Z} e\left(-\sqrt{T^\vir|_Z}\right)\sqrt{\td}\left(T^\vir|_Z\right)\ch_\T(\alpha)\]
where $\sqrt{\td}$ is the the square-root Todd class satisfying
\equa{\sqrt{\td}(T^\vir|_Z)=&\td\left(\sqrt{T^\vir|_Z}\right)\ch\left(\sqrt{\det}\sqrt{T^\vir|_Z}^\vee\right)\\
=&\frac{e\left(\sqrt{T^\vir|_Z}\right)}{\ch\left(\Lambda_{-1}\sqrt{T^\vir|_Z}^\vee\right)}\ch\left(\sqrt{K^\vir}^{\frac12}\right).}
Here we denote
\[K^\vir={\det} (T^{\vir})^\vee, \indent \sqrt{K^\vir}={\det}\sqrt{T^\vir}^{\vee}.\]
Substituting into the above equation, we have
\equa{\hat\chi^\vir(\quot_X(E,n),\alpha)=\sum_{Z\in\quot_X(E,n)^\T}(-1)^{o(\mathcal{L})|_Z} \frac{\ch\left(\sqrt{K^\vir|_Z}^\frac12\right)}{\ch\left({\Lambda_{-1}}\sqrt{T^\vir|_Z}^\vee\right)}\ch(\alpha^{[n]}|_Z).}
Now including the twist of (\ref{def:untwist}), we may define the equivariant Verlinde series as follows.

\begin{definition}
The \emph{equivariant Verlinde series} for a choice of signs $o(\mathcal{L})$ is
\equa{{\mathcal{V}}_X(E,\a;q):=&\sum_{n=0}^\infty q^n\sum_{Z\in\quot_X(E,n)^\T}(-1)^{o(\mathcal{L})|_Z}\frac{ \ch\left(\sqrt{K^\vir|_Z}^\frac12\right)\ch\left(\sqrt{\det}((E^\vee)^{[n]}|_Z)\right)}{\ch\left({\Lambda_{-1}}\sqrt{T^\vir|_Z}^\vee\right)}\ch\left(\text{det}(\alpha^{[n]}|_Z)\right)\\
&\indent \in\frac{\QQ(t_1,t_2,t_3,t_4;y_1,\dots ,y_N;v_1,\dots ,v_{r+s})}{(t_1t_2t_3t_4)}[\![q]\!].
}
\end{definition}


The relation between Segre and Verlinde numbers in the compact case is studied in \cite{Bojko2} using the \emph{Nekrasov genus} for Hilbert schemes, introduced for the 3-fold case by \cite{NO}. We consider the following Quot scheme version from \cite{magcolor}
\equanum{\label{def:nek-genus}{\mathcal{N}}_X(E,\a;q)%:=&\mathcal{N}_\a(w,y,e^{\lambda_1},e^{\lambda_2},e^{\lambda_3})\\
:=&\sum_{n=0}^\infty q^n\sum_{Z\in\quot_X(E,n)^\T}(-1)^{o(\mathcal{L})|_Z}\frac{\ch(\sqrt{K^\vir|_Z}^{\frac12})}{\ch(\Lambda_{-1} \sqrt{T^\vir|_Z}^\vee)}
\ch\left(\frac{\Lambda_{-1}}{\sqrt{\det}} \a^{[n]}|_Z\right).
}
\begin{remark}
Recall in \cite{CKM}, the Nekrasov genus for Hilbert schemes involves a variable $y$ coming from a trivial $\CC^*$-action on $X$. This is exactly the $N=1$ case for the above definition, where we have the parameter $y_1$ from the $\T_1$-action on $E$.
\end{remark}

%From the computations in the next section, we see the Verlinde number and Segre number can be extracted from the Nekrasov genus by taking the following limits:
%\equa{&\mathcal{S}(\alpha;q)=\lim_{\e\->0}\mathcal{N}_{-\a}(w\e^{1-r},e^{\e},e^{\e\lambda_1},e^{\e\lambda_2},e^{\e\lambda_3}),\\
%&\mathcal{V}(\alpha;q)=\lim_{\e\->0}\mathcal{N}_{\a\oplus\a\oplus\O_X}(w\e^{r+\frac12},\e).
%}





\subsection{Vertex formalism}
The invariants in the previous section can be calculated for Hilbert schemes using a vertex formalism developed by \cite{CK2}, based on the method introduced in \cite{MNOP1} for Calabi-Yau 3-folds. We generalize this to Quot schemes using the computations from \cite[§2.1]{Bojko2}. First when $N=1$, the $\T$-fixed points of $\hilb^n(X)$ correspond to monomial ideals of $\CC[x_1,x_2,x_3,x_4]$ \cite[Lemma 3.1]{CK1}, which are labeled by solid partitions $\pi$ of size $n$ where
\[\O_{Z_\pi}=\CC[x_1,x_2,x_3,x_4]/I_{Z_{\pi}}=\Span\{x_1^{a}x_2^bx_3^cx_4^d:(a,b,c,d)\in\pi\}.\]
We denote $Q_\pi$ the character of $\O_{Z_\pi}$
\[Q_\pi=\sum_{(i,j,k,l)\in\pi}t_1^{-a}t_2^{-b}t_3^{-c}t_4^{-d}\in K_{\T}^*(\mathrm{pt})=\frac{\ZZ[t_1^{\pm1},t_2^{\pm1},t_3^{\pm1},t_4^{\pm1}]}{(t_1t_2t_3t_4-1)}.\]
Similar to the surface case, for $E=\oplus_{i=1}^N\O_X\<y_i\>$, the $\T$-fixed points for $\quot_X(E,n)$ are labeled by $N$-colored solid partitions $\pi=(\pi^{(1)},\dots,\pi^{(n)})$ of size $n$, i.e. sequences of form
\[Z_\pi=([Z_1],[Z_2],\dots ,[Z_N])\in\hilb^{n_1}(X)\times\dots \times \hilb^{n_N}(X)\]
such that each $Z_i$ corresponds by solid partition $\pi^{(i)}$. 


Let $Q_i$ be the character of $\O_{Z_i}$. The virtual tangent bundle at $Z_\pi$ is
\equanum{\label{eqn:t-vir-4d}T^\vir_{Z_\pi}=&\ext\left(\bigoplus_{i=1}^N I_{\mathcal{Z}_i}\<y_i\>, \bigoplus_{j=1}^NI_{\mathcal{Z}_j}\<y_j\>\right)_0\\
=&\sum_{i,j=1}^N \O_X\otimes (1- \overline{P(I_{Z_i})}P(I_{Z_j}))y_i^{-1}y_j\\
=&\sum_{i,j=1}^N\left(Q_{j}+t_1t_2t_3t_4\overline{Q_{i}}-t_1t_2t_3t_4P_{1234}\overline{Q_{i}}Q_{j}\right)y_i^{-1}y_j
}
where $P(I)$ is the Poincar\'e polynomial of $I$ defined analogously to (\ref{def:poincare}) from previous section, and $P_{I}=\prod_{i\in I}(1-t_i^{-1})$ for any set of indices $I$. Specializing $t_1t_2t_3t_4=1$, we get the following (non-unique) square root
\[\sqrt{T^\vir_{Z_\pi}}=\sum_{i,j=1}^N\left(Q_{j}-\overline{P_{123}}\overline{Q_{i}}Q_{j}\right)y_i^{-1}y_j.\]
The reason for the above choice of square root is so that
\equa{\ch\left(\sqrt{K^\vir|_{Z_\pi}}^{\frac12}\right)&=\ch\left(\prod_{i,j}\text{det}((\overline{Q_j-\overline{P}_{123}Q_j\overline{Q_i}})y_i^{-1}y_j)^{\frac12})\right)\\
&=\ch\left(\prod_{i,j}\sqrt{\det}(\overline {Q_j})y_iy_j^{-1})\right)\\
&=\frac1{\ch\left(\sqrt{\det}((E^\vee)^{[n]}|_{Z_\pi})\right)}
}
matches our twist in (\ref{def:untwist}), and this simplifies our computation as now we have
\[{\mathcal{V}}_X(E,\a;q)=\sum_{\pi}^\infty q^{|\pi|}(-1)^{o(\mathcal{L})|_{Z_\pi}}\frac{ \ch\left(\text{det}(\alpha^{[n]}|_{Z_\pi})\right)}{\ch\left({\Lambda_{-1}}\sqrt{T^\vir|_{Z_\pi}}^\vee\right)}\]
In this case, the signs $(-1)^{o(\mathcal{L})}$ are described in \cite{monavari,KR-draft} as follows: for any solid partition $\pi$,
\[o(\mathcal{L})|_{Z_\pi}:=|\pi|+\#\{(i,i,i,j)\in\pi:i<j\};\]
and for any $N$-colored solid partition $\pi$,
\[o(\mathcal{L})|_{Z_\pi}:=\sum_{i=1}^No(\mathcal{L})|_{Z_{i}}.\]

The fiber of $V^{[n]}=\oplus_{i=1}^r \O_X^{[n]}\<v_i\>$ over ${Z_\pi}=(Z_1,\dots Z_N)$ is the $rn$-dimensional representation
\[V^{[n]}|_{Z_\pi}=\bigoplus_{i=1}^r\bigoplus_{j=1}^N\O_{Z_{j}}\<v_iy_j\>=\left(\sum_{i=1}^r\sum_{j=1}^N \sum_{(a,b,c,d)\in\pi^{(j)}}v_iy_j t_1^{-a}t_2^{-b}t_3^{-c}t_4^{-d}\right)\in K_\T(\pt).\]
Therefore for any point $Z$ corresponding to an $N$-colored solid partition $\pi$, we have
\equa{c(V^{[n]}|_{Z_\pi})=&\prod_{j=1}^N\prod_{(a,b,c,d)\in\pi^{(j)}}\prod_{i=1}^r(1+w_i+m_j-a\lambda_1-b\lambda_2-c\lambda_3-d\lambda_4)\\
\text{det}(V^{[n]}|_{Z_\pi})=&\prod_{j=1}^N\prod_{(a,b,c,d)\in\pi^{(j)}}\prod_{i=1}^rv_it_1^{-a}t_2^{-b}t_3^{-c}t_4^{-d},\\
\ch(\sqrt{K^\vir|_{Z_\pi}}^{\frac12})
\ch\left(\frac{\Lambda_{-1}}{\text{det}^{\frac12}} V^{[n]}|_{Z_\pi}\right)=&\prod_{j=1}^N\prod_{(a,b,c,d)\in\pi^{(j)}}t_1^{\frac a2}t_2^{\frac b2}t_3^{\frac c2}t_4^{\frac d2}\\
&\cdot \prod_{i=1}^r\left(y^{\frac12}v_i^{\frac12}t_1^{-\frac a2}t_2^{-\frac b2}t_3^{-\frac c2}t_4^{-\frac d2}-y_j^{-\frac12}v_i^{-\frac12}t_1^{\frac a2}t_2^{\frac b2}t_3^{\frac c2}t_4^{\frac d2}\right).
}
Using these expressions, we see the Chern and Verlinde series can be extracted by taking limits of the Nekrasov genus, similar to the surface case. Also, it follows that
\equa{{\mathcal{N}}_X(E,V;q)\in\frac{\QQ(t_1^{\frac12},t_2^{\frac12},t_3^{\frac12},t_4^{\frac12})}{(t_1t_2t_3t_4-1)}[\![q,y_1^{\pm\frac12},\dots ,y_N^{\pm\frac12},v_1^{\pm\frac12},\dots ,v_r^{\pm\frac12}]\!].}
The argument of \cite[Proposition 1.13, 1.15]{CKM} can be applied to show that ${\mathcal{N}}_X(E,V;q)$ in fact lives in $\frac{\QQ(t_1,t_2,t_3,t_4)}{(t_1t_2t_3t_4-1)}[\![q,y_1^{\pm\frac12},\dots ,y_N^{\pm\frac12},v_1^{\pm\frac12},\dots ,v_r^{\pm\frac12}]\!]$. This enables us to talk about admissibility (up to specializing $t_1t_2t_3t_4=1$) in the sense of Definition \ref{def:admissible}. 


\subsection{Factor of \texorpdfstring{$c_3(X)$}{Lg}}\label{sec:factor}
In the surface case, we saw the powers in the universal series of virtual invariants are multiples of $c_1(S)$. In the $X=\CC^4$ case, we shall show that if the universal expressions exist, then they are multiples of $c_3(X)$ by showing show that \equa{e\left(-\sqrt{T^\vir|_{Z_\pi}}\right)}
has $c_3(X)=-(\lambda_1+\lambda_2)(\lambda_1+\lambda_3)(\lambda_2+\lambda_3)$ in its numerator. This factor of $c_3(X)$ and the weak Segre-Verlinde correspondence in the surface case shall motivate Conjecture \ref{con:sv4d-intro}. 

It suffices to show that this term vanishes when we set $\lambda_i=-\lambda_j$ for $i\neq j$ in $\{1,2,3\}$. By symmetry, we may assume $i=1,j=2$. 
%Setting $\lambda_1=-\lambda_2$ is the same as restricting to the 2-dimensional sub-torus of $\T$
%\[\T'=\{(t_1,t_2,t_3,t_4):t_1t_2=1,t_3t_4=1\}\]
%whose representation ring and cohomology are
%\[K_{\T'}(\pt)=\frac{\ZZ[t_1^{\pm1},t_2^{\pm1},t_3^{\pm1},t_4^{\pm1}]}{(t_1t_2-1,t_3t_4-1)},\indent H_{\T'}^*(\pt)=\frac{\CC[\lambda_1,\lambda_2,\lambda_3,\lambda_4]}{(\lambda_1+\lambda_2,\lambda_3+\lambda_4)}\]
Recall that $e$ is the top equivariant Chern class which vanishes if its input has a trivial summand. The process of setting $\lam_1=-\lam_2$ in cohomology is the same as setting $t_1=t_2^{-1}$ in K-theory. Therefore we would like to show that $-\sqrt{T^\vir|_{Z_\pi}}$ has a trivial summand when we set $t_1=t_2^{-1}$, i.e. the character of $\sqrt{T^\vir|_{Z_\pi}}$ in $K_\T(\pt)$ having a strictly negative constant term. This occurs if and only if the image of $T^\vir_{Z_\pi}$ in 
\[\ZZ[t_1^{\pm1},t_2^{\pm1},t_3^{\pm1},t_4^{\pm1}]/(t_1t_2-1,t_3t_4-1)\]
has a strictly negative constant term (which is necessarily a negative even integer). From (\ref{eqn:t-vir-4d}), we see it suffices to show this for the term
\[Q_{\pi}+t_1t_2t_3t_4\overline{Q_{\pi}}-t_1t_2t_3t_4P_{1234}\overline{Q_{\pi}}Q_{\pi}\]
whenever $\pi$ is a non-trivial solid partition. 

\begin{lemma}
For any non-trivial solid partition $\pi$, the expression 
\[Q_{\pi}+t_1t_2t_3t_4\overline{Q_{\pi}}-t_1t_2t_3t_4P_{1234}\overline{Q_{\pi}}Q_{\pi}\]
has a strictly negative constant term when viewed in the quotient ring
\[\ZZ[t_1^{\pm1},t_2^{\pm1},t_3^{\pm1},t_4^{\pm1}]/(t_1t_2-1,t_3t_4-1).\]
\end{lemma}
\begin{proof}
Let $x=t_1=\frac1{t_2}$, $y=t_3=\frac1{t_4}$, so that
\[\ZZ[t_1^{\pm1},t_2^{\pm1},t_3^{\pm1},t_4^{\pm1}]/(t_1t_2-1,t_3t_4-1)=\ZZ[x^{\pm1},y^{\pm1}].\]
Let $P_\pi$ be the image of $Q_\pi$ in $\ZZ[x^{\pm1},y^{\pm1}]$, then
\equanum{\label{eqn:vir-tan-image}T^\vir|_{Z_\pi}=P_\pi + \overline{P_\pi} - P_\pi \overline{P_\pi} (1-x)(1-\frac1x)(1-y)(1-\frac1y).}
Write
\[P_\pi=\sum_{i,j\in\ZZ}p_{i,j}x^iy^j.\]
The image of $\overline{Q_\pi}$ is then
\[\overline{P_\pi}=\sum_{i,j\in\ZZ}p_{i,j}x^{-i}y^{-j}.\]
We see the constant terms of $P_\pi$ and $\overline{P_\pi}$ are both $p_{0,0}$. By definition, all monomial terms in $Q_\pi$ have positive coefficients, and $Q_\pi$ has constant term 1, so $p_{0,0}>0$. We need to find the constant term of $P_\pi \overline{P_\pi} (1-x)(1-\frac1x)(1-y)(1-\frac1y)$.



Observe that
\[(1-x)(1-\frac1x)(1-y)(1-\frac1y)=4-2\left(x+y+\frac1x+\frac1y\right)+\left(xy+\frac1{xy}+\frac xy+\frac yx\right).\]
Write $F=\sum f_{i,j}x^iy^j$, the constant term of $F\cdot (1-x)(1-\frac1x)(1-y)(1-\frac1y)$ is equal to
\equanum{\label{const-term}4f_{0,0}-2(f_{0,1}+f_{1,0}+f_{0,-1}+f_{-1,0})+(f_{1,1}+f_{1,-1}+f_{-1,1}+f_{-1,-1}).}
If we set $F=P_\pi\overline{P_\pi}$, then
\[f_{i,j}=\sum_{\substack{
a-c=i\\
b-d=j}}p_{a,b}p_{c,d}.\]
In particular,
\equa{&f_{0,0}=\sum_{a,b\in\ZZ} p_{a,b}^2,\\
&f_{0,1}+f_{1,0}+f_{0,-1}+f_{-1,0}=\sum_{a,b\in\ZZ} p_{a,b}(p_{{a-1},b}+p_{{a+1},b}+p_{{a},b-1}+p_{{a},b+1}),\\
&f_{1,1}+f_{1,-1}+f_{-1,1}+f_{-1,-1}=\sum_{a,b\in\ZZ} p_{a,b}(p_{{a+1},b+1}+p_{{a+1},b-1}+p_{{a-1},b-1}+p_{{a-1},b+1}).
}
Denote 
\equa{&s_{a,b}=4p_{a,b}-2(p_{{a-1},b}+p_{{a+1},b}+p_{{a},b-1}+p_{{a},b+1});\\
&\indent\indent+(p_{{a+1},b+1}+p_{{a+1},b-1}+p_{{a-1},b-1}+p_{{a-1},b+1}),\\
&s_{a,b}^{++}=p_{a,b}-(p_{{a+1},b}+p_{{a},b+1})+p_{{a+1},b+1},\\
&s_{a,b}^{+-}=p_{a,b}-(p_{{a+1},b}+p_{{a},b-1})+p_{{a+1},b-1},\\
&s_{a,b}^{-+}=p_{a,b}-(p_{{a-1},b}+p_{{a},b+1})+p_{{a-1},b+1},\\
&s_{a,b}^{--}=p_{a,b}-(p_{{a-1},b}+p_{{a},b-1})+p_{{a-1},b-1};\\
}
\equa{
&S^{++}=\sum_{a,b\geq 0}p_{a,b}s_{a,b}^{++}+p_{a+1,b}s_{a+1,b}^{-+}+p_{a,b+1}s_{a,b+1}^{+-}+p_{a+1,b+1}s_{a+1,b+1}^{--},\\
&S^{+-}=\sum_{a\geq 0,b\leq 0}p_{a,b}s_{a,b}^{+-}+p_{a+1,b}s_{a+1,b}^{--}+p_{a,b-1}s_{a,b-1}^{++}+p_{a+1,b-1}s_{a+1,b-1}^{-+},\\
&S^{-+}=\sum_{a\leq 0,b\geq 0}p_{a,b}s_{a,b}^{-+}+p_{a+1,b}s_{a+1,b}^{++}+p_{a,b+1}s_{a,b+1}^{--}+p_{a+1,b+1}s_{a+1,b+1}^{+-},\\
&S^{--}=\sum_{a,b\leq 0}p_{a,b}s_{a,b}^{--}+p_{a+1,b}s_{a+1,b}^{+-}+p_{a,b+1}s_{a,b+1}^{-+}+p_{a+1,b+1}s_{a+1,b+1}^{++}.\\
}
Then (\ref{const-term}) becomes
\equa{\sum_{a,b\in\ZZ}p_{a,b}s_{a,b}=&\sum_{a,b\in\ZZ}p_{a,b}\cdot(s_{a,b}^{++}+s_{a,b}^{+-}+s_{a,b}^{-+}+s_{a,b}^{--})\\
=&S^{++}+S^{+-}+S^{-+}+S^{--}.
}

For the remainder of this proof, we shall show $S^{++}\geq p_{0,0}$. The same will hold for the summands $S^{+-},S^{-+},S^{--}$ by symmetry. We conclude the value of (\ref{const-term}) is at least $4p_{0,0}$. Hence by (\ref{eqn:vir-tan-image}) the constant term of $T^{\vir}|_{Z_\pi}$ is at most $-2p_{0,0}<0$, and we are done.


Recall 
\[Q_\pi=\sum_{(i,j,k,l)\in\pi}t_1^it_2^jt_3^kt_4^l,\]
so
\[P_\pi=\sum_{(i,j,k,l)\in\pi}x^{i-j}y^{k-l},\text{ and}\]
\[p_{a,b}=\#\{(i,j,k,l)\in\pi:i-j=a,k-l=b\}.\]
Fix $k$ and $l$, then the set
$\{(i,j): (i,j,k,l)\in\pi\}$
is a plane partition. 
%Assuming $a\geq 0$, The set of boxes $\square=(i,j)$ such that $i-j=a$ are the ones lined up diagonally, starting from $(0,a)$, going towards the top right-hand side. 
By property (\ref{solid-partition}) of solid partitions, for fixed $b=k-l$, we have $p_{a,b}\geq p_{{a+1},b}$ when $a\geq 0$. For the same reason, we have $p_{a,b}\geq p_{{a},b+1}$ when $b\geq 0$.
Therefore the numbers $(p_{a,b})_{a,b\geq 0}$ are non-increasing as the pair $(a,b)$ move away from the origin. 


We apply induction on $\max\{a:p_{a,0}\neq 0\}$. Suppose for all sequences $(q_{a,b})$ with $\max\{a:q_{a,0}\neq 0\}<\max\{a:p_{a,0}\neq 0\}$, we have
\[S^{++}(q_{a,b})\geq q_{0,0}\]
whenever the sequence $(q_{a,b})_{a,b\geq 0}$ satisfies $q_{a,b}$ is non-increasing in $a,b$. The base case is simply when $q_{a,b}=0$ for all $a,b$, which sums to 0. Let $q_{a,b}=p_{{a+1},b}$, then
\equanum{\label{ineq}S^{++}(p_{a,b})=&\sum_{a,b\geq 0}(p_{a,b}-p_{a+1,b}-p_{a,b+1}+p_{a+1,b+1})^2\\
=&S^{++}(q_{a,b})+\sum_{b\geq 0}(p_{0,b}-p_{1,b}-p_{0,b+1}+p_{1,b+1})^2\\
\geq& p_{1,0}+\sum_{b\geq 0}(p_{0,b}-p_{1,b}-p_{0,b+1}+p_{1,b+1})^2
}
where the first equality follows from the definition and the inequality is by induction hypothesis. 

Now apply another induction on the value of $\max\{b:p_{0,b}\neq 0\}$. The induction hypothesis is that for any sequences $(q_{a,b})$ with $q_{a,b}$ non-increasing in $a,b$ and $\max\{b:q_{0,b}\neq 0\}<\max\{b:p_{0,b}\neq 0\}$, we have
\[\sum_{b\geq 0}(q_{0,b}-q_{1,b}-q_{0,b+1}+q_{1,b+1})^2\geq q_{0,0}-q_{1,0}.\]
Again, the base case is trivial, and we can apply the hypothesis to $q_{a,b}=p_{a,b+1}$, giving us
\[\sum_{b\geq 1}(p_{0,b}-p_{1,b}-p_{0,b+1}+p_{1,b+1})^2\geq p_{0,1}-p_{1,1}\]
So we have the following inequalities
\equa{&(p_{0,0}-p_{1,0}-p_{0,1}+p_{1,1})^2+\sum_{b\geq 1}(p_{0,b}-p_{1,b}-p_{0,b+1}+p_{1,b+1})^2-(p_{0,0}-p_{1,0})\\
\geq& (p_{0,0}-p_{1,0}-p_{0,1}+p_{1,1})^2-(p_{0,0}-p_{1,0}-p_{0,1}+p_{1,1})\\
\geq& 0
}
where the last inequality is due to $p_{0,0}-p_{1,0}-p_{0,1}+p_{1,1}$ being an integer. Therefore
\[\sum_{b\geq 0}(p_{0,b}-p_{1,b}-p_{0,b+1}+p_{1,b+1})^2\geq p_{0,0}-p_{1,0},\]
finishing the second induction. By (\ref{ineq}),
\[S^{++}(p_{a,b})\geq p_{1,0}+(p_{0,0}-p_{1,0})=p_{0,0},\]
which finishes the first induction and the proof.


\end{proof}


\subsection{Cohomological limits}
Recall the proof of Theorem \ref{thm:SV-univ-sieres} mainly involved showing that the genus $I_S$ is admissible in the sense of Definition \ref{def:admissible}. Also, by Proposition \ref{prop:series-admissible}, universal series expressions for the Nekrasov genus, and therefore the Segre and Verlinde series, can be obtained if and only if the Nekrasov genus is admissible. Thus one might ask when the Nekrasov genus is admissible. For the rank $r=N$ case, we shall show that admissibility is a consequence of the following explicit formula conjectured by Nekrasov-Piazzalunga \cite[§2.5]{magcolor}. Denote
\[[x]=x^{\frac12}-x^{-\frac12}.\]

\begin{conjecture}[Nekrasov-Piazzalunga]\label{con:nek2}
There exists some choice of signs $o(\mathcal{L})$ such that for $E=\oplus_{i=1}^N\O_X\<y_i\>$, $V=\oplus_{i=1}^N\O_X\<v_i\>$, 
\[{\mathcal{N}}_X(E,V;q)=\Exp\left(\frac{[t_1t_2][t_2t_3][t_1t_3]}{[t_1][t_2][t_3][t_4]}\frac{[s]}{[s^{\frac12}q][s^{-\frac12}q]}\right)\]
with a change of variable $s=\prod_{i=1}^Ny_iv_i$.
\end{conjecture}

\begin{proposition}
Nekrasov-Piazzalunga's Conjecture \ref{con:nek2} implies that the Nekrasov genus $\mathcal{N}_X$ of rank $r=N$ is admissible with respect to the variables $t_1,t_2,t_3,t_4$.
\end{proposition}
\begin{proof}
Expand the term inside the plethystic exponential, specializing with the relation $t_1t_2t_3t_4=1$, we have
\[\frac{[t_1t_2][t_2t_3][t_1t_3][s]}{[t_1][t_2][t_3][t_4][s^{\frac12}q][s^{-\frac12}q]}=\frac{(1-t_1t_2)(1-t_2t_3)(1-t_1t_3)}{(1-t_1)(1-t_2)(1-t_3)(1-t_4)}\cdot \frac{[s]}{[s^{\frac12}q][s^{-\frac12}q]}\]
Recall Definition \ref{def:admissible}, we have 
\[L=(1-t_1t_2)(1-t_2t_3)(1-t_1t_3)\frac{[s]}{[s^{\frac12}q][s^{-\frac12}q]}\]
is a series in $q,y_1^{\pm\frac12},\dots ,y_N^{\pm\frac12},v^{\pm\frac12}_1,\dots ,v^{\pm\frac12}_r$ whose coefficients are polynomials in $t_1,t_2,t_3,t_4$, as required.
\end{proof}



Lastly, we prove the claim made in the introduction that Conjecture \ref{con:cao-kool-quot} is a consequence of Conjecture \ref{con:nek2} in the $X=\CC^4$ case.

\begin{proposition}\label{prop:nek-con-limit}
Let $X=\CC^4$. If Conjecture \ref{con:nek2} holds for some choice of signs, then Conjecture \ref{con:cao-kool-quot} holds for $Y=X$.

In particular, we may retrieve the following well-known identity from Conjecture \ref{con:nek2}:
\equa{\sum_{n=0}^\infty q^n\int_{[\quot_{\CC^4}(E,n)]^\vir_{o(\mathcal{L})}}1:=&\sum_{n=0}^\infty q^n \sum_{Z\in\quot_X(E,n)^\T}(-1)^{o(\mathcal{L})|_Z}\frac{1}{e_\T\left(\sqrt{T^\vir_Z}\right)}\\
=&\begin{cases}
e^{\frac{(\lambda_1+\lambda_2)(\lambda_1+\lambda_3)(\lambda_2+\lambda_3)}{\lambda_1\lambda_2\lambda_3(\lambda_1+\lambda_2+\lambda_3)}q},&\text{ when $N=1$}\\
\hfil1,&\text{ otherwise.}
\end{cases}}
\end{proposition} 
\begin{remark}
One can compare this to the 3-fold case where \cite[Theorem 7.2]{Quot-DT} states
\[\sum_{n=0}^\infty q^n\int_{[\quot_{\CC^3}(E,n)]^\vir}1=M((-1)^Nq)^{-N\frac{(\lambda_1+\lambda_2)(\lambda_1+\lambda_3)(\lambda_2+\lambda_3)}{\lambda_1\lambda_2\lambda_3}}.\]
Here $M$ denotes the MacMahon function.
\end{remark}
\begin{proof}
We shall compute the following limit using both the definition and the expression from Conjecture \ref{con:nek2}, then compare the two sides:
\[\lim_{\substack{\e\->0\\w_N\->\infty}} {\mathcal{N}}_X\left(E,V;\frac Q{w_N}\right)\bigg\vert_{\lam_i\leadsto\e\lam_i,m_i\leadsto \e(1+m_i),w_i\leadsto\e w_i}.\]
%First recall the identity
%\[\ch\left(\frac{\Lambda_{-1}}{\det^{\frac12}}L^\vee\right)\bigg\vert_{t_i=e^{b\lambda_i},y_i=e^{bm_i},v_i=e^{bw_i}}=e_\T(L)b+\O(b^2).\]
Let $V=\oplus_{i=1}^N\O_X\<v_i\>$ be a rank $N$ bundle, then for any $Z_\pi\in\quot_X(E,n)^\T$, we have
\equa{&\frac{\ch_\T(\sqrt{K^\vir|_{Z_\pi}}^{\frac12})}{\ch_\T(\Lambda_{-1} \sqrt{T^\vir|_{Z_\pi}}^\vee)}\ch_\T\left(\frac{\Lambda_{-1}}{\sqrt{\det}} V^{[n]}|_{Z_\pi}\right)\bigg\vert_{\lam_i\leadsto\e\lam_i,m_i\leadsto \e(1+m_i),w_i\leadsto\e w_i}\\
=&\e ^{Nn-Nn}\frac{e_\T(V^{[n]}|_{Z_\pi})+O(\e )}{e_\T\left(\sqrt{T^\vir_{Z_\pi}}\right)+O(\e )}\\
=&\frac{\prod_{i=1}^N\prod_{j=1}^N\prod_{(a,b,c,d)\in\pi^{(j)}}(1+w_i+m_j+a\lambda_1+b\lambda_2+c\lambda_3+d\lambda_4)+O(\e )}{e_\T\left(\sqrt{T^\vir_{Z_\pi}}\right)+O(\e )}
.}
Take limit $\e \->0$ and let $Q=m_Nq$, then
\equa{&\lim_{\e \->0}\frac{\ch_\T(\sqrt{K^\vir|_{Z_\pi}}^{\frac12})}{\ch_\T(\Lambda_{-1} \sqrt{T^\vir|_{Z_\pi}}^\vee)}\ch_\T\left(\frac{\Lambda_{-1}}{\sqrt{\det}} V^{[n]}|_{Z_\pi}\right)\bigg\vert_{\lam_i\leadsto\e\lam_i,m_i\leadsto \e(1+m_i),w_i\leadsto\e w_i}\cdot q^n\\
=& \frac{\prod_{i=1}^N\prod_{j=1}^N\prod_{(a,b,c,d)\in\pi^{(j)}}(1+w_i+m_j-a\lambda_1-b\lambda_2-c\lambda_3-d\lambda_4)}{e_\T\left(\sqrt{T^\vir_{Z_\pi}}\right)}\cdot\frac{Q^n}{m_N^n}\\
=&\frac{\prod_{i=1}^{N-1}\prod_{j=1}^{N}\prod_{(a,b,c,d)\in\pi^{(j)}}(1+w_i+m_j-a\lambda_1-b\lambda_2-c\lambda_3-d\lambda_4)}{e_\T\left(\sqrt{T^\vir_{Z_\pi}}\right)}\\
&\cdot \prod_{j=1}^N\prod_{(a,b,c,d)\in\pi^{(j)}}\left(1+\frac{m_j}{w_N}-\frac{a\lambda_1}{w_N}-\frac{b\lambda_2}{w_N}-\frac{c\lambda_3}{w_N}-\frac{d\lambda_4}{w_N}\right) Q^n
.}
Now take $w_N\->\infty$ and substitute into Definition \ref{def:nek-genus}. Let $V'=\oplus_{i=1}^{N-1}\O_X\<v_i\>$, then
\equanum{\label{eqn:neklim-def}\lim_{\substack{\e \->0\\w_N\->\infty}} {\mathcal{N}}_X\left(E,V;\frac Q{w_N}\right)\bigg\vert_{\lam_i\leadsto\e\lam_i,m_i\leadsto \e(1+m_i),w_i\leadsto\e w_i}={\mathcal{C}}_X(E,V';Q).}

On the other hand, we apply the same procedure to 
\[{\mathcal{N}}_X(E,V;q)=\Exp\left(\frac{[t_1t_2][t_2t_3][t_1t_3]}{[t_1][t_2][t_3][t_4]}\frac{[s]}{[s^{\frac12}q][s^{-\frac12}q]}\right).\]
For $n\geq 1$, we have
\equa{&\lim_{\substack{\e \->0\\w_N\->\infty}}\frac{[t_1^nt_2^n][t_2^nt_3^n][t_1^nt_3^n]}{[t_1^m][t_2^n][t_3^n][t_4^n]}\frac{[\prod y_i^nv_i^n]}{[\prod y_i^{\frac n2}v_i^{\frac n2}q^n][\prod y_i^{-\frac n2}v_i^{-\frac n2}q^{n}]}\bigg\vert_{\lam_i\leadsto\e\lam_i,m_i\leadsto \e(1+m_i),w_i\leadsto\e w_i}\\
=&\lim_{\substack{\e \->0\\w_N\->\infty}}\frac{(\e n)^3(\lambda_1+\lambda_2)(\lambda_1+\lambda_3)(\lambda_2+\lambda_3)+O(\e ^5)}{(\e n)^4\lambda_1\lambda_2\lambda_3(\lambda_1+\lambda_2+\lambda_3)+O(\e ^5)}\cdot \frac{(\e n)\sum_i(1+m_i+w_i)+O(\e )}{(q^{\frac n2}-q^{-\frac n2})^2}\\
=&\lim_{w_N\->\infty}\frac{(\lambda_1+\lambda_2)(\lambda_1+\lambda_3)(\lambda_2+\lambda_3)}{\lambda_1\lambda_2\lambda_3(\lambda_1+\lambda_2+\lambda_3)}\cdot \frac{\sum_i(1+m_i+w_i)(\frac{Q}{w_N})^n}{(1-(\frac{Q}{w_N})^n)^2}\\
=&\begin{cases}
\frac{(\lambda_1+\lambda_2)(\lambda_1+\lambda_3)(\lambda_2+\lambda_3)}{\lambda_1\lambda_2\lambda_3(\lambda_1+\lambda_2+\lambda_3)}Q,&\text{ when $n=1$}\\
\hfil0,&\text{ otherwise.}
\end{cases}
}
Together with (\ref{eqn:neklim-def}), we have
\equa{{\mathcal{C}}_X(E,V';Q)=&e^{\frac{(\lambda_1+\lambda_2)(\lambda_1+\lambda_3)(\lambda_2+\lambda_3)}{\lambda_1\lambda_2\lambda_3(\lambda_1+\lambda_2+\lambda_3)}Q}\\
=&\exp\left(Q\int_Xc_3(X)\right).
}
This is exactly Conjecture \ref{con:cao-kool-quot}.


With the same method, we can take limits
\[\lim_{\substack{\e \->0\\w_N\->\infty}} {\mathcal{N}}_X\left(E,V;\frac Q{w_Nw_{N-1}\dots w_{N-i+1}}\right)\]
for $1<i\leq N$ and $V$ of rank $N-i$, and get
\[{\mathcal{C}}_X(E,V;Q)=1.\]
In particular, when $i=N$ and $N>1$, we have
\equa{\sum_{n=0}^\infty Q^n\int_{[\quot_{\CC^4}(E,n)]^\vir_{o(\mathcal{L})}}1=1.
}
\end{proof}

