Recall that we use the notation $H_{\mu,\nu,\xi}$ for the series from (\ref{eqn:log-nek2d-vir}) describing the virtual Nekrasov genus for  Quot schemes on $\CC^2$. For a smooth projective surface $S$, a torsion-free sheaf $E$ of rank $N$, and a K-theory class $\a$ of rank $r$, we apply \cite[Equation (4.1), Theorem 4.1]{Bojko2} by setting $f(x)=1-ze^x$, $g(x)=\frac{x}{1-e^{-x}}$ and get
\[\mathcal{N}_S(E,V;q,z)=\left(\prod_{i=1}^NF(H_i)\right)^{c_1(S)c_1(\a)}\left(\prod_{i=1}^NF(H_i)\right)^{\frac{r}{N}c_1(S)c_1(E)}G(R)^{c_1(S)^2}.\]
Here $R=f^rg^N$, $F(x)=\frac{f(x)}{f(0)}$, the series $H_i(q)$ are Newton–Puiseux solutions to 
\[H_i^N=q R(H_i),\]
and $G(R)$ is given by \cite[Equation (4.24)]{Bojko2}. Therefore
\equanum{\label{eqn:general-N}H_{(1),(0),(0)}(q,z)&=\sum_{i=1}^N\log F(H_i)\\
H_{(0),(1),(0)}(q,z)&=\log G(R)\\
H_{(0),(0),(1)}(q,z)&=\frac{r}{N}\sum_{i=1}^N\log F(H_i)
}

Now let $S=\CC^2$. Note that $\mathcal{N}_S$ satisfies
\equa{{\mathcal{N}}_{S}(y_1,\dots,y_N;v_1,\dots,v_r;q,z)&={\mathcal{N}}_{S}(y_1,\dots,y_N;ze^{w_1},\dots,ze^{w_r};q,1)\\
&={\mathcal{N}}_{S}(ze^{m_1},\dots,ze^{m_N};v_1,\dots,v_r;q,1).
}
Applying Lemma \ref{lem:differential} to $H_{\mu,\nu,\xi}$ in the variables $w_1,\dots, w_r$ and gives us for $r>0$,
\[D_z^{k}H_{(0),\nu,\xi}(q,z)=rD_z^{k-1}H_{(1),\nu,\xi}(q,z)=k!\sum_{|\mu|=k}\binom{r}{\mu}H_{\mu,\nu,\xi}(q,z),\]
while applying in the variables $m_1,\dots,m_N$ yields
\equanum{\label{eqn:xi}D_z^{k}H_{\mu,\nu,(0)}(q,z)=ND_z^{k-1}H_{\mu,\nu,(1)}(q,z)=k!\sum_{|\xi|=k}\binom{N}{\xi}H_{\mu,\nu,\xi}(q,z).}
When the rank $r$ is negative, we consider $\a=-[V]$ where $V=\oplus_{i=1}^{-r}\O_S\<v_i\>$, and the same argument applies. Thus for all $r\neq 0$,
\equanum{\label{eqn:mu}D_z^{k}H_{(0),\nu,\xi}(q,z)=|r|D_z^{k-1}H_{(1),\nu,\xi}(q,z)=k!\sum_{|\mu|=k}\binom{|r|}{\mu}H_{\mu,\nu,\xi}(q,z).}


Our goal for this section is to apply Chern and Verlinde limits to (\ref{eqn:xi}) and (\ref{eqn:mu}), together with the explicit expressions (\ref{eqn:general-N}), and obtain relations between the Chern and Verlinde series for various $\mu$, $\nu$ and $\xi$. 



\subsubsection{The Chern limit}
For an arbitrary series $f(q,z)$, we define its \emph{Chern limit of degree $k$ and rank $r$} by
\[\lim_{\e\->0}(-\e)^kf\left((-1)^Nq\e^{N-r}(1+\e)^r,\frac1{1+\e}\right).\]
%and its \emph{Verlinde limit of rank $r$} by
%\[\lim_{\e->0}f((-1)^{r}q\e^{r},\e^{-1}).\]
Note that when applied to $H_{\mu,\nu,\xi}$, the Chern limit of degree $k=|\mu|+|\nu|+|\xi|-1$ in this sense agrees with the Chern limit of Lemma \ref{lem:CV-limit}. These limits can be computed using the following analogue of \cite[Lemma 4.2]{GM}.

\begin{lemma}\label{lem:Chern-limit-arb}
Suppose $f(q,z)=\sum_{m,n\geq 0} f_{m,n}q^mz^n$ such that
\[f_{m,n}=(-1)^np_m(n)\binom{rm}{n}\]
for some polynomial $p_m(x)$ of degree at most $mN+k$. The Chern limit of degree $k$ for $f$ is
\[\lim_{\e\->0}(-\e)^{k} f\left((-1)^Nq\e^{N-r}(1+\e)^r,\frac1{1+\e}\right)=\sum_{m=0}^\infty [x^{mN+k}]p_m(x)(rm)_{(mN+k)}q^m.\]
\end{lemma}
\begin{proof}
First observe that both sides of the identity are polynomials in $r$, so it suffices to prove the equality for $r$ large enough. We shall assume $r>N+k\geq 0$, so $\binom{rm}{n}$ vanishes when $n>rm$. Let $g(q,\e)=f\left((-1)^Nq\e^{N-r}(1+\e)^r,(1+\e)^{-1}\right)$, $f_m(z)=\sum_{n=0}^{rm} f_{m,n}z^n$ then
\[[q^m]g(q,\e)=(-1)^{mN}f_m\left(\frac1{1+\e}\right)(1+\e)^{rm}\e^{mN-rm}=\sum_{i=0}^{rm}c_i\e^{mN-i}\]
for some numbers $c_i$. Substitute $\e=z^{-1}-1$, we have
\equa{\sum_{n=0}^{rm}(-1)^np_m(n)\binom{rm}{n}z^n=&f_m(z)\\
=&(-1)^{mN}\sum_{i=0}^{rm}c_iz^i(1-z)^{rm-i}\\
=&(-1)^{mN}\sum_{i=0}^{rm}c_i\sum_{n=i}^{rm}(-1)^{n-i}\binom{rm-i}{n-i}z^n.
}
Therefore as we assume $r>N+k$, we may write
\[p_m(x)=\sum_{i=0}^{rm}(-1)^{mN+i}\frac{(x)_{(i)}}{(rm)_{(i)}}c_i.\]
Since the degree of $p_m(x)$ must be at most $mN+k$, we can extract the coefficient
\[[x^{mN+k}]p_m(x)=(-1)^{k}\frac{c_{m+k-1}}{(rm)_{(mN+k)}}\]
and get
\equa{\lim_{\e\->0}(-\e)^{k} f\left((-1)^Nq\e^{N-r}(1+\e)^r,\frac1{1+\e}\right)&=\sum_{m=0}^\infty(-1)^{k}c_{mN+k}q^m\\
&=\sum_{m=0}^\infty [x^{mN+k}]p_m(x)(rm)_{(mN+k)}q^m.}

\end{proof}

\subsubsection{The Verlinde limit}
For an arbitrary series $f(q,z)$, its \emph{Verlinde limit of rank $r$} is
\[\lim_{\e\->0}f((-1)^rq\e^r,\e^{-1}).\]

Suppose $r>0$ and $f(q,z)\in\QQ[z][\![q]\!]$. Write $f(q,z)=\sum_{m=0}^\infty f_m(z)q^m$, then this limit makes sense when $f_m$ are polynomials of degree at most $rm$. In this case, we have for each $m>0$,
\[[q^m]\lim_{\e\->0}f((-1)^rq\e^r,\e^{-1})=(-1)^{rm}[z^{rm}]f_m(z)\]
Now suppose $r<0$ and $f\in\QQ[\![q,z]\!]$. For each $m> 0$, the limit extracts the following coefficients term-wise
\[[q^m]\lim_{\e\->0}f((-1)^rq\e^r,\e^{-1})=(-1)^{rm}[z^{-rm}]f_m(z^{-1}).\]
In particular, we have
\equanum{\label{eqn:V-lim}[q^m]\lim_{\e\->0}(D_z^kf)((-1)^rq\e^r,\e^{-1})=(|r|m)^k[q^m]\lim_{\e\->0}f((-1)^rq\e^r,\e^{-1}).}

{\color{red}Should not have abs value here}

\subsubsection{Computations}

We begin with the case $\nu=\xi=(0)$ and apply the Chern limit of Lemma \ref{lem:CV-limit} to $|r|D_z^{k-1}H_{(1),(0),(0)}(q,z)$ for $k> 0$. By \cite[Corollary 2]{bojko-lag}, we have
\[[q^m]H_{(1),(0),(0)}=\frac{1}{m}[t^{mN-1}]\left(-e^tz(1-e^tz)^{rm-1}\left(\frac{t}{1-e^{-t}}\right)^{mN}\right).\]
We may extract the $z^n$ coefficient and get
\equa{%\label{H100znqm}
[z^nq^m]H_{(1),(0),(0)}&=\frac{(-1)^n}{m}\binom{rm-1}{n-1}[t^{mN-1}]\left(e^{tn}\left(\frac{t}{1-e^{-t}}\right)^{mN}\right)\\
&=(-1)^n\frac{n}{rm^2}[t^{mN-1}]\left(e^{tn}\left(\frac{t}{1-e^{-t}}\right)^{mN}\right)\binom{rm}{n}.
}
By the identity $D_z(z^n)=nz^n$, we have
\equa{|r|[z^nq^m]D_z^{k-1}H_{(1),(0),(0)}&=(-1)^n\frac{n^k|r|}{rm^2}[t^{mN-1}]\left(e^{tn}\left(\frac{t}{1-e^{-t}}\right)^{mN}\right)\binom{rm}{n}
}

Since the right-hand side of (\ref{eqn:mu}) consists of universal series whose powers have degree $k-1$ \AC{I know what you mean but this should be phrased a bit better.}, we take the Chern limit of degree $k-1$ via Lemma \ref{lem:Chern-limit-arb} by setting $p_m(x)=\frac{x^k}{rm^2}[t^{mN-1}](e^{tx}(\frac{t}{1-e^{-t}})^{mN})$. Then (\ref{eqn:mu}) gives us
\equa{[q^m]k!\sum_{|\mu|=k}\binom{|r|}{\mu}\log C_{\mu,(0),(0)}(q)&=\frac{|r|(rm)_{(mN+k-1)}}{rm^2}[x^{mN-1}t^{mN-1}]\left(e^{tx}\left(\frac{t}{1-e^{-t}}\right)^{mN}\right)\\
&=\frac{|r|(rm)_{(mN+k-1)}}{rm^2(mN-1)!}\\
&=\frac{|r|}{m}\binom{mr-1}{mN-1}(m(r-N))_{(k-1)},\\
[q^m]\sum_{|\mu|=k}\binom{|r|}{\mu}\log C_{\mu,(0),(0)}(q)&=\frac{|r|}{mk}\binom{mr-1}{mN-1}\binom{m(r-N)}{k-1}.
}
\AC{Shouldn't this be about taking the $x^{mn}$ coefficient without the $-1$  and $[t^{mN-1}]$ separately? In this case, this just looks like $f^m$ and you can use Corollary 2 in my Lagrange inversion paper. }

Observe that the Verlinde series for $\a\in K_\T(S)$ only depends on $c_1(\a)$ by definition. Therefore the universal series are non-trivial only when $\mu=(1)_k:=(1,\dots, 1)$ has $k$ copies of 1. Using (\ref{eqn:V-lim}), we take the Verlinde limit of (\ref{eqn:mu}) and get
\equanum{\label{eqn:V-lim-deg-k}k!|r|^k[q^m]\log B_{(1)_k,(0),(0)}(q)&=(-1)^{rm}|r|(|r|m)^{k-1}\lim_{\e\->0}[q^m]H_{(1),(0),(0)}((-1)^rq\e^r,\e^{-1})\\
&=\frac{|r|(|r|m)^{k-1}}{m}[t^{mN-1}]\left(\frac{e^{rmt}t^{mN}}{(1-e^{-t})^{mN}}\right)\\
&=|r|^km^{k-2}\res_{t=0}\frac{e^{rmt}}{(1-e^{-t})^{mN}}
}
where $\res$ refers to the residue of a meromorphic function. Let $R>0$. We compute the above residue by integrating along the rectangular contour formed by the points $\{\pm R\pm i\pi\}$:
\equa{2\pi i\res_{t=0}\frac{e^{rmt}}{(1-e^{-t})^{mN}}=&\int_{-R}^R\frac{e^{rm(t+i\pi)}}{(1-e^{-(t+i\pi)})^{mN}}dt-\int_{-R}^R\frac{e^{rm(t-i\pi)}}{(1-e^{-(t-i\pi)})^{mN}}dt\\
&+i\int_{-\pi}^\pi\frac{e^{rm(R+it)}}{(1-e^{-(R+it)})^{mN}}dt-i\int_{-\pi}^\pi\frac{e^{rm(-R+it)}}{(1-e^{-(-R+it)})^{mN}}dt\\
}
The first two integrals cancel each other out because $e^{i\pi}=e^{-i\pi}$. Substituting $t=-\theta$ to the third and $t=\theta$ to the fourth integral gives us
\equa{&\indent i\int_{-\pi}^\pi\frac{e^{rm(R+it)}}{(1-e^{-(R+it)})^{mN}}dt-i\int_{-\pi}^\pi\frac{e^{rm(-R+it)}}{(1-e^{-(-R+it)})^{mN}}dt\\
&=i\int_{-\pi}^\pi\frac{e^{-rm(-R+i\theta)}}{(1-e^{-R+i\theta})^{mN}}-\frac{e^{rm(-R+i\theta)}}{(1-e^{-(-R+i\theta)})^{mN}}d\theta\\}
Perform change of variable $z=e^{-R}e^{i\theta}$, we convert this into the following contour integral along a circle of radius $e^{-R}$ centered at 0.
\equa{
&\indent \oint \frac{z^{-rm-1}}{(1-z)^{mN}}- \frac{z^{rm-1}}{(1-z^{-1})^{mN}}dz \\
&=2\pi i \left(\res_{z=0}z^{-rm-1}(1-z)^{-mN}-\res_{z=0}z^{rm-1}(1-z^{-1})^{-mN}\right)\\
&=2\pi i \binom{m(r+N)-1}{mr}
}
Substituting back to (\ref{eqn:V-lim-deg-k}) and dividing both sides by $|r|^k$, we get for $r\neq 0$,
\[[q^m]\log B_{(1)_k,(0),(0)}(q)=\frac{m^{k-2}}{k!}\binom{m(r+N)-1}{mr}.\]

Note that $H_{(0),(0),(1)}=\frac{r}{N}H_{(1),(0),(0)}$. A similar argument using (\ref{eqn:xi}) for the case $\mu=\nu=(0)$ gives us 
\equa{&[q^m]\sum_{|\xi|=k}\binom{N}{\xi}\log C_{(0),(0),\xi}(q)=\frac{r}{mk}\binom{mr-1}{mN-1}\binom{m(r-N)}{k-1},\\&[q^m]\sum_{|\xi|=k}\binom{N}{\xi}\log B_{(0),(0),\xi}(q)=\frac{|r|^{k-1}Nm^{k-2}}{k!}\binom{m(r+N)-1}{mN}.}
Alternatively, when $r>0$, we can combine (\ref{eqn:xi}) and (\ref{eqn:mu}) and obtain for $k_1,k_2\geq 1$,
\equa{[q^m]\sum_{|\mu|=k_1}\sum_{\xi=k_2}\binom{r}{\mu}\binom{N}{\xi}\log C_{\mu,(0),\xi}(q)&=\frac{r(r-N)}{k_1k_2}\binom{mr-1}{mN-1}\binom{m(r-N)-1}{k_1-1,k_2-1},\\
[q^m]\sum_{\xi=k_2}\binom{N}{\xi}\log B_{(1)_{k_1},(0),\xi}(q)&=\frac{r^{k_2}m^{k_1+k_2-2}}{k_1!k_2!}\binom{m(r+N)-1}{mr}.}
We conclude the following Segre-Verlinde relations for terms with $\nu=(0)$.
\begin{theorem}\label{thm:SV2d-deg-pos}
For rank $r\neq 0$ and $n,k>0$, the universal series of Theorem \ref{thm:SV-univ-sieres} satisfy the following identities
\equa{&[q^n]\sum_{|\mu|=k}\binom{|r|}{\mu}\log C_{\mu,(0),(0)}(q)=\frac{|r|}{nk}\binom{nr-1}{nN-1}\binom{n(r-N)}{k-1},\\
&[q^n]\log B_{(1)_k,(0),(0)}(q)=\frac{n^{k-2}}{k!}\binom{n(r+N)-1}{nr},\\
&[q^n]\sum_{|\xi|=k}\binom{N}{\xi}\log C_{(0),(0),\xi}(q)=\frac{r}{nk}\binom{nr-1}{nN-1}\binom{n(r-N)}{k-1},\\&[q^n]\sum_{|\xi|=k}\binom{N}{\xi}\log B_{(0),(0),\xi}(q)=\frac{|r|^{k-1}Nn^{k-2}}{k!}\binom{n(r+N)-1}{nN}\\
}
where $(1)_k=(1,\dots,1)$ is the partition with $k$ copies of $1$.

When $r>0$, we have
\equa{&[q^n]\sum_{|\mu|=k_1}\sum_{|\xi|=k_2}\binom{r}{\mu}\binom{N}{\xi}\log C_{\mu,(0),\xi}(q)=\frac{r(r-N)}{k_1k_2}\binom{nr-1}{nN-1}\binom{n(r-N)-1}{k_1-1,k_2-1},\\&[q^n]\sum_{|\xi|=k_2}\binom{N}{\xi}\log B_{(1)_{k_1},(0),\xi}(q)=\frac{r^{k_2}n^{k_1+k_2-2}}{k_1!k_2!}\binom{n(r+N)-1}{nr}.}

\end{theorem}
By the $k=2$ case of the above theorem, we have the following Segre-Verlinde correspondence.
\begin{corollary}\label{cor:sv2d-deg1}
The universal series of Theorem \ref{thm:SV-univ-sieres} satisfy the following correspondence
\[\left(C^{-r,N}_{(1,1),(0),(0)}(q)\right)^{r^2}\left(C^{-r,N}_{(2),(0),(0)}(q)\right)^{\binom{|r|}{2}}=\left(B^{r,N}_{(1,1),(0),(0)}((-1)^Nq)\right)^{|r|(r+N)}.\]
\[\left(C^{-r,N}_{(0),(0),(1,1)}(q)^{r^2}C^{-r,N}_{(0),(0),(2)}(q)^{\binom{|r|}{2}}\right)^{|r|}=\left(B^{r,N}_{(0),(0),(1,1)}(-q)^{r^2}B^{r,N}_{(0),(0),(2)}(-q)^{\binom{|r|}{2}}\right)^{r+N}.\]
\end{corollary}

\begin{remark}
As mentioned in the introduction, combining Theorem \ref{thm:SV2d-deg-pos} with Theorem \ref{cor:reduced-series} yields the corresponding relations for reduced invariants. In particular, Corollary \ref{cor:sv2d-deg1} implies a correspondence in degree 0 for reduced invariants, which could provide insight into the reduced invariants for K3 surfaces in the compact setting.
\end{remark}




