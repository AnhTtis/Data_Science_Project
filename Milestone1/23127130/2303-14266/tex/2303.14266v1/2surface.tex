\begin{comment}The Hilbert scheme of points $\hilb^n(X)$ of a scheme $X$ parameterizes points on $X$ of length $n$. For a vector bundle $V$ over $X$, the \emph{tautological bundle} $V^{[n]}$ on $\hilb^n(X)$ is 
\[V^{[n]}=p_*(\mathcal{O}_{\mathcal{Z}}\otimes q^*V)\]
where $p:\hilb^n(X)\times X\->\hilb^n(X), q:\hilb^n(X)\times X\->X$ are projections, and $\mathcal{Z}\seq \hilb^n(X)\times X$ is the universal subscheme. This extends to the Grothendieck groups and associates each $\alpha\in K^0(X)$ an $\alpha^{[n]}\in K^0(\hilb^n(X))$. 

\begin{remark}\label{rmk:taut-bundle-local}

\end{remark}
\end{comment}

\subsection{An auxiliary invariant for compact surfaces} 


A general tactic for studying the Segre and Verlinde series is using a more general genus. In the non-virtual surface case this could be \cite[(1.1)]{GM} defined below. In the virtual case, we will use the invariant (\ref{defn:nek2d}). For Calabi-Yau 4-folds, we consider the Nekrasov genus (\ref{def:nek-genus}), introduced by \cite{magcolor}.

For a vector bundle $V$ over $Y$ define
\[\Lambda_{z}(V)=\sum_{i\geq 0}[\Lambda^iV]z^i\in K^0(Y)[z],\indent\Lambda_{z}(-V)=\sum_{i\geq 0}[\sym^iV](-z)^i\in K^0(Y)[\![z]\!]\]
which extends to a homomorphism $\Lambda_z:(K^0(Y),+)\->(K^0(Y)[\![z]\!],\cdot)$. For $\alpha\in K^0(S)$, set
\equanum{\label{eqn:nek2d}I(\a;q,z):=\sum_{n=0}^\infty (-q)^n\chi\left(\hilb^n(S),(\Lambda_{-z}\alpha^{[n]})\otimes \det(\O_S^{[n]})^{-1}\right).}
This invariant is chosen so that the Chern series and Verlinde series can be recovered from it by taking limits. L. G\"ottsche and A. Mellit computed some of its universal series and used them to find universal series for the Segre and Verlinde invariants. We state their theorem for the case of rank $r=2$ for later use.

\begin{theorem}[{G\"ottsche-Mellit \cite[Theorem 1.1]{GM}}]\label{thm:2dNek-series}
For any $r\in\ZZ$, there exist power series $G_0,G_1,G_2,G_3,G_4\in\ZZ[\![q,z]\!]$ such that for all smooth projective surfaces $S$ and $\alpha\in K^0(S)$ of rank $r$, we have
\[I(\a;q,z)=G_0(q,z)^{c_2(\alpha)}G_1(q,z)^{\chi(\det\alpha)}G_2(q,z)^{\frac12\chi(\O_S)}G_3(q,z)^{c_1(\a)K_S-\frac12K_S^2}G_4(q,z)^{K_S^2}.\]
When $r=2$, we have
\equa{&G_0(q,z)=1-qz,\indent
G_1(q,z)=\frac{1-qz^2}{1-qz},\indent
G_2(q,z)=\frac{(1-q)^2}{(1-qz)^2}\\
&G_3(q,z)=1.
}
\end{theorem}

In order to obtain $G_4(q,z)$, we use \cite[Theorem 1.2]{GM} which relates $G_4$ to $B_4$ from (\ref{eqn:univ-Verlinde}) by some change of variables. By \cite[Theorem 5.3]{EGL}, $B_4=1$, so we conclude $G_4(q,z)=1$ in the rank $r=2$ case.


\subsection{Equivariant invariants on Hilbert schemes}
As mentioned in the introduction, the equivariant invariants on $S=\CC^2$ are defined by equivariant localization. The following definition gives a more precise description. Let $\T=(\CC^2)^*$ be the 2-dimensional torus acting on $S=\CC^2$ by scaling:
\equa{(t_1,t_2)\cdot(x_1,x_2)=(t_1x_1,t_2x_2)\in\CC^2.}
This lifts to an action on $\hilb^n(S)$. For a $\T$-equivariant bundle $V$, $V^{[n]}$ is also $\T$-equivariant. The $\T$-fixed locus on $\hilb^n(S)$ is a finite collection of reduced points corresponding to monomial ideals of $\CC[x_1,x_2]$ by Lemma \ref{lem:fixed-points}. Therefore we can define the equivariant Segre and Chern invariants by replacing the usual integration with equivariant integration (\ref{eqn:equivar-int}). Similarly, the Verlinde invariants on $S$ are defined using K-theoretic equivariant localization \cite[Theorem 3.5]{k-local}. Since we are interested in comparing the Segre and Verlinde series, we convert the Verlinde invariants into cohomological invariants using the equivariant Hirzebruch-Riemann-Roch formula \cite[Corollary~3.1]{HRR}.
\begin{definition}\label{def:equi-inv}
When $S=\CC^2$, the \emph{equivariant Chern series} of $\alpha\in K_\T(S)$ is
\equa{I^{\mathcal{C}}(\alpha;q):=&\sum_{n=0}^\infty q^n\int_{\hilb^n(S)}c(\alpha)\\
:=&\sum_{n=0}^\infty q^n\sum_{Z\in\hilb^n(S)^\T}\frac{c(\alpha^{[n]}|_{Z})}{e(T_{Z})}\\
&\in H^*_\T(\pt)_\loc[\![q]\!]=\CC(\lambda_1,\lambda_2)[\![q]\!]
}
where $T_{Z}$ is the Zariski tangent space of $\hilb^n(S)$ at $[Z]$.
The \emph{equivariant Verlinde number} is
\equa{I^{\mathcal{V}}(\alpha;q):=&\sum_{n=0}^\infty q^n\int_{\hilb^n(S)}\ch(\det(\alpha^{[n]}))\td(T)\\
:=&\sum_{n=0}^\infty q^n\sum_{Z\in\hilb^n(S)^\T}\frac{\ch(\det(\alpha^{[n]}|_{Z}))\td(T_{Z})}{e(T_{Z})}\\
=&\sum_{n=0}^\infty q^n\sum_{Z\in\hilb^n(S)^\T}\frac{\ch(\det(\alpha^{[n]}|_{Z}))}{\ch(\Lambda_{-1}T_{Z}^\vee)}\in \left(\prod_{i=0}^\infty H^i_\T(\pt)\right)_\loc[\![q]\!].
}
\end{definition}


\begin{lemma}\label{lem:fixed-points}
The $\T$-fixed locus $\hilb^n(S)^\T$ consists of finitely many reduced points.
\end{lemma}
\begin{proof}
Since $\T$ acts on $\CC^2$ by scaling coordinates, it acts on the coordinate ring $\CC[x_1,x_2]$ by 
\[(t_1,t_2)\cdot x_i=t_i^{-1}x_i.\]
Observe the $\T$-fixed ideals of $\CC[x_1,x_2]$ are exactly the monomial ideals, of which there are only finitely many that have length $n$. For these points to be reduced, it suffices to show the Zariski tangent space in $\hilb^n(S)$ has no $\T$-fixed parts, which follows from the characterization of the tangent space (\ref{eqn:tangent}).
\end{proof}

We shall explain how to calculate the above invariants. First consider the correspondence between monomial ideals of $\CC[x_1,x_2]$ and partitions. For a partition $\mu$, the corresponding point $[Z_{\mu}]\in\hilb^n(S)$ is given by the monomial ideal $I_{Z_{\mu}}$ such that
\[\O_{Z_{\mu}}=\CC[x_1,x_2]/I_{Z_{\mu}}=\Span\{x_1^{c(\square)}x_2^{r(\square)}:\square\in\mu\}.\] 
It follows that the character of $\O_{Z_{\mu}}$ is 
\[\sum_{\square\in\mu} t_1^{-c(\square)}t_2^{-r(\square)}\in K_\T(\pt)=\ZZ[t_1^{\pm},t_2^{\pm}].\]
Let $V=\oplus_{i=1}^r\O_S\<v_i\>$ be a equivariant bundle on $S$ of rank $r$, twisted with weights $v_i$ for $i=1,\dots ,r$. For any point $[Z_\mu]\in \hilb^n(X)^\T$, the fiber of $V^{[n]}$ is
\[V^{[n]}|_{Z_\mu}=p_*(\mathcal{O}_{\mathcal{Z}}\otimes q^*V)|_{Z_\mu}=H^0(X,\O_{Z_\mu}\otimes V)=H^0(V|_{Z_\mu}).\] 
The right-hand side is the following $rn$-dimensional representation in $K_\T(\pt)$:
\equanum{\label{eqn:vb2d}\bigoplus_{i=1}^r\O_{Z_{\mu}}\<v_i\>=\sum_{i=1}^r \sum_{\square\in\mu} v_it_1^{-c(\square)}t_2^{-r(\square)}.}
It was shown in \cite[Lemma 3.2]{Ellingsrud1987} that the Zariski tangent bundle at $[Z_\mu]$ is 
\equanum{\label{eqn:tangent}T_{Z_{\mu}}=\sum_{\square\in\mu}t_1^{a(\square)+1}t_2^{-l(\square)}+t_1^{-a(\square)}t_2^{l(\square)+1}.}
Denote $w_i:=c_1^\T(v_i)$ the equivariant Chern roots of $V$, we may now expand Definition \ref{def:equi-inv} and get
\equa{I^{\mathcal{C}}(V;q)=&\sum_{\mu} q^{|\mu|}\prod_{\square\in\mu}\frac{\prod_{i=1}^r(1+w_i-c(\square)\lam_1-r(\square)\lam_2)}{((a(\square)+1)\lambda_1-l(\square)\lambda_2)((l(\square)+1)\lambda_2-a(\square)\lambda_1)},\\
I^{\mathcal{V}}(V;q)=&\sum_{\mu} q^{|\mu|}\prod_{\square\in\mu}\frac{\prod_{i=1}^rv_it_1^{-c(\square)}t_2^{-r(\square)}}{\left(1-t_1^{-(a(\square)+1)}t_2^{l(\square)}\right)\left(1-t_1^{a(\square)}t_2^{-(l(\square)+1)}\right)}.}
Note the expression for the Verlinde series uses the identification in Remark \ref{rmk:K-H-identification}.

An important tool for studying the Segre and Verlinde series on $S=\CC^2$ used by \cite{GM} is the \emph{master partition function}. For each $r\in\ZZ$, it is defined by
\equa{\Omega(Q;z_1,\dots ,z_r;q_1,q_2):=\sum_{\mu} Q^{|\mu|}\prod_{\square\in\mu}\frac{\prod_{i=1}^r(1-q_1^{c(\square)}q_2^{r(\square)}z_i)}{(q_1^{a(\square)+1}-q_2^{l(\square)})(q_1^{a(\square)}-q_2^{l(\square)+1})}.}
On $S=\CC^2$ with the bundle $V=\oplus_{i=1}^r\O_S\<w_i\>$, the invariant (\ref{eqn:nek2d}) can be defined equivariantly by \[I(V;q,z)=\Omega(q;ze^{w_1},\dots,ze^{w_r};e^{-\lam_1},e^{-\lam_2}).\]
Furthermore, using the explicit expressions of $I^{\mathcal{C}}$ and $I^{\mathcal{V}}$ above, one can show that they are specializations of $\Omega$ as follows.
\begin{proposition}\label{prop:chern-ver-limit}\emph{(\cite[Proposition 3.5]{GM})}
The Chern and Verlinde series satisfy the following limits:
\equa{I^{\mathcal{C}}(V;q)&=\lim_{\e\->0}\Omega\left(-q\e^{2-r}(1+\e)^r;\frac{e^{-\e w_1}}{1+\e},\dots ,\frac{e^{-\e w_r}}{1+\e};e^{\e \lambda_1},e^{\e \lambda_2}\right),\\
I^{\mathcal{V}}(V;q)&=\lim_{\e\->0}\Omega\left((-1)^rq\e^{r+1};\e^{-1}e^{w_1},\dots ,\e^{-1}e^{w_r},\e^{-1};e^{-\lambda_1},e^{-\lambda_2}\right).}
\end{proposition}



\subsection{Relation to projective toric surfaces}\label{sec:proj-reduction}
We consider what the equivariant invariants will be for a projective toric surface $S'$ with a natural action by the torus $\T=(\CC^*)^2$, and compare them with the case $S=\CC^2$. More details on this reduction can be found in \cite[§3.2]{GM}; see also \cite[§6.2]{Arbesfeld} and \cite[§3.2]{lyz}. 

Let the fixed points on $S'$ be $p_1,\dots ,p_M$. Denote the Chern roots of the tangent space at $p_i$ by $a_1^{(i)},a_2^{(i)}$. Let $V'$ be a $\T$-equivariant bundle of rank $r$ on $S'$ with Chern roots $w_1^{(i)},\dots ,w_r^{(i)}$ at $p_i$. By equivariant localization, (\ref{eqn:nek2d}) for $S'$ and $V'$ can be expressed as
\equanum{\label{eqn:proj-reduction}
I(V';q,z)&=\left(\prod_{i=1}^M\Omega(q;ze^{w_1^{(i)}},\dots ,ze^{w_r^{(i)}};e^{-a_1^{(i)}},e^{-a_2^{(i)}})\right)\bigg\vert_{\lambda_1=\lambda_2=0}.
}
As remarked in \cite{Bott-residue}, since $S'$ is compact, the product on the right-hand side lives in $H^*_\T(\pt)=\CC[\lam_1,\lam_2]$. Therefore it is indeed valid to set $\lambda_1=\lambda_2= 0$, and the equality follows from Bott residue formula. This helps us in finding universal series for the $S=\CC^2$ case because the universal series on the left-hand side is already known by Theorem \ref{thm:2dNek-series}. 

Using Macdonald polynomials and results from \cite{mel-HLV}, G\"ottsche and Mellit  \cite[Proposition 2.5]{GM} showed that $\Omega$ is admissible with respect to $q_1,q_2$ in the sense of Definition \ref{def:admissible}. Applying expansion (\ref{eqn:log-partition-E}), we have 
\equanum{\label{eqn:log-total}&\log\Omega(q;ze^{w_1},\dots ,ze^{w_r};e^{-\lambda_1},e^{-\lambda_2})\\
=&\sum_{\substack{ \mu\text{ partition}\\k_1,k_2\geq -1}}H_{\mu,k_1,k_2}(q,z)\cdot{\int_SE_{k_1,k_2}(c_1^\T(S),c_2^\T(S))c^\T_{\mu}(V)}
}
for some series $H_{\mu,k_1,k_2}$. Note that the integrand on the right is a homogeneous rational function in the variables $\lambda_1,\lambda_2$ of total degree $|\mu|+k_1+k_2$. Exponentiating both sides, we get
\[\Omega(q;ze^{w_1},\dots ,ze^{w_r};e^{-\lambda_1},e^{-\lambda_2})=\prod_{\substack{\mu\text{ partition}\\k_1,k_2\geq -1}}H_{\mu,k_1,k_2}(q,z)^{\int_SE_{k_1,k_2}(c_1^\T(S),c_2^\T(S))c^\T_{\mu}(V)}\]
Substituting this into (\ref{eqn:proj-reduction}) yields
\equa{I(V';q,z)=&\left(\prod_{\mu,k_1,k_2\geq -1}H_{\mu,k_1,k_2}(q,z)^{\int_{S'}E_{k_1,k_2}(c_1^\T(S'),c_2^\T(S'))c^\T_{\mu}(V')}\right)\bigg\vert_{\lambda_1=\lambda_2=0}\\
=&\prod_{|\mu|+k_1+k_2=0}H_{\mu,k_1,k_2}(q,z)^{\int_{S'}E_{k_1,k_2}(c_1(S'),c_2(S'))c_{\mu}(V')}.
} 
where the integral in the first line is equivariant integration, while the integral in the second line is the usual non-equivariant integration.
Comparing this expansion with the one in Theorem \ref{thm:2dNek-series}, together with a quick computation that $E_{-1,-1}(x_1,x_2)=1,E_{-1,0}(x_1,x_2)=\frac12x_1, E_{0,0}(x_1,x_2)=x_2, E_{-1,1}(x_1,x_2)=\frac12(x_1^2-2x_2)$, we obtain \cite[(3.16)]{GM}
\equa{&\log G_0(q,z)=H_{(2),-1,-1}(q,z), \indent \log G_1(q,z)=2H_{(1,1),-1,-1}(q,z),\\
&\log G_2(q,z)=24(H_{(0),0,0}(q,z)-2H_{(0),-1,1}(q,z))-4H_{(1,1),-1,-1}(q,z),\\
&\log G_3(q,z)=H_{(1),-1,0}(q,z)+H_{(1,1),-1,-1}(q,z),\\
&\log G_4(q,z)=-H_{(0),0,0}(q,z)+3H_{(0),-1,1}(q,z)-\frac12(H_{(1),-1,0}(q,z)+H_{(1,1),-1,-1}(q,z))
}
The series $G_i(q,z)$ are as in Theorem \ref{thm:2dNek-series}, which give, in rank $r=2$,
\equanum{\label{eqn:deg0}&H_{(2),-1,-1}(q,z)=\log(1-qz),\\ &H_{(1,1),-1,-1}(q,z)=-H_{(1),-1,0}(q,z)=\frac12\log\frac{1-qz^2}{1-qz},\\
&H_{(0),0,0}(q,z)=3H_{(0),-1,1}(q,z)=\frac14\log\frac{(1-q)(1-qz^2)}{(1-qz)^2}.
}


To summarize, we have observed that the universal series for the projective case are exactly the ones for the $S=\CC^2$ case whose powers have degree 0 in $H^*_\T(\pt)$.

\subsection{Virtual invariants}\label{sec:vir-inv2d}
Before defining virtual invariants, we recall the notion of a perfect obstruction theory in the sense of \cite[Definition~4.4]{BF}. For our purposes, we use the following simplified version.

\begin{definition}\label{defn:obs} Let $X$ be a scheme over $\CC$. An \emph{obstruction theory} is a complex of vector bundles
\[E^\bullet=[\dots \->E^{-2}\->E^{-1}\->E^0]\]
for some $a\in \ZZ$, together with a morphism in the derived category $D(\text{QCoh}(X))$ to the cotangent complex
\[\f:E^\bullet\->L_X^\bullet\]
such that $h^0(\f)$ is an isomorphism and $h^{-1}(\f)$ is surjective. It is a \emph{(2-term) perfect obstruction theory} if $E^i=0$ for $i\neq 0,-1$. The \emph{virtual tangent space} $T^{\vir}=E_\bullet=(E^\bullet)^*$ is the class of the dual complex of a given obstruction theory.
\end{definition}


Let $S$ be a surface and $E$ a torsion-free sheaf. It is well known that $\quot_S(E,n)$ admits an obstruction theory given by the dual complex of $\mathbf{R}\shom_{p}(\mathcal{I},\mathcal{F})$, where $\mathcal{I},\mathcal{F}$ are respectively the universal subsheaf and quotient sheaf. When $S$ is a projective surface, \cite[Lemma~1]{MOP1} shows that this obstruction theory is perfect of virtual dimension $nN$. Using this, we can define a virtual fundamental class $[\quot_S(E,n)]^\vir$ via the methods from \cite{BF,LiTian} as well as a virtual structure sheaf $\O^{\vir}$ using \cite{k-local-vir}. Applying the same argument for $S=\CC^2$ gives us a $\T$-equivariant perfect obstruction theory. We note that since $\mathcal{F}$ is compactly supported, the $\ext$-groups are finite dimensional vector spaces, so the steps involving Serre duality still work.




Let $S=\CC^2$ and $E=\oplus_{i=1}^N\O_S\<y_i\>$. Recall from the introduction the following tori:
\[\T_0=(\CC^*)^2, \indent \T_1=(\CC^*)^N,\indent\T_2=(\CC^*)^{r+s}.\] 
Set $\T=\T_0\times \T_1\times\T_2$, with
\equa{K_\T(\pt)&=\ZZ[t_1^{\pm1},t_2^{\pm1};y_1^{\pm1},\dots ,y_N^{\pm1};v_1^{\pm1},\dots ,v_{r+s}^{\pm1}],\\
H^*_\T(\pt)&=\CC[\lambda_1,\lambda_2;m_1,\dots ,m_N;w_1,\dots ,w_{r+s}].}
Under these actions, the $\T_1$-fixed locus of $\quot_S(E,n)$ decomposes into the form
\[0\->\oplus_{i=1}^N I_i\<y_i\>\->\oplus_{i=1}^N \O_S\<y_i\>\->\oplus_{i=1}^N F_i\<y_i\>\->0.\]
Thus the $\T_1$-fixed locus can be identified as
\[\bigsqcup_{n_1+\dots +n_N=n}\hilb^{n_1}(S)\times\dots \times \hilb^{n_N}(S).\]
Consequently, the $\T$-fixed locus $\quot_S(E,n)^\T$ consists of finitely many reduced points of form
\[Z_{\mu}=\left([Z_{1}],[Z_{2}],\dots ,[Z_{N}]\right)\in \hilb^{n_1}(S)\times\dots \times \hilb^{n_N}(S),\]
labeled by $N$-colored partitions $\mu=\left(\mu^{(1)},\dots,\mu^{(N)}\right)$.

The following equivariant invariants are defined similarly to Definition \ref{def:equi-inv}, now motivated by \emph{virtual} equivariant localization \cite{vir-loc}, \emph{virtual} K-theoretic equivariant localization \cite[Theorem 5.3.1]{k-local-vir}, and the \emph{virtual} Hirzebruch-Riemann-Roch formula \cite[Corollary 1.2]{HRR-vir}. 

\begin{definition}
Let $S=\CC^2$ and
\[\alpha=[\oplus_{i=1}^r\O_Y\<v_i\>]-[\oplus_{i=r+1}^{r+s}\O_Y\<v_i\>]\in K_{\T}(S)\]
the \emph{equivariant virtual Segre, Chern, Verlinde series} on Quot schemes are respectively
\equa{&{\mathcal{S}}_S(E,\a;q):=\sum_{n=0}^\infty q^n\sum_{Z\in\quot_S(E,n)^\T}\frac{s(\alpha^{[n]}|_{Z})}{e(T_{Z}^\vir)},\\
&{\mathcal{C}}_S(E,\a;q):=\sum_{n=0}^\infty q^n\sum_{Z\in\quot_S(E,n)^\T}\frac{c(\alpha^{[n]}|_{Z})}{e(T_{Z}^\vir)},\\
&{\mathcal{V}}_S(E,\a;q):=\sum_{n=0}^\infty q^n\sum_{Z\in\quot_S(E,n)^\T}\frac{\ch(\det(\alpha^{[n]}|_{Z}))}{\ch(\Lambda_{-1}(T_{Z}^\vir)^\vee)}.
}
\end{definition}

We shall describe how to calculate these invariants, and refer to \cite[§5.1]{Quot-DT} and \cite[§3.3]{lim} for the following argument. On each $\T_1$-fixed locus 
\[D=\hilb^{n_1}(S)\times\dots \times \hilb^{n_N}(S),\]
the universal subsheaf and universal quotient sheaf of $D$ are $\bigoplus_{i=1}^N I_{\mathcal{Z}_i}\<y_i\>\text{ and } \bigoplus_{j=1}^N \O_{\mathcal{Z}_j}\<y_i\>$
where $\mathcal{Z}_i$ is the universal subscheme of $\hilb^{n_i}(S)$.
The virtual tangent bundle over $D$ 
is then
\[T^\vir_D=\bigoplus_{i,j=1}^N \mathbf{R}\shom_{p}( I_{\mathcal{Z}_i},\O_{\mathcal{Z}_j})\<y_i^{-1}y_j\>\]
where $p:D\times X\->D$ is the projection. Further restricting to each $Z_\mu=([Z_1],[Z_2],\dots ,[Z_N])\in\quot_S(E,n)^\T$ gives the virtual tangent bundle at $Z_{\mu}$ as follows
\equanum{\label{eqn:tangent2}T^\vir_{Z_{\mu}}=\bigoplus_{i,j=1}^N \ext( I_{Z_i},\O_{Z_j})\<y_i^{-1}y_j\>\in K_{\T}(S).}
To give an explicit formula for $T^\vir$, we consider a $\T_0$-equivariant free resolution of $I_{Z_i}$. We refer to \cite[Page 439]{Eisenbud} for the following \emph{Taylor resolution}. Say $I_{Z_i}$ is generated by monomials $m_1,\dots,m_s$. For each $k=0,\dots,s$, let $F_k$ be the free $\CC[x_1,\ldots, x_n]$-module module with basis $\{e_I\}$, indexed by subsets $I\seq \{1,\dots,s\}$ of size $k$. Set
\[m_I=\text{least common multiple of }\{m_i:i\in I\}.\]
For $k=1,\dots, s$, define differential $d_k:F_k\->F_{k-1}$ by 
\[d_k(e_I)=\sum_{j=1}^k(-1)^j\frac{m_I}{m_{I- \{i_j\}}}e_{I- \{i_j\}}\]
for each subset $I=\{i_1,\dots, i_k\}$ such that $i_1<\dots<i_k$. Giving each $e_I$ the weight of $m_I$, we obtain the $T_0$-equivariant free resolution
\[0\->F_s\->\dots \->F_0\->I_{Z_i}\->0\]
where
\[F_k=\bigoplus_{I\seq \{1,\dots, s\},|I|=k}\O_S\<m_I(t)\>\]
for some $d_{kI}\in\ZZ^2$. Define
\equanum{\label{def:poincare}P(I_{Z_i})=\sum_{k,|I|=k}(-1)^km_I(t).}
Note that the character of $\O_S=\CC[x_1,x_2]$ is $\sum_{i,j\geq 0}t_1^{-i}t_2^{-j}=1/(1-t_1^{-1})(1-t_2^{-1})$, so the character of $\O_{Z_i}=\O_S/I_{Z_i}$ is
\[Q_{i}:=\frac{1-P(I_{Z_i})}{(1-t_1^{-1})(1-t_2^{-1})}.\]
Therefore the character of $T^\vir_{Z_{\mu}}$ in $K_{\T}(\pt)$ can be expressed as
\equanum{\label{eqn:Tvir-quot2d}\bigoplus_{i,j=1}^N \ext( I_{Z_i},\O_{Z_j})\<y_i^{-1}y_j\>=&\sum_{i,j=1}^N\sum_{k,|I|=k}(-1)^k \hom(\O_S\<m_I(t)\>,\O_{Z_j})y_i^{-1}y_j\\
=&\sum_{i,j=1}^N\sum_{k,I}(-1)^k\O_{Z_j}\<m_I(t)^{-1}\>y_i^{-1}y_j\\
=&\sum_{i,j=1}^N\overline{P(I_{Z_i})}Q_{j}y_i^{-1}y_j\\
=&\sum_{i,j=1}^N( Q_{j}-(1-t_1)(1-t_2)\overline{Q_{i}} Q_{j})\cdot y_i^{-1}y_j
}
where $\overline{(\cdot)}$ denotes the involution $t_i\mapsto t_i^{-1}$. For the $\T$-equivariant bundle $V=\oplus_{i=1}^r\O_S\<v_i\>$, the fiber of $V^{[n]}$ over ${Z_{\mu}}=(Z_1,\dots Z_N)$ is the $rn$-dimensional representation
\[\bigoplus_{i=1}^r\bigoplus_{j=1}^N\O_{Z_{j}}\<v_iy_j\>=\sum_{i=1}^r \sum_{j=1}^N\sum_{\square\in\mu^{(j)}} v_iy_jt_1^{-c(\square)}t_2^{-r(\square)}.\]

Substituting the above calculations into the definition, we obtain the following expressions for the Chern and Verlinde series of vector bundles
\equanum{\label{eqn:CV2d-local}{\mathcal{C}}_S(E,V;q):=&\sum_{\mu} q^{|\mu|}
\frac{\prod_{j=1}^N\prod_{\square\in\mu^{(j)}}\prod_{i=1}^r\left(1+w_i+m_j-c(\square)\lam_1-r(\square)\lam_2\right)}{e(T^\vir|_{Z_\mu})},\\
{\mathcal{V}}_S(E,V;q):=&\sum_{\mu} q^{|\mu|}
\frac{\prod_{j=1}^N\prod_{\square\in\mu^{(j)}}\prod_{i=1}^rv_iy_jt_1^{-c(\square)}t_2^{-r(\square)}}{\ch(\Lambda_{-1}(T^\vir|_{Z_\mu})^\vee)}.
}


 

\subsection{Universal series expansion}
For projective surfaces, define an auxiliary virtual invariant (c.f. (\ref{eqn:nek2d})) as follows
\equanum{\label{defn:nek2d}{\mathcal{N}}_S(E,\a;q,z):=\sum_{n=0}^\infty q^n\chi^{\vir}\left(\quot_S(E,n),\Lambda_{-z}\a^{[n]}\right).}
where $z$ is considered as the weight of an extra $\CC^*$-action that is trivial on $S$ and $\quot_S(E,n)$. We shall refer to this as the \emph{Nekrasov genus} for Quot schemes of surfaces (c.f. (\ref{def:nek-genus})).

Similar to before, we generalize this to the equivariant setting using virtual equivariant localization. On $S=\CC^2$ for vector bundles, this is given by:
\equanum{\label{eqn:nek2d-local}{\mathcal{N}}_S(E,V;q,z):=&\sum_{\mu} q^{|\mu|}
\frac{\prod_{j=1}^N\prod_{\square\in\mu^{(j)}}\prod_{i=1}^r(1-t_1^{-c(\square)}t_2^{-r(\square)}v_iy_jz)}{\ch(\Lambda_{-1}(T^\vir|_{Z_\mu})^\vee)}\\
&\in\QQ(t_1,t_2;y_1,\dots ,y_N)[\![q,z]\!] .
}

%Sometimes we will write ${\mathcal{N}}_S(E,V;q,z)(\lam_1,\lam_2)$ to indicate the toric parameters we are working with for clarity. 
The choice of this invariant is based on G\"ottsche-Mellit's invariant (\ref{eqn:nek2d}). The following Chern and Verlinde limits are satisfied, analogous to \cite[Proposition 3.5]{GM}.

\begin{lemma}\label{lem:CV-limit}
For $S=\CC^2$, the Chern series and the Verlinde series can be retrieved from ${\mathcal{N}}_S$ by taking limits. We have
\equa{{\mathcal{C}}_S(E,V;q)&=\lim_{\e\->0}
    {\mathcal{N}}_S\left(E,V;(-1)^Nq\e^{N-r}(1+\e)^{r},(1+\e)^{-1}\right)\big|_{\vec\lam\leadsto-\e\vec\lambda,\vec w\leadsto -\e \vec w,\vec m\leadsto-\e\vec m},\\
    {\mathcal{V}}_S(E,V;q)&=\lim_{\e\->0}{\mathcal{N}}_S\left(E,V;(-1)^{r}q\e^{r},\e^{-1}\right).}
\end{lemma}
\begin{proof}
For the Chern limit, first consider the substitutions $\lam_i\leadsto -\e\lam_i,w_i\leadsto -\e w_i,m_i\leadsto -\e m_i$. This turns the term $\prod_{j=1}^N\prod_{\square\in\mu^{(j)}}\prod_{i=1}^r(1-t_1^{-c(\square)}t_2^{-r(\square)}v_iy_j(1+\e)^{-1})$ into 
\equa{&\prod_{j=1}^N\prod_{\square\in\mu^{(j)}}\prod_{i=1}^r1-\frac{e^{-\e(w_i+m_j-c(\square)\lam_1-r(\square)\lam_2)}}{1+\e}\\
=&\frac{1}{(1+\e)^{r|\mu|}}\prod_{j=1}^N\prod_{\square\in\mu^{(j)}}\prod_{i=1}^r(1+\e-e^{-\e(w_i+m_j-c(\square)\lam_1-r(\square)\lam_2)})\\
=&\left(\frac{\e}{1+\e}\right)^{r|\mu|}\prod_{j=1}^N\prod_{\square\in\mu^{(j)}}\prod_{i=1}^r(1-c(\square)\lam_1-r(\square)\lam_2+w_i+m_j+O(\e))
}
For the denominator in the sum (\ref{eqn:nek2d-local}), we note that for a Chern root $x$, substituting it by $-\e x$ turns $1-e^{-x}=x-\frac{x^2}2+...$ into $1-e^{\e x}=-\e(x+O(\e))$. Therefore after the substitution, the denominator $\ch(\Lambda_{-1}(T^\vir|_{Z_\mu})^\vee)$ becomes
\[(-1)^{N|\mu|}\e^{N|\mu|}(e(T^\vir_{Z_\mu})+O(\e))\]
Substituting back into (\ref{eqn:nek2d-local}), the Chern limit becomes the limit of
\equa{&\sum_{\mu} (-1)^{N|\mu|}q^{|\mu|}\e^{(N-r)|\mu|}(1+\e)^{r|\mu|}\cdot \frac{\e^{r|\mu|}}{(-1)^{N|\mu|}\e^{N|\mu|}(1+\e)^{r|\mu|}}\cdot\\
&\frac{\prod_{j=1}^N\prod_{\square\in\mu^{(j)}}\prod_{i=1}^r(1-c(\square)\lam_1-r(\square)\lam_2+w_i+m_j+O(\e))}{(e(T^\vir_{Z_\mu})+O(\e))}\\
=&\sum_{\mu} q^{|\mu|}\frac{\prod_{j=1}^N\prod_{\square\in\mu^{(j)}}\prod_{i=1}^r(1-c(\square)\lam_1-r(\square)\lam_2+w_i+m_j+O(\e))}{(e(T^\vir_{Z_\mu})+O(\e))}
}
which converges to ${\mathcal{C}}_S(E,V;q)$ by (\ref{eqn:CV2d-local}). For the Verlinde series, we have
\equa{&\lim_{\e\->0}{\mathcal{N}}_S(E,V;(-1)^{r}q\e^{r},\e^{-1})\\
=&\lim_{\e\->0} \sum_{\mu} (-1)^{r|\mu|}q^{|\mu|}\e^{r|\mu|}
\cdot \frac{\prod_{j=1}^N\prod_{\square\in\mu^{(j)}}t_1^{c(\square)}t_2^{r(\square)}\prod_{i=1}^r(1-t_1^{-c(\square)}t_2^{-r(\square)}v_iy_j\e^{-1})}{\ch(\Lambda_{-1}(T^\vir|_{Z_\mu})^\vee)}\\
=&\lim_{\e\->0}\sum_{\mu} q^{|\mu|}
\frac{\prod_{j=1}^N\prod_{\square\in\mu^{(j)}}\prod_{i=1}^r(t_1^{-c(\square)}t_2^{-r(\square)}v_iy_j-\e)}{\ch(\Lambda_{-1}(T^\vir|_{Z_\mu})^\vee)}\\
=&{\mathcal{V}}_S(E,V;q).
}

\end{proof}


Before starting the proof for universal series expressions, let us discuss how the expansion of ${\mathcal{N}}_S(E,V;q,z)$ as a formal Laurent series in the variables $\vec{\lam},\vec{m},\vec{w},q,z$ would look like. In \cite[Proposition 3.2]{Arbesfeld}, N. Arbesfeld shows that invariants such as $[q^n]{\mathcal{N}}_S(E,V;q,z)$ can be written as a quotient whose numerator is a Laurent polynomial in $\vec{t},\vec{y},\vec{v},z$, and whose denominator is of the form $\prod_{\mathsf{w}}(1-\mathsf{w})$ for some \emph{non-compact weights} $\mathsf{w}$ in the sense of the following definition.

\begin{definition}\cite[Definition 3.1]{Arbesfeld} Let $M$ be a quasi-projective scheme with an action by some torus $\T$. For a weight $\mathsf{w}\in \T^\vee$, denote $\T_{\mathsf{w}}$ the maximal subtorus of $\T$ containing $\ker\mathsf{w}$. If the fixed locus $M^{\T_\mathsf{w}}$ is proper, then $\mathsf{w}$ is a \emph{compact weight}, otherwise, it is a \emph{non-compact weight}.

\end{definition}

Fasola-Monavari-Ricolfi used this to prove that the K-theoretic Donaldson-Thomas partition functions on $\CC^3$ are Laurent polynomials with respect to the variables $y_1,\dots,y_N$ \cite[Theorem 6.5]{Quot-DT}. We give an outline of their argument, applied to the invariant ${\mathcal{N}}_S$ for $S=\CC^2$. First note that by (\ref{eqn:Tvir-quot2d}), for any $N$-colored partition $\mu$, we have
\[\frac1{\ch(\Lambda_{-1}(T^\vir|_{Z_\mu})^\vee)}=A(\vec{t}\,)\prod_{1\leq i,j\leq N,i\neq j }\frac{\prod_{a\in A_{ij}}(1-y_i^{-1}y_jt^{a})}{\prod_{b\in B_{ij}}(1-y_i^{-1}y_jt^{b})}\]
for some series $A(\vec{t}\,)\in\QQ[\![t_1,t_2]\!]_{\loc}$ and some sets of weights $A_{ij},B_{ij}$. We shall show that the denominator of ${\mathcal{N}}_S$ does not have factors of the form $(1-y_i^{-1}y_jt^b)$ for any $i\neq j$ and $b\in\ZZ^2$. By \cite[Proposition 3.2]{Arbesfeld}, we need to prove $\mathsf{w}=y_i^{-1}y_jt^b$ is a compact weight. Since
\[\ker\mathsf{w}=\{(\vec{t},\vec{y},\vec{v}):y_i=y_jt^b\}\]
is itself a torus, we have $\T_{\mathsf{w}}=\ker\mathsf{w}$. By definition, it suffices to show $\quot_S(E,n)^{\T_{\mathsf{w}}}$ is proper. With the automorphism $\T\->\T$ defined by 
\[(\vec{t},y_1,\dots,y_j,\dots,y_N,\vec{v})\mapsto(\vec{t},y_1,\dots,y_jt^b,\dots,y_N,\vec{v}),\]
we identify the subgroup $\T_{\mathsf{w}}$ to $\T_0\times\{(w_1,\dots,w_N):w_i=w_j\}\times \T_2$, which contains the subgroup $\T_0=\T_0\times\{(1,\dots,1)\}$. This gives us an inclusion
\[\quot_S(E,n)^{\T_{\mathsf{w}}}\xhookrightarrow{}\quot_S(E,n)^{\T_0}.\]
The quotients in the fixed locus $\quot_S(E,n)^{\T_0}$ are all supported at the origin $0\in\CC^2$, so the fixed locus lies inside the punctual Quot scheme $\quot_S(E,n)_0$. The punctual Quot scheme is proper since it is a fiber of the Quot-to-Chow map $\quot_S(E,n)\->\sym^nS$, which is a proper morphism \cite[Remark 3.4]{Quot-DT}. In conclusion, $[q^n]{\mathcal{N}}_S(E,V;q,z)$ is a Laurent polynomial with respect to the variables $y_1,\dots,y_N$, so it can be expanded into a power series with respect to the cohomological parameters $m_1,\dots,m_N$.

Furthermore, if $\mathsf{w}$ is a weight that contains both $t_1$ and $t_2$, then we have $\T_{\mathsf{w}}\cong\{(t_1,t_2):t_1t_2=1\}\times \T_1\times \T_2$. The fixed locus of this subgroup remains the same as that of $\T$, as explained in the next section for reduced invariants. Therefore $\mathsf{w}$ is a compact weight, and the denominator of ${\mathcal{N}}_S$ will not contain factors of the form $(1-t_1^at_2^b)$ for any $a\neq 0$, $b\neq 0$. This means in cohomology, $[q^n]{\mathcal{N}}_S(E,V;q,z)$ can be expanded into a Laurent series in $\lam_1,\lam_2$ whose coefficients are power series in $\vec{m},\vec{w},z$, where the degrees on $\lam_1,\lam_2$ are bounded below individually. We shall see the importance of this lower bound in the proof of the following theorem.

\begin{theorem}\label{thm:SV-univ-sieres}
Let $S=\CC^2$. For any $r\in\ZZ$, $N>0$, there exist universal power series $A_{\mu,\nu,\xi}(q),B_{\mu,\nu,\xi}(q)$, dependent on $N$ and $r$, such that for $E=\oplus_{i=1}^{N}\O_S\<y_i\>$ and $\a\in K_\T(S)$ of rank $r$, the equivariant virtual Segre and Verlinde series on $\quot_S(E,n)$ can be written as the following infinite products
\equa{{\mathcal{S}}_S(E,\alpha;q)=&\prod_{\mu,\nu,\xi\text{ partitions}}A_{\mu,\nu,\xi}(q)^{\int_{S}c_\mu(\a) c_\nu(S)  c_\xi(E)c_1(S)},\\
{\mathcal{V}}_S(E,\alpha;q)=&\prod_{\mu,\nu,\xi\text{ partitions}}B_{\mu,\nu,\xi}(q)^{\int_{S}c_\mu(\a) c_\nu(S)  c_\xi(E)c_1(S)},\\
{\mathcal{C}}_S(E,\alpha;q)=&\prod_{\mu,\nu,\xi\text{ partitions}}C_{\mu,\nu,\xi}(q)^{\int_{S}c_\mu(\a) c_\nu(S)  c_\xi(E)c_1(S)}.
}
\end{theorem}
\begin{proof}
We begin with the case where $\a$ is a vector bundle $V$. Assume $V=\oplus_{i=1}^r\O_S\<v_i\>$, and at end of the proof, we can generalize this to arbitrary $\T$-equivariant bundles by substituting $\T$-weights into the variables $v_1,\dots ,v_r$.

Begin by expanding $\log {\mathcal{N}}_S(E,V;q,z)$ as a Laurent series in $\lam_1,\lam_2$ as follows:
\[\log {\mathcal{N}}_S(E,V;q,z)=\sum_{ (j,k)\in\ZZ^2}H_{j,k}(q,z;\vec{m};\vec{w})\lambda_1^{j}\lambda_2^{k}\] 
for some series $H_{j,k}\in\QQ[\![q,z;m_1,\dots ,m_N;w_1,\dots ,w_r]\!]$. By the symmetry in $w_1,\dots ,w_r$ and the symmetry in $m_1,\dots, m_N$, this expands to 
\equa{\log {\mathcal{N}}_S(E,V;q,z)=&\sum_{\substack{\mu,\xi \text{ partitions}\\j,k\geq -1}}G_{\mu,\xi,j,k}(q,z)\cdot\lam_1^j\lam_2^{k}c_\mu(V)c_\xi(E)\\
&+\sum_{\substack{\mu,\xi \text{ partitions}\\\min\{j,k\}\leq -2}}G_{\mu,\xi,j,k}(q,z)\cdot  \lam_1^j\lam_2^{k}c_\mu(V)c_\xi(E)
}
for some series $G_{\mu,\xi,j,k}\in\QQ[\![q,z]\!]$.


Our goal is to get a universal series expression by exponentiating the above equality. To do so, we first show the terms in the second summation vanish using a similar approach as in Section \ref{sec:proj-reduction}. This proves that ${\mathcal{N}}_S(E,V;q,z)$ is admissible, from which we deduce the desired expressions by taking the limits of Lemma \ref{lem:CV-limit}.


Let $S'$ be a toric projective surface with a natural action by $\T_0=(\CC^*)^2$. Say the fixed points are $p_1,\dots ,p_M$ and the Chern roots of the tangent space of $S'$ at $p_i$ are $a_1^{(i)},a_2^{(i)}$, which live in $H_{\T_0}^*(\pt)=\CC[\lam_1,\lam_2]$. Let $E',V'$ be two arbitrary $\T$-equivariant bundles on $S'$  with Chern roots $b_1^{(i)},\dots ,b_N^{(i)}$ and $c_1^{(i)},\dots ,c_r^{(i)}$ respectively at $p_i$. By a virtual version of the argument in Section \ref{sec:proj-reduction}, this time via the virtual Bott residue formula, we have
\[{\mathcal{N}}_{S'}(E',V';q,z)=\left(\prod_{i=1}^M{\mathcal{N}}_S(E,V;q,z)\Big|_{\vec{\lam}\leadsto\vec{a}^{(i)},\vec m\leadsto\vec{b}^{(i)}, \vec w\leadsto\vec c^{(i)}}\right)\bigg\vert_{\vec{\lam}=\vec{m}=\vec{w}=0}\]
where the symbol $\leadsto$ denotes a substitution of variables. After the substitution inside the bracket of the right-hand side, we know from equivariant integration that the resulting expression is a power series in $\vec\lam,\vec m,\vec w$, which is why we are able to set these variables to 0 and obtain a number. Now we focus on the bundles
\[E'=\bigoplus_{j=1}^N\O_{S'}\<y_j\>,\indent V'=\bigoplus_{j=1}^r\O_{S'}\<v_j\>\]
whose Chern roots at each $p_i$ are $m_1,\dots ,m_N$ and $w_1,\dots ,w_r$ respectively, independent of $i$. Thus
\[{\mathcal{N}}_{S'}(E',V';q,z)=\left(\prod_{i=1}^M{\mathcal{N}}_S(E,V;q,z)|_{\vec\lam\leadsto\vec  a^{(i)}}\right)\bigg\vert_{\vec{\lam}=\vec{w}=\vec{m}=0}.\]
Again, note that the term inside the bracket is a power series with respect to $\lambda_1,\lambda_2$. Substituting the previous expansion of $\log {\mathcal{N}}_S(E,V;q,z)$, we see that
\equa{\log {\mathcal{N}}_{S'}(E',V';q,z)=&\left(\sum_{i=1}^{M}\sum_{\substack{\mu,\xi \text{ partitions}\\\min\{i,j\}\geq -1}}G_{\mu,\xi,i,j}(q,z)\cdot  \lam_1^j\lam_2^kc_\mu(V)c_\xi(E)\bigg\vert_{\vec\lam\leadsto\vec  a^{(i)}}\right)\Bigg\vert_{\vec{\lam}=\vec{w}=\vec{m}=0}\\
&+\left(\sum_{i=1}^{M}\sum_{\substack{\mu,\xi \text{ partitions}\\\min\{i,j\}\leq -2}}G_{\mu,\xi,i,j}(q,z)\cdot  \lam_1^j\lam_2^kc_\mu(V)c_\xi(E)\bigg\vert_{\vec\lam\leadsto \vec a^{(i)}}\right)\Bigg\vert_{\vec{\lam}=\vec{w}=\vec{m}=0}
.
}
Since the elementary symmetric polynomials form a basis for symmetric polynomials, we know the coefficients in front of $q,z$ of the term
\equanum{\label{poly-1}\sum_{i=1}^{M}\sum_{j,k\in\ZZ}G_{\mu,\xi,j,k}(q,z)\cdot  \lam_1^j\lam_2^k\bigg\vert_{\vec\lam\leadsto \vec a^{(i)}}}
are power series in $\lam_1,\lam_2$ for each $\mu,\xi$.

Let $S'=\PP^1\times \PP^1$, with $\T_0$-action
\[(t_1,t_2)\cdot([x_0:x_1],[y_0:y_1])=([x_0:t_1x_1],[y_0,t_2y_1]).\]
We refer to \cite[§3.7]{lyz} for the following computations of equivariant weights. The fixed points are 
\[p_1=p_{00}=([1:0],[1:0]),\indent p_2=p_{01}=([1:0],[0:1]),\]
\[p_3=p_{10}=([0:1],[1:0]),\indent p_4=p_{11}=([0:1],[0:1]).\]
The corresponding weights are $\vec{a}^{(1)}=\vec{a}^{(00)},\vec{a}^{(2)}=\vec{a}^{(01)},\vec{a}^{(3)}=\vec{a}^{(10)},\vec{a}^{(4)}=\vec{a}^{(11)}$ where
\[a_1^{(ij)}=(-1)^i\lam_1,\indent  a_2^{(ij)}=(-1)^j\lam_2\]
for $i,j\in\{0,1\}$. Substituting into (\ref{poly-1}), we see the summands with odd $j$ or $k$ would cancel each other out, leaving us with
\equa{\sum_{j,k\text{ even}}G_{\mu,\xi,j,k}(q,z)\cdot  4\lam_1^j\lam_2^k.
}
Since $\lam_1^j\lam_2^k$ are linearly independent for all distinct $j,k$, and they are not polynomials for $\min\{j,k\}\leq -2$, the coefficients $G_{\mu,\xi,j,k}$ must all be 0 for these $j$ and $k$.

Having dealt with the case where $j,k$ are both even, we would like to apply the same argument to the other cases. To do so we need to solve the problem that the summands vanish whenever one of $j,k$ is odd. The fixed points on $\quot_{S'}(E',n)$ correspond to $M$-tuples of $N$-coloured partitions $(\mu^{(1)},\dots ,\mu^{(M)})$, where $\mu^{(i)}=(\mu^{(i,1)},\dots,\mu_i^{(i,N)})$ and each $\mu^{(i,j)}$ is a partition for $i=1,\dots,M$ and $j=1,\dots,N$. If we replace $w_k$ by $u_k(l+\lam_1+\lam_2)^2$ for some symmetric polynomial $p$ and numbers $l,u_1,\dots ,u_r$, the Chern roots of $(\O_{S'}\<w_k\>)^{[n]}$ would be replaced by
\[\bigcup_{i=1}^M\bigcup_{j=1}^N\bigcup_{\square\in\mu^{(i,j)}}\{u_k\cdot (l+a_1^{(i)}+a_2^{(i)})^2+m_j-c(\square)a_1^{(i)}-r(\square)a_2^{(i)}\}.\]
We claim that taking symmetric series of these Chern roots would result in terms composed of symmetric series of Chern roots of $K_{S'}^{[n]}$, and that of $\O_{S'}^{[n]}$, which are respectively given by the sets
\[\bigcup_{i=1}^M\bigcup_{j=1}^N\bigcup_{\square\in\mu^{(i,j)}}\{a_1^{(i)}+a_2^{(i)}+m_j-c(\square)a_1^{(i)}-r(\square)a_2^{(i)}\},\]
\[\bigcup_{i=1}^M\bigcup_{j=1}^N\bigcup_{\square\in\mu^{(i,j)}}\{m_j-c(\square)a_1^{(i)}-r(\square)a_2^{(i)}\}.\]
This is a result of Lemma \ref{lem:sym-poly} by setting $\vec{x}=(a_1^{(i)}+a_2^{(i)})_{i=1,\dots M},\vec{y}=(m_j-c(\square)a_1^{(i)}-r(\square)a_2^{(i)})_{i=1,\dots,M}^{\square\in\mu^{(i,j)}}$. 
Therefore after replacing $w_k$ by $u_k(l+a_1^{(i)}+a_2^{(i)})$, the resulting invariant is still an integral of characteristic classes of tautological bundles, so the new version of (\ref{poly-1}) remains a power series in $\lam_1,\lam_2$. View $u_1,\dots ,u_r$ as formal variables and replace them with $w_1,\dots ,w_r$, we see in total we have replaced each $w_k$ by $(l+a_1^{(i)}+a_2^{(i)})w_k$ for $k=1,\dots,r$.

As a result of the previous paragraph, the coefficients of
\equa{&\sum_{i=1}^{M}\sum_{j,k}G_{\mu,\xi,j,k}(q,z)\cdot  \lam_1^j\lam_2^k\cdot (l+\lambda_1+\lambda_2)^{2|\mu|}\bigg\vert_{\vec\lambda\leadsto\vec a^{(i)}}\\
=&\sum_{s=0}^{2|\mu|}\sum_{i=1}^M\sum_{j,k}\binom{2|\mu|}{s}G_{\mu,\xi,j,k}(q,z)\cdot\lam_1^j\lam_2^k\cdot l^{2|\mu|-s}(\lam_1+\lam_2)^{s}\bigg\vert_{\vec\lam\leadsto \vec a^{(i)}}}
are power series in $\lam_1,\lam_2$ for any integer $l\geq 0$. When $\mu\neq (0)$, the matrix formed by the vectors \[\left(\binom{2|\mu|}{0}l^{2|\mu|},\binom{2|\mu|}{1}l^{2|\mu|-1},\binom{2|\mu|}{2}l^{2|\mu|-2},\dots ,\binom{2|\mu|}{2|\mu|}l^{0}\right)\] 
for $l=1,2,3,\dots $ has maximal rank, we may take a linear combination of the above expression, and get that
\equa{&\sum_{i=1}^M\sum_{j,k\in\ZZ}G_{\mu,\xi,j,k}(q,z)\cdot\lam_1^j\lam_2^k\cdot (\lam_1+\lam_2)^{s}\bigg\vert_{\vec\lam\leadsto \vec a^{(i)}}}
is a power series in $\lam_1,\lam_2$ for each $s=0,1,\dots ,2|\mu|$.

Take $s=2$, we get
\equa{&\sum_{i=1}^{M}\sum_{j,k}G_{\mu,\xi,j,k}(q,z)\cdot  \lam_1^j\lam_2^k\cdot (\lambda_1+\lambda_2)^{2}\bigg\vert_{\vec\lam\leadsto \vec a^{(i)}}\\
=&\sum_{\substack{j,k\text{ odd}}}G_{\mu,\xi,j,k}(q,z)\cdot  8\lam_1^{j+1}\lam_2^{k+1} .}
Again, since $\lam_1^{j+1}\lam_2^{k+1}$ are linearly independent for distinct $j,k$ and are not polynomials in $\lam_1,\lam_2$ for any $\min\{j,k\}\leq -2$, we know $G_{\mu,\xi,j,k}=0$ whenever $j,k$ are both odd and $\mu\neq (0)$. 

In the case one of $j,k$ is odd and the other is even, continuing to assume $\mu\neq (0)$, we take $s=1$ and get
\equa{&\sum_{i=1}^{M}G_{\mu,\xi,j,k}(q,z)\cdot  \lam_1^j\lam_2^k\cdot (\lam_1+\lam_2)\bigg\vert_{\vec\lam\leadsto \vec a^{(i)}}\\
=&\begin{cases}
G_{\mu,\xi,j,k}(q,z)\cdot  8\lam_1^{j+1}\lam_2^{k}\text{, if $j$ odd, $k$ even}\\
G_{\mu,\xi,j,k}(q,z)\cdot  8\lam_1^{j}\lam_2^{k+1}\text{, if $k$ odd, $j$ even}.
\end{cases}
}
Although these are not polynomials when $\min\{j,k\}\leq -2$, we see there might be some linear dependence, i.e. we could have
\[G_{\mu,\xi,j,k}=-G_{\mu,\xi,j+1,k-1}\]
for $j$ odd and $k$ even, and terms canceling each other out in the sum. To solve this issue, we further apply the argument to the $s=3$ case and obtain the following dependencies
\[G_{\mu,\xi,j,k}=-G_{\mu,\xi,j+3,k-3}\]
for all $j$ odd, $k$ even and $\min\{j,k\}\leq -4$. Combining these relations we see for $\min\{j,k\}\leq -2$, there exist some constants $C_{\mu,\xi,a,b,l}^{\pm}$, labeled by the partitions $\mu,\xi$, integers $a,b,l$ and a sign $\pm$, such that
\equa{G_{\mu,\xi,j,k}(q,z)=&\sum_{a,b}((-1)^j-(-1)^k)C_{\mu,\xi,a,b,j+k}^{\pm}q^{a}z^{b}\\
=&\begin{cases}\sum_{a,b}2C_{\mu,\xi,a,b,j+k}^+q^{a}z^{b}\indent&\text{ if $j$ even, $k$ odd, $j\geq 0$}\\
\sum_{a,b}-2C_{\mu,\xi,a,b,j+k}^+q^{a}z^{b}\indent&\text{ if $j$ odd, $k$ even, $j> 0$,}\\
\sum_{a,b}2C_{\mu,\xi,a,b,j+k}^-q^{a}z^{b}\indent&\text{ if $j$ even, $k$ odd, $j<0$}\\
\sum_{a,b}-2C_{\mu,\xi,a,b,j+k}^-q^{a}z^{b}\indent&\text{ if $j$ odd, $k$ even, $j<0$,}\\
\end{cases}
}
The reason for the superscript $\pm$ is due to cases such as $j=-1,k=0$, where we would have $\min\{j,k\}>-2$, so the dependence does not necessarily hold. Because of this gap, we can not always relate the coefficient when $j\geq 0$ to $j<0$, resulting in separated cases. By the paragraphs preceding this theorem, for a fixed $a$, the degrees $j,k$ on $\lam_1,\lam_2$ of the $[q^{a}]$ coefficient are bounded below. However the above indicates that the constants $C_{\mu,\xi,a,b,l}^\pm$ only depend on the value $l=j+k$, and we can make $j$ or $k$ arbitrarily small. Hence $C_{\mu,\xi,a,b,l}^\pm=0$ whenever $\mu\neq (0)$.


With all the vanishings of $G_{\mu,\xi,j,k}$, we write
\equa{\log {\mathcal{N}}_S(E,V;q,z)=\sum_{\substack{\mu,\xi \text{ partitions}\\j,k\geq -1}}G_{\mu,\xi,j,k}(q,z)\cdot\lam_1^j\lam_2^k\cdot c_\mu(V)\\
+\sum_{\substack{\xi \text{ partition}\\\min\{j,k\}\leq -2}}G_{(0),\xi,j,k}(q,z)\cdot  \lam_1^j\lam_2^k.
}
To deal with the terms $G_{(0),\xi,j,k}(q,z)$ for $\min\{j,k\}\leq -2$, we apply Lemma \ref{lem:differential} and find
\[D_zG_{(0),\xi,j,k}(q,z)=kG_{(1),\xi,j,k}(q,z)=0,\]

so $G_{(0),j,k}$ is constant with respect to the variable $z$. Let us attempt to extract the $[q^{n}\lam_1^j\lam_2^kc_{(0)}(V)c_\xi(E)]$ coefficient of the Chern series from $G_{(0),\xi,j,k}$ using the Chern limit of Lemma \ref{lem:CV-limit}. This results in a limit $\e\->0$ of the term $\e^{n(N-r)}\e^{|\xi|}\e^{j+k}$, which does not make sense when the rank $r$ is sufficiently large, so we must have $G_{(0),\xi,j,k}=0$ for such $r$. To generalize this to arbitrary ranks, we apply \cite[Lemma 3.3]{GM} to ${\mathcal{N}}_S$, which says the coefficients of ${\mathcal{N}}_S$ are polynomials in $r$ when $r\geq 0$. Now we can write
\equa{\log {\mathcal{N}}_S(E,V;q,z)=\sum_{\substack{\mu,\xi \text{ partitions}\\j,k\geq-1}}G_{\mu,\xi,j,k}(q,z)\cdot\lam_1^j\lam_2^k\cdot c_\mu(V)c_\xi(E).
}
As noted in \cite[(31)]{OP}, the obstruction on $\hilb^n(S)$ at a fixed point $[Z_\mu]$ is $(K_S^{[n]})^\vee|_{Z_\mu}$. From (\ref{eqn:vb2d}), we see a copy of $K_S^\vee=t_1t_2$ is in $K_S^{[n]}|_{Z_\mu}$. By (\ref{eqn:Tvir-quot2d}), the obstruction bundle on $\quot_S(E,n)$ at any fixed point has at least one copy of $K_S^\vee$ as a direct summand. For a line bundle $L$, we have
\[\ch(\Lambda_{-1}L^\vee)=1-e^{-c_1(L)}=e(L)\cdot(1+\dots )\] 
where $\dots $ are some omitted terms in $H_\T^{>0}(\pt)$. Therefore $1/\ch(\Lambda_{-1}(T^\vir_Z)^\vee)$ has a factor of $e(K_S^\vee)=c_1(S)=\lambda_1+\lambda_2$ in its numerator. We also note that this factor does not appear in the denominator because if we pass to the subtorus $\{(t_1,t_2):t_1t_2=1\}$, the Zariski tangent space has no $\T$-fixed parts: by (\ref{eqn:tangent2}), the fixed part can only come from the direct summands with $i=j$, which correspond to the Hilbert scheme case; but by (\ref{eqn:tangent}), these summands have no fixed parts because $a(\square),l(\square)\geq 0$ for any box $\square$. Therefore we may extract this factor of $c_1(S)$ and obtain
\equa{\log {\mathcal{N}}_S(E,V;q,z)=\sum_{\substack{\mu,\xi \text{ partitions}\\j,k\geq-1}}H_{\mu,\xi,j,k}(q,z)\cdot\lam_1^j\lam_2^k\cdot c_\mu(V)c_\xi(E)c_1(S).
}
for some series $G_{\mu,\xi,j,k}\in\QQ[\![q,z]\!]$. Furthermore, since $j,k$ are now bounded below by $-1$, multiplying by $\lam_1\lam_2$ would give us a power series expansion in $\lam_1,\lam_2$, allowing us to use the symmetry in $\lam_1,\lam_2$ and write
\equanum{\label{eqn:log-nek2d-vir}\log {\mathcal{N}}_S(E,V;q,z)=\sum_{\mu,\nu, \xi\text{ partitions}}H_{\mu,\nu,\xi}(q,z)\cdot \int_Sc_\mu(V) c_\nu(S)  c_\xi(E) c_1(S).
}
for some series $H_{\mu,\nu,\xi}\in\QQ[\![q,z]\!]$.

Finally, taking Chern and Verlinde limits of $H_{\mu,\nu,\xi}$ as in Lemma \ref{lem:CV-limit}, then exponentiating gives us series $C_{\mu,\nu,\xi},B_{\mu,\nu,\xi}$ such that
\equa{
{\mathcal{C}}_S(E,V;q)=&\prod_{\mu,\nu,\xi\text{ partitions}}C_{\mu,\nu,\xi}(q)^{\int_{S}c_\mu(V) c_\nu(S)  c_\xi(E)c_1(S)},\\
{\mathcal{V}}_S(E,V;q)=&\prod_{\mu,\nu,\xi\text{ partitions}}B_{\mu,\nu,\xi}(q)^{\int_{S}c_\mu(V) c_\nu(S)  c_\xi(E)c_1(S)}.
}
and the fact that ${\mathcal{S}}_S(E,V;q)={\mathcal{C}}_S(E,-V;q)$ implies that there exists series $A_{\mu,\nu,\xi}$ such that
\[{\mathcal{S}}_S(E,V;q)=\prod_{\mu,\nu,\xi\text{ partitions}}A_{\mu,\nu,\xi}(q)^{\int_{S}c_\mu(V) c_\nu(S)  c_\xi(E)c_1(S)}.\]
To generalize this to arbitrary K-theory classes $\a=[V']-[V'']\in K_\T(S)$ for equivariant bundles $V',V''$ of rank $m,l$ respectively, we apply \cite[Lemma 3.3]{GM} once more; it states that the invariants for $\alpha$ are obtained by substituting
\[r\leadsto m-l,\indent p_n(v_1,v_2,\dots ,v_r)\leadsto p_n(v'_1,v'_2,\dots ,v'_m)-p_n(v''_1,v''_2,\dots ,v''_l),\]
where $p_n$ are the power-sum symmetric polynomials. Hence the above universal series expressions hold for all $\a\in K_\T(S)$.
\end{proof}

\begin{lemma}\label{lem:sym-poly}
Let $F(\vec x,\vec y)$ be a polynomial symmetric in $\vec x=(x_1,\dots, x_n)$ and symmetric in $\vec y=(y_1,\dots, y_m)$, then $F$ can be written as a polynomial expression of symmetric functions in $\{y_1,\dots,y_m\}$ and symmetric functions in $\{x_i+y_j\}_{i=1,\dots,n}^{j=1,\dots,m}$.
\end{lemma}
\begin{proof}
Expand $F$ in the elementary symmetric polynomial basis with respect to the variables $y_1,\dots, y_m$: 
\[F(\vec x,\vec y)=\sum_{\mu\text{ partition}}f_\mu(\vec x)e_\mu(\vec y).\]
An induction on the degree of $F$ shows that $f_\mu(\vec x)$ can be written in the desired form for $\mu\neq (0)$. Thus we may assume $F$ is independent of $y$. Furthermore, since if the statement holds for $F$ and $G$, then it holds for $F+G$ and $F\cdot G$, we can assume $F=e_k(\vec x)$. We have
\[e_k(\{x_i+y_j\}_{i=1,\dots,n}^{j=1,\dots,m})=K\cdot e_k(\vec x)+G(\vec x,\vec y)\]
for some constant $K\in\ZZ$, and $G$ is a polynomial symmetric in $\vec x$ and in $\vec y$, and every monomial term in $G$ contains some $y_j$. Apply the same argument to $G$ and we conclude that $G$ satisfies the claim, therefore so does $F$ by the above equation.
\end{proof}

By the non-equivariant Segre-Verlinde correspondence \cite[Theorem 1.7]{Bojko2} and the relations between the non-equivariant series and equivariant series illustrated in Section \ref{sec:proj-reduction}, we have a weak Segre-Verlinde correspondence as the following corollary. The same argument of the following proof also gives us a weak symmetry in the form of Corollary \ref{cor:sv-sym2d-intro}.


\begin{corollary}\label{cor:SV2d-virtual}
In the setting of Theorem \ref{thm:SV-univ-sieres}, we have the following correspondence
\equa{A_{\mu,\nu,\xi}(q)&=B_{\mu,\nu,\xi}((-1)^Nq).
}
whenever one of $\mu,\nu,\xi$ is $(1)$ and the other two are $(0)$. In particular, the degree 0 part satisfies
\[{\mathcal{S}}_{S,0}(E,\alpha;q)-{\mathcal{V}}_{S,0}(E,\alpha;(-1)^Nq)=\sum_{n=2}^\infty \frac{f_n}{(\lambda_1\lambda_2)^{n-2}}\cdot\left(\int_S c_1(S)\right)^2\cdot q^n\]
for some terms $f_n\in H_\T^{2n-2}(\pt)$ dependent on $\a$ through its rank and Chern classes.
\end{corollary}
\begin{proof}
By the argument of Section \ref{sec:proj-reduction}, the universal series in Theorem \ref{thm:SV-univ-sieres}, when passed to a toric projective surface, must give the Segre-Verlinde correspondence of \cite[Theorem 1.7]{Bojko2} in degree 0. Since the degree 0 terms occur only when one of $\mu,\nu,\xi$ is $(1)$ and the other two are $(0)$, we have
\equa{A_{\mu,\nu,\xi}(q)&=B_{\mu,\nu,\xi}((-1)^Nq)}
in those cases.

Note that when we take $\exp$ of (\ref{eqn:log-nek2d-vir}), the total degree 0 part might come from the product of a negative-degree term and a positive-degree term, but since each term in the integrand is accompanied by a copy $c_1(S)$, we know this difference must be a multiple of $c_1(S)^2$. We also see the $[q^n]$ coefficients are sums of products of at most $n$ such integrals, giving a denominator of $\lambda_1^n\lambda_2^n$, so we are done. 

For illustration, we shall extract this difference, and express it explicitly. This is just a standard computation. For a partition $\mu=(\mu_1,\mu_2,\dots ,\mu_L)$ of size $n$ with length $L$, and a sequence of positive integers $\kappa=(k_1,k_2,\dots ,k_L)$, write $\kappa|\mu$ if each $k_i|\mu_i$. We also associate a set of tuples of partitions to $\kappa$ by
\[
M_{\kappa} := \left\{ \left((\mu^{(i)})_{i=1}^{L},(\nu^{(i)})_{i=1}^{L},(\xi^{(i)})_{i=1}^{L}\right)\ \middle\vert \begin{array}{l}
    \text{$\mu^{(i)},\nu^{(i)},\xi^{(i)}$ are partitions for each $i$, s.t.}\\
    \indent \sum_{i=1}^Lk_i(|\mu^{(i)}|+|\nu^{(i)}|+|\xi^{(i)}|-1)=0\text{ and}\\
    \indent \text{$\mu_i=\nu_i=\xi_i=0$ for some $i$.}
  \end{array}\right\}.
\]
For each $n>0$, we would like to find the degree 0 part of the $[q^n]$ coefficient of $\exp$ of (\ref{eqn:log-nek2d-vir}). By expanding the exponential using definition, we observe that these terms come from products of integrals labeled by $\mu^{(i)},\nu^{(i)},\xi^{(i)}$ in $M_\kappa$ for some tuples $\kappa|\pi$ for some partition $\pi$ of size $n$. A more precise description is given by the following equation. Suppose $\log(A_{\mu,\nu,\xi}(q))=\sum_{i=1}^\infty a_{\mu,\nu,\xi,i}q^i$ and $\log(B_{\mu,\nu,\xi}(q))=\sum_{i=1}^\infty b_{\mu,\nu,\xi,i}q^i$, we have
\equa{&[q^n]({\mathcal{S}}_{S,0}(E,\alpha;q)-{\mathcal{V}}_{S,0}(E,\alpha;(-1)^Nq)\\
=&\sum_{\substack{|\pi|=n\\\ell(\pi)>1}}\sum_{\kappa|\pi}\prod_{(\vec\mu,\vec\nu,\vec\xi)\in M_\kappa}\prod_{i=1}^{\ell(\pi)}\frac{1}{k_i!}\left(a_{\mu^{(i)},\nu^{(i)},\xi^{(i)},\pi_i/k_i}\int_Sc_{\mu^{(i)}}(\a)c_{\nu^{(i)}}(S)c_{\xi^{(i)}}(E)c_1(S)\right)^{k_i}\\
&\indent-(-1)^{Nn}\sum_{\substack{|\pi|=n\\\ell(\pi)>1}}\sum_{\kappa|\pi}\prod_{(\vec\mu,\vec\nu,\vec\xi)\in M_\kappa}\prod_{i=1}^{\ell(\pi)}\frac{1}{k_i!}\left(b_{\mu^{(i)},\nu^{(i)},\xi^{(i)},\pi_i/k_i}\int_Sc_{\mu^{(i)}}(\a)c_{\nu^{(i)}}(S)c_{\xi^{(i)}}(E)c_1(S)\right)^{k_i}\\
=&\sum_{\substack{|\pi|=n\\\ell(\pi)>1}}\sum_{\kappa|\pi}\prod_{(\vec\mu,\vec\nu,\vec\xi)\in M_\kappa}\left(\prod_{i=1}^{\ell(\pi)}a_{\mu^{(i)},\nu^{(i)},\xi^{(i)},\pi_i/k_i}^{k_i|M_k|} -(-1)^{Nn}\prod_{i=1}^{\ell(\pi)}b_{\mu^{(i)},\nu^{(i)},\xi^{(i)},\pi_i/k_i}^{k_i|M_k|}\right)\\
&\cdot \prod_{i=1}^{\ell(\pi)}\frac{c_1(S)^{k_i}}{k_i!\lambda_1^{k_i}\lambda_2^{k_i}}\left(c_{\mu^{(i)}}(\a)c_{\nu^{(i)}}(S)c_{\xi^{(i)}}(E)\right)^{k_i}.
}
In the summation we have $\ell(\pi)>1$ because $M_\kappa$ is empty for any $\kappa|\pi$ if $\pi=(n)$ by definition. Therefore $2\leq \sum_{i=1}^{\ell(\pi)}k_i\leq n$. By multiplying some appropriate power of $\lambda_1\lambda_2$ to the denominator and numerator of the right-hand side, we can express it as a rational function in $\lambda_1,\lambda_2$, with denominator $\lambda_1^n\lambda_2^n$ and numerator a multiple of $c_1(S)^2$. Setting this multiple as $f_n\in\CC[\lambda_1,\lambda_2]$ gives the desired expression.

\end{proof}




\begin{example}
The universal series for ${\mathcal{N}}_S$ are known explicitly in the compact case \cite[Theorem 1.2]{Bojko2}. For a smooth projective surface $S$ and $\alpha$ of rank $r$, we apply \cite[Theorem~17, (16),(17)]{KHilb} for $f(x)=1-ze^x,g(x)=\frac{x}{1-e^{-x}}$ and get
\equa{{\mathcal{N}}_S(\O_S,\a;q,z)=\left[\left(\frac{1-zQ}{1-z}\right)^{r}\left(\frac{-rzQ(1-Q)}{1-zQ}+1\right)\right]^{c_1(S)^2}\left[\frac{1-zQ}{1-z}\right]^{c_1(S)\cdot c_1(\alpha)}}
via the substitution 
\[q=\frac{1-Q^{-1}}{(1-zQ)^r}.\]
As mentioned in the introduction, when $N=1$, we shall set $y_1=1$, and omit the subscript $\xi$ for $H_{\mu,\nu,\xi}$. The formula above allows us to compute $H_{(1),(0)}(q,z)$ and $H_{(0),(1)}(q,z)$.  For a smooth projective surface $S$ and $\alpha$ of rank $0$, we have
\[{\mathcal{N}}_S(\O_S,\a,q,z)=\left(\frac{1 - q - z}{(1 - q) (1 - z)}\right)^{ c_1(S)\cdot c_1(\alpha)}.\]
%For $\a$ rank 1, we have
%\[{\mathcal{N}}_S(\O_S,\a,q,z)=\left(1-qz\right)^{K_S\cdot c_1(\alpha)}.\]
The exponent is interpreted as intersection product, which in the toric case corresponds to the equivariant push-forward
\[\int_Sc_1(\alpha)c_1(S).\]
Therefore for rank 0, we have the following series from expansion (\ref{eqn:log-nek2d-vir})
\[H_{(1),(0)}(q,z)=\log\frac{1 - q - z}{(1 - q) (1 - z)},\indent H_{(0),(1)}(q,z)=0,\]
%and for rank 1,
%\[H_{(1),(0)}(q,z)=-\log(1-zq),\indent H_{(0),(1)}(q,z)=0.\]
Take the Chern limit of Lemma \ref{thm:SV-univ-sieres} by substituting $q\leadsto -q\e$, $z\leadsto (1+\e)^{-1}$ and get
\[C_{(1),(0)}(q)={1+q}\]
Replacing $\a$ by $-\a$ in the Chern series to get the Segre series for $\a$, we see
\[A_{(1),(0)}(q)=C_{(1),(0)}(q)^{-1}=\frac{1}{1+q}\]
On the other hand, the Verlinde limit yields
\[B_{(1),(0)}(q)=\frac{1}{1-q}\]
The Segre-Verlinde correspondence of Corollary \ref{cor:SV2d-virtual} is indeed satisfied.
\begin{comment}$A_{(1),(0)}(q),B_{(1),(0)}(q)$ for the following cases:
\begin{itemize}
    \item Segre rank $-1$:\hphantom{Verlinde} $A_{(1),(0)}(q)={1-q};$
    \item Verlinde rank $-1$:\hphantom{Segre} $B_{(1),(0)}(q)={1+q};$
    \item Segre rank $0$:\hphantom{Verlinde$-$} $A_{(1),(0)}(q)=\frac{1}{1+q};$
    \item Verlinde rank $0$:\hphantom{Segre$-$} $B_{(1),(0)}(q)=\frac{1}{1-q}.$
\end{itemize}
\end{comment}
\end{example}


\begin{example}
Let $S=\CC^2$, $n=N=2$, $E=\O_S\<y_1\>\oplus\O_S\<y_2\>$ and $L=\O_S\<v\>$. The $\T_1$-fixed locus of $\quot_S(E,n)$ is the disjoint union of 
\[\hilb^0(S)\times \hilb^2(S),\indent \hilb^1(S)\times \hilb^1(S), \indent\hilb^2(S)\times\hilb^0(S)\]
Denote $Z_\mu$ the point in $\hilb^i(S)^{\T_0}$ corresponding to a partition $\mu$, then the $\T$-fixed points of $\quot_S(\CC^2,2)$ are
\[(Z_\phi,Z_{(2)}),(Z_\phi,Z_{(1,1)}), (Z_{(1)},Z_{(1)}), (Z_{(2)},Z_\phi),(Z_{(1,1)},Z_\phi)\]
Therefore by (\ref{eqn:tangent2}), the virtual tangent bundles at these five points are respectively 
\equa{&(t_1^2 + t_2-t_1^2t_2  + y_1^{-1}y_2)(1 + t_1^{-1})\\
&(t_2^2 + t_1-t_1t_2^2  + y_1^{-1}y_2)(1 + t_2^{-1})\\
&(t_1+t_2-t_1t_2)(1 + y_1^{-1}y_2)(1+y_1y_2^{-1})\\
&(t_1^2 + t_2-t_1^2t_2  + y_1y_2^{-1})(1 + t_1^{-1})\\
&(t_2^2 + t_1-t_1t_2^2  + y_1y_2^{-1})(1 + t_2^{-1})\\
}
\begin{comment}\equa{&-t_1^2t_2 + t_1^2 - t_1t_2 + t_1 + t_2 + y_1^{-1}y_2 + t_1^{-1}t_2 + t_1^{-1}y_1^{-1}y_2,\\
&-t_1t_2^2 - t_1t_2 + t_2^2 + t_1 + t_2 + y_1^{-1}y_2 + t_1t_2^{-1} + t_2^{-1}y_1^{-1}y_2,\\
&-t_1t_2y_1^{-1}y_2 - 2t_1t_2 + t_1y_1^{-1}y_2 + t_2y_1^{-1}y_2 - t_1t_2y_1y_2^{-1} + 2t_1 + 2t_2 + t_1y_1y_2^{-1} + t_2y_1y_2^{-1},\\
&-t_1^2t_2 + t_1^2 - t_1t_2 + t_1 + t_2 + t_1^{-1}t_2 + y_1y_2^{-1} + t_1^{-1}y_1y_2^{-1},\\
&-t_1t_2^2 - t_1t_2 + t_2^2 + t_1 + t_2 + t_1t_2^{-1} + y_1y_2^{-1} + t_2^{-1}y_1y_2^{-1}.}
\end{comment}
The equivariant Chern roots of $\alpha^{[n]}$ at these points are respectively
\[\{m_2 + w, m_2 - \lambda_1 + w\},\{m_2 + w, m_2 - \lambda_2 + w\},\{m_1+w,m_2 + w\}, \{m_1+w, m_1-\lambda_1 + w\},\{m_1+w, m_1-\lambda_2 + w\}\\
\]
The contribution to the Segre numbers at each of these fixed points are
\[\frac{(2\lam_1 + \lam_1)(\lam_1 + \lam_2)}{2(m_1 - m_2 + \lam_1)(m_1 - m_2)(m_2 - \lam_1 + w_1 + 1)(m_2 + w_1 + 1)(\lam_2 - \lam_1)\lam_1^2\lam_2},\]
\[\frac{(\lam_1 + 2\lam_2)(\lam_1 + \lam_2)}{2(m_1 - m_2 + \lam_2)(m_1 - m_2)(m_2 - \lam_2 + w_1 + 1)(m_2 + w_1 + 1)(\lam_1 - \lam_2)\lam_1\lam_2^2},\]
\[\frac{( \lam_1 + \lam_2+m_1 - m_2)(\lam_1 + \lam_2-m_1 + m_2 )(\lam_1 + \lam_2)^2}{(m_1 - m_2 + \lam_1)(m_1 - m_2 - \lam_1)(m_1 - m_2 + \lam_2)(m_1 - m_2 - \lam_2)(m_1 + w1 + 1)(m_2 + w1 + 1)\lam_1^2\lam_2^2},\]
\[\frac{(2\lam_1 + \lam_1)(\lam_1 + \lam_2)}{2(m_1 - m_2 - \lam_1)(m_1 - m_2)(m_1 - \lam_1 + w_1 + 1)(m_1 + w_1 + 1)(\lam_2 - \lam_1)\lam_1^2\lam_2},\]
\[\frac{(\lam_1 + 2\lam_2)(\lam_1 + \lam_2)}{2(m_1 - m_2 - \lam_2)(m_1 - m_2)(m_1 - \lam_2 + w_1 + 1)(m_1 + w_1 + 1)(\lam_1 - \lam_2)\lam_1\lam_2^2}.\]

Summing them up, we have
\[\scriptstyle[q^2]S^2(\a;q)=\frac{{\begin{pmatrix*}[l]\scriptstyle m_{1} m_{2} \lambda_1 - m_{1} \lambda_1^{2} - m_{2} \lambda_1^{2} + \lambda_1^{3} + m_{1} m_{2} \lambda_2 - 3  m_{1} \lambda_1 \lambda_2 - 3  m_{2} \lambda_1 \lambda_2 + 3  \lambda_1^{2} \lambda_2 - m_{1} \lambda_2^{2} - m_{2} \lambda_2^{2}\\\scriptstyle + 3  \lambda_1 \lambda_2^{2} + \lambda_2^{3} + m_{1} \lambda_1 w + m_{2} \lambda_1 w - 2  \lambda_1^{2} w + m_{1} \lambda_2 w + m_{2} \lambda_2 w - 6  \lambda_1 \lambda_2 w - 2  \lambda_2^{2} w\\\scriptstyle + \lambda_1 w^{2} + \lambda_2 w^{2} + m_{1} \lambda_1 + m_{2} \lambda_1 - 2  \lambda_1^{2} + m_{1} \lambda_2 + m_{2} \lambda_2 - 6  \lambda_1 \lambda_2 - 2  \lambda_2^{2} + 2  \lambda_1 w + 2  \lambda_2 w + \lambda_1 + \lambda_2\end{pmatrix*}} {\left(\lambda_1 + \lambda_2\right)}}{2  {\left(m_{1} - \lambda_1 + w + 1\right)} {\left(m_{1} - \lambda_2 + w + 1\right)} {\left(m_{1} + w + 1\right)} {\left(m_{2} - \lambda_1 + w + 1\right)} {\left(m_{2} - \lambda_2 + w + 1\right)} {\left(m_{2} + w + 1\right)} \lambda_1^{2} \lambda_2^{2}}.\]
A similar computation yields another complicated expression for the Verlinde number. We are interested in the total degree 0 part of their difference in the variables $\vec \lam,\vec m,\vec w$, this computes to
\equa{[q^2](S_{S,0}(E,L;q)-V_{S,0}(E,L;q))&=-\frac{{\left(3  m_{1} m_{2} - \lambda_1\lambda_2 + 3  m_{1} w + 3  m_{2} w + 3  w^{2}\right)} {\left(\lambda_1 + \lambda_2\right)}^{2}}{3  \lambda_1^{2} \lambda_2^{2}}\\
&=\left(\frac13c_2(S)-c_2(E)-c_1(E)c_1(V)-c_1(V)^2\right)\left(\int_Sc_1(S)\right)^2.
}
Note that even though the expressions for Segre and Verlinde numbers are complicated, their difference in degree 0 simplifies tremendously and satisfies Corollary \ref{cor:SV2d-virtual}.
\end{example}




\subsection{Reduced virtual classes and invariants}%\label{sec:red-2d}
As mentioned previously, the obstruction for $\quot_S(E,n)$ at $Z$ contains at least one copy of $K_S^\vee$. For $K$-trivial surfaces, this causes the Euler class of $T^\vir$ to vanish. Therefore the virtual Verlinde and Segre numbers both vanish. One can instead study the ``reduced" versions of these invariants. By \cite[Proposition 9]{lim}, when $S$ is a $K$-trivial surface, $n>0$, and $E$ a torsion-free sheaf, there is a \emph{reduced obstruction theory} that is perfect in the sense of Definition \ref{defn:obs}. The reduced (virtual) tangent bundle in this case is given by adding a trivial summand to the usual virtual tangent bundle:
\[T^\redu=T^\vir+\O_{\quot_S(E,n)}.\]

In this section, we study the equivariant analogue where $S=\CC^2$ with the natural action of the the 1-dimensional torus \equa{\T_0=\CC^*=\{(t_1,t_2):t_1t_2=1\}.}
Write
\[H^*_{\T_0}(\pt)=\CC[\lambda_1,\lambda_2]/(\lambda_1+\lambda_2)=\CC[\lambda],\]
\[K_{\T_0}(\pt)=\ZZ[t_1^{\pm1},t_2^{\pm1}]/(t_1t_2-1)=\ZZ[t^{\pm1}].\]


Using the argument of \cite[Lemma~3.1]{CK1}, we see that the $\T=(\T_0\times\T_1\times\T_2)$-fixed locus of $\quot_S(E,n)$ stays unchanged, and the Zariski tangent space at each of the fixed points has no fixed parts by the descriptions (\ref{eqn:tangent}) and (\ref{eqn:tangent2}). The \emph{equivariant reduced Segre and Verlinde series} $\mathcal{S}^\redu_S$ and $\mathcal{V}^\redu_S$ are defined the same as the virtual ones by replacing $T^\vir$ with $T^\redu$. Here we omit the subscript since we are only interested in the $S=\CC^2$ case.
\equa{{\mathcal{S}}^{\redu}(E,\alpha;q):=&\sum_{n>0}^\infty q^n\sum_{Z\in\quot_S(E,n)^\T}\frac{c(\alpha^{[n]}|_{Z})}{e(T_{Z}^\redu)},\\
{\mathcal{V}}^{\redu}(E,\alpha;q):=&\sum_{n>0}^\infty q^n\sum_{Z\in\quot_S(E,n)^\T}\frac{\ch(\det(\alpha^{[n]}|_{Z}))}{\ch(\Lambda_{-1}(T_{Z}^\redu)^\vee)}.}
Note that we do not include the $n=0$ term because condition 2 of \cite[Proposition 9]{lim} is only satisfied when $n>0$.

The same strategy from the previous section can be applied to study these invariants. For $E=\oplus_{i=1}^N\O_S\<y_i\>$ and $V=\oplus_{i=1}^r\O_S\<v_i\>$, define
\equa{{\mathcal{N}}^{\redu}(E,V;q,z):=&\sum_{\mu\neq(0)} q^{|\mu|}\prod_{\square\in\mu}
\frac{\prod_{j=1}^N\prod_{i=1}^r(1-t^{-c(\square)+r(\square)}v_iy_jz)}{\ch(\Lambda_{-1}(T^\redu|_Z)^\vee)}.
}
Again note that the $[q^0]$ coefficient is 0. We can think of the reduced obstruction as removing a copy of $K_S^\vee$ from the usual obstruction in $\ZZ[t_1^{\pm1},t_2^{\pm1}]$, then passing to the quotient ring $\ZZ[t_1^{\pm1},t_2^{\pm1}]/(t_1t_2-1)$. This gives us the following result.
\begin{corollary}
For $n>0$, the $[q^n]$ coefficient of $\mathcal{N}^\redu$ can be obtained from the non-reduced version by taking the following limit:
\equa{[q^n]{\mathcal{N}}^{\redu}(E,V;q,z)&=[q^n]\frac{{\mathcal{N}}_S(E,V;q,z)}{1-e^{-c_1(K_S^\vee)}}\bigg\vert_{-\lambda_2\->\lam_1=\lambda}\\
&=[q^n]\lim_{-\lam_2\->\lam_1=\lam}\frac{{\mathcal{N}}_S(E,V;q,z)}{\lambda_1+\lambda_2}.
}
\end{corollary}


Expand the universal series expression from Theorem \ref{thm:SV-univ-sieres} and obtain
\equa{{\mathcal{S}}_S(E,\a;q)=&\sum_{i=1}^\infty \frac{1}{i!}(\lambda_1+\lambda_2)^i\left(\sum_{\mu,\nu,\xi} \log A_{\mu,\nu,\xi}(q)\cdot  \int_Sc_\mu(\a)c_\nu(S)c_{\xi}(E)\right)^i.
}
Using the above corollary to extract the reduced coefficients, we see the terms with $i>1$ all vanish and
\[{\mathcal{S}}^{\redu}(E,\a;q)=\sum_{\mu,\nu,\xi} \log A_{\mu,\nu,\xi}(q)\cdot  \int_Sc_\mu(\a)c_\nu(S)c_{\xi}(E).\]
The Chern and Verlinde cases are similar, thus we have the following result.

\begin{theorem}\label{cor:reduced-series}
When $S=\CC^2$, the equivariant reduced Segre and Verlinde series for $E=\oplus_{i=1}^N\O_S\<y_i\>$ and $\a\in K_\T(S)$ are
\equa{
{\mathcal{S}}^{\redu}(E,\a;q)=&\sum_{\mu,\nu,\xi}\log\left(A_{\mu,\nu,\xi}(q)\right)\cdot \int_S c_\mu(\a)c_\nu(S)c_\xi(E),\\
{\mathcal{V}}^{\redu}(E,\a;q)=&\sum_{\mu,\nu,\xi}\log\left(B_{\mu,\nu,\xi}(q)\right)\cdot \int_S c_\mu(\a)c_\nu(S)c_\xi(E),\\
{\mathcal{C}}^{\redu}(E,\a;q)=&\sum_{\mu,\nu,\xi}\log\left(C_{\mu,\nu,\xi}(q)\right)\cdot \int_S c_\mu(\a)c_\nu(S)c_\xi(E)\\
}
where $A_{\mu,\nu,\xi}, B_{\mu,\nu,\xi}$ and $C_{\mu,\nu,\xi}$ are the same series from Theorem \ref{thm:SV-univ-sieres}.
\end{theorem}
The integrals in the above theorem labeled by $\mu,\nu,\xi$ have degree $|\mu|+|\nu|+|\xi|-2$, which is one degree lower than the integrals in the non-reduced expressions. Therefore we have a Segre-Verlinde correspondence in degree $-1$ for the reduced setting. However, results for degree $-1$ have no compact analogues since they automatically vanish in the compact setting. In the Section \ref{sec:reduced-computation}, we compute some of the universal series explicitly, giving us some Segre-Verlinde relations in non-negative degrees for the reduced case. 
