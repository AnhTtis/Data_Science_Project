 

\subsection{Equivariant cohomology and K-theory}\label{sec:equivar-coho}
Given a topological group $G$ acting on a topological space $M$, the \emph{equivariant cohomology} $H_G^*(M)$ is defined to be $H^*(EG\times M/G)$, where $EG\->BG$ is the universal principle $G$-bundle on the classifying space $BG$. The map $M\->\text{pt}$ induces a ring homomorphism $H^*_G(\text{pt})\->H^*_G(M)$, making $H^*_G(M)$ a module over $H^*_G(\text{pt})$ for any $M$, and we can view $H^*_G(\text{pt})$ as a ``coefficient ring". 

\begin{definition} Given a $G$ representation $V$, viewed as a vector bundle $V\->\{\text{pt}\}$, we define its \emph{equivariant characteristic classes} by taking the associated bundle 
\[EG\times_GV\-> EG\times_G\{\pt\}=BG\]
and taking its characteristic classes in $H^*(BG)=H^*_G(\pt)$. Denote $c^G_i,e_G,\ch_G,\td_G$ the equivariant versions of the $i$-th Chern class, the Euler class, the Chern character, and the Todd class respectively.
\end{definition}

\begin{example}For the action of a $d$-dimensional torus $\T=(\CC^*)^d=\{(t_1,\dots ,t_d):t_i\neq 0\}$, the coefficient ring is
\[H^*_{\T}(\pt)=H^*(B\T)=H^*((\CC P^\infty)^d)=\CC[\lambda_1,\dots ,\lambda_d]\]
and $\lambda_1,\dots ,\lambda_d$ are exactly the equivariant first Chern classes of 1-dimensional $\T$-representations with weight $t_1,\dots ,t_d$ respectively. In general \cite[Section~3.2]{edidin},
\[c_1^{\T}(\CC\< t_1^{w_1}\dots t_d^{w_d}\>)=w_1\lambda_1+\dots +w_d\lambda_d.\]
For the $d-1$-dimensional subtorus $\T'=\{(t_1,\dots ,t_d)\in \CC^d: t_1\dots t_d=1\}\seq \T$, the inclusion induces the following isomorphism to the quotient ring:
\[H_{\T'}^*(\text{pt})\cong\CC[\lambda_1,\dots ,\lambda_d]/(\lambda_1+\dots +\lambda_d).\]
\end{example}
We can construct the \emph{equivariant K-group} $K_G(M)$ from the $G$-equivariant vector bundles. When $M=\CC^d$,  vector bundles over $M$ are trivial, but they may carry non-trivial $G$-actions. Therefore the equivariant bundles on $\CC^d$ correspond to finite-dimensional $G$-representations. The equivariant characteristic classes of vector bundles can then be extended to the K-theory classes. For example, the Euler class of $\alpha=[V]-[W]\in K_G(\pt)$ is $e^G(\a)=e^G(V)/e^G(W)$, which lives in the ring of fractions $H_G^*(\pt)_\loc$; the Chern character is $\ch_G(\a)=\ch_G(V)-\ch_G(W)$, which lives in $\prod_{i=0}^\infty H^i_G(\pt)$.

\begin{example} When $Y=\CC^d$ with the natural action by $\T=(\CC^*)^d$ or $\T'=(\CC^*)^{d-1}$, we have the following character rings:
\[K_{\T}(Y)\cong \ZZ[t_1^{\pm1},\dots ,t_d^{\pm1}]\cong K_{\T}(\pt),\]
\[K_{\T'}(Y)\cong \frac{\ZZ[t_1^{\pm1},\dots ,t_d^{\pm1}]}{(t_1\cdots t_d-1)}\cong K_{\T'}(\pt) \]
where for any weight $w=(w_1,\dots ,w_d)$, the line bundle $\O_Y\<t^w\>:=\mathcal{O}_Y\otimes t^w$ simply corresponds to its character $t^w=t_1^{w_1}\cdots t_d^{w_d}$.
\end{example}

\begin{remark}\label{rmk:K-H-identification}
We will occasionally identify Chern characters, which are power series in cohomology, with elements in K-theory by
\[t_1^{w_1}\cdots t_d^{w_d}\leftrightarrow \ch_\T(\O_Y\<t_1^{w_1}\cdots t_d^{w_d}\>)=e^{w_1\lambda_1+\dots +w_d\lambda_d}.\]
This allows us to consider certain classes in cohomology as elements of $K_\T(\pt)$. For example, for the line bundle $L=\O_Y\<t_1^{w_1}\cdots t_d^{w_d}\>$, we write
\[\ch_\T(\Lambda_{-1}L^\vee)=1-e^{-c_1^\T(L)}=1-t_1^{-w_1}\cdots t_d^{-w_d}\in K_\T(Y).\]
\end{remark}

The reason we consider equivariant cohomology is for equivariant integration. The integration formula of \cite[Corollary 1]{Bott-residue} via equivariant localization states that on a smooth complete variety $Y$ with the action of a torus $\T$, for $\lambda$ an equivariant cohomological class, we have
\[\pi_{Y\ast}(\lambda)=\sum_{F}\pi_{F\ast}\left(\frac{i_F^*\lambda}{e_\T(N_FY)}\right)\]
where the sum goes through the components $F$ of the fixed locus, $N_FY$ denotes the normal bundle, $\pi$ denotes projection to a point, and $i$ denotes the inclusion map. The right-hand side of this formula can be used to define equivariant integration in general; for $Y$ an arbitrary smooth variety with finitely many fixed (reduced) points, the \emph{equivariant push-forward} of $\pi_Y$ is
\equanum{\label{eqn:equivar-int}\int_Y:H^*_\T(Y)&\->H^*_\T(\pt)_{\loc},\\
\alpha\indent&\mapsto\sum_{x\in Y^\T}\frac{i_x^*\alpha}{e_\T(T_xY)}.}

\begin{example}\label{ex:equi-push} Again let $Y=\CC^d$ with the natural $\T=(\CC^*)^d$-action. The only $\T$-fixed point of $Y$ is the origin. At the origin, the character for the tangent space is $T_{0}Y=t_1+t_2+\cdots+t_d\in K_T(\pt)$, so $e_\T(T_{0}Y)=\lambda_1\cdots\lambda_d$. Substituting into (\ref{eqn:equivar-int}), we have
\[\int_Y\alpha=\frac{\alpha}{\lambda_1\cdots\lambda_d}\]
\end{example}


\subsection{Partitions and solid partitions}\label{sec:partition}
A \emph{partition} $\mu$ is a finite sequence $(\mu_1,\mu_2,\dots ,\mu_\ell)$ of non-increasing positive integers. The size $|\mu|$ is the sum of $\mu_i$'s and we call $\ell=\ell(\mu)$ its length. The empty sequence $(0)$ is the \emph{empty partition} with size $|(0)|=0$. Each partition $\mu$ corresponds to a young diagram which consists of pairs of non-negative integers $(i,j)\in\ZZ_{\geq 0}^2$ as follows
\[\mu\xleftrightarrow{} \{(i,j):j< \mu_{i+1}\}.\]
A pair $\square=(i,j)$ in the above set is called a box in $\mu$, which we denote $\square\in\mu$. The \emph{conjugate partition} $\mu^t$ is defined to be the partition whose boxes are $\{(j,i):(i,j)\in\mu\}$. Denote $c(\square),r(\square),a(\square),l(\square)$ the \emph{column index}, \emph{row index}, \emph{arm length} and \emph{leg length} of $\square=(i,j)\in\mu$, defined explicitly as follows
\equa{&c(\square)=j,\indent r(\square)=i,\\ &a(\square)=\mu_{i+1}-j-1,\indent l(\square)=\mu_{j+1}^t-i-1.
}

When $i,j>0$, a necessary condition for box $(i,j)$ to be in $\mu$ is that both $(i-1,j)$ and $(i,j-1)$ are in $\mu$. When $i=0$ (resp. $j=0$), we only need $(i,j-1)\in\mu$ (resp. $(i-1,j)\in\mu$). 


A \emph{solid partition} $\pi$ is a finite sequence $(\pi_{ijk})_{i,j,k\geq 1}$ of positive integers such that
\equa{\pi_{ijk}\geq \pi_{i+1,j,k},\indent \pi_{ijk}\geq \pi_{i,j+1,k},\indent\pi_{ijk}\geq \pi_{i,j,k+1}.}
The size of $|\pi|$ is the sum of the $\pi_{ijk}$'s. As a 4-dimensional analogue to partitions, the solid partition can also be viewed as a collection of boxes
\[\pi\xleftrightarrow{}\{(i,j,k,l):l< \pi_{i,j,k}\}\seq \ZZ^4_{\geq 0}.\]
Similar to partitions, we have
\equanum{\label{solid-partition}(i,j,k,l)\in\pi\text{ implies }\begin{cases}
(i-1,j,k,l)\in\pi \text{ unless $i=0$,}\\
(i,j-1,k,l)\in\pi \text{ unless $j=0$,}\\
(i,j,k-1,l)\in\pi \text{ unless $k=0$,}\\
(i,j,k,l-1)\in\pi \text{ unless $l=0$.}\\
\end{cases}
}

For a positive integer $N$, an \emph{$N$-colored partition} of size $n$ is an $N$-tuple of partitions $\mu=(\mu^{(1)},\dots ,\mu^{(N)})$ such that $|\mu|:=\sum|\mu^{(i)}|=n$. Figure \ref{fig1} illustrates how the partitions are colored based on their index. Similarly, an \emph{$N$-colored solid partition} is an $N$-tuple of solid partitions.
\begin{figure}
\centering
\includegraphics[width=0.4\textwidth]{colored_partition_ex.png}
\caption{A $3$-colored partition $\mu=(\mu^{(1)},\mu^{(2)},\mu^{(3)})$ of size $|\mu|=19$ where $\mu^{(1)}=(5,3,1)$, $\mu^{(2)}=(4,1)$, $\mu^{(3)}=(3,2)$ are colored by green, blue and yellow respectively}
\label{fig1}
\end{figure}

\subsection{Admissible functions and universal series}%\label{sec:admissible}
We consider the notion of admissibility in the sense of \cite{mel-HLV}, which will be an important condition in finding universal series for equivariant invariants. 



\begin{definition}\label{def:admissible}
Let $F(Q_1,Q_2\dots ;q_1,\dots ,q_d)\in \QQ(q_1,\dots ,q_d)[\![Q_1,Q_2,\dots ]\!]$ be a series in finitely many variables $Q_1,Q_2,\dots $ with constant term equal to 1. Then using the plethystic exponential $\Exp$, we can write
\[F=\Exp\left(\frac{L}{(1-q_1)\cdots(1-q_d)}\right)\]
such that $L$ is a power series in the variables $Q_1,Q_2,\dots$ whose coefficients are rational functions in $q_1,\dots ,q_d$. The series $F$ is called \emph{admissible with respect to the variables} $q_1,\dots, q_d$ if the coefficients of $L$ are polynomials in $q_1,\dots ,q_d$.
\end{definition}


Suppose $F(Q;m_1,\dots ,m_N;w_1,\dots ,w_r;q_1,\dots ,q_d)\in \QQ(q_1,\dots ,q_d)[\![Q;m_1,\dots ,m_N;w_1,\dots ,w_r]\!]$ is admissible with respect to $q_1,\dots,q_d$ with constant term 1, we have the following Laurent expansion
\equa{\log F(Q;\vec{m};\vec{w};;e^{\lambda_1},\dots ,e^{\lambda_d})=\sum_{k_1,\dots ,k_d=-\infty}^\infty H_{k_1,\dots ,k_d}(Q;\vec{m};\vec{w})\lambda_1^{k_1}\dots \lambda_d^{k_d}.}
Since $F$ is admissible, by the definition of plethystic exponential, 
\[(1-q_1)\cdots(1-q_d)\log F(Q;\vec{m};\vec{w};\vec q)\]
is regular in a neighbourhood of $q_1=\dots =q_d=0$ as a power series in $q_1,\dots,q_d$, meaning we have a lower bound $k_1,\dots ,k_d\geq -1$ for the above summation.


Furthermore, suppose $F$ is symmetric in $w_1,\dots ,w_r$ and symmetric in $m_1,\dots ,m_N$, then we can expand in the following elementary symmetric polynomial basis:
\[\log F(Q;\vec{m};\vec{w};e^{\lambda_1},\dots ,e^{\lambda_d})=\sum_{\substack{\mu,\xi\text{ partitions}\\k_1,\dots ,k_d\geq -1}}H_{\mu,\xi,\vec{k}}(Q)\prod_{i=1}^{\ell(\mu)}e_{\mu_i}(\vec w)\prod_{i=1}^{\ell(\xi)}e_{\xi_i}(\vec m)\lambda_1^{k_1}\dots \lambda_d^{k_d}\]
for some series $H_{\mu,\xi,\vec{k}}$.

Let $Y=\CC^d$ and 
\[\T_0=(\CC^*)^d,\T_1=(\CC^*)^N,\T_2=(\CC^*)^r\] 
with the natural actions on $Y,E=\CC^N\otimes\O_Y,V=\CC^r\otimes \O_Y$ respectively. Denote $\T=\T_0\times \T_1\times \T_2$. Say the equivariant cohomology ring of $\T$ is $H_\T^*(\pt)=\CC[\lambda_1,\dots ,\lambda_d;m_1,\dots ,m_N;w_1,\dots ,w_r]$. Then $V$ as a $\T$-equivariant bundle has equivariant Chern roots $w_1,\dots ,w_r$, and $E$ has Chern roots $m_1,\dots ,m_N$, so $e_i(m_1,\dots ,m_N)=c_i^\T(E),e_i(w_1,\dots ,w_r)=c_i^\T(V)$. Therefore
\[\log F(Q;\vec{m};\vec{w};e^{\lambda_1},\dots ,e^{\lambda_d})=\sum_{\substack{\mu,\xi\text{ partitions}\\k_1,\dots ,k_d\geq -1}}H_{\mu,\xi,\vec k}(Q)c_\mu(V)c_\xi(E)\lambda_1^{k_1}\dots \lambda_d^{k_d}.\]

For $\vec k=(k_1,\dots ,k_d)$ where $k_1,\dots ,k_d\geq -1$, there exist polynomials $E_{\vec k}$ such that 
\equa{\frac1{d!}\sum_{\tau\text{ permutation}}\lambda_1^{k_{\tau(1)}}\cdots\lambda_d^{k_{\tau(d)}}=\frac{E_{\vec k}(e_1(\lambda_1,\dots ,\lambda_d),\dots ,e_d(\lambda_1,\dots ,\lambda_d))}{\lam_1\cdots\lam_d}}
Now suppose $F$ is symmetric in the variables $q_1,\dots ,q_d$, so $H_{\mu,k}=H_{\mu,\tau(k)}$ for any permutation $\tau$. Hence
\equa{&\log F(Q;\vec{m};\vec{w};e^{\lambda_1},\dots ,e^{\lambda_d})\\
=&\sum_{\substack{\mu\text{ partition}\\k_1,\dots ,k_d\geq -1}}H_{\mu,\xi,\vec k}(Q)E_{\vec k}(e_1(\lambda_1,\dots ,\lambda_d),\dots ,e_d(\lambda_1,\dots ,\lambda_d))c_\mu(V)c_\xi(E)}
Note the equivariant weights of the tangent space $T_0Y$ are exactly $\lambda_1,\dots ,\lambda_d$, so $e_i(\lambda_1,\dots ,\lambda_d)=c_i^\T(Y)$. By Example  \ref{ex:equi-push}, we have
\equanum{\label{eqn:log-partition-E}\log F(Q;\vec{m};\vec{w};e^{\lambda_1},\dots ,e^{\lambda_d})=\sum_{\substack{\mu\text{ partition}\\k_1,\dots ,k_d\geq -1}}H_{\mu,\xi,\vec{k}}(Q)\int_Y E_{\vec k}(c_1(Y),\dots ,c_d(Y))c_\mu(V)c_\xi(E)}
Redistributing the terms, we get 
\[\log F(Q;\vec{m};\vec{w};e^{\lambda_1},\dots ,e^{\lambda_d})=\sum_{\mu,\nu, \xi \text{ partitions}}H_{\mu,\nu,\xi}(Q)\int_Y c_\nu(Y)c_\mu(V)c_\xi(E)\]
for some series $H_{\mu,\nu,\xi}$. Exponentiate both sides, and we obtain the following \emph{universal series expression} for $F$.



\begin{proposition}\label{prop:series-admissible}
Let $F(Q;\vec{m};\vec{w};\vec{q})\in \QQ(q_1,\dots ,q_d)[\![Q;m_1,\dots ,m_N;w_1,\dots ,w_r]\!]$ be admissible with respect to the variables $q_1,\dots,q_d$. Suppose $F$ is symmetric in $w_1,\dots ,w_r$, in $m_1,\dots ,m_N$, and symmetric in $q_1,\dots ,q_d$, then there exist power series $G_{\mu,\nu,\xi}(Q)$ labeled by partitions $\mu,\nu,\xi$, such that
\[F(Q;\vec{m};\vec{w};e^{\lambda_1},\dots,e^{\lambda_n})=\prod_{\mu,\nu}G_{\mu,\nu,\xi}(Q)^{\int_Y c_\nu(Y)c_\mu(V)c_\xi(E)}.\]
\end{proposition}

