Recall that we use the notation $H_{\mu,\nu,\xi}$ for the series from (\ref{eqn:log-nek2d-vir}) describing the virtual Nekrasov genus for  Quot schemes on $S=\CC^2$. Note that $\mathcal{N}_S$ satisfies
\equa{{\mathcal{N}}_{S}(y_1,\dots,y_N;v_1,\dots,v_r;q,z)&={\mathcal{N}}_{S}(y_1,\dots,y_N;ze^{w_1},\dots,ze^{w_r};q,1)\\
&={\mathcal{N}}_{S}(ze^{m_1},\dots,ze^{m_N};v_1,\dots,v_r;q,1).
}
Applying Lemma \ref{lem:differential} to $H_{\mu,\nu,\xi}$ in the variables $w_1,\dots, w_r$ gives us for $r>0$, 
\[D_z^{k}H_{(0),\nu,\xi}(q,z)=rD_z^{k-1}H_{(1),\nu,\xi}(q,z)=k!\sum_{|\mu|=k}\binom{r}{\mu}H_{\mu,\nu,\xi}(q,z),\]
and applying in the variables $m_1,\dots,m_N$ yields
\equanum{\label{eqn:xi}D_z^{k}H_{\mu,\nu,(0)}(q,z)=ND_z^{k-1}H_{\mu,\nu,(1)}(q,z)=k!\sum_{|\xi|=k}\binom{N}{\xi}H_{\mu,\nu,\xi}(q,z).}

When the rank $r$ is negative, we consider $\a=-[V]$ where $V=\oplus_{i=1}^{-r}\O_S\<v_i\>$. Write 
\[\log {\mathcal{N}}_S(E,-[V];q,z)=\sum_{\mu,\nu, \xi\text{ partitions}}H_{\mu,\nu,\xi}(q,z)\cdot \int_Sc_\mu(V) c_\nu(S)  c_\xi(E) c_1(S).\]
Then the same argument with Lemma \ref{lem:differential} applies, and for all $r\neq 0$,
\equanum{\label{eqn:mu}D_z^{k}H_{(0),\nu,\xi}(q,z)=|r|D_z^{k-1}H_{(1),\nu,\xi}(q,z)=k!\sum_{|\mu|=k}\binom{|r|}{\mu}H_{\mu,\nu,\xi}(q,z).}
Observe that when $r>0$, the Chern limit of $H_{\mu,\nu,\xi}$ returns the universal series for the Chern invariants $\log C_{\mu,\nu,\xi}$, and the Verlinde limit returns $\log B_{\mu,\nu,\xi}$. On the other hand, when $r<0$, the Chern series retrieves the rank $-r$ Segre invariants $\log A_{\mu,\nu,\xi}$, but the Verlinde limit does not give the Verlinde series. This is because $H_{\mu,\nu,\xi}$ is associated to $c_\mu(V)$, whereas $\log B_{\mu,\nu,\xi}$ is associated to $c_\mu(\a)=c_\mu(-[V])$. This results in a change of basis for the symmetric series in the Chern roots of $V$, and Lemma \ref{lem:differential} no longer applies for these negative ranks.

Our goal for this section is to apply Chern and Verlinde limits to (\ref{eqn:xi}) and (\ref{eqn:mu}) for $k>0$, then obtain relations for the Chern and Verlinde series for various $\mu$, $\nu$, and $\xi$. We may obtain explicit expressions for when $|\mu|+|\nu|+|\xi|=1$ from the compact setting. For a smooth projective surface $S'$, a torsion-free sheaf $E'$ of rank $N$, and a K-theory class $\a$ of rank $r$, the universal series structure for $\mathcal{N}_S$ can be obtained from \cite[(5.1), Theorem 5.1]{Bojko2} by setting 
\[f(t)=1-ze^t,\indent g(t)=\frac{t}{1-e^{-t}}.\]
This gives us
\[\mathcal{N}_S(E,\a;q,z)=\left(\prod_{i=1}^NF(H_i)\right)^{c_1(S)c_1(\a)}\left(\prod_{i=1}^NF(H_i)\right)^{\frac{r}{N}c_1(S)c_1(E)}G(R)^{c_1(S)^2},\]
where 
\[R=f^rg^N,\indent F=\frac{f}{f(0)},\]
the series $H_i(q)$ are Newton–Puiseux solutions to 
\[H_i^N=q R(H_i),\]
and $G(R)$ is given explicitly by \cite[(4.24)]{Bojko2}. Therefore 
\equanum{\label{eqn:general-N}H_{(1),(0),(0)}(q,z)&=\frac{r}{|r|}\sum_{i=1}^N\log F(H_i),\\
H_{(0),(1),(0)}(q,z)&=\log G(R),\indent H_{(0),(0),(1)}(q,z)=\frac{r}{N}\sum_{i=1}^N\log F(H_i).
}
The sign $\frac{r}{|r|}$ in the first line appears as a result of $c_1(-V)=-c_1(V)$.


\begin{comment}For this computation, we require the following Lagrange inversion theorem.

\begin{theorem}[Bojko {\cite[Corollary 2]{bojko-lag}}]\label{thm:lag-inv}
Let $R(t)$ be a power series not involving the variable $q$, and let $H_i(q)$ for $i=1,\dots,N$ be the Newton–Puiseux solutions to 
\[H_i^N=qR(H_i).\]
For any power series $\f(t)$ that does not involve $q$, we have
\[\sum_{i=1}^N\f(H_i)=N\f(0)+\sum_{n>0}\frac1nq^n[t^{nN-1}]\f'(t)R^n(t).\]
\end{theorem}
\end{comment}

\subsubsection{The Chern limit}\label{sec:chern-limit}
We begin with the case $\nu=\xi=(0)$. Apply \cite[(4.17)]{Bojko2}, which gives a term-by-term expression for (\ref{eqn:general-N}), by setting $f(t)=1-ze^t$ and $g(t)=\frac{t}{1-e^{-t}}$. We have
\equanum{\label{eqn:100}H_{(1),(0),(0)}(q,z)&=\frac{r}{|r|}\sum_{n=1}^\infty\frac{1}{n}q^n[t^{nN-1}]\left\{-ze^t(1-ze^t)^{nr-1}\left(\frac{t}{1-e^{-t}}\right)^{nN}\right\}}
By (\ref{eqn:xi}) and (\ref{eqn:mu}), we have for $k_1,k_2\geq 0$ and $k:=k_1+k_2>0$,
\equanum{\label{eqn:H-sum}&k_1!k_2!\sum_{|\mu|=k_1}\sum_{|\xi|=k_2}\binom{|r|}{\mu}\binom{N}{\xi}H_{\mu,(0),\xi}\\
=&|r|D^{k-1}_zH_{(1),(0),(0)}\\
=&r\sum_{n=1}^\infty\frac{1}{n}q^n[t^{nN-1}]\left\{D^{k-1}_z\left(-ze^t(1-ze^t)^{nr-1}\right)\left(\frac{t}{1-e^{-t}}\right)^{nN}\right\}\\
=&r\sum_{n=1}^\infty\frac{1}{n}q^n[t^{nN-1}]\left\{(-1)^k(1-ze^t)^{nr-k}p_{n,k}(ze^t)\left(\frac{t}{1-e^{-t}}\right)^{nN}\right\}
}
where $p_{n,k}$ is a polynomial of degree $k$. We may show inductively that $p_{n,k}(1)=(nr-1)_{(k-1)}$. With this expansion, we would like to apply the Chern limit of Lemma \ref{lem:CV-limit}. In the expansion of universal series (\ref{eqn:log-nek2d-vir}), the series $H_{\mu,(0),(0)}$ are multiplied by a term in $\vec{\lam},\vec{w},\vec{m}$ of homogeneous degree $|\mu|-1=k-1$. When taking the Chern limit, we need to make substitutions 
\[\vec\lam\leadsto-\e\vec\lambda,\indent \vec w\leadsto -\e \vec w,\indent \vec m\leadsto-\e\vec m.\]
Therefore we need to multiply by a factor of $(-\e)^{k-1}$ when taking the limit of $H_{\mu,(0),(0)}$. Furthermore, we substitute $q\leadsto (-1)^Nq\e^{N-r}(1+\e)^r$ and $z\leadsto (1+\e)^{-1}$, and the right-hand side of the above expansion becomes
\[r\sum_{n=1}^\infty\frac{1}{n}q^n[t^{nN-1}]\left\{(-1)^{nN-1}\e^{n(N-r)+k-1}(1+\e-e^t)^{nr-k} (1+\e)^{k}p_{n,k}\left(\frac{e^t}{1+\e}\right)\left(\frac{t}{1-e^{-t}}\right)^{nN}\right\}.\]
Since we are extracting the $[t^{nN-1}]$ coefficient of the function inside the curly bracket, we may replace $t\leadsto -\e t$ and divide the function by $(-\e)^{nN-1}$, which gives us
\equa{r\sum_{n=1}^\infty\frac{1}{n}q^n[t^{nN-1}]\left\{\e^{k-nr}(1+\e-e^{-\e t})^{nr-k} (1+\e)^k p_{n,k}\left(\frac{e^{-\e t}}{1+\e}\right)\cdot \left(\frac{-\e t}{1-e^{\e t}}\right)^{nN}\right\}.
}
Taking $\e\->0$, the term $p_{n,k}\left(\frac{\exp(-\e t)}{1+\e}\right)$ converges to $p_{n,k}(1)=(nr-1)_{(k-1)}$. Thus the limit gives us \equa{r\sum_{n=1}^\infty\frac{1}{n}q^n[t^{nN-1}]\left\{(nr-1)_{(k-1)}(1+t)^{nr-k}\right\}=r\sum_{n=1}^\infty\frac{(nr-1)_{(k-1)}}{n}\binom{nr-k}{nN-1}q^n.
}
We further apply the following identity
\equa{(nr-1)_{(k-1)}\binom{nr-k}{nN-1}=(n(r-N))_{(k-1)}\binom{nr-1}{nN-1}.}
The Chern limit of the left-hand side of (\ref{eqn:H-sum}) will be the Chern series of rank $r$ for $r>0$, and the Segre series of rank $-r$ for $r<0$. Therefore for all $r>0$,
\equanum{\label{eqn:segre-mu} k_1!k_2!\sum_{|\mu|=k_1}\sum_{|\xi|=k_2}\binom{r}{\mu}\binom{N}{\xi}\log A_{\mu,(0),\xi}(q)&=-r\sum_{n=1}^\infty\frac{(-n(r+N))_{(k-1)}}{n}\binom{-nr-1}{nN-1}q^n\\
&=-r\sum_{n=1}^\infty\frac{1}{n}q^n[t^{nN-1}]\left\{(-n(r+N))_{(k-1)}(1+t)^{-nr-1}\right\},\\
}
\equa{k_1!k_2!\sum_{|\mu|=k_1}\sum_{|\xi|=k_2}\binom{r}{\mu}\binom{N}{\xi}\log C_{\mu,(0),\xi}(q)&=r\sum_{n=1}^\infty\frac{(n(r-N))_{(k-1)}}{n}\binom{nr-1}{nN-1}q^n.
}

\subsubsection{The Verlinde limit}\label{sec:verlinde-limit}
We apply a similar argument for the Verlinde limit using (\ref{eqn:general-N}). To simplify computation, we consider a different change of variable. Let
\[\tilde{H}_i=1-e^{-H_i}.\]
Then $\tilde{H}_i$ are the Newton-Puiseux solutions to
\[\tilde{H}_i^N=q\frac{(1-\tilde{H}_i-z)^r}{(1-\tilde{H}_i)^r}.\]
Also,
\[F(H_i)=\frac{1-ze^{H_i}}{1-z}=\frac{1-\tilde{H}_i-z}{(1-\tilde{H}_i)(1-z)},\] 
so
\[H_{(1),(0),(0)}(q,z)=\frac{r}{|r|}\sum_{i=1}^N\log F(H_i)=\frac{r}{|r|}\sum_{i=1}^N\log\frac{1-\tilde{H}_i-z}{(1-\tilde{H}_i)(1-z)}.\]
Apply Lagrange inversion theorem \cite[Corollary 2]{bojko-lag} with $\f(t)=\log((1-t-z)/(1-t)(1-z))$ and $R=(1-t-z)^r/(1-t)^{r}$, we have
\equa{&k_1!k_2!\sum_{|\mu|=k_1}\sum_{|\xi|=k_2}\binom{|r|}{\mu}\binom{N}{\xi}H_{\mu,(0),\xi}\\
=&|r|D^{k-1}_zH_{(1),(0),(0)}\\
=&rD^{k-1}_z\sum_{n=1}^\infty\frac{1}{n}q^n[t^{nN-1}]\left\{\f'(t)R(t)^n\right\}\\
=&r\sum_{n=1}^\infty\frac{1}{n}q^n[t^{nN-1}]\left\{D^{k-1}_z\left(-z(1-t-z)^{nr-1}\right)\left(1-t\right)^{-nr-1}\right\}\\
=&r\sum_{n=1}^\infty\frac{1}{n}q^n[t^{nN-1}]\left\{(-1)^k(1-t-z)^{nr-k}q_{n,k}(z)(1-t)^{-nr-1}\right\}.
}
Here $q_{n,k}(z)$ is a polynomial of degree $k$ in $z$ whose coefficients involve the variable $t$, and its leading coefficient is $(nr)^{k-1}$. To take the Verlinde limit for $r>0$, we substitute 
\[q\leadsto (-1)^rq\e^r,\indent z\leadsto \e^{-1},\] 
then take $\e\->0$ and get
\equa{k_1!k_2!\sum_{|\mu|=k_1}\sum_{|\xi|=k_2}\binom{r}{\mu}\binom{N}{\xi}\log  B_{\mu,(0),\xi}&=r\sum_{n=1}^\infty\frac{1}{n}q^n[t^{nN-1}]\left\{(nr)^{k-1}(1-t)^{-nr-1}\right\}.
}
Observe that the Verlinde series for $\a\in K_\T(S)$ only depends on $c_1(\a)$ by definition, so the universal series are non-trivial only when $\mu=(1)_k:=(1,\dots, 1)$ has $k$ copies of 1. We can therefore simplify the left-hand side of the above equation and get
\equanum{\label{eqn:verlinde-mu}k_1!k_2!r^{k_1}\sum_{|\xi|=k_2}\binom{N}{\xi}\log B_{(1)_{k_1},(0),\xi}(q)&=r\sum_{n=1}^\infty\frac{1}{n}q^n[t^{nN-1}]\left\{(nr)^{k-1}(1-t)^{-nr-1}\right\}\\
&=r^k\sum_{n=1}^\infty (-1)^{nN-1}n^{k-2}\binom{-nr-1}{nN-1}q^n.
}

By \cite[Lemma 3.3]{GM}, the universal series are polynomials in $r$ for $r\geq 0$. Therefore the above results hold for the rank $r=0$ case as well.





\subsubsection{The degree $-1$ case}
When $k=0$, the argument from above still applies, where $D_z^{-1}(-)$ denotes taking the anti-derivative of $\frac1z(-)$ with respect to $z$. However this would result in an undetermined constant term from the integration, so we deal with this case separately. We compute $A_{(0),(0),(0)},B_{(0),(0),(0)},C_{(0),(0),(0)}$ using expressions for $H_{(0),(0),(0)}$, which we shall denote as $A,B,C$ and $H$ respectively. Since these series are associated to the part of their respective invariants independent of the weights of $\a$, we have
\[A^{N,r}_{(0),(0),(0)}=C^{N,-r}_{(0),(0),(0)},\]
and by the second part of \cite[Lemma 3.3]{GM}, $A,B,C,H$ are polynomials with respect to $r$ for all $r\in\ZZ$.

By Lemma \ref{lem:differential}, $D_zH=|r|H_{(1),(0),(0)}$. Taking $D_z^{-1}$ of (\ref{eqn:100}) with respect to $z$, we get
\equa{H(q,z)&=H_0(q)+\sum_{n=1}^\infty\frac{1}{n^2}q^n[t^{nN-1}]\left\{(1-ze^t)^{nr}\left(\frac{t}{1-e^{-t}}\right)^{nN}\right\}\\
&=:H_0(q)+H_1(q,z)}
for some $H_0(q)$ independent of the variable $z$.

Apply the result of Section \ref{sec:chern-limit} with $k=0$. We see that $H_1(q)$ admits the following Chern limit
\[\sum_{n=1}^\infty\frac{1}{n^2}\binom{nr}{nN-1}q^n.\]
Write $H_0(q)=\sum_{n=1}^\infty h_nq^n$, then its Chern limit is
\[\lim_{\e\->0}\sum_{n=1}^\infty h_n(-1)^{nN-1}\e^{n(N-r)-1}(1+\e)^{rn}q^n.\]
When $N-r\leq 0$, we must have $h_n=0$ for all $n$ since otherwise we would have negative powers on $\epsilon$ and the limit does not make sense. As each $h_n$ is polynomial in $r$, we conclude that $H_0(q)=0$. Hence for all $r\in\ZZ$
\[\log A(q)=\sum_{n=1}^\infty\frac{1}{n^2}\binom{-nr}{nN-1}q^n,\]
\[\log C(q)=\sum_{n=1}^\infty\frac{1}{n^2}\binom{nr}{nN-1}q^n.\]
When $0\leq r\leq N-1$, the formula for $C(q)$ is consistent with Conjecture \ref{con:cao-kool-quot} and Conjecture \ref{con:low-rank-vanish}. 

Similarly by Section \ref{sec:verlinde-limit}, the Verlinde limit of $H_{(0),(0),(0)}$ is
\[\log B(q)=\sum_{n=1}^\infty\frac1{n^2} (-1)^{nN-1}\binom{-nr-1}{nN-1}q^n.\]

Combining the results of the above sections, we have the following theorem.






\begin{theorem}\label{thm:SV2d-deg-pos}
For rank $r\geq 0$ and integers $k_1,k_2\geq 0$ with $k:=k_1+k_2$, the universal series of Theorem \ref{thm:SV-univ-sieres} satisfy
\equa{k_1!k_2!\sum_{|\mu|=k_1}\sum_{|\xi|=k_2}\binom{r}{\mu}\binom{N}{\xi}\log A_{\mu,(0),\xi}(q)&=-r\sum_{n=1}^\infty\frac{(-n(r+N))_{(k-1)}}{n}\binom{-nr-1}{nN-1}q^n,\\
k_1!k_2!r^{k_1}\sum_{|\xi|=k_2}\binom{N}{\xi}\log B_{(1)_{k_1},(0),\xi}(q)&=-r^{k}\sum_{n=1}^\infty n^{k-2}\binom{-nr-1}{nN-1}\left((-1)^Nq\right)^n.}
Furthermore, we have
\equa{
&k_1!k_2!\sum_{|\mu|=k_1}\sum_{|\xi|=k_2}\binom{r}{\mu}\binom{N}{\xi}\log C_{\mu,(0),\xi}(q)=r\sum_{n=1}^\infty\frac{(n(r-N))_{(k-1)}}{n}\binom{nr-1}{nN-1}q^n,
}
which can be compared to the identities above by replacing $r$ with $-r$.
\end{theorem}

When $k=2$, we have the following Segre-Verlinde correspondences in degree $1$.
\begin{corollary}\label{cor:sv2d-deg1}
For rank $r\geq 0$, the universal series of Theorem \ref{thm:SV-univ-sieres} satisfy the following correspondences
\equa{&\indent A_{(1,1),(0),(0)}(q)^{-r}A_{(2),(0),(0)}(q)^{\frac{-(r-1)}{2}}=B_{(1,1),(0),(0)}\left((-1)^Nq\right)^{r+N},\\
&\indent A_{(1),(0),(1)}(q)^{-r}=B_{(1),(0),(1)}\left((-1)^Nq\right)^{r+N},\text{ and }\\
&\indent A_{(0),(0),(1,1)}(q)^{-rN}A_{(0),(0),(2)}(q)^{\frac{-r(N-1)}{2}}\\
&=B_{(0),(0),(1,1)}\left((-1)^Nq\right)^{N(r+N)}B_{(0),(0),(2)}\left((-1)^Nq\right)^{\frac{(N-1)(r+N)}{2}}.
}
\end{corollary}

\begin{remark}
As mentioned in the introduction, combining Theorem \ref{thm:SV2d-deg-pos} with Theorem \ref{cor:reduced-series} yields the corresponding relations for reduced invariants. In particular, Corollary \ref{cor:sv2d-deg1} implies a correspondence in degree 0 for reduced invariants, which could provide insight into the reduced invariants for K3 surfaces in the compact setting.
\end{remark}


\subsection{Universal series via Lagrange inversion}\label{sec:closed-form}
Let $r>0$. Consider (\ref{eqn:segre-mu}) and (\ref{eqn:verlinde-mu}) from the previous section. The right-hand sides of these identities are linear combinations of series of form
\[\sum_{n=1}^\infty\frac{1}{n}q^n[t^{nN-1}]\left\{n^{k-1}(1+t)^{-nr}\f'\right\}.\]
for $\f(t)=\log(1+t)$ and $k>0$. The goal of this section is to express these series without the process of extracting coefficients. 

By Lagrange inversion theorem \cite[Corollary 2]{bojko-lag}, we have
\equa{&\sum_{n=1}^\infty\frac{1}{n}q^n[t^{nN-1}]\left\{n^{k-1}(1+t)^{-nr}\f'\right\}\\
=&
D_q^{k-1}\sum_{n=1}^\infty\frac{1}{n}q^n[t^{nN-1}]\left\{(1+t)^{-nr}\f'\right\}\\
=&D_q^{k-1}\sum_{i=1}^N\left(\f(H_i)-\f(0)\right)=D_q^{k-1}\sum_{i=1}^N\f(H_i)
}
where $D_q=q\frac{d}{d q}$ and $H_i$ are the Newton-Puiseux solutions to
\equanum{\label{eqn:newton}H_i^N=q(1+H_i)^{-r}.}
Note that 
\[D_q\f(H_i)=q\frac{d}{dq}\f(H_i)=q\f'(H_i)\cdot\frac{dH_i}{dq}=\f'(H_i)\cdot D_qH_i.\]
Here $H_i(q^{\frac1N})$ is a Puiseux series, and by $dH_i/dq$ we mean $(dq/dH_i)^{-1}$. Differentiating both sides of (\ref{eqn:newton}) with respect to $H_i$ yields
\equa{NH_i^{N-1}\frac{dH_i}{dq}&=(1+H_i)^{-r}-rq(1+H_i)^{-r-1}\frac{dH_i}{dq}\\
NqH_i^{-1}\frac{dH_i}{dq}&=1-rq(1+H_i)^{-1}\frac{dH_i}{dq}\\
D_qH_i&=\frac{1}{NH_i^{-1}+r(1+H_i)^{-1}}
}
Let $\psi(t)=Nt^{-1}+r(1+t)^{-1}$, then $D_qH_i=\frac1{\psi(H_i)}$.
Define $D_\psi=\frac1\psi\cdot \frac{d}{dt}$. We conclude
\[D_q\f(H_i)=(D_\psi \f)(H_i)\]
for arbitrary power series $\f(t)$. Therefore
\equa{\sum_{n=1}^\infty\frac{1}{n}q^n[t^{nN-1}]\left\{n^{k-1}(1+t)^{-nr}\f'\right\}=\sum_{i=1}^N(D_\psi^{k-1}\f)(H_i).}
The following corollary follows directly by applying this to (\ref{eqn:segre-mu}) and (\ref{eqn:verlinde-mu}).

\begin{theorem}\label{cor:SV2d-closed}
Let $\f(t)=\log(1+t)$ and $\psi(t)=Nt^{-1}+r(1+t)^{-1}$. Define the differential operator
\[ D_\psi=\frac{1}{\psi}\cdot \frac{d}{dt}.\]
Furthermore, use the notation
\[D_{\mathcal{S}}^{(k)}=(-(r+N)D_\psi)_{(k-1)},\indent D_{\mathcal{V}}^{(k)}=r^{k-1}D_\psi^{k-1}\]
for $k\geq 0$ where $D_\psi^{-1}(-)$ denotes integrating $\psi\cdot (-)$ assuming a constant term 0. In the setting of Theorem \ref{thm:SV2d-deg-pos}, we have the following relations
\equa{
k_1!k_2!\sum_{|\mu|=k_1}\sum_{|\xi|=k_2}\binom{r}{\mu}\binom{N}{\xi}\log A_{\mu,(0),\xi}(q)&=-r\sum_{i=1}^N\left(D_{\mathcal{S}}^{(k)}\f\right)(H_i),\\
k_1!k_2!r^{k_1}\sum_{|\xi|=k_2}\binom{N}{\xi}\log B_{(1)_{k_1},(0),\xi}\left((-1)^Nq\right)&=-r\sum_{i=1}^N\left(D_{\mathcal{V}}^{(k)}\f\right)(H_i)}
where $H_i$ are the Newton-Puiseux solutions to $H_i^N=q(1+H_i)^{-r}$.
\end{theorem}


\begin{example}
We shall compute the $\log B_{(1,1,1),(0),(0)}$ using the above theorem. Set $k_1=3,k_2=0$. We have
\[D_F^{2}\f=\frac{N t \left(1+t\right)}{\left(\left(N+r\right) t+N\right)^{3}}.\]
Hence
\equa{\log B_{(1,1,1),(0),(0)}((-1)^Nq)&=\frac{-1}{3!}\sum_{n=1}^\infty \left(D^2_F\f\right)(H_i)\\
&=\frac{-N}{6}\sum_{i=1}^N\frac{ H_i \left(1+H_i\right)}{\left(\left(N+r\right) H_i+N\right)^{3}}.
}
\end{example}


\begin{example}
Similarly, we may compute $\sum_{|\mu|=3}\binom{r}{\mu}\log A_{\mu,(0),(0)}$.
\equa{&\sum_{|\mu|=3}\binom{r}{\mu}\log A_{\mu,(0),(0)}\\
=&\frac{-r}{3!}\sum_{n=1}^\infty\left(\left((r+N)^2 D_F^2+(r+N)D_F\right)\f\right) (H_i)\\
=&\frac{-r(r+N)}{6}\sum_{i=1}^N\frac{H_i\left((N+r)^2H_i^2+3N(N+r)H_i+N(2N+r)\right)}{\left((N+r)H_i+N\right)^{3}}.
}
Again, $H_i$ are the Newton-Puiseux solutions to $H_i^N=q(1+H_i)^{-r}$.
\end{example}











































