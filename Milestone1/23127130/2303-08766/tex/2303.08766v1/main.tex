\pdfoutput=1
\documentclass[a4paper,11pt]{article}
\usepackage{jcappub} % for details on the use of the package, please see the JINST-author-manual
\usepackage{lineno}
%\linenumbers
\usepackage{caption}
\usepackage{subcaption}
\usepackage[numbers]{natbib}

%\arxivnumber{1234.56789} % Only if you have one
\title{ Modification of the Dipole in Arrival Directions of Ultra-high-energy Cosmic Rays due to the  Galactic Magnetic Field}

% Collaborations

%% [A] If main author
%% \collaboration{\includegraphics[height=17mm]{collabroation-logo}\\[6pt]
%%  XXX collaboration}

%% or
%% [B] If "on behalf of"
%% \collaboration[c]{on behalf of XXX collaboration}


% Authors
% The "\note" macro will give a warning: "Ignoring empty anchor...", you can safely ignore it.

%% [A] simple case: 2 authors, same institution
%% \author[1]{A. Uthor\note{Corresponding author.}}
%% \author{and A. Nother Author}
%% \affiliation{Institution,\\Address, Country}

%% or, e.g.
%% [B] more complex case: 4 authors, 3 institutions, 2 footnotes
%% \author[a,b]{F. Irst,\}
%% \author[c]{S. Econd,}
%% \author[a,1]{T. Hird\note{Also at Some University.}}
%% \author[c,1]{and Fourth}
%% \affiliation[a]{Institution_1,\\Address, Country}
%% \affiliation[b]{Institution_2,\\Address, Country}
%% \affiliation[c]{Institution_3,\\Address, Country}

\author[1]{A. Bakalová,}
\author[1]{J. Vícha,}
\author[1]{P. Trávníček}
\affiliation[1]{FZU - Institute of Physics of the Czech Academy of Sciences,\\
Na Slovance 1999/2, 182 21 Prague 8, Czech Republic}

% E-mail addresses: only for the corresponding author
\emailAdd{bakalova@fzu.cz}

\abstract{The direction and magnitude of the dipole anisotropy of ultra-high-energy cosmic rays with energies above 8 EeV observed by the Pierre Auger Observatory indicate their extragalactic origin. The observed dipole on Earth does not necessarily need to correspond to the anisotropy of the extragalactic cosmic-ray flux due to the effects of propagation in the Galactic magnetic field. We estimate the size of these effects via numerical simulations using the CRPropa 3 package.  The Jansson--Farrar and Terral--Ferrière models of the Galactic magnetic field are used to propagate particles from the edge of the Galaxy to an observer on Earth. We identify allowed directions and amplitudes of the dipole outside the Galaxy that are compatible with the measured features of the dipole on Earth for various mass composition scenarios at the 68\% and 95\% confidence level.}







\begin{document}
\maketitle
\flushbottom

\section{Introduction}
\label{sec:intro}

Our knowledge about ultra-high-energy cosmic rays (above $10^{18}$~eV, UHECRs) grew significantly in the last two decades. Large-area observatories, like the Pierre Auger Observatory \cite{PAO} in the Southern hemisphere and Telescope Array \cite{TA} in the Northern hemisphere measure cosmic rays at ultra-high energies with an unprecedented precision, opening possibilities to study the properties of UHECRs in more detail than ever before. 

Even though the existence of UHECRs was confirmed more than 50 years ago, the sources of these particles and the mechanisms of their acceleration are still unknown and are a subject of abounding studies. Currently, there are strong pieces of evidence suggesting an extragalactic origin of these particles. 
Searches for correlation of arrival directions of UHECRs with astrophysical sources such as active galactic nuclei or starbursts galaxies are being performed \cite{AnisotropyPAO2022, AnisotropyTA2018} with results suggesting an anisotropic distribution of these particles above $\approx 40$~EeV. An argument supporting the extragalactic origin of UHECRs is the fact that no significant anisotropies have been observed at higher energies around the Galactic plane or the Galactic center \cite{Auger2015AD}.
Nonetheless, the true sources are still unknown. The hypothesis about an extragalactic origin of UHECRs is strongly supported by a measurement of dipole anisotropy in the arrival directions of cosmic rays with energies above 8~EeV by the Pierre Auger Observatory \cite{AugerDipole2017, AugerDipole2018}. These measurements show that the direction of the dipole anisotropy points $\approx 125^{\circ}$ away from the Galactic center with uncertainty of about $15^{\circ}$,  which is much more than deflections expected by current models of Galactic magnetic field. The amplitude of the measured dipole in the arrival directions of cosmic rays above 8 EeV is $\approx6.5~\%$ with more than $6\sigma$ significance. At lower energies, in the energy bin from $4$~EeV up to $8$~EeV, the reconstructed dipole has an amplitude of $\approx2.5~\%$ with significance smaller than $3\sigma$ \cite{AugerDipole2018}.

Trajectories of UHECRs are influenced by magnetic fields in the Universe on their long journeys from their sources to Earth. Deflections of cosmic rays in magnetic fields depend on the charge $Z$ and energy $E$ of the particle and the specifics of the magnetic fields they are crossing. The Galactic magnetic field (GMF) is not well understood. However, multiple models of the field have been proposed, for example see \cite{JF12, TF17, PT11}, based on available observations of synchrotron radiation or Faraday rotation measurements. As the GMF is considered to be much stronger than the extragalactic magnetic fields in the Universe \cite{Widrow2002}, it plays a major role in the particle deflection from its source direction. The presence  of the GMF is not only causing deflections at the single-particle level but, as a consequence, it can have a significant influence on the observed large-scale anisotropies as well, like the observed direction of the dipole in arrival directions. The Galactic magnetic field can also change amplitudes of anisotropies and smear them as the field tends to isotropise particle trajectories, especially at lower rigidities ($R=E/Z$). These effects were shown for example in \cite{EichmanmDipole}.

In this work, we investigate the influence of the GMF on the large-scale dipole anisotropy in arrival directions of ultra-high-energy cosmic rays above $8$ EeV employing numerical simulations of cosmic-ray propagation in the GMF using the CRPropa 3 software \cite{CRPropa3}. We assume a dipole anisotropy in the arrival directions of extragalactic cosmic rays at the edge of the Galaxy. We investigate the shift of the dipole direction and the change of the dipole amplitude due to the deflections in the GMF. Specifically, we are searching for dipoles in distributions of extragalactic cosmic rays that are, after propagation in the GMF, compatible with the dipole measured on Earth by the Pierre Auger Observatory. We study two models of the GMF and different mass-composition scenarios of the overall cosmic ray flux.

The paper is structured as follows: We describe the used simulation settings of the cosmic-ray propagation in the GMF and the method of dipole reconstruction in Section~\ref{sec:sims}. The influence of the GMF on the dipole properties in general is described in Section~\ref{sec:GMFinfluence}. In Section~\ref{sec:results}, we derive the sky regions for allowed directions of the extragalactic dipole that are compatible with the measured dipole properties on Earth. We discuss and summarise our findings in Sections~\ref{sec:dis} and \ref{sec:conc}, respectively.

\section{Simulations and Method }
\label{sec:sims}

\subsection{Simulating Cosmic-ray Propagation}

For the study of the influence of the GMF on the dipole anisotropy in arrival directions, we simulate cosmic-ray propagation from the edge of the Galaxy to Earth using the software package CRPropa~3 (version 3.1.6). The particles are emitted isotropically from a shell with a radius 20~kpc around the Galactic center (GC) in coordinates $(x,y,z)=(0,0,0)$~kpc. The isotropic flux is achieved by selecting a random point on the sphere of radius 20~kpc and appointing an isotropic Lambert distributed directions in zenith and azimuth angle. Such a distribution is implemented within the CRPropa 3 package as a class \textit{SourceLambertDistributionOnSphere}.
We collect particles reaching an observer with a radius of 100 pc in coordinates $(x,y,z)=(-8.5,0,0)$~kpc corresponding approximately to the location of the Solar system in the Milky Way (see Figure~\ref{fig:sim}).

\begin{figure}[htp]
     \centering
         \includegraphics[width=0.5\textwidth]{figures/simIllustration2.png}
        \caption{An illustration of the simulation setup. Particles are emitted isotropically from a shell with radius $20\,\rm{kpc}$ and propagated in the GMF. We collect particles hitting an observer with radius $100\,\rm{pc}$ placed in coordinates $(-8.5,0,0)$ kpc. The center of the coordinates system in the simulation corresponds to the Galactic center (GC).}
        \label{fig:sim}
\end{figure}

Propagation of four types of particles is simulated; protons ($^1$H), helium ($^4$He), nitrogen ($^{14}$N), and iron ($^{56}$Fe) nuclei. Each element is simulated with a power-law energy spectrum with spectral index $\gamma=3$, which is close to the observed spectral index of cosmic rays above $8\,\rm{EeV}$ \cite{PAOenergyspectrum, TAenergyspectrum}. The energy range of simulated particles goes from $8$~EeV up to $100$~EeV.

The energy losses on photon backgrounds are neglected since the energy-loss lengths are much larger than the distance traveled by a particle in Galaxy \cite{Batista_2015}, even when accounting for deflections in the GMF. Each simulation starts with $\approx 10^{10}$ injected particles. However, the number of particles reaching the observer is much smaller; for each of the cosmic-ray species the final amount of particles collected on the observer is about $250{,}000$. We set two rejection criteria for the particles to be excluded from the simulation. Firstly, a particle is rejected if its position is further than $20.5\,\rm{kpc}$ from the GC (it will not reach the observer). Secondly, the particle simulation is aborted and not considered if it travelled more than $500\,\rm{kpc}$. The second case might happen for low rigidity particles getting confined in the GMF. The vast majority of the removed particles is due to the first aforementioned criterion.  

We use the Jansson--Farrar model of  the GMF (JF12) \cite{JF12} and the Terral--Ferrière model of the GMF (TF17) \cite{TF17}. In the case of JF12 model, we use regular, striated, and turbulent part of the GMF. The TF17 model has multiple options for the disk field and the halo field. We use the Bd1 model of the disk field together with the bisymmetric halo field C1, as this combination is one of the two best fitted modes of this GMF model \cite{TF17}. Visualization of the field strength in XY and XZ plane for both models taken from the CRPropa~3 software is depicted in Figure~\ref{fig:GMF}.

\begin{figure}[htp]

 \centering
     \begin{subfigure}{0.49\textwidth}
         \centering
          \includegraphics[width=1.0\textwidth]{figures/JF12_XY_new.pdf}
         
         \label{fig:JF12XY}
     \end{subfigure}
     \hfill
     \begin{subfigure}{0.49\textwidth}
         \centering
        \includegraphics[width=1.0\textwidth]{figures/TF17_XY_new.pdf}   
        
         \label{fig:TF17XY}
     \end{subfigure}
  \begin{subfigure}{0.49\textwidth}
         \centering
          \includegraphics[width=1.0\textwidth]{figures/JF12_XZ_new.pdf}
         
         \label{fig:JF12XZ}
     \end{subfigure}
     \hfill
     \begin{subfigure}{0.49\textwidth}
         \centering
        \includegraphics[width=1.0\textwidth]{figures/TF17_XZ_new.pdf}   
        
         \label{fig:TF17XZ}
     \end{subfigure}

        \caption{Strength of the GMF field in XY plane (top) and XZ plane (bottom) for the JF12 model (left) and TF17 model of the GMF (right) with the Galactic center in the coordinates $(x,y,z)=(0,0,0)$~kpc. Note the different scales of the magnetic field strength for the two models by a factor of $\approx 2$.}
        \label{fig:GMF}
\end{figure}

Additionally, a set of simulations is generated for energy range from $4$~EeV to $8$~EeV for all four types of particles with the same setting as for the higher energies in order to investigate the behavior of the dipole at lower energies.



\subsection{Dipole Reconstruction from Simulated Data}

Information about each particle at the injection point (the edge of the Galaxy) and at the observer is kept for particles reaching the observer including their energy, unit momentum vector, and Cartesian coordinates. The isotropic flux of the simulated particles is reweighted according to the initial travel direction of cosmic rays at the edge of the Galaxy in order to obtain a dipole distribution of the particles arriving into Galaxy. A weight $w$ is assigned to every particle reaching the observer according to 
\begin{equation}
w=A_0\,\cos\delta + 1,   
\end{equation}
where $A_0$ is the extragalactic amplitude of the dipole and $\delta$ is the angular distance of the direction of the initial momentum of
the particle and that of the injected extragalactic dipole. We investigate the initial amplitude of the dipole $A_0$ in the range from $6.5$\% up to $20$\% in discrete steps of 2\% (the first step is 1.5 from amplitude 6.5\% to 8\%).  The amplitude $A_0$ of the extragalactic dipole should be equal or higher than the one observed on Earth due to the effect of isotropisation of the cosmic-ray flux in the GMF.  These amplitudes are injected into the particle flux in all possible extragalactic directions of the dipole with a step of $1^{\circ}$ in longitude and $1^{\circ}$ in latitude.



Characteristics of the three-dimensional dipole on the observer level are reconstructed using a procedure from \cite{3Ddipole}. To calculate the direction {$\textbf{D}_{\rm{obs}}$} and amplitude $A_{\rm{obs}}$ of the dipole on the observer we need to estimate discrete version of the zeroth and first moments of the flux on the observer as

\begin{equation}
    S_0=\sum_{k}\frac{1}{w_k}
\,\,\,\,\,\,\,\,\rm{and}\,\,\,\,\,\,\,\, 
    \textbf{S}=\sum_{k}\frac{\textbf{u}_k}{w_k},
\end{equation}
where the sums go over all particles $k$ reaching the observer with weight $w_k$. {$\textbf{u}_k$} is their arrival direction to the observer. Using these we can calculate the amplitude $A_{\rm{obs}}$ and direction ${\textbf{D}_{\rm{obs}}}$ of the dipole on the observer as
\begin{equation}
    A_{\rm{obs}}=3\frac{\left\| \textbf{S} \right\|}{S_0} \,\,\,\,\,\,\,\,\rm{and}\,\,\,\,\,\,\,\, {\textbf{D}_{\rm{obs}}}=\frac{\textbf{S}}{\left\| \textbf{S} \right\|}.
\end{equation}


We are identifying solutions with extragalactic amplitudes of the dipole $A_0$ and extragalactic directions of the dipole $(l_0, b_0)$, where $l_0$ is the galactic longitude and $b_0$  galactic latitude, that are compatible with the measurement by the Pierre Auger Observatory \cite{AugerDipole2018} within $1\sigma$ and $2\sigma$ in both direction and amplitude, i.e. an amplitude of $6.5^{+1.3}_{-0.9}$\% and direction with right ascension $\alpha_{\rm{d}}=(100\pm10)^{\circ} $ and declination $\delta_{\rm{d}}=(-24^{+12}_{-13})^{\circ}$.

%%%%%%%%%%%%%%%%%%%%%%%%%%%%%%%%%%%%%%%%%%%%%%%%%%%%%%%%%%%%%%%%%%%%%%%%%%%
%%%%%%%%%%%%%%%%%%            RESULTS             %%%%%%%%%%%%%%%%%%%%%%%%%
%%%%%%%%%%%%%%%%%%%%%%%%%%%%%%%%%%%%%%%%%%%%%%%%%%%%%%%%%%%%%%%%%%%%%%%%%%%

\section{Influence of the GMF on the Dipole Properties}
\label{sec:GMFinfluence}
\subsection{Dipole Amplitude}
As a result of diffusive propagation of cosmic rays in the GMF, the amplitude of the dipole on the observer $A_{\rm{obs}}$ (Earth) is lower than the dipole amplitude outside of the Galaxy ($A_0$). The level of the decrease of the amplitude depends strongly on the particle rigidity. Particles with lower rigidities are more deflected in the GMF during their travel and the final flux tends to isotropise. The predictions of the decrease of amplitude differ for individual models of GMF. 

\begin{figure}[htp]
     \centering
         \includegraphics[width=0.6\textwidth]{figures/amplitude_new.pdf}
        \caption{Reconstructed dipole amplitude on the observer for different types of particles of energies above 8~EeV with mass A propagated in TF17 (green) and JF12 (blue) model of the GMF. The injected dipole amplitude ($A_{0}$) is 10\% for two extragalactic directions of the dipole; 2MRS in coordinates $(l_0, b_0)=(251^{\circ}, 37^{\circ})$ and $(l_0, b_0)=(40^{\circ},0^{\circ})$.}
        \label{fig:amplitudeChange}
\end{figure}

Not all the injected extragalactic directions of the dipole lead to the same decrease of the amplitude within given GMF model. The change of the amplitude for different cosmic-ray species and two different directions of the dipole are depicted in Figure~\ref{fig:amplitudeChange} for cosmic rays propagated in JF12 and TF17 models of the GMF. The injected extragalactic amplitude is $A_{0}=10$\% in both cases. One of the injected directions corresponds to the 2MRS dipole direction \cite{2MRS} and the other direction, $(l_0, b_0)=(40^{\circ}, 0^{\circ})$, which is chosen in order to demonstrate the extreme case of isotropisation of the flux at lower rigidities and does not correspond to any significant astrophysical system. In the latter case, for the iron-nuclei scenario, the amplitude on the observer decreases to $\approx 1\%$ and the distribution of arrival directions on the observer is compatible with isotropy for both JF12 and TF17 model of the GMF. 


\begin{figure}
     \centering
     \begin{subfigure}{0.49\textwidth}
         \centering
          \includegraphics[width=1.0\textwidth]{figures/amplitudeJF12p.pdf}
         
         \label{fig:amplitudeH_JF12}
     \end{subfigure}
     \hfill
     \begin{subfigure}{0.49\textwidth}
         \centering
        \includegraphics[width=1.0\textwidth]{figures/amplitudeTF17p.pdf}   
        
         \label{fig:amplitudeH_TF17}
     \end{subfigure}
        \caption{Sky map in galactic coordinates of extragalactic dipole directions with initial amplitude of $A_0=10\%$ for protons of energies above 8~EeV. The color scale corresponds to the amplitude of the dipole on the observer after propagation in the JF12 (left) and TF17 (right) model of the GMF.}
        \label{fig:amplitudeH}
\end{figure}



%\begin{figure}[htp]
%     \centering
%         \includegraphics[width=0.5\textwidth]{figures/amplitudeJF12p.pdf}
%        \caption{Sky map in galactic coordinates of extragalactic dipole directions with initial amplitude of $A_0=10\%$ for protons of energies above 8~EeV. The color scale corresponds to the amplitude of the dipole on the observer after propagation in the JF12 model of the GMF.}
%        \label{fig:amplitudeH_JF12}
%\end{figure}

%\begin{figure}[htp]
%     \centering
%         \includegraphics[width=0.5\textwidth]{figures/amplitudeTF17p.pdf}
%        \caption{Same as in Figure~\ref{fig:amplitudeH_JF12}, but using the TF17 model of the GMF.}.
%        \label{fig:amplitudeH_TF17}
%\end{figure}

\begin{figure}
     \centering
     \begin{subfigure}{0.49\textwidth}
         \centering
          \includegraphics[width=1.0\textwidth]{figures/JF12shift_p.pdf}
         
         \label{fig:shiftH_JF12}
     \end{subfigure}
     \hfill
     \begin{subfigure}{0.49\textwidth}
         \centering
        \includegraphics[width=1.0\textwidth]{figures/TF17shift_p.pdf}   
        
         \label{fig:shiftH_TF17}
     \end{subfigure}
  \begin{subfigure}{0.49\textwidth}
         \centering
          \includegraphics[width=1.0\textwidth]{figures/JF12shift_Fe.pdf}
         
         \label{fig:shiftFe_JF12}
     \end{subfigure}
     \hfill
     \begin{subfigure}{0.49\textwidth}
         \centering
        \includegraphics[width=1.0\textwidth]{figures/TF17shift_Fe.pdf}   
        
         \label{fig:shiftFe_TF17}
     \end{subfigure}
     
        \caption{The shift of the dipole direction for different extragalactic directions of the dipole for protons (top) and iron nuclei (bottom) of energies above 8~EeV using JF12 model (left) and TF17 model of the GMF (right). The sky maps are in galactic coordinates.}
        \label{fig:dipoleShift}
\end{figure}


%\begin{figure}[htp]
%     \centering
%         \includegraphics[width=\textwidth]{figures/DipoleShift.png}
%        \caption{The shift of the dipole direction for different extragalactic directions of the dipole for protons (top) and iron nuclei (bottom) of energies above 8~EeV using JF12 model (left) and TF17 model of the GMF (right). The sky maps are in galactic coordinates.}
%        \label{fig:dipoleShift}
%\end{figure}

A sky map of the amplitude on the observer for all possible extragalactic directions of the dipole for protons propagated in JF12 and TF17 model of the GMF is shown in the left and right panels of Figure~\ref{fig:amplitudeH}. Concerning the TF17 model, apart from regions towards and opposite from the Galactic center the suppression of the amplitude is minimal. For particles propagated in the JF12 model, a minimal suppression is found in two extended lobes around longitudes of $\sim\pm90^{\circ}$. The suppression of the dipole amplitude is generally stronger in the JF12 model. In the case of pure protons, the initial amplitude of the extragalactic dipole is suppressed by a maximum of relatively $\approx 40\%$ depending on the direction of the extragalactic dipole and the model of the GMF.


\subsection{Dipole Direction}

As well as the amplitude of the dipole is reduced after propagation in the GMF, the direction of the dipole can change too. Similarly to the previous case, the shift of the dipole direction depends on the model of the GMF used, rigidity of the particles and the extragalactic direction of the dipole. The shift can be in orders of few degrees, nevertheless, it can be as high as $\approx180^{\circ}$ in the case of heavy particles/low rigidities. 



To illustrate the possible shifts of the dipole direction, we show a sky map in galactic coordinates of the angular distance between the injected extragalactic direction of the dipole and the observed direction of the dipole on Earth in Figure~\ref{fig:dipoleShift}. The figure shows the dipole shift for protons and iron nuclei propagated in JF12 and TF17 models of the GMF. Note the different color scales that are used in order to better visualize the shift for individual particles and the two models of GMF. In the case of protons, the maximum angular distance between the dipole direction outside the Galaxy and at the observer is found to be $\approx20^{\circ}$ using JF12 model and $\approx 5^{\circ}$ for the TF17 model. For light particles at energies above $8$~EeV the shift is not that significant. However, taking iron nuclei propagated in JF12 and TF17 models, in some directions of the extragalactic dipole the shift of the dipole can be much larger, going up to the aforementioned $180^{\circ}$.

Depending on the direction of the dipole on Earth and mass composition of the cosmic rays, the true extragalactic direction of the dipole might be located close or very far from the direction at the observer.

\section{Compatibility with Observed Dipole Anisotropy}
\label{sec:results}

In order to find possible parameters of the extragalactic dipole that is observed on Earth by the Pierre Auger Observatory above $8$~EeV, we take into account both the amplitude and direction of the measured dipole. As the mass composition of cosmic rays is not known in large detail, we scan over all possible mass composition mixes with a step of $5\%$ in relative fractions of four types of particles and single-component scenarios of p, He, N and Fe. Parameters of the injected amplitude and directions of the extragalactic dipole for a given composition mix are selected as solutions if the reconstructed amplitude as well as the direction of the dipole on the observer are compatible with the measurement by the Pierre Auger Observatory within $1\sigma$ and $2\sigma$. In all following cases, the dipole character of the anisotropy on the observer is checked by a goodness-of-fit with a dipole function in the right ascension. The distribution of the $\chi^2/ndf$ values for the dipole fit has a mean of 1.2 with a standard deviation of 0.5, while the isotropy fit has a mean value of $\chi^2/ndf$=43 with a standard deviation of 10. We also fit the distribution of arrival directions in the right ascension by first and second harmonic and the results show that the amplitude of the second harmonic is an order of magnitude lower than the first harmonic and is statistically insignificant (under $3\sigma$). 


\subsection{Single-Element Scenario}

\begin{figure*}[htp]
     \centering
         \includegraphics[width=0.78\textwidth]{figures/SolutionsH_new.pdf}
        \caption{The directions of the extragalactic dipole compatible with measured direction and amplitude on Earth at $1\sigma$ and $2\sigma$ level found for JF12 and TF17 models of the GMF for the pure-proton scenario. The results from the Pierre Auger Observatory are indicated in red for 1$\sigma$ c.l. The plot is in Galactic coordinates.}
        \label{fig:Hsingle}
\end{figure*}

\begin{figure*}[htp]
     \centering
         \includegraphics[width=0.78\textwidth]{figures/SolutionsHe_new.pdf}
        \caption{Same as in Figure~\ref{fig:Hsingle}, but for pure-helium scenario.}
        \label{fig:Hesingle}
\end{figure*}

\begin{figure*}[hbt!]
     \centering
         \includegraphics[width=0.78\textwidth]{figures/SolutionsN_new.pdf}
        \caption{Same as in Figure~\ref{fig:Hsingle}, but for pure-nitrogen scenario. No solutions are found for the TF17 model of the GMF at the $1\sigma$ level.}
        \label{fig:Nsingle}
\end{figure*}

\begin{figure*}[hbt!]
     \centering
         \includegraphics[width=0.78\textwidth]{figures/SolutionsFe_new.pdf}
        \caption{Same as in Figure~\ref{fig:Hsingle}, but for pure-iron scenario. No solutions are found for comparison at the $1\sigma$ level.}
        \label{fig:Fesingle}
\end{figure*}

Before we inspect the possible extragalactic directions of the dipole for all the composition mixes, we start with results for the simple case of a single-element scenario. As the simulated energy spectrum is the same for all four types of particles, the influence of the GMF becomes more significant with higher atomic number as the rigidity is decreasing.


We find that at the $1\sigma$ level most of the possible directions of the extragalactic dipole can be found in the case of light elements; proton and helium, with an extragalactic amplitude lower than $10\%$ for both models of the GMF. The sky map of directions of the extragalactic dipole that are compatible with the measurements within 1$\sigma$ and $2\sigma$ is shown in Figure~\ref{fig:Hsingle} and Figure~\ref{fig:Hesingle} for pure protons and pure helium nuclei, respectively. All studied initial amplitudes of the extragalactic dipole are included. 
%All the possible extragalactic directions of dipole are intercepting or in a close vicinity from the  $1\sigma$ confidence region of the dipole direction measured on Earth by the Pierre Auger Observatory. 
In the case of comparison on the $1\sigma$ level, for pure-proton scenario, the most distant solution is within $\approx 20^{\circ}$ from the measured direction of the dipole on Earth for both models of the GMF. In the case of pure-helium, the maximum angular distance of a possible extragalactic direction from the measured direction is $\approx35^{\circ}$ and $\approx25^{\circ}$ for the JF12 and TF17 model, respectively. 

In the case of pure nitrogen nuclei, no possible directions of the extragalactic dipole are found for particles propagated using the TF17 model at the $1\sigma$ level. Nevertheless, few possible directions of the extragalactic dipole are found using the JF12 model with initial amplitude $A_0\geq12\%$. These directions are depicted in Figure~\ref{fig:Nsingle} together with areas of allowed directions of the extragalactic dipole at the $2\sigma$ level. All the directions (JF12 model at $1\sigma$) are located $\approx 70^{\circ}-110^{\circ}$ from the measured direction of the dipole on Earth. The allowed directions of the extragalactic dipole are concentrated in five distinguishable lines. These shapes are caused by the discrete space of investigated initial amplitudes of the dipole as each of the lines corresponds to the same initial amplitude of the dipole going from $12\%$ up to $20\%$ from right to left.

There are no allowed extragalactic directions of the dipole found for pure iron nuclei using neither JF12 or TF17 model at the $1\sigma$ level. The rigidity for iron nuclei is too low and the GMF isotropises the flux of cosmic rays extensively. Even with an initial amplitude as high as $A_0=20\%$ outside the Galaxy, the amplitude on the observer is lower than the one observed on Earth. However, at the $2\sigma$ level we find possible directions of the extragalactic dipole for both models of the GMF that are depicted in Figure~\ref{fig:Fesingle}. Interestingly, these directions are close to the Centaurus~A region $(l,b)=(309.5^{\circ}, 19.4^{\circ})$ where hints for anisotropies have been found \cite{CenA}.


\begin{figure*}[hbt!]
     \centering
         \includegraphics[width=\textwidth]{figures/allSolutions_2sigmacont.pdf}
        \caption{The directions of the extragalactic dipole in Galactic coordinates found for all different mass composition scenarios for the JF12 and TF17 models of GMF within $1\sigma$. Areas of possible directions of the extragalactic dipole compatible with the measurements within 2$\sigma$ are shown by blue and green lines for JF12 and TF17 models, respectively. The 1$\sigma$ contour of the dipole measured by the Pierre Auger Observatory above 8 EeV is shown in red and direction of the 2MRS dipole is displayed with a black triangle marker.}
        \label{fig:mixAll}
\end{figure*}


\subsection{Mixed Composition Scenario}
The experimental results regarding the mass composition of cosmic rays above $8$~EeV by the Pierre Auger Observatory suggest that the mass composition is mixed with the mean $\ln A$ evolving with the energy \cite{ICRC19_YushkovAuger, AugerXmax2014, AugerAnkle2016}. We assume that the abundance of individual elements remains the same in the whole energy range (due to the steeply falling energy spectrum, the dominant contribution to anisotropy comes from the energies just above 8~EeV). We scan through all possible mixes of the four elements with the step in relative fractions of 5\%.



The possible extragalactic directions of the dipole for all composition mixes, including the single-element scenarios, are shown in Figure~\ref{fig:mixAll} for both models of the GMF. Different behavior of the JF12 and TF17 models of the GMF can be seen in this figure. At the $1\sigma$ level, vast majority of the possible extragalactic directions are in a close vicinity around the $1\sigma$ confidence region of the measured dipole. While the JF12 model of the GMF allows most of the extragalactic directions in an extended area around the measured dipole direction (within $\approx40^{\circ}$), the TF17 model rather suggests directions extending to smaller longitudes than the measured dipole direction and a smaller spread in latitude. In the case of the TF17 model, the possible extragalactic directions are within $\approx55^{\circ}$ from the measured direction of the dipole on Earth. However, the areas of allowed directions of the extragalactic dipole at the $2\sigma$ level extend much further for both models of the GMF containing regions like the 2MRS dipole or Centaurus~A region. 



The normalized number of allowed directions of the extragalactic dipole in the $1^{\circ}$ by $1^{\circ}$ grid in longitude and latitude with respect to the mean (variance of) $\ln A$ for different initial amplitudes at the $1\sigma$ level is shown on the left (right) panels in Figure~\ref{fig:mixAll_JF12} and Figure~\ref{fig:mixAll_TF17} for JF12 and TF17 models of the GMF, respectively. 
As expected, with a higher initial amplitude of the dipole outside the Galaxy a heavier composition is required to describe the data well. The heavy elements are needed in order to sufficiently suppress the amplitude of the dipole during propagation in the Galaxy for these high initial amplitudes outside the Galaxy. 




\section{Discussion}
\label{sec:dis}

\textbf{General discussion:} The results show that the direction of the dipole we observe on Earth might be different than the direction of this dipole when cosmic rays are entering the Galaxy. These results are under the assumption of a dipole flux of extragalactic cosmic rays to the Galaxy. Different initial amplitudes $A_0$ of the dipole outside the Galaxy are allowed depending on the mean $\ln A$ of the cosmic-ray flux. Here, we investigate amplitudes from $6.5~\%$ up to $20~\%$ with discrete steps. Higher initial amplitudes of the dipole are not investigated as such high anisotropy in the cosmic-ray flux above 8~EeV would be difficult to explain by distribution of specific astrophysical sources.

\begin{figure*}[htp]
     \centering
         \includegraphics[width=1.0\textwidth]{figures/JF12_solutionsVSlnA_new2.pdf}
        \caption{Number of allowed extragalactic directions of the dipole with respect to mean $\ln A$ (left) and with respect to the variance of the $\ln A$ (right) for different initial amplitudes for JF12 model of the GMF. Only directions with $1\sigma$ agreement with the measurement are shown. }
        \label{fig:mixAll_JF12}
\end{figure*}


\begin{figure*}[hbt!]
     \centering
         \includegraphics[width=1.0\textwidth]{figures/TF17_solutionsVSlnA_new2.pdf}
        \caption{Same as in Figure~\ref{fig:mixAll_JF12}, but using the TF17 model of the GMF.}
        \label{fig:mixAll_TF17}
\end{figure*}

The allowed directions of the extragalactic dipole of cosmic rays above 8 EeV that are presented in the previous section at the $1\sigma$ level mostly suggest directions not far from the measured dipole direction on Earth showing rather smaller angular shifts due to the GMF. Even though identification of sources creating this anisotropy is not the scope of this paper, we would like to mention that close to the allowed directions of the extragalactic dipole presented in the previous section is the direction of the 2MRS dipole \cite{2MRS} (in the case of JF12 model the direction is in the area of allowed directions at $2\sigma$ level). Furthermore, at the $2\sigma$ level, the Centaurus~A region also lies within the region of allowed directions of the extragalactic dipole. Additional deflections of the dipole directions might be caused by the influence of the extragalactic magnetic fields which are not applied in this work. The origin of large-scale anisotropies from the source distribution following a large-scale structure was investigated in \cite{LSSdipole} and shows that a dipole similar to the one we observe on Earth might arise from such a source distribution.

The measurements of the Pierre Auger Observatory above $8$~EeV show the mean $\ln A$ of cosmic rays approximately heavier than helium nuclei \cite{ICRC19_YushkovAuger} with an indication of underestimation of the mean $\ln A$ due to uncertainties of models of hadronic interactions used for the mass-interpretation of measurements \cite{XmaxAdjust_Vicha}. Nevertheless, for all the investigated initial amplitudes solutions with such $\ln A$ can be found. Further constraints on the initial amplitude might be made using an energy dependent mass composition of cosmic rays, which is out of the scope of this current work. 

 \bigskip
\textbf{Dipole at lower energies:} The measurements performed by the Pierre Auger Observatory show that the dipole anisotropy in the arrival directions of cosmic rays at lower energies, in range from 4~EeV up to 8~EeV, has an amplitude of $2.5^{+1.0}_{-0.7}~\%$ and is pointing towards right ascension $(80\pm60)^{\circ}$ and declination $(-75^{+17}_{-8})^{\circ}$ \cite{AugerDipole2018}.  However, this dipole is not statistically significant ($<3\sigma$). In this energy range, cosmic rays are deflected in the GMF more due to their lower rigidities. In order to compare simulations with measured dipole direction in this lower energy bin, we simulate the cosmic-ray propagation in GMF at these lower energies from 4~EeV up to 8~EeV and repeat the same analysis as for the higher energies $>8$~EeV. 

\begin{figure}[htp]
     \centering
         \includegraphics[width=0.9\textwidth]{figures/JF12lowenergy_new.pdf}
        \caption{Directions of the extragalactic  dipole in cosmic-ray arrivals compatible with the measured direction of the dipole within $1\sigma$ by the Pierre Auger Observatory (red contour) with energies $4-8$ EeV (light blue) and above $8$~EeV (dark blue) using the JF12 model of the GMF. The amplitude of the lower energy dipole on Earth is considered to be less than 4\%.}
        \label{fig:lowE_JF12}
\end{figure}

\begin{figure}[htp]
     \centering
         \includegraphics[width=0.9\textwidth]{figures/TF17lowenergy_new.pdf}
        \caption{Same as in Figure~\ref{fig:lowE_JF12}, but using the TF17 model of the GMF.}
        \label{fig:lowE_TF17}
\end{figure}

The sources creating the dipole anisotropy above $8$~EeV can also accelerate particles at lower energies reaching Earth. However, the distribution of sources of cosmic rays at energies ($4-8$)~EeV might be different than the distribution of sources of cosmic rays above $8$~EeV. As the dipole at energies ($4-8$)~EeV has about three times lower amplitude, it is possible that additional sources, that are not capable of accelerating particles to the highest energies, contribute here as well, possibly even as an isotropic component. They thus lower the observed amplitude of the dipole on Earth even more than just the effects of the GMF. For that reason, in the case of the dipole at energies ($4-8$)~EeV we are searching for directions of the extragalactic dipole that would end up with a dipole on the observer level with amplitude lower than $4$\% but still equal or higher than the measured amplitude (Other upper limit amplitudes on the observer are investigated as well, including 2, 3 and 5\%. We report the results for 4\% as an example). The allowed directions of the extragalactic dipole for energies ($4-8$)~EeV and above $8$~EeV at the $1\sigma$ level are shown in Figure~\ref{fig:lowE_JF12} and Figure~\ref{fig:lowE_TF17} for JF12 and TF17 models of the GMF, respectively. The regions of possible directions of the extragalactic dipole obtained from simulations at energies ($4-8$)~EeV are larger as the uncertainties of the measured dipole direction are much larger than for the dipole above $8$~EeV. Nevertheless, there are overlapping regions of possible extragalactic directions of the dipole at energies ($4-8$)~EeV and above $8$~EeV. This is not in contradiction with a possibility of the same origin of these two dipole anisotropies at different energy scales, where one is more smeared by the GMF as the rigidities are lower and possibly suppressed by additional sources. 

 \bigskip
\textbf{Simplifications of the method:} Few simplifications are used in this work. Firstly, we use spectral index $\gamma=3$ as a close approximation of the spectral index of cosmic-ray energy spectrum above 8~EeV measured by current observatories. We verify that small changes of spectral index do not significantly change our results. We confirm that by exploiting spectral index 2.8 and 3.2 for proton primary. The resulting areas of allowed directions of the extragalactic dipole differ then by maximally $2^{\circ}$. 

Secondly, deflections in the extragalactic magnetic field are not taken into account here as we are not trying to identify individual extragalactic sources and therefore the trajectory length of the cosmic rays from a source to our Galaxy is unknown. Moreover, these deflections are much smaller than the deflections caused by the GMF. If we assume that the particles are coming from sources not further than 100 Mpc, the deflection in the extragalactic magnetic field can be estimated as \cite{EGdeflection_waxman, EGdeflection_Farrar}
\begin{equation}
\delta\theta_{EG}\approx0.15^{\circ}\left(\frac{D}{3.8\rm{Mpc}}\cdot\frac{\lambda_{EG}}{100\rm{kpc}}  \right)^{\frac{1}{2}}\left(\frac{B_{EG}}{1\rm{nG}}\cdot\frac{Z}{E_{100}} \right),
\end{equation}
where $D$ is the distance of the source, $\lambda_{EG}$ and $B_{EG}$ are the coherence length and the strength of the extragalactic magnetic field, $Z$ is the charge of the cosmic ray and $E_{100}$ is the energy of the particle in units 100 EeV. This approximation of the total deflection is estimated for a particle experiencing many small deflections in a turbulent magnetic field. Taking a coherence length of the extragalactic magnetic field 100 kpc and strength of  1 nG, a proton of energy 100 EeV originating in a close source at $D=3.8$ Mpc would be deflected by $\approx0.15^{\circ}$, while a particle with the same energy originating in a source at $D=100$ Mpc would be deflected by $\approx 0.8^{\circ}$. Nevertheless, such particle would be also influenced by various energy-loss processes. 





\section{Conclusions}
\label{sec:conc}

We investigate the influence of the Galactic magnetic field on the dipole anisotropy of ultra-high-energy cosmic rays that is measured by the Pierre Auger Observatory \cite{AugerDipole2017}. In our study, we assume an extragalactic origin of the UHECRs above 8~EeV with a dipole distribution in arrival directions when entering the Milky Way. Using simulations of the UHECR propagation in two models of the Galactic magnetic field, JF12 and TF17, we estimate the effects on the amplitude and direction of the dipole anisotropy in the arrival directions of UHECRs.

The comparison of the simulations with the measurement is performed at the $1\sigma$ and $2\sigma$ level. For the $1\sigma$ level, low initial amplitudes of the dipole outside the Galaxy are needed ($\leq10\%$) in the case of light composition (mostly protons and helium nuclei) with extragalactic directions in close vicinity of the measured dipole. Higher initial amplitudes ($10\%-20\%$) result in compatible dipoles in arrival directions for heavier composition mixes. Most of these extragalactic directions of the dipole are within $\approx40^{\circ}$ (JF12), resp. $\approx55^{\circ}$(TF17) from the dipole direction observed on Earth. At the $2\sigma$ level, the sky regions of allowed directions of the extragalactic dipole are much more extended covering interesting regions like 2MRS dipole or Centaurus~A.





\acknowledgments
This work was supported by the Ministry of Education, Youth and Sports of the Czech Republic – Grant No. LTT18004, LM2023032 and CZ.02.1.01/0.0/0.0/16\_013/0001402. We would like to thank our colleagues at FZU - Institute of Physics for numerous comments and suggestions and to Noémie Globus for useful suggestions regarding this work. We are also grateful for the stimulating environment within the Pierre Auger Collaboration.




% Bibliography

%% [A] Recommended: using JHEP.bst file
\bibliographystyle{JHEP}
\bibliography{bibliography.bib}

%% or
%% [B] Manual formatting (see below)
%% (i) We suggest to always provide author, title and journal data or doi:
%% in short all the informations that clearly identify a document.
%% (ii) please avoid comments such as "For a review'', "For some examples",
%% "and references therein" or move them in the text. In general, please leave only references in the bibliography and move all
%% accessory text in footnotes.
%% (iii) Also, please have only one work for each \bibitem.

%\bibliography{bibliography}
%\begin{thebibliography}{99}

%\bibitem{a}
%Author,
%\emph{Title},
%\emph{J. Abbrev.} {\bf vol} (year) pg.

%\bibitem{b}
%Author,
%\emph{Title},
%arxiv:1234.5678.

%\bibitem{c}
%Author,
%\emph{Title},
%Publisher (year).

%\end{thebibliography}
\end{document}

