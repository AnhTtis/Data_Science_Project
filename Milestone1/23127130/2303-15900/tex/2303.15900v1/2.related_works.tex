\section{Related Works}
%Why do we have to use our method for this particular problem?
\subsection{Legged Robot Navigation}

Our work is motivated by autonomous robotic navigation, which is the problem of finding a valid path to reach a target located in an environment. Although promising results have been presented for wheel-based robots~\cite{chaplot2020learning, kadian2020sim2real, hirose2022exaug, yokoyama2021success}, the navigation of legged robots is still at a premature stage without considering joint-level planning. Therefore, legged robot navigation is usually tackled using a hierarchical structure where the high-level controller generates a command variable for a low-level controller~\cite{kareer2022vinl, truong2021learning} who runs at a higher frequency. The low-level controller can be either expert designed ~\cite{kareer2022vinl,truong2021learning,sorokin2022learning} or learned using reinforcement learning~\cite{fu2022coupling, jain2020pixels}. However, most of the previous works only considers environments with flat surfaces, which do not fully exploit the potential of the legged robot. Although a recent work\cite{kareer2022vinl} addresses the legged robot navigation with small obstacles scattered on the surface, the obstacles are only treated as disturbances. In this work, our goal is to generate a feasible motion plan for a legged robot in realistic indoor environments that include stairs and obstacles. 



% I think TO equals MP in quadrupedal robotics.
\subsection{Motion Planning for legged robots}
Motion planning is a research domain that addresses the problem of generating robot motion trajectories for given tasks, which is often approached by trajectory optimization~\cite{winkler2018gait, carius2019trajectory, cebe2021online}. Due to the nonlinearity and high dimensionality of the nature of the robot's dynamics, multiple simplifications are applied to solve the problem, such as a single rigid body model~\cite{ding2019real, di2018dynamic} or inverted pendulum models \cite{green2021learning}. Winkler et al.~\cite{winkler2018gait} reduces the problem space by by using Hermite polynomials for parameterizing the kinodynamic trajectories, where he simplifies further by removing the cost function and treats trajectory optimization as a feasibility problem. Even with these simplifications, solving trajectory optimization problems with long-horizon and complex terrains is still relatively slow and unstable, where the quality and optimization time is highly sensitive to the initial condition. Recent research leverage machine learning methods to accelerate the motion planning process. Mansard et al. \cite{mansard2018using} use deep neural networks to memorize the previous optimized solutions and use them to warm-start the next optimization. Kurtz et al. \cite{kurtz2022mini} train a neural network to plan an entire robot trajectory given the initial state of a robot. The key difference between our method and \cite{kurtz2022mini} is that our generated motion can be modified according to a high-level control variable while \cite{kurtz2022mini} creates a fixed trajectory based only on the initial state of the robot.

\subsection{Kinematic Animation}
Various approaches have been discussed to generate plausible motions from the dataset. One classical method is to construct a motion graph that links different motion clips manually \cite{rose1998verbs} or automatically \cite{kovar2008motion}. The main drawback of these motion graph methods is the inherent discreteness of their underlying mechanisms, where transitions only happen at the end of short motion segments. On the other hand, principle component analysis~(PCA) reduces the dimension of the motion data and produces full-body motions by adjusting variables in a continuous low-dimensional space \cite{howe1999bayesian, safonova2004synthesizing}. Yet, PCA-based methods have difficulty dealing with a large amount of motion data. Recent works have started to leverage deep neural networks and machine learning for motion synthesis. Phase-Functioned Neural Network (PFNN)~\cite{holden2017phase} learns an effective motion controller using its novel network architecture, which conditions its weights on a phase variable. Model-adaptive Neural Network~\cite{zhang2018mode} generates realistic animations of a quadrupedal character by introducing a gated network architecture. Starke and the colleague~\cite{starke2020local} introduce a technique to automatically extract local motion phase signals from the main joints of the skeleton. The work of Li et al.~\cite{li2022ganimator} adopts a generative model to synthesize new motion sequences from a single input clip. 
