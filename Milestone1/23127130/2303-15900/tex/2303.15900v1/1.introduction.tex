% !TeX root = ./main.tex



\section{Introduction}
% 1. Motion planning is a tough problem for high-dof robots, in many different applications
Generating effective and plausible motions for robotic creatures has been one of the most important topics in various autonomous robotic missions, including manipulation, locomotion, and navigation. For instance, a manipulator in cluttered environments needs to find a collision-free path to reach the target object. A legged robot in challenging environments needs to carefully its footsteps to approach the destination while maintaining its balance. However, motion planning becomes more and more challenging for higher-dimensional robots, such as quadruped or humanoid robots, because the search space size is exponential to the number of degrees of freedom (DoFs). Researchers have proposed many approaches, such as simplified models, to find an optimal trajectory in such high-dimensional spaces, but it is still an ongoing research topic in the field of robotics.

\begin{figure}[t]
\centering
    \begin{subfigure}[b]{0.45\textwidth}
        \centering
        \includegraphics[width=0.95\textwidth]{figures/profile_1.png}
    \end{subfigure}
\caption{\small{We present a framework for learning an auto-regressive motion planner (ARMP) to generate physically plausible motion plans for various tasks and environments. Here we show using ARMP as the underlying motion planner for an indoor navigation task.}}
\label{fig: profile}
\vspace{-10pt}
\end{figure}

% 2. This becomes limitation in navigation. Simple models (teleported). Bridging gap.
This challenge of motion planning is also another major hurdle in developing a visual navigation controller for high-dimensional robots, such as 12 DoF quadrupedal robots. Many navigation simulatiors~\cite{xia2018gibson, szot2021habitat} abstract away robotic agents' kinematic and dynamic features by adopting a simple state representation that consists of the global position and orientation, ignoring the transition dynamics. Due to this abstraction, high-level navigation planners cannot utilize the full capability of the quadrupedal robot, such as jumping, walking over obstacles, or climbing stairs. However, it is still not straightforward how we can generate motions effectively due to its expensive computational cost.

% 3. (Literature) quadruped offline vs. online
Over the decades, researchers have proposed various approaches for planning kinematic or dynamic trajectories. One popular approach is trajectory optimization (TO) which solves an open-loop motion plan offline. Offline trajectory optimization can also be formulated in multiple ways: for instance, a collocation method discretizes the trajectory over time and optimizes it with respect to the given task objectives and dynamics constraints. However, offline planning often requires specifying a fixed time horizon, which may not be suitable for navigation tasks with either unknown or infinite time duration. On the other hand, a shooting method leverages simulation to generate trajectories in a forward direction. However, designing such a robust dynamic controller is very challenging despite the recent advances in visual locomotion \cite{kumar2021rma, yu2021visual, jenelten2020perceptive, grandia2022perceptive}. In addition, physics simulation for quadrupedal locomotion typically requires $500$ to $1,000$ Hz simulation, which would be too expensive to train a visual navigation agent.


\begin{figure*}[ht]
\centerline{\includegraphics[width=0.9\linewidth]{figures/ARMP_overview.png}}
\caption{\small{Overview of the framework. In the training phase, a motion library is constructed by conducting trajectory optimization in different settings. In the planning phase, the motion planner leverages an autoregressive method by taking a blended desired trajectory as part of the input for predicting the robot's motion.}}
\label{fig: Overview}
\vspace{-10pt}
\end{figure*}

% 4. We present ARMP... We take inspiration from...
We present a novel autoregressive motion planner (ARMP), which is an efficient approach to synthesizing quadrupedal motion plans from the database. To ensure the physical plausibility of our motion planner, we create a dense set of motion trajectories over various tasks and environments using the traditional offline motion planning algorithm \cite{winkler2018gait}. Then we train a neural network that learns to sequentially generates motion based on previous frames and user control. We take inspiration from the recent advances in character control in computer animation \cite{holden2017phase, zhang2018mode, starke2019neural} and adopt a mixture-of-expert network architecture.%%\ty{Still purple here} \jhc{Their autoregressive architecture enables the planner to easily take account of environment and user control, which is essential for locomotion under complex environments.}



% 5. Our results
We showcase that the proposed ARMP framework can efficiently generate natural and feasible motions of the Aliengo quadrupedal robot \cite{unitree}. It can automatically adapt its gaits to overcome various terrains, including slopes, stairs, and flat terrain with obstacles that require the robot to jump over. We ensure the physical validity of the generated trajectories by reproducing the trajectories in a physics simulator~\cite{raisim}. Because it runs at real-time 
and does not require any internal optimization, ARMP can be easily combined with the existing embodied AI and autonomous navigation framework, Habitat \cite{habitat19iccv, szot2021habitat}, which can be potentially used for future learning of quadrupedal robot navigation skills.