%%%%%%%%%%%%%%%%%%%%%%%%%%%%%%%%%%%%%%%%%%%%%%%%%%%%%%%%%%%%%%%%%%%%%%%%%%%%%%%%
%2345678901234567890123456789012345678901234567890123456789012345678901234567890
%        1         2         3         4         5         6         7         8

\documentclass[letterpaper, 10 pt, conference]{ieeeconf}  % Comment this line out if you need a4paper

%\documentclass[a4paper, 10pt, conference]{ieeeconf}      % Use this line for a4 paper

\IEEEoverridecommandlockouts                              % This command is only needed if 
                                                          % you want to use the \thanks command

\overrideIEEEmargins                                      % Needed to meet printer requirements.

%In case you encounter the following error:
%Error 1010 The PDF file may be corrupt (unable to open PDF file) OR
%Error 1000 An error occurred while parsing a contents stream. Unable to analyze the PDF file.
%This is a known problem with pdfLaTeX conversion filter. The file cannot be opened with acrobat reader
%Please use one of the alternatives below to circumvent this error by uncommenting one or the other
%\pdfobjcompresslevel=0
%\pdfminorversion=4

% See the \addtolength command later in the file to balance the column lengths
% on the last page of the document

% The following packages can be found on http:\\www.ctan.org
%\usepackage{graphics} % for pdf, bitmapped graphics files
\usepackage{epsfig} % for postscript graphics files
%\usepackage{mathptmx} % assumes new font selection scheme installed
%\usepackage{times} % assumes new font selection scheme installed
%\usepackage{amsmath} % assumes amsmath package installed
%\usepackage{amssymb}  % assumes amsmath package installed
\usepackage{xcolor}
\usepackage{bm}
\usepackage{amsmath,amssymb,array,balance,booktabs,cite,changes,color,enumerate,float,graphicx,hyperref,multicol,multirow,siunitx,times,subcaption,mathtools,colortbl,}

%%%%% GENERAL MATH COMMANDS
% Reals
\newcommand{\R}{{\mathbb R}}
% Integers
\newcommand{\Z}{{\mathbb Z}}
% Naturals
\newcommand{\N}{{\mathbb N}}
% Expectation
\DeclareMathOperator*{\E}{\mathbb{E}}
% ^th notation
\newcommand{\tth}{^{\text{th}}}
% Small dots for integer range [a .. b]
\newcommand{\sdots}{\,..\,}
% Vectorized version of matrix
\newcommand{\matvec}{\mbox{vec}}

% := sign
\newcommand{\defeq}{\vcentcolon=}
% Zero function
\newcommand{\zf}{\mathbf{0}}
% Vector of ones
\newcommand{\ones}{\mathbf{1}}

% Argmin and argmax definitions
\DeclareMathOperator*{\argmax}{arg\,max}
\DeclareMathOperator*{\argmin}{arg\,min}


%%%%% PROBLEM STATEMENT NOTATION 
% \newcommandtwoopt{\St}[2][t][]{{S_{#1}^{#2}}} % State
\newcommand{\task}[1][i]{{\mathcal{T}_{#1}}} % Task, optionally takes index
\newcommand{\tasks}{\{ \task \}_{i=1}^N}
\newcommand{\losst}[1][i]{{l_{#1}}}
\newcommand{\lossv}[1][i]{{l_{#1}^{\textrm{val}}}}
\newcommand{\tasktarget}{{\mathcal{T}_{\textrm{target}}}}
\newcommand{\lossttarget}{l_{\textrm{target}}}
\newcommand{\lossvtarget}{l_{\textrm{target}}^{\textrm{val}}}
\newcommand{\lossttargetit}{l_{\textrm{target}}^{(k)}}
\newcommand{\losstotal}{l^{\textrm{total}}}
\newcommand{\lossopt}{l^*}

\newcommand{\thetait}[2]{\theta_{#1}^{(#2)}}
\newcommand{\phit}[1]{\phi^{(#1)}}
\newcommand{\hist}[2]{S_{#1}^{(#2)}}
\newcommand{\grad}[2]{G_{#1}^{(#2)}}

\newcommand{\Alg}{\textup{\textbf{Opt}}}
\newcommand{\MetaAlg}{\textup{\textbf{MetaOpt}}}

%%%%% Theorems
\newtheoremstyle{mytheoremstyle} % name
    {\topsep}                    % Space above
    {\topsep}                    % Space below
    {\itshape}                   % Body font
    {}                           % Indent amount
    {\scshape}                   % Theorem head font
    {.}                          % Punctuation after theorem head
    {.5em}                       % Space after theorem head
    {}  % Theorem head spec (can be left empty, meaning ‘normal’)
\theoremstyle{mytheoremstyle}
\theoremstyle{plain}
\newtheorem{theorem}{Theorem}
\newtheorem{proposition}{Proposition}
\newtheorem{assumption}{Assumption}
\newtheorem{definition}{Definition}
\newtheorem{lemma}{Lemma}
\theoremstyle{remark}
\newtheorem{remark}{Remark}

\definecolor{myblue}{rgb}{0.44, 0.65, 0.82}
\newcommand{\ty}[1] {{\color{myblue} [TY: {#1}]}}
\renewcommand\vec{\mathbf}
\DeclareMathOperator*{\argmin}{arg\,min}
\DeclareMathOperator*{\argmax}{arg\,max}
\DeclareMathOperator*{\minimize}{min}

\title{\LARGE \bf
ARMP: Autoregressive Motion Planning for Quadruped Locomotion and Navigation in Complex Indoor Environments}


\author{Jeonghwan Kim, Tianyu Li, Sehoon Ha
\thanks{Georgia Institute of Technology, Atlanta, GA, 30308, USA}
\thanks{{\tt \small jkim3662@gatech.edu, tli471@gatech.edu, sehoonha@gatech.edu}}%
}


\begin{document}



\maketitle
\thispagestyle{empty}
\pagestyle{empty}

Answering first-order logical (FOL) queries over knowledge graphs (KG) remains a challenging task mainly due to KG incompleteness. 
Query embedding approaches this problem by computing the low-dimensional vector representations of entities, relations, and logical queries. 
KGs exhibit relational patterns such as symmetry and composition and modeling the patterns can further enhance the performance of query embedding models.
However, the role of such patterns in answering FOL queries by query embedding models has not been yet studied in the literature.
In this paper, we fill in this research gap and empower FOL queries reasoning with pattern inference by introducing an inductive bias that allows for learning relation patterns. 
To this end, we develop a novel query embedding method, RoConE, that defines query regions as geometric cones and algebraic query operators by rotations in complex space. RoConE combines the advantages of Cone as a well-specified geometric representation for query embedding, and also the rotation operator as a powerful algebraic operation for pattern inference. 
%Therefore, RoConE enables inferring patterns during the multi-hop reasoning process.
Our experimental results on several benchmark datasets confirm the advantage of relational patterns for enhancing logical query answering task.
\section{Introduction}
Deep learning~\cite{dl} has been highly successful in computer vision~\cite{sg1,od1,app-detection,zhou2024diffdet4sar,li2024predicting,yang2024saratr,LiSARATRX25}, largely due to the availability of large-scale labeled datasets. However, in many practical scenarios, obtaining such large amounts of labeled data is difficult or costly. To address this challenge, Few-shot learning (FSL) aims to enable models to learn new tasks with only a limited number of labeled samples. Consequently, this problem has garnered significant attention in both academia and industry due to its broad real-world applications. While humans can easily distinguish between objects after seeing only a few examples, machines struggle to achieve similar efficiency. In domains such as natural scene images, large datasets are readily available, but FSL is crucial in scenarios where collecting large amounts of data is difficult. Since the problem was first introduced in 2006~\cite{fsl-1}, numerous methods have been proposed to tackle the challenges of FSL~\cite{fslsurvey,fslsurvey22,fslsurvey20,fsl18,fslsurvey1}.

With the development of FSL, challenges such as limited training data, domain variations, and task modifications have led to the emergence of various FSL variants, including semi-supervised FSL~\cite{semifsl}, unsupervised FSL~\cite{ufsl1,ufsl2}, zero-shot learning (ZSL)\cite{zsl1}, and cross-domain FSL (CDFSL)~\cite{feature-wise,bscd-fsl}, among others. These variants represent distinctive cases of FSL in terms of sample availability and domain learning. This paper focuses specifically on CDFSL variants. The traditional FSL problem assumes that both prior knowledge and target tasks come from the same domain, which is often restrictive in real-world applications. CDFSL addresses this issue by overcoming the domain gap between auxiliary data (which provides prior knowledge) and the target data in FSL tasks, as show in Figure~\ref{int}. For instance, in art image recognition tasks involving scribble, cartoon, or sketch images, FSL could theoretically leverage prior knowledge from related domains like cartoons and sketches. However, such data is often scarce due to copyright restrictions and the high cost of collection. As a result, researchers have turned to data-rich domains, such as natural scene images, to address the challenges of few-shot image recognition in the field of art.
However, the significant domain gap between these domains often leads to performance degradation in FSL. CDFSL faces challenges from both transfer learning and FSL, including domain gaps, class shifts, and the scarcity of labeled samples in the target domain, making it a more complex task. Since its formal introduction in 2020~\cite{feature-wise}, CDFSL has garnered widespread attention, with numerous methods published in top venues~\cite{bscd-fsl,st,dynamic,hybrid_1,feature_reweight_6}. Figure~\ref{imaging} presents the milestones of CDFSL technologies from 2019 to the present, showcasing representative CDFSL methods and related benchmarks.
\begin{figure}%[b]
	\centering
  \vspace{-0.3cm}
 	\includegraphics[width=0.9\linewidth]{CDFSLProblem-10.pdf}
  \vspace{-0.3cm}
	\caption{\textcolor{black}{The difference of few-shot learning and cross-domain few-shot learning.}}
 \vspace{-0.4cm}
	\label{int}
\end{figure}


So far, several surveys have provided comprehensive overviews and future directions for FSL~\cite{fsl18,fslsurvey,fslsurvey1,fslsurvey20,fslsurvey22}. For example,\cite{fsl18} categorizes FSL into experiential and conceptual learning, while\cite{fslsurvey} focuses on empirical risk minimization and defines FSL by experience, task, and performance, introducing CDFSL as a branch of FSL. Both~\cite{fslsurvey1} and~\cite{fslsurvey20} highlight CDFSL as a variant of FSL, discussing meta-learning approaches and benchmarks. Lastly,~\cite{fslsurvey22} offers a taxonomy based on prior knowledge and emphasizes that current methods have yet to fully tackle cross-domain challenges. Collectively, these works point to cross-domain learning as a promising area for future FSL research. Currently, there are two elementary surveys on CDFSL~\cite{wang2023survey,deng2023survey}. \cite{wang2023survey} classifies methods into benchmark, single source, and multiple source categories, while~\cite{deng2023survey} categorizes algorithms into data augmentation and feature alignment paradigms. In contrast, to stimulate future research and help newcomers better understand this challenging problem, this paper offers the first classification grounded in theoretical analysis and provides a comprehensive review, offering deeper insights into the core principles of CDFSL. Firstly, this paper compiles and analyzes a broad range of literature on the topic. An analysis of the reference index reveals that even before the formal introduction of CDFSL, some works had already tried to solve cross-domain issues within the FSL framework~\cite{clc, rffl}. Following its formal introduction as a branch of FSL, CDFSL has garnered significant attention. Additionally, we define CDFSL using both machine learning theory~\cite{ml,erm1} and transfer learning principles~\cite{tltheory}. Secondly, our analysis highlights that the unique challenge in CDFSL lies in the unreliable nature of two-stage empirical risk minimization. The details are discussed in Section~\ref{background}. To address these challenges, the paper organizes CDFSL research into four categories: $\mathcal{D}$-Extension, $\mathcal{H}$-Constraint, $\Delta$-Adaptation, and hybrid approaches. We also compile relevant datasets and benchmarks to evaluate the methods, and analyze their performance, as discussed in Sections~\ref{methods} and~\ref{performance}. Finally, we explore future research directions for CDFSL by considering three perspectives, including problem set-ups, applications, and theories, which provide a comprehensive understanding of the field and its potential for future growth. Contributions of this survey can be summarized as follows:

\begin{itemize}
    \item We analyzed existing CDFSL papers and provided a comprehensive survey. We also defined CDFSL formally, connecting it to classic ML~\cite{ml,erm1} and transfer learning theory~\cite{tltheory}. This helps guide future research in the field.
    \item We listed relevant learning problems for CDFSL with examples, clarifying their relation and differences. This helps position CDFSL among various learning problems. We also analyzed unique issues and challenges of CDFSL, helping to explore a scientific taxonomy for CDFSL work.
    \item We conducted an extensive literature review, organizing it into a unified taxonomy based on $\mathcal{D}$-Extension, $\mathcal{H}$-Constraint, $\Delta$-Adaptation, and hybrid approaches. We introduced applicable scenarios for each taxonomy to help discuss its pros and cons. We also presented datasets and benchmarks for CDFSL, summarizing insights from performance results to improve understanding of CDFSL methods.
    \item We proposed promising future directions for CDFSL in problem set-ups, applications, and theories, based on current weaknesses and potential improvements.
\end{itemize}

\begin{figure}
	\centering
  \vspace{-0.3cm}
        \includegraphics[width=\linewidth]{response/crop_fig2.pdf}
 \vspace{-0.5cm}
	\caption{Chronological milestones of CDFSL from 2019 to the present, including representative CDFSL approaches and related benchmarks. Key events include the release of Meta-Dataset~\cite{meta-dataset} and BSCD-FSL~\cite{bscd-fsl} in 2020, the introduction of pioneering works such as~\cite{feature-wise}, and subsequent contributions like~\cite{feature_reweight_1,lscdfsl}. Later works~\cite{st,dynamic,hybrid_1,hybrid_4,hybrid_2} explored new setups, while~\cite{boosting,ata,data_target_1,feature_reweight_5,parameter_weight_2,confess,feature_reweight_9} focused on improving performance. Please see Section~\ref{methods} for details.}
 \vspace{-0.3cm}
	\label{imaging}
\end{figure}

The remainder of this survey is organized as follows: Section \ref{background} provides an overview of CDFSL, including its definition, challenges, and taxonomy. Section \ref{methods} covers approaches to CDFSL in detail, while Section \ref{performance} presents performance results and evaluates methods. Section \ref{future} explores future directions in set-ups, applications, and theories. Finally, Section \ref{conclusion} concludes the survey.
\section{Related Work}
\label{sec:related}

Non-rigid shape matching is a very rich and well-established research area, and a full overview is beyond
the scope of this paper.  Below, we review works that are close to our method and  refer the interested reader to recent surveys \cite{cao2020comprehensive,bronstein2017geometric,guo2016comprehensive,guo2020deep} for a more in-depth treatment.

\tightpara{Functional Maps}
Since its introduction \cite{Ovsjanikov2012}, the functional map pipeline has become a widely-used tool for non-rigid shape matching, being adopted and extended in many follow-up works \cite{ginzburg2019cyclic,Ren2019,eynard2016coupled,Melzi_2019,eynard2016coupled,Nogneng2017,rodola2017partial,poulenard_persistence,sharma2020weakly}. The advantage of this approach is that it reduces the optimization of pointwise maps, which are quadratic in the number of vertices, to optimizing small matrices, thus greatly simplifying computational problems. We refer to \cite{Ovsjanikov2017} for an overview. 

The original functional map pipeline relied on input feature (probe) functions, which were given a priori \cite{Salti2014,sun2009concise,aubry2011wave}.
% , to estimate the underlying functional maps. Most earlier works in this domain used hand-crafted features \cite{Salti2014,sun2009concise,aubry2011wave}. 
Subsequent research has improved the method by adapting it to partial shapes \cite{cosmo2016shrec,rodola2017partial,Litany2017} or using robust regularizers  %estimation pipeline by introducing robust regularizers and penalties 
\cite{Ren2019,Nogneng2017,kovnatsky2013coupled,burghard2017embedding} and proposed efficient refinement techniques \cite{Melzi_2019,Pai_2021_CVPR,jing_maptree}. %, and combined it with extrinsic shape alignment techniques and optimal transport tools \cite{aygun2020unsupervised,Eisenberger2020SmoothSM,pmf}. 
In all of these works, fmaps were computed using hand-crafted probe functions, and any information loss in these descriptors hinders downstream optimization.

More recent works have proposed to solve this problem by learning descriptors (probe functions) directly from the data, using deep neural networks. This line of research was initiated by FMNet \cite{litany2017deep}, and extended in many subsequent works \cite{halimi2019unsupervised,attaiki2023vader,roufosse2019unsupervised,sharma2020weakly,sharp2021diffusion,attaiki2021dpfm,donati2020deep,Eisenberger2021NeuroMorphUS,attaiki2022ncp,eisenberger2020deep,li2022srfeat}. These methods learn the probe features directly from the raw geometry, using a neural network, and supervise the learning with a loss on the functional map in the reduced basis.

% Litany \etal \cite{litany2017deep} proposed to refine the input SHOT descriptors \cite{Salti2014} using a pointwise MLP, and supervised the training with a loss with ground-truth pointwise map. Follow-up approaches, such as \cite{donati2020deep} improved this framework by learning the probe features directly from the raw geometry, using a neural network, and supervising the learning with a loss on the functional map in the reduced basis. Subsequent work further improved the method, making it robust to mesh discretization \cite{sharp2021diffusion} and adapting it to handle partial shapes, achieving state-of-the-art results in many non-rigid matching scenarios.

A parallel line of work has focused on making the learning \textit{unsupervised}, which can be convenient in the absence of ground truth correspondences. Approaches in \cite{halimi2019unsupervised} and \cite{Ginzburg2020} have proposed penalizing either the geodesic distortion of the pointwise map predicted by the network or using the cycle consistency loss.  %However, such approaches require the computation and storage of heavy geodesic matrices, and has shown little generalizability. 
Another line of work \cite{roufosse2019unsupervised,sharma2020weakly,sharp2021diffusion,donati-duo} considered unsupervised training by imposing structural properties on the functional maps in the reduced basis, such as bijectivity, orthonormality, and commutativity of the Laplacian. 
% Eisenberger et al. \cite{eisenberger2020deep} proposed to combine intrinsic and extrinsic alignment, in addition to a refinement of the maps in the network, at the cost of efficiency and computational time. 
The authors of \cite{sharma2020weakly} have shown that feature learning can be done starting from raw 3D geometry in the weakly supervised setting, where shapes are only approximately rigidly pre-aligned. 
% Finally, interpolation was also used a geometric prior to learning robust features in \cite{Eisenberger2021NeuroMorphUS}.

In all of these works, features extracted by neural networks have been used to formulate the optimization problem, from which a functional map is computed. Thus, %feature functions were only used algebraically, as part of an equation to solve the functional maps, and
so far no attention has been paid to the geometric nature or other utility of learned probe functions. In contrast, we aim to analyze the conditions under which probe functions can be used for direct pointwise map computation and use this analysis to design improved map estimation pipelines. 

We also note briefly a very recent work  \cite{li2022srfeat} which has advocated for imposing feature smoothness  when learning for non-rigid shape correspondence. However, that work does not use the in-network functional map estimation and moreover lacks any theoretical analysis or justification for its design choices. % their content in general. Thus, the exact nature of information learned within these 
%
%\souhaib{should we say that sharma requires the shape to be aligned, because we will be using this later}



\tightpara{Recent Advances in Axiomatic Functional Maps}
Our results are also related to recent axiomatic methods for functional map-based methods, which  \textit{couple} optimization for the functional map (fmap) with the associated point map (p2p map) \cite{ren2018continuous,Melzi_2019,Pai_2021_CVPR,Xiang_2021_CVPR}. These techniques typically propose to refine functional maps by \textit{iterating} the conversion from functional to pointwise maps and vice versa. This approach was recently summarized in \cite{discrete_Ren2021}, where the authors introduce the notion of functional map ``properness'' and describe a range of energies that can be optimized via this iterative conversion scheme. 

The common denominator between all these methods is the inclusion of pointwise maps in the process of functional map optimization. In this work, we propose a method and a loss that similarly incorporate pointwise map computation. However, we do so in a learning context and show that this leads to significant improvements in the overall accuracy of the deep functional map pipeline.



\tightpara{Learning on Surfaces}
Multiple methods for deep surface learning have been proposed to address the limitations of handcrafted features in downstream tasks. One type of method is point-based (extrinsic) methods, such as PointNet \cite{qi2017pointnet} and follow-up works \cite{qi2017pointnet++,rethink_ma_22,thomas2019KPConv,dgcnn,pcnn_2018,Wiersma2022DeltaConv}. These methods are simple, effective, and widely applicable. However, they often fail to generalize to new datasets or significant pose changes in deep shape matching. 

Another line of research \cite{poulenard2018multi,sharp2021diffusion,wiersma2020cnns,gong2019spiralnet++,masci2015geodesic,maron2017convolutional} (intrinsic methods) has focused on defining the convolution operator directly on the surface of the input shape. These methods are more suitable for deformable shapes and can leverage the structure of the surface encoded by the mesh, which is ignored by the extrinsic methods.


% To overcome the limitations of handcrafted features in downstream tasks, multiple methods for deep surface learning have been proposed. The first type of method is point-based (extrinsic) methods, pioneered by \cite{qi2017pointnet}, and extended by many works such as \cite{qi2017pointnet++,rethink_ma_22,thomas2019KPConv,dgcnn,pcnn_2018,Wiersma2022DeltaConv} to name a few. These methods are characterized by their simplicity, effectiveness, and applicability in a wide range of domains. Many of these methods have been adapted in the case of deep shape matching \cite{sharma2020weakly,donati2020deep}, and while they perform well in normal scenarios (the test dataset is similar to the training dataset), they often fail to generalize to new datasets or under significant pose changes.

% Another line of research \cite{poulenard2018multi,sharp2021diffusion,wiersma2020cnns,gong2019spiralnet++,masci2015geodesic,maron2017convolutional} (intrinsic methods) has focused on defining the convolution operator directly on the surface of the input shape. The motivation is that these methods are more suitable for deformable shapes, and can benefit from the structure of the surface encoded by the mesh, which is ignored by the first category of methods.

In previous works, intrinsic methods tend to perform better for non-rigid shape matching than extrinsic methods, with DiffusionNet \cite{sharp2021diffusion} being considered the state-of-the-art feature extractor for shape matching. In this work, we will show that a simple modification to extrinsic feature extractors improves their overall performance for shape matching, making them comparable to or better than DiffusionNet. 

\section{Method}
\label{section:A}
\subsection{Overview}
Our goals are two-fold: 1) to promote the interpretability of the feature space for haze removal and 2) to establish a more concise solution space using of contrastive samples. Fig.~\ref{fig:method} illustrates the detailed structure of our C$^2$PNet. To achieve our first goal, we design a physics-aware dual-branch unit that is derived from the atmospheric scattering model. Regarding our second aim, we tailor a contrastive regularization using consensual negatives, along with a self-contained curriculum learning strategy to deal with the learning difficulty. Note that our curricular contrastive regularization is network-agnostic, making it applicable to other dehazing networks.
%\\
%\textbf{Physics-aware Dual-branch Unit.}
\subsection{Physics-aware Dual-branch Unit}
The atmospheric scattering model is commonly used to describe the formation of a hazy image $I$. It can be mathematically formulated as $I(x)=T(x)J(x)+(1-T(x))A$, where $J$ represents the clear image, $T$ is the transmission map, $A$ indicates the atmospheric light, and $x$ denotes the index of pixels. As both $T$ and $A$ are unknown, haze removal is a highly ill-posed problem. Raw space based methods directly estimate the two unknown factors, which can easily lead to cumulative errors. In contrast, imposing physics priors in the feature space can encourage the interpretability that aligns with the hazing process, without relying on the ground truths of $T$ and $A$. Inspired by FDU~\cite{dong2020physics}, we propose a physics-aware dual-branch Unit (PDU) that is derived from the physics model in the feature space, as shown in Fig.~\ref{fig:block}. 
\begin{figure}[t]
	\center
	\includegraphics[width=\linewidth]{fig/PDU.pdf}
	\caption{The architecture of the proposed PDU.}
	\label{fig:block}\vspace{-6mm}	
\end{figure}

To begin with, we reformulate the physics model to represent the clear image $J$ as follows:
\begin{equation}
	\begin{aligned}
		J(x)&=I(x)\frac{1}{T(x)}+A(1-\frac{1}{T(x)})\\
		&=I(x)\frac{1}{T(x)}+A-A\frac{1}{T(x)}.
	\end{aligned}
    \label{equ:scatter}
\end{equation}
Then extracting features via kernel $k$, Eq.~\eqref{equ:scatter} can be reformulated as follows:
\begin{equation}
	k\circledast J=k\circledast(I\odot\frac{1}{T})+k\circledast A-k\circledast(A\odot\frac{1}{T}), 
	\label{equ:conv}
\end{equation}
%Similar to the derivation in FDU, further
where $\circledast$ indicates the convolution operator and $\odot$ denotes the Hadamard product.  Consequently, we respectively introduce the matrix-vector forms of $k$, $J$, $I$, $A$, $\frac{1}{T}$, \ie, $\bm{K}$, $\bm{J}$, $\bm{I}$, $\bm{A}$ and $\bm{D}$, and Eq.~\eqref{equ:conv} can be rewritten as 
\begin{equation}
	\bm{KJ}=\bm{KDI}+\bm{KA}-\bm{KDA}.
\end{equation}
Such a reformulation can be given by a few steps of algebra operations. Note that the diagonal vector of the diagonal matrix $\bm{D}$ corresponds to the vectorized form of $\frac{1}{T}$.

Next, we can decompose the matrix $\bm{KD}$ into a product of two matrices $\bm{QK}$. As $\bm{K}$, $\bm{D}$ and $\bm{Q}$ are all unknown, implementing this decomposition can be indicated as solving an underdetermined system of equations, which can guarantee the existence of $\bm{Q}$. And then, we have
\begin{equation}	
	\bm{KJ}=\bm{Q}(\bm{KI})+\bm{KA}-\bm{Q}(\bm{KA}).
	\label{equ:Q}		
\end{equation}

We can denote $\tilde{\bm{A}}$ as an approximation of the features $\bm{KA}$ that correspond to the atmospheric light and $\tilde{\bm{t}}$ as an approximation of $\bm{Q}$, which is associated with the transmission map. Furthermore, $\bm{KI}$ and $\bm{KJ}$ can be viewed as the extracted features of a hazy image and its corresponding clear image, respectively. Based on Eq.~\eqref{equ:Q}, and assuming that the channel number of the features $\tilde{\bm{t}}$ matches that of the input features $\bm{M}$, we can calculate the physics-aware features $\tilde{\bm{J}}$ by 
\begin{equation}	
	\begin{aligned}
		\tilde{\bm{J}}&=\bm{M}\odot\tilde{\bm{t}}+\tilde{\bm{A}}-\tilde{\bm{A}}\odot\tilde{\bm{t}}\\		&=\bm{M}\odot\tilde{\bm{t}}+\tilde{\bm{A}}(\bm{1}-\tilde{\bm{t}}),
	\end{aligned}
    \label{equ:final}
\end{equation}
where $\bm{1}$ indicates a matrix whose elements are all ones. 

Note that the second term on the right-hand side of Eq.~\eqref{equ:final} involves a synergistic action between $\tilde{\bm{A}}$ and $\tilde{\bm{t}}$ that is ignored by FDU. Then we can explicitly build the PDU based on Eq.~\eqref{equ:final}. One branch in PDU (see the upper part of Fig.~\ref{fig:block}) is used to produce $\tilde{\bm{A}}$. As the atmospheric light is usually assumed to be homogeneous, we use global average pooling (GAP($\cdot$)) to eliminate unnecessary information in the feature space. And $\tilde{\bm{A}}$ is produced by
\begin{equation}	
	\tilde{\bm{A}}=H(\sigma(\textrm{Conv}^N(\textrm{ReLU}(\textrm{Conv}^{\frac{N}{8}}(\textrm{GAP}(\bm{M})))))),		
\end{equation}
where $\sigma(\cdot)$ is the Sigmoid function, $H(\cdot)$ denotes a replication operation, $\textrm{Conv}^N(\cdot)$ is the convolutional layer with $N$ kernels, and $N$ is set to 64.

On the other hand, we cannot apply GAP$(\cdot)$ for the approximation of $\bm{Q}$ due to a loss of information, as the transmission map is non-homogeneous. Therefore, in the lower branch in Fig.~\ref{fig:block}, we choose to extract $\tilde{\bm{t}}$ using a sequence of convolutional layers, which is given by
\begin{equation}	
	\tilde{\bm{t}}=\sigma(\textrm{Conv}^N(\textrm{ReLU}(\textrm{Conv}^{\frac{N}{8}}(\textrm{Conv}^{N}(\bm{M}))))).		
\end{equation}

With the proposed PDU, interpretable features $\tilde{\bm{J}}$ can be generated from the input features $\bm{M}$ for restoring hazy images. Unlike FDU, which uses a shared structure with GAP$(\cdot)$ to predict latent features that are simultaneously correlated to both $T$ and $A$, the PDU attentively incorporates the corresponding physical characteristics of these two factors. This approach allows for more useful features to be estimated in a dual interactive paradigm. 

%\textbf{Curricular Contrastive Regularization.}
\subsection{Curricular Contrastive Regularization} 
Regarding the canonical contrastive regularization for image dehazing, the anchor is the recovered result by the dehazing network, the positive is the ground truth, and the negatives include a hazy input and multiple hazy images that are non-consensual with the positive. The target of this regularization $R$ is to minimize the L1 distance between the embeddings of the anchor and the positive while maximizing their distance from the negatives, which is given by
\begin{equation}
	R=\sum_{i=1}^n\xi_i\frac{||V_i(J)-V_i(f(I,\theta))||_1}{\sum_{q=1}^r||V_i(U_q)-V_i(f(I,\theta))||_1+E_i},		
\end{equation}
where $E_i=||V_i(I)-V_i(f(I,\theta))||_1$, $f(\cdot,\theta)$ indicates the dehazing network with parameters $\theta$, $V_i(\cdot), i=1,2,\cdots,n$ extracts the $i$th hidden features from the pre-trained VGG-19~\cite{simonyan2014very}, the number of non-consensual negatives $\{U_q\}$ is $r$, and $\{\xi_{i}\}$ is the set of hyperparameters. As illustrated in Fig.~\ref{fig:CC}, the introduced contrast between the anchor and non-consensual negatives cannot provide a satisfactory lower bound of the solution space. The non-consensual negatives are typically distantly located from the positive, leading to an under-constricted solution space that limits the quality of the restorations.

Based on our analysis of Fig.~\ref{fig:teaser}, we propose a novel contrastive regularization for haze removal that utilizes negatives in the consensual space, which can be restored results from other dehazing models. Our straightforward aim is to push the anchor far away from better-quality negatives. However, two critical problems arise: 1) how to define the difficulty of different negatives and 2) how to arrange these negatives according to their difficulty during training. 
\begin{figure}[t]
	\center
	\includegraphics[width=\linewidth]{fig/CC.pdf}
	\caption{Illustration of curricular contrastive regularization.}
	\label{fig:CC}\vspace{-4mm}
\end{figure}

To solve both issues, we incorporate a curriculum learning strategy into contrastive regularization. We define the difficulty of the negatives into three levels: easy, hard, and ultra-hard. For easy negative, we use the hazy input consistently. The difficulty levels of the other negatives are dynamically determined during training. Specifically, we measure the average PSNR performance of the network before every epoch begins. In the $t$th epoch, a negative is defined as an ultra-hard sample when its PSNR is higher than the network performance, or as a hard negative otherwise. 

To properly arrange these negatives, we weigh them differently according to their difficulty levels. First, the weight of easy negative is fixed and largest. This is because although hard and ultra-hard negatives may contribute to a more compact solution space, they can also cause learning ambiguity. To ensure that the resultant force is towards the positive such that the anchor is shifted in the desired direction, we give the easy negative a weight that is large enough. In practice, we set this weight to the number of the non-easy negatives $z$. Second, the weight of a non-easy negative $S_q$ at the $t$th epoch is defined as follows: 
\begin{small}
\begin{equation}
	W_t(S_q) = \left\{
	\begin{array}{rcl}
		1+\gamma, &\textrm{avgPSNR}(f(\{I_g\},\theta_{t-1}))\geq \textrm{PSNR}(S_q),\\
		1-\gamma, & \textrm{otherwise},\\
	\end{array} \right. 
	\label{equ:beta}
\end{equation}
\end{small}
where $\{I_g\}$ denotes the hazy input dataset, $q=1,2,\cdots,z$ is the index of the non-easy negatives, and $\gamma$ is a hyperparameter. The weights of the hard and the ultra-hard negatives are set to $1+\gamma$ and $1-\gamma$, respectively. This means that the weight of a hard negative is larger than that of an ultra-hard negative, allowing the hard negative to provide a greater force and alleviating the potential learning ambiguity. Furthermore, the flexibility of this strategy in determining the difficulty levels enables ultra-hard negatives to become hard ones in the later stage of training (see Fig.~\ref{fig:CC}). This makes sense because as the quality of the anchor improves, the ambiguity caused by ultra-hard samples is reduced, and their importance should be strengthened. In this way, the hard and ultra-hard negatives can be viewed as better lower bounds for effectively constraining the solution space. Then, our curricular contrastive regularization $R^*$is formulated as follows:
\begin{small}
\begin{equation}
	R^*=\sum_{i=1}^n\xi_i\frac{||V_i(J)-V_i(f(I,\theta))||_1}{\sum_{q=1}^zW_t(S_q)||V_i(S_q)-V_i(f(I,\theta))||_1+z\cdot E_i}.
	\label{equ:R}	
\end{equation}
\end{small}

Finally, our total objective $\cal L$, which consists of an L1 norm based fidelity term and our contrastive curricular regularization, is given by
\begin{equation}
	{\cal L}=||J-f(I,\theta)||_1+\lambda R^*.	
\end{equation}
%\\


\subsection{Network Architecture}
\begin{table*}[t]
	\caption{Quantitative Evaluations with the state-of-the-art methods on the synthetic and real-world datasets.}
	\centering
	\small
	\begin{tabular}{c||c||c|c||c|c||c|c||c|c||c}
		\toprule
		\multirow{2}*{Method} &\multirow{2}*{Venue\&Year}&\multicolumn{2}{c||}{SOTS-indoor} &\multicolumn{2}{c||}{SOTS-outdoor} &\multicolumn{2}{c||}{Dense-Haze} &\multicolumn{2}{c||}{NH-Haze2} &\multirow{2}*{\#Params} \\
		\cmidrule(lr){3-4}
		\cmidrule(lr){5-6}
		\cmidrule(lr){7-8}
		\cmidrule(lr){9-10}		
		&&PSNR&SSIM&PSNR&SSIM&PSNR&SSIM&PSNR&SSIM&\\
		\midrule
		DCP~\cite{he2010single}&TPAMI2010&16.62&0.8179&19.13&0.8148&11.01&0.4165&11.68&0.6475&-\\
		
		DehazeNet~\cite{cai2016dehazenet}&TIP2016&21.14&0.8472&22.46&0.8514&9.48&0.4383&11.77&0.6217&0.01M\\
		
		AODNet~\cite{li2017aod}&ICCV2017&19.06&0.8504&20.29&0.8765&12.82&0.4683&12.33&0.6311&0.002M\\	
		
		DM2F-Net~\cite{Deng2019}&ICCV2019&34.29&0.9728&34.50&0.9815&14.99&0.5640&20.46&0.8217&92.14M\\
		
		GCANet~\cite{chen2019gated}&WACV2019&30.06&0.9596&22.76&0.8887&12.62&0.4208&18.79&0.7729&0.70M\\
		
		GDN~\cite{liu2019griddehazenet}&ICCV2019&32.16&0.9836&30.86&0.9819&14.96&0.5326&19.26&0.8046&0.96M\\	
		
		MSBDN~\cite{dong2020multi}&CVPR2020&32.77&0.9812&34.81&0.9857&15.13&0.5551&20.11&0.8004&31.35M\\	
		
		FFA-Net~\cite{qin2020ffa}&AAAI2020&36.39&0.9886&33.57&0.9840&12.22&0.4440&20.00&0.8225&4.46M\\	
		
		AECR-Net~\cite{wu2021contrastive}&CVPR2021&37.17&0.9901&-&-&15.80&0.4660&20.68&0.8282&2.61M\\
		
		MAXIM-2S~\cite{tu2022maxim}&CVPR2022&38.11&0.9908&34.19&0.9846&-&-&-&-&14.1M\\	
		
		DeHamer~\cite{guo2022image}&CVPR2022&36.63&0.9881&35.18&0.9860&16.62&0.5602&19.18&0.7939&132.45M\\	
		
		UDN~\cite{hong2022uncertainty}&AAAI2022&38.62&0.9909&34.92&0.9871&-&-&-&-&4.25M\\		
		\midrule
		\textbf{C$^2$PNet}   &&\textbf{42.56}&\textbf{0.9954}&\textbf{36.68}&\textbf{0.9900}&\textbf{16.88}&\textbf{0.5728}&\textbf{21.19}&\textbf{0.8334}&7.17M\\		
		\bottomrule
	\end{tabular}
	\label{tab:quantitative}
\end{table*}
\begin{figure*}[t]
	\centering
	%	\footnotesize
	\setlength{\abovecaptionskip}{0cm}
	\setlength{\tabcolsep}{0.05em}
	\setlength{\fboxrule}{1pt}
	\setlength{\fboxsep}{0pt}
	\begin{tabular}{cccccccc}			   		
		PSNR / SSIM& $18.09 /  0.7459 $ & $31.55 / 0.9793$ & $34.41 / 0.9811$ & $36.69 / 0.9838$ & $37.10 / 0.9825$ & $41.20 / 0.9914$ & $\infty / 1$ \\			
		\includegraphics[width=.12\linewidth]{fig/indoor/rect/hazy.png} &
		\includegraphics[width=.12\linewidth]{fig/indoor/rect/aod.png} &
		\includegraphics[width=.12\linewidth]{fig/indoor/rect/gdn.png} &
		\includegraphics[width=.12\linewidth]{fig/indoor/rect/ffa.png} &
		\includegraphics[width=.12\linewidth]{fig/indoor/rect/maxim.png} &
		\includegraphics[width=.12\linewidth]{fig/indoor/rect/dehamer.png} &
		\includegraphics[width=.12\linewidth]{fig/indoor/rect/ours.png}&
		\includegraphics[width=.12\linewidth]{fig/indoor/rect/clear.png}\\	
		\fcolorbox{red}{red}{\includegraphics[width=.117\linewidth]{fig/indoor/crop/hazy.png}} &
		\fcolorbox{red}{red}{\includegraphics[width=.117\linewidth]{fig/indoor/crop/aod.png}} &
		\fcolorbox{red}{red}{\includegraphics[width=.117\linewidth]{fig/indoor/crop/gdn.png}} &
		\fcolorbox{red}{red}{\includegraphics[width=.117\linewidth]{fig/indoor/crop/ffa.png}}&
		\fcolorbox{red}{red}{\includegraphics[width=.117\linewidth]{fig/indoor/crop/maxim.png}} &
		\fcolorbox{red}{red}{\includegraphics[width=.117\linewidth]{fig/indoor/crop/dehamer.png}}&
		\fcolorbox{red}{red}{\includegraphics[width=.117\linewidth]{fig/indoor/crop/ours.png}}&
		\fcolorbox{red}{red}{\includegraphics[width=.117\linewidth]{fig/indoor/crop/clear.png}}\\
		Hazy Image &AODNet~\cite{li2017aod}&GDN~\cite{liu2019griddehazenet}&FFA-Net~\cite{qin2020ffa}&MAXIM~\cite{tu2022maxim}&DeHamer~\cite{guo2022image}&C$^2$PNet (Ours)&GT
	\end{tabular}
	\caption{Visual results of SOTS-indoor dataset by different methods. (Zoom in for better view.)
	}
	\label{fig:indoor}
\end{figure*}

Our C$^2$PNet adopts an FFA-Net-like backbone because: 1) FFA-Net has a simple structure that cascades several FA blocks without any other redundant modules, and 2) the FA block is simple and has been proven to be practical. Since the proposed PDU mainly focuses on refining spatial information, we deploy it into each FA block by replacing the PA module. In this way, the features are enforced to conform to the hazing process before being fed into the subsequent module. Note that all other network parameters of C$^2$PNet are identical to those of FFA-Net, except for the PDUs.



\section{Hardware optimizations for edge deployment}
%\subsection{Model Performance Evaluation}

% \subsection{Critical Pruning Threshold Exploration}
%\subsection{Cumulative Sparsity Evaluation}

%\subsection{Variable Adapter Size Evaluation}

\subsection{Quantization}

While floating point values are still the most common data representation for DNN training, reduced-precision numerical formats are a good compromise for targeting efficient inference~\cite{hubaraBinarized2016,zhuTrained2017,reagenMinerva2016}, since quantized operands can significantly decrease data storage and movement costs, as well as reduce the complexity of the hardware used for implementing MAC processing units~\cite{carmichaelPerformanceEfficiency2019}. 
%By carefully balancing this trade-off, DNN models can be deployed more efficiently on energy- and storage-limited hardware without losing data accuracy. 
The vanilla ALBERT model has been demonstrated to benefit from quantization down to 16-bit floating-point without negatively impacting its inference performance~\cite{tambeEdgeBERT2021a}. 
For our adapter-ALBERT model, we hypothesize that, given the adapter module's proven ability to compensate for accuracy loss caused by pruning, it may also exhibit similar trends for quantization.

We have designed experiments with quantization configurations in 16-bit and 8-bit fixed-point representations using the CSP-both backbone. Using the conventional notation, $Q_{i,f}$, to represent the quantization scheme using $i$ bits for integer and sign, and $f$ fractional bits, we focus on $Q_{3, 13}$ and $Q_{3, 5}$.  
%Introducing a similar notation as the one used to denote the quantization scheme, we use an adapter size of $A_{64,64}$, where the two numbers in the suffix represent the size of the first and second adapter module in the transformer layer. 
To ensure data consistency, the adapter modules are quantized using the same settings as the backbone. The goal for these experiments is to show to what extent the ability of adapter modules in improving or maintaining the inference accuracy is limited by using lower precision quantization. Moreover, we want to verify if adapter modules can learn reduced data information and recover the loss via retraining of the task-specific parameters. 


\begin{figure}[t]
    \centering
    \includegraphics[width=8.5cm]{fig/qt.pdf}
    \caption{Quantization results on the CSP-all backbone. Although the $Q_{3,13}$ quantization example provides competitive results across all the considered datasets, reducing the number of bits per operand to $Q_{3,5}$ shows a drastic accuracy reduction for some of the QNLI and SST-2 tasks.}
    \label{fig:qt}
\end{figure}

%During the experiments, we have observed unacceptable accuracy degradation when applying quantization methods directly onto the model before inference. 
%As both the model's backbone layers and non-fixed layers are quantified, we speculate that as the information stored in the adapter modules' limited parameter matrices are full-precision data, the adapter modules are very sensitive to any data accuracy deduction caused by quantization if no extra actions are taken.
%We hence re-train the adapter modules while quantifying the entire model to ensure the adapter modules could receive and parameterize information from quantified matrices from the backbone layers.
As shown in Figure~\ref{fig:qt}, the $Q_{3,13}$ configuration provides competitive accuracy results when compared with the single-precision floating point (FP32) baseline. 
On the other hand, the results for $Q_{3, 5}$ quantization are much less consistent, suggesting that the optimal quantization scheme will dependent on the subset of tasks used by an application. 

\begin{table*}[ht]
    \caption{Memory Footprint Breakdown for Vanilla ALBERT and Adapter-ALBERT}
    \label{tab:mem-req}
    %\vskip 0.15in
    \begin{center}
    \begin{small}
    \begin{sc}
        \begin{tabular}{c|c|ccc}
            \toprule
            Model & Quant Config & MLC RRAM & SLC RRAM & SRAM  \\
            \midrule
            \multirow{3}{*}{Vanilla ALBERT}  & FP32 & 3.73MB   & 0.47MB & 9.03MB    \\
                                                & $Q_{3, 13}$ & 1.87MB   & 0.47MB & 4.52MB    \\
                                                & $Q_{3, 5}$ & 0.94MB    & 0.47MB & 2.26MB  \\
            \midrule
            \multirow{3}{*}{Adapter-ALBERT}  & FP32 & 11.13MB   & 1.4MB & 0.43MB    \\
                                                & $Q_{3, 13}$ & 5.57MB   & 1.4MB & 0.22MB    \\
                                                & $Q_{3, 5}$ & 2.79MB    & 1.4MB & 0.11MB  \\
            \bottomrule
        \end{tabular}
    \end{sc}
    \end{small}
    \end{center}
\end{table*}

\subsection{Bitmask Encoding}
Leveraging pruning to improve storage density requires an additional step in how the sparse matrices are mapped into the storage system. Several sparse encoding techniques have been proposed in the past and previous work has highlighted how critical it is to guarantee the robustness of these data structures against faults~\cite{pentecostMaxNVM2019}.  
Among the existing techniques, bitmask encoding is a lightweight approach that can be implemented with minimal encoding and decoding hardware overhead. The non-zero values from the sparse matrix are saved in an ordered array and their location is mapped to a binary matrix. At this point that we apply pruning exclusively to the backbone layers. For this reason, it would be advantageous to store the sparse layers in RRAM. While easy to implement, this solution is susceptible to large errors if any of the bits in the bitmask is flipped. To address this issue and preserve the advantage of the density of RRAMs, we split the bitmask and non-zero value data structures to SLC and MLC RRAM arrays respectively. 

\subsection{Accelerator architecture modeling}
To verify the expected improvements introduced by the adapter-ALBERT model optimizations and the associated on-chip memory architecture, we extrapolate the overall system area, energy, and latency by combining results from NVMExplorer~\cite{pentecostNVMExplorer2021} with a performance model tailored around the EdgeBERT accelerator specifications. In order to minimize the on-chip area overhead and be able to deploy our solution onto a mobile SoC, we select a combination of memory macros with different capacities so that the overall memory size is at the closest value exceeding the application requirements. 
When multiple combinations of memory macros are possible, we evaluate each proposal according to their latency and energy consumption. Since macros of different sizes will display different bandwidths, we compute the overall bandwidth as the weighted average of the bandwidth based on the macro's individual capacity. Off-chip memory access to DRAM are modeled using a similar approach to the one introduced in TETRIS~\cite{gaoTETRIS2017}. In our model, we ignore the energy contribution for the computational units, noting that, with the exception of the adapter computations, the two models will perform the same operations.


The baseline for our evaluations is based on the same accelerator and memory architecture, but for the latter, the capacity requirements consider a different model partition in which only the embedding parameters are stored in RRAM, while the rest of the data uses SRAM. As with adapter-ALBERT, the vanilla ALBERT case uses bitmask encoding for the sparse embeddings. The corresponding memory footprint for these two design options under different quantization configurations are shown in Table \ref{tab:mem-req}.
The model parameters partition of adapter-ALBERT introduces higher storage requirements for SLC and MLC RRAM compared to vanilla ALBERT. This is due to the fact that a larger number of parameters are shared across tasks and therefore are kept in the non-volatile portion of the memory. As a consequence, even though we introduce a parameter overhead associated with adapters, the SRAM storage is reduced by a larger factor. The overall effect is that we can take advantage of the denser RRAM storage for a larger portion of the model, leading to a smaller memory footprint.


The workload scenario we consider is that of a MTI application. The on-chip memory stores only the parameters required to process the current task, and the system performs a parameter update from DRAM whenever a new task needs to be executed. Therefore, we provision the on-chip memory footprint based on the worst-case scenario, \textit{i.e.}, the largest set of parameters for any of the given tasks. Note that this corresponds to the dataset with the largest number of classification labels for the vanilla ALBERT case, and in addition,  
%must consider the largest adapter size from applying VASE to the adapter-ALBERT case.
must consider the largest adapter size adopted by any of the target tasks when using VASE in the adapter-ALBERT case.

%By calculating the model's total number of parameters, removing parameters that are repeatedly reused, and considering sparsity and data representation formats, we obtain the actual memory capacity requirements for RRAMs and SRAM. We then use NVMExplorer to estimate their area, read and write bandwidth, energy, and latency. 

Under this strategy, we compare adapter-ALBERT and vanilla ALBERT's area, energy per inference, and latency per inference under three quantization configurations~\ref{fig:PPA}.
The results are normalized to the FP32 vanilla ALBERT design.
We can observe that for all configurations, the adapter-ALBERT provides all-round advantages against vanilla ALBERT.
% For 8-bit quantization configuration where the two models have the least gaps, the adapter-ALBERT requires 64.9\% area, 1.52\% energy per inference, and 6.37\% latency compared to the vanilla ALBERT model.
% For 32-bit and 16-bit configurations, the advantage of adapter-ALBERT with the memory architecture is more significant.
Compared to the FP32 implementation under a 3-task MTI scenario, adapter-ALBERT enjoys 2.04$\times$, 146.78$\times$, and 2.46$\times$ reductions in area, energy per inference, and latency per inference. The advantage is even more significant when it comes to 16-bit and 8-bit quantified comparisons. For instance, using the $Q_{3, 5}$ configuration, leads to 5.9$\times$, 682$\times$, and 62$\times$ improvements in area, energy and latency.


\begin{figure}[ht]
    \centering
    \includegraphics[width=0.48\textwidth]{fig/ppa.pdf}
    \caption{Area, energy/inference, and latency comparison between adapter-ALBERT and vanilla ALBERT using FP32, $Q_{3,13}$, and $Q_{3, 5}$ data types. The results are normalized to the FP32 vanilla ALBERT design.}
    \label{fig:PPA}
\end{figure}

\section{Conclusion and Future Work}
We present a framework for learning an auto-regressive motion planner (ARMP) to generate physically plausible motion plans for various tasks and environments. We describe how we construct a motion library using a trajectory optimization method. We then illustrate how the motion planner generates physically feasible trajectories via deep learning while following high-level commands using an autoregressive manner. Our evaluation results show that the proposed ARMP can produce a variety of physically feasible motions, including walking, turning, jumping, and stair climbing. Finally, we show that the learned motion planner can be used for complex indoor navigation tasks, such as navigating to goals on different floors or goals behind an obstacle that can not go around, which is not impossible for wheeled robots.

While our work shows promising results, it is still challenging to collect the required trajectories in the motion library due to the difficulty of tasks and the complexity of environments. For instance, it is not straightforward to solve a jumping motion using trajectory optimization, which makes us retarget the captured motion of a real dog. In another example, the environment can be too complex to consider all the possible scenarios, such as stepstones with various gaps or densely populated obstacles. We can approach these challenges by developing a better low-level trajectory optimization module or adaptively sampling the environments based on the learning progress.

If the motion library becomes larger and larger, our learning-based motion planner may not be possible to learn all the trajectories in the database. In this case, it may generate inaccurate trajectories with poor physics or diverged behaviors. One possible solution is to increase the size of the network or to introduce a new architecture that has better consideration of spatial and temporal relationships between frames. We will leave this to our future work.


\section*{Acknowledgement}

This work is supported by MORAI, Inc.

\bibliographystyle{IEEEtran} 
\bibliography{00.bibliography.bib.tex}  

% \input{00.bibliography.tex}

\end{document}
