\documentclass[reprint,superscriptaddress,aps]{revtex4-1}
%\usepackage{amsfonts}
\usepackage{amsmath}
\usepackage{amssymb}
\usepackage{bm}
\usepackage{braket}
\usepackage{color}
%usepackage[usenames,dvipsnames,svgnames,table]{xcolor}
\usepackage[varg]{txfonts}
%\usepackage{wrapfig}
\usepackage{varwidth}
\usepackage{dcolumn}
%\usepackage{mathrsfs}
\usepackage[breaklinks,colorlinks=true,linkcolor=blue,urlcolor=cyan,citecolor=blue]{hyperref}
%\usepackage{soul}
%\usepackage{ulem}
%\usepackage{times}
%\usepackage{subfigure}
\usepackage{here}
\usepackage{graphicx}

\begin{document}

\title{Breathing pyrochlore magnet CuGaCr$_{4}$S$_{8}$: Magnetic, thermodynamic, and dielectric properties}

\author{M. Gen}
\email{masaki.gen@riken.jp}
\affiliation{Department of Advanced Materials Science, The University of Tokyo, Kashiwa 277-8561, Japan}
\affiliation{RIKEN Center for Emergent Matter Science (CEMS), Wako 351-0198, Japan}

\author{H. Ishikawa}
\affiliation{Institute for Solid State Physics, The University of Tokyo, Kashiwa 277-8581, Japan}

\author{A. Miyake}
\affiliation{Institute for Solid State Physics, The University of Tokyo, Kashiwa 277-8581, Japan}

\author{T. Yajima}
\affiliation{Institute for Solid State Physics, The University of Tokyo, Kashiwa 277-8581, Japan}

\author{H. O. Jeschke}
\affiliation{Research Institute for Interdisciplinary Science, Okayama University, Okayama 700-8530, Japan}

\author{H. Sagayama}
\affiliation{Institute of Materials Structure Science, High Energy Accelerator Research Organization, Tsukuba 305-0801, Japan}

\author{A. Ikeda}
\affiliation{Institute for Solid State Physics, The University of Tokyo, Kashiwa 277-8581, Japan}
\affiliation{Department of Engineering Science, University of Electro-Communications, Chofu, Tokyo 182-8585, Japan}

\author{Y. H. Matsuda}
\affiliation{Institute for Solid State Physics, The University of Tokyo, Kashiwa 277-8581, Japan}

\author{K. Kindo}
\affiliation{Institute for Solid State Physics, The University of Tokyo, Kashiwa 277-8581, Japan}

\author{M. Tokunaga}
\affiliation{Institute for Solid State Physics, The University of Tokyo, Kashiwa 277-8581, Japan}

\author{Y. Kohama}
\affiliation{Institute for Solid State Physics, The University of Tokyo, Kashiwa 277-8581, Japan}

\author{T. Kurumaji}
\affiliation{Department of Advanced Materials Science, The University of Tokyo, Kashiwa 277-8561, Japan}

\author{Y. Tokunaga}
\affiliation{Department of Advanced Materials Science, The University of Tokyo, Kashiwa 277-8561, Japan}

\author{T. Arima}
\affiliation{Department of Advanced Materials Science, The University of Tokyo, Kashiwa 277-8561, Japan}
\affiliation{RIKEN Center for Emergent Matter Science (CEMS), Wako 351-0198, Japan}

%%%%%%%%%%%%%%%%%%%%%%%%%%%%%%%%%%%%%%%%%%%%%%%%%%%%%%%%%%
\begin{abstract}

We investigate the crystallographic and magnetic properties of a chromium-based thiospinel CuGaCr$_{4}$S$_{8}$.
From a synchrotron x-ray diffraction experiment and structural refinement, Cu and Ga atoms are found to occupy the tetrahedral {\it A}-sites in an alternate way, yielding breathing pyrochlore Cr network.
CuGaCr$_{4}$S$_{8}$ undergoes a magnetic transition associated with a structural distortion at 31~K in zero magnetic field, indicating that the spin-lattice coupling is responsible for relieving the geometrical frustration.
When applying a pulsed high magnetic field, a sharp metamagnetic transition takes place at 40~T, followed by a 1/2-magnetization plateau up to 103~T. 
These phase transitions accompany dielectric anomalies, suggesting the presence of helical spin correlations in low-field phases.
The density-functional-theory calculation reveals that CuGaCr$_{4}$S$_{8}$ is dominated by antiferromagnetic and ferromagnetic exchange couplings within small and large tetrahedra, respectively, in analogy with CuInCr$_{4}$S$_{8}$.
We argue that {\it A}-site-ordered Cr thiospinels serve as an excellent platform to explore diverse magnetic phases along with pronounced magnetoelastic and magnetodielectric responses.

\end{abstract}

\date{\today}
\maketitle
%%%%%%%%%%%%%%%%%%%%%%%%%%%%%%%%%%%%%%%%%%%%%%%%%%%%%%%%%%

\section{\label{Sec1}Introduction}

Magnetic materials of the pyrochlore lattice, a three-dimensional network of corner-sharing tetrahedra, have been a central research subject in the context of frustrated magnetism \cite{2010_Gar, 2021_Rei}.
In recent years, the {\it breathing} pyrochlore lattice, where up- and down-pointing tetrahedra differ in size, has attracted growing interest from the viewpoint of ground state control \cite{2013_Oka}.
The key concept of this spin model lies in the introduction of inequivalent exchange couplings $J$ and $J'$ in the small and large tetrahedra, respectively [Fig.~\ref{Fig1}(a)].
Depending on the signs and magnitudes of $J$ and $J'$ as well as the nature of spins, various exotic magnetic states and emergent phenomena have been theoretically predicted: e.g., unconventional spin-liquid states and excitations \cite{2015_Ben, 2016_Sav, 2016_Li, 2019_Iqb, 2020_Yan, 2022_Han}, spin-lattice-coupled superlattice long-range orders (LROs) \cite{2019_Aoy, 2021_Aoy}, and magnetic hedgehog-lattice \cite{2021_Aoy_HG, 2022_Aoy_HG}.

A representative realization of the breathing pyrochlore system is a quantum Heisenberg antiferromagnet Ba$_{3}$Yb$_{2}$Zn$_{5}$O$_{11}$, whose magnetism is governed by Yb$^{3+}$ ions with pseudospin-1/2 \cite{2014_Kim}.
In this compound, the breathing ratio $r'/r$, where $r$ ($r'$) represents the nearest-neighbor (NN) bond length in small (large) tetrahedra, amounts to approximately 2, resulting in $J \gg J' > 0$, i.e., close to the decoupled tetrahedron limit \cite{2014_Kim}.
The magnetization, specific heat, and inelastic neutron scattering suggested the singlet formation without any signs of a magnetic LRO at low temperatures \cite{2014_Kim, 2016_Hak}, though the development of intertetrahedral correlations is also pointed out \cite{2016_Rau, 2022_Dis}.

For the larger spin case, {\it A}-site-ordered chromium-based spinels {\it AA'}Cr$_{4}${\it X}$_{8}$, where Cr$^{3+}$ ions with spin-3/2 form a breathing pyrochlore lattice, offer a fertile playground to address an effective spin Hamiltonian with various sets of $J$ and $J'$ \cite{2019_Gho}.
Due to the difference in the ionic radius between {\it A}$^{+}$ and {\it A'}$^{3+}$ cations, their crystallographic ordering like the zinc-blende-type arrangement should modulate the chemical pressure and as a consequence induce the breathing bond-alternation in the Cr pyrochlore network [Fig.~\ref{Fig1}(b)].
This material design was first proposed by Joubert and Durif in 1966 \cite{1966_Jou}.
They prepared two oxides, LiGaCr$_{4}$O$_{8}$ and LiInCr$_{4}$O$_{8}$, and found the missing of an inversion center in their crystal structures, signaling the ordering of Li and Ga/In atoms.
Subsequently, Pinch {\it et al}. synthesized Cr thiospinels {\it AA'}Cr$_{4}$S$_{8}$ with various combinations of {\it A} and {\it A'} atoms: {\it A}=Li, Cu, Ag; {\it A'}=Al, Ga, In \cite{1970_Pin}.
Among them, the {\it A}-site ordering was confirmed for LiGaCr$_{4}$S$_{8}$, LiInCr$_{4}$S$_{8}$, CuAlCr$_{4}$S$_{8}$, and CuInCr$_{4}$S$_{8}$.
Structural refinements and detailed physical property measurements on these compounds have been actively performed in the last decade \cite{2015_Nil, 2016_Lee, 2016_Sah, 2018_Oka, 2018_Pok, 2019_Gen, 2020_Gen, 2020_Kan, 2020_Pok, 2021_Gao, 2022_Sha, 2022_Gen}, triggered by the renewed interest by Okamoto {\it et al.} in 2013 \cite{2013_Oka}.

It is noteworthy that a peculiar combination of antiferromagnetic (AFM) $J$ and ferromagnetic (FM) $J'$ can be realized in {\it AA'}Cr$_{4}${\it X}$_{8}$ due to the competition between the AFM Cr--Cr direct exchange and FM Cr--{\it X}--Cr superexchange interactions \cite{2019_Gho}.
Such spin Hamiltonian can be effectively mapped on the spin-6 Heisenberg antiferromagnet on the face-centered-cubic (FCC) lattice if $J'$ is strong.
Indeed, a cluster excitation was observed in CuInCr$_{4}$S$_{8}$ at low temperatures \cite{2021_Gao}, indicating the development of FM correlations within each large tetrahedron.
Another characteristic feature is the intrinsic strong spin-lattice coupling (SLC) arising from the sensitivity of the strength of $J$ ($J'$) against the NN Cr--Cr bond length, i.e., large $|dJ/dr|$ ($|dJ'/dr'|$).
The SLC can act as a principal perturbation to lift the macroscopic degeneracy and bring magnetostructural transitions at low temperatures \cite{2015_Nil, 2016_Lee, 2016_Sah, 2020_Kan} and in an applied magnetic field \cite{2019_Gen, 2020_Gen, 2021_Aoy}.
Interestingly, CuInCr$_{4}$S$_{8}$ exhibits a fascinating magnetic-field-versus-temperature ($H$-$T$) phase diagram including a robust 3-up-1-down phase associated with a 1/2-magnetization plateau \cite{2020_Gen} as well as a thermal equilibrium phase pocket \cite{2022_Gen} reminiscent of a skyrmion lattice \cite{2009_Muh, 2012_Sek, 2012_Oku}.

In this work, we report the structural, magnetic, thermodynamic, and dielectric properties of CuGaCr$_{4}$S$_{8}$.
Previously, CuGaCr$_{4}$S$_{8}$ was synthesized in Refs.~\cite{1970_Pin, 2005_Kes, 2018_Ami} but no conclusive remark on the {\it A}-site ordering was given because the close proximity of the scattering factors between Cu$^{+}$ and Ga$^{3+}$ made it challenging to judge the presence or absence of forbidden reflections for the {\it A}-site disordered case at the instrumental resolution of the time.
The present synchrotron x-ray diffraction (XRD) measurement provides for the first time a direct proof of the perfect crystallographic ordering of Cu and Ga atoms at the {\it A} site, confirming the formation of a breathing pyrochlore Cr network.
A series of pulsed high-field experiments reveal that CuGaCr$_{4}$S$_{8}$ exhibits a rich $H$-$T$ phase diagram similar to that CuInCr$_{4}$S$_{8}$ \cite{2020_Gen, 2022_Gen}.
The effective spin Hamiltonian of CuGaCr$_{4}$S$_{8}$ is discussed on the basis of the density-functional-theory (DFT) energy mapping as well as the magnetoelastic theory \cite{2020_Gen}.

\begin{figure}[t]
\centering
\includegraphics[width=\linewidth]{Fig1.eps}
\caption{(a) Schematic of a breathing pyrochlore lattice, where small and large tetrahedra with the nearest-neighbor bond lengths, $r$ and $r'$, are characterized by the exchange interactions, $J$ and $J'$, respectively. (b) Crystal structure of the {\it A}-site-ordered Cr spinels. The present focus is CuGaCr$_{4}$S$_{8}$, where nonmagnetic Cu$^{+}$ and Ga$^{3+}$ ions are found to be arranged in the zinc-blende-type structure, yielding a breathing pyrochlore Cr network. The illustrations are drawn with VESTA software \cite{2011_Mom}.}
\label{Fig1}
\end{figure}

\section{\label{Sec2}Methods}
Polycrystalline samples of CuGaCr$_{4}$S$_{8}$ were synthesized by the conventional solid-state reaction method.
Starting ingredients were high-purity gallium ingots (99.999\%) and copper (99.99\%), chromium (99.99\%), and sulfur (99.99\%) powders.
They were mixed in the stoichiometric ratio, sealed in an evacuated quartz tube, and heated at 400~$^{\circ}$C for 24~h and then at 800~$^{\circ}$C for 48~h in a box furnace.
Then, the sintering was repeated twice at 900~$^{\circ}$C for 96~h after grinding and pelletizing the sintered products.

Powder synchrotron x-ray diffraction (XRD) profile was collected at room temperature on Photon Factory BL-8A.
The wavelength was $\lambda = 0.689739~\AA$.
The Rietveld analysis was performed using the RIETAN-FP program \cite{RIETAN}.
The temperature evolution of the powder XRD pattern was measured between 4 and 300~K using a commercial x-ray diffractometer (SmartLab, Rigaku) at the Institute for Solid State Physics (ISSP), University of Tokyo.
The incident x-ray beam was monochromatized by a Johansson-type monochromator with a Ge(111) crystal to select only Cu-$K\alpha$1 radiation.

Magnetization up to 7~T was measured using a SQUID magnetometer (MPMS, Quantum Design).
Magnetization up to 14~T was measured using a vibrating sample magnetometer installed in a physical property measurement system (PPMS, Quantum Design) equipped with a superconducting magnet.
Magnetization up to 57~T was measured by the induction method in a non-destructive (ND) pulsed magnet ($\sim$4~ms duration).
Magnetization up to 140~T was measured by the induction method using a coaxial-type pickup coil in a horizontal single-turn-coil (STC) megagauss generator ($\sim$8~$\mu$s duration) \cite{2020_Gen}.
Thermal expansion was measured by the fiber-Bragg-grating (FBG) method using an optical sensing instrument (Hyperion si155, LUNA) in a cryostat equipped with a superconducting magnet (Spectromag, Oxford) at zero field.
Longitudinal magnetostriction up to 54~T was measured by the FBG method in a ND pulsed magnet ($\sim$36 ms duration), where by the optical filter method was employed to detect the relative sample-length change $\Delta L/L_{\rm 0T}$ \cite{2018_Ike}.
The fiber was adhered to a rod-shaped sintered sample with epoxy Stycast1266.
Dielectric constant at zero field was measured at a frequency of 10~kHz by using an LCR meter (E4980A, Agilent) in PPMS.
Dielectric constant along the field direction ($E \parallel B$) up to 48~T was measured at a frequency of 50~kHz by using a capacitance bridge (1615-A, General Radio) in a ND pulsed magnet ($\sim$36~ms duration) \cite{2020_Miy}.
Silver paste was painted on the two large surfaces of a disk-shaped sintered sample to form electrodes.
Heat capacity was measured by the thermal relaxation method in a PPMS at zero field.
All the pulsed high-field experiments were performed at ISSP.

In our DFT energy mapping \cite{2021_Yam, 2021_Hei, 2022_Her}, we worked with the full potential local orbital basis set \cite{1999_Koe} and generalized gradient approximation (GGA) type exchange and correlation functional \cite{1996_Per}.
Strong electronic correlations on Cr $3d$ orbitals were treated with DFT+U corrections \cite{1995_Lie} where we varied the onsite correlation strength $U$ and fix the Hund's rule coupling to $J_{H} = 0.72$~eV \cite{1996_Miz}.
We use a $2 \times 2 \times 1$ supercell of CuGaCr$_{4}$S$_{8}$ with $Pm$ symmetry and twelve inequivalent Cr$^{3+}$ positions to extract the exchange couplings up to the fifth NN.

\section{\label{Sec3}Basic physical properties}

\subsection{\label{Sec3_1} Structural analysis}

We first show evidence of the ordering of Cu and Ga atoms in CuGaCr$_{4}$S$_{8}$, like the previously reported Li(Ga, In)Cr$_{4}$S$_{8}$ and Cu(Al, In)Cr$_{4}$S$_{8}$ \cite{2018_Oka, 2022_Sha}. 
Figure~\ref{Fig2} shows a synchrotron powder XRD pattern at room temperature.
In addition to the Bragg reflections for the normal spinel structure with the $Fd{\overline 3}m$ space group, weak reflections allowed for $F{\overline 4}3m$, such as 200 and 640, are observed, as shown in the insets.
Rietveld refinements with changing initial parameters confirm the full occupation of Cu and Ga at the $4a$ and $4d$ sites, respectively, with reasonable thermal parameters and small reliability factors.
The crystallographic parameters are summarized in Table~\ref{tab:Lietveld}.
The lattice constant is $a=9.91840(3)~\AA$, in accord with Ref.~\cite{1970_Pin}.
The NN Cr--Cr distance is $r=3.375(4)~\AA$ and $r'=3.639(4)~\AA$.
The breathing ratio $r'/r=1.078$ is larger than those in oxides, 1.035 (LiGaCr$_{4}$O$_{8}$) and 1.049 (LiInCr$_{4}$O$_{8}$) \cite{2013_Oka}, and comparable to those in other sulfides, 1.074 (LiGaCr$_{4}$S$_{8}$), 1.089 (LiInCr$_{4}$S$_{8}$) \cite{2018_Oka}, 1.066 (CuAlCr$_{4}$S$_{8}$) \cite{2022_Sha}, and 1.084 (CuInCr$_{4}$S$_{8}$) \cite{2022_Gen}. 

\begin{figure}[t]
\centering
\includegraphics[width=0.95\linewidth]{Fig2.eps}
\caption{Synchrotron XRD pattern of CuGaCr$_{4}$S$_{8}$ powder sample (red open circle) and calculated XRD pattern obtained by the Rietveld analysis (black solid line). The Rietveld refinement was performed in a range of $10^{\circ} < 2\theta < 120^{\circ}$, where several tiny peaks originating from unknown impurity phases are excluded. Blue vertical bars indicate the nuclear Bragg reflections, and the green line is the difference between the experimental and calculated patterns. Insets show enlarged views of the experimental XRD profiles focusing on 200 and 640 peaks, which are forbidden for $Fd{\overline 3}m$ but allowed for $F{\overline 4}3m$.}
\label{Fig2}
\end{figure}

\begin{table}[t]
\renewcommand{\arraystretch}{1.2}
\caption{Structural parameters of CuGaCr$_{4}$S$_{8}$ at room temperature obtained from the Rietveld analysis. The space group is $F{\overline 4}3m$. The lattice constant is $a=9.91840(3)$~\AA. Reliability factors are $R_{\rm wp}=2.853$, $R_{\rm p}=1.851$, $R_{\rm e}=1.737$, and $S=1.643$.}
\begin{tabular}{ccccccc} \hline\hline
~ & ~ & $x$ & $y$ & $z$ & ~Occup.~ & B (\AA$^{2}$) \\ \hline
~~~Cu~~~ & ~~$4a$~~ & 0 & 0 & 0 & 1 & ~~0.875(95)~~~ \\
~~~Ga~~~ & ~~$4d$~~ & 3/4 & ~~3/4~~ & ~~3/4~~ & 1 & ~~0.835(78)~~~ \\
~~~Cr~~~ & ~~$16e$~~ & ~~0.37033(15)~~ & ~~$x$~~ & ~~$x$~~ & 1 & ~~0.614(22)~~~ \\
~~~S1~~~ & ~~$16e$~~ & ~~0.13305(28)~~ & ~~$x$~~ & ~~$x$~~ & 1 & ~~0.878(51)~~~ \\
~~~S2~~~ & ~~$16e$~~ & ~~0.61709(23)~~ & ~~$x$~~ & ~~$x$~~ & 1 & ~~0.636(48)~~~ \\ \hline\hline
\end{tabular}
\label{tab:Lietveld}
\end{table}

\subsection{\label{Sec3_2} Magnetic and structural transitions at low temperatures}

\begin{figure}[t]
\centering
\includegraphics[width=0.85\linewidth]{Fig3.eps}
\caption{Temperature dependence of (a) magnetic susceptibility $M/H$ at 1 T and (b) heat capacity divided by temperature $C/T$ at 0 T. The inverse magnetic susceptibility $H/M$ and the Curie-Weiss fit above $T^{*} = 160$~K (black) are displayed in the right axis of (a). The solid line in (b) denotes the estimated lattice heat capacity based on the Debye model with the Debye temperature of $\Theta_{\rm D}=430$~K.}
\label{Fig3}
\end{figure}

Figures~\ref{Fig3}(a) and \ref{Fig3}(b) show the magnetic susceptibility $M/H$ measured at 1 T and the heat capacity divided by temperature $C/T$ measured at 0 T as a function of temperature, respectively.
The inverse susceptibility $H/M$ exhibits linear temperature dependence between 160 and 300~K [right axis of Fig.~\ref{Fig3}(a)], following the Curie-Weiss law with the Weiss temperature of $\Theta_{\rm W}=-103$~K and the effective moment $p_{\rm eff}=4.08$~$\mu_{\rm B}$.
The large negative $\Theta_{\rm W}$ indicates dominant AFM exchange couplings, like Cu(Al, In)Cr$_{4}$S$_{8}$ \cite{2022_Sha, 2022_Gen} but unlike Li(Ga, In)Cr$_{4}$S$_{8}$ \cite{2018_Oka, 2018_Pok}.
The estimated $p_{\rm eff}$ is slightly larger than the theoretical value of 3.87~$\mu_{\rm B}$ expected for $S=3/2$ with quenched orbital moments, ensuring a nearly isotropic spin for CuGaCr$_{4}$S$_{8}$.
In order to estimate lattice contributions to $C/T$, we employ the Debye model with the Debye temperature of $\Theta_{\rm D}=430$~K as shown by a black line in Fig.~\ref{Fig3}(b).
The calculated curve fits well to the experimental data above $\sim$120~K.

Remarkably, $M/H$ exhibits an anomaly at $T^{*} \approx 158$~K, below which $H/M$ deviates upward from the linear temperature dependence [Fig.~\ref{Fig3}(a)].
This suggests the onset of AFM short-range correlation.
The similar behavior was also observed for CuInCr$_{4}$S$_{8}$ \cite{2018_Oka}.
Although $C/T$ does not exhibit any visible anomalies around $T^{*}$, it deviates from the estimated lattice heat capacity below $\sim$120~K, suggesting magnetic contributions.
This also supports the development of the magnetic short-range order in this temperature range.
On further cooling, an abrupt $M/H$ drop and a sharp $C/T$ peak are observed at $T_{\rm N}=31$~K, indicating the onset of a magnetic LRO.
The transition temperature is consistent with the previously reported values \cite{1970_Pin, 2005_Kes, 2018_Ami}.

\begin{figure}[t]
\centering
\includegraphics[width=0.85\linewidth]{Fig4.eps}
\caption{Temperature dependence of (a) magnetic susceptibility $M/H$ at 7 T, (b) magnetic heat capacity divided by temperature $C_{\rm mag}/T$ at 0 T, (c) thermal expansion $\Delta L/L_{\rm 2K}$ at 0 T, and (d) dielectric constant $\varepsilon'$ at a frequency of 10 kHz at 0 T. The insets of (a) and (d) show enlarged views of $M/H$ and $\varepsilon'$ around $T_{\rm N}$, respectively. The magnetic entropy $S_{\rm mag}$ calculated by integrating $C_{\rm mag}/T$ with respect to temperature is displayed in the right axis of (b). Black arrows represent the direction of the temperature-sweeping process.}
\label{Fig4}
\end{figure}

Figure~\ref{Fig4} summarizes the temperature dependence of various physical quantities in the vicinity of $T_{\rm N}$.
As shown in the inset of Fig.~\ref{Fig4}(a), $M/H$ shows a clear hysteresis, indicating that the magnetic transition is of the first order.
Figure~\ref{Fig4}(b) shows the magnetic heat capacity divided by temperature $C_{\rm mag}/T$, which is obtained by subtracting the estimated lattice contributions from the experimental $C/T$ as shown in Fig.~\ref{Fig3}(b).
By integrating $C_{\rm mag}/T$ with respect to temperature, the magnetic entropy $S_{\rm mag}$ is found to reach $\sim$5~J/(K~mol-Cr) just above $T_{\rm N}$ \cite{comment}.
Note that we obtain $S_{\rm mag}$ at $T^{*}$ to be $\sim$13~J/(K~mol-Cr)  (not shown), which roughly agrees with the theoretical value $R{\rm ln}4$ = 11.5~J/(K~mol-Cr) for the $S = 3/2$ spin system.
Notably, the thermal expansion $\Delta L/L_{\rm 2K}$ rapidly decreases below 35~K with decreasing temperature, suggesting a significant volume contraction across $T_{\rm N}$.
This behavior is similar with Li(Ga,In)Cr$_{4}$O$_{8}$ \cite{2020_Kan}, where a crystal symmetry lowering was observed \cite{2015_Nil, 2016_Sah}.
As shown next, we confirm the crystal symmetry lowering for CuGaCr$_{4}$S$_{8}$ from the powder XRD measurement at low temperatures.
Moreover, dielectric constant $\varepsilon'$ exhibits a steplike anomaly at $T_{\rm N}$ [inset of Fig.~\ref{Fig4}(d)] as observed in Li(Ga,In)Cr$_{4}$O$_{8}$ \cite{2016_Lee, 2016_Sah}.
A previous powder neutron diffraction study \cite{1976_Wil} proposes that the magnetic structure below $T_{\rm N}$ is an incommensurate spiral state.
We hence infer that the observed dielectric anomaly is of the magnetic origin; in other words, CuGaCr$_{4}$S$_{8}$ could be a type-II multiferroic \cite{2009_Kho}.

To reveal the structural change across the magnetic transition, we investigate the powder XRD patterns at low temperatures.
Figure~\ref{Fig5} shows the temperature evolution of peak profile of some Bragg reflections.
Peak splitting is clearly observed for many reflections below $T_{\rm N}$, signaling that the magnetic transition accompanies a structural transition.
Remarkably, $hhh$ reflections split into two peaks whereas no splitting or broadening is observed for $h00$ reflections [Figs.~\ref{Fig5}(b) and \ref{Fig5}(c)].
These observation can be accounted for by rhombohedral distortion, as opposed to tetragonal or orthorhombic distortion in other Cr spinels \cite{2006_Ued, 2007_Lee, 2009_Yok, 2009_Kan, 2015_Nil, 2016_Sah}.
However, the proposed magnetic modulation vector ${\mathbf Q} = (0.18, 0, 0.80)$ \cite{1976_Wil} may not directly cause rhombohedral distortion, but is compatible with monoclinic distortion.
Indeed, the peak profile of 440 reflection below $T_{\rm N}$ can be well fitted by the superposition of three Lorentzian functions rather than two [see Fig.~\ref{Fig5}(a) and Appendix~\ref{SecA}].
This suggests that the crystal symmetry below $T_{\rm N}$ is lower than rhombohedral.
More detailed crystallographic and magnetic structure analysis is necessary to settle the issue.

\begin{figure}[t]
\centering
\includegraphics[width=\linewidth]{Fig5.eps}
\caption{Waterfall plots of the temperature evolution of the powder XRD pattern focusing on (a) 440, (b) 444, and (c) 800 reflections indexed for the cubic $F{\overline 4}3m$ space group. The data were obtained using a laboratory x-ray diffractometer with monochromatized Cu-$K\alpha$1 radiation. Triangles denote the peak positions obtained by the (multi-) Lorentzian fit to each peak profile.}
\label{Fig5}
\end{figure}

\subsection{\label{Sec3_3} DFT calculation}

Based on the structural parameters at room temperature shown in Table~\ref{tab:Lietveld}, we performed the DFT calculation to estimate the exchange couplings of CuGaCr$_{4}$S$_{8}$.
Figure~\ref{Fig6}(b) shows the DFT energy mapping up to the fifth-NN exchange couplings, which are defined in the Heisenberg Hamiltonian of the form ${{\mathcal{H}}}=\sum_{i<j}J_{ij}{\mathbf S}_{i} \cdot {\mathbf S}_{j}$ [see Fig.~\ref{Fig6}(a)].
Exchange interactions monotonically evolve with onsite Coulomb interaction strength $U$.
The vertical dashed line indicates the $U$ value for which the exchange couplings match the experimental Weiss temperature of $\Theta_{\rm W} = -103$~K.
The obtained parameter set is 
$J = 11.6(7)$~K, $J' = -11.0(6)$~K, $J_{2} = 2.0(5)$~K, $J_{3a} = 5.7(3)$~K, $J_{3b} = 4.3(3)$~K, $J_{4} = -1.1(4)$~K, and $J_{5} = 0.3(2)$~K, representing that CuGaCr$_{4}$S$_{8}$ is characterized by strong AFM $J$ and FM $J'$.

\begin{figure}[t]
\centering
\includegraphics[width=\linewidth]{Fig6.eps}
\caption{(a) Definition of the exchange couplings up to the fifth-NN path in the breathing pyrochlore lattice. (b) Exchange parameters of CuGaCr$_{4}$S$_{8}$ at room temperature obtained by the DFT energy mapping as function of the onsite interaction strength $U$. The corresponding Weiss temperature is denoted by crosses in the right axis. The vertical line indicates the $U$ value where the exchange couplings match the experimental Weiss temperature $\Theta_{\rm W} = -103$~K.}
\label{Fig6}
\end{figure}

\begin{table}[t]
\renewcommand{\arraystretch}{1.2}
\caption{Exchange couplings up to third-NN in four kinds of {\it A}-site-ordered Cr thiospinels estimated by the DFT energy mapping. In the calculation for CuGaCr$_{4}$S$_{8}$, we find that $J_{4}$ and $J_{5}$ are negligibly weak (see text for details).}
\begin{tabular}{ccccccc} \hline\hline
& $J$ & $J'$ & $J_{2}$ & $J_{3a}$ & $J_{3b}$ & Ref. \\
\hline
LiGaCr$_{4}$S$_{8}$~ & ~$-7.7$~K~ & ~$-12.2$~K~ & ~1.2~K~ & ~6.1~K~ & ~3.0~K~ & ~\cite{2019_Gho} \\
LiInCr$_{4}$S$_{8}$~ & ~$-0.3$~K~ & ~$-28.0$~K~ & ~0.7~K~ & ~5.3~K~ & ~2.4~K~ & ~\cite{2019_Gho} \\
CuGaCr$_{4}$S$_{8}$~ & ~11.6~K~ & ~$-11.0$~K~ & ~2.0~K~ & ~5.7~K~ & ~4.3~K~ & ~This work \\
CuInCr$_{4}$S$_{8}$~ & ~14.7~K~ & ~$-26.0$~K~ & ~1.1~K~ & ~6.4~K~ & ~4.5~K~ & ~\cite{2019_Gho} \\ \hline\hline
\end{tabular}
\label{tab:J}
\end{table}

Table~\ref{tab:J} compares the exchange parameters between four kinds of {\it A}-site-ordered Cr thiospinels estimated in the previous works \cite{2019_Gho} and this work.
One can find that $J$ and $J'$ are strongly dependent on the  types of nonmagnetic cations.
The occupation of Li atoms at the $4a$ site leads to FM $J$, whereas that of Cu atoms leads to AFM $J$.
$J'$ is always FM, and its strength is enhanced when In atoms occupy the $4d$ site.
These tendencies are reasonable because monovalent {\it A}$^{+}$ (trivalent {\it A'}$^{3+}$) cations are surrounded by S1 (S2) atoms connecting the short (long) NN Cr--Cr bonds [Fig.~\ref{Fig1}(b)].
The Hamiltonian of CuGaCr$_{4}$S$_{8}$ is qualitatively similar to that of CuInCr$_{4}$S$_{8}$ except that $|J'|$ is much smaller.
In other words, the effects of further-neighbor interactions, especially $J_{3a}$ and $J_{3b}$, are more important for CuGaCr$_{4}$S$_{8}$ than for CuInCr$_{4}$S$_{8}$.
This may be responsible for the difference in the ground state at zero field; a commensurate ${\mathbf Q} = (1, 0, 0)$ state with an $S=6$ spin cluster in the large tetrahedron is realized for CuInCr$_{4}$S$_{8}$ \cite{2021_Gao}, whereas an incommensurate spiral state in which four spins in the large tetrahedra are not parallel with each other for CuGaCr$_{4}$S$_{8}$ \cite{1976_Wil}.

\section{\label{Sec4} Magnetic-field induced phase transitions}

\begin{figure}[t]
\centering
\includegraphics[width=\linewidth]{Fig7.eps}
\caption{(a)(b) Magnetic-field dependence of (a) magnetization $M$ and (b) its field derivative $dM/dH$ measured at various initial temperatures $T_{\rm ini}$ in a non-destructive (ND) pulsed magnet. Thick (thin) lines correspond to the data in field-increasing (decreasing) processes. The curves except for $T_{\rm ini}=1.4$~K are shifted upward for clarity. The inset of (b) is an enlarged view of $dM/dH$ around the lowest-field phase transition for $T_{\rm ini}=1.4$~K. (c) Magnetic-field dependence of $M$ (left) and $dM/dH$ (right) in the field-increasing process measured at $T_{\rm ini} \sim 5$~K in a single-turn-coil (STC) system. The absolute value of $M$ is calibrated by fitting with the $M$--$H$ curve for $T_{\rm ini} = 4.2$~K obtained in a ND pulsed magnet (gray).}
\label{Fig7}
\end{figure}

\subsection{\label{Sec4_1} Magnetization curves}

We here focus on the field-induced properties of CuGaCr$_{4}$S$_{8}$ revealed by pulsed high-field experiments.
Figure~\ref{Fig7}(a) shows magnetization curves measured at various initial temperatures $T_{\rm ini}$ using the ND pulsed magnet.
Although they appear simple at a glance, anomalies in the field derivatives $dM/dH$ show a complicated temperature dependence as shown in Fig.~\ref{Fig7}(b).
For $T_{\rm ini}=4.2$~K, a drastic magnetization jump is observed at around 40~T accompanied by a large hysteresis, where $dM/dH$ exhibits a sharp peak at $\mu_{0}H_{\rm c1}=40.4$~T and a shoulder-like anomaly at $\mu_{0}H_{\rm c2}=42.1$~T in the field-increasing process.
The shoulder-like anomaly does not change its position while the peak moves to a lower field at 37.4~T in the field-decreasing process.
After the metamagnetic transition, $M$ reaches $\sim$1.2~$\mu_{\rm B}$/Cr, which is much smaller than the expected saturation value of $\sim$3~$\mu_{\rm B}$/Cr.
As $T_{\rm ini}$ increases, $H_{\rm c1}$ shifts to a lower field and the hysteresis becomes smaller.
Notably, a broad shoulder-like structure appears on the low-field side of the $dM/dH$ peak at $H_{\rm c1}'$, as denoted by open triangles in Fig.~\ref{Fig7}(b).
As shown in Sec.~\ref{Sec4_2}, this magnetization anomaly is accompanied by a pronounced dielectric response.
Even for $T_{\rm ini}=36$~K ($>T_{\rm N}$), a weak metamagnetic transition from the paramagnetic phase is observed at $\mu_{0}H_{\rm p}=38.3$~T, indicating that the field-induced phase is robust against thermal fluctuations.

For all the measured $T_{\rm ini}$'s below $T_{\rm N}$, a subtle slope change in the magnetization curve is observed at $\mu_{0}H_{\rm c0} \approx 10$~T, which is visible as a $dM/dH$ cusp [inset of Fig.~\ref{Fig7}(b)].
This phase transition is also confirmed by the magnetization measurement up to 14~T in a static magnetic field (see Fig.~\ref{Fig10} in Appendix~\ref{SecB}).
We ascribe these anomalies to a spin-flop transition with the reorientation of magnetic domains of the helical state, as observed for CdCr$_{2}$O$_{4}$ \cite{2007_Mat_1, 2011_Bha, 2019_Ros, 2020_Ros}.
For $T_{\rm ini}=1.4$~K, $dM/dH$ exhibits two peaks around 40~T in the field-increasing process [Fig.~\ref{Fig7}(b)].
We tentatively view this behavior as a splitting of the peak at $H_{\rm c1}$ observed for $T_{\rm ini} \geq 4.2$~K, though its origin is unclear at present.

To get a whole picture of the field-induced phase transitions, we further measure the magnetization at $T_{\rm ini} \sim 5$~K up to $\sim$140~T using the STC system.
As shown in Fig.~\ref{Fig7}(c), a plateau-like behavior is observed between $\mu_{0}H_{\rm c1}=40$~T and $\mu_{0}H_{\rm c3}=103$~T in the field-increasing process.
Note that the transition at $H_{\rm c2}$ is not resolved due to an electromagnetic noise.
Judging from the magnitude of $M$ in this field range ($\sim$1.5), a 3-up-1-down state with a 1/2-magnetization plateau is expected to appear as in Cr spinel oxides \cite{2006_Ued, 2008_Koj, 2011_Miy_JPSJ, 2014_Miy, 2019_Gen, 2007_Mat_2, 2010_Mat} and CuInCr$_{4}$S$_{8}$ \cite{2020_Gen}.
Above $H_{\rm c3}$, $M$ rapidly increases up to the applied maximum field of 140~T, where $M$ reaches $\sim$2.8~$\mu_{\rm B}$/Cr.
This indicates that the saturation field is a bit higher than 140~T.

\begin{table}[t]
\renewcommand{\arraystretch}{1.2}
\caption{Critical fields of the successive phase transitions in CuGaCr$_{4}$S$_{8}$ and CuInCr$_{4}$S$_{8}$ at $\sim$5~K. $H_{\rm c1} \sim H_{\rm c4}$ correspond to the termination fields of {\it X}, {\it Y}, {\it C}, and {\it C'} phases, which can be assigned to canted 2:2, canted 2:1:1, 3-up-1-down, and canted 3:1 phases, respectively, based on the magnetoelastic theory \cite{2020_Gen}, though the real magnetic structures in the canted phases would be incommensurate for CuGaCr$_{4}$S$_{8}$.}
\begin{tabular}{ccccccc} \hline\hline
& $H_{\mathrm{c0}}$ & $H_{\mathrm{c1}}$ & $H_{\mathrm{c2}}$ & $H_{\mathrm{c3}}$ & $H_{\mathrm{c4}}$ & Ref. \\
\hline
CuGaCr$_{4}$S$_{8}$~ & ~10.8~T~ & ~40.4~T~ & ~42.1~T~ & ~103~T~ & ~$>140$~T~ & ~This work \\
CuInCr$_{4}$S$_{8}$~ & ~---~ & ~32~T~ & ~56~T~ & ~112~T~ & ~136~T~ & ~\cite{2020_Gen, 2022_Gen} \\ \hline\hline
\end{tabular}
\label{tab:MH}
\end{table}

The overall magnetization curve of CuGaCr$_{4}$S$_{8}$ at 4.2 or 5~K is similar to that of CuInCr$_{4}$S$_{8}$ \cite{2020_Gen, 2022_Gen}.
For both the compounds, an intermediate-field phase appears just below the 1/2-magnetization plateau.
The magnetic structure in the intermediate-field phase is expected to be a canted 2:1:1 state according to the magnetoelastic theory assuming the effective $S=6$ FCC-lattice model \cite{2020_Gen}, which predicts successive phase transitions from a canted 2:2 to canted 2:1:1, 3-up-1-down, canted 3:1, and a fully polarized phase.
Note that a canted 2:1:1 phase is also observed for ZnCr$_{2}$O$_{4}$ \cite{2011_Miy_JPSJ} and MgCr$_{2}$O$_{4}$ \cite{2014_Miy} in a narrow field range.
An incommensurate spiral component would coexist in the canted spin states in CuGaCr$_{4}$S$_{8}$ due to the presence of sizable further-neighbor exchange couplings and/or the DM interaction, as proposed for CdCr$_{2}$O$_{4}$ \cite {2007_Mat_1, 2020_Ros}.
In the following, we call the field-induced phases of CuGaCr$_{4}$S$_{8}$ at 5~K {\it X}, {\it Y}, {\it C}, and {\it C'} phases in the ascending order of the field [Fig.~\ref{Fig7}(c)].
Table~\ref{tab:MH} summarizes the critical field of each phase transition in the field-increasing process for CuGaCr$_{4}$S$_{8}$ and CuInCr$_{4}$S$_{8}$ \cite{2020_Gen, 2022_Gen}.
Theoretically, as the SLC gets stronger, the 1/2-magnetization plateau expands while the canted 2:1:1 phase gets narrower \cite{2020_Gen}.
Considering $H_{\rm c3}/H_{\rm c2} \sim 2.5$ and 2.0 for CuGaCr$_{4}$S$_{8}$ and CuInCr$_{4}$S$_{8}$, respectively, it can be said that the 3-up-1-down state is more stable in CuGaCr$_{4}$S$_{8}$ than in CuInCr$_{4}$S$_{8}$.
Besides, the field range between $H_{\rm c1}$ and $H_{\rm c2}$ ({\it Y} phase) is much narrower in CuGaCr$_{4}$S$_{8}$ than in CuInCr$_{4}$S$_{8}$.
These trends suggest that the SLC is stronger in CuGaCr$_{4}$S$_{8}$ than in CuInCr$_{4}$S$_{8}$ \cite{2020_Gen}.

\begin{figure}[t]
\centering
\includegraphics[width=\linewidth]{Fig8.eps}
\caption{Magnetic-field dependence of relative changes in (a) the sample length along the field direction $\Delta L/L_{\rm 0T}$ and (b) dielectric constant $\Delta \varepsilon'/\varepsilon'$ measured at various initial temperatures $T_{\rm ini}$. Data presentation is the same as Fig.~\ref{Fig7}.}
\label{Fig8}
\end{figure}

\subsection{\label{Sec4_2} Magnetostrictive and magnetodielectric effects}

Figure~\ref{Fig8}(a) shows the field dependence of the longitudinal magnetostriction and the longitudinal dielectric constant, respectively, measured on a sintered sample at various $T_{\rm ini}$'s using the ND pulsed magnet. 
Here, $\Delta L/L_{\rm 0T}$ and $\Delta \varepsilon'/\varepsilon'_{\rm 0T}$ represent the relative changes normalized by the zero-field values at each $T_{\rm ini}$.
For $T_{\rm ini}=36$~K, $\Delta L/L_{\rm 0T}$ shows a parabolic field dependence at low fields as expected in the paramagnetic state.
For $T_{\rm ini}$ below $T_{\rm N}$, on the other hand, $\Delta L/L_{\rm 0T}$ remains almost constant or exhibits negative magnetostriction in the spiral phase and {\it X} phase.
This tendency is in contrast to Li(Ga, In)Cr$_{4}$S$_{8}$ \cite{2020_Kan} and CuInCr$_{4}$S$_{8}$ \cite{2022_Gen}, where $\Delta L/L_{\rm 0T}$ gradually increases when applying a magnetic field.
A slight slope change at around 10~T observed for $12 \leq T_{\rm ini} \leq 28$~K would reflect the spin-flop transition at $H_{\rm c0}$.
On entering the 1/2-magnetization plateau above $H_{\rm c1}$, a drastic lattice expansion is observed in analogy with other Cr spinels \cite{2022_Gen, 2007_Tan, 2019_Ros}, suggesting that the 3-up-1-down collinear state is stabilized by the exchange striction.

Interestingly, diverse dielectric responses are observed as shown in Fig.~\ref{Fig8}(b).
For $T_{\rm ini}=1.4$ and 4.2~K, $\Delta \varepsilon'/\varepsilon'_{\rm 0T}$ shows a sharp peak with a magnitude of $\sim$4~\% between {\it X} and {\it Y} phases at $H_{\rm c1}$.
As $T_{\rm ini}$ increases toward $T_{\rm N}$, the dielectric anomaly around $H_{\rm c1}$ transforms into a broad valley shape with a reduction of as large as $\sim$8~\%.
For $T_{\rm ini}=12$~K, a valley structure is seen only in the field-decreasing process, presumably suggesting that the actual sample temperature would be higher than $T_{\rm ini}$ in the field-decreasing process due to the magnetocaloric effect \cite{2022_Gen, 2022_Kim}.
Of particular note is the 20-K data, in which a tiny $\Delta \varepsilon'/\varepsilon'$ peak coexist with the valley structure.
This supports the occurrence of another phase transition other than that from {\it X} to {\it Y} phase, as suggested by the magnetization data [Fig.~\ref{Fig7}(b)].
We hereafter call the additional higher-temperature phase ``{\it Z} phase".

The magnetoelectric effect of the {\it A}-site-ordered spinel has long been a subject of interest in view of the inversion symmetry breaking of the crystal structure \cite{2014_Ter, 2021_Sun}.
However, the reported dielectric anomalies associated with magnetic transitions are weak for Li(Ga, In)Cr$_{4}$O$_{8}$ \cite{2016_Lee, 2016_Sah}, where a collinear 2-up-2-down magnetic LRO is induced by the SLC at low temperatures.
We stress that the observed dielectric anomaly in CuGaCr$_{4}$S$_{8}$ is much larger than in the oxide cases thanks to the emergence of incommensurate magnetic LROs \cite{1976_Wil}.
A single-crystal study on CuGaCr$_{4}$S$_{8}$ would be a promising route to seek for remarkable electric polarization changes.

\begin{figure}[t]
\centering
\includegraphics[width=0.65\linewidth]{Fig9.eps}
\caption{$H$-$T$ phase diagram of CuGaCr$_{4}$S$_{8}$ based on physical property measurements performed in the present work. Note that another high-field phase ({\it C'} phase) appears above $\mu_{0}H_{\rm c3} = 103$~T at $\sim$5~K. According to the magnetoelastic theory \cite{2020_Gen}, the magnetic structure of {\it C} phase is expected to be a 3-up-1-down state.}
\label{Fig9}
\end{figure}

\subsection{\label{Sec4_3} {\textit H}-{\textit T} phase diagram}

Based on a series of physical property measurements, we construct an $H$-$T$ phase diagram of CuGaCr$_{4}$S$_{8}$, as shown in Fig.~\ref{Fig9}.
As discussed in Sec.~\ref{Sec4_1}, the magnetic structures of {\it X}, {\it Y}, {\it C} phases are possibly canted 2:2, canted 2:1:1, and 3-up-1-down states, respectively, based on the magnetoelastic theory \cite{2020_Gen}, though in reality an incommensurate spiral component would coexist with the commensurate collinear component.
We also observe a weak spin-flop transition at around 10~T, so that the {\it X} phase may be just a flopped spiral state with ${\mathbf Q} = (0.18, 0, 0.80)$ \cite{1976_Wil}.
An additional magnetization measurement using the STC system reveals that the {\it C} phase terminates at $\mu_{0}H_{\rm c3} = 103$~T, followed by the {\it C'} phase with a canted 3:1 state up to at least 140~T at $\sim$5~K (not shown in Fig.~\ref{Fig9}).

The identified magnetic phase diagram is basically in common with those of Cr spinel oxides \cite{2006_Ued, 2019_Ros, 2008_Koj, 2011_Miy_JPSJ, 2014_Miy, 2022_Kim} and CuInCr$_{4}$S$_{8}$ \cite{2020_Gen, 2022_Gen} where a robust 1/2-magnetization plateau ({\it C} phase) intervenes between spin-canted phases on lower- and higher-field sides.
A characteristic feature of CuGaCr$_{4}$S$_{8}$ is the appearance of field-induced high-temperature phase ({\it Z} phase) immediately below {\it C} phase, in common with another AFM-$J$-FM-$J'$ breathing pyrochlore compound CuInCr$_{4}$S$_{8}$, which hosts {\it A} phase in a closed $H$-$T$ regime around 25--40 T and 10--35 K \cite{2022_Gen}.
In the case of CuInCr$_{4}$S$_{8}$, the appearance of the {\it A} phase is accompanied by negative magnetostriction and the enhancement of magnetocapacitance at a lower phase boundary \cite{2022_Gen}.
These features are not the case for the {\it Z} phase in CuGaCr$_{4}$S$_{8}$.
Thus, we infer that the {\it Z} phase in CuGaCr$_{4}$S$_{8}$ is different form the {\it A} phase in CuInCr$_{4}$S$_{8}$.
The identification of these field-induced high-temperature phases in AFM-$J$-FM-$J'$ breathing pyrochlore systems would be an intriguing issue left for future works.

\section{\label{Sec5}Summary}

We synthesized CuGaCr$_{4}$S$_{8}$ polycrystalline samples and demonstrated the perfect crystallographic ordering of Cu and Ga atoms at the {\it A} site by the synchrotron XRD measurement and the Rietveld analysis.
The DFT calculation shows that the spin Hamiltonian is characterized by AFM $J$, FM $J'$, and relatively strong further-neighbor exchange couplings, so that the system harbors both geometrical frustration and bond frustration.
Thanks to the magnetic frustration as well as the strong SLC, rich magnetic phases, including the incommensurate spiral phase and the 3-up-1-down collinear phase, are induced in high magnetic fields at low temperatures.

\section*{Acknowledgements}
We appreciate Y. Okamoto, M. Mori, and S. Kitou for helpful discussions. 
This work was financially supported by the JSPS KAKENHI Grants-In-Aid for Scientific Research (No. 20J10988).
M.G. was a postdoctoral research fellow of the JSPS.

\appendix

\section{\label{SecA}Lorentzian fit on the peak profile of 440 reflection at low temperatures}

\begin{figure}[H]
\centering
\includegraphics[width=0.75\linewidth]{Fig10.eps}
\caption{Multi-Lorentzian fit (black) to the peak profile of 440 reflection at 4~K (pink), which is obtained by the superposition of three Lorentzian functions (blue).}
\label{Fig10}
\end{figure}

\section{\label{SecB}Low-field phase transition observed in a static magnetic field}

\begin{figure}[h]
\centering
\includegraphics[width=0.85\linewidth]{Fig11.eps}
\caption{Magnetic-field dependence of (a) magnetization $M$ and (b) its field derivative $dM/dH$ measured at various temperatures in a static magnetic field. The curves except for 4.2~K are shifted upward for clarity. Dashed lines in (a) are guides to the eye to make it easier to see the slope change in the $M$--$H$ curves.}
\label{Fig11}
\end{figure}

\begin{thebibliography}{99}
\bibitem{2010_Gar} J. S. Gardner, M. J. P. Gingras, and J. E. Greedan, Magnetic pyrochlore oxides, Rev. Mod. Phys. {\bf 82}, 53 (2010).
\bibitem{2021_Rei} D. Reig-i-Plessis and A. M. Hallas, Frustrated magnetism in fluoride and chalcogenide pyrochlore lattice materials, Phys. Rev. Mater. {\bf 5}, 030301 (2021).
\bibitem{2013_Oka} Y. Okamoto, G. J. Nilsen, J. P. Attfield, and Z. Hiroi, Breathing Pyrochlore Lattice Realized in {\it A}-Site Ordered Spinel Oxides LiGaCr$_{4}$O$_{8}$ and LiInCr$_{4}$O$_{8}$, Phys. Rev. Lett. {\bf 110}, 097203 (2013).
\bibitem{2015_Ben} O. Benton and N. Shannon, Ground State Selection and Spin-Liquid Behaviour in the Classical Heisenberg Model on the Breathing Pyrochlore Lattice, J. Phys. Soc. Jpn. {\bf 84}, 104710 (2015).
\bibitem{2016_Sav} L. Savary, X. Wang, H.-Y. Kee, Y. B. Kim, Y. Yu, and G. Chen, Quantum spin ice on the breathing pyrochlore lattice, Phys. Rev. B {\bf 94}, 075146 (2016).
\bibitem{2016_Li} F.-Y. Li, Y.-D. Li, Y. B. Kim, L. Balents, Y. Yu, and G. Chen, Weyl magnons in breathing pyrochlore antiferromagnets, Nat. Commun. {\bf 7}, 12691 (2016).
\bibitem{2019_Iqb} Y. Iqbal, T. M\"{u}ller, P. Ghosh, M. J. P. Gingras, H. O. Jeschke, S. Rachel, J. Reuther, and R. Thomale, Quantum and Classical Phases of the Pyrochlore Heisenberg Model with Competing Interactions, Phys. Rev. X {\bf 9}, 011005 (2019).
\bibitem{2020_Yan} H. Yan, O. Benton, L. D. C. Jaubert, and N. Shannon, Rank--2 {\it U}(1) Spin Liquid on the Breathing Pyrochlore Lattice, Phys. Rev. Lett. {\bf 124}, 127203 (2020).
\bibitem{2022_Han} S. E. Han, A. S. Patri, and Y. B. Kim, Realization of frac- tonic quantum phases in the breathing pyrochlore lattice, Phys. Rev. B {\bf 105}, 235120 (2022).
\bibitem{2019_Aoy} K. Aoyama and H. Kawamura, Spin ordering induced by lattice distortions in classical Heisenberg antiferromagnets on the breathing pyrochlore lattice, Phys. Rev. B {\bf 99}, 144406 (2019).
\bibitem{2021_Aoy} K. Aoyama, M. Gen, and H. Kawamura, Effects of spin-lattice coupling and a magnetic field in classical Heisenberg antiferromagnets on the breathing pyrochlore lattice, Phys. Rev. B {\bf 104}, 184411 (2021).
\bibitem{2021_Aoy_HG} K. Aoyama and H. Kawamura, Hedgehog-lattice spin texture in classical Heisenberg antiferromagnets on the breathing pyrochlore lattice, Phys. Rev. B {\bf 103}, 014406 (2021).
\bibitem{2022_Aoy_HG} K. Aoyama and H. Kawamura, Hedgehog lattice and field-induced chirality in breathing-pyrochlore Heisenberg antiferromagnets, Phys. Rev. B {\bf 106}, 064412 (2022).
\bibitem{2014_Kim} K. Kimura, S. Nakatsuji, and T. Kimura, Experimental realization of a quantum breathing pyrochlore antiferromagnet, Phys. Rev. B {\bf 90}, 060414(R) (2014).
\bibitem{2016_Hak} T. Haku, K. Kimura, Y. Matsumoto, M. Soda, M. Sera, D. Yu, R. A. Mole, T. Takeuchi, S. Nakatsuji, Y. Kono, T. Sakakibara, L.-J. Chang, and T. Masuda, Low-energy excitations and ground-state selection in the quantum breathing pyrochlore antiferromagnet Ba$_{3}$Yb$_{2}$Zn$_{5}$O$_{11}$, Phys. Rev. B {\bf 93}, 220407(R) (2016).
\bibitem{2016_Rau} J. G. Rau, L. S. Wu, A. F. May, L. Poudel, B. Winn, V. O. Garlea, A. Huq, P. Whitfield, A. E. Taylor, M. D. Lumsden, M. J. P. Gingras, and A. D. Christianson, Phys. Rev. Lett. {\bf 116}, 257204 (2016).
\bibitem{2022_Dis} S. Dissanayake, Z. Shi, J. G.Rau, R. Bag, W. Steinhardt, N. P. Butch, M. Frontzek, A. Podlesnyak, D. Graf, C. Marjerrison, J. Liu, M. J. P. Gingras, and S. Haravifard, Towards understanding the magnetic properties of the breathing pyrochlore compound Ba$_{3}$Yb$_{2}$Zn$_{5}$O$_{11}$ through single-crystal studies, npj Quantum Materials {\bf 7}, 77 (2022).
\bibitem{2019_Gho} P. Ghosh, Y. Iqbal, T. M\"{u}ller, R. Thomale, J. Reuther, M. J. P. Gingras, and H. O. Jeschke, Breathing chromium spinels: a showcase for a variety of pyrochlore Heisenberg Hamiltonians, npj Quantum Mater. {\bf 4}, 63 (2019).
\bibitem{1966_Jou} J.-C. Joubert and A. Durif, Bull. Soc. Fr. Mineral. {\bf 89}, 26 (1966).
\bibitem{1970_Pin} H. L. Pinch, M. J. Woods, and E. Lopatin, Some new mixed A-site chromium chalcogenide spinels, Mat. Res. Bull. {\bf 5}, 425 (1970).
\bibitem{2015_Nil} G. J. Nilsen, Y. Okamoto, T. Masuda, J. Rodriguez-Carvajal, H. Mutka, T. Hansen, and Z. Hiroi, Complex magnetostructural order in the frustrated spinel LiInCr$_{4}$O$_{8}$, Phys. Rev. B {\bf 91}, 174435 (2015).
\bibitem{2016_Lee} S. Lee, S.-H. Do, W.-J. Lee, Y. S. Choi, M. Lee, E. S. Choi, A. P. Reyes, P. L. Kuhns, A. Ozarowski, and K.-Y. Choi, Multistage symmetry breaking in the breathing pyrochlore lattice Li(Ga,In)Cr$_{4}$O$_{8}$, Phys. Rev. B {\bf 93}, 174402 (2016).
\bibitem{2016_Sah} R. Saha, F. Fauth, M. Avdeev, P. Kayser, B. J. Kennedy, and A. Sundaresan, Magnetodielectric effects in {\it A}-site cation-ordered chromate spinels Li{\it M}Cr$_{4}$O$_{8}$ ({\it M}=Ga and In), Phys. Rev. B {\bf 94}, 064420 (2016).
\bibitem{2018_Oka} Y. Okamoto, M. Mori, N. Katayama, A. Miyake, M. Tokunaga, A. Matsuo, K. Kindo, and K. Takenaka, Magnetic and Structural Properties of A-Site Ordered Chromium Spinel Sulfides: Alternating Antiferromagnetic and Ferromagnetic Interactions in the Breathing Pyrochlore Lattice, J. Phys. Soc. Jpn. {\bf 87}, 034709 (2018).
\bibitem{2018_Pok} G. Pokharel, A. F. May, D. S. Parker, S. Calder, G. Ehlers, A. Huq, S. A. J. Kimber, H. Suriya Arachchige, L. Poudel, M. A. McGuire, D. Mandrus, and A. D. Christianson, Negative thermal expansion and magnetoelastic coupling in the breathing pyrochlore lattice material LiGaCr$_{4}$S$_{8}$, Phys. Rev. B {\bf 97}, 134117 (2018).
\bibitem{2019_Gen} M. Gen, D. Nakamura, Y. Okamoto, and S. Takeyama, Ultra-high magnetic field magnetic phases up to 130 T in a breathing pyrochlore antiferromagnet LiInCr$_{4}$O$_{8}$, J. Magn. Magn. Mater. {\bf 473}, 387 (2019).
\bibitem{2020_Gen} M. Gen, Y. Okamoto, M. Mori, K. Takenaka, and Y. Kohama, Magnetization process of the breathing pyrochlore magnet CuInCr$_{4}$S$_{8}$ in ultrahigh magnetic fields up to 150~T, Phys. Rev. B {\bf 101}, 054434 (2020).
\bibitem{2020_Kan} T. Kanematsu, M. Mori, Y. Okamoto, T. Yajima, and K. Takenaka, Thermal Expansion and Volume Magnetostriction in Breathing Pyrochlore Magnets Li{\it A}Cr$_{4}${\it X}$_{8}$ ({\it A}=Ga, In, {\it X}=O, S), J. Phys. Soc. Jpn. {\bf 89}, 073708 (2020).
\bibitem{2020_Pok} G. Pokharel, H. S. Arachchige, T. J. Williams, A. F. May, R. S. Fishman, G. Sala, S. Calder, G. Ehlers, D. S. Parker, T. Hong, A. Wildes, D. Mandrus, J. A. M. Paddison, and A. D. Christianson, Cluster Frustration in the Breathing Pyrochlore Magnet LiGaCr$_{4}$S$_{8}$, Phys. Rev. Lett. {\bf 125}, 167201 (2020).
\bibitem{2021_Gao} S. Gao, A. F. May, M.-H. Du, J. A. M. Paddison, H. S. Arachchige, G. Pokharel, C. dela Cruz, Q. Zhang, G. Ehlers, D. S. Parker, D. G. Mandrus, M. B. Stone, and A. D. Christianson, Hierarchical excitations from correlated spin tetrahedra on the breathing pyrochlore lattice, Phys. Rev. B {\bf 103}, 214418 (2021).
\bibitem{2022_Sha} S. Sharma, M. Pocrnic, B. N. Richtik, C. R. Wiebe, J. Beare, J. Gautreau, J. P. Clancy, J. P. C. Ruff, M. Pula, Q. Chen, S. Yoon, Y. Cai, and G. M. Luke, Synthesis and physical and magnetic properties of CuAlCr$_{4}$S$_{8}$: A Cr-based breathing pyrochlore, Phys. Rev. B {\bf 106}, 024407 (2022).
\bibitem{2022_Gen} M. Gen, H. Ishikawa, A. Ikeda, A. Miyake, Z. Yang, Y. Okamoto, M. Mori, K. Takenaka, H. Sagayama, T. Kurumaji, Y. Tokunaga, T. Arima, M. Tokunaga, K. Kindo, Y. H. Matsuda, and Y. Kohama, Complex magnetic phase diagram with a small phase pocket in a three-dimensional frustrated magnet CuInCr$_{4}$S$_{8}$, Phys. Rev. Research {\bf 4}, 033148 (2022). 
\bibitem{2009_Muh} S. M\"{u}hlbauer, B. Binz, F. Jonietz, C. Pfleiderer, A. Rosch, A. Neubauer, R. Georgii, and P. B\"{o}ni, Skyrmion Lattice in a Chiral Magnet, Science {\bf 323}, 915 (2009).
\bibitem{2012_Sek} S. Seki, X. Z. Yu, S. Ishiwata, and Y. Tokura, Observation of Skyrmions in a Multiferroic Material, Science {\bf 336}, 198 (2012).
\bibitem{2012_Oku} T. Okubo, S. Chung, and H. Kawamura, Multiple-$q$ States and the Skyrmion Lattice of the Triangular- Lattice Heisenberg Antiferromagnet under Magnetic Fields, Phys. Rev. Lett. {\bf 108}, 017206 (2012).
\bibitem{2005_Kes} Ya. A. Kesler, E. G. Zhukov, D. S. Filimonov, E. S. Polulyak, T. K. Menshchikova,
V. A. Fedorov, CuCr$_{2}$S$_{4}$-Based Quaternary Cation-Substituted Magnetic Phases, Inorg. Mater. {\bf 41}, 914 (2005).
\bibitem{2018_Ami} T. G. Aminov, E. V. Busheva, G. G. Shabunina, and V. M. Novotortsev, Magnetic Phase Diagram of Solid Solutions in the CoCr$_{2}$S$_{4}$--Cu$_{0.5}$Ga$_{0.5}$Cr$_{2}$S$_{4}$ System, Russ. J. Inorg. Chem. {\bf 63}, 519 (2018).
\bibitem{2011_Mom} K. Momma and F. Izumi, VESTA 3 for three-dimensional visu- alization of crystal, volumetric and morphology data, J. Appl. Crystallogr. {\bf 44}, 1272 (2011).
\bibitem{RIETAN} F. Izumi and K. Momma, Three-Dimensional Visualization in Powder Diffraction, Solid State Phenom. {\bf 130}, 15 (2007).
\bibitem{2018_Ike} A. Ikeda, Y. H. Matsuda, and H. Tsuda, Note: Optical filter method for high-resolution magnetostriction measurement using fiber Bragg grating under millisecond-pulsed high magnetic fields at cryogenic temperatures, Rev. Sci. Instrum. {\bf 89}, 096103 (2018).
\bibitem{2020_Miy} A. Miyake, H. Mitamura, S. Kawachi, K. Kimura, T. Kimura, T. Kihara, M. Tachibana, and M. Tokunaga, Capacitive detection of magnetostriction, dielectric constant, and magneto-caloric effects in pulsed magnetic fields, Rev. Sci. Instrum. {\bf 91}, 105103 (2020).
\bibitem{2021_Yam} H. Yamamoto, T. Sakakura, H. O. Jeschke, N. Kabeya, K. Hayashi, Y. Ishikawa, Y. Fujii, S. Kishimoto, H. Sagayama, K. Shigematsu, M. Azuma, A. Ochiai, Y. Noda, and H. Kimura, Quantum spin fluctuations and hydrogen bond network in the antiferromagnetic natural mineral henmilite, Phys. Rev. Materials {\bf 5}, 104405 (2021).
\bibitem{2021_Hei} L. Heinze, H. O. Jeschke, I. I. Mazin, A. Metavitsiadis, M. Reehuis, R. Feyerherm, J.-U. Hoffmann, M. Bartkowiak, O. Prokhnenko, A. U. B. Wolter, X. Ding, V. S. Zapf, C. Corval\'{a}n Moya, F. Weickert, M. Jaime, K. C. Rule, D. Menzel, R. Valent\'{i}, W. Brenig, and S. S\"{u}llow, Magnetization process of atacamite: A case of weakly coupled $S = 1/2$ sawtooth chains, Phys. Rev. Lett. {\bf 126}, 207201 (2021).
\bibitem{2022_Her} M. Hering, F. Ferrari, A. Razpopov, I. I. Mazin, R. Valent\'{i}, H. O. Jeschke, and J. Reuther, Phase diagram of a distorted kagome antiferromagnet and application to y-kapellasite, npj Comput. Mater. {\bf 8}, 10 (2022).
\bibitem{1999_Koe} K. Koepernik and H. Eschrig, Full-potential nonorthogonal local-orbital minimum-basis band-structure scheme, Phys. Rev. B {\bf 59}, 1743 (1999).
\bibitem{1996_Per} J. P. Perdew, K. Burke, and M. Ernzerhof, Generalized gradient approximation made simple, Phys. Rev. Lett. {\bf 77}, 3865 (1996).
\bibitem{1995_Lie} A. I. Liechtenstein, V. I. Anisimov, and J. Zaanen, Density-functional theory and strong interactions: Orbital ordering in Mott-Hubbard insulators, Phys. Rev. B {\bf 52}, R5467 (1995).
\bibitem{1996_Miz} T. Mizokawa and A. Fujimori, Electronic structure and orbital ordering in perovskite-type $3d$ transition-metal oxides studied by Hartree-Fock band-structure calculations, Phys. Rev. B {\bf 54}, 5368 (1996).
\bibitem{comment} The present heat capacity measurement using the thermal relaxation method cannot accurately evaluate the latent heat associated with the first-order transition. Therefore, there remains certain umbiguity estimated $S_{\rm N}$
\bibitem{1976_Wil} C. Wilkinson, B. M. Knapp, and J. B. Forsyth, The magnetic structure of Cu$_{0.5}$Ga$_{0.5}$Cr$_{2}$S$_{4}$, J. Phys. C: Solid State Phys. {\bf 9}, 4021 (1976).
\bibitem{2009_Kho} D. Khomskii, Classifying multiferroics: Mechanisms and effects, Physics {\bf 2}, 20 (2009).
\bibitem{2006_Ued} H. Ueda, H. Mitamura, T. Goto, and Y. Ueda, Successive field-induced transitions in a frustrated antiferromagnet HgCr$_{2}$O$_{4}$, Phys. Rev. B {\bf 73}, 094415 (2006).
\bibitem{2007_Lee} S.-H. Lee, G. Gasparovic, C. Broholm, M. Matsuda, J.-H. Chung, Y. J. Kim, H. Ueda, G. Xu, P. Zschack, K. Kakurai, H. Takagi, W. Ratcliff, T. H. Kim, and S.-W. Cheong, Crystal distortions in geometrically frustrated ACr$_{2}$O$_{4}$ (A = Zn, Cd), J. Phys.: Condens. Matter {\bf 19}, 145259 (2007).
\bibitem{2009_Yok} F. Yokaichiya, A. Krimmel, V. Tsurkan, I. Margiolaki, P. Thompson, H. N. Bordallo, A. Buchsteiner, N. St\"{u}$\beta$er, D. N. Argyriou, and A. Loidl, Spin-driven phase transitions in ZnCr$_{2}$Se$_{4}$ and ZnCr$_{2}$S$_{4}$ probed by high-resolution synchrotron x-ray and neutron powder diffraction, Phys. Rev. B {\bf 79}, 064423 (2009).
\bibitem{2009_Kan} Ch. Kant, J. Deisenhofer, T. Rudolf, F. Mayr, F. Schrettle, A. Loidl, V. Gnezdilov, D. Wulferding and P. Lemmens, and V. Tsurkan, Optical phonons, spin correlations, and spin-phonon coupling in the frustrated pyrochlore
magnets CdCr$_{2}$O$_{4}$ and ZnCr$_{2}$O$_{4}$, Phys. Rev. B {\bf 80}, 214417 (2009).
\bibitem{2007_Mat_1} M. Matsuda, M. Takeda, M. Nakamura, and K. Kakurai, A. Oosawa, E. Leli\'{e}vre-Berna, J.-H. Chung, H. Ueda, H. Takagi, S.-H. Lee, Spiral spin structure in the Heisenberg pyrochlore magnet CdCr$_{2}$O$_{4}$, Phys. Rev. B {\bf 75}, 104415 (2007).
\bibitem{2011_Bha} S. Bhattacharjee, S. Zherlitsyn, O. Chiatti, A. Sytcheva, J. Wosnitza, R. Moessner, M. E. Zhitomirsky, P. Lemmens,5 V. Tsurkan, and A. Loidl, Interplay of spin and lattice degrees of freedom in the frustrated antiferromagnet CdCr$_{2}$O$_{4}$: High-field and temperature-induced anomalies of the elastic constants, Phys. Rev. B {\bf 83}, 184421 (2011).
\bibitem{2019_Ros} L. Rossi, A. Bobel, S. Wiedmann, R. K\"{u}chler, Y. Motome, K. Penc, N. Shannon, H. Ueda, and B. Bryant, Negative Thermal Expansion in the Plateau State of a Magnetically Frustrated Spinel, Phys. Rev. Lett. {\bf 123}, 027205 (2019).
\bibitem{2020_Ros} L. Rossi, D. Br\"{u}ing, H. Ueda, Y. Skourski, T. Lorenz, and B. Bryant, Magnetoelectric coupling in a frustrated spinel studied using high-field scanning probe microscopy, Appl. Phys. Lett. {\bf 116}, 262901 (2020).
\bibitem{2005_Chu} J.-H. Chung, M. Matsuda, S.-H. Lee, K. Kakurai, H. Ueda, T. J. Sato, H. Takagi, K.-P. Hong, and S. Park, Statics and Dynamics of Incommensurate Spin Order in a Geometrically Frustrated Antiferromagnet CdCr$_{2}$O$_{4}$, Phys. Rev. Lett. {\bf 95}, 247204 (2005).
\bibitem{2008_Koj} E. Kojima, A. Miyata, S. Miyabe, S. Takeyama, H. Ueda, and Y. Ueda, Full-magnetization of geometrically frustrated CdCr$_{2}$O$_{4}$ determined by Faraday rotation measurements at magnetic fields up to 140 T, Phys. Rev. B {\bf 77}, 212408 (2008).
\bibitem{2011_Miy_JPSJ} A. Miyata, H. Ueda, Y. Ueda, Y. Motome, N. Shannon, K. Penc, and S. Takeyama, 
Novel Magnetic Phases Revealed by Ultra-High Magnetic Field in the Frustrated Magnet ZnCr$_{2}$O$_{4}$, J. Phys. Soc. Jpn. {\bf 80}, 074709 (2011).\bibitem{2014_Miy} A. Miyata, H. Ueda, and S. Takeyama, Canted 2:1:1 Magnetic Supersolid Phase in a Frustrated Magnet MgCr$_{2}$O$_{4}$ as a Small Limit of the Biquadratic Spin Interaction, J. Phys. Soc. Jpn. {\bf 83}, 063702 (2014).
\bibitem{2007_Mat_2} M. Matsuda, H. Ueda, A. Kikkawa, Y. Tanaka, K. Katsumata, Y. Narumi, T. Inami, Y. Ueda, and S.-H. Lee. Spin-lattice instability to a fractional magnetization state in the spinel HgCr$_{2}$O$_{4}$, Nat. Phys. {\bf 3}, 397 (2007).
\bibitem{2010_Mat} M. Matsuda, K. Ohoyama, S. Yoshii, H. Nojiri, P. Frings, F. Duc, B. Vignolle, G. L. J. A. Rikken, L.- P. Regnault, S.-H. Lee, H. Ueda, and Y. Ueda, Universal Magnetic Structure of the Half-Magnetization Phase in Cr-Based Spinels, Phys. Rev. Lett. {\bf 104}, 047201 (2010).
\bibitem{2007_Tan} Y. Tanaka, Y. Narumi, N. Terada, K. Katsumata, H. Ueda, U. Staub, K. Kindo, T. Fukui, T. Yamamoto, R. Kammuri, M. Hagiwara, A. Kikkawa, Y. Ueda, H. Toyokawa, T. Ishikawa, and H. Kitamura, Lattice Deformations Induced by an Applied Magnetic Field in the Frustrated Antiferromagnet HgCr$_{2}$O$_{4}$, J. Phys. Soc. Jpn. {\bf 76}, 043708 (2007).
\bibitem{2022_Kim} S. Kimura, S. Imajo, M. Gen, T. Momoi, M. Hagiwara, H. Ueda, and Y. Kohama, Quantum phase of the chromium spinel oxide HgCr$_{2}$O$_{4}$ in high magnetic fields, Phys. Rev. B {\bf 105}, L180405 (2022).
\bibitem{2014_Ter} N. V. Ter-Oganessian, Cation-ordered A'$_{1/2}$A"$_{1/2}$B$_{2}$X$_{4}$ magnetic spinels as magnetoelectrics, J. Magn. Magn. Mater. {\bf 364}, 47 (2014).
\bibitem{2021_Sun} A. Sundaresan and N. V. Ter-Oganessian, Magnetoelectric and multiferroic properties of spinels, J. Appl. Phys. {\bf 129}, 060901 (2021).

\end{thebibliography}

\end{document}