\documentclass[linenumbers]{aastex631}

\graphicspath{{./}{figs/}}
\usepackage{todonotes}
\usepackage{soul}
\usepackage{verbatim}
\begin{document}

\title{The Winchcombe Fireball---that lucky survivor}

\newcommand{\contrib}[1]{\textcolor{teal}{#1}}
\newcommand{\numb}[1]{\textcolor{orange}{#1}}
\newcommand{\source}[1]{\textsuperscript{\textcolor{blue}{[citation needed]}}}
\newcommand{\missnumber}{\numb{[NUMBER]}}
\newcommand{\checknumber}{\numb{[check number]}}
\newcommand{\densunitSI}{kg\,m$^{-3}$}
\newcommand{\kms}{km\,s$^{-1}$}
\newcommand{\ms}{m\,s$^{-1}$}
\newcommand{\fireballTimeZero}{2021-02-28T21:54:17}
\newcommand{\codename}{DN210228\_02}
\newcommand{\meteorite}{\textit{Winchcombe}}
\newcommand{\DFN}{Desert Fireball Network}
\newcommand{\GFO}{Global Fireball Observatory}

\newcommand{\ukfall}{UK Fireball Alliance (UKFAll)}
\newcommand{\ukfn}{UK Fireball Network (UKFN)}
\newcommand{\gfo}{Global Fireball Observatory (GFO)}
\newcommand{\gmn}{Global Meteor Network (GMN)}
\newcommand{\scamp}{System for the Capture of Asteroid and Meteorite Paths (SCAMP)}

\correspondingauthor{Sarah McMullan}
\email{s.mcmullan16@imperial.ac.uk }

\author[0000-0002-7194-6317]{Sarah McMullan}


\begin{abstract}

\end{abstract}

On the 28th February 2021, at 21:58:16 UTC, a fireball lasting 8 seconds streaked across the UK \citep{king_science_winchcombe}.
It was witnessed by over 1000 people, and captured by many doorbell cameras and dashcams.
It was also captured by 16 dedicated meteor/fireball cameras of the UK Fireball Alliance (UKFAll)---one of the best recorded meteorite falls in history. The UKFAll consortium was established in 2018 as a collaboration between the six meteor camera networks in the UK, with an aim to streamline data sharing and meteorite recovery efforts \citep{2020EPSC...14..705D}.
The precursory work that UKFAll had done prior to this event enabled the team to share data, establish an initial strewn field, and handle press enquiries, all within 12 hours of the fall.
This streamlined process enabled the recovery of the main mass the following day, having been discovered on a driveway in Winchcombe \citep{king_science_winchcombe}. The meteorite was identified as a CM2 carbonaceous chondrite; only 3\% of meteorites globally are carbonaceous chondrites.




Linking a meteorite to its pre-atmospheric orbit is vital for understanding the origins of the Solar System. Carbonaceous chondrites are particularly important to building this understanding, as they have volatile and isotopic compositions, which are thought to be key for the origins of water and life on Earth \citep{marty2012origins, alexander2012provenances}. Prior to the Winchcombe meteorite, only 4 carbonaceous chondrites could be linked to their pre-atmospheric orbit---Tagish Lake \citep{2000Sci...290..320B}, Maribo \citep{haack2012maribo}, Sutter's Mill \citep{2012Sci...338.1583J}, and Flensburg. However, none of these events had anywhere near as many observations as the Winchcombe fireball. 

The \textbf{Tagish Lake} meteorite fell on 18 January 2000 at 16:43 UTC in Canada \citep{2000Sci...290..320B} and was categorised as a C2-ungrouped carbonaceous chondrite with an initial mass of up to 200,000~kg. There were over 70 eyewitness reports, with 24 photos and 5 videos of the dust cloud taken by observers within 1-2 mins of the event. It was also detected by infrared and optical sensors aboard US Department of Defense Satellites and the corresponding shock wave by local seismic and infrasound stations \citep{2002M&PS...37..661B}. More than 500 meteorites were recovered, with the largest fragment 2.3~kg, and a total mass of 16.3~kg \citep{2011M&PS...46.1525P}.
%An initial mass of 50,000--90,000~kg has been estimated, but could have been as much as 200,000~kg \citep{2000Sci...290..320B}, corresponding to a diameter of 4--6~m \citep{2002M&PS...37..661B}. 
%The orbit is classified as a Jupiter Family Comet orbit, with a Tisserand paramter of 3.66 (\Table \ref{tab:carb_chondrites}; see \citet{2000Sci...290..320B} for full orbital information). 

The \textbf{Maribo} meteorite fell on the 17 January 2009, at 18:08:28 UTC in Denmark \citep{haack2012maribo}, and was identified as a CM2 meteorite with a calculated initial meteoroid mass of 2000$\pm$1000~kg \citep{2019M&PS...54.1024B}. There were 550 eyewitness reports and it was captured by a surveillance camera in southern Sweden, in a photo from an all-sky fireball camera in the Netherlands, and by three all-sky meteor radars in Germany. The sonic boom was recorded by 11 seismometers and an infrasound station. Seven radiometers in the Czech Republic were able to measure the radiometric light curve. A single 25.8~g fragment was found six weeks later.

The \textbf{Sutter's Mill} meteorite fell on the 22 April 2012, at 14:51:12 UTC in California and was identified as a CM2 carbonaceous chondrite with a calculated initial meteoroid mass of  \citep{2012Sci...338.1583J}. It was detected by 3 Doppler weather radars, 2 infrasound stations, and 8 seismic stations, along with a set of three photographs from Nevada, and a few videos. 77 meteorites were collected overall (a combined mass of 943~g).

The \textbf{Flensburg} meteorite fell on 12 September 2019, 12:50 UTC over northern Germany, and was identified as a C1-ungrouped carbonaceous chondrite with a calculated initial meteoroid mass of 10,000--20,000~kg \citep{2021M&PS...56..425B}. There were 584 eyewitness reports and was recorded by one AllSky6 camera and 3 casual dashboard cameras. A single meteorite of 24.5~g was recovered, and there was not sufficient data to estimate the total fallen mass or number of fragments.


Each of these events had a limited number of high precision observations, especially from nearby dedicated meteor cameras. Additionally, three out of four were daylight fireballs which made observations more difficult. Due to the variety of observation types, many different techniques were needed to calculate the pre-atmospheric orbits, resulting in large error bars (Table \ref{tab:carb_chondrites}). Winchcombe is the first carbonaceous chondrite fall which was recorded by multiple dedicated meteor/fireball cameras within 150~km of the fireball---observed with unprecedented detail. It was an evening event, providing both an opportunity for high precision recording equipment, as well as being widely witnessed -- over 1000 eyewitness reports -- resulting in much public interest.


The limited number of carbonaceous chondrites that can be linked to their pre-atmospheric orbit is largely due to their poor survivability. Carbonaceous chondrites are the weak, so they fragment and ablate quickly in the Earth's atmosphere. To survive as a meteorite, their passage through the Earth's atmosphere requires specific properties \textit{i.e.} low entry speeds, shallow trajectories, or a large initial mass. Still only a small percentage of the pre-entry mass is likely to survive as a meteorite. Each of the previous events were from large bodies ($>10^3$~kg), with high entry speeds ($>15$~km/s), low trajectory angles($<31^\circ$), breaking up at dynamic pressures $>1$~MPa (Table \ref{tab:carb_chondrites}).


\begin{table*}
\centering
\begin{tabular}{ccccccccc}
\hline
Name & Classification & Max & Initial Mass& Initial & Trajectory  & Semi-major & $T_J$ & Ref. \\
 &  &  Pressure&  Range  &  Velocity &  angle &  Axis &  &  \\
 &  & (MPa) & (kg) & (km/s) &  & (AU) & & \\
\hline 
Tagish Lake & C2 (ungrouped) & 2.2 & $5\times10^4$--$2\times10^5$ & 15.8 & 16.5$^\circ$& $2.1\pm0.2$ & 3.66 & 1  \\
Maribo & CM2 & 5 & $1\times10^3$--$3\times10^3$& 28.3$\pm$0.3 & 31$^\circ$& $2.43\pm0.12$  &$2.95\pm0.11$ & 2, 3 \\
Sutter's Mill & CM2 & \hl{0.9??} & $2\times10^4$--$8\times10^4$ & 28.6$\pm$0.6 & 26.3$\pm$0.5$^\circ$& $2.59\pm0.35$  & $2.81\pm0.32$ & 4 \\
Flensburg & C1-ungrouped & 2 & $1\times10^4$--$2\times10^4$ & 19.43$\pm$0.05 & 24.4$^\circ$& 2.82$\pm$0.03 & $2.89\pm0.02$ & 5 \\
Winchcombe & CM2 & 0.5 & 9--15 &  13.86&$48.940^{\circ}$&$2.5855\pm0.0077$ &$3.1207\pm0.0056$ & 6, 7 \\
\hline
\end{tabular}
\caption{Properties of orbital carbonaceous chondrites, including their Tisserand parameter with Jupiter ($T_j$) where a $2<T_j<3$ defines a Jupiter Family Comet type orbit. References: (1) \citet{2000Sci...290..320B}, (2) \citet{haack2012maribo}, (3) \citet{2019M&PS...54.1024B}, (4) \citet{2012Sci...338.1583J}, (5) \citet{2021M&PS...56..425B}, (6) \citet{king_science_winchcombe}, (7) this work.}
\label{tab:carb_chondrites}
\end{table*}



Firstly we describe the observations and camera networks who contributed to the plethora of data, then we discuss the trajectory and fragmentation modelling, the orbital analysis, the strewn field calculation and compare it with the locations of the meteorites found. Finally, we discuss what we have learnt from the Winchcombe meteorite fall and conclude our findings.









\begin{comment}

\hl{-----------------------------------notes below this line----------------------------------------}

Flensburg:
\citep{2021M&PS...56..425B}
- September 12 2019, 12:50 UTC
northern Germany
- C1 ungrouped carbonaceous chondrite
- one AllSky6 camera and 3 casual dashcam video records of bolide- determined trajectory, velocity, and helliocentric orbit
- combined with USGS bolide energy report - pre-atmospheric diameter 2-3m, mass 10,000-20,000 kg, fragmented heavily 46-37km, under dynamic pressures 0.7-2 MPa
- difference in orbit calculated by cameras and USGS data
- meteoroid was in the 5:2 resonance with Jupiter located at 2.82 AU
- one meteorite recovered - 24.5g, bulk density 1984 pm 16 kgm-3
- entry velocity 91.43 pm 0.05 kms-1, trajectory approx 24.4deg
- available data not sufficient to rigorously determine the total fallen mass and/or number of fragments, although an approximate strewn field area was calculated
- found in garden
- semimajor axis 2.82pm0.03 AU, eccentricty 0.701pm0.003, perihelion distance 0.843pm0.001 AU



Sutter's Mill:
\citep{2012Sci...338.1583J}
- 22 April 2012
-CM2 carbonaceous chondrite
- 20,000-80,000 kg
- daytime fireball over California aqnd Nevada
- 3 weather rads of the U.S. National Climatic Data Center’s NEXRAD network
- 3 meteorites recovered 24 April before heavy rain
- 77 meteorites total found, total mass 943 g, largest 205 g
- 2 infrasound
- 8 seismic stations
- 3 photos Nevada
- videos
- 28.6pm0.6kms-1
- traj 26.3pm0.5
- semimajor axis 2.59pm0.35 AU, eccentricity 0.824pm0.020, perihelion distance 0.456pm0.022 Au


Maribo:

\citep{2019M&PS...54.1024B}
- 2009
- Denmark
- CM2 carbonaceous chondrite, 26g meteorite
- max dynamic pressure 5 MPa
- 28.3pm0.3 kms-1, low density < 2500 kg m-3, both lower chance of meteoroid survival
- trajectory 31 deg
- captured by one security camera in Sweden, and photo from a large distance by an all-sky camera in Netherlands, all-sky meteor radar in Germany, and seven radiometers from Autonomous Fireball Observatories in the Czech Republic which measured the radiometric light curve
- semimajor axis 2.43pm0.12, eccentricity 0.805pm0.010, perihilion distance 0.475pm0.005
- preentry mass of 2000pm1000 kg

\citep{haack2012maribo}
- 17 Jan, 2009, 19:08:28 CET
- many eyewitnesses 550 reports, surveillance camera southern Sweden, all sky camera Netherlands, three meteor radar stations in Germany
- super sonic boom recorded by two seismic stations Dansih island Sjaelland, infrasound station in southern Germnany, nine seismometers in Germany
- single fragment 25.8g found march 4, 2009, CM2


Tagish Lake:
\citep{2000Sci...290..320B}
- 18 January 2000, 16:43 UTC
- Canada
- C2-ungrouped carbonacous chondrite - composition intermediate between CM and CI meteorites
- 50,000-90,000 kg
- upto 200,000 kg
- max dynamic pressure 2.2 MPa, first fragmentation 0.3 MPa
- >500 meteorites recovered from strewn field 16x3km, largest 2.3kg
\citep{2002M&PS...37..661B}
- detected by infrared and optical sensors aboard US Department of Defense satellites
- >70 eyewitnesses interviewed to reconstruct the atmospheric trajectory
- 24 still photos and 5 videos of the dust cloud, subset within 1-2 mins of the event
- registered by satellite network, local seismic stations, infrasound recording, scattered videorecordings of dust trail left in atmosphere
- estimated bulk density 1500 kgm-3 --> 4-6 m diameter
- calculated orbit - typical Earth-crossing Apollo asteroid-type orbit with semimajor axis in middle of asteroid belt. semimajor axis (a) 2.1pm0.2 AU, eccentricity (e) 0.57pm0.05, perihelion distance (q) 0.891pm 0.009 AU
- first meteorites recovered 25 Jan 2000 from frozen lake surface

\citep{2011M&PS...46.1525P}
- material very high porosity 40%
- largest fragment only 10$^{-5}$ of initial mass
- velocity 15.8 kms-1
- trajectory 16.5
- inferred initial mass 65,000 kg, initial diameter 4.2m, density 1.64 gcm-3
- weight expected 100-1000kg, weight recovered 16.3 kg, approx 500 fr4agments, inferred bulk strength 0.3 MPa








\hl{-----------------------------------notes below this line----------------------------------------}



Carbonaceous chondrites have volatile and isotopic compositions that suggest they played a key role in delivery of water and biologically important molecules to Earth (Marty 2012, Alexander et al.,2012). Most meteorites found can't be linked to their source region in the Solar system as there isn't enough information about pre-atmospheric orbit

Fast recovery --> near pristine record of composition of primative asteroids

Orbit confirms originated from an asteroid in the main belt. Models and remote sensing suggest primitive, volatile-rich asteroids originating in the outer Solar System were scattered inwards to the main asteroid belt by the giant planets  \citep{2012M&PS...47.1941W, 2014Natur.505..629D}

The likelihood of CM2 meteorites to survive atmospheric flight depend on several factors



21:58:16(UTC) 
28th February 2021
lasted 8 secs
entry angle $\sim40^\circ$
clear skies
many eyewhitness and doorbell/dashcam footage >1000
detected by 16 UKFAll cameras from 6 camera networks
Main mass discovered the next day on driveway 319.5 g, collected within 12 hours of fall.
Further stones recovered within the following days/weeks
Pre-atmospheric orbit, similar to Sutter's Mill \citep{2012Sci...338.1583J} and Maribo \citep{2019M&PS...54.1024B}, previously reported CM chondrites, distinct from ungrouped carbonaceous chondrites Tagish Lake \citep{2000Sci...290..320B} and Flensburg \citep{2021M&PS...56..425B}

Initial velocity $13.5 \pm 0.01$ km s$^-1$
Initial entry mass 12  \hl{pm 25\%} kg, lowest recorder for carbonaceous chondrite fall, smallest ever by two orders of magnitude
density?
Modelling indicates a surviving mass 0.5 kg, matching what was found 
Fragmented at maximum dynamic pressure 0.5 MPa, lowest ever determined for instrumentally observed fall \citep{2020AJ....160...42B}
All previous observed entries of carbonaceous material have been from objects $>$1m \hl{winchcombe -> 20 cm}.
This is at odds with the fact that most meteorites (wih statistics dominated by ordinary chondrites) come from decimetre-sized meteoroids.
It naturally raises the question of what bias there may exist on the entry survival of carbonaceous material.
For instance, at larger sizes, \citet{2006M&PS...41..607B} have shown that small craters ($<10$ km) on Earth are nearly exclusively made by iron objects, despite the fact that iron asteroids represent $<5$\% of impactors.
In this case, the stronger structure of the iron bodies enables them to survive larger dynamic pressures and reach the ground at hyper velocity.
Stony asteroids in comparison get broken higher in the atmosphere, and their fragments decelerate too much to before reaching the ground, hence make no crater.
A similar selection bias may be at play when comparing carbonaceous and ordinary chondrites.


Survival of Winchcombe --> right conditions: low speed, shallow entry, small carbonaceous can survive luminous phase of atmospheric flight




UKFAll brief history and details on data sharing process. How this resulted in fast orbital/atmospheric modelling and constraints on the fall area
established 2018
Converter program and standard data exchange format \citep{2020EPSC...14..856R}
16 stations captured - most widely instrumentally observed carbonaceous chondrite fall to date
So many cameras, allowed us to select the best data


Tagish lake based on IR and optical from satellite sensors plus photos and videos of dust trail \citep{2000Sci...290..320B}, Sutter's Mill, weather radar and two videos one photo \citep{2012Sci...338.1583J}.
The orbital characteristics and entry conditions of the Maribo (CM2) chondrite were calculated from radar detections and two videos (Borovička et al. 2019), and for the Flensburg (C1ung) chondrite they were determined from one AllSky6 camera and three dashcams videos (Borovička et al. 2021).
Recorded by 16 dedicated meteor optical observation systems, the Winchcombe fireball has been observed with unprecedented detail amongst carbonaceous chondrites.


Seven optical records used for astrometry, most from within 100 km

Backwards integration of orbit method \citep{2021PSJ.....2...98S}
\hl{how good orbit is?? Is there a metric for this?}

Fall line calculated is good fit for the found meteorites


Summary of paper layout





\hl{keep high level, don't say exact results, talk more about implications}

\hl{limits of previous CM2 orbits, lack of close data, --- Winchcombe has lots of instrumental records, with good precision and range of different data sources. Lack of understanding of quality of previously observed CM2 falls.}

The Winchcombe story has been discussed by \citet{king_science_winchcombe}.
In this work, we concentrate on the detailed analysis of the fireball.






\todo[inline]{
Hadrien and Denis, please add to this list anything that you think is important to include in the introduction :) 

- brief story of Winchcombe

- why it is so important/such an unlikely event - survivability of carbonaceous chondrites, recovery time, most observed meteorite dropper

- brief outline on UK networks and their collaboration (both UK and internationally) over the past few years that lead to this. Framework in place to share data and get results quickly.

- previous orbits of carbonaceous chondrites aren't well known/observations terrible. this is the first with accurate observations and trajectories etc. Focus on how good orbit is, not necessarily the freshness of recovery, however, the speed of collaboration was key.

- add in attempting to answer reconstructing history of CM chondrites, helps understand the origin of water on Earth as this is thought to be where most comes from.

- probe preatmospheric structure and strength of this body. Observed fragmentation so well --> accurate reconstruction of fragmentation process, this allows us to understand bodies <1m which has not been previously studied as don't normally survive entry. Current understanding is skewed by the meteorites that survive entry (see \citet{2006M&PS...41..607B}). Maybe more small CMs survived somewhere? Are small objects histories different to larger items? (Discussion points)
}


\end{comment}

\end{document}