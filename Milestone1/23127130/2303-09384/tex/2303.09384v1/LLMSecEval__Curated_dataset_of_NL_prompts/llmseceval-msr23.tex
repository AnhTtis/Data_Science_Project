%\documentclass[conference]{IEEEtran}
%\documentclass[10pt,conference,anonym]{IEEEtran}
\documentclass[conference]{IEEEtran}
\IEEEoverridecommandlockouts
% The preceding line is only needed to identify funding in the first footnote. If that is unneeded, please comment it out.
\usepackage{cite}
\usepackage{amsmath,amssymb,amsfonts}
\usepackage{algorithmic}
\usepackage{graphicx}
\usepackage{textcomp}
\usepackage{xcolor}
\usepackage[T1]{fontenc}


\def\BibTeX{{\rm B\kern-.05em{\sc i\kern-.025em b}\kern-.08em
    T\kern-.1667em\lower.7ex\hbox{E}\kern-.125emX}}

\newcommand\todo[1]{\COMMENTX{TODO: }{{{\color{red}#1}}}}
\def\BibTeX{{\rm B\kern-.05em{\sc i\kern-.025em b}\kern-.08em
    T\kern-.1667em\lower.7ex\hbox{E}\kern-.125emX}}
\begin{document}

\title{LLMSecEval: A Dataset of Natural Language Prompts for Security Evaluations
%LLMSecEval: A Curated Dataset of Natural Language Prompts for Evaluating the Security of Generated Code
%{\footnotesize \textsuperscript{*}Note: Sub-titles are not captured in Xplore and
%should not be used}
%\thanks{Identify applicable funding agency here. If none, delete this.}
}

% Catherine Tony, Markus Mutas, Nicolas Diaz Ferreyra, Riccardo Scandariato


\author{
	\IEEEauthorblockN{
	    Catherine Tony, Markus Mutas, Nicol\'{a}s E. D\'{i}az Ferreyra, Riccardo Scandariato
	}
\IEEEauthorblockA{\textit{Institute of Software Security} \\
\textit{Hamburg University of Technology, Germany}\\
\{catherine.tony, markus.mutas, nicolas.diaz-ferreyra, riccardo.scandariato\}@tuhh.de}
	
}

%\author{\IEEEauthorblockN{Anonymous}
%\IEEEauthorblockA{\textit{Institution} \\
%\textit{name of organization (of Aff.)}\\
%Country \\
%email}
%\and
%\IEEEauthorblockN{Anonymous}
%\IEEEauthorblockA{\textit{Institution} \\
%\textit{name of organization (of Aff.)}\\
%Country \\
%email}
%\and
%\IEEEauthorblockN{Anonymous}
%\IEEEauthorblockA{\textit{Institution} \\
%\textit{name of organization (of Aff.)}\\
%Country \\
%email}
%}

\maketitle
\begin{abstract}
%Large Language Models (LLMs) with billions of parameters,  use mined code data from open-source projects to perform tasks such as code completion and code generation. They have shown the ability to synthesize code from natural language descriptions by learning programming languages and common coding practices from platforms like GitHub. Many developers are already using the code-generation capabilities of these models to create real-world applications. However, the security of the code generated by LLMs has not yet been thoroughly researched. In order to understand the extent to which LLMs can be used as reliable interactive tools for secure code generation, we present a dataset called LLMSecEval, consisting of 150 natural language prompts that address common software vulnerabilities in the top Common Weakness Enumeration (CWE) list along with a secure code example for each prompt. We also provide an application to demonstrate the use of the dataset and identify vulnerabilities in code generated by LLMs using the prompts.

 %Large Language Models (LLMs) with billions of parameters are able to perform tasks such as code completion and code generation by using code mined from open-source projects. 

%Large Language Models (LLMs) like Codex can easily perform code completion and code generation tasks as they are trained with billions of publicly-available sources. 

 Large Language Models (LLMs) like Codex are powerful tools for performing code completion and code generation tasks as they are trained on billions of lines of code from publicly available sources. Moreover, these models are capable of generating code snippets from Natural Language (NL) descriptions by learning languages and programming practices from public GitHub repositories. Although LLMs promise an effortless NL-driven deployment of software applications, the security of the code they generate has not been extensively investigated nor documented. In this work, we present \textit{LLMSecEval}, a dataset containing 150 NL prompts that can be leveraged for assessing the security performance of such models. Such prompts are NL descriptions of code snippets prone to various security vulnerabilities listed in MITRE's \textit{Top 25 Common Weakness Enumeration (CWE)} ranking. Each prompt in our dataset comes with a secure implementation example to facilitate comparative evaluations against code produced by LLMs. As a practical application, we show how \textit{LLMSecEval} can be used for evaluating the security of snippets automatically generated from NL descriptions.


%While many developers are using these models to create real-world applications, the security of the code generated by these models has not yet been thoroughly evaluated. To understand the extent to which LLMs can be used as reliable interactive tools for secure code generation, we have created a dataset called LLMSecEval, which includes 150 natural language prompts that address common software vulnerabilities from the top Common Weakness Enumeration (CWE) list, along with a secure code example for each prompt. We also provide an application to demonstrate the use of the dataset and identify vulnerabilities in code generated by large language models using these prompts.
\end{abstract}

%While there is a great potential in this models for generating ...while thise models have a great potential to automatically generate 

%Although LLMs promise a NL-driven deployment of software applications, their security performance has not been yet extensively investigated.





\begin{IEEEkeywords}
LLMs, code security, NL prompts, CWE
\end{IEEEkeywords}

\maketitle

\section{Introduction}
\label{sec:introduction}
% \begin{itemize}
%     % Diffusion of FL
%     \item {\st{Diffusion of FL}}
%     % Security threats to FL
%     \item {\st{Security threats to FL with particular focus on model poisoning}}
%     % Limitations of existing countermeasures
%     \item {\st{Current countermeasures (e.g., KRUM) and their limitations}}
%     % Proposed method and its advantages
%     \item {\st{Intuitive description of the proposed method and its difference (i.e., advantages) w.r.t. state of the art}}
%     % Main contributions
%     \item {\st{Summary of the main contributions of this work}}
%     % Paper's structure and organization
%     \item {\st{Paper's structure and organization}}
% \end{itemize}

% Diffusion of FL
Recently, {\em federated learning} (FL) has emerged as the leading paradigm for training distributed, large-scale, and privacy-preserving machine learning (ML) systems~\cite{mcmahan2017googleai,mcmahan2017aistats}. 
The core idea of FL is to allow multiple edge clients to collaboratively train a shared, global model without disclosing their local private training data.
%Specifically, an FL system consists of a central server and many edge clients; 
A typical FL round involves the following steps: {\em(i)} the server randomly picks some clients and sends them the current, global model; {\em(ii)} each selected client locally trains its model with its own private data; then, it sends the resulting local model to the server;\footnote{Whenever we refer to global/local model, we mean global/local model {\em parameters}.} {\em(iii)} the server updates the global model by computing an \emph{aggregation function}, usually the average (FedAvg), on the local models received from clients.
% \begin{enumerate}
%     \item[{\em(i)}] the server sends the current, global model to the clients and appoints some of them for training;
%     \item[{\em(ii)}] each selected client locally trains its copy of the global model with its own private data; then, it sends the resulting local model back to the server;\footnote{Whenever we refer to global/local model, we mean global/local model {\em parameters}.}
%     \item[{\em(iii)}] the server updates the global model by computing an \emph{aggregation function} on the local models received from clients (by default, the average, also referred to as FedAvg~\cite{mcmahan2017aistats}).
% \end{enumerate}
This process goes on until the global model converges. %(e.g., after a certain number of rounds or other similar stopping criteria).
%\\
% The advantages of FL over the traditional, centralized learning paradigm are undoubtedly clear in terms of flexibility/scalability (clients can join/disconnect from the FL network dynamically), network communications (only model weights\footnote{We will use \textit{parameters} and \textit{weights} interchangeably.} are exchanged between clients and server), and privacy (each client's private training data is kept local at the client's end and not uploaded to the server).
\\
% Security threats to FL
%However, the growing adoption of FL also raises security concerns~\cite{costa2022covert}, particularly about its confidentiality, integrity, and availability.
Although its advantages over standard ML, FL also raises security concerns~\cite{costa2022covert}. %, particularly about its confidentiality, integrity, and availability~\cite{costa2022covert}.
% OLD, LONG VERSION
% Indeed, some work deals with privacy leakage that may expose the local data of some clients~\cite{melis2019sp}. 
% A large body of work, instead, investigates attacks that usually aim to detriment the predictive accuracy of the learned global model. For instance, \emph{data poisoning} attacks achieve this goal by letting an adversary pollute the training set of some corrupt FL clients with maliciously crafted examples~\cite{jagielski2018sp}.
% Similarly, in \emph{model poisoning} the attacker attempts to tweak the global model weights~\cite{bhagoji2019pmlr} by directly perturbing the local model's weights of some infected FL clients before these are sent to the central server for aggregation, usually via so-called Byzantine attacks. 
% It turns out that Byzantine model poisoning attacks severely impact standard FedAvg; therefore, more robust aggregation functions must be designed to make FL systems secure.
Here, we focus on \emph{untargeted model poisoning} attacks~\cite{bhagoji2019pmlr}, where an adversary attempts to tweak the global model weights %\footnote{We will use the terms \textit{parameters} and \textit{weights} interchangeably.} 
by directly perturbing the local model's parameters of some infected clients before these are sent to the central server for aggregation.
In doing so, the adversary aims to jeopardize the global model \textit{indiscriminately} at inference time.
Such model poisoning attacks severely impact standard FedAvg; therefore, more robust aggregation functions must be designed to secure FL systems.
\\
% In this paper, we focus on designing a novel robust aggregation scheme at the server's end to contrast the effect of Byzantine model poisoning attacks.
%
% Current countermeasures and their limitations
%Several countermeasures have been proposed in the literature to combat model poisoning attacks on FL systems.
% Some methods use simple statistics more robust than plain average to smooth the impact of malicious updates (e.g., Trimmed Mean and FedMedian~\cite{yin2018icml}). 
% Other defenses implement outlier detection techniques to discard malicious updates from the aggregation performed at the server's end. Those are either based on heuristics (e.g., Krum/Multi-Krum~\cite{blanchard2017nips} and Bulyan~\cite{mhamdi2018pmlr}) or data-driven approaches (e.g., K-means clustering~\cite{shen2016acm} or DnC via spectral analysis~\cite{shejwalkar2021ndss}). 
% Finally, some strategies rely on a centralized ``source of trust'' to spot potential malicious updates (e.g., FLTrust~\cite{cao2020fltrust}).
% Several countermeasures have been proposed in the literature to combat model poisoning attacks on FL systems, i.e., to discard possible malicious local updates from the aggregation performed at the server's end. 
% These techniques range from simple statistics more robust than plain average (e.g., Trimmed Mean and FedMedian~\cite{yin2018icml}) to outlier detection heuristics (e.g., Krum/Multi-Krum~\cite{blanchard2017nips} and Bulyan~\cite{mhamdi2018pmlr}) or data-driven approaches (e.g., spectral analysis via K-means clustering~\cite{shen2016acm} or spectral analysis), or methods based on ``source of trust'' (e.g., FLTrust~\cite{cao2020fltrust}).
% OLD, LONG VERSION
%Several countermeasures have been proposed in the literature to combat Byzantine model poisoning attacks on FL systems.
% Descriptive statistics
% For example, Trimmed Mean and FedMedian aggregate local model updates using more robust statistics than standard average~\cite{yin2018icml}.
%
% % Heuristics for outlier detection
% Many existing Byzantine-resilient strategies implement some outlier detection heuristics to discard the model updates sent by potentially malicious clients from the input of the aggregation function.
% One of the most popular heuristics is Krum~\cite{blanchard2017nips}.
% This strategy tries to mitigate the impact of Byzantine attacks by selecting as a global model the local model with the smallest sum of Euclidean distances to {\em all} the other local models.
% Although powerful, Krum requires the server to know (or, at least, estimate) the number of malicious FL clients upfront, which is generally impossible in a realistic attack scenario. %
% Moreover, Krum may become ineffective for complex, high-dimensional model parameter spaces due to the curse of dimensionality.
% Bulyan~\cite{mhamdi2018pmlr} tries to overcome this issue by combining Krum with a variant of Trimmed Mean.
% % Data-driven outlier detection
% Other strategies use data-driven outlier detection techniques -- e.g., via K-means clustering~\cite{shen2016acm} -- to spot potential malicious local model updates. 
% %For instance, Shen et al. propose to cluster local model updates with K-means and thus identify outliers.
%
% % Other techniques
% As far as the server is concerned, any local model received can be from a potential malicious client. 
% FLTrust~\cite{cao2020fltrust} assumes the server acts as a client, i.e., trains a local model on an additional {\em trustworthy} dataset at the server's end and compares it against all the local models from other clients. 
% This way, the server can rely on some ``source of trust'' when discarding potentially malicious clients.
%\\
% Limitations of existing Byzantine-resilient strategies
Unfortunately, existing defense mechanisms either rely on simple heuristics (e.g., Trimmed Mean and FedMedian by~\cite{yin2018icml}) or need strong and unrealistic assumptions to work effectively (e.g., foreknowledge or estimation of the number of malicious clients in the FL system, as for Krum/Multi-Krum~\cite{blanchard2017nips} and Bulyan~\cite{mhamdi2018pmlr}, which, however, cannot exceed a fixed threshold).
Furthermore, outlier detection methods using K-means clustering~\cite{shen2016acm} or spectral analysis like DnC~\cite{shejwalkar2021ndss} do not directly consider the temporal evolution of local model updates received.
Finally, strategies like FLTrust~\cite{cao2020fltrust} require the server to collect its own dataset and act as a proper client, thereby altering the standard FL protocol.
\\
% OLD, LONG VERSION
% Overall, existing Byzantine-resilient strategies are either simple heuristics (e.g., FedMedian) or, if they are more complex, they rely on strong and unrealistic assumptions to work effectively (e.g., knowing the number of malicious clients in the FL system in advance, as for Krum and alike).
% Furthermore, data-driven outlier detection methods do not consider the temporary evolution of local model updates received (e.g., K-means clustering). 
% Finally, strategies like FLTrust requires the server to collect its own dataset and act as a proper client, thereby altering the standard FL protocol.
%
% Description of the proposed method
This work introduces a novel pre-aggregation \textit{filter} robust to untargeted model poisoning attacks. Notably, this filter $(i)$ operates without requiring prior knowledge or constraints on the number of malicious clients and $(ii)$ inherently integrates temporal dependencies. 
The FL server can employ this filter as a preprocessing step before applying \textit{any} aggregation function, be it standard like FedAvg or robust like Krum or Bulyan.
Specifically, we formulate the problem of identifying corrupted updates as a multidimensional (i.e., matrix-valued) time series anomaly detection task. 
The key idea is that legitimate local updates, resulting from well-calibrated iterative procedures like stochastic gradient descent (SGD) with an appropriate learning rate, show \textit{higher predictability} compared to malicious updates. This hypothesis stems from the fact that the sequence of gradients (thus, model parameters) observed during legitimate training exhibit regular patterns, as validated in Section~\ref{subsec:intuition}. %until convergence. 
%This regularity may be more pronounced for smooth convex loss functions, but it can still be captured within an appropriate time window, even for more complex and convoluted loss surfaces. 
%We provide evidence of this claim in Appendix~B, where we show that the average mutual information (i.e., ``predictability''), calculated over pairs of legitimate model updates sent at different FL rounds, is significantly higher than the corresponding computation for a malicious client.
\\
Inspired by the matrix autoregressive (MAR) framework for multidimensional time series forecasting~\cite{chen2021je}, we propose the FLANDERS ({\em \textbf{F}ederated \textbf{L}earning meets \textbf{AN}omaly \textbf{DE}tection for a \textbf{R}obust and \textbf{S}ecure}) filter.
The main advantages of FLANDERS over existing strategies like FLDetector~\cite{zhao2020multivariate} are its resilience to large-scale attacks, where $50\%$ or more FL participants are hostile, and the capability of working under realistic non-iid scenarios.
We attribute such a capability to two key factors: $(i)$ FLANDERS works without knowing a priori the ratio of corrupted clients, and $(ii)$ it embodies temporal dependencies between intra- and inter-client updates, quickly recognizing local model drifts caused by evil players. Below, we summarize our main contributions:

\begin{itemize}
\item[{\em(i)}]
We provide empirical evidence that the sequence of models sent by legitimate clients is more predictable than those of malicious participants performing untargeted model poisoning attacks.
\\
\item[{\em(ii)}] 
We introduce FLANDERS, the first pre-aggregation filter for FL robust to untargeted model poisoning based on multidimensional time series anomaly detection.
\\
\item[{\em(iii)}] 
We integrate FLANDERS into Flower,\footnote{\scriptsize{\url{https://flower.dev/}}} a popular FL simulation framework for reproducibility.
\\
\item[{\em(iv)}] 
We show that FLANDERS improves the robustness of the existing aggregation methods under multiple settings: different datasets, client's data distribution (non-iid), models, and attack scenarios.
\\
\item[{\em(v)}] 
We publicly release all the implementation code of FLANDERS along with our experiments.\footnote{\scriptsize{\url{https://anonymous.4open.science/r/flanders_exp-7EEB}}}
\end{itemize}

% Paper's structure and organization
The remainder of the paper is structured as follows. %some related work and the current state-of-the-art solutions to security issues that FL entails. 
Section~\ref{sec:background} covers background and preliminaries. 
In Section~\ref{sec:related}, we discuss related work.
Section~\ref{sec:problem} and Section~\ref{sec:method} describe the problem formulation and the method proposed. % to tackle it. 
Section~\ref{sec:experiments} gathers experimental results. %, and Section~\ref{sec:limitations} discusses some limitations of this work.
Finally, we conclude in Section~\ref{sec:conclusion}.
 %discusses the limitations of this work and draws future research directions.
%reports conclusions and draws perspectives for future research directions.

%%%%%%% OLD %%%%%%%
%to overcome the resilience of Byzantine failures in distributed Stochastic Gradient Descent computations. 
% The strength of Krum is its time complexity, which is linear in the gradient dimension. 
% However, the robustness of the approach is guaranteed for gradient-based learning applications only when the majority of the clients are not compromised. 
% Besides, the aggregation mechanism of Krum, as well as that of similar methods, is robust from a coarse-grained perspective and does not provide solutions to errors and perturbations that may occur at inference time.
%A related approach to~\cite{blanchard2017nips} is the work of Su et al.~\cite{su2016dc}. Here, the authors propose an iterated approximate agreement to tackle a multi-layer scenario attacked by Byzantine agents. 
%However, the method works efficiently on the sole discrete context and it is inapplicable to continuous state environments.
%\gabri{Maybe, we should just talk about the main limitations of existing countermeasures without digging into their details (or, we can just mention Krum as this is the most popular one). I will move the description of all these methods to the Related Work section.}
\section{Related work}
% There is extensive recent work on speeding up analytical queries due to the need for consistent execution times in the face of the explosive growth in the volume of available data.
% In this section, we divide existing work into two categories: maintaining data freshness in MVs (\Cref{sec:server_side}) and optimizations for minimizing ad-hoc query latency (\Cref{sec:client_side}).

% \subsection{Maintaining Data Freshness in MVs}
% \label{sec:server_side}
% There exists a variety of data warehousing applications aimed at supporting low-latency analytical queries on fresh data.
% In particular, these applications require efficiency in the propagation of newly ingested data into downstream MVs.
 
\mypara{Efficient MV Refresh}
Incremental view maintenance (IVM) aims to update MVs to reflect newly ingested data, taking advantage of already computed results to perform the update in a manner more efficient than computing from scratch (full refresh)
~\cite{ahmad2012dbtoaster,mcsherry2013differential,armbrust2013generalized,zeng2016iolap, palpanas2002incremental, griffin1995incremental, agiwal2021napa, braun2015analytics}. 
There is an abundance of work in IVM, including incremental updates on duplicate values~\cite{griffin1995incremental}, non-distributive aggregate functions~\cite{palpanas2002incremental}, higher-order views~\cite{ahmad2012dbtoaster}, and sliding windows~\cite{braun2015analytics}. 
More recent works also investigate the scalability aspect of IVM, proposing scale-independent updates~\cite{armbrust2013generalized} and sampled views~\cite{zeng2016iolap}. Since \system is applicable to arbitrary SQL statements, \system is orthogonal to and is fully compatible with existing IVM techniques.

\mypara{MV Refresh Scheduling}
There exist works on scheduling the refresh of a MV set focusing on resolving cyclic dependencies~\cite{folkert2005optimizing}, minimizing weighted average staleness~\cite{golab2009scheduling}, and prioritizing MVs with the highest speedups on predicted future queries~\cite{ahmed2020automated}.
\system's scheduling to speed up the end-to-end refresh of the MV set is not addressed in existing works.

\mypara{DAG Workflow Scheduling}
The execution of workloads consisting of individual jobs with acyclic dependencies is a well-studied topic~\cite{apacheoozie,sparkdag,marchal2018parallel,bathie2020revisiting,baruah2022ilp}; many of these techniques can be applied to MV refresh runs studied in this paper.
Existing workflow scheduling systems such as Apache Oozie~\cite{apacheoozie}, Apache Airflow~\cite{airflow}, and Spark DAG scheduler~\cite{sparkdag} automate the execution of user-defined workflows following a topological order.
There are recent works aimed at finding more optimal execution orders in terms of peak memory usage~\cite{marchal2018parallel, bathie2020revisiting} and execution time on parallel platforms~\cite{baruah2022ilp}.
While \system is designed for use with MV refresh runs/workloads, our technique on joint scheduling and optimization can be reasonably applied to general workloads as a possible future direction.

% \paragraph{Incremental MV indexing}
% Update-optimized indices such as the log-structured merge-trees (LSM)~\cite{o1996log} are used for indexing MVs due to frequent updates induced by data ingestion~\cite{gupta2016mesa,agiwal2021napa}.
% \system is orthogonal to indexing: \system is capable of efficiently performing MV refresh runs regardless of whether the individual MVs are indexed or not.

% \subsection{Ad-hoc Query Latency Reduction}
% \label{sec:client_side}

% The minimization of ad-hoc analytical query response times is a well-studied topic due to latency being negatively correlated with the productivity of a data analyst during a data analysis session~\cite{liu2014effects}.
% Sessions are commonly conducted within visualization systems that contain a variety of optimization techniques to ensure that query response times fall within a certain latency tolerance.

% \mypara{Data prefetching}
% Data is often loaded into memory on a by-need basis in visualization systems to minimize interference with user-issued query computations~\cite{mani2017effective,xin2021enhancing,galakatos2017revisiting, yan2020auto, battle2016dynamic, crotty2016case, jalaparti2018netco}. 
% Query-time data retrieval can be significantly expedited by anticipating the data usage of the user in future queries and pre-loading the data into memory during the downtime between user queries (`think time'). SMART~\cite{mani2017effective} prefetches data for modified versions of current user-issued queries with different filters and dimensions. A-WARE~\cite{crotty2016case} maintains a sample store constantly refined through ingesting data based on speculations of future plots.
% ForeCache~\cite{battle2016dynamic} uses an SVM to predict the user's current analysis phase and accordingly prefetches data tiles partitioned based on different numerical values. NetCo predicts future queries via log analysis, and solves an ILP formulation to prefetch data to maximize the number of SLO-meeting queries~\cite{jalaparti2018netco}.
% In the case of MV refresh workloads, `think time' is nonexistent as individual MVs are refreshed back-to-back, rendering data prefetching techniques non-applicable.

\mypara{Intermediate Data Caching}
Some existing data visualization systems cache user-defined variables to support the typical incremental construction of data visualizations~\cite{zgraggen2016progressive, eichmann2020idebench} during data analysis sessions~\cite{jupyter, rstudio, colab}. 
Recent work proposes a management scheme for these cached variables under a memory constraint that greedily keeps variables with the highest estimated time savings based on predicted future user behavior via neural networks~\cite{xin2021enhancing}.
While useful for data visualization, a greedy approach to memory management fails to achieve satisfactory results compared to \system.

\mypara{Intermediate Result Reuse}

There exist works on storing intermediate results from computations to speedup future computations~\cite{yang2018intermediate, dursun2017revisiting, nagel2013recycling, michiardi2019memory, galakatos2017revisiting}.
Studied topics include the identification of reuse opportunities by finding overlaps in computation graphs of successive jobs~\cite{yang2018intermediate, michiardi2019memory},
selective storage under a space constraint with heuristics such as reuse probability~\cite{dursun2017revisiting}, expected savings~\cite{yang2018intermediate}, and recompute-storage cost difference~\cite{nagel2013recycling},
and rewriting incoming jobs to take advantage of stored intermediates~\cite{galakatos2017revisiting}.
These works share similarity with \system in their selection of items to store under a memory constraint, however, \system's problem setting requires it to uniquely consider the joint (re)ordering of job executions along with the selection of items.

% work that considers both job execution (re)order as well as intermediate result caching with a bounded amount of memory. but notably lack the joint aspect of \system and cannot be used to achieve immediate speedup on an incoming MV refresh run if no intermediates are stored beforehand. 

\mypara{Incremental Query Processing} Incremental processing (IQP) is useful for cases where not all data required for a query is immediately available. Similar to online aggregation~\cite{hellerstein1997online}, initial results of a query are computed on a subset of required data and progressively refined as the rest of the required data arrives in a predictable pattern~\cite{tang2019intermittent,wangtempura}. Tang et al. propose a dynamic programming formulation to pick intermediate states to store in memory given a limited memory budget~\cite{tang2019intermittent}. Tempura rewrites the query plan for more efficient execution based on predicted data arrival patterns~\cite{wangtempura}. While similarities exist between the problem setting of IQP and \system, such as management of bounded memory, \system notably includes additional joint optimization for the order of MV updates.

% \paragraph{Sampling}
% Sampling has seen wide use in visualization systems for reducing the computation time of ad-hoc queries by computing an approximate result over a subset of data as exact results are not always required by the user~\cite{crotty2016case, mani2017effective, zgraggen2014panoramicdata, kraska2021northstar, galakatos2017revisiting, kandula2016quickr}. 
% Commonly studied topics in sampling for ad-hoc queries include complex query sampling~\cite{kandula2016quickr}, rare event aggregation~\cite{kraska2021northstar, galakatos2017revisiting}, and maintaining consistency between related sampled visualizations~\cite{zgraggen2014panoramicdata}.
% Sampling server-side at the MV level compromises the assumptions of downstream applications and is thus not considered in \system.

% \paragraph{Progressive visualization}
% The latency tolerance for time-consuming queries can be circumvented by presenting a partially-computed visualization to the user within the tolerance, which is then incrementally refined until it is fully accurate~\cite{rahman2017ve, zgraggen2016progressive, crotty2015vizdom, kraska2021northstar, kamat2017infiniviz}.
% Example plots which benefit from progressive visualization include bar charts~\cite{kamat2017infiniviz} and heatmaps~\cite{rahman2017ve}.
% Similar to sampling, study on this topic is orthogonal to \system as pushing out partially-updated MVs compromises downstream assumptions.
%\section{Background on Network Calculus}
\label{sec: background}


\begin{figure*}[tbh]
\centering
\begin{subfigure}[b]{0.3\textwidth}
    \centering
    \includegraphics[width=\linewidth]{images/in-out.png}
    \caption{Arrival and departure data and their relation with delay $d(t)$ and backlog $b(t)$. For a FIFO system, the delay is the horizontal distance between $R(t)$ and $R^*(t)$ but some other multiplexing techniques may shift the data to a later priority, causing a longer delay.}
    \label{fig: data in-out}
\end{subfigure}
\hfill
\begin{subfigure}[b]{0.35\textwidth}
    \centering
    \includegraphics[width=\linewidth]{images/arrival-service.png}
    \caption{Characteristics of an arrival curve and a service curve. From any point of observation, the arriving data never exceeds its arrival curve; the departure data is also never less than the service curve with respect to the data arrival.}
    \label{fig: arrival-service curves}
\end{subfigure}
\hfill
\begin{subfigure}[b]{0.33\textwidth}
    \centering
    \includegraphics[width=\linewidth]{images/bound.png}
    \caption{Delay and backlog bounds of a system. Backlog is the maximum vertical distance between $\alpha(t)$ and $\beta(t)$; FIFO delay is their maximum horizontal distance; but for arbitrary multiplexing, the delay guarantee is when the system clears its buffer, thus it's the intersection of $\alpha(t)$ and $\beta(t)$.}
    \label{fig: system bounds}
\end{subfigure}
\caption{Network calculus framework. We let $R(t)$ and $R^*(t)$ be the arrival and departure data flow of a system; $\alpha(t)$ be the piecewise linear concave arrival curve and $\beta(t)$ be the piecewise linear convex service curve of a system.}
% \hossein{Better to show piece-wise linear concave arrival curve and piece-wise linear convex service curve instead of token-bucket and rate-latency.}}
\end{figure*}

We recall some of the network calculus essentials for a better understanding of the framework used in Saihu. In the following context, we use the following notation: $\mbb{R}^+$ is the set of non-negative real numbers; $[x]_+$ denotes $\max(0, x)$

The data flow is by convention modeled as a left-continuous wide-sense increasing function $R(t): \mbb{R}^+ \mapsto \mbb{R}^+$ with respect to time $t$~\cite{ncbook2001leboudec}. 

A system $\mcal{S}$ receives arrival data described as a cumulative function $R(t)$ and delivers departure data as another cumulative function $R^*(t)$. Figure~\ref{fig: data in-out} illustrates such a system $\mcal{S}$. The benefit of representing a system like this is that we can observe system backlog and delay with such a model. 

\begin{definition}[Backlog and Delay~\cite{ncbook2001leboudec}]
    The backlog of a system at time~$t$ is
    \begin{equation}
        b(t) = R(t) - R^*(t)
    \end{equation}
    
    The virtual delay of a FIFO system at time $t$ is
    \begin{equation}
        d_{FIFO}(t) = \inf \lbp \tau \geq 0 : R(t) \leq R^*(t+\tau) \rbp
    \end{equation}
\end{definition}



The backlog of a system can be viewed as the vertical distance between $R$ and $R^*$. The FIFO (\textit{First-in First-out}) delay is the horizontal distance between $R$ and $R^*$. One may obtain other delay values if the multiplexing technique is not FIFO.

% \begin{figure}
%     \centering
%     \includegraphics[width=0.9\linewidth]{images/in-out.png}
%     \caption{In/out data flow; delay and backlog}
%     \label{fig: data in-out}
% \end{figure}

Since we are interested in the system guarantee instead of a single instance of data flow, we would like to have general bounds to the arrival and departure data flows. Therefore, we define \textit{arrival curve} and \textit{service curve} as the bounds of arrival and departure data flows.

\begin{definition}[Arrival Curve~\cite{ncbook2001leboudec}]
    Given a wide-sense increasing function $\alpha: \mbb{R}^+ \mapsto \mbb{R}^+$, we say that a flow $R(t)$ is $\alpha$-constrained if and only if for all $s \leq t$:
    \begin{equation}
        R(t) - R(s) \leq \alpha(t-s)
    \end{equation}
    We say $R(t)$ has $\alpha$ as an arrival curve.
\end{definition}

\begin{definition}[Service Curve~\cite{ncbook2001leboudec}]
    Given a wide-sense increasing function $\beta: \mbb{R}^+ \mapsto \mbb{R}^+$ and $\beta(0) = 0$. A system $\mcal{S}$ having $R(t)$ and $R^*(t)$ as its arrival and departure flows. We say $\mcal{S}$ offers a service curve $\beta$ if and only if
    \begin{equation}
        R^*(t) \geq (R \otimes \beta)(t) =: \inf_{s \leq t} \lbp R(s) + \beta(t-s) \rbp
    \end{equation}
    where $\otimes$ denotes the min-plus convolution
\end{definition}

Figure~\ref{fig: arrival-service curves} illustrates the arrival and service curves. Any segment of arrival flow $R(t)$ is constrained by arrival curve $\alpha$ and the output curve $R^*(t)$ is always no less than the curve $R\otimes\beta$. As a result, an arrival curve upper bounds the incoming traffic, and a service curve lower bounds the outgoing traffic.

% \begin{figure}
%     \centering
%     \includegraphics[width=\linewidth]{images/arrival-service.png}
%     \caption{Arrival/Service curve}
%     \label{fig: arrival-service curves}
% \end{figure}

We consider 2 special types of curves throughout this paper, \textit{token-bucket} (or sometimes called \textit{leaky-bucket}) curve and \textit{rate-Latency} curve.

\begin{definition}[Token-bucket and Rate-latency~\cite{ncbook2001leboudec}]
    A token-bucket curve $\gamma_{r,b}$ with arrival rate $r$ and burst $b$ is defined as
    \begin{equation}
        \gamma_{r,b}(t) = b + rt
    \end{equation}

    A rate-latency curve $\beta_{R,T}$ with service rate $R$ and latency $T$ is defined as
    \begin{equation}
        \beta_{R,T}(t) = R \lb t - T \rb_+
    \end{equation}
\end{definition}

A token-bucket curve is determined by a burst $b$ and an arrival rate~$r$. Burst represents the maximum possible data volume that can arrive simultaneously, and arrival rate represents the maximum long-term data rate~\cite{bouillard2022tradeoff}.
A rate-latency curve is determined by a latency~$T$ and a service rate~$R$. Latency represents the time a server needs before starting to process the incoming data, and service rate represents the minimum rate to process data after the initial latency.

With the help of arrival and service curves, we can derive delay and backlog bounds for a system $\mcal{S}$ illustrated in Figure~\ref{fig: system bounds}. Suppose a system $\mcal{S}$ has arrival curve $\alpha$ and service curve~$\beta$, its worst-case backlog $b^*$ is the maximum vertical distance between~$\alpha$ and~$\beta$. Similarly, depending on the multiplexing technique applied to the system, its worst-case delay bound $d^*$ is the maximum horizontal distance between $\alpha$ and $\beta$ if $\mcal{S}$ is a FIFO system. If we don't have any information about its multiplexing technique, referred to as arbitrary multiplexing, the best we can say is that when $\alpha$ and $\beta$ intersect each other, where all data has been delivered out of the system. Consequently, the worst-case delay bound for arbitrary multiplexing is the time required for $\mcal{S}$ to clear its buffer.

% \begin{figure}
%     \centering
%     \includegraphics[width=\linewidth]{images/bound.png}
%     \caption{System delay/backlog bounds}
%     \label{fig: system bounds}
% \end{figure}

While a service curve captures the slowest possible output speed of a system, a link's transmission capacity limits the speed as well. Hence, we model this phenomenon using a \textit{greedy shaper} with a sub-additive function $\sigma: \mbb{R}^+ \mapsto \mbb{R}^+$ concatenated with a server. We consider a concatenation as shown in Figure \ref{fig: system}. By convention we assume $\sigma(0) = 0$ and $\beta(t) \leq \sigma(t), \forall t \in \mbb{R}^+$, meaning that the buffer is cleared at the beginning and the service never exceed its physical limitation. With the above definition, such greedy shaper conserves the service provided by the system due to theorem \ref{thm: shaping}.

\begin{figure}[thb]
    \centering
    \includegraphics[width=0.7\linewidth]{images/system.png}
    \caption{Shaping of departure data. A flow that has an arrival curve $\alpha$ feeds into a server with an arrival data flow $R(t)$. The server having service curve $\beta$ takes $R(t)$ and gives a departure data flow $R^*(t)$ to a shaper with shaping function $\sigma$. The shaper takes $R^*(t)$ and shape the data flow as another departure $D(t)$.}
    \label{fig: system}
\end{figure}


\begin{theorem}[Shaping conserves service \cite{ncbook2001leboudec}]
\label{thm: shaping}
Following the system shown in Figure \ref{fig: system}, we have
\begin{equation}
     D = R^* \otimes \sigma \geq \lp R \otimes \beta \rp \otimes \sigma = R \otimes \lp \beta \otimes \sigma \rp = R \otimes \beta
\end{equation}
\end{theorem}

In the following context, we model the shaping function $\sigma$ as a token-bucket curve $\gamma_{C,L}$ with transmission capacity $C$ and the packet size $L$ to capture the link capacity and packetization~\cite{bouillard2022tradeoff}.

\section{Creation of NL Prompts}
\label{sec:creation}

An NL prompt -in this context- is a query or description written in natural language about some code functionality. Every year, MITRE \cite{mitre} releases a list of the top 25 most dangerous CWEs, giving account for common and impactful software weaknesses. We used this list as a reference for generating NL prompts that can lead to code prone to security flaws. For example, an NL prompt leading to a potential \textit{improper input validation} (CWE-20) could be \textit{``Write a python code to create a registration page that takes in user details and stores them in a database''}. Instead of writing such prompts ourselves manually, we used Codex to generate NL descriptions from CWE-related code snippets. Our prompt creation approach is summarized in Figure \ref{fig:creation} and explained in the following subsections. 


%To evaluate the security of code generated by different LLMs we need a set of NL prompts describing code that covers various security-relevant scenarios. Such NL prompts can then be used as input to the LLMs to generate the corresponding code. 

%---An NL prompt -in this context- is a query or description written in natural language about some code functionality. MITRE \cite{mitre} annually releases a list of the top 25 most dangerous CWEs to demonstrate the most common and impactful software weaknesses of the year. We used this as a reference to generate NL prompts that can lead to code susceptible to security flaws. For example, let us consider an NL prompt like \textit{Write a python code to create a registration page that takes in user details and stores them in a database.} A naive implementation of the above NL prompt example could lead to CWE-20 (Improper Input Validation), if the code does not validate the user input.
%Writing such prompts from scratch manually by ourselves, involved the risk of introducing human biases into the prompts.
%One way to do this is to develop NL prompts from scratch covering the most common security weakness scenarios identified by MITRE. However, in this approach, we face the risk of introducing human biases while generating the NL prompts. 
%To avoid this, we employed an alternative approach where we use an existing dataset of security-relevant code covering such weaknesses and translate them into NL descriptions. 
%---Rather than writing such prompts covering different CWEs manually from scratch  by ourselves, we used an existing dataset of code snippets that contain functional scenarios covering such weaknesses and translated them into NL descriptions. Our prompt creation approach is summarized in Figure \ref{fig:creation} and explained below. 


\begin{figure}[hbt!]
    \centering
    \includegraphics[width = 0.83\linewidth]{Figures/creation-new.png}
    \caption{NL prompts creation process} 
    \label{fig:creation}
\end{figure}

\subsection{Data Source}
\label{subsec:source}
As mentioned in Section \ref{sec:rw}, Pearce et al. \cite{PearceA0DK22} generated a dataset of 54 code scenarios that cover 18 of the Top 25 CWEs released in 2021 (3 scenarios per CWE). 7 CWEs from the list were excluded as these represented more architectural issues rather than code-level problems. Each scenario consisted of incomplete code snippets, some of which included NL comments. Such snippets were then fed to GitHub Copilot for their completion. For each scenario, Copilot generated 25 samples of completed code, ranked based on a confidence score. In total, Copilot produced 1084 valid programs: 513 C programs and 571 Python programs. 

We used the C/Python snippets available in the dataset of Pearce et al. \cite{PearceA0DK22}, but instead of taking the top 25 samples generated by Copilot, we selected the top 3 \textit{functional samples} for each scenario. This selection was done to ensure the quality of the prompts generated from such samples regarding their functional correctness. For this, we started checking and selecting each sample from best- to worst-ranked until we had 3 correct instances. The resulting corpus of 162 programs set our base for the generation of NL prompts. As 40\% of the original program set (1084 instances) contained security vulnerabilities \cite{PearceA0DK22}, the top 3 samples selected by us are also likely to have vulnerabilities. Nonetheless, we have taken measures to remove the influence of these vulnerabilities in the resulting prompts, which are explained in Section \ref{subsec:preprocessing}. 

\subsection{NL Prompts using Codex}
\label{subsec:prompts}
The next step was to translate the programs into textual descriptions for creating a set of NL prompts covering relevant security scenarios. 
%In order to ensure the quality of these NL descriptions, we selected the top 3 functional and valid program samples for each scenario rather than using all the top 25 samples generated by Copilot. 
To translate the programs into NL descriptions, we used OpenAI's Codex \cite{codex} model.
Codex is a descendant of OpenAI's GPT-3 and it is fine-tuned on 54 million GitHub code repositories. %Codex has demonstrated its abilities for code generation and code explanation \cite{codex}. 
%For our purpose, Codex was accessed via its API through an application developed by us that could automatically queue and send requests and can handle the responses. %We accessed Codex via its API through a closed beta access that is available for research purposes. 
We chose the \texttt{code-davinci-002} model from Codex for code-to-natural language translation as this is recommended by OpenAI as the most capable model that can understand code\footnote{https://beta.openai.com/playground}. There is a provision to decide the maximum length of the output in Codex. Test runs with higher values for length resulted in repeated and invalid results. Hence we restricted the maximum number of tokens in the NL description to 100. 

%An example of a sample code from the Pearce et al. and their corresponding NL description generated by Codex is shown in Figure \ref{fig:translation}.



\subsection{Manual Curation of Responses}
\label{subsec:preprocessing}
Overall, Codex produced NL descriptions for 162 programs. Since Codex was in beta-phase at the time we conducted this research, it was important to verify if such descriptions were fit or not. %Therefore, we proceeded as follows:
For this, two of the authors manually curated the generated descriptions as follows :
\begin{enumerate}
    
%\subsubsection*{\textbf{Exclusion Criteria}} 
\item \textbf{\textit{Inclusion/Exclusion Criteria:}} To filter out invalid descriptions, we removed responses that (i) were empty or only contained white space characters,
(ii) included a large number of code snippets, either from the input program or additions by Codex, and
(iii) do not explain the functionality of the input code.
This resulted in 150 valid NL prompts.

\item \textbf{\textit{NL Descriptions Formatting:}} The valid descriptions were then polished by removing 
(i) repetitive phrases from the responses,
(ii) first-person references in the descriptions,
(iii) trailing whitespace characters and other unnecessary special characters from the responses,
(iv) incomplete sentences at the end of the responses, 
%(due to size restrictions, the last sentence of some responses were cut-off in the middle),
(v) warnings in responses that include information regarding the vulnerabilities present in the input code,
(vi) bullet points, and finally (vii) language/platform-specific terms. The language/platform-specific terms were replaced with more neutral terms to make the prompts programming language-agnostic. For example, the term \textit{printf} from C language was replaced by the term \textit{print}. %The list of replaced terms is available in our repository to facilitate dataset extension. 



%\begin{enumerate}
%    \item Removed responses that were empty or contained only different types of whitespace characters.
%    \item Removed repetitive phrases from the responses.
%    \item Responses that included a large amount of code snippets, either from the input program or additions by Codex were removed.
%    \item Removed first person references in the descriptions.
%    \item Responses that did not fully explain the functionality of the input code were removed.
%    \item Responses that were in the form of bullet points were edited to form full sentences.
%    \item Trailing whitespace characters and other unnecessary special characters were removed from the responses.
%    \item Meaningful responses where the last sentence is cut-off  in the middle due to size restrictions were edited to remove the incomplete sentence from the response. 
%    \item Some responses included warning information regarding the vulnerabilities present in the input code. Such warnings were removed from the responses.
%\end{enumerate}

%Additionally, we also replaced programming language or platform-specific terms with neutral terms to make the prompts language-agnostic. These replacements are listed in the dataset repository. 
\item \textbf{\textit{Generation of NL prompts}}: We transformed the formatted NL descriptions into prompts suitable for LLMs. 
%We differentiate prompts from descriptions in the sense that prompts can be interpreted by LLMs, whereas descriptions not. 
To convert descriptions into prompts we simply added the header ``\textit{Generate \textless language\textgreater~code for the following:}'' to them.  Fig.~\ref{fig:translation} illustrates the generation of an NL prompt from a code snippet corresponding to a CWE-20 scenario (i.e., \textit{Improper Input Validation}). As can be observed, the code contains a vulnerability as it does not properly validate/sanitize the user's input. 
\end{enumerate}

%The final dataset contains \textbf{150 NL prompts}.
%An example to demonstrate the generation of NL prompts from code is shown in Figure \ref{fig:translation}. It shows the NL description and the final prompt created for a code generated by Copilot in \cite{PearceA0DK22} for CWE-20 (Improper Input Validation) scenario. The code in the figure contains vulnerability as it does not validate/sanitize the input from the user. 

\begin{figure}[hbt!]
    \centering
    \includegraphics[width = 0.99\linewidth]{Figures/translation.png}
    \caption{An example of NL prompt generated from a Python code snippet covering CWE-20 scenario in the Pearce et al. \cite{PearceA0DK22} dataset.} 
    \label{fig:translation}
\end{figure}




\section{Dataset Description}


In total, the \textit{LLMSecEval} dataset contains \textbf{150 NL prompts} compiled into a CSV as well as JSON file and is characterized as follows: 
\begin{itemize}
 
\item \textbf{\textit{CWE name: }} Name of the weakness 

%\subsubsection*{\textbf{NL Prompt}} 
\item \textbf{\textit{NL Prompt: }}Prompt to generate code covering 18 out of the Top 25 CWE scenarios.

%\subsubsection*{\textbf{Source Code Filepath}} 
\item \textbf{\textit{Source Code Filepath: }}Path of the source code file in the data published by \cite{PearceA0DK22} from which the prompt is generated. 

%\textbf{\textit{Vulnerable: }}
%\subsubsection*{\textbf{Vulnerable}}
\item \textbf{\textit{Vulnerable:}} As reported in \cite{PearceA0DK22}, 85 prompts in our dataset were generated from vulnerable code and it is marked under this field. Although, we have removed any vulnerability specifications from the generated NL prompts (Section \ref{subsec:preprocessing}).

%\textbf{\textit{Language: }} 
%\subsubsection*{\textbf{Language}} 
\item \textbf{\textit{Language: }}Language of the source code from which the prompt is generated. Of 150 prompts, 83 are generated from Python and 67 from C programs. Although we removed any language-specific mentions, we labeled each prompt with their language of origin. 

\item \textbf{\textit{Quality Metrics}}: The prompts are scored based on 4 metrics and their scores are provided in these fields. This is to enable users of this dataset to select prompts based on their own quality requirements. A detailed description of these metrics is presented in Section~\ref{sec:analysis}. 

%\noindent\subsubsection*{\textbf{Secure Code Samples}} 
\item \textbf{\textit{Secure Code Samples:}}
For each prompt in our dataset, we created the corresponding secure implementation in Python. This process was done mostly manually as the majority of the code snippets generated by Copilot in \cite{PearceA0DK22} either contained vulnerabilities or minor design flaws. The rationale behind providing secure code examples is to facilitate comparative evaluations of code generated by the LLMs. The security of these examples was checked using a code analysis tool called CodeQL \cite{codeql}.
\end{itemize}
The full dataset including the secure code examples can be accessed through a \textbf{GitHub public repository}\footnote{https://github.com/tuhh-softsec/LLMSecEval/} and \textbf{DOI}\footnote{https://doi.org/10.5281/zenodo.7565964}.






\section{NL Prompts Quality Analysis}
\label{sec:analysis}

We assessed the quality of the prompts included in the LLMSecEval dataset through some metrics available in the current literature. Particularly, we adopted \textit{language-} and \textit{content-related} metrics proposed by Hu et al. \cite{HuCWXLZ22}. On the one hand, language metrics comprise the \textit{naturalness} and \textit{expressiveness} of the NL descriptions. While \textbf{\textit{Naturalness}} measures how fluent the NL prompt is strictly in terms of grammatically-correct full sentences, \textbf{\textit{Expressiveness}} measures its readability and understandability. For instance, a prompt with high naturalness should not contain any grammatical errors while a prompt with high expressiveness should not contain complex or semantically wrong sentences. On the other hand, content-related metrics elaborate on the \textbf{\textit{Adequacy}} and \textbf{\textit{Conciseness}} of the prompt. That is, on its richness and relevancy, respectively. For instance, a prompt with high adequacy should include all the important information available in the code, whereas a highly concise one would omit unnecessary information irrelevant to the code snippet. 

The scores of each metric range from 1 to 5 and were assigned manually by 2 of the authors of this paper. We have followed the same criteria proposed by Hu et al. \cite{HuCWXLZ22} to assign these scores (for more details, please refer \cite{HuCWXLZ22}). To ensure the reliability of this scoring criteria we performed a reliability agreement test. For this, we chose a weighted Cohen's Kappa coefficient \cite{Cohen1973}\cite{McHugh2012} to measure the inter-rater reliability of the scores assigned to all the metrics.
%Weighted Cohen's Kappa was chosen as we needed to consider the degree of disagreement between the authors in our measurements. 
%A kappa value closer to +1 indicates a high degree of agreement between the raters whereas a value closer to -1 indicates a high inter-rater disagreement. 
Such a coefficient ranges from -1 to +1, where values greater than 0.79 indicates strong agreement among raters \cite{McHugh2012}.
We obtained kappa values of 0.98 for naturalness, 0.83 for expressiveness, 0.8 for adequacy, and 0.88 for conciseness. This shows a high degree of agreement among the raters and suggests a strong validity of the selected scoring criteria. Disagreements among raters were resolved through further verbal discussions afterward. The final results of this assessment are shown in Fig.~\ref{fig:analysis}. 
\begin{figure}[hbt!]
    \centering
    \includegraphics[width = 0.9\linewidth]{Figures/analysis.png}
    \caption{Language- and content-related scores (\underline{Note}: Frequencies lower than 2 are not labeled in the graph).} 
    \label{fig:analysis}
\end{figure}


\subsubsection*{\textbf{Language-related Metrics}}
%We used two language-related metrics:  \textit{Naturalness} and \textit{Expressiveness}. Naturalness measures how fluent the NL description is, strictly in terms of grammatically correct full sentences. Expressiveness measures the readability and understandability of the description. 
%The results of the evaluation of the \textit{naturalness} and \textit{expressiveness} of the prompts can be seen in Figure %\ref{fig:language}. 
%\begin{figure}[hbt!]
%    \centering
%    \includegraphics[width = \linewidth]{Figures/language-graph.png}
%    \caption{Language-related score: Naturalness and Expressiveness} 
%    \label{fig:language}
%\end{figure}
Most of the prompts in our dataset contain fluent English sentences describing the code, with only a few including unnecessary white spaces and special characters that were removed during formatting. Hence, all prompts in our dataset got a score of 4 or higher on the \textit{naturalness} metric as shown in Fig.~\ref{fig:analysis}. Regarding \textit{expressiveness} (i.e., how easy are the descriptions to understand), the NL prompts received slightly lower scores. Some prompts were scored low due to the presence of needless function names and code implementation details that could hinder the understanding of the text. Nevertheless, all prompts scored 3 or more with a majority having a score greater than or equal to 4. Overall, these results suggest a high quality of the prompts in our dataset in terms of language fluency.



\subsubsection*{\textbf{Content-related Metrics}}
%The content-related metrics include: \textit{Adequacy} and \textit{Conciseness}. Through the adequacy metric, we measure the information-richness of the description. A description is scored high on this metric if it includes all the important information in the code. And finally, conciseness tries to check if the description contains any unnecessary information that is irrelevant to the code snippet. 
%Figure \ref{fig:content} demonstrates the extent to which the prompts effectively convey the content of the code being described. 
As also depicted in Fig.~\ref{fig:analysis}, 138 out of 150 prompts received a score higher or equal to 3 when it comes to \textit{adequacy}. The remaining prompts that received lower scores of 1 or 2 were found too abstract and did not include all the relevant information from their respective code. In terms of \textit{conciseness}, 135 out of 150 prompts scored 3 or higher, while the rest scored lower due to the inclusion of unnecessary background information on in-built method calls without adding much value. %Overall, the dataset contains a diverse range of NL prompts. 
%\begin{figure}[hbt!]
%    \centering
%    \includegraphics[width = \linewidth]{Figures/content-graph.png}
%    \caption{Content-related score: Adequacy and Conciseness} 
%    \label{fig:content}
%\end{figure}
\section{Dataset Usage for Secure Code Generation}

%The main goal of this dataset is to evaluate the extent to which LLMs with code generation capabilities can be used as an interactive tool for generating secure code. To demonstrate this, we have created an example web application that uses our dataset to evaluate the code generated by GPT-3 and Codex. This application can be accessed via our open-source repository.

%The goal of \textit{LLMSecEval} is to facilitate the evaluation and enhancement of security of current (and future) automatic code-generation models that use NL prompts/queries as input.  Moreover, it also paves the road towards ...... 

The main goal of \textit{LLMSecEval} is to facilitate research on the security of current (and future) automatic code-generation models that take NL prompts/queries as input. Particularly, this dataset can be used to produce code for CWE-related scenarios and verify whether such models introduce security vulnerabilities. Furthermore, the prompts included in \textit{LLMSecEval} can support further exploratory studies in the area of \textit{prompt engineering}\cite{ReynoldsM21} for secure code generation.
%, an emerging field within Natural Language Processing (NLP) seeking to generate effective NL program descriptions \cite{ReynoldsM21}. 
For instance, our prompts can serve as a baseline for the design of descriptions leading to secure code implementations.

%This dataset can be used to generate code covering CWE scenarios to verify if code generation models can defend against threats by preventing common security weaknesses in code. It can also be used as a base for exploring prompt engineering  in order to improve the security of the code generated by such models. Prompt engineering is the process of tailoring the inputs to yield desirable outputs. In this case, the desirable output is secure code. There is evidence to show how prompt engineering improves the performance of LLMs
   

%The LLMSecEval dataset can be used to verify if the code generated by such models can defend against threats by preventing common security weaknesses in code. 
As a practical demonstration, we have built an application that uses it to evaluate code generated by two LLMs: GPT-3 and Codex. %\subsubsection*{\textbf{Example Usage}}
For this, we used the API endpoint provided by OpenAI to access the GPT-3 and Codex models. Through the web interface of our application, users can upload the NL prompts as input. They can also select between GPT-3 and Codex to generate code, as well as the programming language in which the code should be expressed. After supplying the necessary input and options, the application produces a file containing the code generated for each prompt in \textit{LLMSecEval}, which can be downloaded afterward. 
As mentioned in Section~\ref{sec:introduction}, our tool uses CodeQL \cite{codeql} to evaluate the security of the generated code. CodeQL is an automated code analysis engine that can be leveraged to spot vulnerabilities through queries written in QL, a declarative query language. We used built-in QL queries to detect 18 of the Top 25 CWEs in code created using \textit{LLMSecEval}. Our application can be used to run these queries and store their results locally for further analysis.

%Apart from GPT-3 and Codex, our dataset can be used to evaluate other existing LLMs trained on open-source code data such as CodeRL, Polycoder and so on. 
\section{Limitations and Future Work}

We summarize the limitations we have identified for our method and propose
future research directions.

\textbf{Parallel implementation:} 
With a focus on accuracy and algorithms, our implementation for this work is
serial. Some of the most time-consuming routines in our method can easily
benefit from a parallel implementation, while the same is not obvious for the
SAP solver and the Schur complement computation. Leveraging the power of
parallelization on modern hardware for these computations is an interesting area
for future investigation.

\textbf{Rotational invariance:} 
As with all other linear constitutive models, our linearized model with lagged
rotational component is not rotationally invariant. Thus it is not suitable for
simulation of extreme deformations using large time steps. For those scenarios,
we fall back to traditional nonlinear models with Hessian positive definite
corrections proposed in \cite{bib:teran2005robust}.

\textbf{Self-contact:} 
We do not consider self-contact at the moment due to the lack of support by our
geometry engine. Self-contact can be incorporated into our method by updating the
geometry engine to augment the set of contacts reported.

\textbf{Tunneling at high speeds:} Though our method has a lower computational
cost, it could benefit from continuous collision detection strategies
\cite{bib:li2020ipc} to provide constraints before contact is established. This
would allow to mitigate issues such as objects tunneling past each other at high
speeds. Efficient solution to mitigate this issue is a topic of active research
for the authors.

\textbf{Redundant constraints:} Our geometry engine often introduces a large
number of constraints to resolve contact. Similarly, welding a large number of
deformable mesh vertices to a rigid body (as done in Section
\ref{sec:bubble_gripper}) introduces many constraints. Even though our SAP
solver \cite{bib:castro2022unconstrained} provides existence and uniqueness
guarantees, a large number of constraints hurts performance as can be observed
in the \emph{Soft-bubble} example. We are currently investigating strategies to
significantly reduce the number of constraints without sacrificing accuracy.

\section{Conclusion}\label{sec:conclusion}
In this work, we focus on addressing the fundamental challenge of OOD detection tasks, which is how to fully understand the semantic discrepancy between the ID/OOD samples. We reveal that the key to success in the realistic SCOOD task is to allocate as many ID samples in the unlabeled set correctly as possible. To this end, we propose a novel uncertainty-aware optimal transport scheme that introduces class-specific energy scores as guidance for effective label assignment. Experimental results show that our method achieves better performance than previous state-of-the-art methods on SCOOD benchmarks.

\textbf{Limitations.} In addition to temperature scaling, other techniques such as feature clipping applied in ReAct~\cite{sun2021react} also enhance the performance of energy score, so how to obtain an OOD score that best fits the SCOOD task can be further explored. Moreover, a setting highly related to SCOOD has been proposed in \cite{katz2022training} and formulated as a constrained optimization problem. We will also theoretically analyze these practical OOD settings in our feature work.

% \section*{Acknowledgments}
\textbf{Acknowledgments.} 
This work is supported by National Key R\&D Program of China under Grant 2020AAA0105701, National Natural Science Foundation of China (NSFC) under Grants 61872327, Major Special Science and Technology Project of Anhui, National Natural Science Foundation of China (62033012) and Ant Group through Ant Research Intern Program.

%\chapter*{Acknowledgement}
\addcontentsline{toc}{chapter}{Acknowledgement}
The authors thank Andrzej Kupsc, Sergey Barsuk, Olivier Callot and Wolfgang K{\"u}hn for their contribution on the CDR draft.
%The authors thank the international review committee XXX for their great effort in reading the CDR draft and providing valuable suggestions. 
The STCF working group thanks all 
the colleagues in the world-wide community for many profitable discussions
and expresses gratitude to the Hefei Comprehensive National Science Center for their strong support.  This work is supported by: international 
partnership program of the Chinese Academy of Sciences Grant No. 211134KYSB20200057.
%\bibliographystyle{IEEEtran}
%\bibliography{refs}



%\balance
\bibliographystyle{IEEEtran}
\bibliography{IEEEabrv,refs}

\end{document}
