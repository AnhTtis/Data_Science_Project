\documentclass{amsart}
\usepackage{fullpage}
\usepackage{all2021}
\usepackage{algorithm}
\usepackage{algorithmic}
\newtheorem{example}{Example}
\newtheorem{conjecture}[thm]{Conjecture}
\newcommand{\cA}{\mathcal{A}}
\newcommand{\Log}{\operatorname{Log}}
\newcommand{\ome}{\mathbf{1}}
\newcommand{\R}{\mathbb{R}}
\newcommand{\C}{\mathbb{C}}
\newcommand{\rank}{\mbox{rank}~}
\newcommand{\fcc}{\mbox{fcc}}
\newcommand{\ignore}[1]{}

\DeclareMathOperator{\adim}{adim}

\usepackage{xcolor}
\newcommand{\jojo}[1]{\textbf{\textcolor[rgb]{1,0.41,0.13}{JR: #1}}}

\begin{document}

\title{The amoeba dimension of a linear space}
\author{Jan Draisma}
\address{Mathematical Institute, University of Bern, Sidlerstrasse 5, 3012
Bern, Switzerland; and Department of Mathematics and Computer Science,
Eindhoven University of Technology, P.O.~Box 513, 5600 MB Eindhoven,
The Netherlands}
\email{jan.draisma@unibe.ch}

\author{Sarah Eggleston}
\address{Institut f\"ur Mathematik, Albrechtstra{\ss}e 28a, 49076
Osnabr\"uck, Germany}
\email{sarah.eggleston@uni-osnabrueck.de}

\author{Rudi Pendavingh}
\address{Department of Mathmematics and Computer Science,
Eindhoven University of Technology, P.O.~Box 513, 5600 MB
Eindhoven, The Netherlands}
\email{r.pendavingh@tue.nl}

\author{Johannes Rau}
\address{Departamento de Matem\'aticas, Universidad de los Andes, 
Carrera 1 \# 18A - 12, 111711 Bogot\'a, Colombia}
\email{j.rau@uniandes.edu.co}

\author{Chi Ho Yuen}
\address{Department of Mathematics, University of Oslo,
P.O.~Box 1053, Blindern, 0316 Oslo, Norway}
\email{chihy@math.uio.no}

\thanks{JD was partially supported by Swiss National Science Foundation
(SNSF) project grant 200021\_191981 and by Vici grant 639.033.514 from the
Netherlands Organisation for Scientific Research (NWO).
JR was supported by the FAPA project ``Matroids in tropical geometry'' from the Facultad
de Ciencias, Universidad de los Andes, Colombia.
CHY was supported by the Trond Mohn Foundation project ``Algebraic
and Topological Cycles in Complex and Tropical Geometries''}

\maketitle

\begin{abstract}
Given a complex vector subspace $V$ of $\mathbb{C}^n$, the dimension of the amoeba of $V \cap (\mathbb{C}^*)^n$ depends only on the matroid of $V$. Here we prove that this dimension is given by the minimum of a certain function over all partitions $P_1, \dots, P_k$ of the ground set into nonempty parts $P_i$, as previously conjectured by Rau. We also prove that this formula can be evaluated in polynomial time.
\end{abstract}

\section{Introduction} 

\subsection{The goal}
Let $V \subseteq \CC^n$ be a complex vector subspace, and set $X:=V \cap
(\CC^*)^n$. We assume throughout that $X \neq \emptyset$, i.e., that
$V$ is not contained in any coordinate hyperplane. We denote by $\cA(X)
\subseteq \RR^n$ the {\em amoeba} of $X$, defined as the image $\Log(X)$ of $X$
under the map
\[ \Log:(\CC^*)^n \to \RR^n,\ z=(z_1,\ldots,z_n) \mapsto
(\log(|z_1|),\ldots,\log(|z_n|)). \]
An open dense subset of $\cA(X)$ (in the Euclidean topology) is a real
manifold, and we write $\dim_\RR \cA(X)$ for the dimension of this
manifold. In this paper, we derive a purely combinatorial expression for
$\dim_\RR \cA(X)$. This expression depends only on the {\em
matroid} $M_V$ that
$V$ defines on the ground set $[n]:=\{1,\ldots,n\}$ (see below), and in
fact makes sense for any matroid $M$ on that ground set, not just those
represented by some complex vector space. We show that said expression is
the rank of a different matroid $M'$ on the ground set $[n]$,
and we derive an algorithm for computing the rank of $M'$ that uses a
polynomial number of rank evaluations in $M$. In particular, if $V$ is
given as the row space of an input matrix with, say, Gaussian rational
entries, then $\dim \cA(X)$ can be computed in polynomial time in the
(bit-length of the) input.

\subsection{The matroid of $V$}
For $S \subseteq [n]$ we denote by $\rk_V(S)$ the dimension
of the image of $V$ under the projection $\CC^n \to \CC^S$. 
More explicitly,
if $d:=\dim_\CC V$ and $V$ is presented as the row space of a $d \times
n$-matrix $A$, then $\rk_V(S)$ is the dimension of the span of the
columns of $A$ labelled by $S$.  This is the rank function of the 
matroid $M_V$ defined by $V$. The assumption that $X \neq \emptyset$
is equivalent to the condition that $M_V$ has no loops, and
to the condition that $A$ has no zero column.

\subsection{Main results}
We will prove the following formula, first conjectured by
Rau \cite{Rau20}. 


\begin{thm} \label{thm:Main}
We have 
\begin{equation} \label{eq:Main} 
\dim_\RR \cA(X)=\min_{P_1 \sqcup \dots \sqcup P_k=[n], P_i
\neq \emptyset} \sum_{i=1}^k (2 \rk_V(P_i)-1).
\end{equation}
Here the minimum is over all partitions of $[n]$ into nonempty parts
$P_i$.
\end{thm}

\begin{example} \label{ex:trivial}
The two obvious upper bounds $2d-1$ (cf. Lemma~\ref{lm:2dmin1}) and $n$ (the dimension of the ambient space) can be recovered by choosing the trivial partition and the partition into singletons, respectively.
More generally, if the matroid consists of multiple connected components $M_1,\ldots,M_k$, then choosing the partiton consisting of their respective ground sets shows that the dimension is at most $2d-k$.
\end{example}

The following example (due to Mounir Nisse) shows that the $\dim_\RR \cA(X)$ can drop below $\min\{n,2d-1\}$ even if $M_V$ is connected.

\begin{example}
Consider the $4\times 7$ matrix
$$
\begin{pmatrix}
1 & 0 & 0 & 0 & * & * & *\\
0 & 1 & 0 & 0 & * & * & *\\
0 & 0 & 1 & 0 & 0 & 0 & *\\
0 & 0 & 0 & 1 & 0 & 0 & *
\end{pmatrix},
$$
where each $*$ is a generic complex number.
It is easy to check that the matroid represented by such a matrix is connected.
However, the partition of columns $\{1,2,5,6\},\{3\},\{4\},\{7\}$ indicates that $\dim_\RR \cA(X)\leq (2\cdot 2-1)+3 \cdot (2\cdot 1-1)=6<\min\{7,2\cdot 4-1\}$.

\end{example}

The next result interprets the expression in the
theorem as the rank of a matroid. To this end, 
let $M$ be a loopless matroid on the finite set $E$, with rank function $r:2^E\rightarrow\mathbb{N}$. For subsets $S_1,\ldots, S_k\subseteq E$, put
$$ \tilde{r}(S_1,\ldots S_k):=\sum_{i=1}^k (2r(S_i)-1)$$ and define the function $r':2^{E}\rightarrow\mathbb{N}$ as
 \begin{equation}\label{rprime}
 r'(S):=\min\left\{\tilde{r}(P_1,\ldots, P_k): \{P_1,\ldots,
 P_k\}\text{ is a partition of } S\text{ into nonempty parts}\right\}.\end{equation}

\begin{thm}\label{matroid} Let $M$ be a loopless matroid on
$E$ with rank function $r$, and let $r'$ be defined as in
\eqref{rprime}. Then $r'$ is the rank function of a matroid
$M'$ on $E$.
\end{thm}

Moreover, we will establish the following algorithmic result.

\begin{thm}\label{thm:Algorithm}
The rank function $r'$ of $M'$ can be evaluated on an input set $F
\subseteq E$ by a polynomial number of rank evaluations in
$M$.
\end{thm}

\begin{cor}
There exists an algorithm that on input of a $d \times n$-matrix $A$ with
Gaussian rational entries and with no zero column, computes the amoeba
dimension of $V \cap (\CC^*)^n$, where $V$ is the complex row space of
$A$, in polynomial time in the bit-length of $A$.
\end{cor}

\begin{proof}
This is a direct consequence of Theorems~\ref{thm:Main}
and~\ref{thm:Algorithm}, plus the fact that the rank of the submatrix
$A[S]$ obtained from $A$ by picking the columns labelled by any subset
$S \subseteq [n]$ can be computed in polynomial time in the bit-length
of $A$.
\end{proof}

\subsection{A question} \label{subsection:DRYformula}

From \cite{Draisma18d}, we know that $\dim_\RR \cA(X)$ depends only on
$\Trop(X)$, for any irreducible subvariety $X$ of
$(\CC^*)^n$. 
Here, $\Trop(X) = \lim_{t \to 0} t \cdot \cA(X)$ is ``constant coefficient''
tropicalisation or \emph{Bergman fan} associated to $X$ \cite{Ber-LogarithmicLimitSet}. 
In the current case, $\Trop(X)$ is equal to the \emph{matroid fan} $\Sigma(M_V)$
associated to $M_V$ \cite{Stu-SolvingSystemsPolynomial}, 
so it was already known that the amoeba dimension of $X$
depends only on $M_V$. 
However, this dependence via \cite{Draisma18d} is
rather implicit and not quite sufficient for algorithmic computation:
It states that $\dim_\RR \cA(X) = \adim(M_V)$ with  
\begin{equation} \label{eq:DRYformula} %\nonumber
  \adim(M) := \min\{2\dim_\RR(\Sigma(M)+R)-\dim_\RR(R) \mid R
\subseteq \RR^E \text{ rational subspace}\}.
\end{equation}
Here a {\em rational subspace} of $\RR^E$ is a real subspace spanned by
its intersection with $\QQ^E$, and $\Sigma(M)+R$ is the Minkowski sum of the
$r(E)$-dimensional polyhedral fan $\Sigma(M)$ and the linear
space $R$.
The formula in Theorem~\ref{thm:Main} was perceived based on the conjecture
that the minimum in \eqref{eq:DRYformula} can always be obtained by 
very special subspaces $R$. 

\begin{conjecture} \label{conj:BraidSubspaces}
Let $M$ be a loopless matroid on $E$ and let $\Sigma(M)
\subseteq \RR^E$ be the matroid fan of $M$. 
Then the minimum in \eqref{eq:DRYformula} is obtained 
by subspaces of the braid arrangement, that is, subspaces obtained
as intersections of hyperplanes of the form $x_i = x_j$, $i,j \in E$.
\end{conjecture}

To make the connection with Theorem~\ref{thm:Main}, consider the subspace 
$R$ for which $x_i = x_j$ whenever $i$ and $j$ lie in the same part of the partition 
$\{P_1,\ldots, P_k\}$. 
It is then easy to check that $\tilde{r}(P_1,\ldots, P_k) = 2\dim_\RR(\Sigma(M)+R)-\dim_\RR(R)$.
Hence the conjecture is in fact equivalent to the equality $r'(E) = \adim(M)$. 
As an intermediate step, one might formulate the following conjecture.

\begin{conjecture}
  For $S \subset E$, we set $r''(S) = \adim(M|S)$.
	Then $r''$ is the rank function of a matroid. 
\end{conjecture}

Of course, if the conjecture holds then $r' = r''$ 
and hence the question  is answered by Theorem~\ref{matroid}.


As explained, using \cite{Draisma18d}
it follows that  
Conjecture~\ref{conj:BraidSubspaces} for realizable matroids
is equivalent to Theorem~\ref{thm:Main} and hence holds true. 
However, we emphasize that we give a direct proof of 
Theorem~\ref{thm:Main} that does \emph{not} make
use of the main result in \cite{Draisma18d}. In particular, 
we obtain the realizable case of the conjecture
as a consequence of Theorem~\ref{thm:Main}, 
not the other way around. 
It is intriguing to search for a direct proof of 
the conjecture that works also in the non-realizable case. 
In any case, an important benefit of Theorem~\ref{thm:Main} over 
$\dim_\RR \cA(X) = \adim(M_V)$ is that it allows for efficient computation
of $\dim_\RR \cA(X)$.

\subsection{Amoeba dimensions and algebraic matroids}

Any irreducible variety $X \subseteq (\CC^*)^n$ defines an {\em algebraic}
matroid $M_X$ on $[n]$ by declaring $S \subseteq [n]$ to be independent
if the projection from $X$ to $(\CC^*)^S$ is dominant.

\begin{prop} \label{prop:AmoebaMatroid}
The amoeba $\cA(X)$ also defines a matroid $M_{\cA(X)}$, by declaring
$S \subseteq [n]$ to be independent if the projection from $\cA(X)$
to $\RR^S$ contains a nonempty Euclidean-open subset of $\RR^S$.
\end{prop}

The proof will show that $M_{\cA(X)}$ is, in fact, an {\em algebraic}
matroid. 

\begin{proof}
Let $x_j,y_j$ be the real and imaginary parts, respectively,
of the $j$-th complex coordinate on $(\CC^*)^n$. With this
choice, the Weil restriction $X_\RR$ of $X$ lives in $\Spec
\RR[x_1,\ldots,x_n,y_1,\ldots,y_n][1/\prod_j (x_j^2+y_j^2)]$. The
(complex) points of $X$ form precisely the set $X_\RR(\RR)$ of points
of $X_\RR$ with coordinates in $\RR$. Since $X$ is geometrically
irreducible and $\CC/\RR$ is a finite separable extension, $X_\RR$
is geometrically irreducible. Then so is the image closure $Y \subseteq
\Spec \RR[u_1,\ldots,u_n][1/u_1 \cdots u_n]$ of $X_\RR$ under the morphism
$N$ given by $u_j \mapsto x_j^2 + y_j^2$. It follows that
$Y$ (or its set of complex points in $(\CC^*)^n$) defines a
matroid $M_Y$ on $[n]$. We claim that independence in $M_Y$
agrees with the notion in the proposition. Indeed, suppose
that $S$ is independent in $M_Y$,
and let $\pi: Y \to \Spec \RR[u_i \mid i \in S][1/\prod_{i \in S} x_i]$
be the projection. This is then dominant, and hence its derivative
$d_p \pi$ is surjective for $p$ in an open and dense subset
of the smooth points of $Y$. Since the
real points of $X_\RR$ are dense in $X_\RR$, we may take
$p=N(q)$ for some $q \in X_\RR(\RR)$. In fact, we find more: 
there exists a smooth real
point $q \in X_\RR(\RR)$ such that $d_q (\pi \circ N)$ is surjective.
Unwinding definitions, one finds that the composition of $d_q \Log:T_q
X \to \RR^n$ and projection to $\RR^S$ is surjective, and hence the
projection of $\cA(X)$ contains an open neighbourhood of the image of
$\Log(q)$ in $\RR^S$. The converse implication follows in a similar fashion.
\end{proof}

\begin{prop}
If $X=V \cap(\CC^*)^n$ where $V \subseteq \CC^n$ is a linear space
not contained in any coordinate hyperplane, then the matroid $M_{\cA(X)}$
is the matroid from Theorem~\ref{matroid} with the rank function $r'$
constructed from $M:=M_X=M_V$.
\end{prop}

\begin{proof}
Let $S$ be a subset of $[n]$ and let $X_S$ be the projection of $X$
into $(\CC^*)^S$. Then $S$ is independent in $M_{\cA(X)}$ if and only if
$\dim \cA(X_S)=|S|$, which by Theorem~\ref{thm:Main} applied to $X_S$
is equivalent to the condition that $r'(S)=|S|$.
\end{proof}

We note that the proof above only uses Theorem~\ref{thm:Main} and
not Theorem~\ref{matroid}. Hence when $M=M_V$ is a matroid realisable
over $\CC$, 
Theorem~\ref{thm:Main} and the two propositions above {\em imply}
Theorem~\ref{matroid}. 

Given that, for $X$ arising from a linear space, $M_{\cA(X)}$ has such
a nice description in terms of $M_X$, one might wonder whether the same
holds for general irreducible varieties $X$. In particular, one could ask
whether the amoeba dimension of $X$ is also determined by $M_X$, for
instance via the formula from Theorem~\ref{thm:Main}. 

The answer, however, is no in general. For an extreme counterexample,
let $X$ be a $d$-dimensional subtorus of $(\CC^*)^n$ such that $M_X$
is the uniform matroid $U_{d,n}$ of rank $d$ on $[n]$---this is achieved
by choosing the subtorus $X$ whose Lie algebra, i.e., tangent space at
$(1,\ldots,1)$, is the $\CC$-span $\langle R \rangle_\CC$ of any rational
subspace $R$ of $\QQ^n$ that represents the matroid $U_{d,n}$. Since
$X$ is a subtorus, its amoeba is a linear space, namely, the real span
$\langle R \rangle_\RR$. Hence $\dim_\RR \cA(X)$ is $d$, which is the
minimum among all amoeba dimensions of $d$-dimensional varieties
(actually, this minimum is attained only for translates of subtori;
see \cite[Theorem 4.6]{Nisse22}).  On the
other hand, the tangent space $V$ to $X$ at any point $p \in X$ equals
$p \cdot \langle R \rangle_\CC$ (the Hadamard product) and hence also
represents the matroid $U_{d,n}$. Therefore $\dim \cA(Y)=\min\{n,2d-1\}$,
as is easily seen with the formula in Theorem~\ref{thm:Main}.

\subsection{Organisation of this paper}

In Section~\ref{sec:Proof1}, we prove Theorem~\ref{thm:Main}, in
Section~\ref{sec:AmoebaMatroid} we prove Theorem~\ref{matroid},
and in Section~\ref{sec:Algorithm} we derive the algorithm in
Theorem~\ref{thm:Algorithm}.

\subsection*{Acknowledgments}

This paper grew out of several sources: JR's talk on amoebas \cite{Rau20}
where the formula of Theorem~\ref{thm:Main} was first conjectured, SE's
Master's thesis \cite{Eggleston22} at the University of Bern under the
supervision of JD, work by CHY on a combinatorial analysis of
the Jacobian of $\Log$ at a general point of a linear space, and RP's
work on the matroid $M'$ of Theorems~\ref{matroid}
and~\ref{thm:Algorithm}.
CHY thanks Kris Shaw for suggesting this problem to him.
All authors thank Frank Sottile for discussions on an early version of
this work.

\section{Proof of Theorem~\ref{thm:Main}}
\label{sec:Proof1}

\subsection{Proof of the inequality $\leq$ in
\eqref{eq:Main}}
\label{ssec:ineq1}

As mentioned in Subsection~\ref{subsection:DRYformula}, 
the inequality $\leq$ in
\eqref{eq:Main} is a consequence of 
\cite{Draisma18d} via 
$\dim_\RR \cA(X) = \adim(M_V) \leq r'(E)$.
Nevertheless, we include a short proof which also 
serves as preparation for later arguments. 

Recall from \cite{Draisma18d} that if a closed subvariety $X$ of $(\CC^*)^n$
is stable under a subtorus $T$ of $(\CC^*)^n$, and if we set $Y:=X/T
\subseteq (\CC^*)^n/T \cong (\CC^*)^m$ with $m=n-\dim_\CC(T)$, then we have
a surjective map $\cA(X) \to \cA(Y)$ whose fibres are translates of
$\cA(T)$. It then follows that
\[ \dim_\RR \cA(X)=\dim_\RR \cA(Y) + \dim_\RR
\cA(T)=\dim_\RR \cA(Y) + \dim_\CC T. \]

\begin{lm} \label{lm:2dmin1}
Let $X=V \cap (\CC^*)^n$ where $V$ is a $d$-dimensional complex
subspace of $\CC^n$ not contained in any coordinate
hyperplane. Then $\dim_\RR \cA(X) \leq 2d-1$. 
\end{lm}

\begin{proof}
Since $V$ is closed under scalar multiplication, $X$ is stable under
the one-dimensional torus $T=\{(t,\ldots,t) \mid t \in \CC^*\}$.
Then $Y:=X/T$ has dimension $d-1$, and hence $\dim_\RR
\cA(Y) \leq \dim_\CC Y = 2(d-1)$. It therefore follows from the above that 
\[ \dim_\RR \cA(X)=\dim_\RR \cA(Y) + \dim_\CC T \leq
2(d-1)+1=2d-1, \]
as desired.
\end{proof}

\begin{prop}
In Theorem~\ref{thm:Main}, the inequality $\leq$ holds. 
\end{prop}

\begin{proof}
Let $[n]=P_1 \sqcup \ldots \sqcup P_k$ be a partition into nonempty
parts. Let $V_i$ be the image of $V$ under the projection $\CC^n \to
\CC^{P_i}$, set $d_i:=\dim_\CC V_i$, and let $X_i$ be the intersection
of $V_i$ with $(\CC^*)^{P_i}$. Since $V$ is contained in $\prod_i V_i$,
$X$ is contained in $X':=\prod_i X_i$, and hence also
$\cA(X) \subseteq \cA(X')$. Then we find
\[ \dim_\RR \cA(X) \leq \dim_\RR \cA(X') = \sum_i \dim_\RR
\cA(X_i) \leq \sum_i (2d_i-1) \]
where the equality follows from the fact that the amoeba is
multiplicative on products, and where the last inequality follows from
Lemma~\ref{lm:2dmin1} applied to each $V_i$. 
\end{proof}

\subsection{Proof of the inequality $\geq$ in
\eqref{eq:Main}}

To prove $\geq$ in Theorem~\ref{thm:Main}, we need to construct a
partition for which equality holds. We will do so in an inductive manner.

By definition, $e:=\dim_\RR \cA(X)$ equals the maximum, over all $p
\in X$, of the real rank of the real linear map $d_p \Log: T_p X=V \to
\RR^n$, and this maximum is attained in an open dense subset of $X$
(in the Euclidean topology).

In what follows, for vectors $v,w \in \CC^n$, we write $v \cdot w$ for
their Hadamard product $(v_1w_1,\ldots,v_nw_n)$; and if $w \in (\CC^*)^n$,
then we write $v/w$ for the Hadamard quotient
$(v_1/w_1,\ldots,v_n/w_n)$. We also write $\one \in \CC^n$
for the all-one vector $(1,\ldots,1)$. Furthermore, if $z$
is a complex number or vector of complex numbers, then we
write $\Re(z),\Im(z)$ for the real and imaginary
parts of parts of $z$, respectively. 

\begin{lm} \label{lm:dpLog}
For $p \in X$ and $v \in T_p X=V$ we have 
\[  (d_p \Log)(v)=\Re(v/p)=(\Re(v_1/p_1),\ldots,\Re(v_n/p_n)), \]
where $\Re(z)$ is the real part of the complex number $z$.
\end{lm}

\begin{proof}
Write $v=p\cdot w$ and decompose $w=x + iy$ with $x,y \in \RR^n$. 
For $\epsilon \in \RR$ tending to $0$ we have
\[ \Log(p+\epsilon v)=\Log(p(\one+\epsilon
w))=\Log(p)+\Log(\one+\epsilon (x+iy)) = \Log(p)+\epsilon x +
O(\epsilon^2). \]
This implies $(d_p \Log)(v)=x=\Re(v/p)$, as desired.
\end{proof}

Pick $p_0 \in X$ such that the linear map $a:=d_{p_0} \Log:V
\to \RR^n$ has the
maximal possible rank $e$. This means that $a$ is a point in the real
manifold $Y$ of real linear maps $V \to \RR^n$ of rank precisely $e$. We
will use the following description of the tangent space $T_a Y$:
\[ T_a Y=\{b:V \to \RR^n \text{ linear} \mid b\ker(a)
\subseteq \im(a) \}. \]
Now, for a Euclidean-open neighbourhood $U$ of $p_0$ in $V$,
the map 
\[ \phi:U \to Y, p \mapsto d_p \Log \]
is a map of smooth manifolds. 

\begin{lm} \label{lm:dpphi}
The derivative $d_{p_0} \phi$ is the map 
\[ T_{p_0} U=V \to T_{a} Y, u \mapsto (v \mapsto - \Re((v
\cdot u)/(p_0 \cdot p_0)) \]
where $a:=d_{p_0} \Log:V
\to \RR^n$.
\end{lm}

\begin{proof}
For $u \in V$ write $u=p_0 \cdot w$. 
Then, for $\epsilon$ real and tending to zero, and for $v
\in V$, by Lemma~\ref{lm:dpLog} we have 
\begin{align*} 
\phi(p_0+\epsilon u)(v)&=
(d_{p_0+\epsilon u} \Log)(v)=\Re(v/(p_0+\epsilon u))
=\Re(v/(p_0(\one+\epsilon w)))\\
&=\Re((v/p_0)\cdot(\one-\epsilon w)) + O(\epsilon^2) 
=\Re(v/p_0) - \epsilon \Re((v\cdot w)/p_0) + O(\epsilon^2). \end{align*}
The coefficient of $\epsilon$ is the expression in the
lemma. 
\end{proof}

We simplify the situation as follows: we replace $V$ by $V/p_0$, $X$ by
$X/p_0$, and $p_0$ by $\one$. This only translates the amoeba of $X$,
and it has no effect on the matroid $M_V$.  Consequently, both sides in
\eqref{eq:Main} are unaltered.
In this simplified setting, Lemma~\ref{lm:dpLog} says that
\[ a(v)=(d_\one \Log)(v)=\Re(v). \]
In particular, 
\begin{align}  \label{eq:dim}
e&=\dim_\RR \cA(X)=\dim_\RR V - \dim_\RR \ker(a)=\dim_\RR
V - \dim_\RR (V \cap (i\RR^n))\\ &= \dim_\RR V - \dim_\RR (V
\cap \RR^n). \notag
\end{align}
Furthermore, Lemma~\ref{lm:dpphi} says that $d_{p_0} \phi$ is the linear
map that sends $u$ to the linear map $b_u(v):=-\Re(v \cdot u)$. We now
come to the crucial point in the proof.

\begin{prop} \label{prop:Hadamard}
Under the standing assumption that $p_0=\one \in V$ is a point where
$d_p\Log$ has the maximal rank $e$, the real vector space $V+\RR^n
\subseteq \CC^n$ is closed under Hadamard multiplication with the real
vector space $V \cap \RR^n$.
\end{prop}

\begin{proof}
By the above, for each $u \in V$, the linear map $b_u:V \to \RR^n$
lies in $T_a Y$, i.e., $b_u$ maps $\ker(a)$ into $\im(a)$.  Since $a$
maps a vector to its real part, we have $\ker(a)=V \cap
(i\RR^n)$ and $\im(a)=\Re(V)$. Hence we find that $b_u(v)=-\Re(v
\cdot u)$ is in $\Re(V)$ for all $v \in V \cap (i
\RR^n)$. Then, for $v \in V \cap \RR^n$, we have $i v \in (V
\cap (i\RR^n))$ and hence the real part of $-iv \cdot u$ is in
$\Re(V)$; but this is also the imaginary part of $v \cdot
u$. Furthermore, since $\Re(V)$ is also the set of
imaginary parts of vectors in $V$ (recall that $V$ is a complex vector 
space, so $V$ is invariant under the multiplication by $i$), we
find that for all $u \in V$ and all $v \in V \cap \RR^n$,
the imaginary part of $v \cdot u$ equals that of a vector in
$V$, so that $v \cdot u \in V+\RR^n$. Thus 
\[ (V \cap \RR^n) \cdot V \subseteq V+\RR^n, \]
and since clearly also 
\[ (V \cap \RR^n) \cdot \RR^n \subseteq \RR^n \subseteq
V+\RR^n \]
the proposition follows. 
\end{proof}

\begin{prop} \label{prop:Split}
Assume that $e=\dim_\RR \cA(X)< 2d-1$. Then there exists a partition
$[n]=P_1 \sqcup P_2$ into two nonempty parts with the following
property. Let $V_i$ be the projection of $V$ in $\CC^{P_i}$
and set $X_i:=V_i \cap (\CC^*)^{P_i}$. Then
\[ \dim_\RR \cA(X)=\dim_\RR \cA(X_1) + \dim_\RR \cA(X_2). \]
\end{prop}

\begin{proof}
In this case, $\ker d_{\one} \Log=V \cap (i \RR^n)$ has real dimension at
least $2$. Hence so does $V \cap \RR^n$. Let $v \in V \cap \RR^n$
be a vector linearly independent from $\one$. After adding a suitable
multiple of $\one$, we may assume that all entries of $v$ are positive,
and after scaling we may assume that the maximal entry of $v$ equals
$1$. Let $P_1 \subseteq [n]$ be the positions where $v$ takes this
maximal value $1$, and let $P_2$ be the complement of $P_1$ in $[n]$. By
construction, $P_1$ and $P_2$ are both nonempty.

By Proposition~\ref{prop:Hadamard}, $V+\RR^n$ is preserved under Hadamard
multiplication with $v$. Iterating this multiplication and taking the
limit, we find that $V+\RR^n$ is preserved under setting the coordinates
labelled by $P_2$ to zero. 
Then $V+\RR^n$ is also preserved under
setting the coordinates labelled by $P_1$ to zero.  Hence we
have 
\[ V+\RR^n = (V_1 + \RR^{P_1}) \times (V_2 +
\RR^{P_2}),\]
where $V_1,V_2$ are defined in the proposition.

We assume that for $i=1,2$, the all-one vector $\one_i \in V_i \subseteq
\CC^{P_i}$ has the same property required of $\one$, namely,
that $\dim_\RR \cA(X_i)$ equals the rank of the linear map $d_{\one_i}
\Log:V_i \to \RR^{P_i}$. (This might not follow from the corresponding
property of $\one$, but it may be achieved by picking the original $p_0$
in a suitable dense subset of $X$ and then dividing by that $p_0$.)

Now we have 
\begin{align*} \dim_\RR \cA(X)
&= \dim_\RR V - \dim_\RR (V \cap \RR^n)
= \dim_\RR (V+\RR^n)- \dim_\RR \RR^n\\
&= \dim_\RR (V_1 + \RR^{P_1}) + \dim_\RR (V_2+ \RR^{P_2}) -
|P_1|-|P_2|\\
&= (\dim_\RR (V_1 + \RR^{P_1})-\dim_\RR \RR^{P_1})+
(\dim_\RR (V_2 + \RR^{P_2})-\dim_\RR \RR^{P_2})\\
&= (\dim_\RR V_1 - \dim_\RR (V_1 \cap \RR^{P_1}))
+ (\dim_\RR V_2 - \dim_\RR (V_2 \cap \RR^{P_2}))\\
&= \dim_\RR \cA(X_1) + \dim_\RR \cA(X_2),
\end{align*}
as desired. Here we use three times the dimension formula
for vector subspaces and \eqref{eq:dim}
for $V,V_1,$ and $V_2$.
\end{proof}

\begin{proof}[Proof of Theorem~\ref{thm:Main}.]
The inequality $\leq$ was proved in \S\ref{ssec:ineq1}. For $\geq$
we proceed by induction on $n$; we therefore assume that the
inequality holds for all strictly smaller values of $n$. 

Now if $\dim_\RR \cA(X)=2d-1$, where $d=\dim_\CC V$, then $\geq$ is
witnessed by the partition of $[n]$ into a single part $P_1$.  Otherwise,
by Proposition~\ref{prop:Split}, there is a partition $[n]=P_1 \sqcup P_2$
into two nonempty parts, such that
\[\dim_\RR \cA(X)=\dim_\RR \cA(X_1) + \dim_\RR \cA(X_2) \]
where $X_i:=V_i \cap (\CC^*)^n$ and $V_i$ is the projection
of $V$ onto $\CC^{P_i}$. Since $P_1$ and $P_2$ both have
cardinality strictly smaller than $n$, the induction
hypothesis applies: there exist partitions
$P_i=P_{i1} \sqcup \dots \sqcup P_{ik_i}$ for $i=1,2$, into
nonempty parts, with 
\[\dim_\RR \cA(X_i)=\sum_{j=1}^{k_i} (2 \rk_{V_i} P_{ij} -1
). \]
Then the partition $[n]=P_{11} \sqcup \dots \sqcup P_{1k_1}
\sqcup P_{21} \sqcup \dots \sqcup P_{2k_2}$ does the trick for
$X$.
\end{proof}

\section{The function $r'$ is a rank function} \label{sec:AmoebaMatroid}

In this section, we analyse the right-hand side of the amoeba dimension
formula in Theorem~\ref{thm:Main}, and show that it is the rank function
of another matroid (Theorem~\ref{matroid}). 

\subsection{Preliminaries}

Let $M$ be a loopless matroid on ground set $E$ with rank function
$r:2^E\rightarrow\mathbb{N}$. Recall that, for nonempty subsets
$S_1,\ldots, S_k\subseteq E$, we have defined 
$$ \tilde{r}(S_1,\ldots S_k):=\sum_{i=1}^k 2r(S_i)-1$$ and $r':2^{E}\rightarrow\mathbb{N}$ as
 \begin{equation}
 r'(F):=\min\left\{\tilde{r}(P_1,\ldots, P_k): \{P_1,\ldots,
 P_k\}\text{ is a partition of } F\text{ into nonempty parts}\right\}.\end{equation}
Let $\succeq$ be the partial order on multisets of subsets of $E$ such that $\mathcal{S}\succeq\mathcal{T}$ if and only if there is a sequence of multisets 
 $$\mathcal{S}=\mathcal{S}_1,\ldots,\mathcal{S}_k=\mathcal{T}$$
 so that each $\mathcal{S}_{i+1}$ arises from $\mathcal{S}_i$ by replacing some intersecting pair $S, S'$ of elements from $\mathcal{S}_i$ with the pair $S\cap S', S\cup S'$. Note that by restricting to intersecting pairs, we have $\emptyset\not \in \mathcal{S}\Rightarrow \emptyset\not \in \mathcal{T}$. 
 
 From the submodularity of $r$, it follows that 
 $$\tilde{r}(\mathcal{S}_i)-\tilde{r}(\mathcal{S}_{i+1})=(2r(S)-1)+ (2r(S')-1)- (2r(S\cup S')-1)-(2r(S\cap S')-1)\geq 0$$
 at each stage in such a sequence, so that 
 $$\mathcal{S}\succeq\mathcal{T}\Longrightarrow
 \tilde{r}(\mathcal{S})\geq \tilde{r}(\mathcal{T}).$$
 Moreover, if $c(\mathcal{S})_e$ is defined as the number of sets $S$ in the multiset $\mathcal{S}$ that contain a given $e\in E$ (counting multiple occurrences of $S$) then 
 $$\mathcal{S}\succeq\mathcal{T}\Longrightarrow c(\mathcal{S})_e= c(\mathcal{T})_e.$$
 For any multiset of subsets $\mathcal{S}\subseteq 2^E$, the {\em finest common coarsening} is 
 $$\fcc(\mathcal{S}):= \left\{ T\subseteq \bigcup
 \mathcal{S}: T\text{ is an inclusion-wise minimal nonempty
 set so that } \forall S \in \mathcal{S}: S\cap
 T=\emptyset\text{ or } S\subseteq T\right\}.$$
 We evidently have 
  $$\mathcal{S}\succeq\mathcal{T}\Longrightarrow\fcc(\mathcal{S})=
  \fcc(\mathcal{T}).$$
 Finally, for the number $n(\mathcal{S}):=\sum_{S\in \mathcal{S}} |S|^2$ we have 
 $$\mathcal{S}\succ\mathcal{T}\Longrightarrow n(\mathcal{S}) < n(\mathcal{T}).$$
 Since a multiset $\mathcal{S}$ with $k$ elements (counting multiplicites) has $n(\mathcal{S})\leq k |E|^2$ there are no infinite descending sequences in the partial order $\succeq$. Hence for any multiset $\mathcal{S}$, there is a multiset $\mathcal{T}\preceq \mathcal{S}$ so that 
 $$\mathcal{T}\succeq \mathcal{U}\Longrightarrow \mathcal{T}= \mathcal{U}.$$
Such a $\mathcal{T}$ is {\em cross-free}: there are no sets $T, T'\in \mathcal{T}$ that {\em cross} in the sense that $T\cap T', T\setminus T', T'\setminus T$ are all nonempty. Note that then, $\fcc(\mathcal{T})\subseteq \mathcal{T}$.
 
 For a partition $\mathcal{P}$ of $S$ into nonempty parts
 and a partition $\mathcal{P}'$ of $S'$ into nonempty parts, we write
 $$\mathcal{P}\vee \mathcal{P}':= \fcc(\mathcal{P}\cup \mathcal{P}')\text{ and }\mathcal{P}\wedge \mathcal{P}':=  \{P\cap P': P\in \mathcal{P}, P'\in \mathcal{P}'\}\setminus\{\emptyset\}$$
Then $\mathcal{P}\vee \mathcal{P}'$ is a partition of $S\cup
S'$ into nonempty parts, and $\mathcal{P}\wedge
\mathcal{P}'$ is a partition of $S\cap S'$ into nonempty parts.

\begin{lm} \label{rank}Let $\mathcal{P}, \mathcal{P}'$
partition $S, S'\subseteq E$ into nonempty parts. There is a
partition $\mathcal{Q}$ of $S\cap S'$ into nonempty parts so that $$\tilde{r}(\mathcal{P})+\tilde{r}(\mathcal{P}')\geq \tilde{r}(\mathcal{P}\vee \mathcal{P}')+ \tilde{r}(\mathcal{Q}),$$
 $\mathcal{Q}$ is a coarsening of $\mathcal{P}\wedge \mathcal{P}'$, and $\#\mathcal{P}+\#\mathcal{P}'=\#\mathcal{P}\vee\mathcal{P}'+\#\mathcal{Q}$.
\end{lm}

\proof Let $\mathcal{S}$ be the multiset that arises by taking the union of $\mathcal{P}$ and $\mathcal{P}'$. Let $\mathcal{T}$ be a cross-free multiset so that $\mathcal{S}\succeq \mathcal{T}$. We have 
$$\mathcal{P}\vee \mathcal{P}':= \fcc(\mathcal{P}\cup \mathcal{P}')=\fcc(\mathcal{S})=\fcc(\mathcal{T}),$$
Since $\mathcal{T}$ is cross-free, we have $\fcc(\mathcal{T})\subseteq \mathcal{T}$. Let $\mathcal{Q}$ be the multiset that arises from $\mathcal{T}$ by taking away $\fcc(\mathcal{T})$. 
Then $$c(\mathcal{Q})_e=c(\mathcal{T})_e-c(\fcc(\mathcal{T}))_e=c(\{F, F'\})_e-c(\{F\cup F'\})_e=\left\{\begin{array}{ll} 
1-1=0&\text{if }e\in S\cup S'\text{ and }e\not\in S\cap S'\\
1-0=1&\text{if }e\in S\cap S'\\
0-0=0&\text{if }e\not\in S\cup S'
\end{array}\right.$$
So $\mathcal{Q}$ is a partition of $S\cap S'$, and since each element of $\mathcal{T}$ arises by taking unions and intersections starting from $\mathcal{P}\cup\mathcal{P}'$, $\mathcal{Q}$ is a coarsening of $\mathcal{P}\wedge \mathcal{P}'$. Since $\mathcal{S}\succeq \mathcal{T}$, we have 
$$\tilde{r}(\mathcal{P})+\tilde{r}(\mathcal{P}')=\tilde{r}(\mathcal{S})\geq  \tilde{r}(\mathcal{T})= \tilde{r}(\mathcal{P}\vee \mathcal{P}')+ \tilde{r}(\mathcal{Q})$$
as required. Finally, $\#\mathcal{P}+\#\mathcal{P}'=\#\mathcal{S}=\#\mathcal{T}=\#(\mathcal{P}\vee \mathcal{P}')+ \#\mathcal{Q}.$
\endproof

\subsection{Proof of Theorem~\ref{matroid}}

\proof We show that $r'$
satisfies the matroid rank axioms:
 
\begin{enumerate}
\item 
$r'(S)\geq 0$ for all $S\subseteq E$: Let $\mathcal{P}$ be a partition of $S$ into nonempty parts so that $r'(S)=\tilde{r}(\mathcal{P})$. As $M$ is loopless, we have $r(P)\geq 1$ whenever $P\neq \emptyset$, and so
$$r'(S)=\tilde{r}(\mathcal{P})=\sum_{P\in \mathcal{P}} 2r(P)-1\geq 0,$$ 
as required.

\item 
$r'(S')\leq r'(S)$ whenever $S'\subseteq S\subseteq E$: Let $\mathcal{P}$ be a partition of $S$ into nonempty parts so that $r'(S)=\tilde{r}(\mathcal{P})$. Then $$r'(S')\leq r'(\mathcal{P}\wedge\{S'\})\leq \tilde{r}(\mathcal{P}) = r'(S)$$
as required. 

\item 
$r'(S)+r'(S')\geq r'(S\cup S')+r'(S\cap S')$ for all $S, S'\subseteq E$: Let $\mathcal{P}, \mathcal{P}'$ be partitions of $S, S'$ into nonempty parts so that $r'(S)=\tilde{r}(\mathcal{P}), r'(S')=\tilde{r}(\mathcal{P}')$. By Lemma \ref{rank}, there is a partition $\mathcal{Q}$ of $S\cap S'$ so that
$$r'(S)+r'(S')=\tilde{r}(\mathcal{P})+\tilde{r}(\mathcal{P}')\geq \tilde{r}(\mathcal{P}\vee \mathcal{P}')+ \tilde{r}(\mathcal{Q})\geq r'(S\cup S')+r'(S\cap S'),$$
as required.\qedhere
\end{enumerate}
\endproof

\subsection{Structure of $M'$}
We make a few observations on the structure of the matroid $M'$ in relation to $M$.

If $M$ and $N$ are matroids on a ground set $E$, then $M$ is a {\em quotient} of $N$ if each flat of $M$ is also a flat of $N$ (or equivalently, if $\Sigma(M) \subseteq \Sigma(N)$). 
\begin{lem} %\label{lem }
  The matroid $M$ is a quotient of $M'$.
\end{lem}

More generally, if $P_1, \dots, P_k$ is a partition attaining the minimum,
then $M|P_1 \oplus \dots \oplus M|P_k$ is a quotient of $M'$. 

\begin{proof}
  Let $F$ be a flat of $M$. Assume that $F$ is not closed in $M'$. Then
	there exists $e \in E \setminus F$
	such that $r'(F \cup e) = r'(F)$. Let $P_1, \dots, P_k$ 
	be a minimal partition for $F \cup e$. We may assume $e \in P_1$.
	Note that $\tilde{r}(P_1, \dots, P_k) = r'(F)$ implies that 
	$r(P_1) = r(P_1 \setminus e)$. In particular, $e$ is contained 
	in the closure of $P_1 \setminus e \subset F$, a contradiction
	to $F$ being closed. Hence the claim follows. 
\end{proof}
The {\em truncation} of a matroid $M$ of rank $r>0$ is the matroid $N$ on the same ground set with rank function $r_N(F):=\min\{r_M(F), r-1\}$.
\begin{lem} \label{lem:truncation}If $M$ has rank $r>1$ and $N$ is the truncation of $M$, then $$r'_N(F)=\min\{r'_M(F), 2r-3\}$$ for all $F\subseteq E$.
\end{lem}
\begin{proof}
	Clearly $r'_N(F)\leq r'_M(F)$. Suppose $r'_N(F)< r'_M(F)$. Let $	
	\mathcal{P}$ be an optimal partition of $F$ w.r.t. $N$. Then 
	$$\sum_{P\in \mathcal{P}} 2r_N(P)-1 = \tilde{r}_N(\mathcal{P})=
	r'_N(F)< r'_M(F)\leq  \tilde{r}_M(\mathcal{P})=\sum_{P\in \mathcal{P}} 2r_M(P)-1$$
	hence $r_M(P)> r_N(P):=\min\{r_M(P), r-1\}$ for some $P\in \mathcal{P}$. Then $r_M(P)=r$ and 
	$r_N(P)=r-1$, so that $$2r-3\geq \tilde{r}_N(\{F\})\geq r'_N(F)=\tilde{r}_N(\mathcal{P})\geq 2r-3.$$
	Then $r'_N(F)=2r-3$, as required.
\end{proof}

\begin{example}
Consider the matroid $M$ that arises from the disjoint union of $k$ copies of $U_{2,n}$ by truncating $t$ times. For $t=0$ the matroid $M'$ has rank $3k$ and the partition $\mathcal{P}$ of the ground set $E$ into the $k$ connected components of $M$ attains the optimal value $$\tilde{r}(\mathcal{P})=\sum_{P\in \mathcal{P}} 2r(P)-1 = 3k = r'(E).$$
Applying Lemma \ref{lem:truncation} with $F=E$ we find that for $t=1$ the rank of $M'$ equals  $\min\{3k, 4k-3\}$, and iterating we obtain that the rank of $M'$ equals $\min\{3k, 2(2k-t)-1\}= \min\{\tilde{r}(\mathcal{P}), \tilde{r}(\{E\})\}$ in general. For $0\leq t<(k-1)/2$, the partition $\mathcal{P}$ is optimal, and $\{E\}$ is not. Each truncation increases the connectivity of $M$ by 1 until the rank of $M$ drops below $t$. So $M$ illustrates that in a $t$-connected matroid, an optimal partition may necessarily have $\geq k$ parts. 

Note that as $U_{2,n}$ is linear over $\C$ and taking disjoint unions and truncation preserves linearity over $\C$, the matroid $M$ will be linear over $\C$.
\end{example}

\section{An algorithm for evaluating $r'$}
\label{sec:Algorithm}

We retain the notation of the previous section and continue to explore
the functions $r'$ and $\tilde{r}$ derived from a fixed rank function $r$
of a loopless matroid $M$ on a finite set $E$, aiming for an algorithm
to evaluate $r'(S)$ given any $S\subseteq E$.

\subsection{Optimal partitions form a lattice}

In what follows, we will call a partition $\mathcal{P}$
{\em optimal for $S$} if $\mathcal{P}$ is a partition of $S$
into nonempty parts and $r'(S)=\tilde{r}(\mathcal{P})$.

\begin{lm} \label{lattice}Let $\mathcal{P}, \mathcal{P}'$ be optimal partitions for resp. $F, F'\subseteq E$. 
If $$r'(S)+r'(S')=r'(S\cup S')+r'(S\cap S'),$$ then  $\mathcal{P}\vee \mathcal{P}'$ is optimal for $S\cup S'$ and $\mathcal{P}\wedge \mathcal{P}'$ is optimal for $S\cap S'$.
\end{lm}
\proof
By Lemma \ref{rank}, there exists a partition $\mathcal{Q}$ of $S\cap S'$ that is a coarsening of the partition $\mathcal{P}\wedge \mathcal{P}'$ so that 
$$r'(S)+r'(S')=\tilde{r}(\mathcal{P})+\tilde{r}(\mathcal{P}')\geq \tilde{r}(\mathcal{P}\vee \mathcal{P}')+ \tilde{r}(\mathcal{Q})\geq r'(S\cup S')+r'(S\cap S').$$
By our assumption that  $r'(S)+r'(S')=r'(S\cup S')+r'(S\cap S')$, we have $r'(S\cup S')=\tilde{r}(\mathcal{P}\vee \mathcal{P}')$ and $r'(S\cap S')=\tilde{r}(\mathcal{Q})$. 
Let $\mathcal{Q}^*$ be a coarsening of  $\mathcal{P}\wedge \mathcal{P}'$ so that $r'(S\cap S')=\tilde{r}(\mathcal{Q}^*)$, with $\#\mathcal{Q}^*$ as large as possible. By Lemma \ref{rank} there is a coarsening $\mathcal{Q}'$ of $\mathcal{P}\wedge \mathcal{Q}^*$ so that
$$\tilde{r}(\mathcal{P})+\tilde{r}(\mathcal{Q}^*)=\tilde{r}(\mathcal{P}\vee \mathcal{Q}^*)+ \tilde{r}(\mathcal{Q}')$$
and $\#\mathcal{P}+\#\mathcal{Q}^*=\#(\mathcal{P}\vee \mathcal{Q}^*)+ \#\mathcal{Q}'$. Then $\mathcal{Q}'$ is also a coarsening of  $\mathcal{P}\wedge \mathcal{P}'$ and $r'(S\cap S')=\tilde{r}(\mathcal{Q}')$, so that  $\#\mathcal{Q}'\leq \#\mathcal{Q}^*$ by our choice of $\mathcal{Q}^*$. It follows that $\#\mathcal{P}\geq \#(\mathcal{P}\vee \mathcal{Q}^*)$
and hence each element of $\mathcal{Q}^*$ is a subset of an element of $\mathcal{P}$. Similarly, each element of $\mathcal{Q}^*$ is the subset of an element of $\mathcal{P}'$. Then $\mathcal{Q}^*=\mathcal{P}\wedge \mathcal{P}'$. 
\endproof

\subsection{Coarsest optimal partition and submodular
minimization}

It follows from Lemma \ref{lattice} that for any set $S\subseteq E$ there is a unique coarsest optimal partition 
$$\mathcal{P}^*:=\bigvee_{\mathcal{P}\text{ optimal for }S}\mathcal{P}$$ 
so that each optimal partition for $S$ refines $\mathcal{P}^*$. Similarly, there is a unique finest optimal partition 
$$\mathcal{P}_*:=\bigwedge_{\mathcal{P}\text{ optimal for }S}\mathcal{P}$$
which refines each optimal partition for $S$.


We will describe an algorithm to calculate the coarsest optimal partition for any given $S\subseteq E$. 
The coarsest optimal partition $\mathcal{P}$ for $S$ equals
\begin{equation}\label{parts}\mathcal{Q}:=\{Q\subseteq S: Q\text{ is inclusion-wise maximal so that } r'(Q)=2r(Q)-1\}\end{equation}
To see this, note that because $\mathcal{P}$ is optimal, each partition element $P\in \mathcal{P}$ has $r'(P)=2r(P)-1$ and hence is contained in one of the elements $Q\in\mathcal{Q}$.  The inclusion $P\subseteq Q$ cannot be strict, for then $\mathcal{P}\vee\{Q\}$ would be a coarser optimal partition for $S$ than $\mathcal{P}$. Hence $\mathcal{P}=\mathcal{Q}$.

Because \eqref{parts} refers to $r'(Q)$ it is not so useful for finding the coarsest partition for a general subset $S\subseteq E$. But if $|S|=r'(S)$, then each $Q\subseteq S$ has $r'(Q)=|Q|$, so that we can identify the part $Q$ of the coarsest optimal partition containing a given $e\in S$ as the largest $Q\subseteq S$ such that $e\in Q$ and $2r(Q)-1=|Q|$. Finding such $Q$ can be cast as a submodular function minimization problem.
\begin{lm}\label{test}Let $B\subseteq E$ and $e\in E\setminus B$ be such that $r'(B+e)=|B+e|$. Then the function $f:2^B\rightarrow \mathbb{Q}$ determined by
\begin{equation}\label{cunningham}f(I):=2r(I+e)-1 - |I+e|-\frac{|I|}{2|B|}\end{equation}
is submodular, and  $J$ is a largest subset of $B$ so that $2r(J+e)-1=|J+e|$ if and only $f(J)=\min_I f(I)$.
\end{lm}
\proof Submodularity of $f$ follows the submodularity of the rank function $r$. For each $I\subseteq B$ we have $|I+e|=r'(I+e)\leq \tilde{r}(I+e)=2r(I+e)-1$, hence
$$ f(I)\leq 0 ~ \Leftrightarrow ~ 2r(I+e)-1\leq |I+e| +\frac{|I|}{2|B|} ~ \Leftrightarrow ~ 2r(I+e)-1= |I+e| ~\Leftrightarrow ~f(I) = -\frac{|I|}{2|B|}. $$
Here, the second $\Leftrightarrow$ uses the fact that $0\leq\frac{|I|}{2|B|}\leq\frac{1}{2}$, while every other term is an integer.
Since $f(\emptyset)=0$, we have $0\geq \min_I f(I)$. The lemma follows.\endproof

That a submodular function $f$ obtained from a matroid rank function as in \eqref{cunningham} can be mimimized in polynomial time was first established by Cunningham in \cite{Cunningham1984}. The weakly polynomial time algorithm of Lee, Sidford, and Wang \cite{Lee2015} for submodular set function minimization takes
 $O(k^2\log(k)\cdot\gamma+k^3\log^{O(1)}(k))$ time to mimimize $f$, where $k:=|B|$ and $\gamma$ is the time needed to evaluate $f$.

\subsection{Proof of Theorem~\ref{thm:Algorithm}}

To make sure that an optimal partition for $S$ is coarsest, it will suffice to consider the intersection of that partition with a spanning subset $S'$ of $S$. 

\begin{lm} \label{extend} Suppose that $S'\subseteq S\subseteq E$
are such that $r'(S')=r'(S)$, and let $\mathcal{P}$ be a partition
of  $S$ into nonempty parts.
Then $\mathcal{P}$ is optimal for $S$ if and only if $r(P\cap S')=r(P)$ for all $P\in\mathcal{P}$ and $\mathcal{P}\wedge \{S'\}$ is optimal for $S'$. 
Moreover, $\mathcal{P}$ is the coarsest optimal partition for $S$ if and only if $r(P\cap S')=r(P)$ for all $P\in\mathcal{P}$  and $\mathcal{P}\wedge \{S'\}$ is the coarsest optimal partition for $S'$.
\end{lm}


\proof We have 
$$r'(S)=r'(S')\leq\tilde{r}(\mathcal{P}\wedge \{S'\})=\sum_{P\in \mathcal{P}, P\cap S'\neq \emptyset} 2r(P\cap S')-1\leq \sum_{P\in \mathcal{P}} 2r(P)-1= \tilde{r}(\mathcal{P})$$
so that $\mathcal{P}$ is optimal for $S$ if and only if $r(P\cap S')=r(P)$ for all $P\in \mathcal{P}$ and $\mathcal{Q}=\mathcal{P}\wedge \{S'\}$ is optimal for $S'$. 
In particular, if $\mathcal{P}$ is an optimal partition for $S$, then $\#\mathcal{Q}=\#\mathcal{P}$. 

 If $\mathcal{P}^*$  is a coarser optimal partition than $\mathcal{P}$, then $\mathcal{Q}^*:=\mathcal{P}^*\wedge\{S'\}$ is an optimal partition for $S'$ that is coarser than $\mathcal{Q}$, since  $\#\mathcal{Q}^*=\#\mathcal{P}^*<\#\mathcal{P}=\#\mathcal{Q}$. 
 Conversely, if $\mathcal{Q}^*$ is a coarser optimal partition than $\mathcal{Q}$, then $\mathcal{Q}^*\vee \mathcal{P}$ is coarser than $\mathcal{P}$ since $\#(\mathcal{Q}^*\vee \mathcal{P})\leq \#\mathcal{Q}^*<\#\mathcal{Q}=\#\mathcal{P}$.
\endproof


\begin{algorithm}
\caption{\label{alg:matroid} Coarsest optimal partition $\mathcal{P}$ and basis $B$ for $S\subseteq E$}
%\Input{A matroid $M$ on $E$ with rank function $r$}
%\Output{A partition $\mathcal{P}$ of $E$ so that $r'(E)=\tilde{r}(\mathcal{P})$}
\begin{algorithmic}
\IF{$S=\emptyset$}
\STATE Put $\mathcal{P}= \emptyset$, $B= \emptyset$ and return $\mathcal{P}$, $B$
\ELSE 
\STATE Pick $e\in S$
\STATE Compute the coarsest optimal partition $\mathcal{P}'$ and a basis $B'$ for $S-e$
\IF{$r(P'+e)=r(P')$ for a $P'\in \mathcal{P}'$}
\STATE Put $\mathcal{P}=\mathcal{P}'\vee\{P'+e\}, B=B'$
\STATE return $\mathcal{P}$, $B$
\ELSE
\STATE Pick a largest set $J\subseteq B'$ such that $2r(J+e)-1=|J+e|$
\STATE Put $\mathcal{P}= \mathcal{P}'\vee\{J+e\}$, $B= B'+e$
\STATE return $\mathcal{P}$, $B$
\ENDIF 
\ENDIF
 \end{algorithmic}
\end{algorithm}

\begin{thm} Given a set $S\subseteq E$, Algorithm \ref{alg:matroid} determines the coarsest optimal partition 
$\mathcal{P}$ for  $S$ and a subset $B\subseteq S$ such that $$|B|=r'(B)=r'(S)=\tilde{r}(\mathcal{P})$$ Moreover, the algorithm runs in polynomial time, taking up to $O(nk+ k^3\log(k))$ rank evaluations in $M$, where $n:=|S|, k:=r'(S)$.
\end{thm}
\proof We first argue that the output of the algorithm is correct, using induction on $|S|$. 
The case that $S=\emptyset$ is trivial, so assume that $S\neq \emptyset$, and let $e\in S$. By induction, the algorithm correctly computes the coarsest optimal partition $\mathcal{P}'$ and a subset $B'\subseteq S-e$ such that $|B'|=r'(B)=r'(S-e)=\tilde{r}(\mathcal{P}')$ initially.

If $r'(S)=r'(S-e)$, then  Lemma \ref{extend} applied to $S$ and $S'=S-e$ implies that $\mathcal{P}$ is the coarsest optimal partition for $S$ if and only if $r(P)=r(P-e)$ for all $P\in \mathcal{P}$ and $\mathcal{P}\wedge\{S-e\}=\mathcal{P}'$. Then $\mathcal{P}=\mathcal{P}'\vee\{P'+e\}$ for a $P'\in \mathcal{P}'$ such that $r(P'+e)=r(P')$. Moreover, for $B=B'$ we have $|B|=r'(B)=r'(S-e)=r'(S)$ as required. The algorithm proceeds correctly in this case.

In the remaining case $r'(S)=r'(S-e)+1$. Then for $B=B'+e$, we have
$$r'(B)\geq r'(S)+r'(B')-r'(S-e)=r'(B')+1=|B'|+1=|B|,$$
so that $r'(B)=r'(S)=|B|$. Let  $J\subseteq B'$ be a largest set so that $2r(J+e)-1=|J+e|$. Since 
$$|J+e|=2r(J+e)-1= \tilde{r}(\{J+e\})\geq r'(J+e)=|J+e|$$
 the partition $\{J+e\}$ is optimal for $J+e$, and hence $\mathcal{P}= \mathcal{P}'\vee\{J+e\}$ is optimal for $S$ by Lemma \ref{lattice}. 
If $\mathcal{P}^*$ is an optimal partition for $S$ which is coarser than $\mathcal{P}$, then $\mathcal{Q}^*:=\mathcal{P}^*\wedge\{B\}$ is a coarser optimal partition for $B$ than $\mathcal{Q}:=\mathcal{P}\wedge\{B\}$ by Lemma \ref{extend} applied to $S$ and $S'=B$. Then there is a set $Q\in \mathcal{Q}^*\setminus\mathcal{Q}$. If $e\not \in Q$, then $\mathcal{P}'\vee \{Q\}$ is coarser than $\mathcal{P}'$, a contradiction. So $e\in Q$, hence $J+e$ is properly contained in $Q$. But then $J':=Q-e$ contradicts the maximality of $J$. 
So there is no optimal partition for $S$ that is coarser than $\mathcal{P}$, and the algorithm proceeds correctly in this case as well.

It remains to show that the algorithm takes polynomial time, taking $O(nk+ k^3\log(k))$ rank evaluations. We only count the number of rank evaluations in $M$, the remaining work clearly being less significant by comparison. Observe that for the output $\mathcal{P}, B$ of the algorithm for $S$, we have $\#\mathcal{P}\leq |B|= r'(S)=k$. In the case that $r'(S)=r'(S-e)$, $O(k)$ rank evaluations in $M$ are required to test if $r(P+e)=r(P)$ for a $P\in \mathcal{P}'$. In the case that $r'(S)=r'(S-e)$, one may use submodular function minimization as in Lemma \ref{test}, taking $O(k^2\log(k))$ evaluations of the submodular function $f$, each evaluation of $f$ taking one rank evaluation in $M$. 
The former occurs for the $n-k$ elements of $S\setminus B$, the latter for the $k$ elements of $B$.  \endproof

\bibliographystyle{alpha}
\bibliography{math}

\end{document}

