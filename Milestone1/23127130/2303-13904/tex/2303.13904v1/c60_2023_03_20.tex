\documentclass[%
% aip,
aps,
prb,
% bmf,
% sd,
% rsi,
 amsmath,amssymb,
 superscriptaddress,
%groupedaddress,
preprint,%
% reprint,%
%author-year,%
%author-numerical,%
longbibliography
]{revtex4-1}

\usepackage{graphicx}% Include figure files
\usepackage{dcolumn}% Align table columns on decimal point
\usepackage{bm}% bold math
%\usepackage{hyperref}% add hypertext capabilities
% \usepackage[mathlines]{lineno}% Enable numbering of text and display math
% \linenumbers\relax % Commence numbering lines
\usepackage{textcomp}
\usepackage{xcolor}
\usepackage{bm}% bold math
\usepackage[sectionbib]{bibunits} % Could be used to split the bibliography in parts
% \defaultbibliography{2022c60exciton} 

%\usepackage[showframe,%Uncomment any one of the following lines to test 
%%scale=0.7, marginratio={1:1, 2:3}, ignoreall,% default settings
%%text={7in,10in},centering,
%%margin=1.5in,
%%total={6.5in,8.75in}, top=1.2in, left=0.9in, includefoot,
%%height=10in,a5paper,hmargin={3cm,0.8in},
%]{geometry}

\newcommand{\IPI}{\affiliation{I. Physikalisches Institut, Georg-August-Universit\"at G\"ottingen, Friedrich-Hund-Platz 1, 37077 G\"ottingen, Germany}}
\newcommand{\INAM}{\affiliation{Institute for Numerical and Applied Mathematics, Georg-August-Universit\"at G\"ottingen, Lotzestrasse 16-18, 37083 G\"ottingen, Germany}}
\newcommand{\graz}{\affiliation{Institute of Physics, University of Graz, NAWI Graz, Universitätsplatz 5, 8010 Graz, Austria}}
\newcommand{\KL}{\affiliation{Department of Physics and Research Center OPTIMAS, University of Kaiserslautern,
Erwin-Schr\"odinger-Straße 46, 67663, Kaiserslautern, Germany}}
\newcommand{\ICASEC}{\affiliation{International Center for Advanced Studies of Energy Conversion (ICASEC), University of Göttingen, Göttingen, Germany}}
\newcommand{\CNRS}{\affiliation{Univ. Grenoble Alpes, CNRS, Inst NEEL, F-38042 Grenoble, France}}
\newcommand{\mainz}{\affiliation{Institute of Physics, Johannes Gutenberg-University Mainz, 55128 Mainz, Germany}}


\begin{document}
\newcommand{\csixty}{C\textsubscript{60}}

%\newcommand{\sone}{S\textsubscript{1}}
%\newcommand{\stwo}{S\textsubscript{2}}
%\newcommand{\ctone}{CT\textsubscript{1}}
%\newcommand{\cttwo}{CT\textsubscript{2}}

\newcommand{\sone}{S\textsubscript{1}}
\newcommand{\stwo}{S\textsubscript{4}}
\newcommand{\ctone}{S\textsubscript{2}}
\newcommand{\cttwo}{S\textsubscript{3}}


\newcommand{\ekin}{$E_\mathrm{kin}$}
\newcommand{\ppduration}{40~fs}
\newcommand{\invA}{\AA\textsuperscript{-1}}

\newcommand{\ve}[1]{\boldsymbol{#1}} % Uncomment for BOLD vectors.
\newcommand{\ud}{\mathrm{d}}
% \begin{bibunit}

% \preprint{APS/123-QED}

\title{%Ultrafast imaging of m
Multiorbital exciton formation in an organic semiconductor
}
%%%% Alternative titles:
% Ultrafast orbital tomography of exciton dynamics in an organic semiconductor
% Probing the hole contribution of excitons in an organic semiconductor by photoemission momentum microscopy
% Orbital tomography of exciton wavefunctions in a C$_{60}$ crystal
% Ultrafast photoemission orbital tomography of exciton dynamics in a C$_{60}$ crystal
% Photoemission exciton tomography: probing the electron-hole wavefunction in organic semiconductors
% Probing the electron-hole wavefunction of excitons in organic semiconductors on the femtosecond timescale
% Probing the exciton wavefunction in organic semiconductors on the femtosecond timescale
% Accessing the multiorbital nature of excitons in C60
% Disentangling the electron and hole contribution to excitons by photoemission orbital tomography

\author{Wiebke Bennecke} %
\IPI

\author{Andreas Windischbacher}
\graz

\author{David Schmitt} %
\author{Jan Philipp Bange} %
\IPI
\author{Ralf Hemm} %
\KL
\author{Christian~S. Kern}
\graz

\author{Gabriele D'Avino}
\author{Xavier Blase}
\CNRS

\author{Daniel Steil} %
\author{Sabine Steil} 
\IPI

\author{Martin~Aeschlimann}
\KL

\author{Benjamin Stadtm\"uller} 
\KL
\mainz

\author{Marcel Reutzel}%
\IPI

\author{Peter Puschnig}
\graz

\author{G.~S.~Matthijs~Jansen} \email{gsmjansen@uni-goettingen.de} %
\IPI

\author{Stefan Mathias}

\email{smathias@uni-goettingen.de}%
\IPI
\ICASEC

\date{\today}% It is always \today, today,
             %  but any date may be explicitly specified


\begin{abstract} 
%150 words for Nature Chemistry
Harnessing the optoelectronic response of organic semiconductors requires a thorough understanding of the fundamental light-matter interaction that is dominated by the excitation of correlated electron-hole pairs, i.e. excitons. The nature of these excitons would be fully captured by knowing the quantum-mechanical wavefunction, which, however, is difficult to access both theoretically and experimentally. Here, we use femtosecond photoemission orbital tomography in combination with many-body perturbation theory to gain access to exciton wavefunctions in organic semiconductors. 
We find that the coherent sum of multiple electron-hole pair contributions that typically make up a single exciton can be experimentally evidenced by photoelectron spectroscopy. For the prototypical organic semiconductor buckminsterfullerene (\csixty{}), we show how to disentangle such multiorbital contributions and thereby access key properties of the exciton wavefunctions including localization, charge-transfer character, and ultrafast exciton formation and relaxation dynamics.
\end{abstract}

%\keywords{Suggested keywords}%Use showkeys class option if keyword
                              %display desired
\maketitle

\section{MAIN} % Introduction without header for NatChem: https://www.nature.com/nchem/content
Excitons, quasiparticles consisting of bound electron-hole pairs, are at the heart of the optoelectronic response of all organic semiconductors, and exciton formation and relaxation processes are largely responsible for energy conversion and light harvesting applications in these materials. At the atomic level, excitons are described by a two-particle correlated quantum-mechanical wavefunction that includes both the excited electron and the remaining hole. This wavefunction covers the complete shape of the exciton wave and thus provides access to a number of critical exciton properties such as the orbital character, the degree of (de)localization, the degree of charge separation, and whether this involves charge transfer between molecules. Consequently, in order to fully understand exciton dynamics and to exploit them in, e.g., an organic solar cell, an accurate and complete measurement of the exciton wavefunction would be ideal. Exemplary in this situation is the ongoing work to understand the optoelectronic response of \csixty{}, a prototypical organic semiconductor that is commonly used in organic solar cells \cite{ke_efficient_2015, puente_santiago_tailoring_2020, yu_simplified_2020}. Here, a topic of research has been the optical absorption feature that occurs at 2.8~eV for multilayer and other aggregated structures of \csixty{} \cite{wang_aggregates_1993}. Interestingly, time- and angle-resolved photoelectron spectroscopy and optical absorption spectroscopy studies have indirectly found that this optical transition corresponds to the formation of charge-transfer excitons with significant electron-hole separation \cite{stadtmuller_strong_2019, emmerich_ultrafast_2020, hess_electroabsorption_1996, causa_femtosecond_2018, hahn_role_2016}. Although these hints are supported by time-dependent density functional theory calculations that show the importance of delocalized excitations in \csixty{} clusters \cite{mumthazmuhammed_impact_2022, habeebmokkath_delocalized_2021, kobayashi_wannier-like_2020}, quantitative measurements of the exciton localization and charge separation have so far not been possible. Thus, the \csixty{} case highlights the need for a more direct experimental access to the wavefunctions of the electron-hole pair excitations.

From an experimental point of view, our method of choice to access exciton wavefunctions is time-resolved photoemission orbital tomography (tr-POT, see Methods for the experimental realization used in this work)\cite{puschnig_reconstruction_2009, jansen_efficient_2020, wallauer_tracing_2021, neef_orbital-resolved_2022}. In POT, the comparison with density functional theory calculations (DFT) provides a direct connection between photoemission data and the orbitals of the electrons.\cite{puschnig_reconstruction_2009} The extension to the time-domain promises valuable access also to the spatial information of excited electrons. However, at least for organic semiconductors, it has not been explicitly considered that photoemission of excitons requires the break-up of the two-particle electron-hole pair and that only the photoemitted electron, but not the hole, is directly detected. In fact, it is not clear to what extent (tr-)POT can be reasonably used for the interpretation and analysis of such strongly interacting correlated quasiparticles. Here we address this open question and show how tr-POT can probe the exciton wavefunction in the example system of a \csixty{} multilayer.

\begin{figure*}[tb!]
    \centering
    \includegraphics[width=.7\textwidth]{figures/figure1_v18.png}
    \caption{
    Schematic overview of time-resolved photoemission orbital tomography of exciton states in \csixty{}. \textbf{a, b}, a femtosecond optical pulse (blue pulse and blue arrow in (\textbf{a}) and (\textbf{b}), respectively) excites optically bright excitons in a \csixty{} film. The exciton electron-hole pairs are sketched in (\textbf{a}) as correlated particles (shaded blue-red areas) with a blue sphere for the hole and a red sphere for the electron. We probe the excitons in \csixty{} with extreme ultraviolet (EUV) pulses (purple pulse in (\textbf{a}), $h\nu =$~26.5~eV) that break up the electron-hole pairs and photoemit the corresponding electrons (red spheres), of which we detect the kinetic energies and the momentum emission patterns (yellow-green-colored disks in (\textbf{a})).
    The optical excitation of excitons in \csixty{} is known to lead to the formation of a decay sequence of singlet exciton states with varying charge-transfer character\cite{stadtmuller_strong_2019, emmerich_ultrafast_2020} (see (\textbf{b}), S\textsubscript{i}: i\textsuperscript{th} singlet excited state, S\textsubscript{0}: ground state). We are able to measure these exciton dynamics and the corresponding orbital tomography momentum patterns by adjusting the temporal delay between the optical excitation and the EUV probe pulses.
    }
    \label{fig:exp_excitons}
\end{figure*}


We employ our recently developed setup for photoelectron momentum microscopy\cite{medjanik_direct_2017, keunecke_time-resolved_2020, Keunecke20prb} and use ultrashort laser pulses to optically excite bright excitons in \csixty{} thin films that were deposited on Cu(111) (measurement temperature T $\approx$ 80~K; see Methods and Figure~1a). 
In the time-resolved photoemission experiment, we detect the energy and momentum emission pattern of the photoemitted electrons, which were initially part of the bound electron-hole pairs, i.e. the excitons. Following the time-evolution of the photoelectron spectrum, we can observe how the optically excited states relax to energetically lower-lying dark exciton states with different localization and charge-transfer character\cite{stadtmuller_strong_2019, emmerich_ultrafast_2020} (Figure~1b and data in Extended Fig.~\ref{fig:expdynamics}). 
In addition to the energy relaxation, we collect tr-POT data, and investigate in how far these patterns can be used to access real-space properties of the exciton wavefunctions. Specifically, we will address two questions: which orbitals contribute to the formation of the excitons and how this key information is imprinted in the energy- and momentum-resolved photoemission spectra.



\section{Results \& Discussion}
\subsection{The exciton spectrum of buckminsterfullerene \csixty{}}
To lay the foundation for our study, we first discuss the theoretical electronic properties of the \csixty{} film. On top of a hybrid-functional DFT ground state calculation, we obtain the exciton spectrum by employing the many-body framework of $GW$ and Bethe-Salpeter-Equation ($GW$+BSE) calculations (see Methods for full details). As shown in Fig.~\ref{fig:bse_excitons}a, we model the \csixty{} low-temperature phase by the two symmetry-inequivalent \csixty{} dimers 1-2 and 1-4, respectively\cite{wang_orientational_2001}, which are properly embedded to account for polarization effects in the film. The calculated single-particle energy levels are shown in Fig.~\ref{fig:bse_excitons}b, where we group the electron removal and electron addition energies into four bands, denoted as HOMO-1, HOMO, LUMO, and LUMO+1 according to the orbitals of the parent orbitals of an gas-phase \csixty{} molecule. These manifolds consist of 18, 10, 6, and 6 energy levels per dimer, respectively, originating from the  $g_g$+$h_g$, $h_u$, $t_{1u}$, and $t_{1g}$ irreducible representations of the gas phase \csixty{} orbitals \cite{dresselhaus_c60_1996}. We emphasize that the calculated $GW$ ionization levels of HOMO and HOMO-1 of 6.7~eV and 8.1~eV are in excellent agreement with experimental data for this \csixty{} film (see SI and Ref.~\onlinecite{haag_signatures_2020}).


\begin{figure*}[tbh!]
    \centering
    \includegraphics[width=\linewidth]{figures/Figure2_v12.png}
    \caption{\textit{Ab-initio} calculation of the electronic structure and exciton spectrum of \csixty{} dimers in a crystalline multilayer sample. \textbf{a}, the unit cell for a monolayer of \csixty{}, for which $GW$+BSE calculations for the dimers 1-2 and 1-4 were performed. \textbf{b}, electron addition/removal single-particle energies as retrieved from the self-consistent $GW$ calculation. These energies directly provide $\varepsilon_v$ in Eq.~\eqref{eq:trPOT}. \textbf{c}, results of the full $GW$+BSE calculation, showing as a function of the exciton energy $\Omega$ from bottom to top: the calculated optical absorption, the exciton band assignment \sone{} - \stwo{}, and the relative contributions to the exciton wavefunctions of different electron-hole pair excitations $\phi_v(\ve{r}_h) \chi_c(\ve{r}_e)$. Full details on the calculations are given in the Methods section. \textbf{d}, sketch of the composition of the exciton wavefunction of the \sone{} - \stwo{} bands and their expected photoemission signatures based on Eq.~\eqref{eq:trPOT}. In order to visualize the contributing orbitals, blue holes and red electrons are assigned to the single-particle states as shown in (\textbf{b}).}
    \label{fig:bse_excitons}
\end{figure*}

Building upon the $GW$ single-particle energies, we solve the Bethe-Salpeter equation and compute the energies $\Omega_m$ of all correlated electron-hole pairs (excitons). The resulting absorption spectrum (bottom panel of Fig. \ref{fig:bse_excitons}c) agrees well with literature\cite{wang_aggregates_1993}. In addition, we obtain the weights $X^{(m)}_{v c}$ on the specific electron-hole pairs that coherently contribute to the $m$\textsuperscript{th} exciton state, from which the exciton wavefunctions are constructed in the Tamm-Dancoff approximation as follows:
\begin{equation}
\psi_m(\ve{r}_h, \ve{r}_e) = \sum_{v,c} X^{(m)}_{vc} \phi_v^*(\ve{r}_h) \chi_c(\ve{r}_e).
\label{eq:exciton_state}
\end{equation}
This means that each exciton $\psi_m$ with energy $\Omega_m$ consists of a weighted coherent sum of multiple electron-hole-transitions $\phi_v(\ve{r}_h) \chi_c(\ve{r}_e)$ each containing one electron orbital $\chi_c$ and one hole orbital $\phi_v$.

To gain more insight into the character of the excitons $\psi_m$, we qualitatively classify them according to the most dominant orbital contributions that are involved in the transitions. This is visualized in the four sub-panels above the absorption spectrum in Fig.~\ref{fig:bse_excitons}c. For a given exciton energy $\Omega_m$, the black bars in each sub-panel show the partial contribution $|X_{v,c}^{(m)}|^2$ of characteristic electron-hole transitions $\phi_v(\ve{r}_h) \chi_c(\ve{r}_e)$ to a given exciton $\psi_m$. Looking at individual sub-panels, we see first that characteristic electron-hole transitions can belong to different excitons $\psi_m$ that have very different exciton energies $\Omega_m$. For example, the blue panel in Fig.~2c shows the contributions of HOMO$\rightarrow$LUMO (abbreviated H$\rightarrow$L) transitions as a function of exciton energy $\Omega_m$, and we see that these transitions contribute to excitons that are spread in energy over a scale of more than 1~eV (from $\Omega_m$ $\approx$ 1.7~eV to 3~eV). This spread of H$\rightarrow$L contributions (and also H$-$n$\rightarrow$L+m contributions) is caused by the fact that there are already many orbital energies per dimer (cf. Fig.~2b) which combine to form excitons with different degrees of localization and delocalization of the electrons and holes on one or more molecules.

We now focus on four exciton bands of the \csixty{} film, denoted as \sone{} - \stwo{}, which are centered around $\Omega_\sone{}$, $\Omega_\ctone{}$, $\Omega_\cttwo{}$ and $\Omega_\stwo{}$ at 1.9, 2.1, 2.8 and 3.6~eV, respectively (cf. Ref.~\onlinecite{emmerich_ultrafast_2020}). 
% We now focus on the four exciton bands of the \csixty{} film centered around $\Omega =$~1.8, 2.1, 2.8 and 3.6~eV, which we denote as \sone{} - \stwo{}, respectively (cf. Ref.~\onlinecite{emmerich_ultrafast_2020}). 
It is important to emphasize that each exciton band \sone{} - \stwo{} arises from many individual excitons $\psi_m$ with similar exciton energies $\Omega_m$ within the exciton band. Looking again at the sub-panels, we see that the \sone{} and \ctone{} exciton bands are made up of excitons $\psi_m$ that are almost exclusively composed of transitions from H$\rightarrow$L. On the other hand, the \cttwo{} shows in addition to H$\rightarrow$L also significant contributions from H$\rightarrow$L+1 transitions (pink-dashed panel). The \stwo{} exciton band can be characterized as arising from H$\rightarrow$L+1 (pink-dashed panel) and H$-$1$\rightarrow$L (orange-dash-dotted panel) as well as transitions from the HOMO to several higher lying orbitals denoted as H$\rightarrow$L+n (yellow-dotted panel). We emphasize that although orbitals from several different $GW$ energies contribute, e.g., to an exciton in the \stwo{} band, the exciton energy $\Omega_m$ of each exciton $\psi_m$ has a single well-defined value. 




\subsection{Photoemission signatures of multiorbital contributions}
In the following, we investigate whether these theoretically predicted multiorbital characteristics of the excitons can also be probed experimentally. As will be shown below, time-resolved photoemission spectroscopy can indeed provide access not only to the dark exciton landscape \cite{weinelt_dynamics_2004, dong_direct_2021, wallauer_momentum-resolved_2021, Madeo20sci, schmitt_formation_2022}, but also to the distinct orbital contributions of exciton states. A key step in extracting this information from the experimental data lies in a thorough comparison with simulations that specifically consider the pump-probe photoemission process, a topic that has recently attracted increased attention \cite{Popova2016, DeGiovannini2017, hammon_pump-probe_2021, Reuner2023}. Here, we rely on the formalism of Kern \textit{et al.}\cite{kern_exciton_tomog}, which is based on a common Fermi’s golden rule approach to photoemission \cite{Dauth2014}. Assuming the exciton of Eq.~\eqref{eq:exciton_state} as the initial state and applying the plane-wave final state approximation of POT, the photoemission intensity of the exciton $\psi_m$ is formulated as
\begin{equation}
% \begin{multline}
    % \frac{\ud \sigma^m}{\ud \Omega}\propto \left| \ve{A} \cdot \ve{k} \right|^2
    I_m(E_\mathrm{kin},\ve{k})  \propto  
    \left| \ve{A} \ve{k} \right|^2
    \sum_{v} \left| \sum_{c} X^{(m)}_{v c}  \mathcal{F}\left[\chi_{c} \right] (\ve k)\right|^2 %\nonumber \\
     \times   \delta\left(h\nu - E_\mathrm{kin} - \varepsilon_v + \Omega_m \right).
\label{eq:trPOT}
% \end{multline}
\end{equation}
Here $\ve{A}$ is the vector potential of the incident light field, $\mathcal{F}$ the Fourier transform, $\ve{k}$ the photoelectron momentum, $h\nu$ the probe photon energy, $\varepsilon_v$ the $v$\textsuperscript{th} ionization potential, $\Omega_m$ the exciton energy, and $E_{kin}$ the energy of the photoemitted electron. Note that $\varepsilon_v$ directly indicates the final-state energy of the left-behind hole. In the context of our present study, delving into Eq.~\eqref{eq:trPOT} leads to two striking consequences, which we discuss in the following. 

First, we illustrate the consequences of the multiorbital character of the exciton states on the photoelectron spectrum, and sketch in Fig.~2d the typical single-particle energy level diagrams for the HOMO and LUMO states and then indicate the contributing orbitals to the two-particle exciton state by blue holes and red electrons in these states, respectively. For the \sone{} exciton band (left panel), we already found that the main orbital contributions to the band are of H$\rightarrow$L character (Fig. 2d, left, and cf. Fig. 2c, blue panel). To determine the kinetic energy of the photoelectrons originating from the exciton, we have to consider the correlated nature of the electron-hole pair. The energy conservation expressed by the delta function in Eq.~2 (see also Ref.~\onlinecite{weinelt_dynamics_2004, Zhu14jpcl}) requires that the kinetic energy of the photoelectron depends on the ionization energy of the involved HOMO hole state $\varepsilon_{v}$ = $\varepsilon_{H}$ and the correlated electron-hole pair energy $\Omega \approx \Omega_{S_{1}}$. Therefore, we expect to measure a single photoelectron peak, as shown in the lower part of the left panel of Fig.~2d. In the case of the \ctone{} exciton the situation is similar, since the main orbital contributions are also of H$\rightarrow$L character. However, since the \ctone{} exciton band has a different energy $\Omega_{\ctone{}}$, the photoelectron peak is located at a different kinetic energy with respect to the S$_1$ peak.  

In the case of the \cttwo{} exciton band, we find that in contrast to the \sone{} and \ctone{} excitons not only H$\rightarrow$L, but also H$\rightarrow$L+1 transitions contribute (Fig. 2d, middle panel, and cf. Fig~2c, blue and pink-dashed panels, respectively). However, we still expect a single peak in the photoemission, because the same hole states are involved for both transitions (i.e., same $\varepsilon_v = \varepsilon_{H}$ in the sum in Eq.~2), and all orbital contributions have the same exciton energy $\approx\Omega_{\cttwo{}}$, even though transitions with electrons in energetically very different single-particle LUMO and LUMO+1 states contribute. With other words, and somewhat counter-intuitively, the single-particle energies of the electron orbitals (the LUMOs) contributing to the exciton do not enter the energy conservation term in Eq.~2, and thus do not affect the kinetic energy observed in the experiment. 

Finally, for the \stwo{} exciton band at $\Omega_\stwo{} =$~3.6~eV, we find three major contributions (Fig.~2d, right panel), where not only the electrons but also the holes are distributed over two energetically different levels, namely the HOMO (cf. pink-dashed and yellow-dotted panels in Fig.~2c,d) and the HOMO-1 (cf. orange-dash-dotted panels in Fig.~2c,d). Thus, there are two different final states available for the hole, each with a different binding energy.
Consequently, the photoemission spectrum of \stwo{} is expected to exhibit a double-peak structure with intensity appearing $\approx$~3.6~eV above the HOMO kinetic energy $E_{H}$, and $\approx$~3.6~eV above the HOMO-1 kinetic energy $E_{H-1}$, as illustrated in the right-most panel of Fig.~\ref{fig:bse_excitons}d. Relating this specifically to the single-particle picture of our $GW$ calculations, the two peaks are predicted to have a separation of $\varepsilon_{{H-1}}$ – $\varepsilon_{{H}}$ = 8.1 – 6.7 = 1.4~eV. 

In addition, Eq.~\eqref{eq:trPOT} now also provides the theoretical framework for interpreting momentum-resolved tr-POT data from excitons. Ground state POT can be easily understood in terms of the Fourier transform $\mathcal{F}$ of single-particle orbitals. A naive extension to excitons might imply an incoherent, weighted sum of all LUMO orbitals $\chi_c$ contributing to the exciton wavefunction. However, as Eq.~\eqref{eq:trPOT} shows, such a simple picture proves insufficient. Instead, the momentum pattern of the exciton wavefunction is related to a coherent superposition of the electron orbitals $\chi_c$ weighted by the electron-hole coupling coefficients $X^{(m)}_{v c}$. The implications of this finding are sketched in the $k_x$-$k_y$ plots in Fig.~\ref{fig:bse_excitons}d and are most obvious for the \cttwo{} band. Here, the exciton is composed of transitions with a common hole position, i.e., H$\rightarrow$L and H$\rightarrow$L+1, leading to a coherent superposition of all 12 electron orbitals from the LUMO and LUMO+1 in the momentum distribution. In summary, multiple hole contributions can be identified in a multi-peak structure in the photoemission spectrum, and multiple electron contributions will result in a coherent sum of the electron orbitals that can be identified in the corresponding energy-momentum patterns from tr-POT data.



\subsection{Disentangling multiorbital contributions experimentally}
These very strong predictions about multi-peaked photoemission spectra due to multiorbital excitons can be directly verified in an experiment on \csixty{} by comparing spectra for resonant excitation of either the \cttwo{} or the \stwo{} excitons (cf. Fig.~\ref{fig:bse_excitons}). The corresponding experimental data are shown in Fig.~\ref{fig:exp_S2}a and \ref{fig:exp_S2}c, respectively.  
Starting from the excitation of the \cttwo{} exciton band with $h\nu =$~2.9~eV photon energy (which is sufficiently resonant to excite the manifold of exciton states that make up the \cttwo{} band around $\Omega_{\cttwo{}} =$~2.8~eV), we can clearly identify the direct excitation (at 0~fs delay) of the exciton \cttwo{} feature at an energy of E~$\approx$~2.8~eV above the kinetic energy $E_H$ of the HOMO level. Shortly after the excitation, additional photoemission intensity builds up at $E-E_H \approx$~2.0~eV and $\approx$~1.7~eV, which is known to be caused by relaxation to the \ctone{} and \sone{} dark exciton states\cite{emmerich_ultrafast_2020} and is in good agreement with the theoretically predicted energies of $E-E_H$ $\approx$~2.1~eV and $\approx$1.9~eV.

\begin{figure*}[tbh!]
    \centering
    \includegraphics[width=1.\textwidth]{figures/figure3_v15.png}
    \caption{\textbf{a-d}, comparison of the time-resolved photoelectron spectra of multilayer \csixty{} for $h\nu =$~2.9~eV excitation and $h\nu =$~3.6~eV excitation ((\textbf{a}) and (\textbf{c}), respectively), both normalized and shifted in time to match the intensity of the \cttwo{} signals (see Methods for full details on the data analysis). As can be seen in the difference (\textbf{b}), for $h\nu =$~3.6~eV pump we observe an enhancement of the photoemission yield around $E-E_H \approx$~3.6~eV as well as around $E-E_H \approx$~2.2~eV. We attribute this signal to the \stwo{} exciton band, which has hole contributions stemming from both the HOMO and the HOMO-1. To further quantify the signal of the \stwo{} exciton, (\textbf{d}) shows energy distribution curves for both measurements at early delays, showing the enhancement in the $h\nu =$~3.6~eV measurement. }
    \label{fig:exp_S2}
\end{figure*}

Changing now the pump photon energy to $h\nu =$~3.6~eV for direct excitation of the \stwo{} exciton band (Fig.~\ref{fig:exp_S2}c), two distinct peaks at $\approx$ 3.6~eV above the HOMO and the HOMO-1 are expected from theory. While photoemission intensity at $E-E_H \approx$~3.6~eV above the HOMO level is readily visible in Fig.~\ref{fig:exp_S2}c, the second feature at 3.6~eV above the \mbox{HOMO-1} is expected at $E-E_H \approx$~2.2~eV above the HOMO level (corresponding to $E-E_{H-1}$ $\approx$~3.6~eV) and thus almost degenerate with the aforementioned \ctone{} dark exciton band at about $E-E_H \approx$~2.0~eV, which appears after the optical excitation due to relaxation processes. Therefore, we need to pinpoint this second H$-$1$\rightarrow$L contribution to the \stwo{} exciton at the earliest time of the excitation. Indeed, a closer look around 0~fs delay shows additional photoemission intensity at about $E-E_H \approx$~2.2~eV. 
Using difference maps (Fig.~\ref{fig:exp_S2}b) and direct comparisons of energy-distribution-curves at selected time-steps (Fig.~\ref{fig:exp_S2}d), we clearly find a double-peak structure corresponding to the energy difference of $\approx$1.4~eV of the HOMO and \mbox{HOMO-1} levels. Thereby, we have shown that photoelectron spectroscopy, in contrast to other techniques (e.g., absorption spectroscopy), is indeed able to disentangle different orbital contributions of the excitons. In this way, we have validated the theoretically predicted multi-peak structure of the multiorbital exciton state that is implied by Eq.~\eqref{eq:trPOT}. We also see that the photoelectron energies in the spectrum turn out to be sensitive probes of the corresponding hole contributions of the correlated exciton states.

We note that the signature of the \cttwo{} excitons, even if not directly excited with the light pulse in this measurement, is still visible and moreover with significantly higher intensity than the multiorbital signals of the resonantly excited \stwo{} exciton band. This observation strongly suggests that there is a very fast relaxation from the \stwo{} exciton to the \cttwo{} exciton, with relaxation times well below 50~fs (see Extended Fig.~\ref{fig:expdynamics}). 


\subsection{Time-resolved photoemission orbital tomography of exciton wavefunctions}
Based on the excellent agreement between the experiment and the $GW$+BSE theoretical results, we are now ready to investigate to what extent the momentum patterns from tr-POT data of excitons in organic semiconductors contain information about the real-space spatial distribution of the exciton wavefunction. In the experiment, we once again excite the \cttwo{} exciton band in the \csixty{} film with $h\nu =$~2.9~eV pump energy, and we now use femtosecond tr-POT to collect the momentum fingerprints of the directly excited \cttwo{} excitons around 0~fs and the subsequently built-up dark \ctone{} and \sone{} excitons that appear in the exciton relaxation cascade in the \csixty{} film (see Fig.~\ref{fig:comparison}a-c, where the momentum map of the lowest energy \sone{} exciton band is plotted in (a), the \ctone{} in (b) and the highest energy \cttwo{} exciton band in (c); see Extended Fig.~\ref{fig:expdynamics} for time-resolved traces of the exciton formation and relaxation dynamics). We note that the collection of these data required integration times of up to 70 hours, and that a measurement of the comparatively low-intensity \stwo{} feature when excited with $h\nu =$~3.6~eV has not yet proved feasible. For the interpretation of the collected POT momentum maps from the \sone{}, \ctone{}, and \cttwo{} excitons, we also calculate the expected momentum fingerprints for the wavefunctions obtained from the $GW$+BSE calculation for both dimers, each rotated to all occurring orientations in the crystal. Finally, for the theoretical momentum maps, we sum up the photoelectron intensities of each electron-hole transition in an energy range of 200~meV centered on the exciton band. The results are shown in Fig.~\ref{fig:comparison}d-f below the experimental data for direct comparison.

\begin{figure*}[htb!]
    \centering
    \includegraphics[width=.85\linewidth]{figures/Figure4_v10.png}
    \caption{\textbf{a-f}, comparison of the (\textbf{a-c}) experimental momentum maps acquired for the three exciton bands observed in \csixty{} with the (\textbf{d-f}) predicted momentum maps retrieved from $GW$+BSE.
    Note that the center of the experimental maps could not be analyzed due to a space-charge-induced background signal in this region (gray area, see Methods).
    \textbf{g-i}, isosurfaces of the integrated electron probability density (yellow) within the 1-2 dimer for fixed hole positions on the bottom-left molecule (blue circle) of the dimer for the (\textbf{g}) \sone{}, the (\textbf{h}) \ctone{}, and the (\textbf{i}) \cttwo{} exciton bands. }
    \label{fig:comparison}
\end{figure*}

First, we observe that the experimental momentum maps of the \sone{} and \ctone{} states are largely similar (Fig.~\ref{fig:comparison}a,b), showing six lobes centered at $k_\parallel \approx$~1.2~\invA{}. These features, as well as the energy splitting between \sone{} and \ctone{} (cf. Fig.~\ref{fig:bse_excitons}c), are accurately reproduced by the $GW$+BSE prediction (Fig.~\ref{fig:comparison}d,e). Furthermore, also the $GW$+BSE calculation shows very similar momentum maps for \sone{} and \ctone{}, suggesting a similar spatial structure of the excitons. This is in contrast to a naive application of static POT to the unoccupied orbitals of the DFT ground state of \csixty{}, which does show a similar momentum map for the LUMO, but cannot explain a kinetic energy difference in the photoemission signal, nor give any indication of differences in the corresponding exciton wavefunctions. 
With this agreement between experiment and theory, we now extract the spatial properties of the $GW$+BSE exciton wavefunctions. To visualize the degree of charge-transfer of these two-particle exciton wavefunctions $\psi_m(\ve{r_h},\ve{r_e})$, we integrate the electron probability density over all possible hole positions $\ve{r_h}$, considering only hole positions at one of the \csixty{} molecules in the dimer. This effectively fixes the hole contribution to a particular \csixty{} molecule (blue circles in Fig.~\ref{fig:comparison}g-i indicate the boundary of considered hole positions around one molecule, hole distribution not shown), and provides a probability density for the electronic part of the exciton wavefunction in the dimer, which we visualize by a yellow isosurface (see Fig.~\ref{fig:comparison}g-i). 
Obviously, in the case of \sone{} and \ctone{} (Fig.~\ref{fig:comparison}g,h), when the hole position is restricted to one molecule of the dimer, the electronic part of the exciton wavefunction is localized at the same molecule of the dimer. Our calculations thus suggest that the \sone{} and \ctone{} excitons are of Frenkel-like nature. Their energy difference originates from different excitation symmetries possible for the H$\rightarrow$L transition (namely $t_{1g}$, $t_{2g}$, and $g_g$ for the \sone{} and $h_g$ for the \ctone{}).\cite{kobayashi_wannier-like_2020} 
%More quantitatively, from the exciton wavefunction $\psi_m(\ve{r_h},\ve{r_e})$, we can directly determine the mean electron-hole separation. For the \sone{} and \ctone{} excitons, we calculate a mean electron-hole separation of only 0.05~\AA{} and 0.06~\AA, respectively, which obviously cannot be distinguished in Fig.~\ref{fig:comparison}g and \ref{fig:comparison}h and shows that \sone{} and \ctone{} are of co-localized Frenkel-like nature.

In contrast to the \sone{} and \ctone{} excitons, the momentum map of the \cttwo{} band shows a much more star-shaped POT fingerprint in both theory and experiment (Fig.~\ref{fig:comparison}c,f). This is to be expected, since the electronic part of the \cttwo{} excitons contains not only contributions from the LUMO orbital, but also contributions from the LUMO+1 orbital. Note, however, that we find the experimentally observed star-shaped pattern to be only partially reproduced by the $GW$+BSE calculation. An indication towards the cause of this discrepancy is found by considering the electron-hole separation of the excitons making up the \cttwo{} band. Here, we find that the positions of the electron and the hole contributions are strongly anticorrelated (Fig.~\ref{fig:comparison}i), with the electron confined to the neighboring molecule of the dimer. In fact, the mean electron-hole separation is as large as 7.6~\AA, which is close to the core-to-core distance of the \csixty{} molecules. Although these theoretical results confirm the previously-reported charge-transfer nature of the \cttwo{} excitons \cite{stadtmuller_strong_2019, emmerich_ultrafast_2020}, they also reflect the limitations of the \csixty{} dimer approach. Indeed, the dimer represents the minimal model to account for an intermolecular exciton delocalization effect, but it cannot fully account for dispersion effects\cite{haag_signatures_2020} (cf. Extended Fig. 5), which are required for a quantitative comparison with experimental data. Besides the discrepancy in the \cttwo{} momentum map, this could also be an explanation why the \ctone{} in the present work is of Frenkel-like nature, but could have charge-transfer character according to previous studies\cite{stadtmuller_strong_2019, emmerich_ultrafast_2020}.
However, future developments will certainly allow scaling up of the cluster size in the calculation, so that exciton wavefunctions with larger electron-hole separation can be accurately described.
Most importantly, we find that the present dimer $GW$+BSE calculations are clearly suited to elucidate the multiorbital character of the excitons, which is an indispensable prerequisite for the correct interpretation of tr-POT data of excitons in organic semiconductors. 

\section{CONCLUSION}
In conclusion, we have shown how the energy- and momentum-resolved photoemission spectrum of excitons in an organic semiconductor depends on the multiorbital nature of these excitons. By extending POT to fully-interacting exciton states calculated in the framework of the Bethe-Salpeter equation, we found that the energy of the photoemitted electron of the exciton quasiparticle is determined by the position of the hole and the exciton energy in combination with the probe photon energy. This leads to the prediction of multiple peaks in the photoelectron spectrum, which we verify experimentally, and allows disentangling the different orbital contributions, the wavefunction localization, and the charge-transfer character. Similarly, the momentum fingerprint provides access to the electron states that make up the exciton. Most importantly, we introduce time-resolved photoemission orbital tomography as a key technique for the study of exciton wavefunctions in organic semiconductors.%, which enables unprecedented access to the exciton's localization, charge-transfer character, and the ultrafast exciton formation and relaxation dynamics.

\section{Acknowledgements}
This work was funded by the Deutsche Forschungsgemeinschaft (DFG, German Research Foundation) - 432680300/SFB 1456, project B01 and 217133147/SFB 1073, projects B07 and B10. G.S.M.J. acknowledges financial support by the Alexander von Humboldt Foundation. 
A.W., C.S.K., and P.P acknowledge support from the Austrian Science Fund (FWF) project I 4145. The computational results presented were achieved using the Vienna Scientific Cluster (VSC) and the local high-performance resources of the University of Graz.
R.H., M.A., and B.S. acknowledge financial support by the DFG - 268565370/TRR 173, projects B05 and A02. B.S. acknowledges further support by the Dynamics and Topology Center funded by the State of Rhineland-Palatinate.

\section{Author Contributions}
D.St., M.R., S.S., M.A., B.S., P.P.,  G.S.M.J. and S.M. conceived the research. W.B., D.Sch. and J.P.B. carried out the time-resolved momentum microscopy experiments. W.B. analyzed the data. W.B. and R.H. prepared the samples. A.W., C.S.K., G.D.A, X.B. and P.P. performed the calculations and analyzed the theoretical results. All authors discussed the results. G.S.M.J. and S.M. were responsible for the overall project direction and wrote the manuscript with contributions from all co-authors. 

% \putbib
\bibliography{2022c60exciton} % Produces the bibliography via BibTeX.
% \end{bibunit}

% \begin{bibunit}
\newpage
\section{PoseRAC Model}
\label{sec4}

\begin{figure*}[t]
\centering
\includegraphics[width=1.0\textwidth]{figure5.pdf}
\caption{Overview of our proposed PoseRAC. For a input video, the repetitive count can be obtained through Pose Estimation, Transformer Encoder, Pose Mapping and Action-trigger, where only the Encoder and the Pose Mapping need to be trained. We use Triplet Margin Loss to train the Encoder while Binary Cross Entropy Loss to train both the Encoder and the Pose Mapping. In addition to achieving the state-of-the-art performance so far, the biggest highlight of our PoseRAC is that it is lightweight enough to be easily trained on a CPU.}
\label{fig5}
\end{figure*}

Given a video $V={\{x_i\}}^{T}_{1}\in \mathbb{R}^{C\times H\times W\times T}$ with $T$ RGB frames, repetitive action counting model aims to predict a certain value $Y$, which is the number of repetitive actions. In this section, we will introduce our PoseRAC in detail.

\subsection{Model Overview}

As shown in Figure \ref{fig5}, PoseRAC consists of four parts. 

\begin{itemize}

\item The first is a state-of-the-art and lightweight Pose Estimation Network~($\S\ref{first}$), which is used to estimate the poses represented by lots of human pose key points from each frame of the original video sequence. 

\item The second is a simple Transformer Encoder~($\S\ref{second}$) to embed the key points of poses into high-level feature space, where the same class have similar distances, while the distances of different classes are far apart.

\item The third is a Pose Mapping Module~($\S\ref{third}$), where the unique mapping relationship between the salient poses and the action classes can be learned. Each pose can be mapped to the action class with the highest probability after the previous encoding.

\item The fourth part is a lightweight Action-trigger Module~($\S\ref{fourth}$). When we get the salient action classification results of all frames of the entire video sequence, we can use this module to calculate the repetition count in a short time.

\end{itemize}

\subsection{Pose Estimation Network}
\label{first}
Our model first converts the video sequence into a sequence of human pose key points, which can be defined as: 
\begin{equation}
\begin{split}
&V={\{x_i\}}^{T}_{1}\in \mathbb{R}^{C\times H\times W\times T}\\
&V\xrightarrow{\mathrm{Pose Estimation}} P={\{p_i\}}^{T}_{1}\in \mathbb{R}^{D\times K\times T}
\end{split}
\label{eq1}
\end{equation}
where each $x_i$ represents a single RGB frame, and each $p_i$ represents the key points of each frame. To express the key points of each frame, we use $D\times K$ sequence, which includes two parts, one ($K$) is the number of key points to fully represent the current pose, the other ($D$) is the dimension of each key point, generally three, which are the two coordinates of the planes and the depth estimation.

Here we use state-of-the-art pose estimation models such as Vitpose\cite{xu2022vitpose} and BlazePose\cite{bazarevsky2020blazepose}. The pose estimation algorithms themselves are not designed by us, but we introduce pose information into the action counting task, which is a novel design not explored by previous work.

Moreover, our pose-level poses estimation processes the primitive information of video, which is similar to the feature extraction network in all video-level algorithms such as I3D\cite{carreira2017quo}, VideoSwinTransformer\cite{liu2022video}, and TSN\cite{wang2016temporal}. But the difference is that the result of video-level incorporates all information, while pose-level only produces core information, which greatly improves the performance. Additionally, using pose information can contribute to the lightweight of model. For instance, for a 1024-frame video, video-level feature extraction with an output dimension of 512 would produce a data volume of $1024\times 512=524288$, while using pose information with 33 key points produces a data volume of only $1024\times 33 \times 3=101376$.

\subsection{Encoding Poses with Transformer}
\label{second}
Here we specify our data representation for the Transformer Encoder, which requires input batch size, sequence length, and embedding dimensions. In our pose-level approach, each frame is a batch, the number of key points in each frame is the sequence length, and the feature dimension of each key point is the embedding dimension.

First we get the pose of each frame ${p_i}\in \mathbb{R}^{D\times K}$ through the Pose Estimation Network, where $i\in {1, 2, \dots, T}$ is the frame index, $K$ is the number of key points, and $D$ is the dimension of each key point. We further define $p_i = {\{k_j\}}^{K}_{1}$ to represent each key point, where $k_j\in \mathbb{R}^D$, and we embed it to obtain richer information. Our embedding projection $\mathrm{\bf{E}}$ is a simple MLP network with ReLU as the activation function. These calculations can be defined as:
\begin{equation}
\begin{split}
\mathrm{\bf{Z}}^0 = [\mathrm{\bf{E}}(k_1), \mathrm{\bf{E}}(k_2), \dots, \mathrm{\bf{E}}(k_K)]^T
\end{split}
\end{equation}
where $\mathrm{\bf{E}}(k_j)\in \mathbb{R}^{D^{\prime}}$ is the embedding feature. Then the next Transformer takes $\mathrm{\bf{Z}}^0$ as input and encodes it with self-attention. Given $\mathrm{\bf{Z}}^0\in \mathbb{R}^{K\times D^{\prime}}$ with $K$ key point features, each of which is $D^{\prime}$-dimensional, $\mathrm{\bf{Z}}^0$ is projected using $\mathrm{\bf{W}}_Q\in \mathbb{R}^{D^{\prime}\times D_q}$, $\mathrm{\bf{W}}_K\in \mathbb{R}^{D^{\prime}\times D_k}$, $\mathrm{\bf{W}}_V\in \mathbb{R}^{D^{\prime}\times D_v}$, where $D_k=D_q$, to extract feature representations query($\mathrm{\bf{Q}}$), key($\mathrm{\bf{K}}$) and value($\mathrm{\bf{V}}$), which can be defined as:
\begin{equation}
\begin{split}
&\mathrm{\bf{Q}}=\mathrm{\bf{Z}}^0\times \mathrm{\bf{W}}_Q\\
&\mathrm{\bf{K}}=\mathrm{\bf{Z}}^0\times \mathrm{\bf{W}}_K\\
&\mathrm{\bf{V}}=\mathrm{\bf{Z}}^0\times \mathrm{\bf{W}}_V
\end{split}
\end{equation}
and the output of self-attention can be computed as:
\begin{equation}
\begin{split}
\mathrm{\bf{Attn}}=\mathrm{Softmax}(\frac{\mathrm{\bf{Q}}\mathrm{\bf{K}}^T}{\sqrt{D_q}})\mathrm{\bf{V}}
\end{split}
\end{equation}
where $\mathrm{\bf{Attn}}\in \mathbb{R}^{K\times D^{\prime}}$. Also, we use common multi-head self-attention (MHSA) to make several self-attention operations calculate in parallel.

Now we introduce the overall architecture of Transformer Encoder, which has $L$ layers with each layer consisting of MHSA and MLP blocks. Also, LayerNorm and Residual Connection are applied before and after every MHSA or MLP block, respectively. Because the number of key points of each frame is  a bit less, so our encoder does not include the downsampling module that other models may have. The overall process can be defined as:
\begin{equation}
\begin{split}
&\mathrm{\bf{\hat{Z}}}^l = \mathrm{MHSA}(\mathrm{LN}(\mathrm{\bf{Z}}^{l-1})) + \mathrm{\bf{Z}}^{l-1}\\
&\mathrm{\bf{Z}}^l = \mathrm{MLP}(\mathrm{LN}(\mathrm{\bf{\hat{Z}}}^l)) + \mathrm{\bf{\hat{Z}}}^l
\end{split}
\end{equation}
where $\mathrm{\bf{Z}}^{l-1}$, $\mathrm{\bf{\hat{Z}}}^l$, $\mathrm{\bf{Z}}^l\in \mathbb{R}^{K\times D^{\prime}}$.


\subsection{Pose Mapping}
\label{third}
Taking the Encoder output $\mathrm{\bf{Z}}^L\in \mathbb{R}^{K\times D^{\prime}}$ as input, Pose Mapping module outputs probability scores $\mathrm{\bf{S}}\in \mathbb{R}^{C}$ of the current frame over all action classes. We perform binary classification after Sigmoid activation for each class, with the two salient poses of each class represented by the same bit data. To realize such a module, we use a very lightweight MLP network, which avoids the complexity. First, the two dimensions $K$ and $D^{\prime}$ of $\mathrm{\bf{Z}}^L$ are flattened into $\mathbb{R}^{KD^{\prime}}$, and then it passes through an MLP module, where the output channels is set to $C$, which can be defined as:
\begin{equation}
\begin{split}
\mathrm{\bf{S}} = \sigma(\mathrm{MLP}(\mathrm{Flatten}(\mathrm{\bf{Z}}^L)))
\end{split}
\end{equation}
where $\sigma$ represents the Sigmoid activation function.

With such Pose Mapping, we can obtain the scores of single frame. It should be noted that we extract the poses of all frames, and use the convenience of matrix operations to obtain scores in parallel, which is actually consistent with the idea of mini batch. So at last, we combine the scores of all frames to get the video score matrix $\mathrm{\bf{\hat{S}}}\in \mathbb{R}^{C\times T}$, where $T$ represents the number of frames in the current video. 


\subsection{Action-trigger Module}
\label{fourth}
We use the lightweight Action-trigger Module to obtain the final output $Y$, the repetitive action count, which has a time complexity of $\mathcal{O}(n)$. First, we get the scores $S_c\in \mathbb{R}^T$ of a given action class from $\mathrm{\bf{\hat{S}}}$. Then, we scan all frames and use the action-trigger mechanism to count when the two salient poses of the action class occur sequentially. We set upper and lower bounds to distinguish the scores of the two salient poses, which cluster non-salient poses in the middle and easily classify the salient poses to the two ends.

\subsection{Losses and Metric Learning}

The modules need to be trained are Embedding, Transformer Encoder and Pose Mapping, and because we perform binary classification for each class, so we use the Binary Cross Entropy Loss, which can be defined as follows:
\begin{gather}
\mathcal{L}_{bce} = -\frac{1}{N}\sum\limits_{i=1}^{N}(\frac{1}{C}\sum\limits_{j=1}^{C}loss(i,j))  \\
 loss(i,j)=y_{ij}\log p_{ij} + (1-y_{ij})\log(1-p_{ij})
\end{gather}
where $N$ represents the batch size (in our method, each frame is a batch), $C$ represents the number of classes, $y$ and $p$ are the labels and our predictions, respectively.

Moreover, we use Metric Learning to improve our Encoder and introduce the Pose Triplet Loss. Given a pose, Encoder produces higher-level features $\mathrm{\bf{Z}}^L$, which should be more representative. As shown in Figure \ref{fig5}, we achieve this with Triplet Margin Loss function, which selects anchors, same class positive samples, and different classes negative samples in a batch. It can be expressed as:
\begin{equation}
\begin{split}
\mathcal{L}_{tri} = \mathrm{max}(\mathrm{CS}(a,p)-\mathrm{CS}(a,n)+\mathrm{margin},0)
\end{split}
\end{equation}
where $a$, $p$, $d$ are anchors, positive and negative samples, and $\mathrm{CS}$ represents the Cosine Similarity to measure the distance between features. We pay more attention to hard samples, where the distances between anchors and negative samples are even smaller than those of positive samples. After Metric Learning, the poses of each action can be distinguishable, which cluster in the high-level space.

At last, our overall training combines these two losses:
\begin{equation}
\begin{split}
\mathcal{L} = \mathcal{L}_{bce} + \alpha\mathcal{L}_{tri}
\end{split}
\end{equation}
where $\alpha$ is the weight factor to control the two losses in the same numeric scale.
\subsection{Implementation Details}

\noindent{\bf Training.} We use the \emph{RepCount-pose} and \emph{UCFRep-pose} dataset we created to train our model. Only the frames with salient poses are inputted into the network instead of the entire video to speed up the fitting.

\noindent{\bf Inference.} During inference, the entire video sequence is inputted into the model. The poses of all frames pass through the Encoder and Pose Mapping, and then enter the Action-trigger Module to output the repetitive count.
\section{Extended Figures}

% \subsection{Momentum microscopy of the occupied molecular orbitals}
% For comparison with $GW$ calculation, we present a typical photoelectron spectrum of the occupied molecular orbitals.
\begin{figure}[hp]
    \centering
    \includegraphics[width=\linewidth]{figures/extendedFigure1_v1.png}
    \caption{Static photoelectron momentum microscopy of the multilayer \csixty{} sample, taken from the time-resolved data at -1000~fs. The high contrast and line shape of the occupied molecular orbitals as shown in the energy-momentum cut (\textbf{a}) and integrated energy distribution curve (\textbf{b}) confirm that an influence of the Cu(111) substrate can be ignored. \textbf{c, d:} Furthermore, the clear modulation of the HOMO (\textbf{c}) and HOMO-1 (\textbf{d}) momentum maps confirm the high crystallinity of the multilayer at 80~K\cite{haag_signatures_2020}}
    \label{fig:occupied_states}
\end{figure}


% \subsection{Exciton-resolved temporal dynamics}
\begin{figure}[hp]
    \centering
    \includegraphics[width = \linewidth]{figures/extendedFigure2_v4.png}
    \caption{\textbf{a, b:} Exciton-resolved measurement of the femtosecond relaxation dynamics after $h\nu =$2.9 and $h\nu =$3.6~eV excitation, respectively. The relative intensities were acquired by fitting the model in Eq.~\ref{eq:fitmodel_trPES} and \ref{eq:fitmodel_trPES_extension} to the momentum-integrated trPES data. Analysis of the time dependence indicates a sub 50~fs lifetime of the \stwo{} states. The comparably large \ctone{} population in the $h\nu =$~3.6~eV measurement can potentially be explained by a relaxation of the \stwo{} into both the \cttwo{} and \ctone{} states.}
    \label{fig:expdynamics}
\end{figure}


% Fitting error of the momentum map extraction
\begin{figure}[hp]
    \centering
    \includegraphics[width = \linewidth]{figures/extendedFigure3_v11.png}
    \caption{Amplitude (\textbf{a,b,c,d}) and standard deviation (\textbf{e,f,g,h}) of the extracted momentum maps of the \sone{}, \ctone{}, \cttwo{} and the exponential background ($A_{bg}(k)$), respectively. The data are scaled to the mean value inside the circled area.}
    \label{fig:kmap_errors}
\end{figure}
% \input{derivation_excitonPES}
% \putbib
% \end{bibunit}



\end{document}

