\section{Extended Figures}

% \subsection{Momentum microscopy of the occupied molecular orbitals}
% For comparison with $GW$ calculation, we present a typical photoelectron spectrum of the occupied molecular orbitals.
\begin{figure}[hp]
    \centering
    \includegraphics[width=\linewidth]{figures/extendedFigure1_v1.png}
    \caption{Static photoelectron momentum microscopy of the multilayer \csixty{} sample, taken from the time-resolved data at -1000~fs. The high contrast and line shape of the occupied molecular orbitals as shown in the energy-momentum cut (\textbf{a}) and integrated energy distribution curve (\textbf{b}) confirm that an influence of the Cu(111) substrate can be ignored. \textbf{c, d:} Furthermore, the clear modulation of the HOMO (\textbf{c}) and HOMO-1 (\textbf{d}) momentum maps confirm the high crystallinity of the multilayer at 80~K\cite{haag_signatures_2020}}
    \label{fig:occupied_states}
\end{figure}


% \subsection{Exciton-resolved temporal dynamics}
\begin{figure}[hp]
    \centering
    \includegraphics[width = \linewidth]{figures/extendedFigure2_v4.png}
    \caption{\textbf{a, b:} Exciton-resolved measurement of the femtosecond relaxation dynamics after $h\nu =$2.9 and $h\nu =$3.6~eV excitation, respectively. The relative intensities were acquired by fitting the model in Eq.~\ref{eq:fitmodel_trPES} and \ref{eq:fitmodel_trPES_extension} to the momentum-integrated trPES data. Analysis of the time dependence indicates a sub 50~fs lifetime of the \stwo{} states. The comparably large \ctone{} population in the $h\nu =$~3.6~eV measurement can potentially be explained by a relaxation of the \stwo{} into both the \cttwo{} and \ctone{} states.}
    \label{fig:expdynamics}
\end{figure}


% Fitting error of the momentum map extraction
\begin{figure}[hp]
    \centering
    \includegraphics[width = \linewidth]{figures/extendedFigure3_v11.png}
    \caption{Amplitude (\textbf{a,b,c,d}) and standard deviation (\textbf{e,f,g,h}) of the extracted momentum maps of the \sone{}, \ctone{}, \cttwo{} and the exponential background ($A_{bg}(k)$), respectively. The data are scaled to the mean value inside the circled area.}
    \label{fig:kmap_errors}
\end{figure}