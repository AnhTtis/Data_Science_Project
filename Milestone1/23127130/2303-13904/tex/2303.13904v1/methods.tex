\section{Methods}

\subsection{Femtosecond momentum microscopy of \csixty{}/Cu(111)}
We apply full multidimensional time- and angle-resolved photoelectron spectroscopy (tr-ARPES) to a multilayer \csixty{} crystal evaporated onto Cu(111), where the film thickness was such that no photoemission signature of the underlying Cu(111) could be observed in our experiment. We verified the sample quality by performing momentum microscopy of the occupied HOMO and HOMO-1 states simultaneously to the measurement of the excited states (see Extended Fig.~\ref{fig:occupied_states}). Femtosecond exciton dynamics were induced using $\approx$100~fs, $h\nu =$~2.9~eV or $\approx$100~fs, $h\nu =$~3.6~eV laser pulses derived from the frequency-doubled output of a optical parametric amplifier. The exciton dynamics were probed using our custom photoemission momentum microscope with a 500~kHz ultrafast 26.5~eV extreme ultraviolet (EUV) light source \cite{keunecke_time-resolved_2020} that enables us to map the photoelectron momentum distribution over the full photoemission horizon in a kinetic energy range exceeding 6~eV and an overall time resolution of $\approx$100~fs (the EUV pulse length is about 20~fs). The pump fluence was set to $90(10)$~$\mu$J/cm$^2$ and $20(5)$~$\mu$J/cm$^2$ for the $h\nu =$~2.9~eV and the $h\nu =$~3.6~eV measurement, respectively.
To prevent the free rotation of \csixty{} molecules, we cooled the sample down to $\approx$80~K \cite{wang_orientational_2001}. In addition to the resulting long-range periodic ordering of the \csixty{} crystal, cooling was also observed to prevent light-induced polymerization.  

\subsection{Momentum microscopy data analysis}
\subsubsection{Space-charge effects}
In the momentum microscopy experiment, a balance has to be found between sufficiently low pump and probe light intensities to avoid space-charge effects, but also having sufficient intensity for the optical excitation (pump) and reasonably short integration times (probe). Most of the time, for our settings, small space-charge effects are present in the data, but do not lead to strong distortions in the band-structure data and can be easily corrected. Therefore, the first step in the data analysis was to subtract a space-charge-induced delay- and momentum-dependent kinetic energy shift. For this purpose, the central kinetic energy of the HOMO was determined for normalization. To avoid the influence of the \csixty{} crystal band structure \cite{haag_signatures_2020}, we fitted a two-dimensional Lorentzian \cite{schonhense_multidimensional_2018} and shifted the kinetic energy distribution accordingly, leading to the expected overall flat shape of the molecular orbitals in the ARPES data. 

\subsubsection{Replicas from the 13\textsuperscript{th} harmonic}
%After the space-charge-corrected data we can directly extract time- and energy-dependent photoelecton spectra (trPES) such as shown in Fig.~\ref{fig:exp_S2}a. 
Although we use narrow-band multilayer mirrors to select the 11\textsuperscript{th} harmonic at $h\nu$ = 26.5~eV from our laser-based high-harmonic generation spectrum\cite{keunecke_time-resolved_2020}, we observe subtle replicas of the HOMO, HOMO-1, and HOMO-2 states in the unoccupied regime of the spectrum that are caused by photoemission from the 13\textsuperscript{th} harmonic at at $h\nu$ = 31.2~eV. %To analyze background-free excitonic state signals, 
To quantify these replica signals, we fitted the static reference spectrum above $E-E_H = 1.2$~eV (i.e., in the unoccupied regime of the spectrum) with three Gaussian-shaped peaks for the HOMO replicas and an exponential function to account for residual photoemission intensity in the unoccupied regime that is caused in this spectral region by the much stronger direct 11\textsuperscript{th} harmonic one-photon-photoemission from the HOMO state. After carrying out this fitting routine, we are able to calculate clean 11\textsuperscript{th} harmonic spectra (static and time-resolved) via subtraction of the fitted 13\textsuperscript{th} harmonic HOMO replicas. Note that we only subtract the replica signals, but not the background signal that is caused by one-photon-photoemission with the 11\textsuperscript{th} harmonic from the HOMO state, because this background is time-dependent\cite{stadtmuller_strong_2019}, and needs to be explicitly considered in the fitting procedure. The data shown in Fig.~3 of the main text is processed in the way described above. 

\subsubsection{Fitting procedure for the time-resolved data}
From replica-free trPES data for $h\nu =$~2.9~eV excitation, we determine the amplitude $A_i$, kinetic energy $E_i$, and bandwidth $\Delta E_i$ for the $i$\textsuperscript{th} exciton signature using a global fitting approach. In particular, we apply the model
\begin{equation}
\begin{split}
    I(E,t) & = \frac{A_{\cttwo{}}(t)}{\sqrt{2\pi}\Delta E_{\cttwo{}}} \mathrm{exp}\left[(E-E_{\cttwo{}}(t))^2/\Delta E_{\cttwo{}}^2\right] \\
     & + \frac{A_{\ctone{}}(t)}{\sqrt{2\pi}\Delta E_{\ctone{}}} \mathrm{exp}\left[(E-E_{\ctone{}})^2/\Delta E_{\ctone{}}^2\right]    \\
     & + \frac{A_{\sone{}}(t)}{\sqrt{2\pi}\Delta E_{\sone{}}} \mathrm{exp}\left[(E-E_{\sone{}})^2/\Delta E_{\sone{}}^2\right] \\ 
     & + A_{bg}(t)*\mathrm{exp}\left[-E/\tau\right].
\end{split}
\label{eq:fitmodel_trPES}
\end{equation}
Here, the last term is needed to account for the above-mentioned delay-dependent photoemission intensity that is caused by a transient renormalization of the HOMO state, as found in Ref.~\onlinecite{stadtmuller_strong_2019}.  %The delay-dependence of this background term is a direct consequence of the previously-observed renormalization of occupied states \cite{stadtmuller_strong_2019}. 

The fit results of this model applied to the $h\nu =$~2.9~eV excitation and momentum-integrated data are shown in Table~\ref{tab:fit_trPES}, and Extended Fig.~\ref{fig:expdynamics}a for the time-resolved exciton dynamics.

\begin{table}[h!]
\centering
\begin{tabular}{|c | c | c |} 
\hline
Exciton & Kin. Energy (eV) & Bandwidth (FWHM) (eV) \\ [0.5ex] 
\hline\hline
\cttwo{} (t=0~fs) & 2.768(2) & 0.606(4) \\ 
\ctone{} & 1.978(2) & 0.406(3) \\
\sone{} & 1.667(1) & 0.362(2) \\
\hline
\end{tabular}
\caption{ 
\label{tab:fit_trPES}
Peak parameters for $h\nu =$~2.9~eV excitation, extracted using Eq.~\ref{eq:fitmodel_trPES}. Note that we give the full width at half maximum (FWHM) for the bandwidth. }
\end{table}

For the measurement with $h\nu =$~3.6~eV excitation, we account for the \stwo{} exciton band by extending the model in Eq.~\ref{eq:fitmodel_trPES} with a set of Gaussian peaks with identical temporal evolution, given by
\begin{equation}
\begin{split}
    I(E,t) = ... + \frac{A_{\stwo{}}(t)}{2\sqrt{2\pi}\Delta E_{\stwo{}}} ( &\mathrm{exp}\left[(E-E_{\stwo{},upper})^2/\Delta E_{\stwo{}}^2\right] \\
    +& \mathrm{exp}\left[(E-E_{\stwo{},lower})^2/\Delta E_{\stwo{}}^2\right]),
\end{split}
\label{eq:fitmodel_trPES_extension}
\end{equation}
which follows the same notation as Eq.~\ref{eq:fitmodel_trPES}. Here, we set $E_{\stwo{},upper}$ to be close to 3.6~eV, and following the $GW$+BSE calculation we set $E_{\stwo{},upper} - E_{\stwo{},lower} = 1.4$~eV. 
Fitting this model to the momentum-integrated $h\nu =$~3.6~eV excitation data, we find $E_{\stwo{},upper} = 3.59(1)$~eV, and for the FWHM of the \stwo{} we find $0.58(3)$~eV. The time-resolved amplitudes retrieved using this model are shown in Extended Fig.~\ref{fig:expdynamics}b. Furthermore, this analysis was used in Fig.~3 of the main text to subtract the exponential background $A_{bg}(t)*\mathrm{exp}\left[-E/\tau\right]$ related to the transient broadening of the HOMO state. 

\subsubsection{Fitting procedure for the time- and momentum-resolved data}
In order to analyze the time-resolved data also momentum-resolved and thereby retrieve the momentum patterns that are shown in Fig.~\ref{fig:exp_excitons} and Fig.~\ref{fig:comparison} in the main text, we carry out the fitting routine separately for pixel-resolved energy-distribution curves %extend the model to include a momentum-dependence of the amplitudes following $A \rightarrow A'(k) A(t)$. This model is then applied to each individual pixel 
in the momentum distribution (1 pixel corresponds to $\approx$ 0.02~\AA\textsuperscript{-2}). To optimize the signal-to-noise ratio, we apply a three-fold rotational symmetrization and a mirror symmetrization that match with crystal symmetry. Nevertheless, the signal-to-background ratio in a small region around the center of the photoemission horizon remains below 1, and we therefore exclude this region in our analysis (grey areas in Fig.~4 in the main text). Also, for the momentum-resolved data, the replica HOMO background signals due to the 13\textsuperscript{th} harmonic amounts to 0, 1 or at most 2 counts in the pixel-resolved (momentum-resolved) energy-distribution curves, and can therefore not be fitted and subtracted accurately as described above for the momentum-integrated data. As such, we need to ignore the HOMO replicas from the 13\textsuperscript{th} harmonic in the momentum-resolved analysis. %, and carry out residual background subtraction in the fitting routine by a time-independent linear background of the form $\alpha(k) E + \beta(k)$. %that we attribute to a parasitic signal arising from space-charge effects close to the sample.
We avoid overfitting of the model in Eq.~\ref{eq:fitmodel_trPES} by fixing the energy and bandwidth of the peaks in the fitting routine to the parameters given in Table~\ref{tab:fit_trPES}. %energy and bandwidth to their previously determined values. 
Thus, the set of free parameters in the momentum-resolved fitting procedure is limited to $A_{\cttwo{}}(k)$, $A_{\ctone{}}(k)$, $A_{\sone{}}(k)$ and $A_{bg}(k)$. This approach enables the extraction of reliable momentum distributions also for the partially overlapping energy distributions of the \sone{} and \ctone{}. The $1\sigma$ errors for the full momentum maps are shown in Extended Fig.~\ref{fig:kmap_errors}. 

We note that the overall photoemission intensity of the \stwo{} peak in the $h\nu =$~3.6~eV excitation data is comparably low due to the sub-50~fs decay to the lower-energy \cttwo{} excitons (see Fig.~3 in main text, Extended Fig.~6b). Furthermore, with $h\nu =$~3.6~eV excitation, two-photon photoemission with 2~$*$~3.6~eV = 7.2~eV is sufficient to overcome the work function, so that space-charge effects could only be avoided by considerably reducing the $h\nu =$~3.6~eV pump intensity. Therefore, the signal-to-noise ratio in these measurements was not sufficient for a momentum-resolved analysis of the \stwo{} exciton data. 


\subsection{Calculation of the \csixty{} exciton spectrum}
The \emph{ab initio} calculation of the exciton spectrum of the \csixty{} film was performed in two steps, using a $GW$+BSE approach.
For the static electronic structure, we perform calculations for two unique \csixty{} dimers, which have been extracted from the known structure of the molecular film \cite{david_structure_1991, david_structure_1992} (see Fig.~\ref{fig:bse_excitons}c, dimers 1-2 and 1-4 respectively). Starting from Kohn-Sham orbitals and energies of a ground state DFT calculation (6-311G*/PBE0+D3) \cite{krishnan_basisset_1980, frisch_basisset_1984, adamo_pbe0_1999, grimme_d3_2010} using ORCA 5.0.1 \cite{neese_orca_2012, neese_update5_2022}, we employ the Fiesta code \cite{jacquemin_is_2017} to self-consistently correct the molecular energy levels by quasi-particle self-energy calculations with the $GW$ approximation. To account for polarization effects beyond the molecular dimer, we embed the dimer cluster in a discrete polarizable model using the MESCal program \cite{davino_mescal_2014, li_bse_2016, li_correlated_2017}. We found that mimicking 2 layers of the surrounding \csixty{} film in such a way resulted in the convergence of the band gap within 0.1~eV with a removal energy from the highest valence level of 6.65~eV. The close agreement of this quasi-particle energy with the experimentally determined work function of 6.5~eV gives us additional confidence in the choice of our embedding environment. The calculated quasi-particle energy levels are shown in Fig.~\ref{fig:bse_excitons}a. Here the finite width of the black bars actually arises from multiple energy levels forming bands on the energy axis. We characterize them according to symmetry \cite{dresselhaus_c60_1996} as HOMO-1, HOMO, LUMO, and LUMO+1 bands, each consisting of 18, 10, 6 and 6 energy levels per dimer, respectively. Note that we combine HOMO-1 is made up by states from two different irreducible representations of the isolated gas phase \csixty{} molecule which are practically forming a single band.

Building upon the $GW$ energies, we compute neutral electron-hole excitations by solving the Bethe-Salpeter equation beyond the Tamm-Dancoff approximation (TDA). This yields the excitation energies $\Omega_m$ and the electron-hole coupling coefficients $X^{(m)}_{vc}$ for a series of excitons labelled with $m$. 
We first analyze the resulting optical absorption spectrum which is shown in Fig.~\ref{fig:bse_excitons}c as black solid line. It reveals a prominent absorption band around $h\nu =$~3.6~eV that is well-known from gas-phase spectroscopy\cite{dai_ultravioletvisible_1994}. Secondly, the dimer calculation reveals a strong optical absorption at $h\nu =$~2.8~eV as well as a weakly dipole-allowed transition at $h\nu =$~2~eV. Both of these transitions are known to only appear in aggregated phases of \csixty{} \cite{wang_aggregates_1993}, and cannot be understood by considering only a single \csixty{} molecule. 
% Following existing literature \cite{emmerich_ultrafast_2020}, we refer to the exciton bands around $\Omega =$~1.8, 2.1, 2.8 and 3.6~eV as \sone{}, \ctone{}, \cttwo{} and \stwo{}, respectively. 
We refer to the exciton bands around $\Omega =$~1.9, 2.1, 2.8 and 3.6~eV as \sone{}, \ctone{}, \cttwo{} and \stwo{}, respectively. %Note that the \ctone{} and \cttwo{} states were recently reported to be of charge-transfer nature\cite{emmerich_ultrafast_2020}. 


In line with our classification of the $GW$ energy levels, one can group the composition of the excitons into four categories according to the contributing quasi-particle energy levels. As visualized in Fig.~\ref{fig:bse_excitons}c, this shows that \sone{} and \ctone{} are almost completely described by HOMO to LUMO transitions. On the other hand, \cttwo{} is predicted to have a small contribution of HOMO-1 character, however this contribution is too small to be reliably measured in our experiment. Finally, for \stwo{} we observe a clear and almost equal mixture of HOMO-1 and HOMO contributions, which is also confirmed by our measurements. Contributions from lower lying valence bands are negligibly small in the studied energy window.

\subsection{Calculation of the exciton momentum maps}
Based on our Kohn-Sham orbitals and BSE excitation coefficients, we calculate theoretical momentum maps for each exciton according to Eq.~\ref{eq:trPOT} following the derivation of Kern \textit{et al.}\cite{kern_exciton_tomog}. Note that for better readability, Eqs.~\eqref{eq:exciton_state} and \eqref{eq:trPOT} are given within the TDA, however, can in general be extended to include also de-excitation terms.\cite{kern_exciton_tomog} In the present case, we found the de-excitation contributions to be marginal (below 1\%) without affecting the appearance of the momentum maps and our interpretation. The energy conservation term comprises the BSE excitation energies ($\Omega_m$), the $GW$ quasi-particle energies for electron removal (i.e. ionization potential $\varepsilon_v$), and the probe energy of $h\nu =$~26.5~eV in accordance with our experimental setup. Furthermore, we include an inner potential to correct for the photoemission intensity variation of 3D molecules along the moment vector component perpendicular to the surface. Here, we choose a value of 12.5~eV, which has already been shown to match with experimental \csixty{} data \cite{hasegawa_potential_1998, haag_signatures_2020}. (Note that while considering the inner potential of the film is essential to describe the ARPES fingerprint, a variation of the inner potential between 12 and 14~eV indicated no influence on the interpretation of our results.)
As we exploited the plane-wave approximation, the calculated photoemission intensity is modulated by the momentum-dependent polarization factor $\left| \ve{A} \cdot \ve{k} \right|$, which we modelled as p-polarized light incoming with 68$^{\circ}$ to the surface normal according to experiments. To account for the symmetry of the \csixty{} film, the momentum maps were 3-fold rotated and mirrored. Finally, application of Eq.~\ref{eq:trPOT} provides us with a 4D data set of simulated photoemission intensity as a function of the excitation energy ($\Omega_m$), the kinetic energy ($E_{kin}$) and the momentum components $k_x$ and $k_y$. Analogous to experiment, we referenced the kinetic energy against the energy of the HOMO. The calculated ionization potential was further used to set the photoemission horizon of the theoretical momentum maps. Finally, to arrive at the theoretical momentum maps shown in Fig. \ref{fig:comparison}, we sum up the photoelectron intensities of each contributing electron-hole transition in an kinetic energy range of 200~meV centered on the respective exciton band.
