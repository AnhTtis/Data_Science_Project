% mnras_template.tex 
%
% LaTeX template for creating an MNRAS paper
%
% v3.0 released 14 May 2015
% (version numbers match those of mnras.cls)
%
% Copyright (C) Royal Astronomical Society 2015
% Authors:
% Keith T. Smith (Royal Astronomical Society)

% Change log
%
% v3.0 May 2015
%    Renamed to match the new package name
%    Version number matches mnras.cls
%    A few minor tweaks to wording
% v1.0 September 2013
%    Beta testing only - never publicly released
%    First version: a simple (ish) template for creating an MNRAS paper

%%%%%%%%%%%%%%%%%%%%%%%%%%%%%%%%%%%%%%%%%%%%%%%%%%
% Basic setup. Most papers should leave these options alone.
\documentclass[fleqn,usenatbib]{mnras}

% MNRAS is set in Times font. If you don't have this installed (most LaTeX
% installations will be fine) or prefer the old Computer Modern fonts, comment
% out the following line
\usepackage{newtxtext,newtxmath}
% Depending on your LaTeX fonts installation, you might get better results with one of these:
%\usepackage{mathptmx}
%\usepackage{txfonts}

% Use vector fonts, so it zooms properly in on-screen viewing software
% Don't change these lines unless you know what you are doing
\usepackage[T1]{fontenc}
\usepackage{subfig}
% Allow "Thomas van Noord" and "Simon de Laguarde" and alike to be sorted by "N" and "L" etc. in the bibliography.
% Write the name in the bibliography as "\VAN{Noord}{Van}{van} Noord, Thomas"
\DeclareRobustCommand{\VAN}[3]{#2}
\let\VANthebibliography\thebibliography
\def\thebibliography{\DeclareRobustCommand{\VAN}[3]{##3}\VANthebibliography}


%%%%% AUTHORS - PLACE YOUR OWN PACKAGES HERE %%%%%

% Only include extra packages if you really need them. Common packages are:
\usepackage{graphicx}	% Including figure files
\usepackage{amsmath}	% Advanced maths commands
\usepackage{amssymb}	% Extra maths symbols
\usepackage{chemformula}
\usepackage{float}
\usepackage{subfig}
\usepackage{bm}
\usepackage{hyperref}
\usepackage{gensymb}
\usepackage{amsmath,amssymb}
\usepackage[most]{tcolorbox}
\usepackage{xcolor}
\usepackage{adjustbox}
\usepackage{multicol}
\usepackage{wrapfig}
%\usepackage{enumitem}
\newcommand{\feh}{\ensuremath{\ch{[Fe/H]}} }
\newcommand{\cfe}{\ensuremath{\ch{[C/Fe]}} }
\newcommand{\fcemp}{$F_{\mathrm{CEMP}}$ }
\newcommand{\ag}[1]{\color{red}#1\color{black}}

%%%%%%%%%%%%%%%%%%%%%%%%%%%%%%%%%%%%%%%%%%%%%%%%%%

%%%%% AUTHORS - PLACE YOUR OWN COMMANDS HERE %%%%%

% Please keep new commands to a minimum, and use \newcommand not \def to avoid
% overwriting existing commands. Example:
%\newcommand{\pcm}{\,cm$^{-2}$}	% per cm-squared

%%%%%%%%%%%%%%%%%%%%%%%%%%%%%%%%%%%%%%%%%%%%%%%%%%

%%%%%%%%%%%%%%%%%%% TITLE PAGE %%%%%%%%%%%%%%%%%%%

% Title of the paper, and the short title which is used in the headers.
% Keep the title short and informative.
\title[On the dearth of CEMP-no stars in the Galactic bulge]{On the dearth of C-enhanced metal-poor stars in the Galactic bulge}

\author[G. Pagnini, S. Salvadori, M. Rossi et al.]{
G. Pagnini,$^{1}$\thanks{E-mail: mn@ras.org.uk (KTS)}
S. Salvadori,$^{2,3}$
M. Rossi,$^{2,3}$
D. Aguado,$^{2,3}$
I. Koutsouridou,$^{2,3}$
and A. Skuladottir$^{2,3}$
\\

$^{1}$GEPI, Observatoire de Paris, PSL Research University, CNRS, Place Jules Janssen, 92195 Meudon, France\\
$^{2}$Department, Institution, Street Address, City Postal Code, Country\\
$^{3}$Another Department, Different Institution, Street Address, City Postal Code, Country
}

% These dates will be filled out by the publisher
\date{Accepted XXX. Received YYY; in original form ZZZ}

% Enter the current year, for the copyright statements etc.
\pubyear{2015}

% Don't change these lines
\begin{document}
\label{firstpage}
\pagerange{\pageref{firstpage}--\pageref{lastpage}}
\maketitle

% Abstract of the paper
\begin{abstract}
According to the Stellar Archaeology, the chemical fingerprints of the first stars could be retained within the photospheres of old, low-mass, metal-poor second-generation stars observed in the Local Group. A significant fraction of these objects is represented by stars known as Carbon-Enhanced Metal-Poor (CEMP), characterized by an overabundance of carbon with respect to iron, $\cfe>0.7$. These have been observed in large quantities in the Galactic halo and in the Ultra-Faint Dwarf galaxies (UFDs) with \cfe reaching values up to $+4.5$. Interestingly, although the Milky Way bulge is predicted to host the oldest stars, the dearth of CEMP stars with high \cfe values here is particularly striking, since only one CEMP star has been found with $\cfe\simeq2.2$. In this paper, we explored the possible reasons for this anomaly. We initially performed a statistical analysis on the observations of metal-poor stars within the different regions of the Local Group. Then we focused on theoretical predictions derived from the $\Lambda$CDM cosmological model, through the combination of an N-body simulation and a semi-analytical model, and the use of analytical calculations. Our analysis shows that the scarcity of CEMP stars with high \cfe in the Galactic bulge is not due to the low statistics of metal-poor stars within it, but is the result of a different formation process of this region. In fact, by comparing our theoretical results with the observed sample of metal-poor stars in the Galactic bulge, we indirectly prove the existence of very massive first stars which, exploding as high-energetic pair-instability supernovae, partially washed out the high \cfe values caused by less energetic faint supernovae.
\end{abstract}

% Select between one and six entries from the list of approved keywords.
% Don't make up new ones.
\begin{keywords}
keyword1 -- keyword2 -- keyword3
\end{keywords}

%%%%%%%%%%%%%%%%%%%%%%%%%%%%%%%%%%%%%%%%%%%%%%%%%%

%%%%%%%%%%%%%%%%% BODY OF PAPER %%%%%%%%%%%%%%%%%%

\section{Introduction}
\label{sec:intro}
Today we are surrounded by stars - the Milky Way galaxy alone contains hundreds of billions of stars - but there was a time, billions of years ago, when stars were absent and the Universe was extremely simple: it was mostly neutral and mainly composed by hydrogen and helium produced during the Big Bang Nucleosynthesis.
The first stars have played a fundamental role in the evolution of the Universe, as they have been responsible for the transition from this very simple and early stage to the more complex one visible today. In fact, the first stars were the sources of the first chemical elements heavier than helium and of the first hydrogen-ionizing photons. Hence, they initiated the extended processes of \textit{reionization} and \textit{metal-enrichment}.

Within the standard $\Lambda$CDM model of structure formation, the first stars (referred to as \textit{Population III} or Pop III stars) are predicted to have formed within the first few hundred million years after the Big Bang, corresponding to redshifts of $z \sim 15-20$. 
The primordial birth environment of Pop III stars, characterized by the lack of heavy elements and dust, may have implied a higher Pop III characteristic stellar mass with respect to present-day stars (e.g \cite{tan2004formation}, \cite{hosokawa2011protostellar}) \ag{See my commented note at the end of this paragraph} although sub-solar mass stars might also be able to form (e.g. \cite{Greif_2011}, \cite{Stacy2016}). Since primordial star formation process is still poorly understood, we can state that the Initial Mass Function (IMF) of Pop III stars is almost completely unknown.
%To avoid double brackets in the references (something the journal will not accept I recomend to use \citet (no brackets) and \citep (brackets), so the good way to write it is: \citep[see e.g.,][]{tan2004formation, hosokawa2011protostellar} and the code write everything for you. I would recomend to implement this in the whole paper.
Despite long searches (e.g. \cite{Beers2005}, \cite{caffau2013}) zero-metallicity stars has not yet been observed, confirming the hypothesis of a primordial IMF biased towards more massive stars than the present-day IMF (e.g. \cite{salvadori2007cosmic}, \cite{rossi2021ultra}. Massive Pop III stars explode as supernovae (SNe) polluting the surrounding medium with their chemical products, whose yields depends upon the mass of the progenitor star along with the explosion energy (e.g. \cite{HegerWoosley2010} add nomoto). Hence, even if we cannot directly observe short-lived zero-metallicity stars, we can still catch their long-lived descendants: low-mass (Pop II) stars formed in environments enriched by the chemical products of the first stars up to the critical metallicity value (e.g. \cite{bromm2001fragmentation}, \cite{schneider2002first}). In this context Stellar Archaeology operates: searching for the chemical signatures of the first stellar generations in the photospheres of old ($> 12$ Gyr) and \textit{metal-poor} stars that dwell in our Galaxy and its ancient dwarf satellites. 
\\[3pt]

Which components of the Milky Way should be examined to find the oldest living stars? 
The first surveys looking for first star descendants have attempted to use kinematic information to select metal-poor stars in the Galactic stellar halo (e.g. {\cite{Beers_preston}, \cite{wisotzki2000hamburg}, \cite{christlieb2003finding}}), which is expected to be the most metal-poor component. Typically the iron-abundance\footnote{Throughout this paper we will be using the notation $[A/B] \equiv \log_{10} (N_A/N_B)_\ast - \log_{10} (N_A/N_B)_\odot$, where $N_A$ and $N_B$ refer to the numbers of atoms of elements A and B, respectively.}, [Fe/H], is measured in these surveys and used as a metallicity indicator. The most metal-poor stars are then selected for high-resolution follow-up, often revealing chemically peculiar stars with strong enhancements or deficiencies of particular elements (e.g. \cite{christlieb2003finding};  \cite{caffau2011extremely}; \cite{Keller2014}; \cite{bonifacio2015topos}; \cite{caffau2016topos}; \cite{Francois2018}; \cite{aguado2018j0023+};\cite{gonzalez2020}).  

The best-known type of chemically anomalous \ag{I would not use the word anomalous since CEMP stars became dominant al lower metallicities... Why not "chemically interesting"?} object at [Fe/H]$< -2$ is the \textit{carbon-enhanced metal-poor} (CEMP) class, which has $\cfe>0.7$ (e.g. \cite{Beers2005}). This class can be divided into two main populations: (i) carbon-rich stars that exhibit an excess in heavy elements formed by slow (or rapid) neutron-capture processes, named CEMP-s (CEMP-r) stars, and (ii) carbon-rich stars that do not exhibit such an excess known as CEMP-no stars (e.g. \cite{Beers2005}, \cite{Aoki_2007}). \ag{I would
rather say: (see \citet{Beers2005} or \cite{Aoki_2007} for a detailed taxonomy of EMP stars.} The first class of CEMP-s stars are commonly assumed to be chemically enriched by mass transfer from a companion star that has gone through the asymptotic giant branch (AGB) phase (\citet{abate2015carbon}) since these objects are preferentially found in binary systems (e.g. \cite{suda2004he}, \cite{lucatello2005binary}, \cite{Starkenburg_2014}, \cite{arentsen2019binarity}). 
 On the other hand, CEMP-no stars are rarely found in binary systems (\citet{lucatello2005binary}; \citet{norris2012most}, \citet{Starkenburg_2014}), hence their C-excess \ag{I would use enhancement instead of excess cause we did not assume old star should have same carbon abundance than the sun.} is expected to be most representative of the interstellar medium (ISM) out of which they formed, which was likely polluted by the first stellar generations (e.g. \cite{salvadori15}, \cite{deBen2016limits}). 

The variegate chemical abundance patterns of the most Fe-poor CEMP-no stars are indeed consistent with the yields of Pop III stars exploding as a \textit{faint supernovae} and experiencing mixing and fallback (e.g. \cite{iwamoto2005first}, \cite{Keller2014}) \ag{Is keller14 that you really want to cite here?}. Because of the low explosion energy, indeed, only the outer layers of the Pop III progenitor star, which are rich in CNO, can be expelled by faint SNe while the inner part, which is rich of Fe-peak elements, falls back into the center forming a neutron star or a black hole (e.g. \cite{HegerWoosley2010}). The increase fraction of CEMP-no stars at decreasing [Fe/H] supports such a link with PopIII star pollution (\cite{deBen2016limits}). CEMP-no stars have been found in a significant fraction in the Galactic halo (\cite{Yong_2012}, \cite{placco2014carbon}) and in the faintest satellites of the Milky Way, the so-called ultra-faint dwarf galaxies (\cite{spite2018cemp}, \cite{Norris_2010}, \cite{Lai_2011}, \cite{Gilmore_2013}), which are the oldest galaxies in the Local group (e.g. \cite{Simon_2010}, \cite{gallart2021star}). On the other hand, CEMP-no stars seem to be quite rare in the more luminous classical dwarf spheroidal (dSph) galaxies (\cite{Skul_2015}), which show more complex and longer star formation histories with respect to ultra-faint dwarfs (see \cite{salvadori15} for a global view). Ultimately, these observational results confirm that the descendants of the first stars are preferentially found in purely ancient environments. 

Relying on the $\Lambda$CDM model and hierarchical clustering, \cite{white2000first} first predicted the oldest stars to be in the inner part of the Milky Way, i.e. the Galactic bulge, idea that was confirmed in the following years through different numerical simulations (\cite{diemand2005earth}, \cite{tumlinson2009chemical}, \cite{salvadori2010mining}, \cite{starkenburg2016oldest}).  
Unfortunately, the Galactic bulge is a dusty and overcrowded region, predominantly populated by metal-rich stars, so metal-poor objects are difficult to find (\cite{zoccali2008metal}, \cite{ness2013argos}, \cite{howes2016embla}).  
The EMBLA Survey (Extremely Metal-poor bulge stars with AAOmega), has been the first in attempting to discover candidate metal-poor stars in this inner region (e.g. \cite{howes2014gaia}, \cite{howes2016embla}). Although it successfully identified $\approx 30$ stars with $\feh<-2$, from its observations it is clear that there is a {\it dearth} of CEMP-no stars in this region; in fact, only one CEMP-no star has been found having $\feh=-3.48$ and $\cfe=0.98$ (\cite{Howes2015}). 

More recently, the Pristine Inner Galaxy Survey (PIGS, \cite{arentsen2021pristine}) targeted the Galactic bulge with low/intermediate resolution spectroscopy (R $\approx 1300$ at 3700-5500 \AA, and $R\approx 11000$ at 8400-8800 \AA) collecting a sample of 1900 stars with $\feh < -2.0$, i.e. the largest sample of confirmed very metal-poor stars in the inner Galaxy to date. In the absence of s-process abundance measurements, a different classification has been made to separate the newly discovered CEMP stars based on the absolute carbon abundance\footnote{Elemental abundances can also be referred to as an “absolute” scale, relative to the number of hydrogen atoms, defined as $A(C) \equiv \log_{10} (N_C/N_H)_\ast + 12$.}, A(C), and \feh of the stars. According to \cite{bonifacio2015topos} (see also \cite{spite2013carbon}) CEMP stars can be divided in the two main groups:
(i) the high-carbon band, A(C) > 7.4, largely containing CEMP-r/s stars that typically have higher iron-abundance; (ii) the low-carbon band, A(C) < 7.4, predominantly containing CEMP-no stars at lower [Fe/H]. Using this classification and also correcting for evolutionary effects (e.g. \cite{placco2014carbon}), the PIGS survey identified 24 new CEMP-no candidate stars in the Galactic bulge. Still, the overall fraction of CEMP-no stars at [Fe/H]$<-2$ obtained by PIGS is only $\lesssim 6\%$, i.e. it is much lower than what is found in the Galactic halo ($\approx 20\%$, see \cite{arentsen2021pristine}). Furthermore, $\simeq96\%$ of the CEMP-no candidates observed by PIGS have moderate C-enhancement, $ +0.7 < \cfe < +1.2$, and there is only one CEMP-no star at [Fe/H]$\simeq-3.45$ that shows a high \cfe ($\simeq +2.2$), values that are typically observed both in the Galactic halo and in ultra-faint dwarf galaxies (e.g. Fig. 1 from \cite{salvadori15}).

\textit{Why there is an apparent dearth of CEMP-no stars in the Galactic bulge? And why CEMP-no stars with high \cfe are almost lacking in this ancient environment?} The aim of this paper is to answer these questions, and it is structured as follows. In Section \ref{sec:obs} we will try to understand whether the dearth of C-enhanced stars in the Galactic bulge is due to a statistical effect, by carrying out an analysis on the observational data of metal-poor stars in the different regions of the Local Group. In Section \ref{sec:model} we will illustrate the cosmological model used to follow the evolution of the Galaxy, which consists of a combination of a N-body simulation and a semi-analytical model. The results obtained are described in Sec. \ref{sec:results}. Finally, in Sec. \ref{sec:concl} we will draw the conclusions of our analysis with the associated implications, and we will list the perspectives for future works.
\ag{Introduction is very complete and I like it! Have you considered the chance to include the technical details of the PIGS sample in the next section? Not sure but maybe in a short subsection!}
\section{Observational data analysis}
\label{sec:obs}

To understand why CEMP-no stars are less common in the Galactic bulge with respect to other ancient environments of the Local Group, we should first analyze if this dearth can be just a statistical effect. Since CEMP-no stars are preferentially found at [Fe/H]$< -2$, indeed, the probability to discover them in environments dominated by [Fe/H]$>-2$ stars, such as the Galactic bulge, can be intrinsically very low \citep{salvadori15}.

\begin{figure}
    \centering
    \includegraphics[width=.48\textwidth]{images/fcemp_halo_3_04.png}
    \caption{Fraction of CEMP-no stars belonging to the Galactic halo in different $\ch{[Fe/H]}$ ranges (gray shaded histograms). On the top, the number of stars in each bin is shown. Previous results using the high-resolution sample by \cite{yong2012bmost} are shown with black points and Poissonian errors.}
    \label{fig:fcemp_halo}
\end{figure}

To make this calculation we should first compute the fraction of CEMP-no stars, defined as the ratio of the number of CEMP-no stars over the number of total stars at a given $\feh$. Henceforth we will simply speak of CEMP stars while referring to the subclass of CEMP-no stars, and hence we will refer to the fraction of CEMP-no stars as:

\begin{equation}
    F_{\mathrm{CEMP}}(\feh) = \frac{N_{\mathrm{CEMP}}(\feh)}{N_{*}(\feh)}.
\end{equation}

The Galactic halo has certainly been the best observed ancient environment of the Local Group and for which we have the largest statistics regarding the chemical properties of the population of metal-poor stars \citep[e.g.][]{yong2012bmost,bonifacio}. We compute $F^{halo}_{\mathrm{CEMP}}$ considering the sample of 1413 \textit{halo stars} with carbon measurements from {\texttt{JINAbase}}, which includes the latest observations. {\bf STEF: nel JINAbase c'è anche l'abbondanza di Barium? In altre parole, sai per certo che sono CEMP-no? Questo va chiarito. La cosa che non mi torna è che sembra che la tua Fraction abbia un second peak a -2, mentre tutte le altre sono monotonicamente descrescenti. Perchè è così? Sicura che non hai anche CEMP-unsure or CEMP-r/s?}\ag{DAVID: JINA include pulsanti per selezionare solo diversi tipi di CEMP (-s, -r, -r/s o anche -i) in modo che fare una piccola statistica dovrebbe essere molto facile.} 
The results are shown in Fig.\ref{fig:fcemp_halo}, where for comparison we also display $F_{\mathrm{CEMP}}$ obtained by using the high-resolution Galactic halo measurements for stars at [Fe/H]$< -3$ known 10 years ago and collected by \cite{yong2012bmost}. As we can see, in both cases $F_{\mathrm{CEMP}}$ increases as $\feh$ decreases and reaches the value of $\approx 1$ for $\feh \leq -5$. Because of the large number of CEMP-no stars at $\feh \leq -4 $ discovered during the last 10 years \ag{\citep[see e.g.,][and references therein]{Yong_2012,norris2012most,Keller2014,bonifacio2015topos,agu17II,Dacosta2019,nordlander2019,lardo2021,li2022}}({\bf REFS here: David, please, help us!}), we can now derive $F_{\mathrm{CEMP}}$ with extreme accuracy at all [Fe/H] values, equally spaced every 0.5 dex. Still, we should remind that these values suffer uncertainties. Indeed, from the one hand $F^{halo}_{\mathrm{CEMP}}$ might increase when correcting the carbon measurements to account for the internal depletion of carbon occurring in Red Giant Branch stars \citep{placco2014carbon}. From the other hand, $F^{halo}_{\mathrm{CEMP}}$ might decrease when accounting for non-LTE effects, which can drastically lower the measured [C/Fe] value (Amarsi, Nissen, Skuladottir 2019). %\cite{amarsi2020}. 
Since the two effects compensate and non-LTE corrections are only available for a handful of stars (Amarsi, Nissen, Skuladottir 2019), we will consider the derived $F^{halo}_{\mathrm{CEMP}}$ as the reference fraction of Galactic halo stars at different iron abundances born in C-enhanced environments. Furthermore, we will assume that the fraction of CEMP-no stars is {\it the same in all environments} and equal to the Galactic halo, $F_{\mathrm{CEMP}} \equiv F^{halo}_{\mathrm{CEMP}}$.
\\[2pt]

To understand if CEMP-no stars might be simply hidden in the Galactic bulge, we need to derive the probability to catch CEMP-no stars while blindly observing the bulge and then compare this value with the ones derived for all the other Local group environments \cite{salvadori15}. To this end we analyzed the number of observed metal-poor stars with available carbon measurements in environments with {\it increasing} stellar mass (or luminosity) and {\it increasing} average stellar metallicity. In particular we considered: 

\begin{center}
\begin{itemize}
\item The least luminous ultra-faint dwarf galaxies:\\ 
$ L\leq 10^4 L_\odot$, $\langle\feh\rangle\lesssim-2.2$\;
\item The most luminous ultra-faint dwarf galaxy \textit{Boötes I}\\
($ L \approx 10^{4.5} L_\odot$, $\langle\feh\rangle\sim-2.1$);
\item The dwarf spheroidal galaxy \textit{Sculptor}:\\ 
$ L \approx 10^{6.34} L_\odot$, $\langle\feh\rangle\sim-1.8$, 
\item The dwarf spheroidal galaxy \textit{Fornax}:\\   
$ L \approx 10^{7.25} L_\odot$, $\langle\feh\rangle\sim-1.0$
\item The Galactic {\it bulge}:\\ 
$ L \approx 10^{10} L_\odot$, $\langle\feh\rangle\sim0.0$.
\end{itemize}
\end{center} 

The carbon measurements in the least luminous ultra-faint dwarf galaxies (UFDs) are from high-resolution spectroscopic studies (\textit{Segue 1}: \cite{Norris_2010}; \cite{Frebel_2014}. \textit{Pisces II}: \cite{refId0}. \textit{Ursa Major II and Coma Berenice}: \cite{Frebel_2009}. \textit{Leo IV}: \cite{Simon_2010}). Few stars have been observed in each UFD because these galaxies are faint and distant so that only few Red Giant Branch stars are available for spectroscopic observations. For this reason we consider stars belonging to UFDs all together. 
Stars in \textit{Boötes I}, the most luminous UFD, are studied with low-resolution spectroscopy (\cite{Lai_2011}, \cite{Norris_2010}) with the only exception of seven stars \citep{Gilmore_2013}. 
In the \textit{Sculptor} dwarf spheroidal (dSph) galaxy, many carbon measurements are available from both low- (\cite{Kirby_2015}) and high-resolution spectroscopic studies (\cite{Frebel2010}); \cite{tafel_2010}; \cite{Starkenburg_2014}; \cite{Simon_2015}; \cite{Jablonka_2015}; \cite{Skul_2015}). 
Finally, for the \textit{Fornax} dSph we exploit the low-resolution measurements from \cite{Kirby_2015}.

For the Galactic bulge, we have used data taken from a database of metal-poor stars called {\texttt{JINAbase}}\footnote{\url{https://jinabase.pythonanywhere.com/}}(\cite{abohalima2018jinabase}). The abundance and stellar parameter data collected there is based on high-resolution (resolving power of $R = \lambda/\Delta\lambda \gtrsim 15,000$, with the majority having $R = 30,000 - 40,000$) spectroscopic studies found in the literature. In addition to these data we have also considered those provided by the high-resolution, large multi-object APOGEE spectroscopic survey (Data Release 16, \cite{jonsson2020apogee}. 
%In the APOGEE database the objects defined as bulge stars are those satisfying the following criteria:
%\begin{itemize}
%\item latitude $\rightarrow   -20\degree\lesssim  \textit{l} \lesssim  +20\degree$
%\item longitude $\rightarrow   -25\degree \lesssim  \textit{b} \lesssim  +25\degree$
%\item color cut $\rightarrow   (J-K)_0\geqslant 0.5$
%\item magnitude cut $\rightarrow   H=12.2 $ mag.
%\end{itemize}

\subsection{Are CEMP stars hidden in the Galactic bulge?}
\begin{figure*}
\begin{multicols}{2}
\includegraphics[width=.48\textwidth]{images/mdfs_3_04.png}
\par%
\includegraphics[width=.48\textwidth]{images/pobs_3_04.png}%
\par
\end{multicols}
\caption{The MDF (left panel) and the probability of observing a CEMP star (right panel) for the different environments displayed in order of increasing luminosity: ultra-faint galaxies (orange), Boötes (red), Sculptor (pink), Fornax (green) and the bulge (blue).}
\label{fig:ordered}
\end{figure*} 

For a ``blind'' galaxy survey that does not select the most metal-poor stars, the ability to find CEMP stars results naturally limited by the number of stars that exists at each $\feh$ value i.e. by the Metallicity Distribution Function (MDF, \citet{salvadori15}). In Fig.~2 (upper panel), we show the normalized MDFs for the different environments studied, in order of increasing brightness from the least luminous dwarf galaxies to the bulge. Note that the total number of stars observed in these environments strongly increase with galaxy luminosity: only 19 stars constitute the MDF in the faintest UFDs, while for the Galactic bulge we have $>17000$ stars. 


We then ask ourselves: \textit{if we assume that the fraction of CEMP stars is ``Universal'' and equal to the Galactic halo one, what is the joint probability to observe a star that has a given [Fe/H] value, and that is also carbon-enhanced?} This quantity, named $P_{obs}$, has been computed for each dwarf galaxy and for the Galactic bulge, combining two independent functions: the \textit{halo} fraction of CEMP stars and the normalized MDF derived for each environment:
\begin{equation}
P_{obs}([Fe/H]) = F^{halo}_{\mathrm{CEMP}} (Fe/H]) \times \frac{N_*([Fe/H])}{N_{tot}},
\end{equation}
%For the calculation of $P_{obs}$, we stress that we have used the fraction of carbon-enhanced stars from the halo as it is more reliable due to the higher statistic with which it was calculated, but still explanatory of the general trend.

In the bottom panel of Fig. \ref{fig:ordered} we show $P_{obs}$ for the different environments studied, following the same order of increasing brightness from the least luminous UFDs to the bulge. If we compare these quantities, we note a clear trend: the brighter the galaxy is, the lower and more shifted towards higher $\feh$ values, is the overall probability to observe a metal-poor star that is also carbon-enhanced. 
Since $F_{CEMP}$ used to compute this probability is the same for all environments (the one derived from the Galactic halo), this feature can be explained by the fact that the average MDF is shifted towards higher $\feh$ values as the galaxy luminosity increases (Fig.\ref{fig:ordered}, top panel). In other words, in the most massive galaxies, stars in the low-$\feh$ tail result almost``invisible'' with respect to metal-rich ones since they are much more rare. 
This property is especially evident in the bulge since the peak of the resulting probability ($P_{obs}<0.1\%$) is two order of magnitude lower than the values obtained in UFDs ($P_{obs}\sim 10\%$), where more CEMP stars have been actually discovered. Furthermore, the peak of $P_{obs}$ is shifted towards higher $\feh$ values with respect to all dwarf galaxies.

In conclusion, the scarcity of stars at low-$\feh$, where CEMP stars usually appear (Fig. \ref{fig:fcemp_halo}), makes it very difficult to search for CEMP stars in luminous environments dominated by metal-rich stars such as the Galactic bulge. So the dearth of CEMP stars in the Galactic bulge can be partially ascribed to the reason. However, both EMBLA and PIGS are not a ``blind'' surveys since they specifically search for the most metal-poor stars in the Galactic bulge. PIGS, in particular, obtained a sample of 1900 stars with $\feh<-2$. Thus, if we stack on the hypothesis that the CEMP fraction within the bulge is equal to the one we obtained for the Galactic halo, namely $F^{bulge}_{CEMP}(\feh<-2)=F^{halo}_{CEMP}(\feh<-2)\approx 24\%$, then we can estimate that $\sim450$ c-enhanced stars should have been observed by PIGS. Instead, $<62$ CEMP(-no) candidates have been discovered so far (\cite{arentsen2021pristine}). In agreement with the PIGS papers, we thus conclude that the dearth of CEMP stars in the Galactic bulge cannot just be a consequence of a statistical effects in this metal-rich dominated environment.

In the next section, we will investigate if the fraction of CEMP stars could be intrinsically different within the Galactic bulge by using the combination of a N-body simulation for the Milky Way assembly and a semi-analytical model for the chemical evolution.
\\

\section{Cosmological model description}
\label{sec:model}
In this Section we will briefly recap the cosmological model (\citet{salvadori2010mining, pacucci2017gravitational}) used to identify the first star-forming halos that are currently located in the Galactic bulge and to investigate their properties.

%In this Section we will briefly recap the cosmological model used to follow the evolution of our Galaxy, from the formation of the first stars down to the present-days (\citet{salvadori2010mining, pacucci2017gravitational}) 
%underlying the model assumption/implementations adopted for the purpose of this work.
The model combines a N-body simulation that follows the hierarchical assembly of a MW-like galaxy with a semi-analytical model (\citet{salvadori2007cosmic,salvadori15}) that is required to follow the evolution of baryons, from the formation of the first stars down to the present-days. The semi-analytical model allow us to follow the star formation and metal enrichment history of the Milky Way from the early times ($z\sim20$) up to now ($z=0$) and thus to link the chemical abundances of present-day stars with the properties of the first stellar generations\footnote{Since we want to study the most ancient stars, we are not interested in the formation of the disk that has only recently assembled at $z<2$ (e.g. \cite{mo1998formation})}. More details are provided below.

\label{subsec:nbody}
\begin{itemize}
\item{\bf N-body simulation}
The N-body simulation used to study the hierarchical formation of the Milky Way has a low-resolution region corresponding to a sphere of radius $10h^{-1}$ Mpc and a highest resolution region of radius $1h^{-1}$ Mpc radii, i.e. equal to four times the virial radius of the MW at z = 0 ($r_{vir} = 239$ kpc) (see \cite{scannapieco2006spatial} for details). 
A low-resolution simulation including gas physics and star formation has been used to confirm that the initial conditions will lead to a disk galaxy like the Milky Way.
The system consists on about $10^{6}$ DM particles within $r_{vir}$ with masses of $7.8 \times 10^{5} M_{\odot}$; its virial mass and radius are respectively $M_{vir} = 7.7 \times 10^{11} M_{\odot}$ and $r_{vir} = 239$ kpc, roughly consistent with the observational estimates of the MW ($M_{vir}\approx 10^{12} M_{\odot}$ e.g. \cite{McMillam}). The softening length is 540 kpc. The simulation data is output every 22 Myr between z = 8 - 17 and every 110 Myr for z < 8. At each output, the virialized DM halos have been identified using a \textit{friend-of-friend} group finder with a \textit{linking parameter} b = 0.15, and a threshold number of particles constituting virialized halos equal to 50. The N-body simulation enable to reconstruct the hierarchical history of the MW that proceeds through the consecutive merging of DM halos maintaining the information about the spatial distribution of DM particles belonging to them.
\\[2pt]

\item \textbf{Star formation}
At the initial redshift of the simulation, $z=20$, the gas in DM halos is assumed to have primordial composition and only objects with virial temperature $T_{vir}\geq 10^4$ K are assumed to form stars. This choice, which is equivalent to suppose that the star formation activity is rapidly quenched in mini-halos, is dictated by the limited DM resolution of the N-body simulation \citep{graziani2015galaxy}.
At each time-step, stars are assumed to form in a single bursts, proportional to the available cold gas mass, $M_{gas}$. The constant of proportionality is a redshift-dependent star-formation efficiency $\rm f_{*}(z)=\epsilon_* \; \frac{\Delta t(z)}{t_{ff}(z)}$, where $\rm t_{ff}$ is the free-fall time, $\rm \Delta t(z)$ the N-body time-step and $\epsilon_{*}$ a free parameter of the model, physically corresponding to a \textit{“local” star formation efficiency}.
\\[2pt]
\item \textbf{PopIII stars}
The model simply assumes that the Pop III IMF is a delta function centered either on the average mass value of the PISNe \cite{salvadori2010mining} or faint SNe \cite{pacucci2017gravitational}, i.e:
    \begin{equation}
    \label{eq:pisn_imf}
    \Phi(m) = \frac{dN}{dm} \propto \delta(m_{popIII}).
    \end{equation}
To have an idea of the amount of chemical elements released by these different Pop III stars, an ``average" PISN ($\rm m_{PopIII}=200\,M_{\odot}$) provides an yield of metals, iron, and carbon respectively equal to: $Y\def M_Z/M_{popIII} = 0.45$, $Y_{\ch{Fe}} = 0.022$ and $Y_{\ch{C}} = 0.02$ (\cite{heger2002nucleosynthetic}). Conversely, an average faint SN ($\rm m_{*} =25\,M_{\odot}$) gives $Y \sim 0.1$, $Y_{\ch{Fe}} = 4\times10^{-7}$ and $Y_{\ch{C}} = 9.98\times10^{-3}$ (\cite{iwamoto2005first}; \cite{marassi2014origin}).\\    

\item \textbf{PopIII-to-PopII transition}
Following the critical metallicity scenario (\cite{bromm2001fragmentation}; \cite{schneider2002first}) we assume that the IMF of the newly formed stars depends upon the initial metallicity of the star forming clouds. Therefore, a star forming halo with an initial gas metallicity $Z \leq Z_{cr}$ will host Pop III stars, otherwise, if $Z > Z_{cr}$, it will host PopII/I stars. Exploiting the data driven constraints of \cite{deBen2016limits} we set $Z_{cr} = 10^{-4.5}Z_{\odot}$. Normal Pop II/I stars are assumed to have masses in the range $[0.1, 100]\,M_{\odot}$ and to form according to a standard Larson IMF (\cite{larson1998early}), which peaks at the characteristic mass $\rm m_{ch} = 0.35M_{\odot}$ and rapidly decline with a Salpeter-like shape towards larger masses. 
\\[2pt]

\item \textbf{Metal enrichment}
The newly formed stars are assumed to instantaneously evolve  at the following snapshot of the simulation (IRA approximation) since the time elapsed between two neighbouring steps is larger than the lifetime of the least massive stars evolving as SNe. For Pop III stars exploding as PISN we assume the yields of \cite{heger2002nucleosynthetic}, while for primordial faint SN we assume those of \cite{iwamoto2005first}. 
To not overestimate the contribution of carbon due to AGB stars, which also produce slow-neutron capture process elements and thus CEMP-s stars, we only considered the chemical products of PopII/I that evolve as core-collapse SNe in short timescales ($3-30$ Myr). %Since most AGB stars have lifetimes of several Gyr, we can state that our model is suitable to study the oldest stellar components at $z > 6$.
After SN explosions the newly produced/injected metals are assumed to be instantaneously and homogeneously mixed within the Inter Stellar Medium (ISM) eventually reaching the Inter Galactic Medium (IGM) via supernova driven outflows.
\\[2pt]

\item \textbf{Gas and metals dispersal}
Supernovae may release a high amount of energy, which may overcome the binding energy of the hosting halo leading to partial gas and metals removal from the galaxy itself.
The mass of gas ejected into the IGM, $\rm M^{ej}_{gas}$, depends on the balance between the escape velocity of the halo, $v_{esc}$, and the kinetic energy released during the explosion, namely:
\begin{equation}
\rm M^{ej}_{gas}=(2E_{SN})/v^2_{esc}
\label{eq:m_ej}
\end{equation}
where $\rm E_{SN} = \epsilon_w\,N_{SN}\,\langle E_{SN}\rangle$, with $\rm N_{SN}$ number of SN explosions and $\rm \langle E_{SN}\rangle$ the average explosion energy. For a typical PISN ${\rm \langle E^{PISN}_{200M_{\odot}}\rangle\sim 2.7\times 10^{52}}$ erg (\cite{heger2002nucleosynthetic}) while a typical faint SN provides $\rm \langle E^{faint}_{25M_{\odot}}\rangle\sim 0.7\times 10^{51}$ erg \citep{iwamoto2005first,marassi2014origin}, which is lower than the energy released by a normal   $25\,M_{\odot}$ core-collapse SN, $\rm E^{cc}_{25M_{\odot}}\approx10^{51}$ erg.

The quantity $\epsilon_w$, representing the second free parameter of the model, is the \textit{wind efficiency} i.e. the fraction of the explosion energy converted into kinetic form. 
\\[2pt]

\item \textbf{Chemical evolution}
Due to mechanical feedback, the mass of gas and metals in a halo can decrease substantially. At each simulation time-step the gas mass reservoir in each halo, $\rm M_{gas}$, is updated with respect to the initial gas mass, $\rm M^{in}_{gas}$, to account for the mass of stars locked into newly formed stars, $\rm M_{*}$, and the gas mass ejected out of the halo, $\rm M^{ej}_{gas}$:
\begin{equation}
\label{eq:m_gas}
    \rm M_{gas} = M^{in}_{gas} - (1+R) M_* - M^{ej}_{gas}.
\end{equation}
where $R$ is the returned fraction, which is equal to 1 only for PISN and lower than unity otherwise.
Similarly, the mass of metals, $\rm M_{Z}$, in each hosting halo is updated as follow:
\begin{equation}
    \rm M_{Z} = {M^{in}_{Z}+Y\,M_{*} - Z^{in}_{ISM}M_*- Z^{in}_{ISM}M^{ej}_{gas}}{M_{gas}},
\label{fig:m_Z}
\end{equation}
where $\rm M^{in}_{Z}$ is the initial mass of metals, $\rm Y_{Z}$ the metal yield, and $\rm Z^{in}_{ISM}$ the initial metallicity of the ISM.
\\[2pt]

\item {\bf Model calibration}
To set the best values of the two model free parameters, $\epsilon_*$ and $\epsilon_w$, the observed properties of the MW have been used as a benchmark. In particular, the results of the simulations at redshift $z = 0$ have been compared with the gas/stellar mass and metallicity of the MW, the baryon-to-dark matter ratio, and the metallicity of high-velocity clouds \citep[see][for more details]{salvadori2007cosmic,salvadori2010mining}.
\end {itemize}

%Two ranges of Pop III masses are fundamental to our study on the chemical abundances of ancient present-day stars at $\feh < -1$, namely:
%\begin{enumerate}
%    \item[-] $8\, M_{\odot} \leq m_{PopIII} \leq 40\, M_{\odot}$ i.e. Pop III stars exploding as \textit{faint supernovae};
%    \item[-] $140\, M_{\odot} \leq m_{PopIII} \leq 260\, M_{\odot}$ i.e. Pop III stars exploding as \textit{Pair Instability SNe}.
%\end{enumerate}
%Since the Pop III IMF is unknown, we have explored different IMF shapes which will further explained in the next sections. 
%
%\subsection{Varying the Pop III IMF}
%As we have seen in Section \ref{sec:intro}, the unusual chemical compositions of the most iron-poor and carbon-rich stars can be successfully matched by models of primordial faint SN that experienced mixing and fallback (e.g.\cite{umeda2003first}; \cite{iwamoto2005first}; \cite{joggerst2009mixing}; \cite{marassi2014origin}; \cite{tominaga2014abundance}). 
%In order to include this type of supernova in addition to PISNe in our analysis, and to account for the still unknown shape of the Pop III IMF, we have considered the following cases:
%\begin{enumerate}
%    \item a delta function centered on a stellar mass of $25 M_{\odot}$, 
%    \begin{equation}
 %   \label{eq:faint_imf}
%    \Phi(m) = \frac{dN}{dm} \propto \delta(25M_{\odot}),
%    \end{equation} 
%    to account only the contribution of faint supernovae in the chemical enrichment;
%    \item a Larson IMF,
%    \begin{equation}
%    \Phi(m) = \frac{dN}{dm} \propto m^{-1+x} \mathrm{exp}(-m_{ch}/m),
%    \end{equation}
%    with $x = -1.35$, $m_{ch} \in \{1, 10, 100 \,M_{\odot}\}$ and $m$ in the range $(m_{min}, m_{max}) = [0.8, 1000]\,M_{\odot}$, to account both faint SNe and PISNe in different proportions.
%    \item a Flat IMF, in the range $(m_{min}, m_{max}) = [0.8, 1000]\,M_{\odot}$, to account both faint SNe and PISNe in the same proportions.
%\end{enumerate}

%\subsection{Nucleosynthetic elements}
%We have assumed that stellar lifetime depends not only on the stellar mass but also on the stellar metallicity set by the ISM out of which stars formed (\cite{raiteri1996simulations}).
%To characterize metal-poor stars and the chemical evolution of DM halos, we have computed the abundances of the elements relevant to the present study i.e. Fe and C.
%Since the yields of iron and carbon vary depending on the mass of the star exploding as a supernova, the total mass of the X element is calculated as:
%\begin{equation}
%    M_{\ch{X}}= \sum_0^{N_{*}} \; Y_{\ch{X}}(m_{*})\;m_{*},
%\label{eq:m_x}
%\end{equation}
%where $N_{*}$ is the number of stars exploded with a mass equal to $m_{*}$ and $Y_{\ch{X}}(m_{*})$ is the corresponding yield of a specific element, $\ch{X}$.
%Using eq. \ref{eq:m_x}, the abundance of the X element have been computed as:
%\begin{equation}
%    [\ch{X}/\ch{H}] = \log\bigg(\frac{N_{\ch{X}}}{N_{\ch{H}}}\bigg)- \log\bigg(\frac{N_{\ch{X}}}{N_{\ch{H}}}\bigg)_{\odot}\approx
%    \log\bigg(\frac{M_{\ch{X}}}{M_{gas}}\bigg)- \log\bigg(\frac{M_{\ch{X}}}{M_{\ch{H}}}\bigg)_{\odot},
%    \label{eq:x_h}
%\end{equation}
%where $N_{\ch{X}}$ ($N_{\ch{H}}$) is the number of X element (hydrogen) atoms, $M_{\ch{X}} = N_{\ch{X}}\cdot m_{\ch{X}}$, and $M_{\ch{H}} = N_{\ch{H}} \cdot m_{\ch{H}} \approx M_{gas}$ is the mass of the gas left in the halo after a star formation burst, computed as in Eq. \ref{eq:m_gas}.
%For the Solar values we have used the values of \cite{anders1989abundances}, i.e. $\big(\frac{M_{\ch{Fe}}}{M_{\ch{H}}}\big)_{\odot} = 1.37 \cdot 10^{-3}$, $\big(\frac{M_{\ch{C}}}{M_{\ch{H}}}\big)_{\odot} = 3.07 \cdot 10^{-3}$ and $Z_{\odot} = 0.02$.
%\\[6pt]
\section{Outputs analysis and results}
\label{sec:results}

To understand how much the properties of Pop~III stars can affect the present-day distribution of CEMP stars, we first analyse the $z=0$ outputs of the simulation for two extreme cases: (i) when all Pop~III stars explode as an average PISN of $\rm 200_{\odot}$, or (ii) when they all explode as an average faint SN of $\rm 25_{\odot}$ (see Sec.~\ref{sec:model}). In the first case we find that CEMP stars are never produced by the model, the larger carbon-to-iron value of present-day stars being [C/Fe]$\approx 0.2$ {\bf Please, Giulia, double check this!} This is perfectly consistent with the low [C/Fe] value of an ISM only imprinted by massive $> 160M_{\odot}$ Pop~III stars evolving as PISN \citep[e.g.][left panel of Fig.~2]{salvadori2019probing}.\\ 

Conversely, CEMP stars exist when Pop~III stars are assumed to explode as $25M_{\odot}$ faint SNe and they can have huge C-excess, [C/Fe]$\geq 4$, at the lowest iron abundance values, [Fe/H]$<-4$. This result is in line with the idea that the most iron-poor CEMP(-no) \ag{Do you want to emphasise iron? otherwise in CEMP is also included iron poor: repetition} stars are indeed the descendants of this kind of zero-metallicity SN (Sec.~\ref{sec:intro}). In this hypothesis we can also compute the current distribution of CEMP stars, which is shown in Fig.~\ref{fig:CEMP_distr} in the $\rho - \zeta$ cylindrical coordinate plane. We see that the innermost region is predicted to host the largest mass fraction of CEMP stars with respect to the total. In our extreme assumption of {\it all} Pop~III stars exploding as faint SNe, furthermore, the predicted $\rm F_{CEMP}$ for the Galactic bulge and halo are expected to be very similar.

These results, which are at odd with observations, suggest that to model the early chemical enrichment in the first star forming halos we should consider a more realistic Pop~III IMF, which accounts for the contribution of both faint SNe and PISNe. Furthermore, it seems to advocate towards different physical mechanisms driving the formation of stars currently located in the innermost regions, where faint SN explosions should somehow be less frequent to avoid the formation of many CEMP stars.
\begin{figure}
   \centering
    \includegraphics[width=0.5\textwidth]{images/CEMP_distribution.png}\hfill%
    \caption{Present-day distribution of CEMP stars obtained by assuming that {\it all} Pop~III stars have $\rm m_{popIII}=25M_{\odot}$ and explode as faint SNe.}.
    \label{fig:CEMP_distr}
\end{figure}
%
%\begin{figure}
%   \centering
%    \includegraphics[width=0.45\textwidth]{images/MergerTree.png}\hfill%
%    \caption{Halo merger tree for $z\geq 8.4$. The labels indicate the halo IDs, the circles identify star-forming halos and the stars those hosting Pop~III stars.}
%    \label{Fig:MT}
%\end{figure}

\begin{figure}
   \centering
    \includegraphics[width=0.37\textwidth]{images/maps_particles.jpeg}\hfill%
    \caption{Each panel shows the \textit{current} position of stars formed in a specific halo (identified by the id number and the color in label), at the specified $z$, which decreases from top to bottom. Different Galactic regions are identified concentric circles: the inner halo, $7$ kpc$<r \leq20$ kpc (aquamarine circle), and the bulge, $r\leq2.5$ kpc (light blue circle).}
    \label{fig:halos_z}
\end{figure}
\subsection{The first star forming halos}
%C-enhanced metal-poor stars are among the most ancient Pop II stars that can be observed today, having formed in environments mainly polluted by Pop III stars exploding as faint supernovae. 
In order to identify the first star forming halos and see where the oldest stars are currently located, we reconstructed the hierarchical tree in the MW-analogue by combining the positions of the DM particles within which star formation occurs, and their belonging to a specific halo at different redshifts. This has been done by identifying DM particles in common between two different halos found at consecutive redshifts. If at least 90\% of the particles belonging to the halo at higher redshift ends up in the subsequent one, then we can assume that a merging process has occurred. 

\begin{figure*}
    \centering
    \setlength\fboxsep{0pt}
    \setlength\fboxrule{0.25pt}
    \includegraphics[width=\textwidth]{images/sfr_map.png}\hfill%
    \caption{Current distribution of DM particles in the cylindrical coordinate plane of the simulated galaxy, color-coded with their SFR averaged over the cosmic time between $z=15$ and $z=8$.}
    \label{fig:sfr}
\end{figure*}

Figure \ref{fig:halos_z} shows the \textit{current} position, in the $\rho - \zeta$ cylindrical coordinate plane, of Pop II stars formed at $z \geq 13$, i.e. in the first star-forming halos. Note that the simulation is not exactly centered in the Galactic centre and therefore the bulge is shifted to $\rho \sim 4$ kpc. As we can see from the upper left panel of Fig.~\ref{fig:halos_z}, \textit{the oldest Pop II stars are currently located in the innermost Galactic region, i.e. the bulge}. The position of DM particles belonging to halos that begin forming stars at later redshifts (lower panels of Fig.~\ref{fig:halos_z}) is gradually spread towards the outermost regions, filling first the inner and then the outer Galactic halo. These results confirm the idea that the most ancient stellar populations, some of which should have been formed from the ashes of the first stars, must dwell into the bulge (see Sec.~\ref{sec:intro}). 

Figure \ref{fig:sfr} shows the \textit{current} distribution of DM particles color-coded with their star-formation rate, (SFR), during the first $\approx 800$~Myr, i.e. averaged over the cosmic time between $z=15$ and $z=8$. To compute this quantity we assigned to each DM particle the mean SFR of the halo it belongs to, so that: 
\begin{equation}
\rm \langle SFR\rangle = \frac{\sum_{i=1}^{N}M^{tot}_{*_{i}}}{t(z=8)-t(z=15)},
\label{eq:sfr}
\end{equation}
where $\rm M^{tot}_{*_{i}}$ is the total stellar mass formed in the halo at the \textit{i-th} time-step, and $N$ is the number of time-steps considered between $z=15$ and $z=8$. Then we computed the final $\rm \langle SFR\rangle$ by averaging among the results of all DM particles in the considered pixel. As we can see by inspecting the colorbar in the figure we have obtained a wide range of average star formation rates, $\langle SFR \rangle \approx (10^{-4}-1)\,M_{\odot}/yr$. 
%{\bf Stef: Notare che dipende dal pixel ma quello che sottolineamo è il tremd/gradiente.}\\

As we can see, the spatially resolved region with the highest mean SFR is the innermost one including the galactic bulge. This result, which is in agreement previous findings \citep[e.g.][]{cescutti2011galactic}, suggests that the progenitors of the Galactic bulge experience, on average, a more intense star formation at early times with respect to the progenitors of the Galactic halo.
A higher star formation rate in a metal-poor environment might have strong consequences: indeed, it could allow to produce many rare stellar systems, most likely including the more massive progenitors of PISNe \citep{rossi2021ultra}. 
%However, we must take into account that this is a value averaged over a wide $z$ range and therefore may not be illustrative for the initial star formation epochs when Pop~III stars form (i.e. for $z\approx15$).

\subsection{Pop III stars enrichment}
In this Section we are going to analytically model the Pop~III star enrichment within the first star forming halos of the cosmological simulation by using more realistic Pop III IMFs and accounting for their incomplete sampling. In addition to the delta functions considered so far, we will assume that Pop III stars form according to a \textit{Larson IMF}:
\begin{equation}
    \Phi(m) = \frac{dN}{dm} \propto m^{-2.35} \mathrm{exp}(-m_{ch}/m),
    \end{equation}
 biased towards more massive stars as it is expected for the first stellar generations. In particular, following the latest data-driven results from stellar archaeology \citep{rossi2021ultra}, we assume a minimum mass of Pop~III stars equal to $\rm m_{min}=0.8\,M_{\odot}$, a maximum mass $\rm m_{max}=1000\,M_{\odot}$, and we explore different characteristic masses, $\rm m_{ch} \in \{1, 10, 100 \,M_{\odot}\}$. This allows us to account for the contribution to chemical enrichment of Pop~III stars exploding as both faint SNe and PISNe and to vary their relative proportions. For comparison, we will also consider a Flat IMF with $\rm m_{popIII}=[0.8, 1000]\,M_{\odot}$, where both populations are equally distributed in mass. All the Pop III IMFs considered in this study are shown in Fig. \ref{fig:IMF_popIII}.
\\[2pt]

\begin{figure}
   \centering
    \includegraphics[width=0.48\textwidth]{images/imfs_new.png}\hfill%
    \caption{Different Pop III stars IMFs used in this work.}
    \label{fig:IMF_popIII}
\end{figure}

\begin{figure*}
\centering
\setlength\fboxsep{0pt}
\setlength\fboxrule{0.25pt}
\includegraphics[width=\textwidth]{images/imf_random_1run.png}\hfill%
\caption{Theoretical (solid lines) and effective (coloured shaded histogram) Larson IMFs of Pop III stars with increasing characteristic masses from top to bottom ($m_{ch} = 1,10,100\,M_{\odot}$) obtained from one run of the random sampling. In each panel, the mass range covered by faint SNe (orange shaded region), PISNe (red shaded region), and other stars that do not end their lives as supernovae (grey shaded region), are specified. The mass ratios over the total in the different mass ranges are also listed.}
\label{fig:imfs_1run}
\end{figure*}
To make our analytical calculations we selected the first three star forming halos of the simulation, which have zero-metallicity and whose stars are predicted to dwell in the Galactic bulge at the present time. Two of these objects have a total mass of $\rm M_{halo}\simeq5.5\times10^{7}\,M_{\odot}$ at their first star formation redshift $\rm z=15.3, 14.4$. The third one, at $z=11.9$, is more massive, $\rm M_{halo}\simeq 10^{8}\,M_{\odot}$. The predicted SFR for these first star forming halos is respectively $\rm SFR \sim (10^{-2}, 4\times10^{-3}, 4\times10^{-3}) \,M_{\odot} yr^{-1}$, for first, second, and third halo, which in all cases is less than the threshold value $\rm SFR _{min}\sim 10^{-1} M_{\odot} yr^{-1} $ (\cite{rossi2021ultra}) for a fully populated Pop III IMF. 

For each halo, \textit{we derived the effective Pop~III IMF associated to each burst of primordial star-formation to account for the stochastic and incomplete IMF sampling}. This has been done using a Monte Carlo procedure that, starting from the total mass of Pop III stars formed, enables to generate random numbers of stars, distributed according to the assumed stellar IMF (see \cite{rossi2021ultra} for details). Due to the stochastic nature of this sampling, every time that stars are formed, the effective stellar mass distributions is differently populated, especially at the higher masses (Fig.~\ref{fig:imfs_1run}). Following \cite{rossi2021ultra} we assumed a time-scale for star formation equal to $\Delta t = 1$ Myr and then we computed the total mass of Pop~III stars formed in the burst accordingly, $\rm M^{burst}_{popIII}=SFR\times \Delta t$. Then we applied a statistical approach by averaging among the results of $22$ incomplete random sampling of the IMF and quantifying the scatter among them. In this way we are dividing the mass of the star burst obtained from the simulation (whose timestep is 22 Myr at high redshifts) into 22 sub-regions that form stars in 1 Myr.

%For this reason we have applied a statistical approach by averaging among the results of {$\approx 20$} incomplete random sampling of the IMF and quantifying the scatter among them. The reason why we have chosen to average on 22 random sampling is due to our choice of the star formation $\Delta t = 1$ Myr which is consistent with \cite{rossi2021ultra}. In this way we are dividing the mass of the star burst obtained from the simulation (whose timestep is 22 Myr at high redshifts) into 22 sub-regions that form stars in 1 Myr.

Figure \ref{fig:imfs_1run} shows, for different $m_{ch}$, the comparison between the theoretical Pop~III IMFs and effective ones, which have been obtained in one run of the random sampling. The mass range covered by faint SNe, $\rm (8-40)\,M_{\odot}$, PISNe, $\rm(140-260)\,M_{\odot}$, and other stars that do not end their lives as supernovae are underlined using different colours. Note that the average ratios between the number of stars exploding as faint SNe (PISNe) and the total number of stars in the Pop~III IMF, strongly varies not only with $\rm m_{ch}$ but also as a consequence of the incomplete IMF sampling. While in the mass range of faint SNe the Pop III IMFs are almost completely sampled except for $m_{ch}=100\,M_{\odot}$, in the range of PISNe the Pop III IMF is only partially populated. As expected however, as the characteristic mass increases, the mass range of PISNe becomes more densely populated.
%and their fractions are lower than what is expected for a fully populated IMF: $0.17\%$ instead of $\rm X\%$ for $\rm m_{ch}=1\,M_{\odot}$, $1.14\%$ instead of $\rm X\%$ for $\rm m_{ch}=10\,M_{\odot}$, and $16.02\%$ instead of $\rm X\%$ for $\rm m_{ch}=1\,M_{\odot}$. The two fractions approach each other as the characteristic mass increases, since massive stars are more likely produced and therefore the mass range of PISNe becomes more densely populated.

%\begin{table}
%\caption{Supernovae yields and average SN energy used in the calculation.}
%\label{tab:yields}
%\centering
%\resizebox{\columnwidth}{!}{%
%\begin{tabular}{ccccc}
%\hline
%SN type & stellar mass ($M_{\odot}$) & $Y_{Fe}$ & %$Y_{C}$ & $log_{10}\langle E_{SN} \rangle$ \\
%\hline
%PISN & 140 & $1.6\times10^{-15}$ & $4.9\times10^{-2}$ & 52\\
% & 149 & $2.4\times10^{-4}$ & $3\times10^{-2}$ &52\\
% & 158 & $9.3\times10^{-4}$  &$2.7\times10^{-2}$ &52\\
% & 167 & $1\times10^{-3}$ &$2.6\times10^{-2}$ &52\\
% & 177 & $2.6\times10^{-3}$& $2.4\times10^{-2}$&52\\
% & 195 & $1.6\times10^{-2}$& $2.1\times10^{-2}$&52\\
% & 205 & $2.9\times10^{-2}$&$2\times10^{-2}$ &53\\
% & 214 & $4.6\times10^{-2}$& $1.8\times10^{-2}$&53\\
% & 223 & $6.5\times10^{-2}$&$1.7\times10^{-2}$ &53\\
%& 232 & $8.3\times10^{-2}$&$1.6\times10^{-2}$ &53\\
% & 242 & $0.10$&$1.5\times10^{-2}$ &53\\
% & 251 & $0.13$&$1.4\times10^{-2}$ &53\\
% & 260 & $0.16 $&$1.3\times10^{-2}$&53\\
%\hline
%\end{tabular}
%}
%\end{table}

The mass of metals, iron, and carbon produced by the total number of Pop~III stars effectively formed has been computed by summing up the contribution of Pop~III stars with different masses, i.e. $\rm M_{X} = \sum_{i}^{N} {Y_{X}(m_{popIII,i}){m_{popIII, i}}}$, where $\rm Y_{X}(m_{popIII, i}$ is the yield of the element X produced by Pop~III stars in the $\rm i-th$ mass bins. For faint SNe, we followed \cite{deBen2016limits} and assumed that the yield corresponding to $25\,M_{\odot}$ is simply re-scaled to the mass of the Pop~III star exploding as faint SN, i.e. $\rm Y_X(m_{popIII}) = Y_X(25\,M_{\odot}) \times (m_{popIII}/25\,M_{\odot})$. For PISNe we exploited the yields and SN explosion energies provided by \cite{heger2002nucleosynthetic}. Similarly, we computed the total amount of energy released by SNe by accounting for the number of Pop~III stars effectively formed in different mass bins, $\rm E_{SN} = \epsilon_w \sum_{i}^{N} {N_{SN}(m^{popIII}_i)\times E_{SN}(m^{popIII}_i)}$, as done in \cite{rossi2021ultra}). {\bf Stef: how do you compute $\rm E_{SN}$? Is correct what I wrote?}
Finally, by exploiting Eq.~\ref{eq:m_gas}-\ref{eq:m_Z} we computed the ISM metallicity along with \feh and \cfe after the Pop~III star enrichment. 

In Fig.~\ref{fig:abund_1myr} we show the ISM \feh and \cfe of the three first star-forming halos nowadays dwelling in the Galactic bulge obtained with our incomplete sampling procedure and as a function of the six different Pop~III IMFs assumed (see Fig.~\ref{fig:IMF_popIII}). For comparison, we also show the results obtained considering the fully sampled IMFs. {\bf Stef: clarify tha $Z_{cr}$ has been reached already.}

%The selected halos are identified in the simulation as those forming Pop III stars at the highest redshifts ($z\gtrsim12$); in particular, we have considered two halos with total mass $M_{halo}\simeq5.5\times10^{7}\,M_{\odot}$ and a more massive one with $M_{halo}\simeq1\times10^{8}\,M_{\odot}$.
%
%\subsection{Observational driven constraints}
%The average stellar \feh and \cfe values in the three first star-forming halos are shown respectively as red and orange lines in Fig. \ref{fig:abund_1myr}. These are obtained for the different Pop III IMFs with $m$ in the range $(m_{min}, m_{max}) = [0.8, 1000]\,M_{\odot}$, namely: a delta function centered on a stellar mass of $25 \,M_{\odot}$, Larson type IMFs with increasing $m_{ch}$ ($1, 10, 100 \,M_{\odot}$), a delta function centered on a stellar mass of $200\, M_{\odot}$ and a flat IMF. 
%We also show the results obtained considering the same IMFs but fully sampled, as solid grey lines. T

%The \feh and \cfe observed ranges within the Galactic bulge are also shown for comparison as respectively a red arrow and an orange shaded region. A second red arrow is used to underline the PIGS \feh range while the dark orange shaded area shows the PIGS \cfe range of Group II CEMP stars (see \ref{sec:intro} and \cite{arentsen2021pristine} for the details). 
As we can see, when we only account for the contribution of faint SNe, we obtain a very low value of \feh ($\simeq -5.6$) and an extremely high $\cfe\simeq4$, which is actually near the maximum value observed in the Milky Way halo (\cite{Keller2014}). As soon as the chemical contribution of PISNe is also considered, namely for $m_{ch}=1\,M_{\odot}$, the \cfe value drops dramatically by at least 2 orders of magnitude. Indeed, for the two less massive halos we have $\cfe\simeq+2$, while for the more massive one $\cfe\simeq+0.37$. This result can easily be explained as a consequence of the Pop~III IMF sampling: in the lowest mass halos, which have lower star formation rates, on average only one PISN explodes, thus only partially lowering the \cfe value obtained in the case of faint SNe only. Still, even a single PISN is able to inject into the ISM $50\%$ of its total stellar mass in form of heavy elements equal. In the most massive halo instead more PISNe can actually form and explode, thus lowering more the expected \cfe value. 
{\bf Stef: For increasing characteristic mass...} 
For $m_{ch}=10\,M_{\odot}$, the mean \feh values approach the minimum value observed in the Galactic bulge and the mean \cfe values fall within the range observed in the bulge being around the threshold for carbon enhancement. For $m_{ch}=100\,M_{\odot}$, the average \feh and \cfe values fall within the ranges observed in the bulge. %, and interestingly, the \cfe observed range falls perfectly within the 1$\sigma$ confidence intervals obtained theoretically.
When only pair instability SNe explode, metal-poor stars are no longer expected and \cfe values turn out to be sub-solar. For all the IMFs described so far we can see that the more massive the halo, the more the abundances obtained are close to the case in which the same IMFs are fully populated. The matter changes when considering a flat IMF since, in this case, all the exploding stars have the same weight in the chemical enrichment. 

\subsection{Observational driven constraints}
By comparing our results with the observed abundance ranges we can affirm that massive Pop III stars exploding as PISNe must have existed at least in the Galactic bulge, where they washed out the high \cfe values due to faint SNe.
We can definitely exclude the two extreme cases of IMFs where Pop III stars explode either only as faint SNe or only as PISNe. The IMF that best reproduces the observations is the Larson type with $m_{ch}=10 \,M_{\odot}$ since it explains the large number of new CEMP PIGS candidates with \cfe around the $0.7$ threshold. Furthermore, the observed range of \cfe falls almost entirely within the $1\sigma$ error so even the most extreme values (including the maximum $\cfe\sim2$) would be consistent with our prediction. Also our mean value of \feh, being around the observed minimum, is a promising one if we think that our prediction is right for $z>6$ when all subsequent generations of stars exploded as Type II and/or Type I SNe had not increased the \feh yet.
Despite this, we can not for now exclude neither the $m_{ch}=1-100\,M_{\odot}$ case or a flat IMF.

\begin{figure*}
\centering
\setlength\fboxsep{0pt}
\setlength\fboxrule{0.25pt}
\includegraphics[width=\textwidth]{images/ABUND_RAND_NEW_1MYR.png}\hfill%
\caption{Average stellar \feh (red) and \cfe (orange) values in three first star-forming halos, obtained through the random sampling of different Pop III IMFs. The characteristic masses ($m_{ch}$) are in solar masses ($M_{\odot}$) and error bars are 1$\sigma$ confidence intervals. The values are computed considering a $\Delta t = 1$ Myr of star formation which is consistent with \cite{rossi2021ultra}. For comparison, the grey solid lines are related to the fully sampled IMFs. The \feh observed range within the bulge is shown as a red arrow; an other one is used to underline the PIGS (\cite{arentsen2021pristine}) range of \feh. The \cfe observed range within the bulge is shown as an orange shaded area. The dark orange shaded area shows the \cfe range of Group II CEMP stars in \cite{arentsen2021pristine}.}
\label{fig:abund_1myr}
\end{figure*}


\section{Conclusions}
\label{sec:concl}
The aim of this paper was to investigate the apparent dearth of carbon-enhanced metal-poor (CEMP) stars with high \cfe in the Galactic bulge. 
Could it be due to the low statistics of metal-poor stars in the more metal-rich and dusty bulge? Could it suggest an intrinsic different formation mechanism of this region compared to the other environments? 
In order to answer these questions, we carried out a diversified analysis consisting of three main steps: we started by examining large samples of observational data, then moved on to the use of an N-body simulation, and finally carried out analytical computations. Each of these steps - until the last one - seemed not to solve the problem, and this is why we went ahead with our investigation.
\\[3pt]

We first performed a statistical analysis of the metal-poor stars observed in the Local Group to understand whether the dearth of CEMP stars is due to the limited sample size of such stars within the bulge. To this end we first derived the fraction of CEMP stars at a given \feh for the Galactic halo ($F^{halo}_{\mathrm{CEMP}}$), since it is the environment with the highest statistic on metal-poor stars at $\feh<-2$. Then we assumed this fraction to be ``Universal'', and combined it with the Metallicity Distribution Functions (MDFs) obtained for dwarf galaxies and the bulge to compute the probability of observing a CEMP star in a given \feh range.
Dwarf galaxies and the Galactic bulge are indeed characterized by different luminosities, that range from $<10^4\,L_{\odot}$ for UFDs to $\approx10^{10}\,L_{\odot}$ for the bulge, and different average metallicities, ranging from $\langle \feh \rangle \lesssim -2.2$ (UFDs) to $\langle \feh \rangle \sim 0.0$ (bulge). The data used for the halo and the bulge are taken from a database for metal-poor literature stars called \texttt{JINABASE} and, for the bulge, data from the large multi-object APOGEE spectroscopic survey (Data Release 16) have been also used; for dwarf galaxies we used data from the literature.
\\[3pt]
We then asked whether the fraction of CEMP stars in the Galactic bulge is intrinsically different from the other environments.
To answer this question, we focused on the predictions derived from the $\Lambda$CDM cosmological model, through the use of a N-body simulation that follows the hierarchical formation of a MW-like galaxy (\cite{scannapieco2006spatial}), combined with the semi-analytical model \texttt{GAMETE} (\cite{salvadori2007cosmic}, \cite{salvadori2010mining}, \cite{salvadori15}), required to follow the star formation and metal enrichment history of the Milky Way. The model is data-calibrated, i.e. the best values of the free parameters are set to reproduce the observed properties of the Milky Way at $z=0$.
The N-body simulation gives us information about the positions of the dark matter particles constituting the DM halos, while thanks to the semi-analytical model we are able to follow the evolution of the baryonic component, i.e. gas, stars and metals. %Using the outputs of this simulation, we derived the predicted fraction of CEMP stars in the bulge, inner and outer halo. 
\\[3pt]

To understand if the bulge is really representative of the oldest stellar population, we studied the current position of the oldest stars within the simulated MW-like galaxy. After having identified the first star-forming halos that end up in the bulge region, we followed their chemical evolution and analyzed how the chemical enrichment of their ISM changes by considering different Initial Mass Functions of Pop III stars. To this end we performed an analytical calculation of \feh and \cfe for three of the first star-forming halos using Pop III IMFs with $m$ in the range $(m_{min}, m_{max}) = [0.8, 1000]\,M_{\odot}$, namely: a delta function centered on a stellar mass of $25 M_{\odot}$, Larson type IMFs with increasing $m_{ch}$ ($1, 10, 100 \,M_{\odot}$), a delta function centered on a stellar mass of $200 M_{\odot}$ and a flat IMF. The abundances of \feh and \cfe have been computed after having derived the effective Pop III IMFs carried out through a stochastic Monte Carlo procedure developed by \cite{rossi2021ultra} that generates random numbers of stars distributed according to the theoretical stellar IMF. 
Depending on the Pop III IMF, we can obtain different implications in the enrichment of the ISM that affect the possibility of finding CEMP stars in the Galactic bulge. 

In the following paragraph we summarize the main results obtained thanks to this comprehensive study.

\subsection{Key results \& implications}
From the statistical analysis of the observations in the different environments, we have understood that, assuming an ``Universal'' fraction of CEMP stars, the probability of observing such a star depends basically on the luminosity and metallicity of the environment. In fact, for the bulge region - being the brightest and the most metal-rich among those considered - this probability turns out to be two orders of magnitude lower with respect to ultra-faint dwarf galaxies and also shifted at higher \feh values. \textit{CEMP stars}, given the scarcity of low-\feh observations \textit{in the Galactic bulge, could therefore be hidden} in this region dominated by metal-rich stars, rather than totally absent. 
%However, if this were the case, \textit{$\mathit{\approx 6-7}$ CEMP stars should already be observed in this region, while only one has been observed so far}.
\\[3pt]
To understand if the hypothesis of an ``Universal'' CEMP fraction might be correct for the bulge, we exploited the theoretical prediction of our model which allowed us to study in which region of the simulated MW-like galaxy the oldest stars should now reside. We found that \textit{the oldest second generations of stars are currently located in the Galactic bulge}. At high redshifts, \textit{this region} also resulted as the one\textit{ with the highest predicted mean star-formation rate, having $\mathit{\langle SFR\rangle = 0.9\, M_{\odot}\,yr^{-1}}$}. From these results we have therefore understood that the halos that currently end up in the bulge are also predicted to be the first to form stars. For this reason, we focused on the first star-forming halos and studied their chemical evolution from an analytical point of view.
Within these halos, the SFRs corresponding to the initial bursts of Pop III stars are $\sim 10^{-2}\,M_{\odot}\,yr^{-1}$, i.e. lower than the minimum value needed to fully populate the IMF (\cite{rossi2021ultra}). Therefore, we have  calculated the mean \feh and \cfe abundances resulting from an incomplete sampling of Pop III IMFs. 
By comparing our results with the observed abundance ranges we can affirm that massive Pop III stars exploding as pair instability supernovae must have existed at least in the Galactic bulge, where they washed out the high \cfe values due to faint SNe.
We can also definitely exclude the two extreme cases of IMFs where Pop III stars explode either only as faint SNe or only as PISNe. The IMF that reproduces better the observations is the Larson with $m_{ch}=10 \,M_{\odot}$ even if we can not until now exclude $m_{ch}=1-100\,M_{\odot}$ or a flat IMF. We can therefore conclude that \textit{in the halos that end up in the bulge}, being more star-forming than the UFDs that probably build the halo, \textit{it is possible to form massive Pop III stars}; these, exploding as PISNe, have thus been \textit{able to partially wash out the high \cfe} values caused by faint SNe, \textit{resulting in the dearth of carbon-enhanced metal-poor stars with such high \cfe values} as the ones we actually observed in the Galactic halo. 

What implications can we draw from our analysis? 
First of all, the most obvious one is that massive stars exploding as \textit{PISNe should have existed}.
Furthermore, \textit{the \feh and \cfe ranges of stars observed in the Galactic bulge} can be an \textit{useful tool to give an estimate of the characteristic mass of the Pop III IMFs}.
%This implications for the Pop III IMF, which translate into $m_{max}=1000\,M_{\odot}$ and $m_{ch}\approx1.2\,M_{\odot}$, turned out to be consistent with \cite{rossi2021ultra}, where complementary data have been exploited. 
%Finally, it is noteworthy that, even considering only two halos, the analytical value of \feh corresponding to $\cfe\approx 0.98$, being $\approx -3.8$, is not very far from that of the bulge CEMP star which is $\approx-3.48 \pm 0.1$. This implies that using the dearth of CEMP stars in the Galactic bulge may actually be the right path to put constraints on the Pop III Initial Mass Function.

\subsection{Limitations \& future plans}
The semi-analytical model used to describe the chemical evolution of the MW-like galaxy is based on the Instantaneous Recycling Approximation (IRA). This approximation implies that the gas and metals produced by stars that do not survive to date, are instantaneously and homogeneously mixed within the ISM.
Although this assumption is suitable for studying older stellar components that formed at $z>6$, it is unable to account for an actually gradual enrichment. 
Furthermore, since we are interested in CEMP-no stars, in order to maintain the IRA but not to overestimate the carbon contribution to the enrichment due to AGB stars leading to the formation of CEMP-s stars, we have not considered the contribution to the metal enrichment by stars with $1\,M_{\odot}<M_{*}<7\,M_{\odot}$ that experience an AGB phase in their stellar evolution. We also neglected the contribution of type Ia SN, which typically evolve on long time-scale ($\sim 1$ Gyr) and produce large amount of Fe, enriching the ISM up to $\feh > 0$. This approximation allowed us to correctly describe the early stages of galaxy formation but probably led us to underestimate the \feh for $z < 6$.
In the future, to simulate a more realistic enrichment, we will relax the Instantaneous Recycling Approximation (Koutsouridou et al. in prep.) and then also consider AGB stars and type Ia SNe for completeness.
\\[3pt]

%Another assumption made in the semi-analytical model is that star formation is only triggered in halos with masses corresponding to $T_{vir}>10^4$ K and that it is therefore quenched in mini-haloes that have $T_{vir}<10^4$ K. This assumption does not allow to reproduce the properties of ultra-faint dwarf galaxies since these are predicted to be the living descendants of mini-haloes (e.g. \cite{salvadori15}), but does not affect the conclusions made for our region of interest, i.e. the Galactic bulge, since our analysis shows that this is formed through the merging of the most massive halos at each $z$, corresponding to $\geq3\sigma$ fluctuations of the density peak.
Furthermore, in our analysis we have focused only on C and Fe, but measurements of the abundances of other elements are also available for metal-poor stars in the Galactic bulge. Following the chemical evolution of these elements would give further constraints on Pop III Initial Mass Function.
\\[3pt]

We have seen how this is the perfect time to deepen our studies concerning the Stellar Archaeology within the Galactic bulge and to investigate further its intrinsic and observational properties. The future of the field is indeed promising since great technology developments and new instruments will provide us unprecedented data sets to confirm our theoretical findings.
In the coming years, large samples of metal-poor inner Galaxy stars, such as provided through Pristine Inner Galaxy Survey (PIGS, \cite{arentsen2020pristine}) and other future surveys such as the multi-object spectroscopic 4MOST Milky Way Disc and bulge Low-Resolution Survey (4MIDABLE-LR, \cite{chiappini20194most}) and 4MOST MIlky Way Disc And bulge High-Resolution survey for brighter stars (4MIDABLE-HR,\cite{bensby20194most}), will provide a unique view of the central regions of the Milky Way. 

In addition, we know that it is possible to prove that very massive first stars exploding as Pair Instability Supernovae existed by searching for an under-abundance of [Zn/Fe] in their descendant (\cite{barbuy2015zinc}, \cite{duffau2017gaia}, \cite{salvadori2019probing}). The WEAVE@WHT Galactic Archaelogy survey (\cite{dalton2016weave}) will be soon operational and will be able to provide the chemical abundances needed to find candidates of PISNe descendants.

Incorporating all the analytical studies carried out in this paper into the N-body simulation, will allow us to make more quantitative predictions for future observations. Through the construction of theoretical colour-magnitude diagrams (CMDs) it will be possible to know how bright the simulated long-lived stars are. In this way, we will be able to predict how many second-generation stars we expect to realistically observe at a given magnitude with current facilities, and we will also provide a testable contribution to future surveys of metal-poor stars in the Galactic bulge.

The implications on the Pop III IMF from the CEMP stars in the bulge can also be tested "globally" with observations of the halo and dwarf galaxies. Indeed, the low-Fe tail of the halo is expected to be predominantly due to low-mass ultra-faint dwarf galaxies, which have low star-formation rates (e.g. \cite{bonifacio}). So what do we expect for the Pop III IMF? Could it be consistent with what we have found? 
To investigate the impact of the Pop III IMF on the properties of ultra-faint dwarf galaxies, we should implement our routines in the semi-analytical model and use a high resolution N-body simulation (e.g. Caterpillar project, \cite{griffen2016caterpillar}) able to resolve even mini-halos, which are predicted to be the progenitors of UFDs (e.g. \cite{salvadori15}, \cite{rossi2021ultra}). Within the model we should also include the inhomogeneous reionization that can suppress star formation in these objects (\cite{graziani2015galaxy}).

\section*{Acknowledgements}

NEFERTITI, ask Paola about GEPI etc. 
The Acknowledgements section is not numbered. Here you can thank helpful
colleagues, acknowledge funding agencies, telescopes and facilities used etc.
Try to keep it short.

%%%%%%%%%%%%%%%%%%%%%%%%%%%%%%%%%%%%%%%%%%%%%%%%%%
\section*{Data Availability}

 
The inclusion of a Data Availability Statement is a requirement for articles published in MNRAS. Data Availability Statements provide a standardised format for readers to understand the availability of data underlying the research results described in the article. The statement may refer to original data generated in the course of the study or to third-party data analysed in the article. The statement should describe and provide means of access, where possible, by linking to the data or providing the required accession numbers for the relevant databases or DOIs.




%%%%%%%%%%%%%%%%%%%% REFERENCES %%%%%%%%%%%%%%%%%%

% The best way to enter references is to use BibTeX:

\bibliographystyle{mnras}
\bibliography{example} % if your bibtex file is called example.bib


% Alternatively you could enter them by hand, like this:
% This method is tedious and prone to error if you have lots of references
%\begin{thebibliography}{99}
%\bibitem[\protect\citeauthoryear{Author}{2012}]{Author2012}
%Author A.~N., 2013, Journal of Improbable Astronomy, 1, 1
%\bibitem[\protect\citeauthoryear{Others}{2013}]{Others2013}
%Others S., 2012, Journal of Interesting Stuff, 17, 198
%\end{thebibliography}

%%%%%%%%%%%%%%%%%%%%%%%%%%%%%%%%%%%%%%%%%%%%%%%%%%

%%%%%%%%%%%%%%%%% APPENDICES %%%%%%%%%%%%%%%%%%%%%

%\appendix

%\section{Some extra material}

%If you want to present additional material which would interrupt the flow of the main paper,
%it can be placed in an Appendix which appears after the list of references.

%%%%%%%%%%%%%%%%%%%%%%%%%%%%%%%%%%%%%%%%%%%%%%%%%%


% Don't change these lines
\bsp	% typesetting comment
\label{lastpage}
\end{document}

% End of mnras_template.tex
