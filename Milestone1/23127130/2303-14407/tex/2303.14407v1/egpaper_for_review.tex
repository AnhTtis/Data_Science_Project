\documentclass[10pt,twocolumn,letterpaper]{article}

\usepackage{iccv}
\usepackage{times}
\usepackage{epsfig}
\usepackage{graphicx}
\usepackage{amsmath}
\usepackage{amssymb}

\usepackage{booktabs}
\usepackage{multirow}
\usepackage{float}
\usepackage{subfig}
% Include other packages here, before hyperref.

% If you comment hyperref and then uncomment it, you should delete
% egpaper.aux before re-running latex.  (Or just hit 'q' on the first latex
% run, let it finish, and you should be clear).
\usepackage[pagebackref=true,breaklinks=true,letterpaper=true,colorlinks,bookmarks=false]{hyperref}

 \iccvfinalcopy % *** Uncomment this line for the final submission


\def\httilde{\mbox{\tt\raisebox{-.5ex}{\symbol{126}}}}

% Pages are numbered in submission mode, and unnumbered in camera-ready
\ificcvfinal\pagestyle{empty}\fi


\newcommand{\hb}[1]{{\color{cyan}            {#1}}}
\newcommand{\hbc}[1]{{\color{cyan}            {[HB: #1]}}}

\usepackage{color}
\definecolor{applegreen}{rgb}{0.55, 0.71, 0.0}
\newcommand{\onethousand}[1]{{\color{applegreen}#1}}
\newcommand{\onethousandc}[1]{{\color{magenta}   {[onethousand: #1]}}}

\begin{document}

%%%%%%%%% TITLE
\title{LPFF: A Portrait Dataset for Face Generators Across Large Poses}

\author{Yiqian Wu$^{1,2}$\qquad Jing Zhang$^{1,2}$\qquad Hongbo Fu$^{3}$\qquad Xiaogang Jin$^{1,2}$\thanks{Corresponding author.} \\
$^1$State Key Lab of CAD\&CG, Zhejiang University \\
$^2$ZJU-Tencent Game and Intelligent Graphics Innovation Technology Joint Lab\\
$^3$City University of Hong Kong \\
{\tt\small onethousand@zju.edu.cn,jing\_z99@163.com,hongbofu@cityu.edu.hk,jin@cad.zju.edu.cn}
% For a paper whose authors are all at the same institution,
% omit the following lines up until the closing ``}''.
% Additional authors and addresses can be added with ``\and'',
% just like the second author.
% To save space, use either the email address or home page, not both
}

% \author{Yiqian Wu$^{1,2}$\qquad Jing Zhang$^{1,2}$\qquad Hongbo Fu$^{3}$\qquad Xiaogang Jin$^{1,2}$\thanks{Corresponding author.} \\
% $^1$State Key Lab of CAD\&CG, Zhejiang University \\
% $^2$ZJU-Tencent Game and Intelligent Graphics Innovation Technology Joint Lab\\
% $^3$School of Creative Media, City Univ. of Hong Kong \\

% {\tt\small onethousand@zju.edu.cn,jing\_z99@163.com,hongbofu@cityu.edu.hk,jin@cad.zju.edu.cn}

% }

\maketitle
% Remove page # from the first page of camera-ready.
\ificcvfinal\thispagestyle{empty}\fi


%%%%%%%%% ABSTRACT
\begin{abstract}
The creation of 2D realistic facial images and 3D face shapes using generative networks has been a hot topic in recent years. Existing face generators exhibit exceptional performance on faces in small to medium poses (with respect to frontal faces) but struggle to produce realistic results for large poses. The distorted rendering results on large poses in 3D-aware generators further show that the generated 3D face shapes are far from the distribution of 3D faces in reality. We find that the above issues are caused by the training dataset's pose imbalance.

In this paper, we present \textit{LPFF}, a large-pose Flickr face dataset comprised of 19,590 high-quality real large-pose portrait images. We utilize our dataset to train a 2D face generator that can process large-pose face images, as well as a 3D-aware generator that can generate realistic human face geometry. To better validate our pose-conditional 3D-aware generators, we develop a new FID measure to evaluate the 3D-level performance. Through this novel FID measure and other experiments, we show that \textit{LPFF} can help 2D face generators extend their latent space and better manipulate the large-pose data, and help 3D-aware face generators achieve better view consistency and more realistic 3D reconstruction results.
\end{abstract}


\begin{figure}[h]
    	\centering
    	{\includegraphics[width=0.95\columnwidth]{teaser2.pdf}}
    	\caption{ Image and shape samples generated by EG3D models \cite{Chan_2022_CVPR} trained with the same training strategy but using different datasets (our new dataset \textit{LPFF} and \textit{FFHQ} for (a) and \textit{FFHQ} for (b)). 
      %
      The generators are conditioned by the average camera parameters. 
      %
      Shapes are iso-surfaces extracted from the corresponding density fields using marching cubes. Our dataset helps reduce distorted, ``seam'', ``wall-mounted'', and blurry artifacts exhibited in (b).}
      \vspace{-5pt}
    	\label{fig:teaser}
    \end{figure}




%%%%%%%%% BODY TEXT
%%%%%%%%% BODY TEXT
\section{Introduction}
    \label{sec:intro}
    %2D GANs 
    Since the first introduction by Goodfellow in 2014,
    generative adversarial networks (GANs) \cite{DBLP:conf/nips/GoodfellowPMXWOCB14} have significantly advanced the performance of 2D high-resolution image generation. 
    GANs can accomplish a variety of downstream image editing tasks, particularly face modification \cite{DBLP:journals/tog/AbdalZMW21,nerffaceediting,DBLP:journals/corr/abs-2205-15517,DBLP:conf/cvpr/SunWZLZLW22}, thanks to the excellent image quality and semantic features in its latent space. 
    %3D GANs
    Recently, plenty of 3D-aware generators \cite{DBLP:conf/iclr/GuL0T22,DBLP:journals/corr/abs-2112-11427,Chan_2022_CVPR,DBLP:journals/corr/abs-2110-09788,DBLP:journals/corr/abs-2206-07695,DBLP:conf/cvpr/DengYX022,DBLP:journals/corr/abs-2206-10535,DBLP:journals/corr/abs-2301-09091} have been proposed to learn 3D-consistent face portrait generation from 2D image datasets. 
    % StyleNeRF, StyleSDF, eg3d, CIPS-3D, VoxGraf, Gram, EpiGRAF, BallGAN
    3D-aware generators can describe and represent geometry in their latent space while rendering objects from different camera perspectives using volumetric rendering. Researchers carefully designed generator architectures and training strategies to accelerate training, reduce memory overheads, and increase rendering resolution.
     
    Both the existing 2D and 3D approaches, however, are unable to process large-pose face data.
     %
    Regarding 2D face generators, those large-pose data are actually outside of their latent space, which prevents them from generating reasonable large-pose data, thus causing at least two problems. 
    First, as shown in Fig.~\ref{fig:2d_gans_steep_yaw-1} (left), moving the latent code along the yaw pose editing direction will cause it to reach the edge of the latent space before faces become profile.
    Second, as shown by the results of image inversion
    in Fig.~\ref{fig:2d_gans_steep_yaw-1} (right), it is challenging to project large-pose images to the latent space, let alone perform semantic modification on them.
     %
    One of the goals of 3D-aware generators is to model realistic human face geometry, but existing 3D-aware generators trained on 2D image datasets still have difficulty producing realistic geometry. This issue is more serious when rendering the results at extreme poses.
    As shown in Fig.~\ref{fig:ed_aware_gans_steep_yaw-1}, faces synthesized by those methods have noticeable artifacts, including distortion, blurring, and stratification. 
     %
    In Fig. \ref{fig:teaser} (b), EG3D shows a ``wall-mounted'' and distorted 3D representation without ears. All these indicate that the generated face shapes are not realistic enough.
    

    The above issues in the face generators are mainly caused by the unbalanced camera pose distribution of the {narrow-range} training dataset.
    \textbf{F}lickr-\textbf{F}aces-\textbf{HQ} Dataset (\textbf{FFHQ}) is a popular high-quality face dataset used to train those face generators, but it mainly contains images limited to small to medium poses.
    As a result, 2D and 3D-aware generators cannot learn a correct large-pose face
    distribution without sufficient large-pose data.
    To avoid artifacts under large poses, downstream applications \cite{DBLP:journals/corr/abs-2205-15517,DBLP:journals/corr/abs-2203-13441,DBLP:conf/wacv/KoCCRK23,xie2022high,nerffaceediting,DBLP:conf/siggrapha/JinRKBC22,DBLP:journals/corr/abs-2301-02700,DBLP:journals/corr/abs-2211-16927} based on those face generators typically sample small poses, which limits their application scenarios. 
    
    It is difficult to get a pose-balanced dataset. 
    %
    First, large-pose faces are nearly impossible to detect using Dlib \cite{DBLP:conf/cvpr/KazemiS14}, a popular face detector, and the one used to crop \textit{FFHQ}.
    %
    Second, simply replicating extremely limited large-pose data to balance the pose distribution is insufficient to help extend the camera distribution.
    %
    As a result, it is critical to collect a large number of large-pose, {in-the-wild}, and high-resolution face images, which are lacking in existing datasets.
    
     \begin{figure}[t]
          \centering
          \includegraphics[width=.85\columnwidth]{2d-gans.pdf}
          \caption{
          StyleGAN2 \cite{DBLP:conf/cvpr/KarrasLAHLA20}'s large-pose performance when trained on \textit{FFHQ}. InterfaceGAN \cite{DBLP:conf/cvpr/ShenGTZ20} is used to edit the yaw angle of randomly sampled latent codes. 
          {We use optimization-based GAN inversion to obtain the latent codes of target large-pose real images.}
          }
          \vspace{-5pt}
          \label{fig:2d_gans_steep_yaw-1}
        \end{figure}   
     
     
     %
    In this paper, we propose a novel high-quality face dataset containing \textbf{19,590} real large-pose face images, named \textbf{L}arge-\textbf{P}ose-\textbf{F}lickr-\textbf{F}aces Dataset (\textbf{LPFF}), as a supplement to \textit{FFHQ}, in order to extend the camera pose distribution of \textit{FFHQ} and train 2D and 3D-aware face generators that are free of the aforementioned problems. 
    %
    Given the difficulty of large-pose face detection and the imbalanced distribution of camera poses in real-life photographs, we design a face detection and alignment pipeline that is better suited to large-pose images.
    %
    Our method can also gather large amounts of large-pose data based on pose density. We retrain StyleGAN2-ada \cite{DBLP:conf/nips/KarrasAHLLA20} to demonstrate how our dataset can assist 2D face generators in generating and editing large-pose faces. We retrain EG3D \cite{Chan_2022_CVPR} as an example to demonstrate how our dataset can aid 3D face generators in understanding realistic face geometry and appearance across a wide range of camera poses. 
    {
    In order to better evaluate the 3D-level performance of EG3D models trained on different datasets, we propose a new FID measure for pose-conditional 3D-aware generators.
    }
    Extensive experiments show that our dataset leads to realistic large-pose face generation and manipulation in the 2D generator. Furthermore, our dataset results in more realistic face geometry generation in the 3D-aware generator.

    
    Our paper makes the following major contributions: 
     1) {A novel data processing and filtering method that can collect large-pose face data from the Flickr website according to camera pose distribution, leading to a novel face dataset that contains 19,590 high-quality real large-pose face images.}
     2) A retrained 2D face generator that can process large-pose face images.
     3) A retrained 3D-aware generator that can generate realistic human face geometry.
     4) A new FID measure for pose-conditional 3D-aware generators.



    

     
       \begin{figure}[t] 
          \centering
          \includegraphics[width=.85\columnwidth]{3d-aware-gans.pdf}
          \caption{
          3D-aware generators trained on \textit{FFHQ} (StyleNeRF \cite{DBLP:conf/iclr/GuL0T22}, StyleSDF \cite{DBLP:journals/corr/abs-2112-11427}, EG3D \cite{Chan_2022_CVPR}, and IDE-3D \cite{DBLP:journals/corr/abs-2205-15517}) achieve excellent image synthesis performance on faces in small to medium 
           poses (Top),  but exhibit obvious artifacts at steep angles (Bottom).} 
          \vspace{-5pt}
          \label{fig:ed_aware_gans_steep_yaw-1}
        \end{figure}
        
    
\section{Related Work}
\label{sec:related_work}
\subsection{Co-Speech Gesture Synthesis}
The early approaches for generating co-speech gestures often involve creating linguistic rules to translate speech input into a sequence of pre-collected gesture segments, which are typically referred to as rule-based methods \cite{cassell1994rulefullbody,cassell2001beat,kipp2004gesture,kopp2006bml}. \citet{wagner2014rulereview} provide a comprehensive review of these methods. Rule-based methods produce interpretable and controllable results, but creating gesture datasets and rules requires significant effort. To alleviate the manual effort of designing rules in rule-based methods, data-driven approaches have gradually become predominant in this field. \citet{nyatsanga2023data_driven_gesture_survey} offer a thorough survey of these methods. Early data-driven approaches aim to directly learn mapping rules from data through statistical models \cite{neff2008videogesture,levine2009prosodygesture,levine2010gesturecontroller} and combine them with predefined gesture units for gesture generation. Later, the powerful modeling capability of deep neural networks makes it possible to train complex end-to-end models using raw speech-gesture data directly. One option is deterministic models, such as MLP \cite{kucherenko2020gesticulator}, CNN \cite{habibie2021videogesture}, RNN \cite{yoon2019robot,yoon2020trimodalgesture,bhattacharya2021affectivegesture,liu2022hierarchicalgesture}, and Transformer \cite{bhattacharya2021text2gestures}. Another choice is generative models, including flow-based models \cite{alexanderson2020stylegesture,ye2022styleflowgesture}, VAEs \cite{li2021audio2gesture,ghorbani2022zeroeggs}, and VQ-VAE \cite{yi2022talkshow,yazdian2022gesture2vec,liu2022vqgesturevideo}. Due to the inherent many-to-many relationship between speech and gesture, end-to-end models can generate natural-looking gestures but face challenges in ensuring content matching between speech and generated gestures \cite{yoon2022genea}. To address this issue, some neural systems aim to explicitly model both rhythm and semantics from the perspective of model structure \cite{kucherenko2021speech2properties2gestures,ao2022rhythmicgesticulator,liu2022disco} or training supervision strategy \cite{liang2022seeg}. Furthermore, hybrid systems, such as the combination of deep features and motion graphs \cite{zhou2022gesturemaster}, have been proposed to harness the advantages of different approaches. Recently, diffusion models \cite{sohldickstein2015diffusion,song2020improvedscore,ho2020ddpm} have demonstrated impressive results in image synthesis \cite{ramesh2022dalle2} and human motion generation \cite{tevet2022humanmotiondiffusion, zhang2022motiondiffuse}. Inspired by these works, our system adapts the latent diffusion model \cite{rombach2022latentdiffusion} for the co-speech gesture generation task and achieves appealing results.

\subsection{Style Control for Human Motion}
A typical approach to style control for human motion involves specifying a motion clip as a reference and transferring the reference clip's style to the source motion. This task is also known as \emph{style transfer}. Early works in motion style transfer integrate traditional machine learning techniques with manually defined features to infer motion styles \cite{hsu2005motion_style_translation,ma2010motion_style_transfer,xia2015realtime_motion_style_transfer,yumer2016spectral_motion_style_transfer}. Recently, deep learning-based methods have significantly enhanced motion quality. \citet{holden2016deepmotion} first propose a learning framework enabling motion style control through optimization in the motion manifold space. \citet{du2019stylemotioncvae} improve transfer efficiency by training a conditional VAE. \citet{mason2018few-shot_motion_style_transfer} use few-shot learning to generate stylized locomotion. \citet{aberman2020adain} employ a temporally invariant adaptive instance normalization (AdaIN) layer for target style injection, eliminating the need for paired data during training. \citet{wen2021stylemotionflow} achieve unsupervised style transfer using a flow model. \citet{jang2022motionpuzzle} introduce a method capable of controlling styles for individual body parts.

Previous co-speech gesture synthesis systems with style control can be categorized based on whether or not they require style labels. For methods needing labeled data, early works can only learn an individual style for one generator \cite{levine2010gesturecontroller,neff2008videogesture,ginosar2019stylegesture}. \citet{ahuja2022lowresource} propose a strategy that efficiently adapts the source generator to another speaker style using low-resource data. Some works learn a speaker style embedding space with labeled speaker-motion data, enabling gesture style control by sampling from this space \cite{ahuja2020stylegesture,yoon2020trimodalgesture,bhattacharya2021affectivegesture}. \citet{alexanderson2020stylegesture} aimat controlling fine-grained styles, such as gesturing speed and spatial scope, using preprocessed control signal-motion data. Their later work \cite{alexanderson2022diffusiongesture} utilizes a diffusion model for audio-driven motion synthesis, achieving label-based style control by training the model on labeled data. For methods not requiring style labels, \citet{habibie2022motionmatching} propose a motion matching framework to achieve flexible style control. Other studies achieve arbitrary style control by imitating an example given as a video \cite{liu2022hierarchicalgesture} or a motion clip \cite{ghorbani2022zeroeggs,ye2022styleflowgesture,kuriyama2022tokenizedgestures}.  In this work, we utilize a CLIP-based encoder to extract a style embedding from an arbitrary text prompt and incorporate it into the generator via an AdaIN layer, guiding the synthesis of stylized gestures. Our system supports fine-grained multimodal style prompts as opposed to label-based style control. It employs a self-supervised learning scheme and eliminates the need for labeled data. Additionally, we use an autoregressive model rather than a parallel model, making it potentially suitable for real-time applications.


 
        
\section{Data Preparation}
\label{sec:data_preparation}
In this section, we will introduce how to build our
large-pose face dataset. First, we describe the process for extracting data density from \textit{FFHQ} (Sec.~\ref{subsec:Camera Parameter}). Then, we introduce a novel data processing pipeline that can produce more reasonable realigned results (Sec.~\ref{subsec:Data Processing}). 
In order to filter large-pose face data from the Flickr images according to camera distribution, we propose to employ the pose density function to collect only large-pose data (Sec.~\ref{subsec:Large Pose Data Selection}). Finally, we introduce a novel rebalancing strategy (Sec.~\ref{subsec:Data Rebalance}). 

\subsection{Camera Distribution}
\label{subsec:Camera Parameter}
   {EG3D uses a face reconstruction model \cite{DBLP:conf/cvpr/DengYX0JT19}, denoted as $\mathcal{F}$ in this paper, to extract camera parameters.
   All cameras are assumed to be positioned on a spherical surface with a radius $r=2.7$, and the camera intrinsics are fixed.}
   In this paper, we only consider the camera location and ignore the roll angle of the camera to compute the camera distribution {(detailed in the supplementary file)}. 
   We convert the coordinates of each camera in \textit{FFHQ} from Cartesian coordinates to spherical coordinates and get their $\theta$ and $\phi$ (see Fig. \ref{fig:data_distribution} (a)). Notice that the face with $\theta = 90^\circ$ and $\phi = 90^\circ$ is frontal.

   
    
 
    
\subsection{Data Processing}
\label{subsec:Data Processing}
    {Given the difficulty of large-pose face detection and the imbalanced distribution of camera poses in real-life photographs, we propose a novel mechanism to collect, process, and filter large-pose data.}
    We first collect \textbf{155,720} raw portrait images from Flickr\footnote{\hyperref[]{https://www.flickr.com}} (with permission to copy, modify, distribute, and perform). 
    %
    Then we remove all the raw images that already appeared in \textit{FFHQ}.

    Our pipeline is based on that of EG3D, and we respectively align each raw image according to the image align function in EG3D and StyleGAN.
    %
    In EG3D, the authors first predict the 68 face landmarks of a raw image by Dlib, 
    and then get a realigned image by using the eyes and mouth positions to determine a square crop window for cropping and rotating the raw image.
    The realigned image is denoted as $X_{realigned}$  with the eyes at the horizontal level and the face at the center of the image. 
    Then MTCNN \cite{7553523} is used to get the positions of the eyes, the nose, and the corners of the mouth of $X_{realigned}$, and the 5 feature points are then fed into $\mathcal{F}$ to predict camera parameters. Finally, these positions are used to crop $X_{realigned}$, resulting in the final image.  
    %
    
    In our pipeline, we first use Dlib to get 68 landmarks for each of the 155,720 raw portrait images, and for those images that resist face detection, we additionally apply face alignment \cite{DBLP:conf/iccv/BulatT17} (SFD face detector) to predict landmarks.  
    %
    The face alignment detector achieves better performance on large-pose face detection than Dlib.
    Joining the two landmark predictors can help us detect as many large-pose faces as possible. Then the predicted landmarks are used to get the realigned image $X_{realigned}$. 
    {In this step, we get \textbf{506,262} $X_{realigned}$.}
    
    We find that the MTCNN sometimes cannot predict landmarks for large-pose faces. So instead of using MTCNN, we directly aggregate the 68 landmarks to get the 5 feature points of the eyes, mouth, and nose.

    
    After that, we use $\mathcal{F}$ to predict camera parameters. 
    Then we filter large-pose face data from \textbf{506,262} $X_{realigned}$ (detailed in Sec. \ref{subsec:Large Pose Data Selection}), getting \textbf{208,543}  large pose $X_{realigned}$.
    We automatically filter out low-resolution images and manually examine the rendering results of the reconstructed face models, removing any failed 3D reconstructions (which indicate incorrectly estimated camera parameters), as well as blurry or noisy images. 
    Finally, we get \textbf{19,590}
    high-quality large-pose face images with correctly estimated camera parameters. 
    %
    
    When cropping the final image, we find that some of the 5 feature points (especially when there is a face with eyeglasses) are not accurate enough to crop $X_{realigned}$ properly, but after manual selection,
    the landmarks that $\mathcal{F}$ produces are more aligned with the input faces. 
    %
    {So we use the landmarks of the reconstructed face to crop $X_{realigned}$ according to EG3D and StyleGAN functions and obtain final images. Please refer to the supplement file for an illustration of the image processing pipeline.}


        
\subsection{Large-Pose Data Selection}
%\subsection{Modulating Data Distribution}
    \label{subsec:Large Pose Data Selection}
    {
   To collect only images with ``low density'' (at large poses), we propose using the density function of FFHQ to filter large pose faces.
   %
    Inspired by \cite{DBLP:journals/tog/LeimkuhlerD21}, 
    we estimate the density of the \textit{FFHQ} camera $(\theta, \phi)$ tuples using Gaussian kernel density estimation and  Scott’s rule \cite{DBLP:books/wi/Scott92} as a bandwidth selection strategy. 
   After obtaining $\rho_{ffhq}$, where $density = \rho_{ffhq} (\theta, \phi)$ is the density of the camera at $(\theta, \phi)$, 
   %
   we use $\rho_{ffhq}$ to compute the density of \textbf{506,262} $X_{realigned}$, and filter the images with a density less than 0.4 ($density = \rho_{ffhq} (\theta, \phi)<$  0.4).
   }
    

    \subsection{Data Rebalance}
    \label{subsec:Data Rebalance} 

    After image processing, large pose filtering, and carefully manual selecting, we get \textbf{19,590} large-pose face images as our \textit{LPFF} dataset.
    We use  the \textit{LPFF} dataset as a supplement to \textit{FFHQ}. That is, we combine \textit{LPFF} with \textit{FFHQ}, named \textit{FFHQ+LPFF}. The datasets are augmented by a horizontal flip.
    In Fig.~\ref{fig:data_distribution}, we show the camera distribution for both \textit{FFHQ+LPFF} and \textit{FFHQ}. 
    %Compared to \textit{FFHQ}, \textit{FFHQ+LPFF} has a wider camera distribution.

    
    
    To improve our models' performance on large-pose rendering quality and image inversion, we propose using a resampling strategy to further rebalance our \textit{FFHQ+LPFF} dataset (refer to Sec. \ref{sec:Evaluation} for evaluation).
    %
    In EG3D, in order to increase the sampling probability of the low-density data, the authors rebalanced the \textit{FFHQ} dataset by splitting it into 9 uniform-sized bins across the yaw range and duplicating the images according to the bins (as shown in Fig.~\ref{fig:data_distribution} (b)). We denoted the rebalanced \textit{FFHQ} dataset as \textit{FFHQ-rebal}.
   

    Inspired by EG3D, we also rebalance \textit{FFHQ+LPFF} to help the model focus more on large-pose data.
    %
    Instead of simply splitting the dataset according to yaw angles, we split \textit{FFHQ+LPFF} according to the data densities (Fig.~\ref{fig:data_distribution} (d)). Similar to Sec.~\ref{subsec:Camera Parameter}, we first compute the pose density function of \textit{FFHQ+LPFF} (denoted as $density = \rho_{ffhq+lpff} (\theta, \phi)$), 
    then duplicate our dataset as:
    \begin{equation}
    \centering
        \left\{
            \begin{array}{lr}
                N = \mathop{\min}(\mathop{\max}(\mathrm{round}(\frac{\alpha}{density}), 1), 4),density \geq 0.03&   \\
                N = 5,density \in [0.02,0.03)&  \\
               N = 6,density \in [0,0.02)&    
               %N = 7,density \in [0,0.01)&   
            \end{array}
        \right.
    \end{equation}
    where $\alpha$ is a hyper-parameter (we empirically set $\alpha=0.24$ in our experiments), and $N$ denotes the number of repetitions. The rebalanced \textit{FFHQ+LPFF} is denoted as \textit{FFHQ+LPFF-rebal}.


        
        
% \subsection{Unsplash Pexels Data}
% \label{Subsec: Unsplash Pexels Data}
%     Besides the images from the website Flickr, we additionally  collect 19,321 images from Unsplash\footnote{\hyperref[]{https://unsplash.com}} and Pexels\footnote{\hyperref[]{https://www.pexels.com}}, process them using the same image align method as \textit{LPFF}, denote it as \textit{Unsplash-Pexels}.
%     %
%     However, most of the images in Unsplash and Pexels were taken by professional photographers and processed by image filters, while the images in Flickr are from everyday life scenes taken by users. 
%     %
%     Although the images in \textit{Unsplash-Pexels} are aligned according to the same function as \textit{LPFF}, the domain gap between \textit{Unsplash-Pexels} and \textit{FFHQ} will prevent the models from generating view-consistent results (see Subsec. \ref{subsec:Dataset_style}).
%     %
%     So that we don't use \textit{Unsplash-Pexels} as our training data, but propose this dataset to inspire and facilitate more works in the future.




\section{Training Details}
\label{sec:training_details}
In this section, we will retrain 2D and 3D-aware face generators using our dataset.
%
Regarding the 2D generator (Sec.~\ref{subsec:StyleGAN_retrain}), we retrain StyleGAN2-ada using our dataset before fine-tuning the model using the rebalanced dataset.
%
As for the 3D-aware generator (Sec.~\ref{subsec:EG3D_retrain}), we first use our dataset to retrain EG3D, and then use the rebalanced dataset to fine-tune the model. 
In order to improve image synthesis performance during testing, we further fine-tune the model by setting the camera parameters input to the generator as the average camera.



\subsection{StyleGAN}   
\label{subsec:StyleGAN_retrain}
    \paragraph{Retrain.}
    In the StyleGAN training, we use the StyleGAN2-ada architecture as our baseline, and train it on \textit{FFHQ+LPFF} from scratch. 
    %
    We use the training parameters %that
    defined by \textit{stylegan2} config in StyleGAN2-ada.
    %
     We denote the StyleGAN2-ada model %that 
     trained on \textit{FFHQ} as $S^{FFHQ}_{var1}$, and the model %that
     trained on \textit{FFHQ+LPFF} as $S^{Ours}_{var1}$. 
    %
    Our training time is $\sim$5 days on 8 Tesla V100 GPUs.

    
    \paragraph{Rebalanced dataset fine-tuning.} %fine-tune.}
    We utilize the rebalanced dataset, \textit{FFHQ+LPFF-rebal}, to fine-tune $S^{Ours}_{var1}$, and denote the rebalanced model as $S^{Ours}_{var2}$.
    %
    All training parameters are identical to those of $S^{Ours}_{var1}$.
    %
    Our fine-tuning time is $\sim$18 hours on  8 Tesla V100 GPUs.
    
   
\subsection{EG3D}
\label{subsec:EG3D_retrain}

The mapping network, volume rendering module, and dual discriminator in EG3D~\cite{Chan_2022_CVPR} are all camera pose-dependent. We divided the EG3D model into three modules: Generator $G$, Renderer $R$, and Discriminator $D$, 
%as shown in Fig.~\ref{fig:eg3d_pipeline}. 
{please refer to the supplement file for an illustration of the three modules.}
The attribute correlations between pose and other semantic attributes in the dataset are faithfully modeled by using the camera parameters fed into $G$. $R$ and $D$ are always fed with the same camera specifications. The camera parameters help $D$ ensure multi-view-consistent super resolution and direct $R$ in how to render the final images from various camera views.

    %
    In this paper, we define two types of camera parameters that are inputted into the whole model as:
    \begin{equation}
            c = [c_{g},c_{r}],
            \label{eq:camera_presentation}
        \end{equation} 
    where $c_{g}$ stands for the camera parameters fed into $G$, and $c_{r}$ stands for the camera parameters fed into $R$ and $D$. $c_g$ will influence the face geometry and appearance and should be fixed in testing. 
    %
    The authors of EG3D discover that maintaining $c_{g}=c_{r}$ throughout training can result in a GAN that generates 2D billboards.
    %In training, the authors of EG3D find that always keeping the $c_{g}=c_{r}$ can lead to a GAN that produces 2D billboards.
    To solve this problem, they apply a swapping strategy that randomly swaps %the 
    $c_g$ with another random pose in that dataset with $\beta$ probability, where $\beta$ is a hyper-parameter.
    
% \begin{table}[t]
%     \centering
%     \scalebox{0.8}{
%     \begin{tabular}{|c|c|c|c|}
%     \hline
%     model    & arch  & initiation   & dataset   \\ \hline
%     $E^{FFHQ}_{var1}$  & \makecell[c]{EG3D \\ $\beta = 50\%$}  & / & \textit{FFHQ} \\ \hline
%     $E^{FFHQ}_{var2}$& \makecell[c]{EG3D \\ $\beta = 80\%$} & $E^{FFHQ}_{var1}$  & \textit{FFHQ-rebal}  \\ \hline
%     $E^{Ours}_{var1}$&\makecell[c]{EG3D \\ $\beta = 50\%$} &/ & \textit{FFHQ+LPFF}  \\ \hline
%     $E^{Ours}_{var2}$& \makecell[c]{EG3D \\ $\beta = 80\%$} &$E^{Ours}_{var1}$ & \textit{FFHQ+LPFF-rebal}  \\ \hline
%      $E^{Ours}_{var3}$& \makecell[c]{EG3D \\ $c_g = c_{avg}$}&$E^{Ours}_{var1}$ & \textit{FFHQ+LPFF}  \\ \hline
%     \end{tabular}
%     }
%     \caption{Model settings of EG3D training.}
%     \label{lable::Model-settings}
% \end{table}
    
    \paragraph{Retrain.}
    We use \textit{FFHQ+LPFF} to train EG3D from scratch.
    %
    All the training parameters are identical to those of EG3D,
    %
    where $\beta$ is linearly decayed from 100\%to 50\% over the first 1M images, and then fixed as 50\% in the remaining training.
    We denote the EG3D trained on \textit{FFHQ} as $E^{FFHQ}_{var1}$ (the original EG3D), denote the EG3D trained on \textit{FFHQ+LPFF} as $E^{Ours}_{var1}$.
    %
     Our training time is $\sim$6.5 days on  8 Tesla V100 GPUs. %  6 days + 
        
    
    \paragraph{Rebalanced dataset fine-tuning.}%finetune.}
     In EG3D, the authors use the rebalanced dataset \textit{FFHQ-rebal} to fine-tune $E^{FFHQ}_{var1}$, leading to a more balanced model. We denote the fine-tuned model as $E^{FFHQ}_{var2}$. 
     For a fair comparison, we also use the same fine-tuning strategy as EG3D to  fine-tune our model $E^{Ours}_{var1}$ on our rebalanced dataset \textit{FFHQ+LPFF-rebal}.
     % We also fine-tune our model $E^{Ours}_{var1}$ using our rebalanced dataset \textit{FFHQ+LPFF-rebal}.
    %
    $\beta$ is fixed as 50\% in training, and other training parameters are identical to those of EG3D.
    %
    %See Tab.~\ref{lable::Model-settings}. 
    We denote $E^{Ours}_{var1}$ fine-tuned on \textit{FFHQ+LPFF-rebal} as $E^{Ours}_{var2}$.
    %
    Our fine-tuning time is $\sim$1 day on  8 Tesla V100 GPUs. % 3h + 8h
    
    
    % \paragraph{Avg-camera-conditioned-generator fine-tuning.}
    %  To prevent the rendering results from changing with the camera poses, in testing, the EG3D generator is always conditioned by a fixed camera pose (the average camera pose $c_{avg}$) and the scene is rendered from a moving camera trajectory. 
    %  %
    %  We also find that due to the imbalance in the training dataset, the more extreme the $c_{g}$ is, the face geometry generated from the generator will have more obvious distortion, and the average camera pose $c_{avg}$ can help the generator output more reasonable geometry. 
    %  %
    %  Considering the above points, we fix the camera parameters that are inputted %input 
    %  into  the generator as  $c_g = c_{avg}$, using it as the starting point of our fine-tuning. 
    %  We use \textit{FFHQ+LPFF} as the training dataset to fine-tune $E^{Ours}_{var1}$, and denote the fine-tuned model as $E^{Ours}_{var3}$.
    %  %(see Tab.~\ref{lable::Model-settings}). 
    %  %
    %  Our fine-tuning time is $\sim$1 day on  8 Tesla V100 GPUs.
    
    
    


    


\section{Evaluation}
\label{sec:Evaluation}
{To show that \textit{LPFF} can help 2D and 3D-aware face generators generate realistic results across large poses, we will first evaluate the performance of 2D face generators (Sec.~\ref{sec: eval_stylegan}), and then demonstrate the performance of 3D-aware face generators (Sec. \ref{sec: eval_EG3D}).}
\subsection{StyleGAN}
\label{sec: eval_stylegan}
      \begin{figure}[t] 
          \centering
          \includegraphics[width=.85\columnwidth]{stylegan_gen.pdf}
          \caption{Images produced by our $S^{Ours}_{var1}$ model (Top) and $S^{Ours}_{var2}$ model (Bottom). We apply truncation  with $\psi=0.7$.}
          \vspace{-5pt}
          \label{fig:stylegan_gen}
        \end{figure}
    
    Fig.~\ref{fig:stylegan_gen} shows the uncurated samples of faces generated by the models trained on our dataset, with resolution $1024^2$. Our models synthesize images that are of high quality and have large pose variance.
    
    
    \paragraph{FID and perceptual path length.}  We trained the models using different datasets, so the latent space distributions are different in our experiments. Therefore, we do not compare 
    the Fr\'echet Inception Distance (FID) \cite{DBLP:conf/nips/HeuselRUNH17} and perceptual path length (PPL) \cite{DBLP:conf/cvpr/KarrasLA19} between 
    the models, since they are highly related to dataset distributions. 
    %
    Instead, we respectively measure the FID of $S^{FFHQ}_{var1}$, $S^{Ours}_{var1}$ and  $S^{Ours}_{var2}$ on their training dataset. The FID of $S^{FFHQ}_{var1}$ is 2.71 on \textit{FFHQ}, the FID of $S^{Ours}_{var1}$ is 3.407 on \textit{FFHQ+LPFF}, and the FID of $S^{Ours}_{var2}$ is 3.786 on \textit{FFHQ+LPFF-rebal}. 
    {The comparable FIDs show that the StyleGAN2-ada model can achieve convergence on our datasets as it did on \textit{FFHQ}.}
    %
    We use the PPL metric that is computed based on path endpoints in $W$ latent space, without the central crop.  The PPL of $S^{FFHQ}_{var1}$ is 144.9, the PPL of $S^{Ours}_{var1}$ is 147.6, and the PPL of $S^{Ours}_{var2}$ is 173.0.
    The PPL of $S^{Ours}_{var1}$ is comparable to the PPL of $S^{FFHQ}_{var1}$. The higher PPL of $S^{Ours}_{var2}$ indicates that $S^{Ours}_{var2}$ leads to more drastic image feature changes when performing interpolation in the latent space. We %, we 
    attribute this to the larger pose variance in $S^{Ours}_{var2}$'s latent space and the \textit{FFHQ+LPFF-rebal} dataset.
    % \onethousand{
    % Instead, we respectively measure the FID and perceptual path length of $S^{FFHQ}_{var1}$, $S^{Ours}_{var1}$ and  $S^{Ours}_{var2}$ on their training dataset, as shown in Tab. \ref{tab:fid_ppl}.
    % The comparable FIDs show that the StyleGAN2-ada model can achieve convergence on our datasets as it did on \textit{FFHQ}.
    % %
    % We use the PPL metric that is computed based on path endpoints in $W$ latent space, without the central crop. 
    % The PPL of $S^{Ours}_{var1}$ is comparable to the PPL of $S^{FFHQ}_{var1}$. The higher PPL of $S^{Ours}_{var2}$ indicates that $S^{Ours}_{var2}$ leads to more drastic image feature changes when performing interpolation in the latent space. We %, we 
    % attribute this to the larger pose variance in $S^{Ours}_{var2}$'s latent space and the \textit{FFHQ+LPFF-rebal} dataset.
    % }


  \begin{figure}[t]
    	\centering
    	{\includegraphics[width=0.95\columnwidth]{stylegan_interface_pose_comparison.pdf}}
    	\caption{Pose manipulation comparison between $S^{FFHQ}_{var1}$ (Top), $S^{Ours}_{var1}$ (Middle), and $S^{Ours}_{var2}$ (Bottom). The images highlighted by the blue box are generated from randomly sampled latent codes, and all the samples are linearly moved along the yaw editing direction with the same distance.
    	}
    	\vspace{-5pt}
    	\label{fig:stylegan_interface_pose_comparison}
    \end{figure}
    
%        \begin{table}[]
%        \centering
%         {%
%         \begin{tabular}{ccc}
%         \hline
%         \multicolumn{1}{l}{model} & FID   & PPL  \\ \hline
%         $S^{FFHQ}_{var1}$         & 2.71  & 144.9 \\
%         $S^{Ours}_{var1}$         & 3.41  & 147.6 \\
%         $S^{Ours}_{var2}$         & 3.79 & 173.0 \\ \hline
%         \end{tabular}%
%         }
%             \caption{ 
%             \onethousand{Quantitative evaluation of FID and PPL.}
%             }
%             \vspace{-10pt}
%         \label{tab:fid_ppl}
% \end{table}

    
\paragraph{Pose manipulation.}
    We compare the pose distribution of the latent spaces by displaying the results of linear yaw pose manipulation. 
    %
    {For each model, we label randomly sampled latent codes according to the camera parameters of the corresponding synthesized images (yaw angles $\textgreater 90^{\circ}$ as positive and $\leq 90^{\circ}$ as negative) and use InterfaceGAN \cite{DBLP:conf/cvpr/ShenGTZ20} to compute the yaw editing direction. 
    The pose editing results are then obtained by moving randomly sampled latent codes along the yaw editing direction, as shown in Fig. \ref{fig:stylegan_interface_pose_comparison}.
    %
    Because the linear manipulation method is used without any semantic attribute disentanglement, the results of all models cannot preserve facial identity.}
    %
    As for $S^{FFHQ}_{var1}$, the ``side face'' results are far from a genuine human face, demonstrating that the latent codes have reached the edge of the latent space.
    %
    With regard to  $S^{Ours}_{var1}$ and $S^{Ours}_{var2}$, our models produce reasonable and {comparable} large-pose portraits. 
    The comparison shows that our models' latent spaces are more extensive and better able to represent large-pose data.
  
         
    
 \paragraph{Large-pose data inversion and manipulation.}
    To further show that our models can better represent large-pose data, we project large-pose portraits into the latent spaces of those models (see Fig.~\ref{fig:stylegan_avg_init_projection}), and apply semantic editing to the obtained latent codes.
    %
    We collect the testing images from Unsplash\footnote{\hyperref[]{https://unsplash.com}} and Pexels\footnote{\hyperref[]{https://www.pexels.com}} (independent of both \textit{FFHQ} and \textit{LPFF}). We then employ 500-step latent code optimization in $W+$ latent space to minimize the distance between the synthesized image and the target image. 
    %
    To evaluate the editability of the projected latent codes, we use the attribute classifiers provided by StyleGAN \cite{DBLP:conf/cvpr/KarrasLA19} and employ InterfaceGAN to compute semantic boundaries for each model, and then use the boundaries to edit the projected latent codes.
    %
    We also use the yaw editing directions to try to make the large pose data face forward.
    %
    Please refer to the supplement for those semantic editing results. % in the supplement. 
    %
    As shown in those projection and manipulation results, the models trained on our dataset have fewer artifacts and can better represent the large pose data in their latent spaces. 
    {What's more, $S^{Ours}_{var2}$ outperforms $S^{Ours}_{var1}$ because $S^{Ours}_{var2}$ is trained on a more balanced dataset, which proves the effectiveness of our data rebalance strategy.
    }
    
         \begin{figure}[t]
    	\centering
    	{\includegraphics[width=.8\columnwidth]{stylegan_avg_init_projection.pdf}}
    	\caption{Large-pose data projection comparison between $S^{FFHQ}_{var1}$, $S^{Ours}_{var1}$, and $S^{Ours}_{var2}$. The target images (the first row) are collected from Unsplash and Pexels websites.}
    	\vspace{-5pt}
    	\label{fig:stylegan_avg_init_projection}
    \end{figure}
       
       
       
      % \begin{figure*}[t] 
      %     \centering
      %     \includegraphics[width=0.78\linewidth]{eg3d_gen_curated-v2.pdf}
      %     \caption{
      %     \onethousand{Image-shape pairs produced by $E^{FFHQ}_{var1}$,$E^{FFHQ}_{var2}$, $E^{Ours}_{var1}$, $E^{Ours}_{var2}$, and $E^{Ours}_{var3}$. We apply truncation  with $\psi=0.8$.}
      %     }
      %     \label{fig:eg3d_gen_curated}
      %   \end{figure*}


      \begin{figure*}

\begin{minipage}[b]{.65\linewidth}
    
    \includegraphics[width=1.0\linewidth]{eg3d_gen_curated-v2.pdf}
          \caption{
          {Image-shape pairs produced by $E^{FFHQ}_{var1}$,$E^{FFHQ}_{var2}$, $E^{Ours}_{var1}$, and $E^{Ours}_{var2}$.
          %, and $E^{Ours}_{var3}$. 
          We apply truncation  with $\psi=0.8$.}
          }
          \label{fig:eg3d_gen_curated}
\end{minipage}
\medskip
 \hfill
 \begin{minipage}[b]{.32\linewidth}
    \centering
   \scalebox{0.7}{
    \begin{tabular}{@{}cccc@{}}
    \toprule
    \multicolumn{1}{l}{\multirow{2}{*}{model}} & \multicolumn{1}{l}{$c_g =c_{avg}$} & \multicolumn{1}{l}{$c_g \sim$ FFHQ}  & \multicolumn{1}{l}{$c_g \sim$ LPFF}  \\
    \multicolumn{1}{l}{}   & $c_r \sim$ FFHQ     & $c_r \sim$FFHQ & $c_r \sim$FFHQ    \\ \midrule
    $E^{FFHQ}_{var1}$ & 0.771  &	0.768 &	0.760   	 \\
    $E^{Ours}_{var1}$ &\textbf{0.804}&\textbf{0.792}&\textbf{0.778} \\ \midrule
    $E^{FFHQ}_{var2}$ &0.770&0.769&0.766   \\
    $E^{Ours}_{var2}$ &\textbf{0.789}&\textbf{0.784}&\textbf{0.771}\\ \bottomrule
    \end{tabular}
    }
            \captionof{table}{ 
            {Quantitative evaluation of facial identity consistency ($\uparrow$). }
            }
        \label{tab:Facial_identity}
 \vspace{5pt}
    \scalebox{0.7}{
        \begin{tabular}{@{}cccc@{}}
        \toprule
        \multicolumn{1}{l}{\multirow{2}{*}{model}} & \multicolumn{1}{l}{$c_g =c_{avg}$} & \multicolumn{1}{l}{$c_g \sim$ FFHQ}  & \multicolumn{1}{l}{$c_g \sim$ LPFF}  \\
        \multicolumn{1}{l}{}   & $c_r \sim$ FFHQ     & $c_r \sim$FFHQ & $c_r \sim$FFHQ    \\ \midrule
        $E^{FFHQ}_{var1}$ & 0.134&0.133&0.159	 	 \\
        $E^{Ours}_{var1}$ &\textbf{0.119}&\textbf{0.124}&\textbf{0.134} \\ \midrule
        $E^{FFHQ}_{var2}$ &0.135&0.130&0.142    \\
        $E^{Ours}_{var2}$ &\textbf{0.117}&\textbf{0.122}&\textbf{0.131}  \\  \bottomrule
        \end{tabular}
        }
                \captionof{table}{{
                    Quantitative evaluation of geometry consistency ($\downarrow$).
                }
                }
            \label{tab:Geometry}
\end{minipage} %\par

\end{figure*}



        
\begin{table*}[t]
\centering
\scalebox{0.7}{
    \begin{tabular}{@{}ccccccccc@{}}
    \toprule
    \multicolumn{1}{l}{\multirow{2}{*}{model}} & \multicolumn{1}{l}{$c_g =c_{avg}$} & \multicolumn{1}{l}{$c_g =c_{avg}$} & \multicolumn{1}{l}{$c_g \sim$ FFHQ} & \multicolumn{1}{l}{$c_g \sim$ FFHQ} & \multicolumn{1}{l}{$c_g \sim$ LPFF} & \multicolumn{1}{l}{$c_g \sim$ LPFF} & \multicolumn{1}{l}{$c_g \sim$ FFHQ} & \multicolumn{1}{l}{$c_g \sim$ LPFF}\\
    \multicolumn{1}{l}{}   & $c_r \sim$ FFHQ     & $c_r \sim$ LPFF  & $c_r \sim$FFHQ & $c_r \sim$LPFF   & $c_r \sim$FFHQ & $c_r \sim$LPFF & $c_r =c_g$ &  $c_r =c_g$    \\ \midrule
    $E^{FFHQ}_{var1}$ &\textbf{6.523}&23.598&\textbf{4.273}&22.318&23.698&36.641&\textbf{4.025}&23.301   \\
    $E^{Ours}_{var1}$ & 7.997&\textbf{20.896}&6.623&\textbf{19.738}&\textbf{21.300}&\textbf{22.074}&6.093&\textbf{16.026}  \\ \midrule
    $E^{FFHQ}_{var2}$ &\textbf{6.589}&20.081&\textbf{4.456}&19.983&19.469&30.181&\textbf{4.262}&23.717   \\
    $E^{Ours}_{var2}$ & 9.829&\textbf{16.775}&6.672&\textbf{15.047}&\textbf{13.022}&\textbf{14.836}&6.571&\textbf{12.221} \\ \bottomrule
    \end{tabular}
    }
            \caption{FID ($\downarrow$) for EG3D generators that are trained on different datasets. We calculate the FIDs by sampling 50,000 images using different sampling strategies and different camera distributions. 
            {We compare the models that are trained with the same training strategy ($var1/var2$).
            %$and separately list the results of our novel fine-tuning method ($var3$).
            }
            \vspace{-5pt}
            }
        \label{tab:FID1}
\end{table*}


%   \begin{table}[t]
%             \centering
            
%     \scalebox{0.8}{
%     \begin{tabular}{@{}cccc@{}}
%     \toprule
%     \multicolumn{1}{l}{\multirow{2}{*}{model}} & \multicolumn{1}{l}{$c_g =c_{avg}$} & \multicolumn{1}{l}{$c_g \sim$ FFHQ}  & \multicolumn{1}{l}{$c_g \sim$ LPFF}  \\
%     \multicolumn{1}{l}{}   & $c_r \sim$ FFHQ     & $c_r \sim$FFHQ & $c_r \sim$FFHQ    \\ \midrule
%     $E^{FFHQ}_{var1}$ & 0.771  &	0.768 &	0.760   	 \\
%     $E^{Ours}_{var1}$ &\textbf{0.804}&\textbf{0.792}&\textbf{0.778} \\ \midrule
%     $E^{FFHQ}_{var2}$ &0.770&0.769&0.766   \\
%     $E^{Ours}_{var2}$ &\textbf{0.789}&\textbf{0.784}&\textbf{0.771}\\ \midrule
%     $E^{Ours}_{var3}$ &0.785 &/&/\\ \bottomrule
%     \end{tabular}
%     }
%             \caption{ 
%             {Quantitative evaluation of facial identity consistency ($\uparrow$). Compared to the models trained on \textit{FFHQ}, our models present significant improvements in facial identity consistency. The identity similarity is computed by the ArcFace model \cite{DBLP:conf/cvpr/DengGXZ19}.}
%             }
%         \label{tab:Facial_identity}
% \end{table}
%      \begin{table}[t]
%             \centering
            
%     \scalebox{0.8}{
%         \begin{tabular}{@{}cccc@{}}
%         \toprule
%         \multicolumn{1}{l}{\multirow{2}{*}{model}} & \multicolumn{1}{l}{$c_g =c_{avg}$} & \multicolumn{1}{l}{$c_g \sim$ FFHQ}  & \multicolumn{1}{l}{$c_g \sim$ LPFF}  \\
%         \multicolumn{1}{l}{}   & $c_r \sim$ FFHQ     & $c_r \sim$FFHQ & $c_r \sim$FFHQ    \\ \midrule
%         $E^{FFHQ}_{var1}$ & 0.134&0.133&0.159	 	 \\
%         $E^{Ours}_{var1}$ &\textbf{0.119}&\textbf{0.124}&\textbf{0.134} \\ \midrule
%         $E^{FFHQ}_{var2}$ &0.135&0.130&0.142    \\
%         $E^{Ours}_{var2}$ &\textbf{0.117}&\textbf{0.122}&\textbf{0.131}  \\ \midrule
%         $E^{Ours}_{var3}$ &0.122 &/&/  \\ \bottomrule
%         \end{tabular}
%         }
%                 \caption{{
%                     Quantitative evaluation of geometry consistency ($\downarrow$). Compared to the models trained on \textit{FFHQ}, our models present significant improvements in geometry consistency. The geometry is reconstructed by the face reconstruction model \cite{DBLP:conf/cvpr/DengYX0JT19}.
%                 }
%                 }
%             \label{tab:Geometry}
%     \end{table}



        
    \subsection{EG3D}   
    \label{sec: eval_EG3D}
      Fig.~\ref{fig:eg3d_gen_curated} provides the selected samples that are generated by the models trained {on the \textit{FFHQ} dataset and our dataset}, with resolution $512^2$. Even in large poses, our synthesized images and 3D geometry are high-quality.
      
              \paragraph{FID.}
        \label{subsec:FID}
        In EG3D, the generator is conditioned on a fixed camera pose ($c_{g}$) when rendering from a moving camera trajectory to prevent the scene from changing when the camera ($c_r$) moves during inference. However, EG3D's authors evaluated the FID of EG3D by conditioning the model on $c_{g}$ and rendering results from $c_r = c_g$. This approach cannot demonstrate the performance of multi-view rendering during inference, since
        the generator always ``sees'' the true pose of the rendering camera in evaluation, but omits other poses. 
         %
        For a 3D-aware generator, we are more interested in how a face looks from various camera views (which can indicate the quality of face geometry to some extent).
        So a more reasonable way is to let $c_r$ and $c_g$ be independent of
        each other and sample them from the respective  distributions that are of our interest.
        %
        To achieve this, we propose a novel FID measure, which is based on
        three camera sampling strategies. First, we fix %the 
        $c_{g}$ as $c_{avg}$ and then sample $c_r$ from different datasets. Second, we respectively sample $c_{r}$ and $c_{g}$ from different datasets. Third, we sample $c_{g}$ from different datasets and set $c_r=c_g$ (the one that was used in EG3D). See the calculated FID values in Tab. \ref{tab:FID1}.
        
        
        
        
        Models trained on our datasets exhibit improvements in FID in most cases, particularly when the final results are rendered from large poses ($c_r \sim LPFF$), or when the generator is conditioned on large poses ($c_g \sim LPFF$).  
        {We notice that there is an increased FID when computing $c_g=c_{avg}/c_r, c_r\sim FFHQ$.}
        % The possible reasons are twofold. 
        % First, FID is highly related to the data distribution, and the new addition of \textit{LPFF} data changes the data distribution when rendering from medium poses ($c_r \sim FFHQ$).
        % Second, as explained by the authors of EG3D,  the pre-trained $E^{FFHQ}_{var1}$ and $E^{FFHQ}_{var2}$ were achieved using buggy (XY, XZ, ZX) planes.  
        % In our experiments, since we fix this bug as they suggested using (XY, XZ, ZY),
        % the XZ-plane representation's dimension would be cut in half, thus weakening the expressive capacity for frontal faces.
        As explained by the authors of EG3D,  the pre-trained $E^{FFHQ}_{var1}$ and $E^{FFHQ}_{var2}$ were achieved using buggy (XY, XZ, ZX) planes.  
        In our experiments, since we fix this bug as they suggested using (XY, XZ, ZY),
        the XZ-plane representation's dimension would be cut in half, thus weakening the expressive capacity for frontal faces.
        
    
        
        Thanks to our dataset rebalancing strategy, $E^{Ours}_{var2}$ can pay more attention to large pose data and enhance the rendering quality, thus further improving the FID of $E^{Ours}_{var1}$ on large poses. 
        {When computing FID of $c_g=c_{avg},c_r\sim FFHQ$, we notice that $E^{Ours}_{var2}$ has an increased FID compared to $E^{Ours}_{var1}$, while $E^{FFHQ}_{var2}$ and  $E^{FFHQ}_{var1}$ have comparable results. 
        This is due to the addition of new large-pose data, \textit{LPFF}.
        FID is highly related to the data distribution, and the rebalancd \textit{FFHQ+LPFF-rebal} dataset changes the data distribution when rendering from medium poses.
        }
        
        

         
        %   \begin{figure*}[t]
        % 	\centering
        % 	{\includegraphics[width=0.8\linewidth]{face_scape_projection.pdf}}
        % 	\caption{We use latent code optimization to fit four testing images (left) of a single identity from FaceScape. Then we render the obtained latent codes from four novel views (right). The optimization is performed in $W+$ space, and the generators are conditioned on the average camera parameters.}
        % 	\label{fig:face_scape_projection}
        % \end{figure*}
    
        % \begin{figure*}[h]
        %     \centering
        %     \includegraphics[width=0.9\textwidth]{Unsplash_Pexels_4.pdf}
        %           \caption{
        %           We use HFGI3D \cite{xie2022high} to fit the single-view testing image. Then we render the obtained latent codes from four novel views. The inversion is performed in $W$ space, and the generators are conditioned on the average camera parameters.
        %           }
        %           \label{fig:Unsplash_Pexels_4}
        % \end{figure*}       


          \begin{figure*}[h]
            \centering
            \includegraphics[width=0.85\textwidth]{Unsplash_Pexels_4.pdf}
                  \caption{
                To fit the single-view testing image, we employ HFGI3D \cite{xie2022high}. The obtained latent codes are then rendered using four novel views. The inversion is carried out in $W$ space, and the generators are conditioned on $c_{avg}$.
                   }
                  \label{fig:Unsplash_Pexels_4}
        \end{figure*}       


        
         
        \paragraph{Facial identity consistency.}
         \label{subsec:Facial_Identity}   
        We leverage ArcFace \cite{DBLP:conf/cvpr/DengGXZ19} to measure the models' performance on facial identity maintenance. We render two novel views for 1,024 random faces and use ArcFace to compute the mean identity similarity for all image pairs.
        %
        %See Tab. \ref{tab:Facial_identity}, 
        We employ three sampling strategies for $c_g$ to evaluate the generator's performance on the camera distribution of different datasets. As for $c_r$, we find that the extreme rendering camera views will heavily influence the performance of ArcFace, so we only sample $c_r$ from the \textit{FFHQ} dataset, where most of the faces have small to medium poses. As shown in Tab. \ref{tab:Facial_identity}, our models present significant improvements in facial identity consistency across different sample strategies and datasets.
      

        
          \paragraph{Geometry consistency.}
        \label{subsec:Geometry}   
        We employ $\mathcal{F}$, which outputs 3DMM coefficients to evaluate the geometry consistency.
        We employ the same camera sampling methods as in facial identity consistency computation.
        We first render two novel views for 1,024 random faces. Then
        for each image pair, we compute the mean L2 distance of the face id and expression coefficient.
        As shown in Tab. \ref{tab:Geometry}, our models present improvement in geometry consistency across different sample strategies and datasets.
         %
  

        
      
     
         \paragraph{Image inversion.}
         To evaluate the ability to fit multi-view images, we use FaceScape \cite{DBLP:conf/cvpr/Yang0WHSYC20} as the testing data.
         %
         We use four multi-view images (including one with a small pose) of a single identity as the reference images. We perform latent code optimization to simultaneously project one or four images into $W+$ latent space. Then we use the camera parameters that are extracted from another 4 multi-view images to render novel views. Please refer to the supplement for multi-view image inversion results.
         %
         {
         Because occluded face parts are unavoidable in single-view portraits, we perform single-view image inversion using HFGI3D \cite{xie2022high}, a novel method that combines pseudo-multi-view estimation with visibility analysis.
         }
         %
         As shown in Fig.~\ref{fig:Unsplash_Pexels_4}, the inversion results indicate that $E^{FFHQ}_{var1}$ and $E^{FFHQ}_{var2}$  suffer from the ``wall-mounted'' unrealistic geometry.
         Due to the adhesion between the head and the background, there are missing ears in View 2 and distorted ears and necks in Views 3 and 4 (highlighted by green boxes). A pointed nose exists in View 2 (highlighted by blue boxes).
         %
         Our $E^{Ours}_{var1}$ and $E^{Ours}_{var2}$ models produce reconstructed face geometry that is free from those artifacts, suggesting that the learned 3D prior from our dataset is more realistic. It also shows that after employing the data rebalance in Sec. \ref{subsec:Data Rebalance}, lips are more natural in $E^{Ours}_{var2}$ compared to $E^{Ours}_{var1}$ (highlighted by an orange box).
         
       
    
      
%   \begin{figure}[h]
%     	\centering
%     	{\includegraphics[width=.9\columnwidth]{eg3d_seam.pdf}}
%     	\caption{After employing GAN inversion, we use $E^{FFHQ}_{var1}$ (Top) and $E^{Ours}_{var1}$ (Bottom) to render novel views for the same target image (not shown in this figure). The ``seam" artifacts are highlighted by blue boxes in the results of $E^{FFHQ}_{var1}$.}
%     	\label{fig:eg3d_seam}
%     \end{figure}
    
    
        \paragraph{``Seam'' artifacts.}
        The authors of IDE-3D speculate that the ``seam"  artifacts in EG3D could be caused by the imbalanced camera pose distribution of datasets, and propose a density regularization loss to deal with the ``seam"  artifacts along the edge of the faces. 
        %
        Compared to the IDE-3D, our model $E^{Ours}_{var1}$ is trained without requiring any additional regularization loss or any data rebalance strategy, and is free from the ``seam" artifacts.  Please refer to the supplement for the illustration of ``seam" artifacts.
           
       
        
    
       
   
    

 \section{Conclusion}
 In this paper, we have presented a tactile manipulation system that is able to rotate different objects without vision. We showed an end-to-end reinforcement learning framework to learn tactile dexterity over the proposed system. We carried out experiments both in simulation and real to demonstrate its effectiveness. Our work demonstrated that we are able to achieve tactile dexterity as humans in real for the first time. In the future, there are many promising future directions to investigate, such as exploring the use of a more dense contact sensor array and scaling up the system to solve more diverse tasks. We hope that our work can pave the way for more intelligent robot hands.

{\small
\bibliographystyle{ieee_fullname}
\bibliography{egbib}
}

\end{document}