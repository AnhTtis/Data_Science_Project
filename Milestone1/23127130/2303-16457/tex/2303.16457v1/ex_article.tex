% SIAM Article Template
\documentclass[final,hidelinks,onefignum,onetabnum]{siamart220329}

% Information that is shared between the article and the supplement
% (title and author information, macros, packages, etc.) goes into
% ex_shared.tex. If there is no supplement, this file can be included
% directly.

\usepackage{enumerate}
\usepackage{enumitem}
\usepackage{csquotes}

\usepackage[caption=false]{subfig}

\definecolor{ForestGreen}{RGB}{34,139,34}

\newcommand{\seb}[1]{%
{\leavevmode\color{blue}#1}%
}

\newcommand{\ben}[1]{%
{\leavevmode\color{red}#1}%
}

\newcommand{\cesar}[1]{%
{\leavevmode\color{magenta}#1}%
}

\newcommand{\brian}[1]{%
{\leavevmode\color{ForestGreen}#1}%
}

\newtheorem{rem}{Remark}
\newtheorem{defn}{Definition}
\newtheorem{lem}{Lemma}
\newtheorem{prop}{Proposition}
\newtheorem{cor}{Corollary}
\newtheorem{thm}{Theorem}
\newtheorem{assm}{Assumption}
\newtheorem{exmpl}{Example}
%\newtheorem{claim}{Claim}
% SIAM Shared Information Template
% This is information that is shared between the main document and any
% supplement. If no supplement is required, then this information can
% be included directly in the main document.


% Packages and macros go here
\usepackage{lipsum}
\usepackage{amsfonts}
\usepackage{graphicx}
\usepackage{epstopdf}
\usepackage{algorithmic}
\ifpdf
  \DeclareGraphicsExtensions{.eps,.pdf,.png,.jpg}
\else
  \DeclareGraphicsExtensions{.eps}
\fi

% Prevent itemized lists from running into the left margin inside theorems and proofs
\usepackage{enumitem}
\setlist[enumerate]{leftmargin=.5in}
\setlist[itemize]{leftmargin=.5in}

% Add a serial/Oxford comma by default.
\newcommand{\creflastconjunction}{, and~}

% Used for creating new theorem and remark environments
\newsiamremark{remark}{Remark}
\newsiamremark{hypothesis}{Hypothesis}
\crefname{hypothesis}{Hypothesis}{Hypotheses}
\newsiamthm{claim}{Claim}

% Sets running headers as well as PDF title and authors
\headers{Area formula for spherical polygons via prequantization}{Albert Chern and Sadashige Ishida}

% Title. If the supplement option is on, then "Supplementary Material"
% is automatically inserted before the title.
\title{Area formula for spherical polygons via prequantization 
% \thanks{Submitted to the editors April 12th, 2023. \rev{Revision submitted to the editors March 28th, 2024. 
% }
% % \funding{This project was funded in part by the European Research Council (ERC Consolidator Grant 101045083 \emph{CoDiNA})}
% }
}

% Authors: full names plus addresses.
\author{Albert Chern
\thanks{University of California San Diego
  (\email{alchern@ucsd.edu}).}
\and Sadashige Ishida\thanks{Institute of Science and Technology Austria 
  (\email{sadashige.ishida@ist.ac.at}).}
% \and Jane E. Smith\footnotemark[3]
}

\usepackage{amsopn}
\DeclareMathOperator{\diag}{diag}


%%% Local Variables: 
%%% mode:latex
%%% TeX-master: "ex_article"
%%% End: 


% Optional PDF information
\ifpdf
\hypersetup{
  pdftitle={Towards Understanding the Endemic Behavior of a Competitive  Tri-Virus SIS Networked Model},
  pdfauthor={S.~Gracy, M.~Ye, BDO~Anderson and C.~A.~Uribe}
}
\fi

% The next statement enables references to information in the
% supplement. See the xr-hyperref package for details.

%\externaldocument[][nocite]{ex_supplement}

% FundRef data to be entered by SIAM
%<funding-group specific-use="FundRef">
%<award-group>
%<funding-source>
%<named-content content-type="funder-name"> 
%</named-content> 
%<named-content content-type="funder-identifier"> 
%</named-content>
%</funding-source>
%<award-id> </award-id>
%</award-group>
%</funding-group>

\begin{document}

\maketitle

% REQUIRED
\begin{abstract}
This paper studies the endemic behavior of a multi-competitive networked \break susceptible-infected-susceptible (SIS) model. Specifically, the paper deals with three competing virus systems (i.e., tri-virus systems). First, we show that a tri-virus system, unlike a bi-virus system, is not a monotone dynamical system. Using the Parametric Transversality Theorem, we show that, generically, a tri-virus system has a finite number of equilibria and that the Jacobian matrices associated with each equilibrium are nonsingular.
The endemic equilibria of this system can be classified as follows: a) single-virus endemic equilibria (also referred to as the boundary equilibria), where precisely one %exactly one, o
of the three viruses is alive; b) 2-coexistence equilibria, where exactly two of the three viruses are alive; and c) 3-coexistence equilibria, where all three viruses survive in the network. We provide a necessary and sufficient condition that guarantees local exponential convergence to a boundary equilibrium.  Further, we secure conditions for the nonexistence of 3-coexistence equilibria (resp. for various forms of 2-coexistence equilibria). We also identify sufficient conditions for the existence of a 2-coexistence (resp. 3-coexistence) equilibrium. We identify conditions on the model parameters that give rise to a continuum of coexistence equilibria. More specifically, we establish i) a scenario that  admits the existence and local exponential attractivity of a line of coexistence equilibria; and ii) scenarios that admit the existence of,  and, in the case of one such scenario, global convergence to, a  plane of 3-coexistence equilibria.
\end{abstract}

% REQUIRED
\begin{keywords}
Epidemic processes, competing viruses, %monotone dynamical systems, 
finiteness of equilibria, coexistence equilibrium.
\end{keywords}

% REQUIRED
\begin{MSCcodes}
34D05, 37C75, 92D30
\end{MSCcodes}

%\section{Introduction}\label{sec:introduction}
\section{Introduction}\label{sec:introduction}
Spreading processes are observed in several settings. Prominent examples include the spread of viruses, in particular, pandemics such as COVID-19~\cite{wang2020clinical}, Spanish flu~ \cite{johnson2002updating}; spread of opinions on social media~\cite{wang2019systematic,bradshaw2017troops}; adoption of products in a market \cite{armington1969theory}; and propagation of species in an ecological environment \cite{macarthur1967limiting}. Given these manifestations in several diverse settings, the study of spreading processes has naturally attracted the attention of multiple scientific communities viz. physics \cite{newman2005threshold,karrer2011competing,wang2012dynamics}, biology \cite{laurie2018evidence,wu2020interference}, mathematics \cite{castillo1989epidemiological,carlos2,hethcote2000mathematics}, computer science \cite{prakash2010virus}, to cite a few. 
\par One of the primary objectives behind all of the aforementioned research efforts is to understand the various equilibria of the systems involved and 
the behavior of spread processes in the neighborhood of an equilibrium point.
Several models have been proposed and extensively studied in mathematical epidemiology: notable among them being the susceptible-infected-recovered (SIR) \cite{mei2017dynamics,van2014exact}; susceptible-exposed-infected-recovered (SEIR) \cite{li1995global,arcede2020accounting};  susceptible–asymptomatic–infected–recovered–susceptible (SAIRS) \cite{rothe2020transmission,pare2020modeling};
 susceptible-infected (SI) \cite{matouk2020complex}; and susceptible-infected-susceptible (SIS) models \cite{kermack1932contributions,lajmanovich1976deterministic,van2009virus,khanafer2016stability}.
The aforementioned models provide rigorous guarantees for (some of) the various outcomes of spreading processes; the conditions involved in providing said guarantees are tractable; and finally, they can generate insightful observations due to the low cost of simulations. Thus,  these models are extremely useful in understanding the behavior of spreading processes and, consequently, have gained widespread acceptance in the broader research community. Our focus in the present paper is on networked SIS models, which consider a population of individuals that are divided 
%partitioned 
into subpopulations, called  \emph{agents}. A (possibly) directed graph captures how the various agents are interconnected. Given that there is a virus spreading in the whole population,
%The interconnection between various agents can be represented by a (possibly) directed graph.
%Supposing that there is a prevalent virus in the population,
it can spread both within a subpopulation (i.e.,
between individuals in a subpopulation) and across subpopulations (i.e.,
between individuals belonging to different subpopulations). 
%within a subpopulation and also between subpopulations. 
If none of the individuals in a subpopulation are infected, then this subpopulation is said to be in the susceptible state; otherwise, it is said to be in the infected state. Networked SIS models have been extensively studied in the literature; see, for instance, \cite{FallMMNP07,khanafer2016stability,rami2014switch,peng2010epidemic,wang2003epidemic,gracy2020analysis,pare2020modeling}. 

\par Available models account for the first instance of the spread of only one virus. However,
one encounters scenarios where multiple viruses are simultaneously circulating in a population.
 In such a context, the viruses involved could either be cooperative (see \cite{gracy2022modeling} and references therein for details), or they could be competitive - the present paper focuses on the competitive case. The notion of competing viruses can be observed in several settings; 
some examples include spread of multiple strains of a virus \cite{nowak1991evolution,laurie2018evidence,wu2020interference}, spread of competing opinions on social networks% \cite{sahneh2014competitive},  
\cite{prakash2012winner,wei2013competing}, and that of products competing in a market \cite{trpevski2010model}. More concretely, supposing that two viruses (say virus~1 and virus~2) are competing, an individual can be either infected with virus~1 or with virus~2 or with neither. In contrast, no individual can be infected with \emph{both} viruses simultaneously. 
The dynamics of competing viruses are far richer than the case of single-virus propagation \cite{newman2005threshold}. Specifically, in multi-competitive virus settings, some possibilities include one  virus dominating the rest and all (or a subset of all) viruses existing in some balance -- a phenomenon referred to as \emph{coexistence}. 
%The underlying focus behind most of the research thrusts in this area stems from the crucial observation that the dynamics of multi-virus systems (competitive or cooperative) is far richer  than that of single-virus propagation \cite{newman2005threshold}.  
%As such, the main objectives revolve around understanding when can one  virus dominate the other, thus pushing it to extinction; when can all (or a subset of all) viruses exist in some sort of balance with each other - a phenomenon referred to as \emph{coexistence}.
%\par \seb{
Various mathematical models have been proposed in the literature to deal with multi-competitive viruses. Prime among them is the susceptible-infected-recovered (SIR) model (proposed in \cite{newman2005threshold} and improved upon in \cite{zhang2022networked}) and the susceptible-infected-susceptible (SIS) model (first proposed in \cite{castillo1989epidemiological}). This paper focuses on the continuous-time multi-competitive networked SIS model. 

%\subsection*{Closely-related papers}
The study of multi-competitive networked SIS models has been pursued in a plethora of works starting with \cite{castillo1989epidemiological,carlos2}, and, more recently, in 
\cite{sahneh2014competitive,axel2020TAC,santos2015bi,ye2021convergence,liu2019analysis,ben:brian:opinion:lcss,pare2021multi,anderson2022equilibria,doshi2022convergence,gracy2022trivirus}. Note that the vast majority of the literature in this area, namely \cite{sahneh2014competitive,santos2015bi,doshi2022convergence,axel2020TAC,liu2019analysis,ye2021convergence,anderson2022equilibria}, focuses on the special case when two viruses compete with each other to infect a given population. The more general case where $m>2$ viruses are competing has been relatively less well studied; see \cite{pare2021multi} (resp. \cite{pare2020analysis}) for a preliminary analysis of (resp. discrete-time) multi-competitive networked SIS model. While \cite{pare2021multi,pare2020analysis} provide the general modeling framework for multi-competitive networked SIS models, in that the models therein admit the possibility of more than two viruses competing, the analysis of the possible endemic behavior in said papers is rather limited; in particular most of the results in \cite{pare2020analysis} pertain to the existence  of boundary equilibria. Thus, to our knowledge, an overarching analysis accounting for $m$ competing viruses, where $m$ is some arbitrary but finite positive integer, is not available in the existing literature. Therefore, as a first step, in this direction, the present paper deals with the case where $m=3$ (tri-virus system). %In the rest of this paper, we will consider three competing viruses. 
The equilibria of the tri-virus system can be broadly classified into three categories: the disease-free equilibrium (DFE) (all three viruses have been eradicated); the boundary equilibria (two viruses are dead, and one is endemic); %and coexistence equilibria 
and $k$-coexistence equilibria (where $k \geq 2$ viruses infect separate fractions of every population node in the network).
%(at least two, possibly three, viruses infect separate fractions of every population node in the network).



%\subsection*{Paper Contributions}
%\seb{
We %consider the tri-virus system, and
 make the following key contributions:
\begin{enumerate}[label=\roman*)]
\item We show that it (i.e., the tri-virus system) is not monotone; see Theorem~\ref{prop:tri-virus-not-monotone}. Consequently, one cannot use the existing tools in the literature on competitive bivirus systems, which are deeply rooted %on 
in monotone dynamical systems (see \cite{hirsch1988stability,smith1988systems}) to study the limiting behavior of our model.
\item We show, using the Parametric Transversality Theorem of differential topology \cite[page~145]{lee2013introduction}, that it has a finite number of equilibria, in a generic \footnote{A precise definition of the term \enquote{generic} is provided in Section~\ref{sec:trivirus:monotone}.} sense; see Proposition~\ref{prop:finite:equilibria}. Additionally, Proposition~\ref{prop:finite:equilibria} also establishes that the  Jacobian matrices evaluated at each of the equilibrium points are nonsingular.
\item We identify a sufficient condition, and multiple necessary conditions, for local exponential convergence to a boundary equilibrium; see Theorem~\ref{thm:local}.
\item We identify a sufficient condition for the nonexistence of any 3-coexistence equilibria, and nonexistence of certain forms of 2-coexistence equilibria; see Theorem~\ref{claim:virus3:strongest} and Theorem~\ref{claim:virus1weaker}, respectively. 
\item Leveraging that a bivirus system is monotone, we identify a novel and simpler sufficient condition for a 2-coexistence equilibrium; see Proposition~\ref{prop:2-coexistence}.
\item Using the notion of saturated fixed points, we identify a sufficient condition for the existence of a 3-coexistence equilibrium; see Corollary~\ref{cor:hofbauer}.
\item We identify a special scenario (with respect to the class of system parameters) that gives rise to the existence of, and, subject to the fulfillment of certain conditions, local convergence to, a line of coexistence equilibria; see Theorem~\ref{thm:init:condns}.
\item Assuming that the three viruses are identical, we show that, regardless of the initial non-zero infection levels, the dynamics of the tri-virus system converge to a plane of coexisting equilibria; see Theorem~\ref{thm:global:plane}.
\end{enumerate}
%}
In addition, we make several auxiliary contributions. More specifically, we show that for low-dimensional systems, i.e., $n=1$ or $n=2$, certain kinds of coexistence equilibria can, generically, never exist; see Propositions~\ref{prop:brian:1} and~\ref{prop:brian:2}. Using the notion of saturated fixed points, we prove that either a stable boundary equilibrium exists and/or a saturated 2-coexistence equilibrium and/or a 3-coexistence equilibrium; see Theorem~\ref{thm:hofbauer}. We identify a special setting, but one that subsumes the setting in Theorem~\ref{thm:global:plane}, that permits the existence of a plane of coexistence equilibria; see Proposition~\ref{prop:plane:equilibrium}.

A subset of the results of this paper has been accepted for publication in the Proceedings of the  $2023$ American Control Conference; see~\cite{gracy2022trivirus}. The present paper is a significant extension from~\cite{gracy2022trivirus}; Theorem~\ref{claim:virus3:strongest}, Theorem~\ref{claim:virus1weaker}, Corollary~\ref{cor:hofbauer}, Theorem~\ref{thm:hofbauer}, Propositions~\ref{prop:finite:equilibria}, ~\ref{prop:brian:1}, ~\ref{prop:brian:2}, ~\ref{prop:2-coexistence} and ~\ref{prop:plane:equilibrium} were not provided in \cite{gracy2022trivirus}.
  

\subsection*{Paper Outline}
The paper unfolds in the following manner: We use the rest of this section to collect notations used in the sequel. The formal presentation of the tri-virus model and the questions of interest in the present paper are provided in Section~\ref{sec:prob:formulation}. We investigate the monotonicity of the tri-virus system in Section~\ref{sec:trivirus:monotone}, whereas Section~\ref{sec:finiteness:of:equiibria} deals with the question of the finiteness of the number of equilibria (in a generic sense). Note that the proof for the result in Section~\ref{sec:finiteness:of:equiibria} is relegated to the Appendix since it uses different tools compared to the rest of the results in the paper. Section~\ref{sec:persistence:one:virus} presents a necessary and sufficient condition for local exponential stability of the boundary equilibrium. Sections~\ref{sec:nonexistence:coexistence:equilibria} and~\ref{sec:existence:coexistence} deal with the nonexistence and existence of various kinds of coexistence equilibria. Certain special setting(s), all of which have measure zero, that yield a line (resp. plane) of coexistence equilibria, %along with their stability properties 
are provided in Section~\ref{sec:line:attractivity} (resp. Section~\ref{sec:global:plane}). Simulations illustrating our theoretical findings are provided in Section~\ref{sec:simulations}. A summary of the paper, along with certain future research directions, is provided in Section~\ref{sec:conclusions}.

\subsection*{Notations}
We denote the set of real numbers by $\mathbb{R}$ and the set of nonnegative real numbers by $\mathbb{R}_+$. For any positive integer $n$, we use $[n]$ to denote the set $\{1,2,...,n\}$. The $i^{\rm{th}}$ entry of a vector $x$ is denoted by $x_i$. The element in the $i^{\rm{th}}$ row and $j^{\rm{th}}$ column of a matrix $M$ is denoted by $M_{ij}$. We use $\textbf{0}$ and $\textbf{1}$ to denote the vectors whose entries all equal $0$ and $1$, respectively, and use $I$ to denote the identity matrix, while the sizes of the vectors and matrices are to be understood from the context. For a vector $x$ we denote the square matrix with $x$ along the diagonal by $\diag(x)$. For any two real vectors $a, b \in \mathbb{R}^n$ we write $a \geq b$ if $a_i \geq b_i$ for all $i \in [n]$, $a>b$ if $a \geq b$ and $a \neq b$, and $a \gg b$ if $a_i > b_i$ for all $i \in [n]$. Likewise, for any two real matrices $A, B \in \mathbb{R}^{n \times m}$, we write $A \geq B$ if $A_{ij} \geq B_{ij}$ for all $i \in [n]$, $j \in [m]$, and $A>B$ if $A \geq B$ and $A \neq B$,  and $A \gg B$ if $A_{ij} > B_{ij}$ for all $i \in [n]$, $j \in [m]$.
For a square matrix $M$, we use $\sigma(M)$ to denote the spectrum of $M$, $\rho(M)$ to denote the spectral radius of $M$, and $s(M)$ to denote the largest real part among the eigenvalues of $M$, i.e., $s(M) = \max\{\rm{Re}(\lambda) : \lambda \in \sigma(M)\}$. 
\par A square matrix $A$ is said to be Hurwitz if $s(A)<0$.
A real square matrix $A$ is called Metzler if all its off-diagonal entries are nonnegative. A real square matrix $A$ is said to be a Z-matrix if all of its off-diagonal entries are nonpositive. %All eigenvalues of an
A Z-matrix is an M-matrix if all its eigenvalues have nonnegative real parts. An M-matrix is singular or nonsingular if it has an eigenvalue at the origin or if all its eigenvalues have strictly positive parts. The matrix $A$ is positive semidefinite if $x^\top A x \geq 0$ for all vectors $x$; we denote this by $A\succeq 0$.

\section{Problem Formulation}\label{sec:prob:formulation}
This section presents a model representing the possible spread of multiple competing viruses over a population network. Thereafter, we state the requisite assumptions and definitions needed for the theoretical developments of this paper. Lastly, we will formally state the problems of interest.
% In this section, we detail a model that captures the spread of multiple competing viruses  across a population network. % with a shared resource. %First we establish the premise of the model, from which model variables and dynamics are derived. 
% Subsequently, we detail the  pertinent assumptions and definitions that will be required in the sequel. Finally, we formally specify the problems being investigated.

\subsection{Model}
We consider  a network of $n\geq 2$ %agents,
nodes\footnote{As an aside, multivirus SIS models where $n=1$ have been studied in, among others, \cite[Section~2]{pare2020modeling}.}. Each node represents a well-mixed population of individuals (i.e., any two individuals in the population can interact with the same positive probability) with a large and constant size. One of the central assumptions underpinning this model is that of homogeneity within the population node and (possible) heterogeneity outside the population node. That is, all individuals within a population node have the same infection (resp. healing) rates, but individuals in different population nodes need not necessarily have the same healing (resp. infection) rate~\cite{lajmanovich1976deterministic}. We suppose that $m$ viruses compete with each other to infect the nodes. The fact that the viruses are competing implies that at least two (but possibly more) viruses are simultaneously  circulating in the population.


%\seb{
The case where $m=1$ (meaning there is no competition) has been well-studied in the literature; see, for instance, \cite{khanafer2016stability,lajmanovich1976deterministic,fall2007epidemiological,van2008virus}, whereas the case where $m=2$ (i.e., two competing viruses) has been explored relatively well in recent times, see, for example, \cite{sahneh2014competitive,santos2015bi,ye2021convergence,liu2019analysis,pare2021multi}. %Nonetheless, an overarching analysis accounting for $m$ competing viruses, where $m$ is some arbitrary but finite positive integer, is, to the best of our knowledge, not available in the existing literature. Therefore, as a first step, in this direction, the present paper deals with the case where $m=3$. That is, in the rest of this paper we will consider three competing viruses. 
This paper, for reasons mentioned in Section~\ref{sec:introduction}, focuses on the case where $m=3$.
Hence, within each population node, individuals can be partitioned into four mutually exclusive health compartments: susceptible, infected with virus~1, infected with virus~2, and infected with virus~3. Note that  no individual can be \emph{simultaneously} infected by more than one virus \cite{laurie2018evidence,wu2020interference}. A population node is termed \emph{healthy} if all individuals in said node belong to the susceptible compartment; otherwise, we say it is infected. An individual belonging to population node $i$ (where $i\in [n]$), in the susceptible compartment, as a consequence of coming into contact with its infected neighbors, transitions to the ``infected with virus $k$" (for $k \in [m]$) compartment at a rate $\beta_i^k > 0$. An individual in population~$i$ that is infected with virus~$k$ recovers from it 
based on %its 
said individual's
healing rate with respect to virus~$k$, i.e.,~$\delta_i^k > 0$.
%}
% Consider a network of $n\geq 2$ %agents,
% nodes\footnote{As an aside, multivirus SIS models where $n=1$ have been studied in, among others, \cite[Section~2]{pare2020modeling}.}, where $m$ viruses compete with each other to infect the %agents. 
% nodes. The notion of competition implies the presence of at least two (but possibly more) viruses. %Therefore, 
% Throughout this paper, $m = 3$.
% In context, each node %agent 
% %(hereafter, interchangeably referred to as nodes) 
% represents a well-mixed population of individuals with a large and constant size. A well-mixed population means any two individuals in the population can interact with the same positive probability. A key assumption that underpins this model is that of homogeneity within the population node, and (possible) heterogeneity outside the population node. That is, all individuals within a population node have the same infection (resp. healing) rates, but individuals in different population nodes need not necessarily have the same healing (resp. infection) rate\cite{lajmanovich1976deterministic}.
%an agent could be thought of as an individual or, equivalently, as a subpopulation of individuals (for instance, a district in a big city). 
% An agent, at any given time, could be either infected by one of the $m$ viruses or none; no agent can be infected by more than one virus simultaneously. If, at a given time, none of the individuals in a subpopulation are infected, we say that the agent is healthy; otherwise, we say it is infected. An agent that is healthy could, depending on its infection rate $\beta_i^k$, get infected with virus~$k$ (for $k \in [m]$) as a consequence of coming in direct contact with an infected individual in  a neighboring subpopulation. An agent that is infected with virus~$k$ recovers from it 
% based on %its 
% said agent's
% healing rate with respect to virus~k, i.e., $\delta_i^k$.

% Within each population, individuals can be partitioned into four mutually exclusive health compartments: susceptible, infected with virus 1, infected with virus 2, infected with virus~3. More precisely,  no individual can be \emph{simultaneously} infected by more than one virus \cite{laurie2018evidence,wu2020interference}.
% We say a population node is healthy if all individuals belong to the susceptible compartment; otherwise we say it is infected.
% An individual, belonging to population $i$ (where $i\in [n]$), in the susceptible compartment, can transition to the ``infected with virus $k$" (for $k \in [m]$) compartment at a rate $\beta_i^k > 0$. An individual in population~$i$ that is infected with virus~$k$ recovers from it 
% based on %its 
% said individual's
% healing rate with respect to virus~$k$, i.e., $\delta_i^k > 0$.


\par We model the spread of $m$-competing viruses using an $m$-layer graph $G$; the vertices of the graph represent the population nodes. The contact  graph for the spread of virus~$k$, for each $k \in [m]$, is denoted by the $k^\textrm{th}$ layer. More specifically, for the  graph $G$, there exists a directed edge from node $j$ to node $i$ in layer $k$ if, assuming an individual in population $j$ is infected with virus~$k$, then said individual can infect at least one (but possibly more) healthy individual in node~$i$. The edge set corresponding to the $k^\textrm{th}$ layer of $G$ is denoted by $E^k$, whereas $ A^k$ denotes the weighted adjacency matrix corresponding to layer~$k$ (where $a_{ij}^k \geq 0$). The elements in $A^k$ are in one-to-one correspondence with the existence (or lack thereof) of edges in layer $k$.  That is,  $(i,j)\in E^k$ if, and only if, $a_{ji}^k\neq 0$. We use $x_i^k(t)$ to represent the fraction of individuals infected with virus~$k$ in population~$i$ at time instant $t$. Given that the viruses are circulating in the population nodes, the infection level $x_i^k(t)$ possibly changes over time. 
% \par The spread of $m$-competing viruses can be modeled using an $m$-layer graph $G$, where the vertices of the graph represent the population nodes. The $k^\textrm{th}$ layer denotes the contact graph for the spread of virus~$k$, for each $k \in [m]$. 
% More specifically, for the  graph $G$, there exists a directed edge from node $j$ to node $i$ in layer $k$ if, assuming an individual in population $j$ is infected with virus~$k$, then said individual can infect at least one (but possibly more) healthy individual in node~$i$. Let $E^k$ denote the edge set corresponding to the $k^\textrm{th}$ layer of $G$.
% We denote by $A^k$ (where $a_{ij}^k \geq 0$) the weighted adjacency matrix corresponding to layer~$k$, with the elements in $A^k$ being in one-to-one correspondence with the existence (or lack thereof) of edges in layer $k$.  That is,  $(i,j)\in E^k$ if, and only if, $a_{ji}^k\neq 0$. Let $x_i^k(t)$ denote the fraction of individuals infected with virus~$k$ in population~$i$ at time instant $t$. 
Hence,  the evolution of said fraction can, then, be represented by the following scalar differential equation \cite[Equation~4]{pare2021multi}:
\begin{equation} \label{eq:scalar}
   %\dot{x}_i^k(t) =  \Big{(} \big{(} I - \textstyle \sum_{l=1}^m \diag(x^l(t)) \big{)} B^k  \Big{)} x^k(t), \\
   \dot{x}_i^k(t) = - \delta_i^k x_i^k(t) + \big{(} 1 - \textstyle \sum_{l=1}^m x_i^l(t) \big{)} %\nonumber \\ 
  %\times \big{(} \beta_{iw}^k z^k(t) +
  \textstyle \sum_{j=1}^{n} \beta_{ij}^k x_j^k(t),
  \end{equation}
where $\beta_{ij}^k = \beta_i^ka_{ij}^k$.

Define $x^k(t) = [x_1^k(t), \hdots, x_n^k(t)]^\top$, $D^k =\diag{(\delta_i^k)}$, and $B^k=[\beta_{ij}^k]_{n \times n}$. Therefore, the $n$ coupled equations in~\eqref{eq:scalar} for $i \in [n]$
 %Therefore, ~\eqref{eq:scalar} 
can be written as
\begin{equation} \label{eq:vec}
   \dot{x}^k(t) =  \Big{(} - D^k+ \big{(} I - \textstyle \sum_{l=1}^m \diag(x^l(t)) \big{)} B^k  \Big{)} x^k(t), 
  \end{equation}
Defining $x(t):=[x^1(t),  \dots , x^m(t)]^T$,  and  \break  $R^k(x(t)) := \big{(} - D^k + (I - \textstyle \sum_{l=1}^m \diag(x^l(t)))B^k \big{)}$, the dynamics of the system of all $m$ viruses are given by %\scriptsize 
\begin{gather} \label{eq:full}
 \dot{x}(t)
 =
 \begin{bmatrix}
 R^1 \big{(} x(t) \big{)} & 0 & \dots & 0 \\
 0 & R^2 \big{(} x(t) \big{)} & \dots & 0 \\
 \vdots & \vdots & \ddots & \vdots \\
 0 & 0 & \dots & R^m \big{(} x(t) \big{)}
 \end{bmatrix}
 x(t).
\end{gather}  
  
% Note that by setting $m=1$ in~\eqref{eq:full}, one recovers the classic single-virus SIS model that has been studied extensively in the literature; see \cite{van2008virus,khanafer2016stability,lajmanovich1976deterministic}. Setting $m=2$ yields the classic networked bi-virus SIS model, for which a plethora of results have been provided in \cite{liu2019analysis,sahneh2014competitive,santos2015bi,castillo1989epidemiological,ben:lcss,ye2021convergence,anderson2022equilibria}.
% In this paper, we are interested in the case when $m=3$, i.e., the tri-virus networked competitive spread.

% Consider a system of $m$-competitive SIS viruses. The dynamics of the $k^\textrm{th}$ virus, where $k =1,2,\hdots,m$, is as follows:
% \begin{equation} \label{eq:vec}
%   \dot{x}^k(t) =  \Big{(} \big{(} I - \textstyle \sum_{l=1}^m \diag(x^l(t)) \big{)} B^k - D^k \Big{)} x^k(t), 
%   \end{equation}
%   where $x^k(t)$ denotes the infection level with respect to virus $k$ at time $t$; $D^k =\diag{(\delta_i^k)}$; and $B^k = \bar{B}^k.A^k$ with $\bar{B}^k =\diag{(\beta_i^k)}$. The terms $\delta_i^k $ (resp. $\beta_i^k$) denote the infection rate (resp. healing rate) of agent $i$ with respect to virus $k$, while $A^k$ denotes the adjacency matrix that represents how virus $k$ spreads between different nodes of a given population.  %where $k=1,2,\hdots,m$.
% Defining $x(t):=[x^1(t),  \dots , x^m(t)]^T$,  and  $A^k(x(t)) := \big{(} - D^k + (I - \textstyle \sum_{l=1}^m \diag(x^l(t)))B^k \big{)}$, the dynamics of the system of all $m$ viruses are given by %\scriptsize 
% \begin{gather} \label{eq:full}
%  \dot{x}(t)
%  =
%  \begin{bmatrix}
%  A^1 \big{(} x(t) \big{)} & 0 & \dots & 0 \\
%  0 & A^2 \big{(} x(t) \big{)} & \dots & 0 \\
%  \vdots & \vdots & \ddots & \vdots \\
%  0 & 0 & \dots & A^m \big{(} x(t) \big{)}
%  \end{bmatrix}
%  x(t).
% \end{gather}


\normalsize
%Observe that for the case when $m=2$, system~\eqref{eq:full} is monotone \cite{ye2021convergence}. That is, supposing $x(0)$ and $y(0)$ are initial states of system~\eqref{eq:full} such that  $x(0) \leq y(0)$ then $x(t) \leq y(t)$ for all $t$. It is not known if the same can be said for $m=3$. Consequently, understanding the limiting behavior of tri-virus systems remains open. %This note aims at addressing some of the challenges in this regard.
\par Define, for $k \in [3]$, $X^k=\diag(x^k)$. Based on~\eqref{eq:vec}, the dynamics of the tri-virus system can be written as follows:

\begin{align} 
   \dot{x}^1(t) &=  \Big{(} \big{(} I - (X^1+X^2+X^3) \big{)} B^1 - D^1 \Big{)} x^1(t), \label{eq:x1}\\
      \dot{x}^2(t) &=  \Big{(} \big{(} I - (X^1+X^2+X^3) \big{)} B^2 - D^2 \Big{)} x^2(t) \label{eq:x2}\\
         \dot{x}^3(t) &=  \Big{(} \big{(} I - (X^1+X^2+X^3) \big{)} B^3 - D^3 \Big{)} x^3(t). \label{eq:x3}
  \end{align}
 
\subsection{Assumptions} %and Preliminary Lemmas}  
 We need the following assumptions to ensure that the aforementioned model is well-defined. Further, note that these assumptions are standard in the literature on (multi-competitive) networked SIS models; see, for instance, \cite{lajmanovich1976deterministic,fall2007epidemiological,khanafer2016stability,liu2019analysis}.
 \begin{assm} \label{assum:base}
Suppose that $\delta_i^k>0,  \beta_{ij}^k \geq 0$  for all $i, j \in [n]$ and $k \in [3]$. 
%~$\blacksquare$ 
\end{assm}
%
 
\begin{assm}\label{assum:irreducible}
The matrix $B^k$, for $ k \in [3]$ is irreducible. \end{assm} 
 
Under Assumption~\ref{assum:base},  for all $k \in [3]$, $B^k$ is a nonnegative matrix, and $D^k$ is a positive diagonal matrix. %(and is therefore invertible). %Thus, $(B_w^k - D_w^k)$ is a Metzler matrix, for all $k \in [m]$. 
%Moreover, recall that a square matrix $M$ is said to be irreducible if, replacing the non-zero elements of $M$ with ones and interpreting it as an adjacency matrix, the corresponding graph is strongly connected. 
Moreover, recall that a square nonnegative matrix $M$ has the irreducibility property if, and only if, supposing $M$ is the  (un)weighted adjacency matrix of a graph, the %corresponding 
said graph is strongly connected.
Then, noting that non-zero elements in $B^k$ represent directed edges in the set $E^k$, we see that $B^k$ is irreducible whenever the $k^{\rm{th}}$ layer of the multi-layer network $G$ is strongly connected. %when $B_w^k$ is interpreted as an adjacency matrix representing the set of directed edges $E^k$, $B_w^k$ is said to be irreducible if and only if the $k^{\rm{th}}$ layer of the multi-layer network $G$ is strongly connected.

\par  As a consequence of %these assumptions
Assumption~\ref{assum:base}, %we can
% will be able to 
%restrict 
our analysis of the model in~\eqref{eq:x1}-\eqref{eq:x3} is limited to the sets $\mathcal{D}:= \{x(t): x^k(t) \in [0,1]^n,  \forall k \in [3], \sum_{k=1}^{3}x^k \leq \textbf{1}\}$ and $\mathcal{D}^k:= \{x^k(t) \in [0,1]^n\}$. %for system~\eqref{eq:full} and~\eqref{eq:yk}, respectively. 
%\axel{\textbf{[It would be prudent to restrict S further, so that the sum of all the $p^k(t)$ has be less than one in every component.]}} 
Given that $x_i^k(t)$ is interpreted as a fraction of a population, %and $z^k(t)$ is a nonnegative quantity,
the aforementioned sets represent the sensible domain of the %model 
system. 
That is, %, for any $k \in [m]$, 
if $x^k(t)$ takes values outside of $\mathcal{D}^k$, then those values will not correspond to physical reality. %would lack physical meaning. 
The following lemma shows that $x(t)$ never leaves the set $\mathcal{D}$.
%
%In keeping with this, we can establish the following result for these sets:
%
\begin{lem}{\cite[Lemma~1]{pare2021multi}}\label{lem:pos}
Let Assumption~\ref{assum:base} hold. Then $\mathcal{D}$ is positively invariant with respect to~\eqref{eq:full}.%, and if $y(t) \in \mathcal{D}$ for $t \geq 0$, $\mathcal{D}^k$ is positively invariant with respect to~\eqref{eq:yk} for all $k \in [m]$.
%~$\blacksquare$
\end{lem} 

\begin{lem}\cite[Lemma~7]{axel2020TAC} \label{lem:pos_never_zero}
Let Assumption~\ref{assum:base} hold. Then $\mathcal{D} \setminus \{ \textbf{0} \}$ is positively invariant with respect to system~\eqref{eq:x1}-\eqref{eq:x3}.
\end{lem}

\subsection{Various Equilibria of the Model}\label{ssec:equilibria_types}
Clearly,  $(\textbf{0}, \textbf{0},\textbf{0})$ is an equilibrium of~\eqref{eq:x1}-\eqref{eq:x3}, and is referred to as the DFE. %disease-free equilibrium (DFE). %We now recall
A sufficient condition for global exponential stability (GES) of the DFE is as follows:
\begin{prop}\cite[Theorem~1]{axel2020TAC}\label{prop:exp:convergence}
Consider system~\eqref{eq:x1}-\eqref{eq:x3} under Assumption~\ref{assum:base}. If $s(-D^k+B^k) < 0$, for each $k \in [3]$, then the DFE  is exponentially stable, with a domain of attraction containing~$\mathcal{D}$.
\end{prop}

Clearly, the conditions in Proposition~\ref{prop:exp:convergence} also imply asymptotic convergence to the DFE. However, even when the strict inequalities in the %said
aforementioned proposition are %relaxed 
weakened to allow equality,   asymptotic convergence to the DFE can still be achieved.
%It turns out that by relaxing the strict inequality in Proposition~\ref{prop:exp:convergence}, one can still achieve asymptotic convergence to the DFE. 
%This is formalized in 
The next proposition, which is  a generalization of an analogous result for the bivirus setting (see \cite[Theorem~1]{liu2019analysis}), formalizes the same.
% A sufficient condition for global asymptotic stability (GAS) of the DFE, which is a generalization of an analogous result for the bivirus setting (see \cite[Theorem~1]{liu2019analysis}) is as follows. 
\begin{prop}\cite[Lemma~2]{pare2021multi}
Consider system~\eqref{eq:x1}-\eqref{eq:x3} under Assumption~\ref{assum:base}. If $s(-D^k+B^k) \leq 0$, for each $k \in [3]$, %$s(-D^2+B^2) \leq 0$ and $s(-D^3+B^3) \leq 0$,
then the DFE is the unique equilibrium of system~\eqref{eq:x1}-\eqref{eq:x3}. Moreover, it is asymptotically stable with the domain of attraction $\mathcal D$.\label{prop:phil}
\end{prop}

%While Proposition~\ref{prop:phil} identifies a sufficient condition for asymptotic convergence to the DFE, the next proposition identifies a sufficient condition for \emph{exponential convergence} to the DFE.

% \begin{prop}\cite[Theorem~1]{axel2020TAC}\label{prop:exp:convergence}
% Consider system~\eqref{eq:x1}-\eqref{eq:x3} under Assumption~\ref{assum:base}. If $s(-D^k+B^k) < 0$, for each $k \in [3]$, then the DFE  is exponentially stable, with domain of attraction containing~$\mathcal{D}$.
% \end{prop}

Note that Proposition~\ref{prop:exp:convergence} and~\ref{prop:phil} provide guarantees on convergence to the DFE. It is natural to ask what happens if, for some $k \in [m]$, one of the eigenvalues of the matrix $-D^k+B^k$ has a positive real part. It turns out for each $k \in [m]$, that violates the eigenvalue condition in Proposition~\ref{prop:phil}, there exists an equilibrium of the form $(\textbf{0}, \dots, \Tilde{x}^k, \dots, \textbf{0})$ in $\mathcal{D}$. %where $\Tilde{x}^k$ is the single-virus endemic equilibrium corresponding to virus~$k$.
 The following proposition formalizes this.

%It turns out that for every eigenvalue condition in Proposition~\ref{prop:phil} that is violated, one obtains an equilibrium of the form $(\textbf{0}, \dots, \Tilde{x}^k, \dots, \textbf{0})$ in $\mathcal{D}$, where $\Tilde{x}^k$ is the single-virus endemic equilibrium corresponding to virus~$k$. This is formalized in the following proposition.
%if each of the eigenvalue conditions in Proposition~\ref{prop:phil} were to be violated, then the tri-virus system has at least $4$ equilibria, as we recall in the following proposition.
\begin{prop} %\cite{liu2019analysis,axel2020TAC}
\cite[Theorem~2.1]{fall2007epidemiological} \label{prop:necessity}
Consider system~\eqref{eq:x1}-\eqref{eq:x3} under Assumptions~\ref{assum:base} and~\ref{assum:irreducible}. For each $k \in [3]$, %such that $B^k$ is irreducible, 
there exists a unique %single-virus endemic 
equilibrium $(\textbf{0}, \dots, \Tilde{x}^k, \dots, \textbf{0})$ in $\mathcal{D}$, with $\textbf{0} \ll \Tilde{x}^k \ll \textbf{1}$ if, and only if, $s(B^k - D^k) > 0$.

%and $s(B^k - D^k) > 0$, there is a unique single-virus endemic equilibrium $(\textbf{0}, \dots, \Tilde{x}^k, \dots, \textbf{0})$ in $\mathcal{D}$, with $\textbf{0} \ll \Tilde{x}^k \ll \textbf{1}$.%~$\blacksquare$
\end{prop}
Indeed, it is straightforward to show that $\tilde x^k$ is the endemic equilibrium for the single virus system (i.e., $m=1$) defined by the pair of matrices ($D^k,B^k$).
Assuming $m=1$, analytic methods for computing the %single-virus 
endemic equilibrium\footnote{In the single-virus case, an endemic equilibrium, when it exists, is unique} have been provided in %\cite[Theorem~4.3]{mei2017epidemics_review} and 
\cite[Theorem~5]{van2008virus}. 

Any non-zero equilibrium in $\mathcal D$ is %called
referred to as an \emph{endemic} equilibrium. Endemic equilibria can be further classified as follows:
%The 
Equilibria of the form $(\textbf{0}, \dots, \Tilde{x}^k, \dots, \textbf{0})$ are referred to as the \emph{boundary equilibria}. The equilibria of the form $(\bar{x}^1, \bar{x}^2, \bar{x}^3)$, %with
where at least $\bar{x}^i$ and $\bar{x}^j$ ($i, j \in [3], i\neq j$) 
%being strictly positive vectors 
are nonnegative vectors with at least one positive entry in each of $\bar{x}^i$ and $\bar{x}^j$ are referred to as \emph{coexistence equilibria}. 
The coexistence equilibria can be further classified as i) 2-coexistence equilibria, which are  equilibria of the form $(\bar{x}^1, \bar{x}^2, \bar{x}^3)$ where  $\bar{x}^i$ and $\bar{x}^j$ for some ($i, j \in [3], i\neq j$) 
%being strictly positive vectors 
are nonnegative vectors with at least one positive entry in each of $\bar{x}^i$ and $\bar{x}^j$, and, for $k\neq i, k \neq j$ there holds $\bar{x}^k=\textbf{0}$, and ii) 3-coexistence equilibria, which are  equilibria of the form $(\bar{x}^1, \bar{x}^2, \bar{x}^3)$ where, for each $i \in [3]$,  $\bar{x}^i$ is a nonnegative vector, that has at least one positive entry.
Indeed, such vectors are in fact strictly positive; see \cite[Lemma~6]{axel2020TAC}. That is, at any endemic equilibrium, there is at least one individual in every population node who is infected.
%Lemma~\ref{lem:equi_non-zero_nonone}.
%are referred to as \emph{coexisting equilibria}.


Let  $J(x^1,x^2,x^3)$ denote the Jacobian matrix of system~\eqref{eq:x1}-~\eqref{eq:x3} for an arbitrary  point in the state space. For future reference, it is easily checked that  $J(x^1,x^2,x^3)$ is as given in~\eqref{jacob} below.

%\begin{figure*}[h!]
	{\noindent}
\begin{align}\label{jacob}
&J(x^1,x^2,x^3) = \\
&\scriptsize
\begin{bmatrix}
-D^1+(I-X^1-X^2-X^3)B^1-\diag(B^1x^1) & -\diag(B^1x^1)   & -\diag(B^1x^1)  \\
-\diag(B^2x^2) & -D^2+(I-X^1-X^2-X^3)B^2-\diag(B^2x^2)  & -\diag(B^2x^2)\\
 -\diag(B^3x^3)& -\diag(B^3x^3)& -D^3+(I-X^1-X^2-X^3)B^3-\diag(B^3x^3)  \end{bmatrix}\normalsize.\nonumber
\end{align}
%\capion*{}
%\end{figure*}

 
% \subsection{Monotone dynamical systems and competitive bivirus networked SIS models} 

% Observe that for the case when $m=2$, system~\eqref{eq:full} is, under Assumption~\ref{assum:irreducible}, monotone \cite[Lemma~3.3]{ye2021convergence} (and, assuming homogeneous recovery rates, also in \cite[Theorem~18]{santos2015bi}). 
%  That is, setting $m=2$ for system~\eqref{eq:full}, suppose that $(x_A^1(0), x_A^2(0))$ and $(x_B^1(0), x_B^2(0))$ are two initial conditions in $\textrm{int}(D)$ satisfying i) $x_A^1(0)>x_B^1(0)$ and ii) $x_A^2(0)<x_B^2(0)$. Since the bivirus system is monotone, it follows that, for all $t$, i) $x_A^1(t)\gg x_B^1(t)$ and ii) $x_A^2(t)\ll x_B^2(t)$. 
% Further, it has been shown that, for almost all choices of $D^i$, $B^i$, $i=1,2$, system~\eqref{eq:full} has a finite number of equilibria \cite[Theorem~3.6]{ye2021convergence}. Therefore, from \cite[Theorems~2.5 and~2.6]{smith1988systems} (or \cite[Theorem~9.4]{hirsch1988stability}), we know that for almost all initial conditions in $\mathcal D$, system~\eqref{eq:full} with $m=2$ converges to a stable equilibrium point. The set of initial conditions for which said convergence does not occur (in which case it is  either a)  already at an unstable equilibrium, assuming such an equilibrium exists, or b) in the stable manifold of an unstable equilibria, 
% or c) on a nonattractive limit cycle, or d) in the stable manifold of a nonattractive limit cycle) has measure zero. It is not known if an analogous statement is true for the case when $m=3$. Consequently, understanding the limiting behavior of tri-virus systems remains open if any of the eigenvalue conditions in Proposition~\ref{prop:phil} are violated.

% Further,  for the case when $m=2$, a sufficient condition for local exponential convergence to a boundary equilibrium has been identified in \cite[Theorem~3.10]{ye2021convergence}, whereas for the $m=3$ case no such condition has been identified. Likewise, certain special (nongeneric) scenarios  have been identified which lead to the existence of a continuum of coexistence equilibria; see \cite[Theorems~6 and~7]{liu2019analysis}. Improving upon these results, a broader scenario (but again nongeneric), that accounts for a larger class of parameters, has been identified which leads to not only the existence, but also local exponential attractivity,
% of a  continuum of coexistence equilibria; see \cite[Proposition~3.9]{ye2021convergence}. Analogous results for the $m=3$ case are as yet unavailable. 

 
 \subsection{Preliminary lemmas}

 \begin{lem} \label{lem:eigspec}
\cite[Proposition~1]{liu2019analysis} Suppose that $\Lambda$ is a negative diagonal matrix and $N$ is an irreducible nonnegative matrix. 
%Let $M = \Lambda+N$.
Let $M$ be the irreducible Metzler matrix $M = \Lambda+N$. 
Then, $s(M) < 0$ if and only if $\rho(-\Lambda^{-1} N) < 1, s(M)=0$ if and only if $\rho(-\Lambda^{-1} N) = 1$, and $s(M)>0$ if and only if, $\rho(-\Lambda^{-1} N) > 1$.%~$\blacksquare$
\end{lem}
%
We will also use the following variants of the Perron-Frobenius theorem for irreducible matrices.

\begin{lem} \label{lem:perron_frob}
\cite[Chapter 8.3]{meyer2000matrix} \cite[Theorem~2.7]{varga1999matrix} %(Perron-Frobenius Theorem) 
Suppose that $N$ is an irreducible nonnegative matrix. Then,
%
\begin{enumerate}[label=(\roman*)]
    \item $r = \rho(N)$ is a simple eigenvalue of $N$. \label{item:perfrob_simpleeig}
    \item There is an eigenvector $\zeta \gg \textbf{0}$ corresponding to the eigenvalue $r$. \label{item:perfrob_pos_exists}
    \item $x > \textbf{0}$ is an eigenvector only if $Nx = rx$ and $x \gg \textbf{0}$. %\axel{\textbf{[Not sure this needs to be mentioned]}} 
    \label{item:perfrob_pos_necess}
    \item If $A$ is a nonnegative matrix such that $A < N$, then $\rho(A) < \rho(N)$. \label{item:perfrob_matrix_ineq}%~$\blacksquare$
\end{enumerate}
%
\end{lem}
%
%Comment: Double-check whether we are indeed using all 4 items in Lemma~\ref{lem:perron_frob}]
\begin{lem} \label{lem:perron_frob_metz}
\cite[Lemma~2.3]{varga1999matrix} Suppose that $M$ is an irreducible Metzler matrix. Then $r = s(M)$ is a simple eigenvalue of $M$, and there exists a corresponding eigenvector  $\zeta \gg \textbf{0}$.%~$\blacksquare$
\end{lem}

% Let  $J(x^1,x^2,x^3)$ denote the Jacobian matrix of system~\eqref{eq:x1}-~\eqref{eq:x3} for an arbitrary  point in the state space. Therefore, $J(x^1,x^2,x^3)$ is as given in~\eqref{jacob}.
% \begin{align}\label{jacob}
% &J(x^1,x^2,x^3) = \\
% &\tiny
% \begin{bmatrix}
% -D^1+(I-X^1-X^2-X^3)B^1-\diag(B^1x^1) & -\diag(B^1x^1)   & -\diag(B^1x^1)  \\
% -\diag(B^2x^2) & -D^2+(I-X^1-X^2-X^3)B^2-\diag(B^2x^2)  & -\diag(B^2x^2)\\
% - \hat B^{3} & - \hat B^{3}& -D^3+(I-X^1-X^2-X^3)B^3-\diag(B^3x^3)  \end{bmatrix}\normalsize,\nonumber
% \end{align}

% \begin{figure*}[h!]
% 	{\noindent}
% \begin{align}\label{jacob}
% &J(x^1,x^2,x^3) = \\
% &\scriptsize
% \begin{bmatrix}
% -D^1+(I-X^1-X^2-X^3)B^1-\diag(B^1x^1) & -\diag(B^1x^1)   & -\diag(B^1x^1)  \\
% -\diag(B^2x^2) & -D^2+(I-X^1-X^2-X^3)B^2-\diag(B^2x^2)  & -\diag(B^2x^2)\\
%  -\diag(B^3x^3)& -\diag(B^3x^3)& -D^3+(I-X^1-X^2-X^3)B^3-\diag(B^3x^3)  \end{bmatrix}\normalsize,\nonumber
% \end{align}
% \end{figure*}
%where $\hat{B}^i=\diag(B^i\Tilde{x}^i)$, for $i=1,2,3$.
% \begin{align}
% J_{1,1} &= W B^{1} -D^{1} - \hat B^{1} -\epsilon^1 \hat B^2 \\
% J_{2,2} &= W B^{2} -D^{2} - \hat B^{2} -\epsilon^2 \hat B^1 \\
% J_{3,3} &= -D^1 - D^2 + \epsilon^1 \hat X^1 B^2 + \epsilon^2 \hat X^2 B^1.
% \end{align}

 
% Clearly, the DFE (i.e.,$(\textbf{0}, \textbf{0},\textbf{0})$) is an equilibrium of~\eqref{eq:x1}-\eqref{eq:x3}. %We now recall
% A sufficient condition for global asymptotic stability (GAS) of the DFE, which is a generalization of an analogous result for the bivirus setting is as follows. \cite[Theorem~1]{liu2019analysis}.
% \begin{prop}\cite[Lemma~2]{pare2021multi}
% Consider system~\eqref{eq:x1}-\eqref{eq:x3} under Assumption~\ref{assum:base}. If $s(-D^1+B^1) \leq 0$, $s(-D^2+B^2) \leq 0$ and $s(-D^3+B^3) \leq 0$, then the DFE is the unique equilibrium of system~\eqref{eq:x1}-\eqref{eq:x3}. Moreover, it is asymptotically stable with the domain of attraction $\mathcal D$.\label{prop:phil}
% \end{prop}
%\seb{
\begin{lem}[Restrictions on the trajectories of the tri-virus system] Consider system~\eqref{eq:x1}-\eqref{eq:x3} under Assumption~\ref{assum:base}. Suppose that the initial conditions satisfy a) $x^k(0) \gg \textbf{0}$ for $k \in [3]$, and ii) $(x^1(0), x^2(0), x^3(0)) \in \mathcal D$. Further, suppose that matrix $B^k$ for $k \in [3]$ is irreducible.  For all finite $t>0$, $\textbf{0} \ll x^k(t) \ll \textbf{1}$ for $k \in [3]$; and $x^1(t)+x^2(t)+x^3(t) \ll \textbf{1}$.\label{lem:inward:pointing}
\end{lem}
\par The proof closely follows that of \cite[Lemma~3.2]{ye2021convergence} for the bivirus problem. 
\par \textit{Proof:}  Define $z:=\textbf{1}-x^1-x^2-x^3$, and 
%recall that 
$\hat{B}^i:=\diag(B^ix^i)$. Hence,~\eqref{eq:x1}-\eqref{eq:x3} can be rewritten as:
\begin{align}\label{eq:trajectories}
    \dot{x}^i(t)&=-D^ix^i(t)+\hat{B}^iz(t), i=1,2,3 \nonumber\\
    \dot{z}(t)&= D^1x^1(t)+D^2x^2(t)+ D^3x^3(t)-[\hat{B}^1+\hat{B}^2+\hat{B}^3]z(t)
\end{align}
Suppose that for some $\tau \in \mathbb{R}_{>0}$, and for some $i \in [n]$, $z_i(\tau)=0$, which, since $z_i(\tau)=1-x_i^1(\tau)-x_i^2(\tau)-x_i^3(\tau)$, implies that either $x_i^1(\tau)\neq 0$ and/or $x_i^2(\tau)\neq 0$ and/or $x_i^3(\tau)\neq 0$. Therefore, since a) $D^1, D^2, \text{and } D^3$ are positive diagonal matrices, and b) since, by assumption, $z_i(\tau)=0$, it must be that
 $\dot{z}_i(\tau)>0$. %This implies that $z_i(\tau) > 0$, 
This implies that $z_i(t) > 0$ for all $t > \tau$. This argument holds for all $i$,
and hence we have that $z_i(t)>0$ for all $t$ and $i \in [n]$. As a result, we obtain $z(t) \gg \textbf{0}$ for all $t >\tau$, which implies that $x^1(t)+x^2(t)+x^3(t) \ll \textbf{1}$. Hence, we have $x^k(t) \ll \textbf{1}$ for $k \in [3]$.
\par Suppose that for some $t \in \mathbb{R}_{\geq 0}$, $\ell$ (where $\ell <n$) %positions in 
entries in
$x^1(t)$ equal zero, and let us label these %positions 
entries $i_1, i_2, \hdots, i_\ell$. Since, by assumption, the matrix $B^1$ is irreducible, it follows from the property of irreducible matrices that there is at least one entry %position 
in the vector $B^1x^1$ that is nonzero even though the corresponding %position 
entry in the vector $x^1$ equals zero \cite[Lemma~1]{liu2019analysis}. Let us assume that this occurs for the $i_1^{th}$ entry, i.e., $x^1_{i_1}=0$ yet $[B^1x^1]_{i_1}>0$. Observe that the evolution of the infection level for virus~1 in node $i_1$ is as follows:
\begin{align}\label{ineq:nozeros}
    \dot{x}^1_{i_1}(t)&=[-D^1x^1(t)]_{i_1} +[(I-X^1-X^2-X^3)B^1x^1(t)]_{i_1} \nonumber \\
    &>0,
\end{align}
where the inequality in~\eqref{ineq:nozeros} follows by noting that $x_{i_1}(t)=0$ by assumption, and $[B^1x^1]_{i_1}>0$ as discussed above. This means that there must exist some time instant $t^\prime$, with $t^\prime-t$ not too large, such that $x^1(t^\prime)$ has fewer than $\ell$ zero entries. Repeating the argument for all the other zero entries in the  vector $x^1(t^\prime)$ (and this can be done since the choice of node $i_1$ was arbitrary), we have that $x^1(t) \gg \textbf{0}$ for $t>0$. Analogously, we can prove that $x^2(t), x^3(t) \gg \textbf{0}$ for $t>0$.~\qed
%}
\par Lemma~\ref{lem:inward:pointing}  substantially limits the equilibria that can lie on the boundary of the set $\{x^1, x^2, x^3 \in \mathbb{R}_{\geq 0}\mid x^1+x^2+x^3 \leq \textbf{1}\}$.
 %\brian{I think we need the result also that at an equilibrium (and under the usual assumptions including $B^i$ irreducible), any equilibrium either has $x^k=0$ or $x^k\gg 0$. In my opinion, the proof is almost identical to the proof of the corresponding result for the bivirus case, which can be found in Lemma 3.1 of \cite{ye2021convergence}, and no detail need be given. I will assume a result along these lines has been inserted at this point as a lemma in constructing the argument about finiteness of the number of equilibria. }

It turns out that if $x = (x^1, \dots, x^3) \in \mathcal{D}$ is an equilibrium of~\eqref{eq:x1}-\eqref{eq:x3}, then a certain dichotomy arises for $x^k$, $k=1,2,3$. The following lemma formalizes said dichotomy.
 

\begin{lem}{\cite[Lemma~6]{axel2020TAC}} \label{lem:equi_non-zero_nonone:1}
Consider system~\eqref{eq:x1}-\eqref{eq:x3} under Assumptions~\ref{assum:base} and~\ref{assum:irreducible}. %Suppose that for each $i \in [n]$, $\sum_{j=1}^{n}B^1_{ij} >0$.%and~\ref{assum:irreducible}. %Suppose, for all $k \in [3]$, that $B^k$ is irreducible. 
 If $x = (x^1, \dots, x^3) \in \mathcal{D}$ is an equilibrium of~\eqref{eq:x1}-\eqref{eq:x3}, then, for each $k \in [3]$, either $x^k = \textbf{0}$, or $\textbf{0} \ll x^k \ll \textbf{1}$. Moreover, %we have that 
$\textstyle \sum_{k=1}^3 x^k \ll \textbf{1}$.
\end{lem}

%\cite[Proposition~9]{ye2021convergence}

\subsection{Problem Statements} 

The focus of the present paper is  to answer the following questions:
\begin{enumerate}[label=\roman*)]
  \item Is the tri-virus system monotone?
 \item Does the tri-virus system generically admit a finite number of equilibria?
    \item Can we identify a sufficient condition for local exponential convergence to a boundary equilibrium? 
    \item Can we identify a sufficient condition for the nonexistence of 3-coexistence equilibria (resp. specific forms of 2-coexistence equilibria)?
\item Can we identify a sufficient condition for the existence of a 3-coexistence equilibrium?
    \item Can we identify a sufficient condition for the existence and local attractivity of a line of coexistence equilibria?
    \item Can we identify special case(s) where, irrespective of the non-zero initial infection levels, the tri-virus dynamics converge to a plane of coexistence equilibria?
\end{enumerate}



\section{Monotonicity (or lack thereof) of the tri-virus system} \label{sec:trivirus:monotone}
Note that the system defined by \eqref{eq:x1}-\eqref{eq:x3} is a (networked) nonlinear system. A particular class of nonlinear systems is \emph{monotone dynamical systems} (MDS). For a detailed overview of MDS, see, for instance, \cite{smith2008monotone}. Before investigating whether or not the tri-virus system is monotone, we detail the importance of MDS in the context of the competitive bi-virus SIS networked model.

%In this section, we seek to conclusively answer whether (or not) system~\eqref{eq:full} with $m=3$ is a monotone dynamical system (MDS). \seb{For a detailed overview of monotone systems, see, for instance, \cite{smith2008monotone}. Before answering the question, we detail the importance of the notion of MDS in the context of multi-competitive SIS networked models}.

\subsection{Monotone dynamical systems and competitive bivirus networked SIS models} 

% Note that for the case when $m=2$, system~\eqref{eq:full} is monotone \cite[Lemma~3.3]{ye2021convergence} (and, assuming homogeneous recovery rates, also in \cite[Theorem~18]{santos2015bi}). That is, setting $m=2$ for system~\eqref{eq:full}, supposing $x(0)$ and $y(0)$ are initial states %of system~\eqref{eq:full}
% such that  $x(0) \leq y(0)$, then $x(t) \leq y(t)$ for all $t$. Given that for almost all choices of $D^i$, $B^i$, $i=1,2$, system~\eqref{eq:full} has a finite number of equilibria, \seb{then, because system~\eqref{eq:full} with $m=2$ is monotone,} it follows \seb{from \cite[Theorems~2.5 and~2.6]{smith1988systems}} that for almost all initial conditions in $\mathcal D$, system~\eqref{eq:full} with $m=2$ converges to a stable equilibrium point; the set of initial conditions for which said convergence does not occur (in which case it is on a nonattractive limit cycle) has measure zero \cite[Theorem~3.6]{ye2021convergence}. It is not known if the same can be said for $m=3$. Consequently, understanding the limiting behavior of tri-virus systems remains open.

Notice that if $m=2$, then, under Assumption~\ref{assum:irreducible}, system~\eqref{eq:full} is monotone; see \cite[Lemma~3.3]{ye2021convergence} (and, for $k=1,2$, assuming $\delta_i^k=1$ for $i \in [n]$, also  \cite[Theorem~18]{santos2015bi}). 
 That is, setting $m=2$ for system~\eqref{eq:full}, suppose that $(x_A^1(0), x_A^2(0))$ and $(x_B^1(0), x_B^2(0))$ are two initial conditions in $\textrm{int}(D)$ satisfying i) $x_A^1(0)>x_B^1(0)$ and ii) $x_A^2(0)<x_B^2(0)$. Since the bivirus system is monotone, it follows that, for all $t \in \mathbb{R}_{\geq 0}$, i) $x_A^1(t)\gg x_B^1(t)$ and ii) $x_A^2(t)\ll x_B^2(t)$. 
%Independent of the fact that the bivirus system is monotone, 
Additionally, it is also known that %it has been shown that, 
for almost all \footnote{The term \enquote{almost all} has a precise mathematical meaning: for all but a set of parameter values that has measure zero. This set of exceptional values is defined by an algebraic or semi-algebraic set.} choices of $D^i$, $B^i$, $i=1,2$, system~\eqref{eq:full} has a finite number of equilibria \cite[Theorem~3.6]{ye2021convergence}. Therefore, from \cite[Theorems~2.5 and~2.6]{smith1988systems} (or \cite[Theorem~9.4]{hirsch1988stability}), we know that for almost all initial conditions in $\mathcal D$, system~\eqref{eq:full} with $m=2$ converges to a stable equilibrium point, assuming such an equilibrium  exists. There are initial conditions for which convergence to a stable equilibrium does not occur. In such cases, said initial condition is  either %a)  already at an unstable equilibrium, assuming such an equilibrium exists, or 
a) 
in the stable manifold of an unstable equilibrium, 
%or c) on a nonattractive limit cycle, 
or b) in the stable manifold of a nonattractive limit cycle. The set of initial conditions for which either a) or b) %or c) or d) 
happens has measure zero. 

%The set of initial conditions for which said convergence does not occur (in which case it is  either a)  already at an unstable equilibrium, assuming such an equilibrium exists, or b) in the stable manifold of an unstable equilibrium, 
%or c) on a nonattractive limit cycle, or d) in the stable manifold of a nonattractive limit cycle) has measure zero. It is unknown if an analogous statement is true for the case when $m=3$. Consequently, understanding the limiting behavior of tri-virus systems remains partly open.}

\subsection{The tri-virus system is not monotone}\label{sec:tri:virus:loss:of:generality}
\par It is natural to ask how well the notion of MDS %generalizes 
applies to %for 
multi-competitive networked SIS epidemics; this subsection aims to answer this question conclusively.
%In order to answer the question of \seb{whether the tri-virus system is monotone} 
To this end, we construct a graph associated with the Jacobian \eqref{jacob} of system~\eqref{eq:x1}-\eqref{eq:x3}, say $\bar{G}$. 
% The construction follows the outline provided in \cite{sontag2007monotone}. More specifically,
% the graph $\bar{G}$ has $3$ nodes; with  node~$k$, $k=1,2,3$, representing the sum of the fractions across all population nodes that is infected with virus~$k$.
% %this can be partitioned into groups of $n$ nodes each, thus resulting in $3$ groups. 
% The edges of  $\bar{G}$ are based on the entries in the Jacobian matrix $J(x^1, x^2, x^3)$. Note that the Jacobian $J(x^1, x^2, x^3)$ is a block matrix; blocks along the diagonal represent in-group interactions, whereas all other blocks represent inter-group interactions. If $[J(x^1, x^2, x^3)]_{ij} \leq 0$ for  $i \neq j$, hen we draw an edge labelled with "-" sign;  if  $[J(x^1, x^2, x^3)]_{ij} \geq 0$ for  $i \neq j$, then we draw an edge labelled with "+" sign. Thus,  $\bar{G}$ is a signed graph. Note that $\bar{G}$ has no self-loops. As an aside, also observe that since $x^k(t)\geq 0$ for $k \in [3]$ and $t\in \mathbb{R}_+$, it is immediate that the sign of the elements in $J(x^1, x^2, x^3)$ do not change with the argument.
The construction follows the outline provided in \cite{sontag2007monotone}. More specifically,
the graph $\bar{G}$ has $3n$ nodes. The edges of  $\bar{G}$ are based on the entries in the Jacobian matrix $J(x^1, x^2, x^3)$ in \eqref{jacob}. Specifically, if $[J(x^1, x^2, x^3)]_{ij} <0$ for  $i \neq j$, then we draw an edge labelled with ``-" sign;  if  $[J(x^1, x^2, x^3)]_{ij} > 0$ for  $i \neq j$, then we draw an edge labelled with ``+" sign. Thus,  $\bar{G}$ is a signed graph. Note that $\bar{G}$ has no self-loops. As an aside, also observe that since $x^k(t)\geq 0$ for $k \in [3]$ and $t\in \mathbb{R}_+$, it is immediate that the sign of the elements in $J(x^1, x^2, x^3)$ do not change with the argument so that $\bar G$ %is independent 
is the same for all points in the interior of $\mathcal D$.  %{\color{red} I don't think a special notation like $\mathcal D^{\circ}$ has been proposed.}

\par We also need the following concept from graph theory. A signed graph is considered consistent if every undirected cycle in the graph has a net positive sign, i.e., it has an even number of ``-" signs \cite{sontag2007monotone}. We have the following result.
\begin{thm} \label{prop:tri-virus-not-monotone}
System~\eqref{eq:x1}-\eqref{eq:x3} is not monotone.
\end{thm}
\begin{proof}
%\textit{Proof:} 
Note that the Jacobian $J(x^1, x^2, x^3)$ is a block matrix, with all blocks along the off-diagonal being negative diagonal matrices. Pick any node $i$, where $i \in \{1,2, \hdots, n\}$. Observe that, since  all blocks along the off-diagonal of $J(x^1, x^2, x^3)$ are negative diagonal matrices, it is clear that there exists an edge from node $i$ to node $i+n$, an edge from node $i+n$ to node $i+2n$, and an edge from node $i+2n$ to node $i$. Furthermore, each of these edges has a ``-" sign. Hence, a loop starting from node $i$, traversing through nodes $i+n$, $i+2n$, and back to node $i$ is a 3-length cycle with an odd number of negative signs. Therefore, from \cite[page 62]{sontag2007monotone}, the signed graph $\bar{G}$ is not consistent. 
Consequently, from \cite[page 63]{sontag2007monotone}, it follows that the system~\eqref{eq:x1}-\eqref{eq:x3} is not monotone.\hfill \proofbox%~\qed
\end{proof}
% Since $[J(x^1, x^2, x^3)]_{ij} \leq 0$ for all $i \neq j$ with $i,j=1,2,3$, any (resp. every) edge from group $j$ to group $i$ has a "-" sign.  Note that a loop starting from group  $1$, traversing through groups  $2, 3$ and back to group $1$ is a 3-length cycle that has an odd number of negative signs. Hence, from \cite[page 62]{sontag2007monotone}, the signed graph $\bar{G}$ is not consistent. %which, due to \cite{harary1953notion}, further implies that $\bar{G}$ is not structurally balanced.
% Consequently, from \cite[page 63]{sontag2007monotone}, it follows that the system~\eqref{eq:x1}-\eqref{eq:x3} is not monotone.~$\blacksquare$

%drawn to represent the interactions \emph{between} the $3$ groups, and not within each group. That is, $\bar{G}$ has no self-loops.  
% Define $P^1:=\begin{bmatrix} I_n&&\textbf{0} && \textbf{0}\\
% \textbf{0}&& I_n&& \textbf{0}\\
% \textbf{0}&& \textbf{0} &&I_n 
% \end{bmatrix}$
% \seb{Nonetheless, due to the findings of Theorem~\ref{prop:tri-virus-not-monotone},
% %Even with a proof that the number of equilibria is finite for almost all choices of system parameters (i.e., healing rates and infection rates with respect to each of the three viruses),
% one cannot draw upon the rich literature on monotone dynamical systems (see\cite{smith1988systems}) to study the limiting behavior of system~\eqref{eq:x1}-\eqref{eq:x3}. In general, for non-monotone systems, no dynamical behavior, including chaos, can be definitively ruled out %without additional analysis
% \cite{sontag2007monotone}.}

\par %Theorem~\ref{prop:tri-virus-not-monotone} sheds light on a very interesting phenomenon, namely that the tri-virus system is not monotone. 
The conclusion of Theorem~1 offers several important and interesting insights, including key differences with bi-virus systems (m = 2). %the bivirus system is known to be monotone, see \cite{ye2021convergence}.}
%The fact that the tri-virus system is not monotone is in sharp contrast to the bi-virus setting, which is known to be monotone \cite{ye2021convergence}. 
The fact that a bivirus system is monotone coupled with the fact that for almost all choices of $D^k$, $B^k$, $k=1,2$, the bivirus system has a finite number of equilibria allows one to draw general conclusions on the limiting behavior of bivirus dynamical systems. One can thus focus on the characterization of equilibria given network parameters $D^i$, $B^i$, such as the number of equilibria and their regions of attraction; it is known that bivirus systems can have multiple stable endemic equilibria~\cite{anderson2022equilibria}. A consequence of Theorem~\ref{prop:tri-virus-not-monotone} 
is that one cannot draw upon the rich literature on monotone dynamical systems (see\cite{smith1988systems}) to study the limiting behavior of system~\eqref{eq:x1}-\eqref{eq:x3}. In general, for non-monotone systems, no dynamical behavior, including chaos, can be definitively ruled out %without additional analysis
\cite{sontag2007monotone}. Indeed, and in contrast to the bivirus system, the basic convergence properties of the trivirus system to endemic equilibria are not well understood.

Another possible consequence of the lack of monotonicity is as follows: It is  known that setting $D^k=I$ for $k \in [2]$ has no bearing on either the location of equilibria of system~\eqref{eq:full}
with $m=2$ nor on their (local) stability properties \cite[Lemma~3.7]{ye2021convergence}. That is, consider two bivirus systems, namely $\mathcal S$ and $\hat{\mathcal {S}}$, where $\mathcal S$ is defined by $(B^1,D^1, B^2, D^2)$ and $\hat{\mathcal {S}}$ is defined by $(\hat{B}^1 =(D^1)^{-1}B^1, \hat{D}^1=I,  \hat{B}^2 =(D^2)^{-1}B^2, \hat{D}^2=I)$. Then, the location of equilibria is the same for both bivirus systems. Moreover, local stability of an equilibrium in  bivirus system $\mathcal S$ implies, and is implied by,  that in bivirus system $\hat{\mathcal {S}}$.
%for a) bivirus systems, where $D^k=I$ and $B^k = (D^k)^{-1}B^k$ for $k \in [2]$, and b) bivirus systems, where $D^k$s are arbitrary positive diagonal matrices, the location of equilibria are the same for both bivirus systems a) and b). Plus, local stability of an equilibrium in  bivirus system a) implies, and is implied by,  that in bivirus system b).
%Note that Theorem~\ref{thm:init:condns} is reliant on the assumption that the healing rates for all agents with respect to all viruses are unity. 
For system~\eqref{eq:full} with $m=3$, by extending the arguments from \cite[Lemma~3.7]{ye2021convergence},  it is straightforward to show that the \emph{location} of the equilibria is the same  for an arbitrary system defined by $D^k, B^k$, $k \in [3]$ and the system defined by $\hat{D}^k=I$, and $\hat{B}^k=(D^k)^{-1}B^k$.
%  $k\in [3]$, $D^k=I$, %$B^k = (D^k)^{-1}B^k$,  
% and when $D^k$ %(resp. $B^k$)
%  are arbitrary positive diagonal %(resp. nonnegative) 
% matrices with the $D^k$s not necessarily being equal to each other. 
However, since the tri-virus system is not monotone, the arguments for the stability of equilibria in the proof of \cite[Lemma~3.7]{ye2021convergence} cannot be adapted. %\seb{especially for the 3-coexistence equilibria}. 
Hence, for the tri-virus case,  when the healing rates for all nodes with respect to all viruses are `scaled' in the manner above to become unity, the preservation of stability properties remains an open question.

As a matter of independent interest, in the next section, we ask whether %or not 
for almost all choices of $D^k$, $B^k$, $k=1,2,3$, the trivirus system has a finite number of equilibria. 


\section{Finiteness of equilibria for generic trivirus networks}\label{sec:finiteness:of:equiibria}
%\brian{
In this section, we %argue 
show that for generic tri-virus networks, i.e., for almost all  choices of $D^k,B^k, k=1,2,3$, the number of equilibria is finite, and the associated Jacobian matrices are nonsingular. %\seb{The term \enquote{almost all} has a precise mathematical meaning: for all but a set of parameter values that has measure zero. This set of exceptional values is defined by an algebraic or semi-algebraic set.}
For  bi-virus networks, an argument for the corresponding result based on algebraic geometry ideas was provided in \cite[Theorem~3.6]{ye2021convergence}. That argument becomes much more intricate for tri-virus systems. So an alternative proof is provided, based on a topological tool, termed the Parametric Transversality Theorem, see \cite[p.145]{lee2013introduction} and \cite[p.68]{guillemin2010differential}.
This tool has the advantage that it will apply to variations of the tri-virus dynamics, where the right side of \eqref{eq:scalar} is replaced by $-f(x_i^k) + (1-\sum x_i^k) g(x^1,x^2,x^3)$ for general nonlinear functions $f$ and $g$, such as those arising where feedback control is present, see, e.g., \cite{ye2021_PH_TAC}, or more refined models are constructed, see, e.g., \cite{yang2017bi,doshi2022convergence}.

% \seb{ By extending the algebraic geometry arguments in the proof of \cite[Theorem~3.6]{ye2021convergence}, it is relatively straightforward to show that even for the tri-virus system, for almost all choices of $D^k$, $B^k$, $k=1,2,3$, there exists a finite number of equilibria. Furthermore, by relying on arguments in \cite{anderson2022equilibria}, it can also be shown that for generic trivirus systems, the equilibria are hyperbolic.
%The details are omitted here in the interest of space. 
% Nonetheless, due to the findings of Theorem~\ref{prop:tri-virus-not-monotone},
% %Even with a proof that the number of equilibria is finite for almost all choices of system parameters (i.e., healing rates and infection rates with respect to each of the three viruses),
% one cannot draw upon the rich literature on monotone dynamical systems (see\cite{smith1988systems}) to study the limiting behavior of system~\eqref{eq:x1}-\eqref{eq:x3}. In general, for non-monotone systems, no dynamical behavior, including chaos, can be definitively ruled out %without additional analysis
% \cite{sontag2007monotone}.
\begin{prop}\label{prop:finite:equilibria}
For generic parameters $D^i$, $B^i$, with $i=1,2,3$, the trivirus system~\eqref{eq:x1}-\eqref{eq:x3} has a finite number of equilibria which are all nondegenerate, i.e. the associated Jacobian matrices are nonsingular. %{\color{cyan} Deletable remark: Insert could come from BA somewhere near here about the equilibria being generically hyperbolic, but this would seem peripheral to the main ideas of the paper and so has not been attempted at this point. } 
\end{prop}

The proof of this proposition is contained in the appendix, using as it does very different tools to the rest of the paper. The first component of the proof establishes the nondegeneracy claim. Further, we prove a slight modification, viz. that for any fixed set of $B^k$, almost all choices of the $D^k$ result in a finite number of equilibria. Proving that for fixed $D^k$, almost all choices of $B^k$ result in a finite number of equilibria is achieved with completely analogous calculations, but these calculations are not provided. Together, these two observations establish the proposition. Note that one \textit{cannot} replace the `almost all' requirement with an `all' requirement. As for the bivirus problem, there can exist special values of the parameters for which there is a continuum of equilibria; see Sections~\ref{sec:line:attractivity} and~\ref{sec:global:plane} for details.
%}. 

One of the practical ramifications of Proposition~\ref{prop:finite:equilibria} is recorded in the following remark.
\begin{rem}
The fact that, for almost all choices of infection and recovery rates, the  tri-virus networked SIS model has a finite number of equilibria means that assuming chaos and limit cycles are known to be not present, then,
 even in the absence of interventions for controlling the spread, the range of outcomes that health administration officials need to prepare for are limited (although not necessarily few). 
\end{rem}



%\subsection{Proof of Proposition \ref{prop:finite:equilibria} but to move to appendix by seb}

% The use below of the Parametric Transversality Theorem, the principal tool in the proof, requires the preliminary calculation of certain Jacobians. We will consider the case where the $B^i$ are fixed, to demonstrate that for almost all allowed $D^i$, equilibria are nondegenerate (i.e. the associated Jacobians are nonsingular). Given the bounded nature of the set of interest, viz. $\{\bf{0}}\leq x^k \leq \bf{1}\}\cap \{ x^1+x^2+x^3\leq {\bf{1}} \}$  and continuity of the Jacobian, finiteness of the number of zeros follows straightforwardly from the nondegeneracy conclusion. 

%  It is standard that there is a finite number of equilibria (viz. 1 or 2)), when two of the $x^i$ are zero, the equilibria being expressible using those of a single virus network. If precisely one of the $x^i$ is zero, the equilibria can be defined using a bivirus system, and again, the finiteness for generic $D^k$ is known from \cite{ye2021convergence}. It therefore remains to  separately consider coexistence equilibria with all $x^k$ nonzero, and in this case, as shown in {\color{blue} Lemma something proposed for end of section II}, we can assume $x^k\gg 0, k=1,2,3$.

% Without loss of generality, we will assume there is some fixed positive $\bar d<1$ such that all diagonal entries of the $D^i$ lie in $(\bar d,\bar d^{-1})$. Let $\mathcal D$ denote the manifold defined by the set of allowed $D^i$ consistent with the choice of $\bar d$.

% Let $\mathcal X$ denote the manifold $\{\bf{0}}\ll x^k \ll {\bf{1}}\}\cap \{x^1+x^2+x^3\ll {\bf{1}}\}$. Consider for any $x=[(x^1)^{\top},(x^2)^{\top},(x^3)^{\top}]^{\top}\in\mathcal X$ and $\delta=\mbox{vec}[D^1,D^2,D^3]\in\mathcal D$ the map
% \begin{align*}
% f_{\delta}:&\mathcal X\times\mathcal D\to\mathcal Y,\\ 
% (x,\delta)&\mapsto y=\begin{bmatrix}[-D^1+(I_n-X^1-X^2-X^3)B^1]x^1\\
% [-D^2+(I_n-X^1-X^2-X^3)B^2]x^2\\
% [-D^3+(I_n-X^1-X^2-X^3)B^3]x^3
% \end{bmatrix}
% \end{align*}
% % {\color{red} Change 20220906. I set up two lemmas in the following and eliminated extraneous calculations}
% We now claim:
% \begin{lem}
% With notation as above, the matrix $\frac{\partial f_{\delta}(x,\delta)}{\partial (x,\delta)}$ has full row rank at any coexistence equilibrium. 
% % \begin{small}
% % \begin{align}
% %     &\frac{\partial f_{\delta}(x,\delta)}{\partial (x,\delta)}\\\nonumber
% %     &=\begin{bmatrix}
% %  -D^1+(I-X^1-X^2)B^1-{\rm{diag}}(B^1x^1)&-{\rm{diag}}(B^1x^1)&X^1&0\\
% %   -{\rm{diag}}(B^2x^2)&-D^2+(I-X^1-X^2)B^2-{\rm{diag}}(B^2x^2)&0&X^2   
% %     \end{bmatrix}
% % \end{align}
% % \end{small}
% % and this matrix has full row rank at any coexistence equilibrium.
% \end{lem}

% %\begin{proof}
% \textit{Proof:}
% We first observe that
% \begin{small}
% \begin{align}\label{eq:jacxdelta}
%     \frac{\partial f_{\delta}(x,\delta)}{\partial (x,\delta)}&=\frac{\partial f_{\delta}(x^1,x^2,x^3,D^1,D^2,D^3)}{\partial (x^1,x^2,x^3,D^1,D^2,D^3)}\\
%     &=\begin{bmatrix}\left(\frac{\partial f_{\delta}(x^1,x^2,x^3,D^1,D^2,D^3)}{\partial (x^1,x^2,x^3)}\right)^{\top}\\\nonumber
%     \left(\frac{\partial f_{\delta}(x^1,x^2,x^3,D^1,D^2,D^3)}{\partial (D^1,D^2,D^3)}\right)^{\top}\\
%     \end{bmatrix}^{\top}
% %\end{equation}
% \end{align}
% \end{small}
% % By direct calculation we have 
% % \begin{small}
% % \begin{align*}
% %   & \frac{\partial f_{\delta}(x^1,x^2,D^1,D^2)}{\partial (x^1,x^2)}\\
% %   & \quad=\begin{bmatrix}  
% %   -D^1+(I-X^1-X^2)B^1-{\rm{diag}}(B^1x^1)&-{\rm{diag}}(B^1x^1)\\
% %   -{\rm{diag}}(B^2x^2)&-D^2+(I-X^1-X^2)B^2-{\rm{diag}}(B^2x^2)\end{bmatrix}
% % \end{align*}
% % \end{small}
% Observe next, using the diagonal nature of the $D^i$, that there holds (with $e_j$ denoting the $j$-th unit vector)
% \begin{align}
% \frac{\partial f_{\delta}(x^1,x^2,x^3,D^1,D^2, D^3)}{\partial \delta^1_j}&=x^1_je_j\\\nonumber 
% \frac{\partial f_{\delta}(x^1,x^2,x^3,D^1,D^2,D^3)}{\partial \delta^2_j}&=x^2_je_{j+n}\\\nonumber
% \frac{\partial f_{\delta}(x^1,x^2,x^3,D^1,D^2,D^3)}{\partial \delta^3_j}&=x^3_je_{j+2n}
% \end{align}
% It follows that 
% \begin{align}
%      \frac{\partial f_{\delta}(x^1,x^2,x^3, D^1,D^2, D^3)}{\partial \delta}&=\frac{\partial f_{\delta}(x^1,x^2,x^3, D^1,D^2, D^3)}{\partial (D^1,D^2, D^3)}\\\nonumber
%      &=\begin{bmatrix}
%     X^1&0&0\\0&X^2&0\\0&0&X^3
%     \end{bmatrix}
% \end{align}

% At any equilibrium in $\mathcal X$, this matrix has full row rank, as then does the Jacobian matrix $\frac{\partial f_{\delta}(x,\delta)}{\partial (x,\delta)}$ in \eqref{eq:jacxdelta} of which it is a submatrix. 

% Let $\mathcal Z$ be the submanifold of $\mathcal Y$ consisting of the single point ${\bf{0_{3n}}}$. Evidently, it is in fact the image of those points in $\mathcal X \times \mathcal D$ defined by $f_{\delta}(x,\delta)={\bf{0_{3n}}}$. Hence at such a point, $x^k\gg 0$ or $X^k$ is nonsingular. Hence the Jacobian of $f_{\delta}$ with respect to $(x,\delta)$ has full rank at such a point, which means that the map  $f_{\delta}$ is transversal to $\mathcal Z$. Then by the Parametric Transversality Theorem \cite[see p.145]{lee2013introduction}, \cite[see p.68]{guillemin2010differential}, it follows that for almost all choices $\bar\delta$ of $\delta$ , the mapping $f_{\bar\delta}:\mathcal X\to \mathcal Y, f_{\bar \delta}(x)=f_{\delta}(x,\bar \delta)$  will be transversal to $\mathcal Z$, i.e. the Jacobian $\frac{\partial f_{\delta}(x,\bar \delta)}{\partial x}$  will be of full rank $3n$  at the zeros of $f_{\bar\delta}$. Since this matrix is square and of size $3n\times 3n$, this says that  every zero of $f(x,\bar\delta)$ is nondegenerate, and as a consequence isolated. As already noted, since the zeros occur in a bounded set, this means they are finite in number. The choices of $\delta$ for which they are not finite in number, if indeed such choices exist, define a set of measure zero.
%\end{proof}
%}


 
% Before examining them, we argue that there can be no other equilibrium, $(\hat x^1,\hat x^2)$ say, on the boundary of $\Xi_{2n}$. In particular, there can be no equilibrium for which $\hat x^1_k+\hat x^2_k=1$ for some $k$. For the equilibrium equations, rewritten as $\hat x^i=(D^i)^{-1}(I-\hat X^1-\hat X^2)B^i$, would, using the $k$-th entry on each side, imply $\hat x^1_k=0,\hat x^2_k=0$, a contradiction. Nor can there be an equilibrium in which $\hat x^1+\hat x^2\ll{\bf{1}}_n$ and some but not all entries of $\hat x^1$ are zero. For observe that $(D^1)^{-1}(I-\hat X^1-\hat X^2)B^1=\hat B^1$  will be irreducible and the equilibrium equation will be $\hat x^1=\hat B^1\hat x^1$. By the eigenvalue/eigenvector properties of irreducible matrices, $\hat x^1$ must be positive and not just nonnegative. Likewise, no entry of $\hat x^2$ can be zero. 





% \textit{Proof:}{\color{red} Incomplete at this point} If the Jacobian $J(\bar x^1,\bar x^2,\bar x^3)$ at a particular equilibrium point $(\bar x^1,\bar x^2,\bar x^3)$ is nonsingular, that equilibrium point is isolated. If the Jacobian at every equilibrium point of the trivirus system~\eqref{eq:x1}-\eqref{eq:x3} is nonsingular, all equilibria are isolated, and given the compactness of the region defined by $\{{\bf{0}}\leq x^k\leq{\bf{1}}\cap x^1+x^2+x^3\leq {\bf{1}},k\in[3]\}$, there can only be a finite number of equilibria. Hence it is enough to show that for generic $D^k,B^k$, there is no equilibrium solution of the system equations for which the Jacobian matrix is also singular.

% Notice that for fixed $B^k,D^k$, such a solution, i.e. an equilibrium point with singular Jacobian matrix, would have to satisfy a total of $3n+1$  equations ($3n$ associated with the equilibrium point and a further one associated with the Jacobian determinant being zero) for $3n$ scalar unknowns, viz. the entries of $x^1,x^2,x^3$. 

% Denote the parameter vector associated with the nonzero entries of $B^k$ and diagonal entries of $D^k$ by $p$. The notation $p^*$ will denote a particular value of $p^*$.  For a particular $p^*$, denote the equations by
% \begin{align}\label{eq:specific}
% f_i(x,p^*)&=0\quadi=1,2,\dots,3n\\
% g(x,p^*)&=0
% \end{align}
% These are multivariate polynomial equations in $x$ with real coefficients. 
% One can also record equations satisfied by $x,p$ where the entries of $p$ are regarded as indeterminates:
% \begin{align}\label{eq:field}
% f_i(x,p)&=0\quadi=1,2,\dots,3n\\
% g(x,p)&=0
% \end{align}

% The entries of $p$ occur in an affine fashion in the first $3n$ equations, but as multivariate polynomials in the last equation. Thus the $(3n+1)$ equations can be regarded as multivariate polynoials in $x$ with coefficients in $R[p]$, the ring of multivariate polynomials in $p$. By extension we can regard them as polynomials in $x$ with coefficients in $R(p)$, the field of real rational multivariate functions of $p$. And now in relation to this second set of equations,  algebraic geometry {\color{red} detailed Hodge and Pedoe reference required. One reference might be page 169, Theorem `1: there is a nonzero resultant form for n+1 equations in n+1 unknowns.}} guarantees (since either the $3n+1$ polynomials have a nontrivial common factor, or they do not) that either  {\color{red}} or there exist a set of $(3n+1)$ multivariate polynomials in $x$, denoted by $A_i(x,p)$, and with coefficients in $R(p)$, for which
% \begin{equation}
% \sum_{i=1}^{3n}A_i(x,p)f_i(x,p)+A_{3n+1}(x,p)g(x,p)=1
% \end{equation}




% Observe now that if a particular parameter choice $p^*$ guarantees that the associated equilibrium points have a nonsingular Jacobian matrix, the implicit function theorem shows that for all sufficiently small perturbations of the parameter values away from $p^*$, there will be an associated solution of the equations for which the Jacobian matrix remains nonsingular. 

% This set of $3n+1$ equations in the $3n$ scalar variables  is in fact a set of \textit{multivariate inhomogeneous polynomial} equations, with the coefficients of the monomials appearing in each equation being themselves polynomial (actually affine for the first $3n$ equations) in the entries of $D^k$ and $B^k$, for $k\in[3]$. An associated set of homogeneous equations in $3n+1$ scalar variables, $(x^1,x^2,x^3,u)$ say, where $u$ is the scalar homogenising variable can immediately be formed,  e.g. \cite[see pp. 84 ff.]{cox2006using}. If a solution of the latter equations exists with $u\neq 0$, a solution of the inhomogeneous equation exists, obtained by scaling of the homogeneous equation solution so that $u=1$. The only other nonzero solutions of the homogeneous equations have $u=0$ and are those associated with a nonzero solution of a modification of the inhomogeneous equations obtained by setting all terms other than those of highest degree  to zero (the so-called 'solutions at infinity'). 

% Typically, $(3n+1)$ inhomogeneous equations in $3n$ unknowns might be expected to have no solution.  There is however a condition for the existence of a nonzero solution to the associated homogeneous equations involving a \textit{resultant}; the resultant is a multivariate polynomial in the coefficients appearing in the equations, and there is a nonzero solution to the equations if and only if the resultant evaluates as zero for the particular coefficient values. (It may be identically zero, in which case there are solutions to the equations for all values of the coefficients.) Since the resultant is polynomial in the coefficient values, if it is nonzero for any one set of coefficient values (whatever the signs might be), it will be nonzero for all coefficient values (again with arbitrary signs), with the possible exception of a set of measure zero. Note that complex solutions as well as real solutions are ruled out by a nonzero resultant, even though the coefficients in the equations are real. 

% All this means is that if, for a particular choice of $D^k,B^k$, we can show there is no nonzero solution to the homogeneous equations associated with the $3n+1$ equations obtained from the equilibrium point and zero Jacobian determinant condition, it will follow that the inhomogeneous equations will have no solution for almost all $D^k,B^k$, implying in turn that there is only a finite number of equilibria for almost all $D^k,B^k$. This logic is valid whether or not the $D^k,B^k$ satisfy constraints on the signs of certain elements. 


% We will now demonstrate that for a certain choice of $D^k,B^k$, the homogeneous equations have no nonzero solution. 
% Make the particular choice $D^k=I$ for $k\in[3]$, and take $B^k$ diagonal with, for each $j$, the $j$-th diagonal entries all different. We consider first the solutions of the inhomogeneous equations.  It is easily checked then that apart from the disease free equilibrium, the only equilibria are given by choosing for each $i$ one of the three following possibilities:
% \begin{equation}
% (x^1_i,x^2_i,x^3_i)=(1-1/\beta^1_{ii},0,0), (0,1-1/\beta^2_{ii},0), (0,0,1-1/\beta^3_{ii})
% \end{equation}
% for $i=1,2,\dots,n$. Note that the equilibrium point equations actually comprise $n$ uncoupled sets of three equations with three variables, one set corresponding to each choice of $i$, with the variables $x^1_i,x^2_i,x^3_i$ appearing just in the $i$-th set. One can also verify that the $3\times 3$ Jacobian matrices at the nonzero solutions of each set are nonsingular, being triangular after rearrangement of rows and columns.
% Hence there is no solution of the $3n+1$ inhomogeneous equations. 

% The homogeneous equations derived from the equilibrium conditions are also decoupled, and  are of the form $(x^1_i+x^2_i+x^3_i)x^k_i=0, k=1,2,3$. It is readily checked that the only solution of these equations with $u=0$ (or the solution at infinity of the inhomogeneous equations) is one where $x^1_i+x^2_i+x^3_i=0$. The remaining equation comes from setting the highest degree term in the determinant of the Jacobian to zero. The Jacobian, with a simplication incorporating $x^1_i+x^2_i+x^3_i=0$ is 
% \begin{small}
% \[
% J=\begin{bmatrix}
% -1+b^1_{ii}(1-x^1_i)&-b^1_{ii}x^1_i&-b^1_{ii}x^1_i\\
% -b^2_{ii}x^2_i&-1+b^2_{ii}(1-x^2_i)&-b^2_{ii}x^2_i\\
% -b^3_{ii}x^3_i&-b^3_{ii}x^3_i&-1+b^3_{ii}(1-x^3_i)
% \end{bmatrix}
% \]
% \end{small}
% and one can verify that
% \begin{small}
% \[
% \det J=\\
% -b^1x^1(1-b^2)(1-b^3)-b^2x^2(1-b^1)(1-b^3)-b^3x^3(1-b^1)(1-b^2)
% \]
% \end{small}
% What has to be done is show this has nonzero determinant when $x^1_i+x^2_i+x^3_i=0$\\
% 
%}

% More specifically, suppose that the number of equilibria is finite. Let $J$ denote the Jacobian associated with \eqref{eq:bivirus}. With the right side of \eqref{eq:bivirus} denoted by $f(x)$, then $J(x)=\nabla f(x)$. If $\bar x$ is an equilibrium, it is hyperbolic if and only if $J(\bar x)$ has no eigenvalue with zero real part. Since for almost all parameter choices, there is no zero eigenvalue (because zeros of $f(x)$ are nondegenerate), we only need to rule out the possibility of there being pure imaginary eigenvalues of the Jacobian. However, instead of doing this, we shall rule out a situation of wider scope, viz. the possibility that there are eigenvalue pairs whose sum is zero; obviously, if (generically) there are no eigenvalue pairs summing to zero, this implies as a particular case that there are no pure imaginary eigenvalues, since they necessarily occur in complex conjugate pairs which sum to zero.
% Taking all this together, we need to rule out existence for almost all $B^i$ of a solution 
% $(\bar x,\bar s)$ with $\bar x \in \Xi_{2n}$ and $\bar s \in \mathbb C$  to the following equations:
% \begin{align}\label{eq:zerosummming}
% f(x)&=0\\\nonumber
% |sI_{2n}-\nabla f(x)|&=0\\\nonumber
% |sI_{2n}+\nabla f(x)|&=0
% \end{align}





%  Since the coefficients are affine in the entries of the $B^i$, this means, crucially, that if for any fixed $D^i$, there is a particular choice of entries of the $B^i$ for which the resultant is nonzero then for that choice \textit{and consequently almost all other choices of the $B^i$ entries} the resultant will be nonzero and there can be no solution to the homogeneous equations or no solution $(\bar x,\bar s)$ to the inhomogeneous. equations. The fact that the particular choice of $B^i$ might have certain zero entries, or entries of certain signs, is irrelevant in drawing this conclusion. Complex solutions as well as real solutions are ruled out by a nonzero resultant, even though the coefficients in the equations are real. 

% (The result generalizes the easily understood notion that if two real polynomials $g_1(x,b), g_2(x,b)$ have coefficients affine in some real scalar parameter $b$, and if there is no common zero (real or complex) of the polynomials for a particular $b$, there will be no common zero (real or complex) for almost all $b$.) 

% We can now demonstrate that there exists a choice of the $B^i$ for which there is no solution of \eqref{eq:zerosummming}. Choose both $B^i$ to be diagonal, thus $B^i={\rm{diag}}[\beta^i_{11},\beta^i_{22},\dots, \beta^i_{nn}]$, with also $\beta^i_{jj}>1$ for all $j$. Also, require that $\beta^1_{jj}\neq \beta^2_{jj}$ for any $j$. It is then easily checked that the requirement $f(x)=0$ implies that for each $j$, there holds one of three possibilities, viz. $(\bar x^1_j,\bar x^2_j)=(0,0), (1-(\beta^1_{jj})^{-1}, 0), (0,1-(\beta^2_{jj})^{-1})
% $. We must check that for each possibility, it is impossible, for almost all choices of $\beta^i_{jj}$, for $\nabla f(x)$ to have two eigenvalues summing to zero. To this end, we must evaluate $\nabla f(x)$  }

% \par Theorem~\ref{prop:tri-virus-not-monotone} sheds light on a very interesting phenomenon, namely that the tri-virus system is not monotone. This is in sharp contrast to the bi-virus setting, which is known to be monotone \cite{ye2021convergence}. The fact that a bivirus system is monotone coupled with the fact that for almost all choices of $D^k$, $B^k$, $k=1,2$, the bivirus system has a finite number of equilibria allows one to draw general conclusions on the limiting behavior of bivirus dynamical systems. 
%As such, even if the tri-virus system were to have a finite number of equilibria for almost all choices of system parameters, 
% \seb{Nonetheless, due to the findings of Theorem~\ref{prop:tri-virus-not-monotone},
% %Even with a proof that the number of equilibria is finite for almost all choices of system parameters (i.e., healing rates and infection rates with respect to each of the three viruses),
% one cannot draw upon the rich literature on monotone dynamical systems (see\cite{smith1988systems}) to study the limiting behavior of system~\eqref{eq:x1}-\eqref{eq:x3}. In general, for non-monotone systems, no dynamical behavior, including chaos, can be definitively ruled out %without additional analysis
% \cite{sontag2007monotone}.}

% By extending the algebraic geometry arguments in the proof of \cite[Theorem~3.6]{ye2021convergence}, it is relatively straightforward to show that even for the tri-virus system, for almost all choices of $D^k$, $B^k$, $k=1,2,3$, there exists a finite number of equilibria. The details are omitted here in the interest of space. Nonetheless, due to the findings of Theorem~\ref{prop:tri-virus-not-monotone},
% %Even with a proof that the number of equilibria is finite for almost all choices of system parameters (i.e., healing rates and infection rates with respect to each of the three viruses),
% one cannot draw upon the rich literature on monotone dynamical systems (see\cite{smith1988systems}) to study the limiting behavior of system~\eqref{eq:x1}-\eqref{eq:x3}. In general, for non-monotone systems, no dynamical behavior, including chaos, can be definitively ruled out %without additional analysis
% \cite{sontag2007monotone}.


%{\brian{{\color{blue} Comment to be deleted: I am not sure that the next paragraph is appropriately located in this section, given the title of the section. If it is to be retained, another possibility would be to change the title of the section, and even include in this section the  lemma unwritten but proposed for the end of Section II on the fact that equilibria either have $x^k=0$ or $x^k\gg 0$.}}
%  Another possible consequence of the lack of monotonicity is as follows: It is  known that setting $D^k=I$ for $k \in [2]$ has no bearing on either the location of equilibria of system~\eqref{eq:full}
% with $m=2$ nor on their (local) stability properties \cite[Lemma~3.7]{ye2021convergence}. That is, for a) bivirus systems, where $D^k=I$ and $B^k = (D^k)^{-1}B^k$ for $k \in [2]$, and b) bivirus systems, where $D^k$s are arbitrary positive diagonal matrices, the location of equilibria are the same for both bivirus systems a) and b). Plus, local stability of an equilibrium in  bivirus system a) implies, and is implied by,  that in bivirus system b).
% %Note that Theorem~\ref{thm:init:condns} is reliant on the assumption that the healing rates for all agents with respect to all viruses are unity. 
% For system~\eqref{eq:full} with $m=3$, by extending the arguments from \cite[Lemma~3.7]{ye2021convergence},  it is straightforward to show that the \emph{location} of the equilibria is the same when, for $k\in [3]$, $D^k=I$ and $B^k = (D^k)^{-1}B^k$,  and when $D^k$ (resp. $B^k$) are arbitrary positive diagonal (resp. nonnegative) matrices with the $D^k$s not necessarily being equal to each other. However, since the tri-virus system is not monotone, the arguments for stability of equilibria in the proof of \cite[Lemma~3.7]{ye2021convergence} cannot be adapted. As such, for the tri-virus case, preservation of stability properties remains an open question when the healing rates for all nodes with respect to all viruses are `scaled' in the manner above to become unity.
%In a similar vein to the question of monotonicity, it is natural
%In the sequel, we focus on studying properties of specific nonzero equilibrium points viz. their existence, and stability.

\section{Persistence of one or more viruses}\label{sec:persistence:one:virus}

In this section, we first show that for low-dimensional systems existence of certain kinds of equilibria can be ruled out for almost all choices of system parameters. Further, we identify a sufficient condition for local exponential convergence to a boundary equilibrium.  

 %Brian insert 20220717
\subsection{The special case of low dimension systems}
When $n=1$ or $n=2$, certain types of equilibria are generically impossible, as we now show. %By `generically impossible', we mean for all but a set of parameter values of measure zero. This set of exceptional values is defined by an algebraic or semi-algebraic set. 
%{\color{blue}
\begin{prop}\label{prop:brian:1}
Consider system~\eqref{eq:x1}-\eqref{eq:x3} and suppose that $n=1$. Suppose that $(\tilde x^1,\tilde x^2,\tilde x^3)$ is an equilibrium.  Provided no two values of $(D^k)^{-1}B^k$ (with $k=1,2,3$) are identical, at most one  of the $\tilde x^k$ is nonzero at an equilibrium.
\end{prop}
%\begin{proof}
\textit{Proof:}
At any equilibrium, there holds (recall that the $\tilde x^k$ are scalar, so that $\tilde X^k=\tilde x^k)$
\[
(-D^k+(1-\tilde x^1-\tilde x^2-\tilde x^3)B^k)\tilde x^k=0
\]

 Note that if all $B^i$ are nonzero, there cannot hold $(1-\tilde x^1-\tilde x^2-\tilde x^3)=0,$
 else a contradiction is easily obtained. 

It is immediate that either 
\[(1-\tilde x^1-\tilde x^2-\tilde x^3)^{-1}=(D^k)^{-1})B^k\quad\mbox{or}\quad\tilde x^k=0
\]
The first alternative corresponds to $\tilde x^k\neq 0$, and if, for example, $\tilde x^1\neq 0, \tilde x^2\neq 0$ then the parameters are constrained by $(D^1)^{-1}B^1=(D^2)^{-1}B^2$. \hfill \proofbox
%\end{proof}

For the case $n=2$, we shall appeal to the following lemma.

\begin{lem} 
Consider three positive matrices $ B^i, i=1,2,3$ of dimension $2\times 2$. For generic values of the matrix entries, there cannot exist a diagonal matrix, $\Lambda$ say, such that each of $\Lambda B^i$ has a unity eigenvalue.
\end{lem}
\textit{Proof:}
Denote the diagonal entries of $\Lambda$ by $\lambda_1,\lambda_2$, The eigenvalue condition on $\Lambda B^i$ is det$(I-\Lambda B^i)=0$, or 
\[(\beta^i_{11}\beta^i_{22}-\beta^i_{12}\beta^i_{21})\lambda_1\lambda_2-\beta^i_{11}\lambda_1-\beta^i_{22}\lambda_2+1=0
\]
For fixed $B^i$, the set of $\lambda_i$ satisfying this equation is evidently defined by a  rectangular hyperbola. It is clear that for generic $B^i$, there cannot be a common point of intersection between three such hyperbolae. Algebraic geometry gives methods for calculating the actual semialgebraic set of $B^i$ for which a common point of intersection exists, but this is irrelevant for our purposes. For almost all $B^i$, no such intersection exists.~\proofbox


Now we can state the following.

\begin{prop}\label{prop:brian:2}
Consider system~\eqref{eq:x1}-\eqref{eq:x3} and suppose that $n=2$.  Suppose that $(\tilde x^1,\tilde x^2,\tilde x^3)$ is an equilibrium. Then for almost all values of the parameters $D^k,B^k$ ($k=1,2,3$), at least one $\tilde x^k$  must be equal to $\textbf{0}$.
\end{prop}
\textit{Proof:}
The equilibrium equations yield
\[
(-D^k+(I-\tilde X^1-\tilde X^2-\tilde X^3)B^k)\tilde x^k=0
\]
Set $\Lambda=I-\tilde X^1-\tilde X^2-\tilde X^3$ and observe that the equilibrium equations can be rewritten as 
\[
\tilde x^k=\Lambda[(D^k)^{-1}B^k]\tilde x^k
\]
Hence either $\tilde x^k$ is an eigenvector of $\Lambda[(D^k)^{-1}B^k]$ or it is zero. The preceding lemma applies (with $(D^k)^{-1}B^k$ replacing $B^k$ in the lemma statement) and the conclusion of the proposition is immediate. \hfill  \proofbox
%\end{proof}
%}
\par The proposition indicates that for generic $D^i$, $B^i$, and with $n=2$, the only equilibria of the trivirus system are the healthy equilibrium, those defined by the three associated single virus systems, and those defined by the three associated bivirus systems. Note that for $n=2$, all these equilibria are analytically computable, \cite{ye2021convergence}.



% If one or more of the eigenvalue conditions in Proposition~\ref{prop:phil} is violated, then at least one of the viruses persists in the population.  
% This, in turn, gives rise to a richer possible set of behaviors, as we will see in the rest of this paper. %Towards this end,we recall the following lemmas, %which will be required in the sequel.

%\subsection*{Preliminaries}
% \begin{lem}\cite[Lemma~7]{axel2020TAC} \label{lem:pos_never_zero}
% Let Assumption~\ref{assum:base} hold. Then $\mathcal{D} \setminus \{ \textbf{0} \}$ is positively invariant with respect to system~\eqref{eq:x1}-\eqref{eq:x3}.
% \end{lem}


%We have the following lemma that restricts the trajectories of the tri-virus system.



% The following lemma establishes certain properties of equilibria of system~\eqref{eq:x1}-\eqref{eq:x3}.
% \seb{Recall that Lemma~\ref{lem:inward:pointing}  substantially limits the equilibria that can lie on the boundary of the set $\{x^1, x^2, x^3 \in \mathbb{R}_{\geq 0}\mid x^1+x^2+x^3 \leq \textbf{1}\}$. }
% %An immediate consequence of Lemma~\ref{lem:inward:pointing} 
% \seb{A  consequence of Lemma~\ref{lem:inward:pointing}}
% is the following result, which has also been established using a different proof technique in \cite[Lemma~6]{axel2020TAC}.
It turns out that we cannot have multiple 3-coexistence equilibria that  differ  in the coordinate vector corresponding to only one of the viruses; we formalize the same in the following lemma.
\begin{lem} \label{lem:equi_non-zero_nonone}
Consider system~\eqref{eq:x1}-\eqref{eq:x3} under Assumptions~\ref{assum:base} and~\ref{assum:irreducible}. %Suppose that for each $i \in [n]$, $\sum_{j=1}^{n}B^1_{ij} >0$.%and~\ref{assum:irreducible}. %Suppose, for all $k \in [3]$, that $B^k$ is irreducible. 
%\begin{enumerate}[label=\roman*)]
   % \item \label{stmnt:1} If $x = (x^1, \dots, x^3) \in \mathcal{D}$ is an equilibrium of~\eqref{eq:x1}-\eqref{eq:x3}, then, for each $k \in [3]$, either $x^k = \textbf{0}$, or $\textbf{0} \ll x^k \ll \textbf{1}$. Moreover, %we have that 
%$\textstyle \sum_{k=1}^3 x^k \ll \textbf{1}$. 
    %\item \label{stmnt:2} 
If $(\bar{x}^1, \bar{x}^2, \bar{x}^3)$ and  $(\bar{x}^1, \bar{x}^2, \hat{x}^3)$ are endemic  equilibria (i.e., $\textbf{0}\ll (\bar{x}^1,\bar{x}^2, \bar{x}^3) \ll \textbf{1}$ and $\textbf{0}\ll  \hat{x}^3 \ll \textbf{1}$) of~\eqref{eq:x1}-\eqref{eq:x3}, then $\bar{x}^3 =\hat{x}^3$.
%\end{enumerate}
\end{lem}
\textit{Proof:} 
%Statement~\ref{stmnt:1} is an immediate consequence of Lemma~\ref{lem:inward:pointing}. As an aside, note that statement~\ref{stmnt:1} has also been established using a different proof technique in \cite[Lemma~6]{axel2020TAC}.\\
Suppose that $(\bar{x}^1, \bar{x}^2, \bar{x}^3)$ and  $(\bar{x}^1, \bar{x}^2, \hat{x}^3)$ are endemic equilibria of~\eqref{eq:x1}-\eqref{eq:x3}, then by writing the equilibrium version of equation~\eqref{eq:x1}, we obtain:
\begin{align}
    \textbf{0}&=-D^1\bar{x}^1+((I-\bar{X}^1-\bar{X}^2-\bar{X}^3))B^1\bar{x}^1 \label{eq:x1bar}\\
    \textbf{0}&=-D^1\bar{x}^1+((I-\bar{X}^1-\bar{X}^2-\hat{X}^3))B^1\bar{x}^1 \label{eq:x1hat}
\end{align}
From~\eqref{eq:x1bar} and~\eqref{eq:x1hat}, it is clear that
\begin{align}
    \bar{x}^1&=(D^1)^{-1}((I-\bar{X}^1-\bar{X}^2-\bar{X}^3))B^1\bar{x}^1\nonumber \\
    &=(D^1)^{-1}((I-\bar{X}^1-\bar{X}^2-\hat{X}^3))B^1\bar{x}^1, \nonumber
\end{align}
which, since i) $\bar{x}^1 \gg \textbf{0}$ and ii) by Assumption~\ref{assum:irreducible} it follows that
%assumption, 
for each $i \in [n]$, $\sum_{j=1}^{n}B^1_{ij} >0$,
which implies that $\bar{X}^3=\hat{X}^3$, i.e., $\bar{x}^3=\hat{x}^3$.\hfill \proofbox
\qed
%$\blacksquare$



% It turns out that if each of the eigenvalue conditions in Proposition~\ref{prop:phil} were to be violated, then the tri-virus system has at least $4$ equilibria, as we recall in the following proposition.
% \begin{prop} \cite{liu2019analysis,axel2020TAC} \label{prop:necessity}
% Consider system~\eqref{eq:x1}-\eqref{eq:x3} under Assumption~\ref{assum:base}. For each $k \in [3]$ such that $B^k$ is irreducible and $s(B^k - D^k) > 0$, there is a unique single-virus endemic equilibrium $(\textbf{0}, \dots, \Tilde{x}^k, \dots, \textbf{0})$ in $\mathcal{D}$, with $\textbf{0} \ll \Tilde{x}^k \ll \textbf{1}$.%~$\blacksquare$
% \end{prop}
%

%We need the following lemmas in the sequel.
%\section{Preliminaries} \label{sect:prelims} %stability notions, results from Khalil etc.}
%Comment: Add relevant definitions and theorems on stability analysis from \cite{khalil2002nonlinear}.
%
%In this section, we recall some preliminary results, pertinent to the analysis of system~\eqref{eq:full}.
%A real square matrix is said to be Metzler if all elements outside the diagonal are nonnegative. We require the following results for Metzler matrices.
%
% \begin{lem} \label{lem:metz}
% \cite[Proposition~2]{rantzer2011distributed} Suppose that $M$ is a Metzler matrix such that $s(M) < 0$. Then, there exists a positive diagonal matrix $P$ such that $M^T P + P M \prec 0$.~$\blacksquare$ %\hfill  
% \end{lem}
%
% \begin{lem} \label{lem:metz_irreduc}
% \cite[Lemma A.1]{khanafer2016stability} Suppose that $M$ is an irreducible Metzler matrix such that $s(M) = 0$. Then there exists a positive diagonal matrix $P$ such that $M^T P + P M \preccurlyeq 0$.~$\blacksquare$ \end{lem}
%
% \begin{lem} \label{lem:eigspec}
% \cite[Proposition~1]{liu2019analysis} Suppose that $\Lambda$ is a negative diagonal matrix and $N$ is an irreducible nonnegative matrix. 
% %Let $M = \Lambda+N$.
% Let $M$ be the irreducible Metzler matrix $M = \Lambda+N$. 
% Then, $s(M) < 0$ if and only if $\rho(-\Lambda^{-1} N) < 1, s(M)=0$ if and only if $\rho(-\Lambda^{-1} N) = 1$, and $s(M)>0$ if and only if, $\rho(-\Lambda^{-1} N) > 1$.%~$\blacksquare$
% \end{lem}
% %
% We will also be making use of the following variants of the Perron-Frobenius theorem for irreducible matrices.

% \begin{lem} \label{lem:perron_frob}
% \cite[Chapter 8.3]{meyer2000matrix} \cite[Theorem~2.7]{varga1999matrix} %(Perron-Frobenius Theorem) 
% Suppose that $N$ is an irreducible nonnegative matrix. Then,
% %
% \begin{enumerate}[label=(\roman*)]
%     \item $r = \rho(N)$ is a simple eigenvalue of $N$. \label{item:perfrob_simpleeig}
%     \item There is an eigenvector $\zeta \gg \textbf{0}$ corresponding to the eigenvalue $r$. \label{item:perfrob_pos_exists}
%     \item $x > \textbf{0}$ is an eigenvector only if $Nx = rx$ and $x \gg \textbf{0}$. %\axel{\textbf{[Not sure this needs to be mentioned]}} 
%     \label{item:perfrob_pos_necess}
%     \item If $A$ is a nonnegative matrix such that $A < N$, then $\rho(A) < \rho(N)$. \label{item:perfrob_matrix_ineq}%~$\blacksquare$
% \end{enumerate}
% %
% \end{lem}
% %
% %Comment: Double-check whether we are indeed using all 4 items in Lemma~\ref{lem:perron_frob}]
% \begin{lem} \label{lem:perron_frob_metz}
% \cite[Lemma~2.3]{varga1999matrix} Suppose that $M$ is an irreducible Metzler matrix. Then $r = s(M)$ is a simple eigenvalue of $M$, and a corresponding eigenvector is $\zeta \gg \textbf{0}$.%~$\blacksquare$
% \end{lem}

\subsection{Stability of boundary equilibria}
In this subsection, 
%we identify a sufficient condition for local exponential convergence to a boundary equilibrium. While similar results exist for the case when $m=2$ (see \cite[Theorem~3.10]{ye2021convergence}), to the best of our knowledge, no such result exists for the $m=3$ case. The following theorem addresses this gap, and establishes 
we establish
that the local stability (resp. instability) of the boundary equilibrium corresponding to %virus~$i$ 
virus~1
is dependent on whether (or not) the state matrices, obtained by linearizing the dynamics of viruses~$2$ and~$3$ around the boundary  
%single-virus endemic 
equilibrium of virus~$1$, are Hurwitz.

% corresponding to  %viruses~$j$ and~$k$, 
% viruses~$2$ and~$3$
% linearized around the single-virus endemic equilibrium of virus~$i$, is Hurwitz.

\begin{thm}\label{thm:local}
Consider system~\eqref{eq:x1}-\eqref{eq:x3} under Assumptions~\ref{assum:base} and~\ref{assum:irreducible}. \break%Suppose that for each $k \in [3]$  $B^k$ is irreducible. 
The boundary equilibrium $(\Tilde{x}^1, \textbf{0}, \textbf{0})$ is locally exponentially %asymptotically 
stable if, and only if, 
each of the following conditions is satisfied: 
\begin{enumerate}[label=\roman*)]
    \item $\rho((I-\tilde{X}^1)(D^2)^{-1}B^2)<1$; and
    \item $\rho((I-\tilde{X}^1)(D^3)^{-1}B^3)<1$.
\end{enumerate}
If $\rho((I{-}\tilde{X}^1)(D^2)^{-1}B^2)>1$ or if \mbox{$\rho((I{-}\tilde{X}^1)(D^3)^{{-}1}B^3)>1$}, then $(\Tilde{x}^1, \textbf{0}, \textbf{0})$ is unstable.
\end{thm}
The proof follows the strategy outlined for the $m=2$ case in~\cite[Theorem~3.10]{ye2021convergence}.\\
\textit{Proof:} Consider the equilibrium point $(\Tilde{x}^1, \textbf{0}, \textbf{0})$, and note that the Jacobian evaluated at this point is as follows:
\begin{align}\label{eq:boundaryJac}
&J(\Tilde{x}^1,\textbf{0}, \textbf{0}) = \\
&\scriptsize
\begin{bmatrix}
-D^1+(I-\Tilde{X}^1)B^1-\hat{B}^1  & {-}\hat B^{1}   & {-}\hat B^{1} \\
 \textbf{0} & {-}D^2{+}(I{-}\Tilde{X}^1)B^2 & \textbf{0} \\
\textbf{0}  & \textbf{0} & {-}D^3{+}(I{-}\Tilde{X}^1)B^3 \end{bmatrix}\normalsize,\nonumber
\end{align}
where $\hat{B}^i=\diag(B^i\Tilde{x}^i)$, for $i=1,2,3$.\\
Observe that the matrix $J(\Tilde{x}^1,\textbf{0}, \textbf{0})$ is block upper triangular. Hence, it is Hurwitz if, and only if,  the blocks along the diagonal are Hurwitz. We will now show that this condition is fulfilled as a consequence of the assumptions of Theorem~\ref{thm:local}.
\par Since $(\Tilde{x}^1, \textbf{0}, \textbf{0})$ is an equilibrium point of system~\eqref{eq:x1}-\eqref{eq:x3}, by considering the equilibrium version of equation~\eqref{eq:x1}, we have the following:
\begin{align}\label{eq:eqm:version:1}
    (-D^1+(I-\Tilde{X}^1)B^1)\tilde{x}^1=\textbf{0}.
\end{align}
By Assumption~\ref{assum:base}, we have that $D^1$ is positive diagonal and $B^1$ is nonnegative. Furthermore, by assumption, we know that $B^1$ is irreducible. Moreover, from \cite[Lemma~6]{axel2020TAC}, it follows that $(I-\Tilde{X}^1)$ is positive diagonal, implying that $(I-\Tilde{X}^1)B^1$ is nonnegative irreducible. Thus, we can conclude that the matrix $(-D^1+(I-\Tilde{X}^1)B^1)$ is irreducible Metzler. %SinceF
From \cite[Lemma~6]{axel2020TAC}
%Lemma~\ref{lem:equi_non-zero_nonone} 
we know that $\textbf{0} \ll \tilde{x}^1$. Hence, by applying \cite[Lemma~2.3]{varga1999matrix} to~\eqref{eq:eqm:version:1}, it must be that $\tilde{x}^1$ is, up to scaling, the only eigenvector of $(-D^1+(I-\Tilde{X}^1)B^1)$ with all entries being strictly positive. Furthermore, $\tilde{x}^1$ is the eigenvector that is associated with, and only with, $s(-D^1+(I-\Tilde{X}^1)B^1)$. Therefore,   $s(-D^1+(I-\Tilde{X}^1)B^1)=0$.\\
Define $Q:=D^1-(I-\Tilde{X}^1)B^1$, and note that $Q$ is an M-matrix. Since $s(-Q)=0$ and $B^1$ is irreducible, $Q$ is a singular irreducible M-matrix. Observe that $\hat{B}^1$ is a nonnegative matrix and because $B^1$ is irreducible and $\Tilde{x}^1 \gg \textbf{0}$, it must be that at least one element in $\hat{B}^1$ is strictly positive. Therefore, from \cite[Lemma~4.22]{qu2009cooperative}, it follows that $Q+\hat{B}^1$ is an irreducible non-singular M-matrix, which from \cite[Section~4.3, page~167]{qu2009cooperative} implies that $-Q-\hat{B}^1$ is Hurwitz. Therefore, we have that $s(-D^1+(I-\Tilde{X}^1)B^1-\hat{B}^1)<0$.
\par By assumption, $\rho((I-\tilde{X}^1)(D^2)^{-1}B^2)<1$ and $\rho((I-\tilde{X}^1)(D^3)^{-1}B^3)<1$. Therefore, by noting that $D^2$ (resp. $D^3$) are positive diagonal matrices and $B^2$ (resp. $B^3$) are nonnegative irreducible matrices,
from %Lemma~\ref{lem:eigspec}, 
\cite[Proposition~1]{liu2019analysis} it follows that $s(-D^2+(I-\tilde{X}^1)B^2)<0$ (resp. $s(-D^3+(I-\tilde{X}^1)B^3)<0$). Since each diagonal block of $J(\Tilde{x}^1,\textbf{0}, \textbf{0})$ is Hurwitz, it is clear that $J(\Tilde{x}^1,\textbf{0}, \textbf{0})$ is Hurwitz. Local %asymptotic 
exponential 
stability of $(\Tilde{x}^1,\textbf{0}, \textbf{0})$, then, follows from \cite[Theorem 4.15 and Corollary~4.3]{khalil2002nonlinear}. %\cite[Theorem~4.7, item i)]{khalil2002nonlinear}}.
\par The proof of necessity follows by first noting that if either condition in statement~i) or that in statement~ii) is violated, then, since at least one of the blocks along the diagonal of $J(\Tilde{x}^1,\textbf{0}, \textbf{0})$ is not Hurwitz, the  matrix $J(\Tilde{x}^1,\textbf{0}, \textbf{0})$ is not Hurwitz. Then, by invoking the necessary part of \cite[Theorem 4.15 and Corollary~4.3]{khalil2002nonlinear}, the result follows.
\par The claim for instability can be proved by noting that if either of the eigenvalue conditions is violated, then, since  $J(\Tilde{x}^1,\textbf{0}, \textbf{0})$ is block diagonal, the matrix  $J(\Tilde{x}^1,\textbf{0}, \textbf{0})$ is not Hurwitz. The result follows from \cite[Theorem~4.7, item ii)]{khalil2002nonlinear}.~\hfill \proofbox
\par Analogous results for the boundary equilibria $(\textbf{0}, \Tilde{x}^2, \textbf{0})$ and $(\textbf{0}, \textbf{0},\Tilde{x}^3)$ can be similarly obtained.



\section{Nonexistence of various coexistence equilibria}\label{sec:nonexistence:coexistence:equilibria}
In this section,  we establish conditions on the system parameters that guarantee the nonexistence of different kinds of coexistence equilibria. Specifically, we identify a condition for the nonexistence of a 3-coexistence (resp. 2-coexistence) equilibria. We draw on the stability properties established in the previous section.



% \par \textit{Proof of statement~iii):} Since by assumption $B^2>B^1$, it follows that virus~2 eradicates virus~1. That is, there must exists some time $\tau \in \mathbb{R}_{\geq 0}$ such that for all $t \geq \tau$, $x^1(t)=0$.  This implies that for all $t \geq \tau$, the tri-virus system in~\eqref{eq:x1}-\eqref{eq:x3} boils down to the following bivirus system:
% \begin{align} 
%    %\dot{x}^1(t) &=  \Big{(} \big{(} I - (X^1+X^2+X^3) \big{)} B^1 - D^1 \Big{)} x^1(t), \label{eq:x1}\\
%       \dot{x}^2(t) &=  \Big{(} \big{(} I - (X^2+X^3) \big{)} B^2 - I \Big{)} x^2(t) \label{eq:x2bis}\\
%          \dot{x}^3(t) &=  \Big{(} \big{(} I - (X^2+X^3) \big{)} B^3 - I \Big{)} x^3(t) \label{eq:x3bis}
%   \end{align}
% The proof of statement~iii)  then follows from item ii) in \cite[Proposition~9]{ye2021convergence}}.

%\seb{
\begin{thm}[Nonexistence of 3-coexistence equilibria]\label{claim:virus3:strongest}
Consider system~\eqref{eq:x1}-\eqref{eq:x3} under Assumption~\ref{assum:base}. Suppose that matrix $B^k$ for $k \in [3]$ is irreducible. Suppose that  $\rho((D^k)^{-1}B^k)>1$ for $k \in [3]$. Let $\tilde{x}^1$,  $\tilde{x}^2$, and $\tilde{x}^3$ denote the single-virus endemic equilibrium corresponding to virus~1,~2, and~3, respectively. If %$(D^3)^{-1}B^3>(D^1)^{-1}B^1$ and 
$(D^3)^{-1}B^3>(D^2)^{-1}B^2>(D^1)^{-1}B^1$, then
\begin{enumerate}[label=\roman*)]
\item the equilibrium point $(\textbf{0}, \textbf{0}, \textbf{0})$ is unstable;
\item the equilibrium point $(\tilde{x}^1, \textbf{0}, \textbf{0})$ is unstable;
\item the equilibrium point $(\textbf{0}, \tilde{x}^2, \textbf{0})$ is unstable;
\item the equilibrium point $(\textbf{0}, \textbf{0}, \tilde{x}^3)$ is locally exponentially stable;
\item there does not exist a 3-coexistence equilibrium.
%\item there does not exist any other equilibria
 \end{enumerate}
\end{thm}
\textit{Proof:} 
The proof strategy is quite similar to those in \cite[Theorem~6]{axel2020TAC} and \cite[Corollary~3.10]{ye2021convergence}.\\
\par The proof of statements~i)-iii) are quite similar to that of statements~i) and ii) in Theorem~\ref{claim:virus1weaker}. 
\textcolor{black}{\textit{Proof of statement~i): }%By assumption, $D^1=D^2=D^3=I$. 
 Consider the equilibrium point $(\textbf{0}, \textbf{0}, \textbf{0})$, and observe that the Jacobian computed at this point is as follows: 
 \begin{align}
&J(\textbf{0},\textbf{0}, \textbf{0}) = \\
&\footnotesize
\begin{bmatrix}
-D^1+B^1  & \textbf{0}   & \textbf{0} \\
 \textbf{0} & -D^2+B^2 & \textbf{0} \\
\textbf{0}  & \textbf{0} & -D^3+B^3 \end{bmatrix}\normalsize,\nonumber
\end{align}
% Since $D^1=D^2=D^3=I$, the above can be rewritten as 
%  \begin{align}
% &J(\textbf{0},\textbf{0}, \textbf{0}) = \\
% &\footnotesize
% \begin{bmatrix}
% -I+B^1  & \textbf{0}   & \textbf{0} \\
%  \textbf{0} & -I+B^2 & \textbf{0} \\
% \textbf{0}  & \textbf{0} & -I+B^3 \end{bmatrix}\normalsize,\nonumber
% \end{align}
Since by assumption, $B^k$ is non-negative irreducible and $\rho((D^k)^{-1}B^k)>1$ for $k \in [3]$, it follows from Lemma~\ref{lem:eigspec} that $s(-D^k+B^k)>0$ for $k \in [3]$, which further implies that $s(J(\textbf{0},\textbf{0}, \textbf{0}))>0$. Instability of the equilibrium point $(\textbf{0},\textbf{0}, \textbf{0})$, then, follows from \cite[Theorem~4.7, item ii)]{khalil2002nonlinear}.\\
\textit{Proof of statement~ii):}  Consider the equilibrium point $(\tilde{x}^1, \textbf{0}, \textbf{0})$, and observe that the Jacobian computed at this point is as follows:
\begin{align}
&J(\Tilde{x}^1,\textbf{0}, \textbf{0}) = \\
&\scriptsize
\begin{bmatrix}
-D^1+(I-\Tilde{X}^1)B^1-\hat{B}^1  & -\hat B^{1}   & -\hat B^{1} \\
 \textbf{0} & -D^2+(I-\Tilde{X}^1)B^2 & \textbf{0} \\
\textbf{0}  & \textbf{0} & -D^3+(I-\Tilde{X}^1)B^3 \end{bmatrix}\normalsize,\nonumber
\end{align}
Note that since $\tilde{x}^1$ is the single-virus endemic equilibrium for virus~1, by the definition of an equilibrium point, we have the following:
$$[-D^1+(I-\tilde{X}^1)B^1]\tilde{x}^1=\textbf{0}.$$
Observe that $-D^1+(I-\tilde{X}^1)B^1$ is an irreducible Metzler matrix, and, since $\tilde{x}^1\gg \textbf{0}$, from Lemma~\ref{lem:perron_frob_metz} we have that $s(-D^1+(I-\tilde{X}^1)B^1)=0$. Consequently, from Lemma~\ref{lem:eigspec}, $\rho((I-\tilde{X}^1)(D^1)^{-1}B^1)=1$. By assumption $(D^2)^{-1}B^2> (D^1)^{-1}B^1$, which implies $(I-\tilde{X}^1)(D^1)^{-1}B^1 < (I-\tilde{X}^1)(D^2)^{-1}B^2$. Therefore, since the matrices $(I-\tilde{X}^1)(D^1)^{-1}B^1$ and 
$(I-\tilde{X}^1)(D^2)^{-1}B^2$ are nonnegative, from Lemma~\ref{lem:perron_frob} item iv), we have that $\rho((I-\tilde{X}^1)(D^2)^{-1}B^2)>1$. Consequently, from Lemma~\ref{lem:eigspec}, it must be that $s(-D^2+(I-\tilde{X}^1)B^2)>0$, which implies that $s(J(\Tilde{x}^1,\textbf{0}, \textbf{0}))>0$, and hence the equilibrium point $(\Tilde{x}^1,\textbf{0}, \textbf{0})$ is unstable.\\
 The proof of statement~iii) is analogous to that of statement~ii), and is, therefore, omitted. 
}


\noindent \textit{Proof of statement~iv):} Consider the equilibrium point $(\textbf{0}, \textbf{0}, \tilde{x}^3)$, and observe that the Jacobian is as follows:
\begin{align}
&J(\textbf{0}, \textbf{0}, \Tilde{x}^3) = \\
&\scriptsize
\begin{bmatrix}
-D^1+(I-\Tilde{X}^3)B^1  & \textbf{0}  & \textbf{0}\\
 \textbf{0} & -D^2+(I-\Tilde{X}^3)B^2 & \textbf{0} \\
-\hat B^{3}   & -\hat B^{3}    & -D^3+(I-\Tilde{X}^3)B^3-\hat B^{3}    \end{bmatrix}\normalsize,\nonumber
\end{align}
(Recall that $\hat B^3$ is shorthand for $\diag(B^3\tilde x^3))$. By following similar arguments as in the proof of Theorem~\ref{thm:local} (for establishing that the matrix $-D^1+(I-\Tilde{X}^1)B^1-\hat B^{1}$ is Hurwitz), we can show that the matrix $-D^3+(I-\Tilde{X}^3)B^3-\hat B^{3}$ is Hurwitz, i.e., $s(-D^3+(I-\Tilde{X}^3)B^3-\hat B^{3})<0$. Since $\tilde{x}^3$ is the single-virus endemic equilibrium corresponding to virus~3, we have that
$$\rho((I-\Tilde{X}^3)(D^3)^{-1}B^3)=1.$$
Since, by assumption, $(D^3)^{-1}B^3>(D^2)^{-1}B^2>(D^1)^{-1}B^1$, it is clear that \break
$(D^3)^{-1}B^3>(D^2)^{-1}B^2$ and $(D^3)^{-1}B^3>(D^1)^{-1}B^1$.
Hence, it follows that a) $(I-\Tilde{X}^3)(D^3)^{-1}B^3> (I-\Tilde{X}^3)(D^2)^{-1}B^2$, and b) $(I-\Tilde{X}^3)(D^3)^{-1}B^3> (I-\Tilde{X}^3)(D^1)^{-1}B^1$. Therefore, from Lemma~\ref{lem:perron_frob} item iv), it must be that a) $\rho((I-\Tilde{X}^3)(D^2)^{-1}B^2)<1$, and b) $\rho((I-\Tilde{X}^3)(D^1)^{-1}B^1)<1$, which from Lemma~\ref{lem:eigspec} further implies that a) $s(-D^2+(I-\Tilde{X}^3)B^2)<0$, and b) $s(-D^3+(I-\Tilde{X}^3)B^1)<0$. As a consequence, $s(J(\textbf{0}, \textbf{0}, \Tilde{x}^3))<0$, which, from \cite[Theorem~4.7]{khalil2002nonlinear}, leads us to conclude that the equilibrium point $(\textbf{0}, \textbf{0}, \Tilde{x}^3)$ is locally exponentially stable. \\
\textit{Proof of statement~v):} 
%\par \textit{Proof of item v) (contd.):}
We now show that under the conditions of the Theorem, a 3-coexistence equilibrium (i.e., of the form $\bar{x}=(\bar{x}^1, \bar{x}^2, \bar{x}^3)$ with all $\bar x^i>0$ ) cannot exist. %there are exactly four equilibria, namely the healthy state $(\textbf{0}, \textbf{0}, \textbf{0})$, and the three boundary equilibria $\big((\Tilde{x}^1,\textbf{0}, \textbf{0}),(\textbf{0},\Tilde{x}^2, \textbf{0}), (\textbf{0}, \textbf{0}, \Tilde{x}^3)\big)$. We proceed as follows. 
%First observe that any other equilibrium $\bar{x}=(\bar{x}^1, \bar{x}^2, \bar{x}^3)$ should, in view of Lemma~\ref{lem:equi_non-zero_nonone}, fulfil either a) $\bar{x}^1+\bar{x}^2 \ll \textbf{1}$, or b) $\bar{x}^2+\bar{x}^3 \ll \textbf{1}$, or c) $\bar{x}^1+\bar{x}^3 \ll \textbf{1}$, or d) $\bar{x}^1+\bar{x}^2 + \bar{x}^3 \ll \textbf{1}$.
Suppose that, by way of contradiction, there exists a 3-coexistence %an endemic
equilibrium point $\bar{x}=(\bar{x}^1, \bar{x}^2, \bar{x}^3)$. Therefore, from Lemma~\ref{lem:equi_non-zero_nonone} it must be that $\textbf{0} \ll (\bar{x}^1, \bar{x}^2, \bar{x}^3) \ll \textbf{1}$, and $\bar{x}^1+ \bar{x}^2+ \bar{x}^3 \ll \textbf{1}$. Furthermore, by the definition of an equilibrium point, we also have the following:
\begin{align} 
   \textbf{0} &=  (-D^1 + (I-\bar X^1-\bar X^2-\bar X^3) B^1) \bar x^1, \label{eq:x1_3-equib}\\
      \textbf{0}  &=  (-D^2 + (I-\bar X^1-\bar X^2-\bar X^3) B^2) \bar x^2, \label{eq:x2_3-equib}\\
         \textbf{0}  &=  (-D^3 + (I-\bar X^1-\bar X^2-\bar X^3) B^3) \bar x^3 \label{eq:x3_3-equib}.
  \end{align}
Since $\tilde{x}^k$ is the single-virus endemic equilibrium for virus $k$, $k \in [3]$, we also have the following:
\begin{align} 
   \textbf{0} &=  \Big{(} -D^1+ \big{(} I - \Tilde{X}^1 \big{)} B^1  \Big{)} \tilde{x}^1(t), \label{eq:x1a:3-eqm}\\
    \textbf{0} &=  \Big{(}-D^2+ \big{(} I - \Tilde{X}^2\big{)} B^2 \Big{)} \tilde{x}^2(t) \label{eq:x2a:3-eqm}\\
       \textbf{0} &=  \Big{(}-D^3+ \big{(} I - \Tilde{X}^3\big{)} B^3 \Big{)} \tilde{x}^3(t) \label{eq:x3a:3-eqm}
  \end{align}
Define $\kappa:= \max_{i \in [n]}[\bar{x}^1+ \bar{x}^2+ \bar{x}^3 ]_i/\tilde{x}^3_i$, which implies that $\bar{x}^1+ \bar{x}^2+ \bar{x}^3 \leq \kappa \tilde{x}^3$. Let $j$ be the maximizing value of $i$ in the definition of $\kappa$. Assume by way of contradiction that $\kappa \geq 1$. Consider \scriptsize
\begin{align} 
   (\bar{x}^1_j+ \bar{x}^2_j+ \bar{x}^3_j)&=(1-\bar{x}^1_j-\bar{x}^2_j-\bar{x}^3_j)((D^1)^{-1}B^1\bar{x}^1+(D^2)^{-1}B^2\bar{x}^2+ (D^3)^{-1}B^3\bar{x}^3)_j\nonumber\\
    &< (1-\bar{x}^1_j-\bar{x}^2_j-\bar{x}^3_j)((D^3)^{-1}B^3\bar{x}^1+(D^3)^{-1}B^3\bar{x}^2+ (D^3)^{-1}B^3\bar{x}^3)_j \label{ineq:11bis}\\
    &= (1-\bar{x}^1_j-\bar{x}^2_j-\bar{x}^3_j)((D^3)^{-1}B^3(\bar{x}^1+\bar{x}^2+ \bar{x}^3))_j \nonumber  \\
    &\leq (1-\bar{x}^1_j-\bar{x}^2_j-\bar{x}^3_j)((D^3)^{-1}B^3\kappa\tilde{x}^3)_j  \label{ineq:22bis} \\
    &=\kappa (1-\bar{x}^1_j-\bar{x}^2_j-\bar{x}^3_j)(1-\tilde{x}^3_j)^{-1}\tilde{x}^3_j \label{ineq:33bis} \\
    &=(1-\bar{x}^1_j-\bar{x}^2_j-\bar{x}^3_j)(1-\tilde{x}^3_j)^{-1}(\bar{x}^1_j+ \bar{x}^2_j+ \bar{x}^3_j) \label{ineq:44bis}\\
    %&=(1-\tilde{x}_j^2)(1-\tilde{x}_j^3)^{-1}\tilde{x}_j^2 \label{eq:4bis}\\
    &<(\bar{x}^1_j+ \bar{x}^2_j+ \bar{x}^3_j), \label{ineq:55bis}
\end{align}
\normalsize
where~\eqref{ineq:11bis} follows from the assumption that $(D^3)^{-1}B^3>(D^2)^{-1}B^2$ and \break $(D^3)^{-1}B^3>(D^1)^{-1}B^1$ leads to $(D^3)^{-1}B^3y>(D^2)^{-1}B^2y$ and  \break $(D^3)^{-1}B^3y>(D^1)^{-1}B^1y$ for any $y \gg \textbf{0}$. %and because $\textbf{0} \ll (\bar{x}^1, \bar{x}^2)$
The inequality~\eqref{ineq:22bis} is due to $(\bar{x}^1+\bar{x}^2+ \bar{x}^3) \leq \kappa \tilde{x}^3$;~\eqref{ineq:33bis} is immediate from~\eqref{eq:x3a:3-eqm}; and~\eqref{ineq:44bis} stems from the fact that $\bar{x}^1_j+\bar{x}^2_j+\bar{x}^3_j=\kappa \tilde{x}^3_j$. The inequality~\eqref{ineq:55bis} follows by noting that since by assumption $\kappa \geq 1$, it must be that $(1-\tilde{x}^3_j)>(1-\bar{x}^1_j- \bar{x}^2_j- \bar{x}^3_j)$. Observe that~\eqref{ineq:55bis} is a contradiction, thus implying that $\kappa < 1$. Thus, $\bar{x}^1+ \bar{x}^2+ \bar{x}^3\ll \tilde{x}^3$. %Now suppose that $\bar{x}^1+ \bar{x}^2+ \bar{x}^3 = \tilde{x}^3$.
\par Note that since $\tilde{x}^3 \gg \textbf{0}$, and $\bar{x}^3 \gg \textbf{0}$, it follows from ~\eqref{eq:x3_3-equib} and~~\eqref{eq:x3a:3-eqm} that $\rho((I-\bar X^1-\bar X^2-\bar X^3)(D^3)^{-1} B^3)=1$, and $\rho(\big{(} I - \Tilde{X}^3\big{)}(D^3)^{-1} B^3)=1$. Since $\bar{x}^1+ \bar{x}^2+ \bar{x}^3\ll \tilde{x}^3$, it must be that $(I-\bar{X}^1-\bar{X}^2-\bar{X}^3)> (I-\tilde{X}^3)$, which from Lemma~\ref{lem:perron_frob} item iv) implies that $\rho((I-\bar{X}^1-\bar{X}^2-\bar{X}^3)(D^3)^{-1}B^3)> \rho((I-\tilde{X}^3)(D^3)^{-1}B^3)$, which is a contradiction of the assumption that there exists a 3-coexistence equilibrium. Consequently, there does not exist a 3-coexistence equilibrium.~\hfill \proofbox
%} 
\par %\textcolor{black}
Theorem~\ref{claim:virus3:strongest} identifies a sufficient condition for the non-existence of a 3-coexistence equilibrium. For the bivirus system, a  condition based on the row sum of matrices $B^k$, for $k \in [2]$, guarantees the non-existence of any 2-coexistence equilibria; see \cite[Corollary~3.10, item ii)]{ye2021convergence}. It is unknown if an analogous condition (but for the nonexistence of any 3-coexistence equilibria) can be obtained for the tri-virus system; a rigorous investigation is left for future work. 
It turns out that a condition less restrictive   than that in Theorem~\ref{claim:virus3:strongest} precludes the possibility of the existence of   certain 2-coexistence equilibria, as formalized in the following theorem.
\begin{thm}\label{claim:virus1weaker}
 Consider system~\eqref{eq:x1}-\eqref{eq:x3} under Assumptions~\ref{assum:base} and~\ref{assum:irreducible}. Suppose that  $\rho((D^k)^{-1}B^k)>1$ for $k \in [3]$.
 \begin{enumerate}[label=\roman*)]
\item   The equilibrium point $(\textbf{0}, \textbf{0}, \textbf{0})$ is unstable.
\item If $(D^2)^{-1}B^2> (D^1)^{-1}B^1$ and $(D^3)^{-1}B^3> (D^1)^{-1}B^1$, then
 \begin{enumerate}[label=\alph*)]
        \item the equilibrium point $(\tilde{x}^1, \textbf{0}, \textbf{0})$ is unstable; and 
        \item there does not exist a 2-coexistence equilibrium of the form a) $(\bar{x}^1, \bar{x}^2, \textbf{0})$, or b) $(\bar{x}^1,  \textbf{0}, \bar{x}^3)$, where $\textbf{0}\ll \bar{x}^1 \ll \textbf{1}$, $\textbf{0}\ll \bar{x}^2 \ll \textbf{1}$, and $\textbf{0}\ll \bar{x}^3 \ll \textbf{1}$.
 \end{enumerate}
%\item Suppose that $D^1=D^2=D^3=I$. If $B^2> B^1$ and $B^2=B^3$, then there is  a line of 2-coexistence equilibria; in particular, there exists a unique $z \gg \textbf{0}$ such that all 2-coexistence equilibria of the form $(\textbf{0}, \tilde{x}^2, \tilde{x}^3)$ can be expressed as $(\textbf{0}, \alpha z, (1-\alpha)z)$ for all $\alpha \in [0,1]$.
 \end{enumerate}
\end{thm}
%}
%\seb{ 
 \textit{Proof:} The proof of statement~i) (resp. statement~ii a)) is the same as that of statement~i) (resp. statement~ii)) in Theorem~\ref{claim:virus3:strongest}.
% \textit{Proof of statement~i): }%By assumption, $D^1=D^2=D^3=I$. 
%  Consider the equilibrium point $(\textbf{0}, \textbf{0}, \textbf{0})$, and observe that the Jacobian computed at this point is as follows: 
%  \begin{align}
% &J(\textbf{0},\textbf{0}, \textbf{0}) = \\
% &\footnotesize
% \begin{bmatrix}
% -D^1+B^1  & \textbf{0}   & \textbf{0} \\
%  \textbf{0} & -D^2+B^2 & \textbf{0} \\
% \textbf{0}  & \textbf{0} & -D^3+B^3 \end{bmatrix}\normalsize,\nonumber
% \end{align}
% % Since $D^1=D^2=D^3=I$, the above can be rewritten as 
% %  \begin{align}
% % &J(\textbf{0},\textbf{0}, \textbf{0}) = \\
% % &\footnotesize
% % \begin{bmatrix}
% % -I+B^1  & \textbf{0}   & \textbf{0} \\
% %  \textbf{0} & -I+B^2 & \textbf{0} \\
% % \textbf{0}  & \textbf{0} & -I+B^3 \end{bmatrix}\normalsize,\nonumber
% % \end{align}
% Since by assumption, $B^k$ is non-negative irreducible and $\rho(B^k)>1$ for $k \in [3]$, it follows from Lemma~\ref{lem:eigspec} that $s(-D^k+B^k)>0$ for $k \in [3]$, which further implies that $s(J(\textbf{0},\textbf{0}, \textbf{0}))>0$. Instability of the equilibrium point $(\textbf{0},\textbf{0}, \textbf{0})$, then, follows from \cite[Theorem~4.7, item ii)]{khalil2002nonlinear}. 
% \par \textit{Proof of statement~ii) a):}  Consider the equilibrium point $(\tilde{x}^1, \textbf{0}, \textbf{0})$, and observe that the Jacobian computed at this point is as follows:
% \begin{align}
% &J(\Tilde{x}^1,\textbf{0}, \textbf{0}) = \\
% &\scriptsize
% \begin{bmatrix}
% -D^1+(I-\Tilde{X}^1)B^1-\hat{B}^1  & -\hat B^{1}   & -\hat B^{1} \\
%  \textbf{0} & -D^2+(I-\Tilde{X}^1)B^2 & \textbf{0} \\
% \textbf{0}  & \textbf{0} & -D^3+(I-\Tilde{X}^1)B^3 \end{bmatrix}\normalsize,\nonumber
% \end{align}
% Note that since $\tilde{x}^1$ is the single-virus endemic equilibrium for virus~1, by the definition of an equilibrium point, we have the following:
% $$[-D^1+(I-\tilde{X}^1)B^1]\tilde{x}^1=\textbf{0}.$$
% Observe that $-D^1+(I-\tilde{X}^1)B^1$ is an irreducible Metzler matrix, and, since $\tilde{x}^1\gg \textbf{0}$, from Lemma~\ref{lem:perron_frob_metz} we have that $s(-D^1+(I-\tilde{X}^1)B^1)=0$. Consequently, from Lemma~\ref{lem:eigspec}, $\rho((I-\tilde{X}^1)(D^1)^{-1}B^1)=1$. By assumption $(D^2)^{-1}B^2> (D^1)^{-1}B^1$, which implies $(I-\tilde{X}^1)(D^1)^{-1}B^1 < (I-\tilde{X}^1)(D^2)^{-1}B^2$. Therefore, since the matrices $(I-\tilde{X}^1)(D^1)^{-1}B^1$ and 
% $(I-\tilde{X}^1)(D^2)^{-1}B^2$ are nonnegative, from Lemma~\ref{lem:perron_frob} item iv), we have that $\rho((I-\tilde{X}^1)(D^2)^{-1}B^2)>1$. Consequently, from Lemma~\ref{lem:eigspec}, it must be that $s(-D^2+(I-\tilde{X}^1)B^2)>0$, which implies that $s(J(\Tilde{x}^1,\textbf{0}, \textbf{0}))>0$, and hence the equilibrium point $(\Tilde{x}^1,\textbf{0}, \textbf{0})$ is unstable.
\par \textit{Proof of statement~ii) b):} We now show that, under the conditions of the Theorem,  a 2-coexistence equilibrium of the form $(\bar{x}^1, \bar{x}^2, \textbf{0})$, where $\textbf{0} \ll \bar{x}^1 \ll \textbf{1}$ and $\textbf{0} \ll \bar{x}^2 \ll \textbf{1}$,  cannot exist. Suppose that, by way of contradiction, there exists a 2-coexistence equilibrium of the form $(\bar{x}^1, \bar{x}^2, \textbf{0})$ with $\textbf{0} \ll \bar{x}^1 \ll \textbf{1}$ and $\textbf{0} \ll \bar{x}^2 \ll \textbf{1}$. By taking recourse to the equilibrium version of equations~\eqref{eq:x1}-\eqref{eq:x3}, we have the following:
\begin{align} 
   \textbf{0} &=  \Big{(} \big{(} I - (\bar{X}^1+\bar{X}^2) \big{)} B^1 - D^1 \Big{)} \bar{x}^1, \label{eq:x1a:eqm-1-bar}\\
    \textbf{0} &=  \Big{(} \big{(} I - (\bar{X}^1+\bar{X}^2) \big{)} B^2 - D^2 \Big{)} \bar{x}^2,\label{eq:x2a:eqm-2-bar} %\\
    %  \textbf{0} &=  \Big{(} \big{(} I - (\Tilde{X}^1+\Tilde{X}^2+\Tilde{X}^3)\big{)} B^3 - D^3 \Big{)} \tilde{x}^3(t) \label{eq:x3a:eqm-3}
  \end{align}
By a suitable rearrangement of terms in~\eqref{eq:x1a:eqm-1-bar} and~\eqref{eq:x2a:eqm-2-bar}, and by noting that the matrices $I - (\bar{X}^1+\bar{X}^2) \big{)} B^1 $ and $I - (\bar{X}^1+\bar{X}^2) \big{)} B^2 $ are irreducible, from Lemma~\ref{lem:perron_frob} item i), we have the following:
\begin{align}
    \rho(I - (\bar{X}^1+\bar{X}^2) \big{)}(D^1)^{-1} B^1 )&=1 \label{rho1}\\
    \rho(I - (\bar{X}^1+\bar{X}^2) \big{)}(D^2)^{-1} B^2 )&=1 \label{rho2}
\end{align}

% $$\rho(I - (\bar{X}^1+\bar{X}^2) \big{)} B^1 )=1;$$ 
% $$\rho(I - (\bar{X}^1+\bar{X}^2) \big{)} B^2 )=1.$$ 
Since by assumption $(D^2)^{-1}B^2>(D^1)^{-1}B^1$, it must be that \break $(I - (\bar{X}^1+\bar{X}^2) \big{)}(D^2)^{-1} B^2> (I - (\bar{X}^1+\bar{X}^2) \big{)}(D^1)^{-1} B^1$, which from Lemma~\ref{lem:perron_frob} item iv) further implies that $\rho((I - (\bar{X}^1+\bar{X}^2) \big{)}(D^2)^{-1} B^2)> \rho((I - (\bar{X}^1+\bar{X}^2) \big{)}(D^1)^{-1} B^1)$, which contradicts  the conclusions from~\eqref{rho1} and~\eqref{rho2}. Thus, there cannot exist a 2-coexistence equilibrium  of the form  $(\bar{x}^1, \bar{x}^2, \textbf{0})$ with $\textbf{0} \ll \bar{x}^1 \ll \textbf{1}$ and $\textbf{0} \ll \bar{x}^2 \ll \textbf{1}$. The nonexistence of a 2-coexistence equilibrium of the form $(\bar{x}^1,  \textbf{0}, \bar{x}^3)$ with $\textbf{0} \ll \bar{x}^1 \ll \textbf{1}$ and $\textbf{0} \ll \bar{x}^3 \ll \textbf{1}$ can be shown analogously, by leveraging the assumption that $(D^3)^{-1}B^3> (D^1)^{-1}B^1$.\hfill \proofbox
%}

Note that Theorem~\ref{claim:virus1weaker}  conclusively rules out the existence of a 2-coexistence equilibrium of the form $(\bar{x}^1, \bar{x}^2, \textbf{0})$, and of the form $(\bar{x}^1, \textbf{0}, \bar{x}^3)$. It, however, does not preclude the possibility of the existence of 2-coexistence equilibria of the form $(\textbf{0}, \bar{x}^2, \bar{x}^3)$. Therefore, Theorem~\ref{claim:virus1weaker}   does not set out a condition for the nonexistence of any 2-coexistence equilibrium in a tri-virus system.
\par Observe that the condition in Theorem~\ref{claim:virus3:strongest} implies (but is not implied by) the condition in Theorem~\ref{claim:virus1weaker}, and, consequently, Theorem~\ref{claim:virus3:strongest} also rules out the existence of a 2-coexistence equilibrium of the form $(\bar{x}^1, \bar{x}^2, \textbf{0})$, and of the form $(\bar{x}^1, \textbf{0}, \bar{x}^3)$.  

\section{Existence of coexistence 
%coexistence 
equilibria}\label{sec:existence:coexistence}
Section~\ref{sec:nonexistence:coexistence:equilibria} focuses on the nonexistence of certain equilibria. In this section, we provide conditions for the existence of a 2-coexistence equilibrium and identify circumstances under which the existence of one or more 3-coexistence equilibria can be inferred. Define, for each $i \in [3]$, $\bar X^i: =\diag{(\bar{x}^i)}$, where $\bar x^i$ is as defined in Section~\ref{sec:prob:formulation}.
The following proposition guarantees the existence of a 2-coexistence equilibrium.
%\seb{
\begin{prop}\label{prop:2-coexistence}
Consider system~\eqref{eq:x1}-\eqref{eq:x3} under Assumptions~\ref{assum:base} and~\ref{assum:irreducible}. %Suppose that matrix $B^k$ for $k \in [3]$ are irreducible. 
Suppose that  $\rho((D^k)^{-1}B^k)>1$ for $k \in [3]$. 
%Let $\tilde{x}^1$,  $\tilde{x}^2$, and $\tilde{x}^3$ denote the single-virus endemic equilibrium corresponding to virus~1,~2, and~3, respectively. 
There exists at least one 2-coexistence equilibrium, i.e., $(\bar{x}^1, \bar{x}^2, \bar{x}^3)$ where 
$\textbf{0}\ll \bar{x}^i, \bar{x}^j \ll \textbf{1}$
 for some $i, j \in [3], i\neq j$, 
%being strictly positive vectors 
and, for $\ell \neq i, \ell \neq j$, $\bar{x}^\ell=\textbf{0}$. %Moreover, if $s(-D^3+(I-\bar{X}^1-\bar{X}^2))B^3>0$, $s(-D^2+(I-\bar{X}^1-\bar{X}^3))B^2>0$, and $s(-D^1+(I-\bar{X}^2-\bar{X}^3))B^1>0$, then %any
%every 2-coexistence equilibrium is unstable.
\end{prop}
\textit{Proof:} By assumption, we have that $\rho((D^k)^{-1}B^k)>1$ for $k \in [3]$. Therefore, from Proposition~\ref{prop:necessity} it follows that there exist single-virus endemic equilibria (also referred to as boundary equilibria) corresponding to virus~1,~2, and~3, namely $\tilde{x}^1$,  $\tilde{x}^2$, and $\tilde{x}^3$, respectively. Observe that each of these equilibria could be either stable or unstable, which implies that there must be either i) a pair which are both stable or ii) a pair that are both unstable. We consider the two cases separately.\\
\textit{Case i):} Assume, without loss of generality, that the two stable boundary equilibria are $(\tilde x^1,{\bf{0}},{\bf{0}})$ and $({\bf{0}},\tilde x^2,{\bf{0}})$. Observe that for the bivirus system, defined by neglecting $x^3$, the boundary equilibria are $(\tilde x^1,{\bf{0}})$ and $({\bf{0}},\tilde x^2)$, the existence of which, to reiterate, is guaranteed from the assumption that $\rho(B^k)>1$ for $k=1,2$. The bivirus system is known to be monotone \cite[Lemma~3.3]{ye2021convergence}. Therefore, from \cite[Theorem~2.8]{smith1988systems} and \cite[Proposition~2.9]{smith1988systems}, it follows that there is necessarily an unstable equilibrium $(\bar x^1,\bar x^2)$, where $\textbf{0} \ll (\bar x^1,\bar x^2) \ll \textbf{1}$.  Corresponding to this equilibrium of the bivirus system is the equilibrium $(\bar x^1,\bar x^2,{\bf{0}} )$ of the trivirus system.\\
\textit{Case ii):} Assume, without loss of generality, that the two unstable boundary equilibria are $(\tilde x^1,{\bf{0}},{\bf{0}})$ and $({\bf{0}},\tilde x^2,{\bf{0}})$. Since the bivirus system (defined by neglecting $x^3$) is monotone. Since for monotone systems between any two unstable boundary equilibria, there must  exist a stable equilibrium \cite{smith1988systems}, there must be an equilibrium point $(\bar x^1,\bar x^2)$. Moreover, since from \cite[Lemma~3.1]{ye2021convergence}, it is known that, for the bivirus system, besides the DFE, only the equilibrium points $(\tilde x^1,{\bf{0}})$ and $({\bf{0}},\tilde x^2)$ can be on the boundary of the surface defined by the set $\mathcal D^k$, $k=1,2$; it follows that the equilibrium point $(\bar x^1,\bar x^2)$ must lie in the interior of $\mathcal D^k$, $k=1,2$, which further implies that $\textbf{0} \ll (\bar x^1,\bar x^2) \ll \textbf{1}$,  with $(\bar x^1,\bar x^2,{\bf{0}} )$ again an equilibrium of the trivirus system.

Hence, it follows that in either case, there exists a 2-coexistence equilibrium %namely $(\bar x^1,\bar x^2,{\bf{0}})$, 
for system~\eqref{eq:x1}-\eqref{eq:x3}.\hfill \proofbox
% Let us assume that the  2-coexistence equilbrium that exists is $(\bar x^1,\bar x^2,{\bf{0}})$. 
% By assumption, $s(-D^3+(I-\bar{X}^1-\bar{X}^2)B^3)>0$. Then by evaluating the Jacobian at $(\bar x^1,\bar x^2,{\bf{0}})$, and observing that $J(\bar x^1,\bar x^2,{\bf{0}})$ is block upper triangular, with $-D^3+(I-\bar{X}^1-\bar{X}^2))B^3$  being one of the blocks along the diagonal. Since $s(-D^3+(I-\bar{X}^1-\bar{X}^2)B^3)>0$, it follows that $J(\bar x^1,\bar x^2,{\bf{0}})$ is not Hurwitz, and, consequently, $(\bar x^1,\bar x^2,{\bf{0}})$ is unstable. Observe  that if the 2 coexistence equilibrium that exists is $(\bar x^1,{\bf{0}},\bar x^3)$ (resp. $({\bf{0}}, \bar x^2,,\bar x^3)$) then since by assumption $s(-D^2+(I-\bar{X}^1-\bar{X}^3)B^2)>0$ (resp. $s(-D^1+(I-\bar{X}^2-\bar{X}^3)B^1)>0$), by reasoning analogous to the above, it follows that $(\bar x^1,{\bf{0}},\bar x^3)$ (resp. $({\bf{0}}, \bar x^2,,\bar x^3)$) is unstable.
%}
%~\hfill \qed
\vspace{2mm}
\par We now turn our attention to identifying a sufficient condition for the existence of a 3-coexistence equilibria. To this end, we will borrow some results on systems invariant in $\mathbb R^n_{+}$ with ultimately bounded trajectories, see \cite{hofbauer1990index}; a tri-virus system indeed meets these requirements.
%\seb{Prior to  doing so, 


\par Specifically, and following the terminology of \cite{hofbauer1990index}, we classify equilibria as saturated or unsaturated and derive a counting lower bound on the possible equilibria of tri-virus systems.
% and, subsequently, index properties of the saturated equilibria are considered.  
Observe that for a boundary equilibrium or a 2-coexistence equilibrium, the associated Jacobian is block-triangular; see~\eqref {jacob}. We say that an equilibrium is saturated if the diagonal block corresponding to the zero entries of the equilibrium is Hurwitz and unsaturated otherwise~\cite{hofbauer1990index}. %It turns out that
A boundary equilibrium of~\eqref{eq:x1}-\eqref{eq:x3} is saturated if, and only if, said boundary equilibrium is locally exponentially stable; this follows immediately by noting the structure of the Jacobian matrix, evaluated at a boundary equilibrium, see \eqref{eq:boundaryJac}.
% , and the condition for a fixed point to be saturated provided in \cite{hofbauer1990index}. 
However, a 2-coexistence equilibrium might be unstable but still saturated because we only consider one diagonal block  of the Jacobian to determine if it is saturated, while stability is determined by the  remainder of the Jacobian truncated to remove this diagonal block; for a 2-coexistence equilibrium, this may or may not be Hurwitz. In contrast, a 3-coexistence equilibrium, with no diagonal block in the Jacobian matrix corresponding to zero entries of the equilibrium, is always saturated.

\par The index of an equilibrium, as per the convention in %used by 
\cite{hofbauer1990index}, is defined as the sign, i.e., $\pm 1$, of the negative of the determinant of the Jacobian at the point. (Note that all such Jacobians are nonsingular for generic parameter values, as established in Proposition~\ref{prop:finite:equilibria}).  We will require the following results from \cite{hofbauer1990index}: (a) if a solution beginning in the interior converges to a limit as $t\to\infty$, that limit must be a saturated fixed point, and (b) the sum of the indices {\textit{over saturated equilibria}} is necessarily $+1$.  We will now show that there exists at least one saturated equilibrium, and if the only such saturated equilibrium  is a boundary equilibrium, it must be stable.

%Observe then that if the sum of the indices of the saturated boundary and saturated 2-coexistence equilibria is other than $+1$, there is necessarily a nonzero number of 3-coexistence equilibria.}

% {\color{red} 
%Comment from BA: Here is an outline of a possible inclusion. It could be set up as a proposition. 

% While this section focuses on the nonexistence of certain equilibria, we finish it by recording a further result on the existence as opposed to nonexistence. 
% Suppose $n\geq 2$. Since there are precisely three boundary equilibria, each stable or unstable, there must be a pair which are both stable, or a pair which are both unstable. In the former case, Denote the equilibria by $(\tilde x^1,{\bf{0}},{\bf{0}})$ and $({\bf{0}},\tilde x^2,{\bf{0}})$. For the bivirus system defined by neglecting $x^3$, there will be boundary equilibria $(\tilde x^1,{\bf{0}})$ and $({\bf{0}},\tilde x^2)$. In the first case, there is necessarily an unstable equilibrium $(\bar x^1,\bar x^2)$, because of the monotone nature of the system, and the requirement that ``between" any two stable equilibria, there is an unstable one, \cite{smith1988systems}. In the second case there is necessarily a stable equilibrium $(\bar x^1,\bar x^2)$ since such a equilibrium must exist in a bivirus system, again because of the monotone property, and being precluded from existing on the boundary it is necessarily a coexistence equilibrium In either case, $(\bar x^1,\bar x^2,{\bf{0}})$ becomes and equilibrium of the 3-virus system. 


% \subsection*{Commentary on Hofbauer paper}



% It is possible to obtain a ``counting" or index-based result which for certain patterns of stability of boundary and 2-coexistence equilibria could imply the existence of one or more 3-coexistence equilibria. The key is to use a result available for systems that are invariant in $\mathbb R^n_{+}$ with trajectories that are ultimately bounded, see \cite{hofbauer1990index}. Obviously, these restrictions are met by a trivirus system. 

% First, equilibria have to be classified as saturated, or otherwise. Then the index properties of the saturated equilibria are considered. 

% In more detail, observe that for a boundary equilibrium or a 2-coexistence equilibrium, the associated Jacobian is block triangular, see e.g.  \eqref{eq:boundaryJac}. If the diagonal block corresponding to the zero entries of the equilibrium is Hurwitz, the equilibrium is said to be saturated. A 3-coexistence equilibrium, with no diagonal block,  is also said to be saturated. Note that for a boundary equilibrium, the equilibrium is saturated if and only if the equilibrium is stable. However, a 2-coexistence equilibrium might be unstable, but still saturated. 

% Following more or less standard practice,  the index of an equilibrium is defined as the sign, i.e. $\pm 1$, of the negative of the determinant of the Jacobian at the point. (It is assumed that all such Jacobians are nonsingular, which for generic parameter values is the case, as demonstrated in the proof of Proposition~\ref{prop:finite:equilibria}). 

% The key results are that (a) if a solution beginning in the interior goes to a limit as $t\to\infty$, that limit must be a saturated fixed point, and (b) the sum of the indices {\textit{over saturated equilibria}} is necessarily $+1$. Observe then that if the sum of the indices of the saturated boundary and saturated 2-coexistent equilibria is other than $+1$, there is necesarily a nonzero number of 3-coexistent equilibria.  
% }

%\ben{Ben comment: Providing a comprehensive study of the tri-virus system using \cite{hofbauer1990index} would be really complex, mainly because we need a lot of notation and auxiliary definitions. One simple sufficient condition (although one could argue both ways as to whether it is easy to check) is as follows: 

%For any 2-coexistence equilibrium with $\bar x^i, \bar x^j > \textbf{0}$ and $\bar x^k = 0$, we say that it is saturated if and only if $s(-D^k + (I-\bar X^i-\bar X^j)B^k) < 0$. 

%Theorem: There must always be at least one of i) a stable boundary equilibrium, or ii) a saturated 2-coexistence equilibrium, or iii) a 3-coexistence equilibrium.\\ 
%\seb{
\begin{thm}\label{thm:hofbauer}
 Consider system~\eqref{eq:x1}-\eqref{eq:x3} under Assumption~\ref{assum:base}, and assume, in the light of genericity of the parameters, that all equilibria are nondegenerate. Suppose that, for each $k \in [3]$, $\rho((D^k)^{-1}B^k)>1$.
There exists at least one of the following:
\begin{enumerate}[label=\roman*)]
\item  a stable boundary equilibrium;
\item a saturated 2-coexistence equilibrium; or
\item a 3-coexistence equilibrium.
\end{enumerate}
%either a stable boundary equilibrium, or a saturated 2-coexistence equilibrium, or a 3-coexistence equilibrium.   
\end{thm}

% \textit{Proof sketch:} The proof should follow from \cite[Theorem~2]{hofbauer1990index}, and by noting that an interior fixed point is always saturated.
% \\
\textit{Proof:} We first show that the assumptions of the Theorem guarantee the existence of boundary equilibria and at least one 2-coexistence equilibrium.\\
By assumption, we have that $\rho((D^k)^{-1}B^k)>1$ for $k \in [3]$. Therefore, from Proposition~\ref{prop:necessity} it follows that there exist single-virus endemic equilibria (also  referred to as boundary equilibria) corresponding to virus~1,~2, and~3, namely $\tilde{x}^1$,  $\tilde{x}^2$, and $\tilde{x}^3$, respectively. Moreover, from Proposition~\ref{prop:2-coexistence}, the existence of the said boundary equilibria guarantees the existence of at least one 2-coexistence equilibrium.\\
Observe that Lemma~\ref{lem:pos} guarantees that, for each $k \in [3]$, $x^k(0)>\textbf{0}$ implies that $x^k(t)>\textbf{0}$ for all $t \in \mathbb{R}_{\geq 0}$, and that the set $\mathcal D$ (which is compact) is forward invariant. Therefore, from \cite[Theorem~2]{hofbauer1990index}, it follows that system~\eqref{eq:x1}-\eqref{eq:x3} has at least one saturated fixed point. There are two cases to consider.\\
Case 1: Suppose the aforementioned saturated fixed point is in the interior of $\mathcal D$. 
 Note that any fixed point in the interior of $\mathcal D$ is, due to Lemma~\ref{lem:equi_non-zero_nonone:1}, of the form $(\bar{x}^1, \bar{x}^2, \bar{x}^3)$, where $\textbf{0} \ll (\bar{x}^1, \bar{x}^2, \bar{x}^3) \ll \textbf{1}$, thus implying that $(\bar{x}^1, \bar{x}^2, \bar{x}^3)$ is a 3-coexistence equilibrium. \\%Since any fixed point in the interior of  $\mathcal D$ is saturated, it follows that, if a 3-coexistence equilibrium exists it must be saturated.
Case 2: Suppose there are no fixed points  in the interior of $\mathcal D$. This implies that a saturated fixed point must be on the boundary of $\mathcal D$  \cite{hofbauer1990index}. Therefore,  at least one of the following should be true: a) at least one of the single-virus boundary equilibrium is saturated; or b) at least one of the 2-coexistence equilibria 
%(whose existence is guaranteed by the assumptions of the theorem and Proposition~\ref{prop:2-coexistence}) 
is saturated.~\hfill\proofbox \vspace{2mm}
The following is an immediate consequence of Theorem~\ref{thm:hofbauer}, and \cite[Theorem~1]{hofbauer1990index}.
\begin{cor}\label{cor:hofbauer}
 Consider system~\eqref{eq:x1}-\eqref{eq:x3} under Assumption~\ref{assum:base}  and with generic parameter values ensuring that all equilibria are isolated. Suppose that, for each $k \in [3]$, $\rho((D^k)^{-1}B^k)>1$. Let $(\tilde{x}^1,\textbf{0}, \textbf{0})$,  $(\textbf{0}, \tilde{x}^2,\textbf{0})$ and  $(\textbf{0}, \textbf{0},\tilde{x}^3)$ denote the boundary equilibria corresponding to viruses~1,2 and 3, respectively. Suppose  further for such  equilibria that, for each pair $(i,j)$ such that $i, j \in [3]$ and $i \neq j$, $\rho((I{-}\tilde{X}^i)(D^j)^{-1}B^j)>1$.  Further, let $(\bar{x}^i, \bar{x}^j, \bar{x}^k)$  be a 2-coexistence equilibrium, i.e., such that for some $i, j \in [3]$, $\bar{x}^i, \bar{x}^j>0$, and for some $k \in [3]$, $\bar{x}^k=\textbf{0}$.
  Suppose further that, for every such triplet $(i,j,k)$, $s(-D^k + (I-\bar X^i-\bar X^j)B^k) \geq 0$. Then there exists at least one 3-coexistence equilibrium. %and it is saturated. %and there exists a trajectory from the interior of $\mathcal{D}$ that reaches it. 
In fact, there must be an odd number of 3-coexistence equilibrium.
\end{cor}
\textit{Proof:} By assumption, we have that $\rho((D^k)^{-1}B^k)>1$ for $k \in [3]$. Therefore, from Proposition~\ref{prop:necessity} it follows that there exist boundary equilibria corresponding to virus~1,~2, and~3, namely $\tilde{x}^1$,  $\tilde{x}^2$, and $\tilde{x}^3$, respectively. By assumption, for each pair $(i,j)$ such that $i, j \in [3]$ and $i \neq j$, $\rho((I{-}\tilde{X}^i)(D^j)^{-1}B^j)>1$. Therefore, from Theorem~\ref{thm:local}, each of the boundary equilibrium points, $(\tilde{x}^1,\textbf{0}, \textbf{0})$,  $(\textbf{0}, \tilde{x}^2,\textbf{0})$ and  $(\textbf{0}, \textbf{0},\tilde{x}^3)$,  is unstable,  and, thus, unsaturated. Given the existence of the boundary equilibria, from Proposition~\ref{prop:2-coexistence}, the existence of at least one equilibrium (i.e., 2-coexistence equilibrium), $(\bar{x}^i, \bar{x}^j, \bar{x}^k)$ such that for some $i, j \in [3]$, $\bar{x}^i, \bar{x}^j>0$, and for some $k \in [3]$, $\bar{x}^k=\textbf{0}$, is guaranteed. By assumption, for every such triplet $(i,j,k)$, $s(-D^k + (I-\bar X^i-\bar X^j)B^k) \geq 0$, which implies that no 2-coexistence equilibrium is saturated. Therefore, from Theorem~\ref{thm:hofbauer}, it must be that there exists an equilibrium point $(\bar{x}^i, \bar{x}^j, \bar{x}^k)$ such that   $\bar{x}^i, \bar{x}^j, \bar{x}^k>\textbf{0}$, i.e., a 3-coexistence equilibrium.\hfill \proofbox %Since any equilibrium point $(\bar{x}^i, \bar{x}^j, \bar{x}^k)$ such that   $\bar{x}^i, \bar{x}^j, \bar{x}^k>\textbf{0}$ must necessarily be in the interior of $\mathcal D$, it must be that said equilbrium point is saturated \cite{hofbauer1990index}. \\
%Since the equilibrium point $(\bar{x}^i, \bar{x}^j, \bar{x}^k)$ such that   $\bar{x}^i, \bar{x}^j, \bar{x}^k>\textbf{0}$ is saturated, it follows from \cite[Theorem~1]{hofbauer1990index} that there exists at least one solution $x(t) \in \mathbb{R}^{3n}_{+}$ such that $\lim_{t \rightarrow \infty}x(t)=(\bar{x}^i, \bar{x}^j, \bar{x}^k)$ where   $\bar{x}^i, \bar{x}^j, \bar{x}^k>\textbf{0}$.\\


%\par Corollary: Suppose that every boundary equilibrium is unstable and every 2-coexistence equilibrium is not saturated. Then, there exists at least one 3-coexistence equilibrium and there exists a trajectory from the interior of $\mathcal{D}$ that reaches it. In fact, there must be an odd number of 3-coexistence equilibrium.

% Suppose that every boundary equilibrium $(\textbf{0}, \hdots, \tilde x^k, \hdots, \textbf{0})$ is unstable. Suppose further that, for any 2-coexistence equilibrium with $\bar x^i, \bar x^j > \textbf{0}$ and $\bar x^k = 0$, the corresponding bivirus equilibrium $(\bar x^i, \bar x^j)$ is unstable and $s(-D^k + (I-\bar X^i-\bar X^j)B^k) > 0$. Then, there exists at least one 3-coexistence equilibrium and a trajectory from the interior of $\mathcal{D}$ that reaches it. In fact, there must be an odd number of 3-coexistence equilibrium.
%}


\section{Existence and attractivity of a line of coexistence  equilibria for nongeneric tri-virus networks}\label{sec:line:attractivity}
\par 
%Proposition~\ref{prop:necessity} and Theorem~\ref{thm:local}, respectively, deal with the existence and local exponential convergence to a boundary equilibrium. where one, and only one, virus is alive; the rest have become extinct.  When two competing viruses pervade a population, it is also possible that the two  viruses could exist in some sort of balance with each other \cite{liu2019analysis,pare2021multi,ye2021convergence,axel2020TAC}.
Sections~\ref{sec:persistence:one:virus}-\ref{sec:existence:coexistence} pertain to the existence or nonexistence of various endemic equilibria. Note that in those sections, the discussion centers around the (non)existence (and, in Section~\ref{sec:persistence:one:virus} also stability (or lack thereof)) of \emph{an} equilibrium point. In this section, however, we will identify a scenario that guarantees the existence and local exponential attractivity of a line of coexistence  equilibria. As for the bivirus case \cite{ye2021convergence}, special values of the system parameters are needed.

%In this section, we are interested in identifying a scenario which guarantees the existence and local exponential attractivity of a continuum of coexistence  equilibria.
% (i.e., at least two, but possibly all three, viruses exist in some sort of balance with each other), when all three %competing 
% viruses 
% %are prevalent 
% persist
% in a population. 

% We will consider a special case, namely that of homogeneous spread, which is formalized via the following assumption.
% \begin{assm}\label{assm:homogenous}
% We suppose that
% \begin{enumerate}[label=(\roman*)]
%     \item All three viruses are spreading over the same graph.
%     \item For all $i \in [n]$ $\delta_i^1 =\delta^1$, $\delta_i^2 =\delta^2$, and $\delta_i^3 =\delta^3$.
%     \item For all $i=j \in [n]$ and $(i,j) \in \mathcal E$, $\beta_{ij}^1=\beta^1>0$, $\beta_{ij}^2=\beta^2>0$, and $\beta_{ij}^3=\beta^3>0$.
% \end{enumerate}
 
% \end{assm}
% The following result identifies one such condition, and is an extension of \cite[Theorem~6]{liu2019analysis}.
% \begin{prop}\label{prop:line} Consider system~\eqref{eq:x1}-\eqref{eq:x3} under Assumptions~\ref{assum:base} and~\ref{assm:homogenous}. Further, suppose that the matrix $A$ is irreducible and that $s(A)>\frac{\delta^1}{\beta^1}=\frac{\delta^2}{\beta^2} = \frac{\delta^3}{\beta^3}$.
%     \begin{enumerate}[label=\roman*)]
%         \item \label{no3} If $(\tilde{x}^1, \tilde{x}^2, \textbf{0})$ is an equilibrium of~\eqref{eq:x1}-\eqref{eq:x3}, then $\tilde{x}^1= \alpha_1\tilde{x}^2$ for some $\alpha_1>0$. For each $\alpha_1>0$, there exists a unique pair  $(\tilde{x}^1, \tilde{x}^2)$ such that $\tilde{x}^1= \alpha_1\tilde{x}^2$. Furthermore, if $\tilde{x}^1= \tilde{x}^2$, then 
%         the equilibrium set $(\beta_1\tilde{x}^1, (1-\beta_1)\tilde{x}^1)$ with $\beta_1 \in [0,1]$ is locally exponentially attractive, i.e., for all $x(0) \in \mathcal D$ such that $x(0)$ is close to a point on the line $(\beta_1\tilde{x}^1, (1-\beta_1)\tilde{x}^2)$, $x(t)$ approaches the line $(\beta_1\tilde{x}^1, (1-\beta_1)\tilde{x}^1)$ exponentially fast. 
%         \item If $(\textbf{0},\tilde{x}^2, \tilde{x}^3)$ is an equilibrium of~\eqref{eq:x1}-\eqref{eq:x3}, then $\tilde{x}^2= \alpha_2\tilde{x}^3$ for some $\alpha_2>0$. For each $\alpha_2>0$, there exists a unique pair  $(\tilde{x}^1, \tilde{x}^2)$ such that $\tilde{x}^1= \alpha_2\tilde{x}^2$. Furthermore, if $\tilde{x}^2= \tilde{x}^3$, the equilibrium set $(\beta_2\tilde{x}^2, (1-\beta_2)\tilde{x}^2)$ with $\beta_2 \in [0,1]$ is locally exponentially attractive. 
%           \item If $(\tilde{x}^1, \textbf{0}, \tilde{x}^3)$ is an equilibrium of~\eqref{eq:x1}-\eqref{eq:x3}, then $\tilde{x}^1= \alpha_3\tilde{x}^3$ for some $\alpha_3>0$. For each $\alpha_3>0$, there exists a unique pair  $(\tilde{x}^1, \tilde{x}^3)$ such that $\tilde{x}^1= \alpha_3\tilde{x}^3$. Furthermore, if $\tilde{x}^1= \tilde{x}^3$, the equilibrium set $(\beta_3\tilde{x}^1, (1-\beta_3)\tilde{x}^1)$ with $\beta_3 \in [0,1]$ is locally exponentially attractive.  
%     \end{enumerate}
% \end{prop}
% \textit{Proof:} We prove statement~\ref{no3}. The proofs for the other two statements are same up to a suitable adjustment of notation.\\
% Suppose that $(\tilde{x}^1, \tilde{x}^2, \textbf{0})$ is an equilibrium of~\eqref{eq:x1}-\eqref{eq:x3}. Therefore, by the definition of an equilibrium point, we have the following:
% \begin{align} 
%   \textbf{0} &=  \Big{(} \big{(} I - (\Tilde{X}^1+\Tilde{X}^2) \big{)} B^1 - D^1 \Big{)} \tilde{x}^1(t), \label{eq:x1:eqm}\\
%     \textbf{0} &=  \Big{(} \big{(} I - (\Tilde{X}^1+\Tilde{X}^2)\big{)} B^2 - D^2 \Big{)} \tilde{x}^2(t) \label{eq:x2:eqm}
%   \end{align}
%   Observe that ~\eqref{eq:x1:eqm} and~\eqref{eq:x2:eqm} are equilibrium versions of bivirus system equations. The result then follows by the analysis in the proof of \cite[Theorem~6]{liu2019analysis}.
%   \par By assumption $\tilde{x}^1=\tilde{x}^2$. Hence, exponential  convergence to the set $(\beta_1\tilde{x}^1, (1-\beta_1)\tilde{x}^1)$ with $\beta_1 \in [0,1]$ is due to \cite[Proposition~9]{ye2021convergence}.
%   ~$\blacksquare$
  
 
  
% % More generally, a sufficient condition for the existence of a plane of coexisting equilibria has been identified in \cite[Proposition~1]{bens-notes}.  
% \par We now identify another sufficient condition for the existence of multiple lines of equilibria, but without the assumption of homogeneous spread. We need the following assumption, instead.
% \begin{assm}\label{assm:hetero:samegraph}
% We suppose that
% \begin{enumerate}[label=(\roman*)]
%     \item All three viruses are spreading over the same graph.
%     \item For all $i \in [n]$ $\delta_i^1 =\delta_i^2 =\delta_i^3>0$.
%     \item For all $i=j \in [n]$ and $(i,j) \in \mathcal E$, $\beta_{ij}^1=\beta_{ij}^2=\beta_{ij}^3$.
% \end{enumerate}
% \end{assm}

% \begin{prop}\label{prop:hetero:lines}
% Consider system~\eqref{eq:x1}-\eqref{eq:x3} under Assumptions~\ref{assum:base} and~\ref{assm:hetero:samegraph}. Further, suppose that the matrix $A$ is irreducible and that $s(-D+B)>0$. 
% \begin{enumerate}[label=\roman*)]
%     \item \label{no3:a}If $(\tilde{x}^1, \tilde{x}^2, \textbf{0})$ is an equilibrium of~\eqref{eq:x1}-\eqref{eq:x3}, then $\tilde{x}^1= \alpha_1\tilde{x}^2$ for some $\alpha_1>0$. For each $\alpha_1>0$, there exists a unique pair  $(\tilde{x}^1, \tilde{x}^2)$ such that $\tilde{x}^1= \alpha_1\tilde{x}^2$. Furthermore, if $\tilde{x}^1= \tilde{x}^2$, then the equilibrium set $(\beta_1\tilde{x}^1, (1-\beta_1)\tilde{x}^1)$ with $\beta_1 \in [0,1]$ is locally exponentially attractive. 
%      \item If $(\textbf{0},\tilde{x}^2, \tilde{x}^3)$ is an equilibrium of~\eqref{eq:x1}-\eqref{eq:x3}, then $\tilde{x}^2= \alpha_2\tilde{x}^3$ for some $\alpha_2>0$. For each $\alpha_2>0$, there exists a unique pair  $(\tilde{x}^1, \tilde{x}^2)$ such that $\tilde{x}^1= \alpha_2\tilde{x}^2$. Furthermore, if $\tilde{x}^2= \tilde{x}^3$, the equilibrium set $(\beta_2\tilde{x}^2, (1-\beta_2)\tilde{x}^2)$ with $\beta_2 \in [0,1]$ is locally exponentially attractive. 
%           \item If $(\tilde{x}^1, \textbf{0}, \tilde{x}^3)$ is an equilibrium of~\eqref{eq:x1}-\eqref{eq:x3}, then $\tilde{x}^1= \alpha_3\tilde{x}^3$ for some $\alpha_3>0$. For each $\alpha_3>0$, there exists a unique pair  $(\tilde{x}^1, \tilde{x}^3)$ such that $\tilde{x}^1= \alpha_3\tilde{x}^3$. Furthermore, if $\tilde{x}^1= \tilde{x}^3$, the equilibrium set $(\beta_3\tilde{x}^1, (1-\beta_3)\tilde{x}^1)$ with $\beta_3 \in [0,1]$ is locally exponentially attractive.
% \end{enumerate}
% \end{prop}
% \textit{Proof:} We prove statement~\ref{no3:a}. The proofs for the other two statements are same up to a suitable adjustment of notation.\\
% Suppose that $(\tilde{x}^1, \tilde{x}^2, \textbf{0})$ is an equilibrium of~\eqref{eq:x1}-\eqref{eq:x3}. Therefore, by the definition of an equilibrium point, we have the following:
% \begin{align} 
%   \textbf{0} &=  \Big{(} \big{(} I - (\Tilde{X}^1+\Tilde{X}^2) \big{)} B^1 - D^1 \Big{)} \tilde{x}^1(t), \label{eq:x1a:eqm}\\
%     \textbf{0} &=  \Big{(} \big{(} I - (\Tilde{X}^1+\Tilde{X}^2)\big{)} B^2 - D^2 \Big{)} \tilde{x}^2(t) \label{eq:x2a:eqm}
%   \end{align}
%   Observe that ~\eqref{eq:x1a:eqm} and~\eqref{eq:x2a:eqm} are equilibrium versions of bivirus system equations. The result then follows by the analysis in the proof of \cite[Theorem~7]{liu2019analysis}.
%   \par By assumption $\tilde{x}^1=\tilde{x}^2$. Hence, exponential convergence to the set $(\beta_1\tilde{x}^1, (1-\beta_1)\tilde{x}^1)$ with $\beta_1 \in [0,1]$ is due to \cite[Proposition~9]{ye2021convergence}.

% %Observe that Proposition~\ref{prop:hetero:lines} (and also Proposition~\ref{prop:line}) assumes that the dynamics of virus~3 have been nullified. In other words, Proposition~\ref{prop:hetero:lines} addresses the case where $x_i^3(0)=0$ for $i \in [n]$. For the case where $x_i^3(0) \neq 0$ for  some $i \in [n]$, while one can be assured of the existence of  multiple lines of equilibria, local exponential attractivity to these lines is contingent on the hurwitzness of the linearized (around the equilibrium point $(\tilde{x}^1, \tilde{x}^2, \textbf{0})$) state matrix for virus~3. The following theorem formalizes the same.}

% % 
% % \begin{thm}\label{thm:nonzero:init:condns:dummy}
% % Consider system~\eqref{eq:x1}-\eqref{eq:x3} under Assumptions~\ref{assum:base} and~\ref{assm:hetero:samegraph}. Suppose that the matrix $A$ is irreducible and that $s(-D+B)>0$. Suppose that $D^k=I$ for $k \in [3]$. If $(\tilde{x}^1, \tilde{x}^2, \textbf{0})$ is an equilibrium of~\eqref{eq:x1}-\eqref{eq:x3}, then $\tilde{x}^1= \alpha_1\tilde{x}^2$ for some $\alpha_1>0$. For each $\alpha_1>0$, there exists a unique pair  $(\tilde{x}^1, \tilde{x}^2)$ such that $\tilde{x}^1= \alpha_1\tilde{x}^2$. Furthermore, if $\tilde{x}^1= \tilde{x}^2$, then
% % \begin{enumerate} [label=\roman*)]
% %     \item if $s(-I+(I-\tilde{X}^1-\tilde{X}^2)B^3)<0$, then 
% % the equilibrium set $(\beta_1\tilde{x}^1, (1-\beta_1)\tilde{x}^1, \textbf{0})$ with $\beta_1 \in [0,1]$ is locally exponentially attractive.
% %     \item if $s(-I+(I-\tilde{X}^1-\tilde{X}^2)B^3)>0$, then 
% % the equilibrium set $(\beta_1\tilde{x}^1, (1-\beta_1)\tilde{x}^1, \textbf{0})$ with $\beta_1 \in [0,1]$ is unstable.
% % \end{enumerate}
% % \end{thm}
% % \textit{Proof:} The proof for existence of a line of equilibria of the form $\tilde{x}^1= \alpha_1\tilde{x}^2$ for some $\alpha_1>0$, and that, for each $\alpha_1>0$, there exists a unique pair  $(\tilde{x}^1, \tilde{x}^2)$ such that $\tilde{x}^1= \alpha_1\tilde{x}^2$ is the same as in the proof of Proposition~\ref{prop:hetero:lines}. We focus on proving local exponential attractivity or lack thereof in the sequel.
% % \par \textit{Proof of statement i):} Note that the Jacobian computed at the equilibrium point $(\tilde{x}^1, \tilde{x}^2, \textbf{0})$ is as in equation~\eqref{jacob:thm2}. 
% % \begin{figure*}[h]
% % 	{\noindent}
% % \begin{align}\label{jacob:thm2}
% % &J(\tilde{x}^1, \tilde{x}^2, \textbf{0}) = \\
% % &\scriptsize
% % \begin{bmatrix}
% % -I+(I-\tilde{X}^1-\tilde{X}^2)B^1-\diag(B^1x^1) & -\diag(B^1x^1)   & -\diag(B^1x^1)  \\
% % -\diag(B^2x^2) & -I+(I-\tilde{X}^1-\tilde{X}^2)B^2-\diag(B^2x^2)  & -\diag(B^2x^2)\\
% %  \textbf{0}&  \textbf{0}& -I+(I-\tilde{X}^1-\tilde{X}^2)B^3 \end{bmatrix}\normalsize,\nonumber
% % \end{align}
% % \end{figure*}
% % Hence, we can rewrite $ J(\tilde{x}^1, \tilde{x}^2, \textbf{0})$ as %\scriptsize
% % \begin{align}\label{J:forhatx1:partitioned}
% %     J(\tilde{x}^1, \tilde{x}^2, \textbf{0})
% %     =
% %     & %\scriptsize
% %     \begin{bmatrix} 
% %     \bar{J}(\tilde{x}^1, \tilde{x}^2, \textbf{0}) 
% %       %\diag(B^1\hat{x}^1)
% %       && \hat{J} \\
% %       \mathbf{0} && -I+(I-\tilde{X}^1-\tilde{X}^2)B^3
% %     \end{bmatrix}, 
% % \end{align}
% % \normalsize
% % where
% % \begin{align}\label{jacob:hatx1:22submatrix} 
% % &\bar{J}(\tilde{x}^1, \tilde{x}^2, \textbf{0}) = 
% % & \scriptsize
% % \begin{bmatrix} 
% % -I+(I-\tilde{X}^1-\tilde{X}^2)B^1-\hat{B}^1 & -\hat{B}^1   \\
% % -\hat{B}^2 & -I+(I-\tilde{X}^1-\tilde{X}^2)B^2-\hat{B}^2
% %  \end{bmatrix}, \end{align}
% %  \normalsize
% % while $\hat{J} =  \scriptsize \begin{bmatrix}-\hat{B}^1 \\ -\hat{B}^2 \end{bmatrix}$.
% % \par Note that the matrix $ J(\tilde{x}^1, \tilde{x}^2, \textbf{0})$ is block upper triangular. Hence, it is clear that $s(J(\tilde{x}^1, \tilde{x}^2, \textbf{0})) \leq 0$ if, and only if, i) $s(\bar{J}(\tilde{x}^1, \tilde{x}^2, \textbf{0}) \leq 0$, and ii) $s(-I+(I-\tilde{X}^1-\tilde{X}^2)B^3) \leq 0$. Since, by assumption $s(-I+(I-\tilde{X}^1-\tilde{X}^2)B^3)<0$, it follows that  $s(J(\tilde{x}^1, \tilde{x}^2, \textbf{0})) \leq 0$ if and only if $s(\bar{J}(\tilde{x}^1, \tilde{x}^2, \textbf{0}) \leq 0$. Consider the matrix $\bar{J}(\tilde{x}^1, \tilde{x}^2, \textbf{0})$. Define $P: = \begin{bmatrix} I_n && \textbf{0} \\ \textbf{0} && -I_n \end{bmatrix}$. Therefore, 
% % \begin{align}
% %     P\bar{J}(\tilde{x}^1, \tilde{x}^2, \textbf{0})P
% %     = & \scriptsize
% % \begin{bmatrix} 
% % -I+(I-\tilde{X}^1-\tilde{X}^2)B^1-\hat{B}^1 & \hat{B}^1   \\
% % \hat{B}^2 & -I+(I-\tilde{X}^1-\tilde{X}^2)B^2-\hat{B}^2
% % \end{bmatrix}.
% % \end{align}
% % Note that $P\bar{J}(\tilde{x}^1, \tilde{x}^2, \textbf{0})P$ is irreducible Metzler, and that, by assumption, $\tilde{x}^1=\tilde{x}^2$. Hence, by considering the element-wise positive vector $z:=\begin{bmatrix}\tilde{x}^1 &&\tilde{x}^1\end{bmatrix}^\top$, and by invoking the equilibrium version of the  equation of the single-virus system corresponding to virus~1, it follows that $P\bar{J}(\tilde{x}^1, \tilde{x}^1, \textbf{0})Pz=\textbf{0}$. Hence, from Lemma~\ref{lem:perron_frob_metz}, it follows that $s(P\bar{J}(\tilde{x}^1, \tilde{x}^1, \textbf{0})P)=0$, which implies $s(\bar{J}(\tilde{x}^1, \tilde{x}^1, \textbf{0}))=0$. Consequently, $s(J(\tilde{x}^1, \tilde{x}^1, \textbf{0}))=0$.
% % Furthermore, since $s(\bar{J}(\tilde{x}^1, \tilde{x}^1, \textbf{0}))=0$, it must be also true that the matrix $J(\tilde{x}^1, \tilde{x}^1, \textbf{0})$ has exactly one eigenvalue at the origin, and all other eigenvalues have negative real parts. Therefore, the bivirus equations associated with the line of equilibria define a one-dimensional center manifold along which the Jacobian is singular. Therefore, from \cite[Theorem~8.2]{khalil2002nonlinear} it follows that the line of equilibria is locally attractive.~$\blacksquare$
% % }

Throughout this section, we will assume that $D^k=I$ for $k=1,2,3$ on the one hand, recall, from the discussion in Section~\ref{sec:tri:virus:loss:of:generality}, that whether one can make such an assumption without loss of generality is an open question. On the other hand, said assumption makes it easier to describe the phenomenon as we will see in the sequel.
\par Let $z$ denote the single-virus endemic equilibrium corresponding to virus~1, with $Z=\diag(z)$. Therefore, %assuming 
since, by assumption, $D^1=I$, the vector $z$ fulfils the following:
\begin{equation}\label{eq:z}
    (-I+(I-Z)B^1)z=\textbf{0}.
\end{equation}
Furthermore, since $z$ is an endemic equilibrium, from \cite[Lemma~6]{axel2020TAC} %Lemma~\ref{lem:equi_non-zero_nonone}
it follows that $\textbf{0} \ll z \ll \textbf{1}$. Let $C$ be any nonnegative irreducible matrix for which $z$ is also an eigenvector corresponding to eigenvalue one. That is $Cz=z$. Therefore, from %\cite[Chapter 8.3]{meyer2000matrix},
\cite[Theorem~2.7]{varga1999matrix}, %Lemma~\ref{lem:perron_frob}, 
it follows that $\rho(C)=1$, and that the vector $z$, up to a scaling, is the unique eigenvector of $C$ with all entries being strictly positive. Define
\begin{equation}\label{eq:B2}
    B^2:=(I-Z)^{-1}C.
\end{equation}
%and 
% \begin{equation}\label{eq:B3}
%     B^3:=(I-Z)^{-1}C
% \end{equation}
We have the following result.
\begin{thm}\label{thm:init:condns}
Consider system~\eqref{eq:x1}-\eqref{eq:x3} under Assumption~\ref{assum:base}. Suppose that $D^k=I$ for $k \in [3]$.
    Suppose that 
    $B^1$ and $B^3$ are arbitrary nonnegative irreducible matrices, and the
    vector $z$ and matrix $B^2$ are as defined in~\eqref{eq:z} and~\eqref{eq:B2}, respectively. Then, a set of equilibrium points of the trivirus equations is given by $(\beta_1z, (1-\beta_1)z, \textbf{0})$ for all $\beta_1 \in [0,1]$. Furthermore, \begin{enumerate} [label=\roman*)]
    \item if $s(-I+(I-Z)B^3)<0$, then 
the equilibrium set $(\beta_1z, (1-\beta_1)z, \textbf{0})$, with $\beta_1 \in [0,1]$, is locally exponentially attractive.
    \item if $s(-I+(I-Z)B^3)>0$, then 
the equilibrium set $(\beta_1z, (1-\beta_1)z, \textbf{0})$, with $\beta_1 \in [0,1]$, is unstable.
\end{enumerate}
\end{thm}

\textit{Proof:} We first show that, for all $\beta_1 \in [0,1]$, the  point $(\beta_1z, (1-\beta_1)z, \textbf{0})$ fulfils the equilibrium version of equations~\eqref{eq:x1}-\eqref{eq:x3}. To this end, observe that the right hand side of~\eqref{eq:x1}-\eqref{eq:x3} evaluated at  $(\beta_1z, (1-\beta_1)z, \textbf{0})$ yields:
\begin{align}
    &(-I+(I-\beta_1Z-(1-\beta_1)Z)B^1)\beta_1z \nonumber \\
    &=(-I+(I-Z)B^1)\beta_1z =0, \label{beta1z=0}
\end{align}
where~\eqref{beta1z=0} follows by noting that $\beta_1$ is a scalar, and  $z$ is the single-virus endemic equilibrium corresponding to virus~1. Similarly, 
\begin{align}
    &(-I+(I-\beta_1Z-(1-\beta_1)Z)B^2)(1-\beta_1)z \nonumber \\
    &=(-I+(I-Z)B^2)(1-\beta_1)z \nonumber \\
     &=(-I+(I-Z)(I-Z)^{-1}C)(1-\beta_1)z \nonumber \\
     &=(-I+C)(1-\beta_1)z=0, \label{beta2z=0}
\end{align}
where~\eqref{beta2z=0} follows by noting that $Cz=z$. Thus, from~\eqref{beta1z=0} and~\eqref{beta2z=0}, it is clear that, for every $\beta_1 \in [0,1]$,  $(\beta_1z, (1-\beta_1)z, \textbf{0})$ is an equilibrium point of system~\eqref{eq:x1}-\eqref{eq:x3}; i.e. there is a set of equilibrium points $(\beta_1z, (1-\beta_1)z, \textbf{0})$ with $\beta_1 \in [0,1]$. 
\par Next, observe that, for any $\beta_1 \in [0,1]$, the Jacobian evaluated at $(\beta_1z, (1-\beta_1)z, \textbf{0})$ is as given in~\eqref{jacobian:thm2}.
%\begin{figure*}[ht]
	{\noindent}
\begin{align} \label{jacobian:thm2}
&J(\beta_1z, (1{-}\beta_1)z, \textbf{0}) =
%J(\beta_1z, (1-\beta_1)z, \textbf{0}) =&\scriptsize
\scriptsize\begin{bmatrix}
-I+(I-Z)B^1-\diag(B^1\beta_1z) & -\diag(B^1\beta_1z)   & -\diag(B^1\beta_1z)  \\
-\diag(B^2(1-\beta_1)z) & -I+(I-Z)B^2-\diag(B^2(1-\beta_1)z)  & -\diag(B^2(1-\beta_1)z)\\
 \textbf{0}&  \textbf{0}& -I+(I-Z)B^3 \end{bmatrix}\normalsize
\end{align}
%\end{figure*}
Hence, we can rewrite $J(\beta_1z, (1-\beta_1)z, \textbf{0})$ as %\scriptsize
\begin{align}\label{J:forhatx1:partitioned:1}
    J(\beta_1z, (1-\beta_1)z, \textbf{0})
    =
    & \scriptsize
    \begin{bmatrix} 
    \bar{J}(\beta_1z, (1{-}\beta_1)z, \textbf{0}) 
      %\diag(B^1\hat{x}^1)
      && \hat{J} \\
      \mathbf{0} && {-}I{+}(I{-}Z)B^3
    \end{bmatrix}, 
\end{align}
\normalsize
where
\begin{align}\label{jacob:hatx1:22submatrix:1} 
&\bar{J}(\beta_1z, (1-\beta_1)z, \textbf{0})  \nonumber\\ 
& = \scriptsize
\begin{bmatrix} 
{-}I{+}(I{-}Z)B^1{-}\diag(B^1\beta_1z) & -\diag(B^1\beta_1z)   \\
-\diag(B^2(1-\beta_1)z)& {-}I{+}(I{-}Z)B^2{-}\diag(B^2(1{-}\beta_1)z)
 \end{bmatrix}, \end{align}
 \normalsize
while $\hat{J} =  \scriptsize \begin{bmatrix}-\diag(B^1\beta_1z) \\ -\diag(B^2(1-\beta_1)z)\end{bmatrix}$.
\par Note that the matrix $J(\beta_1z, (1-\beta_1)z, \textbf{0})$ is block upper triangular. Hence, it is clear that $s(J(\beta_1z, (1-\beta_1)z, \textbf{0})) =\max\{s(\bar{J}(\beta_1z, (1-\beta_1)z, \textbf{0}), s(-I+(I-Z)B^3)\}$. \\

\textit{Proof of statement~i):} Consider the matrix  $\bar{J}(\beta_1z, (1-\beta_1)z, \textbf{0})$.  Define $P: = \begin{bmatrix} I_n && \textbf{0} \\ \textbf{0} && -I_n \end{bmatrix}$. Therefore, 
\begin{align}
    &P\bar{J}((\beta_1z, (1-\beta_1)z, \textbf{0})P \nonumber \\
    & = \scriptsize
\begin{bmatrix} 
{-}I{+}(I{-}Z)B^1{-}\diag(B^1\beta_1z) & \diag(B^1\beta_1z)   \\
\diag(B^2(1-\beta_1)z) & {-}I{+}(I{-}Z)B^2{-}\diag(B^2(1{-}\beta_1)z)
\end{bmatrix}.
\end{align}

Note that $P\bar{J}((\beta_1z, (1-\beta_1)z, \textbf{0})P$ is an irreducible Metzler matrix.  Hence, by considering the element-wise positive vector $\begin{bmatrix}z^\top &&z^\top\end{bmatrix}^\top$, and by invoking the equilibrium version of the  equation of the single-virus system corresponding to virus~1, it follows that $P\bar{J}((\beta_1z, (1-\beta_1)z, \textbf{0})Pz=\textbf{0}$. Therefore, from \cite[Lemma~2.3]{varga1999matrix},
%Lemma~\ref{lem:perron_frob_metz}, 
it follows that $s(P\bar{J}(\beta_1z, (1-\beta_1)z, \textbf{0})P)=0$, which implies $s(\bar{J}(\beta_1z, (1-\beta_1)z, \textbf{0})=0$. Consequently, since, by assumption, $s(-I+(I-Z)B^3)<0$, we obtain
$s(J(\beta_1z, (1-\beta_1)z, \textbf{0}))=0$.
Furthermore, since $s(P\bar{J}(\beta_1z, (1-\beta_1)z, \textbf{0})P)=0$, then since \break  $P\bar{J}(\beta_1z, (1-\beta_1)z, \textbf{0})P$ is  an irreducible Metzler matrix, %by Lemma~\ref{lem:perron_frob}, 
by \cite[Theorem~2.7]{varga1999matrix}
we have that the matrix $\bar{J}(\beta_1z, (1-\beta_1)z, \textbf{0})$  has exactly one eigenvalue at the origin, and all other eigenvalues have negative real parts. Therefore, since $s(-I+(I-Z)B^3)<0$, the matrix $J(\beta_1z, (1-\beta_1)z, \textbf{0})$ has exactly one eigenvalue at the origin, and all other eigenvalues have negative real parts. Hence,  the trivirus equations associated with the line of equilibria define a one-dimensional center manifold along which the Jacobian is singular. Therefore, from \cite{khalil2002nonlinear} it follows that the set of equilibrium points $(\beta_1z, (1-\beta_1)z, \textbf{0})$, with $\beta_1 \in [0,1]$, is locally exponentially attractive.

\textit{Proof of statement~ii):} By assumption, $s(-I+(I-Z)B^3)>0$. Hence, \break  $s(J(\beta_1z, (1-\beta_1)z, \textbf{0}))>0$. Therefore, from \cite[Theorem~4.7, statement ii)]{khalil2002nonlinear} it follows that the set of equilibrium points $(\beta_1z, (1-\beta_1)z, \textbf{0})$, with $\beta_1 \in [0,1]$, is unstable.\hfill \proofbox%~$\blacksquare$

%\textit{Proof of statement 2.:} 
\vspace{2mm}
Observe that the key idea behind Theorem~\ref{thm:init:condns} is to fix the matrix $B^1$, and then choose $B^2$ (as given in~\eqref{eq:B2}) so as to obtain a locally exponentially attractive (resp. unstable) equilibrium set, namely $(\beta_1z, (1-\beta_1)z, \textbf{0})$, with $\beta_1 \in [0,1]$.
%Clearly, 
It is straightforward to note that,  using the other two possible pairs, namely $(B^1, B^3)$ and $(B^2, B^3)$, the same idea can be applied so as to secure corresponding locally exponentially attractive (or, in case of condition~ii) in Theorem~\ref{thm:init:condns} being, up to a suitable adjustment of notation, fulfilled, unstable)
% (resp. unstable) 
equilibrium sets.
% Supposing that $B^1$, $B^2$ are arbitrary nonnegative irreducible matrices, and 
% \begin{equation}\label{eq:B3}
%     B^3:=(I-Z)^{-1}C,
% \end{equation}
% where $C$ is chosen as discussed previously.
% %$B^3$ is as defined in~\eqref{eq:B3}. 
% Then, by similar arguments as in the first part of the proof of Theorem~\ref{thm:init:condns}, it can be shown that, for every $\beta_2 \in [0,1]$, the point
% $(\beta_2z,  \textbf{0},(1-\beta_2)z)$ is an equilibrium point of~\eqref{eq:x1}-\eqref{eq:x3}. After performing elementary row and column operations on the Jacobian evaluated at $(\beta_2z,  \textbf{0},(1-\beta_2)z)$, i.e., $J((\beta_2z,  \textbf{0},(1-\beta_2)z))$, the conditions for attractivity (resp. instability) of this set of equilibrium points  are analogous to that provided statement~i) (resp.~ii)). Let $z_1$ denote the single-virus endemic equilibrium corresponding to virus~2. Then, by fixing matrix $B^2$ and then choosing $B^3$  in a manner analogous to that given in~\eqref{eq:B3} (but matrix $C$ fulfilling $Cz_1=z_1$), we can obtain  a locally exponentially attractive (resp. unstable) equilibrium set, namely  $(\textbf{0}, \beta_3z_1, (1-\beta_3)z_1)$, with $\beta_3 \in [0,1]$.% where $z_1$ denotes the single-virus endemic equilibrium corresponding to virus~2. 



% Observe that the key idea behind Theorem~\ref{thm:init:condns} is to fix the matrix $B^1$, and then choose $B^2$ (as given in~\eqref{eq:B2}) so as to obtain a locally exponentially attractive (resp. unstable) equilibrium set, namely $(\beta_1z, (1-\beta_1)z, \textbf{0})$, with $\beta_1 \in [0,1]$. Note that by fixing the matrix $B^1$ (resp. $B^2$) and then choosing $B^3$ (resp. $B^3$) in a manner analogous to that given in~\eqref{eq:B2}, we can obtain  a locally exponentially attractive (resp. unstable) equilibrium set, namely $(\beta_2z,  \textbf{0}, (1-\beta_2)z)$, with $\beta_2 \in [0,1]$ (resp. $(\textbf{0}, \beta_3z_1, (1-\beta_3)z_1)$, with $\beta_3 \in [0,1]$, where $z_1$ denotes the single-virus endemic equilibrium corresponding to virus~2. ).

The %findings of 
result in Theorem~\ref{thm:init:condns} does not contradict the claim for finiteness of equilibria presented in Proposition~\ref{prop:finite:equilibria}.  %Section~\ref{sec:trivirus:monotone}. 
Note that the entries in matrices $D^i$, $B^i$, $i=1,2,3$, are either a priori fixed to a specific value or they are not. In  the case where those are not fixed to a specific value, we refer to those %elements 
as free parameters, in the sense that these are allowed to take any value in $\mathbb{R}_{+}$. The dimension of the space of free parameters equals the number of free parameters in the tri-virus system.
Each  choice of free parameters %yields 
results in a realization of system~\eqref{eq:x1}-\eqref{eq:x3}. The set of choices of free parameters that fall within the special case identified by Theorem~\ref{thm:init:condns} has measure zero.

%The following remark  is in order.
\begin{rem}\label{rem:zero:init:condns}
Theorem~\ref{thm:init:condns} admits a non-zero %spread 
initial infection level in the network for virus~3.  
If one were to assume that 
%there is no spread for virus~3, i.e., 
$x_i^3(0) =0$ for all $i \in [n]$, then the trivirus system of equations (i.e., \eqref{eq:x1}-\eqref{eq:x3})  collapses into the bivirus equation set (i.e., equation~\eqref{eq:full} where $m=2$). Under such a setting, Theorem~\ref{thm:init:condns} coincides with \cite[Proposition~3.9]{ye2021convergence}, which, in turn, subsumes \cite[Theorems~6 and~7]{liu2019analysis}; both with respect to i) admitting a larger class of parameters than \cite[Theorems~6 and~7]{liu2019analysis} (and, assuming the setup in \cite{pare2021multi} is restricted to the bi-virus case,  \cite[Corollaries~2 and~3]{pare2021multi}), and ii) providing guarantees for local exponential attractivity.%~\qed
\end{rem}
% \seb{Note that Theorem~\ref{thm:init:condns} is reliant on the assumption that the healing rates for all agents with respect to all viruses are unity. For system~\eqref{eq:full} with $m=3$, by extending the arguments from \cite[Lemma~3.7]{ye2021convergence},  it is straightforward to show that the \emph{location} of the equilibria is the same when $D^k=I$ for $k\in [3]$ and when $D^k$ are arbitrary positive diagonal matrices with the $D^k$s not necessarily being equal to each other.}
% \seb{
% \begin{rem}[Possible loss of generality in assuming $D^k=I$ for $k=1,2,3$] 
% % Note that Theorem~\ref{thm:init:condns} is reliant on the assumption that the healing rates for all agents with respect to all viruses are unity. For system~\eqref{eq:full} with $m=3$, by extending the arguments from \cite[Lemma~3.7]{ye2021convergence},  it is straightforward to show that the \emph{location} of the equilibria is the same when $D^k=I$ for $k\in [3]$ and when $D^k$ are arbitrary positive diagonal matrices with the $D^k$s not necessarily being equal to each other. 
% As discussed previously, there is no loss of generality in setting $D^k=I$ for $k=1,2,3$ insofar the location of equilibria is concerned.
% However, unlike the case when $m=2$, it is not known if the attractivity properties of the equilibrium set that has been established in Theorem~\ref{thm:init:condns} will hold even if the assumption $D^k=I$ for $k=1,2,3$ were to be relaxed.~\qed
% \end{rem}
% }


\section{Plane of coexistence equilibria for nongeneric trivirus networks}\label{sec:global:plane}
The focus of Section~\ref{sec:line:attractivity} was on the existence and attractivity (resp. instability) of a line of coexistence equilibria. In this section, we identify two  scenarios that permit the existence of a plane of  coexistence equilibria  (more strictly, the equilibria of interest are defined by the intersection of a plane with the positive orthant),  and, in the case of one of those scenarios, 
we provide conditions for global convergence to the associated plane of  coexistence equilibria.
 %furthermore, seek condition(s) that guarantee local (resp. global) stability of such a plane.} 

% Section~\ref{sec:line:attractivity} dealt with the existence and attractivity (resp. instability) of a line of coexistence equilibria. Moving beyond this, it is of natural interest to %consider
% identify scenario(s) where a plane of coexistence equilibria could exist, and furthermore, seek condition(s) that guarantee local (resp. global) stability of such a plane of coexistence equilbria. The present section deals with this issue.


We consider a case where three identical copies of a virus are spreading over the same graph as formalized next.
%{\color{red} I understand there is interest in two things: getting a second example of a plane of equilibria (more accurately, a set of equilibria obtained by intersecting a plane with the positive orthant), analogously to what Liu et al did in AC Trans, and encompassing the example already here, and maybe a second example, in a generalization of the idea for constructing a line of equilibria to become a construction for a plane of equilibria.  For this purpose,  I would look at adding to the definition of $C$ and (36) the following: let $\hat C \neq C$ be a second nonnegative irreducible matrix fo which $z$ is also an iegenvector with eigenvalue 1. Take $B^3=(I-Z)^{-1}\hat C$. As potential plane of equilibria, look at $(\alpha_1z,\alpha_2z,\alpha_3z)$ where $\alpha_i$ sum to 1 and are nonnegative. For this to give an interesting result, one may need $n\geq 3$ or even $n>4$. I suspect that $(I-Z)B^1,C$ and $\hat C$ need to be distinct or one will get a known special case. }

\begin{assm}\label{assm:hetero:samegraph}
We suppose that
\begin{enumerate}[label=(\roman*)]
    \item All three viruses are spreading over the same graph.
    \item For all $i \in [n]$ $\delta_i^1 =\delta_i^2 =\delta_i^3>0$.
    \item For all $i=j \in [n]$ and $(i,j) \in \mathcal E$, $\beta_{ij}^1=\beta_{ij}^2=\beta_{ij}^3$.
\end{enumerate}
\end{assm}

Note that for the  special case identified in Assumption~\ref{assm:hetero:samegraph}, assuming that the setting in \cite{pare2021multi} is restricted to the tri-virus case, the existence of a plane of coexistence equilibrium has been secured by \cite[Corollary~3]{pare2021multi}. However, \cite[Corollary~3]{pare2021multi} does not provide guarantees for even local (let alone global)  convergence to the said plane. To address this shortcoming, first, consider the system%The following theorem addresses this drawback.

%First, consider the system 
\begin{equation}\label{eq:positive_linear_x}
\dot x(t) = (-D+(I-\diag(\tilde x))B)x(t),
\end{equation}
where $\tilde x$ is the unique endemic equilibrium of the single virus SIS system associated with $(D, B)$, and with $B$ irreducible. The matrix $Q: = D-(I-\diag(\tilde x))B$ is a singular irreducible $M$-matrix, with a simple eigenvalue at $0$, and all other eigenvalues have a positive real part. Associated with this simple zero eigenvalue is the right eigenvector $\tilde x \gg {\bf 0}$ and a left eigenvector $\tilde u^\top \gg {\bf 0}^\top$. We assume that $\tilde u^\top$ is normalized to satisfy $\tilde u^\top \tilde x = 1$. Moreover, there exists a positive diagonal matrix $P$ such that $\bar Q := PQ+Q^\top P \succeq 0$. In fact, if we choose $P = \diag(\tilde u_1/\tilde x_1, \hdots, \tilde u_n/\tilde x_n)$, then it can be verified that $\bar Q$ is an irreducible $M$-matrix of rank $n-1$, with the nullvector $\tilde x$ associated with the simple eigenvalue at $0$, see~\cite[Section 4.3.4]{qu2009cooperative}.
% Hence, the Jordan canonical form of $A$, denoted by $J$, can be expressed as
% \begin{equation}
%     P^{-1}AP = J = \begin{bmatrix}
%     0 & \\ & \bar J
%     \end{bmatrix},
% \end{equation}
% where $\bar J$ contains the Jordan blocks associated with eigenvalues that have positive real part, for some matrix $P = [\tilde x \vdots \tilde P]$. Now define $y = P^{-1}x$, and $\bar y = [y_2, y_3, \hdots, y_n]^\top$. That is, $\bar y$ is simply the vector $y$ with its first entry removed. Observe that $\dot y = -P^{-1}APy = Jy$, from which it follows that
% \begin{equation}\label{eq:bar_y}
%     \dot{\bar y} = -\bar J \bar y.
% \end{equation}
% Standard linear systems theory yields that \eqref{eq:bar_y} is exponentially stable at the origin $\bar y = {\bf 0}$. From $x = Py$, it then follows that the system \eqref{eq:positive_linear_x} will obey $\lim_{t\to\infty} x(t) = \alpha \tilde x$ for some $\alpha \in \mathbb R$. Moreover, since \eqref{eq:positive_linear_x} is a positive linear system, it follows that $x(0) > {\bf 0}$ implies $\alpha > 0$. In fact, $\alpha = \tilde u^\top x(0)$, where $\tilde u^\top$ is the positive left eigenvector of $A$ associated with the eigenvalue at the origin, normalized to satisfy $\tilde u^\top \tilde x = 1$.

%\ben{
\begin{thm}\label{thm:global:plane}
Consider system~\eqref{eq:x1}-\eqref{eq:x3} under Assumptions~\ref{assum:base}, \ref{assum:irreducible} and~\ref{assm:hetero:samegraph}. Further, suppose that $\rho(D^{-1}B)> 1$. 
Then
%Consider system~\eqref{eq:x1}-\eqref{eq:x3} under Assumption~\ref{assum:base}. Suppose that $D^1 = D^2 = D^3 = D$ and $B^1 = B^2 = B^3=B$, with $\rho(D^{-1}B)> 1$. Then %the following statements hold true:
\begin{enumerate}[label=\roman*)]
    \item For all initial conditions satisfying $x^1(0) >  {\bf 0}_n$, $x^2(0) > {\bf 0}_n$, and $x^3(0) > {\bf 0}_n$, we have that $\lim_{t\to\infty} (x^1(t),x^2(t),x^3(t)) \in \mathcal{E}$  with exponentially fast convergence rate, where $$\mathcal {E} = \{(x^1, x^2, x^3) | \alpha_1 x^1+\alpha_2 x^2+\alpha_3 x^3 = \tilde x, \textstyle\sum_{i=1}^3 \alpha_i = 1\},$$ and $\tilde x$ is the unique endemic equilibrium of the single virus SIS dynamics defined by $(D,B)$.
    \item Every point on the connected set $\mathcal{E}$ is a coexistence equilibrium.
\end{enumerate}
\end{thm}
\textit{Proof:}
\textit{Proof of statement i)}: 
%Note that, from Assumption~\ref{assm:hetero:samegraph}, we have $B^1=B^2=B^3=B$, and $D^1=D^2=D^3=D$. %We prove the first item. 
The initial conditions guarantee that, at some finite time $t$, we have $x^1(t) \gg  {\bf 0}_n$, $x^2(t) \gg {\bf 0}_n$, and $x^3(t)\gg {\bf 0}_n$. Define $z = x^1 + x^2 + x^3$, and $Z=X^1+X^2+X^3$.
%$Z = \diag(x^1) + \diag(x^2) + \diag(x^3)$. 
%Observe that
Therefore, together with Assumption~\ref{assm:hetero:samegraph}, it follows that
\begin{align}
    \dot{z} & = \Big[-D+\big(I-(X^1(t)+X^2(t)+X^3(t))\big)B\Big] \times \nonumber \\
    &~~~~~~~~~~~~~~~~~~~~~~~~~(x^1(t)+x^2(t)+x^3(t)) \\
    & = \big[-D+(I-Z(t))B\big]z(t).\label{eq:single_virus}
\end{align}
Since $z(t) \gg {\bf 0}_n$, and $\rho(D^{-1}B) > 1$ by hypothesis, it follows from \cite[Theorem~2]{khanafer2016stability}
%\cite[Proposition~3]{liu2019analysis} 
%single virus SIS results
that $\lim_{t\to\infty} z(t) = \tilde x$ exponentially fast. In fact, $\tilde x$ is the exponentially stable equilibrium of \eqref{eq:single_virus}, with domain of attraction $z(0) \in [0,1]^n\setminus {\bf 0}$.  It follows that, for all $z(0) \in [0,1]^n\setminus {\bf 0}$, $\Vert \tilde x - z(t) \Vert \leq ae^{-bt}$ for some positive constants $a, b$, with $\Vert \cdot \Vert$ being the Euclidean norm.

The dynamics for virus~$i$, where $i\in [3]$, can be written as $$\dot{x}^i(t) = -[D+(I-\diag(\tilde x))B]x^i(t) + (\diag(\tilde x) - Z(t))Bx^i(t).$$
Without loss of generality, we consider virus~$1$ and drop the superscript. That is, we study the system
\begin{equation}\label{eq:virus_system_plane}
    \dot{x}(t) = -Qx(t) + (\diag(\tilde x) - Z(t))Bx(t),
\end{equation}
where $Z(t)$ is treated as an external time-varying input, and $Q = D-(I-\diag(\tilde x))B$ is an irreducible singular $M$-matrix, as detailed below \eqref{eq:positive_linear_x}. Define the oblique projection matrix $R = I - \tilde x \tilde u^\top$. Define also $\zeta = Rx$, and note that $\tilde u^\top \zeta = 0$ and $\zeta = {\bf 0} \Leftrightarrow x = \alpha \tilde x$ for some $\alpha \in \mathbb R$. That is, $\zeta$ is always orthogonal to $\tilde u$, and $\zeta$ is the zero vector precisely when $x$ is in the span of $\tilde x$.

Observe that $\dot \zeta(t) = R\dot{x}(t)$. Substituting in the right of \eqref{eq:virus_system_plane} for $\dot{x}(t)$, we obtain
\begin{equation}\label{eq:virus_error_system}
    \dot{\zeta}(t) = -Q\zeta(t) + R(\diag(\tilde x)-Z(t))Bx(t),
\end{equation}
by exploiting the fact that $QR = Q = RQ$. %We shall study the dynamics of the non-autonomous system \eqref{eq:virus_error_system}, under time-varying inputs $Z(t)$ and $x(t)$, with $x(t)$ bounded in $[0,1]^n$. %and $\diag(\tilde x)-Z(t) \to {\bf 0}_{n\times n}$ exponentially fast.

Consider the Lyapunov-like function
\begin{equation}\label{eq:V}
    V = \zeta(t)^\top P \zeta(t),
\end{equation}
with $P$ defined below \eqref{eq:positive_linear_x}. It is positive definite in $\zeta$. Differentiating $V$ with respect to time %along the trajectories of \eqref{eq:virus_error_system}
yields  \footnotesize
\begin{align}\label{eq:V_dot}
    \dot{V} & =  -2\zeta(t)^\top PQ\zeta(t) +2\zeta(t)^\top PR(\diag(\tilde x)-Z(t))Bx(t) \nonumber \\
    & = -\zeta(t)^\top \bar Q \zeta(t) +2\zeta(t)^\top PR(\diag(\tilde x)-Z(t))Bx(t).
\end{align}
\normalsize 
%with $\bar Q$ defined below \eqref{eq:positive_linear_x}. 
Due to submultiplicativity of matrix norms, $\Vert \zeta(t)\Vert \leq \Vert R \Vert \Vert x(t)\Vert$. Hence, since $\Vert x(t)\Vert$ is bounded, \eqref{eq:V_dot} yields  \footnotesize
%Since $\Vert \zeta(t)\Vert \leq \Vert R \Vert \Vert x(t)\Vert$, and $\Vert x(t)\Vert$ is bounded, \eqref{eq:V_dot} yields
\begin{align}
    \dot{V} & \leq -\zeta(t)^\top \bar Q \zeta(t) + \kappa \Vert \tilde x - z(t) \Vert %\nonumber \\
    %&
    \leq -\zeta(t)^\top \bar Q \zeta(t) + \bar a e^{-b t}, \label{eq:V_dot_ineq_1}
\end{align}
\normalsize 
where $\kappa$ and $\bar a$ are positive constants, and the second inequality is since $\Vert \tilde x - z(t) \Vert \leq ae^{-bt}$.

%For the positive semidefinite symmetric matrix $\bar Q$, 
Let $\lambda_2$ denote the %its
smallest strictly positive eigenvalue of $\bar Q$. The Courant-Fischer min-max theorem~\cite[Theorem~8.9]{zhang2011matrix} yields %that
\begin{equation}\label{eq:cf_ineq}
    \frac{\zeta^\top \bar Q \zeta}{\zeta^\top \zeta} \geq \min_{\substack{v^\top \tilde u =  0\\v^\top v = 1}} v^\top \bar Q v = \lambda_2,
\end{equation}
for all $\zeta \neq {\bf 0}$ perpendicular to $\tilde u$. Since $\zeta^\top \tilde u = 0$ holds for all $\zeta \in \mathbb R$ by definition, \eqref{eq:cf_ineq} holds for all $\zeta\neq {\bf 0}$. Next, and recalling the definition of $P$ below \eqref{eq:positive_linear_x}, it follows that
\begin{equation}\label{key:ineq:1:lyap}
    \underline{p}\zeta^\top \zeta \leq \zeta^\top P \zeta \leq \bar p \zeta^\top\zeta,
\end{equation}
for all $\zeta$, where $\underline{p} = \min_{i\in [n]} \tilde u_i/\tilde x_i$ and $\bar p = \max_{i\in[n]} \tilde u_i/\tilde x_i$. 

From~\eqref{eq:V_dot_ineq_1} it follows that $\dot{V} \leq -\lambda_2 \zeta(t)^\top \zeta + \bar a e^{-bt}$, which further implies that $ \dot{V} \leq -\bar \lambda \bar p\zeta(t)^\top \zeta + \bar a e^{-bt}$, where $\bar \lambda = \lambda_2/\bar p$. Consequently, from~\eqref{key:ineq:1:lyap}, it follows that $ \dot{V} \leq -\bar\lambda V + \bar a e^{-bt}$. Thus, and recalling the definition of $\zeta$, it follows that $\lim_{t\to\infty} x(t) = \alpha \tilde x$ exponentially fast, where $\alpha \in (0,1)$ because $x(t) \in (0,1)^n$ for all $t \geq \tau$, for some positive $\tau$.

% It follows that \eqref{eq:V_dot_ineq_1} implies that
% \begin{align}
%     \dot{V} \leq -\lambda_2 \zeta(t)^\top \zeta + \bar a e^{-bt},
% \end{align}
% which further implies that
% \begin{align}
%     \dot{V} \leq -\bar \lambda \bar p\zeta(t)^\top \zeta + \bar a e^{-bt},
% \end{align}
% where $\bar \lambda = \lambda_2/\bar p$. It follows that
% \begin{equation}
%     \dot{V} \leq -\bar\lambda V + \bar a e^{-bt}.
% \end{equation}
% The differential equation $\dot{W} = -\bar\lambda W + \bar ae^{-bt}$ has the solution
% \begin{align}
%     W(t) & = W(0)e^{-\bar\lambda t} + \bar a \int_{0}^t e^{-\bar\lambda(t-s)} e^{-bs} .\text{d}s \\
%     & = W(0)e^{-\bar\lambda t} + \bar a e^{-\bar\lambda t} \int_{0}^t e^{-s(b-\bar\lambda)} .\text{d}s \\
%     & = W(0)e^{-\bar\lambda t} + \frac{\bar a}{b-\bar\lambda} \left(e^{-\bar\lambda t} - e^{-bt}\right).
% \end{align}
% Note that $\frac{\bar a}{b-\bar\lambda}(e^{-\bar\lambda t} - e^{-bt}) \geq 0$ for any positive $b$ and $\bar\lambda$. 
% %This is because $e^{-\bar\lambda t} > e^{-bt}$ for all $t\geq 0$ if and only if $b > \bar\lambda$. 
% Thus, $W(t) \geq 0$ if $W(0) \geq 0$, and $\lim_{t\to\infty} W(t) = 0$ at an exponentially fast rate. By the Comparison Lemma~\cite[Lemma~3.4]{khalil2002nonlinear}, $V(t) \leq W(t)$ and $\lim_{t\to\infty} V(t) = 0$ at an exponentially fast rate. Thus, and recalling the definition of $\zeta$, it follows that $\lim_{t\to\infty} x(t) = \alpha \tilde x$ exponentially fast, where $\alpha \in (0,1)$ because $x(t) \in (0,1)^n$ for all $t \geq \tau$, for some positive $\tau$.}

This analysis holds not only for virus~$1$ but also virus~$2$ and virus~$3$. In other words, $\lim_{t\to\infty} x^i(t) = \alpha_i \tilde x$ for some $\alpha_i \in (0,1)$, for all $i\in [3]$. Recall that $\lim_{t\to\infty} z(t) = \tilde x$, and we immediately conclude that $\sum_{i=1}^n \alpha_i = 1$, thus proving statement i). %The first statement of the theorem is thus proved.

\textit{Proof of statement ii)}: We now prove that every point in $\mathcal{E}$ is an equilibrium (coexistence follows trivially by definition). Consider an arbitrary point $(x^1, x^2, x^3)$ in $\mathcal{E}$. From \eqref{eq:x1}, we have \vspace{-4mm}
\begin{align}
    \dot{x}^1(t) & = (-D+(I-X^1-X^2-X^3)B)x^1 \\
    & = (-D+(I-\diag(\tilde x))B)\alpha_1 \tilde x = {\bf 0}.
\end{align}
By the same arguments, it follows that $\dot{x}^2(t) = {\bf 0}$ and $\dot{x}^3(t) = {\bf 0}$ at the point $(x^1, x^2, x^3)$ in $\mathcal{E}$. In other words, $(x^1, x^2, x^3)$ is an equilibrium of the system system~\eqref{eq:x1}-\eqref{eq:x3}. Since this holds for any arbitrary point in $\mathcal{E}$, the proof of statement~ii) is complete. \hfill \proofbox \\%second claim is proved. %\qed


\vspace{1mm}
We identify yet another special case of the tri-virus model parameters
% but one that subsumes the setting covered in Theorem~\ref{thm:global:plane}, 
that admits the existence of a plane of 3-coexistence equilibrium. To this end, let $B^1$ be some arbitrary nonnegative irreducible matrix with $z$ being as defined in~\eqref{eq:z}, and $B^2$ being as defined in~\eqref{eq:B2}. Let $\hat{C} \neq C$ be a nonnegative irreducible matrix such that $z$ is the eigenvector corresponding to eigenvalue unity, that is, $\hat{C}z=z$. Define \begin{equation}\label{eq:B3}
    B^3:=(I-Z)^{-1}\hat{C}.
\end{equation}
%and 
% \begin{equation}\label{eq:B3}
%     B^3:=(I-Z)^{-1}C
% \end{equation}
We have the following result.

\begin{prop} \label{prop:plane:equilibrium}
%\seb{
Consider system~\eqref{eq:x1}-\eqref{eq:x3} under Assumptions~\ref{assum:base} and~\ref{assum:irreducible}. %Suppose that $D^k=I$ for $k \in [3]$. 
Suppose that 
    $B^1$ is an arbitrary nonnegative irreducible matrix %x
 and the 
    vector $z$ together with matrices $B^2$ and $B^3$ are as defined in~\eqref{eq:z}, and~\eqref{eq:B2} and~\eqref{eq:B3}, respectively. Then, a set of equilibrium points of the trivirus equations is given by $(\alpha_1z, \alpha_2z, \alpha_3z)$,  for all $\alpha_i \geq 0$, %for
 $i=1,2,3$, and  $\sum_{i=1}^3\alpha_i=~1$. 
\end{prop}
\textit{Proof:}
Assume, without loss of generality, that  $D^k=I$ for $k \in [3]$.
Observe, then,  that the right hand side of~\eqref{eq:x1}-\eqref{eq:x3} evaluated at  $(\alpha_1z, \alpha_2z, \alpha_3z)$ yields:
\begin{align}
    &(-I+(I-\alpha_1Z-\alpha_2Z- \alpha_3Z)B^1)\alpha_1z \nonumber \\
    &=(-I+(I-Z)B^1)\alpha_1z =0, \label{alpha1z=0}
\end{align}
where~\eqref{alpha1z=0} follows by noting that by assumption, $\alpha_1$ is a nonnegative scalar, and $\sum_{i=1}^3\alpha_i=1$,  and  $z$ is the single-virus endemic equilibrium corresponding to virus~1. Similarly, 
\begin{align}
    &(-I+(I-\alpha_1Z-\alpha_2Z- \alpha_3Z)B^2)\alpha_2z \nonumber \\
    &=(-I+(I-Z)B^2)\alpha_2z \nonumber \\
     &=(-I+(I-Z)(I-Z)^{-1}C)\alpha_2z \nonumber \\
     &=(-I+C)\alpha_2z=0, \label{alpha2z=0}
\end{align}
where~\eqref{alpha2z=0} follows by noting that $Cz=z$.
Finally, 
\begin{align}
    &(-I+(I-\alpha_1Z-\alpha_2Z- \alpha_3Z)B^3)\alpha_3z \nonumber \\
    &=(-I+(I-Z)B^3)\alpha_3z \nonumber \\
     &=(-I+(I-Z)(I-Z)^{-1}\hat C)\alpha_3z \nonumber \\
     &=(-I+\hat C)\alpha_3z=0, \label{alpha3z=0}
\end{align}
where~\eqref{alpha3z=0} follows by noting that $\hat{C}z=z$.
Thus, from~\eqref{alpha1z=0}, \eqref{alpha2z=0} and~\eqref{alpha3z=0}, it is clear that, for every $\alpha_i\geq 0$ with $i=1,2,3$, such that $\sum_{i=1}^3 \alpha_i=1$,  $(\alpha_1z, \alpha_2z, \alpha_3z)$ is an equilibrium point of system~\eqref{eq:x1}-\eqref{eq:x3}.\hfill \proofbox %i.e. there is a set of equilibrium points $(\beta_1z, (1-\beta_1)z, \textbf{0})$ with $\beta_1 \in [0,1]$.
\vspace{2mm}
\par Note that Proposition~\ref{prop:plane:equilibrium} and, assuming the setting in \cite{pare2021multi} is limited to the tri-virus case, \cite[Corollaries~2 and ~3]{pare2021multi} identify special cases that admit the existence of a plane of coexistence equilibria. Since  Proposition~\ref{prop:plane:equilibrium}  covers a larger class of parameters than that in \cite[Corollaries~2 and ~3]{pare2021multi}, Proposition~\ref{prop:plane:equilibrium} subsumes \cite[Corollaries~2 and ~3]{pare2021multi}.


Proposition~\ref{prop:plane:equilibrium} and Theorem~\ref{thm:global:plane} compare in the following sense: 
% First observe that, for the class of parameters covered by  Theorem~\ref{thm:global:plane}, the existence of a plane of coexistence equilibria has been addressed in \cite[Corollary~3]{pare2021multi}. However, \cite[Corollary~3]{pare2021multi} provides no stability guarantees for said plane; a gap addressed by Theorem~\ref{thm:global:plane}.
On the one hand, note that if $(I-Z)B^1=C=\hat{C}$, then, in view of~\eqref{eq:B2} and~\eqref{eq:B3}, it follows that $B^1=B^2=B^3$, which coincides with the setting of Theorem~\ref{thm:global:plane}. Thus,   
Proposition~\ref{prop:plane:equilibrium} covers a larger class of parameters than Theorem~\ref{thm:global:plane}, and, therefore, with respect to the existence of a set of 3-coexistence equilibria, is more general.  Conversely,  Proposition~\ref{prop:plane:equilibrium} provides no stability guarantees for the said equilibrium set, whereas Theorem~\ref{thm:global:plane} establishes global stability of the  equilibrium set for the smaller class of parameters that it covers. %As a consequence, neither Proposition~\ref{prop:plane:equilibrium} nor Theorem~\ref{thm:global:plane}  subsume one another.

\par With respect to the existence of a continuum of equilibria, Proposition~\ref{prop:plane:equilibrium} is stronger than Theorem~\ref{thm:init:condns} because it adds a larger class of $i=1,2,3$, $D^i$, $B^i$ matrices. That said, note that Theorem~\ref{thm:init:condns} admits  arbitrary $B^3$, whereas Proposition~\ref{prop:plane:equilibrium} insists on $B^3$ obeying a specific functional form, as given in~\eqref{eq:B3}. In conclusion, neither subsumes the other.

\section{Simulations}\label{sec:simulations}

We now present a set of simulations that highlight the key theoretical results of our paper. We choose $D^i = I$ for $i = 1,2,3$, with the following $B^i$ matrices, where $\hat{\beta}^k_{ij}$ are constants that are changed depending on the simulation example being presented. 
\[   B^1 = \scriptsize \begin{bmatrix}
    0 & 0 & 0 & 1.5\\1.5 & 0 & 0 & 0\\0 & 1.5 & 0 & 0\\0 &0 &1.5& 0
    \end{bmatrix},\\
    B^2 = \begin{bmatrix}
    0 & 1.5+\hat{\beta}^2_{12} & 0 & 0\\0 & 0 & 1.5 & 0\\0 & 0 & 0 & 1.5\\1.5& 0& 0& 0
    \end{bmatrix},
\]
\begin{equation}
    % B^1 &= \begin{bmatrix}
    % 0 & 0 & 0 & 1.5\\1.5 & 0 & 0 & 0\\0 & 1.5 & 0 & 0\\0 &0 &1.5& 0
    % \end{bmatrix},\\
    % B^2 &= \begin{bmatrix}
    % 0 & 1.5+\hat{\beta}^2_{12} & 0 & 0\\0 & 0 & 1.5 & 0\\0 & 0 & 0 & 1.5\\1.5& 0& 0& 0
    % \end{bmatrix},\\ 
    B^3 =  \scriptsize \begin{bmatrix}
    1 & 0 & 0.5+\hat{\beta}^{3}_{13} & 0\\0 & 1+\hat{\beta}^{3}_{22} & 0.5 & 0\\0 & 0.5 & 0 & 1\\0.3+\hat{\beta}^{3}_{31}& 0 &1.2& 0
    \end{bmatrix}.\nonumber
\end{equation}
Except as otherwise stated, initial conditions are obtained according to the following procedure. First, for $i\in [n]$ and $s\in[4]$, we sample a value $p_{i}^s$ from a uniform distribution $(0,1)$. Then, for $i\in [n]$ and $k\in[3]$ we set $x_{i}^k(0) = p_{ij}^k/\sum_{s=1}^4 p_{ij}^s$, which ensures that the initial conditions are in $\mathcal D$ but otherwise randomized. 
% Note the logarithmic scale of the time axis, and each plot shows the mean infection state for each virus, $\frac{1}{n}\sum_{i=1}^n x_i^k(t)$; the blue line for $k = 1$, the red line for $k = 2$, and the black line for $k = 3$.

\textit{Example~1:} We set $\hat{\beta}^3_{13} = 0$,  $\hat{\beta}^2_{12}=-0.1$, $\hat{\beta}^3_{22} = -0.1$, and $\hat{\beta}^{3}_{31} = 0.1$. In this example, and following the notation of Section~\ref{ssec:equilibria_types} and Theorem~\ref{thm:local}, we obtain $\rho((I-\tilde X^1)(D^2)^{-1}B^2) = 0.9829$ and $\rho((I-\tilde X^1)(D^3)^{-1}B^3) = 0.99624$. Thus, $(\tilde x^1, \bf{0}, \bf{0})$ is locally exponentially stable, and Fig.~\ref{fig:tri_virus_stable_virus1} shows convergence to $(\tilde x^1, \bf{0}, \bf{0})$. It can be computed that $\rho((I-\tilde X^2)(D^1)^{-1}B^1) = 1.0174$ and $\rho((I-\tilde X^2)(D^3)^{-1}B^3) = 1.0127$ and similarly, $\rho((I-\tilde X^3)(D^1)^{-1}B^1) = 1.003$ and $\rho((I-\tilde X^3)(D^2)^{-1}B^2) = 0.9863$. Hence, in line with Theorem~2, both $({\bf 0},\tilde x^2, {\bf 0})$ and $({\bf 0}, {\bf 0}, \tilde x^3)$ are unstable.

%Hence, according to Theorem~\ref{thm:local}, $({\bf 0},\tilde x^2, {\bf 0})$ and $({\bf 0}, {\bf 0}, \tilde x^3)$ are both unstable. 
% Additional simulations with randomized initial conditions appear to suggest that $(\tilde x^1, {\bf 0}, {\bf 0})$ is globally attractive for initial conditions in the interior of~$\mathcal{D}$, but these are omitted due to space limitations.

% \textit{Example~2:} We set $\hat{\beta}^3_{13} = 0$,  $\hat{\beta}^2_{12}=-0.1$, $\hat{\beta}^3_{22} = -0.1$, and $\hat{\beta}^{3}_{31} = 0.15$. Consequently, we obtain $\rho((I-\tilde x^1)(D^2)^{-1}B^2) = 0.9829$ and $\rho((I-\tilde x^1)(D^3)^{-1}B^3) = 1.0037$. Indeed, we can compute that  $({\bf 0}, {\bf 0}, \tilde x^3)$ is locally exponentially stable according to Theorem~\ref{thm:local}, while the two other boundary equilibria are unstable; see Figure~\ref{fig:tri_virus_unstable_virus1}. 
% Additional simulations with randomized initial conditions appear to suggest that $({\bf 0}, {\bf 0}, \tilde x^3)$ is globally attractive for initial conditions in the interior of $\mathcal{D}$.

\textit{Example~2:} We set $\hat{\beta}^3_{13} = 0.05$,  $\hat{\beta}^2_{12}=\hat{\beta}^3_{22} = \hat{\beta}^{3}_{31} = 0$. In this example, and following the notation of Theorem~\ref{thm:init:condns}, we have a line of equilibria $(\beta_1 z, (1-\beta_1)z, {\bf 0})$, with $z = \frac{1}{3}\bf{1}$ and $\beta_1\in [0,1]$. Since $\rho((I-Z)B^3) = 1.0043$, in line with statement~ii) in Theorem~\ref{thm:init:condns}, this line of equilibria is unstable; see Figure~\ref{fig:tri_virus_unstable_2line}. We can compute that $({\bf 0}, {\bf 0}, \tilde x^3)$ is locally exponentially stable, following reasoning analogous to that in Theorem~\ref{thm:local}. %due to Theorem~\ref{thm:local}. 
% Additional simulations with randomized initial conditions appear to suggest that $({\bf 0}, {\bf 0}, \tilde x^3)$ is globally attractive for initial conditions in the interior of $\mathcal{D}$.
Note, however, that even though the line of equilibria $(\beta_1 z, (1-\beta_1)z, {\bf 0})$ is unstable, it still has an attractive manifold (of measure zero in $\mathcal{D}$); trajectories on this manifold will converge to the line. As established in \cite[Proposition~3.9]{ye2021convergence}, this manifold includes points $(x^1, x^2, {\bf 0})$, where $x^1$ and $x^2$ are in a small open neighbourhood around $\beta_1 z$ and $(1-\beta_1)z$, respectively. 

\textit{Example~3:} We set $\hat{\beta}^3_{13} = -0.1$,  $\hat{\beta}^2_{12}=\hat{\beta}^3_{22} = \hat{\beta}^{3}_{31} = 0$. Similar to Example~2, we have a line of equilibria $(\beta_1 z, (1-\beta_1)z, {\bf 0})$, with $z = \frac{1}{3}\bf{1}$. Now, however, $\rho((I-Z)B^3) = 0.9911$, and this line of equilibria is locally exponentially attractive according to statement i) in Theorem~\ref{thm:init:condns}; see Figure~\ref{fig:tri_virus_stable_2line}. The boundary equilibrium $({\bf 0}, {\bf 0}, \tilde x^3)$ is unstable, following reasoning analogous to that in Theorem~\ref{thm:local}. 
% Additional simulations with randomized initial conditions appear to suggest that $(\beta_1 z, (1-\beta_1)z, {\bf 0})$ is globally attractive for initial conditions in the interior of $\mathcal{D}$, while the value of $\beta_1$ is dependent on the initial conditions.

\textit{Example 4:} We set $\hat{\beta}^3_{13} = \hat{\beta}^2_{12}=\hat{\beta}^3_{22} = \hat{\beta}^{3}_{31} = 0$. The aforementioned choice 
%set 
of $B^i$ satisfies Proposition~\ref{prop:plane:equilibrium}, and hence there is a plane of equilibria of the form $(\alpha_1 z, \alpha_2 z, \alpha_3 z)$, with $\sum_{i=1}^3 \alpha_i = 1$ and $z = \frac{1}{3}\bf{1}$. Although we do not have a theoretical result for convergence to this plane, Fig.~\ref{fig:tri_virus_stable_3plane_a} and \ref{fig:tri_virus_stable_3plane_b} illustrate convergence to this plane of equilibria from different initial conditions; the particular equilibrium reached depends on the choice of initial conditions. 
% Additional simulations suggest that the plane is in fact globally attractive for all initial conditions in the interior of $\mathcal{D}$. 


We now consider a second set of simulations with $n = 5$ nodes. We again set $D^i = I$ for $i = 1, 2, 3$, but we now study a different set of $B^i$ matrices to demonstrate other relevant theoretical results of the paper. Let \scriptsize
\begin{equation}
    B = \begin{bmatrix}
    1 & 0 & 2 & 0 & 0.5\\0.5 & 2 & 0 & 0 & 0\\ 0 & 5 & 0.1 & 0 & 0 \\ 0.1 & 0 & 0 & 0.2 & 0\\0 & 0 & 0 & 0.1 & 0.9
    \end{bmatrix}.
\end{equation}
\normalsize 
Our $B^i$ matrices will be perturbations of $B$, which we describe in the respective simulation examples. To facilitate this, let $e_i$ be the $i$th basis vector of $\mathbb R^5$, i.e., $e_i$ has $1$ in its $i$th entry, and all other entries are $0$. The initial conditions are selected following the same random procedure described above.

\textit{Example 5:} We set $B^i = B$ for all $i=1,2,3$, i.e., we have three identical copies of a virus spreading over the same graph. According to Theorem~\ref{thm:global:plane}, there is a plane of coexistence equilibria, which in this case is defined by the vector $\tilde x = [0.691,0.610,0.758,0.078,0.051]^\top$. Fig.~\ref{fig:trivirus_stable_copyplane_a} and \ref{fig:trivirus_stable_copyplane_b} show convergence to two different equilibria in this plane of equilibria from two different initial conditions. 

\textit{Example 6:} We set $B^1 = B$, $B^2 = B + 0.5 e_1e_4^\top$, and $B^3 = B^2 + 0.1 e_5 e_1^\top$. %Thus, $B^2$ is equal to $B^1$ except its $(1,4)$th entry is greater by $0.5$, and $B^3$ is equal to $B^2$ its $(5,1)$th entry is greater by $0.1$. 
This tri-virus system satisfies the inequality conditions of Theorem~\ref{claim:virus3:strongest}; it follows that there does not exist a $3$-coexistence equilibrium, while the boundary equilibrium $({\bf 0}, {\bf 0}, \tilde x^3)$ is locally exponentially stable. Fig.~\ref{fig:tri_virus_stable_virus3} shows the tri-virus system converging to the $({\bf 0}, {\bf 0}, \tilde x^3)$ equilibrium. 
% Indeed, additional simulations suggests this equilibrium has a large region of attraction in the interior of $\mathcal{D}$ (potentially comprising all of the interior or $\mathcal{D}$).

\textit{Example 7:}  We set $B^1 = B$, $B^2 = B + 2 e_1e_4^\top$, and $B^3 = B + 0.1 e_5 e_1^\top$. This system satisfies the inequality conditions of Theorem~\ref{claim:virus1weaker} but not those of Theorem~\ref{claim:virus3:strongest}. In Fig.~\ref{fig:tri_virus_stable_2virus}, we observe convergence to a $2$-coexistence equilibrium. Consistent with Theorem~\ref{claim:virus1weaker}, this $2$-coexistence equilibrium is such that only virus $2$ and virus $3$ exist in each node of the network, i.e., it has the form $({\bf 0}, \bar x^2, \bar x^3)$.

\textit{Example 8:}  We set $B^1 = B + 0.7e_3 e_2^\top$, $B^2 = B + 2 e_1e_4^\top$, and $B^3 = B + 0.1 e_5 e_1^\top$. This system satisfies the conditions of Corollary~\ref{cor:hofbauer}. The inequalities concerning the boundary equilibria are easily checked. The inequalities involving the $2$-coexistence equilibria are checked by verifying that there are precisely $3$ such equilibria, one unique equilibrium corresponding to each of virus~$1$, virus~$2$, and virus~$3$ being extinct. This uniqueness can be established by a rapid simulation procedure that exploits the fact that bivirus systems are MDS and involves simulation of just two different initial conditions~\cite[Corollary~3.14]{ye2021convergence}. Based on this knowledge, Corollary~\ref{cor:hofbauer} establishes that there is at least one $3$-coexistence equilibrium, and it is saturated. Indeed, Fig.~\ref{fig:tri_virus_stable_3virus} shows convergence to a $3$-coexistence equilibrium.

\textit{Example 9:} In each of the examples above, except for Example 5, the simulations correspond to theoretical results that establish local exponential convergence (either to an isolated equilibrium or a line of equilibria). Extensive additional simulations with many other randomized initial conditions suggest that, for these examples, convergence to the equilibrium of interest (or line/plane of equilibria) occurs for all initial conditions in the interior of $\mathcal{D}$, thereby suggesting that they (i.e., the equilibrium of interest, or line/plane of equilibria), are globally stable. We conclude here with a final example, which serves as a cautionary note highlighting that tri-virus systems can have multiple attractive equilibria with different regions of attraction with nonzero measure in $\mathcal{D}$. First, define the following matrices:
\begin{align*}
    B_{11} & = \begin{bmatrix} 1.6 & 1\\1 & 1.6 \end{bmatrix}, 
    B_{12} = \begin{bmatrix} 2.1 & 0.156\\3.0659 & 1.1 \end{bmatrix}, \\
    B_{21} & = \begin{bmatrix} 1.7 & 1\\ 1.2 & 0.5 \end{bmatrix}, 
    B_{22} = \begin{bmatrix} 1.6 & 1 \\1.2 & 0 \end{bmatrix}, 
    C = 0.001 \times{\bf 1 1}^\top.
\end{align*}
We consider $n = 4$, and set $D^i = I$ for all $i=1,2,3$, and
\begin{equation*}
    B^1 = \begin{bmatrix} B_{11} & C \\ C & B_{21} \end{bmatrix}, B^2 = \begin{bmatrix} B_{12} & C \\ C & B_{22} \end{bmatrix}, 
    B^3 = \begin{bmatrix} B_{22} & C \\ C & B_{11} \end{bmatrix}.
\end{equation*}
Fig.~\ref{fig:tri_virus_stable_multiequib_virus13} and \ref{fig:tri_virus_stable_multiequib_virus23} show convergence to two different $2$-coexistence equilibria from different initial conditions in the interior of $\mathcal{D}$; we verified that the Jacobian matrix at both these equilibria are Hurwitz, i.e., both equilibria are locally exponentially stable. In Fig.~\ref{fig:tri_virus_stable_multiequib_virus13}, convergence occurs to a $2$-coexistence equilibrium where virus $2$ is extinct, whereas in Fig.~\ref{fig:tri_virus_stable_multiequib_virus23}, the $2$-coexistence equilibrium is such that virus~$1$ is extinct.


\begin{figure*}
\begin{minipage}{0.49\linewidth}
\centering
\subfloat[Example~1]{\includegraphics[width=\columnwidth]{tri_virus_stable_virus1.pdf}\label{fig:tri_virus_stable_virus1}}
\end{minipage}
\hfill
% \begin{minipage}{0.49\linewidth}
% \centering\subfloat[Example~2]{\includegraphics[width=\columnwidth]{tri_virus_unstable_virus1.pdf}\label{fig:tri_virus_unstable_virus1}}
% \end{minipage}
% \vfill
\begin{minipage}{0.49\linewidth}
\centering
\subfloat[Example~2]{\includegraphics[width=\columnwidth]{tri_virus_unstable_2line.pdf}\label{fig:tri_virus_unstable_2line}}
\end{minipage}
\vfill
\begin{minipage}{0.49\linewidth}
\centering\subfloat[Example~3]{\includegraphics[width=\columnwidth]{tri_virus_stable_2line.pdf}\label{fig:tri_virus_stable_2line}}
\end{minipage}
\hfill
\begin{minipage}{0.49\linewidth}
\centering
\subfloat[Example~4, First Initial Condition]{\includegraphics[width=\columnwidth]{tri_virus_stable_3plane_a.pdf}\label{fig:tri_virus_stable_3plane_a}}
\end{minipage}
\vfill
\begin{minipage}{0.49\linewidth}
\centering\subfloat[Example~4, Second Initial Condition]{\includegraphics[width=\columnwidth]{tri_virus_stable_3plane_b.pdf}\label{fig:tri_virus_stable_3plane_b}}
\end{minipage}
\hfill
\begin{minipage}{0.49\linewidth}
\centering
\subfloat[Example~5, First Initial Condition]{\includegraphics[width=\columnwidth]{trivirus_stable_copyplane_a.pdf}\label{fig:trivirus_stable_copyplane_a}}
\end{minipage}
\vfill
\begin{minipage}{0.49\linewidth}
\centering\subfloat[Example~5, Second Initial Condition]{\includegraphics[width=\columnwidth]{trivirus_stable_copyplane_b.pdf}\label{fig:trivirus_stable_copyplane_b}}
\end{minipage}
\hfill
\begin{minipage}{0.49\linewidth}
\centering
\subfloat[Example~6]{\includegraphics[width=\columnwidth]{tri_virus_stable_virus3.pdf}\label{fig:tri_virus_stable_virus3}}
\end{minipage}
\caption{Trajectories of the simulated trivirus system \eqref{eq:full}, for different simulation parameters detailed in Section~\ref{sec:simulations}. }\label{fig:tri_virus_simulations}
\end{figure*}

\begin{figure*}
\begin{minipage}{0.49\linewidth}
\centering\subfloat[Example~7]{\includegraphics[width=\columnwidth]{tri_virus_stable_2virus.pdf}\label{fig:tri_virus_stable_2virus}}
\end{minipage}
\hfill
\begin{minipage}{0.49\linewidth}
\centering
\subfloat[Example~8]{\includegraphics[width=\columnwidth]{tri_virus_stable_3virus.pdf}\label{fig:tri_virus_stable_3virus}}
\end{minipage}
\vfill
\begin{minipage}{0.49\linewidth}
\centering\subfloat[Example~9, First Initial Condition]{\includegraphics[width=\columnwidth]{tri_virus_stable_multiequib_virus13.pdf}\label{fig:tri_virus_stable_multiequib_virus13}}
\end{minipage}
\hfill
\begin{minipage}{0.49\linewidth}
\centering\subfloat[Example~9, Second Initial Condition]{\includegraphics[width=\columnwidth]{tri_virus_stable_multiequib_virus23.pdf}\label{fig:tri_virus_stable_multiequib_virus23}}
\end{minipage}
\caption{Trajectories of the simulated trivirus system \eqref{eq:full}, for different simulation parameters detailed in Section~\ref{sec:simulations}. }\label{fig:tri_virus_simulations_b}
\end{figure*}

\section{Conclusion}\label{sec:conclusions}
% This paper analyzed a competitive tri-virus SIS model. We focused on establishing generic properties on its equilibria and dynamics, such as the lack of monotonicity. We then established a range of stability and existence (and nonexistence) results for the various equilibria of interest. Nongeneric situations were also investigated.
This paper analyzed a competitive tri-virus SIS model. Specifically, we established that the tri-virus system is not monotone, unlike the bi-virus system. Subsequently, using the Parametric Transversality Theorem,  we showed that, generically, the tri-virus system has a finite number of equilibria and that the Jacobian matrices associated with each equilibrium are nonsingular. We then identified a sufficient condition for local exponential convergence to a boundary equilibrium. Thereafter, we identified a sufficient condition for the nonexistence of various kinds of coexistence equilibria, followed by the provision of  a sufficient condition for the existence of a 3-coexistence equilibrium (resp. 2-coexistence equilibrium). We identified  
i) a setting that permits the existence and local exponential attractivity of a line of coexistence equilibria; and ii) two settings that permit the existence of, and, in one case, global convergence to, a plane of coexistence equilibria. All of the aforementioned settings have measure zero.
   
\par There are several directions of future research that could be pursued. Recalling the discussion in Section~\ref{sec:tri:virus:loss:of:generality},  a straightforward problem is establishing whether or not the stability properties of a trivirus system defined by $\{D^i,B^i, i=1,2,3\}$ are the same as those defined by $\{I,(D^i)^{-1}B^i, i=1,2,3\}$.
Second, given that the tri-virus system is not monotone, a comprehensive view of the model that, in particular, focuses on its limiting behavior would be extremely relevant since, to provide the said view, novel tools may be required. On a related note, investigating scenarios that admit the existence (resp. exclude the possibility of existence) of limit cycles would provide further insights. 
Third, while the present paper identifies conditions for the existence of a 3-coexistence equilibrium point, no claims are made regarding its uniqueness or its stability (or lack thereof). Hence, one line of investigation could, perhaps by leveraging Morse-Smale inequalities \cite{smale1967differentiable}, focus on providing bounds on the number of 3-coexistence equilibria that a tri-virus system is permitted. Fourth, assuming that one of the competing viruses is benign as compared to the other two, devising
(feedback) control strategies that can drive the virus dynamics to the boundary equilibrium of the benign virus could provide policymakers with another tool for disease mitigation. 

%\bibliography{ReferencesRice}
\bibliographystyle{siamplain}
\bibliography{references}


\section*{Appendix~A: Proof of Proposition~\ref{prop:finite:equilibria}}
The use below of the Parametric Transversality Theorem, the principal tool in the proof, requires the preliminary calculation of certain Jacobians. We will consider the case where the $B^i$ are fixed, to demonstrate that for almost all allowed $D^i$, equilibria are nondegenerate (i.e. the associated Jacobians are nonsingular). Given the bounded nature of the set of interest, viz. $\{\bf{0}\leq x^k \leq \bf{1}\}\cap \{ x^1+x^2+x^3\leq {\bf{1}} \}$  and continuity of the Jacobian, finiteness of the number of zeros follows straightforwardly from the nondegeneracy conclusion. 

 It is standard that there is a finite number of equilibria (viz. 1 or 2)), when two of the $x^i$ are zero, the equilibria being expressible using those of a single virus network. If precisely one of the $x^i$ is zero, the equilibria can be defined using a bivirus system, and again, the finiteness for generic $D^k$ is known from \cite{ye2021convergence}. It, therefore, remains to consider coexistence equilibria with all $x^k$ nonzero separately, and in this case, as shown in 
Lemma~\ref{lem:equi_non-zero_nonone:1}
we can assume $x^k\gg 0, k=1,2,3$.

Without loss of generality, we will assume there is some fixed positive $\bar d<1$ such that all diagonal entries of the $D^i$ lie in $(\bar d,\bar d^{-1})$. Let $\mathcal D$ denote the manifold defined by the set of allowed $D^i$ consistent with the choice of $\bar d$.

Let $\mathcal X$ denote the manifold $\{\bf{0}\ll x^k \ll {\bf{1}}\}\cap \{x^1+x^2+x^3\ll {\bf{1}}\}$. Consider for any $x=[(x^1)^{\top},(x^2)^{\top},(x^3)^{\top}]^{\top}\in\mathcal X$ and $\delta=\mbox{vec}[D^1,D^2,D^3]\in\mathcal D$ the map
\begin{align*}
f_{\delta}:&\mathcal X\times\mathcal D\to\mathcal Y,\\ 
(x,\delta)&\mapsto y=\begin{bmatrix}[-D^1+(I_n-X^1-X^2-X^3)B^1]x^1\\
[-D^2+(I_n-X^1-X^2-X^3)B^2]x^2\\
[-D^3+(I_n-X^1-X^2-X^3)B^3]x^3
\end{bmatrix}
\end{align*}
% {\color{red} Change 20220906. I set up two lemmas in the following and eliminated extraneous calculations}
We now claim:
\begin{lem}
With notation as above, the matrix $\frac{\partial f_{\delta}(x,\delta)}{\partial (x,\delta)}$ has full row rank at any coexistence equilibrium. 
% \begin{small}
% \begin{align}
%     &\frac{\partial f_{\delta}(x,\delta)}{\partial (x,\delta)}\\\nonumber
%     &=\begin{bmatrix}
%  -D^1+(I-X^1-X^2)B^1-{\rm{diag}}(B^1x^1)&-{\rm{diag}}(B^1x^1)&X^1&0\\
%   -{\rm{diag}}(B^2x^2)&-D^2+(I-X^1-X^2)B^2-{\rm{diag}}(B^2x^2)&0&X^2   
%     \end{bmatrix}
% \end{align}
% \end{small}
% and this matrix has full row rank at any coexistence equilibrium.
\end{lem}

%\begin{proof}
\textit{Proof:}
We first observe that
\begin{small}
\begin{align}\label{eq:jacxdelta}
    \frac{\partial f_{\delta}(x,\delta)}{\partial (x,\delta)}&=\frac{\partial f_{\delta}(x^1,x^2,x^3,D^1,D^2,D^3)}{\partial (x^1,x^2,x^3,D^1,D^2,D^3)}\\
    &=\begin{bmatrix}\left(\frac{\partial f_{\delta}(x^1,x^2,x^3,D^1,D^2,D^3)}{\partial (x^1,x^2,x^3)}\right)^{\top}\\\nonumber
    \left(\frac{\partial f_{\delta}(x^1,x^2,x^3,D^1,D^2,D^3)}{\partial (D^1,D^2,D^3)}\right)^{\top}\\
    \end{bmatrix}^{\top}
%\end{equation}
\end{align}
\end{small}
% By direct calculation we have 
% \begin{small}
% \begin{align*}
%   & \frac{\partial f_{\delta}(x^1,x^2,D^1,D^2)}{\partial (x^1,x^2)}\\
%   & \quad=\begin{bmatrix}  
%   -D^1+(I-X^1-X^2)B^1-{\rm{diag}}(B^1x^1)&-{\rm{diag}}(B^1x^1)\\
%   -{\rm{diag}}(B^2x^2)&-D^2+(I-X^1-X^2)B^2-{\rm{diag}}(B^2x^2)\end{bmatrix}
% \end{align*}
% \end{small}
Observe next, using the diagonal nature of the $D^i$, that there holds (with $e_j$ denoting the $j$-th unit vector)
\begin{align}
\frac{\partial f_{\delta}(x^1,x^2,x^3,D^1,D^2, D^3)}{\partial \delta^1_j}&=x^1_je_j\\\nonumber 
\frac{\partial f_{\delta}(x^1,x^2,x^3,D^1,D^2,D^3)}{\partial \delta^2_j}&=x^2_je_{j+n}\\\nonumber
\frac{\partial f_{\delta}(x^1,x^2,x^3,D^1,D^2,D^3)}{\partial \delta^3_j}&=x^3_je_{j+2n}
\end{align}
It follows that 
\begin{align}
     \frac{\partial f_{\delta}(x^1,x^2,x^3, D^1,D^2, D^3)}{\partial \delta}&=\frac{\partial f_{\delta}(x^1,x^2,x^3, D^1,D^2, D^3)}{\partial (D^1,D^2, D^3)}\\\nonumber
     &=\begin{bmatrix}
    X^1&0&0\\0&X^2&0\\0&0&X^3
    \end{bmatrix}
\end{align}

At any equilibrium in $\mathcal X$, this matrix has full row rank, as then does the Jacobian matrix $\frac{\partial f_{\delta}(x,\delta)}{\partial (x,\delta)}$ in \eqref{eq:jacxdelta} of which it is a submatrix. 

Let $\mathcal Z$ be the submanifold of $\mathcal Y$ consisting of the single point ${\bf{0_{3n}}}$. It is, in fact, the image of those points in $\mathcal X \times \mathcal D$ defined by $f_{\delta}(x,\delta)={\bf{0_{3n}}}$. Hence at such a point, $x^k\gg 0$ or $X^k$ is nonsingular. Hence the Jacobian of $f_{\delta}$ with respect to $(x,\delta)$ has full rank at such a point, which means that the map  $f_{\delta}$ is transversal to $\mathcal Z$. Then by the Parametric Transversality Theorem \cite[see p.145]{lee2013introduction}, \cite[see p.68]{guillemin2010differential}, it follows that for almost all choices $\bar\delta$ of $\delta$ , the mapping $f_{\bar\delta}:\mathcal X\to \mathcal Y, f_{\bar \delta}(x)=f_{\delta}(x,\bar \delta)$  will be transversal to $\mathcal Z$, i.e. the Jacobian $\frac{\partial f_{\delta}(x,\bar \delta)}{\partial x}$  will be of full rank $3n$  at the zeros of $f_{\bar\delta}$. Since this matrix is square and of size $3n\times 3n$, this says that  every zero of $f(x,\bar\delta)$ is nondegenerate and as a consequence, isolated. As already noted, since the zeros occur in a bounded set, they are finite in number. The choices of $\delta$ for which they are not finite in number, if indeed such choices exist, define a set of measure zero.




\end{document}
\section{Problem Formulation}\label{sec:prob:formulation}
\seb{In this section, we first present a model that represents the possible spread of multiple competing viruses over a population network. Thereafter, we state the requisite assumptions and definitions needed for the theoretical developments of this paper. Lastly, we will formally state the problems of interest.}
% In this section, we detail a model that captures the spread of multiple competing viruses  across a population network. % with a shared resource. %First we establish the premise of the model, from which model variables and dynamics are derived. 
% Subsequently, we detail the  pertinent assumptions and definitions that will be required in the sequel. Finally, we formally specify the problems being investigated.

\subsection{Model}
\seb{We consider  a network of $n\geq 2$ %agents,
nodes\footnote{As an aside, multivirus SIS models where $n=1$ have been studied in, among others, \cite[Section~2]{pare2020modeling}.}. Each node represents a well-mixed population of individuals (i.e., any two individuals in the population can interact with the same positive probability) with a large and constant size. One of the central assumptions underpinning this model is that of homogeneity within the population node, and (possible) heterogeneity outside the population node. That is, all individuals within a population node have the same infection (resp. healing) rates, but individuals in different population nodes need not necessarily have the same healing (resp. infection) rate\cite{lajmanovich1976deterministic}. We suppose that $m$ viruses compete with each other to infect the nodes. The fact that the viruses are competing implies that there are at least two (but possibly more) viruses circulating in the population.}


\seb{The case where $m=1$ (meaning there is no competition) has been well-studied in the literature, see, for instance, \cite{khanafer2016stability,pare2018epidemic,lajmanovich1976deterministic,fall2007epidemiological,van2008virus}; whereas the case where $m=2$ (i.e., two competing viruses) has been explored relatively well in recent times, see, for example, \cite{sahneh2014competitive,santos2015bi,ye2021convergence,liu2019analysis,pare2021multi}. Nonetheless, an overarching analysis accounting for $m$ competing viruses, where $m$ is some arbitrary but finite positive integer, is, to the best of our knowledge, not available in the existing literature. Therefore, as a first step, in this direction, the present paper deals with the case where $m=3$. That is, in the rest of this paper we will consider three competing viruses. Hence, within each population node, individuals can be partitioned into four mutually exclusive health compartments: susceptible, infected with virus~1, infected with virus~2, infected with virus~3. Note that  no individual can be \emph{simultaneously} infected by more than one virus \cite{laurie2018evidence,wu2020interference}. A population node is termed \emph{healthy} if all individuals in said node belong to the susceptible compartment; otherwise, we say it is infected. An individual, belonging to population node $i$ (where $i\in [n]$), in the susceptible compartment, as a consequence of coming into contact with its infected neighbors, transitions to the ``infected with virus $k$" (for $k \in [m]$) compartment at a rate $\beta_i^k > 0$. An individual in population~$i$ that is infected with virus~$k$ recovers from it 
based on %its 
said individual's
healing rate with respect to virus~$k$, i.e., $\delta_i^k > 0$.}
% Consider a network of $n\geq 2$ %agents,
% nodes\footnote{As an aside, multivirus SIS models where $n=1$ have been studied in, among others, \cite[Section~2]{pare2020modeling}.}, where $m$ viruses compete with each other to infect the %agents. 
% nodes. The notion of competition implies the presence of at least two (but possibly more) viruses. %Therefore, 
% Throughout this paper, $m = 3$.
% In context, each node %agent 
% %(hereafter, interchangeably referred to as nodes) 
% represents a well-mixed population of individuals with a large and constant size. A well-mixed population means any two individuals in the population can interact with the same positive probability. A key assumption that underpins this model is that of homogeneity within the population node, and (possible) heterogeneity outside the population node. That is, all individuals within a population node have the same infection (resp. healing) rates, but individuals in different population nodes need not necessarily have the same healing (resp. infection) rate\cite{lajmanovich1976deterministic}.
%an agent could be thought of as an individual or, equivalently, as a subpopulation of individuals (for instance, a district in a big city). 
% An agent, at any given time, could be either infected by one of the $m$ viruses or none; no agent can be infected by more than one virus simultaneously. If, at a given time, none of the individuals in a subpopulation are infected, we say that the agent is healthy; otherwise, we say it is infected. An agent that is healthy could, depending on its infection rate $\beta_i^k$, get infected with virus~$k$ (for $k \in [m]$) as a consequence of coming in direct contact with an infected individual in  a neighboring subpopulation. An agent that is infected with virus~$k$ recovers from it 
% based on %its 
% said agent's
% healing rate with respect to virus~k, i.e., $\delta_i^k$.

% Within each population, individuals can be partitioned into four mutually exclusive health compartments: susceptible, infected with virus 1, infected with virus 2, infected with virus~3. More precisely,  no individual can be \emph{simultaneously} infected by more than one virus \cite{laurie2018evidence,wu2020interference}.
% We say a population node is healthy if all individuals belong to the susceptible compartment; otherwise we say it is infected.
% An individual, belonging to population $i$ (where $i\in [n]$), in the susceptible compartment, can transition to the ``infected with virus $k$" (for $k \in [m]$) compartment at a rate $\beta_i^k > 0$. An individual in population~$i$ that is infected with virus~$k$ recovers from it 
% based on %its 
% said individual's
% healing rate with respect to virus~$k$, i.e., $\delta_i^k > 0$.


\par \seb{We model the spread of $m$-competing viruses using an $m$-layer graph $G$; the vertices of the graph represent the population nodes. The contact  graph for the spread of virus~$k$, for each $k \in [m]$, is denoted by the $k^\textrm{th}$ layer. More specifically, for the  graph $G$, there exists a directed edge from node $j$ to node $i$ in layer $k$ if, assuming an individual in population $j$ is infected with virus~$k$, then said individual can infect at least one (but possibly more) healthy individual in node~$i$. The edge set corresponding to the $k^\textrm{th}$ layer of $G$ is denoted by $E^k$, whereas the weighted adjacency matrix corresponding to layer~$k$ is denoted by $A^k$ (where $a_{ij}^k \geq 0$). The elements in $A^k$ are in one-to-one correspondence with the existence (or lack thereof) of edges in layer $k$.  That is,  $(i,j)\in E^k$ if, and only if, $a_{ji}^k\neq 0$. We use $x_i^k(t)$ to represent the fraction of individuals infected with virus~$k$ in population~$i$ at time instant $t$. Given that the viruses are circulating in the population nodes, the infection level $x_i^k(t)$ possibly changes over time. 
% \par The spread of $m$-competing viruses can be modeled using an $m$-layer graph $G$, where the vertices of the graph represent the population nodes. The $k^\textrm{th}$ layer denotes the contact graph for the spread of virus~$k$, for each $k \in [m]$. 
% More specifically, for the  graph $G$, there exists a directed edge from node $j$ to node $i$ in layer $k$ if, assuming an individual in population $j$ is infected with virus~$k$, then said individual can infect at least one (but possibly more) healthy individual in node~$i$. Let $E^k$ denote the edge set corresponding to the $k^\textrm{th}$ layer of $G$.
% We denote by $A^k$ (where $a_{ij}^k \geq 0$) the weighted adjacency matrix corresponding to layer~$k$, with the elements in $A^k$ being in one-to-one correspondence with the existence (or lack thereof) of edges in layer $k$.  That is,  $(i,j)\in E^k$ if, and only if, $a_{ji}^k\neq 0$. Let $x_i^k(t)$ denote the fraction of individuals infected with virus~$k$ in population~$i$ at time instant $t$. 
Hence,  the evolution of said fraction can, then, be represented by the following scalar differential equation \cite[Equation~4]{pare2021multi}:
\begin{equation} \label{eq:scalar}
   %\dot{x}_i^k(t) =  \Big{(} \big{(} I - \textstyle \sum_{l=1}^m \diag(x^l(t)) \big{)} B^k  \Big{)} x^k(t), \\
   \dot{x}_i^k(t) = - \delta_i^k x_i^k(t) + \big{(} 1 - \textstyle \sum_{l=1}^m x_i^l(t) \big{)} %\nonumber \\ 
  %\times \big{(} \beta_{iw}^k z^k(t) +
  \textstyle \sum_{j=1}^{n} \beta_{ij}^k x_j^k(t),
  \end{equation}
where $\beta_{ij}^k = \beta_i^ka_{ij}^k$.
}
Define $x^k(t) = [x_1^k(t), \hdots, x_n^k(t)]^\top$, $D^k =\diag{(\delta_i^k)}$, and $B^k=[\beta_{ij}^k]_{n \times n}$. \seb{Therefore, the $n$ coupled equations in~\eqref{eq:scalar} for $i \in [n]$}
 %Therefore, ~\eqref{eq:scalar} 
can be written as
\begin{equation} \label{eq:vec}
   \dot{x}^k(t) =  \Big{(} - D^k+ \big{(} I - \textstyle \sum_{l=1}^m \diag(x^l(t)) \big{)} B^k  \Big{)} x^k(t), 
  \end{equation}
Defining $x(t):=[x^1(t),  \dots , x^m(t)]^T$,  and  $R^k(x(t)) := \big{(} - D^k + (I - \textstyle \sum_{l=1}^m \diag(x^l(t)))B^k \big{)}$, the dynamics of the system of all $m$ viruses are given by %\scriptsize 
\begin{gather} \label{eq:full}
 \dot{x}(t)
 =
 \begin{bmatrix}
 R^1 \big{(} x(t) \big{)} & 0 & \dots & 0 \\
 0 & R^2 \big{(} x(t) \big{)} & \dots & 0 \\
 \vdots & \vdots & \ddots & \vdots \\
 0 & 0 & \dots & R^m \big{(} x(t) \big{)}
 \end{bmatrix}
 x(t).
\end{gather}  
  
% Note that by setting $m=1$ in~\eqref{eq:full}, one recovers the classic single-virus SIS model that has been studied extensively in the literature; see \cite{van2008virus,khanafer2016stability,lajmanovich1976deterministic}. Setting $m=2$ yields the classic networked bi-virus SIS model, for which a plethora of results have been provided in \cite{liu2019analysis,sahneh2014competitive,santos2015bi,castillo1989epidemiological,ben:lcss,ye2021convergence,anderson2022equilibria}.
% In this paper, we are interested in the case when $m=3$, i.e., the tri-virus networked competitive spread.

% Consider a system of $m$-competitive SIS viruses. The dynamics of the $k^\textrm{th}$ virus, where $k =1,2,\hdots,m$, is as follows:
% \begin{equation} \label{eq:vec}
%   \dot{x}^k(t) =  \Big{(} \big{(} I - \textstyle \sum_{l=1}^m \diag(x^l(t)) \big{)} B^k - D^k \Big{)} x^k(t), 
%   \end{equation}
%   where $x^k(t)$ denotes the infection level with respect to virus $k$ at time $t$; $D^k =\diag{(\delta_i^k)}$; and $B^k = \bar{B}^k.A^k$ with $\bar{B}^k =\diag{(\beta_i^k)}$. The terms $\delta_i^k $ (resp. $\beta_i^k$) denote the infection rate (resp. healing rate) of agent $i$ with respect to virus $k$, while $A^k$ denotes the adjacency matrix that represents how virus $k$ spreads between different nodes of a given population.  %where $k=1,2,\hdots,m$.
% Defining $x(t):=[x^1(t),  \dots , x^m(t)]^T$,  and  $A^k(x(t)) := \big{(} - D^k + (I - \textstyle \sum_{l=1}^m \diag(x^l(t)))B^k \big{)}$, the dynamics of the system of all $m$ viruses are given by %\scriptsize 
% \begin{gather} \label{eq:full}
%  \dot{x}(t)
%  =
%  \begin{bmatrix}
%  A^1 \big{(} x(t) \big{)} & 0 & \dots & 0 \\
%  0 & A^2 \big{(} x(t) \big{)} & \dots & 0 \\
%  \vdots & \vdots & \ddots & \vdots \\
%  0 & 0 & \dots & A^m \big{(} x(t) \big{)}
%  \end{bmatrix}
%  x(t).
% \end{gather}


\normalsize
%Observe that for the case when $m=2$, system~\eqref{eq:full} is monotone \cite{ye2021convergence}. That is, supposing $x(0)$ and $y(0)$ are initial states of system~\eqref{eq:full} such that  $x(0) \leq y(0)$ then $x(t) \leq y(t)$ for all $t$. It is not known if the same can be said for $m=3$. Consequently, understanding the limiting behavior of tri-virus systems remains open. %This note aims at addressing some of the challenges in this regard.
\par Define, for $k \in [3]$, $X^k=\diag(x^k)$. Based on~\eqref{eq:vec}, the dynamics of the tri-virus system can be written as follows:

\begin{align} 
   \dot{x}^1(t) &=  \Big{(} \big{(} I - (X^1+X^2+X^3) \big{)} B^1 - D^1 \Big{)} x^1(t), \label{eq:x1}\\
      \dot{x}^2(t) &=  \Big{(} \big{(} I - (X^1+X^2+X^3) \big{)} B^2 - D^2 \Big{)} x^2(t) \label{eq:x2}\\
         \dot{x}^3(t) &=  \Big{(} \big{(} I - (X^1+X^2+X^3) \big{)} B^3 - D^3 \Big{)} x^3(t). \label{eq:x3}
  \end{align}
 
\subsection{Assumptions and Preliminary Lemmas}  
 We need the following assumptions to ensure that the aforementioned model is well-defined. \seb{Further, note that these assumptions are standard in the literature on (multi-competitive) networked SIS models; see, for instance, \cite{lajmanovich1976deterministic,fall2007epidemiological,khanafer2016stability,liu2019analysis}.}
 \begin{assm} \label{assum:base}
Suppose that $\delta_i^k>0,  \beta_{ij}^k \geq 0$  for all $i, j \in [n]$ and $k \in [3]$. 
%~$\blacksquare$ 
\end{assm}
%
 
\begin{assm}\label{assum:irreducible}
The matrix $B^k$, for $ k \in [3]$ is irreducible. \end{assm} 
 
Observe that under Assumption~\ref{assum:base},  for all $k \in [3]$, $B^k$ is a nonnegative matrix and $D^k$ is a positive diagonal matrix. %(and is therefore invertible). %Thus, $(B_w^k - D_w^k)$ is a Metzler matrix, for all $k \in [m]$. 
%Moreover, recall that a square matrix $M$ is said to be irreducible if, replacing the non-zero elements of $M$ with ones and interpreting it as an adjacency matrix, the corresponding graph is strongly connected. 
Moreover, recall that a square nonnegative matrix $M$ has the irreducibility property if and only if, supposing $M$ is the  (un)weighted adjacency matrix of a graph, the corresponding graph is strongly connected.
Then, noting that non-zero elements in $B^k$ represent directed edges in the set $E^k$, we see that $B^k$ is irreducible whenever the $k^{\rm{th}}$ layer of the multi-layer network $G$ is strongly connected. %when $B_w^k$ is interpreted as an adjacency matrix representing the set of directed edges $E^k$, $B_w^k$ is said to be irreducible if and only if the $k^{\rm{th}}$ layer of the multi-layer network $G$ is strongly connected.

\par  As a consequence of %these assumptions
Assumption~\ref{assum:base}, %we can
% will be able to 
%restrict 
our analysis of the model in~\eqref{eq:x1}-\eqref{eq:x3} is limited to the sets $\mathcal{D}:= \{x(t): x^k(t) \in [0,1]^n,  \forall k \in [3], \sum_{k=1}^{3}x^k \leq \textbf{1}\}$ and $\mathcal{D}^k:= \{x^k(t) \in [0,1]^n\}$. %for system~\eqref{eq:full} and~\eqref{eq:yk}, respectively. 
%\axel{\textbf{[It would be prudent to restrict S further, so that the sum of all the $p^k(t)$ has be less than one in every component.]}} 
Given that $x_i^k(t)$ is interpreted as a fraction of a population, %and $z^k(t)$ is a nonnegative quantity,
the aforementioned sets represent the %realistic
sensible domain of the %model 
system. 
That is, %, for any $k \in [m]$, 
if $x^k(t)$ takes values outside of $\mathcal{D}^k$, then those values will not correspond to physical reality. %would lack physical meaning. 
The following lemma shows that $x(t)$ never leaves the set $\mathcal{D}$.
%
%In keeping with this, we can establish the following result for these sets:
%
\begin{lem}{\cite[Lemma~1]{pare2021multi}}\label{lem:pos}
Let Assumption~\ref{assum:base} hold. Then $\mathcal{D}$ is positively invariant with respect to~\eqref{eq:full}.%, and if $y(t) \in \mathcal{D}$ for $t \geq 0$, $\mathcal{D}^k$ is positively invariant with respect to~\eqref{eq:yk} for all $k \in [m]$.
%~$\blacksquare$
\end{lem} 

\begin{lem}\cite[Lemma~7]{axel2020TAC} \label{lem:pos_never_zero}
Let Assumption~\ref{assum:base} hold. Then $\mathcal{D} \setminus \{ \textbf{0} \}$ is positively invariant with respect to system~\eqref{eq:x1}-\eqref{eq:x3}.
\end{lem}


Clearly,  $(\textbf{0}, \textbf{0},\textbf{0})$ is an equilibrium of~\eqref{eq:x1}-\eqref{eq:x3}, and is referred to as the disease-free equilibrium (DFE). %We now recall
A sufficient condition for global exponential stability (GES) of the DFE is as follows:
\begin{prop}\cite[Theorem~1]{axel2020TAC}\label{prop:exp:convergence}
Consider system~\eqref{eq:x1}-\eqref{eq:x3} under Assumption~\ref{assum:base}. If $s(-D^k+B^k) < 0$, for each $k \in [3]$, then the DFE  is exponentially stable, with a domain of attraction containing~$\mathcal{D}$.
\end{prop}

Clearly, the conditions in Proposition~\ref{prop:exp:convergence} also imply asymptotic convergence to the DFE. However, even when the strict inequalities in the %said
aforementioned proposition are %relaxed 
weakened to allow equality,   asymptotic convergence to the DFE can still be achieved.
%It turns out that by relaxing the strict inequality in Proposition~\ref{prop:exp:convergence}, one can still achieve asymptotic convergence to the DFE. 
%This is formalized in 
The next proposition, which is  a generalization of an analogous result for the bivirus setting (see \cite[Theorem~1]{liu2019analysis}), formalizes the same.
% A sufficient condition for global asymptotic stability (GAS) of the DFE, which is a generalization of an analogous result for the bivirus setting (see \cite[Theorem~1]{liu2019analysis}) is as follows. 
\begin{prop}\cite[Lemma~2]{pare2021multi}
Consider system~\eqref{eq:x1}-\eqref{eq:x3} under Assumption~\ref{assum:base}. If $s(-D^k+B^k) \leq 0$, for each $k \in [3]$, %$s(-D^2+B^2) \leq 0$ and $s(-D^3+B^3) \leq 0$,
then the DFE is the unique equilibrium of system~\eqref{eq:x1}-\eqref{eq:x3}. Moreover, it is asymptotically stable with the domain of attraction $\mathcal D$.\label{prop:phil}
\end{prop}

%While Proposition~\ref{prop:phil} identifies a sufficient condition for asymptotic convergence to the DFE, the next proposition identifies a sufficient condition for \emph{exponential convergence} to the DFE.

% \begin{prop}\cite[Theorem~1]{axel2020TAC}\label{prop:exp:convergence}
% Consider system~\eqref{eq:x1}-\eqref{eq:x3} under Assumption~\ref{assum:base}. If $s(-D^k+B^k) < 0$, for each $k \in [3]$, then the DFE  is exponentially stable, with domain of attraction containing~$\mathcal{D}$.
% \end{prop}

\seb{Note that Proposition~\ref{prop:exp:convergence} and~\ref{prop:phil} provide guarantees on convergence to the DFE. It is natural to ask what happens if, for some $k \in [m]$, one of the eigenvalues of the matrix $-D^k+B^k$ has a positive real part. It turns out for each $k \in [m]$, that violates the eigenvalue condition in Proposition~\ref{prop:phil}, there exists an equilibrium of the form $(\textbf{0}, \dots, \Tilde{x}^k, \dots, \textbf{0})$ in $\mathcal{D}$. %where $\Tilde{x}^k$ is the single-virus endemic equilibrium corresponding to virus~$k$.
 The following proposition formalizes this.}

%It turns out that for every eigenvalue condition in Proposition~\ref{prop:phil} that is violated, one obtains an equilibrium of the form $(\textbf{0}, \dots, \Tilde{x}^k, \dots, \textbf{0})$ in $\mathcal{D}$, where $\Tilde{x}^k$ is the single-virus endemic equilibrium corresponding to virus~$k$. This is formalized in the following proposition.
%if each of the eigenvalue conditions in Proposition~\ref{prop:phil} were to be violated, then the tri-virus system has at least $4$ equilibria, as we recall in the following proposition.
\begin{prop} %\cite{liu2019analysis,axel2020TAC}
\cite[Theorem~2.1]{fall2007epidemiological} \label{prop:necessity}
Consider system~\eqref{eq:x1}-\eqref{eq:x3} under Assumptions~\ref{assum:base} and~\ref{assum:irreducible}. For each $k \in [3]$, %such that $B^k$ is irreducible, 
there exists a unique %single-virus endemic 
equilibrium $(\textbf{0}, \dots, \Tilde{x}^k, \dots, \textbf{0})$ in $\mathcal{D}$, with $\textbf{0} \ll \Tilde{x}^k \ll \textbf{1}$ if, and only if, $s(B^k - D^k) > 0$.

%and $s(B^k - D^k) > 0$, there is a unique single-virus endemic equilibrium $(\textbf{0}, \dots, \Tilde{x}^k, \dots, \textbf{0})$ in $\mathcal{D}$, with $\textbf{0} \ll \Tilde{x}^k \ll \textbf{1}$.%~$\blacksquare$
\end{prop}
Analytic methods for computing the single-virus endemic equilibria have been provided in %\cite[Theorem~4.3]{mei2017epidemics_review} and 
\cite[Theorem~5]{van2008virus}.

\seb{Any non-zero equilibrium in $\mathcal D$ is referred to as an \emph{endemic} equilibrium. Endemic equilibria can be further classified as follows:} 
%The 
Equilibria of the form $(\textbf{0}, \dots, \Tilde{x}^k, \dots, \textbf{0})$ are referred to as the \emph{boundary equilibria}. The equilibria of the form $(\bar{x}^1, \bar{x}^2, \bar{x}^3)$, %with
where at least $\bar{x}^i$ and $\bar{x}^j$ ($i, j \in [3], i\neq j$) 
%being strictly positive vectors 
are nonnegative vectors with at least one positive entry in each of $\bar{x}^i$ and $\bar{x}^j$ are referred to as \emph{coexistence equilibria}. 
\seb{The coexistence equilibria can be further classified as i) 2-coexistence equilibria, which are  equilibria of the form $(\bar{x}^1, \bar{x}^2, \bar{x}^3)$ where  $\bar{x}^i$ and $\bar{x}^j$ for some ($i, j \in [3], i\neq j$) 
%being strictly positive vectors 
are nonnegative vectors with at least one positive entry in each of $\bar{x}^i$ and $\bar{x}^j$, and, for $k\neq i, k \neq j$ $\bar{x}^k=\textbf{0}$, and ii) 3-coexistence equilibria, which are  equilibria of the form $(\bar{x}^1, \bar{x}^2, \bar{x}^3)$ where, for each $i \in [3]$,  $\bar{x}^i$ is a nonnegative vector, that has at least one positive entry.}
Indeed, such vectors are in fact strictly positive; see \cite[Lemma~6]{axel2020TAC}. \seb{That is, at any endemic equilibrium, there is at least one individual in every population node, who is infected.}
%Lemma~\ref{lem:equi_non-zero_nonone}.
%are referred to as \emph{coexisting equilibria}.


Let  $J(x^1,x^2,x^3)$ denote the Jacobian matrix of system~\eqref{eq:x1}-~\eqref{eq:x3} for an arbitrary  point in the state space. It is easily checked that  $J(x^1,x^2,x^3)$ is as given in~\eqref{jacob} below.

\begin{figure*}[h!]
	{\noindent}
\begin{align}\label{jacob}
&J(x^1,x^2,x^3) = \\
&\scriptsize
\begin{bmatrix}
-D^1+(I-X^1-X^2-X^3)B^1-\diag(B^1x^1) & -\diag(B^1x^1)   & -\diag(B^1x^1)  \\
-\diag(B^2x^2) & -D^2+(I-X^1-X^2-X^3)B^2-\diag(B^2x^2)  & -\diag(B^2x^2)\\
 -\diag(B^3x^3)& -\diag(B^3x^3)& -D^3+(I-X^1-X^2-X^3)B^3-\diag(B^3x^3)  \end{bmatrix}\normalsize,\nonumber
\end{align}
\caption*{}
\end{figure*}

 
% \subsection{Monotone dynamical systems and competitive bivirus networked SIS models} 

% Observe that for the case when $m=2$, system~\eqref{eq:full} is, under Assumption~\ref{assum:irreducible}, monotone \cite[Lemma~3.3]{ye2021convergence} (and, assuming homogeneous recovery rates, also in \cite[Theorem~18]{santos2015bi}). 
%  That is, setting $m=2$ for system~\eqref{eq:full}, suppose that $(x_A^1(0), x_A^2(0))$ and $(x_B^1(0), x_B^2(0))$ are two initial conditions in $\textrm{int}(D)$ satisfying i) $x_A^1(0)>x_B^1(0)$ and ii) $x_A^2(0)<x_B^2(0)$. Since the bivirus system is monotone, it follows that, for all $t$, i) $x_A^1(t)\gg x_B^1(t)$ and ii) $x_A^2(t)\ll x_B^2(t)$. 
% Further, it has been shown that, for almost all choices of $D^i$, $B^i$, $i=1,2$, system~\eqref{eq:full} has a finite number of equilibria \cite[Theorem~3.6]{ye2021convergence}. Therefore, from \cite[Theorems~2.5 and~2.6]{smith1988systems} (or \cite[Theorem~9.4]{hirsch1988stability}), we know that for almost all initial conditions in $\mathcal D$, system~\eqref{eq:full} with $m=2$ converges to a stable equilibrium point. The set of initial conditions for which said convergence does not occur (in which case it is  either a)  already at an unstable equilibrium, assuming such an equilibrium exists, or b) in the stable manifold of an unstable equilibria, 
% or c) on a nonattractive limit cycle, or d) in the stable manifold of a nonattractive limit cycle) has measure zero. It is not known if an analogous statement is true for the case when $m=3$. Consequently, understanding the limiting behavior of tri-virus systems remains open if any of the eigenvalue conditions in Proposition~\ref{prop:phil} are violated.

% Further,  for the case when $m=2$, a sufficient condition for local exponential convergence to a boundary equilibrium has been identified in \cite[Theorem~3.10]{ye2021convergence}, whereas for the $m=3$ case no such condition has been identified. Likewise, certain special (nongeneric) scenarios  have been identified which lead to the existence of a continuum of coexistence equilibria; see \cite[Theorems~6 and~7]{liu2019analysis}. Improving upon these results, a broader scenario (but again nongeneric), that accounts for a larger class of parameters, has been identified which leads to not only the existence, but also local exponential attractivity,
% of a  continuum of coexistence equilibria; see \cite[Proposition~3.9]{ye2021convergence}. Analogous results for the $m=3$ case are as yet unavailable. 

 
 %\subsection{Preliminary lemmas}
\seb{
 \begin{lem} \label{lem:eigspec}
\cite[Proposition~1]{liu2019analysis} Suppose that $\Lambda$ is a negative diagonal matrix and $N$ is an irreducible nonnegative matrix. 
%Let $M = \Lambda+N$.
Let $M$ be the irreducible Metzler matrix $M = \Lambda+N$. 
Then, $s(M) < 0$ if and only if $\rho(-\Lambda^{-1} N) < 1, s(M)=0$ if and only if $\rho(-\Lambda^{-1} N) = 1$, and $s(M)>0$ if and only if, $\rho(-\Lambda^{-1} N) > 1$.%~$\blacksquare$
\end{lem}
%
We will also be making use of the following variants of the Perron-Frobenius theorem for irreducible matrices.

\begin{lem} \label{lem:perron_frob}
\cite[Chapter 8.3]{meyer2000matrix} \cite[Theorem~2.7]{varga1999matrix} %(Perron-Frobenius Theorem) 
Suppose that $N$ is an irreducible nonnegative matrix. Then,
%
\begin{enumerate}[label=(\roman*)]
    \item $r = \rho(N)$ is a simple eigenvalue of $N$. \label{item:perfrob_simpleeig}
    \item There is an eigenvector $\zeta \gg \textbf{0}$ corresponding to the eigenvalue $r$. \label{item:perfrob_pos_exists}
    \item $x > \textbf{0}$ is an eigenvector only if $Nx = rx$ and $x \gg \textbf{0}$. %\axel{\textbf{[Not sure this needs to be mentioned]}} 
    \label{item:perfrob_pos_necess}
    \item If $A$ is a nonnegative matrix such that $A < N$, then $\rho(A) < \rho(N)$. \label{item:perfrob_matrix_ineq}%~$\blacksquare$
\end{enumerate}
%
\end{lem}
%
%Comment: Double-check whether we are indeed using all 4 items in Lemma~\ref{lem:perron_frob}]
\begin{lem} \label{lem:perron_frob_metz}
\cite[Lemma~2.3]{varga1999matrix} Suppose that $M$ is an irreducible Metzler matrix. Then $r = s(M)$ is a simple eigenvalue of $M$, and there exists a corresponding eigenvector  $\zeta \gg \textbf{0}$.%~$\blacksquare$
\end{lem}
}
% Let  $J(x^1,x^2,x^3)$ denote the Jacobian matrix of system~\eqref{eq:x1}-~\eqref{eq:x3} for an arbitrary  point in the state space. Therefore, $J(x^1,x^2,x^3)$ is as given in~\eqref{jacob}.
% \begin{align}\label{jacob}
% &J(x^1,x^2,x^3) = \\
% &\tiny
% \begin{bmatrix}
% -D^1+(I-X^1-X^2-X^3)B^1-\diag(B^1x^1) & -\diag(B^1x^1)   & -\diag(B^1x^1)  \\
% -\diag(B^2x^2) & -D^2+(I-X^1-X^2-X^3)B^2-\diag(B^2x^2)  & -\diag(B^2x^2)\\
% - \hat B^{3} & - \hat B^{3}& -D^3+(I-X^1-X^2-X^3)B^3-\diag(B^3x^3)  \end{bmatrix}\normalsize,\nonumber
% \end{align}

% \begin{figure*}[h!]
% 	{\noindent}
% \begin{align}\label{jacob}
% &J(x^1,x^2,x^3) = \\
% &\scriptsize
% \begin{bmatrix}
% -D^1+(I-X^1-X^2-X^3)B^1-\diag(B^1x^1) & -\diag(B^1x^1)   & -\diag(B^1x^1)  \\
% -\diag(B^2x^2) & -D^2+(I-X^1-X^2-X^3)B^2-\diag(B^2x^2)  & -\diag(B^2x^2)\\
%  -\diag(B^3x^3)& -\diag(B^3x^3)& -D^3+(I-X^1-X^2-X^3)B^3-\diag(B^3x^3)  \end{bmatrix}\normalsize,\nonumber
% \end{align}
% \end{figure*}
%where $\hat{B}^i=\diag(B^i\Tilde{x}^i)$, for $i=1,2,3$.
% \begin{align}
% J_{1,1} &= W B^{1} -D^{1} - \hat B^{1} -\epsilon^1 \hat B^2 \\
% J_{2,2} &= W B^{2} -D^{2} - \hat B^{2} -\epsilon^2 \hat B^1 \\
% J_{3,3} &= -D^1 - D^2 + \epsilon^1 \hat X^1 B^2 + \epsilon^2 \hat X^2 B^1.
% \end{align}

 
% Clearly, the DFE (i.e.,$(\textbf{0}, \textbf{0},\textbf{0})$) is an equilibrium of~\eqref{eq:x1}-\eqref{eq:x3}. %We now recall
% A sufficient condition for global asymptotic stability (GAS) of the DFE, which is a generalization of an analogous result for the bivirus setting is as follows. \cite[Theorem~1]{liu2019analysis}.
% \begin{prop}\cite[Lemma~2]{pare2021multi}
% Consider system~\eqref{eq:x1}-\eqref{eq:x3} under Assumption~\ref{assum:base}. If $s(-D^1+B^1) \leq 0$, $s(-D^2+B^2) \leq 0$ and $s(-D^3+B^3) \leq 0$, then the DFE is the unique equilibrium of system~\eqref{eq:x1}-\eqref{eq:x3}. Moreover, it is asymptotically stable with the domain of attraction $\mathcal D$.\label{prop:phil}
% \end{prop}
\seb{
\begin{lem}[Restrictions on the trajectories of the tri-virus system] Consider system~\eqref{eq:x1}-\eqref{eq:x3} under Assumption~\ref{assum:base}. Suppose that the initial conditions satisfy a) $x^k(0) \gg \textbf{0}$ for $k \in [3]$, and ii) $(x^1(0), x^2(0), x^3(0)) \in \mathcal D$. Further, suppose that matrix $B^k$ for $k \in [3]$ is irreducible.  For all finite $t>0$, $\textbf{0} \ll x^k(t) \ll \textbf{1}$ for $k \in [3]$; and $x^1(t)+x^2(t)+x^3(t) \ll \textbf{1}$.\label{lem:inward:pointing}
\end{lem}
The proof closely follows that of \cite[Lemma~3.2]{ye2021convergence} for the bivirus problem. 
\par \textit{Proof:}  Define $z:=\textbf{1}-x^1-x^2-x^3$, and 
%recall that 
$\hat{B}^i:=\diag(B^ix^i)$. Hence,~\eqref{eq:x1}-\eqref{eq:x3} can be rewritten as:
\begin{align}
    \dot{x}^i(t)&=-D^ix^i(t)+\hat{B}^iz(t), i=1,2,3 \nonumber\\
    \dot{z}(t)&= D^1x^1(t)+D^2x^2(t)+ D^3x^3(t)-[\hat{B}^1+\hat{B}^2+\hat{B}^3]z(t)
\end{align}
Suppose that for some $\tau \in \mathbb{R}_{>0}$, and for some $i \in [n]$, $z_i(\tau)=0$, which, since $z_i(\tau)=1-x_i^1(\tau)-x_i^2(\tau)-x_i^3(\tau)$, implies that either $x_i^1(\tau)\neq 0$ and/or $x_i^2(\tau)\neq 0$ and/or $x_i^3(\tau)\neq 0$. Therefore, since $D^1, D^2, \text{and } D^3$ are positive diagonal matrices, it must be that $\dot{z}_i(\tau)>0$. This implies that $z_i(\tau) > 0$, and hence we have that $z_i(t)>0$ for all $t$ and $i \in [n]$. As a result, we obtain $z(t) \gg \textbf{0}$ for all $t >0$, which implies that $x^1(t)+x^2(t)+x^3(t) \ll \textbf{1}$. Hence, we have $x^k(t) \ll \textbf{1}$ for $k \in [3]$.
\par Suppose that for some $t \in \mathbb{R}_{\geq 0}$, $\ell$ (where $\ell <n$) %positions in 
entries in
$x^1(t)$ equal zero, and let us label these %positions 
entries $i_1, i_2, \hdots, i_\ell$. Since, by assumption, the matrix $B^1$ is irreducible, it follows from the property of irreducible matrices that there is at least one entry %position 
in the vector $B^1x^1$ that is nonzero even though the corresponding %position 
entry in the vector $x^1$ equals zero \cite[Lemma~1]{liu2019analysis}. Let us assume that this occurs for the $i_1^{th}$ entry, i.e., $x^1_{i_1}=0$ yet $[B^1x^1]_{i_1}>0$. Observe that the evolution of the infection level with respect to virus~1 in node $i_1$ is as follows:
\begin{align}\label{ineq:nozeros}
    \dot{x}^1_{i_1}(t)&=[-D^1x^1(t)]_{i_1} +[(I-X^1-X^2-X^3)B^1x^1(t)]_{i_1} \nonumber \\
    &>0,
\end{align}
where the inequality in~\eqref{ineq:nozeros} follows by noting that $x_{i_1}(t)=0$ by assumption, and $[B^1x^1]_{i_1}>0$ as discussed above. This means that there must exist some time instant $t^\prime$, with $t^\prime-t$ not too large, such that $x^1(t^\prime)$ has fewer than $\ell$ zero entries. Repeating the argument for all the other zero entries in the  vector $x^1(t^\prime)$ (and this can be done, since the choice of node $i_1$ was arbitrary), we have that $x^1(t) \gg \textbf{0}$ for $t>0$. Analogously, we can prove that $x^2(t), x^3(t) \gg \textbf{0}$ for $t>0$.~\qed
}
\par \seb{Lemma~\ref{lem:inward:pointing}  substantially limits the equilibria that can lie on the boundary of the set $\{x^1, x^2, x^3 \in \mathbb{R}_{\geq 0}\mid x^1+x^2+x^3 \leq \textbf{1}\}$}.
% \brian{I think we need the result also that at an equilibrium (and under the usual assumptions including $B^i$ irreducible), any equilibrium either has $x^k=0$ or $x^k\gg 0$. In my opinion, the proof is almost identical to the proof of the corresponding result for the bivirus case, which can be found in Lemma 3.1 of \cite{ye2021convergence}, and no detail need be given. I will assume a result along these lines has been inserted at this point as a lemma in constructing the argument about finiteness of the number of equilibria. }
\seb{
\begin{lem}{\cite[Lemma~6]{axel2020TAC}} \label{lem:equi_non-zero_nonone:1}
Consider system~\eqref{eq:x1}-\eqref{eq:x3} under Assumptions~\ref{assum:base} and~\ref{assum:irreducible}. %Suppose that for each $i \in [n]$, $\sum_{j=1}^{n}B^1_{ij} >0$.%and~\ref{assum:irreducible}. %Suppose, for all $k \in [3]$, that $B^k$ is irreducible. 
 If $x = (x^1, \dots, x^3) \in \mathcal{D}$ is an equilibrium of~\eqref{eq:x1}-\eqref{eq:x3}, then, for each $k \in [3]$, either $x^k = \textbf{0}$, or $\textbf{0} \ll x^k \ll \textbf{1}$. Moreover, %we have that 
$\textstyle \sum_{k=1}^3 x^k \ll \textbf{1}$.
\end{lem}
}
%\cite[Proposition~9]{ye2021convergence}

\subsection{Problem Statements} 
\seb{
The goal of the present paper is  to answer the following questions:
\begin{enumerate}[label=\roman*)]
  \item Is the tri-virus system monotone?
 \item Does the tri-virus system generically admit a finite number of equilbria?
    \item Can we identify a sufficient condition for local exponential convergence to a boundary equilibrium? 
    \item Can we identify a sufficient condition for the nonexistence of 3-coexistence equilibria (resp. specific forms of 2-coexistence equilibria)?
\item Can we identify a sufficient conditions for the existence of a 3-coexistence equilibrium?
    \item Can we identify sufficient condition(s) for the existence and local attractivity of a line of coexistence equilibria?
    \item Can we identify special case(s) where, irrespective of the non-zero initial infection levels, the tri-virus dynamics converge to a plane of coexistence equilibria?
\end{enumerate}
}


\section{Monotonicity (or lack thereof) of the tri-virus system} \label{sec:trivirus:monotone}
\seb{Note that the system defined by \eqref{eq:x1}-\eqref{eq:x3} is a (networked) nonlinear system. A particular class of nonlinear systems is \emph{monotone dynamical systems} (MDS). For a detailed overview of MDSs, see, for instance, \cite{smith2008monotone}. Before investigating whether or not the tri-virus system is monotone, we detail the importance of the notion of MDS in the context of competitive bi-virus SIS networked model.}

%In this section, we seek to conclusively answer whether (or not) system~\eqref{eq:full} with $m=3$ is a monotone dynamical system (MDS). \seb{For a detailed overview of monotone systems, see, for instance, \cite{smith2008monotone}. Before answering the question, we detail the importance of the notion of MDS in the context of multi-competitive SIS networked models}.

\subsection{Monotone dynamical systems and competitive bivirus networked SIS models} 

% Note that for the case when $m=2$, system~\eqref{eq:full} is monotone \cite[Lemma~3.3]{ye2021convergence} (and, assuming homogeneous recovery rates, also in \cite[Theorem~18]{santos2015bi}). That is, setting $m=2$ for system~\eqref{eq:full}, supposing $x(0)$ and $y(0)$ are initial states %of system~\eqref{eq:full}
% such that  $x(0) \leq y(0)$, then $x(t) \leq y(t)$ for all $t$. Given that for almost all choices of $D^i$, $B^i$, $i=1,2$, system~\eqref{eq:full} has a finite number of equilibria, \seb{then, because system~\eqref{eq:full} with $m=2$ is monotone,} it follows \seb{from \cite[Theorems~2.5 and~2.6]{smith1988systems}} that for almost all initial conditions in $\mathcal D$, system~\eqref{eq:full} with $m=2$ converges to a stable equilibrium point; the set of initial conditions for which said convergence does not occur (in which case it is on a nonattractive limit cycle) has measure zero \cite[Theorem~3.6]{ye2021convergence}. It is not known if the same can be said for $m=3$. Consequently, understanding the limiting behavior of tri-virus systems remains open.

\seb{Notice that if $m=2$, then, under Assumption~\ref{assum:irreducible}, system~\eqref{eq:full} is monotone; see \cite[Lemma~3.3]{ye2021convergence} (and, for $k=1,2$, assuming $\delta_i^k=\delta$ for $i \in [n]$, also  \cite[Theorem~18]{santos2015bi}). 
 That is, setting $m=2$ for system~\eqref{eq:full}, suppose that $(x_A^1(0), x_A^2(0))$ and $(x_B^1(0), x_B^2(0))$ are two initial conditions in $\textrm{int}(D)$ satisfying i) $x_A^1(0)>x_B^1(0)$ and ii) $x_A^2(0)<x_B^2(0)$. Since the bivirus system is monotone, it follows that, for all $t \in \mathbb{R}_{\geq 0}$, i) $x_A^1(t)\gg x_B^1(t)$ and ii) $x_A^2(t)\ll x_B^2(t)$. 
Independent of the fact that the bivirus system is monotone, it is known that %it has been shown that, 
for almost all \footnote{ \seb{The term \enquote{almost all} has a precise mathematical meaning: for all but a set of parameter values that has measure zero. This set of exceptional values is defined by an algebraic or semi-algebraic set.}} choices of $D^i$, $B^i$, $i=1,2$, system~\eqref{eq:full} has a finite number of equilibria \cite[Theorem~3.6]{ye2021convergence}. Therefore, from \cite[Theorems~2.5 and~2.6]{smith1988systems} (or \cite[Theorem~9.4]{hirsch1988stability}), we know that for almost all initial conditions in $\mathcal D$, system~\eqref{eq:full} with $m=2$ converges to a stable equilibrium point, assuming such an equilibrium  exists. There are initial conditions for which convergence to a stable equilibrium does not occur. In such cases, said initial condition is  either a)  already at an unstable equilibrium, assuming such an equilibrium exists, or b) in the stable manifold of an unstable equilibrium, 
or c) on a nonattractive limit cycle, or d) in the stable manifold of a nonattractive limit cycle). The set of initial conditions for which either a) or b) or c) or d) happens has measure zero.} %It is natural to ask whether an analogous claim holds for the tri-virus system.}

%The set of initial conditions for which said convergence does not occur (in which case it is  either a)  already at an unstable equilibrium, assuming such an equilibrium exists, or b) in the stable manifold of an unstable equilibrium, 
%or c) on a nonattractive limit cycle, or d) in the stable manifold of a nonattractive limit cycle) has measure zero. It is unknown if an analogous statement is true for the case when $m=3$. Consequently, understanding the limiting behavior of tri-virus systems remains partly open.}

\subsection{The tri-virus system is not monotone}
\par \seb{It is natural to ask how well the notion of MDS generalizes for multi-competitive networked SIS epidemics; this subsection aims to conclusively answer this question.}
%In order to answer the question of \seb{whether the tri-virus system is monotone} 
\seb{To this end,} we construct a graph associated with the Jacobian \eqref{jacob} of system~\eqref{eq:x1}-\eqref{eq:x3}, say $\bar{G}$. 
% The construction follows the outline provided in \cite{sontag2007monotone}. More specifically,
% the graph $\bar{G}$ has $3$ nodes; with  node~$k$, $k=1,2,3$, representing the sum of the fractions across all population nodes that is infected with virus~$k$.
% %this can be partitioned into groups of $n$ nodes each, thus resulting in $3$ groups. 
% The edges of  $\bar{G}$ are based on the entries in the Jacobian matrix $J(x^1, x^2, x^3)$. Note that the Jacobian $J(x^1, x^2, x^3)$ is a block matrix; blocks along the diagonal represent in-group interactions, whereas all other blocks represent inter-group interactions. If $[J(x^1, x^2, x^3)]_{ij} \leq 0$ for  $i \neq j$, hen we draw an edge labelled with "-" sign;  if  $[J(x^1, x^2, x^3)]_{ij} \geq 0$ for  $i \neq j$, then we draw an edge labelled with "+" sign. Thus,  $\bar{G}$ is a signed graph. Note that $\bar{G}$ has no self-loops. As an aside, also observe that since $x^k(t)\geq 0$ for $k \in [3]$ and $t\in \mathbb{R}_+$, it is immediate that the sign of the elements in $J(x^1, x^2, x^3)$ do not change with the argument.
The construction follows the outline provided in \cite{sontag2007monotone}. More specifically,
the graph $\bar{G}$ has $3n$ nodes. The edges of  $\bar{G}$ are based on the entries in the Jacobian matrix $J(x^1, x^2, x^3)$ in \eqref{jacob}. Specifically, if $[J(x^1, x^2, x^3)]_{ij} <0$ for  $i \neq j$, then we draw an edge labelled with ``-" sign;  if  $[J(x^1, x^2, x^3)]_{ij} > 0$ for  $i \neq j$, then we draw an edge labelled with ``+" sign. Thus,  $\bar{G}$ is a signed graph. Note that $\bar{G}$ has no self-loops. As an aside, also observe that since $x^k(t)\geq 0$ for $k \in [3]$ and $t\in \mathbb{R}_+$, it is immediate that the sign of the elements in $J(x^1, x^2, x^3)$ do not change with the argument, so that $\bar G$ %is independent 
is the same for all points in the interior of $\mathcal D$.  %{\color{red} I don't think a special notation like $\mathcal D^{\circ}$ has been proposed.}

\par We also need the following concept from graph theory. A signed graph is said to be consistent if every undirected cycle in the graph has a net positive sign, i.e., it has an even number of ``-" signs \cite{sontag2007monotone}. We have the following result.
\begin{thm} \label{prop:tri-virus-not-monotone}
System~\eqref{eq:x1}-\eqref{eq:x3} is not monotone.
\end{thm}
\textit{Proof:} Note that the Jacobian $J(x^1, x^2, x^3)$ is a block matrix, with all blocks along the off-diagonal being negative diagonal matrices. Pick any node $i$, where $i \in \{1,2, \hdots, n\}$. Observe that, since  all blocks along the off-diagonal of $J(x^1, x^2, x^3)$ are negative diagonal matrices, it is clear that there exists an edge from node $i$ to node $i+n$, an edge from node $i+n$ to node $i+2n$, and an edge from node $i+2n$ to node $i$. Furthermore, each of these edges have a ``-" sign. Hence, a loop starting from node $i$, traversing through nodes $i+n$, $i+2n$ and back to node $i$ is a 3-length cycle that has an odd number of negative signs. Therefore, from \cite[page 62]{sontag2007monotone}, the signed graph $\bar{G}$ is not consistent. 
Consequently, from \cite[page 63]{sontag2007monotone}, it follows that the system~\eqref{eq:x1}-\eqref{eq:x3} is not monotone.~\qed
% Since $[J(x^1, x^2, x^3)]_{ij} \leq 0$ for all $i \neq j$ with $i,j=1,2,3$, any (resp. every) edge from group $j$ to group $i$ has a "-" sign.  Note that a loop starting from group  $1$, traversing through groups  $2, 3$ and back to group $1$ is a 3-length cycle that has an odd number of negative signs. Hence, from \cite[page 62]{sontag2007monotone}, the signed graph $\bar{G}$ is not consistent. %which, due to \cite{harary1953notion}, further implies that $\bar{G}$ is not structurally balanced.
% Consequently, from \cite[page 63]{sontag2007monotone}, it follows that the system~\eqref{eq:x1}-\eqref{eq:x3} is not monotone.~$\blacksquare$

%drawn to represent the interactions \emph{between} the $3$ groups, and not within each group. That is, $\bar{G}$ has no self-loops.  
% Define $P^1:=\begin{bmatrix} I_n&&\textbf{0} && \textbf{0}\\
% \textbf{0}&& I_n&& \textbf{0}\\
% \textbf{0}&& \textbf{0} &&I_n 
% \end{bmatrix}$
% \seb{Nonetheless, due to the findings of Theorem~\ref{prop:tri-virus-not-monotone},
% %Even with a proof that the number of equilibria is finite for almost all choices of system parameters (i.e., healing rates and infection rates with respect to each of the three viruses),
% one cannot draw upon the rich literature on monotone dynamical systems (see\cite{smith1988systems}) to study the limiting behavior of system~\eqref{eq:x1}-\eqref{eq:x3}. In general, for non-monotone systems, no dynamical behavior, including chaos, can be definitively ruled out %without additional analysis
% \cite{sontag2007monotone}.}

\par %Theorem~\ref{prop:tri-virus-not-monotone} sheds light on a very interesting phenomenon, namely that the tri-virus system is not monotone. 
\seb{The conclusion of Theorem~1 offers several important and interesting insights, including key differences with bi-virus systems (m = 2).} %the bivirus system is known to be monotone, see \cite{ye2021convergence}.}
%The fact that the tri-virus system is not monotone is in sharp contrast to the bi-virus setting, which is known to be monotone \cite{ye2021convergence}. 
The fact that a bivirus system is monotone coupled with the fact that for almost all choices of $D^k$, $B^k$, $k=1,2$, the bivirus system has a finite number of equilibria allows one to draw general conclusions on the limiting behavior of bivirus dynamical systems. \seb{A consequence of Theorem~\ref{prop:tri-virus-not-monotone}, then, is that one cannot draw upon the rich literature on monotone dynamical systems (see\cite{smith1988systems}) to study the limiting behavior of system~\eqref{eq:x1}-\eqref{eq:x3}. In general, for non-monotone systems, no dynamical behavior, including chaos, can be definitively ruled out %without additional analysis
\cite{sontag2007monotone}.}

Another possible consequence of the lack of monotonicity is as follows: It is  known that setting $D^k=I$ for $k \in [2]$ has no bearing on either the location of equilibria of system~\eqref{eq:full}
with $m=2$ nor on their (local) stability properties \cite[Lemma~3.7]{ye2021convergence}. \seb{That is, consider two bivirus systems, namely $\mathcal S$ and $\hat{\mathcal {S}}$, where $\mathcal S$ is defined by $(B^1,D^1, B^2, D^2)$ and $\hat{\mathcal {S}}$ is defined by $(\hat{B}^1 =(D^1)^{-1}B^1, \hat{D}^1=I,  \hat{B}^2 =(D^2)^{-1}B^2, \hat{D}^2=I)$. Then, the location of equilibria are the same for both bivirus systems, and moreover, local stability of an equilibrium in  bivirus system $\mathcal S$ implies, and is implied by,  that in bivirus system $\hat{\mathcal {S}}$.}
%for a) bivirus systems, where $D^k=I$ and $B^k = (D^k)^{-1}B^k$ for $k \in [2]$, and b) bivirus systems, where $D^k$s are arbitrary positive diagonal matrices, the location of equilibria are the same for both bivirus systems a) and b). Plus, local stability of an equilibrium in  bivirus system a) implies, and is implied by,  that in bivirus system b).
%Note that Theorem~\ref{thm:init:condns} is reliant on the assumption that the healing rates for all agents with respect to all viruses are unity. 
For system~\eqref{eq:full} with $m=3$, by extending the arguments from \cite[Lemma~3.7]{ye2021convergence},  it is straightforward to show that the \emph{location} of the equilibria is the same when, for $k\in [3]$, $D^k=I$, %$B^k = (D^k)^{-1}B^k$,  
and when $D^k$ %(resp. $B^k$)
 are arbitrary positive diagonal %(resp. nonnegative) 
matrices with the $D^k$s not necessarily being equal to each other. However, since the tri-virus system is not monotone, the arguments for stability of equilibria in the proof of \cite[Lemma~3.7]{ye2021convergence} cannot be adapted. %\seb{especially for the 3-coexistence equilibria}. 
Hence, for the tri-virus case,  when the healing rates for all nodes with respect to all viruses are `scaled' in the manner above to become unity, preservation of stability properties remains an open question.

\seb{As a matter of independent interest, in the next section, we ask whether or not for almost all choices of $D^k$, $B^k$, $k=1,2,3$, the trivirus system has a finite number of equilibria. }



\section{Finiteness of equilibria for generic trivirus networks}\label{sec:finiteness:of:equiibria}
\brian{In this section, we argue that for generic tri-virus networks, i.e. for almost all  choices of $D^k,B^k, k=1,2,3$, the number of equiibria is finite. %\seb{The term \enquote{almost all} has a precise mathematical meaning: for all but a set of parameter values that has measure zero. This set of exceptional values is defined by an algebraic or semi-algebraic set.}
For  bi-virus networks, an argument for the corresponding result based on algebaic geometry ideas was provided in \cite[Theorem~3.6]{ye2021convergence}. That argument becomes much more intricate for tri-virus systems, and so an alternative proof is provided, based on a topological tool termed the Parametric Transversality Theorem, see \cite[p.145]{lee2013introduction} and \cite[p.68]{guillemin2010differential}.  This tool has the advantage that it will apply to other models of trivirus systems beyond those which require a specifically quadratic multinomial structure  of the  equilibrium equations, such as those arising where feedback control is present, \cite{ye2021_PH_TAC} or more refined models are constructed, see e.g. \cite{yang2017bi}.
% \seb{ By extending the algebraic geometry arguments in the proof of \cite[Theorem~3.6]{ye2021convergence}, it is relatively straightforward to show that even for the tri-virus system, for almost all choices of $D^k$, $B^k$, $k=1,2,3$, there exists a finite number of equilibria. Furthermore, by relying on arguments in \cite{anderson2022equilibria}, it can also be shown that for generic trivirus systems, the equilibria are hyperbolic.
%The details are omitted here in the interest of space. 
% Nonetheless, due to the findings of Theorem~\ref{prop:tri-virus-not-monotone},
% %Even with a proof that the number of equilibria is finite for almost all choices of system parameters (i.e., healing rates and infection rates with respect to each of the three viruses),
% one cannot draw upon the rich literature on monotone dynamical systems (see\cite{smith1988systems}) to study the limiting behavior of system~\eqref{eq:x1}-\eqref{eq:x3}. In general, for non-monotone systems, no dynamical behavior, including chaos, can be definitively ruled out %without additional analysis
% \cite{sontag2007monotone}.
\begin{prop}\label{prop:finite:equilibria}
For generic parameters $D^i$, $B^i$, with $i=1,2,3$, the trivirus system~\eqref{eq:x1}-\eqref{eq:x3} has a finite number of equilibria. %{\color{cyan} Deletable remark: Insert could come from BA somewhere near here about the equilibria being generically hyperbolic, but this would seem peripheral to the main ideas of the paper and so has not been attempted at this point. } 
\end{prop}

The proof of this proposition is contained in the appendix, using as it does very different tools to the rest of the paper. Further, we prove a slight modification, viz. that for any fixed set of $B^k$, almost all choices of the $D^k$ result in a finite number of equilibria. Proving that for fixed $D^k$ almost all choices of $B^k$ result in a finite number of equilibria is achieved with completely analogous calculations, but these calculations are not provided. Together, these two observations establish the proposition. Note that one \textit{cannot} replace the `almost all' requirement by an `all' requirement. As for the bivirus problem, there can exist special values of the parameters for which there is a continuum of equilibria; \seb{see Sections~\ref{sec:line:attractivity} and~\ref{sec:global:plane} for details}. }

\seb{One of the practical ramifications of Proposition~\ref{prop:finite:equilibria} is recorded in the following remark.
\begin{rem}
The fact that, for almost all choices of infection and recovery rates, the  tri-virus networked SIS model has a finite number of equilibria means that, even in the absence of interventions for controlling the spread, the range of outcomes that health administration officials need to prepare for are limited (although not necessarily few). 
\end{rem}
}


\section{Persistence of one or more viruses}\label{sec:persistence:one:virus}
\seb{In this section, we first show that for low-dimensional systems existence of certain kinds of equilibria can be ruled out for almost all choices of system parameters. Further, we identify a sufficient condition for local exponential convergence to a boundary equilibrium.  }

{\color{blue} %Brian insert 20220717
\subsection{The special case of low dimension systems}
When $n=1$ or $n=2$, certain types of equilibria are generically impossible, as we now show. %By `generically impossible', we mean for all but a set of parameter values of measure zero. This set of exceptional values is defined by an algebraic or semi-algebraic set. 

\begin{prop}
Consider system~\eqref{eq:x1}-\eqref{eq:x3} and suppose that $n=1$. Suppose that $(\tilde x^1,\tilde x^2,\tilde x^3)$ is an equilibrium.  Provided no two values of $(D^i)^{-1}B^i$ are identical, at most one  of the $\tilde x^i$ is zero at an equilibrium.
\end{prop}
%\begin{proof}
\textit{Proof:}
At any equilibrium, there holds (recall that the $\tilde x^i$ are scalar, so that $\tilde X^i=\tilde x^i)$
\[
(-D^i+(1-\tilde x^1-\tilde x^2-\tilde x^3)B^i)\tilde x^i=0
\]

 Note that if all $B^i$ are nonzero, there cannot hold $(1-\tilde x^1-\tilde x^2-\tilde x^3)=0,$
 else a contradiction is easily obtained. 

It is immediate then that either 
\[(1-\tilde x^1-\tilde x^2-\tilde x^3)^{-1}=(D^i)^{-1})B^i\quad\mbox{or}\quad\tilde x^i=0
\]
The first alternative corresponds to $\tilde x^i\neq 0$, and if, for example, $\tilde x^1\neq 0, \tilde x^2\neq 0$ then the parameters are constrained by $(D^1)^{-1}B^1=(D^2)^{-1}B^2$.
%\end{proof}

For the case $n=2$, we shall appeal to the following lemma.

\begin{lem} 
Consider three positive matrices $ B^i, i=1,2,3$ of dimension $2\times 2$. For generic values of the matrix entries, there cannot exist a diagonal matrix, $\Lambda$ say, such that each of $\Lambda B^i$ has a unity eigenvalue.
\end{lem}
%\begin{proof} 
\textit{Proof:}
Denote the diagonal entries of $\Lambda$ by $\lambda_1,\lambda_2$, The eigenvalue condition on $\Lambda B^i$ is det$(I-\Lambda B^i)=0$, or 
\[(\beta^i_{11}\beta^i_{22}-\beta^i_{12}\beta^i_{21})\lambda_1\lambda_2-\beta^i_{11}\lambda_1-\beta^i_{22}\lambda_2+1=0
\]
For fixed $B^i$, the set of $\lambda_i$ satisfying this equation is evidently defined by a  rectangular hyperbola. It is clear that for generic $B^i$, there cannot be a common point of intersection between three such hyperbola. Algebraic geometry gives methods for calculating the actual semialgebraic set of $B^i$ for which a common point of intersection exists, but this is irrelevant for our purposes. For almost all $B^i$, no such intersection exists. 
%\end{proof}

Now we can state

\begin{prop}
Consider system~\eqref{eq:x1}-\eqref{eq:x3} and suppose that $n=2$.  Suppose that $(\tilde x^1,\tilde x^2,\tilde x^3)$ is an equilibrium. Then for almost all values of the parameters $D^i,B^i$, at least one $\tilde x^i$ must be equal to $\textbf{0}$.
\end{prop}
\textit{Proof:}
%\begin{proof} 
The equilibrium equations yield
\[
(-D^i+(I-\tilde X^1-\tilde X^2-\tilde X^3)B^i)\tilde x^i=0
\]
Set $\Lambda=I-\tilde X^1-\tilde X^2-\tilde X^3$ and observe that the equilibrium equations can be rewritten as 
\[
\tilde x^i=\Lambda[(D^i)^{-1}B^i]\tilde x^i
\]
Hence either $\tilde x^i$ is an eigenvector of $\Lambda[(D^i)^{-1}B^i]$ or it is zero. The preceding lemma applies (with $(D^i)^{-1}B^i$ replacing $B^i$ in the lemma statement) and the conclusion of the proposition is immediate.
%\end{proof}

The proposition indicates that for generic $D^i,B^i$ and with $n=2$, the only equilibria of the trivirus system are the healthy equilibrium, those defined by the three associated single virus systems, and those defined by the three associated bivirus systems. Note that for $n=2$, all these equilibria are analytically computable, \cite{ye2021convergence}.
}


% If one or more of the eigenvalue conditions in Proposition~\ref{prop:phil} is violated, then at least one of the viruses persists in the population.  
% This, in turn, gives rise to a richer possible set of behaviors, as we will see in the rest of this paper. %Towards this end,we recall the following lemmas, %which will be required in the sequel.

%\subsection*{Preliminaries}
% \begin{lem}\cite[Lemma~7]{axel2020TAC} \label{lem:pos_never_zero}
% Let Assumption~\ref{assum:base} hold. Then $\mathcal{D} \setminus \{ \textbf{0} \}$ is positively invariant with respect to system~\eqref{eq:x1}-\eqref{eq:x3}.
% \end{lem}


%We have the following lemma that restricts the trajectories of the tri-virus system.



% The following lemma establishes certain properties of equilibria of system~\eqref{eq:x1}-\eqref{eq:x3}.
% \seb{Recall that Lemma~\ref{lem:inward:pointing}  substantially limits the equilibria that can lie on the boundary of the set $\{x^1, x^2, x^3 \in \mathbb{R}_{\geq 0}\mid x^1+x^2+x^3 \leq \textbf{1}\}$. }
% %An immediate consequence of Lemma~\ref{lem:inward:pointing} 
% \seb{A  consequence of Lemma~\ref{lem:inward:pointing}}
% is the following result, which has also been established using a different proof technique in \cite[Lemma~6]{axel2020TAC}.
\seb{It turns that we cannot have multiple 3-coexistence equilibria that differs only in one of their coordinates; we formalize the same in the following lemma.}
\begin{lem} \label{lem:equi_non-zero_nonone}
Consider system~\eqref{eq:x1}-\eqref{eq:x3} under Assumptions~\ref{assum:base} \seb{and~\ref{assum:irreducible}}. %Suppose that for each $i \in [n]$, $\sum_{j=1}^{n}B^1_{ij} >0$.%and~\ref{assum:irreducible}. %Suppose, for all $k \in [3]$, that $B^k$ is irreducible. 
%\begin{enumerate}[label=\roman*)]
   % \item \label{stmnt:1} If $x = (x^1, \dots, x^3) \in \mathcal{D}$ is an equilibrium of~\eqref{eq:x1}-\eqref{eq:x3}, then, for each $k \in [3]$, either $x^k = \textbf{0}$, or $\textbf{0} \ll x^k \ll \textbf{1}$. Moreover, %we have that 
%$\textstyle \sum_{k=1}^3 x^k \ll \textbf{1}$. 
    %\item \label{stmnt:2} 
If $(\bar{x}^1, \bar{x}^2, \bar{x}^3)$ and  $(\bar{x}^1, \bar{x}^2, \hat{x}^3)$ are endemic  equilibria (i.e., $\textbf{0}\ll (\bar{x}^1,\bar{x}^2, \bar{x}^3) \ll \textbf{1}$ and $\textbf{0}\ll  \hat{x}^3 \ll \textbf{1}$) of~\eqref{eq:x1}-\eqref{eq:x3}, then $\bar{x}^3 =\hat{x}^3$.
%\end{enumerate}
\end{lem}
\begin{proof} 
%Statement~\ref{stmnt:1} is an immediate consequence of Lemma~\ref{lem:inward:pointing}. As an aside, note that statement~\ref{stmnt:1} has also been established using a different proof technique in \cite[Lemma~6]{axel2020TAC}.\\
Suppose that $(\bar{x}^1, \bar{x}^2, \bar{x}^3)$ and  $(\bar{x}^1, \bar{x}^2, \hat{x}^3)$ are endemic equilibria of~\eqref{eq:x1}-\eqref{eq:x3}, then by writing the equilibrium version of equation~\eqref{eq:x1}, we obtain:
\begin{align}
    \textbf{0}&=-D^1\bar{x}^1+((I-\bar{X}^1-\bar{X}^2-\bar{X}^3))B^1\bar{x}^1 \label{eq:x1bar}\\
    \textbf{0}&=-D^1\bar{x}^1+((I-\bar{X}^1-\bar{X}^2-\hat{X}^3))B^1\bar{x}^1 \label{eq:x1hat}
\end{align}
From~\eqref{eq:x1bar} and~\eqref{eq:x1hat}, it is clear that
\begin{align}
    \bar{x}^1&=(D^1)^{-1}((I-\bar{X}^1-\bar{X}^2-\bar{X}^3))B^1\bar{x}^1\nonumber \\
    &=(D^1)^{-1}((I-\bar{X}^1-\bar{X}^2-\hat{X}^3))B^1\bar{x}^1, \nonumber
\end{align}
which, since i) $\bar{x}^1 \gg \textbf{0}$ and ii) by \seb{Assumption~\ref{assum:irreducible} it follows that}
%assumption, 
for each $i \in [n]$, $\sum_{j=1}^{n}B^1_{ij} >0$,
\seb{which} implies $\bar{X}^3=\hat{X}^3$, i.e., $\bar{x}^3=\hat{x}^3$.
\end{proof}
%$\blacksquare$



% It turns out that if each of the eigenvalue conditions in Proposition~\ref{prop:phil} were to be violated, then the tri-virus system has at least $4$ equilibria, as we recall in the following proposition.
% \begin{prop} \cite{liu2019analysis,axel2020TAC} \label{prop:necessity}
% Consider system~\eqref{eq:x1}-\eqref{eq:x3} under Assumption~\ref{assum:base}. For each $k \in [3]$ such that $B^k$ is irreducible and $s(B^k - D^k) > 0$, there is a unique single-virus endemic equilibrium $(\textbf{0}, \dots, \Tilde{x}^k, \dots, \textbf{0})$ in $\mathcal{D}$, with $\textbf{0} \ll \Tilde{x}^k \ll \textbf{1}$.%~$\blacksquare$
% \end{prop}
%

%We need the following lemmas in the sequel.
%\section{Preliminaries} \label{sect:prelims} %stability notions, results from Khalil etc.}
%Comment: Add relevant definitions and theorems on stability analysis from \cite{khalil2002nonlinear}.
%
%In this section, we recall some preliminary results, pertinent to the analysis of system~\eqref{eq:full}.
%A real square matrix is said to be Metzler if all elements outside the diagonal are nonnegative. We require the following results for Metzler matrices.
%
% \begin{lem} \label{lem:metz}
% \cite[Proposition~2]{rantzer2011distributed} Suppose that $M$ is a Metzler matrix such that $s(M) < 0$. Then, there exists a positive diagonal matrix $P$ such that $M^T P + P M \prec 0$.~$\blacksquare$ %\hfill  
% \end{lem}
%
% \begin{lem} \label{lem:metz_irreduc}
% \cite[Lemma A.1]{khanafer2016stability} Suppose that $M$ is an irreducible Metzler matrix such that $s(M) = 0$. Then there exists a positive diagonal matrix $P$ such that $M^T P + P M \preccurlyeq 0$.~$\blacksquare$ \end{lem}
%
% \begin{lem} \label{lem:eigspec}
% \cite[Proposition~1]{liu2019analysis} Suppose that $\Lambda$ is a negative diagonal matrix and $N$ is an irreducible nonnegative matrix. 
% %Let $M = \Lambda+N$.
% Let $M$ be the irreducible Metzler matrix $M = \Lambda+N$. 
% Then, $s(M) < 0$ if and only if $\rho(-\Lambda^{-1} N) < 1, s(M)=0$ if and only if $\rho(-\Lambda^{-1} N) = 1$, and $s(M)>0$ if and only if, $\rho(-\Lambda^{-1} N) > 1$.%~$\blacksquare$
% \end{lem}
% %
% We will also be making use of the following variants of the Perron-Frobenius theorem for irreducible matrices.

% \begin{lem} \label{lem:perron_frob}
% \cite[Chapter 8.3]{meyer2000matrix} \cite[Theorem~2.7]{varga1999matrix} %(Perron-Frobenius Theorem) 
% Suppose that $N$ is an irreducible nonnegative matrix. Then,
% %
% \begin{enumerate}[label=(\roman*)]
%     \item $r = \rho(N)$ is a simple eigenvalue of $N$. \label{item:perfrob_simpleeig}
%     \item There is an eigenvector $\zeta \gg \textbf{0}$ corresponding to the eigenvalue $r$. \label{item:perfrob_pos_exists}
%     \item $x > \textbf{0}$ is an eigenvector only if $Nx = rx$ and $x \gg \textbf{0}$. %\axel{\textbf{[Not sure this needs to be mentioned]}} 
%     \label{item:perfrob_pos_necess}
%     \item If $A$ is a nonnegative matrix such that $A < N$, then $\rho(A) < \rho(N)$. \label{item:perfrob_matrix_ineq}%~$\blacksquare$
% \end{enumerate}
% %
% \end{lem}
% %
% %Comment: Double-check whether we are indeed using all 4 items in Lemma~\ref{lem:perron_frob}]
% \begin{lem} \label{lem:perron_frob_metz}
% \cite[Lemma~2.3]{varga1999matrix} Suppose that $M$ is an irreducible Metzler matrix. Then $r = s(M)$ is a simple eigenvalue of $M$, and a corresponding eigenvector is $\zeta \gg \textbf{0}$.%~$\blacksquare$
% \end{lem}

\subsection{Stability of boundary equilibria}
In this subsection, 
%we identify a sufficient condition for local exponential convergence to a boundary equilibrium. While similar results exist for the case when $m=2$ (see \cite[Theorem~3.10]{ye2021convergence}), to the best of our knowledge, no such result exists for the $m=3$ case. The following theorem addresses this gap, and establishes 
we establish
that the local stability (resp. instability) of the boundary equilibrium corresponding to %virus~$i$ 
virus~1
is dependent on whether (or not) the state matrices obtained by linearizing the dynamics of viruses~$2$ and~$3$ around the single-virus endemic equilibrium of virus~$1$ are Hurwitz.

% corresponding to  %viruses~$j$ and~$k$, 
% viruses~$2$ and~$3$
% linearized around the single-virus endemic equilibrium of virus~$i$, is Hurwitz.

\begin{thm}\label{thm:local}
Consider system~\eqref{eq:x1}-\eqref{eq:x3} under Assumptions~\ref{assum:base} and~\ref{assum:irreducible}. %Suppose that for each $k \in [3]$  $B^k$ is irreducible. 
The boundary equilibrium $(\Tilde{x}^1, \textbf{0}, \textbf{0})$ is locally exponentially %asymptotically 
stable if, and only if, 
each of the following conditions are satisfied: 
\begin{enumerate}[label=\roman*)]
    \item $\rho((I-\tilde{X}^1)(D^2)^{-1}B^2)<1$; and
    \item $\rho((I-\tilde{X}^1)(D^3)^{-1}B^3)<1$.
\end{enumerate}
If $\rho((I{-}\tilde{X}^1)(D^2)^{-1}B^2)>1$ or if \mbox{$\rho((I{-}\tilde{X}^1)(D^3)^{{-}1}B^3)>1$}, then $(\Tilde{x}^1, \textbf{0}, \textbf{0})$ is unstable.
\end{thm}
The proof follows the strategy outlined for the $m=2$ case in~\cite[Theorem~3.10]{ye2021convergence}.\\
\textit{Proof:} Consider the equilibrium point $(\Tilde{x}^1, \textbf{0}, \textbf{0})$, and note that the Jacobian evaluated at this point is as follows:
\begin{align}
&J(\Tilde{x}^1,\textbf{0}, \textbf{0}) = \\
&\scriptsize
\begin{bmatrix}
-D^1+(I-\Tilde{X}^1)B^1-\hat{B}^1  & {-}\hat B^{1}   & {-}\hat B^{1} \\
 \textbf{0} & {-}D^2{+}(I{-}\Tilde{X}^1)B^2 & \textbf{0} \\
\textbf{0}  & \textbf{0} & {-}D^3{+}(I{-}\Tilde{X}^1)B^3 \end{bmatrix}\normalsize,\nonumber
\end{align}
where $\hat{B}^i=\diag(B^i\Tilde{x}^i)$, for $i=1,2,3$.\\
Observe that the matrix $J(\Tilde{x}^1,\textbf{0}, \textbf{0})$ is block upper triangular. Hence, it is Hurwitz if, and only if,  the blocks along the diagonal are Hurwitz. We will now show that this condition is fufilled, as a consequence of the assumptions of Theorem~\ref{thm:local}.
\par Since $(\Tilde{x}^1, \textbf{0}, \textbf{0})$ is an equilibrium point of system~\eqref{eq:x1}-\eqref{eq:x3}, by considering the equilibrium version of equation~\eqref{eq:x1}, we have the following:
\begin{align}\label{eq:eqm:version:1}
    (-D^1+(I-\Tilde{X}^1)B^1)\tilde{x}^1=\textbf{0}.
\end{align}
By Assumption~\ref{assum:base}, we have that $D^1$ is positive diagonal and $B^1$ is nonnegative. Furthermore, by assumption we know that $B^1$ is irreducible. Moreover, from \cite[Lemma~6]{axel2020TAC}, it follows that $(I-\Tilde{X}^1)$ is positive diagonal, implying that $(I-\Tilde{X}^1)B^1$ is nonnegative irreducible. Thus, we can conclude that the matrix $(-D^1+(I-\Tilde{X}^1)B^1)$ is irreducible Metzler. %SinceF
From \cite[Lemma~6]{axel2020TAC}
%Lemma~\ref{lem:equi_non-zero_nonone} 
we know that $\textbf{0} \ll \tilde{x}^1$. Hence, by applying \cite[Lemma~2.3]{varga1999matrix} to~\eqref{eq:eqm:version:1}, it must be that $\tilde{x}^1$ is, up to a scaling, the only eigenvector of $(-D^1+(I-\Tilde{X}^1)B^1)$ with all entries being strictly positive. Furthermore, $\tilde{x}^1$ is the eigenvector that is associated with, and only with, $s(-D^1+(I-\Tilde{X}^1)B^1)$. Therefore,   $s(-D^1+(I-\Tilde{X}^1)B^1)=0$.\\
Define $Q:=D^1-(I-\Tilde{X}^1)B^1$, and note that $Q$ is an M-matrix. Since $s(-Q)=0$ and $B^1$ is irreducible, it follows that $Q$ is a singular irreducible M-matrix. Observe that $\hat{B}^1$ is a nonnegative matrix, and because $B^1$ is irreducible and $\Tilde{x}^1 \gg \textbf{0}$, it must be that at least one element in $\hat{B}^1$ is strictly positive. Therefore, from \cite[Lemma~4.22]{qu2009cooperative}, it follows that $Q+\hat{B}^1$ is an irreducible non-singular M-matrix, which from \cite[Section~4.3, page~167]{qu2009cooperative} implies that $-Q-\hat{B}^1$ is Hurwitz. Therefore, we have that $s(-D^1+(I-\Tilde{X}^1)B^1-\hat{B}^1)<0$.
\par By assumption, $\rho((I-\tilde{X}^1)(D^2)^{-1}B^2)<1$ and $\rho((I-\tilde{X}^1)(D^3)^{-1}B^3)<1$. Therefore, by noting that $D^2$ (resp. $D^3$) are positive diagonal matrices and $B^2$ (resp. $B^3$) are nonnegative irreducible matrices,
from %Lemma~\ref{lem:eigspec}, 
\cite[Proposition~1]{liu2019analysis} it follows that $s(-D^2+(I-\tilde{X}^1)B^2)<0$ (resp. $s(-D^3+(I-\tilde{X}^1)B^3)<0$). Since each diagonal block of $J(\Tilde{x}^1,\textbf{0}, \textbf{0})$ is Hurwitz, it is clear that $J(\Tilde{x}^1,\textbf{0}, \textbf{0})$ is Hurwitz. Local %asymptotic 
exponential 
stability of $(\Tilde{x}^1,\textbf{0}, \textbf{0})$, then, follows from \cite[Theorem 4.15 and Corollary~4.3]{khalil2002nonlinear}. %\cite[Theorem~4.7, item i)]{khalil2002nonlinear}}.
\par The proof of necessity follows by first noting that if either condition in statement~i) or that in statement~ii) is violated, then, since at least one of the blcoks along the diagonal of $J(\Tilde{x}^1,\textbf{0}, \textbf{0})$ is not Hurwitz, the  the matrix $J(\Tilde{x}^1,\textbf{0}, \textbf{0})$ is not Hurwitz. Then, by invoking the necessity part of \cite[Theorem 4.15 and Corollary~4.3]{khalil2002nonlinear}, the result follows.
\par The claim for instability can be proved by noting that if either of the eigenvalue conditions are violated, then, since  $J(\Tilde{x}^1,\textbf{0}, \textbf{0})$ is block diagonal, the matrix  $J(\Tilde{x}^1,\textbf{0}, \textbf{0})$ is not Hurwitz. The result follows from \cite[Theorem~4.7, item ii)]{khalil2002nonlinear}.~\qed
\par Analogous results for the boundary equilibria $(\textbf{0}, \Tilde{x}^2, \textbf{0})$ and $(\textbf{0}, \textbf{0},\Tilde{x}^3)$ can be similarly obtained.


\section{Nonexistence of various coexistence equilibria}\label{sec:nonexistence:coexistence:equilibria}
\seb{In this section,  we establish conditions on the system parameters that guarantee nonexistence of different kinds of coexistence equilibria. Specifically, we identify a condition for the nonexistence of a 3-coexistence (resp. 2-coexistence) equilibria.



% \par \textit{Proof of statement~iii):} Since by assumption $B^2>B^1$, it follows that virus~2 eradicates virus~1. That is, there must exists some time $\tau \in \mathbb{R}_{\geq 0}$ such that for all $t \geq \tau$, $x^1(t)=0$.  This implies that for all $t \geq \tau$, the tri-virus system in~\eqref{eq:x1}-\eqref{eq:x3} boils down to the following bivirus system:
% \begin{align} 
%    %\dot{x}^1(t) &=  \Big{(} \big{(} I - (X^1+X^2+X^3) \big{)} B^1 - D^1 \Big{)} x^1(t), \label{eq:x1}\\
%       \dot{x}^2(t) &=  \Big{(} \big{(} I - (X^2+X^3) \big{)} B^2 - I \Big{)} x^2(t) \label{eq:x2bis}\\
%          \dot{x}^3(t) &=  \Big{(} \big{(} I - (X^2+X^3) \big{)} B^3 - I \Big{)} x^3(t) \label{eq:x3bis}
%   \end{align}
% The proof of statement~iii)  then follows from item ii) in \cite[Proposition~9]{ye2021convergence}}.

\seb{
\begin{thm}[Nonexistence of 3 cocoexistence equilibria]\label{claim:virus3:strongest}
Consider system~\eqref{eq:x1}-\eqref{eq:x3} under Assumption~\ref{assum:base}. Suppose that matrix $B^k$ for $k \in [3]$ are irreducible. Suppose that  $\rho(B^k)>1$ for $k \in [3]$. Let $\tilde{x}^1$,  $\tilde{x}^2$, and $\tilde{x}^3$ denote the single-virus endemic equilibrium corresponding to virus~1,~2, and~3, respectively. If $(D^3)^{-1}B^3>(D^1)^{-1}B^1$ and $(D^3)^{-1}B^3>(D^2)^{-1}B^2$, then
\begin{enumerate}[label=\roman*)]
\item the equilibrium point $(\textbf{0}, \textbf{0}, \textbf{0})$ is unstable;
\item the equilibrium point $(\tilde{x}^1, \textbf{0}, \textbf{0})$ is unstable;
\item the equilibrium point $(\textbf{0}, \tilde{x}^2, \textbf{0})$ is unstable;
\item the equilibrium point $(\textbf{0}, \textbf{0}, \tilde{x}^3)$ is locally exponentially stable;
\item there does not exist a 3-coexistence equilibrium.
%\item there does not exist any other equilibria
 \end{enumerate}
\end{thm}
\textit{Proof:} 
The proof strategy is quite similar to those in \cite[Theorem~6]{axel2020TAC} and \cite[Corollary~3.10]{ye2021convergence}.\\
%\par The proof of statements~i)-iii) are quite similar to that of statements~i) and ii) in Theorem~\ref{claim:virus1weaker}. 
\textcolor{black}{\textit{Proof of statement~i): }%By assumption, $D^1=D^2=D^3=I$. 
 Consider the equilibrium point $(\textbf{0}, \textbf{0}, \textbf{0})$, and observe that the Jacobian computed at this point is as follows: 
 \begin{align}
&J(\textbf{0},\textbf{0}, \textbf{0}) = \\
&\footnotesize
\begin{bmatrix}
-D^1+B^1  & \textbf{0}   & \textbf{0} \\
 \textbf{0} & -D^2+B^2 & \textbf{0} \\
\textbf{0}  & \textbf{0} & -D^3+B^3 \end{bmatrix}\normalsize,\nonumber
\end{align}
% Since $D^1=D^2=D^3=I$, the above can be rewritten as 
%  \begin{align}
% &J(\textbf{0},\textbf{0}, \textbf{0}) = \\
% &\footnotesize
% \begin{bmatrix}
% -I+B^1  & \textbf{0}   & \textbf{0} \\
%  \textbf{0} & -I+B^2 & \textbf{0} \\
% \textbf{0}  & \textbf{0} & -I+B^3 \end{bmatrix}\normalsize,\nonumber
% \end{align}
Since by assumption, $B^k$ is non-negative irreducible and $\rho(B^k)>1$ for $k \in [3]$, it follows from Lemma~\ref{lem:eigspec} that $s(-D^k+B^k)>0$ for $k \in [3]$, which further implies that $s(J(\textbf{0},\textbf{0}, \textbf{0}))>0$. Instability of the equilibrium point $(\textbf{0},\textbf{0}, \textbf{0})$, then, follows from \cite[Theorem~4.7, item ii)]{khalil2002nonlinear}.\\
\textit{Proof of statement~ii):}  Consider the equilibrium point $(\tilde{x}^1, \textbf{0}, \textbf{0})$, and observe that the Jacobian computed at this point is as follows:
\begin{align}
&J(\Tilde{x}^1,\textbf{0}, \textbf{0}) = \\
&\scriptsize
\begin{bmatrix}
-D^1+(I-\Tilde{X}^1)B^1-\hat{B}^1  & -\hat B^{1}   & -\hat B^{1} \\
 \textbf{0} & -D^2+(I-\Tilde{X}^1)B^2 & \textbf{0} \\
\textbf{0}  & \textbf{0} & -D^3+(I-\Tilde{X}^1)B^3 \end{bmatrix}\normalsize,\nonumber
\end{align}
Note that since $\tilde{x}^1$ is the single-virus endemic equilibrium for virus~1, by the definition of an equilibrium point, we have the following:
$$[-D^1+(I-\tilde{X}^1)B^1]\tilde{x}^1=\textbf{0}.$$
Observe that $-D^1+(I-\tilde{X}^1)B^1$ is an irreducible Metzler matrix, and, since $\tilde{x}^1\gg \textbf{0}$, from Lemma~\ref{lem:perron_frob_metz} we have that $s(-D^1+(I-\tilde{X}^1)B^1)=0$. Consequently, from Lemma~\ref{lem:eigspec}, $\rho((I-\tilde{X}^1)(D^1)^{-1}B^1)=1$. By assumption $(D^2)^{-1}B^2> (D^1)^{-1}B^1$, which implies $(I-\tilde{X}^1)(D^1)^{-1}B^1 < (I-\tilde{X}^1)(D^2)^{-1}B^2$. Therefore, since the matrices $(I-\tilde{X}^1)(D^1)^{-1}B^1$ and 
$(I-\tilde{X}^1)(D^2)^{-1}B^2$ are nonnegative, from Lemma~\ref{lem:perron_frob} item iv), we have that $\rho((I-\tilde{X}^1)(D^2)^{-1}B^2)>1$. Consequently, from Lemma~\ref{lem:eigspec}, it must be that $s(-D^2+(I-\tilde{X}^1)B^2)>0$, which implies that $s(J(\Tilde{x}^1,\textbf{0}, \textbf{0}))>0$, and hence the equilibrium point $(\Tilde{x}^1,\textbf{0}, \textbf{0})$ is unstable.\\
 The proof of statement~iii) is analogous to that of statement~ii), and is, therefore, omitted. 
}


\noindent \textit{Proof of statement~iv):} In order to prove statement iv), consider the equilibrium point $(\textbf{0}, \textbf{0}, \tilde{x}^3)$, and observe that the Jacobian is as follows:
\begin{align}
&J(\textbf{0}, \textbf{0}, \Tilde{x}^3) = \\
&\scriptsize
\begin{bmatrix}
-D^1+(I-\Tilde{X}^3)B^1  & \textbf{0}  & \textbf{0}\\
 \textbf{0} & -D^2+(I-\Tilde{X}^3)B^2 & \textbf{0} \\
-\hat B^{3}   & -\hat B^{3}    & -D^3+(I-\Tilde{X}^3)B^3-\hat B^{3}    \end{bmatrix}\normalsize,\nonumber
\end{align}
By following similar arguments as in the proof of Theorem~\ref{thm:local} (for establishing that the matrix $-D^1+(I-\Tilde{X}^1)B^1-\hat B^{1}$ is Hurwitz), we can show that the matrix $-D^3+(I-\Tilde{X}^3)B^3-\hat B^{3}$ is Hurwitz, i.e., $s(-D^3+(I-\Tilde{X}^3)B^3-\hat B^{3})<0$. Since $\tilde{x}^3$ is the single-virus endemic equilibrium corresponding to virus~3, we have that
$$\rho((I-\Tilde{X}^3)(D^3)^{-1}B^3)=1.$$
Since, by assumption, $(D^3)^{-1}B^3>(D^2)^{-1}B^2$ and $(D^3)^{-1}B^3>(D^1)^{-1}B^1$, it follows that a) $(I-\Tilde{X}^3)(D^3)^{-1}B^3> (I-\Tilde{X}^3)(D^2)^{-1}B^2$, and b) $(I-\Tilde{X}^3)(D^3)^{-1}B^3> (I-\Tilde{X}^3)(D^1)^{-1}B^1$. Therefore, from Lemma~\ref{lem:perron_frob} item iv), it must be that a) $\rho((I-\Tilde{X}^3)(D^2)^{-1}B^2)<1$, and b) $\rho((I-\Tilde{X}^3)(D^1)^{-1}B^1)<1$, which from Lemma~\ref{lem:eigspec} further implies that a) $s(-D^2+(I-\Tilde{X}^3)B^2)<0$, and b) $s(-D^3+(I-\Tilde{X}^3)B^1)<0$. As a consequence, $s(J(\textbf{0}, \textbf{0}, \Tilde{x}^3))<0$, which, from \cite[Theorem~4.7]{khalil2002nonlinear}, leads us to conclude that the equilibrium point $(\textbf{0}, \textbf{0}, \Tilde{x}^3)$ is locally exponentially stable. \\
\textit{Proof of statement~v):} 
%\par \textit{Proof of item v) (contd.):}
We now show that under the conditions of the Theorem a 3-coexistence equilibrium (i.e., of the form $\bar{x}=(\bar{x}^1, \bar{x}^2, \bar{x}^3)$) cannot exist. %there are exactly four equilibria, namely the healthy state $(\textbf{0}, \textbf{0}, \textbf{0})$, and the three boundary equilibria $\big((\Tilde{x}^1,\textbf{0}, \textbf{0}),(\textbf{0},\Tilde{x}^2, \textbf{0}), (\textbf{0}, \textbf{0}, \Tilde{x}^3)\big)$. We proceed as follows. 
%First observe that any other equilibrium $\bar{x}=(\bar{x}^1, \bar{x}^2, \bar{x}^3)$ should, in view of Lemma~\ref{lem:equi_non-zero_nonone}, fulfil either a) $\bar{x}^1+\bar{x}^2 \ll \textbf{1}$, or b) $\bar{x}^2+\bar{x}^3 \ll \textbf{1}$, or c) $\bar{x}^1+\bar{x}^3 \ll \textbf{1}$, or d) $\bar{x}^1+\bar{x}^2 + \bar{x}^3 \ll \textbf{1}$.
Suppose that, by way of contradiction, there exists a 3-coexistence %an endemic
equilibrium point $\bar{x}=(\bar{x}^1, \bar{x}^2, \bar{x}^3)$. Therefore, from Lemma~\ref{lem:equi_non-zero_nonone} it must be that $\textbf{0} \ll (\bar{x}^1, \bar{x}^2, \bar{x}^3) \ll \textbf{1}$, and $\bar{x}^1+ \bar{x}^2+ \bar{x}^3 \ll \textbf{1}$. Furthermore, by the definition of an equilibrium point, we also have the following:
\begin{align} 
   \textbf{0} &=  (-D^1 + (I-\bar X^1-\bar X^2-\bar X^3) B^1) \bar x^1, \label{eq:x1_3-equib}\\
      \textbf{0}  &=  (-D^2 + (I-\bar X^1-\bar X^2-\bar X^3) B^2) \bar x^2, \label{eq:x2_3-equib}\\
         \textbf{0}  &=  (-D^3 + (I-\bar X^1-\bar X^2-\bar X^3) B^3) \bar x^3 \label{eq:x3_3-equib}.
  \end{align}
Since $\tilde{x}^k$ is the single-virus endemic equilibrium for virus $k$, $k \in [3]$, we also have the following:
\begin{align} 
   \textbf{0} &=  \Big{(} -D^1+ \big{(} I - \Tilde{X}^1 \big{)} B^1  \Big{)} \tilde{x}^1(t), \label{eq:x1a:3-eqm}\\
    \textbf{0} &=  \Big{(}-D^2+ \big{(} I - \Tilde{X}^2\big{)} B^2 \Big{)} \tilde{x}^2(t) \label{eq:x2a:3-eqm}\\
       \textbf{0} &=  \Big{(}-D^3+ \big{(} I - \Tilde{X}^3\big{)} B^3 \Big{)} \tilde{x}^3(t) \label{eq:x3a:3-eqm}
  \end{align}
Define $\kappa:= \max_{i \in [n]}[\bar{x}^1+ \bar{x}^2+ \bar{x}^3 ]_i/\tilde{x}^3_i$, which implies that $\bar{x}^1+ \bar{x}^2+ \bar{x}^3 \leq \kappa \tilde{x}^3$. Let $j$ be the maximizing value of $i$ in the definition of $\kappa$. Assume by way of contradiction that $\kappa \geq 1$. Consider \tiny
\begin{align} 
   (\bar{x}^1_j+ \bar{x}^2_j+ \bar{x}^3_j)&=(1-\bar{x}^1_j-\bar{x}^2_j-\bar{x}^3_j)((D^1)^{-1}B^1\bar{x}^1+(D^2)^{-1}B^2\bar{x}^2+ (D^3)^{-1}B^3\bar{x}^3)_j\nonumber\\
    &< (1-\bar{x}^1_j-\bar{x}^2_j-\bar{x}^3_j)((D^3)^{-1}B^3\bar{x}^1+(D^3)^{-1}B^3\bar{x}^2+ (D^3)^{-1}B^3\bar{x}^3)_j \label{ineq:11bis}\\
    &= (1-\bar{x}^1_j-\bar{x}^2_j-\bar{x}^3_j)((D^3)^{-1}B^3(\bar{x}^1+\bar{x}^2+ \bar{x}^3))_j \nonumber  \\
    &\leq (1-\bar{x}^1_j-\bar{x}^2_j-\bar{x}^3_j)((D^3)^{-1}B^3\kappa\tilde{x}^3)_j  \label{ineq:22bis} \\
    &=\kappa (1-\bar{x}^1_j-\bar{x}^2_j-\bar{x}^3_j)(1-\tilde{x}^3_j)^{-1}\tilde{x}^3_j \label{ineq:33bis} \\
    &=(1-\bar{x}^1_j-\bar{x}^2_j-\bar{x}^3_j)(1-\tilde{x}^3_j)^{-1}(\bar{x}^1_j+ \bar{x}^2_j+ \bar{x}^3_j) \label{ineq:44bis}\\
    %&=(1-\tilde{x}_j^2)(1-\tilde{x}_j^3)^{-1}\tilde{x}_j^2 \label{eq:4bis}\\
    &<(\bar{x}^1_j+ \bar{x}^2_j+ \bar{x}^3_j), \label{ineq:55bis}
\end{align}
\normalsize
where~\eqref{ineq:11bis} follows from the assumption that $(D^3)^{-1}B^3>(D^2)^{-1}B^2$ and $(D^3)^{-1}B^3>(D^1)^{-1}B^1$ leads to $(D^3)^{-1}B^3y>(D^2)^{-1}B^2y$ and $(D^3)^{-1}B^3y>(D^1)^{-1}B^1y$ for any $y \gg \textbf{0}$. %and because $\textbf{0} \ll (\bar{x}^1, \bar{x}^2)$
The inequality~\eqref{ineq:22bis} is due to $(\bar{x}^1+\bar{x}^2+ \bar{x}^3) \leq \kappa \tilde{x}^3$;~\eqref{ineq:33bis} is immediate from~\eqref{eq:x3a:3-eqm}; and~\eqref{ineq:44bis} stems from the fact that $\bar{x}^1_j+\bar{x}^2_j+\bar{x}^3_j=\kappa \tilde{x}^3_j$. The inequality~\eqref{ineq:55bis} follows by noting that since by assumption $\kappa \geq 1$, it must be that $(1-\tilde{x}^3_j)>(1-\bar{x}^1_j- \bar{x}^2_j- \bar{x}^3_j)$. Observe that~\eqref{ineq:55bis} is a contradiction, thus implying that $\kappa < 1$. Thus, $\bar{x}^1+ \bar{x}^2+ \bar{x}^3\ll \tilde{x}^3$. %Now suppose that $\bar{x}^1+ \bar{x}^2+ \bar{x}^3 = \tilde{x}^3$.
\par Note that since $\tilde{x}^3 \gg \textbf{0}$, and $\bar{x}^3 \gg \textbf{0}$, it follows from ~\eqref{eq:x3_3-equib} and~~\eqref{eq:x3a:3-eqm} that $\rho((I-\bar X^1-\bar X^2-\bar X^3)(D^3)^{-1} B^3)=1$, and $\rho(\big{(} I - \Tilde{X}^3\big{)}(D^3)^{-1} B^3)=1$. Since $\bar{x}^1+ \bar{x}^2+ \bar{x}^3\ll \tilde{x}^3$, it must be that $(I-\bar{X}^1-\bar{X}^2-\bar{X}^3)> (I-\tilde{X}^3)$, which from Lemma~\ref{lem:perron_frob} item iv) implies that $\rho((I-\bar{X}^1-\bar{X}^2-\bar{X}^3)(D^3)^{-1}B^3)> \rho((I-\tilde{X}^3)(D^3)^{-1}B^3)$, which is a contradiction of the assumption that there exists a 3-coexistence equilibrium. Consequently, there does not exist a 3-coexistence equilibrium.~\qed} 
\par \textcolor{black}{Theorem~\ref{claim:virus3:strongest} identifies a sufficient condition for the non-existence of a 3-coexistence equilibrium. For the bivirus system, a  condition based on the row sum of matrices $B^k$, for $k \in [2]$, guarantees the non-existence of any 2-coexistence equilibria; see \cite[Corollary~3.10, item ii)]{ye2021convergence}. It is not known if an analogous condition (but for the nonexistence of any 3-coexistence equilibria) can be obtained for the tri-virus system; 
a rigorous investigation of the same is left for future work. 
It turns out that a condition, less restrictive   than that in Theorem~\ref{claim:virus3:strongest}, precludes the possibility of existence of   certain 2-coexistence equilibria, as formalized in the following theorem.}
\begin{thm}\label{claim:virus1weaker}
 Consider system~\eqref{eq:x1}-\eqref{eq:x3} under Assumptions~\ref{assum:base} and~\ref{assum:irreducible}. Suppose that  $\rho(B^k)>1$ for $k \in [3]$.
 \begin{enumerate}[label=\roman*)]
\item   The equilibrium point $(\textbf{0}, \textbf{0}, \textbf{0})$ is unstable.
\item If $(D^2)^{-1}B^2> (D^1)^{-1}B^1$ and $(D^3)^{-1}B^3> (D^1)^{-1}B^1$, then
 \begin{enumerate}[label=\alph*)]
        \item the equilibrium point $(\tilde{x}^1, \textbf{0}, \textbf{0})$ is unstable; and 
        \item there does not exist a 2-coexistence equilibrium of the form a) $(\bar{x}^1, \bar{x}^2, \textbf{0})$, or b) $(\bar{x}^1,  \textbf{0}, \bar{x}^3)$, where $\textbf{0}\ll \bar{x}^1 \ll \textbf{1}$, $\textbf{0}\ll \bar{x}^2 \ll \textbf{1}$, and $\textbf{0}\ll \bar{x}^3 \ll \textbf{1}$.
 \end{enumerate}
%\item Suppose that $D^1=D^2=D^3=I$. If $B^2> B^1$ and $B^2=B^3$, then there is  a line of 2-coexistence equilibria; in particular, there exists a unique $z \gg \textbf{0}$ such that all 2-coexistence equilibria of the form $(\textbf{0}, \tilde{x}^2, \tilde{x}^3)$ can be expressed as $(\textbf{0}, \alpha z, (1-\alpha)z)$ for all $\alpha \in [0,1]$.
 \end{enumerate}
\end{thm}
}
\seb{  \textit{Proof:} The proof of statement~i) (resp. statement~ii a)) is same as that of statement~i) (resp. statement~ii)) in Theorem~\ref{claim:virus3:strongest}.
% \textit{Proof of statement~i): }%By assumption, $D^1=D^2=D^3=I$. 
%  Consider the equilibrium point $(\textbf{0}, \textbf{0}, \textbf{0})$, and observe that the Jacobian computed at this point is as follows: 
%  \begin{align}
% &J(\textbf{0},\textbf{0}, \textbf{0}) = \\
% &\footnotesize
% \begin{bmatrix}
% -D^1+B^1  & \textbf{0}   & \textbf{0} \\
%  \textbf{0} & -D^2+B^2 & \textbf{0} \\
% \textbf{0}  & \textbf{0} & -D^3+B^3 \end{bmatrix}\normalsize,\nonumber
% \end{align}
% % Since $D^1=D^2=D^3=I$, the above can be rewritten as 
% %  \begin{align}
% % &J(\textbf{0},\textbf{0}, \textbf{0}) = \\
% % &\footnotesize
% % \begin{bmatrix}
% % -I+B^1  & \textbf{0}   & \textbf{0} \\
% %  \textbf{0} & -I+B^2 & \textbf{0} \\
% % \textbf{0}  & \textbf{0} & -I+B^3 \end{bmatrix}\normalsize,\nonumber
% % \end{align}
% Since by assumption, $B^k$ is non-negative irreducible and $\rho(B^k)>1$ for $k \in [3]$, it follows from Lemma~\ref{lem:eigspec} that $s(-D^k+B^k)>0$ for $k \in [3]$, which further implies that $s(J(\textbf{0},\textbf{0}, \textbf{0}))>0$. Instability of the equilibrium point $(\textbf{0},\textbf{0}, \textbf{0})$, then, follows from \cite[Theorem~4.7, item ii)]{khalil2002nonlinear}. 
% \par \textit{Proof of statement~ii) a):}  Consider the equilibrium point $(\tilde{x}^1, \textbf{0}, \textbf{0})$, and observe that the Jacobian computed at this point is as follows:
% \begin{align}
% &J(\Tilde{x}^1,\textbf{0}, \textbf{0}) = \\
% &\scriptsize
% \begin{bmatrix}
% -D^1+(I-\Tilde{X}^1)B^1-\hat{B}^1  & -\hat B^{1}   & -\hat B^{1} \\
%  \textbf{0} & -D^2+(I-\Tilde{X}^1)B^2 & \textbf{0} \\
% \textbf{0}  & \textbf{0} & -D^3+(I-\Tilde{X}^1)B^3 \end{bmatrix}\normalsize,\nonumber
% \end{align}
% Note that since $\tilde{x}^1$ is the single-virus endemic equilibrium for virus~1, by the definition of an equilibrium point, we have the following:
% $$[-D^1+(I-\tilde{X}^1)B^1]\tilde{x}^1=\textbf{0}.$$
% Observe that $-D^1+(I-\tilde{X}^1)B^1$ is an irreducible Metzler matrix, and, since $\tilde{x}^1\gg \textbf{0}$, from Lemma~\ref{lem:perron_frob_metz} we have that $s(-D^1+(I-\tilde{X}^1)B^1)=0$. Consequently, from Lemma~\ref{lem:eigspec}, $\rho((I-\tilde{X}^1)(D^1)^{-1}B^1)=1$. By assumption $(D^2)^{-1}B^2> (D^1)^{-1}B^1$, which implies $(I-\tilde{X}^1)(D^1)^{-1}B^1 < (I-\tilde{X}^1)(D^2)^{-1}B^2$. Therefore, since the matrices $(I-\tilde{X}^1)(D^1)^{-1}B^1$ and 
% $(I-\tilde{X}^1)(D^2)^{-1}B^2$ are nonnegative, from Lemma~\ref{lem:perron_frob} item iv), we have that $\rho((I-\tilde{X}^1)(D^2)^{-1}B^2)>1$. Consequently, from Lemma~\ref{lem:eigspec}, it must be that $s(-D^2+(I-\tilde{X}^1)B^2)>0$, which implies that $s(J(\Tilde{x}^1,\textbf{0}, \textbf{0}))>0$, and hence the equilibrium point $(\Tilde{x}^1,\textbf{0}, \textbf{0})$ is unstable.
\par \textit{Proof of statement~ii) b):} We now show that, under the conditions of the Theorem,  a 2-coexistence equilibrium of the form $(\bar{x}^1, \bar{x}^2, \textbf{0})$, where $\textbf{0} \ll \bar{x}^1 \ll \textbf{1}$ and $\textbf{0} \ll \bar{x}^2 \ll \textbf{1}$,  cannot exist. Suppose that, by way of contradiction, there exists a 2-coexistence equilibrium of the form $(\bar{x}^1, \bar{x}^2, \textbf{0})$ with $\textbf{0} \ll \bar{x}^1 \ll \textbf{1}$ and $\textbf{0} \ll \bar{x}^2 \ll \textbf{1}$. By taking recourse to the equilibrium version of equations~\eqref{eq:x1}-\eqref{eq:x3}, we have the following:
\begin{align} 
   \textbf{0} &=  \Big{(} \big{(} I - (\bar{X}^1+\bar{X}^2) \big{)} B^1 - D^1 \Big{)} \bar{x}^1, \label{eq:x1a:eqm-1-bar}\\
    \textbf{0} &=  \Big{(} \big{(} I - (\bar{X}^1+\bar{X}^2) \big{)} B^2 - D^2 \Big{)} \bar{x}^2,\label{eq:x2a:eqm-2-bar} %\\
    %  \textbf{0} &=  \Big{(} \big{(} I - (\Tilde{X}^1+\Tilde{X}^2+\Tilde{X}^3)\big{)} B^3 - D^3 \Big{)} \tilde{x}^3(t) \label{eq:x3a:eqm-3}
  \end{align}
By a suitable rearrangement of terms in~\eqref{eq:x1a:eqm-1-bar} and~\eqref{eq:x2a:eqm-2-bar}, and by noting that the matrices $I - (\bar{X}^1+\bar{X}^2) \big{)} B^1 $ and $I - (\bar{X}^1+\bar{X}^2) \big{)} B^2 $ are irreducible, from Lemma~\ref{lem:perron_frob} item i), we have the following:
\begin{align}
    \rho(I - (\bar{X}^1+\bar{X}^2) \big{)}(D^1)^{-1} B^1 )&=1 \label{rho1}\\
    \rho(I - (\bar{X}^1+\bar{X}^2) \big{)}(D^2)^{-1} B^2 )&=1 \label{rho2}
\end{align}

% $$\rho(I - (\bar{X}^1+\bar{X}^2) \big{)} B^1 )=1;$$ 
% $$\rho(I - (\bar{X}^1+\bar{X}^2) \big{)} B^2 )=1.$$ 
Since by assumption $(D^2)^{-1}B^2>(D^1)^{-1}B^1$, it must be that $(I - (\bar{X}^1+\bar{X}^2) \big{)}(D^2)^{-1} B^2> (I - (\bar{X}^1+\bar{X}^2) \big{)}(D^1)^{-1} B^1$, which from Lemma~\ref{lem:perron_frob} item iv) further implies that $\rho((I - (\bar{X}^1+\bar{X}^2) \big{)}(D^2)^{-1} B^2)> \rho((I - (\bar{X}^1+\bar{X}^2) \big{)}(D^1)^{-1} B^1)$, which contradicts with the conclusions from~\eqref{rho1} and~\eqref{rho2}. Thus, there cannot exist a 2-coexistence equilibrium  of the form  $(\bar{x}^1, \bar{x}^2, \textbf{0})$ with $\textbf{0} \ll \bar{x}^1 \ll \textbf{1}$ and $\textbf{0} \ll \bar{x}^2 \ll \textbf{1}$. The nonexistence of a 2-coexistence equilibrium of the form $(\bar{x}^1,  \textbf{0}, \bar{x}^3)$ with $\textbf{0} \ll \bar{x}^1 \ll \textbf{1}$ and $\textbf{0} \ll \bar{x}^3 \ll \textbf{1}$ can be shown analogously, by leveraging the assumption that $(D^3)^{-1}B^3> (D^1)^{-1}B^1$.}

Note that Theorem~\ref{claim:virus1weaker}  conclusively rules out the existence of a 2-coexistence equilibrium of the form $(\bar{x}^1, \bar{x}^2, \textbf{0})$, and of the form $(\bar{x}^1, \textbf{0}, \bar{x}^3)$. It, however, does not preclude the possibility of existence of 2-coexistence equilibria of the form $(\textbf{0}, \bar{x}^2, \bar{x}^3)$. Therefore, Theorem~\ref{claim:virus1weaker}  is not a necessary condition for the existence of a 2-coexistence equilibrium in a tri-virus system.
\par Observe that the condition in Theorem~\ref{claim:virus3:strongest} implies (but is not implied by) the condition in Theorem~\ref{claim:virus1weaker}, and, consequently, Theorem~\ref{claim:virus3:strongest} also rules out he existence of a 2-coexistence equilibrium of the form $(\bar{x}^1, \bar{x}^2, \textbf{0})$, and of the form $(\bar{x}^1, \textbf{0}, \bar{x}^3)$.  

\section{Existence of coexistence 
%coexistence 
equilibria}\label{sec:existence:coexistence}
Section~\ref{sec:nonexistence:coexistence:equilibria} focuses on the nonexistence of certain equilibria. In this section, we provide conditions for the existence of a 2-coexistence equilibrium, and identify circumstances under which the existence of one or more 3-coexistence equilibria can be inferred. \seb{Define, for each $i \in [3]$, $\bar X^i: =\diag{(\bar{x}^i)}$, where $\bar x^i$ is as defined in Section~\ref{sec:prob:formulation}.}
The following proposition guarantees the existence of a 2-coexistence equilibrium.
\seb{
\begin{prop}\label{prop:2-coexistence}
Consider system~\eqref{eq:x1}-\eqref{eq:x3} under Assumptions~\ref{assum:base} and~\ref{assum:irreducible}. %Suppose that matrix $B^k$ for $k \in [3]$ are irreducible. 
Suppose that  $\rho(B^k)>1$ for $k \in [3]$. 
%Let $\tilde{x}^1$,  $\tilde{x}^2$, and $\tilde{x}^3$ denote the single-virus endemic equilibrium corresponding to virus~1,~2, and~3, respectively. 
There exists at least one 2-coexistence equilibrium, \seb{i.e., $(\bar{x}^1, \bar{x}^2, \bar{x}^3)$ where 
$\textbf{0}\ll \bar{x}^i, \bar{x}^j \ll \textbf{1}$
 for some $i, j \in [3], i\neq j$, 
%being strictly positive vectors 
and, for $\ell \neq i, \ell \neq j$, $\bar{x}^\ell=\textbf{0}$}. %Moreover, if $s(-D^3+(I-\bar{X}^1-\bar{X}^2))B^3>0$, $s(-D^2+(I-\bar{X}^1-\bar{X}^3))B^2>0$, and $s(-D^1+(I-\bar{X}^2-\bar{X}^3))B^1>0$, then %any
%every 2-coexistence equilibrium is unstable.
\end{prop}
\textit{Proof:} By assumption, we have that $\rho(B^k)>1$ for $k \in [3]$. Therefore, from Proposition~\ref{prop:necessity} it follows that there exists single-virus endemic equilibrium (also refereed to as boundary equilibria) corresponding to virus~1,~2, and~3, namely $\tilde{x}^1$,  $\tilde{x}^2$, and $\tilde{x}^3$, respectively. Observe that each of these equilibria could be either stable or unstable, which implies that there must be either i) a pair which are both stable, or ii) a pair which are both unstable. We consider the two cases separately.\\
\textit{Case i):} Assume, without loss of generality, that the two stable boundary equilibria are $(\tilde x^1,{\bf{0}},{\bf{0}})$ and $({\bf{0}},\tilde x^2,{\bf{0}})$. Observe that for the bivirus system, defined by neglecting $x^3$, the boundary equilibria are $(\tilde x^1,{\bf{0}})$ and $({\bf{0}},\tilde x^2)$; the existence of which, to reiterate, is guaranteed from the assumption that $\rho(B^k)>1$ for $k=1,2$. It is known that the bivirus system is monotone \cite[Lemma~3.3]{ye2021convergence}. Therefore, from \cite[Theorem~2.8]{smith1988systems} and \cite[Proposition~2.9]{smith1988systems}, it follows that there is necessarily an unstable equilibrium $(\bar x^1,\bar x^2)$, where $\textbf{0} \ll (\bar x^1,\bar x^2) \ll \textbf{1}$.\\
\textit{Case ii):} Assume, without loss of generality, that the two unstable boundary equilibria are $(\tilde x^1,{\bf{0}},{\bf{0}})$ and $({\bf{0}},\tilde x^2,{\bf{0}})$. Since the bivirus system (defined by neglecting $x^3$) is monotone, and since for monotone systems between any two unstable boundary equilibria there must  exist a stable equilibria \cite{smith1988systems}, it follows that there must be an equilibrium point $(\bar x^1,\bar x^2)$. Moreover, since from \cite[Lemma~3.1]{ye2021convergence}, it is known that, for the bivirus system, besides the DFE, only the equilibrium points $(\tilde x^1,{\bf{0}})$ and $({\bf{0}},\tilde x^2)$ can be on the boundary of the surface defined by the set $\mathcal D^k$, $k=1,2$, it follows that the equilibrium point $(\bar x^1,\bar x^2)$ must lie in the interior of $\mathcal D^k$, $k=1,2$, which further implies that $\textbf{0} \ll (\bar x^1,\bar x^2) \ll \textbf{1}$. Hence, it follows that in either case there exists a 2-coexistence equilibrium %namely $(\bar x^1,\bar x^2,{\bf{0}})$, 
for system~\eqref{eq:x1}-\eqref{eq:x3}.
% Let us assume that the  2-coexistence equilbrium that exists is $(\bar x^1,\bar x^2,{\bf{0}})$. 
% By assumption, $s(-D^3+(I-\bar{X}^1-\bar{X}^2)B^3)>0$. Then by evaluating the Jacobian at $(\bar x^1,\bar x^2,{\bf{0}})$, and observing that $J(\bar x^1,\bar x^2,{\bf{0}})$ is block upper triangular, with $-D^3+(I-\bar{X}^1-\bar{X}^2))B^3$  being one of the blocks along the diagonal. Since $s(-D^3+(I-\bar{X}^1-\bar{X}^2)B^3)>0$, it follows that $J(\bar x^1,\bar x^2,{\bf{0}})$ is not Hurwitz, and, consequently, $(\bar x^1,\bar x^2,{\bf{0}})$ is unstable. Observe  that if the 2 coexistence equilibrium that exists is $(\bar x^1,{\bf{0}},\bar x^3)$ (resp. $({\bf{0}}, \bar x^2,,\bar x^3)$) then since by assumption $s(-D^2+(I-\bar{X}^1-\bar{X}^3)B^2)>0$ (resp. $s(-D^1+(I-\bar{X}^2-\bar{X}^3)B^1)>0$), by reasoning analogous to the above, it follows that $(\bar x^1,{\bf{0}},\bar x^3)$ (resp. $({\bf{0}}, \bar x^2,,\bar x^3)$) is unstable.
}
~\hfill \qed
{\color{red} 
%Comment from BA: Here is an outline of a possible inclusion. It could be set up as a proposition. 

% While this section focuses on the nonexistence of certain equilibria, we finish it by recording a further result on the existence as opposed to nonexistence. 
% Suppose $n\geq 2$. Since there are precisely three boundary equilibria, each stable or unstable, there must be a pair which are both stable, or a pair which are both unstable. In the former case, Denote the equilibria by $(\tilde x^1,{\bf{0}},{\bf{0}})$ and $({\bf{0}},\tilde x^2,{\bf{0}})$. For the bivirus system defined by neglecting $x^3$, there will be boundary equilibria $(\tilde x^1,{\bf{0}})$ and $({\bf{0}},\tilde x^2)$. In the first case, there is necessarily an unstable equilibrium $(\bar x^1,\bar x^2)$, because of the monotone nature of the system, and the requirement that ``between" any two stable equilibria, there is an unstable one, \cite{smith1988systems}. In the second case there is necessarily a stable equilibrium $(\bar x^1,\bar x^2)$ since such a equilibrium must exist in a bivirus system, again because of the monotone property, and being precluded from existing on the boundary it is necessarily a coexistence equilibrium In either case, $(\bar x^1,\bar x^2,{\bf{0}})$ becomes and equilibrium of the 3-virus system. 


\subsection*{Commentary on Hofbauer paper}

{\textit{I am not sure where this should go. It probably should be set up as a proposition or theorem. It indicates circumstances under which the existence of one or more 3-coexistent equilibria can be inferred. }}

It is possible to obtain a ``counting" or index-based result which for certain patterns of stability of boundary and 2-coexistence equilibria could imply the existence of one or more 3-coexistence equilibria. The key is to use a result available for systems which are invariant in $\mathbb R^n_{+}$ with trajectories which are ultimately bounded, see \cite{hofbauer1990index}. Obviously, these restrictions are met by a trivirus system. 

First, equilibiria have to be classified as saturated, or otherwise. Then the index properties of the saturated equilibria are considered. 

In more detail, observe that for a boundary equilibrium or a coexistence equilibrium, the associated Jacobian is block triangular. If the diagonal block corresponding to the zero entries of the equilibrium is Hurwitz, the equilibrium is said to be saturated, and for a 3-coexistence equilibrium, with no diagonal block, it is also said to be saturated. Note that for a boundary equilibrium, the equilibrium is saturated if and only if the equilibrium is stable. However, a 2-coexistence equilibrium might be unstable, but still saturated. 

Following more or less standard practice,  the index of an equilibrium is defined as the sign, i.e. $\pm 1$, of the negative of the determinant of the Jacobian at the point. (It is assumed that all such Jacobians are nonsingular, which for generic parameter values is the case). 

The key results are that (a) if a solution beginning in the interior goes to a limit as $t\to\infty$, that limit must be a saturated fixed point, and (b) the sum of the indices {\textit{over saturated equilibria}} is necessarily $+1$. Observe then that if the sum of the indices of the saturated boundary and saturated 2-coexistent equilibria is other than $+1$, there is necesarily a nonzero number of 3-coexistent equilibria.  
}

\ben{Ben comment: Providing a comprehensive study of the tri-virus system using \cite{hofbauer1990index} would be really complex, mainly because we need a lot of notation and auxiliary definitions. One simple sufficient condition (although one could argue both ways as to whether it is easy to check) is as follows: 

For any 2-coexistence equilibrium with $\bar x^i, \bar x^j > \textbf{0}$ and $\bar x^k = 0$, we say that it is saturated if and only if $s(-D^k + (I-\bar X^i-\bar X^j)B^k) < 0$. 

Theorem: There must always be at least one of i) a stable boundary equilibrium, or ii) a saturated 2-coexistence equilibrium, or iii) a 3-coexistence equilibrium.\\ 
\seb{
\begin{thm}\label{thm:hofbauer}
 Consider system~\eqref{eq:x1}-\eqref{eq:x3} under Assumption~\ref{assum:base}. Suppose that, for each $k \in [3]$, $\rho(B^k)>1$.
There exists either a stable boundary equilibrium, or a saturated 2-coexistence equilibrium, or a 3-coexistence equilibrium.   
\end{thm}

% \textit{Proof sketch:} The proof should follow from \cite[Theorem~2]{hofbauer1990index}, and by noting that an interior fixed point is always saturated.
% \\
\textit{Proof:} We first show that the assumptions of the Theorem guarantee the existence of boundary equlibria and at least one 2-coexistence equilibria.\\
By assumption, we have that $\rho(B^k)>1$ for $k \in [3]$. Therefore, from Proposition~\ref{prop:necessity} it follows that there exists single-virus endemic equilibrium (also refereed to as boundary equilibria) corresponding to virus~1,~2, and~3, namely $\tilde{x}^1$,  $\tilde{x}^2$, and $\tilde{x}^3$, respectively. Moreover, from Proposition~\ref{prop:2-coexistence} the existence of the said boundary equilibria guarantees the existence of at least one 2-coexistence equilibrium.\\
Observe that Lemma~\ref{lem:pos} guarantees that, for each $k \in [3]$, $x^k(0)>\textbf{0}$ implies that $x^k(t)>\textbf{0}$ for all $t \in \mathbb{R}_{\geq 0}$, and that the set $\mathcal D$ (which is compact) is forward invariant. Therefore, from \cite[Theorem~2]{hofbauer1990index}, it follows that system~\eqref{eq:x1}-\eqref{eq:x3} has at least one saturated fixed point. There are two cases to consider.\\
Case 1: Suppose that the aforementioned saturated fixed point is in the interior of $\mathcal D$. 
 Note that any fixed point in the interior of $\mathcal D$ is, due to Lemma~\ref{lem:equi_non-zero_nonone:1}, of the form $(\bar{x}^1, \bar{x}^2, \bar{x}^3)$, where $\textbf{0} \ll (\bar{x}^1, \bar{x}^2, \bar{x}^3) \ll \textbf{1}$, thus implying that $(\bar{x}^1, \bar{x}^2, \bar{x}^3)$ is a 3-coexistence equilibrium. \\%Since any fixed point in the interior of  $\mathcal D$ is saturated, it follows that, if a 3-coexistence equilibrium exists it must be saturated.
Case 2: Suppose that there are no fixed points  in the interior of $\mathcal D$. This implies that there should be a saturated fixed point on the boundary of $\mathcal D$  \cite{hofbauer1990index}. Therefore,  at least one of the following should be true: a) at least one of the single-virus boundary equilibrium is saturated; or b) at least of the 2-coexistence equilibria 
%(whose existence is guaranteed by the assumptions of the theorem and Proposition~\ref{prop:2-coexistence}) 
is saturated.~\hfill\qed
}\\
\seb{The following is an immediate consequence of Theorem~\ref{thm:hofbauer}, and \cite[Theorem~1]{hofbauer1990index}.
\begin{cor}\label{cor:hofbauer}
 Consider system~\eqref{eq:x1}-\eqref{eq:x3} under Assumption~\ref{assum:base}. Suppose that, for each $k \in [3]$, $\rho(B^k)>1$. Let $(\tilde{x}^1,\textbf{0}, \textbf{0})$,  $(\textbf{0}, \tilde{x}^2,\textbf{0})$ and  $(\textbf{0}, \textbf{0},\tilde{x}^3)$ denote the boundary equilibria corresponding to viruses~1,2 and 3, respectively. Let $(\bar{x}^i, \bar{x}^j, \bar{x}^k)$ such that for some $i, j \in [3]$, $\bar{x}^i, \bar{x}^j>0$, and for some $k \in [3]$, $\bar{x}^k=\textbf{0}$.
Suppose that, for each pair $(i,j)$ such that $i, j \in [3]$ and $i \neq j$, $\rho((I{-}\tilde{X}^i)(D^j)^{-1}B^j)>1$. Suppose that, for every such triplet $(i,j,k)$, $s(-D^k + (I-\bar X^i-\bar X^j)B^k) \geq 0$. Then, there exists at least one 3-coexistence equilibrium and there exists a trajectory from the interior of $\mathcal{D}$ that reaches it. In fact, there must be an odd number of 3-coexistence equilibrium.
\end{cor}
\textit{Proof:} By assumption, we have that $\rho(B^k)>1$ for $k \in [3]$. Therefore, from Proposition~\ref{prop:necessity} it follows that there exists  refereed to as boundary equilibria corresponding to virus~1,~2, and~3, namely $\tilde{x}^1$,  $\tilde{x}^2$, and $\tilde{x}^3$, respectively. By assumption, for each pair $(i,j)$ such that $i, j \in [3]$ and $i \neq j$, $\rho((I{-}\tilde{X}^i)(D^j)^{-1}B^j)>1$. Therefore, from Theorem~\ref{thm:local}, each of the boundary equilibrium points, $(\tilde{x}^1,\textbf{0}, \textbf{0})$,  $(\textbf{0}, \tilde{x}^2,\textbf{0})$ and  $(\textbf{0}, \textbf{0},\tilde{x}^3)$, are unstable. Given the existence of the boundary equilibria, from Proposition~\ref{prop:2-coexistence}, the existence of at least one equilibrium (i.e., 2-coexistence equilibrium), $(\bar{x}^i, \bar{x}^j, \bar{x}^k)$ such that for some $i, j \in [3]$, $\bar{x}^i, \bar{x}^j>0$, and for some $k \in [3]$, $\bar{x}^k=\textbf{0}$, is guaranteed. By assumption, for every such triplet $(i,j,k)$, $s(-D^k + (I-\bar X^i-\bar X^j)B^k) \geq 0$, which implies that no 2-coexistence equilibrium is saturated. Therefore, from Theorem~\ref{thm:hofbauer}, it must be that there exists an equilibrium point $(\bar{x}^i, \bar{x}^j, \bar{x}^k)$ such that   $\bar{x}^i, \bar{x}^j, \bar{x}^k>\textbf{0}$, i.e., a 3-coexistence equilibrium. Since any equilibrium point $(\bar{x}^i, \bar{x}^j, \bar{x}^k)$ such that   $\bar{x}^i, \bar{x}^j, \bar{x}^k>\textbf{0}$ must necessarily be in the interior of $\mathcal D$, it must be that said equilbrium point is saturated \cite{hofbauer1990index}. Since the equilibrium point $(\bar{x}^i, \bar{x}^j, \bar{x}^k)$ such that   $\bar{x}^i, \bar{x}^j, \bar{x}^k>\textbf{0}$ is saturated, it follows from \cite[Theorem~1]{hofbauer1990index} that there exists at least one solution $x(t) \in \mathbb{R}^{3n}_{+}$ such that $\lim_{t \rightarrow \infty}x(t)=(\bar{x}^i, \bar{x}^j, \bar{x}^k)$ where   $\bar{x}^i, \bar{x}^j, \bar{x}^k>\textbf{0}$.\\
COMMENT: To prove the "in fact,..." part of the corollary, we would need to show that for generic parameter values, the Jacobians, evaluated at the equilibria, are nonsingular.
}
\par Corollary: Suppose that every boundary equilibrium is unstable and every 2-coexistence equilibrium is not saturated. Then, there exists at least one 3-coexistence equilibrium and there exists a trajectory from the interior of $\mathcal{D}$ that reaches it. In fact, there must be an odd number of 3-coexistence equilibrium.

% Suppose that every boundary equilibrium $(\textbf{0}, \hdots, \tilde x^k, \hdots, \textbf{0})$ is unstable. Suppose further that, for any 2-coexistence equilibrium with $\bar x^i, \bar x^j > \textbf{0}$ and $\bar x^k = 0$, the corresponding bivirus equilibrium $(\bar x^i, \bar x^j)$ is unstable and $s(-D^k + (I-\bar X^i-\bar X^j)B^k) > 0$. Then, there exists at least one 3-coexistence equilibrium and a trajectory from the interior of $\mathcal{D}$ that reaches it. In fact, there must be an odd number of 3-coexistence equilibrium.
}


\section{Existence and attractivity of a line ofcoexistence  equilibria for nongeneric tri-virus networks}\label{sec:line:attractivity}
\par 
%Proposition~\ref{prop:necessity} and Theorem~\ref{thm:local}, respectively, deal with the existence and local exponential convergence to a boundary equilibrium. where one, and only one, virus is alive; the rest have become extinct.  When two competing viruses pervade a population, it is also possible that the two  viruses could exist in some sort of balance with each other \cite{liu2019analysis,pare2021multi,ye2021convergence,axel2020TAC}.
\seb{Section~\ref{sec:persistence:one:virus}-\ref{sec:existence:coexistence} pertain to the existence or nonexistence of various endemic equilibria. Note that in those sections the discussion center around the (non)existence (and, in Section~\ref{sec:persistence:one:virus} also stability (or lack thereof)) of \emph{an} equilibrium point. In this section, however, we will focus on identifying a scenario which guarantees the existence and local exponential attractivity of a line of coexistence  equilibria.
}
%In this section, we are interested in identifying a scenario which guarantees the existence and local exponential attractivity of a continuum of coexistence  equilibria.
% (i.e., at least two, but possibly all three, viruses exist in some sort of balance with each other), when all three %competing 
% viruses 
% %are prevalent 
% persist
% in a population. 

% We will consider a special case, namely that of homogeneous spread, which is formalized via the following assumption.
% \begin{assm}\label{assm:homogenous}
% We suppose that
% \begin{enumerate}[label=(\roman*)]
%     \item All three viruses are spreading over the same graph.
%     \item For all $i \in [n]$ $\delta_i^1 =\delta^1$, $\delta_i^2 =\delta^2$, and $\delta_i^3 =\delta^3$.
%     \item For all $i=j \in [n]$ and $(i,j) \in \mathcal E$, $\beta_{ij}^1=\beta^1>0$, $\beta_{ij}^2=\beta^2>0$, and $\beta_{ij}^3=\beta^3>0$.
% \end{enumerate}
 
% \end{assm}
% The following result identifies one such condition, and is an extension of \cite[Theorem~6]{liu2019analysis}.
% \begin{prop}\label{prop:line} Consider system~\eqref{eq:x1}-\eqref{eq:x3} under Assumptions~\ref{assum:base} and~\ref{assm:homogenous}. Further, suppose that the matrix $A$ is irreducible and that $s(A)>\frac{\delta^1}{\beta^1}=\frac{\delta^2}{\beta^2} = \frac{\delta^3}{\beta^3}$.
%     \begin{enumerate}[label=\roman*)]
%         \item \label{no3} If $(\tilde{x}^1, \tilde{x}^2, \textbf{0})$ is an equilibrium of~\eqref{eq:x1}-\eqref{eq:x3}, then $\tilde{x}^1= \alpha_1\tilde{x}^2$ for some $\alpha_1>0$. For each $\alpha_1>0$, there exists a unique pair  $(\tilde{x}^1, \tilde{x}^2)$ such that $\tilde{x}^1= \alpha_1\tilde{x}^2$. Furthermore, if $\tilde{x}^1= \tilde{x}^2$, then 
%         the equilibrium set $(\beta_1\tilde{x}^1, (1-\beta_1)\tilde{x}^1)$ with $\beta_1 \in [0,1]$ is locally exponentially attractive, i.e., for all $x(0) \in \mathcal D$ such that $x(0)$ is close to a point on the line $(\beta_1\tilde{x}^1, (1-\beta_1)\tilde{x}^2)$, $x(t)$ approaches the line $(\beta_1\tilde{x}^1, (1-\beta_1)\tilde{x}^1)$ exponentially fast. 
%         \item If $(\textbf{0},\tilde{x}^2, \tilde{x}^3)$ is an equilibrium of~\eqref{eq:x1}-\eqref{eq:x3}, then $\tilde{x}^2= \alpha_2\tilde{x}^3$ for some $\alpha_2>0$. For each $\alpha_2>0$, there exists a unique pair  $(\tilde{x}^1, \tilde{x}^2)$ such that $\tilde{x}^1= \alpha_2\tilde{x}^2$. Furthermore, if $\tilde{x}^2= \tilde{x}^3$, the equilibrium set $(\beta_2\tilde{x}^2, (1-\beta_2)\tilde{x}^2)$ with $\beta_2 \in [0,1]$ is locally exponentially attractive. 
%           \item If $(\tilde{x}^1, \textbf{0}, \tilde{x}^3)$ is an equilibrium of~\eqref{eq:x1}-\eqref{eq:x3}, then $\tilde{x}^1= \alpha_3\tilde{x}^3$ for some $\alpha_3>0$. For each $\alpha_3>0$, there exists a unique pair  $(\tilde{x}^1, \tilde{x}^3)$ such that $\tilde{x}^1= \alpha_3\tilde{x}^3$. Furthermore, if $\tilde{x}^1= \tilde{x}^3$, the equilibrium set $(\beta_3\tilde{x}^1, (1-\beta_3)\tilde{x}^1)$ with $\beta_3 \in [0,1]$ is locally exponentially attractive.  
%     \end{enumerate}
% \end{prop}
% \textit{Proof:} We prove statement~\ref{no3}. The proofs for the other two statements are same up to a suitable adjustment of notation.\\
% Suppose that $(\tilde{x}^1, \tilde{x}^2, \textbf{0})$ is an equilibrium of~\eqref{eq:x1}-\eqref{eq:x3}. Therefore, by the definition of an equilibrium point, we have the following:
% \begin{align} 
%   \textbf{0} &=  \Big{(} \big{(} I - (\Tilde{X}^1+\Tilde{X}^2) \big{)} B^1 - D^1 \Big{)} \tilde{x}^1(t), \label{eq:x1:eqm}\\
%     \textbf{0} &=  \Big{(} \big{(} I - (\Tilde{X}^1+\Tilde{X}^2)\big{)} B^2 - D^2 \Big{)} \tilde{x}^2(t) \label{eq:x2:eqm}
%   \end{align}
%   Observe that ~\eqref{eq:x1:eqm} and~\eqref{eq:x2:eqm} are equilibrium versions of bivirus system equations. The result then follows by the analysis in the proof of \cite[Theorem~6]{liu2019analysis}.
%   \par By assumption $\tilde{x}^1=\tilde{x}^2$. Hence, exponential  convergence to the set $(\beta_1\tilde{x}^1, (1-\beta_1)\tilde{x}^1)$ with $\beta_1 \in [0,1]$ is due to \cite[Proposition~9]{ye2021convergence}.
%   ~$\blacksquare$
  
 
  
% % More generally, a sufficient condition for the existence of a plane of coexisting equilibria has been identified in \cite[Proposition~1]{bens-notes}.  
% \par We now identify another sufficient condition for the existence of multiple lines of equilibria, but without the assumption of homogeneous spread. We need the following assumption, instead.
% \begin{assm}\label{assm:hetero:samegraph}
% We suppose that
% \begin{enumerate}[label=(\roman*)]
%     \item All three viruses are spreading over the same graph.
%     \item For all $i \in [n]$ $\delta_i^1 =\delta_i^2 =\delta_i^3>0$.
%     \item For all $i=j \in [n]$ and $(i,j) \in \mathcal E$, $\beta_{ij}^1=\beta_{ij}^2=\beta_{ij}^3$.
% \end{enumerate}
% \end{assm}

% \begin{prop}\label{prop:hetero:lines}
% Consider system~\eqref{eq:x1}-\eqref{eq:x3} under Assumptions~\ref{assum:base} and~\ref{assm:hetero:samegraph}. Further, suppose that the matrix $A$ is irreducible and that $s(-D+B)>0$. 
% \begin{enumerate}[label=\roman*)]
%     \item \label{no3:a}If $(\tilde{x}^1, \tilde{x}^2, \textbf{0})$ is an equilibrium of~\eqref{eq:x1}-\eqref{eq:x3}, then $\tilde{x}^1= \alpha_1\tilde{x}^2$ for some $\alpha_1>0$. For each $\alpha_1>0$, there exists a unique pair  $(\tilde{x}^1, \tilde{x}^2)$ such that $\tilde{x}^1= \alpha_1\tilde{x}^2$. Furthermore, if $\tilde{x}^1= \tilde{x}^2$, then the equilibrium set $(\beta_1\tilde{x}^1, (1-\beta_1)\tilde{x}^1)$ with $\beta_1 \in [0,1]$ is locally exponentially attractive. 
%      \item If $(\textbf{0},\tilde{x}^2, \tilde{x}^3)$ is an equilibrium of~\eqref{eq:x1}-\eqref{eq:x3}, then $\tilde{x}^2= \alpha_2\tilde{x}^3$ for some $\alpha_2>0$. For each $\alpha_2>0$, there exists a unique pair  $(\tilde{x}^1, \tilde{x}^2)$ such that $\tilde{x}^1= \alpha_2\tilde{x}^2$. Furthermore, if $\tilde{x}^2= \tilde{x}^3$, the equilibrium set $(\beta_2\tilde{x}^2, (1-\beta_2)\tilde{x}^2)$ with $\beta_2 \in [0,1]$ is locally exponentially attractive. 
%           \item If $(\tilde{x}^1, \textbf{0}, \tilde{x}^3)$ is an equilibrium of~\eqref{eq:x1}-\eqref{eq:x3}, then $\tilde{x}^1= \alpha_3\tilde{x}^3$ for some $\alpha_3>0$. For each $\alpha_3>0$, there exists a unique pair  $(\tilde{x}^1, \tilde{x}^3)$ such that $\tilde{x}^1= \alpha_3\tilde{x}^3$. Furthermore, if $\tilde{x}^1= \tilde{x}^3$, the equilibrium set $(\beta_3\tilde{x}^1, (1-\beta_3)\tilde{x}^1)$ with $\beta_3 \in [0,1]$ is locally exponentially attractive.
% \end{enumerate}
% \end{prop}
% \textit{Proof:} We prove statement~\ref{no3:a}. The proofs for the other two statements are same up to a suitable adjustment of notation.\\
% Suppose that $(\tilde{x}^1, \tilde{x}^2, \textbf{0})$ is an equilibrium of~\eqref{eq:x1}-\eqref{eq:x3}. Therefore, by the definition of an equilibrium point, we have the following:
% \begin{align} 
%   \textbf{0} &=  \Big{(} \big{(} I - (\Tilde{X}^1+\Tilde{X}^2) \big{)} B^1 - D^1 \Big{)} \tilde{x}^1(t), \label{eq:x1a:eqm}\\
%     \textbf{0} &=  \Big{(} \big{(} I - (\Tilde{X}^1+\Tilde{X}^2)\big{)} B^2 - D^2 \Big{)} \tilde{x}^2(t) \label{eq:x2a:eqm}
%   \end{align}
%   Observe that ~\eqref{eq:x1a:eqm} and~\eqref{eq:x2a:eqm} are equilibrium versions of bivirus system equations. The result then follows by the analysis in the proof of \cite[Theorem~7]{liu2019analysis}.
%   \par By assumption $\tilde{x}^1=\tilde{x}^2$. Hence, exponential convergence to the set $(\beta_1\tilde{x}^1, (1-\beta_1)\tilde{x}^1)$ with $\beta_1 \in [0,1]$ is due to \cite[Proposition~9]{ye2021convergence}.

% %Observe that Proposition~\ref{prop:hetero:lines} (and also Proposition~\ref{prop:line}) assumes that the dynamics of virus~3 have been nullified. In other words, Proposition~\ref{prop:hetero:lines} addresses the case where $x_i^3(0)=0$ for $i \in [n]$. For the case where $x_i^3(0) \neq 0$ for  some $i \in [n]$, while one can be assured of the existence of  multiple lines of equilibria, local exponential attractivity to these lines is contingent on the hurwitzness of the linearized (around the equilibrium point $(\tilde{x}^1, \tilde{x}^2, \textbf{0})$) state matrix for virus~3. The following theorem formalizes the same.}

% % 
% % \begin{thm}\label{thm:nonzero:init:condns:dummy}
% % Consider system~\eqref{eq:x1}-\eqref{eq:x3} under Assumptions~\ref{assum:base} and~\ref{assm:hetero:samegraph}. Suppose that the matrix $A$ is irreducible and that $s(-D+B)>0$. Suppose that $D^k=I$ for $k \in [3]$. If $(\tilde{x}^1, \tilde{x}^2, \textbf{0})$ is an equilibrium of~\eqref{eq:x1}-\eqref{eq:x3}, then $\tilde{x}^1= \alpha_1\tilde{x}^2$ for some $\alpha_1>0$. For each $\alpha_1>0$, there exists a unique pair  $(\tilde{x}^1, \tilde{x}^2)$ such that $\tilde{x}^1= \alpha_1\tilde{x}^2$. Furthermore, if $\tilde{x}^1= \tilde{x}^2$, then
% % \begin{enumerate} [label=\roman*)]
% %     \item if $s(-I+(I-\tilde{X}^1-\tilde{X}^2)B^3)<0$, then 
% % the equilibrium set $(\beta_1\tilde{x}^1, (1-\beta_1)\tilde{x}^1, \textbf{0})$ with $\beta_1 \in [0,1]$ is locally exponentially attractive.
% %     \item if $s(-I+(I-\tilde{X}^1-\tilde{X}^2)B^3)>0$, then 
% % the equilibrium set $(\beta_1\tilde{x}^1, (1-\beta_1)\tilde{x}^1, \textbf{0})$ with $\beta_1 \in [0,1]$ is unstable.
% % \end{enumerate}
% % \end{thm}
% % \textit{Proof:} The proof for existence of a line of equilibria of the form $\tilde{x}^1= \alpha_1\tilde{x}^2$ for some $\alpha_1>0$, and that, for each $\alpha_1>0$, there exists a unique pair  $(\tilde{x}^1, \tilde{x}^2)$ such that $\tilde{x}^1= \alpha_1\tilde{x}^2$ is the same as in the proof of Proposition~\ref{prop:hetero:lines}. We focus on proving local exponential attractivity or lack thereof in the sequel.
% % \par \textit{Proof of statement i):} Note that the Jacobian computed at the equilibrium point $(\tilde{x}^1, \tilde{x}^2, \textbf{0})$ is as in equation~\eqref{jacob:thm2}. 
% % \begin{figure*}[h]
% % 	{\noindent}
% % \begin{align}\label{jacob:thm2}
% % &J(\tilde{x}^1, \tilde{x}^2, \textbf{0}) = \\
% % &\scriptsize
% % \begin{bmatrix}
% % -I+(I-\tilde{X}^1-\tilde{X}^2)B^1-\diag(B^1x^1) & -\diag(B^1x^1)   & -\diag(B^1x^1)  \\
% % -\diag(B^2x^2) & -I+(I-\tilde{X}^1-\tilde{X}^2)B^2-\diag(B^2x^2)  & -\diag(B^2x^2)\\
% %  \textbf{0}&  \textbf{0}& -I+(I-\tilde{X}^1-\tilde{X}^2)B^3 \end{bmatrix}\normalsize,\nonumber
% % \end{align}
% % \end{figure*}
% % Hence, we can rewrite $ J(\tilde{x}^1, \tilde{x}^2, \textbf{0})$ as %\scriptsize
% % \begin{align}\label{J:forhatx1:partitioned}
% %     J(\tilde{x}^1, \tilde{x}^2, \textbf{0})
% %     =
% %     & %\scriptsize
% %     \begin{bmatrix} 
% %     \bar{J}(\tilde{x}^1, \tilde{x}^2, \textbf{0}) 
% %       %\diag(B^1\hat{x}^1)
% %       && \hat{J} \\
% %       \mathbf{0} && -I+(I-\tilde{X}^1-\tilde{X}^2)B^3
% %     \end{bmatrix}, 
% % \end{align}
% % \normalsize
% % where
% % \begin{align}\label{jacob:hatx1:22submatrix} 
% % &\bar{J}(\tilde{x}^1, \tilde{x}^2, \textbf{0}) = 
% % & \scriptsize
% % \begin{bmatrix} 
% % -I+(I-\tilde{X}^1-\tilde{X}^2)B^1-\hat{B}^1 & -\hat{B}^1   \\
% % -\hat{B}^2 & -I+(I-\tilde{X}^1-\tilde{X}^2)B^2-\hat{B}^2
% %  \end{bmatrix}, \end{align}
% %  \normalsize
% % while $\hat{J} =  \scriptsize \begin{bmatrix}-\hat{B}^1 \\ -\hat{B}^2 \end{bmatrix}$.
% % \par Note that the matrix $ J(\tilde{x}^1, \tilde{x}^2, \textbf{0})$ is block upper triangular. Hence, it is clear that $s(J(\tilde{x}^1, \tilde{x}^2, \textbf{0})) \leq 0$ if, and only if, i) $s(\bar{J}(\tilde{x}^1, \tilde{x}^2, \textbf{0}) \leq 0$, and ii) $s(-I+(I-\tilde{X}^1-\tilde{X}^2)B^3) \leq 0$. Since, by assumption $s(-I+(I-\tilde{X}^1-\tilde{X}^2)B^3)<0$, it follows that  $s(J(\tilde{x}^1, \tilde{x}^2, \textbf{0})) \leq 0$ if and only if $s(\bar{J}(\tilde{x}^1, \tilde{x}^2, \textbf{0}) \leq 0$. Consider the matrix $\bar{J}(\tilde{x}^1, \tilde{x}^2, \textbf{0})$. Define $P: = \begin{bmatrix} I_n && \textbf{0} \\ \textbf{0} && -I_n \end{bmatrix}$. Therefore, 
% % \begin{align}
% %     P\bar{J}(\tilde{x}^1, \tilde{x}^2, \textbf{0})P
% %     = & \scriptsize
% % \begin{bmatrix} 
% % -I+(I-\tilde{X}^1-\tilde{X}^2)B^1-\hat{B}^1 & \hat{B}^1   \\
% % \hat{B}^2 & -I+(I-\tilde{X}^1-\tilde{X}^2)B^2-\hat{B}^2
% % \end{bmatrix}.
% % \end{align}
% % Note that $P\bar{J}(\tilde{x}^1, \tilde{x}^2, \textbf{0})P$ is irreducible Metzler, and that, by assumption, $\tilde{x}^1=\tilde{x}^2$. Hence, by considering the element-wise positive vector $z:=\begin{bmatrix}\tilde{x}^1 &&\tilde{x}^1\end{bmatrix}^\top$, and by invoking the equilibrium version of the  equation of the single-virus system corresponding to virus~1, it follows that $P\bar{J}(\tilde{x}^1, \tilde{x}^1, \textbf{0})Pz=\textbf{0}$. Hence, from Lemma~\ref{lem:perron_frob_metz}, it follows that $s(P\bar{J}(\tilde{x}^1, \tilde{x}^1, \textbf{0})P)=0$, which implies $s(\bar{J}(\tilde{x}^1, \tilde{x}^1, \textbf{0}))=0$. Consequently, $s(J(\tilde{x}^1, \tilde{x}^1, \textbf{0}))=0$.
% % Furthermore, since $s(\bar{J}(\tilde{x}^1, \tilde{x}^1, \textbf{0}))=0$, it must be also true that the matrix $J(\tilde{x}^1, \tilde{x}^1, \textbf{0})$ has exactly one eigenvalue at the origin, and all other eigenvalues have negative real parts. Therefore, the bivirus equations associated with the line of equilibria define a one-dimensional center manifold along which the Jacobian is singular. Therefore, from \cite[Theorem~8.2]{khalil2002nonlinear} it follows that the line of equilibria is locally attractive.~$\blacksquare$
% % }


Let $z$ denote the single-virus endemic equilibrium corresponding to virus~1, with $Z=\diag(z)$. Therefore, assuming $D^1=I$, the vector $z$ fulfils the following:
\begin{equation}\label{eq:z}
    (-I+(I-Z)B^1)z=\textbf{0}.
\end{equation}
Furthermore, since $z$ is an endemic equilibrium, from \cite[Lemma~6]{axel2020TAC} %Lemma~\ref{lem:equi_non-zero_nonone}
it follows that $\textbf{0} \ll z \ll \textbf{1}$. Let $C$ be any nonnegative irreducible matrix for which $z$ is also an eigenvector corresponding to eigenvalue one. That is, $Cz=z$. Therefore, from %\cite[Chapter 8.3]{meyer2000matrix},
\cite[Theorem~2.7]{varga1999matrix}, %Lemma~\ref{lem:perron_frob}, 
it follows that $\rho(C)=1$, and that the vector $z$, up to a scaling, is the unique eigenvector of $C$ with all entries being strictly positive. Define
\begin{equation}\label{eq:B2}
    B^2:=(I-Z)^{-1}C.
\end{equation}
%and 
% \begin{equation}\label{eq:B3}
%     B^3:=(I-Z)^{-1}C
% \end{equation}
We have the following result.
\begin{thm}\label{thm:init:condns}
Consider system~\eqref{eq:x1}-\eqref{eq:x3} under Assumption~\ref{assum:base}. Suppose that $D^k=I$ for $k \in [3]$.
    Suppose that 
    $B^1$ and $B^3$ are arbitrary nonnegative irreducible matrices; and
    vector $z$ and matrix $B^2$ are as defined in~\eqref{eq:z} and~\eqref{eq:B2}, respectively. Then, a set of equilibrium points of the trivirus equations is given by $(\beta_1z, (1-\beta_1)z, \textbf{0})$ for all $\beta_1 \in [0,1]$. Furthermore, \begin{enumerate} [label=\roman*)]
    \item if $s(-I+(I-Z)B^3)<0$, then 
the equilibrium set $(\beta_1z, (1-\beta_1)z, \textbf{0})$, with $\beta_1 \in [0,1]$, is locally exponentially attractive.
    \item if $s(-I+(I-Z)B^3)>0$, then 
the equilibrium set $(\beta_1z, (1-\beta_1)z, \textbf{0})$, with $\beta_1 \in [0,1]$, is unstable.
\end{enumerate}
% \item Suppose that $B^1$ and $B^2$ are arbitrary nonnegative irreducible matrices;
% vector $z$ and matrix $B^3$ are as as defined in~\eqref{eq:z} and~\eqref{eq:B3}, respectively. Then, a set of equilibrium points of the trivirus equations is given by $(\beta_2z, \textbf{0}, (1-\beta_2)z)$ with $\beta_2 \in [0,1]$. Furthermore, \begin{enumerate} [label=\roman*)]
%     \item if $s(-I+(I-Z)B^2)<0$, then 
% the equilibrium set $(\beta_2z,  \textbf{0}, (1-\beta_2)z)$, with $\beta_2 \in [0,1]$, is locally exponentially attractive.
%     \item if $s(-I+(I-Z)B^2)>0$, then 
% the equilibrium set $(\beta_2z,  \textbf{0}, (1-\beta_2)z)$, with $\beta_2 \in [0,1]$, is unstable.
% \end{enumerate}
%\end{enumerate}
\end{thm}

\textit{Proof:} We first show that, for all $\beta_1 \in [0,1]$, the  point $(\beta_1z, (1-\beta_1)z, \textbf{0})$ fulfils the equilibrium version of equations~\eqref{eq:x1}-\eqref{eq:x3}. To this end, observe that the right hand side of~\eqref{eq:x1}-\eqref{eq:x3} evaluated at  $(\beta_1z, (1-\beta_1)z, \textbf{0})$ yields:
\begin{align}
    &(-I+(I-\beta_1Z-(1-\beta_1)Z)B^1)\beta_1z \nonumber \\
    &=(-I+(I-Z)B^1)\beta_1z =0, \label{beta1z=0}
\end{align}
where~\eqref{beta1z=0} follows by noting that $\beta_1$ is a scalar, and  $z$ is the single-virus endemic equilibrium corresponding to virus~1. Similarly, 
\begin{align}
    &(-I+(I-\beta_1Z-(1-\beta_1)Z)B^2)(1-\beta_1)z \nonumber \\
    &=(-I+(I-Z)B^2)(1-\beta_1)z \nonumber \\
     &=(-I+(I-Z)(I-Z)^{-1}C)(1-\beta_1)z \nonumber \\
     &=(-I+C)(1-\beta_1)z=0, \label{beta2z=0}
\end{align}
where~\eqref{beta2z=0} follows by noting that $Cz=z$. Thus, from~\eqref{beta1z=0} and~\eqref{beta2z=0}, it is clear that, for every $\beta_1 \in [0,1]$,  $(\beta_1z, (1-\beta_1)z, \textbf{0})$ is an equilibrium point of system~\eqref{eq:x1}-\eqref{eq:x3}; i.e. there is a set of equilibrium points $(\beta_1z, (1-\beta_1)z, \textbf{0})$ with $\beta_1 \in [0,1]$. 
\par Next, observe that, for any $\beta_1 \in [0,1]$, the Jacobian evaluated at $(\beta_1z, (1-\beta_1)z, \textbf{0})$ is as given in~\eqref{jacobian:thm2}.
\begin{figure*}[h]
	{\noindent}
\begin{align} \label{jacobian:thm2}
&J(\beta_1z, (1{-}\beta_1)z, \textbf{0}) =
%J(\beta_1z, (1-\beta_1)z, \textbf{0}) =&\scriptsize
\scriptsize\begin{bmatrix}
-I+(I-Z)B^1-\diag(B^1\beta_1z) & -\diag(B^1\beta_1z)   & -\diag(B^1\beta_1z)  \\
-\diag(B^2(1-\beta_1)z) & -I+(I-Z)B^2-\diag(B^2(1-\beta_1)z)  & -\diag(B^2(1-\beta_1)z)\\
 \textbf{0}&  \textbf{0}& -I+(I-Z)B^3 \end{bmatrix}\normalsize
\end{align}
\caption*{}
\end{figure*}
Hence, we can rewrite $J(\beta_1z, (1-\beta_1)z, \textbf{0})$ as %\scriptsize
\begin{align}\label{J:forhatx1:partitioned:1}
    J(\beta_1z, (1-\beta_1)z, \textbf{0})
    =
    & \scriptsize
    \begin{bmatrix} 
    \bar{J}(\beta_1z, (1{-}\beta_1)z, \textbf{0}) 
      %\diag(B^1\hat{x}^1)
      && \hat{J} \\
      \mathbf{0} && {-}I{+}(I{-}Z)B^3
    \end{bmatrix}, 
\end{align}
\normalsize
where
\begin{align}\label{jacob:hatx1:22submatrix:1} 
&\bar{J}(\beta_1z, (1-\beta_1)z, \textbf{0})  \nonumber\\ 
& = \scriptsize
\begin{bmatrix} 
{-}I{+}(I{-}Z)B^1{-}\diag(B^1\beta_1z) & -\diag(B^1\beta_1z)   \\
-\diag(B^2(1-\beta_1)z)& {-}I{+}(I{-}Z)B^2{-}\diag(B^2(1{-}\beta_1)z)
 \end{bmatrix}, \end{align}
 \normalsize
while $\hat{J} =  \scriptsize \begin{bmatrix}-\diag(B^1\beta_1z) \\ -\diag(B^2(1-\beta_1)z)\end{bmatrix}$.
\par Note that the matrix $J(\beta_1z, (1-\beta_1)z, \textbf{0})$ is block upper triangular. Hence, it is clear that $s(J(\beta_1z, (1-\beta_1)z, \textbf{0})) =\max\{s(\bar{J}(\beta_1z, (1-\beta_1)z, \textbf{0}), s(-I+(I-Z)B^3)\}$. \\

\textit{Proof of statement~i):} Consider the matrix \break $\bar{J}(\beta_1z, (1-\beta_1)z, \textbf{0})$.  Define $P: = \begin{bmatrix} I_n && \textbf{0} \\ \textbf{0} && -I_n \end{bmatrix}$. Therefore, 
\begin{align}
    &P\bar{J}((\beta_1z, (1-\beta_1)z, \textbf{0})P \nonumber \\
    & = \scriptsize
\begin{bmatrix} 
{-}I{+}(I{-}Z)B^1{-}\diag(B^1\beta_1z) & \diag(B^1\beta_1z)   \\
\diag(B^2(1-\beta_1)z) & {-}I{+}(I{-}Z)B^2{-}\diag(B^2(1{-}\beta_1)z)
\end{bmatrix}.
\end{align}

Note that $P\bar{J}((\beta_1z, (1-\beta_1)z, \textbf{0})P$ is an irreducible Metzler.  Hence, by considering the element-wise positive vector $\begin{bmatrix}z^\top &&z^\top\end{bmatrix}^\top$, and by invoking the equilibrium version of the  equation of the single-virus system corresponding to virus~1, it follows that $P\bar{J}((\beta_1z, (1-\beta_1)z, \textbf{0})Pz=\textbf{0}$. Therefore, from \cite[Lemma~2.3]{varga1999matrix},
%Lemma~\ref{lem:perron_frob_metz}, 
it follows that $s(P\bar{J}(\beta_1z, (1-\beta_1)z, \textbf{0})P)=0$, which implies $s(\bar{J}(\beta_1z, (1-\beta_1)z, \textbf{0})=0$. Consequently, since, by assumption, $s(-I+(I-Z)B^3)<0$, we obtain
$s(J(\beta_1z, (1-\beta_1)z, \textbf{0}))=0$.
Furthermore, since $s(P\bar{J}(\beta_1z, (1-\beta_1)z, \textbf{0})P)=0$, then since $P\bar{J}(\beta_1z, (1-\beta_1)z, \textbf{0})P$ is irreducible Metzler, %by Lemma~\ref{lem:perron_frob}, 
by \cite[Theorem~2.7]{varga1999matrix}
we have that the matrix $\bar{J}(\beta_1z, (1-\beta_1)z, \textbf{0})$  has exactly one eigenvalue at the origin, and all other eigenvalues have negative real parts. Therefore, since $s(-I+(I-Z)B^3)<0$, the matrix $J(\beta_1z, (1-\beta_1)z, \textbf{0})$ has exactly one eigenvalue at the origin, and all other eigenvalues have negative real parts. Hence,  the trivirus equations associated with the line of equilibria define a one-dimensional center manifold along which the Jacobian is singular. Therefore, from \cite[Theorem~8.2]{khalil2002nonlinear} it follows that the set of equilibrium points $(\beta_1z, (1-\beta_1)z, \textbf{0})$, with $\beta_1 \in [0,1]$, is locally exponentially attractive.

\textit{Proof of statement~ii):} By assumption, $s(-I+(I-Z)B^3)>0$. Hence, $s(J(\beta_1z, (1-\beta_1)z, \textbf{0}))>0$. Therefore, from \cite[Theorem~4.7, statement ii)]{khalil2002nonlinear} it follows that set of equilibrium points $(\beta_1z, (1-\beta_1)z, \textbf{0})$, with $\beta_1 \in [0,1]$, is unstable.%~$\blacksquare$

%\textit{Proof of statement 2.:} 
Observe that the key idea behind Theorem~\ref{thm:init:condns} is to fix the matrix $B^1$, and then choose $B^2$ (as given in~\eqref{eq:B2}) so as to obtain a locally exponentially attractive (resp. unstable) equilibrium set, namely $(\beta_1z, (1-\beta_1)z, \textbf{0})$, with $\beta_1 \in [0,1]$.
Clearly, the same idea can be applied using the other two possible pairs, namely $(B^1, B^3)$ and $(B^2, B^3)$, to obtain corresponding locally exponentially attractive (resp. unstable) equilibrium sets.
% Supposing that $B^1$, $B^2$ are arbitrary nonnegative irreducible matrices, and 
% \begin{equation}\label{eq:B3}
%     B^3:=(I-Z)^{-1}C,
% \end{equation}
% where $C$ is chosen as discussed previously.
% %$B^3$ is as defined in~\eqref{eq:B3}. 
% Then, by similar arguments as in the first part of the proof of Theorem~\ref{thm:init:condns}, it can be shown that, for every $\beta_2 \in [0,1]$, the point
% $(\beta_2z,  \textbf{0},(1-\beta_2)z)$ is an equilibrium point of~\eqref{eq:x1}-\eqref{eq:x3}. After performing elementary row and column operations on the Jacobian evaluated at $(\beta_2z,  \textbf{0},(1-\beta_2)z)$, i.e., $J((\beta_2z,  \textbf{0},(1-\beta_2)z))$, the conditions for attractivity (resp. instability) of this set of equilibrium points  are analogous to that provided statement~i) (resp.~ii)). Let $z_1$ denote the single-virus endemic equilibrium corresponding to virus~2. Then, by fixing matrix $B^2$ and then choosing $B^3$  in a manner analogous to that given in~\eqref{eq:B3} (but matrix $C$ fulfilling $Cz_1=z_1$), we can obtain  a locally exponentially attractive (resp. unstable) equilibrium set, namely  $(\textbf{0}, \beta_3z_1, (1-\beta_3)z_1)$, with $\beta_3 \in [0,1]$.% where $z_1$ denotes the single-virus endemic equilibrium corresponding to virus~2. 



% Observe that the key idea behind Theorem~\ref{thm:init:condns} is to fix the matrix $B^1$, and then choose $B^2$ (as given in~\eqref{eq:B2}) so as to obtain a locally exponentially attractive (resp. unstable) equilibrium set, namely $(\beta_1z, (1-\beta_1)z, \textbf{0})$, with $\beta_1 \in [0,1]$. Note that by fixing the matrix $B^1$ (resp. $B^2$) and then choosing $B^3$ (resp. $B^3$) in a manner analogous to that given in~\eqref{eq:B2}, we can obtain  a locally exponentially attractive (resp. unstable) equilibrium set, namely $(\beta_2z,  \textbf{0}, (1-\beta_2)z)$, with $\beta_2 \in [0,1]$ (resp. $(\textbf{0}, \beta_3z_1, (1-\beta_3)z_1)$, with $\beta_3 \in [0,1]$, where $z_1$ denotes the single-virus endemic equilibrium corresponding to virus~2. ).

The findings of Theorem~\ref{thm:init:condns} are not in conflict with the claim for finiteness of equilibria presented in \seb{Proposition~\ref{prop:finite:equilibria}.} %Section~\ref{sec:trivirus:monotone}. 
The elements in matrices $D^i$, $B^i$, $i=1,2,3$, are either a priori fixed to a specific value or they are not. In case of the latter, we refer to those as free parameters, in the sense that these are allowed to take any value in $\mathbb{R}_{+}$. The dimension of the space of free parameters equals the number of free parameters in the tri-virus system.
Each choice of free parameters yields a realization of system~\eqref{eq:x1}-\eqref{eq:x3}. The set of choices of free parameters that fall within the special case identified by Theorem~\ref{thm:init:condns} has measure zero.

%The following remark  is in order.
\begin{rem}\label{rem:zero:init:condns}
Note that Theorem~\ref{thm:init:condns} admits a non-zero %spread 
initial infection level in the network for virus~3.  
If one were to assume that 
%there is no spread for virus~3, i.e., 
$x_i^3(0) =0$ for all $i \in [n]$, then the trivirus system of equations (i.e., \eqref{eq:x1}-\eqref{eq:x3})  collapses into the bivirus equation set (i.e., equation~\eqref{eq:full} where $m=2$). Under such a setting, Theorem~\ref{thm:init:condns} coincides with \cite[Proposition~3.9]{ye2021convergence}, which, in turn, subsumes \cite[Theorems~6 and~7]{liu2019analysis}; both with respect to i) admitting a larger class of parameters than \cite[Theorems~6 and~7]{liu2019analysis} (and, assuming the setup in \cite{pare2021multi} is restricted to the bi-virus case,  \cite[Corollaries~2 and~3]{pare2021multi}), and ii) providing guarantees for local exponential attractivity.%~\qed
\end{rem}
% \seb{Note that Theorem~\ref{thm:init:condns} is reliant on the assumption that the healing rates for all agents with respect to all viruses are unity. For system~\eqref{eq:full} with $m=3$, by extending the arguments from \cite[Lemma~3.7]{ye2021convergence},  it is straightforward to show that the \emph{location} of the equilibria is the same when $D^k=I$ for $k\in [3]$ and when $D^k$ are arbitrary positive diagonal matrices with the $D^k$s not necessarily being equal to each other.}
% \seb{
% \begin{rem}[Possible loss of generality in assuming $D^k=I$ for $k=1,2,3$] 
% % Note that Theorem~\ref{thm:init:condns} is reliant on the assumption that the healing rates for all agents with respect to all viruses are unity. For system~\eqref{eq:full} with $m=3$, by extending the arguments from \cite[Lemma~3.7]{ye2021convergence},  it is straightforward to show that the \emph{location} of the equilibria is the same when $D^k=I$ for $k\in [3]$ and when $D^k$ are arbitrary positive diagonal matrices with the $D^k$s not necessarily being equal to each other. 
% As discussed previously, there is no loss of generality in setting $D^k=I$ for $k=1,2,3$ insofar the location of equilibria is concerned.
% However, unlike the case when $m=2$, it is not known if the attractivity properties of the equilibrium set that has been established in Theorem~\ref{thm:init:condns} will hold even if the assumption $D^k=I$ for $k=1,2,3$ were to be relaxed.~\qed
% \end{rem}
% }

\section{%Global convergence to a 
Plane of coexistence equilibria for nongenric trivirus networks}\label{sec:global:plane}
\seb{The focus of Section~\ref{sec:line:attractivity} was on the existence and attractivity (resp. instability) of a line of coexistence equilibria. In this section, we identify two  scenarios that permit the existence of a plane of  coexistence equilibria, and, in the case of one of those scenarios, 
we provide conditions for global convergence to the associated plane of  coexistence equilibria.}
 %furthermore, seek condition(s) that guarantee local (resp. global) stability of such a plane.} 

% Section~\ref{sec:line:attractivity} dealt with the existence and attractivity (resp. instability) of a line of coexistence equilibria. Moving beyond this, it is of natural interest to %consider
% identify scenario(s) where a plane of coexistence equilibria could exist, and furthermore, seek condition(s) that guarantee local (resp. global) stability of such a plane of coexistence equilbria. The present section deals with this issue.


We consider a case where three identical copies of a virus are spreading over the same graph as formalized next.
%{\color{red} I understand there is interest in two things: getting a second example of a plane of equilibria (more accurately, a set of equilibria obtained by intersecting a plane with the positive orthant), analogously to what Liu et al did in AC Trans, and encompassing the example already here, and maybe a second example, in a generalization of the idea for constructing a line of equilibria to become a construction for a plane of equilibria.  For this purpose,  I would look at adding to the definition of $C$ and (36) the following: let $\hat C \neq C$ be a second nonnegative irreducible matrix fo which $z$ is also an iegenvector with eigenvalue 1. Take $B^3=(I-Z)^{-1}\hat C$. As potential plane of equilibria, look at $(\alpha_1z,\alpha_2z,\alpha_3z)$ where $\alpha_i$ sum to 1 and are nonnegative. For this to give an interesting result, one may need $n\geq 3$ or even $n>4$. I suspect that $(I-Z)B^1,C$ and $\hat C$ need to be distinct or one will get a known special case. }

\begin{assm}\label{assm:hetero:samegraph}
We suppose that
\begin{enumerate}[label=(\roman*)]
    \item All three viruses are spreading over the same graph.
    \item For all $i \in [n]$ $\delta_i^1 =\delta_i^2 =\delta_i^3>0$.
    \item For all $i=j \in [n]$ and $(i,j) \in \mathcal E$, $\beta_{ij}^1=\beta_{ij}^2=\beta_{ij}^3$.
\end{enumerate}
\end{assm}

Note that for the  special case identified in Assumption~\ref{assm:hetero:samegraph}, assuming that the setting in \cite{pare2021multi} is restricted to the tri-virus case, the existence of a plane of coexistence equilibrium has been secured by \cite[Corollary~3]{pare2021multi}. However, \cite[Corollary~3]{pare2021multi} does not provide guarantees for even local (let alone global)  convergence to the said plane. To address this shortcoming, first consider the system%The following theorem addresses this drawback.

%First, consider the system 
\begin{equation}\label{eq:positive_linear_x}
\dot x(t) = (-D+(I-\diag(\tilde x))B)x(t),
\end{equation}
where $\tilde x$ is the unique endemic equilibrium of the single virus SIS system associated with $(D, B)$, and with $B$ irreducible. The matrix $Q: = D-(I-\diag(\tilde x))B$ is a singular irreducible $M$-matrix, with a simple eigenvalue at $0$ and all other eigenvalues have positive real part. Associated with this simple zero eigenvalue is the right eigenvector $\tilde x \gg {\bf 0}$ and a left eigenvector $\tilde u^\top \gg {\bf 0}^\top$. We assume that $\tilde u^\top$ is normalised to satisfy $\tilde u^\top \tilde x = 1$. Moreover, there exists a positive diagonal matrix $P$ such that $\bar Q := PQ+Q^\top P \succeq 0$. In fact, if we choose $P = \diag(\tilde u_1/\tilde x_1, \hdots, \tilde u_n/\tilde x_n)$, then it can be verified that $\bar Q$ is an irreducible $M$-matrix of rank $n-1$, with the nullvector $\tilde x$ associated with the simple eigenvalue at $0$, see~\cite[Section 4.3.4]{qu2009cooperative}.
% Hence, the Jordan canonical form of $A$, denoted by $J$, can be expressed as
% \begin{equation}
%     P^{-1}AP = J = \begin{bmatrix}
%     0 & \\ & \bar J
%     \end{bmatrix},
% \end{equation}
% where $\bar J$ contains the Jordan blocks associated with eigenvalues that have positive real part, for some matrix $P = [\tilde x \vdots \tilde P]$. Now define $y = P^{-1}x$, and $\bar y = [y_2, y_3, \hdots, y_n]^\top$. That is, $\bar y$ is simply the vector $y$ with its first entry removed. Observe that $\dot y = -P^{-1}APy = Jy$, from which it follows that
% \begin{equation}\label{eq:bar_y}
%     \dot{\bar y} = -\bar J \bar y.
% \end{equation}
% Standard linear systems theory yields that \eqref{eq:bar_y} is exponentially stable at the origin $\bar y = {\bf 0}$. From $x = Py$, it then follows that the system \eqref{eq:positive_linear_x} will obey $\lim_{t\to\infty} x(t) = \alpha \tilde x$ for some $\alpha \in \mathbb R$. Moreover, since \eqref{eq:positive_linear_x} is a positive linear system, it follows that $x(0) > {\bf 0}$ implies $\alpha > 0$. In fact, $\alpha = \tilde u^\top x(0)$, where $\tilde u^\top$ is the positive left eigenvector of $A$ associated with the eigenvalue at the origin, normalized to satisfy $\tilde u^\top \tilde x = 1$.

%\ben{
\begin{thm}\label{thm:global:plane}
Consider system~\eqref{eq:x1}-\eqref{eq:x3} under Assumptions~\ref{assum:base}, \ref{assum:irreducible} and~\ref{assm:hetero:samegraph}. Further, suppose that $\rho(D^{-1}B)> 1$. 
Then
%Consider system~\eqref{eq:x1}-\eqref{eq:x3} under Assumption~\ref{assum:base}. Suppose that $D^1 = D^2 = D^3 = D$ and $B^1 = B^2 = B^3=B$, with $\rho(D^{-1}B)> 1$. Then %the following statements hold true:
\begin{enumerate}[label=\roman*)]
    \item For all initial conditions satisfying $x^1(0) >  {\bf 0}_n$, $x^2(0) > {\bf 0}_n$, and $x^3(0) > {\bf 0}_n$, we have that $\lim_{t\to\infty} (x^1(t),x^2(t),x^3(t)) \in \mathcal{E}$ exponentially fast, where $$\mathcal {E} = \{(x^1, x^2, x^3) | \alpha_1 x^1+\alpha_2 x^2+\alpha_3 x^3 = \tilde x, \textstyle\sum_{i=1}^3 \alpha_i = 1\},$$ and $\tilde x$ is the unique endemic equilibrium of the single virus SIS dynamics defined by $(D,B)$.
    \item Every point on the connected set $\mathcal{E}$ is a coexistence equilibrium.
\end{enumerate}
\end{thm}
%\textit{Proof:}
\textit{Proof of statement i)}: 
%Note that, from Assumption~\ref{assm:hetero:samegraph}, we have $B^1=B^2=B^3=B$, and $D^1=D^2=D^3=D$. %We prove the first item. 
The initial conditions guarantee that, at some finite time $t$, we have $x^1(t) \gg  {\bf 0}_n$, $x^2(t) \gg {\bf 0}_n$, and $x^3(t)\gg {\bf 0}_n$. Define $z = x^1 + x^2 + x^3$, and $Z=X^1+X^2+X^3$.
%$Z = \diag(x^1) + \diag(x^2) + \diag(x^3)$. 
%Observe that
Therefore, together with Assumption~\ref{assm:hetero:samegraph}, it follows that
\begin{align}
    \dot{z} & = \Big[-D+\big(I-(X^1(t)+X^2(t)+X^3(t))\big)B\Big] \times \nonumber \\
    &~~~~~~~~~~~~~~~~~~~~~~~~~(x^1(t)+x^2(t)+x^3(t)) \\
    & = \big[-D+(I-Z(t))B\big]z(t).\label{eq:single_virus}
\end{align}
Since $z(t) \gg {\bf 0}_n$, and $\rho(D^{-1}B) > 1$ by hypothesis, it follows from \cite[Theorem~2]{khanafer2016stability}
%\cite[Proposition~3]{liu2019analysis} 
%single virus SIS results
that $\lim_{t\to\infty} z(t) = \tilde x$ exponentially fast. In fact, $\tilde x$ is the exponentially stable equilibrium of \eqref{eq:single_virus}, with domain of attraction $z(0) \in [0,1]^n\setminus {\bf 0}$.  It follows that, for all $z(0) \in [0,1]^n\setminus {\bf 0}$, $\Vert \tilde x - z(t) \Vert \leq ae^{-bt}$ for some positive constants $a, b$, with $\Vert \cdot \Vert$ being the Euclidean norm.

The dynamics for virus~$i$, where $i\in [3]$, can be written as $$\dot{x}^i(t) = -[D+(I-\diag(\tilde x))B]x^i(t) + (\diag(\tilde x) - Z(t))Bx^i(t).$$
Without loss of generality, we consider virus~$1$ and drop the superscript. That is, we study the system
\begin{equation}\label{eq:virus_system_plane}
    \dot{x}(t) = -Qx(t) + (\diag(\tilde x) - Z(t))Bx(t),
\end{equation}
where $Z(t)$ is treated as an external time-varying input, and $Q = D-(I-\diag(\tilde x))B$ is an irreducible singular $M$-matrix, as detailed below \eqref{eq:positive_linear_x}. Define the oblique projection matrix $R = I - \tilde x \tilde u^\top$. Define also $\zeta = Rx$, and note that $\tilde u^\top \zeta = 0$ and $\zeta = {\bf 0} \Leftrightarrow x = \alpha \tilde x$ for some $\alpha \in \mathbb R$. That is, $\zeta$ is always orthogonal to $\tilde u$, and $\zeta$ is the zero vector precisely when $x$ is in the span of $\tilde x$.

Observe that $\dot \zeta(t) = R\dot{x}(t)$. Substituting in the right of \eqref{eq:virus_system_plane} for $\dot{x}(t)$, we obtain
\begin{equation}\label{eq:virus_error_system}
    \dot{\zeta}(t) = -Q\zeta(t) + R(\diag(\tilde x)-Z(t))Bx(t),
\end{equation}
by exploiting the fact that $QR = Q = RQ$. %We shall study the dynamics of the non-autonomous system \eqref{eq:virus_error_system}, under time-varying inputs $Z(t)$ and $x(t)$, with $x(t)$ bounded in $[0,1]^n$. %and $\diag(\tilde x)-Z(t) \to {\bf 0}_{n\times n}$ exponentially fast.

Consider the Lyapunov-like function
\begin{equation}\label{eq:V}
    V = \zeta(t)^\top P \zeta(t),
\end{equation}
with $P$ defined below \eqref{eq:positive_linear_x}. It is positive definite in $\zeta$. Differentiating $V$ with respect to time %along the trajectories of \eqref{eq:virus_error_system}
yields  \footnotesize
\begin{align}\label{eq:V_dot}
    \dot{V} & =  -2\zeta(t)^\top PQ\zeta(t) +2\zeta(t)^\top PR(\diag(\tilde x)-Z(t))Bx(t) \nonumber \\
    & = -\zeta(t)^\top \bar Q \zeta(t) +2\zeta(t)^\top PR(\diag(\tilde x)-Z(t))Bx(t).
\end{align}
\normalsize 
%with $\bar Q$ defined below \eqref{eq:positive_linear_x}. 
Due to submultiplicativity of matrix norms, $\Vert \zeta(t)\Vert \leq \Vert R \Vert \Vert x(t)\Vert$. Hence, since $\Vert x(t)\Vert$ is bounded, \eqref{eq:V_dot} yields  \footnotesize
%Since $\Vert \zeta(t)\Vert \leq \Vert R \Vert \Vert x(t)\Vert$, and $\Vert x(t)\Vert$ is bounded, \eqref{eq:V_dot} yields
\begin{align}
    \dot{V} & \leq -\zeta(t)^\top \bar Q \zeta(t) + \kappa \Vert \tilde x - z(t) \Vert %\nonumber \\
    %&
    \leq -\zeta(t)^\top \bar Q \zeta(t) + \bar a e^{-b t}, \label{eq:V_dot_ineq_1}
\end{align}
\normalsize 
where $\kappa$ and $\bar a$ are positive constants, and the second inequality is due to the fact that $\Vert \tilde x - z(t) \Vert \leq ae^{-bt}$.

%For the positive semidefinite symmetric matrix $\bar Q$, 
Let $\lambda_2$ denote the %its
smallest strictly positive eigenvalue of $\bar Q$. The Courant-Fischer min-max theorem~\cite[Theorem~8.9]{zhang2011matrix} yields %that
\begin{equation}\label{eq:cf_ineq}
    \frac{\zeta^\top \bar Q \zeta}{\zeta^\top \zeta} \geq \min_{\substack{v^\top \tilde u =  0\\v^\top v = 1}} v^\top \bar Q v = \lambda_2,
\end{equation}
for all $\zeta \neq {\bf 0}$ perpendicular to $\tilde u$. Since $\zeta^\top \tilde u = 0$ holds for all $\zeta \in \mathbb R$ by definition, \eqref{eq:cf_ineq} holds for all $\zeta\neq {\bf 0}$. Next, and recalling the definition of $P$ below \eqref{eq:positive_linear_x}, it follows that
\begin{equation}\label{key:ineq:1:lyap}
    \underline{p}\zeta^\top \zeta \leq \zeta^\top P \zeta \leq \bar p \zeta^\top\zeta,
\end{equation}
for all $\zeta$, where $\underline{p} = \min_{i\in [n]} \tilde u_i/\tilde x_i$ and $\bar p = \max_{i\in[n]} \tilde u_i/\tilde x_i$. 

From~\eqref{eq:V_dot_ineq_1} it follows that $\dot{V} \leq -\lambda_2 \zeta(t)^\top \zeta + \bar a e^{-bt}$, which further implies that $ \dot{V} \leq -\bar \lambda \bar p\zeta(t)^\top \zeta + \bar a e^{-bt}$, where $\bar \lambda = \lambda_2/\bar p$. Consequently, from~\eqref{key:ineq:1:lyap}, it follows that $ \dot{V} \leq -\bar\lambda V + \bar a e^{-bt}$. Thus, and recalling the definition of $\zeta$, it follows that $\lim_{t\to\infty} x(t) = \alpha \tilde x$ exponentially fast, where $\alpha \in (0,1)$ because $x(t) \in (0,1)^n$ for all $t \geq \tau$, for some positive $\tau$.

% It follows that \eqref{eq:V_dot_ineq_1} implies that
% \begin{align}
%     \dot{V} \leq -\lambda_2 \zeta(t)^\top \zeta + \bar a e^{-bt},
% \end{align}
% which further implies that
% \begin{align}
%     \dot{V} \leq -\bar \lambda \bar p\zeta(t)^\top \zeta + \bar a e^{-bt},
% \end{align}
% where $\bar \lambda = \lambda_2/\bar p$. It follows that
% \begin{equation}
%     \dot{V} \leq -\bar\lambda V + \bar a e^{-bt}.
% \end{equation}
% The differential equation $\dot{W} = -\bar\lambda W + \bar ae^{-bt}$ has the solution
% \begin{align}
%     W(t) & = W(0)e^{-\bar\lambda t} + \bar a \int_{0}^t e^{-\bar\lambda(t-s)} e^{-bs} .\text{d}s \\
%     & = W(0)e^{-\bar\lambda t} + \bar a e^{-\bar\lambda t} \int_{0}^t e^{-s(b-\bar\lambda)} .\text{d}s \\
%     & = W(0)e^{-\bar\lambda t} + \frac{\bar a}{b-\bar\lambda} \left(e^{-\bar\lambda t} - e^{-bt}\right).
% \end{align}
% Note that $\frac{\bar a}{b-\bar\lambda}(e^{-\bar\lambda t} - e^{-bt}) \geq 0$ for any positive $b$ and $\bar\lambda$. 
% %This is because $e^{-\bar\lambda t} > e^{-bt}$ for all $t\geq 0$ if and only if $b > \bar\lambda$. 
% Thus, $W(t) \geq 0$ if $W(0) \geq 0$, and $\lim_{t\to\infty} W(t) = 0$ at an exponentially fast rate. By the Comparison Lemma~\cite[Lemma~3.4]{khalil2002nonlinear}, $V(t) \leq W(t)$ and $\lim_{t\to\infty} V(t) = 0$ at an exponentially fast rate. Thus, and recalling the definition of $\zeta$, it follows that $\lim_{t\to\infty} x(t) = \alpha \tilde x$ exponentially fast, where $\alpha \in (0,1)$ because $x(t) \in (0,1)^n$ for all $t \geq \tau$, for some positive $\tau$.}

This analysis holds not only for virus~$1$, but also virus~$2$ and virus~$3$. In other words, $\lim_{t\to\infty} x^i(t) = \alpha_i \tilde x$ for some $\alpha_i \in (0,1)$, for all $i\in [3]$. Recall that $\lim_{t\to\infty} z(t) = \tilde x$, and we immediately conclude that $\sum_{i=1}^n \alpha_i = 1$, thus proving statement i). %The first statement of the theorem is thus proved.

\textit{Proof of statement ii)}: We now prove that every point in $\mathcal{E}$ is an equilibrium (coexistence follows trivially by definition). Consider an arbitrary point $(x^1, x^2, x^3)$ in $\mathcal{E}$. From \eqref{eq:x1}, we have \vspace{-4mm}
\begin{align}
    \dot{x}^1(t) & = (-D+(I-X^1-X^2-X^3)B)x^1 \\
    & = (-D+(I-\diag(\tilde x))B)\alpha_1 \tilde x = {\bf 0}.
\end{align}
By the same arguments, it follows that $\dot{x}^2(t) = {\bf 0}$ and $\dot{x}^3(t) = {\bf 0}$ at the point $(x^1, x^2, x^3)$ in $\mathcal{E}$. In other words, $(x^1, x^2, x^3)$ is an equilibrium of the system system~\eqref{eq:x1}-\eqref{eq:x3}. Since this holds for any arbitrary point in $\mathcal{E}$, the proof of statement~ii) is complete. \qed%second claim is proved. %\qed

\seb{We identify yet another special case of the tri-virus model parameters
% but one that subsumes the setting covered in Theorem~\ref{thm:global:plane}, 
that admits the existence of a plane of 3-coexistence equilibrium. To this end, let $B^1$ be some arbitrary nonnegative irreducible matrix with $z$ being as defined in~\eqref{eq:z}, and $B^2$ being as defined in~\eqref{eq:B2}. Let $\hat{C} \neq C$ be a nonnegative irreducible matrix such that $z$ is the eigenvector corresponding to eignevalue unity, that is, $\hat{C}z=z$. Define \begin{equation}\label{eq:B3}
    B^3:=(I-Z)^{-1}\hat{C}.
\end{equation}
%and 
% \begin{equation}\label{eq:B3}
%     B^3:=(I-Z)^{-1}C
% \end{equation}
We have the following result.}
\seb{
\begin{prop} \label{prop:plane:equilibrium}
%\seb{
Consider system~\eqref{eq:x1}-\eqref{eq:x3} under Assumptions~\ref{assum:base} and~\ref{assum:irreducible}. %Suppose that $D^k=I$ for $k \in [3]$. 
Suppose that 
    $B^1$ is an arbitrary nonnegative irreducible matrices; and
    vector $z$, and matrices $B^2$ and $B^3$ are as defined in~\eqref{eq:z}, and~\eqref{eq:B2} and~\eqref{eq:B3}, respectively. Then, a set of equilibrium points of the trivirus equations is given by $(\alpha_1z, \alpha_2z, \alpha_3z)$, with $\alpha_i \geq 0$ for $i=1,2,3$, and  $\sum_{i=1}^3\alpha_i=~1$. 
\end{prop}
\textit{Proof:}
Assume, without loss of generality, that  $D^k=I$ for $k \in [3]$.
Observe, then,  that the right hand side of~\eqref{eq:x1}-\eqref{eq:x3} evaluated at  $(\alpha_1z, \alpha_2z, \alpha_3z)$ yields:
\begin{align}
    &(-I+(I-\alpha_1Z-\alpha_2Z- \alpha_3Z)B^1)\alpha_1z \nonumber \\
    &=(-I+(I-Z)B^1)\alpha_1z =0, \label{alpha1z=0}
\end{align}
where~\eqref{alpha1z=0} follows by noting that by assumption, $\alpha_1$ is a nonnegative scalar, and $\sum_{i=1}^3\alpha_i=1$,  and  $z$ is the single-virus endemic equilibrium corresponding to virus~1. Similarly, 
\begin{align}
    &(-I+(I-\alpha_1Z-\alpha_2Z- \alpha_3Z)B^2)\alpha_2z \nonumber \\
    &=(-I+(I-Z)B^2)\alpha_2z \nonumber \\
     &=(-I+(I-Z)(I-Z)^{-1}C)\alpha_2z \nonumber \\
     &=(-I+C)\alpha_2z=0, \label{alpha2z=0}
\end{align}
where~\eqref{alpha2z=0} follows by noting that $Cz=z$.
Finally, 
\begin{align}
    &(-I+(I-\alpha_1Z-\alpha_2Z- \alpha_3Z)B^3)\alpha_3z \nonumber \\
    &=(-I+(I-Z)B^3)\alpha_3z \nonumber \\
     &=(-I+(I-Z)(I-Z)^{-1}\hat C)\alpha_3z \nonumber \\
     &=(-I+\hat C)\alpha_3z=0, \label{alpha3z=0}
\end{align}
where~\eqref{alpha3z=0} follows by noting that $\hat{C}z=z$.
Thus, from~\eqref{alpha1z=0}, \eqref{alpha2z=0} and~\eqref{alpha3z=0}, it is clear that, for every $\alpha_i\geq 0$ with $i=1,2,3$, such that $\sum_{i=1}^3 \alpha_i=1$,  $(\alpha_1z, \alpha_2z, \alpha_3z)$ is an equilibrium point of system~\eqref{eq:x1}-\eqref{eq:x3}.\hfill \qed %i.e. there is a set of equilibrium points $(\beta_1z, (1-\beta_1)z, \textbf{0})$ with $\beta_1 \in [0,1]$.
\par Note that Proposition~\ref{prop:plane:equilibrium} and, assuming the setting in \cite{pare2021multi} is limited to the tri-virus case, \cite[Corollaries~2 and ~3]{pare2021multi} identify special cases that admit the existence of a plane of co-existence equilibria. Since  Proposition~\ref{prop:plane:equilibrium}  covers a larger class of parameters than that in \cite[Corollaries~2 and ~3]{pare2021multi}, Proposition~\ref{prop:plane:equilibrium} subsumes \cite[Corollaries~2 and ~3]{pare2021multi}.


Proposition~\ref{prop:plane:equilibrium} and Theorem~\ref{thm:global:plane} compare in the following sense: 
% First observe that, for the class of parameters covered by  Theorem~\ref{thm:global:plane}, the existence of a plane of coexistence equilibria has been addressed in \cite[Corollary~3]{pare2021multi}. However, \cite[Corollary~3]{pare2021multi} provides no stability guarantees for said plane; a gap addressed by Theorem~\ref{thm:global:plane}.
On the one hand, note that if $(I-Z)B^1=C=\hat{C}$, then, in view of~\eqref{eq:B2} and~\eqref{eq:B3}, it follows that $B^1=B^2=B^3$, which coincides with the setting of Theorem~\ref{thm:global:plane}. Thus,   
Proposition~\ref{prop:plane:equilibrium} covers a larger class of parameters than Theorem~\ref{thm:global:plane}, and, therefore, with respect to existence of a set of 3-coexistence equiibria, is more general.  On 
 the other hand,  Proposition~\ref{prop:plane:equilibrium} provides no stability guarantees for the said equilbrium set, whereas Theorem~\ref{thm:global:plane} establishes global stability of the  equilbrium set for the smaller class of parameters that it covers. %As a consequence, neither Proposition~\ref{prop:plane:equilibrium} nor Theorem~\ref{thm:global:plane}  subsume one another.
}
\par \seb{Clearly, with respect to the existence of a set of equilibria, Proposition~\ref{prop:plane:equilibrium} is stronger than Theorem~\ref{thm:init:condns}. That said, note that Theorem~\ref{thm:init:condns} admits  arbitrary $B^3$, whereas Proposition~\ref{prop:plane:equilibrium} insists on $B^3$ obeying a specific functional form, as given in~\eqref{eq:B3}. In conclusion, neither subsumes the other.}

\section{Simulations}

\section{Conclusion}

\bibliographystyle{siamplain}
\bibliography{references}
\end{document}
