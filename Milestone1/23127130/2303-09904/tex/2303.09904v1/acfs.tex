% !TEX root = main.tex

\section{Amplitude-like functions}
\label{sec:acfs}
The goal of this section is to review analytic properties of tree-level scattering amplitudes. We work in the setting of ref.~\cite{Remmen:2021zmc}, and we consider a two-to-two scattering of massless scalars described by an amplitude that can be cast in the form:
\beq\label{eq:fac_form}
\cM(s,u) = \cA(s)+\cA(u) = \cA(s)+\cA(-s-t)\,.   
\eeq
To be concrete, we may consider a theory with two distinct massless scalars $\phi_1$ and $\phi_2$, and we consider the scattering $\phi_1\,\phi_2\to\phi_1\,\phi_2$.\footnote{Note that not every theory of this type has necessarily amplitudes that can be cast in the form in eq.~\eqref{eq:fac_form}.}
In the following we discuss some general properties that any such amplitude must have:
\begin{enumerate} 
\item {\underline{Bose symmetry.}} For a scattering of the type $\phi_1\,\phi_2\to\phi_1\,\phi_2$, Bose symmetry implies $\cM(s,u) = \cM(u,s)$. This condition is automatically fulfilled if we work with the factorised form in eq.~\eqref{eq:fac_form}.
\item {\underline{Meromorphicity and simple poles on the positive real line.}}
 It is well known that tree-level amplitudes for scalar scattering are rational functions of the Lorentz invariant products of the four-momenta. 
 The only poles at finite values of the invariants arise from propagators going on shell, and locality dictates that all poles must be simple. In our scenario, this implies that $\cA(s)$ is a meromorphic function of $s$ with simple poles at most at $s=m_n^2\ge 0$. Note that the spectrum $m_n^2$ of exchanged particles cannot have any accumulation point, because a meromorphic function can only have isolated singularities.
 %
 %
\item {\underline{Negative residues at all poles.}} Close to the simple pole at $s=m_n^2$, the amplitude behaves like
%\footnote{We follow the conventions of ref.~\cite{Remmen:2021zmc} and work in the `mostly plus' convention for the metric.}  
$i\cM(s,u)\sim \frac{-ig^2}{s-m_n^2}$, where $g$ denotes the coupling constant of the interaction between the external scalars $\phi_1$, $\phi_2$ and the state $X_n$ of mass $m_n$ exchanged in the $s$-channel. If we want the coupling $\phi_1\phi_2X_n$ to be real, we must have $g^2>0$. Hence, we conclude that we must have
\beq
\textrm{Res}_{s=m_n^2} \cA(s) = \oint\frac{\rd s}{2\pi i}\cA(s) = -g^2 < 0\,,\textrm{~~~for all poles $s=m_n^2$.}
\eeq
%
%
\item {\underline{Polynomial boundedness.}} It is well known that in any QFT scattering amplitudes are polynomially bounded. More precisely, we must have for every fixed value of $t = -s-u$:
\beq
\lim_{s\to\infty}\cM(s,-s-t)s^{-N} = 0\,,\textrm{~~~for some positive integer $N$.}
\eeq
Here this is equivalent to 
\beq
\lim_{s\to\infty}\cA(s)s^{-N} = 0\,,\textrm{~~~for some positive integer $N$.}
\eeq
%
%
\item {\underline{Positivity constraints.}} In ref.~\cite{Adams:2006sv} it was shown that for $t=0$, we must have
%\footnote{We may need to subtract poles due, e.g., the an exchange of massless particles. Since this does not change our argument, we will not dwell on this technical point further.}
\beq
\textrm{Res}_{s=0}\frac{\overline{\cM}(s,-s)}{s^{L+1}}=\oint\frac{\rd s}{2\pi i} s^{-N-1}\overline{\cM}(s,-s) > 0\,,
\eeq
where $\overline{\cM}(s,-s)$ is obtained from ${\cM}(s,-s)$ by subtracting poles at $s=0$.
By Bose symmetry, $\overline{\cM}(s,-s) = \overline{\cM}(-s,s) = \overline{\cA}(s)+\overline{\cA}(-s)$ is an even function (and $\overline{\cA}(s)$ is obtained by subtracting the poles at $s=0$), and so the previous equation only constrains the even Taylor coefficients of $\cA(s)$ around $s=0$. We can of course use the same argument to derive positivity constraints for $u=0$, which allows us to put constraints directly on $\cA(s)$:
\beq
\textrm{Res}_{s=0}\frac{\overline{\cM}(s,0)}{s^{L+1}}=\oint\frac{\rd s}{2\pi i} s^{-N-1}\overline{\cA}(s) > 0\,.
\eeq
It is easy to see this implies that all Taylor coefficients of  $\cA(s)$ around $s=0$ must be positive.
\end{enumerate}

We will refer to a function $\cM(s,u)$ that satisfies these five conditions as an \emph{amplitude-like function}.
In ref.~\cite{Remmen:2021zmc} it is shown that the function $\cM(s,u)$ constructed from the function $\cA(s)$ in eq.~\eqref{eq:cA_Remmen} satisfies these five constraints. As a consequence, $\cM(s,u)$
is amplitude-like.\footnote{In ref.~\cite{Remmen:2021zmc} only the positivity of the even Taylor coefficients was imposed. It is easy to see that  our requirement is stronger and strictly contains the positivity constraint on the even Taylor coefficients. As we will see later on, the function $\cA(s)$ from ref.~\cite{Remmen:2021zmc} (see eq.~\eqref{eq:cA_Remmen}) also satisfies our stronger requirement.}

%%%%%%%%%%%%%%%%%%%%%%%%%%%%%%%%%%%%%%%%%%%%%%
%%%%%%%%%%%%%%%%%%%%%%%%%%%%%%%%%%%%%%%%%%%%%%




