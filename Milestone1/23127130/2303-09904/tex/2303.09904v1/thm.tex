% !TEX root = main.tex

\section{Amplitude-like functions from entire functions}
\label{sec:thm}



In this section we present the main result of our paper, namely we present a general construction of amplitude-like functions from a very large class of entire functions. We start by reviewing some mathematical background on entire functions and we present our result and discuss its consequences in section~\ref{sec:thm_result}. The proof is presented in section~\ref{sec:proof}.

\subsection{Entire functions}
\label{sec:entire_funcs}

In this section we review some standard material in complex analysis, in particular entire functions.
Recall that a function $f:\mathbb{C}\to \mathbb{C}$ is said to be \emph{entire} if it is holomorphic everywhere on the complex plane. Stereotypical examples of entire functions are polynomials and the exponential function. 

As a consequence of Liouville's theorem, every non-constant entire function must be unbounded. It will be useful to consider how an entire function behaves at infinity.
 We say that an entire function $f$ has \emph{order at most $\rho$} if there is $R>0$ and $C\ge0$ such that $|f(z)|<C\exp(|z|^\rho)$ for all $|z|>R$. The smallest such $\rho$ is called the \emph{order of $f$}. If $f$ has order at most $\rho$, then this means that $\log |f(z)|$ grows at most like $|z|^\rho$ for large $|z|$. This implies in particular that $\log|f(z)|$ is polynomially bounded:
 \beq
 \lim_{z\to \infty} z^{-d-1}\,\log|f(z)| = 0\,, 
 \eeq
 where we defined $d := \lfloor \rho\rfloor$, i.e., $d$ is the largest integer less or equal than $\rho$.
 In the following we will only consider entire functions of finite order.


An entire function may have zeroes, and since $f$ is holomorphic, its set of zeroes
%\beq
%Z_f = \{z\in\mathbb{C}:f(z)=0\}\,,
%\eeq
cannot have any accumulation point. Every entire function of finite order $\rho$ can be cast in a standard form using \emph{Hadamard's factorisation theorem}:
\beq\label{eq:hadamard}
f(z) = e^{g(z)}\,z^m\,\prod_{n}\E_d\left(\frac{z}{z_n}\right)\,.
\eeq
where $d=\lfloor\rho\rfloor$, $g$ is a polynomial of degree at most $d$, $m$ is the order of $f$ at 0 and the product runs over all zeroes $z_n\neq0$ of $f$ counted with multiplicity. The function $\E_d(z)$ is the \emph{elementary factor}, defined by
\beq
\E_d(z) := \left\{\begin{array}{ll}
1-z\,, & \textrm{ if } d=0\,,\\
(1-z)\exp\left[\sum_{k=1}^d\frac{z^k}{k}\right]\,,&\textrm{ if } d>0\,.
\end{array}\right.
\eeq
Note that in the case where $f$ has an infinite number of zeroes, the product in eq.~\eqref{eq:hadamard} runs over an infinite number of terms. This case requires some careful consideration regarding the convergence of this infinite product. One can show (see, e.g.,~ref.~\cite{hadamard}) that the infinite product in eq.~\eqref{eq:hadamard} converges if and only if we have 
\beq\label{eq:convergence}
\sum_n\frac{1}{|z_n|^{d+1}}<\infty\,.
\eeq







In the following it will be useful to introduce the following notations.
Hadamard's factorisation theorem in eq.~\eqref{eq:hadamard} implies that we can write every function of finite order in the form
\beq
f(z) = z^m\,\cE_f(z)\,\cP_f(z)\,,
\eeq
where $\cE_f(z):= e^{g(z)}$ has no zeroes and $\cP_f(z) := \prod_{n}\E_d\left(\frac{z}{z_n}\right)$ has the form of an infinite product. 
Note that if $f_1$ and $f_2$ are two entire functions of order at most $\rho$, then so is their product, and we have
\beq
\cE_{f_1f_2}(z) = \cE_{f_1}(z) \cE_{f_2}(z) \textrm{~~~and~~~}\cP_{f_1f_2}(z) = \cP_{f_1}(z) \cP_{f_2}(z) \,.
\eeq
Finally, let us discuss some important consequence of eq.~\eqref{eq:convergence}. Consider a (possibly infinite) sequence $(z_n)_n$ of non-zero complex numbers such that eq.~\eqref{eq:convergence} holds. Then there is an entire function $f$ of order $d<\infty$ with precisely those zeroes. Indeed, Hadamard's theorem allows to easily construct such a function: it is simply the infinite product $\cP_f(z)$. In fact there are infinitely many such functions, and they differ precisely by an exponential factor $\cE_f(z)=e^{g(z)}$, where $g$ is a polynomial of degree at most $d$.



\subsection{The main result}
\label{sec:thm_result}
We now discuss our main result, which generalises the result of ref.~\cite{Remmen:2021zmc} from the entire function $\Xi$ in eq.~\eqref{eq:xi_def} to an infinite class of entire functions. We first need to restrict the class of entire functions $f$ that we will consider. First, the zeroes of $f$ will be related to the poles of the putative amplitude, so $f$ should only have zeroes on the real axis. Second, the propagator poles are related to  squared masses of the exchanged states, so we expect the zeroes to come in pairs $\pm z_n\neq0$. This gives, for functions of order at most $\rho$ (with $d=\lfloor \rho\rfloor$):
\beq\label{eq:E_d_sym}
\cP_{f}(z) = \prod_n\E_d\left(\frac{z}{z_n}\right)\E_d\left(-\frac{z}{z_n}\right) = \prod_n{\E}_{\lfloor d/2\rfloor}\left(\frac{z^2}{z_n^2}\right)\,,
\eeq
where in the last equality the product runs over the distinct zeroes of $f$ located on the positive real axis. Note that in this case $\cP_{f}(z)$ is an even function, $\cP_{f}(-z)=\cP_{f}(z)$, so that $\cP_{f}(\sqrt{z})$ defines an entire function of order at most $\rho/2$, with zeros of order $k_n$ at $z=z_n^2>0$. 

We will from now on focus on even entire functions $f(z)$ with zeroes on the real line. For such functions the zeroes always come in pairs $\pm z_n$, with $z_n>0$. In addition, $f$ may have a pole of order $m$ at $z=0$. Moreover, if $f$ is an even and entire function of order at most $\rho$, then $f(\sqrt{z})$ is en entire function of order at most $\rho/2$ (but it is not necessarily even). We then define (cf.~eq.~\eqref{eq:cA_Remmen}):
%\beq
%\cP_{f}(\sqrt{z}) = \prod_n{\E}_{\lfloor d/2\rfloor}\left(\frac{z}{z_n^2}\right)\,,
%\eeq
%where the product runs over all zeros of $f$ at $z=z_n>0$. We also define:
\beq
\cA_f(z) := -\frac{\rd}{\rd z}\log f(\sqrt{z}) = -\frac{\rd}{\rd z}\log\left[z^{m}\,\cE_f(\sqrt{z})\,\cP_{f}(\sqrt{z})\right]\,,
\eeq
where we use the notation 
\beq\label{eq:E_f_def}
\cE_f({z}) := \exp\left[\sum_{k=0}^{d} g_k\,z^k\right]\,,
\eeq
with $g_k$ some complex numbers. Note that, since $\cA_f(z) = \cA_{cf(z)}$ for every non-zero complex number $c$, we can assume without loss of generality $g_0=0$. 

The following notation will be useful:
\beq\label{eq:c_def}
c_{f,k} = \left\{\begin{array}{ll}
\sum_{n}\frac{1}{z_n^{2(k+1)}}\,, &\textrm{~~if~~}k \ge \lfloor d/2\rfloor\,,\\
0\,, &\textrm{~~if~~}k  <  \lfloor d/2\rfloor\,,
\end{array}\right.
\eeq
where $k$ is a positive integer.
It is easy to see that the series $\sum_{n}\frac{1}{z_n^{2(k+1)}}$ converges for $k\ge \lfloor d/2\rfloor$. Indeed, the only case that needs checking is when $f$ has an infinite number of zeroes $z_n$. Since the set of zeroes of $f$ has no accumulation point, the sequence $(|z_n|^2)_{n}$ is unbounded. Hence, there is some positive integer $N$ such that $|z_n|^2>1$ for $n>N$. This implies $|z_n|^{2(k+1)}\ge |z_n|^{2( \lfloor d/2\rfloor+1)}$, for all $n>N$ and $k\ge  \lfloor d/2\rfloor$. Convergence of $\sum_{n}\frac{1}{|z_n|^{2(k+1)}}$ then follows by comparing with the convergence criterion in eq.~\eqref{eq:convergence} applied to the infinite product in eq.~\eqref{eq:E_d_sym}.

Our main result is summarised in the following theorem:
\begin{thm} Let $f:\mathbb{C}\to\mathbb{C}$ be an even and entire function such that
\begin{enumerate}
\item $f$ has finite order $\rho$,
\item $f$ only has zeroes on the real line,
\item the $g_k$ are real and negative for all $1\le k\le \lfloor\rho\rfloor$.
\end{enumerate}
Then the function $\cM_f(s,u) := \cA_f(s)+\cA_f(u)$ is amplitude-like, i.e., it satisfies the 5 properties given in section~\ref{sec:acfs}.
\end{thm}
The proof of this theorem will be given in section~\ref{sec:proof} below. 
 Let us make some comments about the last condition, which is the most constraining one. First, we note that for an even function $f$, we have $g_{2k+1}=0$, so that the last condition really only applies to the even coefficients $g_{2k}$. Second, it is equivalent to $0\le \cE_f(z) < 1$ for all $z\in \mathbb{R}$. Finally, the last condition is always satisfied for even entire functions of order $\rho<2$. Indeed, if  $\rho<2$, we have $d\le 1$, and so eqs.~\eqref{eq:E_d_sym} and~\eqref{eq:E_f_def} imply:
\beq
\cP_f(z) = \prod_n\left(1-\frac{z^2}{z_n^2}\right)\textrm{~~~and~~~} \cE_f(z) = e^{g_0}=C\,,
\eeq
for some constant non-zero complex number $C$. Equivalently, we can write
\beq
f(z) = C\,z^{2m}\,\prod_n\left(1-\frac{z^2}{z_n^2}\right)\,.
\eeq
From here it is easy to see that the third condition of the theorem is always satisfied in this case, and we have:
\begin{corollary}
Let $f:\mathbb{C}\to\mathbb{C}$ be an even and entire function of finite order $\rho<2$ with zeroes only on the real line. Then the function $\cM_f(s,u)$ is amplitude-like.
\end{corollary}

In the remainder of this section we will discuss some general implications of our theorem.
%
First, we see can see that our theorem contains the results of ref.~\cite{Remmen:2021zmc} for the Riemann zeta function as a special case. Indeed, while the Riemann zeta function $\zeta(z)$ has a pole at $z=1$ (and is thus not entire), the function $\Xi(z)$ is an even and entire function of order 1, and we have:
\beq
\Xi(z) = \prod_{n=1}^{\infty}\left(1-\frac{z^2}{\mu_n^2}\right)\,,
\eeq
where the $\mu_n>0$ are the imaginary parts of the non-trivial zeroes of the Riemann zeta function. Assuming the Riemann hypothesis, the second condition of the theorem is satisfied.
Hence, the theorem applies, and $\cM_{\Xi}(s,u)$ is amplitude-like, in agreement with the findings of ref.~\cite{Remmen:2021zmc}. 


%Our theorem also allows to obtain other examples of amplitude-like functions, and in section~\ref{sec:examples}, we construct infinite classes of examples. This shows that the case of the Riemann zeta function considered in 
%ref.~\cite{Remmen:2021zmc} is not special in any way, and in particular the conclusions of ref.~\cite{Remmen:2021zmc} do not rely on specific properties of the Riemann zeta function (other than the Riemann hypothesis, to ensure that all zeroes lie on the real line), but rather they follow from very general considerations about entire functions. 

We may ask if there are other entire functions that satisfy the assumptions of the theorem. In the following we argue that there are infinitely many such functions. Indeed, consider a sequence of positive real numbers $(z_n)_n$ without accumulation point such that $\sum_{n}\frac{1}{z_n^{d+1}}<\infty$. Then we know from Hadamard's factorisation theorem that there is an entire function $f$ of order at most $\rho$ with $d=\lfloor\rho\rfloor$ and with zeroes precisely at $z=\pm z_n$. We may pick:
\beq
f(z) = \cP_f(z) = \prod_n\E_d\left(\frac{z}{z_n}\right)\E_d\left(-\frac{z}{z_n}\right) = \prod_n{\E}_{\lfloor d/2\rfloor}\left(\frac{z^2}{z_n^2}\right)\,.
\eeq
It is easy to check that this function satisfies all the hypotheses of our theorem, and so the function $\cM_f(s,u)$ is amplitude-like. We thus see there is nothing special about the sequence $(\mu_n)_n$ of non-trivial zeroes of the Riemann $\zeta$ function, but the same conclusion holds for pretty much every sequence of real positive numbers that satisfy eq.~\eqref{eq:convergence}.
%
Finally, we mention that it is easy to see that if $f_1$ and $f_2$ satisfy the hypotheses of our theorem, then so does their product, and we have 
\beq\label{eq:amalgam}
\cA_{f_1f_2} = \cA_{f_1}+\cA_{f_2} \textrm{~~~and~~~} \cM_{f_1f_2} = \cM_{f_1}+\cM_{f_2}\,.
\eeq
Hence, also $\cM_{f_1f_2}$ 
 is amplitude-like.


\subsection{The proof of the theorem}
\label{sec:proof}
In this section we present the proof of our main theorem. We need to show that any $f$ that satisfies the hypotheses of the theorem also satisfies the five properties of section~\ref{sec:acfs}. Bose-symmetry (Property 1) is manifest, and there is nothing to check. 

Let $F:\mathbb{C}\to\mathbb{C}$ be a function analytic on some domain $U$. It is easy to see that
\beq
\frac{\rd}{\rd z}\log F(z) = \frac{F'(z)}{F(z)}
\eeq
has singularities at the zeros of $F$. If $F$ has a zero of order $M$ at $z=z_0\in U$, then $F'$ has a zero of order $M-1$ there, to that $F'/F$ has a simple zero at $z=z_0$. An easy application of the residue theorem shows that
\beq
\oint \frac{\rd z}{2\pi i}\,\frac{F'(z)}{F(z)} =  M > 0\,.
\eeq
We take $F=\cA_f$, and we have:
\beq
\cA_f(z) = -\frac{m}{z} - \frac{\cE_f'(z)}{\cE_f(z)} -\frac{\widetilde\cP_f'({z})}{\widetilde\cP_f({z})}\,,\qquad \widetilde\cP_f({z}):=\cP_f(\sqrt{z})\,.
\eeq
The first term clearly has a simple pole at $z=0$ with negative residue $-m$. The second term has no poles, because $\cE_f(z) = e^{g(z)}$ vanishes nowhere. The last terms has simple poles at $z=z_n^2$, with $z_n>0$. Hence,
\beq
\cA_f(z)  \sim \frac{-k_n}{z-z_n^2}\,,\qquad \textrm{for } z\sim z_n^2\,.
\eeq
We conclude that $\cA_f$ satisfies Properties 2 \& 3 of section~\ref{sec:acfs}.

Let us now check that $\cA_f$ is polynomially bounded (Property 4). Since $f$ has finite order $\rho$ say, we have $\cA_f(z) \sim (\rho-1)|z|^{\rho-1}$, and so $\cA_f$ is clearly bounded by $|z|^{\lfloor \rho\rfloor}$. In other words, the fact that $f$ has finite order immediately translates into $\cA_f$ being polynomially bounded.

It remains to show that $\cA_f$ satisfies Property 5. In other words, we need to show that all the Taylor coefficients of  $\overline{\cA}_f$ around $z=0$ are positive. The Taylor expansion of $\overline{\cA}_f$ is easy to obtain. Indeed, we have
\beq
\overline{\cA}_f(z) = \cA_f(z) + \frac{m}{z} = -\sum_{k=0}^{\lfloor d/2\rfloor-1}(k+1)g_{2k+2}\,z^{k}+\sum_{k=\lfloor d/2\rfloor}^\infty c_{f,k}\,z^{k}\,,
\eeq
We know that the infinite series defining $c_{f,k}$ in eq.~\eqref{eq:c_def} are all convergent and have positive summands, and so $c_{f,k}\ge 0$. Hence, the only non-trivial positivity constraints are those involving $g_{2k+2}$. Those are precisely satisfied if the hypotheses of the theorem hold. This finishes the proof. 

