% !TEX root = main.tex

\section{Examples}
\label{sec:examples}

In the previous section we presented a theorem which allows us to construct an infinite class of amplitude-like functions. Here we present several concrete examples, and we identify in each case a QFT for which $\cM_f(s,u)$ computes a scattering amplitude.

\subsection{Nowhere vanishing entire functions and contact interactions}
Consider a nowhere vanishing even and entire function $f$ of order $d$. By Hadamard's factorisation theorem, such a function is necessarily the exponential of a polynomial of order $d$:
\beq
f(z) = \cE_f(z) = \exp\left[\sum_{k=0}^{\lfloor d/2\rfloor}g_{2k}\,z^{2k}\right]\,.
\eeq
It is easy to see that such a function satisfies the hypotheses of the theorem, provided that $g_{k}\le0$, $1\le k\le d$ (recall that we may assume without loss of generality $g_0=0$). We have
\beq
\cA_f(z) = \sum_{k=0}^{\lfloor d/2\rfloor-1}(k+1)(-g_{2k+2})z^k\,.
\eeq
It is easy to check that the function
\beq
\cM_f(s,u) = \cA_f(s)  + \cA_f(u)  = \sum_{k=0}^{\lfloor d/2\rfloor-1}(k+1)(-g_{2k+2})\,\left(s^k+u^k\right)
\eeq 
computes the tree-level scattering amplitude $\phi_1\phi_2\to\phi_1\phi_2$ in the QFT described by the Lagrangian\footnote{We have computed the Feynman rules for small values of $d$ with FeynRules~\cite{Alloul:2013bka}.}
\beq
\cL = \frac{1}{2}\partial_{\mu}\phi_1\partial^{\mu}\phi_1+\frac{1}{2}\partial_{\mu}\phi_2\partial^{\mu}\phi_2 - \sum_{k=0}^{\lfloor d/2\rfloor-1}\lambda_k\,\phi_1\phi_2(\partial_{\mu_1}\cdots \partial_{\mu_k}\phi_1)(\partial^{\mu_1}\cdots \partial^{\mu_k}\phi_2)\,,
\eeq
with
\beq
\lambda_0 = -\frac{g_1}{4}\textrm{~~~and~~~} \lambda_k = (-1)^k\,2^{k-1}\,(k+1)\,(-g_{k+1})\,,\qquad >0\,.
\eeq
We see that nowhere vanishing entire functions that satisfy the hypotheses of our theorem describe a scattering of four scalars induced by contact interactions, and the dimensions of the contact operators are related to the order of the entire function. For example, the simplest non-constant such function is the exponential $f(z) = e^{-z^2}$ of order 2, and it describes the scattering of four scalars induced by the dimension-four operator $\phi_1^2\phi_2^2$. Another example would be $f(z) = e^{-z^4}$, which has order 4, and it describes a scattering induced by the dimension-six operator $\phi_1\phi_2\partial_\mu\phi_1\partial^{\mu}\phi_2$.





%%%%%%%%%%%%%%%%%%%%%%%%%%%%%%%%%%


\subsection{Entire functions with zeroes}
Let us now turn to the case where $f$ has zeroes at $z=\pm z_n\neq 0$ (plus possibly a zero at the origin). It is sufficient to consider the case 
\beq
f(z) = z^m\cP_f(\sqrt{z}) = z^m\prod_n\E_{d'}\left(\frac{z}{z_n^2}\right)^{k_n}\,,
\eeq
where we defined $d':=\lfloor d/2\rfloor$. The product runs over the distinct zeroes $z_n\neq0$ of $f$ and $k_n$ denotes the multiplicity of that zero. Indeed, we know from the previous section that nowhere vanishing entire functions lead to contact interactions, and we can use eq.~\eqref{eq:amalgam} to construct the corresponding amplitude.

We start by noting that 
\beq
-\frac{\rd}{\rd z}\log\E_{d'}\left(\frac{z}{z_n^2}\right)^{k_n} =\sum_n \left(-\frac{k_n}{z_n^{2d'}}\right)\,\frac{z^{d'}}{z-z_n^2}\,,
\eeq
so that
\beq
\cA_f(z) = -\frac{m}{z}+\sum_n\left(-\frac{k_n}{z_n^{2d'}}\right)\,\frac{z^{d'}}{z-z_n^2}\,.
\eeq

Let us now discuss if there is a QFT with a scattering amplitudes for $\phi_1\phi_2\to \phi_1\phi_2$ given by $\cM_f(s,u) = \cA_f(s)+\cA_f(u)$. We will distinguish the two cases depending on the parity of $d'$.

If $d'=2\delta$ is even, it is straightforward to check that $\cM_f(s,u) = \cA_f(s)+\cA_f(u)$ computes the tree-level scattering $\phi_1\phi_2\to\phi_1\phi_2$ in the QFT described by the Lagrangian
\beq\bsp\label{eq:lag_even}
\cL = \frac{1}{2}\partial_{\mu}\phi_1\partial^{\mu}\phi_1 + \frac{1}{2}\partial_{\mu}\phi_2\partial^{\mu}\phi_2 + \sum_n&\frac{1}{2}\partial_{\mu}X_n\partial^{\mu}X_n - \frac{z_n^2}{2}\,X_n^2- \lambda_{\delta}^{\textrm{even}}\,\ord_{n,\delta}\,,
\esp\eeq
where we defined the operator
\beq\bsp
\ord_{n,\delta} &\,=  X_n\, (\partial_{\mu_1}\cdots\partial_{\mu_{\delta}}\phi_1)(\partial^{\mu_1}\cdots\partial^{\mu_{\delta}}\phi_2)\,,\qquad \lambda_{\delta}^{\textrm{even}} = \frac{2^{\delta}\sqrt{k_n}}{z_n^{2\delta}}\,.
\esp\eeq
We see that, just like in the case of nowhere vanishing entire functions, the order of $f$ is connected to the dimension of the operator. A special case is of course when $f$ is a polynomial.

If $d'$ is odd, then this does not work. Indeed, every Feynman diagram involves two insertions of the operator $\ord_{\delta}$, which necessarily leads to the even powers of $z$ in the numerator. One possibility is to consider instead a non-unitary version of the QFT described by eq.~\eqref{eq:lag_even}:
\beq\bsp
\cL = \frac{1}{2}\partial_{\mu}\phi_1\partial^{\mu}\phi_1 + \frac{1}{2}\partial_{\mu}\phi_2\partial^{\mu}\phi_2 + \sum_n&\partial_{\mu}X_n^{\dagger}\partial^{\mu}X_n - {z_n^2}\,X_n^{\dagger}X_n-X_n^{\dagger}\phi_1\phi_2- \lambda_{d'}^{\textrm{odd}}\ord_{n,d'}\,,
\esp\eeq
with
\beq
\lambda_{d'}^{\textrm{odd}} = \frac{2^{d'}k_n}{z_n^{2d'}}\,.
\eeq
Alternatively, one may consider a scattering of four distinct scalars, e.g., in a theory described by the Lagrangian
\beq
\cL = \sum_{i=1}^4  \frac{1}{2}\partial_{\mu}\phi_i\partial^{\mu}\phi_i + \sum_n\frac{1}{2}\partial_{\mu}X_n\partial^{\mu}X_n - \frac{z_n^2}{2}\,X_n^{2}-X_n\phi_3\phi_4- \lambda_{d'}^{\textrm{odd}}\ord_{n,d'}\,.
\eeq
It is then easy to check that the scattering amplitude for the process $\phi_1\phi_2\to\phi_3\phi_4$ is $\cM_f(s,u) = \cA_f(s)$.

%%%%%%%%%%%%%%%%%%%%%%%







