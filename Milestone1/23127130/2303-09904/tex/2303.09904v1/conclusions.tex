% !TEX root = main.tex

\section{Conclusions}
\label{sec:conclusions}

This paper was motivated by the recent ref.~\cite{Remmen:2021zmc}, where a function was constructed that has the properties of a tree-level scalar two-to-two scattering amplitude via the exchange of a tower of particles with a mass spectrum given by the non-trivial zeroes of the Riemann zeta function (assuming the Riemann hypothesis holds). Since the analytic structure of scattering amplitudes is constrained by physical considerations, the existence of such a function with the required properties seems (at first sight) surprising. This opens the intriguing possibility that there may be a QFT whose mass spectrum is related to the Riemann hypothesis. 
Reference~\cite{Remmen:2021zmc} did not provide any evidence for or against the existence of such a QFT, and answering this question will most likely remain beyond our abilities for a long time. Reference~\cite{Remmen:2021zmc} asked the question if it is possible to extend its construction to other classes of $L$-functions, or even to arbitrary entire functions. 

The purpose of our paper was to provide an answer to these questions. Our main result is a theorem which allows us to construct amplitude-like functions from very large classes of entire functions. The main tool is Hadamard's factorisation theorem, which allows one to represent every entire function $f$ of finite order as a product of an exponential factor and an infinite product that captures the location of the zeroes of $f$. Our theorem states that, under some mild assumptions on $f$, it is possible to construct an amplitude-like function $\cM_f$. The function-theoretic properties of $f$ directly translate into the constraints on scattering amplitudes from physics: the fact that $f$ has finite order implies the polynomial boundedness of $\cM_f$; the location of the zeroes determines the spectrum of exchanged particles; the positivity constraints on effective operators are related to the fact that Taylor coefficients can be represented as convergent series of positive powers. Finally, we showed that the exponential factor in the Hadamard factorisation is related to contact interactions, and the order of $f$ is related to the dimension of the operators that induce the scattering. 

Our theorem contains the result of ref.~\cite{Remmen:2021zmc} as a special case, and it immediately shows how to extend it to other classes of $L$-functions and entire functions.
This begs the question of what the implications are for the existence of a putative QFT with a scattering amplitude as given in ref.~\cite{Remmen:2021zmc}. While our theorem does not allow us to rule out the possibility that such a QFT exists, it implies that most likely there is nothing special about the non-trivial zeroes of the Riemann zeta function in this context, weakening arguments in favour of the existence of such a QFT. Another, and far more exciting possibility, could be that there are QFTs attached much more generally to entire functions, and the QFT attached to the Riemann zeta function is only a special case of a more general mechanism. Which of these two possibilities is correct goes beyond the scope of this paper, and could be the subject of future research.






