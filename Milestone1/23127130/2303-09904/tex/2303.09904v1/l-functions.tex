% !TEX root = main.tex

\section{Amplitude-like functions from $L$-functions}
\label{sec:L-functions}

In section~\ref{sec:thm} we proved our main result, which allows us to construct amplitude-like functions from large classes of even and entire functions. This answers one of the questions asked at the end of ref.~\cite{Remmen:2021zmc}. Reference~\cite{Remmen:2021zmc} also asked the question if it was possible to extend its results from the Riemann zeta function to other (Dirichlet) $L$-functions. In this section we show that the answer to this question is positive and follows directly from our theorem. 


There are various different classes of functions called $L$-functions in the mathematical literature. Many of these functions are defined axiomatically and belong to the so-called \emph{Selberg (S) class}. While we expect that our results related to amplitude-like functions remain true for all functions in the S-class, for simplicity of the exposition we restrict the discussion here to a subset of $L$-functions, the so-called Dirichlet $L$-functions, and we defer the discussion of the general case to appendix~\ref{app:S-class}.


A Dirichlet $L$-function can be defined through a series similar to the definition of the Riemann zeta function in eq.~\eqref{eq:zeta_def}:
\beq\label{eq:L-func}
L(z,\chi) = \sum_{n=1}^{\infty}\frac{\chi(n)}{n^z}\,,\qquad \Re(z)>1\,.
\eeq
Here $\chi(n)$ is a \emph{Dirichlet character}, i.e., a map $\chi: \mathbb{Z}\to \mathbb{C}$ satisfying the following conditions:
\begin{enumerate}
\item $\chi$ is multiplicative: $\chi(m\cdot n) = \chi(m)\chi(n)$\,,
\item $\chi$ is periodic with period $q$: $\chi(n+q) = \chi(n)$\,,
\item $\chi(n)$ is non zero only if $n$ and $q$ are co-prime, i.e., if gcd$(n,q)=1$.
\end{enumerate}
If the smallest period of $\chi$ is $q$, we say that $\chi$ is a character mod $q$. It is easy to check that every Dirichlet character mod $q$ must evaluate to a $q^{\textrm{th}}$ root of unity. Clearly, for the trivial character $\chi(n)=1$, $\forall n\in\mathbb{Z}$, the definition in eq.~\eqref{eq:L-func} reduces to the definition of the Riemann zeta function in eq.~\eqref{eq:zeta_def}. 

If $\chi_1$ is a character mod $q_1$, and if $q_1|q_2$, then we can define a character $\chi_2$ mod $q_2$ by $\chi_2(n) = \chi_1(n)$. The character $\chi_2$ is said to be \emph{induced by $\chi_1$}. A character that is not induced by any other character is called \emph{primitive}. In the following we restrict the discussion to primitive characters (because non-primitive characters are not expected to be in the $S$-class).

It is a fundamental property of $L$-functions that they satisfy functional equations similar to eq.~\eqref{eq:zeta_cont} for the Riemann zeta function.
If $\chi$ is a primitive Dirichlet character mod $q$, then the corresponding $L$-function satisfies the functional equation
\begin{equation}
\label{eq:L-func_eq}
L(z,\chi)=\frac{G(\chi)}{i^{\delta}}2^{z}\pi^{z-1}q^{-z}\sin\left(\frac{\pi}{2}(z+\delta) \right)\Gamma(1-z)L(1-z,\overline{\chi})\,,     
 \end{equation}
 where $\overline{\chi}$ is the complex conjugate of $\chi$, $\delta= \frac{1-\chi(-1)}{2}$, and
 $G(\chi)=\sum_{a=1}^{a=q}\chi(a)e^{2\pi i \frac{a}{q}}$ is the Gauss sum, which in case of primitive characters satisfies $|G(\chi)|=\sqrt{q}$. Just like in the case of the Riemann zeta function, the functional equation can be used to analytically continue the function to values $\Re(z)<1$. The $L$-function has zeroes in the complex plane. The \emph{generalised Riemann hypothesis} expresses the conjecture that the non-trivial zeroes of $L$ (i.e., those that are not captured by the prefactor in the functional equation~\eqref{eq:L-func_eq}) all lie on the critical line $\Re(z)=\frac{1}{2}$. 
 
Just like in the case of the Riemann zeta function, it is possible to construct a function that has zeroes only at the non-trivial zeroes of $L(z,\chi)$:
\begin{equation}
     \xi(z,\chi)=\left(\frac{q}{\pi}\right)^{\frac{z+\delta}{2}}\Gamma\left(\frac{z+\delta}{2}\right)L(z,\chi) \,.
 \end{equation}
 This function is analogous to the function $\xi(z)$ in eq.~\eqref{eq:xi_def}, and one can show that $ \xi(z,\chi)$ defines an entire function of order one (cf., e.g., ref.~\cite{S-class1}).
There is, however, an important difference between $L(z,\chi)$ for a non-trivial character $\chi$ and the Riemann zeta function. For the Riemann zeta function, the zeroes of $\xi(z)$ come in complex conjugate pairs $\frac{1}{2}\pm i\mu_n$, which is a necessary condition for the function $\Xi(z) := \xi\left(\frac{1}{2}+ iz\right)$ to be even. This in turn is one of the hypotheses for our theorem from section~\ref{sec:thm_result} to be applicable. This property does no longer hold for a non-trivial character $\chi$. Instead, if $\rho$ is a zero of $\xi(z,\chi)$, then so is $1-\bar{\rho}$ (and the generalised Riemann hypothesis implies $\rho = 1-\bar{\rho}$). As a consequence, the function $\xi\left(\frac{1}{2}+ iz,\chi\right)$ is in general not an even function of $z$, and so our theorem does not apply. Instead, we can consider the function
\beq
\Xi_{\chi}(z) := \xi\left(\frac{1}{2}+ iz,\chi\right)\,\xi\left(\frac{1}{2}+ iz,\overline{\chi}\right)\,.
\eeq
Since $\xi(z,\chi)$ and $\xi(z,\overline{\chi})$ are entire functions of order 1, the same holds true for $\Xi_{\chi}(z)$.
Using the functional equation~\eqref{eq:L-func_eq}, we can show that $\Xi_{\chi}(z)$ is an even function:
\beq
\Xi_{\chi}(-z) = \xi\left(\frac{1}{2}-iz,\chi\right)\,\xi\left(\frac{1}{2}- iz,\overline{\chi}\right)
= \xi\left(\frac{1}{2}+ iz,\overline{\chi}\right)\,\xi\left(\frac{1}{2}+ iz,{\chi}\right)= \Xi_{\chi}(z)\,,
\eeq
where the second step follows from the functional equation~\eqref{eq:L-func_eq}.
Hence, $\Xi_{\chi}(z)$ is an even entire function of order 1 with the Hadamard product representation
 \begin{equation}
    \Xi_{\chi}(z)=\Xi_{\chi}(0)\prod_{n} \left(1-\frac{z^2}{\mu_{\chi,n}^{2}}\right)^{k_{n}}\,,
\end{equation}
where $\frac{1}{2}+i\mu_{\chi,n}$ are the non-trivial zeroes of $L(z,\chi)$, and $k_n$ denote their multiplicity.
We can apply our theorem from section~\ref{sec:thm_result}, and we see that the following function is amplitude-like:
 \beq\bsp
\mathcal{A}_{\Xi_{\chi}}(s) &=  -\frac{\mathrm{d}}{\mathrm{d}s}\log{\Xi_{\chi}(\sqrt{s})} \\ 
 &= -\frac{i}{2\sqrt{s}}\Biggl[ \frac{L'(\frac{1}{2}+i\sqrt{s},\chi)}{L'(\frac{1}{2}+i\sqrt{s},\chi)}+ \frac{L'(\frac{1}{2}+i\sqrt{s},\overline{\chi})}{L'(\frac{1}{2}+i\sqrt{s},\overline{\chi})}  +\psi\left(\frac{1}{4}+\frac{i\sqrt{s}}{2}+\frac{\delta}{2}\right)+\log{\frac{q}{\pi}}         \Biggr]\\
&= \sum_{n}\frac{k_{n}}{-s+\mu_{\chi,n}^{2}}\,.
\esp\eeq
This shows that it is possible to construct an amplitude-like function not just for the Riemann zeta function, but for general $L$-functions, thereby answering the question asked in ref.~\cite{Remmen:2021zmc}. We emphasise that the functional equation~\eqref{eq:L-func_eq} plays an important role in proving that the function $\Xi_{\chi}$ is even. The discussion here strictly only applies tor Dirichlet $L$-functions with primitive character, which are special instances of the more general $L$-functions from the $S$-class. In appendix~\ref{app:S-class} we show that the arguments presented here can easily be extended to all $L$-functions from the $S$-class.

