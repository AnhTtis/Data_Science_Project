\RequirePackage[table]{xcolor}

%\pdfoutput=1
\documentclass[12pt]{article}
\usepackage{amssymb}   % for math
\usepackage{amsmath}
\usepackage{amsthm}
\usepackage{graphicx}
%\usepackage{axodraw4j}
\usepackage[sort]{cite}

\usepackage{multirow}
%dots in filenames
\usepackage{grffile}

%%% Franz commands:

\def\zbar{\bar{z}}
\def\beq{\begin{equation}}   
\def\eeq{\end{equation}}
\def\bea{\begin{eqnarray}}  
\def\eea{\end{eqnarray}} 
\def\nn{\nonumber}
\def\r{\right} 
\def\l{\left} 
\def\f21{{}_2F_{1}}
%\def\eps{\epsilon}
\def\N{\mathcal{N}}
\def\order{\mathcal{O}}
\def\d{ \mathrm{d}}
  \def\O{\mathcal{O}}
\def\A{\mathcal{A}}
\def\J{\mathcal{J}}
\def\S{\mathrm{S}}
\def\C{\mathrm{C}}
\def\t{\widetilde}
\def\IR{\mathrm{IR}}
\def\m{\mathrm{m}}


%%%%%%%%%%%
%% Added command by Julien for links
%%%%%%%%%%%
\RequirePackage[colorlinks=true
,urlcolor=blue
,anchorcolor=blue
,citecolor=blue
,filecolor=blue
,linkcolor=blue
,menucolor=blue
,pagecolor=blue
,linktocpage=true
,pdfproducer=medialab
,pdfa=true
]{hyperref}


%%%%%%%%%%%
%% Claude's commands
%%%%%%%%%%%

% 
 \newcommand{\eps}{\epsilon}


%%%%%%%% Master names
\newcommand{\refcite}[1]{ref.~\cite{#1}}
\newcommand{\refscite}[1]{refs.~\cite{#1}}
\newcommand{\Eq}[1]{Eq.~\eqref{eq:#1}}
\newcommand{\eq}[1]{eq.~\eqref{eq:#1}}
\newcommand{\eqs}[2]{eqs.~\eqref{eq:#1} and \eqref{eq:#2}}
\renewcommand{\sec}[1]{section~\ref{sec:#1}}
\newcommand{\secs}[2]{sections~\ref{sec:#1} and \ref{sec:#2}}
\newcommand{\subsec}[1]{section~\ref{subsec:#1}}
\newcommand{\fig}[1]{figure~\ref{fig:#1}}
\newcommand{\figs}[2]{figures~\ref{fig:#1} and \ref{fig:#2}}

\newcommand{\df}{\mathrm{d}}
\newcommand{\img}{\mathrm{i}}
\newcommand{\zb}{\bar z}
\newcommand{\xb}{\bar x}
\newcommand{\bn}{{\bar n}}
\newcommand{\bq}{{\bar q}}
\newcommand{\GeV}{\,\mathrm{GeV}}
\newcommand{\cC}{\mathcal{C}}
\newcommand{\cA}{\mathcal{A}}
\newcommand{\cI}{\mathcal{I}}
\newcommand{\cJ}{\mathcal{J}}
\newcommand{\cL}{\mathcal{L}}
\newcommand{\cM}{\mathcal{M}}
\newcommand{\cE}{\mathcal{E}}
\newcommand{\cN}{\mathcal{N}}
\newcommand{\cO}{\mathcal{O}}
\newcommand{\Tau}{\mathcal{T}}
\newcommand{\cP}{\mathcal{P}}
\newcommand{\cS}{\mathcal{S}}
\newcommand{\qt}{{\vec q}_T}
\newcommand{\bt}{{\vec b}_T}
\newcommand{\kt}{{\vec k}_T}
\newcommand{\wa}{{w_1}}
\newcommand{\wb}{{w_2}}
\newcommand{\cut}{{\rm cut}}
\newcommand{\nlim}{\lim\limits_{\mathrm{strict}~n-\mathrm{coll.}}}
\newcommand{\as}{\alpha_s}
\newcommand{\GammaC}{\Gamma_{\rm cusp}}
\newcommand{\MSbar}{\overline{\mathrm{MS}}}
\newcommand{\lqcd}{\Lambda_\mathrm{QCD}}
\newcommand{\sigmatot}{\sigma_\mathrm{tot}}
\newcommand{\SYM}{$\mathcal{N}=4$~sYM}
\newcommand{\EEMC}{\mathrm{EEC}}

\def\beq{\begin{equation}}
\def\eeq{\end{equation}}
\def\bsp#1\esp{\begin{split}#1\end{split}}
\newcommand{\ep}{e^+e^-}
\newcommand{\cF}{{\mathcal{F}}}
\newcommand{\ord}{{\mathcal{O}}}
\newcommand{\rd}{\textrm{d}}

\newcommand{\claudecomment}[1]{\textcolor{red}{\bf [CD: #1]}}

%\newcommand{\BMC}[1]{\textcolor{violet}{\bf [BM comment: #1]}}
%\newcommand{\BM}[1]{\textcolor{violet}{\bf [BM: #1]}}


\setlength{\oddsidemargin}{0pt}
\setlength{\textwidth}{15.8cm}
\setlength{\textheight}{22cm}
\topmargin-0.4cm
\addtolength{\jot}{5pt}
\addtolength{\arraycolsep}{-3pt}
\renewcommand{\arraystretch}{1.25}
\renewcommand{\textfraction}{0}
\renewcommand{\topfraction}{0.9}

\newcommand\barparen[1]{\overset{\textbf{\fontsize{2pt}{2pt}\selectfont(--)}}{#1}}

\DeclareMathOperator{\E}{\textrm{E}}

\newtheorem*{thm}{Theorem}
\newtheorem*{corollary}{Corollary}

%keeps figures in the sections
\usepackage[section]{placeins}

%%%%%%%%%%%%%For comments%%%%%%%%%%%%

\usepackage{color}
\newcounter{RSQ}
\newcommand{\commentRS}[1]{{\bf\textcolor{red}{\stepcounter{RSQ}
[$\bullet\,$ \textcolor{black}{RS Q\theRSQ}\hspace*{0.15cm} #1]}}}

\newcommand{\julien}[1]{\textcolor{blue}{\bf [JB: #1]}}

\newlength{\parwidth}
\newcommand{\cpcsub}[1]
{%
\setlength{\parwidth}{\textwidth}\addtolength{\parwidth}{-2.1em}%
\begin{center}
\begin{tabular}[t]{@{}p{\parwidth}@{}}
#1
\end{tabular}
\end{center}
}%

%%%%%%%%%%%%%%%%%%%%%%%%%%%%%%%%%%%%%%%%%%%%%%%%%%%%%%%%%%%%%%%%%%%%%%%%
% Shamelessly stolen from Thorsten's thohacks.sty
%%%%%%%%%%%%%%%%%%%%%%%%%%%%%%%%%%%%%%%%%%%%%%%%%%%%%%%%%%%%%%%%%%%%%%%%
\catcode`\@=11
\font\manfnt=manfnt
\def\Watchout{\@ifnextchar [{\W@tchout}{\W@tchout[1]}}
\def\W@tchout[#1]{{\manfnt\@tempcnta#1\relax%
  \@whilenum\@tempcnta>\z@\do{%
    \char"7F\hskip 0.3em\advance\@tempcnta\m@ne}}}
\let\foo\W@tchout
\def\dubious{\@ifnextchar[{\@dubious}{\@dubious[1]}}
\let\enddubious\endlist
\def\@dubious[#1]{%
  \setbox\@tempboxa\hbox{\@W@tchout#1}
  \@tempdima\wd\@tempboxa
  \list{}{\leftmargin\@tempdima}\item[\hbox to 0pt{\hss\@W@tchout#1}]}
\def\@W@tchout#1{\W@tchout[#1]}
\catcode`\@=12
%%%%%%%%%%%%%%%%%%%%%%%%%%%%%%%%%%%%%%%%%%%%%%%%%%%%%%%%%%%%%%%%%%%%%%%%



%Begin special definitions for Instructions file






%\received{\today} 		%%
%\revised{}
%\accepted{\today}		%% These are for published papers.

%\preprint{}

			  	% Use \hepth etc. also in bibliography.  


\usepackage{longtable}

\begin{document}


\begin{flushright}
{\small
BONN-TH-2023-02

}
\end{flushright}


\vskip1cm
\begin{center}
{\Large \bf \boldmath Amplitude-like functions from entire functions}
\end{center}



  \vspace{0.5cm}
\begin{center}
\sc Claude~Duhr, \sc Chandrashekhar~Kshirsagar 
\\[6mm]
{\it Bethe Center for Theoretical Physics, Universit\"at Bonn, D-53115, Germany}
\\[0.3cm]
{\it Emails: cduhr@uni-bonn.de, chandra@uni-bonn.de}
\end{center}



\begin{abstract}
Recently a function was constructed that satisfies all known properties of a tree-level scattering of four massless scalars via the exchange of an infinite tower of particles with masses given by the non-trivial zeroes of the Riemann zeta function. A key ingredient in the construction is an even entire function whose only zeroes coincide with the non-trivial zeroes of the Riemann zeta function. In this paper we show that exactly the same conclusions can be drawn for an infinite class of even entire functions with only zeroes on the real line. This shows that the previous result does not seem to be connected to specific properties of the Riemann zeta function, but it applies more generally. As an application, we show that exactly the same conclusions can be drawn for $L$-functions other than the Riemann zeta function.
\end{abstract}


\newpage

\section{Introduction}
\label{sec:introduction}
% \begin{itemize}
%     % Diffusion of FL
%     \item {\st{Diffusion of FL}}
%     % Security threats to FL
%     \item {\st{Security threats to FL with particular focus on model poisoning}}
%     % Limitations of existing countermeasures
%     \item {\st{Current countermeasures (e.g., KRUM) and their limitations}}
%     % Proposed method and its advantages
%     \item {\st{Intuitive description of the proposed method and its difference (i.e., advantages) w.r.t. state of the art}}
%     % Main contributions
%     \item {\st{Summary of the main contributions of this work}}
%     % Paper's structure and organization
%     \item {\st{Paper's structure and organization}}
% \end{itemize}

% Diffusion of FL
Recently, {\em federated learning} (FL) has emerged as the leading paradigm for training distributed, large-scale, and privacy-preserving machine learning (ML) systems~\cite{mcmahan2017googleai,mcmahan2017aistats}. 
The core idea of FL is to allow multiple edge clients to collaboratively train a shared, global model without disclosing their local private training data.
%Specifically, an FL system consists of a central server and many edge clients; 
A typical FL round involves the following steps: {\em(i)} the server randomly picks some clients and sends them the current, global model; {\em(ii)} each selected client locally trains its model with its own private data; then, it sends the resulting local model to the server;\footnote{Whenever we refer to global/local model, we mean global/local model {\em parameters}.} {\em(iii)} the server updates the global model by computing an \emph{aggregation function}, usually the average (FedAvg), on the local models received from clients.
% \begin{enumerate}
%     \item[{\em(i)}] the server sends the current, global model to the clients and appoints some of them for training;
%     \item[{\em(ii)}] each selected client locally trains its copy of the global model with its own private data; then, it sends the resulting local model back to the server;\footnote{Whenever we refer to global/local model, we mean global/local model {\em parameters}.}
%     \item[{\em(iii)}] the server updates the global model by computing an \emph{aggregation function} on the local models received from clients (by default, the average, also referred to as FedAvg~\cite{mcmahan2017aistats}).
% \end{enumerate}
This process goes on until the global model converges. %(e.g., after a certain number of rounds or other similar stopping criteria).
%\\
% The advantages of FL over the traditional, centralized learning paradigm are undoubtedly clear in terms of flexibility/scalability (clients can join/disconnect from the FL network dynamically), network communications (only model weights\footnote{We will use \textit{parameters} and \textit{weights} interchangeably.} are exchanged between clients and server), and privacy (each client's private training data is kept local at the client's end and not uploaded to the server).
\\
% Security threats to FL
%However, the growing adoption of FL also raises security concerns~\cite{costa2022covert}, particularly about its confidentiality, integrity, and availability.
Although its advantages over standard ML, FL also raises security concerns~\cite{costa2022covert}. %, particularly about its confidentiality, integrity, and availability~\cite{costa2022covert}.
% OLD, LONG VERSION
% Indeed, some work deals with privacy leakage that may expose the local data of some clients~\cite{melis2019sp}. 
% A large body of work, instead, investigates attacks that usually aim to detriment the predictive accuracy of the learned global model. For instance, \emph{data poisoning} attacks achieve this goal by letting an adversary pollute the training set of some corrupt FL clients with maliciously crafted examples~\cite{jagielski2018sp}.
% Similarly, in \emph{model poisoning} the attacker attempts to tweak the global model weights~\cite{bhagoji2019pmlr} by directly perturbing the local model's weights of some infected FL clients before these are sent to the central server for aggregation, usually via so-called Byzantine attacks. 
% It turns out that Byzantine model poisoning attacks severely impact standard FedAvg; therefore, more robust aggregation functions must be designed to make FL systems secure.
Here, we focus on \emph{untargeted model poisoning} attacks~\cite{bhagoji2019pmlr}, where an adversary attempts to tweak the global model weights %\footnote{We will use the terms \textit{parameters} and \textit{weights} interchangeably.} 
by directly perturbing the local model's parameters of some infected clients before these are sent to the central server for aggregation.
In doing so, the adversary aims to jeopardize the global model \textit{indiscriminately} at inference time.
Such model poisoning attacks severely impact standard FedAvg; therefore, more robust aggregation functions must be designed to secure FL systems.
\\
% In this paper, we focus on designing a novel robust aggregation scheme at the server's end to contrast the effect of Byzantine model poisoning attacks.
%
% Current countermeasures and their limitations
%Several countermeasures have been proposed in the literature to combat model poisoning attacks on FL systems.
% Some methods use simple statistics more robust than plain average to smooth the impact of malicious updates (e.g., Trimmed Mean and FedMedian~\cite{yin2018icml}). 
% Other defenses implement outlier detection techniques to discard malicious updates from the aggregation performed at the server's end. Those are either based on heuristics (e.g., Krum/Multi-Krum~\cite{blanchard2017nips} and Bulyan~\cite{mhamdi2018pmlr}) or data-driven approaches (e.g., K-means clustering~\cite{shen2016acm} or DnC via spectral analysis~\cite{shejwalkar2021ndss}). 
% Finally, some strategies rely on a centralized ``source of trust'' to spot potential malicious updates (e.g., FLTrust~\cite{cao2020fltrust}).
% Several countermeasures have been proposed in the literature to combat model poisoning attacks on FL systems, i.e., to discard possible malicious local updates from the aggregation performed at the server's end. 
% These techniques range from simple statistics more robust than plain average (e.g., Trimmed Mean and FedMedian~\cite{yin2018icml}) to outlier detection heuristics (e.g., Krum/Multi-Krum~\cite{blanchard2017nips} and Bulyan~\cite{mhamdi2018pmlr}) or data-driven approaches (e.g., spectral analysis via K-means clustering~\cite{shen2016acm} or spectral analysis), or methods based on ``source of trust'' (e.g., FLTrust~\cite{cao2020fltrust}).
% OLD, LONG VERSION
%Several countermeasures have been proposed in the literature to combat Byzantine model poisoning attacks on FL systems.
% Descriptive statistics
% For example, Trimmed Mean and FedMedian aggregate local model updates using more robust statistics than standard average~\cite{yin2018icml}.
%
% % Heuristics for outlier detection
% Many existing Byzantine-resilient strategies implement some outlier detection heuristics to discard the model updates sent by potentially malicious clients from the input of the aggregation function.
% One of the most popular heuristics is Krum~\cite{blanchard2017nips}.
% This strategy tries to mitigate the impact of Byzantine attacks by selecting as a global model the local model with the smallest sum of Euclidean distances to {\em all} the other local models.
% Although powerful, Krum requires the server to know (or, at least, estimate) the number of malicious FL clients upfront, which is generally impossible in a realistic attack scenario. %
% Moreover, Krum may become ineffective for complex, high-dimensional model parameter spaces due to the curse of dimensionality.
% Bulyan~\cite{mhamdi2018pmlr} tries to overcome this issue by combining Krum with a variant of Trimmed Mean.
% % Data-driven outlier detection
% Other strategies use data-driven outlier detection techniques -- e.g., via K-means clustering~\cite{shen2016acm} -- to spot potential malicious local model updates. 
% %For instance, Shen et al. propose to cluster local model updates with K-means and thus identify outliers.
%
% % Other techniques
% As far as the server is concerned, any local model received can be from a potential malicious client. 
% FLTrust~\cite{cao2020fltrust} assumes the server acts as a client, i.e., trains a local model on an additional {\em trustworthy} dataset at the server's end and compares it against all the local models from other clients. 
% This way, the server can rely on some ``source of trust'' when discarding potentially malicious clients.
%\\
% Limitations of existing Byzantine-resilient strategies
Unfortunately, existing defense mechanisms either rely on simple heuristics (e.g., Trimmed Mean and FedMedian by~\cite{yin2018icml}) or need strong and unrealistic assumptions to work effectively (e.g., foreknowledge or estimation of the number of malicious clients in the FL system, as for Krum/Multi-Krum~\cite{blanchard2017nips} and Bulyan~\cite{mhamdi2018pmlr}, which, however, cannot exceed a fixed threshold).
Furthermore, outlier detection methods using K-means clustering~\cite{shen2016acm} or spectral analysis like DnC~\cite{shejwalkar2021ndss} do not directly consider the temporal evolution of local model updates received.
Finally, strategies like FLTrust~\cite{cao2020fltrust} require the server to collect its own dataset and act as a proper client, thereby altering the standard FL protocol.
\\
% OLD, LONG VERSION
% Overall, existing Byzantine-resilient strategies are either simple heuristics (e.g., FedMedian) or, if they are more complex, they rely on strong and unrealistic assumptions to work effectively (e.g., knowing the number of malicious clients in the FL system in advance, as for Krum and alike).
% Furthermore, data-driven outlier detection methods do not consider the temporary evolution of local model updates received (e.g., K-means clustering). 
% Finally, strategies like FLTrust requires the server to collect its own dataset and act as a proper client, thereby altering the standard FL protocol.
%
% Description of the proposed method
This work introduces a novel pre-aggregation \textit{filter} robust to untargeted model poisoning attacks. Notably, this filter $(i)$ operates without requiring prior knowledge or constraints on the number of malicious clients and $(ii)$ inherently integrates temporal dependencies. 
The FL server can employ this filter as a preprocessing step before applying \textit{any} aggregation function, be it standard like FedAvg or robust like Krum or Bulyan.
Specifically, we formulate the problem of identifying corrupted updates as a multidimensional (i.e., matrix-valued) time series anomaly detection task. 
The key idea is that legitimate local updates, resulting from well-calibrated iterative procedures like stochastic gradient descent (SGD) with an appropriate learning rate, show \textit{higher predictability} compared to malicious updates. This hypothesis stems from the fact that the sequence of gradients (thus, model parameters) observed during legitimate training exhibit regular patterns, as validated in Section~\ref{subsec:intuition}. %until convergence. 
%This regularity may be more pronounced for smooth convex loss functions, but it can still be captured within an appropriate time window, even for more complex and convoluted loss surfaces. 
%We provide evidence of this claim in Appendix~B, where we show that the average mutual information (i.e., ``predictability''), calculated over pairs of legitimate model updates sent at different FL rounds, is significantly higher than the corresponding computation for a malicious client.
\\
Inspired by the matrix autoregressive (MAR) framework for multidimensional time series forecasting~\cite{chen2021je}, we propose the FLANDERS ({\em \textbf{F}ederated \textbf{L}earning meets \textbf{AN}omaly \textbf{DE}tection for a \textbf{R}obust and \textbf{S}ecure}) filter.
The main advantages of FLANDERS over existing strategies like FLDetector~\cite{zhao2020multivariate} are its resilience to large-scale attacks, where $50\%$ or more FL participants are hostile, and the capability of working under realistic non-iid scenarios.
We attribute such a capability to two key factors: $(i)$ FLANDERS works without knowing a priori the ratio of corrupted clients, and $(ii)$ it embodies temporal dependencies between intra- and inter-client updates, quickly recognizing local model drifts caused by evil players. Below, we summarize our main contributions:

\begin{itemize}
\item[{\em(i)}]
We provide empirical evidence that the sequence of models sent by legitimate clients is more predictable than those of malicious participants performing untargeted model poisoning attacks.
\\
\item[{\em(ii)}] 
We introduce FLANDERS, the first pre-aggregation filter for FL robust to untargeted model poisoning based on multidimensional time series anomaly detection.
\\
\item[{\em(iii)}] 
We integrate FLANDERS into Flower,\footnote{\scriptsize{\url{https://flower.dev/}}} a popular FL simulation framework for reproducibility.
\\
\item[{\em(iv)}] 
We show that FLANDERS improves the robustness of the existing aggregation methods under multiple settings: different datasets, client's data distribution (non-iid), models, and attack scenarios.
\\
\item[{\em(v)}] 
We publicly release all the implementation code of FLANDERS along with our experiments.\footnote{\scriptsize{\url{https://anonymous.4open.science/r/flanders_exp-7EEB}}}
\end{itemize}

% Paper's structure and organization
The remainder of the paper is structured as follows. %some related work and the current state-of-the-art solutions to security issues that FL entails. 
Section~\ref{sec:background} covers background and preliminaries. 
In Section~\ref{sec:related}, we discuss related work.
Section~\ref{sec:problem} and Section~\ref{sec:method} describe the problem formulation and the method proposed. % to tackle it. 
Section~\ref{sec:experiments} gathers experimental results. %, and Section~\ref{sec:limitations} discusses some limitations of this work.
Finally, we conclude in Section~\ref{sec:conclusion}.
 %discusses the limitations of this work and draws future research directions.
%reports conclusions and draws perspectives for future research directions.

%%%%%%% OLD %%%%%%%
%to overcome the resilience of Byzantine failures in distributed Stochastic Gradient Descent computations. 
% The strength of Krum is its time complexity, which is linear in the gradient dimension. 
% However, the robustness of the approach is guaranteed for gradient-based learning applications only when the majority of the clients are not compromised. 
% Besides, the aggregation mechanism of Krum, as well as that of similar methods, is robust from a coarse-grained perspective and does not provide solutions to errors and perturbations that may occur at inference time.
%A related approach to~\cite{blanchard2017nips} is the work of Su et al.~\cite{su2016dc}. Here, the authors propose an iterated approximate agreement to tackle a multi-layer scenario attacked by Byzantine agents. 
%However, the method works efficiently on the sole discrete context and it is inapplicable to continuous state environments.
%\gabri{Maybe, we should just talk about the main limitations of existing countermeasures without digging into their details (or, we can just mention Krum as this is the most popular one). I will move the description of all these methods to the Related Work section.}
% !TEX root = main.tex

\section{Amplitude-like functions}
\label{sec:acfs}
The goal of this section is to review analytic properties of tree-level scattering amplitudes. We work in the setting of ref.~\cite{Remmen:2021zmc}, and we consider a two-to-two scattering of massless scalars described by an amplitude that can be cast in the form:
\beq\label{eq:fac_form}
\cM(s,u) = \cA(s)+\cA(u) = \cA(s)+\cA(-s-t)\,.   
\eeq
To be concrete, we may consider a theory with two distinct massless scalars $\phi_1$ and $\phi_2$, and we consider the scattering $\phi_1\,\phi_2\to\phi_1\,\phi_2$.\footnote{Note that not every theory of this type has necessarily amplitudes that can be cast in the form in eq.~\eqref{eq:fac_form}.}
In the following we discuss some general properties that any such amplitude must have:
\begin{enumerate} 
\item {\underline{Bose symmetry.}} For a scattering of the type $\phi_1\,\phi_2\to\phi_1\,\phi_2$, Bose symmetry implies $\cM(s,u) = \cM(u,s)$. This condition is automatically fulfilled if we work with the factorised form in eq.~\eqref{eq:fac_form}.
\item {\underline{Meromorphicity and simple poles on the positive real line.}}
 It is well known that tree-level amplitudes for scalar scattering are rational functions of the Lorentz invariant products of the four-momenta. 
 The only poles at finite values of the invariants arise from propagators going on shell, and locality dictates that all poles must be simple. In our scenario, this implies that $\cA(s)$ is a meromorphic function of $s$ with simple poles at most at $s=m_n^2\ge 0$. Note that the spectrum $m_n^2$ of exchanged particles cannot have any accumulation point, because a meromorphic function can only have isolated singularities.
 %
 %
\item {\underline{Negative residues at all poles.}} Close to the simple pole at $s=m_n^2$, the amplitude behaves like
%\footnote{We follow the conventions of ref.~\cite{Remmen:2021zmc} and work in the `mostly plus' convention for the metric.}  
$i\cM(s,u)\sim \frac{-ig^2}{s-m_n^2}$, where $g$ denotes the coupling constant of the interaction between the external scalars $\phi_1$, $\phi_2$ and the state $X_n$ of mass $m_n$ exchanged in the $s$-channel. If we want the coupling $\phi_1\phi_2X_n$ to be real, we must have $g^2>0$. Hence, we conclude that we must have
\beq
\textrm{Res}_{s=m_n^2} \cA(s) = \oint\frac{\rd s}{2\pi i}\cA(s) = -g^2 < 0\,,\textrm{~~~for all poles $s=m_n^2$.}
\eeq
%
%
\item {\underline{Polynomial boundedness.}} It is well known that in any QFT scattering amplitudes are polynomially bounded. More precisely, we must have for every fixed value of $t = -s-u$:
\beq
\lim_{s\to\infty}\cM(s,-s-t)s^{-N} = 0\,,\textrm{~~~for some positive integer $N$.}
\eeq
Here this is equivalent to 
\beq
\lim_{s\to\infty}\cA(s)s^{-N} = 0\,,\textrm{~~~for some positive integer $N$.}
\eeq
%
%
\item {\underline{Positivity constraints.}} In ref.~\cite{Adams:2006sv} it was shown that for $t=0$, we must have
%\footnote{We may need to subtract poles due, e.g., the an exchange of massless particles. Since this does not change our argument, we will not dwell on this technical point further.}
\beq
\textrm{Res}_{s=0}\frac{\overline{\cM}(s,-s)}{s^{L+1}}=\oint\frac{\rd s}{2\pi i} s^{-N-1}\overline{\cM}(s,-s) > 0\,,
\eeq
where $\overline{\cM}(s,-s)$ is obtained from ${\cM}(s,-s)$ by subtracting poles at $s=0$.
By Bose symmetry, $\overline{\cM}(s,-s) = \overline{\cM}(-s,s) = \overline{\cA}(s)+\overline{\cA}(-s)$ is an even function (and $\overline{\cA}(s)$ is obtained by subtracting the poles at $s=0$), and so the previous equation only constrains the even Taylor coefficients of $\cA(s)$ around $s=0$. We can of course use the same argument to derive positivity constraints for $u=0$, which allows us to put constraints directly on $\cA(s)$:
\beq
\textrm{Res}_{s=0}\frac{\overline{\cM}(s,0)}{s^{L+1}}=\oint\frac{\rd s}{2\pi i} s^{-N-1}\overline{\cA}(s) > 0\,.
\eeq
It is easy to see this implies that all Taylor coefficients of  $\cA(s)$ around $s=0$ must be positive.
\end{enumerate}

We will refer to a function $\cM(s,u)$ that satisfies these five conditions as an \emph{amplitude-like function}.
In ref.~\cite{Remmen:2021zmc} it is shown that the function $\cM(s,u)$ constructed from the function $\cA(s)$ in eq.~\eqref{eq:cA_Remmen} satisfies these five constraints. As a consequence, $\cM(s,u)$
is amplitude-like.\footnote{In ref.~\cite{Remmen:2021zmc} only the positivity of the even Taylor coefficients was imposed. It is easy to see that  our requirement is stronger and strictly contains the positivity constraint on the even Taylor coefficients. As we will see later on, the function $\cA(s)$ from ref.~\cite{Remmen:2021zmc} (see eq.~\eqref{eq:cA_Remmen}) also satisfies our stronger requirement.}

%%%%%%%%%%%%%%%%%%%%%%%%%%%%%%%%%%%%%%%%%%%%%%
%%%%%%%%%%%%%%%%%%%%%%%%%%%%%%%%%%%%%%%%%%%%%%





% !TEX root = main.tex

\section{Amplitude-like functions from entire functions}
\label{sec:thm}



In this section we present the main result of our paper, namely we present a general construction of amplitude-like functions from a very large class of entire functions. We start by reviewing some mathematical background on entire functions and we present our result and discuss its consequences in section~\ref{sec:thm_result}. The proof is presented in section~\ref{sec:proof}.

\subsection{Entire functions}
\label{sec:entire_funcs}

In this section we review some standard material in complex analysis, in particular entire functions.
Recall that a function $f:\mathbb{C}\to \mathbb{C}$ is said to be \emph{entire} if it is holomorphic everywhere on the complex plane. Stereotypical examples of entire functions are polynomials and the exponential function. 

As a consequence of Liouville's theorem, every non-constant entire function must be unbounded. It will be useful to consider how an entire function behaves at infinity.
 We say that an entire function $f$ has \emph{order at most $\rho$} if there is $R>0$ and $C\ge0$ such that $|f(z)|<C\exp(|z|^\rho)$ for all $|z|>R$. The smallest such $\rho$ is called the \emph{order of $f$}. If $f$ has order at most $\rho$, then this means that $\log |f(z)|$ grows at most like $|z|^\rho$ for large $|z|$. This implies in particular that $\log|f(z)|$ is polynomially bounded:
 \beq
 \lim_{z\to \infty} z^{-d-1}\,\log|f(z)| = 0\,, 
 \eeq
 where we defined $d := \lfloor \rho\rfloor$, i.e., $d$ is the largest integer less or equal than $\rho$.
 In the following we will only consider entire functions of finite order.


An entire function may have zeroes, and since $f$ is holomorphic, its set of zeroes
%\beq
%Z_f = \{z\in\mathbb{C}:f(z)=0\}\,,
%\eeq
cannot have any accumulation point. Every entire function of finite order $\rho$ can be cast in a standard form using \emph{Hadamard's factorisation theorem}:
\beq\label{eq:hadamard}
f(z) = e^{g(z)}\,z^m\,\prod_{n}\E_d\left(\frac{z}{z_n}\right)\,.
\eeq
where $d=\lfloor\rho\rfloor$, $g$ is a polynomial of degree at most $d$, $m$ is the order of $f$ at 0 and the product runs over all zeroes $z_n\neq0$ of $f$ counted with multiplicity. The function $\E_d(z)$ is the \emph{elementary factor}, defined by
\beq
\E_d(z) := \left\{\begin{array}{ll}
1-z\,, & \textrm{ if } d=0\,,\\
(1-z)\exp\left[\sum_{k=1}^d\frac{z^k}{k}\right]\,,&\textrm{ if } d>0\,.
\end{array}\right.
\eeq
Note that in the case where $f$ has an infinite number of zeroes, the product in eq.~\eqref{eq:hadamard} runs over an infinite number of terms. This case requires some careful consideration regarding the convergence of this infinite product. One can show (see, e.g.,~ref.~\cite{hadamard}) that the infinite product in eq.~\eqref{eq:hadamard} converges if and only if we have 
\beq\label{eq:convergence}
\sum_n\frac{1}{|z_n|^{d+1}}<\infty\,.
\eeq







In the following it will be useful to introduce the following notations.
Hadamard's factorisation theorem in eq.~\eqref{eq:hadamard} implies that we can write every function of finite order in the form
\beq
f(z) = z^m\,\cE_f(z)\,\cP_f(z)\,,
\eeq
where $\cE_f(z):= e^{g(z)}$ has no zeroes and $\cP_f(z) := \prod_{n}\E_d\left(\frac{z}{z_n}\right)$ has the form of an infinite product. 
Note that if $f_1$ and $f_2$ are two entire functions of order at most $\rho$, then so is their product, and we have
\beq
\cE_{f_1f_2}(z) = \cE_{f_1}(z) \cE_{f_2}(z) \textrm{~~~and~~~}\cP_{f_1f_2}(z) = \cP_{f_1}(z) \cP_{f_2}(z) \,.
\eeq
Finally, let us discuss some important consequence of eq.~\eqref{eq:convergence}. Consider a (possibly infinite) sequence $(z_n)_n$ of non-zero complex numbers such that eq.~\eqref{eq:convergence} holds. Then there is an entire function $f$ of order $d<\infty$ with precisely those zeroes. Indeed, Hadamard's theorem allows to easily construct such a function: it is simply the infinite product $\cP_f(z)$. In fact there are infinitely many such functions, and they differ precisely by an exponential factor $\cE_f(z)=e^{g(z)}$, where $g$ is a polynomial of degree at most $d$.



\subsection{The main result}
\label{sec:thm_result}
We now discuss our main result, which generalises the result of ref.~\cite{Remmen:2021zmc} from the entire function $\Xi$ in eq.~\eqref{eq:xi_def} to an infinite class of entire functions. We first need to restrict the class of entire functions $f$ that we will consider. First, the zeroes of $f$ will be related to the poles of the putative amplitude, so $f$ should only have zeroes on the real axis. Second, the propagator poles are related to  squared masses of the exchanged states, so we expect the zeroes to come in pairs $\pm z_n\neq0$. This gives, for functions of order at most $\rho$ (with $d=\lfloor \rho\rfloor$):
\beq\label{eq:E_d_sym}
\cP_{f}(z) = \prod_n\E_d\left(\frac{z}{z_n}\right)\E_d\left(-\frac{z}{z_n}\right) = \prod_n{\E}_{\lfloor d/2\rfloor}\left(\frac{z^2}{z_n^2}\right)\,,
\eeq
where in the last equality the product runs over the distinct zeroes of $f$ located on the positive real axis. Note that in this case $\cP_{f}(z)$ is an even function, $\cP_{f}(-z)=\cP_{f}(z)$, so that $\cP_{f}(\sqrt{z})$ defines an entire function of order at most $\rho/2$, with zeros of order $k_n$ at $z=z_n^2>0$. 

We will from now on focus on even entire functions $f(z)$ with zeroes on the real line. For such functions the zeroes always come in pairs $\pm z_n$, with $z_n>0$. In addition, $f$ may have a pole of order $m$ at $z=0$. Moreover, if $f$ is an even and entire function of order at most $\rho$, then $f(\sqrt{z})$ is en entire function of order at most $\rho/2$ (but it is not necessarily even). We then define (cf.~eq.~\eqref{eq:cA_Remmen}):
%\beq
%\cP_{f}(\sqrt{z}) = \prod_n{\E}_{\lfloor d/2\rfloor}\left(\frac{z}{z_n^2}\right)\,,
%\eeq
%where the product runs over all zeros of $f$ at $z=z_n>0$. We also define:
\beq
\cA_f(z) := -\frac{\rd}{\rd z}\log f(\sqrt{z}) = -\frac{\rd}{\rd z}\log\left[z^{m}\,\cE_f(\sqrt{z})\,\cP_{f}(\sqrt{z})\right]\,,
\eeq
where we use the notation 
\beq\label{eq:E_f_def}
\cE_f({z}) := \exp\left[\sum_{k=0}^{d} g_k\,z^k\right]\,,
\eeq
with $g_k$ some complex numbers. Note that, since $\cA_f(z) = \cA_{cf(z)}$ for every non-zero complex number $c$, we can assume without loss of generality $g_0=0$. 

The following notation will be useful:
\beq\label{eq:c_def}
c_{f,k} = \left\{\begin{array}{ll}
\sum_{n}\frac{1}{z_n^{2(k+1)}}\,, &\textrm{~~if~~}k \ge \lfloor d/2\rfloor\,,\\
0\,, &\textrm{~~if~~}k  <  \lfloor d/2\rfloor\,,
\end{array}\right.
\eeq
where $k$ is a positive integer.
It is easy to see that the series $\sum_{n}\frac{1}{z_n^{2(k+1)}}$ converges for $k\ge \lfloor d/2\rfloor$. Indeed, the only case that needs checking is when $f$ has an infinite number of zeroes $z_n$. Since the set of zeroes of $f$ has no accumulation point, the sequence $(|z_n|^2)_{n}$ is unbounded. Hence, there is some positive integer $N$ such that $|z_n|^2>1$ for $n>N$. This implies $|z_n|^{2(k+1)}\ge |z_n|^{2( \lfloor d/2\rfloor+1)}$, for all $n>N$ and $k\ge  \lfloor d/2\rfloor$. Convergence of $\sum_{n}\frac{1}{|z_n|^{2(k+1)}}$ then follows by comparing with the convergence criterion in eq.~\eqref{eq:convergence} applied to the infinite product in eq.~\eqref{eq:E_d_sym}.

Our main result is summarised in the following theorem:
\begin{thm} Let $f:\mathbb{C}\to\mathbb{C}$ be an even and entire function such that
\begin{enumerate}
\item $f$ has finite order $\rho$,
\item $f$ only has zeroes on the real line,
\item the $g_k$ are real and negative for all $1\le k\le \lfloor\rho\rfloor$.
\end{enumerate}
Then the function $\cM_f(s,u) := \cA_f(s)+\cA_f(u)$ is amplitude-like, i.e., it satisfies the 5 properties given in section~\ref{sec:acfs}.
\end{thm}
The proof of this theorem will be given in section~\ref{sec:proof} below. 
 Let us make some comments about the last condition, which is the most constraining one. First, we note that for an even function $f$, we have $g_{2k+1}=0$, so that the last condition really only applies to the even coefficients $g_{2k}$. Second, it is equivalent to $0\le \cE_f(z) < 1$ for all $z\in \mathbb{R}$. Finally, the last condition is always satisfied for even entire functions of order $\rho<2$. Indeed, if  $\rho<2$, we have $d\le 1$, and so eqs.~\eqref{eq:E_d_sym} and~\eqref{eq:E_f_def} imply:
\beq
\cP_f(z) = \prod_n\left(1-\frac{z^2}{z_n^2}\right)\textrm{~~~and~~~} \cE_f(z) = e^{g_0}=C\,,
\eeq
for some constant non-zero complex number $C$. Equivalently, we can write
\beq
f(z) = C\,z^{2m}\,\prod_n\left(1-\frac{z^2}{z_n^2}\right)\,.
\eeq
From here it is easy to see that the third condition of the theorem is always satisfied in this case, and we have:
\begin{corollary}
Let $f:\mathbb{C}\to\mathbb{C}$ be an even and entire function of finite order $\rho<2$ with zeroes only on the real line. Then the function $\cM_f(s,u)$ is amplitude-like.
\end{corollary}

In the remainder of this section we will discuss some general implications of our theorem.
%
First, we see can see that our theorem contains the results of ref.~\cite{Remmen:2021zmc} for the Riemann zeta function as a special case. Indeed, while the Riemann zeta function $\zeta(z)$ has a pole at $z=1$ (and is thus not entire), the function $\Xi(z)$ is an even and entire function of order 1, and we have:
\beq
\Xi(z) = \prod_{n=1}^{\infty}\left(1-\frac{z^2}{\mu_n^2}\right)\,,
\eeq
where the $\mu_n>0$ are the imaginary parts of the non-trivial zeroes of the Riemann zeta function. Assuming the Riemann hypothesis, the second condition of the theorem is satisfied.
Hence, the theorem applies, and $\cM_{\Xi}(s,u)$ is amplitude-like, in agreement with the findings of ref.~\cite{Remmen:2021zmc}. 


%Our theorem also allows to obtain other examples of amplitude-like functions, and in section~\ref{sec:examples}, we construct infinite classes of examples. This shows that the case of the Riemann zeta function considered in 
%ref.~\cite{Remmen:2021zmc} is not special in any way, and in particular the conclusions of ref.~\cite{Remmen:2021zmc} do not rely on specific properties of the Riemann zeta function (other than the Riemann hypothesis, to ensure that all zeroes lie on the real line), but rather they follow from very general considerations about entire functions. 

We may ask if there are other entire functions that satisfy the assumptions of the theorem. In the following we argue that there are infinitely many such functions. Indeed, consider a sequence of positive real numbers $(z_n)_n$ without accumulation point such that $\sum_{n}\frac{1}{z_n^{d+1}}<\infty$. Then we know from Hadamard's factorisation theorem that there is an entire function $f$ of order at most $\rho$ with $d=\lfloor\rho\rfloor$ and with zeroes precisely at $z=\pm z_n$. We may pick:
\beq
f(z) = \cP_f(z) = \prod_n\E_d\left(\frac{z}{z_n}\right)\E_d\left(-\frac{z}{z_n}\right) = \prod_n{\E}_{\lfloor d/2\rfloor}\left(\frac{z^2}{z_n^2}\right)\,.
\eeq
It is easy to check that this function satisfies all the hypotheses of our theorem, and so the function $\cM_f(s,u)$ is amplitude-like. We thus see there is nothing special about the sequence $(\mu_n)_n$ of non-trivial zeroes of the Riemann $\zeta$ function, but the same conclusion holds for pretty much every sequence of real positive numbers that satisfy eq.~\eqref{eq:convergence}.
%
Finally, we mention that it is easy to see that if $f_1$ and $f_2$ satisfy the hypotheses of our theorem, then so does their product, and we have 
\beq\label{eq:amalgam}
\cA_{f_1f_2} = \cA_{f_1}+\cA_{f_2} \textrm{~~~and~~~} \cM_{f_1f_2} = \cM_{f_1}+\cM_{f_2}\,.
\eeq
Hence, also $\cM_{f_1f_2}$ 
 is amplitude-like.


\subsection{The proof of the theorem}
\label{sec:proof}
In this section we present the proof of our main theorem. We need to show that any $f$ that satisfies the hypotheses of the theorem also satisfies the five properties of section~\ref{sec:acfs}. Bose-symmetry (Property 1) is manifest, and there is nothing to check. 

Let $F:\mathbb{C}\to\mathbb{C}$ be a function analytic on some domain $U$. It is easy to see that
\beq
\frac{\rd}{\rd z}\log F(z) = \frac{F'(z)}{F(z)}
\eeq
has singularities at the zeros of $F$. If $F$ has a zero of order $M$ at $z=z_0\in U$, then $F'$ has a zero of order $M-1$ there, to that $F'/F$ has a simple zero at $z=z_0$. An easy application of the residue theorem shows that
\beq
\oint \frac{\rd z}{2\pi i}\,\frac{F'(z)}{F(z)} =  M > 0\,.
\eeq
We take $F=\cA_f$, and we have:
\beq
\cA_f(z) = -\frac{m}{z} - \frac{\cE_f'(z)}{\cE_f(z)} -\frac{\widetilde\cP_f'({z})}{\widetilde\cP_f({z})}\,,\qquad \widetilde\cP_f({z}):=\cP_f(\sqrt{z})\,.
\eeq
The first term clearly has a simple pole at $z=0$ with negative residue $-m$. The second term has no poles, because $\cE_f(z) = e^{g(z)}$ vanishes nowhere. The last terms has simple poles at $z=z_n^2$, with $z_n>0$. Hence,
\beq
\cA_f(z)  \sim \frac{-k_n}{z-z_n^2}\,,\qquad \textrm{for } z\sim z_n^2\,.
\eeq
We conclude that $\cA_f$ satisfies Properties 2 \& 3 of section~\ref{sec:acfs}.

Let us now check that $\cA_f$ is polynomially bounded (Property 4). Since $f$ has finite order $\rho$ say, we have $\cA_f(z) \sim (\rho-1)|z|^{\rho-1}$, and so $\cA_f$ is clearly bounded by $|z|^{\lfloor \rho\rfloor}$. In other words, the fact that $f$ has finite order immediately translates into $\cA_f$ being polynomially bounded.

It remains to show that $\cA_f$ satisfies Property 5. In other words, we need to show that all the Taylor coefficients of  $\overline{\cA}_f$ around $z=0$ are positive. The Taylor expansion of $\overline{\cA}_f$ is easy to obtain. Indeed, we have
\beq
\overline{\cA}_f(z) = \cA_f(z) + \frac{m}{z} = -\sum_{k=0}^{\lfloor d/2\rfloor-1}(k+1)g_{2k+2}\,z^{k}+\sum_{k=\lfloor d/2\rfloor}^\infty c_{f,k}\,z^{k}\,,
\eeq
We know that the infinite series defining $c_{f,k}$ in eq.~\eqref{eq:c_def} are all convergent and have positive summands, and so $c_{f,k}\ge 0$. Hence, the only non-trivial positivity constraints are those involving $g_{2k+2}$. Those are precisely satisfied if the hypotheses of the theorem hold. This finishes the proof. 


\section{Numerical Example}\label{sec_examples}
Consider a second order SISO flat system of the form in \eqref{BINF}, with $v_k = -\sin(x_{1,k}) + x_{1,k}x_{2,k}^2 - x_{1,k}^3x_{2,k} + u_k.$ In this example, we compare the performance of three nonlinear controllers: (i) An exact linearizing and stabilizing controller designed using basis functions that include $v$ in their span \cite[Cor. 2]{DePersis22}, and two locally stabilizing controllers (ii and iii) designed using the following choice of basis functions\footnote{The method described in \cite[Cor. 2]{DePersis22} requires that the unknown map \eqref{eqn_expressionforv} is linear in $u$, which is why we use the basis functions \eqref{eqn_ex_basisfunctions}. Although the choice of the basis functions in \eqref{eqn_ex_basisfunctions} is different from that in \eqref{eqn_specificchoice}, one can easily see from the proof of Theorem \ref{thm_aprioriFL} that using inputs of the form \eqref{eqn_PEinputSISOflat} also guarantees collective PE of \eqref{eqn_ex_basisfunctions}.} which do not contain $v$ in their span \cite[Cor. 2 and Sec.~III.B]{DePersis22}
\begin{equation}
	\Theta(\xi_k,u_k) = \begin{bmatrix}
		u_k & \xi_k^\top & (\xi_k^2)^\top & (\xi_k^3)^\top
	\end{bmatrix}^{\hspace{-0.5mm}\top}.\label{eqn_ex_basisfunctions}
\end{equation}
For all three controllers, PE of the basis functions of order one is a necessary and sufficient condition for the feasibility of the convex program that is solved to obtain the control gains (cf. \cite[Cor. 2, Thm. 2, and Thm. 5]{DePersis22}). For controllers (i) and (ii), PE is enforced by sampling the input randomly. For controller (iii), PE is enforced \textit{a priori} using the results of Theorem~\ref{thm_aprioriFL}. In this case, we used a straightforward extension of \cite[Cor. 2]{DePersis22} such that collected data from multiple experiments (i.e., collective PE) can be used to design the controller.

Since the system is unstable, the input data (of length $N=21$) for controllers (i) and (ii) had to be sampled from the uniform distribution $U(-0.25,0.25)$, whereas using multiple experiments as in Theorem~\ref{thm_aprioriFL} allowed us to use inputs (each of length $N_j=3$) with larger magnitudes (sampled from $U(-1,1)$). In \cite{vanWaarde20}, a similar observation was made for linear systems. As a result, a larger quantitative level of PE was attained (cf. Remark \ref{remark_qPE} and Table~\ref{table_comparison}).

The performance of the closed-loop system (over $T=20$ time instants) was compared starting from the same initial conditions (randomly sampled from $U(-1,1)\times U(-1,1)$). Table~\ref{table_comparison} shows the average cumulative stabilization errors (defined as $\sum_{k=0}^{T-1}\frac{1}{T}|x_{i,k}|$, for $i=1,2,\,T=20$) for all three controllers over 100 experiments, excluding 5 (respectively 4) unstable experiments for controllers (ii) and (iii). Controller~(i) is the best performing one since it enforces exact nonlinearity cancellation. Controller (iii) is shown to outperform controller (ii), although the same basis functions \eqref{eqn_ex_basisfunctions} were used, potentially suggesting that the region of attraction of controller (iii) is larger compared to (ii). This can be attributed to the fact that larger levels of PE were attained using multiple experiments.
% !TEX root = main.tex

\section{Amplitude-like functions from $L$-functions}
\label{sec:L-functions}

In section~\ref{sec:thm} we proved our main result, which allows us to construct amplitude-like functions from large classes of even and entire functions. This answers one of the questions asked at the end of ref.~\cite{Remmen:2021zmc}. Reference~\cite{Remmen:2021zmc} also asked the question if it was possible to extend its results from the Riemann zeta function to other (Dirichlet) $L$-functions. In this section we show that the answer to this question is positive and follows directly from our theorem. 


There are various different classes of functions called $L$-functions in the mathematical literature. Many of these functions are defined axiomatically and belong to the so-called \emph{Selberg (S) class}. While we expect that our results related to amplitude-like functions remain true for all functions in the S-class, for simplicity of the exposition we restrict the discussion here to a subset of $L$-functions, the so-called Dirichlet $L$-functions, and we defer the discussion of the general case to appendix~\ref{app:S-class}.


A Dirichlet $L$-function can be defined through a series similar to the definition of the Riemann zeta function in eq.~\eqref{eq:zeta_def}:
\beq\label{eq:L-func}
L(z,\chi) = \sum_{n=1}^{\infty}\frac{\chi(n)}{n^z}\,,\qquad \Re(z)>1\,.
\eeq
Here $\chi(n)$ is a \emph{Dirichlet character}, i.e., a map $\chi: \mathbb{Z}\to \mathbb{C}$ satisfying the following conditions:
\begin{enumerate}
\item $\chi$ is multiplicative: $\chi(m\cdot n) = \chi(m)\chi(n)$\,,
\item $\chi$ is periodic with period $q$: $\chi(n+q) = \chi(n)$\,,
\item $\chi(n)$ is non zero only if $n$ and $q$ are co-prime, i.e., if gcd$(n,q)=1$.
\end{enumerate}
If the smallest period of $\chi$ is $q$, we say that $\chi$ is a character mod $q$. It is easy to check that every Dirichlet character mod $q$ must evaluate to a $q^{\textrm{th}}$ root of unity. Clearly, for the trivial character $\chi(n)=1$, $\forall n\in\mathbb{Z}$, the definition in eq.~\eqref{eq:L-func} reduces to the definition of the Riemann zeta function in eq.~\eqref{eq:zeta_def}. 

If $\chi_1$ is a character mod $q_1$, and if $q_1|q_2$, then we can define a character $\chi_2$ mod $q_2$ by $\chi_2(n) = \chi_1(n)$. The character $\chi_2$ is said to be \emph{induced by $\chi_1$}. A character that is not induced by any other character is called \emph{primitive}. In the following we restrict the discussion to primitive characters (because non-primitive characters are not expected to be in the $S$-class).

It is a fundamental property of $L$-functions that they satisfy functional equations similar to eq.~\eqref{eq:zeta_cont} for the Riemann zeta function.
If $\chi$ is a primitive Dirichlet character mod $q$, then the corresponding $L$-function satisfies the functional equation
\begin{equation}
\label{eq:L-func_eq}
L(z,\chi)=\frac{G(\chi)}{i^{\delta}}2^{z}\pi^{z-1}q^{-z}\sin\left(\frac{\pi}{2}(z+\delta) \right)\Gamma(1-z)L(1-z,\overline{\chi})\,,     
 \end{equation}
 where $\overline{\chi}$ is the complex conjugate of $\chi$, $\delta= \frac{1-\chi(-1)}{2}$, and
 $G(\chi)=\sum_{a=1}^{a=q}\chi(a)e^{2\pi i \frac{a}{q}}$ is the Gauss sum, which in case of primitive characters satisfies $|G(\chi)|=\sqrt{q}$. Just like in the case of the Riemann zeta function, the functional equation can be used to analytically continue the function to values $\Re(z)<1$. The $L$-function has zeroes in the complex plane. The \emph{generalised Riemann hypothesis} expresses the conjecture that the non-trivial zeroes of $L$ (i.e., those that are not captured by the prefactor in the functional equation~\eqref{eq:L-func_eq}) all lie on the critical line $\Re(z)=\frac{1}{2}$. 
 
Just like in the case of the Riemann zeta function, it is possible to construct a function that has zeroes only at the non-trivial zeroes of $L(z,\chi)$:
\begin{equation}
     \xi(z,\chi)=\left(\frac{q}{\pi}\right)^{\frac{z+\delta}{2}}\Gamma\left(\frac{z+\delta}{2}\right)L(z,\chi) \,.
 \end{equation}
 This function is analogous to the function $\xi(z)$ in eq.~\eqref{eq:xi_def}, and one can show that $ \xi(z,\chi)$ defines an entire function of order one (cf., e.g., ref.~\cite{S-class1}).
There is, however, an important difference between $L(z,\chi)$ for a non-trivial character $\chi$ and the Riemann zeta function. For the Riemann zeta function, the zeroes of $\xi(z)$ come in complex conjugate pairs $\frac{1}{2}\pm i\mu_n$, which is a necessary condition for the function $\Xi(z) := \xi\left(\frac{1}{2}+ iz\right)$ to be even. This in turn is one of the hypotheses for our theorem from section~\ref{sec:thm_result} to be applicable. This property does no longer hold for a non-trivial character $\chi$. Instead, if $\rho$ is a zero of $\xi(z,\chi)$, then so is $1-\bar{\rho}$ (and the generalised Riemann hypothesis implies $\rho = 1-\bar{\rho}$). As a consequence, the function $\xi\left(\frac{1}{2}+ iz,\chi\right)$ is in general not an even function of $z$, and so our theorem does not apply. Instead, we can consider the function
\beq
\Xi_{\chi}(z) := \xi\left(\frac{1}{2}+ iz,\chi\right)\,\xi\left(\frac{1}{2}+ iz,\overline{\chi}\right)\,.
\eeq
Since $\xi(z,\chi)$ and $\xi(z,\overline{\chi})$ are entire functions of order 1, the same holds true for $\Xi_{\chi}(z)$.
Using the functional equation~\eqref{eq:L-func_eq}, we can show that $\Xi_{\chi}(z)$ is an even function:
\beq
\Xi_{\chi}(-z) = \xi\left(\frac{1}{2}-iz,\chi\right)\,\xi\left(\frac{1}{2}- iz,\overline{\chi}\right)
= \xi\left(\frac{1}{2}+ iz,\overline{\chi}\right)\,\xi\left(\frac{1}{2}+ iz,{\chi}\right)= \Xi_{\chi}(z)\,,
\eeq
where the second step follows from the functional equation~\eqref{eq:L-func_eq}.
Hence, $\Xi_{\chi}(z)$ is an even entire function of order 1 with the Hadamard product representation
 \begin{equation}
    \Xi_{\chi}(z)=\Xi_{\chi}(0)\prod_{n} \left(1-\frac{z^2}{\mu_{\chi,n}^{2}}\right)^{k_{n}}\,,
\end{equation}
where $\frac{1}{2}+i\mu_{\chi,n}$ are the non-trivial zeroes of $L(z,\chi)$, and $k_n$ denote their multiplicity.
We can apply our theorem from section~\ref{sec:thm_result}, and we see that the following function is amplitude-like:
 \beq\bsp
\mathcal{A}_{\Xi_{\chi}}(s) &=  -\frac{\mathrm{d}}{\mathrm{d}s}\log{\Xi_{\chi}(\sqrt{s})} \\ 
 &= -\frac{i}{2\sqrt{s}}\Biggl[ \frac{L'(\frac{1}{2}+i\sqrt{s},\chi)}{L'(\frac{1}{2}+i\sqrt{s},\chi)}+ \frac{L'(\frac{1}{2}+i\sqrt{s},\overline{\chi})}{L'(\frac{1}{2}+i\sqrt{s},\overline{\chi})}  +\psi\left(\frac{1}{4}+\frac{i\sqrt{s}}{2}+\frac{\delta}{2}\right)+\log{\frac{q}{\pi}}         \Biggr]\\
&= \sum_{n}\frac{k_{n}}{-s+\mu_{\chi,n}^{2}}\,.
\esp\eeq
This shows that it is possible to construct an amplitude-like function not just for the Riemann zeta function, but for general $L$-functions, thereby answering the question asked in ref.~\cite{Remmen:2021zmc}. We emphasise that the functional equation~\eqref{eq:L-func_eq} plays an important role in proving that the function $\Xi_{\chi}$ is even. The discussion here strictly only applies tor Dirichlet $L$-functions with primitive character, which are special instances of the more general $L$-functions from the $S$-class. In appendix~\ref{app:S-class} we show that the arguments presented here can easily be extended to all $L$-functions from the $S$-class.


\section{Conclusions}
We consider the phase-extraction problem, and we showed that, given a unitary $U = e^{i\pi H}$ and its inverse $U^{\dag}$, we could implement a block-encoding of $\phi(H)$ for some smooth function $\phi(x)$. The word `smooth' here means existence and continuity of the derivatives: the higher the number of continuous derivatives that a function has, the faster its Fourier sum (and thus the Laurent polynomial on the eigenphases) uniformly converges to that function. We are confident this can have many more applications beyond what is shown in this work. It is also worth remarking that Jackson showed that the convergence rate of a Fourier series is almost-optimal, in the sense that no trigonometric (or, equivalently, complex exponential) series can approximate the desired function faster, up to that $\log d$ factor~\cite[p.\ 21]{jacksonTheoryApproximation1930a}. Also remember that `smoothing' a function, i.e., replacing its derivative with a continuous function, does not give faster convergence for free in general, as its derivative will become steep in the points where we smooth out discontinuities, and this translates to a high Lipschitz constant: a~clear example is given by Eq.~\ref{eq:lipschitz-constant-recurrence-solution}, but in that case, fortunately, nothing depends on the size of the input $N$, and thus does not influence the asymptotic query complexity of Algorithm~\ref{alg:prop-sampling-qsp}, although the constant factor can become large even for $p = 20$. From a theoretical point of view, this work shows that, for any $\eta > 0$, there is an algorithm with query complexity 
$$\Tilde{\bigO}\left(\frac{1}{\bar{c}^{\frac{1}{2} + \eta}} \frac{1}{\epsilon^\eta} \right)$$
solving the proportional-sampling problem. This statement seems to suggest there exists an algorithm which directly solves the problem with $\eta = 0$, and an open question would be to find such algorithm.


It is also interesting to remark that Theorems~\ref{thm:haah-construction},~\ref{thm:haah-completion} indeed allow the construction for any $\phi$, even complex-valued, provided that its absolute value is reciprocal.

One could think that, in Section~\ref{sec:prop-sampling}, instead of using the linear function in the phase-extraction subroutine, we could approximate the square root and then apply the transformation directly on $e^{i \pi c(x)}$. However, in the case of proportional sampling this would be inconvenient, as the derivative of the square root function has a discontinuity with an infinite jump around 0, and we could not choose a constant $\delta$ if we had values of the oracle that are too close to $0$.

\appendix
\section{Amplitude-like functions from the $S$-class}
\label{app:S-class}

In this appendix we show that the arguments of section~\ref{sec:L-functions} can be extended from Dirichlet $L$-functions for primitive characters to all $L$-functions from the $S$-class. In other words, we show that to every $L$-function from the $S$-class that satisfies the Riemann hypothesis we can associate an amplitude-like function. Before proving this statement, we start by briefly defining the $S$-class. 

\subsection{The Selberg class}

The {Selberg ($S$) class} are functions $F(z)$ that satisfy the following axions, called \emph{Selberg axioms} (see, e.g., refs.~\cite{Kaczorowski2006,S-class1}):
\begin{enumerate}
    \item \textbf{Dirichlet Series:} $F(z)$ can be written be written as an absolutely convergent Dirichlet series for $\Re(z)>1$:
 \begin{equation}
F(z)=\sum_{n=1}^{\infty}\frac{a(n)}{n^{z}}   \,. 
\end{equation}
   \item \textbf{Analytic Continuation:} There an integer $m\ge 0$ such that $(z-1)^{m}F(z) $ is an entire function of finite order.
   %
   \item \textbf{Functional Equation:} $F$ satisfies the following functional equation:
   \beq \label{eq:Phi_func_eq}
\Phi(z)=\omega\, \overline{\Phi}(1-z) \,,
\eeq
where we defined $\overline{\Phi}(z) := \overline{\Phi(\bar{z})}$, and
   \begin{flalign}
\Phi(z)&=Q^{z}\prod_{j=1}^{r}\Gamma(\lambda_{j}z+\mu_{j})F(z)\,,
   \end{flalign}
with $\omega\in\mathbb{C}$, $|\omega|=1$, $\Re{\mu_{j}}\geq0$, $Q>0$, $\lambda_{j}>0$ and $r\ge0$.
   
   \item \textbf{Ramanujan hypothesis:} For every $\epsilon>0$, $a(n)\ll n^{\epsilon}$.
   \item \textbf{Euler Product:} For $\Re(z)>1$, 
  \begin{equation}
      \log{F(z)}=\sum_{n=1}^{\infty}\frac{b(n)}{n^{z}}\,,
  \end{equation}
  where $b(n)$ is non-zero only for $n=p^{l}$ where $p$ denotes a prime factor and $l\geq 1$, and $b(n)\ll n^{\theta}$ for $\theta<1/2$.
\end{enumerate}

The $\Gamma$ functions in eq.~\eqref{eq:Phi_func_eq} have poles, and so $F$ must have zeroes at the corresponding locations. These are the trivial zeroes of $F$.
%
Note that axiom 2 implies that $F$ has at most a pole of order $m$ at $z=1$, and it is holomorphic everywhere else. In the following we assume without loss of generality that $m$ is equal to the order of this pole. 
The statement of axiom 2 can actually be made even sharper. One can show that $(z-1)^mF(z)$ is an entire function of order 1~\cite{S-class1}. 

\subsubsection{Amplitude-like from the $S$-class}
Let now $F$ be a function from the $S$ class.
Let us define the following function:
\beq
\Theta(z) := z^m\,(1-z)^m\,\Phi(z)\,.
\eeq
Since $\Phi(z)\propto F(z)$, we can see that $\Theta$ has no pole at $z=1$, and therefore it is an entire function of order at most 1. 
It is also possible to show that $\Theta(z)$ has zeroes only at the non-trivial zeroes of $F$.
Moreover, it is easy to check that $\Theta$ satisfies the functional equation
\beq\label{eq:Theta_func_eq}
\Theta(z) = \omega\,\overline{\Theta}(1-z)\,.
\eeq
However, $\Theta$ will in general not be an even function. Instead, the functional equation~\eqref{eq:Theta_func_eq} implies
\beq
\Theta(z)\overline\Theta(z) = \overline\Theta(1-z)\Theta(1-z)\,.
\eeq
and therefore the function
\beq
\Xi_{F}(z) := \Theta\left(\frac{1}{2}+iz\right) \,\overline{\Theta}\left(\frac{1}{2}+iz\right)
\eeq
is even, $\Xi_{F}(-z) = \Xi_{F}(z)$. It then follows form the Corollary in section~\ref{sec:thm_result} that $\cM_{\Xi_F}(s,u)$ is amplitude-like, provided that the zeroes of $\Theta\left(\frac{1}{2}+iz\right)$ all lie on the real line, i.e., if $F$ satisfies the Riemann hypothesis.  Hence, we see that for all functions $F$ in the $S$-class satisfying these conditions, leads to an amplitude-like function $\cM_{\Xi_F}$.



\section*{Acknowledgments}
The authors are greatful to Florian Loebbert for discussions. This work was co-funded by the European Union (ERC, LoCoMotive, 101043686). Views and opinions expressed are however those of the author(s) only and do not necessarily reflect those of the European Union or the European Research Council. Neither the European Union nor the granting authority can be held responsible for them.



%\bibliography{/Users/robert/Dropbox/Diary/Bibliography/base}
\bibliography{bib}
%\bibliographystyle{unsrt}
\bibliographystyle{JHEP}

\end{document}

