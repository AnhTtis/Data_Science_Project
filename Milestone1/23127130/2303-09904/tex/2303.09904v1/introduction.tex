% !TEX root = main.tex

\section{Introduction and motivation}
\label{sec:intro}

Scattering amplitudes are a cornerstone of modern research in Quantum Field Theory (QFT). First of all, they are one of the main ingredients to many computations for collider and gravitational wave phenomenology, and as such they play an important role when comparing theory and experiment. Second, already at tree-level they may provide important insight into the mathematical structure of QFTs. Sometimes they even serve as a tool to discover new theories with interesting properties and particle spectra, cf.,~e.g.,~ref.~\cite{Cheung:2014dqa}. Prominent examples of this are the Veneziano~\cite{Veneziano:1968yb} and Virasoro~\cite{Virasoro:1969me} amplitudes, which describe the the two-to-two scattering of open and closed strings at tree-level. These amplitudes feature an infinite number of poles, which represent the exchange of an infinite tower of (higher-spin) states with increasing masses.

Amplitudes are functions of the four-momenta of the scattering particles. In particular, for a two-to-two scattering of particles with mass $m_i$, the amplitude is a function of the usual Mandelstam invariants
\beq
s = (p_1+p_2)^2\,,\qquad t = (p_1+p_3)^2\,,\qquad u = (p_2+p_3)^2\,,
\eeq
constrained by 
\beq
s+t+u = \sum_{i=1}^4m_i^2\,.
\eeq
If the external states are scalars, the amplitude is a complex-valued function $\cM(s,u)$. The analytic structure of $\cM(s,u)$ is very much constrained from physics, and not every complex function in two variables may arise as a scattering amplitude in QFT or string theory. At tree-level, the function $\cM(s,u)$ must be meromorphic, with at most simple poles corresponding to the masses of the exchanged intermediate states.  Sometimes these constraints are powerful enough to uniquely determine amplitudes in specific theories. For example, the Veneziano amplitude is (loosely speaking) singled out as the unique amplitude with a prescribed high-energy behaviour and describing an infinite tower of higher-spin exchanges with unbounded mass spectrum~\cite{Caron-Huot:2016icg}. If some constraints are relaxed or changed, one may discover amplitudes in other theories. For example, if one relaxes the condition that the spectrum of exchanged high-spin states is unbounded, there is another solution, called the Coon amplitude~\cite{Coon:1969yw,Baker:1970vxk,Coon:1972qz}, which has received a lot of attention lately, cf.,~e.g.,~refs.~\cite{Figueroa:2022onw,Maldacena:2022ckr,Geiser:2022icl,Chakravarty:2022vrp,Cheung:2022mkw,Geiser:2022exp,Bhardwaj:2022lbz,Cheung:2023adk}.

It was recently proposed~\cite{Remmen:2021zmc} that it is possible to construct a function $\cM(s,u)$ consistent with known constraints on a two-to-two scattering of massless scalars, where the spectrum of exchanged particles is given by the non-trivial zeros of the Riemann zeta function. The latter is defined by
\beq\label{eq:zeta_def}
\zeta(z) = \sum_{n=1}^\infty\frac{1}{n^z}\,.
\eeq
This series converges absolutely for $\Re(z) > 1$. One can extend the definition by analytic continuation to $\mathbb{C}\setminus\{1\}$ via 
\beq\label{eq:zeta_cont}
\zeta(z) = 2^z\,\pi^{z-1}\,\sin(\pi z/2)\,\Gamma(1-z)\,\zeta(1-z)\,.
\eeq
One obtains in this way a meromorphic function with a simple pole at $z=1$, and holomorphic everywhere else. The Riemann zeta function possesses an infinite number of zeroes. First, it is easy to check that $\zeta(-2n)=0$ for every non-negative integer $n$. These are the so-called \emph{trivial zeroes}, and they are the only zeroes on the real line. In addition, there are zeroes in the complex plane (and they must come in complex-conjugate pairs). 
The celebrated \emph{Riemann hypothesis} expresses the remarkable conjecture that all non-trivial zeroes have the form $z_n = \frac{1}{2}\pm i\mu_n$, with $\mu_n$ real and positive. Note that, since $\zeta(z)$ is meromorphic, the set of zeroes must be discrete (and in particular the set of zeroes cannot have any accumulation point), and it is expected that there are infinitely many non-trivial zeroes. It was shown in ref.~\cite{Remmen:2021zmc} that, if we define
\beq\label{eq:cA_Remmen}
\cA(s) := -\frac{\rd}{\rd s}\log\Xi(\sqrt{s})\,,
\eeq
where $\Xi(z)$ is related to the Riemann zeta function via
\beq\label{eq:xi_def}
\Xi(z) := \xi\left({\frac{1}{2}}+iz\right)\,,\qquad \xi(z):=\frac{1}{2}z(z-1)\pi^{-z/2}\Gamma\left(\frac{z}{2}\right)\zeta(z)\,,
\eeq
then the function $\cM(s,t) := \cA(s)+\cA(u)$
has the properties of an amplitude describing the tree-level scattering of 4 massless scalars (we will review the complete set of constraints in section~\ref{sec:acfs}). This raises the intriguing question if there is a QFT for which $\cM(s,u)$ computes a scattering amplitude. 
Note that $\Xi(z)$ is even, $\Xi(-z)=\Xi(z)$, and entire, i.e., $\Xi(z)$ is holomorphic everywhere in the complex plane. It vanishes for $z =\pm\mu_n$, i.e., the zeroes of $\Xi$ are related to the non-trivial zeroes of the Riemann zeta function. This implies that $\cA(s)$ has simples poles at $s = \mu_n^2$. Hence, if there is a QFT for which $\cM(s,u)$ computes a scattering amplitude, then the spectrum of exchanged particles should be equal to the set of non-trivial zeroes of the Riemann zeta function (assuming the Riemann hypothesis holds, since otherwise the masses are complex)! This is in fact not the first time a connection was established between the Riemann hypothesis and the spectrum of a quantum mechanical system, cf.,~e.g.,~refs.~\cite{Montgomery,polya,berry,Srednicki:2011zz,Bender:2016wob,Sierra:2016rgn}.

The question if a QFT exists with mass spectrum $\mu_n$ and a four-point amplitude given by eq.~\eqref{eq:cA_Remmen} is a tough question, which is likely to remain open for a very long time. One may, however, try to answer a simpler question, namely whether one can find other complex functions $f$ from which one can build a function $\cM(s,u)$ that has all the properties of an amplitude in a QFT whose spectrum of exchanged particles is given by the set of zeroes of $f$. Finding such functions, or ruling out that they exist, may elucidate in how far the Riemann zeta function is special and how its properties are reflected in the putative QFT. In particular, ref.~\cite{Remmen:2021zmc} asked the question if it is possible to extend its construction to general Dirichlet $L$-functions (of which the Riemann zeta function is a special case), or even to arbitrary entire functions. However, no answer to these questions was given.

The goal of this paper is to give an answer to these questions. More precisely, we will show that, quite generically, we can associate to very large classes of even and entire functions $f$ with zeroes only on the real line a function $\cM_f(s,u)$ consistent with known constraints on the scattering of four (massless) scalars via the exchange of a spectrum of massive particles given by the zeroes of $f$. Our result allows us in particular to construct such a function $\cM_f$ for every Dirichlet $L$-function in the so-called Selberg class (assuming that the \emph{generalised Riemann hypothesis} holds for $L$-functions), which answers the questions asked in ref.~\cite{Remmen:2021zmc}. Our result shows in particular that the specific properties of the Riemann zeta function do not play an important role in the construction of $\cM_f(s,u)$, but they are a direct consequence of general properties of entire functions.

This paper is organised as follows: In section~\ref{sec:acfs} we review known constraints on two-to-two scattering amplitudes at tree-level, and we introduce the concept of \emph{amplitude-like} functions. Section~\ref{sec:thm} contains the main result of our paper: after some general review of entire functions and Hadamard's factorisation theorem in section~\ref{sec:entire_funcs}, we present our main theorem and its consequences in section~\ref{sec:thm_result}, and we prove our theorem in section~\ref{sec:proof}. In section~\ref{sec:examples} we illustrate our theorem on several examples, and in section~\ref{sec:L-functions} we use it to extend the results of  ref.~\cite{Remmen:2021zmc} from the Riemann zeta function to other types of $L$-functions. Finally, in section~\ref{sec:conclusions} we draw our conclusions.







