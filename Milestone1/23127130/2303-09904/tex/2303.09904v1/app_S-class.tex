\section{Amplitude-like functions from the $S$-class}
\label{app:S-class}

In this appendix we show that the arguments of section~\ref{sec:L-functions} can be extended from Dirichlet $L$-functions for primitive characters to all $L$-functions from the $S$-class. In other words, we show that to every $L$-function from the $S$-class that satisfies the Riemann hypothesis we can associate an amplitude-like function. Before proving this statement, we start by briefly defining the $S$-class. 

\subsection{The Selberg class}

The {Selberg ($S$) class} are functions $F(z)$ that satisfy the following axions, called \emph{Selberg axioms} (see, e.g., refs.~\cite{Kaczorowski2006,S-class1}):
\begin{enumerate}
    \item \textbf{Dirichlet Series:} $F(z)$ can be written be written as an absolutely convergent Dirichlet series for $\Re(z)>1$:
 \begin{equation}
F(z)=\sum_{n=1}^{\infty}\frac{a(n)}{n^{z}}   \,. 
\end{equation}
   \item \textbf{Analytic Continuation:} There an integer $m\ge 0$ such that $(z-1)^{m}F(z) $ is an entire function of finite order.
   %
   \item \textbf{Functional Equation:} $F$ satisfies the following functional equation:
   \beq \label{eq:Phi_func_eq}
\Phi(z)=\omega\, \overline{\Phi}(1-z) \,,
\eeq
where we defined $\overline{\Phi}(z) := \overline{\Phi(\bar{z})}$, and
   \begin{flalign}
\Phi(z)&=Q^{z}\prod_{j=1}^{r}\Gamma(\lambda_{j}z+\mu_{j})F(z)\,,
   \end{flalign}
with $\omega\in\mathbb{C}$, $|\omega|=1$, $\Re{\mu_{j}}\geq0$, $Q>0$, $\lambda_{j}>0$ and $r\ge0$.
   
   \item \textbf{Ramanujan hypothesis:} For every $\epsilon>0$, $a(n)\ll n^{\epsilon}$.
   \item \textbf{Euler Product:} For $\Re(z)>1$, 
  \begin{equation}
      \log{F(z)}=\sum_{n=1}^{\infty}\frac{b(n)}{n^{z}}\,,
  \end{equation}
  where $b(n)$ is non-zero only for $n=p^{l}$ where $p$ denotes a prime factor and $l\geq 1$, and $b(n)\ll n^{\theta}$ for $\theta<1/2$.
\end{enumerate}

The $\Gamma$ functions in eq.~\eqref{eq:Phi_func_eq} have poles, and so $F$ must have zeroes at the corresponding locations. These are the trivial zeroes of $F$.
%
Note that axiom 2 implies that $F$ has at most a pole of order $m$ at $z=1$, and it is holomorphic everywhere else. In the following we assume without loss of generality that $m$ is equal to the order of this pole. 
The statement of axiom 2 can actually be made even sharper. One can show that $(z-1)^mF(z)$ is an entire function of order 1~\cite{S-class1}. 

\subsubsection{Amplitude-like from the $S$-class}
Let now $F$ be a function from the $S$ class.
Let us define the following function:
\beq
\Theta(z) := z^m\,(1-z)^m\,\Phi(z)\,.
\eeq
Since $\Phi(z)\propto F(z)$, we can see that $\Theta$ has no pole at $z=1$, and therefore it is an entire function of order at most 1. 
It is also possible to show that $\Theta(z)$ has zeroes only at the non-trivial zeroes of $F$.
Moreover, it is easy to check that $\Theta$ satisfies the functional equation
\beq\label{eq:Theta_func_eq}
\Theta(z) = \omega\,\overline{\Theta}(1-z)\,.
\eeq
However, $\Theta$ will in general not be an even function. Instead, the functional equation~\eqref{eq:Theta_func_eq} implies
\beq
\Theta(z)\overline\Theta(z) = \overline\Theta(1-z)\Theta(1-z)\,.
\eeq
and therefore the function
\beq
\Xi_{F}(z) := \Theta\left(\frac{1}{2}+iz\right) \,\overline{\Theta}\left(\frac{1}{2}+iz\right)
\eeq
is even, $\Xi_{F}(-z) = \Xi_{F}(z)$. It then follows form the Corollary in section~\ref{sec:thm_result} that $\cM_{\Xi_F}(s,u)$ is amplitude-like, provided that the zeroes of $\Theta\left(\frac{1}{2}+iz\right)$ all lie on the real line, i.e., if $F$ satisfies the Riemann hypothesis.  Hence, we see that for all functions $F$ in the $S$-class satisfying these conditions, leads to an amplitude-like function $\cM_{\Xi_F}$.