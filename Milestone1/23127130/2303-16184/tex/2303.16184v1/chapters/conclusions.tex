\section{Limitations}
There are several limitations inherent to the current VMesh representation that present opportunities for potential future improvements. These limitations include:
(1) The representation can produce inaccurate geometry due to global illumination effects caused by complex materials. For instance, in \cref{fig:real-scenes}, the tabletop displays unexpected holes as a result of the highly reflective, glass-like surface near its center. Our RefBasis texture formulation assumes approximate distant illumination and is therefore unable to accurately account for global illumination effects like self-reflections.
(2) VMesh is unable to precisely capture the appearance of transparent or furry objects. Ideally, these properties should be modeled by the volume part. However, we find that in practice, they are instead baked onto an inaccurate surface instead of being modeled by volume. This may result from using a surface rendering loss for optimization during the contiguous stage (see \cref{eqn:photometric-loss-contiguous}), and it requires a better surface-volume separation strategy to solve this problem.
(3) VMesh currently only facilitates real-time rendering of foreground objects, but there is potential to extend this hybrid formulation to efficiently model both foreground and background.

\section{Conclusions}
In summary, we present a new hybrid mesh-volume representation, VMesh, for efficient view synthesis of objects. VMesh combines the efficiency and flexibility of traditional mesh-based assets and the expressiveness of volumetric representations. We show competitive visual quality with state-of-the-art real-time view synthesis methods while being significantly faster and more efficient in storage. We believe VMesh may inspire future research into the potential of combining surface rendering and volume rendering techniques to produce high-quality view synthesis at a low cost.
