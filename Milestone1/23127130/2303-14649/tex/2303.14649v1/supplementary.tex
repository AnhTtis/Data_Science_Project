\documentclass[12pt,a4paper]{article}
\usepackage[a4paper, total={7.5in, 10in}]{geometry}
\usepackage{layout}
\usepackage[latin1]{inputenc}
\usepackage{amsmath}
\usepackage{amsfonts}
\usepackage{amssymb}
\usepackage{graphicx}
\usepackage{wrapfig}
\usepackage[dvipsnames]{xcolor}
 
% Required package
\usepackage{tikz}
\usetikzlibrary{positioning}
\usetikzlibrary{trees}
\tikzstyle{bag} = [align=center]
\def\beq{\begin{equation}}
\def\eeq{\end{equation}}
\def\ba{\begin{eqnarray}}
\def\ea{\end{eqnarray}}
\def\grad{\vec{\nabla}}
\usepackage[numbers]{natbib}
\usepackage{hyperref}
\hypersetup{
    colorlinks=true,
    linkcolor=blue,
    filecolor=magenta,      
    urlcolor=cyan,
}
 
\urlstyle{same}

\begin{document}
\begin{center}
\textbf{Supplemental Material: Variable range hopping in a non-equilibrium steady state}
\end{center}
\vskip 2pt

\section{Calculation of effective temperature}

In this paper, we have calculated the effective temperature using three different procedures. The details of the procedure are as follows:

\underline{\textit{Using Fermi-Dirac distribution}}: Once the system reaches a steady state, we calculate the site occupation probability ($f_{i}$) as a function of Hartree energy ($\varepsilon_{i}$), where

\begin{equation}
    \label{HE}
    \varepsilon_{i} = \phi_{i} + \sum_{j \neq i} \frac{n_{j}}{r_{ij}}
\end{equation}
We find that around the Fermi-level, $f_{i}$ is well approximated by the Fermi-Dirac distribution with an effective temperature, i.e., 
\begin{equation}
\label{FD_Teff}
f_{i} = \frac{1}{[exp(\varepsilon_{i}/T_{eff}) + 1]}.    
\end{equation}

\begin{figure*}[h]
    \centering
    \includegraphics[scale=0.55]{W2_b20_ness_all_r_xi2.pdf}
    \includegraphics[scale=0.55]{W2_b40_ness_all_r_xi2.pdf}
    \includegraphics[scale=0.55]{W2_b80_ness_all_r_xi2.pdf}
    \caption{Site occupation probability as a function
of Hartree energy. The continuous curves are fit by Fermi-distribution using Eq.(\ref{FD_Teff}).}
    \label{fig:my_label}
\end{figure*}

\underline{\textit{Using conductivity relation}}: As shown in the main text (red circles in Fig.4 in the main text) and Fig.2 here, our data shows that in the ohmic regime, the ES law is satisfied for the temperature range studied here, 
\begin{equation}
    \label{ES_law}
    \sigma  = \sigma_{0} exp[-(T_{0}/T)^{1/2}].
\end{equation}

In Eq.(\ref{ES_law}), we replace $\sigma$ by $\sigma_{NESS}$ and calculate the effective temperature $T_{eff}^{\sigma}$ of the NESS state as a function of bath temperature $T$ and relaxation time $x$.
\begin{equation}
    \label{sigma_ness}
    \sigma_{NESS}(T,x) = \sigma_{0} \, exp\bigg[ -\bigg(\frac{T_{0}}{T_{eff}^{\sigma}} \bigg)^{1/2}\bigg]
\end{equation}

\begin{figure}[h]
    \centering
    \includegraphics[scale=1.0]{sigma_T_eff_W2.pdf}
    \caption{The conductivity of the thermal state as a function of temperature at $W = 2$. We calculate the effective temperature ($T_{eff}^{\sigma}$) of the NESS state here using the functional form $f(x)$ (the solid line giving the best fit here).}
    \label{fig:my_sigma}
\end{figure}

\underline{\textit{Using $g(0)$ vs $T$ relation}}: As shown in the main text (red circles in Fig.5)  and Fig.3 here, our data shows that the density of states at the Fermi-level $g(0)$ of the thermal states follows the following relation
\begin{equation}
    \label{dos_thermal}
    g(0) = c T^{\alpha}
\end{equation}
where $\alpha=1.29$ and $c$ is the proportionality constant. We calculate the effective temperature of the NESS state ($T_{eff}^{g}$) using the relation
\begin{equation}
    \label{dos_ness}
    g_{0}^{NESS}(x,T) = c \, (T_{eff}^{g})^{\alpha}
\end{equation}
where $g_{0}^{NESS}(x, T)$ is the density of states at the Fermi-level of the NESS state as a function of relaxation time $(x)$ and bath temperature $T$. 

\begin{figure}[h]
    \centering
    \includegraphics[scale=1.0]{gT_eff_W2.pdf}
    \caption{The density of states at the Fermi-level of the thermal state as a function of temperature at $W = 2$. We calculate the effective temperature ($T_{eff}^{g}$) of the NESS state here using the functional form $f(x)$ (the solid line giving the best fit here).}
    \label{fig:my_dos}
\end{figure}

\clearpage
\newpage
\underline{\textit{Comparison of the effective temperatures}}: Here, we compare the effective temperatures obtained using Eq.(\ref{FD_Teff}, \ref{sigma_ness} and \ref{dos_ness}) for various physical qualities at a constant temperature $T$ as a function of relaxation times ($x$).
\begin{figure*}[h]
\centering
\includegraphics[scale=0.55]{diff_T_b20.pdf}
\includegraphics[scale=0.55]{ness_b80_Teff.pdf}
\caption{\label{NESS_b80_Teff} The Fermi-Dirac distribution around the Fermi-level (FD), DOS at the Fermi-level ($g(0)$), and conductivity ($\sigma$) at the bath temperature $T$ were used to compute the effective temperature ($T_{eff}^{*}$) of the NESS state as a function of relaxation time ($x$). $T_{eff}^{*}$ here corresponds to $T_{eff}^{\sigma}$ for the blue line, $T_{eff}^{g}$ for the orange line and $T_{eff}^{FD}$ for the green line. }
\end{figure*}
\clearpage
\newpage

\section{Simulation results}

Supplementary table (\ref{table1}) presents the number of upward and downward transitions performed in the Monte Carlo simulations for this work.

For an equilibrium state, the number of downward transitions ($M^{\beta}_{d}$) is approximately equal to the number of upward transitions ($M^{\beta}_{u}$) (we have checked this in our simulations). We now denote $N^{\beta}_{u}(x)$ and $N^{\beta}_{d}(x)$ as the additional number of upward and downward transitions performed in the NESS state, respectively. Note that the total number of upward/downward transitions in a NESS state (n) is

\begin{equation}
 \label{MC_prob}
   n=\begin{cases}
    M^{\beta}_{d} + N^{\beta}_{d}(x) , & \Delta E \leqslant 0.\\
    M^{\beta}_{u} + N^{\beta}_{u}(x) , & \Delta E > 0.
  \end{cases} 
 \end{equation}
 
Two important points can be extracted from Table. (\ref{table1}): 

First, comparing Columns (3-4), we find that the number of downward transitions dominates over the upward transitions in the NESS state ($N^{\beta=80}_{d}(x=1) \textgreater\textgreater N^{\beta=80}_{u}(x=1) $). 
 
Second, if one compares $N^{\beta=80}_{d}(x=1)$ and $N^{\beta=20}_{d}(x=1)$, which corresponds to number of additional downward transitions at $\beta=80$ and $\beta=20$ respectively for relaxation time $x = 1$, we find that the two are approximately the same. This is true for $N^{\beta=40}_{d}(x=1)$ case as well (data not shown). This implies that $N^{\beta}_{d}(x)$ is temperature independent. Since these additional transitions are responsible for the change in conductivity ($\Delta \sigma$) of the system, $\Delta \sigma$ is temperature independent (this is shown in the inset of Fig.3 of the main text). 

\begin{table}[h]
\caption{\label{table1} Simulation parameters at temperature $\beta = 1/T$ at distance $r$ and relaxation time $x$ are presented. The data represents the number of Monte Carlo steps ($\times 10^{6}$) in downward and upward directions.}
\centering
\begin{tabular}{|l|l|l|l|l|}
%\begin{tabular}{ | m{1.8cm}| m{1.8cm} | m{1.8cm} | m{1.8cm} | m{1.8cm} |m{1.8cm} |} 
 \hline
 distance & $M^{\beta=80}_{d}$ & $N^{\beta=80}_{u}(x=1)$ & $N^{\beta=80}_{d}(x=1)$ & $N^{\beta=20}_{d}(x=1)$   \\ \hline
\hline
$r = 1$  & 0.6460 & 0.8377 & 2.3203 & 2.0681  \\
\hline
$1 < r \leq 2$ & 0.1001 & 0.2248 & 1.4432 & 1.3959   \\
\hline
$2 < r \leq 3$ & 0.0194 & 0.0858 & 0.9415 & 0.9246  \\
\hline
$3 < r \leq 4$ & 0.0033 & 0.0290 & 0.3993 & 0.3932  \\
\hline
$4 < r \leq 5$ & 0.0008 & 0.0138 & 0.2076 & 0.2060  \\
\hline
$5 < r \leq 6$ & 0.0001 & 0.0047 & 0.0727 & 0.0719 \\
\hline
$6 < r \leq 7$ & 0.00004 & 0.0019 & 0.0307 & 0.0305  \\
\hline
$7 < r$        & 0.00002 & 0.0015 & 0.0246 & 0.0247  \\
\hline

\end{tabular}
\end{table}

\end{document}
