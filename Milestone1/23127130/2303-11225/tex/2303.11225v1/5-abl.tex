\section{Ablation Studies}


\myparagraph{Ablation Studies on Loss Functions and Datasets.}
%
We present ablation studies on the proposed loss functions and training strategy with hybrid datasets.
We train {\name} with synthetic data alone and compare it to the one trained with hybrid datasets.
%
For coarse shape reconstruction, we investigate the contribution of $\mathcal{L}_{\text{shp}}$ and its sub-term $\mathcal{L}_{\text{kl}}$.
%
For detail reconstruction, we compare {\name} without $\mathcal{L}_{\text{detail}}$ and $\mathcal{L}_{\text{kd}}$, respectively. The results are presented in Fig.~\ref{fig.abl_loss}.







Fig.~\ref{fig.abl_loss} demonstrates that the proposed loss functions and training strategy from hybrid datasets contribute to satisfactory coarse shape and details.
%
First, the model trained with synthetic data alone cannot generalize well to real-world images, which indicates the necessity to train with real-world data.
%
Second, $\mathcal{L}_{\text{shp}}$ improves the coarse shape reconstruction quality. $\mathcal{L}_{\text{shp}}$ is effective in tackling challenging poses and improving alignment. $\mathcal{L}_{\text{kl}}$ can relieve the overfitting risk on the synthetic data and improve the generalization to real-world images.
%
%
Third, without $\mathcal{L}_{\text{detail}}$ or $\mathcal{L}_{\text{kd}}$, the reconstructed details exhibit random noise and cannot faithfully reflect person-specific details. Such noise misses the correspondence to the person-specific identity.



\myparagraph{Ablation Studies on {\module}.}
%
To verify the effectiveness of building bases for static and dynamic details, we present detailed ablation studies on {\module}, by replacing the bases ({\ie}, $\mathbf{B}_{\text{sta}}$ and $\mathbf{B}_{\text{com}}/\mathbf{B}_{\text{str}}$) reconstruction with a U-Net decoder~\cite{ronneberger2015u} (same as DECA~\cite{feng2021learning}). Therefore, the model learns to directly synthesize displacement maps instead of predicting corresponding coefficients like ours.
% 
In Fig.~\ref{fig:abldynamic}, we make comparisons on: 1). directly synthesizing $\mathbf{D}_{\text{dyn}}$ (SD-1), 2). directly synthesizing $\mathbf{D}_{\text{com}}/\mathbf{D}_{\text{str}}$ and interpolating via Eq.~\ref{Eq.dyn_detail} (SD-2), 3). directly synthesizing $\mathbf{D}_{\text{sta}}$ (SD-3), 4). {\module} (Ours).
%
It can be seen that, due to the high diversity and complexity of expression representation, it is hard to directly learn realistic details even with ground-truth labels of $\mathbf{D}_{\text{dyn}}$ from synthetic data (see SD-1, SD-2 and SD-3 in Fig.~\ref{fig:abldynamic}).
More specifically, for the static details, directly synthesizing displacement maps bring much noise (SD-3). For example, the hollow eyebrow is demonstrated in the second row.
For the dynamic details, directly synthesizing displacement maps even leads to unnatural results (SD-1 and SD-2). For example, the reconstructed 3D faces are distorted especially in the second row. We also notice that, directly synthesizing $\mathbf{D}_{\text{dyn}}$ (SD-1) achieves inferior results than directly synthesizing $\mathbf{D}_{\text{com}}/\mathbf{D}_{\text{str}}$ and interpolating via Eq.~\ref{Eq.dyn_detail} (SD-2). This demonstrates that it is beneficial to simplify the expression representation by using interpolation between two displacement maps (\ie, compressed and stretched).
%
In conclusion, these observations further verify our insight on relaxing the challenging detail generation problem into a feasible coefficients regression problem.


\begin{figure}[!t]
    \centering
    % \vspace{-9pt}
    \begin{overpic}[trim=0cm 0cm 0cm 0cm,clip,width=1\linewidth,grid=false]{img/abl_dynamic_v3.jpg}
    \end{overpic}
    \put(-223,97){\bfseries\scriptsize Input}
    \put(-172,97){\scriptsize SD-1}
    \put(-125,97){\scriptsize SD-2}
    \put(-78,97){\scriptsize SD-3}
    \put(-30,97){\bfseries\scriptsize Ours} 
    %
    \vspace{-10pt}
    \caption{\textbf{Ablation studies on {\module}.}
    Results show that directly synthesizing the static or dynamic details is rather challenging, leading to unreasonable coarse shapes and details (SD-1, SD-2, and SD-3). 
    %
    As a comparison, we leverage the statistical bases with {\module} and regard the detail generation problem as a coefficients regression and interpolation problem, leading to more realistic details.}
    \label{fig:abldynamic}
    \vspace{-10pt}
\end{figure}
%




