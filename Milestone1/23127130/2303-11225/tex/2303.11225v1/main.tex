\documentclass[10pt,twocolumn,letterpaper]{article}



%\usepackage[english,russian]{babel}
\usepackage{hyperref}       % hyperlinks
\usepackage{url}            % simple URL typesetting
\usepackage{booktabs}       % professional-quality tables
\usepackage{amsfonts}       % blackboard math symbols
\usepackage{tablefootnote}
\usepackage{verbatim}
\usepackage{nicefrac}       % compact symbols for 1/2, etc.
\usepackage{microtype}      % microtypography
\usepackage{lipsum}
\usepackage{mathtools}

\usepackage{amsmath,amssymb}
\usepackage{algorithm,algorithmic}
\usepackage{pifont}
\usepackage{cases}
\usepackage{subcaption,graphicx}
\usepackage{stackengine}    % circled symbols
\usepackage{wrapfig}
\usepackage{enumitem}

%\newtheorem{theorem}{Theorem}[section]
%\newtheorem{corollary}[theorem]{Corollary}
%\newtheorem{lemma}[theorem]{Lemma}
\newtheorem{assumption}[theorem]{Assumption}
%\newtheorem{definition}[theorem]{Definition}
%\newtheorem{remark}[theorem]{Remark}
%\newtheorem{proposition}[theorem]{Proposition}

\newcommand*{\LargerCdot}{\raisebox{-0.25ex}{\scalebox{2.4}{$\cdot$}}}
\newcommand{\Sum}{\displaystyle\sum\limits}
\newcommand{\Max}{\max\limits}
\newcommand{\Min}{\min\limits}
\newcommand{\fromto}[3]{{#1}=\overline{{#2},\,{#3}}}
\newcommand{\floor}[1]{\left\lfloor{#1}\right\rfloor}
\newcommand{\ceil}[1]{\left\lceil{#1}\right\rceil}

\newcommand{\tild}{\widetilde}
\newcommand{\eps}{\varepsilon}
\newcommand{\lam}{\lambda}
\newcommand{\ol}{\overline}
\newcommand{\one}{\mathbf{1}}
\newcommand{\cset}{\mathcal{C}}
%\newcommand{\Breg}{\mathcal{D}_{h}}
%\newcommand{\PBreg}{\mathbb{D}_{h}}

%\newcommand{\EndProof}{\begin{flushright}$\square$\end{flushright}}

\newcommand{\circledOne}{\text{\ding{172}}}
\newcommand{\circledTwo}{\text{\ding{173}}}
\newcommand{\circledThree}{\text{\ding{174}}}
\newcommand{\circledFour}{\text{\ding{175}}}
\newcommand{\circledFive}{\text{\ding{176}}}
\newcommand{\circledSix}{\text{\ding{177}}}
\newcommand{\circledSeven}{\text{\ding{178}}}
\newcommand{\circledEight}{\text{\ding{179}}}
\newcommand{\circledNine}{\text{\ding{180}}}
\newcommand{\circledTen}{\text{\ding{181}}}
\newcommand{\balashstar}{\stackMath\mathbin{\stackinset{c}{0ex}{c}{0ex}{\text{\ding{83}}}{\bigcirc}}}
\renewcommand\balashstar{\stackMath\mathbin{\stackinset{c}{0ex}{c}{0ex}{\ast}{\bigcirc}}}


\renewcommand{\le}{\leqslant}
\renewcommand{\ge}{\geqslant}
\renewcommand{\hat}{\widehat}

\newcommand{\numberthis}{\addtocounter{equation}{1}\tag{\theequation}}


\DeclareMathOperator*{\argmin}{arg\,min}
\DeclareMathOperator*{\argmax}{arg\,max}
\DeclareMathOperator*{\Argmin}{Arg\,min}
\DeclareMathOperator*{\Argmax}{Arg\,max}
\DeclareMathOperator{\spn}{span}
\DeclareMathOperator{\kernel}{Ker}
\DeclareMathOperator{\image}{Im}
\DeclareMathOperator{\prox}{prox}
\DeclareMathOperator{\proj}{Proj}
\DeclareMathOperator{\col}{col}
\DeclareMathOperator{\diag}{diag}

\newcommand{\N}{\mathbb{N}}
\newcommand{\R}{\mathbb{R}}
\newcommand{\Z}{\mathbb{Z}}
\newcommand{\V}{\mathbb{V}}
\newcommand{\E}{\mathbb{E}}
%\newcommand{\P}{\mathbb{P}}
\newcommand{\I}{\mathbb{I}}
\newcommand{\F}{\mathbb{F}}

\newcommand{\mA}{{\bf A}}
\newcommand{\mB}{{\bf B}}
\newcommand{\mC}{{\bf C}}
\newcommand{\mD}{{\bf D}}
\newcommand{\mE}{{\bf E}}
\newcommand{\mF}{{\bf F}}
\newcommand{\mG}{{\bf G}}
\newcommand{\mH}{{\bf H}}
\newcommand{\mI}{{\bf I}}
\newcommand{\mJ}{{\bf J}}
\newcommand{\mK}{{\bf K}}
\newcommand{\mL}{{\bf L}}
\newcommand{\mM}{{\bf M}}
\newcommand{\mN}{{\bf N}}
\newcommand{\mO}{{\bf O}}
\newcommand{\mP}{{\bf P}}
\newcommand{\mQ}{{\bf Q}}
\newcommand{\mR}{{\bf R}}
\newcommand{\mS}{{\bf S}}
\newcommand{\mT}{{\bf T}}
\newcommand{\mU}{{\bf U}}
\newcommand{\mV}{{\bf V}}
\newcommand{\mW}{{\bf W}}
\newcommand{\mX}{{\bf X}}
\newcommand{\mY}{{\bf Y}}
\newcommand{\mZ}{{\bf Z}}

\newcommand{\cA}{{\mathcal{A}}}
\newcommand{\cB}{{\mathcal{B}}}
\newcommand{\cC}{{\mathcal{C}}}
\newcommand{\cD}{{\mathcal{D}}}
\newcommand{\cE}{{\mathcal{E}}}
\newcommand{\cF}{{\mathcal{F}}}
\newcommand{\cG}{{\mathcal{G}}}
\newcommand{\cH}{{\mathcal{H}}}
\newcommand{\cI}{{\mathcal{I}}}
\newcommand{\cJ}{{\mathcal{J}}}
\newcommand{\cK}{{\mathcal{K}}}
\newcommand{\cL}{{\mathcal{L}}}
\newcommand{\cM}{{\mathcal{M}}}
\newcommand{\cN}{{\mathcal{N}}}
\newcommand{\cO}{{\mathcal{O}}}
\newcommand{\cP}{{\mathcal{P}}}
\newcommand{\cQ}{{\mathcal{Q}}}
\newcommand{\cR}{{\mathcal{R}}}
\newcommand{\cS}{{\mathcal{S}}}
\newcommand{\cT}{{\mathcal{T}}}
\newcommand{\cU}{{\mathcal{U}}}
\newcommand{\cV}{{\mathcal{V}}}
\newcommand{\cW}{{\mathcal{W}}}
\newcommand{\cX}{{\mathcal{X}}}
\newcommand{\cY}{{\mathcal{Y}}}
\newcommand{\cZ}{{\mathcal{Z}}}

\newcommand{\ba}{{\bf a}}
\newcommand{\bb}{{\bf b}}
\newcommand{\bc}{{\bf c}}
\newcommand{\bd}{{\bf d}}
\newcommand{\be}{{\bf e}}
%\newcommand{\bf}{{\bf f}}
\newcommand{\bg}{{\bf g}}
\newcommand{\bh}{{\bf h}}
\newcommand{\bi}{{\bf i}}
\newcommand{\bj}{{\bf j}}
\newcommand{\bk}{{\bf k}}
\newcommand{\bl}{{\bf l}}
\newcommand{\bm}{{\bf m}}
\newcommand{\bn}{{\bf n}}
\newcommand{\bo}{{\bf o}}
\newcommand{\bp}{{\bf p}}
\newcommand{\bq}{{\bf q}}
\newcommand{\br}{{\bf r}}
\newcommand{\bs}{{\bf s}}
\newcommand{\bt}{{\bf t}}
\newcommand{\bu}{{\bf u}}
\newcommand{\bv}{{\bf v}}
\newcommand{\bw}{{\bf w}}
\newcommand{\bx}{{\bf x}}
\newcommand{\by}{{\bf y}}
\newcommand{\bz}{{\bf z}}

\newcommand{\ds}{\displaystyle}
\newcommand{\norm}[1]{\left\| #1 \right\|}
\newcommand{\normtwo}[1]{\left\| #1 \right\|_2}
\newcommand{\sqn}[1]{\norm{#1}_2^2}
\newcommand{\angles}[1]{\left\langle #1 \right\rangle}
\newcommand{\cbraces}[1]{\left( #1 \right)}
\newcommand{\sbraces}[1]{\left[ #1 \right]}
\newcommand{\braces}[1]{\left\{ #1 \right\}}
\def\<#1,#2>{\langle #1,#2\rangle}

\newcommand{\sigmamax}{\sigma_{\max}(\cA)}
\newcommand{\sigmamaxsqr}{\sigma_{\max}^2(\cA)}
\newcommand{\sigmaminplus}{\sigma_{\min}^+(\cA)}
\newcommand{\sigmaminplussqr}{(\sigma_{\min}^+(\cA))^2}

\usepackage[colorinlistoftodos,bordercolor=blue,backgroundcolor=blue!20,linecolor=blue,textsize=scriptsize]{todonotes}
\newcommand{\arogozin}[1]{\todo[inline]{{\textbf{Alexander R.:} \emph{#1}}}}
\newcommand{\schezhegov}[1]{\todo[inline]{{\textbf{Savelii C.:} \emph{#1}}}}


\iccvfinalcopy % *** Uncomment this line for the final submission

\def\httilde{\mbox{\tt\raisebox{-.5ex}{\symbol{126}}}}

% Pages are numbered in submission mode, and unnumbered in camera-ready

\begin{document}

%%%%%%%%% TITLE
\title{{\name}: High-Fidelity 3D Face Reconstruction by\\Learning Static and Dynamic Details}

\author{Zenghao Chai$^{1}$\thanks{Work done when the author was an intern at MSRA.} ~~~~ Tianke Zhang$^{1}$ ~~~~ Tianyu He$^{2}$ ~~~~ Xu Tan$^{2}$\thanks{Corresponding author: Xu Tan (xuta@microsoft.com).} ~~~~ Tadas Baltru\v{s}aitis$^{3}$\\ HsiangTao Wu$^{4}$ ~~~~ Runnan Li$^{4}$ ~~~~ Sheng Zhao$^{4}$ ~~~~ Chun Yuan$^{1}$ ~~~~ Jiang Bian$^{2}$ \\
$^{1}$Tsinghua University ~~~~~ $^{2}$Microsoft Research Asia ~~~~~ \\$^{3}$Microsoft Mixed Reality \& AI Lab ~~~~~ $^{4}$Microsoft Cloud + AI
}

\maketitle
% Remove page # from the first page of camera-ready.
% \ificcvfinal\thispagestyle{empty}\fi


\begin{abstract}
    3D Morphable Models (3DMMs) demonstrate great potential for reconstructing faithful and animatable 3D facial surfaces from a single image.
    The facial surface is influenced by the coarse shape, as well as the static detail ({\eg}, person-specific appearance) and dynamic detail ({\eg}, expression-driven wrinkles). Previous work struggles to decouple the static and dynamic details through image-level supervision, leading to reconstructions that are not realistic. 
    %
    In this paper, we aim at high-fidelity 3D face reconstruction and propose {\name} to explicitly model the static and dynamic details. 
    Specifically, the static detail is modeled as the linear combination of a displacement basis, while the dynamic detail is modeled as the linear interpolation of two displacement maps with polarized expressions. 
    We exploit several loss functions to jointly learn the coarse shape and fine details with both synthetic and real-world datasets, which enable {\name} to reconstruct high-fidelity 3D shapes with animatable details.
    Extensive quantitative and qualitative experiments demonstrate that {\name} presents state-of-the-art reconstruction quality and faithfully recovers both the static and dynamic details. 
    Our project page: \href{https://project-hiface.github.io}{https://project-hiface.github.io}.   
\end{abstract}


\begin{figure}[t!]
\begin{overpic}[trim=0cm 0cm 0cm 0cm,clip,width=1\linewidth,grid=false]{img/teaser_v1.jpg}
\end{overpic}
\vspace{-12pt}
\caption{We propose {\name} to reconstruct high-fidelity 3D face with realistic and animatable details. \textbf{Reconstruction}: given a single image (1st-row), {\name} faithfully reconstructs a coarse shape (2nd-row) with vivid details (3rd-row). \textbf{Animation}: given a source face (yellow box), {\name} can animate the static (4th-row), dynamic (5th-row), or both (6th-row) details of the driving images (green box). Images are taken from FFHQ~\cite{karras2019style} and CelebA~\cite{CelebAMask-HQ}.
}
\vspace{-15pt}
\label{Fig.teaser}
\end{figure}



\section{Introduction}
\IEEEPARstart{T}{he} method Neural Radiance Fields (NeRF)~\cite{mildenhall2020nerf} is proposed for photorealistic novel view synthesis. Given many views of the scene, it creates implicit multi-view geometry and learns for view synthesis. However, it has poor generalizations to new scenes and requires retraining or fine-tuning on each scene. 
 
 Recent work~\cite{Yu_2021_CVPR,Trevithick_2021_ICCV} has explored the ways of using a single image to train NeRF. They introduce a convolutional feature encoder to learn the image representation which gives it some limited generalization abilities to unseen scenes.  But, without fine-tuning, these methods produce many floats and artifacts in rendering novel views. 
 
  Multi-Plane Images (MPI) representation that learns multiple RGB images from a single image is also used in \cite{Wu_2021_ICCV,Tucker_2020_CVPR,wu2022remote} for  novel view synthesis. However, MPI heavily relies on the qualities of the planar images and needs plenty of image planes to avoid blurs. There is no strong 3D geometry constraint and it fails in many complex scenes.
  
  MINE~\cite{Li_2021_ICCV2} introduces the volume rendering of NeRF into the MPI. It runs faster and produces better depth rendering quality compared with single-view NeRFs~\cite{Yu_2021_CVPR,Trevithick_2021_ICCV}. However, the rendering quality heavily relies on the number of image planes. It needs high-resolution 4D volumes to store the 4-channel  (RGB and volume density) image planes that cost a large amount of GPU memory in both training and 
 prediction.  
 

 
 \begin{figure}[t]
\setlength{\abovecaptionskip}{7pt}
\setlength{\belowcaptionskip}{0pt}
	\centering
% 	\subfigure[MINE (PSNR:14.9)]{  % for AAAI
	\subfloat[MINE (PSNR:14.9)]{
%			\centering
			\includegraphics[width=0.23\textwidth]{figure/intro/DJI_20200223_163206_598_0_MINE.png}
%			\label{subfig:pixelnerf}
	}\subfloat[MINE (depth)]{
%			\centering
			\includegraphics[width=0.23\textwidth]{figure/intro/MINE_disp.png}
%			\label{subfig:mpi}
	}
	\\[-3mm]
	\subfloat[Ours (PSNR:17.0)]{
%			\centering
			\includegraphics[width=0.23\textwidth]{figure/intro/DJI_20200223_163206_598_0_ours.png} 
	}\subfloat[Ours (depth)]{
%			\centering
			\includegraphics[width=0.23\textwidth]{figure/intro/ours_disp.png}
	}
	\caption{Comparison with state-of-the-art methods. (a-b) RGB and depth rendering results of  \cite{Li_2021_ICCV2}. It produces many blurs and floats in the occluded regions and at the object/depth edges. 
	(c-d) Our method employs a joint rendering mechanism that preserves more image details and predicts sharp depth edges.}
	\label{fig:performance_illustration}
\end{figure}
 
 In this paper, we propose a joint rendering mechanism that takes the MPI strategy for coarse sampling proposals and the MLP\&volume-based rendering~\cite{mildenhall2020nerf} for fine sampling and rendering. Then, both the coarse point samples and the fine samples are combined according to their geometry distribution to realize a more accurate joint rendering. More importantly, we introduce a depth teacher net that serves as the guidance for the joint rendering. The monocular depth teacher predicts dense pseudo depth maps that assist the consistent 3D geometry learning between the MPI, the fine volume, and the joint rendering. It also boosts the multi-view geometry consistency between the source view and the target novel views that 
helps handle the occlusions, reduce the blurs and floats, and render accurate depths. 
 
In the experiments,  we verify the effectiveness of our method on three challenging real-scene datasets (RealEstate10K~\cite{zhou2018stereo}, NYU~\cite{silberman2012indoor} and  NeRF-LLFF~\cite{mildenhall2020nerf}) for novel view synthesis or depth estimation. Given a single image as input, our method is shown able to produce higher qualities in both the RGB image rendering and depth map prediction. It far outperforms state-of-the-art methods~\cite{Li_2021_ICCV2,Yu_2021_CVPR} with improvements of 5$\sim$20\% in PSNR and SSIM for the RGB rendering and reduces 20$\sim$50\% of the errors for the depth prediction.
\section{Related Work} \label{sec:related work}
\vspace{-0.2cm}
{\noindent \bf Vision-Language Pre-training.} In the early literature, \cite{Mori99,Frome13,Weston11} explore jointly training image-text embeddings using paired text documents. Recently, some studies have further scaled up the training with large-scale web data to form ``the \textbf{foundation} models'', {\em e.g.}, CLIP~\cite{Radford21}, ALIGN~\cite{Jia21}, Florence~\cite{yuan2021florence}, FILIP~\cite{yao2021filip}, VideoCLIP~\cite{xu2021videoclip}, and LiT~\cite{zhai2022lit}. These foundation models usually contain one visual encoder and one textual encoder, which are trained using simple noise contrastive learning for powerful cross-modal representations. They have shown promising potential in many tasks, such as image classification and detection, action recognition, and retrieval. In this paper, we use CLIP for low-shot temporal action localization, but the same technique should be applicable to other foundation models as well.



\vspace{0.1cm}
{\noindent \bf Prompting} refers to leveraging input instructions to steer foundation models for desired outputs. In the NLP domain, early papers~\cite{Gao21,Jiang20,Timo21,Shin20} focus on handcrafted prompt templates. To avoid labor and increase flexibility, some studies~\cite{Lester21,li21-prefixtuning,li2021prefix} propose learnable prompt tuning at the textual stream, showing strong low-shot generalization. In the CV domain, some recent papers~\cite{zhou2019learn,zhou2022conditional,ju2022prompting} introduce such randomly initialized prompt tuning to handle visual tasks, {\em e.g.}, image understanding~\cite{zhu2022prompt,lu2022prompt,yang2022learning,ma2023diffusionseg} and video understanding~\cite{jia2022visual,nag2022zero,ni2022expanding}. However, these studies ignore lexical ambiguity of category names, and cases that are not easy to describe in text. This paper designs novel conditional prompt tuning and language descriptions from LLMs, to solve these issues. 



\vspace{0.1cm}
{\noindent \bf Closed-set Temporal Action Localization} considers to detect and classify action instances from one pre-defined category list. Specifically, existing methods can be divided into two popular supervisions, {\em i.e.}, strong~\cite{zeng2019graph,lin2021learning,qing2021temporal} and weak~\cite{wang2017untrimmednets,ju2023constraint,ju2020point,yudistira2022weakly}. Strong supervision gives precise boundary labels and category labels for training. There are two detailed pipelines: the top-down framework~\cite{shou2016temporal,shou2017cdc,gao2017turn,chao2018rethinking,lin2017single,xu2017r,tan2021relaxed,zhu2021enriching,wang2022rcl,xu2020g} pre-defines extensive anchors, adopts fixed-length sliding windows to produce initial proposals, then regresses to refine boundaries; the bottom-up framework~\cite{zhao2017temporal,lin2018bsn,lin2019bmn,vo2023aoe,zhao2020bottom,bai2020boundary} learns frame-wise boundary detectors for the boundary frames, then groups extreme frames or estimates action lengths for proposal generation. In addition, several works~\cite{gao2018ctap,liu2019multi,yang2020revisiting} used various fusion strategies to complement these frameworks. On the other hand, weak supervision trains without boundary labels to alleviate annotation costs. The video-level setting learns from category labels~\cite{paul2018w,ju2022distilling}, the CAS-based framework~\cite{liu2019completeness,ju2021adaptive,min2020adversarial,narayan2021d2,lee2019background,lee2021weakly,zhao2021soda} and attention-based framework~\cite{nguyen2018weakly,luo2021action,nguyen2019weakly,shi2020weakly,gao2022fine,he2022asm,huang2021foreground,luo2020weakly,ma2022weakly} have been well studied. To generate better results from CAS or attention, some studies~\cite{shou2018autoloc,liu2019weakly} improved post-processing. To balance cost and performance, some papers introduced single-frame annotations~\cite{ju2021divide,ma2020sf,lee2021learning,yang2021background,mettes2019pointly} or instance-number annotations~\cite{narayan20193c,xu2019segregated}. 

Nevertheless, all the above methods assume that action categories remain identical for training and testing, which is an over-simplification of real application scenarios, limiting practical uses of the vision system.



\vspace{0.1cm} 
{\noindent \bf Low-Shot Temporal Action Localization} considers more realistic scenarios: generalize TAL towards action categories that are unseen (zero-shot) or with several support samples (few-shot). Existing methods~\cite{ju2022prompting,nag2022zero,zhang2022ow,bao2022opental} most rely on foundational models pre-trained on large-scale image-caption pairs for help. Typically, E-Prompt~\cite{ju2022prompting} is the first to construct wide baselines with popular prompt tuning~\cite{Lester21,li21-prefixtuning} and vanilla temporal modeling. STALE~\cite{nag2022zero} explores the one-stage framework to further simplify usage. Although promising, all above methods meet two main challenges: (1) For category semantics, the definition may be vague, inaccurate, or incomplete. (2) For visual motions, temporal modeling may be insufficient. In this paper, for detailed category understanding, we design novel language descriptions from LLMs and vision-conditional prompt tuning; for clearer motion understanding, we introduce optical flows to provide explicit motion inputs. 





\vspace{-0.3em}
\section{Method}
\vspace{-0.3em}

Our sensitivity-aware visual parameter-efficient fine-tuning consists of two stages. In the first stage, SPT measures the task-specific sensitivity for the pre-trained parameters (Section~\ref{subsec:sensitivity}). Based on the parameter sensitivity and a given parameter budget, SPT then adaptively allocates trainable parameters to task-specific important positions (Section~\ref{subsec:SPT}).

\vspace{-0.3em}
\subsection{Task-specific Parameter Sensitivity}
\label{subsec:sensitivity}
\vspace{-0.3em}

Recent research has observed that pre-trained backbone parameters exhibit varying feature patterns~\cite{raghu2021vision,naseer2021intriguing} and criticality~\cite{zhang2019all,chatterji2019intriguing} at distinct positions. 
Moreover, when transferred to downstream tasks, their efficacy varies depending on how much pre-trained features are reused and how well they adapt to the specific domain gap~\cite{yosinski2014transferable,kumar2022finetuning,neyshabur2020being}. Motivated by these observations, we argue that not all parameters contribute equally to the performance across different tasks in PEFT and propose a new criterion to measure the sensitivity of the parameters in the pre-trained backbone for a given task.

Specifically, given the training dataset $\gD_t$ for the $t$-th task and the pre-trained model weights $\vw=\left\{w_1, w_2, \ldots, w_N\right\}\in \sR^N$ where $N$ is the total number of parameters, the objective for the task is to minimize the empirical risk: $\min_{\vw} E(\gD_t, \vw)$.
We denote the parameter sensitivity \bohan{set} as $\gS=\{s_1, \ldots, s_N\}$ and the sensitivity $s_n$ for parameter $w_n$ is measured by the empirical risk difference when tuning it:
\begin{equation}
\vspace{-0.3em}
    \begin{aligned}
        s_n = E(\gD_t, \vw)-E(\gD_t, \vw\mid w_n=w_n^*),
    \end{aligned}
\label{eq:sensitivity}
\end{equation}
where $w_n^*=\underset{w_n}{\rm argmin}(E(\gD_t, \vw))$. We can reparameterize the tuned parameters as  $w_n^*=w_n+\Delta_{w_n}$, where $\Delta_{w_n}$ denotes the update for $w_n$ after tuning. Here we individually measure the sensitivity of each parameter, which is reasonable given that most of the parameters are frozen during fine-tuning in PEFT. However, it is still computationally intensive to compute Eq.~(\ref{eq:sensitivity}) for two reasons. Firstly, getting the empirical risk for $N$ parameters requires forwarding the entire network $N$ times, which is time-consuming. Secondly, it is challenging to derive $\Delta_{w_n}$, as we have to tune each individual $w_n$ until convergence.

{\begin{algorithm}[t!]
\caption{\label{alg:tps} Computing task-specific parameter sensitivities}
\begin{algorithmic}
    \STATE \textbf{Input:} Pre-trained model with network parameters $\vw$, training set $\gD_t$ for the $t$-th task, and number of training samples $C$ used to calculate the parameter sensitivities
    \STATE \textbf{Output:} Sensitivity set $\gS=\{s_1, \ldots, s_N\}$
    \STATE Initialize $\gS=\{0\}^N$
    \FOR{$i\in\{1,\ldots,C\}$}
        \STATE Get the $i$-th training sample of $\gD_t$
	    \STATE Compute loss $E$
		\STATE Compute gradients $\vg$
		\FOR{$n\in\{1,\ldots,N\}$}
                \STATE Update sensitivity for the $n$-th parameter: $s_{n} = s_{n} + g_n^2$
		    \ENDFOR
    \ENDFOR
\end{algorithmic}
\end{algorithm}}


\begin{figure*}[t]
\begin{center}
    \includegraphics[width=\linewidth]{main_figure.pdf}
\end{center}\vspace{-2em}
\caption{Overview of our trainable parameter allocation strategy. With the parameter sensitivity \bohan{set} $\gS$, we first get the top-$\tau$ sensitive parameters. Instead of directly tuning these sensitive parameters, we also boost the representational capability by replacing unstructured tuning with structured tuning at sensitive weight matrices that have a large number of sensitive parameters, which can be implemented by an existing structured tuning method, \eg, LoRA~\cite{hu2022lora} and Adapter~\cite{houlsby2019parameter}. Red lines and blocks represent trainable parameters and modules, while blue lines represent frozen parameters.}
\label{fig:main}
\vspace{-1.5em}
\end{figure*}


To overcome the first barrier, we simplify the empirical loss by approximating $s_n$ in the vicinity of $\vw$ by its first-order Taylor expansion
\vspace{-0.3em}
\begin{equation}
\vspace{-0.5em}
    \begin{aligned}
        s_n^{(1)} = -g_n\Delta_{w_n},
    \end{aligned}
\label{eq:first-order}
\end{equation}
where the gradients $\vg=\partial E/\partial\vw$, and $g_n$ is the gradient of the $n$-th element of $\vg$. 
To address the second barrier, following~\cite{liu2018darts,cai2018proxylessnas}, we take the one-step unrolled weight as the surrogate for $w_n^*$ and approximate $\Delta_{w_n}$ in Eq.~(\ref{eq:first-order}) with a single step of gradient descent. We can accordingly get $s_n^{(1)} \approx g_n^2\epsilon$,
where $\epsilon$ is the learning rate. Since $\epsilon$ is the same for all parameters, we can eliminate it when comparing the sensitivity with the other parameters and finally get 
\vspace{-0.5em}
\begin{equation}
\vspace{-0.3em}
    \begin{aligned}
        s_n^{(1)} \approx g_n^2.
    \end{aligned}
\label{eq:first-order-simp}
\end{equation}
Therefore, the sensitivity of a parameter can be efficiently measured by its potential to reduce the loss on the target domain. Note that although our criterion draws inspiration from pruning work~\cite{molchanov2019importance}, it is distinct from it. \cite{molchanov2019importance} measures the parameter importance by the squared change in loss when removing them, \ie, $\left( E(\gD_t, \vw)-E(\gD_t, \vw\mid w_n=0) \right)^2$ and finally derives the parameter importance by $\left( g_n w_n \right)^2$, which is different from our formulations in Eqs.~(\ref{eq:sensitivity}) and~(\ref{eq:first-order-simp}).

In practice, we accumulate $\gS$ from a total number of $C$ training samples ahead of fine-tuning to generate accurate sensitivity as shown in Algorithm~\ref{alg:tps}, where $C$ is a pre-defined hyper-parameter. In Section~\ref{subsec:abl}, we show that employing only 400 training samples is sufficient for getting reasonable parameter sensitivity, which requires only 5.5 seconds with a single GPU for any VTAB-1k dataset with ViT-B/16 backbone~\cite{vit}.

\vspace{-0.3em}
\subsection{Adaptive Trainable Parameters Allocation}
\label{subsec:SPT}
\vspace{-0.2em}

Our next step is to allocate trainable parameters based on the obtained parameter sensitivity set $\gS$ and a desired parameter budget $\tau$. A straightforward solution is to directly tune the top-$\tau$ most sensitive unstructured connections (parameters) \rev{while keeping the rest frozen}, which we name unstructured tuning. Specifically, we select the top-$\tau$ most sensitive weight connections in $\gS$ to form the sensitive weight connection set $\gT$. Then, for \rev{a} weight matrix $\mW\in \sR^{d_{\rm in}\times d_{\rm out}}$, we can get a binary mask $\mM\in \sR^{d_{\rm in}\times d_{\rm out}}$ computed by
\vspace{-0.5em}
\begin{equation}
\vspace{-0.5em}
    {\begin{array}{ll}
    \small
    \begin{aligned}
    \mM^j =
    \left\{\begin{array}{ll} 
    1 ~~~~~ \mW^j \in \gT \\
    0 ~~~~~ \mW^j \notin \gT
    \end{array}\right.
    \end{aligned},
    \small
    \end{array}}
\label{eq:mask}
\end{equation}
where $\mW^j$ and $\mM^j$ are the $j$-th element in $\mW$ and $\mM$, respectively. Accordingly, we can train the sensitive parameters by gradient descent and the updated weight matrix can be formulated as $\mW'\leftarrow \mW - \epsilon\vg_{\mW}\odot\mM$, where $\vg_{\mW}$ is the gradient for $\mW$.

However, considering PEFT approaches generally limit the proportion of trainable parameters to less than 1\%, tuning only a small number of unstructured weight connections might not have enough representational capability to handle the downstream datasets with large domain gaps from the source pre-training data. Therefore, to improve the representational capability, we propose to replace unstructured tuning with structured tuning at the sensitive weight matrices that have a high number of sensitive parameters. To preserve the parameter budget, we can implement structured tuning with an existing efficient structured tuning PEFT method~\cite{hu2022lora,chen2022adaptformer,houlsby2019parameter,jie2022convolutional} that learns to directly adjust \rev{all hidden dimensions at once}. We depict an overview of our trainable parameter allocation strategy in Figure~\ref{fig:main}. For example, we can employ the low-rank reparameterization trick LoRA~\cite{hu2022lora} to the sensitive weight matrices \rev{and the one-step update for $\mW$ can be formulated as}
\vspace{-0.4em}
\begin{equation}
\vspace{-0.4em}
    {\begin{array}{ll}
    \small
    \begin{aligned}
    \mW' = \left\{\begin{array}{ll} 
    \mW + \mW_{\rm down}\mW_{\rm up} & ~~ \text { if } ~~ \sum_{j=0}^{d_{\rm in}\times d_{\rm out}} \mM^j \geq \sigma_{\rm opt} \\
    \mW - \epsilon\vg_{\mW}\odot\mM & ~~ {\rm otherwise}
    \end{array}\right.
    \end{aligned},
    \small
    \end{array}}
\label{eq:weight_updat}
\end{equation}
where $\mW_{\rm down}\in \sR^{d_{\rm in}\times r}$ and $\mW_{\rm up}\in \sR^{r\times d_{\rm out}}$ are two learnable low-rank matrices to approximate the update of $\mW$ and rank $r$ is a hyper-parameter where $r \ll {\rm min}(d_{\rm in},d_{\rm out})$. In this way, we perform structured tuning on $\mW$ when its number of sensitive parameters exceeds $\sigma_{\rm opt}$, whose value depends on the pre-defined type of structured tuning method. For example, since implementing structured tuning with LoRA requires $2\times d_{\rm in} \times d_{\rm out} \times r$ trainable parameters for each sensitive weight matrix, we set $\sigma_{\rm LoRA} \leftarrow 2\times d_{\rm in} \times d_{\rm out} \times r$ to ensure that the number of trainable parameters introduced by structured tuning is always equal to or lower than the number of sensitive parameters.

In this way, our SPT adaptively incorporates both structured and unstructured tuning granularities to enable higher flexibility and stronger representational power, simultaneously. In Section~\ref{subsec:abl}, we show that structured tuning is important for the downstream tasks with larger domain gaps and both unstructured and structured tuning contribute clearly to the superior performance of our SPT.
We have conducted adequate experiments in this section to reveal what kinds of objectness is beneficial for query-based methods in open-world instance segmentation task. 

\begin{table*}[htbp]
% \renewcommand\arraystretch{1.2}
\centering
\caption{Results of cross-category generalizability evaluation on VOC$\to$Non-VOC scenario. $\mathrm{AR^{box}}$ denotes the box AR performance at a budget of 100. $\mathrm{AR}$ without superscript denotes mask AR performance at a budget of 100. \textbf{Bold} scores denote the best performance of each metric. OpenInst$^*$ denotes OpenInst trained with pseudo annotations produced by GGN.}
\label{tab:exp:voc}
\resizebox{\textwidth}{!} 
{
\begin{tabular}{lcccccccc}
\toprule
Methods & Ref & $\mathrm{AR^{box}}$ & $\mathrm{AR}$ & $\mathrm{AR_{0.5}}$ & $\mathrm{AR_{0.75}}$  & $\mathrm{AR_{small}}$ & $\mathrm{AR_{med}}$ & $\mathrm{AR_{large}}$ \\
\midrule
OLN~\cite{oln}        & ICRA22      & \textbf{33.0}          & 26.9          & -             & -                & -              & -             & -       \\
LDET~\cite{ldet}      & ECCV22      & 30.8          & 27.4          & -             & -                & -              & -             & -       \\
GGN~\cite{ggn}        & CVPR22      & 31.5          & 28.7          & -             & -                & -              & -             & -       \\
SOIS~\cite{sois}      & ARXIV22     &  -            & 11.0          &  -            &  -               &  4.9           &  9.2          & 24.8    \\
\textbf{OpenInst}     & -             & 32.0          & 28.2          & 44.8          & 29.4             & \textbf{11.6}           & 33.4          & 54.8    \\
\textbf{OpenInst$^*$} & -             & 33.0          & \textbf{30.1}          & \textbf{47.5}          & \textbf{31.9}             &  7.4           & \textbf{38.5}          & \textbf{64.2}    \\
\bottomrule
\end{tabular}
}
\end{table*}

\subsection{Datasets and Evaluation}
We conduct our experiments on six popular datasets: COCO~\cite{coco}, UVO~\cite{uvo}, Objects365~\cite{objects365}, Mapillary Vistas~\cite{mapillary}, LVIS~\cite{lvis}, and Cityscapes~\cite{cityscapes}, which are widely used in the closed-world instance segmentation task. LVIS, Mapillary Vistas, and Objects365 are only used for evaluation. Cityscapes is only used for training. COCO and UVO are used for both training and evaluation. 

\textbf{COCO} is a widely-used dataset in object detection and instance segmentation.
Following OLN and LDET, we use the train2017 split of COCO for training and the val2017 split for evaluation, which contain 117k and 5k images respectively. COCO covers 80 object categories, which are a superset of categories in the PASCAL VOC dataset.
We use COCO in both cross-category and cross-dataset generalizability evaluations. 
\textbf{UVO} is a class-agnostic and exhaustively labeled dataset. It is specially designed for the open-world instance segmentation task. We use UVO v0.5 for a fair comparison with other methods. UVO v0.5 contains 15315 images for training and 7356 images for validation. Following OLN, GGN, and LDET, we mainly use the validation split for cross-dataset generalizability evaluations. Following SOIS, we also conduct the so-called inner-dataset evaluation, which means using the training set of UVO v0.5 to train the model, and using the validation set of UVO v0.5 for evaluation. 
\textbf{Objects365} is large scale dataset for object detection and has 80k images for validation. Following LDET, we sample 5k images from the validation split of Objects365 for cross-dataset generalizability evaluation. Since Objects365 only has box-level annotations. We only apply open-world object detection evaluation on Objects365.
\textbf{Mapillary Vistas} is a street-centric dataset.
We use the validation split of Mapillary Vistas on version v1.2 for cross-dataset generalizability evaluation.
\textbf{LVIS} is a large-vocabulary dataset for instance segmentation. It contains more than 1200 categories. Following SOIS, we use the validation split (20k images) of LVIS for cross-dataset generalizability evaluation.
\textbf{Cityscapes} consists of urban scene images from 50 different cities. We use the 8 foreground classes from Cityscapes for training, and test the model on the validation split of Mapillary Vistas.


We use the average recall (AR) as the indicator to quantitatively measure the generalizability of different models. Since the AR metric is commonly used in many open-world instance segmentation works~\cite{oln,oln-fcos,ggn,ldet,sois}. Following OLN~\cite{oln}, GGN~\cite{ggn}, and LDET~\cite{ldet}, we also conduct experiments in two main settings: cross-category and cross-dataset.
AR@k denotes the average recall at a budget of k, which means only top k predictions of each image are used for calculating the average recall.

\textbf{Cross-category.}
We conduct cross-category generalizability evaluation only on one scenario: VOC$\to$Non-VOC. We only use annotations belonging to PASCAL VOC~\cite{pascalvoc} categories for supervision in the training phase. And we use annotations belonging to other categories for evaluation. In this experiment, we set the query number to 150. Because predictions belong to $C_{base}$ categories are excluded from the budget of the recall in evaluation.

\textbf{Cross-dataset.}
Cross-dataset generalizability evaluation uses two different datasets for training and testing respectively. The testing dataset contains both $C_{base}$ and $C_{novel}$ categories. We conduct cross-dataset generalizability evaluation on four scenarios: COCO$\to$UVO, COOC$\to$Objects365, COCO$\to$LVIS, and Cityscapes$\to$Mapillary Vistas.

\subsection{Implementations Details}
We build OpenInst on the powerful MMDetection~\cite{mmdet} library. We benchmark our method against the advanced QueryInst~\cite{queryinst} method. We use ResNet-50~\cite{resnet} as the backbone of our model, and leverage BiFPN~\cite{bifpn} instead of vanilla FPN~\cite{fpn} as the neck module. We use box IoU to displace the classification label as the learning objective unless otherwise specified. The box head is trained with L1 loss and GIoU loss~\cite{giou}, whose loss weights are set to 5.0 and 2.0 respectively. The mask head is trained with the dice loss. The number of decoder layer is set to 6 in all experiments.
We use the Adam~\cite{adam} optimizer as our solver with the learning rate set to 1e-4 and weight decay set to 5e-4. For \textbf{1x} configuration, we set the total epoch to 12, and make it decayed at epoch 8 and epoch 11 by 0.1 respectively. For \textbf{3x} configuration, the decay point is set to epoch 27 and epoch 33 respectively. We adopt RandomFlip as the only data augmentation method in our data pipeline for \textbf{1x} configuration and apply \textbf{L}arge \textbf{S}cale \textbf{J}ittor (LSJ) for \textbf{3x} configuration for fair comparison.

% \subsection{Main Results}
\begin{table*}[htbp]
% \renewcommand\arraystretch{1.2}
\centering
\caption{Results of cross-category generalizability evaluation on COCO$\to$UVO scenario. $\mathrm{AR^{box}}$ denotes the box AR performance at a budget of 100. $\mathrm{AR}$ without superscript denotes the mask AR performance at a budget of 100. \textbf{Bold} scores denote the best performance of each metric. Aux. denotes that the method contains an auxiliary model and pseudo annotations.}
\label{tab:exp:uvo}
\resizebox{\textwidth}{!} 
{
\begin{tabular}{lcccccccccc}
\toprule
Methods & Ref & Aux. & Epochs & $\mathrm{AR^{box}}$ & $\mathrm{AR}$ & $\mathrm{AR_{0.5}}$ & $\mathrm{AR_{0.75}}$  & $\mathrm{AR_{small}}$ & $\mathrm{AR_{med}}$ & $\mathrm{AR_{large}}$ \\
\midrule
%OLN       &             & 8 & -             & -             & -             & -                & -              & -     & -      \\
LDET~\cite{ldet}    & ECCV22  &             & 8  & 47.5          & 40.7          & -             & -                & 26.8           & 40.0           & 45.7   \\
GOOD~\cite{good}  & ICLR23    &  \checkmark & 8 & 50.3  & - & - & - & - & - & - \\
GGN~\cite{ggn}    & CVPR22    & \checkmark  & 8  & 52.8          & 43.4          & 71.7          & 44.5             & 23.3           & 44.4           & 50.0   \\
SOIS~\cite{sois}  & ARXIV22    & \checkmark  & 36 & -             & 51.3          & -             & -                & -              & -              & -      \\
\textbf{OpenInst} & -          &             & 12 & 59.1          & 48.7          & 72.6          & 51.4             & 26.4           & 44.3           & 60.4   \\
\textbf{OpenInst} & -          &             & 36 & \textbf{63.0} & \textbf{53.3} & \textbf{76.6} & \textbf{56.8}    & \textbf{31.8}  & \textbf{49.4}  & \textbf{64.3}   \\
\bottomrule
\end{tabular}
}
\end{table*}

\begin{table*}[htbp]
% \renewcommand\arraystretch{1.2}
\centering
\caption{Results of cross-category generalizability evaluation on COCO$\to$Objects365 scenario. $\mathrm{AR^{box}}$ denotes the box AR performance at a budget of 100. \textbf{Bold} scores denote the best performance of each metric.}
\label{tab:exp:obj365}
\begin{tabular}{lcccccc}
\toprule
Methods & $\mathrm{AR^{box}}$ & $\mathrm{AR_{0.5}^{box}}$ & $\mathrm{AR_{0.75}^{box}}$  & $\mathrm{AR_{small}^{box}}$ & $\mathrm{AR_{med}^{box}}$ & $\mathrm{AR_{large}^{box}}$ \\
\midrule
Mask R-CNN~\cite{mask-rcnn}  & 38.5          & -             & -             & 24.0             & 40.1           & 50.2   \\
LDET~\cite{ldet}      & 41.1          & -             & -             & 26.1             & 43.8           & 52.8   \\
% GGN       & 00.0          & 00.0          & 52.9          & 33.5             & 50.7           & 61.9   \\
% SOIS      & 00.0          & 00.0          & \textbf{53.6} & 34.3             & \textbf{51.4}  & \textbf{62.6}   \\
\textbf{OpenInst} & \textbf{50.1}          & 64.1          & 53.0          & \textbf{29.8}             & \textbf{51.8}           & \textbf{66.6}   \\
\bottomrule
\end{tabular}
\end{table*}


\begin{table}[htbp]
% \renewcommand\arraystretch{1.2}
\centering
\caption{Results of cross-category generalizability evaluation on COCO$\to$LVIS scenario. $\mathrm{AR}$ denotes the mask AR performance at a budget of 100. \textbf{Bold} scores denote the best performance of each metric.}
\label{tab:exp:lvis}
\begin{tabular}{ccccccc}
\toprule
Methods & $\mathrm{AR}$ \\
\midrule
Mask R-CNN~\cite{mask-rcnn}    & 22.4  \\
LDET~\cite{ldet}        & 25.1  \\
Mask2Former~\cite{mask2former} & 24.5  \\
SOIS~\cite{sois}        & 25.2  \\
\textbf{OpenInst}   & \textbf{29.3}  \\
\bottomrule
\end{tabular}
\end{table}

\begin{table*}[htbp]
% \renewcommand\arraystretch{1.2}
\centering
\caption{Results of cross-category generalizability evaluation on Cityscapes$\to$Mapillary Vistas scenario. $\mathrm{AR}$ denotes the mask AR performance at a budget of 100. \textbf{Bold} scores denote the best performance of each metric.}
\label{tab:exp:mapillary}
\begin{tabular}{ccccccc}
\toprule
Methods & $\mathrm{AR}$ & $\mathrm{AR_{0.5}}$ & $\mathrm{AR_{0.75}}$  & $\mathrm{AR_{small}}$ & $\mathrm{AR_{med}}$ & $\mathrm{AR_{large}}$ \\
\midrule
% OLN       & 00.0          & 00.0          & 50.8          & \textbf{35.4}    & 49.5           & 57.7   \\
Mask R-CNN~\cite{mask-rcnn}      &  8.4          & 16.3          & -             & -                & -              & -      \\
LDET~\cite{ldet}      & 10.6          & \textbf{21.8}          & -             & -                & -              & -      \\
% GGN       & 00.0          & 00.0          & 52.9          & 33.5             & 50.7           & 61.9   \\
% SOIS      & 00.0          & 00.0          & \textbf{53.6} & 34.3             & \textbf{51.4}  & \textbf{62.6}   \\
\textbf{OpenInst} & \textbf{11.6}          & 18.1          & 11.9          &  3.0             & 12.9           & 33.6   \\
\bottomrule
\end{tabular}
\end{table*}

\subsection{Cross-category Generalizability Evaluation}

\textbf{VOC}$\to$\textbf{Non-VOC.}
In the VOC$\to$Non-VOC scenario, we only concern with the performance of predictions matched with $C_{novel}$. Predictions matched with $C_{base}$ categories are not taken into account when calculating the recall scores. Therefore, we set the query number to 150 for a fair comparison with fellow works. The initialization is also changed from "Image Initialization" to "Random Initialization". Because "Image Initialization" focus more on $C_{base}$ objects in the refining process of the decoder, and thus weakens the localization performance on $C_{novel}$ objects. 
As shown in Fig.~\ref{tab:exp:voc}, OpenInst achieves comparable results against dense proposal-based methods~\cite{oln,ldet,ggn}. When compared with the query-based method SOIS, OpenInst obtains significant advantages. Powered by pseudo annotations, OpenInst$^*$ achieves state-of-the-art results on both boxes as well as mask AR and suppresses other methods by a notable margin. Besides, we have noticed that pseudo annotations mainly help improve the performance of medium and large-size novel objects, while downgrading the performance of small-size objects.


\subsection{Cross-dataset Evaluation}

\textbf{COCO}$\to$\textbf{UVO} is the most important scenario of cross-dataset generalizability evaluation. As shown in Tab.~\ref{tab:exp:uvo}, OpenInst achieves state-of-the-art results when trained with 12 epochs. OpenInst suppresses all dense proposal-based and query-based methods by a large margin. The mask AR score of OpenInst reaches 53.3, which is nearly 10 points higher than the advanced GGN~\cite{ggn}. When compared with query-based methods, OpenInst exceeds SOIS~\cite{sois} by 2 points on the mask AR score. Note that both GGN and SOIS have an auxiliary model and leverage pseudo annotations. Though an auxiliary model and pseudo annotations can enhance the detector, they also make the detector much heavier and the training pipeline more complicated. OpenInst possesses two advantages: better results and a simpler structure. LDET~\cite{ldet} is a dense proposal based without auxiliary models and pseudo annotation. Compared with LDET, we can find that the improvement of OpenInst trained with 12 epochs mainly comes from large objects. LDET suppress OpenInst trained with 12 epochs on $\mathrm{AR_{small}}$ by 0.4. We presume that query-based methods like OpenInst work better on large objects. Because the initialization of query boxes in this scenario is the size of the full images, which means query boxes and features are more likely to notice large objects. LDET leverages densely spread proposals of various sizes, which enables LDET to take objects of all sizes into account. When trained with 36 epochs with LSJ data augmentation, OpenInst obtains significant improvements on all size objects.

\textbf{COCO}$\to$\textbf{Objects365.}
Objects365 only has box-level annotations. We only use the box AR as the metric for evaluation. The original validation split of Objects365 contains 80k images. Following LDET~\cite{ldet}, we use the same subset of the original validation split for evaluation. The evaluation subset consists of 5k images. The taxonomy of Objects365 contains 365 categories and is a superset of the COCO taxonomy (80 categories). Objects365 contains all $C_{base}$ objects and lots of $C_{novel}$ objects. As shown in Tab.~\ref{tab:exp:obj365}, OpenInst outperforms LDET by a large margin. The box AR score of OpenInst reaches 50.1 and suppresses the score of LDET by 9.0 points. From the performance of the three object sizes, we can observe that the principal gap between the performance of LDET and OpenInst comes from large objects. This observation is consistent with that of the COCO$\to$UVO scenario.

\textbf{COCO}$\to$\textbf{LVIS} scenario is introduced by SOIS~\cite{sois}. LVIS is built upon COCO images but has more granular annotations. LVIS has 1203 categories while COCO has only 80. The taxonomy of LVIS is far bigger than that of COCO. Therefore, despite the overlapped images in the training and testing split, annotations of the overlapped images have a huge difference between the training split and the testing split. For those overlapped images, annotations from the training split do not help them cheat on the testing split. Results from COCO to LVIS can reveal the generalizability of the model all the same. As shown in Tab.~\ref{tab:exp:lvis}, OpenInst outperforms all available methods by at least 4.1 AR and achieves state-of-the-art results on the COCO$\to$LVIS scenario.

\textbf{Cityscapes}$\to$\textbf{Mapillary Vistas.}
Following LDET~\cite{ldet}, We train the detector using 8 object-level categories of Cityscapes: \textit{car, bicycles, motorcycle, train, bus, truck, person, rider}. For evaluation, we use the 35 foreground object-level categories of Mapillary Vistas. The number of categories in evaluation is four times as many as in training. As shown in Tab.~\ref{tab:exp:mapillary}, OpenInst achieves 29.3 mask AR@100, which promotes the performance of the query-based method SOIS by 4.1 AR.

\subsection{Ablation Study}

\begin{table}[htbp]
% \renewcommand\arraystretch{1.2}
\centering
\caption{Results on class-agnostic dataset UVO. AR@100 and AR@10 denote mask AR performance at a budget of 100 and 10 respectively. \textbf{Bold} scores denote the best performance of each metric.}
\label{tab:exp:uvo2uvo}
\begin{tabular}{ccc}
\toprule
Methods & $\mathrm{AR@100}$ & $\mathrm{AR@10}$ \\
\midrule
Mask R-CNN~\cite{mask-rcnn}         & 22.8          & 20.0             \\
LDET~\cite{ldet}             & 35.6          & 23.7             \\
SOIS~\cite{sois}             & 41.9          & \textbf{29.2}             \\
\textbf{OpenInst}        & \textbf{43.1} & 20.8             \\
\bottomrule
\end{tabular}
\end{table}


\begin{table*}[htbp]
% \renewcommand\arraystretch{1.2}
\centering
\caption{Effect of different geometric cues on COCO$\to$UVO scenario. $\mathrm{AR}^{box}$ denotes the box AR performance at a budget of 100. $\mathrm{AR}$ without superscript denotes the mask AR performance at a budget of 100. \textbf{Bold} scores denote the best performance of each metric. OpenInst-cls denotes the vanilla QueryInst trained in a class-agnostic way.}
\label{tab:exp:geocue}
\begin{tabular}{lccccccc}
\toprule
Methods & $\mathrm{AR^{box}}$ & $\mathrm{AR}$ & $\mathrm{AR_{0.5}}$ & $\mathrm{AR_{0.75}}$  & $\mathrm{AR_{small}}$ & $\mathrm{AR_{med}}$ & $\mathrm{AR_{large}}$ \\
\midrule
OpenInst-void   & 58.4          & 48.5          & 71.3          & 51.3          & 25.1             & 43.7           & \textbf{60.8}   \\
OpenInst-cls    & 55.7          & 44.9          & 72.1          & 46.7          & 24.0             & 41.8           & 55.2   \\
OpenInst-box    & \textbf{59.1} & \textbf{48.7} & \textbf{72.6} & \textbf{51.4} & \textbf{26.4}    & \textbf{44.3}  & 60.4   \\
OpenInst-mask   & 58.1          & 47.8          & 71.2          & 50.6          & 24.4             & 43.3           & 60.1 \\
OpenInst-fusion & 58.6          & 48.1          & 72.0          & 50.8          & 25.6             & 43.9           & 59.8   \\
\bottomrule
\end{tabular}
\end{table*}


\begin{table*}[htbp]
% \renewcommand\arraystretch{1.2}
\centering
\caption{Effect of different modules on COCO$\to$UVO scenario. $\mathrm{AR}^{box}$ denotes the box AR performance at a budget of 100. $\mathrm{AR}$ without superscript denotes the mask AR performance at a budget of 100. \textbf{Bold} scores denote the best performance of each metric}
\label{tab:exp:module}
\resizebox{\textwidth}{!} 
{
\begin{tabular}{lcccccccc}
\toprule
DCN~\cite{dcnv2} & BiFPN~\cite{bifpn} & $\mathrm{AR^{box}}$ & $\mathrm{AR}$ & $\mathrm{AR_{0.5}}$ & $\mathrm{AR_{0.75}}$  & $\mathrm{AR_{small}}$ & $\mathrm{AR_{med}}$ & $\mathrm{AR_{large}}$ \\
\midrule
           &            & 56.3          & 46.4          & 71.3          & 48.9          & 24.3             & 42.2           & 57.8   \\
\checkmark &            & 57.9          & 47.9          & \textbf{72.8} & 50.5             & 25.3          & 43.8           & 59.4   \\
           & \checkmark & 58.4          & 47.8          & 71.4          & 50.5             & 25.8           & 43.4          & 59.3   \\
\checkmark & \checkmark & \textbf{59.1} & \textbf{48.7} & 72.6          & \textbf{51.4} & \textbf{26.4}    & \textbf{44.3}  & \textbf{60.4}   \\
% OpenInst & &  & 00.0          & 00.0          & 52.9          & 33.5             & 50.7           & 61.9   \\
\bottomrule
\end{tabular}
}
\end{table*}

\textbf{Open-world dataset evaluation.} Following SOIS~\cite{sois}, we conduct extra experiments on the open-world dataset UVO. We use the training split (15k images) and validation split (7856 images) of UVO for training and evaluation respectively. For a fair comparison, we set the image size to 640, to be consistent with SOIS. As shown in Tab.~\ref{tab:exp:uvo2uvo}, OpenInst achieves the best performance on the mask AR@100. OpenInst suppresses the prior query-based SOIS by 1.2 points. Whereas, OpenInst performs poorly on the mask AR@10 metric. The result of OpenInst on AR@10 indicates that using the box IoU as a ranking indicator is inadequate. Box IoU is not a good ranking indicator.

\textbf{Effect of geometric cues.} We use OpenInst-void as the baseline model. Based on this baseline model, we add the classification, box IoU, mask IoU, and the geometric mean of box and mask IoU as the learning objective respectively for training. The combination of OpenInst and classification is a vanilla QueryInst trained in a class-agnostic way. The remaining combinations are illustrated in Sec.~\ref{sec:method}. As shown in Tab.~\ref{tab:exp:geocue}, OpenInst-void has achieved impressive results. These results effectively demonstrate that \textbf{explicitly learning objectness is not crucial in the open-world localization and instance segmentation problem}, which is a highly insightful discovery. When classification is applied as a learning objective, the performances of box and mask AR drop by 2.7 and 3.6 respectively. The performance on $\mathrm{AR_{0.75}}$ experienced a noticeable decrease of 4.3 points. It can be inferred that the classification primarily resulted in insufficient fineness instead of accuracy in the predictions. The impacts of the box and mask IoU is minor than that of the classification. The box IoU has a positive effect on the generalizability while the mask IoU has a negative effect. We presume that learning the mask IoU is more difficult than learning box IoU. Because masks always have irregular shapes and boxes are always a rectangle. Being biased to the difficult mask IoU learning objective degrades the generalizability of the model. From Tab.~\ref{tab:exp:geocue} we can observe that only setting box IoU as the learning objective can slightly improve the performance. Query-based methods without learning any objectness have already been good detectors for the open-world instance segmentation task.

\textbf{Effect of DCN and BiFPN}. As shown in Tab.~\ref{tab:exp:module}, both DCN~\cite{dcnv2} and BiFPN~\cite{bifpn} have a positive effect on all metrics. The deformable mechanism of DCN expands the receptive field of the detector, making it larger and irregular. The detector can find more objects from a larger receptive field and output more precise predictions through irregular shapes. As can be seen in the second row of Tab.~\ref{tab:exp:module}, DCN improves the performance of $\mathrm{AR_{0.5}}$ by an increase of 1.5 points. In comparison, BiFPN brings even greater improvements to small objects. BiFPN designs a dedicated structure that makes full use of multi-scale feature maps. Feature maps with high resolution are enhanced, which makes it easier to locate small objects. After combining the two modules, the mask AR performance is further boosted to 48.7. This shows that DCN and BiFPN are complementary to each other in the open-world instance segmentation task.
% \subsection{Visualization}
% % 

\section{Ablation Studies}


\myparagraph{Ablation Studies on Loss Functions and Datasets.}
%
We present ablation studies on the proposed loss functions and training strategy with hybrid datasets.
We train {\name} with synthetic data alone and compare it to the one trained with hybrid datasets.
%
For coarse shape reconstruction, we investigate the contribution of $\mathcal{L}_{\text{shp}}$ and its sub-term $\mathcal{L}_{\text{kl}}$.
%
For detail reconstruction, we compare {\name} without $\mathcal{L}_{\text{detail}}$ and $\mathcal{L}_{\text{kd}}$, respectively. The results are presented in Fig.~\ref{fig.abl_loss}.







Fig.~\ref{fig.abl_loss} demonstrates that the proposed loss functions and training strategy from hybrid datasets contribute to satisfactory coarse shape and details.
%
First, the model trained with synthetic data alone cannot generalize well to real-world images, which indicates the necessity to train with real-world data.
%
Second, $\mathcal{L}_{\text{shp}}$ improves the coarse shape reconstruction quality. $\mathcal{L}_{\text{shp}}$ is effective in tackling challenging poses and improving alignment. $\mathcal{L}_{\text{kl}}$ can relieve the overfitting risk on the synthetic data and improve the generalization to real-world images.
%
%
Third, without $\mathcal{L}_{\text{detail}}$ or $\mathcal{L}_{\text{kd}}$, the reconstructed details exhibit random noise and cannot faithfully reflect person-specific details. Such noise misses the correspondence to the person-specific identity.



\myparagraph{Ablation Studies on {\module}.}
%
To verify the effectiveness of building bases for static and dynamic details, we present detailed ablation studies on {\module}, by replacing the bases ({\ie}, $\mathbf{B}_{\text{sta}}$ and $\mathbf{B}_{\text{com}}/\mathbf{B}_{\text{str}}$) reconstruction with a U-Net decoder~\cite{ronneberger2015u} (same as DECA~\cite{feng2021learning}). Therefore, the model learns to directly synthesize displacement maps instead of predicting corresponding coefficients like ours.
% 
In Fig.~\ref{fig:abldynamic}, we make comparisons on: 1). directly synthesizing $\mathbf{D}_{\text{dyn}}$ (SD-1), 2). directly synthesizing $\mathbf{D}_{\text{com}}/\mathbf{D}_{\text{str}}$ and interpolating via Eq.~\ref{Eq.dyn_detail} (SD-2), 3). directly synthesizing $\mathbf{D}_{\text{sta}}$ (SD-3), 4). {\module} (Ours).
%
It can be seen that, due to the high diversity and complexity of expression representation, it is hard to directly learn realistic details even with ground-truth labels of $\mathbf{D}_{\text{dyn}}$ from synthetic data (see SD-1, SD-2 and SD-3 in Fig.~\ref{fig:abldynamic}).
More specifically, for the static details, directly synthesizing displacement maps bring much noise (SD-3). For example, the hollow eyebrow is demonstrated in the second row.
For the dynamic details, directly synthesizing displacement maps even leads to unnatural results (SD-1 and SD-2). For example, the reconstructed 3D faces are distorted especially in the second row. We also notice that, directly synthesizing $\mathbf{D}_{\text{dyn}}$ (SD-1) achieves inferior results than directly synthesizing $\mathbf{D}_{\text{com}}/\mathbf{D}_{\text{str}}$ and interpolating via Eq.~\ref{Eq.dyn_detail} (SD-2). This demonstrates that it is beneficial to simplify the expression representation by using interpolation between two displacement maps (\ie, compressed and stretched).
%
In conclusion, these observations further verify our insight on relaxing the challenging detail generation problem into a feasible coefficients regression problem.


\begin{figure}[!t]
    \centering
    % \vspace{-9pt}
    \begin{overpic}[trim=0cm 0cm 0cm 0cm,clip,width=1\linewidth,grid=false]{img/abl_dynamic_v3.jpg}
    \end{overpic}
    \put(-223,97){\bfseries\scriptsize Input}
    \put(-172,97){\scriptsize SD-1}
    \put(-125,97){\scriptsize SD-2}
    \put(-78,97){\scriptsize SD-3}
    \put(-30,97){\bfseries\scriptsize Ours} 
    %
    \vspace{-10pt}
    \caption{\textbf{Ablation studies on {\module}.}
    Results show that directly synthesizing the static or dynamic details is rather challenging, leading to unreasonable coarse shapes and details (SD-1, SD-2, and SD-3). 
    %
    As a comparison, we leverage the statistical bases with {\module} and regard the detail generation problem as a coefficients regression and interpolation problem, leading to more realistic details.}
    \label{fig:abldynamic}
    \vspace{-10pt}
\end{figure}
%





\section{Conclusion and Future Work}
\label{sec:conclusion}

We have presented a novel neural network that successively learns shape sketch and extrusion without any expensive annotations of shape segmentation and labels as the supervision.
%Without the guidance of sketch labels, 
Our approach is able to learn smooth sketches, followed by the differentiable extrusion to reconstruct CAD models that are close to the ground truth. 
We evaluate SECAD-Net using diverse CAD datasets and demonstrate the advantages of our approach by ablation studies and comparing it to the state-of-the-art methods. 
We further demonstrate our method’s applicability in single-image CAD reconstruction. 
Additionally, the CAD shapes generated by our approach can be directly fed into off-the-shelf CAD software for sketch-level or cylinder primitive-level editing. 

% We tested SE-Net on ABC dataset and Fusion 360 dataset. Quantitative results demonstrate that SE-Net can efficiently reconstruct 3D CAD shapes. Qualitative results show that our model can learn fine 2d sketches without any associated ground-truth.


% We propose SE-Net, a network that successively learns shape sketch and extrusion in an unsupervised manner. The CAD shapes generated by the network can be directly sent to off-the-shelf CAD software for sketch-level or cylinder primitive-level editing. SE-Net can be reconstructed to generate smooth sketches and the reconstruction effect is due to the current state-of-the-art, including supervised methods. Additionally, our method is the first to learn sketches from raw shapes without the guidance of sketch labels.

In future work, we plan to extend our approach to learn more CAD-related operations such as \emph{revolve, bevel, and sweep}. %using neural methods. 
Besides, we find that current deep learning models perform poorly on datasets with large differences in shape geometry and structure. %structural and topological variations
Therefore, another promising direction is to explore how to improve the generalization of neural networks and enhance the realism of the generated shapes by learning structural and topological information.


\newpage

\section*{Acknowledgement}
This work was partially supported by the National Key R\&D Program of China (2022YFB4701400/4701402), SZSTC Grant (JCYJ20190809172201639, WDZC2020-0820200655001), Shenzhen Key Laboratory (ZDSYS2021-0623092001004).

%%%%%%%%% REFERENCES
{\small
\bibliographystyle{ieee_fullname}
\bibliography{main}
}

\newpage
\appendix
%
\renewcommand{\thetable}{A\arabic{table}}
\renewcommand{\thefigure}{A\arabic{figure}}
\renewcommand{\theequation}{A\arabic{equation}}
\input{0-appendix}

{\small
\bibliographystylesupp{ieee_fullname}
\bibliographysupp{supp}
}



\end{document}