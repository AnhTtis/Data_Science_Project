\subsection{Connection between Gradient Domination and Connected Superlevel Sets}\label{sec:Connected_GradDom}


% \begin{thm}\label{thm:graddom_nonconnected}
% There exists a convex set $\Xcal$ and a differentiable function $f:\Xcal\rightarrow\mathbb{R}$ such that the set of maximizers
% \[\Xcal^{\star}=\{x^{\star}\in\Xcal:f(x^{\star})\geq f(x),\,\forall x\in\Xcal\}\]
% is disconnected and that all stationary points of $f$ are contained in $\Xcal^{\star}$. 
% In other words, the gradient domination condition does not imply that the superlevel sets are always connected.
% % In other words, the gradient domination condition holds for $f$ on $\Xcal$, but there exist $x_1,x_2\in\Xcal^{\star}$ for which there is no continuous map $p:[0,1]\rightarrow\Xcal^{\star}$ such that $p(0)=x_1$ and $p(1)=x_2$.
% \end{thm}

% The proof of Theorem~\ref{thm:graddom_nonconnected} is simply based on identifying one function that meets the condition. 


% Next, we complement Theorem~\ref{thm:graddom_nonconnected} with the following result, which states that the class of gradient dominated functions is also not subsumed by the class of connected functions. 
% \begin{thm}\label{thm:connected_nongraddom}
% There exists a convex set $\Xcal$ and a differentiable function $f:\Xcal\rightarrow\mathbb{R}$ such that the superlevel set
% \[\Xcal_{\lambda}=\{x\in\Xcal:f(x)\geq \lambda,\,\forall x\in\Xcal\}\]
% is connected for any $\lambda\in\mathbb{R}$ and that there is at least one stationary point of $f$ which is not a global maximizer. In other words, the connectedness of the superlevel sets does not imply the gradient domination condition.
% \end{thm}
We loosely use the term ``gradient domination'' to indicate that a differentiable function does not have any sub-optimal stationary points.
In this section, we use two examples to show that the gradient domination condition in general does not imply or get implied by the connectedness of the superlevel sets. 
The first example is a function that observes the gradient domination condition but has a disconnected set of maximizers (which implies that the superlevel is not always connected).


% \subsection{Proof of Theorem~\ref{thm:graddom_nonconnected}}
% To prove Theorem~\ref{thm:graddom_nonconnected}, we simply need an example of such a function.
Consider $f:[-4,4]\times[-2,0]\rightarrow\mathbb{R}$
\begin{align*}
f(x,y)=\left\{\begin{array}{l}
        f_1(x,y)=-(x-1)^3+3(x-1)-y^2-2y-0.02(y+10)^2(10-x^2), \text { for } x\geq0 \\
        f_2(x,y)=-(-x-1)^3+3(-x-1)-y^2-2y-0.02(y+10)^2(10-x^2), \text { else}
        \end{array}\right.
\end{align*}
It is obvious that the function is symmetric along the line $x=0$ and that $f_1(0,y)=f_2(0,y)$ for all $y\in[-2,0]$. Computing the derivatives of $f_1$ and $f_2$ with respect to $x$, we have
\begin{align*}
    \nabla_x f_1(x,y) &= -3(x-1)^2+3+0.04x(y+10)^2,\\
    \nabla_x f_2(x,y) &= 3(x+1)^2-3+0.04x(y+10)^2.
\end{align*}
We can again verify $\nabla_x f_1(0,y)=\nabla_x f_2(0,y)$ for all $y$, which implies that the function $f$ is everywhere continuous and differentiable. Visualization of $f$ in Fig.~\ref{fig:func_vis} along with simple calculation (solving the system of equations $\nabla_x f(x,y)=0$ and $\nabla_y f(x,y)=0$) show that there are only two stationary points of $f$ on $[-4,4]\times[-2,0]$. The two stationary points are $(3.05,-1.12)$ and $(-3.05,-1.12)$, and they are both global maximizers on this domain, which means that the gradient domination condition is observed. However, the set of maximizers $\{(3.05,-1.12), (-3.05,-1.12)\}$ is clearly disconnected.

% \qed

We next present a function that has connected superlevel sets at all level but does not observe the gradient domination condition (i.e. has sub-optimal stationary points).


% \subsection{Proof of Theorem~\ref{thm:connected_nongraddom}}

% Similar to the proof of Theorem~\ref{thm:graddom_nonconnected}, we just need to find an example function that satisfies the statement of Theorem~\ref{thm:connected_nongraddom}.


\begin{figure}[h]
  \centering
  \includegraphics[width=\linewidth]{Figures/func_vis.png}
  \caption{Visualization of Functions $f$ (Left) and $g$ (Right)}
  \label{fig:func_vis}
\end{figure}


Consider $g:\mathbb{R}^2\rightarrow\mathbb{R}$ defined as
\begin{align*}
    g(x,y)=-(x^2+y^2)^2+4(x^2+y^2).
\end{align*}
This is a volcano-shaped function, which we visualize in Fig.~\ref{fig:func_vis}. It is obvious the superlevel set $\{(x,y):g(x,y)\geq\lambda\}$ is always either a 2D circle (convex set) or a donut-shaped connected set depending on the choice of $\lambda$. However, the gradient domination condition does not hold as $(0,0)$ is a first-order stationary point but not a global maximizer (it is actually a local minimizer).

