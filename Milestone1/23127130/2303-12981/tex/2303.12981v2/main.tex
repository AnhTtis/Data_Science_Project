\documentclass{article}


% if you need to pass options to natbib, use, e.g.:
%     \PassOptionsToPackage{numbers, compress}{natbib}
% before loading neurips_2023


% % ready for submission
% \usepackage{neurips_2023}

% to compile a preprint version, e.g., for submission to arXiv, add add the
% [preprint] option:
%     \usepackage[preprint]{neurips_2023}


% to compile a camera-ready version, add the [final] option, e.g.:
\usepackage[final]{neurips_2023}


% to avoid loading the natbib package, add option nonatbib:
%    \usepackage[nonatbib]{neurips_2023}
% \let\proof\relax
% \let\endproof\relax

\usepackage[utf8]{inputenc} % allow utf-8 input
\usepackage[T1]{fontenc}    % use 8-bit T1 fonts
\usepackage{hyperref}       % hyperlinks
\usepackage{booktabs}       % professional-quality tables
\usepackage{amsfonts,amssymb}       % blackboard math symbols
\usepackage{nicefrac}       % compact symbols for 1/2, etc.
\usepackage{microtype}      % microtypography
\usepackage{bbold}
\usepackage{capt-of}

\usepackage[center]{caption}


\usepackage{cite,epstopdf,color,soul}
% \usepackage{algorithm,algorithmic,tabularx}
% \usepackage{tabularx}
% \usepackage[ruled,vlined]{algorithm2e}
% \allowdisplaybreaks
\usepackage{color}
\usepackage{enumerate}
\usepackage[shortlabels]{enumitem}
\usepackage{multicol}
\usepackage{empheq}
\usepackage{caption} 
% \captionsetup[table]{skip=10pt}



%\usepackage[english,russian]{babel}
\usepackage{hyperref}       % hyperlinks
\usepackage{url}            % simple URL typesetting
\usepackage{booktabs}       % professional-quality tables
\usepackage{amsfonts}       % blackboard math symbols
\usepackage{tablefootnote}
\usepackage{verbatim}
\usepackage{nicefrac}       % compact symbols for 1/2, etc.
\usepackage{microtype}      % microtypography
\usepackage{lipsum}
\usepackage{mathtools}

\usepackage{amsmath,amssymb}
\usepackage{algorithm,algorithmic}
\usepackage{pifont}
\usepackage{cases}
\usepackage{subcaption,graphicx}
\usepackage{stackengine}    % circled symbols
\usepackage{wrapfig}
\usepackage{enumitem}

%\newtheorem{theorem}{Theorem}[section]
%\newtheorem{corollary}[theorem]{Corollary}
%\newtheorem{lemma}[theorem]{Lemma}
\newtheorem{assumption}[theorem]{Assumption}
%\newtheorem{definition}[theorem]{Definition}
%\newtheorem{remark}[theorem]{Remark}
%\newtheorem{proposition}[theorem]{Proposition}

\newcommand*{\LargerCdot}{\raisebox{-0.25ex}{\scalebox{2.4}{$\cdot$}}}
\newcommand{\Sum}{\displaystyle\sum\limits}
\newcommand{\Max}{\max\limits}
\newcommand{\Min}{\min\limits}
\newcommand{\fromto}[3]{{#1}=\overline{{#2},\,{#3}}}
\newcommand{\floor}[1]{\left\lfloor{#1}\right\rfloor}
\newcommand{\ceil}[1]{\left\lceil{#1}\right\rceil}

\newcommand{\tild}{\widetilde}
\newcommand{\eps}{\varepsilon}
\newcommand{\lam}{\lambda}
\newcommand{\ol}{\overline}
\newcommand{\one}{\mathbf{1}}
\newcommand{\cset}{\mathcal{C}}
%\newcommand{\Breg}{\mathcal{D}_{h}}
%\newcommand{\PBreg}{\mathbb{D}_{h}}

%\newcommand{\EndProof}{\begin{flushright}$\square$\end{flushright}}

\newcommand{\circledOne}{\text{\ding{172}}}
\newcommand{\circledTwo}{\text{\ding{173}}}
\newcommand{\circledThree}{\text{\ding{174}}}
\newcommand{\circledFour}{\text{\ding{175}}}
\newcommand{\circledFive}{\text{\ding{176}}}
\newcommand{\circledSix}{\text{\ding{177}}}
\newcommand{\circledSeven}{\text{\ding{178}}}
\newcommand{\circledEight}{\text{\ding{179}}}
\newcommand{\circledNine}{\text{\ding{180}}}
\newcommand{\circledTen}{\text{\ding{181}}}
\newcommand{\balashstar}{\stackMath\mathbin{\stackinset{c}{0ex}{c}{0ex}{\text{\ding{83}}}{\bigcirc}}}
\renewcommand\balashstar{\stackMath\mathbin{\stackinset{c}{0ex}{c}{0ex}{\ast}{\bigcirc}}}


\renewcommand{\le}{\leqslant}
\renewcommand{\ge}{\geqslant}
\renewcommand{\hat}{\widehat}

\newcommand{\numberthis}{\addtocounter{equation}{1}\tag{\theequation}}


\DeclareMathOperator*{\argmin}{arg\,min}
\DeclareMathOperator*{\argmax}{arg\,max}
\DeclareMathOperator*{\Argmin}{Arg\,min}
\DeclareMathOperator*{\Argmax}{Arg\,max}
\DeclareMathOperator{\spn}{span}
\DeclareMathOperator{\kernel}{Ker}
\DeclareMathOperator{\image}{Im}
\DeclareMathOperator{\prox}{prox}
\DeclareMathOperator{\proj}{Proj}
\DeclareMathOperator{\col}{col}
\DeclareMathOperator{\diag}{diag}

\newcommand{\N}{\mathbb{N}}
\newcommand{\R}{\mathbb{R}}
\newcommand{\Z}{\mathbb{Z}}
\newcommand{\V}{\mathbb{V}}
\newcommand{\E}{\mathbb{E}}
%\newcommand{\P}{\mathbb{P}}
\newcommand{\I}{\mathbb{I}}
\newcommand{\F}{\mathbb{F}}

\newcommand{\mA}{{\bf A}}
\newcommand{\mB}{{\bf B}}
\newcommand{\mC}{{\bf C}}
\newcommand{\mD}{{\bf D}}
\newcommand{\mE}{{\bf E}}
\newcommand{\mF}{{\bf F}}
\newcommand{\mG}{{\bf G}}
\newcommand{\mH}{{\bf H}}
\newcommand{\mI}{{\bf I}}
\newcommand{\mJ}{{\bf J}}
\newcommand{\mK}{{\bf K}}
\newcommand{\mL}{{\bf L}}
\newcommand{\mM}{{\bf M}}
\newcommand{\mN}{{\bf N}}
\newcommand{\mO}{{\bf O}}
\newcommand{\mP}{{\bf P}}
\newcommand{\mQ}{{\bf Q}}
\newcommand{\mR}{{\bf R}}
\newcommand{\mS}{{\bf S}}
\newcommand{\mT}{{\bf T}}
\newcommand{\mU}{{\bf U}}
\newcommand{\mV}{{\bf V}}
\newcommand{\mW}{{\bf W}}
\newcommand{\mX}{{\bf X}}
\newcommand{\mY}{{\bf Y}}
\newcommand{\mZ}{{\bf Z}}

\newcommand{\cA}{{\mathcal{A}}}
\newcommand{\cB}{{\mathcal{B}}}
\newcommand{\cC}{{\mathcal{C}}}
\newcommand{\cD}{{\mathcal{D}}}
\newcommand{\cE}{{\mathcal{E}}}
\newcommand{\cF}{{\mathcal{F}}}
\newcommand{\cG}{{\mathcal{G}}}
\newcommand{\cH}{{\mathcal{H}}}
\newcommand{\cI}{{\mathcal{I}}}
\newcommand{\cJ}{{\mathcal{J}}}
\newcommand{\cK}{{\mathcal{K}}}
\newcommand{\cL}{{\mathcal{L}}}
\newcommand{\cM}{{\mathcal{M}}}
\newcommand{\cN}{{\mathcal{N}}}
\newcommand{\cO}{{\mathcal{O}}}
\newcommand{\cP}{{\mathcal{P}}}
\newcommand{\cQ}{{\mathcal{Q}}}
\newcommand{\cR}{{\mathcal{R}}}
\newcommand{\cS}{{\mathcal{S}}}
\newcommand{\cT}{{\mathcal{T}}}
\newcommand{\cU}{{\mathcal{U}}}
\newcommand{\cV}{{\mathcal{V}}}
\newcommand{\cW}{{\mathcal{W}}}
\newcommand{\cX}{{\mathcal{X}}}
\newcommand{\cY}{{\mathcal{Y}}}
\newcommand{\cZ}{{\mathcal{Z}}}

\newcommand{\ba}{{\bf a}}
\newcommand{\bb}{{\bf b}}
\newcommand{\bc}{{\bf c}}
\newcommand{\bd}{{\bf d}}
\newcommand{\be}{{\bf e}}
%\newcommand{\bf}{{\bf f}}
\newcommand{\bg}{{\bf g}}
\newcommand{\bh}{{\bf h}}
\newcommand{\bi}{{\bf i}}
\newcommand{\bj}{{\bf j}}
\newcommand{\bk}{{\bf k}}
\newcommand{\bl}{{\bf l}}
\newcommand{\bm}{{\bf m}}
\newcommand{\bn}{{\bf n}}
\newcommand{\bo}{{\bf o}}
\newcommand{\bp}{{\bf p}}
\newcommand{\bq}{{\bf q}}
\newcommand{\br}{{\bf r}}
\newcommand{\bs}{{\bf s}}
\newcommand{\bt}{{\bf t}}
\newcommand{\bu}{{\bf u}}
\newcommand{\bv}{{\bf v}}
\newcommand{\bw}{{\bf w}}
\newcommand{\bx}{{\bf x}}
\newcommand{\by}{{\bf y}}
\newcommand{\bz}{{\bf z}}

\newcommand{\ds}{\displaystyle}
\newcommand{\norm}[1]{\left\| #1 \right\|}
\newcommand{\normtwo}[1]{\left\| #1 \right\|_2}
\newcommand{\sqn}[1]{\norm{#1}_2^2}
\newcommand{\angles}[1]{\left\langle #1 \right\rangle}
\newcommand{\cbraces}[1]{\left( #1 \right)}
\newcommand{\sbraces}[1]{\left[ #1 \right]}
\newcommand{\braces}[1]{\left\{ #1 \right\}}
\def\<#1,#2>{\langle #1,#2\rangle}

\newcommand{\sigmamax}{\sigma_{\max}(\cA)}
\newcommand{\sigmamaxsqr}{\sigma_{\max}^2(\cA)}
\newcommand{\sigmaminplus}{\sigma_{\min}^+(\cA)}
\newcommand{\sigmaminplussqr}{(\sigma_{\min}^+(\cA))^2}

\usepackage[colorinlistoftodos,bordercolor=blue,backgroundcolor=blue!20,linecolor=blue,textsize=scriptsize]{todonotes}
\newcommand{\arogozin}[1]{\todo[inline]{{\textbf{Alexander R.:} \emph{#1}}}}
\newcommand{\schezhegov}[1]{\todo[inline]{{\textbf{Savelii C.:} \emph{#1}}}}

\input{defs-mlmath.tex}
\newcommand{\qed}{\hfill $\blacksquare$}

\title{Connected Superlevel Set in (Deep) Reinforcement Learning and its Application to Minimax Theorems}
\usepackage{times}

\author{
  Sihan Zeng \\
  Dept. of Electrical and Computer Engineering\\
  Georgia Institute of Technology\\
  Atlanta, GA 30318 \\
  \texttt{szeng30@gatech.edu} \\
  \And
  Thinh T. Doan \\
  Dept. of Electrical and Computer Engineering \\
  Virginia Tech \\
  Blacksburg, VA 24061 \\
  \texttt{thinhdoan@vt.edu} \\
  \And
  Justin Romberg \\
  Dept. of Electrical and Computer Engineering \\
  Georgia Institute of Technology \\
  Atlanta, GA 30318 \\
  \texttt{jrom@ece.gatech.edu} \\
}


% This work was supported in part by the NSF AI Institute AI4OPT, NSF 2112533.



\begin{document}

\maketitle
\begin{abstract}
% The policy optimization problem in reinforcement learning can be formulated as a non-convex optimization program with a recently discovered ``gradient domination’’ condition, which guarantees that every stationary point of the objective function is globally optimal. Apart from this condition, our knowledge on the optimization landscape is still limited. The aim of this paper is to provide more insight on the structure of the policy optimization problem. Specifically, we show that the superlevel set of the objective function with respect to the policy parameter is always a connected set both in the tabular setting and under policies represented by a class of neural networks. To our best knowledge, this is a novel discovery in the literature.

The aim of this paper is to improve the understanding of the optimization landscape for policy optimization problems in reinforcement learning. Specifically, we show that the superlevel set of the objective function with respect to the policy parameter is always a connected set both in the tabular setting and under policies represented by a class of neural networks.
In addition, we show that the optimization objective as a function of the policy parameter and reward satisfies a stronger ``equiconnectedness'' property.
To our best knowledge, these are novel and previously unknown discoveries.


We present an application of the connectedness of these superlevel sets to the derivation of minimax theorems for robust reinforcement learning. We show that any minimax optimization program which is convex on one side and is equiconnected on the other side observes the minimax equality (i.e. has a Nash equilibrium). We find that this exact structure is exhibited by an interesting class of robust reinforcement learning problems under an adversarial reward attack, and the validity of its minimax equality immediately follows. This is the first time such a result is established in the literature.
\end{abstract}




% Importance and appeal of children's drawings
Children's depictions of the human figure are highly expressive and varied.
As one of the very first subjects children attempt to draw, the representation begins as an almost unintelligible cloud of scribbles. 
As the child grows, their representation of the human figure becomes more developed and is extended to graphically represent many different types of characters: people, animals, and even personified objects (see Figure 1).

Who among us has not wished, either as a child or as an adult, to see such figures come to life and move around on the page?
Sadly, while it is relatively fast to produce a single drawing, creating the sequence of images necessary for animation is a much more tedious endeavor, requiring discipline, skill, patience, and sometimes complicated software.
As a result, most of these figures remain static upon the page.

% We built a system to animate them.
Inspired by the importance and appeal of the drawn human figure, we design and build a system to automatically animate it given an in-the-wild photograph of a child's drawing. 
Our system is fast, intuitive, and robust to much of the variation present in these types of drawings, making it well-suited to allow our target audience--children--to see their own characters coming to life.
The system is comprised of four stages: figure detection, segmentation masking, pose estimation/rigging, and animation. 
We describe each stage and identify common causes of failure in each. 
For object detection and pose estimation, we make use of existing computer vision models designed to detect human figures and joints in photographs; we fine-tune these models for use with children's drawings.
For segmentation, we present a straightforward, image processing-based method that, for animation purposes, is more useful and accurate than segmentation masks obtained from a fine-tuned object detection model.
During the animation step, we take advantage of the \textit{twisted perspective} commonly seen in children’s drawings to retarget motion capture data onto the character in a novel and appealing way.

% We use existing machine learning models. However, given the wide domain gap it's not clear how much fine-tuning data was needed. So we ran some experiments to find out and report it.
While our system leverages existing models and techniques, most are not directly applicable to the task due to the many differences between photographic images and simple pen and paper representations. 
To this end, we couple the presentation of our system with a set of experiments exploring the relationship between fine-tuning training set size and success rates.
We also include a perceptual study validating viewer preference for incorporating \textit{twisted perspective} into the motion retargeting step.

We validate the desirability and appeal of our system by building and publicly releasing a version of it as the \AD Demo \,\cite{animateddrawings}.
Launched in December 2021, this demo has been used by millions of people around the world to animate their children's drawings.
Inspired by this reception, our second contribution is The Amateur Drawings Dataset: \hjs{180,000 drawings and user-accepted annotations collected, with consent, through the demo. See Section \ref{sec:UI} for a description of how the annotations were generated.}
We believe this dataset will be a resource to researchers from various fields seeking to better understand the space of amateur drawings, evaluate new algorithms in this domain, or develop new drawing-based tools in general.

To summarize, our contributions are as follows:
\begin{enumerate}
    \item 
    We explore the problem of automatic sketch-to-animation for children's drawings of human figures and present a framework that achieves this effect. We also present a set of experiments determining the amount of training data necessary to achieve high levels of success and a perceptual study validating the usefulness of our motion retargeting technique.
    \item To encourage additional research in the domain of amateur drawings, we present a first-of-its-kind dataset of 180,000 user-submitted amateur drawings, along with user-accepted bounding box, segmentation mask, and joint location annotations.
\end{enumerate}

Upon acceptance of this paper, we plan to publicly release the Amateur Drawings Dataset, project code, and fine-tuned model weights.

\section{Connected Superlevel Set Under Tabular Policy}\label{sec:tabular}

We consider the infinite-horizon average-reward MDP characterized by $\Mcal=(\Scal,\Acal,\Pcal,r)$. We use $\Scal$ and $\Acal$ to denote the state and action spaces, which we assume are finite. The transition probability kernel is denoted by $\Pcal:\Scal\times\Acal\rightarrow \Delta_{\Scal}$, where $\Delta_{\Scal}$ denotes the probability simplex over $\Scal$. The reward function $r:\Scal\times\Acal\rightarrow[0,U_r]$ is bounded for some positive constant $U_r$ and can also be regarded as a vector in $\mathbb{R}^{|\Scal|\times|\Acal|}$. We use $P^{\pi}\in\mathbb{R}^{\Scal\times\Scal}$ to represent the state transition probability matrix under policy $\pi\in\Delta_{\Acal}^{\Scal}$, where $\Delta_{\Acal}^{\Scal}$ is the collection of probability simplexes over $\Acal$ across the state space
\begin{align}
    P^{\pi}_{s',s}=\sum_{a\in\Acal}\Pcal(s'\mid s,a)\pi(a\mid s),\quad\forall s',s\in\Scal.\label{eq:transition_matrix}
\end{align}
% $\mu_{\pi}\in\Delta_{\Scal}$ denotes the stationary distribution of the states induced by policy $\pi$. It is well-known that $\mu_{\pi}$ is an eigenvector of $P^{\pi}$ with the associated eigenvalue equal to $1$, i.e. $\mu_{\pi}=P^{\pi}\mu_{\pi}$. 

We consider the following ergodicity assumption in the rest of the paper, which is commonly made in the RL literature \citep{wang2017primal,wei2020model,wu2020finite}.
\begin{assump}\label{assump:ergodicity}
Given any policy $\pi$, the Markov chain formed under the transition probability matrix $P^{\pi}$ is ergodic, i.e. irreducible and aperiodic.
\end{assump}
Let $\mu_{\pi}\in\Delta_{\Scal}$ denote the stationary distribution of the states induced by policy $\pi$. As a consequence of Assumption~\ref{assump:ergodicity}, the stationary distribution $\mu_{\pi}$ is unique and uniformly bounded away from $0$ under any $\pi$. In addition, $\mu_{\pi}$ is the unique eigenvector of $P^{\pi}$ with the associated eigenvalue equal to $1$, i.e. $\mu_{\pi}=P^{\pi}\mu_{\pi}$. 
Let $\widehat{\mu}_{\pi}\in\Delta_{\Scal\times\Acal}$ denote the state-action stationary distribution induced by $\pi$, which can be expressed as
\begin{align}
    \widehat{\mu}_{\pi}(s,a)=\mu_{\pi}(s)\pi(a\mid s).\label{eq:mu_hat}
\end{align}

We measure the performance of a policy $\pi$ under reward function $r$ by the average cumulative reward $J_r(\pi)$
\begin{align*}
    J_r(\pi)\triangleq\lim_{K \rightarrow \infty} \frac{\sum_{k=0}^{K} r(s_k, a_k)}{K}=\mathbb{E}_{s\sim\mu_{\pi}, a\sim \pi}[r(s_k,a_k)]=\sum_{s,a}r(s,a)\widehat{\mu}_{\pi}(s,a).
\end{align*}

The objective of the policy optimization problem is to find the policy $\pi$ that maximizes the average cumulative reward
\begin{align}
    \max_{\pi\in\Delta_{\Acal}^{\Scal}}J_r(\pi).\label{eq:obj}
\end{align}

The superlevel set of $J_r$ is the set of policies that achieve a value function greater than or equal to a specified level. Formally, given $\lambda\in\mathbb{R}$, the $\lambda$-superlevel set (or superlevel set) under reward $r$ is defined as
\[
    \Ucal_{\lambda,r}\triangleq\{\pi\in\Delta_{\Acal}^{\Scal}\mid J_r(\pi)\geq \lambda\}.
\]

The main focus of this section is to study the connectedness of this set $\Ucal_{\lambda,r}$, which requires us to formally define a connected set.
\begin{definition}\label{def:connectedset}
A set $\Ucal$ is connected if for any $x,y\in\Ucal$ there exists a continuous map $p:[0,1]\rightarrow\Ucal$ such that $p(0)=x$ and $p(1)=y$.
\end{definition}
We say that a function is connected if its superlevel sets are connected at all levels. We also introduce the definition of equiconnected functions. 
\begin{definition}\label{def:equiconnectedfunc}
Given two spaces $\Xcal$ and $\Ycal$, the collection of functions $\{f_y:\Xcal\rightarrow\mathbb{R}\}_{y\in\Ycal}$ is said to be equiconnected if for every $x_1,x_2\in\Xcal$, there exists a continuous path map $p:[0,1]\rightarrow\Xcal$ such that
\begin{align*}
    p(0)=x_1,\quad p(1)=x_2,\quad f_y(p(\alpha))\geq\min\{f_y(x_1),f_y(x_2)\},
\end{align*}
for all $\alpha\in[0,1]$ and $y\in\Ycal$.
\end{definition}
Conceptually, the collection of functions $\{f_y:\Xcal\rightarrow\mathbb{R}\}_{y\in\Ycal}$ being equiconnected requires 1) that $f_y(\cdot)$ is a connected function for all $y\in\Ycal$ (or equivalently, the set $\{x\in\Xcal:f_y(x)\geq\lambda\}$ is connected for all $\lambda\in\mathbb{R}$ and $y\in\Ycal$) and 2) that the path map constructed to prove the connectedness of $\{x\in\Xcal:f_y(x)\geq\lambda\}$ is independent of $y$.


We now present our first main result of the paper, which states that the superlevel set $\Ucal_{\lambda,r}$ is always connected.
\begin{thm}\label{thm:connected_tabular}
    Under Assumption \ref{assump:ergodicity}, the superlevel set $\Ucal_{\lambda,r}$ is connected for any $\lambda\in\mathbb{R}$ and $r\in\mathbb{R}^{|\Scal||\Acal|}$.
    In addition, the collection of functions $\{J_r(\cdot):\Delta_{\Acal}^{\Scal}\rightarrow\mathbb{R}\}_{r\in\mathbb{R}^{|\Scal|\times|\Acal|}}$ is equiconnected.
\end{thm}


Our result here extends easily to the infinite-horizon discounted-reward setting since a discounted-reward MDP can be regarded as an average-reward one with a slightly modified transition kernel \citep{kondathesis}.


The claim in Theorem~\ref{thm:connected_tabular} on the equiconnectedness of $\{J_r\}_{r\in\mathbb{R}^{|\Scal|\times|\Acal|}}$ is a slightly stronger result than the connectedness of $\Ucal_{\lambda,r}$, and plays an important role in the application to minimax theorems discussed later in Section~\ref{sec:application}.


We note that the proof, presented in Section~\ref{sec:thm:connected_tabular} of the appendix, mainly leverages the fact that the value function $J_r(\pi)$ is linear in the state-action stationary distribution $\widehat{\mu}_{\pi}$ and that there is a special connection (though nonlinear and nonconvex) between $\widehat{\mu}_{\pi}$ and the policy $\pi$, which we take advantage of to construct the continuous path map for the analysis. Specifically, given two policies $\pi_1,\pi_2$ with $J_r(\pi_1),J_r(\pi_2)\geq \lambda$, we show that the policy $\pi_{\alpha}$ defined as
\begin{align*}
    \pi_{\alpha}(a\mid s) = \frac{\alpha \mu_{\pi_1}(s)\pi_1(a\mid s)+(1-\alpha)\mu_{\pi_2}(s)\pi_2(a\mid s)}{\alpha \mu_{\pi_1}(s)+(1-\alpha)\mu_{\pi_2}(s)},\quad\forall \alpha\in[0,1]
\end{align*}
is guaranteed to achieve $J_r(\pi_{\alpha})\geq\lambda$ for all $\alpha\in[0,1]$. 

Besides playing a key role in the proof of Theorem~\ref{thm:connected_tabular}, our construction of this path map may inform the design of algorithms in the future.
Given any two policies with a certain guaranteed performance, we can generate a continuum of policies at least as good. As a consequence, if we find two optimal policies (possibly by gradient descent from different initializations) we can generate a range of interpolating optimal policies. If the agent has a preference over these policy (for example, to minimize certain energy like in $H_1$ control, or if some policies are easier to implement physically), then the selection can be made on the continuum of optimal policies, which eventually leads to a more preferred policy.

\section{Connected Superlevel Set Under Neural Network Parameterized Policy}\label{sec:NN}

In real-world reinforcement learning applications, it is common to use a deep neural network to parameterize the policy \citep{silver2016mastering,arulkumaran2017deep}. 
% In general, neural networks break the gradient domination structure of the underlying optimization problem, though under specific assumptions (such as over-parameterization, sufficient width of the layers, and/or small number of the layers) the gradient domination condition may be recovered \citep{liu2019neural,wangneural,cayci2022finite}. 
In this section, we consider the policy optimization problem under a special class of policies represented by an over-parameterized neural network and show that this problem still enjoys the important structure --- the connectedness of the superlevel sets --- despite the presence of the highly complex function approximation. Illustrated in Fig.~\ref{fig:NN}, the neural network parameterizes the policy in a very natural manner which matches how neural networks are actually used in practice.


\begin{figure}[h]
  \centering
  \includegraphics[width=.95\linewidth]{Figures/NN.png}
  \caption{Neural Network Policy Representation}
  \label{fig:NN}
\end{figure}

Mathematically, the parameterization can be described as follows. Each state $s\in\Scal$ is associated with a feature vector $\phi(s)\in\mathbb{R}^d$, which in practice is usually carefully selected to summarize the key information of the state. 
For state identifiability, we assume that the feature vector of each state is unique, i.e.
\begin{align*}
    \phi(s)\neq\phi(s'),\quad \forall s,s'\in\Scal\text{ and }s\neq s'.
\end{align*}

To map a feature vector $\phi(s)$ to a policy distribution over state $s$, we employ a $L$-layer neural network, which in the $k_{\text{th}}$ layer has weight matrix $W_k\in\mathbb{R}^{n_{k-1}\times n_k}$ and bias vector $b_k\in\mathbb{R}^{n_k}$ with $n_0=d$ and $n_L=|\Acal|$. For the simplicity of notation, we use $\Omega_k$ to denote the space of weight and bias parameters $(W_k,b_k)$ of layer $k$, and we write $\Omega=\Omega_1\times\cdots\times\Omega_L$.
We use $\theta$ to denote the collection of the weights and biases
\[\theta=((W_1,b_1),\cdots,(W_L,b_L))\in\Omega\]
We use the same activation function for layers $1$ through $L-1$, denoted by $\sigma:\mathbb{R}\rightarrow\mathbb{R}$, applied in an element-wise fashion to vectors. 
To ensure that the output of the neural network is a valid probability distribution, the activation function for the last layer is a softmax function, denoted by $\psi:\mathbb{R}^{|\Acal|}\rightarrow\Delta_{\Acal}$, i.e. for any vector $v\in\mathbb{R}^{|\Acal|}$
\begin{align*}
    \psi(v)_{i}=\frac{\exp(v_i)}{\sum_{i'=1}^{|\Acal|}\exp(v_{i'})}, \quad \forall i=1,...,|\Acal|.
\end{align*}
With $v\in\mathbb{R}^{d}$ as the input to a neural network with parameters $\theta$, we use $f_k^{\theta}(v)\in\mathbb{R}^{n_k}$ to denote the output of the network at layer $k$. For $k=1,\cdots,L$, $f_k^{\theta}(v)$ is computed as
\begin{align}
    f_k^{\theta}(v) = 
    \left\{\begin{array}{ll}
        \sigma\left(W_1^{\top}v+b_{1}\right) & k=1 \\
        \sigma\left(W_k^{\top}f_{k-1}(v)+ b_{k}\right) & k = 2, 3,..., L-1 \\
        \psi\left(W_{L}^{\top}f_{L-1}(v)+b_{L}\right) & k=L.
    \end{array}\right.\label{eq:def_f_k}
\end{align}
The policy $\pi_{\theta}\in\mathbb{R}^{|\Scal|\times|\Acal|}$ parametrized by $\theta$ is the output of the final layer:
%We define $\pi_{\theta}\in\mathbb{R}^{|\Scal|\times|\Acal|}$ such that
\[\pi_{\theta}(\cdot\mid s)=f_L^{\theta}(\phi(s))\in\Delta_{\Acal},\quad\forall s\in\Scal.\]


Our analysis relies two assumptions about the structure of the neural network.  The first concerns the invertibility of $\sigma(\cdot)$ as well as the continuity and uniqueness of its inverse, which can be guaranteed by the following:
\begin{assump}\label{assump:sigma}
$\sigma$ is strictly monotonic and $\sigma(\mathbb{R})=\mathbb{R}$. In addition, there do not exist non-zero scalars $\{p_i,q_i\}_{i=1}^{m}$ with $q_i\neq q_j,\,\forall i\neq j$ such that for some $m>0$, $\sigma(x)=\sum_{i=1}^{m}p_i\sigma(x-q_i),\,\forall x\in\mathbb{R}$.
\end{assump}
We note that this assumption holds for common activation functions including leaky-ReLU and parametric ReLU \citep{xu2015empirical}.


Our second assumption is that the neural network is sufficiently over-parameterized and that the number of parameters decreases with each layer.  
\begin{assump}\label{assump:network_dimension}
The output of the first layer is wider than $2|\Scal|$, and the width of the network decreases over the layers, i.e.
\begin{align*}
    n_1\geq 2|\Scal|, \text{ and } n_1>n_2>...>n_L=|\mathcal{A}|.
\end{align*}
\end{assump}
%We next consider an assumption on the widths of the layers which ensures the neural network to be sufficiently over-parameterized. 
Neural networks meeting this criteria have a number of weight parameters that is larger than the cardinality of the state space, making them impractical for large $|\mathcal{S}|$.
%which seems to defeat the purpose of using a function approximation in the first place. 
%
While ongoing work seeks to relax or remove this assumption, we point out that similar over-parameterization assumptions are critical and very common in most existing works on the theory of neural networks \citep{zou2019improved,nguyen2019connected,liu2022loss,martinetz2022highly,pandey2023exploring}.


% This assumption is also made in \citet{nguyen2019connected} which studies the connectedness of sublevel sets for deep (convex) supervised learning.





The $\lambda$-superlevel set of the value function with respect to $\theta$ under reward function $r$ is
\begin{align*}
    \Ucal_{\lambda,r}^{\Omega}\triangleq\{\theta\in\Omega\mid J_r(\pi_{\theta})\geq \lambda\}.
\end{align*}

Our next main theoretical result guarantees the connectedness of $\Ucal_{\lambda,r}^{\Omega}$.
\begin{thm}\label{thm:connected_firstlayer2S}
Under Assumptions \ref{assump:ergodicity}-\ref{assump:network_dimension}, the superlevel set $\Ucal_{\lambda,r}^{\Omega}$ is connected for any $\lambda\in\mathbb{R}$. 
In addition, with $J_{r,\Omega}(\theta)\triangleq J_r(\pi_{\theta})$, the collection of functions $\{J_{r,\Omega}(\cdot):\Omega\rightarrow\mathbb{R}\}_{r\in\mathbb{R}^{|\Scal|\times|\Acal|}}$ is equiconnected.
\end{thm}

The proof of this theorem is deferred to the appendix. Similar to Theorem~\ref{thm:connected_tabular}, the claim in Theorem~\ref{thm:connected_firstlayer2S} on the equiconnectedness of $\{J_{r,\Omega}\}_{r\in\mathbb{R}^{|\Scal|\times|\Acal|}}$ is again stronger than the connectedness of $\Ucal_{\lambda,r}^{\Omega}$ and needs to be derived for the application to minimax theorems, which we discuss in the next section.


\section{A Minimax Theorem for Robust Reinforcement Learning}\label{sec:application}



In this section, we discuss the reward poisoning attack considered in \citet{banihashem2021defense}, which can be formulated as a convex-nonconcave minimax optimization program. We show that the minimax equality holds in this optimization problem in the tabular policy setting and under policies represented by a class of neural networks, as a consequence of our results in Sections~\ref{sec:tabular} and \ref{sec:NN}. To our best knowledge, the existence of the Nash equilibrium for this robust RL problem has not been established before even in the tabular case.


We again consider the infinite horizon, average reward MDP $\Mcal=(\Scal,\Acal,\Pcal,r)$ introduced in Section~\ref{sec:tabular}, where $r$ is the true, unpoisoned reward function. 
% We use $\pi^{\star}$ to denote an optimal policy, which is a (not necessarily unique) solution to \eqref{eq:obj}. 
% To reflect the dependency of the value function on the reward, we denote
% \begin{align*}
%     J_r(\pi)&=\lim_{K \rightarrow \infty} \frac{\sum_{k=0}^{K} r(s_k, a_k)}{K}=\mathbb{E}_{s\sim\mu_{\pi}, a\sim \pi}[r(s_k,a_k)]=r^{\top}\widehat{\mu}_{\pi}.
%     % \pi_r^{\star}&=\argmax_{\pi\in\Delta(\Acal)^{\Scal}}J_r(\pi).
% \end{align*}
Let $\Pi^{\text{det}}$ denote the set of deterministic policies from $\Scal$ to $\Acal$. With the perfect knowledge of this MDP, an attacker has a target policy $\pi_{\dagger}\in\Pi^{\text{det}}$ and tries to make the learning agent adopt the policy by manipulating the reward function. 
Mathematically, the goal of the attacker can be described by the function $\operatorname{Attack}(r,\pi_{\dagger},\epsilon_{\dagger})$ which returns a poisoned reward under the true reward $r$, the target policy $\pi_{\dagger}$, and a pre-selected margin parameter $\epsilon_{\dagger}\geq0$. $\operatorname{Attack}(r,\pi_{\dagger},\epsilon_{\dagger})$ is the solution to the following optimization problem
\begin{align}
    \begin{aligned}
    \operatorname{Attack}(r,\pi_{\dagger},\epsilon_{\dagger})\quad=\quad\argmin_{r'}\quad&\sum_{s\in\Scal,a\in\Acal}\left(r'(s,a)-r(s,a)\right)^2\\
    \operatorname{s.t.}\quad& J_{r'}(\pi_{\dagger})\geq J_{r'}(\pi)+\epsilon_{\dagger},\quad\forall \pi\in\Pi^{\text{det}}\backslash\pi_{\dagger}.
    \end{aligned}\label{eq:attack_obj}
\end{align}
In other words, the attacker needs to minimally modify the reward function to make $\pi_{\dagger}$ the optimal policy under the poisoned reward. This optimization program minimizes a quadratic loss under a finite number of linear constraints and is obviously convex.

The learning agent observes the poisoned reward $r_{\dagger}=\operatorname{Attack}(r,\pi_{\dagger},\epsilon_{\dagger})$ rather than the original reward $r$. As noted in \citet{banihashem2021defense}, without any defense, the learning agent solves the policy optimization problem under $r_{\dagger}$ to find $\pi_{\dagger}$, which may perform arbitrarily badly under the original reward. One way to defend against the attack is to maximize the performance of the agent in the worst possible case of the original reward, which leads to solving a minimax optimization program of the form
\begin{align}
    \max_{\pi\in\Delta_{\Acal}^{\Scal}}\min_{r'} J_{r'}(\pi)\quad\operatorname{s.t.}\,\, \operatorname{Attack}(r',\pi_{\dagger},\epsilon_{\dagger})=r_{\dagger}.\label{eq:robustrl_obj}
\end{align}
% When we fix the reward $r'$, \eqref{eq:robustrl_obj} reduces to a standard policy optimization problem, which we have shown is non-convex but has connected superlevel sets. On the other hand, when the policy $\pi$ is fixed, \eqref{eq:robustrl_obj} reduces to 
% \begin{align}
%     \min_{r'} J_{r'}(\pi)\quad\operatorname{s.t.}\,\, \operatorname{Attack}(r',\pi_{\dagger},\epsilon_{\dagger})=r_{\dagger},\label{eq:robustrl_convexside}
% \end{align}
% which has a linear objective function and a convex (and compact) constraint set and is therefore a convex program\footnote{We justify this claim in Section~\ref{sec:robustrl_convexside_proof} of the appendix.}.
When the policy $\pi$ is fixed, \eqref{eq:robustrl_obj} reduces to 
\begin{align}
    \min_{r'} J_{r'}(\pi)\quad\operatorname{s.t.}\,\, \operatorname{Attack}(r',\pi_{\dagger},\epsilon_{\dagger})=r_{\dagger}.\label{eq:robustrl_convexside}
\end{align}
With the justification deferred to Appendix~\ref{sec:robustrl_convexside_proof}, we point out that \eqref{eq:robustrl_convexside} consists of a linear objective function and a convex (and compact) constraint set, and is therefore a convex program. On the other hand, when we fix the reward $r'$, \eqref{eq:robustrl_obj} reduces to a standard policy optimization problem.

We are interested in investigating whether the following minimax equality holds.
\begin{align}
    \max_{\pi\in\Delta_{\Acal}^{\Scal}}\min_{r':\operatorname{Attack}(r',\pi_{\dagger},\epsilon_{\dagger})=r_{\dagger}} J_{r'}(\pi) = \min_{r':\operatorname{Attack}(r',\pi_{\dagger},\epsilon_{\dagger})=r_{\dagger}} \max_{\pi\in\Delta_{\Acal}^{\Scal}}J_{r'}(\pi).
    \label{eq:robustrl_minimax}
\end{align}
This is a fundamental question to ask in minimax optimization, as it is is an important characterization of the optimization landscape. The minimax equality implies the existence of a Nash equilibrium. At the Nash equilibrium solution pair, neither player can achieve a better function value by changing its strategy, which provides a strong notion of equilibrium and global optimality.
The knowledge of the existence of the Nash equilibrium may be useful for designing and analyzing algorithms for solving the problem \citep{kim2008minimax,ricceri2008recent}. 

% It is known that the minimax inequality always holds
% \begin{align*}
%     \max_{\pi}\min_{r':\operatorname{Attack}(r',\pi_{\dagger},\epsilon_{\dagger})=r_{\dagger}} J_{r'}(\pi) \leq \min_{r':\operatorname{Attack}(r',\pi_{\dagger},\epsilon_{\dagger})=r_{\dagger}} \max_{\pi}J_{r'}(\pi).
% \end{align*}
% However, as the program is not convex-concave, it is unclear whether this condition holds as an equality.

% In the rest of the section, we show that the connectedness of the superlevel sets in reinforcement learning implies the minimax equality
% \begin{align}
%     \max_{\pi}\min_{r':\operatorname{Attack}(r',\pi_{\dagger},\epsilon_{\dagger})=r_{\dagger}} J_{r'}(\pi) = \min_{r':\operatorname{Attack}(r',\pi_{\dagger},\epsilon_{\dagger})=r_{\dagger}} \max_{\pi}J_{r'}(\pi).
%     \label{eq:robustrl_minimax}
% \end{align}
% Equivalently, this equality means that there exists a Nash equilibrium solution pair $(\pi^{\star},r^{\star})$.


It is well-known that the minimax equality 
\begin{align}
    \min_{y \in \Ycal} \max_{x \in \Xcal} f(x, y)=\max_{x \in \Xcal} \min_{y \in \Ycal} f(x, y)
\end{align}
holds for function $f:\Xcal\times \Ycal\rightarrow\mathbb{R}$ if 1) $\Xcal,\Ycal$ are finite-dimensional simplexes \citep{neumann1928theorie}, or 2) $f$ is continuous quasiconvex-quasiconcave and $\Xcal,\Ycal$ are convex compact sets \citep{sion1958general,kindler2005simple}. However, the function $J_r(\pi)$ does not fall under either category. This makes the validity of equation \eqref{eq:robustrl_minimax} unclear from the existing literature.




% In the rest of the section, we prove that the connectedness of the superlevel sets in reinforcement learning implies a minimax theorem for \eqref{eq:robustrl_obj}
% \begin{align}
%     \sup_{\theta\in\Omega}\min_{r\in \Rcal}\mathbb{E}[J_r(\pi_{\theta})] = \min_{r\in \Rcal}\sup_{\theta\in\Omega}\mathbb{E}[J_r(\pi_{\theta})].
%     \label{eq:robustrl_minimax}
% \end{align}
% The reason of using supremum rather than the maximum is that the optimal policy may be deterministic. Under a softmax policy parameterization, a deterministic policy can only be achieved in the limit by sending certain parameters to infinity.

In the rest of this section, we establish the equality \eqref{eq:robustrl_minimax} and show that it is a simple consequence of the connectedness of the superlevel sets in reinforcement learning and a minimax theorem adapted from \citet{simons1995minimax} on a special class of convex-nonconcave functions. We now state this minimax theorem and note that this is essentially a simplified version of \citet{simons1995minimax}[Theorem 4].
% specialized to the Euclidean space.


\begin{thm}\label{thm:minimax_simplified}
Consider a separately continuous function $f:\Xcal\times \Ycal\rightarrow\mathbb{R}$, with $\Ycal$ being a convex, compact set. 
Suppose that $f(x,\cdot)$ is convex for all $x\in\Xcal$. Also suppose that the collection of functions $\{f(\cdot,y)\}_{y\in\Ycal}$ is equiconnected. Then, we have
\begin{align}
    \sup_{x \in \Xcal} \min_{y\in \Ycal} f(x,y)=\min_{y\in \Ycal} \sup_{x \in \Xcal} f(x,y).
\end{align}
\end{thm}
Theorem~\ref{thm:minimax_simplified} states that the minimax equality holds under two main conditions (other than the continuity condition, which can easily be verified to hold for $J_{r}(\pi)$). First, the function $f(x,y)$ needs to be convex with respect to the variable $y$ within a convex, compact constraint set. Second, $f(x,y)$ needs to have a connected superlevel set with respect to $x$, and the path function constructed to prove the connectedness of the superlevel set is independent of $y$. As we have shown in this section and earlier in the paper, if we model $J_r(\pi)$ by $f(x,y)$ with $\pi$ and $r$ corresponding to $x$ and $y$, both conditions are observed in the optimization problem \eqref{eq:robustrl_obj}, which allows us to state the following corollary.
\begin{cor}\label{cor:minimax_robustrl_tabular}
Suppose that the Markov chain $\Mcal$ satisfies Assumption~\ref{assump:ergodicity} on ergodicity. Then, the minimax equality \eqref{eq:robustrl_minimax} holds.
\end{cor}

When the neural network presented in Section~\ref{sec:NN} is used to represent the policy, the collection of functions $\{J_{r,\Omega}\}_{r}$ is also equiconnected. This allows us to extend the minimax equality above to the neural network policy class. Specifically, consider the poisoned reward defense problem \eqref{eq:robustrl_obj} where the policy $\pi_{\theta}$ is represented by the parameter $\theta\in\Omega$ as described in Section~\ref{sec:NN}. 
Using $f(x,y)$ to model $J_r(\pi_{\theta})$ with $x$ and $y$ mirroring $\theta$ and $r$, we can easily establish the minimax theorem in this case as a consequence of Theorem~\ref{thm:connected_firstlayer2S} and \ref{thm:minimax_simplified}.

\begin{cor}\label{cor:minimax_robustrl_NN}
Suppose that the Markov chain $\Mcal$ satisfies Assumption \ref{assump:ergodicity} on ergodicity and that the neural policy class satisfies Assumptions~\ref{assump:sigma}-\ref{assump:network_dimension}. Then, we have
\begin{align}
    \sup_{\theta\in\Omega}\min_{r':\operatorname{Attack}(r',\pi_{\dagger},\epsilon_{\dagger})=r_{\dagger}} J_{r'}(\pi_{\theta}) = \min_{r':\operatorname{Attack}(r',\pi_{\dagger},\epsilon_{\dagger})=r_{\dagger}} \sup_{\theta\in\Omega}J_{r'}(\pi_{\theta}).
\end{align}
\end{cor}

Corollary~\ref{cor:minimax_robustrl_tabular} and \ref{cor:minimax_robustrl_NN} establish the minimax equality (or equivalently, the existence of the Nash equilibrium) for the robust reinforcement learning problem under adversarial reward attack for the tabular and neural network policy class, respectively. To our best knowledge, these results are both novel and previously unknown in the existing literature. The Nash equilibrium is an important global optimality notion in minimax optimization, and the knowledge on its existence can provide strong guidance on designing and analyzing algorithms for solving the problem.




%\section{}
%\label{sec:resDir}


\section{Conclusion}
\label{sec:conclusion}
% <>
Since its advent in 1931, Koopman operator theory \cite{koopman:1931} has only recently been actively utilized for solving practical problems, thanks to the introduction of the DMD algorithm in 2008 \cite{schmid:2008}. Since then, a multitude of DMD algorithm variations have risen to prominence and found utility across various fields. A notable feature of our survey paper was reviewing and categorizing the results of over 100 research papers based on both application and algorithm type in smart mobility and vehicle engineering  (see Table~\ref{tab1} and Section~\ref{sec:vehicApp}).  Additionally, this survey paper identified potential research gaps in smart mobility and vehicular engineering applications (Remarks~\ref{remGap1}--\ref{remGap6}). Finally, this review paper discussed theoretical aspects of Koopman operator theory that have been largely neglected by the smart mobility and vehicle engineering community and yet have large potential for contributing to solving open problems in these areas (see Section~\ref{subsec:theorIssue}).

\noindent{\textbf{Future Research Directions.}}	Given the emergence of cyber-threats against connected and autonomous vehicles as well as robotic systems (see, e.g.,~\cite{nekouei2021randomized,mohammadi2022generation}), a future research direction might include utilizing Koopman operator-based algorithms for designing cyber-resilient vehicular and smart mobility applications (see, e.g.,~\cite{taheri2022data} for a related line of research). Another potential research direction is using Koopman operator-based algorithms for predicting the motion of vulnerable road users (VRUs), e.g., pedestrians and cyclists (see, e.g.,~\cite{pool2019context,scholler2020constant}). Finally, rehabilitation robotics and robotic exoskeletons can be the benefactors of the predictive capabilities of Koopman operator-based algorithms for detecting tripping events and/or system  identification in various modes of locomotion (see, e.g.,~\cite{kumar2019extremum,aprigliano2019pre}).



%Fig. 1 depicts the accumulation of such algorithms since 2014, which are particular to vehicle engineering and smart mobility, i.e., the focus of this review. Table 1 summarizes the varieties of relevant algorithms developed in those studies. Furthermore, we have highlighted theoretical issues, whose expansion will have potential applications to the wide research area of smart mobility and vehicle engineering.  

%Although fairly comprehensive, we have found several gaps in this research area. In particular, we could not find any studies related to elevators, robots/vehicles employing crawling, slithering, hopping or peristaltic locomotion, arctic or special-terrain vehicles such as those employing screws or tracks, hovercraft and other amphibious vehicles or subsystems which tolerate flexible environments, classification or guidance systems related to vehicles for drilling or agriculture, or for current-ripple, power-split, battery health monitoring, nuclear propulsion, exoskeletons/prosthetics, personal mobility, motorsports, specialized rovers or similar open problems in emerging areas.  These examples are, of course, not exhaustive.  
%
%The purely data-driven nature of Koopman operators holds the promise of capturing unknown and complex dynamics for reduced-order model generation and system identification, through which the rich machinery of linear control techniques can be utilized. The emergent nature of the smart mobility and vehicular-related applications, where  the Koopman operator  in each particular application needs to be approximated, implies that the development of various Koopman operator approximation  algorithms is expected to grow along with the vehicular problems they aim to solve.  Given the ongoing development of this research area and the many existing open problems in the fields of smart mobility and vehicle engineering, a survey of techniques and open challenges of applying Koopman operator theory to this vibrant area is warranted.  To the best of our knowledge, this survey paper is the \emph{first of its kind} reviewing the applications of Koopman operator theory within a focused research area, namely, smart mobility and vehicle engineering applications. A \emph{notable feature} of our survey paper is reviewing and categorizing the results of over 100 research papers based on both application and algorithm type  (see Tables~\ref{tab1}--~\ref{tab4} and Section~\ref{sec:vehicApp}) that are concerned with the applications of Koopman operator theory to the field of smart mobility and vehicular engineering. Such a \emph{comprehensive and  detailed categorization} will be beneficial to the research practitioners working in the field.  Furthermore, this review paper discusses theoretical aspects of Koopman operator theory that have been largely neglected by the smart mobility and vehicle engineering community and yet have large potential for contributing to solving open problems in these areas. Additionally, our survey paper seeks to \emph{identify gaps} in the smart mobility and vehicle engineering research where new and existing Koopman operator-based methods have the potential to further develop and address unsolved problems  potentially benefiting from the perspectives of nonlinear system identification, control, global linearization, and the predictive powers that Koopman operator theory has to offer (see, e.g., Remarks~\ref{remGap1}--\ref{remGap6}). 


% \section*{Acknowledgement}
% This work was supported in part by the NSF AI Institute AI4OPT, NSF 2112533.

\medskip
\bibliographystyle{plainnat}
\bibliography{references}
\clearpage
\appendix

% \section{Proof of Theorems}


% % \subsection{Proof of Theorem~\ref{thm:connected_tabular}}



% We note that there exists a bijective map between $\pi$ and $\widehat{\mu}_{\pi}$ where $\widehat{\mu}_{\pi}$ is induced by $\pi$ according to \eqref{eq:mu_hat} and conversely
% \begin{align}
%     \pi(a\mid s)=\frac{\widehat{\mu}_{\pi}(s,a)}{\mu_{\pi}(s)}=\frac{\widehat{\mu}_{\pi}(s,a)}{\sum_{a\in\Acal}\widehat{\mu}_{\pi}(s,a)},\label{thm:connected_tabular:proof_eq1}
% \end{align}
% provided that $\mu_{\pi}(s)\neq 0$, which is guaranteed by Assumption \ref{assump:ergodicity}. \eqref{thm:connected_tabular:proof_eq1} inspires the construction of the path map.


% % We define the map $P:\Delta_{\Acal}^{\Scal}\rightarrow\mathbb{R}^{|\Scal|\times|\Scal|}$ as follows
% % \begin{align}
% %     P(\pi)(s',s)=\sum_{a}\Pcal(s'\mid s,a)\pi(a\mid s).
% % \end{align}

% % It is obvious that $P(\pi)$ is linear in $\pi$. In addition, the stationary distribution is the eigenvector of the transition probability matrix associated with eigenvalue $1$, i.e.
% % \begin{align}
% %     \mu_{\pi}=P(\pi)\mu_{\pi}.
% % \end{align}



% To prove that the superlevel set is connected, we show that for any $\lambda\in\mathbb{R}$ and $\pi_1,\pi_2\in \Ucal_{\lambda,r}$, there exists a continuous path map $p:[0,1]\rightarrow \Ucal_{\lambda,r}$ such that $p(0)=\pi_1$ and $p(1)=\pi_2$. 
% % We use $\mu_1,\mu_2$ to denote the stationary distribution under policy $\pi_1$ and $\pi_2$, respectively. 
% % Note that $\mu_1(s),\mu_2(s)>0$ for all $s\in\Scal$ due to Assumption~\ref{assump:ergodicity}.
% % Note that without loss of generality, we assume that there does not exist a state $s$ where $\mu_1(s)=\mu_2(s)=0$, as the state not being visited under the stationary distribution implies that the policy in these state does not play a role in $\mu$. 
% We now construct the path function $p$ by defining
% % \begin{align}
% %     p(\alpha)(a\mid s) = 
% %         \left\{\begin{array}{ll}
% %         \pi_1(a\mid s) & \text { if $\mu_2(s)=0$} \\
% %         \pi_2(a\mid s) & \text { if $\mu_1(s)=0$} \\
% %         \frac{\alpha \mu_1(s)\pi_1(a\mid s)+(1-\alpha)\mu_2(s)\pi_2(a\mid s)}{\alpha \mu_1(s)+(1-\alpha)\mu_2(s)} & \text { else }
% %         \end{array}\right.
% % \end{align}
% \begin{align*}
%     p(\alpha)(a\mid s) = \frac{\alpha \mu_{\pi_1}(s)\pi_1(a\mid s)+(1-\alpha)\mu_{\pi_2}(s)\pi_2(a\mid s)}{\alpha \mu_{\pi_1}(s)+(1-\alpha)\mu_{\pi_2}(s)},
% \end{align*}
% which is well-defined for all $\alpha\in[0,1]$ as $\mu_1(s),\mu_2(s)$ are positive for all $s\in\Scal$ due to Assumption~\ref{assump:ergodicity}
% % 
% It is easy to see that $p(\alpha)\in\Delta_{\Acal}^{\Scal}$ is a continuous in $\alpha$. We use $\pi_{\alpha}$ to denote the policy $\pi_{\alpha}=p(\alpha)$.

% Then, recall the definition of the transition probability matrix in \eqref{eq:transition_matrix}. Let $B\in\mathbb{R}^{|\Scal|}$ be defined as
% \begin{align*}
%     B=P^{\pi_{\alpha}}\cdot\left(\alpha\mu_{\pi_1}+(1-\alpha)\mu_{\pi_2}\right).
% \end{align*}

% Each entry of $B$ can be expressed as
% \begin{align*}
%     B(s') &= \sum_{s,a}\Pcal(s'\mid s,a)\pi_{\alpha}(a\mid s)\left(\alpha\mu_{\pi_1}(s)+(1-\alpha)\mu_{\pi_2}(s)\right)\notag\\
%     &=\sum_{s,a}\Pcal(s'\mid s,a)\frac{\alpha \mu_{\pi_1}(s)\pi_1(a\mid s)+(1-\alpha)\mu_{\pi_2}(s)\pi_2(a\mid s)}{\alpha \mu_{\pi_1}(s)+(1-\alpha)\mu_{\pi_2}(s)}\left(\alpha\mu_{\pi_1}(s)+(1-\alpha)\mu_{\pi_2}(s)\right)\notag\\
%     &=\sum_{s,a}\Pcal(s'\mid s,a)\alpha \mu_{\pi_1}(s)\pi_1(a\mid s)+\sum_{s,a}\Pcal(s'\mid s,a)(1-\alpha)\mu_{\pi_2}(s)\pi_2(a\mid s)\notag\\
%     &=\alpha\sum_{s,a}P^{\pi_1}_{s',s} \mu_{\pi_1}(s)+(1-\alpha)\sum_{s,a}P^{\pi_2}_{s',s}\mu_{\pi_2}(s)\notag\\
%     &=\alpha\mu_{\pi_1}(s')+(1-\alpha)\mu_{\pi_2}(s').
% \end{align*}

% This means that 
% \begin{align}
%     P^{\pi_{\alpha}}\cdot\left(\alpha\mu_{\pi_1}+(1-\alpha)\mu_{\pi_2}\right)=\alpha\mu_{\pi_1}+(1-\alpha)\mu_{\pi_2}.\label{thm:connected_tabular:proof_eq1}
% \end{align}
% A consequence of Assumption \ref{assump:ergodicity} is that for any policy $\pi$ there is a unique eigenvector of $P^{\pi}$ associated with the eigenvalue equal to $1$, and this eigenvector (properly normalized) is the stationary distribution. Therefore, \eqref{thm:connected_tabular:proof_eq1} implies that $\alpha\mu_{\pi_1}+(1-\alpha)\mu_{\pi_2}$ has to be the stationary distribution under policy $\pi_{\alpha}$, i.e.
% \begin{align*}
%     \mu_{\pi_{\alpha}} = \alpha\mu_{\pi_1}+(1-\alpha)\mu_{\pi_2}.
% \end{align*}
% As a result, for all $s\in\Scal,a\in\Acal$
% \begin{align*}
%     \widehat{\mu}_{\pi_{\alpha}}(s,a) &= \mu_{\pi_{\alpha}}(s)\pi_{\alpha}(a\mid s)\\
%     &=\left(\alpha\mu_{\pi_1}(s)+(1-\alpha)\mu_{\pi_2}(s)\right)\frac{\alpha \mu_{\pi_1}(s)\pi_1(a\mid s)+(1-\alpha)\mu_{\pi_2}(s)\pi_2(a\mid s)}{\alpha \mu_{\pi_1}(s)+(1-\alpha)\mu_{\pi_2}(s)}\\
%     &=\alpha \mu_{\pi_1}(s)\pi_1(a\mid s)+(1-\alpha)\mu_{\pi_2}(s)\pi_2(a\mid s)\\
%     &=\alpha\widehat{\mu}_{\pi_1}(s,a)+(1-\alpha)\widehat{\mu}_{\pi_2}(s,a).
% \end{align*}

% Note that $J_r(\pi) = \sum_{s\in\Scal,a\in\Acal}r(s,a)\widehat{\mu}_{\pi}(s,a)$.
% Since $\pi_{\pi_1},\pi_{\pi_2}\in \Ucal_{\lambda,r}$, we know
% \begin{align*}
%     &\sum_{s\in\Scal,a\in\Acal}r(s,a)\widehat{\mu}_{\pi_1}(s,a)\geq \lambda,\notag\\
%     &\sum_{s\in\Scal,a\in\Acal}r(s,a)\widehat{\mu}_{\pi_2}(s,a)\geq \lambda.
% \end{align*}
% Therefore, we have for any $\alpha\in[0,1]$
% \[
%     J_r(\pi_{\alpha})=\sum_{s\in\Scal,a\in\Acal}r(s,a)\widehat{\mu}_{\pi_{\alpha}}(s,a)=\sum_{s\in\Scal,a\in\Acal}r(s,a)\left(\alpha\widehat{\mu}_{\pi_1}(s,a)+(1-\alpha)\widehat{\mu}_{\pi_2}(s,a)\right)\geq \lambda,
% \]
% which implies $\pi_{\alpha}\in \Ucal_{\lambda,r}$. So far we have verifed that the constructed path map $p$ is indeed continuous and maps $\alpha\in[0,1]$ to $\Ucal_{\lambda,r}$ with $p(0)=\pi_1$ and $p(1)=\pi_2$. This concludes the proof on the connectedness of the superlevel set $\Ucal_{\lambda,r}$. The claim on the equiconnectedness simply follows from the fact that the construction of the path map $p$ does not depend on the reward function.

% \qed




\section{Proof of Theorem~\ref{thm:connected_firstlayer2S}}
We use $X$ to denote the concatenation of the feature vectors across all states
\begin{align*}
    X \triangleq \left[\begin{array}{c}\phi(s_1)^{\top} \\ \phi(s_2)^{\top} \\ \vdots \\ \phi(s_{|\Scal|})^{\top}\end{array}\right]\in\mathbb{R}^{|\Scal|\times d}
\end{align*}

In the analysis we may apply the softmax function $\psi$ to a matrix in a row-wise fashion. Specifically, for any $n\geq 1$ and matrix $M\in\mathbb{R}^{n\times|\Acal|}$, we have
\begin{align*}
    \psi(M)_{i,j}=\frac{\exp(M_{i,j})}{\sum_{j'=1}^{|\Acal|}\exp(M_{i,j'})} \quad \forall i=1,...,n.
\end{align*}

The softmax operator $\psi$ can be inverted up to an additive constant factor. We define $\psi_{inv}$ for any matrix $M\in\mathbb{R}^{n\times|\Acal|}$ as
\[
    \psi_{inv}(M)_{i,j}=\log(M_{i,j})+c_i \quad \forall i,j,
\]
with $c_i$ determined such that $\sum_{j=1}^{|\Acal|}\psi_{inv}(M)_{i,j}=0$. Note that $\psi_{inv}$ is a right inverse of $\psi$, i.e. $\psi(\psi_{inv}(M))=M$ for all matrix $A$.

When the input to a neural network with parameter $\theta$ is the feature table $X$, we denote the output of layer $k$ by $F_k^{\theta}\in\mathbb{R}^{|\Scal|\times n_k}$. According to \eqref{eq:def_f_k}, $F_k^{\theta}$ can be expressed as
\begin{align*}
    F_k^{\theta} = 
    \left\{\begin{array}{ll}
        \sigma\left(X W_1+\1_{|\Scal|} b_{1}^{\top}\right) & k=1 \\
        \sigma\left(F_{k-1}^{\theta} W_k+\1_{|\Scal|} b_{k}^{\top}\right) & k = 2, 3,..., L-1 \\
        \psi\left(F_{L-1}^{\theta} W_{L}+\1_{|\Scal|} b_{L}^{\top}\right) & k=L
    \end{array}\right.
\end{align*}
where $\1_{|\Scal|}$ is the all-one vector of dimension $|\Scal|\times 1$. Note that $F_L^{\theta}\in\mathbb{R}^{|\Scal|\times|\Acal|}$ is the policy table produced by the neural network, i.e. $\pi_{\theta}=F_L^{\theta}$.

The proof of Theorem~\ref{thm:connected_firstlayer2S} relies on the following intermediate results, which we now present. The analysis of Proposition~\ref{prop:connected_linearindepedentX} can be found in Appendix~\ref{sec:proof_prop}.




\begin{prop}\label{prop:connected_linearindepedentX}
If $\rank(X)=|\Scal|$, then under Assumption \ref{assump:ergodicity} and \ref{assump:sigma}, the superlevel set $\Ucal_{\lambda,r}^{\Omega}$ is connected for all $\lambda\in\mathbb{R}$.
\end{prop}


\begin{lem}\label{lem:connected_firstlayer2S:lem1}
Let $(X,W,b,V)\in\mathbb{R}^{|\Scal|\times n_0}\times \mathbb{R}^{n_0\times n_1}\times\mathbb{R}^{n_1}\times \mathbb{R}^{n_1\times n_2}$. Let $Z=\sigma(XW+1_{|\Scal|b^{\top}})V$. Suppose $X$ has distinct rows. Then, under Assumption \ref{assump:sigma} and \ref{assump:network_dimension}, there exists a continuous path map $c:[0,1]\rightarrow \mathbb{R}^{n_0\times n_1}\times\mathbb{R}^{n_1}\times \mathbb{R}^{n_1\times n_2}$ with $c(\lambda)=(W(\lambda), b(\lambda), V(\lambda))$ such that

1) c(0) = (W,b,V),

2) $\sigma\left(X W(\lambda)+\mathbf{1}_{|\Scal|} b(\lambda)^{T}\right) V(\lambda)=Z, \forall \lambda \in[0,1]$,

3) $\rank\left(\sigma\left(X W(1)+\mathbf{1}_{|\Scal|} b(1)^{T}\right)\right)=N$.
\end{lem}


\begin{lem}\label{lem:connected_firstlayer2S:lem2}
Let $(X,W,V,W')\in\mathbb{R}^{|\Scal|\times n_0}\times \mathbb{R}^{n_0\times n_1}\times \mathbb{R}^{n_1\times n_2}\times \mathbb{R}^{n_0\times n_1}$. Suppose $\rank(\sigma(XW))=|\Scal|$ and $\rank(\sigma(XW'))=|\Scal|$. Then, under Assumption \ref{assump:sigma} and \ref{assump:network_dimension}, there exists a continuous path map $c:[0,1]\rightarrow \mathbb{R}^{n_0\times n_1}\times \mathbb{R}^{n_1\times n_2}$ with $c(\lambda)=(W(\lambda), V(\lambda))$ such that

1) c(0) = (W,V),

2) $\sigma\left(X W(\lambda)\right) V(\lambda)=\sigma\left(X W\right)V, \forall \lambda \in[0,1]$,

3) $W(1)=W'$.
\end{lem}

To prove Theorem \ref{thm:connected_firstlayer2S}, it suffices to show that for any $\theta_1=(W_{1,l},b_{1,l})_{l=1}^{L}\in \Ucal_{\lambda,r}^{\Omega}$ and $\theta_2=(W_{2,l},b_{2,l})_{l=1}^{L}\in \Ucal_{\lambda,r}^{\Omega}$ there exists a connected path that is completely within $\Ucal_{\lambda,r}^{\Omega}$.

Applying Lemma \ref{lem:connected_firstlayer2S:lem1} with $(X,W_{1,1},b_{1,1},W_{1,2})$ and $(X,W_{2,1},b_{2,1},W_{2,2})$, the problem simplifies to showing the existence of a continuous path within $\Ucal_{\lambda,r}^{\Omega}$ that connects 
\[\theta_1'=((W_{1,1}',b_{1,1}'),(W_{1,2}',b_{1,2}),(W_{1,l},b_{1,l})_{l=3}^{L})\] 
and 
\[\theta_2'=((W_{2,1}',b_{2,1}'),(W_{2,2}',b_{1,2}),(W_{2,l},b_{2,l})_{l=3}^{L})\] 
such that
\[\rank(F_1^{\theta_1'})=\rank(F_1^{\theta_2'})=|\Scal|.\]

Then, we can apply Lemma \ref{lem:connected_firstlayer2S:lem2} with $([X,1_{|\Scal|}],[W_{1,1}'^{\top},b_{1,1}']^{\top},W_{1,2}',[W_{2,1}'^{\top},b_{2,1}']^{\top})$ to show that there is a continuous path between $\theta_1'$ and $\theta_1''$ with $\theta_1''=((W_{2,1}',b_{2,1}'),(W_{1,2}'',b_{1,2}),(W_{1,l},b_{1,l})_{l=3}^{L})$ such that
\[\rank(F_1^{\theta_1''})=\rank(F_1^{\theta_1'})=|\Scal|.\]

As a consequence, now we simply have to show that $\theta_1''$ and $\theta_2'$ is connected by a continuous path within $\Ucal_{\lambda,r}^{\Omega}$.

Note that $\theta_1''$ and $\theta_2'$ have identical first layer parameters and thus the same first layer output, which is full rank. This allows us to treat the layers from $2$ to $L$ as a new network and apply Theorem \ref{prop:connected_linearindepedentX} (which requires the input to be full rank) to the new network to guarantee that there exists a continuous path map $c:[0,1]\rightarrow \Omega_2\times...\times\Omega_k$ such that
$c(0)=((W_{1,2}'',b_{1,2}),(W_{1,l},b_{1,l})_{l=3}^{L})$, $c(1)=((W_{2,2}',b_{1,2}),(W_{2,l},b_{2,l})_{l=3}^{L})$, and
\[
    \min\{J_r(\pi_{\theta_1}),J_r(\pi_{\theta_2})\}\leq J_r(\pi_{((W_{2,1}',b_{2,1}'),c(\alpha))})\leq \max\{J_r(\pi_{\theta_1}),J_r(\pi_{\theta_2})\}
\]
for all $\alpha\in[0,1]$. This implies that there is indeed a continuous path between $\theta_1''$ and $\theta_2'$ within $\Ucal_{\lambda,r}^{\Omega}$.

Similar to the proof of Theorem~\ref{thm:connected_tabular}, the claim on the connectedness simply follows from the fact that the construction of the path map $p$ does not depend on the reward function.
\qed





\section{Proof of Proposition~\ref{prop:connected_linearindepedentX}}\label{sec:proof_prop}

For each layer of the neural network $k=1,\cdots,L$, we define $\Omega_k^{\star}\subseteq\Omega_k$ to be the set of weights $W_k$ and biases $b_k$ of layer $k$ such that $W_k$ is full rank, i.e.
\begin{align}
    \Omega_k^{\star}=\{(W_k,b_k)\in\Omega_k:\text{$W_k$ is full rank}\}.\label{eq:Omegastar_def}
\end{align}
We denote $\Omega^{\star}=\Omega_1^{\star}\times\Omega_2^{\star}\times...\times\Omega_L^{\star}$. Next, we introduce the following lemmas in aid of the analysis.

\begin{cond}
Given $\theta=(W_l,b_l)_{l=2}^L$, $W_l$ has full rank for every $l\in[2,L]$.
\label{cond:Wl_fullrank}
\end{cond}


\begin{lem}
Under Assumption \ref{assump:sigma}, \ref{assump:network_dimension}, and Condition \ref{cond:Wl_fullrank}, given any $k\in[2,L]$ and matrix $F\in\mathbb{R}^{|\Scal|\times n_k}$, there exists a continuous map $h:\Omega_2^{\star}\times...\times\Omega_k^{\star}\times \mathbb{R}^{|\Scal|\times n_k}\rightarrow\Omega_1$ such that

1) Given $((W_2,b_2),...,(W_k,b_k),F)\in\Omega_2^{\star}\times...\times\Omega_k^{\star}\times \mathbb{R}^{|\Scal|\times n_k}$, we have
\begin{align*}
    F_k^{h((W_l,b_l)_{l=2}^{k},F),(W_l,b_l)_{l=2}^{k}}=F.
\end{align*}

2) For any $\theta^{\star}=(W_l^{\star},b_l^{\star})_{l=1}^{L}\in\Omega_1\times\Omega_2^{\star}\times...\times\Omega_L^{\star}$, there exists a continuous path map $p:[0,1]\rightarrow \Omega_1\times\Omega_2^{\star}\times...\times\Omega_L^{\star}$ such that $p(0)=\theta^{\star}$, $p(1)=\left(h((W_l^{\star},b_l^{\star})_{l=2}^{k},F_k^{\theta^{\star}}),(W_l^{\star},b_l^{\star})_{l=2}^{L}\right)$, and $F_L^{p(\alpha)}=F_L^{\theta^{\star}}$ for all $\alpha\in[0,1]$.
\label{lem:fullrank_path}
\end{lem}

\begin{lem}
Given two connected sets $\Acal\subseteq\mathbb{R}^{m_1\times n}$ and $\Bcal\subseteq\mathbb{R}^{n\times m_2}$, the set $\{ab\mid a\in \Acal, b\in \Bcal\}$ is connected. Given two connected sets $\Acal,\Bcal\subseteq\mathbb{R}^{m\times n}$, the set $\{a+b\mid a\in \Acal, b\in \Bcal\}$ is connected.
\label{lem:sum_product_connectedset}
\end{lem}


\begin{lem}
Under Assumption \ref{assump:sigma}, for any $\theta\in\Omega$, there exist $\theta^{\star}\in\Omega^{\star}$ and a continuous path map $p:[0,1]\rightarrow \Omega$ such that $p(0)=\theta$, $p(1)=\theta^{\star}$, and $F_L^{p(\alpha)}=F_L^{\theta}$ for all $\alpha\in[0,1]$.
\label{lem:nonfullrank_to_fullrank_path}
\end{lem}

\begin{lem}\label{lem:matrix_tall_fullrank_connected}
If $n<m$, then the set $\Fcal=\{F\in\mathbb{R}^{m \times n}\mid\rank(F)=n\}$ is connected. In other words, given $F_1,F_2\in\Fcal$, there exists a continuous path map $q:[0,1]\rightarrow\Fcal$ such that $q(0)=F_1$ and $q(1)=F_2$.
\end{lem}




Fix a $\lambda\in\mathbb{R}$. To show the superlevel set $\Ucal_{\lambda,r}^{\Omega}$ is connected, it suffices to show that for any $\theta_1,\theta_2\in \Ucal_{\lambda,r}^{\Omega}$, there exists a continuous path between them that is completely in $\Ucal_{\lambda,r}^{\Omega}$. 

Without any loss of generality, we can safely assume that both $\theta_1=(W_{1,l},b_{1,l})_{l=1}^{L}$ and $\theta_2=(W_{2,l},b_{2,l})_{l=1}^{L}$ satisfy Condition \ref{cond:Wl_fullrank}, since otherwise by Lemma \ref{lem:nonfullrank_to_fullrank_path} we can find a continuous path from $\theta_1$ and $\theta_2$ that leads to one satisfying Condition \ref{cond:Wl_fullrank}. We denote the policies parameterized by $\theta_1,\theta_2$ as $\pi_1,\pi_2$, i.e.
\[
    \pi_1 = F_L^{\theta_1}, \quad \pi_2 = F_L^{\theta_2}.
\]

% Similar to the proof of Theorem \ref{thm:connected_tabular}, we denote the stationary distribution of states under policy $\pi_1,\pi_2$ by $\mu_1,\mu_2\in\Delta_{\Scal}$, and the stationary distribution of state-action pairs by $\widehat{\mu}_1,\widehat{\mu}_2\in\Delta_{\Scal\times\Acal}$.

By Lemma \ref{lem:fullrank_path}, there is a continuous path from $\theta_1/\theta_2$ to $\theta_1'/\theta_2'$ where we define
\begin{align*}
    &\theta_1'=\left(h\left(\left(W_{1,l}, b_{1,l}\right)_{l=2}^{L}, \pi_1\right),(W_{1,l},b_{1,l})_{l=2}^{L}\right),\notag\\
    \text{and}\quad&\theta_2'=\left(h\left(\left(W_{2,l}, b_{2,l}\right)_{l=2}^{L}, \pi_2\right),(W_{2,l},b_{2,l})_{l=2}^{L}\right).
\end{align*}

Now, we just have to show that there exists a continuous path between $\theta_1'$ and $\theta_2'$ that is completely within $\Ucal_{\lambda,r}^{\Omega}$.
% 
By Lemma \ref{lem:matrix_tall_fullrank_connected}, we know that for $l=2,..,L$, there exists a continuous path map $q_l:[0,1]\rightarrow\Omega_l^{\star}$ such that $q_l(1)=W_{1,l}$ and $q_l(0)=W_{2,l}$. Then, we construct the map $q:[0,1]\rightarrow\Omega$
\begin{align*}
    q(\alpha) = \left(h((q_l(\alpha),\alpha b_{1,l}+(1-\alpha) b_{2,l})_{l=2}^{L},\pi_1),(q_l(\alpha),\alpha b_{1,l}+(1-\alpha) b_{2,l})_{l=2}^{L}\right) \quad\forall \alpha\in[0,1].
\end{align*}

It is obvious that $q$ is a continuous map as $h,q_2,...,q_L$ are continuous. In addition, $F_L^{q(\alpha)}=\pi_1$ for all $\alpha\in[0,1]$, and $q(1)=\theta_1'$. We define 
\[
    \theta_1''=q(0)=\left(h\left(\left(W_{2,l}, b_{2,l}\right)_{l=2}^{L}, \pi_1\right),(W_{2,l},b_{2,l})_{l=2}^{L}\right).
\]

Now our aim simplifies to finding a continuous path between $\theta_1''$ and $\theta_2'$ that is completely in $\Ucal_{\lambda,r}^{\Omega}$. To show that this path exists, we construct a continuous map $t:[0,1]\rightarrow\Omega$ as follows
\[
    t(\alpha) = \left(h\left(\left(W_{2,l}, b_{2,l}\right)_{l=2}^{L}, \tilde{\pi}_{\alpha}\right),(W_{2,l},b_{2,l})_{l=2}^{L}\right)\quad\forall \alpha\in[0,1],
\]
where $\tilde{\pi}$ is defined entry-wise
\[
    \tilde{\pi}_{\alpha}(a\mid s) = \frac{\alpha \mu_{\pi_1}(s)\pi_1(a\mid s)+(1-\alpha)\mu_{\pi_2}(s)\pi_2(a\mid s)}{\alpha \mu_{\pi_1}(s)+(1-\alpha)\mu_{\pi_2}(s)}.
\]


It can be seen that $t$ is indeed continuous since $\tilde{\pi}_{\alpha}$ is continuous in $\alpha$, and $t(0)=\theta_2'$ and $t(1)=\theta_1''$. What remains to be shown is that $F_L^{t(\alpha)}\in \Ucal_{\lambda,r}^{\Omega}$, i.e. $J_r(F_L^{t(\alpha)})\geq\lambda$. By the definition of $h$ in Lemma \ref{lem:fullrank_path}, $F_L^{t(\alpha)}=\tilde{\pi}_{\alpha}$. It has been shown in the proof of Theorem \ref{thm:connected_tabular} that indeed $J_r(\tilde{\pi}_{\alpha})\geq\lambda$ provided that $J_r(\pi_1)\geq\lambda$ and $J_r(\pi_2)\geq\lambda$. This concludes the proof of Proposition \ref{prop:connected_linearindepedentX}.


\qed
\section{Proof of Supporting Lemmas}\label{sec:proof_lemmas}

\subsection{Proof of Lemma \ref{lem:connected_firstlayer2S:lem1}}
This lemma is adapted from Lemma 5.2 of \citet{nguyen2019connected}.


\subsection{Proof of Lemma \ref{lem:connected_firstlayer2S:lem2}}
This lemma is adapted from Lemma 5.3 of \citet{nguyen2019connected}.


\subsection{Proof of Lemma \ref{lem:fullrank_path}}
We provide a proof for the case $k=L$. For $k\neq L$, the proof can be found in \citet{nguyen2019connected}[Lemma 3.3]. 

For $((W_2,b_2),...,(W_L,b_L),\pi)\in\Omega_2^{\star}\times...\times\Omega_L^{\star}\times \Delta_{\Acal}^{\Scal}$, we define the map $h$ as follows
\begin{align*}
    h\left((W_l,b_l)_{l=2}^{L},\pi\right)=\left(\widehat{W}_1,\widehat{b}_1\right)
\end{align*}
where $\widehat{W}_1$ and $\widehat{b}_1$ is defined as
\begin{align}
\left\{\begin{array}{l}
{\left[\begin{array}{c}
W_{1} \\
b_{1}^{\top}
\end{array}\right]=\left[X, \mathbf{1}_{|\Scal|}\right]^{\dagger} \sigma^{-1}\left(B_{1}\right),} \\
B_{l}=\left(\sigma^{-1}\left(B_{l+1}\right)-\mathbf{1}_{|\Scal|} b_{l+1}^{\top}\right) W_{l+1}^{\dagger}, \forall l \in[1, k-2] \\
B_{k-1}=
\left(\psi_{inv}(\pi)-\mathbf{1}_{|\Scal|} b_{L}^{\top}\right) W_{L}^{\dagger}
\end{array}\right.
\label{eq:h_def}
\end{align}
where we use $A^{\dagger}$ to denote the Moore-Penrose inverse of a matrix $A$. If A has full column rank, then we have $A^{\dagger}A=I$. If A has full row rank, we have $AA^{\dagger}=I$. We can easily see that the defined $h$ operator is continuous as it is a composition of continuous operators.



Assumption \ref{assump:network_dimension}, and Condition \ref{cond:Wl_fullrank} imply that the matrices $W_2,...,W_L$ all have full column rank, which means $W_l^{\dagger}W_l=I$. We also know that $[X, \1_{|\Scal|}]$ has full row rank by our assumption that $X$ has full row rank, which means $[X, \1_{|\Scal|}][X, \1_{|\Scal|}]^{\dagger}=I$. Therefore, we can layerwise invert \eqref{eq:h_def} and verify that 
\begin{align*}
    F_L^{h((W_l,b_l)_{l=2}^{L},\pi),(W_l,b_l)_{l=2}^{L}}=\pi.
\end{align*}

For every layer $l=2,...,L$, we define the operator $G_l:\mathbb{R}^{|\Scal|\times n_{l-1}}\rightarrow \mathbb{R}^{|\Scal|\times n_{l}}$
\begin{align*}
    G_l(Y) = \left\{\begin{array}{ll}
    \sigma\left(Y W_{l}^{\star}+\1_{|\Scal|}\left(b_{l}^{\star}\right)^{\top}\right) & l \in[2, L-1]\\
    \psi\left(Y W_{L}^{\star}+\1_{|\Scal|}\left(b_{L}^{\star}\right)^{\top}\right) & l=L 
    \end{array}\right.
\end{align*}

We also define the operator $H:\mathbb{R}^{(d+1)\times n_{1}}\rightarrow \mathbb{R}^{|\Scal|\times n_{1}}$
\begin{align*}
    H(Y) = \sigma\left([X,\1_{|\Scal|}]Y\right).
\end{align*}

To show the continuous path claimed in Lemma \ref{lem:fullrank_path} exists, it suffices to show that the set $\{(W_1,b_1):F_L^{(W_1,b_1),(W_l^{\star},b_l^{\star})_{l=2}^{L}}=F_L^{\theta^{\star}}\}$ is connected, which is equivalent to showing that the set $f^{-1}(F_L^{\theta^{\star}})$ is connected where $f$ is defined as
\begin{align*}
    f([W_1^{\top},b_1]^{\top})=G_L\circ ...\circ G_2\circ H([W_1^{\top},b_1]^{\top}).
\end{align*}

Note that the definition of $f$ implies
\begin{align}
    f^{-1}(\pi)=H^{-1}\circ G_2^{-1}\circ ...\circ G_L^{-1}(\pi).
    \label{eq:f_inv}
\end{align}

Note that $G_l^{-1}$ is 
\begin{align*}
    \begin{array}{l}
    G_{l}^{-1}(F)\hspace{-2pt}= \hspace{-2pt}
    \left\{\begin{array}{l}
    \left(\psi_{inv}(F)\hspace{-2pt}+\hspace{-2pt}\{C\mid C_{i,j}\hspace{-2pt}=\hspace{-2pt}C_{i,j'} \,\forall i, j\hspace{-2pt}\neq\hspace{-2pt} j'\}\hspace{-2pt}-\hspace{-2pt}\mathbf{1}_{N} b_{L}^{T}\right)\left(W_{L}^{\star}\right)^{\dagger}+\left\{B \hspace{-2pt}\mid\hspace{-2pt} B W_{L}^{\star}=0\right\},\,\, l=L \\
    \left(\sigma^{-1}(F)-\mathbf{1}_{N} b_{l}^{\star}\right)\left(W_{l}^{\star}\right)^{\dagger}+\left\{B \mid B W_{l}^{\star}=0\right\},\,\, l=2,...,L-1
    \end{array}\right.
    \end{array}
\end{align*}

It is easy to verify that $\{C\mid C_{i,j}=C_{i,j'} \,\forall i, j\neq j'\}$ and $\left\{B \mid B W_{l}^{\star}=0\right\}$ for all $l=2,...,L$. Then, Lemma~\ref{lem:sum_product_connectedset} implies that $G_l^{-1}(F)$ is a connected set for all $F$.

Similarly, $H^{-1}(F)=\left[X, \1_{|\Scal|}\right]^{\dagger} \sigma^{-1}(F)+\left\{B \mid\left[X, \1_{|\Scal|}\right] B=0\right\}$ is also a connected set for all $F$. Therefore, from \eqref{eq:f_inv} we know that $f^{-1}(F)$ is a connected set for any $F$, which concludes the proof of Lemma \ref{lem:fullrank_path}.

\qed

\subsection{Proof of Lemma~\ref{lem:sum_product_connectedset}}

To show that the product of the two connected sets are connected, we consider any $x,y\in\{ab\mid a\in \Acal, b\in \Bcal\}$. Obviously, there exist $a_x,a_y\in\Acal$ and $b_x,b_y\in\Bcal$ such that $x=a_x b_x$, $y=a_y b_y$. The connectedness of $\Acal$ and $\Bcal$ implies that there exists continuous path maps $p_{\Acal}:[0,1]\rightarrow\Acal$ and $p_{\Bcal}:[0,1]\rightarrow\Bcal$ such that $p_{\Acal}(0)=a_x$, $p_{\Acal}(1)=a_y$, $p_{\Bcal}(0)=b_x$, $p_{\Bcal}(1)=b_y$. Define $p(\alpha)=p_{\Acal}p_{\Bcal}$ for $\alpha\in[0,1]$. It is obvious that $p(\alpha)\in\{ab\mid a\in \Acal, b\in \Bcal\}$ for all $\alpha$. Since the product of continuous maps is still continuous, $p:[0,1]\rightarrow\{ab\mid a\in \Acal, b\in \Bcal\}$ is a continuous path map satisfying $p(0)=x$ and $p(1)=y$. This implies that the set $\{ab\mid a\in \Acal, b\in \Bcal\}$ is a connected set.

A similar argument can be used to show that the sum of two connected sets is connected.


\qed


\subsection{Proof of Lemma \ref{lem:nonfullrank_to_fullrank_path}}
Define $\tilde{F}_L((W_l,b_l)_{l=1}^{L})$ as the output of the final layer before the softmax activation 
\[
    \tilde{F}_L^{(W_l,b_l)_{l=1}^{L}} = F_{L-1} W_{L}+\1_{|\Scal|} b_{L}^{\top}.
\]

Then, existing results in the literature (such as Lemma 3.4 of \citet{nguyen2019connected}) show that for any $\theta\in\Omega$, there exists a continuous path map $p:[0,1]\rightarrow \Omega$ such that $p(0)=\theta$, $p(1)=\theta^{\star}\in\Omega^{\star}$, and $\tilde{F}_L(p(\alpha))=\tilde{F}_L(\theta)$ for all $\alpha\in[0,1]$. This leads to our claim.

\qed

\subsection{Proof of Lemma \ref{lem:matrix_tall_fullrank_connected}}

This lemma is adapted from Theorem 4 of \citet{evard1994set}.
\section{Convexity of Optimization Program \eqref{eq:robustrl_convexside}}\label{sec:robustrl_convexside_proof}

In this section, we show that \eqref{eq:robustrl_convexside} is a convex optimization program. First, we note that
\begin{align*}
    J_{r'}(\pi)=\sum_{s,a}r'(s,a)\widehat{\mu}_{\pi}=\widehat{\mu}_{\pi}^{\top}r',
\end{align*}
which means that the objective function is linear in the reward.

The constraint set is obvious closed. It is also bounded as the reward $r(s,a)\in[0,U_r]$. To prove the constraint set is convex, we need to show that for any $r_1,r_2$ such that $\operatorname{Attack}(r_1,\pi_{\dagger},\epsilon_{\dagger})=r_{\dagger}$ and $\operatorname{Attack}(r_2,\pi_{\dagger},\epsilon_{\dagger})=r_{\dagger}$, we have
\begin{align}
    \operatorname{Attack}(\alpha r_1+(1-\alpha)r_2,\pi_{\dagger},\epsilon_{\dagger})=r_{\dagger},\quad\forall \alpha\in[0,1].\label{eq:robustrl_convexside:proof_eq1}
\end{align}

By the optimality condition of \eqref{eq:attack_obj}, $r_{\dagger}$ being the optimal poisoned reward for true reward $r_1$ and $r_2$ is equivalent to
\begin{align*}
    \langle r-r_{\dagger},r_1-r_{\dagger}\rangle\leq 0 \quad \text{and} \quad \langle r-r_{\dagger},r_2-r_{\dagger}\rangle\leq 0
\end{align*}
for all $r$ such that $J_r(\pi_{\dagger})\geq J_r(\pi)+\epsilon_{\dagger},\,\forall \pi\in\Pi^{\text{det}}\backslash\pi_{\dagger}$. By taking the convex combination of these two inequalities, we have for any $\alpha\in[0,1]$
\begin{align}
    \langle r-r_{\dagger},\alpha r_1+(1-\alpha)r_2-r_{\dagger}\rangle\leq 0\label{eq:robustrl_convexside:proof_eq2}
\end{align}
for all $r$ such that $J_r(\pi_{\dagger})\geq J_r(\pi)+\epsilon_{\dagger},\,\forall \pi\in\Pi^{\text{det}}\backslash\pi_{\dagger}$. Again by the optimality condition of \eqref{eq:attack_obj}, \eqref{eq:robustrl_convexside:proof_eq2} is equivalent to \eqref{eq:robustrl_convexside:proof_eq1}.

At this point, we have shown that \eqref{eq:robustrl_convexside} has a linear objective function and a convex (and also compact) constraint set. As a result, the optimization program is convex.






\end{document}
