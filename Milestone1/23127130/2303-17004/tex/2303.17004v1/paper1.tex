\documentclass{article}
\usepackage[utf8]{inputenc}
\usepackage{amsmath}
\usepackage{amssymb}
\usepackage{amsthm}
\usepackage{color}
\usepackage{soul}
\usepackage{verbatim}
\usepackage{url}
\usepackage{geometry}
\usepackage{graphicx}
\usepackage{asymptote}
\usepackage{hyperref}
\usepackage[left]{showlabels}
\usepackage{bbm}
\usepackage[shortlabels]{enumitem}



\newtheorem*{rep@theorem}{\rep@title}
\newcommand{\newreptheorem}[2]{%
\newenvironment{rep#1}[1]{%
 \def\rep@title{#2 \ref{##1}}%
 \begin{rep@theorem}}%
 {\end{rep@theorem}}}
\makeatother






\newcommand{\ds}{\displaystyle}
\newcommand{\cP}{\mathcal{P}}
\newcommand{\cS}{\mathcal{S}}
\newcommand{\cQ}{\mathcal{Q}}
\newcommand{\cR}{\mathcal{R}}
\newcommand{\cI}{\mathcal{I}}
\newcommand{\cD}{\mathcal{D}}
\newcommand{\cO}{\mathcal{O}}
\newcommand{\cC}{\mathcal{C}}

\newcommand{\Z}{\mathbb{Z}}
\newcommand{\Q}{\mathbb{Q}}
\newcommand{\R}{\mathbb{R}}
\newcommand{\C}{\mathbb{C}}
\newcommand{\N}{\mathbb{N}}
\newcommand{\eps}{\varepsilon}
\newcommand{\bI}{\overline{I}}
\newcommand{\bJ}{\overline{J}}
\newcommand{\one}{\mathbbm{1}}
\newcommand{\fkS}{\mathfrak{S}}

\newcommand{\Aut}{\operatorname{Aut}}
\newcommand{\Hilb}{\operatorname{Hilb}}

\newcommand{\wt}{\mathrm{wt}}
\newcommand{\tit}{\textit}
\newcommand{\ccdot}{\cdot \cdot \cdot}
\newcommand{\cm}{\operatorname{CM}}
\newcommand{\Id}{\mathrm{Id}}
\DeclareMathOperator{\ncm}{NCM}
\newcommand{\tw}{\tilde{w}}

\newcommand{\z}[1]{\Z/#1 \Z}

\newcommand{\h}{\hspace{2mm} }


\newcommand{\todo}[1]{\textcolor{blue}{#1}}

\newcommand{\syl}[1]{{\color{red}[#1]}}


\theoremstyle{plain}
\newtheorem*{Thm}{Theorem}
\newtheorem{Lemma}{Lemma}
\newtheorem{Cor}{Corollary}
\newtheorem*{Prop}{Proposition}

\newtheorem{thm}{Theorem}[section]
\newtheorem{prop}[thm]{Proposition}
\newtheorem{cor}[thm]{Corollary}
\newtheorem{lemma}[thm]{Lemma}

\newreptheorem{lemma}{Lemma}



\newtheorem{conjecture}[thm]{Conjecture}
\theoremstyle{definition}
\newtheorem{definition}{Definition}
\theoremstyle{remark}
\newtheorem*{case}{Case}
\newtheorem*{remark}{Remark}
\newtheorem*{example}{Example}
\newtheorem*{note}{Note}
\newtheorem{question}[thm]{Question}


\DeclareMathOperator{\imm}{Imm}
\DeclareMathOperator{\inv}{Inv}
\DeclareMathOperator{\sgn}{sgn}
\DeclareMathOperator{\mat}{Mat}


\title{$\%$-Immanants and Temperley-Lieb Immanants}
\author{Frank Lu, Kevin Ren, Dawei Shen, Siki Wang}
\date{March 2023}

\begin{document}

\maketitle
\begin{abstract}
% In this paper, we investigate a certain class of nice immanants, which we call \%-immanants, and their relationship to Temperley-Lieb immanants, first introduced by Rhoades and Skandera %in their paper
% \cite{RS}. In particular, we classify when a Temperley-Lieb immanant can be written as a linear combination of \%-immanants. To do this, we derive an explicit formula for computing the Temperley-Lieb immanant coming from a $321$-, $1324$-avoiding permutation $w$. %which contains the pattern $2143.$
% We also make progress on extending these results to Kazhdan-Lusztig immanants, a generalization of Temperley-Lieb immanants, such as obtaining a necessary condition for when a Kazhdan-Lusztig immanant is a linear combination of \%-immanants and a classification for when a Kazhdan-Lusztig immanant can be written as a sum of at most two \%-immanants. Finally, we conjecture an explicit formula for computing the Kazhdan-Lusztig immanant coming from a $1324$-, $32154$-, $21543$-avoiding permutation $w$.

In this paper, we investigate the relationship between Temperley-Lieb immanants, which were introduced by Rhoades and Skandera, and \%-immanants, an immanant based on a concept introduced by Chepuri and Sherman-Bennett. Our main result is a classification of when a Temperley-Lieb immanant can be written as a linear combination of \%-immanants. This result uses a formula by Rhoades and Skandera to compute Temperley-Lieb immanants in terms of complementary minors. Using this formula, we also derive an explicit expression for the coefficients of a Temperley-Lieb immanant coming from a $321$-, $1324$-avoiding permutation $w$ containing the pattern $2143,$ which we use to derive our main result.
% We also partially extend our classification to general Kazhdan-Lusztig immanants: we obtain a necessary condition for a Kazhdan-Lusztig immanant to be a linear combination of \%-immanants and a classification for a Kazhdan-Lusztig immanant to be written as a sum of at most two \%-immanants. Finally, we conjecture an explicit formula for computing the Kazhdan-Lusztig immanants coming from a $1324$-, $32154$-, $21543$-avoiding permutation $w$, and using this conjectural formula, we derive expressions for $1324$-, $24153$-, $31524$-, $32154$-, $21543$-, $231564$-, $312645$-, $426153$-avoiding Kazhdan-Lusztig immanants as a sum of \%-immanants.
\end{abstract}
% \syl{Abstract: In this paper, we investigate a class of immanants which we call \%-immanants, and their relationship with TL immanants of Rhoads and Skandera [cite here]. In particular, we classify the TL-immanants that can be interpreted as linear combinations of \%-immanants. As a byproduct, we give an explicit formula for the TL immanant from a $321$-$1324$-avoiding permuation $w$ which contains $2143$. We also make some partial progress on extending our main result to KL immanants which are generalizations of TL immanants.


% }

\tableofcontents
% \syl{Throughout the paper Be consistent with $321-$avoiding v.s. $321$-avoiding. I think the later looks nicer, because the first one might create wired spacing in some places.}
\section{Introduction}
% Here's a suggestion on how to structure the introduction}
% \todo{Edit the introduction.}
Immanants are functions defined on square matrices which are generalizations of the determinant. For a function $f:\mathfrak S_n\to\mathbb C$, we define the immanant associated to $f$, $\imm_f:M_{n\times n}\to \mathbb C$, as
\[\imm_f(X)= \sum\limits_{\sigma \in \mathfrak{S}_n} f(\sigma)x_{1, \sigma(1)}x_{2, \sigma(2)} \cdots x_{n, \sigma(n)},\] where the matrix $X$ has entries $x_{i, j}.$ We may also view this immanant as a polynomial in $\mathbb{C}[x_{i, j}],$ where $1 \leq i, j \leq n.$

We are interested in the following two families of immanants.

% \begin{definition}[Kazhdan-Lusztig immanants]
% Let $w\in \mathfrak{S}_n$. The Kazhdan-Lusztig immanant $\imm_w: \mat_{n\times n}(\C)\to \C$ is given by 
% \[\imm^{KL}_w(M):= \sum_{u\in \mathfrak{S}_n} (-1)^{l(u)-l(w)} P_{w_0 u,w_0 w}(1) m_{1,u_1}m_{2,u_2}\cdots m_{n,u_n} \]
% where $P_{x,y}(q)$ is the Kazhdan-Lusztig polynomial associated to $x,y\in\mathfrak s_n$ and $w_0$ is the longest word in $\mathfrak{S}_n$. See \cite[Section 5.5]{bjorner2006combinatorics} for definitions of Kazhdan-Lusztig polynomials.
% \end{definition}

First, we have Temperley-Lieb immanants, introduced in \cite{RS} by Rhoades and Skandera. These immanants are indexed by $321$-avoiding permutations, and their coefficients are derived from coefficients of expansions of a product within the Temperley-Lieb algebra. They are also a special case of another type of immanant, called Kazhdan-Lusztig immanants, as proven in \cite{HarderRS}. Unlike Temperley-Lieb immanants, Kazhdan-Lusztig immanants are a lot more difficult to understand, as their coefficients are related to Kazhdan-Lusztig polynomials, which in turn are related to each other by a recursive relation. 
%This special subclass satisfies a different type of positivity, called network positivity, related to the evaluation of a path matrix associated with a given planar network. 
% These immanants, unlike Kazhdan-Lusztig immanants, have coefficients that can be defined non-recursively within the Temperley-Lieb algebra.

We then introduce a new class of immanants which we call \%-immanants. These immanants are based on the notion of the determinant formulas from \cite{CSB}, and are significantly easier to compute than Temperley-Lieb immanants.
\begin{definition}[\%-immanants]
Suppose $\lambda / \mu$ is a skew tableau. The \%-immanant associated to this permutation is defined by $$\imm^{\%}_{\lambda/\mu}(X) = \sum_{\sigma \in A} \sgn(\sigma) x_{1, \sigma(1)}x_{2, \sigma(2)} \cdots x_{n, \sigma(n)},$$ where $\sigma \in A$ iff for all $i$, $(i,\sigma(i)) \in \lambda /\mu$. This is related to a `Skew Ferrers Matrix' of \cite{MR2353118}.
\end{definition}

%Recently, Chepuri and Sherman-Bennett \cite{CSB}, in their analysis of $k-$positivity of certain Kazhdan-Lusztig immanants, were able to arrive at a simple formula to compute Kazhdan-Lusztig immanants associated to $2143$ and $1324-$avoiding permutations, deriving from the connections between Kazhdan-Lusztig immanants and Schubert varities. These simple formulas were in terms of another class of immanant, which we refer to as a \%-immanant, which are significantly easier to compute.

Recently, using the connections between Kazhdan-Lusztig immanants and Schubert varieties, Chepuri and Sherman-Bennett \cite{CSB} proved that for a $2143$ and $1324$-avoiding permutation $w$, the corresponding Kazhdan-Lusztig immanant $\imm_w(M)$ is the same as some $\%$-immanant (up to signs). In particular, the same result holds for Temperley-Lieb immanants. Motivated by this result, our paper seeks to answer the following question.

\begin{question}
Which Temperley-Lieb immanants are linear combinations of \%-immanants?
\end{question}

The advantage of specializing to the case of Temperley-Lieb immanants is that determining the coefficients of these immanants, while still difficult, can be done non-recursively, suggesting that such a question is more tenable for Temperley-Lieb immanants than Kazhdan-Lusztig immanants in general. In this paper, we provide a complete answer to the above question. Furthermore, our proofs will be purely combinatorial, based on the relationship between Temperley-Lieb immanants, non-crossing matchings, and colorings. Our main result is the following.

\begin{Thm}
Let $w$ be a $321$-avoiding permutation. The following statements are equivalent:
\begin{enumerate}
    \item The Temperley-Lieb immanant $\imm_w$ is a linear combination of \%-immanants,
    
    \item The signed Temperley-Lieb immanant $\sgn(w) \imm_w$ is a sum of at most two \%-immanants,
    
    \item The permutation $w$ avoids the patterns $1324, 24153, 31524, 231564$, and $312645,$ in addition to avoiding $321.$
\end{enumerate}
\end{Thm}
% The Kazhdan-Lusztig case is harder in general, although we are able to prove some partial results, such as a generalization of one direction of the above theorem, and some analysis of the case of two \%-immanants. In particular, we are able to arrive at the following theorem:

% \begin{Thm}
% Given $w \in S_n,$ we have that $\imm_w$ is a sum of at most two \%-immanants if and only if $w$ avoids the patterns $1324, 24153, 31524, 426153, 312645, 231564, 21543, 32154, 5472163, 5276143, 65872143,$ \\ $65827143, 64872153.$
% \end{Thm}

\par The plan of the paper is as follows. In section 2, we go through the important concepts and introduce some of the important objects that we'll be studying, and review some preliminary results about them. Following that, we cover \%-immanants in section 3, where we also present a result describing which immanants can be written as linear combinations of \%-immanants. 
\par In section 4, we prove that the Temperley-Lieb immanant of a $321$-avoiding permutation $w$ is a \%-immanant (in fact, the \%-immanant associated with $w$) if and only if $w$ avoids the patterns $2143$ and $1324$ using combinatorial methods. Although both directions can be proved relatively easily with known results (e.g. \cite{billey2003maximal}, \cite{CSB}), the setup here will evoke the style of proof we will use in the rest of the paper.
\par We conclude in section 5 by proving that the Temperley-Lieb immanant of a $321$-avoiding permutation is a sum of two \%-immanants, and more generally a linear combination of \%-immanants, if and only if it avoids the patterns $1324, 24153, 31524, 231564, 312645.$ In the course of proving this result, we will also arrive at a relatively simple combinatorial formula for the coefficients of a Temperley-Lieb immanant associated with a $321$- and $1324$-avoiding permutation $w$ that contains the pattern $2143.$ We note that this formula, along with the work in section 4, gives us the coefficients of the Temperley-Lieb immanant of any permutation $w$ that avoids $321$ and $1324.$
% \par In section 6, we turn to Kazhdan-Lusztig immanants in general and show that certain patterns need to be avoided for the associated Kazhdan-Lusztig immanant to be a linear combination of \%-immanants, generalizing one part of the theorem that we obtain in section 5. We consider in this section both the general linear combination case and the sum of two \%-immanant case, and arrive at a classification of when we may write our Kazhdan-Lusztig immanant as a sum of two \%-immanants. Analogous to section 5, we also conjecture a combinatorial formula for the coefficients of the Kazhdan-Lustzig immanants associated to permutations that avoid certain patterns, and show that this conjecture holds when additional pattern avoidance restrictions are assumed, in turn yielding an expression for Kazhdan-Lusztig immanants associated to these permutations in terms of \%-immanants. We conclude the section by arguing that certain \%-immanants are a sum of a Temperley-Lieb immanant and a Kazhdan-Lusztig immanant. We finish in section 7 by briefly discussing how this result allows us to make statements about some notion of positivity of these \%-immanants.
% ][These can be simplified.]
\par Throughout the paper, we will leave the proofs of technical lemmas at the end of each section, so that the reader can first focus on the bigger picture. An interested reader is welcomed to go through the proofs of the lemmas. 
\par We have an upcoming paper that extends our results in this paper to general Kazhdan-Lusztig immanants, i.e., we do not require the associated permutation to be $321$-avoiding.


% In their paper \cite{KL}, Kazhdan and Lusztig introduced a class of polynomials $P_{v, w}(q),$ now referred to as Kazhdan-Lusztig polynomials, in their study of the Hecke algebra and Coxeter group representations. In particular, these polynomials appear as coefficients within a basis, the Kazhdan-Lusztig basis, for the Hecke algebra. Beyond their work in representations, these polynomials have turned out to have many different combinatorial interpretations; see \cite{bjorner2006combinatorics} for many different combinatorial formulas of these Kazhdan-Lusztig polynomials.
% \par Building off of this work in the case of representations of $\mathfrak{S}_n,$ Rhoades and Skandera introduced a class of functions in their paper \cite{HarderRS} called immanants, which are functions of the form $$\imm_f = \sum\limits_{\sigma \in \mathfrak{S}_n} f(\sigma)x_{1, \sigma(1)}x_{2, \sigma(2)} \cdots x_{n, \sigma(n)}$$ for some function $f: \mathfrak{S}_n \rightarrow \mathbb{R}.$ In particular, Rhoades and Skandera defined a Kazhdan-Lusztig immanant associated to any $w \in \mathfrak{S}_n,$ which are dual to the Kazhdan-Lusztig basis. These immanants, as noted in \cite{HarderRS}, formed an important part of studying the dual canonical basis of $\mathcal{O}(GL_n(\mathbb{C})).$ Various positivity properties were proven in \cite{HarderRS}.  Unfortunately, formulas for computing these immanants, and the Kazhdan-Lusztig polynomials more generally, are difficult to come by, due to the recursive nature of the definition of these polynomials. 
% \par Another class of immanants, Temperley-Lieb immanants, were introduced using a different algebra, known as the Temperley-Lieb algebra, in an earlier paper by Rhoades and Skandera, \cite{RS}. It was later proven in \cite{HarderRS} that these immanants, in fact, were a special case of Kazhdan-Lusztig immanants, in the case when $w$ avoided the pattern $321.$ This special subclass satisifed a different type of positivity, called network positivity, related to the evaluation of a path matrix associated with a given planar network. These immanants, unlike Kazhdan-Lusztig immanants, have coefficients that can be defined non-recursively within the Temperley-Lieb algebra.
% \par Recently, Chepuri and Sherman-Bennett in their paper \cite{CSB}, in their analysis of $k-$positivity of certain Kazhdan-Lusztig immanants, were able to arrive at a simple formula to compute Kazhdan-Lusztig immanants associated to $2143$ and $1324-$avoiding permutations, deriving from the connections between Kazhdan-Lusztig immanants and Schubert varities. These simple formulas were in terms of another class of immanant, which we refer to as a \%-immanant, which are significantly easier to compute. It is natural then to ask how one might be able to extend this result, in order to more simply compute other Kazhdan-Lusztig immanants, and what other patterns would need to be avoided for such simple formulas to exist, as well as whether these \%-immanants also obey nice positivity properties.
% \par In this report, we describe what happens in the special case of Temperley-Lieb immanants and detail some partial results for the general case of Kazdhan-Lusztig immanants, using more directly combinatorial tools. Our main result is the following theorem, stating when Temperley-Lieb immanants are linear combinations of \%-immanants.
% \begin{Thm}
% The following are equivalent.
% \begin{enumerate}
%     \item $\imm_w$ is a linear combination of \%-immanants;
    
%     \item $\sgn(w) \imm_w$ is a sum of at most two \%-immanants;
    
%     \item $w$ avoids the patterns $321, 24153, 31524, 231564$, and $312645$.
%     \end{enumerate}
% \end{Thm} 
% This theorem, in other words, completely describes which Temperley-Lieb immanants are linear combinations of \%-immanants. With this result, a natural next question to ask is to figure out what conditions would be needed for the general Kazhdan-Lustzig case, starting with two \%-immanants. Our conjecture is as follows. \begin{conjecture}
% The Kazhdan-Lusztig immanant for a permutation $w\in \mathfrak{S}_n,$ $\imm_w,$ can be expressed as a sum or difference of at most two \%-immanants if and only if $w$ avoids the patterns $$1324, 24153, 31524, 231564, 312645, 426153.$$
% \end{conjecture}
% \par The strategy for this report is as follows. In section 2, we first go through the important concepts and introduce the important objects that we'll be studying, and review some preliminary results about them. Then, in section 3, we prove that the Temperley-Lieb immanant of a $321$-avoiding permutation $w$ is a \%-immanant if and only if $w$ avoids the patterns $2143$ and $1324,$ using more combinatorial methods. Although one direction was already known from the work of Chepuri and Sherman-Bennett in \cite{CSB}, the setup here will both evoke the flavor of the method of proof we will use in section 4.
% \par Following this, in section 4 we prove that the Temperley-Lieb immanant of a $321$-avoiding permutation is a sum of two \%-immanants if and only if it avoids the patterns $1324, 21453, 31524, 231564, 312645.$ Over the course of this proof, we will arrive at a relatively simple combinatorial formula for the coefficients of the Temperley-Lieb immanant of $w,$ for $321$ and $1324-$avoiding permutations $w$ that have the pattern $2143.$ We will then finish by showing that the two \%-immanant case is the ``last" case, in that $w$ needs to avoid the same patterns to even be a linear combination of \%-immanants.
% \par In section 5, we try to answer the same questions that we did in section 4, but now in the case when we deal with Kazhdan-Lusztig immanants in general. In particular, we show that certain patterns need to be avoided to be a linear combination of \%-immanants, although it's clear from our work in section 4 that these conditions aren't sufficient. We finish with a partial result in a converse sort of direction, by showing that certain \%-immanants are a sum of a Temperley-Lieb immanant and a Kazhdan-Lusztig immanant. This in turn allows us to make statements about positivity properties of these \%-immanants. We finish the report with some conjectures that would make for natural next steps from the results so far in Section 6.
\section{Preliminaries}\label{sec:prelims}
% \syl{[Overall feedback for section 2: Break into subsections--- one for \% immanants, one for TL algbera stuff, and one for Bruhat order, one for complementary minors, etc. Add transitional paragraph between the subsections.]}


% \syl{Begin with somethine like this: 
% In this section, we review all the prelimnary and notations that will be used throughout the paper........}
\par In this section, we go through the main preliminary concepts that we extensively use throughout this paper. We start with some discussion of permutations and Bruhat order, and then introduce Temperley-Lieb immanants. We finally discuss complementary minors and their relation to Temperley-Lieb immanants.
\subsection{Permutations}
We begin this section by introducing some notation which will help us talk about permutations.

\par For a finite set $I$, let $|A|$ denote the cardinality of $A$. If furthermore we have a set of indices $I \subset [n] := \{1, 2, \ldots, n\},$ let $\overline{I} = [n] - I.$ 
\par Given integers $a, b,$ let $[a, b] = \{a, a+1, \ldots, b\}$ if $a \leq b$ and the empty set otherwise. Furthermore, we let $[a, b: c, d]$ be $[a, b] \cup [c, d],$ and similarly we let $[a_1, a_2:a_3, a_4: a_5, a_6: \ldots: a_{2n-1}, a_{2n}]$ be the union of the $n$ intervals $[a_{2i - 1}, a_{2i}]$ for $i \in [n].$ Let $s(A)$ be the sum of the elements of a set $A \subset \mathbb{R}.$

\par Let $\fkS_n$ denote the group of permutations of length $n$. A permutation $u \in \fkS_n$ can be viewed as a map $[n] \to [n]$. Thus, given a subset $S \subset [n]$, we can define $u(S)$ to be the image of $S$, or $\{ u(x) : x \in S \}$. %Similarly, if $S = (s_1, s_2, \cdots, s_k)$ is an ordered sequence of elements of $[n]$, then $u(S) = (u(s_1), u(s_2), \cdots, u(s_k))$ is also an ordered sequence of elements of $[n]$.

\par Given $v, w \in \fkS_n,$ let $v \cdot w$ to denote the product of $v, w$ in $\fkS_n,$ given by $v\cdot w(i) = v(w(i)).$ We will sometimes drop the $\cdot$ when writing the product for readability. We furthermore let $(i, j)$ denote the transposition swapping $i$ and $j$. It will be clear by context whether $(i, j)$ denotes a transposition or an ordered pair. 
\par We also have the \textbf{longest word} $w_0 \in \fkS_n,$ which is the permutation with one-line notation $n (n-1) (n-2) \cdots 1.$ We will specify which permutation group we are considering the longest word in when it is unclear from context.
\par A large part of this paper is devoted to extracting substructures from a permutation. We define two mechanisms for doing so. First, we have a notion of a block structure for a permutation, which we define as follows.
\begin{definition}A \textbf{block} is a string of consecutive ascending integers. Given two disjoint blocks $[a]$ and $[b],$ we say that $[a] < [b]$ if the largest element in $[a]$ is smaller than the smallest element of $[b].$
\par From here, given a permutation $w \in \mathfrak{S}_n$ and $v \in \mathfrak{S}_m,$ where $m \leq n,$ we say that $w$ has \textbf{block structure} $[v(1)][v(2)]\ldots [v(m)]$ if the one-line notation for $w$ consists of $m$ blocks, where $[1] < [2] < \ldots < [m].$
\end{definition}
For instance, the permutation with one-line notation $56123784$ has block structure $[3][1][4][2].$ Note that a permutation can have multiple block structures (e.g. the same permutation has block structure $[4][1][2][5][3]$) but a unique block structure with the fewest number of blocks.

\par We will sometimes want to be explicit about the values that are present in a given block. To do this, in analogy with the notation $[a, b]$ for the set $\{a, a+1, \ldots, b\},$ we let $(a..b)$ denote the sequence $(a, a+1, \ldots, b),$ and we let $(a_1..a_2: a_3..a_4: \ldots: a_{2n-1}..a_{2n})$ denote the sequence $$(a_1, a_1 + 1, a_1 + 2, \ldots, a_2, a_3, a_3 + 1, \ldots, a_4, \ldots, a_{2n-1}, a_{2n-1} + 1, \ldots, a_{2n}).$$ If $a > b,$ then we let $(a..b)$ denote the empty sequence. The most common place we will use this notation is for writing the one-line notation of a permutation. We can think of the one line notation as a sequence of elements, where the $i$-th element in the sequence is $w(i).$ For instance, we can write the one line notation $56123784$ using our notation as $(5..6:1..3:7..8:4..4).$ 
\par For brevity, given a sequence $(a_1..a_2, a_3..a_4, \ldots, a_{2n-1}..a_{2n})$ and a function $f$ whose values are well-defined on all of the elements in the sequence, we will denote the sequence $$(f(a_1), f(a_1 + 1), \ldots, f(a_2), f(a_3), f(a_3 + 1), \ldots, f(a_4), \ldots, f(a_{2n-1}), f(a_{2n-1} + 1), \ldots, f(a_{2n}))$$ as $f(a_1..a_2:a_3..a_4:\ldots:a_{2n-1}..a_{2n}).$ The most common situation that we will run into during this paper is the situation where the $a_i$ all lie in $[n],$ and $f \in \fkS_n.$

\par Our other mechanism for obtaining substructures comes from the idea of restriction, in analogy with the notion of restricting the domain of functions. Note that the following definition is the same operation as the flattening operation from \cite[\S 3]{billey2003maximal}. 
\begin{definition}
Given a permutation $w \in \fkS_n$ and a set of indices $I \subset [n]$, \textbf{the restricted permutation} $w|_I$ is the permutation on $\fkS_{|I|}$ defined by the following condition: for $1 \le i \le |I|$, $w$ maps the $i$-th smallest element of $I$ to the $w|_I (i)$-th smallest element of $w(I)$.
\end{definition}



\par The restriction of a permutation allows us to naturally define pattern avoidance.

\begin{definition}
    Given a permutation $w \in \fkS_n$ and $v \in \fkS_m$, with $m \le n$, we say that $w$ \textbf{avoids} the pattern $v$ if there does not exist a subset $I \subset [n]$ with $|I| = m$ such that $w|_I = v$.
\end{definition}

For example, the permutation $31524$ avoids $321$ but does not avoid $123$.

Pattern avoidance (and by contrapositive, containing a pattern) is closed under taking inverses and by application of the longest word. This is the subject of the following two lemmas.
\begin{lemma}\label{lem: restriction inverses}
Let $w \in \fkS_n.$ Then, if $v = w|_I \in \fkS_m$ for some subset $I \subset [1, n]$ (which in particular requires $|I| = m$), then $v^{-1} = (w^{-1})|_{w(I)}.$ In particular, $w$ avoids the pattern $v$ if and only if $w^{-1}$ avoids the pattern $v^{-1}$.
\end{lemma}

\begin{lemma}\label{lem: restriction flips}
Let $w \in \fkS_n.$ Then, if $v = w|_I \in \fkS_m$ for some subset $I \subset [1, n],$ and $w_0 \in \fkS_n, w_0' \in \fkS_m$ are the longest words in their respective permutation groups, then $w_0'v = (w_0w)|_I$ and $vw_0' = (ww_0)|_{w_0(I)}$.
\end{lemma}

Applying the above lemma twice yields the following corollary.
\begin{cor}\label{cor: w_0 conjugation pattern avoidance}
Let $w \in \fkS_n.$ Then, if $v = w|_I \in \fkS_m$ for some subset $I \subset [1, n],$ and $w_0 \in \fkS_n, w_0' \in \fkS_m$ are the longest words in their respective permutation groups, then $w_0'vw_0' = (w_0ww_0)|_{w_0(I)}.$ In particular, $w$ avoids the pattern $v$ if and only if $w_0 w w_0$ avoids the pattern $w_0' v w_0'$.
\end{cor}
We note that for this paper, we will only need Lemma \ref{lem: restriction inverses} and Corollary \ref{cor: w_0 conjugation pattern avoidance}, and we will be mainly applying these to a handful of values of $v$ (namely, $321$ and the five patterns that appear in the statement of the main theorem). %In our next paper, when we deal with the case of Kazhdan-Lusztig immanants, these results, including the stronger statement of Lemma \ref{lem: restriction flips}, will be used more extensively.
\subsection{Bruhat order}
\par We now define the Bruhat order on $\fkS_n$. See \cite{bjorner2006combinatorics} for a detailed reference.

\begin{definition}\label{defn: Bruhat Order}
A \textbf{reduced word} for an element $w \in \fkS_n$ is a decomposition of $w$ into the simple reflections $s_1, s_2, \ldots s_{n-1}$ of $\fkS_n,$ where $s_i$ is the transposition $(i, i+1),$ that is minimal in length among all such decompositions.
\par The length of a permutation $\ell(u)$ is the length of a reduced word for $u$.
\par The \textbf{Bruhat order} on $\fkS_n$ is defined as follows. We say that $u \le v$ if one of the following three equivalent definitions are satisfied:
\begin{enumerate}
    \item \label{bruhat1} Some reduced word for $v$ contains a subword equal to $u$.

    \item \label{bruhat2} There exists a sequence $u = u_1, u_2, \cdots, u_k = v$ such that for each $1 \le i \le k-1$, we can express $u_i^{-1} u_{i+1}$ as a transposition $(a_i, b_i)$ and we also have $\ell(u_{i+1}) = \ell(u_i) + 1$.

    \item \label{bruhat3} For all $1 \le i, j \le n$, we have
$|u([1,i]) \cap [1,j]| \ge |v([1,i]) \cap [1,j]|$.
\end{enumerate}
Using equivalent definition 2, it follows that the Bruhat order is graded by the length function: if $u \le v$, then $\ell(u) \le \ell(v)$. The minimal element of the Bruhat order is the identity permutation, and the maximal element is the longest word $w_0$ in $\fkS_n,$ which has length $\ell(w_0) = \frac{n(n-1)}{2}.$
\end{definition}

We remark that a fourth alternate definition for Bruhat order is given as Theorem 2.1.5 in \cite{bjorner2006combinatorics}. 
\par For example, note that $1423 < 2431,$ since we have the sequence $1423, 2413, 2431,$ with each adjacent pair satisfying the condition stated in definition \ref{bruhat2} above. %Note also that $w_0$ is the largest element in the Bruhat order.
\par In a few places, we wish to compare two permutations that differ on a prescribed set of indices. The following lemma makes this possible.

\begin{lemma}\label{lem:restriction}
Let $v, w \in \fkS_n$. If $I \supset \{ i \mid v(i) \neq w(i) \}$ and $w|_I \le v|_I$, then $w \le v$.
\end{lemma}

We note that one can obtain this lemma by repeatedly using \cite[Lemma 17]{billey2003maximal}. For the sake of completeness, we provide a proof at the end of this section.

\begin{comment}
\begin{proof}
We employ the third equivalent definition of the Bruhat order. Our goal is to show that $|u([1,i]) \cap [1,j]| \ge |v([1,i]) \cap [1,j]|$ for positive integers $i, j$ so $i, j \leq n.$ Clearly $u([1,i] \backslash I) \cap [1,j] = v([1,i] \backslash I) \cap [1,j]$, and $|u([1,i] \cap I) \cap [1,j]| \ge |v([1,i] \cap I) \cap [1,j]|$ follows by applying this third definition of Bruhat order to $u|_I$ and $v|_I$.
\end{proof}

\end{comment}

The following corollary is an immediate consequence of Lemma \ref{lem:restriction}.
\begin{cor}\label{cor:bruhat_inv}
If $(i, j)$ is an inversion of $w$, then $w \cdot (i,j) < w$.
\end{cor}


\begin{remark}
We are tempted to conclude that $w\tau$ has exactly one less inversion than $w$ (i.e. that $w$ covers $w\tau$), but this is not true: if $w$ has one-line notation $321,$ then $(1, 3)$ an inversion of $w,$ but $w \cdot (1, 3)$ is the identity, which is not covered by $w.$ However, if $w$ is $321$-avoiding, we can easily show that $w$ must in fact cover $w\tau$.
\end{remark}

% In particular, for some values of $I$ in Lemma \ref{lem:restriction}, we can actually say that this set $I$ holds for everything within the Bruhat interval $[v, w].$

% \begin{lemma}\label{lem:iso_intervals}
% Given $v, w \in \fkS_n, $ let $I \supset \{i|v(i) \neq w(i)\},$ and suppose that $w|_I \leq v|_I.$ Let $J = v(I) = w(I),$ and suppose that $v([\min(I), \max(I)]) \cap [\min(J), \max(J)] = J.$ Then, $u \in [w, v] \iff u|_I \in [w|_I, v|_I]$ and $u(x) = v(x) = w(x)$ for all $x \not \in I.$
% \end{lemma}
\begin{comment}
\begin{proof}
The if direction holds even without the restrictions on $I.$ Indeed, suppose that $w|_I \leq u|_I \leq v|_I,$ and $u(x) = v(x) = w(x)$ for all $x \not \in I.$ Then, notice that the set $I' = \{i|u(i) \neq w(i)\} \subset I$ by assumption. But then, by Lemma \ref{lem:restriction}, we see that $w \leq u,$ and similarly we can use this argument to argue that $u \leq v.$
\par We now show the other direction. The main difficulty is showing that $u(x) = v(x) = w(x),$ for all $x \not \in I.$ To prove this, suppose for the sake of contradiction that this wasn't the case. Then, there exists some $x$ so that $u(x) \neq v(x) = w(x),$ where $x \not \in I.$ 



\par From here, define $x_0, B$ as follows:
\begin{enumerate}
\item If $u(x) \neq v(x)$ for some $x < \min(I),$ let $x_0$ be the smallest such $x,$ and let $B = \min(u(x), v(x)).$
\item Otherwise, if $u(x) \neq v(x)$ for some $x > \max(I),$ let $x_0$ be the largest such $x,$ and let $B$ again be $\min(u(x), v(x)).$
\item Otherwise, $u = v = w$ outside of $[\min(I), \max(I)],$ so $u([\min(I), \max(I)]) \cap [\min(J), \max(J)] = v([\min(I), \max(I)]) \cap [\min(J), \max(J)] = J.$ Then, there must exist some $x$ where either $x \in I$ and $u(x) \not \in J,$ or $x \not \in I$ and $u(x) \neq v(x) = w(x).$ Let $x_0$ be the smallest $x$ for which either of these is the case. 
\par Furthermore, if $x_0 \in I,$ let $B = u(x)$ if $u(x) < \min(J),$ and $u(x) - 1$ otherwise. Else, $x_0 \not \in I;$ let $B = \min(u(x), v(x))$ if $v(x) < \min(J)$ and $\max(u(x), v(x)) - 1$ otherwise. Notice that $u([\min(I), \max(I)]) \cap [\min(J), \max(J)] = v([\min(I), \max(I)]) \cap [\min(J), \max(J)] = J$ implies that these cases are the only possible cases.
\end{enumerate}
First, we claim that $$|v([1,x_0 - 1]) \cap  [1,B]| = |w([1,x_0 - 1]) \cap  [1,B]|.$$ To prove this, notice that in the first case, $v = w$ on $[1, x_0-1]$ and in the second case $v = w$ on $[x_0, n].$ In the third case, notice that $B < \min(J)$ or $B \geq \max(J),$ meaning that either way, for each $i \in [1, x_0-1],$ either $i \not \in I,$ so $v(i) = w(i),$ or both $v(i), w(i)$ lie in $J$ (so either both lie in $[1, B]$ or neither do). Similarly, we see that $$|v([1,x_0]) \cap  [1,B]| = |w([1,x_0]) \cap  [1,B]|,$$ as either $v(x_0) = w(x_0),$ or $x_0 \in I,$ so $B$ is either disjoint from $J$ or contains $J$ (but $v(x_0), w(x_0)$ both lie in $J$).
\par Therefore, by the definition of Bruhat order we have $$|u([1,x_0 - 1]) \cap  [1,B]| = |v([1,x_0 - 1]) \cap  [1,B]| = |w([1,x_0 - 1]) \cap  [1,B]|$$ and $$|u([1,x_0]) \cap  [1,B]| = |v([1,x_0]) \cap  [1,B]| = |w([1,x_0]) \cap  [1,B]|.$$ Now, notice that in each case, only one of $u(x), v(x)$ lies in $[1, B]$ by construction, implying that $|u([1,x_0 - 1]) \cap  [1,B]| = |v([1,x_0 - 1]) \cap  [1,B]|$ and $|u([1,x_0]) \cap  [1,B]| = |v([1,x_0]) \cap  [1,B]|$ cannot both simultaneously hold, which is a contradiction.
% \begin{enumerate}
%     \item Suppose that there exists an $x$ so $x < \min(I).$ Pick the smallest such $x,$ and consider the quantities $|u([1,x]) \cap [1,\min(u(x), v(x))]|,$ $|v([1,x]) \cap  [1,\min(u(x), v(x))]|,$ and $|w([1,x]) \cap  [1,\min(u(x), v(x))]|$. Since $v, w$ by assumption agree on $[1, x],$ we have that $$|v([1,x]) \cap  [1,\min(u(x), v(x))]| = |w([1,x]) \cap  [1,\min(u(x), v(x))]|,$$ which by Definition \ref{defn: Bruhat Order} means that $$|u([1,x]) \cap [1,\min(u(x), v(x))]| = |v([1,x]) \cap  [1,\min(u(x), v(x))]| = |w([1,x]) \cap  [1,\min(u(x), v(x))]|.$$ 
%     \par By assumption of minimality, we have that $$|u([1,x - 1]) \cap [1,\min(u(x), v(x))]| = |v([1,x - 1]) \cap  [1,\min(u(x), v(x))]| = |w([1,x - 1]) \cap  [1,\min(u(x), v(x))]|.$$ However, only one of $u(x), v(x)$ lies in $[1,\min(u(x), v(x))],$ meaning that $$|u([1,x]) \cap [1,\min(u(x), v(x))]| = |v([1,x]) \cap  [1,\min(u(x), v(x))]| = |w([1,x]) \cap  [1,\min(u(x), v(x))]|$$ cannot hold, contradiction.
%     \item Next, suppose that there exists an $x$ so that $x > \max(I).$ Pick the largest such $x.$ Like in the above case, we see that $|u([1,x - 1]) \cap [1,\min(u(x), v(x))]|,$ $|v([1,x - 1]) \cap  [1,\min(u(x), v(x))]|,$ and $|w([1,x - 1]) \cap  [1,\min(u(x), v(x))]|$ are equal. Note that $$|u([1,x - 1]) \cap [1,\min(u(x), v(x))]| = \min(u(x), v(x)) - |u([x, n]) \cap [1,\min(u(x), v(x))]|,$$ and similarly for $v, w.$ Thus, we have as $|u([x, n]) \cap [1,\min(u(x), v(x))]|,$ $|v([x, n]) \cap [1,\min(u(x), v(x))]|,$ and $|w([x, n]) \cap [1,\min(u(x), v(x))]|$ equal.
%     \par Similarly to the first case, by maximality, $$|u([x + 1, n]) \cap [1,\min(u(x), v(x))]| = |v([x + 1, n]) \cap [1,\min(u(x), v(x))]| = |w([x + 1, n]) \cap [1,\min(u(x), v(x))]|,$$ but $u(x) \neq v(x)$ makes this impossible.
%     \item Suppose now that $u = v = w$ outside of $[\min(I), \max(I)],$ so $u([\min(I), \max(I)]) \cap [\min(J), \max(J)] = v([\min(I), \max(I)]) \cap [\min(J), \max(J)] = J.$ Observe that there must exist some $x$ where either $x \in I$ and $u(x) \not \in J,$ or $x \not \in I$ and $u(x) \neq v(x) = w(x).$ Pick the smallest $x$ for which either of these is the case. We first assume in this case that $x \in I, u(x) \not \in J.$
%     \par Notice that $u([\min(I), \max(I)]) \cap [\min(J), \max(J)] = J$ means that $u(x) < \min(J)$ or $u(x) > \max(J).$ Let $B = u(x)$ if $u(x) < \min(J)$ and $u(x) - 1$ if $u(x) > \max(J).$ Consider now $|u([1, x]) \cap [1,B]|,$ $|v([1, x]) \cap [1,B]|,$ and $|w([1, x]) \cap [1,B]|.$
%     \par Notice that the latter two are equal, since $[1, B]$ either contains $J$ or is disjoint from $J,$ and for each $i \in [1, n],$ either $v(i) = w(i),$ or $i \in I,$ so either both $v(i), w(i)$ lie in $J$ or both don't lie in $J.$ 
%     \par We proceed like in case 1: this implies that all three have to be equal. Furthermore, by minimality we have $$|u([1, x - 1]) \cap [1,B]| = |v([1, x - 1]) \cap [1,B]| = |w([1, x - 1]) \cap [1,B]|,$$ as for each $y \in [1, x-1],$ either $u(y) \in J$ or $u(y) = v(y) = w(y).$ But notice that either $u(x)$ lies in $[1, B],$ or both of $v(x), w(x)$ do (but not at the same time), meaning that $|u([1, x]) \cap [1,B]|,$ $|v([1, x]) \cap [1,B]|,$ and $|w([1, x]) \cap [1,B]|$ cannot all be equal, which again is a contradiction.
%     \item Finally, suppose that this $x$ doesn't lie in $I,$ and $u(x) \neq v(x) = w(x).$ Let $B = \max(v(x), u(x)) - 1$ if $v(x) > \max(J)$ and $\min(v(x), u(x))$ if $v(x) < \min(J).$ Recall that by assumption $v(I) = J,$ so $x \not \in I$ and our assumption that $v([\min(I), \max(I)]) \cap [\min(J), \max(J)] = J$ implies that these are the only two options. Now, consider the values $|u([1, x]) \cap [1,B]|,$ $|v([1, x]) \cap [1,B]|,$ and $|w([1, x]) \cap [1,B]|.$ 
%     \par Again, we see that either $[1, B]$ contains $J$ or is completely disjoint, meaning that for each $y \in [1, x],$ either both $w(y), v(y)$ lie in the set, or neither do, showing that $$|v([1, x]) \cap [1,B]| = |w([1, x]) \cap [1,B]|;$$ we must then have that  $|u([1, x]) \cap [1,B]|$ equals these two values too. Furthermore, by assumption of minimality, we have that $|u([1, x - 1]) \cap [1,B]|,$ $|v([1, x - 1]) \cap [1,B]|,$ and $|w([1, x - 1]) \cap [1,B]|$ are all equal. However, notice that only one of $u(x), v(x)$ can lie in $[1, B]$ by construction, meaning that $|u([1, x]) \cap [1,B]|,$ $|v([1, x]) \cap [1,B]|,$ and $|w([1, x]) \cap [1,B]|$ are not all equal, contradiction. 
%     \end{enumerate}
\par We now see that $u \in [w, v]$ means that $u(x) = v(x) = w(x)$ for all $x \not \in I.$ To prove that $u|_I \in [w|_I, v|_I],$ consider a sequence of inversions to take us from $w$ to $u$ that is increasing in number of inversions, which exists by definition of $w \leq u.$ This yields us with a sequence of permutations $w, u_1, u_2, \ldots, u_k, u.$ By our argument above, each of these $u_i$ agrees with $v$ and $w$ on the complement of $I.$ But then notice that each of the inversions we perform are only inversions among pairs of elements in $I.$
\par Suppose this sequence of inversions was $(i_1, j_1), (i_2, j_2), \ldots, (i_{k+1}, j_{k+1}).$ Applying the map $f$ that sends the $i$-th smallest element of $I$ to $i$ yields a new sequence of inversions, which applied to $w_I$ necessarily yield $u_I.$ Indeed, after one inversion, notice that $u \cdot (i_1, j_1)$ sends $j_1$ to $u(i_1)$ and $i_1$ to $u(j_1),$ and agrees with $u$ everywhere else. Furthermore, $u_I \cdot (f(i_1), f(j_1))$ sends $f(j_1)$ to $u_I(f(i_1)),$ sends $f(i_1)$ to $u_I(f(j_1)),$ and fixes the rest. But this is precisely what $\left(u \cdot (i_1, j_1)\right)|_I$ does. A similar argument can be applied to $u, v,$ giving us the desired inequality $w|_I \leq u|_I \leq v|_I.$
\end{proof}

\end{comment}

% Finally, we say that a permutation that avoids $321$ and $1324$ is called nice. We will see why such permutations are nice in the next two sections. \todo{Remove this and remove places where we refer to nice permutations.}
% \syl{This probably doesn't need to be put in a definition environment. Also maybe move it to a more noticeable place?}
\subsection{The Temperley-Lieb Algebra and Non-Crossing Matchings}
In this subsection, we define the Temperley-Lieb algebra and discuss a few basic properties of the algebra, which we will then use to define the coefficients of Temperley-Lieb immanants. 

% We first define a particular collection of elements in this group algebra.
% \begin{definition}
% Define $z_{i, j} \in \mathbb{C}[\mathfrak{S}_n]$ to be $\sum\limits_{v} v,$ over all permutations $v$ that fix elements not in the set $[i, j].$ For instance, $z_{1, 3} = (1 + s_1 + s_2 + s_1s_2 + s_2s_1 + s_1s_2s_1).$ 
% \end{definition}

\begin{definition}(See \cite[\S 2]{LPP} and \cite[\S 3]{RS})
The \textbf{Temperley-Lieb algebra}, which we denote as $TL_n(2),$ is the $\mathbb{C}-$algebra generated by elements $t_1, t_2, \ldots, t_{n-1}$ with the relations $t_i^2 = 2t_i$ for $i \in [n],$ $t_it_j = t_jt_i$ for $i, j \in [n]$ where $|i - j| \geq 2,$ and $t_it_jt_i = t_i$ for $i, j \in [n]$ where $|i - j| = 1.$
\end{definition}
Notice that the Temperley-Lieb algebra in general depends on a parameter, but throughout this paper we specialize to the case when the parameter equals $2.$ The following result is standard. See \cite[Section 3]{RS}, for instance.
% \syl{Tell the reader that we are using a special case of TL algebra with parameter $=2$.}

\begin{prop}\label{prop:tl}
The $\C$-linear map $\theta: \C(\mathfrak{S}_n) \to TL_n(2)$ defined by $\theta(s_i) = t_i-1$ is a $\C$-algebra homomorphism. The $\C$-linear map $\beta:$ $\C$-span of 321-avoiding permutations in $\fkS_n$ $\to TL_n(2)$ defined by $\beta(s_{i_1} s_{i_2} \cdots s_{i_k}) = t_{i_1} t_{i_2} \cdots t_{i_k}$ for any reduced word $s_{i_1} s_{i_2} \cdots s_{i_k}$ of a $321$-avoiding permutation in $\fkS_n$ is a well-defined vector space isomorphism.
\end{prop}

We use the elements of our Temperley-Lieb algebra in order to construct the first of our objects of interest, the Temperley-Lieb immanant.

\begin{definition}\label{syldef}
Given a permutation $u \in \mathfrak{S}_n,$ let $x_u$ be the monomial $\prod\limits_{i=1}^n x_{i, v(i)},$ living in the set of polynomials on $n^2$ variables $x_{i, j}$ for $1 \leq i, j \leq n.$ If we are furthermore given a 321-avoiding permutation $w \in \fkS_n,$ let $f_w (u)$ be the coefficient of $\beta(w)$ in $\theta(u)$. 
\par From here, for each 321-avoiding permutation $w \in \fkS_n,$ define the \textbf{Temperley-Lieb immanant} of $w$ by Imm$_{w} = \sum_{u \in \mathfrak{S}_n} f_w(u)x_u.$ This is the 
\end{definition}

A lot of the coefficients $f_w (u)$ can be easily computed (and, in particular, a lot are zero), which can be characterized by the following lemma.

\begin{lemma}[Proposition 3.7 in \cite{RS}]\label{lem:basic fwu}
Let $w, u \in \fkS_n$, with $w$ being $321$-avoiding.
\begin{enumerate}
    \item If $u \not\ge w$, then $f_w (u) = 0$.
    
    \item $f_w (w) = 1$.
\end{enumerate}
\end{lemma}
\begin{comment}

\begin{proof}
Let $w = s_{i_1} s_{i_2} \cdots s_{i_k}$ and $u = s_{j_1} s_{j_2} \cdots s_{j_m}$ be reduced words. Consider expanding $\theta(u) = (t_{j_1} - 1)(t_{j_2} - 1) \cdots (t_{j_m} - 1)$ and simplifying each monomial using the Temperley-Lieb algebra relations. The resulting monomial is a subword of the original monomial, which turn is a subword of $t_{j_1} t_{j_2} \cdots t_{j_m}$. Thus, we can only have a $t_{i_1} t_{i_2} \cdots t_{i_k}$ term in the expansion if $s_{i_1} s_{i_2} \cdots s_{i_k}$ is a subword of $s_{j_1} s_{j_2} \cdots s_{j_m}$. Thus, if $f_w (u) \neq 0$, then $u \ge w$, proving part 1 of the Lemma. For part 2, we take $w = u$, and observe that when expanding $(t_{i_1} - 1)(t_{i_2} - 1) \cdots (t_{i_k} - 1)$, the only way to get a monomial of length $k$ is to take the first term of each binomial, and thus the coefficient of $t_{i_1} t_{i_2} \cdots t_{i_k}$ is one.
\end{proof}
\end{comment}

We can relate the coefficients $f_w(u)$ to colorings of a non-crossing matching, which we define below.

\begin{definition}\label{def:ncm}(See \cite[\S 2]{LPP})
A \textbf{non-crossing matching} of the integers in $[2n]$ is a perfect matching of these integers so that there doesn't exist $1 \le a < b < c < d \le 2n$ such that $a$ is paired with $c$ and $b$ is paired with $d$ in the matching.
\par For convenience, we sometimes denote the indices $n+1, n+2, \ldots, 2n$ in a non-crossing matching by $n', (n-1)', \ldots, 1',$ respectively. Also, let $[n]' = \{1', 2', \ldots, n'\}$ and $[a, b]' = \{a', (a+1)', \ldots, b'\}.$
\end{definition}

\begin{remark}
(1) The non-crossing aspect of the condition comes from the following interpretation: if we arrange the integers $1$ through $2n$ in that order around a circle, and draw a chord between paired vertices, then no two chords may intersect.

(2) If $v_1$ and $v_2$ are paired in a non-crossing matching, then there are an even number of vertices between $v_1$ and $v_2$. Thus, the difference of labels $v_2 - v_1$ must be odd. This parity consideration will be used extensively in this paper. \textbf{Warning:} in primed/unprimed notation, where $i' = 2n+1-i$, the difference of labels (when given labels in $[2n]$) between a primed vertex $j'$ and an unprimed vertex $i$ is $2n+1-j-i$, so $j-i$ must be even.
\end{remark}

%The following proposition is also standard, see \cite[Section 3]{RS} and \cite{HKAUFFMAN1987395}.

There is a natural bijection between $321$-avoiding permutations of $[n]$ and non-crossing matchings of $2n$ vertices (see \cite[Section 3]{RS}). We summarize the bijection here:

\begin{prop}\label{prop:ncm}
There is a bijection, which we denote as $\ncm,$ between $321$-avoiding permutations of $[n]$ and non-crossing matchings of $2n$ vertices. Explicitly, for any $321$-avoiding permutation $w$, we construct a wiring diagram $D(w)$ corresponding to a reduced word for $w$, as in \cite{RS}. Fix a coordinate system such that each vertex $i$ is located at $(-1, n-i)$ and vertex $i'$ is located at $(1, n-i)$.

\begin{enumerate}[(a)]
\item For every $i$, the unique shortest path $p_i$ from $i$ to $w(i)'$ along the edges of $D(w)$ has either non-increasing, non-decreasing, or constant $y$-coordinate as a function of $x$, depending on whether $w(i) > i$, $w(i) < i$, or $w(i) = i$, respectively. Furthermore, two paths $p_i$ and $p_j$ intersect in at most one point. Let $S$ be the set of intersection points of some two $p_i, p_j$, $1 \le i < j \le n$.

    \item For each vertex $v$ (one of $i$ or $i'$ for some $i$) of the wiring diagram, there is a unique path $q_v$ along the edges of $D(w)$ starting at $v$ and terminating at another vertex (which we call $f(v)$), with the following properties:
\begin{itemize}
    \item the $y$-coordinate of $q_v$ is either non-increasing or non-decreasing as a function of path coordinate;
    
    \item for each $1 \le i \le n$, no positive length subsegment of the path $p_i$ intersects the path $q_v$ at exactly a single point. In other words, the path $q_v$ ``turns'' when it reaches a point of $S$.
\end{itemize}
Then the map $f$ induces a non-crossing matching of $2n$ vertices, which we call $\ncm(w)$.
\end{enumerate}
%construct a reduced wiring diagram (corresponding to a reduced word for $w$). We may ``undo'' each crossing by replacing the segments $a - (a+1)'$ and $a+1 - a'$ with the curves $a - (a+1)$ and $a' - (a+1)'$. If we discard all loops, then we obtain a non-crossing matching, which is independent of the choice of reduced wiring diagram for $w$. %(See, for instance, the beginning of section 3 of \cite{RS} to get from $s_i$ to $t_i$ representation, and then \cite{HKAUFFMAN1987395} to get to the non-crossing matching).
\end{prop}

\begin{remark}
We abuse notation slightly and let the pairs of a $321$-avoiding permutation $w$ be the pairs that occur in the corresponding non-crossing matching $\ncm(w)$, per the bijection established by Proposition \ref{prop:ncm}.
\end{remark}

% As a corollary, we obtain the following result.
% \begin{lemma}\label{wireTBLemma}
% Suppose we have a permutation $\sigma$ on $[n]$ that fixes all the elements, other than those in the set of integers between $a$ and $b,$ inclusive. Furthermore, suppose that for some $x$ such that $a < x \leq b$ and that $\sigma(x) < \sigma(x+1) < \ldots < \sigma(b) < \sigma(a) < \ldots < \sigma(x - 1)$ are consecutive integers. 
% \par Then, $x + i$ is paired with $x - i - 1,$ and $(\sigma(b) - i)'$ is paired with $(\sigma(b) + i + 1)',$ for integers $i$ so that $0 \leq i \leq \min(x - a - 1, b - x).$ Of the remaining elements within $[a, b]$, the $i$th smallest unpaired element that is unprimed is paired with the $i$th smallest unpaired element that is primed. For $i \notin [a, b]$, $i$ is paired with $i'.$
% \end{lemma}

% As an example, we can see a permutation where we have the permutation that has one line notation $145623.$ Here, we can draw the permutation as follows.
% %This takes a while to load. If possible run the code locally and then attach the resulting output as an image.
% \begin{center}
% \includegraphics[scale = 0.7]{output 3.pdf}
% \end{center}

% We apply Lemma \ref{wireTBLemma} with $a = 2, x = 5, b = 6$. This permutation has the following corresponding perfect matching.

% \begin{center}
% \includegraphics[scale = 0.7]{output 4.pdf}
% \end{center}

% \begin{remark}
% Any crossing in the wiring diagram of a $321$-avoiding permutation corresponds to a $s_i$, which is mapped to a $t_i$ by the map $\beta$. This finite process produces a non-crossing permutation that is well-defined. Furthermore, as noted in \cite{RS}, the correspondence given by replacing $s_i$ with $t_i$ is a bijection between a basis of $TL_n(2)$ and the set of $321$-avoiding permutations. This process  bijectively maps reduced decompositions to each other.
% \end{remark}

For instance, the left figure below is the wiring diagram for the permutation $2341 = s_1 s_2 s_3$ following the convention of \cite{RS}. By following the procedure in Proposition \ref{prop:ncm}, we obtain the non-crossing matching corresponding to $2341$ in the right figure below.

\begin{center}
% \begin{asy}
%     import graph;
%     void Swap(int n, int m, int[] swaps) {
%     for (int i = 0; i < n; ++i){
%         string lab = (string) (i + 1);
%         filldraw(Circle((0,-25*i),2),black,black);
%         label(lab, (0, -25*i), W);
%         filldraw(Circle((50*m,-25*i),2),black,black);
%         label(lab + "'", (50*m, -25*i), E);
%     }
%     int j = 0;
%     for (int z : swaps) {
%         j = j + 50;
%         for (int i = 0; i < z; ++i){
%         draw((j - 50, -25*i)--(j, -25*i));
%         }
%         draw((j - 50, -25*z)--(j, -25*(z+1)));
%         draw((j - 50, -25*(z+1))--(j, -25*z));
%         for (int i = z + 2; i < n; ++i){
%         draw((j - 50, -25*i)--(j, -25*i));
%         }
%     }
%     }
%     int[] array = {2, 1, 0};
%     Swap(4, 3, array); 
% \end{asy}
\includegraphics[]{output29.pdf}
\hspace{2cm}
\includegraphics[]{output1.pdf}
\end{center}

\begin{comment}
\begin{center}
    \begin{asy}
    import graph;
    draw((0, 50)..(50, 0));
    draw((0, 25)..(50, -25));
    draw((0, 0)..(13, -13)..(0, -25));
    draw((50, 50)..(37, 37)..(50, 25));
    filldraw(Circle((0,0),2),black,black);
    label("3", (0, 0), W);
    filldraw(Circle((0,25),2),black,black);
    label("2", (0, 25), W);
    filldraw(Circle((0,50),2),black,black);
    label("1", (0, 50), W);
    filldraw(Circle((0,-25),2),black,black);
    label("4", (0, -25), W);
    filldraw(Circle((50,0),2),black,black);
    label("3'", (50, 0), E);
    filldraw(Circle((50,25),2),black,black);
    label("2'", (50, 25), E);
    filldraw(Circle((50,50),2),black,black);
    label("1'", (50, 50), E);
    filldraw(Circle((50,-25),2),black,black);
    label("4'", (50, -25), E);
\end{asy}
\end{center}

\begin{center}
    \begin{asy}
    import graph;
    void PermDraw(int n, pair[] match) {
    for (int i = 0; i < n; ++i){
        string lab = (string) (i + 1);
        filldraw(Circle((0,-25*i),2),black,black);
        label(lab, (0, -25*i), W);
        filldraw(Circle((50,-25*i),2),black,black);
        label(lab + "'", (50, -25*i), E);
    }
    for (pair z : match) {
        draw((0, -25*z.x)--(50, -25*z.y));
    }
    }
    pair[] array = {(0, 1), (1, 2), (2, 3), (3, 0)};
   PermDraw(4, array); 
\end{asy}
\end{center}
\end{comment}


\subsection{Complementary Minors and Colorings}
Temperley-Lieb immanants are closely related to complementary minors, which in turn can be represented by a coloring. We first describe colorings and how they are associated with complementary minors. 
\par For the purposes of this paper, a coloring of the vertices in $[n] \cup [n]' = \{1, 2, \ldots, n, 1', 2', \ldots, n'\}$ will always be a coloring of these vertices using the colors black and white.
\begin{definition}
A coloring of vertices $[n] \cup [n]'$ is \textbf{compatible} with a non-crossing matching if every black vertex is paired with a white vertex.
\par We also say that a coloring of vertices $[n] \cup [n]'$ is compatible with a permutation $w$ if the coloring is compatible with $\ncm(w).$
\end{definition}

\par In the below matching, for instance, the following is a compatible coloring.

\begin{center}
\includegraphics[]{output2.pdf}
\end{center}

\par We can represent any coloring by $(I, J)$, where $I \subset [n]$ is the set of unprimed black vertices, and $J' \subset [n]'$ is the set of primed white vertices, and thus say that a permutation $w$ is compatible with this pair $(I, J)$ of subsets if it is compatible with the corresponding coloring.
\par A coloring compatible with some non-crossing matching must have $n$ black and $n$ white vertices, which means $|I| = |J|$. As a result, we can associate the coloring $(I, J)$ to the complementary minor $\Delta_{I,J} \Delta_{\bar{I}, \bar{J}}$, defined below.

\begin{definition}
Consider the $n\times n$ matrix $(x_{i,j}),$ the matrix whose $(i, j)$ entry is the variable $x_{i, j}.$ For $I, J \subset [n]$ with $|I| = |J|$, we let $\Delta_{I,J}$ denote the determinant of the minor of this matrix with rows indexed by $I$ and columns indexed by $J$. Then the \textbf{complementary minor} of $I, J$ is $\Delta_{I,J} \Delta_{\bar{I}, \bar{J}}$.
\end{definition}
\par A key result we will use in this paper is a way to express a complementary minor as the sum of the Temperley-Lieb immanants compatible with the associated coloring.
\begin{prop}\cite[Proposition 4.3]{RS}\label{prop:cm equals imm sum}
Let $I,J \subseteq [n]$ with $|I| = |J|$, and consider the coloring $(I, J)$ (i.e. color $I, \bar{J}'$ black and $J', \bar{I}$ white). Then
\begin{equation}\label{cm equals imm sum}
    \Delta_{I,J}\Delta_{\bar{I}, \bar{J}} = \sum_{\substack{w \mathrm{\ compatible}\\ \mathrm{with\ } (I, J)}} \imm_w
\end{equation}
\end{prop}

\begin{remark}
Using the proof of Proposition 4.7 in \cite{RS}, if we treat the equations in \eqref{cm equals imm sum} for all $I, J \subset [n]$ as an (over-determined) linear system in the variables $\imm_w$, we see that the system has a unique solution. In fact, the idea of ``solving'' for $\imm_w$ as a linear combination of certain complementary minors will play a key role in this paper. However, while we know that complementary minors uniquely determine our Temperley-Lieb immanants, it is not entirely clear how to solve this system in general.
\end{remark}

\begin{example}
For $I = \{ 1 \}$ and $J = \{ 1 \}$, we have:
\begin{equation*}
        x_{1,1} \begin{vmatrix}
                x_{2,2} & x_{2,3} \\
                x_{3,2} & x_{3,3}
        \end{vmatrix} = \imm_{123} + \imm_{213}
    \end{equation*}
\begin{center}
    \includegraphics[]{output6.pdf}
    \hspace{2cm}
    \includegraphics[]{output7.pdf}
\end{center}
Note that the left non-crossing matching corresponds to the permuation $123$ and the second corresponds to $213.$
\end{example}

\begin{comment}
\begin{center}
    \begin{asy}
    import graph;
    draw((0, 0)..(50, 50));
    draw((0, 25)..(50, 75));
    draw((0, 50)..(13, 63)..(0, 75));
    draw((50, 0)..(37, 12)..(50, 25));
    filldraw(Circle((0,75),2),black,black);
    label("1", (0, 75), W);
    filldraw(Circle((0,50),2),white,black);
    label("2", (0, 50), W);
    filldraw(Circle((0,25),2),white,black);
    label("3", (0, 25), W);
    filldraw(Circle((0,0),2),white,black);
    label("4", (0, 0), W);
    filldraw(Circle((50,75),2),black,black);
    label("1'", (50, 75), E);
    filldraw(Circle((50,50),2),black,black);
    label("2'", (50, 50), E);
    filldraw(Circle((50,25),2),black,black);
    label("3'", (50, 25), E);
    filldraw(Circle((50,0),2),white,black);
    label("4'", (50, 0), E);
\end{asy}
\begin{asy}
    import graph;
    draw((0, 0)..(50, 50));
    draw((0, 25)..(50, 75));
    draw((0, 50)..(13, 63)..(0, 75));
    draw((50, 0)..(37, 12)..(50, 25));
    filldraw(Circle((0,75),2),black,black);
    label("1", (0, 75), W);
    filldraw(Circle((0,50),2),white,black);
    label("2", (0, 50), W);
    filldraw(Circle((0,25),2),black,black);
    label("3", (0, 25), W);
    filldraw(Circle((0,0),2),black,black);
    label("4", (0, 0), W);
    filldraw(Circle((50,75),2),white,black);
    label("1'", (50, 75), E);
    filldraw(Circle((50,50),2),white,black);
    label("2'", (50, 50), E);
    filldraw(Circle((50,25),2),white,black);
    label("3'", (50, 25), E);
    filldraw(Circle((50,0),2),black,black);
    label("4'", (50, 0), E);
\end{asy}
\begin{asy}
    import graph;
    draw((0, 75)..(50, 75));
    draw((0, 0)..(50, 50));
    draw((0, 25)..(13, 37)..(0, 50));
    draw((50, 0)..(37, 12)..(50, 25));
    filldraw(Circle((0,75),2),black,black);
    label("1", (0, 75), W);
    filldraw(Circle((0,50),2),white,black);
    label("2", (0, 50), W);
    filldraw(Circle((0,25),2),black,black);
    label("3", (0, 25), W);
    filldraw(Circle((0,0),2),black,black);
    label("4", (0, 0), W);
    filldraw(Circle((50,75),2),white,black);
    label("1'", (50, 75), E);
    filldraw(Circle((50,50),2),white,black);
    label("2'", (50, 50), E);
    filldraw(Circle((50,25),2),white,black);
    label("3'", (50, 25), E);
    filldraw(Circle((50,0),2),black,black);
    label("4'", (50, 0), E);
\end{asy}
\end{center}

\begin{center}
    \begin{asy}
    import graph;
    filldraw(Circle((0,75),2),black,black);
    label("1", (0, 75), W);
    filldraw(Circle((0,50),2),white,black);
    label("2", (0, 50), W);
    filldraw(Circle((0,25),2),white,black);
    label("3", (0, 25), W);
    filldraw(Circle((0,0),2),white,black);
    label("4", (0, 0), W);
    filldraw(Circle((50,75),2),black,black);
    label("1'", (50, 75), E);
    filldraw(Circle((50,50),2),black,black);
    label("2'", (50, 50), E);
    filldraw(Circle((50,25),2),black,black);
    label("3'", (50, 25), E);
    filldraw(Circle((50,0),2),white,black);
    label("4'", (50, 0), E);
\end{asy}
\begin{asy}
    import graph;
    filldraw(Circle((0,75),2),black,black);
    label("1", (0, 75), W);
    filldraw(Circle((0,50),2),white,black);
    label("2", (0, 50), W);
    filldraw(Circle((0,25),2),black,black);
    label("3", (0, 25), W);
    filldraw(Circle((0,0),2),black,black);
    label("4", (0, 0), W);
    filldraw(Circle((50,75),2),white,black);
    label("1'", (50, 75), E);
    filldraw(Circle((50,50),2),white,black);
    label("2'", (50, 50), E);
    filldraw(Circle((50,25),2),white,black);
    label("3'", (50, 25), E);
    filldraw(Circle((50,0),2),black,black);
    label("4'", (50, 0), E);
\end{asy}
\end{center}

\begin{center}
    \begin{asy}
    import graph;
    draw((0, 50)..(50, 0));
    draw((0, 25)..(50, -25));
    draw((0, 0)..(13, -13)..(0, -25));
    draw((50, 50)..(37, 37)..(50, 25));
    filldraw(Circle((0,0),2),black,black);
    label("3", (0, 0), W);
    filldraw(Circle((0,25),2),white,black);
    label("2", (0, 25), W);
    filldraw(Circle((0,50),2),black,black);
    label("1", (0, 50), W);
    filldraw(Circle((0,-25),2),white,black);
    label("4", (0, -25), W);
    filldraw(Circle((50,0),2),white,black);
    label("3'", (50, 0), E);
    filldraw(Circle((50,25),2),black,black);
    label("2'", (50, 25), E);
    filldraw(Circle((50,50),2),white,black);
    label("1'", (50, 50), E);
    filldraw(Circle((50,-25),2),black,black);
    label("4'", (50, -25), E);
\end{asy}
\end{center}

\begin{center}
    \begin{asy}
    import graph;
    draw((0, 0)..(25, 0));
    draw((0, 25)..(25, 25));
    draw((0, 50)..(25, 50));
    filldraw(Circle((0,0),2),white,black);
    label("3", (0, 0), W);
    filldraw(Circle((0,25),2),white,black);
    label("2", (0, 25), W);
    filldraw(Circle((0,50),2),black,black);
    label("1", (0, 50), W);
    filldraw(Circle((25,0),2),black,black);
    label("3'", (25, 0), E);
    filldraw(Circle((25,25),2),black,black);
    label("2'", (25, 25), E);
    filldraw(Circle((25,50),2),white,black);
    label("1'", (25, 50), E);
\end{asy}
\begin{asy}
    import graph;
    draw((0, 0)..(25, 0));
    draw((25, 25)..(15,37)..(25, 50));
    draw((0, 25)..(10,37)..(0, 50));
    filldraw(Circle((0,0),2),white,black);
    label("3", (0, 0), W);
    filldraw(Circle((0,25),2),white,black);
    label("2", (0, 25), W);
    filldraw(Circle((0,50),2),black,black);
    label("1", (0, 50), W);
    filldraw(Circle((25,0),2),black,black);
    label("3'", (25, 0), E);
    filldraw(Circle((25,25),2),black,black);
    label("2'", (25, 25), E);
    filldraw(Circle((25,50),2),white,black);
    label("1'", (25, 50), E);
\end{asy}
\end{center}
\end{comment}
\par In this paper, we also make the distinction between products of complementary minors as standalone determinants, and products of complementary minors that are ``embedded" in the matrix.

\begin{definition}\label{def:cm}
Define $\cm_{I,J}$ to be the determinant of the matrix $(y_{i,j})$ given by
\begin{equation*}
    y_{i,j} = \begin{cases}
        x_{i,j}, & i \in I, j \in J \text{ or } i \notin I, j \notin J, \\
        0, & \text{ otherwise.}
    \end{cases}
\end{equation*}
An explicit formula is
\begin{equation*}
    \cm_{I,J} = \sum_{\substack{u \in \fkS_n \\ u(I) = J}} \sgn(u) x_u.
\end{equation*}
\end{definition}

\begin{remark}
    Note that $x_u$ has nonzero coefficient in $\cm_{I,J}$ iff for all $1 \le i \le n$, the vertices $i$ and $u(i)'$ are assigned different colors in the coloring $(I, J)$.
\end{remark}

\begin{lemma}\label{cmminors}
We have $\cm_{I,J} = (-1)^{s(I) + s(J)}\Delta_{I,J} \Delta_{\overline{I},\overline{J}}$, where $s(I) := \sum_{i \in I} i$.
\end{lemma}
\begin{comment}
\begin{proof}
This is clearly true if $I = J = \{ 1, 2, \cdots, k \}$ for some $k$. In the general case, notice that we can reduce to this case by performing $\sum_{i \in I} (i-1) = s(I) - |I|$ row swaps and $\sum_{j \in J} (j-1) = s(J) - |J|$ column swaps, and $|I| = |J|,$ since each swap changes the sign of the complementary minor.
% Consider the coefficient of $x_w$. If $w(i) \notin J$ for some $i \in I$, then both sides are clearly zero. Hence, assume $w(I) \subset J$; then $w(I) = J$ since $|w(I)| = |I| = |J|$. Then the coefficient of $x_w$ in the LHS is $\sgn(w)$, and on the RHS it is $\sgn(w_I) \sgn(w_{\bI}) (-1)^{s(I) + s(J)}$, where $w_I$ and $w_{\bI}$ are the restrictions of $w$ to $I, \bI$. \todo{define restriction}

% Let $S_I = \{ (i, j) \mid i \in I, j \notin I, i < j \}$ and $S_J = \{ (i, j) \mid w(i) \in J, w(j) \notin J, i < j \}$. We will show that $\inv(w)$ has the same parity of $\inv(w_I) + \inv(w_{\bI}) + |S_I| + |S_J|$. Then we will be done since $|S_I| = s(I) - |I|, |S_J| = s(J) - |J|$, and $|I| = |J|$.

% Consider $i < j$.
% \begin{itemize}
%     \item If $i, j \in I$ then $(i, j) \in \inv(w) \iff (i, j) \in \inv(w_I)$, and $(i, j) \notin \inv(w_{\bI}), S, T$.
    
%     \item If $i, j \notin I$ then $(i, j) \in \inv(w) \iff (i, j) \in \inv(w_{\bI})$, and $(i, j) \notin \inv(w_I), S, T$.
    
%     \item If $i \in I, j \notin I$ or $i \notin I, j \in J$, then an even number of $(i, j) \in \inv(w), (i, j) \in S_I, (i, j) \in S_J$ are true, and $(i, j) \notin \inv(w_{\bI}), \inv(w_{\bJ})$.
% \end{itemize}

% This shows the claim and we are done with proof of Lemma.
\end{proof}
\end{comment}
\par As an example, consider the product of complementary minors $$\Delta_{1,4}\Delta_{234,123} = \begin{vmatrix} x_{1,4} \end{vmatrix} \begin{vmatrix} x_{2,1} & x_{2,2} & x_{2,3} \\ x_{3,1} & x_{3,2} & x_{3,3} \\ x_{4,1} & x_{4,2} & x_{4,3} \end{vmatrix},$$ where $I = \{1\}, J = \{4\}.$ Observe that $x_{1,4}x_{2,3}x_{3,2}x_{4,1}$ has coefficient $-1$ in the expansion of $\Delta_{1,4}\Delta_{234,123}$. However, if we instead represented the product of complementary minors by zeroing out entries of the $4 \times 4$ determinant, we would have the following: $$\begin{vmatrix} 0 & 0 & 0 & x_{1,4} \\ x_{2,1} & x_{2,2} & x_{2,3} & 0 \\ x_{3,1} & x_{3,2} & x_{3,3} & 0 \\ x_{4,1} & x_{4,2} & x_{4,3} & 0\end{vmatrix}.$$ Notice here that the coefficient of $x_{1,4}x_{2,3}x_{3,2}x_{4,1}$ is $+1$, and agrees with what we'd expect if we instead took the normal determinant.
\par As such, when we represent the complementary minors by pictures, we refer to this latter picture in which the minors are viewed as being within the larger matrix, rather than standing alone as matrices. The signs in front of the pictures will then correspond to adding or subtracting the corresponding $\cm_{I, J}.$
\par The pictoral representation we use for this complementary minor is the following. 
\begin{center}
\includegraphics[]{output23.pdf}
\begin{comment}
    \begin{asy}
        filldraw((0, 0)--(0,30)--(30, 30)--(30, 0)--cycle, grey);
        draw((0, 0)--(0, 40));
        draw((40, 0)--(40, 40));
        draw((0, 40)--(40, 40));
        draw((0, 0)--(40, 0));
        filldraw((30, 30)--(30, 40)--(40, 40)--(40, 30)--cycle, grey);
    \end{asy}
 \end{comment}
\end{center}
In this case we denote by the shaded region the entries that we include in our complementary minor, and the white areas as the regions that are zeroed out. This convention is followed throughout the rest of the paper.

\par Finally, we record two miscellaneous lemmas that will be useful for us. We first note that the matching diagram admits certain symmetries. Specifically, reflecting a non-crossing matching $\ncm(w)$ about either axis of the diagram also gives a non-crossing matching corresponding to either $\ncm(w^{-1})$ or $\ncm(w_0 ww_0)$, where $w_0$ is the longest word in $\fkS_n$.
\begin{lemma}\label{lem:symmetry}
If $w, u \in \mathfrak{S}_n$ with $w$ $321$-avoiding, then
$f_w (u) = f_{w^{-1}} (u^{-1}) = f_{w_0 w w_0} (w_0 u w_0)$.
\end{lemma}

\par Given a non-crossing matching, we can also construct a compatible coloring with some nice properties.

\begin{lemma}\label{lem:matching_compatibility}
Let $w$ be a $321$-avoiding permutation. Define a coloring on $[n]\cup[n]'$ as follows: for each $1 \le i \le n$, color $i$ black and $w(i)'$ white if $w(i) > i$; color $i$ white and $w(i)'$ black if $w(i) < i$; and arbitrarily color $i$ and $w(i)'$ one white, one black if $w(i) = i$. Then, in $\ncm(w)$, every white vertex $i$ or $i'$ is paired with a black vertex $j$ or $j'$ with $j \leq i$. In particular, $w$ is compatible with the coloring.
\end{lemma}
\begin{remark}
Proposition \ref{prop:ncm} gives a way to explicitly construct $\ncm(w)$ given the one-line notation of $w$. In practice, it is sufficient to know properties of $\ncm(w)$ given in Lemma \ref{lem:matching_compatibility}.
\end{remark}

\begin{comment}
\begin{proof}
From Proposition \ref{prop:cm equals imm sum}, we have for all $I, J \subset [n]$,
\begin{equation}\label{cmeqimm}
    \sgn(u) \one_{u(I) = J} = \sum_{w \mathrm{\ compatible\ with\ } I, J} f_w (u).
\end{equation}
We leverage the fact (see Remark after Proposition \ref{prop:cm equals imm sum}) that for a given $u$, this linear system of equations has a unique solution in the $\{ f_w (u) \}$.
Note that $w$ is compatible with $(I, J)$ iff $w^{-1}$ is compatible with $(J, I)$, so
\begin{equation*}
    \sgn(u^{-1}) \one_{u^{-1} (J) = I} = \sum_{w \mathrm{\ compatible\ with\ } J, I} f_{w^{-1}} (u^{-1}).
\end{equation*}
But this holds for all $J, I$, so by uniqueness of solution to \eqref{cmeqimm} with $u^{-1}$, we have $f_w (u) = f_{w^{-1}} (u^{-1})$.

Define $I^* = \{ n+1 - i \mid i \in I \}$. Note that $w$ is compatible with $(I, J)$ iff $w_0 w w_0$ is compatible with $(I^*, J^*)$, so
\begin{equation*}
    \sgn(w_0 u w_0) \one_{w_0 u w_0 (I^*) = J^*} = \sum_{w \mathrm{\ compatible\ with\ } I^*, J^*} f_{w_0 w w_0} (w_0 u w_0).
\end{equation*}
But this holds for all $I^*, J^*$, so by uniqueness of solution to \eqref{cmeqimm} with $w_0 u w_0$, we have $f_w (u) = f_{w_0 w w_0} (w_0 u w_0)$.

The interested reader is encouraged to find an alternate proof of Lemma \ref{lem:symmetry} directly using Definition \ref{syldef}.
\end{proof}
\end{comment}
% \subsection{Kazhdan-Lusztig Polynomials}
% To define Kazhdan-Lusztig immanants, we first need to define KL polynomials. The following definition is standard; see \cite[Theorem 7.11]{humphreys1990reflection} for a proof.

% \begin{definition}\label{kl-def}
%     For $u, v \in S_n$, define the \textbf{Kazhdan-Lusztig polynomial} $P_{u, v} (q)$ to satisfy the following properties:
%     \begin{enumerate}
%         \item $P_{u,v} (q) = 0$ if $u \not\le v$;
        
%         \item $P_{u,u} (q) = 1$;
        
%         \item $\deg P_{u,v} \le \frac{\ell(u,v) - 1}{2}$ if $u < v$;
        
%         \item if $u \le v$ and $v(i) > v(i+1)$, then
%         \begin{equation*}
%             P_{u,v} (q) = q^{1-c} P_{us_i, vs_i} (q) + q^c P_{u, vs_i} (q) - \sum_{z : z(i) > z(i+1)} q^{l(z, v)/2} \mu(z, vs_i) P_{u,z} (q),
%         \end{equation*}
%         where $c$ is $1$ if $u(i) > u(i+1)$ and $0$ otherwise, $s_i$ is the transposition swapping $i$ and $i+1$, $l(z, v) = l(v) - l(w)$, and $\mu(u, v)$ is the coefficient of $q^{(l(u,v) - 1)/2}$ in $P_{u,v} (q)$ if $u < v$ and $l(u, v)$ is odd, and $0$ otherwise.
%     \end{enumerate}
% \end{definition}

% In this paper, we will usually specialize to $q = 1$, in which case $(3)$ becomes
% \begin{equation}\label{kl-formula}
%     P_{u,v} (1) = P_{us_i, vs_i} (1) + P_{u, vs_i} (1) - \sum_{z : z(i) > z(i+1)} \mu(z, vs_i) P_{u,z} (1).
% \end{equation}

% \subsection{Kazhdan-Lusztig Immanants}

% Kazhdan-Lusztig immanants are a generalization of Temperley-Lieb immanants, first defined by Rhoades and Skandera in \cite{HarderRS}.

% \begin{definition}
% Let $w\in \mathfrak{S}_n$. The \textbf{Kazhdan-Lusztig immanant} $\imm_w: \mat_{n\times n}(\C)\to \C$ is given by 
% \[\imm_w(M):= \sum_{u\in \mathfrak{S}_n} (-1)^{l(u)-l(w)} P_{w_0 u,w_0 w}(1) m_{1,u_1}m_{2,u_2}\cdots m_{n,u_n} \]
% where $P_{x,y}(q)$ is the Kazhdan-Lusztig polynomial and $w_0$ is the longest word in $\mathfrak{S}_n$. See \cite[Section 5.5]{bjorner2006combinatorics} for definitions of Kazhdan-Lusztig polynomials.
% \end{definition}

% % One might be concerned that we had previously defined $\imm_w$ as the Temperley-Lieb immanant, but this turns out to not be an issue. From Rhoades and Skandera's paper \cite{HarderRS}, the following theorem is known: 

% The common notation with Temperley-Lieb immanants is not a coincidence. Indeed, we have the following result from \cite{HarderRS}:
% \begin{prop}[Proposition 5 from \cite{HarderRS}]
% Let $w$ be a $321$-avoiding permutation. Then, the Kazhdan-Lusztig immanant of $w$ equals the Temperley-Lieb immanant of $w.$
% \end{prop}

% In particular, when we write $\imm_w,$ if $w$ is $321$-avoiding, we do not need to worry about whether we are talking about Kazhdan-Lusztig immanants or Temperley-Lieb immanants, as they are equal. As such, in this section when we write $\imm_w,$ we are referring to the more general Kazhdan-Lusztig immanant, which is defined for all permutations $w.$

% Compared to Temperley-Lieb immanants, little is known about Kazhdan-Lusztig immanants, since the polynomials are relatively difficult to compute. 


\subsection{Proofs of Lemmas in Section 2}
In this section, we restate all lemmas in Section 2 and provide their proofs. 

\begin{replemma}{lem: restriction inverses}
Let $w \in \fkS_n.$ Then, if $v = w|_I \in \fkS_m$ for some subset $I \subset [1, n]$ (which in particular requires $|I| = m$), then $v^{-1} = (w^{-1})|_{w(I)}.$ In particular, $w$ avoids the pattern $v$ if and only if $w^{-1}$ avoids the pattern $v^{-1}$.
\end{replemma}

\begin{proof}
Fix $i \in [1, m].$ Let $x$ be the $i$-th smallest element in $w(I)$, and suppose $w^{-1}(x)$ is the $j$-th smallest element of $I$. Then since $w^{-1}(w(I)) = I$ (as $w$ is a permutation, and thus bijection), we have that $(w^{-1})|_{w(I)}(i) = j.$ However, we know that $w$ sends $w^{-1}(x)$ to $x,$ meaning that $w|_I(j) = i,$ or that $v(j) = i.$ Hence, for this $i,$ we have that $v^{-1}(i) = (w^{-1})|_{w(I)}(i).$ But this holds for all $i,$ giving us the desired result.
\end{proof}

\begin{replemma}{lem: restriction flips}
Let $w \in \fkS_n.$ Then, if $v = w|_I \in \fkS_m$ for some subset $I \subset [1, n],$ and $w_0 \in \fkS_n, w_0' \in \fkS_m$ are the longest words in their respective permutation groups, then $w_0'v = (w_0w)|_I$ and $vw_0' = (ww_0)|_{w_0(I)}$.
\end{replemma}

\begin{proof}
First, fix an arbitrary $i \in [1, m].$ Then, if $x$ is the $i$-th smallest element of $I$, suppose it gets sent to the $j$-th smallest element of $w(I).$ By definition of $w|_I$, we have $v(i) = j$. Since $w_0$ has one line notation $n(n-1)(n-2)\ldots 1$, we have $x < y \iff w_0(x) > w_0(y),$ for $x, y \in [1, n].$ Therefore, $w_0w$ sends the $i$-th smallest element of $I$ to the $j$-th largest element of $w_0w(I),$ so $(w_0w)|_I(i) = m + 1 - j.$ But notice that $w_0'(x) = m + 1 - x$ for each $x \in [1, m],$ and $j = v(i).$ Thus, we have that $w_0'v(i) = (w_0w)|_I(i).$ But $i$ is arbitrary, giving the first part of the lemma.
\par The second part proceeds similarly: for each $i \in [1, m],$ if $x$ is the $i$-th smallest element of $w_0(I),$ then $w_0x$ is the $i$-th largest element of $I,$ or the $(m + 1 - i)$-th smallest element of $I.$ Then, say that $j$ is so that $w(w_0(x))$ is the $j$-th smallest element of $w(I).$ We thus have that $(ww_0)|_{w_0(I)}(i) = j.$ Meanwhile, notice that $w$ sends the $(m + 1 - i)$-th smallest element of $I$ to the $j$-th smallest element of $w(I)$, meaning that $vw_0'(i) = v(m + 1 - i) = j.$ But this holds for each $i,$ so $vw_0' = (ww_0)|_{w_0(I)},$ finishing the proof of the lemma.
\end{proof}

\begin{replemma}{lem:restriction}
Let $v, w \in \fkS_n$. If $I \supset \{ i \mid v(i) \neq w(i) \}$ and $w|_I \le v|_I$, then $w \le v$.
\end{replemma}

\begin{proof}
Using equivalent definition \ref{bruhat3} of the Bruhat order, our goal is to show that $|w([1,i]) \cap [1,j]| \ge |v([1,i]) \cap [1,j]|$ for positive integers $i, j$ with $1 \le i, j \leq n.$ To show this, we add up the following two equations:
\begin{itemize}
    \item $w([1,i] \backslash I) \cap [1,j] = v([1,i] \backslash I) \cap [1,j]$ because $w(k) = v(k)$ for $k \notin I$;

    \item $|w([1,i] \cap I) \cap [1,j]| \ge |v([1,i] \cap I) \cap [1,j]|$ follows by applying the equivalent definition \ref{bruhat3} of Bruhat order to $w|_I \le v|_I$. \qedhere
\end{itemize}
\end{proof}

% \begin{replemma}{lem:iso_intervals}
% Given $v, w \in \fkS_n, $ let $I \supset \{i|v(i) \neq w(i)\},$ and suppose that $w|_I \leq v|_I.$ Let $J = v(I) = w(I),$ and suppose that $v([\min(I), \max(I)]) \cap [\min(J), \max(J)] = J.$ Then, $u \in [w, v] \iff u|_I \in [w|_I, v|_I]$ and $u(x) = v(x) = w(x)$ for all $x \not \in I.$
% \end{replemma}
% \begin{proof}
% The if direction holds even without the restrictions on $I.$ Indeed, suppose that $w|_I \leq u|_I \leq v|_I,$ and $u(x) = v(x) = w(x)$ for all $x \not \in I.$ Then, notice that the set $I' = \{i|u(i) \neq w(i)\} \subset I$ by assumption. But then, by Lemma \ref{lem:restriction}, we see that $w \leq u,$ and similarly we can use this argument to argue that $u \leq v.$
% \par We now show the other direction. The main difficulty is showing that $u(x) = v(x) = w(x),$ for all $x \not \in I.$ To prove this, suppose for the sake of contradiction that this wasn't the case. Then, there exists some $x$ so that $u(x) \neq v(x) = w(x),$ where $x \not \in I.$ 
% \par From here, define $x_0, B$ as follows:
% \begin{enumerate}
% \item If $u(x) \neq v(x)$ for some $x < \min(I),$ let $x_0$ be the smallest such $x,$ and let $B = \min(u(x), v(x)).$
% \item Otherwise, if $u(x) \neq v(x)$ for some $x > \max(I),$ let $x_0$ be the largest such $x,$ and let $B$ again be $\min(u(x), v(x)).$
% \item Otherwise, $u = v = w$ outside of $[\min(I), \max(I)],$ so $u([\min(I), \max(I)]) \cap [\min(J), \max(J)] = v([\min(I), \max(I)]) \cap [\min(J), \max(J)] = J.$ Then, there must exist some $x$ where either $x \in I$ and $u(x) \not \in J,$ or $x \not \in I$ and $u(x) \neq v(x) = w(x).$ Let $x_0$ be the smallest $x$ for which either of these is the case. 
% \par Furthermore, if $x_0 \in I,$ let $B = u(x)$ if $u(x) < \min(J),$ and $u(x) - 1$ otherwise. Else, $x_0 \not \in I;$ let $B = \min(u(x), v(x))$ if $v(x) < \min(J)$ and $\max(u(x), v(x)) - 1$ otherwise. Notice that $u([\min(I), \max(I)]) \cap [\min(J), \max(J)] = v([\min(I), \max(I)]) \cap [\min(J), \max(J)] = J$ implies that these cases are the only possible cases.
% \end{enumerate}
% First, we claim that $$|v([1,x_0 - 1]) \cap  [1,B]| = |w([1,x_0 - 1]) \cap  [1,B]|.$$ To prove this, notice that in the first case, $v = w$ on $[1, x_0-1]$ and in the second case $v = w$ on $[x_0, n].$ In the third case, notice that $B < \min(J)$ or $B \geq \max(J),$ meaning that either way, for each $i \in [1, x_0-1],$ either $i \not \in I,$ so $v(i) = w(i),$ or both $v(i), w(i)$ lie in $J$ (so either both lie in $[1, B]$ or neither do). Similarly, we see that $$|v([1,x_0]) \cap  [1,B]| = |w([1,x_0]) \cap  [1,B]|,$$ as either $v(x_0) = w(x_0),$ or $x_0 \in I,$ so $B$ is either disjoint from $J$ or contains $J$ (but $v(x_0), w(x_0)$ both lie in $J$).
% \par Therefore, by the definition of Bruhat order we have $$|u([1,x_0 - 1]) \cap  [1,B]| = |v([1,x_0 - 1]) \cap  [1,B]| = |w([1,x_0 - 1]) \cap  [1,B]|$$ and $$|u([1,x_0]) \cap  [1,B]| = |v([1,x_0]) \cap  [1,B]| = |w([1,x_0]) \cap  [1,B]|.$$ Now, notice that in each case, only one of $u(x), v(x)$ lies in $[1, B]$ by construction, implying that $|u([1,x_0 - 1]) \cap  [1,B]| = |v([1,x_0 - 1]) \cap  [1,B]|$ and $|u([1,x_0]) \cap  [1,B]| = |v([1,x_0]) \cap  [1,B]|$ cannot both simultaneously hold, which is a contradiction.
% \par We now see that $u \in [w, v]$ means that $u(x) = v(x) = w(x)$ for all $x \not \in I.$ To prove that $u|_I \in [w|_I, v|_I],$ consider a sequence of inversions to take us from $w$ to $u$ that is increasing in number of inversions, which exists by definition of $w \leq u.$ This yields us with a sequence of permutations $w, u_1, u_2, \ldots, u_k, u.$ By our argument above, each of these $u_i$ agrees with $v$ and $w$ on the complement of $I.$ But then notice that each of the inversions we perform are only inversions among pairs of elements in $I.$
% \par Suppose this sequence of inversions was $(i_1, j_1), (i_2, j_2), \ldots, (i_{k+1}, j_{k+1}).$ Applying the map $f$ that sends the $i$th smallest element of $I$ to $i$ yields a new sequence of inversions, which applied to $w_I$ necessarily yield $u_I.$ Indeed, after one inversion, notice that $u \cdot (i_1, j_1)$ sends $j_1$ to $u(i_1)$ and $i_1$ to $u(j_1),$ and agrees with $u$ everywhere else. Furthermore, $u_I \cdot (f(i_1), f(j_1))$ sends $f(j_1)$ to $u_I(f(i_1)),$ sends $f(i_1)$ to $u_I(f(j_1)),$ and fixes the rest. But this is precisely what $\left(u \cdot (i_1, j_1)\right)|_I$ does. A similar argument can be applied to $u, v,$ giving us the desired inequality $w|_I \leq u|_I \leq v|_I.$
% \end{proof}

\begin{replemma}{lem:basic fwu}
Let $w, u \in \fkS_n$, with $w$ being $321$-avoiding.
\begin{enumerate}
    \item If $u \not\ge w$, then $f_w (u) = 0$.
    
    \item $f_w (w) = 1$.
\end{enumerate}
\end{replemma}

\begin{proof}
Let $w = s_{i_1} s_{i_2} \cdots s_{i_k}$ and $u = s_{j_1} s_{j_2} \cdots s_{j_m}$ be reduced words. Consider expanding $\theta(u) = (t_{j_1} - 1)(t_{j_2} - 1) \cdots (t_{j_m} - 1)$ and simplifying each monomial using the Temperley-Lieb algebra relations. The resulting monomial is a subword of the original monomial, which turn is a subword of $t_{j_1} t_{j_2} \cdots t_{j_m}$. Thus, we can only have a $t_{i_1} t_{i_2} \cdots t_{i_k}$ term in the expansion if $s_{i_1} s_{i_2} \cdots s_{i_k}$ is a subword of $s_{j_1} s_{j_2} \cdots s_{j_m}$. Thus, if $f_w (u) \neq 0$, then by equivalent definition \ref{bruhat1} of Bruhat order, we get $u \ge w$, proving part 1 of the Lemma. For part 2, we take $w = u$, and observe that when expanding $(t_{i_1} - 1)(t_{i_2} - 1) \cdots (t_{i_k} - 1)$, the only way to get a monomial of length $k$ is to take the first term of each binomial, and thus the coefficient of $t_{i_1} t_{i_2} \cdots t_{i_k}$ is one.
\end{proof}

% \begin{replemma}{lem:matching_compatibility}
% Let $w$ be a $321$-avoiding permutation. Define a coloring on $[n], [n]'$ as follows: for each $i$, color $i$ black and $w(i)'$ white if $w(i) > i$; color $i$ white and $w(i)'$ black if $w(i) < i$; and arbitrarily color $i$ and $w(i)'$ one white, one black if $w(i) = i$. Then, in the non-crossing matching corresponding to $w,$ every white vertex $i$ or $i'$ is paired with a black vertex $j$ or $j'$ with $j \leq i$. In particular, the non-crossing matching corresponding to $w$ is compatible with the coloring.
% \end{replemma}

% \begin{proof}
% If $i$ is white, then $w(i) < i$.
% \end{proof}

% \begin{replemma}{wireTBLemma}
% For $a < x \le b$, let $\sigma$ be the permutation defined by:
% \begin{equation*}
%     \sigma(y) = \begin{cases}
%         y, & y < a \text{ or } y > b, \\
%         y + b - x, & a \le y \le x, \\
%         y + a - x, & x \le y < b.
%     \end{cases}
% \end{equation*}
% (Equivalently, $\sigma$ fixes all indices other than those in the set of integers between $a$ and $b$ inclusive, and $\sigma(x) < \sigma(x+1) < \ldots < \sigma(b) < \sigma(a) < \ldots < \sigma(x - 1)$ are consecutive integers.)
% \par Then, in $\ncm(w)$, $x + i$ is paired with $x - i - 1,$ and $(\sigma(b) - i)'$ is paired with $(\sigma(b) + i + 1)',$ for integers $i$ so that $0 \leq i \leq \min(x - a - 1, b - x).$ Of the remaining elements within $[a, b]$, the $i$th smallest unpaired element that is unprimed is paired with the $i$th smallest unpaired element that is primed. For $i \notin [a, b]$, $i$ is paired with $i'.$
% \end{replemma}
% \begin{proof}
% We may assume $a = 1, b = n$. Construct the coloring in Lemma \ref{lem:matching_compatibility}: then $[1, x - 1], [1, \sigma(b)]'$ are colored black and the remaining vertices are colored white. Now, for convenience, we re-index the $2n$ vertices starting from $x-1$ and go clockwise. That is, 
% \begin{itemize}
%     \item Vertex $i\in [1,x-1]$ has new index $x-i$
%     \item Vertex $i\in [x, n]$ has new index $2n+x-i$
%     \item Vertex $i'\in [1,n]'$ has new index $x-1+i$
% \end{itemize}
% Then $1, 2, \cdots, n$ are colored black and $n+1, n+2, \cdots, 2n$ are colored white under this new indexing. There is a unique non-crossing matching compatible with this coloring, that matches $i$ and $2n+1-i$. Converting back to the old indexing, we recover the desired result.
% \end{proof}

\begin{replemma}{cmminors}
We have $\cm_{I,J} = (-1)^{s(I) + s(J)}\Delta_{I,J} \Delta_{\overline{I},\overline{J}}.$
\end{replemma}

\begin{proof}
This is clearly true if $I = J = [k]$ for some $k$. In the general case, notice that we can reduce to this case by performing $\sum_{i \in I} (i-1) = s(I) - |I|$ many row swaps and $\sum_{j \in J} (j-1) = s(J) - |J|$ many column swaps on the matrix $M$ of $\cm_{I,J}$. To conclude, we note that $|I| = |J|,$ and each row or column swap changes the sign of $M$.
% Consider the coefficient of $x_w$. If $w(i) \notin J$ for some $i \in I$, then both sides are clearly zero. Hence, assume $w(I) \subset J$; then $w(I) = J$ since $|w(I)| = |I| = |J|$. Then the coefficient of $x_w$ in the LHS is $\sgn(w)$, and on the RHS it is $\sgn(w_I) \sgn(w_{\bI}) (-1)^{s(I) + s(J)}$, where $w_I$ and $w_{\bI}$ are the restrictions of $w$ to $I, \bI$. \todo{define restriction}

% Let $S_I = \{ (i, j) \mid i \in I, j \notin I, i < j \}$ and $S_J = \{ (i, j) \mid w(i) \in J, w(j) \notin J, i < j \}$. We will show that $\inv(w)$ has the same parity of $\inv(w_I) + \inv(w_{\bI}) + |S_I| + |S_J|$. Then we will be done since $|S_I| = s(I) - |I|, |S_J| = s(J) - |J|$, and $|I| = |J|$.

% Consider $i < j$.
% \begin{itemize}
%     \item If $i, j \in I$ then $(i, j) \in \inv(w) \iff (i, j) \in \inv(w_I)$, and $(i, j) \notin \inv(w_{\bI}), S, T$.
    
%     \item If $i, j \notin I$ then $(i, j) \in \inv(w) \iff (i, j) \in \inv(w_{\bI})$, and $(i, j) \notin \inv(w_I), S, T$.
    
%     \item If $i \in I, j \notin I$ or $i \notin I, j \in J$, then an even number of $(i, j) \in \inv(w), (i, j) \in S_I, (i, j) \in S_J$ are true, and $(i, j) \notin \inv(w_{\bI}), \inv(w_{\bJ})$.
% \end{itemize}

% This shows the claim and we are done with proof of Lemma.
\end{proof}

\begin{replemma}{lem:symmetry}
If $w, u \in \mathfrak{S}_n$ with $w$ $321$-avoiding, then
$f_w (u) = f_{w^{-1}} (u^{-1}) = f_{w_0 w w_0} (w_0 u w_0)$.
\end{replemma}
\begin{proof}
\par Throughout the proof of this lemma, given $I, J \subset [n],$ we say that $w$ is compatible with $(I, J)$ if $\ncm(w)$ compatible with the coloring where $I, \overline{J}'$ is colored black, and $\overline{I}, J'$ is colored white.
\par From Proposition \ref{prop:cm equals imm sum}, we have for all $I, J \subset [n]$ with $|I| = |J|$ and $u \in \fkS_n$,
\begin{equation}\label{cmeqimm}
    \sgn(u) \one_{u(I) = J} = \sum_{w \mathrm{\ compatible\ with\ } (I, J)} f_w (u),
\end{equation}
where $\one_{u(I) = J}$ is the indicator function that is $1$ on $u$ if $u(I) = J$ and $0$ otherwise.
\par We leverage the fact (see Remark after Proposition \ref{prop:cm equals imm sum}) that for a given $u$, this linear system of equations has a unique solution in the $\{ f_w (u) \}$.
Note that $w$ is compatible with $(I, J)$ iff $w^{-1}$ is compatible with $(J, I)$, and $\sgn(u^{-1}) = \sgn(u)$, so we can rewrite \eqref{cmeqimm} as
\begin{equation*}
    \sgn(u^{-1}) \one_{u^{-1} (J) = I} = \sum_{w \mathrm{\ compatible\ with\ } (J, I)} f_{w^{-1}} (u).
\end{equation*}
Replacing $u$ with $u^{-1}$ and swapping $I, J$, we get (for all $I, J \subset [n]$ with $|I| = |J|$ and $u \in \fkS_n$):
\begin{equation}\label{cmeqimm2}
    \sgn(u) \one_{u(I) = J} = \sum_{w \mathrm{\ compatible\ with\ } (I, J)} f_{w^{-1}} (u^{-1}).
\end{equation}
Notice that \eqref{cmeqimm} and \eqref{cmeqimm2} are the same system of equations with different variables $\{ f_w (u) \}$ and $\{ f_{w^{-1}} (u^{-1}) \}$. By uniqueness of solution to \eqref{cmeqimm}, we have $f_w (u) = f_{w^{-1}} (u^{-1})$.

A similar argument holds to show $f_w (u) = f_{w_0 w w_0} (w_0 u w_0)$. Define $I^* = \{ n+1 - i \mid i \in I \}$. Note that $w$ is compatible with $(I, J)$ iff $w_0 w w_0$ is compatible with $(I^*, J^*)$, so
\begin{equation*}
    \sgn(w_0 u w_0) \one_{w_0 u w_0 (I^*) = J^*} = \sum_{w \mathrm{\ compatible\ with\ } (I^*, J^*)} f_{w_0 w w_0} (u).
\end{equation*}
But this holds for all $I^*, J^*$, and $u$, so by uniqueness of solution to \eqref{cmeqimm}, we have $f_w (u) = f_{w_0 w w_0} (w_0 u w_0)$.

An interested reader is encouraged to find an alternate proof of Lemma \ref{lem:symmetry} directly using Definition \ref{syldef}.
\end{proof}

\begin{replemma}{lem:matching_compatibility}
Let $w$ be a $321$-avoiding permutation. Define a coloring on $[n] \cup [n]'$ as follows: for each $1 \le i \le n$, color $i$ black and $w(i)'$ white if $w(i) > i$; color $i$ white and $w(i)'$ black if $w(i) < i$; and arbitrarily color $i$ and $w(i)'$ one white, one black if $w(i) = i$. Then, in $\ncm(w)$, every white vertex $i$ or $i'$ is paired with a black vertex $j$ or $j'$ with $j \leq i$. In particular, $w$ is compatible with the coloring.
\end{replemma}

\begin{proof}
\par We use the notation of Proposition \ref{prop:ncm}. Additionally, let $y(v)$ be the $y$-coordinate of a vertex. Note that $y(i) = y(i') = n-i$. Thus, we want to show two claims: if $v$ is white, then (1) $f(v)$ is black and (2) $y(v) \le y(f(v))$.

Say a vertex $i$ or $i'$ is fixed if $w(i) = i$. If $v = i$ or $i'$ is fixed, then by Proposition \ref{prop:ncm}(a), $i$ is paired with $i'$, so $f(v) = i'$ or $i$ respectively. Thus, since $v$ is white, our coloring says that $f(v)$ must be black. Thus, we may assume $v$ is not fixed; then neither is $f(v)$.

Given a non-fixed vertex $v$, let $p_v$ be the path $p_i$ if $v = i$ is unprimed and the path $p_{w^{-1} (i)}$ if $v = i'$ is primed. We reformulate the coloring as follows: a non-fixed vertex $v$ is white if and only if when we walk along the path $p_v$ starting from $v$, the path is non-decreasing in $y$-coordinate.

Suppose we start from a non-fixed white vertex $v$, and walk along the piecewise linear path $q_v$ representing the pairing involving this white vertex. Let $p_v$ be the path $p_i$ if $v = i$ is unprimed and the path $p_{w^{-1} (i)}$ if $v = i'$ is primed. Since $v$ is a white vertex, the path $p_v$ is non-decreasing in $y$-coordinate if we walk along it starting at $v$. Since a walk along $q_v$ from $v$ starts by walking along $p_v$, we see by Proposition \ref{prop:ncm}(b) that the path $q_v$ is always non-decreasing in $y$-coordinate when walking from $v$ to $f(v)$. In particular, we have $y(v) \le y(f(v))$, proving our claim 2. Now once we get to $f(v)$, we turn around, and then our path will be non-increasing in $y$-value. But then we will be walking along $q_{f(v)}$, and thus by our coloring rule, we see that $f(v)$ must be black, proving our claim 1. This finishes the proof of the lemma.
\end{proof}

\section{\%-immanants}

\par We now define another subclass of immanant that we study in this paper, which we call \%-immanants. These immanants capture the idea of permutations fitting within skew-tableau lying in $n \times n$ matrices, which has been studied previously in connection to the Bruhat order and Kazhdan-Lusztig immanants. 
\par For instance, \cite{MR2353118} discusses the order ideals of a permutation and how these are related to a `Skew Ferrers Matrix.' In particular, \cite{MR2353118} describes when the set of permutations that fit within a certain skew-tableau in an $n \times n$ matrix is a principal ideal in the Bruhat order. Additionally \cite{CSB}, Chepuri and Sherman-Bennett discuss the ``determinantal formulas" arising from taking the determinant of a matrix where, outside of a skew-tableau, all the entries in the matrix are zero. We can view these formulas as immanants (in fact, our \%-immanants), as such a determinant is a $\mathbb{C}-$linear combination of the monomials $x_{\sigma},$ running over permutations $\sigma \subset \fkS_n.$ 
\par In this section, we define \%-immanants and provide a couple of simple examples. The main result of this section is a classification of the space of immanants generated by these \%-immanants. We re-define a \%-immanant here for convenience.
\begin{definition}
For a skew tableau $\lambda / \mu$, where $\lambda = (\lambda_1, \lambda_2, \ldots, \lambda_n)$ and $\mu = (\mu_1, \mu_2, \ldots, \mu_n)$ are non-increasing sequences of non-negative integers, we say a permutation $\sigma \in \fkS_n$ \textbf{lies in} $\lambda/\mu$ if for all $1 \le i \le n$, we have $\mu_i < \sigma(i) \leq \lambda_i.$ (Geometrically, we have $(i, \sigma(i)) \in \lambda/\mu$.)

Let $A$ is the set of permutations in $\fkS_n$ that lie in $\lambda/\mu$, then define $\imm^{\%}_{\lambda/\mu} = \sum_{\sigma \in A} \sgn(\sigma) x_{\sigma}$. We refer to this polynomial $\imm^{\%}_{\lambda/\mu}$ as a \textbf{\%-immanant} of degree $n$. We say the \textbf{diagram} of $\imm^{\%}_{\lambda/\mu}$ is  $\lambda/\mu$, embedded into a $n \times n$ square.
\end{definition}

Throughout this paper, we orient the $n \times n$ bounding box such that $(1, 1)$ is the upper-left unit square and $(n, 1)$ is the lower-left unit square. For instance, the following is the diagram of $\imm^\%_{(5, 5, 3, 2, 2)/(2, 1)}$ (for convenience, we drop trailing zeroes from $\lambda, \mu$). The areas within the skew shape are shaded. Note that this corresponds to diagrams of left-aligned skew Ferrers matrices in \cite{MR2353118}.
\begin{center}
\includegraphics[]{output21.pdf}
\begin{comment}
\begin{asy}
    draw((0, 0)--(50, 0)--(50, 50)--(0, 50)--cycle);
    filldraw((0, 0)--(0, 30)--(10, 30)--(10, 40)--(20, 40)--(20, 50)--(50, 50)--(50, 30)--(30, 30)--(30, 20)--(20, 20)--(20, 0)--cycle, grey, black);
\end{asy}
\end{comment}
\end{center}

A certain subset of \%-immanants are naturally associated to a permutation $w$ as follows.
\begin{definition}\label{defn:per imm w}
Suppose $w \in \mathfrak{S}_n.$ Let $m_w(i) = \min w([1, i])$ and let $M_w(i) = \max w([i, n]).$ Define $\mu, \lambda$ by $\mu_i = m_w(i)-1$ and $\lambda_i = M_w(i)$ for $1 \le i \le n$. Then, we define $\imm_w^{\%} = \imm_{\lambda/\mu}^{\%}.$
\end{definition}
For instance, the following is the diagram of $\imm^\%_{2143}$, where the positions of $(i, w(i))$ are marked with an ``X," and the areas within the skew shape are shaded. This is also referred to as the left hull of the permutation by \cite{MR2353118}, for instance.
\begin{center}
\includegraphics[]{output22.pdf}
\begin{comment}
\begin{asy}
    draw((0, 0)--(40, 0)--(40, 40)--(0, 40)--cycle);
    filldraw((0, 0)--(0, 30)--(10, 30)--(10, 40)--(40, 40)--(40, 10)--(30, 10)--(30, 0)--cycle, grey, black);
    label("X", (25, 5));
    label("X", (35, 15));
    label("X", (5, 25));
    label("X", (15, 35));
\end{asy}
\end{comment}
\end{center}

Now, let $P_n^{\%}$ be the vector subspace of the space of immanants spanned by the \%-immanants of degree $n$. In order to classify this space, we need the following definition.

\begin{definition}\label{defn:1234-1324}
Two permutations $w, w' \in \fkS_n$ are \textbf{$1324$-adjacent} if $w(i) = w'(i)$ for all but two values $a < b,$ and there exist values $c, d$ with $c < a < b < d,$ such that either $w(c) < w(a) < w(b) < w(d)$ and $w'(c) < w'(b) < w'(a) < w'(d),$ or vice versa for $w, w'.$
\par Two permutations $w, w' \in \fkS_n$ are then said to be \textbf{$1324$-related} if there exists a sequence of permutations $w = w_0, w_1, \ldots, w_m = w' \in \fkS_n$ so that each adjacent pair of permutations are $1324$-adjacent. Being $1324$-related forms an equivalence relation on the permutations in $\fkS_n.$
\end{definition}
For instance, $14235$ is $1324$-adjacent to $13245,$ which in turn is $1324$-adjacent to $12345.$ In fact, one can show that all permutations $w \in \fkS_5$ where $w(1) = 1, w(5) = 5$ are $1324$-related.  
\par The idea behind the name of $1324$-related is that permutations that are $1324$-related can be obtained from each other by turning $1324$ patterns into $1234$ patterns (and vice versa).
\par This condition turns out to be heavily related with coefficients of immanants lying in $P_n^{\%}.$
\begin{thm}\label{thm:classifying space of percent}
The following conditions are equivalent:
\begin{enumerate}
    \item The immanant $\imm_f$ lies in $P_n^{\%},$
    \item The function $f$ satisfies $f(w) = -f(w'),$ for all $w, w' \in \fkS_n$ that are $1324$-adjacent.
\end{enumerate}
\end{thm}
In order to prove this, we first prove three intermediate results. Assume $n$ is fixed throughout.

% One preliminary observation about these \%-immanants is that for $\imm^\%_{\lambda/\mu}$ to be nonzero, the skew tableau has to enclose the antidiagonal. 
% \syl{This is a bit casual, maybe say `the skew tableau' has to enclose the anti-diagonal?}
\begin{lemma}\label{lem:bigtableau}
If $\imm^\%_{\lambda/\mu} \neq 0$, then $(i, n+1-i) \in \lambda/\mu$ for all $1 \le i \le n$.
\end{lemma}

\begin{proof}
% [Proof of Lemma \ref{lem:bigtableau}]
We show the contrapositive assertion. If $(i, n+1-i) \notin \lambda/\mu$ for some $i$, then either $(a, b) \notin \lambda/\mu$ for all $a \le i, b \le n+1-i$ or $(a, b) \notin \lambda/\mu$ for all $a \ge i, b \ge n+1-i$. In the first case, for any permutation $w \in \fkS_n$, there must exist $j \le i$ such that $w(j) \le n+1-i.$ Thus, $(j, w(j)) \notin \lambda/\mu$, and so $\imm^\%_{\lambda/\mu} = 0$. The second case is analogous.
\end{proof}

We can define a partial order on immanants: we say $\imm_{\lambda_1/\mu_1}^\% \ge \imm_{\lambda_2/\mu_2}^\%$ if $\lambda_1/\mu_1$ contains $\lambda_2/\mu_2$.

\begin{lemma}\label{lem:engulfing}
    We have $w \in \fkS_n$ lies in $\lambda/\mu$ iff $\imm_w^\% \le \imm_{\lambda/\mu}^\%$.
\end{lemma}

\begin{proof}
    Let $\imm_w^\% = \imm_{\lambda_w/\mu_w}^\%$. If $w \notin \lambda/\mu$, then there exists $1 \le t \le n$ such that $(t, w(t)) \notin \lambda/\mu$. But $(t, w(t)) \in \lambda_w/\mu_w$, so $\lambda_w/\mu_w \not\subset \lambda/\mu$ and $\imm_w^\% \not\le \imm_{\lambda/\mu}^\%$.

    Now suppose $w \in \lambda/\mu$; then $\mu_i < w(i) \le \lambda_i$ for all $1 \le i \le n$. We claim that $\mu_i < m_w (i)$ for all $1 \le i \le n$. Indeed, if $m_w (i) = w(j)$ for some $1 \le j \le i$, then $m_w (i) = w(j) > \mu_j \ge \mu_i$. Similarly, we have $M_w (i) \le \lambda_i$ for all $1 \le i \le n$. Thus, $\lambda_w/\mu_w \subset \lambda/\mu$ and $\imm_w^\% \le \imm_{\lambda/\mu}^\%$.
\end{proof}

\begin{prop}\label{prop:same percent}
The permutations $w, w' \in \fkS_n$ are $1324$-related if and only if $\imm_w^{\%} = \imm_{w'}^{\%}.$
\end{prop}

\begin{proof}
First, suppose that the permutations are $1324$-related, so there exists a sequence of permutations where each adjacent pair of them are $1324$-adjacent. It suffices to show $\imm_w^{\%} = \imm_{w'}^{\%}$ when $m = 1,$ because the general case follows from the $m=1$ case and transitivity. %when the permutations are actually $1324$-adjacent. To then prove the proposition for $1324$-related permutations, consider a sequence of permutations, where each adjacent pair in the sequence are $1324$-adjacent permuations. We can then apply the $m = 1$ result to each adjacent pair within the sequence to give our desired equality.
Turning to the $m=1$ case, and swapping $w$ and $w'$ if necessary, we conclude there must exist $1 \leq a < b < c < d \leq n$ so that $w(a) < w(b) < w(c) < w(d)$ and $w'(a) < w'(c) < w'(b) < w'(d).$ 
\par To show that $\imm_w^{\%} = \imm_{w'}^{\%},$ we need to prove that $m_w(i) = m_{w'}(i)$ and $M_w(i) = M_{w'}(i)$ for all $i.$ First, notice that $w(j) = w'(j)$ for all $j$ except for $j = b$ and $j = c,$ by definition \ref{defn:1234-1324}. Therefore, if $i < b$ or $i \geq c,$ then $m_w(i) = m_{w'}(i)$ and $M_w(i) = M_{w'}(i)$ because these are the minimum and maximum, respectively, of the same set.
\par Otherwise, if $b \leq i < c,$ since $w(a) = w'(a) < w(b) < w'(b),$ it follows that $m_w(i) \leq w(a) < w(b), w'(b).$ But if $m_w(i) = w(j_i),$ then by definition \ref{defn:1234-1324} this equals $w'(j_i),$ as $j_i \neq b.$ Therefore, we have $m_w(i) \geq m_{w'}(i).$ Swapping the roles of $w, w'$ and repeating the argument yields $m_w(i) \leq m_{w'}(i),$ and thus these two values are equal.
\par Similarly, $M_w(i), M_{w'}(i) \geq w(d) > w(c), w'(c),$ so if $M_w(i) = w(j_i),$ as $j_i$ cannot equal $b$ or $c,$ we have $$M_w(i) = w(j_i) = w'(j_i) \geq M_{w'}(i).$$ Again, we can swap the roles of $w, w'$ to get our equality for all $i.$ This implies that $\imm_w^{\%} = \imm_{w'}^{\%},$ which we wanted.
\par For the other direction, suppose that $\imm_w^{\%} = \imm_{w'}^{\%}.$ We will show the existence of a sequence starting with $w$ and ending with $w'$ and where each pair of adjacent permutations in the sequence are $1324$-adjacent. We induct on the largest value $i$ such that $w(i) \neq w'(i),$ if one exists (and set $i = 0$ if this never holds). Our base case is $i = 0,$ where $w = w'$ and this is obvious.
\par Suppose we have shown that this holds for $i = 0, 1, 2, \ldots, j - 1,$ and $w, w'$ are permutations such that $\imm^{\%}_{w} = \imm^{\%}_{w'},$ and $j$ is the largest value such that $w(j) \neq w'(j).$ In particular, we have that $w(i) = w'(i)$ for $i > j.$ Without loss of generality, suppose that $w(j) > w'(j).$ Let $w'' = (w'(j), w(j)) \cdot w,$ and let $k$ be such that $w(k) = w'(j);$ note that $k < j.$ We will show that $w, w''$ are $1324$-adjacent.
% Consider now the sets $I = w_i([1, j-1])$ and $J = w'([1, j-1]).$ Since $I$ contains a value not in $J,$ namely $w'(j),$ it follows that $J$ contains a value not in $I.$ In addition, since $I, J \subset [n],$ we can write one of these values as $w_i(j'),$ where $j' \geq j.$ But at the same time, $w_i(j') = w'(j') \not \in J$ if $j' > j$ by assumption of maximality, meaning that $j' = j.$ \textcolor{red}{I don't see the point of this paragraph: where do $I$ and $J$ appear later?} 
\par To do this, first observe that $w(x) = w''(x)$ for $x \neq j, k.$ The main difficulty in this proof is finding values $a, d$ such that $a < k < j < d$ and $w(a) < w(j), w(k) < w(d).$ 
\par To do this, by assumption, since $\imm_{w}^{\%} = \imm_{w'}^{\%},$ we require $$M_{w'}(j) = M_{w}(j)$$ and $$m_{w'}(k) = m_{w}(k).$$ We first claim that $w^{-1}(m_w(k)) < k$ and $j < w^{-1}(M_w(j)).$
\par First, note that since $M_w (j) = \max w([j, n])$, we have $w^{-1}(M_w(j)) \geq j$. Suppose that this is an equality; then $M_{w'} (j) = M_w (j) = w(j).$ Since $M_{w'}(j) = \max w'([j, n])$, we have $w(j) \in w'([j, n])$. But observe that $w(j) \neq w(i) = w'(i)$ for $i > j,$ and $w(j) \neq w'(j),$ so $w(j)$ cannot lie in $w'([j, n])$, a contradiction. Thus, $w^{-1}(M_w(j)) > j.$
\par Similarly, $w^{-1}(m_w(k)) \leq k.$ If this is an equality, we have $\min w'([1, k]) = m_{w'} (k) = m_w (k) = w(k) = w'(j).$ But this is impossible, since $j \not \in [1, k]$. Thus, $w^{-1}(m_w(k)) < k.$
\par We now argue that $m_w(k) < w(k) < w(j) < M_w(j).$ By assumption, we have $w(j) > w(k).$ Furthermore, $w(k) \in w([1, k]),$ so $k \neq w^{-1}(m_w(k))$ implies that $w(k) > m_w(k).$ Similarly, $j \neq w^{-1}(M_w(j))$ and $w(j) \in w([j, n]),$ so $w(j) < M_w(j).$
\par Hence, we have that $w^{-1}(m_w(k)) < k < j < w^{-1}(M_w(j)),$ and $m_w(k) < w(k) = w'(j) < w(j) < M_w(j).$ In particular, it follows that $w$ and $w'' = (w'(j), w(k))w$ are $1324$-adjacent, so by the other direction of this proposition, it follows that $\imm_{w''}^{\%} = \imm_w^{\%} = \imm_{w'}^{\%}.$ Furthermore, by construction, we have that $w''(i) = w(i) = w'(i)$ for $i > j,$ and $w''(j) = w(k) = w'(j),$ meaning that the largest value where $w''$ and $w'$ disagree is less than $j.$ By the inductive hypothesis, there exists a sequence of permutations $w'', w_1, w_2, \ldots, w_k, w'$ where each adjacent pair is $1324$-adjacent. But then $w, w'', w_1, w_2, \ldots, w_k, w'$ is a sequence of permutations where each pair is $1324$-adjacent, and therefore $w, w'$ are $1324$-related. This proves the proposition.
% \par We first find our upper bound $d.$ Recall that $M_{w'}(j) = \max w'([j, n])$ and $M_{w_i}(j) = \max w_i([j, n]).$ 
% \par Notice that if $M_{w'}(j) = w'(j) = w_i(k),$ then $M_{w'}(j)$ doesn't lie in $w_i([j, n]).$ We therefore conclude that $M_{w'}(j) \neq M_{w_i}(j),$ since $M_{w_i}(j) \in w_i([j, n])$ by definition. This is a contradiction, and so therefore $M_{w'}(j) \neq w'(j).$ A similar argument tells us that $w_i(j) \neq M_{w_i}(j),$ which in turn implies that $M_{w'}(j) = M_{w_i}(j) > w_i(j), w'(j).$ Therefore, it follows that $w'^{-1}(M_{w'}(j)) = w_i^{-1}(M_{w'}(j)) > j.$ 
% \par We now argue that $w_i^{-1}(m_{w_i}(k)) < k,$ so we may take $a = w_i^{-1}(m_{w_i}(k)).$ To do this, suppose for the sake of contradiction this wasn't the case. By definition, we have that $$w_i^{-1}(m_{w_i}(k)) \leq k,$$ since $m_{w_i}(k)$ is the minimum of the values of $w_i(x)$ for $x = 1, 2, \ldots, w_i^{-1}(w'(j)).$ Therefore, if $w_i^{-1}(m_{w_i}(k)) < k$ doesn't hold, we must have that $$w_i^{-1}(m_{w_i}(k)) = k,$$ and so $m_{w_i}(k) = w_i(k) = w'(j).$ 
% \par But by assumption, the \%-immanants are equal, meaning that $m_{w_i}(k) = w'(j) = m_{w'}(k).$ By the definition of $m_{w'},$ we conclude that $$w'(j) \in w'([1, k]).$$ But this is a contradiction as $k < j,$ so $w'(j)$ cannot occur among these values. 
% \par Therefore, we must have $$w_i^{-1}(m_{w_i}(k)) < k < j < w_i^{-1}(M_{w'}(j)).$$ We now argue that $m_{w_i}(k) < w_i(j)$ and $w_i(k) < M_{w'}(j).$ The latter is not hard, since $w_i(k) = w'(j) < M_{w'}(j).$
% \par Suppose $m_{w_i}(k) > w_i(j)$ (they cannot be equal, as that implies that $j \in [1, k],$ contradiction). Then, 
% If we denote the first as $a,$ the second as $b,$ the third as $c,$ and the fourth as $d,$ note also that $m_{w_i}(k) = w_i(a) < w_i(b), w_i(c) < w_i(d) = M_{w'}(j).$ \textcolor{red}{I don't think you showed $m_{w_i} (k) < w_i (j)$...} Furthermore, since $w_{i+1} = (w_i(j), w'(j))w_i,$ $w_i(b) = w'(j), w_i(c) = w_i(j),$ we have that and $w_{i+1}(a) < w_{i+1}(b), w_{i+1}(c) < w_{i+1}(d),$ and furthermore $w_i, w_{i+1}$ are $1324$-adjacent.
% \par Now, to build our entire sequence, we start with $w = w_0,$ and repeatedly apply our construction. Notice in our construction that for each $i,$ if $j$ is the largest value for which $w_i(j), w'(j)$ aren't equal, then the largest value for which $w_{i+1}(j), w'(j)$ aren't equal, if it exists, is less than $j.$ By considering this monovariant throughout our sequence, we must eventually have have that $w_{i+1}(x) = w'(x)$ for all $x,$ or that $w_{i+1} = w',$ for some $i.$ This is our desired sequence of permutations where each are pairwise $1324$-adjacent, which proves the proposition.
\end{proof}
\begin{remark}
This proposition is the first hint that $\%$-immanants aren't able to distinguish between different $1324$-related permutations (in the sense of statement 2 of Theorem \ref{thm:classifying space of percent}). This idea is also an intuitive explanation for why, when asking which Temperley-Lieb immanants are linear combinations of $\%$-immanants later in the paper, avoiding the pattern $1324$ would be necessary. We will make these ideas more precise in later sections of the paper, once we understand some properties of the coefficients of Temperley-Lieb immanants.
\end{remark}

We now are ready to prove the theorem.

\begin{proof}[Proof of Theorem \ref{thm:classifying space of percent}]
Let $V$ be the vector space of immanants $\imm_f$ for which $f(w) = -f(w')$ for all $w, w'$ that are $1324$-adjacent. We want to show $P_n^{\%} = V$.

First, we prove $P_n^{\%} \subset V$. Since $P_n^{\%}$ is spanned by $\%$-immanants, it suffices to check that $\imm_{\lambda/\mu}^\% \in V$ for all skew tableaux $\lambda/\mu$. Fix $\lambda/\mu$, and for any $w \in \fkS_n$, let $c_w$ denote the coefficient of $x_w$ in $\imm^{\%}_{\lambda/\mu}$. In order to show $\imm_{\lambda/\mu}^\% \in V$, it suffices to check $c_w = -c_{w'}$ for any $w, w' \in \fkS_n$ that are $1324$-adjacent. Note that $c_w = \sgn(w)$ if $w$ lies in $\lambda/\mu$ and $0$ otherwise, and likewise for $w'$. Since $\sgn(w) = -\sgn(w')$, we obtain that $c_w = -c_{w'}$ if we can show the following:

\textbf{Claim.} $w$ lies in $\lambda/\mu$ iff $w'$ lies in $\lambda/\mu$.

By Lemma \ref{lem:engulfing} and Proposition \ref{prop:same percent}, both statements are equivalent to $\imm_w^\% \le \imm_{\lambda/\mu}^\%$. (A direct proof is also possible.)

% \textit{Proof.} Since $w, w'$ are $1324$-adjacent, we can find indices $i < j < k < l$ such that $w \neq w'$ only at $j < k,$ and we have the inequalities $w(i) < w(j), w(k),$ and $w(j), w(k) < w(l).$

% Suppose $w$ lies in $\lambda_p/\mu_p$. Then, we have that each $(t, w(t)) \in \lambda_p/\mu_p$ for $t\in [n]$.  
% % Since all pairs $(x, y)$ that don't lie in $\lambda_p/\mu_p$ concentrate in blocks in the top left and bottom right corner, every $(t, w(t))$ such that $i<t<l$ and $w(i)<w(t)<w(l)$ is in the skew shape. Indeed, 
% Note we can describe $w'$ as \[w'(t) =  \begin{cases} 
%       w(t) & t\not\in \{j,k\} \\
%       w(k) & t=j \\
%       w(j) & t=k
%    \end{cases}
% \]
% So for each $t\not\in\{j,k\} $, $(t,w'(t)) = (t,w(t))$ lies in the skew shape $\lambda_p/\mu_p$. To show that $(j, w'(j))$ and $(k, w'(k))$ also lie in $\lambda_p/\mu_p$, we will more generally consider any pair of $r, s$ so $i < r < l$ and $w(i) < s < w(l).$ Note that in our skew tableau, we have that $s > w(i) > (\mu_p)_i \geq (\mu_p)_r,$ where $(\mu_p)_i$ is the $i$th part of the partition $\mu_p.$ Similarly, $s < w(l) \leq (\lambda_p)_l \leq (\lambda_p)_r,$ so $(r, s) \in \lambda_p/\mu_p.$
% \par Thus, $(k,w'(k)) = (k,w(j)) \in \lambda_p/\mu_p$ because $i<k<l$ and $w(i)<w(j)<w(l)$. Similarly, $(j,w'(j)) \in \lambda_p/\mu_p$. This shows $w'$ lies in $\lambda_p/\mu_p$.

%A similar argument shows that if $w'$ lies in $\lambda_p/\mu_p$, then so does $w$, thus proving the claim and finishing the proof of (2) $\implies$ (1).

% Since $w$ and $w'$ differ by an inversion, they have opposite signs. So thus $x_{w'}$ also appears in the \%-immanant, with the opposite sign.
% \par \textbf{Case 2.} Suppose $c_w = 0$. Then $(t,w(t)) \not\in \lambda_p/\mu_p$ for some $1 \le t \le n$. If $t\not\in\{j,k\}$, then $(t, w'(t)) \not \in \lambda_p/\mu_p$ because $(t,w'(t)) = (t, w(t)) \not \in \lambda_p/\mu_p$. Otherwise, assume without loss of generality that $(j,w(j)) \not \in \lambda_p/\mu_p$. Using the contrapositive of our observation above, since $i<j<l,$ $w(i)<w(j)<w(l),$ and $\lambda_p/\mu_p$ is a skew shape, we have that either $(i,w(i)) \not \in \lambda_p/\mu_p$ or $(l,w(l)) \not \in \lambda_p/\mu_p$. Therefore, $x_{w'}$ isn't a nonzero term in the \%-immanant. In either case, the signs of the monomials $\sgn(w)x_w$ and $\sgn(w') x_{w'}$ have coefficients negative of each other.
% \par Now, let $a_p(w)$ to be $1$ if the monomial associated to $w$ appears as a term in the \%-immanant associated to $\lambda_p/\mu_p,$ and zero otherwise.

% Combining the two cases, we have that

% \begin{align*}
% f(w')&=\sum_{p=1}^d \sgn(w') c_pa_p(w')\\
% &=-\left(\sum_{p=1}^c \sgn(w) c_pa_p(w) \right) \\
% &=-f(w),
% \end{align*}
% which is exactly what we wanted.
\par This proves the Claim and thus we have shown that $P_n^{\%} \subset V$. Now, we will prove that $\dim V = \dim P_n^{\%}$.
\par Let $W$ be a set of representatives for the $1324$-related equivalence classes. For each $w \in W$, let $I_w$ be the set of permutations $w'$ that are $1324$-related to $w,$ and define the immanant $\chi_{I_w} = \sum_{\sigma \in I_w} \sgn(\sigma) x_\sigma$. First, note that the $\chi_{I_w},$ ranging over $w \in W$, span $V$. Indeed, if some immanant $\sum\limits_{w \in \fkS_n} f(w) x_w$ lies in $V,$ then $\sum\limits_{w \in \fkS_n} f(w) x_w =  \sum\limits_{w \in W} f(w) \sgn(w) \chi_{I_w},$ since any other permutation $w' \in \fkS_n$ is $1324$-related to some element $w \in W,$ meaning that $f(w') = \sgn(w)\sgn(w') f(w).$ Thus, $\dim V \le |W|$.

\par Next, we show that $\{ \imm_w^\% \}_{w \in W}$ are linearly independent in $P_n^\%$. Indeed, suppose that there is a subset $W' \subset W$ such that $\sum_{w \in W'} a_w \imm_w^\% = 0$ for nonzero $a_w$. By Proposition \ref{prop:same percent}, we have $\imm_w^\% \neq \imm_v^\%$ for distinct $v, w \in W$. Thus, there is some maximal element $v \in W'$, in the sense that $\imm_w^\% \not\le \imm_v^\%$ for $w \in W'$, $w \neq v$ (using the partial order defined before Lemma \ref{lem:engulfing}). Now consider the coefficient of $x_v$ in $\sum_{w \in W'} a_w \imm_w^\%$. The coefficient of $x_v$ in $\imm_v^\%$ is $\sgn(v)$, but the coefficient of $x_v$ in $\imm_w^\%$ is $0$ by Lemma \ref{lem:engulfing} and maximality of $v$. Thus, we must have $a_v \sgn(v) = 0$, a contradiction since by assumption $a_v \neq 0$. Hence, $\{ \imm_w^\% \}_{w \in W}$ are linearly independent in $P_n^\%$, so $\dim P_n \ge |W|$.

\par Thus, $|W| \le \dim P_n^\% \le \dim V \le |W|$. As a result, $\dim P_n^\% = \dim V$. Since $P_n^\% \subset V$, we have $P_n^\% = V$ and this completes the proof of the theorem. \qedhere

%\par Next, we will show that the immanants $\{ \imm_w^\% \}_{w \in W}$ span $V$. By Proposition \ref{prop:same percent}, the immanants $\{ \imm_w^\% \}_{w \in W}$ are distinct and therefore can be partially ordered (using the partial order defined before Lemma \ref{lem:engulfing}). Let $\{ \imm_w^\% \}_{w \in W} = \{ \imm_1, \imm_2, \cdots, \imm_m \}$ where $w_i \not\ge w_j$ for $i < j$. Let $\imm_i = \imm_{w_i}^\%$ and $\chi_i = \chi_{w_i}$. Now, let $(a_{i, j})$ be the matrix such that $\imm_j = \sum\limits_{j=1}^m a_{i, j}\chi_i;$ this is possible since $\imm_j \in P_n^{\%} \subset V.$ We claim that this matrix is upper-triangular with $1$s all along the diagonal. Indeed, to show $a_{i,j} = 0$ for $i > j$, we compare coefficients of $x_{w_j}$ on both sides. On the LHS, the coefficient is $0$ by Lemma \ref{lem:engulfing} and the fact that $i < j$. On the RHS, the coefficient is $a_{i,j}$ because $\chi_k$ with $k \neq j$ does not contain a $x_{w_j}$ term. \textcolor{red}{check $i < j$ or $i > j$...}
%\par 
%\par It's not hard to see that $a_{i, i} = 1$ for each $i,$ by comparing the coefficients of $x_w$ (in both cases being $\sgn(w)$). Now, suppose that $a_{i, j} \neq 0.$ Then, by definition, if $w_i \in I_i, w_j \in I_j$ we have that $w_j(x) \in [m_{w_i}(x), M_{w_i}(x)]$ for each $x.$ Furthermore, by Proposition \ref{prop:same percent} it doesn't matter which $w_i, w_j$ we choose (all the $m_{w_i}(x), M_{w_i}(x)$ are equal ranging over elements of $I_i$). But then notice that $w_j(x) \geq m_{w_i}(x)$ for all $x,$ and as $m_{w_i}(x) \geq m_{w_i}(x + 1)$ by definition, we have that $m_{w_i}(x) \leq m_{w_j}(x).$ Similarly, $M_{w_i}(x) \geq M_{w_j}(x),$ since $w_j(x) \leq M_{w_i}(x)$ for all $x,$ and $M_{w_i}(x) \geq M_{w_i}(x+1).$
%\par However, this implies that the skew shape associated to $w_j$ is contained within that of $w_i.$ If the skew shapes are equal, this implies that $i = j$ by Proposition \ref{prop:same percent} ($\imm_w = \imm_{w'}$ if and only if they lie in the same $I_i$), and otherwise $j > i.$ In either case, we have that $j \geq i.$ This shows that the matrix is upper triangular. But as we argued that it has $1$s along the diagonal, it follows that our matrix $A$ is invertible. But then this implies that our $\imm_i$ form a basis of $V.$ 
%\par Therefore, it follows that $V \subset P_n^{\%}.$ We have thus shown that $V = P_n^{\%}$ and this compeltes the proof of the theorem. \qedhere
%In other words, every immanant associated to a function $f$ so $f(w) = -f(w')$ for all $w, w'$ that are $1324$-adjacent is a linear combination of \%-immanants. This is precisely what we wanted to prove, and so completes the proof of the theorem. \qedhere

\end{proof}



\section{Temperley-Lieb Immanants as One \%-Immanant}\label{sec:specific-TL}
In this section, we classify the $321$-avoiding permutations $w$ whose Temperley-Lieb immanant is a \%-immanant up to sign. 
\begin{thm}\label{thm:onepercentimm}
Let $w$ be a $321$-avoiding permutation. Then $\imm_w$ is a \%-immanant up to sign if and only if $w$ avoids both $1324$ and $2143$. In that case, $\imm_w = \sgn(w) \imm^\%_w$.
\end{thm}
% \syl{It's worth mentioning the above information that were commented out, but maybe only in a few words.}

\subsection{2143-, 1324-Avoiding Implies $\imm_w = \sgn(w)\imm_w^{\%}$}
This subsection is devoted to proving the if direction of Theorem \ref{thm:onepercentimm}. This is a special case of \cite[Corollary 3.6]{CSB} which drops the assumption that $w$ is $321$-avoiding, whose proof is based on results on Schubert varieties. Here, we present a purely combinatorial proof, using the relationship between Temperley-Lieb immanants, non-crossing matchings, and complementary minors described in \cite{RS} and summarized in Section \ref{sec:prelims}. The ideas in this section will also be used extensively in Section \ref{sec:general_TL}. %The idea is to express a certain $\%$-immanant as a linear combination of products of complementary minors, each of which is then a sum of Temperley-Lieb immanants by Proposition \ref{prop:cm equals imm sum}. Miraculously, an avalanche of cancellation occurs, and we are left with a single Temperley-Lieb immanant. This idea will also be used in Section \ref{sec:general_TL}.

\begin{thm}\label{CSBThm}Special case of \cite[Corollary 3.6]{CSB}
If $w$ is a permutation that avoids the patterns 321, 1324, and 2143, then $\imm_w$ is a \%-immanant up to sign. Specifically, $\imm_w = \sgn(w) \imm_w^{\%}.$
\end{thm}
%Before proving this theorem, we present the following general lemma. We delegate their proofs to subsection \ref{subsec: sec4proof}.

The main method of the proof is as follows. First, we show that $\imm_w^\%$ can be nicely expressed as a sum of complementary minors, each of which is (up to sign) a sum of Temperley-Lieb immanants by Proposition \ref{prop:cm equals imm sum}. This yields a long expression for $\imm_w^{\%}$ in terms of Temperley-Lieb immanants. We then show that almost all of the terms cancel, leaving us with a single Temperley-Lieb immanant $\imm_w$, up to sign. The exact sign can then be extracted by comparing the coefficients of $x_w.$
\par In order to do this, we first proceed by classifying the permutations that avoid the three patterns $321, 2143, 1324$. We begin with the following proposition:
\begin{prop}\label{321-avoiding Rectangles}
If $w$ is a $321$-avoiding permutation, then the complement of the diagram of $\imm^{\%}_w$ consists of two (possibly empty) rectangles in the corners: the rectangle in the upper-left corner is $(w^{-1}(1) - 1)$ by $(w(1) - 1)$, and the one in the lower-right corner is $(n - w^{-1}(n))$ by $(n - w(n))$. In particular, $m_w(i)$ and $M_w(i)$ takes on at most two distinct values across all $1 \le i \le n$. (Recall that $m_w$ and $M_w$ were defined in Definition \ref{defn:per imm w}.)
\end{prop}
\begin{proof}
Notice that if $m_w(i)$ changes values at three times, say at $i_1, i_2, i_3,$ then we have $w(i_1) > w(i_2) > w(i_3),$ which is a contradiction to the assumption that $w$ is $321$-avoiding. Similarly, we see that if $M_w$ changes three times, at $i_1, i_2, i_3,$ we have that $M_w(i_1) = w(i_1) > M_w(i_2) = w(i_2) > M_w(i_3) = w(i_3),$ again a contradiction. Thus, both $m_w (i)$ and $M_w (i)$ take on at most two distinct values across all $1 \le i \le n$. This means the diagram of $\imm_w^\%$ is the complement of two rectangles in the upper-left and lower-right corners. Since $m_w (1) = w(1)$ and $m_w (w^{-1} (1)) = 1$, the dimensions of the upper-left rectangle are $(w^{-1}(1) - 1)$ by $(w(1) - 1)$. Similarly, the rectangle in the lower-right corner has dimensions $(n - w^{-1}(n))$ by $(n - w(n))$. This completes the proof of the Proposition.
\end{proof}
\par Before proceeding with the proof of the proposition, we utilize symmetries of Temperley-Lieb immanants and \%-immanants in order to make some assumptions about our permutation $w.$ The following lemma formalizes the symmetries of these immanants.

\begin{lemma}\label{lem: simplify cases}
Let $S$ be the linear map that sends $x_{\sigma}$ to $x_{\sigma^{-1}},$ and let $T$ be the linear map that sends $x_{\sigma}$ to $x_{w_0 \sigma w_0},$ where $w_0$ is the longest word in $\fkS_n.$ Then, $S$ sends $\imm_w$ to $\imm_{w^{-1}}$ and $\imm^{\%}_w$ to $\imm^{\%}_{w^{-1}},$ and $T$ sends  $\imm_w$ to $\imm_{w_0ww_0}$ and $\imm^{\%}_w$ to $\imm^{\%}_{w_0ww_0}.$
\end{lemma}
\begin{remark}
The proof of this lemma, found in subsection \ref{subsec: sec4proof}, can be easily extended to show that for any skew-tableau $\lambda/\mu$, we have $S(\imm_{\lambda/\mu})$ equals the \%-immanant of the reflection of $\lambda/\mu$ across the main diagonal of the $n \times n$ bounding box, while $T(\imm_{\lambda/\mu})$ equals the \%-immanant of the $180^{\circ}$ rotation of $\lambda/\mu$ about the center of the bounding box.
\end{remark}
Now, we start the proof of Theorem \ref{CSBThm}. Our first step is to use Lemma \ref{lem: simplify cases} to reduce Theorem \ref{CSBThm} to a special case.
\begin{prop}\label{prop:CSBreduction}
    If Theorem \ref{CSBThm} is true for all $321$-, $2143$-, and $1324$-avoiding permutations $w \in \fkS_n$ such that either $w(1) = 1$ or $w(1) = w(n) + 1$, then Theorem \ref{CSBThm} is true for all $321$-, $2143$-, and $1324$-avoiding permutations $w \in \fkS_n$.
\end{prop}

\begin{proof}
    \par Fix a $321$-, $2143$-, and $1324$-avoiding permutation $w \in \fkS_n$. Our goal is to show Theorem \ref{CSBThm} is true for $w$. We divide our analysis according to the following four cases. 
\begin{enumerate}
    \item $w(1) = 1$.
    \item $w(1) \neq 1, w(n) = n.$
    \item $w(1) \neq 1, w(n) \neq n, w(1)> w(n)$
    \item $w(1) \neq 1, w(n) \neq n, w(1) < w(n)$

\end{enumerate}
%\par We are going to reduce them to one of the above two bullets.
\par Case 1: Since $w(1) = 1$, Theorem \ref{CSBThm} is true for $w$.
\par Case 2: Consider $w'=w_0ww_0$.
%If $w(i)=j$, then $w'(n+1-i)=n+1-j$. Thus, the permutation matrix of $w'$ is obtained from that of $w$ by a rotation by $180^{\circ}$.
By Corollary \ref{cor: w_0 conjugation pattern avoidance}, $w'$ is $321$-, $2143$-, and $1324$-avoiding if and only if $w$ is.
%, because these three patterns are all preserved by a $180^{\circ}$ rotation of the permutation matrix. 
Furthermore, by Lemma \ref{lem: simplify cases}, $T$ sends $\imm_w$ to $\imm_{w'}$ and $\imm^{\%}_w$ to $\imm^{\%}_{w'}.$ So $\imm^{\%}_w=\imm_w$ if and only if $\imm^{\%}_{w'}=\imm_{w'}$. Thus, we have reduced showing Theorem \ref{CSBThm} holds for $w$ to showing it holds for $w'$. But now, $w'(1)=n+1-w(n)= 1$, so Theorem \ref{CSBThm} is true for $w'$.
\par Case 3: First suppose that $w(1) > w(n) + 1$. If $c = w^{-1} (w(1) - 1)$, then $1 < c < n$ forms a $321$-pattern because $w(1) > w(c) > w(n)$. This is a contradiction. So we must have $w(1) = w(n) + 1$, which means Theorem \ref{CSBThm} is true for $w$.
\par Case 4: Let $a = w^{-1} (1)$ and $b = w^{-1} (n)$; we claim that $a > b$. Suppose not; then the fact $a < b$ together with the assumption of Case 4 gives $1 < a < b < n$ and $w(a)=1<w(1)<w(n)<n=w(b)$. So $1<a<b<n$ forms a $2143$-pattern, contradicting with our assumption that $w$ avoids $2143$. Thus, we must have that $w^{-1} (1)>w^{-1} (n)$. For simplicity of notation, let $w'=w^{-1}$.
%If $w(i)=j$, then $w'(j)=i$.
%Thus, the permutation matrix of $w'$ is obtained from that of $w$ by a reflection through the main diagonal.
By Lemma \ref{lem: restriction inverses}, $w'$ is $321$-, $2143$-, and $1324$-avoiding if and only if $w$ is.
%, because these three patterns are all preserved by a reflection through the main diagonal.
Furthermore, by Lemma \ref{lem: simplify cases}, $S$ sends $\imm_w$ to $\imm_{w'}$ and $\imm^{\%}_w$ to $\imm^{\%}_{w'}.$ So $\imm^{\%}_w=\imm_w$ if and only if $\imm^{\%}_{w'}=\imm_{w'}$. Thus, we have reduced showing Theorem \ref{CSBThm} holds for $w$ to showing it holds for $w'$. But since $w^{-1} (1)>w^{-1} (n)$, we can use the analysis of Case 3 to conclude that Theorem \ref{CSBThm} holds for $w'$. This completes the proof of Proposition \ref{prop:CSBreduction}
\end{proof}

\begin{comment}
\par Case 4: If $w(1) \neq 1, w(n) \neq n$, then we are in the second bullet if $w^{-1}(1) < w^{-1}(n)$. Otherwise, assume that $w^{-1}(1) > w^{-1}(n)$. Consider $w'=w^{-1}$. If $w(i)=j$, then $w'(j)=i$. Thus, the permutation matrix of $w'$ is obtained from that of $w$ by a reflection through the main diagonal. So $w'$ is $321$-, $2143$-, and $1324$-avoiding if and only if $w$ is, because these three patterns are all preserved by a reflection through the main diagonal. Furthermore, by Lemma \ref{lem: simplify cases}, $S$ sends $\imm_w$ to $\imm_{w'}$ and $\imm^{\%}_w$ to $\imm^{\%}_{w'}.$ So $\imm^{\%}_w=\imm_w$ if and only if $\imm^{\%}_{w'}=\imm_{w'}$. Thus, we have reduced showing Theorem \ref{CSBThm} holds for $w$ to showing it hold for $w'$. But now, $w'(1) \neq 1, w'(n) \neq n$. It suffices to show that $(w')^{-1}(1) < (w')^{-1}(n)$, which implies that $w'$ satisfies the second bullet. Notice that $w^{-1}(1) > w^{-1}(n)$ and $w$ avoids $321$ by assumption. Then $w^{-1}(1) = w^{-1}(n) + 1,$ as otherwise $w^{-1}(n), w^{-1}(n) + 1, w^{-1}(1)$ is a $321$-pattern.
*The paragraph below is Frank's original treatment of case 4*
\par For the second missing case, note that if $w^{-1}(1) > w^{-1}(n),$ then to avoid the $321$ pattern we need $w^{-1}(1) = w^{-1}(n) + 1,$ as otherwise $w^{-1}(n), w^{-1}(n) + 1, w^{-1}(1)$ is a $321$ pattern. Finally, note that if $w$ avoids the patterns, so does $w^{-1}.$ As such, if $w$ doesn't lie in the above case, then $w^{-1}$ does, meaning that $\imm_{w^{-1}} = \imm_{w^{-1}}^{\%}.$ Then, applying $S$ to both sides yields our desired equality.
\end{comment}

\par By Proposition \ref{prop:CSBreduction}, we can assume that either $w(1) = 1$ or $w(1) = w(n) + 1$. We will assume this condition holds throughout the rest of this subsection. %It is easy to verify the theorem when $w=Id$. Therefore, we may assume that we are either in the case where $w(1) = 1, w(n) \neq n,$ or we are in the case where $w$ does not fix $1$ or $n$, and $w(1) = w(n)+1.$ We assume this condition holds throughout the rest of this subsection.
\par To begin with, we have the following proposition, which expresses $\imm_w^\%$ as a sum of certain products of complementary minors.
% \begin{prop}\label{One Rectangle Complementary Minors}
% If $w \in \mathfrak{S}_n$, $w \neq \Id$ so that $w(1) = 1$ and $w$ is $321$-, $2143$- and $1324$-avoiding, then \[\imm_w^{\%} = \sum\limits_{|I| = w(n), [w^{-1}(n) + 1, n] \subset I} \cm_{I, [1, w(n)]}.\] 
% \par Otherwise, suppose that $w(n) = n,$ and $w$ is $2143$-, $1324$-avoiding. Then \[\imm_w^{\%} = \sum\limits_{|I| = n - w(1) + 1, [1, w^{-1}(1) - 1] \subset I } \cm_{I, [w(1), n]}.\] 
% \end{prop}
% \begin{proof}
% %Let $w \in \mathfrak{S}_n$. We will show that $\imm_w = \imm_w^{\%}$ when $w(1) = 1,$ as the case when $w(n) = n$ is analogous.
% We may assume $w(1) = 1$, as the case $w(n) = n$ is analogous. We first show $w(n) \neq n$. Notice that $w(n)=n$ and $w(1)=1$ implies $w(i)=i$ for all $i$. This is because otherwise, there exists $i < j$ with $w(i) > w(j)$, then $(1,i,j,n)$ forms a $1324$ pattern. Assuming $w\not= \Id$ and $w(1)=1$, we must have that $w(n) \neq n$.
% \par Our goal is to show \[\imm_w^{\%} = \sum\limits_{|I| = w(n), [w^{-1}(n) + 1, n] \subset I} \cm_{I, [1, w(n)]}.\] 
% For instance, we can see $\imm_{1423}$ is the following sum of products of complementary minors, which (with the shaded regions representing the complementary minors). 
% \begin{center}
%     \includegraphics[]{output8.pdf}
% \end{center}

% \begin{comment}
% \begin{center}
% \begin{asy}
%     filldraw((0, 0)--(0,30)--(30, 30)--(30, 0)--cycle, gray(0.7));
%     draw((0, 0)--(0, 40));
%     draw((40, 0)--(40, 40));
%     draw((0, 40)--(40, 40));
%     draw((0, 0)--(40, 0));
%     filldraw((30, 30)--(30, 40)--(40, 40)--(40, 30)--cycle, gray(0.7));
%     draw((55, 15)--(55, 25));
%     draw((50, 20)--(60, 20));
%     filldraw((70, 0)--(70,20)--(100, 20)--(100, 0)--cycle, gray(0.7));
%     draw((70, 0)--(70, 40));
%     draw((110, 0)--(110, 40));
%     draw((110, 40)--(70, 40));
%     draw((70, 0)--(110, 0));
%     filldraw((100, 20)--(100, 30)--(110, 30)--(110, 20)--cycle, gray(0.7));
%     filldraw((70, 30)--(70,40)--(100, 40)--(100, 30)--cycle, gray(0.7));
% \end{asy}
% \end{center}
% \end{comment}
% \par To prove the equality, consider the coefficient of the monomial $x_\sigma.$ Observe that if $\sigma^{-1}([1, w(n)]) \neq I,$ then the term $\cm_{I, [1, w(n)]}$ contributes coefficient of zero, since $I$ has the same cardinality as $\sigma^{-1}([1, w(n)]).$ In particular, notice that the coefficient is equal to $0$ if no such $I$ exists. Otherwise, $\sigma$ permutes the elements of $I$ and $\overline{I}$ separately, meaning that the coefficient of $x_{\sigma}$ is equal to $\sgn(\sigma)$ in $\cm_{I, [1, w(n)]}.$ But notice that $I$ is unique if it exists in the above description, meaning that in \[\sum\limits_{|I| = w(n), [w^{-1}(n) + 1, n] \subset I} \cm_{I, [1, w(n)]}\] the coefficient of $x_{\sigma}$ is $\sgn(\sigma)$ if and only if $[w^{-1}(n) + 1, n] \subset \sigma^{-1}([1, w(n)]).$  
% \par But notice that the coefficient of $x_{\sigma}$ in the \%-immanant is $\sgn\sigma$ so long as $[w^{-1}(n) + 1, n] \subset \sigma^{-1}([1, w(n)]),$ (as we see that $m_w(i)$ is always $1,$ and $M_w(i) = n$ for $i \leq w^{-1}(n)$ and equals $w(n)$ after that), and zero otherwise. This is precisely what we wanted to prove. 
% \end{proof}
%Here, notice that $\imm_w^{\%}$ is the \%-immanant where the zeroes are in the entries $a_{ij}$ where either $i > w^{-1}(n)$ and $j > w(n),$ or $i < w^{-1}(1)$ and $j < w(1).$ These inequalities form two rectangles in the upper-left and lower-right corners. We assume that $w$ isn't the identity, since $\imm_w$ is pretty clearly a \%-immanant (namely, equal to the determinant).
\begin{prop}\label{Rectangle Complementary Minors}
Suppose that $w\in \mathfrak{S}_n$ is a $321$-, $1324$-, $2143$-
avoiding permutation so that either $w(1) = 1$ or $w(1) = w(n) + 1$. Then we have \[\imm_w^{\%} = \sum\limits_{\substack{|I| = w(n) \\ [w^{-1}(n) + 1, n] \subset I \subset [w^{-1}(1), n]}} \cm_{I, [1, w(n)]}.\]
% \par Otherwise, $w^{-1}(1) = w^{-1}(n) + 1,$ and so we have \[\imm_w^{\%} = \sum\limits_{|J| = w^{-1}(n), [w(n) + 1, n] \subset J \subset [w(1), n]} \cm_{[1, w^{-1}(n)], J}.\]
\end{prop}
As an example we have the following sum when $w = 3142,$ with the shaded regions representing which complementary minors we take.
\begin{center}
    \includegraphics[]{output9.pdf}
\end{center}
\begin{comment}
\begin{center}
\begin{asy}
    filldraw((0, 0)--(20,0)--(20, 20)--(0, 20)--cycle, gray(0.7));
    draw((0, 0)--(0, 40));
    draw((40, 0)--(40, 40));
    draw((0, 40)--(40, 40));
    draw((0, 0)--(40, 0));
    filldraw((20, 20)--(20, 40)--(40, 40)--(40, 20)--cycle, gray(0.7));
    draw((55, 15)--(55, 25));
    draw((50, 20)--(60, 20));
    filldraw((70, 0)--(70,10)--(90, 10)--(90, 0)--cycle, gray(0.7));
    filldraw((70, 20)--(70,30)--(90, 30)--(90, 20)--cycle, gray(0.7));
    draw((70, 0)--(70, 40));
    draw((110, 0)--(110, 40));
    draw((110, 40)--(70, 40));
    draw((70, 0)--(110, 0));
    filldraw((90, 10)--(90, 20)--(110, 20)--(110, 10)--cycle, gray(0.7));
    filldraw((90, 30)--(90, 40)--(110, 40)--(110, 30)--cycle, gray(0.7));
\end{asy}
\end{center}
\end{comment}
\begin{proof}
\par We first observe the following identity which holds in general. Let $X$ be the determinant of the matrix whose $(i, j)$ entry is the variable $x_{ij}.$ Then, for any $1 \le k \le n,$ 
\begin{equation}\label{eqn:ebm}
    X = \sum_{|I| = k} \cm_{I,[1,k]}.
\end{equation}
(The case $k = 1$ is expansion by minors along the first column.) To prove this, we compare coefficients of $x_\sigma$ on both sides. The coefficient of $x_\sigma$ in $X$ is $\sgn(\sigma)$. For the RHS, there is a unique $I$ for which $\cm_{I,[1,k]}$ has nonzero coefficient for $x_\sigma$: namely when $I = \sigma^{-1} ([1,k])$. Then, by definition of $\cm_{I,J}$, the coefficient of $x_\sigma$ in $\cm_{\sigma^{-1} ([1,k]), [1,k]}$ is $\sgn(\sigma)$. Thus, the coefficients of $x_\sigma$ on both sides of equation \ref{eqn:ebm} are equal, proving the identity.

\par From Proposition \ref{321-avoiding Rectangles}, we have that $\imm_w^{\%}$ is obtained from $X$ by setting $x_{ij} = 0$ for $1 \le i \le w^{-1} (1)-1, 1 \le j \le w(1)-1$ and $w^{-1} (n) + 1 \le i \le n, w(n) \le j \le n$.

\par Now, consider equation \ref{eqn:ebm} with $k = w(n)$, and plug in $x_{ij} = 0$ for $1 \le i \le w^{-1} (1)-1, 1 \le j \le w(1)-1$ and $w^{-1} (n) + 1 \le i \le n, w(n) \le j \le n$. We will show that $\cm_{I,[1,w(n)]}$ vanishes unless $[w^{-1}(n) + 1, n] \subset I \subset [w^{-1}(1), n]$.

\par First, if $i \notin I$ for some $w^{-1} (n) + 1 \le i \le n$, then the $i$-th row in the matrix for $\cm_{I,[1,w(n)]}$ will be all zeros, so $\cm_{I,[1,w(n)]} = 0$. Thus, $\cm_{I,[1,w(n)]}$ vanishes unless $[w^{-1}(n) + 1, n] \subset I$.

\par Next, if $w^{-1} (1) = 1$, then $I \subset [w^{-1}(1), n]$ is vacuously true. If $w^{-1} (1) > 1$, then by assumption we have $w(1) = w(n) + 1.$ Thus, $x_{ij} = 0$ for $1 \le i \le w^{-1} (1) - 1$, $1 \le j \le w(n)$. Now, if $i \in I$ for some $1 \le i \le w^{-1} (1) - 1$, then the $i$-th row of the matrix for $\cm_{I,[1,w(n)]}$ will be all zeros, so $\cm_{I,[1,w(n)]} = 0$. Thus, in either case, $\cm_{I,[1,w(n)]}$ vanishes unless $I \subset [w^{-1}(1), n]$. Hence, we have showed $\cm_{I,[1,w(n)]}$ vanishes unless $[w^{-1}(n) + 1, n] \subset I \subset [w^{-1}(1), n]$.

Finally, if $[w^{-1}(n) + 1, n] \subset I \subset [w^{-1}(1), n]$, then $\cm_{I,[1,w(n)]}$ remains unchanged after setting some $x_{ij}$ to zero. This proves the Proposition.
% \par To prove that \[\imm_w^{\%} = \sum\limits_{|I| = w(n), [w^{-1}(n) + 1, n] \subset I \subset [w^{-1}(1), n]} \cm_{I, [1, w(n)]},\] we consider coefficients of $x_{\sigma}$ on both sides. From Proposition \ref{321-avoiding Rectangles}, we know that $m_w(i)$ takes on at most two distinct values, which therefore have to be $w(1)$ (for $i < w^{-1}(1)$) and $1$ (for $i \geq w^{-1}(1)$). If they are equal, then $m_w(i) = 1$ for all $i.$ Similarly, $M_w(i)$ is either $n$ (for $i \leq w^{-1}(n)$) or $w(n)$ (for $i > w^{-1}(n)$).
% \par Therefore, given a permutation $\sigma,$ we see that the monomial $x_\sigma$ has coefficient $0$ in the \%-immanant unless for $i < w^{-1}(1),$ $w(1) \leq \sigma(i) \leq n,$ for $w^{-1}(1) \leq i \leq w^{-1}(n)$ we have $1 \leq \sigma(i) \leq n,$ and for $i > w^{-1}(n)$ we have $1 \leq \sigma(i) \leq w(n).$ If these conditions are satisfied, the coefficient is $\sgn(\sigma).$ 
% \par We can rewrite these conditions more succinctly as $\sigma^{-1}([1, w(1) - 1]) \subset [w^{-1}(1), n]$ and $\sigma^{-1}([w(n) + 1, n]) \subset [1, w^{-1}(n)].$ But either the former condition is trivial (if $w(1) = 1,$), or $w(n) = w(1) - 1.$ In either case, $\sigma^{-1}([1, w(1) - 1]) \subset [w^{-1}(1), n]$ holds if and only if $\sigma^{-1}([1, w(n)]) \subset [w^{-1}(1), n]$ does. We can also re-write the latter as $[w^{-1}(n) + 1, n] \subset \sigma^{-1}([1, w(n)]).$
% \par Now, consider the complementary minors. Notice that $\cm_{I, [1, w(n)]}$ only includes a $x_\sigma$ term iff $I = \sigma^{-1}([1, w(n)]).$ But such an $I$ appears in the sum if and only if $$[w^{-1}(n) + 1, n] \subset \sigma^{-1}([1, w(n)]) \subset [w^{-1}(1), n].$$ Since $\sigma$ is a permutation, and we have that $w(1) = w(n) + 1,$ the first condition can be re-written as $\sigma^{-1}([w(1), n]) \subset [1, w^{-1}(n)].$ Furthermore, by definition of $\cm_{I, [1, w(n)]},$ the coefficient of $x_\sigma$ in the above complementary minor, where $I = \sigma^{-1}([1, w(n)]),$ is $\sgn(\sigma).$
% \par Thus, in either case, we have the coefficient of $x_{\sigma}$ is $\sgn(\sigma)$ if $\sigma^{-1}([1, w(n)]) \subset [w^{-1}(1), n]$ and $\sigma^{-1}([w(1), n]) \subset [1, w^{-1}(n)],$ and $0$ otherwise, showing that the two are equal.
\end{proof}
From here, we can now express our complementary minors as sums of Temperley-Lieb immanants by Proposition \ref{prop:cm equals imm sum}. Before doing this, we first observe that the pattern avoidance conditions let us conclude that $w$ takes the following form. 
\begin{lemma}\label{lem: w form}
Suppose that $w$ is a $321$-, $1324$-, $2143$- avoiding permutation, and furthermore either $w(1) = 1$ or $w(1) = w(n) + 1$. Then, $w$ is uniquely determined and has one line notation $$(w(n)+1..w^{-1}(1)+w(n)- 1:1..w^{-1}(n) + w(n) - n:w^{-1}(1) + w(n)..n:w^{-1}(n) + w(n) - n+1..w(n)).$$  
% $$w(i) = \begin{cases}
% w(1) + i - 1 & \text{ if } i \leq w^{-1}(1) - 1, \\ i + 1 - w^{-1}(1) & \text{ if } w^{-1}(1) \leq i \leq w^{-1}(1) + w^{-1}(n) + w(n) - n - 1, \\ i - w^{-1}(n) + n & \text{ if } w^{-1}(1) + w^{-1}(n) + w(n) - n \leq i \leq w^{-1}(n), \\ i - n + w(n) & \text{ otherwise.} \end{cases}$$  
\end{lemma}
\begin{remark}
Specifically, $w$ has block structure $[3][1][4][2]$ with block lengths $w^{-1} (1) - 1, w^{-1} (n) + w(n) - n, n - w^{-1} (1) - w(n) + 1$, and $n - w^{-1} (n)$; some of the blocks may be empty.
%\par Also, in the proofs of Lemma \ref{Rectangle Matching} and in Proposition \ref{Rectangle Unique Matching}, we will refer to the first case as a first block, and so forth, of inputs, and similarly for the outputs; for instance, $w([1, w^{-1}(1) - 1])$ is the first block of outputs, although they are not the primed indices with the smallest values. Note that if $w(1) = 1$ that the ``first block" is empty.
\end{remark}

We also need the following lemma about colorings.

\begin{lemma}\label{lem:simple-no-internal-pair}
Suppose $a, b, c$ satisfy $0 \leq a, b, c \leq n$ and $a + b + c = 2n$. There is a unique coloring of the vertices $1, 2, \cdots, 2n$ and a unique compatible non-crossing matching such that vertices $[a+1, a+b]$ are colored black, vertices $[a+b+1, 2n]$ are colored white, and there does not exist an internal pairing within $[1, a]$ (i.e. there do not exist $1 \le i < j \le a$ that are paired).
\end{lemma}

% \begin{lemma}\label{Rectangle Matching}
% Suppose that $w$ is a $321-, 1324-, 2143-$avoiding permutation satisfying the conditions in Proposition \ref{Rectangle Complementary Minors}. Write the sum of complementary minors given in Proposition \ref{Rectangle Complementary Minors} as a combination of Temperley-Lieb immanants. Then, the coefficient of $\imm_w$ is $\pm 1.$
% \end{lemma}
% \begin{proof}
% \par Again, like in Proposition \ref{Two Rectangle Complementary Minors} we assume that $w^{-1}(1) < w^{-1}(n),$ as the other case can be argued by working with $w^{-1}$ first, and then replacing $x_{\sigma}$ with $x_{\sigma^{-1}}$ throughout to get the desired equality . Indeed, the complementary minors we discussed in the above proposition, and the Temperley-Lieb immanant of $w$ is sent to that of $w^{-1}$ since the decomposition of $w^{-1}$ into inversions is the reverse of that of $w.$
% \par We will first show that $w$ is uniquely determined given the pattern avoidance and the values $w(1), w^{-1}(1), w(n), w^{-1}(n).$ Notice that since $w$ is 1324-avoiding and 321-avoiding, $w^{-1}$ is increasing from $1$ to $w(n),$ and increasing from $w(1)$ to $n.$ Notice in particular that $w(w^{-1}(n) + 1), w(w^{-1}(n) + 2), \ldots, n$ are a string of increasing consecutive integers, as are $w(1), w(2), \ldots, w(w^{-1}(1) - 1).$ 
% \par As for the remaining values, notice that $w^{-1}$ needs to be increasing from $w(1)$ to $w(n),$ leaving one unique way to place them (as we have exactly that many open columns left). In particular, we observe that $w(w^{-1}(1) + i) = 1 + i$ for $i \leq w(w^{-1}(n) + 1) - 2,$ and $w(w^{-1}(n) - i) = n - i$ for $i \leq n - w(w^{-1}(1) - 1) - 1.$
% \par Combining these, we have the following: $$w(i) = \begin{cases}
% w(1) + i - 1 \text{ if } i \leq w^{-1}(1) - 1 \\ i + 1 - w^{-1}(1) \text{ if } w^{-1}(1) \leq i \leq w^{-1}(1) + w^{-1}(n) + w(n) - n - 1 \\ i - w^{-1}(n) + n \text{ if } w^{-1}(1) + w^{-1}(n) + w(n) - n \leq i \leq w^{-1}(n) \\ i - n + w(n) \text{ otherwise} \end{cases}.$$ Notice that these cases cover all of the integers from $1$ to $n$ since $w(1) = w(n) + 1.$ We will refer to the first case as a first block, and so forth, of inputs, and similarly for the outputs; for instance, $w([1, w^{-1}(1) - 1])$ is the first block of outputs, although they are not the primed indices with the smallest values. We will also refer to these blocks as being for different permutations; when we do not specify the permutation, this will default to mean $w.$
% \par Similarly to the one rectangle case, we consider the coefficient of each immanant in this sum. We again label two columns for our matchings by $1, 2, \ldots, n$ and $1', 2', \ldots, n'.$ Notice that we can write our permutation $w$ as being the composition of three permutations. The first is $w_1,$ which does the following:
% \begin{enumerate}
%     \item It sends $i$ to $i + 1 - w^{-1}(1)$ if $w^{-1}(1) \leq i \leq w^{-1}(1) + w^{-1}(n) + w(n) - n - 1,$
%     \item sends $i + w^{-1}(n) + w(n) - n$ for $i \leq w^{-1}(1) - 1,$ 
%     \item fixes the rest.
% \end{enumerate} The next is $w_2,$ which sends $i$ to $i - w^{-1}(n) + n$ if $w^{-1}(1) + w^{-1}(n) + w(n) - n \leq i \leq w^{-1}(n),$ to $i + w^{-1}(1) + w(n) - n - 1$ for $i \geq w^{-1}(n) + 1,$ and fixes the rest.
% \par Finally, we have $w_3,$ which sends $i$ to $i - w^{-1}(1)$ if $w^{-1}(n) + w^{-1}(1) + w(n) - n \leq i \leq w^{-1}(1) + w(n) - 1,$ to $i + w$ if $1 + w^{-1}(n) + w(n) - n \leq i \leq w^{-1}(1) + w^{-1}(n) + w(n) - n - 1.$ Notice that we can write $w = w_3w_2w_1.$ It's not too hard to see that the number of inversions in $w$ is equal to the number in $w_1$ plus that in $w_2$ plus that in $w_3;$ if there are $a$ elements in the first block, $b$ in the second, $c$ in the third and $d$ in the fourth, this implies that $w_1$ has $ab$ inversions, $w_2$ has $cd,$ $w_3$ has $bd,$ and $w$ has $a(b + d) + cd,$ which matches the sum.
% \par By Lemma \ref{wireTBLemma}, $w^{-1}(1)$ is paired with $w^{-1}(1) - 1,$ and so forth, until either we pair $1$ with $2w^{-1}(1) - 2$ or $w^{-1}(1) + w^{-1}(n) + w(n) - n - 1$ with $w^{-1}(1) - w^{-1}(n) - w(n) + n.$ A similar argument holds with the third and fourth blocks: either we run out by pairing $n$ with $2w(n) + 1 - n,$ or $w^{-1}(1) + w^{-1}(n) + w(n) - n$ with $n + w(n) + 1 - w^{-1}(n) - w^{-1}(1).$ This also happens with the first and fourth blocks of outputs (the second-highest and third-highest along the column), using Lemma \ref{wireTBLemma} on $\beta(w_3).$
% \par For the remaining ones, we use Lemma \ref{wireTBLemma}. When we consider applying $w_3,$ and consider the effect of multiplying by $\beta(w_3),$ if we are left to pair those in the first and fourth blocks in $w,$ they again will begin to pair up with each other (per the resulting pairing that arises from $w_3$ due to Lemma \ref{wireTBLemma}), until again we have paired all of those in one of the blocks.
% \par To see that happens, recall that for $\beta(w_2w_1),$ the other elements in the first block of inputs are paired with those in the first block in the output (but shifted down), and analogously for the fourth block. Then, the effect of $\beta(w_3),$ given by Lemma \ref{wireTBLemma}, combined with the effect of the pairing $\beta(w_2w_1),$ translates to the stated effect. 
% \par From here, all the remaining unpaired elements that are unprimed will be paired with those that are primed. If they arises from the first or third block, the $i$th largest element left to be paired among the unprimed elements will be paired with the $i$th largest element in the third block of outputs (which are the largest-indexed primed labels), and similarly for the second and fourth blocks. Finally, the remaining primed and unpaired elements are just paired amongst themselves, going outwards in.
% \par As an example, we consider what happens when blocks $1$ and $4$ have size $3,$ and blocks $2$ has size $1,$ block $3$ has size $2.$ The permutation is the next diagram,
% \begin{center}
%     \includegraphics[]{output14.pdf}
% \end{center}
% \begin{comment}
% \begin{center}
%     \begin{asy}
%     import graph;
%     void PermDraw(int n, pair[] match, real startx) {
%     for (int i = 0; i < n; ++i){
%         string lab = (string) (i + 1);
%         filldraw(Circle((startx,-25*i),2),black,black);
%         filldraw(Circle((startx + 100,-25*i),2),black,black);
%     }
%     for (pair z : match) {
%         draw((startx, -25*z.x)--(startx + 100, -25*z.y));
%     }
%     }
%     pair[] array = {(0, 1), (1, 2), (2, 3), (3, 0), (4, 7), (5, 8), (6, 4), (7, 5), (8, 6)};
%     pair[] array2 = {(0, 0), (1, 4), (2, 5), (3, 6), (4, 1), (5, 2), (6, 3), (7, 7), (8, 8)};
%     PermDraw(9, array, 0);
%     PermDraw(9, array2, 100);
% \end{asy}
% \end{center}
% \end{comment}
% with the corresponding matching diagram below.
% \begin{center}
%     \includegraphics[]{output15.pdf}
% \end{center}
% \begin{comment}
% \begin{center}
%     \begin{asy}
%     import graph;
%     void MatchDraw(int n, pair[] match, real startx) {
%     pair[] Coords = array(2*n, (0, 0));
%     for (int i = 0; i < n; ++i){
%         string lab = (string) (i + 1);
%         filldraw(Circle((startx,-25*i),2),black,black);
%         filldraw(Circle((startx + 100,-25*i),2),black,black);
%         Coords[i] = (startx,-25*i);
%         Coords[2*n - 1 - i] = (startx + 100,-25*i);
%     }
%     for (pair z : match) {
%         int index1 = (int) z.x;
%         int index2 = (int) z.y;
%         pair p1 = Coords[index1];
%         pair p2 = Coords[index2];
%         if (p1.x != p2.x) {
%             pair mid = ((p1.x + p2.x)/2, (p1.y + p2.y)/2);
%             draw(p1..mid..p2);
%         }
%         if (p1.x == p2.x) {
%             if (p1.x == startx) {
%                 pair mid = (5*abs(index1 - index2) + startx, (p1.y + p2.y)/2);
%                 draw(p1..mid..p2);
%             }
%             if (p1.x == startx + 100) {
%                 pair mid = (startx + 100 - 5*abs(index1 - index2), (p1.y + p2.y)/2);
%                 draw(p1..mid..p2);
%             }
%         }
%     }
%     }
%     pair[] array = {(0, 15), (1, 14), (2, 3), (4, 7), (5, 6), (8, 13), (9, 12), (10, 11), (16, 17)};
%     pair[] array2 = {(0, 17), (1, 6), (2, 5), (3, 4), (7, 10), (8, 9), (11, 16), (12, 15), (13, 14)};
%   MatchDraw(9, array, 0);
%   MatchDraw(9, array2, 100);
% \end{asy}
% \end{center}
% \end{comment}
% \par From here, notice that the colorings that we have are so that $1', 2', \ldots, w(1)'$ are white, and the rest of the primed ones are black, and furthermore $1, 2, \ldots, w^{-1}(1) - 1$ are white and $w^{-1}(n) + 1, \ldots, n$ are black. But we have then uniquely determined the colors of the remaining blocks to yield a compatible coloring (as we only have to specify the second and third blocks, which must be connected either with a first block element, a fourth block element, or something on the primed elements). Thus, we conclude that $\imm_w$ only appears in one term, meaning that its coefficient is $\sgn(w).$
% \end{proof}

With these two lemmas, we will argue in the next proposition that our sum of complementary minors is (up to sign) a single Temperley-Lieb immanant $\imm_w$.
% \begin{prop}\label{One Rectangle Unique Matching}
% Given the conditions of Proposition \ref{One Rectangle Complementary Minors} hold, the complementary minors given in that Proposition equals $\sgn(w)\imm_w.$
% \end{prop}
% \begin{proof}
% Given Lemma \ref{Rectangle Matching}, it suffices to show that every other $\imm_v$ has coefficient zero when the complementary minors are expanded in terms of the Temperley-Lieb immanants, as then we can compute the global sign factor that we need. Again, we assume that $w(1) = 1.$
% \par Suppose that $v \neq w$ is a $321$- avoiding permutation. We will consider the coefficient of $\imm_v$ in our sum of complementary minors. Notice that given a coloring that corresponds with $v,$ which by our minors has $1'$ to $w(n)'$ are black, and $(w(n) + 1)'$ to $n'$ are white, and $w^{-1}(n) + 1$ to $n$ are white. But then notice that by construction, if $\imm_v$ appears in the sum corresponding to some product of complementary minors, then it has to be consistent with some coloring. From here, observe that this matching cannot be the one that corresponds $i$ to $i',$ meaning that at least two elements that are both not primed must be matched together. 
% \par Now, observe that $(w(n) + i + 1)'$ is paired with $(w(n) - i)'.$ To show this fact, if $(w(n) + i + 1)'$ is paired with another primed vertex, it has to be $(w(n) - i)',$ since considering the possible values, it has to be $j',$ with $1 \leq j \leq w(n);$ but then we require $i$ white vertices and $i$ black vertices within the matchings to allow for a compatible matching. As for a non-primed vertex, notice that if this number is paired with $j,$ there are at least $w(n)$ black elements strictly above the line pairing $(w(n) + i + 1)'$ and $j,$ and at most $w^{-1}(n) - 1$ white ones (since we know that there are $n$ white vertices, but $n - w^{-1}(n)$ of these are below $j,$ and one of these is $(w(n) + i + 1)'$ itself since $j$ being black means that $j < w^{-1}(n)$). But $w^{-1}(n) \leq w(n),$ contradiction. Thus, it follows that $(w(n) - i)'$ and $(w(n) + i + 1)'$ are paired. 
% \par We can also apply this argument from $(2w(n) - n)'$ to $(w^{-1}(n) + w(n) - n)'$ to see that these are paired to $n, n-1, \ldots, n + w^{-1}(n) - w(n),$ by seeing that $(2w(n) - n - i)'$ has to be paired with some element $j$ that is unprimed and white. But then notice that if $j \leq w^{-1}(n),$ then there are at least $2w(n) - n - i$ black vertices at or above the line (the number arising from how many are in the primed column), meaning we have at most $2n - 2w(n) + i$ below the line. Similarly, we have at least $n - w^{-1}(n) + n - w(n)$ white ones below; but we also know that $i < w(n) - w^{-1}(n),$ meaning that $$2n - 2w(n) + i < 2n - w(n) + w^{-1}(n) \leq 2n - w^{-1}(n) - w(n),$$ and  $j > w^{-1}(n).$ But then notice that the number of white vertices below this line is $n - w(n) + n - j,$ and the number of black vertices below this line is $n - w(n) + i.$ For these to equal, we need $j = n - i.$
% \par After performing the above pairings, the elements we have left to pair are from $1$ to $n - w(n) + w^{-1}(n),$ and from $1'$ to $(w^{-1}(n) + w(n) - n)'.$ However, there are $2w^{-1}(n)$ of these elements, so either two elements in $\{1, 2, \ldots, w^{-1}(n) \}$ are paired, or each element of these is paired with an element outside. But since we require the matchings to not cross, there is only one such matching. We will show that this is $w$ by arguing that in the other case, $\imm_v$ has a coefficient zero. 
% \par Indeed, suppose that some pair of elements $(i, j),$ where $1 \leq i < j \leq w^{-1}(n),$ that are connected in the matching for $v.$ In particular, between these two, there must exist some pair of adjacent elements that are connected; let these be $i_v$ and $i_{v} + 1.$ Consider now the following operation on colorings: given a coloring compatible with $v,$ swap the colors of $i_v$ and $i_{v} + 1.$ Notice that this operation is an involution from this subset of compatible colorings to itself. 
% \par But the image of the involution goes from $\cm_{I, J}$ to $\cm_{I', J},$ where only one of $I, I'$ contains $i_v$ and the other contains $i_v + 1.$ However, this requires that $s(I), s(I')$ have different parities. In particular, this means that if $\cm_{I, J}$ is expanded in terms of Temperley-Lieb immanants with a positive coefficient for $\imm_v,$ then for $\cm_{I', J}$ its expansion has $-\imm_v.$ Indeed, all the $\Delta_{I, J}\Delta_{\bar{I}, \bar{J}}$ are sums of Temperley-Lieb immanants, so our argument shows that if $\Delta_{I, J}\Delta_{\bar{I}, \bar{J}}$ is a positive sum, $\Delta_{I', J}\Delta_{\bar{I'}, \bar{J}}$ is a negative sum (so the coefficients of $\imm_v$ within the expansions for these two are opposite).
% \par But then this involution allows us to argue that for each appearance of an $\imm_v,$ there exists another $\imm_v$ with opposite sign, and this is 1-to-1 correspondence. But then the coefficient of $\imm_v$ must then zero. This means that in particular $\imm_w$ isn't in this case, so $w$ is the unique matching that we noted above.
% \par Thus, combined with Lemma \ref{Rectangle Matching}, we've shown that our \%-immanant is equal to $\sgn(w) \imm_w,$ by comparing the coefficients of $x_w$ (which is $\sgn(w)$ in $\imm_w^{\%}$ and $1$ in $\imm_w$), which allows us to finish the proposition.
% \end{proof}

\begin{prop}\label{Rectangle Unique Matching}
Suppose that $w\in \mathfrak{S}_n$ is a $321$-, $1324$-, $2143$-
avoiding permutation so that either $w(1) = 1$ and $w(n) \neq n,$ or $w(1) = w(n) + 1$. Then there exists $c_w \in \R$ such that \[\sum\limits_{|I| = w(n), [w^{-1}(n) + 1, n] \subset I \subset [w^{-1}(1), n]} \cm_{I, [1, w(n)]} = c_w \imm_w.\]
\end{prop}
\begin{proof}
By Lemma \ref{lem:simple-no-internal-pair} and Proposition \ref{prop:ncm}, there exists a unique $\tw$ and a unique coloring $\tilde{\cC}$ compatible with $\ncm(\tw)$ such that $[1, \tw^{-1} (1) - 1]$ and $[1, \tw(n)]'$ are colored white, $[\tw^{-1} (n)+1, n]$ and $[\tw(n)+1, n]'$ are colored black, and there are no internal pairings between vertices in $[\tw^{-1} (1), \tw^{-1} (n)]$.

We first show $\tw = w$. To do this, it suffices to show that $w$ and the following coloring $\cC$
satisfy the conditions for $(\tw, \tilde{\cC})$:
\begin{itemize}
    \item $[1, w^{-1} (1) - 1]$, $[1, w(n)]'$, and $[w^{-1} (1) + w^{-1} (n) + w(n) - n, w^{-1} (n)]$ are colored white;

    \item $[w^{-1} (n)+1, n]$, $[w(n)+1, n]'$, and $[w^{-1} (1), w^{-1} (1) + w^{-1} (n) + w(n) - n - 1]$ are colored black.
\end{itemize}
First, by Lemma \ref{lem: w form}, the coloring $\cC$ satisfies the conditions of Lemma \ref{lem:matching_compatibility}, which means that $\ncm(w)$ and $\cC$ are compatible. Next, we show that there are no internal pairings between vertices in $[w^{-1} (1), w^{-1} (n)]$. There are no internal pairings within $[w^{-1} (1) + w^{-1} (n) + w(n) - n, w^{-1} (n)]$ or $[w^{-1} (1), w^{-1} (1) + w^{-1} (n) + w(n) - n - 1]$ because of color, and there are no pairings between $[w^{-1} (1) + w^{-1} (n) + w(n) - n, w^{-1} (n)]$ and $[w^{-1} (1), w^{-1} (1) + w^{-1} (n) + w(n) - n - 1]$ because by Lemma \ref{lem:matching_compatibility}, any $i \in [w^{-1} (1), w^{-1} (1) + w^{-1} (n) + w(n) - n - 1]$ must be paired with some $j$ or $j'$ with $j \le i$. This shows $(w, \cC)$ satisfy the conditions for $(\tw, \tilde{\cC})$, so $\tw = w$.

Now, we use Proposition \ref{prop:cm equals imm sum} and Lemma \ref{cmminors} to expand the sum of complementary minors in Proposition \ref{Rectangle Complementary Minors} as $\sum_{v} c_v \imm_v$ for some constants $c_v$. We claim that if $v \neq w$, then $c_v = 0$. Suppose not; then since $c_v \neq 0$, there exists a coloring compatible with $\ncm(v)$ such that $[1, v^{-1} (1) - 1]$ and $[1, v(n)]'$ are colored white, $[v^{-1} (n)+1, n]$ and $[v(n)+1, n]'$ are colored black. Since $v \neq \tw$, $\ncm(v)$ must have an internal pairing between two vertices $p, q \in [v^{-1} (1), v^{-1} (n)]$. Since $p, q$ are paired in $\ncm(v)$, we see by Remark 2 after Definition \ref{def:ncm} that $q-p$ must be odd.

For an set $I$ satisfying $[w^{-1} (n)+1, n] \subset I \subset [w^{-1} (1), n]$, define the involution $\iota$ by sending $I$ to $(I - \{p\}) \cup \{q\}$ if $p \in I,$ and $(I - \{q\}) \cup \{p\}$ otherwise. Then, by Lemma \ref{cmminors} and the fact that $q-p$ is odd, the coefficients of $\imm_v$ in $\cm_{I, [1,w(n)]}$ and $\cm_{\iota(I), [1,w(n)]}$ are negatives of each other. Thus, after cancellation, we see that $c_v = 0$ whenever $v \neq w$.
\end{proof}

Putting all of the pieces together yields Theorem \ref{CSBThm}.

\begin{proof}[Proof of Theorem \ref{CSBThm}]
Using Proposition \ref{prop:CSBreduction}, we may assume that either $w(1) = 1$ or $w(1) = w(n) + 1$. Then by Propositions \ref{Rectangle Complementary Minors} and \ref{Rectangle Unique Matching}, we obtain $\imm^\%_w = c_w \imm_w$. Comparing coefficients of $x_w$ on both sides, we have $\sgn(w) = c_w f_w (w)$. But $f_w (w) = 1$ by Lemma \ref{lem:basic fwu}, so $c_w = \sgn(w)$. This proves the theorem.
\end{proof}
%\par From here, we see that Proposition \ref{Rectangle Unique Matching} and our remarks after Lemma \ref{lem: simplify cases} then prove Theorem \ref{CSBThm}.

\subsection{$\imm_w = \sgn(w)\imm_w^{\%}$ Implies 2143-, 1324-Avoiding}
%We are now able to prove the converse of \cite[Corollary 3.6]{CSB}.
In this subsection, we prove the only if direction of Theorem \ref{thm:onepercentimm}.
\begin{thm}\label{2143ConverseThm}
If $w$ is a permutation that avoids $321$ such that $\imm_w$ is a \%-immanant up to sign, then $w$ must avoid both $1324$ and $2143$. %then the zero entries of $\imm_w:=\{a_{ij}\}_{i,j\in [n]} $ are exactly $\{a_{ij} :i < w^{-1}(1), j < w(1)\} \bigcup \{a_{ij} :i > w^{-1}(n), \, j > w(n)\}.$ Furthermore, $w$ must be both 1324 and 2143 avoiding.
\end{thm}
%\par From here, our goal for the rest of this section is to prove that the converse of the theorem holds. We begin by proving another lemma.
Our strategy is to first show $w$ avoids $1324$. Then, we show that if $w$ avoids $1324$, then $w$ must also avoid $2143$. %First, a lemma.
% \begin{lemma}\label{InvLemma}
% Suppose that $w$ is a permutation that is 321-avoiding. Let $(j, k)$ be an inversion, and let $\tau$ be the permutation swapping $w(j), w(k)$ and fixing all other elements. Then $w' = \tau w$ has one less inversion than $w.$
% \end{lemma}
% \begin{proof}
% \par Let $\inv(w)$ be the number of inversions of $w.$ Since $w$ and $w'$ differ only at the $j^{\text{th}}$ and $k^{\text{th}}$ positions, only pairs of the form $(j,k)$, $(j,t)$, and $(t,k)$, for some $j<t<k$, can have different contributions to $\inv(w')$ and $\inv(w)$. Since $w(j)>w(k)$ and $w'(j)<w'(k)$, the position $(j,k)$ has one less contribution to $\inv(w')$. For each $j<t<k$, recall that we have either $w(t)>w(j)$ or $w(t)<w(k)$ because $w$ is assumed to avoid $321$. Then\begin{itemize}
%     \item If $w(t)>w(j)$, $w(t)>w(j)>w(k)=w'(j)$ and $w(t)>w(j)=w'(k)$.
%     \item If $w(t)<w(k)$, $w(t)<w(k)<w(j)=w'(k)$ and $w(t)<w(k)=w'(j)$.
% \end{itemize}   
% Either way, positions $(j,t)$ and $(t,k)$ has the same contribution to $\inv(w)$ and $\inv(w')$, and so proving the claim.
% \end{proof}
%Before proceeding onto the proof of the converse, we prove another related theorem which will be useful to us, both in this section and in the next section.
%Now, we are ready to prove part of Theorem \ref{2143ConverseThm}. The following theorem will show why our definition of nice permutations includes $1324$-avoidance:
\begin{thm}\label{1324Thm}
If $w$ is a $321$-avoiding permutation where $\imm_w$ is a linear combination of \%-immanants, then $w$ must be $1324$-avoiding.
\end{thm}
\begin{proof}
Assume for contradiction that $w$ contains the pattern $1324$. Then we have indices $i<j<k<l$ such that $w(i)<w(k)<w(j)<w(l)$. Define the permutation $w':=w \cdot (j,k)$. By Lemma \ref{cor:bruhat_inv}, $w' \le w$. By Lemma \ref{lem:basic fwu}, we have $f_{w}(w')=0$ and $f_w(w)=1$.  

However, by Theorem \ref{thm:classifying space of percent}, we need $f_w(w') = -f_w(w),$ which is a contradiction, as desired.
% By our assumption, we may write $\ds \imm_w=\sum_{p=1}^d c_p \imm^{\%}_{\lambda_p/\mu_p}$ is a linear combination of \%-immanants. 
% For each generic $\imm^{\%}_{\lambda_p/\mu_p}$ in the sequence of \%-immanants, if $x_w$ appears with nonzero coefficient in the immanant, we have that each $(i, w(i)) \in \lambda_p/\mu_p$ for $t\in [n]$. In particular, $(i,w(i))$ and $(l,w(l))$ lie in the skew shape. Since all zeros in $\lambda_p/\mu_p$ (the pieces not in the skew shape) concentrate in blocks in the top left and bottom right corner, every $(t, w(t))$ such that $i<t<l$ and $w(i)<w(t)<w(l)$ is in the skew shape! Recall that \[w'(t) =  \begin{cases} 
%       w(t) & t\not\in \{j,k\} \\
%       w(k) & t=k \\
%       w(j) & t=j
%   \end{cases}
% \]
% So for each $t\not\in\{j,k\} $, $(t,w'(t))$ and $(t,w(t))$ lie in the skew shape. For $t=k$, $(k,w'(k)), (k,w(j)) \in \lambda_p/\mu_p$ because $i<k<l$ and $w(i)<w(j)<w(l)$. Similarly for $t=j$, $(j,w(j)) \in \lambda_p/\mu_p$. And since $w$ and $w'$ has a length difference of $1$, they have reverse signs. So thus $x_{w'}$ also appears in the \%-immanant, with the opposite sign.

% Otherwise, $x_w$ isn't a term in the \%-immanant. If $(t,w(t)) \not\in \lambda_p/\mu_p$ for some $t\not\in\{j,k\}$, then $(t, w'(t)) \not \in \lambda_p/\mu_p$ because $(t,w'(t)) = (t, w(t)) \not \in \lambda_p/\mu_p$. Otherwise, assume WLOG that $(j,w(j)) \not \in \lambda_p/\mu_p$. Note that $i<j<l$, $w(i)<w(j)<w(l),$ and that $\lambda_p/\mu_p$ is a skew shape. So $(t,w(t)) \not \in \lambda_p/\mu_p$ implies that $(i,w(i)) \not \in \lambda_p/\mu_p$ or $(l,w(l)) \not \in \lambda_p/\mu_p$. So we still have that $x_{w'}$ isn't a nonzero term in the \%-immanant. In either case, the signs of the monomials $\sgn(w)x_w$ and $\sgn(w') x_{w'}$ have coefficients negative of each other. Denote $a_p(w)$ to be $1$ if the monomial associated to $w$ appears as a term in the \%-immanant associated to $\lambda_p/\mu_p,$ and zero otherwise.

% Combining the two cases, we have that

% \begin{align*}
% 0=f_w(w')&=\sum_{p=1}^d \sgn(w') c_pa_p(w')\\
% &=-\left(\sum_{p=1}^c \sgn(w) c_pa_p(w) \right) \\
% &=-f_w(w)=-1.
% \end{align*}
\end{proof}
%In order to prove the converse of \cite[Corollary 3.6]{CSB}, we begin by describing nice $w$ that contain $2143$.

% \begin{proof}
% Now, suppose that one of the two inequalities $$w^{-1}(1) + w(1) \geq n + 2, \quad (n - w^{-1}(n)) + (n - w(n)) \geq n$$ holds. We may WLOG say that $w^{-1}(1) + w(1) \geq n + 2;$ the other inequality is argued analogously. Then, notice that, to avoid the 321-pattern, we require that for $i = 2, 3, \ldots, w^{-1}(1) - 1,$ $w(i) > w(1).$ 
% \par But notice that there are $w^{-1}(1) - 2$ integers that we need to map to integers in the set $w(1) + 1, w(1) + 2, \ldots, n,$ of which there are $n - w(1) \leq w^{-1}(1) - 2.$ We thus see that these are equal, and so in particular $w$ maps the set $\{2, 3, \ldots, w^{-1}(1) - 1\}$ to $\{w(1) + 1, \ldots, n\}.$ But then it follows that $w^{-1}(n) < w^{-1}(1),$ which is a contradiction of the case that we're in.
% \end{proof}

Now, to finish proving Theorem \ref{2143ConverseThm}, we need to show that $w$ avoids $2143$. We begin by analyzing what happens if $w$ doesn't avoid $2143$.

\begin{lemma}\label{w(1) and w(n) inequality for 2143}
Suppose that $w$ contains $2143$.
\begin{enumerate}[(a)]
    \item If $w$ avoids $321$, then $w(1) < w(n)$ and $w^{-1}(1) < w^{-1}(n)$.
    
    \item If $w$ avoids $321$, then $w^{-1}(1) + w(1) \le n + 1$ and $(n+1 - w^{-1}(n)) + (n+1 - w(n)) \le n+1$.
    
    \item If $w$ avoids $1324$, then $w(1) \neq 1$ and $w(n) \neq n.$
\end{enumerate}
%Suppose that $w$ avoids $321,$ but not $2143.$ Then,  
\end{lemma}

Furthermore, if we are given that $\imm_w$ is a \%-immanant up to sign, we would like to know some properties of the \%-immanant, beyond the one given in Lemma \ref{lem:bigtableau}.

\begin{lemma}\label{lem:corner zero}
Suppose $w$ avoids $321, 1324$ and contains $2143$, and $\imm_w$ is a \%-immanant $\imm^\%_{\lambda/\mu}$ up to sign. Then:
\begin{enumerate}[(a)]
    \item $(1, 1), (n, n) \notin \lambda/\mu$;
    
    \item There exists $i$ such that $(1, i), (n+1-i, 1), (n, i), (n+1-i, n) \in \lambda/\mu$.
\end{enumerate}
\end{lemma}

These lemmas, combined with Theorem \ref{1324Thm}, allow us to prove Theorem \ref{2143ConverseThm}.
\begin{proof}[Proof of Theorem \ref{2143ConverseThm}]
Suppose that $\imm_w$ is a \%-immanant $\imm^\%_{\lambda/\mu}$, up to a sign. By Theorem \ref{1324Thm}, $w$ is $1324$-avoiding. Assume for the sake of contradiction that $w$ is not 2143-avoiding. Then, from Lemma \ref{lem:corner zero}, we can find $i'$ such that $(1, i'), (n+1-i', 1), (n, i'), (n+1-i', n) \in \lambda/\mu$ and $(1, 1), (n, n) \notin \lambda/\mu$. %Then, from Lemma \ref{w(1) and w(n) inequality for 2143}, we know that $w(1) < w(n)$ and $w^{-1}(1) < w^{-1}(n).$
%\par We first suppose that there exists some $i$ so that the $i$th row and the $(n + 1 - i)$th column both have no zeros. We let $i$ vary from $w^{-1}(1)$ to $w^{-1}(n).$ 
%\par If this isn't the case, for the first value, it follows that $n + 1 - i \leq w(1) - 1$ or that $n + 1 - i \geq w(n) + 1$ for each $i.$ But as $i$ increases, notice that $n + 1 - i$ decreases. It follows that as $w(n) + 1 > w(1) + 1 > w(1) - 1,$ either the first always holds, or the second always holds. In particular, we demand that either $n + 1 - w^{-1}(1) \leq w(1) - 1$ or $n + 1 - w^{-1}(n) \geq w(n) + 1.$ But this means that either $w^{-1}(1) + w(1) \geq n + 2,$ or that $(n - w^{-1}(n)) + (n - w(n)) \geq n.$ 
%\par Thus, in this case, we may assume that $w^{-1}(1) + w(1) < n + 2,$ so therefore there exists some entry $a_{i, n + 1 - i}$ on the anti-diagonal so that row $i$ and column $n + 1 - i$ have no zeros. Fix this as $i'.$
\par Our approach now is to construct an $n \times n$ matrix $X,$ and then show that $\imm_w$ and $\imm^\%_{\lambda/\mu}$ give different results when evaluated on $X$. Let $X = (x_{i,j})$ where $x_{i, n+1 - i} = 1$ for all $i,$ and also set $x_{1, n + 1 - i'} = x_{1, 1} = x_{i', 1} = x_{i', n} = x_{n, n} = x_{n, n + 1 - i'} = 1.$ Set the rest of the entries to be zero.
\par We now evaluate $\imm^\%_{\lambda/\mu} (X)$. Notice that, by our assumption, this is equal to the determinant of the matrix $X'$ by setting $x_{1, 1}$ and $x_{n, n}$ to zero in $X,$ by Lemmas \ref{lem:bigtableau} and \ref{lem:corner zero}. In other words, we are evaluating the determinant of the following matrix: $$\begin{pmatrix} 0 & 0 & \hdots & 1 & \hdots & 0 & 1 \\ 0 & 0 & \hdots & 0 & \hdots & 1 & 0 \\ \vdots & \vdots & \ddots & \ddots & \ddots & \vdots & \vdots \\ 1 & 0 & \hdots & 1 & \hdots & 0 & 1 \\ \vdots & \vdots & \ddots & \ddots & \ddots & \vdots & \vdots \\ 0 & 1 & \hdots & 0 & \hdots & 0 & 0 \\ 1 & 0 & \hdots & 1 & \hdots & 0 & 0 \end{pmatrix}.$$
\par Notice that we can re-arrange these rows (which only affects the sign) so that the $i'$-th row and the $n$th row become the 2nd and 3rd row, respectively. Then, re-arrange the columns so that the $(n+1-i')$-th column and the $n$th column become the 2nd and 3rd columns. The resulting determinant we are evaluating is then $$\begin{pmatrix} 0 & 1 & 1 & 0 & \hdots & 0 & 0 & 0\\ 1 & 1 & 1 & 0 & \hdots & 0 & 0 & 0 \\ 1 & 1 & 0 & 0 & \hdots & 0 & 0 & 0 \\  0 & 0 & 0 & \ddots & 0 & 0 & 0 & 1  \\  0 & 0 & 0 & \hdots & 0 & 0 & 1 & 0 \\ \vdots & \vdots & \vdots & \ddots & \ddots & \vdots & \vdots & \vdots \\ 0 & 0 & 0 & 0 & 0 & 1 & \hdots & 0 \\ 0 & 0 & 0 & 0 & 1 & 0 & \hdots & 0 \\ 0 & 0 & 0 & 1 & 0 & 0 & \hdots & 0 \end{pmatrix}.$$
\par Notice that this is a block-diagonal matrix, with a matrix with all $1$s along the anti-diagonal in one block and the matrix $\begin{pmatrix}0 & 1 & 1 \\ 1 & 1 & 1 \\ 1& 1 &  0 \end{pmatrix}$ as the other. This means that the determinant of this matrix in total is equal to simply the product of the determinants of these two blocks, which is just $1.$ There are some sign considerations, but this is not important. We conclude $\imm^\%_{\lambda/\mu} (X) = \pm 1$.
\par On the other hand, since $X$ has three equal rows (rows $1, i$ and $n$), Proposition 3.14 from \cite{RS} tells us that $\imm_w X = 0.$ In particular, for our $w,$ we know that $\imm_w X = 0.$ %since $X$ has three copies of the same row, with a $1$ only in the first, $i'$th, and $n$th columns.
But this contradicts $\imm^\%_{\lambda/\mu} (X) = \pm 1$. This proves the theorem.
%\par Now, suppose that one of the two inequalities $$w^{-1}(1) + w(1) \geq n + 2, \quad (n - w^{-1}(n)) + (n - w(n)) \geq n$$ holds. We may WLOG say that $w^{-1}(1) + w(1) \geq n + 2;$ the other inequality is argued analogously. Then, notice that, to avoid the 321-pattern, we require that for $i = 2, 3, \ldots, w^{-1}(1) - 1,$ $w(i) > w(1).$ 
%\par But notice that there are $w^{-1}(1) - 2$ integers that we need to map to integers in the set $w(1) + 1, w(1) + 2, \ldots, n,$ of which there are $n - w(1) \leq w^{-1}(1) - 2.$ We thus see that these are equal, and so in particular $w$ maps the set $\{2, 3, \ldots, w^{-1}(1) - 1\}$ to $\{w(1) + 1, \ldots, n\}.$ But then it follows that $w^{-1}(n) < w^{-1}(1),$ which is a contradiction of the case that we're in.
%\par We've now covered all of the cases, and hence proven the theorem.
\end{proof}
\subsection{Proofs of Lemmas in Section 4}\label{subsec: sec4proof}

In this subsection we present the proofs of the various lemmas throughout this section. We also reproduce each of the lemmas in this subsection.

% \begin{replemma}{lem:matching_compatibility}
% Let $w$ be a $321$-avoiding permutation. Define a coloring on $[n], [n]'$ as follows: for each $i$, color $i$ black and $w(i)'$ white if $w(i) \geq i$, and color $i$ white and $w(i)'$ black if $w(i) < i$. Then, in $\ncm(w)$ (defined in Proposition \ref{prop:ncm}), every white vertex $i$ or $i'$ is paired with a black vertex $j$ or $j'$ with $j \leq i$. In particular, $\ncm(w)$ is compatible with the coloring.
% \end{replemma}

% \begin{proof}
% \par Recall that $\ncm(w)$ can be represented by a diagram as in Proposition \ref{prop:ncm}. Suppose we start from a white vertex, and walk along the piecewise linear path representing the pairing involving this white vertex. Then, by our assumption, we start with increasing along the $y-$direction. Furthermore, every time we reach a transition, the fact that the $y-$direction is increasing doesn't change. In particular, this means that along the final line segment, we are increasing in $y-$value. By our assumption again, this final vertex is black.
% \end{proof}

\begin{replemma}{lem: simplify cases}
Let $S$ be the linear map that sends $x_{\sigma}$ to $x_{\sigma^{-1}},$ and let $T$ be the linear map that sends $x_{\sigma}$ to $x_{w_0 \sigma w_0},$ where $w_0$ is the longest word in $\fkS_n.$ Then, $S$ sends $\imm_w$ to $\imm_{w^{-1}}$ and $\imm^{\%}_w$ to $\imm^{\%}_{w^{-1}},$ and $T$ sends  $\imm_w$ to $\imm_{w_0ww_0}$ and $\imm^{\%}_w$ to $\imm^{\%}_{w_0ww_0}.$
\end{replemma}
\begin{proof}
The claim about $\imm_w$ is Lemma \ref{lem:symmetry}, so we will show the Lemma for $\imm^{\%}_w$.

Note that the coefficient of $x_{\sigma}$ in $\imm^{\%}_{w^{-1}}$ is $\sgn(\sigma)$ if and only if $m_{w^{-1}}(i) \leq \sigma(i) \leq M_{w^{-1}}(i)$ for each $i,$ and is $0$ otherwise. But now, consider $\sigma^{-1}(i).$ Notice that $m_{w^{-1}}(i) \leq \sigma(i)$ implies that $w^{-1}(k) = l \leq \sigma(i)$ for some $k \leq i.$ Furthermore, we know that $w(l) \leq k \leq i,$ meaning that $m_w(\sigma(i)) \leq i$ for each $i,$ or that $m_w(i) \leq \sigma^{-1}(i).$ Similarly, $w^{-1}(p) = q \geq \sigma(i)$ for some $p \geq i,$ so then $w(q) = p \geq i,$ and thus $M_w(\sigma(i)) \geq i$ for each $i.$ We combine these results to get that $m_w(i) \leq \sigma^{-1}(i) \leq M_w(i).$ In particular, $x_{\sigma}$ has coefficient $\sgn(\sigma)$ in $\imm_w^{\%}$ if and only if $x_{\sigma^{-1}}$ has coefficient $\sgn(\sigma^{-1})$ in $\imm_{w^{-1}}^{\%}.$ Thus, $S$ sends $\imm^{\%}_{w^{-1}}$ to $\imm^{\%}_w.$ It's not hard to see that $S$ is an involution, so the reverse also holds. Similarly for $T,$ note that $m_w(i) \leq \sigma(i) \leq M_w(i)$ for each $i$ if and only if $\max \{n + 1 - w(1),  n + 1 - w(2), \ldots,  n + 1 - w(n + 1 - i)\} = M_{w_0ww_0}(i) \geq n + 1 - \sigma(n + 1 - i) \geq \min \{n + 1 - w(n + 1 - i), n + 1 - w(n + 2 - i), \ldots, n + 1 - w(n)\} =  m_{w_0ww_0}(i).$ In other words, $x_{\sigma}$'s coefficient in $\imm_w^{\%}$ is the same as $x_{w_0\sigma w_0}$'s coefficient in $\imm_{w_0ww_0}^{\%}$ for each $u,$ meaning that $T$ sends the former to the latter.
% \par To prove this for $\imm_w,$ the Kazhdan-Lusztig immanant, we note that the coefficient of $x_u$ in $\imm_w$ is $(-1)^{l(u) - l(w)}P_{w_0u, w_0w}(1).$ Meanwhile, the coefficient of $x_{u^{-1}}$ in $\imm_{w^{-1}}$ is $$(-1)^{l(u^{-1}) - l(w^{-1})}P_{w_0u^{-1}, w_0w^{-1}}(1) = (-1)^{l(u) - l(w)}P_{w_0u^{-1}, w_0w^{-1}}(1),$$ as a permutation and its inverse have the same length. Furthermore, by Proposition \ref{properties of KL}, this is equal to $(-1)^{l(u) - l(w)}P_{uw_0, ww_0}(1) = (-1)^{l(u) - l(w)}P_{w_0u, w_0w}(1).$ But this means that $S$ sends $\imm_w$ to $\imm_{w^{-1}},$ the second part of what we wanted. We can also use this logic to see that the coefficient of $x_{w_0uw_0}$ in $\imm_{w_0ww_0}$ is equal to $(-1)^{l(w_0uw_0) - l(w_0ww_0)}P_{uw_0, ww_0}(1).$ Note that $w_0uw_0$ and $u$ have the same parity (as they are both can be written out of the same number of transpositions), so this equals by Proposition \ref{properties of KL} $(-1)^{l(u) - l(w)}P_{uw_0, ww_0}(1) = (-1)^{l(u) - l(w)}P_{w_0u, w_0w}(1).$ This then shows that the map $T,$ which sends $x_{\sigma}$ to $x_{w_0\sigma w_0},$ sends $\imm_w$ to $\imm_{w_0ww_0},$ as the coefficient of $x_{w_0uw_0}$ in $\imm_{w_0ww_0}$ is that of $x_u$ in $\imm_w$ (which is what it would need to be to equal $T(\imm_w)$). This proves the lemma.
\end{proof}
\begin{replemma}{lem: w form}
Suppose that $w$ is a $321$-, $1324$-, $2143$- avoiding permutation, and furthermore either $w(1) = 1$ or $w(1) = w(n) + 1$. Then, $w$ is uniquely determined and has one line notation $$(w(n)+1..w^{-1}(1)+w(n)- 1:1..w^{-1}(n) + w(n) - n:w^{-1}(1) + w(n)..n:w^{-1}(n) + w(n) - n+1..w(n)).$$
%(w(1)..w(1)+w^{-1}(1) - 2:1..w^{-1}(n) + w(n) - n:w^{-1}(1) + w(n)..n:w^{-1}(n) + w(n) - n+1..w(n))
% $$w(i) = \begin{cases}
% w(1) + i - 1 & \text{ if } i \leq w^{-1}(1) - 1 \\ i + 1 - w^{-1}(1) & \text{ if } w^{-1}(1) \leq i \leq w^{-1}(1) + w^{-1}(n) + w(n) - n - 1 \\ i - w^{-1}(n) + n & \text{ if } w^{-1}(1) + w^{-1}(n) + w(n) - n \leq i \leq w^{-1}(n) \\ i - n + w(n) & \text{ otherwise} \end{cases}.$$
\end{replemma}
\begin{proof}
We first show $w^{-1}$ is increasing from $1$ to $w(n).$ Otherwise, if $w^{-1}(i) > w^{-1}(j)$ but $1 \leq i < j \leq w(n),$ then either $w(1) \neq 1,$ and so $1 < w^{-1}(j) < w^{-1}(i)$ forms a $321$-pattern (note that $w(1) = w(n) + 1 > j > i$), or $w(1) = 1,$ where $i, j \neq 1$ yields that 
$1 < w^{-1}(j) < w^{-1}(i) < n$ forms a $1324$ pattern. Similarly, $w^{-1}$ is increasing from $w(n) + 1$ to $n.$
\par We first show $w(1..w^{-1} (1) - 1) = (w(n)+1..w(n) + w^{-1} (1) - 1)$. If $w(1) = 1$ then this statement is trivial. If $w(1) > 1$ then $w(1) = w(n) + 1$, and so by $321$-avoidance of $w$ (or Proposition \ref{321-avoiding Rectangles}), we have $w(i) \ge w(n) + 1$ for $1 \le i \le w^{-1} (1) - 1$. Since $w^{-1}$ is increasing from $w(n)+1$ to $n$, we conclude that $w(1..w^{-1} (1) - 1) = (w(n)+1..w(n) + w^{-1} (1) - 2)$.
\par Next, we show that $w(w^{-1}(n) + 1..n) = (w(n) + w^{-1} (n) - n+1..w(n))$. By $321$-avoidance of $w$, we know that $w(i) \le w(n)$ for $w^{-1}(n) + 1 \le i \le n$. Since $w^{-1}$ is increasing from $1$ to $w(n)$, we conclude that $w(w^{-1}(n) + 1..n) = (w(n) + w^{-1} (n) - n+1..w(n))$.
\par As for the remaining values for $w$, notice that since $w$ avoids $1324$, $w$ needs to be increasing from $w^{-1}(1)$ to $w^{-1}(n)$. We also know
\begin{equation*}
    w([w^{-1} (1), w^{-1} (n)]) = [n] \setminus w([1, w^{-1} (1) - 1:w^{-1}(n) + 1, n]) = [1, w(n)+w^{-1}(n)-n : w(n)+w^{-1} (1), n].
\end{equation*}
Thus, we get $w(w^{-1} (1)..w^{-1} (1)+w(n)+w^{-1}(n)-n-1) = (1..w(n)+w^{-1}(n)-n)$ and $w(w^{-1} (1)+w(n)+w^{-1}(n)-n..w^{-1} (n)) = (w(n)+w^{-1} (1)..n)$, which corresponds to the one line notation 
$$(w(n)+1..w^{-1}(1)+w(n)- 1:1..w^{-1}(n) + w(n) - n:w^{-1}(1) + w(n)..n:w^{-1}(n) + w(n) - n+1..w(n)),$$
proving the lemma.
% $w(w^{-1}(1) + i) = 1 + i$ for $i \leq w(w^{-1}(n) + 1) - 2 = (w(n) - n + w^{-1}(n) + 1) - 2 = w^{-1}(n) + w(n) - n - 1$ and for $w^{-1}(n) - w^{-1}(1) \geq i \geq w^{-1}(n) + w(n) - n$ we have $w(w^{-1}(1) + i) = i + n - w^{-1}(n) + w^{-1}(1).$  
%\par Combining these, we have the following:
% $$w(i) = \begin{cases}
% w(n) + i & \text{ if } i \leq w^{-1}(1) - 1 \\ i + 1 - w^{-1}(1) & \text{ if } w^{-1}(1) \leq i \leq w^{-1}(1) + w^{-1}(n) + w(n) - n - 1\\ i - w^{-1}(n) + n & \text{ if } w^{-1}(1) + w^{-1}(n) + w(n) - n \leq i \leq w^{-1}(n) \\ i - n + w(n) & \text{ otherwise} \end{cases},$$
\end{proof}

\begin{replemma}{lem:simple-no-internal-pair}
Suppose that $a, b, c$ satisfy $0 \le a,b,c \le n$ and $a + b + c = 2n.$ There is a unique coloring of the vertices in $[2n]$ and a unique compatible non-crossing matching such that vertices $[a+1, a+b]$ are colored black, vertices $[a+b+1, 2n]$ are colored white, and there do not exist internal pairings within $[1, a]$ (i.e. there do not exist $1 \le i < j \le a$ that are paired).
\end{replemma}

\begin{proof}
We induct on $a$. For the base case $a = 0$, we have $b = c$ and the unique matching is given by pairing $x$ and $2n-x$ for all $1 \le x \le n$. For the inductive step, suppose the result is true for $a-1$. We may assume $b \le c$; otherwise, relabel the vertices $x \to a+1 - x \pmod {2n}$ and flip black/white. Then $a \ge 1$ implies $2b \le b+c \le 2n-1$, so $b \le n-1$.

We claim that $1$ is paired with $2n$. Suppose not; then $1$ is paired with $x$ for some $a+1 \le x \le 2n-1$. Among the vertices in $[x+1, 2n]$, there must be equal numbers of black vertices and white vertices. We now show this can't be the case. If $x \ge a+b$, then all the vertices are black. If $a+1 \le x < a+b$, then there are $c$ white vertices and at most $b-1$ black vertices. Thus, in fact $1$ must be paired with $2n$, and $1$ must be colored black. If we remove $1$ and $2n$, then the remaining configuration exhibits a coloring and compatible non-crossing matching. By the inductive hypothesis (note that $0 \le a-1, b, c-1 \le n-1$), this coloring and compatible non-crossing matching are unique.
\end{proof}
% \begin{replemma}{Rectangle Matching}
% Suppose that $w$ is a $321-, 1324-, 2143-$avoiding permutation. Write the sum of complementary minors given in Proposition \ref{Rectangle Complementary Minors} as a combination of Temperley-Lieb immanants. Then, the coefficient of $\imm_w$ is $\pm 1.$
% \end{replemma}
% \begin{proof}
% We consider the coefficient of $\imm_w$ in the sum by finding which complementary minors have a coloring consistent with the non-crossing matching corresponding to $w.$ Consider a coloring given in \ref{lem:matching_compatibility} compatible with $w$'s non-crossing matching (by picking $i$ to be colored white if $w(i) = i$). We can take the second block of unprimed indices, namely those where $w^{-1}(1) \leq i \leq w^{-1}(1) + w^{-1}(n) + w(n) - n - 1,$ and have them colored white, meaning that each of these is paired with an index with an index no larger than it (either primed or unprimed). In particular, each of these elements in paired with something in the set $\{1, 2, \ldots, w^{-1}(1) - 1, 1', 2', \ldots, w(n)'\}$ (the second and fourth block of outputs, or the first block of inputs, which are colored black).
% \par However, notice that in each of the complementary minors we have, $1', 2', \ldots, w(n)'$ are colored white, as are $1, 2, \ldots, w^{-1}(1) - 1$ (as they cannot lie in $I$). But this means that, per the above logic, if the complementary minor $\cm_{I, [1, w(n)]}$ produces a coloring consistent with the non-crossing matching for $w,$ that $w^{-1}(1) \leq i \leq w^{-1}(1) + w^{-1}(n) + w(n) - n - 1$ are all colored black. But we also need for $[w^{-1}(n) + 1, n]$ to be colored black, which fixes the coloring of $w^{-1}(n) + w(n) - n + n - w^{-1}(n) = w(n),$ the number of elements in $I.$ In particular, there is at most one such $I$ for which this is the case.
% \par To see that the coefficient is $\pm 1,$ notice that the $I$ that we described is precisely the second and fourth block of inputs being colored black, as are the second and fourth block of outputs being colored white. But this is the exact reverse of the coloring we described from Lemma \ref{lem:matching_compatibility} (which is compatable with the non-crossing matching for $w$), so the coloring associated to this $I$ is also compatible. Thus, there is exactly one complementary minor that has $\imm_w$ when expanded as a sum of Temperley-Lieb immanants, proving the proposition.
% \end{proof}
\begin{replemma}{w(1) and w(n) inequality for 2143}
Suppose that $w$ contains $2143$.
\begin{enumerate}[(a)]
    \item If $w$ avoids $321$, then $w(1) < w(n)$ and $w^{-1}(1) < w^{-1}(n)$.
    
    \item If $w$ avoids $321$, then $w^{-1}(1) + w(1) \le n + 1$ and $(n+1 - w^{-1}(n)) + (n+1 - w(n)) \le n+1$.
    
    \item If $w$ avoids $1324$, then $w(1) \neq 1$ and $w(n) \neq n.$
\end{enumerate}
%Suppose that $w$ avoids $321,$ but not $2143.$ Then,  
\end{replemma}
\begin{proof}
(a) We first show that $w(1) < w(n)$. Since it contains a $2143$ pattern, we can find $a < b < c < d$ such that $w(b) < w(a) < w(d) < w(c)$. Then $w(1) \le w(a)$ (otherwise $1, a, b$ would be a $321$-pattern) and $w(n) \ge w(d)$ (otherwise $c, d, n$ would be a $321$-pattern), so $w(1) \le w(a) < w(d) \le w(n)$.

Now $w^{-1} (1) < w^{-1} (n)$ follows from applying the first part of (a) to $w^{-1}$, which also avoids $321$ and contains $2143$ by Lemma \ref{lem: restriction inverses}.

(b) Note that $w(i) > w(1)$ for $2 \le i \le w^{-1} (1) - 1$ and $i = n$ by $321$-avoidance and part (a). But there are exactly $n-w(1)$ many $i$ with $w(i) > w(1)$, which shows that $w^{-1} (1) - 1 \le n - w(1)$ and hence $w(1) + w^{-1} (1) \le n+1.$

Then $(n+1 - w^{-1}(n)) + (n+1 - w(n)) \le n+1$ follows from applying the first part of (b) to $w_0 w w_0$, which also avoids $321$ and contains $2143$ by Corollary \ref{cor: w_0 conjugation pattern avoidance}.

(c) We first show $w(1) \neq 1$. If $w(1) = 1$, then there exist $1 < a < b < c < d$ such that $w(b) < w(a) < w(d) < w(c)$. Then $1 < a < b < c$ satisfy $w(1) < w(b) < w(a) < w(c)$, which contradicts $w$ being $1324$ avoiding. Thus, $w(1) \neq 1$, and $w(n) \neq n$ follows by applying the first part of (c) to $w_0 w w_0$, which also avoids $1324$ and contains $2143$ by Corollary \ref{cor: w_0 conjugation pattern avoidance}.
\end{proof}

\begin{replemma}{lem:corner zero}
Suppose $w$ avoids $321, 1324$ and contains $2143$, and $\imm_w$ is a \%-immanant $\imm^\%_{\lambda/\mu}$ up to sign. Then:
\begin{enumerate}[(a)]
    \item $(1, 1), (n, n) \notin \lambda/\mu,$
    
    \item There exists $i$ such that $(1, i), (n+1-i, 1), (n, i), (n+1-i, n) \in \lambda/\mu$.
\end{enumerate}
\end{replemma}

\begin{proof}
(a) By Lemma \ref{w(1) and w(n) inequality for 2143}(c), we have $w(1) \neq 1$. If $(1, 1) \in \lambda/\mu$, then consider $w' = w \cdot (1, w^{-1} (1))$. Then since $(1, w^{-1} (1))$ is an inversion of $w'$, we have $w' \not\ge w$ by Corollary \ref{cor:bruhat_inv}. Thus, by Lemma \ref{lem:basic fwu}, we have $f_w (w') = 0$ and $f_w (w) = 1$. However, note that if the coefficient of $x_w$ is nonzero, then in particular we require that $n \leq \lambda_{w^{-1}(n)}$ and $\mu$ to be empty, since $\mu_1 = 0$ by $\lambda/\mu$ containing $(1, 1).$ By Lemma \ref{w(1) and w(n) inequality for 2143}(a), we have that $w^{-1}(1) < w^{-1}(n),$ and so $\lambda_{w^{-1}(1)} = n.$ But then the coefficients of $x_w$ and $x_{w'}$ in $\imm^\%_{\lambda/\mu}$ are negatives of each other, since $w'(i) \leq \lambda_i$ for $i \in [n]$ holds if and only if $w(i) \leq \lambda_i$ for $i\in [n].$ Thus, $(1, 1) \notin \lambda/\mu$, and an analogous argument shows $(n, n) \notin \lambda/\mu$.

(b) Choose $i = \max(w(1), n+1-w^{-1}(n))$. We first observe that since $f_w (w) = 1$, we must have $(j, w(j)) \in \lambda/\mu$ for all $1 \le j \le n$. Thus, since $i \ge w(1)$, we must have $(1, i) \in \lambda/\mu$. Similarly, since $n+1-i \le w^{-1} (n)$, we must have $(n+1-i, n) \in \lambda/\mu$.

\par Now, for $(n, i),$ observe that either $i = w(1) < w(n),$ or $i = n + 1 - w^{-1}(n) \leq w(n)$ by Lemma \ref{w(1) and w(n) inequality for 2143}(a,b). In either case we see that $(n, i) \in \lambda/\mu.$ Finally, we observe that in $(n + 1 - i, 1),$ either $n + 1 - i = w^{-1}(n) > w^{-1}(1),$ or $n + 1 - i = n + 1 - w(1) \geq w^{-1}(1),$ so again in either case $(n+1 - i, 1) \in \lambda/\mu.$ This proves the claim.
% since $w(1) \le w(n)$ and $w(1) \le n+1 - w^{-1} (1)$ (Lemma \ref{w(1) and w(n) inequality for 2143}(b)), we have $w(1) \le i$, and so $(1, i) \in \lambda/\mu$. Similarly, $(n+1-i, n) \in \lambda/\mu$. For $(n, i),$  
\end{proof}

\section{Temperley-Lieb Immanants as Linear Combinations of \%-Immanants}\label{sec:general_TL}
This section is devoted to proving the following theorem. 
\begin{thm}\label{TwoPercentForward}
Let $w$ be a $321$-avoiding permutation. The following statements are equivalent:
\begin{enumerate}
    \item The Temperley-Lieb immanant $\imm_w$ is a linear combination of \%-immanants;
    
    \item The signed Temperley-Lieb immanant $\sgn(w) \imm_w$ is a sum of at most two \%-immanants;
    
    \item The permutation $w$ avoids the patterns $1324, 24153, 31524, 231564$, and $312645,$ in addition to avoiding $321.$
\end{enumerate}
\end{thm}

%If a permutation $w$ is $321$-avoiding, $24153-$avoiding, $31524-$avoiding, $231564-$avoiding, and $312645-$avoiding, then $w$ can be written as the sum of at most two \% immanants.
To prove the theorem, we first classify permutations $w$ that avoid $321$ and $1324$ but contain $2143.$ Then, as in Section \ref{sec:specific-TL}, we will use Proposition \ref{prop:cm equals imm sum} to express $\imm_w$ as an explicit linear combination of certain complementary minors. Finally, we compute the coefficients $f_w (u)$ of the immanant $\imm_w$, which amounts to counting the complementary minors in the sum that have nonzero $x_u$ term.
\subsection{Classifying 321-, 1324-avoiding, 2143-containing permutations}
We will show such permutations satisfy one of two prescribed block structures.
\begin{prop}\label{2143Patterns}
Let $w\in \mathfrak{S}_n$ be a permutation that avoids $321$, $1324$, and contains $2143$. 

Let $a' = w^{-1}(1) - 1, b' = w(1) - 1, c' = n - w(n), d' = n - w^{-1}(n)$. We reserve letters $a, b, c, d$ for the size of finer blocks. We will see that in one case, we will have blocks whose sizes are $a', b', c', d'$ respectively. In the other case, we will break up $a', b', c', d'$ into six values which will form the lengths of our blocks.

\begin{enumerate}
    \item If $a' + b' + c' + d' \leq n,$ let $a = a', b = b', c = c', d = d',$ and let $e=n-a-b-c-d \ge 0$. Then the one-line notation of $w$ is
    \begin{equation*}
        (b+1..b+a : 1..b : b+a+1..b+a+e : n-c+1..n : n-c-d+1..n-c).
    \end{equation*}
In this case, we see $w$ has block structure $[2][1][3][5][4]$ with block lengths $a, b, e, c, d$ and has fixed points at every $i \in [b+a+1, b+a+e]$. Only the $[3]$ block is allowed to be empty.

\item Otherwise, there exist integers $1 \le a, b, c, d \le n$, $0 \le e, f \le n$ with $\max(e, f) \ge 1$, such that $a + e = a', b + f = b', c + e = c'$, $d + f = d'$, $a+b+c+d+e+f=n$, and the one-line notation of $w$ is
\begin{equation*}
    (b+f+1..b+f+a : n-c-e+1..n-c : 1..b : n-c+1..n : b+1..b+f : n-d-c-e+1..n-c-e).
\end{equation*}
In this case, $w$ has block structure $[3][5][1][6][2][4]$ with block lengths $a, e, b, c, f, d$, and the middle two blocks do not contain any numbers in $[b'+1, n-c']$. Only one of the $[5]$ or $[2]$ blocks are allowed to be empty.
\end{enumerate}
\end{prop}

In fact, if $w$ is in the second case, then it must contain one of two prescribed patterns.
\begin{lemma}\label{lem:24153, 31524 pattern}
Suppose that $w$ is a permutation avoiding $1324, 321$ that contains the pattern $2143,$ and define $a', b', c', d'$ as in Proposition \ref{2143Patterns}. Suppose that $a' + b' + c' + d' > n.$ Then, $w$ either contains the pattern $24153$ or $31524.$
\end{lemma}
%To prove this theorem, we first prove a general proposition.
% \begin{prop}\label{2143Patterns}
% Let $w\in \mathfrak{S}_n$ be a permutation that avoids $321$, $1324$, and contains $2143$. 

% Let $a' = w^{-1}(1) - 1, b' = w(1) - 1, c' = n - w(n), d' = n - w^{-1}(n);$ We reserve letters $a, b, c, d$ for the size of finer blocks. We will see that in one case, we will have blocks whose sizes are $a', b', c', d'$ respectively. In the other case, we will break up $a', b', c', d'$ into six values which will form the lengths of our blocks.

% \begin{enumerate}
%     \item If $a' + b' + c' + d' \leq n,$ let $a = a', b = b', c = c', d = d',$ and let $e=n-a-b-c-d$. We have that the one line notation of $w$ has at most 5 ascending strings of consecutive integers given as follows.

% \begin{itemize}
%     \item Block 1 consists of $a$ integers $w(i)=b+i$ for $i\in [1,a].$
%     \item Block 2 consists of $b$ integers $w(i)=i-a$ for $i\in [a+1, b+a].$
%     \item Block 3 consists of $e$ integers $w(i)=i$ for $i\in [b+a+1,b+a+e].$
%     \item Block 4 consists of $c$ integers $w(i)=i+d$ for $i\in [b+a+e+1,b+a+e+c].$
%     \item Block 5 consists of $d$ integers $w(i)=i-c$ for $i\in [b+a+e+c+1,n].$
    
% \end{itemize}  
% In this case, we see $w$ has block structure $[2][1][3][5][4]$.

% \item Otherwise, the one line notation of $w$ has at most six ascending strings of consecutive integers, whose lengths are $a, e, b, c, f, d$ in that order, where $a + e = a', b + f = b', c + e = c',$ and $d + f = d'.$ Furthermore, the middle two strings do not contain any numbers in $[b'+1, n-c'],$ and $w$ has block structure $[3][5][1][6][2][4].$
% \end{enumerate}
% \par Note that some of the blocks in the above cases are allowed to be empty.
% \end{prop}

% \todo{introduce "block structure in section 2"}

\begin{proof}[Proof of Proposition \ref{2143Patterns}]
%\par We first extract information from the pattern avoidance conditions.% First notice that none of the values $a', b', c', d'$ can be zero by Lemma \ref{w(1) and w(n) inequality for 2143}, and furthermore that $w(1) < w(n), w^{-1}(1) < w^{-1}(n).$ In particular, we have that $n - d' > a' + 1$ and $n - c' > b' + 1.$

\par We first extract information from the pattern avoidance conditions on $w$. 
\begin{itemize}
    \item \textbf{Observation 1.} Since $w$ avoids $321, 1324$ and contains $2143$, by Lemma \ref{w(1) and w(n) inequality for 2143}(a,c) we have that $a', b', c', d' \neq 0$, $n - d' > a' + 1$ and $n - c' > b' + 1$.
    
    \item \textbf{Observation 2.} Since $w$ avoids $321$, we have $w([1, a']) \subset [b'+1, n]$ and $w([n-d', n]) \subset [1, n-c']$. For instance, to prove the first claim, notice that if $1 \leq y \leq a'$ satisfies $w(y) \leq b',$ then $1, y, a' + 1$ forms a $321$-pattern. (See also Proposition \ref{321-avoiding Rectangles}.)

    \item \textbf{Observation 3.} Since $w$ avoids $1324$, we have that the following six sequences are increasing:
    \begin{equation*}
        w(1..a'), w(a'+1..n-d'), w(n-d'+1..n), w^{-1}(1..b'), w^{-1} (b'+1..n-c'), w^{-1} (n-c'+1..n).
    \end{equation*}
    For example, if the first sequence $(w(1), w(2), \cdots, w(a'))$ is non-increasing, say we have $1 \le x < y \le a'$ such that $w(x) > w(y)$, then Observation 2 tells us that $x > 1$, and then $1, x, y, n-d'$ forms a $1324$-pattern.

    \item \textbf{Observation 4.} Since $w$ avoids $1324$, there cannot exist $y \in [1, a' : n-d'+1, n]$ and $x \in [a'+1, n-d']$ such that $w(y) \notin [b'+1, n-c']$ and $w(x) \in [b'+1, n-c']$. For if say $y \in [1, a']$, then $w(y) \ge b' + 1$ by Observation 2, and so $w(y) \ge n-c' + 1$. Then $1, y, x, n-d'$ form a $1324$-pattern. A similar contradiction holds if $y \in [n-d'+1, n]$.
\end{itemize}

\par Now, we begin the proof of the proposition in earnest. Let $a \le a'$ be the largest value such that $w(a) \le n-c'$. Then $[1, a] \subset w^{-1} ([b'+1, n-c'])$ and $w^{-1} (b'+1..n-c')$ is increasing by Observation 3, so $(1..a) = w^{-1} (b'+1..b'+a)$ and $w(1..a) = (b'+1..b'+a)$. Now by maximality we have $w(x) > n-c'$ for $a < x \le a'$, so $[a+1, a'] \subset w^{-1} ([n-c'+1, n])$. Since $w^{-1}(n-c'+1..n)$ is increasing, we have $(a+1..a') = w^{-1} (n-c'+1..n-c'+a'-a)$. Thus, $w(a+1..a') = (n-c'+1..n-c'+a'-a)$. In conclusion, so far we have determined the values $w(1)$ through $w(a')$.
\par We can also similarly define $d \le d'$ to be the largest value such that $w(n-d) \ge b'+1$. Then an analogous argument shows that $w(n-d+1..n) = (n-d-c'+1..n-c')$ and $w(n-d'+1..n-d) = (b'-(d'-d)+1..b')$.
\par Since $w([1, a]) = [b'+1,b'+a]$ and $w([n-d+1,n]) = [n-d-c'+1,n-c']$ are disjoint and $b'+1 < n-c'$ by Observation 1, we must have $b'+a \le n-d-c'$.
\par Finally, it suffices to determine the sequence $w(a'+1..n-d')$. By Observation 3, the sequence is increasing and consists of the elements in the set $S = [n] \setminus w([1,a';n-d'+1,n]).$ It will simplify matters to define $b := b' - (d'-d)$ and $c := c' - (a'-a)$. From our computation of $w(1..a':n-d'+1..n)$, we see that $S = [1, b : b'+a+1, n-d-c' : n - c + 1 : n]$. Thus, we can compute $w(a'+1..a' + b) = (1.. b),$
$w(a' + b+1..n-d'-c) = (b'+a+1..n-d-c')$, and
$w(n-d'-c+1..n-d') = (n-c+1..n)$. Since $w(a'+1) = 1$, we must have $b > 0$; similarly, $c > 0$.
We divide into two cases.
\par \textbf{Case 1.} Suppose $a = a'$ and $d = d'$; then $b = b'$, $c = c'$. If we define $e = n - a - b - c - d$, then we get our desired result. In particular, $a' + b' + c' + d' = n-e \le n$.
\par \textbf{Case 2.} Suppose $a < a'$ or $d < d'$. Then at least one of $w(a')$ or $w(d')$ does not lie in $[b'+1, n-c']$, so by Observation 4, we have $w([a'+1, n-d']) \cap [b'+1, n-c'] = \emptyset$. But we know that $w([a'+b+1, n-d-c']) = [b'+a+1,n-d-c']$, which will be a contradiction unless both sides are empty. Thus, $a+b' \ge n-d-c'$, and so $a+b' = n-d-c'$. If $e := a' - a$, $f := d' - d$, then we get $b := b' - f$, and $c := c' - e$, and we recover our desired result. Finally, we note that $a' + b' + c' + d' > a + b' + c' + d = n$.

Thus, we have showed Case 1 happens if and only if $a' + b' + c' + d' \le n$. This completes the proof of the proposition.

%\par We have found four of the six blocks. The remaining two blocks are easy since $w([a'+1, n-d'])$ is increasing.

%\par Now, we begin the proof of the proposition in earnest. We divide the proof into cases as indicated in the statement of the proposition.

% \par \textbf{Case 1.} Suppose that $a' + b'
% + c' + d' \leq n$. Then, we drop the primes and define $a = a', b = b', c = c', d = d'$. Observe that $w([1, a: n - d + 1, n])$ is a set of $a+d$ elements. We claim that it must be a subset of $[b+1, n - c]$. 
% \par To see this, suppose for the sake of contradiction that there existed a $y \in [1, a: n - d + 1, n]$ that does not lie in $w^{-1}([b+1, n-c]).$ Notice that this set contains $n - b - c$ elements. However, by assumption, $n - b - c \geq a + d,$ so $w^{-1}([b+1, n-c])$ cannot be a proper subset of $[1, a: n - d + 1, n].$ Thus, $w^{-1}([b+1, n - c])$ must contain some element $x$ not inside $[1, a] \cup [n - d + 1, n].$ But then $y, x$ will be a contradiction to Observation 4.
% %\par Next, since $w$ is $321$-avoiding, note that $w([1, a]) \subset [b+1, n]$ and $w$ must be increasing on $[1, a].$ Otherwise, if $x, y \in [1, a]$ and $y < x, w(x) < w(y),$ then $y < x < a + 1$ but $w(a+1) = 1 < w(x) < w(y),$ so $y, x, a+1$ forms a $321$ pattern. Similarly, we have $w([n-d+1, n]) \subset [1, n - c],$ and $w$ increases on $[n-d+1, n].$  However, note by assumption that either $y \in [1, a],$ so $1 < y < x < n-d,$ but $b+1 < w(x) < w(y) < n,$ contradicting the fact it is $1324$-avoiding, or $y \in [n-d+1, n],$ so $a+1 < x < y < n$ but $1 < w(y) < w(x) < n-c,$ another contradiction. 
% \par Thus, we require $w([1, a] \cup [n - d + 1, n]) \subset [b+1, n - c]$. In particular, $[1, a] \subset w^{-1} ([b+1, n - c])$. But by Observation 3, $w^{-1} ([b+1, n - c])$ is increasing, so $[1, a] = w^{-1} ([b+1, a+b])$, and so $w([1, a]) = [b + 1, a + b]$. A similar argument gives $w([n-d'+1, n]) = [n-c'-d'+1, n-c']$.

% \par Now since $w([a'+1, n-d'])$ is increasing by Observation 3, they are completely determined. The explicit form is provided in the statement of Case 1 in the Proposition.
% %Notice that $w$ being $1324$-avoiding also implies that $w([1, a]) = [b + 1, a + b].$ If this wasn't the case, let $x$ be the smallest value in $[1, a - 1]$ so $w(x+1) > w(x) + 1.$ Then, $1, x + 1, w^{-1}(w(x) + 1), n$ would form the $1324$-pattern. By the same logic, we must also have that $w([n-d+1, n]) = [n-d-c+1, n-c].$ 
% %\par Notice that we can employ a similar argument as the above with $w^{-1}([1, b], [n-c+1, n])$ to find that $w^{-1}([1, b]) = [a+1, a+b],$ $w^{-1}([n-c+1, n]) = [n - c - d + 1, n - d],$ and that $w^{-1}$ is increasing on the intervals $[1, b]$ and $[a+1, a+b].$ But then this specifies all of the values in $[1, n]$ except for those in $[a + b + 1, n - c - d],$ which must take on values in $[a + b + 1, n - c - d],$ if such values exist. Finally, notice that $1324$-avoiding implies that $w$ must be increasing on the interval, and so we need $w(i) = i$ for $i \in [a + b + 1, n - c - d],$ which gives the first part of the claim.
% \par \textbf{Case 2.} In this case, we have $a' + b' + c' + d' > n,$ meaning that $w([1, a' : n - d' + 1, n])$ cannot lie in $[b' + 1, n-c'].$ Therefore, there exists some $y$ in $[1, a' : n - d' + 1, n]$ so that $w(y) \not \in [b+1, n-c]$. By Observation 4, we thus have $w([a'+1, n-d']) \cap [b+1, n-c] = \emptyset$.
% \par Finally, let $a \le a'$ be the largest value such that $w(a) \le n-c'$. Then $[1, a] \subset w^{-1} ([b'+1, n-c'])$ and $w^{-1} ([b'+1, n-c'])$ is increasing, so $[1, a] = w^{-1} ([b'+1, b'+a])$. Now by maximality we have $w(x) > n-c'$ for $a < x \le a'$, so $[a+1, a'] \subset w^{-1} ([n-c'+1, n])$. Since $w^{-1} ([n-c'+1, n])$ is increasing, we have $[a+1, a'] = w^{-1} ([n-c'+1, n-c'+e])$ where $e := a' - a$. Thus, $w([a+1, a']) = [n-c'+1, n-c'+e]$.
% \par We can also similarly define $d \ge d'$ to be the smallest value such that $w(d) \ge b'+1$. Then an analogous argument shows that $w([n-d+1, n]) = [n-d'-c'+1, n-c']$ and $w([n-d', n-d]) = [b'-f+1, b']$, where $f := b' - b$.
% \par We have found four of the six blocks. The remaining two blocks are easy since $w([a'+1, n-d'])$ is increasing.
%is increasing, there can only be at most one value of $a$ in $[1, a']$ so that $w(a+1) > w(a) + 1,$ and it can't be less by the fact that $w$ is $1324$-avoiding. Indeed, if there were two, with the first being $x$ and the second being $y,$ then observe that either $w(x+1) \leq n - c$ or $w(y) > n - c'.$ In the first case, the values $1, x+1, w^{-1}(w(x) + 1), n$ forms the pattern; in the latter the values $1, y+1, w^{-1}(w(y) + 1), w^{-1}(n)$ form the $1324$-pattern. A similar argument holds on the interval $[n - d' + 1, n].$ 
%\par Thus, since $w$ is $1324$- and $321$- avoiding, $w$ has at most six blocks $$w([1, a]), w([a+1, a+e]), w([a+e+1, a+e+b]),$$ $$w([a+e+b+1, a+e+b+c]), w([a+e+b+c+1, a+e+b+c+f]), w([a+e+b+c+f+1, n]),$$ where all but the second and fifth are guaranteed to be non-empty. Also, $w(a+e+1) = w(a'+1) = 1, w(a+e+b+c) = w(n - d') = n,$ so the $w$ applied to the third interval consists of the smallest set of consecutive numbers ($w$ applied to each other interval yields larger values), and $w$ applied to the fourth consists of the largest set of consecutive numbers. This is enough to prove the proposition.
\end{proof}

% \begin{proof}
% We have $a + b + c + d > n.$ First, notice that $[1, b] \cap w^{-1}([a + 1, n - d])$ and $[b+1, n-c] \cap w^{-1}([n-d+1, n])$ have to be so that one of these sets have size one. Otherwise, similarly to the first case, we have a $231564$ pattern. A similar argument holds for $[b+1, n-c] \cap w^{-1}([1, a-1])$ and $[n-c+1, n] \cap w^{-1}([a+1, n-d]).$ 
% \par Notice also that $[1, b] \cap w^{-1}([a + 1, n - d]), [1, b] \cap w^{-1}([n-d+1, n])$ are two disjoint sets whose sum of cardinality is $b,$ and $[b+1, n-c] \cap w^{-1}([n-d+1, n]), [1, b] \cap w^{-1}([n-d+1, n])$ are two disjoint sets whose sum of cardinality is $d;$ therefore we see that $d \geq b.$  \par Notice, however, that if $a + b + c + d > n,$ then we have that $w([1, b] \cup [n-c+1, n])$ has $b + c$ elements, which is more than $n - a - d,$ or the number of elements in $[a + 1, n-d].$ Therefore, either we have an element $x \in [1, b]$ so $w(x) > n - d,$ or we have an $x \in [n - c + 1, n]$ so $w(x) < a + 1.$ 
% \par In the first instance, notice that $1, x, b, n-c, n$ yields us with $1 = w(b) < w(1) = a + 1 < n - d = w(n) < w(x) < w(n-c) = n,$ and the second instance yields us with $1, b, n-c, x, n$ being so that $1 = w(b) < w(x) < w(1) = a + 1 < w(n) = n - d < w(n-c) = n.$ But the first is a $24153$ pattern and the second is a $31524$ pattern, meaning that we do not actually have this case. This proves the desired theorem.
% \end{proof}

% \begin{remark}
% These two orderings will be heavily and implicitly used throughout the rest of this section.
% \end{remark}

%We are now ready to begin the proof of the theorem.


%In order to show that the converse of this statement holds, we do the following: given a $321-,1324-$avoiding permutation $w$ that contains $2143$, we give an explicit expression of $\imm_w$ as a linear combination of certain complementary minors. 

%In order to compute the coefficients $f_w (u)$ for a $321, 1324$-avoiding, $2143$-containing permutation $w$, we will first express $\imm_w (u)$ as a sum of complementary minors. This is motivated by the following consideration: Proposition \ref{prop:cm equals imm sum} tells us that $\cm_{I,J}$ is a sum of Temperley-Lieb immanants for each $I, J \subset [n]$, and thus we can ``solve'' for a particular Temperley-Lieb immanant as a linear combination of some complementary minors.

%\subsection{Unique Non-crossing Matchings for Partial Colorings}
\subsection{Temperley-Lieb Immanants as a Sum of Complementary Minors}\label{CompliMinors}
% \par Our main goal in this subsection is to obtain a relatively simple formula for $f_w(v)$ for $321-, 1324-$avoiding permutations $w$ with the $2143$-pattern. In particular, this formula will take the form $\sgn(w)\sgn(v) {A + B \choose A}.$ Our main method for arriving at this formula will be to find a good way to express the immanants $\imm_w$ in terms of products of complementary minors. 
\par Let $w \in \fkS_n$ be a permutation that avoids $321$ and $1324$ but contains $2143$. Our main goal in this subsection is to express $\imm_w$ as an explicit sum of certain complementary minors. To give this explicit expression of complementary minors, we split up our argument into two cases, along the lines of the cases provided in Proposition \ref{2143Patterns}. For each of these cases, we closely follow the method in Proposition \ref{Rectangle Unique Matching}. We first prove that a unique non-crossing matching exists for a given condition of colorings (Lemmas \ref{non-crossing matching of case1} and \ref{non-crossing matching of case2}), and then we show that the unique non-crossing matching corresponds to $w$ (Lemmas \ref{ncm equals w case1} and \ref{ncm equals w case2}). Thus, we are able to construct an explicit set $\cS$ of colorings such that for each $u \neq w$, there exists a ``sign-reversing'' involution $\iota_u$ on $\cS$. Finally, we will construct a special linear combination of complementary minors corresponding to colorings in $\cS$ that equals our desired $\imm_w$. To prove the equality, we use $\iota_u$ to pair up opposite terms to generate cancellation (Propositions \ref{sumcm5} and \ref{sumcm6}).

First, we present the following general lemma concerning non-crossing matchings:
\begin{lemma}\label{general non-crossing matching}
Let $a, b, c, d, e \ge 0$ be integers with $n = a + b + c + d + e$. Consider vertices labelled $1, 2, \ldots, 2n.$ Then there exists a unique coloring of these vertices using the colors white and black and a non-crossing matching compatible with this coloring such that the following conditions hold. 
\begin{enumerate}
    \item $[1, b+c+e]$ are colored black,
    
    \item in the interval $[b+c+e+1, a+2b+c+e]$, there are exactly $a$ black and $b$ white vertices, and there are no pairings between two vertices in this interval (which we will refer to as an ``internal pairing''),
    
    \item $[a+2b+c+e+1, a+b+e+n]$ are colored white,
    
    \item in the interval $[a+b+e+n+1, 2n]$, there are exactly $d$ black and $c$ white vertices, and there are no internal pairings in this interval.
\end{enumerate}
\end{lemma}
As an example, this is the case where $a = 2$ and $b = c = d = e = 1,$ with the red lines dividing the four different ranges of indices with the different conditions. 
\begin{center}
    %\includegraphics[scale=0.3]{lemma 4.4.png}
    \includegraphics[scale=0.7]{output27.pdf}
\end{center}
% \begin{center} 
%     \begin{asy} %[scale = 0.7]
%         import graph;
%         pair A1, A2, A3, A4, A5, A6, A7, A8, A9, A10, A11, A12;
%         A1 = (100*cos(0), 100*sin(0));
%         A2 = (100*cos(pi/6), 100*sin(pi/6));
%         A3 = (100*cos(pi/3), 100*sin(pi/3));
%         A4 = (100*cos(pi/2), 100*sin(pi/2));
%         A5 = (100*cos(2*pi/3), 100*sin(2*pi/3));
%         A6 = (100*cos(5*pi/6), 100*sin(5*pi/6));
%         A7 = (100*cos(pi), 100*sin(pi));
%         A8 = (100*cos(7*pi/6), 100*sin(7*pi/6));
%         A9 = (100*cos(4*pi/3), 100*sin(4*pi/3));
%         A10 = (100*cos(3*pi/2), 100*sin(3*pi/2));
%         A11 = (100*cos(5*pi/3), 100*sin(5*pi/3));
%         A12 = (100*cos(11*pi/6), 100*sin(11*pi/6));
%         filldraw(Circle(A1, 2), black, black);
%         label("$1$", A1, unit(A1));
%         filldraw(Circle(A2, 2), black, black);
%         label("$2$", A2, unit(A2));
%         filldraw(Circle(A3, 2), black, black);
%         label("$3$", A3, unit(A3));
%         draw((80*cos(5*pi/12), 80*sin(5*pi/12))--(120*cos(5*pi/12), 120*sin(5*pi/12)), red);
%         filldraw(Circle(A4, 2), white, black);
%         label("$4$", A4, unit(A4));
%         filldraw(Circle(A5, 2), black, black);
%         label("$5$", A5, unit(A5));
%         filldraw(Circle(A6, 2), black, black);
%         label("$6$", A6, unit(A6));
%         draw((80*cos(11*pi/12), 80*sin(11*pi/12))--(120*cos(11*pi/12), 120*sin(11*pi/12)), red);
%         filldraw(Circle(A7, 2), white, black);
%         label("$7$", A7, unit(A7));
%         filldraw(Circle(A8, 2), white, black);
%         label("$8$", A8, unit(A8));
%         filldraw(Circle(A9, 2), white, black);
%         label("$9$", A9, unit(A9));
%         filldraw(Circle(A10, 2), white, black);
%         label("$10$", A10, unit(A10));
%         draw((80*cos(19*pi/12), 80*sin(19*pi/12))--(120*cos(19*pi/12), 120*sin(19*pi/12)), red);
%         filldraw(Circle(A11, 2), black, black);
%         label("$11$", A11, unit(A11));
%         filldraw(Circle(A12, 2), white, black);
%         label("$12$", A12, unit(A12));
%         draw((80*cos(23*pi/12), 80*sin(23*pi/12))--(120*cos(23*pi/12), 120*sin(23*pi/12)), red);
%         draw(A1..(75*cos(23*pi/12), 75*sin(23*pi/12))..A12);
%         draw(A3..(75*cos(5*pi/12), 75*sin(5*pi/12))..A4);
%         draw(A6..(75*cos(11*pi/12), 75*sin(11*pi/12))..A7);
%         draw(A5..(55*cos(11*pi/12), 55*sin(11*pi/12))..A8);
%         draw(A10..(75*cos(19*pi/12), 75*sin(19*pi/12))..A11);
%         draw(A2..(30*cos(21*pi/12), 30*sin(21*pi/12))..A9);
%     \end{asy}
% \end{center}
\begin{remark}
We represent this lemma by drawing these vertices on a circle to illustrate the symmetry. When we then apply this lemma, we will pick two ways of choosing a diameter and letting the vertices on each side of the diameter form a column of $n$ vertices, consistent with the pictures we have previously shown to represent the non-crossing matchings.
\end{remark}

%Before arriving at an expression for $\imm_w$ as a linear combination of complementary minors, we first
\par We specialize Lemma \ref{general non-crossing matching} to our situation. The next two lemmas may seem mysterious at first sight; a curious reader can turn to the discussion before Lemma \ref{ncm equals w case1} for their significance.
%introduce a few lemmas that describe a particular non-crossing matching that corresponds to a particular choice of coloring. 

\begin{lemma}\label{non-crossing matching of case1}
Let $a, b, c, d, e$ be nonnegative integers, so $a, b, c, d \geq 1$ and $a+b+c+d+e=n$. Then, there is a unique non-crossing matching and a coloring compatible with it, such that the coloring satisfies
\begin{enumerate}
    \item $i$ is black for $i\in [a+1,n-d]$
    \item $i'$ is white for $i\in [b+1, n-c]$
    \item There are exactly $a$ black vertices and $b$ white vertices in $[1,a]\cup [1,b]'$
    \item There are exactly $d$ black vertices and $c$ white vertices in $[n-d+1,n]\cup [n-c+1,n]'$ 
    \item There are no pairings between two vertices in $[1,a]\cup [1,b]'$ (which we will refer to as an ``internal pairing")
    \item There are no internal pairings in $[n-d+1,n]\cup [n-c+1,n]'$
\end{enumerate}
\end{lemma}
For instance, here is a complementary coloring, with the corresponding unique matching, with $a = c = 2, e = b = d = 1.$ The boxed vertices are those that are fixed by conditions \textit{1} and \textit{2}.
\begin{center}
    \includegraphics[]{output18.pdf}
\end{center}
\begin{comment}
\begin{center}
    \begin{asy}
    import graph;
    void Example(real x){
    filldraw(Circle((x, 0),2),black,black);
    label("1", (x-5, 0), W);
    filldraw(Circle((x+100,0),2),black,black);
    label("1'", (x+105, 0), E);
    filldraw(Circle((x, -25),2),white,black);
    label("2", (x-5, -25), W);
    filldraw(Circle((x+100,-25),2),white,black);
    label("2'", (x+105, -25), E);
    filldraw(Circle((x, -50),2),black,black);
    label("3", (x-5, -50), W);
    filldraw(Circle((x+100,-50),2),white,black);
    label("3'", (x+105, -50), E);
    filldraw(Circle((x, -75),2),black,black);
    label("4", (x-5, -75), W);
    filldraw(Circle((x+100,-75),2),white,black);
    label("4'", (x+105, -75), E);
    filldraw(Circle((x, -100),2),black,black);
    label("5", (x-5, -100), W);
    filldraw(Circle((x+100,-100),2),white,black);
    label("5'", (x+105, -100), E);
    filldraw(Circle((x, -125),2),black,black);
    label("6", (x-5, -125), W);
    filldraw(Circle((x+100,-125),2),black,black);
    label("6'", (x+105, -125), E);
    filldraw(Circle((x, -150),2),white,black);
    label("7", (x-5, -150), W);
    filldraw(Circle((x+100,-150),2),white,black);
    label("7'", (x+105, -150), E);
    draw((x-5, -130)--(x-5, -40)--(x+5, -40)--(x+5,-130)--cycle); 
    draw((x+95, -105)--(x+95, -15)--(x+105, -15)--(x+105,-105)--cycle); 
    }
    Example(-150);
    Example(0);
    void MatchDraw(int n, pair[] match) {
    pair[] Coords = array(2*n, (0, 0));
    for (int i = 0; i < n; ++i){
        Coords[i] = (0,-25*i);
        Coords[2*n - 1 - i] = (100,-25*i);
    }
    for (pair z : match) {
        int index1 = (int) z.x;
        int index2 = (int) z.y;
        pair p1 = Coords[index1];
        pair p2 = Coords[index2];
        if (p1.x != p2.x) {
            pair mid = ((p1.x + p2.x)/2, (p1.y + p2.y)/2);
            draw(p1..mid..p2);
        }
        if (p1.x == p2.x) {
            if (p1.x == 0) {
                pair mid = (5*abs(index1 - index2), (p1.y + p2.y)/2);
                draw(p1..mid..p2);
            }
            if (p1.x == 100) {
                pair mid = (100 - 5*abs(index1 - index2), (p1.y + p2.y)/2);
                draw(p1..mid..p2);
            }
        }
    }
    }
    pair[] array = {(0, 11), (1, 2), (3, 10), (4, 7), (5, 6), (8, 9), (12, 13)};
    MatchDraw(7, array);
    \end{asy}
\end{center}
\end{comment}

\begin{lemma}\label{non-crossing matching of case2}
Let $a, b, c, d, e, f$ be nonnegative integers where $a, b, c, d, \max(e, f) \geq 1,$ and $a+b+c+d+e+f=n$. Then, there is a unique non-crossing matching and a coloring compatible with it, such that the coloring satisfies
\begin{enumerate}
    \item $i$ is black for $i\in [1,a+e]$
    \item $i$ is white for $i\in [a+e+b + c+1, n]$
    \item $i'$ is black for $i\in [1,b+f]$
    \item $i'$ is white for $i\in [b+f+a+d+1,n]$
    \item There are exactly $c$ black vertices and $b$ white vertices in $[a+e+1,a+e+b+c]$
    \item There are exactly $d$ black vertices and $a$ white vertices in $[b+f+1,b+f+a+d]'$ 
    \item There are no pairings between two vertices in $[a+e+1,a+e+b+c]$ (which we will refer to as an ``internal pairing")
    \item There are no internal pairings in $[b+f+1,b+f+a+d]'$
\end{enumerate}
\end{lemma}
This case has the following diagram, where $a = c = 2, b = d = e = f = 1,$ with the boxed vertices again being those fixed (this time for conditions \textit{1} to \textit{4}).
\begin{center}
    \includegraphics[]{output28.pdf}
\end{center}
% \begin{center}
%     \begin{asy}
%     import graph;
%     void Example(real x){
%     filldraw(Circle((x, 0),2),black,black);
%     label("1", (x-5, 0), W);
%     filldraw(Circle((x+100,0),2),black,black);
%     label("1'", (x+105, 0), E);
%     filldraw(Circle((x, -25),2),black,black);
%     label("2", (x-5, -25), W);
%     filldraw(Circle((x+100,-25),2),black,black);
%     label("2'", (x+105, -25), E);
%     filldraw(Circle((x, -50),2),black,black);
%     label("3", (x-5, -50), W);
%     filldraw(Circle((x+100,-50),2),white,black);
%     label("3'", (x+105, -50), E);
%     filldraw(Circle((x, -75),2),white,black);
%     label("4", (x-5, -75), W);
%     filldraw(Circle((x+100,-75),2),white,black);
%     label("4'", (x+105, -75), E);
%     filldraw(Circle((x, -100),2),black,black);
%     label("5", (x-5, -100), W);
%     filldraw(Circle((x+100,-100),2),black,black);
%     label("5'", (x+105, -100), E);
%     filldraw(Circle((x, -125),2),black,black);
%     label("6", (x-5, -125), W);
%     filldraw(Circle((x+100,-125),2),white,black);
%     label("6'", (x+105, -125), E);
%     filldraw(Circle((x, -150),2),white,black);
%     label("7", (x-5, -150), W);
%     filldraw(Circle((x+100,-150),2),white,black);
%     label("7'", (x+105, -150), E);
%     filldraw(Circle((x, -175),2),white,black);
%     label("8", (x-5, -175), W);
%     filldraw(Circle((x+100,-175),2),white,black);
%     label("8'", (x+105, -175), E);
%     draw((x-5, 10)--(x-5, -60)--(x+5, -60)--(x+5,10)--cycle); 
%     draw((x-5, -185)--(x-5, -140)--(x+5, -140)--(x+5,-185)--cycle); 
%     draw((x+95, 10)--(x+95, -35)--(x+105, -35)--(x+105,10)--cycle);
%     draw((x+95, -185)--(x+95,-115)--(x+105,-115)--(x+105,-185)--cycle); 
%     }
%     Example(-150);
%     Example(0);
%     void MatchDraw(int n, pair[] match) {
%     pair[] Coords = array(2*n, (0, 0));
%     for (int i = 0; i < n; ++i){
%         Coords[i] = (0,-25*i);
%         Coords[2*n - 1 - i] = (100,-25*i);
%     }
%     for (pair z : match) {
%         int index1 = (int) z.x;
%         int index2 = (int) z.y;
%         pair p1 = Coords[index1];
%         pair p2 = Coords[index2];
%         if (p1.x != p2.x) {
%             pair mid = ((p1.x + p2.x)/2, (p1.y + p2.y)/2);
%             draw(p1..mid..p2);
%         }
%         if (p1.x == p2.x) {
%             if (p1.x == 0) {
%                 pair mid = (5*abs(index1 - index2), (p1.y + p2.y)/2);
%                 draw(p1..mid..p2);
%             }
%             if (p1.x == 100) {
%                 pair mid = (100 - 5*abs(index1 - index2), (p1.y + p2.y)/2);
%                 draw(p1..mid..p2);
%             }
%         }
%     }
%     }
%     pair[] array = {(0, 9), (1, 8), (2, 3), (4, 7), (5, 6), (10, 11), (12, 15), (13, 14)};
%     MatchDraw(8, array);
%     \end{asy}
% \end{center}

%From here, if we are given a permutation $w$ that avoids $321$, $1324$ and contains $2143,$ then it fits in one of the two cases of Proposition \ref{2143Patterns}. In this case, the following lemmas argue that the non-crossing matching given in one of the above two lemmas (depending on which case $w$ is in) gives the matching corresponding to $w.$

\par As promised, we reveal that the non-crossing matching in Lemmas \ref{non-crossing matching of case1} and \ref{non-crossing matching of case2} actually belong to one of the $w$'s described in Proposition \ref{2143Patterns}.

\begin{lemma}\label{ncm equals w case1}
Let $w \in \fkS_n$ have block structure $[2][1][3][5][4]$ with block lengths $a,b,e,c,d$, as stated in the first case of Proposition \ref{2143Patterns}. The non-crossing matching of $w$ is exactly the non-crossing matching in Lemma \ref{non-crossing matching of case1}.
\end{lemma}

\begin{lemma}\label{ncm equals w case2}
Let $w \in \fkS_n$ have block structure $[3][5][1][6][2][4]$ with block lengths $a,e,b,c,f,d$, as stated in the second case of Proposition \ref{2143Patterns}. The non-crossing matching of $w$ is exactly the non-crossing matching in Lemma \ref{non-crossing matching of case2}.
\end{lemma}

\begin{comment}
Let $a = w^{-1} (1)-1$ be the size of the first block, $b = w(1)-1$ be the size of the second block, $e$ be the size of the third block, $c = n - w(n)$ be the size of the fourth block, and $d = n - w^{-1} (n)$ be the size of the last block.
\end{comment}

Using Lemmas \ref{non-crossing matching of case1} through \ref{ncm equals w case2}, we are now able to describe the Temperley-Lieb immanant of $w$ in terms of complementary minors. Again, we present the two cases separately.

\begin{prop}\label{sumcm5}
Let $w \in \fkS_n$ have block structure $[2][1][3][5][4]$ with block lengths $a,b,e,c,d$, as stated in the first case of Proposition \ref{2143Patterns}. Define $\cS$ to be the set of all possible colorings $(I, J)$ satisfying conditions \textit{1} through \textit{4} in Lemma \ref{non-crossing matching of case1}.
Then, we have
\begin{equation}\label{antidiagcase1}
    \imm_w = \sgn(w) (-1)^n \sum\limits_{(I,J) \in \cS} (-1)^{|I|} \cm_{I,J},
\end{equation} 
\end{prop}

\begin{remark}
Explicitly, we have
\begin{equation}\label{antidiagcase1'}
    \imm_w = \sgn(w) \sum\limits_{I_1, I_2, I_3, I_4} (-1)^{|I_1| + |I_3|} \cm_{I_1 \cup I_3, I_2 \cup I_4}.
\end{equation}
where the sum runs over all $I_1 \subset [1, a], I_2 \subset [1, b], I_3 \subset [n-d +1, n], I_4 \subset [n-c+1, n]$ satisfying $|I_1| = |I_2|, |I_3| = |I_4|$.

For instance, with $w = 2143,$ we are given that $a = b = c = d = 1,$ and so the possible $(I_1 \cup I_3, I_2 \cup I_4)$ are:
\begin{equation*}
    (\{1,4\}, \{1',4'\}), (\{1\}, \{1'\}), (\{4\}, \{4'\}), (\emptyset, \emptyset).
\end{equation*}
(It is a coincidence in this example that we always have $I_1 = I_2$ and $I_3 = I_4$.) They correspond to the following complementary minors:
\begin{center}
    \includegraphics[]{output24.pdf}
\end{center}
\end{remark}
\begin{comment}
\begin{center}
\begin{asy}
    filldraw((0, 30)--(0,40)--(10, 40)--(10, 30)--cycle, grey);
    filldraw((0, 0)--(0,10)--(10, 10)--(10, 0)--cycle, grey);
    filldraw((30, 30)--(30,40)--(40, 40)--(40, 30)--cycle, grey);
    filldraw((30, 10)--(30,0)--(40, 0)--(40, 10)--cycle, grey);
    draw((0, 0)--(0, 40));
    draw((40, 0)--(40, 40));
    draw((0, 40)--(40, 40));
    draw((0, 0)--(40, 0));
    filldraw((10, 10)--(10, 30)--(30, 30)--(30, 10)--cycle, grey);
    draw((65, 20)--(75, 20));
    filldraw((100, 10)--(100,40)--(130, 40)--(130, 10)--cycle, grey);
    draw((100, 0)--(100, 40));
    draw((140, 0)--(140, 40));
    draw((100, 40)--(140, 40));
    draw((100, 0)--(140, 0));
    filldraw((130, 10)--(130, 0)--(140, 0)--(140, 10)--cycle, grey); 
    draw((165, 20)--(175, 20));
    filldraw((210, 0)--(210,30)--(240, 30)--(240, 0)--cycle, grey);
    draw((200, 0)--(200, 40));
    draw((240, 0)--(240, 40));
    draw((200, 40)--(240, 40));
    draw((200, 0)--(240, 0));
    filldraw((200, 30)--(200, 40)--(210, 40)--(210, 30)--cycle, grey); 
    draw((265, 20)--(275, 20));
    draw((270, 15)--(270, 25));
    filldraw((300, 0)--(300,40)--(340, 40)--(340, 0)--cycle, grey);
\end{asy}
\end{center}
\end{comment}
Notice that $I_1, I_2$ could both be taken to be empty.
\begin{proof}
Let $\beta(I, J) := s(I) + s(J) + |I|$, where $s(I) = \sum_{i \in I} i$. First, we claim for some $\alpha$,
\begin{equation}\label{antidiagcase1eqn}
    \imm_w = \alpha \sum\limits_{(I,J) \in \cS} (-1)^{\beta(I,J)} \Delta_{\overline{I}, \overline{J}} \Delta_{ I, J },
\end{equation}
Here, $\cS$ is the set of all possible colorings $(I, J)$ satisfying conditions \textit{1} through \textit{4} in Lemma \ref{non-crossing matching of case1}. (Recall that in a coloring $(I, J)$, we have that $I, \overline{J}'$ are colored black and $\overline{I}, J'$ are colored white.) %Note the order of the product of these minors means we take coloring where vertices whose labels are in $I_1, I_3$ are colored white, and that, as $I_1 \subset [1, a], I_3 \subset [n - d + 1],$ we necessarily have $[a + 1, n - d] \subset \overline{I_1 \cup I_3}.$ %, where we let $I_1, I_3$ denote the white vertices on the left, and $I_2, I_4$ denote the black vertices on the right.
\par To show \eqref{antidiagcase1eqn}, we expand each product of complementary minors $(-1)^{\beta(I, J)} \Delta_{I,J} \Delta_{\bI,\bJ}$ into a sum of $\imm_u$ terms, where $\imm_u$ appears with coefficient $(-1)^{\beta(I, J)}$ iff $(I, J) \in \cS$ by Proposition \ref{prop:cm equals imm sum}. Thus, the RHS of \eqref{antidiagcase1eqn} can be expressed as $\sum_u c_u \imm_u$, where the sum ranges over $321$-avoiding permutations in $\fkS_n$, and the coefficient $c_u = \sum_{(I, J) \in \cS_u} (-1)^{\beta(I,J)}$, where $\cS_u$ is the set of colorings in $\cS$ compatible with $u$. We want to show $c_u = 0$ unless $u = w$.

\par Suppose there exists $u\not = w$ such that $c_u \neq 0$. Then $\cS_u$ is non-empty, so by Lemmas \ref{non-crossing matching of case1} and \ref{ncm equals w case1}, the non-crossing matching of $u$ does not satisfy both conditions \textit{5} and \textit{6} in Lemma \ref{non-crossing matching of case1}. Fix a pair of vertices $v_1, v_2$ that form an internal pairing (so both vertices are in $[1, a] \cup [1, b]',$ or both are in $[n - d + 1, n] \cup [n - c + 1, n]'$). Now, we define the involution $\iota$ on colorings that swaps the colors of $v_1, v_2$. We will now show that $\iota$ maps $\cS_u$ to $\cS_u$ and $\beta(\iota(I, J)) \equiv \beta(I, J) + 1 \pmod 2$. Indeed, conditions \textit{1} and \textit{2} are preserved under $\iota$ since $v_1, v_2 \notin [a+1, n-d] \cup [b+1, n-c]'$, and conditions \textit{3} and \textit{4} are preserved because $v_1, v_2$ form an internal pairing in either $[1, a] \cup [1, b]'$ or $[n-d+1, n] \cup [n-c+1, n]'$. Furthermore, $v_1$ and $v_2$ are paired in $\ncm(u)$, so swapping the colors of $v_1$ and $v_2$ will preserve compatibility of the coloring with $u$. This shows $\iota$ maps $\cS_u$ to $\cS_u$. For the other claim, we have two main cases to consider.
\par First, if $v_1, v_2$ are both unprimed, then from our non-crossing matching they must be opposite colors; say $v_1 \in I$ and $v_2 \in \overline{I}$. Also, since $v_1, v_2$ are paired, their labels have different parities (when viewed as integers from $1$ to $n$). Now, swapping the colors of $v_1, v_2$ doesn't change the size of $I$ (the number of unprimed black vertices) but replaces $v_1$ with $v_2,$ so this swap changes $\beta(I, J)$ by $v_2 - v_1$. But $v_2 - v_1$ is odd by Remark 2 after Definition \ref{def:ncm}, so $\beta(\iota(I, J)) \equiv \beta(I, J) + 1 \pmod 2$. The same argument holds if both are primed, but now considering $J$.
\par Otherwise, suppose that $v_1$ is primed and $v_2$ is unprimed. Then, $v_1, v_2$ must have the same parity. In this case, swapping the colors of $v_1, v_2$ will add $v_1$ to $I$ and $v_2$ to $J$, or remove each of them from their respective sets; so from the parities of $v_1, v_2,$ the sum of the elements will be the same parity. However, the size $|I|$ also changes by $1$, and so $\beta(I, J)$ changes by an odd number.
% Let the first (in terms of the new indexing introduced in the proof of Lemma \ref{non-crossing matching of case1}) pair be $v_1$ and $v_2$ \todo{This should be stated explicitly since a reader will not have necessarily read this}.
\par In either case, we have $\beta(\iota(I, J)) \equiv \beta(I, J) + 1 \pmod 2$. Thus, since $\iota : \cS_u \to \cS_u$ is an involution,
\begin{equation}\label{eqn:cancellation}
    c_u = \sum_{(I, J) \in \cS_u} (-1)^{\beta(I, J)} = \sum_{(I, J) \in \cS_u} (-1)^{\beta(\iota(I, J))} = \sum_{\iota(I, J) \in \cS_u} (-1)^{\beta(I, J)+1} = -c_u.
\end{equation}
This shows $c_u = 0$, contradicting our original assumption that $c_u \neq 0$. Thus, in fact $c_u = 0$ for $u \neq w$, giving equation \eqref{antidiagcase1eqn} up to a global factor.

Next, conditions \textit{1} through \textit{4} in Lemma \ref{non-crossing matching of case1} tell us that a coloring $(I, J) \in \cS$ can be expressed in the form $\overline{I} = I_1 \cup I_3$ and $\overline{J} = I_2 \cup I_4$, where $I_1 \subset [1, a]$, $I_3 \subset [n-d+1, n]$, $I_2 \subset [1, b]$, $I_4 \subset [n-c+1, n]$, and $|I_1| = |I_2|, |I_3| = |I_4|$. By Lemma \ref{cmminors}, we can rewrite \eqref{antidiagcase1eqn} as
\begin{equation}\label{antidiagcase1eqn2}
    \imm_w = \alpha \sum\limits_{(I,J) \in \cS} (-1)^{|I|} \cm_{I,J} = \alpha \sum\limits_{I_1, I_2, I_3, I_4} (-1)^{n-|I_1| - |I_3|} \cm_{I_1 \cup I_3, I_2 \cup I_4}.
\end{equation}
To determine $\alpha$, we compare coefficients of $x_w$ in $\eqref{antidiagcase1eqn2}$. By the explicit formula of Definition \ref{def:cm}, we know that $\cm_{I_1 \cup I_3, I_2 \cup I_4}$ has a nonzero $x_w$ coefficient iff $w(I_1 \cup I_3) = I_2 \cup I_4$. This forces $I_1 = I_2 = I_3 = I_4 = \emptyset,$ since from Proposition \ref{2143Patterns} we know that $w(I_1) \subset w([1, a]) = [b+1, a+b]$ and $w(I_3) \subset w([n-d+1, n]) = [n-d-c+1, n-c],$ both of which are disjoint from $[1, b] \cup [n-c+1, n] \supset I_2 \cup I_4$. Then since $f_w (w) = 1$, we get $1 = \alpha \sgn(w) (-1)^n$, so $\alpha = \sgn(w) (-1)^n$. With this $\alpha$, \eqref{antidiagcase1eqn2} is equivalent to \eqref{antidiagcase1} and \eqref{antidiagcase1'}, as desired.
\end{proof}

%With these complementary minors, we are able to compute some of the values of the coefficients $f_w(v)$ for our Temperley-Lieb immanants.

\begin{prop}\label{sumcm6}
Let $w \in \fkS_n$ have block structure $[3][5][1][6][2][4]$ with block lengths $a,e,b,c,f,d$, as stated in the second case of Proposition \ref{2143Patterns}. Let $\cS$ be the set of possible colorings satisfying conditions \textit{1} through \textit{6} in Lemma \ref{non-crossing matching of case2}. Then, we have
\begin{equation}\label{antidiagcase2}
    \imm_w = \sgn(w) \sum\limits_{(I,J) \in \cS} \cm_{I,J}.
\end{equation}
\end{prop}

\begin{remark}
Explicitly, we have
\begin{equation}\label{antidiagcase2'}
    \imm_w = \sgn(w) \sum\limits_{I_1,I_2} \cm_{[1,a+e] \cup I_1 , I_2 \cup [b+f+a+d+1,n] },
\end{equation}
where the sum runs over all $I_1\subset [a+e+1,a+e+b+c]$ with $|I_1|=c$ and $I_2\subset [b+f+1,b+f+a+d]$ with $|I_2|=a$. 

For instance, with $w = 24153,$ we are given that $a = b = c = d = f = 1$ and $e = 0,$ and so the possible $([1, a+e] \cup I_1, I_2 \cup [b + f + a + d + 1, n])$ are:
\begin{equation*}
    (\{1,2\}, \{3,5\}), (\{1, 2\}, \{4, 5\}), (\{1, 3\}, \{3, 5\}), (\{1, 3\}, \{4, 5\}).
\end{equation*}
They correspond to the following complementary minors:
\begin{center}
    \includegraphics[]{output25.pdf}
\end{center}
\end{remark}
\begin{comment}
\begin{center}
\begin{asy}
    filldraw((40, 30)--(40,50)--(50, 50)--(50, 30)--cycle, grey);
    filldraw((20, 30)--(20, 50)--(30, 50)--(30, 30)--cycle, grey);
    filldraw((0, 0)--(0,30)--(20, 30)--(20, 0)--cycle, grey);
    filldraw((30, 0)--(30,30)--(40, 30)--(40, 0)--cycle, grey);
    draw((0, 0)--(0, 50));
    draw((50, 0)--(50, 50));
    draw((0, 50)--(50, 50));
    draw((0, 0)--(50, 0));
    draw((70, 25)--(80, 25));
    draw((75, 20)--(75, 30));
    filldraw((130, 30)--(130,50)--(150, 50)--(150, 30)--cycle, grey);
    filldraw((100, 0)--(100,30)--(130, 30)--(130, 0)--cycle, grey);
    draw((100, 0)--(100, 50));
    draw((150, 0)--(150, 50));
    draw((100, 50)--(150, 50));
    draw((100, 0)--(150, 0));
    draw((170, 25)--(180, 25));
    draw((175, 20)--(175, 30));
    filldraw((240, 40)--(240,50)--(250, 50)--(250, 40)--cycle, grey);
    filldraw((220, 20)--(220,30)--(230, 30)--(230, 20)--cycle, grey);
    filldraw((220, 40)--(220,50)--(230, 50)--(230, 40)--cycle, grey);
    filldraw((240, 20)--(240,30)--(250, 30)--(250, 20)--cycle, grey);
    filldraw((200, 0)--(200,20)--(220, 20)--(220, 0)--cycle, grey);
    filldraw((230, 30)--(230,40)--(240, 40)--(240, 30)--cycle, grey);
    filldraw((200, 30)--(200,40)--(220, 40)--(220, 30)--cycle, grey);
    filldraw((230, 0)--(230,20)--(240, 20)--(240, 0)--cycle, grey);
    draw((200, 0)--(200, 50));
    draw((250, 0)--(250, 50));
    draw((200, 50)--(250, 50));
    draw((200, 0)--(250, 0));
    draw((270, 25)--(280, 25));
    draw((275, 20)--(275, 30));
    filldraw((330, 40)--(330,50)--(350, 50)--(350, 40)--cycle, grey);
    filldraw((330, 20)--(330,30)--(350, 30)--(350, 20)--cycle, grey);
    filldraw((300, 0)--(300,20)--(330, 20)--(330, 0)--cycle, grey);
    filldraw((300, 30)--(300,40)--(330, 40)--(330, 30)--cycle, grey);
    draw((300, 0)--(300, 50));
    draw((350, 0)--(350, 50));
    draw((300, 50)--(350, 50));
    draw((300, 0)--(350, 0));
\end{asy}
\end{center}
\end{comment}
\begin{proof}
Let $\beta(I, J) := s(I) + s(J)$. First, we claim for some $\alpha$,
\begin{equation}\label{antidiagcase2eqn}
    \imm_w = \alpha \sum\limits_{(I,J) \in \cS} (-1)^{\beta(I,J)} \Delta_{I, J} \Delta_{\overline{I}, \overline{J}}.
\end{equation}
Here, $\cS$ is the set of possible colorings satisfying conditions \textit{1} through \textit{6} in Lemma \ref{non-crossing matching of case2}.
The proof of \eqref{antidiagcase2eqn} is extremely similar to the proof of \eqref{antidiagcase1eqn}, so we sketch the details. First, we expand each product of complementary minors $(-1)^{\beta(I, J)} \Delta_{I,J} \Delta_{\bI,\bJ}$ into a sum of $\imm_u$ terms, where $\imm_u$ appears with coefficient $(-1)^{\beta(I, J)}$ iff $(I, J) \in \cS$ by Proposition \ref{prop:cm equals imm sum}. Thus, the RHS of \eqref{antidiagcase2eqn} can be expressed as $\sum_u c_u \imm_u$, where the sum ranges over $321$-avoiding permutations in $\fkS_n$, and the coefficient $c_u = \sum_{(I, J) \in \cS_u} (-1)^{\beta(I,J)}$, where $\cS_u$ is the set of colorings in $\cS$ compatible with $u$. We want to show $c_u = 0$ unless $u = w$. Suppose that $u \neq w$ and $c_u \neq 0$; then $\cS_u \neq \emptyset$. By Lemmas \ref{non-crossing matching of case2} and \ref{ncm equals w case2}, the non-crossing matching of $u$ does not satisfy both conditions \textit{7} and \textit{8} in Lemma \ref{non-crossing matching of case2}, so there is some internal pairing of vertices $v_1, v_2;$ note that both are either primed or unprimed.
% Let the first (in terms of the new indexing introduced in the proof of Lemma \ref{non-crossing matching of case2}) pair be $v_1$ and $v_2$. 
\par Then, we know that $v_1, v_2$ must have different parities, so swapping the colors of $v_1, v_2$ will change the parity of $\beta(I,J)$. Also, since $v_1, v_2$ form an internal pairing, conditions \textit{1} through \textit{6} of \ref{non-crossing matching of case2} and compatibility with $u$ still hold after swapping the colors of $v_1, v_2$. In summary, we have constructed a sign-reversing involution $\iota : \cS_u \to \cS_u$. Using \eqref{eqn:cancellation}, we see that $c_u = 0$, a contradiction. Thus in fact $c_u = 0$ whenever $u \neq w$, so we obtain equation \eqref{antidiagcase2eqn} up to a global sign.

\par Next, conditions \textit{1} through \textit{6} in Lemma \ref{non-crossing matching of case2} tell us that a coloring $(I, J) \in \cS$ can be expressed in the form $I = [1,a+e] \cup I_1$ and $J = I_2 \cup [b+f+a+d+1,n])$ where $|I_1| = c$, $I_1 \subset [a+e+1, a+e+b+c]$, $|I_2| = a$, $I_2 \subset [b+f+1, b+f+a+d]$. By Lemma \ref{cmminors}, we can rewrite \eqref{antidiagcase2eqn} as
\begin{equation}\label{antidiagcase2eqn2}
    \imm_w = \alpha \sum\limits_{(I,J) \in \cS} \cm_{I,J} = \alpha \sum\limits_{I_1,I_2} \cm_{[1,a+e] \cup I_1 , I_2 \cup [b+f+a+d+1,n] }
\end{equation}
To determine $\alpha$, we compare coefficients of $x_w$ in $\eqref{antidiagcase2eqn2}.$ Note that $f_w (w) = 1,$ and by the explicit formula of Definition \ref{def:cm}, any complementary minor $\cm_{[1,a+e] \cup I_1 , I_2 \cup [b+f+a+d+1,n] }$ with a nonzero $x_w$ term requires $I_1$ to contain $w^{-1}([b + f + a + d + e + 1, n]) = [a + e + b + 1, a + e + b + c]$ and $I_2$ to contain $w([1, a]) = [b + f + 1, b + f + a]$ (from Proposition \ref{2143Patterns}). But since $|I_1| = c$ and $|I_2| = a$, we must have $I_1 = [a + e + b + 1, a + e + b + c]$ and $I_2 = [b + f + 1, b + f + a]$. Thus, there is a unique choice for $I_1, I_2$ to get a nonzero $x_w$ term, which means $1 = \alpha \sgn(w)$, so $\alpha = \sgn(w)$. With this $\alpha$, \eqref{antidiagcase2eqn2} is equivalent to \eqref{antidiagcase2} and \eqref{antidiagcase2'}, as desired.

%we notice that the coefficient of $x_w$ in the product of complementary minors after multiplying by $(-1)^{\beta(I_1,I_2)}$ is exactly the coefficient of $x_w$ in $\cm_{I, J}$ from Lemma \ref{cmminors}, which is $\sgn(w)$.




\end{proof}

Using the above propositions, we can explicitly express $\imm_w$ as a sum of products of complementary minors. From there, it becomes easy to calculate $f_w(u)$ for each $u\in \mathfrak{S}_n.$


% To begin, given a $321-,1324-$avoiding permutation $w$ that contains $2143,$ define the following values:
% \begin{enumerate}
%     \item If $w(1) \neq 1,$ let $a$ be largest value that is at least $1$ so that $w(1), w(2), \ldots, w(a)$ form a sequence of consecutive increasing integers, and otherwise let $a$ be zero.
%     \item If $w^{-1}(1) \neq 1,$ let $b$ is the largest value so that $w^{-1}(1), w^{-1}(2), \ldots, w^{-1}(b)$ forms a sequence of consecutive increasing integers, and otherwise let $b$ be zero.
%     \item If $w(n) \neq n,$ let $c$ be the largest value so that $w^{-1}(n - c + 1), w^{-1}(n - c + 2), \ldots, w^{-1}(n)$ form a sequence of consecutive increasing integers, and otherwise let $c$ be zero.
%     \item If $w^{-1}(n) \neq n,$ let $d$ be the largest value so that $w(n-d+1), w(n-d + 2), \ldots, w(n)$ form a sequence of consecutive increasing integers, and otherwise let $d$ be zero.
% \end{enumerate}



\begin{thm}\label{AntiDiagCoeff} Let $w \in \fkS_n$ have block structure $[2][1][3][5][4]$ with block lengths $a,b,e,c,d$, as stated in the first case of Proposition \ref{2143Patterns}. Let $u \in \mathfrak{S}_n$. Then the coefficient of $x_u$ in $\imm_w$ is given by
\[
f_w(u) = 
\begin{cases}
0 & \text{if there exists } i \in [1,a] \text{ s.t. } u(i) \in [1,b],\\
& \text{or there exists } i \in [n+1-d, n]\text{ s.t. }  u(i) \in [n+1 - c, n] \\
&\\
\sgn(w)\sgn(u)\binom{A+B}{A} & \text{otherwise, where A = } |[1, a] \cap u^{-1}([n+1-c, n])|, \\
% \# \{i \in [1,a] \text{ s.t. } u(i) \in [n+1 - c,n]\},\\
& B = |[1, b] \cap u^{-1}([n+1-d, n])|.
% \# \{i \in [1,b] \text{ s.t. } u^{-1}(i) \in [n+1 - d,n]\}. 
\end{cases}
\]


\end{thm}

\begin{proof}
%Write $\imm_w$ as a sum of complementary minors by Proposition \ref{sumcm5}. The coefficient of $x_u$ in $CM_{I,J}$ is $0$ unless $u(I) = J$, in which case the coefficient is $\sgn(w) \sgn(u) (-1)^{|I_1| + |I_3|}$.
Let $\cS$ be the set of colorings that satisfy conditions \textit{1} through \textit{4} in Lemma \ref{non-crossing matching of case1}, and let $\cS'_u$ be the set of colorings in $\cS$ such that $i$ and $u(i)'$ have different colors for all $i\in [n]$. (This is different from the $\cS_u$ defined in the proof of Proposition \ref{sumcm5}.)

Consider a $\cm_{I,J}$ term in the sum given in Proposition \ref{sumcm5} (so the coloring $(I, J) \in \cS$). By the remark after Definition \ref{def:cm}, notice that $x_u$ has nonzero coefficient in $\cm_{I,J}$ if and only if the coloring $(I, J) \in \cS_u'$ (in which case the coefficient is $\sgn(u)$). Thus, we get
\begin{equation}\label{eqn:extract_coeff1}
    f_w (u) = (-1)^n\sgn(w) \sgn(u) \sum_{(I, J) \in \cS_u'} (-1)^{|I|}.
\end{equation}

First, if there exists $i \in [1,a]$ such that $u(i) \in [1,b]$, then swapping the colors of $i$ and $u(i)'$ is an involution $\cS_u' \to \cS_u'$ that flips the sign of $(-1)^{|I|}$. By a cancellation argument similar to \eqref{eqn:cancellation}, we get $f_w (u) = 0$. (Alternate algebraic approach: we have $|u([1, a]) \cap [1, b]| = 1 > 0 = |w([1, a]) \cap [1, b]|$ by assumption and Proposition \ref{2143Patterns} case 1, so $u \not\ge w$ by equivalent definition \ref{bruhat3} of Bruhat order. Thus, $f_w (u) = 0$ by Lemma \ref{lem:basic fwu}.) A similar argument holds if there exists $i \in [n+1-d,n]$ such that $u(i) \in [n+1-c, n]$.

Thus, assume neither condition holds. We will show two claims: (1) $|\cS'_u| = \binom{A+B}{A}$ and (2) $(-1)^{n-|I|} = 1$ for each $(I, J) \in \cS'_u$.

Consider the $n$ pairs of vertices $(i, u(i)');$ each pair must consist of one white and one black vertex to have a nonzero $x_u$ term. Then $a-A$ of the vertices in $[1,a]$ are paired with a vertex in $[b+1, n-c]'$ and thus are black. Similarly, $b-B$ of the vertices in $[1,b]'$ are paired with a vertex in $[a+1, n-d]$ and are therefore white. As a result, among the $A+B$ remaining unresolved vertices in $[1,a] \cup [1,b]'$, $A$ must be black and $B$ must be white. There are $\binom{A+B}{A}$ ways of choosing colors for the unresolved vertices. 

Note that each coloring of the unresolved vertices in $[1, a] \cup [1, b]'$ uniquely specifies the entire coloring: since $u^{-1}([1, b]) = [a + 1, n]$ by assumption, we have that the coloring of the unresolved vertices in $[1, b]'$ determines the color of $b - B$ vertices in $[n - d + 1, n].$ But the remaining elements in $[n - d + 1],$ again by assumption, must be sent by $u$ to some element in $[b+1, n-c],$ and so their color is uniquely determined; the same argument holds for unresolved vertices in $[1, a]$ and those in $[n-c+1, n]'.$ Each of these colorings, by construction, colors $i, u(i)'$ different and satisfies conditions \textit{1} through \textit{3} in Lemma \ref{non-crossing matching of case1}. Condition \textit{4} follows from the first three: with $B$ vertices in $[1, b]'$ paired with those in $[n - d + 1, n],$ the remaining $d - B$ are necessarily colored black, and similarly $c - A$ vertices in $[n - c + 1, n]'$ are necessarily colored white. Of the remaining $A + B$ vertices, $A$ are white and $B$ are black (by the reverse condition imposed in $[1, a] \cup [1, b]'$), yielding $c$ white and $d$ black vertices. In summary, each of our choices for the unresolved vertices leads to a valid coloring in $\cS'_u$, and so $|\cS'_u| = \binom{A+B}{A}$, proving claim 1.

Finally, suppose that $C$ of the unresolved vertices in $[1, a]$ are white. Then, $A - C$ of the unresolved vertices in $[1, a]$ are black, since there are $A$ unresolved vertices in $[1, a].$ Thus, there are $A - (A - C)$ many black vertices in $[1,b]',$ since there are $A$ unresolved vertices in $[1, a] \cup [1,b]'$ that are colored black. But since $u^{-1}([1,b])$ lies in $[a+1, n],$ $i'$ being unresolved in $[1, b]'$ means that $u^{-1}(i)$ is unresolved in $[n - d + 1, n],$ and furthermore this is a $1-1$ correspondence. We thus observe that $C$ of the unresolved vertices in $[n-d+1, n]$ are white. Thus, as the only white vertices in $[1, a] \cup [n-d+1, n]$ are unresolved by our above argument, we have that $(-1)^{n-|I|} = (-1)^{2C} = 1,$ since the vertices in $I$ are colored black. This proves Claim 2.

Plugging in Claims 1 and 2 into \eqref{eqn:extract_coeff1}, the coefficient $f_w(u)$ of $x_u$ in $\imm_w$ is equal to $\binom{A + B}{A} \sgn(w)\sgn(u),$ as desired.
\end{proof}

\begin{thm}\label{coefficient of u in w case 2}
Let $w \in \fkS_n$ have block structure $[3][5][1][6][2][4]$ with block lengths $a,e,b,c,f,d$, as stated in the second case of Proposition \ref{2143Patterns}. Let $u\in \mathfrak{S}_n$. Then
\[f_w(u) = \begin{cases}
0 & \text{if there exists } i \in [1,a+e] \text{ s.t. } u(i) \in [1,b+f],\\
& \text{or there exists }i \in [a+e+b+c+1, n]\text{ s.t. }  u(i) \in [b+f+a+d+1, n] \\
&\\
\sgn(w)\sgn(u)\binom{A+B}{A} & \text{otherwise, where} \\
% & A=c- \#\{i\in[a+e+1,a+e+b+c] \text{ s.t. }  u(i)\in [b+f+a+d+1,n] \},\\
& A=c- |[a+e+1,a+e+b+c] \cap u^{-1}([b+f+a+d+1,n])|,\\
% & B= b-\#\{i\in[a+e+1,a+e+b+c] \text{ s.t. }  u(i)\in [1,b+f] \}.
& B= b- |[a+e+1,a+e+b+c] \cap u^{-1}([1,b+f])|.
\end{cases}
\]
% where \[A=c- \Bigg|\Big\{i\in[a+e+1,a+e+b+c] \  \Big| \  u(i)\in [b+f+a+d+1,n] \Big\}\Bigg|\]
% and 
% \[B= b-\Bigg|\Big\{i\in[a+e+1,a+e+b+c] \ \Big| \  u(i)\in [1,b+f] \Big\}\Bigg|\]
\end{thm}

\begin{remark}
The binomial coefficient $\binom{A+B}{A}$ is taken to be zero if $A < 0$ or $B < 0$.
\end{remark}

\begin{proof}
\par Let $\cS$ be the set of colorings that satisfy conditions \textit{1} through \textit{6} in Lemma \ref{non-crossing matching of case2}, and let $\cS'_u$ be the set of colorings in $\cS$ such that $i$ and $u(i)'$ have different colors for all $i\in [n]$.

\par Consider a $\cm_{I,J}$ term in the sum given in Proposition \ref{sumcm6} (so the coloring $(I, J) \in \cS$). By the remark after Definition \ref{def:cm}, notice that $x_u$ has nonzero coefficient in $\cm_{I,J}$ if and only if the coloring $(I, J) \in \cS_u'$ (in which case the coefficient is $\sgn(u)$.) Thus, we get

\begin{equation}\label{eqn:extract_coeff2}
    f_w (u) = \sgn(w) \sgn(u) |\cS_u'|.
\end{equation}

\par First, note that $\cS_u' = \emptyset$ if some $i \in [1,a+e]$ satisfies $u(i) \in [1,b+f]$. This is because both $i$ and $u(i)$ will be black vertices. Similarly, if some $i \in [a+e+b+c+1, n]$ satisfies $u(i) \in [b+f+a+d+1, n]$, then both $i$ and $u(i)$ will be white vertices. In this case, we find that $f_w (u) = 0.$ Thus, assume neither condition holds. We will now count the elements in $\cS_u'$.

Note that $[a+e+1,a+e+b+c] \cap u^{-1}([b+f+a+d+1,n]) $ must all be black and $[a+e+1,a+e+b+c] \cap u^{-1}([1,b+f])$ must all be white, and the vertices in $[a+e+1, a + e + b+c]$ contains $c$ black and $b$ white vertices. Therefore, we have the freedom of choosing exactly $A$ black and $B$ white vertices from the $A+B$ vertices in $[a + e + 1, a + e + b + c],$ each of which uniquely determines the coloring for $[b + f + 1, b + f + a + d + 1]'$ and thus the entire coloring. As a result, $|\cS_u'| = \binom{A+B}{A}$, and hence by \eqref{eqn:extract_coeff2}, $f_w (u) = \sgn(w)\sgn(u)\binom{A + B}{A}$. %\ds\binom{A+B}{A}

%Now, notice that for any choice of coloring such that $x_u$ has nonzero coefficient, their contribution, after multiplying by $(-1)^{ \beta (I_1,I_2)}$ (the factor given in the sum of complementary minors) is exactly $\sgn(u)$, by the same argument in proof of the above theorem. Indeed, recalling Lemma \ref{cmminors}, we can express all of these as being a sum of $\cm_{I, J}$ times $\sgn(w).$ Therefore, the coefficient is $\sgn(w)\sgn(u)\ds\binom{A + B}{A}$. 

\end{proof}

Of special interest is the antidiagonal coefficient, which we promote to its own corollary.

\begin{cor}\label{cor:anti_diag}

Let $w$ avoid the patterns $1324$ and $321$ but not $2143$. Then, $w$ falls into one of the two cases of Proposition \ref{2143Patterns}. Define $a, b, c, d \ge 1$ accordingly (we ignore $e, f$). If $w_0$ is the longest word in $\mathfrak{S}_n,$ then $|f_w(w_0)| = \binom{\min(a, c) + \min(b, d)}{\min(b, d)}.$
\end{cor}
\begin{proof}
\par We divide up our work into the two cases, given by Theorems \ref{coefficient of u in w case 2} and  \ref{AntiDiagCoeff}. 
\par For the first case (covered by Theorem \ref{AntiDiagCoeff}), it's not hard to see that $w_0$ lies in the second case in Theorem \ref{AntiDiagCoeff}, meaning that we need to compute the number of elements in $w_0^{-1}([n + 1 - c, n]) \cap [1, a]$ and $w_0^{-1}([1, b]) \cap [n + 1 - d, n].$ However, notice that $w_0$ sends $[1, b]$ to $[n + 1 - b, n],$ meaning that the size of the latter set is $\min(b, d).$ Similarly, we see that the size of the former set is $\min(a, c),$ giving us that by Theorem \ref{AntiDiagCoeff} that $f_w(w_0) = \binom{\min(a, c) + \min(b, d)}{\min(b, d)}.$
\par For the second case (covered by Theorem \ref{coefficient of u in w case 2}), observe that for $w_0,$ $A$ is equal to $c - |[b + f + a  + d + 1, n] \cap [n - a - e - b - c + 1, n - a - e]|.$ But $n - a - e - b - c = d + f,$ meaning that our set has magnitude $|[b + f + a + d + 1, n - a - e]|.$ But observe that $n - a - e = b + f + c + d,$ meaning that this is equal to $\max(c - a, 0).$ Similarly, for $B,$ this is equal to the size of the set $[d + f, b + f] = \max(b - d, 0)$ subtracted from $b.$
\par Therefore, $A = c - \max(c - a, 0) = \min(a, c), B = b - \max(b - d, 0) = \min(b, d),$ whereby Theorem \ref{coefficient of u in w case 2} gives us the desired result.
\par Combining these together gives us the value of the antidiagonal coefficient, as desired.
\end{proof}


% Now, it suffices to compute the anti-diagonal coefficient of the sum of complementary minors. Again, we divide up into the two cases per Proposition \ref{2143Patterns}. 
% \par \textbf{Case 1:} $w$ has five blocks of ascending consecutive integers. To do so, we compute the number of colorings such that $i$ and $(n+1 - i)'$ are of different colors. If $a \geq c,$ each $i$ with $c+1 \le i \le a$ must be colored black (otherwise, this condition is vacuous), and if $b \geq b,$ each $i'$ with $d+1 \le i \le b$ must be colored white (again, otherwise the condition is vacuous). This leaves us with needing to color $\min(a, c)$ of the $\min(a, c) + \min(b, d)$ undetermined points black. We will show that the signs of the $\binom{\min(a, c) + \min(b, d)}{\min(b, d)}$ compatible complementary minors constructively interfere, proving the result.

% \par Given $2n$ vertices, we recall the following rules on the coloring of the vertices from our work in Subsection \ref{CompliMinors}: we again divide the vertices on the left side into 5 blocks, with size $a,b,e,c,d$ respectively. Among them, block $[1,a]$ and $[n-d+1, n]$ contain possibly both black and white vertices, and the three blocks of size $b, e, c$ in the middle contain only black vertices. Let $I_1$ be the set of black vertices in $[1,a]$, and let $I_3$ be the set of black vertices in $[n-d+1, n]$. Next, for the vertices on the right side, we also divide them into 5 blocks with size $b,a,e,d,c$ respectively. Using $i$ to denote a right vertex and $i'$ to denote a left vertex, we let block $[1',b']$ and $[(n-c+1)',n']$ to have possibly both black and white vertices, and the three blocks of sizes $a, e, d$ in the middle contain only white vertices. Let $I_2$ be the set of black vertices in $[1',b']$, and let $I_4$ be the set of black vertices in $[(n-c+1)',n']$. 

% \par Moreover, let there be $a$ black vertices in the union of $[1,a]$ and $[1',b']$, and $d$ black vertices in the union of $[n-d+1, n]$ and $[(n-c+1)',n']$. By construction, this means that there are $b$ white vertices in the first union and $c$ white vertices in the second union. %Finally, for any vertex $i \in [1,a]$, we will assign vertex $(n+1-i)' \in [(n-c+1)',n']$ to have the opposite coloring so that $i$ and $(n-c+1)'$ can be matched. 

% \par Let $k$ be the number of black vertices on the left. So we have $k = |I_1|+|I_3|+b+c+e$.  Recall, for $I, J \subseteq [n]$, $I, \bar{J}$ are black vertices and $J, \bar{I}$ are white vertices. The coefficient of the product of the anti-diagonal elements in $\Delta_{I,J} \Delta_{\bar{I}, \bar{J}}$ is given by $(-1)^{{k \choose 2} + {n-k \choose 2}}$. Expanding its power, we get 
% \begin{align*}
% {k \choose 2} + {n-k \choose 2} = \frac{k(k-1)}{2} + \frac{(n-k)(n-k-1)}{2} = k^2 - nk +\frac{n^2 - n}{2} \pmod 2
% \end{align*}

% \par We claim that $|I_1| + |I_4| = a$. Indeed, any white vertex in $[1,a]$ has its matching black vertex in $[(n-a+1)',n']$. If the black vertex is not in $[(n-c+1)',n']$, then it is necessarily white. Hence, the black vertex belongs to $I_4$. Also, we already know that $|I_3| + |I_4| = d$. Adding our two equations, we get $|I_1|+|I_3| \equiv |I_1|+|I_3| + 2|I_4| \equiv a + d \pmod 2.$ Hence,
% \begin{align*}
% k &= |I_1|+|I_3|+b+c+e \\
% &\equiv a + d + b + c + e \pmod 2\\
% &\equiv n \pmod 2
% \end{align*}

% This gives us 
% \begin{align*}
% {k \choose 2} + {n-k \choose 2} = k^2 - nk +\frac{n^2 - n}{2}  \equiv \frac{n^2 - n}{2} \pmod 2
% \end{align*}

% \par Consider $s(I_1)+s(I_2)+s(I_3)+s(I_4) + |I_1|+|I_3|$. A vertex $i \in I_1$ contributes $i$ to the sum $s(I_1) + s(I_4)$, and a vertex $j' \in I_4$ contributes $n + 1 - j \equiv n + 1 + j \h (\text{mod }2)$ to the sum $s(I_1) + s(I_4)$. Hence, as for $i \in [1, a]$ either $i \in I_1$ or $(n + 1 - i)' \in I_4,$ we have $s(I_1) + s(I_4) = 1 + 2 + ... + a + (n+1)\cdot |I_4| = \frac{a(a+1)}{2} + (n+1)\cdot |I_4|$, and similarly, $s(I_2) + s(I_3) = (n+1-d) + ... + n + (n+1)\cdot |I_2| = nd - \frac{d(d-1)}{2} + (n+1)\cdot |I_2|$. Since $|I_1| + |I_4| = a$, $|I_1| + |I_2| = a$, we have that $|I_2| = |I_4|$. Computing the sum, we get

% \begin{align*}
% s(I_1)+s(I_2)+s(I_3)+s(I_4) + |I_1|+|I_3| &= \frac{a(a+1)}{2} + (n+1)\cdot |I_4| + nd - \frac{d(d-1)}{2} + (n+1)\cdot |I_2| + a +d \\
% &\equiv \frac{a(a+1)}{2} + nd - \frac{d(d-1)}{2} + a + d \pmod{2}\\
% & = \frac{a^2 + 3a}{2} - \frac{d^2 - 2nd - 3d}{2}
% \end{align*}

% This tells us that the sign of the anti-diagonal coefficient $|f_w(w_0)| = (-1)^{{k \choose 2}+{n - k \choose 2}}\cdot (-1)^{s(I_1)+s(I_2)+s(I_3)+s(I_4) + |I_1| + |I_3|}$ is independent of $I_1, I_2, I_3, I_4$, hence, they constructively interfere. This proves our result.



\begin{remark}
In particular, since $a, b, c, d \ge 1$, we have $|f_w (w_0)| \ge 2$, so $\imm_w$ cannot be expressed as a \%-immanant. This provides an alternate proof for the 2143-avoiding condition in Theorem \ref{2143ConverseThm}.
\end{remark} 
% In particular, the converse of Theorem \ref{TwoPercentForward} falls out as a corollary; there is a bit more work that needs to be done for some of the cases, however.
% \begin{cor}
% A $321$-avoiding permutation $w$ can be expressed as the sum of at most two \% immanants if and only if it avoids the patterns $1324, 24153, 31524, 231564,$ and $312645.$
% \end{cor}
% \begin{proof}
% We heavily employ Proposition \ref{2143Patterns}, since we know from Theorem \ref{2143ConverseThm} that if it avoids $1324, 2143$ that it is equal to a \%-immanant, up to sign. Notice that the $1324$ avoidance is immediate from Theorem \ref{1324Thm}.
% \par From here, we again divide up into cases. First, suppose that $w(1) + w^{-1}(1) + n \leq w(n) + w^{-1}(n) + 2,$ so we have the five ascending strings by Proposition \ref{2143Patterns}. Then, notice that, per Theorem \ref{AntiDiagCoeff}, we see that to be a sum of two \%-immanants, necessarily the anti-diagonal coefficient needs to be at most $2$ in absolute value. But then we see that we require that $\min(a, c) = \min(b, d) = 1,$ as these minima by construction are either both nonzero or both zero.
% \par But then notice that for this to occur, within the pattern outlined in Proposition \ref{2143Patterns}, we require that $w^{-1}(1) = 2$ or $w(n) = n - 1,$ and similarly we require that either $w(1) = 2$ or $w(n) = n - 1.$ WLOG, say that $w^{-1}(1) = 2.$ Then, either $w(1) = 2,$ or $w(n) = n-1.$ In the first case, it follows that the one-line notation for $w$ is $2134\ldots (w(n) + w^{-1}(n) - n) (w(n) + 1) \ldots n (w(n) + w^{-1}(n) - n + 1) \ldots w(n),$ which avoids the four patterns we listed. In the second case, we have that $w(n) = n-1,$ so our pattern is equal to $w(1)123 \ldots (w(1) - 1) (w(1) + 1) (w(1) + 2) \ldots (w(n) - 1) n (w^{-1}(n)) \ldots (n-1),$ which also avoids the patterns we listed.
% \par Now, we suppose that $w(1) + w^{-1}(1) + n > w(n) + w^{-1}(n) + 2,$ which means we're in the second case described by Proposition \ref{2143Patterns}. In this case, it follows that, by similar logic, either the length of the first or the fifth string is $1,$ and the length of the sixth or the third string is $1.$ We may WLOG assume that the first string has length $1.$ 
% \par From Theorem \ref{AntiDiagCoeff}, we see that if $w$ is either a sum or difference of two \%-immanants, notice that, since the coefficient of the anti-diagonal has magnitude at least $2,$ it needs to be a sum (up to an overall sign).
% \par But then notice that one of these \%-immanants has to contain $w;$ this means that in particular one of these \%-immanants is so that there are only zeroes within the two rectangles $i \leq w^{-1}(1) - 1, j \leq w(1) - 1$ and $i \geq w^{-1}(n) + 1, j \geq w(n).$ In particular, this means that the permutation that shifts $[1, n - w(n)]$ to $[w(n) + 1, n]$ and shifts $[n - w(n) + 1, n]$ to $[1, w(n)],$ since we have that this permutation sends $[1, w^{-1}(1) - 1]$ to $[w(n) + 1, w^{-1}(1) + w(n) - 1],$ but $w(n) + 1 \geq w(1) - 1$ (else, we have a $321$- pattern with $w(1), w(n) - 1, w(n)$), and similarly for the other rectangle).
% \par But then notice that, as given in the argument in this case within the proof of Theorem \ref{AntiDiagCoeff}, this permutation is contained in none of the complementary minors, as this would require us to have the colors of at least $\min(a, e) + d$ of the middle primed numbers to be colored black, which is too many (namely, there are $d$), meaning that the coefficient in the TL-immanant is $0;$ but this is a contradiction.
% \par Therefore, we see that in this case, this permutation cannot be the sum of two \%-immanants, as desired.
% \end{proof}
%It turns out that the cases that we've covered are precisely all the possible Temperley-Lieb immanants that can be written as a linear combination of \%-immanants.
% \begin{thm} \label{classification of imm as linear combination}
% Given a $321$-avoiding permutation $w,$ $\imm_w$ is a linear combination of \%-immanants if and only if $w$ avoids the patterns $1324, 24153, 31524, 231564, 312645.$
% \end{thm}

\subsection{Proof of Theorem \ref{TwoPercentForward}}
Using the results of the previous subsection, we are now able to prove Theorem \ref{TwoPercentForward}, which we restate here.
\begin{thm}\label{TwoPercentForward_restate}
Let $w$ be a $321$-avoiding permutation. The following statements are equivalent:
\begin{enumerate}
    \item The Temperley-Lieb immanant $\imm_w$ is a linear combination of \%-immanants;
    
    \item The signed Temperley-Lieb immanant $\sgn(w) \imm_w$ is a sum of at most two \%-immanants;
    
    \item The permutation $w$ avoids the patterns $1324, 24153, 31524, 231564$, and $312645,$ in addition to avoiding $321.$
\end{enumerate}
\end{thm}
% \begin{Thm}
% Let $w$ be a $321$-avoiding permutation. The following statements are equivalent:
% \begin{enumerate}
%     \item The Temperley-Lieb immanant $\imm_w$ is a linear combination of \%-immanants;
    
%     \item The signed Temperley-Lieb immanant $\sgn(w) \imm_w$ is a sum of at most two \%-immanants;
    
%     \item The permutation $w$ avoids the patterns $1324, 24153, 31524, 231564$, and $312645,$ in addition to avoiding $321.$
% \end{enumerate}
% \end{Thm}
\begin{proof}
We will prove $\textit{3} \Rightarrow \textit{2} \Rightarrow \textit{1} \Rightarrow \textit{3}$. The implication $\textit{2} \Rightarrow \textit{1}$ is straightforward.

$\textit{3} \Rightarrow \textit{2}$. Assume that $w$ avoids $321, 1324, 24153, 31524.$ If furthermore, $w$ avoids $2143$, then $\imm_w = \sgn(w)\imm_w^\%$ by Theorem \ref{CSBThm}. Thus, we may further assume that $w$ contains $2143$. By Lemma \ref{lem:24153, 31524 pattern}, we have that $w$ has block structure $[2][1][3][5][4]$, as stated in the first case of Proposition \ref{2143Patterns}. Since $w$ furthermore avoids $231564$ and $312645$, we must have $a = 1$ or $c = 1$, and $b = 1$ or $d = 1$. This is because otherwise, the block structure given in the first case of Proposition \ref{2143Patterns} will result in a pattern of $231564$ or $312645$. We will first consider the case when $a = 1$. This case has two subcases. %Then either
%Also, $a = 1$ or $c = 1$, and $b = 1$ or $d = 1$. The cases $a=b=1$ and $c=d=1$ are similar, and the cases $a=d=1$ and $b=c=1$ are similar. Now, we do the cases $a=b=1$ and $a=d=1$, respectively.

\textbf{Case 1.} Assume $a = b = 1$. We use Theorem \ref{AntiDiagCoeff} to prove that $\sgn(w)\imm_w$ is the sum of the following two \% immanants. 
\begin{itemize}
    \item $\imm^\%_1 = \imm_w^{\%}.$
    \item The \%-immanant $\imm^\%_2$ corresponding to the \% shape where we remove the $d \times c$ rectangle in the lower-right corner, and remove the $(n-d) \times 1$ and $1 \times (n-c)$ rectangles in the upper-left corner (with these two rectangles overlapping at the $1 \times 1$ rectangle in the upper-left corner).
\end{itemize}
\par In particular, in the perfect matching of $w$, block $[1,a]$ is matched with block $[(b+1)', (b+a)']$, and block $[1',b']$ is matched with block $[a+1, a+b],$ letting $a=b=1$ gives us that $w(1)=2$ and $w(2)=1$.

%Similarly, in the case when $c = d = 1$, $f_w(u) = 0$ if $u(n) = n$. 

\textbf{Case 2.} If $a = d = 1,$ we take the following two \%-immanants:
\begin{itemize}
    \item $\imm^\%_1 = \imm_w^{\%}.$
    \item The \%-immanant $\imm^\%_2$ corresponding to the shape where we remove the $1 \times (n-b)$ rectangle in the lower-right corner, and remove the $(n-c) \times 1$ rectangle in the upper-left corner.
\end{itemize} 

To show both of these, we compare the coefficients of $x_u$ for each $u \in \fkS_n.$ First, suppose that $u([1, a])$ is not disjoint from $[1, b],$ or $u([n - d + 1, n])$ is not disjoint from $[n-c + 1, n],$ so we know that $f_w(u) = 0$ by Theorem \ref{AntiDiagCoeff}.
\par Notice that these conditions in particular imply that the coefficient of $x_u$ in $\imm_w^{\%}$ is zero for both cases. This also holds for $\imm_2^{\%}$ as well, since the rectangles that we remove in the latter contain the rectangles we remove in the former. So the coefficient of $x_u$ in $\imm_1^{\%} + \imm_2^{\%}$ is zero as well.
% So for any $u\in \mathfrak{S}_n$ with $u(1) = 1$, the coefficient of $x_u$ in $\imm_w$ is $0;$ by construction we also see that the coefficient of $x_u$ is zero for each of the $\%$-immanants.
% \par Meanwhile, for the second case, notice that whenever $u(i) \in [1',b']$ for some $i \in [1,a]$, or $u(i) \in [(n+1-d), n]$ for some $i \in [(n+1-c)', n']$, we have $u \not \geq w,$ and thus $f_w(u) = 0$. We can also see that in these cases the coefficient of $x_u$ in the sum of the \%-immanants is zero.

\par Now, consider $u \in \mathfrak{S}_n$ where $f_w(u) \neq 0$. Then, $f_w(u) =$ $A+B \choose A$ as in Theorem \ref{AntiDiagCoeff}. Recall that for $w$ to avoid the above patterns, $w$ fall into one of the four cases: $a = 1, b = 1$, or $a = 1, d = 1$, or $c = 1, b = 1$, or $c = 1, d = 1$. Given these constraints, for $A+B \choose A$ to be nonzero, we can have either $B \choose 0$ $= 1$, or $2 \choose 1$ $= 2$. 
\par Suppose that this coefficient is $1,$ so either $A = 0$ or $B = 0.$ Then, when $a = b = 1,$ note that $A = 0$ iff $u(1) \not \in [n-c+1, n],$ or that it doesn't fit in the second pattern, and $B = 0$ iff $u^{-1}(1) \not \in [n-d+1, n],$ or that it doesn't fit in the second pattern. Meanwhile, for $a = d = 1,$ we have $A = 0$ iff $u(1) \not \in [n - c + 1, n]$ (so doesn't fit in the second pattern), and $B = 0$ iff $u^{-1}([1, b])$ is not equal to $n,$ or that it doesn't fit in the second pattern.
\par Otherwise, the coefficient is ${2 \choose 1} = 2$ if and only if $A \neq 0$ and $B \neq 0.$ But then we have that $u(1) \in [n - c + 1, n]$ and $u([n-d + 1, n]) \cap [1, b]$ contains an element. In the case for $a = b = 1,$ we have that $u(1) \not \in [1, n - c]$ and $u([1, n-d]) \neq 1$ (since $u([n-d+1, n])$ contains $1$), and the condition that $f_w(u) \neq 0$ means that $u([n-d+1, n])$ is disjoint from $[n-c+1, n],$ so $u$ fits in both \%-immanant patterns. For the case $a = d = 1,$ again $u(1)$ does not lie in $[1, n-c]$ and $u(n)$ is not in $[b + 1, n],$ so $u$ fits in both $\%$-immanants. This proves our result for $a = 1$.

Otherwise, if $c = 1$, then notice that $w'=w_0 w^{-1} w_0$ will also have block structure $[2][1][3][5][4]$ and the corresponding $a$-value for $w'$ is $1$. This is because taking the inverse is the same as reflecting the matching diagram across the perpendicular bisector of $1$ and $1'$, while taking conjugation by $w_0$ is the same as reflecting the matching diagram across the perpendicular bisector of $1$ and $n$. By the discussion of our previous case, $\sgn(w') \imm_{w'}$ is a sum of two \%-immanants $\imm^\%_1 + \imm^\%_2$. And by Lemma \ref{lem:symmetry}, $\sgn(w) \imm_w = \imm^{*\%}_1 + \imm^{*\%}_2$, where $\imm^{*\%}_i$ is the $\%$-immanant with a zero in $(i, j)$ if and only if $\imm^\%_i$ has a zero in $(n+1-j, n+1-i)$.

%,we may use an argument similar to the $a = 1$ argument. However, we may also use a symmetry argument and directly use the result from $a = 1$ as follows: we know that $w_0 w^{-1} w_0$ will have $a = 1$ and thus $\sgn(w) \imm_{w_0 w^{-1} w_0} = \imm^\%_1 + \imm^\%_2$. Using Lemma \ref{lem:symmetry}, $\sgn(w) \imm_w$ will also be a sum of two \%-immanants $\imm^{*\%}_1 + \imm^{*\%}_2$, where $\imm^{*\%}_i$ is the $\%$-immanant with a zero in $(i, j)$ if and only if $\imm^\%_i$ has a zero in $(n+1-j, n+1-i)$.

$\textit{1} \Rightarrow \textit{3}$. We will show the contrapositive of this statement. Suppose $w$ doesn't avoid one of $1324, 24153, 31524, 231564, 312645$; we would like to show $\imm_w$ is not a linear combination of $\%$-immanants. If $w$ contains $1324$, then we are done by Theorem \ref{1324Thm}, so we can assume $w$ avoids $1324$.

By the symmetry lemma \ref{lem: simplify cases}, we know that $\imm_w$ is a linear combination of $\%$-immanants iff $\imm_{w^{-1}}$ is. Thus, by replacing $w$ with $w^{-1}$ and using Lemma \ref{lem: restriction inverses} if necessary, we may assume without loss of generality that $w$ contains $24153$ or $231564$ and avoids $1324$. In particular, $w$ contains $2143$.

The basic idea is to find two $1324$-adjacent permutations $v, v'$ such that $f_w (v) \neq -f_w (v')$, and use Theorem \ref{thm:classifying space of percent} to conclude that $\imm_w$ is not a linear combination of $\%$-immanants. We divide into cases per Proposition \ref{2143Patterns}. 
\par \textbf{Case 1: $w$ has block form $[2][1][3][5][4]$.} Define $a, b, c, d, e$ as in the first case of Proposition \ref{2143Patterns}. Then by inspection of the block form, we see that $w$ must avoid $24153$, so $w$ contains $231564.$ Further inspection of the block form also yields that $\min(a, c) \geq 2.$
\par Consider the permutation $v$ given by $v(i) = i + (n - a - c)$ for $i \leq a + c,$ and $v(i) = n + 1 - i$ for all other values of $i.$ We first note that $v([1, a]) = [n - a - c+1, n-c],$ which is disjoint from $[1, b]$ and $[n + 1 - c, n],$ and that $v([n + 1 - d, n]) = [1, d],$ which is disjoint from $[n + 1 - c, n]$ as $\max(b, d) \leq n - a - c - e \leq n - c.$ Therefore, by Theorem \ref{AntiDiagCoeff}, we have that $f_w(v) = \sgn(w)\sgn(v),$ since $[1,a] \cap u^{-1}([n+1 - c,n])$ is empty and $\binom{N}{0} = 1$ for any $N$.
\par Now, consider the permutation $vs_a.$ Repeating the computations, observe that $vs_a([1, a])$ is $[n - a - c, n - c - 1] \cup \{n - c + 1\},$ which is disjoint from $[1, b]$ and intersects $[n+1-c, n]$ in one element, and $vs_a([n + 1 - d, n]) = [1, d],$ which is disjoint from $[n + 1 - c, n].$ Again, by Theorem \ref{AntiDiagCoeff}, we find that $f_w(vs_a) = \sgn(w)\sgn(vs_a) \binom{\min(b, d)+1}{1} = \sgn(w)\sgn(vs_a) (\min(b, d)+1),$ since $vs_a([n + 1 - d, n])$ contains $\min(b, d)$ elements that map to elements in $[1, b].$
\par Thus, $f_w (v) = \pm 1$ while $f_w (vs_a) = \mp (\min(b,d) + 1)$. But $v, vs_{a+e}$ are $1324$-adjacent, by considering the inputs $a-1, a, a+1, a+2$ (it is here that the assumption $\min(a, c) \ge 2$ is used). Thus, by Theorem \ref{thm:classifying space of percent}, $\imm_w$ cannot be written as a sum of \%-immanants.
\par \textbf{Case 2: $w$ has block form $[3][5][1][6][2][4]$.} Define $a, b, c, d, e, f$ as in the second case of Proposition \ref{2143Patterns}, with $\max(e, f) > 0$. By replacing $w$ with $w^{-1}$ if needed, we can without loss of generality assume $e \geq 1,$ which means that our permutation contains the pattern $24153.$
\par In this case, consider the permutation given by $v(i) = i + n - a - 2e - c$ for $e \le i \le a + e + c + 1$, and $w(i) = n + 1 - i$ otherwise. Notice that $v([a+e+1, a+e+c+1]) = [n - e - c + 1, n - e + 1] \subset [b + f + a + d + 1, n]$ since $n = a + b + c + d + e + f.$ Therefore, by Theorem \ref{coefficient of u in w case 2}, we have that $f_w(v) = 0,$ since $A = c - (c + 1) < 0.$
\par Now, consider the permutation $vs_{a + e}.$ Observe first that $vs_{a+e}([1, a+e]) = [n - e + 1, n] \cup [n - a - e - c, n - e - c - 1] \cup \{ n - e - c + 1\};$ but $n - a - e - c  = b + f + d \geq b + f + 1.$ Similarly, note that $vs_{a+e}([a + e + b + c + 1, n]) = [1, f + d];$ again note that $f + d < b + f + a + d + 1.$
\par Finally, observe that $vs_{a+e}([a + e + 1, a + e + b + c])$ equals the union of $v([a+e+2, a+e+b+c])$ and $\{v(a+e)\}.$ This in turn equals $[n - e - c + 2, n - e + 1] \cup [n - a - e - b - c + 1 , n - a - e - c - 1] \cup \{n - e - c\} = [b + f + a + d + 2, n - e + 1] \cup [d + f + 1 , b + d + f - 1] \cup \{b + f + a + d\}.$ This contains $c$ elements in $[b + f + a + d + 1, n]$ and $\max(b - d, 0)$ elements of $[1, b + f]$, so by Theorem \ref{coefficient of u in w case 2} we have that $f_w(vs_{a+e}) = \sgn(vs_{a+e})\sgn(w).$
\par Thus, $f_w(vs_{a+e})$ is equal to $\pm 1$ while $f_w (v) = 0$. But $v, vs_{a+e}$ are $1324$-adjacent, by considering the inputs $a + e - 1, a + e, a + e + 1, a + e + 2.$ Thus, by Theorem \ref{thm:classifying space of percent}, $\imm_w$ is not a linear combination of \%-immanants.
% \par To finish, notice that if our permutation $w$ contains one of the four patterns, then either it lies in the second case of Proposition \ref{2143Patterns}, or it lies in the first case. But if it lies in the first case, then notice that from the block structure $[2][1][3][5][4],$ then $w$ cannot contain the pattern $24153$ or $31524.$ 
% \par Indeed, if one of these patterns is exhibited by $i_1, i_2, i_3, i_4, i_5,$ then note that they can't all be in different blocks, due to the ordering of the block structure. Thus, two adjacent indices have to lie in the same block. However, by the definition of a block, if $i < j$ lie in the same block, then $w(i) < w(j)$ and $w([i, j]) = [w(i), w(j)].$ Therefore it follows that that one pair of indices $i_1, i_2, i_3, i_4, i_5$ lies in the same block. In particular, either $w$ contains a $24153$ pattern, and the possible pairs are $(i_1, i_2)$ and $(i_3, i_4),$ or $w$ contains a $31524$ pattern, and the possible pairs are $(i_2, i_3)$ and $(i_4, i_5).$ But note that in the $24153$ case, neither of $w([i_1, i_2]), w([i_3, i_4])$ contains $w(i_5),$ and in the $31524$ case, neither of $w([i_2, i_3]), w([i_4, i_5])$ contains $w(i_1).$
% \par Therefore, if $w$ is described by the first case and contains one of the four patterns, then $w$ contains either $231564$ or $312645,$ exhibited by the indices $i_1, i_2, i_3, i_4, i_5, i_6.$ By the block structure $[2][1][3][5][4],$ notice that $i_1$ lies in the first block. To see this, observe that $w(i_1) > w(i_3)$ means that $i_1$ lies in the first or fourth block, and $w(i_3) < w(i_4)$ and $w(i_4) > w(i_6)$ implies that $i_3$ cannot lie in the last block, so hence $i_1$ is in the first block. This then implies that $i_3$ lies in the second block. Similarly, $i_6$ is in the last block, so $i_4$ lies in the fourth block. 
% \par From here, notice that in the $231564$ pattern, $i_2, i_1$ lie in the first block and $i_4, i_5$ lie in the fourth block, and in the $312645$ pattern, we have that $i_2, i_3$ lie in the second block and $i_5, i_6$ in the last block. These correspond to requiring $a, c \geq 2$ and $b, d \geq 2,$ respectively.
%\par From here, it follows that our above argument covers all permutations $w$ avoiding $321, 1324$ that contain one of $231564, 312645, 24153, 31524.$ Thus, if $w$ avoids $321, 1324$ and contains one of those four patterns, $\imm_w$ cannot be written as a linear combination of \%-immanants.   This proves the theorem.
\end{proof}
% \par In particular, Theorems \ref{CSBThm} and \ref{TwoPercentForward} specify all the possible $321$-avoiding permutations for which the corresponding Temperley-Lieb immanant could be expressed as a linear combination of \%-immanants. 

% \begin{example}
% Let $w = 3416725$. $w$ avoids pattern $1324$ but contains pattern $231564$. The coefficient of $x_{0(n-1)} x_{1(n-2)} \cdots x_{(n-1)0}$ in $\imm_w$ is $-3$, but the same coefficient in a \%-immanant is $\pm 1$.

% \end{example}
\subsection{Proofs of Lemmas in Section 5}
Here we present the proofs for the lemmas introduced in section 5.

\begin{replemma}{lem:24153, 31524 pattern}
Suppose that $w$ is a permutation avoiding $1324, 321$ that contains the pattern $2143,$ and define $a', b', c', d'$ as in Proposition \ref{2143Patterns}. Then, suppose that $a' + b' + c' + d' > n.$ Then, $w$ either contains the pattern $24153$ or $31524.$
\end{replemma}
\begin{proof}
We must be in Case 2 of Proposition \ref{2143Patterns}. If $e \ge 1$, then $w$ contains a $24153$ pattern. If $f \ge 1$, then $w$ contains a $31524$ pattern.
%Notice that, per the statement of proposition \ref{2143Patterns}, that when $a' + b' + c' + d' > n,$ that the six ascending consecutive strings of integers obey a $351624$-pattern if all of these strings are non-empty. 
%\par If the 2nd or 5th string doesn't exist, but the other does, then observe that the block structures we get are, respectively, $[2][4][1][5][3]$ and $[3][1][5][2][4],$ and all of these blocks are non-empty. The only issue we might have is when both the second and fifth strings don't exist; we'll show that this is impossible.
%\par To see this, suppose that the second and fifth strings don't exist. Then, it implies that the first block is of length $a'$ and the fourth block is of length $d'.$ But then observe that the largest element in the first block is $a' + b'$ and the smallest element in the last block is $n - c' - d' + 1.$ But then we need $a' + b' < n - c' - d' + 1$ or that $a' + b' + c' + d' \leq n,$ which is a contradiction. This proves the lemma.
\end{proof}

\begin{replemma}{general non-crossing matching}
Let $a, b, c, d, e \ge 0$ be integers with $n = a + b + c + d + e$. Consider vertices labelled $1, 2, \ldots, 2n.$ Then there exists a unique coloring of these vertices with colors white and black and a non-crossing matching compatible with it such that the following conditions hold. 
\begin{enumerate}
    \item $[1, b+c+e]$ are colored black,
    
    \item in the interval $[b+c+e+1, a+2b+c+e]$, there are exactly $a$ black and $b$ white vertices, and there are no pairings between two vertices in this interval (which we will refer to as an ``internal pairing''),
    
    \item $[a+2b+c+e+1, a+b+e+n]$ are colored white,
    
    \item in the interval $[a+b+e+n + 1, 2n]$, there are exactly $d$ black and $c$ white vertices, and there are no internal pairings in this interval.
\end{enumerate}
\end{replemma}

\begin{proof}
The proof will proceed similarly to the proof of Lemma \ref{lem:simple-no-internal-pair}. We will induct on $c + d$. Our base case is $c = d = 0.$ In this case, we wish to show there is a unique coloring and non-crossing matching compatible with the coloring such that $[1, b + e]$ are colored black, $[b + e + 1, a + 2b + e]$ have no internal pairings and have $a$ black, $b$ white vertices, and $[a + 2b + e, 2n]$ are colored white. But this is just Lemma \ref{lem:simple-no-internal-pair} with the values $a+b, b+e, a+e,$ as $(a + b) + (b + e) + (a + e) = 2n,$ and then shifted by $a + b.$

For the inductive step, assume that we have shown that the lemma is true for $c + d = k-1$. Suppose we are given nonnegative integers $a, b, c, d, e$ with $c + d = k$. We can assume without loss of generality that $c \ge d.$ Otherwise, relabel the vertices such that $x$ is relabelled as $2n + 1 - c - d - x$ for $x \in [1, a + b + e + n]$ and $4n + 1 - c - d - x$ for $x \in [a + b + e + n + 1, 2n],$ and swap the colors black and white. This has the effect of swapping the variables $a$ with $b$ and $c$ with $d$. 

\par From our assumptions that $c \ge d$ and $c + d \ge 1$, we must have $c \ge 1$. 

\par Now, suppose a coloring and a non-crossing matching compatible with the coloring satisfying the properties in the statement of the lemma exist. We claim that $2n$ must be paired with $1$. If not, then $2n$ is paired with $x$ for some $2 \le x \le a + b + e + n = 2n-c-d$. Then among the vertices $[1, x-1]$, there must be equal number of white and black vertices. Otherwise, one of these vertices must be paired with a vertex with index at least $x,$ contradicting the definition of a non-crossing matching. 

\par We now show that this can't happen. If $1 < x \le b+c+e + 1$, then $[1, x-1]$ consists of only black vertices. Next, if $b +c + e + 1 < x \le a+2b+c+e + 1$, then $[1, x-1]$ has at least $b+c+e$ black vertices and at most $b$ white vertices by condition \textit{2}. This is a contradiction since $c \ge 1$, which means $b + c + e > b$. Finally, if $a+2b+c+e + 1 < x \le 2n-c-d$, then $[1, x-1]$ has at least $a+b+c+e$ black vertices and at most $b + (a+d+e-1)$ white vertices by conditions \textit{2} and \textit{3}, which is a contradiction since $c > d-1$. Thus, we must have $2n$ is paired with $1$. In particular, we also know that $2n$ is colored white, since $1$ is colored black.

Now, delete vertices $1$ and $2n,$ and relabel the vertices $[2, 2n-1]$ by reducing their index by $1.$ The remaining configuration must satisfy the following properties.
\begin{enumerate}
    \item $[1, b + c + e - 1]$ are colored black,
    \item in the interval $[b + c + e, a + 2b + c + e - 1],$ there are $a$ black and $b$ white vertices, and there are no internal pairings,
    \item $[a + 2b + c + e, a + b + e + n - 1]$ are colored white,
    \item in the interval $[a + b + e + n, 2(n-2)],$ there are exactly $d$ black and $c - 1$ white vertices, and there are no internal pairing in this interval.
\end{enumerate}
By the inductive hypothesis, this choice of non-crossing matching with compatible coloring on $2n - 2$ vertices is unique. However, this means that the choice of non-crossing matching and compatible coloring on the $2n$ vertices is also unique, since $1$ must be colored black, $2n$ must be colored white, and $1, 2n$ must be paired. This shows that if a coloring and compatible non-crossing matching satisfying the properties in the lemma statement exist, then they must be unique.
\par To see that one such choice of coloring and non-crossing matching exists, take the unique coloring and non-crossing matching on the vertices $[1, 2n - 2]$ (which exists by the inductive hypothesis) satisfying the four properties listed in the previous paragraph. Then, shift the labels of the vertices by $1,$ and add the vertices $1, 2n.$ From here, color $1$ black, $2n$ white, and pair $1$ and $2n.$ This is a coloring and compatible non-crossing matching that satisfies the conditions in the lemma. This finishes the inductive step and hence proves the claim. \qedhere 

% Using the conditions above, we will show that we must arrive at exactly one non-crossing matching and coloring, and that these are compatible.

% To begin, the white vertices in the second interval (that is, $[b+c+e+1,a+2b+c+e]$) cannot be paired with black vertices in the second interval because we do not allow internal pairings. Therefore, they can only be paired with black vertices in the first or the last interval (that is, $[1,b+c+e]$ or $[a+b+e+n+1,2n]$). We will show that the latter cannot hold.

% Assume for contradiction that some white vertex $i$ in the second interval is paired with a black vertex $j$ in the fourth interval. There are exactly $a$ black vertices in the second interval, so there can be at most $a - 1$ black vertices in the second interval with labels larger than $i$; there are exactly $d$ black vertices in the fourth interval, so there can be at most $d-1$ black vertices in the fourth interval with labels smaller than $j$. 

% Therefore, there can be at most $a+d - 2$ black vertices between the pair $ij$, but there are at least $a+d+e$ white vertices between the pair $ij$, since all vertices in the third interval (that is, $[a+2b+c+e+1, a+b+e+n]$) are white. But $a + d + e \geq a + d - 2,$ so any coloring and non-crossing matching satisfying the above conditions would have a white vertex $w$ paired with a vertex $v$ where $v > j$ or $v < i;$ but this is a contradiction since either $w < i < v < j$ or $i < v < j < w.$ Therefore, any non-crossing matching satisfying the above conditions must have every white vertex in the second interval paired with a black vertex in the first interval.

% Next, the black vertices in the fourth interval cannot be paired with white vertices in the fourth interval because we do not allow internal pairings. Therefore, they can only be paired with white vertices in the second or the third interval ($[a + 2b + c + e + 1, a + b + e + n]$). But as we argued before, no white vertex in the second interval can be paired with a black vertex in the fourth interval, so any non-crossing matching satisfying the above conditions must have all the black vertices in the fourth interval paired with white vertices in the third interval.

% Now, assume that there are some black vertex $i$ in the second interval with a smaller label than some white vertex $j$ in the second interval. By the above, $j$ must be paired with some black vertex $k$ in the first interval. Since $i$ is between $jk$, $i$ must be paired with a vertex between $jk$. But this is impossible because the only white vertices before $j$ are in the second interval and we do not allow internal pairing. Therefore, for a coloring satisfying the above conditions, $[b+c+e+1,2b+c+e]$ are all white and $[2b+c+e+1,a+2b+c+e]$ are all black. Furthermore, our non-crossing matching would need to have $i$ is paired with $2b+2c+2e+1-i$ for any $i\in [b + c + e + 1, 2b + c + e],$ and $i$ paired with $2a + 4b + 2c + 2e + 1 - i$ for $i \in [2b + c + e + 1, a + 2b + c + e].$

% Similarly, $[a+b+e+n+1,a+b+d+e+n]$ are all black and $[a+b+d+e+n+1,2n]$ are all white. Furthermore, $i$ is paired with $2a+2b+2e+2n+1-i$ for each $i\in [a+b+e+n+1, a + b + d + e + n].$
 
% Note that the colors of all vertices are determined and some pairs are determined. Notice that the unpaired vertices consists of $a+c+e$ black vertices with labels smaller than each of the $a+c+e$ white vertices, which must be paired with each other by having the $i$th smallest black vertex paired with $i$th largest white vertex. Hence, we've shown that there can be at most one coloring and compatible non-crossing matching satisfying the above conditions. 

% Our proof also constructs the coloring and compatible matching: the only part that isn't immediately obvious is that this is a non-crossing matching. But from our construction, for every pairing $ij,$ where $i < j,$ if $v$ lies in the interval $[i, j],$ then $v$ is paired with another element $w$ that also lies in the interval. This is enough to show that our matching is a non-crossing matching.

% This holds by construction if $i \in [c + e + 1, b + c + e] \cup [2a+2b+c+2e+1,a + b + e + n]$ (namely, pairing the white vertices in the second interval to the $b$ black vertices in the first interval with the largest labels, and pairing the black vertices in the fourth interval with the $d$ white vertices in the third interval with the largest labels). For the remaining $2a + 2c + 2e$ vertices, note that we have $[1, c + e] \cup [2b + c + e + 1, a + 2b + c + e]$ which are black and $[a + 2b + c + e + 1, 2a + 2b + c + 2e] \cup [a + b + d + e + n + 1,2n].$ Suppose that there existed $i, j, v, w$ so that $i, j$ are paired, $v, w$ are paired, and $i < v < j < w.$ Note that $i, v$ cannot both be these remaining vertices, by construction of our matching. But this is a contradiction, since we've shown this cannot happen if one of $i, v$ is not among these $2a + 2c +2e$ vertices.

% This shows that our matching is indeed a non-crossing matching, and hence proves the lemma.
\end{proof}

\begin{replemma}{non-crossing matching of case1}
Let $a, b, c, d, e$ be nonnegative integers, so $a, b, c, d \geq 1$ and $a+b+c+d+e=n$. Then, there is a unique non-crossing matching and a coloring compatible with it, such that the coloring satisfies
\begin{enumerate}
    \item $i$ is black for $i\in [a+1,n-d]$
    \item $i'$ is white for $i\in [b+1, n-c]$
    \item There are exactly $a$ black vertices and $b$ white vertices in $[1,a]\cup [1,b]'$
    \item There are exactly $d$ black vertices and $c$ white vertices in $[n-d+1,n]\cup [n-c+1,n]'$ 
    \item There are no pairings between two vertices in $[1,a]\cup [1,b]'$ (which we will refer to as an ``internal pairing")
    \item There are no internal pairings in $[n-d+1,n]\cup [n-c+1,n]'$
\end{enumerate}
\end{replemma}

\begin{proof}
Relabel the vertices such that $i$ is relabelled as $n - d + 1 - i$ for $i \in [1, n-d],$ relabelled as $3n - d  + 1 - i$ for $i \in [n - d + 1, n],$ and $i'$ is relabelled as $i + n - d$ for $i \in [1, n].$ Then, our conditions can be restated as requiring $[1, b + c + e]$ to be black, $[b + n - d + 1, 2n - c - d] = [a + 2b + c + e + 1, a + b + e + n]$ to be white, with $a$ black and $b$ white vertices in $[b + c + e + 1, a + 2b + c + e],$ and $d$ black and $c$ white vertices in $[2n - c - d + 1,2n] = [a + b + e + n + 1, 2n],$ with no internal pairings in the last two intervals. By Lemma \ref{general non-crossing matching}, there exists a unique coloring and non-crossing matching satisfying these conditions, as desired.
\end{proof}

\begin{comment}
\begin{proof}
Assuming the above conditions are satisfied. Suppose some white vertex $i\in [1,a]\bigcup [1,b]'$ is paired with some black vertex $j\in [a+b+1,a+b+e+c]$. By conditions \textit{1.,3.}, there are at most $b-1$ white vertices to the left of this pair, but at least $b$ black vertices to the left of this pair. $b-1<b$, so we cannot have a non-crossing matching. Suppose that some white vertex $i\in [1,a]\bigcup [1,b]'$ is paired with some black vertex $j\in [a+b+e+c+1,a+b+e+c+d]\bigcup [b+a+e+d+1,b+a+e+d+c]'$. By conditions \textit{1.,3.,4.}, there are at most $b-1+c$ white vertices to the left of this pair, but at least $b+e+c$ black vertices to the left of this pair. $b-1+c<b+c+e$, so we cannot have a non-crossing matching.

Therefore, any white vertex in $[1,a]\bigcup [1,b]'$ must be paired with a black vertex in $[a+1,a+b]$.

Now, for convenience, we re-index the $2n$ vertices starting from $a+b$ and go clockwise. That is, 
\begin{itemize}
    \item Vertex $i\in [1,a+b]$ has new index $a+b+1-i$
    \item Vertex $i\in [a+b+1, n]$ has new index $2n+a+b+1-i$
    \item Vertex $i'\in [1,n]'$ has new index $a+b+i$
\end{itemize}

Let's consider the first black vertex (with respect to the new indexing). Suppose it has new index less than $2b+1$, then by condition \textit{3.}, there exists a white vertex after $i_0$. By the above paragraph, $j_0$ must be paired with a black vertex $k$ in $[a+1,a+b]$ (with new index $[1,b]$). But then, $i_0$ is a black vertex to the left of the pair $j_0 k$, while the only white vertices to the left of the pair $j_0 k$ are in $[1,a]\bigcup [1,b]'$. This contradicts condition \textit{5.} Therefore, vertices with new index $[b+1,2b]$ are all whites and vertices with new index $[2b+1,2b+a]$ are all blacks. Similarly, vertices with new index $[a+b+n-c+1,a+b+n-c+d]'$ are all blacks and $[a+b+n-c+d+1,a+b+n+d]'$ are all whites.

Thus, we divide all $2n$ vertices into $7$ blocks of the same color.
\begin{itemize}
    \item Block $1$ consists of $b$ black vertices with new index $[1,b]$
    \item Block $2$ consists of $b$ white vertices with new index $[b+1,2b]$
    \item Block $3$ consists of $a$ black vertices with new index $[2b+1,2b+a]$
    \item Block $4$ consists of $a+e+d$ white vertices with new index $[2b+a+1,2b+a+a+e+d]$
    \item Block $5$ consists of $d$ black vertices with new index $[2b+2a+e+d+1,2b+2a+e+2d]$
    \item Block $6$ consists of $c$ white vertices with new index $[2b+2a+e+2d+1,2b+2a+e+2d+c]$
    \item Block $7$ consists of $e+c$ white vertices with new index $[2b+2a+e+2d+c+1,2n]$
\end{itemize}

Notice that block $2$ and $3$ cannot pair by condition \textit{5}. Also, block $5$ and $6$ cannot pair by condition \textit{6}. Therefore, the unique possible non-crossing matching satisfying the given conditions is obtained by 
\begin{itemize}
    \item Pairing block $1$ with block $2$
    \item Pairing block $3$ with the first $a$ vertices in block $4$
    \item Pairing block $5$ with the last $d$ vertices in block $4$
    \item Pairing block $6$ with the first $c$ vertices in block $7$
    \item Pairing the remaining $e$ vertices in block $4$ and the last $e$ vertices in block $7$. 
\end{itemize}
where ``first" and ``last" are in terms of the new indexing.
\par For instance, to see that the first two hold, observe that any vertex with new label $i$ in block $2$ must be paired with either something in block $5$ or block $1.$ However, notice that there are at most $a + d$ black vertices between these vertices. But there at least $a + e + d$ white vertices, which is a contradiction.
\par Therefore, we see that anything in block $2$ must be paired with block $1.$ From here, it follows that there is a unique way to form this matching. 
\end{proof}
\end{comment}

\begin{replemma}{non-crossing matching of case2}
Let $a, b, c, d, e, f$ be nonnegative integers where $a, b, c, d, \max(e, f) \geq 1,$ and $a+b+c+d+e+f=n$. Then, there is a unique non-crossing matching and a coloring compatible with it, such that the coloring satisfies
\begin{enumerate}
    \item $i$ is black for $i\in [1,a+e]$
    \item $i$ is white for $i\in [a+e+b + c+1, n]$
    \item $i'$ is black for $i\in [1,b+f]$
    \item $i'$ is white for $i\in [b+f+a+d+1,n]$
    \item There are exactly $c$ black vertices and $b$ white vertices in $[a+e+1,a+e+b+c]$
    \item There are exactly $d$ black vertices and $a$ white vertices in $[b+f+1,b+f+a+d]'$ 
    \item There are no pairings between two vertices in $[a+e+1,a+e+b+c]$ (which we will refer to as an ``internal pairing")
    \item There are no internal pairings in $[b+f+1,b+f+a+d]'$
\end{enumerate}
\end{replemma}
\begin{proof}
Let $\Tilde{a}=d,\Tilde{b}=a,\Tilde{c}=b,\Tilde{d}=c$, and $\Tilde{e}=e+f$. Adopt the setting of Lemma \ref{general non-crossing matching} where we take the five constants to be $\Tilde{a},\Tilde{b},\Tilde{c},\Tilde{d}$, and $\Tilde{e}$.

Relabel the vertices such that $i$ is labelled as $a + e + 1 - i$ for $i \in [1, a + e],$ labelled $2n + a + e + 1 - i$ for $i \in [a + e + 1, n],$ and $i'$ is labelled as $a + e + i$ for $i \in [1, n].$ Then, notice that we are looking for a coloring and non-crossing matching compatible with it such that $[1, a + e + b + f] = [1, \Tilde{b} + \Tilde{c} + \Tilde{e}]$ are black, $[2a + b + d + e + f + 1, n + a + e + d + f] = [\Tilde{a} + 2\Tilde{b} + \Tilde{c} + \Tilde{e} + 1, n + \Tilde{a} + \Tilde{b} + \Tilde{e}]$ are white, and we have $a$ white and $d$ black vertices in $[a +e + b + f + 1, 2a + b + d + e + f] = [\Tilde{b} + \Tilde{c} + \Tilde{e} + 1, \Tilde{a} + 2\Tilde{b} + \Tilde{c} + \Tilde{e}],$ and $c$ black and $b$ white vertices in $[n + a + e + d + f + 1, 2n] = [n + \Tilde{a} + \Tilde{b} + \Tilde{e} + 1,2n],$ with no internal pairings amongst the last two intervals. By Lemma \ref{general non-crossing matching}, there is a unique coloring and non-crossing matching compatible with it, as desired.
\end{proof}

\begin{comment}
\begin{proof}
Assuming the above conditions are satisfied. Suppose that some white vertex $i\in [a+e+1,a+e+b+c]$ is paired with some black vertex $j'\in [b+f+1,b+f+a+d]$. By conditions \textit{1.,3.,5.,6.,} there are at most $b+a-1$ white vertices above this pair but at least $a+e+b+f$ black vertices above this pair. $b+a-1<a+e+b+f$, so we cannot have a non-crossing matching. Therefore, any white vertex in $[a+e+1,a+e+b+c]$ must be paired with a black vertex in $[1,a+e]\bigcup [1,b+f]'$.

Consider the first black vertex (with smallest label) $i_0\in [a+e+1,a+e+b+c]$. If $i_0\in [a+e+1,a+e+b]$, then by condition \textit{5.}, there exists white vertices $j_0\in [i_0+1, a+e+b+c]$ after $i_0$. By the above paragraph, $j_0$ must be paired with a black vertex $k'$ in $[1,a+e]\bigcup [1,b+f]'$. But then, $i_0$ is a black vertex to the left of the pair $j_0 k'$, while the only white vertices to the left of the pair $j_0 k'$ are in $[a+e+1,a+e+b+c]$. This contradicts condition \textit{7.} Hence, $[a+e+1,a+e+b]$ are all white and $[a+e+b+1,a+e+b+c]$ are all black. Similarly, $[b+f+1,b+f+a]'$ are all white and $[b+f+a+1,b+f+a+d]'$ are all black.

For convenience, we now re-index the $2n$ vertices starting from $a+e+b$ and go clockwise. That is, 
\begin{itemize}
    \item Vertex $i\in [1,a+e+b]$ has new index $a+e+b+1-i$
    \item Vertex $i\in [a+e+b+1, n]$ has new index $2n+a+e+b+1-i$
    \item Vertex $i'\in [1,n]'$ has new index $a+e+b+i$
\end{itemize}
We divide all $2n$ vertices into $6$ blocks of the same color.
\begin{itemize}
    \item Block $1$ consists of $b$ white vertices with new index $[1,b]$
    \item Block $2$ consists of $e+a+b+f$ black vertices with new index $[b+1,b+e+a+b+f]$
    \item Block $3$ consists of $a$ white vertices with new index $[b+e+a+b+f+1,b+e+a+b+f+a]$
    \item Block $4$ consists of $d$ black vertices with new index $[b+e+a+b+f+a+1,b+e+a+b+f+a+d]$
    \item Block $5$ consists of $e+c+d+f$ white vertices with new index $[b+e+a+b+f+a+d+1,b+e+a+b+f+a+d+e+c+d+f]$
    \item Block $6$ consists of $c$ black vertices with new index $[b+e+a+b+f+a+d+e+c+d+f+1,2n]$
\end{itemize}
Notice that block $1$ and $6$ cannot pair by condition \textit{7}. Also, block $3$ and $4$ cannot pair by condition \textit{8}. It follows that the unique possible non-crossing matching satisfying the given conditions is obtained by 
\begin{itemize}
    \item Pairing block $1$ with the first $b$ vertices in block $2$
    \item Pairing block $3$ with the last $a$ vertices in block $2$
    \item Pairing block $4$ with the first $d$ vertices in block $5$
    \item Pairing block $6$ with the last $c$ vertices in block $5$
    \item Pairing up the remaining middle $e+f$ vertices in block $2$ and the remaining middle $e+f$ vertices in block $5$ in reverse order. 
\end{itemize}
where the "first", "last", and "reverse order" are in terms of the new indexing.
\par To see this, for instance, notice that if the vertex with new index $i$ is paired with $j,$ notice that $j$ is either in blocks $2$ or $4,$ so $j > i.$ But then notice that if it is paired with something in block $4,$ we have at least $a + e + b + f$ black vertices and at most $a + b$ white vertices; but at least one of $e, f$ is $1,$ which is a contradiction. This gives us the first pairing condition. 
\end{proof}
\end{comment}

\begin{replemma}{ncm equals w case1}
Let $w \in \fkS_n$ have block structure $[2][1][3][5][4]$ with block lengths $a,b,e,c,d$, as stated in the first case of Proposition \ref{2143Patterns}. The non-crossing matching of $w$ is exactly the non-crossing matching in Lemma \ref{non-crossing matching of case1}.
\end{replemma}
\begin{proof}
\par The first part of the proof is to determine $\ncm(w)$. This can be done laboriously using Proposition \ref{prop:ncm}, but for our purposes, it suffices to deduce some structural properties of $\ncm(w)$ that can be directly gleaned from Lemma \ref{lem:matching_compatibility} and Proposition \ref{prop:ncm}(a).

\par Construct the coloring of $\ncm(w)$ in Lemma \ref{lem:matching_compatibility}, with fixed points colored arbitrarily. Then, using the one-line notation for $w$ in Proposition \ref{2143Patterns} case 1, we deduce that the vertices in $B_1 = [1,a] \cup [1,b]'$ are all colored black, while the vertices in $W_1 = [a+1, a+b] \cup [b+1, a+b]'$ are all colored white. Thus, since $\ncm(w)$ is consistent with this coloring, there are no pairings between two vertices in $B_1$. There are also no pairings between two vertices in $W_1$. Finally, by Lemma \ref{lem:matching_compatibility}, a white vertex $i$ or $i'$ in $W_1$ must be paired with some black vertex $j$ or $j'$ with $j \le i \le a+b$. The only black vertices in $[1, a+b] \cup [1, a+b]'$ belong to $B_1$, so each vertex in $W_1$ is paired with some vertex in $B_1$. Since $|W_1| = |B_1| = a+b$, this map is bijective.
%\par Indeed, by Proposition \ref{prop:ncm}, if $i$ is paired with $f(i),$ then we have some path in the wiring diagram $D(w)$ from $i$ to $f(i).$ However, if $x \in [1, a+b],$ and the segment from $x$ to $w(x)'$ intersects with that of $y$ to $w(y)'$ for some $y \neq x,$ then either $y < x$ and $w(x) > w(y),$ or vice versa. But $w([1, a+b]) = [1, a+b]$ requires that $y \in [1, a+b]$ as well. Hence, this path that starts at $i \in [1, a+b]$ only travels along segments connecting elements of $[1, a+b]$ to $[1, a+b]'.$ Thus, each vertex in $[1,a] \bigcup [1,b]'$ is paired with some vertex in $[a+1,a+b] \bigcup [b+1,a+b]'$. 
\par A similar argument shows that each vertex in $W_2 = [n-d+1,n] \cup [n-c+1,n]'$ is paired with some vertex in $B_2 = [n-c-d, n-d] \cup [n-c-d, n-c]'$ (in particular, there are no internal pairings between vertices in $W_2$). Finally, note that for $i \in F := [a + b + 1, n - c - d]$, we have $w(i) = i$ by Proposition \ref{2143Patterns}, so $i$ is paired with $i'$ in $\ncm(w)$ by Proposition \ref{prop:ncm}(a). %To see this, by construction we know that $w(i) = i$ for $i \in [a + b + 1, n - c - d],$ and furthermore if $x > i$ but $w(x) < w(i),$ then $x > i > w(x),$ so $x \not \in [a + b + 1, n - c - d],$ or that $x \in [1, a+b]$ or $[n - c - d + 1, n].$ But then this is impossible: this means that the only path from $i$ along the wiring diagram, for $i \in [a + b + 1, n - c - d],$ must travel along the segment from $i$ to $i',$ and so $i$ is paired with $i'.$

\par Using the properties of $\ncm(w)$ deduced from above, we will create a new coloring compatible with $\ncm(w)$ (\textbf{not} based on Lemma \ref{lem:matching_compatibility}) that satisfies the conditions of Lemma \ref{non-crossing matching of case1}. First, conditions \textit{5} and \textit{6} of Lemma \ref{non-crossing matching of case1} are satisfied from our above work. Next, we follow conditions \textit{1} and \textit{2} of Lemma \ref{non-crossing matching of case1} and color $i$ black for $i \in [a+1, n-d]$ and $i'$ white for $i \in [b+1, n-c]$. This determines the colors of vertices in $W_1, F, F'$, and $B_2$. Since $\ncm$ pairs vertices in $W_1$ with $B_1$, $W_2$ with $B_2$, and $F$ with $F'$, our coloring is consistent so far. Furthermore, we can easily extend our coloring to all of $[1, n] \cup [1, n]'$: for each black vertex $v_1 \in B_1$ paired to some white vertex $v_2 \in W_1$, we assign $v_1$ the opposite color of $v_2$, and likewise for vertices in $W_2$.
%Note that, so far, no two paired vertices in $\ncm(w)$ are colored the same: the only pairs where both vertices have been colored are pairs $(i, i'),$ where $i \in [a + b + 1, n - c - d]$ (as, for instance, the vertices in $[a +1, a+b]$ are paired to those in $[1, a] \cup [1, b]'$). 
%\par Then, there is a unique way to extend the coloring to $[1,n] \bigcup [1,n]'$ such that the coloring is compatible with $\ncm(w),$ since the color of each vertex in $[1, a] \cup [1, b]'$ is determined by the color of the vertex it is paired to, which lies in $[a+1, a+b] \cup [b+1, a+b]'$ (which by construction has already been determined).
\par Now, we verify the remaining conditions of Lemma \ref{non-crossing matching of case1}. Condition \textit{3} is satisfied because there are exactly $a$ white vertices and $b$ black vertices in $W_1$, which will thus pair with $a$ black vertices and $b$ white vertices in $B_1 = [1,a] \cup [1,b]'$. Similarly, condition \textit{4} of Lemma \ref{non-crossing matching of case1} is also satisfied.

% Notice that the permutation $w$ can be thought of as the composition of two permutations that commute with each other. Indeed, observe that $w = w_1w_2,$ where we have that $$w_1(i) = \begin{cases}  i + b \text{ if } i \leq a \\ i - a \text{ if } a < i \leq a + b \\  i \text{ otherwise } \end{cases}$$ and $$w_2(i) = \begin{cases}  i - c \text{ if } i > n-d \\  i + d \text{ if } n - c - d  < i \leq n-d \\  i \text{ otherwise } \end{cases}.$$ But then we may employ Lemma \ref{wireTBLemma} on $w_1, w_2$ to obtain the
% non-crossing matching for $w,$ since $w_1, w_2$ commute, and employing Proposition \ref{prop:tl}.
% \par It's not hard to check then that the properties in Lemma \ref{non-crossing matching of case1} hold. Indeed, we have that $a + i$ is paired with $a - i + 1$ for $i \in [1, \min(a, b)],$ and $(b + i)'$ is paired with $(b - i + 1)'$ for $i \in [1, \min(a, b)].$ If $a = b$ this determines the entire pairing; otherwise without loss of generality, $a > b.$ Then from Lemma \ref{wireTBLemma}, we have that $a - b$ is paired with $(a+b)',$ $a-b-1$ with $(a+b-1)',$ and so forth, until $1$ is paired with $(2b + 1)'.$ But then we see that, given a coloring satisfying conditions 1 and 2, conditions 3 and 5 hold (the white vertices are $a, a-1, \ldots, a-b+1$). The same argument with $c, d$ allow us to conclude that conditions 4, 6 hold.
\par Thus, it follows that $\ncm(w)$ is the unique matching described in Lemma \ref{non-crossing matching of case1}, as desired.
% By uniqueness of the non-crossing matching satisfying the conditions listed in Lemma \ref{non-crossing matching of case1}, we only need to show that
% \begin{itemize}
%     \item The coloring described in the proof of Lemma \ref{non-crossing matching of case1} is compatible with the non-crossing matching of $w$.
%     \item Conditions \textit{5.} and \textit{6.} in Lemma \ref{non-crossing matching of case1} are satisfied.
% \end{itemize} 
% We divide the left $n$ vertices into $5$ blocks in increasing order from $1$ to $n$ given by
% \begin{itemize}
%     \item Block $1$ consists of $a$ vertices
%     \item Block $2$ consists of $b$ black vertices
%     \item Block $3$ consists of $e$ black vertices
%     \item Block $4$ consists of $c$ black vertices
%     \item Block $5$ consists of $d$ vertices
% \end{itemize}
% Similarly, we divide the right $n$ vertices into $5$ blocks in increasing order from $1$ to $n$ given by
% \begin{itemize}
%     \item Block $1'$ consists of $b$ vertices
%     \item Block $2'$ consists of $a$ white vertices
%     \item Block $3'$ consists of $e$ white vertices
%     \item Block $4'$ consists of $d$ white vertices
%     \item Block $5'$ consists of $c$ white vertices
% \end{itemize}
% Then, $w$ is determined by 
% \begin{itemize}
%     \item Block $1$ goes to Block $2'$
%     \item Block $2$ goes to Block $1'$
%     \item Block $3$ goes to Block $3'$
%     \item Block $4$ goes to Block $5'$
%     \item Block $5$ goes to Block $4'$
% \end{itemize}
% such that $(i,j)$ is not an inversion if $i$ and $j$ are in the same block. Notice that each $(i,j)$ is an inversion for $i\in$ Block 1 and $j\in$ Block 2. Therefore, after we resolve the crossings in the diagram of $w$ to form its non-crossing matching, using Lemma \ref{wireTBLemma}, vertices in Block 2 will be paired with the first few (in terms of the new indexing introduced in the proof of Lemma \ref{non-crossing matching of case1}) vertices in Block $1$ and $1'$, matching the non-crossing matching given in Lemma \ref{non-crossing matching of case1}. 
% \par This is the same case for vertices in Block $4$, $5'$, and $4'$. And the remaining $e$ vertices in the middle of both sides are paired with each other in increasing order. This is because for any $i\in$ Block 1 and $j\in$ Block 2, $(i,j)$ is an inversion. Thus, when we resolve the crossings to form the non-crossing matching, $j$ is paired with a vertex above $w(j)$. 
% \par This means that the non-crossing matching of $w$ is indeed compatible with the desired coloring and satisfies the desired non-internal matching property, which proves the lemma.
\end{proof}

\begin{replemma}{ncm equals w case2}
Let $w \in \fkS_n$ have block structure $[3][5][1][6][2][4]$ with block lengths $a,e,b,c,f,d$, as stated in the second case of Proposition \ref{2143Patterns}. The non-crossing matching of $w$ is exactly the non-crossing matching in Lemma \ref{non-crossing matching of case2}.
\end{replemma}
\begin{proof}
% By uniqueness of the non-crossing matching satisfying the conditions listed in Lemma \ref{non-crossing matching of case2}, we only need to show that
% \begin{itemize}
%     \item The coloring described in the proof of Lemma \ref{non-crossing matching of case2} is compatible with the non-crossing matching of $w$.
%     \item Conditions \textit{7} and \textit{8} in Lemma \ref{non-crossing matching of case2} are satisfied.
% \end{itemize}
Consider the following coloring.
\begin{itemize}
    \item $[1, a+e], [a+e+b+1,a+e+b+c], [1, b+f]', [b+f+a+1, b+f+a+d]'$ are black,
    
    \item $[a+e+1, a+e+b], [a+e+b+c+1, n], [b+f+1, b+f+a]', [b+f+a+d+1, n]'$ are white.
\end{itemize}
Using the one-line notation for $w$ in Proposition \ref{2143Patterns} case 2, we check that this coloring satisfies the conditions in Lemma \ref{lem:matching_compatibility}, and so this coloring is compatible with $\ncm(w)$.
\par Furthermore, note that this coloring satisfies conditions \textit{1-6} in Lemma \ref{non-crossing matching of case2}. Next, we check conditions \textit{7-8}. From Lemma \ref{lem:matching_compatibility}, we know that any black vertex has a smaller or equal label than the white vertex it is paired with in $\ncm(w)$. Thus, $\ncm(w)$ cannot have pairings between any two vertices in $[a + e + 1, a + e + b + c],$ since $[a + e + 1, a + e + b]$ are colored white and $[a + e + b + 1, a + e + b + c]$ are colored black. The same holds for the vertices among $[b + f + 1, b + f + a + d]'.$ Thus, conditions \textit{1-8} in Lemma \ref{non-crossing matching of case2} are satisfied, so $\ncm(w)$ is the unique non-crossing matching given in Lemma \ref{non-crossing matching of case2}. \qedhere
% Then, $w$ is determined by 
% \begin{itemize}
%     \item Block $1$ goes to Block $3'$
%     \item Block $2$ goes to Block $5'$
%     \item Block $3$ goes to Block $1'$
%     \item Block $4$ goes to Block $6'$
%     \item Block $5$ goes to Block $2'$
%     \item Block $6$ goes to Block $4'$
% \end{itemize}
% such that $(i,j)$ is not an inversion if $i$ and $j$ are in the same block.

%Notice that each $(i,j)$ is an inversion for $i\in$ Block 1 $\bigcup$ Block 2 and $j\in$ Block 3. But then, after we resolve the crossings in the diagram of $w$ to form its non-crossing matching, using Lemma \ref{wireTBLemma}, vertices in Block 3 will be paired with the last few (in terms of the new indexing introduced in the proof of Lemma \ref{non-crossing matching of case2}) vertices in Block $1$, $2$, and $1'$, matching the non-crossing matching given in Lemma \ref{non-crossing matching of case2}. 
%\par This is the same case for vertices in Block $4$, $3'$, and $4'$. And for the remaining $e+f$ vertices on the ``top", they must be paired with the remaining $e+f$ vertices on the ``bottom." This is because for any $i\in$ Block 1 $\bigcup$ Block 2, $(j,i)$ is not an inversion for any $j<i$. Thus, when we resolve the crossings to form the non-crossing matching, if $i$ is above $j,$ then the vertex $i$ is paired with is above the vertex $j$ is paired with i.e., it ends up pairing with elements larger (in terms of the new indexing introduced in the proof of Lemma \ref{non-crossing matching of case2}) than its image under $w$. 
%\par This means that the non-crossing matching of $w$ is indeed compatible with the desired coloring and satisfies the desired non-internal matching property, which proves the lemma.
\end{proof}

\begin{comment}
Let $a = w^{-1} (1)-1$ be the size of the first block, $b = w(1)-1$ be the size of the second block, $e$ be the size of the third block, $c = n - w(n)$ be the size of the fourth block, and $d = n - w^{-1} (n)$ be the size of the last block.
\end{comment}

\section*{Acknowledgements}
This project was partially supported by RTG grant NSF/DMS-1148634, DMS-1949896, and the Office of Undergraduate Research at Washington University in St. Louis. It was supervised as part of the University of Minnesota School of Mathematics Summer 2021 REU program. 
The authors would like to thank Professor Pavlo Pylyavskyy for introducing the problem and offering helpful directions for research and Sylvester Zhang for their mentorship and helpful comments on the paper. In addition, the authors would like to thank Swapnil Garg and Brian Sun for their algorithmic and coding support for this project.

%This research is carried out as part of the 2021 Combinaotrics REU program at the University of Minnesota, supported by ##Grant number##

\bibliographystyle{amsplain}
\bibliography{paper1.bib}

\end{document}
