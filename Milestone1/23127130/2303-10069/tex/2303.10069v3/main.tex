\documentclass[aps, prl, floatfix,
  reprint, %Formatting option (1/3)
  % 11pt, onecolumn, %Formatting option (2/3)
  % preprint, %Formatting option (3/3)
  % groupedaddress %Title setting (1/2)
  superscriptaddress, %Title setting (2/2)
  longbibliography, noeprint,
  % linenumbers
  % nofootinbib, linenumbers, draft
]{revtex4-2} % for review and submission
\usepackage[T1]{fontenc}
\usepackage[utf8]{inputenc}
\usepackage{microtype}
\usepackage{graphicx, xcolor} % for figures, no options for submitting to arXiv
\usepackage[notrig]{physics} % for equations
\usepackage{amsmath, amssymb}
\usepackage{mathtools} %for \coloneqq
\usepackage{csquotes} % for quotations such as \enquote
\usepackage{capt-of} %for \cptionof
\usepackage{siunitx} 
    \sisetup{group-separator = {,}, group-minimum-digits=4}
\usepackage[pdftex,unicode,pdfusetitle]{hyperref}
\hypersetup{
    colorlinks=true,
    citecolor=blue,
    linkcolor=magenta,
    urlcolor=blue,
% 	bookmarks=true, 
    bookmarksnumbered=true,
    pdfborder={0 0 0},
    bookmarkstype=toc,
    pdfauthor={Shoki Sugimoto, Ryusuke Hamazaki, and Masahito Ueda}
}

%%%%%%%%%%%%%%%%%%%% For Theorem environments %%%%%%%%%%%%%%%%%%%%
\usepackage{amsthm} % for theorem environments
\newtheorem{theorem}{Theorem}
\newtheorem{conjecture}{Conjecture}
\newtheorem{lemma}{Lemma}
\newtheorem{definition}{Definition}
\renewcommand{\qed}{$\hfill\blacksquare$}

\usepackage{tcolorbox}
\tcbuselibrary{theorems, skins, breakable}
\tcbset{thm/.style={enhanced, drop fuzzy shadow, boxrule=1pt, colback=lightgray!20,colframe=gray!90,colbacktitle=blue!10,coltitle=black,fonttitle=\bfseries\strut}}
\newtcbtheorem[number within=section]{framedDefinition}{Definition}{thm}{def}
\newtcbtheorem[number within=section]{framedProposition}{Proposition}{thm}{prop}
%%%%%%%%%%%%%%%%%%%% (END)For Theorem environments %%%%%%%%%%%%%%%%%%%%

\newcommand{\dimLoc}{ d_{\mathrm{loc}} }
\newcommand{\dimTot}{ D }
\newcommand{\particleNumber}{ N }
\newcommand{\EigStateOp}[1]{ \hat{\rho}_{#1} }
\newcommand{\MCE}[1][]{ \hat{\rho}^{(\mathrm{mc})}_{#1} } % Micro-Canonical Ensemble
\newcommand{\opRange}[2][1]{ \norm*{ #2 }_{#1} }
\newcommand{\opSet}{ \mathcal{A} }
\DeclareDocumentCommand{\seminorm}{ s O{\opSet} m m }{%
  \IfBooleanTF{#1}
    {\norm*{#4}_{#3}^{(#2)}}
    {\norm{#4}_{#3}^{(#2)}}%
}
\newcommand{\ETHmeasure}[3][1]{ \Lambda_{#1}^{(#2, #3)} }
\newcommand{\Prob}{\mathbb{P}}
\newcommand{\upperBound}[1][]{#1{\alpha}_{\mathrm{U}}}
\newcommand{\lowerBound}[1][]{#1{\alpha}_{\mathrm{L}}}
\newcommand{\Supple}{Supplementary Information}
\newcommand{\Methods}{Methods}
\usepackage{pifont}% http://ctan.org/pkg/pifont
\newcommand{\cmark}{\ding{51}}%
\newcommand{\xmark}{\ding{56}}%
%%%%%%%%%%%%%%%%%%%%%%%%%%%%%%
\newcommand{\sysSize}{ V }
\newcommand{\subSystem}{ \mathcal{S} }

\DeclareMathOperator*{\argmax}{arg\,max}
\DeclareMathOperator{\vecspan}{span}

\setcounter{dbltopnumber}{1}
%%%%%%%%%%%%%%%%%%%% For improvement %%%%%%%%%%%%%%%%%%%%
% \usepackage{ulem}
% \newcommand{\rhcomment}[1]{\textcolor{magenta}{ #1 }}
% \newcommand{\rhmodify}[2]{\sout{ #1 }\textcolor{magenta}{ #2 }}
% \newcommand{\fix}[2]{\sout{#1} \textcolor{blue}{#2}}
% \newcommand{\red}[1]{\textcolor{red}{#1}}
% \newcommand{\blue}[1]{\textcolor{blue}{#1}}
%%%%%%%%%%%%%%%%%%%% (END)For improvement %%%%%%%%%%%%%%%%%%%%

\usepackage[resetlabels]{multibib}
\newcites{SM}{References}
\usepackage{etoolbox}      % ← 追加
\makeatletter
  \robustify\citeSM
\makeatother
\makeatletter
\def\frontmatter@maketitle{%
  \@author@finish
  \title@column\titleblock@produce
  \suppressfloats[t]%
%  \let\and\relax
%  \let\affiliation\@gobble
%  \let\author\@gobble
%  \let\@AAC@list\@empty
%  \let\@AFF@list\@empty
%  \let\@AFG@list\@empty
%  \let\@AF@join\@AF@join@error
%  \let\email\@gobble
%  \let\@address\@empty
%  \let\maketitle\relax
%  \let\thanks\@gobble
  \let\abstract\@undefined\let\endabstract\@undefined
  \titlepage@sw{%
   \vfil
   \clearpage
  }{}%
  \onecolumn@grid@setup
  \def\set@footnotewidth{\set@footnotewidth@one}%
}%
\makeatother

%%%%%%%%%%%%%%%%%%%%%%%%%%%%%%%%%%%%%%%%%%%%%%%%%%%%%%%%%%%%%%
%%%%%%%%%%%%%%%%%%%%%%%%%%%%%%%%%%%%%%%%%%%%%%%%%%%%%%%%%%%%%%
\begin{document}
\title{
    Bounds on eigenstate thermalization
}
%%%%%%%%%%%%%%%%%%%% Authors and Affiliations %%%%%%%%%%%%%%%%%%%%
\author{Shoki Sugimoto}
  \email{shoki.sugimoto@ap.t.u-tokyo.ac.jp}
  \affiliation{Department of Applied Physics, The University of Tokyo, 7-3-1 Hongo, Bunkyo-ku, Tokyo 113-0033, Japan}
  \affiliation{Nonequilibrium Quantum Statistical Mechanics RIKEN Hakubi Research Team, RIKEN Pioneering Research Institute (PRI), Wako, Saitama 351-0198, Japan}
\author{Ryusuke Hamazaki}
  \affiliation{Nonequilibrium Quantum Statistical Mechanics RIKEN Hakubi Research Team, RIKEN Pioneering Research Institute (PRI), Wako, Saitama 351-0198, Japan}
  \affiliation{RIKEN Center for Interdisciplinary Theoretical and
Mathematical Sciences (iTHEMS), RIKEN, Wako 351-0198, Japan
}
\author{Masahito Ueda}
  \affiliation{Department of Physics, The University of Tokyo, 7-3-1 Hongo, Bunkyo-ku, Tokyo 113-0033, Japan}
  \affiliation{RIKEN Center for Emergent Matter Science (CEMS), Wako 351-0198, Japan}
%%%%%%%%%%%%%%%%%%%% (END) Authors and Affiliations %%%%%%%%%%%%%%%%%%%%
\begin{abstract}
    %
    The eigenstate thermalization hypothesis~(ETH), which asserts that every eigenstate of a many-body quantum system is indistinguishable from a thermal ensemble, plays a pivotal role in understanding thermalization of isolated quantum systems.
    %
    Yet, no evidence has been obtained as to whether the ETH holds for \textit{all} few-body operators in a chaotic system; such few-body operators include key quantities in statistical mechanics, such as the total magnetization, the momentum distributions, and their low-order thermal and quantum fluctuations.
    % 
    % Main result
    Here, we formulate a conjecture that for a generic nonintegrable system the ETH holds for all $m$-body operators with $m < \alpha_{\ast} \particleNumber$ in the thermodynamic limit for some nonzero constant $\alpha_{\ast} > 0$.
    %
    We first prove this statement for systems with Haar-distributed energy eigenstates to analytically motivate our conjecture.
    %
    We then verify the conjecture for generic spin, Bose, and Fermi systems with local and few-body interactions by large-scale numerical calculations.
    % Implications
    Our results imply that generic systems satisfy the ETH for \textit{all} few-body operators, including their thermal and quantum fluctuations.
\end{abstract}
\maketitle
\addtocontents{toc}{\string\tocdepth@munge} % should be placed after \maketitle

%%%%%%%%%%%%%%%%%%%%%%%%%%%%%%%%%%%%%%%%%%%%%%%%%%%%%%%%%%%%%%
%%%%%%%%%%%%%%%%%%%%%%%%%%%%%%%%%%%%%%%%%%%%%%%%%%%%%%%%%%%%%%
% \paragraph{Introduction.---}
% (interest) statistical mechanics
% 
Recent experiments in cold atoms and ions have demonstrated that quantum systems thermalize unitarily without heat reservoirs~\cite{trotzky2012probing, Langen2013-zj, clos2016time, kaufman2016quantum, neill2016ergodic, tang2018thermalization, orioli2018relaxation, langen2015ultracold}.
This finding brings up a striking possibility of integrating statistical mechanics into a single framework of quantum mechanics---a scenario first envisioned by von Neumann about a century ago~\cite{von2010proof}.
% 
It has been argued that a single pure quantum state becomes indistinguishable from a thermal ensemble as far as few-body observables are concerned~\cite{popescu2006entanglement, goldstein2006canonical, reimann2007typicality,
tasaki2016typicality, polkovnikov2011colloquium, d2016quantum, gogolin2016equilibration, mori2018thermalization, deutsch2018eigenstate} due to the interplay between quantum entanglement and physical constraints on observable quantities such as locality or few-bodiness~\cite{popescu2006entanglement, goldstein2006canonical, polkovnikov2011colloquium, d2016quantum, gogolin2016equilibration, mori2018thermalization, deutsch2018eigenstate}.
Depending on the nature of operators used to distinguish between a quantum state and a thermal ensemble,
various notions of quantum-thermal equilibrium, such as microscopic thermal equilibrium~(MITE) and macroscopic thermal equilibrium~(MATE), have been introduced~\cite{popescu2006entanglement, goldstein2006canonical, Goldstein2015, Goldstein2017, tasaki2016typicality, mori2018thermalization}.

The eigenstate thermalization hypothesis (ETH)~\cite{von2010proof, deutsch1991quantum, srednicki1994chaos} is widely believed to be the primary mechanism behind thermalization in isolated quantum systems.
The ETH for an operator $\hat{A}$ means that (i) every energy eigenstate of a system is in thermal equilibrium with respect to the expectation value of $\hat{A}$ and that (ii) off-diagonal elements of $\hat{A}$ in an energy eigenbasis is vanishingly small.
% 
The ETH ensures thermalization of $\hat{A}$ for any initial state with a macroscopically definite energy, barring massive degeneracy in the energy spectrum~\cite{polkovnikov2011colloquium, d2016quantum, gogolin2016equilibration, mori2018thermalization, deutsch2018eigenstate}.
The ETH has been tested numerically for several local or few-body quantities~\cite{rigol2008thermalization, beugeling2014finite, kim2014testing, sugimoto2021test, Sugimoto2022, rigol2009quantum, rigol2009breakdown, rigol2010quantum, biroli2010effect, santos2010localization, steinigeweg2013eigenstate, mondaini2016eigenstate, mondaini2017eigenstate, jansen2019eigenstate, Schonle_PhysRevB2021y}.
However, whether the ETH holds for \textit{all} few-body operators and whether it breaks down for many-body operators have yet to be fully addressed.

Von Neumann~\cite{von2010proof} and Reimann~\cite{reimann2015generalization} proved the ETH for \textit{almost all} Hermitian operators.
However, their results do not imply that the ETH holds true for physically realistic operators because \textit{almost all} operators considered in Refs.~\cite{von2010proof, reimann2015generalization} involve highly nonlocal correlations that are close to $N$-body~\cite{hamazaki2018atypicality} and therefore unphysical.
Some works~\cite{garrison2018does, mori2018thermalization, dymarsky2018subsystem} tested the ETH against several classes of few-body operators, such as local operators of a subsystem; however, these approaches are inherently limited to subsystem observables and are not designed to handle generic few-body operators that act on an entire system, such as the total magnetization, the momentum distribution, and their thermal and quantum fluctuations.


In this Article, we submit a conjecture that the ETH holds true for \textit{all} operators including $\order{N}$-body ones.
More precisely, we formulate the following conjecture:
\begin{conjecture} \label{conj_Conjecture}
    For a generic nonintegrable system, 
    there exists a positive constant $\alpha > 0$ such that the ETH holds for all $m$-body operators with $m < \alpha \particleNumber$ in the thermodynamic limit.
\end{conjecture}
\noindent
Let $\alpha_\ast$ be the largest such constant.
As a theoretical motivation, we first prove this conjecture for systems with Haar-distributed energy eigenstates by establishing explicit bounds, $\lowerBound^{\mathrm{H}} \leq \alpha_{\ast} \leq \upperBound^{\mathrm{H}}$~(Figure~\ref{fig_RegionsOfM}).
While such systems involve highly nonlocal and $\order{\particleNumber}$-body interactions, they constitute an analytically tractable model and serve as a theoretical reference for investigating the ETH measure in systems with local and few-body interactions.
\begin{figure*}[tb]
    \centering
    \includegraphics[width=\linewidth]{Figure1_ETHRegions.pdf}
    \caption{\textbf{Regions of $m/\particleNumber$ where the ETH typically holds and breaks down for systems with Haar-distributed energy eigenstates.}
    The blue region (marked with \enquote{\cmark}) shows where the ETH typically holds for \textit{all} operators in the space $\opSet_{\particleNumber}^{[0,m]}$ of $m$-body operators.
    This region is delimited by $\lowerBound^{\mathrm{H}}$---the largest value of $\alpha\:(\coloneqq m/\particleNumber)$ for which $\dim \opSet_{\particleNumber}^{[0, \alpha \particleNumber]} / D \to 0$ in the thermodynamic limit.
    Here, $D \coloneqq \dim\mathcal{H}$ is the Hilbert-space dimension, $\particleNumber$ is the number of spins or particles, and $V$ denotes the system size (with $N = V$ for spin systems).
    The red region (marked with \enquote{\xmark}) shows where ETH-breaking operators typically exist.
    This region is delimited by $\upperBound^{\mathrm{H}}$---the smallest value of $\alpha$ for which $\dim \opSet_{\particleNumber}^{[0, \alpha \particleNumber]} / D^2$ converges to a nonzero value in the thermodynamic limit.
    The threshold $\alpha_{\ast}$ lies somewhere in the white region (marked with \enquote{?}).
    }
    \label{fig_RegionsOfM}
\end{figure*}

We then test Conjecture~\ref{conj_Conjecture} numerically in generic spin, Bose, and Fermi systems with local and few-body interactions, employing finite-size scaling of the bounds on the ETH measure~(Figure~\ref{fig_Numerics_ShortRangeEnsemble}).
We compare the behavior of these numerical bounds, as we increase $N$ with $m/N$ held fixed, to the corresponding analytic behavior derived for Haar-random systems.
% We compare the behavior of these numerical bounds in increasing $\particleNumber$ with the corresponding analytic behavior derived for Haar-random systems. 
The observed qualitative agreement supports extrapolating the numerical results to the thermodynamic limit.
This yields nontrivial lower bounds on $\alpha_{\ast}$ across approximately the central 40\% of the energy spectrum (Figure~\ref{fig_BoundsEnergyDependence}), thereby verifying Conjecture~\ref{conj_Conjecture} for the systems under investigation.

These results suggest that our method remains effective even in analyzing systems with non-negligible finite-size effects, such as trapped ions and cold atoms, where the thermodynamic limit is inaccessible~\cite{trotzky2012probing, Langen2013-zj, clos2016time, kaufman2016quantum, neill2016ergodic, tang2018thermalization, gring2012relaxation, neyenhuis2017observation}.
Moreover, our result applies directly to any few-body operator of interest in statistical mechanics, including their thermal and quantum fluctuations, without relying on coarse-graining or other approximations.
In particular, the ETH is expected to hold even for \textit{arbitrary} low-order fluctuations of \textit{arbitrary} few-body operators.

%%%%%%%%%%%%%%%%%%%%%%%%%%%%%%%%%%%%%%%%%%%%%%%%%%%%%%%%%%%%%%%%%%%%%%
%%%%%%%%%%%%%%%%%%%%%%%%%%%%%%%%%%%%%%%%%%%%%%%%%%%%%%%%%%%%%%%%%%%%%%
\paragraph{Unified measure for quantum-thermal equilibrium.---}
While several measures and criteria of quantum-thermal equilibrium have been introduced in the literature~\cite{Goldstein2015, Goldstein2017, tasaki2016typicality, mori2018thermalization}, 
they are specifically formulated for particular classes of operators (such as subsystem observables and macroscopic observables) and not applicable to generic few-body operators.
% they cannot be applied to generic few-body operators because they are defined for specific choices of operators, such as those acting only on a subsystem.

To overcome this limitation and to quantitatively study how the ETH depends on physical constraints on observables, 
we introduce the following measure of the closeness to a thermal ensemble, which is applicable to an arbitrary set $\opSet$ of observables.
\begin{definition}[Unified measure of quantum-thermal equilibrium]
    To quantify the distance $\norm*{ \hat{\sigma} - \hat{\rho}_{\mathrm{th}} }$ between a quantum state $\hat{\sigma}$ and a thermal state $\hat{\rho}_{\mathrm{th}}$, we introduce the following (semi-)norm,
    \begin{equation}
        \seminorm*{p}{ \hat{X} }
        \coloneqq \sup_{ \substack{ \hat{A} \in \opSet + \mathbb{R} \hat{I} \\ \hat{A} \neq 0}} 
        \abs{ \tr\qty( \frac{\hat{A}^{\dagger}}{ \opRange[q]{\hat{A}} } \,\hat{X} )  }, 
        \label{def_pseudoDistance1}
    \end{equation}
    where $p^{-1} +q^{-1} = 1$, $\mathbb{R}$ denotes the field of real numbers, and $\hat{I}$ is the identity operator.
    Here, $\norm*{ \hat{A} }_{q} \coloneqq (\tr[(\hat{A}^{\dagger}\hat{A})^{q/2})])^{1/q}$ is the Schatten-$q$ norm~\cite{bhatia2013matrix}.
    The operator $\hat{X}$ is arbitrary and not necessarily Hermitian.
    Then, the (pseudo-)distance $\seminorm{1}{ \hat{\sigma} - \hat{\rho}_{\mathrm{th}} }$ with $\hat{\rho}_{\mathrm{th}}$ being a thermal ensemble serves as a unified measure of quantum-thermal equilibrium.
\end{definition}
In the following, we only use the cases $(p,q) = (1,\infty)$ and $(2,2)$ due to the following two reasons:
First, among $\seminorm{p}{\cdot}$ with $p\geq 1$, only $\seminorm{1}{\cdot}$ can be used to define a measure of quantum-thermal equilibrium, as it satisfies the following essential properties~\cite{sugimoto2021test}:
(i) invariant under a linear transformation$\colon \hat{A} \mapsto a' \hat{A} +b'$,
(ii) dimensionless, and
(iii) thermodynamically intensive, meaning that $\expval*{\hat{A}}^{(\mathrm{mc})}(E) / \norm*{\hat{A}}_{q}$ converges to a finite value in the thermodynamic limit.
Here, $\expval*{\hat{A}}^{(\mathrm{mc})}(E)$ denotes the microcanonical average at energy $E$.
Indeed, for an additive operator $\hat{A}$, we have $\norm*{\hat{A}}_{\infty} \sim \particleNumber$~\cite{Tasaki2018-bz}, ensuring the property (iii) for the 1-norm $\seminorm{1}{\cdot}$.
In contrast, we have $\norm*{\hat{A}}_{q} \sim D^{1/q} \sqrt{N}$ for any $q \in [1,\infty)$, where $D$ is the Hilbert-space dimension~(see Methods).
%  
Second, for general $\opSet$, it is difficult or even impossible to calculate the supremum in $\seminorm{1}{\cdot}$, both numerically and analytically.
However, the 2-norm $\seminorm{2}{\cdot}$ is computable for a given orthonormal basis of $\opSet + \mathbb{R}\hat{I}$~(see \Methods{}) and bounds the 1-norm $\seminorm{1}{\cdot}$ as $\seminorm{2}{\cdot} \leq \seminorm{1}{\cdot} \leq D^{1/2} \seminorm{2}{\cdot}$.

% \seminorm{1} as a unified measure for the proximity to thermal equilibrium
We say that a quantum state $\hat{\sigma}$ is in thermal equilibrium relative to $\opSet$ if $\seminorm{1}{ \hat{\sigma} - \hat{\rho}_{\mathrm{th}} } < \epsilon$ holds for a sufficiently small $\epsilon\, (>0)$.
This definition follows Refs.~\cite{Goldstein2015, Goldstein2017}; however, we here concern only the expectation value of $\hat{A} \in \opSet$ and not the probability distribution over the spectrum of $\hat{A}$.
Nonetheless, our framework can deal with the probability distribution by including sufficiently high powers of $\hat{A}$ in $\opSet$, offering finer control over the precision in observing the distribution of $\hat{A}$.
% 
By appropriately choosing $\opSet$, our notion of quantum-thermal equilibrium unifies previously introduced notions of thermal equilibrium, such as subsystem thermal equilibrium~\cite{popescu2006entanglement, goldstein2006canonical, garrison2018does, dymarsky2018subsystem}, microscopic thermal equilibrium~(MITE), and macroscopic thermal equilibrium~(MATE)~\cite{Goldstein2015, Goldstein2017, mori2018thermalization}~(see \Supple{}~I).


% 
Many of the previous works that numerically tested the ETH with respect to several operators $\hat{A}_{1},\cdots, \hat{A}_{J}$~\cite{rigol2008thermalization, beugeling2014finite, kim2014testing, rigol2009quantum, rigol2009breakdown, rigol2010quantum, biroli2010effect, santos2010localization, steinigeweg2013eigenstate, mondaini2016eigenstate, mondaini2017eigenstate, jansen2019eigenstate, Schonle_PhysRevB2021y} essentially set $\opSet = \Bqty*{ \hat{A}_{1},\cdots, \hat{A}_{J} }$.
However, if one really wants to distinguish an energy eigenstate from $\hat{\rho}_{\mathrm{th}}$ without any exception, one should use not only a limited number of quantities but \textit{all} the quantities compatible with physical constraints under consideration.



\paragraph{Measure of the ETH.---}
As the measure of the ETH, which requires (i) \textit{all} energy eigenstates to be in thermal equilibrium and (ii) \textit{all} off-diagonal elements of an observable in an energy eigenbasis to be vanishingly small, we introduce
\begin{equation*}
    \ETHmeasure[p]{\hat{H}}{\opSet}(E)
    \coloneqq \max_{\ket*{E_{\alpha}}, \ket*{E_{\beta}} \in \mathcal{H}_{E,\Delta E} }
    \seminorm{p}{ \EigStateOp{\alpha\beta} - \MCE[\var{E}](E_{\alpha}) \delta_{\alpha\beta} },
\end{equation*}
where $p=1$ or $2$, $\EigStateOp{\alpha\beta} \coloneqq \dyad*{ E_{\alpha} }{ E_{\beta} }$ with $\ket*{ E_{\alpha} }$ being an energy eigenstate with eigenenergy $E_{\alpha}$, and $\MCE[\var{E}](E_{\alpha})$ is the microcanonical density operator within the energy shell $\mathcal{H}_{E_{\alpha}, \var{E}} \coloneqq \vecspan\Bqty*{ \ket*{E_{\gamma}} \mid \abs*{E_{\gamma} - E_{\alpha}} < \var{E} }$.
Here, the width $\var{E}$ should be chosen sufficiently small as $\var{E} = \order*{N g(N)}$ when testing the ETH at resolution $\ETHmeasure[1]{\hat{H}}{\opSet}(E) = \order{g(N)}$, where $g(N)$ is a function that vanishes for $N\rightarrow\infty$.
% 
As with the norm $\seminorm{p}{\cdot}$, $\ETHmeasure{\hat{H}}{\opSet}$ is a proper measure of the ETH, and $\ETHmeasure[2]{\hat{H}}{\opSet}$ will be used to bound $\ETHmeasure{\hat{H}}{\opSet}$.
% 
We say that the ETH holds for an observable $\hat{A}$ if and only if $\lim_{N \to \infty} \ETHmeasure{\hat{H}}{\Bqty*{\hat{A}}} = 0$. 
Because the leading order in the thermodynamic limit is given by $\norm*{\hat{A}}_{\infty}$, this condition is equivalent to requiring that (i) the expectation values of $\hat{A}$ in $\ket{E_\alpha}$ and $\MCE[\var{E}](E_\alpha)$ agree at the leading order, and (ii) the off-diagonal elements of $\hat{A}$ are negligible at the same order.
% 
This formulation aligns with the standard interpretation in statistical mechanics, whereby thermal ensembles that agree at leading order are regarded as equivalent, even if they differ at subleading orders~\cite{Ruelle1999-sb, Tasaki2018-bz}.
Accordingly, $\ETHmeasure{\hat{H}}{\opSet}$ detects both the validity and breakdown of the ETH in a manner consistent with previous numerical results~(see Methods).
% 
Since $\ETHmeasure{\hat{H}}{\opSet} = \sup_{\hat{A} \in \opSet+\mathbb{R}\hat{I}} \ETHmeasure{\hat{H}}{\Bqty*{\hat{A}}}$, the ETH holds for \textit{all} operators in $\opSet$ if and only if $\lim_{\particleNumber \to \infty} \ETHmeasure{\hat{H}}{\opSet} = 0$.
In particular, any previously reported breakdown of the ETH for specific observables~(e.g., \cite{rigol2009breakdown, santos2010localization, Turner_PhysRevB2018k}) also implies its breakdown under our formalism, since each $\ETHmeasure{\hat{H}}{\Bqty*{\hat{A}}}$ provides a lower bound on $\ETHmeasure{\hat{H}}{\opSet}$.
In this sense, $\ETHmeasure{\hat{H}}{\opSet}$ is the most sensitive ETH measure.
% The ETH holds for \textit{all} operators in $\opSet$ if and only if $\lim_{\particleNumber \to \infty} \ETHmeasure{\hat{H}}{\opSet} = 0$.
% Indeed, if $\ETHmeasure{\hat{H}}{\opSet}$ is sufficiently small, \textit{all} the operators in $\opSet$ give almost the same expectation values for $\ket*{E_{\alpha}}$ and $\MCE[\var{E}](E_{\alpha})$, and their off-diagonal elements within the energy shell $\mathcal{H}_{E_{\alpha}, \var{E}}$ become sufficiently small.
% \red{In other words, if there exists an operator $\hat{A} \in \opSet$ for which either (i) the expectation values of $\hat{A}$ for $\ket*{E_{\alpha}}$ and $\MCE[\var{E}]$ are different at the leading order (i.e. $\norm*{\hat{A}}_{\infty}$), or (ii) some off-diagonal elements of $\hat{A}$ within $\mathcal{H}_{E_{\alpha}, \var{E}}$ remain nonnegligible compared with the leading order (i.e. $\norm*{\hat{A}}_{\infty}$), then $\ETHmeasure{\hat{H}}{\opSet}$ remains finite in the limit $\particleNumber\to\infty$.
% It is immediate that the converse is also true.}

% In that sense, $\ETHmeasure{\hat{H}}{\opSet}$ is the most sensitive ETH measure ever.
% If $\ETHmeasure{\hat{H}}{\opSet}$ remains finite in the limit $\particleNumber\to\infty$, there exists an operator $\hat{A} \in \opSet$ such that the expectation values of $\hat{A}$ for $\ket*{E_{\alpha}}$ and $\MCE[\var{E}]$ are different, or some off-diagonal elements of $\hat{A}$ within $\mathcal{H}_{E_{\alpha}, \var{E}}$ remain nonnegligible. 
% Therefore, the ETH with respect to $\opSet$ breaks down in this case.


%%%%%%%%%%%%%%%%%%%%%%%%%%%%%%%%%%%%%%%%%%%%%%%%%%%%%%%%%%%%%%%%%%%%%%
\paragraph{Distribution of $\ETHmeasure[1]{\hat{H}}{\opSet}(E)$ for systems with Haar-distributed energy eigenstates.---}
Having introduced the measure of the ETH and the bounds on $\seminorm{1}{\cdot}$ in terms of the computable quantity $\seminorm{2}{\cdot}$, we are in a position to discuss the validity of the ETH relative to a set $\opSet$ of test observables.
% 
For systems with Haar-distributed energy eigenstates, we can derive the following theorem.
\begin{theorem}
    \label{th_theorem1}
    Let $\mathcal{G}_{\mathrm{inv}}$ be an invariant random matrix ensemble, the eigenvectors of whose matrices are distributed according to the unitary Haar measure.
    We set $D \coloneqq \dim \mathcal{H}$~(the Hilbert-space dimension) and $M \coloneqq \dim\opSet$.
    Then, for any $D$-independent $\epsilon>0$, we have
    \begin{equation}
        \frac{1}{24} \frac{\sqrt{M}}{D} \leq \ETHmeasure{\hat{H}}{\opSet} + \order{ \frac{ D^{\epsilon} }{ \sqrt{D} } } \leq \sqrt{\frac{M}{D}}
        \label{eq_InequalityForETHmeasure}
    \end{equation}
    and 
    $\ETHmeasure[2]{\hat{H}}{\opSet} = \order{\sqrt{M}/D} + \order{ D^{\epsilon} / \sqrt{D} }$
    for almost all $\hat{H} \in \mathcal{G}_{\mathrm{inv}}$, whose fraction is bounded from below by $1 - \exp\!\qty( -\order{D^{2 \epsilon}} )$.
    When $M \ll D^{2\epsilon}$, the inequality~\eqref{eq_InequalityForETHmeasure} should be interpreted as $\ETHmeasure{\hat{H}}{\opSet} = \order{ D^{\epsilon} / \sqrt{D} }$.
\end{theorem}
\begin{proof}[Outline of the proof]
    The proof goes in three steps. 
    (i)~First, we evaluate the average deviation from thermal equilibrium for individual energy eigenstates.
    For each pair $(\alpha,\beta)$ in the energy window, we quantify the deviation by the $p$-norm as
    \begin{equation*}
        f_{p}^{(\alpha\beta)} \coloneqq \seminorm{p}{ \EigStateOp{\alpha\beta} - \MCE[\var{E}](E_{\alpha}) \delta_{\alpha\beta} } \quad (p = 1,2).
    \end{equation*}
    Calculating the moments of $f_{2}^{(\alpha\beta)}$ over the Haar measure, we obtain $\mathbb{E}[f_{2}^{(\alpha\beta)}] = \order*{\sqrt{M}/D}$.
    % 
    (ii)~Second, we show that these small deviations occur for almost all $\hat{H} \in \mathcal{G}_{\mathrm{inv}}$.
    To this end, we prove that each $f_{p}^{(\alpha\beta)}$ is Lipschitz continuous on $\mathrm{SU}(D)$ with a constant independent of both $D$ and $(\alpha,\beta)$, and apply a concentration inequality for Lipschitz functions~\cite{Ledoux2001,meckes2019random,Mele2024-jg}.
    This step goes beyond Levy's lemma and allows uniform control over all observables in $\opSet$.
    % 
    (iii) Finally, we control the maximal deviation $\max_{\alpha,\beta} f_{p}^{(\alpha\beta)}$ via a union bound over all relevant pairs $(\alpha,\beta)$.
    Here, the strong concentration from Step~(ii) is essential: it implies that even the worst-case deviation should remain small for almost all $\hat{H} \in \mathcal{G}_{\mathrm{inv}}$.
    Combining the resulting bound with the inequality $\seminorm{2}{\cdot} \leq \seminorm{1}{\cdot} \leq \sqrt{D} \seminorm{2}{\cdot}$ yields the desired inequality~\eqref{eq_InequalityForETHmeasure} for almost all $\hat{H} \in \mathcal{G}_{\mathrm{inv}}$.
    % 
    See \Methods{} for the complete proof.
\end{proof}

With Theorem~\ref{th_theorem1}, we can test whether or not the ETH regarding $\opSet$ typically holds in $\mathcal{G}_{\mathrm{inv}}$ by counting the dimension of the operator space $\opSet$.
In the thermodynamic limit, the ETH regarding $\opSet$ typically holds if the upper bound of Eq.~\eqref{eq_InequalityForETHmeasure} vanishes, and an ETH-breaking operator typically exists in $\opSet$ if the lower bound of Eq.~\eqref{eq_InequalityForETHmeasure} converges to a finite value.
We will use this result in the next section to analytically establish Conjecture~\ref{conj_Conjecture} for Haar-random systems.

\begin{figure*}[t]
    \centering
    \includegraphics[width=\linewidth]{Figure2_ShortRangeEnsemble_Measure.pdf}
    \caption{\textbf{Ensemble average of the upper bound of the (diagonal) ETH measure for Hamiltonians with local and few-body interactions.}
    The quantity $\Lambda_{2}^{[0,m]}$ is calculated in the energy window centered at energy $E = (E_{\max} + E_{\min})/2$ with width $2\Delta = 0.05 (E_{\max} - E_{\min})$, where $E_{\mathrm{max/min}}$ is the maximum/minimum of the energy spectrum.
    The width of the microcanonical energy shell is set to $\var{E} = 0.2 (E_{\max} - E_{\min}) / \sysSize$, which is sufficiently small for accessible system sizes~(up to 18 spins for spin systems, and up to 11 particles for Bose and Fermi systems).
    The symbol $\mathbb{E}_{\mathrm{log}}$ denotes the geometric mean.
    % 
    The black solid vertical lines indicate the point $\lowerBound{}$ where the maximum difference between the curves for different system sizes is minimized.
    The blue shaded region indicates the uncertainty in $\lowerBound{}$ due to limited system sizes, characterized by a change in the $\particleNumber$-dependence of the upper bound from monotonic decrease~(left) to increase~(right).
    % 
    The gray dashed curves show the analytic result $\sqrt{M/D}$ for systems with Haar-distributed energy eigenstates. 
    Their intersection defines the gray dashed vertical line at $\lowerBound^{\mathrm{H}}$, which corresponds to the value of $\lowerBound{}$ for Haar-random systems.
    % 
    We consider three ensembles: (a) spin systems consisting of up to 2-local terms, i.e., nearest-neighbor interactions and on-site potentials~(Eq.~\eqref{eq_SpinHamiltonian});
    (b) Bose and (c) Fermi systems consisting of 3-local and 2-body terms, i.e., products of two creation and two annihilation operators acting on three consecutive sites~(Eq.~\eqref{eq_BosonFermionHamiltonian}).
    Different colors show data for different system sizes.
    % 
    There are points where curves for different system sizes almost intersect.
    On the left of the intersections, the upper bound of the ETH measure decreases with increasing system size, indicating that the ETH generically holds in this region.
    These results verify Conjecture~\ref{conj_Conjecture} for generic systems in Eqs.~\eqref{eq_SpinHamiltonian} and \eqref{eq_BosonFermionHamiltonian} by providing a nontrivial lower bound $\lowerBound{} > 0$ on the threshold $\alpha_{\ast}$.
    }
    \label{fig_Numerics_ShortRangeEnsemble}
\end{figure*}

%%%%%%%%%%%%%%%%%%%%%%%%%%%%%%%%%%%%%%%%%%%%%%%%%%%%%%%%%%%%%%%%%%%%%%
\paragraph{Rigorous upper and lower bounds for the ETH for systems with Haar-distributed energy eigenstates.---}
We apply Theorem~\ref{th_theorem1} to test the ETH for few- and many-body observables.
For $\particleNumber$-site spin-$S$ systems, we define the $m$-body operator space $\opSet_{\particleNumber}^{[0, m]}$ as the space of operators that can be expressed as a linear combination of operators acting nontrivially on at most $m$ spins~\cite{hamazaki2018atypicality}.
That is, we define $\opSet_{\particleNumber}^{[0, m]} \coloneqq \bigoplus_{m'=0}^{m} \opSet_{\particleNumber}^{(m')}$ with
\begin{equation*}
    \opSet_{\particleNumber}^{(m')}
    \coloneqq \vecspan\Bqty{ \hat{\sigma}_{x_{1}}^{(p_{1})} \cdots \hat{\sigma}_{x_{m'}}^{(p_{m'})} \mid \substack{ 1 \leq x_{1} < \cdots < x_{m'} \leq \sysSize \\ 0 < p_{j} < \dimLoc^2 } },
\end{equation*}
where $\Bqty*{ \hat{\sigma}_{x}^{(p)} }_{p=0}^{\dimLoc^2-1}$ is an orthonormal basis of the local operator space at site $x$ with $\hat{\sigma}_{x}^{(0)} = \hat{I}/\sqrt{\dimLoc}$ ($\hat{I}$ denotes the identity operator).
Here, $\opSet_{\particleNumber}^{(0)}$ should be understood as $\opSet_{\particleNumber}^{(0)} = \vecspan\!\Bqty*{\hat{I}}$.
% 
For Bose and Fermi systems, we define $\opSet_{\particleNumber}^{[0, m]}$ to be the space of operators that can be written as a linear combination of products of $m$ annihilation operators and $m$ creation operators.
That is, we define $\opSet_{\particleNumber}^{[0, m]} \coloneqq \tilde{\opSet}_{\particleNumber}^{[0, m]} \cap \mathcal{L}(\mathcal{H})$ with 
\begin{equation*}
    \tilde{\opSet}_{\particleNumber}^{[0, m]} 
    \coloneqq \vecspan_{\mathbb{C}}\Bqty{ \hat{a}_{x_{1}}^{\dagger} \cdots \hat{a}_{x_{m}}^{\dagger} \hat{a}_{y_{1}} \cdots \hat{a}_{y_{m}} \mid \substack{ 1 \leq x_{j} \leq V \\ 1 \leq y_{j} \leq V } },
\end{equation*}
where $\hat{a}_{x}$ denotes the bosonic/fermionic annihilation operator at site $x$.
With these definitions, we have $\opSet_{\particleNumber}^{[0, m-1]} \subseteq \opSet_{\particleNumber}^{[0, m]}$ for any $m = 1,\cdots, \particleNumber$, which justifies the superscript $[0, m]$.

By counting the dimension of $\opSet_{\particleNumber}^{[0, m]}$ and that of the total Hilbert space, we can apply Theorem~\ref{th_theorem1} to obtain the following theorem.
\begin{theorem}[Bounds on the ETH for $\hat{H} \in \mathcal{G}_{\mathrm{inv}}$] \label{thm_MainTheorem}
    There exists a nontrivial positive number $\alpha > 0$ such that the ETH with respect to $\opSet_{\particleNumber}^{[0,\alpha \particleNumber]}$ typically holds for Hamiltonians in $\mathcal{G}_{\mathrm{inv}}$ in the thermodynamic limit.
    Let $\alpha_{\ast}$ be the largest such constant.
    Then, there exists a lower bound $\lowerBound^{\mathrm{H}} \in (0,1/2]$ on $\alpha_{\ast}$ (shown in Fig.~\ref{fig_RegionsOfM}), such that the ETH measure for $\opSet_{\particleNumber}^{[0,\alpha \particleNumber]}$ decays exponentially with system size when $\alpha < \lowerBound^{\mathrm{H}}$.
\end{theorem}
\begin{proof}[Outline of the proof]
    We define $\lowerBound^{\mathrm{H}}$ as the largest value such that $\dim \opSet_{\particleNumber}^{[0, \alpha \particleNumber]} / D \to 0$ in the thermodynamic limit.
    Then, Theorem~\ref{th_theorem1} gives $\lowerBound^{\mathrm{H}} \leq \alpha_{\ast}$.
    Direct calculations for each of spin, Bose, and Fermi systems show $\lowerBound^{\mathrm{H}} \in (0,1/2]$ and the exponential decay of the ETH measure when $m/\particleNumber < \lowerBound^{\mathrm{H}}$~(see \Methods{} for details).
\end{proof}
Theorem~\ref{thm_MainTheorem} establishes Conjecture~\ref{conj_Conjecture} for typical systems with Haar-distributed energy eigenstates by providing a nontrivial lower bound $\lowerBound^{\mathrm{H}}$ on $\alpha_\ast$.
Theorem~\ref{th_theorem1} also yields an upper bound $\upperBound^{\mathrm{H}} \geq \alpha_\ast$, defined as the smallest value for which $\dim \opSet_{\particleNumber}^{[0, \alpha \particleNumber]} / D^2$ converges to a positive value in the thermodynamic limit.
(Here, $\lowerBound^{\mathrm{H}} \leq \upperBound^{\mathrm{H}}$ by definition.)
For spin systems and Fermi systems above half-filling, 
this upper bound $\upperBound^{\mathrm{H}}$ takes a nontrivial value less than $1$.
By contrast, in Bose systems and Fermi systems below half-filling, it saturates at the trivial value $\upperBound^{\mathrm{H}} = 1$ due to the rapid growth of the dimension of the $m$-body operator space~(see Fig.~\ref{fig_RegionsOfM}).

The existence of a nontrivial threshold $\alpha_{\ast}$ implies that the ETH holds true even if we observe  $\order{\particleNumber}$-body operators in systems with Haar-distributed energy eigenstates.
Because our result is not restricted to the operators acting only on subsystems, we conclude that the decomposition of the total system into a subsystem and the rest is not essential for the ETH to hold.
For the same reason, our result directly applies to any operators acting on the whole system, which cannot exactly be dealt with in previous works~\cite{mori2018thermalization, garrison2018does}. 
These operators include extensive sums of local operators (e.g., total magnetization), few-body operators (e.g., momentum distributions), and low-order powers of these quantities.
Since the central moments of a few-body operator $\hat{A}$ are polynomials of $\hat{A}$ and their values scale polynomially in $\particleNumber$, the exponential decay of the ETH measure implies that \textit{the ETH typically holds in $\mathcal{G}_{\mathrm{inv}}$ including any low-order fluctuations of arbitrary few-body operators}.

\begin{figure*}[t]
    % \vspace{-1cm}
    \centering
    \includegraphics[width=\linewidth]{Figure3_ShortRangeEnsemble_Bounds_EnergyDependence.pdf}
    \caption{\textbf{Energy dependence of the bounds $\lowerBound$ and $\upperBound$ on the threshold $\alpha_{\ast}$.
    }
    The data points indicate the value of $m/\particleNumber$ where the maximum difference between the curves with different system sizes for a fixed normalized energy $x \coloneqq (E-E_{\min})/(E_{\max} - E_{\min})$, as shown in Fig.~\ref{fig_Numerics_ShortRangeEnsemble}, becomes minimal.
    % 
    The upper/lower end of the error bars shows the value of $\alpha$ above/below which $\mathbb{E}_{\mathrm{log}}[\sqrt{D} \Lambda_{2}^{[0,\alpha \particleNumber]}]$ (or $\mathbb{E}_{\mathrm{log}}[\Lambda_{2}^{[0,\alpha \particleNumber]}]$) monotonically increases/decreases as the system size increases.
    % 
    The system sizes used in this figure are $12\:(12) \leq \particleNumber \leq 18\:(14)$ for $\lowerBound{}\:(\upperBound{})$ of spin systems and $8\:(6) \leq \particleNumber \leq 11\:(9)$ for Bose and Fermi systems.
    % 
    The existence of the nontrivial lower bound $\lowerBound{}\, (>0)$ across approximately the central 40\% of the spectrum supports the validity of Conjecture~\ref{conj_Conjecture} in this energy region.
    }
    \label{fig_BoundsEnergyDependence}
\end{figure*}
%%%%%%%%%%%%%%%%%%%%%%%%%%%%%%%%%%%%%%%%%%%%%%%%%%%%%%%%%%%%%%%%%%%%%%
\paragraph{Numerical tests for locally interacting systems.---}
Theorems~\ref{th_theorem1} and \ref{thm_MainTheorem} are rigorous results that hold for Hamiltonians with Haar-distributed eigenstates.
Ensembles of such Hamiltonians have been used to model chaotic quantum systems.
In particular, the typicality of the ETH regarding several observables has been proved for (nonlocal) Hamiltonians with~\cite{von2010proof, reimann2015generalization, mori2018thermalization} and without~\cite{cipolloni2021eigenstate, Adhikari2024-vj, Erdos2024-zr} Haar-distributed eigenstates.
Theorem~\ref{th_theorem1} generalizes these studies by considering arbitrary sets of test observables, and Theorem~\ref{thm_MainTheorem} illustrates a benefit of such a generalization.
However, Hamiltonians with Haar-distributed eigenstates typically involve highly nonlocal and $\order{N}$-body terms, and thus can lead to predictions different from systems with local and few-body interactions.
Indeed, various studies have reported deviations from chaotic quantum systems with such local interactions~\cite{atas2012multifractality, huang2019universal, backer2019multifractal, pausch2021chaos, huang2021universal, pausch2021chaos2, Haque2022-tc, pausch2022optimal, Huang_IEEEJournalOnSelectedAreasInInformationTheory2024k, hamazaki2018atypicality, sugimoto2021test}.
In addition, a typical unitary sampled from the Haar measure transforms local and few-body operators into highly nonlocal and $\order{N}$-body ones.
Consequently, properties of \textit{individual} test operators, such as locality or few-body nature, do not play a fundamental role in Theorem~\ref{th_theorem1}, where only the dimension of the operator space appears in inequality~\eqref{eq_InequalityForETHmeasure}.

A highly nontrivial aspect of chaotic quantum systems is that some properties derived for such highly non-local and $\order{N}$-body random matrix models remain qualitatively valid even for systems with local and few-body interactions.
Examples include the ETH (as well as Srednicki’s ETH ansatz)~\cite{rigol2008thermalization, beugeling2014finite, kim2014testing, khaymovich2019eigenstate, sugimoto2021test, Sugimoto2022}, Wigner--Dyson level-spacing distributions~\cite{Kolovsky2004-oo, Santos2010-qs}, level-spacing ratios~\cite{Atas_PhysRevLett2013i, Russomanno_PhysRevB2021y}, and spectral form factors~\cite{Kos_PhysicalReviewX2018u, Dong2025-ab}.
Clarifying whether this qualitative validity extends to our conjecture motivated by the Haar-measure analysis requires explicit numerical verification in realistic models.
Thus, we numerically test Conjecture~\ref{conj_Conjecture} for Hamiltonians that explicitly feature the locality and few-body nature of interactions.

We introduce ensembles of Hamiltonians with local and few-body interactions for spin, Bose, and Fermi systems.
For spin systems, we consider the ensemble of Hamiltonians consisting of 2-local terms, i.e., those given by
\begin{equation}
    \hat{H} \coloneqq \sum_{\alpha,\beta = 0}^{3} J_{\alpha\beta} \qty( \sum_{j=1}^{\sysSize} \hat{\sigma}_{j}^{(\alpha)} \hat{\sigma}_{j+1}^{(\beta)} ),
    \label{eq_SpinHamiltonian}
\end{equation}
where $\hat{\sigma}^{(\alpha)}$ ($\alpha = 1, 2, 3$) are the Pauli operators, $\hat{\sigma}^{(0)} \coloneqq \hat{I}/\sqrt{2}$ with $\hat{I}$ being the identity operator, and the coefficients $\Bqty*{ J_{\alpha\beta} }$ are i.i.d. Gaussian random variables with zero mean and unit variance.
Since the Hamiltonian~\eqref{eq_SpinHamiltonian} has translation symmetry, we restrict ourselves to the zero-momentum sector in numerical calculations.
% 
For Bose and Fermi systems, we consider the ensemble of Hamiltonians consisting of 3-local and 2-body terms, i.e., those given by $\hat{H} \coloneqq (\tilde{H} + \tilde{H}^{\dagger} + \hat{P} \tilde{H} \hat{P} + \hat{P} \tilde{H}^{\dagger} \hat{P} ) / 4$, where $\hat{P}$ is the reflection operator defined by $\hat{P} \hat{a}_{j} \hat{P} = \hat{a}_{\sysSize - j +1}$ with $\hat{a}_{j}$ being bosonic or fermionic annihilation operator acting on site $j$, and
\begin{equation}
    \tilde{H} \coloneqq \sum_{x_{1},x_{2},y_{1},y_{2} = 1}^{3} J_{x_{1}x_{2}}^{y_{1}y_{2}} \qty( \sum_{j=1}^{\sysSize} \hat{a}_{j+x_{1}}^{\dagger} \hat{a}_{j+x_{2}}^{\dagger} \hat{a}_{j+y_{1}} \hat{a}_{j+y_{2}} ).
    \label{eq_BosonFermionHamiltonian}
\end{equation}
Here, the coefficients $\Bqty*{ J_{x_{1}x_{2}}^{y_{1}y_{2}} }$ are i.i.d. complex Gaussian random variables satisfying $\mathbb{E}J_{x_{1}x_{2}}^{y_{1}y_{2}} = 0$, $\mathbb{E}\abs*{J_{x_{1}x_{2}}^{y_{1}y_{2}}}^2 = 1$, and $\mathbb{E}(J_{x_{1}x_{2}}^{y_{1}y_{2}})^2 = 0$.
Since the Hamiltonian $\hat{H}$ has translation and reflection symmetries, we restrict ourselves to the zero-momentum and even-parity sector in numerical calculations.
The filling factor $(n \coloneqq \particleNumber/\sysSize)$ is set to $n=1$ for Bose systems and $n=1/2$ for Fermi systems.

For these ensembles, we numerically evaluate the quantity $\Lambda_{2}^{[0,m]}$, which bounds the ETH measure from above and below.
(See \Methods{} for details on the numerical calculations.)
The width $\var{E}$ of the microcanonical shell is set to $\var{E} = 0.2(E_{\max} - E_{\min})/\sysSize$, which is sufficiently small to suppress contributions from energy dependence for the system sizes considered.
Figure~\ref{fig_Numerics_ShortRangeEnsemble} shows the averaged upper bound $\mathbb{E}_{\log}[\sqrt{D} \Lambda_{2}^{[0,m]}]$ of the (diagonal) ETH measure in the middle of the spectrum as a function of the ratio $m/\particleNumber$ for each ensemble.
Here, $\mathbb{E}_{\log}[X] \coloneqq \exp\!\qty( \frac{1}{k} \sum_{j=1}^{k} \log X_j)$ denotes the geometric mean.
We observe that the curves for different system sizes intersect approximately at a particular value, denoted by $\lowerBound$.
To the left of this crossing point~($m/\particleNumber \lesssim \lowerBound{}$), the averaged upper bound decreases with increasing system size, while it increases to the right ($m/\particleNumber \gtrsim \lowerBound{}$).
This $\particleNumber$-dependence qualitatively parallels that of the estimate $\sqrt{D} \Lambda_{2}^{[0,m]} \simeq \sqrt{M/D}$ for Haar-random systems, shown in the figure as dashed curves.
This qualitative agreement provides evidence that the ETH holds for all $m$-body operators with $m/\particleNumber < \lowerBound{}$, yielding a concrete lower bound $\lowerBound{} \leq \alpha_{\ast}$ for the investigated systems and thereby supporting Conjecture~\ref{conj_Conjecture}.
It is noteworthy that such consistency is observed for systems with local and few-body interactions, even though Conjecture~\ref{conj_Conjecture} can be rigorously proved only for systems with Haar-distributed energy eigenstates (Theorem~\ref{thm_MainTheorem}), whose Hamiltonians typically involve highly nonlocal and $\order{\particleNumber}$-body terms.

Moreover, we extend our numerical analysis to varying normalized energies $x = (E - E_{\min})/(E_{\max} - E_{\min})$ and find that our conjecture holds true even for energy densities corresponding to finite temperatures.
That is, for values not far away from the center of the spectrum ($x \approx 0.5$), we observe behavior similar to that in Fig.~\ref{fig_Numerics_ShortRangeEnsemble}: 
the curves of the upper bound $\sqrt{D} \Lambda_{2}^{[0,m]}$ on the (diagonal) ETH measure for different system sizes intersect almost at a single value of $m/\particleNumber$, denoted by $\lowerBound(x)$ (see \Supple{}~III).
% 
As shown in Fig.~\ref{fig_BoundsEnergyDependence}, this intersection point $\lowerBound(x)$ slightly decreases as the normalized energy deviates from the spectral center.
For small values of $x \lesssim 0.25$, the intersection is no longer observed, leaving open the possibility that $\alpha_{\ast} = 0$ in this low-energy region.
% 
A similar crossing behavior is also observed for the lower bound $\Lambda_{2}^{[0,m]}$ of the ETH measure, with the intersection point denoted by $\upperBound(x)$. 
As in the case for $\lowerBound(x)$, $\upperBound(x)$ decreases as the normalized energy decreases (see Fig.~\ref{fig_BoundsEnergyDependence}).
% 
Note that the intersection points $\lowerBound(x)$ (or $\upperBound(x)$) are numerically identified as the value of $\alpha \coloneqq m/\particleNumber$ that minimizes the maximum difference between the bounds on the ETH measure, $\mathbb{E}_{\log}[\sqrt{D} \Lambda_{2}^{[0,\alpha\particleNumber]}]$ (or $\mathbb{E}_{\log}[\Lambda_{2}^{[0,\alpha\particleNumber]}]$), for different system sizes.

We also test Conjecture~\ref{conj_Conjecture} for concrete nonintegrable systems where the ETH is numerically verified to hold, namely, the Ising model with transverse and longitudinal fields~\cite{kim2014testing}, the Bose-Hubbard model at unit filling~\cite{biroli2010effect}, and spinless fermions with next-nearest-neighbor terms~\cite{rigol2009quantum} at half filling.
However, Conjecture~\ref{conj_Conjecture} is neither validated nor invalidated for these concrete models.
For these models, finite-size effects are so significant that we cannot safely determine whether the upper bound of the ETH measure decreases for large system sizes even for $m=1$~(see \Supple{}~IV).
This indicates that the inequality $\seminorm{1}{\cdot} \leq \sqrt{D} \seminorm{2}{\cdot}$ is too loose to enable a conclusive test of Conjecture~\ref{conj_Conjecture} for those concrete physical models.
It remains to be an important future task to validate (or invalidate) Conjecture~\ref{conj_Conjecture} for prototypical nonintegrable systems, e.g., by finding a better upper bound on the ETH measure.

A complementary theoretical expectation also points to the need for an improved lower bound.
Given a plausible argument based on comparing the reduced density operators of energy eigenstates and thermal ensembles on subsystems~\cite{garrison2018does}, we believe that the ETH typically breaks down for $\opSet_{\particleNumber}^{[0,\particleNumber/2]}$, i.e., we expect $\alpha_{\ast} \leq 1/2$ both for realistic systems and systems with Haar-distributed energy eigenstates.
For high-density Fermi systems with filling $n > 2/3$, our results on the Haar-random systems~(Fig.~\ref{fig_RegionsOfM}(c)) yield a tighter bound $\alpha_{\ast} \leq \upperBound^{\mathrm{H}} = (1-n)/n < 1/2$.
However, for spin and Bose systems, our analytical approach for Haar-random systems does not yield a bound of the form $\upperBound^{\mathrm{H}} < 0.5$.
Moreover, the numerical results for systems with local and few-body interactions~(Fig.~\ref{fig_BoundsEnergyDependence}) also do not show $\upperBound{} < 0.5$.
% 
This highlights an analytically challenging yet meaningful direction for future research---developing a tighter lower bound on the ETH measure by explicitly incorporating the locality and the few-body nature of observables.

Beyond quantum thermalization, the (semi-)norm $\seminorm{1}{\cdot}$ introduced in Eq.~\eqref{def_pseudoDistance1} serves more generally as a measure of closeness between quantum states relative to the operator set $\opSet$.
Consequently, this measure has potential utility in broader contexts beyond the ETH, such as comparing a time-evolving state with its steady state or assessing the operational accuracy of state approximations.
Moreover, it can provide a framework for constructing the space of macroscopic states from that of quantum states by identifying quantum states close to each other under the norm $\seminorm{1}{\cdot}$.
Rigorously formulating the correspondence between microscopic and macroscopic states along this line and deriving the macroscopic dynamics from the microscopic one represent important future problems.


%%%%%%%%%%%%%%%%%%%%%%%%%%%%%%%%%%%%%%%%%%%%%%%%%%
%%%%%%%%%%%%%%%%%%%%%%%%%%%%%%%%%%%%%%%%%%%%%%%%%%%%%%%%%%%%%%
\section{Data availability}
The numerical data plotted in the figures of this Article are available from the authors upon reasonable request.

%%%%%%%%%%%%%%%%%%%%%%%%%%%%%%%%%%%%%%%%%%%%%%%%%%%%%%%%%%%%%%
\section{Code availability}
The source code used in the numerical calculations is available at Zenodo:~\url{https://doi.org/10.5281/zenodo.14709223}.

%%%%%%%%%%%%%%%%%%%%%%%%%%%%%%%%%%%%%%%%%%%%%%%%%%
\bibliography{reference}


%%%%%%%%%%%%%%%%%%%%%%%%%%%%%%%%%%%%%%%%%%%%%%%%%%
%%%%%%%%%%%%%%%%%%%%%%%%%%%%%%%%%%%%%%%%%%%%%%%%%%
\clearpage
\section{Methods}
\paragraph{Scaling of the normalization constant $\opRange[q]{ \hat{A} }$.---}
Here, we demonstrate that the quantity $\hat{A} / \norm*{ \hat{A} }_{q}$ in the definition of $\seminorm{1}{\cdot}$ in Eq.~\eqref{def_pseudoDistance1} becomes thermodynamically intensive only when $q=\infty$.
% 
To illustrate this, we analyze the $\particleNumber$-dependence of the normalization constant $\opRange[q]{ \hat{A} }$ for an extensive operator $\hat{M}_{z} \coloneqq \sum_{j=1}^{\particleNumber} \hat{\sigma}^{(z)}_{j}$, where $\hat{\sigma}^{(z)}$ is the Pauli $z$-operator.
Let $\ket{\uparrow}$ and $\ket{\downarrow}$ denote its eigenstates with eigenvalues $+1$ and $-1$, respectively.
The eigenstates of $\hat{M}_{z}$ are tensor products of $\ket{\uparrow}$ and $\ket{\downarrow}$, and their eigenvalues are $m_{j} \coloneqq -\particleNumber+2j \ (j=0,\cdots,\particleNumber)$, where $j$ is the number of up spins.
% 
Thus, we have $\norm*{\hat{M}}_{\infty} = N$, and $\hat{M}_{z} / \norm*{\hat{M}_{z}}_{\infty}$ is intensive by construction.
On the other hand, for any finite $q\, (\geq 1)$, we have
\begin{align}
    \opRange[q]{ \hat{M}_{z} }^{q} &= 
    \sum_{j=0}^{\particleNumber} \binom{\particleNumber}{j} \abs*{-\particleNumber+2j}^{q}.
\end{align}
% 
This sum is equal to the $q$th absolute moment of the sum of $\particleNumber$ independent Rademacher random variables $\Bqty*{ \xi_{j} }_{j=1}^{\particleNumber}$, where $\Prob[\xi_{j} = \pm 1] = 1/2$:
\begin{equation}
    \opRange[q]{ \hat{M}_{z} }^{q} = 2^{\particleNumber} \mathbb{E}[\abs*{S_{\particleNumber}}^q]\qq{where} 
    S_{N} \coloneqq \sum_{k=1}^{\particleNumber} \xi_{k}.
\end{equation}
Indeed, if exactly $j$ of the $\Bqty{\xi_{k}}$ equal $+1$, then $S_{\particleNumber}=-\particleNumber+2j$, which occurs with probability $2^{-\particleNumber} \binom{\particleNumber}{j}$.
The mean and variance of $S_{\particleNumber}$ are $\mathbb{E}[S_{\particleNumber}] = 0$ and $\mathbb{V}[S_{\particleNumber}]=\particleNumber$, respectively.
Thus, the central limit theorem implies that the normalized variable $Z_{\particleNumber} \coloneqq S_{\particleNumber} / \sqrt{\particleNumber}$ converges in distribution to a standard normal variable.
As a result, we obtain
\begin{equation}
    \opRange[q]{ \hat{M}_{z} }^{q} = 2^{\particleNumber} \particleNumber^{\frac{q}{2}}\, \bqty{ \frac{ 2^{q/2} }{\sqrt{\pi}} \Gamma\qty(\frac{q+1}{2}) +\order{\particleNumber^{-1}} }.
\end{equation}
Thus, for any fixed $q$ and sufficiently large $\particleNumber$, the normalization constant $\opRange[q]{ \hat{M}_{z} }$ scales as $\sim 2^{\particleNumber / q} \sqrt{\particleNumber}$.
Consequently, $\hat{M}_{z}/\norm*{ \hat{M}_{z} }_{q}$ becomes thermodynamically intensive only when $q=\infty$.

\bigskip
\paragraph{Validity of the ETH measure $\ETHmeasure{\hat{H}}{\Bqty*{\hat{A}}}$.---}
To explicitly demonstrate the validity of our ETH measure $\ETHmeasure{\hat{H}}{\opSet}$ for the case of a single observable, we compute $\ETHmeasure{\hat{H}}{\Bqty*{\hat{A}}}$ for representative one- and two-body observables in the XYZ spin chain whose Hamiltonian is given by
\begin{equation}
    \hat{H} = \sum_{j=1}^{\sysSize} \sum_{\alpha=1}^{3} J_{\alpha} \hat{\sigma}_{j}^{(\alpha)} \hat{\sigma}_{j+1}^{(\alpha)} + \sum_{j=1}^{\sysSize} (h + \var{h}_{j}) \hat{\sigma}_{j}^{(z)},
\end{equation}
where $\hat{\sigma}^{(\alpha)}$ ($\alpha = 1,2,3$) are the Pauli operators, $\var{h}_{j}$ are i.i.d.\ random variables which are uniformly sampled from $[-W, W]$, with $W$ controlling the disorder strength, and $(J_{x}, J_{y}, J_{z}) = (1, 0.5, 1.5)$.
We consider three regimes: (a) nonintegrable ($h = 1$, $W = 0$, periodic boundary), (b) integrable ($h = 0$, $W = 0$, periodic boundary), and (c) many-body localized ($h = 1$, $W = 8$, open boundary).
For calculations under periodic boundary conditions, we restrict ourselves to the zero-momentum and even-parity sector.
The test observables are chosen as $\hat{M}_{z} \coloneqq \sum_{j=1}^{\sysSize} \hat{\sigma}_{j}^{(3)}$, and $C_{xx}$, $C_{yy}$, and $C_{zz}$, each defined as $\sum_{j=1}^{\sysSize} \hat{\sigma}_{j}^{(\alpha)} \hat{\sigma}_{j+1}^{(\alpha)}$ for $\alpha = 1,2,3$.
As shown in Fig.~\ref{fig_XYZ_ETHmeasure_vsN}, $\ETHmeasure{\hat{H}}{\Bqty*{\hat{A}}}$ decays exponentially in (a) and saturates in (b) and (c), consistent with known ETH behavior.
These results confirm both the validity and the sensitivity of our ETH measure for the case of a single observable.
Furthermore, the results in Fig.~\ref{fig_XYZ_ETHmeasure_vsN} provide a valid lower bound on the ETH measure for any operator set $\opSet$ that includes the tested operators, since $\ETHmeasure{\hat{H}}{\opSet} = \sup_{\hat{A} \in \opSet} \ETHmeasure{\hat{H}}{\{\hat{A}\}}$.
In particular, Fig.~\ref{fig_XYZ_ETHmeasure_vsN} offers direct evidence for the breakdown of the ETH with respect to such operator sets in both integrable and many-body localized systems.
\begin{figure}
    \centering
    \includegraphics[width=\linewidth]{Figure4_XYZ_ETHmeasure_vsN.pdf}
    \caption{\textbf{ETH measure for representative observables in the XYZ model.}
        The ETH measure is plotted against the system size in the
        (a) nonintegrable regime (periodic boundary, $h = 1$), 
        (b) integrable regime (periodic boundary, $h = 0$), and 
        (c) many-body localized regime (open boundary, $h = 1$, $W = 8$).
        %
        For panel~(b), the data for $\hat{M}_{z}$ are shown only at odd system sizes, because its expectation value vanishes at even sizes, where the energy eigenstates are eigenstates of the $\pi$-rotation operator around the $x$ axis.
        % 
        The exponential decay in (a) and the saturation in (b) and (c) demonstrate that our ETH measure correctly detects both the validity and breakdown of the ETH.
    }
    \label{fig_XYZ_ETHmeasure_vsN}
\end{figure}


\bigskip
\paragraph{Computable formula for the 2-norm $\seminorm*{2}{\hat{X}}$.---}
We derive a formula for the 2-norm $\seminorm*{2}{\hat{X}}$, which is computable, given an orthonormal basis $\Bqty*{ \hat{\Lambda}_{\mu} }_{\mu=1}^{M}$ of $\opSet + \mathbb{R} \hat{I}$ with $M \coloneqq \dim(\opSet + \mathbb{R} \hat{I})$.
For an arbitrary operator $\hat{X}$, we introduce $\vec{X} = (X_{1}, X_{2}, \cdots)$ with $X_{\mu} \coloneqq \tr(\hat{\Lambda}_{\mu}^{\dagger} \hat{X})$.
To calculate the supremum in Eq.~\eqref{def_pseudoDistance1}, we use the expansion of $\hat{A} \in \opSet + \mathbb{R} \hat{I}$ as $\hat{A} = \sum_{\mu} c_{\mu} \hat{\Lambda}_{\mu}$, where $c_{\mu} = \tr(\hat{\Lambda}_{\mu}^{\dagger} \hat{A}) \in \mathbb{R}$.
Substituting this expansion in Eq.~\eqref{def_pseudoDistance1} gives
\begin{align}
    \seminorm*{2}{\hat{X}}
    &= \sup_{ \substack{ \hat{A} \in \opSet + \mathbb{R} \hat{I} \\ \hat{A} \neq 0}} 
        \abs{ \tr\qty( \frac{\hat{A}^{\dagger}}{ \opRange[2]{\hat{A}} } \,\hat{X} )  }
    \nonumber \\
    % 
    &= \sup_{ \substack{ \vec{c} \in \mathbb{R}^{M} \colon \vec{c} \neq 0}} 
        \abs{ \sum_{\mu=1}^{M} \frac{ c_{\mu} }{ \norm*{\vec{c}}_{2} } \tr\qty( \hat{\Lambda}_{\mu}^{\dagger} \,\hat{X} )  }
    \nonumber \\
    % 
    &= \sup_{ \substack{ \vec{c} \in \mathbb{R}^{M} \colon \norm*{\vec{c}}_{2} = 1}} 
        \abs{ \vec{c} \cdot \qty( \Re\vec{X} + i \Im\vec{X} )  }
    \nonumber \\
    % 
    &= \sup_{ \substack{ \vec{c} \in \mathbb{R}^{M} \colon \norm*{\vec{c}}_{2} = 1}} 
        \sqrt{ \vec{c}^{T} \mathcal{Q} \vec{c} },
\end{align}
where $\mathcal{Q} \coloneqq (\Re\vec{X})(\Re\vec{X})^{T} + (\Im\vec{X})(\Im\vec{X})^{T}$ is a symmetric matrix.
Since the rank of $\mathcal{Q}$ is at most 2, it is straightforward to obtain the eigenvalues of $\mathcal{Q}$, and we obtain the following formula:
\begin{equation}
    \seminorm*{2}{\hat{X}} = \sqrt{ \frac{ \norm*{ \vec{X} }_{2}^{2} + \abs*{ \vec{X}^{T} \cdot \vec{X} } }{2} }.
    \label{eq_ComputableFormulaFor2Norm}
\end{equation}


\bigskip
\paragraph{Proof of Theorem~\ref{th_theorem1}.---}
As explained in the main text, the proof of Theorem~\ref{th_theorem1} consists of the three steps.
Before explaining each step, we introduce some notations.

Let $\Bqty*{ \ket*{\alpha} }_{\alpha=1}^{D}$ be an arbitrarily fixed orthonormal basis of $\mathcal{H}$.
To see the $p$-norm $f_{p}^{(\alpha\beta)} \coloneqq \seminorm*{p}{ \hat{\rho}_{\alpha\beta} - \MCE[\var{E}](E_{\alpha}) \delta_{\alpha\beta} }$ \ ($p=1,2$) as a function of the energy eigenbasis $\hat{U} \coloneqq (\ket*{E_{1}}\, \ket*{E_{2}}\, \cdots) = \sum_{\alpha=1}^{D} \dyad*{ E_{\alpha} }{ \alpha }$, we introduce the operator
\begin{equation}
    \hat{\Delta}_{I}^{(\alpha\beta)} \coloneqq \dyad*{\alpha}{\beta} - \frac{ \delta_{\alpha\beta} }{ d_{E_{\alpha}, \var{E}} } \sum_{ \ket*{ E_{\gamma} } \in \mathcal{H}_{E_{\alpha},\var{E}} } \dyad*{ {\gamma} },
\end{equation}
where $\mathcal{H}_{E_{\alpha},\var{E}} \coloneqq \vecspan\Bqty*{ \ket*{ E_{\gamma} } \mid \abs*{E_{\gamma} - E_{\alpha} } < \var{E} }$ is the microcanonical energy shell, and $d_{E_{\alpha}, \var{E}} \coloneqq \dim \mathcal{H}_{E_{\alpha},\var{E}}$.
Then, we can write 
$f_{p}^{(\alpha\beta)}(\hat{U}) = \seminorm*{p}{ \hat{U} \hat{\Delta}_{I}^{(\alpha\beta)} \hat{U}^{\dagger} } \ (p=1,2)$.
We will drop the superscripts $(\alpha\beta)$ from $f_{p}^{(\alpha\beta)}(\hat{U})$ and $\hat{\Delta}_{I}^{(\alpha\beta)}$ whenever they are irrelevant.

\textbf{Step i): Estimating the average of $f_{1/2}^{(\alpha\beta)}$.}
Here, we estimate the typical deviation from thermal equilibrium for individual energy eigenstates by calculating the average of $f_{2}^{(\alpha\beta)}$ over the Haar measure.
Substituting the inequality $0 \leq \abs*{ \vec{X}^{T} \cdot \vec{X} } \leq \norm*{ \vec{X} }_{2}^{2}$ into the formula~\eqref{eq_ComputableFormulaFor2Norm} gives 
\begin{equation}
    \frac{1}{\sqrt{2}} \norm*{ \vec{X}_{U} }_{2} \leq \seminorm*{2}{\hat{U} \hat{X} \hat{U}^{\dagger}} \leq \norm*{ \vec{X}_{U} }_{2},
    \label{eq_BoundsOn2Norm}
\end{equation}
where $(\vec{X}_{U})_{\mu} \coloneqq \tr(\hat{\Lambda}_{\mu}^{\dagger} \hat{U} \hat{X} \hat{U}^{\dagger} )$.
The average of $\norm*{ \vec{X}_{U} }_{2}$ can be bounded from above and below by its second and fourth moments.
An upper bound follows from Jensen’s inequality, while a lower bound is obtained via Hölder’s inequality:
\begin{equation}
    \frac{(\mathbb{E}[\norm*{ \vec{X}_{U} }_{2}^2])^{\frac{3}{2}}}{ \sqrt{ \mathbb{E}[\norm*{ \vec{X}_{U} }_{2}^4] } }
    \leq \mathbb{E}\norm*{ \vec{X}_{U} }_{2} \leq \sqrt{ \mathbb{E}[\norm*{ \vec{X}_{U} }_{2}^{2}] }.
    \label{eq_BoundsOnAverageByMoments}
\end{equation}
The second and fourth moments of $\norm*{ \vec{X}_{U} }_{2}$ can be computed using the Weingarten calculus~\cite{Collins_NotAmMathSoc2022c}.
For traceless $\hat{X}$, they are evaluated as
\begin{equation}
    \begin{aligned}
        \mathbb{E}[ \norm*{ \vec{X}_{U} }_{2}^2 ]
        &= \frac{M - 1}{D^2 - 1} \norm*{ \hat{X} }_{2}^{2}\qc \\
        \mathbb{E}[ \norm*{ \vec{X}_{U} }_{2}^4 ] &\leq \frac{9M^2}{D^4} \norm*{ \hat{X} }_{2}^{4} \qty(1+\order{\frac{1}{D^2}}).
    \end{aligned}
    \label{eq_HaarMoments}
\end{equation}
In addition, we can calculate
\begin{align}
    \frac{1}{2} \leq
    \norm*{ \hat{\Delta}_{I}^{(\alpha\beta)} }_{2}^{2} &= 1 - \delta_{\alpha\beta} \frac{1}{d_{E_{\alpha}, \var{E}}} 
    \leq 1.
    \label{eq_ETHOp2Norm}
\end{align}
except for the trivial case of $d_{E_{\alpha}, \var{E}} = 1$, which we do not consider.
We also have $M/2D^2 \leq (M-1)/(D^2-1) \leq M/D^2$ except for the trivial case of $M = 1$, which we do not consider.
From $f_{2}^{(\alpha\beta)}(\hat{U}) = \seminorm*{2}{ \hat{U} \hat{\Delta}_{I}^{(\alpha\beta)} \hat{U}^{\dagger} }$ and Eqs.~\eqref{eq_BoundsOn2Norm}--\eqref{eq_ETHOp2Norm}, we obtain
\begin{equation}
    \frac{1}{24} \frac{\sqrt{M}}{D} \leq \mathbb{E}[f_{2}^{(\alpha\beta)}] \leq \frac{\sqrt{M}}{D}.
    \label{eq_BoundsOnAverageOfF2}
\end{equation}
From the inequality $\seminorm*{2}{\cdot} \leq \seminorm*{1}{\cdot} \leq \sqrt{D} \seminorm*{2}{\cdot}$, we also obtain
\begin{equation}
        \frac{1}{24} \frac{\sqrt{M}}{D} \leq \mathbb{E}[f_{1}^{(\alpha\beta)}] \leq \sqrt{\frac{M}{D}}.
    \label{eq_BoundsOnAverageOfF1}
\end{equation}

\textbf{Step ii): Concentration of each $f_{1/2}^{(\alpha\beta)}(\hat{U})$.}
We show then that these small deviations occur for almost all $\hat{H} \in \mathcal{G}_{\mathrm{inv}}$.
To derive an inequality for $\ETHmeasure[p]{\hat{H}}{\opSet} = \max_{ \ket*{E_{\alpha}}, \ket*{E_{\beta}} \in \mathcal{H}_{E,\Delta} } f_{p}^{(\alpha\beta)}(\hat{U})$ with $p=1,2$, we employ the concentration inequality for the Haar measure~\cite{Ledoux2001,meckes2019random, Mele2024-jg}, which applies to any Lipschitz continuous function of $\hat{U}$.
We here show the Lipschitz continuity of $f_{p}^{(\alpha\beta)}(\hat{U})$ before introducing the concentration inequality.

For this purpose, we show the Lipschitz continuity of the inner product $\tr(\hat{A} \hat{U} \hat{X} \hat{U}^{\dagger})$ for any $\hat{A}$ and $\hat{X}$ independent of $\hat{U}$. 
Since $\norm*{\hat{A}}_{\infty} \leq \norm*{\hat{A}}_{q}$ for any $q\geq 1$, this follows from the triangle and the Hölder inequalities as
\begin{align}
    &\abs{ \tr(\hat{A} \hat{U}_{1} \hat{X} \hat{U}_{1}^{\dagger}) - \tr(\hat{A} \hat{U}_{2} \hat{X} \hat{U}_{2}^{\dagger}) }
    \nonumber \\
    % 
    &\leq \abs{ \tr[ \hat{A} (\hat{U}_{1} - \hat{U}_{2}) \hat{X} \hat{U}_{1}^{\dagger} ] } + \abs{ \tr[ \hat{A} \hat{U}_{2} \hat{X} (\hat{U}_{1}^{\dagger} - \hat{U}_{2}^{\dagger}) ] }
    \nonumber \\
    % 
    &\leq 2 \norm*{ \hat{A} }_{q} \norm*{ \hat{X} }_{2} \norm*{ \hat{U}_{1} - \hat{U}_{2} }_{2}.
    \label{eq_LipschitzContinuityOfTrace}
\end{align}
Since $\abs*{ \max_{x} g(x) - \max_{x} h(x) } \leq \max_{x} \abs*{ g(x) - h(x) }$ for any functions $g$ and $h$, we conclude from Eq.~\eqref{eq_LipschitzContinuityOfTrace} that $f_{p}^{(\alpha\beta)}(\hat{U}) = \seminorm*{p}{\hat{U} \hat{\Delta}^{(\alpha\beta)}_{I} \hat{U}^{\dagger}} \ (p=1,2)$ is Lipschitz continuous with the Lipschitz constant no larger than $2$, which is independent of the indices $(\alpha\beta)$.


Then, we can apply to $f_{p}^{(\alpha\beta)}(\hat{U})$ the concentration inequalities for the Haar measure on $\mathbb{SU}(D)$~\cite{Ledoux2001,meckes2019random, Mele2024-jg}.
It states that for any $\delta  > 0$ that can be $D$-dependent and any Lipschitz continuous function $f$ on $\mathbb{SU}(D)$
with Lipschitz constant $\eta_{f}$, the following inequalities hold:
\begin{align}
    &\Prob\bqty{ f(\hat{U}) \geq \mathbb{E}[f] + \delta } \leq \exp\!\bqty{ -\frac{ \delta^2 \dimTot }{ 4\eta_{f}^2 } }
    \label{eq_ConcentrationSU_Upper}
    \\
    \qq{and} 
    &\Prob\bqty{ f(\hat{U}) \leq \mathbb{E}[f] -\delta } \leq \exp\!\bqty{ -\frac{ \delta^2 \dimTot }{ 4\eta_{f}^2 } }
    \label{eq_ConcentrationSU_Lower}.
\end{align}
This step provides uniform control over all observables in $\mathcal{A}$, going beyond a straightforward application of Levy’s lemma.


\textbf{Step iii): Controlling the maximum over $(\alpha,\beta)$.}
The bounds~\eqref{eq_BoundsOnAverageOfF2} and \eqref{eq_BoundsOnAverageOfF1} on the average of $f_{p}^{(\alpha\beta)} \ (p=1,2)$ and the Lipchitz constant of $f_{p}^{(\alpha\beta)}$ are both independent of $(\alpha,\beta)$.
Then, the strong concentration inequalities in Eq.~\eqref{eq_ConcentrationSU_Upper} and \eqref{eq_ConcentrationSU_Lower} applied to $\Bqty*{f_{p}^{(\alpha\beta)}}$ enable us to derive the concentration inequality for the maximum $\Lambda_{p} = \max_{\alpha,\beta} f_{p}^{(\alpha\beta)}$.
For a family of Lipschitz functions, we can derive the following lemma.
\begin{lemma} \label{lemma_Methods}
    Let $\mathcal{I}$ be a finite set of indices and $\Bqty*{ f_{j}(\hat{U}) }_{j \in \mathcal{I}}$ be a family of Lipschitz functions on $\mathbb{SU}(D)$ with Lipschitz constant no larger than $\eta$, which is independent of the indices $j \in \mathcal{I}$.
    Suppose that there exist constants $C_{\min}$ and $C_{\max}$ that are independent of $j$ and satisfy $C_{\min} \leq \mathbb{E} f_{j} \leq C_{\max}$ for all $j \in \mathcal{I}$.
    Then, for arbitrary $\delta > 0$, we have
    \begin{equation}
        C_{\min} - \delta \leq \max_{j \in \mathcal{I}} f_{j}(\hat{U}) \leq C_{\max} + \delta
        \label{eq_LemmaHighProbabilityBounds}
    \end{equation}
    for a set of $\hat{U}$ whose probability measure is no smaller than $1 - 2 \abs*{\mathcal{I}} \exp[ -\delta^2 D / 4 \eta^2 ]$.
\end{lemma}
\begin{proof}
    Since $\max_{j\in\mathcal{I}} f_{j}(\hat{U}) \geq C_{\max} + \delta$ implies that there exists at least one $j' \in \mathcal{I}$ such that $f_{j'}(\hat{U}) \geq C_{\max} + \delta$, we obtain
    \begin{align}
        \Prob\bqty\Big{\max_{j\in\mathcal{I}} f_{j} \geq C_{\max} +\delta} 
        &\leq \sum_{j\in\mathcal{I}} \Prob\bqty\Big{f_{j} \geq C_{\max} + \delta}
        \nonumber \\
        % 
        &\leq \sum_{j\in\mathcal{I}} \Prob\bqty\Big{f_{j} \geq \mathbb{E}f_{j} + \delta}
        \nonumber \\
        % 
        &\leq \abs*{\mathcal{I}} \exp\!\bqty{ -\frac{\delta^2 D^2}{4 \eta^2} }.
        \label{eq_LemmaUpperBound}
    \end{align}
    where we use $\mathbb{E} f_{j} \leq C_{\max}$ in deriving the second inequality, and use the inequality~\eqref{eq_ConcentrationSU_Upper} in deriving the last inequality.
    In a similar manner, we obtain
    \begin{equation}
        \Prob\bqty\Big{\max_{j\in\mathcal{I}} f_{j} \leq C_{\min} -\delta} 
        \leq \abs*{\mathcal{I}} \exp\!\bqty{ -\frac{\delta^2 D^2}{4 \eta^2} }.
        \label{eq_LemmaLowerBound}
    \end{equation}
    The equations~\eqref{eq_LemmaUpperBound} and \eqref{eq_LemmaLowerBound} complete the proof of Lemma~\ref{lemma_Methods}.
\end{proof}

Let $\mathcal{I}$ be a set of indices $(\alpha\beta)$ such that $\ket*{ E_{\alpha} }, \ket*{ E_{\beta} } \in \mathcal{H}_{E,\Delta}$
% and consider the families of Lipschitz functions $\Bqty*{ f_{p}^{(\alpha\beta)} }_{(\alpha\beta) \in \mathcal{I}}$ with $p=1,2$.
% We have bounds on $\mathbb{E}[f_{p}^{(\alpha\beta)}]$ independent of the indices $(\alpha\beta) \in \mathcal{I}$ in Eqs.~\eqref{eq_BoundsOnAverageOfF2} and \eqref{eq_BoundsOnAverageOfF1} and showed the Lipschitz continuity of each $f_{p}^{(\alpha\beta)}$.
% Therefore, we can 
and apply Lemma~\ref{lemma_Methods} to the families $\Bqty*{ f_{p}^{(\alpha\beta)} }_{(\alpha\beta) \in \mathcal{I}}$ with $p=1,2$.
Since $\delta > 0$ is arbitrary, we can choose $\delta = \order*{\dimTot^{\epsilon} / \sqrt{\dimTot}}$ for any positive $D$-independent constants $\epsilon > 0$.
The inequality~\eqref{eq_LemmaHighProbabilityBounds} with this choice of $\delta$ implies that
\begin{align}
    \frac{1}{24} \frac{\sqrt{M}}{D} - \order{ \frac{\dimTot^{\epsilon}}{\sqrt{\dimTot}} } \leq &\ETHmeasure[2]{\hat{H}}{\opSet} \leq \frac{\sqrt{M}}{D} + \order{ \frac{\dimTot^{\epsilon}}{\sqrt{\dimTot}} }\qc \notag \\
    % 
    \frac{1}{24} \frac{\sqrt{M}}{D} - \order{ \frac{\dimTot^{\epsilon}}{\sqrt{\dimTot}} } \leq &\ETHmeasure[1]{\hat{H}}{\opSet} \leq \sqrt{\frac{M}{D}} + \order{ \frac{\dimTot^{\epsilon}}{\sqrt{\dimTot}} },
    % \label{eq_BoundsOnETHmeasureMethods}
\end{align}
for sets of $\hat{U}$ whose probability measure is no smaller than $1 - 2 \abs*{\mathcal{I}} \exp\!\bqty{ - \order{\dimTot^{2\epsilon}} }$.
This is the precise meaning of the inequality~\eqref{eq_InequalityForETHmeasure} in the main text.
Here, we have
\begin{equation}
    \abs*{\mathcal{I}} e^{ - \order{\dimTot^{2\epsilon}} }
    \leq e^{ - \order{\dimTot^{2\epsilon}} +2\log D } = e^{ -\order{ \dimTot^{2\epsilon} } },
\end{equation}
where we use $\abs*{\mathcal{I}} \leq D^2$.
This is the precise meaning of the statement \enquote{for almost all $\hat{H} \in \mathcal{G}_{\mathrm{inv}}$} below the inequality~\eqref{eq_InequalityForETHmeasure} in the main text.


%%%%%%%%%%%%%%%%%%%%%%%%%%%%%%%%%%%%%%%%%%%%%%%%%%%%%%%%%%%%%%
\bigskip
\paragraph{Proof of Theorem~\ref{thm_MainTheorem} for spin systems.---}
For spin-$S$ systems, the $m$-body operator space is defined as
\begin{equation}
    \opSet_{\particleNumber}^{[0,m]}
    \coloneqq \bigoplus_{\tilde{m} = 0}^{m} \vecspan\!\Bqty{ \hat{\sigma}_{x_{1}}^{(p_{1})} \cdots \hat{\sigma}_{x_{ \tilde{m} }}^{(p_{ \tilde{m} })} \mid \substack{ 1 \leq x_{1} \leq \cdots x_{\tilde{m}} \leq V \\ 1 \leq p_{j} < \dimLoc^2 } }.
\end{equation}
The dimension of this operator space is calculated to be
\begin{gather}
    \dim \opSet_{\particleNumber}^{[0,m]} = \dimTot^2\sum_{j=0}^{m} P_{j},
\end{gather}
where $D_{\particleNumber} \coloneqq \dim\mathcal{H}_{\particleNumber} = (\dimLoc)^{N}$, and $P_{j} \coloneqq \binom{\particleNumber}{j} \frac{ (\dimLoc^2 - 1)^{j} }{\dimLoc^{2\particleNumber}}$ with $\dimLoc \coloneqq 2S+1$ is the probability mass for the binomial distribution $\mathcal{B}(\particleNumber, p)$ with $p = 1 - (\dimLoc)^{-2}$.
As a property of the binomial distribution, we have $P_{m-1} < P_{m}$ for $m < (\particleNumber + 1) p$.
Therefore, we have $\dim \opSet_{\particleNumber}^{[0,m]} \leq (m+1) P_{m}$ for $m/\particleNumber < p$.

To derive the lower bound $\lowerBound^{\mathrm{H}}$, we apply the Stirling's formula to $P_{m}$, obtaining
\begin{align}
    \frac{ \dim \opSet_{\particleNumber}^{[0,m]} }{\dimTot_{\particleNumber}}
    &\leq \exp\!\bqty\Big{ \particleNumber G\qty(\frac{m}{\particleNumber}) +\order{\log N} }
\end{align}
for $m/\particleNumber < p$,
where $G(x) \coloneqq H(x) + x\log(\dimLoc^2 - 1) -\log \dimLoc$, and $H(x) = -x\log x - (1-x) \log(1-x)$.
Here, we define $\lowerBound^{\mathrm{H}}$ as a root of $G(x)$.
Because $G(0) = -\log \dimLoc < 0$, and 
\begin{equation}
    G\qty(\frac{1}{2}) = \frac{1}{2} \log 4\qty(1 - \frac{1}{\dimLoc^2}),
\end{equation}
which is positive for $\dimLoc \geq 2$, we conclude $\lowerBound^{\mathrm{H}} \in (0,1/2)$.
Moreover, it is straightforward to show that $G(x) < 0$ for $x < \lowerBound^{\mathrm{H}}$.
Therefore, the ETH measure decays exponentially with increasing system size when $m/\particleNumber < \lowerBound^{\mathrm{H}}$ as claimed in Theorem~\ref{thm_MainTheorem} in the main text.

To derive the upper bound $\upperBound^{\mathrm{H}}$, we use the fact that $\mathcal{B}(\particleNumber, p)$ converges in distribution to a Gaussian distribution $\mathcal{N}\qty(\particleNumber p, \particleNumber p(1-p))$ for large $\particleNumber$.
This implies that $\dim \opSet_{N}^{[0,Np]} / D_{\particleNumber}^2$ converges to $1/2$ for large $N$.
Then, Theorem~\ref{th_theorem1} implies $\upperBound^{\mathrm{H}} = p = 1 - (\dimLoc)^{-2}$ for spin systems.


%%%%%%%%%%%%%%%%%%%%%%%%%%%%%%%%%%%%%%%%%%%%%%%%%%%%%%%%%%%%%%
\bigskip
\paragraph{Proof of Theorem~\ref{thm_MainTheorem} for Bose and Fermi systems.---}
For $\particleNumber$-particle Bose and spinless Fermion systems on a $V$-site lattice, the $m$-body operator space is defined as 
\begin{equation}
    \opSet_{N,V}^{[0,m]}
    \coloneqq \Bqty{ \hat{A} + \hat{A}^{\dagger},\ i(\hat{A} - \hat{A}^{\dagger}) \mid \hat{A} \in \tilde{\opSet}_{N,V}^{[0,m]} },
\end{equation}
where
\begin{equation}
    \tilde{\opSet}_{N,V}^{[0,m]}
    \coloneqq \vecspan\!\Bqty{ \hat{a}_{x_{1}}^{\dagger} \cdots \hat{a}_{x_{m}}^{\dagger} \hat{a}_{y_{1}} \cdots \hat{a}_{y_{m}} \mid \substack{ 1 \leq x_{j} \leq V \\ 1 \leq y_{j} \leq V } }.
    \label{def_mBodyOpBoseFermi}
\end{equation}
Here, $\hat{a}$ denotes the bosonic annihilation operator $\hat{b}$ or the fermionic annihilation operator $\hat{f}$, and we recall that we consider the Hilbert space with a definite particle number as 
\begin{equation}
    \mathcal{H}_{\particleNumber, \sysSize} = \vecspan\!\Bqty*{ \hat{a}_{x_{1}}^{\dagger} \cdots \hat{a}_{x_{N}}^{\dagger} \ket*{0} \mid 1 \leq x_{j} \leq V }.
\end{equation}

While the numbers of annihilation and creation operators are both required to be equal to $m$ in Eq.~\eqref{def_mBodyOpBoseFermi}, we can show that the space $\tilde{\opSet}_{N,V}^{[0,m]}$ includes $\tilde{\opSet}_{N,V}^{[0,m-1]}$ so that $\tilde{\opSet}_{N,V}^{[0,m-1]} \subset \tilde{\opSet}_{N,V}^{[0,m]}$.
This is because the particle-number conservation makes the particle-number operator $\hat{N} \coloneqq \sum_{x=1}^{N} \hat{a}_{x}^{\dagger} \hat{a}_{x}$ essentially proportional to the identity operator.
Indeed, by denoting the projection operator onto $\mathcal{H}_{\particleNumber, \sysSize}$ by $\hat{\Pi}_{\particleNumber, \sysSize}$, we have
\begin{align}
    &\hat{\Pi}_{\particleNumber, \sysSize}
    ( \hat{a}_{x_{1}}^{\dagger} \cdots \hat{a}_{x_{m-1}}^{\dagger} \hat{a}_{y_{1}} \cdots \hat{a}_{y_{m-1}} )
    \hat{\Pi}_{\particleNumber, \sysSize}
    \nonumber \\
    % 
    &= \hat{\Pi}_{\particleNumber, \sysSize}
    ( \hat{a}_{x_{1}}^{\dagger} \cdots \hat{a}_{x_{m-1}}^{\dagger} \frac{\hat{N}}{N-m+1} \hat{a}_{y_{1}} \cdots \hat{a}_{y_{m-1}} )
    \hat{\Pi}_{\particleNumber, \sysSize}
    \nonumber \\
    % 
    &\propto \sum_{x_{m}=1}^{V}
    \hat{\Pi}_{\particleNumber, \sysSize}
    ( \hat{a}_{x_{1}}^{\dagger} \cdots \hat{a}_{x_{m-1}}^{\dagger} \hat{a}_{x_{m}}^{\dagger} \hat{a}_{x_{m}} \hat{a}_{y_{1}} \cdots \hat{a}_{y_{m-1}} )
    \hat{\Pi}_{\particleNumber, \sysSize},
\end{align}
which proves $\opSet_{N,V}^{[0,m-1]} \subset \opSet_{N,V}^{[0,m]}$.

\textbf{Proof of Theorem~\ref{thm_MainTheorem} for Bose systems.}
The dimension of the $m$-body operator space is calculated to be $\dim \opSet_{\particleNumber, \sysSize}^{[0,m]} = \binom{ V + m - 1 }{ m }^2$, while that of the Hilbert space is given by $D_{\particleNumber, \sysSize} = \dim \mathcal{H}_{\particleNumber, \sysSize} = \binom{V + N - 1}{ N }$.

To derive the lower bound $\lowerBound^{\mathrm{H}}$, we apply the Stirling's formula, obtaining
\begin{equation}
    \frac{ \dim \opSet_{N,V}^{[0,m]} }{ D }
    = \exp\!\bqty{ \sysSize G\qty(\frac{m}{\particleNumber}) + \order{\log \sysSize}},
\end{equation}
where $G(x) \coloneqq 2 (1+n x) H((1+n x)^{-1}) - (1+n) H((1+n)^{-1})$ with $n \coloneqq \particleNumber / \sysSize$.
Here, $G(x)$ is a monotonically increasing function of $x$, and we define $\lowerBound^{\mathrm{H}}$ as a root of $G(x)$.
Then, we have $G(0) = -(1+n) H((1+n)^{-1}) < 0$ and
\begin{equation}
    G\qty(\frac{1}{2}) = \log \frac{\qty(1+\frac{n}{2})^2}{1+n} + n\log \frac{2+n}{1+n} > 0.
\end{equation}
Therefore, we conclude $\lowerBound^{\mathrm{H}} \in (0,1/2)$.
Since $G(x)$ increases monotonically, we have $G(x) < 0$ for $x < \lowerBound^{\mathrm{H}}$.
Therefore, the ETH measure decays exponentially with respect to the system size when $m/\particleNumber < \lowerBound^{\mathrm{H}}$ as claimed in Theorem~\ref{thm_MainTheorem} in the main text.
% 
For Bose systems, the lower bound in Eq.~\eqref{eq_InequalityForETHmeasure} does not give a bound better than the trivial one, i.e., $\upperBound^{\mathrm{H}} = 1$.


%%%%%%%%%%%%%%%%%%%%%%%%%%%%%%%%%%%%%%%%%%%%%%%%%%%%%%%%%%%%%%
\bigskip
\textbf{Proof of Theorem~\ref{thm_MainTheorem} for spinless Fermi systems.}
To calculate the dimension of $\opSet_{N,V}^{[0,m]}$, we consider the particle-hole transformation $\hat{C}$ defined by $\hat{C} \hat{f}_{x} \hat{C}^{\dagger} = \hat{f}_{x}^{\dagger}$ and $\hat{C} \ket*{0} = \hat{f}_{1}^{\dagger} \cdots \hat{f}_{\sysSize}^{\dagger} \ket*{0}$.
Then, we have
\begin{align}
    &\hat{C} \hat{\Pi}_{\particleNumber, \sysSize}
    ( \hat{f}_{x_{1}}^{\dagger} \cdots \hat{f}_{x_{m}}^{\dagger} \hat{f}_{y_{1}} \cdots \hat{f}_{y_{m}} )
    \hat{\Pi}_{\particleNumber, \sysSize} \hat{C}^{\dagger}
    \nonumber \\
    % 
    &= \hat{\Pi}_{\sysSize - \particleNumber, \sysSize}
    ( \hat{f}_{x_{1}} \cdots \hat{f}_{x_{m}} \hat{f}_{y_{1}}^{\dagger} \cdots \hat{f}_{y_{m}}^{\dagger} )
    \hat{\Pi}_{\sysSize - \particleNumber, \sysSize}.
    \label{eq_ParticleHoleOperation}
\end{align}
Since exchanging the order of operators $\hat{f}_{x_{1}} \cdots \hat{f}_{x_{m}}$ and $\hat{f}_{y_{1}}^{\dagger} \cdots \hat{f}_{y_{m}}^{\dagger}$ results in additional terms of at most $(m-1)$-body ones, Eq.~\eqref{eq_ParticleHoleOperation} implies $\dim \opSet_{\particleNumber, \sysSize}^{[0,m]} = \dim(\hat{C} \opSet_{\particleNumber, \sysSize}^{[0,m]} \hat{C}^{\dagger}) \leq \dim \opSet_{\sysSize - \particleNumber, \sysSize}^{[0,m]}$ for $m \leq \min\Bqty*{\particleNumber, \sysSize - \particleNumber}$.
By setting $\particleNumber \mapsto \sysSize - \particleNumber$ in the above argument, we obtain the reversed inequality.
Therefore, we arrive at 
\begin{equation}
    \dim \opSet_{\particleNumber, \sysSize}^{[0,m]} = \dim \opSet_{\sysSize - \particleNumber, \sysSize}^{[0,m]}
    \label{eq_ParticleHoleDim}
\end{equation}
for $m \leq \min\!\Bqty*{\particleNumber, \sysSize-\particleNumber}$.
This relation combined with the inclusion relation $\opSet_{N,V}^{[0,m-1]} \subset \opSet_{N,V}^{[0,m]} \subset \mathcal{L}(\mathcal{H})$ gives $\dim \opSet_{\particleNumber, \sysSize}^{[0,m]} = D_{\particleNumber, \sysSize}^{2}$ for $m \geq \min\Bqty*{\particleNumber, \sysSize - \particleNumber}$.
Because of Eq.~\eqref{eq_ParticleHoleDim}, the remaining task is to calculate $\dim \opSet_{\particleNumber, \sysSize}^{[0,m]}$ for $\particleNumber / \sysSize < 1/2$.
In this case, there are $\binom{\sysSize}{\particleNumber}$ choices for both $\Bqty*{ x_{1}, \cdots, x_{m} }$ and $\Bqty*{ y_{1}, \cdots, y_{m} }$ to have nonzero basis operators.
We can show that different choices for these indices give linearly independent basis operators.~(See \Supple{}~II for the proof.)
Therefore, the dimension of the $m$-body operator space is calculated to be $\dim \opSet_{\particleNumber, \sysSize}^{[0,m]} = \binom{ \sysSize }{ \min\Bqty*{m, \sysSize - \particleNumber} }^2$, while that of the Hilbert space is given by $D_{\particleNumber, \sysSize} = \dim \mathcal{H}_{\particleNumber, \sysSize} = \binom{\sysSize}{\particleNumber}$.

To derive the lower bound $\lowerBound^{\mathrm{H}}$, we apply Stirling's formula, obtaining
\begin{equation}
    \frac{ \dim \opSet_{N,V}^{[0,m]} }{ D }
    = \exp\!\bqty{ \sysSize G\qty(\frac{m}{\particleNumber}) + \order{\log \sysSize}},
\end{equation}
where we set $n\coloneqq \particleNumber/\sysSize \ (\leq 1)$ and
\begin{equation}
    G(x) \coloneqq
    \begin{cases}
        2H(n x) - H(n) & \qty(x \leq \frac{1-n}{n}); \\
        H(n) & \qty(x > \frac{1-n}{n}).
    \end{cases}
\end{equation}
Here, $G(x)$ is a monotonically increasing function of $x$, and we define $\lowerBound^{\mathrm{H}}$ as a root of $G(x)$.
Then, we have $G(0) = -H(n) < 0$ and
\begin{equation}
    G\qty(\frac{1}{2}) =
    \begin{cases}
        \int_{0}^{n} \log \frac{2-s}{1-s} \dd{s} & (n \leq 2/3); \\
        H(n) & (n > 2/3),
    \end{cases}
\end{equation}
which is positive.
Therefore, we conclude $\lowerBound^{\mathrm{H}} \in (0,1/2)$.
Since $G(x)$ increases monotonically, we have $G(x) < 0$ for $x < \lowerBound^{\mathrm{H}}$.
Therefore, the ETH measure decays exponentially with respect to the system size when $m/\particleNumber < \lowerBound^{\mathrm{H}}$ as claimed in Theorem~\ref{thm_MainTheorem} in the main text.

To derive the lower bound $\lowerBound^{\mathrm{H}}$, we use $\dim \opSet_{N,V}^{[0,m]} = D^2$ when $m \geq \min\Bqty*{\particleNumber, \sysSize - \particleNumber}$.
Dividing both sides by  $\particleNumber$, we find that we can take $\upperBound^{\mathrm{H}} = \min\Bqty*{1, (1-n)/n}$.
This completes the proof of Theorem~\ref{thm_MainTheorem}.


%%%%%%%%%%%%%%%%%%%%%%%%%%%%%%%%%%%%%%%%%%%%%%%%%%%%%%%%%%%%%%
\bigskip
\paragraph{Details on the numerical calculation.---}
In our numerical tests of Conjecture~\ref{conj_Conjecture} for systems with local and few-body interactions, we calculate the quantity
\begin{equation}
    \Lambda_{2}^{[0,m]}
    \coloneqq 
    \max_{ \ket*{E_{\alpha}} \in \mathcal{H}_{E_{x}, \Delta} } \seminorm[\opSet_{\particleNumber}^{[0,m]}]{2}{ \dyad*{ E_{\alpha} }{ E_{\alpha} } - \MCE[\var{E}](E_{\alpha}) },
    \label{eqS_quasiETHmeasure}
\end{equation}
which provides a lower bound on the ETH measure and an upper bound on the diagonal ETH measure.
% 
Here, the energy window $\mathcal{H}_{E_{x}, \Delta}$ for testing the ETH is centered at $E_{x} \coloneqq x E_{\max} + (1-x) E_{\min}$ with width $2\Delta \coloneqq 0.05 (E_{\max} - E_{\min})$, where $E_{\max}$ and $E_{\min}$ are the maximum and minimum eigenvalues of $\hat{H}$, respectively. 
Since the spectral width $(E_{\max} - E_{\min})$ is extensive for locally interacting systems~\cite{Tasaki2018-bz}, the normalized energy $x = (E - E_{\min})/(E_{\max} - E_{\min})$ represents the energy density up to linear rescaling.
% 
The width of the microcanonical energy shell is set to be $\var{E} = 0.2 (E_{\max} - E_{\min}) / V$, which we confirm is sufficiently small to exclude contributions to $\Lambda_{2}^{[0,m]}$ from the energy dependence of the microcanonical average for the system sizes considered (up to 18 spins for spin systems, and up to $11$ particles for Bose and Fermi systems at unit and half filling, respectively).

The quantity $\Lambda_{2}^{[0,m]}$ is computed using the explicit formula~\eqref{eq_ComputableFormulaFor2Norm}.
The calculation of the 2-norm for a given $\hat{X}$ costs $\order{M D}$ steps, assuming that an orthogonal basis of $\opSet_{\particleNumber}^{[0,m]}$ consists of sparse matrices in the Fock basis, where $M \coloneqq \dim(\opSet_{\particleNumber}^{[0,m]} + \mathbb{R} \hat{I})$ and $D \coloneqq \dim \mathcal{H}$.
% 
Thus, the computation of the bounds including off-diagonal elements costs $\order{M D^3}$ steps, which is much heavier than the cost $\order{M D^2}$ for the diagonal elements and the cost $\order{D^3}$ of numerical diagonalization of $\hat{H}$.
Therefore, we restrict ourselves to diagonal elements, as in Eq.~\eqref{eqS_quasiETHmeasure}.

The formula~\eqref{eq_ComputableFormulaFor2Norm} requires an orthonormal basis $\Bqty*{ \hat{\Lambda}_{\mu} }_{\mu=1}^{M}$ of $\opSet_{\particleNumber}^{[0,m]} + \mathbb{R} \hat{I}$.
In spin-1/2 systems, such a basis is given by
\begin{equation*}
    \Bqty*{ \hat{\Lambda}_{\mu} } = 2^{-\frac{\sysSize}{2}} \Bqty{ \hat{\sigma}^{\vb{p}} \coloneqq \hat{\sigma}_{1}^{(p_{1})} \hat{\sigma}_{2}^{(p_{2})} \cdots \hat{\sigma}_{\sysSize}^{(p_{\sysSize})} \mid m(\vb{p}) \leq m },
\end{equation*}
where $\vb{p} \coloneqq (p_{1}, p_{2}, \cdots, p_{\sysSize}) \in \Bqty*{0,1,2,3}^{\!\!\sysSize{}}$ with $\hat{\sigma}^{(0)} = \hat{I}$~(identity), and $m(\vb{p})$ denotes the number of nonzero components in $\vb{p}$.
% 
For Bose and Fermi systems, the basis operators $\Bqty*{ \hat{\boldsymbol{\alpha}}(\vb{x},\vb{y}) \coloneqq \hat{a}_{x_{1}}^{\dagger} \cdots \hat{a}_{x_{m}}^{\dagger} \hat{a}_{y_{1}} \cdots \hat{a}_{y_{m}} }$ are not mutually orthogonal under the Hilbert–Schmidt inner product.
Here, $\vb{x} = (x_{1},\cdots,x_{m})$ and $\vb{y} = (y_{1},\cdots,y_{m})$, and $\hat{a}$ denotes either a bosonic or fermionic annihilation operator.
% 
To obtain an orthonormal basis, we construct the Gram matrix $\vb{G}$ given by $G_{(\vb{x},\vb{y}), (\vb{x}',\vb{y}')} \coloneqq \tr_{N}\!\bqty{ \hat{\boldsymbol{\alpha}}(\vb{x},\vb{y})^{\dagger} \hat{\boldsymbol{\alpha}}(\vb{x}',\vb{y}') }$, where $\tr_{N}$ denotes the trace over the $N$-particle Hilbert space.
Since $\vb{G}$ is Hermitian and positive definite, the Cholesky decomposition yields $\vb{G} = \vb{R}^{\dagger} \vb{R}$, where $\vb{R}$ is an upper triangular matrix.
An orthonormal basis is then constructed as $\hat{\Lambda}_{\nu} \coloneqq \sum_{\mu} \hat{\boldsymbol{\alpha}}_{\mu} (\vb{R}^{-1})_{\mu\nu}$.
Thus, the coefficient vector $\vec{X}$ appearing in the formula~\eqref{eq_ComputableFormulaFor2Norm} is obtained as
\begin{align}
    \vec{X} = (\vb{R}^{\dagger})^{-1} \vec{v},\qq{where} v_{\mu} \coloneqq \tr[ \hat{\boldsymbol{\alpha}}_{\mu}^{\dagger} \hat{X} ].
    \label{eqS_2normComputableFormula2}
\end{align}
We employ this expression to calculate the 2-norm for Bose and Fermi systems.

In general, the evaluation of the formula~\eqref{eqS_2normComputableFormula2} costs $\order{D M^2}$ operations to construct the Gram matrix $\vb{G}$, and $\order{M^3}$ operations to compute its Cholesky factor $\vb{R}$.
These costs readily exceed the diagonalization cost of the Hamiltonian, since the operator-space dimension $M$ can be much larger than the Hilbert-space dimension $D$.
% 
However, we find that the Gram matrix $\vb{G}$ has a block-diagonal structure and that the size of each block is reasonably small.
Moreover, for translationally invariant $\hat{X}$ satisfying $\hat{T} \hat{X} \hat{T}^{\dagger} = \hat{X}$, with $\hat{T}\hat{a}_{j} \hat{T}^{\dagger} \coloneqq \hat{a}_{j+1}$, these blocks can be grouped into orbits under the action of the shift operator $\hat{T}$.
Since all blocks in the same orbit contribute identically to $\seminorm*{2}{\hat{X}}$, it suffices to compute a single representative per orbit, further reducing the computational cost by a factor of $1/\sysSize$.

The block-diagonal structure of the Gram matrix $\vb{G}$ arises from the vanishing of many inner products $\tr_{N}[\hat{\boldsymbol{\alpha}}(\vb{x},\vb{y})^{\dagger} \hat{\boldsymbol{\alpha}}(\vb{x}',\vb{y}')]$.
This inner product is nonzero only if there exists a basis state $\ket*{\vb{n}} \coloneqq (\hat{a}_{1}^{\dagger})^{n_{1}} \cdots (\hat{a}_{\sysSize}^{\dagger})^{n_{\sysSize}} \ket{0}$ such that $\hat{\boldsymbol{\alpha}}(\vb{x},\vb{y}) \ket*{\vb{n}} = C_{\vb{n}}\, \hat{\boldsymbol{\alpha}}(\vb{x}',\vb{y}') \ket*{\vb{n}}$ for some $C_{\vb{n}} \neq 0$.
This is because each $\hat{\boldsymbol{\alpha}}$ maps a Fock state to another Fock state.
By tracking the net increase and decrease in occupation numbers at each site, this requirement leads to a simple condition involving the multisets $\mathcal{X} \coloneqq \Bqty*{ x_{j} }_{j=1}^{m}$ and $\mathcal{Y} \coloneqq \Bqty*{ y_{j} }_{j=1}^{m}$, and similarly $\mathcal{X}'$ and $\mathcal{Y}'$:
\begin{equation}
    \mathcal{X}\setminus \mathcal{Y} = \mathcal{X}'\setminus \mathcal{Y}'
    \qq{and}
    \mathcal{Y}\setminus \mathcal{X} = \mathcal{Y}'\setminus \mathcal{X}'.
\end{equation}
This condition partitions the operator basis $\Bqty*{\hat{\boldsymbol{\alpha}}(\vb{x},\vb{y})}$ into disjoint subsets within which the Gram matrix may have nonzero elements, resulting in its block-diagonal structure.
% 
As an illustrative example, we show the block structure of the Gram matrix in two cases with $m = 4$ and $\particleNumber = 11$.
For Bose systems at unit filling, i.e., $\sysSize = \particleNumber = 11$, the total operator-space dimension is $M = \num{1002001}$, and the Gram matrix has the following block structure:
\[
\begin{tabular}{|r|r|r|} \hline
  \text{Block size} &
  \multicolumn{1}{c|}{\shortstack{\strut{} \# of blocks \\ of the same size}} &
  \multicolumn{1}{c|}{\shortstack{\strut{} \# of orbits \\ by the shift operation}} \\ \hline%\hline
  \num{1001} \hspace{1ex} & \num{1}      \hspace{1ex} & \num{1}     \hspace{4ex}  \\ %\hline
   \num{286} \hspace{1ex} & \num{110}    \hspace{1ex} & \num{10}    \hspace{4ex} \\ %\hline
    \num{66} \hspace{1ex} & \num{3080}   \hspace{1ex} & \num{280}   \hspace{4ex} \\ %\hline
    \num{11} \hspace{1ex} & \num{40370}  \hspace{1ex} & \num{3670}  \hspace{4ex} \\ %\hline
     \num{1} \hspace{1ex} & \num{322190} \hspace{1ex} & \num{29290} \hspace{4ex} \\ \hline
\end{tabular}
\]
For spinless Fermi systems at half filling, i.e., $\sysSize = 2 \particleNumber = 22$, the dimension becomes $M = \num{53509225}$, and the corresponding block structure is:
\[
\begin{tabular}{|r|r|r|} \hline
  \text{Block size} &
  \multicolumn{1}{c|}{\shortstack{\strut{} \# of blocks \\ of the same size}} &
  \multicolumn{1}{c|}{\shortstack{\strut{} \# of orbits \\ by the shift operation}} \\ \hline%\hline
  \num{7315} \hspace{1ex} &   \num{1}       \hspace{1ex} &  \num{1}       \hspace{4ex} \\ %\hline
  \num{1140} \hspace{1ex} &  \num{462}      \hspace{1ex} &  \num{21}      \hspace{4ex} \\ %\hline
   \num{153} \hspace{1ex} &  \num{43890}    \hspace{1ex} &  \num{2000}    \hspace{4ex} \\ %\hline
    \num{16} \hspace{1ex} &  \num{1492260}  \hspace{1ex} &  \num{67830}   \hspace{4ex} \\ %\hline
     \num{1} \hspace{1ex} &  \num{22383900} \hspace{1ex} &  \num{1017540} \hspace{4ex} \\ \hline
\end{tabular}
\]
In both cases, the vast majority of blocks are extremely small compared with the total operator-space dimension $M$.
For example, more than 90\% of them have size at most $16$.
This structure of the Gram matrix enables the efficient blockwise evaluation of the 2-norm, making it feasible to test Conjecture~\ref{conj_Conjecture} numerically for Bose and spinless Fermi systems up to $\particleNumber = 11$ particles.

%%%%%%%%%%%%%%%%%%%%%%%%%%%%%%%%%%%%%%%%%%%%%%%%%%%%%%%%%%%%%%
\section{Acknowledgments}
\begin{acknowledgments}
% We are very grateful to Synge Todo and Tilman Hartwig for their help in our numerical calculation.
We are grateful to Takashi Mori for the discussions about the relation of our results to the previous ones.
S.S. is also grateful to Masaya Nakagawa for the valuable and helpful discussions on the dimension counting of the $m$-body space for Fermi systems.
This work was supported by KAKENHI Grant Numbers JP22H01152 from the Japan Society for the Promotion of Science (JSPS). 
% 
S.S. was supported by KAKENHI Grant Number JP22J14935 from the Japan Society for the Promotion of Science (JSPS) and Forefront Physics and Mathematics Program to Drive Transformation (FoPM), a World-leading Innovative Graduate Study (WINGS) Program, the University of Tokyo.
% 
R.H. was supported by JST ERATO-FS Grant Number JPMJER2204 and JST ERATO Grant Number JPMJER2302, Japan, and
%R.H. was supported by JST ERATO-FS Grant Number JPMJER2204, Japan.
JSPS KAKENHI Grant No. JP24K16982.
% 
M.U. is supported by the CREST program \enquote{Quantum Frontiers} (Grant No.~JPMJCR2311) by the Japan Science and Technology Agency.
\end{acknowledgments}

%%%%%%%%%%%%%%%%%%%%%%%%%%%%%%%%%%%%%%%%%%%%%%%%%%%%%%%%%%%%%%
\section{Competing interests}
The authors declare no competing interests.

\section{Author contributions}
All authors contributed to the manuscript writing, the discussion, and the interpretation of the analytical results and the numerical data.
S.S. conceived the project, carried out all the analytical and numerical calculations, and drafted and revised the manuscript.
R.H. regularly reviewed the analytical calculations, suggested improvements, and provided substantial feedback on the revision of the manuscript.
M.U. provided essential feedback on the clarity, structure, and overall presentation of the manuscript, and supervised the project.
% R.H. and M.U. greatly contributed to the improvement of the manuscript.
% M.U. supervised the work.



% Supplementary Information
%%%%%%%%%%%%%%%%%%%%%%%%%%%%%%%%%%%%%%%%%%%%%%%%%%%%%%%%%%%%%%
%%%%%%%%%%%%%%%%%%%%%%%%%%%%%%%%%%%%%%%%%%%%%%%%%%%%%%%%%%%%%%
\addtocontents{toc}{\string\tocdepth@restore} % should be before \clearpage to produce TOC for the supplemental material
\clearpage\clearpage
\makeatletter
   	\c@secnumdepth=4
    \def\@pointsize{11}
	\expandafter\@process@pointsize\expandafter{\@pointsize@default}%
	\appdef\setup@hook{\normalsize}%
	\setup@hook
\makeatother

\tcolorboxenvironment{proof}{% `proof' from `amsthm'
blanker, breakable, borderline west={4pt}{0pt}{lightgray}, left=12pt, top=6pt, bottom=6pt }

\setcounter{equation}{0}
\setcounter{figure}{0}
\setcounter{section}{0}
\setcounter{table}{0}
% Redefine counters for equations, figures, etc.
\renewcommand{\theequation}{S\arabic{equation}}
\renewcommand{\thefigure}{S\arabic{figure}}
% Redefine counters for hyperref accordingly
\renewcommand{\theHequation}{\theequation}
\renewcommand{\theHfigure}{\thefigure}

\renewcommand{\baselinestretch}{1.20}
\normalsize
\title{
    \Large Supplementary Information: \protect\\
    Bounds on eigenstate thermalization
}
\date{\today}
\maketitle
\onecolumngrid
\tableofcontents

\clearpage
% \section{Seminorm $\seminorm{1}{\cdot}$ as a unified measure \protect\\ of quantum-thermal equilibrium}
\refstepcounter{section}
\phantomsection
\addcontentsline{toc}{section}{%
  \protect\numberline{\thesection}%
  Seminorm $\seminorm{1}{\cdot}$ as a unified measure of quantum-thermal equilibrium%
}
\begin{center}
  \normalfont\small\bfseries
  \thesection.\quad \MakeUppercase{Seminorm $\seminorm{1}{\cdot}$ as a unified measure}\\
  \ \MakeUppercase{of quantum-thermal equilibrium}
\end{center}
\vspace{1ex}



As mentioned in the main text, the seminorm $\seminorm{1}{\cdot}$ with an apt choice of $\mathcal{A}$ serves as a unified measure for a rich variety of notions about quantum thermal equilibrium.
In this section, we provide some examples.

\subsection{Subsystem thermal equilibrium}
For any subsystem $\subSystem$ and any $\hat{A}_{\subSystem} \otimes \mathrm{id}_{\subSystem^{c}} \in \mathcal{L}(\mathcal{H}_{\subSystem}) \otimes \mathrm{id}_{\subSystem^{c}}$, we have $\opRange[\infty]{ \hat{A}_{\subSystem} \otimes \mathrm{id}_{\subSystem^{c}} } = \opRange[\infty]{ \hat{A}_{\subSystem} }$.
By setting $\opSet = \mathcal{L}(\mathcal{H}_{\subSystem}) \otimes \mathrm{id}_{\subSystem^{c}}$, we have
\begin{align}
    \seminorm*{1}{\hat{X}}
    = \max_{ \hat{A} \in \mathcal{L}(\mathcal{H}_{\subSystem}) \otimes \mathrm{id}_{\subSystem^{c}} } \abs{ \tr\qty( \frac{ \hat{A} }{ \opRange[\infty]{ \hat{A} } } \hat{X} ) } 
    % \nonumber \\
    % 
    = \max_{ \hat{A}_{\subSystem} \in \mathcal{L}(\mathcal{H}_{\subSystem}) } \abs{ \tr\qty( \frac{ \hat{A}_{\subSystem} }{ \opRange[\infty]{ \hat{A}_{\subSystem} } } \tr_{\subSystem^{c}}( \hat{X} ) ) } 
    % \nonumber \\
    = \norm{ \tr_{\subSystem^{c}}( \hat{X} ) }_{1}.
\end{align}
Thus, the seminorm $\seminorm{1}{\cdot}$ reduces to the trace norm on a subsystem $\subSystem$ for $\opSet = \mathcal{L}(\mathcal{H}_{\subSystem}) \otimes \mathrm{id}_{\subSystem^{c}}$, and the smallness of $\seminorm{1}{ \hat{\sigma} - \hat{\rho}_{\mathrm{th}} }$ for a thermal ensemble $\hat{\rho}_{\mathrm{th}}$ ensures that quantum states $\hat{\sigma}$ are in thermal equilibrium in a subsystem $\subSystem$.

\subsection{Microscopic thermal equilibrium~(MITE)~\citeSM{Goldstein2015SM,Goldstein2017SM,mori2018thermalizationSM}}
The notion of microscopic thermal equilibrium~(MITE) is introduced in Refs.~\citeSM{Goldstein2015SM, Goldstein2017SM} as follows.
\begin{framedDefinition}{Microscopic thermal equilibrium~(MITE)~\citeSM{Goldstein2017SM}}{}
    A state $\hat{\sigma}$ is said to be in microscopic thermal equilibrium~(MITE) on a length scale $l_{0}$ if it satisfies
    \begin{align}
        \norm{ \tr_{\subSystem^{c}}(\hat{\sigma}) - \tr_{\subSystem^{c}}(\hat{\rho}_{\mathrm{th}})  }_{1} < \epsilon
    \end{align}
    for every subsystem $\subSystem$ with $\mathop{\mathrm{Diam}}(\subSystem) \leq l_{0}$, where $\epsilon\ll 1$
    and $\mathop{\mathrm{Diam}}(\subSystem) \coloneqq \sup_{ x,y\in \subSystem } d(x,y)$ defines the diameter of $\subSystem$ for a given distance measure $d$.
\end{framedDefinition}
As mentioned in Ref.~\citeSM{Goldstein2017SM}, MITE can be regarded as the thermal equilibrium relative to
\begin{equation}
    \opSet_{\mathrm{MITE}} = \bigcup_{\subSystem\colon \mathop{\mathrm{Diam}}(\subSystem) \leq l_{0}} \mathcal{L}(\mathcal{H}_{\subSystem}) \otimes \mathrm{id}_{\subSystem^{c}}.
\end{equation}
% 
Then, we have the following proposition.
\begin{framedProposition}{}{}
    \label{prop_RelationToMITE}
    Let $\hat{\sigma}$ be an arbitrary quantum state and $\hat{\rho}_{\mathrm{th}}$ be a thermal ensemble.
    Then, 
    \begin{equation}
        \text{$\hat{\sigma}$ is in MITE.} 
        \iff 
        \seminorm[ \opSet_{\mathrm{MITE}} ]{1}{ \hat{\sigma} - \hat{\rho}_{\mathrm{th}} }
        \leq \epsilon,
    \end{equation}
    where $\epsilon\ll 1$.
\end{framedProposition}
\begin{proof}
    The proof goes as follows:
    \begin{align}
        \seminorm[ \opSet_{\mathrm{MITE}} ]{1}{ \hat{\sigma} - \hat{\rho}_{\mathrm{th}} }
        &= \max_{ \hat{A} \in \opSet_{\mathrm{MITE}} } \abs{ \tr\qty( \frac{ \hat{A} }{ \opRange[\infty]{ \hat{A} } } (\hat{\sigma} - \hat{\rho}_{\mathrm{th}}) ) } \nonumber \\
        % 
        &= \max_{\subSystem\colon \mathop{\mathrm{Diam}}(\subSystem) \leq l_{0}} 
        \max_{ \hat{A}_{S} \in \mathcal{L}(\mathcal{H}_{S}) } \abs{ \tr_{S}\qty( \frac{ \hat{A}_{S} }{ \opRange[\infty]{ \hat{A}_{S} } } \qty\Big( \tr_{S^{c}}(\hat{\sigma}) - \tr_{S^{c}}(\hat{\rho}_{\mathrm{th}}) ) ) } \nonumber \\
        % 
        &= \max_{\subSystem\colon \mathop{\mathrm{Diam}}(\subSystem) \leq l_{0}} 
        \norm{ \tr_{S^{c}}(\hat{\sigma}) - \tr_{S^{c}}(\hat{\rho}_{\mathrm{th}}) }_{1}.
    \end{align}
\end{proof}

Mori \textit{et al.}~\citeSM{mori2018thermalizationSM} extend the notion of MITE by replacing the spatial constraint $\mathrm{Diam}(\subSystem) \leq l_{0}$ with a \enquote{few-body} constraint as $\abs*{\subSystem} = k$ with an integer $k$ of $\order{1}$, i.e., they consider 
\begin{equation}
    \opSet_{\mathrm{MITE}}^{(\mathrm{few})} = \bigcup_{\subSystem\colon \abs*{\subSystem} = k} \mathcal{L}(\mathcal{H}_{\subSystem}) \otimes \mathrm{id}_{\subSystem^{c}}
\end{equation}
in addition to $\opSet_{\mathrm{MITE}}$.
% 
The same proof for Proposition~\ref{prop_RelationToMITE} applies to the MITE with respect to $\opSet_{\mathrm{MITE}}^{(\mathrm{few})}$, and we have the following proposition:
\begin{framedProposition}{}{}
    Let $\hat{\sigma}$ be an arbitrary quantum state and $\hat{\rho}_{\mathrm{th}}$ be a thermal ensemble.
    Then, 
    \begin{equation}
        \text{$\hat{\sigma}$ is in MITE with respect to $\opSet_{\mathrm{MITE}}^{(\mathrm{few})}$} 
        \iff 
        \seminorm[\opSet_{\mathrm{MITE}}^{(\mathrm{few})}]{1}{ \hat{\sigma} - \hat{\rho}_{\mathrm{th}}} 
        \leq \epsilon,
    \end{equation}
    where $\epsilon\ll 1$.
\end{framedProposition}

\subsection{Macroscopic thermal equilibrium~(MATE)~\citeSM{Goldstein2015SM,Goldstein2017SM,mori2018thermalizationSM}}
% \subsection[Macroscopic thermal equilibrium (MATE)]{Macroscopic thermal equilibrium~(MATE)}
The notion of macroscopic thermal equilibrium~(MATE) is introduced in Refs.~\citeSM{Goldstein2015SM, Goldstein2017SM} as follows.
\begin{framedDefinition}{Macroscopic thermal equilibrium~(MATE)~\citeSM{Goldstein2017SM}}{}
    We consider a collection of macro observables $\hat{M}_{1},\cdots, \hat{M}_{K}$.
    By suitably coarse-graining the operators $\hat{M}_{1},\cdots, \hat{M}_{K}$, 
    we expect to obtain a set of mutually commuting operators $\tilde{M}_{1},\cdots, \tilde{M}_{K}$ with $\tilde{M}_{j} \approx \hat{M}_{j}$ for all $j=1,\cdots, K$.
    We take $\tilde{M}_{1}$ as the coarse-grained Hamiltonian, whose eigenspaces are energy shells $\mathcal{H}_{E,\Delta E}$.

    Since $\tilde{M}_{1},\cdots, \tilde{M}_{K}$ commute with each other, we can simultaneously diagonalize them, and the energy shell $\mathcal{H}_{E,\Delta E}$ can be decomposed accordingly as $\mathcal{H}_{E,\Delta E} = \bigoplus_{\nu} \mathcal{H}_{\nu}$.
    Here, $\mathcal{H}_{\nu}$ is called macro-spaces, and we denote the projector onto $\mathcal{H}_{\nu}$ by $\hat{\Pi}_{\nu}$.

    In each $\mathcal{H}_{E,\Delta E}$, it is expected that one macro-space called thermal equilibrium macro-space $\mathcal{H}_{\mathrm{eq}}$ covers the most of the dimensions of $\mathcal{H}_{E,\Delta E}$, i.e.,
    \begin{equation}
        \frac{ \dim \mathcal{H}_{\mathrm{eq}} }{ \dim \mathcal{H}_{E,\Delta E} } = 1 - \tilde{\epsilon},
    \end{equation}
    where $\tilde{\epsilon} \ll 1$.
    Without loss of generality, we set $\mathcal{H}_{\mathrm{eq}} = \mathcal{H}_{\nu=1}$.

    Under these setups, a state $\hat{\sigma} \in \dim \mathcal{H}_{E,\delta E}$ is said to be in macroscopic thermal equilibrium~(MATE) if and only if
    \begin{equation}
        \tr( \hat{\sigma} \hat{\Pi}_{\mathrm{eq}} ) \geq 1 - \delta
    \end{equation}
    for a sufficiently small tolerance $\delta > 0$.
\end{framedDefinition}

As mentioned in Ref.~\citeSM{Goldstein2017SM}, MATE can be regarded as the thermal equilibrium relative to the (coarse-grained) macroscopic observables $\tilde{M}_{1},\cdots, \tilde{M}_{K}$.
Because we focus on the joint distribution of the observed values of $\tilde{M}_{1},\cdots, \tilde{M}_{K}$ in MATE, $\opSet$ for MATE is given by
\begin{equation}
    \opSet_{\mathrm{MATE}} = \Bqty*{ \hat{\Pi}_{\mathrm{eq}} }.
    \label{eq:OpSetForMATE}
\end{equation}
% 
Then, we have the following proposition.
\begin{framedProposition}{}{}
    Let $\hat{\sigma}$ be an arbitrary quantum state.
    Then, 
    \begin{equation}
        \text{$\hat{\sigma}$ is in MATE with tolerance $\delta \, (\geq 2\tilde{\epsilon})$}.
        \iff 
        \seminorm[\opSet_{\mathrm{MATE}}]{1}{ \hat{\sigma}- \MCE[\delta E] } \leq \delta - \tilde{\epsilon}.
    \end{equation}
\end{framedProposition}
\begin{proof}
    For $\opSet_{\mathrm{MATE}} = \Bqty*{ \hat{\Pi}_{\mathrm{eq}} }$, we have
    \begin{align}
        \seminorm[\opSet_{\mathrm{MATE}}]{1}{ \hat{\sigma}- \MCE[\delta E] } 
        &= \abs{ \tr( \hat{\sigma} \hat{\Pi}_{\mathrm{eq}} ) - \frac{ \dim \mathcal{H}_{\mathrm{eq}} }{ \dim \mathcal{H}_{E,\Delta E} } }.
    \end{align}
    Therefore, inequality $\seminorm[\opSet_{\mathrm{MATE}}]{1}{ \hat{\sigma}- \MCE[\delta E] }  \leq \epsilon$ is equivalent to
    \begin{equation}
        1-(\tilde{\epsilon}+\epsilon) 
        \leq \tr( \hat{\sigma} \hat{\Pi}_{\mathrm{eq}} )
        \leq 1+(\epsilon-\tilde{\epsilon}).
    \end{equation}
    Setting $\epsilon \coloneqq \delta - \tilde{\epsilon}\ (\geq \tilde{\epsilon})$, we obtain
    \begin{equation}
        \seminorm[\opSet_{\mathrm{MATE}}]{1}{ \hat{\sigma}- \MCE[\delta E] }
        \leq \delta - \tilde{\epsilon}
        \iff 1 - \delta \leq \tr( \hat{\sigma} \hat{\Pi}_{\mathrm{eq}} ),
    \end{equation}
    which is the desired result.
\end{proof}

Apart from quantum thermalization, the (semi-)norm $\seminorm{1}{\cdot}$ introduced in Eq.~\eqref{def_pseudoDistance1} in the main text serves as the measure of the closeness between two quantum states $\hat{\sigma}$ and $\hat{\rho}$ relative to $\opSet$.
Thus, it can be used in various situations other than quantum thermal equilibrium and the ETH, such as a comparison between the state during time evolution and the steady state.
In particular, it can be used to construct the space of macroscopic states from that of quantum states by identifying quantum states that are very close to each other in terms of $\seminorm{1}{\cdot}$.
It merits further study to formulate the correspondence between microscopic and macroscopic states in this direction, thereby deriving the macroscopic dynamics from the microscopic one.

%%%%%%%%%%%%%%%%%%%%%%%%%%%%%%%%%%%%%%%%%%%%%%%%%%%%%%%%%%%%%%%%%%%%%%
%%%%%%%%%%%%%%%%%%%%%%%%%%%%%%%%%%%%%%%%%%%%%%%%%%%%%%%%%%%%%%%%%%%%%%
% \clearpage
\section{Linear independence of fermionic $m$-body operators}
\label{sect_LinearIndependence}
In this section, we show the linear independence of the basis operators 
\begin{equation}
    \hat{f}_{\vb{x}}^{\dagger} \hat{f}_{\vb{y}} 
    \coloneqq \hat{\Pi}_{\particleNumber,\sysSize} \qty( \hat{f}_{x_{m}}^{\dagger} \cdots \hat{f}_{x_{1}}^{\dagger} \hat{f}_{y_{1}} \cdots \hat{f}_{y_{m}} ) \hat{\Pi}_{\particleNumber,\sysSize}\qc
    \qty(
    \begin{aligned}
        1 \leq x_{1} < \cdots < x_{m} \leq \sysSize \\[0px]
        1 \leq y_{1} < \cdots < y_{m} \leq \sysSize
    \end{aligned}
    )
\end{equation}
of the fermionic $m$-body operator space, where $\vb{x} \coloneqq \Bqty*{ x_{1}, \cdots, x_{m} }$ and $\vb{y} \coloneqq \Bqty*{ y_{1}, \cdots, y_{m} }$.
We decompose $\vb{x}$ and $\vb{y}$ as $\vb{x} = X\sqcup Z$ and $\vb{y} = Y\sqcup Z$, where \enquote{$\sqcup$} denotes the disjoint union, $Z \coloneqq \vb{x} \cap \vb{y}$, $X \coloneqq \vb{x}\setminus Z$, and $Y \coloneqq \vb{y} \setminus Z$.
Accordingly, we rearrange the product $\hat{f}_{y_{1}} \cdots \hat{f}_{y_{m}}$ by introducing
\begin{equation}
    \hat{f}_{Y\sqcup Z} \coloneqq \qty( \prod_{x\in Y} \hat{f}_{x} ) \qty( \prod_{x\in Z} \hat{f}_{x} ),
\end{equation}
where the product $\prod$ of the annihilation operators $\hat{f}_{x}$ is arranged in ascending order in $x$ within each pair of parentheses.

To show the linear independence of the operators 
$
    \Bqty{
        \hat{f}_{\vb{x}}^{\dagger} \hat{f}_{\vb{y}}
        \mid 
        \substack{
            1 \leq x_{1} < \cdots < x_{m} \leq \sysSize,\ \\[1ex] %\\
            1 \leq y_{1} < \cdots < y_{m} \leq \sysSize 
        }
    },
$
we consider the equation
\begin{align}
    0 = 
    \sum_{ \substack{ X,Y \\ X\cap Y = \emptyset \\ \abs*{X} = \abs*{Y} \leq \min\Bqty{ m, \sysSize/2 } }} 
    \sum_{\substack{ Z \\ \abs*{Z} = m - \abs*{X} \\ Z\cap X = \emptyset, Z\cap Y = \emptyset }}
    C_{X,Y}(Z) ( \hat{f}_{X\sqcup Z} )^{\dagger} \hat{f}_{Y\sqcup Z}
    \quad \qty( \eqqcolon \hat{A} ).
    \label{eqS_LinearIndependence}
\end{align}

Given $X$ and $Y$, we label the basis vectors of $\mathcal{H}_{\particleNumber, \sysSize}$ as
\begin{equation}
    \ket*{ \vb{n}_{X}, \vb{n}_{Y}, \vb{n}_{ \vb{V} \setminus (X\sqcup Y) } } 
    \coloneqq \qty( \prod_{ x\in X } (\hat{f}_{x} )^{n_{x}} )^{\dagger} \qty( \prod_{ y\in Y } (\hat{f}_{x} )^{n_{x}} )^{\dagger} \qty( \prod_{ x\in \vb{V} \setminus (X\sqcup Y) } (\hat{f}_{x} )^{n_{x}} )^{\dagger} \ket*{0},
\end{equation}
where $\vb{n}_{W} \coloneqq \Bqty*{ n_{x} \in \Bqty*{0, 1} \mid x\in W }$ with $W$ being an arbitrary subset of $\vb{V} \coloneqq \Bqty*{ 1,\cdots, V }$.
% 
Then, taking the matrix element of the right-hand side of Eq.~\eqref{eqS_LinearIndependence}, we obtain
\begin{align}
    &\quad \matrixel*{ \vb{n}_{X} = \vb{1}, \vb{n}_{Y} = \vb{0}, \vb{n}_{ \vb{V} \setminus (X\sqcup Y) } }{ \hat{A} }{ \vb{n}_{X} = \vb{0}, \vb{n}_{Y} = \vb{1}, \vb{n}_{ \vb{V} \setminus (X\sqcup Y) } }
    \nonumber \\
    % 
    &= \sum_{Z} C_{X,Y}(Z) \matrixel{ \vb{n}_{X} = \vb{1}, \vb{n}_{Y} = \vb{0}, \vb{n}_{ \vb{V} \setminus (X\sqcup Y) } }{ \hat{f}_{X\sqcup Z}^{\dagger} \hat{f}_{Y\sqcup Z} }{ \vb{n}_{X} = \vb{0}, \vb{n}_{Y} = \vb{1}, \vb{n}_{ \vb{V} \setminus (X\sqcup Y) } }
    \nonumber \\
    % 
    &= \sum_{Z} C_{X,Y}(Z) \matrixel*{ \vb{n}_{ \vb{V} \setminus (X\sqcup Y) } }{ \hat{f}_{Z}^{\dagger} \hat{f}_{Z} }{ \vb{n}_{ \vb{V} \setminus (X\sqcup Y) } }
    \nonumber \\
    % 
    &= \sum_{Z} C_{X,Y}(Z) \chi\qty( \forall x\in Z,\ n_{x} = 1 ),
\end{align}
where $\chi(\phi)$ is the indicator function that takes on unity if the proposition $\phi$ is true and zero otherwise.
Therefore, the equation~\eqref{eqS_LinearIndependence} implies
\begin{equation}
    \sum_{\substack{ Z \\ \abs*{Z} = m - \abs*{X} \\ Z\cap X = \emptyset,\, Z\cap Y = \emptyset }} C_{X,Y}(Z) \chi\qty( \forall x\in Z,\ n_{x} = 1 ) = 0,
    \label{eqS_LinearIndependence2}
\end{equation}
which holds for all disjoint subsets $X, Y \subset \vb{V}$ and all configurations $\vb{n}_{\vb{V} \setminus (X \sqcup Y)}$ such that:
\begin{equation}
    X\cap Y = \emptyset\qc
    \abs*{X} = \abs*{Y} \leq m\qc
    \sum_{ x \in \vb{V}\setminus (X\sqcup Y) } n_{x} = \particleNumber - \abs*{X}.
\end{equation}
% for all $X$, $Y$, and $\vb{n}_{ \vb{V}\setminus (X\sqcup Y) }$ with $\abs*{X} = \abs*{Y} \leq m$, $X\cap Y = \emptyset$, and $\sum_{ x \in \vb{V}\setminus (X\sqcup Y) } n_{x} = \particleNumber - \abs*{Y}$.

Here, the range of the summation over $Z$ in Eq.~\eqref{eqS_LinearIndependence2} depends only on $X$ and $Y$ and is independent of $\vb{n}_{ \vb{V} \setminus (X\sqcup Y) }$, and the number of the summation over $Z$ is given by
\begin{equation}
    \binom{ \sysSize - \abs*{X} - \abs*{Y} }{ \abs*{Z} } = \binom{ \sysSize - 2\abs*{X} }{ m - \abs*{X} }.
    \label{eqS_NumberOfCoeffs}
\end{equation}
On the other hand, the number of choices for $\vb{n}_{ \vb{V} \setminus (X\sqcup Y) }$ in Eq.~\eqref{eqS_LinearIndependence2} is 
\begin{equation}
    \binom{ \sysSize - 2\abs*{X} }{ \particleNumber - \abs*{X} }
    \label{eqS_NumberOfEqs}
\end{equation}
for given $X$ and $Y$.
When $\particleNumber \leq \sysSize/2$, we have
\begin{equation}
    m - \abs*{X} \leq \particleNumber - \abs*{X} \leq \frac{\sysSize}{2} - \abs*{X} = \frac{ \sysSize - 2\abs*{X} }{2},
\end{equation}
and therefore
\begin{equation}
    \binom{ \sysSize - 2\abs*{X} }{ m - \abs*{X} } \leq \binom{ \sysSize - 2\abs*{X} }{ \particleNumber - \abs*{X} }
\end{equation}
for any $m\leq \particleNumber$.
This means that the number of equations~\eqref{eqS_LinearIndependence2} for given $X$ and $Y$ is equal to or larger than the number of the coefficients $\Bqty*{ C_{X, Y}(Z) }_{Z}$.
Therefore, Eq.~\eqref{eqS_LinearIndependence} implies $C_{X,Y}(Z) = 0$ for all $X,Y,Z$ when $\particleNumber \leq \sysSize/2$.
Thus, the linear independence of the operators 
$
    \Bqty{
        \hat{f}_{\vb{x}}^{\dagger} \hat{f}_{\vb{y}}
        \mid 
        % \substack{
            1 \leq x_{1} < \cdots < x_{m} \leq \sysSize,\ %\\
            1 \leq y_{1} < \cdots < y_{m} \leq \sysSize 
        % }
    }
$
is proved.

On the other hand, when $N > \sysSize/2$, we have
\begin{equation}
    \sysSize - 2\abs*{X} - (\particleNumber - \abs*{X}) = \sysSize - \particleNumber - \abs*{X} \leq \frac{ \sysSize - 2\abs*{X} }{ 2 }.
\end{equation}
Therefore, for $m > \sysSize-\particleNumber$, the number of equations ~\eqref{eqS_LinearIndependence2} for given $X$ and $Y$ is less than the number of the coefficients $\Bqty*{ C_{X,Y}(Z) }_{Z}$, and the operators in
$
    \Bqty{
        \hat{f}_{\vb{x}}^{\dagger} \hat{f}_{\vb{y}}
        \mid 
        % \substack{
            1 \leq x_{1} < \cdots < x_{m} \leq \sysSize, %\\
            1 \leq y_{1} < \cdots < y_{m} \leq \sysSize 
        % }
    }
$
can be linearly dependent.

%%%%%%%%%%%%%%%%%%%%%%%%%%%%%%%%%%%%%%%%%%%%%%%%%%%%%%%%%%%%%%%%%%%%%%
%%%%%%%%%%%%%%%%%%%%%%%%%%%%%%%%%%%%%%%%%%%%%%%%%%%%%%%%%%%%%%%%%%%%%%
\clearpage
\section{Bounds $\sqrt{D} \Lambda_{2}^{[0,m]}$ and $\Lambda_{2}^{[0,m]}$ at various energies}
Figure~\ref{fig_Numerics_ShortRangeEnsemble} in the main text presents the numerical results for the upper bound $\sqrt{D} \Lambda_{2}^{[0,m]}$ of the diagonal ETH measure at the center of the energy spectrum, $E_{x} = E_{0.5} = (E_{\max} + E_{\min})/2$, where $\Lambda_{2}^{[0,m]}$ is defined in the main text.
Figure~\ref{fig_BoundsEnergyDependence} in the main text shows the resulting lower and upper bounds on the threshold $\alpha_{\ast}$, $\lowerBound{} \leq \alpha_{\ast} \leq \upperBound{}$ across the energy spectrum.
These bounds are derived from the results for $\sqrt{D} \Lambda_{2}^{[0,m]}$ and $\Lambda_{2}^{[0,m]}$ at the corresponding energies.
These results are presented here in Fig.~\ref{figS_UpperBound_EnergyDeps}.

\vspace{1cm}

{
    \centering
    \includegraphics[width=\linewidth]{FigureS1_Bounds_vsM_VariousEnergy.pdf}
    \captionof{figure}{\textbf{Upper and lower bounds on the (diagonal) ETH measure for varying the normalized energy in locally interacting systems.}
    The gray solid vertical line indicates the point $\alpha_{\mathrm{L/U}}{(x)}$ where the maximum difference of the curves for different system sizes is minimized.
    The blue shaded region shows where the $N$-dependence of the bounds ($\mathbb{E}_{\mathrm{log}}[ \sqrt{D} \Lambda_{2}^{[0,\alpha \particleNumber]} ]$ and $\mathbb{E}_{\mathrm{log}}[ \Lambda_{2}^{[0,\alpha \particleNumber]} ]$) qualitatively changes from decreasing~(left) to increasing~(right), indicating the uncertainty of $\lowerBound{}$ and $\upperBound{}$ due to limited system sizes.
    The gray dashed curves and vertical lines show the results for systems with Haar-distributed energy eigenstates.
    % 
    The intersection point $\lowerBound{(x)}$ of $\mathbb{E}_{\mathrm{log}}[ \sqrt{D} \Lambda_{2}^{[0,\alpha \particleNumber]} ]$ is observed for non-small $x$ but not observed for small $x$ ($x \lesssim 0.25$ for spin and Fermi systems and $x \lesssim 0.3$ for Bose systems).
    % 
    The intersection point $\upperBound{(x)}$ of $\mathbb{E}_{\mathrm{log}}[ \Lambda_{2}^{[0,\alpha \particleNumber]} ]$ is also observed for all values of $x$ shown in the figure.
    }
    \label{figS_UpperBound_EnergyDeps}
}

%%%%%%%%%%%%%%%%%%%%%%%%%%%%%%%%%%%%%%%%%%%%%%%%%%%%%%%%%%%%%%%%%%%%%%
%%%%%%%%%%%%%%%%%%%%%%%%%%%%%%%%%%%%%%%%%%%%%%%%%%%%%%%%%%%%%%%%%%%%%%
\clearpage
\section{Numerical results for concrete physical models}
In this section, we show the numerical results for the following prototypical models of nonintegrable systems:
\begin{itemize}
    \item Spin: We consider the mixed-field Ising model, whose Hamiltonian is given by
    \begin{equation}
        \hat{H} = J_{x} \sum_{j=1}^{L} \hat{\sigma}_{j}^{(1)} \hat{\sigma}_{j+1}^{(1)} +B_{z} \sum_{j=1}^{L} \hat{\sigma}_{j}^{(3)} +B_{x} \sum_{j=1}^{L} \hat{\sigma}_{j}^{(1)},
        \label{eqS_MixedFieldIsingModel}
    \end{equation}
    where $\hat{\sigma}_{j}^{(p)} \ (p = 1,2,3)$ are Pauli operators acting on site $j$.
    This model is numerically verified to satisfy the ETH in Ref.~\citeSM{kim2014testingSM} for the parameter $(J_{x}, B_{z}, B_{x}) = (1, 0.9045, 0.8090)$.
    % 
    \item Boson: We consider the Bose-Hubbard model, whose Hamiltonian is given by
    \begin{align}
        \hat{H} 
        &= -J \sum_{x=1}^{L} ( \hat{b}_{x}^{\dagger} \hat{b}_{x+1} + \hat{b}_{x+1}^{\dagger} \hat{b}_{x} ) -U \sum_{x=1}^{L} \frac{ \hat{n}_{x} (\hat{n}_{x} - 1) }{2}.
        \label{eqS_BoseHubbard}
    \end{align}
    This model at unit filling ($\particleNumber/\sysSize = 1$) is numerically verified to satisfy the ETH in Ref.~\citeSM{biroli2010effectSM} for the parameter $(J,U) = (1,1)$, where it is known to be nonintegrable~\citeSM{Kolovsky2004-ooSM}.
    We use these parameters in the numerical calculations.
    % 
     \item Fermion: We consider the spinless fermions with next-nearest hopping and interactions:
    \begin{align}
    \hat{H} 
    &= -t_{1} \sum_{x=1}^{L} ( \hat{f}_{x}^{\dagger} \hat{f}_{x+1} + \hat{f}_{x+1}^{\dagger} \hat{f}_{x} ) -J_{1} \sum_{x=1}^{L} \hat{n}_{x} \hat{n}_{x+1}
    \nonumber \\
    % 
    &\qquad 
    -t_{2} \sum_{x=1}^{L} ( f_{x}^{\dagger}f_{x+2} + f_{x+2}^{\dagger} f_{x} ) -J_{1} \sum_{x=1}^{L} \hat{n}_{x} \hat{n}_{x+2}.
    \label{eqS_FermiHubbardNNN}
    \end{align}
    This model is numerically verified to exhibit quantum chaotic behavior in Ref.~\citeSM{Santos2010-qsSM} for $(t_{1}, J_{1}, t_{2}, J_{2}) = (1, 1, 0.32, 0.32)$ and $\particleNumber/\sysSize = 1/3$.
    In our numerical calculation, we use these values of $(t_{1}, J_{1}, t_{2}, J_{2})$ but set $\particleNumber/\sysSize = 1/2$.
\end{itemize}
We adopt periodic boundary conditions in all the models listed above.
Since these systems have translation and reflection symmetries, we focus on the zero-momentum and even-parity sector.

Figure~\ref{fig_ConcreteModels_UpperBoundVsM} shows the upper bound $\sqrt{D} \Lambda_{2}^{[0,m]}$ on the (diagonal) ETH measure at the center of the energy spectrum as a function of $m/\particleNumber$.
We observe that some of the curves with different system sizes intersect with each other.
However, unlike the case for generic locally interacting systems shown in Fig.~\ref{figS_UpperBound_EnergyDeps}, the point where curves for various system sizes almost intersect is not observed for the nonintegrable models in Eqs.~\eqref{eqS_MixedFieldIsingModel}--\eqref{eqS_FermiHubbardNNN}.
Thus, it remains to be an open question whether the ratio $m_{\ast}/\particleNumber$ converges to a finite value or vanishes in the thermodynamic limit for these models.

It is interesting that the finite-size behaviors of $\sqrt{D} \Lambda_{2}^{[0,m]}$ are different in the nonintegrable models and the generic locally interacting systems presented in the methods section of the main text, even though these systems share the essential physical structure of the locality and the few-body nature of interactions.
It is an important future problem to investigate the underlying causes of these differences. 


Figure~\ref{fig_ConcreteModels_UpperBoundVsL} shows the same data as in Fig.~\ref{fig_ConcreteModels_UpperBoundVsM} as a function of the system size.
For the mixed-field Ising model~\eqref{eqS_MixedFieldIsingModel} and the spinless fermion model~\eqref{eqS_FermiHubbardNNN}, the upper bound $\sqrt{D} \Lambda_{2}^{[0,m]}$ with $m\in[1,4]$ decreases for the three largest system sizes.
However, the number of these data points is too small, and the $\sysSize$-dependence of $\sqrt{D} \Lambda_{2}^{[0,m]}$ is not smooth for the system sizes of the numerical calculation.
% 
For the Bose-Hubbard model~\eqref{eqS_BoseHubbard}, the upper bound $\sqrt{D} \Lambda_{2}^{[0,m]}$ with $m=1$ almost monotonically decreases for $\sysSize \geq 6$.
However, the curves for $m=2,3$ increase as $\sysSize$ increases from $10$ to $11$, which are the two largest system sizes in our calculation.
% 
These facts indicate that the finite-size effects are nonnegligible in determining the behavior of $\sqrt{D} \Lambda_{2}^{[0,m]}$ for large $\sysSize$ in the nonintegrable models in Eqs.~\eqref{eqS_MixedFieldIsingModel}--\eqref{eqS_FermiHubbardNNN}.
% 


% a better upper bound on the ETH measure is needed to determine the existence of the lower bound on the ratio $m_{\ast}/\particleNumber$ in concrete physical models.
% Finding such improved bounds will be another future direction.

The large finite-size effects of $\sqrt{D} \Lambda_{2}^{[0,m]}$ in Figs.~\ref{fig_ConcreteModels_UpperBoundVsM} and \ref{fig_ConcreteModels_UpperBoundVsL} may suggest that the inequality $\seminorm{1}{\cdot} \leq \sqrt{D} \seminorm{2}{\cdot}$ is too loose to test Conjecture~\ref{conj_Conjecture} in the main text for those concrete physical models.
It remains to be an important future task to validate (or invalidate) Conjecture~\ref{conj_Conjecture} for prototypical nonintegrable systems, e.g., by finding better bounds on the ETH measure.
% 
The inequality $\seminorm{2}{\cdot} \leq \seminorm{1}{\cdot}$ may also be too loose because it fails to provide $\upperBound{} < 0.5$ for spin and Bose systems with Haar-distributed energy eigenstates.
For Fermi systems with Haar-distributed energy eigenstates, it gives a better estimate on $\upperBound$ in the high-density region $(n \coloneqq N/V > 2/3)$ as $\upperBound = (1-n)/n < 1/2$, but does not in the other region with $n < 2/3$.
The bound $\upperBound{} < 0.5$ is expected from a plausible argument based on the comparison of the reduced density operators of energy eigenstates and a thermal ensemble on a subsystem~\citeSM{garrison2018doesSM}.
Therefore, it remains to be a future problem to improve the value of $\upperBound{}$ by, for example, finding a better lower bound on $\seminorm{1}{\cdot}$ or quantifying quantum correlations associated with the decomposition of the operator space into accessible part $\opSet{}$ and inaccessible part.
% 
% , we believe that the ETH typically breaks down for $\opSet_{\particleNumber}^{[0,\particleNumber/2]}$, i.e., we expect $m_{\ast} / \particleNumber \leq 1/2$.
% % 
% For systems with Haar-distributed energy eigenstates, our upper bound gives a better upper bound for high-density Fermi systems with $n \coloneqq N/V > 2/3$ as $m_{\ast} / \particleNumber \leq \upperBound = (1-n)/n < 1/2$.
% 
% However, for spin and Bose systems, our method does not provide $\upperBound{} < 0.5$.
% This fact indicates the inequality $\seminorm{2}{\cdot} \leq \seminorm{1}{\cdot}$ is also too loose.


% \clearpage
{
    \centering
    \includegraphics[width=\linewidth]{FigureS2_ConcreteModels_UpperBoundVsM.pdf}
    \captionof{figure}{\textbf{Upper bound on the (diagonal) ETH measure as a function of $m/\particleNumber$ for the nonintegrable models in Eqs.~\eqref{eqS_MixedFieldIsingModel}--\eqref{eqS_FermiHubbardNNN}.}
    The upper bounds are calculated at the center of the energy spectrum.
    While some of the curves intersect with each other, they do not intersect at a single point within the system sizes of the numerical calculation.
    }
    \label{fig_ConcreteModels_UpperBoundVsM}
}\vspace{1cm}
{
    \centering
    \includegraphics[width=\linewidth]{FigureS3_ConcreteModels_UpperBoundVsL.pdf}
    \captionof{figure}{\textbf{Upper bound on the (diagonal) ETH measure as a function of $\sysSize$ for the nonintegrable models in Eqs.~\eqref{eqS_MixedFieldIsingModel}--\eqref{eqS_FermiHubbardNNN}.}
    The same data as in Fig.~\ref{fig_ConcreteModels_UpperBoundVsM} are shown as a function of the system size.
    Different colors indicate different values of $m$ ranging from $1$ to $\particleNumber$.
    The upper bound $\sqrt{D} \Lambda_{2}^{[0,m]}$ with $m\in[1,4]$ decreases as $\sysSize$ increases in the region $\sysSize \geq 16$ for the spin system and $\sysSize \geq 9$ for the spinless fermions.
    However, the number of data points in these regions is too small, and the $\sysSize$-dependence of $\sqrt{D} \Lambda_{2}^{[0,m]}$ is not smooth.
    % 
    For the boson system, the upper bound $\sqrt{D} \Lambda_{2}^{[0,m]}$ with $m=1$ smoothly decreases for $\sysSize \geq 6$.
    However, the curves for $m=2,3$ increase as $\sysSize$ increases from $10$ to $11$, which are the two largest system sizes in our calculation.
    % 
    These facts indicate that the finite-size effects are nonnegligible in determining the behavior of $\sqrt{D} \Lambda_{2}^{[0,m]}$ for large $\sysSize$ in the nonintegrable models in Eqs.~\eqref{eqS_MixedFieldIsingModel}--\eqref{eqS_FermiHubbardNNN}.
    }
    \label{fig_ConcreteModels_UpperBoundVsL}
}

% \clearpage
\bibliographySM{supplement}

\end{document}