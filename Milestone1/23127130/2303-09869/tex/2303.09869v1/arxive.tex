\documentclass[aps,pra,twocolumn,superscriptaddress]{revtex4-1}

\usepackage[utf8]{inputenc} % Required for non-English characters (input)
\usepackage[T1]{fontenc} % Required for non-English characters (output)
\usepackage[english]{babel}
\usepackage[autostyle=true]{csquotes} % Generate language-dependent quotes in the bibliography
\usepackage[dvipsnames]{xcolor}
\usepackage{graphicx}% Include figure files
\usepackage{array}
\usepackage{amsmath} % math formulas
\usepackage{amssymb} % math symbols
\usepackage{amsfonts} % math basic fonts
\usepackage{amsbsy}
\usepackage{mathrsfs} % math RSFS fonts (e.g. \mathscr)
\usepackage{dsfont} % math double strike (e.g. integer set, unity matrix with \mathbb)
\usepackage{dcolumn} % Align table columns on decimal point
\usepackage{upgreek} % Upgreek symbols (e.g. \Upbeta)
\usepackage{bm} % bold math
\usepackage{bbm}
\usepackage{placeins} % Control floats (e.g. \FloatBarrier)
\usepackage{physics} % Include bra and ket

\usepackage[colorlinks,linkcolor=blue,citecolor=blue,urlcolor=blue]{hyperref}
\usepackage[most]{tcolorbox}
\usepackage{subfigure}
\usepackage{float}
%\usepackage{stix}
\usepackage[sort&compress]{natbib}
\usepackage[dvips]{epsfig}
\usepackage{hhline}
\usepackage{multirow}
%
%Colors
\newcommand{\note}[1]{\textcolor{RoyalBlue}{#1}}
\newcommand{\tbc}[1]{\textcolor{Magenta}{#1}}
\newcommand{\amend}[1]{\textcolor{BrickRed}{#1}}
\newcommand{\todo}[1]{\textcolor{Green}{#1}}

% Acronyms
\newcommand{\dfpt}{DFPT}
\newcommand{\tddfpt}{TD--DFPT}
\def\ppe        {P\&p\ }
\def\ai	        {{\em ab--initio}}
\def\MBPT       {many-body perturbation theory}
\def\MB         {many-body}
\newcommand{\qe} {{\normalfont\ttfamily Quantum Espresso}}
\newcommand{\abinit} {{\normalfont\ttfamily Abinit}}
\newcommand{\yambo} {{\normalfont\ttfamily yambo}}
\newcommand{\Yambo} {{\normalfont\ttfamily Yambo}}
\def\GS         {ground state}
%
% Parenthesis and Math symbols
\renewcommand{\[}{\left[}
\renewcommand{\]}{\right]}
\renewcommand{\(}{\left(}
\renewcommand{\)}{\right)}
\def\nl         {\right.\\ \left.}
\def\la         {\left\langle}
\def\ra         {\right\rangle}
\def\dg         {\dagger}
%
% Arrows
\def\dn         {\downarrow}
\def\up         {\uparrow}
\def\lar        {\leftarrow}  
\def\rar        {\rightarrow}  
\def\Lal        {\Leftarrow}
\def\Rar        {\Rightarrow}
\def\lrar       {\Longrightarrow}  
\def\llar       {\Longleftarrow}  
\def\lrar       {\leftrightarrow}  
\def\llrar      {\Longleftrightarrow}  
\newcommand{\olrar}[1]{\overleftrightarrow{#1}}
\newcommand{\orar}[1]{\overrightarrow{#1}}
\newcommand{\olar}[1]{\overleftarrow{#1}}
\newcommand{\tnsrleft}[1] {\overset\leftarrow{#1}}
\newcommand{\tnsrright}[1] {\overset\rightarrow{#1}}
\newcommand{\tnsr}[1] {\overset\leftrightarrow{#1}}
%
% Equations
\newcommand{\eq}[1]{\begin{align}#1\end{align}}
\newcommand{\ml}[1]{\begin{multline}#1\end{multline}}
\newcommand{\eqg}[1]{\begin{gather}#1\end{gather}}
\newcommand{\seq}[1]{\begin{subequations}#1\end{subequations}}
\newcommand{\sst}[2]{\substack{#1\\#2}}
% ... + label
\newcommand{\seql}[2]{\begin{subequations}\label{#1}#2\end{subequations}}
\newcommand{\mll}[2]{\begin{multline}\label{#1}#2\end{multline}}
\newcommand{\eql}[2]{\begin{align}\label{#1}#2\end{align}}
\newcommand{\eqgl}[2]{\seq{\label{#1}\begin{gather}#2\end{gather}}}
%
% Functions
\newcommand\llim[2]{#1\xrightarrow[#2]{}}
\newcommand\diag[2]{\ncline[linewidth=1pt,nodesep=-30pt]{#1}{#2}}
\newcommand{\stkout}[1]{\ifmmode\text{\sout{\ensuremath{#1}}}\else\sout{#1}\fi}
%\newcommand{\braket}[2]{\left\langle #1 \right|\left. #2 \right\rangle}
%\newcommand{\bra}[1]{\mbox{$\langle #1 |$}}
%\newcommand{\ket}[1]{\mbox{$| #1 \rangle$}}
\newcommand{\average}[1]{\left\langle #1 \right\rangle}
\newcommand{\pr}[1]{\left( #1 \right)}
\newcommand{\ps}[1]{\left[ #1 \right]}
\newcommand{\bps}[1]{\biggl[ #1 \biggr]}
\newcommand{\bpr}[1]{\biggl( #1 \biggr)}
\newcommand{\bpg}[1]{\biggl\{ #1 \biggr\}}
\newcommand{\lab}[1]{\label{#1}}
\newcommand{\bs}[1]{\boldsymbol{#1}}
\newcommand{\mc}[1]{\mathcal{#1}}
\newcommand{\ul}[1]{\underline{#1}}
\newcommand{\oo}[1]{\overline{#1}}
\newcommand{\e}[1]{Eq.~\eqref{#1}}
\newcommand{\es}[2]{Eqs.~\eqref{#1}--\eqref{#2}}
\newcommand{\elab}[2]{Eq.(\ref{#1}#2)}
\newcommand{\tab}[1]{Tab.\ref{#1}}
\newcommand{\fig}[1]{Fig.\ref{#1}}
\newcommand{\figlab}[2]{Fig.\ref{#1}#2}
\newcommand{\evalat}[2]{\left.#1\right|_{#2}}
\renewcommand{\sec}[1]{Section\,\ref{#1}}
\newcommand{\app}[1]{Appendix\,\ref{#1}}
\newcommand{\h}[1]{\hat{#1}}
\newcommand{\ti}[1]{\tilde{#1}}
\newcommand{\wh}[1]{\widehat{#1}}
\newcommand{\wt}[1]{\widetilde{#1}}
\renewcommand{\t}[1]{\olrar{#1}}
%\renewcommand{\v}[1]{\vec{#1}}
\newcommand{\dint}[1] {{\rm d}^3 #1}
\renewcommand{\b}[1]{{\bar #1}}
\newcommand{\txtbox}[1]{\fbox{\parbox{\columnwidth}{\it #1}}}
\newcommand{\mr}[1]{{\mathrm #1}}
\newcommand{\ocite}[1]{Ref.\cite{#1}}
%
% Mathematical Symbols
\def\de{\partial}
\def\grad{\mbox{\boldmath $\nabla$}}
\def\Tr{{\rm Tr}}
\def\Re{{\rm Re}}
\def\Im{{\rm Im}}
%
% Bits
\newcommand{\p}{\prime}           % Differential d
\renewcommand{\d}{\mr{d}}           % Differential d
%
%
% Greeks
\def\gk         {\kappa}
\def\gz         {\zeta}
\def\gi         {\iota}
\def\ga         {\alpha}
\def\gb         {\beta}
\def\gc         {\gamma}
\def\gC         {\Gamma}
\def\gd         {\delta}
\def\gD         {\Delta}
\def\gee        {\epsilon}
\def\gl         {\lambda}
\def\gL         {\Lambda}
\def\go         {\omega}
\def\gO         {\Omega}
\def\gr         {\rho}
\def\gs         {\sigma}
\def\gS         {\Sigma}
\def\gT         {\Theta}
\def\gt         {\theta}
\def\gu         {\tau}
\def\gp         {\phi}
\def\gps        {\psi}
%
% Bolds
\def\zero	{{\mathbf 0}}
\def\tt		{{\mathbf t}}
\def\yy		{{\mathbf y}}
\def\ss		{{\mathbf s}}
\def\vv		{{\mathbf v}}
\def\FF		{{\mathbf F}}
\def\AA		{{\mathbf A}}
\def\BB		{{\mathbf B}}
\def\GG		{{\mathbf G}}
\def\RR		{{\mathbf R}}
\def\SS		{{\mathbf S}}
\def\OO		{{\mathbf O}}
\def\TT		{{\mathbf T}}
\def\pp		{{\mathbf p}}
\def\xx		{{\mathbf x}}
\def\rr		{{\mathbf r}}
\def\PP		{{\mathbf P}}
\def\KK		{{\mathbf K}}
\def\QQ		{{\mathbf Q}}
\def\HH		{{\mathbf H}}
\def\II		{{\mathbf I}}
\def\JJ		{{\mathbf J}}
\def\uu		{{\mathbf u}}
\def\UU		{{\mathbf U}}
\def\EE		{{\mathbf E}}
\def\kk		{{\mathbf k}}
\def\qq		{{\mathbf q}}
\def\qqu	{\uu{q}}
\def\jj		{{\mathbf j}}
\def\ee         {{\mathbf e}}
\def\dd         {{\mathbf d}}
\def\Gu         {{\mathbf \tau}}
%
\newcommand{\di}{\mathrm{d}}
\newcommand{\im}{\mathrm{i}}
%
% Bold Greek letters
\def\bga{\mbox{\boldmath $\alpha$}}
\def\bgb{\mbox{\boldmath $\beta$}}
\def\bgg{\mbox{\boldmath $\gamma$}}
\def\bgG{\mbox{\boldmath $\Gamma$}}
\def\bgd{\mbox{\boldmath $\delta$}}
\def\bgD{\mbox{\boldmath $\Delta$}}
\def\bge{\mbox{\boldmath $\epsilon$}}
\def\bgve{\mbox{\boldmath $\varepsilon$}}
\def\bgh{\mbox{\boldmath $\eta$}}
\def\bgth{\mbox{\boldmath $\theta$}}
\def\bgk{\mbox{\boldmath $\kappa$}}
\def\bgl{\mbox{\boldmath $\lambda$}}
\def\bgL{\mbox{\boldmath $\Lambda$}}
\def\bgm{\mbox{\boldmath $\mu$}}
\def\bgn{\mbox{\boldmath $\nu$}}
\def\bgc{\mbox{\boldmath $\xi$}}
\def\bgC{\mbox{\boldmath $\Xi$}}
\def\bgp{\mbox{\boldmath $\pi$}}
\def\bgP{\mbox{\boldmath $\Pi$}}
\def\bgr{\mbox{\boldmath $\rho$}}
\def\bgs{\mbox{\boldmath $\sigma$}}
\def\bgS{\mbox{\boldmath $\Sigma$}}
\def\bgt{\mbox{\boldmath $\tau$}}
\def\bgf{\mbox{\boldmath $\phi$}}
\def\bgvf{\mbox{\boldmath $\varphi$}}
\def\bgF{\mbox{\boldmath $\Phi$}}
\def\bgx{\mbox{\boldmath $\chi$}}
\def\bgw{\mbox{\boldmath $\omega$}}
\def\bgW{\mbox{\boldmath $\Omega$}}
\def\bgq{\mbox{\boldmath $\psi$}}
\def\bgQ{\mbox{\boldmath $\Psi$}}
\def\bgz{\mbox{\boldmath $\zeta$}}

% Italic capital Greek Letters
\def\calgG{\mbox{$\mathit{\Gamma}$}}  
\def\calgD{\mbox{$\mathit{\Delta}$}}  
\def\calgL{\mbox{$\mathit{\Lambda}$}}  
\def\calgC{\mbox{$\mathit{\Xi}$}}  
\def\calgP{\mbox{$\mathit{\Pi}$}}  
\def\calgS{\mbox{$\mathit{\Sigma}$}} 
\def\calgF{\mbox{$\mathit{\Phi}$}} 
\def\calgW{\mbox{$\mathit{\Omega}$}} 
\def\calgQ{\mbox{$\mathit{\Psi}$}} 

% Italic bold capital Greek Letters
\def\bcalgG{\mbox{\boldmath $\mathit{\Gamma}$}}  
\def\bcalgD{\mbox{\boldmath $\mathit{\Delta}$}}  
\def\bcalgL{\mbox{\boldmath $\mathit{\Lambda}$}}  
\def\bcalgC{\mbox{\boldmath $\mathit{\Xi}$}}  
\def\bcalgP{\mbox{\boldmath $\mathit{\Pi}$}}  
\def\bcalgS{\mbox{\boldmath $\mathit{\Sigma}$}} 
\def\bcalgF{\mbox{\boldmath $\mathit{\Phi}$}} 
\def\bcalgW{\mbox{\boldmath $\mathit{\Omega}$}} 
\def\bcalgQ{\mbox{\boldmath $\mathit{\Psi}$}} 

% Bold Latin letters
\def\bla{{\mathbf a}}
\def\blb{{\mathbf b}}
\def\blc{{\mathbf c}}
\def\bld{{\mathbf d}}
\def\bldt{\tilde{{\mathbf d}}}
\def\ble{{\mathbf e}}
\def\blf{{\mathbf f}}
\def\blg{{\mathbf g}}
\def\blh{{\mathbf h}}
\def\bli{{\mathbf i}}
\def\blj{{\mathbf j}}
\def\blk{{\mathbf k}}
\def\bll{{\mathbf l}}
\def\blm{{\mathbf m}}
\def\bln{{\mathbf n}}
\def\blo{{\mathbf o}}
\def\blp{{\mathbf p}}
\def\blq{{\mathbf q}}
\def\blr{{\mathbf r}}
\def\bls{{\mathbf s}}
\def\blt{{\mathbf t}}
\def\blu{{\mathbf u}}
\def\blv{{\mathbf v}}
\def\blw{{\mathbf w}}
\def\blx{{\mathbf x}}
\def\bly{{\mathbf y}}
\def\blz{{\mathbf z}}

% Bold Calligraphic Latin letters
\def\bcalla{\mbox{\boldmath $a$}}
\def\bcallb{\mbox{\boldmath $b$}}
\def\bcallc{\mbox{\boldmath $c$}}
\def\bcalld{\mbox{\boldmath $d$}}
\def\bcalle{\mbox{\boldmath $e$}}
\def\bcallf{\mbox{\boldmath $f$}}
\def\bcallg{\mbox{\boldmath $g$}}
\def\bcallh{\mbox{\boldmath $h$}}
\def\bcalli{\mbox{\boldmath $i$}}
\def\bcallj{\mbox{\boldmath $j$}}
\def\bcallk{\mbox{\boldmath $k$}}
\def\bcalll{\mbox{\boldmath $l$}}
\def\bcallm{\mbox{\boldmath $m$}}
\def\bcalln{\mbox{\boldmath $n$}}
\def\bcallo{\mbox{\boldmath $o$}}
\def\bcallp{\mbox{\boldmath $p$}}
\def\bcallq{\mbox{\boldmath $q$}}
\def\bcallr{\mbox{\boldmath $r$}}
\def\bcalls{\mbox{\boldmath $s$}}
\def\bcallt{\mbox{\boldmath $t$}}
\def\bcallu{\mbox{\boldmath $u$}}
\def\bcallv{\mbox{\boldmath $v$}}
\def\bcallw{\mbox{\boldmath $w$}}
\def\bcallx{\mbox{\boldmath $x$}}
\def\bcally{\mbox{\boldmath $y$}}
\def\bcallz{\mbox{\boldmath $z$}}

% Bold Capital Latin letters
\def\blA{{\mathbf A}}
\def\blB{{\mathbf B}}
\def\blC{{\mathbf C}}
\def\blD{{\mathbf D}}
\def\blE{{\mathbf E}}
\def\blF{{\mathbf F}}
\def\blG{{\mathbf G}}
\def\blH{{\mathbf H}}
\def\blI{{\mathbf I}}
\def\blJ{{\mathbf J}}
\def\blK{{\mathbf K}}
\def\blL{{\mathbf L}}
\def\blM{{\mathbf M}}
\def\blN{{\mathbf N}}
\def\blO{{\mathbf O}}
\def\blP{{\mathbf P}}
\def\blQ{{\mathbf Q}}
\def\blR{{\mathbf R}}
\def\blS{{\mathbf S}}
\def\blT{{\mathbf T}}
\def\blU{{\mathbf U}}
\def\blV{{\mathbf V}}
\def\blW{{\mathbf W}}
\def\blX{{\mathbf X}}
\def\blY{{\mathbf Y}}
\def\blZ{{\mathbf Z}}

% Bold Italic Capital Latin letters
\def\bilA{\mbox{\boldmath $A$}}
\def\bilB{\mbox{\boldmath $B$}}
\def\bilC{\mbox{\boldmath $C$}}
\def\bilD{\mbox{\boldmath $D$}}
\def\bilE{\mbox{\boldmath $E$}}
\def\bilF{\mbox{\boldmath $F$}}
\def\bilG{\mbox{\boldmath $G$}}
\def\bilH{\mbox{\boldmath $H$}}
\def\bilI{\mbox{\boldmath $I$}}
\def\bilJ{\mbox{\boldmath $J$}}
\def\bilK{\mbox{\boldmath $K$}}
\def\bilL{\mbox{\boldmath $L$}}
\def\bilM{\mbox{\boldmath $M$}}
\def\bilN{\mbox{\boldmath $N$}}
\def\bilO{\mbox{\boldmath $O$}}
\def\bilP{\mbox{\boldmath $P$}}
\def\bilQ{\mbox{\boldmath $Q$}}
\def\bilR{\mbox{\boldmath $R$}}
\def\bilS{\mbox{\boldmath $S$}}
\def\bilT{\mbox{\boldmath $T$}}
\def\bilU{\mbox{\boldmath $U$}}
\def\bilV{\mbox{\boldmath $V$}}
\def\bilW{\mbox{\boldmath $W$}}
\def\bilX{\mbox{\boldmath $X$}}
\def\bilY{\mbox{\boldmath $Y$}}
\def\bilZ{\mbox{\boldmath $Z$}}

% Calligraphic Capital Latin letters
\def\callA{\mbox{$\mathcal{A}$}}
\def\callB{\mbox{$\mathcal{B}$}}
\def\callC{\mbox{$\mathcal{C}$}}
\def\callD{\mbox{$\mathcal{D}$}}
\def\callE{\mbox{$\mathcal{E}$}}
\def\callF{\mbox{$\mathcal{F}$}}
\def\callG{\mbox{$\mathcal{G}$}}
\def\callH{\mbox{$\mathcal{H}$}}
\def\callI{\mbox{$\mathcal{I}$}}
\def\callJ{\mbox{$\mathcal{J}$}}
\def\callK{\mbox{$\mathcal{K}$}}
\def\callL{\mbox{$\mathcal{L}$}}
\def\callM{\mbox{$\mathcal{M}$}}
\def\callN{\mbox{$\mathcal{N}$}}
\def\callO{\mbox{$\mathcal{O}$}}
\def\callP{\mbox{$\mathcal{P}$}}
\def\callQ{\mbox{$\mathcal{Q}$}}
\def\callR{\mbox{$\mathcal{R}$}}
\def\callS{\mbox{$\mathcal{S}$}}
\def\callT{\mbox{$\mathcal{T}$}}
\def\callU{\mbox{$\mathcal{U}$}}
\def\callV{\mbox{$\mathcal{V}$}}
\def\callW{\mbox{$\mathcal{W}$}}
\def\callX{\mbox{$\mathcal{X}$}}
\def\callY{\mbox{$\mathcal{Y}$}}
\def\callZ{\mbox{$\mathcal{Z}$}}

% Calligraphic Bold Capital Latin letters
\def\bcallA{\mbox{\boldmath $\mathcal{A}$}}
\def\bcallB{\mbox{\boldmath $\mathcal{B}$}}
\def\bcallC{\mbox{\boldmath $\mathcal{C}$}}
\def\bcallD{\mbox{\boldmath $\mathcal{D}$}}
\def\bcallE{\mbox{\boldmath $\mathcal{E}$}}
\def\bcallF{\mbox{\boldmath $\mathcal{F}$}}
\def\bcallG{\mbox{\boldmath $\mathcal{G}$}}
\def\bcallH{\mbox{\boldmath $\mathcal{H}$}}
\def\bcallI{\mbox{\boldmath $\mathcal{I}$}}
\def\bcallJ{\mbox{\boldmath $\mathcal{J}$}}
\def\bcallK{\mbox{\boldmath $\mathcal{K}$}}
\def\bcallL{\mbox{\boldmath $\mathcal{L}$}}
\def\bcallM{\mbox{\boldmath $\mathcal{M}$}}
\def\bcallN{\mbox{\boldmath $\mathcal{N}$}}
\def\bcallO{\mbox{\boldmath $\mathcal{O}$}}
\def\bcallP{\mbox{\boldmath $\mathcal{P}$}}
\def\bcallQ{\mbox{\boldmath $\mathcal{Q}$}}
\def\bcallR{\mbox{\boldmath $\mathcal{R}$}}
\def\bcallS{\mbox{\boldmath $\mathcal{S}$}}
\def\bcallT{\mbox{\boldmath $\mathcal{T}$}}
\def\bcallU{\mbox{\boldmath $\mathcal{U}$}}
\def\bcallV{\mbox{\boldmath $\mathcal{V}$}}
\def\bcallW{\mbox{\boldmath $\mathcal{W}$}}
\def\bcallX{\mbox{\boldmath $\mathcal{X}$}}
\def\bcallY{\mbox{\boldmath $\mathcal{Y}$}}
\def\bcallZ{\mbox{\boldmath $\mathcal{Z}$}}
\newcommand{\cnrism} {Istituto di Struttura della Materia and Division of Ultrafast Processes in Materials (FLASHit) of the National Research Council, via Salaria Km 29.3, I-00016 Monterotondo Stazione, Italy}
\newcommand{\etsf} {European Theoretical Spectroscopy Facilities (ETSF)}
%\def\smII{Section\,II}
\def\smII{\sec{sec:SM_EOMs}}

%\def\smIII{Section\,III}
\def\smIII{\sec{sec:g_dressing}}

%\def\smIV{Section\,IV}
\def\smIV{\sec{sec:QM_vs_CM}}

%\def\smV{Section\,V}
\def\smV{\sec{sec:SM_ehrenfest}}

%\def\smVI{Section\,VI}
\def\smVI{\sec{sec:SM_damped_ehrenfest}}

%\def\smVII{Section\,VII}
\def\smVII{\sec{sec:SM_model}}

%\newcommand{\mysec}[1]{}
\newcommand{\mysec}[1]{\section{#1}}
%\newcommand{\mysec}[1]{{\em #1}.}

\begin{document}

%%%%%%%%%%%%%%%%%%%%%%%%%%%%%%%%%%%%%%%%%%%%%%%%%%%%%%%%%%%%%%%%%%%%%%%%%%%%%%%%%%%%%%%%%%%%%%%%%%%%%%%%%%%%%%%%%%%%%%%%%%%%%%%%%%%%%%%%%%%%%%%%%%%%%%%%%%%%%%%
\title{
Non--adiabatic effects lead to the breakdown of the classical phonon concept
}
\author{Andrea Marini}
\affiliation{\cnrism}
%%%%%%%%%%%%%%%%%%%%%%%%%%%%%%%%%%%%%%%%%%%%%%%%%%%%%%%%%%%%%%%%%%%%%%%%%%%%%%%%%%%%%%%%%%%%%%%%%%%%%%%%%%%%%%%%%%%%%%%%%%%%%%%%%%%%%%%%%%%%%%%%%%%%%%%%%%%%%%%
\begin{abstract}
The classical phonon and, more in general, the atomic trajectory concepts are based on the assumption that the electrons follow adiabatically the
atoms. Here I extend  the classical phonon definition  to the non--adiabatic case 
and I compare it with the result of a fully quantistic treatment. 
When the atomic vibration is treated as a quantistic boson
the retarded electronic response induces a finite energy indetermination which reflects the internal phonon structure.
The classical phonon, instead, have an infinite lifetime that corresponds to persistent atomic oscillations. 
This difference induces a gap between the quantistic and classical phonon energies that increases with the energy indetermination.
This increasing gap reflects the breakdown of the classical description and
demonstrates that non--adiabatic effects can be captured only by using a fully quantistic approach.
\end{abstract}
\date{\today}
\maketitle

\mysec{Introduction}
%%%%%%%%%%%%%%%%%%%%%%%%%%%%%%%%%%%%%%%%%%%%%%%%%%%%%%%%%%%%%%%%%%%%%%%%%%%%%%%%%%%%%%%%%%%%%%%%%%%%%%%%%%%%%%%%%%%%%%%%%%%%%%%%%%%%%%%%%%%%%
\lab{sec:intro}
The research on the physics induced or mediated by the lattice vibrations is crucial in many and disparate fields of modern solid--state physics: infrared
spectroscopy, Raman, neutron-diffraction spectra, thermal transport are just a few of them~\cite{Stefano2001}.  The classical phonon approach assumes the atoms
to move around the minimum of the Born--Oppenheimer surface~\cite{Tavernelli2015}, with the electrons tightly bound to the atoms during their oscillations.
Non--adiabatic effects correspond to the case where the electronic screening is retarded with respect to the atomic motion.

Density Functional Theory\,(DFT)~\cite{R.M.Dreizler1990} and Density Functional Perturbation Theory(\dfpt)~\cite{Stefano2001,Gonze1995,Gonze1997a} have emerged
as successful and widely used approaches to calculate the structural, electronic properties and also atomic dynamics in a fully \ai, framework. DFT and \dfpt\,
are, nowadays, available in many public scientific codes\cite{Giannozzi2017,Gonze2009} and routinely used to calculate phonon frequencies and related
properties. 

Within DFT and DFPT the atomic degrees of freedom are treated classically, non--adiabatic effects are neglected 
and the total Hamiltonian depends parametrically on the atomic positions. Many--Body
Perturbation Theory\,(MBPT)~\cite{Leeuwen2013} provides an alternative approach to the atomic dynamics both in~\cite{Mahan1990} and beyond~\cite{Harkonen2020}
the harmonic regime. Within MBPT phonons are quasiparticles\cite{Marini2023,Giustino2017,Leeuwen2004a,Marini2015,2303.02102,Marini2018}
and non--adiabatic effects are fully captured.

Experimentally it is well--known that the phonon peaks observed in the inelastic X--Ray scattering~\cite{Shukla2003} or in the Raman
spectra~\cite{Ferrante2018} have an intrinsic energy width.  This energy width is a non--adiabatic effect, due to the retarded reaction of the electronic cloud
surrounding the phonon. In the quantistic case this retardation transforms the phonon in a quasi--phonon\,(QPH), a bosonic quasi--particle that I will define
more precisely in this work. While QPHs are natural concepts within MBPT the actual possibility to describe them in a classical theory like DFPT is still
debated~\cite{Marini2023}.  Some seminal works~\cite{Saitta2008,Calandra2010}, have actually proposed a time--dependent extension of DFPT\,(\tddfpt) that
predicts classical phonons to acquire an energy indetermination providing a pictures conceptually equivalent to MBPT.

The enormous simplicity of \tddfpt\ compared to the more involved MBPT approaches and its availability in many \ai\, codes, has favored the application of the
classical phonon concept well beyond the adiabatic regime. The \tddfpt\ approach has been applied to calculate a wealth of non--adiabatic properties. 
Examples are: phonon widths~\cite{Shukla2003,Lazzeri2005}, dynamical Kohn anomalies\cite{Lazzeri2006,Caudal2007}, non--adiabatic phonon corrections~\cite{Saitta2008} and
Non--adiabatic Born effective charges~\cite{PhysRevLett.128.095901}.

These works have cemented the idea that a classical description is formally equivalent to the more involved  Many--Body approach, with the difference that
\tddfpt\, relies on the change of the electronic density while MBPT requires to solve complicated equations written in terms of non--local phonon and electron
Green's functions.  This formal analogy between \tddfpt\, and MBPT has been mentioned in a recent review of F.\,Giustino~\cite{Giustino2017} where he writes:
{\em The field--theoretical phonon self--energy is in agreement with the expression derived starting from time-dependent density-functional perturbation
theory}.

In this work I critically reexamine the classical phonon concept and formally extend its definition to the non--adiabatic regime.  By using the linearized
Ehrenfest equation of motion I will fully include the electronic response retardation  in the time domain. The solution of the Ehrenfest dynamics, described by
a second--order Volterra integro--differential equation, will formally define the classical phonon energy.  This will be demonstrated to be real, meaning that
small classical atomic oscillation never decay in time and the corresponding atomic dynamics is energy conserving, in agreement with the exact properties of the
Ehrenfest classical approximation.  I will conclude the work by introducing a quasi--particle representation of the phonon self--energy in order to analytically
derive and compare the classical and quantistic phonon frequencies. I will show that the two solutions diverge as the MBPT phonon width increases thus
demonstrating that non--adiabatic effects can be described only by using a fully quantistic approach.

\mysec{The Hamiltonian and the problem}
%%%%%%%%%%%%%%%%%%%%%%%%%%%%%%%%%%%%%%%%%%%%%%%%%%%%%%%%%%%%%%%%%%%%%%%%%%%%%%%%%%%%%%%%%%%%%%%%%%%%%%%%%%%%%%%%%%%%%%%%%%%%%%%%%%%%%%%%%%%%%
\label{sec:H}
Let's start from the generic second quantization form of the  \ai\, Hamiltonian including the linear electron--phonon
interaction~\cite{Marini2023,Marini2015,2303.02102,fields}
\ml{
\h{H}=\sum_{i} \gee_{i} \h{\gr}_{ii}+ 
\sum_{\gl}\bps{\frac{\go_{\gl}}{2}\h{p}^2_\gl+\frac{\(\go_\gl-\Pi_\gl^{st}\)}{2}\h{u}^2_\gl\\+
\sum_{ij} g_{ji}^{\gl} \Delta\h{\gr}_{ij} \h{u}_{\gl}}+\sum_{ij} \Delta V^{Hxc}_{ji} \h{\gr}_{ij}.
\lab{eq:H.1}
}
In \e{eq:H.1} we have: the single--particle electronic\,(phononic) energies, $\gee_i$\,($\go_\gl$) and density matrix $\h{\gr}_{ij}$, the atomic displacement and momentum, $\h{u}_\gl$ and
$\h{p}_\gl$ and the electron--phonon potential $g^\gl_{ji}$. 

In \e{eq:H.1} appears also the variation of the the Hartree plus exchange--correlation potential\,(Hxc), $\Delta V^{Hxc}_{ji}$ induced by the electron--phonon
interaction. Indeed, while when $g^\gl_{ij}=0$ the reference Hxc potential is embodied in the single--particle levels, when the atomic motion induces a change
in the density the Hxc potential will change accordingly\cite{comment}.
$\h{\gr}_{ij}$ is written in terms of  the electronic creation and annihilation operators: $\h{c}^\dag_i/\h{c}_j$ and is $\Delta \h{\gr}_{ij}=\h{c}^\dag_i\h{c}_j-\average{ \h{c}^\dag_i\h{c}_j}$.

In \e{eq:H.1} $\Pi^{st}_\gl$ is the reference static part of the electron--nuclei component of the phonon dynamical matrix. As it has been
explained in \ocite{Marini2023,Marini2015,2303.02102} this terms is already included in the $\go_\gl$ definition and needs to be removed in order to avoid double counting effects.

\mysec{Equations of motion: Quantum vs Classical mechanics}
%%%%%%%%%%%%%%%%%%%%%%%%%%%%%%%%%%%%%%%%%%%%%%%%%%%%%%%%%%%%%%%%%%%%%%%%%%%%%%%%%%%%%%%%%%%%%%%%%%%%%%%%%%%%%%%%%%%%%%%%%%%%%%%%%%%%%%%%%%%%%
\label{sec:QM_and_CM}
The Hamiltonian $\h{H}$ induces a time--dependent dynamics of all operators, electronic and atomic. These equations are derived in details in the Supplementary
Material~(SM), \smII.

In the case of the atomic displacement operator we have
\mll{eq:EOM.1}
{
\h{\callD}_\gl\(t\)\h{u}_{\gl}\(t\)=i\sum_{ij} g_{ji}^\gl \int_{t_0}^t e^{i\Delta\gee_{ij}\(t-\tau\)} \\
\[\ul{\h{\gr}}\(t\),\ul{g}^\gl\h{u}_\gl\(t\)+\ul{\Delta V}^{Hxc}\[\gr\]\]_{ij}
}
with $\h{\callD}_\gl\(t\)=\frac{1}{\go_\gl}\bpr{\frac{\di^2}{\di t^2}-\go_\gl\(\Pi^{st}_\gl-\go_\gl\)}$. $t_0$ is the initial time and underlined quantities are
matrices in the single--particle basis.

The role of the $\ul{\Delta V}^{Hxc}$ potential is to dress the e--p interaction, as demonstrated in the SM, \smIII.  Here we note that we can define a
{\em screened} electron--phonon interaction, $\ti{\ul{g}}^\gl= \ul{\ul{\gee}}^{-1}\ul{g}^\gl$, with $\ul{\ul{\gee}}^{-1}$ the dielectric tensor.  As far as we
are interested in the linear--response regime, $\ul{\Delta V}^{Hxc}$ can be removed from \e{eq:EOM.1} by replacing, inside the commutator,
$\ul{g}^\gl$ with $\ti{\ul{g}}^\gl$.

The corner stone of the  Quantum\,(QM) and Classical Mechanics\,(CM) descriptions is the operator appearing on the r.h.s. of \e{eq:EOM.1}:
$\ul{\h{\gr}}\(t\)\h{u}_\gl\(t\)$. Indeed we can write 
\eql{eq:EOM.2}
{
 \average{\ul{\h{\gr}}\(t\)\h{u}_\gl\(t\)}=\ul{\gr}\(t\)u_\gl\(t\)+\average{\ul{\Delta\h{\gr}}\(t\)\Delta\h{u}_\gl\(t\)},
}
where $\Delta\ul{\h{\gr}}\(t\)=\ul{\h{\gr}}\(t\)- \ul{\gr}\(t\)$ and $\ul{\gr}\(t\)=\average{ \ul{\h{\gr}}\(t\) }$. The first term in \e{eq:EOM.2} corresponds
to the Hartree approximation and the second term represents the correlation, quantistic, correction. 

If we now take the average of \e{eq:EOM.1} neglecting the correlation term, the equation is closed in the space of $u_\gl\(t\)$ and $\ul{\gr}\(t\)$ and reduces
to the well--known Ehrenfest equation of motion. 

\mysec{MBPT phonons}
%%%%%%%%%%%%%%%%%%%%%%%%%%%%%%%%%%%%%%%%%%%%%%%%%%%%%%%%%%%%%%%%%%%%%%%%%%%%%%%%%%%%%%%%%%%%%%%%%%%%%%%%%%%%%%%%%%%%%%%%%%%%%%%%%%%%%%%%%%%%%
\label{sec:mbpt}
The correlation term $\average{\Delta\ul{\h{\gr}}\(t\)\Delta\h{u}_\gl\(t\)}$, instead, can be written in terms of the phonon propagator and
self--energy\,(details are provided in \ocite{Marini2023} and SM, \smIV). We thus have a sharp distinction between the classical (Hartree) and the
quantistic (Hartree+correlation) solutions of \e{eq:EOM.1}. From a physical point of view \e{eq:EOM.2} makes clear that CM describes the
trajectory\,($\average{\h{u}_\gl\(t\)}$) while QM the fluctuations\,($\average{\Delta\ul{\h{\gr}}\(t\)\Delta\h{u}_\gl\(t\)}$) around the classical trajectories.

The QM approach reduces to the solution of the Dyson equation for the phonon Green's function,
$D_\gl\(t,t^\p\)=-\im\average{\callT\{\Delta\h{u}_\gl\(t\)\Delta\h{u}\(t^\p\)\}}$\cite{Leeuwen2013}.  The poles of the Fourier transformed $D^{f}_\gl\(\go\)$
with respect to $\(t-t^\p\)$ are the QM energies and widths, $\gO^{QM}_\gl+\im \gc^{QM}_{\gl}$. Those are, in general, complex and are defined by the condition
\eql{eq:MBPT.1}
{
 \frac{\(\gO^{QM}_{\gl}+\im \gc^{QM}_{\gl}\)^2}{\go_\gl}=\go_\gl+ \Pi_\gl\(\gO^{QM}_{\gl}+\im\gc^{QM}_\gl\)-\Pi^{st}_\gl.
}
The usual interpretation is that while $\gO^{QM}_\gl$ is the renormalized phonon energy, $\gc^{QM}_{\gl}$ defines its energy
indetermination.

In \e{eq:MBPT.1} I have introduced the Fourier transformed phonon self--energy~\cite{Giustino2017,Marini2023}, calculated within
the statically screened approximation\,(SM, \smIII)
\eql{eq:MBPT.2}
{
 \Pi^{f}_\gl\(\go\)=\sum_{ij}g_{ji}^\gl \ti{g}_{ij}^\gl \frac{\Delta f_{ij}}{\go+\im 0^+ +\Delta\gee_{ij}}.
}
In the following I will labels $U^{f}\(\go\)$ and $U^{l}\(\go\)$ the Fourier and Laplace representations of the generic $U\(t\)$ function.

The small positive $\im 0^+$ appearing in \e{eq:MBPT.2} comes, in the case of zero--temperature MBPT theory, from the Gell--Mann\&Low
theorem~\cite{ALEXANDERL.FETTER1971} that allows to send, in \e{eq:EOM.1}, $t_0\rar -\infty$ and use an adiabatic switching on of the electron--phonon
interaction~\cite{Finite_T_note}.  Thanks to this 
basic theorem of MBPT $\Pi^{f}_\gl\(\go\)$ acquires a finite imaginary part and, consequently, provides the phonon with a finite energy indetermination. 

\mysec{TD--DFPT}
%%%%%%%%%%%%%%%%%%%%%%%%%%%%%%%%%%%%%%%%%%%%%%%%%%%%%%%%%%%%%%%%%%%%%%%%%%%%%%%%%%%%%%%%%%%%%%%%%%%%%%%%%%%%%%%%%%%%%%%%%%%%%%%%%%%%%%%%%%%%%
\label{sec:TDDFPT}
\tddfpt\, is based on the linear relation between the  density matrix and the atomic displacement. The same relation is
used in the adiabatic DFPT to derive the phonon secular equation~\cite{Stefano2001}. In \ocite{Calandra2010} this linear relation is 
extended to the time domain and used to derive a frequency dependent secular equation, whose solution leads to 
\eql{eq:tddfpt.2}
{
 \evalat{u_\gl\(t\)}{TD-DFPT}=\sum_{s=\pm} e^{is \gO^{QM}_{\gl}t}u_{\gl s} e^{-\gc^{QM}_\gl t}.
}
In \e{eq:tddfpt.2} $u_\gl^0=u_\gl\(t=0\)$ and $v_\gl^0=\evalat{\frac{\di u_\gl\(t\)}{\di t}}{t=0}$. 
Within \tddfpt\, we can  set $t_0=0$ in \e{eq:EOM.1} as it describes a classical pendulous displaced of $u_\gl^0$ with a given initial velocity $v_\gl^0$ at $t=0$.

\e{eq:tddfpt.2} implies that $\llim{u_\gl\(t\)}{t\rar\infty} 0$, i.e. the small classical oscillations decay in time, with a lifetime that is the inverse of the MBPT width.
Thanks to this \tddfpt\, has emerged as an alternative approach to MBPT cementing the idea that a classical description is formally equivalent, and numerically
much more affordable, to the more involved  Many--Body approach. \e{eq:tddfpt.2} is currently implemented in many \ai\, computational
codes like \qe\,~\cite{Giannozzi2009} and \abinit~\cite{Gonze2009}, for example. 

The {\tddfpt}--{MBPT} equivalence leads, however, to some basic questions. The CM description corresponds to take the Hartree approximation in \e{eq:EOM.2} while
the QM approach requires to evaluate the complicated correlation term\,($\average{\Delta\ul{\h{\gr}}\(t\)\Delta\h{u}_\gl\(t\)}$) that can be written only in terms of the 
phonon Green's function, a genuine quantum object~\cite{Marini2023}.
How is possible, then, that a purely classical approach would give \e{eq:tddfpt.2}, in complete agreement with the fully quantistic, MBPT, approach? In
addition, is the solution \e{eq:tddfpt.2} energy conserving?


\mysec{TD--DFPT as linearized Ehrenfest}
%%%%%%%%%%%%%%%%%%%%%%%%%%%%%%%%%%%%%%%%%%%%%%%%%%%%%%%%%%%%%%%%%%%%%%%%%%%%%%%%%%%%%%%%%%%%%%%%%%%%%%%%%%%%%%%%%%%%%%%%%%%%%%%%%%%%%%%%%%%%%
\label{sec:Ehrenfest}
In order to answer this question we observe that the small classical oscillations dynamics is exactly described by the Ehrenfest equation. This
can be easily seen by taking the first order in the expansion of the r.h.s of \e{eq:EOM.1} in $u_\gl\(t\)$, small by definition in the harmonic regime.  
This corresponds to approximate $\gr_{ij}\(t\)\sim
f_i\gd_{ij}$ and 
\eql{eq:Eh.1}
{
\h{\callD}_\gl\(t\)u_{\gl}\(t\)=\im\sum_{ij}\Delta f_{ij}g_{ji}^\gl \ti{g}^\gl_{ij}\int_{t_0}^t e^{\im\Delta\gee_{ij}\(t-\tau\)}u_\gl\(\tau\)\di\tau.
}
\e{eq:Eh.1} is a second--order Volterra linear integro--differential equation~\cite{Wazwaz2011} with a separable kernel\,($e^{\im\Delta\gee_{ij}\(t-\tau\)}$).
The Volterra equations are the subject of an intense mathematical research activity as they appear in many physical contexts like, for
example, the dynamics of viscoelastic materials~\cite{AlabauBoussouira2008} or applications of physical engineering~\cite{Zak2016,volterra_solvers}. 

In the SM, \smVI, I demonstrate that \e{eq:tddfpt.2} follows from \e{eq:Eh.1} when the Volterra kernel decays in time.  Indeed only in this case the
Fourier transformation is a valid solver~\cite{Wazwaz2011}. In practice \e{eq:tddfpt.2} is valid only if it is assumed from the
beginning that $\llim{u_\gl\(t\)}{t\rar\infty} 0$, i.e., only and only if the oscillations are artificially damped. Here we obtain the same conclusion by using
\e{eq:tddfpt.2} in \e{eq:Eh.1}
\mll{eq:Eh.3}
{
 \h{\callD}_\gl\(t\)\evalat{u_{\gl}\(t\)}{TD-DFPT}= \sum_{ij s} u_{\gl s} \Delta f_{ij} g^\gl_{ji} \ti{g}^\gl_{ij} \\\times
 \[\frac{e^{\im s \gO^{QM}_{\gl}t}e^{-\gc^{QM}_\gl t}-e^{\im\Delta\gee_{ij}t}}{s\gO^{QM}_{\gl}+\im\gc^{QM}_\gl-\Delta\gee_{ij}}\].
}
From \e{eq:Eh.3} we see that  $\evalat{u_{\gl}\(t\)}{TD-DFPT}$ is solution of \e{eq:Eh.1} only if $\llim{e^{\im\Delta\gee_{ij}t}}{t\rar \infty}0$. From a physical
point of view this means that in order for the classical phonon oscillations to decay it is needed that the classical system de--phase the off--diagonal matrix
elements of the density matrix. But within CM this is not possible as de--phasing is a genuine quantistic effect. Indeed \e{eq:Eh.3} can be satisfied only
if we add a small damping $\Delta\gee_{ij}\rar \Delta\gee_{ij}+\im 0^+$. Mathematically this damping forces 
the solution of \e{eq:Eh.1} in the subspace of functions $u_\gl\(t\)\in \callL^1\[t\]$, with $\callL^1\[t\]$ the space of modulus integrable functions,
$\int_{-\infty}^{\infty} |u_\gl\(t\)|=finite$. In this subspace the differential operator $\h{\callD}_\gl\(t\)$ can be Fourier transformed and 
\e{eq:tddfpt.2} easily follows (SM, \smVI).

In addition, and more importantly, the Ehrenfest dynamics is energy conserving. This means that if $E=\average{\h{H}}$ the solution of \e{eq:Eh.1}
must conserve $E$. But if \e{eq:tddfpt.2} is valid than $E$ acquires a time--dependence and $\llim{E\(t\)}{t\rar\infty}\average{\sum_{i} \gee_{i}
\h{\gr}_{ii}}$.  Thus $\evalat{u_{\gl}\(t\)}{TD-DFPT}$ is not a solution of \e{eq:Eh.1} and does not describe the classical atomic oscillations.

The correct energy--conserving solution must be therefore a solution $u_\gl\(t\)\notin \callL^1\[t\]$.
In order to solve \e{eq:Eh.1} in this case we Laplace transform both sides of \e{eq:Eh.1}. It is simple algebra to get
\eql{eq:Eh.4}
{
 u^{l}_\gl\(\go\)=\frac{\go u_\gl^0 +v_\gl^0}{\go^2-\go_\gl\(\Pi_\gl^{st}-\go_\gl\)+\go_\gl \Pi^{l}_\gl\(\go\)},
}
with
\eql{eq:Eh.5}
{
 \Pi^{l}_\gl\(\go\)=-\im \sum_{ij}g_{ji}^\gl \ti{g}_{ij}^\gl \frac{\Delta f_{ij}}{\go-\im\Delta\gee_{ij}}.
}
$u_\gl\(t\)$ can be analytically obtained from $u^{l}_\gl\(\go\)$ by using
the inverse Laplace transformation\,(known as Bromwhich integral~\cite{boas_2015}) that involves a complex plane integral of $u^{l}_\gl\(z\)$ with $z\in{\mathbb C}$. 

We now notice, however, from \e{eq:Eh.5} that $\Pi^l_\gl\(\im\Delta\gee_{km}\)=\infty$ for any $\Delta\gee_{km}$. This means that $u^{l}_\gl\(z\)$ 
 is ill defined when $z$ approaches the electron--hole energies. However let's rewrite $\Pi^{l}_\gl\(\go\)$ as
\eql{eq:Eh.8}
{
 \Pi^{l}_\gl\(\go\)=\(-\im\)\frac{ \sum_{J}R^\gl_J \prod_{I\neq J}\(\go-\im\Delta\gee_{I}\)}{\prod_{I}\(\go-\im\Delta\gee_{I}\)},
}
with $I=\(i,j\)$ and $R_I=g_{ji}^\gl \ti{g}_{ij}^\gl \Delta f_{ij}$. If we plug \e{eq:Eh.8} in \e{eq:Eh.4} we see that,
indeed, $u^{l}_\gl\(\im\Delta\gee_{km}\)=0$. This means that the points $z=\im\Delta\gee_{km}$ can be safely excluded
from the complex plane integral as they do not give any contribution. This implies that we 
can replace $\Pi_\gl\(z\)$ with its {\em Cauchy principal value}, $\Pi_\gl\(z\)\rar  \callP\[\Pi_\gl\(z\)\]$ which leads to a well
defined Bromwhich integral. The final solution is
\eql{eq:Eh.7}
{
 u^{l}_\gl\(z\)=\frac{z u_\gl^0 +v_\gl^0}{z^2-\go_\gl\(\Pi_\gl^{st}-\go_\gl\)-\go_\gl \callP\[\Pi_\gl\(z\)\]}.
}
From \e{eq:Eh.7} and from the fact that 
$\callP\[\Pi^{l}_\gl\(\pm \im \go\)\]=\Re\[\Pi^{f}_\gl\(\go\)\]$ it follows that the solution of \e{eq:Eh.1} is
\eql{eq:Eh.10}
{
 u_\gl\(t\)=\sum_{s=\pm} e^{is \gO^{CM}_{\gl}t}u_{\gl s},
}
with $\gO^{CM}_{\gl}$ solution of the fixed point equation
\eql{eq:Eh.11}
{
 \(\gO^{CM}_{\gl}\)^2=\go_\gl\(\go_\gl+\Re\[\Pi^{f}_\gl\(\gO^{CM}_{\gl}\)\]-\Pi^{st}_\gl\).
}
\es{eq:Eh.10}{eq:Eh.11} are the main result of the present work. They demonstrate that a classical approach has access only to the real part of the Many--Body self-energy.

\e{eq:Eh.10} is an energy conserving solution as it represents an an harmonic oscillator with a renormalized  frequency, $\gO^{CM}_{\gl}$. This means the 
the total energy is constant by definition.

\mysec{Non--adiabatic corrections to the classical description}
%%%%%%%%%%%%%%%%%%%%%%%%%%%%%%%%%%%%%%%%%%%%%%%%%%%%%%%%%%%%%%%%%%%%%%%%%%%%%%%%%%%%%%%%%%%%%%%%%%%%%%%%%%%%%%%%%%%%%%%%%%%%%%%%%%%%%%%%%%%%%
\label{sec:non-adiabatic}
I introduce now a quasi--phonon\,(QPH) approximation for the self--energy. If  $\Delta\Pi^{f}_\gl\(\go\)=\Pi^{f}_\gl\(\go\)-\Pi^{st}_\gl$ 
I define the QPH approximation as
\mll{eq:qph.1}
{
 \evalat{\Delta\Pi^{f}_\gl\(\go\)}{QPH}=i \Im\[\Pi^{f}_\gl\(\go_\gl\)\]\frac{\go}{\go_\gl}\\+\Re\[\Delta\Pi^{f}_\gl\(\go_\gl\)\]\frac{\go^2}{\go_\gl^2}.
}
In order to test the accuracy of \e{eq:qph.1} I use an exactly solvable generalized Fr\"ohlich model, whose Hamiltonian 
is reviewed in the SM, \smVII. In this case there is only one phonon branch and the $\gl$ index corresponds
to the phonon momentum, $\qq$. In the SM I also compare \e{eq:qph.1} with the full self--energy. The result is that
the QPH approximation is very accurate and well describes both the real and the imaginary part of the self--energy.

\begin{figure}[t!]
{\centering
\includegraphics[width=\columnwidth]{Fig1.pdf}
}
\caption{Classical versus \tddfpt\ solutions of the Volterra--Ehrenfest equation in the small oscillation regime. The bare solution\,(dashed line) is compared with the 
present, persistent, energy conserving solution\,(continuous line) and with the \tddfpt, artificially decaying, and energy violating solution\,(dotted line). As predicted by the \e{eq:qph.3} the
classical phonon frequency is increased and, indeed, its period reduced. $T_0$ is defined as $\frac{2\pi}{\go_0}$ with $\go_0=100\,meV$.
}
\label{fig:1}
\end{figure}

In \fig{fig:1} I compare \e{eq:Eh.10} and \e{eq:tddfpt.2} in the case of the Fr\"ohlich model, for a momentum $\qq$ such that $\gc^{QM}_\qq\sim\frac{\go_0}{2}$.
We see that the Laplace solution\,(continuous line) is persistent and energy conserving, while the \tddfpt\ solution\,(dashed line) artificially decays and violates the energy conservation.
We also notice that the the \tddfpt\, and Laplace solutions have different time period, reflecting a different phonon frequency.
By using \e{eq:qph.1} we can not estimate the impact of the MBPT width, $\gc^{QM}_\gl$, on the difference between $\gO^{QM}_\gl$ and $\gO^{CM}_\gl$.

We can now use \e{eq:qph.1} to solve \e{eq:MBPT.1} and \e{eq:Eh.11}. By simple algebra it is possible to demonstrate that 
\eqgl{eq:qph.2}
{
\gc^{QM}_{\gl}=-\frac{Z_\gl}{2}\Im\[\Pi^{f}_\gl\(\go_\gl\)\],\\
\gO^{QM}_{\gl}=\pm\(Z_\gl\go_\gl^2-\(\gc^{QM}_{\gl}\)^2\)^{\frac{1}{2}},
}
with $Z_\gl=\(1-\frac{\Re\[\Delta\Pi^{f}_\gl\(\go_\gl\)\]}{\go_\gl}\)^{-1}$ the phonon renormalization factor. $Z_\gl$ plays a role similar to the
electronic, quasi--particle renormalization factor~\cite{ALEXANDERL.FETTER1971}. Similarly
\eql{eq:qph.3}
{
\gO^{CM}_{\gl}=\pm\(Z_\gl\go_\gl^2\)^{\frac{1}{2}}.
}
From \es{eq:qph.2}{eq:qph.3} it follows that, when $\frac{\gc^{QM}_\gl}{\go_\gl}\ll 1$
\eql{eq:qph.4}
{
 \gO^{QM}_{\gl}\approx \gO^{CM}_{\gl}\(1-\frac{\gc^{QM}_{\gl}}{\gO^{CM}_{\gl}}\).
}
The classical and quantistic phonon energies are compared as a function of $\Im\[\Pi^{f}_\gl\(\go_\gl\)\]$ for the  Fr\"ohlich model in \fig{fig:2}.
We see that the gap between $\gO^{QM}_{\gl}$ and $\gO^{CM}_{\gl}$ increases with
increasing phonon width. Moreover $\gO^{QM}_\gl<\go_\gl$ while $\gO^{CM}_\gl>\go_\gl$.

In the inset of  \fig{fig:2} the dependence of the phonon renormalization factor  is shown as a function of the phonon width. We see that it rapidly moves above
$1$. We note that the usual on--the--mass shell approximation widely used in the literature is obtained by taking $Z_\gl=1$ in \elab{eq:qph.2}{a} 
and by taking the first order term in the expansion of $Z_\gl\sim 1+\frac{\Re\[\Delta\Pi^{f}_\gl\(\go_\gl\)\]}{\go_\gl}$ in \elab{eq:qph.2}{b}.

\begin{figure}[t!]
{\centering
\includegraphics[width=\columnwidth]{Fig2.pdf}
}
\caption{
Comparison of the classical (continuous line) with the quantistic, quasi--phonon, energies (dashed line) for increasing values of the imaginary part of the
self--energy. The classical solution predicts a positive frequency renormalization while the quasi--phonon energy remains always below the bare
phonon energy. This is a signature of the breakdown of the classical phonon pictures caused by increasing non--adiabatic effects. In the inset the 
the phonon renormalization factor is shown. It grows as imaginary part of the self--energy grows, in agreement with \e{eq:qph.3}.
}
\label{fig:2}
\end{figure}

If we consider the case of realistic materials we can take as a reference the list reported in \ocite{Saitta2008}. From Table I of  \ocite{Saitta2008} we can deduce the values of 
$\frac{\Im\[\Pi^{f}_\gl\(\go_\gl\)\]}{\go_\gl}$ for several graphite intercalated compounds that show large non--adiabatic effects. 
If assume the  experimental width to be proportional to $\Im\[\Pi^{f}_\gl\(\go_\gl\)\]$ we see that
in the case of MgB$_2$ 
$\frac{\Im\[\Pi^{f}_\gl\(\go_\gl\)\]}{\go_\gl}\sim 0.36$ which implies a deviation of the classical solution from the QPH energy of about $7-10\,\%$. This confirms
the strong non--adiabaticity of MgB$_2$ that, on the basis of the present results, corresponds to a case where the classical theory breaks down.

\mysec{Conclusions}
%%%%%%%%%%%%%%%%%%%%%%%%%%%%%%%%%%%%%%%%%%%%%%%%%%%%%%%%%%%%%%%%%%%%%%%%%%%%%%%%%%%%%%%%%%%%%%%%%%%%%%%%%%%%%%%%%%%%%%%%%%%%%%%%%%%%%%%%%%%%%
\label{sec:conclusions}
The phonon concept is one of the most well--known concept in physics. It is described in many text--books and it is based on the adiabatic ansatz where the
atoms are assumed to move followed instantaneously by the electrons. The electronic cloud tightly bound to the atoms provides the renormalization of the phonon
energies as predicted, for example, in static \dfpt.

Many materials, however, show strong non--adiabatic effects induced by the delay between the atomic and the electronic motions. In this realm \tddfpt\, has been
proposed as a valid alternative to the more cumbersome MBPT methods. As \tddfpt\, is based on the classical phonon concept it is know widely accepted that the
two approaches, quantistic\,(MBPT) and classical\,(TD--DFPT), actually provide the same physical picture: the phonon energy is renormalized and it also acquires
a lifetime. 

Within  MBPT, however, the imaginary part of the self--energy describes an {\em energy indetermination}, while in \tddfpt\, it is commonly linked to a {\em
decay time}, describing the attenuation of the classical oscillations. Two very different concepts.

In this work I have extended the classical phonon concept to the non--adiabatic regime  by fully including the electronic retarded screening of the atomic
potential.  By solving exactly the small classical oscillations dynamics I have demonstrated that the state--of--the--art \tddfpt\ solution is affected by an
artificial decay induced by the erroneous use of the Fourier transformation to solve the time--dependent secular equation. 
I show, instead, that the calssical phonon energy is purely real, in contrast to what predicted by \tddfpt.
This demonstrates that a classical description has no access to the phonon energy indetermination\cite{light_comment}.

I conclude the work by introducing a quasi--phonon representation to investigate the difference between the classical and the quantistic phonon definitions.
The two solutions are shown to diverge as the imaginary part of the MBPT self--energy increases. The widening of this gap reflects the breakdown of the
classical picture.

The final results is that the classical phonon approach cannot describe non--adiabatic effects. In this case a fully quantistic scheme is needed. The present
results open several questions in many different fields of physics and call for further research on the impact of non--adiabatic effects on the atomic dynamics
within and beyond the harmonic approximation. 

\mysec{Acknowledgments}
%%%%%%%%%%%%%%%%%%%%%%%%%%%%%%%%%%%%%%%%%%%%%%%%%%%%%%%%%%%%%%%%%%%%%%%%%%%%%%%%%%%%%%%%%%%%%%%%%%%%%%%%%%%%%%%%%%%%%%%%%%%%%%%%%%%%%%%%%%%%%
A.M. would like to acknowledge F. Paleari, G. Stefanucci and E. Perfetto for helpful discussions. 
A.M. acknowledges the funding received from the European Union projects: MaX {\em Materials design at the eXascale} H2020-INFRAEDI-2018-2020/H2020-INFRAEDI-2018-1, Grant agreement n. 824143;  
{\em Nanoscience Foundries and Fine Analysis -- Europe | PILOT}  H2020-INFRAIA-03-2020, Grant agreement n. 101007417; 
{\em PRIN: Progetti di Ricerca di rilevante interesse Nazionale} Bando 2020, Prot. 2020JZ5N9M.

%%%%%%%%%%%%%%%%%%%%%%%%%%%%%%%%%%%%%%%%%%%%%%%%%%%%%%%%%%%%%%%%%%%%%%%%%%%%%%%%%%%%%%%%%%%%%%%%%%%%%%%%%%%%%%%%%%%%%%%%%%%%%%%%%%%%%%%%%%%%%
\appendix

\section{Equations of motion for the electronic and bosonic operators}
%%%%%%%%%%%%%%%%%%%%%%%%%%%%%%%%%%%%%%%%%%%%%%%%%%%%%%%%%%%%%%%%%%%%%%%%%%%%%%%%%%%%%%%%%%%%%%%%%%%%%%%%%%%%%%%%%%%%%%%%%%%%%%%%%%%%%%%%%%%%%
\label{sec:SM_EOMs}
The Hamiltonian defined in Eq.(1) of the main text induces a dynamics of all operators, electronic and atomic. These equations have been 
recently reviewed in \ocite{Marini2023,Marini2015,2303.02102}. The second--order time derivative of the displacement is
\eql{eq:SM.eom.1}
{
\frac{\di^2}{\di t^2}\h{u}_{\gl}\(t\)=  \go_\gl\(\Pi^{st}_\gl-\go_\gl\)\h{u}_{\gl}\(t\)-\go_\gl\sum_{ij}g_{ji}^\gl \Delta\h{\gr}_{ij}\(t\),
}
while the equation of motion for the density matrix is
\mll{eq:SM.eom.2}
{
 i\frac{\di}{\di t}\h{\gr}_{ij}\(t\)=-\Delta\gee_{ij}\h{\gr}_{ij}\(t\)\\+\sum_{\gl}
\[\ul{\h{\gr}}\(t\),\ul{g}^\gl\h{u}_\gl\(t\)\]_{ij}+\[\ul{\h{\gr}}\(t\),\Delta\ul{V}^{Hxc}\[\gr\(t\)\]\]_{ij}
}
\es{eq:SM.eom.1}{eq:SM.eom.2} completely solve the Many--Body problem and produce the equations governing the classical atomic motion as an
approximation. 


\section{Electron--phonon interaction dynamical dressing}
%%%%%%%%%%%%%%%%%%%%%%%%%%%%%%%%%%%%%%%%%%%%%%%%%%%%%%%%%%%%%%%%%%%%%%%%%%%%%%%%%%%%%%%%%%%%%%%%%%%%%%%%%%%%%%%%%%%%%%%%%%%%%%%%%%%%%%%%%%%%%
\label{sec:g_dressing}
The role of $V^{Hxc}$ in \e{eq:SM.eom.2} is to  screen the electron--phonon interaction $g^{\gl}_{ji}$. In order to see it we observe
that~\cite{978-3-642-23518-4}
\mll{eq:SM_g_dress.1}
{
 \ul{\Delta V}^{Hxc}\[\gr\]\(t\)= 
 \int \di \tau\, t^\p \ul{\ul{f}}^{Hxc}\(t-\tau\) \gd \ul{\gr}\(\tau\)=\\
 \sum_\gl \int \di \tau\, t^\p \ul{\ul{f}}^{Hxc}\(t-\tau\) \ul{\ul{\chi}}\(\tau-t^\p\) \ul{g}^\gl u_\gl\(t^\p\),
}
with $\ul{\ul{\chi}}$ the tensorial reducible response function and $\ul{\ul{f}}^{Hxc}$ the Hartree plus exchange--correlation kernel, defined as
$\frac{\gd \ul{V}^{Hxc}\[\gr\]\(t\)}{\gd \ul{\gr}\(t^\p\)}$. The single underlined
quantities are matrices\,($\ul{M}_{ij}$) while the doubly underlined are tensors\,($\ul{\ul{M}}_{\sst{ij}{kl}}$) in the electron--hole pairs Fock space. If we now neglect, for
simplicity, the time--dependence of $f^{Hxc}$ and plug \e{eq:SM_g_dress.1} in \e{eq:SM.eom.2} we see that on the r.h.s. it appears
\mll{eq:SM_g_dress.2}
{
 \ul{g}^\gl\h{u}_\gl\(t\)+\ul{\Delta V}^{Hxc}\[\gr\(t\)\]\\=
 \int \di t^\p \[\ul{\ul{\gd}}\gd\(t-t^\p\)+ \ul{\ul{f}}^{Hxc} \ul{\ul{\chi}}\(t-t^\p\)\] \ul{g}^\gl u_\gl\(t^\p\)\\
 =\int \di t^\p \ul{\ul{\gee}}^{-1}\(t-t^\p\)\ul{g}^\gl u_\gl\(t^\p\)\equiv \int \di t^\p \ul{\ti{g}}^{\gl}\(t-t^\p\)u_\gl\(t^\p\).
}
\e{eq:SM_g_dress.2} defines the dynamically screened electron--phonon potential, $\ti{\ul{g}}^{\gl}\(t-t^\p\)$ and demonstrates that the change in the density induced by the atomic motion induce the screening of the
$g^{\gl}$ potential. At this point, following \ocite{Marini2023}, I have used a statically screened 
interaction in all calculations presented in the main body of the work. 

It is instructive to see that  the $\ul{\Delta V}^{Hxc}$ potential screens only the interaction appearing in \e{eq:SM.eom.2} while the same interaction in
\e{eq:SM.eom.1} remains bare. This is an alternative way to introduce the partially screened form of the phonon self--energy proposed in \ocite{Marini2023}.


\section{Quantum vs Classical dynamics}
%%%%%%%%%%%%%%%%%%%%%%%%%%%%%%%%%%%%%%%%%%%%%%%%%%%%%%%%%%%%%%%%%%%%%%%%%%%%%%%%%%%%%%%%%%%%%%%%%%%%%%%%%%%%%%%%%%%%%%%%%%%%%%%%%%%%%%%%%%%%%
\label{sec:QM_vs_CM}
The corner stone of the Quantum Mechanics\,(QM) and Classical Mechanics\,(CM) descriptions is the correlation term appearing on the r.h.s. of \e{eq:SM.eom.2}: $\h{\gr}_{il}\(t\)\h{u}_\gl\(t\)$.
Indeed we have that
\eql{eq:SM.qcm.1}
{
 \(\im\frac{\di}{\di t}+\Delta\gee_{ij}\)\h{\gr}_{ij}\(t\)= \evalat{\im\frac{\di}{\di t}\gr_{ij}\(t\)}{CM}+\evalat{\im\frac{\di}{\di t}\gr_{ij}\(t\)}{QM},
}
with
\eqgl{eq:SM.qcm.2}
{
\im\evalat{\frac{\di}{\di t}\ul{\gr}\(t\)}{CM}= \sum_{\gl} \[\ul{\gr}\(t\),\ul{\ti{g}}^\gl\]u_\gl\(t\),\\
\evalat{\frac{\di}{\di t}\ul{\gr}\(t\)}{QM}= \sum_{\gl} \int \di t^\p \[\ul{\ul{\chi}}^{\(0\)}\(t,t^\p\) \ul{g}^\gl,\ul{\ti{g}}^\gl\] D_{\gl}\(t^\p,t\).
}
The two approaches are somehow alternative in the sense that QM has not access to $u_\gl\(t\)$ and CM does not know anything about $D_{\mu}\(t^\p,t\)$, the 
phonon Green's function. \elab{eq:SM.qcm.2}{b} can be found as Eq.(50a) in \ocite{Marini2023}.


\section{Ehrenfest atomic oscillations in the linear--response regime}
%%%%%%%%%%%%%%%%%%%%%%%%%%%%%%%%%%%%%%%%%%%%%%%%%%%%%%%%%%%%%%%%%%%%%%%%%%%%%%%%%%%%%%%%%%%%%%%%%%%%%%%%%%%%%%%%%%%%%%%%%%%%%%%%%%%%%%%%%%%%%
\label{sec:SM_ehrenfest}
The linear response regime\,(LRR) corresponds to the case of tiny displacements. In this case only the first order term in the r.h.s of \e{eq:SM.eom.2} must be considered.
It follows:
\eql{eq:SM.eh.2}
{
\evalat{\sum_{\gl} \[\ul{\gr}\(t\),\ul{\ti{g}}^\gl\]_{ij}u_\gl\(t\)}{LRR}= \sum_\gl\ti{g}_{ij}^{\gl}\Delta f_{ij}u_\gl\(t\),
}
with $\Delta f_{ij}=f_i-f_j$. 
If we plug \e{eq:SM.eh.2} into \e{eq:SM.eom.2} we get
\eql{eq:SM.eh.3}
{
 \gr_{ij}\(t\)=-i \sum_\gl \ti{g}^{\gl}_{ij}\Delta f_{ij} e^{i\Delta\gee_{ij}t} \int_{0}^t \di\tau e^{-i\Delta\gee_{ij}\tau} u_{\gl}\(\tau\),
}
which, plugged into \e{eq:SM.eom.1} gives
\eql{eq:SM.eh.4}
{
\hat{\callD}_\gl\(t\)u_{\gl}\(t\)=-\int_{0}^t \Pi_\gl\(t-\tau\) u_{\gl}\(\tau\),
}
with
\eql{eq:SM.eh.5}
{
 \Pi_\gl\(t\)= -i \sum_{ij} g_{ji}^\gl \ti{g}_{ij}^{\gl}\Delta f_{ij}  e^{-i\Delta\gee_{ij}t}
}
In \es{eq:SM.eh.4}{eq:SM.eh.5} I have assumed $g_{ji}^\gl \ti{g}_{ij}^{\gl^\p}\sim g_{ji}^\gl \ti{g}_{ij}^{\gl}\gd_{\gl,\gl^\p}$, thus neglecting 
phonon branch renormalizations. Indeed, if $\gl\neq \gl^\p$ the
Volterra integro--differential equations turns in a matrix equation whose solution requires the definition of a self--consistent phonon basis. 
In the case of the  Fr\"ohlich model Hamiltonian, \sec{sec:SM_model}, used in this work, there is only one phonon branch and $\gl=\gl^\p$ by definition.

\section{$u_{\gl}\left(t\right)\in \callL^1\[t\]$: The Fourier artificially decaying solution}
%============================================================================================================================================
\label{sec:SM_damped_ehrenfest}
The Laplace and Fourier transformations are, apparently, very similar:
\eqgl{eq:SM.damp.A.1}
{
 g^{f}\(\go\)=\int_0^\infty e^{\im \go t} g\(t\),\\
 g^{l}\(\go\)=\int_0^\infty e^{-\go t} g\(t\).
}
In \e{eq:SM.damp.A.1} I have assumed $g\(t\)=0$ when $t<0$, which is the case of a classical pendulous displaced from the equilibrium position at $t=0$.

The Fourier transformation is not listed among the solvers of the integro--differential Volterra equation and the reason is simple. To apply
\e{eq:SM.damp.A.1} to \e{eq:SM.eh.4} we need to transform both sides. In the case of r.h.s. we would have
\mll{eq:SM.damp.A.2}
{
 \int_0^\infty e^{\im \go t} \frac{\di^2 u_\gl\(t\)}{\di t^2}\di t=\left. e^{\im \go t} \frac{\di u_\gl\(t\)}{\di t}\right|_{0}^{\infty}\\
 -\im\go\[ \left. e^{\im \go t} u_\gl\(t\)\right|_{0}^{\infty}-\im\go u^f\(\go\)\].
}
From \e{eq:SM.damp.A.2} it is evident that in order for the differential operator to be Fourier transformable we do need to impose the condition
$\evalat{\frac{\di u_\gl\(t\)}{\di t}}{t=\infty}=u_\gl\(t=\infty\)=0$. Therefore the Fourier transformation can be used if and only if 
it is assumed from the beginning that the solution will decay in time, leading to a solution $u_\gl\(t\)\in\callL^1\[t\]$.

As discussed in the main text the constrain $u_{\gl}\(t\)\in\callL^1$ appears even more dramatically when \elab{eq:SM.damp.A.1}{a} is applied to the
r.h.s of \e{eq:SM.eh.4}. In this case the only way to get a converging integral is to add a fictitious damping  such that
\eql{eq:SM.damp.1}
{
 e^{-i\Delta\gee_{ij}t}\rar e^{-i\Delta\gee_{ij}t}e^{-0^+ t}.
}
In this case it is possible  to use the Fourier transformation:
\seql{eq:SM.damp.3}
{
\eqg{
 \int_0^\infty e^{\im \go t}\[ \int \di \tau \Pi_\gl\(t-\tau\) u_\gl\(\tau\) \]= \Pi^f_\gl\(\go\)u^f_\gl\(\go\),\\
 \int_0^\infty e^{\im \go t}\[\frac{d^2\uu_\gl\(t\)}{d t^2}\]=-\go^2 u^f_\gl\(\go\)+\im\go u_\gl^0-v_\gl^0.
}
}
Thanks to \e{eq:SM.damp.3} it is simple algebra to show that \e{eq:SM.eh.4}  can be easily rewritten as
\eql{eq:SM.damp.4}
{
 \[\go^2+\go_\gl\(\Pi^{st}_\gl-\Pi^f_\gl\(\go\)-\go_\gl\)\]u^f_\gl\(\go\)=\im\go u_\gl^0-v_\gl^0.
}
\e{eq:SM.damp.4} defines $\gO_\gl$\,(in general, complex) such that
\eql{eq:SM.damp.5}
{
 \gO_\gl^2+\go_\gl\(\Pi^{st}_\gl-\Pi^f_\gl\(\gO_\gl\)-\go_\gl\)=0.
}
Now \e{eq:SM.damp.5} admits two solutions. If we write them as
\eql{eq:SM.damp.7}
{
\gO_\gl=s \ti{\gO}_\gl-\im  \gc_\gl,
}
with $s=\pm 1$, $\ti{\gO}_\gl\in {\mathbb R}$ and $\gc_\gl \in {\mathbb R}$. We now  approximate $\gO_\gl\approx s \go_\gl$ on the r.h.s. of \e{eq:SM.damp.5} to get
\eql{eq:SM.damp.8}
{
 -2s \ti{\gO}_\gl \gc_\gl= \go_\gl \Im\[\Pi^f_\gl\(s\go_\gl\)\]=-s \go_\gl \left|\Im\[\Pi^f_\gl\(\go_\gl\)\]\right|,
}
where we have used the property of the self--energy
\eql{eq:SM.damp.9}
{
 \Im\[\Pi^f_\gl\(s\go_\gl\)\]=-s \go_\gl \left|\Im\[\Pi^f_\gl\(\go_\gl\)\]\right|.
}
\e{eq:SM.damp.8} demonstrates that the $\llim{u_\gl\(t\)}{t\rar\infty}\sim e^{-\gc_\gl t}$ with
\eql{eq:SM.damp.10}
{
 \gc_\gl\approx \frac{\go_\gl}{2\ti{\gO}_\gl} \left|\Im\[\Pi^f_\gl\(\go_\gl\)\]\right|.
}
\e{eq:SM.damp.10} is in agreement with \ocite{Calandra2010} and demonstrates that the decay of the classical phonon oscillations is artificially induced by the
initial assumption of a decaying $u_\gl\(t\)$. This is a necessary condition to use the Fourier transformation to find the solution of the the Ehrenfest--Volterra
integro--differential equation, \e{eq:SM.eh.4}.

\section{The Fr\"ohlich model Hamiltonian}
%============================================================================================================================================
\label{sec:SM_model}
The generalized Fr\"ohlich model Hamiltonian has been extensively described and tested in \ocite{Marini2023}
and it is characterized by a single phonon with energy $\go_0$ interacting with a gas of free electrons via a Fr\"{o}hlich like, $q$--dependent
e--p interaction $g_q$. In addition to the e--p term it includes the Hartree plus Exchange--Correlation potential so to describe the dynamical screening of
$g_q$, \sec{sec:g_dressing}.
The model Hamiltonian describing this system is
\seq{
\lab{eq:jell.1}
\ml{
 \h{H}_m=\sum_\kk \gee_\kk \h{c}^\dagger_{\kk} \h{c}_\kk+\frac{1}{\gO}\sum_{\qq} V^{Hxc}_{-\qq}\h{\gr}_\qq+\\
  \frac{\go_0}{2}\sum_\qq\[ \(\h{u}^\dagger_{\qq} \h{u}_{\qq} + \h{p}^\dagger_{\qq} \h{p}_{\qq}\)+
  \sqrt{2} g_q  \h{u}_{\qq} \gD\h{\gr}_\qq\],
}
with
\eq
{
 \h{\gr}_\qq= \frac{1}{ \sqrt{N}}\sum_{\kk}\h{c}^\dag_{\kk}\h{c}_{\kk-\qq}.
}
Following \ocite{Nery2018}, I define
\eq
{
 g^2_q=\frac{\ga}{q^2}\frac{2\pi\go_0}{\gO}\sqrt{\(\frac{2\go_0}{m^*}\)},
}
with $\ga$ the a--dimensional e--p Fr\"{o}hlich constant, $q=|\qq|$ and $k=|\kk|$.
The energy levels are assumed to be  $\gee_\kk=\frac{\kk^2}{2 m^*}$ with $m^*$ the effective mass. 
}

The value of the different parameters entering \e{eq:jell.1} are chosen in order to reproduce the $\qq$ dependence of the dielectric properties of solid
MgB$_2$, as described at length in \ocite{Marini2023}.

In \fig{fig:SM_1} the phonon spectral function, $|\Im\[D^f_\qq\(\go\)\]|$, is shown.  
The spectral function is calculated for a momentum such that $\frac{\gC^{MB}_\qq}{\go_0}\sim 0.3$. The comparison of the QPH and the exact solutions of the
Dyson equation shows that the QPH approximation reproduces very well both the real and the imaginary part of the self--energy. 

\begin{figure}[H]
{\centering
\includegraphics[width=\columnwidth]{SM_Fig1.pdf}
}
\caption{
Frame (a): comparison of the calculated phonon self--energy with the QPH representation for the generalized Fr\"ohlich Hamiltonian, \e{eq:jell.1}.
Full (dashed) lines correspond to the imaginary (real)
part. The imaginary part is plotted in modulus as it is negative in the positive energy range. 
The QPH imaginary (real) parts are represented with boxes (circles). The QPH works very well in reproducing the full self--energy and provides
an analytic representation of the spectral function. This is shown in frame (b) where the QPH (dashed line) is compared with the full self--energy.
}
\label{fig:SM_1}
\end{figure}


%%%%%%%%%%%%%%%%%%%%%%%%%%%%%%%%%%%%%%%%%%%%%%%%%%%%%%%%%%%%%%%%%%%%%%%%%%%%%%%%%%%%%%%%%%%%%%%%%%%%%%%%%%%%%%%%%%%%%%%%%%%%%%%%%%%%%%%%%%%%%
\bibliography{paper}

%%%%%%%%%%%%%%%%%%%%%%%%%%%%%%%%%%%%%%%%%%%%%%%%%%%%%%%%%%%%%%%%%%%%%%%%%%%%%%%%%%%%%%%%%%%%%%%%%%%%%%%%%%%%%%%%%%%%%%%%%%%%%%%%%%%%%%%%%%%%%
\end{document}
