\documentclass[11pt,reqno]{amsart}

\pdfoutput=1
\usepackage{amsfonts, amstext, amsmath, amsthm, amscd, amssymb, upgreek, mathtools, soul, float}
\usepackage[pdftex]{graphicx, color}
\usepackage[dvipsnames]{xcolor}
\usepackage[all,cmtip]{xy} 
\usepackage{tikz}
\usepackage{import}
\usepackage[hidelinks,pagebackref]{hyperref}
\usepackage{subfigure, wrapfig, overpic}
\usepackage{comment}
\usepackage[shortlabels]{enumitem}
\usepackage{etoolbox}
\usepackage{natbib}
\usepackage[right]{lineno}
\usepackage{array}
\usepackage[foot]{amsaddr}

\setlength{\textwidth}{6in}
\setlength{\textheight}{8.5in}
\setlength{\topmargin}{0in}
\setlength{\oddsidemargin}{.25in}
\setlength{\evensidemargin}{.25in}
 


\newcommand\restr[2]{{
  \left.\kern-\nulldelimiterspace
  #1
  \vphantom{\big|}
  \right|_{#2}
  }}

\makeatletter
\patchcmd{\@maketitle}
  {\ifx\@empty\@dedicatory}
  {\ifx\@empty\@date \else {\vskip3ex \centering\footnotesize\@date\par\vskip1ex}\fi
   \ifx\@empty\@dedicatory}
  {}{}
\patchcmd{\@maketitle}
  {\ifx\@empty\@date\else \@footnotetext{\@setdate}\fi}
  {}{}{}
\makeatother

\renewcommand{\backreftwosep}{\backrefsep}
\renewcommand{\backreflastsep}{\backrefsep}
\renewcommand*{\backref}[1]{}

\AtBeginDocument{
   \def\MR#1{}
}

\renewcommand{\leq}{\leqslant}
\renewcommand{\geq}{\geqslant}



\theoremstyle{plain}
\newtheorem{theorem}{Theorem}
\newtheorem{corollary}[theorem]{Corollary}
\newtheorem{lemma}[theorem]{Lemma}
\newtheorem{prop}[theorem]{Proposition}
\newtheorem{proposition}[theorem]{Proposition}
\newtheorem{claim}[theorem]{Claim}
\newtheorem{conjecture}[theorem]{Conjecture}

\newtheorem*{namedtheorem}{\theoremname}
\newcommand{\theoremname}{testing}
\newenvironment{named}[1]{\renewcommand{\theoremname}{#1}\begin{namedtheorem}}{\end{namedtheorem}}

\theoremstyle{definition}
\newtheorem{define}{Definition}[theorem]
\newtheorem{definition}[theorem]{Definition}
\newtheorem{question}[theorem]{Question}
\theoremstyle{remark}
\newtheorem{example}{Example}
\newtheorem*{remark}{Remark}



\newcommand{\R}{\mathbb{R}}
\newcommand{\F}{\mathcal{F}}
\newcommand{\fix}{\mathrm{Fix}}
\newcommand{\id}{\mathrm{id}}
\newcommand{\qandq}{\quad \text{and} \quad}



\begin{document}

\title[Fixed points of arbitrarily deep 1-dimensional neural networks]{Fixed points of arbitrarily deep 1-dimensional neural networks}
\author[A.\ Cook]{Andrew Cook$^{\dagger,\ast,1}$}
\thanks{This research is supported by an Australian Government Research Training Program (RTP) Scholarship.}
\author[A.\ Hammerlindl]{Andy Hammerlindl$^{\ast,2}$}
\author[W.\ Tucker]{Warwick Tucker$^{\ast,3}$}

\address{$^{\dagger}$Corresponding author.}
\address{$^{\ast}$School of Mathematics, Monash University, Victoria 3800 Australia.}
\address{Email addresses:}
\address{$^1$\href{mailto:andrew.cook@monash.edu}{andrew.cook@monash.edu}, $^2$\href{mailto:andy.hammerlindl@monash.edu}{andy.hammerlindl@monash.edu}, $^3$\href{mailto:warwick.tucker@monash.edu}{warwick.tucker@monash.edu}.}

\date{March 22, 2023}

\begin{abstract}
In this paper, we introduce a new class of functions on $\R$ that is closed under composition, and contains the logistic sigmoid function. We use this class to show that any 1-dimensional neural network of arbitrary depth with logistic sigmoid activation functions has at most three fixed points. While such neural networks are far from real world applications, we are able to completely understand their fixed points, providing a foundation to the much needed connection between application and theory of deep neural networks.
\end{abstract}

\maketitle

\section{Introduction}

Autoencoders have been part of the deep learning literature since the 1980s \citep{Goodfellow-et-al-2016, 1993hinton}. Autoencoders were initially popular in dimensionality reduction and image processing, while in the last decade, they have been used in speech recognition \citep{2020yang} and as generative models \citep{2014kingma}.

As the name suggests, an {\em autoencoder} is a neural network designed to learn a compression algorithm on a data set. This is achieved by composing an {\em encoder}, $f$, and a {\em decoder}, $g$, where $f$ maps its input to a lower dimensional {\em code}, and $g$ maps the code back to the dimension of the original input. The composition $g \circ f$ is then trained to approximate the identity function on the input data set. Since the raw data does not need to be labelled, autoencoders are an example of an unsupervised learning model.

Designing a neural network involves many decisions: the architecture of the neural network, what loss function and optimisation algorithm to use in training, what activation function to use, and choosing values to assign to other hyperparameters, such as the learning rate. Understanding the effect that these choices have on the dynamics of the training algorithm and the neural network itself could greatly assist the design process, leading to more reliable neural networks in realistic applications.

Autoencoders are examples of dynamical systems---they are continuous maps from $\mathbb{R}^n$ to itself---but there are few studies looking at them from this perspective, and those that do are largely experimental and less theoretical \citep{2021piotrowski, 2022dacruz}. This opens up many dynamics related questions as to how to measure the performance of an autoencoder using a loss function. As when studying any dynamical system, a good place to start is to understand the number and nature of the fixed points and periodic points of the system.

Neural networks used in reality are often required to be very wide in order to approximate some complicated function. Studying the dynamics of such models is very difficult due to their complexity.

In this paper, we study the dynamics of a very simple model of arbitrary depth, but containing only one neuron in each layer. The complexity of such a model is far from that of any neural network used in realistic applications. However, limiting our scope to this simpler problem meant that we were able to solve it completely. Understanding the dynamics of this simple example may play a part in solving more ambitious problems concerning the dynamics of wider neural networks.

Our main result states that any arbitrarily deep neural network of width 1 with logistic sigmoid activation functions can have at most three fixed points. We also introduce a new class $\F$ of differentiable functions on $\R$ that includes the logistic sigmoid function and is closed under composition. While this theory is motivated by problems in neural networks, it also has its place in more pure areas of dynamics.



\section{Statement of main results and definitions}

Mathematically, a neural network whose layers are all one neuron wide is a composition of affine and logistic sigmoid functions. In this paper, an {\bf affine function} is a function $A: \R \to \R$ of the form
\[A(x) \coloneqq a x + b,\]
\noindent
for constants $a, b \in \R$, and $\sigma : \R \to \R$ denotes the {\bf logistic sigmoid} function,
\[\sigma(x) \coloneqq \frac 1 {1 + e^{-x}}.\]

\noindent
An important special case of our main result is the following.

\begin{theorem} \label{thm:main}
For any choice of affine functions $A_1, \ldots, A_n$ with $n > 1$, a function $\Phi$ of the form
\[\Phi \coloneqq A_n \circ \sigma \circ A_{n-1} \circ \sigma \circ \cdots \circ \sigma \circ A_2 \circ \sigma \circ A_1\]
has at most three fixed points.
\end{theorem}

To prove this, we introduce a set $\F$ of $C^1$ functions of the form $f: \R \to \R$ with $f'(x) \neq 0$ for all $x \in \R$. Unless otherwise stated, all functions in this paper will be of this form. To define $\F$, we first need the following definition.

If a function $f$ has two fixed points, $x$ and $y$ such that
\begin{equation} \label{equn:coexpanding}
f'(x)f'(y) > 1,
\end{equation}
then $x$ and $y$ are said to be {\bf coexpanding fixed points} of $f$.

We may generalise this definition to any two points $x, y \in \R$ as follows. Define the function $\chi_f: \R^2 \setminus \Delta \to \R$ ,
\begin{equation*}
\chi_f(x, y) \coloneqq f'(x) f'(y) \left( \frac{x - y}{f(x) - f(y)} \right)^2.
\end{equation*}
\noindent
We say $x$ and $y$ (which are not necessarily fixed points) are {\bf coexpanding} if $\chi_f(x, y) > 1$.

In other words, $x$ and $y$ are coexpanding if there is an affine function $A:\R \to \R$ such that $x$ and $y$ are coexpanding fixed points of $A \circ f$.

If $f$ does not have coexpanding points, it is {\bf nowhere coexpanding}. Define $\F$ to be the set of all nowhere coexpanding functions. Observe that all non-constant affine functions are in $\F$, since $\chi_A(x, y) = 1$ for any such $A$ and distinct $x, y \in \R$.

Section \ref{sec:coexpanding} shows that $\F$ is closed under composition and classifies all possible sets of fixed points for functions in $\F$. Let $\mathrm{Fix}(f) \coloneqq \{x \in \mathbb{R}: x = f(x)\}$. This is summarised in our main result:

\begin{theorem} \label{thm:fix}
If $f \in \F$, then either $\fix(f)$ is a closed interval, or $f$ has at most three fixed points.
\end{theorem}

To prove Theorem \ref{thm:main}, it suffices to show that the logistic sigmoid function $\sigma$ is in $\F$ and that $\fix(\Phi)$ has no interior. To this end, we narrow our study to $C^3$ functions in order to use Schwarzian derivatives. Section \ref{sec:schwarzian} links the existing dynamics literature on Schwarzian derivatives with our class $\F$ in the following result.

\begin{theorem} \label{thm:characterization}
Let $g:\R \to \R$ be a $C^3$ function with non-vanishing derivative. Then $g \in \F$ if and only if its Schwarzian derivative satisfies $S_g(x) \leq 0$ for all $x \in \R$.
\end{theorem}
By calculating Schwarzian derivatives and applying Theorem \ref{thm:characterization}, we can show that the functions $\sigma$, $\tanh$, $\exp$, $\mathrm{erf}$, and $\arctan$ are in $\F$.

Finally, in Section \ref{sec:example}, we conclude with an example showing that $\F$ is not closed under addition. This demonstrates that more work is needed to study the fixed points of more general neural networks.



\section{Fixed points of nowhere coexpanding functions} \label{sec:coexpanding}

In this section, we build a better understanding of the properties of nowhere coexpanding functions. We begin by showing that $\F$ is closed under composition. Following this, we prove a list of necessary conditions for the nowhere coexpanding property. Finally, we use these conditions to prove Theorem \ref{thm:fix}.

\begin{proposition} \label{prop:compositions}
The set $\mathcal{F}$ of nowhere coexpanding functions is closed under composition.
\end{proposition}

To show this, we need the following lemma.

\begin{lemma} \label{lem:affine_composition}
For any non-constant affine functions $A$ and $B$, a function $f$ is nowhere coexpanding if and only if $A \circ f \circ B$ is nowhere coexpanding.
\end{lemma}

\begin{proof}
By the definition of a nowhere coexpanding function, $f \in \F$ if and only if $A \circ f \in \F$. Therefore, it suffices to show that a function $g$ is in $\F$ if and only if $h \coloneqq g \circ B$ is in $\F$. But these are equivalent, since a quick calculation shows that $\chi_{h}(x, y) = \chi_g(B(x), B(y))$ for all distinct $x$ and $y$.
\end{proof}

\begin{proof}[Proof of Proposition \ref{prop:compositions}]
Let $f, g \in \mathcal{F}$, and let $A$ be an affine function such that $h \coloneqq A \circ g \circ f$ has distinct fixed points $x_1$ and $x_2$. Let $y_1 = f(x_1)$, and $y_2 = f(x_2)$, and let $B$ be the affine function taking $x_1$ to $y_1$, and $x_2$ to $y_2$. Now define $\hat{g} \coloneqq A \circ g \circ B$ and $\hat{f} \coloneqq B^{-1} \circ f$, so $h = \hat{g} \circ \hat{f}$. Observe that $x_1$ and $x_2$ are fixed points of $\hat{f}$ and $\hat{g}$. By Lemma \ref{lem:affine_composition}, $\hat{f}, \hat{g} \in \mathcal{F}$, so
\[\hat{f}'(x_1) \hat{f}'(x_2) \leq 1 \qandq \hat{g}'(x_1) \hat{g}'(x_2) \leq 1,\]
and therefore
\[h'(x_1) h'(x_2) \leq 1.\qedhere\]
\end{proof}

We now establish some useful properties of nowhere coexpanding functions.

\begin{lemma} \label{lem:loc_min}
If $f \in \F$, then $f'$ has no local minima.
\end{lemma}

\begin{proof}
Suppose $f'$ has a local minimum at $p$. Then there exist points $a$ and $b$ with $a < p < b$, and such that $f'(a) = f'(b) > f'(x)$ for all $x$ in $(a, b)$. By the mean value theorem,
\[f'(a) > \frac{f(a) - f(b)}{a - b},\]
and therefore
\[f'(a) f'(b) \left(\frac{a - b}{f(a) - f(b)}\right)^2 > 1.\]
Thus $\chi_f(a, b) > 1$, showing that $f \notin \F$.
\end{proof}

\begin{lemma} \label{lem:contr_sandwich}
For a function $f \in \F$, if $\fix(f)$ has no interior and $a < b < c$ are fixed points, then $f'(b) > 1$.
\end{lemma}

\begin{proof}
Suppose $f'(b) \leq 1$. As $[a, b]$ and $[b, c]$ are invariant under $f$, $f'$ must average to unity on these intervals. Since $[a, b]$ and $[b, c]$ cannot be in $\fix(f)$, there exist points $p \in (a, b)$ and $q \in (b, c)$ with $f'(p) > 1$ and $f'(q) > 1$, yielding a local minimum in $f'$ in $(p, q)$ which contradicts Lemma \ref{lem:loc_min}.
\end{proof}

\begin{lemma} \label{lem:discrete}
If $f \in \F$ and $\fix(f)$ has no interior, then $f$ has at most three fixed points.
\end{lemma}

\begin{proof}
If $a$ and $b$ are fixed points, then by Lemma \ref{lem:contr_sandwich}, for any fixed point $p \in (a, b)$, $f'(p) > 1$. Since expanding fixed points cannot be adjacent, there is at most one fixed point in $(a, b)$. As $a$ and $b$ were arbitrarily chosen, $f$ cannot have more than three fixed points.
\end{proof}

\begin{lemma} \label{lem:one_eared_cat}
If $f \in \F$ and $\fix(f)$ has non-empty interior, then $f'(x) \leq 1$ for all $x \in \R$.
\end{lemma}

\begin{proof}
As $\fix(f)$ has an interior, there is an interval $[a, b] \subset \fix(f)$. Since $\F$ contains non-constant affine transformations, it suffices to consider the case where $a < 0 < b$. The result is immediate for $x \in [a, b]$. For $x \in [b, \infty)$,
\begin{equation} \label{equn:di}
\chi_f(0, x) = f'(0) f'(x) \left( \frac{x}{f(x)} \right)^2 \leq 1 \quad\text{and so}\quad f'(x) \leq \left( \frac{f(x)}{x} \right)^2.
\end{equation}
Observe that the solution $u(x) = x$ to the corresponding initial value problem,
\[u'(x) = \left( \frac{u(x)}{x} \right)^2, \quad u(b) = b\]
is unique. It is a standard result (see Chapter 3 of \citep{2002hartman} for instance) that since $f$ satisfies the differential inequality (\ref{equn:di}), it is majorised by $u$. That is, $f(x) \leq x$ for all $x \in [b, \infty)$ and so $f'(x) \leq 1$ follows from (\ref{equn:di}). A similar argument may be used for the case $x \in (-\infty, a]$.
\end{proof}

It is now easy to prove Theorem \ref{thm:fix}.

\begin{proof}[Proof of Theorem \ref{thm:fix}]
If $\fix(f)$ has non-empty interior, then by Lemma \ref{lem:one_eared_cat}, $f'(x) \leq 1$ for all $x \in \R$, and so $\fix(f)$ is connected. Otherwise, Lemma \ref{lem:discrete} applies.
\end{proof}



\section{The Schwarzian derivative} \label{sec:schwarzian}

In this section, we use Schwarzian derivatives to characterise $C^3$ functions in $\F$. We further show that the logistic sigmoid, $\sigma$ is in $\F$, from which it follows that $\Phi$ in Theorem \ref{thm:main} is in $\F$. We then use this to prove Theorem \ref{thm:main}.

Define $C^3_N$ to be the class of $C^3$ functions with non-vanishing derivative. Recall the {\bf Schwarzian derivative} of a $C^3_N$ function $f$ is defined by
\[S_f(x) = \frac{f'''(x)}{f'(x)} - \frac 3 2 \left(\frac{f''(x)}{f'(x)}\right)^2.\]

In complex analysis, the Schwarzian derivative measures how much a function differs from a M{\"o}bius transformation. In the realm of functions on $\mathbb{R}$, the Schwarzian derivative vanishes for affine functions, and is therefore a measure of how much a function varies from being affine. The following are standard results, which we will use in the proof of Theorem \ref{thm:characterization}. For details on Lemma \ref{lem:S6DF}, see for instance \cite{MR190326}.

\begin{lemma} \label{lem:S6DF}
Given a $C^3_N$ function $f$, let
\begin{equation*}
U_f(x, y) \coloneqq \frac{\partial^2}{\partial x \partial y} \log\left|\frac{f(x) - f(y)}{x - y}\right|,
\end{equation*}
for all distinct $x, y \in \R$.
Then
\[S_f(x) = 6 \lim_{y \to x} U_f(x, y),\]
for all $x \in \mathbb{R}$.
\end{lemma}

The next lemma states the chain rule for Schwarzian derivatives as well as some properties that follow from it. The proof is omitted.

\begin{lemma} \label{lem:chain}
For $C^3_N$ functions $f$ and $g$,
\begin{enumerate}
\item For all $x \in \R, \quad S_{g \circ f}(x) = S_g(f(x)) (f'(x))^2 + S_f(x).$
\item If $S_f(x) \leq 0$ and $S_g(x) \leq 0$, for all $x \in \mathbb{R}$, then $S_{g \circ f}(x) \leq 0$ for all $x \in \mathbb{R}$.
\item Suppose $g = B \circ f \circ A$, for non-constant affine functions $A$ and $B$. Then $S_f(x) \leq 0$ for all $x \in \mathbb{R}$ if and only if $S_g(x) \leq 0$ for all $x \in \mathbb{R}$.
\end{enumerate}
Properties (2) and (3) still hold if all the inequalities are changed to be strict.
\end{lemma}

The following lemma is adapted from Chapter 9.4 of \cite{MR3012659}.

\begin{lemma} \label{lem:singer}
Let $f$ be a $C^3_N$ function with fixed points $a < b$. If $f'(a) > 1$ and $f'(b) > 1$, then there exists $p \in (a, b)$ such that $S_f(p) > 0$.
\end{lemma}

\begin{proof}
Assume $S_f(x) \leq 0$ for all $x \in (a, b)$. Since $a$ and $b$ are fixed points, $f$ cannot be expanding for all $x$ in $(a, b)$. Thus, there exists $c \in (a, b)$ such that $f'(c) < 1$. Define $g(x) \coloneqq \frac{d}{dx} \log(f'(x)) = \frac{f''(x)}{f'(x)}$ for all $x$. By the mean value theorem applied to $\log(f'(x))$ on $[a, c]$, there exists $r \in (a, c)$ such that $g(r) < 0$. Similarly, there exists $t \in (c, b)$ such that $g(t) > 0$, and hence there exists $s \in (r, t)$ such that $g(s) = 0$. A simple calculation shows $g'(x) = S_f(x) + \frac 1 2 (g(x))^2$, and by our assumption, $g$ satisfies the differential inequality
\[g'(x) \leq \frac 1 2 (g(x))^2, \quad g(s) = 0.\]
Similar to the proof of Lemma \ref{lem:one_eared_cat}, it follows that $g(x) \leq 0$ for $x \in [s, b]$. This contradicts $g(t) > 0$, so our initial assumption was wrong.
\end{proof}

We are now ready to prove Theorem \ref{thm:characterization}.

\begin{proof}[Proof of Theorem \ref{thm:characterization}]
Suppose $f \in \F$. Evaluating $U_f$ in Lemma \ref{lem:S6DF} yields
\[(x - y)^2 U_f(x, y) = \chi_f(x, y) - 1.\]
Since $\chi_f(x, y) \leq 1$ for all distinct $x$ and $y$, $(x - y)^2 U_f(x, y) \leq 0$, so $U_f(x, y) \leq 0$. Therefore by Lemma \ref{lem:S6DF}, $S_f(x) \leq 0$.

Now suppose $f \notin \F$, so that $A \circ f$ has coexpanding fixed points $a < b$ for some affine function $A$. Let $B$ be the affine function interchanging $a$ and $b$. Then $g \coloneqq A \circ f \circ B \circ A \circ f \circ B$ satisfies $g'(a) > 1$ and $g'(b) > 1$. By Lemma \ref{lem:singer}, there exists $p \in (a, b)$ such that $S_g(p) > 0$. Therefore, by items (2) and (3) of Lemma \ref{lem:chain}, $S_f(q) > 0$ for some $q$.
\end{proof}

With this, the proof of Theorem \ref{thm:main} is straightforward.

\begin{proof}[Proof of Theorem \ref{thm:main}]
The logistic sigmoid function, $\sigma$ satisfies $S_{\sigma}(x) = -\frac 1 2$ for all $x$. It follows by Lemma \ref{lem:chain} (with strict inequalities) that $S_{\Phi}(x) < 0$ for all $x$, and by Theorem \ref{thm:characterization}, $\Phi \in \F$. Since $S_{\Phi}$ is never zero, $\fix(\Phi)$ has no interior, and so by Theorem \ref{thm:fix}, $\Phi$ has at most three fixed points.
\end{proof}



\section{Examples}

Theorem \ref{thm:fix} allows for functions $f \in \F$ with zero, one, two, three, or infinitely many fixed points, where in the last case, $\fix(f)$ is some interval. In this section, we provide examples of functions in $\F$ for each of these cases.

\begin{example} \label{eg:finite}
The following functions in $\F$ have finitely many fixed points. In each case the number of fixed points is robust under $C^1$ small perturbations.

\begin{figure}[h]
\begin{centering}
\begin{tabular}{c c c c}
\includegraphics[width=0.2\linewidth]{eg0.eps}
&
\includegraphics[width=0.2\linewidth]{eg1.eps}
&
\includegraphics[width=0.2\linewidth]{eg2.eps}
&
\includegraphics[width=0.2\linewidth]{eg3.eps}
\\
$f_0(x) = x + 1$ & $f_1(x) = 2 x$ & $f_2(x) = e^x - 2$ & $f_3(x) = \tanh(2 x)$\\
Zero fixed points & One fixed point & Two fixed points & Three fixed points\\
(a) & (b) & (c) & (d)
\end{tabular}
\end{centering}
\caption{Functions in $\F$ plotted with the diagonal in grey.} \label{fig:0to3}
\end{figure}

\noindent Theorem \ref{thm:characterization} shows that the above functions are in $\F$.
\end{example}

We now introduce another way of constructing new functions in $\F$ from other known functions in $\F$. We say a function $f \in \F$ is {\bf glueable} if $f'(0) = 1$ and $|f(x)| \leq |x|$ for all $x$. For glueable functions $f$ and $g$, define
\[(f \star g)(x) \coloneqq \begin{cases} 
    f(x), & x \leq 0\\
    g(x), & x \geq 0
  \end{cases}.
\]

\begin{lemma} \label{lem:glue}
If $f$ and $g$ are glueable, then $f \star g \in \F$.
\end{lemma}

\begin{proof}
Take $x, y > 0$. By the definition of $f \star g$, we may assume without loss of generality that $f$ is odd. It suffices to show $\chi_{f \star g}(-x, y) \leq 1$. Since $f$ and $g$ are glueable, $x f(x) g(y) \leq x f(x) y$ and $y f(x) g(y) \leq y x g(y)$. Adding these together yields $(x + y) f(x) g(y) \leq x y (f(x) + g(y))$, and thus
\[\frac{f(x) g(y)}{f(x) + g(y)} \frac{x + y}{x y} \leq 1, \qandq \left(\frac{f(x)} x \right)^2 \left(\frac{g(y)} y \right)^2 \left(\frac{x + y}{f(x) + g(y)} \right)^2 \leq 1.\]
Since $f, g \in \F$,
\[\chi_f(0, y) = f'(0) f'(y) \left(\frac y {f(y)} \right)^2 \leq 1 \Rightarrow f'(y) \leq \left(\frac{f(y)} y \right)^2\]
and the same for $g$. Therefore
\[\chi_{f \star g}(-x, y) \leq f'(x) g'(y) \left(\frac{x + y}{f(x) + g(y)} \right)^2 \leq 1.\qedhere\]
\end{proof}

\begin{example}
Here, we show how to construct a function $h \in \F$ such that $\fix(h) = [a, b]$ for any prescribed, possibly unbounded, interval $[a, b]$.

Since $\F$ is closed under composition with affine functions, without loss of generality, we only need to find examples of $h$ where $\fix(h)$ is $\R$, $(-\infty, 0]$, and $[0, 1]$. We saw in Example \ref{eg:finite} that $\tanh \in \F$. Observe that $\tanh$ is also glueable. Furthermore, the identity function, $\id$, is the only glueable affine function. We treat each case of $\fix(h)$ separately.

\begin{enumerate}
\item Observe $h = \id \in \F$ is the only choice of $h$ satisfying $\fix(h) = \R$.
\item By Lemma \ref{lem:glue}, $\id \star \tanh \in \F$, so $h = \id \star \tanh$ is a suitable choice such that $\fix(h) = (-\infty, 0]$.
\item Define $g_a(x) \coloneqq (\id \star \tanh)(x - a) + a$ and choose $h = \tanh \star g_1$. Then $h \in \F$ and $\fix(h) = [0, 1]$.
\end{enumerate}
\end{example}



\section{Sums of nowhere coexpanding functions} \label{sec:example}

In real world applications, neural networks have many neurons in each layer, and hence take sums of activation functions. We conclude by demonstrating with an example that $\F$ is not closed under addition. More work is needed to analyse the fixed points of such neural networks.

\begin{example} \label{eg:no_sums}
Consider the function
\[f(x) \coloneqq \tanh(4 x) + \tanh(x/4),\]
shown in Figure \ref{fig:tanh_counterexample}.

\begin{figure}[H]
\begin{center}
\includegraphics[width=0.6\linewidth]{tanh_counterexample.eps}
\end{center}
\caption{The function $f(x) = \tanh(4 x) + \tanh(x/4)$ from Example \ref{eg:no_sums}.} \label{fig:tanh_counterexample}
\end{figure}

\begin{figure}[H]
\begin{center}
\includegraphics[width=0.57\linewidth]{composition_counterexample.eps}
\end{center}
\caption{A plot of the composition $x \mapsto 4 f(f(x + s) - 2 s) + s + 4$ with parameter $s = 0.94$ plotted with the diagonal to show where the five fixed points are.} \label{fig:composition_counterexample}
\end{figure}
\end{example}

\noindent Evaluating $S_{f}$ at $x = 1$ yields $S_{f}(1) > 1$, so by Theorem \ref{thm:characterization}, $f \notin \F$. A plot of $f$ composed with itself and affine functions is shown in Figure \ref{fig:composition_counterexample}. The function is scaled with affine functions to show how extra fixed points can be obtained.

A bounded $C^2$ function $f$ is {\bf sigmoidal} if $f'(x) > 0$ for all $x$, and $f$ has exactly one inflection point \citep{1995han}. There are many examples of sigmoidal functions in $\F$, such as $\tanh$, $\sigma$, $\mathrm{erf}$, and $\arctan$. The function in Example \ref{eg:no_sums} is also sigmoidal, but is not in $\F$.

\bibliographystyle{abbrvnat}
\bibliography{report}

\end{document}
