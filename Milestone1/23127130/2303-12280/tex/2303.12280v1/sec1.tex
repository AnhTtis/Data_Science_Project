\begin{figure}[tb]
\centering
\includegraphics[width=0.45\textwidth]{figure/nlos_setup.pdf}
\caption{(a) Confocal NLOS setup, (b) SPAD system and measured transients, (c) NLOS surface reconstruction with DLCT \cite{Young2020} and (d) with our NLOS-NeuS. Neural implicit surface representation can enhance quality of reconstructed surfaces in NLOS scenes.}
\label{fig:nlos_setup}
\end{figure}

\section{Introduction} \label{sec:introduction}
Many computer-vision applications are being integrated into society.
However, most are intended for visible scenes from cameras.
Non-line-of-sight (NLOS) imaging \cite{Kirmani2011,Velten2012} is used to infer invisible scenes occluded from the cameras.
Figure \ref{fig:nlos_setup}(a) shows a typical confocal NLOS setup \cite{OToole2018}, in which an ultra-fast light source and time-resolved detector are collocated and the detector can only see a diffuse wall.
Pulsed light from the light source is emitted into the visible wall, and reflected light on the wall reaches an NLOS scene.
Reflected light in the NLOS scene then bounces back to the detector via the relay wall.
A time-resolved detectors for such NLOS imaging is a single photon avalanche diode (SPAD) \cite{Buttafava2015} (top in Fig \ref{fig:nlos_setup}(b)).
A SPAD is capable of time-correlated single photon counting with pico-second resolution \cite{Warburton2016}.
Measured data with such a time-resolved detector are called transients, in which the number of counted photons or intensity is recorded at each arrival time (bottom in Fig. \ref{fig:nlos_setup}(b)).
One approach for NLOS imaging is to use the transients on the relay wall as input.
We aim to reconstruct three-dimensional (3D) surfaces in NLOS scenes from the transients, similar to previous studies \cite{Iseringhausen2020,Plack2023,Tsai2017,Tsai2019,Xin2019,Young2020}.

We extend the neural transient field (NeTF) \cite{Shen2021} for neural implicit surface representation in NLOS scenes.
The neural radiance field (NeRF) \cite{Mildenhall2020} is a powerful paradigm for scene representation, in which density and view-dependent color at each scene position are continuously parameterized with a multi-layer perceptron (MLP), which is optimized by reconstructing input multi-view images with volume rendering.
The NeTF is the extended version of the NeRF for transient measurement in an NLOS setup.
However, the NeTF (and also the NeRF) does not support geometric representation, which is necessary for accurate 3D reconstruction.
We incorporate the framework of a neural implicit surface with volume rendering such as NeuS \cite{Wang2021neus} and VolSDF \cite{Yariv2021} into the NeTF for NLOS neural implicit surface reconstruction, i.e., {\it NLOS-NeuS}.

We argue that a simple extension of the NeTF to neural implicit surface representation with a signed distance function (SDF) causes a non-zero level-set surface due to an under-constrained NLOS setup.
We introduce effective training losses to avoid a non-zero level-set surface.
Figure \ref{fig:nlos_setup}(d) shows an example of the surface reconstruction of an occluded object with NLOS-NeuS.
Compared with the results of the state-of-the art NLOS surface-reconstruction method \cite{Young2020} (Fig. \ref{fig:nlos_setup}(c)), the fine details of the target object can be reconstructed because of the nature of its continuous representation.
To the best of our knowledge, this is the first study on a neural implicit surface with volume rendering in NLOS scenes.