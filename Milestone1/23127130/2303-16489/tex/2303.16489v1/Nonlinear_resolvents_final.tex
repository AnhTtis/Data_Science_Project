\documentclass[a4paper,reqno,11pt]{amsart}
%\nonstopmode
\usepackage[english]{babel}
%\usepackage[T1]{fontenc}
%\usepackage[ansinew]{inputenc}
\usepackage{geometry}
%\usepackage{xcolor} 
%\usepackage{ulem}
\geometry{a4paper,left=20mm,right=20mm, top=20mm, bottom=20mm} 
\usepackage{mathrsfs}

%\usepackage[dvipdfmx, colorlinks=true, linkcolor=blue, citecolor=blue, draft=false, bookmarks, bookmarksnumbered=true, plainpages=true]{hyperref}

%\usepackage{lmodern}

\usepackage{graphicx}

\usepackage{amsmath} \numberwithin{equation}{section}
\usepackage{amsthm}
\usepackage{amsfonts}
\usepackage{amssymb}
\usepackage{graphics}
\usepackage{color}
\usepackage{cases}
\usepackage{enumerate}

% theorems
%\newtheorem{satz}{Theorem}
%\renewcommand{\thesatz}{\Alph{satz}}
\theoremstyle{plain}
\newtheorem{theorem}{Theorem}[section]
\newtheorem{lemma}[theorem]{Lemma}
\newtheorem{corollary}[theorem]{Corollary}
\newtheorem{question}[theorem]{Question} 
\newtheorem{proposition}[theorem]{Proposition}    
\theoremstyle{definition}
\newtheorem{definition}[theorem]{Definition}
\newtheorem{example}[theorem]{Example}
\newtheorem{remark}[theorem]{Remark}



% proofs
\newcommand\proofsymbol{\frame{\rule[0pt]{0pt}{8pt}\rule[0pt]{8pt}{0pt}}}

% math
\newcommand\with{\ \vrule\ }  % 'mit'-Symbol in Mengen
\newcommand\abs[1]{\left|#1\right|} % Absolutbetrag
\newcommand\C{\mathbb C}         % K\UTF{FFFD}rper der komplexen Zahlen
\newcommand\R{\mathbb R}         % K\UTF{FFFD}rper der reellen Zahlen
\newcommand\Z{\mathbb Z}         % Ring der ganzen Zahlen
\newcommand\N{\mathbb N}         % Ring der ganzen Zahlen
\newcommand{\uhp}{{\mathbb H}}
\newcommand{\Nev}{{\mathcal N}}
\newcommand{\G}{{\mathscr G}}      
%\newcommand{\G}{{\mathcal G}}      
\newcommand\D{\mathbb D}  
\newcommand\T{\mathbb T}  
\newcommand{\meu}{\alpha} 
\newcommand{\til}{\mu} 
\renewcommand\Im{\,{\mathsf{Im}}\,}
\renewcommand\Re{\,{\mathsf{Re}}\,}
\newcommand{\cP}{{\rm\bf P}}  % Probability measures
\newcommand{\aand}{{\quad\text{and}\quad}}



% other
\newcommand{\eps}{\varepsilon}

\newcommand{\LandauO}{\mathcal{O}} 
\newcommand{\Landauo}{{\scriptstyle\mathcal{O}}}
\newcommand{\diam}{\operatorname{diam}}
\newcommand{\dist}{\operatorname{dist}}
\newcommand{\PSL}{\operatorname{PSL}}
\newcommand{\B}{\mathbb B}
\newcommand{\sphere}{{\widehat{\mathbb C}}}

\newcommand{\supp}{\operatorname{supp}} 

%\usepackage[pdfstartview=FitH]{hyperref}

%\allowdisplaybreaks[1]




\title{Nonlinear resolvents and decreasing Loewner chains}
\author[I. Hotta]{Ikkei Hotta}
\author[S. Schlei{\ss}inger]{Sebastian Schlei{\ss}inger}
\author[T. Sugawa]{Toshiyuki Sugawa}
\subjclass[2020]{Primary 37L05, Secondary 30C45, 46L54}
\keywords{nonlinear resolvents, semigroups, decreasing Loewner chains, free convolution}
\thanks{The first author was supported by JSPS KAKENHI Grant no.20K03632}
\thanks{The second author was supported by the German Research Foundation (DFG), project no. 401281084}
\thanks{The third author was supported by JSPS KAKENHI Grant Number JP17H02847}




\begin{document}
\parindent 0pt 



 
\maketitle

\begin{center}\textit{Dedicated  to  the  memory  of  Gabriela Kohr.}\end{center}

\vspace{5mm}

 \begin{abstract}
In this article we prove that nonlinear resolvents of infinitesimal generators on bounded and
convex subdomains of $\C^n$ are decreasing Loewner chains. Furthermore, we consider the problem of the existence of nonlinear resolvents on unbounded convex domains in $\C$. In the case of the upper half-plane, we obtain a complete solution by using that nonlinear resolvents of certain generators correspond to semigroups of probability measures with respect to free convolution.
 \end{abstract}



 
% 
 \tableofcontents
% 
% \parindent 0pt
% 
  


\section{Introduction}

Let $D\subset \C^n$ be a complete (Kobayashi) hyperbolic domain. The (decreasing) Loewner partial differential equation on $D$ has the form
\begin{equation}\label{intro_1_eq}
 \frac{\partial}{\partial t}f_t(z)=f'_t(z)\cdot G(t,z)\quad \text{for a.e.\ $t\geq 0$}, \quad f_0(z)=z\in D,
\end{equation}
where $f'_t$ denotes the Jacobian of $f_t$ and $G(t,z)$ is a Herglotz vector field on $D$, i.e.\ $z\mapsto G(t,z)$ is an infinitesimal generator of a semigroup on $D$
for almost every $t\geq0$ and $(t,z)\mapsto G(t,z)$ satisfies certain regularity conditions. Equation \eqref{intro_1} has a unique solution of univalent mappings $f_t:D \to D$, which form a decreasing Loewner chain, i.e.\ $f_t(D)\subset f_s(D)$ whenever $s\leq t$, $f_0$ is the identity, and $t\mapsto f_t$ is continuous. This result is well-known, but as the literature usually focuses on increasing Loewner chains, we also provide a proof (Theorem \ref{thm_1}).

Now let us replace $G(t,z)$ by $G(t,f_t(z))$ to obtain the following modified equation:
\begin{equation}\label{intro_2_eq}
 \frac{\partial}{\partial t}f_t(z)=f'_t(z)\cdot G(t,f_t(z))\quad \text{for a.e.\ $t\geq 0$}, \quad f_0(z)=z\in D.
\end{equation}




At first sight, it seems unnatural to look at this equation, for, if $G$ is a generator on $D$ and $\varphi:D\to D$ a holomorphic self-mapping, then $G\circ \varphi$ need not be a generator.
On the other hand, equation \eqref{intro_2_eq} is simpler than \eqref{intro_1_eq} in the following sense. 
While the inverse functions $g_t=f_t^{-1}$ of the solution to \eqref{intro_1_eq} satisfy Loewner's ordinary differential equation 
\[\frac{\partial}{\partial t}g_t(z) = - G(t,g_t(z)),\] the inverse functions $g_t$ of the solution to \eqref{intro_2_eq}, if it exists, 
satisfy \[\frac{\partial}{\partial t}g_t(z) = - G(t,z),\quad \text{i.e.} \quad  g_t(z)=z-\int_0^t G(s,z)ds.\]
Note that we can define $g_t$ by this equation on the whole domain $D$.
Special cases of \eqref{intro_2_eq} appear in different contexts: theory of continuous semigroups of holomorphic functions (\cite{ESS20} and \cite{GKH}), limit of the radial/chordal multiple SLEs (\cite{dMS16}, \cite{dMHS18}, \cite{HS}) and semigroups of probability measures with respect to free convolution (\cite{HS}).
In this article we explain these relations in a more general setting.




In the simplest case, $G(t,z)$ does not depend on $t$, i.e.\ $G(t,z)=G(z)$ for an infinitesimal generator $G$ on $D$.
The solution to \eqref{intro_1_eq} is then simply the semigroup associated to $G$. 
By approximating the general version of \eqref{intro_1_eq} by a piecewise constant Herglotz vector field, we can think of the solution as an infinite composition of 
semigroup mappings which are infinitesimally close to the identity.

Equation \eqref{intro_2_eq}, on the other hand, is related to the nonlinear resolvents of $G$. 
Assume that $D$ is bounded and convex. It is well-known that the nonlinear resolvent equation (at time $t$)
\[w= z - tG(z)\] has a unique solution $f_t(z)$ for every $w\in D$ and every $t\geq 0$. Each $f_t$ is a univalent self-map of $D$.

\begin{theorem}\label{intro_1} Let $D\subset \C^n$ be a domain which is bounded and convex. 
\begin{itemize}
\item[(1)] Let $G$ be an infinitesimal generator on $D$.
Then there exists a unique solution $f_t$ to
\begin{equation}\label{auto} \frac{\partial}{\partial t}f_t(z)=f'_t(z)\cdot G(f_t(z)) \quad \text{for all $t\geq 0$}, \quad f_0(z)=z\in D.
\end{equation}
The mappings $f_t$ are the nonlinear resolvents of $G$ and they form a decreasing Loewner chain.
\item[(2)] More generally, let $G:[0,\infty)\times D\to \C^n$ be such that $z\mapsto G(t,z)$ is holomorphic for a.e.\ $t\geq0$, $t\mapsto G(t,z)$ is locally integrable for all $z\in D$, and $H_t(z):=\int_0^t G(s,z) ds$ 
is an infinitesimal generator on $D$ for every $t\geq 0$. Let $J_t$ be the nonlinear resolvent of $H_t$ at time $1$. Then $(J_t)_{t\geq 0}$ is the unique solution of holomorphic self-mappings of $D$, locally absolutely continuous in $t$, of
\[\frac{\partial}{\partial t}J_t(z)=J_t'(z)\cdot G(t,J_t(z))\quad \text{for a.e.\ $t\geq 0$}, \quad J_0(z)= z\in D.\]

\end{itemize}
\end{theorem}

In contrast to equation \eqref{auto}, the solution in Theorem \ref{intro_1} (2) is not necessarily a decreasing Loewner chain, see Example \ref{no_solution}.

%If $(t,z)\mapsto G(t,J_t(z))$ is a Herglotz vector field, then $(J_t)_{t\geq 0}$ is a decreasing Loewner chain.
%
%\begin{question}Under which conditions on $G(t,z)$ is $G(t,J_t(z))$ a Herglotz vector field?
%\end{question}
%
%If the solution to \eqref{intro_2_eq} is a decreasing Loewner chain, we can think of it as an infinite composition of 
%nonlinear resolvent mappings which are infinitesimally close to the identity.

Convexity of $D$ is a natural assumption for the resolvent equation. 
On the upper half-plane $\uhp$ -- which is unbounded, convex, and still complete hyperbolic -- 
the resolvent equation does not have solutions for all cases, see Example \ref{ex:2t}. The following theorem shows that this is rather an exceptional case.


\begin{theorem}\label{thm:unbdd}
Let $D$ be an unbounded convex domain in $\C$ with $D\ne\C$ and
$G:D\to\C$ be an infinitesimal generator of a semigroup $(F_t)$ on $D.$
If the Denjoy-Wolff point $\sigma$ of $(F_t)$ is finite, then
$G$ admits nonlinear resolvents $J_t$ on $D$ for all $t\ge 0.$
\end{theorem}

A definition of the Denjoy-Wolff point will be given in Section 2.
If the Denjoy-Wolff point is $\infty$, the question concerning the existence of nonlinear resolvents is more complicated. 
In Section \ref{sec_strip} we give a sufficient condition for generators on a horizontal strip having Denjoy-Wolff point $\infty$.
For the case $D=\uhp$, we obtain a complete characterization.

\begin{theorem}\label{intro_2}Let $G:\uhp\to\C$ be an infinitesimal generator on $\uhp$.
\begin{enumerate}

\item[(1)] Assume that $G(\uhp)\subset \uhp\cup \R$. (This is equivalent to either $G(z)\equiv 0$ or the associated semigroup has Denjoy-Wolff point $\infty$.) Let $b=\lim_{y\to\infty}G(iy)/(iy)$. Then
$b$ is a non-negative real number and the resolvent equation for $G$ can be solved if and only if $t\in[0,1/b)$.

If $b=0$, then the resolvents $f_t$ satisfy \[\frac1{f_t(z)}=\int_\R \frac1{z-x}\mu_t(dx),\] where $(\mu_t)_{t\geq0}$ is a semigroup of probability measures on $\R$ with respect to additive free convolution.
\item[(2)] In all other cases, the nonlinear resolvent of $G$ exists for every $t\ge0.$
	%Assume that $G$ generates a semigroup $(\alpha_t)_{t\geq0}$ of automorphisms of $\uhp$. Then the resolvents of $G$ exist for all $t\geq0$ if and only if $\alpha_1$ is not a hyperbolic automorphism whose Denjoy-Wolff point is $\infty$.
\end{enumerate}
\end{theorem}



%\begin{example}
%Let $D=\uhp$ and consider $G(t,z)=z$. \begin{equation}\label{inv_equ_6}
 %\frac{\partial}{\partial t}f_t(z)=Df_t(z)\cdot f_t(z), \quad f_0(z)=z\in D,
%\end{equation}
%$f_t(z) = \frac{z}{1-t}$
%\end{example}

The resolvents in Theorem \ref{intro_2} also satisfy \eqref{intro_2_eq}. Note that Theorem \ref{intro_2} (2) follows directly from Theorem \ref{thm:unbdd}.



For the case of Theorem \ref{intro_2} (1) we also prove that the non-autonomous version 
 has a solution.

\begin{theorem}\label{intro_3} Let $G(t,z)$ be a Herglotz vector field on $\uhp$ such that, for a.e.\ $t\geq0$, 
$z\mapsto G(t,z)$ maps $\uhp$ into $\uhp\cup \R$ and $\lim_{y\to\infty} G(t,iy)/y=0$. 
Then there exists a unique solution $(f_t)_{t\geq0}$, locally absolutely continuous in $t$, to
\begin{equation*} \frac{\partial}{\partial t}f_t(z)=f_t'(z)\cdot G(t,f_t(z))\quad \text{for a.e.\ $t\geq 0$}, \quad f_0(z)=z\in \uhp.
\end{equation*}
The solution can also be written as $1/f_t(z)=\int_\R \frac1{z-x}\mu_t(dx)$ for a family of freely infinitely divisible probability measures on $\R$ and $(f_t)_{t\geq0}$ forms a decreasing Loewner chain on $\uhp$.
\end{theorem}

%In Theorem \ref{intro_2}, the Denjoy-Wolff point is assumed to be either an interior of $\uhp$ or $\infty$. In case of the finite boundary Denjoy-Wolff point, we do not have any general result of existence of resolvents. As a partial answer to this problem, we will show the following facts.
%Remark that the elliptic case below is deduced as a direct corollary of Theorem \ref{intro_2} (1).
%
%
%\begin{theorem}
%\label{intro_3}
%Let $(F_t)_{t\ge0}$ be a continuous semigroup of automorphisms of $\uhp$ with infinitesimal generator $G.$
%Then the followings are true.
%\begin{enumerate}
%\item[(1)] If $(F_{t})_{t \ge 0}$ is hyperbolic, then its resolvent $J_t$ exists for all $t\ge0$ if and only if the Denjoy-Wolff point of the semigroup is finite, namely, lies on $\R$.
%\item[(2)] If $(F_{t})_{t \ge 0}$ is parabolic, then its resolvent $J_t$ exists for all $t\ge0$.
%\item[(3)] If $(F_{t})_{t \ge 0}$ is elliptic, then its resolvent $J_t$ exists for all $t\ge0$.
%\end{enumerate}
%\end{theorem}


The paper is organized as follows. 
\begin{itemize}
	\item In Section \ref{sec_semi} we briefly recall some facts on semigroups and nonlinear resolvents.
	\item The proof of Theorem \ref{thm:unbdd} is provided in Section \ref{sec_interior}.
	\item In Section \ref{sec_deLo} we look at the general theory of decreasing Loewner chains and we prove Theorem \ref{intro_1}.
	\item The proofs of Theorem \ref{intro_3} and part (1) of Theorem \ref{intro_2} are based on notions from free probability theory, which we discuss in Section \ref{sec_free}.
\end{itemize}





 

\section{Semigroups and nonlinear resolvents}\label{sec_semi}

Let $D\subset\C^n$ be a domain. 
For a holomorphic map $f:D\to \C^n$ we denote by $f'(z)$ its (complex) Jacobian matrix at $z$. 
For $n=1$, $f'(z)$ is the ordinary derivative of $f(z).$ 
We denote by $I:\C^n\to\C^n$ the identity mapping.

Let $(F_t)_{t\geq0}$ be a semigroup of holomorphic functions on $D$, i.e.\ $F_0$ is the identity, $F_{t+s}=F_s\circ F_t$, and 
$t\mapsto F_t$ is continuous with respect to locally uniform convergence on $D.$
In the sequel, we sometimes call it a \textit{semigroup} for short.
Then the locally uniform
limit \[\lim_{t\downarrow 0}\frac{F_t(z)-z}{t}=:G(z)\]
exists and the holomorphic function $G:D\to\C^n$ is called the \textit{infinitesimal generator}
(or, simply, the generator)
 of the semigroup, see \cite[Theorem 5]{Aba92}. 
(Sometimes, e.g. in \cite{ESS20}, the function $-G$ is called the infinitesimal generator.)
We denote by $\G(D)$ the set of infinitesimal generators on $D.$

The Berkson-Porta formula \cite{BP78} gives a form of infinitesimal generators on the unit disc $\D.$

\begin{lemma}\label{lem:BP}
Let $G\in\G(\D).$
Then, there exist a point $\tau\in \overline{\D} := \D \cup\partial \D$ and a holomorphic function $p:\D\to\C$ with $\Re(p)\geq0$ such that
\begin{equation}\label{Berkson-Porta}
 G(z) = (\tau-z)(1-\overline{\tau}z)p(z).
 \end{equation}
Conversely, any function $G(z)$ of the form \eqref{Berkson-Porta} is an infinitesimal generator on $\D.$
%Moreover,  $\Re p>0$ if and only if the semigroup does not consist of elliptic automorphisms when $|\tau|<1$ and it does
%not consist of parabolic automorphisms when $|\tau|=1.$
\end{lemma}


If $G(z)\not\equiv 0$, we call $\tau$ the \textit{Denjoy-Wolff point} of the semigroup $(F_t)_{t\geq0}$ generated by $G$. 
Unless $(F_t)_{t\geq0}$ consists of elliptic automorphisms, 
we have $F_t(z)\to \tau$ as $t\to\infty$ locally uniformly on $\D$. 
Here, we recall that an automorphism of $\D$ is called \textit{elliptic} if it has
an interiour fixed point in $\D.$
If $(F_t)_{t\geq0}$ does consist of elliptic automorphisms, then its Denjoy-Wolff point is defined to be
its unique common fixed point, i.e. $\tau\in\D$ and $F_t(\tau)=\tau$ for all $t\geq0$.
See \cite{BCDM20} for details.


Let $\varphi$ be a biholomorphic map of $\D$ onto a
a simply connected domain $D$ in the Riemann sphere $\sphere=\C\cup\{\infty\}.$
It is known that $\varphi$ extends to a continuous map $\overline\D$ to
$\overline D\subset\sphere$ if and only if $\partial D$ is locally connected (see e.g. \cite[Theorem 2.1]{Pom92}).
In this case, we can generalize the notion of the Denjoy-Wolff point as follows.
Let $(F_t)_{t\ge0}$ be a semigroup on $D.$
Suppose that the conjugated semigroup
$\hat F_t=\varphi^{-1}\circ F_t\circ\varphi$ on $\D$ has the Denjoy-Wolff
point at $\tau\in\overline\D.$
We may define $\sigma=\varphi(\tau)\in\overline D$ to be the Denjoy-Wolff point
of the semigroup $(F_t)$ on $D.$
Indeed, $F_t\to \sigma$ locally uniformly on $D$ as $t\to\infty$ unless $(F_t)$ consists
of automorphisms of $D$ with a common interior fixed point at $\sigma.$


We are interested in general characterizations of infinitesimal generators.
One of them is described in terms of nonlinear resolvents as an analogue of the Hille-Yosida theory for semigroups of linear operators on a Banach space.

\begin{definition}
Let $D$ be a domain in $\C^n$ and $G:D\to \C^n$ be a holomorphic map. 
Let $t\geq 0$. If the equation
\begin{equation}\label{nl} w = z - t\cdot G(z) \end{equation}
has a unique solution $z$ in $D$ for each $w\in D$, the mapping $z=J_t(w)=J_t(w,G)$ is called the \textit{nonlinear resolvent} (at ``time'' $t$) of $G$. 
\end{definition}
If $J_t$ exists, it is a holomorphic mapping from $D$ into itself. 

The following characterization of infinitesimal generators is due to Reich and Shoikhet
\cite[Theorem 1.1 and Corollary 1.2]{RS97}.

\begin{theorem}[\cite{RS97}]\label{thm:RS}
Let $D$ be a bounded convex domain in $\C^n.$
A holomorphic map $G:D\to\C^n$ admits nonlinear resolvents $z=J_t(w)$
in \eqref{nl} for all $t\ge0$
if and only if $G$ is an infinitesimal generator of a semigroup $(F_t)$ on $D.$
Moreover, in this case,
$$
F_t=\lim_{n\to\infty}J_{\frac tn}\circ\cdots\circ J_{\frac tn} ~(n\operatorname{-times})
=\lim_{n\to\infty}{J_{\frac tn}}^{\circ n}
$$
\end{theorem}



%\begin{example}\label{ex_H}Let $D=\uhp$ and consider the semigroup $F_t(z)=e^t z$. Then $G(z)=z$ and the equation $w = z - t z$ 
%has no solution in $\uhp$ for $t\geq 1$, but $J_t(w)=\frac{w}{1-t}$ for $t\in[0,1)$.\\
%We can consider the generator $G(z)=z$ also for $D=\C$, where  $J_t$ exists if and only if $t\not=1$.
%\end{example}


\begin{example}\label{ex:2t}
Let
$$
F_{t}(\zeta) = \frac{\zeta+\tanh(t/2)}{1+\zeta\tanh(t/2)} ,\quad \zeta\in\D,\; t\ge0.
$$
Then $(F_{t})_{t\geq0}$ is a semigroup of holomorphic mappings on $\D$ whose Denjoy-Wolff point is 1. The generator $H$ is given by $H(\zeta) = (1-\zeta^{2})/2$, and due to Theorem \ref{thm:RS} the nonlinear resolvent always exists.
On the other hand, consider the conjugate $\hat{F}_{t}$ by $C(z)=(z-i)/(z+i) :\uhp\to\D$, namely $\hat{F}_{t} := C^{-1} \circ F_{t} \circ C$. 
Then 
$$
\hat{F}_{t}(z) = e^{t}z, \quad\quad z\in\uhp,\; t\ge0,
$$
is a semigroup of holomorphic mappings on $\uhp$ with generator $G(z) = z$.
In this case the equation $w = z - t z$ has no solution in $\uhp$ for $t\geq 1$, and the resolvent $J_t(w)=\frac{w}{1-t}$ exists only for $t\in[0,1)$.
We can consider the generator $G(z)=z$ also for $D=\C$, where  $J_t$ exists if and only if $t\not=1$.
\end{example}

%{\color{blue}
%\begin{remark} Example \ref{ex:2t} can be extended in the following way: Let $D$ be a convex sector domain, i.e. it is bounded by two straight half-lines. Then there exists $c\in\C$ such that $D+c$ is a sector domain whose corner is at $0$. In $D+c$, the family $(e^tw)_{t\geq0}$ is a semigroup, which has the form $(e^tz+e^tc-c)_{t\geq0}$ in $D$ by conjugation with $z\mapsto w=z+c$. Its generator is given by $G(z)=z+c$ and the equation $w=z-t(z+c)$ cannot be solved for $t\geq1$. 
%\end{remark}
%%True for any convex domain which is "`positively starlike"', i.e. $z\in D$ implies $rz\in D$ whenever $r>1$.}

\begin{lemma}\label{rm_1}If $D$ is bounded and convex, then $G\circ J_t\in \G(D)$ for all 
$t\geq 0$. 
\end{lemma}
\begin{proof}
This is clearly true for $t=0$. So let $t>0$. Then  $(G\circ J_t)(w)= (J_t(w)-w)/t$.
Now we use the fact that $f-I$, $f:D\to D$ holomorphic, is always an infinitesimal generator on bounded convex domains, see \cite[Proposition 4.3]{RS96}, and the fact that $r\cdot G$ is an infinitesimal generator for every infinitesimal generator $G$ and $r>0$.
\end{proof}

The following lemma can be checked easily.

\begin{lemma}\label{lem:vf}
Let $(F_t)_{t\geq 0}$ be a semigroup on a domain $D\subset\C^n$ with generator $G:D\to \C^n$ and $\varphi:\hat D\to D$ be a biholomorphic mapping.
Then the maps $\hat{F}_t=\varphi^{-1}\circ F_t \circ \varphi$ form a semigroup on $\hat D$ with generator $\hat G(z)=(\varphi'(z))^{-1} G(\varphi(z))$.
Here, $\varphi'(z)$ is the Jacobian matrix of $\varphi$ at $z\in \hat D.$
\end{lemma}

\begin{proof}
Differentiation of both sides of $\varphi\circ\hat F_t=F_t\circ\varphi$ with respect to $t$
gives us the formula
$$
(\varphi'\circ\hat F_t)\frac{d\hat F_t}{dt}=\frac{dF_t}{dt}\circ\varphi.
$$
Letting $t=0,$ we obtain the required formula.
\end{proof}



We now use the Cayley transform $C:\uhp\to\D$, $C(z)=(z-i)/(z+i)$, to obtain a general form of generators on the upper half-plane $\uhp$. 
Let $G\in \G(\D)$. Then $G$ is expressed in the form \eqref{Berkson-Porta}.
By the last lemma, the corresponding generator $\hat G$ on $\uhp$ has the form 
\begin{align*}
\hat G(z) 
&= G(C(z))/C'(z)\\
&= G(C(z))\frac{-i (z+i)^2}{2} \\
%&= \frac{-i (z+i)^2}{2}   (\tau-C(z))(1-\overline{\tau}C(z)) \cdot p(C(z))\\
&=
 -(z+i)^2   (\tau-C(z))(1-\overline{\tau}C(z)) \cdot \frac{i}{2}p(C(z)).
 \end{align*}
If $\tau=1$, this expression reduces to  $G(z) =  2i p(C(z))$. 
If $\tau\not=1$, then the first three factors form a polynomial of degree $2$ with zeros $\sigma := C^{-1}(\tau)\in \uhp\cup \R$ and $\overline{\sigma}$. In this case we obtain
$G(z)= (1-\tau)(1-\overline{\tau})(z-\sigma)(z-\overline{\sigma}) \cdot \frac{i}{2}p(C(z))$. Clearly, if $G(z)\not\equiv 0$, then $\sigma$ is the Denjoy-Wolff point of the semigroup $\hat{F}_t$. Note that $ip(C(z))$ is a \textit{Pick function}, i.e. a holomorphic function from $\uhp$ into $\uhp\cup \R$. We denote by $\Nev$ the set of all Pick functions. We can summarize our calculation as follows (it is essentially contained in \cite{BP78}, where the right half-plane is considered).

\begin{lemma}\label{H_gen}
Every $G\in\G(\uhp)$ has the form 
\[  G(z)=q(z) \qquad \text{or} \qquad   G(z)=(z-\sigma)(z-\overline{\sigma})q(z), \]
with $q\in\Nev$, $\sigma \in \uhp\cup\R$. Conversely, all functions of these forms are infinitesimal generators on $\uhp$.
The first case corresponds to all semigroups whose Denjoy-Wolff point is $\infty$ unless $G=0$. In the second case, $\sigma$ is the Denjoy-Wolff point unless $G=0$.
\end{lemma}



\begin{example}
For the semigroup $F_t(z)=e^{-t}z$ on $\D$ with generator $H(z)=-z$ we obtain the resolvents $J_t(w)=w/(1+t)$ for all $t \ge 0$. 
The conjugated resolvent on $\uhp$ is given by $(C^{-1}\circ J_t \circ C)(z)=i[2z+t(z+i)]/[2i+t(z+i)]$.
On the other hand, by the above argument, the generator of the conjugated semigroup $\hat{F_{t}}$ on $\uhp$ is calculated as $G(z)=\frac{i}{2}(z^2+1) = (z-i)(z+i)\cdot\frac{i}{2}$.
Since one can observe that the equation $w = z - t\cdot G(z)$ does not lead to a M\"obius map, the nonlinear resolvents of a conjugated semigroup are in general not equal to the conjugated resolvents.
\end{example}




 

\section{Existence of nonlinear resolvents in unbounded domains}\label{sec_interior}

%\subsection{Fixed points of holomorphic self-mappings of the unit disc}
%
%
%Here we briefly summarize fundamental properties of fixed points of holomorphic self mappings which are used in this article.
%For a more detailed description, the reader is referred to \cite[Section 1.8]{BCDM20}.
%
%
%
%%In the rest of the present section, we will assume that $f$ is not the identity mapping to exclude the trivial case.
%Due to the well-known Schwarz-Pick Lemma, a non-identity holomorphic self-mapping $f$ of the unit disc $\D$ may have at most one fixed point in $\D$.
%If such a point $\tau$ exists, then this point is unique in $\D$, and 
%the sequence of iterates $\{f^{\circ n}\}_{n \in \N}$ converges to $\tau$ locally uniformly on $\D,$
%unless $f$ is an elliptic automorphism of $\D$ with a fixed point at $\tau.$
%Here and hereafter, $f^{\circ n}$ stands for the $n$-th iterate of $f$ (namely $f^{\circ 1} := f$ and $f^{\circ n} := f \circ f^{\circ(n-1)}$ for $n\ge 2$).
%Otherwise, the Denjoy-Wolff theorem asserts that there exists a unique boundary fixed point $\angle \lim_{z \to \tau}f(z) = \tau \in \partial \D$ such that $\{f^{\circ n}\}_{n \in \N}$ converges to $\tau$ locally uniformly on $\D,$ where $\angle \lim$ denotes the angular (or non-tangential) limit.
%In any case (including the case of an elliptic M\"obius transformation), 
%the point $\tau\in\overline\D$ is called the \textit{Denjoy-Wolff point} of  $\{f^{\circ n}\}$ or,
%more simply, of $f.$
%
%If the Denjoy-Wolff point $\tau$ lies in $\partial\D,$ then the angular derivative $f'(\tau) := \angle \lim_{z \to \tau} (f(z) -f(\tau))/(z-\tau)$ is known to exist and is a real number with
%$0 < f'(\tau) \le 1$.
%
%A non-identity holomorphic self-mapping $f$ of $\D$ is often classified into three cases according to the behavior near the Denjoy-Wolff point $\tau\in\overline\D$ of $f$
%(see \cite[Def.~1.8.5]{BCDM20}).
%That is,
%\begin{enumerate}
%\item[(i)] $f$ is called \textit{hyperbolic} if  $\tau\in\partial \D$ and $0 < f'(\tau) < 1$;
%\item[(ii)] $f$ is called \textit{parabolic} if $\tau\in\partial \D$ and $f'(\tau) = 1$;
%\item[(iii)] $f$ is called \textit{elliptic} if $\tau\in\D$. 
%\end{enumerate}
%This agrees with the classification of the automorphisms of $\D.$
%
%For a semigroup $(f_t)_{t \ge 0}$, all functions share the same Denjoy-Wolff point. 
%In fact, the point $\tau \in \overline{\D}$ appearing in the Berkson-Porta formula \eqref{Berkson-Porta} is the common Denjoy-Wolff point of $f_{t}$ for all $t>0.$
%Furthermore, if there exists $t_{0}>0$ such that $f_{t_{0}}$ is either hyperbolic/parabolic/elliptic, then so are all the other functions of the semigroup as well. 
%
%We denote by $\D(a,r)$ the disk $|z-a|<r$ in the complex plane $\D.$
%For $\tau\in\D$ and $0<\rho<1,$ we define
%$$
%\Delta(\tau, \rho)=\{z\in\D: \big|\tfrac{z-\tau}{1-\bar\tau z}\big|<\rho\}
%=\D\Big(\tfrac{(1-\rho^2)\tau}{1-\rho^2|\tau|^2},\tfrac{(1-|\tau|^2)\rho}{1-\rho^2|\tau|^2}\Big),
%$$
%which is the hyperbolic disc of center $\tau$ with radius $\frac12\log\frac{1+\rho}{1-\rho}.$
%For $\tau\in\partial\D$ and $0<R<+\infty,$ we define
%$$
%E(\tau,R)=\{z\in\D: \tfrac{|z-\tau|^2}{1-|z|^2}<R\}
%=\D\big(\tfrac{\tau}{1+R},\tfrac{R}{1+R}\big),
%$$
%which is often called a horocycle of center $\tau.$
%Note that
%$$
%\bigcup_{0<\rho<1}\Delta(a,\rho)=\D
%\aand
%\bigcup_{0<R<+\infty}\Delta(\tau,R)=\D
%$$
%for each $a\in\D$ and $\tau\in\partial\D.$
%
%
%Let $f:\D\to\D$ be a holomorphic map which is not the identity. Then and let $\tau\in\overline{\D}$ be its unique fixed point.
%
%\begin{lemma}\label{lem:SPDW}
%Let $f:\D\to\D$ be a non-identity holomorphic map and let $\tau\in\overline{\D}$ be its unique fixed point.
%\begin{enumerate}
%\item[(i)]
%When $|\tau|<1,$ $f(\Delta(\tau,\rho))\subset\Delta(\tau,\rho)$ for all $\rho\in(0,1).$
%\item[(ii)]
%When $|\tau|=1,$ $f(E(\tau,R))\subset E(\tau,R)$ for all $R>0.$
%\end{enumerate}
%\end{lemma}
%
%The assertion (i) follows from the Schwarz-Pick lemma and the assertion (ii)
%is contained in the Denjoy-Wolff theorem (see \cite[Theorem 1.8.4]{BCDM20}).


%Passing from $\D$ to the upper half-plane $\uhp$ by the transform $w= i(1+z)/(1-z)$, all of the above notions can be also stated for a holomorphic self-mapping of $\uhp$.
%In particular, if $g$ is a non-identity holomorphic self-mapping of $\uhp$ whose Denjoy-Wolff point is $\infty$, then there exists the \textit{angular derivative at $\infty$},
%$$
%g'(\infty) := \angle \lim_{w \to \infty} \frac{g(w)}{w} = \frac{1}{(C \circ g \circ C^{-1})'(1)} \ge 1,
%$$
%where $C(w)=(w-i)/(w+i).$
%Further, we obtain the upper half-plane version of the Julia-Wolff-Carath\'{e}odory Theorem 
%\begin{equation}
%\label{JWC-Lemma}
%\Im g(w) \ge g'(\infty) \cdot \Im w, \quad w \in \uhp.
%\end{equation}



\subsection{Proof of Theorem \ref{thm:unbdd}}

We denote by $\D(a,r)$ the disk $|z-a|<r$ in the complex plane $\C.$
For $\tau\in\D$ and $0<\rho<1,$ we define
$$
\Delta(\tau, \rho)=\{z\in\D: \big|\tfrac{z-\tau}{1-\bar\tau z}\big|<\rho\}
=\D\Big(\tfrac{(1-\rho^2)\tau}{1-\rho^2|\tau|^2},\tfrac{(1-|\tau|^2)\rho}{1-\rho^2|\tau|^2}\Big),
$$
which is the hyperbolic disc of center $\tau$ with radius $\frac12\log\frac{1+\rho}{1-\rho}.$
For $\tau\in\partial\D$ and $0<R<+\infty,$ we define
$$
E(\tau,R)=\{z\in\D: \tfrac{|z-\tau|^2}{1-|z|^2}<R\}
=\D\big(\tfrac{\tau}{1+R},\tfrac{R}{1+R}\big),
$$
which is often called a horocycle of center $\tau.$
Note that
$$
\bigcup_{0<\rho<1}\Delta(a,\rho)=\D
\aand
\bigcup_{0<R<+\infty}\Delta(\tau,R)=\D
$$
for each $a\in\D$ and $\tau\in\partial\D$.


Let $f:\D\to\D$ be a non-identity holomorphic map. We define the \textit{Denjoy-Wolff point} of $f$ similar to the definition for semigroups. If $f$ is not an elliptic automorphism, then the Denjoy-Wolff point is the unique point $\tau\in\overline{\D}$ such that $f^{\circ n}$, the $n$-th iterate of $f$, converges to $\tau$ locally uniformly on $\D$ as $n\to\infty$. If $f$ is an elliptic automorphism, we define the Denjoy-Wolff point of $f$ as the unique fixed point $\tau\in\D$ of $f$.


If $(f_t)_{t\geq0}$ is a semigroup on $\D$ with Denjoy-Wolff point $\tau$, then the Denjoy-Wolff point of each $f_t$ is equal to $\tau$ as well provided $f_t$ is not the identity.

\begin{lemma}\label{lem:SPDW}
Let $f:\D\to\D$ be a non-identity holomorphic map with Denjoy-Wolff point $\tau\in\overline{\D}$.
\begin{enumerate}
\item[(i)]
When $|\tau|<1,$ $f(\Delta(\tau,\rho))\subset\Delta(\tau,\rho)$ for all $\rho\in(0,1).$
\item[(ii)]
When $|\tau|=1,$ $f(E(\tau,R))\subset E(\tau,R)$ for all $R>0.$
\end{enumerate}
\end{lemma}

The assertion (i) follows from the Schwarz-Pick lemma and the assertion (ii)
is contained in the Denjoy-Wolff theorem (see \cite[Theorem 1.8.4]{BCDM20}).


For the proof of Theorem \ref{thm:unbdd}, we also need the following well-known result due to Study \cite{S11} and Robertson \cite{R36}.

\begin{lemma}\label{lem:convex}
Let $\varphi$ be a holomorphic function mapping $\D$ univalently onto a convex domain.
Then $\varphi(\Delta)$ is convex for every disk $\Delta$ contained in $\D.$
\end{lemma}



%\begin{proof}
%When $\Delta=\D(0,r)$ for some $0<r<1,$ this is an old result of Study
%(Give a reference. See Duren, for instance).
%If $\Delta=\D(c,r)$ with $|c|+r<1,$ then we take a suitable disk automorphism
%$T:\D\to\D$ and a number $0<\rho<1$ so that $T(\D(0,\rho))=\Delta.$
%Now we apply the first case to the map $\varphi\circ T$ to obtain the assertion.
%Finally, if $\Delta=\D(c,r)$ with $|c|+r=1,$ we consider the sequence
%$\Delta_n=\D(c,r-r/n)$ for $n=2,3,\dots.$
%Then $D_n=\varphi(\Delta_n)$ is an increasing sequence of convex domains 
%by the previous case.
%Since $D=\varphi(\Delta)=\cup_n D_n,$ the domain $D$ is also convex.
%\end{proof}

\begin{proof}[Proof of Theorem \ref{thm:unbdd}]
Let $D$ be an unbounded convex domain in $\C$ with $D\ne\C$ and
$G:D\to\C$ be an infinitesimal generator of a semigroup $(F_t)$ on $D.$
Furthermore, we assume that the Denjoy-Wolff point $\sigma$ of $(F_t)$ is finite. We need to show that $G$ admits nonlinear resolvents $J_t$ on $D$ for all $t\ge 0$.

First, let $\varphi:\D\to D$ be a conformal mapping.
As $D$ is convex, it is a Jordan domain in the Riemann sphere unless $D$ is a parallel strip, see \cite[Corollary 2]{K87}.
Therefore, if $D$ is a Jordan domain, $\varphi$ extends to a homeomorphism $\overline\D\to \overline D$ by
a theorem of Carath\'eodory.
Even when $D$ is a parallel strip; i.e., $\varphi=A\log\frac{1+\psi}{1-\psi}+B$ for some disk automorphism $\psi$ and constants $A, B$ with $A\ne0$, $\varphi$ extends to a continuous map $\overline\D\to \overline D$.

We now consider the conjugated semigroup 
$\hat F_t=\varphi^{-1}\circ F_t\circ\varphi$
and denote by $\hat G$ its generator.
Then the Denjoy-Wolff point $\tau$ of $(\hat F_t)$ is mapped to $\sigma$ by $\varphi.$


\medskip
\noindent
Case 1: $\sigma\in D$:
Then $\tau\in\D.$
Let $D_\rho=\varphi(\Delta(\tau,\rho))$ for $0<\rho<1.$
Note that $D_\rho$ is a bounded convex domain by Lemma \ref{lem:convex}.
Then Lemma \ref{lem:SPDW} now implies that $F_t(D_\rho)\subset D_\rho$ for each $\rho$. (If $F_t$ is the identity for some $t$, then trivially $F_t(D_\rho)\subset D_\rho$.)
It enables us to regard $(F_t)$ as a semigroup on the bounded convex domain $D_\rho.$
By Theorem \ref{thm:RS}, we obtain nonlinear resolvents $z=J_t(w)$ on 
$D_\rho$ for all $t\ge 0;$ that is to say, the equation $w=z-t\cdot G(z)$
has a unique solution $z\in D_\rho$ for each $w\in D_\rho.$
Since $D=\cup_{0<\rho<1}D_\rho,$ we readily see that the equation $w=z-t\cdot G(z)$
has a unique solution $z\in D$ for each $w\in D.$
Thus the assertion has been shown in this case.

\medskip
\noindent
Case 2: $\sigma\in\partial D$:
In this case we have $\tau\in\partial\D$ so that we use $E(\tau, R)$ instead.
Let $D_R=\varphi(E(\tau,R))$ for $R>0.$
Then Lemma \ref{lem:SPDW} again implies that $F_t(D_R)\subset D_R$ for each $R.$
Since $\varphi(\tau)=\sigma$ is finite by assumption, $D_R$ is bounded convex
by Lemma \ref{lem:convex}.
We can now show the assertion in the same way as Case 1.
\end{proof}


\subsection{Parallel strips}\label{sec_strip}

As another interesting example, we consider parallel strips here.
Let $D$ be the standard one defined by $|\Im z|<\pi/2.$
First we observe the semigroup $F_t(z)=z+t,\ z\in D, t\ge 0.$
Then the generator is $G(z)=1$ and the nonlinear resolvent exists for each $t$
and is given by $J_t(w)=w+t.$
It is the same as $F_t$ incidentally.
Note that the Denjoy-Wolff point is $\infty.$
%In contrast with the upper half-plane, nonlinear resolvents always exist
%on the parallel strip.
So far, we have the following sufficient condition for existence of nonlinear resolvents
on the parallel strips when the Denjoy-Wolff point is at infinity.
Recall that Theorem \ref{thm:unbdd} covers the case when the Denjoy-Wolff point
is finite.



\begin{theorem}
Let $D=\{z\in\C: |\Im z|<\pi/2\}$ and $G\in\G(D).$
Suppose that the Denjoy-Wolff point $\sigma$ of the semigroup 
generated by $G$ is $\infty.$
Then $G(z)=e^{-z}q(z)$ for a holomorphic function $q$ with $\Re q\ge0.$
Moreover, the nonlinear resolvents of $G$ exist for all 
$t\in[0,\frac{\pi}{2c}),$ where
$$
c=\inf_{x\in\R} |\Im G(x)|=\inf_{x\in\R} e^{-x}|\Im q(x)|.
$$
\end{theorem}

\begin{proof}
The function $\varphi(\zeta)=\log\frac{1+\zeta}{1-\zeta}$ maps $\D$
onto $D.$
%When Denjoy-Wolff point $\sigma$ of the semigroup generated by $G$ is finite,
%the assertion follows from Theorem \ref{thm:unbdd}.
%Thus it is enough to show in the case when $\sigma=\infty.$
We may assume that the point $\tau=1$ corresponds to $\sigma$
under the mapping $\varphi.$
Then, by Lemmas \ref{lem:BP} and \ref{lem:vf}, $\hat G=G\circ\varphi/\varphi'$
has the form $(1-\zeta)^2p(\zeta),$ where $p$ is a holomorphic function on $\D$
with $\Re p\ge 0.$
Hence, for $z=\varphi(\zeta),$ we obtain
$$
G(z)=\hat G(\zeta)\varphi'(\zeta)
=2\frac{1-\zeta}{1+\zeta}\,p(\zeta)
=e^{-z}q(z).
$$
Here, we put $q(z)=2p(\varphi^{-1}(z)).$
For a while, we assume that $q$ satisfies the additional conditions
\begin{enumerate}
\item[(i)]
$q(z)$ is holomorphic on an open set
containing the closed parallel strip $|\Im z|\le \pi/2,$ and
\item[(ii)]
$q$ satisfies the inequality $|q-1|\le k|q+1|$ on $D$ for a constant
$0<k<1.$
\end{enumerate}

Condition (ii) says that the image $q(D)$ is contained in a compact
subset of the right half-plane $\Re w>0$.

Under these conditions, we will show that the nonlinear resolvents
of $G$ exist for all $t\ge 0.$
For a given point $w_0\in D$ and a positive number $t,$ we consider the function
$w=f(z)=z-tG(z)-w_0.$
We need to show that $f$ has a unique zero in $D.$
Note that $|\Im w_0|<\pi/2.$
First we observe that for $x\in\R,$
$
f(x+i\pi/2)=x+i\pi/2+it e^{-x}q(x+i\pi/2)-w
$
so that $\Im f(z)=\pi/2-\Im w_0+t e^{-x}\Re q(z)>0$ on $\Im z=\pi/2.$
Similarly, we see that $\Im f(z)<0$ on $\Im z=-\pi/2.$
When $x=\Re z\to+\infty$ in $D,$ we have $\Re f(z)=x+O(1).$
In particular, $\Re f(z)>0$ if $\Re z$ is large enough.
When $x=\Re z\to-\infty$ in $D,$ we observe
$$
\arg(-f(z))=\arg e^{-z}+\arg(q(z)-e^z(z-w_0)/t)
=-\Im z+\arg(q(z)+o(1)).
$$
Note that $|\arg q|\le 2\arctan k<\pi/2$ on $D$ by assumption.
We fix a number $\alpha$ so that $2\arctan k<\alpha<\pi/2.$
If $-x=-\Re z$ is positive and large enough, then
$|\arg (-f(z))|<\pi/2+\alpha<\pi.$
Now we take the boundary of the rectangle $|\Re z|\le T$
and $|\Im z|\le\pi/2$ as a contour $\Gamma$ for a large enough $T>0.$
Then we see that
$$
\frac1{2\pi i}\int_\Gamma \frac{f'(z)}{f(z)}dz
=\frac1{2\pi}\int_{f(\Gamma)} d\arg w=1
$$
as required.

Next we consider the general $G(z)=e^{-z}q(z)$ and let $0<t<\pi/(2c).$
We first assume that $\Re q>0.$
Take a number $c'$ so that $c<c'\le\pi/(2t).$
Then by the definition of $c,$ we can find a real number $x_0$
so that $t|\Im G(x_0)|<tc'\le \pi/2.$
Thus $w_0=x_0-tG(x_0)\in D.$
Let $p(\zeta)=q(\varphi(\zeta)+x_0).$
Then $p$ is holomorphic and satisfies $\Re p>0$ on $\D.$
For $0<\rho<1,$ we set $G_\rho(z)=e^{-z}q_\rho(z),$
where $q_\rho(z)=p(\rho\varphi^{-1}(z-x_0)).$
Then $G_\rho$ satisfies the conditions (i) and (ii) and $G_\rho(x_0)=G(x_0)$
so that we can find the nonlinear resolvent $g_\rho(w)=J_t(w,G_\rho)$
for each $\rho.$
Note that $g_\rho:D\to D$ is holomorphic and satisfies $g_\rho(w_0)=x_0.$
Since $(g_\rho)$ is a normal family, there exists a local uniform
limit $g=\lim g_{\rho_j}:D\to \overline D$ 
for a suitable sequence $\rho_j\to 1.$
Since $g(w_0)=x_0\in D,$ we see that $g$ maps $D$ into itself.
By definition of the nonlinear resolvents, the functions $g_\rho$
satisfy the relation
$$
w=g_\rho(w)-tG_\rho(g_\rho(w)),\quad w\in D.
$$
We now take the limit through the sequence $\rho_j$ to get
the relation $w=g(w)-tG(g(w)).$


Next we consider the case $\Re q=0$ at some point in $D.$
Then $q=ia$ for a real constant $a.$
In this case, we consider $q_\rho=\rho+ia$ for a positive number $\rho>0.$
Then the same argument works as $\rho\to 0.$
Hence, we have obtained the nonlinear resolvent $g(w)=J_t(w,G).$
\end{proof}




%\section{Upper half-plane I: Denjoy-Wolff point in the interior: Older version; to be removed}
%%\label{sec_interior}
%
%
%%The Berkson-Porta formula may be transformed to the following form.
%%
%%\begin{lemma}
%%Let $\sigma$ be the Denjoy-Wolff point for a semigroup $\{F_t\}$ on $\uhp$
%%with generator $f:\uhp\to\C.$
%%Then there is a Pick function $q$ such that
%%$$
%%f(z)=
%%\begin{cases}
%%(z-\sigma)(z-\bar\sigma)q(z),& \text{if}~\sigma\ne\infty, \\
%%q(z), &\text{if}~ \sigma=\infty.
%%\end{cases}
%%$$
%%\end{lemma}
%
%
%Let us recall the  Nevanlinna representation formula for Pick functions, which corresponds to the Herglotz representation for Carath\'eodory functions.
%
%
%\begin{lemma}[e.g. {\cite[Theorem 1]{cau32}}]
%\label{Nev-expression}
%If $q\in \Nev$, then there
%exist real constants $\alpha\ge0$ and $\beta$
%and a finite non-negative Borel measure $\rho$ on $\R$ such that
%\begin{equation}
%\label{Nev-equation}
%q(z)=\alpha z+\beta+\int_\R \frac{1+tz}{t-z}d\rho(t),\quad z\in\uhp.
%\end{equation}
%The number $\alpha$ and $\beta$ are calculated via $\alpha = \lim_{y\to\infty}f(iy)/(iy)$ and $\beta = \Re q(i)$, respectively.
%Conversely, the function expressed as above is a Pick function.
%\end{lemma}
%
%
%
%
%Consider the set $\Nev_0$ of rational Pick functions of the following form:
%\begin{equation}\label{eq:q}
%q(z)=\alpha z+\beta+\sum_{j=1}^n \rho_j\, \frac{1+t_jz}{t_j-z}
%\end{equation}
%for $\alpha>0, \beta\in\R, t_j\in\R, \rho_j>0.$
%Note that $q(x)\in\R\cup\{\infty\}$ for $x\in\R.$
%
%\begin{lemma}\label{lem:dense}
%The set $\Nev_0$ is dense in $\Nev$ 
%in the topology of local uniform convergence on $\uhp.$
%\end{lemma}
%
%\begin{proof}
%This follows easily from the fact that the set of measures of the form
%$\rho=c_1\delta_{t_1}+\cdots+ c_n \delta_{t_n}$
%with $c_1>0,\dots, c_n>0, c_1+\cdots +c_n=1, t_j\in\R,$ is dence in
%the set of Borel probability measures on $\R$ in the weak topology.
%Here, $\delta_t$ stands for the Dirac measure with unit mass at $t\in\R.$
%\end{proof}
%
%\begin{lemma}
%Assume that $G(z)=(z-\sigma)(z-\bar\sigma)q(z)$
%for a $\sigma\in\uhp\cup\R$ and $q\in\Nev_0.$
%Then the nonlinear resolvent $J_r$ of $G$ exists for every $r\ge 0.$
%\end{lemma}
%
%\begin{proof}
%Suppose that $q$ has the form in \eqref{eq:q} with $\alpha>0, \beta\in\R,$
%$\rho_j>0$ and $t_1<t_2<\dots <t_n$.
%For a fixed $r>0$ and an arbitrary point $w_0\in\uhp,$ the value 
%of the nonlinear resolvent $z=J_r(w_0)$ is the unique solution in $\uhp$ to the
%equation
%$$
%z-rG(z)=w_0,
%$$
%that is a unique zero in $\uhp$ of the rational function
%$$
%F(z)=z-w_0-rG(z)
%=z-w_0-r(z-\sigma)(z-\bar\sigma)q(z).
%$$
%Thus we need to count the number of zeros of $F$ in $\uhp.$
%We note that $F$ has poles at $z=t_j$ and $z=\infty$ and no other poles
%in $\uhp\cup\R.$
%If the equality $t_j=\sigma$ happens for some $j,$ the function $F$ has
%no pole at $t_j$ and therefore such $j$ will be excluded in the consideration below.
%For a large enough $R>0$ and a small enough $\varepsilon>0,$
%we can assume that $F$ has no zeros on $R\le |z|<+\infty$ and
%$0<|z-t_j|\le\varepsilon$ for $j=1,2,\dots, n.$
%Now consider the domain $D=\{z\in\uhp: |z|<R, |z-t_j|>\varepsilon ~(j=1,\dots, n)\}.$
%Note here that $q(\R)\subset\R\cup\{\infty\}.$
%For $x\in\partial D\cap\R,$ we thus have
%\begin{equation}\label{eq:ImF}
%\Im F(x)=\Im\Big\{x-w_0-r|x-\sigma|^2q(x)\Big\}=-\Im w_0<0.
%\end{equation}
%Then, by the argument principle,
%the number $N$ of zeros of $F$ in $D$ is described by
%$$
%N=\frac1{2\pi i}\int_{\partial D}\frac{F'(z)}{F(z)}dz
%=\frac1{2\pi}\int_{\partial D}d\arg F(z)
%=\frac1{2\pi}\int_{F(\partial D)}d\arg w.
%$$
%We divide $\partial D$ into the pieces
%$\gamma_\infty, \delta_0=[-R, t_1-\varepsilon], \gamma_1, 
%\delta_1=[t_1+\varepsilon, t_2-\varepsilon], \gamma_2,
%\dots, \delta_{n-1}=[t_{n-1}+\varepsilon, t_n-\varepsilon],
%\gamma_n, \delta_n=[t_n+\varepsilon, R], $
%where
%$\gamma_\infty$ is the curve defined by $\gamma_\infty(\theta)=Re^{i\theta}~ (0\le\theta\le\pi)$
%and $\gamma_j$ is the curve $\gamma_j(\theta)=t_j+\varepsilon e^{i(\pi-\theta)}~
%(0\le \theta \le \pi)$ for $j=1,2,\dots, n.$
%
%As $z\to\infty,$ by the form of $F(z),$ we see that
%$$
%F(z)=-\alpha r z^3+O(z^2)=-\alpha r z^3(1+o(1)).
%$$
%We choose a continuous branch $u(\theta)$ of $\arg F(\gamma_\infty(\theta))$
%so that $-\pi<u(0)=\arg F(R)<0.$
%Then $u(0)=\arg F(R)=-\pi+o(1), ~ u(\pi)=\arg F(-R)
%=2\pi+o(1)$ as $R\to +\infty$ and $u(\pi)<2\pi.$
%In particular, the image curve $\Gamma_1=F(\gamma_\infty)$ satisfies
%$$
%\int_{\Gamma_1}d\arg w=h(\pi)-h(0)=3\pi+o(1)
%$$
%as $R\to\infty.$
%
%By \eqref{eq:ImF}, we see that the image curve $F(\delta_j)$
%lies entirely in the lower half-plane $\Im w<0.$
%As $z\to t_j,$ the function $F(z)$ has the asymptotic behaviour
%$$
%F(z)=-\frac{A_j}{t_j-z}+B_j+O(z-t_j),.
%$$
%where $A_j=r\rho_j|t_j-\sigma|^2(1+t_j^2)>0$ and
%$B_j$ is some constant.
%In particular, 
%$$
%\Im\big[F(\gamma_j(\theta))-F(\gamma_j(0))\big]
%=\Im\left[\frac{A_j}{\varepsilon e^{i(\pi-\theta)}}-\frac{A_j}{\varepsilon e^{\pi i}}\right]
%+O(\varepsilon)
%=-\frac{A_j}{\varepsilon}\sin\theta+O(\varepsilon)
%$$
%as $\varepsilon\to0$ uniformly in $0\le \theta\le\pi.$
%Noting that $\Im F(\gamma_j(0))=\Im F(t_j-\varepsilon)<-\Im w_0$
%by \eqref{eq:ImF}, we conclude that $F(\gamma_j)$ lies entirely in the lower
%half-plane, as well, for $j=1,2,\dots, n.$
%Hence, for a sufficiently small $\varepsilon>0,$
%the image curve $\Gamma_2=F(\gamma)$ of
%$\gamma=\delta_0+\gamma_1+\delta_1+\cdots
%+\gamma_n+\delta_n$ under the mapping $F$
%lies in the lower half-plane and therefore
%$\Gamma_2$ is homotopic to the line segment $L=[F(-R), F(R)]$
%in the punctured plane $\C\setminus\{0\}.$
%Since $L$ lies in the lower half-plane,
%$$
%\int_{\Gamma_2}d\arg w=\int_{L}d\arg w=-\pi+o(1)
%$$
%as $R\to+\infty$ and $\varepsilon\to 0.$
 %
%We now summarize the above observations as
%$$
%2\pi N=\int_{F(\partial D)}d\arg w
%=\int_{\Gamma_1}d\arg w+\int_{\Gamma_2}d\arg w
%=3\pi+o(1)+(-\pi+o(1))=2\pi+o(1).
%$$
%Since $N$ is an integer, we finally obtain $N=1$ 
%as $R\to+\infty$ and $\varepsilon\to 0.$
%\end{proof}
%
%
%
%We can now prove Theorem \ref{intro_2} (1).
%%Our main result in this section is presented as follows.
%%We will denote by $\G(\uhp)$ the set of generators of semigroups
%%of holomorphic self-maps of $\uhp.$
%
%\begin{theorem}
%\label{finite-point-theorem}
%Let $G$ be an infinitesimal generator on $\uhp$ which has a zero in $\uhp$.
%Then the nonlinear resolvent $J_r=(I-rG)^{-1}:\uhp\to\uhp$ exists
%for every $r\ge0.$
%\end{theorem}
%\begin{proof}
%By Lemma \ref{H_gen} we can write $G$ as $G(z)=(z-\sigma)(z-\overline{\sigma})q(z)$ for some $\sigma\in\uhp$ and $q\in\Nev$. 
%Choose a sequence $(q_n)_{n\in\N}\subset\Nev_0$ with $q_n\to q$ locally uniformly.  Then $G_n(z):=(z-\sigma)(z-\overline{\sigma})q_n(z)$ is a sequence of generators converging to $G$. Fix some $r>0$. By the previous lemma we know that the resolvent $J_{r,n}$ of $G_n$ for the parameter $r$ exists. Furthermore, as $(J_{r,n})_{n\in\N}$ is a normal family, we may assume that $J_{r,n}$ converges locally uniformly to some $J_r:\uhp\to\uhp\cup\R\cup\{\infty\}$.\\
%As $\sigma\in\uhp$, we have $J_{r,n}(\sigma)=\sigma$ for all $n$ and thus $J_r$ maps $\uhp$ into $\uhp$. \\
%By definition,
%$$
%(I-rG_n)\circ J_{r,n}=I
%$$
%on $\uhp.$
%Letting $n\to\infty,$ we obtain $(I-rG)\circ J_r=I$ and we see that $J_r$ is the nonlinear resolvent of $G$
%for the parameter $r$, which proves the assertion.
%\end{proof}


%For the proof, the following convergence lemma.
%
%
%\begin{lemma}
%Let $f_n$ be a sequence in $\G(\uhp)$ converging to $f\in\G(\uhp)$
%locally uniformly on $\uhp.$
%Suppose that $f$ has a zero $\sigma$ in $\uhp.$
%Let $r>0$ be fixed.
%If the nonlinear resolvent $J_n=(I-rf_n)^{-1}:\uhp\to\uhp$ of $f_n$ exists
%for every $n,$ then the nonlinear resolvent $J=(I-rf)^{-1}$ of $f$ exists
%and $J_n\to J$ locally uniformly on $\uhp$ as $n\to\infty.$
%\end{lemma}
%
%\begin{proof}
%
%
%First,  by the Hurwitz theorem,
%we observe that $f_n$ has a zero $\sigma_n$ in $\uhp$
%for large enough $n$
%such that $\sigma_n\to\sigma$ as $n\to\infty.$
%We also note that $J_n(\sigma_n)=\sigma_n.$
%
%Since $\{J_n\}$ is a normal family, we may assume that $J_n$
%converges to some function $J_\infty$ locally uniformly on $\uhp$
%as $n\to\infty.$
%Then either $J_\infty$ maps $\uhp$ into itself or $J_\infty$ is a
%constant in $\R\cup\{\infty\}.$
%Since $J_n(\sigma_n)=\sigma_n,$ the function $J_\infty$ has a fixed point
%at $\sigma.$ Hence, the latter possibility is excluded.
%By definition,
%$$
%(I-rf_n)\circ J_n=I
%$$
%on $\uhp.$
%Letting $n\to\infty,$ we obtain $(I-rf)\circ J_\infty=I.$
%It thus turned out tat $J_\infty$ is the nonlinear resolvent of $f$
%for the parameter $r,$ which proves the assertion.
%\end{proof}
%
%{\bf Rermak for us.
%In the above proof, a weaker assumption may be sufficient to exclude the
%second possibility. Please consider the case when the corresponding semigroup
%has the DW point on $\R.$
%The lemma below supports the possiblity because it allows the case when
%$\sigma\in\R.$
%}
%
%By the above lemma, now it is enough to show the following lemma
%for the proof of our theorem.












 
\section{Decreasing Loewner chains}\label{sec_deLo}

\begin{definition}
 A family $(f_t)_{t\geq0}$ of univalent mappings $f_t:D\to D$ is called a \textit{decreasing Loewner chain} if $f_0$ is the identity, 
 $f_t(D)\subset f_s(D)$ whenever $s\leq t$, and $t\mapsto f_t$ is continuous in the topology of locally uniform convergence on $D$. 
\end{definition}


Decreasing Loewner chains can be obtained by solving a certain Loewner PDE. First, we need the following definition, which was introduced in 
  \cite{MR2507634} for general complete (Kobayashi) hyperbolic manifolds. We will only consider complete  hyperbolic subdomains of $\C^n$. For example, every bounded and convex domain is complete hyperbolic, see \cite[Theorem 1.1]{BS09}.
	
\begin{definition}
Let $D\subset \C^n$ be a complete hyperbolic domain. 
 A \textit{Herglotz vector field} of order $d\in[1,+\infty]$ on $D$ is a mapping $G:[0,\infty)\times D \to \C^n$ with the following properties:
\begin{itemize}
 \item[(1)] The function $t\mapsto G(t, z)$ is measurable on $[0,\infty)$ for all $z\in D.$
\item[(2)] The function $z\mapsto G(t, z)$ is an infinitesimal generator for almost every $t\in[0,\infty).$
\item[(3)] For any compact set $K\subset D$ and $T>0,$ there exists a function $C_{T,K}\in L^d([0,T],\R^+_0)$ such that $\|G(t,z)\|\leq C_{T,K}(t)$ for all $z\in K$ and for almost all $t\in[0,T].$ 
\end{itemize}
\end{definition}
\begin{remark}
In \cite{MR2507634} the definition of a Herglotz vector field involves a condition on the function $G(\cdot,t)$ (for almost all $t$) using the Kobayashi metric of $D$ instead of property (2). In \cite[Theorem 1.1]{MR2887104}, however, it was shown that this condition is equivalent to condition (2).
\end{remark}

\begin{theorem}\label{thm_1}
Let $G(t,z)$ be a Herglotz vector field on a complete hyperbolic domain. Then there exists a decreasing Loewner chain $(f_t)_{t\geq 0}$, locally absolutely continuous in $t$, satisfying the partial differential equation
\begin{equation}\label{inv_equ}
 \frac{\partial}{\partial t}f_t(z)=f'_t(z)\cdot G(t,z) \quad \text{for a.e.\ $t\geq 0$}, \quad f_0(z)=z\in D,
\end{equation}
for all $z\in D$. Conversely, if $(J_t)_{t\geq 0}$ is a family of holomorphic functions $J_t:D\to D$, locally absolutely continuous in $t$, which satisfies \eqref{inv_equ}, then $J_t = f_t$ for all $t\geq0$.
\end{theorem}
\begin{proof}We follow \cite[Theorem 3.2]{CDMG14}, where decreasing Loewner chains are handled by studying evolution families (arising from increasing Loewner chains).




Fix $T>0$ and consider the Loewner equation
\begin{equation*}
 \frac{\partial}{\partial t}g_{s,t}(z)= G(T-t, g_{s,t}(z)), \quad g_{s,s}(z)= z \in D, 0\leq s\leq t \leq T,
\end{equation*}
which has a unique solution of univalent mappings $g_{s,t}:D\to D$ with $g_{s,s}(z)\equiv z$, 
$g_{s,t}=g_{u,t}\circ g_{s,u}$ for all $0\leq s\leq u \leq t \leq T$ and for any compact subset $K\subset D$ and for any $0<S\leq T$ there exists a non-negative function $k_{K,S}\in L^d([0,S],\R^+_0)$ such that for all $0\leq s\leq u\leq t\leq S$ and for all $z\in K,$
$$ \|g_{s,u}(z)-g_{s,t}(z)\| \leq \int_u^t k_{K,S}(\tau)\, d\tau, $$
see \cite[Def. 1.2, Prop. 3.1, Prop. 5.1]{MR2507634}. ($g_{s,t}$ is (a part of) an evolution family of order $d$.) 
In particular $g_{s,T}\circ g_{0,s} = g_{0,T}$. Differentiation with respect to $s$ gives 
\[\frac{\partial}{\partial s} g_{s,T}(g_{0,s}(z)) + g_{s,T}'(g_{0,s}(z)) \cdot G(T-s, g_{0,s}(z)) = 0,\]
i.e. \[ \frac{\partial}{\partial s} g_{s,T}(w) = -  g_{s,T}'(w) \cdot G(T-s, w)\]
for all $w\in g_{0,s}(D)$. However, the right side (and thus its integral w.r.t $s$) can be extended holomorphically to $D$ and thus 
$\frac{\partial}{\partial s} g_{s,T}(w)$ satisfies the PDE for all $w\in D$. 

Now define $f_t = g_{T-t,T}$, $0\leq t\leq T$. 
We have $f_t = g_{T-t,T} = g_{T-s, T} \circ g_{T-t,T-s}=f_s \circ g_{T-t,T-s}$ whenever $s\leq t$. Hence $(f_t)_{0\leq t\leq T}$ is (a part of) a decreasing Loewner chain. We have 
\begin{equation*}
 \frac{\partial}{\partial t}f_t(z)=  g_{T-t,T}'(z) \cdot G(t, z) =   f'_t(z) \cdot G(t, z), \quad f_0(z)= z \in D,\quad  0\leq t \leq T. 
\end{equation*}
 By choosing another $\hat{T}>T$, we obtain a family $(\hat{g}_{s,t})_{0\leq s\leq t\leq\hat{T}}$ with $\hat{g}_{s,\hat{T}} 
=g_{s+T-\hat{T},\hat{T}+T-\hat{T}}=g_{s+T-\hat{T},T}$ for all $s\in [\hat{T}-T, \hat{T}]$. Hence 
$\hat{f}_t := \hat{g}_{\hat{T}-t,\hat{T}} = g_{T-t,T}=f_t$ for all $t \in [0, T]$. 



As we can choose $T>0$ arbitrarily large, we conclude that there exits a decreasing Loewner chain $(f_t)_{t\geq 0}$ satisfying \eqref{inv_equ}.

Finally, let $(J_t)_{t\geq0}$ be a family of holomorphic mappings, locally absolutely continuous in $t$, satisfying \eqref{inv_equ}. Let $T>0$ and define 
$g_{s,t}$ as above. We have \[\frac{\partial}{\partial t}[(J_t(g_{0,T-t}))(z)] = J_t'(g_{0,T-t}(z))\cdot G(t,g_{0,T-t}(z)) - 
J_t'(g_{0,T-t}(z))\cdot G(T, g_{0,T-t}(z)) = 0\] and thus $J_t(g_{0,T-t}(z))=J_0(g_{0,T}(z))=g_{0,T}(z)$ for all $z\in D$ and $0\leq t\leq T$. This implies $J_t = g_{0,T} \circ g^{-1}_{0,T-t}$ on $g_{0,T-t}(D)$. As $g_{0,T-t}(D)$ is an open set, the identity theorem implies 
$J_t = f_t$ on $D$.
\end{proof}

Next we prove Theorem \ref{intro_1}. Part (1) generalizes \cite[Proposition 4.5]{ESS20}, which considers the case of the unit disc and generators with Denjoy-Wolff point $0$. 

\begin{theorem}\label{result1}${}$
\begin{itemize}
\item[(1)] Let $D\subset \C^n$ be a bounded and convex domain and let $G\in\G(D)$ with nonlinear resolvents $J_t:D\to D$. Then $(J_t)_{t\geq0}$ 
is a decreasing Loewner chain satisfying the Loewner partial differential equation
\begin{equation}\label{m}\frac{\partial}{\partial t}J_t(w)=J_t'(w)\cdot G(J_t(w))\quad \text{for all $t\geq 0$}, \quad J_0(w)= w\in D.\end{equation}
The domains $J_t(D)$ contract to the zero set of $G$, i.e.\ $\bigcap_{t\geq 0} J_t(D) = G^{-1}(0)$.
\item[(2)] Let $D\subset \C^n$ be a (possibly unbounded) convex domain and let $G\in\G(D)$. 
Furthermore, let $J_t:D\to D$ be a family of holomorphic functions satisfying \eqref{m}. 
Then $J_t$ are the nonlinear resolvents of $G$ on $D$.
\item[(3)] Let $D\subset \C^n$ be a bounded and convex domain and let $G:[0,\infty)\times D\to \C^n$ be such that $z\mapsto G(t,z)$ is holomorphic for a.e.\ $t\geq0$, $t\mapsto G(t,z)$ is locally integrable for all $z\in D$, and $H_t(z):=\int_0^t G(s,z) ds$ 
is an infinitesimal generator on $D$ for every $t\geq 0$. Let $J_t:D\to D$ be the nonlinear resolvent of $H_t$ at time $1$. Then $(J_t)_{t\geq 0}$ is the unique solution of holomorphic self-mappings of $D$, locally absolutely continuous in $t$, of
\begin{equation}\label{m2}\frac{\partial}{\partial t}J_t(z)=J_t'(z)\cdot G(t,J_t(z)) \quad \text{for a.e.\ $t\geq 0$}, \quad J_0(z)= z\in D.\end{equation}
\end{itemize}
\end{theorem}
\begin{proof}${}$
\begin{itemize}
\item[(1)] 
Let $J_t$ be the resolvents of $G$. From $w = J_t(w) - tG(J_t(w))$ we get by differentiation 
\[J'_t(w) - t G'(J_t(w)) \cdot J'_t(w)= I,\] i.e.\ $J'_t(w) = (I-t G'(J_t(w)))^{-1}$.
Differentiation with respect to $t$ yields \[0 = (I-t G'(J_t(w)))\frac{\partial}{\partial t}J_t(w) - G(J_t(w)),\] which gives us
\[\frac{\partial}{\partial t}J_t(w)=J'_t(w)\cdot G(J_t(w)).\]

Put $\varphi_t(z)=z-tG(z)$. As ($\varphi_t \circ J_t)(z)=z$, each $J_t$ is clearly a univalent function. Moreover, the continuity of 
$t\mapsto \varphi_t$ implies that also $t\mapsto J_t$ is continuous. 


Now let $z\in D$. The statement $z\in J_t(D)$ is equivalent to $\varphi_t(z)\in D$. Hence, as $D$ is convex, we see that 
$z\in J_t(D)$ implies $z\in J_s(D)$ for all $s\in[0,t]$. We conclude that $J_t(D)\subset J_s(D)$ whenever $s\leq t$ and that $(J_t)_{t\geq0}$ is a decreasing Loewner chain. Due to Lemma \ref{rm_1}, we see that $(t,z)\mapsto G(J_t(z))$ is a Herglotz vector field and thus equation \eqref{m} is a Loewner partial differential equation of the form \eqref{inv_equ}. 


(We can also argue as follows: $J_t$ solves \eqref{m} and $G\circ J_t$ is a Herglotz vector field. Hence, Theorem \ref{thm_1} implies that $(J_t)_{t\geq 0}$ is a decreasing Loewner chain.)

Let $z\in D$. If $G(z)=0$, then $J_t(z)=z$ for all $t\geq 0$. If $G(z)\not = 0$, then there exists $T>0$ such that $z-t\cdot G(z)\not\in D$ for all 
$t\geq T$ as $D$ is bounded. Hence, $z\not\in J_t(D)$ for all $t\geq T$ and we conclude $\bigcap_{t\geq 0} J_t(D) = G^{-1}(0)$.

\item[(2)]
Now let $J_t:D\to D$ be a family of holomorphic functions satisfying \eqref{m}, where $D$ is a convex domain. (Note that now, we do not know yet whether $(t,z)\mapsto G(J_t(z))$ is a Herglotz vector field.)
Consider the differential equation 
\[\frac{\partial}{\partial t}\varphi_t(z) = -G(J_t(\varphi_t(z))), \quad \varphi_0(z)=z.\]

Fix $z\in D$. We can solve this equation at least for $t$ small enough.
A small computation shows that $\frac{d}{dt}[J_t(\varphi_t(z))]=0$, i.e.\ $J_t(\varphi_t(z))$ does not depend on $t$ and $J_t(\varphi_t(z))=J_0(\varphi_0(z))=z$. 
Hence $\frac{\partial}{\partial t}\varphi_t(z) = -G(z)$. We conclude that
$t\mapsto \varphi_t(z)$ simply describes a straight line: \[\varphi_t(z)=z-tG(z).\]
In particular, we can now define $\varphi_t(z)$ for all $z\in D$ and all $t\geq0$ by $\varphi_t(z)=z-tG(z)$.



Let $D_t=\{z\in D\,|\, \varphi_t(z)\in D\}$. The convexity of $D$ implies that $D_t\subset D_s$  whenever $s\leq t$. Thus, for all $z\in D_t$, we have 
$J_t(\varphi_t(z))=z$. Applying $\varphi_t$ gives $\varphi_t(J_t(w))=w$ for all $w\in \varphi_t(D_t)$. As the left side extends holomorphically to $D$, we see that $\varphi_t(J_t(w))=w$ for all $w\in D$. Applying $J_t$ gives $J_t(\varphi_t(z))=z$ for all $z\in J_t(D)$. 
%Let $M_t = \varphi_t(D_t)\subset D$. Then $\varphi_t(J_t(z))=z$ for all $z\in M_t$. Note that the function on the left side is in fact defined (and holomorphic) on the whole domain $D$. Thus the identity theorem
%implies $\varphi_t(J_t(z))=z$ for all $z\in D$. Hence $\varphi_t(J_t(D))=D$, which shows 
%$D_t = J_t(D)$ and $M_t=D$. 
We conclude that $J_t$ is the nonlinear resolvent of $G$.
%As $D$ is bounded and convex, the equation $w=z-tG(z)$ can be solved for every $w\in D$.
%Denote the solution by $K_t(z)$. Then $K_t(\varphi_t(z))=z$ on $D_t$ and $D_t$ is non-empty as $K_t(D)\subset D_t$. We conclude that $J_t(\varphi_t(z))=z$ for all $z\in D_t$. The identity theorem implies $J_t = K_t$ on $D$. \\ 
\item[(3)]
 By definition,  $J_t$ is the inverse of $\varphi_t: z\mapsto z- \int_0^t G(s,z) ds$. Clearly, $t \mapsto \varphi_t(z)$ is locally absolutely continuous for any $z\in D$.
Furthermore, as $D$ is bounded, the set $\{J_\tau\,|\, \tau\geq 0\}$ is a normal family and thus $\{J'_\tau\,|\, \tau\geq 0\}$ is locally uniformly bounded.

Let $s,t\geq 0$ and $z\in D$. Put $w=J_t(z)$. If $s$ is close enough to $t$, then $z'(s)=\varphi_s(w)\in D$. 
We have 
\[ J_t(z) - J_s(z)  = J_t(z) - J_s(z'(s)) +  J_s(z'(s)) - J_s(z)= J_s(z'(s)) - J_s(z).\]
From $z'(s)-z = \varphi_s(w) -\varphi_t(w)$, we see that $\tau \mapsto J_\tau(z)$ is locally absolutely continuous for any $z\in D$.
%https://en.wikipedia.org/wiki/Absolute_continuity
$J_t$ solves the equation $w = J_t(w) - \int_0^t G(s,J_t(w)) ds$. 
%Consider $z'=\varphi^{-1}_t(z), w'=\varphi_{t+\eps}^{-1}(z)$. For $\eps<0$ small enough, we can write $w'=\varphi^{-1}_t(z_1), z' = \varphi^{-1}_{t+\eps}(z_2)$. $\varphi^{-1}_t(z)-\varphi_{t+\eps}^{-1}(z)$   
% Let $|z'-w'|<\delta. Then $\varphi_t(z)$
 We get by differentiation 
\[J'_t(w) - \int_0^t G'(s,J_t(w)) \cdot J'_t(w) ds = I,\] i.e.\ $J'_t(w) = (I-\int_0^t G'(s,J_t(w))ds)^{-1}$.
Differentiation with respect to $t$ yields \[0 = (I-\int_0^t G'(s,J_t(w))ds)\frac{\partial}{\partial t}J_t(w) - G(t,J_t(w)),\] which gives us
\[\frac{\partial}{\partial t}J_t(w)=J'_t(w)\cdot G(t, J_t(w)).\]

Next we show uniqueness of the solution. Let $(f_t)_{t\geq 0}$ be another solution of holomorphic mappings $f_t:D\to D$, locally absolutely continuous in $t$, satisfying \eqref{m2}.


Choose some $z_0\in D$. Then we find some open disc $B\subset D$ with center $z_0$ and some $\eps>0$ such that 
$f_t$ is injective on $B$ for all $t\in[0,\eps]$. The inverse functions $g_t$ satisfy
\[ \frac{\partial}{\partial t}g_t(w) = -G(t,w)\]
for a.e.\ $t\in[0,\eps]$ and all $w\in \cap_{t\in[0,\eps]} f_t(B)$ (which is non-empty for $\eps$ small enough). This implies $g_t(w) = w-\int_0^t G(s,w) ds$, which shows $f_t=J_t$ on $B$ for all 
$t\in[0,\eps]$. The identity theorem implies $f_t=J_t$ on $D$ for all $t\in[0,\eps]$. 


In this way, we also see that the set of all 
$t\geq 0$ with $f_t=J_t$ is open (in $[0,\infty)$). At the same time, it is also closed, and thus equal to $[0,\infty)$.
\end{itemize}
\end{proof}

% \begin{remark}
% The conditions in  Theorem \ref{result1} (3) are necessary in the following sense.
%  Assume that $G(t,z)$ satisfies the conditions of Theorem \ref{result1} (3), but $H_t$ is not a generator for some $t>0$. If, in addition, $H_t$ is also bounded, then hen a solution to \eqref{m2} cannot exist until $t$.
% \end{remark}


We can also prove a slight variation of the second part of the theorem.

\begin{lemma}\label{lemma_1}
 Let $D\subset \C^n$ be bounded and convex and let $G:D\to\C^n$ be bounded and holomorphic. Let $(J_t)_{t\geq 0}$ be a family of holomorphic self-mappings of 
 $D$ such that \begin{equation*}\frac{\partial}{\partial t}J_t(w)=J'_t(w)\cdot G(J_t(w))\quad \text{for all $t\geq 0$}, \quad J_0(w)= w\in D.\end{equation*}
Then $G\in\G(D)$.
\end{lemma}
\begin{proof}
% Consider the differential equation 
% \[\frac{\partial}{\partial t}\varphi_t(z) = -G(J_t(\varphi_t(z))), \quad \varphi_0(z)=z.\]
% 
% Fix $z\in D$. We can solve this equation at least for $t$ small enough.
% A small computation shows that $\frac{d}{dt}[J_t(\varphi_t(z))]=0$, i.e. $J_t(\varphi_t(z))$ does not depend on $t$, or: $J_t(\varphi_t(z))=J_0(\varphi_0(z))=z$. 
% Hence $\frac{\partial}{\partial t}\varphi_t(z) = -G(z)$. We conclude that
% $t\mapsto \varphi_t(z)$ simply describes a straight line: \[\varphi_t(z)=z-tG(z).\]
% In particular, we can now define $\varphi_t(z)$ for all $z\in D$ and all $t\geq0$ by $\varphi_t(z)=z-tG(z)$.\\
% Let $D_t=\{z\in D\,|\, \varphi_t(z)\in D\}$. The convexity of $D$ implies that $D_t\subset D_s$  whenever $s\leq t$. Thus, for all $z\in D_t$, we have 
% $J_t(\varphi_t(z))=z$. \\
% Let $M_t = \varphi_t(D_t)\subset D$. Then $\varphi_t(J_t(z))=z$ for all $z\in M_t$. The identity theorem implies $\varphi_t(J_t(z))=z$ for all $z\in D$. Hence $\varphi_t(J_t(D))=D$, which shows 
% $D_t = J_t(D)$ and $M_t=D$. We conclude that the nonlinear resolvent equation of $G$ can be solved for all $t\geq0$.
The proof of part (2) of Theorem \ref{result1} of  shows that the nonlinear resolvent equation of $G$ can be solved for all $t\geq0$. 
It follows from the boundedness of $G$ and \cite[Corollary 1.2]{RS97} that $G\in\G(D)$.
\end{proof}


Finally, we see how \eqref{m} behaves with a different initial value.

\begin{lemma}\label{cor_1}
Let $G:D\to\C^n$ be a holomorphic function on a bounded and convex domain $D\subset \C^n$ and let $\varphi:D\to D$ be univalent such that 
$G \circ \varphi \in \G(D)$.
Then there exists a unique solution to the initial  value problem 
\begin{equation}\label{m3}\frac{\partial}{\partial t}J_t(w)=J'_t(w)\cdot G(J_t(w))\quad \text{for all $t\geq 0$}, \quad J_0 = \varphi.\end{equation}
We have $(J_t)_{t\geq0}=(\varphi \circ K_t)_{t\geq0}$, where $(K_t)_{t\geq0}$ are the nonlinear resolvents of $G\circ \varphi$.

%Furthermore, assume that $G:D\to\C^n$ is holomorphic and bounded, $\varphi:D\to D$ univalent, and let $(J_t)_{t\geq0}$ be a solution to \eqref{m3}. Then 
%$G\circ \varphi$ is an infinitesimal generator on $D$.
\end{lemma}
\begin{proof}
Let $(K_t)_{t\geq0}$ be the nonlinear resolvents of the generator $G\circ \varphi$. They satisfy 
\begin{equation}\label{help_0}\frac{\partial}{\partial t}K_t(z)=K'_t(z)\cdot G(\varphi(K_t(z))), \quad K_0(z) = z.\end{equation}
Now put $J_t = \varphi \circ K_t$. Then 
\begin{equation*}\frac{\partial}{\partial t}J_t(z)=(\varphi(K_t(z)))'\cdot K'_t(z)\cdot G(\varphi(K_t(z))) = J'_t(z) \cdot G(J_t(z)), \quad J_0 = \varphi.\end{equation*}
Next we show uniqueness of the solution. 
Let $(\hat{J}_t)_{t\geq0}$ be another solution of \eqref{m3}. Choose a compact ball $B\subset D$. There exists $\eps>0$ such that for all $t<\eps$ and for all 
$z\in B$ we have $\hat{J}_t(z)\in \varphi(D)$.  
Hence we can define $\hat{K}_t(z)=(\varphi^{-1}\circ \hat{J}_{t})(z)$ for all such $z$ and $t$. A small calculation shows that $t\mapsto \hat{K}_t(z)$ satisfies \eqref{help_0}. But the solution to this equation is unique 
due to Theorem \ref{result1} and we conclude $\hat{K}_t(z) = K_t(z)$ for all $z\in B$ and $t<\eps$. The identity theorem implies $\hat{J}_t(z) = J_t(z)$ for all $z\in D$ and all $t < \eps$.



Let $T>0$ the infimum of all $t$ with $\hat{J}_t\not= J_t$ and assume $T<\infty$. Then $\hat{J}_T=J_T$ due to continuity
 and thus $\hat{J}_T(D)=(\varphi\circ K_T)(D)\subset \varphi(D)$. Now a similar argument shows that $\hat{J}_t = J_t$ for all $t\in[0,T+\eps]$ for some $\eps>0$, a contradiction. 
%Consider the differential equation 
%\[\frac{\partial}{\partial t}\varphi_t(z) = -G(J_t(\varphi_t(z))), \quad \varphi_0(z)=z, z\in D.\]
%
%Fix $z\in D$. We can solve this equation at least for $t$ small enough.
%A small computation shows that $\frac{d}{dt}[J_t(\varphi_t(z))]=0$, i.e. $J_t(\varphi_t(z))$ does not depend on $t$ and thus $J_t(\varphi_t(z))=J_0(\varphi_0(z))=\varphi(z)$. 
%Hence $\frac{\partial}{\partial t}\varphi_t(z) = -G(\varphi(z))$. We conclude that
%$t\mapsto \varphi_t(z)$ simply describes a straight line: \[\varphi_t(z)=z-tG(\varphi(z)).\]
%In particular, we can now define $\varphi_t(z)$ for all $z\in \varphi(D)$ and all $t\geq0$ by $\varphi_t(z)=z-tG(\varphi(z))$.\\
%
%Let $D_t=\{z\in D\,|\, \varphi_t(z)\in D\}$. The convexity of $D$ implies that $D_t\subset D_s$  whenever $s\leq t$. Thus, for all $z\in D_t$, we have 
%$J_t(\varphi_t(z))=\varphi(z)$, or $\varphi^{-1}(J_t(g_t(z)))=z$. \\
%Let $M_t = g_t(D_t)\subset D$. Then $g_t(\varphi^{-1}(J_t(z)))=z$ for all $z\in M_t$. The identity theorem implies $g_t(\varphi^{-1}(J_t(z)))=z$ for all $z\in D$. 
 %We conclude that $\varphi^{-1}\circ J_t$ is the nonlinear resolvent of $G$. It follows from the boundedness of $G$ and \cite[Corollary 1.2]{RS97} that $G$ is an infinitesimal generator on $D$.
\end{proof}


For the rest of this section, we look at the unit disc. First, we construct an example of a vector field $G(t,z)$ as in Theorem \ref{result1} (3) whose solution is not a decreasing Loewner chain.

\begin{example}\label{no_solution}
Let $D=\D$ and consider the two infinitesimal generators 
\[G_1(z) = -z\frac{1 + z}{1 - z}\quad \text{and} \quad G_2(z)=-z\frac{1- z}{1+ z}.\] 
We define the Herglotz vector field $G(t,z)$ by $G(t,z)=G_1(z)$, $t\leq 1$, $G(t,z)=G_2(z)$, $t>1$. 
As any combination $sG_1+tG_2$, $s,t\geq0$, is again a generator on $\D$, the conditions in Theorem \ref{result1} (3) are satisfied and we let $(J_t)_{t\geq 0}$ be the solution to \eqref{intro_2_eq}. 



For $t\leq 1$, $J_t$ is simply the resolvent of $G_1$. For $t=1$, the solution of $w=z+tz(1+z)/(1-z)$ is given by $J_{1}(w)=w/(w+2)$. 



For $t\geq 1$, we have \[\frac{\partial}{\partial t}J_t(w)=J_t'(z)\cdot G_2(J_t(z)),\quad  J_{1}(z)=z/(z+2).\]
Here, the conditions of Lemma \ref{cor_1} are not satisfied: Assume that $H:=G_2\circ J_{1}\in\G(\D)$.


 
We have $H(z)=-J_{1}(z)(1- J_{1}(z))/(1+ J_{1}(z))$. As $H(0)=0$ and $H$ is not constant $0$, the Berkson-Porta formula \eqref{Berkson-Porta} implies 
$\Re\left(-H(z)/z\right)>0$ on $\D$. However, for $z=e^{3i}$ we obtain $\Re\left(J_{1}(z)(1- J_{1}(z))/(1+ J_{1}(z))/z\right) = \Re\left(\frac1{2 + 3 e^{3 i} + e^{6 i}}\right)=-0.47113... < 0$.



The solution $(J_t)_{t\geq0}$ can thus not be a decreasing Loewner chain. This can be seen as follows:



If it was a decreasing Loewner chain, we can define the mappings $(J_{1}^{-1}\circ J_t)_{t\geq 1}$. With the proof of Lemma
 \ref{cor_1}, we see that these functions satisfy the conditions of Lemma \ref{lemma_1} for the function $G_2\circ J_{1}$. Hence, $G_2\circ J_{1}$ 
would be a generator. (In order to apply Lemma \ref{lemma_1}, $G_2\circ J_{1}$ has to be bounded. But we can consider the whole construction with $G_2(z)$ being replaced by the generator $G_2(Rz)$ for
some $R\in(0,1)$ and close enough to $1$.)
\end{example}


\begin{example}\label{solution_exists} If we replace the generator $G_2$ in Example \ref{no_solution} by $G_2(z)=-z$, then $-G_2(J_1(z))/z=J_1(z)/z=1/(z+2)$ has positive real part in 
$\D$. By Lemma \ref{cor_1} and Theorem \ref{result1} (1), equation \eqref{intro_2_eq} has a unique solution, which is a decreasing Loewner chain. 
Hence we see that there do exist vector fields as in Theorem \ref{result1} (3) which are non-constant with respect to time such that the solution to \eqref{intro_2_eq} is a decreasing Loewner chain.
\end{example}

\begin{remark}
Numerical calculations suggest that every Herglotz vector field which jumps at some $T>0$ from $G_1(z) = -z(1 + e^{i\alpha}z)/(1 - e^{i\alpha}z)$ to  $G_2(z)=-z(1+ e^{i\beta}z)/(1- e^{i\beta} z)$, $\alpha,\beta\in\R$, 
leads to the same situation as in Example \ref{no_solution}, i.e.\ the solution is not a decreasing Loewner chain.



It seems to be less clear to decide whether the solution to \eqref{intro_2_eq} is a decreasing Loewner chain if $G(t,z)$ varies continuously with respect to $t$. For example, 
consider the ``slit'' Herglotz vector field \[G(t,z)=-z \frac{\kappa(t)-z}{\kappa(t)+z}\] for a continuous function $\kappa:[0,\infty)\to\partial\D$. 
There is quite some literature on the question as to which conditions on $\kappa$ guarantee that \eqref{intro_1_eq} produces slit mappings, i.e., for all $t>0$, 
the image domain of $f_t$ should have the form $f_t(\D)=\D\setminus \gamma_t$ for a simple curve $\gamma_t\subset \overline{\D}$. (Roughly speaking, if $\kappa$ is smooth enough, e.g. continuously differentiable, then $f$ is a slit mapping. We refer to \cite{ZZ18} and the references therein for such results.)
 

We might ask: for which $\kappa$ does the solution to \eqref{intro_2_eq} form a decreasing Loewner chain?
\end{remark}

In the case of the unit disc, we can obtain some further information about the Herglotz vector field $G\circ J_t$ appearing in \eqref{m}.

\begin{theorem}
\label{expression-G-theorem}
Let $G(z) = (\tau-z)(1-\overline{\tau}z)p(z)\in \G(\D)$ with resolvents $(J_t)_{t\geq 0}$. Then the Herglotz vector field $G_t := G\circ J_t$ can be written as 
\begin{equation}
\label{expression-G-equation}G_t(z)=(\tau-z)(1-\overline{\tau}z)p_t(z) \quad \text{with} \quad 
p_t(z) = \frac{1}{t}\frac{z-J_t(z)}{(z-\tau)(1-\overline{\tau}z)}, \Re p_t \geq 0, \quad \text{for all} \quad t>0.
\end{equation}
\end{theorem}
\begin{proof}If $p(z)\equiv 0$, then $J_t(z)\equiv z$ and the statement trivially holds true. So assume that $p\not\equiv 0$. Then the Herglotz vector field $G_t=G\circ J_t$ can be written as $G_t(z)=\frac{z-J_t(z)}{t}=(z-\tau_t)(1-\overline{\tau_t}z)p_t(z)$ for some   functions $\tau_t:[0,\infty)\to \overline{\D}$, $p_t:[0,\infty)\to \{f:\D\to \C\,|\, \text{$f$ holomorphic}, \Re f \geq 0\}$.


Case 1: $\tau \in \D$. Then $G_t(\tau)=0$ and thus $J_t(\tau)=\tau$, which shows $G_t(\tau)=0$ and thus $\tau_t=\tau$ for all $t\geq0$.


Case 2: $\tau\in \partial \D$. 
It is shown in \cite[Lemma 5.1]{ESS20} (plus the text following Lemma 5.1) 
that the Denjoy-Wolff point of $J_t$ is equal to $\tau$ for each $t>0$. For the sake of completeness, we add the necessary details.


 
By \cite[Theorem 1]{CDMP06} and  \cite[The Grand Iteration Theorem, p. 78]{shapiro}, we know that $G(\tau)=0$ and $G'(\tau)\leq 0$ in the sense of angular limits. Thus, for $t>0$, the function $\varphi_t(z)=z-tG(z)$ satisfies $\varphi_t(\tau)=\tau$ and $\varphi'_t(\tau)\geq 1$.



As $G$ has no zeros in $\D$, $\varphi_t$, and thus also $J_t$, has no fixed points in $\D$. Hence, the Denjoy-Wolff point $\rho$ of $J_t$ lies on $\partial \D$. We have $J_t(\rho)=\rho$ and $J_t'(\rho)\in (0,1]$. This implies $\varphi_t'(\rho)=\rho$ and $\varphi_t'(\rho)\geq 1$ and thus $G(\rho)=0$ and $G'(\rho)\leq 0$. But then $\rho$ is the Denjoy-Wolff point of the semigroup associated to $G$, i.e.\ $\rho=\tau$, again, by \cite[Theorem 1]{CDMP06} and  \cite[The Grand Iteration Theorem, p. 78]{shapiro}.

We conclude $J_t(\tau)=\tau$ and $J_t'(\tau)\in (0,1]$. As $G_t(z)=(J_t(z)-z)/t$, we have $G_t(\tau)=0$ and $G_t'(\tau)\leq 0$, again in the sense of angular limits. This implies that the 
Denjoy-Wolff point of the semigroup associated to $G_t$ is equal to $\tau$. We conclude $\tau_t=\tau$ for all $t\geq 0$.
%Due to \cite[Theorem 2.3]{CDM13}, $\tau_t=\tau$ for almost all $t\geq0$. As $\tau_t$ is continuous, $\tau_t = \tau$ for all $t\geq0$. \\
%Note that \cite[Theorem 2.3]{CDM13} refers to evolution families, but, as shown in the proof of Theorem \ref{theorem_1}, decreasing Loewner chains correspond to evolution families by a simple time reversion. 
%{\color{blue}We have $G'(J_t(\tau)) \cdot J_t'(\tau)=\frac{G'(\tau)}{1-t G'(\tau)}$. Thus, if $\Re p >0$, this Loewner chain should be of divergence type...}
\end{proof}






\section{Upper half-plane and Denjoy-Wolff point at infinity}
\label{sec_free}




In \cite[Section 4 and Appendix A1]{HS}, the authors explain that free semigroups of probability measures lead to 
certain decreasing Loewner chains. By comparing the differential equation of these Loewner chains to \eqref{intro_2_eq}, we see that such semigroups are closely 
related to nonlinear resolvents of certain infinitesimal generators on the upper half-plane $\uhp$.


In the following we will use the Nevanlinna representation formula for Pick functions, which corresponds to the Herglotz representation for Carath\'eodory functions.


\begin{lemma}[e.g. {\cite[Theorem 1]{cau32}}]
\label{Nev-expression}
If $q\in \Nev$, then there
exist real constants $\alpha\ge0$ and $\beta$
and a finite non-negative Borel measure $\rho$ on $\R$ such that
\begin{equation}
\label{Nev-equation}
q(z)=\alpha z+\beta+\int_\R \frac{1+tz}{t-z}d\rho(t),\quad z\in\uhp.
\end{equation}
The number $\alpha$ and $\beta$ are calculated via $\alpha = \lim_{y\to\infty}f(iy)/(iy)$ and $\beta = \Re q(i)$, respectively.
Conversely, the function expressed as above is a Pick function.
\end{lemma}




\subsection{Additive convolution}

Let $\mu$ be a probability measure on $\R$. The Cauchy transform (or Stieltjes transform) of a probability measure $\mu$ on $\R$ is given by 
$$G_{\mu}(z):=\int_\R\frac{1}{z-t}\,\mu(\mathrm{d}t), \quad z\in\uhp=\{z\in\C\,|\, \Im(z)>0\}.$$
The $F$-transform of $\mu$ is simply defined by $F_\mu(z):=1/G_\mu(z)$ for $z\in\uhp$.


\begin{remark}
By the definition, the $F$-transform belongs to $\Nev$. Further it is known (e.g. \cite{M92}) that it has the Nevanlinna representation
$$
F_{\mu}(z)= z+\beta+\int_\R \frac{1+xz}{x-z}\rho(\mathrm{d}x), \quad z \in \uhp,
$$
with $\beta \in \R$ and $\rho$ is a finite non-negative Borel measure on $\R$.
In view of Lemma \ref{Nev-expression}, the above fact implies that $F_\mu$ satisfies $\lim_{y \to \infty}F_{\mu}(iy)/iy = 1$.
\end{remark}



$G_\mu$ maps the upper half-plane $\uhp$ into the lower half-plane.
For some positive real constants $\alpha, \beta>0$, $G_\mu$ is univalent in the set $\Gamma_{\alpha,\beta}=\{z\in \uhp\,|\, \Re(z)<\alpha \Im(z), |z|>\beta \}$ and we can define the compositional right inverse $G_\mu^{-1}$ there. We define $R_\mu(z):=G_\mu^{-1}(z)-\frac1{z}$, which is a holomorphic function that maps some subdomain of the lower half-plane into the lower half-plane or into $\R$; see \cite[Section 5]{BV93}.



 Instead of $R_\mu$, one might also regard the Voiculescu transform $\varphi_\mu(z)$ defined by $\varphi_\mu(z):=F_\mu^{-1}(z)-z=G_\mu^{-1}(1/z)-z=R_\mu(1/z)$. It maps some subdomain of $\uhp$ into the lower half-plane or $\R$.

\begin{example}\label{ex_semicircle}The semicircle distribution $W(0,1)$, with mean $0$ and variance $1$, is given by the density 
\[\frac{1}{2\pi} \sqrt{4 -x^2}, \quad  x\in [-2,2].\] 
We have $G_{W(0,1)}(z)=\frac{z-\sqrt{z^2-4}}{2}$, where the branch of the square root is chosen such that $\sqrt{\cdot}$ maps $\C^2\setminus [-4,\infty)$ into the upper half-plane. 
Solving the equation $\frac{z-\sqrt{z^2-4}}{4}=w$ gives $z=w+\frac{1}{w}$ and thus \[R_{W(0,1)}(z)= z, \quad \varphi_\mu(z)=\frac1{z}.\]
Furthermore, $F_{W(0,1)}(z)=\frac{2}{z-\sqrt{z^2-4}}$, and as $F_{W(0,1)}$ extends continuously to $\uhp\cup \R$, $F_{W(0,1)}(\uhp)$ is the unbounded complement of the curve $\{F_{W(0,1)}(x)\,|\, x\in(-2,2)\}$ in $\uhp$. A small calculation yields $\Re(G_{W(0,1)}(x))=\frac{x}{2}$ and $\Im(G_{W(0,1)}(x))= \frac{-\sqrt{4-x^2}}{2}$ for $x\in[-2,2]$. Thus, for $x\in(-2,2)$, we have 
\[F_{W(0,1)}(x) = \frac1{\frac{x}{2}+i\frac{-\sqrt{4-x^2}}{2}}=
\frac{2}{x-i\sqrt{4-x^2}}=
\frac{2x+2i\sqrt{4-x^2}}{4}=\frac{x}{2}+i\sqrt{1-\left(\frac{x}{2}\right)^2},\]
and we see that the curve is the semicircle $(\partial \D)\cap \uhp$.

 \begin{figure}[ht]
 \begin{center}
 \includegraphics[width=0.9\textwidth]{semi_F.png}
 \caption{$F$-transform of the semicircle distribution $W(0,1)$.}
 \end{center}
 \end{figure}
\end{example}


For two probability measures $\mu$ and $\nu$ on $\R$, the
\textit{additive free convolution} $\boxplus$ is defined by (see \cite{BV93})
\begin{equation*}
\varphi_{\mu \boxplus \nu}(z) = \varphi_{\mu}(z)  + \varphi_{\nu}(z).
\end{equation*}
This convolution arises from the notion of \textit{free independence} introduced by D. Voiculescu. Introductions to free probability theory can be found in \cite{ns06}, \cite{Voi97}.

A probability measure $\mu$ on $\R$ is called \textit{freely infinitely divisible} 
if for every $n\in\N$ there exists a probability measure $\mu_n$ on $\R$ such that 
$\mu = \mu_n \boxplus \cdots \boxplus \mu_n$ ($n$-fold convolution).




\begin{theorem}[Theorem 5.10 in \cite{BV93}] \label{thmBV93}
For a probability measure $\mu$ on $\R$, the following statements are equivalent.
\begin{enumerate}[\rm(1)]
\item $\mu$ is freely infinitely divisible.
\item $\mu=\mu_1$  for a $\boxplus$-semigroup $(\mu_t)_{t\geq0}$
 (i.e.\ $\mu_0=\delta_0,$ $\mu_{t+s}=\mu_t \boxplus \mu_s$ for all $s,t\geq0$ and $t\mapsto\mu_t$ is continuous with respect to weak convergence).
%\item For any $t>0$, there exists a probability measure $\mu^{\boxplus t}$ with the property $\varphi_{\mu^{\boxplus t}}(z) = t\varphi_\mu(z).$
\item $-\varphi_\mu$ extends to a Pick function, i.e.\ a holomorphic function from $\mathbb{H}$ into $\mathbb{H} \cup \R$.
\item\label{FLK1} There exist $a \in \R$ and a finite non-negative measure $\gamma$ on $\R$ such that
\begin{equation}\label{form0}
\varphi_\mu(z)=a +\int_{\mathbb{R}}\frac{1+z x}{z-x} \rho({\rm d}x) ,\qquad z\in \mathbb{H}.
\end{equation}
\end{enumerate}
Conversely, given $a\in\R$ and a finite non-negative measure $\rho$ on $\R$, there exists a unique $\boxplus$-infinitely divisible distribution $\mu$ which has the Voiculescu transform of the form \eqref{form0}.
\end{theorem} 




%Every holomorphic mapping $f$ from $\uhp$ into $\uhp\cup \R$ can be written as
%\begin{equation}\label{nevanlinna}
%f(z)=a + bz + \int_{\mathbb{R}}\frac{1+z x}{x-z} \gamma({\rm d}x) ,\qquad z\in \mathbb{H},
%\end{equation}
%with a non-negative  measure $\gamma$ on $\R$ and some $a\in \R, b\geq 0$ (Nevanlinna representation, see \cite[Theorem 1]{cau32}). The number $b$ can be calculated via 
%$b = \lim_{y\to\infty}f(iy)/(iy)$. Conversely, every such triple $(a, b, \gamma)$ produces a holomorphic function from $\uhp$ into $\uhp\cup \R$.}



\begin{example}The Dirac measure $\mu=\delta_0$ is freely infinitely divisible as $\varphi_{\mu}(z)\equiv 0$. 
Example \ref{ex_semicircle} shows that the semicircle distribution $W(0,1)$ is also freely infinitely divisible. $W(0,1)$ is the ``normal distribution of free probability theory'', i.e.\ the limit distribution in the free central limit theorem, see \cite[Chapter 2]{MS17}.
\end{example}



%The functions with $b=0$ correspond to all holomorphic mappings $f$ from $\uhp$ into $\uhp\cup \R$ with $\lim_{y\to\infty}f(iy)/(iy)=0$.
%\cite[p. 20, Theorem 1]{Don74}.

%\begin{theorem}\label{add_gen}If $J_t = F_{\mu_t}$ for a free $\boxplus$-semigroup of probability measures $\mu_t$ on $\R$, then 
%$J_t$ are the resolvents of the generator $-\varphi_{\mu_1}$.
%%Conversely, let $\phi:\uhp\to \uhp\cup\R$ be holomorphic such that $-\phi$ is of the form \eqref{form0}.
%%The resolvents $J_t$ of $\phi$ on $\uhp$ can be written as $J_t = F_{\mu_t}$ for a free $\boxplus$-semigroup of probability measures $\mu_t$ on $\R$.
%\end{theorem}
%Conversely, let $\phi:\uhp\to \uhp\cup\R$ be holomorphic such that $-\phi$ is of the form \eqref{form0}.
%The resolvents $J_t$ of $\phi$ on $\uhp$ can be written as $J_t = F_{\mu_t}$ for a free $\boxplus$-semigroup of probability measures $\mu_t$ on $\R$.


%\begin{proof}
%Let  $\{\mu_t\}_{0\leq t}$ be a $\boxplus$-semigroup and let $G_t = G_{\mu_t}$. Then $\varphi_{\mu_t}=t\varphi_{\mu_1}$ and each 
%$\varphi_{\mu_t}$ is defined on the whole upper half-plane. We conclude that
%\[ \frac{\partial}{\partial t}G_t(z) = -\frac{\partial}{\partial z}G_t(z) \cdot R_{\mu_1}(G_t(z)), \quad G_0(z)=1/z.\]
%Put $J_t=F_{\mu_t}=1/G_t$. Then
%\[ \frac{\partial}{\partial t}J_t(z) = -\frac{\partial}{\partial z}J_t(z) \cdot \varphi_{\mu_1}(J_t(z)), \quad J_0(z)=z.\]
%As $-\varphi_{\mu_1}$ maps $\uhp$ into $\uhp\cup \R$,  Lemma \ref{result2} now implies 
%that $J_t$ are the resolvents of the generator $-\varphi_{\mu_1}$.
%\end{proof}

Next we prove Theorem  \ref{intro_3}.

 
\begin{theorem}
\label{add_gen_nonaut} 
Let $G(t,z)$ be a Herglotz vector field on $\uhp$ such that, for a.e.\ $t\geq0$, 
$z\mapsto G(t,z)$ maps $\uhp$ into $\uhp\cup \R$ and $\lim_{y\to\infty} G(t,iy)/y=0$. 
Then there exists a unique solution $(f_t)_{t\geq0}$, locally absolutely continuous in $t$, to
\begin{equation}\label{op0} \frac{\partial}{\partial t}f_t(z)=f_t'(z)\cdot G(t,f_t(z))\quad \text{for a.e.\ $t\geq 0$}, \quad f_0(z)=z\in \uhp.
\end{equation}
The solution can also be written as $(f_t)=(F_{\mu_t})$ for a family of freely infinitely divisible probability measures $(\mu_{t})_{t \ge 0}$ on $\R$ and 
$(f_t)_{t\geq0}$ is a decreasing Loewner chain.



If $G(t,z)$ does not depend on $t$, then $G(t,z)=-\varphi_{\mu_1}(z)$ and $(\mu_t)_{t\geq0}$ is a free $\boxplus$-semigroup. Furthermore, 
the family $(f_t)_{t\geq0}$ are the nonlinear resolvents of $-\varphi_{\mu_1}$ in this case.
\end{theorem}



\begin{proof}
Consider $H_t(z):=\int_0^t -G(s,z) ds$, $t\geq 0$. This function is of the form \eqref{form0} 
and Theorem \ref{thmBV93} implies that we find freely infinitely divisible probability measures
 $(\mu_t)_{t\geq 0}$ such that $\varphi_{\mu_t}(z) = H_t(z)$. 
Put $f_t=F_{\mu_t}$. Then $f_t(\varphi_{\mu_t}(z)+z)=z$ for all $z\in\uhp$. Differentiation yields
 \[0=\frac{\partial}{\partial t}f_t(\varphi_{\mu_t}(z)+z) + 
f'_t(\varphi_{\mu_t}(z)+z)\frac{\partial}{\partial t}\varphi_{\mu_t}(z)=\frac{\partial}{\partial t}f_t(\varphi_{\mu_t}(z)+z) - 
f'_t(\varphi_{\mu_t}(z)+z)G(t,z).\] 
Put $w=z+\varphi_{\mu_t}(z)=f_t^{-1}(z)$. Then 
\[\frac{\partial}{\partial t}f_t(w) = f'_t(w) G(t,f_t(w))\]
for all $w$ in the image domain $f_t^{-1}(\uhp)$, and thus, in particular, for all $w\in \uhp$. Clearly, $f_0$ is the identity as 
$\varphi_{\mu_0}=0$. Due to Lemma \ref{H_gen}, $G(t,f_t(z))$ is a Herglotz vector field and Theorem \ref{thm_1} implies that 
$(f_t)$ is a decreasing Loewner chain. 
Uniqueness of the solution is shown as in the proof of Theorem \ref{result1} (3).

If $G(t,z)$ does not depend on $t$, then $G(t,z)=-\varphi_{\nu}$ for some probability measure $\nu$ and 
$\varphi_{\mu_t} = H_t(z) = t\varphi_{\nu}$. This shows that $\nu=\mu_1$ and that $(\mu_t)_{t\geq 0}$ is a free semigroup due to 
Theorem \ref{thmBV93}. Theorem \ref{result1} (2) now implies 
that the functions $(J_t)_{t\geq0}$ are the resolvents of the generator $-\varphi_{\mu_1}$.
\end{proof}

\begin{remark}The fact that the $F$-transforms of a $\boxplus$-semigroup $(\mu_t)_{t\geq 0}$ form a decreasing Loewner chain is also proved in 
\cite[Proposition 5.15]{Jek17}, even for the more general case of operator-valued free probability theory. Also in this higher dimensional 
case, the $F$-transforms, which are holomorphic functions on certain matricial upper half-planes, are the resolvents of $-1$ times
the Voiculescu transform of $\mu_1$ by essentially the same proof.  
\end{remark}

\begin{remark}Let $G$ be a generator of the form $-\varphi_{\mu_1}$. The nonlinear resolvents $(f_t)_{t\geq0}$ of $G$ form a decreasing Loewner chain satisfying the Loewner PDE with Herglotz vector field 
$G(t,z)=(G\circ f_t)(z)$. The semigroup $(g_t)_{t\geq0}$ associated to $G$ satisfies the Loewner PDE with Herglotz vector field $G(t,z)=G(z)$. 
While the mappings $f_t$ are the $F$-transforms of a semigroup with respect to free convolution, the mappings $g_t$ have a quite similar interpretation. For two probability measures $\mu$ and $\nu$, 
the monotone convolution $\mu\rhd \nu$ is defined by $F_{\mu\rhd \nu}(z)=(F_\mu \circ F_\nu)(z),$ $z\in \uhp$.


Now there exists a family of probability measures $(\mu_t)_{t \ge 0}$ on $\R$ such that $g_t = F_{\mu_t}$. 
As $F_{\mu_{s+t}} = F_{\mu_s} \circ F_{\mu_t}$, we see that this family is a semigroup with respect to monotone convolution, see \cite[Section 3]{FHS}, in particular \cite[Proposition 3.11]{FHS}. 
\end{remark}


A consequence of Theorem \ref{add_gen_nonaut} is the following result, which finishes the proof of part (1) of Theorem \ref{intro_2}.

\begin{corollary}
 Let $G:\uhp\to \uhp\cup\R$ be holomorphic with $b=\lim_{y\to\infty}G(iy)/(iy)$. Then the nonlinear resolvents $J_t$ of $G$ exist for all $t\in[0,1/b)$.
 If $b\not=0$, then the resolvent equation cannot be solved for $t\geq 1/b$.
\end{corollary}
\begin{proof}
 If $b=0$, then $G$ generates a free semigroup whose $F$-transforms yield the nonlinear resolvents of $G$ for all $t\geq0$ by Theorem \ref{add_gen_nonaut}.
 
 
 
Now assume that $b\not=0$. Then \eqref{Nev-equation} shows that $\hat{G}(z) = G(z)-bz$ also maps $\uhp$ into $\uhp\cup\R$. 
 The equation $w = z - tG(z)$ becomes $w = (1-tb)z - t\hat{G}(z)$. For $t\geq 1/b$ and $z\in\uhp$, the right side belongs to $-\uhp\cup \R$. Thus it has no solution $z\in\uhp$ for any $w\in \uhp$. 
For $t<1/b$, we can also write $w/(1-tb) = z - t\hat{G}(z)/(1-tb)$ and the previous case applied to the generator $\hat{G}(z)/(1-tb)$ shows the existence of the nonlinear resolvent $J_t$ of $G$. 
\end{proof}


\subsection{Multiplicative convolution}
For the sake of completeness we also discuss the case of multiplicative free convolution of probability measures on 
the unit circle $\T$. These measures again lead to nonlinear resolvents of certain generators, however, not on $\D$, as one might expect 
in the first place. Roughly speaking, the pullbacks of these measures to $\R$ via $x\mapsto e^{ix}$ lead again to nonlinear resolvents 
of generators on $\uhp$.

Let $\mu$ be a probability measure on $\T$. 
The moment generating function of $\mu$ is a holomorphic function on $\D$ defined by
\begin{equation*}
\psi_\mu(z):=\int_{\T}\frac{xz}{1-xz}\, \mu({\rm d}x)=\sum_{n=1}^\infty 
\left(\int_\T x^n \, \mu({\rm d}x)\right) z^n, \quad z \in \D.
\end{equation*}

The classical independence of random variables leads to the classical convolution, 
or Hadamard convolution, $\mu \star \nu$, with $\psi_{\mu \star \nu}=
\sum_{n=1}^\infty 
\left(\int_\T x^n \, \mu({\rm d}x)\right)
\left(\int_\T x^n \, \nu({\rm d}x)\right) z^n$.
Other notions of independence from non-commutative probability theory lead to further 
convolutions. First, we need to define the $\eta$-transform of $\mu$. Let 
\begin{equation*}
\eta_\mu(z):=\frac{\psi_\mu(z)}{1+\psi_\mu(z)}, \quad z \in \D.
\end{equation*}

We denote by $\cP(\T)$ the set of all 
probability measures $\mu$ on $\T$. We let $\cP_{\times}(\T)$ be the set of all $\mu\in \cP(\T)$ with $\eta_\mu'(0)\not=0$,
i.e.\ the first moment of $\mu$ is $\not=0$. Then 
we can invert $\eta_\mu$ in a neighbourhood of $0$. Denote this locally defined function by 
$\eta_\mu^{-1}$.  The $\Sigma$-transform of $\mu$ is defined by 
\begin{equation*}
\Sigma_\mu(z):=\frac{1}{z}\eta_\mu^{-1}(z).
\end{equation*}

For two probability measures $\mu,\nu\in\cP_{\times}(\T)$, Voiculescu \cite{Voi87} characterized
\textit{multiplicative free convolution} $\boxtimes$ by
\begin{equation}\label{FM}
\Sigma_{\mu \boxtimes \nu}(z) = \Sigma_{\mu}(z) \Sigma_{\nu}(z)
\end{equation}
in a neighbourhood of $0$. 


$\mu\in\cP_{\times}(\T)$ is called \textit{freely infinitely divisible} 
if for every $n\in\N$ there exists $\mu_n \in \cP_{\times}(\T)$ such that 
$\mu = \mu_n \boxtimes \cdots \boxtimes \mu_n$ ($n$-fold convolution).
Freely infinitely divisible distributions can be characterized in the following way.

\begin{theorem}[See Lemma 6.6 in \cite{BV92}]\label{MFID}
Let $\mu\in\cP_{\times}(\T)$. Then the following three statements are equivalent.
\begin{enumerate}[\rm(1)]
\item $\mu$ is freely infinitely divisible.
\item\label{CSUMFI} There exists a $\boxtimes$-semigroup $(\mu_t)_{t\geq 0}$ 
(i.e.\ $\mu_0=\delta_1$, $\mu_{t+s}=\mu_t \boxtimes \mu_s$ for all $s,t\geq0$ and $t\mapsto\mu_t$ is continuous with respect to weak convergence) 
such that $\mu_1 = \mu$.
\item\label{UMFI} There exists a holomorphic function $u_\mu\colon\D \to\C$ with $\Re(u_\mu)\geq 0$  such that $\Sigma_\mu(z) = \exp(u_\mu(z))$.
\end{enumerate}
Moreover, the analytic map $u_\mu$ in \eqref{UMFI} can be characterized by the Herglotz representation
\begin{equation}\label{VectorFUMFI}
u_\mu(z) = -i \alpha +\int_{\T} \frac{1+ z \zeta}{1-z\zeta}\rho(\mathrm{d} \zeta),
\end{equation}
where $\alpha \in \R$ and $\rho$ is a finite non-negative Borel measure on $\T$.

Conversely, for any analytic map $u\colon\D \to\C$  with $\Re(u)\geq 0$,  the function $\exp(u(z))$ is the $\Sigma$-transform of some freely infinitely divisible $\mu$.
\end{theorem}

\begin{remark}
 The multiplicative free convolution can also be considered for probability measures $\mu$
 with zero mean. Such a measure is freely infinitely divisible if and only if $\mu$ is the uniform distribution on 
 $\T$, i.e.\ $\psi_\mu(z)\equiv 0$, see \cite[Lemma 6.1]{BV92}.
\end{remark}

%Let $\eta_t=\eta_{\mu_t}$, where $\{\mu_t\}$ is a semigroup as in \eqref{CSUMFI} 
%of Theorem \ref{MFID}. Furthermore, assume that $u_\mu$ does not have the form $u_\mu(z)\equiv xi$ for some $x\in \R$.
 %We exclude this simple case which only leads to 
%simple rotations of point measures.
%Then by \eqref{UMFI} of Theorem \ref{MFID}, we have \[\eta_t^{-1}(z) = z\exp(t u_\mu(z)).\]
%
%%$\frac{\partial}{\partial t}z\cdot\Sigma_t=u_{\mu}\cdot z\Sigma_t$, 
%%i.e. $\frac{\partial}{\partial t}\eta_{t}^{-1}(z)=u_{\mu}\cdot \eta_{t}^{-1}(z)$.
%This yields the differential equation
%\begin{equation*}
%\frac{\partial}{\partial t}\eta_t(z) = -z u_\mu(\eta_t(z)) \cdot \frac{\partial}{\partial z}\eta_t(z).
%\end{equation*}
%
%We put $M_t(z)=\frac{1+\eta_t(z)}{1-\eta_t(z)}=1+2\psi_{\mu_t}$. 
%Then we obtain the partial differential equation
%\begin{equation*}\label{addd}
%\frac{\partial}{\partial t}M_t(z) = -z S_\mu(M_t(z)) \cdot \frac{\partial}{\partial z}M_t(z), 
%\quad M_0(z)\equiv z,
%\end{equation*}
%with $S_\mu(z)=u_{\mu}(\frac{1-z}{1+z})$. 
%Note that $S_\mu$ maps the right half-plane $RH$ holomorphically into itself. \\
%
%For $\Im(z)>0$ define $V_t(z)=iM_t(e^{iz})$. Then $V_t$ maps the upper half-plane into itself and we obtain 
%\begin{equation}\label{mmm}
 %\frac{\partial V_t(z)}{\partial t}= i S(-iV_t(z)) \frac{\partial V_t(z)}{\partial z}=
  %G(V_t(z)) \frac{\partial V_t(z)}{\partial z}, 
%\end{equation}
%with $G(z)=iS(-iz)$, which maps $\uhp$ into itself. Recall that $G$ is an infinitesimal generator of $\uhp$, see Remark \ref{upper}.


\begin{theorem}\label{mult_gen}Let $J_t$ be defined by $(i\frac{1+e^{iz}}{1-e^{iz}})\circ J_t = iM_t(e^{iz})$, where $M_t(z)=\int_\T \frac{x+z}{x-z}\mu_t(dx)$ for a free
$\boxtimes$-semigroup of probability measures $\mu_t$ on $\T$. Then $J_t$ are the resolvents of the infinitesimal generator
$G(z)=iu_{\mu_1}(-e^{iz})$ on $\uhp$.
%Conversely, let $G:\uhp \to \uhp$ be of the form $G(z)=iu(\frac{1+iz}{1-iz})$ for a function $u$ of the form \eqref{VectorFUMFI}.
 %The resolvents $J_t$ of $G$ on $\uhp$ can be written as $J_t = iM_t(e^{iz})$, where $M_t(z)=\int_\T \frac{x+z}{x-z}\mu_t(dx)$ for a free $\boxtimes$-semigroup of probability measures $\mu_t$ on $\T$.
\end{theorem}



\begin{proof}
Let $\eta_t=\eta_{\mu_t}$, where $(\mu_t)_{t\geq0 }$ is a semigroup as in \eqref{CSUMFI} of Theorem \ref{MFID}.
 %Furthermore, assume that $u_\mu$ does not have the form $u_\mu(z)\equiv xi$ for some $x\in \R$.
 %We exclude this simple case which only leads to 
%simple rotations of point measures.
Then by \eqref{UMFI} of Theorem \ref{MFID}, we have \[\eta_t^{-1}(z) = z\exp(t u_\mu(z)).\]

%$\frac{\partial}{\partial t}z\cdot\Sigma_t=u_{\mu}\cdot z\Sigma_t$, 
%i.e. $\frac{\partial}{\partial t}\eta_{t}^{-1}(z)=u_{\mu}\cdot \eta_{t}^{-1}(z)$.
This yields the differential equation
\begin{equation*}
\frac{\partial}{\partial t}\eta_t(z) = -z u_\mu(\eta_t(z)) \cdot \frac{\partial}{\partial z}\eta_t(z).
\end{equation*}

We put $M_t(z)=\frac{1+\eta_t(z)}{1-\eta_t(z)}=1+2\psi_{\mu_t}$. 
Then we obtain the partial differential equation
\begin{equation*}\label{addd}
\frac{\partial}{\partial t}M_t(z) = -z S_\mu(M_t(z)) \cdot \frac{\partial}{\partial z}M_t(z), 
\quad M_0(z)= \frac{1+z}{1-z},
\end{equation*}
with $S_\mu(z)=u_{\mu}(\frac{1-z}{1+z})$. 
Note that $S_\mu$ maps the right half-plane $RH$ holomorphically into $RH\cup i\R$. 

For $\Im(z)>0$ define $V_t(z)=iM_t(e^{iz})$. Then $V_t$ maps the upper half-plane into itself and we obtain 
\begin{equation*}
 \frac{\partial V_t(z)}{\partial t}= i S(-iV_t(z)) \frac{\partial V_t(z)}{\partial z},\quad V_0(z)=i\frac{1+e^{iz}}{1-e^{iz}}. 
\end{equation*}
According to \cite[Theorem 4.7]{HS}, $V_t$ is a decreasing subordination chain and we can write $V_t = V_0 \circ J_t$ with 
$J_t$ satisfying the same equation (see the proof of \cite[Theorem 2.2]{HS}), i.e.
\begin{equation*}
 \frac{\partial J_t(z)}{\partial t}= i S(-iV_t(z)) \frac{\partial J_t(z)}{\partial z}=
 i S(-i (V_0 \circ J_t)(z)) \frac{\partial J_t(z)}{\partial z}=G(J_t(z))\frac{\partial J_t(z)}{\partial z},\quad J_0(z)=z \in \uhp, 
\end{equation*}
 with $G(z)=i S(-i V_0(z))=iu_{\mu_1}(-e^{iz})$, which maps $\uhp$ into $\uhp\cup \R$. Theorem \ref{result1} (2) finishes the proof.
\end{proof}

\begin{remark}
As $iu_{\mu_1}(-e^{iz})$ maps $\uhp$ into $\uhp\cup \R$ and $\lim_{y\to\infty}iu_{\mu_1}(-e^{-y})/(iy)=0$, we see that the generators from Theorem 
\ref{mult_gen} form a subset of the generators appearing in Theorem \ref{intro_2} (1).
\end{remark}


%\section{Geometric properties?}

%In  \cite{dMHS18}, the authors obtain a family of probability measures $(\mu_t)_{t\geq0}$ on $\R$ given by the following differential equation:\\
%
%Put $M_t(z)=\int_\R \frac{1}{z-u}\mu_t(du)$, which is a holomorphic mapping from the upper half-plane $\uhp$ into the lower half-plane. Then
%\begin{equation}
 %\frac{\partial}{\partial t}M_t = -2 M_t(z) \frac{\partial}{\partial z}M_t(z), \quad M_0(z) = \int_\R \frac1{z-u}\mu_0(du).
%\end{equation}
%
%In \cite{HS}, they also consider the more general equation
%
%\begin{equation}
 %\frac{\partial}{\partial t}M_t = R(M_t(z)) \frac{\partial}{\partial z}M_t(z), \quad M_0(z) = \int_\R \frac1{z-u}\mu_0(du),
%\end{equation}
%where $R$ maps the lower half-plane $-\uhp$ holomorphically into $\uhp\cup \R$. This is connected to nonlinear resolvents as follows:
%
%Consider the differential equation 
%\[\frac{\partial}{\partial t}\varphi_t(z) = -R(M_t(\varphi_t(z))), \varphi_0(z)=z.\]
%Locally, we can solve this equation. A small computation shows that $\frac{d}{dt}[M_t(\varphi_t(z))]=0$, i.e. $M_t(\varphi_t(z))$ does not depend on $t$, or: $M_t(\varphi_t(z))=M_0(\varphi_0(z))=M_0(z)$. \\
%We conclude that $t\mapsto \varphi_t(z)$ simply describes a straight line, \[\varphi_t(z)=\varphi_0(z)-t R(M_0(z))=z-tR(M_0(z)).\] In other words: $\varphi_t^{-1}(w)=J_t(w, R\circ M_0)$. 
%(Recall that every holomorphic mapping from $\uhp$ into $\uhp\cup \R$, thus $R\circ M_0$, is an infinitesimal generator.)
%Note that $M_t(z) = M_0(\varphi_t^{-1}(z))$; so the evolution of $M_t$ basically is the evolution given by the decreasing Loewner chain $(\varphi_t^{-1})_{t\geq 0}=(J_t)_{t\geq 0}$, which satisfies
%
%\begin{equation}\label{1}
 %\frac{\partial}{\partial t}J_t =  S(J_t(z)) \frac{\partial}{\partial z}J_t(z), \quad J_0(z) = z,
%\end{equation}
%
%where $S=R\circ M_0$, which maps $\uhp$ into the upper half-plane.

%\item[(1)] {\color{blue} How does $J_t$ transform when passing from $\uhp$ to the unit disc $\D$?}\footnote{In general, it is not true that 
%the nonlinear resolvent of a conjugated semigroup is equal to the conjugation of the nonlinear resolvent of the original semigroup:\\
%Take the semigroup $F_t(z)=e^{-t}z$ with generator $G(z)=-z$ and $J_t(w)=w/(1+t)$. We have $C^{-1}\circ J_t \circ C(z)=i\frac{2z+r(z+i)}{2i+r(z+i)}$. \\
%However, the generator of $C^{-1}\circ F_t \circ C$ is given by $H(z)=\frac{i}{2}(z^2-1)$, leading to a nonlinear resolvent which is not a Moebius transform.}

%old version:
%We can take $J_t$ from \eqref{1} and pass to the unit disc $\D$ by using the Cayley transform. Put $K_t = C\circ J_t \circ C^{-1}$. Then 
%$K_t$ is the nonlinear resolvent of an infinitesimal generator on $\D$. (This should follow immediately from the product formula \cite[Theorem B]{ESS20}). 
%Put $\hat{S}=S\circ C^{-1}$. Then {\color{blue}(this calculation should be rechecked..)}
%\begin{equation}
% \frac{\partial}{\partial t}K_t = - S\circ C^{-1} \circ K_t \cdot   \frac{\partial}{\partial z}K_t(z) / (C^{-1})' = 
% -  \hat{S}(K_t) \cdot  \frac{\partial}{\partial z}K_t(z) \cdot -(i/2) (-1 + z)^2, \quad K_0(z) = z.
%\end{equation}


 %Assume that $G(z)=-zp(z)$, where $p$ maps $\D$ into the right half-plane.
% In \cite[Theorem 4.6.]{ESS20} it is shown by using Betker's theorem that if $G(z)=-zp(z)$ is nice enough, then $J_t$ has quasiconformal extension.  Can we generalize this result? See also hyperbolic convexity..


  %\item[(3)] Consider the general case $G(z) = (\tau-z)(1-\overline{\tau}z)p(z)$. Then the calculations in \cite{ESS20} imply
%\[\frac{\partial}{\partial t}J_t(w) = \frac{\partial}{\partial t}J_t(w) G(J_t(w)).\] 
%So we have the Loewner equation with Herglotz vector field $G(J_t(w))=\frac{w-J_t(w)}{r}=(w-\tau_t)(1-\overline{\tau_t}w)p_t(w)$. \\
%Now we have $\tau_t = \tau$ for all $t\geq0$. {\color{blue}why? We have $J_t(\tau)=\tau$.}\\
%Thus \[p_t(w) = \frac{1}{r}\frac{w-J_t(w)}{(w-\tau_t)(1-\overline{\tau_t}w)}.\]






















%\subsection{Semigroups of automorphisms %(Sugawa)
%}\label{sec_aut}
%
%In this section, we consider continuous semigroups consisting of holomorphic automorphisms
%of the upper half-plane $\uhp.$
%First of all, we note the following, perhaps well-known, result.
%
%\begin{lemma}\label{lem:auto}
%Let $(F_t)_{t\ge0}$ be a continuous semigroup of holomorphic automorphisms of $\uhp.$
%If it contains a non-identity element, then it is M\"obius conjugate to one of the following
%three standard semigroups:
%\begin{enumerate}
%\item[(i)] (hyperbolic case) $z\mapsto e^{\lambda t}z$ on $\uhp$ for a positive constant $\lambda$
%with generator $G_0(z)=\lambda z,$
%\item[(ii)] (parabolic case) $z\mapsto z+\lambda t$ on $\uhp$ for a real constant $\lambda\ne0$ with
%generator $G_0(z)=\lambda,$
%\item[(iii)] (elliptic case) $z\mapsto e^{i\lambda t}z$ on $\D$ for a real constant $\lambda\ne0$
%with generator $G_0(z)=i\lambda z.$
%\end{enumerate}
%\end{lemma}
%
%\begin{proof}
%First note that $F_t$ is continuous as elements in $\PSL(2,\R)$ with respect to $t\ge0.$
%Suppose that $F_{t_0}$ is a non-identity element.
%Then $F_{t_0}\in\PSL(2,\R)$ is either hyperbolic, parabolic or elliptic,
%and it has exactly two fixed points on $\partial\uhp,$ one fixed point on $\partial\uhp$
%or one fixed point in $\uhp,$ respectively.
%Let $\zeta_0$ be such a fixed point.
%Then by the semigroup property,
%$$
%F_{t_0}(F_t(\zeta_0))=F_t(F_{t_0}(\zeta_0))=F_t(\zeta_0),
%$$
%which implies that $F_t(\zeta_0)$ is a fixed point of $F_{t_0}.$
%By the continuity of $F_t(\zeta_0)$ in $0\le t<+\infty,$
%we conclude that $F_t(\zeta_0)=\zeta_0$ for $t\ge0.$
%
%When $F_{t_0}$ is hyperbolic, it has the attracting and repelling fixed points,
%say $\zeta_0$ and $\zeta_1$ respectively.
%Take a M\"obius map $C$ in $\PSL(2,\R)$ so that $C(\zeta_0)=\infty$ and $C(\zeta_1)=0.$
%Then $\hat F_t=C\circ F_t\circ C^{-1}$ has the form $z\mapsto e^{\lambda t}z$
%for a constant $\lambda>0.$
%In this case, we see that $F_t(z)\to \zeta_0$ locally uniformly on $\uhp$ as $t\to+\infty.$
%Hence, the point $\zeta_0$ is nothing but the Denjoy-Wolff point of the semigroup.
%
%When $F_{t_0}$ is parabolic, it has a unique fixed point $\zeta_0$ on $\partial\uhp.$
%Take a M\"obius map $C$ in $\PSL(2,\R)$ so that $C(\zeta_0)=\infty.$
%Then $\hat F_t=C\circ F_t\circ C^{-1}$ has the form $z\mapsto z+\lambda$
%for some $\lambda\in\R$ with $\lambda\ne0.$
%In this case, also $\zeta_0$ is the Denjoy-Wolff point.
%
%When $F_{t_0}$ is elliptic, it has a unique fixed point $\zeta_0$ in $\uhp.$
%Then $C(\zeta)=(\zeta-\zeta_0)/(\zeta-\bar\zeta_0)$ maps $\uhp$ onto $\D$
%in such a way that $C(\zeta_0)=0$ and $C(\bar\zeta_0)=\infty.$
%Then $\hat F_t=C\circ F_t\circ C^{-1}$ has the form $z\mapsto e^{i \lambda t}z$
%for a real constant $\lambda\ne0.$
%Note also that $\zeta_0$ is the Denjoy-Wolff point.
%\end{proof}
%
%We remark that we can further normalize $\lambda=\pm 1$ in the parabolic case.
%However, we cannot normalize so that $\lambda=1$ by a conjugation of an element in $\PSL(2,\R).$
%
%
%
%\subsubsection{Hyperbolic Case}
%
%\begin{theorem}
%Let $(F_t)_{t\ge0}$ be a continuous semigroup of hyperbolic automorphisms of $\uhp$
%with infinitesimal generator $G.$
%Then its resolvent $J_t$ exists for all $t\ge0$ if and only if the Denjoy-Wolff
%point of the semigroup is finite.
%\end{theorem}
%
%\begin{proof}
%By the above lemma, there exist a M\"obius map $\zeta=C(z)=(az+b)/(cz+d)$ in $\PSL(2,\R)$
%(with $ad-bc=1$) and a constant $\lambda>0$ such that $C\circ F_t\circ C^{-1}(z)=e^{\lambda t}z$ for $t\ge0.$
%Then, the generator $G$ of $(F_t)$ is given by
%$$
%G(\zeta)=G_0(z)\frac{d\zeta}{dz}=\frac{\lambda z}{(cz+d)^2}.
%$$
%Therefore, the resolvent $\zeta=J_t(w)$ is described by
%$$
%w=\zeta-tG(\zeta)=\frac{az+b}{cz+d}-\frac{\lambda tz}{(cz+d)^2}=:W_t(z).
%$$
%Since
%$$
%W_t'(z)=\frac{c(1+2t)z-d(1-2t)}{(cz+d)^3},
%$$
%the critical points of $W_t$ lie on the extended real line $\widehat\R=\partial\uhp.$
%Obviously, the function $W_t$ maps $\widehat\R$ into itself.
%By the argument principle, we see that the condition $W_t(\uhp)\cap\uhp\ne\emptyset$
%implies $\uhp\subset W_t(\uhp).$
%If this occurs, we can take an analytic branch of $W_t^{-1}$ and thus
%the resolvent $J_t$ exists and it is given by $J_t(w)=C(W_t^{-1}(w)).$
%
%We next examine the condition $W_t(\uhp)\cap\uhp\ne\emptyset$ for a fixed $t>0.$
%For $z=x+iy$ with $y>0,$ we compute
%\begin{align*}
%\Im W_t(z)&=\frac{y}{|cz+d|^2}-\frac{\lambda t \Im \big[z(c\bar z+d)^2\big]}{|cz+d|^4} \\
%&=\frac{y}{|cz+d|^4}\big[c^2|z|^2+2cdx+d^2-\lambda t(-c^2|z|^2+d^2)\big] \\
%&=\frac{y}{|cz+d|^4}\big[(1+\lambda t)c^2|z|^2+2cdx+(1-\lambda t)d^2\big].
%\end{align*}
%When $c\ne0,$ for a large enough $y>0$ we have $\Im W_t(iy)>0,$ and hence,
%$\uhp\subset W_t(\uhp).$
%When $c=0,$ we see that $\Im W_t(z)<0~(z\in\uhp)$ for $t>1/\lambda.$
%Therefore, the resolvent $J_t$ does not exist for $t>1/\lambda$ in this case.
%The condition $c=0$ is equivalent to saying that the attracting fixed point
%(= the Denjoy-Wolff point) is $\infty.$
%Now the proof is complete.
%\end{proof}
%
%\subsubsection{Parabolic Case}
%
%\begin{theorem}
%Let $(F_t)_{t\ge0}$ be a continuous semigroup of parabolic automorphisms of $\uhp$
%with infinitesimal generator $G.$
%Then its resolvent $J_t$ exists for all $t\ge0.$
%\end{theorem}
%
%\begin{proof}
%By Lemma \ref{lem:auto}, there exist a M\"obius map $\zeta=C(z)=(az+b)/(cz+d)$ in $\PSL(2,\R)$
%(with $ad-bc=1$) and a real constant $\lambda\ne0$ 
%such that $C\circ F_t\circ C^{-1}(z)=z+\lambda t$ for $t\ge0.$
%Then, the generator $G$ of $(F_t)$ is given by
%$$
%G(\zeta)=G_0(z)\frac{d\zeta}{dz}=\frac{\lambda}{(cz+d)^2}.
%$$
%Therefore, the resolvent $\zeta=J_t(w)$ is described by
%$$
%w=\zeta-tG(\zeta)=\frac{az+b}{cz+d}-\frac{\lambda t}{(cz+d)^2}=:W_t(z).
%$$
%Since
%$$
%W_t'(z)=\frac{cz+d+2c\lambda t}{(cz+d)^3},
%$$
%the critical points of $W_t$ lie on $\widehat\R.$
%Exactly by the same reason as before, it is enough to check
%the condition $W_t(\uhp)\cap\uhp\ne\emptyset$ for a fixed $t>0.$
%For $z=x+iy$ with $y>0,$ we compute
%\begin{align*}
%\Im W_t(z)&=\frac{y}{|cz+d|^2}-\frac{\lambda t \Im \big[(c\bar z+d)^2\big]}{|cz+d|^4} \\
%&=\frac{y}{|cz+d|^4}\big[c^2|z|^2+2cdx+d^2-\lambda t(2c^2x+2cd)\big] \\
%&=\frac{y}{|cz+d|^4}\big[c^2|z|^2+2c(d-c\lambda t)x+d(d-2c\lambda t)\big].
%\end{align*}
%When $c\ne0,$ for a large enough $y>0$ we have $\Im W_t(iy)>0.$
%When $c=0,$ we have $\Im W_t(z)=y/d^2>0.$
%At any event, we obtain $\uhp\subset W_t(\uhp).$
%Therefore, the resolvent $J_t$ always exists in the parabolic case.
%Now the proof is complete.
%\end{proof}
%
%\subsubsection{Elliptic Case}
%
%Finally, we consider the elliptic case.
%We also have a positive result in this case.
%
%\begin{theorem}
%Let $(F_t)_{t\ge0}$ be a continuous semigroup of elliptic automorphisms of $\uhp$
%with infinitesimal generator $G.$
%Then its resolvent $J_t$ exists for all $t\ge0.$
%\end{theorem}
%
%\begin{proof}
%By Lemma \ref{lem:auto} and its proof, there is a point $\zeta_0=\xi_0+i\eta_0\in\uhp$
%and a real constant $\lambda\ne0$ such that the transformation 
%$z=C^{-1}(\zeta)=(\zeta-\zeta_0)/(\zeta-\bar\zeta_0)$ maps $\uhp$ onto $\D$
%with $C\circ F_t\circ C^{-1}(z)=e^{i\lambda t}z$ for $t\ge0.$
%Note that $C(z)=(\zeta_0-\bar\zeta_0 z)/(1-z).$
%Then, the generator $G$ of $(F_t)$ is given by
%$$
%G(\zeta)=G_0(z)\frac{d\zeta}{dz}=\frac{-2\eta_0\lambda z}{(1-z)^2}.
%$$
%Therefore, the resolvent $\zeta=J_t(w)$ is described by
%$$
%w=\zeta-tG(\zeta)=\frac{\zeta_0-\bar\zeta_0 z}{1-z}
%+\frac{2\eta_0 \lambda t z}{(1-z)^2}=:W_t(z).
%$$
%In view of the form
%$$
%W_t'(z)=\frac{2i\eta_0}{(1-z)^2}-\frac{4\eta_0\lambda t(1+z)}{(1-z)^3}
%=2i\eta_0\cdot\frac{1+2\lambda ti-(1-2\lambda ti)z}{(1-z)^3},
%$$
%we see that the critical points of $W_t$ lie on the unit circle $|z|=1.$
%In particular, $W_t$ is locally univalent on $\D.$
%Since the function $\zeta-tG(\zeta)$ maps $\widehat\R$ into itself,
%% $W_t$ maps $\partial\D$ into $\widehat\R.$
%Therefore, we observe that $W_t(\D)\cap\uhp\ne\emptyset$ implies
%$\uhp\subset W_t(\D)$ by the same reasoning as before.
%Hence, it is enough to check
%the condition $W_t(\D)\cap\uhp\ne\emptyset$ for a fixed $t>0.$
%Indeed, in view of $W_t(0)=\zeta_0\in\uhp,$ we verify it easily.
%Therefore, the resolvent $J_t$ always exists in the elliptic case.
%Now the proof is complete.
%\end{proof}


%For $z=re^{i\theta}\in\D,$ we compute
%\begin{align*}
%\Im W_t(z)
%&=\frac{\Im\big[(\zeta_0-\bar\zeta_0z)(1-\bar z)\big]}{|1-z|^2}
%+2\eta_0\lambda t\frac{\Im \big[z(1-\bar z)^2\big]}{|1-z|^4} \\
%&=\frac{\eta_0(1-r^2)}{|1-z|^2}
%+2\eta_0\lambda t\frac{(1-r^2)r\sin\theta}{|1-z|^4} \\
%&=\eta_0(1-r^2)\frac{|1-z|^2-2\lambda tr\sin\theta}{|1-z|^4}.
%\end{align*}





%\section{Upper half-plane III: semigroups of automorphisms}
%
%
%%\subsection{Classification of semigroups of automorphisms on $\uhp$}
%
%In the last section, we focus on continuous semigroups consisting of holomorphic automorphisms
%of the upper half-plane $\uhp.$
%
%
%\subsection{Lemma}
%
%First of all, we note the following, perhaps well-known, result.
%
%\begin{lemma}\label{lem:auto}
%Let $(F_t)_{t\ge0}$ be a continuous semigroup of holomorphic automorphisms of $\uhp.$
%If it contains a non-identity element, then it is M\"obius conjugate to one of the following
%three standard semigroups:
%\begin{enumerate}
%\item[(i)] (hyperbolic case) $z\mapsto e^{\lambda t}z$ on $\uhp$ for a positive constant $\lambda$
%with generator $G_0(z)=\lambda z,$
%\item[(ii)] (parabolic case) $z\mapsto z+\lambda t$ on $\uhp$ for a real constant $\lambda\ne0$ with
%generator $G_0(z)=\lambda,$
%\item[(iii)] (elliptic case) $z\mapsto e^{i\lambda t}z$ on $\D$ for a real constant $\lambda\ne0$
%with generator $G_0(z)=i\lambda z.$
%\end{enumerate}
%\end{lemma}
%
%\begin{proof}
%First note that $F_t$ is continuous as elements in $\PSL(2,\R)$ with respect to $t\ge0.$
%Suppose that $F_{t_0}$ is a non-identity element.
%Then $F_{t_0}\in\PSL(2,\R)$ is either hyperbolic, parabolic or elliptic,
%and it has exactly two fixed points on $\partial\uhp,$ one fixed point on $\partial\uhp$
%or one fixed point in $\uhp,$ respectively.
%Let $\zeta_0$ be such a fixed point.
%Then by the semigroup property,
%$$
%F_{t_0}(F_t(\zeta_0))=F_t(F_{t_0}(\zeta_0))=F_t(\zeta_0),
%$$
%which implies that $F_t(\zeta_0)$ is a fixed point of $F_{t_0}.$
%By the continuity of $F_t(\zeta_0)$ in $0\le t<+\infty,$
%we conclude that $F_t(\zeta_0)=\zeta_0$ for $t\ge0.$
%
%When $F_{t_0}$ is hyperbolic, it has the attracting and repelling fixed points,
%say $\zeta_0$ and $\zeta_1$ respectively.
%Take a M\"obius map $C$ in $\PSL(2,\R)$ so that $C(\zeta_0)=\infty$ and $C(\zeta_1)=0.$
%Then $\hat F_t=C\circ F_t\circ C^{-1}$ has the form $z\mapsto e^{\lambda t}z$
%for a constant $\lambda>0.$
%In this case, we see that $F_t(z)\to \zeta_0$ locally uniformly on $\uhp$ as $t\to+\infty.$
%Hence, the point $\zeta_0$ is nothing but the Denjoy-Wolff point of the semigroup.
%
%When $F_{t_0}$ is parabolic, it has a unique fixed point $\zeta_0$ on $\partial\uhp.$
%Take a M\"obius map $C$ in $\PSL(2,\R)$ so that $C(\zeta_0)=\infty.$
%Then $\hat F_t=C\circ F_t\circ C^{-1}$ has the form $z\mapsto z+\lambda$
%for some $\lambda\in\R$ with $\lambda\ne0.$
%In this case, also $\zeta_0$ is the Denjoy-Wolff point.
%
%When $F_{t_0}$ is elliptic, it has a unique fixed point $\zeta_0$ in $\uhp.$
%Then $C(\zeta)=(\zeta-\zeta_0)/(\zeta-\bar\zeta_0)$ maps $\uhp$ onto $\D$
%in such a way that $C(\zeta_0)=0$ and $C(\bar\zeta_0)=\infty.$
%Then $\hat F_t=C\circ F_t\circ C^{-1}$ has the form $z\mapsto e^{i \lambda t}z$
%for a real constant $\lambda\ne0.$
%Note also that $\zeta_0$ is the Denjoy-Wolff point.
%\end{proof}
%
%We remark that we can further normalize $\lambda=\pm 1$ in the parabolic case.
%However, we cannot normalize so that $\lambda=1$ by a conjugation of an element in $\PSL(2,\R).$
%
%
%
%\subsection{Hyperbolic Case}
%
%
%
%\begin{theorem}
%Let $(F_t)_{t\ge0}$ be a continuous semigroup of hyperbolic automorphisms of $\uhp$
%with infinitesimal generator $G.$
%Then its resolvent $J_t$ exists for all $t\ge0$ if and only if the Denjoy-Wolff
%point of the semigroup is finite.
%\end{theorem}
%
%\begin{proof}
%By the above lemma, there exist a M\"obius map $\zeta=C(z)=(az+b)/(cz+d)$ in $\PSL(2,\R)$
%(with $ad-bc=1$) and a constant $\lambda>0$ such that $C\circ F_t\circ C^{-1}(z)=e^{\lambda t}z$ for $t\ge0.$
%Then, the generator $G$ of $(F_t)$ is given by
%$$
%G(\zeta)=G_0(z)\frac{d\zeta}{dz}=\frac{\lambda z}{(cz+d)^2}.
%$$
%Therefore, the resolvent $\zeta=J_t(w)$ is described by
%$$
%w=\zeta-tG(\zeta)=\frac{az+b}{cz+d}-\frac{\lambda tz}{(cz+d)^2}=:W_t(z).
%$$
%Since
%$$
%W_t'(z)=\frac{c(1+2t)z-d(1-2t)}{(cz+d)^3},
%$$
%the critical points of $W_t$ lie on the extended real line $\widehat\R=\partial\uhp.$
%Obviously, the function $W_t$ maps $\widehat\R$ into itself.
%By the argument principle, we see that the condition $W_t(\uhp)\cap\uhp\ne\emptyset$
%implies $\uhp\subset W_t(\uhp).$
%If this occurs, we can take an analytic branch of $W_t^{-1}$ and thus
%the resolvent $J_t$ exists and it is given by $J_t(w)=C(W_t^{-1}(w)).$
%
%We next examine the condition $W_t(\uhp)\cap\uhp\ne\emptyset$ for a fixed $t>0.$
%For $z=x+iy$ with $y>0,$ we compute
%\begin{align*}
%\Im W_t(z)&=\frac{y}{|cz+d|^2}-\frac{\lambda t \Im \big[z(c\bar z+d)^2\big]}{|cz+d|^4} \\
%&=\frac{y}{|cz+d|^4}\big[c^2|z|^2+2cdx+d^2-\lambda t(-c^2|z|^2+d^2)\big] \\
%&=\frac{y}{|cz+d|^4}\big[(1+\lambda t)c^2|z|^2+2cdx+(1-\lambda t)d^2\big].
%\end{align*}
%When $c\ne0,$ for a large enough $y>0$ we have $\Im W_t(iy)>0,$ and hence,
%$\uhp\subset W_t(\uhp).$
%When $c=0,$ we see that $\Im W_t(z)<0~(z\in\uhp)$ for $t>1/\lambda.$
%Therefore, the resolvent $J_t$ does not exist for $t>1/\lambda$ in this case.
%The condition $c=0$ is equivalent to saying that the attracting fixed point
%(namely the Denjoy-Wolff point) is $\infty.$
%Now the proof is complete.
%\end{proof}
%
%\subsection{Parabolic Case}
%
%In the parabolic case, we obtain the following theorem.
%
%\begin{theorem}
%Let $(F_t)_{t\ge0}$ be a continuous semigroup of parabolic automorphisms of $\uhp$
%with infinitesimal generator $G.$
%Then its resolvent $J_t$ exists for all $t\ge0.$
%\end{theorem}
%
%\begin{proof}
%By Lemma \ref{lem:auto}, there exist a M\"obius map $\zeta=C(z)=(az+b)/(cz+d)$ in $\PSL(2,\R)$
%(with $ad-bc=1$) and a real constant $\lambda\ne0$ 
%such that $C\circ F_t\circ C^{-1}(z)=z+\lambda t$ for $t\ge0.$
%Then, the generator $G$ of $(F_t)$ is given by
%$$
%G(\zeta)=G_0(z)\frac{d\zeta}{dz}=\frac{\lambda}{(cz+d)^2}.
%$$
%Therefore, the resolvent $\zeta=J_t(w)$ is described by
%$$
%w=\zeta-tG(\zeta)=\frac{az+b}{cz+d}-\frac{\lambda t}{(cz+d)^2}=:W_t(z).
%$$
%Since
%$$
%W_t'(z)=\frac{cz+d+2c\lambda t}{(cz+d)^3},
%$$
%the critical points of $W_t$ lie on $\widehat\R.$
%Exactly by the same reason as before, it is enough to check
%the condition $W_t(\uhp)\cap\uhp\ne\emptyset$ for a fixed $t>0.$
%For $z=x+iy$ with $y>0,$ we compute
%\begin{align*}
%\Im W_t(z)&=\frac{y}{|cz+d|^2}-\frac{\lambda t \Im \big[(c\bar z+d)^2\big]}{|cz+d|^4} \\
%&=\frac{y}{|cz+d|^4}\big[c^2|z|^2+2cdx+d^2-\lambda t(2c^2x+2cd)\big] \\
%&=\frac{y}{|cz+d|^4}\big[c^2|z|^2+2c(d-c\lambda t)x+d(d-2c\lambda t)\big].
%\end{align*}
%When $c\ne0,$ for a large enough $y>0$ we have $\Im W_t(iy)>0.$
%When $c=0,$ we have $\Im W_t(z)=y/d^2>0.$
%At any event, we obtain $\uhp\subset W_t(\uhp).$
%Therefore, the resolvent $J_t$ always exists in the parabolic case.
%Now the proof is complete.
%\end{proof}
%
%\subsection{Elliptic Case}
%
%%Finally, we consider the elliptic case.
%
%
%\begin{theorem}
%Let $(F_t)_{t\ge0}$ be a continuous semigroup of elliptic automorphisms of $\uhp$
%with infinitesimal generator $G.$
%Then its resolvent $J_t$ exists for all $t\ge0.$
%\end{theorem}
%
%\begin{proof}
%This is immediately obtained from Theorem \ref{finite-point-theorem}.
%We, however, give a direct proof of it for readers.
%
%By Lemma \ref{lem:auto} and its proof, there is a point $\zeta_0=\xi_0+i\eta_0\in\uhp$
%and a real constant $\lambda\ne0$ such that the transformation 
%$z=C^{-1}(\zeta)=(\zeta-\zeta_0)/(\zeta-\bar\zeta_0)$ maps $\uhp$ onto $\D$
%with $C\circ F_t\circ C^{-1}(z)=e^{i\lambda t}z$ for $t\ge0.$
%Note that $C(z)=(\zeta_0-\bar\zeta_0 z)/(1-z).$
%Then, the generator $G$ of $(F_t)$ is given by
%$$
%G(\zeta)=G_0(z)\frac{d\zeta}{dz}=\frac{-2\eta_0\lambda z}{(1-z)^2}.
%$$
%Therefore, the resolvent $\zeta=J_t(w)$ is described by
%$$
%w=\zeta-tG(\zeta)=\frac{\zeta_0-\bar\zeta_0 z}{1-z}
%+\frac{2\eta_0 \lambda t z}{(1-z)^2}=:W_t(z).
%$$
%In view of the form
%$$
%W_t'(z)=\frac{2i\eta_0}{(1-z)^2}-\frac{4\eta_0\lambda t(1+z)}{(1-z)^3}
%=2i\eta_0\cdot\frac{1+2\lambda ti-(1-2\lambda ti)z}{(1-z)^3},
%$$
%we see that the critical points of $W_t$ lie on the unit circle $|z|=1.$
%In particular, $W_t$ is locally univalent on $\D.$
%Since the function $\zeta-tG(\zeta)$ maps $\widehat\R$ into itself,
%% $W_t$ maps $\partial\D$ into $\widehat\R.$
%Therefore, we observe that $W_t(\D)\cap\uhp\ne\emptyset$ implies
%$\uhp\subset W_t(\D)$ by the same reasoning as before.
%Hence, it is enough to check
%the condition $W_t(\D)\cap\uhp\ne\emptyset$ for a fixed $t>0.$
%Indeed, in view of $W_t(0)=\zeta_0\in\uhp,$ we verify it easily.
%Therefore, the resolvent $J_t$ always exists in the elliptic case.
%Now the proof is complete.
%\end{proof}




 

\begin{thebibliography}{CDMG14}
\providecommand{\bysame}{\leavevmode\hbox to3em{\hrulefill}\thinspace}


\bibitem[Aba92]{Aba92} M.\ Abate, \textit{The Infinitesimal Generators of Semigroups of Holomorphic Maps}, 
Annali di Matematica pura ed applicata (IV), Vol. CLXI (1992), 167--180. 

\bibitem[AB11]{MR2887104}
L.\ Arosio, F.\ Bracci, \textit{Infinitesimal generators and the
  {L}oewner equation on complete hyperbolic manifolds}, Anal. Math. Phys.
  1 (2011), no.~4, 337--350.

\bibitem[BV92]{BV92} H.\ Bercovici, D.\ Voiculescu, 
\textit{L\'{e}vy-Hin\v{c}in type theorems for multiplicative and additive free convolution}, 
Pacific J.\ Math. 153 (1992), no.\ 2, 217--248.	

 \bibitem[BV93]{BV93} \bysame, \textit{Free convolution of measures with unbounded support}, Indiana Univ. Math. J. 42 (1993), no. 3, 733--773.

\bibitem[BP78]{BP78} E.\ Berkson, H.\ Porta, \textit{Semigroups of analytic functions and composition operators}, Michigan Math. J. 25 (1978), no.~1, 101--115.

\bibitem[BS09]{BS09} F.\ Bracci, A.\ Saracco,  \textit{Hyperbolicity in unbounded convex domains}, Forum Math. 21 (2009), 815--825.

\bibitem[BCDM09]{MR2507634}
F.\ Bracci, M. D.\ Contreras, S.\ D{\'{\i}}az-Madrigal,
  \textit{Evolution families and the {L}oewner equation. {II}. {C}omplex
  hyperbolic manifolds}, Math. Ann. 344 (2009), no.~4, 947--962.
  
\bibitem[BCDM20]{BCDM20}
\bysame,
  \textit{Continuous semigroups of holomorphic self-maps of the unit disc}, 
 Springer Monographs in Mathematics. Springer, Cham, 2020. xxvii+566 pp.  
	
\bibitem[CDM13]{CDM13}  T.\ Casavecchia, S.\ D\'{i}az-Madrigal, \textit{A Non-Autonomous Version of the Denjoy-Wolff Theorem}, Complex Analysis and Operator Theory 7 (2013), 1457--1479.

\bibitem[Cau32]{cau32} W.\ Cauer, \textit{The Poisson integral for functions with positive real part}, Bull. Amer. Math. Soc. 38 (1932), 
713--717. 

\bibitem[CDMP06]{CDMP06} M. D.\ Contreras, S.\ D\'\i az-Madrigal, C.\ Pommerenke, \textit{On  boundary  critical  points  for  semigroups of analytic functions}, Math. Scand. 98 (2006), 125--142.


\bibitem[CDMG10]{CDMG10}
M. D.\ Contreras, S.\ D\'\i az-Madrigal, P.\ Gumenyuk, \textit{Loewner chains in the unit disk}, Rev. Mat. Iberoam. 26 (2010), no.~3, 975--1012.

\bibitem[CDMG14]{CDMG14}
\bysame, \textit{Local duality in Loewner equations}, J. Nonlinear Convex Anal. 15 (2014), no. 2, 269--297.

\bibitem[dMHS18]{dMHS18} A.\ del Monaco, I.\ Hotta, S.\ Schlei{\ss}inger, \textit{Tightness results for infinite-slit limits of the chordal Loewner equation}, Comput. Methods Funct. Theory 18 (2018), no. 1, 9--33. 

\bibitem[dMS16]{dMS16}
A. del Monaco, S. Schlei{\ss}inger,
\textit{Multiple SLE and the complex Burgers equation},
Math. Nachr. 289 (2016), 2007--2018.

\bibitem[ESS20]{ESS20} M.\ Elin, D.\ Shoikhet, T.\ Sugawa, \textit{Geometric properties of the nonlinear resolvent of holomorphic generators}, J. Math. Anal. Appl. 483 (2020).
 
\bibitem[FHS20]{FHS} U.\ Franz, T.\ Hasebe, S.\ Schlei{\ss}inger,  \textit{Monotone Increment Processes, Classical Markov Processes, and Loewner Chains}, Dissertationes Math. 552 (2020), 119 pp.  

\bibitem[GKH20]{GKH} I. Graham, H. Hamada, G. Kohr, \textit{Loewner chains and nonlinear resolvents of the Carath{\'e}odory family on the unit ball in $\mathbb{C}^n$}. J. Math. Anal. Appl. 491 (2020), no. 1, 124289, 29 pp.

\bibitem[HS20]{HS} I.\ Hotta, S.\ Schlei{\ss}inger, \textit{Limits of radial multiple SLE and a Burgers-Loewner differential equation}, J. Theoret. Probab. 34 (2021), no. 2, 755--783.

\bibitem[Jek20]{Jek17}D.\ Jekel, \textit{Operator-valued chordal Loewner chains and non-commutative probability}, J. Funct. Anal. 278 (2020), no. 10, 108452, 100 pp.

\bibitem[Koe87]{K87} W. Koepf, \textit{Convex functions and the Nehari univalence criterion}, in: Laine, I., Sorvali, T., Rickman, S., Complex Analysis Joensuu 1987, Lecture Notes in Mathematics, vol 1351, Springer, Berlin, Heidelberg, 1987.

\bibitem[Maa92]{M92} H. Maassen, \textit{Addition of freely independent random variables}, 
J. Funct. Anal. 106 (1992), no. 2, 409--438.

\bibitem[MS17]{MS17}
J. A.\ Mingo, R.\ Speicher, \textit{Free Probability and Random Matrices}, Springer, Berlin, 2017. 

\bibitem[NS06]{ns06}  A.\ Nica, R.\ Speicher, \textit{Lectures on the Combinatorics of Free
Probability}, London Mathematical Society, Lecture Note Series 335, Cambridge
University Press, 2006.

\bibitem[Pom92]{Pom92}
Ch. Pommerenke, \textit{Boundary behaviour of conformal maps}, Grundlehren der
  Mathematischen Wissenschaften, vol. 299, Springer-Verlag, Berlin, 1992.

\bibitem[RS96]{RS96} S.\ Reich, D.\ Shoikhet, \textit{Generation theory for semigroups of holomorphic mappings in Banach spaces}, Abstr. Appl. Anal. 1 (1996), no. 1, 1--44. 

\bibitem[RS97]{RS97}  S.\ Reich, D.\ Shoikhet, \textit{Semigroups and generators on convex domains with the hyperbolic metric}, 
Atti della Accademia Nazionale dei Lincei. Classe di Scienze Fisiche, Matematiche e Naturali. Rendiconti Lincei. Matematica e Applicazioni, Serie 9, Vol. 8 (1997), no. 4, p. 231--250.

\bibitem[Rob36]{R36} M.\ S.\ Robertson, \textit{On the theory of univalent functions}, Ann. Math. 37 (1936), no. 2, 374-408.

\bibitem[Sha93]{shapiro} J. H.\ Shapiro, \textit{Composition Operators and Classical Function Theory}, Springer, 1993.

\bibitem[Stu11]{S11} E.\ Study, \textit{Konforme Abbildung einfach-zusammenh\"angender Bereiche}, Volume 2 of Vorlesungen \"uber ausgew\"ahlte Gegenst\"ande der Geometrie,  Leipzig, Berlin, B.G. Teubner, 1911-13.
%\bibitem[Sho01]{Shoikhet:2001}
%D.~Shoikhet, \textit{Semigroups in geometrical function theory}, Kluwer Academic Publishers, 2001.

\bibitem[Voi87]{Voi87} D.\ Voiculescu, \textit{Multiplication of certain noncommuting
random variables}, J.\ Operator Theory 18 (1987), 223--235.

\bibitem[Voi97]{Voi97} \bysame, \textit{Free Probability Theory}, Fields Inst. Commun. 12, Amer. Math. Soc., 1997.

\bibitem[ZZ18]{ZZ18} H.\ Zhang, M.\ Zinsmeister, \textit{Local Analysis of Loewner Equation}, arXiv:1804.03410.
\end{thebibliography}

\end{document}
