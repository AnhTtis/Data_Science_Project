% CVPR 2023 Paper Template
% based on the CVPR template provided by Ming-Ming Cheng (https://github.com/MCG-NKU/CVPR_Template)
% modified and extended by Stefan Roth (stefan.roth@NOSPAMtu-darmstadt.de)

\documentclass[10pt,twocolumn,letterpaper]{article}

%%%%%%%%% PAPER TYPE  - PLEASE UPDATE FOR FINAL VERSION
% \usepackage[review]{cvpr}      % To produce the REVIEW version
\usepackage{cvpr}              % To produce the CAMERA-READY version
%\usepackage[pagenumbers]{cvpr} % To force page numbers, e.g. for an arXiv version

% Include other packages here, before hyperref.
\usepackage{graphicx}
\usepackage{amsmath}
\usepackage{amssymb}
\usepackage{booktabs}
\usepackage{multirow}

% It is strongly recommended to use hyperref, especially for the review version.
% hyperref with option pagebackref eases the reviewers' job.
% Please disable hyperref *only* if you encounter grave issues, e.g. with the
% file validation for the camera-ready version.
%
% If you comment hyperref and then uncomment it, you should delete
% ReviewTempalte.aux before re-running LaTeX.
% (Or just hit 'q' on the first LaTeX run, let it finish, and you
%  should be clear).
\usepackage[pagebackref,breaklinks,colorlinks]{hyperref}


% Support for easy cross-referencing
\usepackage[capitalize]{cleveref}
\crefname{section}{Sec.}{Secs.}
\Crefname{section}{Section}{Sections}
\Crefname{table}{Table}{Tables}
\crefname{table}{Tab.}{Tabs.}
\newcommand{\weidi}[1]{{\textcolor{magenta}{Weidi: #1} }}


%%%%%%%%% PAPER ID  - PLEASE UPDATE
\def\cvprPaperID{7851} % *** Enter the CVPR Paper ID here
\def\confName{CVPR}
\def\confYear{2023}

\newcommand{\highlightChange}{\color{red}}
\def\HC{\highlightChange}
\newcommand{\Note}[1]{{\color{blue} \bf \small [NOTE: #1]}}


\begin{document}

%%%%%%%%% TITLE - PLEASE UPDATE
% \title{Cameras Beating LiDAR 3D detection with Collaboration}
% \title{Collaboration Is What You Need for Camera to Overtake LiDAR in 3D Detection}
\title{Collaboration Helps Camera Overtake LiDAR in 3D Detection}
% \author{First Author\\
% Institution1\\
% Institution1 address\\
% {\tt\small firstauthor@i1.org}
% % For a paper whose authors are all at the same institution,
% % omit the following lines up until the closing ``}''.
% % Additional authors and addresses can be added with ``\and'',
% % just like the second author.
% % To save space, use either the email address or home page, not both
% \and
% Second Author\\
% Institution2\\
% First line of institution2 address\\
% {\tt\small secondauthor@i2.org}
% }
\author{Yue~Hu$^1$ \quad Yifan~Lu$^{1}$ \quad Runsheng~Xu$^2$ \quad Weidi~Xie$^{1,3}$ \quad Siheng Chen$^{1,3}$\footnotemark[1] \quad Yanfeng Wang$^{3,1}$ \\
$^{1}${Cooperative Medianet Innovation Center, Shanghai Jiao Tong University} \\ \quad $^{2}${University of California, Los Angeles} \quad $^{3}${Shanghai AI Laboratory} \\ 
$^{1}${\tt\small \{18671129361, yifan\_lu, weidi, sihengc, wangyanfeng\}@sjtu.edu.cn}  \quad 
$^{2}${\tt\small \{rxx3386\}@ucla.edu} \\
}

% \author{%
%   Yue~Hu^1, Yifan Lu^1, Runsheng Xu^2, {Weidi Xie}^1, {Siheng ~Chen}^1\thanks{Corresponding author}, {Yanfeng Wang}^1 \\
%   ^{1} Cooperative Medianet Innovation Center, Shanghai Jiao Tong University, Shanghai AI Laboratory \\
%    ^{2} University of California, Los Angeles \\
%    \texttt{18671129361, yifan\_lu,weidi, sihengc, wangyanfeng}@sjtu.edu.cn \\
%   \texttt{rxx3386}@ucla.edu \\
% }

% \author{%
%   Yue~Hu \hspace{1.75cm} Yifan Lu \\
%   Cooperative Medianet Innovation Center, Shanghai Jiao Tong University \\
%   \texttt{\{18671129361, yifan\_lu}@sjtu.edu.cn \\
%   Runsheng Xu University of California, Los Angeles 
%   \texttt{rxx3386@ucla.edu} \\
%   \and 
%    Weidi Xie, Siheng ~Chen\thanks{Corresponding author}, Yanfeng Wang\\
%   Shanghai Jiao Tong University,  Shanghai AI Laboratory \\
%   \texttt{weidi, sihengc, wangyanfeng}@sjtu.edu.cn
% }
\maketitle
\renewcommand{\thefootnote}{\fnsymbol{footnote}}
\footnotetext[1]{Corresponding author.}

\begin{abstract}
As models continue to grow in size, the development of memory optimization methods (MOMs) has emerged as a solution to address the memory bottleneck encountered when training large models. To comprehensively examine the practical value of various MOMs, we have conducted a thorough analysis of existing literature from a systems perspective. 
% Furthermore, we have evaluated the most widely adopted MOMs employed in mainstream frameworks for both vision and language models.
Our analysis has revealed a notable challenge within the research community: the absence of standardized metrics for effectively evaluating the efficacy of MOMs. The scarcity of informative evaluation metrics hinders the ability of researchers and practitioners to compare and benchmark different approaches reliably. Consequently, drawing definitive conclusions and making informed decisions regarding the selection and application of MOMs becomes a challenging endeavor.
To address the challenge, this paper summarizes the scenarios in which MOMs prove advantageous for model training. We propose the use of distinct evaluation metrics under different scenarios. By employing these metrics, we evaluate the prevailing MOMs and find that their benefits are not universal. We present insights derived from experiments and discuss the circumstances in which they can be advantageous.

\end{abstract}
\section{Introduction}


Recent years have witnessed the rise of human digitization~\cite{habermannDeepCapMonocularHuman2020,alexanderCREATINGPHOTOREALDIGITAL,pengNeuralBodyImplicit2021,alldieckDetailedHumanAvatars2018, rajANRArticulatedNeural2020}. This technology greatly impacts the entertainment, education, design, and engineering industry.
There is a well-developed industry solution for this task.
High-fidelity reconstruction of humans can be achieved either with full-body laser scans~\cite{saitoSCANimateWeaklySupervised2021}, dense synchronized multi-view cameras~\cite{xiangModelingClothingSeparate2021a,xiangDressingAvatarsDeep2022a}, or light stages~\cite{alexanderCREATINGPHOTOREALDIGITAL}.
However, these settings are expensive and tedious to deploy and consist of a complex processing pipeline, preventing the technology's democratization.

Another solution is to view the problem as inverse rendering and learn digital humans directly from custom-collected data.
Traditional approaches directly optimize explicit mesh representation~\cite{loperSMPLSkinnedMultiperson2015, fangRMPERegionalMultiperson2018, pavlakosExpressiveBodyCapture2019} which suffers from the problems of smooth geometry and coarse textures~\cite{prokudinSMPLpixNeuralAvatars2020,alldieckVideoBasedReconstruction2018}. Besides, they require professional artists to design human templates, rigging, and unwrapped UV coordinates.
Recently, with the help of volumetric-based implicit representations~\cite{mildenhallNeRFRepresentingScenes2020, parkDeepSDFLearningContinuous2019, meschederOccupancyNetworksLearning2019} and neural rendering~\cite{laineModularPrimitivesHighPerformance2020, liuSoftRasterizerDifferentiable2019, thiesDeferredNeuralRendering2019}, 
one can easily digitize a quality-plausible human avatar from video footage~\cite{jiangNeuManNeuralHuman2022,wengHumanNeRFFreeviewpointRendering}.
Particularly, volumetric-based implicit representations~\cite{mildenhallNeRFRepresentingScenes2020, pengNeuralBodyImplicit2021} can reconstruct scenes or objects with much higher fidelity against previous neural renderer~\cite{thiesDeferredNeuralRendering2019,prokudinSMPLpixNeuralAvatars2020}, and is more user-friendly as it does not need any human templates, pre-set rigging, or UV coordinates.
Captured visual footage and corresponding skeleton tracking are enough for training.
However, better reconstructions and more friendly usability are at the expense of the following factors.
1) \textbf{Inefficiency:}
They require longer optimization times (typically tens of hours or days) and inference slowly.
Volume rendering~\cite{mildenhallNeRFRepresentingScenes2020,lombardiNeuralVolumesLearning2019} formulates images by querying the densities and colors of millions of spatial coordinates. 
In the training stage, due to memory constraints, only a small fraction of points are sampled which leads to slow convergence speed.
2) \textbf{Entangled representations}:
The geometry, materials, and motion dynamics are entangled in the neural networks. 
Due to the implicit nature of neural nets, one can hardly edit one property without touching the others~\cite{yuanNeRFEditingGeometryEditing2022a,liuEditingConditionalRadiance2021}.
3) \textbf{Graphics incompatibility}:
Volume rendering is incompatible with the current popular graphic pipeline,
which renders triangular/quadrilateral meshes efficiently with the rasterization technique.
Many downstream applications require mesh rasterization in their workflow (\eg, editing~\cite{foundationBlenderOrgHome}, simulation~\cite{benderPositionBasedSimulationMethods2015}, real-time rendering~\cite{akenine2019real}, ray-tracing~\cite{waldRTXRayTracing}).
Although there are approaches~\cite{lorensenMarchingCubesHigh,labelleIsosurfaceStuffingFast2007} can convert volumetric fields into meshes, the gaps from discrete sampling degrade the output quality in terms of both meshes and textures.


To address these issues, we present \textbf{EMA}, a method based on \textbf{E}fficient \textbf{M}eshy neural fields to reconstruct animatable human \textbf{A}vatars.
Our method enjoys flexibility from implicit representations and efficiency from explicit meshes, yet still maintains high-fidelity reconstruction quality.
Given video sequences and the corresponding pose tracking, our method digitizes humans in terms of canonical triangular meshes, physically-based rendering (PBR) materials, and skinning weights \textit{w.r.t.} skeletons.
We jointly learn the above components via inverse rendering~\cite{laineModularPrimitivesHighPerformance2020,chenDIBRLearningPredict2021,chenLearningPredict3D2019} in an end-to-end manner.
Each of them is derived from a separate neural field, which relaxes the requirements of a preset human template, rigging, or UV coordinates.
Specifically, we predict a canonical mesh out of a signed distance field (SDF) by differentiable marching tetrahedra~\cite{shenDeepMarchingTetrahedra2021,gaoGET3DGenerativeModel,gaoLearningDeformableTetrahedral2020,munkbergExtractingTriangular3D2022}, then we extend the marching tetrahedra~\cite{shenDeepMarchingTetrahedra2021} for spatial-varying materials by utilizing a neural field to predict PBR materials \textit{on the mesh surfaces} after rasterization~\cite{munkbergExtractingTriangular3D2022,hasselgrenShapeLightMaterial2022,laineModularPrimitivesHighPerformance2020}.
To make the canonical mesh animatable, we take another neural field to model the forward linear blend skinning for the meshes. 
Given a posed skeleton, the canonical mesh is then transformed into the corresponding poses.
Finally, we shade the mesh with a rasterization-based differentiable renderer~\cite{laineModularPrimitivesHighPerformance2020} and train our models with a photo-metric loss.
After training, we export the mesh with materials and discard the neural fields.

\looseness=-1
There are several merits of our method design.
1) \textbf{Efficiency}:
Powered by efficient mesh rendering, our method can render in real-time.
Besides, the training speed is boosted as well, 
since we compute loss holistically on the whole image and the gradients only flow on the mesh surface. In contrast, volume rendering takes limited pixels for loss computation and back-propagates the gradients in the whole space.
Our method only needs about an hour of training and minutes of optimization are enough for plausible avatar reconstruction.
2) \textbf{Disentangled representations}:
Our shape, materials, and motion modules are disentangled naturally by design, which facilitates editing. 
Besides, Canonical meshes with forward skinning modeling handle the out-of-distribution poses better.
3) \textbf{Graphics compatibility}:
Our derived mesh representation is compatible with 
the prominent graphic pipeline, which leads to instant downstream applications (\eg, the shape and materials can be edited directly in design software~\cite{foundationBlenderOrgHome}).
To further improve reconstruction quality, we additionally optimize image-based environment lights and non-rigid motions.


We conduct extensive experiments on standards benchmarks H36M~\cite{ionescuHuman36MLarge2014b} and ZJU-MoCap~\cite{pengNeuralBodyImplicit2021}.
Our method achieves very competitive performance for novel view synthesis, generalizes better for novel poses, 
and significantly improves both training time and inference speed against previous arts.
Our research-oriented code reaches real-time inference speed (100+ FPS for rendering $512\times512$ images).
We in addition showcase applications including novel pose synthesis, material editing, and relighting.
\section{Related Work} \label{part 2}

\subsection{Deep learning for time series forecasting}

\label{part dl related}

Time series forecasting has been studied for decades. The field has been dominated for a long time by statistical tools such as ARIMA, Exponential Smoothing (ES), or (S)ARIMAX, this last model allowing the use of exogenous variables. It now opens itself to deep learning models \citep{9461796}. These new models recently achieved great performances on many datasets. Three main parts compose typical DNNs: an input layer, several hidden layers and an output layer. In this paper we apply our framework to optimize the hidden layers for a given time series forecasting task (see Figure~\ref{fig:metamodel_monash}). In this part, we introduce usual DNN layers for time series forecasting, which can be used in our search space.

The first layer type from our search space is the fully-connected layer, or Multi-Layer Perceptron (MLP). The input vector is multiplied by a weight matrix. Most architectures use such layers as simple building blocks for dimension matching, input embedding or output modelling. The N-Beats model is a well-known example of a DNN based on fully-connected layers for time series forecasting \citep{NBeats}.

The second layer type \citep{lecun2015deep} is the convolution layer (CNN). Inspired by the human brain's visual cortex, it has mainly been popularised for computer vision. The convolution layer uses a discrete convolution operator between the input data and a small matrix called a filter. The extracted features are local and time-invariant if the considered data are time series. Many architectures designed for time series forecasting are based on convolution layers such as WaveNet \citep{oord2016wavenet} and Temporal Convolution Networks \citep{lea2017temporal}.

The third layer type is the recurrent layer (RNN), specifically designed for sequential data processing, therefore, particularly suitable for time series. These layers scan the sequential data and keep information from the sequence past in memory to predict its future. A popular model based on RNN layers is the Seq2Seq network \citep{seq2seq}. Two RNNs, an encoder and a decoder, are sequentially connected by a fixed-length vector. Various versions of the Seq2Seq model have been introduced in the literature, such as the DeepAR model \citep{salinas2020deepar}, which encompasses an RNN encoder in an autoregressive model. The major weakness of RNN layers is the modelling of long-term dynamics due to the vanishing gradient. Long Short-Term Memory (LSTM) and Gated Recurrent Unit (GRU) layers have been introduced \citep{hochreiter1997long, chung2014empirical} to overcome this problem.

Finally, the layer type from our search space is the attention layer. The attention layer has been popularized within the deep learning community as part of Vaswani's transformer model \citep{vaswani2017attention}. The attention layer is more generic than the convolution. It can model the dependencies of each element from the input sequence with all the others. In the vanilla transformer \citep{vaswani2017attention}, the attention layer does not factor the relative distance between inputs in its modelling but rather the element's absolute position in the sequence. The Transformer-XL \citep{dai2019transformer}, a transformer variant created to tackle long-term dependencies tasks, introduces a self-attention version with relative positions. \citet{cordonnier2019relationship} used this new attention formulation to show that, under a specific configuration of parameters, the attention layers could be trained as convolution layers. Within our search space, we chose this last formulation of attention, with the relative positions.

The three first layers (i.e. MLP, CNN, RNN) were frequently mixed into DNN architectures. Sequential and parallel combinations of convolution, recurrent and fully connected layers often compose state-of-the-art DNN models for time series forecasting. Layer diversity enables the extraction of different and complementary features from input data to allow a better prediction. Some recent DNN models introduce transformers into hybrid DNNs. In \citet{lim2021temporal}, the authors developed the Temporal Fusion Transformer, a hybrid model stacking transformer layers on top of an RNN layer. With this in mind, we built a flexible search space which generalizes hybrid DNN models including MLPs, CNNs, RNNs and transformers.

\subsection{Search spaces for automated deep learning}

Designing an efficient DNN for a given task requires choosing an architecture and tuning its many hyperparameters. It is a difficult, fastidious, and time-consuming optimization task. Moreover, it requires expertise and restricts the discovery of new DNNs to what humans can design. Research related to the automatic design and optimization of DNNs has therefore risen this last decade \citep{talbi2021automated}. The first challenge in automatic deep learning (AutoDL), and more specifically neural architecture search (NAS), is search space design. Typical search spaces for Hyperparameters Optimization (HPO) are a product space of a mixture of continuous and categorical dimensions (e.g. learning rate, number of layers, batch size), while NAS focuses on optimizing the topology of the DNN \citep{white2023neural}. Encoding a DNN topology is a complex task because the encoding should not be too broad and allow too many architectures to keep the search efficient. On the contrary, if the encoding is too restrictive, we may miss promising solutions and novel architectures. This means before creating the search space we need to choose which DNNs or type of DNNs are relevant or not to the problem at hand. Once we have decided on this broad set of DNNs, we define the search space following a set of rules \citep{talbi2021automated}:

\begin{itemize}
\item Completeness: all (or almost all) relevant DNNs from this broad set should be encoded in the search space.
\item Connectedness: a path should always be possible between two encoded DNNs in the search space.
\item Efficiency: the encoding should be easy to manipulate by the search operators (i.e. neighbourhoods, variation operators) of the search strategy.
\item Constraint handling: the encoding should facilitate the handling of the various constraints to generate feasible DNNs.
\end{itemize}

A complete classification of encoding strategies for NAS is presented in \citet{talbi2021automated} and reproduced in Figure \ref{fig:classi_encoding}. We can discriminate between direct and indirect encodings. With direct strategies, the DNNs are completely defined by the encoding, while indirect strategies need a decoder to find the architecture back. Amongst direct strategies, one can discriminate between two categories: flat and hierarchical encodings. In flat encodings, all layers are individually encoded \citep{loni2020deepmaker, sun2018particle, wang2018evolving, wang2019evolving}. The global architecture can be a single chain, with each layer having a single input and a single output, which is called chain structured \citep{assuncao_denser_2018}, but more complex patterns such as multiple outputs, skip connections, have been introduced in the extended flat DNNs encoding \citep{chen_scale-aware_2021}. For hierarchical encodings, they are bundled in blocks \citep{pham2018efficient, shu2019understanding, liu2017hierarchical, zhang2019d}. If the optimization is made on the sequencing of the blocks, with an already chosen content, this is referred to as inner-level fixed \citep{camero2021bayesian, white2021bananas}. If the optimization is made on the blocks' content with a fixed sequencing, it is called outer level fixed. A joint optimization with no level fixed is also an option \citep{liu2019auto}. Regarding the indirect strategies, one popular encoding is the one-shot architecture \citep{bender2018understanding, brock2017smash}. One single large network resuming all candidates from the search space is trained. Then the architectures are found by pruning some branches. Only the best promising architectures are retrained from scratch.

\begin{figure*}[htbp]
\centering
    \begin{tikzpicture}
        \draw node[] (SE) {Solution Encoding};
        \draw node[below left =0.5cm and 1.8cm of SE] (Direct) {Direct};
        \draw node[below right =0.5cm and 1.8cm of SE] (Indirect) {Indirect};
        \draw node[below left =0.5cm and 2cm of Direct] (Flat) {Flat};
        \draw node[below right =0.5cm and 2cm of Direct] (Hierarchical) {Hierarchical};
        \draw node[below left =0.5cm and 0cm of Flat, align=center] (Chain) {Chain\\structured};
        \draw node[below right =0.5cm and 0cm of Flat, align=center] (Extended) {Extended\\Flat DNNs};
        \draw node[below =0.5cm of Hierarchical, align=center] (Outer) {Outer\\Level fixed};
        \draw node[left =0.5cm of Outer, align=center] (Inner) {Inner\\Level Fixed};
        \draw node[right =0.5cm of Outer, align=center] (No) {No level\\fixed};
        \draw node[below =of Indirect, align=center] (One) {One-shot};
        \draw[->] (SE) -- (Direct);
        \draw[->] (SE) -- (Indirect);
        \draw[->] (Direct) -- (Flat);
        \draw[->] (Direct) -- (Hierarchical);
        \draw[->] (Flat) -- (Chain);
        \draw[->] (Flat) -- (Extended);
        \draw[->] (Hierarchical) -- (Inner);
        \draw[->] (Hierarchical) -- (Outer);
        \draw[->] (Hierarchical) -- (No);
        \draw[->] (Indirect) -- (One);
    \end{tikzpicture}
    \caption{Classification of encoding strategies for NAS \citep{talbi2021automated}.}
    \label{fig:classi_encoding}
\end{figure*}

Our search space can be categorized as a direct and extended flat encoding. It is based on the representation of DNNs by DAGs. This representation is very popular among the NAS community and is used by cell-based search spaces such as NAS-Bench-101 inspired by the ResNet architecture \citep{ying2019bench}, as well as one-shot representation such as the DARTS framework (for Differentiable Architecture Search) proposed by \citet{liu2018darts}. In cell-based search spaces, DNNs are represented by repeated cells encoded as DAGs, where each node is an operation belonging to a well-defined list, typically: convolution of size 1, 3, or 5, pooling of size 3, skip connection, or zeroed operation for an image classification task for example. The graphs are then represented either as vectors using path encoding, or as adjacency matrices. In the case of path encoding, different search algorithms can be used, such as Bayesian optimization \citep{white_bananas_2020}, reinforcement learning \citep{zoph2018learning}, particle swarm optimization \citep{wang_evolving_2019}, or evolutionary algorithms \citep{xie2017genetic}, for which classical mutation and crossover operators are usually used and consist in modifying the elements of the path. Adjacency matrices, on the other hand, are more complex objects to optimize. The matrix itself represents the connections within the graph and is usually accompanied by a list representing the nodes content. In the literature, these matrices have been optimized directly with random search algorithms \citep{irwin2019graph} or indirectly with neural predictors based on auto-encoders (see for example \citet{zhang2019d} or \citet{chatzianastasis2021graph}). In the case of one-shot representations, an initial large graph containing all the considered DNN is pruned with a certain search algorithm to only keep the best possible subgraph (and thus the best possible DNN). Various search algorithms can be used to simplify this meta-graph \citep{bender2018understanding} like evolutionary algorithm \citep{guo2020single}. One of the most widely used techniques is DARTS \citep{liu2018darts}, where each edge is associated with a candidate operation, assigned to a probability of being retained in the final subgraph, optimized by gradient descent. The candidate operations can be very traditional, such as for cell-based search spaces, but \citet{chen_scale-aware_2021} proposes to use DARTS with other types of operations, such as inter-variable attention, for multivariate time series prediction. While such search spaces have proven to be efficient for tasks like image classification or language processing, \citet{white2023neural} points out that current NAS search spaces are not very expressive and prevents finding highly novel architectures. This problem is amplified when dealing with tasks for which no known architectures have yet been found.

 Compared to these search spaces, the one we define in this paper is more flexible. We address the optimization of both the architecture and the hyperparameters. We do not fix a list of possible operations with fixed hyperparameters, as is done in these works, but leave the user free to use any operation coded as \textit{PyTorch nn.Module} and to optimize any chosen parameters. Furthermore, we do not fix the generic form of our graph, we do not fix a maximum number of incoming or outgoing edges and we allow to expand or reduce the graphs. DRAGON is capable of generating innovative, original, yet well-performing DNNs. This flexibility may hinder the framework's ability to find good DNNs compared to the NAS state-of-the-art for well-known tasks such as image classification or language processing. However, in cases where DNNs have not been extensively studied and well-performing architectures have not yet been found, such as time series prediction, DRAGON may be more useful and powerful. Finally, we encode our DAGs using their adjacency matrices and provide evolutionary operators to directly modify this representation. To our knowledge, neither such a large search space nor such operators have been used in the literature.

%**************************************************
\subsection{AutoML for time series forecasting}
%**************************************************

The automated design of DNNs called Automated Deep Learning (AutoDL), belongs to a larger field \citep{hutter2019automated} called Automated Machine Learning (AutoML). AutoML aims to automatically design well-performing machine learning pipelines, for a given task. Works on model optimization for time series forecasting mainly focused on AutoML rather than AutoDL \citep{alsharef2022review}. The optimization can be performed at several levels: input features selection, extraction and engineering, model selection and hyperparameters tuning. Initial research works used to focus on one of these subproblems, while more recent works offer complete optimization pipelines.

The first subproblems, input features selection, extraction and engineering, are specific to our learning task: time series forecasting. This tedious task can significantly improve the prediction scores by giving the model relevant information about the data. Methods to select the features are among computing the importance of each feature on the results or using statistical tools on the signals to extract relevant information. Next, the model selection aims at choosing among a set of diverse machine learning models the best-performing one on a given task. Often, the models are trained separately, and the best model is chosen. In general, the selected model has many hyperparameters, such as the number of hidden layers, activation function or learning rate. Their optimization usually allows for improving the performance of the model.

Nowadays, many research works implement complete optimization pipelines combining those subproblems for time series forecasting. The Time Series Pipeline Optimization framework \citep{dahl2020tspo}, is based on an evolutionary algorithm to automatically find the right features thanks to input signal analysis, then the model and its related hyperparameters. AutoAI-TS \citep{shah2021autoai} is also a complete optimization pipeline, with model selection performed among a wide assortment of models: statistical models, machine learning, deep learning models and hybrids models. Closer to our work, the framework Auto-Pytorch-TS \citep{deng2022efficient} is specific to deep learning models optimization for time series forecasting. The framework uses Bayesian optimization with multi-fidelity optimization. Finally, a recent work from Amazon \citep{shchur2023autogluon} introduces a time series version to their AutoML framework, AutoGluon, leveraging ensembles of statistical and machine learning forecasters.

Except for AutoPytorch-TS, cited works covering the entire optimization pipeline for time series do not deepen model optimization and only perform model selection and hyperparameters optimization. However, time series data becomes more complex, and there is a growing need for more sophisticated and data-specific DNNs. Our framework DRAGON, presented in this paper, only tackles the model selection and hyperparameters optimization parts of the pipeline. We made this choice to show the effectiveness of our framework for designing better DNNs. If we had implemented feature selection, it would have been harder to determine whether the superiority of our results came from the input features pool or the model itself.

\vspace{-0.3em}
\section{Method}
\vspace{-0.3em}

Our sensitivity-aware visual parameter-efficient fine-tuning consists of two stages. In the first stage, SPT measures the task-specific sensitivity for the pre-trained parameters (Section~\ref{subsec:sensitivity}). Based on the parameter sensitivity and a given parameter budget, SPT then adaptively allocates trainable parameters to task-specific important positions (Section~\ref{subsec:SPT}).

\vspace{-0.3em}
\subsection{Task-specific Parameter Sensitivity}
\label{subsec:sensitivity}
\vspace{-0.3em}

Recent research has observed that pre-trained backbone parameters exhibit varying feature patterns~\cite{raghu2021vision,naseer2021intriguing} and criticality~\cite{zhang2019all,chatterji2019intriguing} at distinct positions. 
Moreover, when transferred to downstream tasks, their efficacy varies depending on how much pre-trained features are reused and how well they adapt to the specific domain gap~\cite{yosinski2014transferable,kumar2022finetuning,neyshabur2020being}. Motivated by these observations, we argue that not all parameters contribute equally to the performance across different tasks in PEFT and propose a new criterion to measure the sensitivity of the parameters in the pre-trained backbone for a given task.

Specifically, given the training dataset $\gD_t$ for the $t$-th task and the pre-trained model weights $\vw=\left\{w_1, w_2, \ldots, w_N\right\}\in \sR^N$ where $N$ is the total number of parameters, the objective for the task is to minimize the empirical risk: $\min_{\vw} E(\gD_t, \vw)$.
We denote the parameter sensitivity \bohan{set} as $\gS=\{s_1, \ldots, s_N\}$ and the sensitivity $s_n$ for parameter $w_n$ is measured by the empirical risk difference when tuning it:
\begin{equation}
\vspace{-0.3em}
    \begin{aligned}
        s_n = E(\gD_t, \vw)-E(\gD_t, \vw\mid w_n=w_n^*),
    \end{aligned}
\label{eq:sensitivity}
\end{equation}
where $w_n^*=\underset{w_n}{\rm argmin}(E(\gD_t, \vw))$. We can reparameterize the tuned parameters as  $w_n^*=w_n+\Delta_{w_n}$, where $\Delta_{w_n}$ denotes the update for $w_n$ after tuning. Here we individually measure the sensitivity of each parameter, which is reasonable given that most of the parameters are frozen during fine-tuning in PEFT. However, it is still computationally intensive to compute Eq.~(\ref{eq:sensitivity}) for two reasons. Firstly, getting the empirical risk for $N$ parameters requires forwarding the entire network $N$ times, which is time-consuming. Secondly, it is challenging to derive $\Delta_{w_n}$, as we have to tune each individual $w_n$ until convergence.

{\begin{algorithm}[t!]
\caption{\label{alg:tps} Computing task-specific parameter sensitivities}
\begin{algorithmic}
    \STATE \textbf{Input:} Pre-trained model with network parameters $\vw$, training set $\gD_t$ for the $t$-th task, and number of training samples $C$ used to calculate the parameter sensitivities
    \STATE \textbf{Output:} Sensitivity set $\gS=\{s_1, \ldots, s_N\}$
    \STATE Initialize $\gS=\{0\}^N$
    \FOR{$i\in\{1,\ldots,C\}$}
        \STATE Get the $i$-th training sample of $\gD_t$
	    \STATE Compute loss $E$
		\STATE Compute gradients $\vg$
		\FOR{$n\in\{1,\ldots,N\}$}
                \STATE Update sensitivity for the $n$-th parameter: $s_{n} = s_{n} + g_n^2$
		    \ENDFOR
    \ENDFOR
\end{algorithmic}
\end{algorithm}}


\begin{figure*}[t]
\begin{center}
    \includegraphics[width=\linewidth]{main_figure.pdf}
\end{center}\vspace{-2em}
\caption{Overview of our trainable parameter allocation strategy. With the parameter sensitivity \bohan{set} $\gS$, we first get the top-$\tau$ sensitive parameters. Instead of directly tuning these sensitive parameters, we also boost the representational capability by replacing unstructured tuning with structured tuning at sensitive weight matrices that have a large number of sensitive parameters, which can be implemented by an existing structured tuning method, \eg, LoRA~\cite{hu2022lora} and Adapter~\cite{houlsby2019parameter}. Red lines and blocks represent trainable parameters and modules, while blue lines represent frozen parameters.}
\label{fig:main}
\vspace{-1.5em}
\end{figure*}


To overcome the first barrier, we simplify the empirical loss by approximating $s_n$ in the vicinity of $\vw$ by its first-order Taylor expansion
\vspace{-0.3em}
\begin{equation}
\vspace{-0.5em}
    \begin{aligned}
        s_n^{(1)} = -g_n\Delta_{w_n},
    \end{aligned}
\label{eq:first-order}
\end{equation}
where the gradients $\vg=\partial E/\partial\vw$, and $g_n$ is the gradient of the $n$-th element of $\vg$. 
To address the second barrier, following~\cite{liu2018darts,cai2018proxylessnas}, we take the one-step unrolled weight as the surrogate for $w_n^*$ and approximate $\Delta_{w_n}$ in Eq.~(\ref{eq:first-order}) with a single step of gradient descent. We can accordingly get $s_n^{(1)} \approx g_n^2\epsilon$,
where $\epsilon$ is the learning rate. Since $\epsilon$ is the same for all parameters, we can eliminate it when comparing the sensitivity with the other parameters and finally get 
\vspace{-0.5em}
\begin{equation}
\vspace{-0.3em}
    \begin{aligned}
        s_n^{(1)} \approx g_n^2.
    \end{aligned}
\label{eq:first-order-simp}
\end{equation}
Therefore, the sensitivity of a parameter can be efficiently measured by its potential to reduce the loss on the target domain. Note that although our criterion draws inspiration from pruning work~\cite{molchanov2019importance}, it is distinct from it. \cite{molchanov2019importance} measures the parameter importance by the squared change in loss when removing them, \ie, $\left( E(\gD_t, \vw)-E(\gD_t, \vw\mid w_n=0) \right)^2$ and finally derives the parameter importance by $\left( g_n w_n \right)^2$, which is different from our formulations in Eqs.~(\ref{eq:sensitivity}) and~(\ref{eq:first-order-simp}).

In practice, we accumulate $\gS$ from a total number of $C$ training samples ahead of fine-tuning to generate accurate sensitivity as shown in Algorithm~\ref{alg:tps}, where $C$ is a pre-defined hyper-parameter. In Section~\ref{subsec:abl}, we show that employing only 400 training samples is sufficient for getting reasonable parameter sensitivity, which requires only 5.5 seconds with a single GPU for any VTAB-1k dataset with ViT-B/16 backbone~\cite{vit}.

\vspace{-0.3em}
\subsection{Adaptive Trainable Parameters Allocation}
\label{subsec:SPT}
\vspace{-0.2em}

Our next step is to allocate trainable parameters based on the obtained parameter sensitivity set $\gS$ and a desired parameter budget $\tau$. A straightforward solution is to directly tune the top-$\tau$ most sensitive unstructured connections (parameters) \rev{while keeping the rest frozen}, which we name unstructured tuning. Specifically, we select the top-$\tau$ most sensitive weight connections in $\gS$ to form the sensitive weight connection set $\gT$. Then, for \rev{a} weight matrix $\mW\in \sR^{d_{\rm in}\times d_{\rm out}}$, we can get a binary mask $\mM\in \sR^{d_{\rm in}\times d_{\rm out}}$ computed by
\vspace{-0.5em}
\begin{equation}
\vspace{-0.5em}
    {\begin{array}{ll}
    \small
    \begin{aligned}
    \mM^j =
    \left\{\begin{array}{ll} 
    1 ~~~~~ \mW^j \in \gT \\
    0 ~~~~~ \mW^j \notin \gT
    \end{array}\right.
    \end{aligned},
    \small
    \end{array}}
\label{eq:mask}
\end{equation}
where $\mW^j$ and $\mM^j$ are the $j$-th element in $\mW$ and $\mM$, respectively. Accordingly, we can train the sensitive parameters by gradient descent and the updated weight matrix can be formulated as $\mW'\leftarrow \mW - \epsilon\vg_{\mW}\odot\mM$, where $\vg_{\mW}$ is the gradient for $\mW$.

However, considering PEFT approaches generally limit the proportion of trainable parameters to less than 1\%, tuning only a small number of unstructured weight connections might not have enough representational capability to handle the downstream datasets with large domain gaps from the source pre-training data. Therefore, to improve the representational capability, we propose to replace unstructured tuning with structured tuning at the sensitive weight matrices that have a high number of sensitive parameters. To preserve the parameter budget, we can implement structured tuning with an existing efficient structured tuning PEFT method~\cite{hu2022lora,chen2022adaptformer,houlsby2019parameter,jie2022convolutional} that learns to directly adjust \rev{all hidden dimensions at once}. We depict an overview of our trainable parameter allocation strategy in Figure~\ref{fig:main}. For example, we can employ the low-rank reparameterization trick LoRA~\cite{hu2022lora} to the sensitive weight matrices \rev{and the one-step update for $\mW$ can be formulated as}
\vspace{-0.4em}
\begin{equation}
\vspace{-0.4em}
    {\begin{array}{ll}
    \small
    \begin{aligned}
    \mW' = \left\{\begin{array}{ll} 
    \mW + \mW_{\rm down}\mW_{\rm up} & ~~ \text { if } ~~ \sum_{j=0}^{d_{\rm in}\times d_{\rm out}} \mM^j \geq \sigma_{\rm opt} \\
    \mW - \epsilon\vg_{\mW}\odot\mM & ~~ {\rm otherwise}
    \end{array}\right.
    \end{aligned},
    \small
    \end{array}}
\label{eq:weight_updat}
\end{equation}
where $\mW_{\rm down}\in \sR^{d_{\rm in}\times r}$ and $\mW_{\rm up}\in \sR^{r\times d_{\rm out}}$ are two learnable low-rank matrices to approximate the update of $\mW$ and rank $r$ is a hyper-parameter where $r \ll {\rm min}(d_{\rm in},d_{\rm out})$. In this way, we perform structured tuning on $\mW$ when its number of sensitive parameters exceeds $\sigma_{\rm opt}$, whose value depends on the pre-defined type of structured tuning method. For example, since implementing structured tuning with LoRA requires $2\times d_{\rm in} \times d_{\rm out} \times r$ trainable parameters for each sensitive weight matrix, we set $\sigma_{\rm LoRA} \leftarrow 2\times d_{\rm in} \times d_{\rm out} \times r$ to ensure that the number of trainable parameters introduced by structured tuning is always equal to or lower than the number of sensitive parameters.

In this way, our SPT adaptively incorporates both structured and unstructured tuning granularities to enable higher flexibility and stronger representational power, simultaneously. In Section~\ref{subsec:abl}, we show that structured tuning is important for the downstream tasks with larger domain gaps and both unstructured and structured tuning contribute clearly to the superior performance of our SPT.
\begin{table*}[t!]
\begin{minipage}{0.175\linewidth}
\centering
% \hspace{1.8mm}
\captionof{table}{\small Datasets statistics \label{graph_datasets}}
\begin{tiny}
\begin{tabular}{c||c|c}
      {\bf Graph} & {\bf \#nodes} & {\bf \#edges} \\ \hline
      {\em FL} & 80\,513     & 5\,899\,882 \\
      {\em YT} & 1\,138\,499 & 2\,990\,443 \\
      {\em LJ} & 2\,238\,731 &14\,608\,137 \\
      {\em OR} & 3\,072\,441 & 117\,185\,083 \\
      {\em TW} & 41\,652\,230 & 1\,468\,365\,182 \\
\end{tabular}
\end{tiny}
\end{minipage}%
 \quad
 \begin{minipage}{.265\linewidth}
\centering
\tabcolsep=0.05cm

\captionof{table}{\small Avg. memory footprint (GB) of {\sf DistGER} and {\sf KnightKing} on each machine, where $\sigma$ is the standard deviation}
\label{Memory_usage}
\begin{tiny}
\newcommand{\tabincell}[2]{\begin{tabular}{@{}#1@{}}#2\end{tabular}}
  % \caption{\small {\color{blue} Avg. memory footprint (GB) of {\sf DistGER} and {\sf KnightKing} on each machine, where $\sigma$ is the standard deviation.}}
  \begin{tabular}{c|cc|cc}
    %\hline
    { }&\multicolumn{2}{c|}{\bfseries{ Sampling}}&\multicolumn{2}{c}{\bfseries{Training}}\\
    \hline
    {\bf{Graph}} &{\sf KnightKing} &{\sf DistGER} &{\sf KnightKing} &{\sf DistGER} \\
    \hline
     {\em FL} & 0.66($\pm$0.06)	&{\bf 0.41($\pm$0.02)}	&1.31($\pm$0.17) 	&{\bf 0.86($\pm$0.06)} 	\\

     {\em YT} &4.11($\pm$0.55)	&{\bf 1.36($\pm$0.23)} 	&4.73($\pm$0.72) 	&{\bf 4.26($\pm$0.63)} \\

     {\em LJ} & 7.65($\pm$0.82)	&{\bf 1.95($\pm$0.16)}	&6.38($\pm$0.97) 	&{\bf 5.49($\pm$0.85)} 	\\

     {\em CO} &10.98($\pm$1.03)	&{\bf 3.27($\pm$0.79)} 	&8.52($\pm$1.01) 	&{\bf 6.86($\pm$0.69)} 	\\

     {\em TW} & out-of-memory	&{\bf 20.18($\pm$3.62)} 	&out-of-memory 	& {\bf 67.16($\pm$5.18)} 	\\
  %\hline
\end{tabular}
\end{tiny}

\end{minipage}
\quad
\begin{minipage}{.25\linewidth}
    \centering
    \includegraphics[width= 1.85in, height = 1.2in]{./Figures/Dist_total_time_partition.eps}%
    \captionof{figure}
      {\small Efficiency: {\sf PBG} \cite{PBG_2019}, {\sf DistDGL} \cite{DistDGL_2020}, {\sf KnightKing} \cite{KnighKing_2019}, {\sf HuGE-D} (baseline), {\sf DistGER} (ours)
        \label{overall_performance}
      }
\end{minipage}%\hfill
\quad
\begin{minipage}{.25\linewidth}
    \centering
    \includegraphics[width= 1.85in, height = 1.2in]{./Figures/Dist_scalability_partition.eps}%
    \captionof{figure}
      {\small Scalability: {\sf PBG} \cite{PBG_2019}, {\sf DistDGL} \cite{DistDGL_2020}, {\sf KnightKing} \cite{KnighKing_2019}, {\sf HuGE-D} (baseline), {\sf DistGER} (ours)
        \label{Dist_scalability}
      }
\end{minipage}
\end{table*}


\section{Experimental Results}
\label{sec:experiments}
We evaluate the efficiency (\S \ref{sec:overall}) and scalability (\S \ref{sec:scalability}) of our proposed method, {\sf DistGER}
by comparing with {\sf HuGE-D} (baseline),
{\sf KnightKing} \cite{KnighKing_2019}, {\sf PyTorch-BigGraph} ({\sf PBG}) \cite{PBG_2019}, and {\sf Distributed DGL}
({\sf DistDGL}) \cite{DistDGL_2020}. We also compare the effectiveness (\S \ref{sec:effectiveness}) of generated embeddings
on link prediction.
% and multi-label classification tasks. 
Finally, we analyze efficiency due to individual
parts of {\sf DistGER} (\S \ref{sec:individual})
and the generality of {\sf DistGER} for other random walk-based embeddings (\S \ref{sec:generality}).
Our codes and datasets are at \cite{code}.
%
\subsection{Experimental Setup}
\label{sec:setup}
%
\spara{Environment.} We conduct experiments on a cluster of 8 machines with 2.60GHz Intel $^\circledR$ Xeon $^\circledR$ Gold 6240 CPU with 72 cores (hyper-threading)
in a dual-socket system, and each machine is equipped with 192GB DDR4 memory and connected by a 100Gbps network.
The machines run Ubuntu 16.04 with Linux kernel 4.15.0. We use GCC v9.4.0 for compiling {\sf DistGER}, {\sf KnightKing}, and {\sf HuGE-D},
and use Python v3.6.15 and torch v1.10.2 as the backend deep learning framework for {\sf Pytorch-BigGraph} and {\sf DistDGL}.

\spara{Datasets. } We employ five widely-used, real-world graphs
(Table~\ref{graph_datasets}): {\em Flickr} (FL) \cite{Flickr_Youtube_Graph},
{\em Youtube} (YT) \cite{Flickr_Youtube_Graph},
{\em LiveJournal} (LJ) \cite{BlogCatalog_Twitter_LiveJournal_Graph},
{\em Com-Orkut} (OR) \cite{com-orkut_2012}, and {\em Twitter} (TW) \cite{twitter_2010}.
The first two graphs are selected for multi-label node classification with distinct number of node labels 195 and 47, respectively, %in {\em Flickr} and {\em Youtube},
where labels in {\em Flickr} represent interest groups of users, and {\em Youtube}'s labels represent groups of viewers that enjoy common video genres. The last four graphs are used in link prediction. We also use synthetic graphs \cite{RMAT_2004} (up to 1 billion nodes, 10 billion edges) and a real-world {\em UK graph} \cite{BSVLTAG} (100M nodes, 3.7B edges) to assess the scalability of {\sf DistGER}.
Considering the default settings of popular random walk-based methods (e.g., Deepwalk, node2vec, HuGE), we use their undirected version.

\spara{Competitors.} We compare {\sf DistGER} against three state-of-the-art distributed graph embedding frameworks: the distributed random walk engine, {\sf KnightKing} {\scriptsize\url{https://github.com/KnightKingWalk/KnightKing}}
\cite{KnighKing_2019}; the distributed multi-relations based graph embedding system, {\sf PyTorch-BigGraph} ({\sf PBG})
{\scriptsize\url{https://github.com/facebookresearch/PyTorch-BigGraph}} \cite{PBG_2019} -- designed by Facebook; and
the distributed graph neural networks-based system, {\sf DistDGL} {\scriptsize\url{https://github.com/dmlc/dgl}}
\cite{DistDGL_2020} -- recently proposed by Amazon. We also implement {\sf HuGE-D}, a distributed version of
information-centric random walk-based graph embedding ({\sf HuGE} \cite{HuGE_2021}), on top of {\sf KnightKing},
served as our baseline. Since {\sf KnightKing} and {\sf HuGE-D} provide distributed support only for
random walk without that for embedding learning, we generate their node embeddings using
{\sf Pword2vec} {\scriptsize\url{https://github.com/IntelLabs/pWord2Vec}} \cite{Pword2vec_2019},
the most popular distributed {\sf Skip-Gram} system released by Intel.
%We find that {\sf pSGNScc} \cite{pSGNSCC_2017} (\S \ref{sec:learning})
%only provides a single-machine implementation, thus we do not include it in our distributed experiments.

\spara{Parameters.} For {\sf DistGER} and {\sf HuGE-D} random walks, we set
parameters $\mu$=0.995, $\delta$=0.001 based on information measurements (\S \ref{sec:preliminaries}),
while {\sf KnightKing} uses $L$=80 and $r$=10 that are routine configurations in the traditional
random walk-based graph embedding \cite{node2vec_2016, DeepWalk_2014, KnighKing_2019}. For {\sf DistGER}, {\sf KnightKing}, and {\sf HuGE-D} training,
we set the sliding window size $w$=10, number of negative samples $K$=5, and synchronization period=0.1 sec \cite{Pword2vec_2019},
and additionally, multi-windows number=2, $\gamma$=2 for {\sf DisrGER}.
%For {\sf Pytorch-BigGraph} ({\sf PBG}), we set the number of partitions to 16 following \cite{PBG_2019}, that is, using $2m$ partitions for the number of machines $m$ = 8
%in our case. %For {\sf DistDGL}, the deployed {\sf GaphSAGE} model uses three graph convolutional layers.
For fair comparison across all systems, %the efficiency performance of all systems involved in the experiments,
we set the embedding dimension $d$=128 that is commonly used \cite{HuGE_2021,node2vec_2016,DeepWalk_2014,Line_2015,Verse_2018,ProNE_2019},
and report the average running time for each epoch. For task effectiveness evaluations,
we find the best results from a grid search over learning rates from 0.001-0.1, \# epochs from 1-30,
and \# dimensions from 128-512.


%
\eat{
\begin{table}
\newcommand{\tabincell}[2]{\begin{tabular}{@{}#1@{}}#2\end{tabular}}
  \caption{\small Avg. memory footprint (GB) of {\sf DistGER} and {\sf KnightKing} on each machine, where $\sigma$ is the standard deviation.}
  \label{Memory_usage}
  \begin{center}
   \footnotesize
  \begin{tabular}{c|cc|cc}
    %\hline
    { }&\multicolumn{2}{c|}{\bfseries{ Sampling}}&\multicolumn{2}{c}{\bfseries{Training}}\\
    \hline
    {\bf{Graph}} &{\sf KnightKing} &{\sf DistGER} &{\sf KnightKing} &{\sf DistGER} \\
    \hline
     {\em Flickr} & 0.66($\pm$0.06)	&{\bf 0.41($\pm$0.02)}	&1.31($\pm$0.17) 	&{\bf 0.86($\pm$0.06)} 	\\

     {\em Youtube} &4.11($\pm$0.55)	&{\bf 1.36($\pm$0.23)} 	&4.73($\pm$0.72) 	&{\bf 4.26($\pm$0.63)} \\

     {\em LiveJournal} & 7.65($\pm$0.82)	&{\bf 1.95($\pm$0.16)}	&6.387($\pm$0.97) 	&{\bf 5.49($\pm$0.85)} 	\\

     {\em Com-Orkut} &10.98($\pm$1.03)	&{\bf 3.27($\pm$0.79)} 	&8.52($\pm$1.01) 	&{\bf 6.86($\pm$0.69)} 	\\

     {\em Twitter} & out-of-memory	&{\bf 37.1($\pm$5.28)} 	&out-of-memory 	& {\bf 79.5($\pm$7.27)} 	\\
  %\hline
\end{tabular}
\end{center}
\end{table}
%
}
%


%
\subsection{Efficiency and Memory Use w.r.t. Competitors}
\label{sec:overall}
%\begin{figure}
%  \centering
%  \includegraphics[width= 3 in]{Dist_total_time.eps}
%  \caption{\small Overall performance of PBG, DistDGL, KnightKing, HuGE-D and DistGER for generating embeddings on different read-word graphs, {\color{blue}for Twitter graph, DistDGL cannot finish in one day, and KnightKing fails to perform due to memory issue, where the y axis is in log-scale.}}
%  \label{overall_performance}
%\end{figure}
%
We report the end-to-end running times of {\sf PBG}, {\sf DistDGL}, {\sf KnightKing}, {\sf HuGE-D}, and {\sf DistGER}
on five real-world graphs with the cluster of 8 machines in Figure~\ref{overall_performance}.
The reported end-to-end time includes the running time of partitioning, random walks (for random walk-based frameworks), and training procedures.
%{\color{blue} Noted that the reported end-to-end time in our experiments excludes the partition time for all evaluated frameworks due to the all used partition schemes are executed as a preprocessing component, and we separately evaluate the partition efficiency in Section 6.5, thus the end-to-end time refers to the running time of random walk (only for random walk-based framework) and training procedure.}
{\sf DistGER} significantly outperforms the competitors
on all these graphs, achieving a speedup ranging from 2.33$\times$ to 129$\times$. %, by an average acceleration of $39.78 \times$.
Recall that {\sf DistGER} is a similar type of system as {\sf KnightKing} and {\sf HuGE-D},
and our key improvements are discussed in \S \ref{sec:DistGER} and in \S \ref{sec:learning}.
Analogously, Figure~\ref{overall_performance} exhibits that our system, %designs are more effective (see more evaluation details in \S 6.3),
{\sf DistGER} achieves an average speedup of 9.25$\times$ and 6.56$\times$ compared with {\sf KnightKing} and {\sf HuGE-D}.
Notice that we fail to run {\sf KnightKing} on the largest {\em Twitter} dataset
because its routine random walk strategy requires more main memory space.
%Although Huge-D achieves comparable performance,
The advantage of information-centric random walk in {\sf HuGE} is almost wiped out in {\sf HuGE-D}
due to on-the-fly information measurements and the higher communication costs in a distributed setting.
The multi-relation-based {\sf PBG} leverages a parameter server to synchronize embeddings between clients,
resulting in more load on the communication network. As a result, {\sf PBG} is on average
26.22$\times$ slower than {\sf DistGER}. For graph neural network-based system {\sf DistDGL},
due to the long running time of graph sampling (e.g., taking 80\% of the overhead for the {\sf GraphSAGE}),
it is highly inefficient than other systems. For the billion-edge {\em Twitter} graph, it does not terminate in 1 day.
%
%Considering the resource consumption that affects scalability,
{Table ~\ref{Memory_usage}} shows {\sf DistGER}'s average memory footprint on each machine of the 8-machine cluster. %from a cluster of 8 machines.
%and the standard deviation %($\sigma$) %of the results
%in 
%Table ~\ref{Memory_usage}. 
Compared
to %other methods, %with the 
same type of system
{\sf KnightKing}, 
% that is of the same system type, 
{\sf DistGER} requires less memory for sampling and training.


\subsection{Scalability w.r.t. Competitors}
\label{sec:scalability}
%
%\begin{figure}
%  \centering
%  \includegraphics[width= 3.2 in]{Dist_scalability.eps}
%  \caption{\small Scalability comparison on LiveJournal graph, where the y axis is in log-scale.}
% \label{Dist_scalability}
%\end{figure}
%
Figure~\ref{Dist_scalability} shows end-to-end running times of all competing
systems on the {\em LiveJournal} graph, as we increase \# machines
from 1 to 8 to evaluate scalability. {\sf DistGER} achieves better scalability than the other
four distributed systems.
%Due to space limitation, we omit results on other graph datasets,
%which exhibit similar trends.
{\sf PBG} leverages a parameter server and a shared network filesystem
to synchronize the parameters in the distributed model. %The edges are partitioned into $m^2$ buckets
%and training can be performed in parallel using up to $m/2$ machines. After one bucket completes
%the training, it needs to communicate with the parameter server.
When the number of machines increases, {\sf PBG} puts more load
on the communications network, resulting in poor scalability. Likewise, {\sf DistDGL}
is bounded by the synchronization overhead for gradient updates,
limiting its scalability.
%Since {\sf DistDGL} uses mini-batches for sampling, %features %for GraphSAGE,
%if the mini-batch samples cannot be generated on time, the trainer will be delayed on the forward pass, and all other
%machines need to wait before starting the backward pass. Thus, increasing the number of
%machines also affects the efficiency of backward pass. %Being the random walk-based distributed systems,
Both {\sf KnightKing} and {\sf HuGE-D} suffer from higher communication costs during random walks,
due to their only workload-balancing partitioning scheme (\S \ref{sec:dRand}, \S \ref{sec:individual}).
%Their scalability is relatively poor as the number of machines increases.
%{\sf KnightKing} partitions the graph by a workload-balancing scheme, inevitably introducing higher
%cross-machine communications due to the randomness inherent in the random walking procedure (\S \ref{sec:dRand}, \S \ref{sec:individual}).
%With more machines, the inefficiency of the partitioning scheme is further magnified.
Since {\sf HuGE-D} is implemented on top of {\sf KnigtKing},
it exhibits worse scalability due to high communication costs and on-the-fly information measurements in a distributed setting (\S \ref{sec:HUGED}).
%In contrast, its performance is much better than all the competitors in a single machine.
In comparison, {\sf DistGER} incorporates multi-proximity-aware streaming graph partitioning and incremental computations
to reduce both communication and computation costs, it also employs hotness-block based parameters synchronization
during training to dramatically reduce the pressure on network bandwidth. Hence, {\sf DistGER} achieves better scalability than other systems.
Due to space limitations, we omit {\sf DistGER}'s scalability results on other graphs, which exhibit similar trends. On {\em Twitter}, the end-to-end running times {\sf DistGER} on 1, 2, 4, and 8 machines are 3090s, 1739s, 1197s, and 746s, respectively,
while on {\em Com-Orkut}, the results are 304s, 204s, 149s, and 89s, respectively. 
The results show a good linear relationship.
% The results demonstrate a desired scalability with the increase of the machines.

\begin{table}
\quad
\begin{minipage}{0.46\linewidth}
    \centering
    \includegraphics[width= 1.6 in]{./Figures/Dist_scalability_datasize.eps}%
    \captionof{figure}
      {\small {Scalability of {\sf DistGER} on synthetic graphs, where Y-axis is in log-scale}}
      %The lines depict the running time required for random walk (blue line) and training (red line), respectively. Pentagrams show the time cost of six real-world graphs,
        \label{Dist_scalability_data}
      
\end{minipage}\hfill
\quad
\begin{minipage}{.46\linewidth}
    \centering
    \includegraphics[width= 1.6 in]{./Figures/Dist_time_auc.eps}%
    \captionof{figure}
      {\small {The influence of running time on embedding quality for {\sf DistGER} and competitors}}
        \label{Dist_time_auc}
\end{minipage}
\end{table}


To further assess the scalability of {\sf DistGER}, we generate synthetic graphs \cite{RMAT_2004} with a fixed node degree of 10 and the number of nodes from $10^5$ to $10^9$. Figure~\ref{Dist_scalability_data} presents the running times for random walks and training on these synthetic graphs using a cluster of 8 machines, suggesting that the running time increases linearly with the size of a graph, and {\sf DistGER} has the capability to handle even billion-node graphs. Moreover, the running times for six real-world graphs (including the {\em UK graph} with $|E|=3.7B$, $|V|=100M$, for which the competing systems do not terminate in 1 day or crash due to hardware and memory limitation) are inserted into the plot, which is consistent with the trend on synthetic data.

%
%
\subsection{Effectiveness w.r.t. Competitors}
\label{sec:effectiveness}
%
\spara{Link prediction.} To perform link prediction on a given graph $G$, following \cite{HuGE_2021,node2vec_2016,Verse_2018,NRP_2020},
we first uniformly at random remove 50\% edges as positive test edges, and the rest are used as positive training edges.
We also provide negative training and test edges by considering those node pairs between which no edge exists in $G$.
We ensure that the positive and negative set sizes are similar. %For a pair of nodes $(u, v)$, let $\varphi(u)$ and
%$\varphi(v)$ be the vectors learned by embedding methods.
The link prediction is conducted as a classification task
based on the similarity of $u$ and $v$, i.e., $\varphi(u)\cdot\varphi(v)$.
The effectiveness of link prediction is measured via the $AUC$ (Area Under Curve) score \cite{AUC_kdd} -- the higher the better.
We repeat this procedure 50 times to offset the randomness of edge removal and report the average $AUC$ in
Table~\ref{AUC_results}.
%shows $AUC$ for all the methods on five real-world graphs.
%, respectively, where a ``$-$'' indicates that the method fails due to the limitation of computing resources or because its running time exceeds 1 day.
{\sf DistGER} outperforms all competitors on these graphs, except for {\sf PBG} on {\em Com-Orkut}, where {\sf DistGER} ranks second.
On average, {\sf DistGER} has an 11.7\% higher $AUC$ score compared with the other three systems, thanks to our
information-centric random walks. {\sf PBG} is the best on {\em Com-Orkut} because this graph is much denser
and is friendly to the multi-relationship-based model in {\sf PBG}.
Figure~\ref{Dist_time_auc} exhibits accuracy-efficiency tradeoffs of {\sf DistGER} and competitors, i.e., their $AUC$ convergence curves w.r.t. increasing running times of random walks and training, over {\em LiveJournal}, further indicating
that {\sf DistGER} has better efficiency and effectiveness than the competitors.
%As a system of the same type, DistGER achieves better accuracies on all graphs than KnightKing which leverages the routine random walk configuration, thanks to its information-centric random walk strategies. We do not report the effectiveness of HuGE-D here because it uses the same random walk model as DistGER.
%
\begin{table}[h!]
\newcommand{\tabincell}[2]{\begin{tabular}{@{}#1@{}}#2\end{tabular}}
  \caption{\small $AUC$ scores of {\sf DistGER} and competitors for link prediction}
  \label{AUC_results}
  \begin{center}
  \footnotesize
  \begin{tabular}{cccccc}
%    \hline
    {Method}&\tabincell{c}{Youtube}&{LiveJournal}&\tabincell{c}{Com-Orkut}&{ Twitter}\\
    \hline
    {\sf PBG}        & 0.753           &0.882            &\bfseries{0.955} &0.912\\

    {\sf DistDGL}    &0.894            &0.718            &0.815            & running time $>$ 1 day \\

    {\sf KnightKing} &0.904            &0.963            & $0.918$         & out-of-memory\\

    {\sf DistGER}    &\bfseries{0.966} &\bfseries{0.976} &0.921            &\bfseries{0.919}\\
%  \hline
\end{tabular}
\end{center}
\end{table}

% \eat{
\spara{Multi-label node classification.}
This task predicts one or more labels for each graph node and has applications in %modern applications ranging from
text categorization \cite{zhang2006multilabel} and bioinformatics \cite{zhang2018ontological}.
We use embedding vectors and a one-vs-rest logistic regression classifier
with L2 regularization \cite{MLC_LIBLINEAR_2008}, %(using the LIBLINEAR library),
then evaluate the effectiveness by micro-averaged F1 ($Micro-F1$) and macro-averaged F1 ($Macro-F1$) \cite{WangC016}
scores, where $Micro-F1$ gives equal weight to each test instance and $Macro-F1$ assigns equal weight to each label category \cite{keikha2018community}.
%To train a classifier, nodes are uniformly at random split into training and test sets.
Following \cite{HuGE_2021,node2vec_2016,DeepWalk_2014,Line_2015,Verse_2018},
we select 10\% to 90\% training data ratio on {\em Flickr}, and 1\% to 9\% training ratio on {\em Youtube}.
%and the remaining nodes for testing.
We report the averaged $Macro-F1$ and $Micro-F1$ scores from 50 trials in Figure~\ref{Dist_MLC_mac_mic_F1}.
% shows the $Macro-F1$ and $Micro-F1$ scores achieved by each system as a
%function of the training ratio variation, respectively.
We find that {\sf DistGER} has better $Macro-F1$ and $Micro-F1$ scores
than existing frameworks, %on these graphs, %. In particular, compared with the KnightKing,
%DistGER consistently outperforms the other random walk-based systems on all graphs in $Macro-F1$ and $Micro-F1$ scores,
gaining 9.2\% and 3.3\% average improvements, respectively, due to its more effective information-centric random walks.
%Definition of $Macro-F1$ and $Micro-F1$ are as the following:

\begin{figure}[h!]
  \centering
  \includegraphics[width= 3.45 in]{./Figures/Dist_MLC_mac_mic_F1_1.eps}
  \caption{\small $Macro-F1$ (a1, b1) and $Micro-F1$ (a2, b2) scores for multi-label node classification. $X$-axis: training data ratio}
  \label{Dist_MLC_mac_mic_F1}
\end{figure}

% }
%\begin{equation}
%Precision = \frac{\sum\nolimits_{i}^{K}TP(i)}{\sum\nolimits_{i}^{K}(TP(i)+FP(i))}
%\end{equation}
%
%\begin{equation}
%Recall = \frac{\sum\nolimits_{i}^{K}TP(i)}{\sum\nolimits_{i}^{K}(TP(i)+FN(i))}
%\end{equation}
%
%\begin{equation}
%Micro-F1 = \frac{2\times Precision\times Recall}{Precision+Recall}
%\end{equation}
%
%\begin{equation}
%Macro-F1 = \frac{\sum\nolimits_{i}^{K}Micro-F1(i)}{|K|}
%\end{equation}
%
%where $TP(i)$, $FP(i)$ and $FN(i)$ are the number of true positives, false positives and false negatives in the instances which are predicted as $i$, respectively. Suppose $K$ is the overall label set, $Micro-F1$($i$) and $Macro-F1$ are the measure of $Micro-F1$ and $Macro-F1$ for the label $i$, respectively.
%\begin{table}
%\setlength{\abovecaptionskip}{0.cm}
%\setlength{\belowcaptionskip}{-0.cm}
%\newcommand{\tabincell}[2]{\begin{tabular}{@{}#1@{}}#2\end{tabular}}
%  \caption{$Macro-F1$ and $Micro-F1$ for multi-label classification on Flickr and Youtube graph, where train ratio is 0.5.}
%  \label{cluster_results}
%  \begin{center}
%  \small
%  \begin{tabular}{ccccc}
%    \hline
%    { }&\multicolumn{2}{c}{\bfseries{ \scriptsize Flickr}}&\multicolumn{2}{c}{\bfseries{\scriptsize Youtube}}\\
%    \hline
%    { }&Macro-F1 &Micro-F1&Macro-F1 &Micro-F1\\
%
%    \hline
%    \small PBG & 0.225	&0.387 	&0.295 	&0.406 	\\
%
%    \small DistDGL &0.205 	&0.378 	&0.283 	&0.403 	\\
%
%    \small KnightKing &0.239    &0.386 &0.285 	&0.402 	\\
%
%    \small DistGER  &\bfseries{0.277} &\bfseries{0.409}&\bfseries{0.298} &\bfseries{0.417}\\
%
%  \hline
%\end{tabular}
%\end{center}
%\end{table}
%

\begin{figure}
  \centering
  \includegraphics[width= 3.2 in]{./Figures/Dist_sampling_training_Mpad_efficiency.eps}
  \caption{\small {(a) Random walk efficiency, (b) training efficiency, (c) \# cross-machine messages, (d) random walk efficiency for {\sf MPGP} (ours) and workload-balancing scheme ({\sf KnightKing})}}
  \label{Dist_efficiency_sampling_training_MPGP}
\end{figure}

\begin{table*}[t!]
\begin{minipage}{0.275\linewidth}
\centering
\renewcommand\arraystretch{1.2}
\captionof{table}{\small Performance evaluation of partitioning for {\sf DistGER} and Competitors } %{\sf PBG} and {\sf DistDGL}
\label{Partition_sechme_overhead}
\begin{scriptsize}
\begin{tabular}{ccccc}
    \multicolumn{5}{c}{{\bfseries (a) Partitioning time for {\sf DistGER} and competitors }} \\
    \hline
    {\bf graph} & {\sf PBG} & {\sf DistDGL} & {\sf DistGER}\\
                &           & ({\sf METIS}) &  ({\sf MPGP}) \\
    \hline
    {\sf FL} & 383.28 s & 127.72 s & \bfseries{15.96 s} \\
    {\sf YT} & 349.15 s & 116.30 s & \bfseries{13.56 s} \\
    {\sf LJ} & 458.52 s & 425.19 s & \bfseries{36.42 s} \\
    {\sf OR} & 2662.62 s & 2761.25 s &\bfseries{294.68 s}\\
    {\sf TW} & 22 hour s & $>$ 1 day &\bfseries{9 hours}\\
    \hline
%    \multicolumn{5}{c}{} \\
    \multicolumn{5}{c}{{\bfseries (b) Evaluation of {\sf Parallel MPGP} }} \\
    \hline
    {\bf graph} & {\sf Streaming} & {\sf Partitioning} & {\sf Walking}\\
    \hline
  %  {\sf MPGP}   &DFS+deg  & 9 hours & \bfseries{575.22 s} \\
    \multirow{2}{*}{\sf LJ} &DFS+deg & 21.86 s & \bfseries{23.78 s} \\
           & BFS+deg & \bfseries{21.25 s} & 24.79 s \\
    \multirow{2}{*}{\sf OR} &DFS+deg & \bfseries{151.29 s} & 77.12 s \\
           & BFS+deg & 156.37 s & \bfseries{46.55 s} \\
    \multirow{2}{*}{\sf TW} &DFS+deg & \bfseries{1940.65} s & 683.81 s \\
           & BFS+deg & 2034.21 s & \bfseries{590.36 s}
\end{tabular}
\end{scriptsize}
\end{minipage}%\hfill
\quad
\begin{minipage}{.3\linewidth}
    \centering
    \includegraphics[width= 2.5in, height = 1.45 in]{./Figures/Dist_Mpad_streaming_vertex_time.eps}%
    \captionof{figure}
      {\small The distribution of local computations and cross-machine communications for different streaming orders on {\em LiveJournal}. The top table reports their running times for partitioning and random walks
        \label{Dist_MPaD_streaming}
      }
\end{minipage}%\hfill
\qquad
\begin{minipage}{.37\linewidth}
    \centering
    \includegraphics[width= 2.5 in, height = 1.45 in]{./Figures/Dist_generality_table_HuGE+.eps}%
    \captionof{figure}
      {\small Generality of {\sf DistGER} vs. {\sf KnightKing}. The bars show random walk efficiency ($-R$) and training efficiency ($-T$) for {\sf Deepwalk} ({\sf DW}), {\sf node2vec} ({\sf n2v}) and {\sf HuGE+}. The top table shows the ratio $\frac{\text{{\em AUC} for {\sf DistGER}}}{\text{{\em AUC} of {\sf KnightKing}}}$, with {\sf DW} and {\sf n2v}, task: link prediction
        \label{Dist_generality}
      }
\end{minipage}
\end{table*}


\subsection{Efficiency due to Individual Parts of DistGER}
\label{sec:individual}
\spara{Random walk and training efficiency.}
To evaluate the system design of {\sf DistGER} (\S \ref{sec:DistGER}, \S \ref{sec:learning}),
we first compare the efficiency of random walks and training with those of {\sf KnighKing} and {\sf HuGE-D}.
%For fair comparison, the running times that we reported for {\sf KnightKing} and {\sf HuGE-D} exclude the
%time of vocabulary table construction, since it is a serial process in {\sf Pwode2vec}, while {\sf DistGER}
%pipelines the construction during random walks.
For random walks (Figure~\ref{Dist_efficiency_sampling_training_MPGP}(a)),
{\sf DistGER} significantly outperforms {\sf KnightKing} and {\sf HuGE-D} on all our graph
datasets, achieving an average speedup of $3.32\times$ and $3.88\times$, respectively.
Although {\sf HuGE-D} implements information-oriented random walks on {\sf KnightKing},
due to additional computation and communication overheads during on-the-fly information
measurements (\S \ref{sec:HUGED}), its efficiency can be lower than that of {\sf KnightKing}.
We also notice that the random walk lengths ($L$) and the number of random walks ($r$) reduce (on average)
63.2\% and 18\%, respectively, in our information-oriented random walks, compared to {\sf KnightKing}'s
routine random walk configuration.
%which supports the traditional
%random walk methods. %To provide a straightforward adaptation for the information-oriented approach, DistGER leverages the incremental information-centric computation mechanism to mitigate the redundant computation and high communication cost in HuGE-D, then it achieves an average speedup of $3.32\times$ and $3.88\times$ in random walk procedure compared to KnightKing and HuGE-D.

Another benefit of information-centric random walks is that it generates concise and effective corpus to improve 
training efficiency. Compared to {\sf KnightKing}, {\sf DistGER} achieves $17.37\times$-$27.95\times$ acceleration
in training over all our graphs. Next, considering the same corpus size, we compare the training efficiency of {\sf Pword2vec} and {\sf DSGL}
(trainer in {\sf DistGER}). Figure~\ref{Dist_efficiency_sampling_training_MPGP}(b) shows that {\sf DSGL} achieves $4.31\times$ average speedup
compared to {\sf Pword2vec}. We also notice that the average throughput (number of nodes processed per second) for {\sf DSGL} is up to 49.5 million/s,
while that of {\sf Pword2vec} is only up to 16.1 million/s. These results indicate that our distributed {\sf Skip-Gram} learning model (\S \ref{sec:learning})
is more efficient than {\sf Pword2vec}.
%
%\begin{figure}
%  \centering
%  \includegraphics[width= 2.5 in]{Dist_Mpad_efficiency.eps}
%  \caption{\small (a) exhibits the number of cross-machine computation for DistGER on workload-balancing and MPGP partition scheme, respectively, and (b) shows the random walk time of DistGER on the two schemes, where y axis is in log-scale.}
%  \label{Dist_efficiency_MPaD}
%\end{figure}

\spara{Partitioning efficiency.} Considering the %large number cross-machine computing introduced by the
randomness inherent in random walks, the partitioning scheme is critical to overall efficiency. %of the distributed framework.
%To validate the efficiency of our multi-proximity-aware streaming graph partitioning (MPGP),
%we deploy the workload balancing scheme used in KnightKing and MPGP on DistGER,
%respectively,
%and report the number of cross-machine computations during the random walk procedure for the two schemes.
%We also present the efficiency performance of MPGP compared with the workload-balancing scheme.
For {\sf DistGER},
Figure~\ref{Dist_efficiency_sampling_training_MPGP}(c) exhibits that our multi-proximity-aware streaming graph partitioning ({\sf MPGP})
significantly reduces (avg. reduction $45\%$) the number of cross-machine messages than the workload-balancing partition of {\sf KnightKing}
on five graphs. Moreover, it improves the efficiency by 38.9\% for the random walking procedure
(Figure~\ref{Dist_efficiency_sampling_training_MPGP}(d)) over the same set of walks.
We report in Table~\ref{Partition_sechme_overhead}(a) the time required for graph partitioning in competing systems,
where {\sf DistDGL} uses the {\sf METIS} algorithm \cite{METIS_1998} for partitioning.
The results show that {\sf MPGP} performs partitioning with very little overhead in most cases, and
the partitioning efficiency is on average $25.1\times$ faster than competitors.
In Figure~\ref{Dist_MPaD_streaming}, we exhibit the distribution of local computations and cross-machine communications
on four machines for different streaming orders, and the top table reports their running times for partitioning and random walks.
For sequential {\sf MPGP}, we find that the {\sf DFS+degree}-based streaming order (\S \ref{sec:partition}) is more efficient than other streaming orders,
and it also strikes the best balance between cross-machine communications reduction and workload balancing.
Table~\ref{Partition_sechme_overhead}(b) exhibits the performance evaluation of {\sf parallel MPGP} on the small- ({\em LiveJournal}), medium- ({\em Com-Orkut}) and large-scale ({\em Twitter}) graphs. The results show that {\sf DFS+Degree} in {\sf parallel MPGP} is still the best or comparable in terms of partition time, due to the same reason as stated in our third optimization scheme (\S \ref{sec:partition}). On the other hand, {\sf BFS+Degree} in {\sf parallel MPGP} works the best in terms of random walk time due to preserving the locality of the graph structure (our fourth optimization scheme in \S \ref{sec:partition}).
%as using its streaming order to parallel partitioning can reduce the influence of relevance between each segment.
We ultimately recommend {\sf BFS+Degree} for {\sf parallel MPGP}, since it reduces the partition time greatly, while the random walk time is comparable to that obtained from sequential {\sf MPGP}.
%
%\begin{figure}
%  \centering
%  \includegraphics[width= 3 in]{Dist_Mpad_streaming_vertex_time.eps}
%  \caption{\small The distribution of local computations and cross-machine communications for different streaming orders on {\em LiveJournal}. The top table reports their running times for partitioning and random walks.}
%  \label{Dist_MPaD_streaming}
%\end{figure}
%
%\begin{table}
%\setlength{\abovecaptionskip}{0.cm}
%\setlength{\belowcaptionskip}{-0.cm}
%\newcommand{\tabincell}[2]{\begin{tabular}{@{}#1@{}}#2\end{tabular}}
%  \caption{\small Time execution time (seconds) of the partition scheme in PBG, DistDGL, and DistGER, ``$-$'' means the scheme fails under constrains of computation resource.}
%  \label{Partition_sechme_overhead}
%  \begin{center}
%  \small
%  \begin{tabular}{ccccc}
%    \hline
%    {Graph}&{PBG}&{DistDGL(METIS)}&{DistGER}\\
%    \hline
%    Flickr& 383.28 &127.72 &\bfseries{15.96} \\
%
%    Youtube& 349.15 &116.30&\bfseries{13.56} \\
%
%    LiveJournal& 458.52 &425.19 &\bfseries{36.42} \\
%
%    Com-Orkut& 2662.62 &2761.25 &\bfseries{294.68}\\
%
%    Twitter&78986.85 &$-$&\bfseries{35500.41}\\
    %\hline
%    \multicolumn{5}{l}{* HuGE+ generates the smallest corpus size for training among all methods tested.} \\
%
%  \hline
%\end{tabular}
%\end{center}
%\end{table}
%


\subsection{Generality of DistGER}
\label{sec:generality}
%\begin{figure}
%  \centering
%  \includegraphics[width= 3 in, height= 1.65 in]{Dist_generality_table.eps}
%  \caption{\small Generality comparison for DistGER and KnightKing, %on real-word graphs,
%  The bars display random walk (denoted as $-R$) and training efficiency (denoted as $-T$) for {\sf Deepwalk} (DW) and {\sf node2vec} (n2v), respectively.%, and the y axis is in log-scale.
%  Top table shows the ratio $\frac{\text{{\em AUC} for {\sf DistGER}}}{\text{{\em AUC} of {\sf KnightKing}}}$, both with Deepwalk and node2vec, respectively, considering link prediction.}
%  \label{Dist_generality}
%\end{figure}
%
%Since our proposed information-oriented random walk framework DistGER aims to address the redundant computations and high communication cost introduced by the effectiveness measurement of the generated walking information in distributed setting, it provides a good systematic support for the information-centric approach HuGE as shown by the previous experimental results. A natural question arises: can DistGER also support the traditional random-walk-based methods?
To demonstrate the generality of {\sf DistGER}, we deploy {\sf Deepwalk} \cite{DeepWalk_2014}, {\sf node2vec} \cite{node2vec_2016} and {
\sf HuGE+} \cite{HuGE+_2022}
on {\sf DistGER}. While the original {\sf Deepwalk} and {\sf node2vec} follow
traditional random walks, in {\sf DistGER} the walk length and the number of walks are decided via information-centric measurements.
Next, we also deploy both {\sf Deepwalk} and {\sf node2vec} on {\sf KnightKing} which supports the routine configuration random walk.
Figure~\ref{Dist_generality} illustrates that {\sf DistGER} reduces the random walks time by 41.1\% and 51.6\% on average for
{\sf Deepwalk} and {\sf node2vec}, respectively. For training, {\sf DistGER} is on average $17.7\times$ and $21.3\times$ faster than {\sf KnightKing}+{\sf Pword2vec}
for {\sf Deepwalk} and {\sf node2vec}, respectively.
Moreover, we also show the {\em AUC} ratio of {\sf DistGER} and {\sf KnightKing}, considering {\sf Deepwalk} and {\sf node2vec}, for link prediction.
% tasks, where performing multi-label classification on Flickr graph  and link prediction on other graphs, the accuracy metric for the two task are $Miro-F1$ and $AUC$ score, respectively, it can be found from
Our results depict that {\sf DistGER} has comparable (in most cases, higher) {\em AUC} scores, while it improves the efficiency significantly
even for traditional random walk-based graph embedding methods.
{\sf HuGE+} is an extension of {\sf HuGE}, and it uses the same {\sf HuGE} information-centric method to determine the walk length and the number of walks per node. Figure~\ref{Dist_generality} exhibits the compatibility of {\sf HuGE+} on {\sf DistGER} via its general API.
%%%%%%%%%%%%%%%%%%%%%%%%%%%%%%%%%%%%%%%%%%%%%%%%%

\section{Conclusion} 
We introduce \algname{}, a novel object detection method that achieves highly efficient inference speed while also improving zero-shot generalization compared with existing methods. The prompt-based decoding approach reduces the computational burden of object queries. The RoI-based masked attention and RoI pruning techniques allow us to efficiently leverage a large ViT-based CLIP model, enhancing detection performance through classification prediction ensembling. Comprehensive experiments show that \algname{} is $21.2$ times faster than OV-DETR while achieving comparable or higher APs on base and novel classes compared to two-stage OVD methods. %We believe that our work will inspire future work to explore the benefits of using Transformers.


%\paragraph{Ethics Statement.} 
%The focus of this paper is on open-vocabulary object detection. Our approach involves the integration of Transformer-based object detector and CLIP. We have not identified any foreseeable negative social impact associated with our work to share our findings with the scientific community. Nonetheless, we will continue to monitor and consider any potential concerns that may arise. 




%%%%%%%%% REFERENCES
{\small
\bibliographystyle{ieee_fullname}
\bibliography{egbib}
}

\clearpage
\section{Proofs}
\subsection{Derivation of GSS ELBO}
We provide the proof of Eq.~\eqref{equ:elbo_sem} in the main paper here.
We rewrite the log-likelihood of semantic segmentation $\log{p(c|x)}$ by introducing  a discrete L-dimension latent distribution $q(z|c)$ (with $z\in \mathbb{Z}^L$).
\begin{align*}
    \notag\log{p(c|x)} &= \log{\int p(c,z|x)} \text{\ d}z\\
    \notag &= \log{\int p(c,z|x)} \frac{q(z|c)}{q(z|c)}\text{\ d}z\\
    \notag &=\log\mathbb{E}_{q(z|c)}\left[\frac{p(c,z|x)}{q(z|c)}\right]\\
    \notag &\geq \mathbb{E}_{q(z|c)}\left[\log \frac{p(c,z|x)}{q(z|c)} \right]\\
    \notag \left( \text{as} \right.-log(\cdot)& \text{ is convex, by Jensen's Inequality: }\\
    \notag f(\sum_i\lambda_i x_i)& \leq \sum_i \lambda_if(x_i)\text{, where } \lambda_i\geq 0, \sum_i\lambda_i=1 \left. \right)\\
    \notag  &= \mathbb{E}_{q(z|c)}\left[\log \frac{p(c|z)p(z|x)}{q(z|c)} \right]\\
    \notag  &= \mathbb{E}_{q(z|c)}\left[\log p(c|z) \right] + \mathbb{E}_{q(z|c)}\left[\log \frac{p(z|x)}{q(z|c)} \right]\\
    \notag  = \mathbb{E}_{q(z|c)}&\left[ \log p(c|z) \right] -D_{KL}(q(z|c), p(z|x)) \\
    \notag  = \mathbb{E}_{q_\phi(z|c)}&\left[ \log p_\theta(c|z) \right] -D_{KL}(q_\phi(z|c), p_\psi(z|x)).
\end{align*}

Different from ELBO~\cite{kingma2013auto} in VAE, the latent variable we introduce here is $q(z|c)$, rather than $q(z)$ to solve the conditioned mask generation problem.

\subsection{Derivation of latent posterior learning}
We provide the proof of Eq.~\eqref{eq:posterior} in the main paper here.
As stated in main paper, the first stage latent posterior training is conducted by a MSE loss
\begin{equation*}
    \min_{\theta, \phi}\sum_c \mathbb{E}_{q_\phi(z|c)}\|p_\theta(c|z)-c\|.
\end{equation*}

Let us denote $\hat{c}=p_\theta(c|z)$ is the reconstructed mask.
Then, we define a linear transform $x^{(c)}=\mathcal{X}_\beta(c)=c\beta$, where $\beta\in\mathbb{R}^{K\times 3}$ and an arbitrary inverse transform $\hat{c}=\mathcal{X}^{-1}_\gamma(\hat{x}^{(c)})$.
Noted that the parameter $\gamma$ can be non-linear.
$x^{(c)}$ is called \texttt{maskige} and $\hat{x}^{(c)}$ is the reconstructed \texttt{maskige} produced by the {\tt maskige} decoder $\hat{x}^{(c)}=\mathcal{D}_{{\theta}}(\hat{z})$. 
The transformed latent parameter $\hat{z}$ preserves the probability for the linear transformation,
\begin{figure}[tb]
	\begin{center}
\includegraphics[width=1.0\linewidth]{figures/sup_obj_drawio.pdf}
	\end{center}
	\caption{
	\textbf{An illustration of our transformed objective.} \texttt{Rec.} stands for reconstruction.}
	\label{fig:sup_obj}
\end{figure}
\begin{equation}
\label{equ:qhatz}
    q_{\hat{\phi}}(\hat{z}|x^{(c)}) =q_{\hat{\phi}}(\hat{z}|c\beta)= q_\phi(z|c).
\end{equation}
Then, we have
\begin{align}
    \notag &\min_{\theta, \phi}\sum_c \mathbb{E}_{q_\phi(z|c)}2\|\hat{c}-c\|\\
    \notag =&\min_{\theta, \phi, \gamma}\sum_c \mathbb{E}_{q_\phi(z|c)}\left[\|\mathcal{X}^{-1}(\hat{x}^{(c)})-c\| + \|\hat{c}-c\|\right].
\end{align}
For the first term, since $\mathcal{X}^{-1}(\hat{x}^{(c)})=\mathcal{X}^{-1}(\mathcal{D}_{{\theta}}(\hat{z}))$ which is not related to $\theta$ and $\phi$.
Therefore, we have
\begin{align}
    \notag &\min_{\theta, \phi, \gamma}\sum_c \mathbb{E}_{q_\phi(z|c)}\|\mathcal{X}^{-1}(\hat{x}^{(c)})-c\|\\
     \notag=&\min_{\gamma}\sum_c \mathbb{E}_{q_\phi(z|c)}\|\mathcal{X}^{-1}(\mathcal{D}_{\theta}(\hat{z}))-c\|\\
    \label{equ:t1}
    =&\min_{\gamma,\beta}\sum_c \mathbb{E}_{q_{\hat{\phi}}(\hat{z}|x^{(c)})}\|\mathcal{X}^{-1}(\mathcal{D}_{{\theta}}(\hat{z}))-c\|\ (\text{Eq.~\eqref{equ:qhatz}}).
\end{align}
\begin{figure*}[tb]
	\begin{center}
		\includegraphics[width=0.9\linewidth]{figures/sup_architecture_drawio.pdf}
	\end{center}
	\caption{
\textbf{Encoder-decoder style architecture of \model{}.} ``Pred." stands for prediction.}
	\label{fig:sup_arch}
\end{figure*}

For the second term, $\hat{c}=\mathcal{D}_\theta(z)$ is not related to $\gamma$, which can be rewritten as
\begin{align}
    \notag &\min_{\theta, \phi}\sum_c \mathbb{E}_{q_\phi(z|c)}\|\hat{c}-c\|\\
    \notag &=\min_{\theta, \phi, s.t. \|\beta\|=1}\sum_c \mathbb{E}_{q_\phi(z|c)}\|\hat{c}-c\|\|\beta\|\\
    \notag &=\min_{\theta, \phi, s.t. \|\beta\|=1}\sum_c \mathbb{E}_{q_\phi(z|c)}\|\hat{c}\beta-c\beta\|\\
    % \notag &=\min_{\theta, \phi, s.t. \|\beta\|=1}\sum_c \mathbb{E}_{q_\phi(z|c)}\|\hat{c}\beta-\hat{x}^{(c)} + \hat{x}^{(c)} - x^{(c)} + x^{(c)} -c\beta\|\\
    \notag &= \min_{\theta, \phi, s.t. \|\beta\|=1} \sum_c
    \mathbb{E}_{q_\phi(z|c)}\|(\hat{c}\beta-\hat{x}^{(c)}) \ (\text{equal to 0 by def.})\\
    \notag &\quad\quad\quad\quad\quad\quad\quad\quad+ (\hat{x}^{(c)} - x^{(c)})\\
    \notag&\quad\quad\quad\quad\quad\quad\quad\quad+ (x^{(c)} -c\beta)\|\ (\text{equal to 0 by def.})\\
    \notag &=\min_{\theta, \phi, s.t. \|\beta\|=1}\sum_{x^{(c)}} \mathbb{E}_{q_\phi(z|c)} \|\hat{x}^{(c)} - x^{(c)}\|\ (\text{not related to }\theta)\\
    \notag &=\min_{{\theta}, \phi, s.t. \|\beta\|=1}\sum_{x^{(c)}} \mathbb{E}_{q_\phi(z|c)} \|\mathcal{D}_{{\theta}}(\hat{z}) - x^{(c)}\|\\
    \notag &=\min_{{\theta}, \hat{\phi}, \beta, s.t. \|\beta\|=1}\sum_{x^{(c)}} \mathbb{E}_{q_{\hat{\phi}}(\hat{z}|x^{(c)})}
    \label{equ:t2}\|\mathcal{D}_{{\theta}}(\hat{z}) - x^{(c)}\|\ (\text{Eq.~\eqref{equ:qhatz}}).
\end{align}
%
Combining Eq.~\eqref{equ:t1} and Eq.~\eqref{equ:t2}, our final objective is 
\begin{align}
    \notag& \min_{\hat{\phi},{\theta}, \beta, s.t. \|\beta\|=1}\sum_{x^{(c)}} \mathbb{E}_{q_{\hat{\phi}}(\hat{z}|x^{(c)})}\|\mathcal{D}_{{\theta}}(\hat{z})-x^{(c)}\|\\
    + &\quad \min_{\gamma, \beta}\sum_c\mathbb{E}_{q_{\hat{\phi}}(\hat{z}|x^{(c)})}\|\mathcal{X}^{-1}(\mathcal{D}_{{\theta}}(\hat{z}))-c\|.
\end{align}
For the first term, it is a VQVAE~\cite{van2017neural} reconstruction objective for \texttt{maskige}.
Therefore, a VQVAE pretrained by DALL$\cdot$E~\cite{ramesh2021zero} with a large-scale OpenImage dataset can 
readily offer a good 
lower bound for the first term.

As such, only the second term is left for optimization.
We can optimize the $\gamma$ with gradient descent,
corresponding to {\model{}-\gssc{}\&\gsse{}}. 
Besides, we can solve this problem more efficiently by a {linear assumption}, \ie{} $\mathcal{X}^{-1}(\hat{x}^{(c)}) = \hat{x}^{(c)}\gamma$ where $\gamma\in\mathbb{R}^{3\times K}$.
We denote the $\hat{X}^{(c)}$ is a matrix with each row an reconstructed \texttt{maskige} and $C$ is a matrix with each row an input mask.
We solve the optimization with least square error
\begin{align}
    \notag&\|\mathcal{X}^{-1}(\hat{X}^{(c)})-C\|^2\\
    \notag=&\|\hat{X}^{(c)}\gamma-C\|^2\\
    \notag=&\|(\hat{X}^{(c)}-X^{(c)}+X^{(c)})\gamma-C\|^2\\
    \label{equ:opt_gamma_beta_mid}
    \leq&\left(\|\hat{X}^{(c)}-C\beta\|\|\gamma\|+\|X^{(c)}\gamma-C\|\right)^2.
\end{align}
The optimization over both $\beta$ and $\gamma$ is non-convex (as shown by the poor performance with {\model{}-\gsse{}}), so we optimize them sequentially in {\model{}-\gssb{}}\&{\gssc{}\&\gssd{}}.
For {\model{}-\gssb{}}, we use a hand-crafted optimized $\beta$.
\begin{align}
    \notag&\left(\|\hat{X}^{(c)}-C\beta\|\|\gamma\|+\|X^{(c)}\gamma-C\|\right)^2\\
    \notag\leq&\left(\tau\|\gamma\|+\|X^{(c)}\gamma-C\|\right)^2\\
    \notag=&\left(\tau\|\gamma\|+\|C\beta\gamma-C\|\right)^2\\
    \label{equ:opt_gamma_beta_mid2}
    \leq&\left(\tau\|\gamma\|+\|C\|\|\beta\gamma-\mathbb{1}\|\right)^2.
\end{align}
where $\tau=\|\hat{X}^{(c)}-C\beta\|$ is bounded and unrelated to $\gamma$ by provided VQVAE and $\beta$.
Our objective then changes to minimize the upper bound.
\begin{align}
    \notag \min_{\gamma, s.t. \|\gamma\|=1} \text{RSS}(\gamma)&=\min_{\gamma, s.t. \|\gamma\|=1} \|\beta\gamma-\mathbb{1}\|^2\\
    \label{equ:rss}
    &=\min_{\gamma, s.t. \|\gamma\|=1}(\beta\gamma-\mathbb{1})^\top(\beta\gamma-\mathbb{1}).
\end{align}
We take the derivative of Eq.~\eqref{equ:rss}, then
\begin{equation}
\label{equ:drss}
    \frac{\partial \text{RSS}}{\partial \gamma}=2\beta^\top(\beta\gamma-\mathbb{1})=0.
\end{equation}
The unique solution of Eq.~\eqref{equ:drss} is 
\begin{align}
    \notag\beta^\top\beta\gamma&=\beta^\top\\
    \notag(\Rightarrow)\quad (\beta^\top\beta)^{-1}(\beta^\top\beta)\gamma &= (\beta^\top\beta)^{-1}\beta^\top\\
    \label{equ:gamma_mse}
    (\Rightarrow)\quad\quad\quad\quad\quad\quad\quad\ \ \gamma &= (\beta^\top\beta)^{-1}\beta^\top.
\end{align}

For the special design {\model{}-\gssd{}}, we use a cascaded optimization to automatically optimize $\beta$ and $\gamma$.
\begin{align}
    \notag\beta_{t+1} &= \mathop{\arg\min}\limits_{\beta}\left(\|\hat{X}^{(c)}-C\beta\|\|\gamma_t\|+\|C\beta\gamma_t-C\|\right)^2\\
    \notag\gamma_{t+1} &= (\beta_{t+1}^\top\beta_{t+1})^{-1}\beta_{t+1}^\top.
\end{align}
The $\notag\beta_{t+1}$ is optimized by one mini-batch step of gradient descent.

\begin{table*}[htb]
\tablestyle{3.8pt}{1.05}
\centering
\renewcommand\tabcolsep{3.9pt}
\renewcommand\arraystretch{1.2}
\small
\begin{tabular}{llcccccc|c}
\hline

\hline

\hline

\hline
Method & Iteration &VOC~\cite{everingham2010pascal} & Context~\cite{mottaghi2014role} & CamVid~\cite{brostow2009semantic} & WildDash~\cite{zendel2018wilddash}   & KITTI~\cite{geiger2013vision}  & ScanNet~\cite{dai2017scannet} & \textit{h.\ mean} \\
\hline

\hline
\emph{- Discriminative modeling:}\\
\shline
CCSA~\cite{motiian2017unified} & 500k &48.9 & - & 52.4 & 36.0 & - & 27.0 & 39.7 \\
MGDA~\cite{sener2018multi} & 500k &69.4 & - & 57.5 & 39.9 & - & 33.5 & 46.1 \\
MSeg-w/o relabel~\cite{lambert2020mseg}  & 500k & 70.2 & 42.7 & 82.0 & 62.7 &  65.5 & 43.2 & 57.6\\
MSeg~\cite{lambert2020mseg}  &  500k & 70.7 & 42.7 & 83.3 & 62.0 & 67.0 & 48.2 & 59.2\\
MSeg-480p~\cite{lambert2020mseg} & 1,500k & 76.4 & 45.9 & 81.2 & 62.7 & \textbf{68.2} & 49.5 & 61.2 \\
MSeg-720p~\cite{lambert2020mseg} & 1,500k & 74.7 & 44.0 & 83.5 & 60.4 & 67.9 & 47.7 & 59.8 \\
MSeg-1080p~\cite{lambert2020mseg} & 1,500k & 72.0 & 44.0 & \textbf{84.5} & 59.9 & 66.5 & 49.5 & 59.8 \\
\hline

\hline
 \emph{- Generative modeling:}\\
 \shline
\rowcolor[gray]{.9}
\model-\gssb{}~(Ours) & 160k & 78.7 & 45.8 & 74.2 & 61.8 & 65.4 & 46.9 & 59.5\\
\rowcolor[gray]{.9}
\model-\gssc{}-W~(Ours) & 160k &
\textbf{79.5} & \textbf{47.7} & 75.9 & \textbf{65.3} & {68.0} &  \textbf{49.7} & \textbf{61.9}\\
\hline

\hline

\hline

\hline
\end{tabular}
\caption{\textbf{Additional cross-domain semantic segmentation performance on MSeg dataset {\tt test} split~\cite{lambert2020mseg}:}  
We add performance of Mseg-480p, Mseg-720p and Mseg-1080p~\cite{lambert2020mseg} results to Table~\ref{tab:mseg} of the main paper. \textbf{No improved versions of our methods are included.}
``480p", ``720p" and ``1080p" mean all test images are resized to 480p (the shorter side is 480 pixel), 720p and 1080p, respectively, when MSeg model inference.
}
\label{tab:supp_mseg}
% \vspace{-1em}
\end{table*}

\section{{\tt Maskige} optimization designs}

As illustrated above, $\beta$ is a linear projection applied on the ground-truth mask, \ie{} $x^{(c)}=c\beta$.
%
As is shown in Eq.~\eqref{equ:opt_gamma_beta_mid} and Eq.~\eqref{equ:opt_gamma_beta_mid2}, the quality of $\|\hat{X}^{(c)}-C\beta\|$ directly affects the quality of the following $\gamma$ optimization, so the optimization of $\beta$ will affect the difficulty degree of our learning of $\mathcal{X}_\gamma^{-1}$.
%
The cascaded optimization in {\model{}-\gssd{}} (84.37\% reconstruction mIoU) provides an upper-bound for $\beta$ optimization. 
However, the gradient descent optimization costs another $5$ GPU hours according to Table~\ref{tab:vqvae} of main paper.
%
Therefore, we propose a hand-crafted optimization of $\beta$ in {\model-\gssb{}\&\gssc{}\&\gssc{}-W} that achieves satisfying performance without requiring extra training time, based on the \emph{maximal distance assumption}.

To understand this assumption, we can consider the linear projection parameter $\beta \in \mathbb{R}^{K \times 3}$ as a colorization process, where each category is assigned an \texttt{rgb} color. The idea behind the {\em maximal distance assumption} is to maximize the color difference between the encoding of each category. For instance, if two different categories are assigned similar colors, the model may struggle to differentiate between them. Therefore, by maximizing the distance between color embeddings, we can improve the model's ability to distinguish between categories.
%
We interpret the parameter $\beta$ as R, G, B color sequences $\mathcal{A}^r, \mathcal{A}^g, \mathcal{A}^b$ assigned to each category. To better satisfy the \emph{maximal distance assumption}, we will try different ways to construct these sequences, \ie., assigning colors to each category.

\emph{\textbf{(i)} Arithmetic sequence on R/G/B channels}: 
Designing three arithmetic sequences $\mathcal{A}^r, \mathcal{A}^g, \mathcal{A}^b$ for R/G/B channels respectively. Then we have
\begin{equation}
\mathcal{A}^m = \left \{a_1^m, a_2^m, \dots, a_i^m, \dots, a_n^m \right \}, m \in \left \{r, g, b\right \}.
\end{equation}
For the $i$-th color value,
\begin{equation}
      a_i^m = a_1^m + (i-1)\cdot k^m,k^m \in N^+, 
\end{equation}
where color channel $m \in \left \{r, g, b\right \}$, the interval of arithmetic sequence $k^m$ can be difference between channels, $a_1^m$ default is $0$.
The set of colors is the Cartesian product of these three series, 
\begin{equation}
    \mathcal{C} =  \mathcal{A}^r \times \mathcal{A}^g \times \mathcal{A}^b.
\end{equation}
\Eg, if the interval of R, G, B channel $k=45$, the color set $\mathcal{C}$ will be $\left \{(0, 0, 0)\right .$, $(0, 0, 45), \dots,$ $\left .(225, 225, 225) \right \}$.

\emph{\textbf{(ii)} Misalignment start points}: The original starting point of the arithmetic sequence is 0, 0, 0 for R/G/B respectively. In order to avoid duplication of values, we let R/G/B have different starting points, 
\begin{equation}
    a_1^{r}\ne a_1^{g} \ne a_1^{b}.
\end{equation}
In practice, we simply set to $a_1^r=0, a_1^g=1, a_1^b=2$.

\emph{\textbf{(iii)} Random additive factors}: Adding three independent random factors $t \in [0, T]$ on the R/G/B arithmetic sequence respectively, to avoid repetition of several same values,
\begin{equation}
    a_i^m = a_1^m + (i-1)\cdot k^m + t_i^m.
\end{equation}
\Eg, a color sequence with random additive factors: $\left \{(1, 7, 3)\right .$, $(4, 2, 45)$, $\dots$, $\left .(235, 215, 232)\right \}$. In practical terms, $T$ is set to $15$.

\emph{\textbf{(iv)} Category-specific refinement}: We equip the lower IoU categories with values where the R/G/B values vary at large degrees (\eg, we replace $(128, 128, 128)$ with $(0, 128, 255)$). In addition, we keep the color away from gray as possible, because gray is located in the center of the color space, thus being close to many categories and giving rise to a harder learning problem.
Such an category-specific refinement allows each category to
be possibly furthest from the others as possible.

\begin{table}[t]
\tablestyle{1.8pt}{1.05}

% \vspace{-0.4cm}
\renewcommand\tabcolsep{13.1pt}
\renewcommand\arraystretch{1.2}
\small
\begin{center}
 
\begin{tabular}{l|cc}
\hline

\hline

\hline

\hline
{Colorization technique} & {mIoU} & {aAcc} \\
\hline

\hline
Arithmetic sequence & 85.99 & 94.37  \\
~~~+ Misalignment start points & 86.12 & 94.45  \\
~~~+ Random additive factors & 87.42 & 95.08  \\
\rowcolor[gray]{.9}
~~~+ Category-specific refinement & \textbf{87.73} & \textbf{95.29}  \\

\hline

\hline

\hline

\hline
\end{tabular}
\end{center}
\caption{
\textbf{Ablation on \emph{maximal distance assumption}:} The \texttt{maskige} reconstruction performance (mIoU and aAcc) of {\model{}-\gssb} on ADE20K {\tt val} split under different Mask-to-\texttt{maskige} transformations $\fx$.
}
\label{tab:supp_colorization}
\end{table}

\paragraph{Results}

As shown in Table~\ref{tab:supp_colorization},
it is evident that the colorization design for \texttt{maskige} generation
presents a good amount of impact on the reconstruction performance.
In particular, the last design {category-specific refinement}
yields the best results, conforming our intuition and design consideration.

\paragraph{Visualization}
For visual understanding,
in Figure~\ref{fig:supp_color_ade_mda} and Figure~\ref{fig:supp_color_ade_opt} we visualize the 150 colors corresponding to all the categories of ADE20K~\cite{zhou2019semantic} generated by 
the \emph{maximal distance assumption} (hand-designed) and gradient descent optimization (learned), respectively.
We observe that the hand-designed method
produces the colors with enhanced contrast and greater vibrancy.
Instead, the colors learned 
are vibrant for the more frequent categories and relatively dark for the less frequent categories.

\section{Overall architecture}
Following \cite{esser2021taming, ramesh2021zero}, the modeling of latent prior learning is formulated by an encoder-decoder architecture (See Figure~\ref{fig:sup_arch}).
%
For the image encoder $\mathcal{I}_\psi$, we take the advantage of hierarchical shifted window transformer~\cite{liu2021swin}
for extracting the multi-scale information~\cite{wang2021pyramid} and memory efficiency~\cite{lu2021soft}.
This is different from UViM \cite{kolesnikov2022uvim}, which uses a single-scale and full-range Transformer as the encoder.
To implement the image encoder, we use the Swin-Large architecture \cite{liu2021swin}, pre-trained on ImageNet-22K \cite{deng2009imagenet}, as the backbone.
As shown in Figure~\ref{fig:sup_arch}, we use four-scale feature maps ($1/4$, $1/8$, $1/16$, $1/32$) and upsample all the lower-resolution features to $1/4$ scale, then concatenate four features across the channel \cite{xie2021segformer}.
The multi-level aggregation consists of an MLP and $D$ layers of hierarchical shifted-window Transformer \cite{liu2021swin}, with the swin window size set to 7, the number of attention heads to 16, the embedding dimension to 512, and the FFN dimension to 1024. 
For the implementation version with resnet as the backbone, $D=6$. However, for models with strong Swin Transformer backbones, fewer MLA layers are needed, and thus $D=2$.
We implement the {\tt maskige} decoder $\mathcal{D}_\theta$ as a fixed VQVAE decoder \cite{van2017neural}.

\section{More training details}
\noindent \emph{\textbf{(i)} Latent posterior learning}: 
As illustrated before, the latent posterior learning is simplified as:
\begin{equation}
    \min_{\fxpi}\mathbb{E}_{q_{\hat{\phi}}(\hat{z}|\mathcal{X}(c))}\|\mathcal{X}^{-1}(\hat{x}^{(c)})-c\|.
\end{equation} 
The target can be interpreted as minimizing the distance between a ground-truth segmentation mask and the predicted mask.
Following \cite{kolesnikov2022uvim, chen2022generalist}, we use cross-entropy loss instead of euclidean distance for a better minimization between segmentation masks.

\noindent \emph{\textbf{(ii)} Latent posterior learning for $\fx$}: For GSS variants whose $\mathcal{X}$ dose not require training, such as {\model{}-\gssb{}}\&{\gssc{}}\&{\gssc{}-W}, we assign a 3-channel encoding to each category directly based on the {\em maximum distance assumption}. 
For the GSS variants that require training, including {\model{}-\gssd{}}\&{\gssd{}}\&, we freeze the parameters of the VQVAE of the DALL-E pretrain and train $\fx$ for 4,000 iterations using \texttt{SGD} optimizer with a batch size of 16. 
By the way, the training process of \model{}-\gsse{} also optimizes the $\fxpi$ function.

\noindent \emph{\textbf{(iii)}~Latent posterior learning for $\fxpi$}: 
We propose a method for training an $\fxpi$ that is more robust to noise (used in {\model{}-\gssc{}}\&{\gssc{}-W}). We found that training $\fxpi$ with \textit{noisy \texttt{maskige}} helps it learn to be more robust. To provide noisy {\tt maskige}, we can use the trained $\mathcal{I}_\psi$, DALL$\cdot$E pre-trained $\mathcal{D}_\theta$, and $\fxpi$ to directly predict noisy {\tt maskige} predictions. In practice, we use $\mathcal{I}_\psi$ that trained up to the middle checkpoint (\eg, 32,000 iterations) or final checkpoint in latent prior learning. $\fxpi$ is trained using cross-entropy loss and optimized with \texttt{AdamW}, with a batch size of 16. We trained \model{}-{\gssc{}-W} for 40,000 iterations, while \model{}-\gssc{} was trained for only 3,000 iterations due to its fast convergence.
The non-linear function $\fxpi$ is implemented using either a convolutional or Swin block structure. 
Specifically, for \model{}-\gssc{}, the structure comprises two conv $1 \times 1$ layers enclosing a conv $3 \times 3$ layer. 
However, this approach is superseded by \model{}-\gssc{}-W, the final model, which employs a group of Swin blocks with a number of heads of 4, a Swin window size of 7, and an embedding channel of 128 to realize $\fxpi$. 
Regardless of the specific implementation, $\fxpi$ relies on local RGB information in the predicted mask to deduce the category of each pixel.

\noindent \emph{\textbf{(iv)}~Latent prior learning}:
For the optimization of~~$\mathcal{I}_\psi$, we use the AdamW optimizer and implement a polynomial learning rate decay schedule\cite{zhao2017pyramid} with a minimum learning rate of $0.0$. We set the initial learning rate to $1.5 \times 10^{-3}$ for Cityscapes and $1.2 \times 10^{-4}$ for ADE20K and Mseg.

\section{Domain generic \texttt{maskige} and image encoder}
We did two tests by deriving a {\bf \em general maskige} on MSeg~\cite{lambert2020mseg}.

\noindent \text{\textbf{(i)}}
As shown in Table~\ref{tab:share_maskiage_between_cityscapes_and_mseg}, we applied our general \texttt{maskige} to the Cityscapes dataset and achieved a mIoU score of 79.5, which is only slightly lower than the mIoU score of 80.5 obtained using the Cityspaces specific \texttt{maskige}. This result demonstrates the versatility of our \texttt{maskige} across different datasets.

\noindent \text{\textbf{(ii)}} To further evaluate the effectiveness of our domain-generic approach, we shared the image encoder $\mathcal{I}_\psi$ between MSeg and Cityscapes and trained our model on the training split of MSeg. We then evaluated the model on the {\bf \em zero-shot} test split consisting of 6 unseen datasets. As shown in Table~\ref{tab:mseg}, our GSS outperforms other state-of-the-art methods on the MSeg dataset. These experiments demonstrate that our \texttt{maskige} is domain-generic and has the potential for open-world settings.

\section{Additional quantitative results}
We additionally compare the improved versions of MSeg~\cite{lambert2020mseg} with 1,500k longer training on the cross-domain benchmark.
As shown in Table~\ref{tab:supp_mseg},
despite using short training, our model still achieves 
better performance.
This verifies the advantage of our method
in terms of training efficiency, in addition to the accuracy.

\section{Additional qualitative results}
For further qualitative evaluation,
we visualize the prediction results of our GSS on both single-domain segmentation datasets~\cite{cordts2016cityscapes,zhou2019semantic} and cross-domain segmentation dataset~\cite{lambert2020mseg}.

As shown in Figure~\ref{fig:supp_visualization_stage_2_cityscapes}, our \model{} has an accurate perception of buses, trucks and pedestrians in distance, whilst also splitting the dense and slim poles. 
In Figure~\ref{fig:supp_visualization_stage_2_ade20k}, we see that \model{} correctly recognises a wide range of furniture such as curtains, cabinets, murals, doors and toilets;
This suggests that our {\texttt maskige} generative approach can accurately represent a wide range of semantic entities. 
Figure~\ref{fig:supp_visualization_stage_2_mseg_main} and Figure~\ref{fig:supp_visualization_stage_2_meg_kitti} show the cross-domain segmentation performance
on images from previously unseen domains (Mseg {\tt test} datasets). 
It can be seen that \model{} performs well in all five datasets in the MSeg {\tt test} split~\cite{lambert2020mseg}, further validating that our generative algorithm has strong cross-domain generalization capabilities.

\section{Reproduced semantic segmentation version of UViM~\cite{kolesnikov2022uvim}}
We reproduce UViM with {\em mmsegmentation} and follow the hyperparameter and structure in the paper~\cite{kolesnikov2022uvim}.
To achieve a fair comparison with our approach, we have made some modifications: 
\textbf{(i)} we implement Swin-Large~\cite{liu2021swin} pretrained on ImageNet 22K~\cite{deng2009imagenet} as the Language model $LM$ as ours; 
\textbf{(ii)} we generate the Guiding code straightforwardly in a non-autoregressive manner; 
\textbf{(iii)} we trained 80k iterations in the first stage of UViM~\cite{kolesnikov2022uvim} and 160k iterations in the second stage. 
These modifications are necessary to ensure \textbf{a fair comparison}.

\section{Societal impact}
Given that the strong cross-domain generalization capability,  we consider our model has the potential to be used in a wide range of visual scenarios. This is desired in practical applications
due to the benefits of reducing the demands of per-domain model training and easier deployment and system management.
This is meaningful and advantageous in both economics and environment. 
%
On the other hand, there exist the potential to spawn abuse of our algorithm and unexpected undesirable uses. 
%
Therefore, it is necessary to strengthen the regulation and supervision of algorithm applications, in order to guarantee that new algorithms including ours can be used safely and responsibly for the good of the humanity and society.

\section{Limitations and Future Work}
While our study represents a significant step forward for generative segmentation, our models still fall short of the performance achieved by top discriminative models. 
One contributing factor is that decision boundaries for generative models are often less precise than those of discriminative models, resulting in less accurate object edges in segmentation. 
Another drawback is that generative models require larger amounts of data to achieve good performance, because discriminative models only learn decision boundaries, while generative models need to learn the distribution of the entire sample space. In our experiments, the performance of MSeg is better compared to Cityscapes and ADE20K, which roughly indicates this point. 

Additionally, since we convert all categories to colors, the color space is limited, and as the number of categories increases, the colors become more crowded. This can lead to confusion when using $\mathcal{X}^{-1}$ to query and predict the closest pre-defined color for each category from {\tt maskige}, especially near object edges. Therefore, it is worth trying to expand this space to higher dimensions.

Looking ahead, there are several avenues for future research in generative semantic segmentation. One promising direction is instance-level segmentation, which would enable more precise identification and separation of individual objects within an image. Additionally, we believe that it would be valuable to explore a unified model that can perform multiple vision tasks, such as segmentation, 2D object detection, depth prediction, 3D detection, and more.

Given that the second stage training of \model{} focuses on latent prior learning, new vision tasks could be inclusively added by incorporating a new posterior distribution of latent variables, without requiring any changes to the model architecture. By pursuing these directions, we believe that significant advances can be made in the field of generative semantic segmentation.

\begin{figure*}[th]
	\def \imwidth {5.6cm}
	% \def \adeheight {2.0cm}
	\def \cityheight {3cm}
	\centering
	\includegraphics[height=\cityheight, width=\imwidth, align=b]{figures/supp_visualization_cityscapes/img/frankfurt_000000_003357_leftImg8bit.png}
	\includegraphics[height=\cityheight, width=\imwidth, align=b]{figures/supp_visualization_cityscapes/gt/frankfurt_000000_003357_leftImg8bit_rec.png}
	\includegraphics[height=\cityheight, width=\imwidth, align=b]{figures/supp_visualization_cityscapes/pred/frankfurt_000000_003357_leftImg8bit_pred.png}\\
	\includegraphics[height=\cityheight, width=\imwidth, align=b]{figures/supp_visualization_cityscapes/img/frankfurt_000000_005898_leftImg8bit.png}
	\includegraphics[height=\cityheight, width=\imwidth, align=b]{figures/supp_visualization_cityscapes/gt/frankfurt_000000_005898_leftImg8bit_rec.png}
	\includegraphics[height=\cityheight, width=\imwidth, align=b]{figures/supp_visualization_cityscapes/pred/frankfurt_000000_005898_leftImg8bit_pred.png}
	\\
 	\includegraphics[height=\cityheight, width=\imwidth, align=b]{figures/supp_visualization_cityscapes/img/frankfurt_000000_010763_leftImg8bit.png}
	\includegraphics[height=\cityheight, width=\imwidth, align=b]{figures/supp_visualization_cityscapes/gt/frankfurt_000000_010763_leftImg8bit_rec.png}
	\includegraphics[height=\cityheight, width=\imwidth, align=b]{figures/supp_visualization_cityscapes/pred/frankfurt_000000_010763_leftImg8bit_pred.png}
	\\
  	\includegraphics[height=\cityheight, width=\imwidth, align=b]{figures/supp_visualization_cityscapes/img/frankfurt_000000_011007_leftImg8bit.png}
	\includegraphics[height=\cityheight, width=\imwidth, align=b]{figures/supp_visualization_cityscapes/gt/frankfurt_000000_011007_leftImg8bit_rec.png}
	\includegraphics[height=\cityheight, width=\imwidth, align=b]{figures/supp_visualization_cityscapes/pred/frankfurt_000000_011007_leftImg8bit_pred.png}
	\\
  	\includegraphics[height=\cityheight, width=\imwidth, align=b]{figures/supp_visualization_cityscapes/img/frankfurt_000000_001016_leftImg8bit.png}
	\includegraphics[height=\cityheight, width=\imwidth, align=b]{figures/supp_visualization_cityscapes/gt/frankfurt_000000_001016_leftImg8bit_rec.png}
	\includegraphics[height=\cityheight, width=\imwidth, align=b]{figures/supp_visualization_cityscapes/pred/frankfurt_000000_001016_leftImg8bit_pred.png}
	\\
  	\includegraphics[height=\cityheight, width=\imwidth, align=b]{figures/supp_visualization_cityscapes/img/frankfurt_000000_002196_leftImg8bit.png}
	\includegraphics[height=\cityheight, width=\imwidth, align=b]{figures/supp_visualization_cityscapes/gt/frankfurt_000000_002196_leftImg8bit_rec.png}
	\includegraphics[height=\cityheight, width=\imwidth, align=b]{figures/supp_visualization_cityscapes/pred/frankfurt_000000_002196_leftImg8bit_pred.png}
	\\
	\rotatebox{0}{\textcolor{white}{---}Image}
	\rotatebox{0}{\textcolor{white}{------------------------------------}Ground Truth}
	\rotatebox{0}{\textcolor{white}{----------------------------------}Prediction}
	\caption{Qualitative results of semantic segmentation on  Cityscapes {\tt val} split~\cite{cordts2016cityscapes}.
	}
	\label{fig:supp_visualization_stage_2_cityscapes}
\end{figure*}
\begin{figure*}[th]
	\def \imwidth {5.6cm}
	\def \adeheight {3.6cm}
	% \def \cityheight {3cm}
	\centering

	\includegraphics[height=\adeheight, width=\imwidth, align=b]{figures/supp_visualization_ade20k/img/ADE_val_00000086.jpg}
	\includegraphics[height=\adeheight, width=\imwidth, align=b]{figures/supp_visualization_ade20k/gt/ADE_val_00000086_gt.png}
	\includegraphics[height=\adeheight, width=\imwidth, align=b]{figures/supp_visualization_ade20k/pred/ADE_val_00000086_learnable_pred_our_color.png}\\
	\includegraphics[height=\adeheight, width=\imwidth, align=b]{figures/supp_visualization_ade20k/img/ADE_val_00000622.jpg}
	\includegraphics[height=\adeheight, width=\imwidth, align=b]{figures/supp_visualization_ade20k/gt/ADE_val_00000622_gt.png}
	\includegraphics[height=\adeheight, width=\imwidth, align=b]{figures/supp_visualization_ade20k/pred/ADE_val_00000622_learnable_pred_our_color.png}
	\\
 	\includegraphics[height=\adeheight, width=\imwidth, align=b]{figures/supp_visualization_ade20k/img/ADE_val_00000713.jpg}
	\includegraphics[height=\adeheight, width=\imwidth, align=b]{figures/supp_visualization_ade20k/gt/ADE_val_00000713_gt.png}
	\includegraphics[height=\adeheight, width=\imwidth, align=b]{figures/supp_visualization_ade20k/pred/ADE_val_00000713_learnable_pred_our_color.png}\\
	\includegraphics[height=\adeheight, width=\imwidth, align=b]{figures/supp_visualization_ade20k/img/ADE_val_00000714.jpg}
	\includegraphics[height=\adeheight, width=\imwidth, align=b]{figures/supp_visualization_ade20k/gt/ADE_val_00000714_gt.png}
	\includegraphics[height=\adeheight, width=\imwidth, align=b]{figures/supp_visualization_ade20k/pred/ADE_val_00000714_learnable_pred_our_color.png}
	\\
	\includegraphics[height=\adeheight, width=\imwidth, align=b]{figures/supp_visualization_ade20k/img/ADE_val_00001135.jpg}
	\includegraphics[height=\adeheight, width=\imwidth, align=b]{figures/supp_visualization_ade20k/gt/ADE_val_00001135_gt.png}
	\includegraphics[height=\adeheight, width=\imwidth, align=b]{figures/supp_visualization_ade20k/pred/ADE_val_00001135_learnable_pred_our_color.png}\\
	\includegraphics[height=\adeheight, width=\imwidth, align=b]{figures/supp_visualization_ade20k/img/ADE_val_00001282.jpg}
	\includegraphics[height=\adeheight, width=\imwidth, align=b]{figures/supp_visualization_ade20k/gt/ADE_val_00001282_gt.png}
	\includegraphics[height=\adeheight, width=\imwidth, align=b]{figures/supp_visualization_ade20k/pred/ADE_val_00001282_learnable_pred_our_color.png}
	% \\
	\rotatebox{0}{\textcolor{white}{------}Image}
	\rotatebox{0}{\textcolor{white}{--------------------------------------}Ground Truth}
	\rotatebox{0}{\textcolor{white}{--------------------------------}Prediction}
	\caption{Qualitative results of semantic segmentation on  ADE20K {\tt val} split~\cite{zhou2019semantic}.
	}
% 	\vspace{-1em}
	\label{fig:supp_visualization_stage_2_ade20k}
\end{figure*}
\begin{figure*}[th]
	\def \imwidth {2.8cm}
    \def \vochight {2.0cm}
	\def \adeheight {2.0cm}
	\def \cityheight {1.5cm}
    \def \contexthight {2.0cm}
	\centering
	\includegraphics[height=\vochight, width=\imwidth, align=b]{figures/supp_visualization_mseg/voc2012/2007_000762.jpg}
	\includegraphics[height=\vochight, width=\imwidth, align=b]{figures/supp_visualization_mseg/voc2012/2007_000762_gt.png}
	\includegraphics[height=\vochight, width=\imwidth, align=b]{figures/supp_visualization_mseg/voc2012/2007_000762_learnable_pred.png}
	\includegraphics[height=\vochight, width=\imwidth, align=b]{figures/supp_visualization_mseg/voc2012/2007_001763.jpg}
	\includegraphics[height=\vochight, width=\imwidth, align=b]{figures/supp_visualization_mseg/voc2012/2007_001763_gt.png}
	\includegraphics[height=\vochight, width=\imwidth, align=b]{figures/supp_visualization_mseg/voc2012/2007_001763_learnable_pred.png}
	\\
  	\includegraphics[height=\contexthight, width=\imwidth, align=b]{figures/supp_visualization_mseg/pascal-context-60/2008_000423.jpg}
	\includegraphics[height=\contexthight, width=\imwidth, align=b]{figures/supp_visualization_mseg/pascal-context-60/2008_000423_gt.png}
	\includegraphics[height=\contexthight, width=\imwidth, align=b]{figures/supp_visualization_mseg/pascal-context-60/2008_000423_learnable_pred.png}
 	\includegraphics[height=\contexthight, width=\imwidth, align=b]{figures/supp_visualization_mseg/pascal-context-60/2008_000268.jpg}
	\includegraphics[height=\contexthight, width=\imwidth, align=b]{figures/supp_visualization_mseg/pascal-context-60/2008_000268_gt.png}
	\includegraphics[height=\contexthight, width=\imwidth, align=b]{figures/supp_visualization_mseg/pascal-context-60/2008_000268_learnable_pred.png}
	\\
	\includegraphics[height=\adeheight, width=\imwidth, align=b]{figures/supp_visualization_mseg/scannet-20/001600.jpg}
	\includegraphics[height=\adeheight, width=\imwidth, align=b]{figures/supp_visualization_mseg/scannet-20/001600_gt.png}
	\includegraphics[height=\adeheight, width=\imwidth, align=b]{figures/supp_visualization_mseg/scannet-20/001600_learnable_pred.png}
	\includegraphics[height=\adeheight, width=\imwidth, align=b]{figures/supp_visualization_mseg/scannet-20/001700.jpg}
	\includegraphics[height=\adeheight, width=\imwidth, align=b]{figures/supp_visualization_mseg/scannet-20/001700_gt.png}
	\includegraphics[height=\adeheight, width=\imwidth, align=b]{figures/supp_visualization_mseg/scannet-20/001700_learnable_pred.png}
	\\
 	\includegraphics[height=\cityheight, width=\imwidth, align=b]{figures/supp_visualization_mseg/camvid-11/0016E5_07961.png}
	\includegraphics[height=\cityheight, width=\imwidth, align=b]{figures/supp_visualization_mseg/camvid-11/0016E5_07961_gt.png}
	\includegraphics[height=\cityheight, width=\imwidth, align=b]{figures/supp_visualization_mseg/camvid-11/0016E5_07961_learnable_pred.png}
	\includegraphics[height=\cityheight, width=\imwidth, align=b]{figures/supp_visualization_mseg/camvid-11/0016E5_07995.png}
	\includegraphics[height=\cityheight, width=\imwidth, align=b]{figures/supp_visualization_mseg/camvid-11/0016E5_07995_gt.png}
	\includegraphics[height=\cityheight, width=\imwidth, align=b]{figures/supp_visualization_mseg/camvid-11/0016E5_07995_learnable_pred.png}
	\\
  	\includegraphics[height=\cityheight, width=\imwidth, align=b]{figures/supp_visualization_mseg/wilddash-19/cn0000_100000.png}
	\includegraphics[height=\cityheight, width=\imwidth, align=b]{figures/supp_visualization_mseg/wilddash-19/cn0000_100000_gt.png}
	\includegraphics[height=\cityheight, width=\imwidth, align=b]{figures/supp_visualization_mseg/wilddash-19/cn0000_100000_learnable_pred.png}
	\includegraphics[height=\cityheight, width=\imwidth, align=b]{figures/supp_visualization_mseg/wilddash-19/fi0005_100000.png}
	\includegraphics[height=\cityheight, width=\imwidth, align=b]{figures/supp_visualization_mseg/wilddash-19/fi0005_100000_gt.png}
	\includegraphics[height=\cityheight, width=\imwidth, align=b]{figures/supp_visualization_mseg/wilddash-19/fi0005_100000_learnable_pred.png} \\
	\includegraphics[height=\cityheight, width=\imwidth, align=b]{figures/supp_visualization_mseg/wilddash-19/si0001_100000.png}
	\includegraphics[height=\cityheight, width=\imwidth, align=b]{figures/supp_visualization_mseg/wilddash-19/si0001_100000_gt.png}
	\includegraphics[height=\cityheight, width=\imwidth, align=b]{figures/supp_visualization_mseg/wilddash-19/si0001_100000_learnable_pred.png}
 	\includegraphics[height=\cityheight, width=\imwidth, align=b]{figures/supp_visualization_mseg/wilddash-19/fi0008_100000.png}
	\includegraphics[height=\cityheight, width=\imwidth, align=b]{figures/supp_visualization_mseg/wilddash-19/fi0008_100000_gt.png}
	\includegraphics[height=\cityheight, width=\imwidth, align=b]{figures/supp_visualization_mseg/wilddash-19/fi0008_100000_learnable_pred.png}
	\rotatebox{0}{\textcolor{white}{---------}Image}
	\rotatebox{0}{\textcolor{white}{-------------}Ground Truth}
	\rotatebox{0}{\textcolor{white}{---------}Prediction}
	\rotatebox{0}{\textcolor{white}{---------------}Image}
	\rotatebox{0}{\textcolor{white}{-----------}Ground Truth}
	\rotatebox{0}{\textcolor{white}{-----------}Prediction\textcolor{white}{-------}}
	\caption{Qualitative results of semantic segmentation on MSeg {\tt test} datasets~\cite{lambert2020mseg}. From top to bottom: Pascal VOC~\cite{everingham2010pascal}, Pascal Context~\cite{mottaghi2014role}, ScanNet-20~\cite{dai2017scannet}, CamVid~\cite{brostow2009semantic} and WildDash~\cite{zendel2018wilddash} (the last two rows).}
	\label{fig:supp_visualization_stage_2_mseg_main}
\end{figure*}
\begin{figure*}[th]
	\def \imwidth {8.6cm}
    \def \vochight {2.5cm}
	\centering
	\includegraphics[height=\vochight, width=\imwidth, align=b]{figures/supp_visualization_mseg/kitti-19/000008_10.png} 
 	\includegraphics[height=\vochight, width=\imwidth, align=b]{figures/supp_visualization_mseg/kitti-19/000149_10.png} 
  \\
	\includegraphics[height=\vochight, width=\imwidth, align=b]{figures/supp_visualization_mseg/kitti-19/000008_10_gt.png} 
 	\includegraphics[height=\vochight, width=\imwidth, align=b]{figures/supp_visualization_mseg/kitti-19/000149_10_gt.png} 
 \\
	\includegraphics[height=\vochight, width=\imwidth, align=b]{figures/supp_visualization_mseg/kitti-19/000008_10_learnable_pred.png} 
	\includegraphics[height=\vochight, width=\imwidth, align=b]{figures/supp_visualization_mseg/kitti-19/000149_10_learnable_pred.png}
	\\

	\caption{Qualitative results of semantic segmentation on MSeg {\tt test} dataset~\cite{lambert2020mseg} (KITTI dataset~\cite{geiger2013vision}). The $1^{st}$ row is input image, the $2^{rd}$ row is Ground Truth, and the $3^{rd}$ row is prediction result.}
% 	\vspace{-1em}
	\label{fig:supp_visualization_stage_2_meg_kitti}
\end{figure*}
\begin{figure*}[htb]
	\begin{center}
		\includegraphics[width=12.86cm, height=22cm]{figures/color_ade_mda_f.pdf}
	\end{center}
	\caption{
Visualization of \texttt{maskige} for each category in ADE20K~\cite{zhou2019semantic} dataset under \emph{maximal distance assumption}.
	}
	\label{fig:supp_color_ade_mda}
\end{figure*}
\begin{figure*}[htb]
	\begin{center}
		\includegraphics[width=12.86cm, height=22cm]{figures/color_ade_opt_f.pdf}
	\end{center}
	\caption{
Visualization of \texttt{maskige} for each category in ADE20K~\cite{zhou2019semantic} dataset under gradient descent optimization.
	}
	\label{fig:supp_color_ade_opt}
\end{figure*}


\end{document}
