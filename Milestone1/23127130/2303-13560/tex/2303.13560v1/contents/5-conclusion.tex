% \vspace{-2mm}
\section{Conclusion and Limitation}
% \vspace{-2mm}

We propose \texttt{CoCa3D}, a novel collaborative camera-only 3D detection that approaches holistic 3D detection. The core idea is to introduce multi-agent collaboration to improve the detection ability of cameras. Meanwhile, the communication cost is optimized, and each agent carefully selects the spatially sparse yet critical depth messages to share. Extensive experiments covering both real-world and simulation scenarios, and multi-type agents (cars, drones, and infrastructures) show that \texttt{CoCa3D} not only achieves state-of-the-art perception-bandwidth trade-off, but overtakes LiDAR-based detectors with a sufficient number of collaborative agents on OPV2V+.

\noindent
\textbf{Limitation and future work.} 
It is expensive to collect a real-world multi-agent perception dataset. So far, DAIR-V2X is the sole public real-world dataset, which only has one vehicle and one roadside unit. This work mainly leverages simulation data to validate the proposed novel methods and sketch a promising research direction. We advocate more resources for real-world data collection.

\noindent
\textbf{Acknowledgment.}
This research is supported by the National Key R\&D Program of China under Grant 2021ZD0112801, NSFC under Grant 62171276 and the Science and Technology Commission of Shanghai Municipal under Grant 21511100900 and 22DZ2229005.
\clearpage