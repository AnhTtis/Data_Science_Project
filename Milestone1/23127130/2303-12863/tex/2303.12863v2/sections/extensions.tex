% \section{Extensions}

\section{Sparsity Extension}
% \subsection{Delay Affine Structure}

\label{sec:delay_affine}

% A more general structure is when dynamics are affine in nonzero delays,
\urg{May send to arxiv copy in case we run out of room.}
Dynamics $f(t, x_0, \ldots, x_r)$ have \textit{current-delay-affine} structure if they can be separated into the following form,
\begin{equation}
\label{eq:dynamics_aff_non}
    \dot{x}(t) = \textstyle\sum_{i=1}^r f_i(t, x(t), x(t - \tau_i))
\end{equation}
The special structure of \eqref{eq:dynamics_aff_non} may be used to express weak solutions with measures in fewer variables \cite{lasserre2006convergent}. 

A correlative sparsity graph such as in \cite{waki2006sums} may be realized with $n + 1$ vertices corresponding to the variables $(t, x_0, x_1, \ldots, x_r)$. An edge is drawn between vertices if they appear in the same (non-additive) function. The Lie derivative term associated with \eqref{eq:dynamics_aff_non} is,
\begin{equation*}
    \Lie_f v(t, x_0) = \partial_t v(t, x_0) + \textstyle\sum_{i=1}^r f_i(t, x_0, x_i) \cdot \nabla_{x_0} v(t, x_0).
% \end{equation}
\end{equation*}

Under this definition, the cliques of the correlative sparsity graph are $(t, x_0, x_i)$ for all $i = 1, \ldots, r$ each corresponding to the term $f_i(t, x_0, x_i) \cdot \nabla_{x_0}v(t, x_0)$.
Joint occupation measures $\bar{\mu}^i \in \Mp{[0, T] \times X^2}$ may be defined on each clique, such that the overlapping marginal $\pi^{t x_0}_\# \bar{\mu}^i$ is identical for all cliques.
The decomposed Liouville constraint for all $v(t, x) \in C^1([0, T] \times X)$ is,
\begin{align}
\label{eq:liou_sparse}
    \inp{v(t, x)}{\mu_p} &= v(0, x_h(0)) + \inp{\partial_t v(t, x_0)}{\bar{\mu}^1} \\
    &+ \textstyle \sum_{i=1}^r\inp{f_i(t, x_0, x_1) \cdot v(t, x_0)}{\bar{\mu}^i}.\nonumber
\end{align}

The sparse consistency constraints for $i=1, \ldots, r$ with test functions $\phi_i(t, x) \in C([0, T], \times X)$ are,
\begin{align}
     \inp{\phi_i(t, x_i)}{\bar{\mu}^i} &= \textstyle \int_{t=-\tau_i}^0 \phi_i(t, x_h(t+\tau_i)) dt \\
     &+  \inp{\phi_i(t+\tau_i,x)}{\textstyle \sum_{j=-i}^{r-i} \nu_j }.\nonumber
\end{align}

The zero-lag consistency constraint also pins the marginal overlaps between joint occupation measures $\forall \phi_0 \in C([0, T] \times X), i =1, \ldots, r$,
\begin{align}
    \inp{ \phi_0(t, x_0)}{\bar{\mu}^i} &= \inp{\phi_0(t, x)}{\textstyle\sum_{j=0}^{r} \nu_j} .
\end{align}

With the correlatively sparse decomposition, the number of variables has dropped from $[1+n(r+1)]$ in $\bar{\mu}$ to $[1+2n]$ in $\bar{\mu_i} \ \forall i = 1, \ldots, r$. The complexity of \acp{LMI} generated from these measures correspondingly decrease in a polynomial manner as the number of variables in the largest measure falls.
\urg{will need a demo showing the timing impact}
% \subsection{Uncertainty}
% \urg{Briefly mention time-dependent and time-independent uncertainty, analogous to \cite{miller2021uncertain}.This uncertainty is an instance of an adversarial optimal control problem. Bounding optimal control costs in time delay systems using occupation measures will be the subject of a sequel work.}