\section{Peak Estimation of DDEs}
\urg{ Revise and Excise this content}
% \section{Peak Estimation of Time-Delay Systems}

% Free-terminal time weak solutions may be employed to produce upper bounds on the peak estimation problem \eqref{eq:peak_delay_traj}.

\subsection{Measure Program}

Problem \eqref{eq:peak_delay_traj} may be upper-bounded by a measure \ac{LP} program to produce an objective $p^* \geq P^*$. This measure program is based on the multiple-history free-time weak solution $(\mu_0, \mu_T, \bar{\mu}, \{\nu_j\}_{j=-r}^r, \{\hat{\nu_i}\}_{i=1}^r)$ from Definition \ref{defn:weak_multiple_free} (with the convention that $\hat{\nu}_0 = 0$),

\begin{subequations}
\label{eq:peak_delay_meas}
\begin{align}
p^* = & \ \textrm{max} \quad \inp{p(x)}{\mu_p} \label{eq:peak_delay_meas_obj} \\
    & \mu_p = \delta_0 \otimes \mu_0 + \pi^{tx_0}_\# \Lie_f^\dagger \bar{\mu} \label{eq:peak_delay_meas_flow}\\
    & \inp{1}{\mu_0} = 1 \label{eq:peak_delay_meas_prob}\\
    & \pi^{t x_i}_\# \bar{\mu} + \hat{\nu}_i = (t + \tau_i, x)_\# \left( \textstyle\sum_{j=-i}^{r-i} \nu_{j} \right) & & \forall i = 0, \ldots, r \label{eq:peak_delay_meas_delay} \\
    & \pi^t_\# \left( \textstyle\sum_{i=1}^r \nu_{-i} \right) = \lambda_{[-\tau_r, 0]} \label{eq:peak_delay_meas_history} \\
    & \mu_0 \in \Mp{X_0} \label{eq:peak_delay_meas_init} \\
    & \mu_p \in \Mp{[0, T] \times X} \label{eq:peak_delay_meas_peak}\\
    & \bar{\mu} \in \Mp{[0, T] \times X^{r+1}} \label{eq:peak_delay_meas_occ} \\
    &\nu_{j} \in \Mp{\Omega_j} & &  \forall j = -r, \ldots, r \\
    & \hat{\nu}_i \in \Mp{[0, T] \times X} \label{eq:peak_delay_meas_complement}& &  \forall i = 1, \ldots, r.
\end{align}
\end{subequations}
 

Constraint \eqref{eq:peak_delay_meas_delay} is an equivalent (weak) expression to the free-time multiple-history consistency constraint in \eqref{eq:consistency_multi_free}. Shaping constraints may optionally be added as per Section \ref{sec:shaping} to describe valid trajectories in the function class $\mathcal{H}$.

For $r$ time-delays and $n$ states, there will be 1 measure with $(r+1)n + 1$ variables ($\mu$), $3r-1$ measures with $n+1$ variables (in order: components $\nu_{i}$, complements $\hat{\nu}_{i}$ from absolute continuity of components, and $\mu_p$), and 1 measure of size $n$ ($\mu_0$). LMI relaxations of program \eqref{eq:peak_delay_meas} will scale combinatorially with the number of delay terms $r$, and the bottleneck in optimization will be the joint occupation measure $\bar{\mu}$. 
 
\subsection{Function Program}
Dual variables must be introduced to formulate the Lagrangian of problem \eqref{eq:peak_delay_meas}. Let $v(t,x) \in C^1([0, T] \times X), \ \gamma \in \R, \  \phi_i(t, x_i) \in C([0, T] \times X) \ \forall i = 0, \ldots, r,$ and $h(t) \in C([-\tau_r, 0])$ be dual variables for constraints \eqref{eq:peak_delay_meas_flow}-\eqref{eq:peak_delay_meas_delay}. 
The Lagrangian of problem \eqref{eq:peak_delay_meas} is,

\begin{align}
    \scL &= \inp{p(x)}{\mu_p} + \inp{v(t,x_0)}{\delta_0 \otimes \mu_0 + \pi^{t x_0}_\# \Lie_f^\dagger \bar{\mu} - \mu_p} + \gamma(1 - \inp{1}{\mu_0})\\\label{eq:peak_lagrangian}
    & + \textstyle \sum_{i=0}^r \inp{\phi_i(t, x)}{\pi^{t x_i}_\# \bar{\mu} + \hat{\nu}_i -(t + \tau_i, x)_\# \left( \textstyle\sum_{j=-i}^{r-i} \nu_{j} \right)} + \inp{h(t)}{\lambda_{[-\tau_r, 0]} - \left(\textstyle\sum_{i=1}^r \nu_{-i} \right) }. \nonumber
\end{align}

The dual problem (function program) of \eqref{eq:peak_delay_meas} formed by taking a saddle point to $\scL$ is,
\begin{subequations}
\label{eq:peak_delay_cont}
\begin{align}
    d^* = & \ \min_{\gamma + \in \R} \gamma + \textstyle \int_{t=-\tau_r}^0 h(t) dt & & \\
    & \gamma \geq v(0, x)  & & \forall x \in  X_0 \label{eq:peak_delay_cont_init}\\
    & v(t, x) \geq p(x) & & \forall (t, x) \in [0, T] \times X \label{eq:peak_delay_cont_p} \\
    & 0 \geq \Lie_f v(t, x_0) + \textstyle \sum_{i=0}^{r} \phi_i(t, x_i)& & \forall (t, \cup_{i=0}^r x_i)\in [0, T] \times X^{r+1} \\
    & \phi_i(t, x) \leq 0 & & \forall (t, x) \in \Omega_{i}, \ i = 1, \ldots, r \\ 
    & \textstyle \sum_{i=j}^{r} \phi_i(t + \tau_i, x) \geq 0 & & \forall (t, x) \in \Omega_j, \ j = 0, \ldots, r \\
    & \textstyle\sum_{i=0}^{r-j} \phi_i(t + \tau_i, x)  + h(t)\geq 0 & & \forall (t, x) \in \Omega_{-j}, \ j = 1, \ldots, r \\
    & v(t,x) \in C^1([0, T]\times X) \label{eq:peak_delay_cont_v}& & \\
    &\phi_i(t, x) \in C([0, T] \times X) & &  \forall i=0, \ldots, r \label{eq:peak_delay_cont_phi} \\
    &h(t) \in C([-\tau_r, 0]).
\end{align}
\end{subequations}
The variable $v(t, x)$ is an auxiliary function that upper bounds the cost $p(x)$ over the area of interest $[0, T] \times X$. The $\phi_i$ terms are introduced by delay dynamics, and the $h(t)$ function is required to ensure the  sanctity of histories (Lebesgue marginal in time). Without time-delays, this problem is equal to the standard peak estimation measure formulation because the multipliers $\phi_i(t, x)$ will be set to $0$. 

Problems \eqref{eq:peak_delay_meas} and \eqref{eq:peak_delay_cont} achieve strong duality such that $p^* = d^*$ when the set $[-\tau_r, T] \times X$ is compact by Theorem 3.10 of \cite{anderson1987linear} \urg{write in appendix, send to Arxiv}. 
The measures in problem \eqref{eq:peak_delay_meas} form a free-time multiple-history weak solution and therefore have bounded mass when the set $[-\tau_r, T] \times X$ is compact. The constraints defining this weak solution are closed in the weak-$*$ topology, so the conditions for strong duality by Theorem 3.10 of \cite{anderson1987linear} are therefore satisfied.

