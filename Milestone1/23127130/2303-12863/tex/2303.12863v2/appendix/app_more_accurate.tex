\section{Improved Accuracy of Control Problems}
\label{app:more_accurate}

This appendix lays out methods to reduce the conservatism of \ac{DDE} \acp{OCP} from appendix \ref{app:ocp} by adding new infintie-dimensional nonnegativity constraints. 

All approaches discussed in this appendix may be applied to peak estimation, but are presented here in the simplified fixed-terminal-time single-history \ac{OCP} setting.

\subsection{Spatial Partitioning}

The constraints \eqref{eq:u_cont_term}-\eqref{eq:u_cont_f_end} must hold in support sets defined by $[0, T] \times X \times U$. Assume that there exists a decomposition of the state  spaces $X = \cup_k X_k$ such that $\forall k: \textrm{dim}(X_k) = n$ and  $ \forall k, k': \textrm{int}(X_k \cap X_{k'}) = \varnothing$ (cells $X_k$ are full-dimensional and their intersections are not full dimensional). Further assume a similar decomposition exists for the control set 
 $U = \cup_\ell U_k$.
 
Let $(v_k(t, x), \phi_k(t,x))$ be functions associated with each space $X^k$. A space-control partition of \eqref{eq:u_cont} is:
\begin{subequations}
\label{eq:u_cont_space}
\begin{align}
    d^* = & \ \sup \quad \sum_k \left(I_{X_k}(x_h(0)) v_k(0, x_h(0)) \right)+ \textstyle\int_{-\tau}^{0} \phi(t + \tau, x_h(t))  dt\label{eq:u_cont_obj_space}& \\
    & \forall x \in X_k: \nonumber \\
    & \qquad J_T(x) - v_k(T, x) \geq 0 \label{eq:u_cont_term_space}  \\
    & \forall (t, x_0, x_1 ,u) \in [0, T-\tau]  X_k \times X_{k'} \times U_\ell: \nonumber \\
    & \qquad \Lie_f v_k + J(t, x_0, u) -  \phi_{k'}(t, x_1)  +  \phi_k(t+\tau, x_0) \geq 0 \label{eq:u_cont_f_space}  \\
    & \forall (t, x_0, x_1 ,u) \in [T-\tau, T] \times X_k \times X_{k'} \times U_\ell: \nonumber \\        
    & \qquad  \Lie_f v_k + J(t, x_0, u) -  \phi_{k'}(t, x_1) \geq 0 \label{eq:u_cont_f_end_space} & &  \\
    & \forall (t, x) \in [0, T] \times (X_k \cap X_{k'}): \nonumber \\
    & \qquad v_k(t, x) = v_{k'}(t, x) \label{eq:u_cont_agreement_space} \\
    &\forall k: v_k \in C^1([0, T] \times X_k)  \label{eq:u_cont_v_space} \\
    &\forall k: \phi_k \in C([0, T] \times X_k)  \label{eq:u_cont_phi_space}.
\end{align}
\end{subequations}


The $v_k$ terms agree on boundary regions between state cells by \eqref{eq:u_cont_agreement_space}. The $\phi_k$ terms remain continuous (bounded measurable), but this partitioning has an impact when evaluating the finite-degree  \acp{SDP}.  
%The term $\phi(t, x)$ in the integral \eqref{eq:subvalue} is much more delicate, and I do not know if it can be incorporated into the spatial partition. Further details about this term will take place in the following temporal partition section.

\subsection{Temporal Partitioning}

We utilize the following lemma to provide conditions for temporal partitioning:
\begin{lem}
\label{lem:suff_int}
A sufficient condition for $\int_{t_0}^{t_1} g_1(t) dt \geq \int_{t_0}^{t_1} g_2(t) dt$ is that \begin{equation}
    \forall t \in [t_0, t_1]: g_1(t) \geq g_2(t).
\end{equation}
\end{lem}
Define the following time breaks (partition) arranged in sorted order as
\begin{equation}
    T_{break} = \{0, t_1, \ldots, t_{k-1}, t_k=T-\tau, t_{k+1}, \ldots, t_{k+\ell-1}, t_{k+\ell} = T\}.
\end{equation}

Let $[t_{b}, t_{b+1}]$ and $[t_{b-1}, t_{b}]$ be regions in $T_{break}$. The subvalue functionals from \eqref{eq:subvalue} defined in this region must satisfy 
\begin{subequations}
\label{eq:subvalue_break_decrease}
\begin{align}
    V_b(t_b, x, z(\cdot)) &\geq V_{b-1}(t_b, x, z(\cdot)) \\
    v_b(t_b, x) + \int_{t_b}^{\min(t_b+\tau, T)} \phi_b(s, z(s-\tau)) ds & \geq v_{b-1}(t_b, x) + \int_{t_b}^{\min(t_b+\tau, T)} \phi_{b-1}(s, z(s-\tau)) ds.
\end{align}
\end{subequations}

The sufficient condition in Lemma \ref{lem:suff_int} may be used to accomplish this relation in \eqref{eq:subvalue_break_decrease}, ensuring that the subvalue function will always decrease when traversing a time break:
\begin{subequations}
\label{eq:time_suff}
\begin{align}
    & v_b(t_b, x) \geq v_{b-1}(t_b, x) & & \forall x \in X\\
    & \phi_b(t, x) \geq \phi_{b-1}(t, x) & & \forall (t, x) \in [t_b, \min(t_b + \tau, T)] \times X.
\end{align}
\end{subequations}

The resultant time-partitioned \ac{LP} is,
\begin{subequations}
\label{eq:u_cont_time}
\begin{align}
    d^* = & \ \sup \quad v(0, x_h(0)) +\textstyle \textstyle\int_{-\tau}^{0} \phi_{k+\ell}(t + \tau, x_h(t)) dt\label{eq:u_cont_obj_time}& \\
    & \forall x \in X:  \nonumber \\
    &\qquad J_T(x) - v_{k+\ell}(T, x) \geq 0 \label{eq:u_cont_term_time} \\
    & \forall (t, x_0, x_1 ,u) \in [t_{k'}, t_{k'+1}] \times X^2 \times U, k'=0..k-1: \nonumber \\
    & \qquad  \Lie_f v_{k'} + J(t, x_0, u) -  \phi_{k'}(t, x_1)  +  \phi_{k'}(t+\tau, x_0) \geq 0 \label{eq:u_cont_f_time} \\
    &  \forall (t, x_0, x_1 ,u) \in [t_{k'}, t_{k'+1}] \times X^2 \times U, k'=k..k+\ell: \nonumber \\
    & \qquad \Lie_f v_{k'} + J(t, x_0, u) -  \phi_{k'}(t, x_1) \geq 0 \label{eq:u_cont_f_end_time} \\
    & \forall x \in X,  k'=1..k+\ell-1: \nonumber \\
    & \qquad  v_{k'}(t_{k'}, x) \leq v_{k'+1}(t_{k'}, x)  \\
    & \forall (t, x) \in [t_{k'}, \min(t_{k'} + \tau, T)] \times X, k'=1..k+\ell-1 \nonumber\\
    &  \qquad  \phi_{k'}(t, x) \leq \phi_{k'+1}(t, x)\\
    & \forall k'=0..k+\ell: \\
    &\qquad v_{k'} \in C^1([t_{k'}, t_{k'+1}] \times X) & & \label{eq:u_cont_v_time} \\
    &\qquad \phi_{k'} \in C([t_{k'}, \min(t_{k'+1}+\tau, T)] \times X).  \label{eq:u_cont_phi_time}
\end{align}
\end{subequations}


\subsection{Double Integral Functionals}

The subvalue functional in \eqref{eq:subvalue} has a single integral term for each delay. 
Some Lyapunov-Krasovskii or Barrier methods for \ac{DDE}  analysis employ double integrals, such as the following functional for a single delay $\tau$ \cite{papachristodoulou2005tutorial},
\begin{align}
    V(t, z, w) &= v(t, z) + \int_{t}^{\min(t+\tau, T)} \phi_i(s, w(s-\tau-t))ds \nonumber \\
    &+ \int_{t}^{\min(t+\tau, T)}\int_{-\tau}^{0}\psi(s, q, w(s-\tau-t), w(q))dq ds. \label{eq:subvalue_double}
\end{align}

The time derivative of \eqref{eq:subvalue_double} in the time span $t \in [0, T-\tau)$ is
\begin{align}
        \dot{V}(t,z,w) &= \Lie_f v(t, z) + \phi(t+\tau, w(0)) - \phi(t, w(-\tau))\\ &+\textstyle\int_{-\tau}^{0}\psi(t+\tau, q, w(0), w(q))dq \nonumber\\
    &-\textstyle\int_{-\tau}^{0}\psi(t, q, w(-\tau), w(q))dq, \label{eq:double_span_start}
\end{align}
and between $t \in (T-\tau, 0]$ is
\begin{align}
    \dot{V}(t,z,w) &= \Lie_f v(t, z)  - \phi(t, w(-\tau)) -\textstyle\int_{-\tau}^{0}\psi(t, q, w(-\tau), w(q))dq. \label{eq:double_span_end}
\end{align}

The derivative $\dot{V}$ has a discontinuity present at $t=T-\tau$, just as described in Section \eqref{sec:continuity_subvalue}.


A sufficient condition for the inequality \eqref{eq:hjb_lie} to be fulfilled is that the following functions associated with \eqref{eq:double_span_start} and  \eqref{eq:double_span_end}  (moving all terms under the $dq$ integral) are nonnegative:
\begin{align}
&\forall t\in [0, T-\tau], (x_0, x_1, \tilde{x}) \in X^3, q \in [-\tau, 0]: \nonumber \\
    &\qquad \tau^{-1}\left(\Lie_f v(t, z) + J(t, x_0, u) + \phi(t+\tau, x_0) - \phi(t, x_1) \right) \nonumber \\
    &\qquad + \psi(t+\tau, q, x_0, \tilde{x}) - \psi(t, q, x_0, \tilde{x}) \geq 0  \label{eq:double_nonneg_start} \\
&\forall t\in [T-\tau, T], (x_0, x_1, \tilde{x}) \in X^3, q : 
 \nonumber \\
    &\qquad \tau^{-1} \left(\Lie_f v(t, z) + J(t, x_0, u)  - \phi(t, x_1)\right) - \psi(t, q, x_0, \tilde{x}) \geq 0. \label{eq:double_nonneg_end}
\end{align}

Lemma \ref{lem:suff_int} is utilized to enforce nonnegativity of the integral terms in \eqref{eq:double_span_start} and \eqref{eq:double_span_end}. The $\tau^{-1}$ scale factor arises from placing a $q$-independent term (such as $J(t, x_0, u)$) inside the integral. The variable $q \in [-\tau, 0]$ is the integrated (swept) time, and $\tilde{x} \in X$ abstracts out the swept state $w(q)$. The dual formulation of constraints \eqref{eq:double_nonneg_start} and \eqref{eq:double_nonneg_end} involve occupation measures $\bar{\mu}_0 \in \Mp{[0, T-\tau] \times X^3 \times U}$ and $\bar{\mu}_1 \in \Mp{[T-\tau, T] \times X^3 \times U}$. This construction may be generalized to \acp{DDE} with $r$ delays by adding a double-integral term for each delay.
