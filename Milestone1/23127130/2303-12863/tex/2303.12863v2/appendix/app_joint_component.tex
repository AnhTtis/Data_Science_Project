\section{Joint+Component Measure}
\label{app:joint_component}

This appendix details an alternate notion of \ac{MV}-solutions for \acp{DDE}. Solving peak estimation problems through these methods will return more conservative but quicker-executing 
 programs (computationally) as compared to the results in Section \ref{sec:peak_lp}.

\subsection{Measure Program}
The \ac{MV}-solution involves a joint occupation measure $\bmu$ and component measures $\omega_0, \omega_1$:
\begin{subequations}
\label{eq:weak_solution_jc}
    \begin{align}
        &\textrm{History}  & \mu_h &\in \Mp{H_0} \label{eq:weak_solution_jc_history} \\
       & \textrm{Initial} & \mu_0 &\in \Mp{X_0} \\
       &\textrm{Peak}  &\mu_p &\in \Mp{[0, T] \times X} \\
       &\textrm{Time-Slack} & \nu &\in \Mp{[0, T] \times X} \label{eq:weak_solution_jc_slack} \\
        &\textrm{Joint Occupation}  & \bmu &\in \Mp{[0, T] \times X^2} \label{eq:weak_solution_jc_occ} \\        
        &\textrm{Component Start }  & \omega_0 &\in \Mp{[0, T-\tau] \times X} \\
        &\textrm{Component End }  & \omega_1 &\in \Mp{[T-\tau, T] \times X}.
    \end{align}
\end{subequations}

The peak estimation \ac{LP} for the Joint+Component framework is
\begin{subequations}
\label{eq:peak_delay_jc_meas}
    \begin{align}
        p^* = & \ \sup \quad \inp{p}{\mu_p} \label{eq:peak_delay_jc_obj} \\
    & \inp{1}{\mu_0} = 1 \label{eq:peak_delay_jc_prob}\\    
    & \pi^{t}_\# \mu_h = \lambda_{[-\tau, 0]} \label{eq:peak_delay_jc_hist}\\   
    & \mu_p = \delta_0 \otimes\mu_0 + \pi^{t x_0}_\# \Lie_f^\dagger \bmu \label{eq:peak_delay_jc_flow}\\   
    & \pi^{t x_0}_\# \bmu = \omega_0 + \omega_1 \label{eq:peak_delay_jc_cons_start}\\ 
    & \pi^{t x_1}_\# \bmu + \nu = S^\tau_\#(\mu_h + \omega_0) \label{eq:peak_delay_jc_cons_end}\\ 
    & \textrm{Measure Definitions from  \eqref{eq:weak_solution_jc}.} \label{eq:peak_delay_jc_def}
    \end{align}
\end{subequations}

The history-validity and Liouville constraints in \eqref{eq:peak_delay_jc_meas} are the same as in \eqref{eq:peak_delay_meas} under the relation $\bmu = \bmu_0 + \bmu_1$. The consistency constraint in the Joint+Component formulation is split up into the pair \eqref{eq:peak_delay_jc_cons_start}-\eqref{eq:peak_delay_jc_cons_end}.

\begin{thm} 
Program \eqref{eq:peak_delay_jc_meas} returns an upper bound on \eqref{eq:peak_delay_traj}.
\end{thm}
\begin{proof}
Let $x_h \in \mathcal{H}$ be a history that generates the trajectory $x(t \mid x_h)$, and let $t^* \in [0, T]$ be a stopping time. 

Just as in Theorem \ref{thm:delay_upper_bound}, measures can be picked as $\mu_0 = \delta_{x = x_h(0^+ \mid x_h)}$, $\mu_p = \delta_{t=t^*} \otimes \delta_{x=x(t^* \mid x_h)}$, $\mu_h$ as the occupation measure of $x_\xi(t)$ in the times $[-\tau, 0]$, and $\bmu$ as the occupation measure of $z(t) = (x(t \mid x_h), x(t-\tau \mid x_h))$ in the times $[0, t^*]$.

When $t^* \in [0, T-\tau]$, then $\omega_0$ is the occupation measure of $x(t \mid x_h)$ in times $[0, t^*]$, $\omega_1$ is the zero measure, and $\nu$ is the occupation measure of $x(t-\tau \mid x_h)$ in times $[t^*, t^*+\tau]$. When $t^* \in (T-\tau, T]$, then $\omega_0$ is the occupation measure of $x(t \mid x_h)$  in the  times $[0, T-\tau]$, $\omega_1$ is the occupation measure of $x(t \mid x_h)$ in the times $[T-\tau, t^*$, and $\nu$ is the occupation measure of $x(t-\tau \mid x_h)$ in the times $[T-\tau, T]$.

All measures inside \eqref{eq:weak_solution_jc} have been defined for a valid trajectory, proving that \eqref{eq:peak_delay_jc_meas} upper-bounds \eqref{eq:peak_delay_traj}.
\end{proof}

\begin{thm}
    The Joint+Component measure program \eqref{eq:peak_delay_jc_meas} is also an upper bound on \eqref{eq:peak_delay_meas}.
\end{thm}
\begin{proof}
    Let $(\mu_h, \mu_0, \mu_p, \nu, \bmu_0, \bmu_1)$ be a feasible set of measures for the constraints of \eqref{eq:peak_delay_meas}.

    After performing the following definitions,
    \begin{align}
        \bmu &= \bmu_0 + \bmu_1 & \omega_0 &= \pi^{t x_0} \bmu_0 & \omega_1 &= \pi^{t x_0} \bmu_1,
    \end{align}
    the measures $(\mu_h, \mu_0, \mu_p, \nu, \bmu, \omega_0, \omega_1)$ are feasible solutions for the constraints of \eqref{eq:peak_delay_jc_meas}.    
\end{proof}


Note how the Joint+Component \ac{MV}-solution involves only one measure involving $(t, x_0, x_1)$ together ($\bmu$ in \eqref{eq:weak_solution_jc_slack}), while the solution in \eqref{eq:weak_solution} has two measures ($\bmu_0, \bmu_1$). Application of the moment-\ac{SOS} hierarchy towards solving problems in \eqref{eq:weak_solution_jc_slack} result in only one Gram matrix of maximal size $\binom{1+2n+d}{d}$.

\subsection{Function Program}

The gap between \eqref{eq:peak_delay_jc_meas} and \eqref{eq:peak_delay_meas} can most easily be observed by examining the dual program of  \eqref{eq:peak_delay_jc_meas}:

\begin{subequations}
\label{eq:peak_delay_jc_cont}
\begin{align}
    d^* = & \ \inf_{\gamma \in \R} \gamma + \textstyle \int_{t=-\tau}^0 \xi(t) dt & & \\
    & \gamma \geq v(0, x)  & & \forall x \in  X_0 \label{eq:peak_delay_jc_cont_init}\\
    & v(t, x) \geq p(x) & & \forall (t, x) \in [0, T] \times X \label{eq:peak_delay_jc_cont_p} \\
    & \xi(t) + \phi_1(t+\tau, x) & & \forall (t, x) \in H_0 \\
    & 0 \geq \Lie_f v(t, x_0) + \phi_0(t, x_0) + \phi_1(t, x_1) & & \forall (t, x_0, x_1)\in [0, T] \times X^{2} \label{eq:jc_omega_Lie} \\
    & \phi_1(t, x) \leq 0 & & \forall (t, x) \in [0, T] \\ 
    & \phi_0(t, x) + \phi_1(t+\tau, x) \geq 0 & & \forall (t, x) \in [0, T-\tau] \times X \label{eq:jc_omega_0} \\
    & \phi_0(t, x)\geq 0 & & \forall (t, x) \in [T-\tau, T] \times X \label{eq:jc_omega_1} \\
    & v(t,x) \in C^1([0, T]\times X) \label{eq:peak_delay_jc_cont_v}& & \\
    &\phi_0(t, x), \phi_1(t, x) \in C([0, T] \times X)  \label{eq:peak_delay_jc_cont_phi} \\
    &\xi(t) \in C([-\tau, 0]).
\end{align}
\end{subequations}

Adding together \eqref{eq:jc_omega_Lie} and \eqref{eq:jc_omega_0} yields constraint \eqref{eq:peak_delay_cont_lie0} in $[0, T-\tau] \times X^2$. Similarly, the addition of \eqref{eq:jc_omega_Lie} and \eqref{eq:jc_omega_1} forms constraint \eqref{eq:peak_delay_cont_lie1}. The dual formulation in \eqref{eq:peak_delay_jc_cont} enforces nonnegativity of addends inside whole-terms of \eqref{eq:peak_delay_meas}. The constraints of \eqref{eq:peak_delay_jc_cont} are stricter than of \eqref{eq:peak_delay_cont}, resulting in a lowered infimum/upper bound on peak value.

\subsection{Joint+Component Example}

Table \eqref{tab:jc_compare_flow} compares moment-\ac{SOS} \acp{SDP} associated to programs \eqref{eq:peak_delay_meas} and \eqref{eq:peak_delay_jc_meas} for the delayed Flow example in Section \ref{ex:delay_flow}.

\begin{table}[h]
\caption{\label{tab:jc_compare_flow}Comparison of \eqref{eq:peak_delay_meas} and \eqref{eq:peak_delay_jc_meas} \ac{SDP} bounds for the delayed Flow system}
\centering
\begin{tabular}{llllll}
degree $d$               & 1    & 2      & 3      & 4    & 5  \\ \hline
Joint+Component \eqref{eq:peak_delay_jc_meas} & 1.25 & 1.223  & 1.1937 & 1.1751 & 1.1636\\
Standard \eqref{eq:peak_delay_meas}  & 1.25 & 1.2183 & 1.1913 & 1.1727 & 1.1630 
\end{tabular}
\end{table}

\begin{table}[h]
\caption{\label{tab:jc_compare_flow_time}Time (seconds) to obtain \ac{SDP} bounds in Table \ref{tab:jc_compare_flow}}
\centering
\begin{tabular}{llllll}
degree $d$  & 1    & 2      & 3      & 4    & 5  \\ \hline
Joint+Component \eqref{eq:peak_delay_jc_meas} & 0.782 & 0.991  & 5.271 & 31.885& 336.509\\
Standard \eqref{eq:peak_delay_meas}  & 0.937 & 1.190 & 9.508 & 105.777 & 552.496
\end{tabular}
\end{table}