\section{Peak Measure Program}
\label{sec:peak_lp}
This section will formulate a measure-valued \ac{LP} which upper-bounds Problem \eqref{eq:peak_delay_traj} in objective.



\subsection{Assumptions}

The following assumptions will be imposed on the peak estimation Problem \eqref{eq:peak_delay_traj}:
\begin{enumerate}
    \item[A1] The set $[-\tau, T] \times X$ is compact with $\tau<T$;
    \item[A2] The function $f$ is Lipschitz inside $[0, T] \times X^2$;
    \item[A3] Any trajectory $x(\cdot \mid x_h)$ with $x_h \in \hs$ such that $x(t\mid x_h) \not\in X$ for some $t \in [0, T]$ also satisfies $x(t'\mid x_h) \not\in X$ for all $t' \geq t$; 
    \item[A4] The objective $p$ is continuous;
    \item[A5] The history class $\hs$ is graph-constrained by $H_0 \subset [-\tau, 0] \times X$.
        % Trajectories satisfy $x(t \mid x_h) \in X$ for all times  \\$t \in [0, T]$ and histories $x_h \in \mathcal{H}$.
\end{enumerate}

In the case where $\tau > T$, the delayed state $t \mapsto x(t-\tau)$ is fully specified in time $[0, T]$ without requiring dynamics information, and \eqref{eq:peak_delay_traj} reduces to a peak estimation problem over \acp{ODE}.
All tracked histories in $\hs$ are bounded due to assumption A1 (since the range $X$ is compact). The nonreturn assumption A3 ensures that a trajectory cannot leave and then return to $X$ to produce a lower value of $p$, given that the occupation-measure-based techniques used in this paper can only track trajectories while they are in $X$.
% Assumption A3 is a requirement of nonreturn that allows for a trajectory to escape $X$ while still leaving the solution to \eqref{eq:peak_delay_traj} well defined (when restricting to trajectories evolving in $X$). \MK{Milan: I am not sure why we need A3.}

\subsection{Measure-Valued Solution}




The initial set $X_0$ is the $t=0^+$ slice of $H_0$. Equation  \eqref{eq:weak_solution} describes the measures $(\mu_h, \mu_0, \mu_p, \bmu_0, \bmu_1, \nu)$ that will be used to form a free-terminal-time \ac{MV}-solution to the \ac{DDE} \eqref{eq:delay_dynamics} with multiple histories (in $\hs$):
\begin{subequations}
\label{eq:weak_solution}
    \begin{align}
        &\textrm{History}  & \mu_h &\in \Mp{H_0} \label{eq:weak_solution_history} \\
       & \textrm{Initial} & \mu_0 &\in \Mp{X_0} \\
       &\textrm{Peak}  &\mu_p &\in \Mp{[0, T] \times X} \\
        &\textrm{Occupation Start }  & \bmu_0 &\in \Mp{[0, T-\tau] \times X^2} \\        
        &\textrm{Occupation End }  & \bmu_1 &\in \Mp{[T-\tau, T] \times X^2} \\
        &\textrm{Time-Slack} & \nu &\in \Mp{[0, T] \times X} \label{eq:weak_solution_slack}
    \end{align}
\end{subequations}

The joint (relaxed) occupation measure $\bmu \in \Mp{[0, T] \times X^2}$ is constructed from the sum $\bmu = \bmu_0 + \bmu_1$. An \ac{MV} solution to the \ac{DDE} in \eqref{eq:delay_dynamics} is a set of measures from \eqref{eq:weak_solution} that satisfy three types of constraints: History-Validity, Liouville, Consistency.

\subsubsection{History-Validity}
\label{sec:history_validity}
The first History-Validity constraint is that $\mu_0$ should be a probability distribution over the initial state condition (at $t=0$). The second is that the history measure $\mu_h$ should represent an averaged occupation measure of histories that are defined between $[-\tau, 0]$, which implies that the $t$-marginal of $\mu_h$ should be Lebesgue-distributed. The two History-Validity constraints are,
\begin{align}
    \inp{1}{\mu_0} &= 1, &   \pi^{t}_\# \mu_h &= \lambda_{[-\tau, 0]}. \label{eq:History-Validity}
\end{align}

\subsubsection{Liouville}
The true occupation measure $(t,x_0,x_1) \mapsto \bmu(t, x_0, x_1)$ has a time $t$, a current state $x_0 = x(t \mid x_h)$, and an external input $x_1 \in X$ with $x_1(t) = x(t-\tau \mid x_h)$. 
Use of the Liouville equation in \eqref{eq:liou}  applied to the joint occupation measure $\bmu = \bmu_0 + \bmu_1$ leads to
\begin{align}
    & \mu_p = \delta_0 \otimes\mu_0 + \pi^{t x_0}_\# \Lie_f^\dagger (\bmu_0 + \bmu_1). \label{eq:peak_delay_liou}
\end{align}

\subsubsection{Consistency}
The $x_1$ input of $f$ from the Liouville equation \eqref{eq:peak_delay_liou} is not arbitrary; it should be equal to a time-delayed  $x_1(t) = x_0(t-\tau)$. This requirement will be imposed by a Consistency constraint. 
% \begin{lem}
% \label{lem:consistency_fixed}
% Let $x(t)$ be a solution to \eqref{eq:delay_dynamics} for some history $x_h$. The following two integrals are equal for all  $\phi \in C([0, T] \times X)$
% \begin{align}
% \label{eq:change_limits}
%     \int_{0}^T \phi(t, x(t-\tau))dt = \left(\int_{-\tau}^{0} + \int_{0}^{T-\tau} \right)\phi(t+\tau, x(t)) dt.
% \end{align}
% \end{lem}

\begin{lem}
\label{lem:consistency_free}
    % Let $x(t)$ be a solution to \eqref{eq:delay_dynamics} (same setting as Lemma \ref{lem:consistency_fixed}) 
    Let $x(\cdot)$ be a solution to \eqref{eq:delay_dynamics} for some history $x_h$ with an initial time of $0$ and a stopping time of $t^* \in [0, T]$. Then the following two integrals are equal for all  $\phi \in C([0, T] \times X)$:
    \begin{align}
    \label{eq:change_limits_free}
        &\left(\int_{0}^{t^*} + \int_{t^*}^{\min(T, t^*+\tau)}\right) \phi(t, x(t-\tau))dt  \nonumber\\
        &\qquad = \left(\int_{-\tau}^{0} + \int_{0}^{\min(t^*, T-\tau)} \right)\phi(t'+\tau, x(t)) dt'.
    \end{align}
\end{lem}
\begin{proof}
    This follows from 
a change of variable with $t' \leftarrow t - \tau$.
\end{proof}
% The free-terminal-time expression in \eqref{eq:change_limits_free} adds a new integral symbol in the left-hand-side expression as compared to \eqref{eq:change_limits}. 
Equation \eqref{eq:change_limits_free} inspires a consistency constraint for the free-terminal-time \ac{MV}-solution in \eqref{eq:weak_solution}. The left-hand-side of \eqref{eq:change_limits_free} may be generalized to 
\begin{equation}
\label{eq:consistency_lhs}
    \inp{\phi(t, x_1)}{\bmu_0(t, x_0, x_1) + \bmu_1(t, x_0, x_1)} + \inp{\phi(t, x)}{\nu(t, x)},
\end{equation}
in which $\bmu_0$ is supported in times $[0, \min(t^*, T-\tau)]$,  $\bmu_1$ is supported in times $[T-\tau, t^*]$ if $t^* > T-\tau$, and the slack measure $\nu$ implements the $[t^* , \min(T, t^*+\tau)]$ limits. The right-hand-side of \eqref{eq:change_limits_free} may be interpreted as \begin{equation}
\label{eq:consistency_rhs}
    \inp{\phi(t+\tau, x)}{\mu_h(t, x)} + \inp{\phi(t+\tau, x_0)}{\bmu_0(t, x_0, x_1)}.
\end{equation}

Define $S^\tau$  as the shift operator $S^\tau \phi(t, x) = \phi(t+\tau, x)$. With an abuse of notation, the pushforward operation $S^\tau_\#$ applied to a measure (such as $\mu_h$) will have the expression
\begin{align}
     \inp{\phi}{S^\tau_\#\mu_h} = \inp{S^\tau \phi}{\mu_h} =  \inp{\phi(t+\tau, x)}{\mu_h(t, x)}.
\end{align}
The Consistency constraint inspired by Lemma \ref{lem:consistency_free} is
\begin{equation}
\label{eq:consistency_free}
    \pi^{t x_1}_\# (\bmu_0 + \bmu_1) + \nu = S^\tau_\#(\mu_h + \pi^{t x_0}_\# \bmu_0). 
\end{equation}

\begin{rmk}
    Equation \eqref{eq:consistency_free} may also be written as $\pi^{t x_1}_\# (\bmu_0 + \bmu_1) \leq S^\tau_\#(\mu_h + \pi^{t x_0}_\# \bmu_0)$. The associated slack measure is $\nu$.
\end{rmk}
% The measure $S^\tau_\#(\mu_h + \pi^{t x_0}_\# \bmu_0)$ dominates $\pi^{t x_1}_\# (\bmu_0 + \bmu_1)$, and the slack (orthogonal complement) from its Jordan decomposition is $\nu$. \MK{Milan: Again, this is not really a Jordan decomposition here.}

% Let $\tilde{\mu} \in \Mp{[0, T] \times X}$ be an occupation measure of a trajectory in \eqref{eq:delay_dynamics} between times $[0, T]$. 

% \begin{align*}
%       % \inp{\phi_i(t, x_i)}{\tilde{\mu}} &= \textstyle\int_{0}^{T} \phi_i(t, x(t  - \tau_i\mid x_h)) dt    & \\
%         &= \textstyle\int_{-\tau_i}^{T - \tau_i} \phi_i(t + \tau_i, x(t \mid x_h)) dt  \\
%          &= \textstyle\left(\int_{-\tau_i}^{0} + \int_{0}^{T - \tau_i} \right) \phi_i(t + \tau_i, x(t \mid x_h)) dt.
% \end{align*}

\subsection{Measure Program}

An infinite-dimensional \ac{LP} in terms of the measures from  \eqref{eq:weak_solution} to upper-bound Problem \eqref{eq:peak_delay_traj} is,
\begin{subequations}
\label{eq:peak_delay_meas}
    \begin{align}
        p^* = & \ \sup \quad \inp{p}{\mu_p} \label{eq:peak_delay_obj} \\
    & \inp{1}{\mu_0} = 1 \label{eq:peak_delay_prob}\\    
    & \pi^{t}_\# \mu_h = \lambda_{[-\tau, 0]} \label{eq:peak_delay_hist}\\   
    & \mu_p = \delta_0 \otimes\mu_0 + \pi^{t x_0}_\# \Lie_f^\dagger (\bmu_0 + \bmu_1) \label{eq:peak_delay_flow}\\        
    & \pi^{t x_1}_\# (\bmu_0 + \bmu_1) + \nu = S^\tau_\#(\mu_h + \pi^{t x_0}_\# \bmu_0) \label{eq:peak_delay_cons}\\ 
    & \textrm{Measure Definitions from  \eqref{eq:weak_solution}.} \label{eq:peak_delay_def}
    \end{align}
\end{subequations}
\begin{rmk}
    Membership in the history class $\hs$ is imposed  by the History-Validity constraint \eqref{eq:peak_delay_hist} and through support of $\mu_h$ in \eqref{eq:weak_solution_history}.
\end{rmk}
% \MK{Milan: I would mention that the history is imposed through the support of $\mu_h$.}

\begin{defn} 
\label{defn:mv_solution}
An \ac{MV}-solution to the \ac{DDE} \eqref{eq:delay_dynamics} with free-terminal-time and histories in $\hs$ is a tuple of measures that satisfy \eqref{eq:peak_delay_prob}-\eqref{eq:peak_delay_def} and \eqref{eq:weak_solution_history}-\eqref{eq:weak_solution_slack}.
\end{defn}


\begin{thm}
\label{thm:delay_upper_bound}
Under assumptions A1-A5, \eqref{eq:peak_delay_meas} will upper bound \eqref{eq:peak_delay_traj} with $p^* \geq P^*$ when $\hs$ is graph-constrained.
\end{thm}
\begin{proof}
This proof will proceed by demonstrating that every $(t^*, x_h)$ candidate from \eqref{eq:peak_delay_traj} may be expressed by a unique \ac{MV}-solution from Defn. \ref{defn:mv_solution}. The history measure $\mu_h$ is the $[-\tau, 0]$ occupation measure of $x(t)$, and the initial measure $\mu_0$ is the Dirac-delta $\delta_{x_h(0^+)}$. The peak measure $\mu_p$ is the Dirac-delta $\delta_{t=t^*} \otimes \delta_{x = x(t^* \mid x_h)}$. 
The relaxed occupation measures $(\bmu_0, \bmu_1, \nu)$ will now be considered. For convenience, define $z(t) = (t, x(t \mid x_h), x(t-\tau \mid x_h))$ as the delay embedding of the trajectory $x(t \mid x_h)$. In the case where $t^* \in [0, T-\tau]$, then $\bmu_0$ is the $[0, t^*]$ occupation measure of $z(t)$, $\bmu_1$ is the zero measure, and $\nu$ is the $[t^*, t^*+\tau]$ occupation measure of $(t, x(t-\tau \mid x_h))$. Alternatively when $t^* \in (T-\tau, T]$, $\bmu_0$ is the $[0, T-\tau]$ occupation measure of  $z(t)$, $\bmu_1$ is the $[T-\tau, t^*]$ occupation measure of $z(t)$, and $\nu$ is the $[t^*, T]$ occupation measure of $(t, x(t-\tau \mid x_h))$.
All of the measures in \eqref{eq:weak_solution} have been defined for each input $(t^*, x_h)$, which proves that $p^*\geq P^*$. 
\end{proof}

% \urg{Arxiv extension: strong duality goes here}

% \MK{Milan: To make it into a proper proof, you should argue that the measures constructed in that way will satisfy the constraints of \eqref{eq:peak_delay_meas}, at least the less obvious ones like 16d and 16f. Alternatively, we can leave proof details to a journal paper.}