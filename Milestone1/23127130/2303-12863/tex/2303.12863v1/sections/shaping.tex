
\subsection{Shaping Constraints}
\label{sec:shaping}
\urg{possibly put this on Arxiv}.
The weak solutions \eqref{eq:weak_multi_fixed} and \eqref{eq:weak_multi_free} assume that the history $x_h(t)$ can vary arbitrarily in time within the support set $H_0$.
% Structure may 
Some applications involve structure in the function class $\mathcal{H}$, such as requiring that the histories in $\mathcal{H}$ are constant in time between $t \in [-\tau_r, 0]$. Examples of these constant histories for the system in \eqref{eq:time_delay_fig} staring within the black box ($H_0$) are plotted in Figure \ref{fig:delay_shape_const}. 

\begin{figure}[ht]
    \centering
    \includegraphics[width=\linewidth]{fig/shaping_constant_history.png}
    \caption{Constant histories in the black box}
    \label{fig:delay_shape_const}
\end{figure}


These types of structure in histories may be realized by adding constraints to the weak derivatives of $\mu_h$. One example is enforcing that $x_h(t)$ is constant on $(-\tau_r, 0)$ by,
\begin{align}
    \inp{v(0, x)}{\mu_0} &= \inp{\partial_t v(t, x)}{\textstyle\sum_{i=1}^r\nu_{-i}} +\inp{v(-\tau, x)}{\mu_0}& \forall v \in C([-\tau_r, 0]).
    \label{eq:weak_shaping_open}
\end{align}
A Liouville equation with occupation measure $\mu_h$ may be used to force constant histories on the closed interval $[-\tau_r, 0]$,
\begin{align}
\inp{v(0, x)-v(-\tau_r, x)}{\mu_0}&= \inp{\partial_t v(t, x)}{\textstyle\sum_{i=1}^r\nu_{-i}}  & \forall v \in C([-\tau_r, 0]).
\end{align}

% Content about Lipschitz constraints and continuity, can't figure it out right now.
% Trajectories in $\mathcal{$
% A Lipschitz constraint with Lipschitz constant $L$ in each coordinate $x_k$ 
% and requiring that 
The histories $x_h(t)$ may be filled with lines by setting all second spatial weak derivatives to zero,
\begin{align}
    0 = \inp{\partial_{x_k, x_k'} v(t, x)}{\textstyle\sum_{i=1}^{r} \nu_{-i}}   \qquad \qquad \forall 1 \leq k \leq k' \leq n, \forall v \in C([-\tau_r, 0]).
\end{align}

% Continuity of histories by $-C \leq \partial_t \norm{x_h(t)}_\infty \leq C$ may be realized by the weak derivative shaping constraints,
% \begin{align}
%     -C \leq \inp{\partial_t v(t, x_k)}{\mu_h}   \qquad \qquad \forall  k=1, \ldots,  n, \forall v \in C([-\tau_r, 0] \times ).
% \end{align}


% Lipschitz constraints 



% \subsection{Multiple History Example}

