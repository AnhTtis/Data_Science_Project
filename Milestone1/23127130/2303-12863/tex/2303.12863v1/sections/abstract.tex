\begin{abstract}
This work proposes a method to compute the maximum value obtained by a state function along trajectories of \iac{DDE}. An example of this task is finding the maximum number of infected people in an epidemic model with a nonzero incubation period. The variables of this peak estimation problem include the stopping time and the original history (restricted to a class of admissible histories). The original nonconvex \ac{DDE} peak estimation problem is approximated by an infinite-dimensional \ac{LP} in occupation measures, inspired by existing measure-based methods in peak estimation and optimal control.
This \ac{LP} is approximated from above by a sequence of \acp{SDP} through the moment-\ac{SOS} hierarchy. Effectiveness of this scheme in providing peak estimates for \acp{DDE} is demonstrated with provided examples.
\acresetall
\end{abstract}

% This work proposes a method to compute the maximum value obtained by a state function along trajectories of a Delay Differential Equation (DDE). An example of this task includes finding the maximum number of infected people in an epidemic model with a nonzero incubation period. The variables of this peak estimation problem include the stopping time and the original history (restricted to a class of admissible histories). The original nonconvex DDE peak estimation problem is approximated by an infinite-dimensional Linear Program (LP) in occupation measures, inspired by existing measure-based methods in peak estimation and optimal control. This LP is approximated from above by a sequence of Semidefinite Programs through the moment-Sum-of-Squares hierarchy. Effectiveness of this scheme in providing peak estimates for DDEs is demonstrated with provided examples.
