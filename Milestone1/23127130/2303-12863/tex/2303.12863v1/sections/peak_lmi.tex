\section{Peak Moment Program}
\label{sec:moment}
This section will briefly review the moment-\ac{SOS} hierarchy \cite{lasserre2009moments} in order to approximate-from-above Program \eqref{eq:peak_delay_meas} by a sequence of finite-dimensional \acp{SDP}.
\subsection{Review of Moment-SOS Hierarchy}
% \urg{TODO: fill this in}

Let $\mu\in \Mp{X}$ be a measure, and let $\alpha \in \N^n$ be a multi-index. The $\alpha$-moment of $\mu$ is the pairing $\bm_\alpha = \inp{x^\alpha}{\mu}$. The moment sequence $\bm = \{\bm_\alpha\}_{\alpha \in \N^n}$ is the infinite collection of moments of $\mu$. A unique (Riesz) linear functional $L_\bm$ exists operating on each polynomials $p =\sum_{\alpha \in \mathcal{J}} p_\alpha x^\alpha \in \R[x]$  as $L_\bm(p) = \sum_{\alpha \in \mathcal{J}} p_\alpha \bm_\alpha$ for a finite index set $\mathcal{J} \subset \N^n$.

A set is \ac{BSA} if it is defined by a finite number of polynomial inequality constraints, such as by $\K = \{x \in \R^n \mid g_k(x) \geq 0: \ k = 1 \ldots N_c\} \subseteq \R^n$. The measure $\mu$ is supported on $\K$ if $\mu \in \Mp{\K}$.
Given a polynomial $g = \sum_{\gamma \in \mathcal{J}} g_\gamma x^\gamma$, the localizing matrix $\M[g \bm]$ induced by the  constraint $g(x) \geq 0$ with respect to the moment sequence $\bm$ is the infinite-dimensional matrix indexed by $\alpha, \beta \in \N^n$ as $\M[g \bm]_{\alpha, \beta} = L_{\bm}(x^{\alpha+\beta} g) = \sum_{\gamma \in \R^n} g_{\gamma} \bm_{\alpha+\beta+\gamma}$. The moment matrix $\M[ \bm]$ is the localizing matrix associated with $g = 1$. The matrix $\M[\K \bm]$ is the block-diagonal matrix comprised of $\M[\bm]$ and $\M[g_k \bm]$ for $k=1 \ldots N_c$.

Let $\{\tilde{\bm}_{\alpha}\}_{\alpha \in \N^n}$ be a sequence of real numbers. If there exists some measure $\tilde{\mu} \in \Mp{\K}$ such that $\forall \alpha \in \N^n: \ \inp{x^\alpha}{\mu} = \tilde{\bm}_\alpha$ then $\tilde{\mu}$ is a representing measure for $\tilde{\bm}$, and $\tilde{\bm}$ is a moment-sequence for $\tilde{\mu}$. Such a representing measure (if it exists) could be nonunique. The stronger condition that there is a unique representing measure for $\tilde{\bm}$ is called moment determinacy. 
A necessary condition for $\tilde{\bm}$ to have a representing measure is that the block-diagonal matrix $\M[\tilde \bm]$ is \ac{PSD}. This necessary condition is also sufficient if $\K$ satisfies an \textit{Archimedean} condition (stronger than compactness, equivalent after a ball constraint $R - \norm{x}_2^2 \geq 0$ is added to $\K$ for sufficiently large $R>0$ if $\K$ is compact). 
In general we will call $\tilde \bm$ a \textit{pseudo-moment} sequence.

The order-$d$ truncation of $\M[\K \bm]$ (for $d \in \N$ and expressed as $\M_d[\K \bm]$) keeps entries of degree $\leq 2d$, and preserves the top-corner of each matrix in the block-diagonal. 
The moment matrix $\M_d[\bm]$ is a \ac{PSD} matrix of size $\binom{n+d}{d}$ assuming a monomial basis for $x$ is employed. 
The size of each truncated localizing matrix $\M_d[g_k \bm]$ is $\binom{n+d-\ceil{d_k/2}}{d-\ceil{d_k/2}}$, where $d_k = \deg g_k$. 
The moment-\ac{SOS} hierarchy is the process of increasing the degree $d \rightarrow \infty$ when forming moment programs associated to measure \acp{LP}.

\subsection{Moment Program}

Additional assumptions are required in order to approximate \eqref{eq:peak_delay_meas} using the moment-\ac{SOS} hierarchy:
\begin{itemize}
    \item[A6] The sets $H_0$, $X_0$, and $X$ are Archimedean \ac{BSA} sets.
    \item[A7] Both $p$ and $f$ are polynomials.
\end{itemize}


Let the measures $(\mu_h, \mu_0, \mu_p, \bmu_0, \bmu_1, \nu)$ have associated pseudo-moment sequences $(\bm^h, \bm^0, \bm^p, \bar{\bm}^0, \bar{\bm}^1, \bm^\nu)$ respectively. Let $\alpha \in \N^n$ and $\beta \in \N$ be multi-indices that define monomial test functions $x_0^\alpha t^\beta$. For each multi-index 
tuple $(\alpha, \beta)$, the operator $\textrm{Liou}_{\alpha \beta}(\bm^0, \bm^p, \bar{\bm}^0, \bar{\bm}^1)$  may be derived from the linear relations induced by the Liouville equation \eqref{eq:peak_delay_flow} (in which $\delta_{\beta0}=1$ is a Kronecker delta):  
 % \begin{align}
 % 0 &= L_{\bm^0}(x^\alpha) \delta_{\beta 0} + L_{\bar{\bm}^0}(\Lie( x_0^\alpha t^\beta))+ L_{\bar{\bm}^1}(\Lie( x_0^\alpha t^\beta)) - L_{\bm^\tau}(x^\alpha t^\beta).
 %     \end{align}
 \begin{align}
 0 &= \inp{x^\alpha}{\mu_0} \delta_{\beta 0} + \inp{\Lie( x_0^\alpha t^\beta)}{\bmu^0 + \bmu^1} - \inp{x^\alpha t^\beta}{\mu_\tau}.
     \end{align}

Similarly, the operator     $\textrm{Cons}_{\alpha \beta}(\bm^h, \bm^\nu, \bar{\bm}^0, \bar{\bm}^1,)$ may be derived from the consistency constraint \eqref{eq:peak_delay_cons} by
\begin{align}
    0 = & \inp{x_1^\alpha t^\beta}{\bmu^0 + \bmu^1} + \inp{x^\alpha t^\beta}{\nu} - \inp{x^\alpha(t+\tau)^\beta}{\mu_h} \\
    &  -\inp{x_0^\alpha(t+\tau)^\beta}{\bmu^0}. \nonumber
\end{align}

Given a degree $d \in \N$, the dynamics degree $\tilde{d} \geq d$  may be defined as $\tilde{d}=d + \floor{\deg f /2}.$

\begin{prob}
Program \eqref{eq:peak_delay_meas} is upper-bounded by the following order-$d$ \ac{LMI} in pseudo-moments:
\begin{subequations}
\label{eq:peak_delay_lmi}
\begin{align}
    p^*_{d} = & \max \quad  L_{\bm^p}(p)  \label{eq:peak_delay_lmi_obj} \\
        & \bm^0_0 = 1 \label{eq:peak_delay_lmi_prob}\\
    &  \forall (\alpha, \beta) \in \N^{n+1}_{\leq 2d}: \nonumber \\
        & \qquad \bm^h_{\beta} = \textstyle \int_{-\tau}^0 t^\beta dt = -(-\tau)^{\beta+1}/(\beta+1) \label{eq:peak_delay_lmi_leb}  \\
&\qquad \textrm{Liou}_{\alpha \beta}(\bm^0, \bm^p, \bar{\bm}^0, \bar{\bm}^1) = 0  \label{eq:peak_delay_lmi_flow} \\
    &\qquad \textrm{Cons}_{\alpha \beta}(\bm^h, \bm^\nu, \bar{\bm}^0, \bar{\bm}^1,)= 0 \label{eq:peak_delay_lmi_cons}   \\
    & \M_d((X_0) \bm^0), \ \M_{\tilde{d}}((H_0)\bm^h) \succeq 0   \label{eq:peak_delay_lmi_supp_beg} \\
    &\M_d(([0, T] \times X)\bm^p) \succeq 0 \\
    &\M_{\Tilde{d}}(([0, T-\tau] \times X^2)\bar{\bm}^0) \succeq 0 \\
    &\M_{\Tilde{d}}(([T-\tau, T] \times X^2)\bar{\bm}^1) \succeq 0 \\
    &\M_{\tilde{d}}(([0, T] \times X)\bm^\nu) \succeq 0. \label{eq:peak_delay_lmi_supp_end} 
\end{align}
\end{subequations}
\end{prob}
The objective \eqref{eq:peak_delay_lmi_obj} is the pseudo-moment version of $\inp{p}{\mu_p}$. 
Constraints \eqref{eq:peak_delay_lmi_leb} and \eqref{eq:peak_delay_lmi_prob} are History-Validity constraints from \eqref{eq:History-Validity} when applied to the pseudo-moments $(\bm^\nu, \bm^0)$. Constraints \eqref{eq:peak_delay_lmi_flow} and \eqref{eq:peak_delay_lmi_cons} are the Liouville and Consistency constraints respectively. Constraints \eqref{eq:peak_delay_lmi_supp_beg}-\eqref{eq:peak_delay_lmi_supp_end} are  support constraints necessary for the pseudo-moments to have representing measures.

% The following boundedness result is required to ensureconvergence
Boundedness of all moments of measures in \eqref{eq:weak_solution} is required to obtain convergence of \eqref{eq:peak_delay_lmi} to \eqref{eq:peak_delay_meas} as $d \rightarrow \infty$.
\begin{lem}
\label{lem:moment_bound}
All measures from \eqref{eq:weak_solution}  in an \ac{MV}-solution (Defn. \ref{defn:mv_solution}) are bounded under assumptions A1-A7.
\end{lem}

\begin{proof}
Boundedness of a measure's mass and support is a  sufficient condition that all of the measure's moments are bounded. Assumption A1 ensures compactness, with the requirement from Defn. \ref{defn:graph_constrained} that $H_0 \subseteq [-\tau, X]$ and $X_0 \subseteq X$. The remainder of this proof will involve finding upper bounds on the masses of all measures in \eqref{eq:weak_solution}.

The initial measure $\mu_0$ has a mass of 1, and the history measure $\mu_h$ has a mass of $\tau$ by the History-Validity constraints \eqref{eq:peak_delay_prob} and \eqref{eq:peak_delay_hist}.
Substitution of the test function $v(t, x) = 1$ in the Liouville \eqref{eq:peak_delay_flow} leads to $\inp{1}{\mu_p} = \inp{1}{\mu_0} = 1$. Since $T$ is finite, the moment $\inp{t}{\mu_p} \leq \inp{1}{\mu_p} \; (\sup_{t \in [0, T]} t) = T$ is also finite. Use of the test function $v(t, x) = t$ into the Liouville \eqref{eq:peak_delay_flow}  yields $\inp{t}{\mu_p} = \inp{1}{\bmu_0 +\bmu_1} \leq T$. 
Because $\bmu_0$ and $\bmu_1$ are both nonnegative Borel measures, it holds that $\inp{1}{\bmu_0} \leq T$ and $\inp{1}{\bmu_1} \leq T$. The final constraint involves substitution of $\phi(t, x) = 1$ into the Consistency \eqref{eq:peak_delay_cons}, resulting in
\begin{align}
\label{eq:consistency_substitute}
\inp{1}{\bmu_0 + \bmu_1} + \inp{1}{\nu} &= \inp{1}{\mu_h} + \inp{1}{\bmu_0} \\
\inp{1}{\nu} &= \inp{1}{\mu_h} - \inp{1}{\bmu_1} = \tau - \inp{1}{\bmu_1}. \nonumber
\end{align}
Given that $\bmu_1$ and $\nu$ are nonnegative Borel measures and cannot have negative masses, the mass $\inp{1}{\nu}$ is constrained within $[0, \tau]$. 
All masses are demonstrated to be finite, thus proving boundedness.
\end{proof}

\begin{rmk}
Neglecting the History-Validity constraint \eqref{eq:peak_delay_hist} allows for $\mu_h$ in \eqref{eq:consistency_substitute} to have infinite mass, violating the boundedness principle.
\end{rmk}


\begin{thm} The optima in \eqref{eq:peak_delay_lmi} will converge as $\lim_{d \rightarrow \infty} p^*_d = p^*$ to \eqref{eq:peak_delay_meas} under assumptions A1-A6.\end{thm}
\begin{proof}
    This follows from Corollary 8 of \cite{tacchi2022convergence} under the boundedness condition in Lemma \ref{lem:moment_bound}.
\end{proof}

\begin{rmk}
    Assumption $A6$ can be generalized to cases where the sets ($H_0$, $X_0$, $X$) are the unions of \ac{BSA} sets. As an example, consider $H_0 = H_0^1 \cup H_0^2$ in which $\pi^t H_0^1 = [-\tau, -\tilde{\tau}]$ and $\pi^t H_0^2= [-\tilde{\tau}, 0]$ for some $\tilde{\tau} \in (0, \tau)$. Then the pseudo-moments $\bm^h = \bm^h_1 + \bm^h_2$ can be implicitly constructed from $\M_d((H_0^1)\bm_1^h), \;\M_d((H_0^2)\bm_2^h) \succeq 0$.
\end{rmk}

\subsection{Computational Complexity}

% \urg{Document the sizes of the Gram matrices}

Table \ref{tab:moment_size} lists the size of the  order-$d$ \ac{PSD} moment matrices associated with the pseudo-moment sequences $(\bm^h, \bm^0, \bm^p, \bar{\bm}^0, \bar{\bm}^1, \bm^\nu)$.

\begin{table}[h]
    \centering
    \caption{Size of Moment Matrices in \ac{LMI} \eqref{eq:peak_delay_lmi}}
    \begin{tabular}{rccc}
         Matrix: &$\M_d{(\bm^0)}$ & $\M_{\tilde{d}}{(\bm^p)}$ & $\M_d{(\bm^h)}$  \\
         Size: & $\binom{n+d}{d}$ &$\binom{n+1+d}{d}$ & $\binom{n+1+\tilde{d}}{\tilde{d}}$ \\ \\
         Matrix: &$\M_d{(\bar{\bm}^0)}$ & $\M_{\tilde{d}}{(\bar{\bm}^1)}$ & $\M_d{(\bm^{\nu})}$  \\
         Size: & $\binom{2n+1+\tilde{d}}{\tilde{d}}$ &$\binom{2n+1+\tilde{d}}{\tilde{d}}$ & $\binom{n+1+\tilde{d}}{\tilde{d}}$ \\
    \end{tabular}
    \label{tab:moment_size}
\end{table}
%
The largest size written in Table \ref{tab:moment_size} is $\binom{2n+1+\tilde{d}}{\tilde{d}}$, which occurs with the pseudo-moment sequences $(\bar{\bm}^0, \bar{\bm}^1)$ associated to the two joint occupation measures $(\bmu_0, \bmu_1)$. Equality constraints between entries of the moment matrices must be added to convert the \ac{LMI} into \iac{SDP} for use in symmetric-cone Interior Point Methods. The per-iteration complexity of solving an \ac{SDP}  derived from an order-$d$ \ac{LMI} involved in the moment-\ac{SOS} hierarchy  scales as $O(n^{6d})$ \cite{lasserre2009moments} with $n$. In the case of \ac{LMI} \eqref{eq:peak_delay_lmi}, the computational complexity of solving \eqref{eq:peak_delay_lmi} will scale approximately as $(2n+1)^{6 \tilde{d}}$  (based on $\bar{\bm}^0, \bar{\bm}^1$).
