\section{Numerical Examples}
\label{sec:delay_examples}
All experiments were developed in MATLAB 2021a, and code is available at \url{https://github.com/Jarmill/timedelay}. Dependencies include Gloptipoly \cite{henrion2003gloptipoly}, YALMIP \cite{Lofberg2004}, and Mosek \cite{mosek92} in order to formulate and solve moment-\ac{SOS} \acp{LMI} and \acp{SDP}. 

In this section, a notational convention where $(x_1, x_2)$ correspond to coordinates of $x \in X$ will be used. All sampled histories in visualizations are piecewise-constant inside $H_0$ with 10 randomly-spaced jumps between $[-\tau, 0]$.

% \urg{I will need to write a more comprehensive sampler to generate trajectories}

% \subsection{Peak Estimation}

% \section{Examples of Peak Estimation for Time-Delay Systems}

\label{sec:delay_delay_traj_analysis_peak_example}

% \urg{Load in the peak estimation figures from the BrainPOP presentation.}

\subsection{Epidemic Model}
This section provides an example of \iac{MV}-solution and peak estimation given a single history in a compartmental epidemic model.
Many diseases have incubation periods during which there is a delay between initial infection and infectious potential. In the current COVID-19 pandemic, this incubation period appears to be between 2-14 days, with a median of 5 days \cite{lauer2020incubation}.  The epidemic  dynamics with time delays are
\begin{subequations}
\label{eq:sir_delay}
\begin{align}
    S'(t) &= -\beta S(t) I(t) \\
    I'(t) &= \beta S(t-\tau) I(t-\tau) - \gamma I(t) \\
    R'(t) &= \gamma I(t)
\end{align}
\end{subequations}
There exists also exists a `latent' state $L'(t) = \beta S(t) I(t) - \beta  S(t - \tau) I (t - \tau)$ such that $S + I + R + L = 1$. The setting discussed in this section is $\beta = 0.4, \ \gamma = 0.1, \  \ T = 30$. 

Figures \ref{fig:sir_01} and \ref{fig:sir_02} display simulations of this epidemic model as $\tau$ changes under a constant state history with $R=0$. The black curve in Figures \ref{fig:sir_01} and \ref{fig:sir_02} is the plot of $I(t)$ at $\tau=0$. As the incubation period $\tau$, the time $t^*$ at which the peak is achieved is delayed (moves rightwards) in a monotonically increasing manner. The other colored curves in each plot have delays $\tau \in 1..9$.
In Figure \ref{fig:sir_01} with $I_h = 0.1$, the peak infected population decreases as the delay $\tau$ increases. Conversely in Figure \ref{fig:sir_02} with $I_h = 0.2$, the peak infected population increases as the delay increases.


\begin{figure}[ht]
     \centering
     \begin{subfigure}[b]{0.48\linewidth}
         \centering
         \includegraphics[width=\linewidth]{fig/sir_peak_fall.pdf}
         \caption{\label{fig:sir_01} $I_h = 0.1$, peak decreases}
         
     \end{subfigure}
     \;
     \begin{subfigure}[b]{0.48\linewidth}
         \centering
         \includegraphics[width=\linewidth]{fig/sir_peak_rise.pdf}
         \caption{\label{fig:sir_02} $I_h = 0.2$, peak increases}
     \end{subfigure}
      \caption{\label{fig:sir_peak} Peak infected population vs. time delay}
\end{figure}

For the peak estimation example, a  constant history is assumed with  an initial infection rate of $I_h = 0.2$ and an incubation period of $\tau = 9$, forming the initial history $S(t) = 1-I_h, I(t) = I_h, R(t) = 0$ between $t \in [-9, 0]$.


For numerical purposes the dynamics are scaled such that $\tilde{t} \in [0, 1]$ with an effective delay of $\tilde{\tau} = \tau / T = 0.3$.
Only the $x = (S, I)$ subsystem is considered to form the state set $X = \{S \geq 0, I \geq 0, S+I \leq 1\}$, and the joint occupation measures $(\bar{\mu}_0, \bar{\mu}_1)$ have variables  $(t, S_0, I_0, S_1, I_1)$. 

% Given a monomial $t^\beta S^{\alpha_1} I^{\alpha_2}$, the LMI estimated moment is $\hat{\mathbf{m}}_{\alpha \beta}: \inp{t^\beta S^{\alpha_1} I^{\alpha_2}}{\bar{\mu}}$.
% The reference moments $\mathbf{m}_{\alpha \beta} = \int_{\tilde{t}=0}^1 (\tilde{t})^\beta S(T \tilde{t})^{\alpha_1} I(T \tilde{t})^{\alpha_2} d\tilde{t}$ were calculated by Simpson's rule on 1000 equally spaced points in $\tilde{t} \in [0, 1]$.
% Figure \ref{fig:sir_weak} shows errors $\abs{\mathbf{m}_{\alpha \beta} - \hat{\mathbf{m}}_{\alpha \beta}}$ in LMI moment approximations to the trajectory from times $\tilde{t} \in [0, 1]$, indexed as the monomial powers increase $(\alpha, \beta)$ in a graded lexicographical order w.r.t $(\tilde{T}, S, I)$. 
% The largest PSD block at order-3 has size 126, and at order-4 the largest PSD block has size 252.

% \begin{figure}[h]
% \centering
%      \begin{subfigure}[b]{0.48\linewidth}
%          \centering
%          \includegraphics[width=\linewidth]{fig/sir_3_mom_approx.pdf}
%          \caption{Order-3 moments, 2-norm$= 1.3062\times 10^{-4}$. }
%          \label{fig:sir_weak_3}
%      \end{subfigure}
%      \;
%      \begin{subfigure}[b]{0.48\linewidth}
%          \centering
%          \includegraphics[width=\linewidth]{fig/sir_4_mom_approx.pdf}
%          \caption{Order-4 moments, 2-norm $= 4.2534 \times 10^{-5}$}
%          \label{fig:sir_weak_4}
%      \end{subfigure}
%       \caption{\label{fig:sir_weak} Error in moment approximation}
% \end{figure}

Peak estimation is employed to bound the maximum infection rate over the course of the epidemic. This peak estimation program maximizes $\inp{I}{\mu_p}$ under the constraint that $(\mu_p, \bar{\mu}, \{\nu_0, \nu_1\}, \hat{\nu}_1)$ is a free-time \ac{MV}-solution of dynamics \eqref{eq:sir_delay}.

Figure \ref{fig:sir_peak_recovery} displays the output of peak estimation, where the order-3 LMI relaxation bounds the maximal infection rate at 56.89\%. The moment matrix $\M_3[y_p]$ is approximately rank-1 (second largest eigenvalue of $\M_3[y_p] = 2.448\times 10^{-5}$), and the extracted optimum from $\M_3[y_p]$ by Algorithm 1 of \cite{miller2020recovery} is $(S^*, I^*) = (0.0561, 0.5689)$ occurring at $t^*=15.636$ days.
\begin{figure}[h]
    \centering
    \includegraphics[width=0.6\linewidth]{fig/sir_3_peak_traj_recover.pdf}
    \caption{\label{fig:sir_peak_recovery} SIR peak estimation and recovery at order 3}
    
\end{figure}
\subsection{Delayed Flow System}
\label{ex:delay_flow}
A time-delayed version of the Flow system from \cite{prajna2004safety} is
\begin{equation}
\label{eq:flow_delay}
    \dot{x}(t) = \begin{bmatrix}x_2(t) \\ -x_1(t-\tau) - x_2(t) + x_1(t)^3/3
        \end{bmatrix}.
\end{equation}

Figure \ref{fig:flow_delay_comparision} plots the delayed Flow system \eqref{eq:flow_delay} without lag ($\tau=0$ in blue) and with a lag ($\tau=0.75$ in orange) starting from the constant initial history $x_h(t) = (1.5, 0), \ \forall t \in [-\tau, 0]$ (black circle). 
% The system with lag spirals 
    \begin{figure}[ht]
        \centering
        \includegraphics[width=0.5\linewidth]{fig/flow_constant_compare.pdf}
        \caption{\label{fig:flow_delay_comparision}Comparison of delayed Flow systems \eqref{eq:flow_delay} with lags $\tau=0$ and $\tau= 0.75$ in times $t \in [0, 20]$ }
    \end{figure}
    


The time-zero set of allowable histories is $X_0 = \{x \in \R^2 \mid  \ (x_1-1.5)^2 + x_2^2 \leq 0.4^2\}$. The history class $\hs$ will be the set of functions $x_h \in PC([-\tau, 0])$ whose graphs $(t, x(t))$ are contained within  the cylinder $H_0 = [-0.75, 0] \times X_0$. No further requirements of continuity are posed on histories in $\hs$.     
The considered peak estimation aims to find the minimum value of $x_2$ (maximize $p(x)=-x_2)$
for trajectories following \eqref{eq:flow_delay} starting from $H_0$, within the state set 
$X = [-1.25, 2.5] \times [-1.25, 1.5]$ and time horizon $T=5$. The first five bounds on the maximum value of $-x_2$ by solving \eqref{eq:peak_delay_lmi} are $p^*_{1:5}=[1.25, 1.2183, 1.1913, 1.1727, 1.1630]$.

Figure \ref{fig:flow_delay_peak} plots trajectories and peak information associated with this example. The black circle is the initial set $X_0$. The initial histories inside $X_0$ are plotted in grey. These sampled histories are piecewise constant with 10 uniformly spaced jumps (moving to a new point uniformly sampled in $X_0$) within $[-0.75, 0]$. The cyan curves are the \ac{DDE} trajectories of \eqref{eq:flow_delay} starting from the grey histories. The red dotted line is the $p^*_5$ bound on the minimum vertical coordinate of a point on any trajectory starting from $\hs$ up to $T=5$.
    \begin{figure}[ht]
        \centering
        \includegraphics[width=0.5\linewidth]{fig/peak_flow_5_T_5.png}
        \caption{\label{fig:flow_delay_peak} Minimize $x_2$ on the delayed Flow system \eqref{eq:flow_delay}}
    \end{figure}
    

Distance estimation is performed on the Flow system \ref{eq:flow_delay} with an $L_2$ metric, a time horizon of $T=8$, arbitrarily varying histories in $H_0$, a time horizon of $\tau=0.5$, and a half-circle unsafe set  $X_u = \{x \mid 0.5^2 \geq (x_1+0.5)^2 + (x_2+1)^2, (1.5+x_1+x_2)$. The recovered distance estimates up to degree 4 from \ac{SDP} relaxations of \eqref{eq:dist_delay_meas} are $c^*_{1:4} = [1.1897 \times 10^{-4}, \  4.0420 \times 10^{-4}, \ 0.1572, \ 0.1820]$. 
Figure \ref{fig:flow_delay_dist} plots the set $X_u$ in red along with its $c^*_4 = 0.1820$ certified distance contour.
    \begin{figure}[ht]
        \centering
        \includegraphics[width=0.5\linewidth]{fig/distance_dde_tmax_8_order_4.png}
        \caption{\label{fig:flow_delay_dist} Minimize $c(x; X_u)$ on the delayed Flow system \eqref{eq:flow_delay}}
    \end{figure}

    
    % Future work will involve more examples of peak estimation in time-delay systems.
    
    \subsection{Delayed Time-Varying System}
    
    This example involves peak estimation of \iac{DDE} version of the time-varying Example 2.1 of \cite{fantuzzi2020bounding} with
    \begin{equation}
\label{eq:time_var_delay}
    \dot{x}(t) = \begin{bmatrix}x_2(t) t - 0.1 x_1(t) - x_1(t-\tau) x_2(t-\tau)  \\ -x_1(t) t - x_2(t) + x_1(t) x_1(t-\tau)
        \end{bmatrix}.
\end{equation}

The considered support parameters are $\tau = 0.75, \ T = 5, $ and $X = [-1.25, 1.25] \times [-0.75, 1.25]$.
The time-zero set is the disk $X_0 = \{x\in \R^2 \mid (x_1+0.75)^2 + x_2^2 \leq 0.3^2\}$. The only restriction on allowable histories $\hs$ is that their graphs are contained in the history set $H_0 = [-0.75, 0] \times X_0$. 

Solving the \ac{SDP} associated with the \ac{LMI} \eqref{eq:peak_delay_lmi} to maximize the peak function $p = x_1$ yields the sequence of five bounds $p^*_{1:5} = [1.25, 1.25, 1.1978, 0.8543, 0.718264618]$. 
Figure \ref{fig:time_var_peak} plots system trajectories and the $p^*_5$ bound on $x_1$ using the same visual convention as Figure \ref{fig:flow_delay_peak} (black circle $X_0$, grey histories $x_h(t)$, cyan trajectories $x(t \mid x_h)$, red dotted line $x_1 = p^*_5$).

    \begin{figure}[ht]
        \centering
        \includegraphics[width=0.5\linewidth]{fig/peak_time_var_5.png}
        \caption{\label{fig:time_var_peak} Maximize $x_1$ on the delayed time-varying \eqref{eq:time_var_delay}}
    \end{figure}

Figure \ref{fig:time_var_peak_3d} plots the corresponding trajectory and bound information in 3d $(t, x_1, x_2)$. The black circles denote the boundary of $H_0$. The history structure inside $H_0$ between times $[-0.75, 1]$ is clearly visible in grey.

        \begin{figure}[ht]
        \centering
        \includegraphics[width=0.6\linewidth]{fig/time_var_3d_hist_circ.png}
        \caption{\label{fig:time_var_peak_3d} A 3d plot of \eqref{eq:time_var_delay} and its  $x_1$ bound }
    \end{figure}

The peak estimation of $p = x_2$ under the same system parameters leads to the sequence of five bounds $p^*_{1:5} = [1.25, 1.25, 0.9557, 0.9138, 0.9112].$ 