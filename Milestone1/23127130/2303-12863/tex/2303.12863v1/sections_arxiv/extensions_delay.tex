\section{Extensions}
\label{sec:extensions_delay}

This section discusses several extensions to the \ac{DDE} peak estimation framework.

\subsection{Shaping Constraints}

% \subsection{Shaping Constraints}
\label{sec:delay_shaping}
% \urg{possibly put this on Arxiv}.
Assumption A5 imposes that the class $\mathcal{H}$ is graph-constrained. 
% Structure may 
Some applications involve further structure in the function class $\mathcal{H}$, such as requiring that the histories in $\mathcal{H}$ are constant in time between $t \in [-\tau_r, 0]$. Examples of these constant histories for the system in \eqref{eq:time_delay_fig} staring within the black box ($H_0$) are plotted in Figure \ref{fig:delay_shape_const}. 

\begin{figure}[ht]
    \centering
    \includegraphics[width=0.5\linewidth]{fig/shaping_constant_history.png}
    \caption{Constant histories in the black box}
    \label{fig:delay_shape_const}
\end{figure}


These types of structure in histories may be realized by adding constraints to $\mu_h$. A method to ensure that the histories in $\mu_h$ are constant in time between $t \in [-\tau, 0]$ is by requiring $\mu_h$ to be the occupation measure of the system $\dot{x}=0$ through a Liouville equation
\begin{align}
    \inp{v(0, x)}{\mu_0} &= \inp{\partial_t v(t, x)}{\mu_h} +\inp{v(-\tau, x)}{\mu_0}& \forall v \in C([-\tau_r, 0]).
    \label{eq:weak_shaping_open}
\end{align}

\subsection{Multiple Time Delays}

\Iac{DDE} with multiple time-delays $0 < \tau_1 < \tau_2 < \ldots < \tau_r$ for $(r, \tau_r)$ finite and a history $x_h \in PC([-\tau_r, 0], X)$ is
\begin{align}
    \dot{x}(t) &= f(t, x(t), x(t-\tau_1), \ldots, x(t - \tau_r)) \label{eq:delay_multiple}\\
    x(t) &= x_h(t), \quad \forall t \in [-\tau_r, 0]. \nonumber
\end{align}

A peak estimation problem for \eqref{eq:delay_multiple} with history class $\mathcal{H}$ and objective $p(x)$ is
\begin{subequations}
\label{eq:peak_delay_multi_traj}
    \begin{align}
    P^* = & \sup_{t^* \in [0, T], \; x_h(\cdot)} p(x(t^* \mid x_h)) &\\
    & \dot{x} =  f(t, x(t), x(t-\tau_1), \ldots x(t-\tau_r)) & & \forall t \in [0, T] \label{eq:delay_dynamics_multi} \\
    & x(t) = x_h(t) & & \forall t \in [-\tau_r, 0]\\
     & x_h(\cdot) \in \mathcal{H}.
    \end{align}
\end{subequations}

A multiple-time-delay \ac{MV}-solution for the peak estimation problem \eqref{eq:peak_delay_multi_traj} is (for  all $i=1..r$):
\begin{subequations}
\label{eq:weak_solution_multi}
    \begin{align}
        &\textrm{History}  & \mu_{h i} &\in \Mp{H_0 \cap ([-\tau_{i}, -\tau_{i-1}] \times X)}\label{eq:weak_solution_multi_history} \\
       & \textrm{Initial} & \mu_0 &\in \Mp{X_0} \\
       &\textrm{Peak}  &\mu_p &\in \Mp{[0, T] \times X} \\
       &\textrm{Time-Slack} & \nu_i &\in \Mp{[0, T] \times X} \label{eq:weak_solution_multi_slack} \\
        &\textrm{Occupation Start }  & \bmu_0 &\in \Mp{[0, T-\tau] \times X^2} \\        
        &\textrm{Occupation End }  & \bmu_i &\in \Mp{[T-\tau_{i}, T-\tau_{i-1}] \times X^2}.
    \end{align}
\end{subequations}

The Lie derivative operator $\Lie_f$ with respect to \eqref{eq:delay_multiple} for $v \in C^1([0, T] \times X)$ is
\begin{equation}
    \Lie_f v(t, x_0) = \partial_t v(t, x_0) + f(t, x_0, x_1, \ldots, x_r) \cdot \nabla_{x_0} v(t, x_0).
\end{equation}

The multiple-time-delay peak estimation \ac{LP} for \eqref{eq:peak_delay_multi_traj} problem of $p(x)$ is

\begin{subequations}
\label{eq:peak_delay_multi_meas}
    \begin{align}
        p^* = & \ \sup \quad \inp{p}{\mu_p} \label{eq:peak_delay_multi_obj} \\
    & \inp{1}{\mu_0} = 1 \label{eq:peak_delay_multi_prob}\\    
    & \pi^{t}_\# \mu_{hi} = \lambda_{[-\tau_{i}, -\tau_{i-1}]} & & \forall i =1..r \label{eq:peak_delay_multi_hist}\\   
    & \mu_p = \delta_0 \otimes\mu_0 + \pi^{t x_0}_\# \Lie_f^\dagger (\bmu_0 + \textstyle \sum_{i=1}^r \bmu_i) \label{eq:peak_delay_multi_flow}\\        
    & \pi^{t x_1}_\# (\bmu_0 + \textstyle \sum_{i=1}^r \bmu_i) + \nu_i = S^{\tau_i}_\#(\sum_{j=1}^i \mu_{h j} + \pi^{t x_0}_\# (\bmu_0 + \sum_{j=1}^{i-1} \bmu_{i} ))  & & \forall i = 1..r\label{eq:peak_delay_multi_cons}\\ 
    & \textrm{Measure Definitions from  \eqref{eq:weak_solution_multi}.} \label{eq:peak_delay_multi_def}
    \end{align}
\end{subequations}

Theorem \ref{thm:delay_upper_bound} can be extended to the multiple-time-delay case to prove that $P^* \leq p^*$ between  \eqref{eq:peak_delay_multi_traj} and \eqref{eq:peak_delay_multi_meas}. Even if Conjecture \ref{conj:delay} holds in the single-delay case, it is unlikely the conjecture is satisfied in the multiple-delay case due to findings in \cite{rosenblueth1992proper}.


\subsection{Uncertainty}

This extension subsection will discuss three types of uncertainty that can affect \acp{DDE} dynamics: time-independent, time-dependent, and unknown-delay.

\subsubsection{Time-independent Uncertainty}

Time-independent uncertainty $\theta \in \Theta$ for a set $\Theta$ can be added to dynamics by adjoining the state $\theta$ following $\dot{\theta}=0$ to \eqref{eq:delay_dynamics}. This same process occurs in \cite{miller2021uncertain} for the \ac{ODE} case.

\subsubsection{Time-dependent Uncertainty}

Time-dependent may be implemented by a Young Measure approach. Given dynamics $\dot{x}(t) = f(t, x(t), x(t-\tau), w(t))$ for $w(t) \in W$, the joint occupation measures representing trajectories are $\bmu_0 \in \Mp{[0, T-\tau] \times X^2 \times W}$ and  $\bmu_1 \in \Mp{[T-\tau, T] \times X^2 \times W}$. No substantial changes are required to the Liouville nor consistency constraints.

Input delays may also be introduced into dynamics with $\dot{x}(t) = f(t, x(t), x(t-\tau), w(t), w(t-\tau))$ under the state history $x_h$ and input history $w_h$ (defined in times $t \in [-\tau, 0]$). The associated joint occupation measures are now $\bmu_0 \in \Mp{[0, T-\tau] \times X^2 \times W^2}$ and   $\bmu_1 \in \Mp{[T-\tau, T] \times X^2 \times W^2}$, each involving variables $(t, x_0, x_1, w_0, w_1)$. The state-input history class is $\hs \in PC([-\tau, 0], X \times W)$, and its history occupation measure now includes an input component $\mu_h \in \Mp{[-\tau, 0] \times X \times W}$.

While the Liouville equation stays the same as \eqref{eq:peak_delay_liou}, the consistency constraint ensures that the $w_1$ coordinate contains a delayed copy of $w_0$,
\begin{align}
    \pi^{t x_1 w_1}_\#(\bmu_0 + \bmu_1) = S^\tau_\#(\mu_h + \pi^{t x_0 w_0} \bmu_0).
\end{align}

\subsubsection{Unknown Delays}

This extension focuses on dynamics where the time-independent (constant) delay is unknown but fixed in the finite range of $\tau \in [\ubar{\tau}, \bar{\tau}]$. The unknown delay $\tau$ must be treated as an additional state with $\dot{\tau} = 0$.

The following sets may be defined:
\begin{subequations}
\begin{align}
    \Omega_{h} &= \{(\tau, t, x) \mid \tau \in [\ubar{\tau}, \bar{\tau}], \ (t, x) \in H_0\mid_\tau \}  \\
    \Omega_{0} &= \{(\tau, t, x_0, x_1) \mid \tau \in [\ubar{\tau}, \bar{\tau}], \ t \in [0, T-\tau], (x_0, x_1) \in X^2 \}\\
    \Omega_{1} &= \{(\tau, t, x_0, x_1) \mid \tau \in [\ubar{\tau}, \bar{\tau}], \ t \in [T-\tau, T], (x_0, x_1) \in X^2 \}.
\end{align}
\end{subequations}

\Iac{MV}-solution in the unknown-delay case has the form:
\begin{subequations}
\label{eq:weak_solution_unknown}
    \begin{align}
        &\textrm{History}  & \mu_h &\in \Mp{\Omega_{h}} \label{eq:weak_solution_unknown_history} \\
        &\textrm{History Slack}  &\bar{\mu}_h &\in \Mp{\Omega_{h}} \label{eq:weak_solution_unknown_history_slack} \\
       & \textrm{Initial} & \mu_0 &\in \Mp{X_0 \times [\ubar{\tau}, \bar{\tau}]} \\
       &\textrm{Peak}  &\mu_p &\in \Mp{[0, T] \times X\times [\ubar{\tau}, \bar{\tau}]} \\
       &\textrm{Time-Slack} & \nu &\in \Mp{[0, T] \times X\times [\ubar{\tau}, \bar{\tau}]} \label{eq:weak_solution_unknown_slack} \\
        &\textrm{Occupation Start }  & \bmu_0 &\in \Mp{\Omega_0} \\        
        &\textrm{Occupation End }  & \bmu_1 &\in \Mp{\Omega_1}.
    \end{align}
\end{subequations}

The Liouville and Consistency constraints in the unknown-delay case are unchanged as compared to the known-delay system (with the new state $\dot{\tau}=0$).

However, the history-validity constraints have the following form,
\begin{subequations}
\label{eq:hist_valid_unknown}
\begin{align}
& \inp{1}{\mu_0} = 1 \label{eq:hist_valid_unknown_mass}\\
&\delta_{t=0} \otimes (\pi^\tau_\# \mu_0) = \delta_{t=-\bar{\tau}} \otimes (\pi^{\tau}_\# \mu_0) + (\partial_t)_\# (\pi^{t \tau}_\# (\mu_h + \hat{\mu}_h)) \label{eq:hist_valid_unknown_const}\\
& \pi^{t}(\mu_h + \hat{\mu}_h) = \lambda_{[-\bar{\tau}, 0]}. \label{eq:hist_valid_unknown_lambda}
\end{align}
\end{subequations}
Constraint \eqref{eq:hist_valid_unknown_mass} ensures that the initial distribution $\mu_0$ is a probability measure. Constraint \eqref{eq:hist_valid_unknown_const} imposes that $\tau(t)$ is constant in time between $t=[-\bar{\tau}, 0]$. Constraint \eqref{eq:hist_valid_unknown_lambda} is a domination term that requires the history $x_h$ to be defined in times $[-\bar{\tau}, 0]$.

It is an open problem to extend consistency constraints and \ac{MV}-solutions towards cases where the delay $\tau(t)$ is time-dependent  (such as $\dot{\tau(t)} \in [-B, B]$).
