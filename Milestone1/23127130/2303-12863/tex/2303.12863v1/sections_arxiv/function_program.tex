Appendix \ref{app:delay_structure} uses these methods to form \ac{MV}-solutions to systems with other delay structures (proportional delay, long-delay discrete-time systems). 

\begin{rmk}
    The proof of Theorem  \ref{thm:delay_upper_bound} provides a unique \ac{MV} solution for each \ac{DDE} trajectory. Additionally, each \ac{DDE} trajectory given an initial condition $x_h$ is unique under the Lipschitz assumption A2.

    We note that \ac{MV} solutions are not necessarily unique (for a given terminal time distribution $\pi^t_\# \mu_p$) when the history measure $\mu_h$ is supported on the graph of more than one curve. As an example, Figure \ref{fig:same_occ} shows two sets of curves under the dynamics $\dot{x}(t) = -2x(t) - 3x(t-1)$ in the times $t\in [0, 5]$. The history occupation measure $\mu_h = 0.5 \lambda_{[-1, 0]} \otimes \delta_{x=1} + 0.5 \lambda_{[-1, 0]} \otimes \delta_{x=-1}$ is supported in the set $\mu_h \in \Mp{[-1, 0] \times \{-1, 1\}}$. The superposition of each set of red and blue curves each have the same history measure $\mu_h$, but the switch that takes place on the bottom plot (e.g. blue: $x_h(t) = 1$ for $t \in [-1, -0.5)$, \ $x_h(t) = -1$ for $t \in [0.5, 0]$) yields a different trajectory going forward in time.
    
    % arising from the history occupation measure supported in $H_0 = [-1, 0] \times \{-1, 1\}$ 

\begin{figure}[h]
    \centering
    \includegraphics[width=0.7\linewidth]{fig/same_history_occ.m.png}
    \caption{The same $\mu_h$ leads to different trajectories in times $(0, 5]$}
    \label{fig:same_occ}
\end{figure}
    
\end{rmk}

\subsection{Function Program}

The dual program of \eqref{eq:peak_delay_meas} with variables $(\gamma, \xi, v, \phi)$ is
\begin{subequations}
\label{eq:peak_delay_cont}
    \begin{align}
        d^* = & \inf_{\gamma \in \R} \ \gamma + \int_{-\tau}^0 \xi(t) dt \label{eq:peak_delay_cont_obj} \\
        & \xi(t) + \phi(t+\tau, x) \geq 0 & & \forall (t, x) \in H_0 \\ 
        & \gamma \geq v(0, x)  & & \forall x \in X_0 \\ 
        & v(t, x) \geq  p(x)  & & \forall (t, x) \in [0, T] \times X\\ 
        & \phi(t, x) \leq 0 & & \forall (t, x) \in [0, T] \times X\\ 
        & \Lie_f v(t, x_0) + \phi(t, x_1) \leq \phi(t+\tau, x_0) & & \forall (t, x_0, x_1) \in [0, T-\tau] \times X^2 \label{eq:peak_delay_cont_lie0}\\ 
        & \Lie_f v(t, x_0) + \phi(t, x_1) \leq 0 & & \forall (t, x_0, x_1) \in [T-\tau, T] \times X^2 \label{eq:peak_delay_cont_lie1}\\
        & \xi \in C([-\tau, 0]) \\
        & v \in C^1([0, T] \times X) \\
        & \phi \in C([0, T] \times X).
    \end{align}
\end{subequations}

\begin{thm}
    There is no duality gap between \eqref{eq:peak_delay_meas} and \eqref{eq:peak_delay_cont}.
\end{thm}
\begin{proof}
    See Appendix \ref{app:strong_duality_delay}.
\end{proof}

We pose the following conjecture based on \cite{lewis1980relaxation, rosenblueth1991relaxation}:
\begin{conj}
\label{conj:delay}
Assume that A1-A5 hold. Additionally, assume that $T > \tau>0$ and the image-set $f(t, x_0, X)$ is convex for all fixed $t \in [0, T], \ x_0 \in X$. Then there is no relaxation gap between \eqref{eq:peak_delay_traj} and \eqref{eq:peak_delay_meas} ($p^*=P^*$).
\end{conj}

Proving Conjecture \ref{conj:delay} is the subject of ongoing work.

Appendix \ref{app:ocp} contains a further discussion of the continuity and structural aspects of the dual solution in \eqref{eq:peak_delay_cont} as applied to bounding costs on \ac{DDE} \acp{OCP}.