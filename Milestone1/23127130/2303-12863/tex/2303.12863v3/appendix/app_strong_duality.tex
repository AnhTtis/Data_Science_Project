\section{Proof of Strong Duality}
\label{app:strong_duality_delay}

This appendix will prove strong duality between programs \eqref{eq:peak_delay_meas} and \eqref{eq:peak_delay_cont} for peak estimation. The general pattern for cone conventions and affine maps of \cite{tacchi2021thesis} will be followed.

% \urg{Use Theorem 2.6 of \cite{tacchi2021thesis}, as is standard practice.}

The signed measure spaces of \eqref{eq:weak_solution} are
\begin{align}
     \mathcal{X} = &\mathcal{M}(H_0) \times \mathcal{M}(X_0) \times \mathcal{M}([0, T]\times X)^2 \nonumber \\
     &\times \mathcal{M}([0, T-\tau] \times X^2) \times \mathcal{M}([T-\tau, T] \times X^2) \label{eq:dual_spaces_delay} \\
     \mathcal{X}' = &C(H_0) \times C(X_0) \times C([0, T]\times X)^2 \nonumber \\
     &\times C([0, T-\tau] \times X^2) \times C([T-\tau, T] \times X^2). \nonumber
\end{align}

Their nonnegative subcones (with \eqref{eq:weak_solution} membership) are topological duals under A1 and have definitions
\begin{align}
     \mathcal{X}_+ = &\mathcal{M}_+(H_0) \times \mathcal{M}_+(X_0) \times \mathcal{M}_+([0, T]\times X)^2 \nonumber  \\
     & \times \mathcal{M}_+([0, T-\tau] \times X^2) \times \mathcal{M}_+([T-\tau, T] \times X^2) \label{eq:dual_cones_delay} \\
     \mathcal{X}'_+= &C_+(H_0) \times C_+(X_0) \times C_+([0, T]\times X)^2  \nonumber \\
    &\times C_+([0, T-\tau] \times X^2) 
      \times C_+([T-\tau, T] \times X^2). \nonumber
\end{align}

The collection of measures in \eqref{eq:weak_solution} will be denoted as $\boldsymbol{\mu} = (\mu_h, \mu_0, \mu_p, \nu, \bmu_0, \bmu_1)$ and is a member of $\mathcal{X}_+$. 

The constraint spaces of \eqref{eq:peak_delay_prob}-\eqref{eq:peak_delay_cons} are
\begin{align}
    \mathcal{Y} &= \R \times C([-\tau, 0]) \times  C^1([0, T] \times X) \times C([0, T] \times X) \\
    \mathcal{Y}' &= 0 \times \mathcal{M}([-\tau, 0])  \times C^1([0, T] \times X)' \times  \mathcal{M}([0, T] \times X).
\end{align}
The space $\mathcal{X}$ has the weak-* topology and 
$\mathcal{Y}$ has a sup-norm bounded weak topology.
Because there are no affine-inequality constraints present in \eqref{eq:peak_delay_prob}-\eqref{eq:peak_delay_cons}, we write $\mathcal{Y}_+ = \mathcal{Y}$ and $\mathcal{Y}_+' = \mathcal{Y}'$ to match the notation used in \cite{tacchi2021thesis}. 

The variables of \eqref{eq:peak_delay_cont} with $\boldsymbol{\ell} = (\gamma, \xi,  v, \phi)$ satisfy $\boldsymbol{\ell} \in \mathcal{Y}_+'$.

A pair of adjoint linear operators $\A: \mathcal{X}_+ \rightarrow \mathcal{Y}_+$ and $\A': \mathcal{Y}_+' \rightarrow \mathcal{X}_+'$  induced from \eqref{eq:peak_delay_prob}-\eqref{eq:peak_delay_cons} are,
\begin{align}
    \A(\boldsymbol{\mu}) &=\begin{bmatrix}\inp{1}{\mu_0}\\ \pi^t_\# \mu_h \\ \mu_p- \delta_0 \otimes\mu_0 -\Lie_f^\dagger \mu  \\ S^\tau_\#(\mu_h + \pi^{t x_0}_\# \bmu_0)- \pi^{t x_1}_\# (\bmu_0 + \bmu_1) - \nu  \end{bmatrix}\\ 
    \A'(\boldsymbol{\ell}) &=\begin{bmatrix}\xi(t) + \phi(t+\tau, x)\\ \gamma - v(0, x) \\ v(t, x) \\ -\phi(t, x) \\ -\Lie_f v(t, x_0) - \phi(t, x_1) + \phi(t+\tau, x_0) \\ -\Lie_f v(t, x_0) - \phi(t, x_1) \end{bmatrix}.\nonumber
\end{align}
The cost and answer vectors are
\begin{align}
    \mathbf{c} &= [0, 0, p, 0, 0, 0] \\
    \mathbf{b} &= [1, \lambda_{[-\tau, 0]}, 0, 0].
\end{align}
Problem \ref{eq:peak_delay_meas} may be expressed as the standard-form \ac{LP}:
\begin{align}
    p^* =& \sup_{\boldsymbol{\mu} \in \mathcal{X}_+}\inp{\mathbf{c}}{\boldsymbol{\mu}} = \inp{p}{\mu_p}, & & \mathbf{b} - \A(\boldsymbol{\mu}) \in \mathcal{Y}_+. \label{eq:peak_delay_meas_std}\\
\intertext{The standard-form dualization of \eqref{eq:peak_delay_meas_std}  is }
    d^* = &\inf_{\boldsymbol{\ell} \in \mathcal{Y}'_+} \inp{\boldsymbol{\ell}}{\mathbf{b}} = \gamma + \textstyle \int_{t=-\tau}^0 \xi(t) dt,
    & &\A'(\boldsymbol{\ell}) - \mathbf{c} \in \mathcal{X}_+. \label{eq:peak_delay_cont_std}
\end{align}
The standard-form \eqref{eq:peak_delay_cont_std} may be expanded into \eqref{eq:peak_delay_cont}.

Given that all sets are compact (A1), measures in $\boldsymbol{\mu}$ are bounded (Lemma \ref{lem:moment_bound}), functions in $(c, \mathcal{A})$ are continuous (A2, A4, $v\in C^1([0, T] \times X) \implies \Lie_f v \in C([0, T] \times X)$ ), and there exists a feasible measure solution (Theorem \ref{thm:delay_upper_bound}); it holds that strong duality between \eqref{eq:peak_delay_meas} and \eqref{eq:peak_delay_cont} is achieved by Theorem 2.6 of \cite{tacchi2021thesis}.

% \old{
% The sufficient conditions for strong duality and attainment of optimality between \eqref{eq:dist_meas_std} and  \eqref{eq:dist_cont_std} as outlined in Theorem 2.6 of \cite{tacchi2021thesis} are that:
% \begin{enumerate}
%     \item[R1] All support sets are compact (A1)
%     \item[R2] All measure solutions have bounded mass (Lemma \ref{lem:bounded})
%     \item[R3] All functions involved in the definitions of  $c$ and $\A$ are continuous (A2, A3)
%     \item[R4] There exists a $\rev{\boldsymbol{\mu}}_{\textrm{feas}} \in \rev{\mathcal{X}}_+$ with $\rev{\mathbf{b}} - \A(\rev{\boldsymbol{\mu}}_\textrm{feas}) \in \rev{\mathcal{Y}}_+$ 
% \end{enumerate}

% The requirements R1 and R2 hold by Assumption A1 and Lemma \ref{lem:bounded} respectively. R3 is valid given that $c(x, y)$ is $C^0$ (A3), the projection map $\pi^{x}$ is continuous, and the mapping $(t,x) \mapsto \Lie_f v(t,x)$ is $C^0$ for $v\in C^1$ and $f$ Lipschitz (continuous) (A2). A feasible measure $\rev{\boldsymbol{\mu}}_\textrm{feas}$ may be constructed from the process in Theorem \ref{thm:meas_lower} from a tuple $\mathcal{T}$, therefore satisfying R4.

% Strong duality between \eqref{eq:dist_meas} and \eqref{eq:dist_cont} is therefore proven after satisfaction of all four requirements.




% Therefore, strong duality holds between \eqref{eq:dist_meas} and \eqref{eq:dist_cont}.

% }