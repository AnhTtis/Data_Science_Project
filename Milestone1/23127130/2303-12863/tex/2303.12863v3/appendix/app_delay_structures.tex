\section{Delay Structures}
\label{app:delay_structure}
This chapter has focused on supremizing $p(x)$ in \eqref{eq:peak_delay_traj} over continuous-time systems with a discrete delay $x(t-\tau)$.
This subsection will discuss peak estimation of $p(x)$ with respect to other types of dynamics and delay structures.

\subsection{Proportional Time-Delays}

A system with a proportional delay is defined with respect to a scaling term $\kappa \in [0, 1)$:
\begin{align}
    \dot{x}(t) = f(t, x(t), x(\kappa t)). \label{eq:dynamics_prop}
\end{align}

Proportional time delays are observed in the current collection of a pantograph on a streetcar \cite{ockendon1971dynamics}. References on functional differential equation with proportional time delay include \cite{carr19762, iserles1991generalized, iserles1994nonlinear, liu1996asymptotic}.
% This framework may be applied to neural networks with proportional delays \cite{zhou2015global}.
MATLAB uses the command \texttt{ddesd} to solve DDEs with time-dependent delays by an RK4 algorithm \cite{shampine2005solving}. 
Other numerical algorithms specifically for proportional delays include \cite{ali2009spectral, sedaghat2012numerical, bahcsi2015numerical}. 

The peak estimation problem over \eqref{eq:dynamics_prop} is,
\begin{subequations}
\label{eq:peak_delay_prop_traj}
    \begin{align}
    P^* = & \sup_{t^* \in [0, T], \; x_0 \in X_0} p(x(t^* \mid x_0)) &\\
    & \dot{x} =  f(t, x(t), x(\kappa t)) & & \forall t \in [0, T] \label{eq:delay_dynamics_prop}.
    \end{align}
\end{subequations}
\Iac{MV}-solution for proportional time delays is
\begin{subequations}
\label{eq:weak_solution_prop}
    \begin{align}
       & \textrm{Initial} & \mu_0 &\in \Mp{X_0} \\
       &\textrm{Peak}  &\mu_p &\in \Mp{[0, T] \times X} \\
       &\textrm{Time-Slack} & \nu &\in \Mp{[0, T] \times X} \label{eq:weak_solution_slack_prop} \\
        &\textrm{Occupation Start }  & \bmu_0 &\in \Mp{[0, \kappa T] \times X^2} \label{eq:weak_solution_start_prop}
         \\        
        &\textrm{Occupation End }  & \bmu_1 &\in \Mp{[\kappa T, T] \times X^2}. \label{eq:weak_solution_end_prop}
    \end{align}
\end{subequations}

Note how the \ac{MV}-solution \eqref{eq:weak_solution_prop} lacks a history measure $\mu_h$ as compared with \eqref{eq:weak_solution}, and also how the limits on \eqref{eq:weak_solution_start_prop}-\eqref{eq:weak_solution_end_prop} are $[0, \kappa T]$ and $[\kappa T, T]$ respectively.

The Lie derivative operator $\Lie$ with $\Lie v= (\partial_t + f(t, x_0, x_1) \cdot \nabla_{x_0}) v(t, x_0)$ is the same as in the discrete-delay case \eqref{eq:peak_delay_liou}, but under dynamics \eqref{eq:dynamics_prop}.

The consistency constraint follows from a modification of Lemma \ref{lem:consistency_free}:
\begin{lem}
\label{lem:consistency_free_prop}
    % Let $x(t)$ be a solution to \eqref{eq:delay_dynamics} (same setting as Lemma \ref{lem:consistency_fixed}) 
    Let $x(\cdot)$ be a solution to \eqref{eq:delay_dynamics_prop} with an initial condition of $x_0 \in X_0$ and a stopping time of $t^* \in [0, T]$. The following pairs of integral are equal for all  $\phi \in C([0, T] \times X)$:
    \begin{align}
    \label{eq:change_limits_free_prop}
        &\int_{0}^{ t^*} \phi(t, x(\kappa t))dt = \frac{1}{\kappa} \int_{0}^{\min(t^*/\kappa, T)} \phi(t'/\kappa, x(t)) dt'.
    \end{align}
\end{lem}
\begin{proof}
   This relation is due to a change of variable with $t' \leftarrow \kappa t$.
\end{proof}

The resultant consistency constraint w.r.t. the measures in \eqref{eq:weak_solution_prop} is
\begin{align}
\label{eq:consistency_prop}
\inp{\phi(t, x_1)}{\bmu_0 + \bmu_1} + \inp{\phi(t, x)}{\nu} &= \inp{\phi(t/\kappa, x_0) / \kappa}{\bmu_0}.
\end{align}

Expressing the linear expansion operator $E_\kappa$ as $E_\kappa \phi(t, x) = \phi(t/\kappa, x_0)\kappa$, the measure \ac{LP} for problem \eqref{eq:peak_delay_prop_traj} is,
\begin{subequations}
\label{eq:peak_delay_prop_meas}
    \begin{align}
    p^* = & \ \sup \quad \inp{p}{\mu_p} \label{eq:peak_delay_prop_obj} \\    
    & \inp{1}{\mu_0} = 1 \label{eq:peak_delay_prop_prob}\\    
    & \mu_p = \delta_0 \otimes\mu_0 + \pi^{t x_0}_\# \Lie_f^\dagger (\bmu_0 + \bmu_1) \label{eq:peak_delay_prop_flow}\\     
    & \pi^{t x_1}_\# (\bmu_0 + \bmu_1) + \nu = E^\kappa_\#(\pi^{t x_0}_\# \bmu_0) \label{eq:peak_delay_prop_cons}\\ 
    & \textrm{Measure Definitions from  \eqref{eq:weak_solution_prop}.} \label{eq:peak_delay_prop_def}
    \end{align}
\end{subequations}

Problem \eqref{eq:peak_delay_prop_meas} upper-bounds \eqref{eq:peak_delay_prop_traj} by following the reasoning from Theorem \ref{thm:delay_upper_bound} for the proportional-delay case.

\begin{rmk}
Proportional and discrete delays can be applied together to form dynamics,
\begin{align}
\label{eq:prop_combined}
    \dot{x}(t) = f(t, x(t), x(\kappa t - \tau).
\end{align}
Causalness of \eqref{eq:prop_combined} requires that $\kappa \in [0, 1)$ and $\tau \geq 0$. A consistency constraint may be posed using an integral relation
\begin{align}
    \int_{0}^T \phi(t, x(\kappa t - \tau)) dt = \int_{-\tau}^{\kappa T - \tau} \phi((t+\tau)/\kappa, x(t))/\kappa dt,
\end{align}
used as a step towards forming Lemmas \ref{lem:consistency_free} and \ref{lem:consistency_free_prop}.
\end{rmk}

\subsection{Discrete-Time Systems}

This subsection will concentrate on a discrete-time system with a long time delay. The discrete-time system $x[t]$ is defined w.r.t. a delay $\tau \in \N$, and a time horizon $T \in \N$ under the assumption that $\tau < T$.

The peak estimation program for a system with discrete-time dynamics and one time delay $\tau$ is:
\begin{subequations}
\label{eq:peak_delay_disc_traj}
    \begin{align}
    P^* = & \sup_{t^* \in 0..T, \; x_h[\cdot]} p(x[t \mid x_h]) &\\
    & \dot{x} =  f(t, x[t], x[t-\tau]) & & \forall t \in 1..T \label{eq:delay_dynamics_disc} \\
    & x[t] = x_h[t] & & \forall t \in [-\tau, 0]\\
     & x_h[\cdot] \in \mathcal{H}.
    \end{align}
\end{subequations}

Delayed dynamics \eqref{eq:delay_dynamics_disc} may be implemented as a non-delayed discrete system by state inflation in terms of $x[t-(0..\tau)]$ \cite{fridman2014intro}. Such state augmentation could lead to a large number of variables in systems analysis and result in intractably large computational problems.

This subsection will define \iac{MV}-solution using the variables from \eqref{eq:weak_solution}, in which the measures with the maximum number of variables are $(\bmu_0(t, x_0, x_1), \bmu_1(t, x_0, x_1))$.

\subsubsection{History-Validity}

The history-validity constraint for discrete-time systems will separate the history $x_h[t]$ into a time-zero component ($\mu_0$) and a history component $t \in -\tau..-1$ ($\mu_h$). The time-zero component is $\mu_0 \in X_0$, as in Section \ref{sec:history_validity}.

The history measure $\mu_h$ should represent a history $x_h[t]$ defined between $t \in -\tau..-1$. This may be imposed by setting the $t$-marginal of $\mu_h$ to a train of Dirac-deltas supported at sample times $-\tau..-1$,
\begin{align}
    \pi^{t}_\# \mu_h &= \textstyle \sum_{t=-\tau}^1 \delta_{t}.
\end{align}

\subsubsection{Liouville}

The discrete-time Liouville equation \cite{magron2017discrete} applied to the dynamics \eqref{eq:delay_dynamics_disc} for all test functions $v \in C([0, T+1] \times X)$ is,
\begin{align}
    \inp{v(t, x)}{\mu_p} &= \inp{v(0, x)}{\mu_0} + \inp{v(t+1, f(t, x_0, x_1) - v(t, x_0, x_1)}{\bmu_0 + \bmu_1}. \label{eq:delay_disc_push_int}
    \intertext{The Liouville constraint in \eqref{eq:delay_disc_push_int} will be abbreviated (using the identity operator $Id(x)=x$),}
    \mu_p &= \delta_0 \otimes \mu_0 + \pi^{t x_0}_\# ((t+1, f)_\# - Id_\#)(\bmu_0 + \bmu_1).\label{eq:delay_disc_push}
\end{align}



% The Liouville equation 
\subsubsection{Consistency}

The consistency constraint for dynamics \eqref{eq:delay_dynamics_disc} may be derived from the following Lemma,
\begin{lem}
\label{lem:consistency_free_disc}
    % Let $x(t)$ be a solution to \eqref{eq:delay_dynamics} (same setting as Lemma \ref{lem:consistency_fixed}) 
    Let $x[\cdot]$ be a trajectory of \eqref{eq:delay_dynamics_disc} given an initial history $x_h$ and a stopping time of $t^* \in 0..T$. It follows that the below pair of summations are equal for all  $\phi \in C([0, T] \times X)$:
    \begin{align}
    \label{eq:change_limits_free_discrete}
        &\left(\sum_{t=0}^{t^*} + \sum_{t=t^*}^{\min (T, t^* + \tau)} \right) \phi(t, x[t-\tau])dt = \sum_{t'=-\tau}^{\min(T-\tau, t^*)} \phi(t'+\tau, x[t]).
    \end{align}
\end{lem}
\begin{proof}
   The index of summation is exchanged as $t' \rightarrow t-\tau$.
\end{proof}

The resultant consistency constraint from Lemma \ref{lem:consistency_free_disc} has an identical form as  \eqref{eq:consistency_free} with
\begin{align}
    \pi^{t x_1}_\#(\bmu_0 + \bmu_1) + \nu = S^\tau_\#(\mu_h + \pi^{t x_0} \bmu_0).
\end{align}


\subsubsection{Measure Program}

The peak estimation measure \ac{LP} that upper-bounds \eqref{eq:peak_delay_disc_traj} is,
\begin{subequations}
\label{eq:peak_delay_disc_meas}
    \begin{align}
        p^* = & \ \sup \quad \inp{p}{\mu_p} \label{eq:peak_delay_disc_obj} \\
    & \inp{1}{\mu_0} = 1 \label{eq:peak_delay_disc_prob}\\    
    & \pi^{t}_\# \mu_h = \textstyle \sum_{t'=-\tau}^{-1} \delta_{t=t'} \label{eq:peak_delay_disc_hist}\\   
    & \mu_p = \delta_0 \otimes \mu_0 + \pi^{t x_0}_\# ((t+1, f, x_1)_\# - Id_\#)(\bmu_0 + \bmu_1)\label{eq:peak_delay_disc_flow}\\        
    & \pi^{t x_1}_\# (\bmu_0 + \bmu_1) + \nu = S^\tau_\#(\mu_h + \pi^{t x_0}_\# \bmu_0) \label{eq:peak_delay_disc_cons}\\ 
    & \textrm{Measure Definitions from  \eqref{eq:weak_solution}.} \label{eq:peak_delay_disc_def}
    \end{align}
\end{subequations}

This upper-bound also follows from constructing measures from trajectories as in Theorem 
\ref{thm:delay_upper_bound}.