\section{Introduction}
\label{sec:introduction}

% \urg{Introduce the paper}
This paper presents an algorithm to upper bound extreme values of a state function attained along trajectories of a \ac{DDE}. 
 The dynamics of a \ac{DDE} depend on a history of the state, in contrast to an \ac{ODE} in which the dynamics are a function only of the present values of state \cite{hale1971functional,kuang1993delay,bellen2013numerical,fridman2014intro}. This paper will involve analysis of \acp{DDE} in a state space $X \subset \R^n$ over a time horizon $T < \infty$ with a single fixed discrete bounded delay $\tau \in (0, T)$.

% Analysis in this paper will be restricted to continuous-time $n$-state dynamics in $r$ discrete bounded time delays $0 = \tau_0 < \tau_1 < \ldots < \tau < \infty$ with dynamics,
% \begin{equation}
% \label{eq:delay_dynamics}
%     \dot{x}(t) = f(t, x(t), x(t-\tau_1), x(t - \tau_2), \; \ldots, \  x(t - \tau)).
% \end{equation}
% Dynamics in \eqref{eq:delay_dynamics} are an instance of a functional differential equation in which the state evolution $\dot{x}(t)$ at time $t$ also depends on the prior states in times $[t-\tau, t)$.
 Trajectory evolution of \iac{DDE}  depends on an initial history $x_h: [-\tau, 0] \rightarrow X$ rather than simply an initial condition $x_0 \in X$ for a corresponding \ac{ODE}. The  evaluation at time $t$ for a trajectory starting with a history $x_h$ will be denoted as $x(t \mid x_h)$. 
 A function class $\mathcal{H}$ of histories may be defined, allowing for the definition of differential inclusions of \acp{DDE}. 
 % An example of one such class $\hs$ is a set of histories whose graph $\{(t, x_h(t)) \mid t \in [-\tau, 0]\}$ is contained in a given subset of $[-\tau, 0] \times X$.
A peak estimation problem may be defined on a time-delay system to find the maximum value of a state function $p$ along system trajectories given a class of initial histories $\mathcal{H}$ as
% The initial state histories are contained in the region $H_0$.
\begin{subequations}
\label{eq:peak_delay_traj}
    \begin{align}
    P^* = & \sup_{t^* \in [0, T], \; x_h(\cdot)} p(x(t^* \mid x_h)) &\\
    & \dot{x} =  f(t, x(t), x(t-\tau)) & & \forall t \in [0, T] \label{eq:delay_dynamics} \\
    & x(t) = x_h(t) & & \forall t \in [-\tau, 0]\\
     & x_h(\cdot) \in \mathcal{H}.
    \end{align}
\end{subequations}
% \MK{Milan: The constraint $x_h(\cdot) \in \mathcal{H}$ written explicitly  improves readability I believe you don't have to speak about graph constrained everywhere.}

The variables in Problem \eqref{eq:peak_delay_traj} are the stopping time $t^*$ and the initial history $x_h$.
Problem \eqref{eq:peak_delay_traj} is a \ac{DDE} version of the (generically nonconvex) \ac{ODE} peak estimation program studied in \cite{cho2002linear, fantuzzi2020bounding}. The peak estimation task in \eqref{eq:peak_delay_traj} is an instance of \iac{DDE} \ac{OCP} with a free terminal time and a zero running (integrated) cost.

This work uses measure-theoretic methods in order to provide certifiable upper bounds on the peak value $P^*$ from \eqref{eq:peak_delay_traj}. The first application of measure-theoretic methods towards \acp{DDE} was in \cite{warga1974optimal}, in which the control input was relaxed into a Young Measure \cite{young1942generalized} (probability distribution at each point in time) \cite{warga2014optimal}. This Young-Measure-based relaxed control yields the \ac{OCP} optimal value in the case of a single discrete time delay under convexity, regularity, and compactness assumptions. However, the Young Measure control programs may result in a lower bound when there are two or more delays in the system dynamics (there exist Young-Measure solutions that do not correspond to \ac{OCP} solutions) \cite{rosenblueth1991relaxation, rosenblueth1992strongly}. Adding new measures and constraints allows for the construction of tight Young Measure \ac{OCP} approximations 
% \MK{Milan: Same here} 
at the cost of significantly more complicated programs \cite{rosenblueth1992proper}. 
% Further information about Young Measure relaxation procedure for \acp{DDE} and more general functional differential equations is available at \cite{warga2014optimal}.


Occupation measures are nonnegative Borel measures that contain all possible information about trajectory behavior, and are a step beyond Young Measures in terms of abstraction and relaxation. The work of \cite{lewis1980relaxation} proves that a convex infinite-dimensional \ac{LP} in occupation measures for an \ac{ODE} \ac{OCP} has the same optimal value as the original \ac{OCP} under compactness, convexity, and regularity conditions. The problem of estimation of the peak of the expected value of a given state function for stochastic processes may be solved using occupation measures under these same conditions \cite{cho2002linear}. The Moment-\ac{SOS} hierarchy offers a sequence of outer approximations (lower bounds on \ac{OCP}/upper bounds on peak estimates) as found through solving \acp{SDP} of increasing size \cite{lasserre2009moments}. The moment-\ac{SOS} hierarchy has been applied to dynamical problems including barrier functions \cite{prajna2004safety}, \acp{OCP}  \cite{henrion2008nonlinear,papa2009sosdelay}, peak estimation \cite{fantuzzi2020bounding, miller2020recovery}, region of attraction estimation \cite{korda2013inner}, reachable set estimation \cite{magron2017discrete}  and distance estimation \cite{miller2022distance_short}. 

Use of the moment-\ac{SOS} hierarchy towards analysis of \acp{DDE} includes finding stability and safety certificates \cite{papachristodoulou2005tutorial, prajna2005methods, papa2009sosdelay}.
Prior work on using occupation measures for problems in time delays includes \ac{ODE}-PDE models in \cite{marx2018entropy, korda2018momentspde}, a Riesz-frame system in \cite{magron2020optimal}, and a gridded \ac{LP} framework for optimal control given a single history $x_h$ in \cite{barati2012optimal}. Peak estimation has been conducted on specific time-delay systems such as the forced Li\'{e}nard model \cite{suresh2018forced} and compartmental epidemic models \cite{sadeghi2021universal}.