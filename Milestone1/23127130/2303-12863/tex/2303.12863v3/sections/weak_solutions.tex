\section{Weak Solutions of DDEs}

\label{sec:weak_solutions}
\urg{ Revise and Excise this content}

This section develops a measure-valued (weak) solution to a \ac{DDE} of the form in Equation \eqref{eq:delay_dynamics} given a single bounded \ac{PC} history $x_h$. It is further assumed that the terminal time $T$ is finite. 

% \subsection{Motivating Visual Example}
A visual example used throughout this section will be the following trajectory of Equation \eqref{eq:time_delay_fig} with history $x_h(t) = 1 - t/2$ in times $ t \in [-\tau, 0] = [-1, 0]$. This trajectory is plotted in Figure \ref{fig:tri_component} for a time horizon of $T=5$. The blue section between $t \in [-\tau, 0] = [-1, 0]$ is the history $x_h(t)$. The red and green sections of the trajectory are defined in the regions $[0, T-\tau] = [0, 4]$ and $[T-\tau, T] = [4, 5]$ respectively. 

\begin{figure}[h]
    \centering
    \includegraphics[width=0.9\linewidth]{fig/triangle_component_decomposition.pdf}
    \caption{Example trajectory of \eqref{eq:time_delay_fig} to explain \ac{DDE} weak solutions}
    \label{fig:tri_component}
\end{figure}

Figure \ref{fig:tri_stack} shows the trajectory in current time $x(t)$ and in delayed time $x(t-\tau)=x(t-1)$. Note that the data in the red section is identical between current time: $t \in [0, 4]$ and delayed time $t \in [1, 5]$.
\begin{figure}[h]
    \centering
    \includegraphics[width=0.9\linewidth]{fig/triangle_component_stack.pdf}
    \caption{Trajectory in current time and delayed time}
    \label{fig:tri_stack}
\end{figure}

\subsection{Fixed Terminal Time}
\label{sec:weak_fixed}
A weak solution of a controlled time-delay system may be encoded by a tuple of measures $(\mu_T, \bar{\mu}, \{\nu_{j}\}_{j=0}^r)$. The names and supports of these measures are listed in Table \ref{tab:weak_solution}.

\begin{table}[h]
\begin{center}
\caption{\label{tab:weak_solution_old} Description of measures in \ac{DDE} weak solutions}
\begin{tabular}{l l}
    \textbf{Name} & \textbf{Measure and support}  \\
    Terminal & $\mu_T \in \Mp{X_T}$ \\
    Component (start)  & $\nu_0 \in \Mp{[0, T-\tau_r] \times X}$ \\
    Component (end)  & $\nu_{i=1:r} \in \Mp{[T - \tau_{r-i}, T-\tau_{r-i+1}] \times X}$ \\
     Joint Occupation & $\bar{\mu}  \in \Mp{[0, T] \times X^{r+1}}$\\
\end{tabular}
\end{center}
\end{table}
For notational convenience, define the supports of $\nu_j$ as listed in Table 
\ref{tab:weak_solution} as,
\begin{align}
\label{eq:omega_support}
    \Omega_0 &= [0, T-\tau_r] \times X \\
    \Omega_i &= [T - \tau_{r-i}, T-\tau_{r-i+1}] \times X & \forall i= 1, \ldots, r \nonumber.
\end{align}


These measures are related together by a Liouville equation and a set of consistency constraints.

\subsubsection{Liouville Equation}
 
The joint occupation measure $\bar{\mu}(t, x_0, x_1, \ldots, x_r)$ is a Young measure \cite{young1942generalized} 
in terms of time $t$, current state $x_0$, and delayed states $x_i$ for $i=1, \ldots, r$. A joint occupation $\bar{\mu} \in \Mp{[0, T] \times X^{r+1}}$ measure may be formed from every solution $x(t \mid x_h)$ of \eqref{eq:delay_dynamics} satisfying for all test functions $\bar{v} \in C([0, T] \times X^{r+1})$,
\begin{equation}
\label{eq:joint_occ_description}
    \inp{\bar{v}}{\bar{\mu}}=\int_{t=0}^T \bar{v}(t, x(t - \tau_0 \mid x_h), \ldots, x(t - \tau_r \mid x_h)) dt.
\end{equation}

% The correspo
The corresponding terminal measure $\mu_T$ for such a trajectory would be $\mu_T = \delta_{x = x(T \mid x_h)}$.
% A Liouville equation 
The 3-dimensional delay-embedding $(t, x(t), x(t-\tau))$ of the example trajectory is plotted in black in Figure \ref{fig:tri_embedding}. 
% A joint occupation measure $\bar{\mu}$ 
\begin{figure}[h]
    \centering
    \includegraphics[width=\linewidth]{fig/triangle_delay_embedding.pdf}
    \caption{Delayed embedding of trajectory in black}
    \label{fig:tri_embedding}
\end{figure}

% A Liou
The Liouville equation in \eqref{eq:liou_meas}-\eqref{eq:liou_int} may be generalized to time-delayed dynamics in \eqref{eq:delay_dynamics}. A Lie derivative for dynamics in \eqref{eq:delay_dynamics} is,
\begin{equation}
    \Lie_f v(t, x_0) = \partial_t v(t,x_0) + f(t, x_0, \ldots, x_r) \cdot \nabla_{x_0} v(t, x_0) \label{eq:delay_lie}
\end{equation}

The time-delay Liouville equation holding for all test functions $v(t, x) \in C^1([0, T] \times X)$ using the Lie derivative in \eqref{eq:delay_lie} is,
\begin{align}
    \inp{v(T, x)}{\mu_T} &= v(0, x_h(0)) + \inp{\Lie_f(t, x_0, \ldots, x_r) v(t, x_0)}{\bar{\mu}} \label{eq:liou_int_delay}\\
    \delta_T \otimes \mu_T &= \delta_0 \otimes \delta_{x=x_h(0)} + \pi^{t x_0}_\# \Lie_f^\dagger \bar{\mu} \label{eq:liou_meas_delay}
\end{align}

% Eq. \eqref{eq:liou_meas_delay} is the Liouville Equation of an 
The Lie derivative $\Lie_f v(t, x)$ in \eqref{eq:liou_int_delay} was expanded to highlight how the delay terms $x(t - \tau_i)$ in dynamics \eqref{eq:delay_dynamics} are replaced by variables $x_i$ in the joint occupation measure $\bar{\mu}$.

\subsubsection{Consistency Constraints}

% Consistency constraints are based on the change-of-variable relation,
% \begin{equation}
%     \int_{0}^T \phi_i(t, x(t - \tau_i)) dt = \int_{-\tau_i}^{T-\tau_i}
%     \end{equation}

The projection of the delay-embedding on the $(t, x(t))$ and $(t, x(t-\tau))$ planes align with the components shown in Figure \ref{fig:tri_stack}. In the measure framework, this projection onto axes corresponds to marginalizations of the joint occupation measures such as $\pi^{t x_i}_\# \bar{\mu}$. Component measures $\{\nu_j\}_{j=0}^r$ are occupation measures in $(t, x)$ containing information about the trajectory in a specific time range, in which the supports are recorded in Table \ref{tab:weak_solution}. As an example with $r=5$ delays, the component measure $\nu_3 \in \Mp{[T-\tau_{2}, T-\tau_{3}]}$ given a trajectory $x(t \mid x_h)$ is the unique occupation measure satisfying $\inp{c_3}{\nu_3} = \int_{T-\tau_3}^{T-\tau_2} \phi_3(t, x(t \mid x_h)) dt$ for all test functions $c_3 \in C([T-\tau_{2}, T-\tau_{3}] \times X)$.

Consistency constraints are imposed between the marginals of $\bar{\mu}$ and time-shifted copies of the component measures $\nu = \{\nu_j\}_{j=0}^r$.  The consistency constraints between $(\bar{\mu}, \nu)$ for lag $i = 0, \ldots, r$ under the convention that $\tau_0 = 0$ may be derived for all $\phi_i \in C([0, T]\times X)$ as,
\begin{align*}
      \inp{\phi_i(t, x_i)}{\bar{\mu}} &= \textstyle\int_{0}^{T} \phi_i(t, x(t  - \tau_i\mid x_h)) dt    & \\
        &= \textstyle\int_{-\tau_i}^{T - \tau_i} \phi_i(t + \tau_i, x(t \mid x_h)) dt  \\
         &= \textstyle\left(\int_{-\tau_i}^{0} + \int_{0}^{T - \tau_i} \right) \phi_i(t + \tau_i, x(t \mid x_h)) dt.
\end{align*}

The final formula for the consistency constraint at lag $i$ is,
\begin{equation}
\label{eq:consistency_single}
    \inp{\phi_i(t, x_i)}{\bar{\mu}} = \textstyle\int_{-\tau_i}^{0} \phi_i(t + \tau_i, x_h(t)) dt + \inp{\phi_i(t + \tau_i, x)}{\textstyle \sum_{j=0}^{r-i} \nu_i}
\end{equation}

% The $x_h$ integration term in \eqref{eq:consistency_single} will evaluate to $0$ by continuity of $\phi_0$ in the current state consistency constraint of $i=0$.


For the trajectory example in Fig \ref{fig:tri_embedding}, the first consistency constraint which holds for all test functions $\phi_0\in C([0, T] \times X)$ is,
    \begin{align*}
        \inp{\phi_0(t, x_0)}{\bar{\mu}} &= \textstyle\int_{0}^{T} \phi_0(t, x(t \mid x_h)) dt   \\
       &= \left(\textstyle\int_{0}^{T-\tau} + \int_{T-\tau}^T \right) \phi_0(t, x(t \mid x_h)) dt \\
       &= \inp{\phi_0(t, x)}{\cz{\nu_0} + \cp{\nu_1}}.
    \end{align*}

The second consistency constraint holding for all test functions $\phi_1\in C([0, T] \times X)$ is,
    \begin{align*}
        \inp{\phi_1(t, x_1)}{\bar{\mu}} &= \textstyle\int_{0}^{T} \phi_1(t, x(t  - \tau\mid x_h)) dt   \\
        &= \textstyle\left(\int_{-\tau}^{0} + \int_{0}^{T - \tau}\right) \phi_1(t + \tau, x(t \mid x_h)) dt  \\
       &=\textstyle \int_{-\tau}^{0} \phi_1(t+\tau, \cn{x_h}(t)) dt  +  \inp{\phi_1(t + \tau, x)}{\cz{\nu_0}}
    \end{align*}

In this manner, the data between times $t \in [0, 4]$ in $\cz{\nu_0}$ is shared between the marginals $\pi_\#^{t x_0} \bar{\mu}$ in current state and $\pi_\#^{t x_1} \bar{\mu}$ in delayed state (visible in Figure \ref{fig:tri_embedding}).

\begin{defn}
A fixed terminal time weak solution of the \ac{DDE} \eqref{eq:delay_dynamics} $(\mu_T, \bar{\mu}, \{\nu_j\}_{j=0}^r)$ given a history $x_h(t): t \in [-\tau_r, 0]$ is a tuple of measures $(\mu_T, \bar{\mu}, \{\nu_j\}_{j=0}^r)$ satisfying the Liouville equation \eqref{eq:liou_meas_delay} and consistency constraints \eqref{eq:consistency_single} for all $i= 0, \ldots, r$.
\end{defn}

\begin{thm}
\label{thm:construct_weak_single_fixed}
Every trajectory $x(t \mid x_h)$ of \eqref{eq:delay_dynamics} has a corresponding weak solution $(\mu_T, \bar{\mu}, \{\nu_j\}_{j=0}^r)$
\end{thm}
\begin{proof}
The proof of the above theorem is interspersed throughout this subsection. The terminal measure may be chosen as $\mu_T = \delta_{x = x(T \mid x_h)}$.  The joint occupation measure is the unique measure obeying Equation \eqref{eq:joint_occ_description}. The zeroth (start) component measure is the unique occupation measure satisfying $\inp{c_0}{\nu_0} = \int_{t=0}^{T-\tau_r} c_0(t, x_h(t)) dt$ for all test functions $c_0\in C([0, T-\tau_r] \times X)$, and the higher (end) component measures $\{\nu_i\}_{i=1}^r$ obey $\inp{c_i}{\nu_i} = \int_{t=T-\tau_{r-i}}^{T-\tau_{r-i+1}} c_i(t, x_h(t)) dt$ for all test functions $c_i\in C([T-\tau_{r-i}, T-\tau_{r-i+1}] \times X)$ and for all $i = 1, \ldots, r$.
\end{proof}

% \begin{thm}
% Every weak solution $(\mu_T, \bar{\mu}, \{\nu_j\}_{j=0}^r)$ given $x_h$ is supported on (portions or embeddings as appropriate of) the trajectory $x(t \mid x_h)$. 
% \label{thm:weak_steps}
% \end{thm}
% \begin{proof}
% The proof of this statement relies on induction through the Method of Steps (Equation \eqref{eq:method_of_steps}). 
% % The joint occupation measure $\bar{\mu}$ may be disentangled into $\bar{\mu}(t, x_0, \ldots, x_r) = \mu_t$

% Define $k_{max} = \ceil{T / \tau_1}$ as the number of steps required to solve a trajectory of \eqref{eq:delay_dynamics}. The joint occupation measure $\bar{\mu}$ may be split into $\bar{\mu} = \sum_{k=1}^{k_{max}} \bar{\mu}^{(k)}$ by absolute continuity where $\bar{\mu}^{(k)} \in \Mp{[(k-1)\tau_1, k \tau_1] \times X^{r+1}}$ for $k < k_{max}$ and $\bar{\mu}^{(k_{max})} \in \Mp{[(k_{max}-1)\tau_1, T] \times X^{r+1}}$.

% Each marginal $\pi^{t x_i}_\# \bar{\mu}^{(1)}$ is supported on the graph $(t, x_h(t - \tau_i))$ between times $[0, \tau_1]$ by consistency constraints \eqref{eq:consistency_single}. Given that the history $x_h: [-\tau, 0] \rightarrow X$ is single-valued for each $t$, the joint marginal $\pi^{t x_1, \ldots x_r} \bar{\mu}^{(k)}$ generated by gluing together the marginals $\pi^{t x_i}_\# \bar{\mu}^{(1)}$ along $t$  is therefore supported on the graph of $(t, x_h(t - \tau_1), \ldots, x_h(t - \tau_r))$.

% The ODE dynamics $F^{(1)}(t, x) = f(t, x, x_h(t - \tau_1), \ldots , x_h(t - \tau_r))$ inspire an ODE Liouville equation \eqref{eq:liou_meas} with initial measure $\mu_0^{(1)} = \delta_{x = x_h(0)}$, terminal measure $\mu_0^{(2)} \in \Mp{X}$, and occupation measure $\mu^{(1)}$,
% \begin{equation}
% \label{eq:step_1_proof}
%     \delta_{t=\tau_1} \otimes \mu_0^{(2)} = \delta_{t=0}\otimes \mu_0^{(1)} + \Lie_{F^{(1)}}^\dagger \mu^{(1)}.
% \end{equation}
% There is a unique solution $(\mu^{(1)}, \mu_0^{(2)})$ to \eqref{eq:step_1_proof} by Theorem 3.1 of \cite{ambrosio2003lecture}.
% Let $x^{(1)}(t)$ be the trajectory starting from $x_h(0)$ obeying
% $\dot{x}^{(1)}(t) = F^{(1)}(t, x)$
% between times $[0, \tau_1]$. 
% The trajectory $x^{(1)}(t)$ is a single-valued function of $t$ for $t \in [0, \tau_1]$.  
% The  measure solution is therefore $\mu_0^{(2)} = \delta_{x = x^{(1)}(\tau_1)}$ and $\mu^{(1)}$ is the occupation measure supported on $(t, x^{(1)}(t))$. 
% The marginal $\pi_\#^{t x_1 \ldots x_0}$ and the occupation measure $\mu^{(1)}$ may be glued together along $t$ to produce the unique occupation measure supported on $(t, x^{(1)}(t), x_h(t - \tau_1), \ldots, x_h(t - \tau_r))$, which is $\bar{\mu}^{(1)}$. 

% Further steps $k=2, \ldots, k_{max}$ may proceed by the same process, patching together a unique weak solution with measures supported on the (portioned and embedded) graphs of the single unique trajectory $x(t \mid x_h)$. For reference, the Liouville equation used in step $k < k_{max}$ is,
% \begin{equation}
% \label{eq:step_k_proof}
%     \delta_{t=k \tau_1} \otimes \mu_0^{(k+1)} = \delta_{t=(k-1) \tau_1}\otimes \mu_0^{(k)} + \Lie_{F^{(k)}}^\dagger \mu^{(k)}.
% \end{equation}


% \end{proof}
% Fo
\subsection{Free Terminal Time}
\label{sec:weak_free}
The weak solutions $(\mu_T, \bar{\mu}, \{\nu_j\}_{j=0}^r)$ described in Section \ref{sec:weak_fixed} have a fixed terminal time $T$. A terminal measure for free terminal time is a probability measure $\mu_p \in \Mp{[0, T] \times X}$, just as in the ODE peak estimation sections. The Liouville equation associated with the free terminal time setting is,
\begin{align}
    \inp{v(t, x)}{\mu_p} &= v(0, x_h(0)) + \inp{\Lie_{f(t, x_0, \ldots, x_r)} v(t, x_0)}{\bar{\mu}} \label{eq:liou_int_delay_free}\\
 \mu_p &= \delta_0 \otimes \delta_{x=x_h(0)} + \pi^{t x_0}_\# \Lie_f^\dagger \bar{\mu}. \label{eq:liou_meas_delay_free}
\end{align}

The consistency constraints in \eqref{eq:consistency_single} must change to accommodate the free terminal time. To explain why, consider the example in Figure \ref{fig:stop_traj}. 

\begin{figure}[h]
    \centering
    \includegraphics[width=0.9\linewidth]{fig/thesis_stop.pdf}
    \caption{Halted execution at $t^*=3$ }
    \label{fig:stop_traj}
\end{figure}

The system \eqref{eq:time_delay_fig} as seen in Figure \ref{fig:tri_stack} is stopped at time $t^* = 3$. The top panel of Figure \ref{fig:stop_traj} shows the trajectory running for $t^*=3$ time units, terminating in the black star. The bottom panel of Figure \ref{fig:stop_traj} is in delayed time units, and the same black star is at time $t^* + \tau = 3+1 = 3$. Let $\bar{\mu}$ be a joint occupation measure obeying consistency constraints \eqref{eq:consistency_single} with respect to $\cz{\nu_0}$ and $\cn{x_h}$. The masses of the marginals (elapsed times) as plotted in Figure \ref{fig:stop_traj} are $\inp{1}{\pi_\#^{t x_0} \bar{\mu}} = 4$ and $\inp{1}{\pi_\#^{t x_1} \bar{\mu}} = 4$. This is a contradiction, since the common $t$-marginals $\pi_\#^t \pi_\#^{t x_0} \bar{\mu}$ and $\pi_\#^t \pi_\#^{t x_1} \bar{\mu}$ should be equal to each other.

This contradiction may be patched by stopping the delayed trajectory $x(t-\tau)$ at time $t^* - \tau$ rather than at time $t^*$. The black star on the bottom panel of Figure \ref{fig:stop_traj_delay} remains at time $t^*+\tau = 4$, but now there is a black circle located at the point $(t^*, x(t^* - \tau\mid x_h)) = (3, x(2))$. This black circle is the stopping point for the delayed process. With the early stop in the black circle, an equal amount of elapsed time occurs in the top plot and the bottom plot. The marginals $t$-marginals $\pi_\#^t \pi_\#^{t x_0} \bar{\mu}$ and $\pi_\#^t \pi_\#^{t x_1} \bar{\mu}$ with delay-stopping are now equal to each other, and a joint occupation measure $\bar{\mu}$ supported on the graph $(t, x(t), x(t - \tau))$ for $t \in [0, t^*]$ may be defined. The dotted red line between the black circle and black star on the bottom plot is the trajectory between times $[t^*-\tau, t^*] = [2, 3]$ in absolute time.

\begin{figure}[h]
    \centering
    \includegraphics[width=0.9\linewidth]{fig/thesis_stop_free_slack.pdf}
    \caption{Halted execution at $t^*=3$ with delay-stopping}
    \label{fig:stop_traj_delay}
\end{figure}




% \subsection{Single Slacks

The term $\inp{\phi_0(t,x)}{\bar \mu_2}$ may be subtracted from both sides of the zeroth consistency constraint \eqref{eq:consistency_single} at $i=0$, forming the expression,
\begin{equation}
\label{eq:consistency_zero}
    \inp{\phi_0(t,x_0)}{\bar \mu_1} = \inp{\phi_0(t, x)}{\textstyle\sum_{j=0}^r \nu_j} - \inp{\phi_0(t,x_0)}{\bar \mu_2}.
\end{equation}

The supports of the component measures $\{\nu_j\}_{j=0}^r$ form a temporal partition of $[0, T] \times X$. 

By absolute continuity in equation \eqref{eq:consistency_zero}, there exists measures $\nu_j' \in \Mp{\Omega_j}, \ \forall j = 0, \ldots, r$ such that,
\begin{align}
    \inp{\psi_j(t,x)}{\nu_j'} &= \inp{\psi_j(t,x)}{\nu_j} - \int_{\Omega_j} \psi_j(t,x_j)d\bar \mu_2 & \forall \psi_j \in C(\Omega_j). \label{eq:zero_nu_abscont}
\end{align}
The following zeroth consistency constraint no longer involves $\bar{\mu_2}$,
\begin{align}
    \inp{\phi_0(t,x_0)}{\bar \mu_1} &= \inp{\phi_0(t, x)}{\textstyle\sum_{j=0}^r \nu_j'} & \forall \phi_0 \in C([0, T] \times X).
\end{align}

A $\tau_i$-shifted and $[0, T]$-truncated copy of $\pi^{t x_i}_\# \bar{\mu_2}$ (containing trajectory information in times $[t^*+\tau_i, T]$) may now be subtracted from both sides of consistency constraints \eqref{eq:consistency_single} for all $i = 1, \ldots, r$,
\begin{subequations}
\label{eq:consistency_shift}
\begin{align}
    \Phi_i &= \textstyle\inp{\phi_i(t, x_i)}{\bar{\mu_1} + \bar{\mu}_2} - \int_{[0, T-\tau_i] \times X} \phi_i(t + \tau_i, x_0) d \bar{\mu} \label{eq:consistency_shift_lhs}\\
    &= \textstyle\int_{t=-\tau_i}^0 \phi_i(t+\tau_i, x_h(t)) dt + \inp{\phi_i(t+\tau_i, x)}{ \sum_{j=0}^{r-i} \nu_i} - \int_{[0, T-\tau_i] \times X} \phi_i(t + \tau_i, x_0) d \bar{\mu}_2. \label{eq:consistency_shift_rhs}
\end{align}
\end{subequations}
The bottom term \eqref{eq:consistency_shift_rhs} may be reformulated using the $\{\nu_j\}_{j=0}^r$ expressions from \eqref{eq:zero_nu_abscont},
\begin{equation}
    \Phi_i = \textstyle\int_{t=-\tau_i}^0 \phi_i(t+\tau_i, x_h(t)) dt + \inp{\phi_i(t+\tau_i, x)}{ \sum_{j=0}^{r-i} \nu_i'}.
\end{equation}

The top term may also be reformulated via absolute continuity. There exists a nonnegative measure $\hat{\nu}_i \in \Mp{[0, T] \times X}$ such that,
\begin{equation}
    \inp{\phi_i(t, x_i)}{\bar{\mu_2}} = \inp{\phi_i(t, x)}{\hat{\nu}_i} +\int_{[0, T-\tau_i] \times X} \phi_i(t + \tau_i, x_0) d \bar{\mu}_2.
    \label{eq:i_nu_abscont}
\end{equation}

Let $t^*$ be a stopping time in the support of $\pi^t_\# \mu_p$. The measure $\pi^{t x_i}_\# \bar{\mu_2}$ in the left hand side of \eqref{eq:i_nu_abscont} is supported in times $[t^*, T]$. The $\tau_i$-shifted and truncated copy of $\pi^{t x_0}_\# \bar{\mu_2}$ (rightmost integral term of \eqref{eq:i_nu_abscont}) is supported in time $[t^*+\tau_i, T]$. The remaining slack term $\hat{\nu}_i$ is therefore supported on the trajectory in times $[t^*, t^*+\tau_i]$. This support is illustrated by the red dotted line after the black circle in the bottom plot of Figure \ref{fig:stop_traj_delay}. The red curve on the top plot of Figure \ref{fig:stop_traj_delay} is the support of $\cz{\nu_0'}$ in [0, 4], and $\cp{\nu_1'}$ in this example is the zero measure.

An expression for the free-time consistency constraint \eqref{eq:consistency_shift} for all $i = 1, \ldots, r$ and $\phi_i \in C([0, T] \times X)$ is,
\begin{equation}
\label{eq:consistency_slack}
    \Phi_i = \inp{\phi_i(t, x)}{\bar{\mu}_1}+ \inp{\phi_i(t, x)}{\hat{\nu}_i} = \textstyle\int_{t=-\tau_i}^0 \phi_i(t+\tau_i, x_h(t)) dt + \inp{\phi_i(t+\tau_i, x)}{ \sum_{j=0}^{r-i} \nu_i'}.
\end{equation}

There is a single slack measure $\hat{\nu}_i$ in each consistency constraint \eqref{eq:consistency_slack} for $i=1, \ldots, r$. 
% The joint occupation measure $\bar{\mu}_2$ over arbitrary dynamics has been completely eliminated from the consistency constraint expression.

\begin{defn}
A tuple $(\mu_p, \ \bar{\mu}, \  \{\nu_j\}_{j=0}^r, \ \{\hat{\nu}_i\}_{i=1}^r)$ is a free-time weak solution to a system with dynamics \eqref{eq:delay_dynamics} and a given history $x_h$ if it satisfies constraints,
\begin{subequations}
\begin{align}
&\forall v \in C^1([0, T] \times X):  \\
    & \qquad \inp{v(t,x)}{\mu_p} = v(0, x_h(0)) + \inp{\Lie_f v(t, x)}{ \bar{\mu}}  \nonumber  \\
    & \forall \phi_0 \in C([0, T] \times X):   \\
    & \qquad \inp{\phi_0(t, x_0)}{\bar{\mu}} = \inp{\phi_0(t, x)}{ \textstyle \sum_{j=0}^{r} \nu_i}\nonumber \\
    & \forall \phi_i \in C([0, T] \times X), \forall i=1, \ldots, r: \\
    &\qquad \inp{\phi_i(t, x)}{\bar{\mu}}+ \inp{\phi_i(t, x)}{\hat{\nu}_i} = \textstyle\int_{t=-\tau_i}^0 \phi_i(t+\tau_i, x_h(t)) dt + \inp{\phi_i(t+\tau_i, x)}{ \sum_{j=0}^{r-i} \nu_i} \nonumber \\
    &\mu_p \in \Mp{[0, T] \times X}, \ \bar{\mu} \in \Mp{[0, T] \times X^{r+1}}, \  \nu_0 \in \Mp{[0, T- \tau_r] \times X} \\
    & \forall i = 1, \ldots, r: \nu_i \in \Mp{[T - \tau_{r-i}, T - \tau_{r-i+1}] \times X}, \ \hat{\nu}_i \in \Mp{[0, T] \times X}.
\end{align}
\end{subequations}
\end{defn}

\begin{thm}
\label{thm:delay_single_free}
A free-time weak solution may be constructed from every trajectory $x(t\mid x_h)$ starting from $x_h$ stopping at time $t^* \in [0, T]$.
\end{thm}
\begin{proof}
% The content of this proof is dispersed in prior sections. 
The terminal measure may be chosen to be $\mu_p = \delta_{t = t^*} \otimes \delta_{x = x(t^* \mid x_h)}$. The component measures satisfy $\inp{\phi_0(t, x)}{\sum_{j=0}^r \nu_j} \ \int_{t=0}^{t^*} \phi_0(t, x(t \mid x_h)) dt$ for all $\phi_0 \in C([0, T] \times X)$. The joint occupation measure $\bar{\mu}$ is the occupation measure supported on the graph of $(t, x(t \mid x_h), x(t - \tau_1 \mid x_h), \ldots, x(t - \tau_r \mid x_h))$ in times $t \in [0, t^*]$. For each $i = 1, \ldots, r$, the slack measure $\hat{\nu}_i$ is the unique occupation measure satisfying $\inp{\phi_i(t, x)}{\hat{\nu}_i} =  \ \int_{t=t^*}^{\min(T, t^* + \tau_i)} \phi_i(t, x(t - \tau_i \mid x_h)) dt$.
\end{proof}  

% \begin{thm}
% \label{thm:delay_single_free_parameterized}
% Every free-time weak solution is supported on a family of trajectories parameterized by the stopping time $t^*$ of \eqref{eq:delay_dynamics} starting from $x_h$.
% \end{thm}
% \begin{proof}
% % The proof of this theorem relies on undoing the work in this subsection to recover a Two-Liouville expression. A measure $\bar{\mu_2} \in $

% This theorem may be proved by varying the Method of Steps construction used in the proof of Theorem \ref{thm:weak_steps}. 
% The terminal measure $\mu_p \in \Mp{[0, T] \times X}$ may be split into the sum of measures $\mu_p^{(k)} \in \Mp{[(k-1)\tau_1, k \tau_1]}$ and $\mu_p^{(k_{max})} \in \Mp{[(k_{max}-1)\tau_1, T] \times X}$ by absolute continuity.
% The step between times $t \in [0, \tau_1]$ obeys ODE dynamics  $\dot{x}(t) = F^{(1)}(t, x) = f(t, x, x_h(t - \tau_1), \ldots , x_h(t - \tau_r))$. An ODE Liouville equation \eqref{eq:liou_meas} with initial measure $\mu_0^{(1)} = \delta_{x = x_h(0)}$, terminal measure $\mu_0^{(2)} \in \Mp{X}$, mid-time terminal measure $\mu_p^{(1)} \in \Mp{[0, \tau_1] \times X}$ from $\mu_p$ (split by absolute continuity), and occupation measure $\mu^{(1)} \in \Mp{[0, \tau] \times X}$,
% \begin{equation}
% \label{eq:step_1_proof_free}
%     \delta_{t=\tau_1} \otimes \mu_0^{(2)} + \mu_p^{(1)} = \delta_{t=0}\otimes \mu_0^{(1)} + \Lie_{F^{(1)}}^\dagger \mu^{(1)}.
% \end{equation}

% The free-terminal measure $\mu_p^{(1)}$ captures the distribution (endpoints) of trajectories that stop in times $t \in [0, T]$. Delayed versions of these terminated trajectories are recorded until times $t^* + \tau_i$ (where $t^*$ is distributed as $\pi_\#^t \mu_p^{(1)}$) by the slacks $\hat{\nu}_i$, as enforced by the consistency constraint \eqref{eq:consistency_slack}. The Liouville equation at step $k < k_{max}$ is,
% \begin{equation}
% \label{eq:step_k_proof_free}
%     \delta_{t=k\tau_1} \otimes \mu_0^{(2)} + \mu_p^{(k)} = \delta_{t=(k-1) \tau_1}\otimes \mu_0^{(k)} + \Lie_{F^{(k)}}^\dagger \mu^{(k)},
% \end{equation}
% and the Liouville equation at step $k_{max}$ is,
% \begin{equation}
% \label{eq:step_k_proof_free_liou}
%     \mu_p^{(k_{max})} = \delta_{t=(k_{max}-1) \tau_1}\otimes \mu_0^{(k_{max})} + \Lie_{F^{(k_{max})}}^\dagger \mu^{(k_{max})}.
% \end{equation}
% By ODE uniqueness arguments from Theorem 3.1 of \cite{ambrosio2003lecture} in each step, the single-slack free-time weak solution is supported on a family of trajectories starting from $x_h$. The parameterizing stopping times $t^*$ are distributed as the $t$-marginal $\pi^t_\# \mu_p$. 

% \end{proof}
% As described in the previous derivations, every free-time stopped trajectory may be represented as a weak solution, and every weak solution is supported on a family of trajectories starting from $x_h$.
\section{Weak Solutions with Multiple Histories}

\label{sec:weak_multiple}

The weak solutions that have previously been developed in this section were defined with a single given history $x_h(t): \ t \in [-\tau_r, 0]$. This section will develop weak solutions where the initial histories $x_h$ are members of a function class $\mathcal{H} \subset PC([-\tau, 0] \times X)$. An example could be where $\mathcal{H}$ is the set of histories defined between times $t \in [-2, 0]$ with $1 \leq x_h(t) \leq 2$ and containing no further constraints on continuity of histories.
% are drawn from a distribution of histories (occupation measures between times $t \in [-\tau_r, 0]$).

Let the set $H_0 \subset [-\tau_r, 0] \times X$
contain the graph of all admissible initial state histories,
\begin{equation}
    (t, x_h(t)) \in H_0 \qquad \forall t \in [-\tau_r, 0], \qquad \forall x_h \in \mathcal{H}.
\end{equation}
Further define the set $X_0$ of initial conditions at time $t=0$ as the slice,
\begin{equation}
    X_0 = \{x \mid (0^+, x) \in H_0\}.
\end{equation}

\subsection{Fixed Terminal Time}
In this weak solution framework, the admissible trajectories $x_h(t) \in \mathcal{H}$ are distributed (averaged)
% The histories $x_h(t)$ are distributed  
according to an occupation measure $\mu_h \in \Mp{H_0}$. In the case of a single history $x_h$, this history occupation measure is the unique measure satisfying $\inp{v(t, x)}{\mu_h} = \int_{-\tau_r}^0 v(t, x_h(t)) dt$.

Define the support sets $\{\Omega_{-i}\}_{i=1}^r$ and history components $\{\nu_{-i}\}_{i=1}^r$,
\begin{align}
    \Omega_{-i} &= ([-\tau_i, -\tau_{i-1}] \times X) \cap H_0  & & i = 1, \ldots, r \\
    \nu_{-i} &\in \Mp{\Omega_{-i}}.
\end{align}
The support sets $\Omega_{-i}$ complement the existing $\Omega_{\geq 0}$ definition in Equation \eqref{eq:omega_support}.
The history occupation measure $\mu_h$ may be split by absolute continuity into the sum of history component measures $\mu_h = \sum_{i=1}^r \nu_{-i}$.

Consistency constraints with history occupation measures and fixed terminal time are,
\begin{align}
\label{eq:consistency_multi_fixed}
    \inp{\phi_i(t, x_i)}{ \bar{\mu}} &= \inp{\phi_i(t, x)}{\textstyle\sum_{j=-i}^{r-i} \nu_j} & \forall \phi_i \in C([0, T] \times X), i=0, \ldots, r.
\end{align}

An equivalent expression for constraint \eqref{eq:consistency_multi_fixed} is,
\begin{align}
\pi^{t x_i}_\# \bar{\mu} &= \textstyle \sum_{j=-i}^{r-i} \nu_j & \forall i = 0, \ldots, r.
\end{align}

The following definition will assume that the class $\mathcal{H}$ is the set of bounded \ac{PC} histories whose graph is inside $H_0$ with no further constraints.
\begin{defn}
\label{defn:weak_multiple_fixed}
A fixed terminal time weak solution with histories in $H_0$ is a tuple $(\mu_0, \mu_T, \bar{\mu}, \{\nu_j\}_{j=-r}^r)$ subject to,
\begin{subequations}
\label{eq:weak_multi_fixed}
\begin{align}
    &\inp{v(T, x)}{\mu_T} = \inp{v(0, x)}{\mu_0} + \inp{\Lie_f v(t, x)}{\bar{\mu}} & &\forall v \in C^1([0, T] \times X)  \\
    &\inp{\phi_i(t, x_i)}{\bar{\mu}} = \inp{\phi_i(t+\tau_i, x)}{\textstyle \sum_{j=-i}^{r-i} \nu_j} & &\forall \phi_i \in C([0, T] \times X), \ i = 0, \ldots, r \\
    & \inp{1}{\mu_0} = 1 \\
    &\mu_0 \in \Mp{X_0}, \ \mu_{T} \in \Mp{X} \\
    &\bar{\mu} \in \Mp{[0, T] \times X^{r+1}} \\
    &\nu_j \in \Mp{\Omega_j} & &\forall j = -r, \ldots, r.
\end{align}
\end{subequations}
\end{defn}

\begin{thm}
\label{thm:delay_multiple_fixed}
A multiple-history weak solution may be constructed from every trajectory $x_h(t) \in \mathcal{H}$ (satisfying $(t, x_h(t)) \in H_0 \  \forall t \in [-\tau_r, 0]$).
\end{thm}
\begin{proof}
This construction follows the method in Theorem \ref{thm:construct_weak_single_fixed} to form the measures $\mu_T, \bar{\mu}, \{\nu_j\}_{j=0}^r$ from the given history $x_h$. The initial measure may be chosen $\mu_0 = \delta_{x=x_h(0)}$, and the history components respect $\inp{v(t, x)}{\nu_{-i}} = \int_{-\tau_{i}}^{-\tau_{i-1}} v(t, x_h(t)) dt$ for all $v \in C([-\tau_i, -\tau_{i-1}] \times X)$ and $i = 1, \ldots, r$.
Conic combinations of weak solutions formed by histories $x_h^1, x_h^2, \ldots \in \mathcal{H}$ will also be a multiple-history weak solution by convexity of constraints in \eqref{eq:weak_multi_fixed}.
\end{proof}


% Free terminal time consistency constraints with multiple occupation measures have a term $\inp{\phi_i(t, x_i)}{ \bar{\mu} + \nu_i}$ on the left hand side of \eqref{eq:consistency_multi_fixed}, just as in Equation \eqref{eq:consistency_slack}.

% Future work includes finding out under what conditions will multiple-history weak solutions \eqref{eq:weak_multi_fixed}  be supported along \ac{DDE} trajectories of \eqref{eq:delay_dynamics}. A first step from \cite{lewis1980relaxation} may be that the image of the map $X^r \rightarrow f(t, x_0, X, X, X, \ldots, X)$ is convex for each fixed $(t, x_0) \in [0, T] \times X$. 
% If this convexity condition holds, $[0, T] \times X$ is compact, and $f$ is Lipschitz, then weak solutions \eqref{eq:weak_multi_fixed} are supported on the graphs of trajectories $x_0' = f(t, x_0, x_1, x_2, \ldots x_r)$ with external inputs $x_1(t), \ldots x_r(t)$ \cite{korda2014convex}. It is unknown at this point whether the consistency constraints \eqref{eq:consistency_multi_fixed} are strong enough to force the external inputs to align with delayed trajectories as $x_i(t) = x_0(t - \tau_i)$.

\subsection{ Free Terminal Time}

Free terminal time weak solutions with multiple histories require a set of additional constraints in order for all measures to remain bounded when $\mu_h$ is not given a priori. To explain why, consider free-time consistency constraint with multiple histories at delay $i = 1, \ldots, r$,
\begin{align}
\label{eq:consistency_multi_free}
    \inp{\phi_i(t, x_i)}{\bar{\mu}} + \inp{\phi_i(t, x)}{\nu_i} &= \inp{\textstyle \phi_i(t+\tau_i, x)}{ \sum_{j=-i}^{r-i} \nu_j} & \forall \phi_i \in C([0, T] \times X).
\end{align}

The case with $r=1$ will lead to the following expressions,
\begin{subequations}
\label{eq:free_time_example}
\begin{align}
\text{Initial Prob. Measure:} & & \inp{1}{\mu_0} &= 1 \label{eq:free_time_example_prob}\\
\text{Liouville $v(t, x)=1$:} & & \inp{1}{\mu_p} &= \inp{\partial_t 1}{\bar{\mu}} + \inp{1}{\mu_0} =  \inp{1}{\mu_0} = 1 \\
    \text{Liouville $v(t, x)=t$:} & & \inp{t}{\mu_p} &= 0 *\inp{1}{\mu_0}+\inp{\partial_t t}{\bar{\mu}} =  \inp{1}{\bar{\mu}} \leq T \\
    \text{Consistency $\phi_0(t, x)=1$:} & & \inp{1}{\bar{\mu}} &= \inp{1}{\nu_0 + \nu_1} \label{eq:free_time_example_consistency}\\
    \text{Consistency $\phi_1(t, x)=1$:} & & \inp{1}{\bar{\mu}} &= \inp{1}{\nu_0} + (\inp{1}{\nu_{-1}} - \inp{1}{\hat{\nu}_1}) \label{eq:free_time_example_error}.
\end{align}
\end{subequations}
The masses of $\mu_0, \mu_p, \bar{\mu}, \nu_0, \nu_1$ are all bounded by constraints \eqref{eq:free_time_example_prob}-\eqref{eq:free_time_example_consistency} when $T$ is finite. Line \eqref{eq:free_time_example_error} enforces a mass constraint involving the difference
$\inp{1}{\nu_{-1}} - \inp{1}{\hat{\nu}_1}$.
The mass $\inp{1}{\hat{\nu}_1}$ is only present and constrained  in this free-time consistency expression, so therefore the masses of $\nu_{-1}$ and $\hat{\nu}_1$ may each be unbounded. 

More generally, the masses of $\{\nu_{-i}, \hat{\nu}_i\}_{i=1}^r$ may be unbounded in a na\"ive porting of the multiple-history weak solution \eqref{eq:weak_multi_fixed} to free terminal time.

This issue of unbounded masses may be fixed by adding a Lebesgue constraint to the time marginals of the history components. The history occupation measure $\mu_h = \sum_{i=1}^r \nu_{-i}$ should represent a distribution of histories $x_h \in PC(H_0)$, where each history $x_h(t)$ is defined in times $t \in [-\tau_r, 0]$. The $t$-marginal $\pi^t_\# \mu_h$ should therefore be equal to the Lebesgue measure $\lambda_{[-\tau_r, 0]}$,
% \begin{align}
%     \inp{w(t)}{\sum_{i=1}^r \nu_{-i}} &= \int_{t = -\tau_r}^0 w(t) dt & \forall w \in C([-\tau_r, 0]).
% \end{align}
As a consequence, the mass $\inp{1}{\mu_h} = \tau_r$ under this Lebesgue constraint would equal the elapsed time in histories between $[-\tau_r, 0]$. The history components have mass $\nu_{-i} = \tau_i - \tau_{i-1} \ \forall i = 1, \ldots, r$.

\begin{defn}
\label{defn:weak_multiple_free}
A free terminal time weak solution with histories supported in $H_0$ is a tuple $(\mu_0, \mu_T, \bar{\mu}, \{\nu_j\}_{j=-r}^r, \{\hat{\nu_i}\}_{i=1}^r)$ subject to,
\begin{subequations}
\label{eq:weak_multi_free}
\begin{align}
    &\inp{v(T, x)}{\mu_T} = \inp{v(0, x)}{\mu_0} + \inp{\Lie_f v(t, x)}{\bar{\mu}} & &\forall v \in C^1([0, T] \times X)  \\
    &\inp{\phi_i(t, x)}{\hat{\nu}_i \pi^{t x_i}_\#\bar{\mu}} = \inp{\phi_i(t+\tau_i, x)}{\textstyle \sum_{j=-i}^{r-i} \nu_j} & &\forall \phi_i \in C([0, T] \times X), \ i = 0, \ldots, r \\
    & \inp{1}{\mu_0} = 1 \\
    &\inp{h(t)}{\textstyle\sum_{i=1}^r \nu_{-i}} = \textstyle\int_{t = -\tau_r}^0 h(t) dt & & \forall h \in C([-\tau_r, 0]). \\
    &\mu_0 \in \Mp{X_0}, \ \mu_{T} \in \Mp{X} \\
    &\bar{\mu} \in \Mp{[0, T] \times X^{r+1}} \\
    &\nu_j \in \Mp{\Omega_j} & &\forall j = -r, \ldots, r.
\end{align}
\end{subequations}
\end{defn}

\begin{thm}
\label{thm:delay_multiple_free}
A multiple-history free-time weak solution stopping at time $t^*$ may be built from every trajectory $x_h(t) \in \mathcal{H}$.
\end{thm}
\begin{proof}
The measures $\mu_T, \bar{\mu}, \{\nu_j\}_{j=0}^r, \{\hat{\nu_i}\}_{i=1}^r$ may be formed per the proof of Theorem \ref{thm:delay_single_free}. The measures $\{\hat{\nu_{-i}}\}_{i=1}^r$ and $\mu_0$ may be created based on the proof of Theorem \ref{thm:delay_multiple_fixed}. The measure $\mu_h = \sum_{i=1}^r \nu_{-i}$ is the unique occupation measure representing the history $x_h(t): t \in [-\tau_r, 0]$, so it automatically satisfies the time-Lebesgue constraint $\pi^t_\# \mu_h = \lambda_{[-\tau_r, 0]}$.
\end{proof}

It is as of yet unknown whether the converse of Theorem \ref{thm:delay_multiple_free} holds, whether every multiple-history free-time weak solution is supported on a family of \ac{DDE} trajectories. An  answer of validity to the converse of Theorem \ref{thm:delay_multiple_fixed} on fixed terminal time is required before working on the free terminal time case.

% \urg{Try out multiple histories on the SIR demo with constant histories. Randomly sample constant histories $x_h(t) = (1-I_0, I_0)$ as $I_0$ is drawn from a uniform distribution. Compute expected moments of the occupatoin measure by simulation as well as through the LMI.}
