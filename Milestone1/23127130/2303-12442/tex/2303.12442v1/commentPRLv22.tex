% ****** Start of file aipsamp.tex ******
%
%   This file is part of the AIP files in the AIP distribution for REVTeX 4.
%   Version 4.1 of REVTeX, October 2009
%
%   Copyright (c) 2009 American Institute of Physics.
%
%   See the AIP README file for restrictions and more information.
%
% TeX'ing this file requires that you have AMS-LaTeX 2.0 installed
% as well as the rest of the prerequisites for REVTeX 4.1
% 
% It also requires running BibTeX. The commands are as follows:
%
%  1)  latex  aipsamp
%  2)  bibtex aipsamp
%  3)  latex  aipsamp
%  4)  latex  aipsamp
%
% Use this file as a source of example code for your aip document.
% Use the file aiptemplate.tex as a template for your document.
\documentclass[%
aip,
% jmp,
% bmf,
% sd,
% rsi,
amsmath,amssymb,
%preprint,%
reprint,%
%author-year,%
%author-numerical,%
% Conference Proceedings
]{revtex4-1}

\usepackage{graphicx}% Include figure files
\usepackage{dcolumn}% Align table columns on decimal point
\usepackage{bm}% bold math
%\usepackage[mathlines]{lineno}% Enable numbering of text and display math
%\linenumbers\relax % Commence numbering lines

\usepackage[utf8]{inputenc}
\usepackage[T1]{fontenc}
\usepackage{mathptmx}
\usepackage{etoolbox}

%% Apr 2021: AIP requests that the corresponding 
%% email to be moved after the affiliations
\makeatletter
\def\@email#1#2{%
	\endgroup
	\patchcmd{\titleblock@produce}
	{\frontmatter@RRAPformat}
	{\frontmatter@RRAPformat{\produce@RRAP{*#1\href{mailto:#2}{#2}}}\frontmatter@RRAPformat}
	{}{}
}%
\makeatother
\begin{document}
	
	\preprint{AIP/123-QED}
	
	\title{Comment on "Rotating Spin and Giant Splitting: Unoccupied Surface Electronic Structure of Tl/Si(111)"}
	\author{A. F. Campos, K. Wang, A. Tejeda}
	\affiliation{Laboratoire de Physique des Solides, CNRS, Universit{\'e} Paris-Saclay, 91405 Orsay, France.}
	
	\date{\today}
	
	\begin{abstract}

	\end{abstract}
	
	\maketitle
	
	The Letter \cite{stolwijk2013rotating} studies the unoccupied states of Tl/Si(111) by spin-resolved inverse photoemission (IPES) using a rotatable electron source \cite{stolwijk2014rotatable}. Authors claim that the spin polarization rotates from an in-plane Rashba polarization at $\overline{\Gamma}$ to the surface normal at $\overline{K} (\overline{K'})$ valleys. This Comment first indicates that the in-plane polarization does not vanish at the valleys despite the claims and the data representation. Second, it highlights that the out-of-plane polarization cannot be straightforwardly measured. Hypothesis and/or unspecified data treatment are needed to experimentally conclude on the polarization rotation and the completely out-of-plane spin-polarized valleys. 
	
	The Letter claims an in-plane polarization vanishing around $\overline{K}$ and "completely out-of-plane spin-polarized valleys in the vicinity of the Fermi level". These claims seem to be supported by the absence of experimental points near the $\overline{K}$ valley in Fig. 3a of Ref. \onlinecite{stolwijk2013rotating}. This is at odds with the spectra around $\overline{K}$ ($60^{\circ} \le \theta \le 70^{\circ}$) showing differences in spin up/down (Fig. 2a of Ref.\onlinecite{stolwijk2013rotating}, from wich Fig. 3a is extracted), i.e. a non-vanishing in-plane polarization. More importantly, these points with non-vanishing in-plane polarization are absent in Fig. 3a, so Fig. 3a suggests that the in-plane component vanishes at $\overline{K}$. Note that both panels in Fig. 2 have the same angular range (0 to 70$^{\circ}$) and when $\theta$ is converted to $k$, the same wavevector range should be found in Fig. 3 panels. However $k^{max}_{\bot}<1.2$ \AA$^{-1}$ (Fig. 3b) while $k^{max}_{||}<1.1$ \AA$^{-1}$ (Fig. 3a), i.e. before $\overline{K}$.  
	
\begin{figure}[ht!]
	\includegraphics[width=0.43\textwidth]{DonathCoupling}
	\caption{(a) Electron source configuration for in-plane polarization spectra along $\overline{\Gamma}\overline{K}$ by varying $\theta$. (b) The configuration in (a) provides an in-plane electron beam polarization, perpendicular to $\overline{\Gamma}\overline{K}$ and with constant surface projection when varying $\theta$. (c) IPES spectra in (a) configuration, strictly sensitive to in-plane polarization. (d) Electron source configuration for some sensitivity to the out-of-plane polarization. (e) The configuration in (d) measures IPES spectra coupling in- and out-of-plane components of the electron beam polarization. The polarization is projected along $\overline{\Gamma}\overline{K}$ and its out-of-plane component varies with $\theta$. 	(f) A deconvolution is needed to obtain the out-of-plane component from measurements in configuration (d). Assumptions and/or a delicate analysis of spectral intensities are needed. }
	\label{fig:C1}
\end{figure}

Besides the unclear data representation, the setup cannot directly measure the out-of-plane polarization. In-plane polarization IPES spectra correspond to the situation shown in our Fig. 1 (top). Rotating the electron source (Fig. 1, bottom), provides spectra with some out-of-plane sensitivity. It can be seen that the pristine out-of-plane spectra could only be measured at strictly grazing incidence (Fig. 1e), which is experimentally unreachable. The difficulty of measuring the out-of-plane component near $\overline{\Gamma}$ ($k=\theta=0$) is also evident. Instead, out-of-plane spectra and dispersions close to $\overline{\Gamma}$ are shown (Fig. 2 and 3 in Ref. \onlinecite{stolwijk2013rotating}). 
		


			
Fig. 1 shows the necessity of assumptions to estimate the out-of-plane spectra. Measurements must be deconvolved, requiring information on the in-plane spectra. One can: (1) assume a vanishing in-plane component, or (2) introduce in-plane spectra as an input. (1) is not an experimental determination of the out-of-plane contribution. (2) relies on  a complex data treatment: (i) in-plane spectra measurement (ii) subtraction of in-plane spectra (vertical electron source) from raw spectrum with out-of-plane sensitivity (horizontal electron source). This involves a delicate spectral intensity comparison measured at different geometries, complicated by intensity changes when the electron source is rotated by 90$^{\circ}$: (a) the macroscopic rotation of the source displaces the electron beam from the optical axis and it must be realigned\cite{stolwijk2014rotatable}. (b) the beam polarization changes its direction with respect to surface orbitals and $\overline{\Gamma}\overline{K}$ ($||$ to $\bot$, see Fig. 1). (c) the experimental geometry affects the IPES matrix elements. Finally, surface freshness evolves with time, affecting the spectral intensity of the measurements in the two different configurations. There are thus many factors affecting intensities when turning the electron source, when varying $\theta$ or $\textbf{P}$. Their impact on the intensity variation is difficult to quantify. The Letter concludes on a constant spectral intensity instead.
		
In summary, conclusions on the spin polarization vector rotation and the completely out-of-plane spin-polarized valleys should be carefully reviewed, since rotatable spin-polarized electron sources cannot measure the out-of-plane polarization of electronic states. Reliable results require sources allowing to measure in- and out-of-plane spectra without assumptions or without data treatment involving spectral intensities \cite{Campos22a, Campos22b, Campos22c}. 
\bibliography{aipsamp}
		
\end{document}
