\clearpage

%% CONTINUUM and C18O IMAGES %%%%%%%%%%%%%%%%%%%%%%%%%%%%%%%%%%%%%%%%%%%%%%%%%%%%%%%%%%%%%%%%%%%%%%%%%%%%%%

\section{Assessing the significance of the CS 7-6 emission offset}
\label{sec:appendix_continuum_c18o}

Here, we assess whether the observed offset in the peak of the CS 7-6 emission is related to a physical phenomenon, or the result of an observational error. We begin by comparing the CS emission to the continuum and C$^{18}$O 3-2 observations from our ACA dataset (Figure \ref{fig_offset}). Both the continuum and C$^{18}$O 3-2 line are detected at high signal-to-noise (see Section \ref{sec:methods}). A previous study has suggested that both the continuum and C$^{18}$O 2-1 emission may be slightly asymmetric, with emission peaks that are offset from the star by $\sim 0.08$" and $\sim 0.11$" respectively \citep{miley_2019}. However, such deviations are significantly smaller than the observed CS offset of $\sim1$" (by factors of $\sim 12.5$ and $\sim 9$). At the scale of our observations (using a $\sim 5$" synthesized beam and pixel size of 0.5"), any offset in the continuum or C$^{18}$O emission will be contained entirely within the central pixel. It is therefore reasonable to utilise the continuum and C$^{18}$O emission peaks as reference locations for the host star. We also note that the C$^{18}$O 3-2 spectrum presented here (Figure \ref{fig_offset}, bottom panel) is symmetric about the source velocity, in contrast to the C$^{18}$O 2-1 line presented in \citep{miley_2019}. This suggests that the layer traced by C$^{18}$O 3-2 emission is more symmetric that that traced by the 2-1 emission, and strengthens the case to use it as a frame of reference for the host star. By comparision, it is clear that the CS 7-6 emission is significantly offset from the central star, and the offset cannot therefore be related to the absolute astronometric precision of the observation. We created a residual map by subtracting the CS 7-6 integrated intensity map from the C$^{18}$O 3-2 integrated intensity map. A 1$\sigma$ clip was first applied to the CS data, and the C$^{18}$O peak flux was then scaled to match that of the CS. The residuals were then divided by the rms of the CS data.

Next, we calculate the uncertainty in the beam centroid, given by the following equation \citep{condon_1998}:

\begin{equation}
    \begin{split}
        \sigma_P \approx \frac{1}{2} \sigma \theta / S_P
    \end{split}
    \label{eq:condon98_eq1_1}
\end{equation}

where $\sigma_P$ is the positional uncertainty, $\sigma$ is the rms, $\theta$ is the beam FWHM, and $S_P$ is the peak flux density. Using our reported values of $\sigma = 0.1$ Jy beam$^{-1}$, $\theta = 4.78$", and $S_P = 0.9$ Jy beam$^{-1}$, we find:

\begin{equation}
    \begin{split}
        \sigma_P \approx \frac{1}{2} \times 0.1 \times 4.78 / 0.9 \approx 0.27"
    \end{split}
    \label{eq:condon98_eq1_2}
\end{equation}

We therefore find the CS 7-6 emission peak to be offset by $\sim 1 \pm 0.27$". Since the offset is a factor of $\sim 4$ times the uncertainty, we can be confident that it is at least partly related to a physical characteristic of the source.

Next, we assess the offset in the visibility plane, using the CASA task \emph{uvmodelfit} to fit a model directly to the UV-data. We first split out the spectral windows containing the CS 7-6 emission from continuum-subtracted measurement set, using only the channels centered $\pm 10$ km s$^{-1}$ from the source velocity. We then fit a Gaussian to the data, where the initial guess for shape is based on the synthesized beam properties, and the initial guess for the emission peak is the image centre (ie. location of the star). We run 20 iterations, with the model converging after $\sim 8$. The offset in the emission peak is found to be $0.88 \pm 0.23$" west and $0.68 \pm 0.25$" south, corresponding to a total absolute offset of $1.11 \pm 0.33$". Since \emph{uvmodelfit} can be sensitive to the initial conditions, we repeat the fitting using a range of initial guesses for the location of the emission peak. In all cases the models converge to close to the same result, deviating by a maximum of $\sim8$ \% from the value reported here.

Finally, we looked at data in the ALMA archive for other spectral line observations which might display similar asymmetries. We uncovered an unpublished observation of C$_2$H at 349 GHz (ALMA project 2017.1.00845.S, PI: E. Bergin), obtained at high angular resolution (beam size $\sim 0.3$", approximately 30 au). C$_2$H is a powerful tracer of the C/O ratio, with numerous studies having linked strong C$_2$H emission to chemical environments where C/O > 1 \citep[e.g.][]{Bergin_2016, miotello_2019, bergner_2019, maps_7_bosman2021}. An emission map created from the pipeline data products reveals that C$_2$H is concentrated in a ring at $\sim 230$ au, which approximately coincides with the outer edge of a millimeter dust ring. The total flux on the red-shifted side of the disk is a factor of $\sim 1.5$ greater that that on the shifted side, which may be indicative of an elevated C/O ratio. This is in agreement with our CS 7-6 observation, where the emission is significantly offset towards the red shifted side of the disk. We do not formally present the C$_2$H observatiom here, since a full analysis is well beyond the scope of this work. However, we note that the pipeline data products are freely available in the ALMA archive.


%% CBFs and ABUNDANCE MAPS %%%%%%%%%%%%%%%%%%%%%%%%%%%%%%%%%%%%%%%%%%%%%%%%%%%%%%%%%%%%%%%%%%%
\section{Abundance maps and contribution functions}
\label{sec:appendix_cbfs}

Figure \ref{fig_cbfs} shows CS and SO abundance maps, for both the C/O=0.5 region and C/O>1 wedge . Contribution functions for the CS 7-6 and SO 7$_7$-6$_6$+7$_8$-6$_7$ transitions are overlaid (25\% and 75\% contours).





%% WEDGE SIZE AND POSITION VARIATIONS %%%%%%%%%%%%%%%%%%%%%%%%%%%%%%%%%%%%%%%%%%%%%
\section{Varying the size and position of the high-C/O wedge}
\label{sec:appendix_wedge_variations}

Figure \ref{fig_wedge_variations} shows the effect of varying the size and position of the high-C/O wedge on the modelled spectra. The wedge size is described by the arc subtended by the angle $\theta$. The wedge position is described using the azimuthal angle $\phi$ (where $\phi=0$ points due west, and is measured positively anti-clockwise). A given angle $\phi$ corresponds to the position of the arc centre.

We vary the angular size of the high-C/O wedge between $\theta=20^\circ$ to $180^\circ$, while keeping the centre of the wedge arc fixed at $\phi=0$ (Figure \ref{fig_wedge_variations}, top panel). Asymmetries in the SO spectra are more pronounced for larger wedge angles, with increasing wedge sizes leading to decreasing peak fluxes on the red-shifted side of the spectrum. The CS spectrum remains highly asymmetric about the source velocity, regardless of wedge size, with larger wedges result in broader line profiles and higher peak fluxes.

We vary the angular position of the high-C/O wedge between $\phi = -60^\circ$ to $60^\circ$, while keeping the angular size fixed at $\theta=60^\circ$ (Figure \ref{fig_wedge_variations}, bottom panel). Asymmetries in the SO spectra are most pronounced when the wedge points towards $60^\circ$, since this corresponds to the region of the disk in which line-of-sight velocities are most highly red-shifted. Rotating the arc position clockwise through to $-60^\circ$ results in a incrementally smaller contribution to the SO spectra from the highly red-shifted part of the disk, and therfore a more symmetric double-peaked structure. Similarly, the CS spectrum is most highly red-shifted when the high-C/O wedge is positioned at $60^\circ$, with a narrow profile and high peak flux. As the arc position is rotated clockwise through to $-60^\circ$, the line becomes broader with smaller peak flux, with the peak of the line profile moving closer to the source velocity.





%% TEMPERATURE / DENSITY PLOTS %%%%%%%%%%%%%%%%%%%%%%%%%%%%%%%%%%%%%%%%%%%%%%%%%%%%%%%%%%%%%%%%%%%%%%%%%%%%%%
\section{Temperature and density maps}
\label{sec:appendix_tempdens}

Figure \ref{fig_temp_density} shows the temperature and density maps for dust and gas in the HD 100546 disk model.





%% TIMESCALES %%%%%%%%%%%%%%%%%%%%%%%%%%%%%%%%%%%%%%%%%%%%%%%%%%%%%%%%%%%%%%%%%%%%%%%%%%%%%%
\section{Cooling, freeze-out, and chemical timescales}
\label{sec:appendix_timescales}

% Intro paragraph...
We propose that material associated with a protoplanet in HD 100546's inner dust cavity casts a shadow on a region of the outer disk, resulting in the azimuthal chemical variations described in this work. In order for such material to cast a shadow covering an angular region of $60^\circ$, as prescribed by our model, the azimuthal extent of the material must be much larger than the host star. At a radial distance of 13 au, which is equivalent to the outer edge of the dust cavity, the necessary azimuthal extent of the shadowing material is $\sim 15$ au. Additionally, the material must extend vertically such that the shadow enables freeze-out of H$_2$O in the upper layers of the disk atmosphere. Taking $r=100$ au and $z/r=0.25$ as a representative location for excess H$_2$O freeze-out (via comparison with the modelled H$_2$O snow surface, see Figure \ref{fig_tdust_snowsurface}), the necessary vertical extent of the shadowing material is $\sim 3.2$ au. This is reasonable, since hydrodynamic simulations of disks show that dust concentrations associated with large embedded protoplanets can be highly elongated, especially in the $\phi$ direction \citep[e.g.][]{zhu_2014}.

If the material responsible for shadowing orbits at 13 au, it will complete one orbit in $\sim 30$ years. A shadow covering an azimuthal region of $60^\circ$ will therefore transit a given region of the disk in $30 \times (60/360) = 5$ years. However, since the shadowed material is in orbit itself, the total time spent in shadow will vary as a function of orbital separation. Shadowing timescales are longest in the inner disk, where the orbital period of the material being shadowed is comparible to the orbital period of the shadow. This falls off rapidly with radius, approaching the minimum shadowing time of $\sim5$ years in the outer disk. Since reproducing the offset in the CS emission requires a large increase in the CS abundance in the outer disk, we take 5 years as our upper limit on the time available for changes in the chemistry to occur such that the C/O ratio becomes super-solar. To assess the feasibility of the shadowing scenario, we present a simple analytical model that takes into consideration three important timescales; cooling, freeze-out, and chemical conversion.

% Cooling...
The cooling timescale for a single dust grain is given by:

\begin{equation}
    \begin{split}
        t_\text{cooling} &= \frac{mC\Delta T}{Q_\text{rad}P}
    \end{split}
    \label{eq:cooling_timescale}
\end{equation}

where $m$ is the mass of the dust grain, $C$ is the specific heat capacity, $\Delta T$ is the change in temperature, and $P=A\sigma T^4$ is the radiative power of the dust grain (where A is the grain surface area, and $\sigma$ is the Stefan-Boltzmann constant). $Q_\text{rad}$ is the Planck mean efficiency for dust grains at temperatures between 10-250 K (where $Q_\text{rad} = 1.25 \times 10^{-5} T_d^2 (r/1\mu m)$, where $r$ is the grain radius \citep{tielens_book}).

We make the simplifying assumption that gas in the disk atmosphere is coupled to a population of small dust grains of equal size $r=1\mu m$. Taking $\rho = 4$ g cm$^{-3}$ as a typical value for grain density (based on silicate materials such as olivine), the grain mass is then $m=\rho(4/3 \pi r^3) \approx 1.68 \times 10^{-14}$ kg. We take $C=2000$ J kg$^{-1}$ K$^{-1}$, which is again typical of silicate material. Finally, we make the simplification that all H$_2$O is in gaseous form at 100 K and in ice at 50 K i.e. $\Delta T = 50$ K. Note that these approximate values are lower than the canonical freezeout temperature for H$_2$O to account for the effects of photodesorption. For grains with radius $r=1 \mu m$ at 100 K, $Q_\text{rad} = 0.125$. The cooling timescale can then be calculated as:

\begin{equation}
    \begin{split}
        t_\text{cooling} &= \frac{1.68 \times 10^{-14} \times 2000 \times (100-50)}{0.125 \times 4\pi (10^{-6})^2 \times 5.67 \times 10^{-8} \times 100^4} \\
          &\approx 189 \text{ s}
    \end{split}
    \label{eq:cooling_timescale2}
\end{equation}


% Freeze-out...
We now consider the freeze-out timescale, which is dictated by the efficiency with which gas-phase H$_2$O molecules collide with cooled grains. The mean free path of a gas molecule is given by:

\begin{equation}
    \mu = \frac{1}{n\pi r^2}
    \label{eq:mean_free_path}
\end{equation}

where $n$ is the number density of the dust grains, and $r$ is the grain radius. We extract values for $n$ directly from our chemical model, from a range of grid cells that trace the location of the H$_2$O freeze-out in the upper disk atmosphere, then reduce the value by a factor of 100 to account for dust settling to the midplane.

The average gas speed of a molecule is given by:

\begin{equation}
    v_\text{rms} = \sqrt{\bar{v}^2} = \sqrt{\frac{3kT}{m}}
    \label{eq:gas_speed}
\end{equation}

where $k$ is the Boltzmann constant and $m$ is the mass of the H$_2$O molecule. The freeze-out timescale, which is equal to the collision timescale, is then given by:

\begin{equation}
    \begin{split}
        t_\text{freeze} &= \frac{\mu}{v_\text{rms}} \\
    \end{split}
    \label{eq:freeze-out}
\end{equation}

Note that it is not necessary for the bulk of the gas to cool down to the dust temperature for freezeout to occur when the gas and dust are thermally decoupled, as is the case in the disk atmosphere (Extended Data Fig. 3). The increase in dust temperature due to a collision will be extremely small, and since the dust cooling timescale is extremely short this can be considered negligible. The timescale for the dust and gas to reach thermal equilibrium will be longer that the timescale derived here, but our calculation is reasonable when considering freezeout in the conditions described above.

We compute $t_\text{freeze}$ for a range of gas-to-dust ratios between $\Delta_\text{g/d} = 10 - 390$, where the lower limit is taken from \citep{Kama2016b} and the upper limit is based on constraints from modelling HD lines (see Section \ref{subsec:chemical_modelling}). We find that $t_\text{freeze}$ can vary from $\sim 47$ hours to $4500$ years between $r=13$ to $300$ au, depending on $\Delta_\text{g/d}$. These values are subject to large uncertainties because $n_\text{dust}$ can vary quite dramatically depending on the precise location of the disk atmosphere from which it is extracted, in addition to uncertainties in $\Delta_\text{g/d}$ due to dust settling and radial drift. While the simple calculations we present here relate to a single grain, these can reasonably be used as an approximation for the entire local grain population in the optically thin regime, which is a reasonable assumption for material in the upper surface layers of the disk.


% Chemical...
Finally, we consider the relevant chemical timescales. We simulate the effects of shadowing by first extracting the final elemental and molecular abundances from our `unshadowed' C/O=0.5 model, using these as an input for the shadow model. The total elemental gas-phase oxygen is then depleted by setting the input H$_2$O abundance to zero (representing total freeze-out). This complete removal of H$_2$O is not enough to increase the gas-phase C/O above unity, since the majority of element oxygen is tied up in atomic O and CO. However, our chemical models show that when H$_2$O is removed from the gas phase on such a large scale, the following chemical rebalance leads to a significant production of new gas-phase H$_2$O via the following pathway:

\begin{gather}
    \text{H}_2 \text{ + O} \rightarrow \text{OH + H}
    \label{eq:chem1}\\
    \text{H}_2 \text{ + OH} \rightarrow \text{H}_2 \text{O + H}
    \label{eq:chem2}
\end{gather}

The subsequent freeze-out of this newly-produced gas-phase H$_2$O therefore provides a route by with atomic oxygen can be removed from the gas phase, further elevating the C/O ratio. Additional oxygen (and carbon) may be removed by the conversion of gas-phase CO into CO$_2$ ice, via a surface reaction with OH radicals produced by UV irradiation of amorphous solid water:

\begin{gather}
    \text{CO + OH} \rightarrow trans-\text{HOCO}
    \label{eq:co2_1}\\
    trans-\text{HOCO} \rightarrow cis-\text{HOCO}
    \label{eq:co2_2}\\
    cis-\text{HOCO} \rightarrow \text{CO}_2 \text{ + H}
    \label{eq:co2_3}
\end{gather}

We can expect reactions \ref{eq:co2_1} - \ref{eq:co2_3} to be enhanced in the shadowed region, since the lower temperature locks a larger fraction of H$_2$O into ices. Laboratory experiments show that this pathway can permanently sequester gas-phase CO into CO$_2$ ice at temperatures between 40-60 K (well above the canonical CO freeze-out temperature), with an efficiency of up to 27\% \citep{van_scheltinga_2022}. While reactions \ref{eq:co2_1} - \ref{eq:co2_3} are not included in our chemical network, we take them into account by manually adjusting the abundances as follows. We adjust the initial conditions of our shadow model, depleting not only H$_2$O, but also a fraction of atomic O and CO, such that the resulting gas-phase C/O > 1.0 (see Section \ref{subsec:chemical_modelling}). We let the chemistry evolve and extract the CS and SO abundances as a function of time from key regions of the disk. We find that the newly elevated C/O ratio favours both the formation of CS and destruction of SO, primarily via the reactions:

\begin{gather}
    \text{C + SO} \rightarrow \text{CS + O}
    \label{eq:chem3}\\
    \text{C + SO} \rightarrow \text{CO + S}
    \label{eq:chem4}
\end{gather}

These reactions have no activation barrier and so can proceed in cold gas. CS is formed at a faster rate than SO is destroyed, since there are prominent CS formation pathways in addition to reaction \ref{eq:chem3}:

\begin{gather}
    \text{CH}_2 \text{ + S} \rightarrow \text{CS + H}_2
    \label{eq:chem5}\\
    \text{C + HS} \rightarrow \text{CS + H}
    \label{eq:chem6}
\end{gather}

The effect of this time evolution on the CS and SO abundances and disk integrated line fluxes is illustrated in Figure \ref{fig_chemistry}. The CS 7-6 disk-integrated line flux steadily increases with time in the super-stellar C/O environment, with the SO 7$_7$-6$_6$ disk-integrated line flux showing the opposite trend. To approximate the timescale at which this phenomenon becomes apparent in observations, we define the chemical timescale as the period in which the CS abundance increases by a factor of two, and the SO abundance decreases by a factor of two, whichever is longest (see Figure \ref{fig_chemistry}). We calculate the chemical timescale at different radial regions in the disk, by extracting the variations in CS and SO abundance from key grid cells in the model (Figure \ref{fig_chemistry}, lower panels). We find that the chemical timescale varies between $\sim 2 \times 10^{-4}$ to $\sim 5$ years, for radial regions between 13-300 au.

The combined results for cooling, freeze-out, and chemical conversion are illustrated in Figure \ref{fig_timescales}. The total time is almost entirely dependent on the freeze-out timescale, since the cooling and chemical timescales are negligable compared to the shadow transit time. We show that, within the $\sim 5$ year period it takes the shadow to transit a given region of the disk, the gas-phase C/O ratio can become super-solar within the inner $\sim 200$ au of the disk (within $\sim 150$ au for $\Delta_\text{g/d}=100$). Considering the fact that prominent emission from both CS and SO emanates from inside of $\sim 200$ au (\ref{sec:appendix_cbfs}), and the uncertainties associated with the freeze-out timescale, we conclude that shadowing is a plausible means by which the local C/O ratio can become elevated, leading to the spatial and spectral asymmetries presented in this work. 

A more sophisticated model could take into account the reverse timescale, and the cyclical nature of disk shadowing, by addressing the effects of many transits over the lifetime of the disk. Such an investigation is beyond the scope of this work.



%% CS CHANNEL MAPS %%%%%%%%%%%%%%%%%%%%%%%%%%%%%%%%%%%%%%%%%%%%%%%%%%%%%%%%%%%%%%%%%%%%%%%%%%%%%
\section{CS 7-6 channel maps}
\label{sec:appendix_channel_maps}

Figure \ref{fig_channel_maps} shows the individual channel maps for the CS 7-6 detection with the ACA.



%% CO and CI SPECTRA %%%%%%%%%%%%%%%%%%%%%%%%%%%%%%%%%%%%%%%%%%%%%%%%%%%%%%%%%%%%%%%%%%%%%%%%%%%%%
\section{CO 6-5, CO 3-2, and [CI] spectra}
\label{sec:appendix_co_ci}

Figure \ref{fig_coci} shows the modelled and observed CO 6-5, CO 3-2, and [CI] 1-0 spectral lines, used to constrain the level of CO depletion in the high-C/O wedge (see Section \ref{subsec:chemical_modelling}).


%% MODEL PARAMETERS %%%%%%%%%%%%%%%%%%%%%%%%%%%%%%%%%%%%%%%%%%%%%%%%%%%%%%%%%%%%%%%%%%%
\section{Disk model parameters}
\label{sec:appendix_modelparams}

The parameters used in our disk models are listed in Table \ref{table:modelparameters}.




%% OBSERVATIONAL DATA %%%%%%%%%%%%%%%%%%%%%%%%%%%%%%%%%%%%%%%%%%%%%%%%%%%%%%%%%%%%%%%%%%%%%%%%%%%%%%
\section{Observational data}
\label{sec:appendix_obs_data}

The observational parameters for each of the spectral windows in the ACA dataset are given in Table \ref{table:aca_observations}. The full list of line fluxes and upper limits used to constrain our model is given in Table \ref{table:model_constraints}.

\clearpage





%%%%%%%%%%%%%%%%%%%%%%%%%%%%%%%%%%%%%%%%%%%%%%%%%%%%%%%%%%%
%%%%%%%%%%%%%%%%%%%%% FIGURES %%%%%%%%%%%%%%%%%%%%%%%%%%%%%
%%%%%%%%%%%%%%%%%%%%%%%%%%%%%%%%%%%%%%%%%%%%%%%%%%%%%%%%%%%


% CONTINUUM and C18O
\begin{figure*}[h!]
\centering
\includegraphics[width=\textwidth]{fig_offset_new.png}
\caption{\textbf{Comparison between continuum, C$^{18}$O, and CS ACA data}. \emph{Top left: }Continuum emission map. \emph{Top right: } C$^{18}$O 3-2 integrated intensity map. \emph{Middle left: }CS 7-6 integrated intensity map. \emph{Middle right: }Residual map created by subtracting the CS emission from the C$^{18}$O emission. The C$^{18}$O flux was scaled to match the peak flux of the CS emission, and a 1$\sigma$ clip was applied to the CS emission. The residuals are divided by the rms of the CS data, from which it is evident that the CS emission is significantly offset from the star (denoted with `x'). \emph{Bottom: }C$^{18}$O 3-2 spectrum.}
\label{fig_offset}
\end{figure*}

\clearpage



% CBFs and ABUNDANCE MAPS
\begin{figure*}[hbt!]
\centering
\includegraphics[width=\textwidth]{fig_abundance_maps_viridis.png}
\caption{\textbf{Abundance maps and contribution functions for the modelled CS and SO emission in HD 100546}. Each panel shows an abundance map overlaid with contours representing 25\% and 75\% line emission (white). Top: CS 7-6 emission from the C/O=0.5 region (\emph{left}) and C/O>1 wedge (\emph{right}). Bottom: SO $7_7-6_6$+$7_8-6_7$ emission from the C/O=0.5 region (\emph{left}) and C/O>1 wedge (\emph{right}).}
\label{fig_cbfs}
\end{figure*}

\clearpage


% WEDGE VARIATIONS
\begin{figure*}[hbt!]
\centering
\includegraphics[clip=,width=\linewidth]{fig_wedge_variations.png}
\caption{\textbf{Effect of varying the high-C/O wedge size and position on the modelled spectra}. \emph{Top panel: }SO 7$_7$-6$_6$+7$_8$-6$_7$ (left) and CS $7-6$ (right) spectra for variations in wedge size ($\theta$), centered on position $\phi=0$. \emph{Bottom panel: }SO 7$_7$-6$_6$+7$_8$-6$_7$ (left) and CS $7-6$ (right) spectra for variations in wedge position ($\phi$), for a fixed angular size $\theta=60^\circ$.}
\label{fig_wedge_variations}
\end{figure*}

\clearpage


% TEMPERATURE / DENSITY PLOTS
\begin{figure*}[hbt!]
\centering
\includegraphics[width=\textwidth]{fig_temp_density.png}
\caption{\textbf{Temperature and density maps for the baseline HD 100546 disk model (C/O=0.5)}.\emph{ Top left: }Gas number density. \emph{Bottom left: }Dust number density. \emph{Top right: }Gas temperature. \emph{Bottom right: }Dust temperature.}
\label{fig_temp_density}
\end{figure*}

\clearpage


% TIMESCALES
\begin{figure*}[ht!]
\centering
\includegraphics[clip=,width=0.9\linewidth]{fig_chemistry.png}
\caption{\textbf{Time-dependent disk-integrated fluxes and abundances in the high C/O wedge}. \emph{Top panel: }Disk-integrated CS $7-6$ (green) and SO $7_7-6_6$ (purple) fluxes, as a function of time in the region where C/O>1. \emph{All other panels: } CS (green) and SO (purple) abundances at different radial locations, as a function of time in the region where C/O>1. All values are extracted from the model at z/r $\sim 0.25$. Dotted lines represent the point in time where the CS flux increases by a factor of two from its initial values (green), and the SO flux decreases by a factor of two from its initial value (purple). The longer of these two times is taken as the chemical timescale at that particular radius.}
\label{fig_chemistry}
\end{figure*}

\clearpage


% CHANNEL MAPS
\begin{figure*}
\centering
\includegraphics[clip=,width=1\linewidth]{fig_channelmaps_cs_viridis.png}
\caption{\textbf{CS 7-6 channel maps at a spectral resolution of 0.5 km s$^{-1}$}. Contours are at 35\%, 50\%, 65\%, and 80\% peak flux. The source velocity is $V_\text{LSRK} = 5.7 \text{ km s}^{-1}$. The white cross denotes the position of the star.}
\label{fig_channel_maps}
\end{figure*}

\clearpage


% CO and CO SPECTRA
\begin{figure*}
\centering
\includegraphics[clip=,width=0.5\linewidth]{fig_co_ci_spectra.png}
\caption{\textbf{Modelled and observed CO and [CI] spectra}. Modelled CO 6-5 (blue), CO 3-2 (green), and [CI] 1-0 (purple) spectra. APEX observations in grey. The model reproduces the asymmetry of the CO 6-5 spectral line (but underpredicts the peak CO 6-5 flux, as in previous studies of HD~100546 \citep{Kama2016b}).}
\label{fig_coci}
\end{figure*}

\clearpage



% MODEL PARAMETERS
\begin{table*}
\caption{HD~100546 disk model parameters.}             
\label{table:modelparameters}      
\centering
\begin{tabular}{l l l}     % 
\hline\hline       
                      
Parameter & Description & Fiducial\\ 
\hline                    
   \rsub                & Sublimation radius                       & 0.25 au                               \\
   \rgap                & Inner disk size                          & 1.0 au                                \\
   \rcav                & Cavity radius                            & 13 au                                 \\
   $R_\text{out}$       & Disk outer radius                        & 1000 au                               \\
   $R_c$                & Critical radius for surface density      & 50 au                                 \\
   \deltagas            & Gas depletion factor inside cavity       & 1                                     \\
   \deltadust           & Dust depletion factor inside cavity      & $10^{-4}$                             \\
   $\gamma$             & Power law index of surface density profile   & 1.0                                   \\
   $\chi$               & Dust settling parameter                      & 0.2                                   \\
   $f$                  & Large-to-small dust mixing parameter         & 0.85                                  \\
   $\Sigma_c$           & $\Sigma_\text{gas}$ at $R_c$                 & 82.75 g cm$^{-2}$                     \\
   $h_c$                & Scale height at $R_c$                        & 0.10                                  \\
   $\psi$               & Power law index of scale height              & 0.20                                  \\
   \gasdust             & Gas-to-dust mass ratio                       & 100                                   \\
   $L_*$                & Stellar luminosity                           & $36\; L_\odot$                        \\
   $L_X$                & Stellar X-ray luminosity                     & $7.94 \times 10^{28} \text{ erg s}^{-1}$    \\
   $T_X$                & X-ray plasma temperature                     & $7.0 \times 10^{7}$ K                 \\
   $\zeta_\text{cr}$    & Cosmic ray ionization rate                   & $3.0 \times 10^{-17}$ s$^{-1}$        \\
   $M_\text{gas}$       & Disk gas mass                                & $1.45 \times 10^{-1}$ \msun           \\
   $M_\text{dust}$      & Disk dust mass                               & $1.12 \times 10^{-3}$ \msun           \\
   $\text{[C/H]}_\text{gas}$        & Initial carbon abundance (relative to hydrogen)  & $1.0 \times 10^{-5}$ (where C/O=0.5)\\
   $\text{[O/H]}_\text{gas}$        & Initial oxygen abundance (relative to hydrogen)   & $2.0 \times 10^{-5}$ (where C/O=0.5)\\
   $t_\text{chem}$      & Timescale for time-dependent chemistry   & 5 Myr (where C/O=0.5) \\
\hline                  
\end{tabular}
\end{table*}







% OBSERVATIONAL DATA
\begin{table*}
\caption{Observational parameters for molecular species in HD 100546 from the ACA (program 2016.1.01339.S). Upper limits are at 3$\sigma$ level, denoted by `<'. Detections are given for Keplerian masked cubes (\emph{a}) and elliptically masked cubes (\emph{b}).}
\label{table:aca_observations}      

\centering

\begin{tabular}{l l l l l l l l l}     % 9 columns 
\hline\hline       
                      
Molecule & Transition & $\nu$ (GHz) & $\Delta_{\nu}$ (GHz) & E$_\text{up}$ (K) & A$_\text{ul}$ (s$^{-1}$) & Beam Size & RMS (Jy beam$^{-1}$ km s$^{-1}$) & Flux (Jy km s$^{-1}$)\\
\hline
  SO                & $2_1$-$1_0$              &  $329.385$   & $0.250$ & $15.8$ & $1.423 \times 10^{-5}$   & $4.83" \times 4.08"$ &    $0.14$     &  $< 0.66$   \\
  SO$_2$            & $4_{3,1}$-$3_{2,2}$      &  $332.505$   & $0.250$ & $31.3$ & $3.290 \times 10^{-4}$   & $4.84" \times 4.07"$ &    $0.10$     &  $< 0.48$   \\
  C$^{36}$S         & $7-6$                    &  $332.510$   & $0.250$ & $63.9$ & $6.951 \times 10^{-4}$   & $4.78" \times 4.07"$ &    $0.10$     &  $< 0.46$   \\
  HCS$^+$           & $8-7$                    &  $341.339$   & $0.250$ & $73.7$ & $8.352 \times 10^{-4}$   & $4.62" \times 4.03"$ &    $0.14$     &  $< 0.71$   \\
  CS                & $7-6$                    &  $342.883$   & $0.250$ & $65.8$ & $8.368 \times 10^{-4}$   & $4.78" \times 4.06"$ &    $0.10$     &  $0.62^a$, $1.02^b$      \\
  H$_2$CS           & $10_{0,10}$-$9_{0,9}$    &  $342.946$   & $0.250$ & $90.6$ & $6.080 \times 10^{-4}$   & $4.73" \times 3.90"$ &    $0.31$     &  $< 1.50$    \\
  $^{34}$SO         & $2_3$-$2_1$              &  $343.851$   & $0.250$ & $20.9$ & $1.382 \times 10^{-7}$   & $4.63" \times 3.89"$ &    $0.31$     &  $< 1.60$    \\
  SO                & $8_8$-$7_7$              &  $344.310$   & $0.250$ & $87.5$ & $5.188 \times 10^{-4}$   & $4.62" \times 3.89"$ &    $0.30$     &  $< 1.50$    \\
  C$^{18}$O         & $3-2$                    &  $329.330$   & $0.250$ & $31.6$ & $2.172 \times 10^{-6}$   & $4.96" \times 4.21"$ &    $0.30$     &  $5.41^a$, $8.22^b$\\
\hline                  
\end{tabular}
\end{table*}



\clearpage

 \onecolumn

 
\begin{table*}
\centering
\caption{Disk-integrated line fluxes and upper limits used to constrain our model.}\label{table:model_constraints}
\scalebox{0.7}{
\begin{tabular}{llll}
\toprule
Molecule          & Transition     & Flux (W m$^{-2}$)               & Reference \\
\midrule 
CO                & 3-2            & $1.72\pm 0.04 \times 10^{-18}$   & \citep{Kama2016b} \\
CO                & 6-5            & $1.61\pm 0.08 \times 10^{-17}$   & \citep{Kama2016b} \\
CO                & 7-6            & $2.35\pm 0.27 \times 10^{-17}$   & \citep{vanderwiel2014} \\
CO                & 8-7            & $3.31\pm 0.35 \times 10^{-17}$   & \citep{vanderwiel2014} \\
CO                & 9-8            & $4.53\pm 0.41 \times 10^{-17}$   & \citep{vanderwiel2014} \\
CO                & 10-9           & $5.53\pm 0.37 \times 10^{-17}$   & \citep{vanderwiel2014} \\
CO                & 11-10          & $5.13\pm 0.45 \times 10^{-17}$   & \citep{vanderwiel2014} \\
CO                & 12-11          & $5.83\pm 0.33 \times 10^{-17}$   & \citep{vanderwiel2014} \\
CO                & 13-12          & $6.07\pm 0.47 \times 10^{-17}$   & \citep{vanderwiel2014} \\
CO                & 14-13          & $6.00\pm 0.83 \times 10^{-17}$   & \citep{meeus2012} \\
CO                & 15-14          & $8.83\pm 0.97 \times 10^{-17}$   & \citep{meeus2012} \\
CO                & 16-15          & $5.88\pm 0.97 \times 10^{-17}$   & \citep{meeus2012} \\
CO                & 17-16          & $7.39\pm 1.0 \times 10^{-17}$    & \citep{meeus2012} \\
CO                & 18-17          & $7.15\pm 0.69 \times 10^{-17}$   & \citep{meeus2012} \\
CO                & 19-18          & $6.47\pm 0.85 \times 10^{-17}$   & \citep{meeus2012} \\
CO                & 20-19          & $4.99\pm 0.57 \times 10^{-17}$   & \citep{meeus2012} \\
CO                & 21-20          & $6.50\pm 0.88 \times 10^{-17}$   & \citep{meeus2012} \\
CO                & 22-21          & $\leq4.2 \times 10^{-17}$        & \citep{meeus2012} \\
CO                & 23-22          & $7.84\pm 1.1 \times 10^{-17}$    & \citep{meeus2012} \\
CO                & 24-23          & $7.13\pm 1.2 \times 10^{-17}$    & \citep{meeus2012} \\
CO                & 25-24          & $\leq7.71 \times 10^{-17}$       & \citep{meeus2012} \\
CO                & 28-27          & $8.15\pm 1.1 \times 10^{-17}$    & \citep{meeus2012} \\
CO                & 29-28          & $7.86\pm 1.9 \times 10^{-17}$    & \citep{meeus2012} \\
CO                & 30-29          & $7.34\pm 1.5 \times 10^{-17}$    & \citep{meeus2012} \\
CO                & 31-30          & $\leq1.40 \times 10^{-16}$       & \citep{meeus2012} \\
CO                & 32-31          & $\leq6.53 \times 10^{-17}$       & \citep{meeus2012} \\
CO                & 33-32          & $\leq8.45 \times 10^{-17}$       & \citep{meeus2012} \\
CO                & 34-33          & $5.31\pm 1.4 \times 10^{-17}$    & \citep{meeus2012} \\
CO                & 35-34          & $\leq4.41 \times 10^{-17}$       & \citep{meeus2012} \\
CO                & 36-35          & $5.29\pm 1.3 \times 10^{-17}$    & \citep{meeus2012} \\
CO                & 37-36          & $\leq8.29 \times 10^{-17}$       & \citep{meeus2012} \\
CO                & 38-37          & $\leq1.07 \times 10^{-16}$       & \citep{meeus2012} \\
$^{13}$CO         & 3-2            & $\leq 6.6 \times 10^{-19}$       & \citep{panic2010} \\
$^{13}$CO         & 6-5            & $\leq 7.5 \times 10^{-18}$       & \citep{vanderwiel2014} \\
$^{13}$CO         & 7-6            & $\leq 7.2 \times 10^{-18}$       & \citep{vanderwiel2014} \\
$^{13}$CO         & 8-7            & $\leq 1.02 \times 10^{-17}$      & \citep{vanderwiel2014} \\
$^{13}$CO         & 9-8            & $\leq 1.44 \times 10^{-17}$      & \citep{vanderwiel2014} \\
$^{13}$CO         & 11-10          & $\leq 1.68 \times 10^{-17}$      & \citep{vanderwiel2014} \\
OI                & 145$\mu$m      & $3.57\pm 0.13 \times 10^{-16}$   & \citep{meeus2012, fedele2013} \\
\toprule
\end{tabular}

\hspace{2em}

\begin{tabular}{llll}
\toprule
Molecule          & Transition     & Flux (W m$^{-2}$)               & Reference \\
\midrule 
OI                & 63$\mu$m       & $5.54\pm 0.05 \times 10^{-15}$   & \citep{meeus2012,fedele2013} \\
CI                & 1-0            & $6.60\pm 2.0 \times 10^{-19}$    & \citep{Kama2016b} \\
CI                & 2-1            & $\leq 3.58 \times 10^{-18}$      & \citep{Kama2016b} \\
CII               & 158$\mu$m      & $1.35\pm 0.15 \times 10^{-16}$   & \citep{meeus2012,fedele2013} \\
HD                & 112$\mu$m      & $\leq2.7 \times 10^{-16}$        & \citep{Kama2016b} \\
HD                & 56$\mu$m       & $\leq1.6 \times 10^{-17}$        & \citep{fedele2013} \\
C$_2$H            & 29             & $\leq4.8\times 10^{-21}$         & \citep{Kama2016b} \\
C$_2$H            & 30             & $\leq4.8\times 10^{-21}$         & \citep{Kama2016b} \\
C$_2$H            & 31             & $\leq4.8\times 10^{-21}$         & \citep{Kama2016b} \\
C$_2$H            & 32             & $\leq4.8\times 10^{-21}$         & \citep{Kama2016b} \\
C$_2$H            & 33             & $\leq4.8\times 10^{-21}$         & \citep{Kama2016b} \\
C$_2$H            & 34             & $\leq4.8\times 10^{-21}$         & \citep{Kama2016b} \\
C$_2$H            & 35             & $\leq4.8\times 10^{-21}$         & \citep{Kama2016b} \\
C$_2$H            & 36             & $\leq4.8\times 10^{-21}$         & \citep{Kama2016b} \\
C$_2$H            & 37             & $\leq4.8\times 10^{-21}$         & \citep{Kama2016b} \\
C$_2$H            & 38             & $\leq4.8\times 10^{-21}$         & \citep{Kama2016b} \\
C$_2$H            & 39             & $\leq4.8\times 10^{-21}$         & \citep{Kama2016b} \\
HCO$^+$           & 1-0            & $\leq7.0\times 10^{-20}$         & \citep{Kama2016b} \\
HCO$^+$           & 4-3            & $6.81\pm 1.0 \times 10^{-20}$    & \citep{Kama2016b} \\
H$_2$O            & 63.32          & $\leq2.59\times 10^{-17}$        & \citep{meeus2012} \\
H$_2$O            & 71.946         & $\leq5.09\times 10^{-17}$        & \citep{meeus2012} \\
H$_2$O            & 78.74          & $\leq5.70\times 10^{-17}$        & \citep{meeus2012} \\
H$_2$O            & 179.52         & $3.05\pm 4.8\times 10^{-17}$     & \citep{sturm_2010}\\
H$_2$O            & 180.42         & $\leq1.71\times 10^{-17}$        & \citep{meeus2012} \\
H$_2$O            & 90.00          & $1.160\pm 0.161 \times 10^{-16}$ & \citep{sturm_2010}\\
H$_2$O            & 557 GHz        & $1.41\pm 0.0259 \times 10^{-18}$ & \citep{Du_2017} \\
H$_2$O            & 1113 GHz       & $5.09\pm 0.0844 \times 10^{-18}$ & \citep{Du_2017} \\
H$_2$O            & 1153 GHz       & $\leq1.98\times 10^{-17}$        & \citep{Du_2017} \\
SO                & 7$_7$-6$_6$          & $1.25 \times 10^{-21}$     & \citep{booth_2022} \\
SO                & 7$_8$-6$_7$          & $1.45 \times 10^{-21}$     & \citep{booth_2022} \\
C$^{18}$O         & 3-2                       & 9.03$\times 10^{-20}$         & This work \\
SO                & 2$_1$-1$_0$               & $\leq 7.25\times 10^{-21}$    & This work \\
SO$_2$            & 4$_{3,1}$-3$_{2,2}$       & $\leq 5.32\times 10^{-21}$    & This work \\
C$^{36}$S         & 7-6                       & $\leq 5.10\times 10^{-21}$    & This work \\
HCS$^+$           & 8-7                       & $\leq 8.08\times 10^{-21}$    & This work \\
CS                & 7-6                       & 1.167$\times 10^{-20}$        & This work \\
H$_2$CS           & 10$_{0,10}$-9$_{0,9}$     & $\leq 1.72\times 10^{-20}$    & This work \\
$^{34}$SO         & 2$_3$-2$_1$               & $\leq 1.84\times 10^{-20}$    & This work \\
SO                & 8$_8$-7$_7$               & $\leq 1.72\times 10^{-20}$    & This work \\
\toprule
\end{tabular}}
\end{table*}



 
\twocolumn



\clearpage


