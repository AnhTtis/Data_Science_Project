\section{Introduction}
\label{sec:intro}
\begin{figure}[t!]
  \centering
     \includegraphics[width=0.78
     \linewidth]{./figs/fig1-09.pdf}
  \caption{(a). Illustrations of four different domain shifts in medical images. (b). Overview of different UDA settings and how MAPSeg can fit into different scenarios.}
  \label{fig:overview}
  \vspace{-1.5em}
\end{figure}

Quantitative measures from medical scans serve as biomarkers for various types of medical research and clinical practice. For instance, neurodevelopmental studies utilize metrics such as brain volume and cortex thickness/surface area from infant brain magnetic resonance imaging (MRI) to investigate the early brain development and neurodevelopmental disorders~\cite{cruz2023cortical, shen2022subcortical, baribeau2019structural, hazlett2017early}. Therefore, robust segmentation of medical images acquired from large-scale, multi-center, and longitudinal studies is desired, yet often challenged by the domain shifts across different imaging techniques and even within a single modality (\hyperref[fig:overview]{Fig.1a}). For example, computed tomography (CT) and MRI provide markedly different signals for the same structure (\eg, cardiac regions, \hyperref[fig:overview]{Fig.1a}). MRI, a widely adopted radiation-free imaging technique, bears various types of inherent heterogeneity, including cross-sequence (\eg, distinct contrasts for the same tissue in T1/T2 sequences) and cross-site (\eg, contrast of the same tissue in the same sequence varies  with acquisition scanner and setup). Moreover, subject-dependent physiological changes also lead to domain shift. For example, contrasts of white matter and grey matter vary while the human brain undergoes significant growth and expansion within both cortical and subcortical regions during early postnatal years~\cite{gilmore2018imaging}, which contributes to the cross-age domain shift (\hyperref[fig:overview]{Fig.1a}).

The prevalent heterogeneities in medical images lead to suboptimal performance when deep neural networks trained in one source domain are applied to another target domain. To address this challenge, we introduce a \textit{unified} unsupervised domain adaptation (UDA) framework for volumetric and heterogeneous medical image segmentation, named Masked Autoencoding and Pseudo-Labeling Segmentation (\textbf{MAPSeg}). To the best of our knowledge, MAPSeg is the first framework that can be used in \textbf{centralized}, \textbf{federated}, and \textbf{test-time} UDA for volumetric medical image segmentation while maintaining comparable performance. This versatility is particularly advantageous in the field of medical image segmentation, where data sharing is restricted and annotations are expensive. While centralized UDA delivers the best performance in most cases, the strict requirement of co-located data limits its application in multi-institutional studies due to regulations such as the Health Insurance Portability and Accountability Act (HIPAA) and EU General Data Protection Regulation (GDPR)~\cite{MAL-083, insight_from_GDPR}. MAPSeg circumvents this restriction with federated and test-time adaptation, enabling clinical and research collaboration across different medical centers. In contrast, some previous studies, despite showing promising results in one scenario, may become infeasible or suffer significant performance drop in others due to the requirement for co-located data or synchronous adaptation.

In addition, we conduct extensive experiments on a private infant brain MRI dataset, which includes expert-provided annotations, to evaluate MAPSeg on cross-sequence, cross-site, and cross-age adaptation tasks. MAPSeg is also compared with previously reported state-of-the-art (SOTA) results on a public cardiac CT $\rightarrow$ MRI segmentation task. MAPSeg consistently outperforms previous SOTA methods by a large margin (10.5 Dice improvement on the private MRI dataset and 5.7 on the public CT-MRI dataset in the centralized UDA setting). While previous studies have separately explored one of the abovementioned domain shifts~\cite{iseg,9741336, Cui_structure_driven}, they may not generalize to others. For example, cross-age domain shift is mainly composed of changes in brain size and contrast, and methods based on image-translation fail to handle it as they also change the size when translating data from target domain to source domain, leading to segmentation errors. We systematically evaluate MAPSeg across various domain shifts and imaging modalities, demonstrating its consistent and generalizable effectiveness. 

Moreover, in all three UDA settings, MAPSeg does not rely on any target labels for model validation and selection. On the contrary, some previous studies on cardiac CT $\rightarrow$ MRI segmentation~\cite{8988158, Chen_Dou_Chen_Qin_Heng_2019} validate and select the best model using labeled target data, which may not be readily available in real-world problems. We demonstrate that MAPSeg surpasses the previous SOTA results without using any target label for validation, and the performance drop between using and without using target label is minor (0.9 mean Dice). This further justifies its practical value in real-world medical image segmentation tasks. The  contributions of this study are multi-fold: 
\begin{enumerate}
  \item We propose MAPSeg, a unified UDA framework capable of handling various domain shifts in medical image segmentation. 
  \item MAPSeg is suitable for universal UDA scenarios, suggesting its versatility and practical value for real-world problems.  
  \item MAPSeg is extensively evaluated on both private and public datasets, outperforming previous SOTA methods by a large margin. We conduct detailed ablation studies to investigate the impact of each component of MAPSeg.
\end{enumerate}
