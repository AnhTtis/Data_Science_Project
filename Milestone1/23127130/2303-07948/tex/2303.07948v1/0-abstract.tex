\begin{abstract}
This paper introduces VR Haptics at Home, a method of repurposing everyday objects in the home to provide casual and on-demand haptic experiences. Current VR haptic devices are often expensive, complex, and unreliable, which limits the opportunities for rich haptic experiences outside research labs. In contrast, we envision that, by repurposing everyday objects as passive haptics props, we can create engaging VR experiences for casual uses with minimal cost and setup. To explore and evaluate this idea, we conducted an in-the-wild study with eight participants, in which they used our proof-of-concept system to turn their surrounding objects such as chairs, tables, and pillows at their own homes into haptic props. The study results show that our method can be adapted to different homes and environments, enabling more engaging VR experiences without the need for complex setup process. Based on our findings, we propose a possible design space to showcase the potential for future investigation.
\end{abstract}

% This paper introduces VR Haptics at Home, a method of repurposing everyday objects at home to provide room-scale casual haptic experience on demand.
% which significantly limits the opportunity for VR haptic experiences for casual consumer users.
% We also present design considerations for future developers and designers to create interactions using our technique
% Through a guided setup process, the user can turn everyday objects and environment, such as walls, tables, chairs, doors and pillows in their room into haptic props, which can be matched to in-game elements. We present a design space of commonly available physical objects, and propose a series of VR game applications that demonstrate our concept, where the user can touch, grasp, stroke, and rotate real-world objects for haptic feedback in VR games. We evaluated our prototype with eight participants to understand how the difference in objects affects the haptic experience and how easy the setup process is. Based on the quantitative and qualitative feedback, we draw design recommendations for VR designers in the future.

