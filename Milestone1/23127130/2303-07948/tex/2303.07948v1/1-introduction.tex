\section{Introduction}
Consumer virtual reality (VR) devices have shown great promise in creating visually immersive environments, and with devices like the Meta Quest~\footnote{\url{https://www.meta.com/quest/products/quest-2/}}, users are no longer tethered to the computer with wires and thus can play with VR wherever they wish. However, VR \textit{haptic} experiences are not the case for such casual and on-demand uses.
Today's haptic devices that go beyond simple vibrational feedback are not widely available for consumer VR users, as they are often costly, bulky, and complex, which significantly limits the opportunities for rich haptic experiences outside of research labs. 

What if, instead, we could turn our everyday environment into an adaptable, easy-to-use, and body-scale VR haptic environment?
In this paper, we explore VR Haptics at Home, an idea of repurposing everyday objects and environment to create casual, on-demand, yet engaging VR haptic experiences at home without the need for any special equipment.
The idea of leveraging passive haptic props itself is not new, as it has been explored in the literature (mostly known as Substitutional Reality~\cite{simeone2015substitutional}). However, most of the previous work has been studied in a controlled lab setting, leaving questions about how this approach can be adapted to various rooms in real-world environments.

To fill this gap, this paper contributes an \textbf{\textit{in-the-wild study}} conducted in each participant's home, rather than in a research lab, to better understand how such ideas could scale and how difficult the configuration process could be.
To this end, we developed a proof-of-concept prototype that allows users to configure their own haptic props on-demand through a simple setup, which allows us to investigate how this method can be adapted to different homes and environments.
In our study, we asked eight participants to use our system to turn their surrounding objects into haptic props for interactive VR games. More specifically, they used their own \textit{table as a ground} for a whack-a-mole game, \textit{chair as a canon} for a shooting game, and \textit{pillow as a cat} for petting.
The study results show that our method can be adapted to objects in different homes, and the extra steps to configure the objects are considered to be minimal and easy to follow. In addition, while some tasks were more negatively affected by imprecision in hand-tracking than others, overall, our method can effectively provide haptic feedback and enable more engaging VR experiences.
Through the study and iterative design explorations, we also learned that the concept of VR haptics at home can go beyond these three object modalities and there are a lot of untapped opportunities for this idea. For example, we found that not only the shape but also the affordances of objects, such as compliance, mechanism, and texture, could enhance the rich tactile feedback for passive haptics. Moreover, we can reuse a part of the object, rather than focusing on the entire form, to broaden the adaptability and generalizability of such haptic props.
To showcase such potential, we discuss a possible design space with accompanying exemplary scenarios, which we hope could inspire the research community for further investigation.

% We envision that, by repurposing everyday objects as passive haptics props, we can create engaging VR experiences for casual users with minimal cost and setup.
% We develop the proof-of-concept demonstration and its in-the-wild study to investigate how various props in different homes and environments can serve as passive haptic props for interactive VR experiences. 
% in which VR environments are designed to match with surrounding physical environments. 
% We also present design considerations for future developers and designers to create interactions using our technique.

% We explore a method of using furniture (e.g., chairs and cushions) and surfaces (e.g., walls and tabletops) at home as simple passive haptic props, which we argue can greatly enhance the sense of presence in VR with minimal cost and efforts. 
% and our idea is built on top of these existing concepts. 


% still largely rely on graphics and audio, with very limited haptic feedback through vibrational sensations from the hand-held controllers.
% Haptic technologies that render tactile and force feedback have been developed in research laboratories~\cite{benko2016normaltouch, choi2017grabity,lee2019torc}, but they are not made ready and adapted for consumer use due to high cost and mechanical complexity. Furthermore, most of the existing haptic devices focus on tactile sensations for hands or fingertips, making them still difficult to simulate room-scale haptic experience that affect the body as a whole. 

% most prior work has focused on creating visual-haptic illusions that reuse  of a physical prop. Instead, we propose to leverage parts of the physical object with appropriate functional affordances. Third, 

% rather than solely focusing on their . 
% For example, we propose the use of these objects through mobility and functional affordances, such as compliance, mechanism, texture, etc. 
% the following three new perspectives which fill each gap discussed above.
% {\bf 1) Design Space Exploration: } 
% To illustrate the rich interaction space of VR Haptics at Home technique of integrating indoor items as haptic props, we created a design space diagram with accompanying exemplary scenarios, which could inspire the research community for further investigation.
% {\bf 2) Repurposing based on Affordances, rather than Shapes: }
% We investigate everyday objects for VR haptic proxy based on {\it affordacnes} of objects, rather than solely focusing on their {\it shapes}. 
% For example, we propose the use of these objects through mobility and functional affordances, such as compliance, mechanism, texture, etc. 
% {\bf 3) In-the-Wild Study: }
% To evaluate and better understand how such ideas work, we contribute with an {\it in-the-wild study} at participants' homes, instead of the evaluation study at our research lab.
% In summary, the contributions of this work are:
% \begin{itemize}
%     \item The design and implementation of \system{} system.
%     \item The design space and a set of exemplary VR experiences enabled by \system{}, focusing on functional affordances of everyday objects.
%     \item The results and insights based on in-situ user evaluations, where we investigated the adaptability and effectiveness of our technique in different homes.
% \end{itemize}

% current research on passive haptics and substitutional reality still has many gaps to be filled. First, the design of such applications and the choice of proxy objects are mostly decided by researchers. There still does not exist an extensive design space exploration to understand what common objects we can use and what kind of affordances each object can provide. Second, most prior work has focused on creating visual-haptic illusions that reuse the entire form of a physical prop. Instead, we propose to leverage parts of the physical object with appropriate functional affordances. Third, 
