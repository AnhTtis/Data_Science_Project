\documentclass[sigconf]{acmart}

%% NOTE that a single column version may be required for 
%% submission and peer review. This can be done by changing
%% the \doucmentclass[...]{acmart} in this template to 
%% \documentclass[manuscript,screen,review]{acmart}
%% 
%% To ensure 100% compatibility, please check the white list of
%% approved LaTeX packages to be used with the Master Article Template at
%% https://www.acm.org/publications/taps/whitelist-of-latex-packages 
%% before creating your document. The white list page provides 
%% information on how to submit additional LaTeX packages for 
%% review and adoption.
%% Fonts used in the template cannot be substituted; margin 
%% adjustments are not allowed.
%%
%% \BibTeX command to typeset BibTeX logo in the docs
\AtBeginDocument{%
  \providecommand\BibTeX{{%
    \normalfont B\kern-0.5em{\scshape i\kern-0.25em b}\kern-0.8em\TeX}}}

%% Rights management information.  This information is sent to you
%% when you complete the rights form.  These commands have SAMPLE
%% values in them; it is your responsibility as an author to replace
%% the commands and values with those provided to you when you
%% complete the rights form.
\copyrightyear{2023}
\acmYear{2023}
\setcopyright{rightsretained}
\acmConference[CHI EA '23]{Extended Abstracts of the 2023 CHI Conference on Human Factors in Computing Systems}{April 23--28, 2023}{Hamburg, Germany}
\acmBooktitle{Extended Abstracts of the 2023 CHI Conference on Human Factors in Computing Systems (CHI EA '23), April 23--28, 2023, Hamburg, Germany}\acmDOI{10.1145/3544549.3585871}
\acmISBN{978-1-4503-9422-2/23/04}


\begin{document}


%%
%% The "title" command has an optional parameter,
%% allowing the author to define a "short title" to be used in page headers.
\newcommand{\ryo}[1]{\textcolor{purple}{[Ryo: #1]}}

\newcommand{\system}{VR Haptics at Home}

\title[\system{}: Repurposing Everyday Objects and Environment \protect\\ for Casual and On-Demand VR Haptic Experiences]{\system{}: Repurposing Everyday Objects and Environment for Casual and On-Demand VR Haptic Experiences}


%%
%% The "author" command and its associated commands are used to define
%% the authors and their affiliations.
%% Of note is the shared affiliation of the first two authors, and the
%% "authornote" and "authornotemark" commands
%% used to denote shared contribution to the research.


\author{Cathy Mengying Fang}
\email{catfang@media.mit.edu}
\affiliation{
  \institution{MIT Media Lab}
  \city{Cambridge}
  \country{USA}
}


\author{Ryo Suzuki}
\email{ryo.suzuki@ucalgary.ca}
\affiliation{
  \institution{University of Calgary}
  \city{Calgary}
  \country{Canada}
}

\author{Daniel Leithinger}
\email{daniel.leithinger@colorado.edu}
\affiliation{
  \institution{University of Colorado Boulder}
  \city{Boulder}
  \country{USA}
}



%%
%% By default, the full list of authors will be used in the page
%% headers. Often, this list is too long, and will overlap
%% other information printed in the page headers. This command allows
%% the author to define a more concise list
%% of authors' names for this purpose.

\begin{abstract}
As models continue to grow in size, the development of memory optimization methods (MOMs) has emerged as a solution to address the memory bottleneck encountered when training large models. To comprehensively examine the practical value of various MOMs, we have conducted a thorough analysis of existing literature from a systems perspective. 
% Furthermore, we have evaluated the most widely adopted MOMs employed in mainstream frameworks for both vision and language models.
Our analysis has revealed a notable challenge within the research community: the absence of standardized metrics for effectively evaluating the efficacy of MOMs. The scarcity of informative evaluation metrics hinders the ability of researchers and practitioners to compare and benchmark different approaches reliably. Consequently, drawing definitive conclusions and making informed decisions regarding the selection and application of MOMs becomes a challenging endeavor.
To address the challenge, this paper summarizes the scenarios in which MOMs prove advantageous for model training. We propose the use of distinct evaluation metrics under different scenarios. By employing these metrics, we evaluate the prevailing MOMs and find that their benefits are not universal. We present insights derived from experiments and discuss the circumstances in which they can be advantageous.

\end{abstract}

\begin{CCSXML}
<ccs2012>
   <concept>
       <concept_id>10003120.10003121.10003124.10010866</concept_id>
       <concept_desc>Human-centered computing~Virtual reality</concept_desc>
       <concept_significance>500</concept_significance>
       </concept>
 </ccs2012>
\end{CCSXML}

\ccsdesc[500]{Human-centered computing~Virtual reality}
%%
%% Keywords. The author(s) should pick words that accurately describe
%% the work being presented. Separate the keywords with commas.
\keywords{Virtual Reality, Interaction Techniques, Passive Haptics}

\begin{teaserfigure}
\centering
\includegraphics[width=\textwidth]{Figures/teaser.png}
% \includegraphics[width=.8\textwidth]{Figures/title3.png}
\caption{We propose an idea that repurposes everyday objects (e.g., a chair) as passive haptic props (e.g., a cannon) to provide haptic feedback for virtual reality experiences (A). Figures B and C illustrate the potential scenarios of VR haptics at home --- by leveraging the functional affordances of various common indoor objects (B), users can transform an indoor space into an escape room (C) where they use complete tasks in virtual reality with physical objects.}
\label{fig:teaser}
\end{teaserfigure}


% \teaser{
% \centering
% \resizebox{1\textwidth}{!}{
% \includegraphics[height=\textwidth]{pictures/teaser-draft-1.jpg}
% \hspace{1cm}
% \includegraphics[height=\textwidth]{pictures/teaser-mockup-2.jpg}
% }
% \caption{caption. with one sketch to illustrate the concept and another sketch for escape room?}
% \label{fig:teaser}
% }


\maketitle


\section{Introduction}


Recent years have witnessed the rise of human digitization~\cite{habermannDeepCapMonocularHuman2020,alexanderCREATINGPHOTOREALDIGITAL,pengNeuralBodyImplicit2021,alldieckDetailedHumanAvatars2018, rajANRArticulatedNeural2020}. This technology greatly impacts the entertainment, education, design, and engineering industry.
There is a well-developed industry solution for this task.
High-fidelity reconstruction of humans can be achieved either with full-body laser scans~\cite{saitoSCANimateWeaklySupervised2021}, dense synchronized multi-view cameras~\cite{xiangModelingClothingSeparate2021a,xiangDressingAvatarsDeep2022a}, or light stages~\cite{alexanderCREATINGPHOTOREALDIGITAL}.
However, these settings are expensive and tedious to deploy and consist of a complex processing pipeline, preventing the technology's democratization.

Another solution is to view the problem as inverse rendering and learn digital humans directly from custom-collected data.
Traditional approaches directly optimize explicit mesh representation~\cite{loperSMPLSkinnedMultiperson2015, fangRMPERegionalMultiperson2018, pavlakosExpressiveBodyCapture2019} which suffers from the problems of smooth geometry and coarse textures~\cite{prokudinSMPLpixNeuralAvatars2020,alldieckVideoBasedReconstruction2018}. Besides, they require professional artists to design human templates, rigging, and unwrapped UV coordinates.
Recently, with the help of volumetric-based implicit representations~\cite{mildenhallNeRFRepresentingScenes2020, parkDeepSDFLearningContinuous2019, meschederOccupancyNetworksLearning2019} and neural rendering~\cite{laineModularPrimitivesHighPerformance2020, liuSoftRasterizerDifferentiable2019, thiesDeferredNeuralRendering2019}, 
one can easily digitize a quality-plausible human avatar from video footage~\cite{jiangNeuManNeuralHuman2022,wengHumanNeRFFreeviewpointRendering}.
Particularly, volumetric-based implicit representations~\cite{mildenhallNeRFRepresentingScenes2020, pengNeuralBodyImplicit2021} can reconstruct scenes or objects with much higher fidelity against previous neural renderer~\cite{thiesDeferredNeuralRendering2019,prokudinSMPLpixNeuralAvatars2020}, and is more user-friendly as it does not need any human templates, pre-set rigging, or UV coordinates.
Captured visual footage and corresponding skeleton tracking are enough for training.
However, better reconstructions and more friendly usability are at the expense of the following factors.
1) \textbf{Inefficiency:}
They require longer optimization times (typically tens of hours or days) and inference slowly.
Volume rendering~\cite{mildenhallNeRFRepresentingScenes2020,lombardiNeuralVolumesLearning2019} formulates images by querying the densities and colors of millions of spatial coordinates. 
In the training stage, due to memory constraints, only a small fraction of points are sampled which leads to slow convergence speed.
2) \textbf{Entangled representations}:
The geometry, materials, and motion dynamics are entangled in the neural networks. 
Due to the implicit nature of neural nets, one can hardly edit one property without touching the others~\cite{yuanNeRFEditingGeometryEditing2022a,liuEditingConditionalRadiance2021}.
3) \textbf{Graphics incompatibility}:
Volume rendering is incompatible with the current popular graphic pipeline,
which renders triangular/quadrilateral meshes efficiently with the rasterization technique.
Many downstream applications require mesh rasterization in their workflow (\eg, editing~\cite{foundationBlenderOrgHome}, simulation~\cite{benderPositionBasedSimulationMethods2015}, real-time rendering~\cite{akenine2019real}, ray-tracing~\cite{waldRTXRayTracing}).
Although there are approaches~\cite{lorensenMarchingCubesHigh,labelleIsosurfaceStuffingFast2007} can convert volumetric fields into meshes, the gaps from discrete sampling degrade the output quality in terms of both meshes and textures.


To address these issues, we present \textbf{EMA}, a method based on \textbf{E}fficient \textbf{M}eshy neural fields to reconstruct animatable human \textbf{A}vatars.
Our method enjoys flexibility from implicit representations and efficiency from explicit meshes, yet still maintains high-fidelity reconstruction quality.
Given video sequences and the corresponding pose tracking, our method digitizes humans in terms of canonical triangular meshes, physically-based rendering (PBR) materials, and skinning weights \textit{w.r.t.} skeletons.
We jointly learn the above components via inverse rendering~\cite{laineModularPrimitivesHighPerformance2020,chenDIBRLearningPredict2021,chenLearningPredict3D2019} in an end-to-end manner.
Each of them is derived from a separate neural field, which relaxes the requirements of a preset human template, rigging, or UV coordinates.
Specifically, we predict a canonical mesh out of a signed distance field (SDF) by differentiable marching tetrahedra~\cite{shenDeepMarchingTetrahedra2021,gaoGET3DGenerativeModel,gaoLearningDeformableTetrahedral2020,munkbergExtractingTriangular3D2022}, then we extend the marching tetrahedra~\cite{shenDeepMarchingTetrahedra2021} for spatial-varying materials by utilizing a neural field to predict PBR materials \textit{on the mesh surfaces} after rasterization~\cite{munkbergExtractingTriangular3D2022,hasselgrenShapeLightMaterial2022,laineModularPrimitivesHighPerformance2020}.
To make the canonical mesh animatable, we take another neural field to model the forward linear blend skinning for the meshes. 
Given a posed skeleton, the canonical mesh is then transformed into the corresponding poses.
Finally, we shade the mesh with a rasterization-based differentiable renderer~\cite{laineModularPrimitivesHighPerformance2020} and train our models with a photo-metric loss.
After training, we export the mesh with materials and discard the neural fields.

\looseness=-1
There are several merits of our method design.
1) \textbf{Efficiency}:
Powered by efficient mesh rendering, our method can render in real-time.
Besides, the training speed is boosted as well, 
since we compute loss holistically on the whole image and the gradients only flow on the mesh surface. In contrast, volume rendering takes limited pixels for loss computation and back-propagates the gradients in the whole space.
Our method only needs about an hour of training and minutes of optimization are enough for plausible avatar reconstruction.
2) \textbf{Disentangled representations}:
Our shape, materials, and motion modules are disentangled naturally by design, which facilitates editing. 
Besides, Canonical meshes with forward skinning modeling handle the out-of-distribution poses better.
3) \textbf{Graphics compatibility}:
Our derived mesh representation is compatible with 
the prominent graphic pipeline, which leads to instant downstream applications (\eg, the shape and materials can be edited directly in design software~\cite{foundationBlenderOrgHome}).
To further improve reconstruction quality, we additionally optimize image-based environment lights and non-rigid motions.


We conduct extensive experiments on standards benchmarks H36M~\cite{ionescuHuman36MLarge2014b} and ZJU-MoCap~\cite{pengNeuralBodyImplicit2021}.
Our method achieves very competitive performance for novel view synthesis, generalizes better for novel poses, 
and significantly improves both training time and inference speed against previous arts.
Our research-oriented code reaches real-time inference speed (100+ FPS for rendering $512\times512$ images).
We in addition showcase applications including novel pose synthesis, material editing, and relighting.
\section{Related Work} \label{sec:related work}
\vspace{-0.2cm}
{\noindent \bf Vision-Language Pre-training.} In the early literature, \cite{Mori99,Frome13,Weston11} explore jointly training image-text embeddings using paired text documents. Recently, some studies have further scaled up the training with large-scale web data to form ``the \textbf{foundation} models'', {\em e.g.}, CLIP~\cite{Radford21}, ALIGN~\cite{Jia21}, Florence~\cite{yuan2021florence}, FILIP~\cite{yao2021filip}, VideoCLIP~\cite{xu2021videoclip}, and LiT~\cite{zhai2022lit}. These foundation models usually contain one visual encoder and one textual encoder, which are trained using simple noise contrastive learning for powerful cross-modal representations. They have shown promising potential in many tasks, such as image classification and detection, action recognition, and retrieval. In this paper, we use CLIP for low-shot temporal action localization, but the same technique should be applicable to other foundation models as well.



\vspace{0.1cm}
{\noindent \bf Prompting} refers to leveraging input instructions to steer foundation models for desired outputs. In the NLP domain, early papers~\cite{Gao21,Jiang20,Timo21,Shin20} focus on handcrafted prompt templates. To avoid labor and increase flexibility, some studies~\cite{Lester21,li21-prefixtuning,li2021prefix} propose learnable prompt tuning at the textual stream, showing strong low-shot generalization. In the CV domain, some recent papers~\cite{zhou2019learn,zhou2022conditional,ju2022prompting} introduce such randomly initialized prompt tuning to handle visual tasks, {\em e.g.}, image understanding~\cite{zhu2022prompt,lu2022prompt,yang2022learning,ma2023diffusionseg} and video understanding~\cite{jia2022visual,nag2022zero,ni2022expanding}. However, these studies ignore lexical ambiguity of category names, and cases that are not easy to describe in text. This paper designs novel conditional prompt tuning and language descriptions from LLMs, to solve these issues. 



\vspace{0.1cm}
{\noindent \bf Closed-set Temporal Action Localization} considers to detect and classify action instances from one pre-defined category list. Specifically, existing methods can be divided into two popular supervisions, {\em i.e.}, strong~\cite{zeng2019graph,lin2021learning,qing2021temporal} and weak~\cite{wang2017untrimmednets,ju2023constraint,ju2020point,yudistira2022weakly}. Strong supervision gives precise boundary labels and category labels for training. There are two detailed pipelines: the top-down framework~\cite{shou2016temporal,shou2017cdc,gao2017turn,chao2018rethinking,lin2017single,xu2017r,tan2021relaxed,zhu2021enriching,wang2022rcl,xu2020g} pre-defines extensive anchors, adopts fixed-length sliding windows to produce initial proposals, then regresses to refine boundaries; the bottom-up framework~\cite{zhao2017temporal,lin2018bsn,lin2019bmn,vo2023aoe,zhao2020bottom,bai2020boundary} learns frame-wise boundary detectors for the boundary frames, then groups extreme frames or estimates action lengths for proposal generation. In addition, several works~\cite{gao2018ctap,liu2019multi,yang2020revisiting} used various fusion strategies to complement these frameworks. On the other hand, weak supervision trains without boundary labels to alleviate annotation costs. The video-level setting learns from category labels~\cite{paul2018w,ju2022distilling}, the CAS-based framework~\cite{liu2019completeness,ju2021adaptive,min2020adversarial,narayan2021d2,lee2019background,lee2021weakly,zhao2021soda} and attention-based framework~\cite{nguyen2018weakly,luo2021action,nguyen2019weakly,shi2020weakly,gao2022fine,he2022asm,huang2021foreground,luo2020weakly,ma2022weakly} have been well studied. To generate better results from CAS or attention, some studies~\cite{shou2018autoloc,liu2019weakly} improved post-processing. To balance cost and performance, some papers introduced single-frame annotations~\cite{ju2021divide,ma2020sf,lee2021learning,yang2021background,mettes2019pointly} or instance-number annotations~\cite{narayan20193c,xu2019segregated}. 

Nevertheless, all the above methods assume that action categories remain identical for training and testing, which is an over-simplification of real application scenarios, limiting practical uses of the vision system.



\vspace{0.1cm} 
{\noindent \bf Low-Shot Temporal Action Localization} considers more realistic scenarios: generalize TAL towards action categories that are unseen (zero-shot) or with several support samples (few-shot). Existing methods~\cite{ju2022prompting,nag2022zero,zhang2022ow,bao2022opental} most rely on foundational models pre-trained on large-scale image-caption pairs for help. Typically, E-Prompt~\cite{ju2022prompting} is the first to construct wide baselines with popular prompt tuning~\cite{Lester21,li21-prefixtuning} and vanilla temporal modeling. STALE~\cite{nag2022zero} explores the one-stage framework to further simplify usage. Although promising, all above methods meet two main challenges: (1) For category semantics, the definition may be vague, inaccurate, or incomplete. (2) For visual motions, temporal modeling may be insufficient. In this paper, for detailed category understanding, we design novel language descriptions from LLMs and vision-conditional prompt tuning; for clearer motion understanding, we introduce optical flows to provide explicit motion inputs. 






\begin{figure*}[t!]
\includegraphics[width=\textwidth]{Figures/process.png}
\caption{(A) The user places the object of interest at a desirable location. (B) The user aligns the controller against the object to place the virtual element. (C) The user places the controllers aside somewhere safe and looks at their hand to switch to hand-tracking mode. (D) The user feels the haptic feedback from the physical object while in VR (E) sees the hand touch a virtual cat.}
\label{fig:setup-process}
% \vspace{-10px}
\end{figure*}



\section{VR Haptics at Home: Proof-of-Concept System}
To investigate the VR Haptics at Home concept, we developed a proof-of-concept prototype with the Unity 3D engine. 
Our system consists of three interactive VR games: 1) pet a cat, 2) whack-a-mole, and 3) shoot monsters (See Figure \ref{fig:setup-process}, ~\ref{fig:demo-table}, and~\ref{fig:demo-chair} respectively).
We deployed our program on the Meta Quest and used Meta's hand tracking API (1.40). We found the accuracy of hand tracking to be sufficient to create a proof-of-concept. We note that our implementation is not limited to Meta's tracking technology. To repurpose physical objects, the user first prepares the object to be within reach (Figure \ref{fig:setup-process}A), and in the headset draws the guardian boundary to include the object. They place the controller against the physical object and press the trigger to place the virtual interactable object (Figure \ref{fig:setup-process}B). Once all the virtual objects have been placed, users then place the controllers face down somewhere safe and allow the system to automatically switch to the hand-tracking mode (Figure \ref{fig:setup-process}C). Lastly, the user can directly interact with the virtual object with their hands (Figure \ref{fig:setup-process}D). We have also considered leveraging hand tracking to, e.g., pinch to place the virtual interactable object, but to ensure reliable performance during the study, we used controllers as the reliable setup method.



% \begin{figure}[b!]
% \centering
% \includegraphics[width=.8\linewidth]{Figures/mole_eg.png}
% \caption{The user aligns the controller against a table surface to place the game platform (A). They can tap on the table and receive haptic feedback (B) as they hit the virtual moles in the Whack-a-Mole game (C).}
% \label{fig:demo-table}
% \end{figure}




% \subsection{Exemplary scenarios}
% We selected and built three scenarios to demonstrate objects with various affordances. {\bf Whack-a-Mole} – A Whack-a-Mole game is a game where the player hits the “moles” that peeks through the game platform before they disappear. {\bf Pet a Cat} – In this VR experience, users interact with a virtual pet. When a user reaches out their hand and wants to pet the animal, their hand normally feels nothing. {\bf Shoot Monsters} – The user controls a cannon to aim and shoot the monsters to win points. As the monsters appear in different locations, they would need to grab the handle of the cannon to rotate and aim at the monsters. Given the space constraint, please refer to the Video Figure and Appendix for a detailed description for each.

% {\bf Whack-a-Mole} – A Whack-a-Mole game is a game where the player hits the “moles” that peeks through the game platform before they disappear. In a typically VR experience, the user’s hand feels nothing when hitting a virtual button or the mole in mid-air, and the hand also goes through the virtual game platform. In this scenario, we leverage the flat surface of a tabletop to be a game platform. When the user presses the virtual button to initiate the game and taps the mole to win points, their hand is stopped by the tabletop surface.  For implementation, we attach the game platform to the front of the controller but the platform is hidden at first. When the user places the ring of the controller flush against the flat surface and presses the trigger, the game surface is enabled and placed where the controller ring is, which then aligns with the physical surface (Figure \ref{fig:demo-table}).



% {\bf Pet a Cat} – In this VR experience, users interact with a virtual pet. When a user reaches out their hand and wants to pet the animal, their hand normally feels nothing. For this scenario, we repurpose a cushion as a cat. When the user pets the cat, they feel both the texture and the compliance of the cushion that mimics the haptic feedback from that of a cat’s belly. To implement this, similarly, we attach a virtual cat to controller. When the user aligns the controller against the center of a cushion and confirms with a trigger, the top of the cushion aligns with the cat’s belly. When the user pets the cat, they hear the cat cry and purr and also see it wiggle its body as a gesture of affinity (Figure \ref{fig:setup-process}E).


% % \begin{figure}[t!]
% % \centering
% % \includegraphics[width=.8\linewidth]{Figures/cat_eg.png}
% % \caption{The user presses the controller against a cushion to place the cat (A). They can touch the cushion and receive haptic feedback (B) as they pet the virtual cat (C).}
% % \label{fig:demo-cushion}
% % \end{figure}


% % \begin{figure}[b!]
% % \centering
% % \includegraphics[width=.8\linewidth]{Figures/cannon_eg.png}
% % \caption{The user aligns the controller against the back of a chair to place the cannon (A). They can rest their hands on the chair, move it around and receive haptic feedback (B) as they rotate the virtual cannon to aim at the monsters (C).}
% % \label{fig:demo-chair}
% % \end{figure}


% {\bf Shoot Monsters} – The user controls a cannon to aim and shoot the monsters to win points. As the monsters appear in different locations, they would need to grab the handle of the cannon to rotate and aim at the monsters. If the user controls the cannon with their hands instead of a joy stick, their hands would go through the handle and feel no haptic feedback that would give them a sense of control of the cannon. Here we use a chair and leverage the movement constraints to enhance this VR experience. The cannon appears to be heavy and cannot be easily picked up, and so is the chair. The cannon has wheels that allows it to move along the ground, and its handles signals that it can be rotated. Even though the shape of the chair does not match the virtual cannon, the back of the chair affords grabbing and the rotating motion of the moveable chair matches the same motion constraint of the cannon. For implementation, we press the controller against the top edge of the back of the chair, such that the handle of the cannon then aligns with the chair’s back. The virtual handles are grabbable objects that follow the users' hands, which allows the movement of the virtual cannon and the physical chair to be in sync (Figure \ref{fig:demo-chair}).



% \begin{figure}[h!]
% \includegraphics[width=\linewidth]{title3.png}
% \caption{A user repurposes everyday furniture and surface for VR objects e.g., (a) the back of the chair as the handle of a cannon, (b) a table surface as a whack-a-mole game platform, and (c) a couch cushion as a pet cat.}
% \label{fig:demo}
% \end{figure}












\begin{figure}[b!]
% \vspace{-5px}
    \centering
    \includegraphics[width=\linewidth]{Figures/mole_eg.png}
    % \vspace{-10px}
    \caption{The user aligns the controller against a table surface to place the game platform (A). They can tap on the table and receive haptic feedback (B) as they hit the virtual moles in the Whack-a-Mole game (C).}
    \label{fig:demo-table}
\end{figure}


\section{In-the-Wild User Study}
We tested with 8 participants (male: 5; female: 3; mean age 21.4; all with VR experience) and paid each a \$15 gift card. Participants used their own Meta Quest (first or second generation) and performed the tasks remotely at their homes or offices. 
%We chose one scenario from each mobility category (see Appendix, Figures 2–5, and Video Figure for detailed description) for our study: 
The three scenarios created for the study include “Whack-a-mole” where the participants hit the mole 10 times, “Pet a cat” where the participants rub the cat’s belly 10 times, and “Shoot monsters”, where the participants aim and hit 20 monsters. There were two conditions: the {\it control, no-haptics condition} which every task was performed in mid-air, and the {\it haptic condition} where participants followed instructions in VR to configure the physical objects. The order of the scenarios and conditions was counterbalanced. Following the instructions in a Google Doc, participants set up the Quest headset for hand tracking and drew the guardian boundary in a space where "there is a clean tabletop surface, a moveable chair, and a small cushion or pillow" (the exact instruction given). After each task, they took off the headset and filled out a seven-question Likert-scale questionnaire about the task in a Google Form. Finally, the investigator conducted a semi-structured interview via voice call where participants elaborated on their preferences and experiences during the setup process and interactive scenarios.



\section{Experimental Results}\label{sec:experiments}

We presented in the previous section how to leverage watermarks for detection and identification of images generated from text prompts.
We now present more general results on robustness and image quality for different generative tasks.
We also compare Stable Signature to other watermarking algorithms applied post-generation.

\begin{SCtable*}
    \centering
    \footnotesize
    \setlength{\tabcolsep}{4pt}
        \begin{tabular}{ c l @{\hspace{2pt}} l  *{2}{c} *{4}{p{15pt}}}
        \toprule
       & & \multirow{2}{*}{}          & \multirow{2}{*}{PSNR / SSIM $\uparrow$} & \multirow{2}{*}{FID $\downarrow$} &\multicolumn{4}{c}{Bit accuracy $\uparrow$ on:} \\ 

    &                                         &                           &           &                                            &  None & Crop & Brigh. & Comb.  \\ \midrule
\multirow{6}{*}{\rotatebox[origin=c]{90}{Tasks}}  
        & Text-to-Image                      & LDM~\cite{rombach2022ldm}             & $30.0$ / $0.89$   & $19.6$ \color{orange}{($-0.3$)} & $0.99$ & $0.95$ & $0.97$ & $0.92$ \\ \cmidrule{2-9}
       & Image Edition                       & DiffEdit~\cite{couairon2022diffedit}                  & $31.2$ / $0.92$ & $15.0$ \color{orange}{($-0.3$)}       & $0.99$ & $0.95$ & $0.98$ & $0.94$ \\ \cmidrule{2-9}
       & Inpainting - Full          & \multirow{2}{*}{Glide~\cite{nichol2021glide}}   & $31.1$ / $0.91$  & $16.8$ \color{orange}{($+0.6$)} & $0.99$ & $0.97$ & $0.98$ & $0.93$ \\ 
       & {\color{white}Inpa} - Mask only          &                                   & $37.8$ / $0.98$  & $9.0$~~ \color{orange}{($+0.1$)} & $0.89$ & $0.76$ & $0.84$ & $0.78$\\ \cmidrule{2-9}
       & Super-Resolution & LDM~\cite{rombach2022ldm}  & $34.0$ / $0.94$ & $11.6$ \color{orange}{($+0.0$)}      & $0.98$ & $0.93$ & $0.96$ & $0.92$ \\ 
    \midrule \rule{0pt}{8pt} \rule{0pt}{8pt} 
\multirow{7}{*}{\rotatebox[origin=c]{90}{WM Methods}} 
           & \emph{Post generation} \\
           & Dct-Dwt \cite{cox2007digital}                      & $0.14$ (s/img)      &  $39.5$ / $0.97$  & $19.5$ \color{orange}{($-0.4$)} & $0.86$ & $0.52$ & $0.51$ & $0.51$ \\ 
           & SSL Watermark \cite{fernandez2022sslwatermarking}  & $0.45$ (s/img)      &  $31.1$ / $0.86$  & $20.6$ \color{orange}{($+0.7$)} & $1.00$ & $0.73$ & $0.93$ & $0.66$ \\ 
           & FNNS \cite{kishore2021fixed}                       & $0.28$ (s/img)      &  $32.1$ / $0.90$  & $19.0$ \color{orange}{($-0.9$)} & $0.93$ & $0.93$ & $0.91$ & $0.93$ \\ 
           & HiDDeN \cite{zhu2018hidden}                        & $0.11$ (s/img)      &  $32.0$ / $0.88$  & $19.7$ \color{orange}{($-0.2$)} & $0.99$ & $0.97$ & $0.99$ & $0.98$ \\ \cmidrule{2-9}
           & \emph{Merged in generation} \\
           & Stable Signature                            & $0.00$ (s/img)             &  $30.0$ / $0.89$ & $19.6$ \color{orange}{($-0.3$)}     & $0.99$ & $0.95$ & $0.97$ & $0.92$ \\ 
        \bottomrule \vspace*{-0.2cm}
    \end{tabular}
    \caption{
        Generation quality and comparison to post-hoc watermarking on 512$\times$512 images and $48$-bit signatures.
        PSNR and SSIM are computed between generations of the original and watermarked generators.
        For FID, we show in {\color{orange} (color)} the difference with regards to original.
        Post-hoc watermarking is evaluated on text-generated images.
        (App.~\ref{sec:supp-robustness} gives results on more transformations, and App.~\ref{app:implementation-details} gives more details on the evaluations.)
        Overall, Stable Signature has minimal impact on generation quality. It has comparable robustness to post-hoc methods while being rooted in the generation itself.
        \vspace*{-0.2cm}
    }\label{tab:quality-watermarking} 
\end{SCtable*}



\subsection{Tasks \& evaluation metrics}
Since our method only involves the LDM decoder, it makes it compatible with many generative tasks. 
We evaluate text-to-image generation and image edition on the validation set of MS-COCO~\cite{lin2014microsoft}, super-resolution and inpainting on the validation set of ImageNet~\cite{deng2009imagenet} 
(all evaluation details are available in App.~\ref{app:evaluation}).

We evaluate the image distortion with the Peak Signal-to-Noise Ratio (PSNR), which is defined as $\mathrm{PSNR}(x,x') = -10\cdot \log_{10} (\mathrm{MSE}(x,x'))$, for $x,x'\in [0,1]^{c\times h\times w}$, as well as Structural Similarity score (SSIM)~\cite{wang2004image}.
They compare images generated with and without watermark. 
On the other hand, we evaluate the diversity and quality of the generated images with the Fr\'echet Inception Distance (FID)~\cite{heusel2017gans}.
The bit accuracy -- the percentage of bits correctly decoded -- evaluates the watermarks' robustness.



\subsection{Image generation quality}
\autoref{fig:qualitative} shows qualitative examples of how the image generation is altered by the latent decoder's fine-tuning. 
The difference is very hard to perceive even for a trained eye. 
This is surprising for such a low PSNR, especially since the watermark embedding is not constrained by any Human Visual System like in professional watermarking techniques. 
Most interestingly, the LDM decoder has indeed learned to add the watermark signal only over textured areas where the human eyes are not sensitive, while the uniform backgrounds are kept intact (see the pixel-wise difference).
% More visual results are available in App.~\ref{app:qualitative}.

\autoref{tab:quality-watermarking} presents a quantitative evaluation of image generation quality on the different tasks.
We report the FID, and the average PSNR and SSIM that are computed between the images generated by the fine-tuned LDM and the original one.
The results show that no matter the task, the watermarking has very small impact on the FID of the generation.

The average PSNR is around $30$~dB and SSIM around $0.9$ between images generated by the original and a  watermarked model.
They are a bit low from a watermarking perspective because we do not explicitly optimize for them.
Indeed, in a real world scenario, one would only have the watermarked version of the image. 
Therefore we don't need to be as close as possible to the original image but only want to generate artifacts-free images. Without access to the image generated by the original LDM, it is very hard to tell whether a watermark is present or not.


\begin{figure}[b]
    \centering
    \scriptsize
    \setlength{\tabcolsep}{0pt}
    \resizebox{0.99\linewidth}{!}{
    \begin{tabular}{c @{\hspace{0.1cm}} c @{\hspace{0.1cm}} c}
        \toprule
        Generated with original & Generated with watermark & Pixel-wise difference ($\times 10$) \\
        \midrule
        \includegraphics[width=0.33\linewidth]{figs/qual/01_nw.png} &
        \includegraphics[width=0.33\linewidth]{figs/qual/01_w.png} &
        \includegraphics[width=0.33\linewidth]{figs/qual/01_diff.png} \\ 
        \rule{0pt}{0.2cm}
        \includegraphics[width=0.33\linewidth]{figs/qual/02_nw.png} &
        \includegraphics[width=0.33\linewidth]{figs/qual/02_w.png} &
        \includegraphics[width=0.33\linewidth]{figs/qual/02_diff.png} \\       
        \bottomrule\\
    \end{tabular}
    }
    % \captionsetup{font=small}
    \caption{
    Images generated with Stable Diffusion. 
    The PSNR is $35.4$\,dB in the first row and $28.6$\,dB in the second.
    Images generated with Stable Signature look natural because modified areas are located where the eye is not sensitive.
    More examples in App.~\ref{app:qualitative}.
    }
    \label{fig:qualitative}
\end{figure}

\begin{table}[t]
    \centering
    \caption{
        Watermark robustness on image transformations applied before decoding, details of which are available in App.~\ref{app:evaluation}.
        We report the bit accuracy, averaged over $10\times1$k images generated from COCO prompts with $10$ different keys.
        }\label{tab:robustness}
        \footnotesize
        \vspace*{0.2cm}
        \setlength{\tabcolsep}{4pt}
        \resizebox{0.96\linewidth}{!}{
            \begin{tabular}{ll|ll|ll}
                \toprule
                \bf{Attack}             & \bf{Bit acc.}     & Comb.         & $0.92$    & Sharpness $2.0$ & $0.99$ \\
                None & $0.99$                               & Bright. $2.0$  & $0.97$    & Med. Filter $k$=7  & $0.94$ \\
                Crop $0.1$ & $0.95$                         & Cont. $2.0$    & $0.98$    & Resize $0.7$  & $0.91$  \\
                JPEG $50$ & $0.88$                          & Sat. $2.0$  & $0.99$      & Text overlay    & $0.99$ \\
                \bottomrule
            \end{tabular}
        }
        \vspace*{-0.3cm}
    \end{table}

\subsection{Watermark robustness}\label{subsec:robustness}
We evaluate the robustness of the watermark to different image transformations applied before extraction.
For each task, we generate $1$k images with $10$ models fine-tuned for different messages, and report the average bit accuracy in \autoref{tab:quality-watermarking}.
Additionally, \autoref{tab:robustness} reports results on more image transformations for images generated from COCO prompts.
The main evaluated transformations are presented in Fig.~\ref{fig:transformations} (more evaluation details are available in App.~\ref{app:evaluation}).

We see that the watermark is indeed robust for several tasks and across transformations.
The bit accuracy is always above $0.9$, except for inpainting, when replacing only the masked region of the image (between $1-50$\% of the image, with an average of $27\%$ across masks).
Besides, the bit accuracy is not perfect even without edition, mainly because there are images that are harder to watermark (\eg the ones that are very uniform, like the background in Fig.~\ref{fig:qualitative}) and for which the accuracy is lower.

Note that the robustness comes even without any transformation during the LDM fine-tuning phase:
it is due to the watermark extractor.
If the watermark embedding pipeline is learned to be robust against an augmentation, then the LDM will learn how to produce watermarks that are robust against it during fine-tuning.




\subsection{Comparison to post-hoc watermarking}\label{subsec:watermarking}

An alternative way to watermark generated images is to process them after the generation (post-hoc). 
This may be simpler, but less secure and efficient than Stable Signature.
We compare our method to a frequency based method, DCT-DWT~\cite{cox2007digital},
iterative approaches (SSL Watermark~\cite{fernandez2022sslwatermarking} and FNNS~\cite{kishore2021fixed}), and an encoder/decoder one like HiDDeN~\cite{zhu2018hidden}.
We choose DCT-DWT since it is employed by the original open source release of Stable Diffusion~\cite{2022stablediffusion}, and the other methods because of their performance and their ability to handle arbitrary image sizes and number of bits.
We use our implementations (see details in App.~\ref{app:watermarking}).

\autoref{tab:quality-watermarking} compares the generation quality and the robustness over $5$k generated images.
Overall, Stable Signature achieves comparable results in terms of robustness. 
HiDDeN's performance is a bit higher but its output bits are not i.i.d. meaning that it cannot be used with the same guarantees as the other methods.
We also observe that post-hoc generation gives worse qualitative results, images tend to present artifacts (see Fig.~\ref{fig:supp-watermark} in the supplement).
One explanation is that Stable Signature is merged into the high-quality generation process with the LDM auto-encoder model, which is able to modify images in a more subtle way.


% \vspace{0.3cm}
\subsection{Can we trade image quality for robustness?}\label{subsec:quality-tradeoff}

We can choose to maximize the image quality or the robustness of the watermark thanks to the weight $\lambda_i$ of the perceptual loss in~\eqref{eq:loss2}.
We report the average PSNR of $1$k generated images, as well as the bit accuracy obtained on the extracted message for the `Combined' editing applied before detection (qualitative results are in App.~\ref{sec:supp-percep-loss}).
A higher $\lambda_i$ leads to an image closer to the original one, but to lower bit accuracies on the extracted message, see \autoref{tab:tradeoff}.

\begin{table}[t]
        \centering
        \caption{Quality-robustness trade-off during fine-tuning.}\label{tab:tradeoff}
        \resizebox{0.9\linewidth}{!}{
        \begin{tabular}{l *{7}{c@{\hspace*{8pt}}}}
            \toprule
            $\lambda_i$ for fine-tuning     & $0.8$ & $0.4$ & $0.2$ & $0.1$ & $0.05$ & $0.025$ \\ \midrule
            \rule{0pt}{2ex}
            PSNR $\uparrow$ & $31.4$ & $30.6$ & $29.7$ & $28.5$ & $26.8$ & $24.6$ \\ 
            \rule{0pt}{2ex}
            Bit acc. $\uparrow$ on `comb.' & $0.85$ & $0.88$ & $0.90$ & $0.92$ & $0.94$ & $0.95$ \\ 
            \bottomrule 
            \vspace*{-0.4cm}
        \end{tabular}
        }
\end{table}



% \vspace{-0.2cm}
\subsection{Attack simulation layer}\label{subsec:message-decoder}

\begin{table}[t]
    \centering
    \caption{Role of the attack simulation layer at pre-training.}\label{tab:asl}
    % \vspace{0.2cm}
    \resizebox{0.95\linewidth}{!}{
    \begin{tabular}{c@{\hspace*{4pt}} *{5}{c@{\hspace*{8pt}}}}
        \toprule
        \multirow{2}{*}{ \shortstack{ Seen at  \\ $\mathcal{W}$ training \vspace*{-4pt}} } & \multicolumn{5}{c}{Bit accuracy $\uparrow$ at test time:} \\ \cmidrule{2-6}
            & Crop $0.1$ & Rot. $90$ &JPEG $50$ & Bright. $2.0$ & Res. $0.7$  \\
        \midrule
        \xmark     & 1.00 & 0.56 & 0.50 & 0.99 & 0.48 \\
        \cmark     & 1.00 & 0.99 & 0.90 & 0.99 & 0.91 \\
        \bottomrule
        \vspace*{-0.8cm} 
    \end{tabular} 
    }
\end{table}

Watermark robustness against image transformations depends solely on the watermark extractor.
here, we pre-train them with or without specific transformations in the simulation layer, on a shorter schedule of $50$ epochs, with $128\times 128$ images and $16$-bits messages.
From there, we plug them in the LDM fine-tuning stage and we generate $1$k images from text prompts.
We report the bit accuracy of the extracted watermarks in \autoref{tab:asl}.
The extractor is naturally robust to some transformations, such as crops or brightness, without being trained with them, while others, like rotations or JPEG, require simulation during training for the watermark to be recovered at test time.
Empirically we observed that adding a transformation improves results for the latter, but makes training more challenging.



\begin{figure*}[t!]
% \vspace{-5px}
% \resizebox{1\textwidth}{!}{
% \includegraphics[height=\textwidth]{Figures/designSpace1.png}
% \hspace{.5cm}
% \includegraphics[height=\textwidth]{Figures/designSpace2.png}
% }
\includegraphics[width=\textwidth]{Figures/dspace2.png}
% \includegraphics[width=.8\textwidth]{Figures/dspace3.png}
\caption{The design space of our idea. We illustrated two scenarios for each object to show the rich design opportunity.}
\label{fig:design-space}
\end{figure*}

\section{Design Space}
Through the participants' feedback and iterative design process, we also learned that the idea of VR haptics at home can go beyond the three demonstrations we developed. Therefore, based on our exploration, this section explores the possible design space by illustrating exemplary VR experiences for future investigation (Figure \ref{fig:design-space}). 
Moreover, we discuss how the \textit{affordance} of physical objects, rather than similarities in shapes, can affect virtual haptic experiences.

% For each interaction, we illustrate 
% how the affordance of virtual and physical objects rather than similarities in shapes, which we discuss below with exemplary VR experiences (Figure \ref{fig:design-space}). 
% We propose a technique that, instead of precluding indoor furniture and objects, incorporates and repurposes these objects to serve as passive haptic props for more immersive VR experiences. 
% Prior work explored repurposing physical objects for ones with a similar form and suggested the less mismatch in shape the better~\cite{simeone2015substitutional}. 




\subsection{Mobility}
% We define the mobility of an object as the ability to be relocated easily by a single person during the setup process and VR experience. 
For VR haptic scenarios, mobility depends on the object’s weight, size, and relation to the indoor environment, which provides constraints to the possible interactions. 
% In addition, tracking every single object is impractical at home (e.g., attaching optical trackers to objects in the home), thus mobility becomes an affordance uniquely important to VR, as misplacement can cause immediate break-in-presence. 


\subsubsection{Static Objects and Surfaces}
Static objects or surfaces are large and structural and thus are most suitable for setting reliable boundaries. 
% Couches can be used as a fort to lean against or a platform for stepping on and reaching things. 
Going beyond repurposing the entirety of the object, we can focus on the flat surface as an affordance. For example, tangible interactions such as organizing a digital photo gallery where the user swipes and moves the photos presented on a surface. We can also simulate haptic interactions such as playing the piano where the hand encounters a surface when touching the virtual interactables. Flat surfaces can be a constraint for mobile objects. A table air hockey striker can only move along the plane of an air hockey table.


\subsubsection{Semi-Static Objects}
We define semi-static objects as heavy objects that can be moved within certain constraints. Beyond being used for sitting, a chair can also be pushed along the floor. We can leverage the constrained motion and repurpose the chair as a lawnmower, a similarly heavy object that one does not pick up and only moves across the ground. If the chair swivels, the constrained rotating motion allows the chair to be repurposed as a game cannon. Likewise, a door naturally serves as a portal to a different world, but we could also leverage constrained motion around the hinge.

\subsubsection{Mobile Objects}
These objects are lightweight and can afford interactions such as grasping and gripping along with swift movements due to their mobility. A water bottle that can be grabbed with the whole hand and moved around is suitable for replacing a saber. In addition, mobile objects can afford bimanual manipulations. Once again, a water bottle can be grabbed with both hands and acts as a baseball bat. A mobile object is not constrained by its location. A cushion placed on the ground can be repurposed as a CPR dummy, but when it’s placed on the lap, it can be replaced as a pet cat.

\subsection{Functional Affordance}
Furthermore, building on the findings in Substitutional Reality~\cite{simeone2015substitutional} where the similarity in affordance allows for a more believable substitution, we focus on leveraging the functional affordance of everyday objects. 

\subsubsection{Compliance}
The home environment conveniently provides compliant objects that are otherwise difficult to simulate using active haptic devices. A cushion is a convenient substitute for a small animal that allows the user to feel the softness when touching and petting the animal. In addition, compliant objects can absorb impact from large forces or fast actions. For example, a couch or a cushion could be used for CPR training where the user exerts a large downward force, or for a boxing game where the player hits the punching bag at a high speed. 
% A large compliant object, such as a couch, can also safely provide whole-body feedback when, e.g., the user leans on or steps on the couch.


\subsubsection{Mechanism}
Owing to their purposes, objects or components in the room have functional mechanisms that can be repurposed in games and VR environments.
For example, rotating knobs and switches are common gameplay elements that players interact with to control the scene. 
Beyond using the mechanisms for their original purpose, we can focus on the constrained motion that they afford. A door handle affords a lever that can be repurposed as the handle of a lemon squeezer. The door affords a rotating motion constrained by the hinge that can only be moved along the edge of a circular plane at the height of the door handle. A drawer can only be pushed or pulled linearly.
% We can repurpose a doorknob and a light switch in the room to control elements in the VR game. 
% We can repurpose these physical counterparts to provide accurate feedback. 
% These elements are typically controlled by hand-held controllers or would require specialized active haptic devices to simulate, but 

\subsubsection{Temperature}
One may think about the thermal conductivity of a material as an affordance to design experiences that involve temperature. Touching glass windows generates a cooling sensation that can be repurposed for a cold environment like an igloo. On the other end of the spectrum, being wrapped around in a blanket over time feels warm and can simulate sitting near a fireplace. User-generated actions can also help create a breeze at different temperatures. Using a hair dryer can generate warm air and using a fan or fanning oneself with a book can generate a cool breeze.



\subsubsection{Texture}
Texture plays a key role in providing information about the material of an object, and at the same time, it is a challenging feature to recreate using active haptic devices. Existing objects in the room provide a rich library of textures. A fuzzy cushion is similar to an animal's fur, and a carpet is analogous to a grass field. With the help of visual cues, it is not difficult to imagine touching the curtain as feeling a willow tree or a giant's long hair.



% \subsubsection{Fragility}

% glass surface, mirrors and LCD screens are flat surfaces but fragile, so they are not suitable for performing hitting or actions that require large forces and or at high speed.

% glassware or ceramics require care and can be visually indicated as fragile gameplay elements.


% \subsection{Combined affordances}
% Book + pillow
% water bottle on table
% water bottle on top of pillow -> kneeding dough

% combining mobility and various functional affordances



\section{Limitation and Future Work}

\subsubsection*{Object Tracking and Recognition} 
To reduce the manual effort for configuration, we are interested in exploring real-time object tracking using LiDAR and object recognition. In this case, once an object has been located and configured using a controller, we assume that the object is either stationary or follows the hands’ movement. Being able to adapt to the physical objects properties and features would further enhance the immersivity. For example, being able to recognize the texture of a cushion and adapt the cat's fur to the texture or change the curvature of the cannon's handle to match that of the back of the chair.

% , which would still be limiting when the objects are partially occluded

% \subsubsection*{Hand Tracking} For the most part, current hand tracking technologies work reliably well for interacting with static objects and interactions that does not require self-occluding hand gestures. The "Shoot monsters" scenario, on the other hand, suffered from the lack of accuracy in detecting grasps. The scenario involves moving of the object, and as discussed in the above section, our current approach relies on hand tracking technology to synchronize the movement between the virtual and physical object. At the moment, most state-of-the-art hand tracking focuses on the pinching gesture, whereas in reality, pinching is not suitable for a range of natural interactions (e.g., grabbing, squeezing, and clenching), but we have no doubt that the hand tracking technology will improve and allow for even more immersive experiences using our technique.

\subsubsection*{Variability in Objects} Even though we chose common objects that one can find in a room, we expected variations in the objects in our participants’ space. 
% The variation in the chair had the most distinct effect on the experience from participants’ feedback. 
For example, the back of the chair had variations in curvature and some participants’ chairs could not swivel, which limits how the virtual cannon could be manipulated. 
Also, from the participants’ feedback, the mismatch between the flat table surface and the binary state of the “mole” was prevalent and affected the experience more than other factors. 
Future work should investigate how to address these variations of physical objects through, for example, automated virtual shape deformation.
% Future work should investigate how to address these variations of the physical objects through, for example, an automated virtual shape deformation.
% There were variations in the texture of the cushion, and some are closer to the texture of a cat’s fur than others. 
% We did not receive any feedback about the table’s features so we could not make any conclusive insights, but 

% When asked about other scenarios that could use this technique, participants reported games that involve small, mobile objects such as the pillow or a book, modifying or substituting controllers to fit the context of the game, and sitting or standing on couches during the VR experience.

\subsubsection*{Fill the Gap of the Design Space} 
Finally, as discussed in the Design Space section, there is an interesting design challenge in leveraging \textit{functional affordances}, rather than the shape of an object itself. Moreover, our design space exploration is just an initial investigation, and we did not demonstrate or prototype most of the exemplary scenarios illustrated in Figure~\ref{fig:design-space}. Future work should further investigate and demonstrate other possible VR haptic applications by filling the gap in the current design space, and then examine how these affordances allow for richer haptic experiences in real-world environments. In addition, electric or motorized appliances such as an air-conditioner and Roomba as in MoveVR\cite{wang2020movevr}, could afford even more dynamic and diverse virtual applications. 


% \subsubsection*{Common Mechanisms} We only focused on mobility as one variable for our design space. Building on the findings in \cite{simeone2015substitutional} where the similarity in affordance allows for a more believable substitution, future work can leverage the functional affordance of everyday objects, and here we describe a few exemplary uses for different types of functional affordances. The compliance of a couch or a cushion could be used for CPR training where the user exerts a large downward force, or for a boxing game where the player hits the punching bag at a high speed. We can repurpose a doorknob and a light switch in the room to control elements in the VR game. Beyond using the mechanisms for their original purpose, we can focus on the constrained motion that they afford. A door handle affords a lever that can be repurposed as the handle of a lemon squeezer. The door affords a rotating motion constrained by the hinge that can only be moved along the edge of a circular plane at the height of the door handle. A drawer can only be pushed or pulled linearly. One may also think about the heat capacity of a material as an affordance to design experiences that involve temperature. User-generated actions can also help create a breeze at different temperatures. Using a hair dryer can generate warm air and using a fan or fanning oneself with a book can generate a cool breeze. Texture plays a key role in providing information about the material of an object yet challenging to recreate in VR. A fuzzy cushion is similar to an animal's fur, and a carpet is analogous to a grass field. With the help of visual cues, it is not difficult to imagine touching the curtain as feeling a willow tree or a giant's long hair.


% Many gameplay elements involve the use of pulling a drawer open and toggle on and off switches, which can be found in a physical room. Compared to static objects, these mechanisms are more difficult to track and configure, and interactions with these mechanisms require more precise, fine motor control. For future work, we hope to include these mechanisms to provide more dynamic kinesthetic haptic feedback. Furthermore, some of the everyday objects in our design space are less common and accessible than others. For example, walls, tables, chairs, and cushions are common household items, but a handheld vacuum is not guaranteed to be found in a home. Designers could consider multiple options of objects for repurposing. 


% {\it Shared experience} – We have only explored adapting to a single user's environment. Merging and sharing of the same VR experience but configured in different physical environments is another opportunity for remote collaborative gameplay in VR. Future work can explore whether bringing physical objects in multi-user experience could enhance collaboration and a feeling of shared presence.

% {\it Safety} – Conventionally, users are asked to clear up a space and set up a guardian boundary where no obstacles are included within. This safety measure is important to prevent harming the user during an immersive VR experience, and it should still be respected for our technique. For our implementation, we only included one physical object at a time in the guardian boundary, and the physical object is actively involved during the experience such that it is not left unattended and potentially forgotten. Future work should investigate how to provide visual guidance in such a virtual-physical hybrid environment to safe guard the users and to convey expected haptic feedback.







\section{Conclusion}

In this paper, we investigated the design space and the recognition method of voice-accompanying hand-to-face (VAHF) gestures to enhance voice interaction with  parallel gesture channels. To design VAHF gestures, we first conducted an elicitation study, resulting in a total proposal of 15 gestures, followed by a hierarchical analysis process to output the most salient 8 gestures with the least ambiguity and physical confusion. Then we proposed a novel cross-device sensing method fusing different sensor channels to recognize para-linguistic hand-to-face gestures, achieving a high recognition accuracy of 97.3\% for 3+1\revision{(empty)} gestures and 91.5\% for 8+1\revision{(empty)} gestures recognition on our cross-device VAHF dataset. The uniqueness of our work is that we explored a broadened and scalable VAHF-gesture-based interaction space, which remains under-researched, to facilitate voice interaction in a more diverse manner (e.g., defining a shortcut or parsing parameters). Compared with prior work \cite{10.1145/3411764.3445687,Yan-UIST-2019} where a specific gesture (e.g., bringing the phone to the mouth\cite{10.1145/3411764.3445687}) was designed and recognized for 1-bit modality control (e.g., activating the voice assistant), our multi-device sensing framework is not only capable for recognizing up to 8 VAHF gestures simultaneously \revision{from the hand-off "empty" gesture}, but also benefits from the scalability (e.g., adding a device or adding a gesture is easy under our framework). As mobile devices and scenarios are becoming prevalent these years, voice input has become an essential modality of pervasive interaction. We envision our work would further enhance the efficiency and capability of current voice interaction and serve an important role in the future voice interaction of various scenarios like AR and IoT.





%\bibliographystyle{abbrv}
% \bibliographystyle{abbrv-doi}
%\bibliographystyle{abbrv-doi-narrow}
%\bibliographystyle{abbrv-doi-hyperref}
%\bibliographystyle{abbrv-doi-hyperref-narrow}
\bibliographystyle{ACM-Reference-Format}

\bibliography{template}

\appendix

\section{Exemplary scenarios}
We selected and built three scenarios to demonstrate objects with various affordances.

\subsection{Whack-a-Mole} A Whack-a-Mole game is a game where the player hits the “moles” that peeks through the game platform before they disappear. In a typically VR experience, the user’s hand feels nothing when hitting a virtual button or the mole in mid-air, and the hand also goes through the virtual game platform. In this scenario, we leverage the flat surface of a tabletop to be a game platform. When the user presses the virtual button to initiate the game and taps the mole to win points, their hand is stopped by the tabletop surface.  For implementation, we attach the game platform to the front of the controller but the platform is hidden at first. When the user places the ring of the controller flush against the flat surface and presses the trigger, the game surface is enabled and placed where the controller ring is, which then aligns with the physical surface (Figure \ref{fig:demo-table}).



\subsection{Pet a Cat} In this VR experience, users interact with a virtual pet. When a user reaches out their hand and wants to pet the animal, their hand normally feels nothing. For this scenario, we repurpose a cushion as a cat. When the user pets the cat, they feel both the texture and the compliance of the cushion that mimics the haptic feedback from that of a cat’s belly. To implement this, similarly, we attach a virtual cat to controller. When the user aligns the controller against the center of a cushion and confirms with a trigger, the top of the cushion aligns with the cat’s belly. When the user pets the cat, they hear the cat cry and purr and also see it wiggle its body as a gesture of affinity (Figure \ref{fig:setup-process}E).


\subsection{Shoot Monsters} The user controls a cannon to aim and shoot the monsters to win points. As the monsters appear in different locations, they would need to grab the handle of the cannon to rotate and aim at the monsters. If the user controls the cannon with their hands instead of a joy stick, their hands would go through the handle and feel no haptic feedback that would give them a sense of control of the cannon. Here we use a chair and leverage the movement constraints to enhance this VR experience. The cannon appears to be heavy and cannot be easily picked up, and so is the chair. The cannon has wheels that allows it to move along the ground, and its handles signals that it can be rotated. Even though the shape of the chair does not match the virtual cannon, the back of the chair affords grabbing and the rotating motion of the moveable chair matches the same motion constraint of the cannon. For implementation, we press the controller against the top edge of the back of the chair, such that the handle of the cannon then aligns with the chair’s back. The virtual handles are grabbable objects that follow the users' hands, which allows the movement of the virtual cannon and the physical chair to be in sync (Figure \ref{fig:demo-chair}).


% \section{User Study Results}


\end{document}
\endinput
