
\begin{figure*}[t!]
% \vspace{-5px}
% \resizebox{1\textwidth}{!}{
% \includegraphics[height=\textwidth]{Figures/designSpace1.png}
% \hspace{.5cm}
% \includegraphics[height=\textwidth]{Figures/designSpace2.png}
% }
\includegraphics[width=\textwidth]{Figures/dspace2.png}
% \includegraphics[width=.8\textwidth]{Figures/dspace3.png}
\caption{The design space of our idea. We illustrated two scenarios for each object to show the rich design opportunity.}
\label{fig:design-space}
\end{figure*}

\section{Design Space}
Through the participants' feedback and iterative design process, we also learned that the idea of VR haptics at home can go beyond the three demonstrations we developed. Therefore, based on our exploration, this section explores the possible design space by illustrating exemplary VR experiences for future investigation (Figure \ref{fig:design-space}). 
Moreover, we discuss how the \textit{affordance} of physical objects, rather than similarities in shapes, can affect virtual haptic experiences.

% For each interaction, we illustrate 
% how the affordance of virtual and physical objects rather than similarities in shapes, which we discuss below with exemplary VR experiences (Figure \ref{fig:design-space}). 
% We propose a technique that, instead of precluding indoor furniture and objects, incorporates and repurposes these objects to serve as passive haptic props for more immersive VR experiences. 
% Prior work explored repurposing physical objects for ones with a similar form and suggested the less mismatch in shape the better~\cite{simeone2015substitutional}. 




\subsection{Mobility}
% We define the mobility of an object as the ability to be relocated easily by a single person during the setup process and VR experience. 
For VR haptic scenarios, mobility depends on the object’s weight, size, and relation to the indoor environment, which provides constraints to the possible interactions. 
% In addition, tracking every single object is impractical at home (e.g., attaching optical trackers to objects in the home), thus mobility becomes an affordance uniquely important to VR, as misplacement can cause immediate break-in-presence. 


\subsubsection{Static Objects and Surfaces}
Static objects or surfaces are large and structural and thus are most suitable for setting reliable boundaries. 
% Couches can be used as a fort to lean against or a platform for stepping on and reaching things. 
Going beyond repurposing the entirety of the object, we can focus on the flat surface as an affordance. For example, tangible interactions such as organizing a digital photo gallery where the user swipes and moves the photos presented on a surface. We can also simulate haptic interactions such as playing the piano where the hand encounters a surface when touching the virtual interactables. Flat surfaces can be a constraint for mobile objects. A table air hockey striker can only move along the plane of an air hockey table.


\subsubsection{Semi-Static Objects}
We define semi-static objects as heavy objects that can be moved within certain constraints. Beyond being used for sitting, a chair can also be pushed along the floor. We can leverage the constrained motion and repurpose the chair as a lawnmower, a similarly heavy object that one does not pick up and only moves across the ground. If the chair swivels, the constrained rotating motion allows the chair to be repurposed as a game cannon. Likewise, a door naturally serves as a portal to a different world, but we could also leverage constrained motion around the hinge.

\subsubsection{Mobile Objects}
These objects are lightweight and can afford interactions such as grasping and gripping along with swift movements due to their mobility. A water bottle that can be grabbed with the whole hand and moved around is suitable for replacing a saber. In addition, mobile objects can afford bimanual manipulations. Once again, a water bottle can be grabbed with both hands and acts as a baseball bat. A mobile object is not constrained by its location. A cushion placed on the ground can be repurposed as a CPR dummy, but when it’s placed on the lap, it can be replaced as a pet cat.

\subsection{Functional Affordance}
Furthermore, building on the findings in Substitutional Reality~\cite{simeone2015substitutional} where the similarity in affordance allows for a more believable substitution, we focus on leveraging the functional affordance of everyday objects. 

\subsubsection{Compliance}
The home environment conveniently provides compliant objects that are otherwise difficult to simulate using active haptic devices. A cushion is a convenient substitute for a small animal that allows the user to feel the softness when touching and petting the animal. In addition, compliant objects can absorb impact from large forces or fast actions. For example, a couch or a cushion could be used for CPR training where the user exerts a large downward force, or for a boxing game where the player hits the punching bag at a high speed. 
% A large compliant object, such as a couch, can also safely provide whole-body feedback when, e.g., the user leans on or steps on the couch.


\subsubsection{Mechanism}
Owing to their purposes, objects or components in the room have functional mechanisms that can be repurposed in games and VR environments.
For example, rotating knobs and switches are common gameplay elements that players interact with to control the scene. 
Beyond using the mechanisms for their original purpose, we can focus on the constrained motion that they afford. A door handle affords a lever that can be repurposed as the handle of a lemon squeezer. The door affords a rotating motion constrained by the hinge that can only be moved along the edge of a circular plane at the height of the door handle. A drawer can only be pushed or pulled linearly.
% We can repurpose a doorknob and a light switch in the room to control elements in the VR game. 
% We can repurpose these physical counterparts to provide accurate feedback. 
% These elements are typically controlled by hand-held controllers or would require specialized active haptic devices to simulate, but 

\subsubsection{Temperature}
One may think about the thermal conductivity of a material as an affordance to design experiences that involve temperature. Touching glass windows generates a cooling sensation that can be repurposed for a cold environment like an igloo. On the other end of the spectrum, being wrapped around in a blanket over time feels warm and can simulate sitting near a fireplace. User-generated actions can also help create a breeze at different temperatures. Using a hair dryer can generate warm air and using a fan or fanning oneself with a book can generate a cool breeze.



\subsubsection{Texture}
Texture plays a key role in providing information about the material of an object, and at the same time, it is a challenging feature to recreate using active haptic devices. Existing objects in the room provide a rich library of textures. A fuzzy cushion is similar to an animal's fur, and a carpet is analogous to a grass field. With the help of visual cues, it is not difficult to imagine touching the curtain as feeling a willow tree or a giant's long hair.



% \subsubsection{Fragility}

% glass surface, mirrors and LCD screens are flat surfaces but fragile, so they are not suitable for performing hitting or actions that require large forces and or at high speed.

% glassware or ceramics require care and can be visually indicated as fragile gameplay elements.


% \subsection{Combined affordances}
% Book + pillow
% water bottle on table
% water bottle on top of pillow -> kneeding dough

% combining mobility and various functional affordances
