\documentclass[preprint,showpacs,preprintnumbers,amsmath,amssymb]{revtex4-1}
\usepackage{graphicx}
\bibliographystyle{naturemag}
\setcitestyle{super}
\usepackage{graphicx}
\usepackage{color}
\usepackage{bm}
\usepackage{braket}
\usepackage{amsmath}
\usepackage{natbib}
\def\cblue{\color{blue}}
\def\cred{\color{red}}

\begin{document}

\title{Supplementary information of exotic heavy fermion superconductivity in atomically thin CeCoIn$_5$ films}
\author{L. Peng$^{1}$, M. Naritsuka$^{1,2}$,  S. Akutagawa$^{1}$,  S. Suetsugu$^{1}$,  M. Haze$^{3}$, Y. Kasahara$^{1}$, T. Terashima $^{1}$, R. Peters$^1$,  Y. Matsuda$^{1}$, and T. Asaba$^{1*}$}

\affiliation{
$^1$Department of Physics, Kyoto University, Kyoto 606-8502, Japan.\\
$^2$School of Physics and Astronomy, University of St Andrews, North Haugh, St Andrews, Fife KY16 9SS, UK.\\
$^3$Institute for Solid State Physics, University of Tokyo, Kashiwanoha 5-1-5, Kashiwa, Chiba 277-8581,  Japan.\\
}
\date{\today}
\maketitle

\newpage
\renewcommand\thefigure{S\arabic{figure}}    


%\section*{Materials and Methods}
\noindent
 {\bf Formation of hybridization gap}
	
In the main text, we showed the asymmetric hybridization gap from atomically thin films of CeCoIn$_5$. The hybridization gap is asymmetric because of the gap formation process. Shown in Fig.\,S1 is the schematic diagram of the hybridization gap formation. Below the Kondo temperature, localized f-electrons hybridize with conduction electrons, opening the hybridization gap (Fig.\,S1). Since dI/dV is proportional to the local density of states, the gap becomes symmetric with respect to the Fermi energy only when (1) the DOS is symmetric to the gap center and (2) the Fermi energy locates at the gap center. Generally, neither condition is satisfied, resulting in an asymmetric hybridization gap. 



\noindent
 {\bf Derivative of the spectra}
	
For the second derivative curves, we have also tried to first fit the data using Gaussian+polynomial, then take the second derivative of the fitted curves(Fig.\,S2). The particle-hole symmetry is also observed.


%%%%%%%%%Figure1$$$$$$$$$$$
\begin{figure}[h]
	\centering
	\includegraphics[width=0.9\linewidth]{SIFigureBand.eps}
	\caption{Schematic image of the formation of hybridization gap.}
\end{figure}
%%%%%%%%%Figure1$$$$$$$$$$$S



%%%%%%%%%Figure1$$$$$$$$$$$
\begin{figure}[h]
	\centering
	\includegraphics[width=0.9\linewidth]{SIFigureFit.eps}
	\caption{(left) The Gaussian+third-order polynomial fitting of the trilayer data. (right) The second derivative of fitted dI/dV curves.}
\end{figure}
%%%%%%%%%Figure1$$$$$$$$$$$S


%%%%%%%%%Figure1$$$$$$$$$$$
\begin{figure}[h]
		\centering
	\includegraphics[width=0.9\linewidth]{SIFigure1.eps}
	\caption{\label{SI_Fig:basic}  (a)-(c) STM images ($V_s = $100 mV, $I_t = $50 pA) taken on terrace A, B and C in Fig. 1(c), respectively.  For (a) and (c), bright spots that form the square lattice represent the In atoms in the Ce-In plane. For (b), bright spots represent the Co atoms.  Black bars denote $1$\,nm. (d)-(f) Fourier transform images of (a)-(c), respectively. Scale bar: 1.9\,nm$^{-1}$.}
\end{figure}
%%%%%%%%%Figure1$$$$$$$$$$$S

%%%%%%%%%Figure2$$$$$$$$$$$
\begin{figure}[h]
		\centering
	\includegraphics[width=\linewidth]{SIFigure2.eps}
	\caption{ (a)(b) Large scale STM images. The right-bottom corner of (a) is connected to the left-top corner of (b). White bars denote 40\,nm. (c) A line section along the red arrow in (b). Each step height is consistent with the $c$-axis lattice constant of CeCoIn$_5$. Image conditions: $V_s = $2.0 V, $I_t = $50 pA.}
\end{figure}
%%%%%%%%%Figure2$$$$$$$$$$$

%%%%%%%%%Figure2$$$$$$$$$$$
\begin{figure}[h]
		\centering
	\includegraphics[width=0.5\linewidth]{SIFigure3.eps}
	\caption{Temperature dependence of tunnelling conductance d$I$/d$V$ spectra in zero magnetic field. The d$I$/d$V$ spectra are normalized by that at 1.6\,K, then normalized by d$I$/d$V$ at V=-0.5\,mV.   Light blue region corresponds to superconducting gap area.  The gray lines represent the background obtained by cubic fitting in the high bias regime.}
\end{figure}
%%%%%%%%%Figure2$$$$$$$$$$$

%%%%%%%%%Figure2$$$$$$$$$$$
\begin{figure}[h]
		\centering
	\includegraphics[width=0.9\linewidth]{FittingCurves_Filed.eps}
	\caption{ (a) Field dependence of d$I$/d$V$ spectra at $T$=0.4\.K. The d$I$/d$V$ spectra are normalized by that at 1.6\,K, then normalized by d$I$/d$V$ at V=-0.5\,mV.   Light blue region corresponds to the superconducting gap area.  The gray lines represent the background obtained by polynomial fitting in the high bias regime. The SC gap still survives even at 11\,T. (b) Magnetic field dependence of the SC gap area. %The calculated SC gap area gradually decreases monotonously as magnetic filed.
	}
\end{figure}
%%%%%%%%%Figure2$$$$$$$$$$$

%%%%%%%%%Figure2$$$$$$$$$$$
\begin{figure}[h]
		\centering
	\includegraphics[width=0.9\linewidth]{SIFigure5.eps}
	\caption{ (a)-(d) The applied field is $\mu_0 H $= 3, 7, 10 and 11\,T, respectively. Spectra are taken on the same site on the trilayer CeCoIn$_5$ surface. The spectra are vertically shifted for clarity.  Tunnelling parameters: $V_s=2$ mV, $I_t=50$ pA, $V_{mod}=30$$ \mu $V.}
\end{figure}
%%%%%%%%%Figure2$$$$$$$$$$$

%%%%%%%%%Figure2$$$$$$$$$$$
\begin{figure}[h]
		\centering
	\includegraphics[width=0.9\linewidth]{SIFigure6.eps}
	\caption{Temperature dependence of the superconducting gap area $S(T)$ at different magnetic fields. We fit the data by using $S(T) \propto \sqrt{1-(T/T_c)^3}$ at each field and obtain  $T_c$=1.22, 1.15, 0.98 and 0.95\,K at $\mu_0H$=3, 7, 10 and 11\,T, respectively. The data at zero temperature are normalized to 1 ($S (T\,=\,0)\,=\,1$). }
\end{figure}
%%%%%%%%%Figure2$$$$$$$$$$$

%%%%%%%%%Figure2$$$$$$$$$$$
\begin{figure}[h]
		\centering
	\includegraphics[width=0.9\linewidth]{SIFigure7.eps}
	\caption{Robust superconductivity against magnetic fields in multiple samples. In addition to the sample shown in the main text, the survival of superconductivity at 11\,T was observed from two other samples.  As the position of the STM tip differs with different magnetic fields, the gap shape is influenced by inhomogeneity.%The universal enhancement of  $H_{c2\perp}^{orb}$ indicates the intrinsic origin of the  that the enhancement originates from an intrinsic nature.
	}
\end{figure}
%%%%%%%%%Figure2$$$$$$$$$$$

%%%%%%%%%Figure2$$$$$$$$$$$
\begin{figure}[h]
		\centering
	\includegraphics[width=0.8\linewidth]{SIFigure8.pdf}
	\caption{Resistivity curve of a trilayer CeCoIn$_5$ superlattice. Three layers of CeCoIn$_5$ are sandwitched by five layers of YbCoIn$_5$. The total thickness of the superlattice is $\sim$ 120\,nm. A clear superconducting transition with onset Tc$\sim$1.3\,K is observed.  %The universal enhancement of  $H_{c2\perp}^{orb}$ indicates the intrinsic origin of the  that the enhancement originates from an intrinsic nature.
	}
\end{figure}
%%%%%%%%%Figure2$$$$$$$$$$$

%%%%%%%%%Figure2$$$$$$$$$$$
\begin{figure}[h]
	\centering
	\includegraphics[width=0.8\linewidth]{SIFigureLinecut.eps}
	\caption{(a) dI/dV spectra of monolayer CeCoIn$_5$ taken along the line. (b) dI/dV spectra of the edge between YbCoIn$_5$ and CeCoIn$_5$ monolayer taken along the line.  The feature around zero bias on the bottom curve is either due to the experimental error or associated with the inhomogeneity of the sample.
	}
\end{figure}
%%%%%%%%%Figure2$$$$$$$$$$$





\end{document}