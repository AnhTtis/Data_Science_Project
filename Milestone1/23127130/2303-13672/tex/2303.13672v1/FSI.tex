

\hypertarget{FSI-problem}{%
\subsection{Beam Support Design}\label{FSI-problem}}

The design of a beam support in a fluid channel is optimized in this section with the setup in \cite{Jenkins2016}. The goal is to demonstrate the generality of the method which is shown here by its capacity to solve a multiphysics problem with complex boundary conditions. Density methods do not extend naturally to handle such problems and are faced with difficulty in obtaining accurate coupling between the fluid and structure since the representation of the interface is spread across cells in the vicinity of the boundary. 

\begin{figure}
	\centering
	\incfig[0.55]{FSISetup}
	\caption{The beam support problem setup. An inlet velocity is prescribed on the left side, a homogenous Dirichlet condition on the top and bottom walls and a homogenous Neumann condition on the right wall $\Omega_N$. 
	The design region supporting the beam in light grey can be either fluid or solid. 
	The solid domain $\Omega_{\mathrm{in}}$ is composed of the dark grey beam and the solid part of the design region.
	The fluid region $\Omega_{\mathrm{out}}$ is composed of the remainder of the channel, including the non-solid part of the design region.}
	\label{fig:FSISetup}
\end{figure}



The problem setup is seen in Figure \ref{fig:FSISetup}. 
For this problem, we set the fluid parameters as $\mu=1 m^2s^{-1}$ and $\alpha_u=2.5 \mu / 0.01^2 $ and use a parabolic velocity profile on the inlet with an average velocity of $0.01 ms^{-1}$.
For the structural parameters, we set $E=1 Pa$ and $\nu=0.3$ and use a $0.45$ volume fraction. 
We use the same network as in the poisson problem but set $w = (12,12,24,46,96,96)$ and $l = (24,24,48,96,192,192)$. The optimized geometry for problem is seen in Figure \ref{fig:FSI-result}.
	


%\begin{comment}
	\begin{figure}
		\centering
		%\includegraphics[width=84mm]{./figures/FSI_temp.png}
		% neuralFSI3.5
		\setlength{\abovecaptionskip}{-10pt plus 1pt minus 20pt }
		\incfig[1]{FSIResult}%}{fsi9del}
		\caption{Optimized geometry for the beam support problem. Streamlines are plotted to represent the velocity field and pressure is indicated in the legend. } 
		\label{fig:FSI-result}
		\end{figure}
%\end{comment}


