

\section{Numerical Discretization}\label{simulation}


\begin{figure}
  \centering
  \incfig[0.55]{DomainDefinition}
  \caption{Illustration of the problem domains. The background domain is segmented into the domains $\Omega_{\rm in}$ and $\Omega_{\rm out}$. In $\Omega_{\rm in}$, the physical terms $\mathcal{R}(\boldsymbol{u},\boldsymbol{v})$ are integrated. In $\Omega_{\rm out}$, the stabilization terms $ \mathcal{R}^{\mathrm{stb}}(\boldsymbol{u},\boldsymbol{v})$ are integrated. }
  \label{fig:domain-definition}
\end{figure}

Here we formulate the numerical discretization of the PDEs being solved in the simulation step of the optimization loop. We consider a background polyhedral bounded domain $\Omega$ with boundary $\partial \Omega$. A \ac{ls} function $\phi(x)$ is used to split $\Omega$ into two subdomains $\Omega_{\rm in}(\phi)$ and $\Omega_{\rm out}(\phi)$ as follows (see Figure \ref{fig:domain-definition}):
\begin{equation}
  \begin{aligned}
     & \Omega_{\rm in}(\phi) = \{ x \in \Omega :  \phi(x) > 0 \},
    \quad \Omega_{\rm out}(\phi) = \{ x \in \Omega :  \phi(x) < 0 \}.
  \end{aligned}
  \label{eq:Omega_def}
\end{equation}
We denote the interface between these two subdomains as $\Gamma(\phi) \doteq \partial \Omega_{\mathrm{in}}(\phi) \cap \partial \Omega_{\mathrm{out}}(\phi)$.
Let us assume that $\partial \Omega_{\mathrm{in}}(\phi) \cap \partial \Omega \neq \emptyset$. The domain $\Omega_{\mathrm{in}}(\phi)$ is the one in which we consider our PDE problem. The weak form of the continuous problem can be stated as follows: find $\boldsymbol{u} \in V$ such that
\begin{equation}
  \int_{\Omega_{\mathrm{in}}(\phi)} \mathcal{R}(\boldsymbol{u},\boldsymbol{v}) \mathrm{d} \Omega = 0, \quad \forall v \in V.
  \label{eq:R}
\end{equation}
where $V$ is a Hilbert space in which the problem is well-posed. We consider zero flux Neumann boundary conditions on $\partial \Omega_{\mathrm{in}}(\phi) \setminus \partial \Omega$ and (for simplicity)  homogeneous boundary conditions on  $\partial\Omega_{\mathrm{in}}(\phi) \cap \partial \Omega$; the generalization to non-homogeneous Dirichlet boundary conditions is straightforward.

The domain $\Omega_{\mathrm{in}}(\phi)$ will change along the optimization process. As a result, it is not practical to compute body-fitted unstructured meshes for the geometrical discretization of $\Omega_{\mathrm{in}}(\phi)$. Instead, we consider a background mesh, which can simply be a Cartesian background mesh of $\Omega$  and make use of an unfitted \ac{fe} discretization. Unfitted (or embedded) discretizations relax the geometrical constraints but pose additional challenges to the numerical discretization \cite{Burman2010,Badia2018}. The first issue is the integration over cut cells (for details, see \cite{Badia2022-Geometrical}). The other issue is the so-called \emph{small cut cell} problem. Cut cells with arbitrary small support lead to ill-conditioned systems \cite{dePrenter2017}. Various techniques can be used to stabilize the problem including the \ac{cutfem} \cite{Burman2010}, the \ac{agfem} \cite{Badia2018} and the 
\ac{fcm} \cite{Parvizian2007}. 
We select the \ac{fcm} method in this case because it is differentiable with respect to the \ac{ls} 
everywhere in the domain. The \ac{cutfem} and \ac{agfem} are conversely not differentiable. %(see Proposition \ref{thm:CutFEM-nondiff}). %\sbcom{\hl{acronyms, I would use a package to be sure that we define it the first time we use them}}. The idea of \ac{fcm} method is to add a penalty term in the artificial domain $\Omega_{\mathrm{out}}$ that makes the discrete linear system solvable. Even though this method is not consistent and intrinsically low order, it is suitable for \ac{to} applications, which are intrinsically low-order.

In order to state the discrete form of the continuous problem, we introduce the unfitted \ac{fe} space. Let $\mathcal{T}_h$ represent a conforming, quasi-uniform and shape regular partition (mesh) of $\Omega$, $h$ being a characteristic mesh size. $\Omega$ can be a trivial geometry, e.g., a square or cube, and $\mathcal{T}_h$ can be a Cartesian mesh. We define a nodal Lagrangian \ac{fe} space of order $q \geq 1$ on $\mathcal{T}_h$ as:
\begin{equation}
  V_{h}^q = \{ \boldsymbol{v}_h \in \mathcal{C}^0(\Omega) : \boldsymbol{v}_h\vert_{K} \in \mathcal{X}_q(K) \ \forall K \in \mathcal{T}_h \},
\end{equation}
where $\mathcal{X}_q(K)$ is the space $\mathcal{Q}_q(K)$ of  polynomials with maximum degree $q$ for each variable when $\mathcal{T}_h$ is a quadrilateral or hexahedral mesh and the space $\mathcal{P}_q(K)$ of polynomials of total degree $q$ when $\mathcal{T}_h$ is a simplicial mesh. In this work, we consider low order spaces, which is the most reasonable choice for \ac{to} applications.

The weak formulation of the problem solved by the method is now described. Let us represent with $V_{h,0}^q = V_h^q \cap \mathcal{C}_{0}^{0}(\bar{\Omega})$ the nodal \ac{fe} space that vanishes on the boundary $\partial \Omega$. 

Now, we can define a first-order \ac{fcm} discretization of (\ref{eq:R}) as follows: find $\boldsymbol{u}_h \in V_h^1$ such that
\begin{equation}
  \int_{\Omega_{\mathrm{in}}(\phi)} \mathcal{R}(\boldsymbol{u}_h,\boldsymbol{v}_h) \mathrm{d} \Omega +
  \int_{\Omega \setminus \Omega_{\mathrm{in}}(\phi)} \alpha_{\mathrm{out}} \mathcal{R}^{\mathrm{stb}}(\boldsymbol{u}_h,\boldsymbol{v}_h) \mathrm{d} \Omega = 0, \quad \forall v \in V_h^1,
\label{eq:RFCM}
\end{equation}
where $\alpha_{\mathrm{out}} \ll 1$ is the penalty parameter and $\mathcal{R}^{\mathrm{stb}}$ is a \emph{stabilizing}  differential operator on the artificial domain.



\newcommand{\ubs}{\boldsymbol{u}}

\subsection{Poisson Equation}\label{poisson-equations}


The poisson equation is used to model the temperature $\theta(\boldsymbol{x})$ 
that satisfies
\begin{equation}\protect\hypertarget{eq:T}{}{
    -\boldsymbol{\nabla}\cdot (\boldsymbol{\kappa} \boldsymbol{\nabla} \theta) = f \quad \text{in}\ \Omega_{\mathrm{in}}(\phi),
  }\label{eq:T}\end{equation}
where $\boldsymbol{\kappa}$  is the thermal conductivity tensor of the material and $f$ is a thermal source.
A zero Dirichlet condition ($\theta=0$) is prescribed on $\Omega_{\mathrm{in}}(\phi) \cap \partial \Omega$ and a zero flux condition ($\boldsymbol{n}\cdot(\boldsymbol{\kappa}\boldsymbol{\nabla}\theta) = 0$) is prescribed on $\Gamma(\phi)$. 

The \ac{fcm} approximation of this problem without a source term reads as: find $\theta_h \in V_{h,0}^{1}$ such that  
  \begin{equation}
  \int_{\Omega_{\mathrm{in}}(\phi)} \boldsymbol{\kappa} \boldsymbol{\nabla}\theta_h \cdot \boldsymbol{\nabla}v_h  \mathrm{d} \Omega +
  \int_{\Omega \setminus \Omega_{\mathrm{in}}(\phi)} \alpha_{\mathrm{out}} \boldsymbol{\kappa} \boldsymbol{\nabla}\theta_h \cdot \boldsymbol{\nabla}v_h \mathrm{d} \Omega = 0, \quad \forall v_h \in V_{h,0}^1.
  \label{eq:RT}
\end{equation}
One can readily check that this method is weakly enforcing the zero flux condition on $\Gamma(\phi)$ as $\alpha_{\mathrm{out}} \rightarrow 0$.  We observe that we use the same differential operator in the artificial domain for stabilisation purposes (times the scaling coefficient $\alpha_{\mathrm{out}}$).

We consider the \ac{to} problem in which we aim at finding a level-set $\phi$ that minimizes the integral of the temperature: %\sbcom{\hl{name?}}
\begin{equation}
  J(\phi,\theta_h(\phi))  = \int_{\Omega(\phi)} \theta_h(\phi) \  \mathrm{d}\Omega,
\end{equation}
where $\theta_h(\phi)$ is the solution of (\ref{eq:RT}) given $\phi$. 






\hypertarget{linear-elasticity}{%
  \subsection{Linear elasticity}\label{linear-elasticity}}

We want to obtain the displacement $\boldsymbol{d}(\boldsymbol{x})$ that satisfies the linear elasticity equation  
\begin{equation}\protect\hypertarget{eq:d}{}{
    \left.
    \begin{aligned}
      -\boldsymbol{\nabla}	\cdot \bm{\sigma} (\boldsymbol{d})  & = \boldsymbol{f} &  & \text{in} \   \Omega_{\rm in}(\phi),
    \end{aligned}
    \right.
  }\label{eq:d}\end{equation} where $ \bm{\sigma} = \lambda \text{tr}(\bm{\varepsilon})I+2\mu \bm{\varepsilon} $ is the stress tensor, $\bm{\varepsilon} = \frac{1}{2}(\boldsymbol{\nabla}\boldsymbol{d} + (\boldsymbol{\nabla}{\boldsymbol{d}})^\top)$ is the symmetric gradient, $ \(\boldsymbol{d}\)$ is the displacement, $\lambda$ and $\mu$ are the Lam\'e parameters given by $\lambda = (E\nu)/((1+\nu)(1-2\nu)) $ and $ \mu=E/(2(1+\nu)) $
and $\boldsymbol{f}$ is the forcing term. A zero Dirichlet condition ($\boldsymbol{d}=\boldsymbol{0}$) is prescribed on $\Omega_{\mathrm{in}}(\phi) \cap \partial \Omega$ and a zero stress condition ($\boldsymbol{n}\cdot \boldsymbol{\sigma}(\boldsymbol{d}) = 0$) is prescribed on $\Gamma(\phi)$.

The \ac{fcm} approximation of this problem without a forcing term reads as: find $\boldsymbol{d}_h \in \boldsymbol{V}_{h,0}^{1} \doteq  [V_{h,0}^1 ]^D$ such that  
  \begin{equation}
  \int_{\Omega_{\mathrm{in}}(\phi)} \boldsymbol{\sigma}(\boldsymbol{d}_h) : \boldsymbol{\varepsilon}(\boldsymbol{v}_h)  \mathrm{d} \Omega +
  \int_{\Omega \setminus \Omega_{\mathrm{in}}(\phi)} \alpha_{\mathrm{out}} \boldsymbol{\sigma}(\boldsymbol{d}_h) : \boldsymbol{\varepsilon}(\boldsymbol{v}_h)  \mathrm{d} \Omega = 0, \quad \forall \boldsymbol{v}_h \in \boldsymbol{V}_{h,0}^1.
  \label{eq:Rd}
\end{equation}
It is easy to check that the zero-stress condition on $\Gamma(\phi)$ is recovered as $\alpha_{\mathrm{out}} \rightarrow 0$. For the linear elasticity equation, we again use the same differential operator in the artificial domain for stabilisation purposes.


A typical \ac{to} problem in solid mechanics is the minimization of the strain energy. In this case, we aim at finding a level-set $\phi$ that minimizes the cost function 
\begin{equation}
  J(\phi,\boldsymbol{d}_h(\phi))  = \int_{\Omega(\phi)} \bm{\sigma}(\boldsymbol{d}_h):\bm{\varepsilon}(\boldsymbol{d}_h) \  \mathrm{d}\Omega,
  \label{eq:Jd}
\end{equation}
where $\boldsymbol{d}_h(\phi)$ is the solution of (\ref{eq:Rd}) given $\phi$. 



\subsection{Linear Elasticity with Fluid Forcing Terms}\label{FSI}

Once again, we want to obtain the displacement $\boldsymbol{d}_h(\boldsymbol{x})$ that satisfies the linear elasticity formulation in (\ref{eq:d}). In this case, however, we consider the surface traction exerted by the fluid:
\begin{equation}\protect\hypertarget{eq:rdGammas}{}{
  \begin{aligned}
    \int_{\Gamma(\phi)}
      ( \boldsymbol{n} \cdot \boldsymbol{\nabla}\boldsymbol{u}_h  - {p}_h \ \boldsymbol{n} ) \cdot \boldsymbol{v}
    {\rm d}x,
  \end{aligned}
}\label{eq:rdGammas}\end{equation}
where the fluid velocity $\boldsymbol{u}_h$ and pressure field ${p}_h$ are obtained by solving a fluid problem in the domain $\Omega_{\rm out}$. 
These fields are obtained by solving the Stokes equations with a Brinkmann penalization as in \cite{Borrvall2002} but without intermediate interpolation of permeabilities at the boundary.

In order to approximate the fluid problem, we use a mixed \ac{fe} method, namely the equal order pair $\boldsymbol{V}_{h,0}^{1} \times {V}_h^1$.


We find \((\boldsymbol{u}_h,{p}_h)\in \boldsymbol{V}_{h,0}^1 \times V_{h,0}^1 \)
such that:
\begin{equation}\protect\hypertarget{eq:rupOmega}{}{
  \begin{aligned}
    %\mathscr{R}_{\boldsymbol{u},p}^\Omega \doteq
    \int_{\Omega}%\cup\Omega_s\cup\Omega_f}
    &\left[
      \alpha \boldsymbol{u}_h \cdot \boldsymbol{\psi}_h +
      \mu \boldsymbol{\nabla}\boldsymbol{u}_h \cdot\boldsymbol{\nabla}\boldsymbol{\psi}_h 
      -{p}_h (\boldsymbol{\nabla}\cdot \boldsymbol{\psi}_h)
      -(\boldsymbol{\nabla}\cdot \boldsymbol{u}_h) {q}_h
      - h^2 \boldsymbol{\nabla}p_h \cdot \boldsymbol{\nabla}{q_h}
      \right] 
    {\rm d}x
    =0, \quad 
    \\
    &\forall \boldsymbol{\psi}_h,{q}_h \in \boldsymbol{V}_{h,0}^{1} \times {V}_h^1,  %, where $\mathscr{R}_{\boldsymbol{u},p}^\Omega$ is given by:
  \end{aligned}
}\label{eq:rupOmega}\end{equation}
where
\begin{equation}
  \left\lbrace
  \begin{aligned}
    \alpha & = 0        &  & \text{in } \Omega_{\rm out} \\%\cup \Omega_f,  \\
    \alpha & = \alpha_u &  & \text{in } \Omega_{\rm in} \\%\cup \Omega_s. \\
  \end{aligned}
  \right.
\end{equation}
using an artificial porosity $\alpha_u$ to make the fluid problem well posed in the solid domain $\Omega_{\rm in}$ and enforce the no-slip boundary condition. In the fluid domain $\Omega_{\rm out}$ we recover the Stokes equations.

The \ac{to} problem once again involves finding a level-set $\phi$ that minimizes the elastic strain using (\ref{eq:Jd}). 

\subsection{Differentiability of the unfitted \ac{fe} solver}\label{sec:}

An important property for the convergence of a \ac{to} strategy is the notion of shape differentiability of the cost function. 
A functional under a PDE constraint is considered shape differentiable if the mapping $\phi \rightarrow J(\boldsymbol{u_h}(\phi),\phi)$ 
is differentiable at the admissable set of domains in $\Omega$ defined by $\phi$.
In this section, we discuss how the choice of \ac{fe} stabilization can effect this property. 

The model problem in \eqref{eq:RT} with a \ac{fcm} stabilization is equivalent to that of a typical two phase conductivity problem. 
With a solution $\boldsymbol{u}_h\in H^1(\Omega)$, the functional $J(\boldsymbol{u_h},\phi)$ for this problem can be proven to be shape differentiable, see \cite[Theorem 4.9]{Allaire2021}. 

Conversely, unfitted techniques involving stabilization only in the vicinity of the boundary are in general not shape differentiable. 
Regions of non differentiability arise (typically when the boundary crosses over mesh nodes) harming the convergence of the geometry to optimized solutions \cite{Sharma2016}. To see why this is the case, we investigate shape pertubations under the \ac{cutfem} and \ac{agfem} formulations.



\begin{figure}
  \centering
  \incfig[0.55]{Pertubation}
  \caption{A small perturbation with size $\epsilon$ to the \ac{ls} to form a new domain $\Omega_{\rm in}^\epsilon$. }
  \label{fig:pertubation}
\end{figure}



If we were to use a restricted space for the solution and add ghost penalty terms in the vicinity of the interface following the CutFEM method \cite{CutFEM2015}, the problem is to find $u_h\in W_{h,0}^1.$ such that:
\begin{equation}
  \begin{aligned}
    \int_{\Omega_{\mathrm{in}}(\phi)} \mathcal{R}(\boldsymbol{u}_h,\boldsymbol{v}_h) \mathrm{d} \Omega_{\Omega_{\rm in}} +
    j(\boldsymbol{u}_h,\boldsymbol{v}_h)_{\Gamma_G} = 0, \quad \forall v_h \in W_{h,0}^1,
  \end{aligned}
\end{equation}
for a ghost penalty term $j$ on a ghost skeleton triangulation $\Gamma_G$ using the space $W_{h,0}^1$ as in \cite{CutFEM2015}. 
Consider the change to the domain $\Omega_{\rm in}$ from Figure \ref{fig:domain-definition} caused by a perturbation $\epsilon \delta$, where $\epsilon \in \mathbb{R}$ and $\delta\in V^1_h$, to the \ac{ls} function $\phi$. The resulting domain $\Omega_{\rm in}^\epsilon$ may be as in Figure \ref{fig:pertubation}. The ghost penalty term in this formualtion does not depend on $\epsilon$ and instead changes depending on which cells are cut. Specifically, if $\Gamma^\epsilon$ crosses over a mesh node as in Figure \ref{fig:ghost-skeleton}, the ghost skeleton triangulation includes the faces of a new element. Non-zero terms are integrated on this triangulation introducing discontinuity to the problem with respect to the shape, since:
\begin{equation}
  \lim_{\epsilon \to 0} \  [ j(\boldsymbol{u}_h,\boldsymbol{v}_h)_{\Gamma_G^\epsilon} - j(\boldsymbol{u}_h,\boldsymbol{v}_h)_{\Gamma_G} ] \  \neq 0.
\end{equation}
With different terms added to the linear system which do not go to zero with $\epsilon$, we can see that the derivative of the solutions with respect to the shape can be ill-defined. A cost function operating on the solution could not, in general, be shape differentiable at these points.

\begin{figure}
  \centering
  \incfig[0.55]{GhostSkeleton}
  \caption{The change in a portion of the ghost skeleton triangulation $\Gamma_G$, depicted by the red faces, before and after a perturbation to the boundary.}
  \label{fig:ghost-skeleton}
\end{figure}

In the \ac{agfem}, the problem is to find $u_h\in V^{agg}_{h,0}$ such that:
\begin{equation}
  \begin{aligned}
	  \mathscr{R}(\mathcal{E}(\boldsymbol{u}_h),\mathcal{E}(\boldsymbol{v}_h))_{\Omega_{\rm in}} \quad \forall v_h \in V^{agg}_{h,0},
  \end{aligned}
\end{equation}
for the extension operator $\mathcal{E}$ and space $V^{agg}_{h,0}$ as defined in \cite{Badia2018}. Similar to the CutFEM, regions of non-differentiability in the problem exist when the zero isosurface of the \ac{ls} crosses over mesh nodes:
\begin{equation}
  \begin{aligned}
    \lim_{\epsilon \to 0} \ 
    [
      \mathscr{R}(\mathcal{E}(\boldsymbol{u}_h),\mathcal{E}(\boldsymbol{v}_h))_{\Omega_{\rm in}} 
    -
    \mathscr{R}(\mathcal{E}^\epsilon(\boldsymbol{u}_h),\mathcal{E}(\boldsymbol{v}_h))_{\Omega_{\rm in}^\epsilon} 
    ]
    \neq 0,
  \end{aligned}
\end{equation}
since $\mathcal{E}^\epsilon(\boldsymbol{u}) \neq \mathcal{E}(\boldsymbol{u}_h)$ in general because the support for cut cells potentially changes depending on which cells are cut. Following the same reasoning as above, the method is therefore not shape differentiable. The same is true for other methods which use stabilization approaches that act only in the vicinity of cut cells, e.g. \cite{Lang2014}, using similar branching strategies when crossing a node or reaching a certain threshold.
\qed
%\end{proof}




