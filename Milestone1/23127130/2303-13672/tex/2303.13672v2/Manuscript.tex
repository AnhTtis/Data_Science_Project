\documentclass[oneside,reqno,11pt,a4paper]{amsart}
\usepackage[T1]{fontenc}
\usepackage[margin=2.5cm]{geometry}
\usepackage{bm}
\usepackage{amsmath,amsfonts,amssymb,amscd,amsthm}
\usepackage[usenames,dvipsnames,svgnames]{xcolor}
%\usepackage[utf8x]{inputenc}
\usepackage{graphicx}
\graphicspath{}
\usepackage{caption}
\usepackage{subcaption}
\usepackage{doi}
% \usepackage{autonum}
\usepackage{makecell}
\usepackage{hyperref}
\usepackage{acronym}
\usepackage{listings}
\usepackage{textgreek}
\usepackage{url}
\usepackage{color}
\usepackage{framed}
\usepackage{verbatim}
\usepackage{fancyvrb}
% new times roman
%\usepackage{newtxtext}
%\usepackage{newtxmath} 
\usepackage{microtype} 
\usepackage[final]{pdfpages}
\usepackage{tikz}
\usepackage{booktabs}
\usepackage{multirow}
\usepackage{algorithm}
\usepackage{algpseudocode}

\usepackage[backend=biber,
defernumbers=true,
maxbibnames=5,
style=numeric-comp,
isbn=false,
bibencoding=utf8,
safeinputenc, % https://github.com/plk/biblatex/issues/819
url=false,
doi=true,
giveninits=true,
sorting=none,
]{biblatex}

%%%% Standard Packages
%%<additional latex packages if required can be included here>
\usepackage{textcomp}
\usepackage{cancel}
\usepackage{mathrsfs}
%%%% For importing extra symbols ( imported llbracket (square double bracket ))
\usepackage{stmaryrd}
\usepackage{parskip} %stopping indentation
%%%% figure support
\usepackage{import}
\usepackage{xifthen}
\usepackage{pdfpages}
\usepackage{hyperref}
%%%%%%%%usepackage{subfigure}
\usepackage{float}
\usepackage{graphicx}
\usepackage{siunitx}
\usepackage{placeins}
\usepackage[labelfont=bf]{caption}
\usepackage{enumitem}
\usepackage{cleveref}
\usepackage{import}
\usepackage{transparent}
\usepackage{xcolor}
\usepackage{longtable}
%%%%%%%\usepackage[mathletters]{ucs}
%\usepackage[utf8x]{inputenc}
\catcode`\|=12\relax
\usepackage{pgfplots}
\pgfplotsset{compat=newest}
\usepackage{program}
\usepackage{tikzscale}
%\usetikzlibrary{external}
%\tikzexternalize[prefix=tikz-tex/] % activate and define figures/ as cache folder
%%%%
%\usepackage{subfig}
\usepackage{soul}
\usepackage{pmboxdraw}

\newcommand{\mybasiclinewidth}{semithick}
%source is page 126 in the manual
\tikzset{
every picture/.style={
line width = 0.3mm} %or use: "`line width=1 pt,"<-- note:if you write line width, you must use a value with unit
}
\pgfplotsset{
every axis/.append style={
line width = 0.3mm,
grid style={
    line width = 0.3mm,
},
tick style={
    line width = 0.3mm,
},
},
}

%\addbibresource{/home/mal199/Documents/GitHub/ConnorMallon-Thesis/doc/thesis/Preamble/bibliography.bib}

\makeatletter
\let\blx@rerun@biber\relax
\makeatother

%\addbibresource{"refs.bib"}
\hypersetup{
%bookmarks=true,
breaklinks=true,
bookmarksopen=true,
pdftitle={Template article},    % insert title
pdfauthor={BadiaLab},     % insert author
colorlinks=true,       % false: boxed links; true: colored links
linkcolor=black,          % color of internal links (change box color with linkbordercolor)
citecolor=blue,        % color of links to bibliography
filecolor=black,      % color of file links
urlcolor=blue           % color of external links
}
\definecolor{bg}{rgb}{0.93,0.93,0.93}
\renewcommand\theadfont{\normalfont}
% Here add your article-wise definitions

\newtheorem{theorem}{Theorem}[section]
\newtheorem{lemma}[theorem]{Lemma}
\newtheorem{proposition}[theorem]{Proposition}
\newtheorem{corollary}[theorem]{Corollary}
\newtheorem{definition}[theorem]{Definition}
\newtheorem{hypothesis}[theorem]{Hypothesis}
\newtheorem{assumption}[theorem]{Assumption}
\newtheorem{remark}[theorem]{Remark}
\newtheorem{example}[theorem]{Example}
\newtheorem{method}[theorem]{Algorithm}

% Examples of acronyms
\acrodef{pde}[PDE]{partial differential equation}
\acrodef{fe}[FE]{finite element}
\acrodef{fem}[FEM]{finite element method}
\acrodef{fcm}[FCM]{finite cell method}
\acrodef{DOF}[DOF]{degrees of freedom}
%\acrodefplural{DOF}[DOFs]{degrees of freedom}
%\acrodef{agfe}[agFE]{aggregated finite element}
\acrodef{agfem}[AgFEM]{aggregated finite element method}
\acrodef{cutfem}[CutFEM]{cut finite element method}
%\acrodef{cg}[CG]{continuous Galerkin}
%\acrodef{dg}[DG]{discontinuous Galerkin}
\acrodef{ls}[LS]{level set}
\acrodef{to}[TO]{topology optimization}
\acrodef{nn}[NN]{Neural network}

% jump in DG
\newcommand{\tnor}[1]{{\left\vert\kern-0.25ex\left\vert\kern-0.25ex\left\vert #1 
\right\vert\kern-0.25ex\right\vert\kern-0.25ex\right\vert}}

% The following commands are extracted from https://arxiv.org/format/1411.1607
%\definecolor{shadecolor}{gray}{.92}
%\definecolor{incolor}{rgb}{0,0,.7}
%\definecolor{outcolor}{rgb}{.65,0,0}
%\definecolor{syntaxcolor}{rgb}{.65,0,0}
%\newcounter{jcounter}
%\newenvironment{jinput}[1][]{\vspace{-0.2cm}\ttfamily\hspace*{-.0in}\noindent\begin{minipage}[t]{0.95\textwidth}\vskip-0ex\begin{shaded}}{\end{shaded}\vspace{0.1cm}\end{minipage}\par}
%\newcommand{\sh}[1]{\textcolor{syntaxcolor}{#1}}

% comments
\newcommand{\sbcom}[1]{{{{#1}}}}

%\addbibresource{references.bib}
%\renewenvironment{proof}{{\bfseries Proof.}}{\qedsymbol}
\renewcommand{\qedsymbol}{\square}

\begin{document}


%\newcommand{\incfig}[2][1]{%
%    \def\svgwidth{#1\linewidth}
%    \includegraphics{./#2_converted.pdf}
%}

\newcommand{\incfig}[2][1]{%
    \def\svgwidth{#1\linewidth}
    \import{./figures/}{#2.pdf_tex}
}

\newcommand{\mycomment}[1]{}

\renewcommand{\multicitedelim}{\addcomma}

\title[Neural Level Set Topology Optimization Using Unfitted Finite Elements]{Neural Level Set Topology Optimization Using unfitted Finite Elements}

%%=============================================================%%
%% Prefix	-> \pfx{Dr}
%% GivenName	-> \fnm{Joergen W.}
%% Particle	-> \spfx{van der} -> surname prefix
%% FamilyName	-> \sur{Ploeg}
%% Suffix	-> \sfx{IV}
%% NatureName	-> \tanm{Poet Laureate} -> Title after name
%% Degrees	-> \dgr{MSc, PhD}
%% \author*[1,2]{\pfx{Dr} \fnm{Joergen W.} \spfx{van der} \sur{Ploeg} \sfx{IV} \tanm{Poet Laureate} 
%%                 \dgr{MSc, PhD}}\email{iauthor@gmail.com}
%%=============================================================%%

\author[]{Connor N. Mallon$^{1*}$}
\author[]{Aaron W. Thornton$^{2}$}
\author[]{Matthew R. Hill$^{1,2}$}
\author[]{Santiago  Badia$^{3*}$}
\thanks{\null\ 
$^{1}$ Department of Chemical and Biological Engineering, Monash University, Wellington Rd Clayton, 3800, Victoria, Australia.\ 
$^{2}$ CSIRO, Research Way Clayton, 3168, Victoria, Australia,\ 
$^{3}$ School of Mathematics, Monash University, Wellington Rd Clayton, 3800, Victoria, Australia.\ 
$^*$ Corresponding authors\ 
E-mails: {\tt connor.mallon@monash.edu} (Connor Mallon, Department of Chemical and Biological Engineering, Monash University, Wellington Rd Clayton, 3800, Victoria, Australia), {\tt santiago.badia@monash.edu} (Santiago Badia, School of Mathematics, Monash University, Wellington Rd Clayton, 3800, Victoria, Australia)}


%%==================================%%
%% sample for unstructured abstract %%
%%==================================%%

\begin{abstract}
To facilitate the widespread adoption of automated engineering design techniques, existing methods must become more efficient and generalizable. In the field of topology optimization, this requires the coupling of modern optimization methods with solvers capable of handling arbitrary problems. In this work, a topology optimization method for general multiphysics problems is presented. We leverage a convolutional neural parameterization of a level set for a description of the geometry and use this in an unfitted finite element method that is differentiable with respect to the level set everywhere in the domain. We construct the parameter to objective map in such a way that the gradient can be computed entirely by automatic differentiation at roughly the cost of an objective function evaluation. Without handcrafted initializations, the method produces regular topologies close to the optimal solution for standard benchmark problems whilst maintaining the ability to solve a more general class of problems than standard methods, e.g., interface-coupled multiphysics.
\end{abstract}

%%================================%%
%% Sample for structured abstract %%
%%================================%%

% \abstract{\textbf{Purpose:} The abstract serves both as a general introduction to the topic and as a brief, non-technical summary of the main results and their implications. The abstract must not include subheadings (unless expressly permitted in the journal's Instructions to Authors), equations or citations. As a guide the abstract should not exceed 200 words. Most journals do not set a hard limit however authors are advised to check the author instructions for the journal they are submitting to.
% 
% \textbf{Methods:} The abstract serves both as a general introduction to the topic and as a brief, non-technical summary of the main results and their implications. The abstract must not include subheadings (unless expressly permitted in the journal's Instructions to Authors), equations or citations. As a guide the abstract should not exceed 200 words. Most journals do not set a hard limit however authors are advised to check the author instructions for the journal they are submitting to.
% 
% \textbf{Results:} The abstract serves both as a general introduction to the topic and as a brief, non-technical summary of the main results and their implications. The abstract must not include subheadings (unless expressly permitted in the journal's Instructions to Authors), equations or citations. As a guide the abstract should not exceed 200 words. Most journals do not set a hard limit however authors are advised to check the author instructions for the journal they are submitting to.
% 
% \textbf{Conclusion:} The abstract serves both as a general introduction to the topic and as a brief, non-technical summary of the main results and their implications. The abstract must not include subheadings (unless expressly permitted in the journal's Instructions to Authors), equations or citations. As a guide the abstract should not exceed 200 words. Most journals do not set a hard limit however authors are advised to check the author instructions for the journal they are submitting to.}

%%\pacs[JEL Classification]{D8, H51}

%%\pacs[MSC Classification]{35A01, 6510, 65L12, 65L20, 65L70}

%\keywords{neural, unfitted, \ac{ls}, topology optimization}

\maketitle




\section{Introduction}\label{introduction}
After the birth of \acp{to} in the field of structural design \cite{Bendse1989}, efforts have been made to increase the effectiveness of such automated design approaches and allow for their deployment on a more general class of problems \cite{Sigmund2013,Guo2010}.   

A plethora of \ac{to} strategies exist through the literature, the most common of which being density-based methods using the so-called SIMP (Solid isotropic microstructure with penalization for intermediate densities) method \cite{Sigmund2001}. These involve varying a material distribution continuously 
between 0 and 1 to introduce an artificial representation of the
boundary. Although simple for basic structural problems, a way to represent intermediate design variables arising at
the boundary must be included, which becomes increasingly
complex in multi-physics applications and makes imposing
arbitrary boundary conditions non-trivial \cite{Yoon2014}. 

%para 4 : \ac{ls} topopt
An alternative technique that can overcome some of the problems presented by density methods and tackle a more general class of problems (e.g., interface-coupling multiphysics and problems that involve surface PDEs on boundaries) is the \ac{ls} \ac{to} method \cite{Osher1988,SethianJamesAlbert1999Lsma}
Using this approach, the boundary is
described by the zero iso-surface of a \ac{ls} function. It is instead this \ac{ls} function that is
varied to obtain optimized designs. A precise location of the boundary
is then available. 

A variety of alternative implementations of the \ac{ls} \ac{to} method
have been made \cite{vanDijk2013}. They can be distinguished, among other things, by how they update the topology at each iteration and their means of geometry mapping.
The methods to update geometries involve either updating the solution of Hamilton-Jacobi equations by a velocity field based on the sensitivity information \cite{Osher2001, Burman2018} or using a parameterization of the topology that is an explicit function of the design variables of a steepest descent optimization scheme. The latter approach allows one to leverage well-established nonlinear programming techniques and is the method selected for this work.
Types of geometry mappings include using the \ac{ls} function to define a conformal mesh to the boundary, see e.g. \cite{Ha2008,Yamasaki2011}, which requires re-meshing at each iteration, density-based mappings (see e.g. \cite{Allaire2004,Wang2003,Dugast2020}), which recover some of the issues related to density methods, or unfitted/immersed boundary techniques (see, e.g., \cite{Parvizian2011,Burman2015,badia_stokes_2018}). unfitted methods rely on a fixed background mesh and capture the precise location of the boundary in the model using integration on sub-triangulations aligned with the zero iso-surface of the \ac{ls} function. By doing so, re-meshing and the introduction of intermediate densities are avoided.

A known issue with unfitted techniques is the ill-conditioning problem associated with small cut elements. The common XFEM \cite{Kreissl2012,Villanueva2017} approach uses a \ac{fe} space restricted to the interior domain and cut cells for the solution and requires stabilization in the vicinity of the boundary by, for example, ghost penalty terms \cite{Burman2010} or cell aggregation \cite{badia_aggregated_2017,Badia2022-linking}. These methods are consistent and can provide high-order approximation \cite{Badia2022-high} however the support of the stabilization terms change depending on the location of the cut cells, leading to potential non-differentiability in the optimization problem which can harm the convergence of gradient-based optimization algorithms. The specific unfitted \ac{to} technique used in this work is instead a version of the \ac{fcm} \cite{Parvizian2011}, in which a non-consistent penalty term is added everywhere in the fictitious domain (outside the physical domain) to provide robustness. This stabilization is suitable for \ac{to} because it is differentiable with respect to the level set parameterization (see Section \ref{simulation}). An implementation of the \ac{fcm} for \ac{to} is made in \cite{Parvizian2011}, which uses a refined grid for the material boundary compared to the solution to capture fine-scale geometry. We instead use subgrid triangulations using the \ac{ls} function as in \cite{Kreissl2012} to capture fine-scale structure in the integration and thus avoid the need to increase the number of design variables parameterizing the geometry. The loss of consistency of the \ac{fcm} is not an issue in TO, where high-order approximations are not very relevant. 

It is natural to use a \ac{fe} function for a discrete representation of the \ac{ls}. Doing so, a parameterization is obtained with a user-controlled resolution. A common approach to the optimization problem is to take the  \acp{DOF} of this \ac{ls} as the design variables \cite{Kreissl2011,Kreissl2012,Dijk2012}. This choice, however, means that each parameter is only capable of a local influence on the \ac{ls} function to the surrounding cells. As an alternative, we introduce a
neural parameterization of the geometry. We set our design variables in
this case to be the parameters of a particular artificial \ac{nn} that outputs the \ac{ls} function  \acp{DOF}. Performing this step, we obtain control over the optimization
problem by controlling the connectivity of parameters, allowing them to influence cells spread across the domain. Although the expressivity is unchanged, since the \ac{ls}s are in the same space, the parameters controlling the evolution act to optimize the geometry at multiple scales. The ultimate objective is then for the optimization process to unveil regularized geometries with good performance. 

The combination of machine learning and \ac{to} was explored as early as the 1990s \cite{Adeli1995} but has gained massive momentum in recent years \cite{Zhang2021,Woldseth2022}. \acp{nn} and other ML techniques can be incorporated into the \ac{to} process in many ways.
Common data-driven approaches attempt to train networks to map problem descriptions directly to a geometry \cite{Hoang2022,Yu2018,Li2019,Zheng2021}. These however require pre-training on already optimized samples and suffer from a lack of generalisability \cite{Woldseth2022}. Others replace some or all of the optimization loop for accelerated convergence by training an auxiliary network \cite{Kallioras2020,Joo2021}. These approaches are based on the premise, which in general is not necessarily true, that early iterations of the optimization contain the information to produce performant optimal geometries. An alternative method is the inclusion of a \ac{nn} as an alternative parameterization of the geometry \cite{Deng2020,Chandrasekhar2020,Hoyer2019}. These approaches typically optimize the parameters of a \ac{nn} representing a continuous function that maps positions in space to a density. These approaches tend to focus on reducing the dimensionality of the design space assuming that \acp{nn} can efficiently achieve expressiveness with a small number of parameters \cite{Barron1994}. A reduction in parameters does not, however, necessarily lead to faster convergence for \acp{nn} \cite{Chandrasekhar2020} compared to the standard SIMP approach. 

Instead of focusing on a neural parameterization that reduces the dimensionality of the problem, we select a network description of the geometry which is specifically designed to learn effectively on problems involving the segmentation of a domain. The network used in this case is a modification of the U-Net convolutional network. The U-Net architecture was originally developed for biomedical
image segmentation tasks \cite{ronneberger2015unet} but has proven
successful for a variety of applications in which multi-scale features
and spatial correlation is important \cite{Ulyanov2020}. These networks are typically composed of encoding and decoding halfs. The encoding section maps the context of input images into a low dimensional latent space which is localized in the upsampling section to provide segmentation at the desired resolution. The work in \cite{Hoyer2019} exploited the properties of this network showing
improved performance with a U-Net density parameterization for a SIMP
structural optimization problem. Similar to \cite{Hoyer2019}, we
use a trainable input vector for the network and feed this into the
up-sampling half of the U-Net. In contrast to most applications of this
network, we have no input image and therefore do not need the
encoding of half of the network. It is the up-sampling (or decoding) part of
the network the one that provides the parameterization of the
multi-scale features which are important in this context.  

The Julia \cite{Julia-2017} programming language is used to implement all aspects of this project with the \ac{fe} toolbox Gridap \cite{Badia2020,Verdugo2022} being the main package utilized. We also use the Julia machine learning library Flux \cite{Flux.jl-2018} for the implementation of the U-Net. Using these foundations, we implement a routine combining \acp{nn} with an unfitted \ac{fe} based TO. The main contributions of this work are the presentation of:
\begin{itemize}
	\item An unfitted \ac{ls} \ac{to} method with a \ac{nn} parameterization that assists to achieve simple optimized geometries with similar or better performance compared to baseline methods and 
	\item a fully automatically differentiable unfitted \ac{ls} \ac{to} method for multiphysics problems with complex boundary conditions.
\end{itemize}
We present the overall framework as follows. First, in Section \ref{optimsation-problem}, we present the entire
optimization loop at a high level. We then go into more detail about
various stages in the loop. Details of the architecture of the neural
network are found in Section \ref{neural-architecture}, details of the geometry processing are presented in Section \ref{level-set-function-processing}, the numerical discretization of the problem is presented in Section \ref{simulation} and the gradient implementation in Section \ref{backwards-pass-implementation}. We then benchmark the method against baseline methods and show the generality of the method with an application to a multiphysics problem with complex boundary conditions in Section \ref{numerical-experiments}.



\section{optimization Problem}\label{optimsation-problem}

%\the\linewidth

%\begin{document}
In this section, we provide a succinct overview of the overall \ac{to} algorithm proposed in this work. 
We aim to solve the problem:
	\begin{equation}
		\begin{aligned}
			\underset{\mathbf{p}}{\text{min}} & \ J(\boldsymbol{u}(\mathbf{p}),\mathbf{p})&  \\
			\text{s.t} & \ \mathscr{R}(\boldsymbol{u}(\mathbf{p}),\mathbf{p}) &= 0 ,\\
				   & \ \quad \quad \  \mathscr{V}(\mathbf{p})     &= 0 ,\\
		\end{aligned}
	\label{eq:opt-problem}
	\end{equation}

where 
$\mathbf{p}$ are the parameters that describe our geometry,
  $\mathscr{V}$  is an equality constraint (e.g. for the volume),
  $\mathscr{R}$  is the PDE residual,
  $\boldsymbol{u}$ is the solution of the PDE and
  $J$  is the objective.
If desired, further equality and inequality constraints can then be imposed by adding penalty terms to the objective function. 


\begin{figure}
	\centering
	\incfig[0.55]{FlowChart}
	\caption{Computational graph of the optimization loop. We start with an input to the system $\mathbf{p}$, and perform the forward pass by descending through the blue boxes on the right side to obtain a performance measure $J$ as explained in Section \ref{forward-pass}. %We evaluate the network to obtain $\mathbf{\varphi}$, apply the constraint to obtain $\phi$ and solve the residual equation to obtain $\boldsymbol{u}$.
 	The convergence criteria are used to decide whether this $J$ represents an acceptable minimum. If not, the backward pass is performed to compute an update for the parameters as explained in Section \ref{backwards-pass} and the loop is continued. %\sbcom{\hl{I would change the second box to $\phi = \mathscr{H}(\mathbf{\varphi})$, where $\mathscr{H}$ is an operator that takes the NN and generates a suitable  function (e.g., after re-distancing and volume correction by \emph{traslation}).}} } 
	}
 	\label{fig:FlowChart}
\end{figure}

\subsection{Optimization Loop}\label{optimization-loop}
To optimize the parameters $\mathbf{p}$, we make use of a gradient-based optimization strategy. To do so, we establish a map between $\mathbf{p}$ and $J$ and a means to compute the gradient $\frac{dJ}{d\mathbf{p}}$ for parameter updates.
For the neural \ac{ls} \ac{to} method, $\mathbf{p}$ represents the parameters of a particular \ac{nn} that outputs a vector $\mathbf{\varphi}$. This vector is processed using the operator $\mathscr{H}$ to obtain the \ac{ls} $\phi$ used in the PDE and objective function. %\sbcom{\hl{As I say in the figure, I would add some notation.}}. 
It is also only through $\phi$ that the PDE and objective depend on the parameter vector $\mathbf{p}$. 
With these definitions, we present the optimization loop for solving (\ref{eq:opt-problem}) in Figure \ref{fig:FlowChart}. 


\subsubsection{Forward Pass}\label{forward-pass}

To solve the forward problem and get a performance measure $J$ for a set of parameters $\mathbf{p}$ we descend through the light blue boxes on the right hand side of Figure \ref{fig:FlowChart} by performing the following steps:
 
\begin{enumerate}	
	\item In the first light blue box, we evaluate the network $\boldsymbol{\mathcal{N}}: \mathbf{p} \in \mathbb{R}^{N_p} \mapsto  \mathbf{\varphi} \in \mathbb{R}^{N}$, where $N_p$ is the number of parameters and $\boldsymbol{\mathcal{N}}$ is as defined in Section \ref{neural-architecture}. 
	\item 
	In the second light blue box, we process the output of the network $\mathbf{\varphi}$ using the operator $\mathscr{H}: \mathbf{\varphi} \in \mathbb{R}^{N} \mapsto \phi \in V_h^1$ to obtain a suitable \ac{ls} description of the geometry $\phi $, where $V_h^1$ is the \ac{fe} space for the \ac{ls} defined in Section \ref{level-set-function-processing}. This involves an interpolation on a \ac{fe} space, smoothing and the inclusion of an equality constraint on the geometry so that the final \ac{ls} function satisfies $\mathscr{V}(\phi)=0$. This step is broken down in Section \ref{level-set-function-processing}. We enforce the equality constraint here to allow for the use of an unconstrained optimization method suitable for \ac{nn}s.	
	\item In the third light blue box, we solve the \ac{fe} problem associated with the weak form of the residual $\mathscr{R} (\boldsymbol{u}_h(\phi),\phi)=0$ on the domain segmentation defined by $\phi$ for an approximate solution $\boldsymbol{u}_h$ obtained using a \ac{fe} discretization. Details of the \ac{fcm} used to solve the problem are given in Section \ref{simulation}. We can then evaluate the objective $J(\boldsymbol{u}_h(\phi),\phi) \in \mathbb{R}$.
		
		
\end{enumerate}

\subsubsection{Backwards Pass}\label{backwards-pass}

To perform the update for the parameters using a steepest descent optimization strategy, we require an evaluation of the gradient $\frac{\partial J}{\partial \mathbf{p}}$. 
To do this efficiently at a cost roughly matching that of the forward pass, we use reverse mode differentiation and define rules to propagate sensitivities through each of the steps in the forward pass. We use an adjoint rule for the PDE and use automatic differentiation for all of the partial derivatives, including the derivative of integrals with respect to the \ac{ls}.
The derivative is then passed onto a chosen optimizer to update the parameters $\mathbf{p}$. The implementation of the gradient computation is discussed in Section \ref{backwards-pass-implementation}.


\begin{figure}
   \vspace{-0.5cm}
   \center
   \includegraphics[width=0.48\textwidth]{misc/figures/Figure_revised_activation_layer.pdf}
   \vspace{-0.3cm}
   \caption{
   Our CIM installs an ``extension'' on the backbone (e.g., ResNet-18~\cite{he2016deep,rebuffi2017icarl} with four blocks) by adding a learnable activation function (e.g., PAU~\cite{molina2019pade}) at the position of the original activation function (i.e., ReLU~\cite{nair2010rectified}). Such ``extension'' results in a new network branch (in \textcolor[rgb]{0.576, 0.769, 0.490}{green}), whose weight layer parameters are directly copied from the original branch (in \textcolor[rgb]{0.427, 0.620, 0.922}{blue}).
   }
   \label{figure: revised_activation_layer}
   \vspace{-0.4cm}
\end{figure}   


\section{\ac{ls} Function Processing}\label{level-set-function-processing}

In this section, we detail the \ac{ls} description of the geometry used in the method. We explain how the output vector from the \ac{nn} $\mathbf{\varphi}$ is used to define the \ac{ls} to be used in the numerical method. The computation of the \ac{ls} function is performed in four steps. We introduce an interpolation step in Section \ref{interpolation} to obtain a first \ac{ls} function $\phi_{n(1)}$. In Section \ref{smoothing}, we smooth out this \ac{ls} to obtain a smooth \ac{ls} function $\phi_{f(2)}$. Then, we perform a reinitialization of that \ac{ls} in Section \ref{reinitialization} to obtain an \ac{ls} function $\phi_{s(3)}$. Finally, in Section \ref{translation}, we propose a volume correction strategy to end up with the final \ac{ls} function $\phi_{b(4)}$. In the subsequent sections, we will refer to this final \ac{ls} function $\phi_{b(4)}$ as $\phi$.


\subsection{Interpolation}\label{interpolation}

 In this method, we work with discrete \ac{ls} functions $\phi \in V_h^1$. The  \ac{DOF} of $V_h^1$ are the values of the function at the vertices of $\mathcal{T}_h$, which we denote with $\{\boldsymbol{x}_i\}_{i=1}^N$, where $N$ is the number of mesh nodes. Thus, there is an isomorphism between $V_h^1$ and $\mathbb{R}^N$. The output image vector of the \ac{nn} $\varphi$ are the \ac{DOF} values that uniquely determine the \ac{ls} \ac{fe} function.

\subsection{Smoothing}\label{smoothing}
Next, we convolve the function with a linear filter for smoothing:
\begin{equation}
  {\phi_{f(2)}}_i = (\sum^N_{j=1} w_{ij} )^{-1} ( \sum^N_{j=1} w_{ij} {\phi_{n(1)}}_j),
\end{equation}
where $w_{ij}=\max(0,r_f-\vert x_i-x_j \vert )$,  ${\phi_{n(1)}}_j$ is the $j^{th}$ degree of freedom of the function $\phi_{n(1)}$, ${\phi_{f(2)}}_i$ is the $i^{th}$ degree of freedom of the function $\phi_{f(2)}$ and $r_f$ is the smoothing radius.

\subsection{Reinitialization}\label{reinitialization}
We then reinitialize the \ac{ls} as a signed distance function. This is often done to gain control over the spatial gradient of the \ac{ls} function to improve convergence \cite{vanDijk2013}. In our case, it is of even greater importance as it also guarantees that when we apply the translation to the \ac{ls} to satisfy the volume constraint, we do not artificially introduce volumes into the domain far from the boundary where the \ac{ls} is close to zero. This would add discontinuity to the problem harming convergence. To perform this step, we solve the reinitialization equation in \cite{Xing2009} to obtain a signed distance function $\phi_{s(3)}\in V_h^1$:
\begin{equation}
  \frac{ \partial \phi_{s(3)} }{\partial \tau} + \text{sign}(\phi_{s(3)})(\vert \boldsymbol{\nabla}\phi_{s(3)} \vert - 1 ) = 0.
\end{equation}
The problem is solved using Picard iterations at steady state using the initial point $\phi_{f(2)}$ so that only one solve of the adjoint equation is required in the backward pass. Artificial viscosity is added to the problem for stabilization and a surface penalty term is integrated on the embedded boundary to prevent movement of the zero iso-surface in the reinitialization. The weak form of the problem is then to find the solution $\phi_{s(3)}\in V_h^1$ of the equation:
\begin{equation}
  \int_\Omega ( \boldsymbol{w}\cdot\boldsymbol{\nabla}(\phi_{s(3)})  v  + c_a h \vert \boldsymbol{w} \vert \boldsymbol{\nabla}(\phi_{s(3)}) \boldsymbol{\nabla} v  - \text{sign}(\phi_{s(3)}) v  ) d\Omega + \int_\Gamma (\phi_{s(3)}, v )d\Gamma = 0
\label{eq:Rs}
\end{equation}
for all $ v \in V_h^1$, where $\boldsymbol{w} = \text{sign}(\phi_{s(3)})(\boldsymbol{\nabla}(\phi_{s(3)})/ \vert\boldsymbol{\nabla}(\phi_{s(3)}) \vert )$, $c_a$ is the stabilization coefficient, set to $3$, and $h$ is the element size. 

\subsection{Translation}\label{translation}
The next task is to impose the volume constraint. This is done by applying a translation to the entire \ac{ls} function. Here we solve the nonlinear equation $\mathscr{V}$  for the scalar bias  $b$ to obtain a translation to the \ac{ls} function such that the domain $\Omega_{\rm in}$ satisfies the volume constraint. A bisection method is used to find the root $b$ of the equation:
\begin{equation}
  \mathscr{V} =  \int_{\Omega_{\rm in}} d\Omega - \mathscr{V}_0 = 0,
\end{equation}
where $\Omega_{\rm in}$ is defined by the \ac{ls} $\phi_{b(4)} = \phi_{s(3)} + b$ and $\mathscr{V}_0$ is the volume fraction given by the constraint. This is the \ac{ls} which is then used to define the boundary of the \ac{fe} problem. Details of how the \ac{ls} is used to define domains is given in the next Section.




\section{Numerical Discretization}\label{simulation}


\begin{figure}
  \centering
  \incfig[0.55]{DomainDefinition}
  \caption{Illustration of the problem domains. The background domain is segmented into the domains $\Omega_{\rm in}$ and $\Omega_{\rm out}$. In $\Omega_{\rm in}$, the physical terms $\mathcal{R}(\boldsymbol{u},\boldsymbol{v})$ are integrated. In $\Omega_{\rm out}$, the stabilization terms $ \mathcal{R}^{\mathrm{stb}}(\boldsymbol{u},\boldsymbol{v})$ are integrated. }
  \label{fig:domain-definition}
\end{figure}

Here we formulate the numerical discretization of the PDEs being solved in the simulation step of the optimization loop. We consider a background polyhedral bounded domain $\Omega$ with boundary $\partial \Omega$. A \ac{ls} function $\phi(x)$ is used to split $\Omega$ into two subdomains $\Omega_{\rm in}(\phi)$ and $\Omega_{\rm out}(\phi)$ as follows (see Figure \ref{fig:domain-definition}):
\begin{equation}
  \begin{aligned}
     & \Omega_{\rm in}(\phi) = \{ x \in \Omega :  \phi(x) > 0 \},
    \quad \Omega_{\rm out}(\phi) = \{ x \in \Omega :  \phi(x) < 0 \}.
  \end{aligned}
  \label{eq:Omega_def}
\end{equation}
We denote the interface between these two subdomains as $\Gamma(\phi) \doteq \partial \Omega_{\mathrm{in}}(\phi) \cap \partial \Omega_{\mathrm{out}}(\phi)$.
Let us assume that $\partial \Omega_{\mathrm{in}}(\phi) \cap \partial \Omega \neq \emptyset$. The domain $\Omega_{\mathrm{in}}(\phi)$ is the one in which we consider our PDE problem. The weak form of the continuous problem can be stated as follows: find $\boldsymbol{u} \in V$ such that
\begin{equation}
  \int_{\Omega_{\mathrm{in}}(\phi)} \mathcal{R}(\boldsymbol{u},\boldsymbol{v}) \mathrm{d} \Omega = 0, \quad \forall v \in V.
  \label{eq:R}
\end{equation}
where $V$ is a Hilbert space in which the problem is well-posed. We consider zero flux Neumann boundary conditions on $\partial \Omega_{\mathrm{in}}(\phi) \setminus \partial \Omega$ and (for simplicity)  homogeneous boundary conditions on  $\partial\Omega_{\mathrm{in}}(\phi) \cap \partial \Omega$; the generalization to non-homogeneous Dirichlet boundary conditions is straightforward.

The domain $\Omega_{\mathrm{in}}(\phi)$ will change along the optimization process. As a result, it is not practical to compute body-fitted unstructured meshes for the geometrical discretization of $\Omega_{\mathrm{in}}(\phi)$. Instead, we consider a background mesh, which can simply be a Cartesian background mesh of $\Omega$  and make use of an unfitted \ac{fe} discretization. Unfitted (or embedded) discretizations relax the geometrical constraints but pose additional challenges to the numerical discretization \cite{Burman2010,Badia2018}. The first issue is the integration over cut cells (for details, see \cite{Badia2022-Geometrical}). The other issue is the so-called \emph{small cut cell} problem. Cut cells with arbitrary small support lead to ill-conditioned systems \cite{dePrenter2017}. Various techniques can be used to stabilize the problem including the \ac{cutfem} \cite{Burman2010}, the \ac{agfem} \cite{Badia2018} and the 
\ac{fcm} \cite{Parvizian2007}. 
We select the \ac{fcm} method in this case because it is differentiable with respect to the \ac{ls} 
everywhere in the domain. The \ac{cutfem} and \ac{agfem} are conversely not differentiable. %(see Proposition \ref{thm:CutFEM-nondiff}). %\sbcom{\hl{acronyms, I would use a package to be sure that we define it the first time we use them}}. The idea of \ac{fcm} method is to add a penalty term in the artificial domain $\Omega_{\mathrm{out}}$ that makes the discrete linear system solvable. Even though this method is not consistent and intrinsically low order, it is suitable for \ac{to} applications, which are intrinsically low-order.

In order to state the discrete form of the continuous problem, we introduce the unfitted \ac{fe} space. Let $\mathcal{T}_h$ represent a conforming, quasi-uniform and shape regular partition (mesh) of $\Omega$, $h$ being a characteristic mesh size. $\Omega$ can be a trivial geometry, e.g., a square or cube, and $\mathcal{T}_h$ can be a Cartesian mesh. We define a nodal Lagrangian \ac{fe} space of order $q \geq 1$ on $\mathcal{T}_h$ as:
\begin{equation}
  V_{h}^q = \{ \boldsymbol{v}_h \in \mathcal{C}^0(\Omega) : \boldsymbol{v}_h\vert_{K} \in \mathcal{X}_q(K) \ \forall K \in \mathcal{T}_h \},
\end{equation}
where $\mathcal{X}_q(K)$ is the space $\mathcal{Q}_q(K)$ of  polynomials with maximum degree $q$ for each variable when $\mathcal{T}_h$ is a quadrilateral or hexahedral mesh and the space $\mathcal{P}_q(K)$ of polynomials of total degree $q$ when $\mathcal{T}_h$ is a simplicial mesh. In this work, we consider low order spaces, which is the most reasonable choice for \ac{to} applications.

The weak formulation of the problem solved by the method is now described. Let us represent with $V_{h,0}^q = V_h^q \cap \mathcal{C}_{0}^{0}(\bar{\Omega})$ the nodal \ac{fe} space that vanishes on the boundary $\partial \Omega$. 

Now, we can define a first-order \ac{fcm} discretization of (\ref{eq:R}) as follows: find $\boldsymbol{u}_h \in V_h^1$ such that
\begin{equation}
  \int_{\Omega_{\mathrm{in}}(\phi)} \mathcal{R}(\boldsymbol{u}_h,\boldsymbol{v}_h) \mathrm{d} \Omega +
  \int_{\Omega \setminus \Omega_{\mathrm{in}}(\phi)} \alpha_{\mathrm{out}} \mathcal{R}^{\mathrm{stb}}(\boldsymbol{u}_h,\boldsymbol{v}_h) \mathrm{d} \Omega = 0, \quad \forall v \in V_h^1,
\label{eq:RFCM}
\end{equation}
where $\alpha_{\mathrm{out}} \ll 1$ is the penalty parameter and $\mathcal{R}^{\mathrm{stb}}$ is a \emph{stabilizing}  differential operator on the artificial domain.



\newcommand{\ubs}{\boldsymbol{u}}

\subsection{Poisson Equation}\label{poisson-equations}


The poisson equation is used to model the temperature $\theta(\boldsymbol{x})$ 
that satisfies
\begin{equation}\protect\hypertarget{eq:T}{}{
    -\boldsymbol{\nabla}\cdot (\boldsymbol{\kappa} \boldsymbol{\nabla} \theta) = f \quad \text{in}\ \Omega_{\mathrm{in}}(\phi),
  }\label{eq:T}\end{equation}
where $\boldsymbol{\kappa}$  is the thermal conductivity tensor of the material and $f$ is a thermal source.
A zero Dirichlet condition ($\theta=0$) is prescribed on $\Omega_{\mathrm{in}}(\phi) \cap \partial \Omega$ and a zero flux condition ($\boldsymbol{n}\cdot(\boldsymbol{\kappa}\boldsymbol{\nabla}\theta) = 0$) is prescribed on $\Gamma(\phi)$. 

The \ac{fcm} approximation of this problem without a source term reads as: find $\theta_h \in V_{h,0}^{1}$ such that  
  \begin{equation}
  \int_{\Omega_{\mathrm{in}}(\phi)} \boldsymbol{\kappa} \boldsymbol{\nabla}\theta_h \cdot \boldsymbol{\nabla}v_h  \mathrm{d} \Omega +
  \int_{\Omega \setminus \Omega_{\mathrm{in}}(\phi)} \alpha_{\mathrm{out}} \boldsymbol{\kappa} \boldsymbol{\nabla}\theta_h \cdot \boldsymbol{\nabla}v_h \mathrm{d} \Omega = 0, \quad \forall v_h \in V_{h,0}^1.
  \label{eq:RT}
\end{equation}
One can readily check that this method is weakly enforcing the zero flux condition on $\Gamma(\phi)$ as $\alpha_{\mathrm{out}} \rightarrow 0$.  We observe that we use the same differential operator in the artificial domain for stabilisation purposes (times the scaling coefficient $\alpha_{\mathrm{out}}$).

We consider the \ac{to} problem in which we aim at finding a level-set $\phi$ that minimizes the integral of the temperature: %\sbcom{\hl{name?}}
\begin{equation}
  J(\phi,\theta_h(\phi))  = \int_{\Omega(\phi)} \theta_h(\phi) \  \mathrm{d}\Omega,
\end{equation}
where $\theta_h(\phi)$ is the solution of (\ref{eq:RT}) given $\phi$. 






\hypertarget{linear-elasticity}{%
  \subsection{Linear elasticity}\label{linear-elasticity}}

We want to obtain the displacement $\boldsymbol{d}(\boldsymbol{x})$ that satisfies the linear elasticity equation  
\begin{equation}\protect\hypertarget{eq:d}{}{
    \left.
    \begin{aligned}
      -\boldsymbol{\nabla}	\cdot \bm{\sigma} (\boldsymbol{d})  & = \boldsymbol{f} &  & \text{in} \   \Omega_{\rm in}(\phi),
    \end{aligned}
    \right.
  }\label{eq:d}\end{equation} where $ \bm{\sigma} = \lambda \text{tr}(\bm{\varepsilon})I+2\mu \bm{\varepsilon} $ is the stress tensor, $\bm{\varepsilon} = \frac{1}{2}(\boldsymbol{\nabla}\boldsymbol{d} + (\boldsymbol{\nabla}{\boldsymbol{d}})^\top)$ is the symmetric gradient, $ \(\boldsymbol{d}\)$ is the displacement, $\lambda$ and $\mu$ are the Lam\'e parameters given by $\lambda = (E\nu)/((1+\nu)(1-2\nu)) $ and $ \mu=E/(2(1+\nu)) $
and $\boldsymbol{f}$ is the forcing term. A zero Dirichlet condition ($\boldsymbol{d}=\boldsymbol{0}$) is prescribed on $\Omega_{\mathrm{in}}(\phi) \cap \partial \Omega$ and a zero stress condition ($\boldsymbol{n}\cdot \boldsymbol{\sigma}(\boldsymbol{d}) = 0$) is prescribed on $\Gamma(\phi)$.

The \ac{fcm} approximation of this problem without a forcing term reads as: find $\boldsymbol{d}_h \in \boldsymbol{V}_{h,0}^{1} \doteq  [V_{h,0}^1 ]^D$ such that  
  \begin{equation}
  \int_{\Omega_{\mathrm{in}}(\phi)} \boldsymbol{\sigma}(\boldsymbol{d}_h) : \boldsymbol{\varepsilon}(\boldsymbol{v}_h)  \mathrm{d} \Omega +
  \int_{\Omega \setminus \Omega_{\mathrm{in}}(\phi)} \alpha_{\mathrm{out}} \boldsymbol{\sigma}(\boldsymbol{d}_h) : \boldsymbol{\varepsilon}(\boldsymbol{v}_h)  \mathrm{d} \Omega = 0, \quad \forall \boldsymbol{v}_h \in \boldsymbol{V}_{h,0}^1.
  \label{eq:Rd}
\end{equation}
It is easy to check that the zero-stress condition on $\Gamma(\phi)$ is recovered as $\alpha_{\mathrm{out}} \rightarrow 0$. For the linear elasticity equation, we again use the same differential operator in the artificial domain for stabilisation purposes.


A typical \ac{to} problem in solid mechanics is the minimization of the strain energy. In this case, we aim at finding a level-set $\phi$ that minimizes the cost function 
\begin{equation}
  J(\phi,\boldsymbol{d}_h(\phi))  = \int_{\Omega(\phi)} \bm{\sigma}(\boldsymbol{d}_h):\bm{\varepsilon}(\boldsymbol{d}_h) \  \mathrm{d}\Omega,
  \label{eq:Jd}
\end{equation}
where $\boldsymbol{d}_h(\phi)$ is the solution of (\ref{eq:Rd}) given $\phi$. 



\subsection{Linear Elasticity with Fluid Forcing Terms}\label{FSI}

Once again, we want to obtain the displacement $\boldsymbol{d}_h(\boldsymbol{x})$ that satisfies the linear elasticity formulation in (\ref{eq:d}). In this case, however, we consider the surface traction exerted by the fluid:
\begin{equation}\protect\hypertarget{eq:rdGammas}{}{
  \begin{aligned}
    \int_{\Gamma(\phi)}
      ( \boldsymbol{n} \cdot \boldsymbol{\nabla}\boldsymbol{u}_h  - {p}_h \ \boldsymbol{n} ) \cdot \boldsymbol{v}
    {\rm d}x,
  \end{aligned}
}\label{eq:rdGammas}\end{equation}
where the fluid velocity $\boldsymbol{u}_h$ and pressure field ${p}_h$ are obtained by solving a fluid problem in the domain $\Omega_{\rm out}$. 
These fields are obtained by solving the Stokes equations with a Brinkmann penalization as in \cite{Borrvall2002} but without intermediate interpolation of permeabilities at the boundary.

In order to approximate the fluid problem, we use a mixed \ac{fe} method, namely the equal order pair $\boldsymbol{V}_{h,0}^{1} \times {V}_h^1$.


We find \((\boldsymbol{u}_h,{p}_h)\in \boldsymbol{V}_{h,0}^1 \times V_{h,0}^1 \)
such that:
\begin{equation}\protect\hypertarget{eq:rupOmega}{}{
  \begin{aligned}
    %\mathscr{R}_{\boldsymbol{u},p}^\Omega \doteq
    \int_{\Omega}%\cup\Omega_s\cup\Omega_f}
    &\left[
      \alpha \boldsymbol{u}_h \cdot \boldsymbol{\psi}_h +
      \mu \boldsymbol{\nabla}\boldsymbol{u}_h \cdot\boldsymbol{\nabla}\boldsymbol{\psi}_h 
      -{p}_h (\boldsymbol{\nabla}\cdot \boldsymbol{\psi}_h)
      -(\boldsymbol{\nabla}\cdot \boldsymbol{u}_h) {q}_h
      - h^2 \boldsymbol{\nabla}p_h \cdot \boldsymbol{\nabla}{q_h}
      \right] 
    {\rm d}x
    =0, \quad 
    \\
    &\forall \boldsymbol{\psi}_h,{q}_h \in \boldsymbol{V}_{h,0}^{1} \times {V}_h^1,  %, where $\mathscr{R}_{\boldsymbol{u},p}^\Omega$ is given by:
  \end{aligned}
}\label{eq:rupOmega}\end{equation}
where
\begin{equation}
  \left\lbrace
  \begin{aligned}
    \alpha & = 0        &  & \text{in } \Omega_{\rm out} \\%\cup \Omega_f,  \\
    \alpha & = \alpha_u &  & \text{in } \Omega_{\rm in} \\%\cup \Omega_s. \\
  \end{aligned}
  \right.
\end{equation}
using an artificial porosity $\alpha_u$ to make the fluid problem well posed in the solid domain $\Omega_{\rm in}$ and enforce the no-slip boundary condition. In the fluid domain $\Omega_{\rm out}$ we recover the Stokes equations.

The \ac{to} problem once again involves finding a level-set $\phi$ that minimizes the elastic strain using (\ref{eq:Jd}). 

\subsection{Differentiability of the unfitted \ac{fe} solver}\label{sec:}

An important property for the convergence of a \ac{to} strategy is the notion of shape differentiability of the cost function. 
A functional under a PDE constraint is considered shape differentiable if the mapping $\phi \rightarrow J(\boldsymbol{u_h}(\phi),\phi)$ 
is differentiable at the admissable set of domains in $\Omega$ defined by $\phi$.
In this section, we discuss how the choice of \ac{fe} stabilization can effect this property. 

The model problem in \eqref{eq:RT} with a \ac{fcm} stabilization is equivalent to that of a typical two phase conductivity problem. 
With a solution $\boldsymbol{u}_h\in H^1(\Omega)$, the functional $J(\boldsymbol{u_h},\phi)$ for this problem can be proven to be shape differentiable, see \cite[Theorem 4.9]{Allaire2021}. 

Conversely, unfitted techniques involving stabilization only in the vicinity of the boundary are in general not shape differentiable. 
Regions of non differentiability arise (typically when the boundary crosses over mesh nodes) harming the convergence of the geometry to optimized solutions \cite{Sharma2016}. To see why this is the case, we investigate shape pertubations under the \ac{cutfem} and \ac{agfem} formulations.



\begin{figure}
  \centering
  \incfig[0.55]{Pertubation}
  \caption{A small perturbation with size $\epsilon$ to the \ac{ls} to form a new domain $\Omega_{\rm in}^\epsilon$. }
  \label{fig:pertubation}
\end{figure}



If we were to use a restricted space for the solution and add ghost penalty terms in the vicinity of the interface following the CutFEM method \cite{CutFEM2015}, the problem is to find $u_h\in W_{h,0}^1.$ such that:
\begin{equation}
  \begin{aligned}
    \int_{\Omega_{\mathrm{in}}(\phi)} \mathcal{R}(\boldsymbol{u}_h,\boldsymbol{v}_h) \mathrm{d} \Omega_{\Omega_{\rm in}} +
    j(\boldsymbol{u}_h,\boldsymbol{v}_h)_{\Gamma_G} = 0, \quad \forall v_h \in W_{h,0}^1,
  \end{aligned}
\end{equation}
for a ghost penalty term $j$ on a ghost skeleton triangulation $\Gamma_G$ using the space $W_{h,0}^1$ as in \cite{CutFEM2015}. 
Consider the change to the domain $\Omega_{\rm in}$ from Figure \ref{fig:domain-definition} caused by a perturbation $\epsilon \delta$, where $\epsilon \in \mathbb{R}$ and $\delta\in V^1_h$, to the \ac{ls} function $\phi$. The resulting domain $\Omega_{\rm in}^\epsilon$ may be as in Figure \ref{fig:pertubation}. The ghost penalty term in this formualtion does not depend on $\epsilon$ and instead changes depending on which cells are cut. Specifically, if $\Gamma^\epsilon$ crosses over a mesh node as in Figure \ref{fig:ghost-skeleton}, the ghost skeleton triangulation includes the faces of a new element. Non-zero terms are integrated on this triangulation introducing discontinuity to the problem with respect to the shape, since:
\begin{equation}
  \lim_{\epsilon \to 0} \  [ j(\boldsymbol{u}_h,\boldsymbol{v}_h)_{\Gamma_G^\epsilon} - j(\boldsymbol{u}_h,\boldsymbol{v}_h)_{\Gamma_G} ] \  \neq 0.
\end{equation}
With different terms added to the linear system which do not go to zero with $\epsilon$, we can see that the derivative of the solutions with respect to the shape can be ill-defined. A cost function operating on the solution could not, in general, be shape differentiable at these points.

\begin{figure}
  \centering
  \incfig[0.55]{GhostSkeleton}
  \caption{The change in a portion of the ghost skeleton triangulation $\Gamma_G$, depicted by the red faces, before and after a perturbation to the boundary.}
  \label{fig:ghost-skeleton}
\end{figure}

In the \ac{agfem}, the problem is to find $u_h\in V^{agg}_{h,0}$ such that:
\begin{equation}
  \begin{aligned}
	  \mathscr{R}(\mathcal{E}(\boldsymbol{u}_h),\mathcal{E}(\boldsymbol{v}_h))_{\Omega_{\rm in}} \quad \forall v_h \in V^{agg}_{h,0},
  \end{aligned}
\end{equation}
for the extension operator $\mathcal{E}$ and space $V^{agg}_{h,0}$ as defined in \cite{Badia2018}. Similar to the CutFEM, regions of non-differentiability in the problem exist when the zero isosurface of the \ac{ls} crosses over mesh nodes:
\begin{equation}
  \begin{aligned}
    \lim_{\epsilon \to 0} \ 
    [
      \mathscr{R}(\mathcal{E}(\boldsymbol{u}_h),\mathcal{E}(\boldsymbol{v}_h))_{\Omega_{\rm in}} 
    -
    \mathscr{R}(\mathcal{E}^\epsilon(\boldsymbol{u}_h),\mathcal{E}(\boldsymbol{v}_h))_{\Omega_{\rm in}^\epsilon} 
    ]
    \neq 0,
  \end{aligned}
\end{equation}
since $\mathcal{E}^\epsilon(\boldsymbol{u}) \neq \mathcal{E}(\boldsymbol{u}_h)$ in general because the support for cut cells potentially changes depending on which cells are cut. Following the same reasoning as above, the method is therefore not shape differentiable. The same is true for other methods which use stabilization approaches that act only in the vicinity of cut cells, e.g. \cite{Lang2014}, using similar branching strategies when crossing a node or reaching a certain threshold.
\qed
%\end{proof}




 



\section{Gradient Implementation}\label{backwards-pass-implementation}

To the best of our knowledge, there is no existing implementation of an unfitted \ac{ls} \ac{to} method that accepts arbitrary residuals defining the PDE and computes the entire gradient $\frac{dJ}{d\mathbf{p}}$ by automatic differentiation. Making use of a backwards pass, we do this efficiently by defining differentiation rules for each of the steps in the method. 

\subsection{Integral Differentiation Operator}
The backward pass is mainly composed of gradients of integrals with respect to the  \acp{DOF} of \ac{fe} functions. To make the derivative computation efficient, we exploit the fact that the  \acp{DOF} only have an effect on surrounding cells and utilize the optimizations exploiting sparsity in the \ac{fe} library Gridap \cite{Verdugo2021}. Integrals in the domain can be divided into cell-wise components:
\begin{equation}
\mathscr{I}(\boldsymbol{u},\boldsymbol{v},\phi) = \sum_{K\in\mathcal{T}_h} \mathscr{I}^K(\mathbf{u}^K,\mathbf{v}^K,\mathbf{\phi}^K), 
\end{equation}
where $\mathbf{u}^K \in \mathbb{R}^{\Sigma_u},\mathbf{v}^K \in \mathbb{R}^{\Sigma_v}$ and $\phi^K\in\mathbb{R}^{\Sigma_\phi}$ are the \acp{DOF} parameterising the restrictions of $u,v$ and $\phi$ to the cell $K$ and $\Sigma u$, $\Sigma v$ and $\Sigma \phi$ are the number of \acp{DOF} in $K$ for the respective functions. 
Gradients can then be computed at roughly the cost of an integral evaluation for each cell $K$:
\begin{equation}
	\frac{\partial \mathscr{I} }{ \partial \phi }^K = \nabla^F_{\phi} \mathscr{I}^K ( \mathbf{u}^K,\mathbf{v}^K, \phi^K ),
\end{equation}
where the operator $\nabla^F_{\phi}$ represents taking the gradient with respect to $\phi^K$ using a vectorized forward propogation of dual numbers \cite{ForwardDiff}. 
To make taking derivatives in this way possible for the \ac{ls}, we implement the integrals so that each $\phi^K$ is accepted as the argument to compute the contribution $\mathscr{I}^K$:
\begin{equation}
	\mathscr{I}^K: \phi^K \in \mathbb{R}^{\Sigma_\phi} \mapsto \mathscr{I}_K(\mathbf{u}^K,\mathbf{v}^K,\phi^K) \in \mathbb{R}.
\end{equation}
where
\begin{equation}
	\mathscr{I}^K(\mathbf{u}^K,\mathbf{v}^K,\phi^K)=\int_{K(\phi^K)} \mathcal{I}(\mathbf{u}^K,\mathbf{v}^K)dK
\end{equation}

A key point is that the integral function subroutines, including all the unfitted \ac{fe} tools, are implemented in such a way as to allow the propagation of dual numbers through the code. We also make use of a reverse mode operator $\nabla^R$ for the backwards propogation of derivatives used where appropriate, e.g., for the \ac{nn}.

\subsection{Backwards Pass Routine}
We now present the backwards pass in detail. 
To compute the sensitivity of the objective with respect to the parameters, we start with the seed $\frac{dJ}{dJ}=1$ and propogate derivatives in reverse mode using the chain rule:
\begin{equation}
	\frac{dJ}{d\mathbf{p}} = 	
	\frac{dJ}{dJ} \left(
	\frac{\partial{J}}{\partial{\phi }} +
	\frac{\partial{J}}{\partial{\boldsymbol{u}}} 
\frac{d\boldsymbol{u}}{d\phi } \right)
	\frac{d\phi }{d\mathbf{\varphi}} 
	\frac{d\mathbf{\varphi}}{d{\mathbf{p}}} 
\end{equation}
where an adjoint method on the problem residual $\mathscr{R}$ is used to differentiate through the PDE:
\begin{equation}
	\frac{\partial{J}}{\partial{\boldsymbol{u}}} \frac{d\boldsymbol{u}}{d\phi }  =  {-\lambda^T\frac{d\mathscr{R}}{d\phi }} \ (\text{ here we solved  } \frac{d\mathscr{R}}{d\boldsymbol{u}}^{T} \lambda =  \frac{dJ}{d\boldsymbol{u}}^{T}). 
\end{equation}
We then use the chain rule to differentiate through the \ac{ls} function processing steps:
\begin{equation}
	\frac{d\phi }{d\mathbf{\varphi}} = 
	\frac{d\phi }{d{\phi_{s(3)}}}  
	\frac{d{\phi_{s(3)}}}{d{\phi_{f(2)}}}  
	\frac{d{\phi_{f(2)}}}{d\mathbf{\varphi} }.
\end{equation}
The volume constraint here involved a root finding method. To differentiate through this step, we utilize the implicit function theorem:
\begin{equation}
	\frac{d\phi }{d{\phi_{s(3)}}} =  \frac{\partial \phi } { \partial {\phi_{s(3)}}}  - \frac{\partial \phi  }{ \partial b} \frac{\partial \mathscr{V}}{\partial b}^{-1} \frac{\partial\mathscr{V }}{\partial {\phi_{s(3)}}},
\end{equation}
and to differentiate through the signed distance map, we use the adjoint method once again for the residual $\mathscr{R}_s$ equal to the integral in (\ref{eq:Rs}). 
Finally, we use standard backpropagation to compute the derivative with respect to the parameters of the \ac{nn}. The steps of the backward pass are presented explicitly in Algorithm \ref{al:bp}.



\begin{algorithm}
	\caption{Backwards Pass}\label{alg:cap}
	\begin{algorithmic}
	\State Initialize $\frac{dJ}{dJ} \gets 1$ 
	\State Extract $\phi^K \in \mathbb{R}^{\Sigma \phi}$, $\mathbf{u}^K \in \mathbb{R}^{\Sigma u}$ from $\phi$,$\boldsymbol{u}$ $\forall{K} \in \mathcal{T}_h$.
	\For{ $ K \in \mathcal{T}_h$ }
	\State $\frac{\partial J }{ \partial \boldsymbol{u} }^K \gets \nabla^F_{u} J ( \mathbf{u}^K, \phi^K )$ 
		\State $\frac{\partial J }{ \partial \phi }^K \gets \nabla^F_{\phi} J ( \mathbf{u}^K, \phi^K )$
	\EndFor
	\State Assemble the gradients $ \frac{\partial J }{ \partial \boldsymbol{u} } \in \mathbb{R}^{N_u}$ and $ \frac{\partial J }{ \partial \phi } \in \mathbb{R}^{N}$

	\State Assemble the sparse jacobian associated with the residual $ \frac{\partial \mathscr{R} }{ \partial \boldsymbol{u} } \in \mathbb{R}^{N_u,N_u}$
	\State Solve the adjoint equation $\frac{\partial \mathscr{R} }{ \partial \boldsymbol{u} } \lambda = \frac{\partial J }{ \partial \boldsymbol{u} }$ for $\lambda\in\mathbb{R}^{N_u}$
	\State Extract $\lambda^K \in \mathbb{R}^{\Sigma u}$  from $\lambda$ $\forall  K \in \mathcal{T}_h$ 
	\For{ $ K \in \mathcal{T}_h$ }	
	\State $\frac{\partial J }{ \partial \boldsymbol{u} }\frac{\partial \boldsymbol{u}}{\partial \phi}^K \gets \nabla^F_{u} \mathscr{R} ( \mathbf{u}^K, \lambda^K, \phi^K )$ 
	\EndFor
	\State Assemble the gradient $ \frac{\partial J }{ \partial \boldsymbol{u} }\frac{\partial \boldsymbol{u}}{\partial \phi} \in \mathbb{R}^N $ 
\State $\frac{dJ}{d\phi} \gets \frac{\partial J }{ \partial \phi } +  \frac{\partial J }{ \partial \boldsymbol{u} }\frac{\partial \boldsymbol{u}}{\partial \phi} $
	\State Compute the vector-jacobian-products: 
		\State $ \frac{\partial J }{ \partial b } \gets \frac{d J }{ d \phi } \nabla^R_b \phi  (\phi_{s(3)},b)  $ 
		\State $ \frac{\partial J }{ \partial \phi } \gets \frac{d J }{ d \phi } \nabla^R_{\phi} \phi  (\phi_{s(3)},b) $ 
	%\For{$ K \in \mathcal{T}_h$ }
	\State Compute the gradients:
		\State $\frac{\partial V }{ \partial \phi } \gets \nabla^R_{u} V ( \phi_{s(3)},b )$ 
		\State $\frac{\partial V }{ \partial b } \gets \nabla^F_{u} V ( \phi_{s(3)},b )$ 
	\State $\frac{dJ }{d{\phi_{s(3)}}} \gets \frac{\partial J } { \partial {\phi_{s(3)}}}  - \frac{\partial J  }{ \partial b} \frac{\partial \mathscr{V}}{\partial b}^{-1} \frac{\partial\mathscr{V }}{\partial {\phi_{s(3)}}}$
	
	\State Assemble the sparse jacobian associated with the residual $ \frac{\partial \mathscr{R}_s }{ \partial \phi_{s(3)} } \in \mathbb{R}^{N,N}$
	\State Solve the adjoint equation $\frac{\partial \mathscr{R}_s }{ \partial \phi_{s(3)} } \lambda_s = \frac{\partial J }{ \partial \phi_{s(3)} }$ for $\lambda_s\in\mathbb{R}^N$
	\For{ $ K \in \mathcal{T}_h$  }\\
		$\frac{d \mathscr{R}_s }{ d \phi_{f (2)} }^K \gets \nabla^F_{\phi_{f (2)} } \mathscr{R}_s ( \phi_{s(3)}^K, \lambda_s^K, \phi_{ f(2)}^K )$ 
	\EndFor
	\State Assemble the gradient $ \frac{d J }{ d \phi_{f(2)} }  \in \mathbb{R}^N$ 

	\State Compute the vector-jacobian-products: 
	\State $ \frac{d J }{d \varphi } \gets \frac{d J }{ d \phi_{f(2)} } \nabla^R_\varphi (\phi_{f(2)}(\varphi)) $ 
	
	\State $ \frac{d J }{d p } \gets \frac{d J }{ d \varphi } \nabla^R_p (N(\mathbf{p})) $ 
	\end{algorithmic}
	\label{al:bp}
\end{algorithm}




\section{Results}
\label{results}

\begin{figure*}[ht]
    \centering
    \includegraphics[scale=0.15,trim={0 2.5cm 0 5cm},clip]{images/aoi-single_burst}
    \caption{The time average peak Age of Information with burst and \gls{soa} loss values against the dynamic reliability logic for different network topologies.}
    \label{fig:aoi_burst}\vspace{-0.4cm}
\end{figure*}


This paper focuses on both transport layer and application layer metrics to determine the feasibility of dynamic reliability. For this, we have selected the session packet volume, as transmitted, retransmitted, lost and backlogged packets as \glspl{kpi} for the transport layer; while focusing on the \gls{aoi} for the application layer. The \gls{aoi} was chosen as a crucial indicator for the freshness of packets in real-time applications. More specifically, this work adopts the time average peak \gls{aoi} equation \cite{aoi_equation} depicted in Eq. \ref{aoi}, where $\Delta(r_{i+1})$ is the $i$th update at the time it was received at the server, for a session time period of $\tau$.

\begin{equation}
    \label{aoi}
    \gls{aoi}_\tau = \frac{1}{n-1}\sum_{i=1}^{n-1} \Delta(r_{i+1})
\end{equation}

We include a comparison between the vanilla QUIC implementation which does not enjoy the dynamic reliability extension, with a number of dynamic reliability policies. The tests were run a number of times for statistical significance, with the mean value of vanilla implementation used as a baseline for comparison. The topology utilised both random loss and bursty loss to explore the bounds of dynamic reliability. The \gls{soa} loss in the figures correspond to the loss values presented in Table. \ref{tab:path_char}, for ease of comparison between bursty and random loss scenarios.

\subsection{Transport-Layer KPIs}

To analyse the performance gain at the transport layer due to dynamic reliability, the volume of transmitted and backlogged packets is examined. The figures are in the form of boxplots, which take the vanilla implementation as a benchmark, depicted as the red dashed line.

As seen in Fig. \ref{fig:sent_burst}, the loss plays a crucial role in the performance of the reliability policies. The policies under random loss did incredibly well for the networks with a larger capacity, namely \gls{mmwave} and Sub-6~GHz, whereas for burst loss, the lower network capacities had a larger packet reduction. With the increase in burst loss, the behaviour of the set split reliable policies became unpredictable, if a reliable assignment happened to coincide with a burst loss, the number of transmitted packets increases, and vice versa. On the other hand, in smarter policies, such as Loss-Aware, the performance lightly matched the vanilla baseline, as the reliable assignment dominated the session to compensate for a higher burst loss. Not only that but, the burst loss also impacted the variance of the transmitted packets for the policies.

Unsurprisingly, the unreliable focused policy, 80-20 split, outperformed other policies for all topologies in random and bursty loss scenarios, with an approximate reduction of 80\%. That being said, the majority of the policies reduced the transmitted packets on the link by approximately 70\% for random loss, while the reduction started at $\approx 15\%$ and decreased as the loss increased for the burst loss scenario.

The retransmitted and lost packets, not shown due to space limitations, followed the same trend as the transmitted packets for the random loss scenarios. However, for the burst loss scenarios, the larger capacity networks had a lower reduction in the retransmitted and lost packets. This can be seen as a favorable outcome since the lower capacity networks are scarce on resources. It is important to note that the Loss-Aware policy mimicked the vanilla approach as the burst loss increased, signifying the overwhelming appointment of reliable packets in adapting to the harsh burst loss conditions.
 
Alternatively, Fig. \ref{fig:backlog_burst} clearly shows a stark comparison between the policies and loss scenario in the reduction of the backlogged packets. The Loss-Aware policy for random loss scenario reduced the backlogged packets by up to 50\%, beating all other policies by approximately 30\%. Furthermore, it is clear that the unreliability focused policies resulted in the lowest backlog for the session. In comparison, we notice that the burst loss and the backlogged frequency have a positive correlation, where the maximum reduction of the backlogged packets for the policies is at most 20\%. Much like the transmitted packets, the probability of a burst loss occurrence plays a vital role in the number of retransmissions sent and by extension the number of backlogged packets. Thus, we can conclude that the stress placed on the buffer is a result of the reliable packets which is tightly coupled with the congestion on the session. Whereas, unreliable focused policies did not encounter such a phenomenon regardless if it was experiencing a burst loss.


\subsection{Application-Layer KPIs}

The feasibility of dynamic reliability for real-time applications can be determined by the \gls{aoi}, with comparison across different topologies and policies. If we take a strict approach and consider anything below $10$~ms is real-time \cite{real-time}, then all the reliability policies passed that requirement, which is attractive for real-time applications, as shown in Fig. \ref{fig:aoi_burst}. Utilising the median as an estimate of the runs, the policies in the WLAN and Sub-6~GHz topology with random loss floated around $4-5$~ms with negligible difference, while the \gls{aoi} for \gls{mmwave} was $\approx 2-3$~ms. It is clear that the \gls{aoi} and the network capacity have a negative correlation, as the network capacity decreases, the \gls{aoi} increases. The same correlation is extended to the bursty loss scenarios, where \gls{mmwave} dominated the other topologies. That being said, it is crucial to note that the \gls{aoi} for the reliability policies is often slightly better than or equal to the \gls{aoi} of the vanilla implementation, proving that dynamic reliability reduces the congestion of the session at no cost to the \gls{aoi}.
 

\section{Conclusion}\label{sec:conclusion}
In this work, we focus on addressing the fundamental challenge of OOD detection tasks, which is how to fully understand the semantic discrepancy between the ID/OOD samples. We reveal that the key to success in the realistic SCOOD task is to allocate as many ID samples in the unlabeled set correctly as possible. To this end, we propose a novel uncertainty-aware optimal transport scheme that introduces class-specific energy scores as guidance for effective label assignment. Experimental results show that our method achieves better performance than previous state-of-the-art methods on SCOOD benchmarks.

\textbf{Limitations.} In addition to temperature scaling, other techniques such as feature clipping applied in ReAct~\cite{sun2021react} also enhance the performance of energy score, so how to obtain an OOD score that best fits the SCOOD task can be further explored. Moreover, a setting highly related to SCOOD has been proposed in \cite{katz2022training} and formulated as a constrained optimization problem. We will also theoretically analyze these practical OOD settings in our feature work.

% \section*{Acknowledgments}
\textbf{Acknowledgments.} 
This work is supported by National Key R\&D Program of China under Grant 2020AAA0105701, National Natural Science Foundation of China (NSFC) under Grants 61872327, Major Special Science and Technology Project of Anhui, National Natural Science Foundation of China (62033012) and Ant Group through Ant Research Intern Program.


\printbibliography  

\end{document}
