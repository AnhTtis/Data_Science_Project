
\section{\ac{ls} Function Processing}\label{level-set-function-processing}

In this section, we detail the \ac{ls} description of the geometry used in the method. We explain how the output vector from the \ac{nn} $\mathbf{\varphi}$ is used to define the \ac{ls} to be used in the numerical method. The computation of the \ac{ls} function is performed in four steps. We introduce an interpolation step in Section \ref{interpolation} to obtain a first \ac{ls} function $\phi_{n(1)}$. In Section \ref{smoothing}, we smooth out this \ac{ls} to obtain a smooth \ac{ls} function $\phi_{f(2)}$. Then, we perform a reinitialization of that \ac{ls} in Section \ref{reinitialization} to obtain an \ac{ls} function $\phi_{s(3)}$. Finally, in Section \ref{translation}, we propose a volume correction strategy to end up with the final \ac{ls} function $\phi_{b(4)}$. In the subsequent sections, we will refer to this final \ac{ls} function $\phi_{b(4)}$ as $\phi$.


\subsection{Interpolation}\label{interpolation}

 In this method, we work with discrete \ac{ls} functions $\phi \in V_h^1$. The  \ac{DOF} of $V_h^1$ are the values of the function at the vertices of $\mathcal{T}_h$, which we denote with $\{\boldsymbol{x}_i\}_{i=1}^N$, where $N$ is the number of mesh nodes. Thus, there is an isomorphism between $V_h^1$ and $\mathbb{R}^N$. The output image vector of the \ac{nn} $\varphi$ are the \ac{DOF} values that uniquely determine the \ac{ls} \ac{fe} function.

\subsection{Smoothing}\label{smoothing}
Next, we convolve the function with a linear filter for smoothing:
\begin{equation}
  {\phi_{f(2)}}_i = (\sum^N_{j=1} w_{ij} )^{-1} ( \sum^N_{j=1} w_{ij} {\phi_{n(1)}}_j),
\end{equation}
where $w_{ij}=\max(0,r_f-\vert x_i-x_j \vert )$,  ${\phi_{n(1)}}_j$ is the $j^{th}$ degree of freedom of the function $\phi_{n(1)}$, ${\phi_{f(2)}}_i$ is the $i^{th}$ degree of freedom of the function $\phi_{f(2)}$ and $r_f$ is the smoothing radius.

\subsection{Reinitialization}\label{reinitialization}
We then reinitialize the \ac{ls} as a signed distance function. This is often done to gain control over the spatial gradient of the \ac{ls} function to improve convergence \cite{vanDijk2013}. In our case, it is of even greater importance as it also guarantees that when we apply the translation to the \ac{ls} to satisfy the volume constraint, we do not artificially introduce volumes into the domain far from the boundary where the \ac{ls} is close to zero. This would add discontinuity to the problem harming convergence. To perform this step, we solve the reinitialization equation in \cite{Xing2009} to obtain a signed distance function $\phi_{s(3)}\in V_h^1$:
\begin{equation}
  \frac{ \partial \phi_{s(3)} }{\partial \tau} + \text{sign}(\phi_{s(3)})(\vert \boldsymbol{\nabla}\phi_{s(3)} \vert - 1 ) = 0.
\end{equation}
The problem is solved using Picard iterations at steady state using the initial point $\phi_{f(2)}$ so that only one solve of the adjoint equation is required in the backward pass. Artificial viscosity is added to the problem for stabilization and a surface penalty term is integrated on the embedded boundary to prevent movement of the zero iso-surface in the reinitialization. The weak form of the problem is then to find the solution $\phi_{s(3)}\in V_h^1$ of the equation:
\begin{equation}
  \int_\Omega ( \boldsymbol{w}\cdot\boldsymbol{\nabla}(\phi_{s(3)})  v  + c_a h \vert \boldsymbol{w} \vert \boldsymbol{\nabla}(\phi_{s(3)}) \boldsymbol{\nabla} v  - \text{sign}(\phi_{s(3)}) v  ) d\Omega + \int_\Gamma (\phi_{s(3)}, v )d\Gamma = 0
\label{eq:Rs}
\end{equation}
for all $ v \in V_h^1$, where $\boldsymbol{w} = \text{sign}(\phi_{s(3)})(\boldsymbol{\nabla}(\phi_{s(3)})/ \vert\boldsymbol{\nabla}(\phi_{s(3)}) \vert )$, $c_a$ is the stabilization coefficient, set to $3$, and $h$ is the element size. 

\subsection{Translation}\label{translation}
The next task is to impose the volume constraint. This is done by applying a translation to the entire \ac{ls} function. Here we solve the nonlinear equation $\mathscr{V}$  for the scalar bias  $b$ to obtain a translation to the \ac{ls} function such that the domain $\Omega_{\rm in}$ satisfies the volume constraint. A bisection method is used to find the root $b$ of the equation:
\begin{equation}
  \mathscr{V} =  \int_{\Omega_{\rm in}} d\Omega - \mathscr{V}_0 = 0,
\end{equation}
where $\Omega_{\rm in}$ is defined by the \ac{ls} $\phi_{b(4)} = \phi_{s(3)} + b$ and $\mathscr{V}_0$ is the volume fraction given by the constraint. This is the \ac{ls} which is then used to define the boundary of the \ac{fe} problem. Details of how the \ac{ls} is used to define domains is given in the next Section.
