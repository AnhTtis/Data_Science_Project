\documentclass{article}
\usepackage[utf8]{inputenc}

\usepackage[letterpaper, portrait, margin=1in]{geometry}

\usepackage{parskip}
\usepackage{newpxtext,newpxmath}
\usepackage{csquotes}
\usepackage{graphicx}
\usepackage{hyperref}
\usepackage{amsmath}
\usepackage{braket}
\usepackage{mathdots}
\usepackage{color}
\usepackage[noadjust]{cite}
\usepackage[affil-it]{authblk}

\usepackage{tikzit}
\usetikzlibrary{decorations.pathreplacing}
\input{default.tikzstyles}

\newcommand{\symp}[2]{\left\langle #1, #2 \right\rangle}
\DeclareMathOperator{\Tr}{Tr}
\DeclareMathOperator{\rank}{rank}
\newcommand{\bgs}[1]{\textcolor{magenta}{(brian: #1)}}
\newcommand{\vlb}[1]{\textcolor{teal}{(valerie: #1)}}

\title{NoRA: A Tensor Network Ansatz for Volume-Law Entangled Equilibrium States of Highly Connected Hamiltonians}
\author{Val\'erie Bettaque \\ \href{mailto:vbettaque@brandeis.edu}{vbettaque@brandeis.edu} 
   \and Brian Swingle \\ \href{mailto:bswingle@brandeis.edu}{bswingle@brandeis.edu}}
\affil{Department of Physics, Brandeis University, Waltham, MA 02453}
\date{}

\begin{document}

\maketitle

\begin{abstract}

Motivated by the ground state structure of quantum models with all-to-all interactions such as mean-field quantum spin glass models and the Sachdev-Ye-Kitaev (SYK) model, we propose a tensor network architecture which can accomodate volume law entanglement and a large ground state degeneracy. We call this architecture the non-local renormalization ansatz (NoRA) because it can be viewed as a generalization of MERA, DMERA, and branching MERA networks with the constraints of spatial locality removed. We argue that the architecture is potentially expressive enough to capture the entanglement and complexity of the ground space of the SYK model, thus making it a suitable variational ansatz, but we leave a detailed study of SYK to future work. We further explore the architecture in the special case in which the tensors are random Clifford gates. Here the architecture can be viewed as the encoding map of a random stabilizer code. We introduce a family of codes inspired by the SYK model which can be chosen to have constant rate and linear distance at the cost of some high weight stabilizers. We also comment on potential similarities between this code family and the approximate code formed from the SYK ground space.

\end{abstract}

\tableofcontents

\section{Introduction}

The increasing complexity of source code poses a key challenge to the reliability of large-scale software systems. Software bugs in these systems can lead to safety issues~\cite{bug_safety} for users around the world as well as cause non-negligible financial losses~\cite{bug_loss}. As such, developers have to spend a large amount of time and effort on bug fixing. Consequently, \aprfull (\apr), designed to automatically generate patches to fix software bugs, has attracted wide attention from both academia and industry~\cite{long2016prophet, legoues2012genprog, long2015spr, lou2020can, tufano2018empstudy}. 


To achieve \apr, one popular approach is known as Generate-and-Validate (G\&V)~\cite{qi2015gv, ghanbari2019prapr, lou2020can, le2016hdrepair, legoues2012genprog, wen2018capgen, hua2018sketchfix, martinez2016astor, koyuncu2020fixminder, liu2019tbar, liu2019avatar}, which is typically based on the following pipeline: First, fault localization techniques~\cite{wong2016fl, abreu2007ochiai, zhang2013injecting, papadakis2015metallaxis, li2019deepfl, li2017transforming} are applied to determine the suspicious locations in programs where bugs are likely to exist. Then, the buggy locations are used by the \apr tools to generate a list of patches that replace buggy lines with correct lines. Afterward, each patch is validated against the original test suite to identify any \emph{plausible patches} (i.e., passing all tests in the test suite). Finally, to determine the \emph{correct patches}, developers examine the list of plausible patches to see if any of them can correctly fix the bug. 

Traditional \apr tools can mainly be categorized into heuristic-based~\cite{legoues2012genprog, le2016hdrepair, wen2018capgen}, constraint-based~\cite{mechtaev2016angelix, le2017s3, demacro2014nopol, long2015spr} and \template~\cite{ghanbari2019prapr, hua2018sketchfix, martinez2016astor, liu2019tbar, liu2019avatar}. Among these traditional tools, \template \apr tools~\cite{ghanbari2019prapr, liu2019tbar, benton2020effectiveness} have been able to achieve state-of-the-art results. \Template \apr tools typically leverage pre-defined templates (e.g., adding a nullness check) for bug fixing. However, since these fix templates are typically handcrafted, the number and types of bugs they are able to fix can be limited. 



To address the limitations of traditional \apr, researchers have proposed various \learning \apr tools~\cite{li2020dlfix, chen2018sequencer, jiang2021cure, lutellier2020coconut, zhu2021recoder, ye2022rewardrepair} based on the \nmtfull (\nmt) architecture~\cite{sutskever2014mt} where the input is the buggy code snippets and the goal is to translate the buggy code snippets into a fixed version. To accomplish this, \learning \apr tools require supervised training datasets with pairs of both buggy and fixed code snippets in order to learn how to perform this translation step. These training data are usually obtained by mining historical bug fixes using heuristics/keywords~\cite{dallmeier2007benchmark}, which can be imprecise for identifying bug-fixing commits; even the actual bug-fixing commits can include irrelevant code changes, leading to further pollution in the dataset~\cite{xia2022alpharepair}.
% 
Moreover, it can be hard for such \apr tools to generalize and fix bug types unseen during training. 



To better leverage recent advances in \plmfull{s} (\plm{s}), researchers~\cite{xia2022alpharepair, xia2023repairstudy, kolak2022patch, prenner2021codexws} have directly applied \plm{s} to generate patches without bug-fixing datasets. These \llm-based \apr tools work by either directly generating a complete code function~\cite{prenner2021codexws, xia2023repairstudy} or predict/infill the correct code snippet given its surrounding context~\cite{xia2022alpharepair, xia2023repairstudy}. By directly using \llm{s} that are pre-trained on billions of open-source code snippets, \llm-based \apr tools can achieve state-of-the-art performance on many repair datasets~\cite{xia2022alpharepair}. 


% 
%
%

Traditional \apr tools have long used the insight of the \emph{plastic surgery hypothesis}~\cite{barr2014plastic} where it states that the code ingredients to fix a bug already exist within the same project. Traditional \apr tools have manually designed pattern-~\cite{ghanbari2019prapr, saha2017elixir} or heuristic-based~\cite{jiang2018simfix, legoues2012genprog} approaches to finding and using such relevant code ingredients to generate fixes for bugs. However, the plastic surgery hypothesis has been largely ignored in \llm-based \apr. In fact, \llm provides a unique opportunity to fully automate the plastic surgery hypothesis idea via fine-tuning (learning project-specific information via model updates from the buggy project) and prompting (directly providing relevant code ingredients to the model), and make it directly applicable to different languages (since the \llm{s} are typically multi-lingual).%
Moreover, despite the intensive manual efforts involved, traditional \apr tools still cannot fully leverage project-specific information due to large search space for leveraging/composing existing code ingredients. In contrast, the project-specific information can effectively leveraged by \llm{s} due to their power in code understanding/vectorization, e.g., even partial/imprecise information may still guide \llm{s} in correct patch generation!
 To this end, we ask the question: \emph{How useful is the plastic surgery hypothesis in the era of \plm{s}}?








\mypara{Our Work.} To answer the question, we present \ourtech{\xspace} -- a \llm-based approach that automatically utilizes the plastic surgery hypothesis by systematically combining multiple fine-tuning and prompting strategies for \apr. \ourtech fine-tunes \plm{s} using two novel domain-specific training strategies: \textbf{\epfinetune} -- we fine-tune using the original buggy project by aggressively masking out a high percentage of tokens, which allows \plm to learn project-specific code tokens and programming styles; and \textbf{\rofinetune} -- which only masks out a single continuous code sequence per training sample, allowing the model to get used to the final \csapr task of predicting a single continuous code sequence. Furthermore, we directly leverage the ability for \plm{s} to understand natural language instructions and introduce a novel prompting strategy, \textbf{\idprompting}, which uses information retrieval and static analysis to obtain a list of relevant identifiers for the buggy lines. While such relevant identifiers are critical for fixing some difficult bugs, they may not be seen by the \llm during inference due to limited context window size. Through the use of prompting, we directly tell the model to use these extracted identifiers (relevant code ingredients) to generate the correct code. Finally, to perform repair, we combine all four model variants (including the base model, both fine-tuned models and the base model with prompting) for the final repair.





While our insight of leveraging the plastic surgery hypothesis for \llm-based \apr is generalizable across different types of \plm{s}, to implement \ourtech, we choose a recent \plm{\xspace}, \ctfive~\cite{wang2021codet5}, which is pre-trained on millions of open-source code snippets. \ctfive is an encoder-decoder model trained using \mspfull (\msp) objective where a percentage of tokens are masked out and each continuous masked token sequence is referred to as a masked span. Also, although we only extract relevant identifiers from the current buggy project (since this paper focuses on the plastic surgery hypothesis), our work can be easily extended to obtain other code information (such as relevant statements or functions) from other sources, such as  the massive pre-training corpora~\cite{husain2020codesearchnet} or historical bug-fixing datasets~\cite{jiang2019infer}, which can provide more coding knowledge for \llm{s}. Besides, although we mainly focus on using traditional string comparison algorithms for information retrieval in this paper, these techniques can be easily replaced by other frequency-based retrieval~\cite{robertson2009probabilistic} and neural search (or embedding-based search)~\cite{reimers2019sentence}.
  In summary, this paper makes the following contributions:


%


\begin{itemize}[noitemsep, leftmargin=*, topsep=0pt]
    \item \textbf{Dimension.} This paper is the first to revisit the important plastic surgery hypothesis in the era of \llm{s}. It opens up a new dimension for \llm-based \apr to incorporate previously neglected information from the buggy project itself to boost \apr performance. Furthermore, it demonstrates the promising future of retrieval-based prompting for modern \llm-based \apr.
    \item \textbf{Implementation.} We implement \ourtech based on the recent \ctfive model. We augment the model using two novel fine-tuning strategies: \epfinetune and \rofinetune, along with a novel prompting strategy based on information retrieval and static analysis: \idprompting. We combine the patches generated by all four models together and perform patch ranking to speed up \apr.% 
    \item \textbf{Evaluation Study.} We conduct an extensive evaluation against state-of-the-art \apr tools. On the widely studied \dfj 1.2 and 2.0 datasets~\cite{just2014dfj}, \ourtech is able to achieve the new state-of-the-art results of 89 and 44 correct bug fixes (15 and 8 more than best baseline) respectively.  Furthermore, we perform a broad ablation study to justify our design. \ourtech demonstrates for the first time that the plastic surgery hypothesis can substantially boost \llm-based \apr and advance state-of-the-art \apr, while being fully automated and general. Moreover, even partial/imprecise code ingredients may still effectively guide \llm{s} for \apr!
\end{itemize}



\section{Network Architecture}
\label{sec:arch}

Throughout this section we work with general qudits of local dimension $d$. We first describe the general structure of the class of NoRA networks we consider, then we specialize to a particular network structure inspired by scaling and renormalization group (RG) considerations. We analyze both the entanglement and complexity of the scaling-adapted ground state network and discuss an extension to describe excited states. In particular, we show that a natural choice of energy scales in a toy model Hamiltonian can give rise to a power-law temperature dependence of the thermodynamic entropy and heat capacity.

\subsection{General Structure}

The NoRA network is defined by $L$ layers as in Fig.~\ref{fig:arch}, where we refer to the bottom qudits as \emph{ground state} qudits and the other qudits as \emph{excited state or thermal} qudits. When we set the thermal qudits to some fixed product state, $\ket{0}$, we obtain the \emph{ground state network} as in Fig.~\ref{fig:arch}. This nomenclature is chosen because we can view the network as a variational ansatz for the ground space of a mean-field model. From this point of view, the ground state qudits parameterize a space of states that would be identified with the degenerate ground space of the concrete model of interest.

\begin{figure}[htb]
    \centering
    \scalebox{0.8}{\tikzfig{figures/syk_circuit}}
    \caption{Basic architecture of the proposed NoRA tensor network ansatz. A code word $\ket{\psi_{\textrm{code}}}$ consisting of $n_0 \equiv k$ (logical) \enquote{ground-state} qudits is embedded  as $\ket{\Psi_{\textrm{phys}}}$ in the (physical) ground space of the $d^N$-dimensional many-body Hilbert space by the means of $L$ layers of some given depth $D$ quantum circuits. For each layer $1 \leq \ell \leq L$ the circuit $D_{\ell}$ acts on the $n_{\ell - 1}$ qudit output from the previous layer and an additional $\Delta n_{\ell}$ new ancillary \enquote{thermal} qudits initialized in state $\ket{0}$. We stress that the layer circuits $D_\ell$ do not have to respect locality structure depicted by the 1d arrangement of qudit lines.}
    \label{fig:arch}
\end{figure}

One way to think about the network is as a ``fine-graining'' circuit moving upwards from the bottom ground state qudits. This is the inverse of a conventional RG transformation since we are adding degrees of freedom. We start with $k$ of these ground state qudits. Then at each layer $\ell$ we add $\Delta n_{\ell}$ thermal qudits in the fixed state $\ket{0}^{\otimes \Delta n_{\ell}}$ and apply a depth $D$ quantum circuit to all the qudits in that layer. This circuit could also be generalized to be time evolution with a suitably normalized all-to-all Hamiltonian for a constant time (proportional to $D$). The next layer takes all the qudits from the previous layer and adds more thermal qudits to generate the hierarchical structure in Fig.~\ref{fig:arch}. The total number of qudits at layer $\ell$ is denoted $n_{\ell}$ and given by
\begin{equation}
    n_{\ell} = k + \sum_{\ell'=1}^{\ell} \Delta n_{\ell'}.
\end{equation}
The total number of qudits is therefore
\begin{equation}
    N \equiv n_L = k + \sum_{\ell=1}^L \Delta n_{\ell}.
\end{equation}


\subsection{Scaling Specialization}

As is, we have described a fairly general architecture. Motivated by scaling and renormalization group considerations, we will primarily consider the special case where $n_{\ell} \sim k + r^{\ell}$, so that the number of thermal qudits is increasing exponentially with each layer up from the bottom. Viewing the top layer as the UV or microscopic degrees of freedom and the bottom layer as the IR or emergent degrees of freedom, moving from the UV to IR (top to bottom) mimics a renormalization group transformation where we remove some fraction of the thermal degrees of freedom at each step. Indeed, borrowing the language of MERA and DMERA and viewing the circuit from top to bottom, the individual layers are like disentanglers that leave behind some decoupled degrees of freedom, the thermal qudits added at that layer. In this scheme, we choose the number of qudits at layer $\ell$ to be
\begin{equation}
    k + r^{\ell} \stackrel{!}{=} n_{\ell} = k + \sum_{\ell'=1}^{\ell} \Delta n_{\ell},
\end{equation}
implying that the number of new thermal qudits for each layer must be
\begin{align}
\begin{split}
    \Delta n_{\ell > 1} &= r^{\ell} - r^{\ell - 1}, \\
    \Delta n_{1} &= r.
\end{split}
\end{align}
For the case of $r=2$, which we primarily consider in this work, this simplifies to approximately $\Delta n_{\ell} = 2^{\ell - 1}$ for all layers $\ell$.

\subsection{Entanglement and Complexity}

We next discuss the entanglement and complexity of the RG-inspired network. There are $\text{O}(N)$ non-trivial bonds in the circuit, of which $N$ bonds connect to the same constant-depth circuit in the last layer. It is therefore straightforward to establish that the network has the potential to encode volume law entanglement for sub-regions of a \enquote{typical} UV state. We also explicitly demonstrate that this is achievable within the Clifford model discussed below in Sections~\ref{sec:clifford}, \ref{sec:numerical}, and \ref{sec:code_structure}.

Turning to the complexity, we take the number of gates in the network as an estimate of the circuit complexity of the UV state, although in general this is only an upper bound. For a layer $\ell$ with $n_{\ell}$ total qubits in it, we apply $D$ rounds of $\lfloor n_{\ell}/q \rfloor$ $q$-qudit gates, so the number of gates of layer $\ell$ is
\begin{equation}
   \text{gates at layer } \ell = D \cdot \lfloor n_{\ell}/q \rfloor.
\end{equation}
Summing this result over all layers and assuming that $q$ divides $n_{\ell}$ without remainder gives a total number of gates equal to
\begin{equation}
\label{eq:circuit_complexity}
    \text{total gates} = \frac{D}{q} \sum_{\ell=1}^L n_{\ell} = \frac{D}{q} \left(L \cdot k + \frac{r^{L+1} - r}{r-1} \right).
\end{equation}
In sections \ref{sec:numerical} and \ref{sec:code_structure} we will cast this result into simpler leading-order expressions that correspond to the respective types of ground space scaling being considered.

\subsection{Extension to Excited States}

Let us conclude this section by extending the ground state network we have so far discussed to the case of excited states. As we have repeatedly emphasized, the discussion so far is general and does not consider a particular physical Hamiltonian. We are simply trying to match certain qualitative features of the entanglement and complexity expected for mean-field models. A structure similar to what we will consider here was recently studied for non-interacting fermions and advocated for as a general approach to approximating thermal states~\cite{sewell_thermal_multi_scale_2022}.

The idea is to introduce a toy Hamiltonian for which the above network is an exact ground state for any choice of state on the $k$ ground state qudits. In other words, the toy Hamiltonian has an exactly degenerate ground space. The Hamiltonian is constructed in a standard way by introducing projectors $P = |0\rangle \langle 0|$ for each thermal qudit and defining corresponding projectors acting on the UV qudits by conjugating these elementary projectors with the network circuit. Let $\tilde{P}_i$ denote the projector for thermal qudit $i$ conjugated by the network circuit. The toy Hamiltonian is
\begin{equation}
\label{eq:stab_hamiltonian}
    H = - \sum_i J_i \tilde{P}_i,
\end{equation}
where $J_i$ are a set of free parameters that determine the energy scale associated with each thermal qudit. Note that -- just like the circuit it encodes -- this Hamiltonian is highly non-local and not necessarily few-body, thus limiting the potential for physical interpretation. The setup is described in more detail in appendix \ref{sec:entropy_scaling}. 

Again motivated by RG considerations, in which the energy scale of excitations decreases by a fixed factor after every RG step (top to bottom), we take the $J_i$ to be equal within a layer and to depend on the layer index $\ell$ as
\begin{equation}
\label{eq:energy_scaling}
    J_{\ell} = \Lambda \cdot e^{-\gamma (L-\ell)}.
\end{equation}
In this way, the UV energy scale is $\Lambda$ and the energy of excitations decreases exponentially with the layer index decreasing towards the IR. The free parameter $\gamma$ controls the rate of decrease.

As computed in appendix \ref{sec:entropy_scaling}, the entropy for the Gibbs ensemble associated to said toy Hamiltonian describing our tensor network ansatz (and for general scaling of $J_{\ell}$) is
\begin{equation}
\label{eq:stab_entropy}
    S = \log\left(d^{k} \cdot (d-1)^{\braket{N-k}}\right) + \sum_{\ell} \Delta n_{\ell} \cdot S(p_{\ell}),
\end{equation}
where we defined a probability, 
\begin{equation}
    p_{\ell} = \frac{d - 1}{e^{\beta J_{\ell}} + d - 1},
\end{equation}
$S(p_{\ell})$ is the classical binary entropy function,
\begin{equation}
    S(p_{\ell}) = - p_{\ell} \cdot \log(p_{\ell}) - (1-p_{\ell}) \cdot \log(1-p_{\ell}),
\end{equation}
and 
\begin{equation}
    \braket{N-k} = \sum_i p_i = \sum_{\ell} \Delta n_{\ell} \, p_{\ell}.
\end{equation}
Note that in the case of qubits ($d=2$), $p_{\ell}$ coincides with the ordinary Fermi-Dirac distribution, in which case $\braket{N-k}$ is analogous to a sum of occupation numbers.

Plugging in \eqref{eq:energy_scaling} and going to the low-temperature regime (relative to the energy scale $\Lambda$), \eqref{eq:stab_entropy} can be approximated in the continuum limit as
\begin{align}
\label{eq:stab_entropy_approx}
\begin{split}
    & S - k \cdot \log(d) \\ \lessapprox{}& (d-1) (N-k) \cdot \frac{\alpha}{\gamma} \cdot \Gamma\left(\frac{\alpha}{\gamma} + 1\right) (\beta \Lambda)^{-\alpha/\gamma} \\
    \propto{}& (T/\Lambda)^{\alpha/\gamma},
\end{split}
\end{align}
with $N = k + r^L$ and $\alpha = \log(r)$. This together with the specific example depicted in figure \ref{fig:entropy_scaling} confirms that in this limit the entropy does obey a power law. By choosing the parameters $\alpha$ and $\gamma$ suitable, one could even match the precise low-temperature behavior of the SYK heat capacity $C_V$ (which is proportional to $T$) due to $dS = \frac{C_V}{T} dT$:
\begin{equation}
    C_V = T \left( \frac{dS}{dT} \right) \propto (T/\Lambda)^{\alpha/\gamma}.
\end{equation}

\begin{figure}[htb]
    \centering
    \includegraphics{figures/entropy_scaling.pdf}
    \caption{Logarithmic scaling of the exact Gibbs entropy $S_{\textrm{stab}}$ associated to $H$, and the low-temperature approximation $S_{\textrm{approx}}$ for $L=20$, $r=2$, $k=1$, $d=2$, $\Lambda=1$ and $\gamma = 0.4$. Both match almost exactly for our choice of parameters and small $T/\Lambda$, confirming the existence of a scaling law. The same is also true for other choices of $\gamma$ (the only significant free parameter), as seen in figure \ref{fig:entropy_scaling_appendix}.}
    \label{fig:entropy_scaling}
\end{figure}

\subsection{Summary}

Starting from the general architecture in Figure~\ref{fig:arch}, we introduced the RG-inspired network in which the number of qudits at layer $\ell$ is $k + r^\ell$. In the special case where $k=0$, i.e. a non-degenerate ground space, the number of qudits decreases by a factor from one layer to the next into the IR. This decrease is analogous to a block decimation RG procedure applied to a quantum state. The case of $k\neq 0$ describes a generalization of such an RG procedure. The entanglement entropy of the physical states produced by the RG-inspired network can be volume-law, as expected for mean-field models. We also showed that the ground state network can be extended to provide a model of thermal excitations in which the thermodynamic heat capacity has a power-law temperature dependence at low temperature. These general features are all chosen to match characteristics of the SYK model, which also features a nearly degenerate space of highly entangled ground states and a power-law heat capacity at low temperature.

%
\section{Analysis of the SYK-Inspired Code}
\label{sec:code_structure}

We now consider in more detail the properties of the SYK inspired code with $k=r^a$ and $L=a+b$ for two integers $a$ and $b$. Recall that the total number of qudits is 
\begin{equation}
    N = r^a + r^{a+b},
\end{equation}
and the ratio between ground state qudits and the total number of qudits (i.e. the rate) is therefore
\begin{equation}
    \frac{k}{N} = \frac{1}{1 + r^b},
\end{equation}
which is independent of $a$. The thermodynamic limit $a \rightarrow \infty$ gives a family of codes with non-zero rate. We already established in Section~\ref{sec:numerical} that this code can be highly entangled. It is also interesting to consider its complexity.

In this case, the complexity sum \eqref{eq:circuit_complexity} can be rewritten as
\begin{equation}
    \text{total gates} = \frac{D}{q \cdot (1 + r^b)} \cdot N \log_r N + \mathcal{O}(N).
\end{equation}
This leading $N \log N$ scaling with the total number of degrees of freedom can be compared to holographic complexity conjectures applied to JT gravity~\cite{jt_complexity}; one also gets $N \log N$ by studying, for example, the volume (length) of the wormhole dual to the thermofield double state with temperature of order $1/N$. The key point is that the throat of the wormhole is long, of order $\log N$, at this temperature. Hence, the circuit complexity of our SYK-inspired encoding also resembles that obtain from holographic models dual to SYK.

For the estimates discussed below, we continue to assume that the layers are composed of random 2-qudit Clifford gates applied to random pairs of qudits. We caution that this is certainly not correct for the actual SYK model: the gates must act on fermionic degrees of freedom and will not be Clifford (or the fermionic analogue of Clifford) generically. Here we continue to focus on the Clifford case for ease of analysis and for its interpretation in terms of an exact quantum error correcting code. Below we comment briefly on the potential similarities and differences with the actual SYK model.

\subsection{Distance Estimate and Stabilizer Weights}
\label{sec:syk_distance_weight}

We know the rate of our SYK-inspired code. To estimate the distance, we need to understand how logical operators grow as they pass from the IR to the UV. Let us assume that a typical operator grows in size by a factor of $g^D$ after passing through one layer (i.e. being conjugated by that layer unitary), up to a maximum size set by the total number of qudits. A way to estimate $g$ when the layer unitary is a random Clifford circuit can be found in appendix \ref{sec:growth_factor}. At the same time, the number of qudits is also growing, going from $k+r^{\ell-1}$ to $k+r^\ell$. The distance depends on whether the size of operators grows faster or slower than the number of qudits. Note that we saw already a manifestation of this competition in the discussion in Section~\ref{sec:numerical}; here we explain in more detail the issues.

From a given random circuit layer, we expect operators to grow by a factor of $g^D$ provided they are not close to maximum weight. If they are close to maximum weight, then they will grow by a reduced factor. We must compare this operator growth to the rate of qudit increase. The ratio $R_\ell$ between the number of qudits in successive layers is 
\begin{equation}
      R_\ell =  \frac{k+r^\ell}{k+r^{\ell-1}} = 1 + \frac{r-1}{1 + k/r^{\ell-1}}, 
\end{equation}
which monotonically increases with $\ell$. As logical operators evolve from layer to layer into the UV, the relative weight of the operator either increases or decreases depending on whether $g^D > R_\ell$ or $g^D < R_\ell$. The dynamics of this process, iterated over all $L$ layers, gives an estimate for the size of non-trivial logical operators.

\subsubsection{Warmup: Small Fixed $k$}

To illustrate the key competition, consider first the case in which $k$ is small and fixed. In this case, the ratio $R_\ell \rightarrow r$ as $\ell$ increases, so most of the evolution corresponds to a fixed ratio of $r$. In terms of the parameters above, we can achieve this regime by taking $b$ large at fixed $a$.

Suppose $g^D > r$. Then operator growth is the fastest process and logical operators will reach saturation. In this case, we expect the distance to be linear in $N$. The distance will not exactly saturate the singleton bound, but it may come close for large $D$.

Now suppose $g^D < r$. In this case, we are adding qudits faster than operators can grow, so the logical operators are ultimately supported on a dilute fraction of all the sites. Indeed, the size of a typical logical operator will be $g^{DL}$, whereas the total number of qudits is $N =  r^L (1 + r^{-b}) \approx r^L$. Expressed in terms of $N$, the size of a typical logical operator is
\begin{equation}
    g^{DL} \sim N^c
\end{equation}
where $c = \frac{\ln g^D}{\ln r} < 1$. Hence, we expect a distance that scales as a sublinear function of $N$.

\subsubsection{SYK-Like Scaling}

Now we turn to the case where $k = r^a$ is large and $b$ is fixed. Here, when $\ell$ is small, the $R_\ell$ ratio is close to one and the number of qudits is barely increasing from layer to layer. In this regime, operator growth is completely dominant. In contrast, at the most UV layer, where $\ell=L=a+b$, the ratio is
\begin{equation}
   R_L = 1 + \frac{r-1}{1+r^{1-b}} < r. 
\end{equation}

Suppose $g^D > R_L$. Then operator growth always dominates over qudit growth. However, because the initial number of qudits (the ground state qudits) is large, we still have to compare the total operator size, $g^{D L}$, to the total number of qudits, $N = r^L ( 1+ r^{-b})$. We see again that if $g^D >r$, then this naive estimate gives an operator weight larger than $N$, meaning that the operator growth actually saturated at something proportional to $N$. If $g^D < r$, then we are again in the situation where $g^{DL} \sim N^c$.


Suppose $g^D < R_L$. Then there will be some layer $\ell^*$ such that operator growth and qudit growth switch dominance as $\ell$ increases through $\ell^*$. We may approximately determine this crossover scale from
\begin{equation}
    g^D = R_{\ell^*},
\end{equation}
noting that this $\ell^*$ is not typically an integer. In the thermodynamic limit $a \rightarrow \infty$, we must have $\ell^* = a + b^*$ for some constant $b^*$ since the ratio $R_\ell$ is essentially unity until $r^{\ell}$ is comparable to $k$.

Now between $\ell=1$ and $\ell = \ell^*$, logical operators will grow faster than the number of qudits. Assuming they don't reach saturation, they will grow by roughly a factor of $g^{D\ell^*}$. By contrast, the number of qudits at layer $\ell^*$ is
\begin{equation}
    n_{\ell^*} = r^a (1 + r^{b^*}),
\end{equation}
so the ratio of operator size to number of qudits is
\begin{equation}
    \left( \frac{g^D}{r} \right)^a \frac{g^{D b^*}}{1+r^{b^*}}.
\end{equation}
This ratio vanishes as $a\rightarrow \infty$ since we are assuming that $g^D < R_L$ and $R_L < r$. Hence, $g^{D\ell^*} \sim (n_{\ell^*})^c$ as above.

There are a fixed number of layers from $\ell^*$ to $L$ since $b$ and $b^*$ are fixed as $a\rightarrow \infty$. Therefore operators and the number of qudits grow by an additional factor independent of $N$ from $\ell^*$ to $L$. Hence, the scaling of $g^{DL}$ with $N$ is the same as the scaling of $g^{D\ell^*}$ with $n_{\ell^*}$, that is $g^{DL} \sim N^c$.




\subsubsection{Stabilizer Weights}

We expect that the stabilizer weights will display a similar pattern as in Figure~\ref{fig:stab_weights}. In particular, a non-zero fraction of all the stabilizers will have constant weight. These arise from the UV most layer. Then as we descend in the network towards the IR, there are fewer stabilizers but of increasing weight. In particular, there are at least a few stabilizers of very high weight, similar to the weight of logical operators. 

\subsection{Comparison to SYK} 

We now compare features of the SYK-inspired code to those of the actual SYK model. To be precise, we will compare a particular realization of the SYK Hamiltonian (with $q=4$), $H_{\text{SYK}}$, with a particular realization of the toy code Hamiltonian, $H_{\text{code}}$, for the SYK-inspired code (see Section~\ref{sec:arch}). (It is also interesting to consider supersymmetric generalizations~\cite{Fu_2017}.)
\begin{itemize}

   \item{[Hamiltonian structure]} $H_{\text{SYK}}$ is composed of $O(N^4)$ weight-$4$ fermion terms (all possible such terms). These terms do not all commute and they enter $H_{\text{SYK}}$ with random coefficients. $H_{\text{code}}$ is composed of $O(N)$ commuting terms with fixed coefficients. The weight of the terms varies, with many having low-weight but a significant fraction having high weight, comparable to the distance of the corresponding code.
   
    \item{[Ground space]}  $H_{\text{SYK}}$ has $e^{s_0 N}$ approximate ground states which are approximately degenerate with level spacing $e^{-\alpha N}$. $\alpha$ and $s_0$ are constants, independent of $N$. Similarly, $H_{\text{code}}$ has $d^k = e^{ \frac{\ln d}{1+r^b} N}$ exactly degenerate ground states.
    
    \item{[Low temperature thermodynamics]} The SYK model has a low temperature heat capacity proportional to temperature $T$. Similarly, the parameters of $H_{\text{code}}$ can be chosen so that its low temperature heat capacity is proportional to $T$.
    
    \item{[Fine-grained spectrum]} The fine-grained energy spectrum of $H_{\text{SYK}}$ is random-matrix-like~\cite{Cotler_2017,saad2019semiclassical}. The fine-grained energy spectrum of $H_{\text{code}}$ is not random-matrix-like because $H_{\text{code}}$ is a commuting projector Hamiltonian.
    
    \item{[Entanglement]} Both models feature energy eigenstates with volume-law entanglement. The entanglement spectrum will, however, be quite different between the two kinds of states. In particular, eigenstates of $H_{\text{code}}$, being stabilizer states, have a flat entanglement spectrum.
    
    \item{[Complexity]} We only have estimates here. Using the duality to JT gravity and holographic complexity/geometry conjectures, the circuit complexity of the SYK approximate ground states is estimated to be $O(N \ln N)$. We have an explicit estimate (and upper bound) of $O(N \ln N)$ for circuit complexity of the ground space of $H_{\text{code}}$.
    
 
\end{itemize}

The many similarities between $H_{\text{SYK}}$ and $H_{\text{code}}$ are the basis for our conjecture that the architecture in Figure~\ref{fig:arch} has the potential to describe the physics of the SYK model once the tensors in the network have been adapted to a particular SYK instance, for example, using a variational approach. However, there are also crucial differences between the two. Two that stand out are the different scalings of the weights of Hamiltonian terms with system size and the exact versus approximate nature of the ground state degeneracy. The fine-grained energy spectrum is also very different in the two cases. Thus, it will be informative in the future to explore our network architecture as a variational ansatz for the SYK ground space.

\subsection{SYK Ground Space as an Approximate Code}

Here we want to comment on another possibility raised by the similarities above. For $H_{\text{code}}$, we have seen explicitly that the ground space can be viewed as an error correcting code with constant relative distance and constant rate (provided $D$ is big enough). In particular, it is an exact stabilizer code. This naturally raises the possibility that the approximate ground space of the SYK model could have interesting properties as an approximate quantum error correction code.\footnote{The network architecture presented here was first considered by one of the authors in fall 2019 during their stay at the Institute for Advanced Study and later presented in preliminary form, along with the potential code interpretation, in January 2020 at UCSB. Independently, the code properties of SYK in the thermal regime have been studied~\cite{Chandrasekaran_2022}.}

Thus we consider a code defined by the full approximate ground space of some particular $H_{\text{SYK}}$ realization. By construction this code has a constant rate as $N\rightarrow \infty$ which is given by ground state entropy density $s_0$. This code is not a stabilizer code, but it does have a sort of ``low weight'' definition via the SYK Hamiltonian. 

What is not immediately clear is the distance of this code. Moreover, since the code is approximate, we must specify precisely what we mean by the distance. We will defer a full discussion to a future work, but here let us note that if the architecture in Figure~\ref{fig:arch} does indeed provide a good approximation to the ground space of the SYK realization, then the same kind of scaling analysis discussed above for the random Clifford code would also provide an estimate for the operator size of logical operators.

In this case, it would be important to understand the analog of $r$ and $g^D$ in the SYK case. As one approach, we could fix $r=2$ and then adjust the layer circuits so that we get a good approximation to the ground space. The parameter $g^D$ would then be determined by the properties of these circuit. A simple random operator growth model may be too crude to capture the detailed physics, but continuing with this estimate for now, if the resulting $g^D$ were greater than $r$, then we have logical operators of weight proportional to $N$ and potentially distance proportional to $N$. Alternatively, if $g^D < r$, then the distance could be some power of $N$, $N^c$. It would be interesting to understand which of two cases is realized; this should be related to the spectrum of the scaling dimensions in the theory since these are related to the mixing properties of the scaling superoperator~\cite{PhysRevLett.101.110501}. Given the relatively low scaling dimension of the fermion operators, it may be that one is effectively in the $g^D<r$ regime.

\subsection{Summary}

We gave analytical estimates of the distance for a family of SYK-inspired codes in the thermodynamic limit of many qudits. This code family shares a number of similarities with known properties of the actual SYK model, although there are crucial differences as well. Viewing the approximate ground space of SYK as an approximate quantum code, the analysis of the SYK-inspired model suggests that the actual SYK ground space code, which has constant rate as $N\rightarrow \infty$, could have a distance $N^c$ for some constant $0<c\leq 1$.


\section{Clifford Ansatz}
\label{sec:clifford}

Having laid out the scaling-inspired architecture in the previous section and shown that it can capture some expected features of mean-field models, especially the SYK model, we now consider a concrete version of the network built from Clifford gates. We would also like to use the network as a variational ansatz to study physical mean-field models, but for the reasons outlined in the introduction, in this paper we focus on the Clifford model as an example where we can also classically simulate the network properties. A review of the Clifford group and how it can be implemented is provided in appendix \ref{sec:phase_space}. 

If the circuits in Figure~\ref{fig:arch} are composed of Clifford gates, then the network can be interpreted as an encoding circuit for a stabilizer quantum error correcting code~\cite{gottesman_stabilizer_codes_quantum_1997}. The ground state qudits then correspond to the logical qudits of the code. We focus in particular on the distance of the code and the weight of the stabilizers, as they provide a good heuristic for probing the entanglement structure and give us a glimpse at the network's potential as an error-correcting code. The purpose of this section is to review this error correction interpretation and setup the subsequent calculations in Sections~\ref{sec:numerical} and \ref{sec:code_structure}.

\subsection{The Clifford Group}


Let us briefly recall the motivation for Clifford circuits. In general, simulating quantum circuits on a classical computer architecture becomes difficult with increasing number of qudits due to the exponential scaling of the Hilbert space dimension with the number of qudits. However, we can still compute certain quantities efficiently on classical computers by restricting ourselves to a subgroup of the full unitary group that only scales linearly in the number of qudits ~\cite{gottesman_heisenberg_representation_quantum_1998, aaronson_improved_simulation_stabilizer_2004}. This group is called the \emph{Clifford group} and is defined as the subgroup of unitary operators that map Pauli strings to Pauli strings~\cite{gottesman_heisenberg_representation_quantum_1998}. Elements of the Clifford group can then be represented as \emph{Clifford circuits}, which are circuits composed of successive (elementary) Clifford gates acting on a bounded number of qudits at a time. An example of such a Clifford circuit is depicted in Figure \ref{fig:layer_circuit} in the context of random scrambling.

The Clifford group has a variety of applications in quantum information. For example, in the context of generating random states, the Clifford group is useful because it forms a $k$-design of the Haar measure of random unitaries. This means that quantities averaged over random choices of gates/states only start to differ between Clifford and Haar in probability moments higher than $k$, where we have $k=1$ for all possible qudit dimensions $d$, $k=2$ for all $d$ that are powers of primes, and $k=3$ when said base prime is 2. \cite{zhu_multiqubit_clifford_groups_2017, graydon_clifford_groups_are_2021}.

The reason the Clifford group for $N$ qudits with local dimension $d$ can be efficiently simulated resides in the fact that it is a projective representation of the symplectic group $\text{Sp}(2N, \mathbb{F}_d)$, where $2N$ is the vector space dimension the group acts on and $\mathbb{F}_d$ is the (unique) finite field with $d$\footnote{In this case $d$ must be the power of a prime.} elements. As mentioned before, the space of Pauli strings therefore scales linearly, with operators mapping between them being represented (up to a phase) by $2N \times 2N$ symplectic matrices over $\mathbb{F}_d$. Sampling Cliffords therefore can be achieved by sampling symplectic matrices, for which efficient algorithms exist \cite{koenig_how_efficiently_select_2014}.  A more detailed description of this framework is provided in Appendix \ref{sec:phase_space}.

\subsection{Random Layer Circuits}

To define a precise model based on our architecture, we have to make an explicit choice for the depth $D$ circuits $D_{\ell}$ that are applied at each layer $\ell$. Inspired by SYK, our approach is to apply $n_{\ell} / q$ randomly sampled Clifford gates\footnote{In general $q$ does not have to divide $n_{\ell}$ without remainder, in which case one can simply leave $n_{\ell} \mod{q}$ qudits unchanged at each sub-layer. In our computations we always chose our parameters such that this is not necessary. } to randomly chosen non-intersecting sets of $q$ qudits for each sub-layer $1 \leq m \leq D$ of the total layer circuit. Such a Clifford circuit is depicted in figure \ref{fig:layer_circuit} for $q=2$. Heuristically, this ansatz can be interpreted as a Trotterization of the SYK Hamiltonian, although with qudits instead of Majorana fermions \footnote{In an actual variational calculation with the SYK model, we might expect that the layer circuits are unitarites generated by SYK-Hamiltonian-like operators (although not necessarily the SYK Hamiltonian itself). The choice of random Clifford layers is thus loosely inspired by our expectations for the SYK model.} .

\begin{figure}[hbt]
    \centering
    \tikzfig{figures/wall_circuit}
    \caption{An example of a layer circuit acting on 10 qudits with $q=2$ and depth $D=3$. Each unmarked gate represents a randomly sampled Clifford element acting on two qudits, while the gates $\pi_m$ for $m=1,2,3$ are random permutations of the qudits. While such a circuit does not exhibit a causal cone, this non-locality of interactions is expected from mean-field models.}
    \label{fig:layer_circuit}
\end{figure}

With that we can then view the resulting network as an encoding circuit for a quantum stabilizer code. The $k$ ground state qudits are the logical qudits and the $N$ UV qudits are the physical qudits. We now briefly review stabilizer codes and the important notion of distance, which captures aspects of the entanglement structure discussed above in Section~\ref{sec:arch}.

\subsection{Stabilizer Codes}
\label{sec:stab}

A $[[N, k, \delta]]$ \emph{stabilizer code} that encodes $k$ logical qudits into $N$ physical qudits with distance $\delta$ is defined in terms of a \emph{stabilizer group} $S$, which is an abelian subgroup of the (generalized) Pauli group $P_d(N)$ i.e.\ the group generated by all possible $N$-element tensor products of ordinary Pauli operators ($d = 2$)
\begin{equation}
    X = \begin{pmatrix}
        0 & 1 \\ 1 & 0
    \end{pmatrix}, \,
    Y = \begin{pmatrix}
        0 & -i \\ i & 0
    \end{pmatrix}, \,
    Z = \begin{pmatrix}
        1 & 0 \\ 0 & -1
    \end{pmatrix},
\end{equation}
or their higher-dimensional counterparts ($d > 2$), which are defined in appendix \ref{sec:weyl_rep} \cite{gottesman_stabilizer_codes_quantum_1997}. The stabilizer group must therefore be generated by $N - k$ independent and commuting elements of $P_d(N)$. A \emph{code word} then is a state vector $\ket{\psi} \in \mathbb{C}^{d^N}$ that satisfies $s \ket{\psi} = \ket{\psi}$ for all $s \in S$. The space spanned by all possible code words is called the \emph{code space} and has dimension $d^k$ due to the rank of the group being $N-k$. The operators mapping logical states to other logical states are called \emph{logical operators} and must therefore commute with all elements of the stabilizer group and hence form the centralizer of the stabilizer group in in $P_d(N)$.

%\subsubsection{Error Correction}

%The purpose of stabilizer codes is to facilitate the detection and correction of errors that might occur in a quantum state without destroying its entanglement structure. One cannot use simple redundancy, as this is impossible in general due to the no-cloning theorem. In principle, this can be accomplished by performing projective measurements according to the basis elements of the stabilizer group, to check if the state still has unit eigenvalue as initially required. Repeating this with all stabilizer basis elements produces a $N-k$ dimensional vector called the \emph{syndrome} associated to the given error that occurred. If the syndrome does show an error for a certain stabilizer, one can apply a so-called \emph{destabilizer} to correct that error. Each destabilizer anti-commutes with a single stabilizer and commutes with all others, allowing us to flip a single localized error without affecting the other stabilizers or the code word. \bgs{destabilizers are only defined up to local operators, right?}

%However there are limits to this approach: Once the number of singular errors crosses the threshold given by the distance $\delta$, it is by definition impossible to reconstruct the original logical state since it has been affected by too many errors. It is therefore essential to find codes that maximize the scaling of $\delta$ with the total system size $N$ because the probability of errors will increase as well. Additionally, it can occur that an error exactly mimics one of the logical operators (up to an additional error that doesn't), meaning it can't be detected. However (for codes with a \enquote{good} distance) the probability of this happening quickly approaches zero with increasing $N$ due to the spreading of the encoded information implying that the logical operators must act non-trivially on almost all physical qudits, decreasing the chance of them randomly occurring.

\subsubsection{Decoupling \& Code Distance}

The code distance is a measure of how robust the code is to errors on the physical qubits. Determining the distance for a stabilizer code is in general a computationally intensive problem due to the potential for complex patterns of entanglement. We use an \emph{adversarial approach}, which is based on analyzing the mutual information
\begin{equation}
\label{eq:mut_inf}
    I(A, R) = S(A) + S(R) - S(AR)
\end{equation}
between all possible subsystems $A$ of the physical qudits and some external reference $R$ which is maximally entangled with the code space. A depiction of the setup can be found in figure \ref{fig:decoupling}.
\begin{figure}[htb]
    \centering
    \tikzfig{figures/decoupling}
    \caption{Circuit representation of state in which the code space is maximally entangled with a reference $R$. Here $U_{MN}$ is a unitary that takes states of the form $\ket{\psi_{\text{anc}}}_M \otimes \ket{\psi_{\text{code}}}_N$ and maps $\ket{\psi_{\text{code}}}$ to the code space of the chosen stabilizer code. $\ket{\psi_{\text{anc}}}$ is the all $0$ state of the ancillary qubits. If a region $A$ has zero mutual information with $R$, then it has no access to the encoded information. The code distance $\delta$ is the biggest integer such that all regions $A$ of size $|A| < \delta$ have $I(A,R)=0$.}
    \label{fig:decoupling}
\end{figure}

Because $R$ is maximally entangled with the code space, it is effectively tracking the encoded information. Therefore the question is how much of the system does an adversary need access to in order to be correlated with $R$ and thus have (at least partial) access to the encoded information. This correlation can be detected using the aforementioned mutual information \eqref{eq:mut_inf}, which becomes non-zero in such a case.
%In that case one says that $A$ is \emph{decoupled} from the rest of the physical qudit space, meaning that the state can be written as a tensor product on $A$ and its complement $MN / A$. 
The code distance $\delta$ is the biggest integer such that all regions $A$ with $|A|<\delta$ have $I(A,R)=0$. %is therefore
%\begin{equation}
%    \delta = |A^*|,
%\end{equation}
%where $A^*$ is one of the largest subsets of physical qudits which still satisfy $I(A^*, R) = 0$. 
%Since the code word is assumed to be maximally entangled with $R$, this statement is true for all possible choices of $\ket{\psi_{\text{code}}}$.

Implementing this approach as an algorithm is time-consuming though, since iterating through all possible choices for $A$ is combinatorically intensive. A way to simplify the procedure at the cost of only getting an upper bound approximation for the code distance is by randomly sampling choices for $A$ and determining the largest one which has vanishing mutual information. This Monte Carlo approach is the method we use.

\subsubsection{Stabilizer Weights}

It is also interesting to ask about the weights of the stabilizers. The weight of a Pauli string is defined as the number of elements of $P_d(1)$ in the tensor product representation of the operator that are not proportional to the identity operator $I$. Since $P_d(1)$ contains $d^2$ elements, there are $d^2 - 1$ such nontrivial operators.  If the stabilizer group has a generating set containing only Pauli strings of bounded weight, then we say the code has constant weight. The code space can always be obtained as the ground space of a Hamiltonian built from a generating set of the stabilizer group, and if the code has constant weight then there is such a Hamiltonian which contains only terms acting on a bounded number of qudits at a time.



\subsection{Summary}

Here we reviewed the notion of a stabilizer code and defined the random Clifford gate version of our architecture. In the following two sections, \ref{sec:numerical} and \ref{sec:code_structure}, we consider random stabilizer codes built from random Clifford layers inserted in the RG-inspired architecture (Figure~\ref{fig:arch} and Section~\ref{sec:arch}). We investigate the distance and stabilizer weights both numerically and via analytic arguments. We verify that these codes can be highly entangled, for example, with a distance proportional to $N$. We also study the distribution of stabilizer weights and show that some stabilizers do have high weight proportional to $N$. As such, they are not constant weight codes in general.




\section{Numerical examples}
\label{sect: numerical experiments}




In this section we will assess the efficiency of the Bubbles approach with the Dirac-Frenkel-MacLachlan approach against a spectral method in the two-dimensional case.



\subsection{Spectral scheme}

We start by discussing the spectral method we shall use to compare with the results of Algorithm \ref{algo: DFMP -- solve approximately cNLS}. We refer to \cite{fornbergPracticalGuidePseudospectral1996} for a general introduction to spectral methods for the Schr{\"o}dinger equation, and to \cite{antoineComputationalMethodsDynamics2013} for grid-based schemes applied to the Gross-Pitaevskii equation. 

We now present a method which can be understood as the application of \cite{bernierExactSplittingMethods2021} to a simpler equation, namely the Harmonic Oscillator. We use a splitting method to simulate the linear part \eqref{eqn: cNLS with psi -- linear part}, and thanks to \cite{bernierExactSplittingMethods2020,alphonsePolarDecompositionSemigroups2021} we have:
\begin{align}
    e^{-it (-\Delta + |x|^2)} 
    &= e^{- \frac{1}{2} \tanh(it) |x|^2} e^{\frac{1}{2} \sinh(2it)\Delta_x} e^{- \frac{1}{2} \tanh(it) |x|^2} \nonumber\\
    &= e^{- \frac{i}{2} \tan(t) |x|^2} e^{\frac{i}{2} \sin(2t)\Delta_x} e^{- \frac{i}{2} \tan(t) |x|^2}
    \label{eqn: num -- exact time splitting of HO}
.\end{align}
%
We can cite \cite{jinMathematicalComputationalMethods2011} which also presents a spectral method based on the Fourier transform with time splitting, however our method is different in that \eqref{eqn: num -- exact time splitting of HO} is exact and hence we do not have any time-splitting error.


The first and third exponentials on the RHS are straightforward to compute on a grid. For the second one, we use a Fourier transform: \( e^{\frac{i}{2} \sin(2t)\Delta_x} \) is the propagator of the following equation:
\begin{equation*}
    \partial_{t} \psi = i\cos(2t) \Delta_x \psi
.\end{equation*}
%
By using a Fourier transform, we get
\begin{equation*}
    \partial_{t} \mathcal{F}(\psi)(\xi)
    = i\cos(2t) \mathcal{F} \left( \Delta_x \psi\right)(\xi)
    = -i\cos(2t) |\xi|^2 \mathcal{F} \left( \psi \right)(\xi)
.\end{equation*}
%
Hence,
\begin{equation*}
    \mathcal{F}(\psi(t, \cdot))(\xi) = e^{-\frac{i}{2} \sin(2t) |\xi|^2} \mathcal{F}(\psi(0, \cdot))(\xi)
.\end{equation*}
%
The RHS exponential is straightforward to compute in the Fourier space. 
Hence, an exact-time spectral approximation of the solution to \eqref{eqn: cNLS with psi -- linear part} is given by Algorithm \ref{algo: num -- spectral solver linear part}.
From this, it is easy to obtain an algorithm which simulates \eqref{eqn: cNLS with psi} with interactions.
It consists in using a Strang splitting method on \eqref{eqn: cNLS with psi}, by splitting the linear part \eqref{eqn: cNLS with psi -- linear part} and the nonlinear part \eqref{eqn: cNLS with psi -- nonlinear part}. 
The linear part is approximated via Algorithm \ref{algo: num -- spectral solver linear part}, and the computation of interactions is explicit thanks to the fact that \( |u(t, x)|^2 \) does not depend on time (see e.g. \cite[Sect.~2.2]{faouGeometricNumericalIntegration2012}).
This fully describes Algorithm \ref{algo: num -- spectral solver}.

\begin{algorithm}
    \caption{Spectral solver for \eqref{eqn: cNLS with psi -- linear part}, with an exact time resolution for each splitting step.}
    \label{algo: num -- spectral solver linear part}
    \begin{algorithmic}
        \State Discretize the initial data \( \eta \) on a \( \text{Grid} \subset \mathbb{R}^d \).
        \For{ Each timestep of size \( \Delta t \) } 
            \For{ \( x \in \textsc{Grid} \) }
                \Comment{\(x \in \mathbb{R}^d\).}
                \State Multiply \( \eta(x) \) by \( e^{- \frac{i}{2} \tan(\Delta t) |x|^2} \).
            \EndFor
            %
            \State Apply a FFT to \( \eta \).
            \Comment{FFT: Fast Fourier Transform.}
            %
            \For{ \( \xi \in \textsc{Fourier Grid} \) }
                \Comment{\( \xi \in \mathbb{R}^d\).}
                \State Multiply \( \mathcal{F}(\eta)(\xi) \) by \( e^{-\frac{i}{2} \sin(2\Delta t) |\xi|^2} \).
            \EndFor
            %
            \State Apply an inverse FFT to \( \mathcal{F}(\eta) \).
            %
            \For{ \( x \in \textsc{Grid} \) }
                \State Multiply \( \eta(x) \) by \( e^{- \frac{i}{2} \tan(\Delta t) |x|^2} \).
            \EndFor
        \EndFor
    \end{algorithmic}
\end{algorithm}



\begin{algorithm}
    \caption{Spectral solver for \eqref{eqn: cNLS with psi}, with a Strang Splitting method.}
    \label{algo: num -- spectral solver}
    \begin{algorithmic}
        \State Discretize the initial data \( \eta \) on a \( \text{Grid} \subset \mathbb{R}^d \).
        \For{ Each timestep of size \( \Delta t \) } 
            \State Use Algorithm \ref{algo: num -- spectral solver linear part} with a stepsize \( \Delta t/2 \).
            \For{ \( x \in \textsc{Grid} \) } 
                \Comment{Add interactions.}
                \State Multiply \( \eta(x) \) by \( e^{-i\,\Delta t\, |\eta(x)|^2} \).
            \EndFor
            \State Use Algorithm \ref{algo: num -- spectral solver linear part} with a stepsize \( \Delta t/2 \).
        \EndFor
    \end{algorithmic}
\end{algorithm}




Of course, in pratical applications one is not able to define a grid over \( \mathbb{R}^d \). 
Hence, Algorithms \ref{algo: num -- spectral solver linear part} and \ref{algo: num -- spectral solver} have to be modified by defining \textsc{Grid} as a discretization of a finite-volume subset of \( \mathbb{R}^d \), typically a product of intervals in each dimension.
For all of our numerical examples, this will \( [-15, 15]\times[-15, 15] \), discretized using \( N_x\times N_y \) points.
In order to have an easily computable FFT, one has to use a spatial uniform grid, which then defines the \textsc{Fourier Grid}.
Special care has to be paid when choosing the number of points: if we have Fourier frequencies larger than the \emph{Nyquist frequency}, then we will observe a phenomenon known as \emph{aliasing}. 
This may not be problematic for the Harmonic Oscillator \eqref{eqn: cNLS with psi -- linear part} depending on the initial condition, but will eventually become an issue when simulating \eqref{eqn: cNLS with psi} because it involves interactions and hence an infinite number of frequencies.
Moreover, by using a FFT-based algorithm we implicitely impose periodic boundary conditions.





\subsection{Discretization into a sum of Bubbles}

We need to decompose any arbitrary function into a finite sum of \( N \) bubbles.
A solution to this question has been proposed in \cite{qianFastGaussianWavepacket2010}, but it involves integrals over the whole phase space, which is something we want to avoid.

We could also use a nonlinear least squares approach, but our experimental results showed that it tends to yield spread out gaussians, which may present huge overlaps between them. The overlaps cause issues with the DFMP, for instance a blow-up of the conservative quantities. This has been observed during our experiments but the results are not reported in this paper.
The issue of discretizing an arbitrary function into a sum of bubble without too much overlaps is not the main concern of this paper, hence we will use a visual trial-and-error discretization. Another possible way of discretizing the initial data is outlined in \cite{adamowiczLaserinducedDynamicAlignment2022}.




\subsection{Observables}

In order to compare the bubbles scheme against the spectral method, we compare them in absence of interactions, i.e. on the Harmonic Oscillator \eqref{eqn: cNLS with psi -- linear part}, as well as in the presence of nonlinear interactions, i.e. on \eqref{eqn: cNLS with psi}.
We showed in Lemma \ref{lemma: conserved quantities in HO} the conservation of some quantities for \eqref{eqn: cNLS with psi -- linear part} and \eqref{eqn: cNLS with psi}, we will focus on mass, energy and momentum.
When computing the observables for the spectral solution, we noted that the approximation of the gradient using finite differences with periodic boundary conditions yielded very rough results while the gradient approximation using the Fast Fourier Transform gave more accurate results. We use the latter approximation in the Figures of Section \ref{subsect: num results}.
In the case of bubbles, we compute every integral by hand thanks to the assumption \( v(s,y) = e^{-\frac{|y|^2}{2}} \), some details are given in Appendix \ref{appendix: Miscellaneous computations}. When reporting the results in the following \( \log \)-plots, all values with an amplitude smaller than \( 10^{-16} \) were set to be equal to \( 10^{-16} \).


For all of the results shown, the spectral scheme is supplied with the exact initial condition and not a projection on the grid of the bubbles discretization.





\subsection{Results}
\label{subsect: num results}


We consider examples adapted from \cite{baoNumericalSolutionGross2003}.


\subsubsection{Test case 1: Zero phase initial data}

The initial condition reads 
\begin{equation}
    \psi(t=0, x) = \pi e^{-\frac{|x-\mu_1|^2}{2} } + 2e^{-\frac{|x-\mu_2|^2}{2} }, \quad x\in \mathbb{R}^2,\quad \mu_1 = (0, 2), \ \mu_2 = (1, 0)
.\end{equation}
%




\begin{figure}
    \begin{subfigure}{\textwidth}
        \centering
        \inputtikz{1}{bubblesVSspectral_conservative_quantities_dt-0.001_coeffs-1.0-0.0_testcase1_Nx128_Ny129}
        \caption{\centering Approximate solution to the Harmonic Oscillator \eqref{eqn: cNLS with psi -- linear part}.}
    \end{subfigure}
    \begin{subfigure}{\textwidth}
        \centering
        \inputtikz{1}{bubblesVSspectral_conservative_quantities_dt-0.001_coeffs-1.0-1.0_testcase1_Nx128_Ny129}
        \caption{\centering Approximate solution to the Schrödinger equation \eqref{eqn: cNLS with psi} using DFMP Algorithm.}
    \end{subfigure}
    \caption{\centering Test case 1. Relative evolution of mass, energy and momentum with bubbles and spectral methods. \( \Delta t = 10^{-3} \). Time-integrator for the nonlinear part of the splitting: Runge-Kutta of order 4. Spectral scheme with \( N_x = 128, N_y = 129 \).}
    \label{fig: num -- test case 1}
\end{figure}




% \begin{figure}
%     \centering
%     \inputtikz{0.4}{courbe_cv_Nmin=16_Nmax=2048_testcase=1_HO=false_T=1.0_dt=0.005}
%     \caption{\centering Test case 1. Evolution of the \( L^2 \) norm of the difference between spectral and bubble schemes against the number of points in each direction \( N_x = N_y \), at time \( T = 1 \) with \( dt=5\cdot 10^{-3} \), relative w.r.t. the exact \( L^2 \) norm of the discretized initial data.}
%     \label{fig: num -- test case 1 - evolution of relative l2 norm}
% \end{figure}




The results are displayed in Figure \ref{fig: num -- test case 1}.
The solution approximated with the DFMP approach globally outperforms the spectral method on both the Harmonic Oscillator and the cubic NonLinear Schrödinger equations.
On the Harmonic Oscillator, the solution obtained with the Bubbles scheme is about one order of magnitude better than the spectral scheme. When we compare them on \eqref{eqn: cNLS with psi}, i.e. when adding interactions, the \( \mathbb{L}^2 \) norm is better conserved for the spectral scheme, but the other conservative quantities are better conserved on a long time for the Bubbles scheme.

The ``jumps'' in the DFMP approach may be explained by an ill-conditioned Gram matrix, which would then yield a very rough approximation of the modulation parameters.






\subsubsection{Test case 2: Weak interactions}

The initial condition reads 
\begin{equation}
    \psi(t=0, x) = e^{-|x - \mu_3|^2} e^{i \cosh |x - \mu_3|}, \quad x\in \mathbb{R}^2, \quad \mu_3 = (1, 1)
.\end{equation}
%

The approximation of this function as a sum of bubbles is pretty straightforward: we know that for \( x \) small, \( \cosh x \approx 1 + \frac{x^2}{2}  \), hence 
\begin{equation*}
    \psi(t=0, x) \approx e^{-|x - \mu_3|^2} e^{i + i\frac{|x - \mu_3|^2}{2} }, \quad x\in \mathbb{R}^2
.\end{equation*}
%



\begin{figure}
    \begin{subfigure}{\textwidth}
        \centering
        \inputtikz{1}{bubblesVSspectral_conservative_quantities_dt-0.001_coeffs-1.0-0.0_testcase2_Nx128_Ny129}
        \caption{\centering Approximate solution to the Harmonic Oscillator \eqref{eqn: cNLS with psi -- linear part}.}
    \end{subfigure}
    \begin{subfigure}{\textwidth}
        \centering
        \inputtikz{1}{bubblesVSspectral_conservative_quantities_dt-0.001_coeffs-1.0-1.0_testcase2_Nx128_Ny129}
        \caption{\centering Approximate solution to the Schrödinger equation \eqref{eqn: cNLS with psi} using DFMP Algorithm.}
    \end{subfigure}
    \caption{\centering Test case 2. Relative evolution of mass, energy and momentum with bubbles and spectral methods. \( \Delta t = 10^{-3} \). Time-integrator for the nonlinear part of the splitting: Runge-Kutta of order 4. Spectral scheme with \( N_x = 128, N_y = 129 \).}
    \label{fig: num -- test case 2}
\end{figure}





% \begin{figure}
%     \centering
%     \inputtikz{0.4}{courbe_cv_Nmin=16_Nmax=2048_testcase=2_HO=false_T=1.0_dt=0.005}
%     \caption{\centering Test case 2. Evolution of the \( L^2 \) norm of the difference between spectral and bubble schemes against the number of points in each direction \( N_x = N_y \), at time \( T = 1 \) with \( dt=5\cdot 10^{-3} \), relative w.r.t. the exact \( L^2 \) norm of the discretized initial data.}
%     \label{fig: num -- test case 2 - evolution of relative l2 norm}
% \end{figure}


The results are displayed in Figure \ref{fig: num -- test case 2}.
This example shows the performance of the DFMP approach in its most efficient setting: it only has one bubble. This explains the very good conservation results obtained: the Bubbles scheme outperforms the spectral scheme on both \eqref{eqn: cNLS with psi -- linear part} and \eqref{eqn: cNLS with psi}, except for the energy on \eqref{eqn: cNLS with psi}. However, even in this case, the error of the DFMP method remains generally less than one order of magnitude larger than the error from the spectral method.








\subsubsection{Test case 3: Strong interactions}

The initial condition reads 
\begin{equation}
    \psi(t=0, x) = 
    \begin{cases}
        \sqrt{M^2 - |x|^2} e^{i\cosh \sqrt{x_1^2 + x_2^2}}, & |x|^2 < M^2 \\
        0 & \text{otherwise}
    \end{cases}, \quad M = 4.
\end{equation}
%

We apply the same approximation for the complex exponential as previously explained, and use a ``visual trial-and-error'' discretization of the square root. It yields a number of \( N=9 \) bubbles. 
We emphasize the fact that this discretization may be far from being the best one achievable, however the process of discretizing an arbitrary function into a sum of bubbles is not the main concern of this paper. The discretization of the initial square root is given in Figure \ref{fig: num -- approximation of sqrt with bubbles}.


\begin{figure}
    \centering
    \inputtikz{0.7}{approx_sqroot_with_bubbles}
    \caption{\centering Approximation of \( x\mapsto \sqrt{M^2 - |x|^2} \) as a sum of bubbles}
    \label{fig: num -- approximation of sqrt with bubbles}
\end{figure}



\begin{figure}
    \begin{subfigure}{\textwidth}
        \centering
        \inputtikz{1}{bubblesVSspectral_conservative_quantities_dt-0.001_coeffs-1.0-0.0_testcase3_Nx128_Ny129}
        \caption{\centering Approximate solution to the Harmonic Oscillator \eqref{eqn: cNLS with psi -- linear part}.}
    \end{subfigure}
    \begin{subfigure}{\textwidth}
        \centering
        \inputtikz{1}{bubblesVSspectral_conservative_quantities_dt-0.001_coeffs-1.0-1.0_testcase3_Nx128_Ny129}
        \caption{\centering Approximate solution to the Schrödinger equation \eqref{eqn: cNLS with psi} using DFMP Algorithm.}
    \end{subfigure}
    \caption{\centering Test case 3. Relative evolution of mass, energy and momentum with bubbles and spectral methods. \( \Delta t = 10^{-3} \). Time-integrator for the nonlinear part of the splitting: Runge-Kutta of order 4. Spectral scheme with \( N_x = 128, N_y = 129 \).}
    \label{fig: num -- test case 3}
\end{figure}



% \begin{figure}
%     \centering
%     \inputtikz{0.4}{courbe_cv_Nmin=16_Nmax=2048_testcase=3_HO=false_T=1.0_dt=0.005}
%     \caption{\centering Test case 3. Evolution of the \( L^2 \) norm of the difference between spectral and bubble schemes against the number of points in each direction \( N_x = N_y \), at time \( T = 1 \) with \( dt=5\cdot 10^{-3} \), relative w.r.t. the exact \( L^2 \) norm of the discretized initial data.}
%     \label{fig: num -- test case 3 - evolution of relative l2 norm}
% \end{figure}


The results are displayed in Figure \ref{fig: num -- test case 3}.
This example is by far the most interesting of the three test cases presented in this paper, because it shows that with the discretization given in Figure \ref{fig: num -- approximation of sqrt with bubbles} the conservation properties are pretty good for the Bubbles scheme, even when there are a lot of interactions between bubbles. 
The spectral scheme is globally outperformed by the Bubbles scheme, except for the \( \mathbb{L}^2 \) norm in the presence of interactions, which is better conserved by the spectral scheme. Even in this case, the conservation of this quantity with the DFMP is relatively correct.

The ``jumps'' in the relative evolution of conservative quantities may be explained by an ill-conditioned Gram matrix in DFMP. It also has to be noted that if the discretization presents too much overlap between the gaussian functions, then the DFMP approach fails and the conservative quantities blow up: this has been observed with other discretizations of the same initial data, and is not reported here.



\section{Analysis of the SYK-Inspired Code}
\label{sec:code_structure}

We now consider in more detail the properties of the SYK inspired code with $k=r^a$ and $L=a+b$ for two integers $a$ and $b$. Recall that the total number of qudits is 
\begin{equation}
    N = r^a + r^{a+b},
\end{equation}
and the ratio between ground state qudits and the total number of qudits (i.e. the rate) is therefore
\begin{equation}
    \frac{k}{N} = \frac{1}{1 + r^b},
\end{equation}
which is independent of $a$. The thermodynamic limit $a \rightarrow \infty$ gives a family of codes with non-zero rate. We already established in Section~\ref{sec:numerical} that this code can be highly entangled. It is also interesting to consider its complexity.

In this case, the complexity sum \eqref{eq:circuit_complexity} can be rewritten as
\begin{equation}
    \text{total gates} = \frac{D}{q \cdot (1 + r^b)} \cdot N \log_r N + \mathcal{O}(N).
\end{equation}
This leading $N \log N$ scaling with the total number of degrees of freedom can be compared to holographic complexity conjectures applied to JT gravity~\cite{jt_complexity}; one also gets $N \log N$ by studying, for example, the volume (length) of the wormhole dual to the thermofield double state with temperature of order $1/N$. The key point is that the throat of the wormhole is long, of order $\log N$, at this temperature. Hence, the circuit complexity of our SYK-inspired encoding also resembles that obtain from holographic models dual to SYK.

For the estimates discussed below, we continue to assume that the layers are composed of random 2-qudit Clifford gates applied to random pairs of qudits. We caution that this is certainly not correct for the actual SYK model: the gates must act on fermionic degrees of freedom and will not be Clifford (or the fermionic analogue of Clifford) generically. Here we continue to focus on the Clifford case for ease of analysis and for its interpretation in terms of an exact quantum error correcting code. Below we comment briefly on the potential similarities and differences with the actual SYK model.

\subsection{Distance Estimate and Stabilizer Weights}
\label{sec:syk_distance_weight}

We know the rate of our SYK-inspired code. To estimate the distance, we need to understand how logical operators grow as they pass from the IR to the UV. Let us assume that a typical operator grows in size by a factor of $g^D$ after passing through one layer (i.e. being conjugated by that layer unitary), up to a maximum size set by the total number of qudits. A way to estimate $g$ when the layer unitary is a random Clifford circuit can be found in appendix \ref{sec:growth_factor}. At the same time, the number of qudits is also growing, going from $k+r^{\ell-1}$ to $k+r^\ell$. The distance depends on whether the size of operators grows faster or slower than the number of qudits. Note that we saw already a manifestation of this competition in the discussion in Section~\ref{sec:numerical}; here we explain in more detail the issues.

From a given random circuit layer, we expect operators to grow by a factor of $g^D$ provided they are not close to maximum weight. If they are close to maximum weight, then they will grow by a reduced factor. We must compare this operator growth to the rate of qudit increase. The ratio $R_\ell$ between the number of qudits in successive layers is 
\begin{equation}
      R_\ell =  \frac{k+r^\ell}{k+r^{\ell-1}} = 1 + \frac{r-1}{1 + k/r^{\ell-1}}, 
\end{equation}
which monotonically increases with $\ell$. As logical operators evolve from layer to layer into the UV, the relative weight of the operator either increases or decreases depending on whether $g^D > R_\ell$ or $g^D < R_\ell$. The dynamics of this process, iterated over all $L$ layers, gives an estimate for the size of non-trivial logical operators.

\subsubsection{Warmup: Small Fixed $k$}

To illustrate the key competition, consider first the case in which $k$ is small and fixed. In this case, the ratio $R_\ell \rightarrow r$ as $\ell$ increases, so most of the evolution corresponds to a fixed ratio of $r$. In terms of the parameters above, we can achieve this regime by taking $b$ large at fixed $a$.

Suppose $g^D > r$. Then operator growth is the fastest process and logical operators will reach saturation. In this case, we expect the distance to be linear in $N$. The distance will not exactly saturate the singleton bound, but it may come close for large $D$.

Now suppose $g^D < r$. In this case, we are adding qudits faster than operators can grow, so the logical operators are ultimately supported on a dilute fraction of all the sites. Indeed, the size of a typical logical operator will be $g^{DL}$, whereas the total number of qudits is $N =  r^L (1 + r^{-b}) \approx r^L$. Expressed in terms of $N$, the size of a typical logical operator is
\begin{equation}
    g^{DL} \sim N^c
\end{equation}
where $c = \frac{\ln g^D}{\ln r} < 1$. Hence, we expect a distance that scales as a sublinear function of $N$.

\subsubsection{SYK-Like Scaling}

Now we turn to the case where $k = r^a$ is large and $b$ is fixed. Here, when $\ell$ is small, the $R_\ell$ ratio is close to one and the number of qudits is barely increasing from layer to layer. In this regime, operator growth is completely dominant. In contrast, at the most UV layer, where $\ell=L=a+b$, the ratio is
\begin{equation}
   R_L = 1 + \frac{r-1}{1+r^{1-b}} < r. 
\end{equation}

Suppose $g^D > R_L$. Then operator growth always dominates over qudit growth. However, because the initial number of qudits (the ground state qudits) is large, we still have to compare the total operator size, $g^{D L}$, to the total number of qudits, $N = r^L ( 1+ r^{-b})$. We see again that if $g^D >r$, then this naive estimate gives an operator weight larger than $N$, meaning that the operator growth actually saturated at something proportional to $N$. If $g^D < r$, then we are again in the situation where $g^{DL} \sim N^c$.


Suppose $g^D < R_L$. Then there will be some layer $\ell^*$ such that operator growth and qudit growth switch dominance as $\ell$ increases through $\ell^*$. We may approximately determine this crossover scale from
\begin{equation}
    g^D = R_{\ell^*},
\end{equation}
noting that this $\ell^*$ is not typically an integer. In the thermodynamic limit $a \rightarrow \infty$, we must have $\ell^* = a + b^*$ for some constant $b^*$ since the ratio $R_\ell$ is essentially unity until $r^{\ell}$ is comparable to $k$.

Now between $\ell=1$ and $\ell = \ell^*$, logical operators will grow faster than the number of qudits. Assuming they don't reach saturation, they will grow by roughly a factor of $g^{D\ell^*}$. By contrast, the number of qudits at layer $\ell^*$ is
\begin{equation}
    n_{\ell^*} = r^a (1 + r^{b^*}),
\end{equation}
so the ratio of operator size to number of qudits is
\begin{equation}
    \left( \frac{g^D}{r} \right)^a \frac{g^{D b^*}}{1+r^{b^*}}.
\end{equation}
This ratio vanishes as $a\rightarrow \infty$ since we are assuming that $g^D < R_L$ and $R_L < r$. Hence, $g^{D\ell^*} \sim (n_{\ell^*})^c$ as above.

There are a fixed number of layers from $\ell^*$ to $L$ since $b$ and $b^*$ are fixed as $a\rightarrow \infty$. Therefore operators and the number of qudits grow by an additional factor independent of $N$ from $\ell^*$ to $L$. Hence, the scaling of $g^{DL}$ with $N$ is the same as the scaling of $g^{D\ell^*}$ with $n_{\ell^*}$, that is $g^{DL} \sim N^c$.




\subsubsection{Stabilizer Weights}

We expect that the stabilizer weights will display a similar pattern as in Figure~\ref{fig:stab_weights}. In particular, a non-zero fraction of all the stabilizers will have constant weight. These arise from the UV most layer. Then as we descend in the network towards the IR, there are fewer stabilizers but of increasing weight. In particular, there are at least a few stabilizers of very high weight, similar to the weight of logical operators. 

\subsection{Comparison to SYK} 

We now compare features of the SYK-inspired code to those of the actual SYK model. To be precise, we will compare a particular realization of the SYK Hamiltonian (with $q=4$), $H_{\text{SYK}}$, with a particular realization of the toy code Hamiltonian, $H_{\text{code}}$, for the SYK-inspired code (see Section~\ref{sec:arch}). (It is also interesting to consider supersymmetric generalizations~\cite{Fu_2017}.)
\begin{itemize}

   \item{[Hamiltonian structure]} $H_{\text{SYK}}$ is composed of $O(N^4)$ weight-$4$ fermion terms (all possible such terms). These terms do not all commute and they enter $H_{\text{SYK}}$ with random coefficients. $H_{\text{code}}$ is composed of $O(N)$ commuting terms with fixed coefficients. The weight of the terms varies, with many having low-weight but a significant fraction having high weight, comparable to the distance of the corresponding code.
   
    \item{[Ground space]}  $H_{\text{SYK}}$ has $e^{s_0 N}$ approximate ground states which are approximately degenerate with level spacing $e^{-\alpha N}$. $\alpha$ and $s_0$ are constants, independent of $N$. Similarly, $H_{\text{code}}$ has $d^k = e^{ \frac{\ln d}{1+r^b} N}$ exactly degenerate ground states.
    
    \item{[Low temperature thermodynamics]} The SYK model has a low temperature heat capacity proportional to temperature $T$. Similarly, the parameters of $H_{\text{code}}$ can be chosen so that its low temperature heat capacity is proportional to $T$.
    
    \item{[Fine-grained spectrum]} The fine-grained energy spectrum of $H_{\text{SYK}}$ is random-matrix-like~\cite{Cotler_2017,saad2019semiclassical}. The fine-grained energy spectrum of $H_{\text{code}}$ is not random-matrix-like because $H_{\text{code}}$ is a commuting projector Hamiltonian.
    
    \item{[Entanglement]} Both models feature energy eigenstates with volume-law entanglement. The entanglement spectrum will, however, be quite different between the two kinds of states. In particular, eigenstates of $H_{\text{code}}$, being stabilizer states, have a flat entanglement spectrum.
    
    \item{[Complexity]} We only have estimates here. Using the duality to JT gravity and holographic complexity/geometry conjectures, the circuit complexity of the SYK approximate ground states is estimated to be $O(N \ln N)$. We have an explicit estimate (and upper bound) of $O(N \ln N)$ for circuit complexity of the ground space of $H_{\text{code}}$.
    
 
\end{itemize}

The many similarities between $H_{\text{SYK}}$ and $H_{\text{code}}$ are the basis for our conjecture that the architecture in Figure~\ref{fig:arch} has the potential to describe the physics of the SYK model once the tensors in the network have been adapted to a particular SYK instance, for example, using a variational approach. However, there are also crucial differences between the two. Two that stand out are the different scalings of the weights of Hamiltonian terms with system size and the exact versus approximate nature of the ground state degeneracy. The fine-grained energy spectrum is also very different in the two cases. Thus, it will be informative in the future to explore our network architecture as a variational ansatz for the SYK ground space.

\subsection{SYK Ground Space as an Approximate Code}

Here we want to comment on another possibility raised by the similarities above. For $H_{\text{code}}$, we have seen explicitly that the ground space can be viewed as an error correcting code with constant relative distance and constant rate (provided $D$ is big enough). In particular, it is an exact stabilizer code. This naturally raises the possibility that the approximate ground space of the SYK model could have interesting properties as an approximate quantum error correction code.\footnote{The network architecture presented here was first considered by one of the authors in fall 2019 during their stay at the Institute for Advanced Study and later presented in preliminary form, along with the potential code interpretation, in January 2020 at UCSB. Independently, the code properties of SYK in the thermal regime have been studied~\cite{Chandrasekaran_2022}.}

Thus we consider a code defined by the full approximate ground space of some particular $H_{\text{SYK}}$ realization. By construction this code has a constant rate as $N\rightarrow \infty$ which is given by ground state entropy density $s_0$. This code is not a stabilizer code, but it does have a sort of ``low weight'' definition via the SYK Hamiltonian. 

What is not immediately clear is the distance of this code. Moreover, since the code is approximate, we must specify precisely what we mean by the distance. We will defer a full discussion to a future work, but here let us note that if the architecture in Figure~\ref{fig:arch} does indeed provide a good approximation to the ground space of the SYK realization, then the same kind of scaling analysis discussed above for the random Clifford code would also provide an estimate for the operator size of logical operators.

In this case, it would be important to understand the analog of $r$ and $g^D$ in the SYK case. As one approach, we could fix $r=2$ and then adjust the layer circuits so that we get a good approximation to the ground space. The parameter $g^D$ would then be determined by the properties of these circuit. A simple random operator growth model may be too crude to capture the detailed physics, but continuing with this estimate for now, if the resulting $g^D$ were greater than $r$, then we have logical operators of weight proportional to $N$ and potentially distance proportional to $N$. Alternatively, if $g^D < r$, then the distance could be some power of $N$, $N^c$. It would be interesting to understand which of two cases is realized; this should be related to the spectrum of the scaling dimensions in the theory since these are related to the mixing properties of the scaling superoperator~\cite{PhysRevLett.101.110501}. Given the relatively low scaling dimension of the fermion operators, it may be that one is effectively in the $g^D<r$ regime.

\subsection{Summary}

We gave analytical estimates of the distance for a family of SYK-inspired codes in the thermodynamic limit of many qudits. This code family shares a number of similarities with known properties of the actual SYK model, although there are crucial differences as well. Viewing the approximate ground space of SYK as an approximate quantum code, the analysis of the SYK-inspired model suggests that the actual SYK ground space code, which has constant rate as $N\rightarrow \infty$, could have a distance $N^c$ for some constant $0<c\leq 1$.


%%%%%%%%%%%%%%%%
\section{Outlook}
\label{sec:Out}
%%%%%%%%%%%%%%%%

In this review we have attempted to present the main ideas behind the hypothesis
that the applicability of fluid dynamics to early phases of QGP dynamics can be
explained by a far-from-equilibrium hydrodynamic attractor. We have emphasised
the role of the kinematic setting specific to heavy-ion collisions, which makes
it plausible that such an attractor occurs also in QCD.  At the conceptual level
this picture is rather compelling. However, Bjorken flow is a very restrictive
setting, and despite some existing applications using the attractor in practical
calculations of phenomenologically interesting observables requires developing
effective methods of working with models with a large number of degrees of
freedom. Progress is likely to come gradually, by learning how to deal with
models of increasing complexity, extending the studies reviewed in Sections
\ref{sec:PhaseSpace} and \ref{sec:beyond}. 

The idea of hydrodynamic attractors has been closely connected with the
divergence of the hydrodynamic gradient expansion. In this review we have not
discussed this connection beyond its utilitarian aspects, but on a conceptual
level there have been some important developments in recent times. This includes
a proof of the generic divergence of the gradient expansion at the linearised
level without any symmetry assumptions, and its connection with the properties
of dispersion relations~\cite{Withers:2018srf,Heller:2020uuy}.  At the nonlinear
level, some of the results for Bjorken flow have been generalised to a much
wider class of flows, called longitudinal flows. In particular, the gradient
expansion has been shown to diverge for this class of
solutions~\cite{Heller:2021oxl}. It was also found that the large order
behaviour of the gradient series can be expressed in terms of new degrees of
freedom, the singulant fields, which track transient
effects~\cite{Heller:2021yjh}. The relevance of these advances to the study of
attractors remains an interesting challenge for the future. 

A number of issues were not addressed in this review. One is the presence of
other degrees of freedom, such as those  connected with chiral symmetry
breaking, and their possible effect on the attractor dynamics of
QGP~\cite{Mitra:2020mei,Mitra:2020hbj} (see also Ref.~\cite{Mitra:2022uhv}).
Another such issue is the role of fluctuations, which has been mostly neglected
in the attractor literature, with the notable exception of
Refs.~\cite{Akamatsu:2016llw,Chen:2022ryi}. 

It would also be very interesting to understand the role of quantum effects in
the early-time dynamics. Of course the hydrodynamic picture implicitly contains
them, but in a rather opaque way. On the other hand, the kinetic theory
description arises from quantum field theory through a series of
approximations~\cite{Mueller:2002gd,Jeon:2004dh} and it should be possible to
study the origin and robustness of the kinetic theory attractor in a framework
which allows for a systematic incorporation of quantum corrections. This is
connected with other approaches to early-time dynamics, including those
involving ideas such as the Color Glass Condensate or non-thermal
attractors~\cite{Berges:2008wm,Mazeliauskas:2018yef,Brewer:2019oha,Brewer:2022ifw,Brewer:2022vkq,Berges:2020fwq}.
It is not yet fully understood how they are related to the ideas reviewed here,
and clarifying this appears to be a very promising avenue for future research. 

Finally, there is the more general question about far-from-equilibrium
attractors in nonequilibrium systems.  In the context of heavy-ion collisions
the specific kinematic circumstances have lead us to consider boost-invariant
expansion where far-from-equilibrium hydrodynamic behaviour was first noted, but
there could be other situations where analogous phenomena might
appear~\cite{Baggioli:2021tzr}, perhaps even in the non-relativistic
domain~\cite{Le:2022ntg}.  Another context where far from equilibrium
hydrodynamic attractors can occur is the dynamics of systems in nontrivial
spacetime backgrounds (see e.g. Ref.~\cite{Vyas:2022hkm}). 




\bibliographystyle{jhep}
\bibliography{refs,codes}

\appendix

\section{Scaling of the Stabilizer Entropy}
\label{sec:entropy_scaling}

\subsection{The Stabilizer Hamiltonian}

Every set of stabilizers fixing a quantum state or a space of quantum states can be expressed as a projective Hamiltonian that has said states as part of its (degenerate) ground space.

Let $N$ be the total number of physical qudits and $k \leq N$ the number of logical qudits needed to represent the code word $\ket{\psi}_{\text{code}}$. The case of $N = k$ is not interesting to us so we assume we have $N - k > 0$ ancillary qudits. Every quantum code can then be written as a unitary $U$ satisfying
\begin{equation}
    \ket{\Psi} = U \left( \ket{\psi}_{\text{code}} \otimes \ket{0}_{\text{anc}}^{\otimes \, (N - k)} \right),
\end{equation}
where $\ket{\Psi}$ is the code word encoded in the space of physical qudits. The ancillary thermal qudits can each be fixed to be $\ket{0}$ without loss of generality.

Before applying the encoding unitary, it is easy to see that the generating set of stabilizers fixing the code space spanned by all possible choices of $\ket{\psi}_{\text{code}} \otimes \ket{0}_{\text{anc}}^{\otimes \, (N-k)}$ is given by
\begin{equation}
\label{eq:pre_code_stab}
    Z_i \equiv I_{\text{code}} \otimes I^{\otimes \, (i-1)} \otimes Z \otimes  I^{\otimes \, (N-k-i)}, \quad i = 1,\ldots,N-k, 
\end{equation}
where the $Z$ acts on the $i$th qudit of the ancillary system. From this it immediately follows that the stabilizers acting on the physical qudits can be retrieved by applying the code unitary such that
\begin{equation}
    \widetilde{Z}_i = U \, Z_i \, U^{\dagger}.
\end{equation}

To construct the stabilizer Hamiltonian though, we have to use the (disjoint) projectors associated to our chosen stabilizer basis. Analogously to the previous case, before encoding the state they are
\begin{equation}
\label{eq:pre_code_proj}
    P_i \equiv I_{\text{code}} \otimes I^{\otimes \, (i-1)} \otimes \ket{0}\bra{0} \otimes I^{\otimes \, (N-k-i)}, \quad i = 1,\ldots,N-k, 
\end{equation}
and after the encoding they become
\begin{equation}
    \widetilde{P}_i = U \, P_i \, U^{\dagger}.
\end{equation}

With that, the general stabilizer Hamiltonian has the form of
\begin{equation}
\label{eq:stab_hamil}
    H = - \sum_{i = 1}^{N-k} J_i \cdot \widetilde{P}_i, \quad J_i > 0
\end{equation}
The coefficients $J_i$ can be arbitrarily chosen and determine the energy scales of the system, but since they are necessarily positive-definite, this do not affect the space of ground states i.e. the space of valid physical qudit states. Excitations away from a ground states then correspond to errors being present in the state, which is because of the one-to-one relation between projectors and stabilizers.

\subsection{General Thermodynamic Quantities}

Using the Hamiltonian derived in the previous section, we can now compute the associated Gibbs state
\begin{equation}
    \rho_{\beta} = \frac{1}{Z} e^{-\beta H}, \quad Z = \Tr[e^{-\beta H}]
\end{equation}
and some of its properties, including the entropy.

First, it is straightforward to show that
\begin{align}
\begin{split}
\label{eq:gibbs_exp}
    e^{-\beta H} &= \exp\left( \beta \sum_{i = 1}^{N-k} J_i \cdot \widetilde{P}_i \right) \\
    &= \prod_{i = 1}^{N-k} \exp\left(\beta J_i \cdot \widetilde{P}_i\right) \\
    &= \prod_{i = 1}^{N-k} \left[ \sum_{n=0}^{\infty} \frac{1}{n!} \left(\beta J_i \cdot \widetilde{P}_i \right)^n \right] \\
    &= \prod_{i = 1}^{N-k} \left[ I + \sum_{n=1}^{\infty} \frac{1}{n!} \left(\beta J_i\right)^n \cdot \widetilde{P}_i\right] \\
    &= \prod_{i = 1}^{N-k} \left[ I + \left(e^{\beta J_i} - 1 \right) \cdot \widetilde{P}_i\right] \\
    &= \sum_{n=0}^{N-k} \, \sum_{1 \leq i_1 < \ldots < i_n \leq N-k} \, \prod_{\{i_a\}} \left( e^{ \beta J_{i_a}} - 1\right)\cdot \widetilde{P}_{i_a},
\end{split}
\end{align}
where in the last line we used a generalization of the binomial theorem and the fact that the projection operators commute by definition. Computing the partition function $Z$ using the final expression in \eqref{eq:gibbs_exp} can be done in the following way:
\begin{align}
\begin{split}
    Z &= \Tr[e^{-\beta H}] \\
    &= \sum_{n=0}^{N-k} \, \sum_{1 \leq i_1 < \ldots < i_n \leq N-k} \, \prod_{\{i_a\}} \left( e^{ \beta J_{i_a}} - 1 \right) \cdot \Tr \bigg[ \prod_{\{i_a\}}\widetilde{P}_{i_a} \bigg] \\
    &= \sum_{n=0}^{N-k} \, \sum_{1 \leq i_1 < \ldots < i_n \leq N-k} d^{N-n} \cdot \prod_{\{i_a\}} \left( e^{ \beta J_{i_a}} - 1 \right) \\
    &= d^{k} \cdot \sum_{n=0}^{N-k} \, \sum_{1 \leq i_1 < \ldots < i_n \leq N-k} d^{N-k-n} \cdot \prod_{\{i_a\}} \left( e^{ \beta J_{i_a}} - 1 \right) \\
    &= d^k \cdot \prod_{i=1}^{N-k} \left( e^{ \beta J_i} + d - 1 \right).
\end{split}
x\end{align}
Note that in the second line we used the definition \eqref{eq:pre_code_proj} for the projection operators, which implies that $\Tr[\widetilde{P}_{i_1} \cdots \widetilde{P}_{i_n}] = d^{N-n}$ given that none of the indices $i_a$ coincide. Going from the penultimate line to the last one we then again applied the generalized binomial theorem.

From the partition function it is then easy to determine all other thermodynamic quantities, of which the most important one for us is the von-Neumann entropy
\begin{align}
\begin{split}
    S &\equiv - \Tr[\rho_{\beta} \log(\rho_{\beta})] \\
    &= \beta \cdot \braket{E}_{\beta} + \log(Z) \\
    &= (1 - \beta \cdot \partial_{\beta}) \log(Z),
\end{split}
\end{align}
where the second and third lines are well-known equivalent expressions and we assume $\log$ to refer to the natural logarithm. Therefore, by using the fact that
\begin{align}
    \log(Z) &= k \cdot \log(d) + \sum_{i=1}^{N-k} \log\left( e^{ \beta J_i} + d - 1 \right), \\
    - \beta \cdot \partial_{\beta} \log(Z) &= - \sum_{i=1}^{N-k} \frac{\beta J_i \cdot e^{\beta J_i}}{e^{\beta J_i} + d - 1},
\end{align}
and after doing some rearranging, we arrive at
\begin{equation}
    S_{\text{stab}} = \left( k \log(d) + \sum_{i=1}^{N-k} p_i \log(d-1) \right) + \sum_{i=1}^{N-k} S(p_i),
\end{equation}
where
\begin{equation}
    S(p_i) \equiv - p_i \cdot \log(p_i)) - (1-p_i) \cdot \log(1-p_i)
\end{equation}
is the binary Shannon entropy associated to the probability distributions $\{p_i, 1-p_i\}_i$ which are defined in terms of
\begin{equation}
\label{eq:shannon_prob}
    p_i \equiv \frac{d-1}{e^{\beta J_i} + d - 1} \in \left(0, \frac{d-1}{d}\right).
\end{equation}
Note that in the case of qubits ($d = 2$), $p_i$ is the Fermi-Dirac distribution associated to $J_i$. Hence we can interpret the sum in the leading term as an occupation number such that
\begin{equation}
    \braket{N-k} \equiv \sum_{i=1}^{N-k} p_i, \quad S_{\text{stab}} = \log\left(d^k \cdot (d-1)^{\braket{N-k}}\right) + \sum_{i=1}^{N-k} S(p_i).
\end{equation}
Ignoring that leading term, the total entropy of the Gibbs ensemble therefore decouples into a sum of entropies associated with each energy level $J_i$ and therefore each element of the stabilizer basis \eqref{eq:pre_code_stab}. This is not unexpected though, as each term in the stabilizer Hamiltonian \eqref{eq:stab_hamil} commutes with every other one, making the system completely diagonalizable.

\subsection{Entropy Scaling for the NoRA Model}

So far all the calculations we did hold for error-correcting stabilizer codes in general. To actually get some results unique to the NoRA network discussed in this paper, we have to make some assumptions about the distribution of energy levels $J_i$.

One obvious such assumption is that the level distribution should only depend strongly on the layer $\ell$ at which associated stabilizer elements are first acted on in a non-trivial way by the encoding unitary. Hence we move from $J_i$ to $J_{\ell}$ (and therefore from $p_i$ to $p_{\ell}$),  ignoring (for now) that the energy might actually vary slightly for different stabilizers at the same level. Because of this the expression for the entropy becomes
\begin{equation}
\label{eq:level_entr_discrete}
    S_{\text{stab}} = \log\left(d^k \cdot (d-1)^{\braket{N-k}}\right) + \sum_{\ell=1}^{L} \Delta n_{\ell} \cdot S(p_{\ell}),
\end{equation}
where $n_{\ell}$ is the number of stabilizer basis elements with the same associated energy level:
\begin{align}
\begin{split}
\label{eq:stab_distribution}
    \Delta n_{\ell=1} &= r, \\
    \Delta n_{\ell > 1} &= r^{\ell} - n_{\ell - 1} = (r - 1) \cdot r^{\ell - 1}.
\end{split}
\end{align}
with $1 \leq \ell \leq L$ and $r^L = N - k$. It is easy to see that this distribution therefore does indeed satisfy $\sum_{\ell} \Delta n_{\ell} = N - k$.

The other assumption we are making is that the distribution of energies $J_{\ell}$ increases exponentially with increasing $\ell$, giving it the form of
\begin{equation}
\label{eq:energy_distribution_disc}
    J_{\ell} = \Lambda \cdot e^{-\gamma \cdot (L - \ell)}
\end{equation}
for some UV energy scale $\Lambda > 0$ and rate of increase $\gamma > 0$. This is an artificial but reasonable choice because we want the circuit to obey renormalization invariance while going from the IR to UV limit in the same was as MERA networks generally do.

\subsubsection{Moving to the Continuum Limit}

To determine the scaling of the entropy close to the zero temperature (i.e. $\beta \rightarrow \infty$) limit, it is useful to consider the continuum limit of \eqref{eq:level_entr_discrete} in addition to the other assumptions we made. The stabilizer difference $\Delta n_{\ell}$ therefore becomes the stabilizer density
\begin{equation}
    \rho(\ell) = 
    \rho_0 \cdot e^{\alpha \cdot \ell}, \quad \ell \in [0, L],
\end{equation}
where $\alpha > 0$ can be chosen arbitrarily\footnote{One could of course choose $\alpha = \log(r)$ in the spirit of \eqref{eq:stab_distribution}, but we will refrain from making a specific choice here for the sake of generality. This specific case will be considered later when comparing the approximation with the actual entropy formula.} and $\rho_0$ is fixed by the density having to satisfy
\begin{equation}
    N-k \stackrel{!}{=} \int_0^L d\ell \, \rho(\ell) = \frac{\rho_0}{\alpha} \left( e^{\alpha \cdot L} - 1 \right) \quad \Longleftrightarrow \quad \rho_0 = \frac{\alpha \cdot (N-k)}{e^{\alpha \cdot L} - 1}.
\end{equation}
Because the distribution of the energy levels \eqref{eq:energy_distribution_disc} can be left untouched when moving to the continuum limit, the stabilizer entropy can be naively approximated as
\begin{equation}
\label{eq:level_entr_cont}
    S_{\text{stab}} \approx S_{\text{cont}} = \log\left(d^{k} \cdot (d-1)^{\braket{N-k}}\right) + \int_0^L d\ell \, \rho(\ell) \cdot S(p(\ell)),
\end{equation}
with $p(\ell)$ being of the same form as $p_{\ell}$ in \eqref{eq:shannon_prob}, but now considered as a continuous function of $\ell$. But to make the upcoming calculations easier, we perform a change of variables, integrating over $J = J(\ell)$ instead of $\ell$. To do that, we first note that from \eqref{eq:energy_distribution_disc} it follows that
\begin{equation}
    \ell(J) = L + \frac{1}{\gamma} \cdot \log\left( \frac{J}{\Lambda} \right),
\end{equation}
and hence
\begin{equation}
    d \ell = \frac{d \ell}{d J} \, dJ = \frac{dJ}{\gamma \cdot J} .
\end{equation}
This also allows us to express the stabilizer density as a function dependent on $J$:
\begin{equation}
    \rho(J) = \rho_0 \cdot e^{\alpha L} \cdot \left( \frac{J}{\Lambda} \right)^{\alpha/\gamma}.
\end{equation}
Finally, the continuous entropy as an integral over $J$ is
\begin{align}
\begin{split}
\label{eq:S_cont}
    S_{\text{cont}} &= \log\left(d^{k} \cdot (d-1)^{\braket{N-k}}\right) + \int_{
    \Lambda \cdot e^{-\gamma  L}}^{\Lambda} dJ \, \frac{\rho(J)}{\gamma \cdot J} \cdot S(p(J)) \\
    &= \log\left(d^{k} \cdot (d-1)^{\braket{N-k}}\right) + \frac{\rho_0 }{\gamma} \cdot e^{\alpha  L} \cdot \int_{
    \Lambda \cdot e^{-\gamma L}}^{\Lambda} \frac{dJ}{\Lambda}  \left( \frac{J}{\Lambda} \right)^{\alpha/\gamma - 1} \cdot S(p(J)).
\end{split}
\end{align}
Note that the lower integration bound acts as an effective IR cutoff for the integral. This is necessary for us to be able to make the following approximations..

\subsubsection{Low-Temperature Limit}

Computing the integral in \eqref{eq:S_cont} is in general hard, but since we are only interested in the limit of small $T/J$ (or equivalently large $\beta J$), we can approximate the binary entropy $S(p(J))$ that occurs in the integral as
\begin{align}
\begin{split}
\label{eq:binS_approx}
    S(p(J)) &= - \frac{d-1}{e^{\beta J} + d - 1} \cdot \log \left( \frac{d-1}{e^{\beta J} + d - 1} \right) -  \frac{e^{\beta J}}{e^{\beta J} + d - 1} \cdot \log \left( \frac{e^{\beta J}}{e^{\beta J} + d - 1} \right) \\ 
    &\stackrel{\beta J \rightarrow \infty}{=} (d-1) \cdot \frac{\beta J}{e^{\beta J}} + \mathcal{O}(e^{-\beta J}),
\end{split}
\end{align}
which is straightforward to prove. To realize this limit it is necessary to choose the right parameters since it follows from \eqref{eq:energy_distribution_disc} that
\begin{equation}
    \beta J = \beta \Lambda \cdot e^{- \gamma (L-\ell)} \gg 1 \quad \forall \, \ell
\end{equation}
and hence
\begin{equation}
    \beta \Lambda \cdot e^{-\gamma L} \gg 1 \quad \Longleftrightarrow \quad \gamma L \ll \log(\beta \Lambda).
\end{equation}

Plugging \eqref{eq:binS_approx} into \eqref{eq:S_cont} and noting that $\braket{N-k} = \sum_i p_i = 0$ in that limit then leaves us with an expression that can be further simplified using a change of variables:
\begin{align}
\begin{split}
\label{eq:S_cont_2}
    S_{\text{cont}} &\approx k \log(d) + (d-1) \cdot \frac{\rho_0 \cdot e^{\alpha L}}{\gamma}  \int_{
    \Lambda \cdot e^{-\gamma L}}^{\Lambda} \frac{dJ}{\Lambda} \frac{\beta J}{e^{\beta J}} \left( \frac{J}{\Lambda} \right)^{\alpha/\gamma - 1} \\
    &= k \log(d) + (d-1) \cdot \frac{\rho_0 \cdot e^{\alpha L}}{\gamma} \cdot (\beta \Lambda)^{-\alpha/\gamma} \int_{\beta \Lambda \cdot e^{-\gamma L}}^{\beta \Lambda} dt \, t^{\alpha/\gamma} \cdot e^{-t} \\
    &= k \log(d) + (d-1)(N-k) \cdot \frac{\alpha}{\gamma} \cdot \frac{e^{\alpha L}}{e^{\alpha L} - 1} \cdot (\beta \Lambda)^{-\alpha/\gamma} \int_{\beta \Lambda \cdot e^{-\gamma L}}^{\beta \Lambda} dt \, t^{\alpha/\gamma} \cdot e^{-t} \\
    &\stackrel{\alpha L \gg 1}{\approx} k \log(d) + (d-1) (N-k) \cdot \frac{\alpha}{\gamma} \cdot (\beta \Lambda)^{-\alpha/\gamma} \int_{\beta \Lambda \cdot e^{-\gamma L}}^{\beta \Lambda} dt \, t^{\alpha/\gamma} \cdot e^{-t}
\end{split}
\end{align}
Let's consider the trailing integral. Up to the integration bounds it is the same as the gamma function $\Gamma(\alpha/\gamma + 1)$, whose integrand is positive everywhere. We can therefore get an upper bound for $S_{\text{cont}}$ (that we also expect to be approximately saturated for certain domains of $\beta \Lambda$) by substituting the \enquote{incomplete} gamma function with the proper one. Thus we have
\begin{equation}
\label{eq:S_cont_3}
    S_{\text{cont}} \lessapprox  k \log(d) + (d-1) (N-k) \cdot \frac{\alpha}{\gamma} \cdot \Gamma\left(\frac{\alpha}{\gamma} + 1\right) \cdot (\beta \Lambda)^{-\alpha/\gamma}, 
\end{equation}
which only scales with $(\beta \Lambda)^{-\alpha/\gamma} = (T/\Lambda)^{\alpha/\gamma}$, indicating that the entropy could indeed follow a power law, at least for certain low-temperature regimes. To show how well both continuous approximations hold up against the discrete stabilizer entropy with equivalent parameters ($N-k = r^L$, $\alpha=\log(r)$), we display both in logarithmic plots over $\log(T/\Lambda)$ and with different choices of $\gamma$, which is the only significant free parameter. These plots are depicted in figure \ref{fig:entropy_scaling_appendix} and indeed confirm that our low-temperature approximations are good at predicting aspects of the actual entropy, including its power-law growth.

\begin{figure}[hbt]
    \centering
    \includegraphics[width=\textwidth]{figures/entropy_scaling_appendix.png}
    \caption{Logarithmic scaling of exact stabilizer entropies $S_{\text{stab}}$ and their continuous approximations $S_{\text{cont}}$ (with and without the gamma function correction) for $L=20$, $N-k=r^L$, $k=1$, $d=2$, $r=2$, $\alpha = \log(r)$, $\Lambda=1$ and $\gamma \in \{0.1, 0.4, 1, 3\}$. In the first two figures it can be seen that our continuous approximation from \eqref{eq:S_cont_2} matches almost exactly with the discrete stabilizer entropy for $\gamma \ll 1$ and small $T/\Lambda$. Even though the second figure shows less behavior than the first one, we expect that it will behave similarly for even lower relative temperatures. While the last two approximations with $\gamma \geq 0$ also receive their primary contribution from the polynomial term, it is more apparent that they don't completely align with the actual data anymore. Especially in the last figure where $\gamma = 2$ the trend of the stabilizer entropy is not strictly polynomial anymore. Still, each figure has at least a regime where its growth is either exactly polynomial or follows a polynomial trend that aligns with our theoretical predictions up to a total constant factor.}  
    \label{fig:entropy_scaling_appendix}
\end{figure}




\section{Estimating the Layer Growth Factor}
\label{sec:growth_factor}

Given a generic string of $n$ generalized Pauli operators with local dimension $d$ and initial weight $w_0 \ll n$, we can estimate the relative weight growth $g$ the string experiences from one layer of $n/q$ random Cliffords being applied to random disjoint substrings of length $q$. The weight $w_k$ at the $k$th layer can therefore be estimated as
\begin{equation}
    w_k \approx g \cdot w_{k-1}.
\end{equation}

\subsection{Single Layer}

To find $g$ it is helpful to look at a single substring of length $q$ being scrambled by a single random Clifford. In that case, as long as a substring's weight $w_{k-1}(q)$ is not zero, we can expect its weight in the next layer to be on average
\begin{equation}
    \label{eq:substring_weight}
    w_k(q) = q \cdot \frac{d^2-1}{d^2},
\end{equation}
regardless of how the initial string looked\footnote{Remember that the generalized Pauli group of dimension $d$ has $d^2$ different elements, up to phases. Of those only the identity has zero weight.}. To extend this argument to the whole Pauli string we can therefore distinguish between two extreme cases:
\begin{itemize}
    \item All $w_{k-1}$ non-trivial Pauli operators are contained in as few substrings as possible, namely $\lceil w_{k-1}/q \rceil$. Since each such substring will on average have the weight \eqref{eq:substring_weight} after the Clifford layer, we find that
    \begin{equation}
    \label{eq:growth_saturated}
        w_k = \lceil w_{k-1}/q \rceil \cdot q \cdot \frac{d^2-1}{d^2} \approx w_{k-1} \cdot \frac{d^2-1}{d^2}.
    \end{equation}
    \item Each (non-trivial) substring contains exactly one non-trivial Pauli operator, meaning that in the next layer we have $w_{k-1}$ substrings each having the average weight \eqref{eq:substring_weight}. The total weight is therefore
    \begin{equation}
        w_k = w_{k-1} \cdot q \cdot \frac{d^2-1}{d^2}
    \end{equation}
\end{itemize}
Hence we can provide approximate upper and lower bounds for $g$ by
\begin{equation}
    \frac{d^2-1}{d^2} \lesssim g \lesssim q \cdot \frac{d^2-1}{d^2}.
\end{equation}
However, for our purposes we will always have $w_{k-1} \ll n$ (see next section), which makes it more likely for $g$ to be closer to the upper bound. Hence we can assume that
\begin{equation}
\label{eq:growth_factor}
    g \approx q \cdot \frac{d^2-1}{d^2}.
\end{equation}

\subsection{Multiple Layers}
\label{sec:max_depth}

Usually the scrambling circuits will be composed of more than one layer of random $q$-party Clifford gates. Therefore, given that we start with $w_0 \ll n$ and keep $g$ fixed to be \eqref{eq:growth_factor}, what is the approximate maximum depth $D$ for which $w_D = g^D \cdot w_0$ gives a good estimate for the total operator weight at the end?

Due to our previous arguments, we can expect the approximation to not hold anymore by the time $w_D$ is of order $n$ since then the case of \eqref{eq:growth_saturated} will dominate. In this case we say that the weight is \emph{saturated}, and we can estimate the order of magnitude of the saturation depth $D_{\text{sat}}$ by requiring that $g^{D_{\text{sat}}} \cdot w_0 \lessapprox n$, leading to:
\begin{equation}
\label{eq:D_max}
    D_{\text{sat}} \approx \log_g \left( \frac{n}{w_0} \right).
\end{equation}
A tighter bound can also be achieved by using $\log_q$ instead of $\log_g$. Both options are shown for a specific simulated example in Figure \ref{fig:growth_factor}.

\begin{figure}[htb]
    \centering
    \includegraphics[width=0.9\textwidth]{figures/growth_factor.png}
    \caption{Averaged relative weight growth $w_D / w_{D-1}$ of a single Pauli string ($d=3, n=128, w_0 = 1$) subjected to a random 2-local Clifford with increasing circuit depth $D$ (1000 repetitions). Depicted are also the estimates of the effect growth factor $g$ \eqref{eq:growth_factor} and the saturation depth(s) $D_{\text{sat}}$ \eqref{eq:D_max} for which it can be considered to hold. While \eqref{eq:D_max} indeed provides a good maximum circuit depth for which the data and our estimate for $g = 16/9$ approximately coincide, a tighter bound can be achieved by instead using $q = 2$ as the base of the logarithm.}
    \label{fig:growth_factor}
\end{figure}

\section{Phase Space Formalism}
\label{sec:phase_space}

\subsection{Weyl Representation}
\label{sec:weyl_rep}

Given a Hilbert space $\mathcal{H}$ of prime dimension $d > 2$ \footnote{The case of $d=2$ is excluded here since our choice of representation requires the existence of a 2-element in the group sucht that $\frac{1}{2} = \frac{d + 1}{2}$, which is only the case for d>2. This should not affect the phsyics though as systems of different qudits can always be mapped to each other.}, we choose a basis $\{\ket{0},\ket{1}, \ldots, \ket{d-1}\}$ with its states being labeled by the elements of the associated finite (Galois) field GF($d$)\footnote{Finite fields also exist for powers of primes i.e.\ GF($d^k$), but addition and multiplication does not happen mod $d^k$ in these cases. One can achieve the same group order though by instead using $k$ qudits with each being represented by a copy of GF($d$)}.  One can then introduce \emph{clock and shift operators} $Z, X$ which act on the basis states according to \cite{hudson}
\begin{equation}
\label{eq:boost_shift}
    Z^p \ket{k} = \chi(p \cdot k) \ket{k}, \quad X^q \ket{k} = \ket{k + q},
\end{equation}
where $p, q, k \in \text{GF}(d)$ and $\chi(k) = e^{2 \pi i k / d}$. Note that addition and multiplication happens over GF($d$) and is thus mod $d$. This is also respected by our choice for $\chi(k)$ since $\chi(k + d) = \chi(k)$ even for addition without modulo.

We are now able to define the so-called \emph{Weyl operators} for a single qudit, which provide a generalisation of the Pauli operators on a qubit:
\begin{equation}
\label{eq:weyl_single}
    w(p, q) = \chi\left(-\frac{p \cdot q}{2}\right) \, Z^p \, X^q, 
    \quad p,q \in \text{GF}(d).
\end{equation}
Extending this definition to $n$ qudits is as easy as tensoring $n$ copies of \eqref{eq:weyl_single}, which we write as
\begin{align}
\label{eq:weyl_multi}
    \begin{split}
        w(v) &= w(p_1, q_1, \ldots, p_n, q_n) \\
        &= w(p_1, q_1) \otimes \ldots \otimes w(p_n, q_n).
\end{split}
\end{align}
Each Weyl operator is therefore uniquely represented by an element $v$ of a $2n$-dimensional vector space $V$ over the field GF($d$). Using the commutation relations of $Z^p$ and $X^q$ that arise from their definition in \eqref{eq:boost_shift}, it also follows that
\begin{equation}
\label{eq:weyl_mul}
    w(v) \, w(w) = \chi \left( \frac{\symp{v}{w}}{2} \right) \, w(v + w),
\end{equation}
where $\symp{\cdot}{\cdot}$ is the \emph{symplectic product} on $V$, which obeys $\symp{v}{w} = -\symp{w}{v}$ and can be expressed as a matrix product:
\begin{equation}
\label{eq:symp_prod}
    \symp{v}{w} = v^T J w, \quad J = \begin{pmatrix}
        0 & 1 \\ -1 & 0
    \end{pmatrix}^{\oplus n}.
\end{equation}
Because of that the Weyl operators form a projective representation of the associated vector space $V$ equipped with a symplectic product. It is also noteworthy that \eqref{eq:weyl_mul} implies that two Weyl operators $w(v), w(w)$ commute if and only if the corresponding symplectic product $\symp{v}{w}$ vanishes.

Another useful indentity which we will use later is the fact that
only the identity $I_n = w(0)$ has a non-vanishing trace:
\begin{equation}
    \label{eq:weyl_trace}
        \Tr[w(v)] = d^n \delta_{v,0}.
\end{equation}
This is trivial to show for $X^q$ but requires using the fact that the Kronecker delta can be written as
\begin{equation}
\label{eq:kronecker_sum}
    \delta_{p,0} = \frac{1}{d} \sum_{k = 0}^{d-1} e^{\frac{2 \pi i k}{d} p}
\end{equation}
to prove it for $Z^p$ as well. 

\subsection{The Clifford Group}

The Clifford group is a subset of the unitary group which maps Weyl operators to other Weyl operators (up to a factor):
\begin{equation}
\label{eq:clifford_def}
    U w(v) U^{\dagger} = c(v) \, w(S(v)),
\end{equation}
for some $c: V \rightarrow \mathbb{C}$ and $S: V \rightarrow V$. Because $S$ therefore has to be compatible with \eqref{eq:weyl_mul}, it is easy to see that it has to be linear and preserve the symplectic product:
\begin{equation}
    \symp{S v}{S w} = \symp{v}{w} \quad \forall \, v,w \in V.
\end{equation}
In matrix representation, one can also equivalently state this property as $S^T J S = J$. Such a function is called \emph{symplectic}. The set of all symplectic functions for a given vector space $V$ forms the so-called \emph{symplectic group}\footnote{Note the similarities to the definition of the orthogonal group. In fact, the column entries of a symplectic matrix also form as (symplectic) basis $(e_1, f_1, \ldots, e_n, f_n)$ of $V$ which satisfies $\symp{e_i}{e_j} = 0 = \symp{f_i}{f_j}$ and $\symp{e_i}{f_j} = \delta_{ij}$ for all $i,j = 1, \ldots, n$. Applying a symplectic is therefore equivalent to a change of basis. \label{fn:symp_basis}}. 

In general, the structure of the Clifford group is completely determined by the following statements:
\begin{enumerate}
    \item For any symplectic $S$ there is a unitary operator $\mu(S)$ satisfying
    \begin{equation}
        \mu(S) w(v) \mu(S)^{\dagger} = w(S v) \quad \forall \, v \in V.
    \end{equation}
    \item $\mu(S)$ is a projective representation of the symplectic group, meaning
    \begin{equation}
        \mu(S) \mu(T) = e^{i \phi} \mu(S T)
    \end{equation}
    for some phase $\phi$.
    \item Up to a phase, any Clifford operator is of the form
    \begin{equation}
        U = w(a) \mu(S)
    \end{equation}
    for a suitable $a \in V$ and symplectic $S$.
\end{enumerate}
A proof of these statements can be found in \cite{hudson}. Note that this also fixes the factor from \eqref{eq:clifford_def} to be $c(v) = \chi(\symp{a}{Sv})$. 

\subsection{Stabilizer States and Codes}

As mentioned before, a vanishing symplectic product $\symp{v}{w}$ is equivalent to a vanishing commutator $[w(v), w(w)]$. One can therefore construct a set
\begin{equation}
    w(M) = \{ m \,|\, m \in M\}
\end{equation}
containing only commuting Weyl operators by choosing $M$ to be a subspace of $V$ satisfying
\begin{equation}
    \symp{m_i}{m_j} = 0 \quad \forall \, m_i, m_j \in M
\end{equation}
Such a subspace is called \emph{isotropic} and it is easy to see that it also forms a group under vector addition since the symplectic product is bilinear. The cardinality of isotropic subspaces can range between 0 and $d^n$ as there are at most $n$ elements with mutually vanishing symplectic product in a $2n$-dimensional symplectic basis (see footnote \ref{fn:symp_basis} for the reason). We will refer to $M$ having maximal cardinality as \emph{maximally isotropic}.

In general it is convenient to write the basis elements of an isotropic subspace as a $k \times 2n$ (or $2n \times k$) matrix over $GF(d)$, where $k = \log_d(M)$ is the size of the basis. In the literature this is called the \emph{stabilizer matrix}, although there it is often written in terms of the actual Pauli/Weyl operators and not their symplectic representation.

Isotropy of $M$ allows one to (at least partially) diagonalize the Weyl operators contained in $w(M)$, even completely if $M$ is maximally isotropic. In the latter case it is therefore possible to define a unique quantum state $\ket{M, v}$ in terms of the elements in $w(M)$ acting on it as stabilizers:
\begin{equation}
\label{eq:stab_def}
    \chi(\symp{v}{m}) w(m) \ket{M, v} = \ket{M, v} \quad \forall \, m \in M.
\end{equation}
The vector $v \in V$ therefore determines the phase differences between the eigenstates assocated with $w(M)$. A state satisfying \eqref{eq:stab_def} is called a \emph{stabilizer state} and can be written as
\begin{equation}
\label{eq:stab_state}
    \ket{M, v}\bra{M, v} = \frac{1}{d^n} \sum_{m \in M} \chi(\symp{v}{m})\, w(m).
\end{equation}
It is easy to show that \eqref{eq:stab_state} is a projection operator and has unit trace by applying \eqref{eq:weyl_trace} and using the fact that $M$ is a group and thus satisfies $M + m = M$ for all $m \in M$.

In fact, even for a non-maximally isotropic subspace $M$ would \eqref{eq:stab_state} still be a projector (up to normalization), but not a quantum state anymore. In this more general case we have
\begin{equation}
\label{eq:stab_proj}
    \Pi(M,v) = \frac{1}{|M|} \sum_{m \in M} \chi(\symp{v}{m})\, w(m)
\end{equation}
with $\Tr[\Pi(M,v)] = \frac{d^n}{|M|}$. All states in the subspace which $\Pi(M,v)$ projects onto therefore satisfy \eqref{eq:stab_def}, meaning that they form a code space. We can therefore identify this case as being a stabilizer code since it satisfies the definition in section \ref{sec:stab}. Even though finding stabilizer codes therefore just amounts to making a choice for $M$ and $v$, it does not ensure that the resulting code is good in the sense that its Hamming distance might be small or does not scale well.

\subsection{Entanglement Entropy of Stabilizer States}
\label{sec:stab_entropy}

Thanks to the structure of the symplectic product \eqref{eq:symp_prod} and the multi-particle Weyl operators defined in \eqref{eq:weyl_multi}, one can easily take the partial trace of \eqref{eq:stab_state} over a desired subsystem $B$ by writing $v = v_A \oplus v_B$ (same for $m$) and $w(m) = w(m_A) \otimes w(m_B)$ for all $m \in M$ and applying \eqref{eq:weyl_trace} to the latter term in the tensor product. The resulting reduced state is then
\begin{align}
\begin{split}
\label{eq:stab_reduced}
    \rho_A &= \Tr_B[\rho] \\
    &= \frac{d^{n_B}}{d^n} \sum_{m_A \in M_A} \chi(\symp{v_A}{m_A})\, w(m_A) \\
    &\equiv \frac{|M_A|}{d^{n_A}} \, \Pi(M_A, v_A),
\end{split}
\end{align}
where we made use of the fact that $n = n_A + n_B$ and identified \eqref{eq:stab_proj}, but this time in terms of $v_A$ and
\begin{equation}
\label{eq:M_reduced}
    M_A = \{ m_A \,|\, m_A \oplus 0_B \in M\}.
\end{equation}
This is possible since the definition of $M_A$ ensures that it is again a group (although not necessarily maximally isotropic)\footnote{Naively computing $M_A$ using \eqref{eq:M_reduced} is not efficient as such an algorithm would have $\mathcal{O}(d^n)$ runtime. A runtime that is polynomial in the system size can be achieved by instead permuting the sites that are to be traced out to the front the stabilizer matrix and then computing its reduced row echolon form. The basis vectors $b = b_A \oplus b_B$ for which $b_B \neq 0$ are then removed and for the remaining elements only $b_A$ is being considered.}.

The fact that even after tracing out a subsystem the resulting reduced state is still proportional to a projection operator makes computing the entanglement entropy straightforward. While it is possible to just directly evaluate the Von-Neumann entropy $S(A) = \Tr[\rho_A \log_d(\rho_A)]$, a more elegant and insightful approach can be made by instead considering the \emph{R\'enyi entropies}
\begin{equation}
    S^{(n)}(A) = \frac{1}{1 - n} \log_d \Tr[\rho_A^n],
\end{equation}
which have the property that
\begin{equation}
    \log_d(d^{n_A}) = S^{(0)}(A) \geq S(A) \geq S^{(2)}(A) \geq \ldots 
\end{equation}
where $S(A) = S^{(1)}(A) = \lim_{n \rightarrow 0} S^{(n)}(A)$ reproduces the ordinary Von-Neumann entropy. What makes the R\'enyi entropies interesting here is that they satisfy $S^{(n)}(A) = \log_d (\rank\rho_A)$ for all $n>0$ if the state being considered has a flat entanglement spectrum i.e.\ it is proportional to a projection operator\footnote{The proof is straightforward: Let $\rho_A = \alpha \cdot \Pi_A$, then $S^{(n)}(A) = \frac{1}{1 - n} \log_d \Tr[(\alpha \cdot \Pi_A))^n] = \frac{1}{1 - n} \log_d (\alpha^n \cdot \Tr[\Pi_A]) = \frac{1}{1 - n} \log_d (\alpha^n \cdot \rank\rho_A)$. Since $\alpha = (\rank \rho_A)^{-1}$ because of normalization we have $S^{(n)}(A) = \frac{1}{1 - n} \log_d (\rank\rho_A)^{1-n} = \log_d (\rank\rho_A)$.}. Since this is the case for the reduced stabilizer state we can use the fact that $\rank \rho_A = \frac{d^{n_A}}{|M_A|}$ to show that
\begin{equation}
    S(A) = n_A - \log_d |M_A|.
\end{equation}
If the number of basis vectors $k_A = \log_d |M_A|$ is known, then computing $S(A) = n_A - k_A$ is straightforward and numerically stable\footnote{As a sanity check, note that if $\rho_A$ is pure and therefore has $S(A) = 0$ it implies that $k_A = n_A$, which is the requirement for $M_A$ to be a maximally isotropic subspace and thus to define a (pure) stabilizer state.}.

\end{document}