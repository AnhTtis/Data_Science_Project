
\section{Analysis of the SYK-Inspired Code}
\label{sec:code_structure}

We now consider in more detail the properties of the SYK inspired code with $k=r^a$ and $L=a+b$ for two integers $a$ and $b$. Recall that the total number of qudits is 
\begin{equation}
    N = r^a + r^{a+b},
\end{equation}
and the ratio between ground state qudits and the total number of qudits (i.e. the rate) is therefore
\begin{equation}
    \frac{k}{N} = \frac{1}{1 + r^b},
\end{equation}
which is independent of $a$. The thermodynamic limit $a \rightarrow \infty$ gives a family of codes with non-zero rate. We already established in Section~\ref{sec:numerical} that this code can be highly entangled. It is also interesting to consider its complexity.

In this case, the complexity sum \eqref{eq:circuit_complexity} can be rewritten as
\begin{equation}
    \text{total gates} = \frac{D}{q \cdot (1 + r^b)} \cdot N \log_r N + \mathcal{O}(N).
\end{equation}
This leading $N \log N$ scaling with the total number of degrees of freedom can be compared to holographic complexity conjectures applied to JT gravity~\cite{jt_complexity}; one also gets $N \log N$ by studying, for example, the volume (length) of the wormhole dual to the thermofield double state with temperature of order $1/N$. The key point is that the throat of the wormhole is long, of order $\log N$, at this temperature. Hence, the circuit complexity of our SYK-inspired encoding also resembles that obtain from holographic models dual to SYK.

For the estimates discussed below, we continue to assume that the layers are composed of random 2-qudit Clifford gates applied to random pairs of qudits. We caution that this is certainly not correct for the actual SYK model: the gates must act on fermionic degrees of freedom and will not be Clifford (or the fermionic analogue of Clifford) generically. Here we continue to focus on the Clifford case for ease of analysis and for its interpretation in terms of an exact quantum error correcting code. Below we comment briefly on the potential similarities and differences with the actual SYK model.

\subsection{Distance Estimate and Stabilizer Weights}
\label{sec:syk_distance_weight}

We know the rate of our SYK-inspired code. To estimate the distance, we need to understand how logical operators grow as they pass from the IR to the UV. Let us assume that a typical operator grows in size by a factor of $g^D$ after passing through one layer (i.e. being conjugated by that layer unitary), up to a maximum size set by the total number of qudits. A way to estimate $g$ when the layer unitary is a random Clifford circuit can be found in appendix \ref{sec:growth_factor}. At the same time, the number of qudits is also growing, going from $k+r^{\ell-1}$ to $k+r^\ell$. The distance depends on whether the size of operators grows faster or slower than the number of qudits. Note that we saw already a manifestation of this competition in the discussion in Section~\ref{sec:numerical}; here we explain in more detail the issues.

From a given random circuit layer, we expect operators to grow by a factor of $g^D$ provided they are not close to maximum weight. If they are close to maximum weight, then they will grow by a reduced factor. We must compare this operator growth to the rate of qudit increase. The ratio $R_\ell$ between the number of qudits in successive layers is 
\begin{equation}
      R_\ell =  \frac{k+r^\ell}{k+r^{\ell-1}} = 1 + \frac{r-1}{1 + k/r^{\ell-1}}, 
\end{equation}
which monotonically increases with $\ell$. As logical operators evolve from layer to layer into the UV, the relative weight of the operator either increases or decreases depending on whether $g^D > R_\ell$ or $g^D < R_\ell$. The dynamics of this process, iterated over all $L$ layers, gives an estimate for the size of non-trivial logical operators.

\subsubsection{Warmup: Small Fixed $k$}

To illustrate the key competition, consider first the case in which $k$ is small and fixed. In this case, the ratio $R_\ell \rightarrow r$ as $\ell$ increases, so most of the evolution corresponds to a fixed ratio of $r$. In terms of the parameters above, we can achieve this regime by taking $b$ large at fixed $a$.

Suppose $g^D > r$. Then operator growth is the fastest process and logical operators will reach saturation. In this case, we expect the distance to be linear in $N$. The distance will not exactly saturate the singleton bound, but it may come close for large $D$.

Now suppose $g^D < r$. In this case, we are adding qudits faster than operators can grow, so the logical operators are ultimately supported on a dilute fraction of all the sites. Indeed, the size of a typical logical operator will be $g^{DL}$, whereas the total number of qudits is $N =  r^L (1 + r^{-b}) \approx r^L$. Expressed in terms of $N$, the size of a typical logical operator is
\begin{equation}
    g^{DL} \sim N^c
\end{equation}
where $c = \frac{\ln g^D}{\ln r} < 1$. Hence, we expect a distance that scales as a sublinear function of $N$.

\subsubsection{SYK-Like Scaling}

Now we turn to the case where $k = r^a$ is large and $b$ is fixed. Here, when $\ell$ is small, the $R_\ell$ ratio is close to one and the number of qudits is barely increasing from layer to layer. In this regime, operator growth is completely dominant. In contrast, at the most UV layer, where $\ell=L=a+b$, the ratio is
\begin{equation}
   R_L = 1 + \frac{r-1}{1+r^{1-b}} < r. 
\end{equation}

Suppose $g^D > R_L$. Then operator growth always dominates over qudit growth. However, because the initial number of qudits (the ground state qudits) is large, we still have to compare the total operator size, $g^{D L}$, to the total number of qudits, $N = r^L ( 1+ r^{-b})$. We see again that if $g^D >r$, then this naive estimate gives an operator weight larger than $N$, meaning that the operator growth actually saturated at something proportional to $N$. If $g^D < r$, then we are again in the situation where $g^{DL} \sim N^c$.


Suppose $g^D < R_L$. Then there will be some layer $\ell^*$ such that operator growth and qudit growth switch dominance as $\ell$ increases through $\ell^*$. We may approximately determine this crossover scale from
\begin{equation}
    g^D = R_{\ell^*},
\end{equation}
noting that this $\ell^*$ is not typically an integer. In the thermodynamic limit $a \rightarrow \infty$, we must have $\ell^* = a + b^*$ for some constant $b^*$ since the ratio $R_\ell$ is essentially unity until $r^{\ell}$ is comparable to $k$.

Now between $\ell=1$ and $\ell = \ell^*$, logical operators will grow faster than the number of qudits. Assuming they don't reach saturation, they will grow by roughly a factor of $g^{D\ell^*}$. By contrast, the number of qudits at layer $\ell^*$ is
\begin{equation}
    n_{\ell^*} = r^a (1 + r^{b^*}),
\end{equation}
so the ratio of operator size to number of qudits is
\begin{equation}
    \left( \frac{g^D}{r} \right)^a \frac{g^{D b^*}}{1+r^{b^*}}.
\end{equation}
This ratio vanishes as $a\rightarrow \infty$ since we are assuming that $g^D < R_L$ and $R_L < r$. Hence, $g^{D\ell^*} \sim (n_{\ell^*})^c$ as above.

There are a fixed number of layers from $\ell^*$ to $L$ since $b$ and $b^*$ are fixed as $a\rightarrow \infty$. Therefore operators and the number of qudits grow by an additional factor independent of $N$ from $\ell^*$ to $L$. Hence, the scaling of $g^{DL}$ with $N$ is the same as the scaling of $g^{D\ell^*}$ with $n_{\ell^*}$, that is $g^{DL} \sim N^c$.




\subsubsection{Stabilizer Weights}

We expect that the stabilizer weights will display a similar pattern as in Figure~\ref{fig:stab_weights}. In particular, a non-zero fraction of all the stabilizers will have constant weight. These arise from the UV most layer. Then as we descend in the network towards the IR, there are fewer stabilizers but of increasing weight. In particular, there are at least a few stabilizers of very high weight, similar to the weight of logical operators. 

\subsection{Comparison to SYK} 

We now compare features of the SYK-inspired code to those of the actual SYK model. To be precise, we will compare a particular realization of the SYK Hamiltonian (with $q=4$), $H_{\text{SYK}}$, with a particular realization of the toy code Hamiltonian, $H_{\text{code}}$, for the SYK-inspired code (see Section~\ref{sec:arch}). (It is also interesting to consider supersymmetric generalizations~\cite{Fu_2017}.)
\begin{itemize}

   \item{[Hamiltonian structure]} $H_{\text{SYK}}$ is composed of $O(N^4)$ weight-$4$ fermion terms (all possible such terms). These terms do not all commute and they enter $H_{\text{SYK}}$ with random coefficients. $H_{\text{code}}$ is composed of $O(N)$ commuting terms with fixed coefficients. The weight of the terms varies, with many having low-weight but a significant fraction having high weight, comparable to the distance of the corresponding code.
   
    \item{[Ground space]}  $H_{\text{SYK}}$ has $e^{s_0 N}$ approximate ground states which are approximately degenerate with level spacing $e^{-\alpha N}$. $\alpha$ and $s_0$ are constants, independent of $N$. Similarly, $H_{\text{code}}$ has $d^k = e^{ \frac{\ln d}{1+r^b} N}$ exactly degenerate ground states.
    
    \item{[Low temperature thermodynamics]} The SYK model has a low temperature heat capacity proportional to temperature $T$. Similarly, the parameters of $H_{\text{code}}$ can be chosen so that its low temperature heat capacity is proportional to $T$.
    
    \item{[Fine-grained spectrum]} The fine-grained energy spectrum of $H_{\text{SYK}}$ is random-matrix-like~\cite{Cotler_2017,saad2019semiclassical}. The fine-grained energy spectrum of $H_{\text{code}}$ is not random-matrix-like because $H_{\text{code}}$ is a commuting projector Hamiltonian.
    
    \item{[Entanglement]} Both models feature energy eigenstates with volume-law entanglement. The entanglement spectrum will, however, be quite different between the two kinds of states. In particular, eigenstates of $H_{\text{code}}$, being stabilizer states, have a flat entanglement spectrum.
    
    \item{[Complexity]} We only have estimates here. Using the duality to JT gravity and holographic complexity/geometry conjectures, the circuit complexity of the SYK approximate ground states is estimated to be $O(N \ln N)$. We have an explicit estimate (and upper bound) of $O(N \ln N)$ for circuit complexity of the ground space of $H_{\text{code}}$.
    
 
\end{itemize}

The many similarities between $H_{\text{SYK}}$ and $H_{\text{code}}$ are the basis for our conjecture that the architecture in Figure~\ref{fig:arch} has the potential to describe the physics of the SYK model once the tensors in the network have been adapted to a particular SYK instance, for example, using a variational approach. However, there are also crucial differences between the two. Two that stand out are the different scalings of the weights of Hamiltonian terms with system size and the exact versus approximate nature of the ground state degeneracy. The fine-grained energy spectrum is also very different in the two cases. Thus, it will be informative in the future to explore our network architecture as a variational ansatz for the SYK ground space.

\subsection{SYK Ground Space as an Approximate Code}

Here we want to comment on another possibility raised by the similarities above. For $H_{\text{code}}$, we have seen explicitly that the ground space can be viewed as an error correcting code with constant relative distance and constant rate (provided $D$ is big enough). In particular, it is an exact stabilizer code. This naturally raises the possibility that the approximate ground space of the SYK model could have interesting properties as an approximate quantum error correction code.\footnote{The network architecture presented here was first considered by one of the authors in fall 2019 during their stay at the Institute for Advanced Study and later presented in preliminary form, along with the potential code interpretation, in January 2020 at UCSB. Independently, the code properties of SYK in the thermal regime have been studied~\cite{Chandrasekaran_2022}.}

Thus we consider a code defined by the full approximate ground space of some particular $H_{\text{SYK}}$ realization. By construction this code has a constant rate as $N\rightarrow \infty$ which is given by ground state entropy density $s_0$. This code is not a stabilizer code, but it does have a sort of ``low weight'' definition via the SYK Hamiltonian. 

What is not immediately clear is the distance of this code. Moreover, since the code is approximate, we must specify precisely what we mean by the distance. We will defer a full discussion to a future work, but here let us note that if the architecture in Figure~\ref{fig:arch} does indeed provide a good approximation to the ground space of the SYK realization, then the same kind of scaling analysis discussed above for the random Clifford code would also provide an estimate for the operator size of logical operators.

In this case, it would be important to understand the analog of $r$ and $g^D$ in the SYK case. As one approach, we could fix $r=2$ and then adjust the layer circuits so that we get a good approximation to the ground space. The parameter $g^D$ would then be determined by the properties of these circuit. A simple random operator growth model may be too crude to capture the detailed physics, but continuing with this estimate for now, if the resulting $g^D$ were greater than $r$, then we have logical operators of weight proportional to $N$ and potentially distance proportional to $N$. Alternatively, if $g^D < r$, then the distance could be some power of $N$, $N^c$. It would be interesting to understand which of two cases is realized; this should be related to the spectrum of the scaling dimensions in the theory since these are related to the mixing properties of the scaling superoperator~\cite{PhysRevLett.101.110501}. Given the relatively low scaling dimension of the fermion operators, it may be that one is effectively in the $g^D<r$ regime.

\subsection{Summary}

We gave analytical estimates of the distance for a family of SYK-inspired codes in the thermodynamic limit of many qudits. This code family shares a number of similarities with known properties of the actual SYK model, although there are crucial differences as well. Viewing the approximate ground space of SYK as an approximate quantum code, the analysis of the SYK-inspired model suggests that the actual SYK ground space code, which has constant rate as $N\rightarrow \infty$, could have a distance $N^c$ for some constant $0<c\leq 1$.