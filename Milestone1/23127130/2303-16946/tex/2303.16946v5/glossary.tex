\section*{Glossary}

\begin{tabularx}{\columnwidth}{r X}
    $d$ & Local qudit bond dimension. \\
    $L$ & Number of layers in the NoRA network. \\
    $n_{\ell}$ & Number of qudits being acted on non-trivially at layer $\ell$. \\
    $N$ & Number of physical qudits. Equal to $n_L$. \\
    $k$ & Number of (ground-state) logical qudits. Equal to $n_0$. \\
    $\Delta n_{\ell}$ & Number of new (thermal) qudits introduced at layer $\ell$. Equal to $n_{\ell} - n_{\ell - 1}$. \\
    $D$ & Circuit depth for a single layer of NoRA (layer-independent). \\
    $r$ & Qudit growth rate of the NoRA network with specified scaling. \\
    $q$ & Locality of the circuit at a single later (layer-independent).  \\
    $w_{\ell}$ & Operator weight of a (stabilizer) Pauli string at layer $\ell$. \\
    $g$ & Effective weight growth rate for a single random $q$-local Clifford. Approximately $q \cdot (d^2 - 1)/d^2$. \\
    $D_{\text{sat}}$ & Minimum layer circuit depth necessary to achieve approximate weight saturation. Approximately $ \approx \log_g r$. \\
    $\delta$ & Code distance of a $[[N, k, \delta]]$ error-correcting code. \\
    $\delta_{\text{qsb}}$ & Maximum possible code distance as predicted by the quantum singleton bound. Equal to $(N - k)/2 + 1$. \\
    $\delta / N$ & Relative code distance.
    
\end{tabularx}
