\section{Introduction}
\label{sec:intro}

In recent years, quantum information theory and more specifically quantum error correction have quickly become some of the most-studied subjects in both experimental and theoretical phsyics. This is not only because of the anticipated advent of large-scale quantum computing devices and with them the need for efficient and scalable protection against noise, but also the uncovered connections between certain gravitational and quantum mechanical systems \cite{Maldacena_1999}. Further work on what is now called the AdS/CFT correspondence suggests that maybe even the universe itself exhibits some form of fundamental self-correcting properties in the form of \emph{holography} \cite{Harlow_2016}. 

But while methods to prevent data corruption due to outside noise have long been a staple in modern computer science and are nowadays used in almost all computational devices, trying to accomplish the same for quantum systems is still a relatively new field of research \cite{shor} \cite{gottesmann}. While many concept can indeed be taken from previous work, translating error correction to quantum systems poses a couple of problems that are not present in the classical counterpart. One major issue is that many classical codes employ some form of redundancy to make the information safe against noise. This is not feasible for their quantum analogue as quantum mechanics forbids the cloning of quantum information and manually preparing copies of states is not an efficient method. Additionally, the most naive way of checking if an error occured would involve having to measure the current state of the quantum computer, thus completely or at least partially collapsing the wavefunction and destroying the superpositions.

While a solution to these problems has been found in \emph{stabilizer codes} \cite{gottesmann}, it is not entirely clear yet what kind of codes are feasible and scale well with the system size and the number of qudits one wants to encode. Recent advances \cite{Panteleev_2022} seem to have proven though that there exist so-called \enquote{good} quantum low-density-parity-check (LDPC) codes whose distance scales almost with the system size. This would mean a serious step towards large-scale quantum error correction and bring us towards the cutting edge of classical computer science, where good LDPC codes are nowadays used in a lot of devices.

In this paper we try to set the groundwork for evaluating if such a (good) qLDPC code is realized by the Sachdev-Ye-Kitaev (SYK) model, a system of Majorana fermions with randomly distributed interaction coefficients. Given that the model indeed exhibits error-correcting properties (which is thought to be the case \cite{syk_qec}), we expect that the corresponding code might be LDPC because the defining Hamiltonian has low operator weight. But what makes this model also interesting is that it exhibits an almost conformal symmetry in certain limits and is dual to 2-dimensional Jackiw-Teitelboim (JT) gravity, which is one of the simplest and most well-understood examples for AdS/CFT. If the model therefore can correct errors, this should also be the case for the gravity side of the duality. To explore this idea we constructed a MERA-like tensor network that exhibits some important properties of SYK and tested it for its error correction capabilities.

On a side note, we also employed the phase space formalism reviewed in this paper to evaluate the entanglement content of tripartite Clifford stabilizer states. Considerations from stabilizer tensor networks and AdS/CFT lead to the popular lore that the entanglement content of holographic spacetimes should be dominated by bipartite entanglement. This is because in the ER=EPR picture the wormholes related to tri- or multipartite entanglement are not classical solutions of general relativity and are therefore suppressed by the gravitational path integral \cite{Balasubramanian_2014} \cite{susskind2014}.
Usually those considerations are made for a large number of qudits as we naturally expect the Hilbert space of the bulk/boundary to be unimagineable in size. But it is also of interest for us if similar statements can be made for systems where the number of qudits is more manageable (e.g.\ condensed matter systems or quantum computers).