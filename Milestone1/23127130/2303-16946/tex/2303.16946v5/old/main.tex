\documentclass{article}
\usepackage[utf8]{inputenc}

\usepackage[letterpaper, portrait, margin=1in]{geometry}

\usepackage{parskip}
\usepackage{newpxtext,newpxmath}
\usepackage{csquotes}
\usepackage{graphicx}
\usepackage{hyperref}
\usepackage{amsmath}
\usepackage{braket}
\usepackage{mathdots}
\usepackage{color}

\usepackage{tikzit}
\usetikzlibrary{decorations.pathreplacing}
\input{default.tikzstyles}

\newcommand{\symp}[2]{\left\langle #1, #2 \right\rangle}
\DeclareMathOperator{\Tr}{Tr}
\DeclareMathOperator{\rank}{rank}
\newcommand{\bgs}[1]{\textcolor{magenta}{(brian: #1)}}

\title{SYK-Inspired Quantum Codes}
\author{Valérie Bettaque, Brian Swingle \\ Brandeis University}
\date{July 2022}

\begin{document}

\maketitle



Over the past few years, there has been a significant amount of research focused on studying the ReLU activation function, with the aim of achieving neural network convergence through over-parametrization. However, recent developments in the field of Large Language Models (LLMs) have sparked interest in the use of exponential activation functions, specifically in the attention mechanism.

Mathematically, we define the neural function $F: \R^{d \times m} \times  \mathbb{R}^d \rightarrow \mathbb{R}$ using an exponential activation function. Given a set of data points with labels $\{(x_1, y_1), (x_2, y_2), \dots, (x_n, y_n)\} \subset \mathbb{R}^d \times \mathbb{R}$ where $n$ denotes the number of the data. Here $F(W(t),x)$ can be expressed as $F(W(t),x) := \sum_{r=1}^m a_r \exp(\langle w_r, x \rangle)$, where $m$ represents the number of neurons, and $w_r(t)$ are weights at time $t$. It's standard in literature that $a_r$ are the fixed weights and it's never changed during the training. We initialize the weights $W(0) \in \mathbb{R}^{d \times m}$ with random Gaussian distributions, such that $w_r(0) \sim \mathcal{N}(0, I_d)$ and initialize $a_r$ from random sign distribution for each $r \in [m]$.

Using the gradient descent algorithm, we can find a weight $W(T)$ such that $\| F(W(T), X) - y \|_2 \leq \epsilon$ holds with probability $1-\delta$, where $\epsilon \in (0,0.1)$ and $m = \Omega(n^{2+o(1)}\log(n/\delta))$. To optimize the over-parametrization bound $m$, we employ several tight analysis techniques from previous studies [Song and Yang arXiv 2019, Munteanu, Omlor, Song and Woodruff ICML 2022]. 

 


\tableofcontents

\section*{Proposed Paper Structure}

\bgs{new outline}

Tensor network architecture:
\begin{itemize}
    \item desiderata: a representation of ground states of all-to-all models analogous to mera, should encode volume law entanglement, should also allow for degenerate ground states
    \item proposed architecture in its simplest form (figure)
    \item $n_0$ ground state qubits, increase "thermal" qubits by a factor of $r$ each layer, mix dof with a $D$ layer circuit
    \item entanglement, complexity, thermal entropy and comparison to syk
\end{itemize}

\noindent Clifford model:
\begin{itemize}
    \item take tensors to be random clifford gates 
    \item distance analysis
    \item stabilizer weight analysis
    \item comparison to quantum singleton bound 
    \item is the code degenerate or non-degenerate? if non-degenerate, compare to quantum Hamming bound
    \item since it is stabilizer, can also compare to gilbert-varshamov bound
\end{itemize}

\noindent Outlook
\begin{itemize}
    \item generalizations: analogue of branching MERA, most general architecture within our class
    \item possible future work on syk, majorana code extensions
    \item possible relation to emergent space in holography, e.g. JT gravity
    \item possible relation to emergence of bulk in low-dimensional holography
\end{itemize}



\section{Introduction}
\label{sec:introduction}
% \begin{itemize}
%     % Diffusion of FL
%     \item {\st{Diffusion of FL}}
%     % Security threats to FL
%     \item {\st{Security threats to FL with particular focus on model poisoning}}
%     % Limitations of existing countermeasures
%     \item {\st{Current countermeasures (e.g., KRUM) and their limitations}}
%     % Proposed method and its advantages
%     \item {\st{Intuitive description of the proposed method and its difference (i.e., advantages) w.r.t. state of the art}}
%     % Main contributions
%     \item {\st{Summary of the main contributions of this work}}
%     % Paper's structure and organization
%     \item {\st{Paper's structure and organization}}
% \end{itemize}

% Diffusion of FL
Recently, {\em federated learning} (FL) has emerged as the leading paradigm for training distributed, large-scale, and privacy-preserving machine learning (ML) systems~\cite{mcmahan2017googleai,mcmahan2017aistats}. 
The core idea of FL is to allow multiple edge clients to collaboratively train a shared, global model without disclosing their local private training data.
%Specifically, an FL system consists of a central server and many edge clients; 
A typical FL round involves the following steps: {\em(i)} the server randomly picks some clients and sends them the current, global model; {\em(ii)} each selected client locally trains its model with its own private data; then, it sends the resulting local model to the server;\footnote{Whenever we refer to global/local model, we mean global/local model {\em parameters}.} {\em(iii)} the server updates the global model by computing an \emph{aggregation function}, usually the average (FedAvg), on the local models received from clients.
% \begin{enumerate}
%     \item[{\em(i)}] the server sends the current, global model to the clients and appoints some of them for training;
%     \item[{\em(ii)}] each selected client locally trains its copy of the global model with its own private data; then, it sends the resulting local model back to the server;\footnote{Whenever we refer to global/local model, we mean global/local model {\em parameters}.}
%     \item[{\em(iii)}] the server updates the global model by computing an \emph{aggregation function} on the local models received from clients (by default, the average, also referred to as FedAvg~\cite{mcmahan2017aistats}).
% \end{enumerate}
This process goes on until the global model converges. %(e.g., after a certain number of rounds or other similar stopping criteria).
%\\
% The advantages of FL over the traditional, centralized learning paradigm are undoubtedly clear in terms of flexibility/scalability (clients can join/disconnect from the FL network dynamically), network communications (only model weights\footnote{We will use \textit{parameters} and \textit{weights} interchangeably.} are exchanged between clients and server), and privacy (each client's private training data is kept local at the client's end and not uploaded to the server).
\\
% Security threats to FL
%However, the growing adoption of FL also raises security concerns~\cite{costa2022covert}, particularly about its confidentiality, integrity, and availability.
Although its advantages over standard ML, FL also raises security concerns~\cite{costa2022covert}. %, particularly about its confidentiality, integrity, and availability~\cite{costa2022covert}.
% OLD, LONG VERSION
% Indeed, some work deals with privacy leakage that may expose the local data of some clients~\cite{melis2019sp}. 
% A large body of work, instead, investigates attacks that usually aim to detriment the predictive accuracy of the learned global model. For instance, \emph{data poisoning} attacks achieve this goal by letting an adversary pollute the training set of some corrupt FL clients with maliciously crafted examples~\cite{jagielski2018sp}.
% Similarly, in \emph{model poisoning} the attacker attempts to tweak the global model weights~\cite{bhagoji2019pmlr} by directly perturbing the local model's weights of some infected FL clients before these are sent to the central server for aggregation, usually via so-called Byzantine attacks. 
% It turns out that Byzantine model poisoning attacks severely impact standard FedAvg; therefore, more robust aggregation functions must be designed to make FL systems secure.
Here, we focus on \emph{untargeted model poisoning} attacks~\cite{bhagoji2019pmlr}, where an adversary attempts to tweak the global model weights %\footnote{We will use the terms \textit{parameters} and \textit{weights} interchangeably.} 
by directly perturbing the local model's parameters of some infected clients before these are sent to the central server for aggregation.
In doing so, the adversary aims to jeopardize the global model \textit{indiscriminately} at inference time.
Such model poisoning attacks severely impact standard FedAvg; therefore, more robust aggregation functions must be designed to secure FL systems.
\\
% In this paper, we focus on designing a novel robust aggregation scheme at the server's end to contrast the effect of Byzantine model poisoning attacks.
%
% Current countermeasures and their limitations
%Several countermeasures have been proposed in the literature to combat model poisoning attacks on FL systems.
% Some methods use simple statistics more robust than plain average to smooth the impact of malicious updates (e.g., Trimmed Mean and FedMedian~\cite{yin2018icml}). 
% Other defenses implement outlier detection techniques to discard malicious updates from the aggregation performed at the server's end. Those are either based on heuristics (e.g., Krum/Multi-Krum~\cite{blanchard2017nips} and Bulyan~\cite{mhamdi2018pmlr}) or data-driven approaches (e.g., K-means clustering~\cite{shen2016acm} or DnC via spectral analysis~\cite{shejwalkar2021ndss}). 
% Finally, some strategies rely on a centralized ``source of trust'' to spot potential malicious updates (e.g., FLTrust~\cite{cao2020fltrust}).
% Several countermeasures have been proposed in the literature to combat model poisoning attacks on FL systems, i.e., to discard possible malicious local updates from the aggregation performed at the server's end. 
% These techniques range from simple statistics more robust than plain average (e.g., Trimmed Mean and FedMedian~\cite{yin2018icml}) to outlier detection heuristics (e.g., Krum/Multi-Krum~\cite{blanchard2017nips} and Bulyan~\cite{mhamdi2018pmlr}) or data-driven approaches (e.g., spectral analysis via K-means clustering~\cite{shen2016acm} or spectral analysis), or methods based on ``source of trust'' (e.g., FLTrust~\cite{cao2020fltrust}).
% OLD, LONG VERSION
%Several countermeasures have been proposed in the literature to combat Byzantine model poisoning attacks on FL systems.
% Descriptive statistics
% For example, Trimmed Mean and FedMedian aggregate local model updates using more robust statistics than standard average~\cite{yin2018icml}.
%
% % Heuristics for outlier detection
% Many existing Byzantine-resilient strategies implement some outlier detection heuristics to discard the model updates sent by potentially malicious clients from the input of the aggregation function.
% One of the most popular heuristics is Krum~\cite{blanchard2017nips}.
% This strategy tries to mitigate the impact of Byzantine attacks by selecting as a global model the local model with the smallest sum of Euclidean distances to {\em all} the other local models.
% Although powerful, Krum requires the server to know (or, at least, estimate) the number of malicious FL clients upfront, which is generally impossible in a realistic attack scenario. %
% Moreover, Krum may become ineffective for complex, high-dimensional model parameter spaces due to the curse of dimensionality.
% Bulyan~\cite{mhamdi2018pmlr} tries to overcome this issue by combining Krum with a variant of Trimmed Mean.
% % Data-driven outlier detection
% Other strategies use data-driven outlier detection techniques -- e.g., via K-means clustering~\cite{shen2016acm} -- to spot potential malicious local model updates. 
% %For instance, Shen et al. propose to cluster local model updates with K-means and thus identify outliers.
%
% % Other techniques
% As far as the server is concerned, any local model received can be from a potential malicious client. 
% FLTrust~\cite{cao2020fltrust} assumes the server acts as a client, i.e., trains a local model on an additional {\em trustworthy} dataset at the server's end and compares it against all the local models from other clients. 
% This way, the server can rely on some ``source of trust'' when discarding potentially malicious clients.
%\\
% Limitations of existing Byzantine-resilient strategies
Unfortunately, existing defense mechanisms either rely on simple heuristics (e.g., Trimmed Mean and FedMedian by~\cite{yin2018icml}) or need strong and unrealistic assumptions to work effectively (e.g., foreknowledge or estimation of the number of malicious clients in the FL system, as for Krum/Multi-Krum~\cite{blanchard2017nips} and Bulyan~\cite{mhamdi2018pmlr}, which, however, cannot exceed a fixed threshold).
Furthermore, outlier detection methods using K-means clustering~\cite{shen2016acm} or spectral analysis like DnC~\cite{shejwalkar2021ndss} do not directly consider the temporal evolution of local model updates received.
Finally, strategies like FLTrust~\cite{cao2020fltrust} require the server to collect its own dataset and act as a proper client, thereby altering the standard FL protocol.
\\
% OLD, LONG VERSION
% Overall, existing Byzantine-resilient strategies are either simple heuristics (e.g., FedMedian) or, if they are more complex, they rely on strong and unrealistic assumptions to work effectively (e.g., knowing the number of malicious clients in the FL system in advance, as for Krum and alike).
% Furthermore, data-driven outlier detection methods do not consider the temporary evolution of local model updates received (e.g., K-means clustering). 
% Finally, strategies like FLTrust requires the server to collect its own dataset and act as a proper client, thereby altering the standard FL protocol.
%
% Description of the proposed method
This work introduces a novel pre-aggregation \textit{filter} robust to untargeted model poisoning attacks. Notably, this filter $(i)$ operates without requiring prior knowledge or constraints on the number of malicious clients and $(ii)$ inherently integrates temporal dependencies. 
The FL server can employ this filter as a preprocessing step before applying \textit{any} aggregation function, be it standard like FedAvg or robust like Krum or Bulyan.
Specifically, we formulate the problem of identifying corrupted updates as a multidimensional (i.e., matrix-valued) time series anomaly detection task. 
The key idea is that legitimate local updates, resulting from well-calibrated iterative procedures like stochastic gradient descent (SGD) with an appropriate learning rate, show \textit{higher predictability} compared to malicious updates. This hypothesis stems from the fact that the sequence of gradients (thus, model parameters) observed during legitimate training exhibit regular patterns, as validated in Section~\ref{subsec:intuition}. %until convergence. 
%This regularity may be more pronounced for smooth convex loss functions, but it can still be captured within an appropriate time window, even for more complex and convoluted loss surfaces. 
%We provide evidence of this claim in Appendix~B, where we show that the average mutual information (i.e., ``predictability''), calculated over pairs of legitimate model updates sent at different FL rounds, is significantly higher than the corresponding computation for a malicious client.
\\
Inspired by the matrix autoregressive (MAR) framework for multidimensional time series forecasting~\cite{chen2021je}, we propose the FLANDERS ({\em \textbf{F}ederated \textbf{L}earning meets \textbf{AN}omaly \textbf{DE}tection for a \textbf{R}obust and \textbf{S}ecure}) filter.
The main advantages of FLANDERS over existing strategies like FLDetector~\cite{zhao2020multivariate} are its resilience to large-scale attacks, where $50\%$ or more FL participants are hostile, and the capability of working under realistic non-iid scenarios.
We attribute such a capability to two key factors: $(i)$ FLANDERS works without knowing a priori the ratio of corrupted clients, and $(ii)$ it embodies temporal dependencies between intra- and inter-client updates, quickly recognizing local model drifts caused by evil players. Below, we summarize our main contributions:

\begin{itemize}
\item[{\em(i)}]
We provide empirical evidence that the sequence of models sent by legitimate clients is more predictable than those of malicious participants performing untargeted model poisoning attacks.
\\
\item[{\em(ii)}] 
We introduce FLANDERS, the first pre-aggregation filter for FL robust to untargeted model poisoning based on multidimensional time series anomaly detection.
\\
\item[{\em(iii)}] 
We integrate FLANDERS into Flower,\footnote{\scriptsize{\url{https://flower.dev/}}} a popular FL simulation framework for reproducibility.
\\
\item[{\em(iv)}] 
We show that FLANDERS improves the robustness of the existing aggregation methods under multiple settings: different datasets, client's data distribution (non-iid), models, and attack scenarios.
\\
\item[{\em(v)}] 
We publicly release all the implementation code of FLANDERS along with our experiments.\footnote{\scriptsize{\url{https://anonymous.4open.science/r/flanders_exp-7EEB}}}
\end{itemize}

% Paper's structure and organization
The remainder of the paper is structured as follows. %some related work and the current state-of-the-art solutions to security issues that FL entails. 
Section~\ref{sec:background} covers background and preliminaries. 
In Section~\ref{sec:related}, we discuss related work.
Section~\ref{sec:problem} and Section~\ref{sec:method} describe the problem formulation and the method proposed. % to tackle it. 
Section~\ref{sec:experiments} gathers experimental results. %, and Section~\ref{sec:limitations} discusses some limitations of this work.
Finally, we conclude in Section~\ref{sec:conclusion}.
 %discusses the limitations of this work and draws future research directions.
%reports conclusions and draws perspectives for future research directions.

%%%%%%% OLD %%%%%%%
%to overcome the resilience of Byzantine failures in distributed Stochastic Gradient Descent computations. 
% The strength of Krum is its time complexity, which is linear in the gradient dimension. 
% However, the robustness of the approach is guaranteed for gradient-based learning applications only when the majority of the clients are not compromised. 
% Besides, the aggregation mechanism of Krum, as well as that of similar methods, is robust from a coarse-grained perspective and does not provide solutions to errors and perturbations that may occur at inference time.
%A related approach to~\cite{blanchard2017nips} is the work of Su et al.~\cite{su2016dc}. Here, the authors propose an iterated approximate agreement to tackle a multi-layer scenario attacked by Byzantine agents. 
%However, the method works efficiently on the sole discrete context and it is inapplicable to continuous state environments.
%\gabri{Maybe, we should just talk about the main limitations of existing countermeasures without digging into their details (or, we can just mention Krum as this is the most popular one). I will move the description of all these methods to the Related Work section.}
\section{Quantum Error Correction}
\label{section:disc}

\subsection{Stabilizer Codes}
\label{sec:stab}

A (quantum) stabilizer code $[[n, k, d]]$ that encodes $k$ physical qudits into $n$ physical qudits with code distance $d$ is defined in terms of a \emph{stabilizer group} $S$, which is an abelian subgroup of the Pauli group $P_n(d)$ i.e.\ the group generated by all possible $n$-element tensor products of the Pauli operators ($d = 2$)
\begin{equation}
    X = \begin{pmatrix}
        0 & 1 \\ 1 & 0
    \end{pmatrix}, \,
    Y = \begin{pmatrix}
        0 & -i \\ i & 0
    \end{pmatrix}, \,
    Z = \begin{pmatrix}
        1 & 0 \\ 0 & -1
    \end{pmatrix}
\end{equation}
or their higher-dimensional counterparts ($d > 2$), which are defined in section \ref{sec:weyl_rep} \cite{qldpc}. The stabilizer group must therefore be generated by $(n - k)$ independent elements of $P_n$. A \emph{code word} is a state vector $\ket{\psi} \in \mathbb{C}^{d^n}$\ footnote{$d$ here is the dimension of the Hilbert space of a single qudit and not the code distance.} that satisfies $s \ket{\psi} = \ket{\psi}$ for all $s \in S$. The space spanned by all possible code words is called the \emph{code space} and has dimension $k$. The operators mapping logical states to other logical states are called \emph{logical operators} and must therefore commute with all elements of the stabilizer group.

Some important quantities in relation to stabilizer codes are
\begin{itemize}
    \item \textbf{operator weight:} The number of elements of $P_1(d)$ in the tensor product of the operator that are not proportional to the identity operator $I$.
    \item \textbf{code weight:} The largest weight an element of the corresponding stabilizer group exhibits.
    \item \textbf{code distance:} The minimal weight of all non-trivial logical operators. Equivalently it can be defined as the minimum number of qudits that have to be changed to arrive at another code word.
\end{itemize}
A \emph{quantum low-density-parity-check} (qLDPC) code is a code with constant weight, regardless of the code length $n$. This means that each stabilizer only acts on a constant number of qudits and each qudit is acted on by only a constant number of stabilizer elements. A \emph{good} qLDPC code has its code space dimension $k$ and distance $d$ scale linearly with the length $n$.

\subsection{Decoupling \& Code Distance}

Determining the distance for a stabilizer code and how it scales with the number of physical qudits is not a trivial thing to do. We present here one way, called the \emph{adversarial approach}, which makes use of analysing the mutual information
\begin{equation}
\label{eq:mut_inf}
    I(A, R) = S(A) + S(R) - S(AR)
\end{equation}
between all possible subsystems $A$ of the physical qudits and some external adversary $R$ who is initially maximally entangled with the code word. A depiction of the setup can be found in figure \ref{fig:decoupling}.
\begin{figure}[hbt]
    \centering
    \tikzfig{figures/decoupling}
    \caption{Circuit representation of the adversarial approach for determining the distance of a stabilizer code. Here $U_{MN}$ is a unitary that takes states of the form $\ket{\psi_{\text{anc}}}_M \otimes \ket{\psi_{\text{code}}}_N$ and maps $\ket{\psi_{\text{code}}}$ to the code space of the chosen stabilizer state. $\ket{\psi_{\text{anc}}}$ is an arbitrary state of ancilla qudits and can therefore be ignored. The code distance is then one less than the size of the smallest choice for $A$ that has non-vanishing mutual information $I(A, R)$ with $R$. The interpretation is that $R$ then has access to enough qudits such that they can change the initial code word.}
    \label{fig:decoupling}
\end{figure}

As can be seen there, $R$ remains maximally entangled with the logical state $\ket{\psi_{\text{code}}}$ even after it has been encoded into the code space by an appropiate unitary $U_{MN}$. The question is therefore what the minimum number of physical qudits $A$ is that the adversary $R$ has to \enquote{steal} to get access to one logical qudit, thus being able to change it. A measure that indicates for which choices of $A$ this is the case is the aforemention mutual information \eqref{eq:mut_inf}, which becomes non-zero in such a case. In that case one says that $A$ is \emph{decoupled} from the rest of the physical qudit space, meaning that the state can be written as a tensor product on $A$ and its complement $MN / A$. The code distance is therefore
\begin{equation}
    d = |A^*| - 1,
\end{equation}
where $A^*$ is the smallest set of physical qudits which satisfies $I(A^*, R) \neq 0$. Since the code word is assumed to be maximally entangled with $R$, this statement is true for all possible choices of $\ket{\psi_{\text{code}}}$.

Implementing this approach as an algorithm is problematic though since iterating about all possible choices for $A$ is combinatorically intensive. A way to simplify the procdeure at the cost of only getting an upper bound approximation for the code distance is by randomly sampling choices for A and determining the smallest one which has non-vanishing mutual information.
\section{Motivating the SYK Code}
\label{section:syk}

\subsection{The SYK Model}

As is well known \cite{Maldacena_2016}, the SYK model consists of $N$ Majorana fermions $\psi_i$ which have their dynamics determinded by the Hamiltonian
\begin{equation}
    H = (i)^{q/2} \sum_{1 \leq i_1 \leq i_2 \leq \ldots \leq i_q \leq N} j_{i_1 i_2 \ldots i_q} \psi_{i_1} \psi_{i_2} \ldots \psi_{i_q},
\end{equation}
where $q$ is an even number. The power of $i$ is necessary to ensure that $H$ is hermitean if $q = 2 \mod{4}$. This implies that the system is not symmetric under time reversal if $q/2$ is odd. If one therefore restricts themselves to the time reversal symmetric case, then the case $q=4$ dominates the interactions at low energy. 

Additionally, the coefficients $j_{i_1 i_2 \ldots i_q}$ are drawn from a random Gaussian distribution of real-valued variables such that they have zero mean and variance
\begin{equation}
    \braket{j^2_{i_1, i_2, \ldots, i_q}} = \frac{J^2 (q-1)!}{N^{q-1}},
\end{equation}
where $J$ is a scalar which is taken to be the same for all coefficients. The additonal factors involving $q$ and $N$ are chosen such that it simplifies the large $N$ limit.

\subsection{The MERA Tensor Network}

An important property of the SYK model is that it becomes approximately conformal in the large $N$ limit, which is the reason why AdS/CFT tells us that it is dual to JT gravity. But this connection between SYK models and CFTs is also interesting since the latter can be approximated as a MERA tensor network. One can therefore ask if something similar can be done for SYK models in general.

Looking at an example for a MERA network in figure \ref{fig:mera} it is clear that the circuit exhibits some form of self-similarity. This is necessary to ensure the field theory being approximated is renormalizable. A fine graining from the top of the circuit towards the bottom is also noticeable and can be interepreted as some code space with small Hilbert space dimension being encoded in a larger system of physical qudits. Since this is what we want to achieve, the SYK circuit has to obey these heuritic properties as well.
\begin{figure}[hbt]
    \centering
    \includegraphics[width=.9\linewidth]{figures/mera.png}
    \caption{A visual representation of a MERA tensor network approximating a conformal field theory \cite{Bao_2015}. The triangles and squares can be thought of as creating entanglement and fine-graining the encoded information (by adding more ancillary states) respectively when looked at from the moving-up perspective. Because of the structure of the circuit the information can't distribute faster than indicated by the causal cone (shaded in red).} 
    \label{fig:mera}
\end{figure}

Another important property is the scaling of the (thermodynamic) entropy with the temperature $T$ when an IR cutoff scale $\Lambda$\footnote{This can be interpreted as \enquote{cuting off} all the lower layers of the tensor network starting from a certain layer.} is introduced. In the case of the MERA network this scaling is given by
\begin{equation}
\label{eq:mera_scaling}
    S_{\text{th}} \sim \frac{n}{2^{\ell}} \sim n \left(\frac{T}{\Lambda}\right)^{\#},
\end{equation}
where $n$ is the number of qudits on the lowest layer and $\ell$ is the number of layers remaining after the cutoff. The latter can be found to be
\begin{equation}
\label{eq:layers}
    \ell = \log_2\left( \frac{\Lambda}{T} \right),
\end{equation}
which leads to the second expression in \eqref{eq:mera_scaling} ($\#$ is just an unspecified exponent).

\subsection{The SYK-Inspired Tensor Network Code}

\begin{equation}
    \widetilde{Z}_{\ell, i} = U_{\ell} \, Z_i \, U^{\dagger}_{\ell}
\end{equation}

\begin{equation}
    \Pi_{\ell, i} = I - \widetilde{Z}_{\ell, i}
\end{equation}

\begin{equation}
    \Pi_{\ell} = \bigotimes_{i = 1}^{2^{\ell}} \Pi_{\ell, i}
\end{equation}




\subsection*{OLD}

The task is now to construct a tensor network model that has approximately the same properties as its MERA counterpart, but with the correct scalings for the thermal entropy (and complexity) as expected from estimates for SYK and JT gravity \cite{Brown_2019}. We propose that such a network is given by the circuit in figure \ref{fig:syk_circuit}, where each unitary is modeled by a scrambling circuit with depth $D$ as seen in \ref{fig:wall_circuit}. 
\begin{figure}[hbt]
    \centering
    \tikzfig{figures/syk_circuit}
    \caption{The SYK tensor network taking a code word $\ket{\psi_{\text{code}}}$ and $n_{\text{anc}}$ ancilla qudits and scarmabling them into a system of physical qudits $\ket{\Psi_{\text{phys}}}$. We assume that for this paper that we have $n_{\text{anc}} = 2^{\ell}$ such that all input qudits experience some scrambling in the end. The scrambling unitary at each step is of the form depicted in figure \ref{fig:wall_circuit}.} 
    \label{fig:syk_circuit}
\end{figure}

Assuming an IR cutoff was already performed at the lowest level, $\ell$ is again the number of layers in the circuit and also determined by \eqref{eq:layers}. Self-similarity is then achieved if one sets $n_{\text{anc}} = 2^{\ell}$. Going upwards from the bottom the information is here too entangled by the unitaries and fine-grained due to the stepwise addition of new ancillary states.

\begin{figure}[hbt]
    \centering
    \tikzfig{figures/wall_circuit}
    \caption{An example quantum circuit with depth $D = 5$ used to simulate the rapid scrambling of quantum information. Each gate represents a randomly sampled unitary operator acting on two qudits. If the circuit wires are fixed then this code also exhibits some form of causal cone. But one can also randomly choose which pairs of qudits to scramble at each layer of the circuit.}
    \label{fig:wall_circuit}
\end{figure}

What remains to be checked is the scaling of the entropy, which can be easily seen to satisfy
\begin{equation}
    S_{\text{th}} \sim n_{\text{code}} + \frac{n}{k^{\ell}} \sim n_{\text{code}} + n_{\text{anc}} \left(\frac{T}{\Lambda}\right)^{\#},
\end{equation}
which is similar to what one would expect according to \cite{Brown_2019} \cite{Brown_2016}. We therefore argue that the network shown in \ref{fig:syk_circuit} does indeed satisfy the requirements for modelling a SYK system.


\section{Results}
\label{results}

\begin{figure*}[ht]
    \centering
    \includegraphics[scale=0.15,trim={0 2.5cm 0 5cm},clip]{images/aoi-single_burst}
    \caption{The time average peak Age of Information with burst and \gls{soa} loss values against the dynamic reliability logic for different network topologies.}
    \label{fig:aoi_burst}\vspace{-0.4cm}
\end{figure*}


This paper focuses on both transport layer and application layer metrics to determine the feasibility of dynamic reliability. For this, we have selected the session packet volume, as transmitted, retransmitted, lost and backlogged packets as \glspl{kpi} for the transport layer; while focusing on the \gls{aoi} for the application layer. The \gls{aoi} was chosen as a crucial indicator for the freshness of packets in real-time applications. More specifically, this work adopts the time average peak \gls{aoi} equation \cite{aoi_equation} depicted in Eq. \ref{aoi}, where $\Delta(r_{i+1})$ is the $i$th update at the time it was received at the server, for a session time period of $\tau$.

\begin{equation}
    \label{aoi}
    \gls{aoi}_\tau = \frac{1}{n-1}\sum_{i=1}^{n-1} \Delta(r_{i+1})
\end{equation}

We include a comparison between the vanilla QUIC implementation which does not enjoy the dynamic reliability extension, with a number of dynamic reliability policies. The tests were run a number of times for statistical significance, with the mean value of vanilla implementation used as a baseline for comparison. The topology utilised both random loss and bursty loss to explore the bounds of dynamic reliability. The \gls{soa} loss in the figures correspond to the loss values presented in Table. \ref{tab:path_char}, for ease of comparison between bursty and random loss scenarios.

\subsection{Transport-Layer KPIs}

To analyse the performance gain at the transport layer due to dynamic reliability, the volume of transmitted and backlogged packets is examined. The figures are in the form of boxplots, which take the vanilla implementation as a benchmark, depicted as the red dashed line.

As seen in Fig. \ref{fig:sent_burst}, the loss plays a crucial role in the performance of the reliability policies. The policies under random loss did incredibly well for the networks with a larger capacity, namely \gls{mmwave} and Sub-6~GHz, whereas for burst loss, the lower network capacities had a larger packet reduction. With the increase in burst loss, the behaviour of the set split reliable policies became unpredictable, if a reliable assignment happened to coincide with a burst loss, the number of transmitted packets increases, and vice versa. On the other hand, in smarter policies, such as Loss-Aware, the performance lightly matched the vanilla baseline, as the reliable assignment dominated the session to compensate for a higher burst loss. Not only that but, the burst loss also impacted the variance of the transmitted packets for the policies.

Unsurprisingly, the unreliable focused policy, 80-20 split, outperformed other policies for all topologies in random and bursty loss scenarios, with an approximate reduction of 80\%. That being said, the majority of the policies reduced the transmitted packets on the link by approximately 70\% for random loss, while the reduction started at $\approx 15\%$ and decreased as the loss increased for the burst loss scenario.

The retransmitted and lost packets, not shown due to space limitations, followed the same trend as the transmitted packets for the random loss scenarios. However, for the burst loss scenarios, the larger capacity networks had a lower reduction in the retransmitted and lost packets. This can be seen as a favorable outcome since the lower capacity networks are scarce on resources. It is important to note that the Loss-Aware policy mimicked the vanilla approach as the burst loss increased, signifying the overwhelming appointment of reliable packets in adapting to the harsh burst loss conditions.
 
Alternatively, Fig. \ref{fig:backlog_burst} clearly shows a stark comparison between the policies and loss scenario in the reduction of the backlogged packets. The Loss-Aware policy for random loss scenario reduced the backlogged packets by up to 50\%, beating all other policies by approximately 30\%. Furthermore, it is clear that the unreliability focused policies resulted in the lowest backlog for the session. In comparison, we notice that the burst loss and the backlogged frequency have a positive correlation, where the maximum reduction of the backlogged packets for the policies is at most 20\%. Much like the transmitted packets, the probability of a burst loss occurrence plays a vital role in the number of retransmissions sent and by extension the number of backlogged packets. Thus, we can conclude that the stress placed on the buffer is a result of the reliable packets which is tightly coupled with the congestion on the session. Whereas, unreliable focused policies did not encounter such a phenomenon regardless if it was experiencing a burst loss.


\subsection{Application-Layer KPIs}

The feasibility of dynamic reliability for real-time applications can be determined by the \gls{aoi}, with comparison across different topologies and policies. If we take a strict approach and consider anything below $10$~ms is real-time \cite{real-time}, then all the reliability policies passed that requirement, which is attractive for real-time applications, as shown in Fig. \ref{fig:aoi_burst}. Utilising the median as an estimate of the runs, the policies in the WLAN and Sub-6~GHz topology with random loss floated around $4-5$~ms with negligible difference, while the \gls{aoi} for \gls{mmwave} was $\approx 2-3$~ms. It is clear that the \gls{aoi} and the network capacity have a negative correlation, as the network capacity decreases, the \gls{aoi} increases. The same correlation is extended to the bursty loss scenarios, where \gls{mmwave} dominated the other topologies. That being said, it is crucial to note that the \gls{aoi} for the reliability policies is often slightly better than or equal to the \gls{aoi} of the vanilla implementation, proving that dynamic reliability reduces the congestion of the session at no cost to the \gls{aoi}.

\section{Conclusion}\label{sec:conclusion}
In this work, we focus on addressing the fundamental challenge of OOD detection tasks, which is how to fully understand the semantic discrepancy between the ID/OOD samples. We reveal that the key to success in the realistic SCOOD task is to allocate as many ID samples in the unlabeled set correctly as possible. To this end, we propose a novel uncertainty-aware optimal transport scheme that introduces class-specific energy scores as guidance for effective label assignment. Experimental results show that our method achieves better performance than previous state-of-the-art methods on SCOOD benchmarks.

\textbf{Limitations.} In addition to temperature scaling, other techniques such as feature clipping applied in ReAct~\cite{sun2021react} also enhance the performance of energy score, so how to obtain an OOD score that best fits the SCOOD task can be further explored. Moreover, a setting highly related to SCOOD has been proposed in \cite{katz2022training} and formulated as a constrained optimization problem. We will also theoretically analyze these practical OOD settings in our feature work.

% \section*{Acknowledgments}
\textbf{Acknowledgments.} 
This work is supported by National Key R\&D Program of China under Grant 2020AAA0105701, National Natural Science Foundation of China (NSFC) under Grants 61872327, Major Special Science and Technology Project of Anhui, National Natural Science Foundation of China (62033012) and Ant Group through Ant Research Intern Program.

\chapter*{Acknowledgement}
\addcontentsline{toc}{chapter}{Acknowledgement}
The authors thank Andrzej Kupsc, Sergey Barsuk, Olivier Callot and Wolfgang K{\"u}hn for their contribution on the CDR draft.
%The authors thank the international review committee XXX for their great effort in reading the CDR draft and providing valuable suggestions. 
The STCF working group thanks all 
the colleagues in the world-wide community for many profitable discussions
and expresses gratitude to the Hefei Comprehensive National Science Center for their strong support.  This work is supported by: international 
partnership program of the Chinese Academy of Sciences Grant No. 211134KYSB20200057.

\bibliographystyle{plain}
\bibliography{refs}

\appendix

\section{Phase Space Formalism}
\label{sec:phase_space}

\subsection{Weyl Representation}
\label{sec:weyl_rep}

Given a Hilbert space $\mathcal{H}$ of prime dimension $d > 2$ \footnote{The case of $d=2$ is excluded here since our choice of representation requires the existence of a 2-element in the group sucht that $\frac{1}{2} = \frac{d + 1}{2}$, which is only the case for d>2. This should not affect the phsyics though as systems of different qudits can always be mapped to each other.}, we choose a basis $\{\ket{0},\ket{1}, \ldots, \ket{d-1}\}$ with its states being labeled by the elements of the associated finite (Galois) field GF($d$)\footnote{Finite fields also exist for powers of primes i.e.\ GF($d^k$), but addition and multiplication does not happen mod $d^k$ in these cases. One can achieve the same group order though by instead using $k$ qudits with each being represented by a copy of GF($d$)}.  One can then introduce \emph{clock and shift operators} $Z, X$ which act on the basis states according to \cite{hudson}
\begin{equation}
\label{eq:boost_shift}
    Z^p \ket{k} = \chi(p \cdot k) \ket{k}, \quad X^q \ket{k} = \ket{k + q},
\end{equation}
where $p, q, k \in \text{GF}(d)$ and $\chi(k) = e^{2 \pi i k / d}$. Note that addition and multiplication happens over GF($d$) and is thus mod $d$. This is also respected by our choice for $\chi(k)$ since $\chi(k + d) = \chi(k)$ even for addition without modulo.

We are now able to define the so-called \emph{Weyl operators} for a single qudit, which provide a generalisation of the Pauli operators on a qubit:
\begin{equation}
\label{eq:weyl_single}
    w(p, q) = \chi\left(-\frac{p \cdot q}{2}\right) \, Z^p \, X^q, 
    \quad p,q \in \text{GF}(d).
\end{equation}
Extending this definition to $n$ qudits is as easy as tensoring $n$ copies of \eqref{eq:weyl_single}, which we write as
\begin{align}
\label{eq:weyl_multi}
    \begin{split}
        w(v) &= w(p_1, q_1, \ldots, p_n, q_n) \\
        &= w(p_1, q_1) \otimes \ldots \otimes w(p_n, q_n).
\end{split}
\end{align}
Each Weyl operator is therefore uniquely represented by an element $v$ of a $2n$-dimensional vector space $V$ over the field GF($d$). Using the commutation relations of $Z^p$ and $X^q$ that arise from their definition in \eqref{eq:boost_shift}, it also follows that
\begin{equation}
\label{eq:weyl_mul}
    w(v) \, w(w) = \chi \left( \frac{\symp{v}{w}}{2} \right) \, w(v + w),
\end{equation}
where $\symp{\cdot}{\cdot}$ is the \emph{symplectic product} on $V$, which obeys $\symp{v}{w} = -\symp{w}{v}$ and can be expressed as a matrix product:
\begin{equation}
\label{eq:symp_prod}
    \symp{v}{w} = v^T J w, \quad J = \begin{pmatrix}
        0 & 1 \\ -1 & 0
    \end{pmatrix}^{\oplus n}.
\end{equation}
Because of that the Weyl operators form a projective representation of the associated vector space $V$ equipped with a symplectic product. It is also noteworthy that \eqref{eq:weyl_mul} implies that two Weyl operators $w(v), w(w)$ commute if and only if the corresponding symplectic product $\symp{v}{w}$ vanishes.

Another useful indentity which we will use later is the fact that
only the identity $I_n = w(0)$ has a non-vanishing trace:
\begin{equation}
    \label{eq:weyl_trace}
        \Tr[w(v)] = d^n \delta_{v,0}.
\end{equation}
This is trivial to show for $X^q$ but requires using the fact that the Kronecker delta can be written as
\begin{equation}
\label{eq:kronecker_sum}
    \delta_{p,0} = \frac{1}{d} \sum_{k = 0}^{d-1} e^{\frac{2 \pi i k}{d} p}
\end{equation}
to prove it for $Z^p$ as well. 

\subsection{The Clifford Group}

The Clifford group is a subset of the unitary group which maps Weyl operators to other Weyl operators (up to a factor):
\begin{equation}
\label{eq:clifford_def}
    U w(v) U^{\dagger} = c(v) \, w(S(v)),
\end{equation}
for some $c: V \rightarrow \mathbb{C}$ and $S: V \rightarrow V$. Because $S$ therefore has to be compatible with \eqref{eq:weyl_mul}, it is easy to see that it has to be linear and preserve the symplectic product:
\begin{equation}
    \symp{S v}{S w} = \symp{v}{w} \quad \forall \, v,w \in V.
\end{equation}
In matrix representation, one can also equivalently state this property as $S^T J S = J$. Such a function is called \emph{symplectic}. The set of all symplectic functions for a given vector space $V$ forms the so-called \emph{symplectic group}\footnote{Note the similarities to the definition of the orthogonal group. In fact, the column entries of a symplectic matrix also form as (symplectic) basis $(e_1, f_1, \ldots, e_n, f_n)$ of $V$ which satisfies $\symp{e_i}{e_j} = 0 = \symp{f_i}{f_j}$ and $\symp{e_i}{f_j} = \delta_{ij}$ for all $i,j = 1, \ldots, n$. Applying a symplectic is therefore equivalent to a change of basis. \label{fn:symp_basis}}. 

In general, the structure of the Clifford group is completely determined by the following statements:
\begin{enumerate}
    \item For any symplectic $S$ there is a unitary operator $\mu(S)$ satisfying
    \begin{equation}
        \mu(S) w(v) \mu(S)^{\dagger} = w(S v) \quad \forall \, v \in V.
    \end{equation}
    \item $\mu(S)$ is a projective representation of the symplectic group, meaning
    \begin{equation}
        \mu(S) \mu(T) = e^{i \phi} \mu(S T)
    \end{equation}
    for some phase $\phi$.
    \item Up to a phase, any Clifford operator is of the form
    \begin{equation}
        U = w(a) \mu(S)
    \end{equation}
    for a suitable $a \in V$ and symplectic $S$.
\end{enumerate}
A proof of these statements can be found in \cite{hudson}. Note that this also fixes the factor from \eqref{eq:clifford_def} to be $c(v) = \chi(\symp{a}{Sv})$. 

\subsection{Stabilizer States and Codes}

As mentioned before, a vanishing symplectic product $\symp{v}{w}$ is equivalent to a vanishing commutator $[w(v), w(w)]$. One can therefore construct a set
\begin{equation}
    w(M) = \{ m \,|\, m \in M\}
\end{equation}
containing only commuting Weyl operators by choosing $M$ to be a subspace of $V$ satisfying
\begin{equation}
    \symp{m_i}{m_j} = 0 \quad \forall \, m_i, m_j \in M
\end{equation}
Such a subspace is called \emph{isotropic} and it is easy to see that it also forms a group under vector addition since the symplectic product is bilinear. The cardinality of isotropic subspaces can range between 0 and $d^n$ as there are at most $n$ elements with mutually vanishing symplectic product in a $2n$-dimensional symplectic basis (see footnote \ref{fn:symp_basis} for the reason). We will refer to $M$ having maximal cardinality as \emph{maximally isotropic}.

In general it is convenient to write the basis elements of an isotropic subspace as a $k \times 2n$ (or $2n \times k$) matrix over $GF(d)$, where $k = \log_d(M)$ is the size of the basis. In the literature this is called the \emph{stabilizer matrix}, although there it is often written in terms of the actual Pauli/Weyl operators and not their symplectic representation.

Isotropy of $M$ allows one to (at least partially) diagonalize the Weyl operators contained in $w(M)$, even completely if $M$ is maximally isotropic. In the latter case it is therefore possible to define a unique quantum state $\ket{M, v}$ in terms of the elements in $w(M)$ acting on it as stabilizers:
\begin{equation}
\label{eq:stab_def}
    \chi(\symp{v}{m}) w(m) \ket{M, v} = \ket{M, v} \quad \forall \, m \in M.
\end{equation}
The vector $v \in V$ therefore determines the phase differences between the eigenstates assocated with $w(M)$. A state satisfying \eqref{eq:stab_def} is called a \emph{stabilizer state} and can be written as
\begin{equation}
\label{eq:stab_state}
    \ket{M, v}\bra{M, v} = \frac{1}{d^n} \sum_{m \in M} \chi(\symp{v}{m})\, w(m).
\end{equation}
It is easy to show that \eqref{eq:stab_state} is a projection operator and has unit trace by applying \eqref{eq:weyl_trace} and using the fact that $M$ is a group and thus satisfies $M + m = M$ for all $m \in M$.

In fact, even for a non-maximally isotropic subspace $M$ would \eqref{eq:stab_state} still be a projector (up to normalization), but not a quantum state anymore. In this more general case we have
\begin{equation}
\label{eq:stab_proj}
    \Pi(M,v) = \frac{1}{|M|} \sum_{m \in M} \chi(\symp{v}{m})\, w(m)
\end{equation}
with $\Tr[\Pi(M,v)] = \frac{d^n}{|M|}$. All states in the subspace which $\Pi(M,v)$ projects onto therefore satisfy \eqref{eq:stab_def}, meaning that they form a code space. We can therefore identify this case as being a stabilizer code since it satisfies the definition in section \ref{sec:stab}. Even though finding stabilizer codes therefore just amounts to making a choice for $M$ and $v$, it does not ensure that the resulting code is good in the sense that its Hamming distance might be small or does not scale well.

\subsection{Entanglement Entropy of Stabilizer States}
\label{sec:stab_entropy}

Thanks to the structure of the symplectic product \eqref{eq:symp_prod} and the multi-particle Weyl operators defined in \eqref{eq:weyl_multi}, one can easily take the partial trace of \eqref{eq:stab_state} over a desired subsystem $B$ by writing $v = v_A \oplus v_B$ (same for $m$) and $w(m) = w(m_A) \otimes w(m_B)$ for all $m \in M$ and applying \eqref{eq:weyl_trace} to the latter term in the tensor product. The resulting reduced state is then
\begin{align}
\begin{split}
\label{eq:stab_reduced}
    \rho_A &= \Tr_B[\rho] \\
    &= \frac{d^{n_B}}{d^n} \sum_{m_A \in M_A} \chi(\symp{v_A}{m_A})\, w(m_A) \\
    &\equiv \frac{|M_A|}{d^{n_A}} \, \Pi(M_A, v_A),
\end{split}
\end{align}
where we made use of the fact that $n = n_A + n_B$ and identified \eqref{eq:stab_proj}, but this time in terms of $v_A$ and
\begin{equation}
\label{eq:M_reduced}
    M_A = \{ m_A \,|\, m_A \oplus 0_B \in M\}.
\end{equation}
This is possible since the definition of $M_A$ ensures that it is again a group (although not necessarily maximally isotropic)\footnote{Naively computing $M_A$ using \eqref{eq:M_reduced} is not efficient as such an algorithm would have $\mathcal{O}(d^n)$ runtime. A runtime that is polynomial in the system size can be achieved by instead permuting the sites that are to be traced out to the front the stabilizer matrix and then computing its reduced row echolon form. The basis vectors $b = b_A \oplus b_B$ for which $b_B \neq 0$ are then removed and for the remaining elements only $b_A$ is being considered.}.

The fact that even after tracing out a subsystem the resulting reduced state is still proportional to a projection operator makes computing the entanglement entropy straightforward. While it is possible to just directly evaluate the Von-Neumann entropy $S(A) = \Tr[\rho_A \log_d(\rho_A)]$, a more elegant and insightful approach can be made by instead considering the \emph{R\'enyi entropies}
\begin{equation}
    S^{(n)}(A) = \frac{1}{1 - n} \log_d \Tr[\rho_A^n],
\end{equation}
which have the property that
\begin{equation}
    \log_d(d^{n_A}) = S^{(0)}(A) \geq S(A) \geq S^{(2)}(A) \geq \ldots 
\end{equation}
where $S(A) = S^{(1)}(A) = \lim_{n \rightarrow 0} S^{(n)}(A)$ reproduces the ordinary Von-Neumann entropy. What makes the R\'enyi entropies interesting here is that they satisfy $S^{(n)}(A) = \log_d (\rank\rho_A)$ for all $n>0$ if the state being considered has a flat entanglement spectrum i.e.\ it is proportional to a projection operator\footnote{The proof is straightforward: Let $\rho_A = \alpha \cdot \Pi_A$, then $S^{(n)}(A) = \frac{1}{1 - n} \log_d \Tr[(\alpha \cdot \Pi_A))^n] = \frac{1}{1 - n} \log_d (\alpha^n \cdot \Tr[\Pi_A]) = \frac{1}{1 - n} \log_d (\alpha^n \cdot \rank\rho_A)$. Since $\alpha = (\rank \rho_A)^{-1}$ because of normalization we have $S^{(n)}(A) = \frac{1}{1 - n} \log_d (\rank\rho_A)^{1-n} = \log_d (\rank\rho_A)$.}. Since this is the case for the reduced stabilizer state we can use the fact that $\rank \rho_A = \frac{d^{n_A}}{|M_A|}$ to show that
\begin{equation}
    S(A) = n_A - \log_d |M_A|.
\end{equation}
If the number of basis vectors $k_A = \log_d |M_A|$ is known, then computing $S(A) = n_A - k_A$ is straightforward and numerically stable\footnote{As a sanity check, note that if $\rho_A$ is pure and therefore has $S(A) = 0$ it implies that $k_A = n_A$, which is the requirement for $M_A$ to be a maximally isotropic subspace and thus to define a (pure) stabilizer state.}.
\section{Scaling of the Stabilizer Entropy}
\label{sec:entropy_scaling}

\subsection{The Stabilizer Hamiltonian}

Every set of stabilizers fixing a quantum state or a space of quantum states can be expressed as a projective Hamiltonian that has said states as part of its (degenerate) ground space.

Let $N$ be the total number of physical qudits and $k \leq N$ the number of logical qudits needed to represent the code word $\ket{\psi}_{\text{code}}$. The case of $N = k$ is not interesting to us so we assume we have $N - k > 0$ ancillary qudits. Every quantum code can then be written as a unitary $U$ satisfying
\begin{equation}
    \ket{\Psi} = U \left( \ket{\psi}_{\text{code}} \otimes \ket{0}_{\text{anc}}^{\otimes \, (N - k)} \right),
\end{equation}
where $\ket{\Psi}$ is the code word encoded in the space of physical qudits. The ancillary thermal qudits can each be fixed to be $\ket{0}$ without loss of generality.

Before applying the encoding unitary, it is easy to see that the generating set of stabilizers fixing the code space spanned by all possible choices of $\ket{\psi}_{\text{code}} \otimes \ket{0}_{\text{anc}}^{\otimes \, (N-k)}$ is given by
\begin{equation}
\label{eq:pre_code_stab}
    Z_i \equiv I_{\text{code}} \otimes I^{\otimes \, (i-1)} \otimes Z \otimes  I^{\otimes \, (N-k-i)}, \quad i = 1,\ldots,N-k, 
\end{equation}
where the $Z$ acts on the $i$th qudit of the ancillary system. From this it immediately follows that the stabilizers acting on the physical qudits can be retrieved by applying the code unitary such that
\begin{equation}
    \widetilde{Z}_i = U \, Z_i \, U^{\dagger}.
\end{equation}

To construct the stabilizer Hamiltonian though, we have to use the (disjoint) projectors associated to our chosen stabilizer basis. Analogously to the previous case, before encoding the state they are
\begin{equation}
\label{eq:pre_code_proj}
    P_i \equiv I_{\text{code}} \otimes I^{\otimes \, (i-1)} \otimes \ket{0}\bra{0} \otimes I^{\otimes \, (N-k-i)}, \quad i = 1,\ldots,N-k, 
\end{equation}
and after the encoding they become
\begin{equation}
    \widetilde{P}_i = U \, P_i \, U^{\dagger}.
\end{equation}

With that, the general stabilizer Hamiltonian has the form of
\begin{equation}
\label{eq:stab_hamil}
    H = - \sum_{i = 1}^{N-k} J_i \cdot \widetilde{P}_i, \quad J_i > 0
\end{equation}
The coefficients $J_i$ can be arbitrarily chosen and determine the energy scales of the system, but since they are necessarily positive-definite, this do not affect the space of ground states i.e. the space of valid physical qudit states. Excitations away from a ground states then correspond to errors being present in the state, which is because of the one-to-one relation between projectors and stabilizers.

\subsection{General Thermodynamic Quantities}

Using the Hamiltonian derived in the previous section, we can now compute the associated Gibbs state
\begin{equation}
    \rho_{\beta} = \frac{1}{Z} e^{-\beta H}, \quad Z = \Tr[e^{-\beta H}]
\end{equation}
and some of its properties, including the entropy.

First, it is straightforward to show that
\begin{align}
\begin{split}
\label{eq:gibbs_exp}
    e^{-\beta H} &= \exp\left( \beta \sum_{i = 1}^{N-k} J_i \cdot \widetilde{P}_i \right) \\
    &= \prod_{i = 1}^{N-k} \exp\left(\beta J_i \cdot \widetilde{P}_i\right) \\
    &= \prod_{i = 1}^{N-k} \left[ \sum_{n=0}^{\infty} \frac{1}{n!} \left(\beta J_i \cdot \widetilde{P}_i \right)^n \right] \\
    &= \prod_{i = 1}^{N-k} \left[ I + \sum_{n=1}^{\infty} \frac{1}{n!} \left(\beta J_i\right)^n \cdot \widetilde{P}_i\right] \\
    &= \prod_{i = 1}^{N-k} \left[ I + \left(e^{\beta J_i} - 1 \right) \cdot \widetilde{P}_i\right] \\
    &= \sum_{n=0}^{N-k} \, \sum_{1 \leq i_1 < \ldots < i_n \leq N-k} \, \prod_{\{i_a\}} \left( e^{ \beta J_{i_a}} - 1\right)\cdot \widetilde{P}_{i_a},
\end{split}
\end{align}
where in the last line we used a generalization of the binomial theorem and the fact that the projection operators commute by definition. Computing the partition function $Z$ using the final expression in \eqref{eq:gibbs_exp} can be done in the following way:
\begin{align}
\begin{split}
    Z &= \Tr[e^{-\beta H}] \\
    &= \sum_{n=0}^{N-k} \, \sum_{1 \leq i_1 < \ldots < i_n \leq N-k} \, \prod_{\{i_a\}} \left( e^{ \beta J_{i_a}} - 1 \right) \cdot \Tr \bigg[ \prod_{\{i_a\}}\widetilde{P}_{i_a} \bigg] \\
    &= \sum_{n=0}^{N-k} \, \sum_{1 \leq i_1 < \ldots < i_n \leq N-k} d^{N-n} \cdot \prod_{\{i_a\}} \left( e^{ \beta J_{i_a}} - 1 \right) \\
    &= d^{k} \cdot \sum_{n=0}^{N-k} \, \sum_{1 \leq i_1 < \ldots < i_n \leq N-k} d^{N-k-n} \cdot \prod_{\{i_a\}} \left( e^{ \beta J_{i_a}} - 1 \right) \\
    &= d^k \cdot \prod_{i=1}^{N-k} \left( e^{ \beta J_i} + d - 1 \right).
\end{split}
x\end{align}
Note that in the second line we used the definition \eqref{eq:pre_code_proj} for the projection operators, which implies that $\Tr[\widetilde{P}_{i_1} \cdots \widetilde{P}_{i_n}] = d^{N-n}$ given that none of the indices $i_a$ coincide. Going from the penultimate line to the last one we then again applied the generalized binomial theorem.

From the partition function it is then easy to determine all other thermodynamic quantities, of which the most important one for us is the von-Neumann entropy
\begin{align}
\begin{split}
    S &\equiv - \Tr[\rho_{\beta} \log(\rho_{\beta})] \\
    &= \beta \cdot \braket{E}_{\beta} + \log(Z) \\
    &= (1 - \beta \cdot \partial_{\beta}) \log(Z),
\end{split}
\end{align}
where the second and third lines are well-known equivalent expressions and we assume $\log$ to refer to the natural logarithm. Therefore, by using the fact that
\begin{align}
    \log(Z) &= k \cdot \log(d) + \sum_{i=1}^{N-k} \log\left( e^{ \beta J_i} + d - 1 \right), \\
    - \beta \cdot \partial_{\beta} \log(Z) &= - \sum_{i=1}^{N-k} \frac{\beta J_i \cdot e^{\beta J_i}}{e^{\beta J_i} + d - 1},
\end{align}
and after doing some rearranging, we arrive at
\begin{equation}
    S_{\text{stab}} = \left( k \log(d) + \sum_{i=1}^{N-k} p_i \log(d-1) \right) + \sum_{i=1}^{N-k} S(p_i),
\end{equation}
where
\begin{equation}
    S(p_i) \equiv - p_i \cdot \log(p_i)) - (1-p_i) \cdot \log(1-p_i)
\end{equation}
is the binary Shannon entropy associated to the probability distributions $\{p_i, 1-p_i\}_i$ which are defined in terms of
\begin{equation}
\label{eq:shannon_prob}
    p_i \equiv \frac{d-1}{e^{\beta J_i} + d - 1} \in \left(0, \frac{d-1}{d}\right).
\end{equation}
Note that in the case of qubits ($d = 2$), $p_i$ is the Fermi-Dirac distribution associated to $J_i$. Hence we can interpret the sum in the leading term as an occupation number such that
\begin{equation}
    \braket{N-k} \equiv \sum_{i=1}^{N-k} p_i, \quad S_{\text{stab}} = \log\left(d^k \cdot (d-1)^{\braket{N-k}}\right) + \sum_{i=1}^{N-k} S(p_i).
\end{equation}
Ignoring that leading term, the total entropy of the Gibbs ensemble therefore decouples into a sum of entropies associated with each energy level $J_i$ and therefore each element of the stabilizer basis \eqref{eq:pre_code_stab}. This is not unexpected though, as each term in the stabilizer Hamiltonian \eqref{eq:stab_hamil} commutes with every other one, making the system completely diagonalizable.

\subsection{Entropy Scaling for the NoRA Model}

So far all the calculations we did hold for error-correcting stabilizer codes in general. To actually get some results unique to the NoRA network discussed in this paper, we have to make some assumptions about the distribution of energy levels $J_i$.

One obvious such assumption is that the level distribution should only depend strongly on the layer $\ell$ at which associated stabilizer elements are first acted on in a non-trivial way by the encoding unitary. Hence we move from $J_i$ to $J_{\ell}$ (and therefore from $p_i$ to $p_{\ell}$),  ignoring (for now) that the energy might actually vary slightly for different stabilizers at the same level. Because of this the expression for the entropy becomes
\begin{equation}
\label{eq:level_entr_discrete}
    S_{\text{stab}} = \log\left(d^k \cdot (d-1)^{\braket{N-k}}\right) + \sum_{\ell=1}^{L} \Delta n_{\ell} \cdot S(p_{\ell}),
\end{equation}
where $n_{\ell}$ is the number of stabilizer basis elements with the same associated energy level:
\begin{align}
\begin{split}
\label{eq:stab_distribution}
    \Delta n_{\ell=1} &= r, \\
    \Delta n_{\ell > 1} &= r^{\ell} - n_{\ell - 1} = (r - 1) \cdot r^{\ell - 1}.
\end{split}
\end{align}
with $1 \leq \ell \leq L$ and $r^L = N - k$. It is easy to see that this distribution therefore does indeed satisfy $\sum_{\ell} \Delta n_{\ell} = N - k$.

The other assumption we are making is that the distribution of energies $J_{\ell}$ increases exponentially with increasing $\ell$, giving it the form of
\begin{equation}
\label{eq:energy_distribution_disc}
    J_{\ell} = \Lambda \cdot e^{-\gamma \cdot (L - \ell)}
\end{equation}
for some UV energy scale $\Lambda > 0$ and rate of increase $\gamma > 0$. This is an artificial but reasonable choice because we want the circuit to obey renormalization invariance while going from the IR to UV limit in the same was as MERA networks generally do.

\subsubsection{Moving to the Continuum Limit}

To determine the scaling of the entropy close to the zero temperature (i.e. $\beta \rightarrow \infty$) limit, it is useful to consider the continuum limit of \eqref{eq:level_entr_discrete} in addition to the other assumptions we made. The stabilizer difference $\Delta n_{\ell}$ therefore becomes the stabilizer density
\begin{equation}
    \rho(\ell) = 
    \rho_0 \cdot e^{\alpha \cdot \ell}, \quad \ell \in [0, L],
\end{equation}
where $\alpha > 0$ can be chosen arbitrarily\footnote{One could of course choose $\alpha = \log(r)$ in the spirit of \eqref{eq:stab_distribution}, but we will refrain from making a specific choice here for the sake of generality. This specific case will be considered later when comparing the approximation with the actual entropy formula.} and $\rho_0$ is fixed by the density having to satisfy
\begin{equation}
    N-k \stackrel{!}{=} \int_0^L d\ell \, \rho(\ell) = \frac{\rho_0}{\alpha} \left( e^{\alpha \cdot L} - 1 \right) \quad \Longleftrightarrow \quad \rho_0 = \frac{\alpha \cdot (N-k)}{e^{\alpha \cdot L} - 1}.
\end{equation}
Because the distribution of the energy levels \eqref{eq:energy_distribution_disc} can be left untouched when moving to the continuum limit, the stabilizer entropy can be naively approximated as
\begin{equation}
\label{eq:level_entr_cont}
    S_{\text{stab}} \approx S_{\text{cont}} = \log\left(d^{k} \cdot (d-1)^{\braket{N-k}}\right) + \int_0^L d\ell \, \rho(\ell) \cdot S(p(\ell)),
\end{equation}
with $p(\ell)$ being of the same form as $p_{\ell}$ in \eqref{eq:shannon_prob}, but now considered as a continuous function of $\ell$. But to make the upcoming calculations easier, we perform a change of variables, integrating over $J = J(\ell)$ instead of $\ell$. To do that, we first note that from \eqref{eq:energy_distribution_disc} it follows that
\begin{equation}
    \ell(J) = L + \frac{1}{\gamma} \cdot \log\left( \frac{J}{\Lambda} \right),
\end{equation}
and hence
\begin{equation}
    d \ell = \frac{d \ell}{d J} \, dJ = \frac{dJ}{\gamma \cdot J} .
\end{equation}
This also allows us to express the stabilizer density as a function dependent on $J$:
\begin{equation}
    \rho(J) = \rho_0 \cdot e^{\alpha L} \cdot \left( \frac{J}{\Lambda} \right)^{\alpha/\gamma}.
\end{equation}
Finally, the continuous entropy as an integral over $J$ is
\begin{align}
\begin{split}
\label{eq:S_cont}
    S_{\text{cont}} &= \log\left(d^{k} \cdot (d-1)^{\braket{N-k}}\right) + \int_{
    \Lambda \cdot e^{-\gamma  L}}^{\Lambda} dJ \, \frac{\rho(J)}{\gamma \cdot J} \cdot S(p(J)) \\
    &= \log\left(d^{k} \cdot (d-1)^{\braket{N-k}}\right) + \frac{\rho_0 }{\gamma} \cdot e^{\alpha  L} \cdot \int_{
    \Lambda \cdot e^{-\gamma L}}^{\Lambda} \frac{dJ}{\Lambda}  \left( \frac{J}{\Lambda} \right)^{\alpha/\gamma - 1} \cdot S(p(J)).
\end{split}
\end{align}
Note that the lower integration bound acts as an effective IR cutoff for the integral. This is necessary for us to be able to make the following approximations..

\subsubsection{Low-Temperature Limit}

Computing the integral in \eqref{eq:S_cont} is in general hard, but since we are only interested in the limit of small $T/J$ (or equivalently large $\beta J$), we can approximate the binary entropy $S(p(J))$ that occurs in the integral as
\begin{align}
\begin{split}
\label{eq:binS_approx}
    S(p(J)) &= - \frac{d-1}{e^{\beta J} + d - 1} \cdot \log \left( \frac{d-1}{e^{\beta J} + d - 1} \right) -  \frac{e^{\beta J}}{e^{\beta J} + d - 1} \cdot \log \left( \frac{e^{\beta J}}{e^{\beta J} + d - 1} \right) \\ 
    &\stackrel{\beta J \rightarrow \infty}{=} (d-1) \cdot \frac{\beta J}{e^{\beta J}} + \mathcal{O}(e^{-\beta J}),
\end{split}
\end{align}
which is straightforward to prove. To realize this limit it is necessary to choose the right parameters since it follows from \eqref{eq:energy_distribution_disc} that
\begin{equation}
    \beta J = \beta \Lambda \cdot e^{- \gamma (L-\ell)} \gg 1 \quad \forall \, \ell
\end{equation}
and hence
\begin{equation}
    \beta \Lambda \cdot e^{-\gamma L} \gg 1 \quad \Longleftrightarrow \quad \gamma L \ll \log(\beta \Lambda).
\end{equation}

Plugging \eqref{eq:binS_approx} into \eqref{eq:S_cont} and noting that $\braket{N-k} = \sum_i p_i = 0$ in that limit then leaves us with an expression that can be further simplified using a change of variables:
\begin{align}
\begin{split}
\label{eq:S_cont_2}
    S_{\text{cont}} &\approx k \log(d) + (d-1) \cdot \frac{\rho_0 \cdot e^{\alpha L}}{\gamma}  \int_{
    \Lambda \cdot e^{-\gamma L}}^{\Lambda} \frac{dJ}{\Lambda} \frac{\beta J}{e^{\beta J}} \left( \frac{J}{\Lambda} \right)^{\alpha/\gamma - 1} \\
    &= k \log(d) + (d-1) \cdot \frac{\rho_0 \cdot e^{\alpha L}}{\gamma} \cdot (\beta \Lambda)^{-\alpha/\gamma} \int_{\beta \Lambda \cdot e^{-\gamma L}}^{\beta \Lambda} dt \, t^{\alpha/\gamma} \cdot e^{-t} \\
    &= k \log(d) + (d-1)(N-k) \cdot \frac{\alpha}{\gamma} \cdot \frac{e^{\alpha L}}{e^{\alpha L} - 1} \cdot (\beta \Lambda)^{-\alpha/\gamma} \int_{\beta \Lambda \cdot e^{-\gamma L}}^{\beta \Lambda} dt \, t^{\alpha/\gamma} \cdot e^{-t} \\
    &\stackrel{\alpha L \gg 1}{\approx} k \log(d) + (d-1) (N-k) \cdot \frac{\alpha}{\gamma} \cdot (\beta \Lambda)^{-\alpha/\gamma} \int_{\beta \Lambda \cdot e^{-\gamma L}}^{\beta \Lambda} dt \, t^{\alpha/\gamma} \cdot e^{-t}
\end{split}
\end{align}
Let's consider the trailing integral. Up to the integration bounds it is the same as the gamma function $\Gamma(\alpha/\gamma + 1)$, whose integrand is positive everywhere. We can therefore get an upper bound for $S_{\text{cont}}$ (that we also expect to be approximately saturated for certain domains of $\beta \Lambda$) by substituting the \enquote{incomplete} gamma function with the proper one. Thus we have
\begin{equation}
\label{eq:S_cont_3}
    S_{\text{cont}} \lessapprox  k \log(d) + (d-1) (N-k) \cdot \frac{\alpha}{\gamma} \cdot \Gamma\left(\frac{\alpha}{\gamma} + 1\right) \cdot (\beta \Lambda)^{-\alpha/\gamma}, 
\end{equation}
which only scales with $(\beta \Lambda)^{-\alpha/\gamma} = (T/\Lambda)^{\alpha/\gamma}$, indicating that the entropy could indeed follow a power law, at least for certain low-temperature regimes. To show how well both continuous approximations hold up against the discrete stabilizer entropy with equivalent parameters ($N-k = r^L$, $\alpha=\log(r)$), we display both in logarithmic plots over $\log(T/\Lambda)$ and with different choices of $\gamma$, which is the only significant free parameter. These plots are depicted in figure \ref{fig:entropy_scaling_appendix} and indeed confirm that our low-temperature approximations are good at predicting aspects of the actual entropy, including its power-law growth.

\begin{figure}[hbt]
    \centering
    \includegraphics[width=\textwidth]{figures/entropy_scaling_appendix.png}
    \caption{Logarithmic scaling of exact stabilizer entropies $S_{\text{stab}}$ and their continuous approximations $S_{\text{cont}}$ (with and without the gamma function correction) for $L=20$, $N-k=r^L$, $k=1$, $d=2$, $r=2$, $\alpha = \log(r)$, $\Lambda=1$ and $\gamma \in \{0.1, 0.4, 1, 3\}$. In the first two figures it can be seen that our continuous approximation from \eqref{eq:S_cont_2} matches almost exactly with the discrete stabilizer entropy for $\gamma \ll 1$ and small $T/\Lambda$. Even though the second figure shows less behavior than the first one, we expect that it will behave similarly for even lower relative temperatures. While the last two approximations with $\gamma \geq 0$ also receive their primary contribution from the polynomial term, it is more apparent that they don't completely align with the actual data anymore. Especially in the last figure where $\gamma = 2$ the trend of the stabilizer entropy is not strictly polynomial anymore. Still, each figure has at least a regime where its growth is either exactly polynomial or follows a polynomial trend that aligns with our theoretical predictions up to a total constant factor.}  
    \label{fig:entropy_scaling_appendix}
\end{figure}




\end{document}
