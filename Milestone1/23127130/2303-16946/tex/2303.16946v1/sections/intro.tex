\section{Introduction}
\label{sec:intro}

Tensor networks are a powerful tool in the study of geometrically local quantum systems which have proven particularly useful for one-dimensional systems \cite{Or_s_2019}. In quantum many-body physics, they first appeared in the guise of \enquote{finitely-correlated states} \cite{cmp/1104249404} and were later understood to underlie the functioning of a powerful numerical technique, the density matrix renormalization group (DMRG), which gave unprecedented access to ground states of 1d Hamiltonians~\cite{2011AnPhy.326...96S}. It was understood that DMRG worked because the ground states of interest had limited entanglement and could be effectively compressed to a much smaller space parameterized by so-called matrix product states, a simple kind of 1d tensor network. The use of these tools has since broadened, and there is now a large family of tensor network architectures that are used for both analytical and numerical purposes, both with classical computers and, potentially, quantum computers, with approaches including \cite{PhysRevX.12.011047,https://doi.org/10.48550/arxiv.1711.07500, Verstraete_2004, Swingle_2016, Evenbly_2015, Czech_2016, Swingle_2016_2, Haegeman_2018,novikov2017exponential,stoudenmire2017supervised,Liu_2019,Han_2018,Huggins_2019}.

In contrast, such network representations have not been much explored for mean-field quantum models which are characterized by all-to-all interactions amongst their degrees of freedom. This is presumably because ground states of such models are expected to be volume-law entangled (e.g.~\cite{2018PhRvB..97x5126L,2019PhRvD.100d1901H}), and such a high degree of entanglement is costly to represent using existing tensor networks. In this paper, we address this problem by proposing a class of tensor networks which have the potential to represent the highly entangled ground states of mean-field models.

The networks we consider can be viewed as generalizations of MERA, DMERA, and branching MERA networks where the requirement of spatial locality is removed \cite{Vidal_2007,PhysRevLett.101.110501, Evenbly_2014, https://doi.org/10.48550/arxiv.1711.07500}. As we show below, such networks can accommodate volume law entanglement as is expected for ground states of mean-field models. However, without the imposition of additional structure it is not possible to efficiently contract these networks on a classical computer. Nevertheless, they provide a number of conceptual advantages and can still form the basis for variational quantum algorithms, e.g.~\cite{Peruzzo_2014,McClean_2016}. 

We are particularly motivated to consider these networks in light of the physics of the Sachdev-Ye-Kitaev (SYK) model \cite{PhysRevLett.70.3339,kitaev_talk,Polchinski_2016,Maldacena_2016}. This is a model of all-to-all interacting fermions with a number of unusual features, including an extensive ground state degeneracy and a power-law temperature dependence of the heat capacity at low temperature. Moreover, these curious low energy properties are related to the existence of a dual description in terms of a low-dimensional theory of gravity known as Jackiw-Teitelboim (JT) gravity \cite{kitaev_talk}. It is desirable to better understand the emergence of this gravitational physics, especially for a fixed realization of the couplings, in both the SYK model and beyond. Following earlier ideas relating tensor networks and holography, a small sampling of which is \cite{Swingle_2012,Pastawski_2015,Molina_Vilaplana_2014,2017arXiv171103109J,Bhattacharyya_2018,Bao_2019,Jahn_2021,Jahn_2022,chen2022exact}, a tensor network model of SYK may also provide useful information about the emergence of the bulk.

Informed by these properties, we consider a class of networks which can encode an extensively degenerate space of highly entangled ground states. Figure~\ref{fig:arch} illustrates the network architecture, dubbed the non-local renormalization ansatz (NoRA), which should be viewed as a quantum circuit ansatz for the ground space of a suitable class of Hamiltonians. To justify this architecture as a potential model of SYK, we estimate its entanglement and circuit complexity and find qualitative agreement with SYK expectations. In addition to constructing the ground space, the network also provides a skeleton on which we can build a model of excitations~\cite{2022arXiv221016419S}. For an appropriate choice of parameters this model can exhibit a power-law temperature dependence of the thermodynamic entropy (and therefore the heat capacity). These features are the key desiderata underlying our construction, and we discuss them in detail in Section~\ref{sec:arch}.

A natural next step would be to explore the NoRA network as a variational ansatz for SYK. This is complicated by two issues: we need to generalize the network structure to fermionic degrees of freedom, and we need to find a way to efficiently contract the network (or use a quantum computer). Given this extra complexity, we have elected to first explore the architecture in a simpler setting where the elementary gates are not variationally chosen but instead are taken to be random Clifford gates. This enables us to study the network properties using the stabilizer formalism~\cite{gottesman_stabilizer_1997} without needing to explicitly contract the network. Moreover, this setting yields a class of stabilizer codes in which the logical space is identified with the ground state degrees of freedom and the network represents an encoding circuit for the code. We study the stabilizer weights and distance of the resulting codes as a function of the layer depth $D$ and the total system size $N$ (see Figure~\ref{fig:arch}). We find that the network can produce good quantum codes~\cite{calderbank_good_1996}, meaning code families where the distance and number of logical qudits are both proportional to the number of physical qudits. However, these codes are not low-density parity check (LDPC) codes~\cite{gallager_low-density_1962,breuckmann_quantum_2021} since some of the stabilizers are high weight. We hypothesize that by further fine-tuning the gates, our network architecture could also yield encoding circuits for the recently discovered classes of good quantum LDPC codes~\cite{panteleev_asymptotically_2022, dikstein_locally_2020,dinur_good_2022,lin_good_2022,gu_efficient_2022,anshu_nlts_2022}.  

%\newpage

The rest of this paper is organized as follows: In Section~\ref{sec:arch} we describe the architecture in detail and discuss its key properties. In Section~\ref{sec:clifford} we define a family of random Clifford networks based on our architecture and discuss their interpretation as encoding circuits for stabilizer quantum error correcting codes. In Section~\ref{sec:numerical} we report a numerical study of several different realizations of the architecture falling within the stabilizer code ansatz. We describe in detail how the distance and stabilizer weights of the resulting codes depend on the model parameters. In Section~\ref{sec:code_structure} we discuss a particular thermodynamic limit which is inspired by the structure of SYK. We compare the entanglement and complexity to expectations from holographic calculations and comment on the code properties. Finally, in Section~\ref{sec:outlook} we give an outlook and discuss ongoing and future work.

\begin{figure}[h]
    \centering
    \tikzfig{figures/syk_circuit}
    \caption{Basic architecture of the proposed NoRA tensor network ansatz. A code word $\ket{\psi_{\text{code}}}$ consisting of $n_0 \equiv k$ (logical) \enquote{ground-state} qudits is embedded  as $\ket{\Psi_{\text{phys}}}$ in the (physical) ground space of the $d^N$-dimensional many-body Hilbert space by the means of $L$ layers of some given depth $D$ quantum circuits. For each layer $1 \leq \ell \leq L$ the circuit $D_{\ell}$ acts on the $n_{\ell - 1}$ qudit output from the previous layer and an additional $\Delta n_{\ell}$ new ancillary \enquote{thermal} qudits initialized in state $\ket{0}$. We stress that the layer circuits $D_\ell$ do not have to respect locality structure depicted by the 1d arrangement of qudit lines.}
    \label{fig:arch}
\end{figure}