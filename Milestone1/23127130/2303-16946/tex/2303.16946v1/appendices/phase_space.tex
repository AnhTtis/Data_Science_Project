\section{Phase Space Formalism}
\label{sec:phase_space}

\subsection{Weyl Representation}
\label{sec:weyl_rep}

Given a Hilbert space $\mathcal{H}$ of prime dimension $d > 2$ \footnote{The case of $d=2$ is excluded here since our choice of representation requires the existence of a 2-element in the group sucht that $\frac{1}{2} = \frac{d + 1}{2}$, which is only the case for d>2. This should not affect the phsyics though as systems of different qudits can always be mapped to each other.}, we choose a basis $\{\ket{0},\ket{1}, \ldots, \ket{d-1}\}$ with its states being labeled by the elements of the associated finite (Galois) field GF($d$)\footnote{Finite fields also exist for powers of primes i.e.\ GF($d^k$), but addition and multiplication does not happen mod $d^k$ in these cases. One can achieve the same group order though by instead using $k$ qudits with each being represented by a copy of GF($d$)}.  One can then introduce \emph{clock and shift operators} $Z, X$ which act on the basis states according to \cite{hudson}
\begin{equation}
\label{eq:boost_shift}
    Z^p \ket{k} = \chi(p \cdot k) \ket{k}, \quad X^q \ket{k} = \ket{k + q},
\end{equation}
where $p, q, k \in \text{GF}(d)$ and $\chi(k) = e^{2 \pi i k / d}$. Note that addition and multiplication happens over GF($d$) and is thus mod $d$. This is also respected by our choice for $\chi(k)$ since $\chi(k + d) = \chi(k)$ even for addition without modulo.

We are now able to define the so-called \emph{Weyl operators} for a single qudit, which provide a generalisation of the Pauli operators on a qubit:
\begin{equation}
\label{eq:weyl_single}
    w(p, q) = \chi\left(-\frac{p \cdot q}{2}\right) \, Z^p \, X^q, 
    \quad p,q \in \text{GF}(d).
\end{equation}
Extending this definition to $n$ qudits is as easy as tensoring $n$ copies of \eqref{eq:weyl_single}, which we write as
\begin{align}
\label{eq:weyl_multi}
    \begin{split}
        w(v) &= w(p_1, q_1, \ldots, p_n, q_n) \\
        &= w(p_1, q_1) \otimes \ldots \otimes w(p_n, q_n).
\end{split}
\end{align}
Each Weyl operator is therefore uniquely represented by an element $v$ of a $2n$-dimensional vector space $V$ over the field GF($d$). Using the commutation relations of $Z^p$ and $X^q$ that arise from their definition in \eqref{eq:boost_shift}, it also follows that
\begin{equation}
\label{eq:weyl_mul}
    w(v) \, w(w) = \chi \left( \frac{\symp{v}{w}}{2} \right) \, w(v + w),
\end{equation}
where $\symp{\cdot}{\cdot}$ is the \emph{symplectic product} on $V$, which obeys $\symp{v}{w} = -\symp{w}{v}$ and can be expressed as a matrix product:
\begin{equation}
\label{eq:symp_prod}
    \symp{v}{w} = v^T J w, \quad J = \begin{pmatrix}
        0 & 1 \\ -1 & 0
    \end{pmatrix}^{\oplus n}.
\end{equation}
Because of that the Weyl operators form a projective representation of the associated vector space $V$ equipped with a symplectic product. It is also noteworthy that \eqref{eq:weyl_mul} implies that two Weyl operators $w(v), w(w)$ commute if and only if the corresponding symplectic product $\symp{v}{w}$ vanishes.

Another useful indentity which we will use later is the fact that
only the identity $I_n = w(0)$ has a non-vanishing trace:
\begin{equation}
    \label{eq:weyl_trace}
        \Tr[w(v)] = d^n \delta_{v,0}.
\end{equation}
This is trivial to show for $X^q$ but requires using the fact that the Kronecker delta can be written as
\begin{equation}
\label{eq:kronecker_sum}
    \delta_{p,0} = \frac{1}{d} \sum_{k = 0}^{d-1} e^{\frac{2 \pi i k}{d} p}
\end{equation}
to prove it for $Z^p$ as well. 

\subsection{The Clifford Group}

The Clifford group is a subset of the unitary group which maps Weyl operators to other Weyl operators (up to a factor):
\begin{equation}
\label{eq:clifford_def}
    U w(v) U^{\dagger} = c(v) \, w(S(v)),
\end{equation}
for some $c: V \rightarrow \mathbb{C}$ and $S: V \rightarrow V$. Because $S$ therefore has to be compatible with \eqref{eq:weyl_mul}, it is easy to see that it has to be linear and preserve the symplectic product:
\begin{equation}
    \symp{S v}{S w} = \symp{v}{w} \quad \forall \, v,w \in V.
\end{equation}
In matrix representation, one can also equivalently state this property as $S^T J S = J$. Such a function is called \emph{symplectic}. The set of all symplectic functions for a given vector space $V$ forms the so-called \emph{symplectic group}\footnote{Note the similarities to the definition of the orthogonal group. In fact, the column entries of a symplectic matrix also form as (symplectic) basis $(e_1, f_1, \ldots, e_n, f_n)$ of $V$ which satisfies $\symp{e_i}{e_j} = 0 = \symp{f_i}{f_j}$ and $\symp{e_i}{f_j} = \delta_{ij}$ for all $i,j = 1, \ldots, n$. Applying a symplectic is therefore equivalent to a change of basis. \label{fn:symp_basis}}. 

In general, the structure of the Clifford group is completely determined by the following statements:
\begin{enumerate}
    \item For any symplectic $S$ there is a unitary operator $\mu(S)$ satisfying
    \begin{equation}
        \mu(S) w(v) \mu(S)^{\dagger} = w(S v) \quad \forall \, v \in V.
    \end{equation}
    \item $\mu(S)$ is a projective representation of the symplectic group, meaning
    \begin{equation}
        \mu(S) \mu(T) = e^{i \phi} \mu(S T)
    \end{equation}
    for some phase $\phi$.
    \item Up to a phase, any Clifford operator is of the form
    \begin{equation}
        U = w(a) \mu(S)
    \end{equation}
    for a suitable $a \in V$ and symplectic $S$.
\end{enumerate}
A proof of these statements can be found in \cite{hudson}. Note that this also fixes the factor from \eqref{eq:clifford_def} to be $c(v) = \chi(\symp{a}{Sv})$. 

\subsection{Stabilizer States and Codes}

As mentioned before, a vanishing symplectic product $\symp{v}{w}$ is equivalent to a vanishing commutator $[w(v), w(w)]$. One can therefore construct a set
\begin{equation}
    w(M) = \{ m \,|\, m \in M\}
\end{equation}
containing only commuting Weyl operators by choosing $M$ to be a subspace of $V$ satisfying
\begin{equation}
    \symp{m_i}{m_j} = 0 \quad \forall \, m_i, m_j \in M
\end{equation}
Such a subspace is called \emph{isotropic} and it is easy to see that it also forms a group under vector addition since the symplectic product is bilinear. The cardinality of isotropic subspaces can range between 0 and $d^n$ as there are at most $n$ elements with mutually vanishing symplectic product in a $2n$-dimensional symplectic basis (see footnote \ref{fn:symp_basis} for the reason). We will refer to $M$ having maximal cardinality as \emph{maximally isotropic}.

In general it is convenient to write the basis elements of an isotropic subspace as a $k \times 2n$ (or $2n \times k$) matrix over $GF(d)$, where $k = \log_d(M)$ is the size of the basis. In the literature this is called the \emph{stabilizer matrix}, although there it is often written in terms of the actual Pauli/Weyl operators and not their symplectic representation.

Isotropy of $M$ allows one to (at least partially) diagonalize the Weyl operators contained in $w(M)$, even completely if $M$ is maximally isotropic. In the latter case it is therefore possible to define a unique quantum state $\ket{M, v}$ in terms of the elements in $w(M)$ acting on it as stabilizers:
\begin{equation}
\label{eq:stab_def}
    \chi(\symp{v}{m}) w(m) \ket{M, v} = \ket{M, v} \quad \forall \, m \in M.
\end{equation}
The vector $v \in V$ therefore determines the phase differences between the eigenstates assocated with $w(M)$. A state satisfying \eqref{eq:stab_def} is called a \emph{stabilizer state} and can be written as
\begin{equation}
\label{eq:stab_state}
    \ket{M, v}\bra{M, v} = \frac{1}{d^n} \sum_{m \in M} \chi(\symp{v}{m})\, w(m).
\end{equation}
It is easy to show that \eqref{eq:stab_state} is a projection operator and has unit trace by applying \eqref{eq:weyl_trace} and using the fact that $M$ is a group and thus satisfies $M + m = M$ for all $m \in M$.

In fact, even for a non-maximally isotropic subspace $M$ would \eqref{eq:stab_state} still be a projector (up to normalization), but not a quantum state anymore. In this more general case we have
\begin{equation}
\label{eq:stab_proj}
    \Pi(M,v) = \frac{1}{|M|} \sum_{m \in M} \chi(\symp{v}{m})\, w(m)
\end{equation}
with $\Tr[\Pi(M,v)] = \frac{d^n}{|M|}$. All states in the subspace which $\Pi(M,v)$ projects onto therefore satisfy \eqref{eq:stab_def}, meaning that they form a code space. We can therefore identify this case as being a stabilizer code since it satisfies the definition in section \ref{sec:stab}. Even though finding stabilizer codes therefore just amounts to making a choice for $M$ and $v$, it does not ensure that the resulting code is good in the sense that its Hamming distance might be small or does not scale well.

\subsection{Entanglement Entropy of Stabilizer States}
\label{sec:stab_entropy}

Thanks to the structure of the symplectic product \eqref{eq:symp_prod} and the multi-particle Weyl operators defined in \eqref{eq:weyl_multi}, one can easily take the partial trace of \eqref{eq:stab_state} over a desired subsystem $B$ by writing $v = v_A \oplus v_B$ (same for $m$) and $w(m) = w(m_A) \otimes w(m_B)$ for all $m \in M$ and applying \eqref{eq:weyl_trace} to the latter term in the tensor product. The resulting reduced state is then
\begin{align}
\begin{split}
\label{eq:stab_reduced}
    \rho_A &= \Tr_B[\rho] \\
    &= \frac{d^{n_B}}{d^n} \sum_{m_A \in M_A} \chi(\symp{v_A}{m_A})\, w(m_A) \\
    &\equiv \frac{|M_A|}{d^{n_A}} \, \Pi(M_A, v_A),
\end{split}
\end{align}
where we made use of the fact that $n = n_A + n_B$ and identified \eqref{eq:stab_proj}, but this time in terms of $v_A$ and
\begin{equation}
\label{eq:M_reduced}
    M_A = \{ m_A \,|\, m_A \oplus 0_B \in M\}.
\end{equation}
This is possible since the definition of $M_A$ ensures that it is again a group (although not necessarily maximally isotropic)\footnote{Naively computing $M_A$ using \eqref{eq:M_reduced} is not efficient as such an algorithm would have $\mathcal{O}(d^n)$ runtime. A runtime that is polynomial in the system size can be achieved by instead permuting the sites that are to be traced out to the front the stabilizer matrix and then computing its reduced row echolon form. The basis vectors $b = b_A \oplus b_B$ for which $b_B \neq 0$ are then removed and for the remaining elements only $b_A$ is being considered.}.

The fact that even after tracing out a subsystem the resulting reduced state is still proportional to a projection operator makes computing the entanglement entropy straightforward. While it is possible to just directly evaluate the Von-Neumann entropy $S(A) = \Tr[\rho_A \log_d(\rho_A)]$, a more elegant and insightful approach can be made by instead considering the \emph{R\'enyi entropies}
\begin{equation}
    S^{(n)}(A) = \frac{1}{1 - n} \log_d \Tr[\rho_A^n],
\end{equation}
which have the property that
\begin{equation}
    \log_d(d^{n_A}) = S^{(0)}(A) \geq S(A) \geq S^{(2)}(A) \geq \ldots 
\end{equation}
where $S(A) = S^{(1)}(A) = \lim_{n \rightarrow 0} S^{(n)}(A)$ reproduces the ordinary Von-Neumann entropy. What makes the R\'enyi entropies interesting here is that they satisfy $S^{(n)}(A) = \log_d (\rank\rho_A)$ for all $n>0$ if the state being considered has a flat entanglement spectrum i.e.\ it is proportional to a projection operator\footnote{The proof is straightforward: Let $\rho_A = \alpha \cdot \Pi_A$, then $S^{(n)}(A) = \frac{1}{1 - n} \log_d \Tr[(\alpha \cdot \Pi_A))^n] = \frac{1}{1 - n} \log_d (\alpha^n \cdot \Tr[\Pi_A]) = \frac{1}{1 - n} \log_d (\alpha^n \cdot \rank\rho_A)$. Since $\alpha = (\rank \rho_A)^{-1}$ because of normalization we have $S^{(n)}(A) = \frac{1}{1 - n} \log_d (\rank\rho_A)^{1-n} = \log_d (\rank\rho_A)$.}. Since this is the case for the reduced stabilizer state we can use the fact that $\rank \rho_A = \frac{d^{n_A}}{|M_A|}$ to show that
\begin{equation}
    S(A) = n_A - \log_d |M_A|.
\end{equation}
If the number of basis vectors $k_A = \log_d |M_A|$ is known, then computing $S(A) = n_A - k_A$ is straightforward and numerically stable\footnote{As a sanity check, note that if $\rho_A$ is pure and therefore has $S(A) = 0$ it implies that $k_A = n_A$, which is the requirement for $M_A$ to be a maximally isotropic subspace and thus to define a (pure) stabilizer state.}.