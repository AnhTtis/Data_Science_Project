\section{Estimating the Layer Growth Factor}
\label{sec:growth_factor}

Given a generic string of $n$ generalized Pauli operators with local dimension $d$ and initial weight $w_0 \ll n$, we can estimate the relative weight growth $g$ the string experiences from one layer of $n/q$ random Cliffords being applied to random disjoint substrings of length $q$. The weight $w_k$ at the $k$th layer can therefore be estimated as
\begin{equation}
    w_k \approx g \cdot w_{k-1}.
\end{equation}

\subsection{Single Layer}

To find $g$ it is helpful to look at a single substring of length $q$ being scrambled by a single random Clifford. In that case, as long as a substring's weight $w_{k-1}(q)$ is not zero, we can expect its weight in the next layer to be on average
\begin{equation}
    \label{eq:substring_weight}
    w_k(q) = q \cdot \frac{d^2-1}{d^2},
\end{equation}
regardless of how the initial string looked\footnote{Remember that the generalized Pauli group of dimension $d$ has $d^2$ different elements, up to phases. Of those only the identity has zero weight.}. To extend this argument to the whole Pauli string we can therefore distinguish between two extreme cases:
\begin{itemize}
    \item All $w_{k-1}$ non-trivial Pauli operators are contained in as few substrings as possible, namely $\lceil w_{k-1}/q \rceil$. Since each such substring will on average have the weight \eqref{eq:substring_weight} after the Clifford layer, we find that
    \begin{equation}
    \label{eq:growth_saturated}
        w_k = \lceil w_{k-1}/q \rceil \cdot q \cdot \frac{d^2-1}{d^2} \approx w_{k-1} \cdot \frac{d^2-1}{d^2}.
    \end{equation}
    \item Each (non-trivial) substring contains exactly one non-trivial Pauli operator, meaning that in the next layer we have $w_{k-1}$ substrings each having the average weight \eqref{eq:substring_weight}. The total weight is therefore
    \begin{equation}
        w_k = w_{k-1} \cdot q \cdot \frac{d^2-1}{d^2}
    \end{equation}
\end{itemize}
Hence we can provide approximate upper and lower bounds for $g$ by
\begin{equation}
    \frac{d^2-1}{d^2} \lesssim g \lesssim q \cdot \frac{d^2-1}{d^2}.
\end{equation}
However, for our purposes we will always have $w_{k-1} \ll n$ (see next section), which makes it more likely for $g$ to be closer to the upper bound. Hence we can assume that
\begin{equation}
\label{eq:growth_factor}
    g \approx q \cdot \frac{d^2-1}{d^2}.
\end{equation}

\subsection{Multiple Layers}
\label{sec:max_depth}

Usually the scrambling circuits will be composed of more than one layer of random $q$-party Clifford gates. Therefore, given that we start with $w_0 \ll n$ and keep $g$ fixed to be \eqref{eq:growth_factor}, what is the approximate maximum depth $D$ for which $w_D = g^D \cdot w_0$ gives a good estimate for the total operator weight at the end?

Due to our previous arguments, we can expect the approximation to not hold anymore by the time $w_D$ is of order $n$ since then the case of \eqref{eq:growth_saturated} will dominate. In this case we say that the weight is \emph{saturated}, and we can estimate the order of magnitude of the saturation depth $D_{\text{sat}}$ by requiring that $g^{D_{\text{sat}}} \cdot w_0 \lessapprox n$, leading to:
\begin{equation}
\label{eq:D_max}
    D_{\text{sat}} \approx \log_g \left( \frac{n}{w_0} \right).
\end{equation}
A tighter bound can also be achieved by using $\log_q$ instead of $\log_g$. Both options are shown for a specific simulated example in Figure \ref{fig:growth_factor}.

\begin{figure}[htb]
    \centering
    \includegraphics[width=0.9\textwidth]{figures/growth_factor.png}
    \caption{Averaged relative weight growth $w_D / w_{D-1}$ of a single Pauli string ($d=3, n=128, w_0 = 1$) subjected to a random 2-local Clifford with increasing circuit depth $D$ (1000 repetitions). Depicted are also the estimates of the effect growth factor $g$ \eqref{eq:growth_factor} and the saturation depth(s) $D_{\text{sat}}$ \eqref{eq:D_max} for which it can be considered to hold. While \eqref{eq:D_max} indeed provides a good maximum circuit depth for which the data and our estimate for $g = 16/9$ approximately coincide, a tighter bound can be achieved by instead using $q = 2$ as the base of the logarithm.}
    \label{fig:growth_factor}
\end{figure}