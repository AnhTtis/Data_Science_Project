\section{Motivating the SYK Code}
\label{section:syk}

\subsection{The SYK Model}

As is well known \cite{Maldacena_2016}, the SYK model consists of $N$ Majorana fermions $\psi_i$ which have their dynamics determinded by the Hamiltonian
\begin{equation}
    H = (i)^{q/2} \sum_{1 \leq i_1 \leq i_2 \leq \ldots \leq i_q \leq N} j_{i_1 i_2 \ldots i_q} \psi_{i_1} \psi_{i_2} \ldots \psi_{i_q},
\end{equation}
where $q$ is an even number. The power of $i$ is necessary to ensure that $H$ is hermitean if $q = 2 \mod{4}$. This implies that the system is not symmetric under time reversal if $q/2$ is odd. If one therefore restricts themselves to the time reversal symmetric case, then the case $q=4$ dominates the interactions at low energy. 

Additionally, the coefficients $j_{i_1 i_2 \ldots i_q}$ are drawn from a random Gaussian distribution of real-valued variables such that they have zero mean and variance
\begin{equation}
    \braket{j^2_{i_1, i_2, \ldots, i_q}} = \frac{J^2 (q-1)!}{N^{q-1}},
\end{equation}
where $J$ is a scalar which is taken to be the same for all coefficients. The additonal factors involving $q$ and $N$ are chosen such that it simplifies the large $N$ limit.

\subsection{The MERA Tensor Network}

An important property of the SYK model is that it becomes approximately conformal in the large $N$ limit, which is the reason why AdS/CFT tells us that it is dual to JT gravity. But this connection between SYK models and CFTs is also interesting since the latter can be approximated as a MERA tensor network. One can therefore ask if something similar can be done for SYK models in general.

Looking at an example for a MERA network in figure \ref{fig:mera} it is clear that the circuit exhibits some form of self-similarity. This is necessary to ensure the field theory being approximated is renormalizable. A fine graining from the top of the circuit towards the bottom is also noticeable and can be interepreted as some code space with small Hilbert space dimension being encoded in a larger system of physical qudits. Since this is what we want to achieve, the SYK circuit has to obey these heuritic properties as well.
\begin{figure}[hbt]
    \centering
    \includegraphics[width=.9\linewidth]{figures/mera.png}
    \caption{A visual representation of a MERA tensor network approximating a conformal field theory \cite{Bao_2015}. The triangles and squares can be thought of as creating entanglement and fine-graining the encoded information (by adding more ancillary states) respectively when looked at from the moving-up perspective. Because of the structure of the circuit the information can't distribute faster than indicated by the causal cone (shaded in red).} 
    \label{fig:mera}
\end{figure}

Another important property is the scaling of the (thermodynamic) entropy with the temperature $T$ when an IR cutoff scale $\Lambda$\footnote{This can be interpreted as \enquote{cuting off} all the lower layers of the tensor network starting from a certain layer.} is introduced. In the case of the MERA network this scaling is given by
\begin{equation}
\label{eq:mera_scaling}
    S_{\text{th}} \sim \frac{n}{2^{\ell}} \sim n \left(\frac{T}{\Lambda}\right)^{\#},
\end{equation}
where $n$ is the number of qudits on the lowest layer and $\ell$ is the number of layers remaining after the cutoff. The latter can be found to be
\begin{equation}
\label{eq:layers}
    \ell = \log_2\left( \frac{\Lambda}{T} \right),
\end{equation}
which leads to the second expression in \eqref{eq:mera_scaling} ($\#$ is just an unspecified exponent).

\subsection{The SYK-Inspired Tensor Network Code}

\begin{equation}
    \widetilde{Z}_{\ell, i} = U_{\ell} \, Z_i \, U^{\dagger}_{\ell}
\end{equation}

\begin{equation}
    \Pi_{\ell, i} = I - \widetilde{Z}_{\ell, i}
\end{equation}

\begin{equation}
    \Pi_{\ell} = \bigotimes_{i = 1}^{2^{\ell}} \Pi_{\ell, i}
\end{equation}




\subsection*{OLD}

The task is now to construct a tensor network model that has approximately the same properties as its MERA counterpart, but with the correct scalings for the thermal entropy (and complexity) as expected from estimates for SYK and JT gravity \cite{Brown_2019}. We propose that such a network is given by the circuit in figure \ref{fig:syk_circuit}, where each unitary is modeled by a scrambling circuit with depth $D$ as seen in \ref{fig:wall_circuit}. 
\begin{figure}[hbt]
    \centering
    \tikzfig{figures/syk_circuit}
    \caption{The SYK tensor network taking a code word $\ket{\psi_{\text{code}}}$ and $n_{\text{anc}}$ ancilla qudits and scarmabling them into a system of physical qudits $\ket{\Psi_{\text{phys}}}$. We assume that for this paper that we have $n_{\text{anc}} = 2^{\ell}$ such that all input qudits experience some scrambling in the end. The scrambling unitary at each step is of the form depicted in figure \ref{fig:wall_circuit}.} 
    \label{fig:syk_circuit}
\end{figure}

Assuming an IR cutoff was already performed at the lowest level, $\ell$ is again the number of layers in the circuit and also determined by \eqref{eq:layers}. Self-similarity is then achieved if one sets $n_{\text{anc}} = 2^{\ell}$. Going upwards from the bottom the information is here too entangled by the unitaries and fine-grained due to the stepwise addition of new ancillary states.

\begin{figure}[hbt]
    \centering
    \tikzfig{figures/wall_circuit}
    \caption{An example quantum circuit with depth $D = 5$ used to simulate the rapid scrambling of quantum information. Each gate represents a randomly sampled unitary operator acting on two qudits. If the circuit wires are fixed then this code also exhibits some form of causal cone. But one can also randomly choose which pairs of qudits to scramble at each layer of the circuit.}
    \label{fig:wall_circuit}
\end{figure}

What remains to be checked is the scaling of the entropy, which can be easily seen to satisfy
\begin{equation}
    S_{\text{th}} \sim n_{\text{code}} + \frac{n}{k^{\ell}} \sim n_{\text{code}} + n_{\text{anc}} \left(\frac{T}{\Lambda}\right)^{\#},
\end{equation}
which is similar to what one would expect according to \cite{Brown_2019} \cite{Brown_2016}. We therefore argue that the network shown in \ref{fig:syk_circuit} does indeed satisfy the requirements for modelling a SYK system.

