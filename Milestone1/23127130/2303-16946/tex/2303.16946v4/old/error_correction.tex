\section{Quantum Error Correction}
\label{section:disc}

\subsection{Stabilizer Codes}
\label{sec:stab}

A (quantum) stabilizer code $[[n, k, d]]$ that encodes $k$ physical qudits into $n$ physical qudits with code distance $d$ is defined in terms of a \emph{stabilizer group} $S$, which is an abelian subgroup of the Pauli group $P_n(d)$ i.e.\ the group generated by all possible $n$-element tensor products of the Pauli operators ($d = 2$)
\begin{equation}
    X = \begin{pmatrix}
        0 & 1 \\ 1 & 0
    \end{pmatrix}, \,
    Y = \begin{pmatrix}
        0 & -i \\ i & 0
    \end{pmatrix}, \,
    Z = \begin{pmatrix}
        1 & 0 \\ 0 & -1
    \end{pmatrix}
\end{equation}
or their higher-dimensional counterparts ($d > 2$), which are defined in section \ref{sec:weyl_rep} \cite{qldpc}. The stabilizer group must therefore be generated by $(n - k)$ independent elements of $P_n$. A \emph{code word} is a state vector $\ket{\psi} \in \mathbb{C}^{d^n}$\ footnote{$d$ here is the dimension of the Hilbert space of a single qudit and not the code distance.} that satisfies $s \ket{\psi} = \ket{\psi}$ for all $s \in S$. The space spanned by all possible code words is called the \emph{code space} and has dimension $k$. The operators mapping logical states to other logical states are called \emph{logical operators} and must therefore commute with all elements of the stabilizer group.

Some important quantities in relation to stabilizer codes are
\begin{itemize}
    \item \textbf{operator weight:} The number of elements of $P_1(d)$ in the tensor product of the operator that are not proportional to the identity operator $I$.
    \item \textbf{code weight:} The largest weight an element of the corresponding stabilizer group exhibits.
    \item \textbf{code distance:} The minimal weight of all non-trivial logical operators. Equivalently it can be defined as the minimum number of qudits that have to be changed to arrive at another code word.
\end{itemize}
A \emph{quantum low-density-parity-check} (qLDPC) code is a code with constant weight, regardless of the code length $n$. This means that each stabilizer only acts on a constant number of qudits and each qudit is acted on by only a constant number of stabilizer elements. A \emph{good} qLDPC code has its code space dimension $k$ and distance $d$ scale linearly with the length $n$.

\subsection{Decoupling \& Code Distance}

Determining the distance for a stabilizer code and how it scales with the number of physical qudits is not a trivial thing to do. We present here one way, called the \emph{adversarial approach}, which makes use of analysing the mutual information
\begin{equation}
\label{eq:mut_inf}
    I(A, R) = S(A) + S(R) - S(AR)
\end{equation}
between all possible subsystems $A$ of the physical qudits and some external adversary $R$ who is initially maximally entangled with the code word. A depiction of the setup can be found in figure \ref{fig:decoupling}.
\begin{figure}[hbt]
    \centering
    \tikzfig{figures/decoupling}
    \caption{Circuit representation of the adversarial approach for determining the distance of a stabilizer code. Here $U_{MN}$ is a unitary that takes states of the form $\ket{\psi_{\text{anc}}}_M \otimes \ket{\psi_{\text{code}}}_N$ and maps $\ket{\psi_{\text{code}}}$ to the code space of the chosen stabilizer state. $\ket{\psi_{\text{anc}}}$ is an arbitrary state of ancilla qudits and can therefore be ignored. The code distance is then one less than the size of the smallest choice for $A$ that has non-vanishing mutual information $I(A, R)$ with $R$. The interpretation is that $R$ then has access to enough qudits such that they can change the initial code word.}
    \label{fig:decoupling}
\end{figure}

As can be seen there, $R$ remains maximally entangled with the logical state $\ket{\psi_{\text{code}}}$ even after it has been encoded into the code space by an appropiate unitary $U_{MN}$. The question is therefore what the minimum number of physical qudits $A$ is that the adversary $R$ has to \enquote{steal} to get access to one logical qudit, thus being able to change it. A measure that indicates for which choices of $A$ this is the case is the aforemention mutual information \eqref{eq:mut_inf}, which becomes non-zero in such a case. In that case one says that $A$ is \emph{decoupled} from the rest of the physical qudit space, meaning that the state can be written as a tensor product on $A$ and its complement $MN / A$. The code distance is therefore
\begin{equation}
    d = |A^*| - 1,
\end{equation}
where $A^*$ is the smallest set of physical qudits which satisfies $I(A^*, R) \neq 0$. Since the code word is assumed to be maximally entangled with $R$, this statement is true for all possible choices of $\ket{\psi_{\text{code}}}$.

Implementing this approach as an algorithm is problematic though since iterating about all possible choices for $A$ is combinatorically intensive. A way to simplify the procdeure at the cost of only getting an upper bound approximation for the code distance is by randomly sampling choices for A and determining the smallest one which has non-vanishing mutual information.