\def\cvprPaperID{7892}
\def\confName{CVPR}
\def\confYear{2023}

\def\paperTitle{Paper Title}

\def\authorBlock{
        Kun Su$^1$\thanks{Work done while interning at MIT-IBM Watson AI Lab} \qquad
    Kaizhi Qian$^2$ \qquad
    Eli Shlizerman$^1$ \qquad
    Antonio Torralba$^3$ \qquad
    Chuang Gan$^{2,4}$ \qquad \\
    $^1$University of Washington \qquad
    $^2$MIT-IBM Watson AI Lab \qquad
    $^3$MIT \qquad
    $^4$UMass Amherst\\
    % {\tt\small \{email, addresses\}@inst.edu}
}

% Compilation vars
\newif\ifreview \newcommand{\review}{\reviewtrue}
\newif\ifarxiv \newcommand{\arxiv}{\arxivtrue}
\newif\ifcamera \newcommand{\cameraready}{\cameratrue}
\newif\ifrebuttal \newcommand{\rebuttal}{\rebuttaltrue}

% \review % \review OR \arxiv OR \cameraready
\arxiv
\pdfoutput=1
\documentclass[10pt,twocolumn,letterpaper]{article}
\ifreview \usepackage[review]{cvpr} \fi
\ifarxiv \usepackage[pagenumbers]{cvpr} \fi
\ifrebuttal \usepackage[rebuttal]{cvpr} \fi
\ifcamera \usepackage{cvpr} \fi

\usepackage{graphicx}
\usepackage{amsmath}
\usepackage{amssymb}
\usepackage{booktabs}

%% PACKAGES (also see cvpr_header.tex)
\usepackage{times}
\usepackage{microtype}
\usepackage{epsfig}
\usepackage[dvipsnames,table,xcdraw]{xcolor}
\usepackage{caption}
\usepackage{float}
\usepackage{placeins}
\usepackage{color, colortbl}
\usepackage{stfloats}
\usepackage{enumitem}
\usepackage{tabularx}
\usepackage{xstring}
\usepackage{multirow}
\usepackage{xspace}
\usepackage{url}
\usepackage{soul}
\usepackage{subcaption}
\usepackage{xcolor}
\usepackage[hang,flushmargin]{footmisc}
\usepackage{amssymb}% http://ctan.org/pkg/amssymb
\usepackage{pifont}% http://ctan.org/pkg/pifont
\usepackage[accsupp]{axessibility}

\newcommand{\cmark}{\ding{51}}%
\newcommand{\xmark}{\ding{55}}%

\captionsetup{skip=2pt}
\setlength{\textfloatsep}{8.0pt plus 2.0pt minus 2.0pt}
\setlength{\floatsep}{8.0pt plus 2.0pt minus 2.0pt}
\setlength{\intextsep}{8.0pt plus 2.0pt minus 2.0pt}
\setlength{\dbltextfloatsep}{8.0pt plus 2.0pt minus 2.0pt}
\setlength{\dblfloatsep}{8.0pt plus 2.0pt minus 2.0pt}
% \titlespacing*{\section}{0pt}{*1}{*0}
% \titlespacing*{\subsection}{0pt}{*1}{*0}
% \titlespacing*{\subsubsection}{0pt}{*1}{*0}

\makeatletter
\renewcommand{\paragraph}{%
  \@startsection{paragraph}{4}%
  {\z@}{1ex \@plus .2ex \@minus .2ex}{-1em}%
  {\normalfont\normalsize\bfseries}%
}
\makeatother

% Unfortunately, this package interferes with arxiv's stamp
\ifcamera \usepackage[accsupp]{axessibility} \fi

%% MACROS

% \newcommand{\authorname}[1]{{\textcolor{blue}{[Author: #1]}}}
% ...

% \newcommand{\commandname}{string\xspace}
% \definecolor{colorname}{rgb}{0.92,0.49,0.19}

% General

\newcommand{\nbf}[1]{{\noindent \textbf{#1.}}}

\newcommand{\supp}{supplemental material\xspace}
\ifarxiv \renewcommand{\supp}{appendix\xspace} \fi

\newcommand{\todo}[1]{{\textcolor{red}{[TODO: #1]}}}

% Reviewer commands (1 to 5), e.g. \R{1}, \R{2}
\newcommand{\R}[1]{{%
    \textbf{%
        \ifstrequal{#1}{hmSm}{\textcolor{red}{#1}}{%
        \ifstrequal{#1}{1}{\textcolor{red}{R#1}}{%
        \ifstrequal{#1}{GrqZ}{\textcolor{blue}{#1}}{%
        \ifstrequal{#1}{2}{\textcolor{blue}{R#1}}{%
        \ifstrequal{#1}{GTBY}{\textcolor{teal}{#1}}{%
        \ifstrequal{#1}{3}{\textcolor{teal}{R#1}}{%
        \ifstrequal{#1}{4}{\textcolor{magenta}{R#1}}{%
                           \textcolor{cyan}{R#1}%
        }}}}}}}%
    }%
}}

\newenvironment{packed_enum}{
\begin{enumerate}
  \setlength{\itemsep}{1pt}
  \setlength{\parskip}{2pt}
  \setlength{\parsep}{0pt}
}{\end{enumerate}}

\newenvironment{packed_item}{
\begin{itemize}
  \setlength{\itemsep}{1pt}
  \setlength{\parskip}{2pt}
  \setlength{\parsep}{0pt}
}{\end{itemize}}

\def\eg{\emph{e.g.}}
\def\Eg{\emph{E.g.}}
\def\ie{\emph{i.e.}}
\def\etal{\emph{et al.}}
\def\etc{\emph{etc.}}
\definecolor{attr}{RGB}{192,0,0}
\definecolor{comp}{RGB}{0,164,74}
\definecolor{obj}{RGB}{0,112,192}
\definecolor{tabcolor}{RGB}{237,244,254}
\definecolor{branchone}{RGB}{85,131,53}
\definecolor{branchtwo}{RGB}{197,91,17}
\newcommand{\framework}{ADE\xspace}
% \newcommand{\hsznew}[1]{\textcolor{blue}{#1}}
\newcommand{\hsznew}[1]{#1}
\newcommand{\hsz}[1]{#1}
% \newcommand{\R1}{\textcolor{red}{hmSm}}
% \newcommand{\R2}{\textcolor{blue}{GrqZ}}
% \newcommand{\R3}{\textcolor{Green}{GTBY}}

\usepackage[ruled,vlined]{algorithm2e}
\setlength{\algomargin}{0pt}
\definecolor{commentcolor}{RGB}{110,154,155}   % define comment color
\newcommand{\PyComment}[1]{\ttfamily\textcolor{commentcolor}{\# #1}}  % add a "#" before the input text "#1"
\newcommand{\PyCode}[1]{\ttfamily\textcolor{black}{#1}} % \ttfamily is the code font

\usepackage[normalem]{ulem}
\newcommand{\li}{\uline{\hspace{0.5em}}}  %

\usepackage{xr-hyper}

\makeatletter
\newcommand*{\addFileDependency}[1]{
  \typeout{(#1)}
  \@addtofilelist{#1}
  \IfFileExists{#1}{}{\typeout{No file #1.}}
}

\makeatother
\newcommand*{\myexternaldocument}[1]{
    \externaldocument{#1}
    \addFileDependency{#1.tex}
    \addFileDependency{#1.aux}
}



\makeatletter
\AfterEndEnvironment{algorithm}{\let\@algcomment\relax}
\AtEndEnvironment{algorithm}{\kern2pt\hrule\relax\vskip3pt\@algcomment}
\let\@algcomment\relax
\newcommand\algcomment[1]{\def\@algcomment{\footnotesize#1}}
\renewcommand\fs@ruled{\def\@fs@cfont{\bfseries}\let\@fs@capt\floatc@ruled
  \def\@fs@pre{\hrule height.8pt depth0pt \kern2pt}%
  \def\@fs@post{}%
  \def\@fs@mid{\kern2pt\hrule\kern2pt}%
  \let\@fs@iftopcapt\iftrue}
\makeatother

\definecolor{citecolor}{RGB}{0, 113, 188}
\definecolor{rosepink}{RGB}{255, 0, 127}
\usepackage[pagebackref,breaklinks,colorlinks, citecolor=citecolor]{hyperref}

\usepackage[capitalize]{cleveref}
\crefname{section}{Sec.}{Secs.}
\crefname{table}{Table}{Tables}
\crefname{figure}{Fig.}{Figs.}




\frenchspacing

\begin{document}
%% TITLE
\title{3D Concept Learning and Reasoning from Multi-View Images}
%%

\author{Yining Hong\textsuperscript{1},  Chunru Lin\textsuperscript{2}, Yilun Du\textsuperscript{3}, \\ Zhenfang Chen\textsuperscript{5}, Joshua B. Tenenbaum\textsuperscript{3}, Chuang Gan\textsuperscript{4, 5}
 \\ 
\textsuperscript{1}UCLA, \textsuperscript{2,4}Shanghai Jiaotong University, \textsuperscript{3}MIT CSAIL,\\
\textsuperscript{4}UMass Amherst, \textsuperscript{5}MIT-IBM Watson AI Lab\\
\url{https://vis-www.cs.umass.edu/3d-clr/}
% \texttt{xuanli1@math.ucla.edu, yilingq@umd.edu, pyc@csail.mit.edu,}\\
% \texttt{jkrishna@mit.edu, lin@cs.unc.edu, cffjiang@math.ucla.edu,}\\ \texttt{chuangg@umass.edu}
}


\twocolumn[{
\renewcommand\twocolumn[1][]{#1}
\maketitle
\begin{center}
%\centering
    \includegraphics[width=\linewidth]{figures/teaser_v6.pdf}
    \vspace{-0.1in}
    \captionof{figure}{An exemplar scene with multi-view images and question-answer pairs of our 3DMV-VQA dataset. 3DMV-VQA contains four question types: concept, counting, relation, comparison. \textcolor{orange}{Orange} words denote semantic concepts; \textcolor{blue}{blue} words denote the relations.} \label{fig:teaser}
    \vspace{0.15in}
\end{center}
}]





\begin{abstract}
% \vspace{-1em}
The diffusion-based generative models have achieved remarkable success in text-based image generation. However, since it contains enormous randomness in generation progress, it is still challenging to apply such models for real-world visual content editing, especially in videos. 
In this paper, we propose \texttt{FateZero}, a zero-shot text-based editing method on real-world videos without per-prompt training or use-specific mask. 
\RM{Specifically, different from a pipeline of two independent inversion and then generation stages, we find the intermediate attention maps during inversions store better structure and motion information. We thus reform them to temporally casual attention and replace them in the generation progress. To further reduce the unnecessary semantic leakage of source video and enhance the editing quality, we then remix the temporally casual attentions via the cross-attention features of the source prompt as the mask.}
To edit videos consistently, we propose several techniques based on the pre-trained models. Firstly, in contrast to the straightforward DDIM inversion technique, our approach captures intermediate attention maps during inversion, which effectively retain both structural and motion information. These maps are directly fused in the editing process rather than generated during denoising. To further minimize semantic leakage of the source video, we then fuse self-attentions with a blending mask obtained by cross-attention features from the source prompt. Furthermore, we have implemented a reform of the self-attention mechanism in denoising UNet by introducing spatial-temporal attention to ensure frame consistency.
Yet succinct, our method is the first one to show the ability of zero-shot text-driven video style and local attribute editing from the trained text-to-image model. We also have a better zero-shot shape-aware editing ability based on the text-to-video model~\cite{tuneavideo}. \RM{Besides video, our unified method also achieves state-of-the-art performance in zero-shot image editing.\chenyang{Need exp or remove the zero-shot image}} Extensive experiments demonstrate our superior temporal consistency and editing capability than previous works.
% The code will be released.
% \chenyang{emphasize: our observation at inversion time} \xiaodong{replacing the bold part to the actual pipeline: \textbf{Specifically, we work on replacing and mixing the attention maps between the inversion and generation since the self-attention map keeps the structure of the original natural image and the cross-attention is semantic-related, after remixing, we replace them in the corresponding generation steps for denoising.}}
% \footnote{Since there is no general video diffusion model is publicly available, we use one-shot video generation method~(Tune-A-Video~\cite{tuneavideo}) as the pretrained video diffusion model for zero-shot video editing\xiaodong{can be removed if we actually zero-shot on video}.}.
\end{abstract}
\vspace{-3mm}
Humans are able to accurately reason in 3D by gathering multi-view observations of the surrounding world.  Inspired by this insight, we introduce a new large-scale benchmark for 3D multi-view visual question answering (3DMV-VQA). This dataset is collected by an embodied agent actively moving and capturing RGB images in an environment using the Habitat simulator.  In total, it consists of approximately 5k scenes, 600k images, paired with 50k questions. We evaluate various state-of-the-art models for visual reasoning on our benchmark and find that they all perform poorly. We suggest that a principled approach for 3D reasoning from multi-view images should be to infer a compact 3D representation of the world from the multi-view images, which is further grounded on open-vocabulary semantic concepts, and then to execute reasoning on these 3D representations. As the first step towards this approach, we propose a novel 3D concept learning and reasoning (3D-CLR) framework that seamlessly combines these components via neural fields, 2D pre-trained vision-language models, and neural reasoning operators. Experimental results suggest that our framework outperforms baseline models by a large margin, but the challenge remains largely unsolved. We further perform an in-depth analysis of the challenges and highlight potential future directions. 


\section{Introduction}

\label{sec:intro}

% \textit{"Drawing and colour are not separate at all; in so far as you paint, you draw. The more the colour harmonizes, the more exact the drawing becomes."} - Paul Cezanne.

Art is a reflection of the figments of human imagination. 
While many are limited in their practical creative capabilities, by pushing the boundaries of digital media, new ways can be created for casual artists and experts alike to express their ideas. At the same time, current neural generative art takes away much of the control from humans. In this work, we attempt to take a step towards restoring some of that control, enabling neural networks to complement users and naturally extend their skills rather than taking hold over the generative process.

% \orr{TBD - make the opening colorful : 1. Add quote:  2. Elaborate: art is a rendering of figments of imaginations of humans. Most people are limited in their drawing capabilities, and by pushing the boundaries we allow new ways for casual artists and experts alike in expressing ideas. At the same time, neural generative art takes a lot of the control away. Here, we want to give back some of this control to humans, such that neural networks complement them and compensate their lack of skills, rather than replacing them.}

% The field of image synthesis has been significantly propelled by neural generative models, particularly by the latest text-to-image models that predominantly rely on large language-image models ~\cite{balaji2022eDiff-I, ramesh2022dalle, rombach2021highresolution, imagen2022saharia}. These models have revolutionized the field of computer vision as they can produce astonishing visual outcomes from text prompts only.

The field of image synthesis has been significantly propelled by neural generative models, particularly by the latest text-to-image models that predominantly rely on large language-image models ~\cite{balaji2022eDiff-I, ramesh2022dalle, rombach2021highresolution, imagen2022saharia}. These models have revolutionized the field of computer vision, as they can produce astonishing visual outcomes from text prompts alone.

The ability of text-to-image models has sparked a wave of editing methods that utilize these models. Many of these techniques rely on prompt editing ~\cite{ fu2022shapecrafter, hertz2022prompt, kawar2022imagic,lin2022magic3d,mokady2022null, poole2022dreamfusion}. Nevertheless, simplifying the interface to text alone means users lack the necessary level of granularity to produce their exact desired outcomes.
% which is} insufficient for effectively editing local content. 
% editing and manipulating visual content, as users lack the necessary level of control to achieve their desired outcomes
Sketch-guided editing, on the other hand, provides intuitive control that aligns with user's conventional drawing and painting skills. By incorporating user-guided sketches into text-to-image models, powerful editing systems can be created, offering a high degree of flexibility and fine-grained control for manipulating visual content~\cite{zhang2023controlnet, voynov2022sketch}.

Although sketch-guided and text-driven methods have proven successful in generating and manipulating 2D images \cite{meng2022sdedit, voynov2022sketch, cheng2023wacv}, it immediately raises the intriguing question of whether a similar approach could be developed to edit 3D shapes. 
Since direct text-to-3D models require an abundance of data to scale, state-of-the-art 3D generative models, such as DreamFusion~\cite{poole2022dreamfusion} and Magic3D~\cite{lin2022magic3d}, which build on the capabilities of text-to-image models, may be considered as an alternative.
% Due to the difficulty of scaling general direct text-to-3D models, incorporating conditions into a text-to-3D model is not straightforward. Thus, state of the art 3D generative models, such as DreamFusion~\cite{poole2022dreamfusion} \orrc{and Magic3D~\cite{lin2022magic3d}}, which build on the capabilities of text-to-image models, may be considered as an alternative.
However, maintaining control via conditioning with such models remains a challenging task, as these generative pipelines optimize a Neural Radiance Field (NeRF) \cite{mildenhall2020nerf} by amortizing gradients from a multitude of 2D views. In particular, providing consistent sketches across all possible views presents a hurdle for users. Instead, a plausible user interface should act with guidance from as few views as possible, e.g. up to two or three.


In this paper, we present \textbf{SKED}, a \textbf{SK}etch-guided 3D \textbf{ED}iting technique. Our method acts on reconstructed or generated NeRF models. We assume a text prompt and a minimum of two sketches as input and provide output edits over the neural field faithful to the input conditions.
Meeting all input requirements can be challenging as the text prompt may not match the sketch's semantics, and sketch views may lack coherence.
To undertake this complex task, we conceptually break it down into two subtasks that are easier to handle: one that depends on pure geometric reasoning and the other that exploits the rich semantic knowledge of the generative model. These two subtasks work together, with the former providing a coarse estimate of location and boundary, and the latter adding and refining geometric and texture details through fine-grained operations.


Our experiments highlight the effectiveness of our approach for editing various pretrained NeRF instances. We introduce assorted accessories, objects, and artifacts, which are generated and blended into the original neural field seamlessly.
Finally, we validate our method through quantitative evaluations and ablation studies to assert the contribution of individual components in our method. 
% By presenting examples in the paper, we illustrate that our method can generate realistic 3D artifacts with accurate texture and geometry using only a few basic sketches.



% Due to the absence of a direct text-to-3D model, incorporating conditions into a text-to-3D model is not straightforward. Thus, 3D generative models, such as DreamFusion~\cite{poole2022dreamfusion}, which build on the capabilities of text-to-image models, may be considered as an alternative.
% However, this is a challenging task since DreamFusion generates a NeRF by integrating many different 2D views. It is very hard to provide consistent sketches across all possible views. The challenge is to use sketches as a guide on only a few views (e.g., two or three) and generate 3D edit of the existing NeRF that is subject to being edited. 

% In this paper, we present \textbf{SKED}, a \textbf{SK}etch-guided 3D \textbf{ED}iting technique, that takes as input a text prompt and a few (two or more) sketches and edits a 3D given object represented as a NeRF in a geometrically plausible and controlled way. 
% We acknowledge the difficulty of this task, as there are no existing text-to-3D generative models available for manipulating the geometry of the existing object based on a text prompt. 
% To undertake this complex task, we conceptually break it down into two simpler subtasks that are easier to handle: one that depends on pure geometric reasoning and the other that exploits the rich semantic knowledge of generative model. These two subtasks work together, with the former providing a coarse estimate of location and the latter adding and refining geometric and texture details through fine-grained operations.

% Our experiments showcase the effectiveness of our approach in performing sketch-guided text-based edits on different base nerfs by introducing various accessories, objects, and artifacts. We also conduct ablation studies and experiments to evaluate the performance of individual components in our method. By presenting examples in the paper, we illustrate that our method can generate realistic 3D artifacts with accurate texture and geometry using only a few basic sketches.

%\dcc{Add here the traditional paragraph that tell about what we achieved and evaluated}
  While  submodular optimization problems are generally NP-hard, the celebrated greedy algorithm \cite{nemhauser1978analysis} attains a $(1-1/e)$ approximation ratio for  submodular maximization subject to uniform matroids and a $1/2$ approximation ratio for general matroid constraints. As discussed in the introduction, the  continuous greedy algorithm \cite{calinescu2011maximizing} restores the $(1-1/e)$ approximation ratio by lifting the discrete problem to the continuous domain via the multilinear relaxation. %It is worth to mention here that the multilinear relaxation is a DR-submodular function, a.k.a. a continuous function with the diminishing returns property.

Stochastic submodular maximization, in which the objective is expressed as an expectation, has gained a lot of interest in the recent years \cite{asadpour2008stochastic, zhang2022stochastic, chen2018online}. Karimi et al. \cite{karimi2017stochastic} use a concave relaxation method that achieves the $(1-1/e)$ approximation guarantee, but only  for the class of submodular coverage functions. Hassani et al.~\cite{hassani2017gradient} provide projected gradients methods for the general case of stochastic submodular problems that achieve $1/2$ approximation guarantee.  Mokhtari et al. \cite{mokhtari2020stochastic} propose stochastic  conditional gradient methods for solving both minimization and maximization  stochastic submodular optimization problems. Their method for maximization, Stochastic Continous Greedy (SCG) can be interpreted as a stochastic variant of the continuous greedy algorithm \cite{vondrak2008optimal, calinescu2011maximizing} and achieves a tight $(1-1/e)$ approximation guarantee for monotone and submodular functions. %However, all these methods suffer from two sources of randomness (one comes from sampling the objective function and the other comes from estimating the multilinear relaxation via sampling its inputs).

Our work builds upon and relies on the approach by  \"{O}zcan et al.~\cite{ozcan2021submodular}, who studied ways of accelerating the computation of gradients via a polynomial estimator. Extending on the work of Mahdian et al.~\cite{mahdian2020kelly},  \"{O}zcan et al. show that submodular functions that can be written as compositions of (a) an analytic function and (b) a multilinear function can be arbitrarily well approximated via Taylor polynomials; in turn, this gives rise to a method for approximating their multilinear relaxation in a closed form, without sampling. We leverage this method in the context of stochastic submodular optimization, showing that it can also be applied in combination with SCG of Mokhtari et al.~\cite{mokhtari2020stochastic}: this eliminates one of the two sources of randomness, thereby reducing variance at the expense of added bias. From a technical standpoint, this requires controlling the error introduced by the bias of the polynomial estimator, while simultaneously accounting for the variance inherent in SCG, due to sampling instances.   %: this eliminates the latter source of randomness by utilizing the properties of deep submodular models that result from composition over multiple layers. In order to do so, we combine the stochastic continuous greedy algorithm proposed by Mokthari et al. \cite{mokhtari2020stochastic} with the deterministic estimator proposed by
\section{\ours}
\label{sec:dataset}
\begin{figure}[t]
    \centering
    \hfill
        \includegraphics[width=\linewidth]{figures/data_sample.pdf}
    \hfill
    \vskip -7mm
    \caption{\textbf{A sample from the \ours dataset.} Each sample of the dataset consists of two input patches and the corresponding annotations. \textbf{Left} shows the large FoV patch $x_{l}$ with tissue segmentation annotation $y_{l}^{t}$, where green denotes the cancer area. \textbf{Right} shows the small FoV patch $x_{s}$ with cell point annotation $y_{s}^{c}$, where blue and yellow dots denote \textit{tumor} and \textit{background} cells, respectively. The red box indicates the size and location of the $x_{s}$ with respect to the $x_{l}$.}
    \label{fig:data_sample}
    \vspace{-0.3cm}
\end{figure}

In this section, we introduce \ours, a histopathology dataset specifically built to enable the development of methods that leverage cell and tissue relationships. Each sample of the \ours dataset $\mathcal{D}$ is composed of six components,

\begin{equation}
\mathcal{D} = \left\{\left(x_{s}, y_s^{c}, x_l, y_l^{t}, c_x, c_y\right)_i\right\}_{i=1}^{N}
\end{equation}

\noindent where $x_s, x_l$ are the small and large FoV patches extracted from the WSI, $y_s^{c}, y_l^{t}$ refer to the corresponding cell and tissue annotations, respectively, and $c_x, c_y$ are the relative coordinates of the center of $x_s$ within $x_l$. We drop the sample index $i$ for simplicity. \autoref{fig:data_sample} shows the visualization of a sample in \ours. More details about the dataset including data collection and statistics can be found in the following sub-sections. The dataset is publicly available at \href{https://lunit-io.github.io/research/publications/ocelot/}{https://lunit-io.github.io/research/publications/ocelot/}.

\subsection{Data Collection}
\label{ssec:data-collect}
We collect 306 TCGA~\cite{HUTTER2018283} WSIs from a total of 6 different organs: \textit{kidney}, \textit{head-and-neck}, \textit{prostate}, \textit{stomach}, \textit{endometrium}, and \textit{bladder}. 
From each of the WSIs, we select 1 to 3 large Regions of Interest (ROIs) for the tissue segmentation task. 
Finally, for the cell detection task, we randomly choose a smaller ROI that is fully contained within the larger tissue ROI. 
As a result, \ours includes 673 paired patches from 6 organs. The numbers of WSIs and pairs of patches per organ are detailed in \autoref{tab:dst_size}.


Some natural image datasets, such as ImageNet~\cite{deng2009imagenet} or Pascal VOC~\cite{everingham2010pascal}, include thousands of annotated images. However, annotating histopathology images is more challenging and expensive due to the scarcity of expert pathologists~\cite{c3det}. Furthermore, acquiring dense annotations for cell detection and tissue segmentation is especially time-demanding compared to higher-level tasks such as image classification. Nonetheless, in \autoref{tab:label_stats_new}, we observe that \ours is roughly double the size of the recent TIGER dataset with respect to the annotated tissue area and the number of annotated cells. 



\vspace{-4mm}
\paragraph{Patch configuration.} 
Cell detection tasks benefit from fine-grained spatial information to better capture detailed cell properties (e.g. border, shape, color, and opacity). In contrast, tissue segmentation requires a larger context to enable a better understanding of the overall structural information. Therefore, we define the FoV sizes of $x_s$ (cell detection) and $x_l$ (tissue segmentation) as 1024 $\times$ 1024 and 4096 $\times$ 4096 pixels, respectively, at a resolution of 0.2 Microns-per-Pixel (MPP). Finally, the large FoV patches and tissue annotations ($x_l$, $y_l^{t}$) are down-sampled by a factor of 4, resulting in a size of 1024 $\times$ 1024 pixels. 


\vspace{-4mm}
\paragraph{Annotation.} 
All cell-tissue pairs of patches are annotated by board-certified pathologists. Cells are labeled as points, with associated 2D coordinates and class labels. We denote the annotations in a given cell-level patch, $x_s$, as $y_s^{c}$, and consider two classes: Tumor Cell (\textit{TC}) and Background Cell (\textit{BC})\footnote{\textit{BC} includes any of the following cell categories: lymphocyte, macrophage, fibroblast, endothelial, or other remaining cell types.}.
\textit{TC} and \textit{BC} class ratios are 35.01\% and 64.99\%, respectively.
Regarding the tissue patches, $x_l$, pathologists annotate the pixel-wise segmentation maps $y_l^{t}$ with either Cancer Area (\textit{CA}) or Background (\textit{BG}) labels. A minority of pixels where the tissue class was uncertain were labeled as Unknown (\textit{UNK}). \textit{BG}, \textit{CA}, and \textit{UNK} class ratios are 55.77\%,  40.17\%, and 4.06\%, respectively. The amount of annotated cells and tissue pixels, per data split, can be found in the \supple. The detection of \textit{TC} and \textit{BC} has clinical relevance. For example, tumor purity \cite{azimi2017breast}, computed as the tumor/non-tumor cell ratio in a WSI, has a correlation with cancer prognosis \cite{mao2018low, zhang2017tumor, gong2020tumor}.

\vspace{-4mm}
\paragraph{Dataset splits.} The dataset is divided into three subsets: \textit{training}, \textit{validation}, and \textit{test}, following a $6$:$2$:$2$ ratio. To prevent information leaking among the data subsets, we randomly split the dataset per WSI, so that different patches from the same WSI are not included in multiple subsets. We maintain consistent cancer-type ratios in each subset.

\begin{table}
\small

\centering
\setlength{\tabcolsep}{0.2em}
{
\begin{tabular}{lrrrrrr}
\toprule
\multirow{2}{*}{\textbf{Organs}} & \multicolumn{3}{c}{\textbf{\# Slides}} & \multicolumn{3}{c}{\textbf{\# Patch Pairs}} \\ \cmidrule(lr){5-7} \cmidrule(lr){2-4} 
& \textbf{Train} & \textbf{Val} & \textbf{Test} &  \textbf{Train} & \textbf{Val} & \textbf{Test} \\ \midrule
Kidney & 48 & 15 & 18 & 125 & 41 & 41 \\ 
Head-neck & 13 & 5 & 6 & 27 & 9 & 10 \\ 
Prostate & 26 & 12 & 10 & 50 & 17 & 16 \\ 
Stomach & 15 & 6 & 5 & 36 & 12 & 12 \\ 
Endometrium & 38 & 13 & 13 & 86 & 29 & 25 \\ 
Bladder & 35 & 14 & 14 & 82 & 29 & 26 \\ \midrule
\textbf{Total} & 175 & 65 & 66 & 406 & 137 & 130 \\ \bottomrule
\end{tabular}
}
\vskip -2mm
\caption{\textbf{Dataset size per organ and data subset.}}
\label{tab:dst_size}
\vspace{-4mm}
\end{table}
\section{Method}
% \label{sec:method}

\begin{figure*}[t]
\centering
\includegraphics[width=\linewidth]{figures/framework_v3.pdf}
\vspace{-5mm}
\caption{An overview of our 3D-CLR framework. First, we learn a 3D compact scene representation from multi-view images using neural fields (I). Second, we use CLIP-LSeg model to get per-pixel 2D features (II). We utilize a 3D-2D alignment loss to assign features to the 3D compact representation (III). By calculating the dot-product attention between the 3D per-point features and CLIP language embeddings, we could get the concept grounding in 3D (IV). Finally, the reasoning process is performed via a set of neural reasoning operators, such as \textsc{Filter}, \textsc{Get\_Instance} and \textsc{Count\_Relation} (V). Relation operators are learned via relation networks.}
\vspace{-5mm}
\label{fig:framework}
\end{figure*}

Fig.~\ref{fig:framework} illustrates an overview of our framework. Specifically, our framework consists of three steps.  First, we learn a 3D compact representation from multi-view images using neural field. And then we propose to leverage pre-trained 2D vision-and-language model to ground concepts on 3D space. This is achieved by 1) generating 2D pixel features using CLIP-LSeg; 2) aligning the features of 3D voxel grid and 2D pixel features from CLIP- LSeg~\cite{li2022language}; 3) dot-product attention between the 3D features and CLIP language features~\cite{li2022language}. Finally, to perform visual reasoning, we propose neural reasoning operators, which execute the question step by step on the 3D compact representation and outputs a final answer. For example, we use \textsc{Filter} operators to ground semantic concepts on the 3D representation, \textsc{Get\_Instance} to get all instances of a semantic class, and \textsc{Count\_Relation} to count how many pairs of the two semantic classes have the queried relation.
% \gc{metion all the neural operators.}
% Works from linguistic and cognitive science suggest that semantic concepts are diverse and open-vocabulary, while relational concepts describing 3D objects' relationships can be very limited and thus can be considered a close-class vocabulary \cite{Landau1993WhatA, Hayward1995SpatialLA}. Therefore, it's unrealistic to learn the embeddings of all the concepts in the question-answering pairs, while it's more natural to learn the relation embeddings. Inspired by this, we propose to leverage 2D pretrained vision-language model (\textit{i.e.,} CLIP) for open-vocabulary semantic concept learning, while proposing a neural relation module network for relational reasoning. 

\subsection{Learning 3D Compact Scene Representations}

% Since 3D-related reasoning works on 3D compact representations rather than 2D images, we first propose to use a neural field to extract 3D representations from multi-view images. The next step is to learn the 3D features for visual reasoning. However, 3D assets are limited in diversity and scale, posing challenges for training large-scale 3D foundation models, while there's much progress on large-scale 2D pretrained models which provide decent features\cite{Radford2021LearningTV, Ramesh2021ZeroShotTG}. Since neural field maps a 2D pixel to several 3D points along the ray, it's natural to get 3D features for 2D per-pixel features. We apply CLIP-LSeg\cite{Li2022LanguagedrivenSS} to learn per-xel 2D features, and use an alignment loss to align 3D features with 2D features.

% \paragraph{3D Compact Representation from neural field.} 
Neural radiance fields  \cite{mildenhall2020nerf} are capable of learning a 3D representation that can reconstruct a volumetric 3D scene representation from a set of images. Voxel-based methods \cite{Garbin2021FastNeRFHN, Hedman2021BakingNR, Yu2021PlenOctreesFR, Sun2022DirectVG} speed up the learning process by explicitly storing the scene properties (\textit{e.g.}, density, color and feature) in its voxel grids. We leverage Direct Voxel Grid Optimization (DVGO) \cite{Sun2022DirectVG} as our backbone for 3D compact representation for its fast speed. DVGO stores the learned density and color properties in its grid cells. The rendering of multi-view images is by interpolating through the voxel grids to get the density and color for each sampled point along each sampled ray, and integrating the colors based on the rendering alpha weights calculated from densities according to quadrature rule \cite{Max1995OpticalMF}. The model is trained by minimizing the L2 loss between the rendered multi-view images and the ground-truth multi-view images. By extracting the density voxel grid, we can get the 3D compact representation (\textit{e.g.,} By visualizing points with density greater than 0.5, we can get the 3D representation as shown in Fig. \ref{fig:framework} I. ) 

\subsection{3D Semantic Concept Grounding}
Once we extract the 3D compact representation of the scene, we need to ground the semantic concepts for reasoning from language. 
Recent work from \cite{hong20223d} has proposed to ground concepts from paired 3D assets and question-answers. Though promising results have been achieved on synthetic data, it is not feasible for open-vocabulary 3D reasoning in real-world data, since it is hard to collect large-scale 3D vision-and-language paired data.  To address this challenge, our idea is to leverage  pre-trained 2D vision and language model \cite{Radford2021LearningTV, Ramesh2021ZeroShotTG} for 3D concept grounding in real-world scenes.  But how can we map 2D concepts into 3D neural field representations? Note that 3D compact representations can be learned from 2D multi-view images and that each 2D pixel actually corresponds to several 3D points along the ray. Therefore, it's possible to get 3D features from 2D per-pixel features. Inspired by this, we first add a feature voxel grid representation to DVGO, in addition to density and color, to represent 3D features. 
% it's natural to utilize 2D VLMs to ground semantic concepts on the 3D representations. 
 We then apply CLIP-LSeg\cite{li2022language} to learn per-pixel 2D features, which can be attended to by CLIP concept embeddings. We use an alignment loss to align 3D features with 2D features so that we can perform concept grounding on the 3D representations.
% Since 3D voxel grids and 2D pixels are aligned via alpha compositing, we add one L1 loss to force the features of 3D voxel grids to align with the 2D LSeg pixels based on the alpha values. 

\noindent\textbf{2D Feature Extraction.}
To get per-pixel features that can be attended by concept embeddings, we use the features from language-driven semantic segmentation (CLIP-LSeg) \cite{li2022language}, which learns 2D per-pixel features from a pre-trained vision-language model (\textit{i.e.,} \cite{Radford2021LearningTV}). Specifically, it
uses the text encoder from CLIP, trains an image encoder to produce an embedding vector for each pixel, and calculates the scores of word-pixel correlation by dot-product. By outputting the semantic class with the maximum score of each pixel, CLIP-LSeg is able to perform zero-shot 2D semantic segmentation.

\noindent\textbf{3D-2D Alignment.}
In addition to density and color, we also store a 512-dim feature in each grid cell in the compact representation. To align the 3D per-point features with 2D per-pixel features, we calculate an L1 loss between each pixel and each 3D point sampled on the ray of the pixel. The overall L1 loss along a ray is the weighted sum of all the pixel-point alignment losses, with weights same as the rendering weights: $\mathcal{L}_{\text {feature}}=\sum_{i=1}^K w_i(\|\boldsymbol{f_i}-F(\boldsymbol{r})\|),$
where $\boldsymbol{r}$ is a ray corresponding to a 2D pixel, $F(\boldsymbol{r})$ is the 2D feature from CLIP-LSeg, $K$ is the total number of sampled points along the ray and $\boldsymbol{f_i}$ is the feature of point $i$ by interpolating through the feature voxel grid, $w_i$ is the rendering weight.
% \gc{add equations.} 

\noindent\textbf{Concept Grounding through Attention.}  Since our feature voxel grid representation is learnt from CLIP-LSeg, by calculating the dot-product attention $<\boldsymbol{f}, \boldsymbol{v}> $ between per-point 3D feature $\boldsymbol{f}$ and the CLIP concept embeddings $\boldsymbol{v}$, we can get zero-shot view-independent concept grounding and semantic segmentations in the 3D representation, as is presented in Fig. \ref{fig:framework} IV. 
% \gc{add equations.}

\subsection{Neural Reasoning Operators}
Finally, we use the grounded semantic concepts for 3D reasoning from language. We first transform questions into a sequence of operators that can be executed on the 3D representation for reasoning. We adopt a LSTM-based semantic parser   \cite{Yi2018NeuralSymbolicVD} for that. As \cite{Mao2019TheNC, hong20223d}, we further devise a set of operators which can be executed on the 3D representation.  Please refer to \textbf{Appendix} for a full list of operators.

\noindent\textbf{Filter Operators.}  We filter all the grid cells with a certain semantic concept.

\noindent\textbf{Get\_Instance Operators.} We implement this by utilizing DBSCAN \cite{Ester1996ADA}, an unsupervised algorithm which assigns clusters to a set of points. Specifically, given a set of points in the 3D space, it can group together the points that are closely packed together for instance segmentation.

\noindent\textbf{Relation Operators.} We cannot directly execute the relation on the 3D representation as we have not grounded relations. Thus, we represent each relation using a distinct neural module (which is practical as the vocabulary of relations is limited \cite{Landau1993WhatA}). We first concatenate the voxel grid representations of all the referred objects and feed them into the relation network.
% \yd{Do we do something afterwards -- we first concatenate, then what?} 
The relation network consists of three 3D convolutional layers and then three 3D deconvolutional layers. A score is output by the relation network indicating whether the objects have the relationship or not. Since vanilla 3D CNNs are very slow, we use Sparse Convolution \cite{spconv2022} instead. Based on the relations asked in the questions, different relation modules are chosen. 

% \subsection{Learning 3D Compact Representation}
% In recent years, neural field models(\textit{e.g.,} \cite{mildenhall2020nerf}) have gained much popularity since they can reconstruct a volumetric 3D scene representation from a set of images. Recent works \cite{Garbin2021FastNeRFHN, Hedman2021BakingNR, Yu2021PlenOctreesFR, Sun2022DirectVG} have pushed it further by using classic voxel-grids to explicitly store the scene properties (\textit{e.g.}, density, color and feature) for rendering, which allows for real-time rendering. Since concept grounding and relation learning are expected to work on the per-point features in the 3D space \cite{hong20223d} of thousands of scenes, it's more suitable to use voxel-grid-based methods since they store explicit properties in each point which can be directly used for reasoning, and super-fast convergence makes it feasible to train thousands of scenes. Specifically, we use the fine reconstruction process of Direct Voxel Grid Optimization \cite{Sun2022DirectVG} as our backbone for 3D compact representation for its fast speed. 

% A compact voxel-grid representation models the modalities of interest (\textit{e.g.,} density, color or feature) explicitly in its grid cells. To query the properties at any given 3D point, interpolation is used:
% \begin{equation}
% \operatorname{interp}(\boldsymbol{x}, \boldsymbol{V}):\left(\mathbb{R}^3, \mathbb{R}^{C \times N_x \times N_y \times N_z}\right) \rightarrow \mathbb{R}^C
% \end{equation}
% where $\boldsymbol{x}$ is the queried 3D point,  $\boldsymbol{V}$ is the voxel grid, and $C$ is
% the dimension of one of the modalities, and $N_x, N_Y, N_z$ is the number of voxels. We first predict the density of a specified point by interpolating the density grid. This is crucial for the geometric reconstruction of the scene.  
% \begin{equation}
% \sigma=\operatorname{interp}\left(\boldsymbol{x}, \boldsymbol{V}^{(\text {density })}\right)
% \end{equation}
% where $\sigma$ is the volume density at position $\boldsymbol{x}$. For the modeling of color emission, we use an explicit-implicit hybrid representation where  a shallow MLP is placed after the color voxel grid interpolation process:
% \begin{equation}
% \boldsymbol{c}=\operatorname{MLP}^{(\mathrm{rgb})}\left(\operatorname{interp}\left(\boldsymbol{x}, \boldsymbol{V}^{(\mathrm{color})}\right), \boldsymbol{x}, \boldsymbol{d}\right)
% \end{equation}
% where $\boldsymbol{c}$ is the view-dependent color emission at position $\boldsymbol{x}$ viewing from direction $\boldsymbol{d}$.

% To render the color $\hat{C}(\boldsymbol{r})$ of ray $r$, K points are sampled on ray $r$ with densities and colors $\left\{\left(\sigma_i, \boldsymbol{c}_i\right)\right\}_{i=1}^K$. The K results are accumulated by the quadrature rule by Max \cite{Max1995OpticalMF}:
% \begin{align}
% \hat{C}(\mathbf{r})=\sum_{i=1}^K T_i\left(1-\exp \left(-\sigma_i \delta_i\right)\right) \mathbf{c}_i, 
% &\\
% T_i=\exp \left(-\sum_{j=1}^{i-1} \sigma_j \delta_j\right)
% \end{align}
% where $\delta_i=t_{i+1}-t_i$ is the distance between adjacent points along a ray, and $\alpha_i=1-\exp \left(-\sigma_i \delta_i\right)$ is the alpha value for traditional alpha compositing.

% The backbone is trained by minimizing the mean
% square error between the rendered and observed color. 

% \begin{equation}
% \mathcal{L}_{\text {color }}=\|\hat{C}(\boldsymbol{r})-C(\boldsymbol{r})\|_2^2
% \end{equation}

% By extracting the density values of the voxel grid $\boldsymbol{V}^{(\text {density })} \in \mathbb{R}^{1 \times N_x \times N_y \times N_z}$, we can get the compact 3D representation of the scene, as shown in the middle of Figure 2.

% We refer the readers to \cite{Sun2022DirectVG} for more details about the Direct Voxel Grid Optimization.


% \subsection{3D Semantic Concept Grounding}
% In \cite{hong20223d}, a Neural Descriptor Field (NDF) \cite{simeonov2021neural} which gives a feature vector for each 3D coordinate a feature vector is used for concept grounding by aligning the feature vector with the learned concept embeddings. Drawing inspiration from this, we also propose to use a feature voxel-grid  (in addition to density voxel grid and color voxel grid) used for concept grounding. The compact 3D feature representation is composed of one feature voxel-grid representation plus one view-independent shallow MLP: 

% \begin{equation}
% \boldsymbol{f}=\operatorname{MLP}^{(\mathrm{feature})}\left(\operatorname{interp}\left(\boldsymbol{x}, \boldsymbol{V}^{(\mathrm{feature})}\right), \boldsymbol{x}, \boldsymbol{d}\right)
% \end{equation}

% However, the drawback of \cite{hong20223d} is that the embeddings of concepts are learnt from sratch, which is unrealistic in the open-vocabulary reasoning in real-world data. Furthermore, compared to 2D data, 3D assets are limited in diversity and scale, posing challenges for training large vision-language models (VLMs) on 3D-and-language data. Therefore, there's no large-scale 3D VLMs that can be directly used for concept grounding. On the contrary, there's much progress on large-scale 2D VLMs \cite{Radford2021LearningTV, Ramesh2021ZeroShotTG} thanks to the countless image-caption data on the internet. Since we obtain 3D compact representations from 2D multi-view images, it's natural to utilize 2D VLMs to ground semantic concepts on the 3D representations. Based on the CLIP model \cite{Radford2021LearningTV}, LSeg\cite{Li2022LanguagedrivenSS} manages to ground semantic concepts on each 2D pixel (and thus each ray $r$). We denote the feature of ray $r$ as $F(\boldsymbol{r})$.
% Since 3D voxel grids and 2D pixels are aligned via alpha compositing, we add one L1 loss to force the features of 3D voxel grids to align with the 2D LSeg pixels based on the alpha values. Specifically,

% \begin{equation}
% \mathcal{L}_{\text {feature}}=\sum_{i=1}^K T_i\left(1-\exp \left(-\sigma_i \delta_i\right)\right)(\|\boldsymbol{f}-F(\boldsymbol{r})\|)
% \end{equation}

% Assuming we have a set of concepts $P$, the similarities between a concept $\boldsymbol{p} \in P$ and a feature $\boldsymbol{f}$ is calculated as $\langle \boldsymbol{f}, \boldsymbol{p} \rangle$. We define a \textsc{Filter} operator. Specifically, the 3D compact representation for a semantic class $p$ after filtering out that class is:

% \begin{equation}
% \boldsymbol{V}_{\boldsymbol{p}} =  min(\langle\boldsymbol{V}^{(\mathrm{feature})}, \boldsymbol{p}\rangle, \boldsymbol{V}^{(\mathrm{density})}) 
% \end{equation}

% In practice, we only set the values of voxel grids with densities < 0.5 to 0, since we find that those points are irrelevant to the 3D geometry of the scene.

% To get each instance of the objects of the same category, we use DBSCAN \cite{Ester1996ADA} to implement the \textsc{Get\_Instance} operator which assigns clusters to all true values of $\boldsymbol{V}_{\boldsymbol{p}}$. The DBSCAN takes the 3D coordinates as input.







\section{Experiments}
\label{sec:exp}

 \subsection{Experimental Setup}
% \gc{talk about training setup, evaluation metric, implementation details of your our method}
\noindent\textbf{Evaluation Metric.} We report the visual question answering accuracy on the proposed 3DMV-VQA dataset
w.r.t the four types of questions. The train/val/test split is 7:1:2. 

\noindent\textbf{Implementation Details} For 3D compact representations, we adopt the same architectures as DVGO, except skipping the coarse reconstruction phase and directly training the fine reconstruction phase. After that, we freeze the density voxel grid and color voxel grid, for the optimization of the feature voxel grid only. The feature grid has a world size of 100 and feature dim of 512. We train the compact representations for 100,000 iterations and the 3D features for another 20,000 iterations. For LSeg, we use the official demo model, which has the ViT-L/16 image encoder and CLIP’s ViT-B/32 text encoder. We follow the official script for inference and use multi-scale inference. For DBSCAN, we use an epsilon value of 1.5, minimum samples of 2, and we use L1 as the clustering method. For the relation networks, each relation is encoded into a three-layer sparse 3D convolution network with hidden size 64. The output is then fed into a one-layer linear network to produce a score, which is normalized by sigmoid function. We use cross-entropy loss to train the relation networks, and we use the one-hop relational questions with ``yes/no" answers to train the relation networks.

\subsection{Baselines}
Our baselines range from vanilla neural networks, attention-based methods, fine-tuned from large-scale VLM, and graph-based methods, to neural-symbolic methods.
\begin{itemize}
[align=right,itemindent=0em,labelsep=2pt,labelwidth=1em,leftmargin=*,itemsep=0em]
\item \textbf{LSTM}. The question is transferred to word embeddings which are input into a word-level LSTM \cite{Hochreiter1997LongSM}. The last LSTM hidden state is fed into a multi-layer perceptron (MLP) that outputs a distribution
over answers. This method is able to model question-conditional bias since it uses no image information.
\item \textbf{CNN+LSTM}. The question is encoded by the final hidden states from LSTM. We use a resnet-50 to extract frame-level features of images and average them over the time dimension. The
features are fed to an MLP to predict the final answer. This is a simple baseline that
examines how vanilla neural networks perform on 3DMV-VQA.
\item \textbf{3D-Feature+LSTM}. We use the 3D features we get from 3D-2D alignment and downsample the voxel grids using 3D-CNN as input, concatenated with language features from LSTM and fed to an MLP.
\item  \textbf{MAC} \cite{Hudson2018CompositionalAN}. MAC utilizes a Memory, Attention and Composition cell to perform iterative reasoning process. Like CNN+LSTM, we use the average pooling over multi-view images as the feature map. 

\item \textbf{MAC(V)}. We treat the multi-view images along a trajectory as a video. We modify the MAC model by applying a temporal attention unit across the video frames to generate a latent encoding for the video.
\item \textbf{NS-VQA}\cite{Yi2018NeuralSymbolicVD}. This is a 2D version of our 3D-CLR model. We use CLIP-LSeg to ground 2D semantic concepts from multi-view images, and the relation network also takes the 2D features as input. We execute the operators on each image and max pool from the answers to get our final predictions.
\item \textbf{ALPRO} \cite{Li2022AlignAP}. ALPRO is a video-and-language pre-training framework. A transformer model is pretrained  on large webly-source video-text pairs and can be used for downstream tasks like Video Question answering.
\item \textbf{LGCN} \cite{Huang2020LocationAwareGC}. LGCN represents the contents in the video as a location-aware graph by incorporating the location information of an
object into the graph construction.
% \item Human Evaluation
\end{itemize}

\subsection{Experimental Results}

\begin{table*}[t]
	\begin{center}\small
 
	\begin{tabular}{lccccc}
	\toprule
      Methods   & Concept & Counting & Relation& Comparison & Overall\\ 
     
    \midrule
        Q-type (rand.)  &49.4 &10.7 &21.6 & 49.2 &26.4\\ 
        Q-type (freq.)  &50.8 &11.3  &23.9 &50.3 &28.2\\
        LSTM           &53.4&15.3&24.0&55.2 &29.8\\
        \midrule
        CNN+LSTM  &57.8&22.1&35.2&59.7 &37.8\\ 
        MAC &62.4&19.7&47.8&62.3 &46.7\\    
        MAC(V) &60.0&24.6&51.6&65.9 &50.0\\
        NS-VQA &59.8&21.5&33.4&61.6 &38.0\\ 
        ALPRO &65.8 &12.7&42.2&68.2 &43.3\\
        LGCN &56.2&19.5&35.5&66.7 &39.1\\
        3D-Feature+LSTM  &61.2 &22.4 & 49.9 & 61.3 &48.2\\ 
        \midrule
        3D-CLR (Ours)  & \textbf{66.1} & \textbf{41.3} &\textbf{57.6}& \textbf{72.3} &\textbf{57.7}\\ 

    \bottomrule
	\end{tabular}
	\end{center}
	\vspace{-13pt}
	\caption{Question-answering accuracy of 3D visual reasoning baselines on different question types.}
	\vspace{-12pt}
	\label{tab:reasoning}
\end{table*}

\noindent\textbf{Result Analysis.} We summarize the performances for each question type of baseline models in Table \ref{tab:reasoning}. All models are trained on the training set until convergence, tuned on the validation set, and evaluated on the test set. We provide detailed analysis below.


First, for the examination of language-bias of the dataset, we find that the performance of LSTM is only slightly higher than random and frequency, and all other baselines outperform LSTM a lot. This suggests that there's little language bias in our dataset. Second, we observe that encoding temporal information in MAC (\textit{i.e.,} MAC(V)) is better than average-pooling of the features, especially in counting and relation. This suggests that average-pooling of the features may cause the model to lose information from multi-view images, while attention on multi-view images helps boost the 3D reasoning performances. Third, we also find that fine-tuning on large-scale pretrained model (\textit{i.e.,} ALPRO) has relatively high accuracies in concept-related questions, but for counting it's only slightly higher than the random baseline, suggesting that pretraining on large-scale video-language dataset may improve the model's perception ability, but does not provide the model with the ability to tackle with more difficult reasoning types such as counting. Next, we find that LGCN has poor performances on the relational questions, indicating that building a location-aware graph over 2D objects still doesn't equip the model with 3D location reasoning abilities. Last but not least, we find that 3D-based baselines are better than their 2D counterparts. 3D-Feature+LSTM performs well on the 3D-related questions, such as counting and relation, than most of the image-based baselines. Compared with 3D-CLR, NS-VQA can perform well in the conceptual questions. However, it underperforms 3D-CLR a lot in counting and relation, suggesting that these two types of questions require the holistic 3D understanding of the entire 3D scenes. Our 3D-CLR outperforms other baselines by a large margin, but is still far from satisfying. From the accuracy of the conceptual question, we can see that it can only ground approximately 66\% of the semantic concepts. This indicates that our 3DMV-VQA dataset is indeed very challenging.

\noindent\textbf{Qualitative Examples.} In Fig. \ref{fig:qualitative}, we show four qualitative examples. From the examples, we show that our 3D-CLR can infer an accurate 3D representation from multi-view images, as well as ground semantic concepts on the 3D representations to get the semantic segmentations of the entire scene.  Our 3D-CLR can also learn 3D relationships such as ``close", ``largest", ``on top of" and so on. However, 3D-CLR also fails on some questions. For the third scene in the qualitative examples, it fails to ground the concepts ``mouse" and ``printer". Also, it cannot accurately count the instances sometimes. We give detailed discussions below. 
\begin{figure*}[t]
\centering
%\includegraphics[width=\linewidth]{figures/qual_v5.pdf}
\includegraphics[width=\linewidth]{figures/qualitative_v2_1.pdf}
\vspace{-6mm}
\caption{Qualitative examples of our 3D-CLR. We can see that 3D-CLR can ground most of the concepts and answer most questions correctly. However, it still fails sometimes, mainly because it cannot separate close object instances and ground small objects. }
\vspace{-4mm}
\label{fig:qualitative}
\end{figure*}

% \begin{figure}[t]
% \centering
% \includegraphics[width=\linewidth]{figures/qual_v2.pdf}
% \caption{}
% \label{fig:qualitative}
% \end{figure}

\subsection{Discussions}
\begin{figure}[t]
\centering
\includegraphics[width=0.98\linewidth]{figures/abs_bar.png}
\vspace{-3mm}

\caption{Model diagnosis of our 3D-CLR. }
\vspace{-5mm}

\label{fig:ablative}
\end{figure}


We perform an in-depth analysis to understand the challenge of this dataset. We leverage the modular design of our 3D-CLR, replacing individual components of the framework with ground-truth annotations for model diagnosis. The result is shown in Fig \ref{fig:ablative}. 3D-CLR w/ Semantic denotes our model with ground-truth semantic concepts from HM3DSem annotations. 3D-CLR w/ Instance denotes that 
we have ground-truth instance segmentations of semantic concepts. From Fig.~\ref{fig:qualitative} and Fig.~\ref{fig:ablative}, we summarize several key challenges of our benchmark:

\noindent \textbf{Very close object instances} From Fig.~\ref{fig:ablative}, we can see that even with ground-truth semantic labeling of the 3D points, 3D-CLR still has unsatisfying results on counting questions. This suggests that the instance segmentations provided by DBSCAN are not accurate enough. From the top two qualitative examples in Fig.~\ref{fig:qualitative}, we can also see that if two chairs contact each other, DBSCAN will not tell them apart and thus have poor performance on counting. One crucial future direction is to improve unsupervised instance segmentations on very close object instances.

\noindent \textbf{Grounding small objects}
Fig.~\ref{fig:ablative} suggests that 3D-CLR fails to ground a large portion of the semantic concepts, which hinders the performance. From the last example in Fig. ~\ref{fig:qualitative}, we can see that 3D-CLR fails to ground small objects like ``computer mouse". Further examination indicates there are two possible reasons: 1) CLIP-LSeg fails to assign the right features to objects with limited pixels; 2) The resolution of feature voxel grid is not high enough and therefore small objects cannot be represented in the compact representation. An interesting future direction would be learning exploration policies that enable the agents to get closer to uncertain objects that cannot be grounded.
% \gc{talk about learning policy to get closer look at uncertain objects.}

\noindent \textbf{Ambiguity on 3D relations} 
Even with ground-truth semantic and instance segmentations, the performance of the relation network still needs to be improved. We find that most of the failure cases are correlated to the ``inside" relation. From the segmentations in Fig.~\ref{fig:qualitative}, we can see that 3D-CLR is unable to ground the objects in the cabinets. A potential solution can be joint depth and segmentation predictions. 
% \noindent\textbf{Ground small objects}

% \noindent\textbf{Sefig:ablativete very close object instances}

% \noindent\textbf{Ambiguity on 3D relations}

% \subsection{Discussions}







% \newpage
% page 7
% \newpage
% page 8

% \newpage
% \subsection{Limitation}
% % \xiaodong{ablation study of the temporal-conv, failed case when changing bird to the flying dinosaur.}
% % \chenyang{Move this section to supp? It does not present our contribution. }
% % page 8.5
% % \newpage

% While our method achieves impressive results in video editing, it still has some limitations. During shape editing, since the motion is leaned by the one-shot video diffusion model~\cite{tuneavideo}, it is difficult to generate totally new motion~(\eg, `swimming' $\xrightarrow{}$ `fly' ) or very different shape~(\eg, `swan' $\xrightarrow{}$ 'pterosaur'). We believe a stronger video diffusion model might solve these problems.
% (b) Our editing capacity is bounded by the performance of pretrained-model, which bring obstacles to generate diverse movie styles as gen1~\cite{gen1} (\eg, claymation)


\section{Conclusion}
In this paper, we propose a new text-driven video editing framework \texttt{FateZero} that performs temporal consistent zero-shot editing of attribute, style, and shape. 
We make the first attempt to study and utilize the cross-attention and spatial-temporal self-attention during DDIM inversion, which provides fine-grained motion and structure guidance at each denoising step.
A new Attention Blending Block is further proposed to enhance the shape editing performance of our framework.
Our framework benefits \textbf{video} editing using widely existing \textbf{image} diffusion models, which we believe will contribute to a lot of new video applications. 

\noindent \textbf{Limitation \& Future Work.}
While our method achieves impressive results,
% in video editing, 
it still has some limitations. During shape editing, since the motion is produced by the one-shot video diffusion model~\cite{tuneavideo}, it is difficult to generate totally new motion~(\eg,`swim'$\xrightarrow{}$`fly' ) or very different shape~(\eg,`swan' $\xrightarrow{}$`pterosaur'). We will test our method on the generic pretrained video diffusion model for better editing abilities.

% We leave the application of our techniques to other pretrained image diffusion models~\cite{controlnet} as future work.
\label{sec:conclusion}




{\small
\bibliographystyle{ieee_fullname}
\bibliography{11_references}
}

\ifarxiv \clearpage \appendix
\label{sec:appendix}

\section{Sensitivity Analysis on Hyper-parameters}
\par
 The hyper-parameters in our method including $\gamma$ in Eq.~8, $\epsilon$ in Eq.~10 and $\lambda$ in Eq.~12. In Figure~\ref{supp_fig1}, a sensitivity study of them on 3-bit ResNet-18 is performed. For $\gamma$, we choose several small values from 0 to 2 since the optimization in data generation process of ImageNet is not as easy as other small-scale datasets. We keep $\epsilon$ on a small magnitude so that the perturbation is not too large, and keep $\lambda$ on a large magnitude so that the magnitude of two loss terms are consistent. 
% The results show that HSAT is not very sensitive to these hyper-parameters, although there is some effect. 
The results show that the performance of HAST is somewhat sensitive to these hyper-parameters, but most of these results ($50.12\% \sim 51.15\%$) are comparable with that of the model fine-tuned on real data ($51.95\%$). Note that the worse results in Figure~\ref{supp_fig1} outperforms the quantized model obtained by the state-of-the-art ZSQ method ($45.51\%$) in a large margin.
% Similar experiments can be conducted to find out the optimal value of these hyper-parameters on other datasets, as listed in Sec.~\ref{Implementation Details}.
We conduct similar experiments to find out the optimal value of these hyper-parameters on other datasets.



\section{Sample Difficulty Promotion Details}
\label{SDP Details}
\par
\textbf{Perturbation Direction Calculation.} In the main paper, we calculate the perturbation $\delta$ by maximizing the sample difficulty, which is closely related to the loss. However, there are two loss terms, i.e., the Kullback-Leibler (KL) loss and the feature alignment (FA) loss in the fine-tuning process. Thus we conduct a further experiment to select the optimal loss for perturbation direction calculation. The experimental results are shown in Table~\ref{supp_table1}. We observe that the choice of loss for calculating the perturbation direction has a certain impact on the performance. Though not optimal for all settings, we choose KL+FA to calculate perturbation direction since it shows the best in most settings.


\begin{table}[h]   
\begin{center}   
\resizebox{\columnwidth}{!}{
\begin{tabular}{c c c c c c}   
\hline 
Dataset & Model & Bit-width & KL & FA & KL+FA \\ \hline
\multirow{2}*{Cifar-10}& \multirow{2}*{ResNet-20} & W4A4 & 92.43 & 92.29 & 92.36 \\ 
\multirow{2}*{}& \multirow{2}*{} & W3A3 & 88.29 & 87.68 & 88.34 \\ \hline
\multirow{2}*{Cifar-100}& \multirow{2}*{ResNet-20} & W4A4 & 66.69 & 66.50 & 66.68 \\ 
\multirow{2}*{}& \multirow{2}*{} & W3A3 & 55.61 & 55.13 & 55.67 \\ \hline
\multirow{2}*{ImageNet}& \multirow{2}*{ResNet-18} & W4A4 & 66.90 & 66.69 & 66.91 \\ 
\multirow{2}*{}& \multirow{2}*{} & W3A3 & 51.06 & 50.87 & 51.15  \\ \hline
\end{tabular} 
}
\caption{Performance of our HAST when calculating perturbation direction with diferent losses. We maximize the gradient of KL, FA and KL+FA respectively to calculate perturbation direction.}
\label{supp_table1} 
\end{center}   
\end{table}


\par
\textbf{loss weights.} We apply sample difficulty promotion to the synthetic samples obtained by hard sample synthesis for more difficult samples. Then both of them are used to fine-tune the quantized model with the same loss weights. Further experiments on the loss weights of the original synthetic samples and the promotional samples are conducted. Experimental results are shown in table~\ref{supp_table2}. The loss weight of the original synthetic samples is denoted as $a$, and that of the promotional samples is denoted as $b$. We perform 3-bit quantization on CIFAR-10 and ImageNet. For CIFAR-10, we achieve the best accuracy of 88.34\% by setting both the weights to 1. When it comes to ImageNet, better performance than that reported in the main paper is obtained by increasing the weight of promotional samples.

\begin{table}[ht]   
\begin{center}   
\resizebox{\columnwidth}{!}{
\begin{tabular}{c c c c c c}   
\hline 
\multirow{2}{*}{$a,b$} & ResNet-20 & ResNet-18 & \multirow{2}{*}{$a,b$} & ResNet-20 & ResNet-18  \\ \cline{2-3} \cline{5-6}
 & Cifar-10 & ImageNet &  & Cifar-10 & ImageNet  \\ \hline
 1,0 & 86.17 & 47.94 & 0,1 & 88.19 & 48.55 \\
 3,1 & 85.92 & 50.52 & 1,4 & 86.69 & 52.14 \\
 2,1 & 87.53 & 50.97 & 1,3 & 86.94 & 53.12 \\
 1,1 & 88.34 & 51.15 & 1,2 & 87.73 & 52.69 \\ \hline
\end{tabular} 
}
\caption{Ablation results of loss weights in W3A3 setting. The loss weights of original synthetic samples and promotional samples are denoted as $a,b$ respectively.}
\label{supp_table2} 
\end{center}   
\end{table}

\section{Feature Alignment Analysis}
\par 
\textbf{Direct feature alignment vs. relaxed feature alignment.} Direct feature alignment~\cite{featurealignment} is an easy and effective way to transfer feature representations by directly using mean square error to align the feature. However, we use attention vector~\cite{AttentionTransfer} to relax the feature alignment constraint due to the limited capacity of quantized model. In this section, we provide the performance comparison of our HAST between using direct feature alignment (DFA) and using relaxed feature alignment (RFA). Table~\ref{supp_table3} shows the experimental results. The relaxed feature alignment obtains better performance in any settings over direct feature alignment. Significant improvements can be observed from 3-bit quantization. This shows that it is harmful for low-precision quantized model to learn the feature representations of full-precision model directly.

\begin{table}[ht]   
\begin{center}   
\resizebox{\columnwidth}{!}{
\begin{tabular}{c c c c c}   
\hline 
Dataset & Model & Bit-width & HAST(DFA) & HAST(RFA) \\ \hline
\multirow{2}*{Cifar-10}& \multirow{2}*{ResNet-20} & W4A4 & 91.99 & 92.36 \\ 
\multirow{2}*{}& \multirow{2}*{} & W3A3 & 83.92 & 88.34 \\ \hline
\multirow{2}*{Cifar-100}& \multirow{2}*{ResNet-20} & W4A4 & 66.53 & 66.68 \\ 
\multirow{2}*{}& \multirow{2}*{} & W3A3 & 51.50 & 55.67 \\ \hline
\multirow{2}*{ImageNet}& \multirow{2}*{ResNet-18} & W4A4 & 66.49 & 66.91 \\ 
\multirow{2}*{}& \multirow{2}*{} & W3A3 & 45.52 & 51.15 \\ \hline
\end{tabular} 
}
\caption{Performance of our HAST with direct feature alignment and relaxed feature alignment.}
\label{supp_table3} 
\end{center}   
\end{table}



\begin{figure*}[ht]
\centering
\begin{minipage}[b]{\linewidth}
        \centering
        \includegraphics[width=2.1in]{parameter_gamma.pdf}
        \includegraphics[width=2.1in]{parameter_epsilon.pdf}
        \includegraphics[width=2.1in]{parameter_lambda.pdf}
\end{minipage}
\vspace{-8mm}
\caption{Sensitivity analysis on hyper-parameters. We report the top-1 accuracy of 3-bit ResNet-18 on ImageNet.}
\label{supp_fig1}
\end{figure*}


\begin{figure*}[ht]
\centering
    \subfloat[Gradient cosine similarity.]{
        \label{supp_fig2.a}
        \includegraphics[width=1.6in]{gradient_cosine_similiarity.pdf}
    }
    \quad
    \subfloat[Distribution of eigenvalues.]{
        \label{supp_fig2.b}
        \includegraphics[width=1.6in]{density_eigenvalue_of_CE.pdf}
        \includegraphics[width=1.6in]{density_eigenvalue_of_KL.pdf}
        \includegraphics[width=1.6in]{density_eigenvalue_of_FA.pdf}
    }
\caption{Further experiments on feature alignment. (a)Gradient cosine similarity of two terms in loss function. (b)Distribution of the eigenvalues for different loss.}
\label{supp_fig2}
\end{figure*}

\par 
\textbf{Cooperation with KL.} Gradient cosine similarity was used in~\cite{AIT} to measure the cooperation ability of multiple loss terms. The authors found that the cross-entropy (CE) loss does not work well with the Kullback-Leibler (KL) loss in network fine-tuning process. We apply this metric in our work. Specifically, we fine-tune the 3-bit ResNet-20 using baseline (CE+KL)~\cite{IntraQ} and our HAST (FA+KL) respectively and measure the cosine similarity of the gradient of two distinct loss terms. As shown in Figure~\ref{supp_fig2.a}, the cosine distance between CE and KL takes negative values throughout the fine-tuning, while that of FA+KL is positive. This implies that the combinations of FA and KL cooperate well, and using them together could enhance each other, which is opposite to the combinations of CE and KL.

\par \textbf{Generalizability.}  Hessian matrix was used in~\cite{AIT} to measure the local curvature of the loss surface and compare the generalizability of the two distinct loss terms. Since Hessian matrix itself is enormous in size and computations involving its entirety is considered almost infeasible, analyzing the eigenvalues of the matrix is often the most preferred way to study its characteristics. Figure~\ref{supp_fig2.b} plots the distribution of the eigenvalues of the Hessian matrix, approximated by PyHessian~\cite{PyHessian}. We separate Hessian calculation for each loss of CE, KL and FA. A huge difference in the local curvature of the loss terms can be observed. While CE has longer tail for high eigenvalues, KL and FA has more concentration to lower eigenvalues, which means the local curvature of loss surface of KL and FA is smaller than that of CE, leading to better genrealizability according to the finding that  smaller local curvature improves generalization~\cite{AIT}.

\section{Results with smaller number of samples}
Table~\ref{supp_table4} shows the ablation on amount of the synthetic samples. The performance drops as the number of samples decreases. However, HAST with only 256 samples still performs better than previous methods, such as IntraQ with 45.51\% using 5120 samples.


\begin{table}[ht]
\centering
    \resizebox{0.45\textwidth}{!}{
    \large
    \begin{tabular}{*{10}{c}}
        \toprule
        Amount & IntraQ(5120) & 256 &  1280 & 2560 & 5120 \\
        \midrule
        ACC(\%) & 45.51 & 49.17 & 49.95 & 50.23 & 51.15 \\
        \bottomrule
    \end{tabular}
    }
\caption{Results with smaller number of samples.}
\label{supp_table4}
\end{table} \fi

\end{document}
