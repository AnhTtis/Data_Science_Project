%\documentclass[linenumbers,twocolumn,trackchanges]{aastex631}
\documentclass[twocolumn,trackchanges]{aastex631}
\newcommand{\vdag}{(v)^\dagger}
\newcommand\aastex{AAS\TeX}
\newcommand\latex{La\TeX}

\def\reference{In: Pei, J., Tseng, V.S., Cao, L., Motoda, H., Xu, G. (eds) Advances in Knowledge
Discovery and Data Mining. Lecture Notes in Computer Science(), vol 7819. Springer, Berlin,
Heidelberg.}

\usepackage{color}

\shorttitle{The LMC globular cluster NGC~2005}
\shortauthors{A.E. Piatti and Y. Hirai}

\begin{document}

\title{The Origin of the Large Magellanic Cloud Globular Cluster NGC~2005}

\author[0000-0002-8679-0589]{Andr\'es E. Piatti}
\affiliation{Instituto Interdisciplinario de Ciencias B\'asicas (ICB), CONICET-UNCUYO, Padre J. Contreras 1300, M5502JMA, Mendoza, Argentina}
\affiliation{Consejo Nacional de Investigaciones Cient\'{\i}ficas y T\'ecnicas (CONICET), Godoy Cruz 2290, C1425FQB,  Buenos Aires, Argentina}

\author[0000-0002-5661-033X]{Yutaka Hirai}
\altaffiliation{JSPS Research Fellow}
\affiliation{Department of Physics and Astronomy, University of Notre Dame,
225 Nieuwland Science Hall, Notre Dame, IN 46556, USA}
\affiliation{Astronomical Institute, Tohoku University,
6-3 Aoba, Aramaki, Aoba-ku, Sendai, Miyagi 980-8578, Japan}
\affiliation{Joint Institute for Nuclear Astrophysics, Center for the Evolution of the Elements (JINA-CEE), USA}

\correspondingauthor{Andr\'es E. Piatti}
\email{e-mail: andres.piatti@unc.edu.ar}

\begin{abstract}

The ancient Large Magellanic Cloud (LMC) globular cluster NGC~2005 has recently been reported
to have an \textit{ex-situ} origin, thus, setting precedents that the LMC could have partially
formed from smaller merged dwarf galaxies. We here provide additional arguments from which
we conclude that is also fairly plausible an \textit{in-situ} origin of NGC~2005, based on the
abundance spread of a variety of chemical elements measured in dwarf galaxies,
their minimum mass in order to form globular
clusters, the globular cluster formation imprints kept in their kinematics, and the recent
modeling showing that explosions of supernovae are responsible for the observed chemical abundance
spread in dwarf galaxies. The present analysis
points to the need for further development of numerical simulations and observational indices that can 
help us to differentiate between two mechanisms of galaxy formation for the LMC, namely, a primordial
dwarf or an initial merging event of smaller dwarfs.

\end{abstract}

\keywords{Dwarf galaxies(416) --- Magellanic Clouds(990) --- Star clusters(1567) --- Globular star clusters(656)
}

\section{Introduction} \label{sec:intro}
\citet{mucciarellietal2021} recently reported that the Large Magellanic Cloud (LMC) old
globular cluster NGC~2005 is the unique surviving relic of a low star formation efficiency
dwarf galaxy that merged into the LMC in the past. The \textit{ex-situ} origin of NGC~2005 was 
claimed from its deficient abundance of some chemical species -- forming from different
nucleosynthesis channels -- with respect to those of five LMC old globular clusters of 
similar metallicities ($-$1.75 $<$ [Fe/H]\footnote{ {[A/B] = $\log_{10}({N_{\mathrm{A}}}/{N_{\mathrm{B}}})-\log_{10}({N_{\mathrm{A}}}/{N_{\mathrm{B}}})_{\odot}$, where $N_{\mathrm{A}}$ and 
$N_{\mathrm{B}}$ are the number densities of elements A and B, respectively.}} $<$ $-$1.69). 
A Fornax-like progenitor of 
NGC~2005 was suggested by arguing that such a dwarf spheroidal galaxy matches the 
peculiar chemical composition of NGC~2005. NGC~2005 is a 13.77$\pm$4.90 Gyr old 
globular cluster \citep{wagnerkaiseretal2017}, with a overall metallicity [Fe/H] =
 $-$1.75$\pm$0.10 \citep{setal92,beetal2002,mucciarellietal2010}, and a total mass of
  log($M/M_{\odot}$) =   5.49$\pm$0.16 \citep{mg2003}.

While the proposed scenario for the formation of NGC~2005 results plausible in the context 
of the hierarchical assembly of galaxies according to the standard cosmological model
\citep[e.g.,][]{mooreetal1999}, there are a couple of inferences made by \citet{mucciarellietal2021} 
in order to conclude on the \textit{ex-situ} origin of NGC~2005 that may allow another interpretation.


Precisely, this work aims to introduce them so that they can trigger further analysis.
The arguments in this work imply that the \citet{mucciarellietal2021}' results would not be conclusive but greatly enrich 
the debate on the NGC~2005 origin. The possible \textit{in-situ} or \textit{ex-situ} formation of NGC~2005 points to the need for a better
understanding of galaxy formation. Particularly, whether the LMC partly formed through
the accretion of smaller galactic systems or from a purely gaseous outside-in formation scenario
\citep{carreraetal2011,pg13} is still under debate. Furthermore, the analysis of the 
origin of NGC~2005 can shed light on some distinctive features that an LMC-like galaxy 
formed as a primordial dwarf should have with respect to an LMC-like galaxy partially built 
from the accretion and merging of smaller sub-units. In this context, it is worth studying
whether there is a minimum mass budget to differentiate the above two modes of 
galaxy formation. As far as we are aware, there are no simulations testing whether it is
possible to distinguish both modes of galaxy formation.

%\deleted{As overviewed by \citet{goswami2020}, chemical elements heavier than lithium were 
%produced in the very early Universe by stellar nucleosynthesis in evolving and exploding stars 
%and were ejected into the interstellar medium when they died. The next generation of
%stars formed out of this metal-rich gas further enriched
%the interstellar medium when they died. Stars of different masses and metallicities go through 
%different nuclear cycles and hence produce different elements. A large amount of work exists on 
%stellar evolution and on the elements they synthesize. Stellar yields, i.e., the amount of newly 
%produced metals that are ejected into the interstellar medium during and at the end of the life 
%cycle of stars, are adopted in chemical evolution studies to interpret observed abundances in 
%terms of galaxy evolution properties.}

In this work, we gathered relevant works available in the literature about
mechanisms of nucleosynthesis that take place during the early life of galaxies in 
order to show that there exists an alternative interpretation for the origin of NGC~2005 
to that suggested by \citet{mucciarellietal2021}. The present results do not discredit 
the possible \textit{ex-situ} origin of this ancient LMC globular cluster but pose the issue in a 
broader context. These results point to the need for detailed simulations exploring
the space of similarities and differences of galaxy formation processes for different
galaxy masses. In order to provide a conclusive answer about the origin
of NGC~2005, further spectroscopic observation campaigns of LMC field stars are needed, as well
as globular cluster formation modeling in the context of galaxy formation with higher resolution and 
precision of the orbital integration. Nevertheless, the present somehow qualitative
arguments shed light on a more comprehensive analysis of the origin of NGC~2005.


\section{Analysis}

The first piece of analysis that led us to support a possible \textit{in-situ} origin of NGC~2005 is
a  rigorous statistical treatment  of the abundances used by \citet{mucciarellietal2021} 
to conclude on the \textit{ex-situ} origin.
\citet{mucciarellietal2021} showed that 13 chemical abundances measured in NGC~2005
are systematically lower  than the values for five LMC globular clusters 
(NGC~1786, 1835, 1916, 2210, 2257)  with metallicities ([Fe/H])  similar to that of NGC~2005. 
However, if the uncertainties are taken into account, those differences change as a function
of the chemical element.


Figure~\ref{fig1} shows a more comprehensive picture in this respect. 
In order to build it, 
we first computed the difference ($\Delta$, absolute value) in abundance ratios between the mean 
abundance ratios for the aforementioned five LMC globular clusters ([X/Fe]$_{\rm mean}$) and 
that of NGC~2005 ([X/Fe]$_{\rm NGC2005}$),
using values kindly provided by A. Mucciarelli. We note that \citet{mucciarellietal2021}
only included the values for [Si/Fe], [Ca/Fe], [Cu/Fe], and [Zn/Fe] in their table~1,  because they 
focused on elements with predictions of stellar yields that are representative of different 
nucleosynthesis channels. Particularly, they rely their analysis on the [Zn/Fe] ratio, for which they
found the largest mean difference (0.68 dex).

\begin{figure}
\includegraphics[width=\columnwidth, bb = 0 0 450 350]{fig1.png}
%\plotone{fig1.png}
\caption{Diagnostic diagram built to illustrate the quality of the measurements
of different chemical abundances ratios in \citet{mucciarellietal2021}.}
\label{fig1}
\end{figure}

For completeness purposes, we included in
Table~\ref{tab1} all the [X/Fe] ratios used in that work. Then, we added  in quadrature their
respective uncertainties $\sigma$[X/Fe]$_{\rm mean}$ and $\sigma$[X/Fe]$_{\rm NGC2005}$,
and calculated $\eta$=$\Delta$/$\sqrt{\sigma [\rm{X/Fe}]_{\rm mean}^2+\sigma [\rm{X/Fe}]_{\rm NGC2005}^2}$, which
 we plotted in Figure~\ref{fig1} as a function of $\Delta$, and included them in Table~\ref{tab1}. 
As can be seen,  [Sc/Fe],  [Co/Fe], [La/Fe], [Zn/Fe], and [Eu/Fe]  ratios show differences $\Delta$ larger than 
3 times the sum of their respective uncertainties, which means these chemical element abundances in 
NGC~2005 and the other five globular clusters are different. We note that $\eta$ values $<$ 3 do not 
warrant a real difference so that
the conclusion on distinctive chemical patterns between NGC~2005 and the other five LMC
globular clusters for  [Si/Fe], [Ca/Fe], and several other chemical abundances should be taken with 
caution. Therefore, if only some chemical elements show real abundance differences between NGC~2005
and 5 LMC globular clusters, both \textit{ex-situ} and \textit{in-situ} formation scenarios are feasible. 


\begin{deluxetable}{lccccc}
\tablecaption{[X/Fe] values from \citet[private communication]{mucciarellietal2021}.}
\label{tab1}
\tablewidth{0pt}
%\begin{tabular}{@{}lccccc}\hline
\tablehead{\colhead{[X/Fe]} & \colhead{NGC~2005} & \colhead{$<$LMC$>$} & \colhead{$\Delta$} & \colhead{$\sigma$} & \colhead{$\eta$}}
           %&    (dex)    & (dex) & (dex) & (dex) & \\\hline
 \startdata
 Si	& 0.08$\pm$0.09 & 0.32$\pm$0.05 & 0.24 & 0.10 & 2.40 \\
 Ca &  0.01$\pm$0.05 & 0.15$\pm$0.03 & 0.14 & 0.06 & 2.33\\
 Sc	& $-$0.39$\pm$0.07 &0.06$\pm$0.04 & 0.45 & 0.08 & 5.62 \\
Ti	& $-$0.06$\pm$0.10 & 0.24$\pm$0.04 & 0.30 & 0.11 & 2.73 \\
V	& $-$0.34$\pm$0.10 & $-$0.10$\pm$0.04 & 0.24 & 0.11 & 2.18\\
Mn	& $-$0.61$\pm$0.09 & $-$0.53$\pm$0.02 &0.08 & 0.09 & 0.89 \\
Co	& $-$0.29$\pm$0.08 & $ $0.05$\pm$0.03 & 0.34 & 0.09 & 3.78 \\
Ni	& $-$0.07$\pm$0.05 & $-$0.01$\pm$0.02 & 0.06 & 0.05 & 1.20 \\
Cu	& $-$1.10$\pm$0.14 & $-$0.67$\pm$0.09 & 0.43 & 0.17& 2.53 \\
Zn	& $-$0.80$\pm$0.20 & $-$0.12$\pm$0.07 & 0.68 & 0.21 & 3.24 \\
Ba	& 0.09$\pm$0.09 & 0.30$\pm$0.11 & 0.21 & 0.14 & 1.50 \\
La	& $-$0.22$\pm$0.07 & 0.39$\pm$0.11 & 0.61 & 0.13 & 4.69 \\
Eu	& 0.28$\pm$0.06 & 0.70$\pm$0.11 & 0.42 & 0.13 & 3.23 \\%\hline
\enddata
%\end{tabular}
%\end{table}
\end{deluxetable}

Figure~\ref{fig1} shows that  the 13 different chemical abundances in 
NGC~2005 and the other five LMC  clusters would not seem to be equally distinguishable, 
and those differences are only exhibited for a couple of the 13  chemical species analyzed
by \citet{mucciarellietal2021}. The unavoidable question arises: what does the difference
in these few chemical species mean? \citet{mucciarellietal2021}
argued that NGC~2005 formed in a Fornax-like dwarf galaxy that was accreted onto the LMC.
According to them, the  Fornax-like dwarf galaxy would have left negligible consequences
in the LMC in the form of relics (galaxy mass ratio $<$ 0.01), except only NGC~2005.
The different chemical abundance features would be a signature that NGC~2005 formed
\textit{ex-situ} the LMC.

However, the global chemical pattern of NGC~2005  is not statistically peculiar 
compared to that 
of the LMC. In order to quantify this, we figured out that we measured 1000 times the
abundance of each of the 13 elements  of Table~\ref{tab1}  in NGC~2005 and in the other 
five LMC globular clusters; then we gathered the 1000 measurements of each element,
and looked at the obtained distributions. We assume, as expected, that this experimental 
exercise will provide normal distribution functions, so that we represented them by
generating 1000 points following a gaussian distribution for each element in NGC~2005
and in the other five LMC globular clusters. In order to do this, we used the {\it random.normal}
library within Numpy\footnote{\url{https://www.numpy.org/}} using the mean values and
errors quoted in Table~\ref{tab1} as {\it loc} and {\it scale} parameter values, respectively
(see Figure~\ref{fig2}).  Note that most of the points are concentrated within 1$\sigma$.
As can be seen, there
are some elements in NGC~2005 and in the five LMC globular clusters
whose point distributions totally overlap (Mn, Ni); other elements with 
a partial overlap (e.g., Ca, V, Ba), and a few ones which look different (Sc, La).
The total overlap of the point distributions for some chemical elements means that 
any possible measure of that element in NGC~2005 can also be obtained for the other
five LMC globular clusters, or role reversal. A similar reasoning can be used for those
elements with a partial or null overlap, respectively. We note that the comparison of these
distributions for each element in NGC~2005 and in the five LMC globular clusters
is more meaningful than the use of the respective $\eta$ 
values (see Table~\ref{tab1}),  although the latter has the advantage of providing
a quantitative measure.


 If we considered altogether the 13 element distributions of NGC~2005 
and compared it with that of the five LMC globular clusters, we would obtain
a measure of the level of similitude between them.
We then
considered the 13 chemical elements together using the 26000 points of Figure~\ref{fig2};
 most of the points distributed within 1$\sigma$ as provided by the normal distribution
law. We built two tables, one for NGC~2005 and another for the five LMC globular clusters
containing the respective 13000 generated previously. Then we statistically estimated the similarity
between these two tables - in a scale from 0 to 1, where 0 means totally different and 1 
means totally equals - between
the 13 chemical abundances in NGC~2005  (grey points in Fig.~\ref{fig2}) and the LMC 
 (orange points in Fig.~\ref{fig2}) 
 using different statistical 
methods, namely: Jaccard similarity (0.47); cosine distance (0.59); S{\o}rensen-Dice statistic (0.64);
Levenshtein, Hamming, Jaro, and Jaro-Winkler distances (0.52); Pearson correlation (0.83); Spearman
correlation (0.81). We used  Python language v3.8.10\footnote{\url{http://www.python.org/}}, 
and the following packages: Numpy; 
Scipy\footnote{\url{https://scipy.org}}; and PyPi\footnote{\url{https://pypi.org/}}, also found in GitHub 
repository\footnote{\url{https://github.com/}}. The Appendix shows the Python scripts used
in order to applied the aforementioned statistics. The resulting similarities between both
samples are given within parenthesis above, following the name of the respective method.
As can be seen, the general consensus of these
statistics is that the chemical abundance pattern of NGC~2005 can partially overlap 
that of the other 5 LMC globular clusters.

\begin{figure}
\plotone{fig2.png}
\caption{[X/Fe] ratios derived by \citet{mucciarellietal2021}
(see Table~\ref{tab1}) for NGC~2005 (grey points) and the
other five LMC globular clusters (orange dots), respectively. Each [X/Fe] ratio is
represented by 1000 points following a gaussian distribution. For the seek of the
reader we included the corresponding 1$\sigma$ error bars represented by black
and red segments, respectively.}
\label{fig2}
\end{figure}

%\begin{figure}
%\includegraphics[width=\columnwidth]{fig2.png}
%\caption{Mean [Zn/Fe] versus [Fe/H] relationship for dwarf galaxies: Carina (yellow), Draco (magenta), Fornax (green),
%Sagittarius (cyan), Sextants (orange), Sculptor (blue), Ursa Minor (red), and the LMC (dark-grey). Filled 
%squares represent field stars of
%dwarf galaxies with mean relationships traced using the same color. The light-grey region
%represents the mean observed field star dispersion, while the dark-grey and cyan squares with error bars
%re the globular clusters NGC~2005 and Pal~12 \citep{sbordoneetal2007},  respectively.}
%\label{fig2}
%\end{figure}

In order to explore such a possibility more deeply, we first thoroughly searched the literature for [X/Fe] ratios measured in the LMC. \citet{hasselquistetal2021} used APOGEE abundances 
\citep{majewskietal2017} to uncover the chemical abundance patterns in massive Milky Way satellites, including the LMC. They carefully selected galaxy stars based
on APOGEE data quality cuts and proper motions. 
For Si, Ca, and Ni (the only three 
elements overlapping those in \citet{mucciarellietal2021}), they obtained the median abundances
and $\pm$1 $\sigma$ uncertainties listed in Table~\ref{tab2}. When comparing these values
with those of five LMC globular clusters in Table~\ref{tab1}, we found that abundances
of Ca and Ni are  less than 1$\eta$ and that of Si is different at 2.5$\eta$ level.
We assume that this result supports that both metallicity scales are similar within the quoted
uncertainties, inhomogeneities and/or systematic effects, if present, being smaller.
We carried out the same statistical
comparison for the values of NGC ~2005,  and found that abundances of Si and Ni are less 
than 1$\eta$, but Ca is different at 1.6$\eta$. They suggest that these chemical elements in
NGC~2005 and the LMC have similar abundances.

\begin{deluxetable*}{lcccccccc}
%\begin{table*}
\tablewidth{0pt}
\tablecaption{Mean [X/Fe] values in dex at the NGC~2005 metallicity ([Fe/H]=$-$1.75$\pm$0.10 dex)
for the observed [X/Fe] vs. [Fe/H] distributions  (References in parenthesis).}
\label{tab2}
%\begin{small}
%\begin{tabular}{@{}lcccccccc}\hline
\tablehead{\colhead{X}	       &  \colhead{LMC}		& \colhead{Fornax} 			&\colhead{Sagittarius} 		&	\colhead{Sextants} 		& \colhead{Sculptor} 		& \colhead{Ursa Minor}}
\startdata
Si	&0.17$\pm$0.03 (1) 	&				&0.17$\pm$0.04 (1)	&				& 	&	\\
Ca	&0.12$\pm$0.05 (1) 	&				&0.02$\pm$0.07 (1)	&				&	&	\\
Sc	&	 	 		&$-$0.07$\pm$0.15 (2)	&$-$0.10$\pm$0.02 (2)	&$-$0.28$\pm$0.17 (2)	&0.00$\pm$0.10 (2)	&$-$0.30$\pm$0.10 (2)\\
Ti	&	 			&0.25$\pm$0.10 (2)	&0.25$\pm$0.10 (2)	&0.25$\pm$0.15 (2)	&0.20$\pm$0.25 (2)	&0.25$\pm$0.15 (2)	\\
V	&	 			&				&				&				&				&	\\
Mn	&	 			&$-$0.60$\pm$0.10 (2)	&$-$0.75$\pm$0.15 (2)	&$-$0.55$\pm$0.10 (2) &$-$0.35$\pm$0.10 (2)&$-$0.70$\pm$0.05 (2)	\\
Co	&	 			&				&				&				&				&	\\
Ni	&$-$0.03$\pm$0.03 (1) &0.00$\pm$0.05 (2)	&$-$0.03$\pm$0.04 (1) & 0.00$\pm$0.10 (2)&$-$0.10$\pm$0.13 (2)  &$-$0.10$\pm$0.07 (2)	\\
	&				&				&$-$0.05$\pm$0.06 (2)&				&				&				\\
Cu	&	 			&				&				&				&				&	\\
Zn	&	 			&				&$-$0.03$\pm$0.12 (2)	&				&$-$0.25$\pm$0.20 (2) &$-$0.20$\pm$0.19 (2)	\\
Ba	&	 			&$-$0.10$\pm$0.07 (2)	&0.01$\pm$0.04 (2)	&0.00$\pm$0.20 (2)	&$-$0.10$\pm$0.20 (2) &$-$0.05$\pm$0.15 (2)	\\
La	&	 			&				&				&				&	&	\\
Eu	&				&				&0.50$\pm$0.05 (2)	&0.55$\pm$0.05 (2)	&0.45$\pm$0.04 (2)&0.50$\pm$0.04 (2)	\\%\hline
%\end{tabular}
\enddata
\tablecomments{References: (1) \citet{hasselquistetal2021}; (2) \citet{reichertetal2020}.}
%\citet{delosreyesetal2022}; (2) \citet{pw2021}; (3) \citet{skuladottiretal2020}; 
%(6) \citet{skuladottiretal2017}.
%\end{table*}
\end{deluxetable*}

We also derived the mean values of different chemical elements of dwarf
galaxies using the homogenized
analysis carried out by \citet{reichertetal2020}, which is, as far as we are aware, the largest 
compilation of these quantities for this type of object. The calculated values are listed in Table~\ref{tab2}. 
As can be seen, the abundance of Ni in Sagittarius is in excellent agreement
with that of \citet{hasselquistetal2021} so that we assumed that both results are in the same
scale within the quoted uncertainties.


For the chemical elements showing $\eta$$>$ 3 
in Figure~\ref{fig1} (Sc, Zn, and Eu), 
we repeated the statistical analysis described above  by comparing the values in Table~\ref{tab1}
with those in Table~\ref{tab2}. We found that: 1) the 5 LMC globular
clusters have  $\eta <$ 3 for Sc, Zn, and Eu with respect to all the
dwarfs included in Table~\ref{tab2},  with the exception of Sc for Sagittarius and Ursa Minor; 2) 
NGC~2005 has only Sc abundance different from Sagittarius and Sculptor and Zn abundance 
different from Sagittarius. The above results show that chemical element abundances that appear 
to be different in NGC~2005 with respect to LMC are also found to be different, unevenly, 
between the LMC and other dwarfs, as well as between NGC~2005 and other dwarfs.

%The derived resulting differences lead us to favor an inhomogeneity-based origin.


 Note that we performed this statistical analysis based on chemical abundances derived by the different sets of analysis, i.e., we adopt chemical abundances of \citet{mucciarellietal2021} and \citet{hasselquistetal2021}. On the other hand, \citet{mucciarellietal2021}' chemical abundances were derived with the same analysis. This difference may introduce additional inhomogeneity and systematic effects.

We would expect that if NGC~2005 formed in a dwarf (Fornax-like) galaxy  as proposed by \citet{mucciarellietal2021}, their chemical abundance patterns 
should be similar.  \citet{mucciarellietal2021} compiled from the literature 
abundances for Si, Ca, Cu, and Zn in Fornax (see their supplementary figure 5). Unfortunately, none of these chemical elements are in
the compilation by \citet{reichertetal2020} to compare one to the other. Nevertheless,
we found that \citet{letarteetal2006} measured Ba, Ni, and Ti abundances for Fornax's globular
clusters. When extrapolating their [X/Fe] vs. [Fe/H] relationships up to [Fe/H] = $-$1.75, we
found that the values for Ba and Ni are similar to those in Table~\ref{tab2}, while that
for Ti is somewhat different. Therefore, according to the compilation by \citet{reichertetal2020},
the chemical element abundances in NGC~2005 and Fornax would not seem to be
clearly different. We note that it would be worth performing further measurements for more 
chemical elements in the LMC and other dwarfs to make a more comprehensive
comparison with the values obtained for NGC~2005.


%The second piece of analysis that also supports a possible in-situ origin of NGC~2005 is
%the anaysis of Zn. Zn has been extensively
%used in the literature to trace the chemical evolution history of dwarf galaxies and of the Milky Way.
%We built Figure~\ref{fig2} with the aim of  illustrating the evolution of Zn with metallicity for some of them.
%In order to do that, we used data from Figure~4 of \citet{skuladottiretal2018}, and from Figures~2 and 7 of 
% \citet[][for the LMC and Fornax, respectively]{mucciarellietal2021}. Figure~\ref{fig2} shows that 
%the analyzed dwarfs would seem to  show a noticeable dispersion. Such a dispersion is found in dwarfs with or without
%globular clusters (see below), so that chemical tagging  would not seem to be a reliable
%indicator to support an ex-situ origin of a globular cluster.  
\section{Discussion}
In this study, we explore whether a scenario of \textit{in-situ} formation of NGC 2005 is still allowed. We have shown that most elements show $\eta$ between 2 and 3, meaning that the probability that the difference in 
chemical abundance is real is more than 95$\%$, it can not be considered insignificant. 
Although this argument would suffer from systematics in different observations, the following discussion of chemo-dynamical properties of NGC 2005 could support the \textit{in-situ} formation scenario.
\subsection{Chemical Abundances}
The analysis of
chemical abundances and their production channels support a possible \textit{in-situ} origin of NGC~2005 even if chemical abundances in NGC 2005 and other LMC's globular clusters are different.
Dispersion of the [Zn/Fe] ratios in dwarf galaxies can be caused by the inhomogeneity of the interstellar medium. \citet{hiraietal2018} performed a series of chemo-dynamical simulations of dwarf galaxies with the Zn enrichment. Their models assume that Zn is synthesized by electron-capture supernovae and hypernovae, while Fe is from core-collapse supernovae (CCSNe) and type Ia supernovae (SNe Ia). They found that scatters of [Zn/Fe] for [Fe/H] $> -$2.5 reflect the inhomogeneity of [Zn/Fe] ratios caused by SNe Ia. As shown in their figures 9 and 11, several stars with [Fe/H] $> -$2 have [Zn/Fe] $< -$1. These stars are formed from gas clouds heavily enriched by SNe Ia. These results mean that inhomogeneity caused by SNe Ia could produce low [X/Fe] ratios at relatively high metallicity in dwarf galaxies.


Characteristics of the [X/Fe] in NGC~2005 suggest that it was formed from the gas cloud heavily affected by the ejecta of SNe Ia. Since they synthesize a large amount of Fe, star clusters formed in gas containing ejecta of SNe Ia tend to show low [X/Fe] ratios if these types of supernovae do not largely synthesize the element X. A notable example is [Eu/Fe], which shows $\Delta$ = 0.42. Eu is almost entirely synthesized by the $r$-process, which does not occur in SNe Ia \citep[e.g.,][]{hiraietal2015, hiraietal2017, wanajoetal2021}.

On the other hand, the difference of the [X/Fe] for elements synthesized by SNe Ia tends to be smaller. The double detonation (CSDD-L) model of sub-Chandrasekhar (sub-$M_{\rm{Ch}}$) white dwarfs in \citet{lachetal2020} synthesizes a large amount of Ca, Mn, and Ni but not much for Co, Cu, and Zn. As shown in Figure \ref{fig1}, $\Delta$ of [Ca/Fe], [Mn/Fe], and [Ni/Fe] are relatively small compared to [Co/Fe], [Cu/Fe], and [Zn/Fe]. \citet{delosreyesetal2022} found the possible contribution of sub-$M_{\rm{Ch}}$ SNe Ia to the Sculptor dwarf galaxy from [Mn/Fe] ratios. These results mean that NGC 2005 could be formed from the gas cloud heavily affected by the ejecta of SNe Ia.

%\deleted{\citet{lachetal2020} analyzed 
% different burning regimes and production sites of chemical elements
%and found a wide variety of SNe Ia whose explosions produce different levels of chemical element
% abundances. For instance, their figure~3 shows that the levels of chemical abundance 
% of Si, Ca, Sc, Ti, Co, Ni, Cu, and Zn measured in 
%NGC~2005 and in the other five LMC clusters (the values listed in table~1 of \citet{mucciarellietal2021} and in
%the present Table~\ref{tab1}; see also Figure~\ref{fig1}) can be produced by a double detonation 
%mechanism and by a delayed detonation for V.  This means that
%these particular types of supernovae distributed across the LMC body could chemically enrich the 
%galaxy at its earliest epoch differently, so that  NGC~2005 and the other five LMC globular
%clusters formed from different enriched gas. In other words, the LMC region where NGC~2005 formed
%was inefficiently enriched compared with those where the other five LMC globular clusters formed.}


The lack of well-mixed gas during the formation of the 
LMC is documented in the case of Fe by the extensive range of [Fe/H] values of the 15 LMC globular clusters 
and field stars ($-$2.0 $\leq$ [Fe/H] $\leq$ $-$1.3), all of them formed in a relatively short timescale 
\citep[$\Delta$(age) $\sim$2 Gyr;][]{pg13,pm2018,piattietal2018c}. Since SNe Ia can be occurred in $\sim$1 Gyr \citep[e.g.,][]{strolger2020}, this timescale is enough to cause SNe Ia in the progenitors of the LMC.

%\deleted{However, further measurements of the abundances of
%different elements across the LMC are still needed to comprehensively understand its star formation
%history.}

%\deleted{According to the recent results gathered in Table~\ref{tab2},
%[X/Fe] spreads are not only found in dwarf galaxies without globular clusters but also in those with
%globular cluster populations (e.g. Sagittarius, Fornax). This implies that the [X/Fe]
%values of globular clusters could span the observed [X/Fe] ranges of their
%progenitor dwarf galaxies.}


%\deleted{The abundance spread in chemical elements is produced along the chemo-dynamical
%enrichment history of a galaxy.  Supernova explosions 
%at different
%timescale can produce these abundance spreads in the chemical elements, whose amplitudes
%depend on the metallicity level, the particular chemical element and the dwarf environment 
% \citep{hiraietal2018,tn2018,skuladottiretal2019,hiraietal2019,grimmeretal2020,palla2022}. Inhomogeneous chemical
% enrichment and stochastic feedback have recently been found to play an important role during the
% formation of dwarf galaxies \citep{chenetal2022}. Particularly, \citet{cr2022}
% showed that stellar feedback  directly impacts galaxy properties, such as their star formation 
% histories and
% metal contents, among others.}
 


Among the 15 LMC globular clusters, four are metal-poorer than NGC~2005; five are of comparable
metallicity, and
other five clusters are metal-richer; the whole globular cluster population spanning the
[Fe/H] range from $-$2.0 up to $-$1.3 \citep{piattietal2019}.  
If we considered the gas cloud metallicities similarly distributed as the
metallicity distribution of the LMC globular clusters (27\% metal-poorer, 40\% similar, and 33\% metal-richer than
[Fe/H]=$-$1.75), then we would find that $\sim$ 27\%,
40\% and 33\% of the whole gathered gas cloud was metal-poorer, with similar
metallicity, and metal-richer than NGC~2005, respectively. The portion of the gas cloud out of which
the five globular clusters with metallicities ([Fe/H]) similar to that of NGC~2005 and NGC~2005 itself were formed
(40\% of the whole gas cloud),  should also have the 13 chemical elements analyzed by \citet{mucciarellietal2021}
distributed similarly as these 6 globular clusters, i.e., five sharing a similar pattern and NGC~2005 with a somewhat
different one. This means that 1/6 of that gas cloud portion (40\%/6 $\approx$ 6\% of the whole cloud, 
$\sim$12$\,\times\,$10$^6$$M_\odot$) should have had the chemical 
abundance pattern found in NGC~2005. This percentage explains that only NGC 2005 has a 
different chemical abundance from other globular clusters.


This estimate is consistent with the gas mass affected by SNe Ia around dwarf galaxies formed in a cosmological zoom-in simulation. Here we analyze the high-resolution cosmological zoom-in simulation of a Milky Way-like galaxy in \citet{hiraietal2022}. This simulation assumes the initial mass function of \citet{chabrier2003} from 0.1 $M_{\odot}$ to 100 $M_{\odot}$ with the nucleosynthesis yields of \citet{nomotoetal2013} for CCSNe and the N100 model of \citet{seitenzahl2013} for SNe Ia. They also adopt a turbulence-induced metal mixing model to compute chemical inhomogeneity correctly \citep{hiraisaitoh2017}. We pick up the most massive satellite dwarf galaxy from this simulation with a total stellar mass of 2.1$\,\times\,10^7\,M_{\odot}$ at $z$ = 0. Although this simulation does not have LMC-mass systems, this satellite is large enough to discuss the inhomogeneity caused by the SNe Ia. As shown in \citet{reichertetal2020}, Fornax dwarf spheroidal galaxy, which has a similar mass to this galaxy, also has significant variations of chemical abundances.

We then compute a gas affected by SNe Ia at the lookback time of 11.5 Gyr within the virial radius of the progenitor of this galaxy. By this analysis, we found that 2.6$\,\times\,10^7\,M_{\odot}$ of gas shows $-2\,<\,$[Fe/H]$\,<\,-1$ and [Mg/Fe]$\,<\,0$, which indicates these gas clouds are affected by SNe Ia. The total gas mass of this galaxy in this epoch is 8.8$\,\times\,10^7\,M_{\odot}$.  Since globular clusters are collisional systems, galaxy formation simulations assuming collisionless systems cannot correctly resolve the formation and evolution of globular clusters. Even though there are such numerical difficulties, this result suggests that there is enough gas around a dwarf galaxy to form a globular cluster affected by SNe Ia together with globular clusters with different chemical abundances.

In addition to the analysis of the cosmological zoom-in simulation, we estimate the star formation rates (SFRs) and the number of SNe Ia ($N_{\rm{Ia}}$) to explain the chemical abundances obtained by \citet{mucciarellietal2021}. Here we estimate the SFRs and $N_{\rm{Ia}}$ using closed box chemical evolution  model \citep{hiraisaitoh2017}. In this model, we adopt exponentially declining SFR, i.e., SFRs are proportional to exp($-t/\tau$), where $\tau\,=\,2\times10^9$ yr. The initial gas mass of this system is $5\times10^9M_{\sun}$. These values result in a model consistent with the LMC's metallicity distribution and star formation timescale. Since this model is to roughly estimate SFRs and $N_{\rm{Ia}}$, we ignore gaseous inflow and outflow. For nucleosynthesis yields, we adopt \citet{nomotoetal2013} for CCSNe and the W7 model of \citet{iwamoto1999} for SNe Ia. Since this yield set tends to overproduce Si and Ca, the resulting [Si/Fe] and [Ca/Fe] are shifted $-$0.2 dex \citep{timmesetal1995,prantzosetal2018}. This shift is done within the uncertainties of nucleosynthesis.

According to this model, SFRs for $\lesssim$1 Gyr are 1 $M_{\sun}\,\rm{yr}^{-1}$ while they are decreased to $10^{-3}\,M_{\sun}\,\rm{yr}^{-1}$ at 13.8 Gyr. The final stellar mass of this system is $3\times10^9M_{\sun}$, consistent with the stellar mass of the LMC. \citet{mucciarellietal2021} also estimated that the LMC globular clusters were formed with 1--1.5 $M_{\sun}\,\rm{yr}^{-1}$ in the early phase.

We have counted the number of type Ia supernovae in this model. When the system’s metallicity is [Fe/H] = $-$1.75, the system is polluted by $5.5\times10^5$ of SNe Ia. At this time, [Si/Fe] and [Ca/Fe] are 0.36 and 0.11, respectively. These values are consistent with the average [X/Fe] values in LMC (Table \ref{tab1}).

We further estimate $N_{\rm{Ia}}$ to explain [Si/Fe] ratios of NGC 2005 by the scenario of the local inhomogeneity. We assume that NGC 2005, with the stellar mass of $3\times10^5M_{\sun}$ was formed from the gas cloud of $5\times10^7M_{\sun}$. This assumption is based on the average star formation efficiency (0.006) of the giant molecular cloud \citep{murray2011}.  By adopting the solar system abundance of \citet{asplundetal2009}, we estimate that there are 1200 $M_{\sun}$ of Fe and Si in the cloud with [Fe/H] = $-$1.75 and [Si/Fe] = 0.32. To decrease the [Si/Fe] to the value of NGC 2005 ([Si/Fe] = 0.08), we estimate that d$M_{\rm{Fe}}$ = 860 $M_{\sun}$ of Fe should be added to the cloud. We then apply the Fe yield ($Y_{\rm{Fe}}$ = 0.75 $M_{\sun}$ for each SNIa) of the W7 model of \citet{iwamoto1999} and ignore the production of Si in SNe Ia. By dividing d$M_{\rm{Fe}}$ by $Y_{\rm{Fe}}$, $N_{\rm{Ia}}$ required to reproduce [Si/Fe] in NGC 2005 is ~1100. This number is only 0.2\% of the total number of SNe Ia in the whole region.


This result means that if there is a region around LMC enriched slightly excess in the ejecta of SNe Ia, a globular cluster with alpha-element abundance similar to NGC 2005 could be formed. Since this is a simple estimation, we ignore the increase of [Fe/H] by additional SNe Ia. We also refrain from doing this estimate on other elements due to the significant uncertainties of nucleosynthesis.


%\deleted{We note that NGC~2005 could have been formed ex-situ the LMC,
%in another dwarf that later was accreted to the LMC, if the merging of smaller dwarfs were 
%confirmed for the earliest epochs of the LMC formation.}

\subsection{Kinematics}
Kinematics of globular clusters let us consider an  \textit{in-situ} origin for NGC~2005.
%We reach here a point where the complement of kinematics information to chemical elements 
%analysis could help to shed light on the origin of NGC~2005. 
Indeed, the kinematics of 
Milky Way globular clusters have been used to disentangle different accretion events
\citep{massarietal2019,kruijssenetal2019} since their motions have kept along their lifetimes' imprints of their origins \citep{piatti2019}. Similarly, 
\citet{bennetetal2022} performed a 6D phase-space analysis from multiple independent analysis techniques
of 31 LMC globular clusters using {\it Gaia} EDR3 \citep{gaiaetal2020b} and {\it Hubble Space Telescope} data.
They found that the system of globular clusters rotates like in a stellar disk with one-dimensional velocity dispersions of order 30 
km\,s$^{-1}$, similar to that of the LMC old stellar disk population. From these results, they argued that most, if not
all, LMC globular clusters formed through a single formation mechanism in the LMC disk, albeit their
significant dispersion in age and metallicity, any accretion signature being absent within the
involved uncertainties. Similarly to outer halo Milky Way globular clusters, which are associated with
dwarf galaxy accretion events \citep[e.g., Sagittarius, Gaia-Enceladus, Sequoia, etc.,][]{forbes2020},
the LMC halo globular cluster should be those with more chances to have an 
\textit{ex-situ} origin, but at present, there is no chemical signature hinting at it. Note that
NGC~2005 is placed in the inner LMC disk. 

Several recent studies support the \textit{in-situ} scenario.
\citet{shaoetal2021} showed that accreted globular clusters in the Milky Way and Fornax
are less centrally concentrated than those formed \textit{in-situ}. Moreover, globular clusters that
escape dwarf satellites of the Milky Way are found orbiting the latter \citep{rostamietal2022}.
\citet{piattietal2019} derived mean proper motions of the 15 LMC globular clusters, and from
existent radial velocities, they computed their velocity vectors. They found that LMC globular
clusters are distributed in two different kinematics groups, namely: those moving in the LMC disk
and others in a spherical component. Since globular clusters in both kinematic–structural components share 
similar ages and metallicities, they concluded that their origin occurred 
through a fast collapse that formed a halo and disk concurrently. NGC~2005 resulted in being 
a disk globular cluster, while among the other 5 LMC clusters, three and two are in the disk and halo,
respectively.


In addition to kinematics, the mass required to form globular clusters would support the \textit{in-situ} formation scenario. \citet{eadieetal2021} estimated a minimum
galaxy stellar mass required to form globular clusters of $\sim$10$^7$$M_\odot$ 
We found from Table~\ref{tab3} that only Sagittarius and Fornax could form globular clusters.
Carina, Draco, and Ursa Minor do not have globular clusters (either formed \textit{in-situ}
or \textit{ex-situ}). With a total galaxy stellar masses of
$\sim$10$^{5.4}$$M_\odot$ it is probable that these galaxies are primordial dwarfs, i.e.,
they did not form from the merger of smaller galaxies.

\begin{deluxetable}{lcc}
\tablewidth{0pt}
\tablecaption{Stellar mass of dwarf galaxies using the absolute $M_V$ magnitudes compiled by \citet{drlicawagneretal2020} and interpolating them in figure~5 of \citet{georgievetal2010}.}
\label{tab3}
%\begin{small}
%\begin{tabular}{@{}lcc}\hline
\tablehead{\colhead{Name} 		& \colhead{$M_V$ (mag)} & \colhead{log(stellar mass /$M_\odot$)} }%\\\hline
\startdata
Carina		&	$-$9.43		&	5.6$^{+0.3}_{-0.2}$ \\
Draco		&	$-$8.71		&	5.3$^{+0.3}_{-0.2}$ \\
Fornax		&	$-$13.46	&	7.4$^{+0.3}_{-0.2}$ \\
Sagittarius	&    $-$13.50		&	7.4$^{+0.3}_{-0.2}$ \\
Sextants		&	$-$8.72		&	5.3$^{+0.3}_{-0.2}$ \\
Sculptor		&	$-$10.82	&	6.2$^{+0.2}_{-0.1}$ \\
Ursa Minor	&	$-$9.03		&	5.4$^{+0.3}_{-0.2}$ \\%\hline
\enddata
%\end{tabular}
\end{deluxetable}
%\end{table}
We also computed the LMC mass for a lookback time of 11.5 Gyr, when
all its globular clusters formed \citep{piattietal2019}, using the SFRs derived
by \citet{mazzietal2021} and \citet{massanaetal2022}. We obtained an LMC mass of
$\sim$10$^8$$M_\odot$. Thus, the pre-enriched gas cloud out of which the LMC globular clusters formed
could have been a gathering of smaller pieces, each with a particular chemical
enrichment history.

 Following these discussions, we anticipate that stars with chemical abundances similar to NGC 2005 would be formed if there are gas clouds with enough mass ($\sim$10$^7\,M_{\sun}$) enriched by SNe Ia larger than the other region. On the other hand, in \citet{mucciarellietal2021}’ scenario, NGC 2005 would be formed around a Fornax-like dwarf galaxy and later accreted to the LMC. Our discussion suggests that NGC 2005 could be formed \textit{in-situ} without imposing unphysical assumptions.
\section{Conclusions}
The present analysis shows that the 13 chemical elements employed by
\citet{mucciarellietal2021} to claim an \textit{ex-situ} origin of NGC~2005
are not all of the same accuracy. Consequently, they cannot be used
indistinctly to support abundance differences between the [X/Fe]
values derived for NGC~2005 and for five LMC globular clusters with 
similar metallicities. Nevertheless, the abundance differences measured 
for some chemical elements ($>$ 3$\sigma$) are yielded by 
SNe Ia. Different 
dwarf galaxies with studied chemical enrichment histories show 
abundances spread of the considered chemical elements that
encompass the mean values of NGC~2005, which means that NGC~2005
could have been born in any of these galaxies, including the LMC.
The five LMC globular clusters with [Fe/H] values similar to that
of NGC~2005 belong to the inner disk (3) and the outer halo (2) 
of the LMC, and they have similar individual [X/Fe] ratios.
The LMC globular clusters span a wide range of metallicities, and 
that range is verified from those populating the kinematically 
different disk and halo substructures, respectively \citep{piattietal2019}.  
Therefore, the presence of NGC~2005 in the inner LMC disk should not catch
our attention to differentiate it from the remaining LMC globular
cluster population. Recent modeling has also shown a widespread abundance of chemical species at the metalicity level of NGC~2005
can be produced by supernova explosions, as has also been probed
in the LMC.

\begin{acknowledgments}
We thank the referee for the thorough reading of the manuscript and
timely suggestions to improve it.
We thank Alessio Mucciarelli for
providing data for Table~\ref{tab1}, and Giuseppina Battaglia for
valuable comments. Y.H. is supported in part by JSPS KAKENHI Grant Numbers JP21J00153, JP20K14532, JP21H04499, JP21K03614, JP22H01259, MEXT as ``Program for Promoting Researches on the Supercomputer Fugaku" (Toward a unified view of the universe: from large scale structures to planets, Grant No. JPMXP1020200109), JICFuS, and grants PHY 14-30152; Physics Frontier Center/JINA Center for the Evolution of the Elements (JINA-CEE), and OISE-1927130: The International Research Network for Nuclear Astrophysics (IReNA), awarded by the U.S. National Science Foundation. Numerical computations and analysis were carried out on Cray XC50 and computers at the Center for Computational Astrophysics, the National Astronomical Observatory of Japan, and the Yukawa Institute Computer Facility. This research has made use of NASA's Astrophysics Data System Bibliographic Services.
\end{acknowledgments}


%\software{Numpy \citep{harrisetal2020},  
%          Scipy \citep{virtanenetal2020}
%          }


%\bibliographystyle{aasjournal}
%\bibliography{paper} % if your bibtex file is called paper.bib
%%\documentclass[10pt,twocolumn,twoside]{IEEEtran}
%\documentclass[11pt,onecolumn,twoside,draftcls]{IEEEtran}
%\documentclass[5p]{elsarticle}
%\documentclass[preprint,11pt]{elsarticle}
\documentclass[1p,11pt]{elsarticle}
%\documentclass[journal]{IEEEtran}
%\documentclass[conference]{IEEEtran}
%\IEEEoverridecommandlockouts

\usepackage{amsmath}
\usepackage{amssymb}
\usepackage{amsthm}
\usepackage{graphicx}
\usepackage{epstopdf}
\usepackage{bm}
\usepackage{color}
%\usepackage{cite}
\usepackage{subcaption}
\usepackage{dsfont}
\usepackage{algorithm}
\usepackage{algpseudocode}
\usepackage{enumitem}

\biboptions{sort&compress}

\newtheorem{theorem}{Theorem}
\newtheorem{lemma}[theorem]{Lemma}
\newtheorem{corollary}[theorem]{Corollary}
\newtheorem{definition}{Definition}
%\newtheorem{proposition}{Proposition}
\newtheorem{assumption}{Assumption}
\newtheorem{remark}[theorem]{Remark}
\newtheorem{example}[theorem]{Example}

\allowdisplaybreaks

%\journal{Signal Processing}

\begin{document}

\begin{frontmatter}

\title{Estimation of Scalar Field Distribution in the \\Fourier Domain} 

\author[1]{Alex S. Leong\corref{cor1}%
%\fnref{fn1}
}
\ead{alex.leong@defence.gov.au}
\author[2]{Alexei T. Skvortsov%\fnref{fn2}
}
\ead{alexei.skvortsov@defence.gov.au}

\cortext[cor1]{Corresponding author}

\affiliation{organization={Defence Science and Technology Group},
%addressline={506 Lorimer St},
city={Fishermans Bend},
postcode={Vic. 3207},
country={Australia}} 

%\maketitle

\begin{abstract}
In this paper we consider the problem of estimation of scalar field distribution collected from noisy measurements. The field is modelled as a sum of Fourier components/modes, where the number of modes retained and estimated determines in a natural way the approximation quality. An algorithm for estimating the modes using an online optimization approach is presented, under the assumption that the noisy measurements are quantized. The algorithm can estimate time-varying fields through the introduction of a forgetting factor. Simulation studies demonstrate the effectiveness of the proposed approach. 
\end{abstract}

\end{frontmatter}

\section{Introduction}

Estimation of scalar field distribution from a set of point measurements is an important problem often emerging in ecology, geophysics, and many technological applications. Examples include concentration of pollutant, radiation, temperature in urban areas, carbon dioxide emission, methane sources, and many others, see 
\cite{HutchinsonOh,HutchinsonLiu,NeumannBennetts_advanced_robotics,ThomsonHirst,RisticMorelandeGunatilaka,Selvaratnam_CDC,EslingerMendez,NewazJeong,WeidmannHirst,LiChen,MartinPayton,LaSheng,LaShengChen,MorelandeSkvortsov,RazakSukumarChung_journal,LeongZamani_SP,LeongZamaniShames,TranGarratt} and references therein. This approach is often used for indirect inference of scalar fields (pressure, temperature, radiation)  in inaccessible locations where the direct measurements are prohibited due to some geometrical or physical constraints (blocking obstacles, high temperature, or exposure to hazards). The methods of source localisation \cite{HutchinsonOh,HutchinsonLiu,NeumannBennetts_advanced_robotics,ThomsonHirst,RisticMorelandeGunatilaka,Selvaratnam_CDC,EslingerMendez,NewazJeong,WeidmannHirst,LiChen} and mapping \cite{MartinPayton,LaSheng,LaShengChen,MorelandeSkvortsov,RazakSukumarChung_journal,LeongZamani_SP,LeongZamaniShames,TranGarratt} employing remote (and noisy) measurements have attracted increasing attention in recent years due to tremendous progress in instrumentation for aerial and remote sensing using unmanned aerial vehicles (UAVs) and  unmanned ground vehicles (UGVs). This technological advancement necessitates the development and evaluation of some statistical methods and algorithms that can be applied for the timely estimation of the structure (map) of the scalar field in the environment from an ever-increasing set of noisy measurements acquired in a  sequential or concurrent manner (e.g.,  sensing signals from UAVs and UGVs operating over the hazardous area, the intermittent concentration of methane leaked from the ocean floor, oil surface concentration due to androgenic spill, trigger signals from meteorological stations,  etc). These algorithms may become critical for backtracking and characterization of the main sources of the scalar field in the environment which is important for the  remediation effectiveness and retrospective forensic analysis.   This was the main motivation for the present study.

Conventionally, in work on estimation of scalar fields, the field is modelled as a sum of radial basis functions (RBFs) or Gaussian mixture models, see, e.g., \cite{LaSheng,LaShengChen,MorelandeSkvortsov,RazakSukumarChung_journal,LeongZamani_SP,LeongZamaniShames,TranGarratt}. Field estimation then reduces to a problem of estimating the parameters of these models. In the current work, we assume the field to be an arbitrary 2D function which can be viewed in the Fourier domain using, e.g., the discrete Fourier transform (DFT) or the discrete cosine transform (DCT) \cite{BritanikYipRao}. For intuition of this approach, suppose we  regard the plot of the field as an image. From image processing, it is well-known that the most important parts of an image are concentrated in the lowest (spatial) frequency components/modes. Our approach to field estimation is then to estimate the low frequency Fourier components.\footnote{We will use the terms Fourier component and DCT component interchangeably in this paper.} One of the advantages for using this Fourier component approach compared to the RBF approach is that it offers a perhaps more natural way to control the accuracy of the approximation, e.g., by controlling the number of Fourier modes used/retained. Furthermore, if one wants to refine the field estimate by estimating more modes, existing estimates of the lower order modes can be reused. 

The main contributions of this paper are:
\begin{itemize}
    \item Rather than the use of radial basis function field models, we model the 2D scalar field in the Fourier domain as a sum of Fourier components. 
    \item A numerical comparison of the approximation capabilities of the Fourier components and RBF field models is carried out.
    \item We show that the RBF field model and Fourier component field model have similar forms, which allows one to leverage existing algorithms for field estimation developed for the RBF field model to estimate the Fourier components under various different measurement models.
    \item For the quantized measurements model, we present in detail how Fourier component estimation can be carried out using an online optimization approach similar to \cite{LeongZamaniShames}. We further extend the approach of \cite{LeongZamaniShames} from binary measurements to multi-level quantized measurements, and from static to time-varying fields. 
\end{itemize}

The organization of the paper is as follows: Section \ref{sec:preliminaries} gives preliminaries on the DCT and motivation for its use in field modelling. Section \ref{sec:system_model} presents our field model, as well as various different sensor measurement models. Section~\ref{sec:DCT_RBF_comparison} compares our Fourier component field model with the RBF field model in terms of approximation performance. Section \ref{sec:DCT_estimation} first relates our field model to the RBF field models considered in previous works, and then considers in detail the estimation of Fourier components using  quantized measurements. Numerical studies  are presented in Section \ref{sec:numerical}. 

\section{Preliminaries}
\label{sec:preliminaries}

Consider a region of interest $\mathcal{S} = [X_{\textnormal{min}}, X_{\textnormal{max}}] \times [Y_{\textnormal{min}}, Y_{\textnormal{max}}]$. Discretize $[X_{\textnormal{min}}, X_{\textnormal{max}}]$ into $N_x$ points and $ [Y_{\textnormal{min}}, Y_{\textnormal{max}}]$ into $N_y$ points as
\begin{align*}
\mathcal{X}_d \triangleq \left\{X_{\textnormal{min}} + \Big(\frac{1}{2} + I_x \Big) \Delta_x:   I_x \in \{0, \dots, N_x - 1\} \right\} \\
\mathcal{Y}_d \triangleq \left\{Y_{\textnormal{min}} + \Big(\frac{1}{2} + I_y \Big) \Delta_y:  I_y \in \{0, \dots, N_y - 1\} \right\},
\end{align*}
where 
$$\Delta_x \triangleq \frac{X_{\textnormal{max}} - X_{\textnormal{min}}}{N_x}, \quad \Delta_y \triangleq \frac{Y_{\textnormal{max}} - Y_{\textnormal{min}}}{N_y}. $$


Our aim is the estimation of 2D distribution of scalar field $\phi(x,y)$, $(x,y) \in \mathcal{S}$,  which is assumed either static or slowly varying. We define
$$\phi_d(I_x, I_y) \triangleq \phi\Big(X_{\textnormal{min}} + \left(1/2 + I_x \right) \Delta_x, Y_{\textnormal{min}} + \left(1/2 + I_y \right) \Delta_y \Big)$$
as the field value at the discretized position $\big(X_{\textnormal{min}} + \left(1/2 + I_x \right) \Delta_x, Y_{\textnormal{min}} + \left(1/2 + I_y \right) \Delta_y \big) \in \mathcal{X}_d \times \mathcal{Y}_d$. 
Recall the (Type-II) discrete cosine transform (DCT), see, e.g., \cite{BritanikYipRao,Strang_DCT}: 
\begin{align*}
C(u,v) &= \sum_{I_x=0}^{N_x-1} \sum_{I_y=0}^{N_y-1} \alpha_x(u) \alpha_y(v) \phi_d(I_x,I_y) \cos \left(\frac{(2I_x+1)\pi u}{2 N_x} \right) \cos \left(\frac{(2I_y+1)\pi v}{2 N_y} \right), \\ &\quad\quad u=0,\dots,N_x-1, \quad v=0,\dots,N_y-1,
\end{align*}
where 
$$ \alpha_x(u) \triangleq \left\{\begin{array}{ll} \sqrt{\frac{1}{N_x}}, & u = 0 \\ 
\sqrt{\frac{2}{N_x}}, & u \neq 0
\end{array} \right., \quad 
\alpha_y(v) \triangleq \left\{\begin{array}{ll} \sqrt{\frac{1}{N_y}}, & v = 0 \\
\sqrt{\frac{2}{N_y}}, & v \neq 0.
\end{array} \right. $$

The inverse DCT is given by:
\begin{align}
\phi_d(I_x,I_y) &= \sum_{u=0}^{N_x-1} \sum_{v=0}^{N_y-1} \alpha_x(u) \alpha_y(v) C(u,v) \cos \left(\frac{(2I_x+1)\pi u}{2 N_x} \right) \cos \left(\frac{(2I_y+1)\pi v}{2 N_y} \right), \label{eqn:inverse_DCT} \\ &\quad\quad I_x=0,\dots,N_x-1, \quad I_y=0,\dots,N_y-1. \nonumber
\end{align}
It will be convenient for our purposes to rewrite \eqref{eqn:inverse_DCT} as
\begin{align}
\phi_d(I_x,I_y) &= \sum_{(u,v) \in \mathcal{U}}  \alpha_x(u) \alpha_y(v) C(u,v) \cos \left(\frac{(2I_x+1)\pi u}{2 N_x} \right) \cos \left(\frac{(2I_y+1)\pi v}{2 N_y} \right), \label{eqn:inverse_DCT_single_summation} \\ &\quad\quad I_x=0,\dots,N_x-1, \quad I_y=0,\dots,N_y-1, \nonumber
\end{align}
where 
$$\mathcal{U} \triangleq \{(u,v): u \in \{0, \dots, N_x-1\}, v \in \{0, \dots, N_y-1\} \}.$$

The most important information about the field distribution is concentrated in the low order modes, i.e. the components corresponding to $\cos \left(\frac{(2I_x+1)\pi u}{2 N_x} \right) \cos \left(\frac{(2I_y+1)\pi v}{2 N_y} \right)$ with $u$ and $v$ small, while higher order modes define the fine structure of the field distribution. See Figs. \ref{fig:field_seed341_DCT} and \ref{fig:field_seed343_DCT} for examples of how retaining different numbers of modes affects the quality of the approximation to the true field. 


\section{System Model}
\label{sec:system_model}

\subsection{Field Model}
Motivated by the above discussion, we propose to approximate \eqref{eqn:inverse_DCT_single_summation} by
\begin{equation}
\label{field_model} 
\begin{split}
\phi_d(I_x,I_y) & \approx \sum_{(u,v)\in \tilde{\mathcal{U}}} \alpha_x(u) \alpha_y(v) C(u,v) \cos \left(\frac{(2I_x+1)\pi u}{2 N_x} \right) \cos \left(\frac{(2I_y+1)\pi v}{2 N_y} \right), \\ & \quad\quad I_x=0,\dots,N_x-1, \quad I_y=0,\dots,N_y-1 \\ & \triangleq \tilde{\phi}_d(I_x,I_y),
\end{split}
\end{equation}
where $\tilde{\mathcal{U}} \subseteq \mathcal{U}$ is the subset of low order modes that we wish to retain.\footnote{In general one could use in \eqref{field_model} coefficients $\tilde{C}(u,v)$ which are not necessarily equal to $C(u,v)$. One reason for taking the coefficients to be equal to $C(u,v)$ is given in Lemma \ref{lemma:optimal_C_DCT}.}

For example, we could  retain the first $\tilde{N}_x \times \tilde{N}_y$ modes, with $\tilde{N}_x \leq N_x, \tilde{N}_y \leq N_y$, so that 
\begin{equation}
\label{eqn:U_tilde_rect}
 \tilde{\mathcal{U}} = \{(u,v): u \in \{0, \dots, \tilde{N}_x-1\}, v \in \{0, \dots, \tilde{N}_y-1\} \}.
 \end{equation}
The total number  of modes retained $\tilde{N} $ is thus equal to $\tilde{N} = \tilde{N}_x  \tilde{N}_y$.

Another possibility is the following:
\begin{equation}
\label{eqn:U_tilde_largest}
\tilde{\mathcal{U}} = \{ \tilde{N} \textnormal{ pairs } (u,v) \textnormal{ with smallest values of } (u+1)^2 + (v+1)^2 \}
\end{equation}
which tries to retain the $\tilde{N}$ ``largest'' (in magnitude) modes.\footnote{This is of course an approximation, as exactly determining the $\tilde{N}$ largest modes depends on and requires knowledge of the very field that we are trying to estimate.} The motivation for \eqref{eqn:U_tilde_largest} comes from a result that the DCT coefficients $C(u,v)$ decay as $O \big(\frac{1}{(u+1)^2 + (v+1)^2} \big)$ for $u,v \rightarrow \infty$ \cite{YamataniSaito}. Thus the larger components will usually have smaller values of  $(u+1)^2 + (v+1)^2$, leading to the choice \eqref{eqn:U_tilde_largest}. In numerical simulations, we have found \eqref{eqn:U_tilde_largest} to give better approximations than \eqref{eqn:U_tilde_rect} (for the same number of retained modes $\tilde{N}$) in many, though not all, cases. 


\subsection{Measurement Models}
At position $\big(X_{\textnormal{min}} + \left(1/2 + I_x \right) \Delta_x, Y_{\textnormal{min}} + \left(1/2 + I_y \right) \Delta_y \big)$, we have noisy measurements of the field
$$z(I_x,I_y) = h(\phi_d(I_x,I_y), n(I_x, I_y)),$$
where $h(\bm{\cdot},\bm{\cdot})$ is a (in general non-linear) function, and  $n(\bm{\cdot},\bm{\cdot})$ is random noise.

For example, we could have additive noise
\begin{equation}
\label{additive_noise_model}
z(I_x,I_y) = \phi_d(I_x,I_y) + n(I_x, I_y),
\end{equation}
similar to \cite{LaSheng,LaShengChen}. 

One could also further quantize \eqref{additive_noise_model}
\begin{equation}
\label{quantized_measurement_model}
z(I_x,I_y) = q(\phi_d(I_x,I_y) + n(I_x, I_y))
\end{equation}
where $q(\bm{\cdot})$ is a quantizer of $L$ levels, say $\{0, 1, \dots, L-1\}$. The quantizer can be expressed in the form 
\begin{equation}
\label{eqn:quantizer}
q(x) = \left\{\begin{array}{cc} 0, & x < \tau_0 \\ 1, & \tau_0 \leq x < \tau_1 \\ \vdots & \vdots \\ L-2, & \tau_{L-3} \leq x < \tau_{L-2} \\ L-1, & x \geq \tau_{L-2}    \end{array} \right. 
\end{equation}
where the quantizer thresholds $\{\tau_0,\dots,\tau_{L-2}\}$ satisfy $\tau_0 \leq \tau_1 \leq \dots \leq \tau_{L-2}$.


The special case of \eqref{quantized_measurement_model}-\eqref{eqn:quantizer} corresponding to a 1-bit quantizer, or binary measurements, is considered in \cite{LeongZamani_SP,LeongZamaniShames,TranGarratt}. It can be expressed as
\begin{equation}
\label{binary_measurement_model}
z(I_x,I_y) = \mathds{1env} \big(\phi_d(I_x,I_y) + n(I_x, I_y) > \tau \big),
\end{equation}
where $\tau$ is the quantizer threshold, and $\mathds{1}(\bm{\cdot})$ is the indicator function that returns 1 if its argument is true and 0 otherwise. 

Another measurement model which has been considered are Poisson measurements \cite{MorelandeSkvortsov}. Define $\mathbf{x} \triangleq (x,y)$. Then in this model
$$z(\mathbf{x}) \sim \texttt{Poisson}(\lambda(\textbf{x})),$$
where 
$$\lambda(\textbf{x}) = \int k(\mathbf{x}' - \mathbf{x}) \phi(\mathbf{x}') d\mathbf{x}'$$
and 
$$k (\mathbf{x}) = \left\{ \begin{array}{cc} \frac{1}{R^2}, & ||\mathbf{x}|| \leq R \\  \frac{1}{||\mathbf{x}||^2}, & ||\mathbf{x}|| \geq R \end{array} \right.$$
for some constant $R$.

\subsection{Problem Statement}
\label{sec:problem_statement}
The problem we wish to consider in this paper is to estimate the coefficients\footnote{When we refer to \emph{estimation of components/modes} in this paper, we specifically mean estimation of the coefficients $C(u,v)$.}
$$C(u,v), \,\,(u,v) \in \tilde{\mathcal{U}}$$
from noisy measurements $\{z(I_x,I_y)\}$ of the field $\phi_d(I_x,I_y)$. The estimation should be done in an online manner such that the estimates are continually updated as new measurements are collected.


\section{Comparison with RBF Field Model}
\label{sec:DCT_RBF_comparison}
Before we consider the problem of estimating the coefficients $C(u,v)$ (which will be studied in Section \ref{sec:DCT_estimation}), we will in this section compare the use of our Fourier component model
\eqref{field_model}
with the radial basis function model considered in \cite{RazakSukumarChung_journal, LeongZamani_SP,LeongZamaniShames,TranGarratt} (see also \cite{LaSheng,LaShengChen,MorelandeSkvortsov} for similar models), in terms of how well they can approximate a field for a given number of modes (for the Fourier component model) or basis functions (for the RBF model). 

\subsection{Fourier Component Field Model}
Define the mean squared error (MSE):
$$ \textnormal{MSE} \triangleq \frac{1}{N_x N_y} \sum_{I_x=0}^{N_x-1} \sum_{I_y=0}^{N_y-1} \Big( \phi_d (I_x, I_y) - \tilde{\phi}_d (I_x, I_y)  \Big)^2,$$
where 
%$$\phi_d(I_x,I_y) = \sum_{(u,v) \in \mathcal{U}} \alpha_x(u) \alpha_y(v) C(u,v) \cos \left(\frac{(2I_x+1)\pi u}{2 N_x} \right) \cos \left(\frac{(2I_y+1)\pi v}{2 N_y} \right)$$
 $\phi_d(I_x,I_y)$ is the (discretized) true field given by \eqref{eqn:inverse_DCT_single_summation} and
$$\tilde{\phi}_d (I_x, I_y)   \triangleq \sum_{(u,v) \in \tilde{\mathcal{U}} } \alpha_x(u) \alpha_y(v) \tilde{C}(u,v) \cos \left(\frac{(2I_x+1)\pi u}{2 N_x} \right) \cos \left(\frac{(2I_y+1)\pi v}{2 N_y} \right)$$
is the approximation of the true field using a subset of modes $\tilde{U}$ and coefficients $\tilde{C}(u,v)$. The expression for $\tilde{\phi}_d (I_x, I_y) $ is the same as \eqref{field_model} except that the coefficients $\tilde{C}(u,v)$ may be different from $C(u,v)$. However, it turns out that setting $\tilde{C}(u,v)$ to be equal to $C(u,v)$ will minimize the MSE. 

\begin{lemma}
\label{lemma:optimal_C_DCT}
Given a subset of modes $\tilde{U}$, the optimal values of $\tilde{C}(u,v)$ that minimize the MSE satisfy
$$ \tilde{C}^*(u,v) = C(u,v), \, \forall (u,v) \in \tilde{U}.$$
\end{lemma}
\begin{proof}
By definition, 
%\begin{equation}
%\label{eqn:MSE_derivation_DCT}
%\begin{split}
\begin{align}
\textnormal{MSE} &= \frac{1}{N_x N_y} \sum_{I_x=0}^{N_x-1} \sum_{I_y=0}^{N_y-1} \Big( \phi_d (I_x, I_y) - \tilde{\phi}_d (I_x, I_y)  \Big)^2  \nonumber \\
& = \frac{1}{N_x N_y} \sum_{I_x=0}^{N_x-1} \sum_{I_y=0}^{N_y-1}  \Bigg( \sum_{(u,v) \in \mathcal{U}} \alpha_x(u) \alpha_y(v) C(u,v) \cos \left(\frac{(2I_x+1)\pi u}{2 N_x} \right) \cos \left(\frac{(2I_y+1)\pi v}{2 N_y} \right) \nonumber \\
& \quad - \sum_{(u,v) \in \tilde{\mathcal{U}} } \alpha_x(u) \alpha_y(v) \tilde{C}(u,v) \cos \left(\frac{(2I_x+1)\pi u}{2 N_x} \right) \cos \left(\frac{(2I_y+1)\pi v}{2 N_y} \right) \Bigg) ^2 \nonumber \\
&= \frac{1}{N_x N_y} \sum_{I_x=0}^{N_x-1} \sum_{I_y=0}^{N_y-1}  \Bigg( \sum_{(u,v) \in \tilde{\mathcal{U}}} \alpha_x(u) \alpha_y(v) \big(C(u,v) - \tilde{C}(u,v) \big) \nonumber  \\ 
& \quad \quad \times \cos \left(\frac{(2I_x+1)\pi u}{2 N_x} \right) \cos \left(\frac{(2I_y+1)\pi v}{2 N_y} \right) \nonumber \\
& \quad + \sum_{(u,v) \in \mathcal{U} \setminus \tilde{\mathcal{U}} } \alpha_x(u) \alpha_y(v) C(u,v) \cos \left(\frac{(2I_x+1)\pi u}{2 N_x} \right) \cos \left(\frac{(2I_y+1)\pi v}{2 N_y} \right) \Bigg) ^2 \nonumber \\
& = \frac{1}{N_x N_y} \sum_{I_x=0}^{N_x-1} \sum_{I_y=0}^{N_y-1}  \Bigg[ \Bigg( \sum_{(u,v) \in \tilde{\mathcal{U}}} \alpha_x(u) \alpha_y(v) \big(C(u,v) - \tilde{C}(u,v) \big) \nonumber \\ 
& \quad \quad \times \cos \left(\frac{(2I_x+1)\pi u}{2 N_x} \right) \cos \left(\frac{(2I_y+1)\pi v}{2 N_y} \right) \Bigg)^2 \nonumber \\
& \quad + 2 \Bigg( \sum_{(u,v) \in \tilde{\mathcal{U}}} \alpha_x(u) \alpha_y(v) \big(C(u,v) - \tilde{C}(u,v) \big) \cos \left(\frac{(2I_x+1)\pi u}{2 N_x} \right) \cos \left(\frac{(2I_y+1)\pi v}{2 N_y} \right) \Bigg) \nonumber \\
& \quad \quad \times \Bigg( \sum_{(u,v) \in \mathcal{U} \setminus \tilde{\mathcal{U}} } \alpha_x(u) \alpha_y(v) C(u,v) \cos \left(\frac{(2I_x+1)\pi u}{2 N_x} \right) \cos \left(\frac{(2I_y+1)\pi v}{2 N_y} \right) \Bigg) \nonumber \\
& \quad + \Bigg(\sum_{(u,v) \in \mathcal{U} \setminus \tilde{\mathcal{U}} } \alpha_x(u) \alpha_y(v) C(u,v) \cos \left(\frac{(2I_x+1)\pi u}{2 N_x} \right) \cos \left(\frac{(2I_y+1)\pi v}{2 N_y} \right) \Bigg)^2 \Bigg] \nonumber \\
& = \frac{1}{N_x N_y} \sum_{I_x=0}^{N_x-1} \sum_{I_y=0}^{N_y-1}  \Bigg[ \Bigg( \sum_{(u,v) \in \tilde{\mathcal{U}}} \alpha_x(u) \alpha_y(v) \big(C(u,v) - \tilde{C}(u,v) \big) \nonumber \\ 
& \quad \quad \times \cos \left(\frac{(2I_x+1)\pi u}{2 N_x} \right) \cos \left(\frac{(2I_y+1)\pi v}{2 N_y} \right) \Bigg)^2 \nonumber \\
& \quad + \Bigg(\sum_{(u,v) \in \mathcal{U} \setminus \tilde{\mathcal{U}} } \alpha_x(u) \alpha_y(v) C(u,v) \cos \left(\frac{(2I_x+1)\pi u}{2 N_x} \right) \cos \left(\frac{(2I_y+1)\pi v}{2 N_y} \right) \Bigg)^2 \Bigg]  \label{eqn:MSE_derivation_DCT}
\end{align}
%\end{split}
%\end{equation}
The last equality follows since 
\begin{align*}
\sum_{I_x=0}^{N_x-1} \sum_{I_y=0}^{N_y-1} & \alpha_x(u) \alpha_y(v) \big(C(u,v) - \tilde{C}(u,v) \big) \cos \left(\frac{(2I_x+1)\pi u}{2 N_x} \right) \cos \left(\frac{(2I_y+1)\pi v}{2 N_y} \right) \\
& \quad \times \alpha_x(u') \alpha_y(v') C(u',v') \cos \left(\frac{(2I_x+1)\pi u'}{2 N_x} \right) \cos \left(\frac{(2I_y+1)\pi v'}{2 N_y} \right)  
\end{align*}
is equal to zero for all $(u,v) \in \mathcal{U}$ and $ (u',v') \in \mathcal{U} \setminus \tilde{\mathcal{U}}$, by orthogonality of the DCT basis vectors \cite{AhmedNatarajanRao,Strang_DCT}. To conclude the proof, we note that the expression for the MSE given in the last equality of \eqref{eqn:MSE_derivation_DCT} is  clearly minimized when $ \tilde{C}(u,v) = C(u,v), \, \forall (u,v) \in \tilde{U}.$
\end{proof}


\subsection{RBF Field Model}
The following RBF field model is used in \cite{RazakSukumarChung_journal, LeongZamani_SP,LeongZamaniShames,TranGarratt}:
\begin{equation}
\label{field_model_RBF}
\phi(\mathbf{x}) \approx \sum_{j=1}^J \beta_j K_j(\mathbf{x}),
\end{equation}
where $\mathbf{x} \triangleq (x,y)$ and $K_j(\mathbf{x}), j=1,\dots,J$ are radial basis functions. In particular, we consider the choice
 \begin{equation}
 \label{eqn:Gaussian_RBF}
 K_j(\mathbf{x}) = \exp \left(- \frac{\|\mathbf{c}_j-\mathbf{x}\|^2}{\sigma_j^2}\right), \quad j=1,\dots,J,
 \end{equation}
which results in a Gaussian mixture model \cite{MorelandeSkvortsov}.
For a given number of basis functions $J$, we assume that the $\mathbf{c}_j$'s and $\sigma_j$'s are chosen,\footnote{The case where the $\mathbf{c}_j$'s and $\sigma_j$'s are also estimated has been considered, but was found to suffer from identifiability issues and sometimes give very unreliable results \cite{LeongZamani_SP}.} while the $\beta_j$'s are free parameters. Algorithms for estimating the $\beta_j$'s are studied in, e.g., \cite{RazakSukumarChung_journal, LeongZamani_SP,LeongZamaniShames,TranGarratt}. Here we consider instead the problem of finding the optimal  $\beta_j$'s in order to minimize the mean squared error, to see how good the RBF model can be when approximating a field for a given set of basis functions. Define
$$ \textnormal{MSE}_{RBF} \triangleq \frac{1}{|\mathcal{S}_d|} \sum_{\mathbf{x} \in \mathcal{S}_d} \Big( \phi (\mathbf{x}) - \sum_{j=1}^J \beta_j K_j(\mathbf{x}) \Big)^2,$$
where $\phi (\mathbf{x}) $ is the true field value at position~$\mathbf{x}$, $\mathcal{S}_d$ is a discretized set of points in the search region $\mathcal{S}$, and $|\mathcal{S}_d|$ is the cardinality of $\mathcal{S}_d$. 

\begin{lemma}
\label{lemma:optimal_beta_RBF}
Given a set of radial basis functions $\{K_1(.), \dots, K_j(.)\}$ and an ordering $\{\mathbf{x}_1, \dots, \mathbf{x}_{|\mathcal{S}_d|} \}$ of the elements in $\mathcal{S}_d$, the optimal values of $(\beta_1, \dots, \beta_J)$ that minimize $ \textnormal{MSE}_{RBF}$ satisfy
$$ \bm{\beta}^* = \left( \mathcal{K}^T \mathcal{K} \right)^{-1} \mathcal{K}^T \bm{\phi},$$
where $\bm{\beta} = \begin{bmatrix} \beta_1 & \dots & \beta_J \end{bmatrix}^T$, $\bm{\phi} = \begin{bmatrix} \phi(\mathbf{x}_1), \dots, \phi(\mathbf{x}_{|\mathcal{S}_d|}) \end{bmatrix}^T$, and 
$$\mathcal{K} = \begin{bmatrix}
K_1(\mathbf{x}_1) & \dots & K_J(\mathbf{x}_1) \\
\vdots & \ddots & \vdots \\
K_1(\mathbf{x}_{|\mathcal{S}_d|}) & \dots & K_J(\mathbf{x}_{|\mathcal{S}_d|})
\end{bmatrix}.$$
\end{lemma}

\begin{proof}
This is a standard application of the optimal solution to a linear least squares / linear regression problem \cite{CalafioreElGhaoui,Murphy_book1}.    
\end{proof}

\subsection{Numerical Experiments}

\begin{figure}[t!]
\centering 
\includegraphics[scale=0.35]{field_seed341_DCT.pdf} 
\caption{True field and approximations obtained by retaining different numbers of modes}
\label{fig:field_seed341_DCT}
\end{figure} 

\begin{figure}[t!]
\centering 
\includegraphics[scale=0.35]{field_seed341_RBF.pdf} 
\caption{True field and approximations obtained by using different numbers of basis functions}
\label{fig:field_seed341_RBF}
\end{figure} 

\begin{figure}[t!]
\centering 
\includegraphics[scale=0.35]{field_seed343_DCT.pdf} 
\caption{True field and approximations obtained by retaining different numbers of modes}
\label{fig:field_seed343_DCT}
\end{figure} 

\begin{figure}[t!]
\centering 
\includegraphics[scale=0.35]{field_seed343_RBF.pdf} 
\caption{True field and approximations obtained by using different numbers of basis functions}
\label{fig:field_seed343_RBF}
\end{figure} 

In Figs.\ref{fig:field_seed341_DCT}-\ref{fig:field_seed343_RBF}  we show two example fields, and the field approximations that are obtained when various different numbers of modes (for Fourier component model) or radial basis functions (for RBF model) are used. The discretization in the true fields is set as $N_x = N_y = 100$ (so that there are $100^2 = 10000$ modes in total). For the RBF model we set $\mathcal{S}_d = \mathcal{X}_d \times \mathcal{Y}_d$, so that the discretized set of points are the same in the MSE calculations. 
For the Fourier component model, we choose $\tilde{\mathcal{U}}$ as in \eqref{eqn:U_tilde_largest} to retain the $\tilde{N}$ ``largest'' modes. For the RBF model, we use $J = J_x \times J_y$ radial basis functions with $\mathbf{c}_j$'s in \eqref{eqn:Gaussian_RBF} placed uniformly on a grid at locations $\mathcal{X}_{RBF} \times \mathcal{Y}_{RBF}$, where
\begin{align*}
\mathcal{X}_{RBF} \triangleq \left\{X_{\textnormal{min}} + \Big(\frac{1}{2} + i_x \Big) \delta_x:   i_x \in \{0, \dots, J_x - 1\} \right\} \\
\mathcal{Y}_{RBF} \triangleq \left\{Y_{\textnormal{min}} + \Big(\frac{1}{2} + i_y \Big) \delta_y:  i_y \in \{0, \dots, J_y - 1\} \right\}
\end{align*}
and 
$$\delta_x \triangleq \frac{X_{\textnormal{max}} - X_{\textnormal{min}}}{J_x}, \quad \delta_y \triangleq \frac{Y_{\textnormal{max}} - Y_{\textnormal{min}}}{J_y}. $$
The $\sigma_j$'s in \eqref{eqn:Gaussian_RBF} are chosen to be equal to $\sigma_j = \max(\delta_x,\delta_y), \forall j$. The $\beta_j$'s used in \eqref{field_model_RBF} are the optimal values computed according to Lemma \ref{lemma:optimal_beta_RBF}. 

In the figures we show two performance measures, 1) the MSE, and 2) the structural similarity (SSIM) index, which originated in \cite{WangBovikSheikhSimoncelli} and has been widely adopted in the image processing community. The structural similarity index is a measure of the similarity between two images. In our case, we can regard $\Phi = \{\phi_d(I_x, I_y): I_x = 0,\dots, N_x - 1, I_y = 0, \dots, N_y - 1 \}$ and $\tilde{\Phi} = \{\tilde{\phi}_d(I_x, I_y): I_x = 0,\dots, N_x - 1, I_y = 0, \dots, N_y - 1 \}$ as the image representations of the true and approximated fields respectively, and compute the SSIM between these two images. The SSIM gives a scalar value between 0 and 1, with $\textnormal{SSIM} = 1$ if the two images to be compared are identical. 
We refer to \cite{WangBovikSheikhSimoncelli,WangBovik_MSE} for the specific equations used to compute the SSIM. 

We see from Figs.\ref{fig:field_seed341_DCT}-\ref{fig:field_seed343_RBF} that as more modes (for Fourier component model) or basis functions (for RBF model) are used, the approximations to the true field improves.  When using a smaller number of modes / basis functions the RBF model seems to give better approximations than the Fourier component model, while for larger numbers of modes / basis functions the two approaches perform similarly. We also observe that for these two examples, using a relatively small number of modes (when compared to the total number of modes of 10000) or basis functions will still result in a qualitatively reasonable approximation to the true field.

Although the Fourier component model does not seem to offer a significant advantage in terms of approximation quality, there are other reasons where one may consider its use. One advantage of the Fourier model is that it provides a natural way to control the number of model parameters (the coefficients $C(u,v)$) that need to be estimated, in that we simply choose however many modes we wish to retain, whereas with the RBF model one would need to also choose the locations $\mathbf{c}_j$ to place the basis functions and what the values of $\sigma_j$ should be. Additionally, if we want to refine our field model with finer structure by including more model parameters, in the Fourier component model we can reuse any previous estimates  (and further improve them) of the lower order modes, since these remain the same in a model with more modes, whereas in the RBF model one would likely need  to recalculate the estimates of all the parameter values when more basis functions are utilized.

\section{Estimation of Fourier Components}
\label{sec:DCT_estimation}
We now return to the problem of estimating the coefficients $C(u,v), \, (u,v) \in \tilde{\mathcal{U}}$ stated in Section~\ref{sec:problem_statement}. 
Given a set of modes to be retained $\tilde{\mathcal{U}}$, of cardinality $\tilde{N}$, define an ordering on $\tilde{\mathcal{U}}$ indexed by $j \in \{0, \dots, \tilde{N}-1\}$. For instance, the elements of $\tilde{\mathcal{U}}$ could be sorted in lexicographic order.  
Denote the $j$-th element under this ordering  by $(u_j, v_j)$, and define 
$$ C_j \triangleq C(u_j,v_j).$$

%Given $u \in \{0,\dots,\tilde{N}_x-1\}$ and $v \in \{0, \dots, \tilde{N}_y-1\}$, define Alternatively, one can define $j \triangleq v + u \tilde{N}_y$, and conversely $u_j  \triangleq j \textnormal{ mod } \tilde{N}_y $, $v_j \triangleq \lfloor j/\tilde{N}_y \rfloor$. 
%$$j \triangleq u + v \tilde{N}_x$$
%and note that $j \in \{0,\dots, \tilde{N}_x  \tilde{N}_y-1\}$. Conversely, given  $j \in \{0,\dots, \tilde{N}_x  \tilde{N}_y-1\}$, define
%$$ u_j \triangleq  \lfloor j/\tilde{N}_x \rfloor, \quad v_j \triangleq j \textnormal{ mod } \tilde{N}_x$$ 
%Now define
%$$ C_j \triangleq C(u_j,v_j).$$
%Forming $C_j = C(u_j, v_j), j = 0, \dots, \tilde{N}_x \tilde{N}_y - 1$ then corresponds to applying the $\mathtt{vec}$ operation \cite{HornJohnson2} on the matrix with entries $C(u,v)$.

Then we can express
\begin{align*}
\tilde{\phi}_d(I_x,I_y) & = \sum_{(u,v) \in \tilde{\mathcal{U}}} \alpha_x(u) \alpha_y(v) C(u,v) \cos \left(\frac{(2I_x+1)\pi u}{2 N_x} \right) \cos \left(\frac{(2I_y+1)\pi v}{2 N_y} \right)
\end{align*}
in the alternative form 
\begin{equation}
\label{field_model_vector}
 \tilde{\phi}_d(I_x,I_y)  = \sum_{j=0}^{\tilde{N}-1} \alpha_x(u_j) \alpha_y(v_j) C_j \cos \left(\frac{(2I_x+1)\pi u_j}{2 N_x} \right) \cos \left(\frac{(2I_y+1)\pi v_j}{2 N_y} \right),
\end{equation}
which is a linear function of $(C_0, \dots, C_{\tilde{N} -1})$.

%In \cite{LeongZamani_SP,LeongZamaniShames,TranGarratt} (see also \cite{LaSheng,LaShengChen,MorelandeSkvortsov,RazakSukumarChung_journal} for similar models), the field is modelled as \eqref{field_model_RBF}.  
Comparing \eqref{field_model_vector} with the RBF field model \eqref{field_model_RBF}, we see that they are both linear functions of the parameters that are to be estimated. Thus the algorithms developed in e.g. \cite{LaSheng,LaShengChen,RazakSukumarChung_journal, LeongZamani_SP,LeongZamaniShames,TranGarratt,MorelandeSkvortsov}  for estimating fields can in principle also be adapted to work for our field model \eqref{field_model_vector}, under their various assumed measurement models.

\begin{remark}
\label{remark:param_magnitudes}
The DCT coefficients which we are trying to estimate can be of substantially different orders of magnitude, with the higher order components being much smaller in magnitude than the ``DC'' component corresponding to  $u=v=0$, due to the result that the DCT coefficients decay as $O \big(\frac{1}{(u+1)^2 + (v+1)^2} \big)$ \cite{YamataniSaito}.
%For instance, the field in Fig. \ref{fig:field_1_modes} has $C(0,0) = 56.31$, $C(1,1) = 7.749$, $C(2,2) = -2.065$, $C(3,3) = 1.040$, $C(4,4) = 0.544$, $C(5,5) = 0.0156$, $C(6,6) = 0.0284$, $C(7,7) = 0.0075$, \dots. 
In order to estimate parameters with such large differences in magnitude, it is desirable to appropriately scale the parameters that are to be estimated, see \eqref{eqn:param_scaling} below. 
\end{remark}

\subsection{Estimation of Fourier Components Using Quantized Measurements}
\label{sec:DCT_estimation_quantized_measurements}
In this subsection we describe an approach to estimating the parameters $C(u,v), \, u=0, \dots, \tilde{N} - 1$, which assumes the quantized measurement model \eqref{quantized_measurement_model}-\eqref{eqn:quantizer}, with the parameters estimated recursively. 
%A sequential Monte Carlo (SMC) approach is presented in Section \ref{sec:SMC_approach}, while an online optimization approach  is presented in Section \ref{sec:online_optim_approach}. The algorithms are similar to those of \cite{LeongZamani_SP} and \cite{LeongZamaniShames} respectively,  
The algorithm uses an online optimization approach similar to \cite{LeongZamaniShames},
however in this paper we will generalize \cite{LeongZamaniShames} from binary measurements to multi-level quantized measurements, and also extend the approach to handle time-varying fields. 

%\subsection{Sequential Monte Carlo approach}
%\label{sec:SMC_approach}
%In the measurement model \eqref{quantized_measurement_model}-\eqref{eqn:quantizer}, assume the noise has Gaussian distribution
%$n(.,.) \sim \mathcal{N}(0, \sigma_n^2)$. The noise variance $\sigma_n^2$ is assumed unknown as the algorithm can also estimate $\sigma_n^2$. Recalling the observation in Remark \ref{remark:param_magnitudes}, we consider the following scaling of the DCT coefficients:
%\begin{equation}
%\label{eqn:param_scaling}
%\theta_j \triangleq \big((u_j+1)^2 + (v_j+1)^2 \big) C_j.
%\end{equation}
%We then define 
%$$\bm{\theta} \triangleq (\theta_0, \dots, \theta_{\tilde{N}_x \tilde{N}_y-1}, \log \sigma_n)$$
%as the vector of parameters that are estimated. 

%Let $z_k$ denote the measurement,  and $(I_{x,k}, I_{y,k})$ the position index, at time/iteration $k$. For notational compactness we also denote
%\begin{equation}
%\label{eqn:I_x_vector}
%\bm{I}_{\textbf{x},k} \triangleq (I_{x,k}, I_{y,k})
%\end{equation}
%and  
%\begin{equation}
%\label{eqn:K_vector}
%\mathbf{K}(\bm{I}_{\textbf{x},k} ) \triangleq \left[\begin{array}{cccc} K_0 (\bm{I}_{\textbf{x},k} ) & K_1 (\bm{I}_{\textbf{x},k} ) &  \dots & K_{\tilde{N}_x \tilde{N}_y-1}(\bm{I}_{\textbf{x},k} ) \end{array} \right]^T,
%\end{equation}
%where
%\begin{equation}
%\label{eqn:K_vector_components}
%K_j (\bm{I}_{\textbf{x},k} ) \triangleq \frac{\alpha_x(u_j) \alpha_y(v_j) }{(u_j\!+\!1)^2 \!+\! (v_j\!+\!1)^2} \cos \Big(\frac{(2I_{x,k}+1)\pi u_j}{2 N_x} \Big) \cos \Big(\frac{(2I_{y,k}+1)\pi v_j}{2 N_y} \Big).
%\end{equation}
%Denote 
%\begin{equation}
%\label{eqn:z_all}
%z_{1:k} \triangleq \{z_1, \dots, z_k\}    
%\end{equation}
%as the set of measurements collected up to time $k$, with corresponding position indices
%\begin{equation}
%\label{eqn:I_x_all}
%\bm{I}_{\textbf{x},1:k} \triangleq \{(I_{x,1:k}, I_{y,1:k})\}.
%\end{equation}
 
%\begin{algorithm}
%\caption{Estimation of Fourier components using SMC approach}
%\label{alg:DCT_SMC_time_varying}
%\begin{algorithmic}[1]
%\State \textbf{Algorithm Parameters}: $N \in \mathbb{N}$, $a \in (0,1)$, $h = \sqrt{1-a^2}$, prior pdf $p_0(\bm{\theta})$, $b \in [0,1]$, distribution of new parameter values $p_{\bm{\theta}_{k-1}}(.)$
%\State \textbf{Inputs}: Initial location index $\bm{I}_{\textbf{x},1}$
%\State \textbf{Outputs}: Particles $\{\bm{\theta}_k^{(i)}\}$ and weights $\{ \bm{w}_k^{(i)} \}$

%\State Sample particles $\bm{\theta}_0^{(i)}, i=1,\dots,N$ from $p_0(\bm{\theta})$, and assign weights $ \bm{w}_0^{(i)} = \frac{1}{N}, i=1,\dots,N$
%\For{$k=1,2,\dots,$}
%	\State Observe $z_k(\bm{I}_{\textbf{x},k})$ at location index $\bm{I}_{\textbf{x},k}$
%	\For{$i=1,\dots,N$}  \label{line_loop_start:Nemeth}
%		\State Compute $ \textbf{m}_{k-1}^{(i)} = a \bm{\theta}_{k-1}^{(i)}  + (1-a) \bm\bar{\bm{\theta}}_{k-1}$ where $ \bar{\bm{\theta}}_{k-1}  = \sum_{i=1}^N \bm{w}_{k-1}^{(i)} \bm{\theta}_{k-1}^{(i)}$
		 
%		\State Assign $\bm{w}_{1,k}^{(i)} \propto  p(z_k(\bm{I}_{\textbf{x},k})|\textbf{m}_{k-1}^{(i)}; \bm{I}_{\textbf{x},k}) \bm{w}_{k-1}^{(i)}$ \label{line:Nemeth:w_1}
%		\State Sample $\bm{\gamma}_k^{(i)} \sim p_{\bm{\theta}_{k-1}^{(i)}} (.) $
%		\State Assign $\bm{w}_{2,k}^{(i)} \propto  p(z_k(\bm{I}_{\textbf{x},k})| \bm{\gamma}_k^{(i)}; \bm{I}_{\textbf{x},k}) \bm{w}_{k-1}^{(i)}$ \label{line:Nemeth:w_2}
%\EndFor
%	\State Normalize $\{\bm{w}_{1,k}^{(i)} \}$ and $\{\bm{w}_{2,k}^{(i)} \}$ such that $\sum_{i=1}^N \bm{w}_{1,k}^{(i)} = 1$ and $\sum_{i=1}^N \bm{w}_{2,k}^{(i)} = 1$ 
%	\State Sample $N$ times with replacement a set of indices $\{i^-: i=1,\dots,N\} $, from a distribution with probabilities $$\mathbb{P}(i^- = j) = \left\{ \begin{array}{cl} (1-b) \bm{w}_{1,k}^{(j)}, & j \in \{1, \dots,N\} \\  b \bm{w}_{2,k}^{(j-N)}, & j \in \{N+1, \dots,2N\}\end{array} \right.$$
%	\For{$i^- \in \{1,\dots,N\}$}
%		\State Sample  $\bm{\theta}_k^{(i)} \sim \mathcal{N}(\textbf{m}_{k-1}^{(i^-)}, h^{2} \textbf{V}_{k-1})$, where 		$\textbf{V}_{k-1}  = \sum_{i=1}^N \bm{w}_{k-1}^{(i)} (\bm{\theta}_{k-1}^{(i)} - \bar{\bm{\theta}}_{k-1}) (\bm{\theta}_{k-1}^{(i)} - \bar{\bm{\theta}}_{k-1})^T$  
%		\State Assign weights $ \bm{w}_k^{(i)} \propto \frac{p(z_k(\bm{I}_{\textbf{x},k})|\bm{\theta}_k^{(i)}; \bm{I}_{\textbf{x},k})}{p(z_k(\bm{I}_{\textbf{x},k})|\textbf{m}_{k-1}^{(i^-)}; \bm{I}_{\textbf{x},k})}$	\label{line:Nemeth_weights1}
%	\EndFor
%	\For{$i^- \in \{N+1,\dots,2N\}$}
%		\State Set  $\bm{\theta}_k^{(i)} = \bm{\gamma}_k^{(i^- - N)}$
%		\State Assign weights $ \bm{w}_k^{(i)} \propto \frac{p(z_k(\bm{I}_{\textbf{x},k}))|\bm{\theta}_k^{(i)}; \bm{I}_{\textbf{x},k})}{p(z_k(\bm{I}_{\textbf{x},k})|\bm{\gamma}_{k}^{(i^- -N)}; \bm{I}_{\textbf{x},k})}$	\label{line:Nemeth_weights2}
%	\EndFor
%	\State Normalize $\{ \bm{w}_k^{(i)} \}$ such that $\sum_{i=1}^N \bm{w}_{k}^{(i)} = 1$   \label{line_loop_end:Nemeth}
%	\State Determine  $\bm{I}_{\textbf{x},k+1} = \texttt{ActiveSensingSMC}(\bm{I}_{\textbf{x},k}, \{\bm{\theta}_k^{(i)}\})$  using Algorithm \ref{alg:active_sensing_SMC} 
%\EndFor
%\end{algorithmic}
%\end{algorithm} 

%We present in Algorithm \ref{alg:DCT_SMC_time_varying} a sequential Monte Carlo (SMC) \cite{Doucet_book} based algorithm that approximates the posterior pdf $ p(\bm{\theta} | z_{1:k}; \bm{I}_{\textbf{x},1:k})$ by a set of particles $\bm{\theta}_k^{(i)}, i=1,\dots,N$, and associated weights $\bm{w}_k^{(i)}, i=1,\dots,N$. 
%Denoting $\bm{\theta}_k^{(i)} \triangleq (\theta_{0,k}^{(i)}, \dots, \theta_{\tilde{N}_x \tilde{N}_y,k}^{(i)})$,  conditional mean estimates at iteration $k$ can be computed from the particle approximation as:
% \begin{align*}
% \hat{C}_{j,k} &= \sum_{i=1}^N \frac{ \bm{w}_{k}^{(i)} \theta_{j,k}^{(i)} }{(u_j+1)^2 + (v_j+1)^2}, \quad j = 0,\dots,\tilde{N}_x \tilde{N}_y - 1,   \quad \quad \hat{\sigma}_{n,k}^2 = \sum_{i=1}^N \bm{w}_k^{(i)} \exp(2 \theta_{\tilde{N}_x \tilde{N}_y,k}^{(i)}). \end{align*}
%In lines \ref{line:Nemeth_weights1} and \ref{line:Nemeth_weights2} of Algorithm \ref{alg:DCT_SMC_time_varying}, the likelihood functions are computed as:
%\begin{align} 
%p(z_k(\bm{I}_{\textbf{x},k}) = 0|\bm{\theta}_k^{(i)}; \bm{I}_{\textbf{x},k} )  &  =  \Phi \bigg( \frac{\tau_0 - \bm{\theta}_k^{(i)T} \mathbf{K}(\bm{I}_{\textbf{x},k} )}{\exp(\theta_{\tilde{N}_x \tilde{N}_y, k}^{(i)})}\bigg)\nonumber \\
% p(z_k(\bm{I}_{\textbf{x},k}) = l|\bm{\theta}_k^{(i)}; \bm{I}_{\textbf{x},k} ) & =  \Phi \bigg( \frac{\tau_l - \bm{\theta}_k^{(i)T} \mathbf{K}(\bm{I}_{\textbf{x},k} )}{\exp(\theta_{\tilde{N}_x \tilde{N}_y, k}^{(i)})}\bigg) -  \Phi \bigg( \frac{\tau_{l-1} - \bm{\theta}_k^{(i)T} \mathbf{K}(\bm{I}_{\textbf{x},k} )}{\exp(\theta_{\tilde{N}_x \tilde{N}_y, k}^{(i)})}\bigg), \quad l = 1, \dots, L-2 \nonumber 
%\\
% p(z_k(\bm{I}_{\textbf{x},k}) = L-1 | \bm{\theta}_k^{(i)} ; \bm{I}_{\textbf{x},k})  & = 1 -  \Phi \bigg( \frac{\tau_{L-2} - \bm{\theta}_k^{(i)T} \mathbf{K}(\bm{I}_{\textbf{x},k} )}{\exp(\theta_{\tilde{N}_x \tilde{N}_y, k}^{(i)})}\bigg), \label{likelihood_fn_multilevel}
%\end{align}
%and similarly for $p (z_k(\bm{I}_{\textbf{x},k}) | \textbf{m}_k^{(i)} ; \bm{I}_{\textbf{x},k}) $ and $p (z_k(\bm{I}_{\textbf{x},k}) | \bm{\gamma}_k^{(i)} ; \bm{I}_{\textbf{x},k}) $, where $\Phi(x) \triangleq \int_{-\infty}^{x} \frac{1}{\sqrt{2\pi}} \exp\left( - \frac{t^2}{2} \right) dt$ is the cdf of the standard normal distribution $\mathcal{N}(0,1)$. Expression \eqref{likelihood_fn_multilevel} generalizes the likelihood functions for the case of binary measurements considered in \cite{LeongZamani_SP}.
%Finally, we point out that Algorithm \ref{alg:DCT_SMC_time_varying} can be used to estimate slowly varying fields or fields with occasional abrupt changes in parameters \cite{NemethFearnheadMihaylova,LeongZamani_SP}.

%\subsubsection{Measurement location selection using active sensing}

%For choosing the locations in which to make measurements, ``active sensing'' algorithms can be used. Given position $\mathbf{x} \in \mathcal{S}$, denote $\bm{I}_{\textnormal{closest}} (\mathbf{x})$ as the closest location index $(I_x, I_y)$ to $\mathbf{x}$. In Algorithm \ref{alg:active_sensing_SMC} we present the active sensing mechanism from \cite{LeongZamani_SP}, which is an information maximization based approach \cite{RisticSkvortsovGunatilaka}, suitably generalized to the multi-level quantized measurement model \eqref{quantized_measurement_model}-\eqref{eqn:quantizer}. It will return the next location index $\bm{I}_{\textbf{x},k+1}$ to travel to, given the current location index $\bm{I}_{\textbf{x},k}$ and current set of particles $\{\bm{\theta}_k^{(i)}\}$. 

%\begin{algorithm}[t]
%\caption{Active sensing algorithm for SMC approach: $\bm{I}_{\textbf{x},k+1} = \texttt{ActiveSensingSMC}(\bm{I}_{\textbf{x},k}, \{\bm{\theta}_k^{(i)}\})$}
%\label{alg:active_sensing_SMC}
%\begin{algorithmic}[1]
%\State \textbf{Algorithm Parameters}: $\varepsilon \geq 0$, $\alpha \in [0,\infty) \backslash \{1\}$, $\rho_0 \geq 0$, $N_\rho \in \mathbb{N}$, $N_d \in \mathbb{N}$, search region $\mathcal{S}$
%\State \textbf{Inputs}:    $\bm{I}_{\textbf{x},k}$, $\{\bm{\theta}_k^{(i)}\}$
%\State \textbf{Output}: Next measurement location index $\bm{I}_{\textbf{x},k+1}$
%\State Set $\mathbf{x}_k = \big(X_{\textnormal{min}} + \left(1/2 + I_{x,k} \right) \Delta_x, Y_{\textnormal{min}} + \left(1/2 + I_{y,k} \right) \Delta_y \big)$
%	\State With probability $\varepsilon$ set $\bm{I}_{\textbf{x},k+1} $ to a random location index in $\{0,\dots,\tilde{N}_x-1\} \times \{0,\dots,\tilde{N}_y-1\}$, otherwise set $$\bm{I}_{\textbf{x},k+1} = \textrm{arg} \max\limits_{\bm{I}_{\textbf{x}'} \in \mathcal{I}_k} \frac{1}{\alpha-1} \sum_{z_{k+1}=0}^{L-1} \gamma_1(z_{k+1}|\bm{I}_{\textbf{x}'}) \ln  \frac{\gamma_\alpha(z_{k+1}|\bm{I}_{\textbf{x}'})}{(\gamma_1(z_{k+1}|\bm{I}_{\textbf{x}'}))^\alpha}$$ 
%	where
%	\begin{align*}
%	&\mathcal{X}_k = \bigg\{ \mathbf{x}_k + \Big(n \rho_0 \cos\Big( \frac{2\pi\ell}{N_d}\Big), n \rho_0 \sin \Big( \frac{2\pi\ell}{N_d} \Big) \Big): %\\ &\quad\quad\quad\quad\quad 
%  n \in\{ 0,\dots,N_\rho\}, \ell \in \{ 0, 1, \dots, N_d - 1\} \bigg\} \cap \mathcal{S} \\  & \mathcal{I}_k = \{\bm{I}_{\textnormal{closest}} (\mathbf{x}):  \mathbf{x} \in \mathcal{X}_k  \}
% \\	&\gamma_\alpha(z_{k+1} | \bm{I}_{\textbf{x}'}) = \frac{1}{N} \sum_{i=1}^N p(z_{k+1}|\bm{\theta}_k^{(i)}; \bm{I}_{\textbf{x}'})^\alpha
%	\end{align*}
%\end{algorithmic}
%\end{algorithm} 

%\subsection{Online optimization approach}
%\label{sec:online_optim_approach}
For the measurement model \eqref{quantized_measurement_model}-\eqref{eqn:quantizer}, $n(\bm{\cdot},\bm{\cdot})$ is taken as zero mean noise (not necessarily Gaussian). 
Recalling the observation in Remark \ref{remark:param_magnitudes}, we consider the following scaling of the DCT coefficients:
\begin{equation}
\label{eqn:param_scaling}
\beta_j \triangleq \big((u_j+1)^2 + (v_j+1)^2 \big) C_j,
\end{equation}
and define 
$$\bm{\beta} \triangleq (\beta_0, \dots, \beta_{\tilde{N}-1})$$
as the vector of parameters that are to be estimated. 

We first introduce some notation. 
Let $z_k$ denote the measurement,  and $(I_{x,k}, I_{y,k})$ the position index, at time/iteration $k$. For notational compactness we also denote
\begin{equation}
\label{eqn:I_x_vector}
\bm{I}_{\textbf{x},k} \triangleq (I_{x,k}, I_{y,k})
\end{equation}
and  
\begin{equation}
\label{eqn:K_vector}
\mathbf{K}(\bm{I}_{\textbf{x},k} ) \triangleq \left[\begin{array}{cccc} K_0 (\bm{I}_{\textbf{x},k} ) & K_1 (\bm{I}_{\textbf{x},k} ) &  \dots & K_{\tilde{N}-1}(\bm{I}_{\textbf{x},k} ) \end{array} \right]^T,
\end{equation}
where
\begin{equation}
\label{eqn:K_vector_components}
K_j (\bm{I}_{\textbf{x},k} ) \triangleq \frac{\alpha_x(u_j) \alpha_y(v_j) }{(u_j\!+\!1)^2 \!+\! (v_j\!+\!1)^2} \cos \Big(\frac{(2I_{x,k}+1)\pi u_j}{2 N_x} \Big) \cos \Big(\frac{(2I_{y,k}+1)\pi v_j}{2 N_y} \Big).
\end{equation}
Denote 
\begin{equation}
\label{eqn:z_all}
z_{1:k} \triangleq \{z_1, \dots, z_k\}    
\end{equation}
as the set of measurements collected up to time $k$, with corresponding position indices
\begin{equation}
\label{eqn:I_x_all}
\bm{I}_{\textbf{x},1:k} \triangleq \{(I_{x,1:k}, I_{y,1:k})\}.
\end{equation}

The idea is to recursively estimate $\bm{\beta}$ by trying to minimize a cost function 
$$J_k (\bm{\beta}; \bm{I}_{\textbf{x},1:k}, z_{1:k}) = \sum_{t=0}^k g_t (\bm{\beta}; \bm{I}_{\textbf{x},t}, z_t)$$ using online optimization techniques \cite{LesageLandryTaylorShames}.
For binary measurements \eqref{binary_measurement_model}, the following per stage cost function from \cite{LeongZamaniShames} can be used:
\begin{equation}
\label{eqn:per_stage_cost_binary}
g_t (\bm{\beta}; \bm{I}_{\textbf{x}}, z) = \left\{\begin{array}{cc} \log(1+\exp(\eta (\bm{\beta}^T \mathbf{K}(\bm{I}_{\textbf{x}}) - \tau))), & z = 0 \\ 
\log(1+\exp(-\eta (\bm{\beta}^T \mathbf{K}(\bm{I}_{\textbf{x}}) - \tau))), & z = 1 
\end{array}
\right.
\end{equation}
where $\eta>0$ is a parameter in the logistic function $\ell(x) \triangleq 1/(1+\exp(\eta x))$, where larger values of $\eta$ will more closely approximate the function $\mathds{1env}(x > 0)$. The cost function \eqref{eqn:per_stage_cost_binary} is similar to cost functions used in binary logistic regression problems \cite[p.516]{CalafioreElGhaoui}. In the current work, we wish to define a cost suitable for multi-level quantized measurements. Note that there are cost functions used in multinomial logistic regression problems \cite{Murphy_book1}, however they are unsuitable for our problem as they usually involve multiple sets of parameters for each possible output $z$, whereas here we just have a single set of parameters $\bm{\beta}$. 

To motivate our cost function, let us look more closely at the binary measurements cost function \eqref{eqn:per_stage_cost_binary}. In the case where the measurement  $z_t$ at time $t$ and position index $\bm{I}_{\textbf{x},t}$ is equal to 0, the cost $g_t (\bm{\beta}; \bm{I}_{\textbf{x},t}, z_t)$ will be small if  $\bm{\beta}^T \mathbf{K}(\bm{I}_{\textbf{x},t}) $ is less than the quantizer threshold $\tau$, and large otherwise. Similarly, when  $z_t=1$,  $g_t (\bm{\beta}; \bm{I}_{\textbf{x},t}, z_t)$ will be small if  $\bm{\beta}^T \mathbf{K}(\bm{I}_{\textbf{x},t}) $ is greater than $\tau$, and large otherwise. For the case of multi-level quantized measurements  with $L$ levels  given by \eqref{eqn:quantizer}, we would like to have a cost function such that 1) when $z_t=0$, $g_t (\bm{\beta}; \bm{I}_{\textbf{x},t}, z_t)$ is small for $\bm{\beta}^T \mathbf{K}(\bm{I}_{\textbf{x},t}) < \tau_0$, and large otherwise, 2)  when $z_t=l, \, l \in \{1,\dots,L-2\}$, $g_t (\bm{\beta}; \bm{I}_{\textbf{x},t}, z_t)$ is small for $\tau_{l-1} \leq \bm{\beta}^T \mathbf{K}(\bm{I}_{\textbf{x},t}) < \tau_l$, and large otherwise, and 3) when $z_t=L-1$,  $g_t (\bm{\beta}; \bm{I}_{\textbf{x},t}, z_t)$ is small for $\bm{\beta}^T \mathbf{K}(\bm{I}_{\textbf{x},t}) > \tau_{L-2}$, and large otherwise. In this paper we will choose the following per stage cost function, which can be easily checked to satisfy these three requirements:
\begin{equation}
\label{eqn:per_stage_cost_multilevel}
g_t (\bm{\beta}; \bm{I}_{\textbf{x}}, z) \triangleq \left\{\begin{array}{ll} \log(1+\exp(\eta (\bm{\beta}^T \mathbf{K}(\bm{I}_{\textbf{x}}) - \tau_0))), & z = 0 \\ 
\log(1+\exp(-\eta (\bm{\beta}^T \mathbf{K}(\bm{I}_{\textbf{x}}) - \tau_{l-1}))) & \\
 \quad + \log(1+\exp(\eta (\bm{\beta}^T \mathbf{K}(\bm{I}_{\textbf{x}}) - \tau_{l}))), & z = l \in \{1,\dots,L-2\}\\
\log(1+\exp(-\eta (\bm{\beta}^T \mathbf{K}(\bm{I}_{\textbf{x}}) - \tau_{L-2}))), & z = L-1. 
\end{array}
\right.
\end{equation}
We remark that \eqref{eqn:per_stage_cost_multilevel} reduces to \eqref{eqn:per_stage_cost_binary} when the measurements are binary. 

Now that the per stage cost \eqref{eqn:per_stage_cost_multilevel} has been defined, we will present the online estimation algorithm. First, the gradient of $g_t(\bm{\cdot};\bm{\cdot},\bm{\cdot})$ can be derived as 
\begin{equation}
\label{eqn:per_stage_gradient}
\nabla g_t (\bm{\beta}; \bm{I}_{\textbf{x}}, z) = \left\{\begin{array}{ll} 
\frac{\eta}{1+\exp(-\eta (\bm{\beta}^T \mathbf{K}(\bm{I}_{\textbf{x}}) - \tau_0))} \mathbf{K}(\bm{I}_{\textbf{x}}), & z = 0 \\
\Big( \frac{-\eta}{1+\exp(\eta (\bm{\beta}^T \mathbf{K}(\bm{I}_{\textbf{x}}) - \tau_{l-1}))} & \\ \quad \quad + \frac{\eta}{1+\exp(-\eta (\bm{\beta}^T \mathbf{K}(\bm{I}_{\textbf{x}}) - \tau_l))} \Big) \mathbf{K}(\bm{I}_{\textbf{x}}), & z = l \in \{1,\dots,L-2\}\\
 \frac{-\eta}{1+\exp(\eta (\bm{\beta}^T \mathbf{K}(\bm{I}_{\textbf{x}}) - \tau_{L-2}))} \mathbf{K}(\bm{I}_{\textbf{x}}), & z = L-1, 
\end{array}
\right.
\end{equation}
while the Hessian of $g_t(\bm{\cdot};\bm{\cdot},\bm{\cdot})$ can be derived as 
\begin{align}
& \nabla^2 g_t (\bm{\beta}; \bm{I}_{\textbf{x}}, z) \nonumber \\& = \left\{\begin{array}{ll} \frac{\eta^2 \exp(-\eta (\bm{\beta}^T \mathbf{K}(\bm{I}_{\textbf{x}}) - \tau_0))}{(1+\exp(-\eta (\bm{\beta}^T \mathbf{K}(\bm{I}_{\textbf{x}}) - \tau_0)))^2} \mathbf{K}(\bm{I}_{\textbf{x}}) \mathbf{K}(\bm{I}_{\textbf{x}})^T, & z = 0 \\
\Big( \frac{\eta^2 \exp(\eta (\bm{\beta}^T \mathbf{K}(\bm{I}_{\textbf{x}}) - \tau_{l-1}))}{(1+\exp(\eta (\bm{\beta}^T \mathbf{K}(\bm{I}_{\textbf{x}}) - \tau_{l-1})))^2} & \\ \quad \quad + \frac{\eta^2 \exp(-\eta (\bm{\beta}^T \mathbf{K}(\bm{I}_{\textbf{x}}) - \tau_l))}{(1+\exp(-\eta (\bm{\beta}^T \mathbf{K}(\bm{I}_{\textbf{x}}) - \tau_l)))^2} \Big)  \mathbf{K}(\bm{I}_{\textbf{x}}) \mathbf{K}(\bm{I}_{\textbf{x}})^T, & z = l \in \{1,\dots,L-2\}\\
 \frac{\eta^2 \exp(\eta (\bm{\beta}^T \mathbf{K}(\bm{I}_{\textbf{x}}) - \tau_{L-2}))}{(1+\exp(\eta (\bm{\beta}^T \mathbf{K}(\bm{I}_{\textbf{x}}) - \tau_{L-2})))^2} \mathbf{K}(\bm{I}_{\textbf{x}}) \mathbf{K}(\bm{I}_{\textbf{x}})^T, & z = L-1. 
\end{array}
\right. \nonumber \\
& = \left\{\begin{array}{ll} 
\frac{\eta^2 \exp(\eta (\bm{\beta}^T \mathbf{K}(\bm{I}_{\textbf{x}}) - \tau_0))}{(1+\exp(\eta (\bm{\beta}^T \mathbf{K}(\bm{I}_{\textbf{x}}) - \tau_0)))^2} \mathbf{K}(\bm{I}_{\textbf{x}}) \mathbf{K}(\bm{I}_{\textbf{x}})^T, & z = 0 \\
\Big( \frac{\eta^2 \exp(\eta (\bm{\beta}^T \mathbf{K}(\bm{I}_{\textbf{x}}) - \tau_{l-1}))}{(1+\exp(\eta (\bm{\beta}^T \mathbf{K}(\bm{I}_{\textbf{x}}) - \tau_{l-1})))^2} & \\ \quad \quad + \frac{\eta^2 \exp(\eta (\bm{\beta}^T \mathbf{K}(\bm{I}_{\textbf{x}}) - \tau_l))}{(1+\exp(\eta (\bm{\beta}^T \mathbf{K}(\bm{I}_{\textbf{x}}) - \tau_l)))^2} \Big)  \mathbf{K}(\bm{I}_{\textbf{x}}) \mathbf{K}(\bm{I}_{\textbf{x}})^T, & z = l \in \{1,\dots,L-2\}\\
 \frac{\eta^2 \exp(\eta (\bm{\beta}^T \mathbf{K}(\bm{I}_{\textbf{x}}) - \tau_{L-2}))}{(1+\exp(\eta (\bm{\beta}^T \mathbf{K}(\bm{I}_{\textbf{x}}) - \tau_{L-2})))^2} \mathbf{K}(\bm{I}_{\textbf{x}}) \mathbf{K}(\bm{I}_{\textbf{x}})^T, & z = L-1. 
\end{array} \right.  \label{eqn:per_stage_Hessian}
\end{align}
An approximate online Newton method \cite{LeongZamaniShames} for estimating the parameters $\bm{\beta}$ is now given by:
\begin{equation}
\label{eqn:approx_ONM1}
 \hat{\bm{\beta}}_{k+1} = \hat{\bm{\beta}}_k - \left( H_k (\hat{\bm{\beta}}_k; \bm{I}_{\textbf{x},1:k}, z_{1:k})  \right)^{-1} G_k (\hat{\bm{\beta}}_k; \bm{I}_{\textbf{x},k}, z_k), 
\end{equation}
where
\begin{align}
G_k (\hat{\bm{\beta}}_k; \bm{I}_{\textbf{x},k}, z_k) & = \nabla g_k (\hat{\bm{\beta}}_k; \bm{I}_{\textbf{x},k}, z_k)  \nonumber \\
H_k(\hat{\bm{\beta}}_{k}; \bm{I}_{\textbf{x},1:k}, z_{1:k}) & =  H_{k-1}(\hat{\bm{\beta}}_{k-1}; \bm{I}_{\textbf{x},1:k-1}, z_{1:k-1}) + \nabla^2 g_k (\hat{\bm{\beta}}_k; \bm{I}_{\textbf{x},k}, z_k) \nonumber \\
H_0 (\hat{\bm{\beta}}_{0}) & = \varsigma I. \label{eqn:approx_ONM2}
\end{align}
The initialization $H_0 (\hat{\bm{\beta}}_{0}) = \varsigma I$  is a Levenberg-Marquardt type modification \cite{ChongZak} to ensure that the matrices $\{H_k\}$ are always non-singular.\footnote{In \cite{LeongZamaniShames} this is equivalently expressed as a full rank initialization  on $\left(H_0(\hat{\bm{\beta}}_{0})\right)^{-1}$.}

In the case where the field (and hence the parameters $\bm{\beta})$ is time-varying, the algorithm \eqref{eqn:approx_ONM1}-\eqref{eqn:approx_ONM2} may not be able to respond quickly to changes in $\bm{\beta}$, due to all past Hessians (including Hessians from old fields) being used in the computation of $H_k(\hat{\bm{\beta}}_{k}; \bm{I}_{\textbf{x},1:k}, z_{1:k}) $ in \eqref{eqn:approx_ONM2}. To overcome this problem, we  will introduce a \emph{forgetting factor} \cite{ManolakisIngleKogon} into the algorithm, 
where the forgetting factor $\delta$ satisfies $0 < \delta \leq 1$, and typically chosen to be close to one. The final estimation procedure is summarized as Algorithm~\ref{alg:DCT_optim_time_varying}. Compared to \eqref{eqn:approx_ONM2}, we note that the Levenberg-Marquardt modification in Algorithm \ref{alg:DCT_optim_time_varying} is done at every time step by adding $\varsigma I$ to $\tilde{H}_k$, as we found that only doing it once at the beginning can lead to algorithm instability due to exponential decay of initial conditions when using a forgetting factor. We also remark that Algorithm \ref{alg:DCT_optim_time_varying} reduces to \eqref{eqn:approx_ONM1}-\eqref{eqn:approx_ONM2} when the forgetting factor $\delta = 1$. 

\begin{algorithm}
\caption{Estimation of Fourier components using online optimization approach}
\label{alg:DCT_optim_time_varying}
\begin{algorithmic}[1]
\State \textbf{Algorithm Parameters}:  Logistic function parameter $\eta > 0$, Levenberg-Marquardt parameter $\varsigma > 0$, forgetting factor $\delta \in (0,1]$
\State \textbf{Inputs}: Initial position index $\bm{I}_{\textbf{x},0}$
\State \textbf{Outputs}: Parameter estimates $\{ \hat{\bm{\beta}}_k \}$
\State Initialize $\tilde{H}_0(\hat{\bm{\beta}}_{0}) = \mathbf{0}$
\For{$k=0,1,2,\dots,$}
	\State Update estimates 
 \begin{align*}
\hat{\bm{\beta}}_{k+1} &= \hat{\bm{\beta}}_k - \left( H_k (\hat{\bm{\beta}}_k; \bm{I}_{\textbf{x},1:k}, z_{1:k}) \right)^{-1} G_k (\hat{\bm{\beta}}_k; \bm{I}_{\textbf{x},k}, z_k) \nonumber \\
G_k (\hat{\bm{\beta}}_k; \bm{I}_{\textbf{x},k}, z_k) & = \nabla g_k (\hat{\bm{\beta}}_k; \bm{I}_{\textbf{x},k}, z_k) \nonumber \\
\tilde{H}_k(\hat{\bm{\beta}}_{k}; \bm{I}_{\textbf{x},1:k}, z_{1:k}) & =  \delta \tilde{H}_{k-1}(\hat{\bm{\beta}}_{k-1}; \bm{I}_{\textbf{x},1:k-1}, z_{1:k-1}) + \nabla^2 g_k (\hat{\bm{\beta}}_k; \bm{I}_{\textbf{x},k}, z_k) \nonumber \\
H_k(\hat{\bm{\beta}}_{k}; \bm{I}_{\textbf{x},1:k}, z_{1:k}) & = \tilde{H}_k(\hat{\bm{\beta}}_{k}; \bm{I}_{\textbf{x},1:k}, z_{1:k})  + \varsigma I, \label{eqn:approx_ONM_time_varying}
\end{align*}
\,\,\,\,\,\,\, where $\nabla g_k (\bm{\cdot}; \bm{\cdot}, \bm{\cdot})$ and $ \nabla^2 g_k(\bm{\cdot}; \bm{\cdot}, \bm{\cdot})$ are computed using \eqref{eqn:per_stage_gradient}-\eqref{eqn:per_stage_Hessian}	
	
 \State Determine  $\bm{I}_{\textbf{x},k+1} =\texttt{ActiveSensing}(\bm{I}_{\textbf{x},k}, \hat{\bm{\beta}}_{k+1})$  using Algorithm \ref{alg:active_sensing_online_opt}
\EndFor
\end{algorithmic}
\end{algorithm} 

\subsection{Measurement Location Selection Using Active Sensing}
For choosing the positions in which to take measurements, an ``active sensing'' approach \cite{Kreucher_active_sensing,LaShengChen,RisticSkvortsovGunatilaka} can be used, which aims to cleverly choose the next position given information collected so far, in order to more quickly obtain a good estimate of the field. 

In the case of binary measurements, a method for choosing the next measurement location is proposed in \cite{LeongZamaniShames}, that tries to maximize the minimum eigenvalue of an ``expected Hessian'' term $H^+(\bm{I}_{\textbf{x}'})$ over candidate future position indices $\bm{I}_{\textbf{x}'}$. Formally, the problem is:
$$\bm{I}_{\textbf{x},k+1} = \textnormal{arg} \max\limits_{\bm{I}_{\textbf{x}'} \in \mathcal{I}_{k+1}} \lambda_{\min} (H^+(\bm{I}_{\textbf{x}'})),$$
where $\lambda_{\min} (H^+(\bm{I}_{\textbf{x}'}))$ is the minimum eigenvalue of  $H^+(\bm{I}_{\textbf{x}'})$, $\mathcal{I}_{k+1}$ is the set of possible future position indices\footnote{The set $ \mathcal{I}_{k+1}$ may, e.g., capture the set of reachable positions from the current state of the mobile sensor platform.}
and
\begin{equation}
\label{eqn:expected_Hessian_binary}
\begin{split}
 H^+(\bm{I}_{\textbf{x}'}) & \triangleq H_k(\bm{\hat{\beta}}_{k}; \bm{I}_{\textbf{x},1:k}, z_{1:k})  +  \frac{ \eta^2 \exp(\eta (\bm{\hat{\beta}}_{k+1}^T \mathbf{K}(\bm{I}_{\textbf{x}'}) -  \tau)) }{\big(1+\exp(\eta (\bm{\hat{\beta}}_{k+1}^T \mathbf{K}(\bm{I}_{\textbf{x}'}) -  \tau)) \big)^2}  \mathbf{K}(\bm{I}_{\textbf{x}'}) \mathbf{K}(\bm{I}_{\textbf{x}'})^T \mathbb{P}(z' = 0) \\
 & \quad +  \frac{ \eta^2 \exp(\eta (\bm{\hat{\beta}}_{k+1}^T \mathbf{K}(\bm{I}_{\textbf{x}'}) -  \tau)) }{\big(1+\exp(\eta (\bm{\hat{\beta}}_{k+1}^T \mathbf{K}(\bm{I}_{\textbf{x}'}) -  \tau)) \big)^2}  \mathbf{K}(\bm{I}_{\textbf{x}'}) \mathbf{K}(\bm{I}_{\textbf{x}'})^T \mathbb{P}(z' = 1) \\
 & = H_k  (\bm{\hat{\beta}}_{k}; \bm{I}_{\textbf{x},1:k}, z_{1:k}) +  \frac{ \eta^2 \exp(\eta (\bm{\hat{\beta}}_{k+1}^T \mathbf{K}(\bm{I}_{\textbf{x}'}) -  \tau)) }{\big(1+\exp(\eta (\bm{\hat{\beta}}_{k+1}^T \mathbf{K}(\bm{I}_{\textbf{x}'}) -  \tau)) \big)^2}  \mathbf{K}(\bm{I}_{\textbf{x}'}) \mathbf{K}(\bm{I}_{\textbf{x}'})^T, 
\end{split}
\end{equation}
The last line of \eqref{eqn:expected_Hessian_binary} holds since $ \mathbb{P}(z' = 0)  +  \mathbb{P}(z' = 1)  = 1$, irrespective of the distribution of the noise $n(\bm{\cdot},\bm{\cdot})$. 

If we attempt to generalize \eqref{eqn:expected_Hessian_binary} to multi-level measurements, we find that there will be terms $ \mathbb{P}(z' = 0), \mathbb{P}(z' = 1), \dots,  \mathbb{P}(z' = L-1)  $ which cannot all be cancelled, and we will need to specify a noise distribution in order to compute these terms. Since exact knowledge of the noise distribution is usually unavailable in practice, we will instead consider a slightly different objective to optimize, namely a ``predicted Hessian''
\begin{equation}
\label{eqn:predicted_Hessian}
\begin{split}
 \hat{H}(\bm{I}_{\textbf{x}'}) & \triangleq H_k(\bm{\hat{\beta}}_{k}; \bm{I}_{\textbf{x},1:k}, z_{1:k})  +  \nabla^2 g_{k+1}(\bm{\hat{\beta}}_{k+1}; \bm{I}_{\textbf{x}'}, \hat{z}')
\end{split}
\end{equation}
where 
$\hat{z}' \triangleq q\big(\bm{\hat{\beta}}_{k+1}^T \mathbf{K}(\bm{I}_{\textbf{x}'})\big)$ is the predicted future measurement, with the quantizer $q(\bm{\cdot})$ given by \eqref{eqn:quantizer}. Note that \eqref{eqn:predicted_Hessian} reduces to \eqref{eqn:expected_Hessian_binary} in the case of binary measurements. We then maximize the minimum eigenvalue of the predicted Hessian to determine the next measurement location target:
\begin{equation}
\label{prob:maxmin_eig}
\bm{I}_{\textbf{x}}^{\textnormal{target}} = \textnormal{arg} \max\limits_{\bm{I}_{\textbf{x}'} \in \mathcal{I}_{k+1}} \lambda_{\min} (\hat{H}(\bm{I}_{\textbf{x}'})).
\end{equation}

For the set of candidate future position indices  $ \mathcal{I}_{k+1}$, one possible choice could be positions distributed uniformly on a grid within the search region $\mathcal{S}$. 
Once a new location target $\bm{I}_{\textbf{x}}^{\textnormal{target}} $ has been determined, we head in that direction. We will collect measurements and update $\hat{\bm{\beta}}$ along the way, where we collect a new measurement after every $\rho_0$ in distance has been travelled until $ \bm{I}_{\textbf{x}}^{\textnormal{target}}$ is reached, at which time a new location target is determined. 
%$ \bm{I}_{\tilde{k}}^{\textnormal{target}}$ is determined. The index $\tilde{k}$ is the new iteration index, which may be larger than $k+2$ if intermediate measurements have been collected on the way to $ \bm{I}_{k+1}^{\textnormal{target}}$.
The procedure is summarized in Algorithm \ref{alg:active_sensing_online_opt},  where $\bm{I}_{\textnormal{closest}} (\mathbf{x})$ denotes the closest position index $(I_x, I_y)$ to $\mathbf{x} \in \mathcal{S}$.
The condition in line \ref{line:at_target} of Algorithm \ref{alg:active_sensing_online_opt} means the location target has been reached, so that a new location target is determined and a location index $\bm{I}_{\textbf{x},k+1}$ in the direction of the new target is returned. Some random exploration is also included in the algorithm, such that the new location target is random with probability $\varepsilon$, similar to $\varepsilon$-greedy algorithms used in reinforcement learning \cite{SuttonBarto}.  The condition in line \ref{line:within_range_target} means that the agent is within $\rho_0$ of the target, which will be reached at the next time step, while the condition in line \ref{line:outside_range_target} means  the agent will continue heading towards the target and collect measurements along the way. 

%An alternative method is used in \cite{LeongZamaniShames}, where a convex combination of the old and new directions is then determined, and a distance of $\rho_0$ is travelled when a new measurement is collected, $\hat{\bm{\beta}}$ is updated, and problem \eqref{prob:maxmin_eig} solved again. 

\begin{algorithm}[t]
\caption{Active sensing algorithm for online optimization approach: $\bm{I}_{\textbf{x},k+1} = \texttt{ActiveSensing}(\bm{I}_{\textbf{x},k}, \hat{\bm{\beta}}_{k+1})$}
\label{alg:active_sensing_online_opt}
\begin{algorithmic}[1]
\State \textbf{Algorithm Parameters}: Distance $\rho_0 \geq 0$, candidate position indices $\mathcal{I}_{k+1}$, search region $\mathcal{S}$, exploration probability $\varepsilon$
\State \textbf{Inputs}:    $\bm{I}_{\textbf{x},k}$, $\hat{\bm{\beta}}_{k+1}$
\State \textbf{Output}: Next position index $\bm{I}_{\textbf{x},k+1}$
\If{$k=0$}
    \State Initialize $\bm{I}_{\textbf{x}}^{\textnormal{target}}  = \bm{I}_{\textbf{x},0}$
\EndIf
\If{$\bm{I}_{\textbf{x},k} = \bm{I}_{\textbf{x}}^{\textnormal{target}}$} \label{line:at_target}
    \State With probability $\varepsilon$, set new $\bm{I}_{\textbf{x}}^{\textnormal{target}}$ to a random location index in $\{0, \dots, N_x - 1\} \times \{0, \dots, N_y-1\}$, otherwise compute new $\bm{I}_{\textbf{x}}^{\textnormal{target}} = \textnormal{arg} \max\limits_{\bm{I}_{\textbf{x}'} \in \mathcal{I}_{k+1}} \lambda_{\min} (\hat{H}(\bm{I}_{\textbf{x}'})),$ where $\hat{H}(\bm{I}_{\textbf{x}'})$ is given by \eqref{eqn:predicted_Hessian}  \label{line:random_exploration}
    \State Set $\mathbf{x}_{k+1} = \mathbf{x}_k + \rho_0 (\mathbf{x}^\textnormal{target} - \mathbf{x}_k)/||\mathbf{x}^{\textnormal{target}} - \mathbf{x}_k||$ and return $\bm{I}_{\textbf{x},k+1} = \bm{I}_{\textnormal{closest}}(\mathbf{x}_{k+1})$
\ElsIf{$||\mathbf{x}_k - \mathbf{x}^{\textnormal{target}}|| < \rho_0$} \label{line:within_range_target}
    \State Set $\mathbf{x}_{k+1} = \mathbf{x}^{\textnormal{target}}$ and return $\bm{I}_{\textbf{x},k+1} =\bm{I}_{\textbf{x}}^{\textnormal{target}}$
\Else \label{line:outside_range_target}
    \State Set $\mathbf{x}_{k+1} = \mathbf{x}_k + \rho_0 (\mathbf{x}^\textnormal{target} - \mathbf{x}_k)/||\mathbf{x}^{\textnormal{target}} - \mathbf{x}_k||$ and return $\bm{I}_{\textbf{x},k+1} = \bm{I}_{\textnormal{closest}}(\mathbf{x}_{k+1})$
\EndIf

\end{algorithmic}
\end{algorithm} 

\section{Numerical Studies}
\label{sec:numerical}
For performance evaluation of the field estimation algorithms, we will consider two performance measures, the mean squared error (MSE) and structural similarity index (SSIM). These are defined similar to Section \ref{sec:DCT_RBF_comparison}, except that we replace the approximated field with the estimated field 
$$\hat{\phi}_d (I_x, I_y)   \triangleq \sum_{(u,v) \in \tilde{\mathcal{U}} } \alpha_x(u) \alpha_y(v) \hat{C}(u,v) \cos \left(\frac{(2I_x+1)\pi u}{2 N_x} \right) \cos \left(\frac{(2I_y+1)\pi v}{2 N_y} \right).$$


\subsection{Static Fields}
We consider estimation of the (true) field shown in Fig. \ref{fig:true_field_seed355}, with search region $\mathcal{S} = [0,100] \times [0,100]$. The field is discretized using $N_x = 100$ and $N_y = 100$. We use \eqref{eqn:U_tilde_largest} to select the largest modes that we wish to retain and estimate. 
\begin{figure}[t!]
\centering 
\includegraphics[scale=0.6]{true_field_seed355.pdf} 
\caption{Static field}
\label{fig:true_field_seed355}
\end{figure} 

We use Algorithm \ref{alg:DCT_optim_time_varying} with $\eta=5$ and $\varsigma = 1/5000$. As the field is assumed static, the forgetting factor is set to $\delta = 1$. The initial position index is set to $\bm{I}_{\textbf{x},0} = (50,50)$, close to the center of the search region~$\mathcal{S}$. 
A four level quantizer is used with quantizer thresholds $\tau_0=1, \tau_1=2, \tau_2 = 3$. The measurement noise $n(\bm{\cdot},\bm{\cdot})$ is i.i.d. Gaussian with zero mean and variance equal to 0.1. For choosing the measurement locations, we use Algorithm \ref{alg:active_sensing_online_opt} with $\rho_0 = 10$. The candidate position indices $\mathcal{I}_{k+1}$ are chosen to correspond to 36 points placed uniformly on a grid within the search region $\mathcal{S}$. The exploration probability is chosen as $\varepsilon = 0.1$.

\begin{figure}[t!]
\centering 
\includegraphics[scale=0.6]{MSE_time_stepped_seed355.pdf} 
\caption{Static field: MSE vs. $k$}
\label{fig:MSE_time_stepped_seed355}
\end{figure} 

\begin{figure}[t!]
\centering 
\includegraphics[scale=0.6]{SSIM_time_stepped_seed355.pdf} 
\caption{Static field: SSIM vs. $k$}
\label{fig:SSIM_time_stepped_seed355}
\end{figure} 

Fig. \ref{fig:MSE_time_stepped_seed355}  shows the MSE vs $k$ (corresponding to the number of measurements collected), when various numbers of modes are estimated. Fig.  \ref{fig:SSIM_time_stepped_seed355} shows the SSIM vs $k$. Each point in Figs. \ref{fig:MSE_time_stepped_seed355} and \ref{fig:SSIM_time_stepped_seed355} is obtained by averaging over 10 runs. We see from the figures that there is a trade-off between the estimation quality, number of modes/parameters that need to be estimated, and number of measurements collected. If a lot of measurements can be collected, then estimating more modes will allow for a better estimate of the field.\footnote{For example, if multiple mobile agents can be utilized \cite{LeongZamani_SP} or one has a sensor network, then more measurements can be collected in a limited amount of time.} On the other hand, if fewer measurements are available, estimating fewer modes more accurately may give a better field estimate than estimating lots of modes inaccurately.  
In Fig. \ref{fig:estimated_field_seed355} we show a sample plot of the estimated field when 80 modes are estimated, after 2000 measurements have been collected. 
\begin{figure}[t!]
\centering 
\includegraphics[scale=0.6]{estimated_field_seed355.pdf} 
\caption{Static field: Estimated field using 2000 measurements}
\label{fig:estimated_field_seed355}
\end{figure} 

\subsection{Time-varying Fields}
We now consider an example with time-varying fields. Suppose the true field is the same of that of Fig. \ref{fig:field_seed341_DCT} for the first 1000 iterations, but then switches to the true field in Fig. \ref{fig:field_seed343_DCT} for the next 1000 iterations. We will use Algorithms \ref{alg:DCT_optim_time_varying} and \ref{alg:active_sensing_online_opt}  with forgetting factor $\delta = 0.995$, with other parameters the same as in the previous example. 

Figs. \ref{fig:MSE_time_varying_seed341_343} and \ref{fig:SSIM_time_varying_seed341_343} show respectively the MSE and SSIM vs. $k$, when the 60 largest modes are estimated. 
We see that after the field changes at $k=1000$ the accuracy of the field estimate drops, but Algorithm~\ref{alg:DCT_optim_time_varying} is able to recover  and estimate the new field as more measurements are collected. 

For comparison, the MSE and SSIM obtained using forgetting factor $\delta = 1$ are also shown. In this case, as there is no forgetting of old information, the field estimates will take much longer to adjust to the new field. 

\begin{figure}[t!]
\centering 
\includegraphics[scale=0.6]{MSE_time_varying_seed341_343.pdf} 
\caption{Time varying field: MSE vs. $k$}
\label{fig:MSE_time_varying_seed341_343}
\end{figure} 

\begin{figure}[t!]
\centering 
\includegraphics[scale=0.6]{SSIM_time_varying_seed341_343.pdf} 
\caption{Time varying field: SSIM vs. $k$}
\label{fig:SSIM_time_varying_seed341_343}
\end{figure} 

\section{Conclusion}
This paper has studied the estimation of scalar fields, where a field is viewed in the Fourier domain. An algorithm has been presented for estimating the lower order modes of the field under the assumption of noisy quantized measurements. Our approach assumed an agent or agents travelling around a region in order to collect measurements. Future work will consider the use of a sensor network for field estimation, with algorithms constrained by local communication and distributed computation. 


\section*{Acknowledgment}
The authors thank Mr. Shintaro Umeki for suggesting the use of structural similarity as a performance measure while working at DST Group as a summer vacation student. 

\bibliography{IEEEabrv,source_localization}
\bibliographystyle{IEEEtran} 




\end{document}
 

\begin{thebibliography}{}
\expandafter\ifx\csname natexlab\endcsname\relax\def\natexlab#1{#1}\fi
\providecommand{\url}[1]{\href{#1}{#1}}
\providecommand{\dodoi}[1]{doi:~\href{http://doi.org/#1}{\nolinkurl{#1}}}
\providecommand{\doeprint}[1]{\href{http://ascl.net/#1}{\nolinkurl{http://ascl.net/#1}}}
\providecommand{\doarXiv}[1]{\href{https://arxiv.org/abs/#1}{\nolinkurl{https://arxiv.org/abs/#1}}}

\bibitem[{{Beasley} {et~al.}(2002){Beasley}, {Hoyle}, \&
  {Sharples}}]{beetal2002}
{Beasley}, M.~A., {Hoyle}, F., \& {Sharples}, R.~M. 2002, \mnras, 336, 168,
  \dodoi{10.1046/j.1365-8711.2002.05714.x}

\bibitem[{{Bennet} {et~al.}(2022){Bennet}, {Alfaro-Cuello}, {del Pino},
  {Watkins}, {van der Marel}, \& {Sohn}}]{bennetetal2022}
{Bennet}, P., {Alfaro-Cuello}, M., {del Pino}, A., {et~al.} 2022, arXiv
  e-prints, arXiv:2207.13100.
\newblock \doarXiv{2207.13100}

\bibitem[{{Carrera} {et~al.}(2011){Carrera}, {Gallart}, {Aparicio}, \&
  {Hardy}}]{carreraetal2011}
{Carrera}, R., {Gallart}, C., {Aparicio}, A., \& {Hardy}, E. 2011, \aj, 142,
  61, \dodoi{10.1088/0004-6256/142/2/61}

\bibitem[{{Chabrier}(2003)}]{chabrier2003}
{Chabrier}, G. 2003, \pasp, 115, 763, \dodoi{10.1086/376392}

\bibitem[{{de los Reyes} {et~al.}(2022){de los Reyes}, {Kirby}, {Ji}, \&
  {Nu{\~n}ez}}]{delosreyesetal2022}
{de los Reyes}, M. A.~C., {Kirby}, E.~N., {Ji}, A.~P., \& {Nu{\~n}ez}, E.~H.
  2022, \apj, 925, 66, \dodoi{10.3847/1538-4357/ac332b}

\bibitem[{{Drlica-Wagner} {et~al.}(2020){Drlica-Wagner}, {Bechtol}, {Mau},
  {McNanna}, {Nadler}, {Pace}, {Li}, {Pieres}, {Rozo}, {Simon}, {Walker},
  {Wechsler}, {Abbott}, {Allam}, {Annis}, {Bertin}, {Brooks}, {Burke},
  {Rosell}, {Carrasco Kind}, {Carretero}, {Costanzi}, {da Costa}, {De Vicente},
  {Desai}, {Diehl}, {Doel}, {Eifler}, {Everett}, {Flaugher}, {Frieman},
  {Garc{\'\i}a-Bellido}, {Gaztanaga}, {Gruen}, {Gruendl}, {Gschwend},
  {Gutierrez}, {Honscheid}, {James}, {Krause}, {Kuehn}, {Kuropatkin}, {Lahav},
  {Maia}, {Marshall}, {Melchior}, {Menanteau}, {Miquel}, {Palmese}, {Plazas},
  {Sanchez}, {Scarpine}, {Schubnell}, {Serrano}, {Sevilla-Noarbe}, {Smith},
  {Suchyta}, {Tarle}, \& {DES Collaboration}}]{drlicawagneretal2020}
{Drlica-Wagner}, A., {Bechtol}, K., {Mau}, S., {et~al.} 2020, \apj, 893, 47,
  \dodoi{10.3847/1538-4357/ab7eb9}

\bibitem[{{Eadie} {et~al.}(2021){Eadie}, {Harris}, \&
  {Springford}}]{eadieetal2021}
{Eadie}, G.~M., {Harris}, W.~E., \& {Springford}, A. 2021, arXiv e-prints,
  arXiv:2110.15376.
\newblock \doarXiv{2110.15376}

\bibitem[{{Forbes}(2020)}]{forbes2020}
{Forbes}, D.~A. 2020, \mnras, 493, 847, \dodoi{10.1093/mnras/staa245}

\bibitem[{{Gaia Collaboration} {et~al.}(2020){Gaia Collaboration}, {Luri},
  {Chemin}, {Clementini}, {Delgado}, {McMillan}, {Romero-G{\'o}mez},
  {Balbinot}, {Castro-Ginard}, {Mor}, {Ripepi}, {Sarro}, {Cioni}, {Fabricius},
  {Garofalo}, {Helmi}, {Muraveva}, {Brown}, {Vallenari}, {Prusti}, {de},
  {Babusiaux}, {Biermann}, {Creevey}, {Evans}, {Eyer}, {Hutton}, {Jansen},
  {Jordi}, {Klioner}, {Lammers}, {Lindegren}, {Mignard}, {Panem}, {Pourbaix},
  {Randich}, {Sartoretti}, {Soubiran}, {Walton}, {Arenou}, {Bailer-Jones},
  {Bastian}, {Cropper}, {Drimmel}, {Katz}, {Lattanzi}, {van Leeuwen}, {Bakker},
  {Casta{\~n}eda}, {De}, {Ducourant}, {Fouesneau}, {Fr{\'e}mat}, {Guerra},
  {Guerrier}, {Guiraud}, {Jean-Antoine}, {Masana}, {Messineo}, {Mowlavi},
  {Nicolas}, {Nienartowicz}, {Pailler}, {Panuzzo}, {Riclet}, {Roux},
  {Seabroke}, {Sordo}, {Tanga}, {Th{\'e}venin}, {Gracia-Abril}, {Portell},
  {Teyssier}, {Altmann}, {Andrae}, {Bellas-Velidis}, {Benson}, {Berthier},
  {Blomme}, {Brugaletta}, {Burgess}, {Busso}, {Carry}, {Cellino}, {Cheek},
  {Damerdji}, {Davidson}, {Delchambre}, {Dell'Oro},
  {Fern{\'a}ndez-Hern{\'a}ndez}, {Galluccio}, {Garc{\'\i}a-Lario},
  {Garcia-Reinaldos}, {Gonz{\'a}lez-N{\'u}{\~n}ez}, {Gosset}, {Haigron},
  {Halbwachs}, {Hambly}, {Harrison}, {Hatzidimitriou}, {Heiter},
  {Hern{\'a}ndez}, {Hestroffer}, {Hodgkin}, {Holl}, {Jan{\ss}en}, {Jevardat de
  Fombelle}, {Jordan}, {Krone-Martins}, {Lanzafame}, {L{\"o}ffler}, {Lorca},
  {Manteiga}, {Marchal}, {Marrese}, {Moitinho}, {Mora}, {Muinonen}, {Osborne},
  {Pancino}, {Pauwels}, {Recio-Blanco}, {Richards}, {Riello}, {Rimoldini},
  {Robin}, {Roegiers}, {Rybizki}, {Siopis}, {Smith}, {Sozzetti}, {Ulla},
  {Utrilla}, {van Leeuwen}, {van Reeven}, {Abbas}, {Abreu}, {Accart}, {Aerts},
  {Aguado}, {Ajaj}, {Altavilla}, {{\'A}lvarez}, {{\'A}lvarez Cid-Fuentes},
  {Alves}, {Anderson}, {Anglada Varela}, {Antoja}, {Audard}, {Baines}, {Baker},
  {Balaguer-N{\'u}{\~n}ez}, {Balog}, {Barache}, {Barbato}, {Barros}, {Barstow},
  {Bartolom{\'e}}, {Bassilana}, {Bauchet}, {Baudesson-Stella}, {Becciani},
  {Bellazzini}, {Bernet}, {Bertone}, {Bianchi}, {Blanco-Cuaresma}, {Boch},
  {Bombrun}, {Bossini}, {Bouquillon}, {Bragaglia}, {Bramante}, {Breedt},
  {Bressan}, {Brouillet}, {Bucciarelli}, {Burlacu}, {Busonero}, {Butkevich},
  {Buzzi}, {Caffau}, {Cancelliere}, {C{\'a}novas}, {Cantat-Gaudin}, {Carballo},
  {Carlucci}, {Carnerero}, {Carrasco}, {Casamiquela}, {Castellani}, {Castro
  Sampol}, {Chaoul}, {Charlot}, {Chiavassa}, {Comoretto}, {Cooper}, {Cornez},
  {Cowell}, {Crifo}, {Crosta}, {Crowley}, {Dafonte}, {Dapergolas}, {David},
  {David}, {de Laverny}, {De Luise}, {De March}, {De Ridder}, {de Souza}, {de
  Teodoro}, {de Torres}, {del Peloso}, {del Pozo}, {Delgado}, {Delisle}, {Di
  Matteo}, {Diakite}, {Diener}, {Distefano}, {Dolding}, {Eappachen}, {Enke},
  {Esquej}, {Fabre}, {Fabrizio}, {Faigler}, {Fedorets}, {Fernique}, {Fienga},
  {Figueras}, {Fouron}, {Fragkoudi}, {Fraile}, {Franke}, {Gai}, {Garabato},
  {Garcia-Gutierrez}, {Garc{\'\i}a-Torres}, {Gavras}, {Gerlach}, {Geyer},
  {Giacobbe}, {Gilmore}, {Girona}, {Giuffrida}, {Gomez}, {Gonzalez-Santamaria},
  {Gonz{\'a}lez-Vidal}, {Granvik}, {Guti{\'e}rrez-S{\'a}nchez}, {Guy},
  {Hauser}, {Haywood}, {Hidalgo}, {Hilger}, {H{\l}adczuk}, {Hobbs}, {Holland},
  {Huckle}, {Jasniewicz}, {Jonker}, {Juaristi}, {Julbe}, {Karbevska},
  {Kervella}, {Khanna}, {Kochoska}, {Kontizas}, {Kordopatis}, {Korn},
  {Kostrzewa-Rutkowska}, {Kruszy{\'n}ska}, {Lambert}, {Lanza}, {Lasne}, {Le
  Campion}, {Le Fustec}, {Lebreton}, {Lebzelter}, {Leccia}, {Leclerc},
  {Lecoeur-Taibi}, {Liao}, {Licata}, {Lindstr{\o}m}, {Lister}, {Livanou},
  {Lobel}, {Madrero}, {Managau}, {Mann}, {Marchant}, {Marconi}, {Marcos},
  {Marinoni}, {Marocco}, {Marshall}, {Polo}, {Mart{\'\i}n-Fleitas}, {Masip},
  {Massari}, {Mastrobuono-Battisti}, {Mazeh}, {Messina}, {Michalik}, {Millar},
  {Mints}, {Molina}, {Molinaro}, {Moln{\'a}r}, {Montegriffo}, {Morbidelli},
  {Morel}, {Morris}, {Mulone}, {Munoz}, {Murphy}, {Musella}, {Noval},
  {Ord{\'e}novic}, {Orr{\`u}}, {Osinde}, {Pagani}, {Pagano}, {Palaversa},
  {Palicio}, {Panahi}, {Pawlak}, {Pe{\~n}alosa}, {Penttil{\"a}}, {Piersimoni},
  {Pineau}, {Plachy}, {Plum}, {Poggio}, {Poretti}, {Poujoulet}, {Pr{\v{s}}a},
  {Pulone}, {Racero}, {Ragaini}, {Rainer}, {Raiteri}, {Rambaux}, {Ramos},
  {Ramos-Lerate}, {Re}, {Regibo}, {Reyl{\'e}}, {Riva}, {Rixon}, {Robichon},
  {Robin}, {Roelens}, {Rohrbasser}, {Rowell}, {Royer}, {Rybicki}, {Sadowski},
  {Sagrist{\`a}}, {Sahlmann}, {Salgado}, {Salguero}, {Samaras}, {Sanchez},
  {Sanna}, {Santove{\~n}a}, {Sarasso}, {Schultheis}, {Sciacca}, {Segol},
  {Segovia}, {S{\'e}gransan}, {Semeux}, {Siddiqui}, {Siebert}, {Siltala},
  {Slezak}, {Smart}, {Solano}, {Solitro}, {Souami}, {Souchay}, {Spagna},
  {Spoto}, {Steele}, {Steidelm{\"u}ller}, {Stephenson}, {S{\"u}veges},
  {Szabados}, {Szegedi-Elek}, {Taris}, {Tauran}, {Taylor}, {Teixeira},
  {Thuillot}, {Tonello}, {Torra}, {Torra}, {Turon}, {Unger}, {Vaillant}, {van},
  {Vanel}, {Vecchiato}, {Viala}, {Vicente}, {Voutsinas}, {Weiler}, {Wevers},
  {Wyrzykowski}, {Yoldas}, {Yvard}, {Zhao}, {Zorec}, {Zucker}, {Zurbach}, \&
  {Zwitter}}]{gaiaetal2020b}
{Gaia Collaboration}, {Luri}, X., {Chemin}, L., {et~al.} 2020, arXiv e-prints,
  arXiv:2012.01771.
\newblock \doarXiv{2012.01771}

\bibitem[{{Georgiev} {et~al.}(2010){Georgiev}, {Puzia}, {Goudfrooij}, \&
  {Hilker}}]{georgievetal2010}
{Georgiev}, I.~Y., {Puzia}, T.~H., {Goudfrooij}, P., \& {Hilker}, M. 2010,
  \mnras, 406, 1967, \dodoi{10.1111/j.1365-2966.2010.16802.x}

\bibitem[{{Hasselquist} {et~al.}(2021){Hasselquist}, {Hayes}, {Lian},
  {Weinberg}, {Zasowski}, {Horta}, {Beaton}, {Feuillet}, {Garro}, {Gallart},
  {Smith}, {Holtzman}, {Minniti}, {Lacerna}, {Shetrone}, {J{\"o}nsson},
  {Cioni}, {Fillingham}, {Cunha}, {O'Connell}, {Fern{\'a}ndez-Trincado},
  {Mu{\~n}oz}, {Schiavon}, {Almeida}, {Anguiano}, {Beers}, {Bizyaev},
  {Brownstein}, {Cohen}, {Frinchaboy}, {Garc{\'\i}a-Hern{\'a}ndez}, {Geisler},
  {Lane}, {Majewski}, {Nidever}, {Nitschelm}, {Povick}, {Price-Whelan},
  {Roman-Lopes}, {Rosado}, {Sobeck}, {Stringfellow}, {Valenzuela}, {Villanova},
  \& {Vincenzo}}]{hasselquistetal2021}
{Hasselquist}, S., {Hayes}, C.~R., {Lian}, J., {et~al.} 2021, \apj, 923, 172,
  \dodoi{10.3847/1538-4357/ac25f9}

\bibitem[{{Hirai} {et~al.}(2022){Hirai}, {Beers}, {Chiba}, {Aoki}, {Shank},
  {Saitoh}, {Okamoto}, \& {Makino}}]{hiraietal2022}
{Hirai}, Y., {Beers}, T.~C., {Chiba}, M., {et~al.} 2022, arXiv e-prints,
  arXiv:2206.04060.
\newblock \doarXiv{2206.04060}

\bibitem[{{Hirai} {et~al.}(2015){Hirai}, {Ishimaru}, {Saitoh}, {Fujii},
  {Hidaka}, \& {Kajino}}]{hiraietal2015}
{Hirai}, Y., {Ishimaru}, Y., {Saitoh}, T.~R., {et~al.} 2015, \apj, 814, 41,
  \dodoi{10.1088/0004-637X/814/1/41}

\bibitem[{{Hirai} {et~al.}(2017){Hirai}, {Ishimaru}, {Saitoh}, {Fujii},
  {Hidaka}, \& {Kajino}}]{hiraietal2017}
---. 2017, \mnras, 466, 2474, \dodoi{10.1093/mnras/stw3342}

\bibitem[{{Hirai} \& {Saitoh}(2017)}]{hiraisaitoh2017}
{Hirai}, Y., \& {Saitoh}, T.~R. 2017, \apjl, 838, L23,
  \dodoi{10.3847/2041-8213/aa6799}

\bibitem[{{Hirai} {et~al.}(2018){Hirai}, {Saitoh}, {Ishimaru}, \&
  {Wanajo}}]{hiraietal2018}
{Hirai}, Y., {Saitoh}, T.~R., {Ishimaru}, Y., \& {Wanajo}, S. 2018, \apj, 855,
  63, \dodoi{10.3847/1538-4357/aaaabc}

\bibitem[{{Iwamoto} {et~al.}(1999){Iwamoto}, {Brachwitz}, {Nomoto},
  {Kishimoto}, {Umeda}, {Hix}, \& {Thielemann}}]{iwamoto1999}
{Iwamoto}, K., {Brachwitz}, F., {Nomoto}, K., {et~al.} 1999, \apjs, 125, 439,
  \dodoi{10.1086/313278}

\bibitem[{{Kruijssen} {et~al.}(2019){Kruijssen}, {Pfeffer}, {Reina-Campos},
  {Crain}, \& {Bastian}}]{kruijssenetal2019}
{Kruijssen}, J.~M.~D., {Pfeffer}, J.~L., {Reina-Campos}, M., {Crain}, R.~A., \&
  {Bastian}, N. 2019, \mnras, 486, 3180, \dodoi{10.1093/mnras/sty1609}

\bibitem[{{Lach} {et~al.}(2020){Lach}, {R{\"o}pke}, {Seitenzahl}, {Cot{\'e}},
  {Gronow}, \& {Ruiter}}]{lachetal2020}
{Lach}, F., {R{\"o}pke}, F.~K., {Seitenzahl}, I.~R., {et~al.} 2020, \aap, 644,
  A118, \dodoi{10.1051/0004-6361/202038721}

\bibitem[{{Letarte} {et~al.}(2006){Letarte}, {Hill}, {Jablonka}, {Tolstoy},
  {Fran{\c{c}}ois}, \& {Meylan}}]{letarteetal2006}
{Letarte}, B., {Hill}, V., {Jablonka}, P., {et~al.} 2006, \aap, 453, 547,
  \dodoi{10.1051/0004-6361:20054439}

\bibitem[{{Mackey} \& {Gilmore}(2003)}]{mg2003}
{Mackey}, A.~D., \& {Gilmore}, G.~F. 2003, \mnras, 338, 85,
  \dodoi{10.1046/j.1365-8711.2003.06021.x}

\bibitem[{{Majewski} {et~al.}(2017){Majewski}, {Schiavon}, {Frinchaboy},
  {Allende Prieto}, {Barkhouser}, {Bizyaev}, {Blank}, {Brunner}, {Burton},
  {Carrera}, {Chojnowski}, {Cunha}, {Epstein}, {Fitzgerald}, {Garc{\'\i}a
  P{\'e}rez}, {Hearty}, {Henderson}, {Holtzman}, {Johnson}, {Lam}, {Lawler},
  {Maseman}, {M{\'e}sz{\'a}ros}, {Nelson}, {Nguyen}, {Nidever}, {Pinsonneault},
  {Shetrone}, {Smee}, {Smith}, {Stolberg}, {Skrutskie}, {Walker}, {Wilson},
  {Zasowski}, {Anders}, {Basu}, {Beland}, {Blanton}, {Bovy}, {Brownstein},
  {Carlberg}, {Chaplin}, {Chiappini}, {Eisenstein}, {Elsworth}, {Feuillet},
  {Fleming}, {Galbraith-Frew}, {Garc{\'\i}a}, {Garc{\'\i}a-Hern{\'a}ndez},
  {Gillespie}, {Girardi}, {Gunn}, {Hasselquist}, {Hayden}, {Hekker}, {Ivans},
  {Kinemuchi}, {Klaene}, {Mahadevan}, {Mathur}, {Mosser}, {Muna}, {Munn},
  {Nichol}, {O'Connell}, {Parejko}, {Robin}, {Rocha-Pinto}, {Schultheis},
  {Serenelli}, {Shane}, {Silva Aguirre}, {Sobeck}, {Thompson}, {Troup},
  {Weinberg}, \& {Zamora}}]{majewskietal2017}
{Majewski}, S.~R., {Schiavon}, R.~P., {Frinchaboy}, P.~M., {et~al.} 2017, \aj,
  154, 94, \dodoi{10.3847/1538-3881/aa784d}

\bibitem[{{Massana} {et~al.}(2022){Massana}, {Ruiz-Lara}, {No{\"e}l},
  {Gallart}, {Nidever}, {Choi}, {Sakowska}, {Besla}, {Olsen}, {Monelli},
  {Dorta}, {Stringfellow}, {Cassisi}, {Bernard}, {Zaritsky}, {Cioni},
  {Monachesi}, {van der Marel}, {de Boer}, \& {Walker}}]{massanaetal2022}
{Massana}, P., {Ruiz-Lara}, T., {No{\"e}l}, N.~E.~D., {et~al.} 2022, \mnras,
  513, L40, \dodoi{10.1093/mnrasl/slac030}

\bibitem[{{Massari} {et~al.}(2019){Massari}, {Koppelman}, \&
  {Helmi}}]{massarietal2019}
{Massari}, D., {Koppelman}, H.~H., \& {Helmi}, A. 2019, \aap, 630, L4,
  \dodoi{10.1051/0004-6361/201936135}

\bibitem[{{Mazzi} {et~al.}(2021){Mazzi}, {Girardi}, {Zaggia}, {Pastorelli},
  {Rubele}, {Bressan}, {Cioni}, {Clementini}, {Cusano}, {Rocha}, {Gullieuszik},
  {Kerber}, {Marigo}, {Ripepi}, {Bekki}, {Bell}, {de Grijs}, {Groenewegen},
  {Ivanov}, {Oliveira}, {Sun}, \& {van Loon}}]{mazzietal2021}
{Mazzi}, A., {Girardi}, L., {Zaggia}, S., {et~al.} 2021, \mnras, 508, 245,
  \dodoi{10.1093/mnras/stab2399}

\bibitem[{{Moore} {et~al.}(1999){Moore}, {Ghigna}, {Governato}, {Lake},
  {Quinn}, {Stadel}, \& {Tozzi}}]{mooreetal1999}
{Moore}, B., {Ghigna}, S., {Governato}, F., {et~al.} 1999, \apjl, 524, L19,
  \dodoi{10.1086/312287}

\bibitem[{{Mucciarelli} {et~al.}(2021){Mucciarelli}, {Massari}, {Minelli},
  {Romano}, {Bellazzini}, {Ferraro}, {Matteucci}, \&
  {Origlia}}]{mucciarellietal2021}
{Mucciarelli}, A., {Massari}, D., {Minelli}, A., {et~al.} 2021, Nature
  Astronomy, \dodoi{10.1038/s41550-021-01493-y}

\bibitem[{{Mucciarelli} {et~al.}(2010){Mucciarelli}, {Origlia}, \&
  {Ferraro}}]{mucciarellietal2010}
{Mucciarelli}, A., {Origlia}, L., \& {Ferraro}, F.~R. 2010, \apj, 717, 277,
  \dodoi{10.1088/0004-637X/717/1/277}

\bibitem[{{Nomoto} {et~al.}(2013){Nomoto}, {Kobayashi}, \&
  {Tominaga}}]{nomotoetal2013}
{Nomoto}, K., {Kobayashi}, C., \& {Tominaga}, N. 2013, \araa, 51, 457,
  \dodoi{10.1146/annurev-astro-082812-140956}

\bibitem[{{Piatti}(2019)}]{piatti2019}
{Piatti}, A.~E. 2019, \apj, 882, 98, \dodoi{10.3847/1538-4357/ab3574}

\bibitem[{{Piatti} {et~al.}(2019){Piatti}, {Alfaro}, \&
  {Cantat-Gaudin}}]{piattietal2019}
{Piatti}, A.~E., {Alfaro}, E.~J., \& {Cantat-Gaudin}, T. 2019, \mnras, 484,
  L19, \dodoi{10.1093/mnrasl/sly240}

\bibitem[{{Piatti} \& {Geisler}(2013)}]{pg13}
{Piatti}, A.~E., \& {Geisler}, D. 2013, \aj, 145, 17,
  \dodoi{10.1088/0004-6256/145/1/17}

\bibitem[{{Piatti} {et~al.}(2018){Piatti}, {Hwang}, {Cole}, {Angelo}, \&
  {Emptage}}]{piattietal2018c}
{Piatti}, A.~E., {Hwang}, N., {Cole}, A.~A., {Angelo}, M.~S., \& {Emptage}, B.
  2018, \mnras, 481, 49, \dodoi{10.1093/mnras/sty2324}

\bibitem[{{Piatti} \& {Mackey}(2018)}]{pm2018}
{Piatti}, A.~E., \& {Mackey}, A.~D. 2018, \mnras, 478, 2164,
  \dodoi{10.1093/mnras/sty1048}

\bibitem[{{Prantzos} {et~al.}(2018){Prantzos}, {Abia}, {Limongi}, {Chieffi}, \&
  {Cristallo}}]{prantzosetal2018}
{Prantzos}, N., {Abia}, C., {Limongi}, M., {Chieffi}, A., \& {Cristallo}, S.
  2018, \mnras, 476, 3432, \dodoi{10.1093/mnras/sty316}

\bibitem[{{Reichert} {et~al.}(2020){Reichert}, {Hansen}, {Hanke},
  {Sk{\'u}lad{\'o}ttir}, {Arcones}, \& {Grebel}}]{reichertetal2020}
{Reichert}, M., {Hansen}, C.~J., {Hanke}, M., {et~al.} 2020, \aap, 641, A127,
  \dodoi{10.1051/0004-6361/201936930}

\bibitem[{{Rostami Shirazi} {et~al.}(2022){Rostami Shirazi}, {Haghi}, {Khalaj},
  {Asl}, \& {Hasani Zonoozi}}]{rostamietal2022}
{Rostami Shirazi}, A., {Haghi}, H., {Khalaj}, P., {Asl}, A.~F., \& {Hasani
  Zonoozi}, A. 2022, \mnras, 513, 3526, \dodoi{10.1093/mnras/stac1070}

\bibitem[{{Seitenzahl} {et~al.}(2013){Seitenzahl}, {Ciaraldi-Schoolmann},
  {R{\"o}pke}, {Fink}, {Hillebrandt}, {Kromer}, {Pakmor}, {Ruiter}, {Sim}, \&
  {Taubenberger}}]{seitenzahl2013}
{Seitenzahl}, I.~R., {Ciaraldi-Schoolmann}, F., {R{\"o}pke}, F.~K., {et~al.}
  2013, \mnras, 429, 1156, \dodoi{10.1093/mnras/sts402}

\bibitem[{{Shao} {et~al.}(2021){Shao}, {Cautun}, {Frenk}, {Reina-Campos},
  {Deason}, {Crain}, {Kruijssen}, \& {Pfeffer}}]{shaoetal2021}
{Shao}, S., {Cautun}, M., {Frenk}, C.~S., {et~al.} 2021, \mnras, 507, 2339,
  \dodoi{10.1093/mnras/stab2285}

\bibitem[{{Strolger} {et~al.}(2020){Strolger}, {Rodney}, {Pacifici}, {Narayan},
  \& {Graur}}]{strolger2020}
{Strolger}, L.-G., {Rodney}, S.~A., {Pacifici}, C., {Narayan}, G., \& {Graur},
  O. 2020, \apj, 890, 140, \dodoi{10.3847/1538-4357/ab6a97}

\bibitem[{{Suntzeff} {et~al.}(1992){Suntzeff}, {Schommer}, {Olszewski}, \&
  {Walker}}]{setal92}
{Suntzeff}, N.~B., {Schommer}, R.~A., {Olszewski}, E.~W., \& {Walker}, A.~R.
  1992, \aj, 104, 1743, \dodoi{10.1086/116356}

\bibitem[{{Timmes} {et~al.}(1995){Timmes}, {Woosley}, \&
  {Weaver}}]{timmesetal1995}
{Timmes}, F.~X., {Woosley}, S.~E., \& {Weaver}, T.~A. 1995, \apjs, 98, 617,
  \dodoi{10.1086/192172}

\bibitem[{{Wagner-Kaiser} {et~al.}(2017){Wagner-Kaiser}, {Mackey},
  {Sarajedini}, {Chaboyer}, {Cohen}, {Yang}, {Cummings}, {Geisler}, \&
  {Grocholski}}]{wagnerkaiseretal2017}
{Wagner-Kaiser}, R., {Mackey}, D., {Sarajedini}, A., {et~al.} 2017, \mnras,
  471, 3347, \dodoi{10.1093/mnras/stx1702}

\bibitem[{{Wanajo} {et~al.}(2021){Wanajo}, {Hirai}, \&
  {Prantzos}}]{wanajoetal2021}
{Wanajo}, S., {Hirai}, Y., \& {Prantzos}, N. 2021, \mnras, 505, 5862,
  \dodoi{10.1093/mnras/stab1655}

\end{thebibliography}


\begin{appendix} 

\section{Python scripts used to perform different statistics}

In what follows y1 and y2 represent values of the
13 element abundances in NGC~2005 and in the five LMC
globular clusters, respectively (see text for details). \\



$\#$ Spearman correlation\\

spearman= stats.spearmanr(y1,y2, axis=0)

print (spearman) \\


$\#$ S{\o}rensen-Dice statistic\\

def dice(a, b):

    $\_$a = set(a)

    $\_$b = set(b)

    return (2*len($\_$a.intersection($\_$b))) / (len($\_$a) + len($\_$b))

dice = dice(y1, y2)

print (dice) \\



%chi = stats.chisquare(y1, y2, ddof=0, axis=0)

%print (chi) \\



$\#$ Jaccard similarity \\
%#from math import*

def jaccard$\_$similarity(x,y):

intersection$\_$cardinality = len(set.intersection(*[set(x), set(y)]))

union$\_$cardinality = len(set.union(*[set(x), set(y)]))

return intersection$\_$cardinality/float(union$\_$cardinality)

jaccard = jaccard$\_$similarity(y1,y2)

print (jaccard) \\



%$\#$ Euclidean distance\\

%def euclidean$\_$distance(x,y):

%  return np.sqrt(sum(pow(a-b,2) for a, b in zip(x, y)))

%euclidean=euclidean$\_$distance(y1,y2)

%print (euclidean) \\



$\#$ Cosine similarity\\

def square$\_$rooted(x):

   return round(np.sqrt(sum([a*a for a in x])),3)

  def cosine$\_$similarity(x,y):

 numerator = sum(a*b for a,b in zip(x,y))

 denominator = square$\_$rooted(x)*square$\_$rooted(y)

 return round(numerator/float(denominator),3)  

cosine = cosine$\_$similarity(y1,y2)

print (cosine) \\


%####t-test
%#tt = stats.ttest_ind(y1,y2)
%#print (tt)


$\#$ Pearson correlation\\

my$\_$rho = np.corrcoef(y1,y2)

print (my$\_$rho) \\


$\#$ Levenshtein distance\\

import Levenshtein as lev

ratio=lev.ratio(y1,y2)

jaro = lev.jaro(y1,y2)

jw = lev.jaro$\_$winkler(y1,y2)

ham = lev.hamming(y1,y2)

print (ratio, jaro, jw, ham)



\end{appendix}

\end{document}

% End of file `sample631.tex'.
