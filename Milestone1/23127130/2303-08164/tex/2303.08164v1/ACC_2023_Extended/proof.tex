\documentclass{article}
\usepackage[utf8]{inputenc}

\usepackage{graphicx}  

\usepackage{subcaption}
\usepackage{balance}

% \usepackage{stfloats}

\usepackage{amsmath,amsfonts,amssymb,amscd}
\usepackage[ruled,vlined]{algorithm2e} % For writing the algorithm
\usepackage{mathtools,cleveref}

\usepackage{graphicx}
\usepackage{color}
\usepackage{pdfsync}

\usepackage{tikz, pgfplots, circuitikz}
\usetikzlibrary{arrows,automata,positioning,shapes,intersections,calc}
%\usepgfplotslibrary{groupplots}
\pgfplotsset{compat=newest}

\newlength\figureheight
\newlength\figurewidth

\newtheorem{remark}{\bfseries Remark}
\newtheorem{theorem}{\bfseries Theorem}
\newtheorem{lemma}{\bfseries Lemma}
\newtheorem{assump}{\bfseries Assumption}
\newtheorem{corollary}{\bfseries Corollary}
\newtheorem{example}{\bfseries Example}
\newtheorem{ModelEx}{Example}
\newtheorem{defn}{\bfseries Definition}
\newtheorem{prop}{\bfseries Proposition}
\newtheorem{case}{Case}
%\newtheorem{corollary}{\bfseries Corollary}
\hyphenation{op-tical net-works semi-conduc-tor}
\DeclareMathOperator{\argmin}{argmin}


\newcommand{\qed}{\hfill \ensuremath{\blacksquare}}
\newcommand{\qedex}{\hfill \ensuremath{\square}}
\newcommand{\triend}{\hfill \ensuremath{\lhd}}
\newcommand{\bmatrixt}[1]{\begin{bmatrix}#1\end{bmatrix}^\mathrm{T}}
\newcommand{\norm}[1]{\left\lVert#1\right\rVert}

\usepackage{url,changebar,bm,xspace,dsfont}
\let\mathbb=\mathds % I much prefer the dsfont over the bbfont
\def\diag{\mathop{\mathrm{diag}}}  % For diagonal matrices
\def\det{\mathop{\mathrm{det}}}  % For determinants
\def\Co{\mathop{\mathrm{Co}}}  % For convex Hull
\def\d{\mathrm{d}} % for differentials

%\newcommand{\mrt}{{\it mRT}\xspace}
\newcommand{\hinf}{$\mathcal{H}_{\infty}$\xspace}
\DeclareMathOperator{\esssup}{ess\,sup}
\DeclareMathOperator{\Tr}{Tr}

\newcommand{\T}{^{\mbox{\tiny T}}}
\def\R{\mathbb{R}}
\def\Z{\mathbb{Z}}
\def \defin{\stackrel{\triangle}{=}}

\title{Proof}

\begin{document}
\maketitle


\begin{proof}
Define \begin{align*}
    f(\varphi) &\coloneqq \frac{1}{2} \text{E}_{\Xi_k} \left[\big \| v^*  - h_k\big(\varphi,\Xi_{k}\big) \big \|_2^2\right],\\
    g_{k-1} &\coloneqq \frac{\partial f(\varphi)}{\partial \varphi} \Bigg |_{ \varphi=q_{k-1}} = -S_{\varphi}^\top(v^*-S_{\varphi}q_{k-1}-\mu_{k-1}),
\end{align*}
$\delta_{k-1} \coloneqq (S_{\varphi}-\widehat{S}_{k-1})^\top \Delta  v^*_{k}$; then,  it follows from \eqref{updated_limits} -- \eqref{q_update_final_2} that
\begin{align}
q_{k} &= \Big[q_{k-1} + \gamma_k\widehat{S}_{k-1}^\top\Delta  v^*_{k}+\gamma_k\alpha_{k}\nu_{k}\Big]_{\underline{q}_{k} +\underline{\eta}_{k} }^{\overline{ q}_{k} -\overline{\eta}_{k}}\nonumber\\
    &=\Big[q_{k-1} + \gamma_k\big(S_{\varphi}+(\widehat{S}_{k-1}-S_{\varphi})\big)^\top\Delta  v^*_{k}\nonumber\\&\quad+\gamma_k\alpha_{k}\nu_{k}\Big]_{\underline{q}_{k} +\underline{\eta}_{k}  }^{\overline{ q}_{k} -\overline{\eta}_{k}}\nonumber\\
    &= \Big[q_{k-1} - \gamma_k(g_{k-1}+S_{\varphi}^\top S_w\xi_{k-1} +\delta_{k-1}-\alpha_{k}\nu_{k})\Big]_{\underline{q}_{k} +\underline{\eta}_{k} }^{\overline{ q}_{k} -\overline{\eta}_{k}}.
b\end{align} 
Define $\zeta_k \coloneqq S_{\varphi}^\top S_w\xi_{k-1} +\delta_{k-1}-\alpha_{k}\nu_{k}$; then, by   using the non-expansiveness property of the Euclidean projection operator, we have that
\begin{align}
    \|q_{k}-q^*\|^2&\leq \big\|q_{k-1}-q^* - \gamma_k\big(g_{k-1}+\zeta_{k-1}\big)\big\|^2\nonumber\\
    &= \|q_{k-1}-q^*\|^2 - 2\gamma_k\big(g_{k-1}+\zeta_{k-1}\big)^\top (q_{k-1} - q^*)\nonumber\\&\quad + \gamma_k^2\|g_{k-1}+\zeta_{k-1}\|^2.
    \label{main_ineq1}
\end{align}
It follows from the convexity property that
\begin{align}
    -g_{k-1}^\top (q_{k-1}-q^*) \leq f(q^*) - f(q_{k-1})\label{grad_ineq}.
\end{align}
By applying \eqref{grad_ineq} to \eqref{main_ineq1}, we obtain that
\begin{align}
    \|q_{k}-q^*\|^2 
    &\leq\|q_{k-1}-q^*\|^2 -2\gamma_k(f(q_{k-1}) - f(q^*))\nonumber\\&\quad-2\gamma_k\zeta_{k-1}^\top (q_{k-1} - q^*) + \gamma_k^2\|g_{k-1}+\zeta_{k-1}\|^2.
    \label{main_ineq2}
\end{align}
Define \[g^* \coloneqq \frac{\partial f(\varphi)}{\partial \varphi} \Bigg |_{ \varphi=q^*};\]
then, the following Lipschitz condition holds trivially for some $L>0$:
\begin{align}
    \|g_{k-1}-g^*\| = \|S_{\varphi}^\top S_{\varphi}(q_{k-1}-q^*)\| \leq L\|(q_{k-1}-q^*)\|.
    \label{Lipschitz}
\end{align}
By using \eqref{Lipschitz} and the fact that $2x^\top y\leq x^2+y^2$, for any $x,y\in\mathds{R}^n$, we have that
\begin{align}
    \|g_{k-1}+\zeta_{k-1}\|^2 &= \big\|g_{k-1}-g^*+g^*+\zeta_{k-1}\big\|^2\nonumber\\
    &\leq 2\|g_{k-1}-g^*\|^2+2\|g^*+\zeta_{k-1}\|^2\nonumber\\
    &\leq 2L^2\|q_{k-1}-q^*\|^2+2\|g^*+\zeta_{k-1}\|^2.
    \label{lip_ineq}
\end{align}
% Let $\mathcal{F}_k$ denote the accumulated collection of states, $\{(v_t,q_t)\}_{t=0}^{k}$.
Since $\text{E}\big[\delta_{k-1}\,|\,\mathcal{F}_{k-1}\big] = 0$, $\text{E}\big [S_{\varphi}^\top S_w\xi_{k-1} \,| \, \mathcal{F}_{k-1} \big ] = 0$, and $\text{E}\big [\alpha_{k}\nu_{k}\,|\,\mathcal{F}_k\big ] = 0$, we have that
\begin{align}
    \text{E}\big [\zeta_{k-1}\,|\,\mathcal{F}_{k-1}\big ] = \text{E}\big [S_{\varphi}^\top S_w\xi_{k-1} +\delta_{k-1}-\alpha_{k}\nu_{k}\,|\,\mathcal{F}_{k-1}\big ] = 0.\label{zeta_ineq}
\end{align}
Then, by taking an expectation of \eqref{main_ineq2} and applying \eqref{lip_ineq} and \eqref{zeta_ineq}, we obtain that
\begin{align}
    \text{E}\left[\|q_{k}-q^*\|^2\,|\,\mathcal{F}_{k-1}\right]
    &\leq(1+2L^2\gamma_k^2)\|q_{k-1}-q^*\|^2\nonumber\\
    &\quad -2\gamma_k(f(q_{k-1}) - f(q^*))\nonumber\\
    &\quad -2\gamma_k \text{E}[\zeta_{k-1}|\mathcal{F}_{k-1}]^\top (q_{k-1} - q^*)\nonumber\\
    &\quad+2\gamma_k^2 \text{E}\left[\|g^*+\zeta_{k-1}\|^2|\mathcal{F}_{k-1}\right])\nonumber\\
    &=(1+2L^2\gamma_k^2)\|q_{k-1}-q^*\|^2\nonumber\\
    &\quad  -2\gamma_k(f(q_{k-1}) - f(q^*))\nonumber\\&\quad+2\gamma_k^2 \text{E}\left[\|g^*+\zeta_{k-1}\|^2\,|\,\mathcal{F}_{k-1}\right]).
    \label{main_ineq3}
\end{align}
Further, it can be easily shown that
\begin{align}\text{E}\left[\|g^*+\zeta_{k-1}\|^2 \, |\, \mathcal{F}_{k-1}\right]<\infty. \label{noise_bound}
\end{align}
By applying the Robbins-Siegmund Theorem (see, e.g., \cite[Lemma~11]{Polyak}) to \eqref{main_ineq3}, we conclude that $q_k$ converges almost surely to some point in $\mathcal{X}_k^*$.
\end{proof}

The next result provides the cost error bound at the time instant when the sequence $\{\Delta q_k\}_{k \geq 1}$ is no longer persistently exciting. This indeed will be the case if the rLSE in \eqref{sensitivity_update_eqns} is used to generate the sequence $\{\widehat{S}_{k-1}\}_{k \geq 1}$ as $\widehat{S}_{k-1}$ will cease to be an unbiased estimate of the sensitivity matrix, $S_\varphi$, as $k \to \infty$.

\begin{prop}
Suppose we have for some $\alpha$ and $T>0$ that 
\begin{align}
\sum_{l=1}^T\lambda^{T-l}\Delta q_{l}\Delta q_{l}^\top < \alpha I.\label{excitation_loss}    
\end{align}
Suppose that $q_T>\underline{q}_{T}+\underline{\eta}_{T}$ and $q_T<\overline{q}_{T}-\overline{\eta}_{T}$.
Then, the following relation holds for any $q^*\in\mathcal{X}_T^*$:
\begin{align}
    f(q_T)-f(q^*) < (\overline{q}_T-\underline{q}_T)\frac{\sqrt{m\alpha}+B}{\gamma_T},\label{prop_result}
\end{align}
where $B$ is an upper bound for $\|\zeta_l\|$, $l\geq 1$.
\end{prop}
\begin{proof}
It follows from \eqref{excitation_loss} that $\alpha I - \sum_{l=1}^T\lambda^{T-l}\Delta q_{l}\Delta q_{l}^\top$ is a positive definite matrix. Hence, by \cite[Corollary~7.1.5]{Horn_Johnson}, its trace is positive, namely,
\begin{align}
    \Tr\left(\alpha I -  \sum_{l=1}^T\lambda^{T-l}\Delta q_{l}\Delta q_{l}^\top\right) > 0,\label{trace_ineq}
\end{align}
where $\Tr(A) = \sum_i A_{ii}$ denotes the trace of matrix $A$.
Then, by using the assumption that $q_T>\underline{q}_{T}+\underline{\eta}_{T}$ and $q_T<\overline{q}_{T}-\overline{\eta}_{T}$, and applying the triangle inequality, we have that
\begin{align}
    \sqrt{m\alpha} &> \sqrt{\Tr\left(\sum_{l=1}^T\lambda^{T-l}\Delta q_{l}\Delta q_{l}^\top\right)}\nonumber\\ 
    &\geq \|\gamma_T(g_T+\zeta_T)\|\geq \gamma_T(\|g_T\|-\|\zeta_T\|)\nonumber\\
    &\geq\gamma_T\|g_T\|-B,
\end{align}
Hence,
\begin{align}
    \|g_T\|< \frac{\sqrt{m\alpha}+B}{\gamma_T}.\label{grad_ineq2}
\end{align}
By using \eqref{grad_ineq} and the facts that $q_T \in [\underline{q}_T,\overline{q}_T]$, and $q^* \in [\underline{q}_T,\overline{q}_T]$, we obtain that
\begin{align}
    \|g_T\|(\overline{q}_T-\underline{q}_T)\geq f(q_T)-f(q^*).\label{grad_ineq3}
\end{align}
By combining \eqref{grad_ineq3} with \eqref{grad_ineq2}, we obtain that \eqref{prop_result} holds.
\end{proof}


\bibliographystyle{IEEEtran}

\bibliography{references}
\end{document}