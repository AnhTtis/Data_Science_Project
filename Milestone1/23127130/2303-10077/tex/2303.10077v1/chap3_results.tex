\chapter{Results}
\label{chap:results}

In this chapter, we present the results of our bulge-disk decomposition pipeline. This is mainly a catalogue of parameters for the (components of) galaxies in our sample along with a wealth of ancillary data and quality control metrics. We investigate some of these metrics here. A more detailed quality control of the catalogue is then presented in Chapter~\ref{chap:QC}, where we compare our work to previous works in the field and perform a detailed study of systematic uncertainties. Since the quality control focuses on \texttt{v04} of the \texttt{BDDecomp} DMU, we concentrate on this version here, too (following \citealt{Casura2022}). However, we expand on \texttt{v04} with the new results from \texttt{v05} that have not been presented previously, highlighting and discussing differences where relevant. This anchores the \texttt{v05} results to those of \texttt{v04}, thereby benefitting from the detailed quality control of that version presented in Chapter~\ref{chap:QC}. Results from DMU versions prior to \texttt{v04} are not presented since they were somewhat preliminary with limited quality control and were superseded entirely by \texttt{v04}. 


\section{\texttt{BDDecomp} DMU}
\label{sec:bddecompdmu}

Our main result is the \texttt{BDDecomp} DMU on the GAMA database. It contains two catalogues per band: \texttt{BDInputs} with the most important outputs of the preparatory work pipeline (segmentation, PSF estimation, initial guesses) and \texttt{BDModelsAll} with the output from the galaxy fitting and post-processing (model selection, flagging of bad fits and truncating to segment radii). From \texttt{v03} onwards, where we include several bands in the analysis, the catalogue names also specify the band to which they refer (e.g. \texttt{BDModelsAllR}). Up to \texttt{v04}, all images of the same object in the same band (data matches) are listed separately since they were also treated individually. For \texttt{v05}, they were fitted jointly so each galaxy only has one entry in the \texttt{BDModelsAll} table. The matches remain separate in the \texttt{BDInputs} tables since the preparatory work was still carried out on each image individually. 

In addition to these band-specific tables, \texttt{BDModels} gives the most important columns of the \texttt{BDModelsAll} table(s), combining results from all bands from \texttt{v03} onwards. It also has a few additional joint columns (mainly joint model selection). Finally, the table \texttt{BDModelsAlt} (added from \texttt{v03} onwards) presents the same information as \texttt{BDModels} just with the different bands arranged in rows instead of columns. This results in a total of three tables up to \texttt{v02} ($r$-band only), eight tables for \texttt{v03} and \texttt{v04} ($g$, $r$, $i$) and 20 tables for \texttt{v05} ($u, g, r, i, Z, Y, J, H, K_s$). 

Each table is accompanied by comprehensive documentation including descriptions of all columns, details on the processing steps and practical tips for using the catalogue. The DMU also provides all input data used for the fitting (i.e. image cutouts, masks, error maps, segmentation maps, sky estimates, PSFs) as well as various diagnostic plots of the fit results on the GAMA file server, where detailed descriptions of these files can be found.

Sections~\ref{sec:v01} to~\ref{sec:v06} summarise the key changes between the different DMU versions (also listed on the GAMA database) and give the basis for each DMU version, as labelled in the directory tree of our local machines. Furthermore, they provide the release date and the versions of \texttt{ProFit} and \texttt{ProFound} used. Individual aspects of this pipeline evolution are pointed out throughout Chapter~\ref{chap:pipeline} and we refer the reader to this chapter for details. 

In the remainder of this chapter we then present an overview over the contents of the main catalogue, \texttt{BDModels}, in \texttt{v04} and \texttt{v05}.

\subsection{\texttt{BDDecomp v01}}
\label{sec:v01}

\texttt{v01} is the first release of our bulge-disk decomposition catalogue, published on 2019-02-08. It is limited to the $r$-band only and is based on the preparatory work run labelled \texttt{"run3"} and the galaxy fitting \texttt{"run5"}. All previous runs as well as numerous test runs were not published on the GAMA database. \texttt{ProFit} and \texttt{ProFound} versions used were from 2018-04-24.

\subsection{\texttt{BDDecomp v02}}
\label{sec:v02}

\texttt{v02} is only a minor update of \texttt{v01} (published 2019-02-27) with two small mistakes fixed in the post-processing of the \texttt{BDModels} and \texttt{BDModelsAll} tables:
\begin{itemize}
\item The \texttt{*\_OUTLIER\_FLAG}s were fixed for 297 single component fits, 
   2211 double component fits and 3437 1.5-component fits.
\item The \texttt{P\_BDQUAL\_FLAG}s were fixed for all 1.5-component fits.
\end{itemize}
The preparatory work and galaxy fitting results did not change; and both are based on the same runs as \texttt{v01}. 

\subsection{\texttt{BDDecomp v03}}
\label{sec:v03}

\texttt{v03} is a major update which was released on 2019-12-03 and is based on preparatory work \texttt{"run4"} and fitting \texttt{"run6"}. We included the $g$ and $i$ bands in addition to the $r$-band, increasing the number of catalogues from three to eight (two per band plus two joint ones). This also resulted in a slight re-labelling of the catalogues themselves and their column names to distinguish all values between the three bands. At the same time, we upgraded to KiDS DR4.0 (which had become available in the meantime), including the photometric homogenisation given in the \texttt{DMAG} header keyword. We also updated \texttt{ProFit} and \texttt{ProFound} to the versions from 2019-08-19, which saw major changes since the \texttt{v01}/\texttt{v02} DMU release. All of the above, as well as our own investigations and feedback from both collaborators and users of the first catalogue versions resulted in numerous changes to the pipeline.

In the preparatory work pipeline (see the first half of Section~\ref{sec:pipelinedevelopment} for details):
\begin{itemize}
\item We turned off our own routine for segmentation map fixing since the new version of \texttt{ProFound} was much less prone to ``shredding" galaxies and our routine tended to include faint secondary objects in the segments.
\item We increased the \texttt{skycut} value in \texttt{profoundProFound} from 1 to 2 resulting in smaller segments with smoother borders.
\item We added one additional dilation of the galaxy segment after the \texttt{profoundProFound} run, which ensures that the edges are smooth and unbiased (without this, the segment border can be very jagged including noisy, slightly positive pixels but excluding slightly negative pixels; especially for large bright objects where the number of curve of growth iterations is often zero or one). The dilated segments are then approximately the same size as they were in the previous run due to the higher \texttt{skycut} value.
\item We defined the segmentation maps on stacked images of the $g, r, i$ bands (with corresponding stacked masks) and used these segmentation maps in all bands. 
\item We defined and used two different segmentation maps for the sky estimation and object fitting; the one used for sky estimation is more dilated to exclude faint sources and extended wings from the sky (aggressive object masking and lower \texttt{skycut}). 
\item The star fraction cut to identify candidate stars and the chi-squared cut to exclude bad fits from PSF modelling were adjusted slightly to account for the new segments.
\item We stopped using the \texttt{profoundSkySplitFFT} routine and instead take the sky estimate from \texttt{profoundProFound} directly, as we found that to be more robust. 
\end{itemize}

In the galaxy fitting and post-processing (see the second half of Section~\ref{sec:pipelinedevelopment} for details):
\begin{itemize}
\item We increased the limits for the S\'ersic index fitting to 0.1 to 20 (previously 1 to 12) allowing for flatter bulges and causing fewer fits hitting the limits.
\item We improved the swapping criteria slightly (both for the selection of fits to enter the swapping procedure and for selecting the better of two swapped fits) based on new visual inspections.
\item We updated the outlier rejection criteria to reflect the new S\'ersic index limit; and we included a new criterion based on the difference between the magnitude within the segment and within the segment radius (also based on new visual inspections). 
\item We re-calibrated the model selection for each band (new visual inspections) and additionally provided a joint model selection that is based on the summed DICs for all bands. 
\item We added several alternative measurements of the magnitudes, effective radii and bulge-to-total (B/T) ratios for each fit, resulting in the catalogues having more columns: 
\begin{itemize}
\item The S\'ersic magnitude and effective radius.
\item The magnitude contained within the segment and the corresponding effective radius (containing half of the segment flux).
\item The magnitude and effective radius within the ``segment radius", which is defined as the maximum distance between the centre of the fit and the edge of the segment. This is the upper limit to which our model fits are valid, everything beyond that is extrapolated and unconstrained. We recommend using these values for magnitude, effective radius and bulge-to-total flux ratios because many of our fits have high S\'ersic indices with unphysically large wings beyond the segment borders. This is because we opted for relatively tight segments focusing on fitting the central regions well rather than forcing the models to zero at large radii (Section~\ref{sec:postprocessing}). For reproducibility and ease of comparison to other catalogues, the segment radius (\texttt{RAD\_SEG}) is provided in a separate column as well. 
\item The magnitude and effective radius within 10 S\'ersic effective radii. Note that these values may still include unphysical results (see above) and were provided for completeness only. 
\end{itemize}
\end{itemize}

\subsection{\texttt{BDDecomp v04}}
\label{sec:v04}

\texttt{v04} is again a relatively minor update with no changes to the statistical properties of the galaxy parameters. It is based on preparatory work \texttt{"run5"} and fitting \texttt{"run7"}, used \texttt{ProFit} and \texttt{ProFound} versions from 2019-08-19 and was released on 2021-06-25. Its major effort was in better characterising the systematic uncertainties. This is the version that \citet{Casura2022} is based on and is the currently newest version on the GAMA database. 

In detail:
\begin{itemize}
\item Parameter errors labelled \texttt{*\_ERR} now include our best estimate of systematic uncertainties. The purely random MCMC errors have been re-labelled to \texttt{*\_ERR\_MCMC}. The \texttt{*\_ERR} values are obtained from the \texttt{*\_ERR\_MCMC} values by multiplying with the ``error underestimate" factors for each parameter listed in table 4 of \citet{Casura2022} and reproduced here in Table~\ref{tab:errorunderestimate}. We do not apply the bias corrections listed in the same table since they are only applicable to the averages of large statistical samples, not individual galaxies. Since the error understimates are slightly different for \texttt{*\_SEGRAD} values, these values now also have associated uncertainties (\texttt{*\_SEGRAD\_ERR}). Note the error underestimates are derived from single component $r$-band fits. We apply them to the $g$ and $i$ bands and bulge and disk parameters for 1.5 and double component fits as well, since that should result in more realistic errors than the random ones alone. However, it is likely that the true errors are even larger for component parameters (in particular for the non-dominant component) and/or the $g$ and $i$ bands (also for single S\'ersic fits, since the $r$-band is the deepest and best-resolved).
\item B/T values now also have errors (*\_BT\_ERR) given for convenience. They are derived from the bulge and disk magnitude errors assuming independent variables (which is not actually true) and hence should be taken as lower limits.
\item We fixed a small bug in the preparatory work pipeline where in \texttt{v03} we accidentally performed the KiDS zeropoint homogenisation correction (with the \texttt{DMAG} header keyword) twice, resulting in initial guesses for magnitudes being biased by about 0.03 mag on average. Since the MCMC fits are not strongly dependent on the initial guesses, this did not affect the fits (the zeropoint was correct during the actual fitting). Nonetheless, we re-ran the entire preparatory work and fitting pipelines. 
\item Position angles are now all in the interval 0 to 180\degr\ (which they were already supposed to be in \texttt{v03}, but were not in all cases). 
\item Galaxies are now also flagged as outliers if the single S\'ersic fit failed (even if the 1.5- or double component fits were successful) because then the segment radius (and thus all \texttt{*\_SEGRAD} columns) are missing; and selecting good single S\'ersic fits with \texttt{NCOMP}\,>\,0 fails. This affects only one fit in the $g$-band and none in the other bands. 
\item For consistency we renamed the columns \texttt{*\_FLUX\_ERR\_SEG} to \texttt{*\_FLUX\_SEG\_ERR} and \texttt{*\_RAD\_SEG} to \texttt{*\_SEGRAD}.
\item We added a quick-start guide to the DMU description on the GAMA database to cover the most important aspects in catalogue usage. 
\end{itemize}

\subsection{\texttt{BDDecomp v05}}
\label{sec:v05}

\texttt{v05} is another major update, to be released on the GAMA database alongside this thesis. It uses preparatory work \texttt{"run6"} and fitting \texttt{"run8"} with \texttt{ProFit} and \texttt{ProFound} versions last updated on 2022-02-11. In addition, we moved from Ubuntu 16.04 to Ubuntu 20.04 on all of our local machines, which necessitated updating \texttt{R} along with all of its packages and replacing the \texttt{astro} package by the (relatively new) \texttt{Rfits} package for writing FITS files. On the scientific side, the major upgrades are that we now include the KiDS $u$ and VIKING $Z, Y, J, H$ and $K_s$ bands and do multi-frame fitting. The corresponding changes are described in detail in Section~\ref{sec:pipelineupdates}. 

In summary:
\begin{itemize}
\item We now process all nine KiDS and VIKING bands, using the science tiles for KiDS but individual detector chips preprocessed by \citet{Wright2019} for VIKING. This increases the number of catalogues from eight to 20 (two per band plus two joint ones). 
\item We now do multi-frame fitting, i.e. when there is more than one data match to the same galaxy in the same band, they are fitted jointly rather than separately (assuming the astrometric solutions to be the ground truth). For the preparatory work, the matches are still treated individually and also listed individually in the corresponding \texttt{BDInputs} tables; but for the \texttt{BDModels} tables, there is now only one entry per galaxy. This means that \texttt{CATAID} is a unique identifier again (along with the \texttt{OBJECTID}s which we keep for compatibility with earlier versions). 
\item To reflect these changes, the \texttt{BDQUAL\_FLAG}s and \texttt{CUTOUT\_FLAG}s have changed in meaning slightly, the \texttt{BEST\_IMG} column (identifying the best image to use in the case of multiple matches) has been replaced with the \texttt{N\_MATCH} and \texttt{N\_FIT} columns (giving the number of data matches and the number of images actually used for fitting). The pixel scale (which varies for VIKING) has been added in the \texttt{PIXSC} column to give meaning to all columns with units of pixels. \texttt{JOINT\_*} columns have been renamed into \texttt{GRI\_*} (or whichever bands exactly they refer to, to differentiate between the different versions of joint model selection). 
\item We define the KiDS $g, r$ and $i$ bands as our core bands and transfer the segmentation maps defined on a stack of those core bands to all other bands. 
\item The galaxy segment sizes are generally larger than in \texttt{v04} due to a default change in \texttt{profoundProFound}. We use the new default of \texttt{SBdilate}\,=\,2, but discarded the \texttt{skycut} default change. 
\item We re-calibrated the model selection with new visual inspections in all bands. 
\end{itemize}

\subsection{\texttt{BDDecomp v06}}
\label{sec:v06}

We envision future versions of the DMU to provide simultaneous multi-band fits for all KiDS and VIKING observations of our sample.




\section{Catalogue statistics}
\label{sec:statistics}

We begin the presentation of the contents of the \texttt{BDModels} catalogue with an overview of the numbers of galaxies classified in each model selection category, including outliers and skipped fits. For this, Section~\ref{sec:statsoverview} presents two large tables (Table~\ref{tab:results} for \texttt{v04} and~\ref{tab:resultsv05} for \texttt{v05}) detailing the evolution of the numbers of galaxies at each step in our pipeline, with the most important information condensed in Figure~\ref{fig:ncompstats}. The \texttt{v04} versions of this are heavily based on \citet{Casura2022}. In Sections~\ref{sec:outlierstats} and~\ref{sec:modelseldiffs} we then investigate the outlier statistics and model selection statistics in more detail, with a particular focus on comparing \texttt{v04} and \texttt{v05}. 

\subsection{Statistics overview}
\label{sec:statsoverview}

Table~\ref{tab:results} gives an overview of the fit and post-processing results for \texttt{v04}. Table~\ref{tab:resultsv05} presents the analogous table for \texttt{v05}. Starting with our full sample (13096 galaxies from the combination of our main and SAMI samples, see Section~\ref{sec:sampleselection}), we show how the number of galaxies evolves through all steps of the pipeline in both versions. The results are split per-band and per-model where necessary. At some steps, we also include percentages of galaxies lost or remaining (grey font). In short, for \texttt{v04} we lose nearly 20\,\% of our sample to masking and a further almost 10\,\% to the flagging of bad fits; where the former is a random subset while the latter preferentially affects certain types of galaxies (e.g. mergers and irregulars). These fractions are similar for \texttt{v05} in the core bands, but increase for the KiDS $u$ and the VIKING bands - especially the longer wavelength ones - due to missing data (VIKING only), additional masking, a higher fraction of PSF failures ($u$-band only) and a larger fraction of outliers mainly caused by the shallower data. The former three effects are explained in Section~\ref{sec:jointfitting}, while the latter one is investigated in more detail in Section~\ref{sec:outlierstats}. For \texttt{v05}, the numbers of fit failures also increases significantly in all bands (including the core bands), but this does not affect the final statistics since preferentially those fits failed that would not have been selected in the model selection anyway (Section~\ref{sec:modelseldiffs}). 
 
\afterpage{%
\clearpage%
\begin{landscape}
\begin{table*}
	\centering
	\caption{Fit results for \texttt{v04} of the \texttt{BDDecomp} DMU: numbers (black) and percentages (grey) of galaxies remaining or lost at each step in our pipeline, split per-band and per-model where necessary.$^1$ Individual steps are numbered through, see text for details.}
	\label{tab:results}
	\setlength{\tabcolsep}{4pt} % 6pt seems to be the default
	\begin{tabu}{l rrr c rrr c rrr c rrr} % number of columns, alignment for each
	    \hline
	    band
	    & \multicolumn{3}{c}{$g$}
	    && \multicolumn{3}{c}{$r$}
	    && \multicolumn{3}{c}{$i$}
	    && \multicolumn{3}{c}{joint $gri$}
	    \\
	    model (components)
	    & \multicolumn{1}{c}{1} & \multicolumn{1}{c}{1.5} & \multicolumn{1}{c}{2} 
	    && \multicolumn{1}{c}{1} & \multicolumn{1}{c}{1.5} & \multicolumn{1}{c}{2} 
	    && \multicolumn{1}{c}{1} & \multicolumn{1}{c}{1.5} & \multicolumn{1}{c}{2} 
	    && \multicolumn{1}{c}{1} & \multicolumn{1}{c}{1.5} & \multicolumn{1}{c}{2} 
	    \\
	    \hline
	    number of: 
	    \\
	    1) unique objects (galaxies)
	    & \multicolumn{15}{c}{13096}
	    \\
	    2) images (independent fits)
	    & \multicolumn{15}{c}{14966}
	    \\
	    3) images not masked
	    & \multicolumn{15}{c}{11989}
	    \\
	    \rowfont{\color{gray}}
	    lost due to masking (\%)
	    & \multicolumn{15}{c}{20}
	    \\
	    \\	    
	    4) successful PSFs
	    & \multicolumn{3}{c}{11838}
	    && \multicolumn{3}{c}{11872}
	    && \multicolumn{3}{c}{11946}
	    && \multicolumn{3}{c}{11683}
	    \\
	    \rowfont{\color{gray}}
	    lost due to PSF fails (\%)
	    & \multicolumn{3}{c}{1}
	    && \multicolumn{3}{c}{0.8}
	    && \multicolumn{3}{c}{0.3}
	    && \multicolumn{3}{c}{2}
	    \\
	    \\
		5) successful fits 
		& 11837 & 11837 & 11831 
		&& 11872 & 11870 & 11861 
		&& 11946 & 11943 & 11945
		&& 11682 & 11678 & 11665
		\\		
		\rowfont{\color{gray}}
		lost due to fit fails (\%)
		& <\,0.01 & <\,0.01 & 0.05 
		&& 0 & 0.01 & 0.07 
		&& 0 & 0.02 & <\,0.01
		&& <\,0.01 & 0.03 & 0.12
		\\		
		6) fits not flagged 
		& 10951 & 7122 & 8022 
		&& 11025 & 8164 & 8759 
		&& 11086 & 7620 & 7775
		&& 10680 & 6446 & 5870
		\\		
		\rowfont{\color{gray}} 
	    not flagged/successful (\%) 
		& 93 & 60 & 68 
		&& 93 & 69 & 74 
		&& 93 & 64 & 65 
		&& 91 & 55 & 50
		\\		
		7) selected fits 
		& 8294 & 740 & 1743 
		&& 7061 & 585 & 2935 
		&& 7411 & 662 & 2663 
		&& 7308 & 621 & 2009
		\\		
		\rowfont{\color{gray}} 
		selected/successful (\%)
		& 70 & 6 & 15 
		&& 59 & 5 & 25 
		&& 62 & 6 & 22 
		&& 63 & 5 & 17
		\\		
		\\
		total number (per band) of:\\
		8a) good\,|\,flagged\,|\,skipped fits
		& \multicolumn{3}{c}{10777\,|\,1061\,|\,3128} 
		&& \multicolumn{3}{c}{10581\,|\,1291\,|\,3094} 
		&& \multicolumn{3}{c}{10736\,|\,1210\,|\,3020} 
		&& \multicolumn{3}{c}{9938\,|\,1745\,|\,3283} 
		\\
		\rowfont{\color{gray}} 
		good\,|\,f.\,|\,s./all images (\%)
		& \multicolumn{3}{c}{72\,|\,7\,|\,21} 
		&& \multicolumn{3}{c}{71\,|\,9\,|\,21} 
		&& \multicolumn{3}{c}{72\,|\,8\,|\,20} 
		&& \multicolumn{3}{c}{66\,|\,12\,|\,22}
		\\
		\\
		8b) good\,|\,flagged\,|\,skipped gal.
		& \multicolumn{3}{c}{9722\,|\,935\,|\,2439}
		&& \multicolumn{3}{c}{9545\,|\,1145\,|\,2406} 
		&& \multicolumn{3}{c}{9687\,|\,1059\,|\,2350} 
		&& \multicolumn{3}{c}{8998\,|\,1559\,|\,2539} 
		\\
		\rowfont{\color{gray}} 
		good\,|\,f.\,|\,s./unique objects (\%) 
		& \multicolumn{3}{c}{74\,|\,7\,|\,19} 
		&& \multicolumn{3}{c}{73\,|\,9\,|\,18} 
		&& \multicolumn{3}{c}{74\,|\,8\,|\,18} 
		&& \multicolumn{3}{c}{69\,|\,12\,|\,19}
		\\
        \hline
        \multicolumn{16}{l}{$^1$Based on information given in the \texttt{*\_BDQUAL\_FLAG}, \texttt{*\_OUTLIER\_FLAG} and \texttt{*\_NCOMP} columns of the \texttt{BDModels} catalogue.}
        \\
	\end{tabu}
 \end{table*}
\end{landscape}
\clearpage
}


\afterpage{%
\clearpage%
\begin{landscape}
\begin{table*}
	\centering
	\caption{Fit results for \texttt{v05} of the \texttt{BDDecomp} DMU: numbers (black) and percentages (grey) of galaxies remaining or lost at each step in our pipeline, split per-band and per-model where necessary.$^1$ Individual steps are numbered through, see text for details.}
	\label{tab:resultsv05}
	\setlength{\tabcolsep}{4pt} % 6pt seems to be the default
	\begin{tabu}{l rrr c rrr c rrr c rrr} % number of columns, alignment for each
	    \hline
	    band
	    & \multicolumn{3}{c}{$u$}
	    && \multicolumn{3}{c}{$g$}
	    && \multicolumn{3}{c}{$r$}
	    && \multicolumn{3}{c}{$i$}
	    \\
	    model (components)
	    & \multicolumn{1}{c}{1} & \multicolumn{1}{c}{1.5} & \multicolumn{1}{c}{2} 
	    && \multicolumn{1}{c}{1} & \multicolumn{1}{c}{1.5} & \multicolumn{1}{c}{2} 
	    && \multicolumn{1}{c}{1} & \multicolumn{1}{c}{1.5} & \multicolumn{1}{c}{2} 
	    && \multicolumn{1}{c}{1} & \multicolumn{1}{c}{1.5} & \multicolumn{1}{c}{2} 
	    \\
	    \hline
	    number of: 
	    \\
	    1) unique objects (galaxies)
	    & \multicolumn{3}{c}{13096}
	    && \multicolumn{3}{c}{13096}
	    && \multicolumn{3}{c}{13096}
	    && \multicolumn{3}{c}{13096}
	    \\
	    2a) images (data matches)
	    & \multicolumn{3}{c}{14738}
	    && \multicolumn{3}{c}{14966}
	    && \multicolumn{3}{c}{14966}
	    && \multicolumn{3}{c}{14966}
	    \\
	    2b) objects with >\,0 images
	    & \multicolumn{3}{c}{13096}
	    && \multicolumn{3}{c}{13096}
	    && \multicolumn{3}{c}{13096}
	    && \multicolumn{3}{c}{13096}
	    \\
	    \rowfont{\color{gray}}
	    lost due to missing data (\%)
	    & \multicolumn{3}{c}{0}
	    && \multicolumn{3}{c}{0}
	    && \multicolumn{3}{c}{0}
	    && \multicolumn{3}{c}{0}
	    \\
	    \\
	    3) objects not masked
	    & \multicolumn{3}{c}{10540}
	    && \multicolumn{3}{c}{10768}
	    && \multicolumn{3}{c}{10768}
	    && \multicolumn{3}{c}{10768}
	    \\
	    \rowfont{\color{gray}}
	    lost due to masking (\%)
	    & \multicolumn{3}{c}{20}
	    && \multicolumn{3}{c}{18}
	    && \multicolumn{3}{c}{18}
	    && \multicolumn{3}{c}{18}
	    \\	    
	    4) successful PSFs
	    & \multicolumn{3}{c}{9141}
	    && \multicolumn{3}{c}{10768}
	    && \multicolumn{3}{c}{10752}
	    && \multicolumn{3}{c}{10731}
	    \\
	    \rowfont{\color{gray}}
	    lost due to PSF fails (\%)
	    & \multicolumn{3}{c}{11}
	    && \multicolumn{3}{c}{0}
	    && \multicolumn{3}{c}{0.1}
	    && \multicolumn{3}{c}{0.3}
	    \\
	    \\
		5) successful fits 
		& 9087 & 8902 & 8504  
		&& 10715 & 10640 & 9805 
		&& 10706 & 10609 & 9845
		&& 10674 & 10566 & 9875
		\\		
		\rowfont{\color{gray}}
		lost due to fit fails (\%)
		& 0.4 & 2 & 5
		&& 0.4 & 1 & 7
		&& 0.4 & 1 & 7
		&& 0.4 & 1 & 7
		\\		
		6) fits not flagged 
		& 6165 & 2408 & 880 
		&& 9762 & 6096 & 5913 
		&& 9803 & 7079 & 7000
		&& 9568 & 6527 & 4865
		\\		
		\rowfont{\color{gray}} 
	    not flagged/successful (\%) 
		& 68 & 27 & 10 
		&& 91 & 57 & 60 
		&& 92 & 67 & 71 
		&& 90 & 62 & 49
		\\		
		7) selected fits 
		& 5888 & 254 & 57 
		&& 7213 & 607 & 1691 
		&& 5522 & 682 & 3201 
		&& 7207 & 537 & 1686
		\\		
		\rowfont{\color{gray}} 
		selected/successful (\%)
		& 65 & 3 & 0.7 
		&& 67 & 6 & 17 
		&& 52 & 6 & 33 
		&& 68 & 5 & 17
		\\		
		\\
		total number (per band) of:\\
		8) good\,|\,flagged\,|\,skipped fits
		& \multicolumn{3}{c}{6199\,|\,2936\,|\,3961} 
		&& \multicolumn{3}{c}{9511\,|\,1249\,|\,2336} 
		&& \multicolumn{3}{c}{9405\,|\,1345\,|\,2346} 
		&& \multicolumn{3}{c}{9430\,|\,1298\,|\,2368} 
		\\
		\rowfont{\color{gray}} 
		good\,|\,f.\,|\,s./all objects (\%)
		& \multicolumn{3}{c}{47\,|\,22\,|\,30} 
		&& \multicolumn{3}{c}{73\,|\,10\,|\,18} 
		&& \multicolumn{3}{c}{72\,|\,10\,|\,18} 
		&& \multicolumn{3}{c}{72\,|\,10\,|\,18}
		\\
        \hline
	\end{tabu}
 \end{table*}

\begin{table*}\ContinuedFloat
	\centering
	\setlength{\tabcolsep}{4pt} % 6pt seems to be the default
	\begin{tabu}{l rrr c rrr c rrr c rrr} % number of columns, alignment for each
	    \hline
	    band
	    & \multicolumn{3}{c}{$Z$}
	    && \multicolumn{3}{c}{$Y$}
	    && \multicolumn{3}{c}{$J$}
	    && \multicolumn{3}{c}{$H$}
	    \\
	    model (components)
	    & \multicolumn{1}{c}{1} & \multicolumn{1}{c}{1.5} & \multicolumn{1}{c}{2} 
	    && \multicolumn{1}{c}{1} & \multicolumn{1}{c}{1.5} & \multicolumn{1}{c}{2} 
	    && \multicolumn{1}{c}{1} & \multicolumn{1}{c}{1.5} & \multicolumn{1}{c}{2} 
	    && \multicolumn{1}{c}{1} & \multicolumn{1}{c}{1.5} & \multicolumn{1}{c}{2} 
	    \\
	    \hline
	    number of: 
	    \\
	    1) unique objects (galaxies)
	    & \multicolumn{3}{c}{13096}
	    && \multicolumn{3}{c}{13096}
	    && \multicolumn{3}{c}{13096}
	    && \multicolumn{3}{c}{13096}
	    \\
	    2a) images (data matches)
	    & \multicolumn{3}{c}{46225}
	    && \multicolumn{3}{c}{47678}
	    && \multicolumn{3}{c}{95396}
	    && \multicolumn{3}{c}{45918}
	    \\
	    2b) objects with >\,0 images
	    & \multicolumn{3}{c}{12437}
	    && \multicolumn{3}{c}{12437}
	    && \multicolumn{3}{c}{12436}
	    && \multicolumn{3}{c}{12429}
	    \\
	    \rowfont{\color{gray}}
	    lost due to missing data (\%)
	    & \multicolumn{3}{c}{5}
	    && \multicolumn{3}{c}{5}
	    && \multicolumn{3}{c}{5}
	    && \multicolumn{3}{c}{5}
	    \\
	    \\
	    3) objects not masked
	    & \multicolumn{3}{c}{9989}
	    && \multicolumn{3}{c}{9969}
	    && \multicolumn{3}{c}{9961}
	    && \multicolumn{3}{c}{9945}
	    \\
	    \rowfont{\color{gray}}
	    lost due to masking (\%)
	    & \multicolumn{3}{c}{19}
	    && \multicolumn{3}{c}{19}
	    && \multicolumn{3}{c}{19}
	    && \multicolumn{3}{c}{19}
	    \\	    
	    4) successful PSFs
	    & \multicolumn{3}{c}{9942}
	    && \multicolumn{3}{c}{9894}
	    && \multicolumn{3}{c}{9898}
	    && \multicolumn{3}{c}{9825}
	    \\
	    \rowfont{\color{gray}}
	    lost due to PSF fails (\%)
	    & \multicolumn{3}{c}{0.4}
	    && \multicolumn{3}{c}{0.6}
	    && \multicolumn{3}{c}{0.5}
	    && \multicolumn{3}{c}{0.9}
	    \\
	    \\
		5) successful fits 
		& 9899 & 9825 & 8943 
		&& 9861 & 9721 & 9148 
		&& 9848 & 9768 & 8929
		&& 9687 & 9534 & 8770
		\\		
		\rowfont{\color{gray}}
		lost due to fit fails (\%)
		& 0.3 & 0.9 & 8 
		&& 0.3 & 1 & 6
		&& 0.4 & 1 & 7
		&& 1 & 2 & 8
		\\		
		6) fits not flagged 
		& 8796 & 5305 & 3868 
		&& 8261 & 4414 & 2538 
		&& 8120 & 5081 & 2938
		&& 7368 & 4529 & 2299
		\\		
		\rowfont{\color{gray}} 
	    not flagged/successful (\%) 
		& 89 & 54 & 43 
		&& 84 & 45 & 28 
		&& 82 & 52 & 33 
		&& 76 & 48 & 26
		\\		
		7) selected fits 
		& 6363 & 949 & 1395 
		&& 6247 & 610 & 1239 
		&& 6336 & 475 & 1257 
		&& 5597 & 712 & 1024
		\\		
		\rowfont{\color{gray}} 
		selected/successful (\%)
		& 64 & 10 & 16 
		&& 63 & 6 & 14 
		&& 64 & 5 & 14 
		&& 58 & 7 & 12
		\\		
		\\
		total number (per band) of:\\
		8) good\,|\,flagged\,|\,skipped fits
		& \multicolumn{3}{c}{8707\,|\,1221\,|\,3168} 
		&& \multicolumn{3}{c}{8096\,|\,1786\,|\,3214} 
		&& \multicolumn{3}{c}{8068\,|\,1820\,|\,3208} 
		&& \multicolumn{3}{c}{7333\,|\,2405\,|\,3358} 
		\\
		\rowfont{\color{gray}} 
		good\,|\,f.\,|\,s./all objects (\%)
		& \multicolumn{3}{c}{66\,|\,9\,|\,24} 
		&& \multicolumn{3}{c}{62\,|\,14\,|\,25} 
		&& \multicolumn{3}{c}{62\,|\,14\,|\,24} 
		&& \multicolumn{3}{c}{56\,|\,18\,|\,26}
		\\
        \hline
	\end{tabu}
 \end{table*}
 
 \begin{table*}\ContinuedFloat
	\centering
	\setlength{\tabcolsep}{4pt} % 6pt seems to be the default
	\begin{tabu}{l rrr c rrr c rrr} % number of columns, alignment for each
	    \hline
	    band
	    & \multicolumn{3}{c}{$K_s$}
	    && \multicolumn{3}{c}{$gri$}
	    && \multicolumn{3}{c}{$ugriZYJHK_s$}
	    \\
	    model (components)
	    & \multicolumn{1}{c}{1} & \multicolumn{1}{c}{1.5} & \multicolumn{1}{c}{2} 
	    && \multicolumn{1}{c}{1} & \multicolumn{1}{c}{1.5} & \multicolumn{1}{c}{2} 
	    && \multicolumn{1}{c}{1} & \multicolumn{1}{c}{1.5} & \multicolumn{1}{c}{2} 
	    \\
	    \hline
	    number of: 
	    \\
	    1) unique objects (galaxies)
	    & \multicolumn{3}{c}{13096}
	    && \multicolumn{3}{c}{13096}
	    && \multicolumn{3}{c}{13096}
	    \\
	    2a) images (data matches)
	    & \multicolumn{3}{c}{45993}
	    && \multicolumn{3}{c}{14966\,$\times$\,3}
	    && \multicolumn{3}{c}{340846}
	    \\
	    2b) objects with >\,0 images
	    & \multicolumn{3}{c}{12438}
	    && \multicolumn{3}{c}{13096}
	    && \multicolumn{3}{c}{12426}
	    \\
	    \rowfont{\color{gray}}
	    lost due to missing data (\%)
	    & \multicolumn{3}{c}{5}
	    && \multicolumn{3}{c}{0}
	    && \multicolumn{3}{c}{5}
	    \\
	    \\
	    3) objects not masked
	    & \multicolumn{3}{c}{9955}
	    && \multicolumn{3}{c}{10768}
	    && \multicolumn{3}{c}{9738}
	    \\
	    \rowfont{\color{gray}}
	    lost due to masking (\%)
	    & \multicolumn{3}{c}{19}
	    && \multicolumn{3}{c}{18}
	    && \multicolumn{3}{c}{21}
	    \\
	    4) successful PSFs
	    & \multicolumn{3}{c}{9800}
	    && \multicolumn{3}{c}{10715}
	    && \multicolumn{3}{c}{8082}
	    \\
	    \rowfont{\color{gray}}
	    lost due to PSF fails (\%)
	    & \multicolumn{3}{c}{1}
	    && \multicolumn{3}{c}{0.4}
	    && \multicolumn{3}{c}{13}
	    \\
	    \\
		5) successful fits 
		& 9660 & 9516 & 8694 
		&& 10597 & 10315 & 8635 
		&& 7964 & 7149 & 4457
		\\		
		\rowfont{\color{gray}}
		lost due to fit fails (\%)
		& 1 & 2 & 8
		&& 0.9 & 3 & 16
		&& 0.9 & 7 & 28
		\\		
		6) fits not flagged 
		& 6687 & 4136 & 1910 
		&& 9371 & 5378 & 3464 
		&& 4312 & 1443 & 357
		\\		
		\rowfont{\color{gray}} 
	    not flagged/successful (\%) 
		& 69 & 43 & 22 
		&& 88 & 52 & 40 
		&& 54 & 20 & 8 
		\\		
		7) selected fits 
		& 5366 & 343 & 929 
		&& 7202 & 604 & 1097 
		&& 4051 & 37 & 141 
		\\		
		\rowfont{\color{gray}} 
		selected/successful (\%)
		& 56 & 4 & 11
		&& 68 & 6 & 13 
		&& 51 & 0.5 & 3 
		\\		
		\\
		total number (per band) of:\\
		8) good\,|\,flagged\,|\,skipped fits
		& \multicolumn{3}{c}{6638\,|\,3093\,|\,3365} 
		&& \multicolumn{3}{c}{8903\,|\,1788\,|\,2405} 
		&& \multicolumn{3}{c}{4229\,|\,3912\,|\,4955} 
		\\
		\rowfont{\color{gray}} 
		good\,|\,f.\,|\,s./all objects (\%)
		& \multicolumn{3}{c}{51\,|\,24\,|\,26} 
		&& \multicolumn{3}{c}{68\,|\,14\,|\,18} 
		&& \multicolumn{3}{c}{32\,|\,30\,|\,38} 
		\\
        \hline
        \multicolumn{12}{l}{$^1$Based on the \texttt{*\_N\_MATCH}, \texttt{*\_BDQUAL\_FLAG}, \texttt{*\_OUTLIER\_FLAG} and \texttt{*\_NCOMP} columns.}
        \\
	\end{tabu}
 \end{table*}
 
\end{landscape}
\clearpage
}


Note that for both versions, we used stacked $gri$ images for segmentation and masking (plus additional masking in the non-core bands), but then treated the galaxies independently in all bands except for the model selection, where we performed both a per-band and a joint version. Therefore, the column ``joint $gri$" in Table~\ref{tab:results} always gives the number of galaxies that were ``good" in all three bands (hence why numbers are generally lower), except for the model selection, where it shows the results of the joint model selection (cf. Section~\ref{sec:postprocessing}). The same is true for the corresponding ``$gri$" and ``$ugriZYJHK_s$" columns in Table~\ref{tab:resultsv05}, except that for step 2a) we now - somewhat arbitrarily - give the sum of all data matches in all respective bands since there is no one-to-one correspondence between data matches in the KiDS and VIKING bands, so it is impossible to define the number of joint data matches in all bands. For the remaining rows, however, we then list the numbers of objects that were ``good" in all bands again, resulting in very low numbers for the 9-band joint selection. Also note that for \texttt{v05}, all numbers except those in step 2a) refer to the numbers of galaxies (unique objects) since all images (data matches) of a galaxy were fitted jointly (per band). For \texttt{v04}, individual images were fitted separately, so the numbers in steps 2) to 8a) refer to individual data matches, not unique objects. 

Explanations of each step as numbered through in Tables~\ref{tab:results} and~\ref{tab:resultsv05}:
\begin{enumerate}
\item The full sample results from the combination of our main and SAMI samples (Section~\ref{sec:sampleselection}). 
\item For KiDS bands, some galaxies have been imaged more than once due to overlap regions between the tiles. For VIKING bands, most galaxies have been covered by more than one exposure since we work at the individual chip level. For \texttt{v04}, we treat these duplicate observations of the same physical object independently throughout our pipeline, so numbers in all subsequent rows of Table~\ref{tab:results} refer to individual images, not unique objects. For \texttt{v05}, all images of the same object in the same band are fitted jointly, so all numbers in Table~\ref{tab:resultsv05} refer to unique objects. For the VIKING bands, about 5\,\% of objects are not covered by any chip due to small gaps in the data coverage (Section~\ref{sec:jointfitting}). 
\item For the core bands, we use the associated KiDS masks, combining the three bands. \emph{Images} for which the central galaxy pixel is masked ($\sim$\,20\,\%) are skipped during the fitting (Section~\ref{sec:preparatorysteps}). This results in $\sim$\,18\,\% of \emph{unique objects} being skipped in the core bands. Since we use the $gri$ segmentation maps for all other bands, too, these objects are automatically skipped in all bands. 1-2\,\% of objects are additionally skipped in the KiDS $u$ and VIKING bands due to the masking in the respective bands (Section~\ref{sec:jointfitting}).
\item For each image in each band, a PSF is then estimated by fitting nearby stars. If the PSF estimation fails, the image is skipped during the fit (Section~\ref{sec:preparatorysteps}). Note that technically, we estimate PSFs also for galaxies that are masked in step 3, but we do not list those here. For \texttt{v05}, an object is only skipped if all of its data matches are either masked or have a failed PSF. We count those for which all matches are masked in step 3 (irrespective of whether they also have failed PSFs or not); and those for which all matches failed the PSF as well as those for for which there was a mixture between matches that were masked (but have a PSF) and failed the PSF (but were not masked) here in step 4. 
\item For each non-masked image with a successful PSF estimate, we attempt 3 fits: a single S\'ersic (1), a pointsource + exponential (1.5) and a S\'ersic + exponential (2). In \texttt{v05}, we attempt a joint fit of all images for each object that has at least one non-masked image with a successful PSF estimate. Sometimes, the fit attempts fail with an error (Section~\ref{sec:galaxyfitting}).
\item Each fit (for each model independently) is passed through our outlier flagging process, identifying bad fits (Section~\ref{sec:postprocessing}). We further assess the differences between \texttt{v04} and \texttt{v05} in Section~\ref{sec:outlierstats}. 
\item Of the non-flagged (i.e. good) fits, we then select the most appropriate one during model selection (Section~\ref{sec:postprocessing}). \texttt{v04} and \texttt{v05} model selection results are compared in detail in Section~\ref{sec:modelseldiffs}. 
\item Summing up the selected fits for each model (step 7) gives the total number of good fits. The difference between the good and successful fits (step 5) stems from the outlier flagging. Skipped fits are due to missing data, masking, PSF or fit fails (steps 2b, 3, 4, 5). For \texttt{v04}, the sum of good, flagged and skipped fits in step 8a gives the total number of independent fits (step 2). Removing duplicate fits for the same physical objects gives the number of good, flagged and skipped galaxies in step 8b, which sum to the number of unique objects (step 1). For this, we always use the best available result for each galaxy, i.e. it is counted as ``good" if at least one of the multiple fits was ``good". For \texttt{v05}, there is only one fit per galaxy, so the good, flagged and skipped fits from step 8 sum directly to the number of unique objects (step 1).
\end{enumerate}


\begin{figure}
\centering
	\includegraphics[width=0.65\textwidth]{plots/ncompstats}
	
	\bigskip
	\includegraphics[width=\textwidth]{plots/ncompstatsv05}
    \caption{The number of components assigned in our model selection procedure for individual bands and the joint analysis for \texttt{v04} (\textbf{top}) and \texttt{v05} (\textbf{bottom}) of the pipeline. 1, 1.5 and 2 mean single S\'ersic, point source bulge + exponential disk and S\'ersic bulge + exponential disk models respectively. Negative values indicate that the chosen (best) fit was flagged as unreliable (mostly irregular or partly masked galaxies). -999 is assigned to skipped fits, either because the galaxy centre is masked (most cases), the galaxy falls into a gap in the data (VIKING bands only) or because the PSF estimation failed. For \texttt{v04}, the lighter (higher) bars show the number of images, whereas saturated bars indicate the number of unique objects in both figures; see text and Tables~\ref{tab:results} and~\ref{tab:resultsv05} for details. Horizontal grey lines in the background are placed at equal intervals in both figures to ease the direct comparison.}		
    \label{fig:ncompstats}
\end{figure}

Figure~\ref{fig:ncompstats} visualises the most important information given in Tables~\ref{tab:results} and~\ref{tab:resultsv05}, namely the final number of objects classified in each category. For \texttt{v04} (top panel), lighter bars in the background refer to individual fits (total 14966) with the number of unique galaxies (total 13096) overplotted. When several fits to the same galaxy were classified in different categories, we allocate it to the highest of those,\footnote{This means that a galaxy is classified as ``outlier" if \emph{all} fits to it are outliers and it is ``skipped" only if \emph{all} fits are skipped. Galaxies with good fits are allocated to the most complex model of the available fits (assuming that one of the images was deeper and allowed to constrain more components than the other(s)), while within the outlier categories we allocate it to the simplest model. Note that in Table~\ref{tab:results} we only show the total number of flagged fits and do not split them into the different outlier categories.} which is consistent with Table~\ref{tab:results}. For \texttt{v05} (bottom panel), we only show the number of unique objects, since all images were fitted jointly.

\texttt{NCOMP}\,=\,$-999$ means the object was skipped (not fitted) because it is masked or the PSF estimation failed (usually because of large masked areas in the immediate vicinity of the object). For VIKING bands, objects can also be skipped due to small gaps in the sky coverage. \texttt{NCOMP}\,=\,1, 1.5 or 2 indicates that this is a good fit classified as single, 1.5- or double component fit. \texttt{NCOMP}\,=\,$-1$, $-1.5$ or $-2$ indicates that this is a bad fit (outlier) which would have been classified as single, 1.5- or double component fit if it were not an outlier (most often these are mergers/irregular galaxies for which our models are not appropriate; or galaxies that are partly masked). We keep these three classes separate since automated outlier identification can never be perfect; and what should be considered a bad fit will depend on the use case. The flagging of fits is hence only intended as a guide and all available information in the catalogue is retained for all fitted objects. We analyse Figure~\ref{fig:ncompstats} in more detail in Sections~\ref{sec:outlierstats} and~\ref{sec:modelseldiffs}, where we compare the outlier flagging and model selection statistics for \texttt{v04} and \texttt{v05} of the \texttt{BDDecomp} DMU. 

Examples of fits classified in each category can be found throughout Chapter~\ref{chap:pipeline}, see e.g. Figure~\ref{fig:examplefit} for a double component fit, Figure~\ref{fig:examplefit1.5} for a 1.5-component object, Figure~\ref{fig:examplefitnormvst4} for a single S\'ersic galaxy and Figure~\ref{fig:examplefit-2} for an outlier. 


\subsection{Outlier statistics}
\label{sec:outlierstats}

In this section, we analyse the numbers of outliers in Figure~\ref{fig:ncompstats} and the corresponding Tables~\ref{tab:results} and~\ref{tab:resultsv05} in more detail. Since the outlier flagging was calibrated on \texttt{v04} (cf. Section~\ref{sec:manualcalibrationchanges}), we put particular emphasis on comparing the \texttt{v05} results to those of \texttt{v04} for the core bands ($g$, $r$, $i$) and compare all other bands against those three within \texttt{v05}. The dif\-fe\-ren\-ces in the numbers of skipped fits (\texttt{NCOMP}\,=\,$-999$) are explained in Section~\ref{sec:jointfitting}. In short, the $g, r$ and $i$ bands show the same number of skipped fits in \texttt{v04} and \texttt{v05} since that is mainly due to masking, which did not change. The $u$-band has a higher number of skipped fits due to its smaller footprint, shallower data and the additional masking. VIKING bands have higher numbers of skipped fits mainly due to small gaps in the data coverage. All objects that are skipped in at least one of the bands are counted as skipped for the joint categories, which is the reason for the higher number of skipped objects for the 9-band joint analysis. The differences in the model selection are investigated in Section~\ref{sec:modelseldiffs}. In this section, we focus on differences in the outlier categories, i.e. \texttt{NCOMP}\,=\,$-1$, $-1.5$ and $-2$. 

\begin{table}[t!]
\caption{The percentage of fits flagged as outlier according to each criterion (see text for details) in all bands in \texttt{v04} and \texttt{v05} of the \texttt{BDDecomp} DMU. All values are in per cent of the total number of non-skipped fits, except for the ``extreme B/T ratio" criterion, which is in per cent of the total number of non-skipped 1.5- and double component fits (since it does not apply to single S\'ersic fits). Grey font indicates cautionary flags that are not considered during the final outlier flagging. Bold font in the \texttt{v05} table highlights major changes in the fractions relative to \texttt{v04} (for $g$, $r$ and $i$ bands) and relative to the (\texttt{v05}) $r$-band for all other bands.}
\label{tab:outlierstats}

\begin{subtable}{1\textwidth}
\centering
\caption{Outlier statistics in \texttt{v04}.}
\label{tab:v04outlierstats}
	\begin{tabu}{lrrr} % number of columns, alignment for each
		\hline
		flag & $g$ & $r$ & $i$ \\
		\hline
		irregular segment & 5.48 & 5.37 & 5.27\\ % flag value 2
		extreme B/T ratio & 0.00 & 0.05 & 0.04\\ % flag values 4 and 8
		numerical problems & 0.08 & 0.20 & 0.17\\ % flag value 16
		rel. param hit limit & 3.67 & 5.82 & 4.94\\ % flag value 32
		\rowfont{\color{gray}}
		any param hit limit & 4.60 & 6.46 & 8.09\\ % flag value 64 (cautionary)
		\rowfont{\color{gray}}
		small or large error & 2.14 & 2.06 & 4.13\\ % flag values 128 and 256 (cautionary)
		poor $\chi^2$ statistics & 0.04 & 0.14 & 0.05\\ % flag values 512 and 1024
		position offset >\,2\arcsec\ & 0.42 & 0.35 & 0.39\\ % flag value 2048
		\rowfont{\color{gray}} 
		position offset >\,1\arcsec\ & 1.50 & 1.29 & 1.41\\ % flag value 4096 (cautionary)
		flux in seg <\,20\,\% & 1.42 & 1.42 & 1.29\\ % flag value 8192
		\rowfont{\color{gray}}
		flux in seg <\,50\,\% & 8.04 & 9.32 & 9.33\\ % flag value 16384 (cautionary)
        \hline
        \textbf{total outliers} & \textbf{8.95} & \textbf{10.87} & \textbf{10.13}\\ % flag value 1
        \hline
	\end{tabu}
\end{subtable}

\bigskip
\begin{subtable}{1\textwidth}
\centering
\caption{Outlier statistics in \texttt{v05}.}
\label{tab:v05outlierstats}
	\begin{tabu}{lrrrrrrrrr} % number of columns, alignment for each
		\hline
		flag & $u$ & $g$ & $r$ & $i$ & $Z$ & $Y$ & $J$ & $H$ & $K_s$ \\
		\hline
		irregular segment & 7.21 & \textbf{6.85} & \textbf{6.83} & \textbf{6.68} & 7.06 & 6.81 & 7.88 & 7.65 & 7.71\\ % flag value 2
		extreme B/T ratio & 0.36 & 0.07 & 0.04 & 0.10 & 0.13 & 0.18 & 0.19 & 0.50 & 0.98\\ % flag values 4 and 8
		numerical problems & 1.38 & 0.31 & 0.62 & 0.34 & 0.36 & 0.83 & 0.64 & 1.56 & 2.90\\ % flag value 16
		rel. param hit limit & \textbf{25.40} & \textbf{4.91} & 5.60 & \textbf{5.67} & 5.25 & \textbf{11.18} & \textbf{10.99} & \textbf{17.11} & \textbf{23.60}\\ % flag value 32
		\rowfont{\color{gray}}
		any param hit limit & 28.66 & 5.61 & 6.23 & 6.71 & 6.42 & 13.47 & 13.01 & 20.48 & 28.78\\ % flag value 64 (cautionary)
		\rowfont{\color{gray}}
		small or large error & 28.86 & 5.73 & 6.26 & 6.73 & 6.51 & 13.37 & 12.91 & 20.45 & 28.94\\ % flag values 128 and 256 (cautionary)
		poor $\chi^2$ statistics & 0.10 & 0.09 & 0.17 & 0.11 & 0.08 & 0.13 & 0.10 & 0.15 & 0.15\\ % flag values 512 and 1024
		position offset >\,2\arcsec\ & \textbf{1.02} & 0.39 & 0.32 & 0.34 & 0.48 & 0.54 & 0.59 & \textbf{1.12} & \textbf{1.93}\\ % flag value 2048
		\rowfont{\color{gray}} 
		position offset >\,1\arcsec\ & 3.65 & 1.49 & 1.25 & 1.40 & 1.84 & 2.08 & 2.17 & 3.20 & 4.59\\ % flag value 4096 (cautionary)
		flux in seg <\,20\,\% & 1.54 & 1.09 & 1.27 & 1.14 & 0.86 & 0.84 & 1.13 & 1.51 & 1.71\\ % flag value 8192
		\rowfont{\color{gray}}
		flux in seg <\,50\,\% & 6.23 & 5.25 & 6.45 & 5.77 & 3.88 & 3.59 & 4.88 & 5.51 & 6.39\\ % flag value 16384 (cautionary)
        \hline
        \textbf{total outliers} & \textbf{32.14} & \textbf{11.61} & \textbf{12.50} & \textbf{12.10} & \textbf{12.30} & \textbf{18.06} & \textbf{18.41} & \textbf{24.70} & \textbf{31.79}\\ % flag value 1
        \hline
	\end{tabu}
\end{subtable}

\end{table}

The core bands in general have similar numbers of flagged fits in \texttt{v05} with respect to \texttt{v04}. The VIKING $Z$-band is also comparable to the core bands. For the longer wavelength bands and the KiDS $u$-band, the number of outliers steadily increases. To investigate the origin of this further, Table~\ref{tab:outlierstats} gives the percentages of fits flagged according to each criterion for all bands in \texttt{v04} and \texttt{v05}. To decouple the analysis from differences in the model selection, we show the total number of outliers, i.e. categories \texttt{NCOMP}\,=\,$-2$, $-1.5$ and $-1$ combined. A detailed description of all criteria is given in Section~\ref{sec:postprocessing}, where the corresponding $r$-band \texttt{v04} percentages are also indicated. All values are given in percent of the total number of non-skipped fits; counting only those in the respective model selection category.\footnote{E.g. the 7.21\,\% of $u$-band fits with an irregular segment are composed of the fraction of objects that had this flag raised for their single S\'ersic fit \emph{and} were classified as single S\'ersic fits or single S\'ersic outlier (absolute value of \texttt{NCOMP} equal to one), plus the corresponding fractions of 1.5- and double component fits. For fractions of flagged fits for each model independently, see step 6 of Tables~\ref{tab:results} and~\ref{tab:resultsv05}. Note also that the total number of flagged fits in step 8 of those tables is given as a fraction of the total number of objects, whereas Table~\ref{tab:outlierstats} gives all values as percentages of non-skipped objects.}
Note that the total number of outliers is smaller than the sum of the individual flag values since bad fits frequently fall into multiple outlier categories. Grey font indicates cautionary flags that are not taken as criteria for flagging outliers. We only list these for completeness and do not comment on them further. In Table~\ref{tab:v05outlierstats}, bold font highlights values that show large differences with respect to the \texttt{v04} percentages for the $g$, $r$ and $i$ bands; and with respect to the \texttt{v05} $r$-band for all other bands. 

The fractions of outliers in the $g$, $r$ and $i$ bands in \texttt{v05} are approximately 12\,\%, 13\,\% and 12\,\% respectively. These are 2\,\% to 3\,\% more outliers than in \texttt{v04}. The main driver of this is the ``very irregular fitting segment (irregular segment)" criterion (+1.5\,\%). The $g$ and $i$ bands also seem to have a higher number of (reliable) parameters hitting their fit limits (+1\,\%), which is not observed in the $r$-band. In addition, there are increases in the numerical problems and to a lesser extent in the extreme B/T ratio objects, but the overall numbers of fits classified as outliers according to these criteria are so low that they do not affect the final outlier fraction much. 

There are no differences in the data used between \texttt{v04} and \texttt{v05} and the numbers of skipped fits are nearly identical for the core bands. The only possible reasons for the increased fraction of outliers are differences in the processing, namely the increased segment sizes and the joint fitting of all images (Section~\ref{sec:pipelineupdates}). Considering only galaxies with a single data match (in the core bands) does not change any flag values significantly. This suggests that the joint fitting is not the cause of the increased outlier fractions and instead it is the slightly different (larger) segments. Since the segments used are the same in all bands, the fraction of fits flagged according to the ``irregular segment" criterion are very similar across all bands (with slight differences between bands arising from the definition of this criterion, which also depends on the fit itself, cf. Section~\ref{sec:postprocessing}). 

\begin{figure}
\begin{center}
\includegraphics[width=0.8\textwidth]{plots/examplesegflag}
\caption{The $r$-band segmentation map (dark green contour) for galaxy 14912, which was flagged as having an irregular fitting segment in \texttt{v05} (\textbf{right}), but not in \texttt{v04} (\textbf{left}). Purple contours indicate masked regions. }
\label{fig:examplesegflag}
\end{center}
\end{figure}


Visual inspection of a number of fits, which were flagged for having an irregular segment in \texttt{v05} but not in \texttt{v04}, showed that they were all borderline cases, mostly with large fractions of the segment cut off by a masked region or neighbouring object. An example is shown in Figure~\ref{fig:examplesegflag} for the single S\'ersic object 14912: both segments (dark green contours) are cut off by the bright star mask (purple contour). The segment for \texttt{v05} (right panel) is not ``worse" than that for \texttt{v04} (left panel), suggesting that the ``irregular segment" flag could benefit from a re-calibration in \texttt{v05}. However, since it only affects a relatively small number of objects, all of which are borderline cases, we decided against a re-calibration at this stage. Since we provide all modelling results and flag values for all objects in the catalogue, this can still be revised in the future.


\begin{figure}[t!]
\begin{center}
\includegraphics[width=0.8\textwidth]{plots/examplefitshallow1}
\caption{The $u$-band single S\'ersic fit to galaxy 16537, which is classified as an outlier since it hit its fitting limits in several parameters. Panels in the top two rows are the same as those in Figure~\ref{fig:examplefit}, while the bottom row shows the one-dimensional fit only, corresponding to the rightmost panel of the bottom row in Figure~\ref{fig:examplefit}.}
\label{fig:examplefitshallow1}
\end{center}
\end{figure}

\begin{figure}[t!]
\begin{center}
\includegraphics[width=0.8\textwidth]{plots/examplefitshallow2}
\caption{The $r$-band single S\'ersic fit to galaxy 16537, classified as a single S\'ersic object, for direct comparison to Figure~\ref{fig:examplefitshallow1}.}
\label{fig:examplefitshallow2}
\end{center}
\end{figure}

The fraction of outliers in the VIKING $Z$-band (12\,\%) is comparable to that in the core bands, while the KiDS $u$ and the VIKING $Y, J, H, K_s$ bands have significantly higher fractions, ranging between 18\,\% and 32\,\%. Part of the reason for these higher fractions are the lower numbers of non-skipped fits (i.e. lower normalisation), but as evident from Figure~\ref{fig:ncompstats}, the absolute numbers of outliers increase, too. In all cases, the dominant reason for this are (reliable) parameters hitting their fit limits, with maybe a slight secondary contribution from the position offset flag (Table~\ref{tab:v05outlierstats}). All other flag values show only small increases that can be attributed to the lower normalisation. For the VIKING bands, the fractions of fits flagged generally increases from $Z$ through to $K_s$ for all flag values (in part again due to the higher number of skipped fits from $Z$ through to $K_s$).  

The parameter that hits its fit limit most frequently for single S\'ersic fits is the axial ratio, followed by the S\'ersic index (with a large overlap between the two) and to a lesser extent the effective radius. Almost all of the parameters hitting their fit limits also have a suspiciously large error (meaning they are an outlier in their respective error distribution), which is a cautionary flag. Both of these are indications that the parameters are ill-constrained and can be explained by the decreasing data quality (in particular depth) as a function of wavelength for VIKING (Section~\ref{sec:viking}) and the KiDS $u$-band (Section~\ref{sec:kids}; which was also the reason why the $u$-band was not considered during \texttt{v04} and is not used as a ``core" band in \texttt{v05}). The shallowest bands ($u$ and $K_s$) are hence not only too shallow to constrain two components in most cases (cf. Section~\ref{sec:manualcalibrationchanges}), but also too shallow to constrain a single S\'ersic fit for approximately one quarter of our sample and therefore become flagged as outliers. As we discuss in Section~\ref{sec:manualcalibrationchanges}, this is a fundamental problem that can only be solved with deeper data or simultaneous multi-band fits. 

For the same reason, the fractions of non-flagged fits provided in Table~\ref{tab:resultsv05} decrease drastically as a function of data quality and model complexity. For example, in the $u$-band, 90\,\% of the attempted double component fits are flagged as outliers. The vast majority of these, however, were not classified as double component objects since the data can equally well be represented by a single S\'ersic fit. What drives the higher fraction of outliers in these shallow bands are therefore mostly the objects for which even the single S\'ersic fits cannot be constrained anymore. Figure~\ref{fig:examplefitshallow1} shows an example of such a fit where the $u$-band data is of insufficient quality to constrain a single S\'ersic model. The fit was flagged as an outlier since it hit its parameter limits for the axial ratio and S\'ersic index. For comparison, we show the $r$-band version of the same galaxy in Figure~\ref{fig:examplefitshallow2}. This fit was classified as a (good) single S\'ersic fit. 


\subsection{Model selection differences}
\label{sec:modelseldiffs}


We now turn to the model selection differences between \texttt{v04} and \texttt{v05}, as shown in Figure~\ref{fig:ncompstats} and Tables~\ref{tab:results} and~\ref{tab:resultsv05}. The model selection differences between individual bands in \texttt{v05} have already been discussed in Section~\ref{sec:manualcalibrationchanges} where we also give full confusion matrices against visual inspection for the model selection in all bands and the joint model selection in \texttt{v05}. Corresponding \texttt{v04} confusion matrices are provided in Sections~\ref{sec:postprocessing} and~\ref{sec:swappingandoutliers}. Here, we supplement these by a new type of confusion matrix, namely that between the \texttt{v04} and \texttt{v05} model selections for the core bands. For the most direct comparison, we consider only those galaxies that had a single data match and were not skipped in any bands in either catalogue (leaving 8900 objects), and to decouple the analysis from outlier flagging differences (Section~\ref{sec:outlierstats}), we take absolute values of \texttt{NCOMP}. 

\begin{table}
\centering
\caption{The confusion matrices between the \texttt{v04} and \texttt{v05} model selections in percent of the total number of objects that have a single data match and were not skipped in any of the core bands in either catalogue. Bold font highlights those galaxies classified in the same category for both versions. Total fractions of objects classified into each of the three cateogories (i.e. the sum of each row/column) are also given; as well as the total percentage of objects for which the model selection is in agreement between the two versions (sum of the diagonal).}
\label{tab:modelselv04vsv05}
\begin{subtable}{1\textwidth}
\centering
	\begin{tabu}{lcrrrcrcccrrrcr} % number of columns, alignment for each
	& & \multicolumn{5}{c}{$g$-band} & &
	& & \multicolumn{5}{c}{$r$-band}\\
	\hline
	& & \multicolumn{3}{c}{\texttt{v05 NCOMP}} &&&&
	& & \multicolumn{3}{c}{\texttt{v05 NCOMP}} &&\\
		 
		\texttt{v04 NCOMP} & 
		& 1 & 1.5 & 2 && total &&& 
		& 1 & 1.5 & 2 && total\\
		\hline
		
		1 &
		& \textbf{69.2} & 0.9 & 2.3 && 72.4 &&&
		& \textbf{53.1} & 2.1 & 5.0 && 60.2\\
		
		1.5 &
		& 0.8 & \textbf{4.2} & 2.4 && 7.4 &&&
		& 0.2 & \textbf{3.0} & 2.2 && 5.5\\
		
		2 &
		& 1.9 & 1.5 & \textbf{16.8} && 20.2 &&& 
		& 1.6 & 2.0 & \textbf{30.8} && 34.3\\\\
		
		total &
		& 71.8 & 6.7 & 21.5 && \textbf{90.2} &&& 
		& 54.8 & 7.1 & 38.1 && \textbf{86.9}\\
		
        \hline
	\end{tabu}
\end{subtable}%

\bigskip
\begin{subtable}{1\textwidth}
\centering
	\begin{tabu}{lcrrrcrcccrrrcr} % number of columns, alignment for each
	& & \multicolumn{5}{c}{$i$-band} & &
	& & \multicolumn{5}{c}{$gri$ joint}\\
	\hline
	& & \multicolumn{3}{c}{\texttt{v05 NCOMP}} &&&&
	& & \multicolumn{3}{c}{\texttt{v05 NCOMP}} &&\\
		 
		\texttt{v04 NCOMP} &  
		& 1 & 1.5 & 2 && total &&&
		& 1 & 1.5 & 2 && total\\
		\hline
		
		1 &
		& \textbf{63.3} & 0.1 & 0.3 && 63.8 &&& 
		& \textbf{63.4} & 0.4 & 0.8 && 64.6\\
		
		1.5 &
		& 2.5 & \textbf{3.5} & 0.2 && 6.2 &&& 
		& 1.6 & \textbf{2.8} & 2.0 && 6.3 \\
		
		2 &
		& 8.9 & 1.8 & \textbf{19.4} && 30.1 &&& 
		& 11.1 & 3.3 & \textbf{14.7} && 29.1 \\\\
		
		total &
		& 74.7 & 5.3 & 19.9 && \textbf{86.2} &&&
		& 76.0 & 6.5 & 17.5 && \textbf{80.9} \\
		
        \hline
	\end{tabu}
\end{subtable}%
\end{table}


\begin{table}
\centering
\caption{The same as Table~\ref{tab:modelselv04vsv05} but using the \texttt{v04}-calibrated $\Delta$DIC cuts for the \texttt{v05} model selection.}
\label{tab:modelselv04vsv05b}
\begin{subtable}{1\textwidth}
\centering
	\begin{tabu}{lcrrrcrcccrrrcr} % number of columns, alignment for each
	& & \multicolumn{5}{c}{$g$-band} & &
	& & \multicolumn{5}{c}{$r$-band}\\
	\hline
	& & \multicolumn{5}{c}{\texttt{v05 NCOMP (v04 cal.)}} &&
	& & \multicolumn{5}{c}{\texttt{v05 NCOMP (v04 cal.)}} \\
		 
		\texttt{v04 NCOMP} & 
		& 1 & 1.5 & 2 && total &&& 
		& 1 & 1.5 & 2 && total\\
		\hline
		
		1 &
		& \textbf{69.3} & 1.5 & 1.7 && 72.4 &&&
		& \textbf{57.7} & 0.7 & 1.8 && 60.2\\
		
		1.5 &
		& 0.8 & \textbf{6.1} & 0.5 && 7.4 &&&
		& 0.3 & \textbf{4.6} & 0.5 && 5.5\\
		
		2 &
		& 1.6 & 1.7 & \textbf{16.9} && 20.2 &&& 
		& 1.9 & 2.1 & \textbf{30.3} && 34.3\\\\
		
		total &
		& 71.6 & 9.3 & 19.1 && \textbf{92.3} &&& 
		& 60.0 & 7.4 & 32.6 && \textbf{92.6}\\
		
        \hline
	\end{tabu}
\end{subtable}%

\bigskip
\begin{subtable}{1\textwidth}
\centering
	\begin{tabu}{lcrrrcrcccrrrcr} % number of columns, alignment for each
	& & \multicolumn{5}{c}{$i$-band} & &
	& & \multicolumn{5}{c}{$gri$ joint}\\
	\hline
	& & \multicolumn{5}{c}{\texttt{v05 NCOMP (v04 cal.)}} &&
	& & \multicolumn{5}{c}{\texttt{v05 NCOMP (v04 cal.)}} \\
		 
		\texttt{v04 NCOMP} &  
		& 1 & 1.5 & 2 && total &&&
		& 1 & 1.5 & 2 && total\\
		\hline
		
		1 &
		& \textbf{61.4} & 0.9 & 1.5 && 63.8 &&& 
		& \textbf{62.1} & 1.1 & 1.4 && 64.6\\
		
		1.5 &
		& 0.5 & \textbf{5.3} & 0.4 && 6.2 &&& 
		& 0.6 & \textbf{5.4} & 0.3 && 6.3 \\
		
		2 &
		& 1.7 & 1.7 & \textbf{26.6} && 30.1 &&& 
		& 2.9 & 3.9 & \textbf{22.2} && 29.1 \\\\
		
		total &
		& 63.6 & 7.9 & 28.5 && \textbf{93.2} &&&
		& 65.6 & 10.4 & 24.0 && \textbf{89.7} \\
		
        \hline
	\end{tabu}
\end{subtable}%
\end{table}

Table~\ref{tab:modelselv04vsv05} shows the resulting confusion matrices for all bands and the joint $gri$ model selection, with bold colours highlighting the galaxies that were classified in the same category in both catalogue versions. For better overview, we also show the sums of each row and column (total numbers of objects classified in each model selection category for each version independently) and the sum of the diagonal, i.e. the total number of objects for which the model selection agrees between the two versions. All values are in percent of the total number of objects.

The overall agreement between the model selections in \texttt{v04} and \texttt{v05} ranges between $\sim$\,81\,\% for the joint model selection to 90\,\% for the $g$-band. The largest rate of confusion is observed between single and double component objects. For the $r$-band, there are more double and 1.5-component fits in \texttt{v05} than in \texttt{v04} (also evident from Figure~\ref{fig:ncompstats}). This is reversed in the $i$-band and the joint model selection, where the number of single component fits is higher in \texttt{v05}. For the $g$-band, the relative numbers of 1.5-/double component fits and single S\'ersic objects is approximately the same in both versions. 

Comparing the $\Delta$DIC$_{1-2}$ cuts shown in Tables~\ref{tab:v04diccuts} and~\ref{tab:v05diccuts}, we can see that for the $g$-band, the $\Delta$DIC cuts are similar in \texttt{v04} and \texttt{v05}.\footnote{Note that the absolute DIC values in \texttt{v05} are generally a factor of about two larger than in \texttt{v04} due to the larger segments. However, the DIC \emph{differences} are comparable since all models are equally affected by the larger segments. We therefore would expect the $\Delta$DIC cuts to be similar in both catalogue versions.} 
For the $r$-band, the \texttt{v05} cut is slightly lower than in \texttt{v04}, although still within the uncertainty region. The $i$-band and the joint $gri$ model selection both have significantly higher $\Delta$DIC$_{1-2}$ cuts in \texttt{v05} than in \texttt{v04}, with no overlap between the ``unsure" regions. This explains the low confusion rate between \texttt{v04} and \texttt{v05} in the $g$-band, the slightly higher number of double components fits in \texttt{v05} in $r$, and the significantly lower number of double component fits in the $i$-band and joint model selection. 

Since the procedure for the model selection calibration is identical in \texttt{v04} and \texttt{v05}, these differences must stem from differences in the fits themselves or in the manual (re-)calibrations. To judge the effect of differences in the fits (either from the random nature of the MCMC sampling or from systematic differences e.g. due to the larger segments), we additionally performed the model selection on \texttt{v05} using the $\Delta$DIC cuts from \texttt{v04}. The resulting confusion matrices between the model selection categories for \texttt{v04} and \texttt{v05} (with \texttt{v04} calibration) are given in Table~\ref{tab:modelselv04vsv05b}. The total fractions of fits classified in the same model selection category (sum of the diagonals) rise to 92.3\,\%, 92.6\,\%, 93.2\,\% and 89.7\,\% respectively for the $g, r, i$ individual and $gri$ joint model selection, compared to 90.2\,\%, 86.9\,\%, 86.2\,\% and 80.9\,\% in Table~\ref{tab:modelselv04vsv05}. The relative fractions of single S\'ersic and 1.5-/double component fits in the two versions are now within 0.2\,\% of each other for all bands ($\sim$\,1\,\% for the joint selection). The fractions of 1.5-component fits are about 2\,\% higher in \texttt{v05} than in \texttt{v04} for all bands (4\,\% for the joint $gri$ selection), with the double component fractions correspondingly lower, despite using the same $\Delta$DIC cuts. 

This indicates that confusion between models on the level of approximately 7-8\,\% (10\,\% for the joint selection) stems from differences between the fits themselves (to the same galaxies), either of random nature or due to the different fitting procedures in \texttt{v04} and \texttt{v05}, most likely the larger segments (see Section~\ref{sec:pipelineupdates}). This introduces no systematic differences in the classification between the single S\'ersic and 1.5-/double component models, but generally increases the numbers of 1.5-component fits relative to the double component fits. The reasons for the latter remain to be investigated.

The differences between Tables~\ref{tab:modelselv04vsv05} and~\ref{tab:modelselv04vsv05b} are caused by differences in the $\Delta$DIC cuts for \texttt{v04} and \texttt{v05}, which in turn must be caused by differences in the visual calibrations. Reasons could be statistical fluctuations due to the randomly selected calibration samples of galaxies (that differ in \texttt{v04} and \texttt{v05}) as well as human error, both amplified by the relatively small sample size of 200 objects per band in \texttt{v05}. Since the visual classifications were performed several years apart, there could also be systematic differences due to a deepened understanding of the fit results, modelling limitations, systematic uncertainties and model selection caveats based on the detailed investigations of the \texttt{v04} results in the meantime. In particular, the \texttt{v05} calibration put special emphasis on classifying as few fits as possible in the ``unsure" or ``outlier" categories since that further reduces the already relatively low number of objects available for model selection calibration. We also explicitly de-coupled model selection from outlier rejection, the identification of swapped fits and segmentation failures. If, for example, the double component fit (within the segment) is significantly better than the other two models, in the \texttt{v05} calibration we labelled this object as a double component object regardless of whether it is swapped, would be better classified as an outlier or the segmentation map failed (e.g. containing secondary objects). This means that the two categories ``unsure" and ``unfittable/outlier" became equivalent (as they always were in the $\Delta$DIC cut calibration, where both are ignored). Conversely, if the object clearly has a bulge but the 1.5- and double component models failed to fit it and are not better than the single S\'ersic fit, then the object was labelled as ``single". While the general notions were the same during the \texttt{v04} visual inspections - driven by the limitations of a model selection based on a statistical measure - the criteria were not as sharply defined. 

\begin{figure}[t!]
\includegraphics[width=0.5\textwidth]{plots/modelselv04v05g}
\includegraphics[width=0.5\textwidth]{plots/modelselv04v05r}
\includegraphics[width=0.5\textwidth]{plots/modelselv04v05i}
\includegraphics[width=0.5\textwidth]{plots/modelselv04v05gri}
\caption{The visual classification during manual sorting plotted against $\Delta$DIC$_{1-2}$ for \texttt{v04} (blue points) and \texttt{v05} (orange points) in the $g$ (\textbf{top left}), $r$ (\textbf{top right}) and $i$ (\textbf{bottom left}) bands as well as for the $gri$ joint model selection (\textbf{bottom right}; with DICs summed for all bands). Vertical solid and dashed lines indicate the corresponding $\Delta$DIC$_{1-2}$ cuts for \texttt{v04} (blue) and \texttt{v05} (orange).}
\label{fig:modelselv04v05}
\end{figure}

To investigate such manual calibration differences, Figure~\ref{fig:modelselv04v05} shows the visual classifications in \texttt{v04} (blue) and \texttt{v05} (orange) as a function of the DIC difference between the single and double component models, where we see most confusion in Table~\ref{tab:modelselv04vsv05}. Note that the galaxies shown are not the same and the number of objects is higher for \texttt{v04} ($\sim$\,700 per band; 2000 in the joint selection) than for \texttt{v05} (200 per band, 600 in the joint selection). The corresponding calibrated $\Delta$DIC$_{1-2}$ cuts with their ``unsure" regions (solid and dashed lines) are also indicated. The visual inspection categories that are relevant for calibrating the $\Delta$DIC$_{1-2}$ cut are ``single", ``double" and ``1.5 or double" (cf. Section~\ref{sec:postprocessing}); we show the other categories for completeness only. 

Overall, the distribution of blue and orange points in each panel is similar, indicating that there are no fundamental differences in the visual calibrations for \texttt{v04} and \texttt{v05}. For the $g$ and $r$ bands, the DIC difference cuts are consistent between the two versions, as already mentioned above when comparing Tables~\ref{tab:v04diccuts} and~\ref{tab:v05diccuts}. Hence, the confusion rates in these bands is low with especially the $g$-band coming close to the confusion from fit differences alone (Tables~\ref{tab:modelselv04vsv05} and~\ref{tab:modelselv04vsv05b}). The cuts in the $i$-band and joint selection are not consistent, although they are still relatively close to each other compared to the vast range the DIC differences observed. Therefore, the confusion rates between \texttt{v04} and \texttt{v05} in these bands are still moderate despite the numerical differences in the $\Delta$DIC cuts (Tables~\ref{tab:modelselv04vsv05} as well as~\ref{tab:v04diccuts} and~\ref{tab:v05diccuts}). 

The higher cut in \texttt{v05} in the $i$-band appears to be caused by a cluster of orange points around a $\Delta$DIC$_{1-2}$ value of $10^3$, that has no blue counterpart (remember that there are a factor of 3.5 more blue points than orange ones) and is also not observed in the $g$ and $r$ bands. It is possible that this is simply caused by an unfortunate selection of galaxies or random human error and would be downweighted by a larger sample of objects used for calibration. Alternatively, it could be systematic due to the sharpened criteria for the \texttt{v05} calibration. Visual inspection of this cluster of datapoints reveals them all to be borderline cases. Figures~\ref{fig:examplefitdicdiff1} and~\ref{fig:examplefitdicdiff2} show an example, where the double component fit is better than the single S\'ersic fit, but it is debatable whether it is ``significantly" better in the sense that it justifies the increased number of parameters. In the \texttt{v05} calibration we decided that this was not the case, but maybe the decision would have been different during the \texttt{v04} calibration; or even when inspecting the same galaxy at a later or earlier time in the \texttt{v05} calibration, as there is always a certain amount of scatter associated with visual inspection. %\\

\begin{figure}
\begin{center}
\includegraphics[width=0.8\textwidth]{plots/examplefitdicdiff1}
\caption{The single S\'ersic fit to galaxy 585561 in the $i$-band, which was visually classified as a single S\'ersic object despite its relatively high $\Delta$DIC$_{1-2}$ value of $\sim$\,1500. Panels in the top two rows are the same as those in Figure~\ref{fig:examplefit}, while the bottom row shows the one-dimensional fit only, corresponding to the rightmost panel of the bottom row in Figure~\ref{fig:examplefit}.}
\label{fig:examplefitdicdiff1}
\end{center}
\end{figure}

\begin{figure}[t!]
\begin{center}
\includegraphics[width=0.8\textwidth]{plots/examplefitdicdiff2}
\caption{The double component fit to galaxy 585561 in the $i$-band, for direct comparison to Figure~\ref{fig:examplefitdicdiff1}. Panels are the same as those in Figure~\ref{fig:examplefit}.}
\label{fig:examplefitdicdiff2}
\end{center}
\end{figure}

The last point of the model selection that we would like to return to are the high confusion rates against visual inspection for the $g$ and $r$ bands in \texttt{v05} that were apparent from Table~\ref{tab:modelselconfusionv05}. As a reminder, the \texttt{v05} $g$ and $r$ bands and the two versions of the joint model selections have approximately 15\,\%, 17\,\%, 17\,\% and 18\,\% of fits classified wrongly by the automated procedure compared to visual inspection, mostly due to \texttt{NCOMP}\,=\,1 objects that were visually classified as doubles. For all other bands and also for the \texttt{v04} versions of the same bands, the confusion rate is 9\,\% at most. Based on the above analysis of the \texttt{v04} and \texttt{v05} model selection differences, we believe that this is mainly due to the lower number of fits classified as ``unsure" in the \texttt{v05} model selection. 

Going back to Figure~\ref{fig:modelselv04v05}, the objects responsible for the high confusion are orange points in the ``double" manual sorting category but to the left of the orange $\Delta$DIC cut. There is a much higher number of those in the $g$ and $r$ bands than in $i$, partly due to the generally lower number of double component objects in $i$. Most blue points (\texttt{v04} model selection) in that region of $\Delta$DIC$_{1-2}$ instead populate the ``unsure" category, where there is generally a very low number of orange points due to our explicit attempt to classify as few galaxies as possible as ``unsure" in the \texttt{v05} calibration. From this - and the consistency of the \texttt{v04} and \texttt{v05} $\Delta$DIC$_{1-2}$ cuts in the $g$ and $r$ bands - we conclude that the automated model selection itself is not ``worse" in \texttt{v05} than it was in \texttt{v04}. Instead, the higher confusion statistics is simply caused by a large number of ambiguous objects that were more or less randomly assigned to one of the two categories in the \texttt{v05} visual inspection, whereas they were classified as ``unsure" and ignored during \texttt{v04} model selection. 


Considering all of the above differences and uncertainties, we conclude that the confusion rates between the model selections in \texttt{v04} and \texttt{v05} as well as those between the \texttt{v05} automated and manual classifications are acceptable. The quality of the \texttt{v05} model selection is comparable to that of \texttt{v04} despite the lower number of visual classifications per band. 


\section{Parameter distributions}

After the analysis of the outlier rejection and model selection statistics, we take a closer look at the fitted structural parameters, considering only good fits within the respective model selection category for each band individually. We do not use the joint model selection or take a combined sample across all bands due to the differences in data quality between the bands that result in vastly different relative numbers of galaxies in each category. Section~\ref{sec:v04parameterdistributions} shows the parameter distributions for \texttt{v04} of the catalogue, taken directly from \citet{Casura2022}. Section~\ref{sec:v05parameterdistributions} adds the new \texttt{v05} versions (not included in \citealt{Casura2022} or elsewhere) and compares them to those of \texttt{v04}. Finally, Section~\ref{sec:colours} focuses on galaxy and component colours, again based on \citet{Casura2022}. 

\subsection{\texttt{v04} parameter distributions}
\label{sec:v04parameterdistributions}

\begin{figure}
    \includegraphics[width=\textwidth]{plots/resultshists}
    \caption{The distribution of the main parameters (limited to segment radii) for all bands and models in \texttt{v04}. Left, middle and right columns show magnitude, effective radius and S\'ersic index while top, middle and bottom rows show the $g$-band, $r$-band and $i$-band respectively. The solid yellow lines are the single S\'ersic (S) values for those galaxies which were classified as single component systems, dotted red and dashed blue lines show bulges (B) and disks (D), respectively, for those objects classified as 1.5- or double component systems. For reference the solid black line shows the single S\'ersic fits for all galaxies with \texttt{NCOMP}\,>\,0 (i.e. including those classified as 1.5- or double component systems). The number of objects in each histogram is given in the legends, where the number of bulges and disks differs for effective radii and S\'ersic indices because these parameters do not exist for 1.5-component fits (point source bulge). We do not show disk S\'ersic indices since they were fixed to 1 (exponential).}	
    \label{fig:resultshists}
\end{figure}

Figure~\ref{fig:resultshists} shows the distribution of the main parameters - magnitude, effective radius and S\'ersic index - for \texttt{v04} of the \texttt{BDDecomp} DMU in all three bands ($g$, $r$ and $i$) for single S\'ersic fits, bulges and disks. The single S\'ersic fit distributions are shown for all galaxies with \texttt{NCOMP}\,>\,0 (i.e. all non-outliers) in black and for those galaxies which were actually classified as single component systems (\texttt{NCOMP}\,=\,1) in yellow. Red dotted and blue dashed lines show bulges and disks, respectively. For disks we show the 1.5-component fits and double component fits combined (i.e. the 1.5-component parameters for objects with \texttt{NCOMP}\,=\,1.5 and double component parameters for those with \texttt{NCOMP}\,=\,2 added into one histogram); the S\'ersic index is not shown since it was fixed to 1. Bulge magnitudes are also shown for 1.5- and double component fits combined; effective radii and S\'ersic indices are only shown for the double component fits since they do not exist in the point source model. The legend indicates the numbers of objects in each histogram, which can also be inferred from Table~\ref{tab:results}. Magnitudes and effective radii are truncated at the segment radii which we found to give more robust results than using the S\'ersic values extrapolated to infinity (see Sections~\ref{sec:postprocessing}, \ref{sec:comparelee} and \ref{sec:systematics}). 

The first thing apparent from Figure~\ref{fig:resultshists} is that the distributions in the three bands are generally very similar, which is reassuring given that the fits were performed independently. Looking at the first column, the single S\'ersic number counts increase up to a sharp drop just before 20\,mag in all bands, which is not surprising given the GAMA survey limit of 19.8\,mag. The faintest of these objects are all classified as single component galaxies (the yellow lines are on top of the black lines), while some of the brighter objects are successfully decomposed into bulges and disks. Disks are generally slightly brighter than bulges. The bulges show a second, smaller peak at very faint magnitudes which we found to be the ones from the 1.5-component fits (unresolved, faint bulges). There is a slight trend for magnitudes to become brighter moving from $g$ to $r$ to $i$ for all components, as expected from typical galaxy colours. We investigate the colours further in Section~\ref{sec:colours}. 

From the middle column it becomes obvious that bulges tend to be smaller than disks by a factor $\sim$\,2, while single S\'ersic fits span a wide range of sizes. Similar to the trend observed in the magnitudes, the smallest objects are classified as single component systems, while some of the larger galaxies can be successfully decomposed. 

\begin{figure}[t!]
\begin{center}
    \includegraphics[width=0.65\textwidth]{plots/BThist}
    \caption{The distribution of the bulge to total flux ratio (limited to segment radii) for the 1.5- and double component fits in all bands in \texttt{v04}. Dashed green, solid light red and dotted dark pink lines refer to the $g$, $r$ and $i$ bands, respectively. The histograms have been normalised by their respective total number of fits (cf. Figure~\ref{fig:resultshists}) to make the bands directly comparable.}	
    \label{fig:BThist}
\end{center}
\end{figure}

\begin{figure}[h!]
\begin{center}
    \includegraphics[width=0.8\textwidth]{plots/magrecovery}
    \caption{The difference between the single S\'ersic magnitude and the total magnitude derived from the double or 1.5-component fits for those galaxies that were classified as such, for all bands in \texttt{v04} (all magnitudes limited to segment radii). The scatter plot shows the difference between the two magnitudes against the single S\'ersic magnitude for all three bands with the running medians and 1$\sigma$-percentiles overplotted. The top and right panels show the respective marginal distributions. Dashed green, solid light red and dotted dark pink lines refer to the $g$, $r$ and $i$ bands respectively.}	
    \label{fig:magrecovery}
\end{center}
\end{figure}

The S\'ersic indices of single component systems show a clear peak around a value of 1 (exponential), a sharp drop-off at lower values and a longer tail towards higher values. Interestingly, the single S\'ersic distributions showing all systems (black lines) have a secondary ``bump" around a value of 4 or 5 (classical de Vaucouleurs bulge), which is not apparent in those galaxies classified as single component systems (yellow line). Hence most of those high S\'ersic index objects were found to contain bulges and were classified as double component systems. The bulges themselves show a wide range of S\'ersic indices with (at least in $r$ and $i$ bands) a slightly double-peaked nature around values of 1 and 4-6. At this point, we would like to remind the reader that we use the term ``bulge" to refer to all kinds of central components of galaxies, including classical bulges, pseudo-bulges, bars and AGN (cf. Section~\ref{sec:galaxyfitting}). Hence the ``bulge" distribution will include a variety of physical components and their combinations, leading to the wide spread of values. In addition, the S\'ersic index tends to be the parameter with the largest uncertainty, with typical galaxies showing relative errors on their bulge S\'ersic index of 1-10\,\%, adding further scatter to the distribution.%\\

Since the bulge to total flux ratio is a derived parameter that is frequently of interest, we additionally show it in Figure~\ref{fig:BThist} for all three bands; for those galaxies that were classified as a 1.5- or double component fit in the respective band. The majority of systems have intermediate values of B/T with only a few percent at the extreme end above 0.8. The secondary peak at very low B/T values around 0.02 stems from the 1.5-component fits. The B/T ratio generally increases from $g$ to $r$ to $i$, as expected (see Section~\ref{sec:prevcols}). 

Finally, as a first consistency check, we show the difference between the single S\'ersic magnitude and the total magnitude derived from the double or 1.5-component fits, all limited to segment radii, in Figure~\ref{fig:magrecovery}. The distributions for all three bands are highly peaked around zero, with the vast majority of objects having total magnitudes consistent with the single S\'ersic magnitudes within 0.1\,mag (over the entire magnitude range). We only show galaxies that were classified as 1.5- or double component fits here to ensure reliable bulge and disk magnitudes, but note that the distribution is very similar when including objects classified as single S\'ersic fits. This indicates that the total magnitude is well-constrained even in the case when the individual component magnitudes are not (see also Section~\ref{sec:colours}). 


\subsection{\texttt{v05} parameter distributions}
\label{sec:v05parameterdistributions}

\begin{figure}
    \includegraphics[width=\textwidth]{plots/resultshistsv05a}
    \phantomcaption
\end{figure}

\begin{figure}\ContinuedFloat
    \includegraphics[width=\textwidth]{plots/resultshistsv05b}
    \caption{The distribution of the main parameters (limited to segment radii) for all bands and models in \texttt{v05}. Panels are the same as in Figure~\ref{fig:resultshists}, except that we now show all nine bands ($u$, $g$, $r$, $i$, $Z$, $Y$, $J$, $H$, $K_s$ from top to bottom). The axis scales and bin sizes are the same for all bands and also with respect to Figure~\ref{fig:resultshists}.}	
      \label{fig:resultshistsv05}
\end{figure}

Figure~\ref{fig:resultshistsv05} shows the \texttt{v05} equivalent of Figure~\ref{fig:resultshists}: the distribution of the three main S\'ersic parameters in all nine bands for single S\'ersic fits, bulges and disks. The layout of the plot, the colour coding of the lines and the scales of the axes are identical to those in Figure~\ref{fig:resultshists}. A more detailed, direct comparison between the results in \texttt{v04} and \texttt{v05} is shown in Figures~\ref{fig:resultsv04vsv05_R_S} to~\ref{fig:resultsv04vsv05_R_D_SEGRAD} at the end of this section.

Generally, the distributions of the parameters in Figure~\ref{fig:resultshistsv05} are similar in all bands and similar to those in Figure~\ref{fig:resultshists}. The total number of fits in the core bands is lower in \texttt{v05} than in \texttt{v04} due to the joint fitting: in \texttt{v04} we show all fits obtained to individual images of the same galaxy, while in \texttt{v05} there is only one fit per galaxy. For all other bands, the total numbers are lower due to the increased fractions of skipped fits and outliers, cf. Sections~\ref{sec:jointfitting} and~\ref{sec:outlierstats}. The number of objects shown for individual components varies greatly due to the differences in the model selection between bands (Sections~\ref{sec:manualcalibrationchanges} and~\ref{sec:modelseldiffs}), where - as for \texttt{v04} - the brightest and largest objects, often with high single S\'ersic indices, are most frequently decomposed successfully. 

Magnitudes become systematically brighter as a function of wavelength due to typical galaxy colours. Effective radii and S\'ersic indices do not show obvious trends as a function of wavelength, with no indications of inconsistencies in the analysis between different datasets or bands. For all bands, disks tend to be brighter than bulges, with the bulge magnitudes showing a double-peaked nature due to the 1.5- and double component fits. Disks are also typically larger than bulges. The bulge S\'ersic indices again show a large range of values, although with a slight trend towards higher values for longer wavelength bands. %\\


\begin{figure}[t!]
\begin{center}
    \includegraphics[width=0.8\textwidth]{plots/BThistv05}
    \caption{The distribution of the bulge to total flux ratio (limited to segment radii) for the 1.5- and double component fits in all bands in \texttt{v05}. This figure is equivalent to Figure~\ref{fig:BThist}, except that we now show density plots instead of normalised histograms for better visibility due to the higher number of bands.}	
      \label{fig:BThistv05}
\end{center}
\end{figure}

The \texttt{v05} equivalent to Figure~\ref{fig:BThist} is shown in Figure~\ref{fig:BThistv05}. It shows the distribution of the bulge to total flux ratio for all nine bands as a density plot on a logarithmic $x$-axis scale. The general trend observed in Figure~\ref{fig:BThist} persists: the B/T ratio increases as a function of wavelength. The double-peaked nature is again due to the 1.5- and double component fits, where the former populate the region of very low B/T values and the latter the intermediate and high values. The $u$-band is very different from the others since it has a factor of $\sim$\,10 less B/T ratio measurements available than the core bands; and almost all of those are 1.5-component fits (see first row in Figure~\ref{fig:resultshistsv05}). Notable exceptions from the general trend of increased B/T with wavelength are observed for the $Z$ and $H$ bands, both of which have exceptionally high numbers of 1.5-component fits relative to the double component fits (see relative numbers in rows 5 and 8 in Figure~\ref{fig:resultshistsv05}; also visible as larger ``bumps" at low magnitudes). This results in lower B/T values on average for these bands. 


\begin{figure}[t!]
\begin{center}
    \includegraphics[width=0.8\textwidth]{plots/magrecoveryv05}
    \caption{The difference between the single S\'ersic magnitude and the total magnitude derived from the double or 1.5-component fits for those galaxies that were classified as such, for all bands in \texttt{v05} (all magnitudes limited to segment radii). The scatter plot shows the difference between the two magnitudes against the single S\'ersic magnitude for all bands with the running medians and 1$\sigma$-percentiles overplotted. The top and right panels show the respective marginal distributions. This figure is equivalent to Figure~\ref{fig:magrecovery} for \texttt{v04}.}	
      \label{fig:magrecoveryv05}
\end{center}
\end{figure}

Figure~\ref{fig:magrecoveryv05} shows the last of the \texttt{v05} equivalents to Section~\ref{sec:v04parameterdistributions}, namely the difference between the single S\'ersic magnitudes and the total magnitudes derived from the sum of the bulge and disk fluxes (corresponding \texttt{v04} version in Figure~\ref{fig:magrecovery}). For all bands, these two versions of the total magnitude are consistent within 0.1\,mag for the vast majority of galaxies over the entire magnitude range. There is, however, a slight positive bias especially for the $u, i, J, H$ and $K_s$ bands (see the histogram on the right of Figure~\ref{fig:magrecoveryv05}), i.e. the single S\'ersic magnitudes tend to be slightly brighter than the total magnitudes obtained from adding the bulge and disk fluxes. A hint of this is also visible in Figure~\ref{fig:magrecovery}.%\\ 

\begin{figure}
    \includegraphics[width=1\textwidth]{plots/resultsv04vsv05_R_S}
    \caption{The difference $\Delta$ or quotient Q (for scale parameters) between the \texttt{v05} and \texttt{v04} fits plotted against the \texttt{v04} fits for all single S\'ersic parameters in the $r$-band. Black dots show all fits that had a single data match and were neither skipped nor flagged as outliers in either catalogue, red dots with error bars show the running median and its error in evenly spaced bins and horizontal blue lines indicate no difference between the fits. The numbers in the top left corners of the first row of panels show the median and 1$\sigma$-quantile of the respective distribution in the $y$-direction (which is identical for all panels of a row). Results for the $g$ and $i$ bands are not shown since they are almost indistinguishable from those in the $r$-band.}
\label{fig:resultsv04vsv05_R_S}
\end{figure}

\begin{figure}[t!]
\begin{center}
    \includegraphics[width=0.5\textwidth]{plots/resultsv04vsv05_R_S_SEGRAD}
    \caption{The same as Figure~\ref{fig:resultsv04vsv05_R_S}, but for the three main parameters magnitude, effective radius and S\'ersic index only, truncating the former two to the \texttt{v04} segment radii for both catalogues. The axis scales are the same as in Figure~\ref{fig:resultsv04vsv05_R_S} (central three by three panels).}
\label{fig:resultsv04vsv05_R_S_SEGRAD}
\end{center}
\end{figure}

\begin{figure}[hb!]
\begin{center}
    \includegraphics[width=0.5\textwidth]{plots/resultsv04vsv05_R_P_SEGRAD}
    \caption{The same as Figure~\ref{fig:resultsv04vsv05_R_S}, but for the main parameters of the 1.5-component fits (bulge and disk magnitudes and disk effective radius); all truncated to the \texttt{v04} segment radii for both catalogues (except for the point source magnitude since the point source flux is entirely contained within the segment by definition). The axis scales are the same as those for the corresponding single S\'ersic parameters in Figures~\ref{fig:resultsv04vsv05_R_S} (central three by three panels) and~\ref{fig:resultsv04vsv05_R_S_SEGRAD}. The sample is limited to objects classified as 1.5-component fits in both catalogues in the $r$-band.}
\label{fig:resultsv04vsv05_R_P_SEGRAD}
\end{center}
\end{figure}

\begin{figure}[ht!]
\begin{center}
    \includegraphics[width=0.8\textwidth]{plots/resultsv04vsv05_R_D_SEGRAD}
    \caption{The same as Figure~\ref{fig:resultsv04vsv05_R_S}, but for the main parameters of the double component fits (bulge and disk magnitudes, bulge and disk effective radii and bulge S\'ersic index); with magnitudes and effective radii truncated to the \texttt{v04} segment radii for both catalogues. The axis scales are the same as those for the corresponding single S\'ersic parameters in Figures~\ref{fig:resultsv04vsv05_R_S} (central three by three panels) and~\ref{fig:resultsv04vsv05_R_S_SEGRAD}. The sample is limited to objects classified as double component fits in both catalogues in the $r$-band.}
\label{fig:resultsv04vsv05_R_D_SEGRAD}
\end{center}
\end{figure}

To finish this Section, Figures~\ref{fig:resultsv04vsv05_R_S} to~\ref{fig:resultsv04vsv05_R_D_SEGRAD} provide a direct comparison of the fitted parameters in \texttt{v04} and \texttt{v05} in the $r$-band. For all single S\'ersic parameters, we show the difference between the two fits (\texttt{v05} - \texttt{v04} for position RA and Dec, magnitude $m$, and position angle PA; and \texttt{v05} / \texttt{v04} for the scale parameters effective radius $R_e$, S\'ersic index $n$, and axial ratio $b/a$) against the \texttt{v04} fits (in logarithmic units for scale parameters) in Figure~\ref{fig:resultsv04vsv05_R_S}. Running medians and 1$\sigma$ quantiles are shown as red points with error bars with numerical values indicated in the top left corners of each row. The sample is limited to those objects that had a single data match in the core bands and were neither skipped nor flagged as outlier in either catalogue. The corresponding $g$ and $i$ band plots are nearly indistinguishable with only very slightly increased scatter in the S\'ersic index (1$\sigma$-quantiles of 0.08 for both bands) and position angle (1$\sigma$-quantiles of 0.53 in $g$ and 0.55 in $i$), so we do not show them. 

Galaxy positions in RA and Dec (top two rows of Figure~\ref{fig:resultsv04vsv05_R_S}) are recovered near-perfectly (the $y$-axis range corresponds to 0\farcs01; or 5\,\% of a pixel). The only prominent feature are the three GAMA II equatorial survey regions that are clearly visible in RA. The small gaps in the distribution in Dec at intervals of 1\degr\ (the KiDS tile size) are due to the limitation to objects with a single match, excluding the overlap sample. There are no deviations of RA or Dec as a function of any other parameter (first two rows); nor does any other parameter show systematic trends as a function of RA and Dec (first two columns). Similarly consistent results between \texttt{v04} and \texttt{v05} are obtained for the axial ratio and position angle (bottom two rows; and last two columns), which agree to within $\sim$\,2\,\% and half a degree respectively. 

The three main parameters (central three by three panels in Figure~\ref{fig:resultsv04vsv05_R_S}) show more deviations and also systematic trends: magnitudes are on average 0.01\,mag fainter in \texttt{v05} than in \texttt{v04}; while effective radii and S\'ersic indices are 4\,\% and 3\,\% smaller respectively. Bright and large objects with high S\'ersic indices are particularly strongly affected. These differences are caused by the larger segments in \texttt{v05} as explained in Sections~\ref{sec:postprocessing} and~\ref{sec:segchoices}. We also investigate this effect further in Section~\ref{sec:comparelee}, where we compare the \texttt{v04} results to previous work. 

Truncating the effective radius and magnitude to the \texttt{v04} segment radii (the smaller segments) for both catalogue versions removes the systematic trends in those parameters as shown in Figure~\ref{fig:resultsv04vsv05_R_S_SEGRAD}. The S\'ersic index remains different between the catalogue versions since it cannot be corrected for different segment sizes. For reasons outlined in Section~\ref{sec:v05segmentation}, we still opted for slightly larger segments in \texttt{v05}. 


For completeness, Figures~\ref{fig:resultsv04vsv05_R_P_SEGRAD} and~\ref{fig:resultsv04vsv05_R_D_SEGRAD} show the analogous plots for the 1.5- and double component fits, for objects that were classified as such in both catalogues. We show only the three main S\'ersic parameters (for bulges and disks where relevant) and limit magnitudes and effective radii to segment radii. The sample sizes are much smaller (especially in the 1.5-component plot), and the scatter generally increases, in particular for the bulge parameters. The overall agreement between the \texttt{v04} and \texttt{v05} parameters remains high, with systematic trends mostly limited to S\'ersic indices. The average offsets in S\'ersic index decreases to 2\,\% (from 4\,\% for single S\'ersic fits) and the offset in the disk effective radius disappears entirely. This is most likely due to the generally better model fit of the double component model, since the differences only arise when the model cannot represent the data adequately (cf. Section~\ref{sec:postprocessing}).



\subsection{Galaxy and component colours}
\label{sec:colours}

\begin{figure}[t!]
    \includegraphics[width=\textwidth]{plots/colourplots}
    \caption{\textbf{Left panel:} The Galactic extinction-corrected $g-r$ colour distributions (limited to segment radii) for galaxies and their components. The colour coding of the lines is the same as for Figures~\ref{fig:resultshists} and~\ref{fig:resultshistsv05}, although with a few additions: The solid black line shows single S\'ersic fits for all galaxies with \texttt{NCOMP}\,>\,0 in the joint model selection; the thinner dark red and light blue solid lines split this sample into those with $n$\,>\,2.5 and $n$\,<\,2.5 in the $r$-band. The solid yellow line gives the single S\'ersic values for those galaxies which were classified as single component systems, dotted red and dashed blue lines show bulges and disks, respectively, for those objects classified as 1.5 or double component systems (always in the joint model selection). The dot-dashed green histogram gives the total galaxy colour (derived from the addition of bulge and disk flux) for 1.5 and double component systems. The number of objects in each histogram is given in the legend.
\textbf{Right panel:} The Galactic extinction-corrected $g-r$ vs. $M_r$ colour-magnitude diagram (limited to segment radii) for galaxies and their components. The colour coding of the lines is the same as for the left panel, although note we do not show the single S\'ersic contours here for clarity. Contours include 10, 25, 50, 75 and 90\,\% of the sample.}	
    \label{fig:colourplots}
\end{figure}

For the study of colours, we focus on the core bands since those are the deepest exposures and most directly comparable in our analysis. Additionally, they benefit from the joint $gri$ model selection, resulting in a relatively high number of objects with reliable bulge and disk magnitudes. To maximise similarity in terms of depth and seeing, we use $g-r$ colours and $r$-band absolute magnitudes, $M_r$. Results for $g-i$ and/or $M_i$ are qualitatively similar, albeit a bit more noisy. Using other bands reduces the number of available components considerably and further increases the scatter (which is already substantial for $g-r$ as we will see below). Following \citet{Casura2022}, we use \texttt{v04} of the \texttt{BDDecomp} DMU. Using \texttt{v05} instead does not change the results. 

The left panel of Figure~\ref{fig:colourplots} shows the distribution of $g-r$ colours for galaxies and their components. The colours are corrected for Galactic extinction, but not for dust attenuation in the emitting galaxy. The Galactic extinction was obtained from \texttt{v03} of the \texttt{GalacticExtinction} catalogue accompanying the equatorial input catalogue on the GAMA database.

The solid black line shows the colour distribution for all single S\'ersic fits that were not classified as outliers in the joint model selection (\texttt{NCOMP}\,>\,0). It is clearly bimodal, with redder colours typically belonging to higher S\'ersic index objects as indicated by the thinner dark red and light blue lines splitting the distribution at $n$\,=\,2.5 (in the $r$-band). Not entirely surprisingly (given the distribution of S\'ersic indices in Figure~\ref{fig:resultshists}), the distribution of single S\'ersic objects actually classified as such (\texttt{NCOMP}\,=\,1, solid yellow line) mostly follows the distribution of low S\'ersic index objects; while the high S\'ersic index objects tend to be classified as double component systems. For the latter, we show total colours with a dash-dotted green line, bulge colours with a dotted red and disk colours with a dashed blue line. As expected, bulges tend to be redder than disks, although the scatter is large. %\\

The right panel of Figure~\ref{fig:colourplots} shows the corresponding colour-magnitude diagram. Colours and absolute magnitudes are both corrected for Galactic extinction but not for dust at\-tenua\-tion in the emitting galaxy. The absolute magnitude was calculated using the distance modulus provided in v14 of the \texttt{DistancesFrames} catalogue from the GAMA database which we also used to obtain redshifts for the sample selection. 

The grey density plot in the background shows the single S\'ersic fits for all non-outlier (\texttt{NCOMP}\,>\,0) galaxies, corresponding to the black line in the left panel of Figure~\ref{fig:colourplots}. The bimodality of the distribution is even clearer here, with the red sequence and blue cloud being well-separated. The green contours indicate the part of the sample that was classified as 1.5 or double component object\footnote{To be precise, the green contours were derived by adding the respective bulge and disk fluxes of the 1.5 or double component objects (for consistency with the bulge and disk contours), while the grey density plot is based on the single S\'ersic fits (for robustness at low magnitudes). These two versions of the total galaxy magnitude are generally very similar as evidenced by Figure~\ref{fig:magrecovery}.}: as expected, this is concentrated towards the bright end of the galaxy distribution and hence encompasses mostly galaxies located in the red sequence. Correspondingly, bulges and disks are both relatively red, with bulges on average slightly redder than the total galaxies and disks slightly bluer (while both components - obviously - are fainter than the total galaxy). However, both components show a large scatter and overlap with each other: both faint blue bulges exist as well as bright red disks. 

A detailed study of component colours and the different populations in the right panel of Figure~\ref{fig:colourplots} is beyond the scope of this thesis. However, we note that the total galaxy colours show much less scatter; indicating that the scatter results from a different splitting of the light into bulge and disk components in the $g$ and $r$ bands, while the total amount of light is well-constrained (cf. also Figure~\ref{fig:magrecovery}). A further brief investigation into extreme systems (blue bulges with red disks and also excessively red bulges with very blue disks) suggests that they are caused by a variety of remaining uncertainties in our analysis, e.g. swapped components in one of the two bands (Section~\ref{sec:galaxyfitting}), small faint bulges that are barely detected in the $r$-band and missed in $g$, the ``bulge" component dominating both small and large radii in one of the two bands (cf. Section~\ref{sec:modelselectioncaveats}) or failures in the flagging of bad fits, all combined with model selection uncertainties and the necessity of joint model selection to compromise between the bands. While each of these processes by itself only affects a small number of galaxies, in sum across both bands they do reach the 10-20\,\% level. Still, on average our colours do follow the expected trends, as we show in Section~\ref{sec:prevcols} with an overview of similar studies in the literature. We will study the colours of galaxies and their components in more detail in forthcoming work, also including the other bands ($uZYJHK_s$) and taking full account of inclination effects due to dust in the emitting galaxies \citep[see, e.g.][]{Driver2008}. We will then also assess trends in other parameters, such as the component effective radii, with wavelength; and use these to constrain the nature and distribution of dust in galaxy disks. 





\section{Catalogue limitations}
\label{sec:cataloguelimitations}

We finish this chapter by pointing out a few limitations of our results that users of the catalogue should be aware of. Most of these have been discussed before and are only summarised here, with references to the relevant sections. 

\subsection{Model limitations}

All of our models are axially symmetric and monotonically decreasing in intensity from the centre. We are unable to capture asymmetries such as spiral arms, offset bulges, tidal tails, mergers, star-forming regions etc.; or disk features such as rings, bumps, truncations or flares. If such features are present in the data, they may bias or skew the model parameters. We also remind the reader that when we talk about ``bulges", what we really mean are the central components. This could be a classical bulge, a pseudo-bulge, an AGN, a bar, or any combination (sometimes resulting in the model trying to fit a mixture between e.g. a bar and a bulge). We make no attempt to distinguish between these cases. 

\subsection{Model selection caveats}
\label{sec:modelselectioncaveats}

Model selection is accurate to >\,90\,\% compared to what could be achieved
by visual classification (Section~\ref{sec:postprocessing}). However, it is important to note that our aim in the model selection is to determine which one of our three models is most appropriate to use for the given data; and not how many physically distinct components an object consists of. The reason for this is that for a given galaxy, the data quality will strongly influence how many fitting parameters can be meaningfully constrained and using more model parameters will inevitably overfit the data and lead to unphysical results. Hence, even in the joint model selection, we base our visual classification on the fit and residuals in individual bands (which is what we fit to), rather than e.g. colour images. Due to the different depths and resolutions of the bands, it is common for the same galaxy to be classified as double component in one band, but single component in another (cf. also discussions in Sections~\ref{sec:manualcalibrationchanges} and~\ref{sec:modelseldiffs}).  

In an attempt to make fitting parameters more directly comparable across bands, we introduced the joint $gri$ model selection (Section~\ref{sec:postprocessing}), yet this is necessarily a compromise between the different bands. For example, we lose bulges that are resolved in the $r$-band but not in $g$ and $i$ due to the larger PSFs; or there may be some ill-constrained $i$-band fitting parameters for an extended low-surface brightness disk that is visible in $r$ and $g$ but not in the shallower $i$-band image. There are also more skipped fits and outliers in the joint model selection than in the band-specific ones because all objects that are skipped or flagged in at least one of the three bands are skipped or flagged in the joint model selection. These problems become significantly worse for the nine-band joint model selection encompassing all of $ugriZYJHK_s$ due to the larger range in depth and seeing for individual bands.%\\

Irrespective of the result of the model selection, we provide all fitted parameters of all models in the catalogue (along with the postage stamps of all fits and a flag indicating the preferred model). This allows users to perform their own selection if desired; but also requires care as not all provided parameters will be meaningful. While single S\'ersic fits to double component objects are mostly reasonable; double component fits to true single component galaxies will have unconstrained and potentially unphysical parameters for at least one of the components.

\begin{figure}
\begin{center}
	\includegraphics[width=0.8\textwidth]{plots/examplefithighbt}
    \caption{The double component fit to galaxy 549706, classified as double component object but with a very high B/T ratio of 0.71 in the KiDS $r$-band. Panels are the same as those in Figure~\ref{fig:examplefit}.} 
    \label{fig:examplefithighbt}
\end{center}
\end{figure}

We are also aware of a population of objects that are classified as double component fits but have the bulge component dominating both the centre and the outskirts, with the disk only dominating at intermediate radii or even staying ``below" the bulge at all radii. We believe these are essentially single component systems that do not follow a S\'ersic law (e.g. S\'ersic index would be higher at centre than outskirts); and so the freedom of the disk is used to offset this. This population is easily identifyable by the high bulge-to-total ratio (B/T\,$\gtrsim$\,0.6 or 0.7). The single S\'ersic fits may be more appropriate to use in these cases \citep[see also the discussion of this issue in][]{Allen2006}. An example is shown in Figure~\ref{fig:examplefithighbt}. 



\subsection{Drawbacks of tight fitting segments}

As detailed in Section~\ref{sec:postprocessing}, we use relatively tight segments around the galaxies for fitting, which results in the best possible fit of the inner regions of the galaxy but can lead to large, unphysical wings. Hence we recommend using only integrated properties, i.e. the summed flux/magnitude within the region that was fitted and the corresponding effective radii and bulge-to-total ratios as given by the corresponding \texttt{*\_SEGRAD} properties in the catalogue. For comparisons to other catalogues using larger fitting segments, their profiles should also be appropriately truncated (see details in Sections~\ref{sec:postprocessing}, ~\ref{sec:segchoices} and \ref{sec:comparelee}). 

\subsection{Sources of systematic uncertainties}

We provide errors for each fitted parameter in the catalogue including our best estimate of systematic uncertainties taken from Table~\ref{tab:errorunderestimate}. However, we do not apply the (small) bias corrections given in the same table, since they are only applicable to large random samples of our galaxies and not to individual objects. In addition, we would like to point out that the systematic errors were estimated from single S\'ersic $r$-band fits. We expect that individual components as well as other bands are affected by similar systematics, but we did not test for this. Also, there are some systematic uncertainties that we do not account for in our simulations, most obviously galaxy features that cannot be captured by our models. For these reasons, the given errors should still be considered as lower limits of the true errors. 

\subsection{GAMA-KiDS RA/Dec offset}

We observed an average offset between the input and output (fitted) positions of galaxies in both RA and Dec of approx. 0.4\,pix (0$\farcs$08). This is due to an offset between the GAMA (SDSS) and KiDS positions; the same offset can
be seen when comparing the KiDS source catalogue with the Gaia catalogue; see also figure~15 in \citet{Kuijken2019}. We correct for this during the outlier rejection, but give the original (uncorrected) fitted values for position otherwise. 


\subsection{Completeness limits}

Due to our sample selection (Section~\ref{sec:sampleselection}), our spectroscopic completeness is 100\,\% and even the faintest objects in our sample are well-resolved and bright enough to allow for robust single S\'ersic fits in the core bands. However, this is not the case for the KiDS $u$ and the longest wavelength VIKING bands, as discussed in Section~\ref{sec:outlierstats}: in the shallowest bands ($K_s$ and $u$), we lose approximately 25\,\% of our sample entirely; and robust double component fits can only be obtained for a minority of objects. Also, there is a systemic limit to the component magnitude in that the samples of bulges and disks with magnitudes fainter than the GAMA limit ($r$\,<\,19.8\,mag) are incomplete. For example, a bulge with a magnitude of 22\,mag in the $r$-band will only be contained in our sample if the corresponding disk is bright enough such that the total magnitude is below 19.8\,mag. Hence, the sample of bulges with 22\,mag is incomplete. This applies almost exclusively to the faint bulges from the 1.5-component fits as can be seen in the first column of Figures~\ref{fig:resultshists} and~\ref{fig:resultshistsv05}.

