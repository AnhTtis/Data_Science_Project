\chapter{Summary, conclusion and outlook}
\label{chap:conclusion}

In this thesis we presented our pipeline for the single S\'ersic fits and bulge-disk decompositions of 13096 galaxies at redshifts $z$\,<\,0.08 in the GAMA II equatorial survey regions in the KiDS $u$, $g$, $r$, $i$ and the VIKING $Z, Y, J, H, K_s$ bands. The galaxy modelling is done using \texttt{ProFit}, the Bayesian two-dimensional surface profile fitting code of \citet{Robotham2017}, fitting three models to each galaxy:
\begin{enumerate} 
\item{a single S\'ersic component,}
\item{a two-component model consisting of a S\'ersic bulge plus exponential disk and }
\item{a two-component model consisting of a point source bulge plus exponential disk (for unresolved bulges).}
\end{enumerate} 
The preparatory work (image segmentation, background subtraction and obtaining initial parameter guesses) is carried out using the sister package \texttt{ProFound} \citep{Robotham2018}; with the PSF estimated by fitting nearby stars using a combination of \texttt{ProFound} and \texttt{ProFit}. Segmentation maps are defined on joint $gri$-images, while the remaining analysis is performed individually in each band except for the model selection, for which we offer both a per-band and a joint version. The analysis is fully automated and self-contained with no dependency on additional tools. 

In addition to the galaxy fitting, we performed a number of post-processing steps including the flagging of bad fits and model selection. An overview of the number of galaxies successfully fitted in each band as well as the number classified in each category is given in Tables~\ref{tab:results} and~\ref{tab:resultsv05} as well as Figure~\ref{fig:ncompstats}. For our planned applications of the catalogue, which involves the statistical study of dust attenuation effects, we need fits that are most directly comparable to each other. Hence, we choose to model a maximum of two components for each galaxy even if more features may be present; and focus on achieving good fits in the high signal-to-noise regions of the galaxies by using relatively small segments for fitting. Consequently, we recommend using truncated magnitudes and effective radii for all analyses instead of the S\'ersic values which are extrapolated to infinity. The quality of the fits was ensured by visual inspection, comparing to previous works \citep{Kelvin2012, Lange2015}, studying independent fits of galaxies in the overlap regions of KiDS tiles and bespoke simulations. The latter two were also used for a detailed analysis of how systematic uncertainties affect our fits. %\\

We found that the combination of \texttt{ProFound} and \texttt{ProFit} is well-suited to the automated analysis of large datasets. The fully Bayesian MCMC treatment enabled by \texttt{ProFit} is able to overcome the main shortcomings of traditionally used downhill-gradient based optimisers, namely their susceptibility to initial guesses and their inability to easily derive realistic error estimates. The watershed deblending algorithm used by \texttt{ProFound} is less prone to catastrophic segmentation failures and allows us to extract more complex object shapes than other commonly used algorithms based on elliptical apertures; while still preserving the total flux well. With its wealth of utility functions, it not only facilitates the robust segmentation of large sets of images but also provides sky background estimates and reasonable initial guesses for the MCMC fitting. 

These characteristics, in combination with our own routines for quality assurance, led to results that are robust across a variety of galaxy types and image qualities and in reasonable agreement with previous studies given the different data, code and focus of the analysis. The outlier rejection routine efficiently identifies objects for which none of our models is appropriate such as irregular galaxies or those compromised by masked areas. Model selection is based on a $\Delta$DIC cut and accurate to >\,90\,\% compared to what could be achieved by visual inspection. There is a minimal bias in the fitted magnitude, effective radius and S\'ersic index of approximately 0.01\,mag, 1\,\% and 1\,\% respectively (on average across the full sample) caused by excess flux from nearby other objects. The errors obtained from the MCMC chains are underestimated with respect to the true errors by factors of typically between 2 and 3 (see Table~\ref{tab:errorunderestimate}) and can easily be corrected for statistically large samples of galaxies.%\\


All results are integrated into the GAMA database as part of the \texttt{BDDecomp} DMU. The DMU consists of a number of catalogues giving the results of the preparatory work, the 2D surface brightness distribution fits and the post-processing of all 13096 galaxies in our full sample ($z$\,<\,0.08 in the GAMA II equatorial survey regions) in all bands; with additional diagnostic plots and all fit inputs available on the GAMA file server (see Section~\ref{sec:bddecompdmu} for details). So far, four DMU versions have been released, while \texttt{v05} will be made available alongside the publication of this thesis. The full DMU is currently available to GAMA team members with a version restricted to SAMI galaxies available to the SAMI team. It will be made publicly available in one of the forthcoming GAMA data releases. Readers interested in using (parts of) the catalogue before it is publicly released are encouraged to contact the authors to explore the possibilities for a collaboration\footnote{\url{http://www.gama-survey.org/collaborate/}}.

Both the GAMA and the SAMI teams have actively made use of the catalogue versions already released, with many more studies in progress and planned (see Chapter~\ref{chap:intro}). We therefore detailed not only the final version, but also different stages during pipeline development, including quality control steps of the preparatory work, in Chapter~\ref{chap:pipeline}. Since \texttt{v05} of the catalogue has not been previously published, a particular focus is placed on ensuring the reliability of those results by comparing it to \texttt{v04} (Chapter~\ref{chap:results}). The latter has benefitted from an extensive quality control detailed in Chapter~\ref{chap:QC}. %\\

In addition to the projects of collaborators mentioned above, we have many own plans and ideas for scientific analyses on the basis of our bulge-disk decomposition results. These include studying bulge and disk colours and trends of structural parameters with wavelength as well as deriving component stellar masses with the aims of investigating the stellar mass functions, stellar mass-to-light ratios and scaling relations such as the size-mass relation for individual components as a function of wavelength. By comparison with dust radiative transfer models, we will then also be able to constrain the nature and distribution of dust in galaxy disks, thereby contributing to the understanding of systematic uncertainties affecting studies of galaxy structure, formation and evolution, which in turn are vital for our understanding of the universe as a whole (see Section~\ref{sec:scienceaims}). 

On the technical side, further development of the pipeline foresees a number of minor improvements in the near future, for example considering the uncertainty of the astrometric solution for the joint fit, simultaneously fitting nearby sources and improving our treatment of the VIKING data by accounting for the slight differences in pixel size between frames, using the provided confidence maps and generating bright star masks. In the longer term, we will exploit the relatively new multi-band fitting functionality of \texttt{ProFit} - or better yet the new package \texttt{ProFuse} - to achieve simultaneous fits across all wavelength bands with physically motivated variations of structural parameters. Improvements to the preparatory work pipeline foresee the creation of a joint segmentation map including all bands, eliminating all manually calibrated tuning parameters and simultaneously fitting all suitable stars for more robust PSF estimates. In the post-processing, the main focus of development is on minimising manual intervention - especially the re-calibration of model selection - to allow easy scaling of the code to larger samples of galaxies from potentially different bands and datasets.%\\

To summarise, we obtained a catalogue of robust structural parameters for the components of a sample of 13096 nearby GAMA galaxies while at the same time contributing to the advancement of image analysis, surface brightness fitting and post-processing routines for quality assurance in the context of automated large-scale bulge-disk decomposition studies. The further development of such methods and new approaches is vital to fully exploit the data of future sky surveys that will provide multi-wavelength imaging for millions of galaxies at unprecedented depth and resolution. The resulting measured parameters in turn are crucial to test and improve theoretical models and simulations and ultimately understand the formation, structure, composition and evolution of our universe.



\newpage
\thispagestyle{plain}
\section*{Acknowledgements}

% GAMA acknowledgement
GAMA is a joint European-Australasian project based around a spectroscopic campaign using the Anglo-Australian Telescope. The GAMA input catalogue is based on data taken from the Sloan Digital Sky Survey and the UKIRT Infrared Deep Sky Survey. Complementary imaging of the GAMA regions is being obtained by a number of independent survey programmes including GALEX MIS, VST KiDS, VISTA VIKING, WISE, Herschel-ATLAS, GMRT and ASKAP providing UV to radio coverage. GAMA is funded by the STFC (UK), the ARC (Australia), the AAO, and the participating institutions. The GAMA website is \url{http://www.gama-survey.org/}.

% KiDS DR4.0 need to cite Kuijken 2019 A&A and cite the following
Based on observations made with ESO Telescopes at the La Silla Paranal Observatory under programme IDs 177.A-3016, 177.A-3017, 177.A-3018 and 179.A-2004, and on data products produced by the KiDS consortium. The KiDS production team acknowledges support from: Deutsche Forschungsgemeinschaft, ERC, NOVA and NWO-M grants; Target; the University of Padova, and the University Federico II (Naples).

% VIKING DR4 needs this (from Edge2020)
This publication has made use of data from the VIKING survey from VISTA at the ESO Paranal Observatory, programme ID 179.A-2004. Data processing has been contributed by the VISTA Data Flow System at CASU, Cambridge and WFAU, Edinburgh.

% and for the climate
The total computing time for all runs and test runs performed in the context of this thesis sums to approximately 676\,700 hours on CPU. With an electricity usage of 12\,W per CPU, our total energy consumption for the project is 8120\,kWh. Assuming an average CO$_2$ equivalent of 500\,g\,/\,kWh (corresponding to the average of the German energy mix during the period of the project\footnote{\url{https://www.umweltbundesamt.de/presse/pressemitteilungen/bilanz-2019-co2-emissionen-pro-kilowattstunde-strom}}), results in 4\,tons of CO$_2$ emitted by this PhD thesis. This is comparable to the worldwide average of CO$_2$ emissions per person per year.\footnote{\url{https://de.statista.com/statistik/daten/studie/167877/umfrage/co-emissionen-nach-laendern-je-einwohner/}} It is still a lower limit since it does not count any non-automated calculations nor failed runs; and completely ignores other activities related to the project, such as travelling.  

\newpage
\thispagestyle{plain}
\section*{Thanks}

First of all, I would like to thank my supervisor Jochen Liske for going through this project with me start to end. Jochen is a very skilled and enthusiastic scientist who is never short of ideas and was a great source of inspiration for me. On the more technical side, much support came from Aaron Robotham. I am very grateful to him for always being available and helping me out countless times during the course of my PhD. Furthermore, let me thank Robi Banerjee, Peter Hauschildt and Dieter Horns for being part of my defense committee. 

It was great to be welcomed into the GAMA team, a very friendly and supportive group of scientists. Working with the SAMI and KiDS teams was also a pleasure, with members of both collaborations always happy to help out. My stays at ICRAR will remain in good memory due to the hospitality and friendlyness of everybody I met there. While it is impossible to name everyone who contributed to the success of this project, a few people who come to my mind - apart from Jochen and Aaron of course - are Dan Taranu, Angus Wright, Jarkko Laine, Robin Cook, Hosein Hashemizadeh, Stefania Barsanti, Ben Henderson, Sree Oh, Simon Driver, Boris H{\"au{\ss}ler, Benne Holwerda, Ned Taylor, Amanda Moffett, Konrad Kuijken, Luke Davies, Michelle Cluver and Christos Georgiou. 

A special thanks goes to Denis Wittor and Volker Heesen for proof-reading this thesis and suggesting many improvements. Janis Kummer was so kind to provide his thesis to use as a template and J\"org Knoche offered much needed emergency IT-support on several occassions. Together with all other current and former members of the Hamburger Sternwarte - many of whom I count to my friends - they made my PhD time a lot of fun and one of the best periods of my life. 

Last but not least, my thanks goes to my family and friends - especially my parents, but also my sister, grandparents, some of the wider family and many close friends - all of whom always gave me the feeling that I could achieve anything if only I wanted to. I am extremely lucky to have Lars, the best husband that I could have wished for and an amazing dad to our son Tim. Without his unlimited support - and the happy nature of Tim - I could have never finished this project. 

