\chapter{Introduction}
\label{chap:intro}
\pagenumbering{arabic} % needed in first chapter


Most of the information we obtain about the universe is in the form of electromagnetic radiation. The majority of electromagnetic radiation is emitted by galaxies, which are one of the most important constituents of the universe. Studying the galaxy population, their structure and evolution is therefore one of the main pathways to understand our universe \citep[e.g.][]{ZeilikandGregory, Hammerbook, Kembhavi2020, Haeussler2022}. 

Galaxies are complex objects consisting mainly of stars, gas, dust, dark matter and a supermassive black hole, sometimes including an active galactic nucleus (AGN). They differ greatly in their physical properties including their size, mass and its distribution in different components, their age, composition (relative amounts of gas and dust, types of stars contained, abundances and distribution of different elements, dark matter component), star formation activity, assembly and evolutionary history. Most of those properties can only be inferred indirectly and suffer from many uncertainties \citep[e.g.][see also Section~\ref{sec:galaxystructure} for an overview over different galaxy structures]{ZeilikandGregory, Vika2013, Hammerbook}.

The picture becomes even more complex when considering the interactions between the different constituents of galaxies as well as their evolution over time. For example, stars can form from gas via collapse and then feed back into the environment from which they formed via radiation and mass-loss processes \citep[e.g.][and references therein]{Kelvin2012, Driver2016}. A significant fraction of the radiation emitted by stars is absorbed by dust and re-emitted at longer wavelengths \citep{Driver2007, Popescu2011}. The visual appearance of a galaxy at any given point in time is therefore not only determined by the relative amounts of stars, gas and dust it contains, but also their relative distributions and the galaxy geometry and orientation with respect to the line of sight. Further, most galaxies are not isolated systems but exhibit a range of interactions with their environments. Exactly how all of these aspects are related and lead to the formation and evolution of a variety of galaxies is still not fully understood, with much debate on the relative importance of different processes such as mergers, feedback or secular processes \citep{Driver2009, Lange2016, Hammerbook, Nedkova2021}. %\\

To constrain the formation and evolution of galaxies, observations are key. The two main branches of observations of galaxies are photometry and spectroscopy. While the latter has the advantage of providing the intensity as a function of wavelength, the former offers larger samples at higher spatial resolution and can also include at least some colour information via multi-wavelength observations in different broad-band filters \citep{ZeilikandGregory, Vika2014, Haeussler2022}. For both methods, the optical part of the electromagnetic spectrum is of special importance. In part, this is because optical telescopes (and the analysis methods of corresponding data) have reached greater maturity than their counterparts at other wavelengths due to their much longer history. However, there is also physical motivation for studying galaxies at (near-)visible wavelengths since this is the range in which most stars emit most of their radiation \citep{ZeilikandGregory, Driver2012, Vika2013}.

Along with technological advances, observations of galaxies have become more sophisticated over time \citep{Driver2009, Haeussler2022}. To fully exploit the increased data quality and quantity of modern galaxy surveys and maximise their science return, corresponding advances in analysis methods are needed \citep{Hammerbook, Robotham2017, Robotham2022, Haeussler2022}. Quantitative measurements of the sizes and structures of the galaxy population are particularly important to constrain theory and simulations, which - in turn - constrain the formation and evolution of galaxies and the universe as a whole \citep[e.g.][and see also Section~\ref{sec:SBmodelling} for a more comprehensive overview of galaxy surveys and corresponding analysis methods]{Driver2009, Simard2011, Kelvin2012, Nedkova2021}.%\\

This thesis presents the analysis of the photometric images of 13096 galaxies from the Galaxy and Mass Assembly \citep[GAMA;][]{Driver2009} survey in nine broad-band optical and near-infrared (NIR) filters. We fit several models to the surface brightness distribution of each galaxy in each band, in order to obtain information about their internal stellar structure. The resulting catalogue can be used to analyse the properties of galaxies as a function of wavelength, morphology or other parameters and serves as a basis for the comparison to theory and simulations. At the same time, we have contributed to the advancement of software, methodologies and analysis techniques in the field of large automated bulge-disk decomposition studies of galaxies.

We provide more details on the aims and achievements of this thesis at the end of Section~\ref{sec:backgroundandaims}, after introducing the more general context. The remainder of this chapter then presents the data (Section~\ref{sec:dataandsample}) as well as the analysis methods and code packages used (Section~\ref{sec:methodsandcode}), including a discussion of the distinguishing features of this study compared to previous work. Chapter~\ref{chap:pipeline} details the pipeline we developed for the bulge-disk decomposition of our sample of galaxies, including preparatory work and post-processing; and the evolution of all steps over time. The main results of this pipeline are then shown in Chapter~\ref{chap:results}. Chapter~\ref{chap:QC} focuses on the quality control of the fits by comparison to previous work and a detailed investigation into systematic uncertainties and biases. We conclude with a summary and outlook in Chapter~\ref{chap:conclusion}. We assume a standard cosmology of $H_0$\,=\,70\,km\,s$^{-1}$\,Mpc$^{-1}$, $\Omega_m$\,=\,0.3 and $\Omega_{\lambda}$\,=\,0.7 throughout. 

Parts of this work have already been submitted for publication in \citet{Casura2022}. In particular, Section~\ref{sec:SBmodelling}, parts of Section~\ref{sec:scienceaims}, most of Sections~\ref{sec:gama}, \ref{sec:kids}, \ref{sec:sampleselection}, \ref{sec:profit} and \ref{sec:profound}, Section~\ref{sec:pipelineoverview}, parts of Section~\ref{sec:statsoverview}, Sections~\ref{sec:v04parameterdistributions}, \ref{sec:colours} and~\ref{sec:cataloguelimitations}, Chapter~\ref{chap:QC} and the majority of Chapter~\ref{chap:conclusion} are either heavily based on or directly taken from \citet{Casura2022} with only minor modifications. This includes Figures~\ref{fig:exampleseg}, \ref{fig:examplefit}, \ref{fig:tightseg}, \ref{fig:examplepsfoutput} and~\ref{fig:examplefit-2}, the top panel of Figure~\ref{fig:ncompstats}, Figures~\ref{fig:resultshists}, \ref{fig:BThist}, \ref{fig:magrecovery}, \ref{fig:colourplots}, \ref{fig:examplefithighbt}, \ref{fig:colourvskennedy}, \ref{fig:sizemass} and \ref{fig:comparelee}, the right panels of Figures~\ref{fig:compareoverlap} and~\ref{fig:comparesimulations}, Figures~\ref{fig:examplebadsim}, \ref{fig:exampleoutliersim}, \ref{fig:exampledoublesim} and~\ref{fig:differrnorm} as well as Tables~\ref{tab:fittingparameters}, \ref{tab:singlefitlimits}, \ref{tab:modelselconfusionr}, \ref{tab:modelselconfusiongijoint}, \ref{tab:results}, \ref{tab:stats} and \ref{tab:errorunderestimate}. The remaining work of this thesis has not been presented elsewhere to date. Where we have taken large amounts of content from \citet{Casura2022}, we also indicate this at the beginning of the corresponding chapter or section. 

The written work presented here is accompanied by several data products which are bundled in the \texttt{BDDecomp} data management unit (DMU) on the GAMA team database\footnote{\url{http://www.gama-survey.org/db/}}. It includes several catalogues containing the results of the preparatory work, the galaxy fitting and the post-processing as well as a detailed description of all processing steps and the columns contained in each catalogue. Various diagnostic plots for each galaxy fit and all inputs used for the fitting (i.e. the outputs of the preparatory work) are stored on the GAMA file server. To date, there have been four releases of the \texttt{BDDecomp} DMU (\texttt{v01} to \texttt{v04}), with the fifth (\texttt{v05}) to be published alongside this thesis. More information on the \texttt{BDDecomp} DMU and its different releases is given in Section~\ref{sec:bddecompdmu}. In addition, all of the results stored on the GAMA database as well as many test runs and the corresponding code can be found on the local machines at Hamburg observatory, specifically on \texttt{/hs/fs11/data/gama/profit/} and the corresponding directories on the \texttt{fs12}, \texttt{fs13} and \texttt{fs14} machines. The ``readme"-files in those directories provide orientation. The (publicly available) KiDS and (preprocessed) VIKING data are also on the \texttt{fs11} machine, in the \texttt{/hs/fs11/data/gama/data/imaging/} directory. 


\section{Background and aims}
\label{sec:backgroundandaims}

In this section, we explain the context and motivation for the current work. We begin with a more detailed description of the different types of galaxies and their structure (Section~\ref{sec:galaxystructure}), followed by a brief overview on galaxy surface brightness modelling including available tools and common practices in the field (Section~\ref{sec:SBmodelling}). This provides the background for the aims of this thesis, which are described in Section~\ref{sec:scienceaims}. Most of Section~\ref{sec:SBmodelling} and parts of Section~\ref{sec:scienceaims} are taken from the introduction of \citet{Casura2022}.

\subsection{The structure of galaxies}
\label{sec:galaxystructure}

Here, we give a brief overview over the different types of present-day (low $z$) galaxies and their main properties, mostly following the discussion in \citet{ZeilikandGregory}. We note that the picture drawn here is somewhat simplified and galaxies are more complex in detail, with many exceptions to the rule. 

Traditionally, three main types of galaxies are distinguished according to their visual appearance: elliptical galaxies, spiral galaxies and irregular galaxies. These are represented in the Hubble sequence introduced by \citet{Hubble1926} with a schematic shown in Figure~\ref{fig:hubblesequence}. 

\begin{figure}
  \begin{center}
	\includegraphics[width=0.8\textwidth]{plots/Hubblesequence.png}
    \caption{The Hubble sequence. Elliptical galaxies are labelled $E$ and shown in the left part of the diagram, with increasing numbers referring to increasing ellipticities. Spiral galaxies are shown on the right, split into ``ordinary" $S$ (top) and barred $SB$ (bottom) spirals. Letters $a$ to $c$ represent the transition from ``early-types" to ``late-types", characterised by an increased dominance of the disk relative to the bulge combined with less tightly wound spiral arms. Image credit: Ville Koistinen, Wikimedia Commons [\url{https://commons.wikimedia.org/wiki/File:Hubble_sequence_photo.png}].} 
    \label{fig:hubblesequence}
  \end{center}
\end{figure}

Elliptical galaxies have a smooth appearance with near to no substructure, gas or dust. Their stellar population is relatively old with little ongoing star formation such that they appear reddish in colour. They are pressure-supported with random stellar motions and typically follow a \citet{deVaucouleurs1948} radial profile, i.e. the intensity $I$ drops approximately as $10^{r^{-1/4}}$ with $r$ being the radius from the centre of the galaxy. 

Spiral galaxies show more structure than elliptical galaxies. They classically have two main stellar components: a bulge and a disk. The bulge sits at the centre and is similar to an elliptical galaxy in its properties, due to which bulges and ellipticals are often classed together and named spheroids. The disk is flat and rotationally supported, contains gas and dust and has a younger stellar population with ongoing star formation and bluer colours. It typically follows an exponential decrease in intensity from the centre, although with local variations due to substructure like spiral arms. 

As can be seen in Figure~\ref{fig:hubblesequence}, spiral galaxies are subdivided into two branches according to whether or not they have a central bar. The bar is disk-like in its stellar composition, but, as the name suggests, has an elongated shape. There are also spiral galaxies that host a pseudo-bulge instead of the classical bulge described above. Pseudo-bulges, like classical bulges, appear approximately spherical and are located at the centre of the galaxy. However, pseudo-bulges are younger and bluer in colour than classical bulges and show a less pronounced increase in intensity towards the centre \citep[e.g.][]{Kennedy2016, Lange2016, Haeussler2022}. 

The distinct differences between spheroids and disks can most easily be explained by different formation channels \citep[e.g.][]{Driver2013, Lange2016}. In the two-phase model, spheroids form in a so-called ``hot" mode, whereby they rapidly assemble their mass through collapse and mergers at early times. This leads to (now) old stellar populations with a high concentration at the nucleus and a generally spheroidal, pressure-supported shape. Disks are formed later and more slowly in the ``cold" mode via the accretion of gas regulated by feedback processes. Therefore, disk stellar populations are younger than spheroids, flattened due to the conservation of angular momentum and rotationally supported. Bars and pseudo-bulges can form from disk instabilities. %\\

In between spiral and elliptical galaxies lie S0 or lenticular galaxies. They have both a dominant bulge and disk, but no spiral arms or other prominent substructure. Their colour and composition is intermediate in between that of spiral galaxies and elliptical galaxies; their formation channel is still being debated \citep[e.g.][]{Barsanti2021}. 

The last of the three main categories of galaxies are irregulars, which show no clear symmetry. They tend to be bluer in colour than spiral galaxies and can also contain substructure, gas and dust. They are thought to be the result of interactions and mergers between galaxies, which can disturb the regular morphology and trigger star formation. 

In addition to these categories of giant galaxies, there are dwarf galaxies. Dwarfs can be elliptical or irregular in shape. They are the most numerous type, but contribute only a small fraction to the total stellar mass and light in the universe. They are also difficult to detect due to their faintness.%\\

The relative fractions of galaxies that fall into each category depend strongly on the observational limits. For magnitude-limited surveys, spirals are usually dominant in number, followed by ellipticals since these two classes of galaxies are brightest. In the local volume, however, most galaxies are small irregular galaxies with spirals only contributing around 33\,\% and ellipticals 13\,\% \citep{ZeilikandGregory}. 

For the GAMA survey, \citet{Driver2022} have derived morphological classifications for all galaxies at a redshift of $z$\,<\,0.08. They find that approximately 10\,\% of their sample are elliptical galaxies, 45\,\% bulge-disk systems (i.e. spirals and S0) and another 45\,\% late-type systems with only a single component, including asymmetric systems (irregulars and disk-only systems with no discernible bulge). Since this sample of galaxies closely matches our sample selection (see Section~\ref{sec:sampleselection}), we might expect a very similar distribution of galaxy types for our catalogue. 




\subsection{Galaxy surface brightness modelling}
\label{sec:SBmodelling}


To obtain physical quantities from the observations of galaxies provided by surveys, they need to be modelled. Due to its importance in understanding the history of the universe, the quantitative modelling of galaxy surface brightness distributions has a long history dating back to \citet{deVaucouleurs1948}, \citet{Sersic1963} and even earlier works; see \citet{Graham2013b} for a review of the development of light profile models. While the early works focused on azimuthally averaged galaxy profiling with a single functional form \citep[e.g.][]{Kormendy1977}, modern codes allow users to decompose galaxies into several distinct components (e.g. bulges, disks) and to take into account the full two-dimensional information. To this end, there are many different techniques, methods and code packages, all of which have become increasingly sophisticated as the quality and quantity of available astronomical data have grown. 

Broadly, they can be divided into parametric and non-parametric modelling, as well as one-dimensional and two-dimensional methods. Which of these is most appropriate to use depends on the science case and the available data. This work falls into the regime of large-scale automated analyses of galaxies with often barely resolved components, for which we want to obtain structural parameters that are easily comparable between galaxies. Hence, two-dimensional parametric analysis is most appropriate (see also the discussion in \citealt{Robotham2017} and references therein). 

Examples of such two-dimensional, parametric fitting tools used for large-scale automated analyses include \texttt{GIM2D} \citep{Simard2002}, \texttt{BUDDA} \citep{deSouza2004}, \texttt{GALFIT3} \citep{Peng2010}, \texttt{GALFITM} \citep{Vika2013}, \texttt{IMFIT} \citep{Erwin2015}, \texttt{ProFit} \citep{Robotham2017} and \texttt{PHI} \citep{Argyle2018}. 
% 
Each of these tools comes with its own advantages and disadvantages, which goes to show how difficult the problem of galaxy modelling is, especially when automated for large samples of the very diverse galaxy population. Usually, some form of post-processing is needed to assess the influence of systematic uncertainties, judge the convergence, exclude bad fits and identify the most appropriate model to use for each galaxy. This can be achieved via visual inspection (for small enough samples), logical filters, frequentist statistics such as the $F$-test, Bayesian inference, or similar methods \citep[see, e.g.,][]{Allen2006, Gadotti2009, Simard2011, Vika2014, Meert2015, Lange2016, Mendez-Abreu2017}. 

Despite the associated difficulties (e.g. convergence and quality of fit metrics), many authors have performed two-dimensional surface brightness profile fitting for large numbers of galaxies, modelling the radial light profile as a simple functional form, most often a S\'ersic function \citep[][to name just a few]{Blanton2003, Blanton2005, Barden2005, Trujillo2006, Hyde2009, LaBarbera2010, Kelvin2012, vanderWel2012, Haeussler2013, Shibuya2015, Sanchez-Janssen2016}. 
%
The results of such analyses have been used to derive a number of key relations between different galaxy properties, their formation and evolutionary history, and interactions with the environment. 
% 
For example, many works have studied the distribution of, and relation between, size and mass or luminosity for different galaxy types (split by e.g. S\'ersic index or colour), sometimes including morphology, surface brightness, internal velocity, environment, wavelength, colour, or redshift effects \citep[e.g.][]{Shen2003, Barden2005, Blanton2005, Trujillo2006, Hyde2009, LaBarbera2010, Kelvin2014, vanderWel2014, Lange2015, Shibuya2015, Nedkova2021}.%\\ 
%


With improving data quality of surveys, the fitting of more than one component - i.e. decomposing galaxies - has become more common. While some authors, such as \citet{Gadotti2009}, \citet{Salo2015} or \citet{Gao2017} also account for bars, central point sources, spiral arms or other additional morphological features, most works focus on the bulge and disk. The focus on only two components is especially true when running automated analyses of large samples, since in many cases the data quality is not sufficient to meaningfully constrain more than one or two components, or it would require extensive manual tuning based on visual inspection. From a more physical point of view, the majority of the stellar mass in the local universe resides in ellipticals, disks and classical bulges, with pseudo-bulges and bars only contributing a few percent (\citealt{Gadotti2009}; and see also Section~\ref{sec:galaxystructure}). Hence, for automated analyses it is common practice to fit only two components, where the term ``bulge" is used to describe the central component, irrespective of whether it is a classical bulge, pseudo-bulge, bar, lens, AGN, or a mixture thereof, while ``disk" refers to a more extended component with typically lower surface brightness and potential additional structure such as spiral arms, breaks, flares or rings. 
% 
 
Examples of large bulge-disk decomposition studies include \citet{Simard2002, Simard2011, Allen2006, Benson2007, Gadotti2009, Lackner2012, Fernandez-Lorenzo2014, Head2014, Mendel2014, Vika2014, Meert2015, Meert2016, Kennedy2016, Kim2016, Lange2016, Dimauro2018, Bottrell2019, Cook2019, Barsanti2021, DominguezSanchez2022, Haeussler2022, Hashemizadeh2022}; and \citet{Robotham2022}. 
% 
Such catalogues can then be used to determine the relative numbers of different galaxy components as well as their luminosity or stellar mass functions, size-mass or size-luminosity relations, including their redshift evolution and dependence on other properties of the galaxy and its environment (similar to the studies of entire galaxies mentioned earlier). For example, this has been done by \citet{Driver2007b, Dutton2011, Tasca2014, Kennedy2016, Lange2016, Moffett2016} and \citet{Dimauro2019}.
%  


In addition, quantitative measures for the components of galaxies aid the comparison of observational data to theory and simulations. Bulges and disks are often decisively different not only in their visual appearance but also in their structure, dynamics, stellar populations, gas and dust content and are thought to have different formation pathways (Section~\ref{sec:galaxystructure} and \citealt{Cole2000, Cook2009, Driver2013, Lange2016, Dimauro2018, Lagos2018, Oh2020}). Consequently, bulge-disk decomposition studies provide stringent constraints on the formation and evolutionary histories of galaxies and their physical properties that are not easily measured directly such as the dark matter halo, the build-up of stellar mass (in different components) over time, or merger histories 
\citep[examples include][]{Driver2013, Bottrell2017b, Bluck2019, Rodriguez-Gomez2019, deGraaff2022}. 
% 
Hence, consistently measuring the structure of the stellar components is essential to make full use of current and future large-scale observational surveys such as the Kilo-Degree Survey \citep[KiDS;][]{deJong2013} and the VISTA Kilo-Degree INfrared Galaxy \citep[VIKING;][]{Edge2013} Survey or the Legacy Survey of Space and Time \citep[LSST;][]{Ivezic2019}, % also DES, HSC  
and of cosmological hydrodynamical simulations such as Illustris \citep{Vogelsberger2014} and IllustrisTNG \citep[The Next Generation;][]{Pillepich2018} or Evolution and Assembly of GaLaxies and their Environments \citep[EAGLE;][]{Schaye2015}.%\\ 


\subsection{Science aims}
\label{sec:scienceaims}

This work is the first in a series of planned contributions to the field of galaxy structure and evolution, with the final aim of the series being to constrain the nature and distribution of dust in galaxy disks. \citet{Driver2007} have shown that in the $B$-band of the Millennium Galaxy Catalogue (MGC), an average of 37\,\% of photons produced in the disk and 71\,\% of photons produced in the bulge are absorbed before even leaving their galaxy of origin. Accounting for this effect of internal dust attenuation is therefore vital to avoid biases in structure formation and galaxy evolution studies, especially since the dust properties evolve as well. Due to the geometry of the system (see also Section~\ref{sec:galaxystructure}), bulges and disks are affected differently and the severity of the attenuation depends strongly on the inclination angle of the galaxy relative to our line of sight (see figure 11 in \citealt{Driver2007}). Bulges and disks therefore need to be studied separately. 

A sufficiently large sample of robust structural parameters for these two components can be used to constrain the nature and distribution of dust. This can be achieved by comparing the distribution of bulges and disks in the luminosity-size plane as a function of inclination to dust radiative transfer models such as those presented in \citet{Popescu2011} and preceding papers of this series. The analysis is particularly powerful if observations in multiple filters are available since dust attenuation also varies strongly as a function of wavelength \citep{Popescu2011}. 

This thesis presents the first step towards achieving this final goal: we obtain single S\'ersic fits and bulge-disk decompositions for 13096 GAMA galaxies in the KiDS $u$, $g$, $r$ and $i$ bands and the VIKING $Z$, $Y$, $J$, $H$ and $K_s$ bands. We choose \texttt{ProFit} (see Section~\ref{sec:profit}) as our modelling software due to its Bayesian nature (allowing full MCMC treatment including more realistic error estimates), its suitability to large-scale automated analyses and its ability, in combination with \texttt{ProFound} (Section~\ref{sec:profound}), to serve as a fully self-contained package covering all steps of the analysis from image segmentation through to model fitting. We supplement this functionality with our own routines for the rejection of unsuitable fits, model selection, and a characterisation of systematic uncertainties. The aim of our future work is to use those structural parameters for the bulges and disks to significantly expand the analysis of \citet{Driver2007}, exploiting our more and better data at several wavelengths. 

With these science aims in mind, we are most interested in obtaining structural parameters that are directly comparable amongst each other, i.e. consistent within the dataset; and correctly represent the statistical properties of the entire sample, with less emphasis placed on capturing all aspects of the detailed structure of individual galaxies. Consequently, we choose to model a maximum of two components for each galaxy and use the terms ``bulge" and ``disk" in their widest senses, in line with previous automated decompositions of large samples. In particular the ``bulges" we obtain are often mixtures of classical or pseudo-bulges, bars, lenses and AGN. Similarly, we place more emphasis on the central, high surface brightness regions of galaxies by modelling only a relatively tight region around each galaxy of interest. While most of the fits we obtain are not perfect (because galaxies are more complex than two simple components), they do achieve the aims specified above and are comparable to similar studies.

While much of our procedure was focused towards obtaining suitable fits for our final goal of studying dust properties, there are many other analyses that can be built on our results. Some of the most obvious of these include deriving the stellar mass functions of bulges and disks, studying component colours and investigating the trends of structural parameters such as bulge or disk size with wavelength, all of which belong to our plans for future work. Additionally, (an earlier version of) the resulting catalogue has been used to aid the kinematic bulge-disk decomposition of a sample of galaxies in the Sydney-AAO Multi-object Integral-field spectroscopy (SAMI) Galaxy Survey \citep{Oh2020}, to examine the properties of galaxy groups \citep{Cluver2020}, to investigate the difference between ionised gas and stellar velocity dispersions \citep{Oh2022}, to cross-check the results of other multi-wavelength bulge-disk decomposition studies \citep{Haeussler2022, Robotham2022}, to study the alignment of galaxy spin axes with filaments of the cosmic web as a function of different galaxy (component) properties (Barsanti et al., in prep.) and to validate a cosmological galaxy simulation against observations using an unsupervised machine learning technique (Turner et al., in prep.). Several student projects, including Bachelor's and Master's theses, have also made use of our results. For example, \citet{Roschlaub2022} tested the usage of a new deconvolution algorithm presented in \citet{Nammour2021} in the context of galaxy fitting, Targaczewski (in prep.) is working on estimating the supermassive black hole mass from the bulge S\'ersic index for a sample of AGN in our catalogue, Ehbrecht (in prep.) is investigating under which circumstances (seeing and noise) a given S\'ersic profile can be reasonably constrained, Porter-Temple (in prep.) is going to look at the difference the number of spiral arms make on the bulge-to-disk flux ratio and Porter (in prep.) will compare the bulge-to-disk flux ratios of void galaxies to those of the remaining GAMA sample. 

Apart from these scientific insights enabled by the final product of our pipeline (the catalogue), advancements on the technical side should not be neglected. As briefly mentioned before, improvements in methodology are crucial to make full use of current and future datasets of large-scale galaxy surveys; and we have contributed to this aspect in several ways. First of all, \texttt{ProFit}, and even more so \texttt{ProFound}, were under active development during the time of our own pipeline development, with this thesis project being one of the first large-scale automated applications of both packages. Consequently, our work has contributed to improving both packages by discovering bugs, suggesting additional features and serving as inspiration for the implementation of various automated procedures. In addition, we have added routines for the swapping of bulge and disk components (see Section~\ref{sec:galaxyfitting}) and post-processing of fits (mainly model selection and flagging of bad fits, Section~\ref{sec:postprocessing}); and have performed numerous tests with varying combinations of \texttt{ProFound} and \texttt{ProFit} routines and their tuning parameters for the preparatory work (image segmentation, background subtraction, point spread function estimation, obtaining initial guesses) as well as the galaxy fitting. The resulting procedures, as well as the numerous alternatives that we found to be less optimal are described in detail in Chapter~\ref{chap:pipeline} and can serve as guides for similar bulge-disk decomposition works on current and future large-scale galaxy surveys. Our detailed analysis of systematic uncertainties based on the overlap sample and bespoke simulations (Section~\ref{sec:systematics}) can also give orientation towards constraining and quantifying the dominant sources of error in other analyses. 

The goal of this thesis therefore is not only to present our final catalogue of robust structural parameters for the components of galaxies, but also the large amounts of technical work surrounding the pipeline development. 


\section{Data and sample}
\label{sec:dataandsample} 

After the general introduction of the context and aims of this thesis, we now present the data products that we use from GAMA (Section~\ref{sec:gama}), KiDS (Section~\ref{sec:kids}) and VIKING (Section~\ref{sec:viking}) followed by the selection of our sample of galaxies (Section~\ref{sec:sampleselection}). Sections~\ref{sec:gama}, \ref{sec:kids} and~\ref{sec:sampleselection} are largely based on \citet[their section 2]{Casura2022}. 

\subsection{GAMA}
\label{sec:gama}

The Galaxy and Mass Assembly (GAMA)\footnote{\label{foot:gama}\url{http://www.gama-survey.org}} survey is a large low-redshift spectroscopic survey covering $\sim$\,238\,000 galaxies in 286\,deg$^2$ of sky (split into 5 survey regions) out to a redshift of approximately 0.6 and a depth of $r$\,<\,19.8\,mag. The observations were taken using the AAOmega spectrograph on the Anglo-Australian Telescope and were completed in 2014. The survey strategy and spectroscopic data reduction are described in detail in \citet{Driver2009, Baldry2010, Robotham2010, Driver2011, Hopkins2013, Baldry2014} and \citet{Liske2015}. 

In addition to the spectroscopic data, the GAMA team collected imaging data on the same galaxies from a number of independent surveys in more than 20 bands with wavelengths between 150\,nm and 500\,$\mu$m. Details of the imaging surveys and the photometric data reduction are given in \citet{Liske2015, Driver2016, Driver2022}; and relevant publications of the corresponding independent surveys. The combined spectroscopic and multiwavelength photometric data at this depth, resolution and completeness provide a unique opportunity to study a variety of properties of the low-redshift galaxy population.

In this work, we focus on the KiDS and VIKING imaging data in the nine optical and near-infrared filters $u, g, r, i, Z, Y, J, H, K_s$ (see Sections~\ref{sec:kids} and~\ref{sec:viking}) in the GAMA II equatorial survey regions, which are 3 regions of size 12\degr\,$\times$\,5\degr\ located along the equator at 9, 12 and 14.5 hours in right ascencion (the G09, G12 and G15 regions). For our sample selection, we make use of the equatorial input catalogue\footnote{For the sake of reproducibility, we always give the exact designation of a catalogue on the GAMA database in parentheses: the data management unit (DMU) that produced the catalogue (e.g. \texttt{EqInputCat}) followed by the catalogue name (e.g. \texttt{TilingCat}) and the version used (e.g. \texttt{v46}).} \citep[\texttt{EqInputCat:TilingCatv46},][]{Baldry2010} and the most recent version of the redshifts originally described by \citet{Baldry2012} (\texttt{LocalFlowCorrection:DistancesFramesv14}), see details in Section~\ref{sec:sampleselection}. For the stellar mass-size relation (Section~\ref{sec:sizemass}), we also use the Data Release (DR) 3 version of the stellar mass catalogue first presented in \citet{Taylor2011} (\texttt{StellarMasses:StellarMassesv19}); for the comparison to previous work (Section~\ref{sec:comparelee}) we use the single S\'ersic fits of \citet{Kelvin2012} (\texttt{SersicPhotometry:SersicCatSDSSv09}); and in order to correct galaxy colours for Galactic extinction, we use the corresponding table provided along with the equatorial input catalogue (\texttt{EqInputCat:GalacticExtinctionv03}). All of these catalogues can be obtained from the GAMA database.$^{\ref{foot:gama}}$ 

\subsection{KiDS}
\label{sec:kids}

The Kilo-Degree Survey \citep[KiDS,][]{deJong2013} is a wide-field imaging survey in the Southern sky using the VLT Survey Telescope (VST) at the ESO Paranal Observatory. 1350\,deg$^2$ are mapped in the optical broad-band filters $u, g, r, i$; while the VIKING Survey (\citealt{Edge2013}, Section~\ref{sec:viking}) provides the corresponding near-infrared data in the $Z, Y, J, H, K_s$ bands. The GAMA II equatorial survey regions have been covered as of DR3.0.  

KiDS provides $\sim$\,1\degr\,$\times$\,1\degr\ science tiles calibrated to absolute values of flux with associated weight maps (inverse variance) and binary masks. The science tiles are composed of 5 dithers (4 in $u$) totalling 1000, 900, 1800 and 1200\,s exposure time in $u,g,r,i$, with all dithers aligned in the right ascenscion and declination axes (i.e. no rotational dithers) and taken in immediate succession. The $r$-band observations were performed during the best seeing conditions in dark time; while $g$, $u$ and $i$ have progressively worse seeing and $i$ was additionally taken during grey time or bright moon. During co-addition, the dithers across all four bands were re-gridded onto a common pixel scale of $0\farcs$2. The magnitude zeropoint of the science tiles is close to zero with small corrections given in the image headers for DR4.0. The $r$-band point spread function (PSF) size is typically $0\farcs$7 and the limiting magnitudes in $u,g,r,i$ are $\sim$\,24.2, 25.1, 25.0, 23.7\,mag respectively (5$\sigma$ in a 2\arcsec aperture). This high image quality, depth, survey size and wide wavelength coverage in combination with VIKING make KiDS data unique. For details, see \citet{Kuijken2019}. 

For this work, we use the the $u$, $g$, $r$ and $i$ band science tiles, weight maps and masks from KiDS DR4.0 \citep{Kuijken2019}, which are publicly available\footnote{\url{http://kids.strw.leidenuniv.nl/DR4/index.php}} for our selected sample of galaxies (Section~\ref{sec:sampleselection}). Our primary band, used for most analyses during pipeline development, is the $r$-band since it is the deepest and was taken during the best seeing conditions. Observations in $g$ and $i$ are of comparable quality, such that these three bands together form our core bands where we obtain the best and most comparable fits (these are also what \citealt{Casura2022} is based on). The $u$-band is of considerable worse data quality especially in terms of depth and therefore we place less emphasis on its analysis. More details are given in Section~\ref{sec:pipelinedevelopment}, also explaining our choice of which of the KiDS data products to use (Section~\ref{sec:otherprepworkchoices}). 


\subsection{VIKING}
\label{sec:viking}

The VISTA Kilo-degree INfrared Galaxy (VIKING) survey is a wide-field, intermediate-depth near-infrared imaging survey using the Visible and Infrared Survey Telescope for Astronomy (VISTA) at the ESO Paranal Observatory. 1350\,deg$^2$ were mapped in the broad-band filters $Z, Y, J, H, K_s$ over two areas of sky matched to the KiDS footprint. The observations are complete and the fourth and final public data release is described in \citet{Edge2020}. 

VIKING provides astrometrically and photometrically calibrated tiles of size $\sim$\,1.5\degr\ in right ascension (RA) and $\sim$\,1\degr\ in declination (Dec) composed of 6 pawprints each, arranged in a fashion to cover the gaps between detector chips. Each tile is observed twice, with observations sometimes years apart: first in the $Z, Y, J$ filters with total exposure times of 480, 400 and 200\,s; and then again in the $J, H, K_s$ filters with exposure times of 200, 300 and 480\,s. Combining the two $J$-band observations results in median magnitude limits of 21.4, 20.6, 20.1, 19.0 and 18.6\,mag in the $Z, Y, J, H$ and $K_s$ bands respectively, although some fields can be shallower by up to 0.3\,mag which is the quality threshold applied by the VIKING team. In addition to the stacked tiles, the pawprints are publicly available. Both pawprints and stacks are approximately aligned in RA and Dec, have a pixel size close to $0\farcs$34 and various Vega magnitude zeropoints around 30, with the exact values given in the image headers. They also both have associated confidence maps which give the per-pixel exposure time and exclude the ``bad patches" of two detectors that have flat-fielding issues. More details are given in \citet{Edge2020}.

For this work, we use the $Z, Y, J, H$ and $K_s$ individual detector images from the pawprints. In particular, we use the preprocessed versions from \citet{Wright2019} that were kindly provided to us by Angus Wright and the KiDS team. These have the advantage that they are specifically processed in order to allow a consistent analysis in combination with KiDS data: they have been rotated slightly to be exactly aligned in RA and Dec, corrected for atmospheric extinction, re-calibrated to remove the exposure time from the image units and re-scaled onto a common AB magnitude zeropoint of 30. In addition, \citet{Wright2019} also perform background subtraction, produce weight maps and perform a quality control of each chip. More details and the reasons for our choice to use these preprocessed individual chips instead of the VIKING stacked tiles are given in Section~\ref{sec:vikingdataproducts}. 

\subsection{Sample selection}
\label{sec:sampleselection}

Our main sample consists of all GAMA II equatorial region main survey targets with a reliable redshift in the range 0.005\,<\,$z$\,<\,0.08, which are a total of 12958 objects.\footnote{In detail, we select all targets with NQ\,$\geq$\,3, SURVEY\textunderscore CLASS\,$\geq$\,4 and 0.005\,<\,Z\textunderscore CMB\,<\,0.08 from \texttt{EqInputCat:TilingCatv46} joined to \texttt{LocalFlowCorrection:DistancesFramesv14} on CATAID.} In addition, we include all 2404 targets of the ``GAMA sample" of the SAMI Galaxy Survey\footnote{Taking the CATIDs listed in sami\textunderscore sel\textunderscore 20140413\textunderscore v1.9\textunderscore publiclist from \url{https://sami-survey.org/data/target_catalogue}} \citep{Bryant2015}, the majority of which are already in our main sample. The combination of both results in the full sample of 13096 unique physical objects, which were imaged a total of 14966 times in each of the KiDS $g$, $r$ and $i$ bands due to small overlap regions between the tiles. 11301, 1742, 31 and 22 objects were imaged once, twice, three and four times respectively. For versions of our bulge-disk decomposition pipeline up to \texttt{BDDecomp} \texttt{v04} \citep{Casura2022}, we keep these multiple data matches to the same physical object separate during all processing steps to serve as an internal consistency check (Chapter~\ref{chap:QC}). For \texttt{v05}, where we add the KiDS $u$ and the five VIKING bands to the analysis, we then changed this to instead fit all data matches in the same band jointly. This is necessary due to our decision to work at the individual detector chip level for VIKING (see Section~\ref{sec:viking}), resulting in many data matches to the same physical objects (more than 20 for some objects). See Section~\ref{sec:pipelineupdates} for details. 



\section{Methods and code}
\label{sec:methodsandcode}

With the galaxy sample and input data defined, we now turn towards the methods used in the analysis of those. After a brief introduction into the theoretical background of Bayesian analysis and S\'ersic modelling in Sections~\ref{sec:bayesiananalysis} and~\ref{sec:sersicmodels}, we present the two main code packages used for the galaxy modelling and preparatory work respectively, \texttt{ProFit} and \texttt{ProFound} (Sections~\ref{sec:profit} and~\ref{sec:profound}; taken from section~2 of \citet{Casura2022}). 

\subsection{Bayesian analysis}
\label{sec:bayesiananalysis}

Bayesian probability theory, originally introduced by Reverend Tho\-mas Bayes in the 18$^\mathrm{th}$ century and further developed by Laplace in the early 19$^\mathrm{th}$ century, has recently experienced a ``re-discovery" with rapidly increasing popularity. In the alternative (for a long time more popular) frequentist approach, probability is defined as the ``long-run relative frequency" of an outcome and hence requires many repetitions of an experiment. In contrast to this, Bayesian probability is defined as a ``degree of belief", which has proven to be a powerful approach especially in fields such as astronomy, where the repeatability of experiments is often very limited \citep{Sivia2006, Gregory2005}. We briefly review its most basic concepts here following \citet{Sivia2006}. We refer the reader to this and similar works for a more detailed treatment. 

Bayesian analysis can be used both for parameter estimation and model selection. Its most basic building block is Bayes' theorem, which can be expressed as
\begin{equation}
\label{eq:bayes}
\mathrm{p}(B \vert A, C) = \frac{\mathrm{p}(A \vert B, C) \mathrm{p}(B \vert C)}{\mathrm{p}(A \vert C)}. 
\end{equation}
Here, $\mathrm{p}(B \vert A, C)$ denotes the probability p of statement B given information A and C. 

In the context of parameter estimation, this becomes
\begin{equation}
\label{eq:bayesparameter}
\mathrm{p}(\theta \vert data, M) = \frac{\mathrm{p}(data \vert \theta, M) \mathrm{p}(\theta \vert M)}{\mathrm{p}(data \vert M)}, 
\end{equation}
where $\theta$ denotes one or several parameters of model M. It gives the posterior probability of the parameter(s) [$\mathrm{p}(\theta \vert data, M)$, the quantity that is desired in parameter estimation], as the product of the probability of the data given the parameter values and the model [$\mathrm{p}(data \vert \theta, M)$, which is easily computed] with the prior of the parameter [$\mathrm{p}(\theta \vert M)$, its probability distribution before taking any data] and a normalisation constant that depends only on the data and the model, not the parameter [$\mathrm{p}(data \vert M)$, called evidence]. Note that if the model contains several parameters, the posterior probability is a joint probability for all of those parameters and one needs to marginalise (integrate) over all other parameters to obtain the correct one-dimensional posterior probability of any given parameter of choice. In practice, the most efficient way to achieve this usually is sampling the likelihood space via Markov Chain Monte Carlo (MCMC) or similar methods. Provided convergence is achieved, the MCMC chain points are representative samples from the joint posterior, allowing easy projection of the distribution onto any axis (parameter) of interest. 

For model selection, Bayes' theorem (Equation~\ref{eq:bayes}) takes the following form: 
\begin{equation}
\label{eq:bayesmodel}
\mathrm{p}(M_k \vert data, I) = \frac{\mathrm{p}(data \vert M_k, I) \mathrm{p}(M_k \vert I)}{\mathrm{p}(data \vert I)}. 
\end{equation}
This now expresses the posterior probability of Model k as the product of the model likelihood with the model prior, normalised by the probability of the data given any relevant background information $I$. The ratio of two model posteriors gives the odds ratio between two models: 
\begin{equation}
\label{eq:oddsratio}
O_{1:2} = \frac{\mathrm{p}(M_1 \vert data, I)}{\mathrm{p}(M_2 \vert data, I)} = \frac{\mathrm{p}(data \vert M_1, I)}{\mathrm{p}(data \vert M_2, I)}\frac{ \mathrm{p}(M_1 \vert I)}{\mathrm{p}(M_2 \vert I)},
\end{equation}
where the normalisation constant cancelled out since it is the same for both models. This gives the relative probability of the two models (for any parameter values) given the same set of data and background information. The last term on the right hand side is the prior odds ratio, i.e. the relative probability of the two models before taking any data. Unless there is a strong reason to prefer one model over the other (e.g. previous data contained in $I$), this ratio should be set to 1 to allow a fair comparison of models. 

The most relevant term is hence the first term on the right hand side, called the Bayes factor. It consists of the likelihoods for each model, which is the probability of the data given the model, marginalised (integrated) over all possible parameter values: 
\begin{equation}
\label{eq:marglike}
\begin{split}
\mathrm{p}(data \vert M_k, I) &= \int \mathrm{p}(data, \theta \vert M_k, I) d\theta\\
&= \int \mathrm{p}(data \vert \theta, M_k, I) \mathrm{p}(\theta \vert M_k, I) d\theta. 
\end{split}
\end{equation}
Due to this integral over all parameters, the marginalised likelihood will generally become smaller when adding parameters that do not improve the fit significantly (since the likelihood $\mathrm{p}(data \vert \theta, M_k, I)$ is then approximately constant while the integral over the prior probabi\-lities for the parameters $\mathrm{p}(\theta \vert M_k, I)$ decreases due to the larger prior range). Equation~\ref{eq:oddsratio} hence naturally implements Ockham's factor (also called Ockham's razor) which states that if two models can explain the data equally well, then the simpler one should be preferred (i.e. to avoid overfitting). 
Note that the model likelihood in Equation~\ref{eq:marglike} is also the same as the evidence in parameter estimation (the denominator in Equation~\ref{eq:bayesparameter}) and for this reason the Bayes factor can also be called the ratio of evidences or the ratio of marginalised likelihoods. Computing the model likelihood is often non-trivial especially for models with many para\-meters, since it requires high-dimensional integrals. We return to this in Section~\ref{sec:postprocessing}. 

 
\subsection{S\'ersic models}
\label{sec:sersicmodels}


The most important component for both parameter estimation and model selection - apart from the data - is the model $M$ with its parameters $\theta$. If the model is wrong in the sense that it cannot represent the data adequately, then any estimated parameter values will have limited validity. Comparing two inadequate models consequently also produces an odds ratio with little meaning. We hence carefully explain our choice of models (and parameters) here, including its implications.

The most common radial profile to fit the surface brightness distribution of galaxies with is the \citet{Sersic1963} function. It gives the intensity $I$ as a function of radius $r$:
\begin{equation}
    I(r)=I_e \exp\left[-b_n \left(\left(\frac{r}{R_e}\right)^{1/n}-1\right)\right].
	\label{eq:sersic}
\end{equation}
Here, $n$ is the S\'ersic index (the main shape parameter), $R_e$ is the effective radius where half of the total flux is included (the size), $I_e$ is the intensity at $R_e$ (the overall normalisation) and $b_n$ is a normalisation constant which can be calculated from $n$. The S\'ersic function becomes a Gaussian for $n$\,=\,0.5, exponential for $n$\,=\,1 and a \citet{deVaucouleurs1948} profile for $n$\,=\,4 and can be adjusted to account for ellipticity and/or boxyness in two dimensions \citep{Robotham2017}. Due to this flexibility, the S\'ersic function can fit a wide variety of galaxy (component) shapes and types such as exponential disks or classical de Vaucouleurs bulges (Section~\ref{sec:galaxystructure}) and correspondingly is extremely popular in the galaxy fitting community \citep[and references therein]{Graham2005}. It is, however, a purely empirically derived function with no profound physical meaning. 

Adding several S\'ersic profiles together allows to fit multiple galaxy components. Common examples include a de Vaucouleurs bulge plus an exponential disk (i.e. S\'ersic functions with $n$ fixed to 4 and 1 respectively), a (free $n$) S\'ersic bulge plus exponential disk or a double S\'ersic profile. In all cases, the bulge may or may not be forced to be round (i.e. circular instead of elliptical in two dimensions) and there can additionally be further constraints on the parameters such that for example the centres of the bulge and disk components must align. Further morphological features, such as bars, can also be accounted for by adding more (S\'ersic or other) profiles. 

However, there are also problems associated with increasing numbers of components and fitting parameters: parameter degeneracies become more common, convergence is more difficult to achieve and when the fitting algorithm has converged it needs to be ensured that the solution is also physical, i.e. that the different model components do indeed represent dif\-fe\-rent physical components of the galaxy. As briefly mentioned in Section~\ref{sec:SBmodelling}, this problem is particularly pronounced for automated analyses of large samples of galaxies with varying properties, where manual intervention and control of the fits is limited. Depending on the image quality, physical properties and redshift of the galaxy, it is possible that only one component can reasonably be constrained by the data even though the galaxy physically consists of several. Others may need three of more components to adequately represent all morphological features. Conversely, there are also a number of galaxies (e.g. ellipticals) that physically contain just a single component, leaving any additional model components unconstrained. While in theory these should be easily recognized during model selection, it is often non-trivial in practice due to overlapping point sources, neighbouring objects, imperfect PSF estimates or sky subtraction, image artifacts and similar issues. 

For these reasons, we decided to follow most other works in the field of large automated ana\-lyses of galaxy surface brightness fitting (Section~\ref{sec:SBmodelling}) and fit a maximum of two components to each galaxy. For the disk component, we use an exponential profile that is elliptical in two dimensions. This ensures that we can correctly capture disks with differing brightnesses, sizes and inclination angles. We do not allow the S\'ersic index to vary (i.e. we fit exponential disks instead of S\'ersic disks) as most disks do follow an exponential profile on average \citep{ZeilikandGregory}, with spiral arms only contributing local perturbations. Other deviations from the exponential profile, such as disk breaks and flares are most common in the disk outskirts, where we place less emphasis by considering only a relatively tight region around each galaxy for fitting. For the bulge profile, however, we leave the S\'ersic index free in order to not only capture classical de Vaucouleurs bulges (with S\'ersic indices around 4) but also pseudo-bulges (with S\'ersic indices usually below 2) and bars (also low S\'ersic indices and additionally elongated shapes; see Section~\ref{sec:galaxystructure}). For the same reason we also do not constrain the bulge - or, more precisely, the central component - to be round. However, we do constrain the bulge and disk to lie precisely on top of each other to avoid one of the components wandering off to fit overlapping point sources. 

In addition to this two-component model, we fit two simpler ones: a single S\'ersic model to represent those galaxies that either physically have just one component or where the data quality is not sufficient to constrain more than one component; and a ``1.5-component" point source bulge plus exponential disk model. While the former is also routinely done in the literature since it is a wise decision in general to start with a simple model before adding complexity (also for Bayesian model selection), the latter is less common. The reason for us adding this model is that after fitting, we identified a population of bulges that were clearly present, but had ill-constrained parameters due to their faintness and small sizes. To obtain reliable magnitudes for these bulges, we considered it best to limit the freedom of the fitting parameters to a minimum and fit the point source model (consisting of only a magnitude and a position) instead of the S\'ersic bulge model. 


In summary, we fit three models to each galaxy: a single S\'ersic, a S\'ersic plus exponential and a point source plus exponential. More details are given in Sections~\ref{sec:galaxyfitting} and~\ref{sec:modellingdecisions}. We note that according to the morphological classification performed by \citet{Driver2022} (see Section~\ref{sec:galaxystructure}), the majority of galaxies in our sample can be approximately represented by one of these three models. However, in a statistical sense, our models are not an appropriate representation of the detailed structure of most galaxies. We therefore expect highly correlated residuals caused by - for example - spiral arms, rings, nuclear lenses, bars, AGN and bulges (especially if several of these features are present and the ``bulge" S\'ersic function is forced to compromise between fitting them) as well as asymmetries and irregular features of all kinds. These model shortcomings need to be considered when assessing the fit quality and for model selection, see Sections~\ref{sec:postprocessing} and \ref{sec:swappingandoutliers}.



\subsection{ProFit}
\label{sec:profit}

Once the data are obtained and the model and its parameters are defined, the decision comes to the software to use for fitting. We opt to use \texttt{ProFit}\footnote{\url{https://github.com/ICRAR/ProFit}} (v1.3.2) which is a free and open-source, fully Bayesian two-dimensional profile fitting code specifically developed to fit the surface brightness distributions of galaxies \citep{Robotham2017}. \texttt{ProFit} offers great flexibility: there are several built-in profiles to choose from, it is easy to add several components of the same or different profiles, there is a choice of likelihood calculations and optimisation algorithms that can be used (various downhill gradient options, genetic algorithms, over 60 variants of MCMC methods), parameters can be fitted in linear or logarithmic space, it is possible to add complex priors for each, as well as constraints relating several parameters; and much more. The pixel integrations are performed using a standalone C++ library (\texttt{libprofit}), making it both faster and more accurate than other commonly used algorithms such as \texttt{GALFIT} (\citealt{Peng2010}; see detailed comparison in \citealt{Robotham2017}). This allows us to fit galaxies with the computationally more expensive MCMC algorithms, overcoming the main problems of downhill gradient based optimisers: their susceptibility to initial guesses and their inability to easily derive realistic error estimates \citep[e.g.][]{Lange2016}. This makes \texttt{ProFit} highly suitable for the decomposition of large sets of galaxies with little user intervention.

\subsection{ProFound}
\label{sec:profound} 

\texttt{ProFit} (Section~\ref{sec:profit}) requires a number of inputs apart from the (sky-subtracted) science image and the chosen model to fit, most importantly initial parameter guesses, a segmentation map specifying which pixels to fit, a sigma (error) map and a PSF image. To provide these inputs in a robust and consistent manner, the sister package \texttt{ProFound}\footnote{\url{https://github.com/asgr/ProFound/}} \citep{Robotham2018} was developed, which also serves as a stand-alone source finding and image analysis tool. The main novelties of \texttt{ProFound} compared to other commonly used free and open-source packages such as \texttt{Source Extractor} \citep{Bertin1996} are that, rather than elliptical apertures, \texttt{ProFound} uses dilated ``segments" (collections of pixels of arbitrary shape) with watershed de-blending across saddle-points in flux. This means that the flux from each pixel is attributed to exactly one source (or the background) and apertures are never overlapping or nested. It also allows for extracting more complex object shapes than ellipses while still capturing the total flux due to the segment dilation (expansion) process. This makes it less prone to catastrophic segmentation failures (such as fragmentation of bright sources or blending of several sources into one aperture), reducing the need for manual intervention or multiple runs with ``hot" and ``cold" deblending settings, making \texttt{ProFound} particularly suitable for large-scale automated analysis of deep extragalactic surveys \citep{Robotham2018, Davies2018, Bellstedt2020}.

Apart from the segmentation map, the main function of the package, \texttt{profoundProFound}, also returns estimated sky and sky-RMS maps (if not given as inputs) and a wealth of ancillary data including a list of segments and their properties such as their size, ellipticity and the flux contained. The latter is particularly useful to obtain reasonable initial parameter guesses for galaxy fitting; or for identifying certain types of sources (e.g. stars for PSF estimation). The package also contains many additional functions for further image analysis and processing, all within the same framework. In addition, combining \texttt{ProFound} with \texttt{ProFit} allows the user to estimate a PSF (see Section~\ref{sec:preparatorysteps}), entirely removing any dependence on external tools. Finally, both packages come with comprehensive documentation and many extended examples and vignettes which serve as great resources for newcomers to the fields of source extraction and galaxy fitting. 

We use \texttt{ProFound} (v1.9.2) along with \texttt{ProFit} (v1.3.2) for all preparatory steps (image segmentation/source identification, sky subtraction, initial parameter estimates and PSF determination; see Section~\ref{sec:preparatorysteps} for details) producing the inputs needed for the galaxy fitting with \texttt{ProFit}. 
