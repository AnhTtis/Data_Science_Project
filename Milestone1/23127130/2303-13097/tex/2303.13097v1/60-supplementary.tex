\twocolumn[{
\begin{center}
\textbf{\Large Supplementary Materials for \\
CP$^3$: Channel Pruning Plug-in for Point Cloud Network}
\end{center}
}]
In the supplementary material, we provide more experimental comparisons on object detection in Sec.~\ref{supp:od} and segmentation tasks in Sec.~\ref{supp:ss} , and we showcase the effectiveness of the proposed knowledge recycling module in Sec.~\ref{supp:kr}, we also analyzed the pruning rates of different layers in Sec.~\ref{supp:al}.

\section{More Experimental Results on 3D Object Detection}
\label{supp:od}
To further illustrate the effectiveness of our proposed \cpt, we incorporated CP$^3$ with pruning methods HRank~\cite{jia2021arank} and CHIP~\cite{sui2021chip} to prune VoteNet~\cite{qi2019deep} on ScanNetV2~\cite{dai2017scannet} and SUN RGB-D~\cite{song2015sun} for 3D object detection.



\begin{table}[h!]
\centering
\caption{Comparisons of object detection performance on the ScanNetV2 dataset.
The baseline PNN model is VoteNet.}\label{tabl1_vsn_supp}
\resizebox{\columnwidth}{!}{
\begin{tabular}{lcccc}
\toprule
\textbf{Method} & \textbf{mAP@0.25} & \textbf{mAP@0.50} & \textbf{Params. (K)} & \textbf{GFLOPs ($\downarrow$\%)}      \\ \midrule
Baseline     & 62.34    & 40.82   & 641.92       & 5.78 (--)     \\ \cmidrule(lr){1-5}
HRank        & 61.04    & 37.99   & 249.82       & 2.46 (57.4)    \\
HRank+CP$^3$ & 61.66    & 39.25   & 239.43       & 2.44 (57.8)    \\
HRank        & 59.46    & 35.98   & 178.16       & 1.87 (67.7)    \\
HRank+CP$^3$ & 60.51    & 39.15   & 169.87       & 1.80 (68.9)    \\ \cmidrule(lr){1-5}
CHIP         & 62.17    & 41.37   & 247.72       & 2.49 (56.9)    \\
CHIP+CP$^3$  & 62.33    & 41.49   & 245.45       & 2.45 (57.6)    \\
CHIP         & 60.86    & 39.94   & 176.88       & 1.89 (67.3)    \\
CHIP+CP$^3$  & 61.55    & 40.43   & 172.78       & 1.87 (67.6)    \\
\bottomrule
\end{tabular}
}
\end{table} 

\para{ScanNetv2}
Tab.~\ref{tabl1_vsn_supp} shows the comparison results of directly applying advanced pruning methods (HRank, CHIP) and implementation them with \cpt.
%
Overall, CP$^3$ consistently improved the performance of existing advanced CNN pruning methods under different pruning rates.
%
For instance, in the case of applying HRank with 67.7\% FLOPs reduction, by incorporating \cpt, the mAP@0.50 increased 3.17\% (35.95\% vs. 39.15\%) while achieving 1.2\% more FLOPs reduction (67.7\% vs. 68.9\%).


\begin{table}[h!]
\centering
\caption{Comparisons of object detection performance on the SUN RGB-D dataset. The baseline PNN model is VoteNet.}\label{table2_vsun_supp}
\resizebox{\columnwidth}{!}{
\begin{tabular}{lcccc}
\toprule
\textbf{Method} & \textbf{mAP@0.25} & \textbf{mAP@0.50} & \textbf{Params. (K)} & \textbf{GFLOPs ($\downarrow$\%)}      \\ \midrule
Baseline        & 59.78             & 35.77             & 641.92               & 5.78 (--)            \\  \cmidrule(lr){1-5}
HRank           & 59.22             & 34.26             & 249.82               & 2.46 (57.4)          \\
HRank+CP$^3$    & 60.21             & 34.96             & 245.32               & 2.44 (57.8)           \\
HRank           & 57.68             & 31.30             & 178.88               & 1.87 (67.7)          \\
HRank+CP$^3$    & 59.22             & 33.18             & 176.03               & 1.85 (68.0)           \\ \cmidrule(lr){1-5}
CHIP            & 59.54             & 35.74             & 248.31               & 2.49 (56.9)          \\
CHIP+CP$^3$     & 59.88             & 35.84             & 242.12               & 2.43 (58.0)            \\
CHIP            & 58.63             & 35.07             & 176.23               & 1.89 (67.3)          \\
CHIP+CP$^3$     & 59.13             & 35.32             & 172.02               & 1.87 (67.6)            \\
\bottomrule
\end{tabular}
}
\end{table} 

\para{SUN RGB-D}
We reported the comparison results on the {SUN RGB-D} dataset in Tab.~\ref{table2_vsun_supp}.
%
For both HRank and CHIP, the implementation with {\cpt} achieved higher accuracy performance with higher FLOPs reduction, similar to our observations on other tasks and datasets.


\begin{figure*}[t]
    \centering
    \includegraphics[width=\linewidth]{figs/variance.pdf}
    \caption{The variance of the importance score of each channel in the 10-th layer of PointNeXt-S. The x-axis represents channel indices, and the y-axis represents the variances of each channel importance scores, which are calculated by 2,468 input samples.}
    \label{variance}
\end{figure*}

\section{More Experimental Results on Semantic Segmentation}
\label{supp:ss}

To investigate the generality of our work, we extended the comparisons on semantic segmentation. We conducted the experiment on the S3DIS~\cite{DBLP:conf/cvpr/ArmeniSZJBFS16} dataset of PointNeXt-S and PointNeXt-XL, and two advanced pruning methods are evaluated.

\begin{table}[h!]
\centering
\caption{Comparisons of semantic segmentation performance on the S3DIS dataset (evaluated in Area-5) with PointNeXt-S~\cite{qian2022pointnext}.}
\label{seg_s3dis_supp_ps}
\resizebox{.95\linewidth}{!}{
\begin{tabular}{l|ccccc}
\toprule
\multirow{2.5}{*}{\textbf{Method}}
           & \multicolumn{5}{c}{\textbf{PointNeXt-S}}  \\ \cmidrule(lr){2-6}
           & \textbf{OA}    & \textbf{mAcc}  & \textbf{mIoU}  & \textbf{Params. (M)} & \textbf{GFLOPs ($\downarrow\%$)}  \\ \midrule
Baseline     &88.20   &70.70   &64.20   &0.80       &3.60 (--)   \\  \cmidrule(lr){1-6}
HRank        &85.89  &67.27   &60.49  &0.33     &1.53 (57.5)  \\
HRank+{\cpt} &86.18  &67.65   &61.04  &0.31     &1.52 (57.8)  \\
HRank        &84.92  &65.37   &58.73  &0.17     &0.77 (78.6)  \\
HRank+{\cpt} &85.12  &67.48   &60.12  &0.16     &0.74 (79.4)  \\ \cmidrule(lr){1-6}
CHIP         &84.45  &67.72   &61.16  &0.32     &1.53 (57.5)  \\
CHIP+{\cpt}  &84.53  &70.52   &63.62  &0.33     &1.48 (58.9)  \\
CHIP         &84.39  &67.29   &60.63  &0.15     &0.74 (79.4)  \\
CHIP+{\cpt}  &85.04  &69.02   &61.45  &0.16     &0.73 (79.7)  \\ \bottomrule
\end{tabular}}
\end{table}


\para{PointNext-S}
Compared to other PointNeXt zoos, besides the fewer parameters, PointNeXt-S is designed with no InvResMLP blocks and is a simpler network architecture.
%
The comparison result in Tab.~\ref{seg_s3dis_supp_ps} showed that the consistent outperformance of {\cpt} compared to directly using HRank and CHIP \textbf{without} \cpt.
%
In the case of applying CHIP with 57.5\% FLOPs reduction, by incorporating \cpt, the mAcc increases 2.8 \% while achieving 1.4 \% more FLOPs reduction.
%
The result indicated that even for a simpler network architecture, the accuracy performance degradation still occurred when directly implementation of CNN pruning methods, verifying the necessity of the implementation with \cpt.


\begin{table}[h!]
\centering
\caption{Comparisons of semantic segmentation performance on the S3DIS dataset (evaluated in Area-5) with PointNeXt-XL~\cite{qian2022pointnext}.}
\label{seg_s3dis_supp}
\resizebox{.95\linewidth}{!}{
\begin{tabular}{l|ccccc}
\toprule
\multirow{2.5}{*}{\textbf{Method}}
           & \multicolumn{5}{c}{\textbf{PointNeXt-XL}}  \\ \cmidrule(lr){2-6}
           & \textbf{OA}    & \textbf{mAcc}  & \textbf{mIoU}  & \textbf{Params. (M)} & \textbf{GFLOPs ($\downarrow\%$)}  \\ \midrule
Baseline     &91.00   &77.20    &71.10   &41.60   &84.80 (--)         \\  \cmidrule(lr){1-6}
HRank        &90.02   &74.50   &68.22  &13.03  &26.71 (68.5)      \\
HRank+{\cpt} &90.80   &75.85   &69.98  &12.57  &25.78 (69.6)      \\
HRank        &89.79   &74.40   &68.09  &8.80   &18.05 (78.7)      \\
HRank+{\cpt} &90.48   &74.45   &68.41  &8.42   &17.37 (79.5)      \\ \cmidrule(lr){1-6}
CHIP         &89.98   &75.43   &69.21  &17.54  &35.97 (57.6)      \\
CHIP+{\cpt}  &90.57   &75.90   &69.40  &17.00  &34.68 (59.1)      \\
CHIP         &89.15   &74.54   &68.07  &8.42   &17.37 (79.5)      \\
CHIP+{\cpt}  &90.03   &74.61   &68.26  &8.05   &16.62 (80.4)      \\ \bottomrule
\end{tabular}}
\end{table} 


\para{PointNext-XL}
We further investigated on the more complex and larger network PointNext-XL. Similar observations on the improvement by {{\cpt}} can be found in Tab.~\ref{seg_s3dis_supp}.
For instance, our approach can achieve 69.8 \% and 69.9  \% storage and computation reductions, respectively, with a 1.3 \% and 1.7\% accuracy increase for mAcc and mIoU over the baseline model.


\begin{table}[h]
\centering
\caption{Comparisons on the SemanticKITTI with RandLA-Net.}
\label{tab_2}
\resizebox{0.42\textwidth}{!}{
\begin{tabular}{l|ccc}
\toprule
\multicolumn{1}{l|}{\multirow{1}{*}{Method}}
& \multicolumn{1}{c}{mIoU} & \multicolumn{1}{c}{Params. (M)} & \multicolumn{1}{c}{FLOPs (\%)}
 \\ \midrule
Baseline (RandLA-Net)          &50.30   &0.95     &100                   \\ \cmidrule{1-4}
CHIP               &49.12  &0.20    &21.7                   \\
CHIP+CP$^3$        &50.21  &0.18    &19.8 \\ \bottomrule
\end{tabular}}
\end{table}

\para{Outdoor Experiments}
%
{\cpt} focuses on point-based networks (PNNs), and by following prevailing PNN works, we have experimented on the popular large-scale datasets such as ScanObjectNN and S3DIS and achieved promising results in the paper. To further illustrate the validity of our method, we experimented on a outdoor dataset (SemanticKITTI) with RandLA-Net and CHIP. The results are shown in Tab.~\ref{tab_2}.


\section{Exploration on Knowledge Recycling}
\label{supp:kr}

In this section, we took a deeper look into the Knowledge Recycling~(KR) module.
%
To verify the effectiveness of KR, we performed pruning methods and statistically analyzed the positive effect of KR.
We took the PointNeXt-S with dataset ModelNet40 as an example.
We performed the comparison between directly implementing HRank and HRank with \cpt.
%
And we focus on the KR scores generated by the KR module (with discarded points) and HRank scores \textbf{without} KR module.
We calculated the variances of scores to justify the robustness of \cpt.
Fig.~\ref{variance} shows the statistics comparison results on the 10-th layer with 512 channels on 2468 test meshed CAD models.
%
Among 512 channels in the 10-th layer, the percentage is 93.2\% in the case of the KR score variances lower than the original score variances. For instance, in the case of the 25-th channel, the KR score variance is much lower than the original score variance (0.21 vs. 0.89).
%
Similar results can be found on other layers. These results verify that the KR module enabled the channel importance calculation to be more stable and robust.


\section{Analysis of the layer-wise redundancy}
\label{supp:al}
We take the pruned PointNet++ on ScanObjectNN as an example and show the pruning rates for each layer in Fig.~\ref{fig_channel}.
Pruning with {\cpt} eliminates more redundancy on shallow layers and less on $6$th and $7$th layers.
{\cpt} on ResRep achieved higher accuracy (84.80\% vs 83.79\%) with higher FLOPs reduction (84.8\% vs 83.0\%), indicating pruning with CP$^3$ effectively identifies the redundancy in point-based networks.
\begin{figure}[t]
\centering
\includegraphics[width=0.45\textwidth]{figs/channel.pdf}
\caption{Different layer pruning rate. } \label{fig_channel}
\end{figure} 
