\UseRawInputEncoding
\documentclass[aps,prl,reprint]{revtex4-1}
\usepackage{blindtext}
\usepackage{babel}

\usepackage[utf8]{inputenc}
\usepackage{graphicx}% Include figure files
\usepackage{dcolumn}% Align table columns on decimal point
\usepackage{bm}% bold math
\usepackage{soul}
\newcommand{\vdag}{(v)^\dagger}

\usepackage{xcolor}
\newcommand{\gi}[1]{{\color{violet}#1}}   % Giulio
\newcommand{\fc}[1]{{\color{blue}#1}}  %Francesco
\newcommand{\lr}[1]{{\color{red}#1}} %Laurence
\newcommand{\gb}[1]{{\color{orange}#1}} %Gerard*
\newcommand{\bg}[1]{{\color{green}#1}} %Gerard

% \newcommand{\green}[1]{{\color{green}#1}} %Gerard
% \newcommand{\fcgreen}[1]{{\color{lime}#1}} %Gerard
% \newcommand{\lrgreen}[1]{{\color{olive}#1}} %Gerard

\DeclareUnicodeCharacter{00A0}{}

\newcommand{\figref}[1]{Fig. \ref{#1}}
\begin{document}
\title{Experimental evidence of the role of non-gyrotropy in magnetopause equilibrium}
\author{Giulio Ballerini}
\affiliation{LPP, CNRS/Sorbonne Université/Université Paris-Saclay/Observatoire de Paris/Ecole Polytechnique Institut Polytechnique
de Paris, Palaiseau, France}
\affiliation{Dipartimento di Fisica, University of Pisa, Italy}

\author{Gerard Belmont}
\affiliation{LPP, CNRS/Sorbonne Université/Université Paris-Saclay/Observatoire de Paris/Ecole Polytechnique Institut Polytechnique
de Paris, Palaiseau, France}

\author{Laurence Rezeau}
\affiliation{LPP, CNRS/Sorbonne Université/Université Paris-Saclay/Observatoire de Paris/Ecole Polytechnique Institut Polytechnique
de Paris, Palaiseau, France}


\author{Francesco Califano}
\affiliation{Dipartimento di Fisica, University of Pisa, Italy}

\begin{abstract}
Thin current layers separating different plasmas are common in space. In the simplest approximation, compressive and rotational characteristics of these discontinuities are mutually exclusive, apart in the case of a “tangential discontinuity”. This case requires the normal component of the magnetic field and the velocity with respect to the structure to be strictly null. However, mixed compressive and rotational characteristics are observed at the Earth magnetopause, although the assumption of a completely impermeable boundary is to be questioned.
In this letter, we discuss the role of anisotropy in discontinuities between different plasmas and show a typical example of Earth's magnetopause crossing data, which shows that Finite Larmor Radius effects have to be taken into account to describe such thin layers.

\end{abstract}

\maketitle


\noindent {In space physics, one observe a natural tendency of the medium to self-organize into distinct cells, separated by thin layers.} Examples can be given at several scales. Notable ones are planetary magnetospheres, which are bubbles in the solar wind flow and which are separated from it by bow shocks and magnetopauses \cite{parks_physics_2019, kivelson_introduction_1995, belmont_collisionless_2014}. The interaction of the solar wind with unmagnetized bodies as comets gives rise as well to similar bubbles \cite{coates_ionospheres_1997, bertucci_structure_2005}. The Solar System itself is a bubble in the flow of the local interstellar cloud, and it is separated from it by the terminal solar wind shock, the heliopause and a bow shock \cite{lallement_2001, richardson_observations_2022}. {Similar cells and thin layers can also form spontaneously, far from any boundary condition as in the context of a turbulent medium.}

At each boundary, the downstream and upstream physical quantities are always connected by the basic conservation laws: mass, momentum, energy and magnetic flux. In the simplest case, the number of conservation laws is equal to the number of parameters characterizing the plasma state. It therefore  determines in a unique way the possible downstream states as functions of the upstream one, whatever the phenomena, generally non-ideal, occurring inside the layer. In particular, it is possible to describe pressure variations without any closure equation. In this case, the jumps of all quantities are determined by a single scalar parameter (namely the "shock parameter" in neutral gas). Hereafter we will refer to the "Classic Theory of Discontinuities"  (CTD) as for the theory corresponding to this simplest case used both  for neutral mediums and magnetised plasmas. CTD  is characterized by the following simplifying assumptions: a stationary layer, 1D variation and isotropic pressure on both sides. For plasmas, the additional assumption of an ideal Ohm's law on both sides is considered \cite{belmont_introduction_2019}. When the hypothesis of isotropy is relaxed \cite{hudson_rotational_1971}, the number of conservation equations is no longer adequate to determine a unique downstream state for a given upstream one. As a consequence, the global result depends on the non-ideal phenomena occurring inside the layer. In the case of thin boundaries between different plasmas, even for a 1D stationary layer with the ideal Faraday/Ohm's law valid everywhere across the layer, one can suspect that Finite Larmor Radius (FLR) effects break the gyrotropy of the pressure tensor, thus providing the non-ideal effects accounting for the jumps. This paper provides experimental evidence concerning the crucial role of FLR effects for quasi-tangential layers at the terrestrial magnetopause.
The main results of the CTD have been known for decades \cite{landau_electrodynamics_1961, landau_fluid}. A more recent presentation provides a synthetic view of the main properties of the CTD solutions \cite{belmont_introduction_2019}. In CTD, the conservation laws provide a system of jump equations called Rankine-Hugoniot in neutrals and generalized Rankine-Hugoniot in plasmas. The systems used when calculating the linear modes from HD and MHD models much resemble these systems of jump equations just because the HD and MHD models rely on the same conservation laws as Rankine-Hugoniot and generalized Rankine-Hugoniot respectively. A direct consequence is that many properties are shared by the solutions of the two kinds of systems: linear modes and discontinuities. For a neutral medium, the linear sound wave solution corresponds to the well-known sonic shock solution, while for a magnetized plasma, the two magnetosonic waves correspond to the two main kinds of MHD shocks: fast and slow (although an additional discontinuity solution, the intermediate shock, has no linear counterpart). In addition, a non-compressional solution exists in both approaches: the linear shear Alfv\'en mode corresponding to the "rotational discontinuity" solution.\\
Focusing on plasma physics, an important feature of the CTD solutions (as well as of the linear MHD mode solutions) is that the compressional and rotational characters are mutually exclusive: the shock solutions are purely compressional, without any rotation of the tangential velocity or the tangential magnetic field (this is called the "coplanarity property"), while the rotational discontinuity does imply such a rotation but without any compression of the particle density or the magnetic field magnitude. This separation persists whatever the fluxes along the discontinuity normal, even when the normal components $V_n$ and $B_n$ of the velocity and the magnetic field are arbitrarily small. The only exception, which appears as a singular case in CTD, is the "tangential discontinuity", when both normal fluxes are strictly zero. We will investigate here the transitional case of "quasi-tangential discontinuities" when the normal fluxes tend toward zero without being strictly null. In this case, even small departures from CTD assumptions can be expected to draw increasing departures in the corresponding solution since all the usual terms of CTD equations, apart from those of the normal pressure equilibrium, are proportional to $V_n$ or $B_n$ and therefore tend to zero in quasi-tangential conditions.\\
In the solar wind, several authors have done statistics for a long time to determine the proportion, among the discontinuities observed, of the tangential and rotational ones, generally concluding that tangential discontinuities ($i.e.$ with $B_n$ small enough to be hardly measurable) are the most ubiquitous (see \cite{Colburn1966} for a pioneer work in this domain and \cite{Liu2022} and references therein for more recent contributions). In such statistics, rotational discontinuities are identified only when $B_n$ is large enough. Extending these studies in the range of small $B_n$, where all discontinuities are not necessarily "tangential discontinuities" in the CTD sense,  demands to investigate the quasi-tangential case. It is the goal of the present paper.\\
For testing experimentally the discontinuity theories, the terrestrial magnetopause has a pivotal role, thanks to \textit{in-situ} observations \cite{chou_statistical_2012}, which allow for a detailed description. Over its entire surface, this boundary presents both a rotation of the magnetic field \cite{sonnerup_magnetopause_1974}, a feature typical of rotational discontinuities, and variations of the density and of the magnetic field magnitude, usually considered as shock signatures \cite{dorville_rotationalcompressional_2014}. The very radical hypothesis of a magnetopause completely impermeable to mass and magnetic flux, with strictly null $V_n$ and $B_n$ everywhere and therefore quasi-independent media on both sides (apart from the normal pressure equilibrium), clearly deserves to be questioned both experimentally and theoretically. What is known is that at magnetopause-like boundaries the components $B_n$ and $V_n$ are always very small. This is observed experimentally at the terrestrial magnetopause, but these measurements usually cannot make the difference between $B_n=0$ and $B_n \simeq 0$, as there are always uncertainties, mainly due to the determination of the direction of the normal \cite{Rezeau2017,haaland2004, dorville_magnetopause_2015}.\\
The fact that the $B_n$ and $V_n$ components are always very small is actually related to the ideal MHD frozen-in property, which governs the flow at large scales in magnetized plasmas and which is responsible for the very existence of these boundaries: if this property was valid everywhere, including inside the boundaries, it would prevent the different media from mixing, constraining the media on each side of a boundary to remain on its own side, making it fully impermeable.
However, in the same way that a neutral shock cannot be infinitely thin due to viscosity, these boundaries cannot be infinitely thin either, their width being regulated by ion and electron kinetic effects (that come into play here well before collisions). For "small" boundary thicknesses the frozen-in flux property can be violated inside the layer, which allows for small fluxes across it, therefore making it non fully-impermeable. In the absence of collisions, the implied small scales are the collisionless ion scales: the ion inertial length $d_i$ and the Larmor radius $\rho_i$. One can guess that the thickness of the boundary adapts itself to these non-ideal "collisionless" scales.
At those scales, the ideal Faraday/Ohm's law can be violated and the pressure tensor can become non-gyrotropic (FLR effects). As mentioned above, out of the CTD conditions, in particular for anisotropic plasmas, the jump from upstream to downstream depends on non-ideal effects inside the layer.  The possible types of discontinuities in an anisotropic plasma have been discussed in several papers \cite{lynn_discontinuities_1967,abraham-shrauner_propagation_1967, chao_interplanetary_1970, neubauer_jump_1970}, and the present one improves the analysis in the light of the new experimental possibilities.
Even if the phenomena invoked are small-scale, they have significant feed-backs on the large-scale properties of the discontinuities.  This problem closely resembles that of magnetic reconnection where non-ideal, small-scale effects violating the frozen-in condition have a strong impact on the large-scale ideal-MHD dynamics. The present problem is however different since non-ideal effects invoked here, under stationary and one-dimensional conditions, are not related to any X-point geometry and are not the driver of any significant change in magnetic connectivity.\\
The letter is organised as follow: Section 1 focuses on the theoretical study of  the role of kinetic effects in quasi-tangential discontinuities while in Section 2 we analyse a crossing of the terrestrial magnetopause by the Magnetospheric Multiscale (MMS) Mission \cite{burch_phan_2016}.

\section{The role of pressure}
In CTD, the separation between the compressional and rotational discontinuities arises from only two equations, the tangential projections of the momentum equation:
\begin{equation}
\rho\frac{d \mathbf{v}}{d t}+\nabla\cdot\mathbf{P}_i+\nabla\cdot\mathbf{P}_e=\mathbf{J}\times \mathbf{B}
\end{equation}
and the Faraday/Ohm's law:
\begin{equation}
    \mathbf{\nabla}\times \mathbf{E}=-\frac{\partial \mathbf{B}}{\partial t}
\end{equation}
\begin{equation}
    \mathbf{E}+\mathbf{v}\times\mathbf{B}=\frac{1}{ne}\mathbf{J}\times\mathbf{B}-\frac{1}{ne}\nabla\cdot\mathbf{P}_e
\end{equation}
These equations lead to:
\begin{equation}
    (V_{n2} -V_{n0})\mathbf{B}_{t2} = (V_{n1} -V_{n0})\mathbf{B}_{t1}
    \label{eq:isotropic}
\end{equation}
where:
\begin{equation}
   V_{n0}=\frac{B_{n}^2} {\mu_0 \rho V_n} = \mathrm{cst}
\end{equation}

The $n$ and $t$ indices indicate the normal and the tangential component, respectively, while the indices 1 and 2 indicate the upstream and downstream regions, respectively. Moreover, $V_n$ indicates the plasma normal velocity across the layer, in the frame where this one is stationary. Equation (\ref{eq:isotropic}) directly leads to the distinction between shocks, where the direction of the tangential magnetic field is unchanged between the two mediums, and rotational discontinuities, which requires the terms inside the parentheses to be equal to zero. Rotational discontinuities correspond to a propagation velocity equal to the normal Alfv\'en velocity, and it implies $V_{n1}=V_{n2}=V_{n0}$, $i.e.$ an absence of compression of the plasma.

In the CTD treatment, the pressure tensor is assumed to be isotropic so that the divergence of the pressure tensor in the momentum equation is purely along the normal direction. It is the reason why the $\nabla \cdot \mathbf{P}$ term does not appear in Eq. \ref{eq:isotropic} from which both rotational discontinuities and shock solutions are derived. This term comes into play if conditions are met for the pressure tensor to become anisotropic, as shown in \cite{hudson_rotational_1971}, and \textit{a fortiori} in the non-gyrotropic case.

In the anisotropic case, it has been shown in paper \cite{hudson_rotational_1971} that the $\nabla \cdot \mathbf{P}$ term introduces a new coefficient:
\begin{equation}
    \alpha = 1-\frac{p_{\parallel}-p_{\perp}}{ B^2 / \mu_0}
    \label{alpha}
\end{equation}

This coefficient has been interpreted as a change in the Alfv\'en velocity $V'^2_{An}=\alpha V_{An}^2$, but it appears more basically as a change in Eq.(\ref{eq:isotropic}):
\begin{equation}
    (V_{n2} -\alpha_2 V_{n0})\mathbf{B}_{t2} = (V_{n1} -\alpha_1 V_{n0})\mathbf{B}_{t1}
    \label{eq:anisotropic}
\end{equation}

This equation shows that, in the anisotropic case, coplanar solutions are still present ($\mathbf{B}_{t2}$ and $\mathbf{B}_{t1}$ collinear), but that the equivalent of the rotational discontinuity now implies compression whenever $\alpha_2$ is not equal to $\alpha_1$:

\begin{equation}
    V_{n2} =\alpha_2 V_{n0} \ne  V_{n1} =\alpha_1 V_{n0}
    \label{eq:rotational_anisotropic}
\end{equation}
The variation of $V_n$ explains why the modified rotational discontinuity can be "evolutionary"\cite{jeffrey_non-linear_1964}, the non linear steepening being counter-balanced at equilibrium by some internal non-ideal effects for a thickness comparable with the characteristic scales of these effects.

There is actually no additional conservation equation available that would allow determining the jump of the anisotropy coefficient $\alpha$. Consequently, there is no universal result that would give the downstream state as a function of the upstream one, independently of the phenomena inside the layer. Since the layers are thin with respect to the ion scales  ({$L \sim d_i$, $L \sim \rho_i$, $L$ being the characteristic length}), the FLR effects, which make the pressure tensor non-gyrotropic, are to be taken into account to describe consistently these internal phenomena. Such effects have been already reported in the context of reconnection \cite{aunai_electron_2013, aunai_proton_2011} and in kinetic modeling of purely tangential layers \cite{belmont_kinetic_2012, dorville_asymmetric_2015} but they have never been treated yet in the context of quasi-tangential discontinuities and we will show in the next sections that it is actually crucial in this case.

\section{Data}
In order to investigate the variations within the magnetopause layer of the different plasma quantities, we use data from the Magnetospheric Multiscale (MMS) Mission \cite{burch_phan_2016}. For a given time interval inside the magnetopause, the normal direction is obtained by a method derived from the Minimum Derivative (MDD) method \cite{shi_dimensional_2005,Rezeau2017, denton2021}. In particular, when using the magnetic field measurements, the MDD method is based on the experimental estimation of the tensor $\mathbf{G} = \mathbf{\nabla} \mathbf{B}$ from multispacecraft measurements and the derivation of the normal direction from the properties of this tensor via a fit procedure. Furthermore, one can also use particle data such as $\rho_i \mathbf{V}_i$, which provide an independent determination of the normal. Different constraints can be invoked in the fit procedure in order to restrict as much as possible the normal uncertainty (such as $\mathbf{\nabla}\cdot\mathbf{B}=0$). Thanks to the MDD method and to the high quality of the MMS data, a normal to the structure can be obtained with an unprecedented accuracy, which is necessary when studying the normal components $V_n$ and $B_n$. Indeed, due to their small values with respect to the magnitude of the velocity and the magnetic field, respectively, reaching a very high degree of accuracy is a pivotal element for the success of the analysis. Furthermore, this normal  is calculated at each time step inside the magnetopause, allowing one to observe its local variations. The tangential directions are obtained at the same time instants. 
\begin{figure}[h]
\includegraphics[width=\columnwidth]{Hodogram_frecce.png}% Here is how to import EPS art
\caption{Hodogram in the tangential plane of the magnetic field for a magnetopause crossing by MMS in 28.12.2015 from 22:12:04 to 22:12:08. See text for the significance of the arrows. }
\label{fig:hodogram}
\end{figure}
Several crossings have been analyzed. For each one we first study the hodogram of the magnetic field in the tangential plane with respect to the structure. In this plane, a rotational discontinuity would correspond to a circular arc with constant radius while a shock would correspond to a radial line. In several crossings, however, a linear (yet not radial) variation is observed as shown in Figure \ref{fig:hodogram} for the crossing of the 28.12.2015 at 22:12:04 which we consider typical. This non-radial variation of the magnetic field is quite interesting since it is not predicted by CTD.

We note that the time interval corresponds to the intersection of the intervals when both the ion mass flux and the magnetic field have a 1D structure with the same normal, using the dimensionality parameters presented in \cite{Rezeau2017}. This underlines that this observed property is not likely to be due to a  non-unidimensional geometry of the layer.

To further analyze the causes of this disagreement, we must compare the different terms of the tangential momentum equation and Faraday/Ohm's law. This is the object of Figure \ref{fig:Ohm_momentum}. The normal components are also shown for reference.

Concerning the Ohm's law (Figure \ref{fig:Ohm_momentum}a), we see that the electric field is well counter-balanced by the $\mathbf{U}\times\mathbf{B}$ and $\mathbf{J}\times\mathbf{B}/nq$ terms (ideal and Hall terms), while the $\nabla\cdot \mathbf{P}_e$ is negligible in all directions. Outside the layer, on both sides, the ideal Ohm's law is verified, as assumed in CTD. \\
Concerning the momentum equation (Figure \ref{fig:Ohm_momentum}b), we observe that, in the normal direction, the $\mathbf{J}\times\mathbf{B}$ term is counter-balanced by the divergence of the ion pressure, as expected. But, if the isotropic condition assumed in CTD was valid, we would expect the divergence of the ion pressure tensor to be null in the two tangential directions, or at least negligible with respect to the inertial term $\rho d\mathbf{V}/dt$. On the contrary, we observe in these two directions that the $\mathbf{J}\times\mathbf{B}$ term is of the same order of magnitude as the divergence of the ion pressure tensor, and one order of magnitude larger than all the other terms. It can be noticed that, due to the limited accuracy attainable when estimating the pressure tensor divergence, the force equilibrium is not exactly verified everywhere. Despite this, since the inertial term is one order of magnitude lower than the $\mathbf{J} \times \mathbf{B}$ term, these results prove that the $\nabla \cdot \mathbf{P}_i$ term plays a fundamental role in the magnetopause equilibrium. 
This point can be emphasized also by analyzing the hodogram of Figure \ref{fig:hodogram}. The arrows are along the directions of the tangential plane that are physically relevant for the problem: - the tangent to the hodogram (green), which indicates the total variation of $\mathbf B_t$; - the radial direction (red), which corresponds to the plasma compression; - the $\nabla \cdot \mathbf{P}_{it}$ direction (blue), which is absent in CTD. These directions are averaged in two sub-intervals (bold hodogram). The striking result is that the total variation is mainly determined by the non-classic term $\nabla \cdot \mathbf{P}_{it}$ and not by the radial classic one. This explains the very recurrent (even if not reported in the literature hitherto) feature that the hodograms are almost linear but not radial.


\begin{figure*}
\includegraphics[width=\textwidth]{SingleFig.png}% Here is how to import EPS art
\caption{\label{fig:Ohm_momentum} Terms of ($a$) the Ohm equation (units of m$V/m$) and ($b$) the momentum equation (units of $10^{-15}kg\,m/s^2$), projected in the normal direction (\textit{n}) and in the two tangential directions ($T_1$ and $T_2$).  To reduce the noise, a running average with a time window of 0.35s is done on the electric field measurements. Dotted line represents the sum of all the other terms, which is smaller that other terms in most of the time interval in all three directions.. \textit{N.b.} The terms of the tangential Faraday/ Ohm's law used in the text are just the derivatives of the ones in ($a$) (apart from a $\pi/2$ rotation).}
\end{figure*}
\section{Conclusions}
In this letter, the transitional case of a “quasi-tangential” discontinuity has been investigated, unveiling the crucial role of the full pressure tensor in the layer equilibrium. We emphasize that, in case of anisotropy, the phenomena occurring inside the layer play a fundamental role to determine the relation between downstream and upstream quantities. In particular, for thin current layers, the FLR effects, which make the pressure tensor non-gyrotropic, have to be taken into account. We present here the results for a reference crossing. This can be considered as representative since the features described here have been observed in most of the other cases that we have analyzed, (including the most documented case of 16.10.2015, already analyzed in \cite{burch_phan_2016,Rezeau2017,manuzzo2019}). For this reference crossing, the "linear" hodogram in the tangential plane shows how the boundary escapes the CTD results. This discrepancy with CTD can be then explained by studying the tangential components of the momentum equation, which evidences the role of the pressure tensor. This opens to the importance of the FLR effects in quasi-tangential discontinuities. It is well-known that the linear version of the rotational discontinuity is the MHD shear Alfv\'en wave. It appears here that the magnetopause-like "quasi-tangential" discontinuities correspond in the same way to the quasi-perpendicular "Kinetic Alfv\'en Waves"  \cite{Hasegawa,belmont_rezeau1987,cramer}.

%\pagebreak
\bibliographystyle{apsrev4-1} % Tell bibtex which bibliography style to use
\bibliography{1702.bib}



\end{document}


