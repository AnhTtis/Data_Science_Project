%%%%%%%%%%%%%%%%%%%%%%%%%%%%%%%%%%%%%%%%%%%%%%%%%%%%%%%%%%%%%%%%%%%%%%%%%%%%
% AGUJournalTemplate.tex: this template file is for articles formatted with LaTeX
%
% This file includes commands and instructions
% given in the order necessary to produce a final output that will
% satisfy AGU requirements, including customized APA reference formatting.
%
% You may copy this file and give it your
% article name, and enter your text.
%
%
% Step 1: Set the \documentclass
%
%

%% To submit your paper:
\documentclass[draft]{agujournal2019}
\usepackage{url} %this package should fix any errors with URLs in refs.
\usepackage{lineno}
\usepackage[inline]{trackchanges} %for better track changes. finalnew option will compile document with changes incorporated.
\usepackage{soul}
\usepackage{blindtext}
\usepackage{babel}

\usepackage[utf8]{inputenc}
\usepackage{graphicx}% Include figure files
\usepackage{dcolumn}% Align table columns on decimal point
\usepackage{bm}% bold math
\usepackage{soul}
\newcommand{\vdag}{(v)^\dagger}

\usepackage{xcolor}
\newcommand{\gi}[1]{{\color{violet}#1}}   % Giulio
\newcommand{\fc}[1]{{\color{blue}#1}}  %Francesco
\newcommand{\lr}[1]{{\color{red}#1}} %Laurence
\newcommand{\gb}[1]{{\color{orange}#1}} %Gerard*
\newcommand{\gbd}[1]{{\color{orange}}\st{#1}} % deleted by Gerard
%\newcommand{\bg}[1]{{\color{green}#1}} %Gerard
%\linenumbers
%%%%%%%
% As of 2018 we recommend use of the TrackChanges package to mark revisions.
% The trackchanges package adds five new LaTeX commands:
%
%  \note[editor]{The note}
%  \annote[editor]{Text to annotate}{The note}
%  \add[editor]{Text to add}
%  \remove[editor]{Text to remove}
%  \change[editor]{Text to remove}{Text to add}
%
% complete documentation is here: http://trackchanges.sourceforge.net/
%%%%%%%

\draftfalse

%% Enter journal name below.
%% Choose from this list of Journals:
%
% JGR: Atmospheres
% JGR: Biogeosciences
% JGR: Earth Surface
% JGR: Oceans
% JGR: Planets
% JGR: Solid Earth
% JGR: Space Physics
% Global Biogeochemical Cycles
% Geophysical Research Letters
% Paleoceanography and Paleoclimatology
% Radio Science
% Reviews of Geophysics
% Tectonics
% Space Weather
% Water Resources Research
% Geochemistry, Geophysics, Geosystems
% Journal of Advances in Modeling Earth Systems (JAMES)
% Earth's Future
% Earth and Space Science
% Geohealth
%
% ie, \journalname{Water Resources Research}

\journalname{Enter journal name here}


\begin{document}

%% ------------------------------------------------------------------------ %%
%  Title
%
% (A title should be specific, informative, and brief. Use
% abbreviations only if they are defined in the abstract. Titles that
% start with general keywords then specific terms are optimized in
% searches)
%
%% ------------------------------------------------------------------------ %%

% Example: \title{This is a test title}

\title{Experimental evidence of the role of non-gyrotropy in magnetopause equilibrium}


\authors{Giulio Ballerini\affil{1,2}, Gerard Belmont\affil{1}, Laurence Rezeau\affil{1}, Francesco Califano\affil{2}}

\affiliation{1}{LPP, CNRS/Sorbonne Université/Université Paris-Saclay/Observatoire de Paris/Ecole Polytechnique Institut Polytechnique de Paris, Paris, France}
\affiliation{2}{Dipartimento di Fisica, University of Pisa, Italy}
%(repeat as many times as is necessary)

%% Corresponding Author:
% Corresponding author mailing address and e-mail address:

% (include name and email addresses of the corresponding author.  More
% than one corresponding author is allowed in this LaTeX file and for
% publication; but only one corresponding author is allowed in our
% editorial system.)

% Example: \correspondingauthor{First and Last Name}{email@address.edu}

\correspondingauthor{Giulio Ballerini}{giulio.ballerini@sorbonne-universite.fr}

%% Keypoints, final entry on title page.

%  List up to three key points (at least one is required)
%  Key Points summarize the main points and conclusions of the article
%  Each must be 140 characters or fewer with no special characters or punctuation and must be complete sentences



\begin{keypoints}
\item Open or close Magnetopause
\item Rotational and tangential Discontinuities
\item Pressure tensor and Finite Larmor Radius effects
\end{keypoints}



\begin{abstract}

Is the magnetosphere closed or open?  This is one of the oldest questions in space physics still at the basis of the hottest topics in the field today. With respect to the magnetopause boundary, this translates into: "is this boundary a tangential discontinuity or not?". We revisit this question, and show that the small thickness of the magnetopause escapes classic theory, from which comes the notion of tangential discontinuity. For such "quasi-tangential" discontinuities, both compressive and rotational properties coexist as observed, without requiring strictly null normal fluxes, while they are mutually exclusive when Finite Larmor radius effects are ignored. This change is similar to the change from a shear Alfvén wave to a Kinetic Alfvén wave for linear modes.
We present a typical example of MMS data from a magnetopause crossing. We show that the non-gyrotropic pressure tensor does indeed play a very important role in the boundary equilibrium.


\end{abstract}

\section{Introduction}
\noindent {In space physics, there is a natural tendency of the medium to self-organize into distinct cells, separated by thin layers.} Examples can be given at various scales. Notable examples are planetary magnetospheres, which are bubbles in the solar wind stream and which are separated from it by bow shocks and magnetopauses \cite{parks_physics_2019, kivelson_introduction_1995, belmont_collisionless_2014}. The interaction of the solar wind with unmagnetized bodies such as comets also produces similar bubbles \cite{coates_ionospheres_1997, bertucci_structure_2005}. The Solar System itself is a bubble in the flow of the local interstellar cloud, and it is separated from it by the terminal solar wind shock, the heliopause and a bow shock \cite{lallement_2001, richardson_observations_2022}. {Similar cells and thin layers can also form spontaneously, far from any boundary condition as in the context of a turbulent medium.}

At each boundary, the downstream and upstream physical quantities are always linked by the fundamental conservation laws: mass, momentum, energy and magnetic flux. In the simplest case, the number of conservation laws is equal to the number of parameters characterizing the plasma state. It therefore uniquely determines the possible downstream states as a function of the upstream state, regardless of the generally non-ideal phenomena occurring within the layer. In particular, it is possible to describe pressure variations without any closure equation. In this case, the jumps of all quantities are determined by a single scalar parameter (namely the "shock parameter" in neutral gas). In the following, we will refer to the "Classic Theory of Discontinuities" (CTD) as for the theory corresponding to this simplest case, which is used both for neutral media and magnetized plasmas. CTD  is characterized by the following simplifying assumptions: a stationary layer, 1D variation and isotropic pressure on both sides. For plasmas, the additional assumption of an ideal Ohm's law on both sides is considered \cite{belmont_introduction_2019}. When the isotropy hypothesis is relaxed \cite{hudson_rotational_1971}, the number of conservation equations is no longer sufficient to determine a unique downstream state for a given upstream one. As a consequence, the global result depends on the non-ideal phenomena occurring within the layer. In addition to the simple anisotropy effects,  Finite Larmor Radius (FLR) effects can be expected to break the gyrotropy of the pressure tensor in the case of thin boundaries between different plasmas,  and thus provide the non-ideal effects that account for the jumps, even for a 1D stationary layer with the ideal Faraday/Ohm's law valid throughout the layer.  \\
This paper provides experimental evidence concerning the crucial role of FLR effects for quasi-tangential layers at the terrestrial magnetopause. The main results of the CTD have been known for decades \cite{landau_electrodynamics_1961, landau_fluid}. A more recent presentation gives a synthetic view of the main properties of the CTD solutions \cite{belmont_introduction_2019}. In CTD, the conservation laws provide a system of jump equations called Rankine-Hugoniot in neutrals and generalized Rankine-Hugoniot in plasmas. The equation systems used to compute the linear modes from HD and MHD models are very similar to these jump equation systems simply because the HD and MHD models rely on the same conservation laws as Rankine-Hugoniot and generalized Rankine-Hugoniot respectively. A direct consequence is that many properties are shared by the solutions of the two types of systems: linear modes and discontinuities. For a neutral medium, the linear sound wave solution corresponds to the well-known sonic shock solution, while for a magnetized plasma, the two magnetosonic waves correspond to the two main types of MHD shock: fast and slow (although an additional discontinuity solution, the intermediate shock, has no linear counterpart). In addition, a non-compressional solution exists in the two types of systems: the linear shear Alfv\'en mode corresponding, in MHD, to the "rotational discontinuity" solution.\\
Focusing on plasma physics, an important feature of the CTD solutions (as well as of the linear MHD mode solutions) is that the compressional and rotational characters are mutually exclusive: the shock solutions are purely compressional, without any rotation of the tangential velocity or the tangential magnetic field (this is called the "coplanarity property"), while the rotational discontinuity does imply such a rotation but without any compression of the particle density or the magnetic field magnitude. This separation persists whatever the fluxes along the discontinuity normal, even when the normal components $V_n$ and $B_n$ of the velocity and the magnetic field are arbitrarily small. The only exception, which appears as a singular case in CTD, is the "tangential discontinuity", when both normal fluxes are strictly zero. We will investigate here the transitional case of "quasi-tangential discontinuities" when the normal fluxes tend toward zero without being strictly null. In this case, even small departures from CTD assumptions can be expected to draw increasing departures in the corresponding solution since all the usual terms of CTD equations, apart from those of the normal pressure equilibrium, are proportional to $V_n$ or $B_n$ and therefore tend to zero in quasi-tangential conditions.\\
In the solar wind, several authors have performed statistics for a long time to determine the proportion of tangential and rotational discontinuities among the observed discontinuities, generally concluding that tangential discontinuities ($i.e.$ with $B_n$ small enough to be barely measurable) are the most ubiquitous (see \cite{Colburn1966}, for a pioneering work in this domain and \cite{Liu2022}, and references therein for more recent contributions). In such statistics, rotational discontinuities are only identified  when $B_n$ is large enough. Extending these studies in the range of small $B_n$, where all discontinuities are not necessarily "tangential discontinuities" in the CTD sense,  requires the study of the quasi-tangential case. This is the aim of the present paper.\\
For the experimental testing of the discontinuity theories, the terrestrial magnetopause has a pivotal role, thanks to \textit{in-situ} observations \cite{chou_statistical_2012} that allow for a detailed description. 
The usual paradigm is that the magnetopause is always a tangential discontinuity (with  $B_n$ and $V_n$ strictly null everywhere), and that it becomes "open" only exceptionally at a few points where reconnection occurs, making 2D the magnetopause structure. The existence of open zones is indeed necessary since it allows a small part of the solar wind to penetrate the magnetosphere, leading to the variety of magnetospheric phenomena observed, such as substorms and auroras. 
This paradigm is based on CTD. We will show, experimentally and theoretically, why this theory fails at the magnetopause, leading to a questioning of this paradigm: rotation and compression can coexist even in the 1D case, with finite $B_n$ and $V_n$. Such a paradigm change may be reminiscent of a similar improvement in the theoretical modeling of the magnetotail in the 70's studies. In that case the authors demonstrated that even a very weak component of the magnetic field across the current layer was sufficient to completely modify the stability properties of the plasmasheet, so that the finite value of $B_n$ was to be taken into account, contrary to the pioneer versions of the tearing instability theories(\cite{coppi_dynamics_1966, galeev_reconnection_1979} and refferences therein).   

It is known that this boundary exhibits, over its entire surface,  both a rotation of the magnetic field \cite{sonnerup_magnetopause_1974} and a density variation \cite{Otto2005}, features typical of rotational discontinuities and shocks respectively, which are mutually exclusive in the CTD with the exception of the tangential case \cite{dorville_rotationalcompressional_2014}. Does it mean that it justifies the very radical hypothesis of a magnetopause completely impermeable to mass and magnetic flux, with strictly null $V_n$ and $B_n$ and quasi-independent media on both sides (apart from the normal pressure equilibrium)? What is known is that at magnetopause-like boundaries the components $B_n$ and $V_n$ are always very small. This is observed experimentally at the terrestrial magnetopause, but these measurements usually cannot tell the difference between $B_n=0$ and $B_n \simeq 0$, because there are always uncertainties, mainly due to the determination of the direction of the normal \cite{Rezeau2017,haaland2004, dorville_magnetopause_2015}.\\
The fact that the $B_n$ and $V_n$ components are always very small is actually related to the ideal MHD frozen-in property that governs the flow at large scales in magnetized plasmas and which is responsible for the very existence of these boundaries: if this property was valid everywhere, including inside the boundaries, it would prevent the different media from mixing, forcing the media on each side of a boundary to remain on their own side, making it completely impermeable.
However, in the same way that a neutral shock cannot be infinitely thin due to viscosity, these boundaries cannot be infinitely thin either, their width being regulated by ion and electron kinetic effects (which come into play here well before collisions). For "small" boundary thicknesses the frozen-in flux property can be violated inside the layer, allowing for small fluxes across the layer, and thus making it non-completely impermeable. In the absence of collisions, the implied small scales are the collisionless ion scales: the ion inertial length $d_i$ and the Larmor radius $\rho_i$. One can guess that the thickness of the boundary will adapt itself to these non-ideal "collisionless" scales.
At these scales, the ideal Faraday/Ohm's law can be violated and the pressure tensor can become non-gyrotropic (FLR effects). As mentioned above, out of the CTD conditions, especially for anisotropic plasmas, the jump from upstream to downstream depends on non-ideal effects within the layer.  The possible types of discontinuities in an anisotropic plasma have been discussed in several papers \cite{lynn_discontinuities_1967,abraham-shrauner_propagation_1967, chao_interplanetary_1970, neubauer_jump_1970}, and the present paper improves the analysis in the light of the new experimental possibilities.
Even if the invoked phenomena are small-scale, they have significant feedback on the large-scale properties of the discontinuities. This problem closely resembles that of magnetic reconnection where non-ideal, small-scale effects violating the frozen-in condition have a strong impact on the large-scale ideal-MHD dynamics. However, the present problem is different since the non-ideal effects invoked here, under stationary and one-dimensional conditions, are not related to any X-point geometry and are not the driver of any significant change in magnetic connectivity.\\
The letter is organised as follows: Section 1 focuses on the theoretical study of  the role of kinetic effects in quasi-tangential discontinuities while in Section 2 we analyse a crossing of the terrestrial magnetopause by the Magnetospheric Multiscale (MMS) Mission \cite{burch_phan_2016}.

\section{The role of pressure}
In CTD, only the tangential projection of two equations, provide the separation between the compressional and rotational discontinuities,the momentum equation:
\begin{equation}
\rho\frac{d \mathbf{v}}{d t}+\nabla\cdot\mathbf{P}_i+\nabla\cdot\mathbf{P}_e=\mathbf{J}\times \mathbf{B}
\end{equation}
and the Faraday/Ohm's law:
\begin{equation}
    \mathbf{\nabla}\times \mathbf{E}=-\frac{\partial \mathbf{B}}{\partial t}
\end{equation}
\begin{equation}
    \mathbf{E}+\mathbf{v}\times\mathbf{B}=\frac{1}{ne}\mathbf{J}\times\mathbf{B}-\frac{1}{ne}\nabla\cdot\mathbf{P}_e
\end{equation}
These equations link the upstream and downstream quantities as expressed by the following relationships:
\begin{equation}
    (V_{n2} -V_{n0})\mathbf{B}_{t2} = (V_{n1} -V_{n0})\mathbf{B}_{t1}
    \label{eq:isotropic}
\end{equation}
where:
\begin{equation}
   V_{n0}=\frac{B_{n}^2} {\mu_0 \rho V_n} = \mathrm{cst}
\end{equation}

The $n$ and $t$ indices indicate the normal and the tangential component, respectively. The indices 1 and 2 indicate the upstream and downstream regions, respectively. Moreover, $V_n$ indicates the normal plasma velocity across the layer, in the frame where it is stationary. Equation (\ref{eq:isotropic}) directly leads to the distinction between shocks, where the tangential magnetic field direction is unchanged between the two mediums, and rotational discontinuities, where the terms inside the brackets to be equal to zero. Rotational discontinuities correspond to a propagation velocity equal to the normal Alfv\'en velocity, and imply $V_{n1}=V_{n2}=V_{n0}$, that is an absence of compression of the plasma.

In the CTD treatment, the pressure tensor is assumed to be isotropic so that the divergence of the pressure tensor in the momentum equation is purely along the normal direction. This is the reason why the $\nabla \cdot \mathbf{P}$ term does not appear in Eq. \ref{eq:isotropic} from which both rotational discontinuities and shock solutions are derived. This term comes into play when conditions are met for the pressure tensor to become anisotropic, as shown in \cite{hudson_rotational_1971}, and \textit{a fortiori} in the non-gyrotropic case.

In the simple anisotropic case, without non-gyrotropy, it was shown in the paper \cite{hudson_rotational_1971} that the $\nabla \cdot \mathbf{P}$ term just introduces a new coefficient:
\begin{equation}
    \alpha = 1-\frac{p_{\parallel}-p_{\perp}}{ B^2 / \mu_0}
    \label{alpha}
\end{equation}

This coefficient has been interpreted as a change in the Alfv\'en velocity $V'^2_{An}=\alpha V_{An}^2$, but it appears more basically as a change in Eq.(\ref{eq:isotropic}):
\begin{equation}
    (V_{n2} -\alpha_2 V_{n0})\mathbf{B}_{t2} = (V_{n1} -\alpha_1 V_{n0})\mathbf{B}_{t1}
    \label{eq:anisotropic}
\end{equation}

This equation shows that, in this relatively simple anisotropic case, coplanar solutions still exist ($\mathbf{B}_{t2}$ and $\mathbf{B}_{t1}$ collinear), but that whenever $\alpha_2$ is not equal to $\alpha_1$, the equivalent of the rotational discontinuity now implies compression:

\begin{equation}
    V_{n2} =\alpha_2 V_{n0} \ne  V_{n1} =\alpha_1 V_{n0}
    \label{eq:rotational_anisotropic}
\end{equation}
The variation of $V_n$ explains why the modified rotational discontinuity can be "evolutionary"\cite{jeffrey_non-linear_1964}, the non linear steepening being counter-balanced at equilibrium by some internal non-ideal effects for a thickness comparable with the characteristic scales of these effects.

There is actually no additional conservation equation available that would allow the jump of the anisotropy coefficient $\alpha$ to be determined. Consequently, there is no universal result that gives the downstream state as a function of the upstream one, independently of the phenomena within the layer. This remains valid for the full anisotropic case, with non-gyrotropy. As soon as the the two ion scales $d_i$ and $\rho_i$ are not negligible with respect to the layer characteristic scale $L$, the FLR effects, which make the pressure tensor non-gyrotropic, must be taken into account to describe self-consistently the internal phenomena. Then, the effect of the divergence of the pressure tensor is no longer reduced to the introduction of a coefficient $\alpha$ since its tangential component is no longer collinear with $B_t$. Such effects have been already reported and analyzed in the context of reconnection \cite{aunai_electron_2013, aunai_proton_2011} and in kinetic modeling of purely tangential layers \cite{belmont_kinetic_2012, dorville_asymmetric_2015}. They have also been investigated in the case of linear modes, where they are known to change the shear Alfvén wave into the Kinetic Alfvén Wave \cite{Hasegawa,belmont_rezeau1987,cramer}. On the other hand, they have never been treated yet in the context of quasi-tangential discontinuities. We will show in the next sections that this point is actually crucial.

\section{Data}
In order to study the behaviour of the different plasma quantities within the magnetopause layer, we use data from the Magnetospheric Multiscale (MMS) mission \cite{burch_phan_2016}. The FluxGate Magnetometers (FGM) \cite{russell_magnetospheric_2016} provides the magnetic field data, the Electric Double Probe \cite{lindqvist_spin-plane_2016, ergun_axial_2016} those of the electric field, the plasma data are measured by the Dual Ions and Electrons Spectrometer instrument (DIS, DES)\cite{pollock_fast_2016}.  For a given time interval, the normal direction inside the magnetopause is obtained by a method derived from the Minimum Directional Derivative (MDD) technique \cite{shi_dimensional_2005,Rezeau2017, denton2021}. In this method, the calculation is based on the experimental estimation of the dyadic tensor $\mathbf{G} = \mathbf{\nabla} \mathbf{B}$ ($\textbf B$ can be the magnetic field vector or any other vector of the data). This tensor gradient can be obtained from multispacecraft measurements using the reciprocal vector method \cite{chanteur_spatial_1998}. In the original MDD method, the normal direction is obtained at each time step as the eigenvector corresponding to the maximum eigenvalue of the $\mathbf{G} \cdot \mathbf{G}^T$ matrix. Nevertheless, the small scale fluctuations, due to waves and turbulence, may invalidate the estimation of $\textbf G$ and cause strong variations in the normal obtained. To avoid this, we have embedded the original MDD method in a fitting procedure with short running windows. With this procedure we also add a constraint to the fit. In particular, when using the magnetic field data, the new constraint corresponds to impose $\mathbf{\nabla}\cdot\mathbf{B}=0$. We so reduce as much as possible the uncertainty on the normal direction.   

In addition, one can use also particle data such as $\rho_i \mathbf{V}_i$, providing an independent way to determine the normal.  If the two normals are not identical, it shows that the geometry of the particle structure is different from the structure of the magnetic field one. We will restrict here to intervals where the two normals are similar enough, a signature that the whole physics is really 1D. Thanks to the MDD method and to the high quality of the MMS data, a normal to the structure is so obtained with an unprecedented accuracy. Reaching such a high degree of accuracy is actually a pivotal element for the success of the analysis because the normal components $V_n$ and $B_n$ have very small values with respect to the magnitudes of the velocity and the magnetic field, respectively. Furthermore, this normal  is calculated at each time step inside the magnetopause, allowing one to observe local variations. The tangential directions are obtained at the same time instants from the normal vectors. Since the normal is almost constant during the time interval, so are the directions of the tangential directions. 
\begin{figure}[h]
\includegraphics[width=0.5\columnwidth]{Hodogram_frecce.png}% Here is how to import EPS art
\caption{Hodogram in the tangential plane of the magnetic field for a magnetopause crossing by MMS in 28.12.2015 from 22:12:04 to 22:12:08. See text for the significance of the arrows. $B_{T1}$ and $B_{T2}$ are the projections of \textbf{B} along the tangential directions.}
\label{fig:hodogram}
\end{figure}
Several crossings have been analyzed. For each one we first study the hodogram of the magnetic field in the tangential plane with respect to the structure. In this plane, a rotational discontinuity would correspond to a circular arc with constant radius while a shock would correspond to a radial line. In several crossings, however, a linear (yet not radial, i.e. not passing through the origin) variation is observed. This is shown in Figure \ref{fig:hodogram} for the crossing of the 28.12.2015 at 22:12:04 which we consider as typical. This non-radial variation of the magnetic field is quite interesting being not predicted by CTD.

We note that the time interval that we analysed was selected as a sub-interval of the whole crossing. It has been chosen so that both the  ion mass flux and the magnetic field have a 1D structure with the same normal for all the selected times. The dimensionality is obtained using the dimensionality parameters presented in \cite{Rezeau2017}. This underlines that this observed property is not likely to be due to a  non-unidimensional geometry of the layer. 

To further analyze the causes of this disagreement, we must compare the different terms of the tangential momentum equation and Faraday/Ohm's law. This is the object of Figure \ref{fig:Ohm_momentum}. The current is a by-product of the method used since its components simply derive from the components of the $\nabla \textbf{B}$ fit. The normal components are also shown for reference.

Concerning the Ohm's law (Figure \ref{fig:Ohm_momentum}, panel 1), we see that the electric field is well counter-balanced by the $\mathbf{U}\times\mathbf{B}$ and $\mathbf{J}\times\mathbf{B}/nq$ terms (ideal and Hall terms), while the $\nabla\cdot \mathbf{P}_e$ is negligible in all directions. Outside the layer, on both sides, the ideal Ohm's law is verified, as assumed in CTD. \\
Concerning the momentum equation, shown in panel 2 of Figure \ref{fig:Ohm_momentum}, we observe that, in the normal direction, the $\mathbf{J}\times\mathbf{B}$ term is counter-balanced by the divergence of the ion pressure, as expected. But, if the isotropic condition assumed in CTD was valid, we would expect the divergence of the ion pressure tensor to be null in the two tangential directions, or at least negligible with respect to the inertial term $\rho d\mathbf{V}/dt$. On the contrary, we observe in these two directions that the $\mathbf{J}\times\mathbf{B}$ term is of the same order of magnitude as the divergence of the ion pressure tensor, and one order of magnitude larger than all the other terms. Considering the sum of the different terms would show that it is actually not zero as it should be showing that the accuracy presently attainable is not yet sufficient for such a detailed check. Nevertheless, this accuracy is sufficient to establish that the inertial term is one order of magnitude lower than the $\mathbf{J} \times \mathbf{B}$ term. This proves that the $\nabla \cdot \mathbf{P}_i$ term plays a fundamental role in the magnetopause equilibrium. 
This point can be emphasized also by analyzing the hodogram of Figure \ref{fig:hodogram}. In the figure, the arrows are directed along the directions of the tangential plane that are physically relevant for the problem: - the tangent to the hodogram (green), which indicates the total variation of $\mathbf B_t$; - the radial direction (red), which corresponds to the plasma compression; - the $\nabla \cdot \mathbf{P}_{it}$ direction (blue), which is absent in CTD. The relative lengths of the arrows are chosen proportional to the corresponding term magnitudes. These directions are averaged in two sub-intervals (bold hodogram). The striking result is that the total variation is mainly determined by the non-classic term $\nabla \cdot \mathbf{P}_{it}$ and not by the radial classic one. This explains the very recurrent (even if not reported in the literature hitherto) feature that the hodograms are almost linear but not radial.


\begin{figure*}
\includegraphics[width=\textwidth]{SingleFig_GRL_nosum.png}% Here is how to import EPS art
\caption{\label{fig:Ohm_momentum} Terms of the Ohm equation (panel 1, units of m$V/m$) and the momentum equation (panel 2, units of $10^{-15}kg\,m/s^2$), projected in the normal direction \textit{n} ($a$) and in the two tangential directions ($T_1$ ($b$) and $T_2$ ($c$)). To reduce the noise, a running average with a time window of 0.35s is done on the electric field measurements. Dotted line represents the sum of all the other terms, which is smaller that other terms in most of the time interval in all three directions. \textit{N.b.} The terms of the tangential Faraday/ Ohm's law used in the text are just the derivatives of the ones in ($a$) (apart from a $\pi/2$ rotation).}
\end{figure*}

\section{Conclusions}

The study of the properties of the magnetopause is an important issue for the understanding of the penetration of the solar wind plasma into the magnetosphere. In this letter, we have shown that to study the magnetopause we need to investigate the case of a “quasi-tangential” discontinuity, revealing the crucial role of the full pressure tensor in the layer equilibrium. We emphasize that, in presence of anisotropy, the physics processes occurring inside the layer play a fundamental role being responsible for the set up of the relation between downstream and upstream quantities. In particular, for thin current layers, the FLR corrections, corresponding to the non-gyrotropic pressure tensor components, must be taken into account. We have presented here the results for a crossing observed by MMS. This can be considered as representative since the main features described here were observed in most of the other cases that we have analyzed, (including the most documented case of 16.10.2015, see \cite{burch_phan_2016,Rezeau2017,manuzzo2019}). For this crossing, the "linear" hodogram in the tangential plane shows how the boundary differs from the CTD results. This discrepancy with CTD can be then explained by studying the tangential components of the momentum equation, which evidences the role of the pressure tensor. This confirms the important role of the FLR effects in quasi-tangential discontinuities. It is well-known that the linear version of the rotational discontinuity is the MHD shear Alfv\'en wave. Here it appears that the magnetopause-like "quasi-tangential" discontinuities correspond in the same way to the quasi-perpendicular "Kinetic Alfv\'en Waves"  \cite{Hasegawa,belmont_rezeau1987,cramer}.
\section*{Open Research}
Magnetospheric  Multiscale  satellite  data  were  accessed  through  the  MMS  Science  Data  Center, \url{https://lasp.colorado.edu/mms/sdc/public/}. Furthermore, all the softwares employed, from interpolation of the data to the analysis itself,  can be found at \url{https://github.com/GiulioBallerini/
Notebooks_FLR.git}.
%\pagebreak
%\bibliographystyle{apsrev4-1} % Tell bibtex which bibliography style to use
\bibliography{1702.bib}

\begin{thebibliography}{}

\bibitem [\protect \citeauthoryear {%
Abraham-Shrauner%
}{%
Abraham-Shrauner%
}{%
{\protect \APACyear {1967}}%
}]{%
abraham-shrauner_propagation_1967}
\APACinsertmetastar {%
abraham-shrauner_propagation_1967}%
\begin{APACrefauthors}%
Abraham-Shrauner, B.%
\end{APACrefauthors}%
\unskip\
\newblock
\APACrefYearMonthDay{1967}{}{}.
\newblock
{\BBOQ}\APACrefatitle {Propagation of hydromagnetic waves through an
  anisotropic plasma} {Propagation of hydromagnetic waves through an
  anisotropic plasma}.{\BBCQ}
\newblock
\APACjournalVolNumPages{Journal of Plasma Physics}{1}{3}{361--378}.
\newblock
\begin{APACrefURL}
  [{2023-01-20}]\url{https://www.cambridge.org/core/product/identifier/S0022377800003354/type/journal_article}
  \end{APACrefURL}
\newblock
\begin{APACrefDOI} \doi{10.1017/S0022377800003354} \end{APACrefDOI}
\PrintBackRefs{\CurrentBib}

\bibitem [\protect \citeauthoryear {%
Aunai%
, Hesse%
\BCBL {}\ \BBA {} Kuznetsova%
}{%
Aunai%
\ \protect \BOthers {.}}{%
{\protect \APACyear {2013}}%
}]{%
aunai_electron_2013}
\APACinsertmetastar {%
aunai_electron_2013}%
\begin{APACrefauthors}%
Aunai, N.%
, Hesse, M.%
\BCBL {}\ \BBA {} Kuznetsova, M.%
\end{APACrefauthors}%
\unskip\
\newblock
\APACrefYearMonthDay{2013}{}{}.
\newblock
{\BBOQ}\APACrefatitle {Electron nongyrotropy in the context of collisionless
  magnetic reconnection} {Electron nongyrotropy in the context of collisionless
  magnetic reconnection}.{\BBCQ}
\newblock
\APACjournalVolNumPages{Physics of Plasmas}{20}{9}{092903}.
\newblock
\begin{APACrefURL}
  [{2023-01-27}]\url{http://aip.scitation.org/doi/10.1063/1.4820953}
  \end{APACrefURL}
\newblock
\begin{APACrefDOI} \doi{10.1063/1.4820953} \end{APACrefDOI}
\PrintBackRefs{\CurrentBib}

\bibitem [\protect \citeauthoryear {%
Aunai%
\ \protect \BOthers {.}}{%
Aunai%
\ \protect \BOthers {.}}{%
{\protect \APACyear {2011}}%
}]{%
aunai_proton_2011}
\APACinsertmetastar {%
aunai_proton_2011}%
\begin{APACrefauthors}%
Aunai, N.%
, Retinò, A.%
, Belmont, G.%
, Smets, R.%
, Lavraud, B.%
\BCBL {}\ \BBA {} Vaivads, A.%
\end{APACrefauthors}%
\unskip\
\newblock
\APACrefYearMonthDay{2011}{}{}.
\newblock
{\BBOQ}\APACrefatitle {The proton pressure tensor as a new proxy of the proton
  decoupling region in collisionless magnetic reconnection} {The proton
  pressure tensor as a new proxy of the proton decoupling region in
  collisionless magnetic reconnection}.{\BBCQ}
\newblock
\APACjournalVolNumPages{Annales Geophysicae}{29}{9}{1571--1579}.
\newblock
\begin{APACrefURL}
  [{2023-01-27}]\url{https://angeo.copernicus.org/articles/29/1571/2011/}
  \end{APACrefURL}
\newblock
\begin{APACrefDOI} \doi{10.5194/angeo-29-1571-2011} \end{APACrefDOI}
\PrintBackRefs{\CurrentBib}

\bibitem [\protect \citeauthoryear {%
Belmont%
, Aunai%
\BCBL {}\ \BBA {} Smets%
}{%
Belmont%
\ \protect \BOthers {.}}{%
{\protect \APACyear {2012}}%
}]{%
belmont_kinetic_2012}
\APACinsertmetastar {%
belmont_kinetic_2012}%
\begin{APACrefauthors}%
Belmont, G.%
, Aunai, N.%
\BCBL {}\ \BBA {} Smets, R.%
\end{APACrefauthors}%
\unskip\
\newblock
\APACrefYearMonthDay{2012}{}{}.
\newblock
{\BBOQ}\APACrefatitle {Kinetic equilibrium for an asymmetric tangential layer}
  {Kinetic equilibrium for an asymmetric tangential layer}.{\BBCQ}
\newblock
\APACjournalVolNumPages{Physics of Plasmas}{19}{2}{022108}.
\newblock
\begin{APACrefURL}
  [{2023-01-27}]\url{http://aip.scitation.org/doi/10.1063/1.3685707}
  \end{APACrefURL}
\newblock
\begin{APACrefDOI} \doi{10.1063/1.3685707} \end{APACrefDOI}
\PrintBackRefs{\CurrentBib}

\bibitem [\protect \citeauthoryear {%
Belmont%
, Grappin%
, Mottez%
, Pantellini%
\BCBL {}\ \BBA {} Pelletier%
}{%
Belmont%
\ \protect \BOthers {.}}{%
{\protect \APACyear {2014}}%
}]{%
belmont_collisionless_2014}
\APACinsertmetastar {%
belmont_collisionless_2014}%
\begin{APACrefauthors}%
Belmont, G.%
, Grappin, R.%
, Mottez, F.%
, Pantellini, F.%
\BCBL {}\ \BBA {} Pelletier, G.%
\end{APACrefauthors}%
\unskip\
\newblock
\APACrefYear{2014}.
\newblock
\APACrefbtitle {Collisionless plasmas in astrophysics} {Collisionless plasmas
  in astrophysics}.
\newblock
\APACaddressPublisher{}{Wiley}.
\PrintBackRefs{\CurrentBib}

\bibitem [\protect \citeauthoryear {%
Belmont%
\ \BBA {} Rezeau%
}{%
Belmont%
\ \BBA {} Rezeau%
}{%
{\protect \APACyear {1987}}%
}]{%
belmont_rezeau1987}
\APACinsertmetastar {%
belmont_rezeau1987}%
\begin{APACrefauthors}%
Belmont, G.%
\BCBT {}\ \BBA {} Rezeau, L.%
\end{APACrefauthors}%
\unskip\
\newblock
\APACrefYearMonthDay{1987}{}{}.
\newblock
{\BBOQ}\APACrefatitle {{Finite Larmor radius effects: the two-fluid approach}}
  {{Finite Larmor radius effects: the two-fluid approach}}.{\BBCQ}
\newblock
\APACjournalVolNumPages{{Annales Geophysicae}}{}{}{vol. 5, no2, pp. 59-69}.
\newblock
\begin{APACrefURL} \url{https://hal.science/hal-00408549} \end{APACrefURL}
\PrintBackRefs{\CurrentBib}

\bibitem [\protect \citeauthoryear {%
Belmont%
, Rezeau%
, Riconda%
\BCBL {}\ \BBA {} Zaslavsky%
}{%
Belmont%
\ \protect \BOthers {.}}{%
{\protect \APACyear {2019}}%
}]{%
belmont_introduction_2019}
\APACinsertmetastar {%
belmont_introduction_2019}%
\begin{APACrefauthors}%
Belmont, G.%
, Rezeau, L.%
, Riconda, C.%
\BCBL {}\ \BBA {} Zaslavsky, A.%
\end{APACrefauthors}%
\unskip\
\newblock
\APACrefYear{2019}.
\newblock
\APACrefbtitle {Introduction to plasma physics} {Introduction to plasma
  physics}.
\newblock
\APACaddressPublisher{}{{ISTE} Press}.
\PrintBackRefs{\CurrentBib}

\bibitem [\protect \citeauthoryear {%
Bertucci%
}{%
Bertucci%
}{%
{\protect \APACyear {2005}}%
}]{%
bertucci_structure_2005}
\APACinsertmetastar {%
bertucci_structure_2005}%
\begin{APACrefauthors}%
Bertucci, C.%
\end{APACrefauthors}%
\unskip\
\newblock
\APACrefYearMonthDay{2005}{}{}.
\newblock
{\BBOQ}\APACrefatitle {Structure of the magnetic pileup boundary at Mars and
  Venus} {Structure of the magnetic pileup boundary at mars and venus}.{\BBCQ}
\newblock
\APACjournalVolNumPages{Journal of Geophysical Research}{110}{}{A01209}.
\newblock
\begin{APACrefURL}
  [{2023-01-27}]\url{http://doi.wiley.com/10.1029/2004JA010592}
  \end{APACrefURL}
\newblock
\begin{APACrefDOI} \doi{10.1029/2004JA010592} \end{APACrefDOI}
\PrintBackRefs{\CurrentBib}

\bibitem [\protect \citeauthoryear {%
Burch%
\ \BBA {} Phan%
}{%
Burch%
\ \BBA {} Phan%
}{%
{\protect \APACyear {2016}}%
}]{%
burch_phan_2016}
\APACinsertmetastar {%
burch_phan_2016}%
\begin{APACrefauthors}%
Burch, J\BPBI L.%
\BCBT {}\ \BBA {} Phan, T\BPBI D.%
\end{APACrefauthors}%
\unskip\
\newblock
\APACrefYearMonthDay{2016}{}{}.
\newblock
{\BBOQ}\APACrefatitle {Magnetic reconnection at the dayside magnetopause:
  Advances with MMS} {Magnetic reconnection at the dayside magnetopause:
  Advances with mms}.{\BBCQ}
\newblock
\APACjournalVolNumPages{Geophysical Research Letters}{43}{16}{8327-8338}.
\newblock
\begin{APACrefURL}
  \url{https://agupubs.onlinelibrary.wiley.com/doi/abs/10.1002/2016GL069787}
  \end{APACrefURL}
\newblock
\begin{APACrefDOI} \doi{https://doi.org/10.1002/2016GL069787} \end{APACrefDOI}
\PrintBackRefs{\CurrentBib}

\bibitem [\protect \citeauthoryear {%
Chanteur%
}{%
Chanteur%
}{%
{\protect \APACyear {1998}}%
}]{%
chanteur_spatial_1998}
\APACinsertmetastar {%
chanteur_spatial_1998}%
\begin{APACrefauthors}%
Chanteur, G.%
\end{APACrefauthors}%
\unskip\
\newblock
\APACrefYearMonthDay{1998}{}{}.
\newblock
{\BBOQ}\APACrefatitle {Spatial Interpolation for Four Spacecraft: Theory}
  {Spatial interpolation for four spacecraft: Theory}.{\BBCQ}
\newblock
\APACjournalVolNumPages{{ISSI} Scientific Reports Series}{1}{}{349--370}.
\newblock
\begin{APACrefURL}
  [{2023-05-15}]\url{https://ui.adsabs.harvard.edu/abs/1998ISSIR...1..349C}
  \end{APACrefURL}
\newblock
\APACrefnote{{ADS} Bibcode: 1998ISSIR...1..349C}
\PrintBackRefs{\CurrentBib}

\bibitem [\protect \citeauthoryear {%
Chao%
}{%
Chao%
}{%
{\protect \APACyear {1970}}%
}]{%
chao_interplanetary_1970}
\APACinsertmetastar {%
chao_interplanetary_1970}%
\begin{APACrefauthors}%
Chao, J.%
\end{APACrefauthors}%
\unskip\
\newblock
\APACrefYear{1970}.
\newblock
\APACrefbtitle {Interplanetary collisionless shock waves} {Interplanetary
  collisionless shock waves}.
\newblock
\APACaddressPublisher{}{Vita. Bibliography: leaves 148-150. Sc D.}
\PrintBackRefs{\CurrentBib}

\bibitem [\protect \citeauthoryear {%
Chou%
\ \BBA {} Hau%
}{%
Chou%
\ \BBA {} Hau%
}{%
{\protect \APACyear {2012-08}}%
}]{%
chou_statistical_2012}
\APACinsertmetastar {%
chou_statistical_2012}%
\begin{APACrefauthors}%
Chou, Y\BHBI C.%
\BCBT {}\ \BBA {} Hau, L\BHBI N.%
\end{APACrefauthors}%
\unskip\
\newblock
\APACrefYearMonthDay{2012-08}{}{}.
\newblock
{\BBOQ}\APACrefatitle {A statistical study of magnetopause structures:
  Tangential versus rotational discontinuities: {THE} {STRUCTURE} {OF}
  {MAGNETOPAUSE} {CURRENT}} {A statistical study of magnetopause structures:
  Tangential versus rotational discontinuities: {THE} {STRUCTURE} {OF}
  {MAGNETOPAUSE} {CURRENT}}.{\BBCQ}
\newblock
\APACjournalVolNumPages{Journal of Geophysical Research: Space
  Physics}{117}{}{n/a--n/a}.
\newblock
\begin{APACrefURL}
  [{2023-01-20}]\url{http://doi.wiley.com/10.1029/2011JA017155}
  \end{APACrefURL}
\newblock
\begin{APACrefDOI} \doi{10.1029/2011JA017155} \end{APACrefDOI}
\PrintBackRefs{\CurrentBib}

\bibitem [\protect \citeauthoryear {%
Coates%
}{%
Coates%
}{%
{\protect \APACyear {1997}}%
}]{%
coates_ionospheres_1997}
\APACinsertmetastar {%
coates_ionospheres_1997}%
\begin{APACrefauthors}%
Coates, A.%
\end{APACrefauthors}%
\unskip\
\newblock
\APACrefYearMonthDay{1997}{}{}.
\newblock
{\BBOQ}\APACrefatitle {Ionospheres and magnetospheres of comets} {Ionospheres
  and magnetospheres of comets}.{\BBCQ}
\newblock
\APACjournalVolNumPages{Advances in Space Research}{20}{2}{255--266}.
\newblock
\begin{APACrefURL}
  \url{https://linkinghub.elsevier.com/retrieve/pii/S0273117797005437}
  \end{APACrefURL}
\newblock
\begin{APACrefDOI} \doi{10.1016/S0273-1177(97)00543-7} \end{APACrefDOI}
\PrintBackRefs{\CurrentBib}

\bibitem [\protect \citeauthoryear {%
Colburn%
\ \BBA {} Sonett%
}{%
Colburn%
\ \BBA {} Sonett%
}{%
{\protect \APACyear {1966}}%
}]{%
Colburn1966}
\APACinsertmetastar {%
Colburn1966}%
\begin{APACrefauthors}%
Colburn, D\BPBI S.%
\BCBT {}\ \BBA {} Sonett, C\BPBI P.%
\end{APACrefauthors}%
\unskip\
\newblock
\APACrefYearMonthDay{1966}{}{}.
\newblock
{\BBOQ}\APACrefatitle {Discontinuities in the solar wind} {Discontinuities in
  the solar wind}.{\BBCQ}
\newblock
\APACjournalVolNumPages{Space Science Reviews}{5}{}{439-506}.
\PrintBackRefs{\CurrentBib}

\bibitem [\protect \citeauthoryear {%
Coppi%
, Laval%
\BCBL {}\ \BBA {} Pellat%
}{%
Coppi%
\ \protect \BOthers {.}}{%
{\protect \APACyear {1966}}%
}]{%
coppi_dynamics_1966}
\APACinsertmetastar {%
coppi_dynamics_1966}%
\begin{APACrefauthors}%
Coppi, B.%
, Laval, G.%
\BCBL {}\ \BBA {} Pellat, R.%
\end{APACrefauthors}%
\unskip\
\newblock
\APACrefYearMonthDay{1966}{}{}.
\newblock
{\BBOQ}\APACrefatitle {Dynamics of the Geomagnetic Tail} {Dynamics of the
  geomagnetic tail}.{\BBCQ}
\newblock
\APACjournalVolNumPages{Physical Review Letters}{}{26}{1207--1210}.
\newblock
\begin{APACrefURL}
  [{2023-05-15}]\url{https://link.aps.org/doi/10.1103/PhysRevLett.16.1207}
  \end{APACrefURL}
\newblock
\begin{APACrefDOI} \doi{10.1103/PhysRevLett.16.1207} \end{APACrefDOI}
\PrintBackRefs{\CurrentBib}

\bibitem [\protect \citeauthoryear {%
Cramer%
}{%
Cramer%
}{%
{\protect \APACyear {2001}}%
}]{%
cramer}
\APACinsertmetastar {%
cramer}%
\begin{APACrefauthors}%
Cramer, N\BPBI F.%
\end{APACrefauthors}%
\unskip\
\newblock
\APACrefYear{2001}.
\newblock
\APACrefbtitle {The Physics of Alfvén Waves} {The physics of alfvén waves}.
\newblock
\APACaddressPublisher{}{John Wiley and Sons, Ltd}.
\newblock
\begin{APACrefDOI} \doi{doi.org/10.1002/3527603123.fmatter} \end{APACrefDOI}
\PrintBackRefs{\CurrentBib}

\bibitem [\protect \citeauthoryear {%
Denton%
\ \protect \BOthers {.}}{%
Denton%
\ \protect \BOthers {.}}{%
{\protect \APACyear {2021}}%
}]{%
denton2021}
\APACinsertmetastar {%
denton2021}%
\begin{APACrefauthors}%
Denton, R\BPBI E.%
, Torbert, R\BPBI B.%
, Hasegawa, H.%
, Genestreti, K\BPBI J.%
, Manuzzo, R.%
, Belmont, G.%
\BDBL {}Giles, B\BPBI L.%
\end{APACrefauthors}%
\unskip\
\newblock
\APACrefYearMonthDay{2021}{}{}.
\newblock
{\BBOQ}\APACrefatitle {Two-Dimensional Velocity of the Magnetic Structure
  Observed on July 11, 2017 by the Magnetospheric Multiscale Spacecraft}
  {Two-dimensional velocity of the magnetic structure observed on july 11, 2017
  by the magnetospheric multiscale spacecraft}.{\BBCQ}
\newblock
\APACjournalVolNumPages{Journal of Geophysical Research: Space
  Physics}{126}{3}{e2020JA028705}.
\newblock
\begin{APACrefURL}
  \url{https://agupubs.onlinelibrary.wiley.com/doi/abs/10.1029/2020JA028705}
  \end{APACrefURL}
\newblock
\begin{APACrefDOI} \doi{https://doi.org/10.1029/2020JA028705} \end{APACrefDOI}
\PrintBackRefs{\CurrentBib}

\bibitem [\protect \citeauthoryear {%
Dorville%
, Belmont%
, Aunai%
, Dargent%
\BCBL {}\ \BBA {} Rezeau%
}{%
Dorville%
, Belmont%
\BCBL {}\ \protect \BOthers {.}}{%
{\protect \APACyear {2015}}%
}]{%
dorville_asymmetric_2015}
\APACinsertmetastar {%
dorville_asymmetric_2015}%
\begin{APACrefauthors}%
Dorville, N.%
, Belmont, G.%
, Aunai, N.%
, Dargent, J.%
\BCBL {}\ \BBA {} Rezeau, L.%
\end{APACrefauthors}%
\unskip\
\newblock
\APACrefYearMonthDay{2015}{}{}.
\newblock
{\BBOQ}\APACrefatitle {Asymmetric kinetic equilibria: Generalization of the
  {BAS} model for rotating magnetic profile and non-zero electric field}
  {Asymmetric kinetic equilibria: Generalization of the {BAS} model for
  rotating magnetic profile and non-zero electric field}.{\BBCQ}
\newblock
\APACjournalVolNumPages{Physics of Plasmas}{22}{9}{092904}.
\newblock
\begin{APACrefURL} \url{http://aip.scitation.org/doi/10.1063/1.4930210}
  \end{APACrefURL}
\newblock
\begin{APACrefDOI} \doi{10.1063/1.4930210} \end{APACrefDOI}
\PrintBackRefs{\CurrentBib}

\bibitem [\protect \citeauthoryear {%
Dorville%
, Belmont%
, Rezeau%
, Grappin%
\BCBL {}\ \BBA {} Retinò%
}{%
Dorville%
\ \protect \BOthers {.}}{%
{\protect \APACyear {2014}}%
}]{%
dorville_rotationalcompressional_2014}
\APACinsertmetastar {%
dorville_rotationalcompressional_2014}%
\begin{APACrefauthors}%
Dorville, N.%
, Belmont, G.%
, Rezeau, L.%
, Grappin, R.%
\BCBL {}\ \BBA {} Retinò, A.%
\end{APACrefauthors}%
\unskip\
\newblock
\APACrefYearMonthDay{2014}{}{}.
\newblock
{\BBOQ}\APACrefatitle {Rotational/compressional nature of the magnetopause:
  Application of the {BV} technique on a magnetopause case study}
  {Rotational/compressional nature of the magnetopause: Application of the {BV}
  technique on a magnetopause case study}.{\BBCQ}
\newblock
\APACjournalVolNumPages{Journal of Geophysical Research: Space
  Physics}{119}{3}{1898--1908}.
\newblock
\begin{APACrefURL}
  [{2023-01-20}]\url{http://doi.wiley.com/10.1002/2013JA018927}
  \end{APACrefURL}
\newblock
\begin{APACrefDOI} \doi{10.1002/2013JA018927} \end{APACrefDOI}
\PrintBackRefs{\CurrentBib}

\bibitem [\protect \citeauthoryear {%
Dorville%
, Haaland%
, Anekallu%
, Belmont%
\BCBL {}\ \BBA {} Rezeau%
}{%
Dorville%
, Haaland%
\BCBL {}\ \protect \BOthers {.}}{%
{\protect \APACyear {2015}}%
}]{%
dorville_magnetopause_2015}
\APACinsertmetastar {%
dorville_magnetopause_2015}%
\begin{APACrefauthors}%
Dorville, N.%
, Haaland, S.%
, Anekallu, C.%
, Belmont, G.%
\BCBL {}\ \BBA {} Rezeau, L.%
\end{APACrefauthors}%
\unskip\
\newblock
\APACrefYearMonthDay{2015}{}{}.
\newblock
{\BBOQ}\APACrefatitle {Magnetopause orientation: Comparison between generic
  residue analysis and {BV} method: {GRA}/{BV} {COMPARISON}} {Magnetopause
  orientation: Comparison between generic residue analysis and {BV} method:
  {GRA}/{BV} {COMPARISON}}.{\BBCQ}
\newblock
\APACjournalVolNumPages{Journal of Geophysical Research: Space
  Physics}{120}{5}{3366--3379}.
\newblock
\begin{APACrefURL}
  [{2023-01-27}]\url{http://doi.wiley.com/10.1002/2014JA020806}
  \end{APACrefURL}
\newblock
\begin{APACrefDOI} \doi{10.1002/2014JA020806} \end{APACrefDOI}
\PrintBackRefs{\CurrentBib}

\bibitem [\protect \citeauthoryear {%
Ergun%
\ \protect \BOthers {.}}{%
Ergun%
\ \protect \BOthers {.}}{%
{\protect \APACyear {2016}}%
}]{%
ergun_axial_2016}
\APACinsertmetastar {%
ergun_axial_2016}%
\begin{APACrefauthors}%
Ergun, R\BPBI E.%
, Tucker, S.%
, Westfall, J.%
, Goodrich, K\BPBI A.%
, Malaspina, D\BPBI M.%
, Summers, D.%
\BDBL {}Cully, C\BPBI M.%
\end{APACrefauthors}%
\unskip\
\newblock
\APACrefYearMonthDay{2016}{}{}.
\newblock
{\BBOQ}\APACrefatitle {The Axial Double Probe and Fields Signal Processing for
  the {MMS} Mission} {The axial double probe and fields signal processing for
  the {MMS} mission}.{\BBCQ}
\newblock
\APACjournalVolNumPages{Space Science Reviews}{199}{1}{167--188}.
\newblock
\begin{APACrefURL}
  [{2023-05-15}]\url{http://link.springer.com/10.1007/s11214-014-0115-x}
  \end{APACrefURL}
\newblock
\begin{APACrefDOI} \doi{10.1007/s11214-014-0115-x} \end{APACrefDOI}
\PrintBackRefs{\CurrentBib}

\bibitem [\protect \citeauthoryear {%
Galeev%
}{%
Galeev%
}{%
{\protect \APACyear {1979}}%
}]{%
galeev_reconnection_1979}
\APACinsertmetastar {%
galeev_reconnection_1979}%
\begin{APACrefauthors}%
Galeev, A.%
\end{APACrefauthors}%
\unskip\
\newblock
\APACrefYearMonthDay{1979}{}{}.
\newblock
{\BBOQ}\APACrefatitle {Reconnection in the magnetotail} {Reconnection in the
  magnetotail}.{\BBCQ}
\newblock
\APACjournalVolNumPages{Space Science Reviews}{23}{3}{}.
\newblock
\begin{APACrefURL} [{2023}]\url{http://link.springer.com/10.1007/BF00172248}
  \end{APACrefURL}
\newblock
\begin{APACrefDOI} \doi{10.1007/BF00172248} \end{APACrefDOI}
\PrintBackRefs{\CurrentBib}

\bibitem [\protect \citeauthoryear {%
Haaland%
\ \protect \BOthers {.}}{%
Haaland%
\ \protect \BOthers {.}}{%
{\protect \APACyear {2004}}%
}]{%
haaland2004}
\APACinsertmetastar {%
haaland2004}%
\begin{APACrefauthors}%
Haaland, S\BPBI E.%
, Sonnerup, B\BPBI U\BPBI O.%
, Dunlop, M\BPBI W.%
, Balogh, A.%
, Georgescu, E.%
, Hasegawa, H.%
\BDBL {}Vaivads, A.%
\end{APACrefauthors}%
\unskip\
\newblock
\APACrefYearMonthDay{2004}{}{}.
\newblock
{\BBOQ}\APACrefatitle {Four-spacecraft determination of magnetopause
  orientation, motion and thickness: comparison with results from
  single-spacecraft methods} {Four-spacecraft determination of magnetopause
  orientation, motion and thickness: comparison with results from
  single-spacecraft methods}.{\BBCQ}
\newblock
\APACjournalVolNumPages{Annales Geophysicae}{22}{4}{1347--1365}.
\newblock
\begin{APACrefURL} \url{https://angeo.copernicus.org/articles/22/1347/2004/}
  \end{APACrefURL}
\newblock
\begin{APACrefDOI} \doi{10.5194/angeo-22-1347-2004} \end{APACrefDOI}
\PrintBackRefs{\CurrentBib}

\bibitem [\protect \citeauthoryear {%
Hasegawa%
\ \BBA {} Uberoi%
}{%
Hasegawa%
\ \BBA {} Uberoi%
}{%
{\protect \APACyear {1982}}%
}]{%
Hasegawa}
\APACinsertmetastar {%
Hasegawa}%
\begin{APACrefauthors}%
Hasegawa, A.%
\BCBT {}\ \BBA {} Uberoi, C.%
\end{APACrefauthors}%
\unskip\
\newblock
\APACrefYear{1982}.
\newblock
\APACrefbtitle {The Alfvén wave} {The alfvén wave}.
\newblock
\APACaddressPublisher{}{Oak Ridge, TN (USA): U.S. Department of Energy
  Technical Information Center}.
\PrintBackRefs{\CurrentBib}

\bibitem [\protect \citeauthoryear {%
Hudson%
}{%
Hudson%
}{%
{\protect \APACyear {1971}}%
}]{%
hudson_rotational_1971}
\APACinsertmetastar {%
hudson_rotational_1971}%
\begin{APACrefauthors}%
Hudson, P.%
\end{APACrefauthors}%
\unskip\
\newblock
\APACrefYearMonthDay{1971}{}{}.
\newblock
{\BBOQ}\APACrefatitle {Rotational discontinuities in an anisotropic plasma}
  {Rotational discontinuities in an anisotropic plasma}.{\BBCQ}
\newblock
\APACjournalVolNumPages{Planetary and Space Science}{19}{12}{1693--1699}.
\newblock
\begin{APACrefURL}
  [{2023-01-03}]\url{https://linkinghub.elsevier.com/retrieve/pii/0032063371901292}
  \end{APACrefURL}
\newblock
\begin{APACrefDOI} \doi{10.1016/0032-0633(71)90129-2} \end{APACrefDOI}
\PrintBackRefs{\CurrentBib}

\bibitem [\protect \citeauthoryear {%
Jeffrey%
\ \BBA {} Taniuti%
}{%
Jeffrey%
\ \BBA {} Taniuti%
}{%
{\protect \APACyear {1964}}%
}]{%
jeffrey_non-linear_1964}
\APACinsertmetastar {%
jeffrey_non-linear_1964}%
\begin{APACrefauthors}%
Jeffrey, A.%
\BCBT {}\ \BBA {} Taniuti, T.%
\end{APACrefauthors}%
\unskip\
\newblock
\APACrefYear{1964}.
\newblock
\APACrefbtitle {Non-linear Wave Propagation: With Applications to Physics and
  Magnetohydrodynamics (Mathematics in science and engineering ; v. 9)}
  {Non-linear wave propagation: With applications to physics and
  magnetohydrodynamics (mathematics in science and engineering ; v. 9)}.
\newblock
\APACaddressPublisher{}{Academic Press}.
\PrintBackRefs{\CurrentBib}

\bibitem [\protect \citeauthoryear {%
Kivelson%
\ \BBA {} Russell%
}{%
Kivelson%
\ \BBA {} Russell%
}{%
{\protect \APACyear {1995}}%
}]{%
kivelson_introduction_1995}
\APACinsertmetastar {%
kivelson_introduction_1995}%
\begin{APACrefauthors}%
Kivelson, M\BPBI G.%
\BCBT {}\ \BBA {} Russell, C\BPBI T.%
\end{APACrefauthors}%
\ (\BEDS).
\unskip\
\newblock
\APACrefYear{1995}.
\newblock
\APACrefbtitle {Introduction to space physics} {Introduction to space physics}.
\newblock
\APACaddressPublisher{}{Cambridge University Press}.
\newblock
\APACrefnote{{OCLC}: 1124679918}
\PrintBackRefs{\CurrentBib}

\bibitem [\protect \citeauthoryear {%
Lallement%
}{%
Lallement%
}{%
{\protect \APACyear {2001}}%
}]{%
lallement_2001}
\APACinsertmetastar {%
lallement_2001}%
\begin{APACrefauthors}%
Lallement, R.%
\end{APACrefauthors}%
\unskip\
\newblock
\APACrefYearMonthDay{2001}{}{}.
\newblock
{\BBOQ}\APACrefatitle {Heliopause and asteropauses} {Heliopause and
  asteropauses}.{\BBCQ}
\newblock
\APACjournalVolNumPages{Astrophysics and Space Science}{277}{}{205--2017}.
\newblock
\begin{APACrefDOI} \doi{10.1023/A:1012284420156} \end{APACrefDOI}
\PrintBackRefs{\CurrentBib}

\bibitem [\protect \citeauthoryear {%
Landau%
\ \BBA {} Lifshitz%
}{%
Landau%
\ \BBA {} Lifshitz%
}{%
{\protect \APACyear {1987}}%
}]{%
landau_fluid}
\APACinsertmetastar {%
landau_fluid}%
\begin{APACrefauthors}%
Landau, L\BPBI D.%
\BCBT {}\ \BBA {} Lifshitz, E\BPBI M.%
\end{APACrefauthors}%
\unskip\
\newblock
\APACrefYear{1987}.
\newblock
\APACrefbtitle {Fluid mechanics} {Fluid mechanics}.
\newblock
\APACaddressPublisher{}{Pergamon Press}.
\PrintBackRefs{\CurrentBib}

\bibitem [\protect \citeauthoryear {%
Landau%
, Lifshitz%
\BCBL {}\ \BBA {} King%
}{%
Landau%
\ \protect \BOthers {.}}{%
{\protect \APACyear {1961}}%
}]{%
landau_electrodynamics_1961}
\APACinsertmetastar {%
landau_electrodynamics_1961}%
\begin{APACrefauthors}%
Landau, L\BPBI D.%
, Lifshitz, E\BPBI M.%
\BCBL {}\ \BBA {} King, A\BPBI L.%
\end{APACrefauthors}%
\unskip\
\newblock
\APACrefYearMonthDay{1961}{}{}.
\newblock
{\BBOQ}\APACrefatitle {Electrodynamics of Continuous Media} {Electrodynamics of
  continuous media}.{\BBCQ}
\newblock
\APACjournalVolNumPages{American Journal of Physics}{29}{9}{647--648}.
\newblock
\begin{APACrefURL}
  [{2022-12-19}]\url{http://aapt.scitation.org/doi/10.1119/1.1937882}
  \end{APACrefURL}
\newblock
\begin{APACrefDOI} \doi{10.1119/1.1937882} \end{APACrefDOI}
\PrintBackRefs{\CurrentBib}

\bibitem [\protect \citeauthoryear {%
Lindqvist%
\ \protect \BOthers {.}}{%
Lindqvist%
\ \protect \BOthers {.}}{%
{\protect \APACyear {2016}}%
}]{%
lindqvist_spin-plane_2016}
\APACinsertmetastar {%
lindqvist_spin-plane_2016}%
\begin{APACrefauthors}%
Lindqvist, P\BHBI A.%
, Olsson, G.%
, Torbert, R\BPBI B.%
, King, B.%
, Granoff, M.%
, Rau, D.%
\BDBL {}Tucker, S.%
\end{APACrefauthors}%
\unskip\
\newblock
\APACrefYearMonthDay{2016}{}{}.
\newblock
{\BBOQ}\APACrefatitle {The Spin-Plane Double Probe Electric Field Instrument
  for {MMS}} {The spin-plane double probe electric field instrument for
  {MMS}}.{\BBCQ}
\newblock
\APACjournalVolNumPages{Space Science Reviews}{199}{1}{137--165}.
\newblock
\begin{APACrefURL} \url{http://link.springer.com/10.1007/s11214-014-0116-9}
  \end{APACrefURL}
\newblock
\begin{APACrefDOI} \doi{10.1007/s11214-014-0116-9} \end{APACrefDOI}
\PrintBackRefs{\CurrentBib}

\bibitem [\protect \citeauthoryear {%
Liu%
\ \protect \BOthers {.}}{%
Liu%
\ \protect \BOthers {.}}{%
{\protect \APACyear {2022}}%
}]{%
Liu2022}
\APACinsertmetastar {%
Liu2022}%
\begin{APACrefauthors}%
Liu, Y\BPBI Y.%
, Fu, H\BPBI S.%
, Cao, J\BPBI B.%
, Wang, Z.%
, He, R\BPBI J.%
, Guo, Z.%
\BDBL {}Yu, Y.%
\end{APACrefauthors}%
\unskip\
\newblock
\APACrefYearMonthDay{2022}{}{}.
\newblock
{\BBOQ}\APACrefatitle {Magnetic Discontinuities in the Solar Wind and
  Magnetosheath: Magnetospheric Multiscale Mission (MMS) Observations}
  {Magnetic discontinuities in the solar wind and magnetosheath: Magnetospheric
  multiscale mission (mms) observations}.{\BBCQ}
\newblock
\APACjournalVolNumPages{The Astrophysical Journal}{930}{}{}.
\PrintBackRefs{\CurrentBib}

\bibitem [\protect \citeauthoryear {%
Lynn%
}{%
Lynn%
}{%
{\protect \APACyear {1967}}%
}]{%
lynn_discontinuities_1967}
\APACinsertmetastar {%
lynn_discontinuities_1967}%
\begin{APACrefauthors}%
Lynn, Y\BPBI M.%
\end{APACrefauthors}%
\unskip\
\newblock
\APACrefYearMonthDay{1967}{}{}.
\newblock
{\BBOQ}\APACrefatitle {Discontinuities in an Anisotropic Plasma}
  {Discontinuities in an anisotropic plasma}.{\BBCQ}
\newblock
\APACjournalVolNumPages{Physics of Fluids}{10}{10}{2278}.
\newblock
\begin{APACrefURL}
  [{2023-01-20}]\url{https://aip.scitation.org/doi/10.1063/1.1762025}
  \end{APACrefURL}
\newblock
\begin{APACrefDOI} \doi{10.1063/1.1762025} \end{APACrefDOI}
\PrintBackRefs{\CurrentBib}

\bibitem [\protect \citeauthoryear {%
Manuzzo%
, Belmont%
, Rezeau%
, Califano%
\BCBL {}\ \BBA {} Denton%
}{%
Manuzzo%
\ \protect \BOthers {.}}{%
{\protect \APACyear {2019}}%
}]{%
manuzzo2019}
\APACinsertmetastar {%
manuzzo2019}%
\begin{APACrefauthors}%
Manuzzo, R.%
, Belmont, G.%
, Rezeau, L.%
, Califano, F.%
\BCBL {}\ \BBA {} Denton, R\BPBI E.%
\end{APACrefauthors}%
\unskip\
\newblock
\APACrefYearMonthDay{2019}{}{}.
\newblock
{\BBOQ}\APACrefatitle {Crossing of Plasma Structures by Spacecraft: A Path
  Calculator} {Crossing of plasma structures by spacecraft: A path
  calculator}.{\BBCQ}
\newblock
\APACjournalVolNumPages{Journal of Geophysical Research: Space
  Physics}{124}{12}{10119-10140}.
\newblock
\begin{APACrefDOI} \doi{https://doi.org/10.1029/2019JA026632} \end{APACrefDOI}
\PrintBackRefs{\CurrentBib}

\bibitem [\protect \citeauthoryear {%
Neubauer%
}{%
Neubauer%
}{%
{\protect \APACyear {1970}}%
}]{%
neubauer_jump_1970}
\APACinsertmetastar {%
neubauer_jump_1970}%
\begin{APACrefauthors}%
Neubauer, F\BPBI M.%
\end{APACrefauthors}%
\unskip\
\newblock
\APACrefYearMonthDay{1970}{}{}.
\newblock
{\BBOQ}\APACrefatitle {Jump relations for shocks in an anisotropic magnetized
  plasma} {Jump relations for shocks in an anisotropic magnetized
  plasma}.{\BBCQ}
\newblock
\APACjournalVolNumPages{Zeitschrift für Physik A Hadrons and
  nuclei}{237}{3}{205--223}.
\newblock
\begin{APACrefURL}
  [{2023-01-20}]\url{http://link.springer.com/10.1007/BF01398634}
  \end{APACrefURL}
\newblock
\begin{APACrefDOI} \doi{10.1007/BF01398634} \end{APACrefDOI}
\PrintBackRefs{\CurrentBib}

\bibitem [\protect \citeauthoryear {%
Otto%
}{%
Otto%
}{%
{\protect \APACyear {2005}}%
}]{%
Otto2005}
\APACinsertmetastar {%
Otto2005}%
\begin{APACrefauthors}%
Otto, A.%
\end{APACrefauthors}%
\unskip\
\newblock
\APACrefYearMonthDay{2005}{}{}.
\newblock
{\BBOQ}\APACrefatitle {The Magnetosphere} {The magnetosphere}.{\BBCQ}
\newblock
\BIn{} K.~Scherer, H.~Fichtner, B.~Heber\BCBL {}\ \BBA {} U.~Mall\ (\BEDS),
  \APACrefbtitle {Space Weather: The Physics Behind a Slogan} {Space weather:
  The physics behind a slogan}\ (\BPGS\ 133--192).
\newblock
\APACaddressPublisher{Berlin, Heidelberg}{Springer Berlin Heidelberg}.
\newblock
\begin{APACrefURL} \url{https://doi.org/10.1007/978-3-540-31534-6_5}
  \end{APACrefURL}
\newblock
\begin{APACrefDOI} \doi{10.1007/978-3-540-31534-6_5} \end{APACrefDOI}
\PrintBackRefs{\CurrentBib}

\bibitem [\protect \citeauthoryear {%
Parks%
}{%
Parks%
}{%
{\protect \APACyear {2019}}%
}]{%
parks_physics_2019}
\APACinsertmetastar {%
parks_physics_2019}%
\begin{APACrefauthors}%
Parks, G\BPBI K.%
\end{APACrefauthors}%
\unskip\
\newblock
\APACrefYear{2019}.
\newblock
\APACrefbtitle {Physics of space plasmas: an introduction} {Physics of space
  plasmas: an introduction}.
\newblock
\APACaddressPublisher{}{Routledge, Taylor \& Francis Group}.
\PrintBackRefs{\CurrentBib}

\bibitem [\protect \citeauthoryear {%
Pollock%
\ \protect \BOthers {.}}{%
Pollock%
\ \protect \BOthers {.}}{%
{\protect \APACyear {2016}}%
}]{%
pollock_fast_2016}
\APACinsertmetastar {%
pollock_fast_2016}%
\begin{APACrefauthors}%
Pollock, C.%
, Moore, T.%
, Jacques, A.%
, Burch, J.%
, Gliese, U.%
, Saito, Y.%
\BDBL {}Zeuch, M.%
\end{APACrefauthors}%
\unskip\
\newblock
\APACrefYearMonthDay{2016}{}{}.
\newblock
{\BBOQ}\APACrefatitle {Fast Plasma Investigation for Magnetospheric Multiscale}
  {Fast plasma investigation for magnetospheric multiscale}.{\BBCQ}
\newblock
\APACjournalVolNumPages{Space Science Reviews}{199}{1}{331--406}.
\newblock
\begin{APACrefURL}
  [{2023-05-15}]\url{http://link.springer.com/10.1007/s11214-016-0245-4}
  \end{APACrefURL}
\newblock
\begin{APACrefDOI} \doi{10.1007/s11214-016-0245-4} \end{APACrefDOI}
\PrintBackRefs{\CurrentBib}

\bibitem [\protect \citeauthoryear {%
Rezeau%
, Belmont%
, Manuzzo%
, Aunai%
\BCBL {}\ \BBA {} Dargent%
}{%
Rezeau%
\ \protect \BOthers {.}}{%
{\protect \APACyear {2017}}%
}]{%
Rezeau2017}
\APACinsertmetastar {%
Rezeau2017}%
\begin{APACrefauthors}%
Rezeau, L.%
, Belmont, G.%
, Manuzzo, R.%
, Aunai, N.%
\BCBL {}\ \BBA {} Dargent, J.%
\end{APACrefauthors}%
\unskip\
\newblock
\APACrefYearMonthDay{2017}{12}{}.
\newblock
{\BBOQ}\APACrefatitle {Analyzing the Magnetopause Internal Structure: New
  Possibilities Offered by MMS Tested in a Case Study} {Analyzing the
  magnetopause internal structure: New possibilities offered by mms tested in a
  case study}.{\BBCQ}
\newblock
\APACjournalVolNumPages{Journal of Geophysical Research: Space
  Physics}{123}{}{}.
\newblock
\begin{APACrefDOI} \doi{10.1002/2017JA024526} \end{APACrefDOI}
\PrintBackRefs{\CurrentBib}

\bibitem [\protect \citeauthoryear {%
Richardson%
\ \protect \BOthers {.}}{%
Richardson%
\ \protect \BOthers {.}}{%
{\protect \APACyear {2022}}%
}]{%
richardson_observations_2022}
\APACinsertmetastar {%
richardson_observations_2022}%
\begin{APACrefauthors}%
Richardson, J\BPBI D.%
, Burlaga, L\BPBI F.%
, Elliott, H.%
, Kurth, W\BPBI S.%
, Liu, Y\BPBI D.%
\BCBL {}\ \BBA {} von Steiger, R.%
\end{APACrefauthors}%
\unskip\
\newblock
\APACrefYearMonthDay{2022}{}{}.
\newblock
{\BBOQ}\APACrefatitle {Observations of the Outer Heliosphere, Heliosheath, and
  Interstellar Medium} {Observations of the outer heliosphere, heliosheath, and
  interstellar medium}.{\BBCQ}
\newblock
\APACjournalVolNumPages{Space Science Reviews}{218}{4}{35}.
\newblock
\begin{APACrefURL}
  [{2023-01-27}]\url{https://link.springer.com/10.1007/s11214-022-00899-y}
  \end{APACrefURL}
\newblock
\begin{APACrefDOI} \doi{10.1007/s11214-022-00899-y} \end{APACrefDOI}
\PrintBackRefs{\CurrentBib}

\bibitem [\protect \citeauthoryear {%
Russell%
\ \protect \BOthers {.}}{%
Russell%
\ \protect \BOthers {.}}{%
{\protect \APACyear {2016}}%
}]{%
russell_magnetospheric_2016}
\APACinsertmetastar {%
russell_magnetospheric_2016}%
\begin{APACrefauthors}%
Russell, C\BPBI T.%
, Anderson, B\BPBI J.%
, Baumjohann, W.%
, Bromund, K\BPBI R.%
, Dearborn, D.%
, Fischer, D.%
\BDBL {}Richter, I.%
\end{APACrefauthors}%
\unskip\
\newblock
\APACrefYearMonthDay{2016}{}{}.
\newblock
{\BBOQ}\APACrefatitle {The Magnetospheric Multiscale Magnetometers} {The
  magnetospheric multiscale magnetometers}.{\BBCQ}
\newblock
\APACjournalVolNumPages{Space Science Reviews}{199}{1}{189--256}.
\newblock
\begin{APACrefURL}
  [{2023-05-15}]\url{http://link.springer.com/10.1007/s11214-014-0057-3}
  \end{APACrefURL}
\newblock
\begin{APACrefDOI} \doi{10.1007/s11214-014-0057-3} \end{APACrefDOI}
\PrintBackRefs{\CurrentBib}

\bibitem [\protect \citeauthoryear {%
Shi%
\ \protect \BOthers {.}}{%
Shi%
\ \protect \BOthers {.}}{%
{\protect \APACyear {2005}}%
}]{%
shi_dimensional_2005}
\APACinsertmetastar {%
shi_dimensional_2005}%
\begin{APACrefauthors}%
Shi, Q\BPBI Q.%
, Shen, C.%
, Pu, Z\BPBI Y.%
, Dunlop, M\BPBI W.%
, Zong, Q\BHBI G.%
, Zhang, H.%
\BDBL {}Balogh, A.%
\end{APACrefauthors}%
\unskip\
\newblock
\APACrefYearMonthDay{2005}{}{}.
\newblock
{\BBOQ}\APACrefatitle {Dimensional analysis of observed structures using
  multipoint magnetic field measurements: Application to Cluster: {STRUCTURE}
  {DIMENSIONALITY} {DETERMINATION}} {Dimensional analysis of observed
  structures using multipoint magnetic field measurements: Application to
  cluster: {STRUCTURE} {DIMENSIONALITY} {DETERMINATION}}.{\BBCQ}
\newblock
\APACjournalVolNumPages{Geophysical Research Letters}{32}{12}{n/a--n/a}.
\newblock
\begin{APACrefURL}
  [{2023-01-05}]\url{http://doi.wiley.com/10.1029/2005GL022454}
  \end{APACrefURL}
\newblock
\begin{APACrefDOI} \doi{10.1029/2005GL022454} \end{APACrefDOI}
\PrintBackRefs{\CurrentBib}

\bibitem [\protect \citeauthoryear {%
Sonnerup%
\ \BBA {} Ledley%
}{%
Sonnerup%
\ \BBA {} Ledley%
}{%
{\protect \APACyear {1974}}%
}]{%
sonnerup_magnetopause_1974}
\APACinsertmetastar {%
sonnerup_magnetopause_1974}%
\begin{APACrefauthors}%
Sonnerup, B\BPBI U\BPBI O.%
\BCBT {}\ \BBA {} Ledley, B\BPBI G.%
\end{APACrefauthors}%
\unskip\
\newblock
\APACrefYearMonthDay{1974}{}{}.
\newblock
{\BBOQ}\APACrefatitle {Magnetopause rotational forms} {Magnetopause rotational
  forms}.{\BBCQ}
\newblock
\APACjournalVolNumPages{Journal of Geophysical Research}{79}{28}{4309--4314}.
\newblock
\begin{APACrefURL}
  [{2023-01-20}]\url{http://doi.wiley.com/10.1029/JA079i028p04309}
  \end{APACrefURL}
\newblock
\begin{APACrefDOI} \doi{10.1029/JA079i028p04309} \end{APACrefDOI}
\PrintBackRefs{\CurrentBib}

\end{thebibliography}





\end{document}


