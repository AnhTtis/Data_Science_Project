\section{Dataset}
\label{sec:dataset}

\begin{figure*}[htb]
  \centering
  \includegraphics[width=.98\linewidth]{image/fig_dataset_gallery.jpeg}
  \caption{The \textbf{gallery} of the representative examples sampled from \textit{3DBiCar}. Each collected reference image is followed by the T-pose model and the posed model, created by professional artists. \textit{3DBiCar} contains 1,500 topologically consistent, textured and skinned 3D high-quality models with paired 2D images, which covers 15 species and 4 image styles.
  %Dataset Gallery of \textit{3DBiCar}: Here are 12 examples from all styles of images in \textit{3DBiCar}. Each example is represented by three graphs, a 2D picture, a 3D T-Pose model, and a 3D posed model, respectively, from left to right.
  }
  \label{fig_dataset_gallery}
\end{figure*}
\begin{figure}[htb]
  \centering
  \includegraphics[width=.98 \linewidth]{image/fig_table.pdf}
  \caption{\textbf{Data distribution.} Chart (a) illustrates the number of 15 species of bipedal cartoon characters in \textit{3DBiCar}. Chart (b) shows the number of four styles of reference images collected in our dataset.}
  \label{fig_datainfo}
\end{figure}

% \iffalse
% \begin{figure*}[htb]
%   \centering
%   \includegraphics[width= .98 \linewidth]{image/fig_dataset_pipeline.pdf}
%   \caption{Dataset Building Pipeline: The generation process of a biped character model from an image.
%   }
%   \label{fig_dataset_pipeline}
% \end{figure*} \fi

\begin{figure}[htb]
  \centering
  \includegraphics[width=.98 \linewidth]{image/fig_template_design.pdf}
  \caption{\textbf{Template.} The models in the center are the predefined template mesh with landmarks. It can be seen that we refine the structure on specific regions, where a complex nose or tail may exist. The colored regions and delineated lines denote the landmarks. These landmarks represent specific components of the character's body, such as elbow and eye socket. During model crafting, artists are required to deform the template model while keeping the landmarks in the position where the original body components are.
  }
  \vspace{-0.3cm}
  \label{fig_template_design}
\end{figure}




%Recently, researchers have made significant progress in digitizing realistic and articulated human characters. It is already possible to generate relatively accurate 3D real humans from simple inputs, even a single-view image or sparse strokes. However, no existing works focused on the efficient generation of 3D biped cartoon characters, which a great demand in gaming and filming. 
Considerable progress has been made in digitizing realistic and articulated human characters. However, efficiently creating visually plausible biped cartoon characters remains demanding and challenging, mainly due to the lack of data. In this work, we propose to fill this gap by introducing \textit{3DBiCar}, the first large-scale full-body 3D biped character data. We build \textit{3DBiCar} following three rules:

% The lack of large-scale 3D character datasets in the past has led to the stagnation of this field, so we build \textit{3DBiCar}, a large-scale topologically consistent 3D biped cartoon character dataset. In the following part, We will elaborate on the pipeline to build \textit{3DBiCar}, and for a brief dataset demonstration, please refer to Fig.~\ref{fig_dataset_gallery}. 

% image collection varies
% We start establishing our \textit{3DBiCar} by searching diverse images. We carefully select 1,500  biped character images among 17 species in 4 styles from the Internet and e-books. They are vital references for later 3D model crafting and challenging inputs for possible reconstruction tasks. %precious resource?

\textbf{Diversity.} \textit{3DBiCar} spans a wide range of 3D biped cartoon characters, containing 1,500 high-quality 3D models. First, we carefully collect images of 2D full-body biped cartoon characters with diverse identities, shape, and textural styles from the Internet, resulting in 15 character species and 4 image styles, as shown in Fig.~\ref{fig_datainfo}. Then we recruit six professional artists to create 3D corresponding character models according to the collected reference images. The modeling result is required to be matched with the reference images as much as possible. The representative image-model pairs sampled from our dataset are shown in Fig.~\ref{fig_dataset_gallery}. 

\textbf{Topological-consistency.} 
The key to building a linear parametric shape model is keeping a unified mesh topology. Traditional human parametric models utilize a template mesh to register different human body scans with 3D landmarks to keep topologically uniform. Inspired by this, we first create a template mesh with several 3D colored landmarks as shown in Fig.~\ref{fig_template_design}. All six artists are required to craft 3D models by deforming the above-predefined template under the constraints of these obvious landmarks. We set up a review committee of 10 to check these models based on the landmarks, ensuring the consistency of mesh topology. The landmarks could also be used to compute the position of models' joints for body posing or character animation. The topological consistency of \textit{3DBiCar} paves the way to learn a skinned parametric model, which we will discuss in Sec.~\ref{sec:algorithm}.


\textbf{Richness.} We provide various forms of data for each character. There are not only the 3D shape meshes and UV-space textures carefully crafted by artists but also collected reference images. For each character, artists are asked first to create a T-pose mesh and then deform it to match the reference pose. Furthermore, all the models are rigged and skinned using a predefined skeleton and skinning weight matrix, which enables further animation production for characters. In addition, each character contains two separate eyeballs specifically designed for facial animation. The body mesh of each character comprises 38,726 vertices and 77,448 faces, while each eyeball consists of 1,025 vertices and 2,046 faces.

% The artists are asked to produce the corresponding 3D shape, pose and texture according to the reference images. Therefore, \textit{3DBiCar} provides the shape, pose, texture map, and the corresponding 2D reference image for each model simultaneously, which could be directly applied to several vital tasks in visual computing such as single-view reconstruction, pose tracking, and texture synthesis.