\section{Details of Sketch-based Modeling}

\noindent\textbf{Data preparation.} We first sample 12,000 shape vectors randomly and feed them to \textit{RaBit} to generate 3D cartoon characters with diversified shapes. Then the suggestive contour~\cite{decarlo2003suggestive, han2017deepsketch2face} is applied to render the front-view sketches with different abstraction levels and obtain 108,000 sketch-model pairs. Fig.~\ref{fig_sketch_data} shows examples of rendered sketches. 

\begin{figure}[htbp]
  \centering
  \includegraphics[width=.98\linewidth]{supp_image/shape_data_2.pdf}
  \caption{An illustration of rendered sketches used for training.}
  \label{fig_sketch_data}
\end{figure}

\noindent\textbf{Implementations.} As shown in Fig.~\ref{fig_sketch_pipeline}, we first adopt one ResNet-50 module and three MLPs as the encoder-decoder architecture, mapping the input sketch $512 \times 512$ to 100-dimensional shape parameters. Then the generated shape parameters are fed to \textit{RaBit} to reconstruct the corresponding 3D model. We train the network with a batch size of $100$ and a learning rate of $3 \times 10^{-4}$ with the Adam optimizer. Moreover, we use the $L_1$ loss to measure the difference between the predicted shape parameters and the ground truth. Our sketch-based modeling interface is implemented with the QT framework. CGAL is adopted for 3D geometry processing. As shown in the video, running on a personal computer with an Intel i7-7700 CPU, 16GB RAM, and a single Nvidia GTX 2080Ti GPU, our modeling application supports real-time feedback. 

\begin{figure}[htbp]
  \centering
  \includegraphics[width=.95 \linewidth]{supp_image/pipeline_sketch2.pdf}
  \caption{The pipeline of our sketch-based modeling. Given a sketch $512 \times 512$ as input, we employ one ResNet-50 module and three MLPs to embed the input to 100-dimensional shape parameters. The output shape parameters are fed to \textit{RaBit} to reconstruct the corresponding 3D model.}
  \label{fig_sketch_pipeline}
\end{figure}