\begin{figure}[htbp]
\centering
\begin{subfigure}[b]{0.11\textwidth}
        \centering
        \includegraphics[width=0.9\textwidth]{image/SMCL_process/fig_pred1.pdf}
        \caption{$\bar{M_S}$}
\end{subfigure}
\hfill %\hspace{0mm}
\begin{subfigure}[b]{0.11\textwidth}
        \centering
        \includegraphics[width=0.9\textwidth]{image/SMCL_process/fig_pred2.pdf}
        \caption{$F_S(B)$}
\end{subfigure}
\hfill %\hspace{0mm}
\begin{subfigure}[b]{0.11\textwidth}
        \centering
        \includegraphics[width=0.9\textwidth]{image/SMCL_process/fig_pred3.pdf}
        \caption{$F_P(M_S,\Theta)$}
\end{subfigure}
\hfill %\hspace{0mm}
\begin{subfigure}[b]{0.11\textwidth}
        \centering
        \includegraphics[width=0.9\textwidth]{image/SMCL_process/fig_pred4.pdf}
        \caption{$F_T(M_P,T)$}
\end{subfigure}
\caption{\textit{\textbf{RaBit}.} (a) indicates the mean T-pose model $\bar{M}_{S}$. Given shape parameter $B$ for body, a specific T-pose biped character is next reconstructed as (b) shows. (c) demonstrates the posed model after applying pose parameter $\Theta$. Finally, a complete textured model (d) is obtained with texture parameter $T$.}
\label{fig_SMBL_process}
\end{figure}