% \iffalse
% \begin{figure*}[htb]
%   \centering
%   \includegraphics[width= .98 \linewidth]{image/fig_dataset_pipeline.pdf}
%   \caption{Dataset Building Pipeline: The generation process of a biped character model from an image.
%   }
%   \label{fig_dataset_pipeline}
% \end{figure*} \fi

\begin{figure}[htb]
  \centering
  \includegraphics[width=.98 \linewidth]{image/fig_template_design.pdf}
  \caption{\textbf{Template.} The models in the center are the predefined template mesh with landmarks. It can be seen that we refine the structure on specific regions, where a complex nose or tail may exist. The colored regions and delineated lines denote the landmarks. These landmarks represent specific components of the character's body, such as elbow and eye socket. During model crafting, artists are required to deform the template model while keeping the landmarks in the position where the original body components are.
  }
  \vspace{-0.3cm}
  \label{fig_template_design}
\end{figure}


