\def\cvprPaperID{7892}
\def\confName{CVPR}
\def\confYear{2023}

\def\paperTitle{Paper Title}

\def\authorBlock{
        Kun Su$^1$\thanks{Work done while interning at MIT-IBM Watson AI Lab} \qquad
    Kaizhi Qian$^2$ \qquad
    Eli Shlizerman$^1$ \qquad
    Antonio Torralba$^3$ \qquad
    Chuang Gan$^{2,4}$ \qquad \\
    $^1$University of Washington \qquad
    $^2$MIT-IBM Watson AI Lab \qquad
    $^3$MIT \qquad
    $^4$UMass Amherst\\
    % {\tt\small \{email, addresses\}@inst.edu}
}

% Compilation vars
\newif\ifreview \newcommand{\review}{\reviewtrue}
\newif\ifarxiv \newcommand{\arxiv}{\arxivtrue}
\newif\ifcamera \newcommand{\cameraready}{\cameratrue}
\newif\ifrebuttal \newcommand{\rebuttal}{\rebuttaltrue}


% \cameraready % \review OR \arxiv OR \cameraready
\arxiv
\pdfoutput=1
\documentclass[10pt,twocolumn,letterpaper]{article}

% \usepackage{graphicx}
% \usepackage{amsmath}
% \usepackage{booktabs}
% \usepackage{multirow}
% \usepackage{xcolor}
% \usepackage{amssymb}
% \usepackage{diagbox}
% \usepackage{subfigure}
% \usepackage{graphicx}
% \usepackage[colorlinks,linkcolor=red]{hyperref}
% \usepackage[accsupp]{axessibility}

% \newcommand{\hc}[1]{{\color{red}            {[HC: #1]}}}
% \newcommand{\cscCom}[1]{{\color{purple}            {[SC: #1]}}}
% \newcommand{\zj}[1]{{\color{blue}            {#1}}}

\ifreview \usepackage[review]{cvpr} \fi
\ifarxiv \usepackage[pagenumbers]{cvpr} \fi
\ifrebuttal \usepackage[rebuttal]{cvpr} \fi
\ifcamera \usepackage{cvpr} \fi

\usepackage{graphicx}
\usepackage{amsmath}
\usepackage{amssymb}
\usepackage{booktabs}

%% PACKAGES (also see cvpr_header.tex)
\usepackage{times}
\usepackage{microtype}
\usepackage{epsfig}
\usepackage[dvipsnames,table,xcdraw]{xcolor}
\usepackage{caption}
\usepackage{float}
\usepackage{placeins}
\usepackage{color, colortbl}
\usepackage{stfloats}
\usepackage{enumitem}
\usepackage{tabularx}
\usepackage{xstring}
\usepackage{multirow}
\usepackage{xspace}
\usepackage{url}
\usepackage{soul}
\usepackage{subcaption}
\usepackage{xcolor}
\usepackage[hang,flushmargin]{footmisc}
\usepackage{amssymb}% http://ctan.org/pkg/amssymb
\usepackage{pifont}% http://ctan.org/pkg/pifont
\usepackage[accsupp]{axessibility}

\newcommand{\cmark}{\ding{51}}%
\newcommand{\xmark}{\ding{55}}%

\captionsetup{skip=2pt}
\setlength{\textfloatsep}{8.0pt plus 2.0pt minus 2.0pt}
\setlength{\floatsep}{8.0pt plus 2.0pt minus 2.0pt}
\setlength{\intextsep}{8.0pt plus 2.0pt minus 2.0pt}
\setlength{\dbltextfloatsep}{8.0pt plus 2.0pt minus 2.0pt}
\setlength{\dblfloatsep}{8.0pt plus 2.0pt minus 2.0pt}
% \titlespacing*{\section}{0pt}{*1}{*0}
% \titlespacing*{\subsection}{0pt}{*1}{*0}
% \titlespacing*{\subsubsection}{0pt}{*1}{*0}

\makeatletter
\renewcommand{\paragraph}{%
  \@startsection{paragraph}{4}%
  {\z@}{1ex \@plus .2ex \@minus .2ex}{-1em}%
  {\normalfont\normalsize\bfseries}%
}
\makeatother

% Unfortunately, this package interferes with arxiv's stamp
\ifcamera \usepackage[accsupp]{axessibility} \fi

%% MACROS

% \newcommand{\authorname}[1]{{\textcolor{blue}{[Author: #1]}}}
% ...

% \newcommand{\commandname}{string\xspace}
% \definecolor{colorname}{rgb}{0.92,0.49,0.19}

% General

\newcommand{\nbf}[1]{{\noindent \textbf{#1.}}}

\newcommand{\supp}{supplemental material\xspace}
\ifarxiv \renewcommand{\supp}{appendix\xspace} \fi

\newcommand{\todo}[1]{{\textcolor{red}{[TODO: #1]}}}

% Reviewer commands (1 to 5), e.g. \R{1}, \R{2}
\newcommand{\R}[1]{{%
    \textbf{%
        \ifstrequal{#1}{hmSm}{\textcolor{red}{#1}}{%
        \ifstrequal{#1}{1}{\textcolor{red}{R#1}}{%
        \ifstrequal{#1}{GrqZ}{\textcolor{blue}{#1}}{%
        \ifstrequal{#1}{2}{\textcolor{blue}{R#1}}{%
        \ifstrequal{#1}{GTBY}{\textcolor{teal}{#1}}{%
        \ifstrequal{#1}{3}{\textcolor{teal}{R#1}}{%
        \ifstrequal{#1}{4}{\textcolor{magenta}{R#1}}{%
                           \textcolor{cyan}{R#1}%
        }}}}}}}%
    }%
}}

\newenvironment{packed_enum}{
\begin{enumerate}
  \setlength{\itemsep}{1pt}
  \setlength{\parskip}{2pt}
  \setlength{\parsep}{0pt}
}{\end{enumerate}}

\newenvironment{packed_item}{
\begin{itemize}
  \setlength{\itemsep}{1pt}
  \setlength{\parskip}{2pt}
  \setlength{\parsep}{0pt}
}{\end{itemize}}

\def\eg{\emph{e.g.}}
\def\Eg{\emph{E.g.}}
\def\ie{\emph{i.e.}}
\def\etal{\emph{et al.}}
\def\etc{\emph{etc.}}
\definecolor{attr}{RGB}{192,0,0}
\definecolor{comp}{RGB}{0,164,74}
\definecolor{obj}{RGB}{0,112,192}
\definecolor{tabcolor}{RGB}{237,244,254}
\definecolor{branchone}{RGB}{85,131,53}
\definecolor{branchtwo}{RGB}{197,91,17}
\newcommand{\framework}{ADE\xspace}
% \newcommand{\hsznew}[1]{\textcolor{blue}{#1}}
\newcommand{\hsznew}[1]{#1}
\newcommand{\hsz}[1]{#1}
% \newcommand{\R1}{\textcolor{red}{hmSm}}
% \newcommand{\R2}{\textcolor{blue}{GrqZ}}
% \newcommand{\R3}{\textcolor{Green}{GTBY}}

\usepackage[ruled,vlined]{algorithm2e}
\setlength{\algomargin}{0pt}
\definecolor{commentcolor}{RGB}{110,154,155}   % define comment color
\newcommand{\PyComment}[1]{\ttfamily\textcolor{commentcolor}{\# #1}}  % add a "#" before the input text "#1"
\newcommand{\PyCode}[1]{\ttfamily\textcolor{black}{#1}} % \ttfamily is the code font

\usepackage[normalem]{ulem}
\newcommand{\li}{\uline{\hspace{0.5em}}}  %

\usepackage{xr-hyper}

\makeatletter
\newcommand*{\addFileDependency}[1]{
  \typeout{(#1)}
  \@addtofilelist{#1}
  \IfFileExists{#1}{}{\typeout{No file #1.}}
}

\makeatother
\newcommand*{\myexternaldocument}[1]{
    \externaldocument{#1}
    \addFileDependency{#1.tex}
    \addFileDependency{#1.aux}
}



\makeatletter
\AfterEndEnvironment{algorithm}{\let\@algcomment\relax}
\AtEndEnvironment{algorithm}{\kern2pt\hrule\relax\vskip3pt\@algcomment}
\let\@algcomment\relax
\newcommand\algcomment[1]{\def\@algcomment{\footnotesize#1}}
\renewcommand\fs@ruled{\def\@fs@cfont{\bfseries}\let\@fs@capt\floatc@ruled
  \def\@fs@pre{\hrule height.8pt depth0pt \kern2pt}%
  \def\@fs@post{}%
  \def\@fs@mid{\kern2pt\hrule\kern2pt}%
  \let\@fs@iftopcapt\iftrue}
\makeatother

\definecolor{citecolor}{RGB}{0, 113, 188}
\definecolor{rosepink}{RGB}{255, 0, 127}
\usepackage[pagebackref,breaklinks,colorlinks, citecolor=citecolor]{hyperref}

\usepackage[capitalize]{cleveref}
\crefname{section}{Sec.}{Secs.}
\crefname{table}{Table}{Tables}
\crefname{figure}{Fig.}{Figs.}




\frenchspacing


\usepackage[accsupp]{axessibility} % Improves PDF readability for those with disabilities.

\begin{document}
%% TITLE
\title{\emph{RaBit}: Pa\textit{ra}metric Modeling of 3D \textit{Bi}ped Car\textit{t}oon Characters \\ with a Topological-consistent Dataset}
\author{\authorBlock}
%\maketitle
%%

\twocolumn[{%
  \renewcommand\twocolumn[1][]{#1}%
\maketitle
\begin{center}
  \newcommand{\teaserwidth}{\textwidth}
  \vspace{-0.15in}
  \centerline{
    \includegraphics[width=.95\teaserwidth]{image/fig_teaser4.png}
    }
    % \vspace{-10pt}
    \captionof{figure}{
     We present \textit{\textbf{3DBiCar}}, the first large-scale repository of 3D biped cartoon characters. It contains 1,500 topologically consistent, textured, and skinned 3D high-quality meshes manually created by professional artists, covering 15 species. Further, we propose \textit{RaBit}, the first cartoon character parametric model simultaneously parameterizing shape, pose, and texture. %With \textit{3DBiCar} and \textit{RaBit}, we conduct various applications to demonstrate their promising potential for efficient biped character digitization.
    }
  %\vspace{-0.05in}
  \label{fig_teaser}
 \end{center}%
    }]
\authorFootnote

% \begin{center}
\small
\setlength{\fboxrule}{1pt}
\setlength{\tabcolsep}{2pt}
\begin{tabular}{ccc}
\centeredtab{\begin{tikzpicture}
 \foreach \X [count=\Z]in {fig/teaser/input/000000.jpg,fig/teaser/input/000194.jpg,fig/teaser/input/000395.jpg,fig/teaser/input/000600.jpg,fig/teaser/input/000799.jpg,fig/teaser/input/000989.jpg,fig/teaser/input/001081.jpg}
 {\node[opacity=1] at (\Z/5,-\Z/2.5,0) {\fbox{\includegraphics[width=3cm]{\X}}};}
\end{tikzpicture} \\
Input: casually captured long video \\
\\ \\ \\
\includegraphics[width=5cm]{fig/teaser/teaser_plot3.png} \\
Output: jointly estimated camera poses \\ and local radiance fields
} &
\setlength{\tabcolsep}{1pt}
\renewcommand{\arraystretch}{0.8}
\begin{tabular}{cccc}
{\includegraphics[width=3.9cm]{fig/teaser/000020.jpg}} & 
{\includegraphics[width=3.9cm]{fig/teaser/000530.jpg}} & 
{\includegraphics[width=3.9cm]{fig/teaser/001190.jpg}} \\
\multicolumn{3}{c}{LocalRF (ours): high-quality novel view synthesis} \\
\rule{0pt}{2.35cm}{\includegraphics[width=3.9cm]{fig/teaser/rgb_2.jpg}} & 
{\includegraphics[width=3.9cm]{fig/teaser/rgb_53.jpg}} & 
{\includegraphics[width=3.9cm]{fig/teaser/rgb_119.jpg}} \\
\multicolumn{3}{c}{BARF~\cite{lin2021barf}: the estimated poses often fall into local minima for long sequences} \\
\rule{0pt}{2.35cm}{\includegraphics[width=3.9cm]{fig/teaser/color_002.jpg}} & 
{\includegraphics[width=3.9cm]{fig/teaser/color_053.jpg}} & 
{\includegraphics[width=3.9cm]{fig/teaser/color_119.jpg}} \\
\multicolumn{3}{c}{Mip-NeRF360~\cite{barron2022mipnerf360}: the spatial resolution is often limited throughout the video}
\end{tabular}
\end{tabular}%
\captionof{figure}{
\label{fig:teaser}
\textbf{High-quality novel view synthesis from a long casually captured video.} 
We jointly optimize camera poses and a scene representation using a progressive scheme that dynamically allocates local radiance fields (blue boxes).
Our method robustly handles casual hand-held captures, scales to processing arbitrarily long videos with limited memory usage, and maintains high resolution throughout the entire video.
}
\end{center}



\begin{abstract}
% \vspace{-1em}
The diffusion-based generative models have achieved remarkable success in text-based image generation. However, since it contains enormous randomness in generation progress, it is still challenging to apply such models for real-world visual content editing, especially in videos. 
In this paper, we propose \texttt{FateZero}, a zero-shot text-based editing method on real-world videos without per-prompt training or use-specific mask. 
\RM{Specifically, different from a pipeline of two independent inversion and then generation stages, we find the intermediate attention maps during inversions store better structure and motion information. We thus reform them to temporally casual attention and replace them in the generation progress. To further reduce the unnecessary semantic leakage of source video and enhance the editing quality, we then remix the temporally casual attentions via the cross-attention features of the source prompt as the mask.}
To edit videos consistently, we propose several techniques based on the pre-trained models. Firstly, in contrast to the straightforward DDIM inversion technique, our approach captures intermediate attention maps during inversion, which effectively retain both structural and motion information. These maps are directly fused in the editing process rather than generated during denoising. To further minimize semantic leakage of the source video, we then fuse self-attentions with a blending mask obtained by cross-attention features from the source prompt. Furthermore, we have implemented a reform of the self-attention mechanism in denoising UNet by introducing spatial-temporal attention to ensure frame consistency.
Yet succinct, our method is the first one to show the ability of zero-shot text-driven video style and local attribute editing from the trained text-to-image model. We also have a better zero-shot shape-aware editing ability based on the text-to-video model~\cite{tuneavideo}. \RM{Besides video, our unified method also achieves state-of-the-art performance in zero-shot image editing.\chenyang{Need exp or remove the zero-shot image}} Extensive experiments demonstrate our superior temporal consistency and editing capability than previous works.
% The code will be released.
% \chenyang{emphasize: our observation at inversion time} \xiaodong{replacing the bold part to the actual pipeline: \textbf{Specifically, we work on replacing and mixing the attention maps between the inversion and generation since the self-attention map keeps the structure of the original natural image and the cross-attention is semantic-related, after remixing, we replace them in the corresponding generation steps for denoising.}}
% \footnote{Since there is no general video diffusion model is publicly available, we use one-shot video generation method~(Tune-A-Video~\cite{tuneavideo}) as the pretrained video diffusion model for zero-shot video editing\xiaodong{can be removed if we actually zero-shot on video}.}.
\end{abstract}
\section{Introduction}

\label{sec:intro}

% \textit{"Drawing and colour are not separate at all; in so far as you paint, you draw. The more the colour harmonizes, the more exact the drawing becomes."} - Paul Cezanne.

Art is a reflection of the figments of human imagination. 
While many are limited in their practical creative capabilities, by pushing the boundaries of digital media, new ways can be created for casual artists and experts alike to express their ideas. At the same time, current neural generative art takes away much of the control from humans. In this work, we attempt to take a step towards restoring some of that control, enabling neural networks to complement users and naturally extend their skills rather than taking hold over the generative process.

% \orr{TBD - make the opening colorful : 1. Add quote:  2. Elaborate: art is a rendering of figments of imaginations of humans. Most people are limited in their drawing capabilities, and by pushing the boundaries we allow new ways for casual artists and experts alike in expressing ideas. At the same time, neural generative art takes a lot of the control away. Here, we want to give back some of this control to humans, such that neural networks complement them and compensate their lack of skills, rather than replacing them.}

% The field of image synthesis has been significantly propelled by neural generative models, particularly by the latest text-to-image models that predominantly rely on large language-image models ~\cite{balaji2022eDiff-I, ramesh2022dalle, rombach2021highresolution, imagen2022saharia}. These models have revolutionized the field of computer vision as they can produce astonishing visual outcomes from text prompts only.

The field of image synthesis has been significantly propelled by neural generative models, particularly by the latest text-to-image models that predominantly rely on large language-image models ~\cite{balaji2022eDiff-I, ramesh2022dalle, rombach2021highresolution, imagen2022saharia}. These models have revolutionized the field of computer vision, as they can produce astonishing visual outcomes from text prompts alone.

The ability of text-to-image models has sparked a wave of editing methods that utilize these models. Many of these techniques rely on prompt editing ~\cite{ fu2022shapecrafter, hertz2022prompt, kawar2022imagic,lin2022magic3d,mokady2022null, poole2022dreamfusion}. Nevertheless, simplifying the interface to text alone means users lack the necessary level of granularity to produce their exact desired outcomes.
% which is} insufficient for effectively editing local content. 
% editing and manipulating visual content, as users lack the necessary level of control to achieve their desired outcomes
Sketch-guided editing, on the other hand, provides intuitive control that aligns with user's conventional drawing and painting skills. By incorporating user-guided sketches into text-to-image models, powerful editing systems can be created, offering a high degree of flexibility and fine-grained control for manipulating visual content~\cite{zhang2023controlnet, voynov2022sketch}.

Although sketch-guided and text-driven methods have proven successful in generating and manipulating 2D images \cite{meng2022sdedit, voynov2022sketch, cheng2023wacv}, it immediately raises the intriguing question of whether a similar approach could be developed to edit 3D shapes. 
Since direct text-to-3D models require an abundance of data to scale, state-of-the-art 3D generative models, such as DreamFusion~\cite{poole2022dreamfusion} and Magic3D~\cite{lin2022magic3d}, which build on the capabilities of text-to-image models, may be considered as an alternative.
% Due to the difficulty of scaling general direct text-to-3D models, incorporating conditions into a text-to-3D model is not straightforward. Thus, state of the art 3D generative models, such as DreamFusion~\cite{poole2022dreamfusion} \orrc{and Magic3D~\cite{lin2022magic3d}}, which build on the capabilities of text-to-image models, may be considered as an alternative.
However, maintaining control via conditioning with such models remains a challenging task, as these generative pipelines optimize a Neural Radiance Field (NeRF) \cite{mildenhall2020nerf} by amortizing gradients from a multitude of 2D views. In particular, providing consistent sketches across all possible views presents a hurdle for users. Instead, a plausible user interface should act with guidance from as few views as possible, e.g. up to two or three.


In this paper, we present \textbf{SKED}, a \textbf{SK}etch-guided 3D \textbf{ED}iting technique. Our method acts on reconstructed or generated NeRF models. We assume a text prompt and a minimum of two sketches as input and provide output edits over the neural field faithful to the input conditions.
Meeting all input requirements can be challenging as the text prompt may not match the sketch's semantics, and sketch views may lack coherence.
To undertake this complex task, we conceptually break it down into two subtasks that are easier to handle: one that depends on pure geometric reasoning and the other that exploits the rich semantic knowledge of the generative model. These two subtasks work together, with the former providing a coarse estimate of location and boundary, and the latter adding and refining geometric and texture details through fine-grained operations.


Our experiments highlight the effectiveness of our approach for editing various pretrained NeRF instances. We introduce assorted accessories, objects, and artifacts, which are generated and blended into the original neural field seamlessly.
Finally, we validate our method through quantitative evaluations and ablation studies to assert the contribution of individual components in our method. 
% By presenting examples in the paper, we illustrate that our method can generate realistic 3D artifacts with accurate texture and geometry using only a few basic sketches.



% Due to the absence of a direct text-to-3D model, incorporating conditions into a text-to-3D model is not straightforward. Thus, 3D generative models, such as DreamFusion~\cite{poole2022dreamfusion}, which build on the capabilities of text-to-image models, may be considered as an alternative.
% However, this is a challenging task since DreamFusion generates a NeRF by integrating many different 2D views. It is very hard to provide consistent sketches across all possible views. The challenge is to use sketches as a guide on only a few views (e.g., two or three) and generate 3D edit of the existing NeRF that is subject to being edited. 

% In this paper, we present \textbf{SKED}, a \textbf{SK}etch-guided 3D \textbf{ED}iting technique, that takes as input a text prompt and a few (two or more) sketches and edits a 3D given object represented as a NeRF in a geometrically plausible and controlled way. 
% We acknowledge the difficulty of this task, as there are no existing text-to-3D generative models available for manipulating the geometry of the existing object based on a text prompt. 
% To undertake this complex task, we conceptually break it down into two simpler subtasks that are easier to handle: one that depends on pure geometric reasoning and the other that exploits the rich semantic knowledge of generative model. These two subtasks work together, with the former providing a coarse estimate of location and the latter adding and refining geometric and texture details through fine-grained operations.

% Our experiments showcase the effectiveness of our approach in performing sketch-guided text-based edits on different base nerfs by introducing various accessories, objects, and artifacts. We also conduct ablation studies and experiments to evaluate the performance of individual components in our method. By presenting examples in the paper, we illustrate that our method can generate realistic 3D artifacts with accurate texture and geometry using only a few basic sketches.

%\dcc{Add here the traditional paragraph that tell about what we achieved and evaluated}
  While  submodular optimization problems are generally NP-hard, the celebrated greedy algorithm \cite{nemhauser1978analysis} attains a $(1-1/e)$ approximation ratio for  submodular maximization subject to uniform matroids and a $1/2$ approximation ratio for general matroid constraints. As discussed in the introduction, the  continuous greedy algorithm \cite{calinescu2011maximizing} restores the $(1-1/e)$ approximation ratio by lifting the discrete problem to the continuous domain via the multilinear relaxation. %It is worth to mention here that the multilinear relaxation is a DR-submodular function, a.k.a. a continuous function with the diminishing returns property.

Stochastic submodular maximization, in which the objective is expressed as an expectation, has gained a lot of interest in the recent years \cite{asadpour2008stochastic, zhang2022stochastic, chen2018online}. Karimi et al. \cite{karimi2017stochastic} use a concave relaxation method that achieves the $(1-1/e)$ approximation guarantee, but only  for the class of submodular coverage functions. Hassani et al.~\cite{hassani2017gradient} provide projected gradients methods for the general case of stochastic submodular problems that achieve $1/2$ approximation guarantee.  Mokhtari et al. \cite{mokhtari2020stochastic} propose stochastic  conditional gradient methods for solving both minimization and maximization  stochastic submodular optimization problems. Their method for maximization, Stochastic Continous Greedy (SCG) can be interpreted as a stochastic variant of the continuous greedy algorithm \cite{vondrak2008optimal, calinescu2011maximizing} and achieves a tight $(1-1/e)$ approximation guarantee for monotone and submodular functions. %However, all these methods suffer from two sources of randomness (one comes from sampling the objective function and the other comes from estimating the multilinear relaxation via sampling its inputs).

Our work builds upon and relies on the approach by  \"{O}zcan et al.~\cite{ozcan2021submodular}, who studied ways of accelerating the computation of gradients via a polynomial estimator. Extending on the work of Mahdian et al.~\cite{mahdian2020kelly},  \"{O}zcan et al. show that submodular functions that can be written as compositions of (a) an analytic function and (b) a multilinear function can be arbitrarily well approximated via Taylor polynomials; in turn, this gives rise to a method for approximating their multilinear relaxation in a closed form, without sampling. We leverage this method in the context of stochastic submodular optimization, showing that it can also be applied in combination with SCG of Mokhtari et al.~\cite{mokhtari2020stochastic}: this eliminates one of the two sources of randomness, thereby reducing variance at the expense of added bias. From a technical standpoint, this requires controlling the error introduced by the bias of the polynomial estimator, while simultaneously accounting for the variance inherent in SCG, due to sampling instances.   %: this eliminates the latter source of randomness by utilizing the properties of deep submodular models that result from composition over multiple layers. In order to do so, we combine the stochastic continuous greedy algorithm proposed by Mokthari et al. \cite{mokhtari2020stochastic} with the deterministic estimator proposed by
\section{Dataset}
\label{sec:dataset}

\begin{figure*}[htb]
  \centering
  \includegraphics[width=.98\linewidth]{image/fig_dataset_gallery.jpeg}
  \caption{The \textbf{gallery} of the representative examples sampled from \textit{3DBiCar}. Each collected reference image is followed by the T-pose model and the posed model, created by professional artists. \textit{3DBiCar} contains 1,500 topologically consistent, textured and skinned 3D high-quality models with paired 2D images, which covers 15 species and 4 image styles.
  %Dataset Gallery of \textit{3DBiCar}: Here are 12 examples from all styles of images in \textit{3DBiCar}. Each example is represented by three graphs, a 2D picture, a 3D T-Pose model, and a 3D posed model, respectively, from left to right.
  }
  \label{fig_dataset_gallery}
\end{figure*}
\begin{figure}[htb]
  \centering
  \includegraphics[width=.98 \linewidth]{image/fig_table.pdf}
  \caption{\textbf{Data distribution.} Chart (a) illustrates the number of 15 species of bipedal cartoon characters in \textit{3DBiCar}. Chart (b) shows the number of four styles of reference images collected in our dataset.}
  \label{fig_datainfo}
\end{figure}

% \iffalse
% \begin{figure*}[htb]
%   \centering
%   \includegraphics[width= .98 \linewidth]{image/fig_dataset_pipeline.pdf}
%   \caption{Dataset Building Pipeline: The generation process of a biped character model from an image.
%   }
%   \label{fig_dataset_pipeline}
% \end{figure*} \fi

\begin{figure}[htb]
  \centering
  \includegraphics[width=.98 \linewidth]{image/fig_template_design.pdf}
  \caption{\textbf{Template.} The models in the center are the predefined template mesh with landmarks. It can be seen that we refine the structure on specific regions, where a complex nose or tail may exist. The colored regions and delineated lines denote the landmarks. These landmarks represent specific components of the character's body, such as elbow and eye socket. During model crafting, artists are required to deform the template model while keeping the landmarks in the position where the original body components are.
  }
  \vspace{-0.3cm}
  \label{fig_template_design}
\end{figure}




%Recently, researchers have made significant progress in digitizing realistic and articulated human characters. It is already possible to generate relatively accurate 3D real humans from simple inputs, even a single-view image or sparse strokes. However, no existing works focused on the efficient generation of 3D biped cartoon characters, which a great demand in gaming and filming. 
Considerable progress has been made in digitizing realistic and articulated human characters. However, efficiently creating visually plausible biped cartoon characters remains demanding and challenging, mainly due to the lack of data. In this work, we propose to fill this gap by introducing \textit{3DBiCar}, the first large-scale full-body 3D biped character data. We build \textit{3DBiCar} following three rules:

% The lack of large-scale 3D character datasets in the past has led to the stagnation of this field, so we build \textit{3DBiCar}, a large-scale topologically consistent 3D biped cartoon character dataset. In the following part, We will elaborate on the pipeline to build \textit{3DBiCar}, and for a brief dataset demonstration, please refer to Fig.~\ref{fig_dataset_gallery}. 

% image collection varies
% We start establishing our \textit{3DBiCar} by searching diverse images. We carefully select 1,500  biped character images among 17 species in 4 styles from the Internet and e-books. They are vital references for later 3D model crafting and challenging inputs for possible reconstruction tasks. %precious resource?

\textbf{Diversity.} \textit{3DBiCar} spans a wide range of 3D biped cartoon characters, containing 1,500 high-quality 3D models. First, we carefully collect images of 2D full-body biped cartoon characters with diverse identities, shape, and textural styles from the Internet, resulting in 15 character species and 4 image styles, as shown in Fig.~\ref{fig_datainfo}. Then we recruit six professional artists to create 3D corresponding character models according to the collected reference images. The modeling result is required to be matched with the reference images as much as possible. The representative image-model pairs sampled from our dataset are shown in Fig.~\ref{fig_dataset_gallery}. 

\textbf{Topological-consistency.} 
The key to building a linear parametric shape model is keeping a unified mesh topology. Traditional human parametric models utilize a template mesh to register different human body scans with 3D landmarks to keep topologically uniform. Inspired by this, we first create a template mesh with several 3D colored landmarks as shown in Fig.~\ref{fig_template_design}. All six artists are required to craft 3D models by deforming the above-predefined template under the constraints of these obvious landmarks. We set up a review committee of 10 to check these models based on the landmarks, ensuring the consistency of mesh topology. The landmarks could also be used to compute the position of models' joints for body posing or character animation. The topological consistency of \textit{3DBiCar} paves the way to learn a skinned parametric model, which we will discuss in Sec.~\ref{sec:algorithm}.


\textbf{Richness.} We provide various forms of data for each character. There are not only the 3D shape meshes and UV-space textures carefully crafted by artists but also collected reference images. For each character, artists are asked first to create a T-pose mesh and then deform it to match the reference pose. Furthermore, all the models are rigged and skinned using a predefined skeleton and skinning weight matrix, which enables further animation production for characters. In addition, each character contains two separate eyeballs specifically designed for facial animation. The body mesh of each character comprises 38,726 vertices and 77,448 faces, while each eyeball consists of 1,025 vertices and 2,046 faces.

% The artists are asked to produce the corresponding 3D shape, pose and texture according to the reference images. Therefore, \textit{3DBiCar} provides the shape, pose, texture map, and the corresponding 2D reference image for each model simultaneously, which could be directly applied to several vital tasks in visual computing such as single-view reconstruction, pose tracking, and texture synthesis.
\section{Method}
% \label{sec:method}

\begin{figure*}[t]
\centering
\includegraphics[width=\linewidth]{figures/framework_v3.pdf}
\vspace{-5mm}
\caption{An overview of our 3D-CLR framework. First, we learn a 3D compact scene representation from multi-view images using neural fields (I). Second, we use CLIP-LSeg model to get per-pixel 2D features (II). We utilize a 3D-2D alignment loss to assign features to the 3D compact representation (III). By calculating the dot-product attention between the 3D per-point features and CLIP language embeddings, we could get the concept grounding in 3D (IV). Finally, the reasoning process is performed via a set of neural reasoning operators, such as \textsc{Filter}, \textsc{Get\_Instance} and \textsc{Count\_Relation} (V). Relation operators are learned via relation networks.}
\vspace{-5mm}
\label{fig:framework}
\end{figure*}

Fig.~\ref{fig:framework} illustrates an overview of our framework. Specifically, our framework consists of three steps.  First, we learn a 3D compact representation from multi-view images using neural field. And then we propose to leverage pre-trained 2D vision-and-language model to ground concepts on 3D space. This is achieved by 1) generating 2D pixel features using CLIP-LSeg; 2) aligning the features of 3D voxel grid and 2D pixel features from CLIP- LSeg~\cite{li2022language}; 3) dot-product attention between the 3D features and CLIP language features~\cite{li2022language}. Finally, to perform visual reasoning, we propose neural reasoning operators, which execute the question step by step on the 3D compact representation and outputs a final answer. For example, we use \textsc{Filter} operators to ground semantic concepts on the 3D representation, \textsc{Get\_Instance} to get all instances of a semantic class, and \textsc{Count\_Relation} to count how many pairs of the two semantic classes have the queried relation.
% \gc{metion all the neural operators.}
% Works from linguistic and cognitive science suggest that semantic concepts are diverse and open-vocabulary, while relational concepts describing 3D objects' relationships can be very limited and thus can be considered a close-class vocabulary \cite{Landau1993WhatA, Hayward1995SpatialLA}. Therefore, it's unrealistic to learn the embeddings of all the concepts in the question-answering pairs, while it's more natural to learn the relation embeddings. Inspired by this, we propose to leverage 2D pretrained vision-language model (\textit{i.e.,} CLIP) for open-vocabulary semantic concept learning, while proposing a neural relation module network for relational reasoning. 

\subsection{Learning 3D Compact Scene Representations}

% Since 3D-related reasoning works on 3D compact representations rather than 2D images, we first propose to use a neural field to extract 3D representations from multi-view images. The next step is to learn the 3D features for visual reasoning. However, 3D assets are limited in diversity and scale, posing challenges for training large-scale 3D foundation models, while there's much progress on large-scale 2D pretrained models which provide decent features\cite{Radford2021LearningTV, Ramesh2021ZeroShotTG}. Since neural field maps a 2D pixel to several 3D points along the ray, it's natural to get 3D features for 2D per-pixel features. We apply CLIP-LSeg\cite{Li2022LanguagedrivenSS} to learn per-xel 2D features, and use an alignment loss to align 3D features with 2D features.

% \paragraph{3D Compact Representation from neural field.} 
Neural radiance fields  \cite{mildenhall2020nerf} are capable of learning a 3D representation that can reconstruct a volumetric 3D scene representation from a set of images. Voxel-based methods \cite{Garbin2021FastNeRFHN, Hedman2021BakingNR, Yu2021PlenOctreesFR, Sun2022DirectVG} speed up the learning process by explicitly storing the scene properties (\textit{e.g.}, density, color and feature) in its voxel grids. We leverage Direct Voxel Grid Optimization (DVGO) \cite{Sun2022DirectVG} as our backbone for 3D compact representation for its fast speed. DVGO stores the learned density and color properties in its grid cells. The rendering of multi-view images is by interpolating through the voxel grids to get the density and color for each sampled point along each sampled ray, and integrating the colors based on the rendering alpha weights calculated from densities according to quadrature rule \cite{Max1995OpticalMF}. The model is trained by minimizing the L2 loss between the rendered multi-view images and the ground-truth multi-view images. By extracting the density voxel grid, we can get the 3D compact representation (\textit{e.g.,} By visualizing points with density greater than 0.5, we can get the 3D representation as shown in Fig. \ref{fig:framework} I. ) 

\subsection{3D Semantic Concept Grounding}
Once we extract the 3D compact representation of the scene, we need to ground the semantic concepts for reasoning from language. 
Recent work from \cite{hong20223d} has proposed to ground concepts from paired 3D assets and question-answers. Though promising results have been achieved on synthetic data, it is not feasible for open-vocabulary 3D reasoning in real-world data, since it is hard to collect large-scale 3D vision-and-language paired data.  To address this challenge, our idea is to leverage  pre-trained 2D vision and language model \cite{Radford2021LearningTV, Ramesh2021ZeroShotTG} for 3D concept grounding in real-world scenes.  But how can we map 2D concepts into 3D neural field representations? Note that 3D compact representations can be learned from 2D multi-view images and that each 2D pixel actually corresponds to several 3D points along the ray. Therefore, it's possible to get 3D features from 2D per-pixel features. Inspired by this, we first add a feature voxel grid representation to DVGO, in addition to density and color, to represent 3D features. 
% it's natural to utilize 2D VLMs to ground semantic concepts on the 3D representations. 
 We then apply CLIP-LSeg\cite{li2022language} to learn per-pixel 2D features, which can be attended to by CLIP concept embeddings. We use an alignment loss to align 3D features with 2D features so that we can perform concept grounding on the 3D representations.
% Since 3D voxel grids and 2D pixels are aligned via alpha compositing, we add one L1 loss to force the features of 3D voxel grids to align with the 2D LSeg pixels based on the alpha values. 

\noindent\textbf{2D Feature Extraction.}
To get per-pixel features that can be attended by concept embeddings, we use the features from language-driven semantic segmentation (CLIP-LSeg) \cite{li2022language}, which learns 2D per-pixel features from a pre-trained vision-language model (\textit{i.e.,} \cite{Radford2021LearningTV}). Specifically, it
uses the text encoder from CLIP, trains an image encoder to produce an embedding vector for each pixel, and calculates the scores of word-pixel correlation by dot-product. By outputting the semantic class with the maximum score of each pixel, CLIP-LSeg is able to perform zero-shot 2D semantic segmentation.

\noindent\textbf{3D-2D Alignment.}
In addition to density and color, we also store a 512-dim feature in each grid cell in the compact representation. To align the 3D per-point features with 2D per-pixel features, we calculate an L1 loss between each pixel and each 3D point sampled on the ray of the pixel. The overall L1 loss along a ray is the weighted sum of all the pixel-point alignment losses, with weights same as the rendering weights: $\mathcal{L}_{\text {feature}}=\sum_{i=1}^K w_i(\|\boldsymbol{f_i}-F(\boldsymbol{r})\|),$
where $\boldsymbol{r}$ is a ray corresponding to a 2D pixel, $F(\boldsymbol{r})$ is the 2D feature from CLIP-LSeg, $K$ is the total number of sampled points along the ray and $\boldsymbol{f_i}$ is the feature of point $i$ by interpolating through the feature voxel grid, $w_i$ is the rendering weight.
% \gc{add equations.} 

\noindent\textbf{Concept Grounding through Attention.}  Since our feature voxel grid representation is learnt from CLIP-LSeg, by calculating the dot-product attention $<\boldsymbol{f}, \boldsymbol{v}> $ between per-point 3D feature $\boldsymbol{f}$ and the CLIP concept embeddings $\boldsymbol{v}$, we can get zero-shot view-independent concept grounding and semantic segmentations in the 3D representation, as is presented in Fig. \ref{fig:framework} IV. 
% \gc{add equations.}

\subsection{Neural Reasoning Operators}
Finally, we use the grounded semantic concepts for 3D reasoning from language. We first transform questions into a sequence of operators that can be executed on the 3D representation for reasoning. We adopt a LSTM-based semantic parser   \cite{Yi2018NeuralSymbolicVD} for that. As \cite{Mao2019TheNC, hong20223d}, we further devise a set of operators which can be executed on the 3D representation.  Please refer to \textbf{Appendix} for a full list of operators.

\noindent\textbf{Filter Operators.}  We filter all the grid cells with a certain semantic concept.

\noindent\textbf{Get\_Instance Operators.} We implement this by utilizing DBSCAN \cite{Ester1996ADA}, an unsupervised algorithm which assigns clusters to a set of points. Specifically, given a set of points in the 3D space, it can group together the points that are closely packed together for instance segmentation.

\noindent\textbf{Relation Operators.} We cannot directly execute the relation on the 3D representation as we have not grounded relations. Thus, we represent each relation using a distinct neural module (which is practical as the vocabulary of relations is limited \cite{Landau1993WhatA}). We first concatenate the voxel grid representations of all the referred objects and feed them into the relation network.
% \yd{Do we do something afterwards -- we first concatenate, then what?} 
The relation network consists of three 3D convolutional layers and then three 3D deconvolutional layers. A score is output by the relation network indicating whether the objects have the relationship or not. Since vanilla 3D CNNs are very slow, we use Sparse Convolution \cite{spconv2022} instead. Based on the relations asked in the questions, different relation modules are chosen. 

% \subsection{Learning 3D Compact Representation}
% In recent years, neural field models(\textit{e.g.,} \cite{mildenhall2020nerf}) have gained much popularity since they can reconstruct a volumetric 3D scene representation from a set of images. Recent works \cite{Garbin2021FastNeRFHN, Hedman2021BakingNR, Yu2021PlenOctreesFR, Sun2022DirectVG} have pushed it further by using classic voxel-grids to explicitly store the scene properties (\textit{e.g.}, density, color and feature) for rendering, which allows for real-time rendering. Since concept grounding and relation learning are expected to work on the per-point features in the 3D space \cite{hong20223d} of thousands of scenes, it's more suitable to use voxel-grid-based methods since they store explicit properties in each point which can be directly used for reasoning, and super-fast convergence makes it feasible to train thousands of scenes. Specifically, we use the fine reconstruction process of Direct Voxel Grid Optimization \cite{Sun2022DirectVG} as our backbone for 3D compact representation for its fast speed. 

% A compact voxel-grid representation models the modalities of interest (\textit{e.g.,} density, color or feature) explicitly in its grid cells. To query the properties at any given 3D point, interpolation is used:
% \begin{equation}
% \operatorname{interp}(\boldsymbol{x}, \boldsymbol{V}):\left(\mathbb{R}^3, \mathbb{R}^{C \times N_x \times N_y \times N_z}\right) \rightarrow \mathbb{R}^C
% \end{equation}
% where $\boldsymbol{x}$ is the queried 3D point,  $\boldsymbol{V}$ is the voxel grid, and $C$ is
% the dimension of one of the modalities, and $N_x, N_Y, N_z$ is the number of voxels. We first predict the density of a specified point by interpolating the density grid. This is crucial for the geometric reconstruction of the scene.  
% \begin{equation}
% \sigma=\operatorname{interp}\left(\boldsymbol{x}, \boldsymbol{V}^{(\text {density })}\right)
% \end{equation}
% where $\sigma$ is the volume density at position $\boldsymbol{x}$. For the modeling of color emission, we use an explicit-implicit hybrid representation where  a shallow MLP is placed after the color voxel grid interpolation process:
% \begin{equation}
% \boldsymbol{c}=\operatorname{MLP}^{(\mathrm{rgb})}\left(\operatorname{interp}\left(\boldsymbol{x}, \boldsymbol{V}^{(\mathrm{color})}\right), \boldsymbol{x}, \boldsymbol{d}\right)
% \end{equation}
% where $\boldsymbol{c}$ is the view-dependent color emission at position $\boldsymbol{x}$ viewing from direction $\boldsymbol{d}$.

% To render the color $\hat{C}(\boldsymbol{r})$ of ray $r$, K points are sampled on ray $r$ with densities and colors $\left\{\left(\sigma_i, \boldsymbol{c}_i\right)\right\}_{i=1}^K$. The K results are accumulated by the quadrature rule by Max \cite{Max1995OpticalMF}:
% \begin{align}
% \hat{C}(\mathbf{r})=\sum_{i=1}^K T_i\left(1-\exp \left(-\sigma_i \delta_i\right)\right) \mathbf{c}_i, 
% &\\
% T_i=\exp \left(-\sum_{j=1}^{i-1} \sigma_j \delta_j\right)
% \end{align}
% where $\delta_i=t_{i+1}-t_i$ is the distance between adjacent points along a ray, and $\alpha_i=1-\exp \left(-\sigma_i \delta_i\right)$ is the alpha value for traditional alpha compositing.

% The backbone is trained by minimizing the mean
% square error between the rendered and observed color. 

% \begin{equation}
% \mathcal{L}_{\text {color }}=\|\hat{C}(\boldsymbol{r})-C(\boldsymbol{r})\|_2^2
% \end{equation}

% By extracting the density values of the voxel grid $\boldsymbol{V}^{(\text {density })} \in \mathbb{R}^{1 \times N_x \times N_y \times N_z}$, we can get the compact 3D representation of the scene, as shown in the middle of Figure 2.

% We refer the readers to \cite{Sun2022DirectVG} for more details about the Direct Voxel Grid Optimization.


% \subsection{3D Semantic Concept Grounding}
% In \cite{hong20223d}, a Neural Descriptor Field (NDF) \cite{simeonov2021neural} which gives a feature vector for each 3D coordinate a feature vector is used for concept grounding by aligning the feature vector with the learned concept embeddings. Drawing inspiration from this, we also propose to use a feature voxel-grid  (in addition to density voxel grid and color voxel grid) used for concept grounding. The compact 3D feature representation is composed of one feature voxel-grid representation plus one view-independent shallow MLP: 

% \begin{equation}
% \boldsymbol{f}=\operatorname{MLP}^{(\mathrm{feature})}\left(\operatorname{interp}\left(\boldsymbol{x}, \boldsymbol{V}^{(\mathrm{feature})}\right), \boldsymbol{x}, \boldsymbol{d}\right)
% \end{equation}

% However, the drawback of \cite{hong20223d} is that the embeddings of concepts are learnt from sratch, which is unrealistic in the open-vocabulary reasoning in real-world data. Furthermore, compared to 2D data, 3D assets are limited in diversity and scale, posing challenges for training large vision-language models (VLMs) on 3D-and-language data. Therefore, there's no large-scale 3D VLMs that can be directly used for concept grounding. On the contrary, there's much progress on large-scale 2D VLMs \cite{Radford2021LearningTV, Ramesh2021ZeroShotTG} thanks to the countless image-caption data on the internet. Since we obtain 3D compact representations from 2D multi-view images, it's natural to utilize 2D VLMs to ground semantic concepts on the 3D representations. Based on the CLIP model \cite{Radford2021LearningTV}, LSeg\cite{Li2022LanguagedrivenSS} manages to ground semantic concepts on each 2D pixel (and thus each ray $r$). We denote the feature of ray $r$ as $F(\boldsymbol{r})$.
% Since 3D voxel grids and 2D pixels are aligned via alpha compositing, we add one L1 loss to force the features of 3D voxel grids to align with the 2D LSeg pixels based on the alpha values. Specifically,

% \begin{equation}
% \mathcal{L}_{\text {feature}}=\sum_{i=1}^K T_i\left(1-\exp \left(-\sigma_i \delta_i\right)\right)(\|\boldsymbol{f}-F(\boldsymbol{r})\|)
% \end{equation}

% Assuming we have a set of concepts $P$, the similarities between a concept $\boldsymbol{p} \in P$ and a feature $\boldsymbol{f}$ is calculated as $\langle \boldsymbol{f}, \boldsymbol{p} \rangle$. We define a \textsc{Filter} operator. Specifically, the 3D compact representation for a semantic class $p$ after filtering out that class is:

% \begin{equation}
% \boldsymbol{V}_{\boldsymbol{p}} =  min(\langle\boldsymbol{V}^{(\mathrm{feature})}, \boldsymbol{p}\rangle, \boldsymbol{V}^{(\mathrm{density})}) 
% \end{equation}

% In practice, we only set the values of voxel grids with densities < 0.5 to 0, since we find that those points are irrelevant to the 3D geometry of the scene.

% To get each instance of the objects of the same category, we use DBSCAN \cite{Ester1996ADA} to implement the \textsc{Get\_Instance} operator which assigns clusters to all true values of $\boldsymbol{V}_{\boldsymbol{p}}$. The DBSCAN takes the 3D coordinates as input.





\section{Single-View Reconstruction}
\label{sec:experiment-svr}
% \begin{figure*}[htbp]
%   \centering
%   \includegraphics[width=0.92 \linewidth]{image/fig_pipeline2.pdf}
%   \caption{%The architecture of our baseline method. 
%   \textbf{Single view reconstruction baseline.}
%   Given an image as input, we firstly adopt two CNNs and three MLPs as the encoders, mapping the input to three low-dimension latent codes respectively, i.e., the parameters of shape, pose and texture. Then these three parameters are fed into our \textit{SMCL} model to generate the predicted mesh with texture and pose. Finally, we take the ground truth model from \textit{3DBiCar} as supervision to train our neural network. Our \textit{SMCL} is not only responsible for the shape, pose and texture generation, but also takes charge of the eyes computing.
%   %Overview of our baseline pipeline: Input images are fed to the encoder, and output shape, pose, and texture parameters. With \textit{SMBL}, posed textured models are reconstructed with predicted parameters.} %In our pipeline, we invlove in prevent shape and pose parameter from influencing mutually
%   }
%   \label{reconstruction:pipeline}
% \end{figure*}

\begin{figure}[htbp]
  \centering
  \includegraphics[width=.96\linewidth]{image/pipeline_svr.pdf}
  \caption{\textit{\textbf{BiCarNet}.} Given a masked image, we first map it to the parametric vectors. The vectors are then fed to our \textit{RaBit} to generate a posed body mesh, two eyeballs, and a global UV texture. A part-sensitive reasoner is utilized to perceive local regions and generate the detailed UV texture map. Finally, a vivid 3D cartoon character is obtained with our \textit{BiCarNet}.}
  \label{fig_pipeline_svr}
\end{figure}

Single-view reconstruction (SVR) is one of the most popular tasks of efficient 3D content generation, and recent work has made noticeable progress on human reconstruction based on parametric model of human characters (e.g., SMPL). %However, SVR for 3D biped cartoon characters still needs to be solved and is worth probing.
To verify the practicality of our proposed \emph{3DBiCar} and \emph{RaBit}, we conduct SVR for bipled cartoon characters. A baseline learning-based method is presented, which is termed as \textit{BiCarNet}. 

%We thus present a baseline method named \textit{BiCarNet} for efficient biped cartoon characters reconstruction from a single image.

%Single-view reconstruction (SVR) is a classic task in computer vision and computer graphics, which aims to assist people in creating 3D content efficiently. In this section, we present a baseline method for reconstructing 3D biped cartoon characters from a single image, with the help of \textit{RaBit}.
% Single view reconstruction is a popular and basic application for parametric model and human dataset (e.g. SMPL). With our \textit{RaBit} and \textit{3DBiCar}, single view reconstruction on cartoon can be handled.

\subsection{\textit{\textbf{BiCarNet}}}

%With   As shown in Fig.~\ref{fig_pipeline_svr}, given the input image, we first adopt an \textit{Encoder}, for instance, HMR~\cite{kanazawa2018end}, to map the input to the three parametric vectors of \textit{RaBit} on shape, pose, and texture, respectively. Next the vectors are fed to our \textit{RaBit} model to generate a posed body mesh, two eyeballs, and a coarse texture map. 

%\textbf{\textit{BiCarNet}.} 
Given a single masked image of cartoon characters, our \textit{BiCarNet} aims to reconstruct the corresponding 3D shape, pose, and texture. %As \textit{Rabit} spans a large space of textured models. 
The key problem is to build an encoder to map the input image to the parametric space of \textit{Rabit}. As shown in the upper part of Fig.~\ref{fig_pipeline_svr}, 
%With the fixed parametric model \textit{Rabit}, an Encoder is utilized to map the image to parametric space of shape, pose, and texture, as the upper part of Fig~\ref{fig_pipeline_svr} shows.
We adopt the learning block in HMR~\cite{kanazawa2018end} as our Encoder. Once these parametric vectors are learned, we can feed them to our \textit{RaBit} model to generate a posed body mesh, two eyeballs, and a UV texture (we name it global for convenience to introduce the following method description). 

During our preliminary experiments, we find that the shape reconstruction of characters, \ie the eyes and body, is satisfactory, while the inferred UV tends to lose detailed appearances of some small yet significant areas, such as the nose and ears. We thus adopt a part-sensitive texture reasoner (PSR) to address the above issue, as the lower part of Fig.~\ref{fig_pipeline_svr} shows. Specifically, we design five individual UV-mappings for these significant parts of the nose, ears, horns, eyes, and mouth. Five lightweight encoder-decoder branches are next introduced to learn the appearances of these local regions from the input image, respectively. The learned part UVs could be remapped to the corresponding area on the global UV map to produce a blended texture. However, a direct blending tends to cause seam artifacts. Thus we further adopt a Fuser to address the artifacts as Fig.~\ref{fig_pipeline_svr} illustrates. Please refer to the Supplementary for thorough implementations of \textit{BiCarNet}.


% Move this part to Supplementary
% !!!!!!!!!!!!!!
\iffalse 
\textbf{Implementations.} In our implementation, for shape and pose regression modules, inspired by HMR~\cite{kanazawa2018end}, we utilize two ResNet-50 architectures to embed the $512 \times 512$ input image to a 100-dimensional shape vector, and a 69-dimensional pose vector, respectively. For \textit{texture-module}, we adopt the pSp-encoder~\cite{richardson2021encoding} to learn a 512-dimensional texture vector from the image. As for the part-sensitive texture reasoner, we utilizes the pSp~\cite{richardson2021encoding} as the basic building block to learn multiple $256 \times 256$ local UV textures from the input. pix2pixHD~\cite{wang2018pix2pixHD} is employed as the fusion module, which takes $1024 \times 1024$ coarse texture maps as input and output $1024 \times 1024$ fine texture maps.
\fi
% !!!!!!!!!!!!!!

\subsection{Experiments}

\textbf{Data preparation.} We first split \textit{3DBiCar} into a training set (1,050 image-model pairs) and a testing set (450 pairs). To support a stable training of \textit{BiCarNet}, we next generate a large number of synthetic paired data with the help of \textit{RaBit}, which are highly diversified in shape, posture, and texture. To be specific, a lot of 3D textured models are first sampled from the \textit{RaBit} space, which are then rendered into images from different camera views. 
This finally produce 13,650 pairs for training. Note that, \emph{BiCarNet} takes an image with foreground masked as input. All synthetic images naturally have no background. For real images, all the foreground masks are manually annotated. 
%the synthetic images have no background and we also manually mask the foreground for all real images. 

%These augmentation data are rendered to multiple images with different camera angles. With \textit{3DBiCar} and the augmentation data, we finally obtain 13,650 pairs for training. Notice that we also separate the foreground character from the image background.
 
 % mention this on the Supplementary

% a topology-preserving inference method for shape and pose.

\textbf{Result gallery.} % results of \textit{BiCarNet} are available in Fig.~\ref{fig_wildresult}. 
Fig.~\ref{fig_wildresult} shows representative results generated by \textit{BiCarNet}. As illustrated, our \textit{BiCarNet} can generate vivid 3D cartoon characters loyal to individual cartoon images in shape, pose, and texture. We believe that our work opens the door to producing 3D biped cartoon characters from easy-to-obtain inputs.
\begin{figure}[htbp]
  \centering
  \includegraphics[width=.98 \linewidth]{image/wild_result.pdf}
  \caption{\textbf{Result gallery of \textit{BiCarNet}.} Our \textit{BiCarNet} is capable of generating vivid 3D cartoon characters with only single-view image input.
  }
  \label{fig_wildresult}
\end{figure}



\textbf{Results on Shape Reconstruction.} 
%Our target is reconstructing topological-consistent 3D biped cartoon characters from single images to facilitate future animation. 
As mentioned above, \textit{BiCarNet} utilizes HMR-like blocks and \textit{RaBit} for shape and pose learning. Currently, other reconstruction methods could also be used for topology-consistent geometry inference, such as GCNN-based methods~\cite{lin2021-mesh-graphormer} and UV-based methods~\cite{zeng20203d}. We choose two representative methods for comparison, i.e., Mesh-Graphormer~\cite{lin2021-mesh-graphormer,lin2021end-to-end} and DecoMR~\cite{zeng20203d}. Mesh-Graphormer combines graph convolutions and self-attentions in a transformer for 3D human reconstruction from a single image. DecoMR reconstructs 3D human mesh from single images by regressing a UV-based location map. Tab.~\ref{tab_mesh_result} shows the quantitative comparisons of the above three methods on MPVE, MPJPE, and PA-MPJPE. We also provide qualitative comparisons in Fig.~\ref{fig_mesh_result}. Both quantitative and qualitative results demonstrate that the HMR-like method achieves the highest performance on geometry inference and provides more accurate reconstructions closer to ground truths. As noted, both Mesh-Graphormer and DecoMR outperform HMR for SMPL-based human reconstruction. It is interestingly found that they perform worse in our settings. One possible reason is that our biped cartoon meshes own an extremely larger amount of vertices than SMPL to model more complex geometry. This greatly increases the challenge of vertices regression in Mesh-Graphormer and DecoMR. Thus, in our setting, directly regressing the low-dimension parameters performs better. 

%Due to the high diversity of cartoon characters, it is challenging for neural networks to directly regress a large number of points (38,726 vertices) or high-resolution position maps from a single image. In our HMR-based modules, \textit{RaBbit} models the shape and pose into low-dimensional parametric spaces, significantly reducing the learning difficulty of neural networks.

% To conduct comprehensive experiments of SVR, we compare our \textit{BiCarNet} with current advanced mesh reconstruction methods, including GCNN-based method Mesh Graphormer~\cite{lin2021-mesh-graphormer} and UV-based method DecoMR~\cite{zeng20203d}. Mesh Graphormer combines graph convolutions and self-attentions in a transformer for 3D human reconstruction from a single image. DecoMR reconstructs 3D human mesh from single images by regressing a UV-based location map. 
% We adopt HMR block~\cite{kanazawa2018end} as \textit{Encoder} for mesh geometry learning based on \textit{RaBit} and compare our method with aformentioned two reconstruction methods. 
% We use MPVE, MPJPE and PA-MPJPE as our evaluation metrics for mesh reconstruction as Table~\ref{tab_mesh_result} shows. It can be seen that our method as one setting of \textit{BiCarNet} outperforms other methods in all metrics. We also illustrate the reconstruction results of these methods as Fig.~\ref{fig_mesh_result} shows and our results are visually relatively close to the ground truths. This demonstrates the quantitative and qualitative superiority of \textit{Rabit} and \textit{BiCarNet} for accurate cartoon character reconstruction. % Both quantitative and qualitative results demonstrate that our method achieves the highest performance on geometry inference and provides more accurate reconstructions that are closer to ground truths. The main reason are: 1) It is challenging for neural networks to directly regress large number of points (38,726 vertices) or high-resolution position-maps from a single-image; 2) In our HMR-based modules, \textit{RaBbit} model the shape and pose into low-dimensional parametric spaces, which greatly reduce the learning difficulty of neural networks.
%Our target is reconstructing topological-consistent 3D biped cartoon characters from single images to facilitate future animation. 

%As mentioned above, \textit{BiCarNet} utilizes HMR-like blocks and \textit{RaBit} for shape and pose learning. 

%There are some optional methods that could also be used for topology-consistent geometry inference, such as GCNN-based methods~\cite{lin2021-mesh-graphormer,lin2021end-to-end,kolotouros2019cmr} and UV-based methods~\cite{zeng20203d,alldieck2019tex2shape,feng2018prn}. We choose two representative methods of them for comparison, i.e., Mesh-Graphormer~\cite{lin2021-mesh-graphormer} and DecoMR~\cite{zeng20203d}. 


%\iffalse
\begin{table}[htbp]
    \renewcommand{\arraystretch}{1.3}
    \centering
    \resizebox{\columnwidth}{!}{
        \begin{tabular}{l|c|c|c}
        % \hline 
        \toprule
        Method & MPVE $\downarrow$ & MPJPE $\downarrow$ & PA-MPJPE $\downarrow$ \\
        \midrule
        DecoMR\cite{zeng20203d} & 85.74 & 81.23 & 47.23\\
        Mesh-Graphormer\cite{lin2021-mesh-graphormer} & 63.31 & 47.15 &	34.12\\
        Ours (HMR\cite{kanazawa2018end} + \textit{RaBit}) & \textbf{51.46} & 	\textbf{37.80} & \textbf{25.97}\\
        \bottomrule
        \end{tabular}
    }
    \caption{\textbf{Quantitative results of shape reconstruction.} Our method achieves the best results in terms of MPVE, MPJPE and PA-MPJPE. Note that all metrics are measured in a unit $10^{-3}$m.} %P-MPJPE reports the MPJPE error after alignment with 3d keypoint of ground truth in translation and rotation. Theta reports MAE between predition and ground truth. T-rec reports the point-wise MSE between predition and ground truth model with Tpose. Projection error between manual 2D annotation on image and 2D projections of models after alignment
    \label{tab_mesh_result}
\end{table}
%\fi

% 	input	mPVE	mPJPE	PAmPJPE
% hmr	224	9.28164	7.73912	--
% Mesh Graphormer	224	63.31	47.15	34.12
% decomr	224	19.0676	6.967964	4.84606

\begin{figure}[htp]
  \vspace{-0.2cm}
  \centering
  \includegraphics[width= 0.98\linewidth]{image/mesh_result_2.pdf}
  \caption{\textbf{Qualitative results of shape reconstruction.} From left to right, each row contains (a) the input image, reconstructed meshes of (b) Mesh Graphormer, (c) DecoMR, (d) our method, and (e) the GT mesh.}
  \vspace{-1.2mm}
  \label{fig_mesh_result}
\end{figure}

\textbf{Results on Texture Inference.} 
%Our \textit{BiCarNet} contains a novel part-sensitive texture reasoner to perceive the detailed appearances of local areas, which consists of several individual . We also design a Fuser to avoid artifacts on boundaries. 
To demonstrate the capability of our proposed GAN-based texture generator, we first compare our \emph{RaBit}-based texture inference (i.e., \emph{BiCarNet}) with PCA-based inference. Specifically, for PCA-based method, we utilize the same learning architecture to map the input image into the PCA-based texture space. Furthermore, %with the traditional PCA-based method. Specifically, 
to evaluate the effectiveness of the proposed texture inference modules, we also conduct ablative analysis on \textit{BiCarNet} without Fuser and \textit{BiCarNet} without Part-sensitive Reasoner (PSR). Table~\ref{tab_texutre_result} shows the quantitative results of different texture inference methods on MSE, PSNR and FID and our proposed method achieves the highest scores compared with all other methods. Moreover, Fig.~\ref{fig_texture_result} illustrates the qualitative results of these methods for texture inference. Fig.~\ref{fig_texture_fusion} shows the results without and with Fuser, which demonstrates our fusion module can deal with unnature seam-like artifacts. We can observe that the part-sensitive texture reasoner and the Fuser help to capture the local regions of characters and recover their detailed appearances. %Our \textit{BiCarNet} is capable of recovering reasonable 3D cartoon texture from the input image. 

% \textit{BiCarNet} designs a novel part-sensitive texture reasoner to perceive the appearances of important local areas. To evaluate its effectiveness, we conduct comparison analysis on the PCA-based method, \textit{BiCarNet} without TR (Texture Reasoner), \textit{BiCarNet} without Fuser and \textit{BiCarNet}. Tab.~\ref{tab_texutre_result} shows the quantitative results of different methods on our testing dataset. Our proposed approach achieves the highest accuracy on texture inference compared with the other methods. Fig.~\ref{fig_texture_result} shows the qualitative results of different methods. As illustrated, the results generated by our approach are closer to ground truths. With the help of the part-sensitive texture reasoner, \textit{BiCarNet} is able to capture the local regions of characters and recover their detailed appearances, such as the nose and ears.

% We choose the principal component analysis (PCA) method as the baseline method. In addition, we evaluate and compare the whole-based style-GAN method and the part-based style-GAN method.
% In the baseline method, PCA is first applied to reduce the dimension of UV, and then the neural network is used to fit the reduced vector. In both PCA and style-GAN methods, the dimension of $T$ is fixed to be 512.
% We evaluate four proposed methods on our dataset. Each is defined as:
% MSE, PSNR and FID are used for evaluation. As shown in Tab. \ref{tab_texutre_result}), the styleGAN method outperformed the PCA method in all three metrics. 
% The qualitative result in Fig. \ref{fig_texture_result} illustrates that the UV from the styleGAN method is sharper, less artifact, and closer to ground truth. 

%\iffalse
\begin{table}[htbp]
    \renewcommand{\arraystretch}{1.3}
    \centering
    \resizebox{0.45\textwidth}{!}{
        \begin{tabular}{l|c|c|c}
        % \hline 
        \toprule
        Method & MSE($\times 10^{-1}$)$\downarrow$ & PSNR($\times 10^2$) $\uparrow$ & FID $\downarrow$ \\
        \midrule
        PCA  & 0.2309 & 0.2254 & 0.4642 \\
        % \hline
        % StyleGan  & 
        % \textbf{0.1630} & \textbf{0.2578} & \textbf{0.3553} \\
        % \midrule
        \textit{BiCarNet}  & \textbf{0.1093} & \textbf{0.2458} & \textbf{0.1133} \\
        
        \midrule
        \textit{BiCarNet} w/o Fuser  & 0.1108 & 0.2397 & 0.1407 \\
        \textit{BiCarNet} w/o PSR  & 0.1346 & 0.2361 & 0.4024 \\
        % \hline
        % BP  & 0.1328 & 0.2364 & 0.3103 \\
        % \hline
        % BPH  & 0.1325 & 0.2349 & 0.2716 \\
        % \hline
        % pSp  & 0.1281 & 0.2285 & 0.4869 \\
        % \midrule
        
        % \hline
        % p2p_2  & 0.1373 & 0.2272 & 0.1107 \\
        % \hline
        \bottomrule
        \end{tabular}
        %result is check!
    }
    \caption{\textbf{Quantitative results on texture inference.} PCA denotes linear modeling method for texture and the last two rows indicate the results of \textit{BiCarNet} respectively without two designed module. Our \textit{BiCarNet} outperforms others methods in all metrics.}
    \label{tab_texutre_result}
\end{table}
% \vspace{-0.6cm}
%\fi
% \vspace{-0.3cm}
\begin{figure}[htp]
  \vspace{-0.2cm}
  \centering
  \includegraphics[width=1.0 \linewidth]{image/texture_result2.png}
  \caption{\textbf{Qualitative comparisons on texture inference.} The input image (a) is followed by the textured models from (b) PCA, (c) \textit{BiCarNet} w/o PSR, (d) \textit{BiCarNet} and (e) the ground truth. Note that we use the same shape and focus on the difference of textures.}
  \label{fig_texture_result}
\end{figure}
\begin{figure}[htbp]
%   \vspace{-0.8cm}
  \centering
  \includegraphics[width=.95\linewidth]{image/texture_fusion.pdf}
  \caption{\textbf{Qualitative ablation on Fuser in Texture inference.} Left: \textit{BiCarNet} w/o Fuser. Right: \textit{BiCarNet} with Fuser.}
  \label{fig_texture_fusion}
\end{figure}
% \vspace{-0.6cm}

\section{More Applications}
\label{sec:application}
\subsection{Sketch-based Modeling}
Customizing 3D biped cartoon characters usually requires a heavy workload with commercial tools, even for experienced artists. Sketch-based modeling enables amateur users to get involved in 3D shape customization in a simple and intuitive fashion. We build a sketch-based modeling application with the help of \textit{3DBiCar} and \textit{RaBit}. 

We first sample a series of shape vectors randomly and feed them to \textit{RaBit} to generate 3D cartoon characters with diversified shapes, resulting in 12,000 T-pose models. Then we apply suggestive contour~\cite{han2017deepsketch2face} to render the front-view sketches with different abstraction levels and obtain 108,000 sketch-model pairs. Given a sketch as input, we employ ResNet-50 and three MLPs as the encoder and decoder to map the input sketch to 100-dimensional shape parameters. The generated shape parameters are next fed to \textit{RaBit} to reconstruct the corresponding 3D model. Please refer to the Supplemental materials for more details. Note that users only need to depict a 3D character with T-pose on a 2D canvas while the output characters of our method are animation-ready and can be directly applied to other commercial tools. Fig.~\ref{fig_sketch} displays the sketches created by users with little knowledge of modeling as well as the corresponding models generated by our system. It can be seen that our sketch-based modeling system offers a smart approach for amateur users to create 3D biped cartoon characters with diversified shapes. We will further explore the support of shape reconstruction from  sketches with arbitrary poses, and texture painting in the future.
 
% \textbf{Implementations.} For character shape, as Fig~\ref{fig_sketch} shows, given a sketch as input, we first employ ResNet-50 and three MLPs as encoder and decoder to map the input sketch to 100-dimensional shape parameters in \textit{BiCar}. Finally, we use the 3D character data we constructed as ground truth to train our neural network. For texture generation, taken as input a painting sketch, we first map it into the unfolding UV map. Then, we follow pSp-Encoder~\cite{richardson2021encoding} and encode the input incomplete texture map. Next, apply the \textit{BiCar} to decode and finally output the completed texture map. 


% \textbf{Results and Analysis.} TBD.
\begin{figure}[ht]
  \centering
  \includegraphics[width=.99\linewidth]{image/sketch_result.pdf}
  \caption{\textbf{Result gallery of sketch-based modelling.} The sketches created by amateur users denotes on the left and the generated models on the right.}
  \label{fig_sketch}
\end{figure}
\vspace{-0.5cm}


\subsection{3D Character Animation}
%In previous sections, we have demonstrated the effectiveness of \textit{3DBiCar} and \textit{RaBit} in supporting 3D biped cartoon characters generation from easy-to-obtain inputs, i.e., single-view images and sketches. This section further demonstrates the usability of our generated models for character animation. 

Following the recent advance of human recovering method and parametric model~\cite{wang2022live,SMPL:2015,SMPL-X:2019}, we extract the human from input video frames and adopt a temporal-aware encoder to recover the sequence of human poses~\cite{wang2022live}. Then, a motion retargeting method~\cite{hsieh2005motion} is used to convert the poses on the human skeleton to the motion of our cartoon characters. As shown in Fig.~\ref{fig_anim}, animation-ready characters generated by our \textit{RaBit} can be directly applied to 3D animation. Please refer to the supplementary for animation video.
% qldd nb!
% jin zong nb!
%Animating 3D characters is critical in filming and gaming, yet labor-intensive and time-consuming for animators. With recent advances in deep learning, it is already possible to efficiently animate 3D humans from 2D motion videos with the help of parametric models. Similar to SMPL~\cite{SMPL:2015}, our parametric model \textit{RaBit} can also be animated by controlling the pose parameters. We build a simple application for automatically transferring motion from 2D human videos to \textit{RaBit}. Inspired by~\cite{wang2022live}, we use Yolov3\cite{redmon2018yolov3} to detect the bounding box of a human in every frame, then use the Temporal Encoder provided by TePose~\cite{wang2022live} to extract the sequence of human pose parameters. These pose parameters could be directly applied to animate \textit{RaBit} due to its similar skeleton to SMPL. Thus any cartoon character generated by \textit{RaBit} could be animated from an input video in this way. Fig.~\ref{fig_anim} shows the representative results of transferring motion from an input video to three characters generated by \textit{RaBit}. As seen, \textit{RaBit} could generate high-fidelity posed mesh with the input pose parameters.

\begin{figure}[htbp]
  \centering
  \includegraphics[width=.98 \linewidth]{image/fig_anim.pdf}
  \caption{\textbf{Transferring motion of a human video to animate characters.} (a) denotes the input frames. (b), (c), and (d) indicate three corresponding posed cartoon characters.}
  \label{fig_anim}
\end{figure}
\vspace{-0.3cm}
% video-based approach to 3D motion capture
% We propose a simple algorithm for automatic transfer of facial expressions, from videos to a 3D character, as well as between distinct 3D characters through their rendered animations. 
% Animating 3D character is a great demand in filming and gaming. However, traditional commercial 

% The mature video-based approach to 3D motion capture has recently led to the boom of vtubers, but model-making is still costly. However, after the appearance of \textit{RaBit}, the history of time-and-money-consuming drivable 3D models has passed. We set up a character animation application with video input in this section. 

% This application is the merge of video-based human body reconstruction and motion retargeting. Following TePose \cite{wang2022live}, we train a temporal 3D human pose estimation to provide a real-time pose parameter sequence. Due to the skeleton structure similar to SMPL, we can reproduce the motion in the video with our models with painless artifacts, by directly mapping and transferring pose parameters from correspondance joints in SMPL. We displace several frames of video and output models in Fig. \ref{fig_anim}.

% In the film industry, the motion fidelity of virtual characters is in high demand. Similar to SMPL\cite{SMPL:2015}, our parametric model \textit{RaBit} can also animate the model by controlling the pose parameters. Based on actual human pose datasets, placing our cartoon characters in various poses is painless, even creating animations. Fig. \ref{fig_anim} displays several frames of character poses extracting from human video and migrating to random models in \textit{3DBiCar}.

% Following TePose\cite{wang2022live}, we use Yolov3\cite{redmon2018yolov3} to detect the bounding box of a human in one frame, then use the Temporal Encoder provided by TePose to extract the human pose parameters. Applying these pose parameters to our parametric model \textit{RaBit},  we can place any cartoon character in the same pose as the human in the frame. In this way, any cartoon character created by \textit{RaBit} can be directly applied to video-based motion capture.

% % \subsection{Cartoon Model Editing}
% \iffalse
% \subsection{Part Based SMCL}
% \label{part_SMCL}
% We provide another impletement on mesh of SMCL. For specific part $p$, which contains $N_p$ points, the linear model can be expressed as:
% \begin{equation}
%     M_{Tbody}^p = F_B^p(B^p, S^p) = \bar{M}_{Tbody}^p + \sum_i^{|B^p|} \beta^p_i s^p_i
% \end{equation}
% where $\bar{M}_{Tbody}^p$ is the average mesh of part $p$, $B^p=[\beta^p_0,\beta^p_1,...,\beta^p_{|B^p|}] \in \mathbb R^{|B|}$ is shape coefficients for part $p$, $s^p_i \in \mathbb R^{3N_p}$ is one of the orthonormal principal components of shape offset, $S^p\in \mathbb R^{3N_p \times |B^p|}$ is the shape offset matrix getting from PCA. As a result, we can parameterize the final model as:
% \begin{equation}
%     M_{Tbody} = \bigoplus_p M_{Tbody}^p
% \end{equation}
% where $\oplus$ represents merging operation.
% In detail, we provide a simple way to implement the merging operation. When splitting the model, we preserve common segments for different parts on junctions. Different parts can align together by calculating translation $t$ with common segments:

% \begin{equation}
%     t = argmin_t \sum_{k\in \mathcal I_{p} \cap \mathcal I_{p'} }\|v_{p, k} - v_{p', k} \|_2^2 
% \end{equation}
% where $\mathcal I_{p}$ and $\mathcal I_{p'}$ are index sets of points from different parts. $v_{p, k}$ is the position of the point with k-th index in part $p$. 

% Although aligning the parts is able to reach a proper position, common segments of the different parts are not likely to match perfectly. We use a weighting function to merge the common segments:
% \begin{equation}
% \begin{split}
%     v_k =& u_{p, k}*v_{p, k} + u_{p', k}*v_{p', k} \\
%     =& \frac{d(u_{k}, p')^{\alpha}*v_{p, k}}{d(u_{k}, p')^{\alpha}+d(u_{k}, p)^{\alpha}} + \frac{d(u_{k}, p)^{\alpha}*v_{p', k}}{d(u_{k}, p')^{\alpha}+d(u_{k}, p)^{\alpha}}
% \end{split}
% \end{equation}
% where vertex with index $k$ is the comment segment of $p$ and $p'$. $v_{p, k}$ and $v_{p', k}$ are the vertices with index $k$ in part $p$ and $p'$ respectively. $d(u_k, p)$ is the shortest distance from $v_k$ to part $p$(exclude the common segment) on mesh. $\alpha$ is a hyperparameter to control smoothness, which is set as 1.3 in our impletement.
% \fi
% % % \begin{figure}[htbp]
  \centering
  \includegraphics[width=.96\linewidth]{supp_image/fig_interpolation2.pdf}
  \caption{An illustration of interpolated shapes. Models from the top row and left column are from \textit{3DBiCar}. Other models with blue backgrounds are obtained by interpolating the leftmost and uppermost models with the help of \textit{RaBit}.}
  \label{fig_interpolation}
\end{figure}
% \begin{figure}
     \centering
     \begin{subfigure}[b]{0.1\textwidth}
         \centering
         \includegraphics[width=\textwidth]{supp_image/part-aware-sample/fig_assemble1.pdf}
         \caption{Arms}
     \end{subfigure}
     \hfill\hspace{-4mm}
     \begin{subfigure}[b]{0.1\textwidth}
         \centering
         \includegraphics[width=\textwidth]{supp_image/part-aware-sample/fig_assemble2.pdf}
         \caption{Legs}
     \end{subfigure}
     \hfill\hspace{-4mm}
     \begin{subfigure}[b]{0.1\textwidth}
         \centering
         \includegraphics[width=\textwidth]{supp_image/part-aware-sample/fig_assemble3.pdf}
         \caption{Head}
     \end{subfigure}
     \hfill\hspace{-4mm}
     \begin{subfigure}[b]{0.1\textwidth}
         \centering
         \includegraphics[width=\textwidth]{supp_image/part-aware-sample/fig_assemble4.pdf}
         \caption{Body}
     \end{subfigure}
     \hfill\hspace{-4mm}
     \begin{subfigure}[b]{0.1\textwidth}
         \centering
         \includegraphics[width=\textwidth]{supp_image/part-aware-sample/fig_assemble5.pdf}
         \caption{Concat}
     \end{subfigure}
     \caption{\textbf{Part-aware Sample}. By concating embeddings representing differet parts from different model, we can sample a new model(e) which assemble parts (a-d) from different models. In column (e), yellow texture means the collision between two different parts.}
\end{figure}
\begin{figure}[htp]
  \vspace{-0.2cm}
  \centering
  \includegraphics[width= 0.98\linewidth]{image/mesh_result_2.pdf}
  \caption{\textbf{Qualitative results of shape reconstruction.} From left to right, each row contains (a) the input image, reconstructed meshes of (b) Mesh Graphormer, (c) DecoMR, (d) our method, and (e) the GT mesh.}
  \vspace{-1.2mm}
  \label{fig_mesh_result}
\end{figure}
\begin{figure}[htp]
  \vspace{-0.2cm}
  \centering
  \includegraphics[width=1.0 \linewidth]{image/texture_result2.png}
  \caption{\textbf{Qualitative comparisons on texture inference.} The input image (a) is followed by the textured models from (b) PCA, (c) \textit{BiCarNet} w/o PSR, (d) \textit{BiCarNet} and (e) the ground truth. Note that we use the same shape and focus on the difference of textures.}
  \label{fig_texture_result}
\end{figure}

\section{Experiments}
\label{sec:experiment}
In this section, we conduct comprehensive experiments to demonstrate the capacity of \textit{SMCL} and \textit{3DBiCar}. In \ref{sec:experiment-sampleing}, we present the parameterization of \textit{SMCL} with different sampling methods. In \ref{sec:experiment-svr}, we elabrate the implementation detail and further demonstrate the qualitative and quantitative result of our single view reconstruction. 
%\subsection{Shape Morphing}




\begin{figure}[htbp]
  \centering
  \includegraphics[width=.98 \linewidth]{image/wild_result.pdf}
  \caption{\textbf{Result gallery of \textit{BiCarNet}.} Our \textit{BiCarNet} is capable of generating vivid 3D cartoon characters with only single-view image input.
  }
  \label{fig_wildresult}
\end{figure}


\subsection{Single View Reconstruction}
\label{sec:experiment-svr}


%In this section, we will show training details and quantitative and qualitative results of our method.

To obtain reconstruction networks for single view reconstruction. We split \textit{3DBiCar} into a training set (including 1532 models) and a testing set (including 114 models). In order to reduce domain shift during the evaluation, we keep the distribution of test samples as consistent with the distribution of 3DBiCar. % as possible.

%Even though we pay much effort into \textit{3DBiCar}, compared with other datasets, the train set is so limited that the neural network may not converge well. 

We next render synthetic images for each model and apply image augmentation such as color jitter for data augmentation. Moreover, we also make use of these sampling methods as Sec.~\ref{sec:experiment-sampleing} to generate synthetic model-image pairs. To minimize the impact of data imbalances, we resample the training set to maintain the balance between raw and synthetic data.

\textbf{Mesh Reconstruction.} In part-based \textit{SMCL}, we perform PCA on symmetric structures simultaneously (e.g., left and right hands) to reduce the size of parts from 6 to 4. To make fair comparision between whole-based and part-based linear models, we fix the dimension of $\Theta$ to 72 and $B$ to 200 (74 dimensions for head part and 42 for others).
%50 dimensions for every part of the part-based model). 
%\iffalse
\begin{table}[htbp]
    \renewcommand{\arraystretch}{1.3}
    \centering
    \resizebox{\columnwidth}{!}{
        \begin{tabular}{l|c|c|c}
        % \hline 
        \toprule
        Method & MPVE $\downarrow$ & MPJPE $\downarrow$ & PA-MPJPE $\downarrow$ \\
        \midrule
        DecoMR\cite{zeng20203d} & 85.74 & 81.23 & 47.23\\
        Mesh-Graphormer\cite{lin2021-mesh-graphormer} & 63.31 & 47.15 &	34.12\\
        Ours (HMR\cite{kanazawa2018end} + \textit{RaBit}) & \textbf{51.46} & 	\textbf{37.80} & \textbf{25.97}\\
        \bottomrule
        \end{tabular}
    }
    \caption{\textbf{Quantitative results of shape reconstruction.} Our method achieves the best results in terms of MPVE, MPJPE and PA-MPJPE. Note that all metrics are measured in a unit $10^{-3}$m.} %P-MPJPE reports the MPJPE error after alignment with 3d keypoint of ground truth in translation and rotation. Theta reports MAE between predition and ground truth. T-rec reports the point-wise MSE between predition and ground truth model with Tpose. Projection error between manual 2D annotation on image and 2D projections of models after alignment
    \label{tab_mesh_result}
\end{table}
%\fi

% 	input	mPVE	mPJPE	PAmPJPE
% hmr	224	9.28164	7.73912	--
% Mesh Graphormer	224	63.31	47.15	34.12
% decomr	224	19.0676	6.967964	4.84606


MAE errors of $\Theta$ and P-MPJPE are used in the pose parameter for evaluation. We also report the Projection error (P-Proj) between manual 2D annotation and 2D projections of models after alignment to verify the pose consistency between our results and given pictures. To evaluate shape reconstruction, we use reconstruction loss under T-pose (T-rec) as a model shape metric instead of using reconstruction directly because of the strong coupling between the pose-related metric and the reconstruction metric. Our quantitative result in Tab. \ref{tab_mesh_result} demonstrate that our part-based model outperformed the whole-based model in all metrics. We also visualize the results as shown in Fig.~\ref{fig_mesh_result}. The results of the whole-based model are acceptable, while the part-based models are closer to ground truth models.

%\iffalse
\begin{table}[htbp]
    \renewcommand{\arraystretch}{1.3}
    \centering
    \resizebox{0.45\textwidth}{!}{
        \begin{tabular}{l|c|c|c}
        % \hline 
        \toprule
        Method & MSE($\times 10^{-1}$)$\downarrow$ & PSNR($\times 10^2$) $\uparrow$ & FID $\downarrow$ \\
        \midrule
        PCA  & 0.2309 & 0.2254 & 0.4642 \\
        % \hline
        % StyleGan  & 
        % \textbf{0.1630} & \textbf{0.2578} & \textbf{0.3553} \\
        % \midrule
        \textit{BiCarNet}  & \textbf{0.1093} & \textbf{0.2458} & \textbf{0.1133} \\
        
        \midrule
        \textit{BiCarNet} w/o Fuser  & 0.1108 & 0.2397 & 0.1407 \\
        \textit{BiCarNet} w/o PSR  & 0.1346 & 0.2361 & 0.4024 \\
        % \hline
        % BP  & 0.1328 & 0.2364 & 0.3103 \\
        % \hline
        % BPH  & 0.1325 & 0.2349 & 0.2716 \\
        % \hline
        % pSp  & 0.1281 & 0.2285 & 0.4869 \\
        % \midrule
        
        % \hline
        % p2p_2  & 0.1373 & 0.2272 & 0.1107 \\
        % \hline
        \bottomrule
        \end{tabular}
        %result is check!
    }
    \caption{\textbf{Quantitative results on texture inference.} PCA denotes linear modeling method for texture and the last two rows indicate the results of \textit{BiCarNet} respectively without two designed module. Our \textit{BiCarNet} outperforms others methods in all metrics.}
    \label{tab_texutre_result}
\end{table}
% \vspace{-0.6cm}
%\fi

\textbf{Texture Reconstruction.} We choose the principal component analysis (PCA) method for comparison. In this method, PCA is first applied to reduce the dimension of UV, and then the neural network is used to fit the reduced vector. In PCA method and styleGAN method, the dimension of $T$ is fixed to be 512.

MSE, PSNR and FID are used for evaluation. As shown in Tab. \ref{tab_texutre_result}), the styleGAN method outperformed the PCA method in all three metrics. The qualitative result in Fig.~\ref{fig_texture_result} illustrates that the UV from the styleGAN method is sharper, less artifact, and closer to ground truth. 

\textbf{Result of wild image} To show both mesh reconstruction and texture reconstruction, we collect some images and generate textured posed models after extracting main character from images(Fig.~\ref{fig_wildresult}).

This work presented a simulation approach that centers around finding iteratively an approximation of the evolution of the algebraic variables in the power system \glspl{DAE}. The approximation of the dynamic state evolutions by NNs, instead of classical explicit numerical integration schemes, allows larger time-steps to be realized while being fast to execute. This work aimed at providing a proof of concept, it is foreseeable that future work on this method shares many typical questions with established \gls{DAE} solvers, hence, by applying various existing techniques the computational performance and scalability of the approach should improve significantly.

{\small
\bibliographystyle{ieee_fullname}
\bibliography{11_references}
}

\ifarxiv \clearpage \appendix
\label{sec:appendix}

\section{Sensitivity Analysis on Hyper-parameters}
\par
 The hyper-parameters in our method including $\gamma$ in Eq.~8, $\epsilon$ in Eq.~10 and $\lambda$ in Eq.~12. In Figure~\ref{supp_fig1}, a sensitivity study of them on 3-bit ResNet-18 is performed. For $\gamma$, we choose several small values from 0 to 2 since the optimization in data generation process of ImageNet is not as easy as other small-scale datasets. We keep $\epsilon$ on a small magnitude so that the perturbation is not too large, and keep $\lambda$ on a large magnitude so that the magnitude of two loss terms are consistent. 
% The results show that HSAT is not very sensitive to these hyper-parameters, although there is some effect. 
The results show that the performance of HAST is somewhat sensitive to these hyper-parameters, but most of these results ($50.12\% \sim 51.15\%$) are comparable with that of the model fine-tuned on real data ($51.95\%$). Note that the worse results in Figure~\ref{supp_fig1} outperforms the quantized model obtained by the state-of-the-art ZSQ method ($45.51\%$) in a large margin.
% Similar experiments can be conducted to find out the optimal value of these hyper-parameters on other datasets, as listed in Sec.~\ref{Implementation Details}.
We conduct similar experiments to find out the optimal value of these hyper-parameters on other datasets.



\section{Sample Difficulty Promotion Details}
\label{SDP Details}
\par
\textbf{Perturbation Direction Calculation.} In the main paper, we calculate the perturbation $\delta$ by maximizing the sample difficulty, which is closely related to the loss. However, there are two loss terms, i.e., the Kullback-Leibler (KL) loss and the feature alignment (FA) loss in the fine-tuning process. Thus we conduct a further experiment to select the optimal loss for perturbation direction calculation. The experimental results are shown in Table~\ref{supp_table1}. We observe that the choice of loss for calculating the perturbation direction has a certain impact on the performance. Though not optimal for all settings, we choose KL+FA to calculate perturbation direction since it shows the best in most settings.


\begin{table}[h]   
\begin{center}   
\resizebox{\columnwidth}{!}{
\begin{tabular}{c c c c c c}   
\hline 
Dataset & Model & Bit-width & KL & FA & KL+FA \\ \hline
\multirow{2}*{Cifar-10}& \multirow{2}*{ResNet-20} & W4A4 & 92.43 & 92.29 & 92.36 \\ 
\multirow{2}*{}& \multirow{2}*{} & W3A3 & 88.29 & 87.68 & 88.34 \\ \hline
\multirow{2}*{Cifar-100}& \multirow{2}*{ResNet-20} & W4A4 & 66.69 & 66.50 & 66.68 \\ 
\multirow{2}*{}& \multirow{2}*{} & W3A3 & 55.61 & 55.13 & 55.67 \\ \hline
\multirow{2}*{ImageNet}& \multirow{2}*{ResNet-18} & W4A4 & 66.90 & 66.69 & 66.91 \\ 
\multirow{2}*{}& \multirow{2}*{} & W3A3 & 51.06 & 50.87 & 51.15  \\ \hline
\end{tabular} 
}
\caption{Performance of our HAST when calculating perturbation direction with diferent losses. We maximize the gradient of KL, FA and KL+FA respectively to calculate perturbation direction.}
\label{supp_table1} 
\end{center}   
\end{table}


\par
\textbf{loss weights.} We apply sample difficulty promotion to the synthetic samples obtained by hard sample synthesis for more difficult samples. Then both of them are used to fine-tune the quantized model with the same loss weights. Further experiments on the loss weights of the original synthetic samples and the promotional samples are conducted. Experimental results are shown in table~\ref{supp_table2}. The loss weight of the original synthetic samples is denoted as $a$, and that of the promotional samples is denoted as $b$. We perform 3-bit quantization on CIFAR-10 and ImageNet. For CIFAR-10, we achieve the best accuracy of 88.34\% by setting both the weights to 1. When it comes to ImageNet, better performance than that reported in the main paper is obtained by increasing the weight of promotional samples.

\begin{table}[ht]   
\begin{center}   
\resizebox{\columnwidth}{!}{
\begin{tabular}{c c c c c c}   
\hline 
\multirow{2}{*}{$a,b$} & ResNet-20 & ResNet-18 & \multirow{2}{*}{$a,b$} & ResNet-20 & ResNet-18  \\ \cline{2-3} \cline{5-6}
 & Cifar-10 & ImageNet &  & Cifar-10 & ImageNet  \\ \hline
 1,0 & 86.17 & 47.94 & 0,1 & 88.19 & 48.55 \\
 3,1 & 85.92 & 50.52 & 1,4 & 86.69 & 52.14 \\
 2,1 & 87.53 & 50.97 & 1,3 & 86.94 & 53.12 \\
 1,1 & 88.34 & 51.15 & 1,2 & 87.73 & 52.69 \\ \hline
\end{tabular} 
}
\caption{Ablation results of loss weights in W3A3 setting. The loss weights of original synthetic samples and promotional samples are denoted as $a,b$ respectively.}
\label{supp_table2} 
\end{center}   
\end{table}

\section{Feature Alignment Analysis}
\par 
\textbf{Direct feature alignment vs. relaxed feature alignment.} Direct feature alignment~\cite{featurealignment} is an easy and effective way to transfer feature representations by directly using mean square error to align the feature. However, we use attention vector~\cite{AttentionTransfer} to relax the feature alignment constraint due to the limited capacity of quantized model. In this section, we provide the performance comparison of our HAST between using direct feature alignment (DFA) and using relaxed feature alignment (RFA). Table~\ref{supp_table3} shows the experimental results. The relaxed feature alignment obtains better performance in any settings over direct feature alignment. Significant improvements can be observed from 3-bit quantization. This shows that it is harmful for low-precision quantized model to learn the feature representations of full-precision model directly.

\begin{table}[ht]   
\begin{center}   
\resizebox{\columnwidth}{!}{
\begin{tabular}{c c c c c}   
\hline 
Dataset & Model & Bit-width & HAST(DFA) & HAST(RFA) \\ \hline
\multirow{2}*{Cifar-10}& \multirow{2}*{ResNet-20} & W4A4 & 91.99 & 92.36 \\ 
\multirow{2}*{}& \multirow{2}*{} & W3A3 & 83.92 & 88.34 \\ \hline
\multirow{2}*{Cifar-100}& \multirow{2}*{ResNet-20} & W4A4 & 66.53 & 66.68 \\ 
\multirow{2}*{}& \multirow{2}*{} & W3A3 & 51.50 & 55.67 \\ \hline
\multirow{2}*{ImageNet}& \multirow{2}*{ResNet-18} & W4A4 & 66.49 & 66.91 \\ 
\multirow{2}*{}& \multirow{2}*{} & W3A3 & 45.52 & 51.15 \\ \hline
\end{tabular} 
}
\caption{Performance of our HAST with direct feature alignment and relaxed feature alignment.}
\label{supp_table3} 
\end{center}   
\end{table}



\begin{figure*}[ht]
\centering
\begin{minipage}[b]{\linewidth}
        \centering
        \includegraphics[width=2.1in]{parameter_gamma.pdf}
        \includegraphics[width=2.1in]{parameter_epsilon.pdf}
        \includegraphics[width=2.1in]{parameter_lambda.pdf}
\end{minipage}
\vspace{-8mm}
\caption{Sensitivity analysis on hyper-parameters. We report the top-1 accuracy of 3-bit ResNet-18 on ImageNet.}
\label{supp_fig1}
\end{figure*}


\begin{figure*}[ht]
\centering
    \subfloat[Gradient cosine similarity.]{
        \label{supp_fig2.a}
        \includegraphics[width=1.6in]{gradient_cosine_similiarity.pdf}
    }
    \quad
    \subfloat[Distribution of eigenvalues.]{
        \label{supp_fig2.b}
        \includegraphics[width=1.6in]{density_eigenvalue_of_CE.pdf}
        \includegraphics[width=1.6in]{density_eigenvalue_of_KL.pdf}
        \includegraphics[width=1.6in]{density_eigenvalue_of_FA.pdf}
    }
\caption{Further experiments on feature alignment. (a)Gradient cosine similarity of two terms in loss function. (b)Distribution of the eigenvalues for different loss.}
\label{supp_fig2}
\end{figure*}

\par 
\textbf{Cooperation with KL.} Gradient cosine similarity was used in~\cite{AIT} to measure the cooperation ability of multiple loss terms. The authors found that the cross-entropy (CE) loss does not work well with the Kullback-Leibler (KL) loss in network fine-tuning process. We apply this metric in our work. Specifically, we fine-tune the 3-bit ResNet-20 using baseline (CE+KL)~\cite{IntraQ} and our HAST (FA+KL) respectively and measure the cosine similarity of the gradient of two distinct loss terms. As shown in Figure~\ref{supp_fig2.a}, the cosine distance between CE and KL takes negative values throughout the fine-tuning, while that of FA+KL is positive. This implies that the combinations of FA and KL cooperate well, and using them together could enhance each other, which is opposite to the combinations of CE and KL.

\par \textbf{Generalizability.}  Hessian matrix was used in~\cite{AIT} to measure the local curvature of the loss surface and compare the generalizability of the two distinct loss terms. Since Hessian matrix itself is enormous in size and computations involving its entirety is considered almost infeasible, analyzing the eigenvalues of the matrix is often the most preferred way to study its characteristics. Figure~\ref{supp_fig2.b} plots the distribution of the eigenvalues of the Hessian matrix, approximated by PyHessian~\cite{PyHessian}. We separate Hessian calculation for each loss of CE, KL and FA. A huge difference in the local curvature of the loss terms can be observed. While CE has longer tail for high eigenvalues, KL and FA has more concentration to lower eigenvalues, which means the local curvature of loss surface of KL and FA is smaller than that of CE, leading to better genrealizability according to the finding that  smaller local curvature improves generalization~\cite{AIT}.

\section{Results with smaller number of samples}
Table~\ref{supp_table4} shows the ablation on amount of the synthetic samples. The performance drops as the number of samples decreases. However, HAST with only 256 samples still performs better than previous methods, such as IntraQ with 45.51\% using 5120 samples.


\begin{table}[ht]
\centering
    \resizebox{0.45\textwidth}{!}{
    \large
    \begin{tabular}{*{10}{c}}
        \toprule
        Amount & IntraQ(5120) & 256 &  1280 & 2560 & 5120 \\
        \midrule
        ACC(\%) & 45.51 & 49.17 & 49.95 & 50.23 & 51.15 \\
        \bottomrule
    \end{tabular}
    }
\caption{Results with smaller number of samples.}
\label{supp_table4}
\end{table} \fi

\end{document}
