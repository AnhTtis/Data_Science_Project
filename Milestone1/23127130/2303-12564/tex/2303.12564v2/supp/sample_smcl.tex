\label{sec:appendix}
\section{Parameterization of SMCL}
\label{sec:experiment-sampleing}
% Sample on mesh: Interpolated Sample,Sample one specific axis, part aware sample % Sample on pose: pose sample 
% This section shows several sampling methods to show the capacity and \textit{SMCL}.

To demonstrate the generalization of \textit{SMCL}, we sample on parameter $(B,e,\Theta,T)$ as equation shows in section \ref{sec:algorithm}. 

\begin{equation}
\begin{split}
    M =& F(B, e, \Theta, T)\\
      =& F_T(F_P(F_S(B,e),\Theta), T)
\label{eq:1}
\end{split}
\end{equation}

We first sample shape parameter $B$ of \textit{SMCL} in Fig. \ref{fig_interpolation}. Interpolations of two models in the dataset building reasonable models shows the stability of our parameterization on mesh. 

Next, we adopt part-aware sampling in Fig. \ref{fig_assemble} through concating parameters of differents parts. It can be seen that \textit{SMCL} can generate diverse models with fully decoupled parts.

%Part-aware sampling in Fig. \ref{fig_assemble} aims t to generate the hybrids derived from various models. This sampling method suggests that \textit{SMCL} can generate diverse models with fully decoupled parts.

We further adopt poses from Human3.6M\cite{Ionescu2014Human36M} to our pose parameter  space $\Theta$ as Fig~\ref{fig_poseaug}. Notice that we can generate an animation video from a pose sequence.

%As for pose sampling, a similar skeleton structure allows us to transfer the poses from other human body datasets to our model (Fig. \ref{fig_poseaug}). Moreover, with a sequence pose and a specific model, we can output an animation video for corresponding model.

In the end, we interpolate the parameter $T$ in textures of two samples. The UV maps and models with UV map are shown in Fig. \ref{fig_uv_interpolation}, which demonstrates the capability of our texture parameterization.
% \section{}
% \begin{figure}[htbp]
  \centering
  \includegraphics[width=.96\linewidth]{supp_image/fig_interpolation2.pdf}
  \caption{An illustration of interpolated shapes. Models from the top row and left column are from \textit{3DBiCar}. Other models with blue backgrounds are obtained by interpolating the leftmost and uppermost models with the help of \textit{RaBit}.}
  \label{fig_interpolation}
\end{figure}
\begin{figure}
     \centering
     \begin{subfigure}[b]{0.1\textwidth}
         \centering
         \includegraphics[width=\textwidth]{supp_image/part-aware-sample/fig_assemble1.pdf}
         \caption{Arms}
     \end{subfigure}
     \hfill\hspace{-4mm}
     \begin{subfigure}[b]{0.1\textwidth}
         \centering
         \includegraphics[width=\textwidth]{supp_image/part-aware-sample/fig_assemble2.pdf}
         \caption{Legs}
     \end{subfigure}
     \hfill\hspace{-4mm}
     \begin{subfigure}[b]{0.1\textwidth}
         \centering
         \includegraphics[width=\textwidth]{supp_image/part-aware-sample/fig_assemble3.pdf}
         \caption{Head}
     \end{subfigure}
     \hfill\hspace{-4mm}
     \begin{subfigure}[b]{0.1\textwidth}
         \centering
         \includegraphics[width=\textwidth]{supp_image/part-aware-sample/fig_assemble4.pdf}
         \caption{Body}
     \end{subfigure}
     \hfill\hspace{-4mm}
     \begin{subfigure}[b]{0.1\textwidth}
         \centering
         \includegraphics[width=\textwidth]{supp_image/part-aware-sample/fig_assemble5.pdf}
         \caption{Concat}
     \end{subfigure}
     \caption{\textbf{Part-aware Sample}. By concating embeddings representing differet parts from different model, we can sample a new model(e) which assemble parts (a-d) from different models. In column (e), yellow texture means the collision between two different parts.}
\end{figure}
% 
% \begin{figure}[htbp]
%   \centering
%   \includegraphics[width=.98 \linewidth]{placeholder/fig_poseaug.pdf} %pose aug
%   \caption{Pose Sampling: Tpose indicates T-pose model. Pose\* indicate pose of coresponding model in \textit{3DBiCar}. Pose1 and Pose2 is two poses transfer from Human3.6M \cite{Ionescu2014Human36M}}
%   \label{fig_poseaug}
% \end{figure}


\begin{figure}[htbp]
  \centering
  \includegraphics[width=.90 \linewidth]{supp_image/fig_poseaug.pdf} %pose aug
%   \caption{Diverse poses from  Human3.6M \cite{Ionescu2014Human36M} are transfered to a model in \textit{3DBiCar}.}
  \caption{An illustration of diverse poses transferred from pose datasets.}
  \label{fig_poseaug}
\end{figure}

\begin{figure}[htbp]
  \centering
  \includegraphics[width=.98 \linewidth]{supp_image/texture-inter.pdf} %assemble.png
  \caption{An illustration of synthesized texture maps. For each row, the leftmost and the rightmost textures are from \textit{3DBiCar}, while the other three textures are interpolated results generated by \textit{RaBit} under different weights.}
  \label{fig_uv_interpolation}
\end{figure}