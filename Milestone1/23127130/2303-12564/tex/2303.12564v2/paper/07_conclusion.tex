\section{Conclusion}
\label{sec:conclusion}

In this work, we introduce \textit{3DBiCar}, the first large-scale 3D biped cartoon character dataset. It contains 1,500 textured and skinned models with a consistent mesh topology. Based on \textit{3DBiCar}, we propose the first 3D full-body cartoon parametric model \textit{RaBit} for biped character modeling. Furthermore, we build a baseline method \textit{BiCarNet} for reconstructing 3D textured models from a single image with cartoon characters. Experimental results demonstrate the capability of \textit{3DBiCar} and \textit{RaBit} as well as the effectiveness of \textit{BiCarNet}. Last but not least, two further applications, i.e., sketch-based modeling and 3D character animation, demonstrate the usability and practicality of our dataset and parametric model. %We hope that our work can open a door for further researches in the area of cartoon character digitalization.
We hope that our work will contribute to the development of 3D biped cartoon character modeling and inspire future works in this area.

\noindent\textbf{Acknowledgements.} The work was supported in part by NSFC with Grant No.~62293482, the Basic Research Project No.~HZQB-KCZYZ-2021067 of Hetao Shenzhen-HK S\&T Cooperation Zone. It was also partially supported by Shenzhen General Project with No.~JCYJ20220530143604010, the National Key R\&D Program of China with grant No.~2018YFB1800800, by Shenzhen Outstanding Talents Training Fund 202002, by Guangdong Research Projects No.~2017ZT07X152 and No.~2019CX01X104, by the Guangdong Provincial Key Laboratory of Future Networks of Intelligence (Grant No.~2022B1212010001), and by Shenzhen Key Laboratory of Big Data and Artificial Intelligence (Grant No.~ZDSYS201707251409055).
% To demonstrate the practicality of \textit{3DBiCar} and \textit{RaBit}, 

% we conduct various applications, including single-view reconstruction, sketch-based modeling, and 3D cartoon animation. Furthermore, a novel 

% Furthermore, we firstly introduce a biped cartoon parametric model \textit{RaBit} to build an efficient bipedal cartoon character representation. To demonstrate the strengths of \textit{RaBit}, we implement various applications including single-view cartoon character reconstruction, 3D cartoon model editing, sketch-based character modeling and 3D cartoon animation. 

% Moreover, experiments show the advantages of our proposed non-linear texture parameterization. 





% In this paper, we presented the first large-scale 3D bipedal cartoon character dataset \textit{3DBiCar}. 

%We find it demanding to reconstruct fine texture compare to mesh reconstruction for 3D cartoon characters because building texture from a 2D image is subjective to artists. Exploring a more reasonable texture could be the future interest for 3D cartoon datasets. 


% Even through in most cases, mesh reconstruction is relatively satisfying, while texture reconstruction affects people's subjective impression of the model to a large extent. Thus, how to generate a more reasonable texture is an issue for future research to explore. 

% For this task, there is still a long way to go. In human reconstruction, datasets and algorithms of upstream tasks about humans are relatively mature. However, in character bipedal reconstruction, a lack of data and a suitable algorithm is a big problem to add additional constrain or prior for more accurate results. We hope this area can boom in the future with the effort of every researcher.

% emotion???
