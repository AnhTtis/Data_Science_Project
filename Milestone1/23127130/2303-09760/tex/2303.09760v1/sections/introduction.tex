\section{Introduction}
\begin{figure}[ht]
    \centering
    \includegraphics[width=.95\linewidth]{img/figures_general/topology_example.pdf} 
    \caption{Minimum Compliance Topology Optimization problem. 
    Given a set of constraints (loads, boundary conditions, volume fraction), find the optimal topology minimizing functional performance (structural compliance).}
    \label{fig:intro_to}
\end{figure}

Topology optimization~(TO \citep{bendsoe1988generating, sigmund2013topology}) is an essential engineering tool that enables finding the optimal material distribution for meeting performance objectives while satisfying constraints. Constraints can take various forms, such as loads, boundary conditions, and volume fraction. Topology optimization can lead to significant improvements in the efficiency, safety, reliability, and durability of structures across various fields, including aerospace, automotive, civil, and mechanical engineering. The demand for improvements in topology optimization techniques is ever-growing to enhance design engineering for a new age of engineering discovery.
Topology optimization uses Finite Element Analysis (FEA), with gradient-based~\cite{bendsoe1988generating} and gradient-free methods~\cite{ahmed2013constructive} being two primary approaches. The Solid Isotropic Material with Penalization (SIMP) method~\cite{bendsoe1989optimal, rozvany1992generalized} is one of the most widely used gradient-based optimization algorithms. SIMP utilizes iterative optimization methods over continuous, density-based representation to determine the optimal material distribution. This class of methods iteratively adjusts the material density in a given domain to check whether the design meets the desired constraints. 




Iterative optimization-based topology optimization (TO) methods have great potential benefits but face challenges in practical applications, particularly for large-scale problems due to their computational complexity. Iterative algorithms, like SIMP, are computationally expensive and prone to getting trapped in local optima, limiting the search for a global optimum. To address these challenges, there has been a recent surge of interest in learning-based methods for topology optimization, which use machine learning (ML) algorithms, such as deep generative models, to speed up the optimization process and generate more diverse structural topologies. By leveraging large datasets of existing designs, ML models can quickly identify patterns and generate new designs that meet specified constraints. Compared to traditional optimization methods, ML methods can handle high-dimensional data, explore a broader range of design solutions, and provide more diverse results that may not have been considered previously.
Companies like SolidWorks and start-ups like nTopology are integrating deep generative topology optimization in their products, adding flexibility and creativity for the designers and speeding up the candidate generation phase. 




\paragraph{Deep Generative Models for TO.}
Deep Generative models (DGMs~\cite{bond2021deep}), have shown an impressive ability to model high-dimensional and complex data modalities, such as images or text, with high fidelity and diversity. This capability opens the door for significant improvements in creativity and productivity in various fields in the coming years.
Recent advancements in large vision models~\cite{rombach2021high}) and language models~\cite{brown2020language}) have greatly increased our capacity to process unstructured data, leading to the development of multimodal generation techniques~\cite{ramesh2021zero}. These models hold great promise for enhancing engineering design~\cite{regenwetter2022deep, song2023multi}.
Building on these successes, DGMs have also been applied in the engineering field to problems with constraints, with a focus on improving the design process. However, these applications have primarily focused on metrics like reconstruction quality and have not fully addressed the fulfillment of engineering requirements.

DGMs for topology optimization pose challenges in terms of both performance and manufacturability, which TopologyGAN~\citep{nie2021topologygan}, addresses by conditioning on physics-based information. By introducing physical fields such as the Von Mises stress, strain energy density, and displacement fields, TopologyGAN improves the adherence of the generated samples to the underlying engineering problem. However, computing the physical fields requires solving a computationally expensive Finite Element Analysis (FEA) routine for each configuration of load and boundary condition considered, even though sampling new topologies with the base model (a conditional GAN) is fast.
This approach has been successful but still optimizes for a pixel-wise, reconstruction-based loss that does not account for engineering requirements such as high performance and feasibility.
Despite conditioning on physical fields, many designs generated by these models still suffer from floating materials that impede manufacturability, as well as limited diversity and generalizability out-of-distribution.

TopoDiff~\cite{maze2022topodiff} proposes a conditional diffusion model, conditioning on fields similarly to~\cite{nie2021topologygan}, and an additional guidance mechanism to encourage the generative process to sample in regions with high manufacturability and high performance (low compliance in this specific case). This work solves some of the issues connected with performance and feasibility, at the cost of slow sampling, still relying on computationally expensive physical fields, and introducing surrogate models to account for performance and manufacturability.
During inference, the model must compute the strain and force fields for each constraint configuration using FEA and use this information to condition the model, enabling it to generate optimized topologies that meet the specified requirements. This conditioning step is crucial to ensure that the model performs well and generates output that is both feasible and optimized but, at the same time, expensive and not easy to scale to higher dimensionality, complex structures, and 3D domains.

To summarize, GAN-based methods such as those proposed in \cite{nie2021topologygan, Behzadi_Ilies2021} can generate a large number of topologies efficiently but may produce un-manufacturable topologies that violate soft constraints such as volume fraction errors and lead to higher compliance structures. On the other hand, diffusion-based approaches like TopoDiff~\cite{maze2022topodiff} generate samples that satisfy constraints more accurately but are computationally expensive due to iterative sampling, reliance on physical fields, and surrogate models with auxiliary labeled data.

\paragraph{}
To overcome the limitations of existing approaches, we propose a novel method that addresses the issues of slow sampling, reliance on physical fields, and the need for additional surrogate models. Our approach involves reducing the number of steps required for sampling, approximating physical fields using a computationally inexpensive kernel based on classic ODE solutions, and integrating optimization methods like SIMP for refining the generated topology. By explicitly guiding the generated topology to regions with high manufacturability and low compliance with only a few optimization steps, we can efficiently sample good topologies without the need for external auxiliary models, FEM solvers for pre-processing, or additional labeled data.



\paragraph{Contribution.} Our contributions are the following:

\begin{itemize}
    \item We decrease the computational cost of generative topology optimization while maintaining high performance. To achieve this, we explore more efficient sampling methods for TopoDiff reducing the number of steps required for generation by an order of magnitude while ensuring minimal loss in performance across all models.
    \item We explore alternative conditioning techniques that eliminate the need to compute force and energy strain fields, which can be a major bottleneck in the optimization process. As a result, we have reduced the inference time for generation by 53.93\% compared to the baselines.
    \item We propose a generative optimization method that integrates a conditional diffusion model with traditional topology optimization algorithms. The fast conditional diffusion model predicts a preliminary topology, which is then refined using traditional topology optimization-based methods in just a few steps (5-10 iterations). This approach improves manufacturability and increases performance by 23.81\% and 25.64\%, respectively, for in- and out-of-distribution constraints.
\end{itemize}

