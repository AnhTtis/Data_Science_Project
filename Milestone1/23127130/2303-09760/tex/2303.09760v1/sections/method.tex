\section{Method}

\begin{figure}
    \centering
    \includegraphics[width=.9\linewidth]{img/figures_general/kernel_constraints_field.pdf}
    \caption{Field (left) vs Kernel (right) conditioning. Where computing the fields requires solving an expensive iterative FEA problem for each new configuration, computing the kernel relaxation is a single-step, computationally inexpensive approximation that does not rely on domain knowledge and scales to any resolution or domain structure.}
    \label{fig:kernel-field}
\end{figure}

\begin{table}
    \centering
    \resizebox{0.99\linewidth}{!}{
    \begin{tabular}{c| c c c}
        Class & Metrics & Goal & Challenging?\\
        \toprule
        Hard-constraint       & Loads Disrespect & Feasibility & No     \\
        Hard-constraint       & Floating Material & Manufacturability & Yes \\
        Soft-constraint       & Volume Fraction   & Min Cost & No    \\
        Functional Performance & Compliance Error & Max Performance & Yes\\
        Modeling Requirements  & Sampling Time & Fast Inference & Yes \\
        \bottomrule
    \end{tabular}
    }
    \caption{Design and Modelling requirements for a constrained generative model for topology optimization. Our goal is to improve the requirements that are challenging to fulfill.
    In this work, we focus on improving Floating Material, reducing Compliance Error, and reducing Sampling Time.}
    \label{tab:table-constraints}
\end{table}

Our three main objectives are: improving inference efficiency, improving the sampling time for diffusion-based topology generation while still satisfying the design requirements with a minimum decrease in performance; minimizing reliance on force and strain fields as conditioning information, reducing the computation burden at inference time and the need for ad-hoc conditioning mechanisms for each problem and domain; merging together learning-based and optimization-based methods, refining the topology generated using a conditional diffusion model, improving the final solution in terms of manufacturability and performance. 
We aim to improve generative models with constraints and eliminate the need for costly FEM solutions as conditioning mechanisms.
We refer to our approach as TopoDiff-FF, short for Topology-Diffusion-Fields-Free, and to our approach with optimization-based refinement as TopoDiff-FF + SIMP.

\paragraph{Conditioning on Hard and Soft Constraints.}
All models are subject to conditioning based on loads, boundary conditions, and volume fractions. In addition, TopoDiff and TopoDiff-GUIDED undergo conditioning based on force field and energy strain, while TopoDiff-FF is conditioned based on kernel relaxation, which is discussed in the following paragraph (see Fig.~\ref{fig:kernel-field}). 


\paragraph{Green's Functions.}
To improve the efficiency of diffusion-based topology generation and minimize reliance on force and strain fields, we aim to relax boundary conditions and loads by leveraging kernels as approximations for the way such constraints act on the domain. One possible choice of kernel structure is inspired by Green's method~\cite{garabedian1960partial}, which defines integral functions that are solutions to the time-invariant Poisson's Equation~\cite{hale2013introduction}, a generalization of Laplace's Equation for point sources excitations.
Poisson's Equation can be written as $\nabla^2_x f(x) = h$, where $h$ is a forcing term and $f$ is a generic function defined over the domain $\mathcal{X}$. This equation governs many phenomena in nature, and a special case is a forcing part $h=0$, which yields the Laplace's Equation formulation commonly employed in heat transfer problems.
Green's method is a mathematical construction to solve partial differential equations without prior knowledge of the domain. The solutions obtained with this method are known as Green's functions~\cite{keldysh1951characteristic}. While solutions obtained with this method can be complex in general, for a large class of physical problems involving constraints and forces that can be approximated with points, a simple functional form can be derived by leveraging the idea of source and sink.
Consider a laminar domain (e.g., a beam or a plate) constrained in a feasible way. If a point source is applied to this domain (e.g., a downward force on the edge of a beam or on the center of a plate) in $x_f$, such force can be described using the Dirac delta function, $\delta(x - x_f)$. The delta function is highly discontinuous but has powerful integration properties. In particular $\int f(x) \delta(x - x_f) dx = f(x_f)$ over the domain $\mathcal{X}$. The solution of the time-invariant Poisson's Equation with point concentrated forces can be written as a Green's function solution, where the solution depends only on the distance from the force application point. In particular:

\begin{equation}
        \mathcal{G}(x, x') = - \dfrac{1}{4 \pi} \dfrac{1}{|x - x'|},
    \end{equation}
where $r = |x - x'| = \sqrt{|x_i - x^{'}_{i}|^2 + |x_j - x^{'}_{j}|^2}$.
We propose to approximate the forces and loads applied to our topologies using a kernel relaxation built using Green's functions. While this formulation may not provide a correct solution for generic loads and boundary conditions, it allows us to provide computationally inexpensive conditioning information that respects the original physical and engineering constraints. By leveraging these ideas, we aim to increase the amount of information provided to condition the model, ultimately improving generative models with constraints.


\paragraph{Kernel Relaxation.}
We can use these kernels to construct a kernel relaxation method to condition generative models (Fig.~\ref{fig:kernel-field}). The idea is to use the kernels as approximations of the way boundary conditions and loads act on the domain. Specifically, we can use the kernels to represent the effects of the boundary conditions and loads as smooth functions across the domain. This approach avoids the need for computationally expensive and time-consuming finite element Analysis (FEA) to provide conditioning information.
To apply the kernel relaxation method, we first determine the locations of the sources and sinks in the domain. Then, we compute the corresponding kernels for the loads and boundary conditions, respectively. Finally, we use the resulting kernels as conditioning information for generative models.
In particular, we consider loads as sources and boundary conditions as sinks. 
For a load or source $p$, and $r = |x - x^p| = \sqrt{|x_i - x^{p}_{i}|^2 + |x_j - x^{p}_{j}|^2}$ we have:
\begin{equation}
    K_{l}(x, x^p; \alpha, \beta) \propto (1 - e^{- \alpha/r^{\beta}}) ~ \bar p,
\end{equation}
where $\bar p$ is the module of a generic force in 2D.
Notice how, for $r \rightarrow 0$, $K_l(x, x^p) \rightarrow p$, and $r \rightarrow \infty$, $K_l(x, x^p) \rightarrow 0$.
For a boundary condition or sink:
\begin{equation}
    K_{bc}(x, x^{bc}; \alpha, \beta) \propto e^{- \alpha/r^{\beta}}.
\end{equation}
We notice how closer to the boundary the kernel is null, and farther from the boundary the kernel tends to 1.
Note that the choice of $\alpha$ and $\beta$ parameters in the kernels affect the smoothness and range of the kernel functions. By adjusting these parameters, we can control the trade-off between accuracy and computational cost. Furthermore, these kernels are isotropic, meaning that they do not depend on the direction in which they are applied.
Overall, the kernel relaxation method offers a computationally inexpensive way to condition generative models on boundary conditions and loads, making them more applicable in practical engineering and design contexts.

