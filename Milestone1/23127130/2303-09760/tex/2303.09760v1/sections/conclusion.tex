\section{Discussion} 
It is essential to acknowledge the limitations of purely data-driven approaches, such as TopoDiff-FF, which struggle to generalize to out-of-distribution scenarios since these models lack a way to generalize to new situations not encountered during training. In contrast, TopoDiff incorporates additional conditioning information obtained from numerical analysis and the underlying physics of the problem, which enables the model to generalize to regions of the domain not included in the training data.
The kernel proposed for TopoDiff-FF is versatile and can be easily applied to any domain, resolution, and material, making it a valuable tool for engineering design. Contrary, FEM conditioning requires a thorough understanding of the problem, and it does not scale well with resolution and problem complexity. Therefore, TopoDiff-FF represents a powerful alternative for engineering design, particularly in situations where FEM conditioning is not feasible or effective.
Furthermore, our results demonstrate that combining the topology generated by TopoDiff-FF with a few iterations of SIMP can produce better results than either data-driven or physics-informed methods, indicating a way to increase expressivity for such models in engineering design.


\paragraph{Limitations.}
Deep Generative models for topology optimization are a promising direction to improve efficiency, scalability, and variety in engineering design but they still present limitations ~\citep{woldseth2022use}.
In general, three main objectives can be identified for data-driven approaches: direct design, speedup, and upsampling. 
The authors in~\cite{woldseth2022use} are critical of end-to-end approaches for direct design, stating that most of the methods proposed in the literature produce poor designs and are expensive and limited in the variety of problems and mesh resolutions they can handle. This is indeed a good critique because most methods focus on a specific domain and resolution. 
The authors argue that the iterative nature of topology optimization is not what makes it impractical but rather the computational load of the costly components within each iteration. They suggest that the focus should shift towards alleviating this computational load. 
The authors also suggest that the main contribution of the learning methods should be in speeding up the computations in the intermediate steps of the iterative optimization process or post-processing optimized results for manufacturability.
Where most of these critiques are accurate, there is a new wave of generative models that are tackling such challenges and solving or alleviating most of these issues, from sampling efficiency to reliance on a specific domain; to optimization-based refinement and direct design with performance awareness~\cite{regenwetter2022deep}.

\section{Conclusion}
In summary, our method of utilizing kernel approximation as conditioning for generative models and refining with optimization techniques has shown promise in enhancing precision, constraint satisfaction, manufacturability, and performance. To further improve our approach, future research could explore methods for maintaining diversity while preserving precision and constraint satisfaction. Furthermore, the scalability of our approach to larger and more complex problems should be investigated. In conclusion, we believe that our work offers new possibilities for advancing generative design in topology optimization and engineering design.