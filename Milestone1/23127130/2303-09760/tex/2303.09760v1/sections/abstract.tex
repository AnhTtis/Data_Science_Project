
\begin{abstract}
Topology Optimization seeks to find the best design that satisfies a set of constraints while maximizing system performance. Traditional iterative optimization methods like SIMP can be computationally expensive and get stuck in local minima, limiting their applicability to complex or large-scale problems. Learning-based approaches have been developed to accelerate the topology optimization process, but these methods can generate designs with floating material and low performance when challenged with out-of-distribution constraint configurations. Recently, deep generative models, such as Generative Adversarial Networks and Diffusion Models, conditioned on constraints and physics fields have shown promise, but they require extensive pre-processing and surrogate models for improving performance. To address these issues, we propose a Generative Optimization method that integrates classic optimization like SIMP as a refining mechanism for the topology generated by a deep generative model. We also remove the need for conditioning on physical fields using a computationally inexpensive approximation inspired by classic ODE solutions and reduce the number of steps needed to generate a feasible and performant topology. Our method allows us to efficiently generate good topologies and explicitly guide them to regions with high manufacturability and high performance, without the need for external auxiliary models or additional labeled data. We believe that our method can lead to significant advancements in the design and optimization of structures in engineering applications, and can be applied to a broader spectrum of performance-aware engineering design problems.
\end{abstract}

