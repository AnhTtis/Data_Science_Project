\section{Related Work}
\begin{figure}[ht]
    \centering
    \includegraphics[width=.95\linewidth]{img/figures_general/topodiff_intro.pdf}  
    \caption{TopoDiff-GUIDED. TopoDiff is a conditional diffusion model guided by a classifier and a regression score. The conditioning mechanism $c$ involves loads $l$, volume fraction $v$, and stress and energy fields $f$. i.e. $c=h(l, v, f)$. The guidance mechanism involves a classifier for the presence or absence of floating material and a regressor to minimize compliance error between generated and optimized topologies.}
    \label{fig:topodiff-intro}
\end{figure}
\vspace{5pt}
\paragraph{Topology Optimization.}
Engineering design is the process of creating solutions to technical problems under engineering requirements~\cite{shigley1985mechanical}. The goal is to create designs that are highly performant given the required constraints.
Topology Optimization (TO~\citep{bendsoe1988generating}) is a branch of engineering design and is a critical component of the design process in many industries, including aerospace, automotive, manufacturing, and software development.
From the inception of the homogenization method for TO, a number of different approaches have been proposed, including density-based~\citep{bendsoe1989optimal, rozvany1992generalized}, level-set~\citep{allaire2002level}, derivative-based~\citep{sokolowski1999topological}, evolutionary~\citep{xie1997basic}, and others~\cite{bourdin2003design}. 
The density-based methods are widely used and use a nodal-based representation where the level-set leverages shapes derivative to obtain the optimal topology.
Topology Optimization has evolved as a computationally intensive discipline, with the availability of efficient open-source code~\cite{hunter2017topy, liu2014efficient}. See~\cite{liu2014efficient} for more on this topic and ~\cite{sigmund2013topology} for a comprehensive review of the Topology Optimization field.


\paragraph{Deep Learning for Topology Optimization.}
Following the success of Deep Learning (DL) in vision, a surging interest arose recently for transferring these methods to the engineering field.
In particular, DL methods have been employed for direct-design~\cite{Abueiddaetal2020}, accelating the optimization process~\cite{Bangaetal2018}, optimizing the shape~\cite{Hertleinetal2021}, super-resolusion~\cite{Elingaard2021, Napieretal2020}, and 3D topologies~\cite{kench2021generating}. 
Among these methods, Deep Generative Models (DGMs) are especially appealing to improve design diversity in engineering design\cite{JiangChenFan2021_nano2, RawatShen2018}.
Additionally, DGMs have been used for Topology Optimization problems conditioning on constraints (loads, boundary conditions, volume fraction for the structural case), directly generating topologies~\cite{rawat2019application, sharpe2019topology} training dataset of optimized topologies, leveraging superresolution methods to improve fidelity~\cite{yu2019deep}, using filtering and iterative design approaches~\cite{berthelot2017began} to improve quality and diversity. Methods for 3D topologies have also been proposed~\citep{Behzadi_Ilies2021}.
Recently, GAN-based approaches conditioning on constraints and physical information have had success in modeling the TO problem~\cite{nie2021topologygan}.
For a comprehensive review and critique of the field, see~\cite{woldseth2022use}.

\paragraph{Conditional Diffusion Models.}
Improving sampling speed for diffusion models is an active research topic~\cite{kong2021fast}. Recently, distillation has been used to greatly reduce sampling steps~\cite{meng2022distillation}.
Methods to condition DDPM have been proposed, conditioning at inference time~\cite{choi2021ilvr}, learning a class-conditional score~\cite{song2020score}, explicitly conditioning on class information~\cite{nichol2021improved}, set-based features~\cite{giannone2022few}, and physical properties~\cite{xie2021crystal}.
Text-to-image diffusion models~\cite{ramesh2022hierarchical} have been recently proposed for a guided generation.
Recently, TopoDiff~\citep{maze2022topodiff} has shown that conditional diffusion models are effective for generating topologies that fulfill the constraints, and have high manufacturability and high performance. TopoDiff relies on physics information and surrogate models to guide the sampling of novel topologies with good performance.