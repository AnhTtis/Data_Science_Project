In this paper, we have presented a Geometric Deep Learning (GDL) approach for classifying and down-selecting in graph-based engineering design problems towards sets of better-performing solutions.
Using the electrical circuit graph-design case study where all the graphs were known but not their expensive optimization-based performance values, it was shown the potential capabilities of GDL in helping classify ``good'' and ''bad' graphs based on limited performance data, as well as some insights into the key hyperparameters that need to be tuned.
Additionally, the inclusion of the graph-based features eigenvalue centrality and betweenness centrality improved many of the key metrics.

The results showed that GDL can be used as an effective and efficient method for narrowing the graphs sets for this case study as the model was able to achieve a total set classification accuracy of 80\% in using only 10\% of the graphs, as well as the iterative GDL classification approach identifying all of the top 10 and 87 of the top 100 graphs at 25\% the computational cost of complete enumeration.

Key future work items include iteratively adding new graphs to the dataset as mentioned in \sref{exp4r} and investigating this approach on other larger datasets in different engineering fields, such as ones with directed graphs and multiple performance metrics.
The future of GDL is promising for a wide application in engineering graph-design problems.


