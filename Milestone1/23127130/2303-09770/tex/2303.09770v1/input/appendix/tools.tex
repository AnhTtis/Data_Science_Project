The primary tool used is PyTorch-Geometric (PyG) \cite{b16} because of the extensive applications that PyG can perform on graph data.
It is built on top of PyTorch \cite{b15}, an open-source machine learning framework, which also contains an extensive library of tools for model manipulation and data analysis.
Since both tools are used primarily with Python \cite{b57}, that will be the language of choice for the model.
We also include Networkx, a Python package for the creation and manipulation of complex networks \cite{b17}.
This decision is because PyG requires the data to be of a certain instance, whether a SciPy sparse matrix or a Trimesh instance; it needs to be in a form readable by PyG.
Networkx was chosen because of the information that can be added to the graph, which can be easily transformed into a PyG instance.
We then use Pandas \cite{b77} for its data organization capabilities to import the data and prepare it to be passed on to Networkx and, finally, PyG.
For a full list of the tools used and their versions, please see \tref{tooltab}.

All the tools were utilized on a personal workstation consisting of an Intel Core i9-9900k CPU @ 3.60GHz, 32GB installed memory, and an Nvidia GeForce RTX 2060 Super GPU.

\begin{table}[h]
\centering
\caption{A list of the primary tools and their versions.}
\label{tooltab} 
\begingroup
\setlength{\tabcolsep}{6pt} % Default value: 6pt
\renewcommand{\arraystretch}{1.1} % Default value: 1
\begin{tabular}{rl}
\hline \hline
\textbf{Tool} & \textbf{Version} \\
\hline
Python & 3.9\\
Networkx & 2.8.7\\
PyTorch & 1.12.1\\
PyTorch-Geometric & 2.1.0\\
SciPy & 1.9.1\\
Pandas & 1.5.0\\
\hline \hline
\end{tabular}
\endgroup
\end{table}