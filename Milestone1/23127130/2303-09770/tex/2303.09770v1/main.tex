\documentclass[10pt,twocolumn,cleanfoot,cleanhead]{config/asme2e}
\special{papersize=8.5in,11in}

% to use roman numerals for table references
% only required if using the endfloat package
% \renewcommand{\theposttbl}{\Roman{posttbl}}
% same for figures
% \renewcommand{\thepostfig}{\Roman{postfig}}

% change font size for title
% \Huge for 10pt is 24.88 (first number), with baselineskip = 1.2 * fontsize, 29.856
% no number for 9pt, use 24.88
% font sizes https://tex.stackexchange.com/questions/24599/what-point-pt-font-size-are-large-etc
\def\titlefontsize{\fontsize{24.88}{29.856}\selectfont}
% to change font size for bibliography
% when using biblatex
% default is \normalsize
% font list:
% \tiny, \scriptsize, \footnotesize, \small, \normalsize, \large, \Large, \LARGE, \huge and \Huge
\def\bibliographyfontsize{\fontsize{7.3}{8.3pt}\selectfont}


\title{On the Use of Geometric Deep Learning Towards the Evaluation of Graph-Centric Engineering Systems}

\author{}
\author{Anthony Sirico Jr.\thanks{Corresponding author, \texttt{\href{mailto:anthony.sirico@colostate.edu}{anthony.sirico@colostate.edu}}}, Daniel~R.~Herber
\affiliation{
Department of Systems Engineering \\
Colorado State University \\
Fort Collins, CO 80523 \\
Email:~\texttt{\{\href{mailto:anthony.sirico@colostate.edu}{anthony.sirico}, \href{mailto:daniel.herber@colostate.edu}{daniel.herber}\}{@}colostate.edu}
}
}
\author{}



\usepackage{amsmath}

%---------------------------------------------------------------------
\begin{document}
 \setlength{\parskip}{0pt}
 \setlength{\parsep}{0pt}
 \setlength{\headsep}{0pt}

%\setlength{\topskip}{0pt}
\setlength{\topsep}{0pt}
%\setlength{\partopsep}{0pt}

% equations
\abovedisplayshortskip=3pt
\belowdisplayshortskip=3pt
\abovedisplayskip=3pt
\belowdisplayskip=3pt

\titlespacing*{\section}{0pt}{18pt plus 1pt minus 1pt}{3pt plus 0.5pt minus 0.5pt}

\titlespacing*{\subsection}{0pt}{9pt plus 1pt minus 0.5pt}{1pt plus 0.5pt minus 0.5pt}

\titlespacing*{\subsubsection}{0pt}{9pt plus 1pt minus 0.5pt}{1pt plus 0.5pt minus 0.5pt}

\maketitle

%---------------
\begin{abstract}\noindent
\textit{Many complex engineering systems can be represented in a topological form, such as graphs.
This paper utilizes a machine learning technique called Geometric Deep Learning (GDL) to aid designers with challenging, graph-centric design problems. 
The strategy presented here is to take the graph data and apply GDL to seek the best realizable performing solution effectively and efficiently with lower computational costs. 
This case study used here is the synthesis of analog electrical circuits that attempt to match a specific frequency response within a particular frequency range.
Previous studies utilized an enumeration technique to generate 43,249 unique undirected graphs presenting valid potential circuits.
Unfortunately, determining the sizing and performance of many circuits can be too expensive.
To reduce computational costs with a quantified trade-off in accuracy, the fraction of the circuit graphs and their performance are used as input data to a classification-focused GDL model.
Then, the GDL model can be used to predict the remainder cheaply, thus, aiding decision-makers in the search for the best graph solutions.
The results discussed in this paper show that additional graph-based features are useful, favorable total set classification accuracy of 80\% in using only 10\% of the graphs, and iteratively-built GDL models can further subdivide the graphs into targeted groups with medians significantly closer to the best graph and containing 87 of the top 100 best performing graphs.
}
\end{abstract}

\vspace{1ex}
\noindent Keywords: machine learning, geometric deep learning, graph classification, graph-based design, circuit synthesis

%---------------
\newpage
\section{INTRODUCTION}
\section{Introduction}

The increasing complexity of source code poses a key challenge to the reliability of large-scale software systems. Software bugs in these systems can lead to safety issues~\cite{bug_safety} for users around the world as well as cause non-negligible financial losses~\cite{bug_loss}. As such, developers have to spend a large amount of time and effort on bug fixing. Consequently, \aprfull (\apr), designed to automatically generate patches to fix software bugs, has attracted wide attention from both academia and industry~\cite{long2016prophet, legoues2012genprog, long2015spr, lou2020can, tufano2018empstudy}. 


To achieve \apr, one popular approach is known as Generate-and-Validate (G\&V)~\cite{qi2015gv, ghanbari2019prapr, lou2020can, le2016hdrepair, legoues2012genprog, wen2018capgen, hua2018sketchfix, martinez2016astor, koyuncu2020fixminder, liu2019tbar, liu2019avatar}, which is typically based on the following pipeline: First, fault localization techniques~\cite{wong2016fl, abreu2007ochiai, zhang2013injecting, papadakis2015metallaxis, li2019deepfl, li2017transforming} are applied to determine the suspicious locations in programs where bugs are likely to exist. Then, the buggy locations are used by the \apr tools to generate a list of patches that replace buggy lines with correct lines. Afterward, each patch is validated against the original test suite to identify any \emph{plausible patches} (i.e., passing all tests in the test suite). Finally, to determine the \emph{correct patches}, developers examine the list of plausible patches to see if any of them can correctly fix the bug. 

Traditional \apr tools can mainly be categorized into heuristic-based~\cite{legoues2012genprog, le2016hdrepair, wen2018capgen}, constraint-based~\cite{mechtaev2016angelix, le2017s3, demacro2014nopol, long2015spr} and \template~\cite{ghanbari2019prapr, hua2018sketchfix, martinez2016astor, liu2019tbar, liu2019avatar}. Among these traditional tools, \template \apr tools~\cite{ghanbari2019prapr, liu2019tbar, benton2020effectiveness} have been able to achieve state-of-the-art results. \Template \apr tools typically leverage pre-defined templates (e.g., adding a nullness check) for bug fixing. However, since these fix templates are typically handcrafted, the number and types of bugs they are able to fix can be limited. 



To address the limitations of traditional \apr, researchers have proposed various \learning \apr tools~\cite{li2020dlfix, chen2018sequencer, jiang2021cure, lutellier2020coconut, zhu2021recoder, ye2022rewardrepair} based on the \nmtfull (\nmt) architecture~\cite{sutskever2014mt} where the input is the buggy code snippets and the goal is to translate the buggy code snippets into a fixed version. To accomplish this, \learning \apr tools require supervised training datasets with pairs of both buggy and fixed code snippets in order to learn how to perform this translation step. These training data are usually obtained by mining historical bug fixes using heuristics/keywords~\cite{dallmeier2007benchmark}, which can be imprecise for identifying bug-fixing commits; even the actual bug-fixing commits can include irrelevant code changes, leading to further pollution in the dataset~\cite{xia2022alpharepair}.
% 
Moreover, it can be hard for such \apr tools to generalize and fix bug types unseen during training. 



To better leverage recent advances in \plmfull{s} (\plm{s}), researchers~\cite{xia2022alpharepair, xia2023repairstudy, kolak2022patch, prenner2021codexws} have directly applied \plm{s} to generate patches without bug-fixing datasets. These \llm-based \apr tools work by either directly generating a complete code function~\cite{prenner2021codexws, xia2023repairstudy} or predict/infill the correct code snippet given its surrounding context~\cite{xia2022alpharepair, xia2023repairstudy}. By directly using \llm{s} that are pre-trained on billions of open-source code snippets, \llm-based \apr tools can achieve state-of-the-art performance on many repair datasets~\cite{xia2022alpharepair}. 


% 
%
%

Traditional \apr tools have long used the insight of the \emph{plastic surgery hypothesis}~\cite{barr2014plastic} where it states that the code ingredients to fix a bug already exist within the same project. Traditional \apr tools have manually designed pattern-~\cite{ghanbari2019prapr, saha2017elixir} or heuristic-based~\cite{jiang2018simfix, legoues2012genprog} approaches to finding and using such relevant code ingredients to generate fixes for bugs. However, the plastic surgery hypothesis has been largely ignored in \llm-based \apr. In fact, \llm provides a unique opportunity to fully automate the plastic surgery hypothesis idea via fine-tuning (learning project-specific information via model updates from the buggy project) and prompting (directly providing relevant code ingredients to the model), and make it directly applicable to different languages (since the \llm{s} are typically multi-lingual).%
Moreover, despite the intensive manual efforts involved, traditional \apr tools still cannot fully leverage project-specific information due to large search space for leveraging/composing existing code ingredients. In contrast, the project-specific information can effectively leveraged by \llm{s} due to their power in code understanding/vectorization, e.g., even partial/imprecise information may still guide \llm{s} in correct patch generation!
 To this end, we ask the question: \emph{How useful is the plastic surgery hypothesis in the era of \plm{s}}?








\mypara{Our Work.} To answer the question, we present \ourtech{\xspace} -- a \llm-based approach that automatically utilizes the plastic surgery hypothesis by systematically combining multiple fine-tuning and prompting strategies for \apr. \ourtech fine-tunes \plm{s} using two novel domain-specific training strategies: \textbf{\epfinetune} -- we fine-tune using the original buggy project by aggressively masking out a high percentage of tokens, which allows \plm to learn project-specific code tokens and programming styles; and \textbf{\rofinetune} -- which only masks out a single continuous code sequence per training sample, allowing the model to get used to the final \csapr task of predicting a single continuous code sequence. Furthermore, we directly leverage the ability for \plm{s} to understand natural language instructions and introduce a novel prompting strategy, \textbf{\idprompting}, which uses information retrieval and static analysis to obtain a list of relevant identifiers for the buggy lines. While such relevant identifiers are critical for fixing some difficult bugs, they may not be seen by the \llm during inference due to limited context window size. Through the use of prompting, we directly tell the model to use these extracted identifiers (relevant code ingredients) to generate the correct code. Finally, to perform repair, we combine all four model variants (including the base model, both fine-tuned models and the base model with prompting) for the final repair.





While our insight of leveraging the plastic surgery hypothesis for \llm-based \apr is generalizable across different types of \plm{s}, to implement \ourtech, we choose a recent \plm{\xspace}, \ctfive~\cite{wang2021codet5}, which is pre-trained on millions of open-source code snippets. \ctfive is an encoder-decoder model trained using \mspfull (\msp) objective where a percentage of tokens are masked out and each continuous masked token sequence is referred to as a masked span. Also, although we only extract relevant identifiers from the current buggy project (since this paper focuses on the plastic surgery hypothesis), our work can be easily extended to obtain other code information (such as relevant statements or functions) from other sources, such as  the massive pre-training corpora~\cite{husain2020codesearchnet} or historical bug-fixing datasets~\cite{jiang2019infer}, which can provide more coding knowledge for \llm{s}. Besides, although we mainly focus on using traditional string comparison algorithms for information retrieval in this paper, these techniques can be easily replaced by other frequency-based retrieval~\cite{robertson2009probabilistic} and neural search (or embedding-based search)~\cite{reimers2019sentence}.
  In summary, this paper makes the following contributions:


%


\begin{itemize}[noitemsep, leftmargin=*, topsep=0pt]
    \item \textbf{Dimension.} This paper is the first to revisit the important plastic surgery hypothesis in the era of \llm{s}. It opens up a new dimension for \llm-based \apr to incorporate previously neglected information from the buggy project itself to boost \apr performance. Furthermore, it demonstrates the promising future of retrieval-based prompting for modern \llm-based \apr.
    \item \textbf{Implementation.} We implement \ourtech based on the recent \ctfive model. We augment the model using two novel fine-tuning strategies: \epfinetune and \rofinetune, along with a novel prompting strategy based on information retrieval and static analysis: \idprompting. We combine the patches generated by all four models together and perform patch ranking to speed up \apr.% 
    \item \textbf{Evaluation Study.} We conduct an extensive evaluation against state-of-the-art \apr tools. On the widely studied \dfj 1.2 and 2.0 datasets~\cite{just2014dfj}, \ourtech is able to achieve the new state-of-the-art results of 89 and 44 correct bug fixes (15 and 8 more than best baseline) respectively.  Furthermore, we perform a broad ablation study to justify our design. \ourtech demonstrates for the first time that the plastic surgery hypothesis can substantially boost \llm-based \apr and advance state-of-the-art \apr, while being fully automated and general. Moreover, even partial/imprecise code ingredients may still effectively guide \llm{s} for \apr!
\end{itemize}



% \clearpage
\section{BACKGROUND}
\section{Introduction}

The increasing complexity of source code poses a key challenge to the reliability of large-scale software systems. Software bugs in these systems can lead to safety issues~\cite{bug_safety} for users around the world as well as cause non-negligible financial losses~\cite{bug_loss}. As such, developers have to spend a large amount of time and effort on bug fixing. Consequently, \aprfull (\apr), designed to automatically generate patches to fix software bugs, has attracted wide attention from both academia and industry~\cite{long2016prophet, legoues2012genprog, long2015spr, lou2020can, tufano2018empstudy}. 


To achieve \apr, one popular approach is known as Generate-and-Validate (G\&V)~\cite{qi2015gv, ghanbari2019prapr, lou2020can, le2016hdrepair, legoues2012genprog, wen2018capgen, hua2018sketchfix, martinez2016astor, koyuncu2020fixminder, liu2019tbar, liu2019avatar}, which is typically based on the following pipeline: First, fault localization techniques~\cite{wong2016fl, abreu2007ochiai, zhang2013injecting, papadakis2015metallaxis, li2019deepfl, li2017transforming} are applied to determine the suspicious locations in programs where bugs are likely to exist. Then, the buggy locations are used by the \apr tools to generate a list of patches that replace buggy lines with correct lines. Afterward, each patch is validated against the original test suite to identify any \emph{plausible patches} (i.e., passing all tests in the test suite). Finally, to determine the \emph{correct patches}, developers examine the list of plausible patches to see if any of them can correctly fix the bug. 

Traditional \apr tools can mainly be categorized into heuristic-based~\cite{legoues2012genprog, le2016hdrepair, wen2018capgen}, constraint-based~\cite{mechtaev2016angelix, le2017s3, demacro2014nopol, long2015spr} and \template~\cite{ghanbari2019prapr, hua2018sketchfix, martinez2016astor, liu2019tbar, liu2019avatar}. Among these traditional tools, \template \apr tools~\cite{ghanbari2019prapr, liu2019tbar, benton2020effectiveness} have been able to achieve state-of-the-art results. \Template \apr tools typically leverage pre-defined templates (e.g., adding a nullness check) for bug fixing. However, since these fix templates are typically handcrafted, the number and types of bugs they are able to fix can be limited. 



To address the limitations of traditional \apr, researchers have proposed various \learning \apr tools~\cite{li2020dlfix, chen2018sequencer, jiang2021cure, lutellier2020coconut, zhu2021recoder, ye2022rewardrepair} based on the \nmtfull (\nmt) architecture~\cite{sutskever2014mt} where the input is the buggy code snippets and the goal is to translate the buggy code snippets into a fixed version. To accomplish this, \learning \apr tools require supervised training datasets with pairs of both buggy and fixed code snippets in order to learn how to perform this translation step. These training data are usually obtained by mining historical bug fixes using heuristics/keywords~\cite{dallmeier2007benchmark}, which can be imprecise for identifying bug-fixing commits; even the actual bug-fixing commits can include irrelevant code changes, leading to further pollution in the dataset~\cite{xia2022alpharepair}.
% 
Moreover, it can be hard for such \apr tools to generalize and fix bug types unseen during training. 



To better leverage recent advances in \plmfull{s} (\plm{s}), researchers~\cite{xia2022alpharepair, xia2023repairstudy, kolak2022patch, prenner2021codexws} have directly applied \plm{s} to generate patches without bug-fixing datasets. These \llm-based \apr tools work by either directly generating a complete code function~\cite{prenner2021codexws, xia2023repairstudy} or predict/infill the correct code snippet given its surrounding context~\cite{xia2022alpharepair, xia2023repairstudy}. By directly using \llm{s} that are pre-trained on billions of open-source code snippets, \llm-based \apr tools can achieve state-of-the-art performance on many repair datasets~\cite{xia2022alpharepair}. 


% 
%
%

Traditional \apr tools have long used the insight of the \emph{plastic surgery hypothesis}~\cite{barr2014plastic} where it states that the code ingredients to fix a bug already exist within the same project. Traditional \apr tools have manually designed pattern-~\cite{ghanbari2019prapr, saha2017elixir} or heuristic-based~\cite{jiang2018simfix, legoues2012genprog} approaches to finding and using such relevant code ingredients to generate fixes for bugs. However, the plastic surgery hypothesis has been largely ignored in \llm-based \apr. In fact, \llm provides a unique opportunity to fully automate the plastic surgery hypothesis idea via fine-tuning (learning project-specific information via model updates from the buggy project) and prompting (directly providing relevant code ingredients to the model), and make it directly applicable to different languages (since the \llm{s} are typically multi-lingual).%
Moreover, despite the intensive manual efforts involved, traditional \apr tools still cannot fully leverage project-specific information due to large search space for leveraging/composing existing code ingredients. In contrast, the project-specific information can effectively leveraged by \llm{s} due to their power in code understanding/vectorization, e.g., even partial/imprecise information may still guide \llm{s} in correct patch generation!
 To this end, we ask the question: \emph{How useful is the plastic surgery hypothesis in the era of \plm{s}}?








\mypara{Our Work.} To answer the question, we present \ourtech{\xspace} -- a \llm-based approach that automatically utilizes the plastic surgery hypothesis by systematically combining multiple fine-tuning and prompting strategies for \apr. \ourtech fine-tunes \plm{s} using two novel domain-specific training strategies: \textbf{\epfinetune} -- we fine-tune using the original buggy project by aggressively masking out a high percentage of tokens, which allows \plm to learn project-specific code tokens and programming styles; and \textbf{\rofinetune} -- which only masks out a single continuous code sequence per training sample, allowing the model to get used to the final \csapr task of predicting a single continuous code sequence. Furthermore, we directly leverage the ability for \plm{s} to understand natural language instructions and introduce a novel prompting strategy, \textbf{\idprompting}, which uses information retrieval and static analysis to obtain a list of relevant identifiers for the buggy lines. While such relevant identifiers are critical for fixing some difficult bugs, they may not be seen by the \llm during inference due to limited context window size. Through the use of prompting, we directly tell the model to use these extracted identifiers (relevant code ingredients) to generate the correct code. Finally, to perform repair, we combine all four model variants (including the base model, both fine-tuned models and the base model with prompting) for the final repair.





While our insight of leveraging the plastic surgery hypothesis for \llm-based \apr is generalizable across different types of \plm{s}, to implement \ourtech, we choose a recent \plm{\xspace}, \ctfive~\cite{wang2021codet5}, which is pre-trained on millions of open-source code snippets. \ctfive is an encoder-decoder model trained using \mspfull (\msp) objective where a percentage of tokens are masked out and each continuous masked token sequence is referred to as a masked span. Also, although we only extract relevant identifiers from the current buggy project (since this paper focuses on the plastic surgery hypothesis), our work can be easily extended to obtain other code information (such as relevant statements or functions) from other sources, such as  the massive pre-training corpora~\cite{husain2020codesearchnet} or historical bug-fixing datasets~\cite{jiang2019infer}, which can provide more coding knowledge for \llm{s}. Besides, although we mainly focus on using traditional string comparison algorithms for information retrieval in this paper, these techniques can be easily replaced by other frequency-based retrieval~\cite{robertson2009probabilistic} and neural search (or embedding-based search)~\cite{reimers2019sentence}.
  In summary, this paper makes the following contributions:


%


\begin{itemize}[noitemsep, leftmargin=*, topsep=0pt]
    \item \textbf{Dimension.} This paper is the first to revisit the important plastic surgery hypothesis in the era of \llm{s}. It opens up a new dimension for \llm-based \apr to incorporate previously neglected information from the buggy project itself to boost \apr performance. Furthermore, it demonstrates the promising future of retrieval-based prompting for modern \llm-based \apr.
    \item \textbf{Implementation.} We implement \ourtech based on the recent \ctfive model. We augment the model using two novel fine-tuning strategies: \epfinetune and \rofinetune, along with a novel prompting strategy based on information retrieval and static analysis: \idprompting. We combine the patches generated by all four models together and perform patch ranking to speed up \apr.% 
    \item \textbf{Evaluation Study.} We conduct an extensive evaluation against state-of-the-art \apr tools. On the widely studied \dfj 1.2 and 2.0 datasets~\cite{just2014dfj}, \ourtech is able to achieve the new state-of-the-art results of 89 and 44 correct bug fixes (15 and 8 more than best baseline) respectively.  Furthermore, we perform a broad ablation study to justify our design. \ourtech demonstrates for the first time that the plastic surgery hypothesis can substantially boost \llm-based \apr and advance state-of-the-art \apr, while being fully automated and general. Moreover, even partial/imprecise code ingredients may still effectively guide \llm{s} for \apr!
\end{itemize}



% \clearpage
\section{METHODOLOGY}
\label{method}
\section{Introduction}

The increasing complexity of source code poses a key challenge to the reliability of large-scale software systems. Software bugs in these systems can lead to safety issues~\cite{bug_safety} for users around the world as well as cause non-negligible financial losses~\cite{bug_loss}. As such, developers have to spend a large amount of time and effort on bug fixing. Consequently, \aprfull (\apr), designed to automatically generate patches to fix software bugs, has attracted wide attention from both academia and industry~\cite{long2016prophet, legoues2012genprog, long2015spr, lou2020can, tufano2018empstudy}. 


To achieve \apr, one popular approach is known as Generate-and-Validate (G\&V)~\cite{qi2015gv, ghanbari2019prapr, lou2020can, le2016hdrepair, legoues2012genprog, wen2018capgen, hua2018sketchfix, martinez2016astor, koyuncu2020fixminder, liu2019tbar, liu2019avatar}, which is typically based on the following pipeline: First, fault localization techniques~\cite{wong2016fl, abreu2007ochiai, zhang2013injecting, papadakis2015metallaxis, li2019deepfl, li2017transforming} are applied to determine the suspicious locations in programs where bugs are likely to exist. Then, the buggy locations are used by the \apr tools to generate a list of patches that replace buggy lines with correct lines. Afterward, each patch is validated against the original test suite to identify any \emph{plausible patches} (i.e., passing all tests in the test suite). Finally, to determine the \emph{correct patches}, developers examine the list of plausible patches to see if any of them can correctly fix the bug. 

Traditional \apr tools can mainly be categorized into heuristic-based~\cite{legoues2012genprog, le2016hdrepair, wen2018capgen}, constraint-based~\cite{mechtaev2016angelix, le2017s3, demacro2014nopol, long2015spr} and \template~\cite{ghanbari2019prapr, hua2018sketchfix, martinez2016astor, liu2019tbar, liu2019avatar}. Among these traditional tools, \template \apr tools~\cite{ghanbari2019prapr, liu2019tbar, benton2020effectiveness} have been able to achieve state-of-the-art results. \Template \apr tools typically leverage pre-defined templates (e.g., adding a nullness check) for bug fixing. However, since these fix templates are typically handcrafted, the number and types of bugs they are able to fix can be limited. 



To address the limitations of traditional \apr, researchers have proposed various \learning \apr tools~\cite{li2020dlfix, chen2018sequencer, jiang2021cure, lutellier2020coconut, zhu2021recoder, ye2022rewardrepair} based on the \nmtfull (\nmt) architecture~\cite{sutskever2014mt} where the input is the buggy code snippets and the goal is to translate the buggy code snippets into a fixed version. To accomplish this, \learning \apr tools require supervised training datasets with pairs of both buggy and fixed code snippets in order to learn how to perform this translation step. These training data are usually obtained by mining historical bug fixes using heuristics/keywords~\cite{dallmeier2007benchmark}, which can be imprecise for identifying bug-fixing commits; even the actual bug-fixing commits can include irrelevant code changes, leading to further pollution in the dataset~\cite{xia2022alpharepair}.
% 
Moreover, it can be hard for such \apr tools to generalize and fix bug types unseen during training. 



To better leverage recent advances in \plmfull{s} (\plm{s}), researchers~\cite{xia2022alpharepair, xia2023repairstudy, kolak2022patch, prenner2021codexws} have directly applied \plm{s} to generate patches without bug-fixing datasets. These \llm-based \apr tools work by either directly generating a complete code function~\cite{prenner2021codexws, xia2023repairstudy} or predict/infill the correct code snippet given its surrounding context~\cite{xia2022alpharepair, xia2023repairstudy}. By directly using \llm{s} that are pre-trained on billions of open-source code snippets, \llm-based \apr tools can achieve state-of-the-art performance on many repair datasets~\cite{xia2022alpharepair}. 


% 
%
%

Traditional \apr tools have long used the insight of the \emph{plastic surgery hypothesis}~\cite{barr2014plastic} where it states that the code ingredients to fix a bug already exist within the same project. Traditional \apr tools have manually designed pattern-~\cite{ghanbari2019prapr, saha2017elixir} or heuristic-based~\cite{jiang2018simfix, legoues2012genprog} approaches to finding and using such relevant code ingredients to generate fixes for bugs. However, the plastic surgery hypothesis has been largely ignored in \llm-based \apr. In fact, \llm provides a unique opportunity to fully automate the plastic surgery hypothesis idea via fine-tuning (learning project-specific information via model updates from the buggy project) and prompting (directly providing relevant code ingredients to the model), and make it directly applicable to different languages (since the \llm{s} are typically multi-lingual).%
Moreover, despite the intensive manual efforts involved, traditional \apr tools still cannot fully leverage project-specific information due to large search space for leveraging/composing existing code ingredients. In contrast, the project-specific information can effectively leveraged by \llm{s} due to their power in code understanding/vectorization, e.g., even partial/imprecise information may still guide \llm{s} in correct patch generation!
 To this end, we ask the question: \emph{How useful is the plastic surgery hypothesis in the era of \plm{s}}?








\mypara{Our Work.} To answer the question, we present \ourtech{\xspace} -- a \llm-based approach that automatically utilizes the plastic surgery hypothesis by systematically combining multiple fine-tuning and prompting strategies for \apr. \ourtech fine-tunes \plm{s} using two novel domain-specific training strategies: \textbf{\epfinetune} -- we fine-tune using the original buggy project by aggressively masking out a high percentage of tokens, which allows \plm to learn project-specific code tokens and programming styles; and \textbf{\rofinetune} -- which only masks out a single continuous code sequence per training sample, allowing the model to get used to the final \csapr task of predicting a single continuous code sequence. Furthermore, we directly leverage the ability for \plm{s} to understand natural language instructions and introduce a novel prompting strategy, \textbf{\idprompting}, which uses information retrieval and static analysis to obtain a list of relevant identifiers for the buggy lines. While such relevant identifiers are critical for fixing some difficult bugs, they may not be seen by the \llm during inference due to limited context window size. Through the use of prompting, we directly tell the model to use these extracted identifiers (relevant code ingredients) to generate the correct code. Finally, to perform repair, we combine all four model variants (including the base model, both fine-tuned models and the base model with prompting) for the final repair.





While our insight of leveraging the plastic surgery hypothesis for \llm-based \apr is generalizable across different types of \plm{s}, to implement \ourtech, we choose a recent \plm{\xspace}, \ctfive~\cite{wang2021codet5}, which is pre-trained on millions of open-source code snippets. \ctfive is an encoder-decoder model trained using \mspfull (\msp) objective where a percentage of tokens are masked out and each continuous masked token sequence is referred to as a masked span. Also, although we only extract relevant identifiers from the current buggy project (since this paper focuses on the plastic surgery hypothesis), our work can be easily extended to obtain other code information (such as relevant statements or functions) from other sources, such as  the massive pre-training corpora~\cite{husain2020codesearchnet} or historical bug-fixing datasets~\cite{jiang2019infer}, which can provide more coding knowledge for \llm{s}. Besides, although we mainly focus on using traditional string comparison algorithms for information retrieval in this paper, these techniques can be easily replaced by other frequency-based retrieval~\cite{robertson2009probabilistic} and neural search (or embedding-based search)~\cite{reimers2019sentence}.
  In summary, this paper makes the following contributions:


%


\begin{itemize}[noitemsep, leftmargin=*, topsep=0pt]
    \item \textbf{Dimension.} This paper is the first to revisit the important plastic surgery hypothesis in the era of \llm{s}. It opens up a new dimension for \llm-based \apr to incorporate previously neglected information from the buggy project itself to boost \apr performance. Furthermore, it demonstrates the promising future of retrieval-based prompting for modern \llm-based \apr.
    \item \textbf{Implementation.} We implement \ourtech based on the recent \ctfive model. We augment the model using two novel fine-tuning strategies: \epfinetune and \rofinetune, along with a novel prompting strategy based on information retrieval and static analysis: \idprompting. We combine the patches generated by all four models together and perform patch ranking to speed up \apr.% 
    \item \textbf{Evaluation Study.} We conduct an extensive evaluation against state-of-the-art \apr tools. On the widely studied \dfj 1.2 and 2.0 datasets~\cite{just2014dfj}, \ourtech is able to achieve the new state-of-the-art results of 89 and 44 correct bug fixes (15 and 8 more than best baseline) respectively.  Furthermore, we perform a broad ablation study to justify our design. \ourtech demonstrates for the first time that the plastic surgery hypothesis can substantially boost \llm-based \apr and advance state-of-the-art \apr, while being fully automated and general. Moreover, even partial/imprecise code ingredients may still effectively guide \llm{s} for \apr!
\end{itemize}



% \clearpage
\section{CASE STUDY: ELECTRIC CIRCUIT FREQUENCY RESPONSE GRAPH-DESIGN PROBLEM}
\label{case}
\section{Introduction}

The increasing complexity of source code poses a key challenge to the reliability of large-scale software systems. Software bugs in these systems can lead to safety issues~\cite{bug_safety} for users around the world as well as cause non-negligible financial losses~\cite{bug_loss}. As such, developers have to spend a large amount of time and effort on bug fixing. Consequently, \aprfull (\apr), designed to automatically generate patches to fix software bugs, has attracted wide attention from both academia and industry~\cite{long2016prophet, legoues2012genprog, long2015spr, lou2020can, tufano2018empstudy}. 


To achieve \apr, one popular approach is known as Generate-and-Validate (G\&V)~\cite{qi2015gv, ghanbari2019prapr, lou2020can, le2016hdrepair, legoues2012genprog, wen2018capgen, hua2018sketchfix, martinez2016astor, koyuncu2020fixminder, liu2019tbar, liu2019avatar}, which is typically based on the following pipeline: First, fault localization techniques~\cite{wong2016fl, abreu2007ochiai, zhang2013injecting, papadakis2015metallaxis, li2019deepfl, li2017transforming} are applied to determine the suspicious locations in programs where bugs are likely to exist. Then, the buggy locations are used by the \apr tools to generate a list of patches that replace buggy lines with correct lines. Afterward, each patch is validated against the original test suite to identify any \emph{plausible patches} (i.e., passing all tests in the test suite). Finally, to determine the \emph{correct patches}, developers examine the list of plausible patches to see if any of them can correctly fix the bug. 

Traditional \apr tools can mainly be categorized into heuristic-based~\cite{legoues2012genprog, le2016hdrepair, wen2018capgen}, constraint-based~\cite{mechtaev2016angelix, le2017s3, demacro2014nopol, long2015spr} and \template~\cite{ghanbari2019prapr, hua2018sketchfix, martinez2016astor, liu2019tbar, liu2019avatar}. Among these traditional tools, \template \apr tools~\cite{ghanbari2019prapr, liu2019tbar, benton2020effectiveness} have been able to achieve state-of-the-art results. \Template \apr tools typically leverage pre-defined templates (e.g., adding a nullness check) for bug fixing. However, since these fix templates are typically handcrafted, the number and types of bugs they are able to fix can be limited. 



To address the limitations of traditional \apr, researchers have proposed various \learning \apr tools~\cite{li2020dlfix, chen2018sequencer, jiang2021cure, lutellier2020coconut, zhu2021recoder, ye2022rewardrepair} based on the \nmtfull (\nmt) architecture~\cite{sutskever2014mt} where the input is the buggy code snippets and the goal is to translate the buggy code snippets into a fixed version. To accomplish this, \learning \apr tools require supervised training datasets with pairs of both buggy and fixed code snippets in order to learn how to perform this translation step. These training data are usually obtained by mining historical bug fixes using heuristics/keywords~\cite{dallmeier2007benchmark}, which can be imprecise for identifying bug-fixing commits; even the actual bug-fixing commits can include irrelevant code changes, leading to further pollution in the dataset~\cite{xia2022alpharepair}.
% 
Moreover, it can be hard for such \apr tools to generalize and fix bug types unseen during training. 



To better leverage recent advances in \plmfull{s} (\plm{s}), researchers~\cite{xia2022alpharepair, xia2023repairstudy, kolak2022patch, prenner2021codexws} have directly applied \plm{s} to generate patches without bug-fixing datasets. These \llm-based \apr tools work by either directly generating a complete code function~\cite{prenner2021codexws, xia2023repairstudy} or predict/infill the correct code snippet given its surrounding context~\cite{xia2022alpharepair, xia2023repairstudy}. By directly using \llm{s} that are pre-trained on billions of open-source code snippets, \llm-based \apr tools can achieve state-of-the-art performance on many repair datasets~\cite{xia2022alpharepair}. 


% 
%
%

Traditional \apr tools have long used the insight of the \emph{plastic surgery hypothesis}~\cite{barr2014plastic} where it states that the code ingredients to fix a bug already exist within the same project. Traditional \apr tools have manually designed pattern-~\cite{ghanbari2019prapr, saha2017elixir} or heuristic-based~\cite{jiang2018simfix, legoues2012genprog} approaches to finding and using such relevant code ingredients to generate fixes for bugs. However, the plastic surgery hypothesis has been largely ignored in \llm-based \apr. In fact, \llm provides a unique opportunity to fully automate the plastic surgery hypothesis idea via fine-tuning (learning project-specific information via model updates from the buggy project) and prompting (directly providing relevant code ingredients to the model), and make it directly applicable to different languages (since the \llm{s} are typically multi-lingual).%
Moreover, despite the intensive manual efforts involved, traditional \apr tools still cannot fully leverage project-specific information due to large search space for leveraging/composing existing code ingredients. In contrast, the project-specific information can effectively leveraged by \llm{s} due to their power in code understanding/vectorization, e.g., even partial/imprecise information may still guide \llm{s} in correct patch generation!
 To this end, we ask the question: \emph{How useful is the plastic surgery hypothesis in the era of \plm{s}}?








\mypara{Our Work.} To answer the question, we present \ourtech{\xspace} -- a \llm-based approach that automatically utilizes the plastic surgery hypothesis by systematically combining multiple fine-tuning and prompting strategies for \apr. \ourtech fine-tunes \plm{s} using two novel domain-specific training strategies: \textbf{\epfinetune} -- we fine-tune using the original buggy project by aggressively masking out a high percentage of tokens, which allows \plm to learn project-specific code tokens and programming styles; and \textbf{\rofinetune} -- which only masks out a single continuous code sequence per training sample, allowing the model to get used to the final \csapr task of predicting a single continuous code sequence. Furthermore, we directly leverage the ability for \plm{s} to understand natural language instructions and introduce a novel prompting strategy, \textbf{\idprompting}, which uses information retrieval and static analysis to obtain a list of relevant identifiers for the buggy lines. While such relevant identifiers are critical for fixing some difficult bugs, they may not be seen by the \llm during inference due to limited context window size. Through the use of prompting, we directly tell the model to use these extracted identifiers (relevant code ingredients) to generate the correct code. Finally, to perform repair, we combine all four model variants (including the base model, both fine-tuned models and the base model with prompting) for the final repair.





While our insight of leveraging the plastic surgery hypothesis for \llm-based \apr is generalizable across different types of \plm{s}, to implement \ourtech, we choose a recent \plm{\xspace}, \ctfive~\cite{wang2021codet5}, which is pre-trained on millions of open-source code snippets. \ctfive is an encoder-decoder model trained using \mspfull (\msp) objective where a percentage of tokens are masked out and each continuous masked token sequence is referred to as a masked span. Also, although we only extract relevant identifiers from the current buggy project (since this paper focuses on the plastic surgery hypothesis), our work can be easily extended to obtain other code information (such as relevant statements or functions) from other sources, such as  the massive pre-training corpora~\cite{husain2020codesearchnet} or historical bug-fixing datasets~\cite{jiang2019infer}, which can provide more coding knowledge for \llm{s}. Besides, although we mainly focus on using traditional string comparison algorithms for information retrieval in this paper, these techniques can be easily replaced by other frequency-based retrieval~\cite{robertson2009probabilistic} and neural search (or embedding-based search)~\cite{reimers2019sentence}.
  In summary, this paper makes the following contributions:


%


\begin{itemize}[noitemsep, leftmargin=*, topsep=0pt]
    \item \textbf{Dimension.} This paper is the first to revisit the important plastic surgery hypothesis in the era of \llm{s}. It opens up a new dimension for \llm-based \apr to incorporate previously neglected information from the buggy project itself to boost \apr performance. Furthermore, it demonstrates the promising future of retrieval-based prompting for modern \llm-based \apr.
    \item \textbf{Implementation.} We implement \ourtech based on the recent \ctfive model. We augment the model using two novel fine-tuning strategies: \epfinetune and \rofinetune, along with a novel prompting strategy based on information retrieval and static analysis: \idprompting. We combine the patches generated by all four models together and perform patch ranking to speed up \apr.% 
    \item \textbf{Evaluation Study.} We conduct an extensive evaluation against state-of-the-art \apr tools. On the widely studied \dfj 1.2 and 2.0 datasets~\cite{just2014dfj}, \ourtech is able to achieve the new state-of-the-art results of 89 and 44 correct bug fixes (15 and 8 more than best baseline) respectively.  Furthermore, we perform a broad ablation study to justify our design. \ourtech demonstrates for the first time that the plastic surgery hypothesis can substantially boost \llm-based \apr and advance state-of-the-art \apr, while being fully automated and general. Moreover, even partial/imprecise code ingredients may still effectively guide \llm{s} for \apr!
\end{itemize}



% \clearpage
\section{CASE STUDY EXPERIMENTS AND RESULTS}
\label{results}
\section{Results}
\label{results}

\begin{figure*}[ht]
    \centering
    \includegraphics[scale=0.15,trim={0 2.5cm 0 5cm},clip]{images/aoi-single_burst}
    \caption{The time average peak Age of Information with burst and \gls{soa} loss values against the dynamic reliability logic for different network topologies.}
    \label{fig:aoi_burst}\vspace{-0.4cm}
\end{figure*}


This paper focuses on both transport layer and application layer metrics to determine the feasibility of dynamic reliability. For this, we have selected the session packet volume, as transmitted, retransmitted, lost and backlogged packets as \glspl{kpi} for the transport layer; while focusing on the \gls{aoi} for the application layer. The \gls{aoi} was chosen as a crucial indicator for the freshness of packets in real-time applications. More specifically, this work adopts the time average peak \gls{aoi} equation \cite{aoi_equation} depicted in Eq. \ref{aoi}, where $\Delta(r_{i+1})$ is the $i$th update at the time it was received at the server, for a session time period of $\tau$.

\begin{equation}
    \label{aoi}
    \gls{aoi}_\tau = \frac{1}{n-1}\sum_{i=1}^{n-1} \Delta(r_{i+1})
\end{equation}

We include a comparison between the vanilla QUIC implementation which does not enjoy the dynamic reliability extension, with a number of dynamic reliability policies. The tests were run a number of times for statistical significance, with the mean value of vanilla implementation used as a baseline for comparison. The topology utilised both random loss and bursty loss to explore the bounds of dynamic reliability. The \gls{soa} loss in the figures correspond to the loss values presented in Table. \ref{tab:path_char}, for ease of comparison between bursty and random loss scenarios.

\subsection{Transport-Layer KPIs}

To analyse the performance gain at the transport layer due to dynamic reliability, the volume of transmitted and backlogged packets is examined. The figures are in the form of boxplots, which take the vanilla implementation as a benchmark, depicted as the red dashed line.

As seen in Fig. \ref{fig:sent_burst}, the loss plays a crucial role in the performance of the reliability policies. The policies under random loss did incredibly well for the networks with a larger capacity, namely \gls{mmwave} and Sub-6~GHz, whereas for burst loss, the lower network capacities had a larger packet reduction. With the increase in burst loss, the behaviour of the set split reliable policies became unpredictable, if a reliable assignment happened to coincide with a burst loss, the number of transmitted packets increases, and vice versa. On the other hand, in smarter policies, such as Loss-Aware, the performance lightly matched the vanilla baseline, as the reliable assignment dominated the session to compensate for a higher burst loss. Not only that but, the burst loss also impacted the variance of the transmitted packets for the policies.

Unsurprisingly, the unreliable focused policy, 80-20 split, outperformed other policies for all topologies in random and bursty loss scenarios, with an approximate reduction of 80\%. That being said, the majority of the policies reduced the transmitted packets on the link by approximately 70\% for random loss, while the reduction started at $\approx 15\%$ and decreased as the loss increased for the burst loss scenario.

The retransmitted and lost packets, not shown due to space limitations, followed the same trend as the transmitted packets for the random loss scenarios. However, for the burst loss scenarios, the larger capacity networks had a lower reduction in the retransmitted and lost packets. This can be seen as a favorable outcome since the lower capacity networks are scarce on resources. It is important to note that the Loss-Aware policy mimicked the vanilla approach as the burst loss increased, signifying the overwhelming appointment of reliable packets in adapting to the harsh burst loss conditions.
 
Alternatively, Fig. \ref{fig:backlog_burst} clearly shows a stark comparison between the policies and loss scenario in the reduction of the backlogged packets. The Loss-Aware policy for random loss scenario reduced the backlogged packets by up to 50\%, beating all other policies by approximately 30\%. Furthermore, it is clear that the unreliability focused policies resulted in the lowest backlog for the session. In comparison, we notice that the burst loss and the backlogged frequency have a positive correlation, where the maximum reduction of the backlogged packets for the policies is at most 20\%. Much like the transmitted packets, the probability of a burst loss occurrence plays a vital role in the number of retransmissions sent and by extension the number of backlogged packets. Thus, we can conclude that the stress placed on the buffer is a result of the reliable packets which is tightly coupled with the congestion on the session. Whereas, unreliable focused policies did not encounter such a phenomenon regardless if it was experiencing a burst loss.


\subsection{Application-Layer KPIs}

The feasibility of dynamic reliability for real-time applications can be determined by the \gls{aoi}, with comparison across different topologies and policies. If we take a strict approach and consider anything below $10$~ms is real-time \cite{real-time}, then all the reliability policies passed that requirement, which is attractive for real-time applications, as shown in Fig. \ref{fig:aoi_burst}. Utilising the median as an estimate of the runs, the policies in the WLAN and Sub-6~GHz topology with random loss floated around $4-5$~ms with negligible difference, while the \gls{aoi} for \gls{mmwave} was $\approx 2-3$~ms. It is clear that the \gls{aoi} and the network capacity have a negative correlation, as the network capacity decreases, the \gls{aoi} increases. The same correlation is extended to the bursty loss scenarios, where \gls{mmwave} dominated the other topologies. That being said, it is crucial to note that the \gls{aoi} for the reliability policies is often slightly better than or equal to the \gls{aoi} of the vanilla implementation, proving that dynamic reliability reduces the congestion of the session at no cost to the \gls{aoi}.


% \clearpage
\section{CONCLUSION}
\label{conclusion}
\section{Conclusion}\label{sec:conclusion}
In this work, we focus on addressing the fundamental challenge of OOD detection tasks, which is how to fully understand the semantic discrepancy between the ID/OOD samples. We reveal that the key to success in the realistic SCOOD task is to allocate as many ID samples in the unlabeled set correctly as possible. To this end, we propose a novel uncertainty-aware optimal transport scheme that introduces class-specific energy scores as guidance for effective label assignment. Experimental results show that our method achieves better performance than previous state-of-the-art methods on SCOOD benchmarks.

\textbf{Limitations.} In addition to temperature scaling, other techniques such as feature clipping applied in ReAct~\cite{sun2021react} also enhance the performance of energy score, so how to obtain an OOD score that best fits the SCOOD task can be further explored. Moreover, a setting highly related to SCOOD has been proposed in \cite{katz2022training} and formulated as a constrained optimization problem. We will also theoretically analyze these practical OOD settings in our feature work.

% \section*{Acknowledgments}
\textbf{Acknowledgments.} 
This work is supported by National Key R\&D Program of China under Grant 2020AAA0105701, National Natural Science Foundation of China (NSFC) under Grants 61872327, Major Special Science and Technology Project of Anhui, National Natural Science Foundation of China (62033012) and Ant Group through Ant Research Intern Program.

%--------------
%\clearpage
% \begin{acknowledgment}
% \blindtext[1]
% \end{acknowledgment}

%--------------------------
% \clearpage
\renewcommand{\refname}{REFERENCES}
\bibliographystyle{config/asmems4}
\begin{mySmall}
% %\nocite{*} % remove later, displays all references
\bibliography{References}
\end{mySmall}
%\input{Ref.bbl}
%---------------
\appendix
\section{TOOLS AND COMPUTING ARCHITECTURE}
\label{appa}
\definecolor{olivegreen}{HTML}{2E6930}

\begin{table}[]
\tiny
\caption{Common tools employed for data extraction and graph construction based on static or dynamic analysis.}
\centering
\begin{tabular}{@{}l|p{0.35\textwidth}@{}}\toprule
\textbf{Representation} & \textbf{Tools} \\
\midrule
CFG                     & \textcolor{olivegreen}{Androguard} \cite{androguard}, radare2 \cite{radare2}, IDA Pro \cite{ida}, Ghidra \cite{ghidra}  \\ 
FCG                     & \textcolor{olivegreen}{Androguard}, \textcolor{olivegreen}{Apktool} \cite{apktool}, \textcolor{olivegreen}{graph4apk} \cite{graph4apk}, \textcolor{olivegreen}{WALA} \cite{wala}, Angr \cite{angr}, radare2, IDA Pro, cuckoo \cite{cuckoo}, Ghidra  \\ 
PDG                     & \textcolor{olivegreen}{Androguard}, Ghidra \\
Syscalls                & strace \cite{strace}, SystemTap \cite{systemtap}, ltrace \cite{systemtap} \\ 
System entities         & cuckoo, Any.Run \cite{anyrun} \\ 
Network flows           & Argus \cite{argus}, Zeek \cite{zeek}, Splunk \cite{splunk}, Joy \cite{joy}  \\ 
\bottomrule
\end{tabular}
\begin{tablenotes}
  \setlength\labelsep{0pt}
    \footnotesize
    \parbox{\textwidth}{Green refers to tools specifically designed for Android APKs.}
\end{tablenotes}
\label{table:tools}
\end{table}

\end{document}
