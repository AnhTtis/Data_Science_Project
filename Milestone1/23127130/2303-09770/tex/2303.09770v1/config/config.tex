
\newcommand{\bmmax}{2}

\usepackage{empheq, amssymb}

\usepackage{lmodern}
\usepackage{amsmath}
\usepackage{amsfonts}
\usepackage{amssymb}
% \usepackage{amsthm,thmtools}

\usepackage{bm} % bold for greek symbols

\usepackage{txfonts}

\usepackage[T1]{fontenc}
\usepackage[utf8]{inputenc} % accept different input encodings

\usepackage[dvipsnames]{xcolor} % defines colors

\usepackage[american]{babel} %  culturally-determined typographical rules

% \usepackage{amsthm}

\usepackage{graphicx}

\usepackage{epstopdf}

%\usepackage{booktabs}
%\usepackage{makecell}
% \usepackage{balance}
%\usepackage{float}

% \restylefloat{table}

\usepackage[noadjust]{cite} % grouped citations

\usepackage{ltxcmds}

% \usepackage{nicefrac}

% \usepackage[export]{adjustbox} % for image frames

\usepackage{mathtools}

% \usepackage{varwidth} % for specifying max widths

\usepackage{subcaption}% for subfigures

\usepackage{multirow} %for tables 

% \usepackage{relsize}

% \usepackage{tkz-linknodes}

% \usepackage{rotating}

\usepackage{blindtext}

% \usepackage[hyphens]{hyperref}
\usepackage{hyperref}

% \usepackage{tikz} % for drawings
% \usetikzlibrary{shapes.geometric, arrows, topaths, shapes,decorations.text, positioning,shadows,trees,shapes.arrows, fadings,arrows.meta,graphs,graphs.standard}


% \usetikzlibrary{external}
% \tikzexternalize[prefix=tikz/]


% \usepackage{algorithm2e} % environment for writing algorithms 

\usepackage{enumitem}

\usepackage{ifthen}

% \usepackage{blkarray}
% \usepackage{showframe}
% \usepackage{nomencl}

\usepackage{etoolbox}

\usepackage{cancel}

\usepackage{empheq}

\usepackage{fontawesome}

% \usepackage{titlesec}

\usepackage{soul}

% \usepackage{flushend}

\usepackage{flushend} % add at the end

% \usepackage{balance}

\usepackage{textcomp}

\usepackage{lipsum}

% \usepackage[thmmarks, amsmath, thref, hyperref, empheq]{ntheorem}


\DeclareMathOperator*{\argmax}{arg\,max}
\DeclareMathOperator*{\argmin}{arg\,min}
\DeclarePairedDelimiter\abs{\lvert}{\rvert}
\DeclarePairedDelimiter\norm{\lVert}{\rVert}
\DeclarePairedDelimiter\bracket{[}{]}
\DeclarePairedDelimiter\paren{(}{)}
\DeclarePairedDelimiter\curl{\lbrace}{\rbrace}
\DeclarePairedDelimiter\ceil{\lceil}{\rceil}
\DeclarePairedDelimiter\floor{\lfloor}{\rfloor}


% create font sizes
\newcommand{\mysmall}{\fontsize{8}{9}\selectfont}
\newcommand{\mySmall}{\fontsize{8.8}{9.5}\selectfont}

% colors
\definecolor{light-gray}{gray}{0.4}
\definecolor{box-gray}{gray}{1}

% editing commands
\newcommand{\xcolor}[1]{\textsl{\textsf{#1}}}
% \newcommand{\xrev}[1]{\textcolor{red}{#1}}
% \newcommand{\xneed}[1]{\textcolor{red}{#1}}
% \newcommand{\xwork}[1]{\textcolor{red}{#1}}
% \newcommand{\xchange}[1]{\textcolor{blue}{#1}}

%
\newcommand{\tran}{^{\mkern-1.5mu\mathsf{T}}}

% url style
\urlstyle{rm}

% hyperlink commands (doi, url, arxiv)
\newcommand{\xdoi}[1]{{doi: \href{https://doi.org/#1}{#1}}\rmFullStop}
\newcommand{\xurl}[1]{{url: \href{#1}{#1}}\rmFullStop}
\newcommand{\xarxiv}[1]{{arXiv:\href{https://arxiv.org/abs/#1}{#1}}\rmFullStop}

% hyperref
\hypersetup{
    unicode=false,          % non-Latin characters in Acrobat’s bookmarks
    pdftoolbar=true,        % show Acrobat’s toolbar?
    pdfmenubar=true,        % show Acrobat’s menu?
    pdffitwindow=false,     % window fit to page when opened
    pdfstartview={FitV},    % fits the width of the page to the window
    pdftitle={On the Use of Geometric Deep Learning Towards the Evaluation of Graph-Centric Engineering Systems},    % title
    pdfauthor={Anthony Sirico Jr. and Daniel R. Herber},     % author
    pdfsubject={},   % subject of the document
    pdfkeywords = {}, % list of keywords
    pdfnewwindow=true,      % links in new window
    colorlinks=true,
    allcolors=blue
}

\usepackage{nomencl}

\renewcommand\nomgroup[1]{%
  \item[\bfseries
  \ifstrequal{#1}{V}{ Variables}{%
  \ifstrequal{#1}{B}{ Subscripts}{%
  \ifstrequal{#1}{P}{ Notation}{%
  \ifstrequal{#1}{A}{ Acronyms}{}}}}]
}

\renewcommand*{\nompreamble}{\markboth{\nomname}{\nomname}}

\newcommand{\nomdescr}[1]{\parbox[t]{4cm}{\RaggedRight #1}}
\newcommand{\nomwithdim}[5]{\nomenclature[#1]{#2}%
{\nomdescr{#3}\DimensUnits{#4}{#5}}}

\renewcommand{\nomname}{Nomenclature}
\makenomenclature

% https://tex.stackexchange.com/questions/284313/how-do-i-tag-a-subequations-environment-as-a-whole
\makeatletter
\newenvironment{taggedsubequations}[1]
 {%
  % \end{subequations} will advance `equation`
  \addtocounter{equation}{-1}%
  \begin{subequations}%
  % set the current label
  \def\@currentlabel{#1}%
  % redefine \theequation
  \renewcommand{\theequation}{#1.\alph{equation}}%
 }
 {\end{subequations}}
\makeatother %not \makeatletter

\usepackage{dblfloatfix}
\usepackage{float}
% 

\definecolor{block-gray}{gray}{0.95}

% \usepackage{amsthm}
\usepackage[framemethod=TikZ]{mdframed}

\usepackage{xpatch}

\makeatletter
\xpatchcmd{\endmdframed}
  {\aftergroup\endmdf@trivlist\color@endgroup}
  {\endmdf@trivlist\color@endgroup\@doendpe}
  {}{}
\makeatother

%
\newcommand{\bz}{\ensuremath{\bm{z}}}
\newcommand{\bx}{\ensuremath{\bm{x}}}

\newcommand{\xa}{\ensuremath{^{\textcolor{black}{a}}}}
% \newcommand{\xa}{\ensuremath{^{\textcolor{black}{(a)}}}}
% \newcommand{\xa}{\ensuremath{^{\textcolor{black}{\circ}}}}
% \newcommand{\xa}{\ensuremath{^{\textcolor{black}{\Join}}}}
% 	\newcommand{\xa}{\ensuremath{^{\textcolor{black}{\boxplus}}}}

\newcommand{\asym}{\ensuremath{\boxplus}}

\newcommand{\xaz}{\ensuremath{_{\textcolor{black}{a_z}}}}

\newcommand{\eref}[1]{Eq.~\eqref{#1}}        % cite equation
\newcommand{\fref}[1]{Fig.~\ref{#1}}   % cite figure
\newcommand{\cref}[1]{Chap.~\ref{#1}}  % cite chapter
\newcommand{\sref}[1]{Sec.~\ref{#1}}  % cite section/sub(sub)section
\newcommand{\aref}[1]{App.~\ref{#1}} % cite appendix
\newcommand{\tref}[1]{Table~\ref{#1}}    % cite table
\newcommand{\rref}[1]{Ref.~\cite{#1}}

% \newcommand{\xneed}[1]{\textcolor{red}{#1}}

\newcommand{\xrev}[1]{\textcolor{red}{#1}}
\newcommand{\xdrh}[1]{\textcolor{red}{#1}}
\newcommand{\xz}[1]{\textcolor{lightgray}{#1}}
% \usepackage{showframe}

\usepackage{fontawesome}

\usepackage{titlesec}

\input{config/doiCmd.tex}


\usepackage{colortbl}

\definecolor{light-gray}{gray}{0.6}
\newcommand{\grayline}{\arrayrulecolor{light-gray}\hline\arrayrulecolor{black}}

\newcommand{\xsection}[1]{\section[#1]{\MakeUppercase{#1}}}

\usepackage{accents}

\newcommand{\pmin}{\underaccent{\bar}{\bm{P}}}
\newcommand{\pmax}{\bar{\bm{P}}}

\newcommand{\rmin}{\underaccent{\bar}{\bm{R}}}
\newcommand{\rmax}{\bar{\bm{R}}}

\newcommand{\tpmin}{\underaccent{\bar}{T}_p}
\newcommand{\tpmax}{\bar{T}_p}

\newcommand{\trmin}{\underaccent{\bar}{T}_r}
\newcommand{\trmax}{\bar{T}_r}

\newcommand{\xub}[1]{\underaccent{\bar}{#1}}
\newcommand{\xob}[1]{\bar{#1}}

\newcommand{\gline}{\arrayrulecolor{light-gray}\hline\arrayrulecolor{black}}

\definecolor{needcolor}{HTML}{C62828}
% \newcommand{\xneed}[1]{\textcolor{needcolor}{#1}}
% \newcommand{\xrev}[1]{\textcolor{red}{#1}}
% \newcommand{\xchecked}[1]{\textcolor{needcolor}{#1}}

\newtheorem{theorem}{Theorem}
\newtheorem{definition}{Definition}

\renewcommand{\qed}{\hfill\ensuremath{\blacksquare}}

\newcommand{\xtran}{\ensuremath{^\mathsf{T}}}
% \newcommand{\xtran}{\ensuremath{^\prime}}

\newcommand{\openFAST}{OpenFAST}

\providecommand{\keywords}[1]
{
  \small	
  \textbf{\textit{Keywords---}} #1
}

\newcommand\MyBox[1]{%
  \fbox{\parbox[c][1.5cm][c]{1.5cm}{\centering #1}}%
}

\newcommand\MyVBox[1]{%
  \parbox[c][1.7cm][c]{1cm}{\centering\bfseries #1}%
} 

\newcommand\MyHBox[2][\dimexpr1.5cm+2\fboxsep\relax]{%
  \parbox[c][1cm][c]{#1}{\centering\bfseries #2}%
} 
\newcommand\MyTBox[3]{%
  \MyVBox{#1}\MyBox{#2}\hspace*{-\fboxrule}%
  \MyBox{#3}\hspace*{-\fboxrule}%
  \par\vspace{-\fboxrule}
}  

% \newcommand\CM[6]{%
%     \begin{center}
%     \offinterlineskip
%     %\raisebox{-5cm}[0pt][0pt]{\rotatebox[origin=c]{90}{\parbox[c][0pt][c]{1cm}{\textbf{Source2}\\[20pt]}}}\par

%     \begin{figure}[!h]
%         \centering
%         \hspace*{1cm}\MyHBox{T}\MyHBox{F}\par
%         \MyTBox{P}{$T_P$\\#1}{$F_P$\\#2}
%         \MyTBox{N}{$F_N$\\#3}{$T_N$\\#4}
%         \caption{#5}
%         \label{#6}
%     \end{figure}
%     \end{center}
% }

\newcommand\subCM[6]{%
    \begin{subfigure}[b]{0.475\textwidth}
        \centering
        \hspace*{1cm}\MyHBox{T}\MyHBox{F}\par
        \MyTBox{P}{$T_P$\\#1}{$F_P$\\#2}
        \MyTBox{N}{$F_N$\\#3}{$T_N$\\#4}
        \caption{#5}
        \label{#6}
    \end{subfigure}
}


\usepackage{amsmath}
\usepackage{bm}
\usepackage{nicematrix}

\definecolor{poscolor}{HTML}{1e88e5}
\definecolor{negcolor}{HTML}{e53935}
\definecolor{offcolor}{HTML}{A93C93}

\newcommand{\CM}[6]{%
\begin{table}[t]
\centering
\caption{#5}
\label{#6}
\vspace{-0.15in}
\scalebox{0.8}{
\begingroup
\setlength{\tabcolsep}{6pt} % Default value: 6pt
\renewcommand{\arraystretch}{1.2} % Default value: 1
\begin{NiceTabular}{cccc}

& & \Block{1-2}{\textit{Data}} \\
 & & \Block[c,color=poscolor,draw=black,respect-arraystretch]{}{Actually\\ Positive (1)} & \Block[c,color=negcolor,draw=black,respect-arraystretch]{}{Actually\\ Negative (0)} \\
\Block{2-1}{\rotate \textit{Model}} & \Block[c,color=poscolor,draw=black,respect-arraystretch]{}{Predicted\\ Positive (1)} & \Block[c,color=poscolor,draw=black,fill=poscolor!5]{}{#1} & \Block[c,color=offcolor,draw=black,fill=offcolor!5]{}{#2} \\
& \Block[c,color=negcolor,draw=black,respect-arraystretch]{}{Predicted\\ Negative (0)} & \Block[c,color=offcolor,draw=black,fill=offcolor!5,respect-arraystretch]{}{#3} & \Block[c,color=negcolor,draw=black,fill=negcolor!5,respect-arraystretch]{}{#4} \\
\end{NiceTabular}%
\endgroup
}
\end{table}
}