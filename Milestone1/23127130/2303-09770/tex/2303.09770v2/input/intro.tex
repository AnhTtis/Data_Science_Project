\label{intro}
Mathematical graphs can be used to represent many systems and decisions because of their ability to capture discrete compositional and relational information. 
For decades, studies have employed different graph representations to capture their respective problems \cite{b5, b6, b37, b95, b96, b97, b102, b105}, and for well over a century, researchers have utilized graph enumeration to understand engineering design problems to help in decision making \cite{b3, b59, b6, b37, b38}. 
Today, many engineering design problems, including the construction of the ``system architecture'', are increasing in scope and complexity to a point where traditional discrete and continuous presentations are insufficient to represent the system \cite{b6}.

The system architecture is a conceptual model capturing the structure, behavior, rules, etc., of a product, process, or element \cite{b18} and is often the foundation on which it is designed, built, and operated. 
Many studies have concentrated on the effective representation of system architectures using graph theory \cite{b29, b91, b92}, where the goal is often a set of useful architectures that are feasible with respect to constraints.
Furthermore, one or more value metrics (e.g.,~performance or cost) might be determined for the different architectures so that the designer might sort through the candidates.
One method for generating all these options is graph enumeration, where a complete and ordered listing of the potential graphs is produced for some prescribed structure \cite{b3, b30, b31}.
% all the items in the collection of feasible and unique architectures \cite{b3, b30, b31}.
However, this approach can lead to an enormous amount of potential solutions depending on the problem at hand and its selected representation.
Paired with the increasing complexity of modern systems, the result is an exponential increase in computational costs making the decision-making process for the designer or system architect increasingly challenging.
% It can be extremely costly when synthesizing all generated candidate architectures.
These challenges drive the need for an approach that facilitates the decision-making for larger and more complex graph-centric design problems.

In this paper, we consider deep learning, specifically \textit{Geometric Deep Learning (GDL)}, as a potential strategy to address these issues.
Deep learning is defined as machine learning models composed of multiple processing layers capable of learning data representations with multiple levels of abstraction \cite{b21}, and
% What does this mean?
``deep'' refers to a larger number of hidden layers within the neural network. 
% For example, a typical machine learning model contains one to three hidden layers.
% You have your input layer, which is the starting point for your network, and it carries that information into the model for further processing.
% The hidden layers then calculate the weighted sum of the inputs and weights and execute an activation function.
% Deep neural networks have four or more hidden layers capable of learning features directly from the data without the need for manual intervention.
% One of the more popular types of deep learning networks is the Convolutional Neural Network (CNN), which uses layers that contain filters to help extract features for the input data, such as images.
Now, GDL is an umbrella term encompassing an emerging technique that generalizes neural networks to Euclidean and non-Euclidean domains, such as graphs, manifolds, meshes, or string representations \cite{b19}, and uses \textit{Graph Neural Networks (GNNs)}.
% We discuss this further in \sref{gdl} and \sref{gnn}.
In essence, GDL encompasses approaches that incorporate information on the input variables' structure space and symmetry properties and leverage it to improve the quality of the data captured by the model. 
GDL has immense potential and is widely used in other scientific communities, including molecular representations \cite{b20, b23, b24}, materials science \cite{b25}, architecture \cite{b28}, and the medical field \cite{b26, b27}.

Perhaps due to the lack of relevant graph-based design datasets, there is limited usage of GDL or other similar graph-based, machine-learning approaches in engineering design, despite the availability of the tools and documented success in other fields \cite{b101}.
Many studies utilize machine learning for the representation and selection of objects in 3D space, often in computer-aided design (CAD) applications \cite{b103, b104}.
However, there are some key differences between meshes and other graph representations, including size, structure, locality, and geometric interpretation (or lack thereof).
% The primary difference between meshes and graphs is that meshes are primarily used in the field of computer-aided design (CAD), while this study focuses more on the representation of systems from an architectural standpoint.
It is the intent of this study to provide another type of graph design example, and data set \cite{gdlgithub} illustrating how GDL can aid in the engineering design processes. %  by showing the statistical data from the case study used in this paper and providing access to the dataset for further studies.


% In this paper, we propose how GDL can help the decision-making process when applied to graph-centric design problems by effectively and efficiently reducing computational costs by learning from data structured like how humans perceive the world.

Here we are particularly motivated by graphic-centric design problems where generating many potential graphs is feasible, but determining the value or performance of each option is too expensive.
The proposed approach uses GDL to reduce the computational expense of classifying what might be ``good'' or ``bad'' options with a trade-off in classification accuracy.

The remainder of the paper is organized as follows:
Section~\ref{background} discusses the necessary background information required to have a basic understanding of GDL.
In \sref{method}, we go over the proposed approach, including graph classification, the machine learning model, and the metrics used to determine the performance of the models.
Section~\ref{case} discusses the case study on electric circuit frequency response
matching graph design, and \sref{results} describes several experiments conducted to explore GDL on this problem.
Lastly, we conclude in \sref{conclusion} with the final discussions and future work.






