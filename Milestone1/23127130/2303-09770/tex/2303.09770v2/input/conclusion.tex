In this paper, we have presented a Geometric Deep Learning (GDL) approach for classifying and down-selecting graph-based engineering design problems towards sets of better-performing solutions.
Using the electrical circuit graph-design case study where all the graphs were known but not their expensive optimization-based performance values, it was shown the potential capabilities of GDL in helping classify ``good'' and ``bad'' graphs based on limited performance data, as well as some insights into the key hyperparameters that need to be tuned.
Additionally, the inclusion of the graph-based features eigenvalue centrality and betweenness centrality improved many of the key metrics.

The results showed that GDL can be used as an effective and efficient method for narrowing the graphs sets for this case study as the model was able to achieve a total set classification accuracy of 80\% in using only 10\% of the graphs.
Furthermore, the iterative GDL classification approach identified 9.0 of the top 10,  88.2 of the top 100 graphs, and 751.2 out of the top 1000 at 25\% the computational cost of complete enumeration.

% an effective and efficient method for the down-select process toward graph-based engineering design problems in order to find the best realizable performing solution.
% In \sref{background}, we discussed the necessary background information that is required for one to become familiar with this new method.
% Then, in \sref{method}, we discussed what graph classification is, the model used to perform graph classification and the metrics that were used to determine how well the models performed.

% We then presented you with a case study that proved to be an excellent example to show the capabilities of Geometric Deep Learning.
% We showed how, if your data is limited, to add graph-based features to help each individual graph stand out more.
% We demonstrated how not only adding graph-based features saves time when the need for more data is needed, but how adjusting specific parameters within the model itself can help save computational costs.
% We showed GDL can efficiently train on a small fraction of the data, and still have high-performing results.
% Lastly, we demonstrated how GDL could effectively narrow down the graphs to the top 10\% with only a fraction of the given data.

Key future work items include iteratively adding new graphs to the dataset as mentioned in \sref{exp4r}, and investigating transfer learning on similar problems~\cite{b5}.
There is also the task of investigating this approach on other larger datasets in different engineering fields, such as ones with directed graphs and multiple performance metrics.
The promising results shown for this particular large dataset full of small engineering graphs indicate a promising future for iterative classification-based GDL for a wide application in engineering graph-design problems.



% will include taking GDL to a completely different problem within a different field of engineering.
% There, we will demonstrate how versatile GDL is by not only changing the problem basis, but also increasing the complexity of the graphs as well as using DIRECTED graphs as the input data.