% \subsection{Experiment 3: Decrease Batch-Size \& Epochs}
% \label{exp3}
% As mentioned earlier in this section, the goal is to keep a high level of accuracy while increasing efficiency.
% We attempt to answer the following two questions:
% \begin{enumerate}[nolistsep]
%     \item What if we could have achieved a higher accuracy level if the model had trained for more epochs?
%     \item What is the minimum training set size required to keep the high accuracy achieved in the \textit{3 feature} model so that we may decrease training time?
% \end{enumerate}

% Both previous experiments trained on 34,599 graphs for 170 epochs.
% To answer the first question, we will set the number of epochs to 1,000 and train on various random-sized training sets to see if there is a convergence at a specific epoch.
% If there is, that is the number of epochs that will be used for the remainder of the experiments in this case study.
% The reason behind this is to decrease the training time, thus increasing the efficiency of the study.

% To answer the second question, we will reduce the batch size by order of $50\%$ and train for the number of epochs found above.
% The expected outcome for this experiment is that as batch size decreases, the time it takes to train will decrease, thus increasing efficiency.
% But, there is going to be an expected give and take with this experiment.
% As we reduce the number of data points that the model can train on, we run the risk of decreasing model accuracy.
% So, the goal is to increase accuracy and find a compromise between sacrificing accuracy for efficiency.
% We begin the experiment with a training set size of $80\%$ of the data.
% The validation set size is $10\%$ of the size of the corresponding training set.
% Lastly, the test set size is simply the remainder of the data.
% Looking at \tref{exp3graphs}, we show the number of graphs per dataset.
% For experiment results, see \sref{exp3r}.

% \begin{table*}[t]
% \centering
% \caption{The batch sizes for experiment 3.}
% \label{exp3graphs}
% \begin{tabular}{c|c|c|c|c|c|c|c|c|c}
%     \hline \hline
%     \textbf{Dataset} & $80\%$ & $40\%$ & $20\%$ & $10\%$ & $5\%$ & $2.5\%$ & $1.25\%$ & $0.625\%$ & $0.3125\%$ \\
%     \hline
%     Training & 34,599 & 17,299 & 8,649 & 4,324 & 2,162 & 1,081 & 540 & 270 & 135\\
%     \hline
%     Validation & 3,459 & 1,729 & 846 & 432 & 216 & 108 & 54 & 27 & 13\\
%     \hline
%     Test & 5,191 & 24,221 & 33,736 & 38,493 & 40,871 & 42,060 & 42,655 & 42,952 & 43,101\\
%     \hline \hline
% \end{tabular}
% \end{table*}
