% In the following sections, we briefly review the enumeration of circuits followed by the problem description.
% Then, we discuss how the data is presented and prepared to be passed through the model.
% Lastly, we go over each experiment that is conducted on the dataset.

% To explore the capabilities of GDL in graph-based engineering design problems, we utilize the results of a frequency response matching study \cite{b5}. % , which was found to be an excellent problem to demonstrate the effectiveness of GDL considering the high demand for circuit designs within a specific frequency domain. 

% In this section, we discuss the design problem that is the focus of the case study and outline the experiments conducted to understand better how to train the model for this application effectively and how the proposed approach would be helpful to designers.

% \subsection{Case Study: Electric Circuit Frequency Response Matching Graph-design Problem}

To explore the capabilities of GDL in graph-based engineering design problems, we utilize the results of a frequency response matching study \cite{b5}.
The dataset and code to replicate these studies are available at Ref.~\cite{gdlgithub}.

% However, like many problems of this nature, it is both complex and expensive. 
% The successful synthesis of fully defined analog circuits involves decisions related to both the topology and sizing of the components, but the issue is that synthesis methods must navigate the enormous design space within these particular problems \cite{b47}, which leads researchers to be unable to find the best solutions efficiently. 
% As a result, most attempted solutions to these problems have aimed to navigate the immense design space by sampling from the complete design space.

\subsection{Enumeration of Circuit Graphs}

As discussed in \sref{enumeration}, we will be considering the graph enumeration of electric circuits from Ref.~\cite{b5}.
In particular, the set $\mathcal{G}$ with 43,249 undirected graphs includes all topologies that have up to 6 impedance subcircuits with $RC$ components and a required connection to the ground \cite{b5}.
Many researchers have utilized the enumeration technique of circuit problems \cite{b38, b48, b49} and in other problem areas where the data can be represented as graphs and other enumerable objects \cite{b51, b52, b53}. % , and the desired outcome is a list of all possible candidates.
As already mentioned, each circuit here is represented by a vertex-labeled graph, meaning every vertex has an associated label representing some circuit concept.
The size of each graph varied between 6 and 20 nodes, all of which had guaranteed different topologies (i.e., no labeled graph isomorphisms from \sref{gt}), which is relatively small for GDL.

% In the \sref{enumeration}, we discussed the enumeration process\footnote{To see this process in more detail, please refer to the original study in Ref. \cite{b5}.}.

% In this section, we go over how the enumeration process was applied to circuits.


% Each circuit component is then assigned a number of ports, or connections that component will have, and the upper bound on the number of graphs created is calculated by $(N-1)!!$, where $N$ is the total number of ports and $!!$ represents the double factorial.


% Now that we have discussed the generation of the circuits, it is time to apply our constraints, and in this case, the constraints are a specific frequency response shown in \eref{freq} over a specific frequency range seen in \eref{freqrange}.

\subsection{Frequency Response Matching Optimization}
The performance value $J(G_i)$ here is the error between the desired frequency response and the one a selected circuit provides, based originally on the study in \rref{b54}.
A circuit graph $G_i$ does not have an intrinsic value for $J$, rather it is a function of the \textit{tunable values} for the $RLC$ coefficients in the given graph and a selected \textit{optimization problem} with an objective and constraints.
From Ref.~\cite{b5}, we consider the ``set 1'' problem defined as:
\begin{subequations}\label{eq:cir-opt}
\begin{align}
\underset{\mathbf{z}_i = \{\mathbf{R}_i, \mathbf{C}_i\} }{\text{minimize:}} \quad & J = \sum_{k} \left( \log|H_i(j\omega_k,\mathbf{z}_i)| - \log|F(j\omega_k)|  \right)^2 \\
\text{subject to:} \quad & 10^{-2} \leq R_j \leq 10^{0} \quad \text{ for all $R_j$ in $G_i$} \\
& 10^{-2} \leq C_j \leq 10^{0} \quad \text{ for all $C_j$ in $G_i$} \\
\text{where:} \quad & |F(j \omega)| = \sqrt{\frac{2\pi}{10\omega}} \quad 0.2 \leq \frac{\omega}{2\pi}\leq 5
\end{align}
\end{subequations}

\noindent where $|F(j \omega)|$ is the desired frequency response, $H_i(j\omega_k,\mathbf{z}_i)$ is the frequency response for circuit graph $G_i$, $\mathbf{z}_i$ is the collection of optimization variables for the resistors $R$ and capacitors $C$ in graph $G_i$, and $\omega$ is sampled at 500 logarithmically-spaced evaluation points.

%
% Why this frequency response and range? 
% Because, according to Grimbleby in \rref{b54}, no formal design method exists for such a response. 
% These circuits were synthesized within the desired structure space, all topologies with up to 6 impedance sub-circuits and a required connection to the ground. 

Solving this nonlinear constrained least squares optimization problem can be expensive; the original study required over 8 hours to optimize each graph to determine $J(G_i)$.
Therefore, it is desirable to reduce the computational costs to discern good and bad circuit graphs.
% \footnote{Work is currently being done to replicate this study for a fairer computational comparison.}

% Finally, each graph had an associated label representing its “performance” $(P_i)$ ranging from approximately 

% The problem to be evaluated here is the evaluation of the generated circuits, where in\rref{b5} had taken over 8 hours to accomplish, totaling over 12 hours of computational cost.

The performance values here have a large range, between $4 \times 10^{-4}$ to $2 \times 10^{2}$.
Classification is done based on a sampled $\mathcal{G}_{known}$ using the median value of the known performance values, as discussed in \sref{sec:datasets}.
As we have the performance values for all graphs, this problem can serve an a good example to explore the potential effectiveness of GDL in these kinds of problems, so we can compare to classification using $\mathcal{G}_{all}$.

% PTMTorrrent
\newcommand{\numberOfModelHub}{5\xspace}

\newcommand{\TotalNumberOfPackages}{{15,913}\xspace}
% 12401 from Hugging Face
% 185 from ONNX
% 33 from Model Hub
% 3245 from Model Zoo
% 49 from PyTorch Hub
% SUM (by Nick): 15,913

\newcommand{\HFNumberOfPackages}{{12,401}\xspace}
\newcommand{\HFNumberOfPackagesMetadata}{{124,427}\xspace}
\newcommand{\MZNumberOfPackages}{3,245\xspace}
\newcommand{\PHNumberOfPackages}{{49}\xspace}
\newcommand{\MHNumberOfPackages}{{33}\xspace}
\newcommand{\ONNXNumberOfPackages}{{185}\xspace}

\newcommand{\TotalDataSize}{\textasciitilde{61TB}\xspace}
\newcommand{\HFDataSize}{{61TB}\xspace}
\newcommand{\MZDataSize}{{115GB}\xspace}
\newcommand{\PHDataSize}{{1.5GB}\xspace}
\newcommand{\MHDataSize}{{721MB}\xspace}
\newcommand{\ONNXDataSize}{{441MB}\xspace}
%%%



% ICSE submission - HFTorrent v1

\newcommand{\PTMDatasetNPackages}{63,182\xspace}
\newcommand{\PTMDatasetPercentage}{{99.7\%}\xspace}
\newcommand{\PTMDatasetFailedPackages}{{186}\xspace}
\newcommand{\PTMDatasetFailedPercentage}{{0.3\%}\xspace}

\newcommand{\PTMDatasetNReposWithSignedCommits}{{132}\xspace}
\newcommand{\PTMDatasetPercentOfSignedCommits}{{0.208\%}\xspace}


\newcommand{\PercentOfVerifiedOrgs}{{3.188\%}\xspace}
\newcommand{\NOrganizations}{{6,243}\xspace}
\newcommand{\NVerifedOrgs}{{199}\xspace}

\newcommand{\NOfRepositoriesWithMalware}{{1}\xspace}
\newcommand{\PercentageOfRepositoriesWithMalware}{{0.002\%}\xspace}
\newcommand{\TotalRepositoriesForMalwareScanning}{{63,366}\xspace}

% \subsection{Experiment 1: Establish a Baseline}
% \label{exp1}

% In the first experiment, we start by creating a baseline model for future comparisons.
% Here we consider a 80\% (34,599 graphs) in the training set $\mathcal{G}_{training}$, 2\% (865) in the validation set $\mathcal{G}_{validation}$, and the remaining 18\% (7,785) in  $\mathcal{G}_{unknown}$.
% These are by no means recommended distributions, but rather a nearly best case scenario that will be used to explore what is possible and what might be done to increase accuracy while maintaining efficiency.


% with the raw data to establish a baseline.
% Why would we need to establish a baseline?
% This creates a frame of reference so that we know early on how well the model trains on the data "as is" without further manipulation.
% We did an 80/20 split for the batch sizes: the model trained on 80\% of the data points, and we tested the model on the remaining 20\%.
% Since we were also using a validation dataset, 865 graphs were deducted from the training set and designated for validation.
% The remaining graphs were used as the "unseen" dataset to test the model's accuracy.
% See table \ref{exp1tab} for an exact breakdown of the datasets.

% \begin{table}[t]
% \centering
% \caption{The initial breakdown of the datasets to establish a baseline.}
% \label{exp1tab}
% \begin{tabular}{c|r}
%     \hline \hline
%     \textbf{Dataset} & \textbf{\# of Graphs}\\
%     \hline
%     Training & 34,599 \\
%     Validation & 865 \\
%     Test & 7,785 \\
%     \hline \hline
% \end{tabular}
% \end{table}

% Once the baseline, or frame of reference, is established, we can now determine what we need to do in order to increase accuracy while maintaining efficiency.
% For the results of this experiment, please see \sref{exp1r}.


% \subsection{Experiment 2: Addition of Graph-Based Features}
% The goal of this experiment is not only to increase the models' performance but also to maintain efficiency.
% What do we mean by efficiency?
% One way of increasing model accuracy is to train the model for more epochs, which takes more time while also running the risk of overfitting.
% Increasing efficiency means training for the same amount of epochs or less.
% To do that, we added graph-based features to each node: eigenvector centrality and betweenness centrality. 
% This is known as feature engineering and feature selection, where the features give a higher ability to explain the variance in the training data.

% \subsubsection{Eigenvector Centrality.}
% Eigenvector centrality computes a node's centrality based on its neighbors' centrality. 
% The eigenvector centrality for node $v$ is the $i$-th element of the vector $x$ defined by \eref{evc}, where $A$ is the adjacency matrix with eigenvalue $\lambda$ \cite{b60}.
% %
% \begin{equation}
%     Ax=\lambda x
%     \label{evc}
% \end{equation}

% \subsubsection{Betweenness Centrality.}
% Betweenness centrality of a node is the sum of the fraction of all-pairs shortest paths that pass through that node as shown in Eq.~\eqref{bc} where $V$ is the set of nodes, $\sigma(s,t)$ is the number of shortest paths, and $\sigma(s,t|v)$ is the number of those paths passing through some node $v$ other than $s,t$ \cite{b61}.
% %
% \begin{equation}
%     c_B(v) = \sum_{s,t\in V} \frac{\sigma(s,t|v)}{\sigma(s,t)}
%     \label{bc}
% \end{equation}

% With the additional features, the new features matrix $\textbf{X}$ for each individual graph has gone from $\textbf{X} \in \mathbb{R}^{n \times 1}$ to $\textbf{X} \in \mathbb{R}^{n \times 3}$.
% The expected outcome for this experiment is that we will be able to achieve higher scores from the metrics described in \sref{metrics}.

% \subsection{Experiment 3: Decrease Batch-Size \& Epochs}
% \label{exp3}
% As mentioned earlier in this section, the goal is to keep a high level of accuracy while increasing efficiency.
% We attempt to answer the following two questions:
% \begin{enumerate}[nolistsep]
%     \item What if we could have achieved a higher accuracy level if the model had trained for more epochs?
%     \item What is the minimum training set size required to keep the high accuracy achieved in the \textit{3 feature} model so that we may decrease training time?
% \end{enumerate}

% Both previous experiments trained on 34,599 graphs for 170 epochs.
% To answer the first question, we will set the number of epochs to 1,000 and train on various random-sized training sets to see if there is a convergence at a specific epoch.
% If there is, that is the number of epochs that will be used for the remainder of the experiments in this case study.
% The reason behind this is to decrease the training time, thus increasing the efficiency of the study.

% To answer the second question, we will reduce the batch size by order of $50\%$ and train for the number of epochs found above.
% The expected outcome for this experiment is that as batch size decreases, the time it takes to train will decrease, thus increasing efficiency.
% But, there is going to be an expected give and take with this experiment.
% As we reduce the number of data points that the model can train on, we run the risk of decreasing model accuracy.
% So, the goal is to increase accuracy and find a compromise between sacrificing accuracy for efficiency.
% We begin the experiment with a training set size of $80\%$ of the data.
% The validation set size is $10\%$ of the size of the corresponding training set.
% Lastly, the test set size is simply the remainder of the data.
% Looking at \tref{exp3graphs}, we show the number of graphs per dataset.
% For experiment results, see \sref{exp3r}.

% \begin{table*}[t]
% \centering
% \caption{The batch sizes for experiment 3.}
% \label{exp3graphs}
% \begin{tabular}{c|c|c|c|c|c|c|c|c|c}
%     \hline \hline
%     \textbf{Dataset} & $80\%$ & $40\%$ & $20\%$ & $10\%$ & $5\%$ & $2.5\%$ & $1.25\%$ & $0.625\%$ & $0.3125\%$ \\
%     \hline
%     Training & 34,599 & 17,299 & 8,649 & 4,324 & 2,162 & 1,081 & 540 & 270 & 135\\
%     \hline
%     Validation & 3,459 & 1,729 & 846 & 432 & 216 & 108 & 54 & 27 & 13\\
%     \hline
%     Test & 5,191 & 24,221 & 33,736 & 38,493 & 40,871 & 42,060 & 42,655 & 42,952 & 43,101\\
%     \hline \hline
% \end{tabular}
% \end{table*}


% \subsection{Experiment 4: }
% \label{exp4}

% In this experiment, the goal is to see if the GDL model can narrow the dataset down to the actual top 10\% of the dataset through multiple iterations.
% Here, we take the original dataset and split it into the KNOWN and UNKNOWN sets, and once again, use the performance median of the KNOWN set as the threshold for the class splitting of ones and zeros.

% We then take the KNOWN dataset and using an 80/20 split, divide the set into training and validation sets to train the first model, which is then used to predict the classes of the UNKNOWN dataset.
% Once the predictions are completed, we determine which circuits were predicted as zero and remove them under the assumption that those circuits were predicted to be "bad."
% We also remove the classified zeros from the known set since we already have classified them as "bad."
% Now, we are left with our original KNOWN set of "good" graphs, and the newly predicted ones from the UNKNOWN set, and we concatenate these two sets to create a new KNOWN2 set.
% Next, we train a new model on the KNOWN2 dataset using the KNOWN2's performance median as the new class threshold.

% The expected outcome of this experiment is that after each iteration of training on the newly formed KNOWN dataset, the class threshold will continue to reduce to reveal the top candidates for the best realizable performing solution.

% In the three previous experiments, the number of ones and zeros was split 50/50.
% In this experiment, after the select number of graphs was synthesized, the result was an imbalance between the two classes, reflecting closer to a real-world scenario where the user does not know the outcome of the performance values.
% The intended purpose of this is to show that GDL still has a high level of accuracy, even when performed in a case of this nature.

% There are two approaches to this experiment, both of which the user pre-selects some graphs to synthesize; in this case, we chose 50\% of the data and 10\% of the data to synthesize.

% \subsubsection*{The first approach: Create a 50/50 Split Training Set}
% After the data is synthesized, you end up with your ones ($n_1$) and zeros ($n_0$) where $n_0 \gg n_1$.
% It is now time to split the data into the training and validation sets, and in this approach, we take a percentage of the ones for training ($p_{training}$) where $n_{1,training} = n_1 p_{training}$, and then use the same number of zeros where $n_{0,training} \subset n_0(n_{1,training})$.
% This would create a training set $N_{training} = n_{1,training} \cup n_{0,training}$ that has a 50/50 split between classes, and the remainder of the classes would be placed into the validation set $N_{validation} = (n_1 - n_{1,training}) \cup (n_0 - n_{0,training})$.
% The purpose of this approach is to mimic the previous experiments, where there is an equal number in each class.
% The downside to this approach is that the remaining graphs that the model would later predict may have a much more significant class imbalance

% \subsubsection*{The second approach: Shuffle the Data} 
% In a similar situation, after synthesizing the data, you end up with an imbalance of classes where $n_0 \gg n_1$, except this time, we shuffle the data and split the shuffled result into training and validation sets.
% The hypothesis behind this method is: having fewer ones within the training set can help those "top performing" graphs stand out more among the zeros, making them more distinguishable.
% The downside to using this approach, one could end up with $N_{training}$ having little to no ones depending on the size of the synthesized dataset.

