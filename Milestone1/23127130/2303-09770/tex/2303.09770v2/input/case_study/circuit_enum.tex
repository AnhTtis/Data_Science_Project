\subsection{Enumeration of Circuit Graphs}

As discussed in \sref{enumeration}, we will be considering the graph enumeration of electric circuits from Ref.~\cite{b5}.
In particular, the set $\mathcal{G}$ with 43,249 undirected graphs includes all topologies that have up to 6 impedance subcircuits with $RC$ components and a required connection to the ground \cite{b5}.
Many researchers have utilized the enumeration technique of circuit problems \cite{b38, b48, b49} and in other problem areas where the data can be represented as graphs and other enumerable objects \cite{b51, b52, b53}. % , and the desired outcome is a list of all possible candidates.
As already mentioned, each circuit here is represented by a vertex-labeled graph, meaning every vertex has an associated label representing some circuit concept.
The size of each graph varied between 6 and 20 nodes, all of which had guaranteed different topologies (i.e., no labeled graph isomorphisms from \sref{gt}), which is relatively small for GDL.

% In the \sref{enumeration}, we discussed the enumeration process\footnote{To see this process in more detail, please refer to the original study in Ref. \cite{b5}.}.

% In this section, we go over how the enumeration process was applied to circuits.


% Each circuit component is then assigned a number of ports, or connections that component will have, and the upper bound on the number of graphs created is calculated by $(N-1)!!$, where $N$ is the total number of ports and $!!$ represents the double factorial.