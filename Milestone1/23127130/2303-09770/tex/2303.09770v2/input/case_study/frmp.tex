
% Now that we have discussed the generation of the circuits, it is time to apply our constraints, and in this case, the constraints are a specific frequency response shown in \eref{freq} over a specific frequency range seen in \eref{freqrange}.

\subsection{Frequency Response Matching Optimization}
The performance value $J(G_i)$ here is the error between the desired frequency response and the one a selected circuit provides, based originally on the study in \rref{b54}.
A circuit graph $G_i$ does not have an intrinsic value for $J$, rather it is a function of the \textit{tunable values} for the $RLC$ coefficients in the given graph and a selected \textit{optimization problem} with an objective and constraints.
From Ref.~\cite{b5}, we consider the ``set 1'' problem defined as:
\begin{subequations}\label{eq:cir-opt}
\begin{align}
\underset{\mathbf{z}_i = \{\mathbf{R}_i, \mathbf{C}_i\} }{\text{minimize:}} \quad & J = \sum_{k} \left( \log|H_i(j\omega_k,\mathbf{z}_i)| - \log|F(j\omega_k)|  \right)^2 \\
\text{subject to:} \quad & 10^{-2} \leq R_j \leq 10^{0} \quad \text{ for all $R_j$ in $G_i$} \\
& 10^{-2} \leq C_j \leq 10^{0} \quad \text{ for all $C_j$ in $G_i$} \\
\text{where:} \quad & |F(j \omega)| = \sqrt{\frac{2\pi}{10\omega}} \quad 0.2 \leq \frac{\omega}{2\pi}\leq 5
\end{align}
\end{subequations}

\noindent where $|F(j \omega)|$ is the desired frequency response, $H_i(j\omega_k,\mathbf{z}_i)$ is the frequency response for circuit graph $G_i$, $\mathbf{z}_i$ is the collection of optimization variables for the resistors $R$ and capacitors $C$ in graph $G_i$, and $\omega$ is sampled at 500 logarithmically-spaced evaluation points.

%
% Why this frequency response and range? 
% Because, according to Grimbleby in \rref{b54}, no formal design method exists for such a response. 
% These circuits were synthesized within the desired structure space, all topologies with up to 6 impedance sub-circuits and a required connection to the ground. 

Solving this nonlinear constrained least squares optimization problem can be expensive; the original study required over 8 hours to optimize each graph to determine $J(G_i)$.
Therefore, it is desirable to reduce the computational costs to discern good and bad circuit graphs.
% \footnote{Work is currently being done to replicate this study for a fairer computational comparison.}

% Finally, each graph had an associated label representing its “performance” $(P_i)$ ranging from approximately 

% The problem to be evaluated here is the evaluation of the generated circuits, where in\rref{b5} had taken over 8 hours to accomplish, totaling over 12 hours of computational cost.

The performance values here have a large range, between $4 \times 10^{-4}$ to $2 \times 10^{2}$.
Classification is done based on a sampled $\mathcal{G}_{known}$ using the median value of the known performance values, as discussed in \sref{sec:datasets}.
As we have the performance values for all graphs, this problem can serve an a good example to explore the potential effectiveness of GDL in these kinds of problems, so we can compare to classification using $\mathcal{G}_{all}$.