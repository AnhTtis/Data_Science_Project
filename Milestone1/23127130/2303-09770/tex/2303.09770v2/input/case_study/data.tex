% \subsection{The Circuit Data}
% In \aref{appa}, we discussed how the data was presented and how we went from graph data to data capable of passing through a GNN.
% In this section, we will go over the process as it pertains to circuits.
% The original data was 43,249 circuits represented as undirected graphs.
% The size of each graph varied between 6 and 20 nodes, all of which had guaranteed different topologies.
% In addition, each node had its label that represented a component within the circuit. 



% If we look at \fref{perc}, we can see approximately how many graphs are in each percentile group.
% The top $20\%$ of graphs start at $P_{8649} = 0.0138$ and end at $P_{1} = 4.88 \times 10^{5}$.
% While the bottom $20\%$ start at $P_{34599} = 9.367$ and end at $P_{43249} = 145.44$.
 
% For comparison purposes, the generation of these circuit topologies as graphs took approximately 4 hours\footnote{Performed on a single workstation with an i7-6800K at 3.8 GHz, 32 GB DDR4 3200 MHz RAM, Windows 10 64-bit, and MATLAB 2017a}. 
 


% Since the original study was to find a correlation between performance and circuit complexity (number of impedance components) and determine which topologies were the best, we needed to assign the graphs their binary values based on whether the graph was good or bad, as discussed in \sref{gclass}.
% To do that, we needed to determine the splitting index to assign the graph label.
% For the first three experiments to be conducted, it was decided to split the data down the middle and assign the top 50\% with a one and the remaining with a zero.
% The reason for this is to have a balance in classes, where an imbalance can produce a model that resulted in high accuracy, but in fact is just relying on the more frequent class, as discussed in \sref{metrics}.
% We conduct an experiment with class imbalances in experiment 4.

% After passing the data through the script, the outcome is one dataset of PyG instances, where the top 50\% are assigned ones and the remaining zeros.
%

%
% \fref{perc} shows an example of a 10\% portion of the data would be divided into its classes.
% In this graphic, there are a total of 432 graphs where the blue lines represent the selected ones, and the red represents the zeros.
% The green lines represent the where and at what values the data was divided.

% In the following sections, we discuss the experiments conducted on this dataset in detail.
