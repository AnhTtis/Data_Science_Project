This paper presents a Geometric Deep Learning (GDL) approach for classifying and down-selecting graph-based analog circuits towards sets of better-performing solutions based on their performance evaluation in an engineering design problem.
% \st{engineering design problems towards sets of better-performing solutions}.
% \st{Using the electrical circuit} 
In the presented graph-design case study where all the graphs were known but not their expensive optimization-based performance values, we have demonstrated the potential capabilities of GDL in helping classify ``good'' and ``bad'' graphs based on limited performance data, as well as some insights into the key hyperparameters that need to be tuned.
Additionally, including the graph-based features eigenvalue centrality and betweenness centrality improved many of the key metrics.

The results showed that GDL can be used as an effective and efficient method for narrowing the graph sets for this case study, as the model achieved a total set classification accuracy of 80\% % in using only 10\% % of the graphs.
Furthermore, the iterative GDL classification approach identified 9.0 of the top 10,  88.2 of the top 100 graphs, and 751.2 out of the top 1000 at 25\% the computational cost of complete enumeration.

Key future work items include investigating iteratively adding new graphs to the dataset as mentioned in \sref{exp4r}, alternative approaches to predict the classes of $\mathcal{G}_{unknown}$ using the trained GDL model (changing what graphs would be removed at each iteration), and transfer learning on similar problems~\cite{b5}.
There is also the task of investigating this approach on other datasets in different engineering fields, such as ones with directed graphs and multiple performance metrics.
% \xrev{the application of this approach towards another complex design problem that would now involve multiple objectives and \textit{directed} graphs.
As is the case with many GNN-based approaches, there is still work to be done to explore the best neural network architecture and hyperparameters to reduce the data input requirements and improve model accuracy.
Furthermore, similar to Sec.~\ref{exp2r}, adding other graph-based features could enable more favorable outcomes (e.g.,~current-flow closeness centrality, closeness vitality, and harmonic centrality \cite{Koschutzki2005, boldi2013axioms}).
The promising results for this large dataset of small engineering graphs indicate a bright future for iterative classification-based GDL for a broader application in other engineering graph-design problems.
