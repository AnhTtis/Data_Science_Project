\label{intro}
Mathematical graphs can be used to represent many systems and decisions because of their ability to capture discrete compositional and relational information. 
For decades, studies have employed different graph representations to capture their respective problems \cite{b5, b6, b37, b95, b96, b97, b102, b105}, and for well over a century, researchers have utilized graph enumeration to understand engineering design problems to help in decision making \cite{b3, b59, b6, b37, b38}. 
Today, many engineering design problems, including the construction of the ``system architecture'', are increasing in scope and complexity to a point where traditional discrete and continuous presentations are insufficient to represent the system \cite{b6}.

The system architecture is a conceptual model capturing the structure, behavior, rules, etc., of a product, process, or element \cite{b18} and is often the foundation on which it is designed, built, and operated. 
Many studies have concentrated on the effective representation of system architectures using graph theory \cite{b29, b91, b92}, where the goal is often a set of useful architectures that are feasible with respect to constraints.
Furthermore, one or more value metrics (e.g.,~performance or cost) might be determined for the alternative architectures so that the designer might sort through the candidates.
One method for generating all these options is graph enumeration, where a complete and ordered listing of the potential graphs is produced for some prescribed structure \cite{b3, b30, b31}.
However, this approach can lead to an enormous amount of potential solutions depending on the problem at hand and a selected representation.
Paired with the increasing complexity of modern systems, the result is an exponential increase in computational costs making the decision-making process for the designer or system architect increasingly challenging.
These challenges drive the need for an approach that facilitates the decision-making for larger and more complex graph-centric design problems.

In this paper, we consider deep learning, specifically \textit{Geometric Deep Learning (GDL)}, as a potential strategy to address these issues.
Deep learning is defined as machine learning models composed of multiple processing layers capable of learning data representations with multiple levels of abstraction \cite{b21}, and
``deep'' refers to a larger number of hidden layers within the neural network. 
Now, GDL is an umbrella term encompassing an emerging technique that generalizes neural networks to Euclidean and non-Euclidean domains, such as graphs, manifolds, meshes, or string representations \cite{b19}, and uses \textit{Graph Neural Networks (GNNs)}.
In essence, GDL encompasses approaches that incorporate information on the input variables' structure space and symmetry properties and leverage it to improve the quality of the data captured by the model. 
GDL has immense potential and is widely used in other scientific communities, including molecular representations \cite{b20, b23, b24}, materials science \cite{b25}, architecture \cite{b28}, and the medical field \cite{b26, b27}.
Within engineering design, there have been applications of GNNs to airfoil design \cite{b106}, structural mechanics using mesh-based physics simulations \cite{b107}, wind-farm
power estimation \cite{Park2019a}, distributed circuit design \cite{b110}, and decision-making processes involved with shared mobility systems \cite{b108}.
Additionally, GNNs have aided in component function classification by training on assembly and flow relationships \cite{b109}.

Most of these engineering problems presented do not have extensive datasets, and perhaps due to this lack of relevant graph-based design datasets, there is limited usage of GDL or other similar graph-based, machine-learning approaches in the engineering design down-select process, despite the availability of the tools and documented success in other fields \cite{b101}.
Many studies utilize machine learning for the representation and selection of objects in 3D space, often in computer-aided design (CAD) applications \cite{b103, b104}.
However, there are some key differences between meshes and other graph representations, including size, structure, locality, and geometric interpretation (or lack thereof).
It is the intent of this study to provide another type of graph design example, and data set \cite{gdlgithub} illustrating how GDL can aid in the engineering design processes. 

Here we are particularly motivated by graph-centric design problems where generating many potential graphs is feasible, but determining the value or performance of each option is too expensive.
Previous work has shown promising results on the specific dataset used in this article but required a significant training set size and time \cite{Guo2019a}.
Here, the proposed approach uses GDL to reduce the computational expense of classifying what might be ``good'' or ``bad'' options with a trade-off in classification accuracy, utilizing much smaller training sets.

The remainder of the paper is organized as follows:
Section~\ref{background} discusses the necessary background for graph design and GDL.
In \sref{method}, we go over the proposed approach, including graph classification, the machine learning model, and the metrics used to determine the performance of the models.
Section~\ref{case} discusses the case study on electric circuit frequency response
matching graph design, and \sref{results} describes several experiments conducted to explore GDL on this problem.
Lastly, we conclude in \sref{conclusion} with the final discussions and future work.