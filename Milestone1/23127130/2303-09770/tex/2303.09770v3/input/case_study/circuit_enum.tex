\subsection{Enumeration of Circuit Graphs}

Automating the design synthesis of analog circuits has long been a problem of interest with numerous attempts \cite{501647, 4208908, grimbleby}.
Still, these approaches can often fail to meet the needs when encountering unfamiliar and complex design problems.
Even with the addition of heuristics \cite{1083985} and knowledge bases \cite{44506}, determining global solutions, sets of alternatives, and patterns is not straightforward, which leads us to consider enumeration.


As discussed in \sref{enumeration}, we will consider the graph enumeration of electric circuits from Ref.~\cite{b5}.
In particular, the set $\mathcal{G}$ with 43,249 undirected graphs includes all topologies with up to 6 impedance subcircuits with $RC$ components and a required connection to the ground \cite{b5}.
Two examples are shown in Fig.~\ref{fig:pvc} with $RC$ values optimally selected for the design problem presented in the next section.

Many researchers have utilized the enumeration technique of circuit problems \cite{b38, b48, b49} and in other problem areas where the data can be represented as graphs and other enumerable objects \cite{b51, b52, b53}. 
As mentioned, each circuit here is represented by a vertex-labeled graph, meaning every vertex has an associated label representing some circuit concept.
The size of each graph varied between 6 and 20 nodes, all of which had guaranteed different topologies (i.e., no labeled graph isomorphisms from \sref{gt}), which is relatively small for GDL.
