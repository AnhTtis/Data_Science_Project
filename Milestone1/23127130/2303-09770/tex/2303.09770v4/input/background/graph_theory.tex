\subsection{Graph Theory}
\label{gt}
A \textit{graph} $G$ is a pair of sets $(V,E)$ where $E \subseteq [V]^2$ (i.e., $E$ is a two-element subset of $V$). $E$ represents the edges of the graph, while $V$ is the set of its vertices or nodes \cite{b7}.
A graph has various properties.
For example, a graph's order, denoted by $n = |G|$, is the number of vertices in the graph.
Each of those vertices has an associated degree value, which is the number of neighbors or edges to a vertex and is denoted by $d_G(v) = d(v)$.

While there are different ways to represent a graph mathematically, here we focus on one known as the \textit{adjacency matrix} $\textbf{A} = (a_{i,j})_{n \times n}$ and is defined as:
\begin{align}
a_{i,j} \coloneqq 
\begin{cases}
1, & \text{ if } (v_i,v_j) \in E \\
0, & \text{ otherwise}
\end{cases}
\label{adjeq}
\end{align}

\noindent For simple, undirected graphs, the matrix is symmetric and the diagonal of $\textbf{A}$ will be all zeros, which indicates that the graph has no self-loops.

Another important concept is \textit{graph isomorphism}.
Say we have two graphs, $G_1 = (V_1,E_1)$ and $G_2 = (V_2,E_2)$.
These two graphs are isomorphic, denoted $G1 \cong G_2$, if there is a bijection, $\varphi$, from $V_1 \rightarrow V_2$ such that $(v_i,v_j) \in E_1 \leftrightarrow (\varphi(v_i),\varphi(v_j))\in E_2$ for all $v_i,v_j \in V$ \cite{b78}.
We also consider a feature or \textit{labeled graph isomorphism}.
Here we consider the same two graphs from the previous definition, but they contain a third feature.
Specifically, we will consider this feature as a vertex label, denoted $X$, where $G_1 = (V_1,E_1,X_1)$ and $G_2 = (V_2,E_2,X_2)$.
The graphs will be isomorphic as long as the vertex label property is preserved under some valid bijection $\varphi$.
More generally, a features matrix $\textbf{X} \in \mathbb{R}^{n \times c}$ can have as many columns as necessary to represent the $c$ features associated with each vertex.

\subsubsection{Representing Electric Circuits as Graphs.}

\begin{figure}[t]
\centering
\includegraphics[scale=0.95]{input/figures/circuit.pdf}
\caption{Electrical circuit schematic represented as a vertex-labeled graph.}
\label{circgraph}
\end{figure}

The concepts from the previous section can be used to represent engineering systems.
The particular type of system used in the case study here is electric $RLC$ circuits, so they will be used to illustrate their representation as a graph.
Looking at the left side of \fref{circgraph}, we can see a basic $RLC$ circuit schematic.
Then on the right side, we have the same circuit represented pictorially as a graph.
Each vertex in the graph is now labeled as the corresponding component where $I$ and $O$ represent the input and output nodes, $R$ is a resistor, $L$ is an inductor, $C$ is a capacitor, and lastly, $N$ represents a voltage node with constant voltage. Also, other graph representations of the same circuit are possible.

For the graph in \fref{circgraph}, an adjacency matrix $\mathbf{A}$ representing its structure and vertex list $\mathbf{V}$ representing the vertex labels are:
\begin{align}
\begingroup % keep the change local
\setlength\arraycolsep{3pt}
\setcounter{MaxMatrixCols}{20}
\textbf{A} = \begin{bmatrix}  
\xz{0} & 1 & \xz{0} & \xz{0} & \xz{0} & \xz{0} & \xz{0} & \xz{0} & \xz{0} & \xz{0} & \xz{0} & \xz{0} & \xz{0}\\ 1 & \xz{0} & 1 & \xz{0} & \xz{0} & \xz{0} & \xz{0} & \xz{0} & \xz{0} & \xz{0} & \xz{0} & \xz{0} & \xz{0}\\ \xz{0} & 1 & \xz{0} & 1 & \xz{0} & \xz{0} & \xz{0} & \xz{0} & \xz{0} & \xz{0} & \xz{0} & \xz{0} & \xz{0}\\ \xz{0} & \xz{0} & 1 & \xz{0} & 1 & \xz{0} & \xz{0} & \xz{0} & 1 & 1 & \xz{0} & \xz{0} & \xz{0}\\ \xz{0} & \xz{0} & \xz{0} & 1 & \xz{0} & 1 & \xz{0} & \xz{0} & \xz{0} & \xz{0} & \xz{0} & \xz{0} & \xz{0}\\ \xz{0} & \xz{0} & \xz{0} & \xz{0} & 1 & \xz{0} & 1 & \xz{0} & \xz{0} & \xz{0} & \xz{0} & \xz{0} & \xz{0}\\ \xz{0} & \xz{0} & \xz{0} & \xz{0} & \xz{0} & 1 & \xz{0} & 1 & 1 & \xz{0} & \xz{0} & \xz{0} & \xz{0}\\ \xz{0} & \xz{0} & \xz{0} & \xz{0} & \xz{0} & \xz{0} & 1 & \xz{0} & \xz{0} & \xz{0} & \xz{0} & \xz{0} & \xz{0}\\ \xz{0} & \xz{0} & \xz{0} & 1 & \xz{0} & \xz{0} & 1 & \xz{0} & \xz{0} & \xz{0} & \xz{0} & \xz{0} & \xz{0}\\ \xz{0} & \xz{0} & \xz{0} & 1 & \xz{0} & \xz{0} & \xz{0} & \xz{0} & \xz{0} & \xz{0} & 1 & \xz{0} & \xz{0}\\ \xz{0} & \xz{0} & \xz{0} & \xz{0} & \xz{0} & \xz{0} & \xz{0} & \xz{0} & \xz{0} & 1 & \xz{0} & 1 & \xz{0}\\ \xz{0} & \xz{0} & \xz{0} & \xz{0} & \xz{0} & \xz{0} & \xz{0} & \xz{0} & \xz{0} & \xz{0} & 1 & \xz{0} & 1\\ \xz{0} & \xz{0} & \xz{0} & \xz{0} & \xz{0} & \xz{0} & \xz{0} & \xz{0} & \xz{0} & \xz{0} & \xz{0} & 1 & \xz{0} \end{bmatrix}
\qquad 
\textbf{X} = \begin{bmatrix} I\\R\\L\\N\\C\\R\\L\\N\\O\\R\\L\\C\\G\end{bmatrix}
\label{adjmat}
\endgroup
\end{align}

\noindent where $1$ represents a connection between two vertices.
