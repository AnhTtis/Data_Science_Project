\begin{table}[tb]
\centering
%\vspace{-2mm}
\scalebox{0.95}{

\begin{tabular}{c|c|cccc}
\toprule[1pt]
\multicolumn{2}{c|}{} & \multicolumn{4}{c}{$T^{max}$} \\
\cline{3-6}
 \multicolumn{2}{c|}{} &  0.7 & 0.6 & 0.5 & 0.4\\
 \hline
\multirow{4}{*}{$T^{min}$} & 0.4 & 38.3/39.5 & 38.0/39.5 & 38.0/39.5 & 37.9/39.3 \\
 & 0.3 & 38.8/39.4 & 38.7/39.4 & 38.5/39.4 & 38.6/39.5 \\
 & 0.2 & 38.9/39.3 & 39.0/39.3 & 38.7/39.0 & 38.4/38.8 \\
 & 0.1 & 38.8/39.0 & 38.9/39.1 & 38.6/38.8 & 38.3/38.3 \\
\bottomrule[1pt]
\end{tabular}
}
\caption{The object detection results on COCO \texttt{val} set with different configurations of $T^{max}$ and $T^{min}$. The values before and after ‘/’ denote the results without and with NMS in inference. As shown in the table, a small $T^{min}$ is helpful to narrow down the performance gap between end-to-end detection (without NMS) and non end-to-end detection (with NMS). A large $T^{max}$ can improve the overall performance without NMS. We set $T^{max}$ and $T^{min}$ to 0.6 and 0.2, respectively, in the rest experiments.}
\label{table3}
\end{table}