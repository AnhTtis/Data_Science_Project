\documentclass{WileyMSP-template}
\usepackage{amssymb}
\usepackage{braket}
\usepackage[super,square,compress,sort]{natbib}
\usepackage{setspace}
\onehalfspacing
\usepackage{amsmath}
\usepackage{graphicx}
\usepackage{epstopdf}

\setcounter{equation}{0}
\setcounter{figure}{0}
\setcounter{table}{0}
\setcounter{page}{1}
\makeatletter
\renewcommand{\theequation}{S\arabic{equation}}
\renewcommand{\thefigure}{S\arabic{figure}}
\renewcommand{\bibnumfmt}[1]{[S#1]}
\renewcommand{\citenumfont}[1]{S#1}



\def  \bsig   {\mbox{\boldmath$\sigma $}}
\def  \bnu    {\mbox{\boldmath$\nu $}}
\def  \btau   {\mbox{\boldmath$\tau $}}
\def  \bbet   {\mbox{\boldmath$\beta $}}
\def  \bxi   {\mbox{\boldmath$\xi $}}
\def  \bmu   {\mbox{\boldmath$\mu $}}
\def  \bmatA   {\mbox{\boldmath$\mathcal{A}$}}
\def  \bmatB   {\mbox{\boldmath$\mathcal{B}$}}



\begin{document}
%\pagestyle{fancy}
%\rhead{\includegraphics[width=2.5cm]{vch-logo.png}}

\title{Supplementary materials: Topological monopole's gauge field induced anomalous Hall effect in artificial honeycomb lattice}

\maketitle

\author{Jiasen Guo}
\author{Vitalii Dugaev}
\author{Arthur Ernst}
\author{George Yumnam}
\author{Pousali Ghosh}
\author{Deepak Kumar Singh$^\ast$}


\begin{affiliations}
J. Guo, G. Yumnam, P. Ghosh, Prof. D. K. Singh\\
Department of Physics and Astronomy\\
University of Missouri, Columbia \\
Columbia, Missouri 65211, USA \\
E-mail: singhdk@missouri.edu\\
\medskip
Prof. V. Dugaev\\
Department of Physics and Medical Engineering\\
Rzeszów University of Technology\\
Rzeszów, 35-959, Poland\\
\medskip 
Prof. A. Ernst\\
Institut f\"{u}r Theoretisce Physik\\
Johannes Kepler Universit\"{a}t\\
Linz, 4040, Austria\\
Max-Planck-Institut f\"{u}r Mikrostrukturphysik\\
Halle, 06120, Germany\\
\end{affiliations}
\clearpage

\justify
\section{Berry phase due to the magnetization profile in the ring}

Herewith we propose a possible mechanism of experimentally observed
magnetization jumps with the variation of the applied magnetic field.
This effect is related to the Berry phase due to inhomogeneity of the
in-plane magnetization in the rings. The idea is to account for the
energy of an electron system under the gauge potential related to the
magnetization and the vector potential of the external field
${\bf B}$.  Indeed, the total energy $E$ of the system includes the
magnetic energy $E_m$ and the energy of electron system $E_{el}$. Both
of them are depending on the magnetic field ${\bf B}$ and the
magnetization distribution ${\bf M}({\bf r})$. At low temperatures,
one realizes a ``saddle-point" magnetization profile
${\bf M}_{sp}({\bf r})$ corresponding to the minimum of the total
energy $E$ at a given ${\bf B}$.  A nonlinear dependence of
${\bf M}_{sp}({\bf r})$ on the external field ${\bf B}$ can be the
reason of the magnetization jumps.

To make it clear we consider a model with electrons on the magnetic
ring. In the absence of an external field, the
magnetization is in-plane along the ring due to magnetic anisotropy,
so that $M_z=0$. When we apply the field ${\bf B}$ along axis $z$, the
magnetization can acquire a $z$-component $M_z$ (see Figure 1a). To estimate the energy of the electron system we consider electrons on the
quantum ring. The corresponding Hamiltonian can be presented as  
\begin{eqnarray}
\label{1}
H=\int d^2{\bf r}\; \psi ^\dag \Big[ -\frac{\hbar ^2}{2m}\Big( \boldsymbol{\nabla} -\frac{ie{\bf A}}{\hbar c}\Big)
+W({\bf r})-gM_0\, \bsig \cdot {\bf n}({\bf r})\Big] \, \psi ,
\end{eqnarray} 
where $\psi ^\dag ({\bf r})$ and $\psi ({\bf r})$ are the spinor
creation and annihilation operators, ${\bf A}({\bf r})$ is the
electromagnetic vector potential, ${\bf n}({\bf r})$ is the unit vector of magnetization, ${\bf M}({\bf r})=M_0{\bf n}({\bf r})$, and $g$ is the coupling constant. The potential $W({\bf r})$ bounds electrons to the ring. One can use a local unitary transformation $U({\bf r})$, $\psi \to U\psi $ such that  
\begin{eqnarray}
\label{2}
U\, (\bsig \cdot {\bf n})\, U^{-1}=\bsig \cdot {\bf n}_0\, .
\end{eqnarray}
where ${\bf n}_0$ is a constant unit vector, and we choose ${\bf n}_0=(n_{0x},\, 0,\, n_{0z}$).
Then after $U$-transformation we get
\begin{eqnarray}
\label{3}
H=\int d^2{\bf r}\; \psi ^\dag \Big[ -\frac{\hbar ^2}{2m}\, 
\Big( \boldsymbol{\nabla} -i\bmatA -\frac{ie{\bf A}}{\hbar c}\Big) ^2
+W({\bf r})-gM_0\, \bsig \cdot {\bf n_0}\Big] \, \psi \, ,
\end{eqnarray}
where
\begin{eqnarray}
\label{4}
\bmatA ({\bf r})=iU\, (\boldsymbol{\nabla} U^{-1})	
\end{eqnarray}
is the gauge potential. Hamiltonian (S3) describes electrons moving in a homogeneous magnetization field $M_0{\bf n}_0$ under magnetic field ${\bf B}$ and the gauge potential $\bmatA ({\bf r})$. 


In the case of magnetization along the ring with a constant value of $n_z$ (as shown in Figure 1a), the matrix of unitary transformation is
\begin{eqnarray}
\label{5}
U({\bf r})=e^{i\alpha \sigma _z} ,
\end{eqnarray}
where $\alpha ({\bf r})$ is the angle around circle. 
Then using Equation (S4) we obtain
\begin{eqnarray}
\label{6}
\bmatA ({\bf r})=\sigma _z\, \boldsymbol{\nabla}\, \alpha . 	
\end{eqnarray}
Using the angular coordinate $\alpha $ at the ring we present the one-particle Hamiltonian in form
\begin{eqnarray}
\label{7}
H=-\frac{\hbar ^2}{2mR^2}\, \Big( \frac{d}{d\alpha }-i\sigma _z-\frac{ieA_l}{\hbar c}\Big) ^2
-gM_0\, \bsig \cdot {\bf n}_0\, , 	
\end{eqnarray}
where $R$ is the ring radius and $A_l=BR/2$ is the longitudinal component (along the ring) of the vector potential ${\bf A}$. 
The eigenfunction of operator (S7) has the form $\psi (\alpha )=e^{is\alpha }\psi _s$, where $s$ is the integer quantum number. Correspondingly, the Schr\"odinger equation for spinor $\psi _s$ is 
\begin{eqnarray}
\label{8}
\Big[ \frac{\hbar ^2}{2mR^2}\, \Big( s-\sigma _z-\frac{eBR^2}{2\hbar c}\Big) ^2
-gM_0\, \bsig \cdot {\bf n}_0-\varepsilon \Big] \psi _s=0,  	
\end{eqnarray}
from which we obtain the energy of electron in $s$-state
\begin{eqnarray}
\label{9}
\varepsilon _s=\frac{\hbar ^2(\tilde{s}^2+1)}{2mR^2}\pm
\left[ \Big( \frac{\hbar ^2\tilde{s}}{mR^2}-gM_z\Big) ^2+g^2M_l^2\right] ^{1/2} ,	
\end{eqnarray}
where $\tilde{s}=s-\Phi /\Phi _0$, $\Phi =\pi R^2B$ is the magnetic flux through the ring and $\Phi _0=hc/e$ is the elementary flux.
It can be also presented as a dependence of $\varepsilon _s$ on the angle $\theta $ between vector ${\bf M}$ and axis $z$
\begin{eqnarray}
\label{10}
\varepsilon _s(\theta )=\frac{\hbar ^2(\tilde{s}^2+1)}{2mR^2}\pm
\left[ \Big( \frac{\hbar ^2\tilde{s}}{mR^2}-gM_0\cos \theta \Big) ^2+g^2M_0^2\sin ^2\theta \right] ^{1/2} .	
\end{eqnarray}
We can use the notations
$\varepsilon _0=\hbar ^2/mR^2,\; \xi =gM_0/\varepsilon _0$, and $t=\cos \theta $.
The parameter $\xi $ describes a ratio of the spin splitting to the size quantization splitting $\varepsilon _0$.   
%
Then we get
\begin{eqnarray}
\label{11}
\frac{\varepsilon _s}{\varepsilon _0}
=\frac{\tilde{s}^2+1}2 \pm\big( \tilde{s}^2-2\tilde{s}\xi t +\xi ^2\big) ^{1/2}.
\end{eqnarray}  
As we see, the energy of electron at the quantum level $s$ is depending on the orientation of magnetization of the ring. 
Figure 2b shows how the energy is depending on magnetic field at different orientations of magnetization. The dependence of $\varepsilon _s$ with $s=1$ on the parameter $t$ for different values of field is presented in Figure 2a. We see that when the magnetic field is small, the electron energy has a minimum at $t=0$ (for in-plane magnetization). But with the increasing field, after certain critical value $\Phi _c$ of the flux, the electron energy gets smaller if $t=1$. Thus, there appear an energy gain related to electron system, which makes it favorable the magnetization jump -- the magnetization changes its orientation from in-plane to out-of-plane along axis $z$. Besides, as the jump is associated with the critical flux $\Phi _c$, this imposes a relation between the magnetic field $B_c$ and the cell size $S$. For the parameters in Figure 2a, we have a critical point $B_c\simeq \Phi _0/S$. For example, if we have two different contours with $S_2=7S_1$ (it means that $S_1$ is one-cell area and $S_2$ has 7 cells) then $B_{c1}/B_{c2}=7$).

The obtained results can be interpreted in terms of the Berry phase of electron moving along the contour C within the ring. Indeed, the wave function of electron moving adiabatically in a non-homogeneous magnetization field ${\bf M}({\bf r})$ acquires the Berry phase $\gamma ^g_C=\int _C \mathcal{A}_l\, dl$, where $\mathcal{A} _l$ is the longitudinal component of gauge potential (S6). For the closed contour (ring) we find $\gamma ^{g}_C=\oint _C \mathcal{A}_l\, dl=2\pi $, which is one-half of the full surface of Berry sphere (the mapping space of vector field ${\bf n}({\bf r})$). This corresponds to the flux of topological (gauge) field $\bmatB =\boldsymbol\nabla \times \bmatA $ created by the monopole with topological charge $e_m=1$ in the center of Berry sphere.     

The magnetic field ${\bf B}$ generates an additional (Aharonov-Bohm) phase $\gamma ^m_C=\oint _C A_l\, dl$ at the same closed contour C, so that the total flux $\gamma _C=\gamma ^g_C+\gamma ^m_C$ through the one-half of Berry sphere is not $2\pi $ but is depending on ${\bf B}$. This can be viewed as a variation of monopole charge, which changes the total flux through one-half of the Berry sphere.  







\section{Mechanism of anomalous Hall effect in a 2D electron system with magnetized honeycomb lattice}   

Here we propose a possible mechanism of anomalous Hall effect in a two-dimensional electron system on a magnetic network with the inhomogeneous in-plane magnetization. The key role of proposed effect is a certain chirality of the magnetic structure. For definiteness we consider the model with Hamiltonian
\begin{eqnarray}
\label{12}
H=\int d^2{\bf r}\; \psi ^\dag ({\bf r})\left[ 
-\frac{\hbar ^2(\nabla _x^2+\nabla _y^2)}{2m}+g\, \bsig \cdot {\bf M}({\bf r})\right] \psi ({\bf r})	
\end{eqnarray}
assuming that magnetization field ${\bf M}({\bf r})$ is forming a lattice of magnetic rings presented in Figure~\ref{fig:spin_solid_model},
\begin{eqnarray}
\label{eq:mag}
{\bf M}({\bf r})=\sum _i {\bf m}({\bf r}-{\bf R}_i^1)+{\bf m}({\bf r}-{\bf R}_i^2),
\end{eqnarray}
where the magnetization of a single ring 
\begin{eqnarray}
\label{14}
{\bf m}({\bf r})=\frac{\lambda _0}{r^2}\, (\hat{\bf z}\times {\bf r})
\end{eqnarray}
for $r_1<r<r_2$ and ${\bf m}({\bf r})=0$ otherwise. Vector ${\bf R}_i^1$ and ${\bf R}_i^2$ in Equation~(\ref{eq:mag}) determines the locations of the center of single rings of opposite chiralities in the $i$th unit cell. The Fourier transformation of Equation~(\ref{eq:mag}) is  
\begin{eqnarray}
\label{15}
{\bf M}({\bf q})=i\lambda _q \sum _i ( e^{-i{\bf q}\cdot {\bf R}_i^1} - e^{-i{\bf q}\cdot {\bf R}_i^2})\, (\hat{\bf z}\times {\bf n}_{\bf q}), 
\end{eqnarray}
where $\hat{\bf z}$  is the unit vector along axis $z$ perpendicular to 2D plane, ${\bf n}_{\bf q}$ is the unit vector along ${\bf q}$ and we denoted 
\begin{eqnarray}
\label{16}
\lambda _q=\frac{2\pi \lambda _0}{q} \int _{qr_1}^{qr_2} J_1(x)\, dx.
\end{eqnarray}
In Equation~(S16), $J_1(x)$ is the Bessel function. Using Equation (S16) we can present the magnetization profile in the following form
\begin{eqnarray}
\label{17}
{\bf M}({\bf r})=i\int \frac{d^2{\bf q}}{(2\pi )^2}\, \lambda _q
\sum _i (e^{i{\bf q}\cdot ({\bf r}-{\bf R}_i^1)} - e^{i{\bf q}\cdot ({\bf r}-{\bf R}_i^2)} )\, (\hat{\bf z}\times {\bf n}_{\bf q}) .
\end{eqnarray}
Correspondingly, the matrix element of perturbation related to magnetic lattice
\begin{eqnarray}
\label{18}
V_{\bf kk'}=\frac{ig\lambda _{{\bf k-k'}}}{\Omega _0}\sum _i (e^{-i({\bf k-k'})\cdot {\bf R}_i^1} - e^{-i({\bf k-k'})\cdot {\bf R}_i^2})\; 
\bsig \cdot (\hat{\bf z}\times {\bf n}_{\bf k-k'}), 
\end{eqnarray}
where $\Omega _0$ is the sample area.

Now we calculate the current along axis $y$ assuming the electric field applied along axis $x$. 
Using the Kubo method we present the transverse current 
\begin{eqnarray}
\label{19}
j_{y\omega }=-\frac{ie^2\hbar ^2A_{x\omega }}{m^2c}\, {\rm Tr} 
\int d^2{\bf r}_1\int \frac{d\varepsilon }{2\pi }\; 
\nabla _y\; \tilde{G}({\bf r,r}_1;\varepsilon +\hbar \omega )\; \nabla _x\; 
\tilde{G}({\bf r}_1,{\bf r'};\varepsilon )\Big| _{{\bf r'}={\bf r}} ,
\end{eqnarray}
where $G({\bf r,r'}; \varepsilon )$ is the Green's function of Hamiltonian (S12). One can calculate the transverse current taking the magnetic lattice in second-order approximation of perturbation theory as presented in diagram of Figure~\ref{fig:diagram}. \cite{note1} Substituting $A_{x\omega }=cE_{x\omega }/i\omega $ and taking the limit $\omega \to 0$ we obtain 
\begin{eqnarray}
\label{20}
j_y=\frac{e^2\hbar ^2E_x\Omega _0}{m^2\omega }\, \lim _{\omega \to 0} {\rm Tr} 
\int \frac{d^2{\bf k}}{(2\pi )^2} \frac{d^2{\bf k'}}{(2\pi )^2} \frac{d\varepsilon }{2\pi }\; 
k_y\, G_{0k}(\varepsilon +\hbar \omega )\, V_{\bf kk'}\, G_{0k'}(\varepsilon +\hbar \omega )\, k'_x\,
G_{0k'}(\varepsilon )\, V_{\bf k'k}\, G_{0k}(\varepsilon ) ,
\end{eqnarray}
 where 
\begin{eqnarray}
\label{21}
G_{0k}(\varepsilon )=\big( \varepsilon -\varepsilon _k+\mu +i\Gamma \, {\rm sgn}\, \varepsilon \big) ^{-1} 
\end{eqnarray}
is the Green's function of free electron, $\Gamma =\hbar /2\tau $ is the relaxation rate, $\tau $ the electron relaxation time, $\varepsilon _k=\hbar ^2k^2/2m$, and $\mu $ is the chemical potential. The integral over $\varepsilon $ is not zero if the Green's function poles are in different halfplanes of complex $\varepsilon $. Then we get
\begin{eqnarray}
\label{22}
j_y=\frac{e^2\hbar ^3E_x\Omega _0}{2\pi m^2}\, {\rm Tr}
\int \frac{d^2{\bf k}}{(2\pi )^2} \frac{d^2{\bf k'}}{(2\pi )^2}\,
k_y\, G^R_{0k}\, V_{\bf kk'}\, G^R_{0k'}\, k'_x\,
G^A_{0k'}\, V_{\bf k'k}\, G^A_{0k}\, ,
\end{eqnarray}
where $G^{R,A}_k$ are the retarded and advanced Green's functions at $\varepsilon =0$. Using Equation (S18) we find 
\begin{equation}
\label{eq:conduct}
\begin{split}
\sigma _{yx}=\frac{e^2\hbar ^3g^2N}{\pi m^2\Omega _0}
&\int \frac{d^2{\bf k}}{(2\pi )^2}\, 
\int \frac{d^2{\bf k'}}{(2\pi )^2}\,
\lambda ^2_{\bf k-k'}\,
k_y\, G^R_k\, G^A_k\,
k_x'\, G^R_{k'}\, G^A_{k'}\,\\
&\sum _{nn'}(e^{-i({\bf k}-{\bf k'})\cdot (n {\bf a} + n'{\bf b})}(2-e^{-i({\bf k}-{\bf k'})\cdot \boldsymbol{\delta}}-e^{-i({\bf k'}-{\bf k})\cdot \boldsymbol{\delta}} ),
\end{split}
\end{equation}
where $N$ is the number of cells. Basis vectors of the honeycomb lattice presented in Figure~\ref{fig:spin_solid_model} are
\begin{eqnarray}
\label{eq:lattice_vectors}
{\bf a}=a_0\; (1,\, 0),\hskip0.3cm
{\bf b}=\frac{a_0}{2}\, (1,\, \sqrt{3}), \hskip0.3cm \boldsymbol{\delta}=\frac{a_0}{2}(1,\, \frac{\sqrt{3}}{3}),
\end{eqnarray}
where $a_0$ is the lattice constant of magnetic structure and $\boldsymbol{\delta} = {\bf R}_i^1 - {\bf R}_j^2$, $n$ and $n'$ are integers. Then we obtain the transverse conductivity consists of three terms
\begin{eqnarray}
\label{25}
\sigma _{yx} = \sigma_{yx}^1 - \sigma_{yx}^2 -\sigma_{yx}^3,
\end{eqnarray}
where
\begin{eqnarray}
\label{26}
\sigma_{yz}^1=\frac{e^2\hbar ^3g^2}{\pi m^2a_0^2}\sum _{nn'}
\int \frac{d^2{\bf k}}{(2\pi )^2}\,
2e^{-\frac{ika_0}2 ((2n+n')\cos \varphi +\sqrt{3}n'\sin \varphi )}\; 
k\sin \varphi \; G^R_k\, G^A_k
\nonumber \\ \times
\int \frac{d^2{\bf k'}}{(2\pi )^2}\, \lambda ^2_{\bf k-k'}\,
e^{\frac{ika_0}2 ((2n+n')\cos \psi +\sqrt{3}n'\sin \psi)}\; 
k'\cos \psi \; G^R_{k'}\, G^A_{k'}\, .
\end{eqnarray}
Then after integrating over $k$ we get
\begin{equation}
\label{27}
\begin{split}
\sigma _{yx}^1
\simeq 2\frac{e^2g^2\lambda ^2k_F^2\tau ^2}{4\pi ^3\hbar ^3a_0^2}\sum _{nn'}&
\int _0^{2\pi } d\varphi \; e^{-\frac{ik_Fa_0}2 ((2n+n')\cos \varphi +\sqrt{3}'n\sin \varphi)} \sin \varphi \\
&\int _0^{2\pi } d\psi \; e^{\frac{ik_Fa_0}2 ((2n+n')\cos \psi +\sqrt{3}n'\sin \psi)} \cos \psi ,
\end{split}
\end{equation}
where we introduced the mean square of $\lambda ^2_{\bf k-k'}$ for $k,k'=k_F$
\begin{eqnarray}
\label{28}
\lambda ^2=\frac1{\pi }\int _0^\pi d\phi \, \lambda ^2_{{\bf k}_F-{\bf k}'_F}
=\frac1{\pi }\int _0^\pi d\phi \; \lambda ^2_{2k_F|\sin (\phi /2)|}.
\end{eqnarray}
Substituting Equation~(\ref{eq:lattice_vectors}) into~(\ref{eq:conduct}) results in similar expressions for $\sigma_{yx}^2$ and $\sigma_{yx}^3$.
The result of calculation is presented as a dependence of 
$(\sigma _{yx}/\sigma _0)$ on $k_Fa_0$. We denoted
$\sigma _0=e^2g^2\lambda ^2\tau ^2/4\pi ^3\hbar ^3a_0^4$.
The calculation is performed for a finite structure with $n$ and $n'$ running from $-10$ to $10$. As we see from Figure 5a, $(\sigma _{yx}/\sigma _0)$ oscillates as a function of $k_Fa_0$.
Thus, it is expected that any external perturbation, such as magnetic field application, that alters the Fermi level of the system or the magnetic unit cell size will lead to oscillatory transverse current. We have performed a similar calculations on another magnetization distribution, where each magnetic unit cell contains one magnetic ring of the same chirality, presented in Figure~\ref{fig:simple_model}. As we see from Figure~\ref{fig:simple_model_conduct}, $(\sigma _{yx}/\sigma _0) \times (k_Fa_0)^3$ is an oscillating function decreasing with $a_0$ as $1/a_0^3$.  
Thus, the transverse current can be rather strong in structures with small magnetic cells.   

The physical mechanism of anomalous Hall effect can be also understood from Equation~(S3). The gauge field $\bmatA $ comes to this equation like the vector potential of electromagnetic field ${\bf A}$. It affects the energy spectrum and electron wavefunctions of the system. On the other hand, nonzero circulation of the gauge field $\bmatA $ along the ring is equivalent to the flux of Berry curvature $\bmatB =\boldsymbol{\nabla} \times \bmatA $ penetrating the ring. Thus, the gauge field $\bmatB $ is acting like external magnetic field,  inducing the Hall current.        




\section{Persistent currents}

Here we calculate the equilibrium persistent current in a single magnetic ring. We assume that in the equilibrium state without external field, the magnetic moments are in-plane along the ring ($M_z=0$). The Hamiltonian of electrons in the ring of radius $R$ is
\begin{eqnarray}
\label{29}
H=Rd_0\int d\alpha \; \psi ^\dag (\alpha )
\left[ -\frac{\hbar ^2}{2mR^2}\, \frac{d^2}{d\alpha ^2}
+gM\left( \begin{array}{cc} 0 & -ie^{-i\alpha } \\ ie^{i\alpha } & 0 \end{array}\right)  \right] \psi (\alpha ),
\end{eqnarray}
where $d_0$ is the width of ring. The eigenfunctions and eigenvectors can be found from the Schr\"odinger equation 
\begin{eqnarray}
\label{30}
\left( \begin{array}{cc} 
-\frac{\hbar ^2}{2mR^2}\, \frac{d^2}{d\alpha ^2} -\varepsilon & 
-igMe^{-i\alpha } \\ igMe^{i\alpha } & 
-\frac{\hbar ^2}{2mR^2}\, \frac{d^2}{d\alpha ^2}-\varepsilon 
\end{array}\right)  
\left( \begin{array}{c} c_1\, e^{in\alpha } \\ c_2\, e^{i(n+1)\alpha }\end{array}\right) =0,
\end{eqnarray}
from which we find eigenenergies
\begin{eqnarray}
\label{31}
\varepsilon _n=\frac{\hbar ^2}{4mR^2}\, \big[ n^2+(n+1)^2\big]
\pm \Big( \frac{\hbar ^4}{16m^2R^4}\, \big[ n^2-(n+1)^2\big] ^2+g^2M^2\Big) ^{1/2} 
\end{eqnarray}
and the relation between coefficients $c_1$ and $c_2$
\begin{eqnarray}
\label{32}
c_2=-\frac{i}{gM}\, \Big( \frac{\hbar ^2n^2}{2mR^2}-\varepsilon _n\Big) \, c_1.
\end{eqnarray}
Then using the normalization condition, $|c_1|^2+|c_2|^2=1$, we find
\begin{eqnarray}
\label{33}
|c_1|^2=\frac1{1+\frac1{g^2M^2} \Big( \frac{\hbar ^2n^2}{2mR^2}-\varepsilon _n\Big) ^2}\, ,
\\
|c_2|^2=\frac{\frac1{g^2M^2} \Big( \frac{\hbar ^2n^2}{2mR^2}-\varepsilon _n\Big) ^2}
{1+\frac1{g^2M^2} \Big( \frac{\hbar ^2n^2}{2mR^2}-\varepsilon _n\Big) ^2}\, .
\end{eqnarray}
%
Using the current operator, corresponding to electric current along the ring
\begin{eqnarray}
\label{34}
\hat{j}_\alpha =-\frac{ie\hbar \nabla _\alpha }{mR} 
\end{eqnarray}
and the eigenfunctions of Hamiltonian (S29) we can calculate the expectation value of $\hat{j}_\alpha $
\begin{eqnarray}
\label{35}
j_\alpha =-\frac{ie\hbar }{mR}
\sum _n f(\varepsilon _n)\, \big( \psi ^\dag _n\, \nabla _\alpha \, \psi _n\big)
=\frac{e\hbar }{mR}\sum _n\, \big[ n|c_1|^2+(n+1)|c_2|^2\big] \, f(\varepsilon _n),    	
\end{eqnarray}
where $\psi _n^T=\big( c_1\, e^{in\alpha },\, c_2\, e^{i(n+1)\alpha }\big) $ and $f(\varepsilon _n)$ is the Fermi-Dirac function. The dependence of current $j_\alpha $ (in units $e\hbar /mR$) on the chemical potential $\mu $ for $gM=0.01$~eV is presented in Figure 5b.  

The Berry phase at the contour along the ring
\begin{eqnarray}
\label{36}
\phi _n=\int _0^{2\pi }d\alpha \, A_n(\alpha ),	
\end{eqnarray}
where $A_n(\alpha )$ is the gauge potential
\begin{eqnarray}
\label{37}
A_n(\alpha )=-i \psi ^\dag _n\, \nabla _\alpha \, \psi _n\, .	
\end{eqnarray} 
Using Equation~(S35) we can present the equilibrium current by the Berry phase
\begin{eqnarray}
\label{38}
j_\alpha =\frac{e\hbar }{2\pi mR}\sum _n f(\varepsilon _n)\, \phi _n\, . 	
\end{eqnarray}
The existence of equilibrium current in the nano ring is related to inhomogeneous magnetization. Indeed, in the noncollinear magnetic state one appears the spin torque acting on magnetic moments. The torque transfer can be viewed as the spin current of propagating electrons, which, in its turn, is accompanied by the charge current due to the imbalance of electrons with different spin polarization.

\clearpage


\begin{figure}%[!ht]
\centering
\includegraphics[width=\linewidth]{Figure_S1.png}
\caption{The spin solid model of 2D electron gas (2DEG) with magnetized rings of alternating chiralities in a given magnetic unit cell.}
\label{fig:spin_solid_model}
\end{figure}


\begin{figure}%[!ht]
\centering
\includegraphics[width=\linewidth]{Figure_S2.png}
\caption{Kubo diagram for the conductivity.}
\label{fig:diagram}
\end{figure}

\begin{figure}%[!ht]
\centering
\includegraphics[width=\linewidth]{Figure_S3.png}
\caption{The model of 2DEG with a lattice of magnetized rings of the same chirality. Each magnetic unit cell contains one magnetized ring.}
\label{fig:simple_model}
\end{figure}

\begin{figure}%[!ht]
\centering
\includegraphics[width=\linewidth]{Figure_S4.png}
\caption{First principle calculation of the transverse conductivity based on the 2DEG model presented in Figure~\ref{fig:simple_model}.}
\label{fig:simple_model_conduct}
\end{figure}

\clearpage
\begin{figure}%[!ht]
\centering
\includegraphics[width=\linewidth]{Figure_S5.png}
\caption{Semiconducting-type linear resistance fitted with Arrhenius law. The activation energies are found to be around 17 K and 25 K in temperature unit for (a) Py honeycomb and (b) Py-Pt honeycomb, respectively.}
\label{fig:resistivity_fitting}
\end{figure}

\clearpage
\begin{thebibliography}{99}

%\bibitem{vonsovsky}
%S. V. Vonsovsky, {\em Magnetism} (Nauka, Moscow, 1971).

\bibitem{note1}
Strictly speaking, the higher-order terms can be also important for the calculation of Hall current.

\end{thebibliography}



\end{document}




