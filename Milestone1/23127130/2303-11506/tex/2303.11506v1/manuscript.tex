\documentclass{WileyMSP-template}
\usepackage{amsmath}
\usepackage{amssymb}
\usepackage{braket}
\usepackage[super,square,compress,sort]{natbib}

\usepackage{setspace}
\onehalfspacing

\usepackage{lineno}
\renewcommand\linenumberfont{\normalfont\bfseries\small}



\begin{document}


%\pagestyle{fancy}
%\rhead{\includegraphics[width=2.5cm]{vch-logo.png}}


\title{Topological monopole’s gauge field induced anomalous Hall effect in artificial honeycomb lattice}

\maketitle

\author{Jiasen Guo}
\author{Vitalii Dugaev}
\author{Arthur Ernst}
\author{George Yumnam}
\author{Pousali Ghosh}
\author{Deepak Kumar Singh$^\ast$}


\begin{affiliations}
J. Guo, G. Yumnam, P. Ghosh, Prof. D. K. Singh\\
Department of Physics and Astronomy\\
University of Missouri, Columbia \\
Columbia, Missouri 65211, USA \\
E-mail: singhdk@missouri.edu\\
\medskip
Prof. V. Dugaev\\
Department of Physics and Medical Engineering\\
Rzeszów University of Technology\\
Rzeszów, 35-959, Poland\\
\medskip 
Prof. A. Ernst\\
Institut f\"{u}r Theoretisce Physik\\
Johannes Kepler Universit\"{a}t\\
Linz, 4040, Austria\\
Max-Planck-Institut f\"{u}r Mikrostrukturphysik\\
Halle, 06120, Germany\\
\end{affiliations}


% Keywords: Please provide a minimum of three and a maximum of seven keywords, separated by commas
\section*{Key Points}
\begin{enumerate}
    \item{Demonstration of gauge field flux due to vortex magnetism.}
    \item{Anomalous Hall effect in magnetic honeycomb lattice.}
    \item{Topological monopole induced magnetic flux.}
\end{enumerate}
\keywords{Anomalous Hall effect, vortex magnetism, topological monopole, magnetic honeycomb}
\clearpage

\justify
\section*{Abstract}
% Abstract should be written in the present tense and impersonal style (i.e., avoid we), and be at most 200 words long
\begin{abstract}
Vortex magnetic structure in artificial honeycomb lattice provides a unique platform to explore emergent properties due to the additional Berry phase curvature imparted by chiral magnetization to circulating electrons via direct interaction. We argue that while the perpendicularly-aligned magnetic component leads to the quantized flux of monopole at the center of the Berry sphere, the in-plane vortex circulation of magnetization gives rise to unexpected non-trivial topological Berry phase due to the gauge field transformation. The unprecedented effect signifies the importance of vector potential in multiply-connected geometrical systems. Experimental confirmations to proposed hypotheses are obtained from Hall resistance measurements on permalloy honeycomb lattice. Investigation of the topological gauge transformation due to the in-plane chirality reveals anomalous quasi-oscillatory behavior in Hall resistance $R_{xy}$ as function of perpendicular field. The oscillatory nature of $R_{xy}$ is owed to the fluctuation in equilibrium current as a function of Fermi wave-vector $k_F$, envisaged under the proposed new formulation in this article. Our synergistic approach suggests that artificially tunable nanostructured material provides new vista to the exploration of topological phenomena of strong fundamental importance.
\end{abstract}




\section*{Introduction}
Electrons traversing a metallic loop can result in the acquisition of a net geometric phase to its wavefunction when a parameter, such as the potential energy, is slowly varied.\cite{Xiao,McDonald,Liu,Son} The geometric phase acquired by the particle is equal to the flux of a new field called the Berry curvature through the surface defined by this loop.\cite{Son,Haldane,Weng} In magnetic materials with chiral moment arrangement, electron’s scattering from magnetic moment renders additional Berry curvature due to the Dirac’s magnetic monopole flux through the surface defined by the closed path.\cite{Chen,Zhang,Felser} Unlike the Aharonov-Bohm effect, which provides theoretical foundation for the observation of monopoles in real space, the Berry curvature strictly occurs in reciprocal space.\cite{McDonald,Rosch} The elegant formulation of Berry curvature arising due to the divergence of flux from the Dirac’s monopole led to broad synergistic exploration of the fundamental entity in reciprocal space in solid state materials. It includes the investigation of simple perovskite SrRuO$_3$ to unconventional magnets, such as chiral hedgehog, skyrmions and spin ice (although the underlying physical mechanism behind Dirac’s magnetic charge in spin ice is arguably different).\cite{Zhang,Tokura,Volovik,Milde,Nayak,Bramwell}\\

The Berry curvature can also have its origin in the topological spin chirality, described by the scalar triple product.\cite{Nayak,Nagaosa} For instance, the observation of anomalous Hall effect (AHE) in the molybdate pyrochlore revealed the topological nature of Hall signal due to the Berry curvature.\cite{Nagaosa,Tokura2} The AHE effect with topological characteristic was also envisaged to be manifested in a two-dimensional array of nanomagnetic cylinders.\cite{Vitalii} The magnetization direction is aligned along the cylinder’s length, perpendicular to the thin metallic film where Hall conductivity is measured. Now, imagining a ring-shaped magnetic system, consisting of both an in-plane vortex magnetic configuration and a canted chiral moment structure perpendicular to the plane of the ring, can in principle lead to the detection of both the AHE associated to the net Berry curvature and the magnetic monopole’s quantization directly. While the AHE is due to the vortex in-plane magnetization pattern, the magnetic monopole’s quantization can be associated to the canted moment configuration.\cite{Vitalii2} The physical mechanism is schematically described in Figure 1a and b. Figure 1a describes the closed contour C along the ring in the $x$-$y$ plane, a mapping of which on the Berry sphere is shown in Figure 1b. In the figure, magnetization orientation has two components – along the ring (in-plane) and along the $z$-axis (out-of-plane). Such unique scenario can be realized in artificially created two-dimensional magnetic (permalloy) honeycomb lattice,\cite{Heyderman} made of nanoscopic connected elements ($\sim$ 12 nm in length).\cite{Summers,Artur} The strong shape anisotropy in permalloy element ensures that the system maintains an in-plane component of magnetization even in the large perpendicular field application (it would require a field larger than the magnitude of 1/2 $\mu V^2$, $\mu$ $\sim$ 100,000 erg/cm$^2$ for permalloy, to completely align the moment to the perpendicular field direction).\cite{Heyderman,Blundell}


\section*{Methods}
\subsection*{Sample Fabrication} 
We create Permalloy (Py) and Permalloy-Platinum (Py-Pt) honeycomb lattice samples using diblock template method, which results in large throughput sample with ultra-small connecting elements of $\sim$ 12 nm in length. Details about the synthesis of diblock template can be found somewhere else.\cite{Russell} The diblock template is fabricated by spin coating copolymer PS-b-P4VP on silicon substrate, followed by solvent vapor annealing. The resulting diblock template resembles a honeycomb lattice with a typical element size of 12 nm (length) and 5 nm (width), as is shown in Figure 1c. This topographical property was exploited to create metallic honeycomb lattice by depositing permalloy, Ni$_{0.81}$Fe$_{0.19}$, in near parallel configuration in an electron-beam evaporator. The substrate was rotated at a moderate constant speed about its axis during the deposition process to create uniformity in the film thickness. In the case of Py-Pt sample, thin layer of Pt is deposited on top of Py layer without breaking the vacuum, thus ensuring the clean contact between two layers. 
\subsection*{Electrical Measurement}
Hall probe measurements are performed on a 2$\times$2 mm$^2$ sample using the standard contact configuration, as prescribed by National Institute of Standards and Technology,\cite{NIST} see inset in Figure 1. Electrical measurements were performed in a cryogen-free 9 T magnet with a base temperature of $\sim$ 5 K, using a set of current source and nanovoltmeter from Keithley. 


\section*{Results}
The circulation of gauge field $\boldsymbol{\mathcal{A}}$ along the contour C is the topological Berry phase, generated by the uniform magnetization. Typically, the Berry phase is related to the mapping $\mathbf{r} \rightarrow \mathbf{n(r)}$ of contour C to the Berry sphere (see Figure 1b), with the magnetic monopole in the center of this sphere.\cite{Xiao,He,McDonald} Consequently, the flux $\gamma_C$ of $\boldsymbol{\mathcal{B}}$ field through the surface, limited by the contour C is given by
\begin{equation}
    \gamma_c=\oint{\boldsymbol{\mathcal{A}}\cdot d\mathbf{l}}=\int{\boldsymbol{\mathcal{B}}\cdot d\mathbf{S}}
\end{equation}
where $\boldsymbol{\mathcal{B}}(\mathbf{r})=\boldsymbol{\nabla}\times \boldsymbol{\mathcal{A}}(\mathbf{r})$ is the Berry curvature. The gauge field $\boldsymbol{\mathcal{A}}$ is a matrix in spin space, defined by $\boldsymbol{\mathcal{A}}(\mathbf{r})=iU(\boldsymbol{\nabla} U^{-1})$ where $U(\mathbf{r})$ is a unitary transformation such that $U(\boldsymbol{\sigma}\cdot \mathbf{n})U^{-1}=\boldsymbol{\sigma}\cdot \mathbf{n_0}$, $\mathbf{n_0}$ is a constant vector (see Supplementary Materials for detail). In the case of magnetization along the ring with a constant value of $n_z$ (see Figure 1b), the matrix of unitary transformation is $U=e^{(i\alpha\sigma_z)}$ where $\alpha (\mathbf{r})$ is the angle around the circular contour. Now, in the presence of external magnetic field $\mathbf{B_{ex}}$, the quantity $\boldsymbol{\mathcal{A}}$ undergoes a gauge transformation of 
\begin{equation}
    \boldsymbol{\mathcal{A}}\rightarrow \boldsymbol{\mathcal{A}}+\frac{eA_l}{\hbar c},
\end{equation}
where $A_l=B_{ex}R/2$ is the longitudinal component (along the ring) of the electromagnetic field vector potential $\mathbf{A}$. Correspondingly, the net Berry phase is replaced by
\begin{equation}
    \gamma_C\rightarrow \gamma_C+\frac{e\Phi}{\hbar c}=2\pi\left ( 1+\frac{\Phi}{\Phi_0} \right ),
\end{equation}
where magnetic flux $\Phi$ due to the external magnetic field $\mathbf{B_{ex}}$ is given by $\pi R^2B_{ex}$ and $\Phi_0$ is the quantized flux of $hc/e$. The change in Berry phase in applied field basically infers a quantized change in the topological magnetic charge at the center of the Berry sphere. A mathematical confirmation to this effect is obtained by calculating the energy of electron in gauge transformed vector potential $\boldsymbol{\mathcal{A}}$. As described in Supplementary Materials, the energy of electron in a quantum level $s$ depends on the angle $\theta$ between magnetization vector $\mathbf{M}$ and the $z$-axis via the following expression:
\begin{equation}
    \frac{\varepsilon_s}{\varepsilon_0}=\frac{\tilde{s}^2+1}{2}\pm(\tilde{s}^2-2\tilde{s}\xi t+\xi^2)^{1/2},
\end{equation}
where 
\begin{equation*}
\varepsilon_0=\frac{\hbar^2}{mR^2},\qquad \xi=\frac{gM_0}{\varepsilon_0},\qquad t=\cos(\theta). 
\end{equation*}
The quantum state $\tilde{s}$ is defined as $\tilde{s}=s-\Phi/\Phi_0$. The dependence of $\varepsilon_s$ with $s = 1$ on $t$ for different values of field is presented in Figure 2a. In Figure 2a, we see that when the magnetic field is small, the electron energy has a minimum at $t = 0$ (for in-plane magnetization). But with the increasing of the field, after certain critical value $\Phi_c$ of the flux, there appear an energy gain related to electron system, which makes it favorable for magnetization to cant out of in-plane configuration. Besides, the jump associated to the critical flux $\Phi_c$ imposes a relation between magnetic field $B_c$ and the cell size $S$. The obtained results can be interpreted in terms of the Berry phase of electron moving along the contour C within the ring. Indeed, the wave function of electron moving adiabatically in a non-homogeneous magnetization field $\mathbf{M(r)}$ acquires a net Berry phase, given by, $\gamma_C^g=\int_C\mathcal{A}_l\cdot dl$, where $\mathcal{A}_l$ is the longitudinal component of gauge potential. For the closed contour (ring) we find, $\gamma_C^g=\oint \mathcal{A}_l\cdot dl=2\pi$, which is one-half of the full surface of Berry sphere (the mapping space of vector field $\mathbf{n(r)}$). This corresponds to the flux of topological (gauge) field $\boldsymbol{\mathcal{B}}=\boldsymbol{\nabla}\times \boldsymbol{\mathcal{A}}$ created by the monopole with topological charge $e_m = 1$ in the center of the Berry sphere. The magnetic field $\mathbf{B}$ generates an additional (Aharonov-Bohm) phase $\gamma_C^m=\oint A_l\cdot dl$ across the same closed contour C such that the total flux $\gamma_C=\gamma_C^g+\gamma_C^m$. This can be viewed as a variation of monopole charge, which changes the total flux through one-half of the Berry sphere. 

As discussed in the previous paragraph, the gauge field arising due to the vortex structure of magnetic configuration can lead to the topological effect of net Berry phase accumulation. Experimental confirmation to this effect can be obtained from the Hall effect measurements. We have performed Hall measurements on permalloy (Py) honeycomb samples (see Sample Fabrication Section for details about the nanofabrication procedure). Hall effect magnetoresistance measurements on a $2\times2$ mm$^2$ square size permalloy honeycomb sample reveal symmetric quasi-oscillatory responses in $R_{xy}$ as a function of magnetic field at low temperature $T$ = 5 K, see Figure 3a. The Hall resistance $R_{xy}$ increases with field, peaking around $H$ $\sim$ 1.5 T before decreasing again. The observation is in stark contrast to the linear $R_{xx}$, as shown in Figure 3b, which manifests surprising negative magnetoresistance as a function of field. We notice a sharp decrement in $R_{xx}$ below $H$ $\sim$ 0.5 T followed by the gradual decline as magnetic field increases. While a clear explanation to this effect is lacking, it is most likely arising due to the weak localization of electrons, interacting with magnetization $\mathbf{m}$ via $\boldsymbol{\sigma}\cdot \mathbf{m}$, in the honeycomb element.\cite{McCann,Vitalii3} Note that localization effects are strongly enhanced in nanostructures if the phase relaxation length $l_\phi$ is larger than the nanoelement thickness $d$. In magnetic honeycomb lattice, $d$ $\sim$ 5 nm. The magnetic field application suppresses the localization corrections, thus leading to the negative magnetoresistance. Unlike the field dependence, $R_{xy}$ and $R_{xx}$ don’t exhibit any unusual behavior as a function of temperature. As shown in Figure 3c and d, the system manifests semiconducting characteristic in both zero and applied field. 

The quasi-oscillatory behavior in $R_{xy}$ becomes conspicuously prominent in permalloy sample coated with a thin layer ($\sim$ 2 nm) of platinum (Py-Pt), see Figure 4a. The pronounced quasi-oscillatory characteristic of Hall resistance in Py-Pt sample can be attributed to the spin-orbital (SO) coupling due to the Pt film (as discussed below).\cite{Huda,Vitalii3} However, the change in Hall resistance is modest compared to that found in only Py honeycomb sample.  In the Py-Pt sample, the Hall resistance does not seem to be symmetric in field. Rather, the peak in $R_{xy}$ occurs at $H$ $\sim$ 0.5 T on the positive side and at $H$ $\sim$ $-3.8$ T on the negative side. Unlike the Py sample, the linear magnetoresistance in Py-Pt sample exhibits positive enhancement as a function of field, following a negative tendency at $H$ $<$ 0.5 T, see Figure 4b. The experimental observation is in accord with the conventional understanding of Py thin film, which is known to exhibit positive magnetoresistance. Pt coating seems to alter the electrical characteristic of honeycomb lattice sample at low temperature. Unlike the Py sample, manifesting semiconducting behavior, Py-Pt sample shows weakly metallic property as temperature reduces, see Figure 4c and d. The observation could be attributed to the weak localization effect, in addition to the SO coupling.\cite{Vitalii3} Due to the SO coupling, the system tends to reflect the metallic characteristic of Pt film. However, the strong semiconducting behavior of Py honeycomb lattice dominates at low temperature. 

\section*{Discussion}
There are two conceptual issues here: the oscillatory behavior in $R_{xy}$ and a negative linear magnetoresistance in field. The oscillatory behavior in Hall resistance and its possible topological origin can be understood from the first principle calculations. For this purpose, we consider the model Hamiltonian of
\begin{equation}
    \left [ -\frac{\hbar^2(\nabla_x^2+\nabla_y^2)}{2m}+g\boldsymbol{\sigma}\cdot \mathbf{m(r)}-\varepsilon  \right ]\psi(\mathbf{r}) = 0,
\end{equation}
where $\mathbf{m(r)}=\frac{\lambda_0}{r^2}(-y,x)=\frac{\lambda_0}{r^2}\mathbf{\hat{z}}\times \mathbf{r}$ represents local magnetization in the loop state, shown in Figure 1d. The solution to this equation is $\psi^T_n(\alpha)=(c_1e^{in\alpha}, c_2e^{i(n+1)\alpha})$. Correspondingly, the Berry phase on the contour along the ring can be written as
\begin{equation}
    \phi_n=\int_0^{2\pi}{ A_n(\alpha)d\alpha},
\end{equation}
where $A_n(\alpha)$ is the gauge potential, given by $A_n(\alpha)=-i\psi_n^\dagger\nabla_\alpha\psi_n$. The Berry phase is used to calculate the energy spectrum, equilibrium current and off-diagonal conductivity (Hall resistance) as functions of chemical potential and the parameter $k_Fa_0$. Plots of $\sigma_{xy}$ vs. $k_Fa_0$ and  $j_\alpha$ vs. $\mu$ are shown in Figure 5 (see Supporting Information for detail). Now, the Hall resistance $R_{xy}$ is dependent on $\sigma_{xy}$ via the following relation
\begin{equation}
   R_{xy}\simeq -\frac{\sigma_{yx}}{\sigma^2_{xx}}\simeq -\sigma_{yx}R_0^2\left (  1+\frac{2\delta R(H)}{R_0}  \right ), 
\end{equation}
where 
\begin{equation}
\sigma_{xx}=\frac{1}{R_{xx}}\simeq \frac{1}{R_0}-\frac{\delta R(H)}{R_0^2} 
\end{equation}
with $R_{xx}=R_0+\delta R(H)$. In the experiment, the observed absolute $\delta R(H) \ll R_0$, thus, the oscillatory behavior in $\sigma_{xy}$ due to the vortex configuration of in-plane magnetization, as shown in Figure 5a, is reflected in the oscillatory Hall resistance $R_{xy}$, albeit weakly. Also, it is worth noting that magnetic field application upends the chemical potential, hence the Fermi parameter $k_F$ and the energy spectrum $\varepsilon_k$. Cumulatively, it can be concluded that the vortex configuration of in-plane magnetization, as found in magnetic honeycomb lattice at low temperature, manifests topological characteristic of Berry phase accumulation. 

\section*{Conclusion}
Finally, we summarize the main results of the paper. The synergistic study presented here not only elucidates the occurrence of magnetic charge induced gauge transformed flux through a ring-shaped contour of perpendicular moment in honeycomb lattice but also reveals a new mechanism behind the topological nature of chiral vortex circulation due to in-plane magnetization. It should be noted that the proposed effect of the gauge field, related to inhomogeneous magnetization in the ring, cannot be reduced to the gauge-field-induced AHE. The latter mechanism would be similar to magnetic-field-induced classical Hall effect. As we demonstrated, the gauge field in the nanoring is substantially affecting the wave functions and the energy spectrum of electrons. As a result, we come to a quantum AHE in the magnetic nanostructure. This is reminiscent of the quantum Hall effect, in which the field-induced transverse current is transferred by the quantum excitations of 2D electron system (not by free electrons) in the magnetic field. The vortex magnetic configuration causes an additional flux to the one due to magnetic charge at the center of the corresponding three-dimensional Berry sphere. The net gain in Berry phase of cycling electrons due to the in-plane magnetization is highly surprising. The non-trivial effect suggests strong implication of the vector potential $\mathbf{A}$ in electrodynamics problems. Typically, the AHE due to the Berry phase, as found in pyrochlore compounds,\cite{Nagaosa} is ascribed to the chirality due to the scalar triple product of magnetic spins. In the case of in-plane vortex moment configuration, the scalar triple product is zero. However, as we have seen, the net Berry phase is non-zero, which gives rise to the AHE signal. It clearly suggests that a new mechanism is at the play. We believe this effect can be realized in other two-dimensional multiply connected systems as well. 


\medskip
\noindent\textbf{Supporting Information} \par
\noindent Supporting Information is available from the Wiley Online Library or from the author.


\medskip
\noindent\textbf{Acknowledgements} \par
\noindent DKS thankfully acknowledges the support by US Department of Energy, Office of Science, Office of Basic Energy Sciences under the grant no. DE-SC0014461. VKD thankfully acknowledges the support by the National Science Center in Poland, research project No. DEC-2017/27/B/ST3/02881.

\medskip
\noindent\textbf{Author Contributions} \par
\noindent DKS envisaged the research concept and supervised every aspect of research. JG and PG fabricated the magnetic honeycomb samples. JG and GY performed the Hall effect measurements. VKD and AE performed the first principle calculations. DKS prepared the manuscript where everyone contributed. 



\medskip
\noindent\textbf{Data Availability Statement}\par
\noindent The data that supports the findings of the study are available from the corresponding author upon reasonable request.


\medskip
\noindent\textbf{Conflict of Interest}\par
\noindent The authors declare no potential conflicts of interest.

\medskip
\noindent\textbf{ORCID}\par
\noindent Jiasen Guo, https://orcid.org/0000-0001-7708-1476\par
\noindent Vitalii Dugaev, https://orcid.org/0000-0002-0999-4051 \par
\noindent Arthur Ernst, https://orcid.org/0000-0003-4005-6781 \par
\noindent George Yumnam, https://orcid.org/0000-0001-9462-7434\par
\noindent Pousali Ghosh, https://orcid.org/0000-0002-3393-9941 \par
%\noindent Deepak Kumar Singh, \par


% References
\medskip

\begin{thebibliography}{9}
\bibitem{Xiao} Xiao D, Chang MC, Niu Q. Berry phase effects on electronic properties. \textit{Rev. Mod. Phys.} 2010;82(3):1959-2007. doi:10.1103/RevModPhys.82.1959


\bibitem{McDonald} McDonald AH, Niu Q. New twist for magnetic monopoles. \textit{Phys. World} 2004;17(1):18-19. doi:10.1088/2058-7058/17/1/24



\bibitem{Liu} Liu F, Yamamoto M, Wakabayashi K. Topological edge states of honeycomb lattices with zero Berry curvature. \textit{J. Phys. Soc. Jpn.} 2017;86(12):123707-123710. doi:10.7566/JPSJ.86.123707


\bibitem{Son} Son DT, Yamamoto N. Berry curvature, triangle anomalies, and the chiral magnetic effect in Fermi liquids. \textit{Phys. Rev. Lett.} 2012;109(18):181602-181605. doi:10.1103/PhysRevLett.109.181602



\bibitem{Haldane}Haldane F. Berry curvature on the Fermi surface: anomalous Hall effect as a topological Fermi-liquid property. \textit{Phys. Rev. Lett.} 2004;93(20):206602-206605. doi:10.1103/PhysRevLett.93.206602



\bibitem{Weng} Weng HM, Yu R, Hu X, Dai X, Fang Z. Quantum anomalous Hall effect and related topological electronic states. \textit{Adv. Phys.} 2015;64(3):227-282. doi:10.1080/00018732.2015.1068524 



\bibitem{Chen} Chen JW, Pu S, Wang Q, Wang XN. Berry curvature and four-dimensional monopoles in the relativistic chiral kinetic equation. \textit{Phys. Rev. Lett.} 2013;110(26):262301-262305. doi:10.1103/PhysRevLett.110.262301


\bibitem{Zhang} Fang Z, Nagaosa N, Takahashi KS et al. The Anomalous Hall effect and magnetic monopoles in momentum space. \textit{Science} 2003;302(5642):92-95. doi:10.1126/science.1089408 



\bibitem{Felser} Manna K, Sun Y, Muechler L, K\"{u}bler J, Felser C, Heusler, Weyl and Berry. \textit{Nat. Rev. Mat.} 2018;3(8):244-256. doi:10.1038/s41578-018-0036-5 





\bibitem{Rosch} Sch\"{u}tte C, Rosch A. Dynamics and energetics of emergent magnetic monopoles in chiral magnets. \textit{Phys. Rev. B} 2014;90(17):174432-174438. doi:10.1103/PhysRevB.90.174432



\bibitem{Tokura} Fujishiro Y, Kanazawa N, Nakajima T et al. Topological transitions among skyrmion and hedgehog lattice states in cubic chiral magnets. \textit{Nat. Commun.} 2019;10: 1059-1066. doi.org/10.1038/s41467-019-08985-6





\bibitem{Volovik} Volovik GE. Monopoles and fractional vortices in chiral superconductors. \textit{Proc. Nat. Acad. Sci.} 2000;97(6):2431-2436. doi:


\bibitem{Milde} Milde P et al. Unwinding of a skyrmion lattice by magnetic monopoles. \textit{Science} 2013;340(6136):1076-1080. doi:10.1126/science.1234657



\bibitem{Nayak} Nayak A et al. Large anomalous Hall effect driven by a nonvanishing Berry curvature in the noncolinear antiferromagnet Mn$_3$Ge. \textit{Sci. Adv.} 2016;2(4):1501870-1501874. doi:10.1126/sciadv.1501870



\bibitem{Bramwell} Fennell T, Deen PP, Wildes AR, Schmalzl K et al. Magnetic coulomb phase in the spin ice Ho$_2$Ti$_2$O$_7$. \textit{Science} 2009;326(5951):415-417. doi:10.1126/science.1177582



\bibitem{Nagaosa} Nagaosa N, Sinova J, Onoda S, MacDonald AH, and Ong NP. Anomalous Hall effect. \textit{Rev. Mod. Phys.} 2010;82(2):1539-1592. doi:10.1103/RevModPhys.82.1539


\bibitem{Tokura2} Taguchi Y, Oohara Y, Yoshizawa H, Nagaosa N, Tokura Y. Spin chirality, Berry phase, and anomalous Hall effect in a frustrated ferromagnet. \textit{Science} 2001;291(5513):2573-2578. doi:10.1126/science.1058161



\bibitem{Vitalii} Bruno P, Dugaev VK, Taillefumier M. Topological Hall effect and Berry phase in magnetic nanostructures. \textit{Phys. Rev. Lett.} 2004;93(9):096806-096809. doi:10.1103/PhysRevLett.93.096806


\bibitem{Vitalii2} Dugaev VK, Cr\'epieux A, Bruno P. Localization corrections to the anomalous Hall effect in a ferromagnet. \textit{Phys. Rev. B} 2001;64(10):104411-104416. doi:10.1103/PhysRevB.64.104411


\bibitem{Heyderman} Skj\ae rv\o{} SH, Marrows CH, Stamps RL, Heyderman LJ. Advances in artificial spin ice. \textit{Nat. Rev. Phys.} 2020;2(1):13-28. 10.1038/s42254-019-0118-3


\bibitem{Summers} Summers B, Debeer-Schmitt L, Dahal A, Glavic A, Kampshroeder P, Gunasekera J, Singh DK. Temperature-dependent magnetism in artificial honeycomb lattice of connected elements. \textit{Phys. Rev. B} 2018;97(1):014401-014406. doi:10.1103/PhysRevB.97.014401


\bibitem{Artur} Glavic A, Summers B, Dahal A, Kline J, Van Herck W, Sukhov A, Ernst A, Singh DK. Spin solid versus magnetic charge ordered state in artificial honeycomb lattice of connected elements. \textit{Adv. Sci.} 2018;5(4):1700856-1700862. doi:10.1002/advs.536



\bibitem{Blundell} Blundell S. Magnetism in condensed matter, 1st ed. Oxford University Press; 2001. 

\bibitem{Russell} Park S, Kim B, Yavuzcetin O, Tuominen MT, Russell TP, Ordering of PS-\textit{b}-P4VP on patterned silicon surfaces. \textit{ACS Nano} 2008;2(7):1363-1370. doi:10.1021/nn800073f

\bibitem{NIST} Hall Effect Measurements. December 6, 2019. https://www.nist.gov/pml/nanoscale-device-characterization-division/popular-links/hall-effect


\bibitem{He} He Y, Moore J, Varma CM. Berry phase and anomalous Hall effect in a three-orbital tight-binding Hamiltonian. \textit{Phys. Rev. B} 2012;85(15):155106-155112. doi:10.1103/PhysRevB.85.155106


\bibitem{McCann} McCann E, Kechedzhi K, Fal’ko VI, Suzuura H, Ando T, Altshuler BL. Weak-localization magnetoresistance and valley symmetry in graphene. \textit{Phys. Rev. Lett.} 2006;97(14):146805-146808. doi:10.1103/PhysRevLett.97.146805


\bibitem{Vitalii3} Dugaev VK, Bruno P, Barna\'{s} J. Weak localization in ferromagnets with spin-orbit interaction. \textit{Phys. Rev. B} 2001;64(14):144423-144435. doi:10.1103/PhysRevB.64.144423 


\bibitem{Huda} Huda MN, Niranjan MK, Sahu BR, Kleinman L. Effect of spin-orbit coupling on small platinum nanoclusters
. \textit{Phys. Rev. A} 2006;73(5):053201-053205. doi:10.1103/PhysRevA.73.053201



\end{thebibliography}

\clearpage

\begin{figure}
    \centering
    \includegraphics[width=\linewidth]{Figure_1.png}
    \caption{\textbf{Berry phase due to magnetization profile in a ring.} (a) Magnetization along the ring in magnetic field applied perpendicular to the plane of the system. Magnetization has both in-plane and perpendicular components. (b) Mapping of the contour due to perpendicular component across ring on the Berry sphere. (c) Atomic force micrograph of a nanoscopic honeycomb lattice. The typical size of permalloy element, used in this study, is 12 nm. (length) 5 nm (width). For the Py only sample, $\sim$ 6 nm Py are deposited; for the Py-Pt sample, $\sim$ 6 nm Py and 2 nm Pt are deposited. (d)  Permalloy honeycomb lattice is known to manifest spin solid state due to vortex magnetic loop at low temperature, schematically described here. (e) Perpendicular field application causes moments to cant out of the plane. However, strong shape anisotropy in permalloy competes against field-induced alignment of moment along $z$-axis. This unique scenario allows for the direct demonstrations of monopole’s induced gauge transformed flux due to Berry phase in vortex magnetic configuration, which causes anomalous Hall effect in permalloy honeycomb lattice. Inset shows the Van der Pauw configuration used for the Hall effect measurement.}
    \label{fig:Figure_1}
\end{figure}

\begin{figure}
    \centering
    \includegraphics[width=\linewidth]{Figure_2.png}
    \caption{\textbf{Energetics behind monopole’s gauge transformed flux.} (a) Electron energy levels in the state with $s$ = 1 as a function of magnetization orientations for different magnetic fields. The parameter $\xi$ = 0.5. (b) Electron energy in the state with $s$ = 1 as a function of magnetic field for different magnetization orientations. The parameter $\xi$ = 0.2. }
    \label{fig:Figure_2}
\end{figure}


\begin{figure}
    \centering
    \includegraphics[width=\linewidth]{Figure_3_twopanel.png}
    \caption{\textbf{Anomalous Hall effect due to topological characteristic of vortex loop of in-plane moment structure.} In a non-trivial depiction of vector potential’s role in generating Berry phase due to vortex magnetic profile, we show Hall resistance $R_{xy}$ data (Figure a) and linear resistance (Figure b) from permalloy (Py) honeycomb lattice at $T$ = 5 K. The red lines are for eye guidance. Oscillatory behavior is detected in Hall data in ($R_{xy}(H)$ – $R_{xy}$(0)) plot, while the linear resistance manifests negative MR as a function of field. Measurements of both $R_{xy}$ and $R_{xx}$ reveal semiconducting characteristic of Py honeycomb lattice system. }
    \label{fig:Figure_3}
\end{figure}


\begin{figure}
    \centering
    \includegraphics[width=\linewidth]{Figure_4_twopanel.png}
    \caption{\textbf{Anomalous Hall effect in Py-Pt system.} Plots of Hall resistance $R_{xy}$ (Figure a) and linear resistance (Figure b) measurements on Py-Pt sample at $T$ = 5 K. Unlike Py sample, quasi-oscillation behavior in ($R_{xy}(H)$ – $R_{xy}$(0)) is more apparent in Py-Pt sample due to the spin-orbital coupling. Also, contrasting magnetoresistance tendency is observed in Py-Pt honeycomb in linear resistance measurements of $R_{xx}$ (Figure b). The red lines are for eye guidance. Additionally, Py-Pt sample exhibits weak metallic characteristic (Figure c) in $R_{xy}$ data as a function of temperature, which is different from the Py honeycomb sample. Figure d shows the semiconducting behavior in $R_{xx}$ vs. $T$ (K) plot, similar to Py sample.}
    \label{fig:Figure_4}
\end{figure}


\begin{figure}
    \centering
    \includegraphics[width=\linewidth]{Figure_5.png}
    \caption{\textbf{Theoretical manifestation of oscillatory behavior in Hall conductivity.}  (a) First principal calculation reveals the oscillatory behavior in transverse conductivity as a function of $k_Fa_0$. (b) Equilibrium current in the magnetized ring as a function of chemical potential.}
    \label{fig:Figure_5}
\end{figure}



\end{document}
