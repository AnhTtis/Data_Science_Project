
Let $G_{sc} = \Sp(V)$ be simply connected of type $C_{\ell}$, and let $b$ be the $G_{sc}$-invariant alternating bilinear form defining $G_{sc}$. We have $\g_{sc} \cong S^2(V)^*$ by Lemma \ref{lemma:LieSpS2}, so for every unipotent element $u \in G$ the Jordan block sizes of $u$ on $\g_{sc}$ are known by the results described in Section \ref{section:tensor}. In this section, we will describe the Jordan block sizes of unipotent elements $u \in G$ on $\g_{sc}/Z(\g_{sc})$, in terms of Jordan block sizes of $u$ on $\g_{sc}$. 

Throughout this section, we will denote by $\varphi$ the linear map $\varphi: S^2(V) \rightarrow K$ defined by $\varphi(xy) = b(x,y)$ for all $x,y \in V$. Since $b$ is $G_{sc}$-invariant, it follows that $\varphi$ is a surjective morphism of $G_{sc}$-modules. By Lemma \ref{lemma:kerphiisdual}, the Jordan normal form of a unipotent element $u \in G$ on $\g_{sc}/Z(\g_{sc})$ is the same as on $\Ker \varphi$.

We will then describe the Jordan block sizes of unipotent elements $u \in G$ on $\Ker \varphi$. By Lemma \ref{jordanrestrictionNIL}, this amounts to finding the largest integer $m \geq 0$ such that $\Ker(\widetilde{u}-1)^m \subseteq \Ker \varphi$, where $\widetilde{u}$ is the action of $u$ on $S^2(V)$.

\begin{lemma}\label{lemma:unipS2decomp}
Let $u \in \Sp(V)$ be unipotent. Suppose that we have a orthogonal decomposition $V = U_1 \perp \cdots \perp U_t$ as $K[u]$-modules. Denote the action of $u$ on $S^2(V)$ by $\widetilde{u}$, and the action of $u$ on $S^2(U_i)$ by $\widetilde{u_i}$. Let $m \geq 0$ be an integer. 

Then $\Ker(\widetilde{u}-1)^m \subseteq \Ker \varphi$ if and only if $\Ker(\widetilde{u_i}-1)^m \subseteq \Ker \varphi$ for all $1 \leq i \leq t$.
\end{lemma}

\begin{proof}The orthogonal decomposition $V = U_1 \perp \cdots \perp U_t$ into $K[u]$-modules induces a decomposition $$S^2(V) = \bigoplus_{1 \leq i \leq t} S^2(U_i) \perp \bigoplus_{1 \leq i < j \leq t} U_iU_j$$ into $K[u]$-modules, where $U_iU_j = \{xy : x \in U_i, y \in U_j\}$. We have $U_iU_j \subseteq \Ker \varphi$ for all $1 \leq i < j \leq t$, from which the lemma follows.\end{proof}


With Lemma \ref{lemma:unipS2decomp}, we reduce to the case where $V \downarrow K[u]$ is orthogonally indecomposable. We first consider the case where $V \downarrow K[u] = V(2\ell)$, so $n = 2\ell$. Let $v_1$, $\ldots$, $v_n$ be a basis as in the definition of $V(2\ell)$ (Definition \ref{def:V2lUNIP}). Then $b(v_i,v_j) = 1$ if $i+j = n+1$, and $0$ otherwise. Furthermore, the action of $u$ is defined by \begin{align*}
uv_1 &= v_1, \\
uv_i &= v_i + v_{i-1} + \cdots + v_1 \text{ for all } 2 \leq i \leq \ell+1, \\
uv_i &= v_i + v_{i-1} \text{ for all } \ell+1 < i \leq n.
\end{align*} We will denote $v_j = 0$ for all $j \leq 0$ and $j > n$.

Define $\gamma = \sum_{1 \leq i \leq \ell} v_iv_{n+1-i}$. We have a short exact sequence $$0 \rightarrow V^{[2]} \rightarrow S^2(V) \rightarrow \wedge^2(V) \rightarrow 0,$$ where $V^{[2]}$ is the subspace generated by $v^2$ for all $v \in V$. As noted in \cite[Section 9]{Korhonen2020Hesselink}, the image of $\gamma$ in $\wedge^2(V)$ is fixed by the action of $\Sp(V)$. Thus $u \cdot \gamma = \gamma + x$ for some $x \in V^{[2]}$, and more precisely we have the following.

\begin{lemma}\label{lemma:unipS2delta}
Let $u$ and $\gamma$ be as above. Then $u \cdot \gamma = \gamma + \sum_{1 \leq i \leq \ell} v_i^2$, and $\gamma + v_{\ell+1}^2$ is fixed by the action of $u$.
\end{lemma}

\begin{proof}
Denote $s_i := \sum_{j \geq 0} v_{i-j} v_{n-i}$. First we note that \begin{align} \notag u \cdot \sum_{1 \leq i < \ell} v_i v_{n-i+1} &= \sum_{1 \leq i < \ell} \left( \sum_{j \geq 0} v_{i-j} \right)\left(v_{n-i+1} + v_{n-i} \right) \\ \notag &= \sum_{1 \leq i < \ell} \left(s_i + s_{i-1} + v_i v_{n-i+1}\right) \\ \label{eq:firstsum} &= s_{\ell-1} + \sum_{1 \leq i < \ell} v_i v_{n-i+1}. \end{align} By another calculation, we get \begin{equation} \label{eq:secondsum}  u \cdot v_{\ell}v_{\ell+1} = \left(\sum_{1 \leq j \leq \ell} v_j \right) \left(\sum_{1 \leq j \leq \ell+1} v_j \right) = s_{\ell-1} + v_{\ell}v_{\ell+1} +  \sum_{1 \leq j \leq \ell} v_j^2.\end{equation} Adding~\eqref{eq:firstsum} and~\eqref{eq:secondsum} together, we conclude that $u \cdot \gamma = \gamma + \sum_{1 \leq i \leq \ell} v_i^2$.

Since $u \cdot v_{\ell+1}^2 = \left(\sum_{1 \leq i \leq \ell+1} v_i\right)^2 = \sum_{1 \leq i \leq \ell+1} v_i^2$, it follows that $\gamma + v_{\ell+1}^2$ is fixed by the action of $u$.
\end{proof}

\begin{lemma}\label{lemma:S2indecompV2l}
Let $u \in \Sp(V)$ be unipotent such that $V \downarrow K[u] = V(2\ell)$. Let $\widetilde{u}$ be the action of $u$ on $S^2(V)$, and denote $\alpha = \nu_2(\ell)$. Then the following hold:
	\begin{enumerate}[\normalfont (i)]
		\item $\Ker(\widetilde{u}-1)^{2^{\alpha}-1} \subseteq \Ker \varphi$.
		\item $\Ker(\widetilde{u}-1)^{2^{\alpha}} \not\subseteq \Ker \varphi$.
	\end{enumerate}
\end{lemma}

\begin{proof}
By Lemma \ref{lemma:smallesblockunipS2} the smallest Jordan block size of $\widetilde{u}$ is $2^{\alpha}$, so (i) follows from Lemma \ref{jordanrestrictionNIL}. 

For (ii), we first consider the case where $\alpha = 0$, so $\ell$ is odd. In the notation used before the lemma, by Lemma \ref{lemma:unipS2delta} the vector $\gamma + v_{\ell+1}^2$ is fixed by the action of $u$. Furthermore $\varphi(\gamma + v_{\ell+1}^2) = \ell \neq 0$, so $\Ker(\widetilde{u}-1) \not\subseteq \Ker \varphi$, as claimed.

Suppose then that $\alpha > 0$. We have $V \downarrow K[u^{2^{\alpha}}] = V(\ell/2^{\alpha-1})^{2^{\alpha}}$ by \cite[Lemma 6.13]{Korhonen2020Hesselink}. Combining this with Lemma \ref{lemma:unipS2decomp} and the fact that $(\widetilde{u}-1)^{2^{\alpha}} = \widetilde{u}^{2^{\alpha}} - 1$, the claim follows from the $\alpha = 0$ case.\end{proof} 

\begin{lemma}\label{lemma:S2indecompWl}
Let $u \in \Sp(V)$ be unipotent such that $V \downarrow K[u] = W(\ell)$. Let $\widetilde{u}$ be the action of $u$ on $S^2(V)$, and denote $\alpha = \nu_2(\ell)$. Then the following hold:
	\begin{enumerate}[\normalfont (i)]
		\item $\Ker(\widetilde{u}-1)^{2^{\alpha}-1} \subseteq \Ker \varphi$.
		\item $\Ker(\widetilde{u}-1)^{2^{\alpha}} \not\subseteq \Ker \varphi$.
	\end{enumerate}
\end{lemma}

\begin{proof}
We have $V = W \oplus W^*$ as $K[u]$-modules, where $W$ and $W^*$ are totally isotropic subspaces. This induces a decomposition $$S^2(V) = S^2(W) \oplus S^2(W^*) \oplus WW^*$$ of $K[u]$-modules, where $WW^* \cong W \otimes W^*$ is the subspace $\{xy : x \in W, y \in W^*\}$. We have $S^2(W), S^2(W^*) \subseteq \Ker \varphi$ and the smallest Jordan block size in $W \otimes W^*$ is $2^{\alpha}$ \cite[Lemma 4.2]{KorhonenJordanGood}, so (i) follows from Lemma \ref{jordanrestrictionNIL}.

For (ii), we first consider the case where $\alpha = 0$, so $\ell$ is odd. Choose a basis $v_1$, $\ldots$, $v_{\ell}$ of $W$, and let $w_1$, $\ldots$, $w_{\ell}$ be the corresponding dual basis in $W^*$, so $b(v_i,w_j) = \delta_{i,j}$ for all $1 \leq i,j \leq \ell$. Then $\gamma' = \sum_{1 \leq i \leq \ell} v_iw_i$ is fixed by the action of $u$, see for example \cite[Lemma 3.7]{KorhonenJordanGood}. We have $\varphi(\gamma') = \ell \neq 0$, so $\Ker(\widetilde{u}-1) \not\subseteq \Ker \varphi$, as claimed.

Suppose then that $\alpha > 0$. We have $(\widetilde{u}-1)^{2^{\alpha}} = \widetilde{u}^{2^{\alpha}} - 1$, and furthermore $V \downarrow K[u^{2^{\alpha}}] = W(\ell/2^{\alpha})^{2^{\alpha}}$ by \cite[Lemma 6.12]{Korhonen2020Hesselink}. Thus as in the proof of Lemma \ref{lemma:S2indecompV2l}, the claim follows from Lemma \ref{lemma:unipS2decomp} and the $\alpha = 0$ case.\end{proof}


\begin{prop}\label{prop:unipKeraction}
Let $u \in \Sp(V)$ be unipotent, with orthogonal decomposition $$V \downarrow K[u] = \sum_{1 \leq i \leq t} W(m_i) \perp \sum_{1 \leq j \leq s} V(2k_j).$$ Denote by $\alpha \geq 0$ the largest integer such that $2^{\alpha} \mid m_i,k_j$ for all $i$ and $j$. Then \begin{align*}
		\g_{sc} &\cong V_{2^{\alpha}} \oplus V' \\
		\g_{sc}/Z(\g_{sc}) &\cong V_{2^{\alpha}-1} \oplus V' \end{align*} for some $K[u]$-module $V'$.
\end{prop}

\begin{proof}
Let $\widetilde{u}$ be the action of $u$ on $S^2(V)$. By Lemma \ref{lemma:LieSpS2}, Lemma \ref{lemma:kerphiisdual}, and Lemma \ref{jordanrestrictionNIL}, the proposition is equivalent to the statement that $\Ker(\widetilde{u}-1)^{2^{\alpha}-1} \subseteq \Ker \varphi$ and $\Ker(\widetilde{u}-1)^{2^{\alpha}} \not\subseteq \Ker \varphi$. 

By Lemma \ref{lemma:unipS2decomp}, Lemma \ref{lemma:S2indecompV2l}, and Lemma \ref{lemma:S2indecompWl}, we have $\Ker(\widetilde{u}-1)^{2^{\alpha}-1} \subseteq \Ker \varphi$. Next note that either $\alpha = \nu_2(m_i)$ for some $i$, or $\alpha = \nu_2(k_j)$ for some $j$.  It follows then from Lemma \ref{lemma:unipS2decomp}, together with Lemma \ref{lemma:S2indecompWl} (if $\alpha = \nu_2(m_i)$) or Lemma \ref{lemma:S2indecompV2l} (if $\alpha = \nu_2(k_j)$) that $\Ker(\widetilde{u}-1)^{2^{\alpha}} \not\subseteq \Ker \varphi$.\end{proof}

