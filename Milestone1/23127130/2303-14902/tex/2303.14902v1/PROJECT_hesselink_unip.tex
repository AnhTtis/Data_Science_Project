
Consider $\Sp(V)$ with $\dim V = 2\ell$. In this section, we will recall the description of unipotent conjugacy classes in $\Sp(V)$, due to Hesselink \cite{Hesselink}.  For more details, see for example \cite{Hesselink}, \cite[Chapter 4, Chapter 6]{LiebeckSeitzClass}, or \cite[Section 6]{Korhonen2020Hesselink}.

For a group $G$, a bilinear $K[G]$-module $(W, \beta)$ is a finite-dimensional $K[G]$-module $W$ equipped with a $G$-invariant bilinear form $\beta$. Two bilinear $K[G]$-modules $(W,\beta)$ and $(W', \beta')$ are said to be isomorphic, if there exists an isomorphism $W \rightarrow W'$ of $K[G]$-modules which is also an isometry.

Let $u,u' \in \Sp(V)$ be unipotent. Let $q$ be a power of two such that $u^q = (u')^q = 1$, so that $V \downarrow K[u]$ and $V \downarrow K[u']$ are $K[C_q]$-modules. Then $u$ and $u'$ are conjugate in $\Sp(V)$ if and only if $V \downarrow K[u] \cong V \downarrow K[u']$ as bilinear $K[C_q]$-modules. 

For $u \in \Sp(V)$ unipotent, it is clear that we can write $V \downarrow K[u] = U_1 \perp \cdots \perp U_t$, where $U_i$ are \emph{orthogonally indecomposable} $K[u]$-modules. Here orthogonally indecomposable means that if $U_i = U' \perp U''$ as $K[u]$-modules, then $U' = 0$ or $U'' = 0$. There are two basic types of orthogonally indecomposable $K[u]$-modules, which we can define as follows. (Similar definitions are given in \cite[Section 6.1]{LiebeckSeitzClass}.)

\begin{maar}\label{def:V2lUNIP}
For $\ell \geq 1$, we define the module $V(2\ell)$ as follows. Let $n = 2\ell$, and suppose that $V$ has basis $v_1$, $\ldots$, $v_{n}$ with $b(v_i,v_j) = 1$ if $i+j = n+1$ and $0$ otherwise. Define $u: V \rightarrow V$ by \begin{align*} u v_1 &= v_1 \\ u v_i &= v_{i} + v_{i-1} + \cdots + v_1 \text{ for all } 1 < i \leq \ell + 1 \\ u v_i &= v_i + v_{i-1} \text{ for all } \ell+1 < i \leq n.\end{align*} Then $u \in \Sp(V)$, and we define $V(2\ell)$ as the bilinear $K[u]$-module $V \downarrow K[u]$.
\end{maar}

\begin{maar}\label{def:WlUNIP}
For $\ell \geq 1$, we define the module $W(\ell)$ as follows. Let $n = 2\ell$, and suppose that $V$ has basis $v_1$, $\ldots$, $v_{n}$ with $b(v_i,v_j) = 1$ if $i+j = n+1$ and $0$ otherwise. Define $u: V \rightarrow V$ by \begin{align*} u v_1 &= v_1 \\ u v_i &= v_i + v_{i-1} + \cdots + v_1 \text{ for all } 1 < i \leq \ell \\ u v_{\ell+1} &= v_{\ell+1} \\ u v_i &= v_i + v_{i-1} \text{ for all } \ell+1 < i \leq n.\end{align*} Then $u \in \Sp(V)$, and we define $W(\ell)$ as the bilinear $K[u]$-module $V \downarrow K[u]$.
\end{maar}

The fact that the modules are isomorphic to those described by Hesselink \cite[Proposition 3.5]{Hesselink} is seen as follows.

	\begin{itemize}
		\item For the module $V(2\ell)$ in Definition \ref{def:V2lUNIP} we have $V(2\ell) \downarrow K[u] = V_{2\ell}$, so this agrees with \cite[Proposition 3.5]{Hesselink}.
		\item In Definition \ref{def:WlUNIP}, we have totally singular decomposition $V = W \oplus Z$, where $W = \langle v_1, \ldots, v_{\ell} \rangle$ and $Z = \langle v_{\ell+1}, \ldots, v_n \rangle$ are $K[u]$-modules with $W \cong Z \cong V_{\ell}$. From this it follows that $V$ is isomorphic to the module $W(\ell)$ defined by Hesselink, see for example \cite[Lemma 6.12]{Korhonen2020Hesselink}.
	\end{itemize}

The classification of unipotent conjugacy classes in $\Sp(V)$ is based on the following result.

\begin{lause}[{\cite[Proposition 3.5]{Hesselink}}]\label{thm:hesselinkUNIPindecomp}
Let $u \in \Sp(V)$ be unipotent such that $V \downarrow K[u]$ is orthogonally indecomposable. Then $V \downarrow K[u]$ is isomorphic to $V(2\ell)$ or $W(\ell)$, where $\dim V = 2\ell$.
\end{lause}

By Theorem \ref{thm:hesselinkUNIPindecomp}, for every unipotent $u \in \Sp(V)$ we have an orthogonal decomposition $$V \downarrow K[u] = U_1 \perp \cdots \perp U_t,$$ where for all $1 \leq i \leq t$ we have $U_i \cong V(2\ell_i)$ or $U_i \cong W(\ell_i)$ for some integer $\ell_i \geq 1$.

In general there can be several different ways to decompose $V \downarrow K[u]$ into orthogonally indecomposable summands, and even the number of summands is not uniquely determined. This is due to the fact that for even $m$, we have an isomorphism $$W(m) \perp V(m) \cong V(m) \perp V(m) \perp V(m)$$ of bilinear $K[u]$-modules.  However, there are normal forms which are uniquely determined, such as the \emph{Hesselink normal form} \cite[3.7]{Hesselink} \cite[Theorem 6.4]{Korhonen2020Hesselink} or the \emph{distinguished normal form} defined by Liebeck and Seitz in \cite[p. 61]{LiebeckSeitzClass}.

