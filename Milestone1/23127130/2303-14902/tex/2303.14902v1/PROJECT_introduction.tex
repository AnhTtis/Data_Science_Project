
Let $G$ be a simple algebraic group over an algebraically closed field $K$, and denote the Lie algebra of $G$ by $\g$. Recall that an element $u \in G$ is \emph{unipotent}, if $f(u)$ is a unipotent linear map for every rational representation $f: G \rightarrow \GL(W)$. Similarly $e \in \g$ is said to be \emph{nilpotent}, if $\D f(e)$ is a nilpotent linear map for every rational representation $f: G \rightarrow \GL(W)$.

We denote the adjoint representations of $G$ and $\g$ by $\Ad: G \rightarrow \GL(\g)$ and $\ad: \g \rightarrow \mathfrak{gl}(V)$, respectively. In this paper, we will consider the following two problems.

	\begin{prob}\label{prob:unipAD}
		Let $u \in G$ be a unipotent element. What is the Jordan normal form of $\Ad(u)$?
	\end{prob}
	
	\begin{prob}\label{prob:nilAD}
		Let $e \in \g$ be a nilpotent element. What is the Jordan normal form of $\ad(e)$?
	\end{prob}

In most cases, the answer to both questions is known. When $G$ is simply connected of exceptional type, the Jordan block sizes were computed in the unipotent case by Lawther \cite{Lawther, LawtherCorrection} and in the nilpotent case by Stewart \cite{StewartNilpotentBlocks}. In the case where $G$ is exceptional of adjoint type, both questions are easily settled using the results of Lawther and Stewart, see \cite[Lemma 3.1]{KorhonenStewartThomas}.

When $G$ is of type $A_{\ell}$, solutions to both questions follow from the main results of \cite{KorhonenJordanGood} and \cite{KorhonenNilpotent}, see \cite[Remark 1.3]{KorhonenNilpotent}. For $G$ of type $B_{\ell}$, $C_{\ell}$, or $D_{\ell}$ in good characteristic, the Jordan block sizes are described by well-known results on decompositions of tensor products, symmetric squares, and exterior squares --- see for example \cite[Remark 3.5]{KorhonenNilpotent}. 

Thus it remains to solve Problem \ref{prob:unipAD} and Problem \ref{prob:nilAD} in the case where $G$ is of type $B_{\ell}$, $C_{\ell}$, or $D_{\ell}$ in characteristic $\chr K = 2$. In this paper, we consider the case where $G$ is of type $C_{\ell}$. We make the following assumption for the rest of this paper.
	\begin{center} \emph{Assume that $\chr K = 2$.}\end{center} 
	
Let $G$ be a simple algebraic group of type $C_{\ell}$ over $K$. As our main result, we determine the Jordan block sizes of $\Ad(u)$ and $\ad(e)$ for all unipotent $u \in G$ and nilpotent $e \in \g$. In type $C_{\ell}$ either $G$ is simply connected or $G$ is adjoint; we deal with these two possibilities as follows.

Let $G_{sc}$ be simply connected and simple of type $C_{\ell}$ with Lie algebra $\g_{sc}$, and let $G_{ad}$ be simple of adjoint type $C_{\ell}$ with Lie algebra $\g_{ad}$. There exists an isogeny $\varphi: G_{sc} \rightarrow G_{ad}$, which induces a bijection between the unipotent variety of $G_{sc}$ and $G_{ad}$. Similarly the differential $\D \varphi: \g_{sc} \rightarrow \g_{ad}$ induces a bijection between the nilpotent cone of $\g_{sc}$ and $\g_{ad}$.

Thus for the solution of Problem \ref{prob:unipAD} and Problem \ref{prob:nilAD} in type $C_{\ell}$, it will suffice to consider the Jordan block sizes of unipotent $u \in G_{sc}$ on $\g_{sc}$ and $\g_{ad}$, and the Jordan block sizes of nilpotent $e \in \g_{sc}$ on $\g_{sc}$ and $\g_{ad}$.

We can assume that $G_{sc} = \Sp(V)$ with Lie algebra $\g_{sc} = \mathfrak{sp}(V)$, where $\dim V = 2\ell$. It is well known (Lemma \ref{lemma:LieSpS2}) that $\g_{sc} \cong S^2(V)^*$ as $G_{sc}$-modules. The Jordan block sizes of the action of unipotent $u \in \GL(V)$ on $S^2(V)$ and nilpotent $e \in \mathfrak{gl}(V)$ on $S^2(V)$ are described in \cite{KorhonenSymExt2021}. Taking the dual does not change the Jordan block sizes, so the Jordan block sizes on $\g_{sc}$ are known by previous results.

In our main results, we will describe the Jordan block sizes on $\g_{ad}$ in terms of the Jordan block sizes on $\g_{sc}$. To state the result in the unipotent case, we first need describe the classification of unipotent conjugacy classes in $\Sp(V)$, which in characteristic two is due to Hesselink \cite{Hesselink}. We discuss this in some more detail in Section \ref{section:hesselinkunipotent}.

Let $C_q$ be a cyclic group of order $q$, where $q = 2^{\alpha}$ for some $\alpha \geq 0$. Then there are a total of $q$ indecomposable $K[C_q]$-modules $V_1$, $\ldots$, $V_q$ up to isomorphism, where $\dim V_i = i$ and a generator of $C_q$ acts on $V_i$ with a single $i \times i$ unipotent Jordan block. For convenience we will denote $V_0 = 0$. If $W$ is a vector space over $K$, we denote $W^0 = 0$, and $W^d = W \oplus \cdots \oplus W$ ($d$ summands) for an integer $d > 0$.

Suppose then that $u \in \GL(V)$ is unipotent, so $u$ has order $q$ for some $q = 2^{\alpha}$. We denote the group algebra of $\langle u \rangle$ by $K[u]$. Then $V \downarrow K[u] \cong V_{d_1}^{n_1} \oplus \cdots \oplus V_{d_t}^{n_t}$ for some integers $0 < d_1 < \cdots < d_t$, where $n_i > 0$ for all $1 \leq i \leq t$. Equivalently, the Jordan normal form of $u$ has Jordan blocks of sizes $d_1$, $\ldots$, $d_t$, and a block of size $d_i$ has multiplicity $n_i$.

For unipotent $u \in \Sp(V)$, we have a decomposition $V \downarrow K[u] = U_1 \perp \cdots \perp U_t$, where $U_i$ are \emph{orthogonally indecomposable} $K[u]$-modules. Here orthogonally indecomposable means that if $U_i = U' \perp U''$ as $K[u]$-modules, then $U' = 0$ or $U'' = 0$. The orthogonally indecomposable $K[u]$-modules fall into two types: one denoted by $V(m)$ (for $m$ even) and another denoted by $W(m)$ (for $m \geq 1$). We define these in Section \ref{section:hesselinkunipotent}, for now we only mention that $V(m) \cong V_m$ and $W(m) \cong V_m \oplus V_m$ as $K[u]$-modules.

Our first main result is the following. (Below we denote by $\nu_2$ the $2$-adic valuation on the integers, so $\nu_2(a)$ is the largest integer $k \geq 0$ such that $2^k$ divides $a$.)

\begin{lause}\label{thm:unipGSCtoGAD}
Let $u \in \Sp(V)$ be unipotent, with orthogonal decomposition $$V \downarrow K[u] = \sum_{1 \leq i \leq t} W(m_i) \perp \sum_{1 \leq j \leq s} V(2k_j).$$ Denote by $\alpha \geq 0$ the largest integer such that $2^{\alpha} \mid m_i,k_j$ for all $i$ and $j$. Then the following hold:
	\begin{enumerate}[\normalfont (i)]
		\item Suppose that $s = 0$. Then:
			\begin{enumerate}[\normalfont (a)]
				\item If $\alpha = 0$, then $\g_{sc} \cong \g_{ad}$ as $K[u]$-modules.
				\item If $\alpha > 0$, then \begin{align*}
		\g_{sc} &\cong V_{2^{\alpha}} \oplus V' \\
		\g_{ad} &\cong V_{1} \oplus V_{2^{\alpha}-1} \oplus V' \end{align*} for some $K[u]$-module $V'$.
			\end{enumerate}
		\item Suppose that $s > 0$, and let $\beta = \max_{1 \leq j \leq s} \nu_2(k_j)$. Then:
			\begin{enumerate}[\normalfont (a)]
				\item If $\beta = \nu_2(k_j)$ for all $1 \leq j \leq s$ and $\nu_2(m_i) > \beta$ for all $1 \leq i \leq t$, then $\g_{sc} \cong \g_{ad}$ as $K[u]$-modules.
				\item If $\beta > \nu_2(k_j)$ for some $1 \leq j \leq s$, or $\nu_2(m_i) \leq \beta$ for some $1 \leq i \leq t$, then  \begin{align*}
		\g_{sc} &\cong V_{2^{\alpha}} \oplus V_{2^{\beta}} \oplus V' \\
		\g_{ad} &\cong V_{2^{\alpha}-1} \oplus V_{2^{\beta}+1} \oplus V' \end{align*} for some $K[u]$-module $V'$.		
			\end{enumerate}
	\end{enumerate}
\end{lause}

To describe our results in the nilpotent case, we recall the classification of nilpotent orbits in $\mathfrak{sp}(V)$, due to Hesselink \cite{Hesselink}. For more details we refer to Section \ref{section:hesselinknilpotent}.

Suppose that $q = 2^{\alpha}$ for some $\alpha \geq 0$. Let $\mathfrak{w}_q$ be the abelian $2$-Lie algebra over $K$ generated by a single nilpotent element $e \in \mathfrak{w}_q$ such that $e^{[2^{\alpha}]} = 0$ and $e^{[2^{\alpha-1}]} \neq 0$. There are a total of $q$ indecomposable $\mathfrak{w}_q$-modules $W_1$, $\ldots$, $W_q$ up to isomorphism, where $\dim W_i = i$ and $e$ acts on $W_i$ with a single $i \times i$ nilpotent Jordan block. Throughout we will denote $W_0 = 0$.

Consider then a nilpotent linear map $e \in \mathfrak{gl}(V)$, and denote the $2$-Lie subalgebra generated by $e$ with $K[e]$. Then $K[e] \cong \mathfrak{w}_q$, where $q$ is the smallest power of two such that $e^q = 0$. We have $V \downarrow K[e] \cong W_{d_1}^{n_1} \oplus \cdots \oplus W_{d_t}^{n_t}$ for some integers $0 < d_1 < \cdots < d_t$, where $n_i > 0$ for all $1 \leq i \leq t$. As in the unipotent case, this amounts to the statement that the Jordan normal form of $e$ has Jordan blocks of sizes $d_1$, $\ldots$, $d_t$, and a block of size $d_i$ has multiplicity $n_i$.

For nilpotent $e \in \mathfrak{sp}(V)$, we have an orthogonal decomposition $V \downarrow K[e] = U_1 \perp \cdots \perp U_t$, where the $U_i$ are orthogonally indecomposable $K[e]$-modules. Here orthogonally indecomposable is defined as in the unipotent case. For the orthogonally indecomposable $K[e]$-modules, there are several different types: $V(m)$ (for $m$ even), $W_k(m)$ (for $0 < k < m/2$ and $m > 2$), and $W(m)$ (for $m \geq 1$). We give the definitions and more details in Section \ref{section:hesselinknilpotent}, for now we just note that $V(m) \cong W_m$, $W_k(m) \cong W_m \oplus W_m$, and $W(m) \cong W_m \oplus W_m$ as $K[e]$-modules.

Our main result in the nilpotent case is the following.

\begin{lause}\label{thm:nilGSCtoGAD}
Let $e \in \mathfrak{sp}(V)$ be nilpotent, with orthogonal decomposition $$V \downarrow K[e] = \sum_{1 \leq i \leq t} W(m_i) \perp \sum_{1 \leq j \leq t'} W_{k_j}(\ell_j) \perp \sum_{1 \leq r \leq t''} V(2d_r).$$ Denote by $\alpha \geq 0$ the largest integer such that $2^{\alpha} \mid m_i,\ell_j,2d_r$ for all $i$, $j$, and $r$. Then the following statements hold:
	\begin{enumerate}[\normalfont (i)]
		\item Suppose that $V \downarrow K[e] = \sum_{1 \leq i \leq t} W(m_i)$. Then:	
			\begin{enumerate}[\normalfont (a)]
				\item If $\alpha = 0$, then $\g_{sc} \cong \g_{ad}$ as $K[e]$-modules.
				\item If $\alpha > 0$, then \begin{align*}
		\g_{sc} &\cong W_{2^{\alpha}} \oplus V' \\
		\g_{ad} &\cong W_{1} \oplus W_{2^{\alpha}-1} \oplus V' \end{align*} for some $K[e]$-module $V'$.
			\end{enumerate}
		\item Suppose that $V \downarrow K[e]$ is not of the form $\sum_{1 \leq i \leq t} W(m_i)$. Then:
			\begin{enumerate}[\normalfont (a)]
				\item If $\alpha = 1$, then $\g_{sc} \cong \g_{ad}$ as $K[e]$-modules.
				\item If $\alpha \neq 1$, then \begin{align*}
		\g_{sc} &\cong W_1 \oplus W_{2^{\alpha}} \oplus V' \\
		\g_{ad} &\cong W_{2} \oplus W_{2^{\alpha}-1} \oplus V' \end{align*} for some $K[e]$-module $V'$.
			\end{enumerate}
	\end{enumerate}
\end{lause}

\begin{esim}\label{example:tables}
In Table \ref{table:UNIPexamples} and Table \ref{table:NILexamples}, we illustrate Theorem \ref{thm:unipGSCtoGAD} and Theorem \ref{thm:nilGSCtoGAD} in the case where $G$ is of type $C_{\ell}$ for $2 \leq \ell \leq 4$. In the tables we have also included the Jordan block sizes on $[\g_{ad},\g_{ad}]$, which we prove as an intermediate result in Proposition \ref{prop:unipKeraction} and Proposition \ref{prop:nilKeraction}. Furthermore, the values of $\alpha$ and $\beta$ that appear in Theorem \ref{thm:unipGSCtoGAD} and Theorem \ref{thm:nilGSCtoGAD} are included.

In the tables, we use the notation $$d_1^{n_1}, \ldots, d_t^{n_t}$$ for $0 < d_1 < \cdots < d_t$ and $n_i > 0$ to denote that the Jordan block sizes are $d_1$, $\ldots$, $d_t$, and a block of size $d_i$ has multiplicity $n_i$. For the orthogonal decompositions in the first column, notation such as $V(m)^k$ denotes an orthogonal direct sum $V(m) \perp \cdots \perp V(m)$ ($k$ summands).\end{esim}

In particular, from Theorem \ref{thm:unipGSCtoGAD} and Theorem \ref{thm:nilGSCtoGAD} we get the number of Jordan blocks of $\Ad(u)$ and $\ad(e)$, which is equal to the dimension of the Lie algebra centralizer of $u$ and $e$, respectively.

\begin{seur}\label{cor:centralizerdimUNIP}
Let $u \in \Sp(V)$ be unipotent and denote $\alpha$, $s$ as in Theorem \ref{thm:unipGSCtoGAD}. Then the following hold:
	\begin{enumerate}[\normalfont (i)]
		\item Suppose that $s = 0$. Then $$\dim \g_{ad}^u = \begin{cases} \dim \g_{sc}^u,& \text{ if } \alpha = 0. \\ \dim \g_{sc}^u+1,& \text{ if } \alpha > 0.\end{cases}$$		
		\item Suppose that $s > 0$. Then $$\dim \g_{ad}^u = \begin{cases} 
		\dim \g_{sc}^u-1,& \text{ if } \alpha = 0 \text{ and } \nu_2(k_j) > 0 \text{ for some } j.  \\ 
		\dim \g_{sc}^u-1,& \text{ if } \alpha = 0 \text{ and } \nu_2(m_i) = 0 \text{ for some } i. \\ 
		\dim \g_{sc}^u,& \text{ if } \alpha = 0 \text{ and } \nu_2(k_j) = 0, \nu_2(m_i) > 0 \text{ for all } i \text{ and } j. \\
		\dim \g_{sc}^u,& \text{ if } \alpha > 0.\end{cases}$$			
	\end{enumerate}
\end{seur}

\begin{seur}\label{cor:centralizerdimNIL}
Let $e \in \mathfrak{sp}(V)$ be nilpotent and denote $\alpha$ as in Theorem \ref{thm:unipGSCtoGAD}. Then the following hold:
	\begin{enumerate}[\normalfont (i)]
		\item Suppose that $V \downarrow K[e] = \sum_{1 \leq i \leq t} W(m_i)$. Then $$\dim \g_{ad}^e = \begin{cases} \dim \g_{sc}^e,& \text{ if } \alpha = 0. \\ \dim \g_{sc}^e+1,& \text{ if } \alpha > 0.\end{cases}$$
		\item Suppose that $V \downarrow K[e]$ is not of the form $\sum_{1 \leq i \leq t} W(m_i)$. Then $$\dim \g_{ad}^e = \begin{cases} \dim \g_{sc}^e-1,& \text{ if } \alpha = 0 .\\ \dim \g_{sc}^e,& \text{ if } \alpha > 0.\end{cases}$$
	\end{enumerate}
\end{seur}

In the case of regular elements, this amounts to the following. (An element $x \in G$ or $x \in \g$ is \emph{regular}, if $\dim C_G(x) = \rank G$. In the case of $G = \Sp(V)$, a regular unipotent element is characterized by $V \downarrow K[u] = V(2\ell)$, and a regular nilpotent element is characterized by $V \downarrow K[e] = V(2\ell)$.)

\begin{seur}\label{cor:regularcentralizerdim}
Suppose that $G$ is of type $C_{\ell}$ in characteristic two (adjoint or simply connected). Let $u \in G$ be a regular unipotent element and $e \in \g$ a regular nilpotent element. Then $\dim \g^u = \ell+1$ and $\dim \g^e = 2\ell$.
\end{seur}

For the proofs of our main results, our basic approach is as follows. We will first describe the Jordan block sizes of unipotent $u \in G_{sc}$ and nilpotent $e \in \g_{sc}$ on $\g_{sc}/Z(\g_{sc})$, in terms of the Jordan block sizes on $\g_{sc}$. This is attained in Section \ref{section:unipgmodZ} and Section \ref{section:nilgmodZ}.

It is known that $\g_{sc}/Z(\g_{sc}) \cong [\g_{ad}, \g_{ad}]$ as $G_{sc}$-modules (Lemma \ref{lemma:gmodZisgadgad}), so we then have the Jordan block sizes on $[\g_{ad}, \g_{ad}]$ as well. In Section \ref{section:unipGAdZ} and Section \ref{section:nilGAdZ} we will describe the Jordan block sizes of unipotent $u \in G_{sc}$ and nilpotent $e \in \g_{sc}$ on $\g_{ad}$, in terms of the Jordan block sizes on $[\g_{ad}, \g_{ad}]$. Combining these results, we get the Jordan block sizes of $u$ and $e$ on $\g_{ad}$, in terms of the Jordan block sizes on $\g_{sc}$ (Theorem \ref{thm:unipGSCtoGAD}, Theorem \ref{thm:nilGSCtoGAD}).

The other sections of this paper are organized as follows. The notation, terminology, and some preliminary results are stated in Section \ref{section:notation} and Section \ref{section:prelim}. In Section \ref{section:tensor}, we state results on Jordan block sizes of unipotent and nilpotent elements on tensor products, symmetric squares, and exterior squares. The classification of unipotent classes in $\Sp(V)$ and nilpotent orbits in $\mathfrak{sp}(V)$ is described in Section \ref{section:hesselinkunipotent} and Section \ref{section:hesselinknilpotent}, respectively.

In Section \ref{section:chevalley}, we discuss the Chevalley construction of simple algebraic groups, and in particular the action of $G_{sc} = \Sp(V)$ on $\g_{ad}$. We then make some observations about the structure of $\g_{sc}$ and $\g_{ad}$ as a $G_{sc}$-module in Section \ref{section:typeClie}. 

As mentioned earlier, in Section \ref{section:unipgmodZ} and Section \ref{section:nilgmodZ} we describe the Jordan block sizes of unipotent and nilpotent elements on $\g_{sc}/Z(\g_{sc})$, in terms of the Jordan block sizes on $\g_{sc}$. In Section \ref{section:unipGAdZ} and Section \ref{section:nilGAdZ} we similarly describe the Jordan block sizes on $[\g_{ad}, \g_{ad}]$, in terms of Jordan block sizes on $\g_{ad}$. These results allow us to prove our main results, and the proofs of the results stated in this introduction are given in Section \ref{section:final}.

\begin{table}[!htbp]
\centering

\caption{For $G_{sc} = \Sp(V)$ simply connected of type $C_{\ell}$ with $2 \leq \ell \leq 4$ and $\chr K = 2$, Jordan block sizes of unipotent $u \in G_{sc}$ on $\g_{sc}$, $[\g_{ad}, \g_{ad}]$, and $\g_{ad}$. See Example \ref{example:tables}.}\label{table:UNIPexamples}
\begin{tabular}{lllllll}
\hline

&&&&&&\\[-10pt]
$\ell$ & $V \downarrow K[u]$          & $\g_{sc} \downarrow K[u]$ & $[\g_{ad}, \g_{ad}] \downarrow K[u]$ & $\g_{ad} \downarrow K[u]$ & $\alpha$ & $\beta$ \\ \hline
&&&&&&\\[-5pt]
$2$   & $V(4)$ &        $ 2, 4^{2} $ &         $ 1, 4^{2} $ &         $ 2, 4^{2} $      & $1$ & $1$ \\
& $V(2)^2$ &      $ 1^{2}, 2^{4} $ &     $ 1, 2^{4} $ &     $ 1^{2}, 2^{4} $            & $0$ & $0$ \\
& $W(1) \perp V(2)$ & $ 1^{4}, 2^{3} $ &     $ 1^{3}, 2^{3} $ &     $ 1^{2}, 2^{4} $    & $0$ & $0$ \\
& $W(2)$ &        $ 1^{2}, 2^{4} $ &     $ 1^{3}, 2^{3} $ &     $ 1^{4}, 2^{3} $        & $1$ & $-$ \\
& $W(1)^2$ &      $ 1^{10} $ &           $ 1^{9} $ &           $ 1^{10} $               & $0$ & $-$ \\
\\
&&&&&&\\		
$3$	   & $V(6)$ &          $ 1, 4, 8^{2} $ &      $ 4, 8^{2} $ &      $ 1, 4, 8^{2} $      & $0$ & $0$ \\
& $V(2) \perp V(4)$ &   $ 1, 2^{2}, 4^{4} $ &  $ 2^{2}, 4^{4} $ &  $ 2, 3, 4^{4} $         & $0$ & $1$ \\
& $V(2)^3$ &        $ 1^{3}, 2^{9} $ &     $ 1^{2}, 2^{9} $ &     $ 1^{3}, 2^{9} $         & $0$ & $0$ \\
& $W(1)^2 \perp V(2)$ & $ 1^{11}, 2^{5} $ &    $ 1^{10}, 2^{5} $ &    $ 1^{9}, 2^{6} $     & $0$ & $0$ \\
& $W(1) \perp V(4)$ &   $ 1^{3}, 2, 4^{4} $ &  $ 1^{2}, 2, 4^{4} $ &  $ 1^{2}, 3, 4^{4} $  & $0$ & $1$ \\
& $W(1) \perp V(2)^2$ & $ 1^{5}, 2^{8} $ &     $ 1^{4}, 2^{8} $ &     $ 1^{3}, 2^{9} $     & $0$ & $0$ \\
& $W(3)$ &          $ 1, 2^{2}, 4^{4} $ &  $ 2^{2}, 4^{4} $ &  $ 1, 2^{2}, 4^{4} $         & $0$ & $-$ \\
& $W(1) \perp W(2)$ &   $ 1^{5}, 2^{8} $ &     $ 1^{4}, 2^{8} $ &     $ 1^{5}, 2^{8} $     & $0$ & $-$ \\
& $W(1)^3$ &        $ 1^{21} $ &           $ 1^{20} $ &           $ 1^{21} $               & $0$ & $-$ \\
&&&&&&\\
$4$  & $V(8)$ &               $ 4, 8^{4} $ &         $ 3, 8^{4} $ &         $ 4, 8^{4} $                                      & $2$ & $2$ \\
& $V(2) \perp V(6)$ &        $ 1^{2}, 2, 4, 6^{2}, 8^{2} $ & $ 1, 2, 4, 6^{2}, 8^{2} $ & $ 1^{2}, 2, 4, 6^{2}, 8^{2} $        & $0$ & $0$ \\
& $V(4)^2$ &             $ 2^{2}, 4^{8} $ &     $ 1, 2, 4^{8} $ &     $ 2^{2}, 4^{8} $                                        & $1$ & $1$ \\
& $V(2)^2 \perp V(4)$ &      $ 1^{2}, 2^{5}, 4^{6} $ & $ 1, 2^{5}, 4^{6} $ & $ 1, 2^{4}, 3, 4^{6} $                           & $0$ & $1$ \\
& $V(2) \perp W(3)$ &        $ 1^{2}, 2^{5}, 4^{6} $ & $ 1, 2^{5}, 4^{6} $ & $ 2^{6}, 4^{6} $                                 & $0$ & $0$ \\
& $W(1) \perp V(2)^3$ &      $ 1^{6}, 2^{15} $ &    $ 1^{5}, 2^{15} $ &    $ 1^{4}, 2^{16} $                                  & $0$ & $0$ \\
& $W(1)^3 \perp V(2)$ &      $ 1^{22}, 2^{7} $ &    $ 1^{21}, 2^{7} $ &    $ 1^{20}, 2^{8} $                                  & $0$ & $0$ \\
& $W(2) \perp V(4)$ &        $ 1^{2}, 2^{5}, 4^{6} $ & $ 1^{3}, 2^{4}, 4^{6} $ & $ 1^{3}, 2^{3}, 3, 4^{6} $                   & $1$ & $1$ \\
& $V(2)^4$ &             $ 1^{4}, 2^{16} $ &    $ 1^{3}, 2^{16} $ &    $ 1^{4}, 2^{16} $                                      & $0$ & $0$ \\
& $W(1)^2 \perp V(4)$ &      $ 1^{10}, 2, 4^{6} $ & $ 1^{9}, 2, 4^{6} $ & $ 1^{9}, 3, 4^{6} $                                 & $0$ & $1$ \\
& $W(1)^2 \perp V(2)^2$ &    $ 1^{12}, 2^{12} $ &   $ 1^{11}, 2^{12} $ &   $ 1^{10}, 2^{13} $                                 & $0$ & $0$ \\
& $W(1) \perp V(6)$ &        $ 1^{4}, 4, 6^{2}, 8^{2} $ & $ 1^{3}, 4, 6^{2}, 8^{2} $ & $ 1^{2}, 2, 4, 6^{2}, 8^{2} $          & $0$ & $0$ \\
& $W(1) \perp V(2) \perp V(4)$ & $ 1^{4}, 2^{4}, 4^{6} $ & $ 1^{3}, 2^{4}, 4^{6} $ & $ 1^{3}, 2^{3}, 3, 4^{6} $               & $0$ & $1$ \\
& $W(4)$ &               $ 2^{2}, 4^{8} $ &     $ 2^{2}, 3, 4^{7} $ &     $ 1, 2^{2}, 3, 4^{7} $                              & $2$ & $-$ \\
& $W(1) \perp W(3)$ &        $ 1^{4}, 2^{2}, 3^{4}, 4^{4} $ & $ 1^{3}, 2^{2}, 3^{4}, 4^{4} $ & $ 1^{4}, 2^{2}, 3^{4}, 4^{4} $ & $0$ & $-$ \\
& $W(2)^2$ &             $ 1^{4}, 2^{16} $ &    $ 1^{5}, 2^{15} $ &    $ 1^{6}, 2^{15} $                                      & $1$ & $-$ \\
& $W(1)^2 \perp W(2)$ &      $ 1^{12}, 2^{12} $ &   $ 1^{11}, 2^{12} $ &   $ 1^{12}, 2^{12} $                                 & $0$ & $-$ \\
& $W(1)^4$ &             $ 1^{36} $ &           $ 1^{35} $ &           $ 1^{36} $                                             & $0$ & $-$ \\
&&&&&&\\ \hline
\end{tabular}
\end{table}


\begin{table}[!htbp]
\centering
\caption{For $G_{sc} = \Sp(V)$ simply connected of type $C_{\ell}$ with $2 \leq \ell \leq 4$ and $\chr K = 2$, Jordan block sizes of nilpotent $e \in \g_{sc}$ on $\g_{sc}$, $[\g_{ad}, \g_{ad}]$, and $\g_{ad}$. See Example \ref{example:tables}.}\label{table:NILexamples}
\begin{tabular}{llllll}
\hline
&&&&&\\[-10pt]
$\ell$ & $V \downarrow K[e]$          & $\g_{sc} \downarrow K[e]$ & $[\g_{ad}, \g_{ad}] \downarrow K[e]$ & $\g_{ad} \downarrow K[e]$ & $\alpha$ \\ \hline
&&&&&\\[-5pt]
$2$    & $V(4)$ &        $ 1^{2}, 4^{2} $ &     $ 1^{2}, 3, 4 $ &     $ 1, 2, 3, 4 $          & $2$ \\
       & $V(2)^2$ &      $ 1^{2}, 2^{4} $ &     $ 1^{3}, 2^{3} $ &     $ 1^{2}, 2^{4} $       & $1$ \\
       & $W(1)  \perp  V(2)$ & $ 1^{4}, 2^{3} $ &     $ 1^{3}, 2^{3} $ &     $ 1^{2}, 2^{4} $ & $0$ \\
       & $W(2)$ &        $ 1^{2}, 2^{4} $ &     $ 1^{3}, 2^{3} $ &     $ 1^{4}, 2^{3} $       & $1$ \\
       & $W(1)^2$ &      $ 1^{10} $ &           $ 1^{9} $ &           $ 1^{10} $              & $0$ \\
&&&&&\\	
	
$3$	   & $V(6)$ &        $ 1^{3}, 2, 8^{2} $ &  $ 1^{4}, 8^{2} $ &     $ 1^{3}, 2, 8^{2} $    & $1$ \\
& $W_1(3)$ &      $ 1^{5}, 4^{4} $ &     $ 1^{4}, 4^{4} $ &     $ 1^{3}, 2, 4^{4} $           & $0$ \\
& $V(2) \perp V(4)$ &   $ 1^{3}, 2, 4^{4} $ &  $ 1^{4}, 4^{4} $ &     $ 1^{3}, 2, 4^{4} $     & $1$ \\
& $W(1)^2 \perp V(2)$ & $ 1^{11}, 2^{5} $ &    $ 1^{10}, 2^{5} $ &    $ 1^{9}, 2^{6} $        & $0$ \\
& $W(1) \perp V(2)^2$ & $ 1^{5}, 2^{8} $ &     $ 1^{4}, 2^{8} $ &     $ 1^{3}, 2^{9} $        & $0$ \\
& $W(1) \perp V(4)$ &   $ 1^{5}, 4^{4} $ &     $ 1^{4}, 4^{4} $ &     $ 1^{3}, 2, 4^{4} $     & $0$ \\
& $V(2)^3$ &      $ 1^{3}, 2^{9} $ &     $ 1^{4}, 2^{8} $ &     $ 1^{3}, 2^{9} $              & $1$ \\
& $W(1) \perp W(2)$ &   $ 1^{5}, 2^{8} $ &     $ 1^{4}, 2^{8} $ &     $ 1^{5}, 2^{8} $        & $0$ \\
& $W(3)$ &        $ 1^{5}, 4^{4} $ &     $ 1^{4}, 4^{4} $ &     $ 1^{5}, 4^{4} $              & $0$ \\
& $W(1)^3$ &      $ 1^{21} $ &           $ 1^{20} $ &           $ 1^{21} $                    & $0$ \\
&&&&&\\

$4$  & $V(8)$ &           $ 1^{4}, 8^{4} $ &     $ 1^{4}, 7, 8^{3} $ &     $ 1^{3}, 2, 7, 8^{3} $                             & $3$ \\
     & $W_1(4)$ &         $ 1^{4}, 4^{8} $ &     $ 1^{4}, 3, 4^{7} $ &     $ 1^{3}, 2, 3, 4^{7} $                             & $2$ \\
     & $W(1) \perp W_1(3)$ &    $ 1^{6}, 2^{3}, 4^{6} $ & $ 1^{5}, 2^{3}, 4^{6} $ & $ 1^{4}, 2^{4}, 4^{6} $                   & $0$ \\
     & $V(2) \perp V(6)$ &      $ 1^{4}, 2^{2}, 6^{2}, 8^{2} $ & $ 1^{5}, 2, 6^{2}, 8^{2} $ & $ 1^{4}, 2^{2}, 6^{2}, 8^{2} $  & $1$ \\
     & $V(4)^2$ &         $ 1^{4}, 4^{8} $ &     $ 1^{4}, 3, 4^{7} $ &     $ 1^{3}, 2, 3, 4^{7} $                             & $2$ \\
     & $V(2)^2 \perp V(4)$ &    $ 1^{4}, 2^{4}, 4^{6} $ & $ 1^{5}, 2^{3}, 4^{6} $ & $ 1^{4}, 2^{4}, 4^{6} $                   & $1$ \\
     & $W(1) \perp V(6)$ &      $ 1^{6}, 2, 6^{2}, 8^{2} $ & $ 1^{5}, 2, 6^{2}, 8^{2} $ & $ 1^{4}, 2^{2}, 6^{2}, 8^{2} $      & $0$ \\
     & $W(1) \perp V(2) \perp V(4)$ & $ 1^{6}, 2^{3}, 4^{6} $ & $ 1^{5}, 2^{3}, 4^{6} $ & $ 1^{4}, 2^{4}, 4^{6} $             & $0$ \\
     & $W(1)^2 \perp V(4)$ &    $ 1^{12}, 4^{6} $ &    $ 1^{11}, 4^{6} $ &    $ 1^{10}, 2, 4^{6} $                            & $0$ \\
     & $W(1)^2 \perp V(2)^2$ &  $ 1^{12}, 2^{12} $ &   $ 1^{11}, 2^{12} $ &   $ 1^{10}, 2^{13} $                              & $0$ \\
     & $W(1)^3 \perp V(2)$ &    $ 1^{22}, 2^{7} $ &    $ 1^{21}, 2^{7} $ &    $ 1^{20}, 2^{8} $                               & $0$ \\
     & $W(2) \perp V(4)$ &      $ 1^{4}, 2^{4}, 4^{6} $ & $ 1^{5}, 2^{3}, 4^{6} $ & $ 1^{4}, 2^{4}, 4^{6} $                   & $1$ \\
     & $V(2)^4$ &         $ 1^{4}, 2^{16} $ &    $ 1^{5}, 2^{15} $ &    $ 1^{4}, 2^{16} $                                     & $1$ \\
     & $W(1) \perp V(2)^3$ &    $ 1^{6}, 2^{15} $ &    $ 1^{5}, 2^{15} $ &    $ 1^{4}, 2^{16} $                               & $0$ \\
		 & $V(2) \perp W(3)$ &      $ 1^{6}, 2^{3}, 4^{6} $ & $ 1^{5}, 2^{3}, 4^{6} $ & $ 1^{4}, 2^{4}, 4^{6} $                   & $0$ \\
     & $W(1)^2 \perp W(2)$ &    $ 1^{12}, 2^{12} $ &   $ 1^{11}, 2^{12} $ &   $ 1^{12}, 2^{12} $                              & $0$ \\
     & $W(2)^2$ &         $ 1^{4}, 2^{16} $ &    $ 1^{5}, 2^{15} $ &    $ 1^{6}, 2^{15} $                                     & $1$ \\
     & $W(1) \perp W(3)$ &      $ 1^{8}, 3^{4}, 4^{4} $ & $ 1^{7}, 3^{4}, 4^{4} $ & $ 1^{8}, 3^{4}, 4^{4} $                   & $0$ \\
     & $W(4)$ &           $ 1^{4}, 4^{8} $ &     $ 1^{4}, 3, 4^{7} $ &     $ 1^{5}, 3, 4^{7} $                                & $2$ \\
     & $W(1)^4$ &         $ 1^{36} $ &           $ 1^{35} $ &           $ 1^{36} $                                            & $0$ \\
&&&&&\\ \hline
\end{tabular}
\end{table}









