
Let $q = 2^{\alpha}$ for some $\alpha \geq 0$. In this section, we state some basic results about the decomposition of tensor products, exterior squares, and symmetric squares of $K[C_q]$-modules and $\mathfrak{w}_q$-modules.

\subsection{Unipotent case} Let $u$ be a generator for the cyclic group $C_q$. To describe the indecomposable summands of tensor products $V \otimes W$ of $K[C_q]$-modules, it is clear that it suffices to do so in the case where $V$ and $W$ are indecomposable. 

The decomposition of $V_m \otimes V_n$ into indecomposable summands has been studied extensively in all characteristics, see for example (in chronological order) \cite{Srinivasan}, \cite{Ralley}, \cite{McFall}, \cite{Renaud}, \cite{Norman}, \cite{Hou}, \cite{Norman2}, \cite{IimaIwamatsu2009}, and \cite{Barry}. Although there is no closed formula in general, there are various recursive formulae which can be used to determine the indecomposable summands efficiently. We are assuming $\chr K = 2$, in which case we have the following result.

\begin{lause}[{\cite[(2.5a)]{GreenModular}, \cite[Lemma 1, Corollary 3]{GowLaffey}}]\label{thm:tensordecompchar2}
Let $0 < m \leq n \leq q$ and suppose that $q/2 < n \leq q$. Then the following statements hold:
\begin{enumerate}[\normalfont (i)]
\item If $n = q$, then $V_m \otimes V_n \cong V_q^m$ as $K[C_q]$-modules.
\item If $m+n > q$, then $V_m \otimes V_n \cong V_q^{n+m-q} \oplus (V_{q-n} \otimes V_{q-m})$ as $K[C_q]$-modules.
\item If $m+n \leq q$, then $V_m \otimes V_n \cong V_{q-d_t} \oplus \cdots \oplus V_{q-d_1}$ as $K[C_q]$-modules, where $V_m \otimes V_{q-n} \cong V_{d_1} \oplus \cdots \oplus V_{d_t}$.
\end{enumerate}
\end{lause}

With Theorem \ref{thm:tensordecompchar2}, every tensor product $V_m \otimes V_n$ is either described explicitly (case (i)), or in terms of another tensor product $V_{m'} \otimes V_{n'}$ with $0 < m' \leq n' < n$.  Thus by repeated applications of Theorem \ref{thm:tensordecompchar2}, we can rapidly find the indecomposable summands of $V_m \otimes V_n$ for any given $m$ and $n$.

For exterior squares and symmetric squares, similarly it suffices to consider the indecomposable case. This follows from the fact that for all $K[C_q]$-modules $V$ and $W$, we have isomorphisms \begin{align*} \wedge^2(V \oplus W) &\cong \wedge^2(V) \oplus \wedge^2(W) \oplus V \otimes W \\ S^2(V \oplus W) &\cong S^2(V) \oplus S^2(W) \oplus V \otimes W\end{align*} as $K[C_q]$-modules. For the decomposition of $\wedge^2(V_n)$ and $S^2(V_n)$, we have the following results.

\begin{lause}[{\cite[Theorem 2]{GowLaffey}}]\label{thm:unipext1}
Suppose that $q/2 < n \leq q$. Then we have $$\wedge^2(V_n) \cong \wedge^2(V_{q-n}) \oplus V_q^{n-q/2-1} \oplus V_{3q/2-n}$$ as $K[C_q]$-modules.
\end{lause}

\begin{lause}[{\cite[Theorem 1.3]{KorhonenSymExt2021}}]\label{thm:unipsym1}
Suppose that $q/2 < n \leq q$. Then we have $$S^2(V_n) \cong \wedge^2(V_{q-n}) \oplus V_q^{n-q/2} \oplus V_{q/2}$$ as $K[C_q]$-modules.
\end{lause}

Similarly to Theorem \ref{thm:tensordecompchar2}, with Theorem \ref{thm:unipext1} and Theorem \ref{thm:unipsym1} we can quickly decompose $\wedge^2(V_n)$ and $S^2(V_n)$ for any given $n$.

\begin{lemma}\label{lemma:fixpdimsymwedge}
Let $n > 0$ be an integer. Then the following hold:
	\begin{enumerate}[\normalfont (i)]
		\item $\dim \wedge^2(V_n)^u = \lfloor \frac{n}{2} \rfloor$.
		\item $\dim S^2(V_n)^u = \lfloor \frac{n}{2} \rfloor + 1$.
	\end{enumerate}
\end{lemma}

%\cite[Example 4.9]{Korhonen2020Hesselink}
\begin{proof}
We first consider (i). If $n = 1$, then $\wedge^2(V_1) = 0$ has dimension $0$ and thus (i) holds. Suppose then that $n > 1$ and proceed by induction on $n$. Let $q$ be a power of two such that $q/2  < n \leq q$. If $n = q$, it follows from Theorem \ref{thm:unipext1} that $\wedge^2(V_n) \cong V_q^{n-q/2-1} \oplus V_{q/2}$, so $\dim \wedge^2(V_n)^u = n - q/2 = n/2$. If $q/2 < n < q$, we have $$\wedge^2(V_n) \cong \wedge^2(V_{q-n}) \oplus V_{q}^{n-q/2-1} \oplus V_{3q/2-n}$$ by Theorem \ref{thm:unipext1}. Then by induction $$\dim \wedge^2(V_n)^u = \left\lfloor \frac{q-n}{2} \right\rfloor + n-q/2 = \left\lfloor \frac{n}{2} \right\rfloor,$$ as claimed by (i).

Next we will prove (ii). If $n = 1$, then $S^2(V_1) = V_1$ so (ii) holds. Suppose then that $n > 1$, and let $q$ be a power of two such that $q/2 < n \leq q$. If $n = q$, then $S^2(V_n) \cong V_q^{n-q/2} \oplus V_{q/2}$, and thus $\dim S^2(V_n)^u = n-q/2+1 = n/2+1$. Suppose then that $q/2 < n < q$. It follows from Theorem \ref{thm:unipsym1} and (i) that $$\dim S^2(V_n)^u = \left\lfloor \frac{q-n}{2} \right\rfloor + n-q/2+1 = \left\lfloor \frac{n}{2} \right\rfloor + 1,$$ as claimed by (ii).\end{proof}

\begin{lemma}\label{lemma:smallesblockunipS2}
Let $\ell > 0$ and define $\alpha = \nu_2(\ell)$. Then the smallest Jordan block size in $S^2(V_{2\ell})$ is $2^{\alpha}$, occurring with multiplicity one.
\end{lemma}

\begin{proof} (cf. \cite[Lemma 4.12]{Korhonen2020Hesselink}) If $\ell = 2^{\alpha}$, it follows from Theorem \ref{thm:unipsym1} that $S^2(V_{2\ell}) = V_{2^{\alpha+1}}^{2^{\alpha}} \oplus V_{2^{\alpha}}$, so the claim holds. If $\ell \neq 2^{\alpha}$, we have $q/2 < 2\ell < q$ for some $q = 2^{\beta}$. Then \begin{equation}\label{eq:s2decompinlemma}S^2(V_{2\ell}) = \wedge^2(V_{q-2\ell}) \oplus V_q^{2\ell-q/2} \oplus V_{q/2}\end{equation} by Theorem \ref{thm:unipsym1}. 

Now $\nu_2(q-2\ell) = 2^{\alpha+1}$ since $q > 2^{\alpha+1}$, so by \cite[Lemma 4.12]{Korhonen2020Hesselink} the smallest Jordan block size in $\wedge^2(V_{q-2\ell})$ is $2^{\alpha}$, occurring with multiplicity one. Furthermore $2^{\alpha} < q/2$ because $q/2 < 2\ell < q$, so the lemma follows from~\eqref{eq:s2decompinlemma}.\end{proof}

\subsection{Nilpotent case} As in the previous section, to determine the indecomposable summands of tensor products, exterior squares, and symmetric squares of $\mathfrak{w}_q$-modules, it suffices to do so in the indecomposable case.

For the indecomposable summands of $W_m \otimes W_n$, it turns out that we get the same decomposition as for $V_m \otimes V_n$ in the unipotent case.

\begin{prop}[{\cite[Section III]{Fossum}, \cite[Corollary 5 (a)]{NormanTwoRelated}}]\label{prop:uninilsim}
Let $0 < n,m \leq q$ and suppose that we have $V_m \otimes V_n \cong V_{r_1} \oplus \cdots \oplus V_{r_t}$ as $K[C_q]$-modules for some $r_1, \ldots, r_t > 0$. Then $W_m \otimes W_n \cong W_{r_1} \oplus \cdots \oplus W_{r_t}$ as $\mathfrak{w}_q$-modules.
\end{prop}

Thus we can apply Theorem \ref{thm:tensordecompchar2} to find the decomposition of $W_m \otimes W_n$ into indecomposable summands.

Following \cite[p. 231]{GlasbyPraegerXiapart}, we call the \emph{consecutive-ones binary expansion} of an integer $n > 0$ the alternating sum $n = \sum_{1 \leq i \leq r} (-1)^{i+1} 2^{\beta_i}$ such that $\beta_1 > \cdots > \beta_r \geq 0$ and $r$ is minimal. (Here $\beta_{r-1} > \beta_r + 1$ if $r > 1$.)

The decomposition of $V_n \otimes V_n$ into indecomposable summands can be given explicitly in terms of the consecutive-ones binary expansion of $n$ \cite[Theorem 15]{GlasbyPraegerXiapart}. Such descriptions can also be given for $\wedge^2(V_n)$ and $S^2(V_n)$, by using Theorem \ref{thm:unipext1} and Theorem \ref{thm:unipsym1}. 

For the decomposition of $W_n \otimes W_n$, $\wedge^2(W_n)$, and $S^2(W_n)$ we have the following result.

\begin{lause}[{\cite[Theorem 1.6, Theorem 1.7, Theorem 3.7]{KorhonenSymExt2021}}]\label{thm:extsymnilpotent}
Let $n > 0$ be an integer, with consecutive-ones binary expansion $n = \sum_{1 \leq i \leq r} (-1)^{i+1} 2^{\beta_i}$, where $\beta_1 > \cdots > \beta_r \geq 0$. For $1 \leq k \leq r$ , define $d_k := 2^{\beta_k} + \sum_{k < i \leq r} (-1)^{k+i} 2^{\beta_i+1}$. Then \begin{align*}W_n \otimes W_n &\cong \bigoplus_{1 \leq k \leq r} W_{2^{\beta_k}}^{d_k} \\ \wedge^2(W_n) &\cong \bigoplus_{\substack{1 \leq k \leq r \\ \beta_k > 0}} W_{2^{\beta_k}-1}^{d_k/2} \\ S^2(W_n) &\cong W_1^{\lceil n/2 \rceil} \oplus \bigoplus_{\substack{1 \leq k \leq r \\ \beta_k > 0}} W_{2^{\beta_k}}^{d_k/2}\end{align*} as $\mathfrak{w}_q$-modules.\end{lause}


