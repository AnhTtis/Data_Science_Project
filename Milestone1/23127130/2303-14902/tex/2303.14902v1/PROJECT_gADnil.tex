
Continuing with the setup of the previous section, in this section we describe the Jordan block sizes of nilpotent $e \in \g_{sc}$ on $\g_{ad}$, in terms of the Jordan block sizes on $[\g_{ad},\g_{ad}]$. The basic approach is similar to the previous section, but the proofs will be more simple due to the fact that $V \downarrow K[e^2]$ is always has an orthogonal decomposition of the form $\sum_{1 \leq i \leq t} W(m_i)$ (Lemma \ref{lemma:nilsquaredecomp}).

\begin{lemma}\label{lemma:NILgadsingular}
Let $e \in \g_{sc}$ be nilpotent such that $V \downarrow K[e] = \sum_{1 \leq i \leq t} W(m_i)$, and let $\widetilde{e}$ be the action of $e$ on $\g_{ad}$. Then $\Ker \widetilde{e} \not\subseteq [\g_{ad}, \g_{ad}]$.
\end{lemma}

\begin{proof}
Let $e_{\Z} = e_{\Z,1} + \cdots + e_{\Z,t} \in \g_{\Z}$ be as in Section \ref{subsection:nilpotentroots}, so that $e$ is the reduction modulo $p$ of $e_{\Z}$. Then $$e_{\Z,i} = X_{\alpha_{t_i+1}} + \cdots + X_{\alpha_{t_i+m_i-1}},$$ where $t_1 = 0$ and $t_i = m_1 + \cdots + m_{i-1}$ for all $1 < i \leq t$. 

As noted earlier (proof of Lemma \ref{lemma:deltaWLunip}), we have $X_{\alpha_i} \cdot \delta = 0$ for all $1 \leq i < \ell$. Therefore $e_{\Z} \cdot \delta = 0$, from which it follows that $\Ker \widetilde{e} \not\subseteq [\g_{ad}, \g_{ad}]$ (Lemma \ref{lemma:nilpinUZgen}).
\end{proof}

\begin{lemma}\label{lemma:NILgadV2lWkl}
Let $e \in \g_{sc}$ be nilpotent with $V \downarrow K[e] = V(2\ell)$ or $V \downarrow K[e] = W_k(\ell)$ for some $0 < k < \ell/2$. Let $\widetilde{e}$ be the action of $e$ on $\g_{ad}$. Then $\Ker \widetilde{e} \subseteq [\g_{ad}, \g_{ad}]$.
\end{lemma}

\begin{proof}
Let $e_{\Z} \in \g_{\Z}$ be as in Section \ref{subsection:nilpotentroots}, so that $e$ is the reduction modulo $p$ of $e_{\Z}$. Then $$e_{\Z} = X_{\alpha_1} + \cdots + X_{\alpha_{\ell-1}} + X_{2 \varepsilon_r},$$ where $r = \ell$ if $V \downarrow K[e] = V(2\ell)$, and $r = k$ if $V \downarrow K[e] = W_k(\ell)$.

Suppose that $\Ker \widetilde{e} \not\subseteq [\g_{ad}, \g_{ad}]$. Then it follows from Lemma \ref{lemma:nilpinUZgen} that there exists $v \in L_{sc}$ such that $e_{\Z} \cdot (\delta + v) \in 2L_{ad}$. We have $e_{\Z} \cdot \delta = \frac{1}{2} v_r^2,$ so \begin{equation}\label{eq:12vr2}\frac{1}{2} v_r^2 = -e_{\Z} \cdot v + 2w\end{equation} for some $w \in L_{ad}$.

Note that $$e_{\Z} \cdot \frac{1}{2} v_i^2 = \pm v_i v_{i-1}$$ for all $1 \leq i \leq n$. Therefore $e_{\Z} L_{sc} \subseteq S^2(V_{\Z})$. Since also $2L_{ad} \subseteq S^2(V_{\Z})$, it follows from~\eqref{eq:12vr2} that $\frac{1}{2} v_r^2 \in S^2(V_{\Z})$, which is impossible. We have a contradiction, so we conclude that $\Ker \widetilde{e} \subseteq [\g_{ad}, \g_{ad}]$.\end{proof}

\begin{lemma}\label{lemma:nonsingularkernelNILGAD}
Let $e \in \g_{sc}$ be nilpotent, and let $\widetilde{e}$ be the action of $e$ on $\g_{ad}$. Then $\Ker \widetilde{e} \subseteq [\g_{ad}, \g_{ad}]$, except possibly when $V \downarrow K[e] = \sum_{1 \leq i \leq t} W(\ell_i)$ for some $\ell_1$, $\ldots$, $\ell_t$.\end{lemma}

\begin{proof}
Write $V \downarrow K[e] = U_1 \perp \cdots \perp U_t$, where $U_i$ is orthogonally indecomposable for all $1 \leq i \leq t$, and $\dim U_i = 2\ell_i$ with $\ell_i > 0$. Let $e_{\Z} = e_{\Z,1} + \cdots + e_{\Z,t} \in \g_{\Z}$ be as in Section \ref{subsection:nilpotentroots}, so that $e$ is the reduction modulo $p$ of $e_{\Z}$. 

The result follows from Lemma \ref{lemma:GADreduceNIL} (ii) (with $X_i = e_{\Z,i}$) and Lemma \ref{lemma:NILgadV2lWkl}.\end{proof}

\begin{lemma}\label{lemma:nonsingularkernel2NILGAD}
Let $e \in \g_{sc}$ be nilpotent, and let $\widetilde{e}$ be the action of $e$ on $\g_{ad}$. Then $\Ker (\widetilde{e})^2 \not\subseteq [\g_{ad}, \g_{ad}]$.
\end{lemma}

\begin{proof}
We have $V \downarrow K[e^2] = \sum_{1 \leq i \leq t} W(m_i)$ by Lemma \ref{lemma:nilsquaredecomp}, so the result follows from Lemma \ref{lemma:NILgadsingular}.
\end{proof}

\begin{prop}\label{prop:nilgadaction}
Let $e \in \g_{sc}$ be nilpotent. Then the following hold:

	\begin{enumerate}[\normalfont (i)]
		\item If $V \downarrow K[e] = \sum_{1 \leq i \leq t} W(m_i)$, then $$\g_{ad} \cong W_1 \oplus [\g_{ad}, \g_{ad}]$$ as $K[e]$-modules.

		\item If $V \downarrow K[e]$ is not of the form $\sum_{1 \leq i \leq t} W(m_i)$, then \begin{align*}
		\g_{ad} &\cong W_{2} \oplus V' \\
		[\g_{ad},\g_{ad}] &\cong W_{1} \oplus V' \end{align*} for some $K[e]$-module $V'$.
	\end{enumerate}
	
\end{prop}

\begin{proof}In case (i), the claim follows from Lemma \ref{lemma:NILgadsingular} and Lemma \ref{jordanrestrictionNIL}. Similarly (ii) follows from Lemma \ref{lemma:nonsingularkernelNILGAD}, Lemma \ref{lemma:nonsingularkernel2NILGAD}, and Lemma \ref{jordanrestrictionNIL}.\end{proof}

