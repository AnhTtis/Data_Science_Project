
 We will use the following notation and terminology throughout the paper.

\subsection{Generalities} We will always assume that $K$ is an algebraically closed field of characteristic two.

For a $K$-vector space $V$ and integer $n > 0$, we denote $V^n = V \oplus \cdots \oplus V$, where $V$ occurs $n$ times. Furthermore, we denote $V^0 = 0$.

We will always use $G$ to denote a group, and all the $K[G]$-modules that we consider will be finite-dimensional. Suppose that $V = U_1 \supset U_2 \supset \cdots \supset U_t \supset U_{t+1} = 0$ is the socle filtration of a $K[G]$-module $V$, so $Z_i := U_i/U_{i+1} = \soc(V/U_{i+1})$ for all $1 \leq i \leq t$. Then we will denote this by $V = Z_1|Z_2|\cdots|Z_t$, and $V$ is said to be \emph{uniserial} if $Z_i$ is irreducible for all $1 \leq i \leq t$.

We denote by $\nu_2$ the $2$-adic valuation on the integers, so $\nu_2(a)$ is the largest integer $k \geq 0$ such that $2^k$ divides $a$.


\subsection{Algebraic groups} Suppose that $G$ is a simple linear algebraic group over $K$. In the context of algebraic groups, the notation that we use will be as in \cite{JantzenBook}. We will denote the Lie algebra of $G$ by $\g$.

When $G$ is an algebraic group, by a $G$-module we will always mean a finite-dimensional rational $K[G]$-module. Fix a maximal torus $T$ of $G$ with character group $X(T)$, and a base $\Delta = \{ \alpha_1, \ldots, \alpha_{\ell} \}$ for the root system of $G$, where $\ell = \operatorname{rank} G$. We will always use the standard Bourbaki labeling of the simple roots $\alpha_i$, as given in \cite[11.4, p. 58]{Humphreys}. We denote the dominant weights with respect to $\Delta$ by $X(T)^+$, and the fundamental dominant weight corresponding to $\alpha_i$ is denoted by $\varpi_i$. For a dominant weight $\lambda \in X(T)^+$, we denote the rational irreducible $K[G]$-module with highest weight $\lambda$ by $L_G(\lambda)$.

An element $u \in G$ is \emph{unipotent}, if $f(u)$ is a unipotent linear map for every rational representation $f: G \rightarrow \GL(W)$. Since $\chr K = 2$, equivalently $u$ is unipotent if and only if $u$ has order $2^{\alpha}$ for some $\alpha \geq 0$.

Similarly $e \in \g$ is said to be \emph{nilpotent}, if $\D f(e)$ is a nilpotent linear map for every rational representation $f: G \rightarrow \GL(W)$. Equivalently $e$ is nilpotent if and only if $e^{[2^\alpha]} = 0$ for some $\alpha > 0$, where $x \mapsto x^{[2]}$ is the canonical $p$-mapping on $\g$.

We will mostly be concerned with the case where $G = \Sp(V)$, in which case $G$ is of type $C_{\ell}$, where $\dim V = 2\ell$. In this case $u \in \Sp(V)$ is unipotent if and only if $u$ is a unipotent linear map on $V$. Furthermore $\g = \mathfrak{sp}(V)$, and $e \in \g$ is nilpotent if and only if $e$ is a nilpotent linear map on $V$.

\subsection{Actions of unipotent elements}\label{section:defunipotent} 

Let $q = 2^{\alpha}$ for some $\alpha \geq 0$. Then a cyclic group $C_q = \langle u \rangle$ has $q$ indecomposable $K[C_q]$-modules, up to isomorphism. We denote them by $V_1$, $\ldots$, $V_q$, where $\dim V_i = i$ and $u$ acts on $V_i$ as a single $i \times i$ unipotent Jordan block. For convenience we denote $V_0 = 0$.

Let $u$ be an element of a group $G$ and let $V$ be a finite-dimensional $K[G]$-module on which $u$ acts as a unipotent linear map. We denote $V^u := \{v \in V : uv = v\}$, which is the fixed point space of $u$ on $V$. Note that $\dim V^u$ is the number of Jordan blocks of $u$ on $V$. 

If $u \in \GL(V)$ is unipotent, we denote by $K[u]$ the group algebra of $\langle u \rangle$. Then $V \downarrow K[u] = V_{d_1}^{n_1} \oplus \cdots \oplus V_{d_t}^{n_t}$, where $0 < d_1 < \cdots < d_t$ are the Jordan block sizes of $u$ on $V$, and $n_i$ is the multiplicity of a Jordan block of size $d_i$.

\subsection{Actions of nilpotent elements}\label{section:defnilpotent} Let $q = 2^{\alpha}$ for some $\alpha \geq 0$. We denote by $\mathfrak{w}_q$ the abelian $2$-Lie algebra over $K$ generated by a nilpotent element $e$ with $e^{[2^{\alpha}]} = 0$ and $e^{[2^{\alpha-1}]} \neq 0$. Then as a $K$-vector space $$\mathfrak{w}_q = \bigoplus_{0 \leq i < \alpha} \langle e^{[2^i]} \rangle.$$ There are a total of $q$ indecomposable $\mathfrak{w}_q$-modules, up to isomorphism. We denote them by $W_1$, $\ldots$, $W_q$, where $\dim W_i = i$ and $e$ acts on $W_i$ as a single $i \times i$ nilpotent Jordan block. For convenience we denote $W_0 = 0$.

If $V$ is a finite-dimensional module for a Lie algebra $\g$ and $e \in \g$ acts on $V$ as a nilpotent element, we denote $V^e := \{v \in V : ev = 0\}$. Then $\dim V^e$ is the number of Jordan blocks of $e$ on $V$.

If $e \in \mathfrak{gl}(V)$ is nilpotent, we denote by $K[e]$ the $2$-Lie algebra generated by $e$. Then $K[e] \cong \mathfrak{w}_q$, where $q$ is the smallest power of two such that $e^q = 0$. Furthermore $V \downarrow K[e] = W_{d_1}^{n_1} \oplus \cdots \oplus W_{d_t}^{n_t}$, where $0 < d_1 < \cdots < d_t$ are the Jordan block sizes of $e$ on $V$, and $n_i$ is the multiplicity of a Jordan block of size $d_i$.



