
In the notation of Section \ref{subsection:chevalleySP}, let $G_{sc} = \Sp(V)$ be a simply connected simple algebraic group of type $C_{\ell}$, with Lie algebra $\g_{sc} = \mathfrak{sp}(V)$. In this section, we will make some initial observations about the structure of $\g_{sc}$ and $\g_{ad}$ as $G_{sc}$-modules.

\begin{lemma}\label{lemma:LieSpS2}
Let $G_{sc} = \Sp(V)$ with Lie algebra $\g_{sc} = \mathfrak{sp}(V)$. Then $\g_{sc} \cong S^2(V)^*$ as $G_{sc}$-modules.
\end{lemma}

\begin{proof}A proof is given in \cite[5.4]{DowdSin}, alternatively this follows from the isomorphism $K \otimes_{\Z} L_{sc} \cong \g_{sc}$ given in Section \ref{subsection:symsquareC}.\end{proof}

\begin{lemma}\label{lemma:gmodZisgadgad}
We have $\g_{sc}/Z(\g_{sc}) \cong [\g_{ad}, \g_{ad}]$ as $G_{sc}$-modules.
\end{lemma}

\begin{proof}Let $G_{ad}$ be a group of adjoint type $C_{\ell}$, and let $\psi: G_{sc} \rightarrow G_{ad}$ be an isogeny as in Lemma \ref{lemma:GSCtoGAD}. Then the map $\D \psi: \g_{sc} \rightarrow \g_{ad}$ is a morphism of $G_{sc}$-modules, with $\Ker \D \psi = Z(\g_{sc})$ \cite[Lemma 2.2]{Hogeweij}. The image of $\D \psi$ contains all root elements of $\g_{ad}$, so by \cite[Table 1]{Hogeweij} it is equal to $[\g_{ad}, \g_{ad}]$. Therefore $\g_{sc}/ \Ker \D \psi = \g_{sc}/Z(\g_{sc}) \cong [\g_{ad}, \g_{ad}]$ as $G_{sc}$-modules.\end{proof}

Let $\varphi$ the linear map $\varphi: S^2(V) \rightarrow K$ defined by $\varphi(xy) = b(x,y)$ for all $x,y \in V$. Since $b$ is $G_{sc}$-invariant, it follows that $\varphi$ is a surjective morphism of $G_{sc}$-modules. 

\begin{lemma}\label{lemma:kerphiisdual}
Let $G_{sc} = \Sp(V)$ with Lie algebra $\g_{sc} = \mathfrak{sp}(V)$. Then $\Ker \varphi \cong \left( \g_{sc}/Z(\g_{sc}) \right)^*$ as $G_{sc}$-modules. 
\end{lemma}

\begin{proof}
It is clear that $\varphi$ is surjective, so we have a short exact sequence $$0 \rightarrow \Ker \varphi \rightarrow S^2(V) \rightarrow K \rightarrow 0$$ of $G_{sc}$-modules. This induces a short exact sequence $$0 \rightarrow K \rightarrow S^2(V)^* \rightarrow \left( \Ker \varphi \right)^* \rightarrow 0$$ of $G_{sc}$-modules. Now $S^2(V)^* \cong \g_{sc}$ as $G_{sc}$-modules (Lemma \ref{lemma:LieSpS2}), and $\g_{sc}$ has a unique trivial submodule since $Z(\g_{sc}) \cong K$ \cite[Table 1]{Hogeweij}. Thus we conclude that $\left( \Ker \varphi \right)^* \cong \g_{sc}/Z(\g_{sc})$ as $G_{sc}$-modules, from which the lemma follows.
\end{proof}

\begin{lemma}\label{lemma:uniqueuniserial}
Let $G = \Sp(V)$ be simply connected and simple of type $C_{\ell}$. Then:

	\begin{enumerate}[\normalfont (i)]
		\item There exists a uniserial $G$-module $W$ with $W = L_G(0)|L_G(2\varpi_1)|L_G(0)$.
		\item A $G$-module $W$ as in (i) is unique up to isomorphism.
		\item Let $u \in G$ be a unipotent element. Then $$\dim W^u = \begin{cases} \dim V^u + 1,& \text{ if } \dim V^u \text{ is odd.}\\ \dim V^u + 2,& \text{ if } \dim V^u \text{ is even.} \end{cases}$$
	\end{enumerate}
\end{lemma}

\begin{proof}
Let $G' = \SO(V')$ with $\dim V' = 2\ell+1$, so $G'$ is a simple algebraic group of adjoint type $B_{\ell}$. Let $\tau: G \rightarrow G'$ be an exceptional isogeny as in \cite[Theorem 28]{SteinbergNotesAMS}. We can embed $G'$ into a simple algebraic group $\SO(W)$ of type $D_{\ell+1}$ as the stabilizer of a nonsingular vector, see for example \cite[Section 6.8]{LiebeckSeitzClass}. Here $\dim W = 2\ell+2$, and as in \cite[Section 6.8]{LiebeckSeitzClass}, we identify $V' = \langle v \rangle^\perp \subset W$, where $v \in W$ is a nonsingular vector. 

It is straightforward to see that $W$ is a uniserial $G'$-module with $$W = L_{G'}(0)|L_{G'}(\varpi_1')|L_{G'}(0),$$ where $\varpi_1'$ is the first fundamental highest weight for $G'$. Then the twist of $W$ by $\tau$ is a uniserial $G$-module $W^\tau$ as in (i). For a unipotent element $u \in G$, the fixed point space of $u$ on the Frobenius twist $L_G(2\varpi_1) \cong L_G(\varpi_1)^{[1]}$ is the same as on $L_G(\varpi_1) \cong V$, because the Frobenius endomorphism preserves unipotent conjugacy classes. Therefore (iii) holds for $W$ by \cite[Lemma 3.8]{KorhonenUP}.

It remains to check that $W$ is unique. To this end, note first that $$\Ext_G^1(K, L_G(2 \varpi_1)) \cong \Hom_G(\wedge^2(V)^*, K)$$ \cite[II.2.14]{JantzenBook} and \cite[5.4]{DowdSin}. Here $\Hom_G(\wedge^2(V)^*, K) \cong \wedge^2(V)^G \cong K$, since $V$ has a unique $G$-invariant alternating bilinear form up to a scalar. Thus there exists a unique nonsplit extension $$0 \rightarrow K \rightarrow Z \rightarrow L_G(2\varpi_1) \rightarrow 0,$$ up to isomorphism of $G$-modules. 

Since $\Ext_G^1(K,K) = \Ext_G^2(K,K) = 0$ \cite[II.4.11]{JantzenBook} and $\Ext_G^1(K, L_G(2\varpi_1)) \cong K$, we have $\Ext_G^1(K,Z) \cong K$. Hence there exists a unique nonsplit extension $$0 \rightarrow Z \rightarrow W \rightarrow K \rightarrow 0,$$ up to isomorphism of $G$-modules. Every $W$ as in (i) is such an extension, so we conclude that $W$ is unique up to isomorphism.\end{proof}


\begin{lemma}\label{lemma:exactseqvarpi2gAD}
Let $G_{sc} = \Sp(V)$, so $G_{sc}$ is simply connected and simple of type $C_{\ell}$. Assume that $\ell$ is even. Then there is a short exact sequence $$0 \rightarrow L_G(\varpi_2) \rightarrow \g_{ad} \rightarrow W \rightarrow 0$$ of $G_{sc}$-modules, where $W$ is as in Lemma \ref{lemma:uniqueuniserial} (i).
\end{lemma}

\begin{proof}
As observed in \cite[5.6]{DowdSin}, in this case as a $G_{sc}$-module $\g_{ad}$ is uniserial with $\g_{ad} = L_G(\varpi_2)|L_G(0)|L_G(2\varpi_1)|L_G(0)$. Thus the result follows from Lemma \ref{lemma:uniqueuniserial} (ii).
\end{proof}

\begin{lemma}\label{lemma:regularl2mod4ineq}
Let $G = \Sp(V)$ be simply connected and simple of type $C_{\ell}$. Assume that $\ell \equiv 2 \mod{4}$ and let $u \in G$ be unipotent with $V \downarrow K[u] = V(2\ell)$. Then $\dim \g_{ad}^u \leq \dim \g_{sc}^u$.
\end{lemma}

\begin{proof}
It follows from Lemma \ref{lemma:fixpdimsymwedge} that $\dim \wedge^2(V)^u = \ell$. Because $\ell \equiv 2 \mod{4}$, the smallest Jordan block size of $u$ on $\wedge^2(V)$ is equal to $2$ \cite[Lemma 4.12]{Korhonen2020Hesselink}, and by \cite[Theorem B]{Korhonen2020Hesselink} we have $\dim L_G(\varpi_2)^u = \ell - 1$. 

For $W$ as in Lemma \ref{lemma:exactseqvarpi2gAD} (i), we have $\dim W^u = 2$ by Lemma \ref{lemma:exactseqvarpi2gAD} (iii). Thus it follows from Lemma \ref{lemma:exactseqvarpi2gAD} that $\dim \g_{ad}^u \leq \dim L_G(\varpi_2)^u + \dim W^u = \ell+1$. By Lemma \ref{lemma:LieSpS2} and Lemma \ref{lemma:fixpdimsymwedge} (ii) we have $\dim \g_{sc}^u = \ell+1$, so the result follows.\end{proof}

\begin{remark}
As a corollary of our results, we will see later that equality $\dim \g_{ad}^u = \dim \g_{sc}^u$ holds in Lemma \ref{lemma:regularl2mod4ineq}, without assumptions on $\ell$ (Corollary \ref{cor:regularcentralizerdim}).
\end{remark}



