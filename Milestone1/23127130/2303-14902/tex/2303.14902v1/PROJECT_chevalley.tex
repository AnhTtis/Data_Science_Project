
In this section, we will recall basics of the Chevalley construction of Lie algebras and simple algebraic groups, and in particular how it applies for groups of type $C_{\ell}$ in characteristic two. For more details, see for example \cite{SteinbergNotesAMS} or \cite[Chapter VII]{Humphreys}. We will also make some preliminary observations about the actions of unipotent and nilpotent elements on the adjoint Lie algebra $\g_{ad}$ of type $C_{\ell}$. 

\subsection{Chevalley construction}  Let $\mathfrak{g}$ be a finite-dimensional simple Lie algebra over $\C$. Fix a Cartan subalgebra $\mathfrak{h}$ of $\mathfrak{g}$, and let $\Phi$ be the corresponding root system, so $$\mathfrak{g} = \mathfrak{h} \oplus \bigoplus_{\alpha \in \Phi} \mathfrak{g}_\alpha,$$ where $\mathfrak{g}_\alpha = \{ X \in \mathfrak{g} : [H, X] = \alpha(H)X \text{ for all } H \in \mathfrak{h}. \}.$ The Killing form $\kappa$ is non-degenerate on $\mathfrak{h}$, so for all $\alpha \in \Phi$ there exists $H_\alpha' \in \mathfrak{h}$ such that $\kappa(H, H_\alpha') = \alpha(H)$ for all $H \in \mathfrak{h}$. We define $H_\alpha := \frac{2}{\kappa(H_\alpha', H_\alpha')} H_\alpha'$ for all $\alpha \in \Phi$. 

It was shown by Chevalley \cite[Th\'eor\`eme 1]{ChevalleyTohoku} that one can choose $X_\alpha \in \mathfrak{g}_\alpha$ such that the following properties hold:		

\begin{enumerate}[(a)]
\item $[X_\alpha, X_{-\alpha}] = H_{\alpha}$ for all $\alpha \in \Phi$,
\item If $\alpha, \beta \in \Phi$ and $\alpha+\beta \in \Phi$, then $[X_\alpha, X_\beta] = \pm(r+1) X_{\alpha+\beta}$, where $r \geq 0$ is the largest integer such that $\beta-r\alpha$ is a root.
\item If $\alpha, \beta \in \Phi$ and $\alpha+\beta \not\in \Phi$, then $[X_\alpha, X_\beta] = 0$.
\end{enumerate}

For a choice of root vectors $X_\alpha$ satisfying (a) -- (c) above, we define $\mathfrak{g}_\Z$ to be the $\Z$-span of all $X_\alpha$ and $H_\alpha$ for $\alpha \in \Phi$. Let $\Delta$ be a base for $\Phi$ and let $\Phi^+$ be the corresponding system of positive roots. Then $\{X_\alpha : \alpha \in \Phi\} \cup \{H_\alpha : \alpha \in \Delta\}$ is a $\Z$-basis of $\mathfrak{g}_\Z$, called a \emph{Chevalley basis} for $\mathfrak{g}$.

Fix a Chevalley basis for $\mathfrak{g}$. Let $\mathscr{U}_{\Z}$ be the corresponding Kostant $\Z$-form, which is the subring of the universal enveloping algebra generated by $1$ and $\frac{X_{\alpha}^k}{k!}$ for all $\alpha \in \Phi$ and $k \geq 1$.

Let $V$ be a faithful finite-dimensional $\mathfrak{g}$-module over $\C$. We denote the set of weights of $\mathfrak{h}$ in $V$ by $\Lambda(V)$. Then $\Z \Phi \subseteq \Z \Lambda(V) \subseteq \Lambda$, where $\Lambda$ is the weight lattice.

A \emph{lattice} in $V$ is the $\Z$-span of a basis of $V$. We say that a lattice $L \subseteq V$ is \emph{admissible}, if $L$ is $\mathscr{U}_{\Z}$-invariant. 

Let $F$ be an algebraically closed field of characteristic $p > 0$, and $L$ an admissible lattice in $V$. Set $V_F := F \otimes_{\Z} L$. Then for all $X \in \mathscr{U}_{\Z}$, the \emph{reduction modulo $p$} of $X$ is the $F$-linear map defined by $1 \otimes X' : V_F \rightarrow V_F$, where $X': L \rightarrow L$ is the action of $X$ on $L$.

In particular, for all $\alpha \in \Phi$ and $k \geq 0$ we have a linear map $X_{\alpha,k} : V_F \rightarrow V_F$ which is the reduction modulo $p$ of $X_{\alpha}^k/k!$ on $V_F$. Since $X_{\alpha}$ acts nilpotently on $V$, we can define for all $t \in F$ the root element $x_{\alpha}(t)$ as the exponential $x_{\alpha}(t) := \sum_{k \geq 0} t^k X_{\alpha,k}$. We have $x_{\alpha}(t) \in \GL(V_F)$, and the \emph{Chevalley group} (over $F$) corresponding to $V$ and $L$ is defined as $$G(V,L) := \langle x_{\alpha}(t) : \alpha \in \Phi, t \in F \rangle.$$ 

Then $G(V,L)$ is a simple algebraic group over $F$ with root system $\Phi$ \cite[Theorem 6]{SteinbergNotesAMS}. Furthermore, we have a maximal torus $$T = \langle h_{\alpha}(t) : \alpha \in \Delta, t \in K^\times \rangle,$$ where $h_{\alpha}(t)$ is defined as in \cite[Lemma 19, p. 22]{SteinbergNotesAMS}. Then the weights of $T$ on $V$ can be identified with $\Lambda(V)$, and the character group $X(T)$ can be identified with $\Z \Lambda(V)$ \cite[p. 39]{SteinbergNotesAMS}.

We say that $G(V,L)$ is \emph{simply connected} if $\Z \Lambda(V) = \Lambda$, and \emph{adjoint} if $\Z \Lambda(V) = \Z \Phi$.

\begin{lemma}[{\cite[Corollary 1, p. 41]{SteinbergNotesAMS}}]\label{lemma:GSCtoGAD}
Suppose that $V'$ is another faithful finite-dimensional $\mathfrak{g}$-module with admissible lattice $L'$, and denote the corresponding root elements in $G(V',L')$ by $x_{\alpha}'(t)$. Then:
	\begin{enumerate}[\normalfont (i)]
		\item If $G(V',L')$ is simply connected, there exists a morphism $\varphi: G(V',L') \rightarrow G(V,L)$ of algebraic groups with $x_{\alpha}'(t) \mapsto x_{\alpha}(t)$ for all $\alpha \in \Phi$ and $t \in F$.
		\item If $G(V',L')$ is adjoint, there exists a morphism $\varphi: G(V,L) \rightarrow G(V',L')$ of algebraic groups with $x_{\alpha}(t) \mapsto x_{\alpha}'(t)$ for all $\alpha \in \Phi$ and $t \in F$.
	\end{enumerate}
\end{lemma}
                                                                            
The Lie algebra of $G(V,L)$ is identified as follows. The stabilizer of $L$ in $\g$ and $\mathfrak{h}$ is given by \cite[Corollary 2, p. 16]{SteinbergNotesAMS} \begin{align*} \mathfrak{h}^L &= \{ H \in \mathfrak{h} : \mu(H) \in \Z \text{ for all } \mu \in \Lambda(V) \}, \\ \g^L &= \mathfrak{h}^L \oplus \bigoplus_{\alpha \in \Phi} \Z X_{\alpha}.\end{align*} In particular $\g^L$ and $\mathfrak{h}^L$ only depend on the weights in $V$, not the choice of the admissible lattice $L$. Furthermore, under the adjoint action $\g^L$ is an admissible lattice in $\g$ \cite[Proposition 27.2]{Humphreys}.

Then $\g^L$ is a $\Z$-Lie algebra which acts on $L$. The \emph{Chevalley algebra} $$\g(V,L) := F \otimes_{\Z} \g^L$$ is a Lie algebra over $F$ which acts faithfully on $V_F$, and $\g(V,L) \subseteq \mathfrak{gl}(V_F)$ is precisely the Lie algebra of $G(V,L)$. The adjoint action of $G(V,L)$ can be realized by applying the Chevalley construction with $V' = \g$ and $L' = \g^L$, and then taking the morphism $G(V,L) \rightarrow G(V',L')$ of Lemma \ref{lemma:GSCtoGAD} (ii).

Note that then the Lie algebra $\g(V,L)$ is a $G_{sc}$-module for a simply connected Chevalley group $G_{sc}$ with root system $\Phi$ (Lemma \ref{lemma:GSCtoGAD} (i)). In all cases, the structure of $\g(V,L)$ as a Lie algebra and as a $G_{sc}$-module is described by Hogeweij in \cite{Hogeweij}. Here the composition factors of $\g(V, L)$ are completely determined by $V$, but the submodule structure depends on the choice of the admissible lattice $L$.

In the simply connected case we have $\g^L = \g_{\Z}$. In the adjoint case we denote $\g_{\Z}^{ad} := \g^L$, so $\g_{\Z}^{ad}$ is the $\Z$-span of $\g_{\Z}$ and $\mathfrak{h}^{ad} := \{ H \in \mathfrak{h} : \alpha(H) \in \Z \text{ for all } \alpha \in \Phi \}$.

\subsection{Chevalley construction for type $C_{\ell}$}\label{subsection:chevalleyC}

We will now setup the Chevalley construction for groups of type $C_{\ell}$ over $K$. (Recall that by $K$ we always denote an algebraically closed field of characteristic two.) 

Let $V_{\C}$ be a $\C$-vector space of dimension $n = 2\ell$, with basis $v_1$, $\ldots$, $v_n$. We define a non-degenerate alternating bilinear form $(-,-)$ on $V_{\C}$ by $(v_i,v_{n-i+1}) = 1 = -(v_{n-i+1},v_i)$ for all $1 \leq i \leq \ell$, and $(v_i,v_j) = 0$ if $i+j \neq n+1$.

Let $\mathfrak{sp}(V_\C) = \{X \in \mathfrak{gl}(V_\C) : (Xv, w) + (v, Xw) = 0 \text{ for all } v,w \in V_{\C} \}$, so $\mathfrak{sp}(V_\C)$ is a simple Lie algebra of type $C_{\ell}$. Let $\mathfrak{h}_{\C}$ be the Cartan subalgebra formed by the diagonal matrices in $\mathfrak{sp}(V_\C)$. Then $\mathfrak{h}_{\C} = \{ \operatorname{diag}(h_1, \ldots, h_{\ell}, -h_{\ell}, \ldots, -h_1) : h_i \in \C \}$.

For $1 \leq i \leq \ell$, define linear maps $\varepsilon_i : \mathfrak{h}_{\C} \rightarrow \C$ by $\varepsilon_i(h) = h_i$ where $h$ is a diagonal matrix with diagonal entries $(h_1, \ldots, h_{\ell}, -h_{\ell}, \ldots, -h_1)$. Then $$\Phi = \{ \pm(\varepsilon_i \pm \varepsilon_j) : 1 \leq i < j \leq \ell \} \cup \{ \pm 2\varepsilon_i : 1 \leq i \leq \ell \}$$ is the root system for $\mathfrak{sp}(V_\C)$, and $$\Phi^+ = \{ \varepsilon_i \pm \varepsilon_j : 1 \leq i < j \leq \ell \} \cup \{ 2 \varepsilon_i : 1 \leq i \leq \ell \}$$ is a system of positive roots. The set of simple roots corresponding to $\Phi^+$ is $\Delta = \{\alpha_1, \ldots, \alpha_{\ell}\}$, where $\alpha_i = \varepsilon_i - \varepsilon_{i+1}$ for $1 \leq i < {\ell}$ and $\alpha_{\ell} = 2 \varepsilon_{\ell}$.

For all $i,j$ let $E_{i,j}$ be the linear endomorphism on $V_{\C}$ such that $E_{i,j}(v_j) = v_i$ and $E_{i,j}(v_k) = 0$ for $k \neq j$. Throughout we will use the following Chevalley basis of $\mathfrak{sp}(V_\C)$, which is taken from \cite[Section 11, p. 38]{JantzenThesis}. \begin{align*} 
X_{\varepsilon_i - \varepsilon_j} &= E_{i,j} - E_{n-j+1,n-i+1} & \text{ for all } i \neq j, \\
X_{\varepsilon_i + \varepsilon_j} &= E_{j, n-i+1} + E_{i, n-j+1} & \text{ for all } i \neq j, \\
X_{-(\varepsilon_i + \varepsilon_j)} &= E_{n-j+1, i} + E_{n-i+1, j} & \text{ for all } i \neq j, \\
X_{2 \varepsilon_i} &= E_{i, n-i+1} & \text{ for all } i, \\
X_{-2 \varepsilon_i} &= E_{n-i+1, i} & \text{ for all } i. \\
H_{\alpha_i} &= [X_{\alpha_i}, X_{-\alpha_i}] & \text{ for all } 1 \leq i \leq \ell.
\end{align*}

Let $\Lambda \subset \mathfrak{h}_{\C}^*$ be the weight lattice. The maps $\varepsilon_1$, $\ldots$, $\varepsilon_{\ell}$ form $\Z$-basis for $\Lambda$, and for $1 \leq i \leq \ell$, the $i$th fundamental highest weight is equal to $\varpi_i := \varepsilon_1 + \cdots + \varepsilon_i$.

As in the previous subsection, we denote by $\g_{\Z}$ be the $\Z$-span of the Chevalley basis above. Then $\g_{\Z}^{ad}$ is the $\Z$-span of $\g_{\Z}$ and $H$, where $H \in \mathfrak{h}_{\C}$ is the diagonal matrix $H = \diag(1/2, \ldots, 1/2, -1/2, \ldots, -1/2)$. In terms of the Chevalley basis, we have $$H = \frac{1}{2} \left(H_{\alpha_1} + 2 H_{\alpha_2} + \cdots + \ell H_{\alpha_{\ell}} \right).$$ 

\subsection{Simply connected groups of type $C_{\ell}$}\label{subsection:chevalleySP} Let $V_{\Z}$ be the $\Z$-span of the basis $v_1$, $\ldots$, $v_n$ of $V_{\C}$. It is clear that $V_{\Z}$ is an admissible lattice. Denote $V := K \otimes_{\Z} V_{\Z}$. By abuse of notation we denote $v_i := 1 \otimes v_i$ for all $1 \leq i \leq n$, so $v_1$, $\ldots$, $v_n$ is a basis of $V$. The alternating bilinear form $(-,-)$ induces an alternating bilinear form $b: V \times V \rightarrow K$ on $V$, with $b(v_i,v_j) = 1$ if $i+j = n+1$ and $b(v_i,v_j) = 0$ otherwise.

Then the Chevalley group corresponding to $V_{\C}$ and $V_{\Z}$ is simply connected of type $C_{\ell}$, and it is precisely the symplectic group $\Sp(V)$ corresponding to $b$ \cite[5]{Ree}. We denote $G_{sc} = \Sp(V)$. Then the Lie algebra of $G_{sc}$ is $$\g_{sc} = \mathfrak{sp}(V) = \{X \in \mathfrak{gl}(V) : b(Xv,w) + b(v,Xw) = 0 \text{ for all } v,w \in V\}.$$

We denote by $\g_{ad} := K \otimes_{\Z} \g_{\Z}^{\ad}$ the adjoint Lie algebra of type $C_{\ell}$. For any $\mathfrak{sp}(V_{\C})$-module $W$ with admissible lattice $L$, we will consider $K \otimes_{\Z} L$ as a $G_{sc}$-module via the action provided by Lemma \ref{lemma:GSCtoGAD} (i).

\subsection{Lie algebra of adjoint type $C_{\ell}$}\label{subsection:symsquareC} To compute with the action of $G_{sc} = \Sp(V)$ on $\g_{ad}$, it will be convenient to use the following well-known identification of $\mathfrak{sp}(V_{\C})$ with the symmetric square $S^2(V_{\C})$.

\begin{lemma}\label{lemma:overCisomorphism}
For $x,y \in V_{\C}$, define a linear map $\psi_{x,y}: V_{\C} \rightarrow V_{\C}$ by $v \mapsto (y,v)x + (x,v)y$. Then we have an isomorphism $S^2(V_{\C}) \rightarrow \mathfrak{sp}(V_{\C})$ of $\mathfrak{sp}(V_{\C})$-modules, defined by $xy \mapsto \psi_{x,y}$ for all $x,y \in V_{\C}$.
\end{lemma}

\begin{proof}We have an isomorphism $\tau: V_{\C} \otimes V_{\C}^* \rightarrow \mathfrak{gl}(V_{\C})$ of $\mathfrak{gl}(V_{\C})$-modules, where $\tau(v \otimes f)$ is the linear map $w \mapsto f(w)v$ for all $v \in V_{\C}$ and $f \in V_{\C}^*$.  Moreover, we have an isomorphism $\tau': V_{\C} \otimes V_{\C} \rightarrow V_{\C} \otimes V_{\C}^*$ of $\mathfrak{sp}(V_{\C})$-modules defined by $x \otimes y \mapsto x \otimes f_y$, where $f_y(v) = (y,v)$ for all $v \in V$. 

Now identifying $S^2(V_{\C}) \subseteq V_{\C} \otimes V_{\C}$ via $xy \mapsto x \otimes y + y \otimes x$, the restriction of $\tau\tau'$ to $S^2(V_{\C})$ is a map $\psi: S^2(V_{\C}) \rightarrow \mathfrak{gl}(V_{\C})$ defined by $xy \mapsto \psi_{x,y}$. Then $\psi$ is an injective morphism of $\mathfrak{sp}(V_{\C})$-modules, and it is clear that $\psi_{x,y} \in \mathfrak{sp}(V_{\C})$ for all $x,y \in V_{\C}$. Since $S^2(V_{\C})$ and $\mathfrak{sp}(V_{\C})$ have the same dimension, we conclude that $\psi$ defines an isomorphism $S^2(V_{\C}) \rightarrow \mathfrak{sp}(V_{\C})$ of $\mathfrak{sp}(V_{\C})$-modules.\end{proof}

We denote by $\psi: S^2(V_{\C}) \rightarrow \mathfrak{sp}(V_{\C})$ the isomorphism $\psi(xy) = \psi_{x,y}$ as in Lemma \ref{lemma:overCisomorphism}. Let $1 \leq i, j \leq \ell$ with $i \neq j$. Then under the map $\psi$, the root vectors $X_{\pm \varepsilon_i \pm \varepsilon_j}$ in the Chevalley basis correspond to elements of $S^2(V_{\C})$ as follows. \begin{align*}
		\psi\left(v_i v_{n-j+1}\right) &= - X_{\varepsilon_i - \varepsilon_j} \\
		\psi\left(v_i v_j\right) &= X_{\varepsilon_i + \varepsilon_j} \\
		\psi\left(v_{n-i+1} v_{n-j+1}\right) &= -X_{-(\varepsilon_i + \varepsilon_j)} \\
		\psi\left(\frac{1}{2} v_i^2\right) &= X_{2\varepsilon_i} \\
		\psi\left(\frac{1}{2} v_{n-i+1}^2\right) &= -X_{-2\varepsilon_i}
	\end{align*} Furthermore, define $$\delta := \frac{1}{2} \sum_{i = 1}^{\ell} v_i v_{n-i+1}.$$ Then $\psi(\delta) = -H$.
	
We define \begin{align*} L_{sc} &:= \Z \text{-span of } v_iv_j \text{ and } \frac{1}{2}v_i^2 \text{ for all } 1 \leq i,j \leq n; \\ L_{ad} &:= \Z \text{-span of } L_{sc} \text{ and } \delta;\end{align*} so that $\psi(L_{sc}) = \g_{\Z}$ and $\psi(L_{ad}) = \g_{\Z}^{ad}$.

Then $L_{sc} \subset L_{ad}$ are admissible lattices in $S^2(V_{\C})$, since $\g_{\Z}$ and $\g_{\Z}^{ad}$ are admissible lattices in $\mathfrak{sp}(V_{\C})$. 

From the Chevalley construction $K \otimes_{\Z} \g_{\Z} = \g_{sc}$ and $K \otimes_{\Z} \g_{\Z}^{ad} = \g_{ad}$ as $G_{sc}$-modules, so $\psi$ induces isomorphisms \begin{equation}\label{eq:psiprimemorph}\begin{aligned} &\psi': \g_{sc} \rightarrow K \otimes_{\Z} L_{sc} \\ &\psi'': \g_{ad} \rightarrow K \otimes_{\Z} L_{ad}\end{aligned}\end{equation} of $G_{sc}$-modules. 

By \cite[Lemma 2.2]{Hogeweij} we have $\dim Z(\g_{sc}) = 1$ and $Z(\g_{sc})$ is spanned by $1 \otimes 2H$. Similarly by \cite[Table 1]{Hogeweij}, we have $\dim \g_{ad}/[\g_{ad},\g_{ad}] = 1$ and $[\g_{ad},\g_{ad}]$ is spanned by $1 \otimes v$ with $v \in \g_{\Z}$. Thus under the isomorphisms in~\eqref{eq:psiprimemorph}, we get \begin{equation}\label{eq:psiprimemorph2}\begin{aligned} \psi'(Z(\g_{sc})) &= \langle 1 \otimes 2\delta \rangle, \\ \psi''([\g_{ad},\g_{ad}]) &= \langle 1 \otimes v : v \in L_{sc} \rangle.\end{aligned}\end{equation}

Note that \begin{equation}\label{eq:ladlscmod2}L_{ad}/L_{sc} = \{L_{sc}, \delta + L_{sc}\} \cong \Z/2\Z,\end{equation} so for the linear map $\pi: K \otimes_{\Z} L_{sc} \rightarrow K \otimes_{\Z} L_{ad}$ induced by the inclusion $L_{sc} \subset L_{ad}$, we have $\Ker \pi = \langle 1 \otimes (2\delta) \rangle$ and $\operatorname{Im} \pi = \langle 1 \otimes v : v \in L_{sc} \rangle$.

\subsection{Action of unipotent elements on $\g_{ad}$, orthogonally indecomposable case}\label{subsection:unipotentroots} Let $G_{sc} = \Sp(V)$ as in the previous sections, and let $u \in \Sp(V)$ be a unipotent element. 

We will describe how to construct a representative for the conjugacy class of $u$ in terms of root elements, and how to compute the action of $u$ on $\g_{ad} \cong K \otimes_{\Z} L_{ad}$. We first do this in the case where $V \downarrow K[u]$ is orthogonally indecomposable, so $V \downarrow K[u] = V(2\ell)$ or $V \downarrow K[u] = W(\ell)$ (Theorem \ref{thm:hesselinkUNIPindecomp}). In this case it is well known that by replacing $u$ with a conjugate, we can take $$u = \begin{cases} x_{\alpha_1}(1) \cdots x_{\alpha_{\ell}}(1),& \text{ if } V \downarrow K[u] = V(2\ell). \\ x_{\alpha_1}(1) \cdots x_{\alpha_{\ell-1}}(1),& \text{ if } V \downarrow K[u] = W(\ell).\end{cases}$$ 

For $\alpha \in \Phi$, let $$x_{\Z}^{(\alpha)} := 1 + X_{\alpha} + \frac{X_{\alpha}^2}{2!} \in \mathscr{U}_{\Z}.$$ Then $x_{\alpha}(1)$ is the reduction modulo $p$ of the action of $x_{\Z}^{(\alpha)}$ on $V_{\Z}$. Furthermore, since $X_{\alpha}^3 \cdot S^2(V_{\C}) = 0$, the action of $x_{\alpha}(1)$ on $\g_{ad} \cong K \otimes_{\Z} L_{ad}$ is the reduction modulo $p$ of the action of $x_{\Z}^{(\alpha)}$ on $L_{ad}$. 

Therefore we define $$u_{\Z} := \begin{cases} x_{\Z}^{(\alpha_1)} \cdots x_{\Z}^{(\alpha_\ell)},& \text{ if } V \downarrow K[u] = V(2\ell). \\ x_{\Z}^{(\alpha_1)} \cdots x_{\Z}^{(\alpha_{\ell-1})},& \text{ if } V \downarrow K[u] = W(\ell).\end{cases}$$ If $V \downarrow K[u] = V(2\ell)$, we get \begin{align*} u_{\Z} \cdot v_1 &= v_1, \\ u_{\Z} \cdot v_i &= v_i + v_{i-1} + \cdots + v_1 \text{ if } 1 < i \leq \ell+1, \\ u_{\Z} \cdot v_i &= v_i - v_{i-1} \text{ if } \ell+1 < i \leq 2\ell.\end{align*} Similarly for $V \downarrow K[u] = W(\ell)$, we get \begin{align*} u_{\Z} \cdot v_1 &= v_1, \\ u_{\Z} \cdot v_i &= v_i + v_{i-1} + \cdots + v_1 \text{ if } 1 < i \leq \ell, \\ u_{\Z} \cdot v_{\ell+1} &= v_{\ell+1} \\ u_{\Z} \cdot v_i &= v_i - v_{i-1} \text{ if } \ell+1 < i \leq 2\ell.\end{align*} Thus the action of $u$ on $V$ is exactly as described in Definition \ref{def:V2lUNIP} or Definition \ref{def:WlUNIP}. 

Since $X_{\alpha_i}^3 \cdot S^2(V_{\C}) = 0$, for the action on $S^2(V_{\C})$ we have $$x_{\Z}^{(\alpha)} \cdot (vw) = (x_{\Z}^{(\alpha)}v)(x_{\Z}^{(\alpha)}w)$$ for all $v,w \in V_{\C}$. Thus $$u_{\Z} \cdot (vw) = (u_{\Z}v)(u_{\Z}w)$$ for all $v,w \in V_{\C}$.

\subsection{Action of unipotent elements on $\g_{ad}$, general case}\label{subsection:unipotentrootsGEN} For all unipotent elements $u \in G_{sc}$, in general we can proceed as follows. We have an orthogonal decomposition $V \downarrow K[u] = U_1 \perp \cdots \perp U_t$, where $\dim U_i > 0$ and $U_i$ is an orthogonally indecomposable $K[u]$-module for all $1 \leq i \leq t$. Write $\dim U_i = n_i = 2\ell_i$ for all $1 \leq i \leq t$.

Then $u = u_1 \cdots u_t$, where $u_i \in \Sp(U_i)$ is the action of $u$ on $U_i$. For all $1 \leq i \leq t$, we have $U_i \downarrow K[u_i] \cong W(\ell_i)$ or $U_i \downarrow K[u_i] \cong V(2\ell_i)$ (Theorem \ref{thm:hesselinkUNIPindecomp}). 

Relabel the basis $v_1$, $\ldots$, $v_n$ of $V_{\C}$ as $$v_1^{(1)}, \ldots, v_{n_1}^{(1)}, \ldots, v_1^{(t)}, \ldots, v_{n_t}^{(t)},$$ where $(v_i^{(j)}, v_{i'}^{(j')}) = 1$ if $j = j'$ and $i+i' = n_j$, and $0$ otherwise. Then the Cartan subalgebra $\mathfrak{h}$ consists of diagonal matrices of the form $$h = \operatorname{diag}(a_1^{(1)}, \ldots, a_{\ell_1}^{(1)}, -a_{\ell_1}^{(1)}, \ldots, -a_{1}^{(1)},\ \ldots,\ a_1^{(t)},\ldots, a_{\ell_t}^{(t)}, -a_{\ell_t}^{(t)}, \ldots, -a_{1}^{(t)}).$$ We define then for $1 \leq i \leq t$ and $1 \leq k \leq \ell_i$ the linear map $\varepsilon_k^{(i)} : \mathfrak{h} \rightarrow \C$ by $h \mapsto a_k^{(i)}$.

Denote by $V_{\Z}^{(i)}$ (resp. $V_{\C}^{(i)}$) the $\Z$-span (resp. $\C$-span) of $v_1^{(i)}, \ldots, v_{n_i}^{(i)}$, so we have \begin{align*}V_{\Z} &= V_{\Z}^{(1)} \oplus \cdots \oplus V_{\Z}^{(t)}, \\ V_{\C} &= V_{\C}^{(1)} \perp \cdots \perp V_{\C}^{(t)}.\end{align*} Since $V = K \otimes_{\Z} V_{\Z}$, we can assume that $U_i = K \otimes_{\Z} V_{\Z}^{(i)}$ for all $1 \leq i \leq t$. 

We have $\mathfrak{sp}(V_{\C}^{(i)}) \subseteq \mathfrak{sp}(V_{\C})$, and $\mathfrak{sp}(V_{\C}^{(i)})$ has a root system $\Phi^{(i)}$ of type $C_{\ell_i}$ with simple roots $\Delta^{(i)} = \{\beta_1, \ldots, \beta_{\ell_i} \}$, where $$\beta_j = \begin{cases} \varepsilon_j^{(i)} - \varepsilon_{j+1}^{(i)},& \text{ if } 1 \leq j < \ell_i, \\ 2 \varepsilon_{\ell_i}^{(i)},& \text{ if } j = \ell_i.\end{cases}$$ Note that $\{X_{\alpha} : \alpha \in \Phi^{(i)} \} \cup \{ H_{\alpha} : \alpha \in \Delta^{(i)} \}$ is a Chevalley basis of $\mathfrak{sp}(V_{\C}^{(i)})$, and the corresponding Kostant $\Z$-form $\mathscr{U}_{\Z}^{(i)}$ is a subring of $\mathscr{U}_{\Z}$. Then $V_{\Z}^{(i)}$ is an admissible lattice for $\mathfrak{sp}(V_{\C}^{(i)})$, and applying the Chevalley construction we get $\Sp(U_i)$.

As in the orthogonally indecomposable case, we define $$u_{\Z,i} := \begin{cases} x_{\Z}^{(\beta_1)} \cdots x_{\Z}^{(\beta_{\ell_i})},& \text{ if } U_i \downarrow K[u_i] = V(2\ell_i). \\ x_{\Z}^{(\beta_1)} \cdots x_{\Z}^{(\beta_{\ell_i-1})},& \text{ if } U_i \downarrow K[u_i] = W(\ell_i).\end{cases}$$ Then $u_{\Z,i} \in \mathscr{U}_{\Z}^{(i)}$, and the reduction modulo $p$ of the action of $u_{\Z,i}$ on $V_{\Z}^{(i)}$ is precisely $u_i \in \Sp(U_i)$.

Thus we can define $$u_{\Z} := u_{\Z,1} \cdots u_{\Z,t} \in \mathscr{U}_{\Z},$$ so that $u$ is the reduction modulo $p$ of the action of $u_{\Z}$ on $V_{\Z}$. Furthermore, the action of $u$ on $\g_{ad}$ is the reduction modulo $p$ of the action of $u_{\Z}$ on $L_{ad}$.

\subsection{Action of nilpotent elements on $\g_{ad}$}\label{subsection:nilpotentroots} Let $e \in \mathfrak{sp}(V)$ be nilpotent. Similarly to the unipotent case, we can find $e_{\Z} \in \g_{\Z}$ such that $e$ is the reduction modulo $p$ of the action of $e_{\Z}$ on $V_{\Z}$.

We first consider the orthogonally indecomposable case, so $V \downarrow K[e] = V(2\ell)$, $V \downarrow K[e] = W(\ell)$, or $V \downarrow K[e] = W_k(\ell)$ for some $0 < k < \ell/2$ (Theorem \ref{thm:hesselinkNILindecomp}). In this case we define $$e_{\Z} := \begin{cases} X_{\alpha_1} + \cdots + X_{\alpha_{\ell}}, & \text{ if } V \downarrow K[e] = V(2\ell). \\ X_{\alpha_1} + \cdots + X_{\alpha_{\ell-1}}, & \text{ if } V \downarrow K[e] = W(\ell). \\ X_{\alpha_1} + \cdots + X_{\alpha_{\ell-1}} + X_{2\varepsilon_k}, & \text{ if } V \downarrow K[e] = W_k(\ell).\end{cases}$$ Then for the reduction modulo $p$ of $e_{\Z}$, the action of $V = K \otimes_{\Z} V_{\Z}$ is exactly as in Definition \ref{def:V2l} -- \ref{def:Wkl}. (Here the expression for $W_k(\ell)$ is taken from \cite[Lemma 3.4]{KorhonenStewartThomas}.)

In the general case, we have an orthogonal decomposition $V \downarrow K[e] = U_1 \perp \cdots \perp U_t$, where $U_i$ is an orthogonally indecomposable $K[e]$-module, with $\dim U_i = 2\ell_i$. We have $e = e_1 + \cdots + e_t$, where $e_i \in \mathfrak{sp}(U_i)$ is the action of $e$ on $U_i$. In the notation of the previous section, we take $U_i = K \otimes_{\Z} V_{\Z}^{(i)}$. We can then define $$e_{\Z} := e_{\Z,1} + \cdots + e_{\Z,t},$$ where $e_{\Z,i} \in \g_{\Z} \cap \mathfrak{sp}(V_{\C}^{(i)})$ is defined as in the orthogonally indecomposable case, such that $e_i$ is the reduction modulo $p$ of $e_{\Z,i}$. 

Then $e$ is the reduction modulo $p$ of $e_{\Z}$. Moreover, given any irreducible $\mathfrak{sp}(V_{\C})$-module $W_{\C}$ and admissible lattice $L \subset W_{\C}$, the action of $e$ on $K \otimes_{\Z} L$ is precisely the reduction modulo $p$ of the action of $e_{\Z}$ on $L$.
  
\subsection{Elements of $\mathscr{U}_{\Z}$ acting nilpotently} In this section, we consider $X \in \mathscr{U}_{\Z}$ which act nilpotently on $S^2(V_{\C})$. Then by reduction modulo $p$, we get an action $\widehat{X}$ on $\g_{sc}$, and $\widetilde{X}$ on $\g_{ad}$. We will describe when $\operatorname{Im}(\widehat{X}) \supseteq Z(\g_{sc})$ and when $\operatorname{Ker}(\widetilde{X}) \subseteq [\g_{ad}, \g_{ad}]$. For our purposes this is mostly relevant for $X = (u_{\Z} - 1)^m$ in the unipotent case, and $X = e_{\Z}^m$ in the nilpotent case, for integer $m \geq 1$.

\begin{lemma}\label{lemma:nilpinUZ2}
Let $X \in \mathscr{U}_{\Z}$ be such that $X$ acts nilpotently on $S^2(V_{\C})$, and denote the action of $X$ on $\g_{sc}$ by $\widehat{X}$. Then $\operatorname{Im}(\widehat{X}) \supseteq Z(\g_{sc})$ if and only if there exists $v \in L_{sc}$ such that $X \cdot v \equiv 2 \delta \mod 2L_{sc}$.
\end{lemma}

\begin{proof}
Identifying $\g_{sc} = K \otimes_{\Z} L_{sc}$ via~\eqref{eq:psiprimemorph}, we have $Z(\g_{sc}) = \langle 1 \otimes 2 \delta \rangle$ by~\eqref{eq:psiprimemorph2}. Thus if $v \in L_{sc}$ is such that $X \cdot v \equiv 2 \delta \mod 2L_{sc}$, we have $\widehat{X} \cdot (1 \otimes v) = 1 \otimes 2 \delta$.

Conversely, suppose that $\operatorname{Im}(\widehat{X}) \supseteq Z(\g_{sc})$. Then there exists $w \in \g_{sc}$ with $\widehat{X} \cdot w = 1 \otimes 2\delta$. Since $\widehat{X} = 1 \otimes X'$ where $X'$ is the action of $X$ on $L_{sc}$, we can assume that $w = 1 \otimes v$ with $v \in L_{sc}$. Then $$1 \otimes 2\delta = \widehat{X} \cdot w = 1 \otimes (X \cdot v)$$ implies that $X \cdot v \equiv 2 \delta \mod 2L_{sc}$.\end{proof}

For actions of unipotent elements, Lemma \ref{lemma:nilpinUZ2} gives the following.

\begin{lemma}\label{lemma:imZforUZ}
Let $u \in G_{sc}$ be unipotent, and suppose that $u$ is the reduction modulo $p$ of $u_{\Z} \in \mathscr{U}_{\Z}$. Denote the action of $u$ on $\g_{sc}$ by $\widehat{u}$. Let $m \geq 1$ be an integer. Then $\operatorname{Im}(\widehat{u} - 1)^m \supseteq Z(\g_{sc})$ if and only if there exists $v \in L_{sc}$ such that $(u_{\Z} - 1)^m \cdot (v) \equiv 2 \delta \mod 2L_{sc}$.
\end{lemma}

\begin{proof}
Follows from Lemma \ref{lemma:nilpinUZ2}, with $X = (u_{\Z} - 1)^m$.
\end{proof}


\begin{lemma}\label{lemma:nilpinUZgen}
Let $X \in \mathscr{U}_{\Z}$ be such that $X$ acts nilpotently on $S^2(V_{\C})$, and denote the action of $X$ on $\g_{ad}$ by $\widetilde{X}$. Then the following statements are equivalent:
	\begin{enumerate}[\normalfont (i)]
		\item $\Ker(\widetilde{X}) \not\subseteq [\g_{ad}, \g_{ad}]$ .
		\item There exists $v \in L_{sc}$ such that $X \cdot (\delta + v) \in 2 L_{ad}$.
		\item There exists $v \in L_{sc}$ such that one of the following holds:
			\begin{enumerate}[\normalfont (a)] 
				\item $X \cdot (\delta + v) \equiv 0 \mod{2 L_{sc}}$.
				\item $X \cdot (\delta + v) \equiv 2\delta \mod{2 L_{sc}}$.
			\end{enumerate}
	\end{enumerate}
\end{lemma}

\begin{proof}
We first prove that (i) and (ii) are equivalent. We identify $\g_{ad} = K \otimes_{\Z} L_{ad}$ via~\eqref{eq:psiprimemorph}, in which case  $\g_{ad} = \langle 1 \otimes \delta, [\g_{ad}, \g_{ad}] \rangle$ as $K$-vector spaces. Here $[\g_{ad}, \g_{ad}]$ is the subspace spanned by $1 \otimes v$ with $v \in L_{sc}$, as seen in~\eqref{eq:psiprimemorph2}.

If there exists $v \in L_{sc}$ such that $X \cdot (\delta + v) \in 2 L_{ad}$, we have $$\widetilde{X} \cdot (1 \otimes \delta + 1 \otimes v) = 1 \otimes (X \cdot (\delta + v)) = 0,$$ so $\Ker(\widetilde{X}) \not\subseteq [\g_{ad}, \g_{ad}]$. 

Conversely, suppose that $\Ker(\widetilde{X}) \not\subseteq [\g_{ad}, \g_{ad}]$. Then there exists $w \in [\g_{ad}, \g_{ad}]$ such that $\widetilde{X} \cdot (1 \otimes \delta + w) = 0$. We have $\widetilde{X} = 1 \otimes X'$ where $X'$ is the action of $X$ on $L_{ad}$, so we can assume that $w = 1 \otimes v$ for some $v \in L_{sc}$. Then $$0 = \widetilde{X} \cdot (1 \otimes \delta + 1 \otimes v) = 1 \otimes (X \cdot (\delta+v))$$ implies that $X \cdot (\delta+v) \in 2L_{ad}$.

Next we show that (ii) and (iii) are equivalent. It follows from~\eqref{eq:ladlscmod2} that $2L_{ad}/2L_{sc} \cong L_{ad}/L_{sc} \cong \Z/2\Z$ contains only two elements, $2L_{sc}$ and $2\delta + 2L_{sc}$. Therefore $X \cdot (\delta + v) \in 2L_{ad}$ if and only if (iii)(a) or (iii)(b) holds.
\end{proof}

Next we make some observations that will allow us to reduce the proofs of our main results to the orthogonally indecomposable case. Continue with the notation as in Section \ref{subsection:unipotentrootsGEN}, so $V = U_1 \perp \cdots \perp U_t$ with $U_i = K \otimes_{\Z} V_{\Z}^{(i)}$, and $\mathscr{U}_{\Z}^{(i)}$ is the Kostant $\Z$-form for $\mathfrak{sp}(V_{\C}^{(i)})$. Define $$\delta_i = \frac{1}{2} \sum_{1 \leq j \leq \ell_i} v_j^{(i)}v_{2\ell_i-j+1}^{(i)}$$ for all $1 \leq i \leq t$. (Note that $\delta = \delta_1 + \cdots + \delta_t$.) Furthermore, for $1 \leq i \leq t$ we define \begin{align*} L_{sc}^{(i)} &:= \Z \text{-span of } v_j^{(i)}v_k^{(i)}\text{ and } \frac{1}{2}\left(v_j^{(i)}\right)^2 \text{ for all } 1 \leq j,k \leq n_i. \\ L_{ad}^{(i)} &:= \Z \text{-span of } L_{sc}^{(i)} \text{ and } \delta_i.\end{align*} Then \begin{equation}\label{eq:S2overZdecomp}L_{sc} = L_{sc}^{(1)} \oplus \cdots \oplus L_{sc}^{(t)} \oplus \bigoplus_{i \neq i'} V_{\Z}^{(i)} V_{\Z}^{(i')}\end{equation} as $\Z$-modules.

Moreover $L_{ad}^{(i)} \subset S^2(V_{\C}^{(i)})$ is an admissible lattice for the action of $\mathfrak{sp}(V_{\C}^{(i)})$, and $$\g_{ad}^{(i)} := K \otimes_{\Z} L_{ad}^{(i)}$$ is an adjoint Lie algebra of type $C_{\ell_i}$. Here $\Sp(U_i)$ acts on $\g_{ad}^{(i)}$ via the Chevalley construction.

\begin{lemma}\label{lemma:GADreduceNIL}
Let $X = X_1 + \cdots + X_t$, where $X_i \in \mathscr{U}_{\Z}^{(i)}$ acts nilpotently on $S^2(V_{\C})$ for all $1 \leq i \leq t$. Denote by $\widetilde{X}$ the linear map acting on $\g_{ad}$, given by reducing the action of $X$ on $L_{ad}$ modulo $p$. For $1 \leq i \leq t$, let $\widetilde{X_i}$ be the linear map acting on $\g_{ad}^{(i)}$, given by reducing the action of $X_i$ on $L_{ad}^{(i)}$ modulo $p$. 

Then the following statements hold.

	\begin{enumerate}[\normalfont (i)]
		\item $\Ker(\widetilde{X}) \not\subseteq [\g_{ad}, \g_{ad}]$ if and only if there exist $w_1$, $\ldots$, $w_t$ with $w_i \in L_{sc}^{(i)}$ such that one of the following holds:
			\begin{enumerate}[\normalfont (a)]
				\item $X_i \cdot (\delta_i + w_i) \equiv 0 \mod{2 L_{sc}^{(i)}}$ for all $1 \leq i \leq t$.
				\item $X_i \cdot (\delta_i + w_i) \equiv 2 \delta_i \mod{2 L_{sc}^{(i)}}$ for all $1 \leq i \leq t$.
			\end{enumerate}
		\item If $\Ker(\widetilde{X}) \not\subseteq [\g_{ad}, \g_{ad}]$, then $\Ker(\widetilde{X_i}) \not\subseteq [\g_{ad}^{(i)}, \g_{ad}^{(i)}]$ for all $1 \leq i \leq t$.
	\end{enumerate}
\end{lemma}

\begin{proof}
For (i), suppose first that there exist $w_1$, $\ldots$, $w_t$ with $w_i \in L_{sc}^{(i)}$ such that (i)(a) or (i)(b) holds. For $w = w_1 + \cdots + w_t$, we have $$X \cdot (\delta + w) = X_1 \cdot (\delta_1 + w_1) + \cdots + X_t \cdot (\delta_t + w_t).$$ Thus $X \cdot (\delta + w) \equiv 0 \mod{2 L_{sc}}$ if (a) holds, and $X \cdot (\delta + w) \equiv 2 \delta \mod{2 L_{sc}}$ if (b) holds. Therefore $\Ker(\widetilde{X}) \not\subseteq [\g_{ad}, \g_{ad}]$ by Lemma \ref{lemma:nilpinUZgen}.

Conversely, suppose that $\Ker(\widetilde{X}) \not\subseteq [\g_{ad}, \g_{ad}]$. Then by Lemma \ref{lemma:nilpinUZgen}, there exists $w \in L_{sc}$ such that $X \cdot (\delta + w) \equiv 0 \mod{2 L_{sc}}$ or $X \cdot (\delta + w) \equiv 2 \delta \mod{2 L_{sc}}$. By~\eqref{eq:S2overZdecomp} we can write $w = w_1 + \cdots + w_t + w'$, where $w_i \in L_{sc}^{(i)}$ for all $1 \leq i \leq t$ and $w' \in \bigoplus_{i \neq i'} V_{\Z}^{(i)} V_{\Z}^{(i')}$.

Note that $L_{sc}^{(1)} \oplus \cdots \oplus L_{sc}^{(t)}$ and $\bigoplus_{i \neq i'} V_{\Z}^{(i)} V_{\Z}^{(i')}$ are $\mathscr{U}_{\Z}^{(i)}$-invariant. Thus $$X \cdot (\delta + w) = X_1 \cdot (\delta_1 + w_1) + \cdots + X_t \cdot (\delta_t + w_t) + X \cdot w'$$ with $X_i \cdot (\delta_i + w_i) \in L_{ad}^{(i)}$ for all $1 \leq i \leq t$, and $X \cdot w' \in \bigoplus_{i \neq i'} V_{\Z}^{(i)} V_{\Z}^{(i')}$. Since $2\delta \in L_{sc}^{(1)} \oplus \cdots \oplus L_{sc}^{(t)}$, it follows that $X \cdot w' = 0$. Moreover we have assumed that $X \cdot (\delta + w) \equiv 0 \mod{2 L_{sc}}$ or $X \cdot (\delta + w) \equiv 2 \delta \mod{2 L_{sc}}$, so either (i)(a) or (i)(b) holds.

Claim (ii) follows from (i) and Lemma \ref{lemma:nilpinUZgen}.
\end{proof}

For the action of unipotent elements, we have the following result.

\begin{lemma}\label{lemma:GADreduce}
Let $u = u_1 \cdots u_t \in \Sp(V)$ be unipotent with $u_{\Z} = u_{\Z,1} \cdots u_{\Z,t} \in \mathscr{U}_{\Z}$ as in Section \ref{subsection:unipotentrootsGEN}. Denote the action of $u$ on $\g_{ad}$ by $\widetilde{u}$, and denote the action of $u_i$ on $\g_{ad}^{(i)}$ by $\widetilde{u_i}$.

Let $m \geq 1$ be an integer. Then the following statements hold.

\begin{enumerate}[\normalfont (i)]
	\item $\Ker(\widetilde{u}-1)^m \not\subseteq [\g_{ad},\g_{ad}]$ if and only if there exists $w_1$, $\ldots$, $w_t$ with $w_i \in L_{sc}^{(i)}$ such that one of the following holds:
		\begin{enumerate}[\normalfont (a)]
			\item $(u_{\Z,i} - 1)^m \cdot (\delta_i + w_i) \equiv 0 \mod{2 L_{sc}^{(i)}}$ for all $1 \leq i \leq t$.
			\item $(u_{\Z,i} - 1)^m \cdot (\delta_i + w_i) \equiv 2 \delta_i \mod{2 L_{sc}^{(i)}}$ for all $1 \leq i \leq t$.
		\end{enumerate}
	\item If $\Ker(\widetilde{u}-1)^m \not\subseteq [\g_{ad},\g_{ad}]$, then $\Ker(\widetilde{u_i} -1)^m \not\subseteq [\g_{ad}^{(i)}, \g_{ad}^{(i)}]$ for all $1 \leq i \leq t$.
\end{enumerate}
\end{lemma}

\begin{proof}It follows from Lemma \ref{lemma:nilpinUZgen} that $\Ker(\widetilde{u}-1)^m \not\subseteq [\g_{ad},\g_{ad}]$ if and only if there exists $w \in L_{sc}$ such that $(u_{\Z} - 1) \cdot (\delta + w) \equiv 0 \mod{2 L_{ad}}$. 

By~\eqref{eq:S2overZdecomp} we can write every $w \in L_{sc}$ in the form $w = w_1 + \cdots + w_t + w'$, where $w_i \in L_{sc}^{(i)}$ for all $1 \leq i \leq t$ and $w' \in \bigoplus_{i \neq i'} V_{\Z}^{(i)} V_{\Z}^{(i')}$. Since $u_{\Z,i}$ acts trivially on $V_{\Z}^{(i')}$ for $i \neq i'$, we have $$(u_{\Z} - 1)^m \cdot (\delta + w) = (u_{\Z,1} - 1) \cdot (\delta_1 + w_1) + \cdots + (u_{\Z,t} - 1) \cdot (\delta_t + w_t) + (u_{\Z} - 1) \cdot w'.$$ Here $(u_{\Z,i} - 1) \cdot (\delta_i + w_i) \in L_{ad}^{(i)}$ for all $1 \leq i \leq t$, and $(u_{\Z} - 1) \cdot w' \in \bigoplus_{i \neq i'} V_{\Z}^{(i)} V_{\Z}^{(i')}$. Thus the result follows similarly to the proof of Lemma \ref{lemma:GADreduceNIL}.\end{proof}

