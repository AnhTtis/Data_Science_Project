
We continue with the setup of the previous section. Let $G_{sc} = \Sp(V)$ be simply connected of type $C_{\ell}$ with Lie algebra $\g_{sc} = \mathfrak{sp}(V)$. In this section, we will describe the Jordan block sizes of nilpotent elements $e \in \mathfrak{sp}(V)$ on $\g_{sc}/Z(\g_{sc}) \cong (\Ker \varphi)^*$ (Lemma \ref{lemma:kerphiisdual}). As in the previous section, we describe the Jordan block sizes of $e$ on $\Ker \varphi$ in terms of Jordan block sizes of $e$ on $S^2(V)$. Note that the Jordan normal form of $e$ on $S^2(V)$ is known by the results described in Section \ref{section:tensor}. 

We begin by reducing the calculation to the orthogonally indecomposable case. After this we consider the different orthogonally indecomposable $K[e]$-modules in turn, and by combining the results we obtain Proposition \ref{prop:nilKeraction}, which is analogous to Proposition \ref{prop:unipKeraction}.

\begin{lemma}\label{lemma:NILS2decomp}
Let $e \in \mathfrak{sp}(V)$ be unipotent. Suppose that we have a orthogonal decomposition $V = U_1 \perp \cdots \perp U_t$ as $K[e]$-modules. Denote the action of $e$ on $S^2(V)$ by $\widetilde{e}$, and the action of $e$ on $S^2(U_i)$ by $\widetilde{e_i}$. Let $m \geq 0$ be an integer. 

Then $\Ker(\widetilde{e})^m \subseteq \Ker \varphi$ if and only if $\Ker(\widetilde{e_i})^m \subseteq \Ker \varphi$ for all $1 \leq i \leq t$.
\end{lemma}

\begin{proof}Follows with the same proof as Lemma \ref{lemma:unipS2decomp}.\end{proof}


\begin{lemma}\label{lemma:S2indecompWlNIL}
Let $e \in \mathfrak{sp}(V)$ be nilpotent such that $V \downarrow K[e] = W(\ell)$. Let $\widetilde{e}$ be the action of $e$ on $S^2(V)$, and denote $\alpha = \nu_2(\ell)$. Then the following hold:
	\begin{enumerate}[\normalfont (i)]
		\item $\Ker(\widetilde{e})^{2^{\alpha}-1} \subseteq \Ker \varphi$.
		\item $\Ker(\widetilde{e})^{2^{\alpha}} \not\subseteq \Ker \varphi$.
	\end{enumerate}
\end{lemma}

\begin{proof}We proceed similarly to the proof of Lemma \ref{lemma:S2indecompWl}. By definition of $W(\ell)$, we have a totally singular decomposition $V = W \oplus W^*$, where $W$ and $W^*$ are $K[e]$-modules on which $e$ acts with a single Jordan block of size $\ell$. This induces a decomposition $$S^2(V) = S^2(W) \oplus S^2(W^*) \oplus WW^*$$ of $K[e]$-modules, where $WW^* \cong W \otimes W^*$. Thus $$\Ker(\widetilde{e}) = \Ker(\widetilde{e}_{S^2(W)}) \oplus \Ker(\widetilde{e}_{S^2(W)}) \oplus \Ker(\widetilde{e}_{WW^*}).$$ We have $S^2(W), S^2(W^*) \subseteq \Ker \varphi$ since $W$ and $W^*$ are totally singular. Furthermore, the smallest Jordan block size in $W \otimes W^*$ is $2^{\alpha}$ by Proposition \ref{prop:uninilsim} and \cite[Lemma 4.2]{KorhonenJordanGood}, so (i) follows from Lemma \ref{jordanrestrictionNIL}.

Next we consider (ii). We have $V \downarrow K[e^{2^{\alpha}}] = W(\ell/2^{\alpha})^{2^{\alpha}}$ by Lemma \ref{lemma:nilsquaredecomp}. Therefore by Lemma \ref{lemma:NILS2decomp}, it suffices to consider the case where $\alpha = 0$, in which case $\ell$ is odd. In this case, choose a basis $v_1$, $\ldots$, $v_{\ell}$ of $W$, and let $w_1$, $\ldots$, $w_{\ell}$ be the corresponding dual basis in $W^*$, so $b(v_i,w_j) = \delta_{i,j}$ for all $1 \leq i,j \leq \ell$. Then $\gamma = \sum_{1 \leq i \leq \ell} v_iw_i$ is annihilated by $e$ and $\varphi(\gamma) = \ell \neq 0$, so $\Ker(\widetilde{e}) \not\subseteq \Ker \varphi$, as required.\end{proof}

\begin{lemma}\label{lemma:S2indecompWlkNIL}
Let $e \in \mathfrak{sp}(V)$ be nilpotent such that $V \downarrow K[e] = W_k(\ell)$, where $0 < k < \ell/2$. Let $\widetilde{e}$ be the action of $e$ on $S^2(V)$, and denote $\alpha = \nu_2(\ell)$. Then the following hold:
	\begin{enumerate}[\normalfont (i)]
		\item $\Ker(\widetilde{e})^{2^{\alpha}-1} \subseteq \Ker \varphi$.
		\item $\Ker(\widetilde{e})^{2^{\alpha}} \not\subseteq \Ker \varphi$.
	\end{enumerate}
\end{lemma}

\begin{proof}Let $v_1$, $\ldots$, $v_n$ be the basis used in the definition of $W_k(\ell)$. Then $b(v_i,v_j) = 1$ if $i+j = n+1$ and $0$ otherwise. Furthermore, the action of $e$ is defined by \begin{align*}
ev_1 &= 0, \\
ev_{\ell+1} &= 0 \\
ev_i &= v_{i-1} \text{ for all } i \not\in \{1,\ell+1,n-k+1\} \\
ev_{n-k+1} &= v_{n-k} + v_{k}. \end{align*} We will denote $v_j = 0$ for all $j \leq 0$ and $j > n$.

Let $W = \langle v_i : 1 \leq i \leq \ell \rangle$ and $Z = \langle e^i v_n : 0 \leq i < \ell \rangle$. Then $V = W \oplus Z$, where $W$ and $Z$ are $K[e]$-modules on which $e$ acts as a single Jordan block of size $\ell$. We take $w_1$, $\ldots$, $w_{\ell}$ as a basis of $Z$, where $w_i := e^{i-1} v_n$ for all $1 \leq i \leq \ell$.

For the claims, we will first consider the case where $\alpha = 0$, so $\ell$ is odd. In this case (i) is trivial. For (ii), note that the vector $\gamma = \sum_{1 \leq j \leq \ell} v_j w_{j}$ is annihilated by the action of $e$. For all $1 \leq j \leq \ell$, we have $w_{j} = v_{n-j+1}$ or $w_{j} = v_{n-j+1} + v_r$ for some $1 \leq r \leq k$. Therefore $\varphi(\gamma) = \ell \neq 0$, so $\Ker(\widetilde{e}) \not\subseteq \Ker \varphi$, as claimed.

Suppose then that $\alpha > 0$. For (i), we will first prove that $\Ker(\widetilde{e}) \subseteq \Ker \varphi$. To this end, note that the decomposition $V = W \oplus Z$ induces a decomposition $$S^2(V) = S^2(W) \oplus S^2(Z) \oplus WZ$$ of $K[e]$-modules, where $WZ = \{ wz : w \in W, z \in Z\}$ is isomorphic to $W \otimes Z$. It follows from \cite[Proof of Theorem 1.6]{KorhonenSymExt2021} that $S^2(W)^e = W^{[2]}$ and $S^2(Z)^e = Z^{[2]}$, so $$\Ker(\widetilde{e}) = W^{[2]} \oplus Z^{[2]} \oplus \Ker(\widetilde{e}_{WZ}).$$ It is clear that $W^{[2]}, Z^{[2]} \subseteq \Ker \varphi$, and $\Ker(\widetilde{e}_{WZ}) \subseteq \Ker \varphi$ since the smallest Jordan block size in $W \otimes Z$ is $2^{\alpha}$ (Proposition \ref{prop:uninilsim} and \cite[Lemma 4.2]{KorhonenJordanGood}).

Thus $\Ker(\widetilde{e}) \subseteq \Ker \varphi$. This also proves (i) in the case where $\alpha = 1$, so suppose next that $\alpha > 1$. We have $V \downarrow K[e^{2^{\alpha-1}}] = W(\ell/2^{\alpha-1})^{2^{\alpha-1}}$ by Lemma \ref{lemma:nilsquaredecomp}. Since $\ell/2^{\alpha-1}$ is even, it follows from Lemma \ref{lemma:S2indecompWlNIL} and Lemma \ref{lemma:NILS2decomp} that \begin{equation}\label{eq:wklalpha1}\Ker(\widetilde{e})^{2^{\alpha-1}} \subseteq \Ker \varphi.\end{equation}

In $S^2(W)$, $S^2(Z)$, and $W \otimes Z$ all Jordan block sizes of $e$ are powers of two (Theorem \ref{thm:extsymnilpotent}). In particular $\widetilde{e}$ has no Jordan blocks of size $2^{\alpha-1} < d < 2^{\alpha}$, so by~\eqref{eq:wklalpha1} and Lemma \ref{jordanrestrictionNIL} we conclude that $\Ker(\widetilde{e})^{2^{\alpha}-1} \subseteq \Ker \varphi$, as required.

It remains to prove (ii) in the case where $\alpha > 0$. To this end, note first that we have $V \downarrow K[e^{2^{\alpha}}] = W(\ell/2^{\alpha})^{2^{\alpha}}$ by Lemma \ref{lemma:nilsquaredecomp}. It follows then from Lemma \ref{lemma:S2indecompWlNIL} and Lemma \ref{lemma:NILS2decomp} that $\Ker(\widetilde{e})^{2^{\alpha}} \not\subseteq \Ker \varphi$.\end{proof}

\begin{lemma}\label{lemma:S2indecompV2lNIL}
Let $e \in \mathfrak{sp}(V)$ be nilpotent such that $V \downarrow K[e] = V(2\ell)$. Let $\widetilde{e}$ be the action of $e$ on $S^2(V)$, and denote $\alpha = \nu_2(2\ell)$. Then the following hold:
	\begin{enumerate}[\normalfont (i)]
		\item $\Ker(\widetilde{e})^{2^{\alpha}-1} \subseteq \Ker \varphi$.
		\item $\Ker(\widetilde{e})^{2^{\alpha}} \not\subseteq \Ker \varphi$.
	\end{enumerate}
\end{lemma}

\begin{proof}
We have $\Ker(\widetilde{e}) = V^{[2]}$ by \cite[Proof of Theorem 1.6]{KorhonenSymExt2021}, so $\Ker(\widetilde{e}) \subseteq \Ker \varphi$. Since $e$ has no Jordan blocks of size $1 < d < 2^{\alpha}$ on $S^2(V)$ (Theorem \ref{thm:extsymnilpotent}), it follows from Lemma \ref{jordanrestrictionNIL} that $\Ker(\widetilde{e})^{2^{\alpha}-1} \subseteq \Ker \varphi$.

Next we consider claim (ii). Since $\alpha = \nu_2(2\ell) > 0$, we have $V \downarrow K[e^{2^{\alpha}}] = W(\ell/2^{\alpha-1})^{2^{\alpha-1}}$ by Lemma \ref{lemma:nilsquaredecomp}. Because $\ell/2^{\alpha-1}$ is odd, we conclude from Lemma \ref{lemma:S2indecompWlNIL} and Lemma \ref{lemma:NILS2decomp} that $\Ker(\widetilde{e})^{2^{\alpha}} \not\subseteq \Ker \varphi$.\end{proof}

\begin{prop}\label{prop:nilKeraction}
Let $e \in \mathfrak{sp}(V)$ be nilpotent, with orthogonal decomposition $$V \downarrow K[e] = \sum_{1 \leq i \leq t} W(m_i) \perp \sum_{1 \leq j \leq t'} W_{k_j}(\ell_j) \perp \sum_{1 \leq r \leq t''} V(2d_r).$$ Denote by $\alpha \geq 0$ the largest integer such that $2^{\alpha} \mid m_i,\ell_j,2d_r$ for all $i$, $j$, and $r$.

Then \begin{align*}
		\g_{sc} &\cong W_{2^{\alpha}} \oplus V' \\
		\g_{sc}/Z(\g_{sc}) &\cong W_{2^{\alpha}-1} \oplus V' \end{align*} for some $K[e]$-module $V'$.
\end{prop}

\begin{proof}Let $\widetilde{e}$ be the action of $e$ on $S^2(V)$. By Lemma \ref{lemma:LieSpS2}, Lemma \ref{lemma:kerphiisdual}, and Lemma \ref{jordanrestrictionNIL}, the proposition is equivalent to the statement that $\Ker(\widetilde{e})^{2^{\alpha}-1} \subseteq \Ker \varphi$ and $\Ker(\widetilde{e})^{2^{\alpha}} \not\subseteq \Ker \varphi$. 

Thus similarly to the proof of Proposition \ref{prop:unipKeraction}, the result follows from Lemma \ref{lemma:NILS2decomp}, together with Lemma \ref{lemma:S2indecompWlNIL}, Lemma \ref{lemma:S2indecompWlkNIL}, and Lemma \ref{lemma:S2indecompV2lNIL}.\end{proof}

