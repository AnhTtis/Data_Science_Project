
We consider $G = \Sp(V)$ with Lie algebra $\mathfrak{sp}(V)$, where $\dim V = 2\ell$. We recall the classification of nilpotent orbits in $\mathfrak{sp}(V)$ due to Hesselink \cite{Hesselink}. For more details, we refer to \cite{Hesselink} and \cite[Chapter 4, Chapter 5]{LiebeckSeitzClass}.

For a nilpotent element $e \in \mathfrak{sp}(V)$, define the \emph{index function} $\chi_V: \Z_{\geq 0} \rightarrow \Z_{\geq 0}$ corresponding to $e$ by $$\chi_V(m) := \operatorname{min} \{n \geq 0 : b(e^{n+1}v, e^{n}v) = 0 \text{ for all } v \in \Ker e^m \}.$$ Let $0 < d_1 < \cdots < d_t$ be the Jordan block sizes of $e$, and let $n_i$ be the multiplicity of Jordan block size $d_i$ for $e$. By a result of Hesselink \cite[Theorem 3.8]{Hesselink}, the nilpotent orbit of $e$ is determined by the integers $d_i$, $n_i$, and the function $\chi_V$. 

Hesselink also proved that it suffices to only consider the values of $\chi_V$ on the Jordan block sizes $d_1$, $\ldots$, $d_t$.

\begin{lause}[{\cite[3.9]{Hesselink}}]\label{thm:hesselinkmainthm}
Let $e \in \mathfrak{sp}(V)$ be nilpotent with index function $\chi = \chi_V$ on $V$. Let $0 < d_1 < \cdots < d_t$ be the Jordan block sizes of $e$, and let $n_i$ be the multiplicity of a Jordan block of size $d_i$ for $e$. Then the following statements hold:
	\begin{enumerate}[\normalfont (i)]
		\item The nilpotent orbit of $e$ is determined by the symbol $({d_1}_{\chi(d_1)}^{n_1}, \ldots, {d_t}_{\chi(d_t)}^{n_t})$.
		\item $\chi(d_1) \leq \cdots \leq \chi(d_t)$ and $d_1 - \chi(d_1) \leq \cdots \leq d_t - \chi(d_t)$.
		\item $0 \leq \chi(d_i) \leq d_i/2$ for all $1 \leq i \leq t$.
		\item $\chi(d_i) = d_i/2$ if $n_i$ is odd.
	\end{enumerate}
\end{lause}

\begin{remark}
Conversely, consider integers $d_i$, $n_i$, $\chi(d_i)$ with $0 < d_1 < \cdots < d_t$ and $\sum_{i = 1}^t n_i d_i = \dim V$, such that conditions (ii) -- (iv) of Theorem \ref{thm:hesselinkmainthm} hold. Then there exists a nilpotent element $e \in \mathfrak{sp}(V)$ with corresponding symbol $({d_1}_{\chi(d_1)}^{n_1}, \ldots, {d_t}_{\chi(d_t)}^{n_t})$ \cite[3.9]{Hesselink}.
\end{remark}

Similarly to the unipotent case, we can phrase the classification in terms of bilinear modules. For a Lie algebra $\mathfrak{w}$, a \emph{bilinear $\mathfrak{w}$-module} $(W, \beta)$ is a finite-dimensional $\mathfrak{w}$-module $W$ equipped with a $\mathfrak{w}$-invariant bilinear form $\beta$, so $\beta(Xv,w) + \beta(v,Xw) = 0$ for all $X \in \mathfrak{w}$ and $v,w \in W$. Two bilinear $\mathfrak{w}$-modules $(W,\beta)$ and $(W',\beta')$ are said to be isomorphic, if there exists an isomorphism $W \rightarrow W'$ of $\mathfrak{w}$-modules which is also an isometry.

Let $e, e' \in \mathfrak{sp}(V)$ be nilpotent. Choose a power of two $q$ such that $e^q = (e')^q = 0$, so that $V \downarrow K[e]$ and $V \downarrow K[e']$ are $\mathfrak{w}_q$-modules. Then $e$ and $e'$ are conjugate under the action of $\Sp(V)$ if and only if $V \downarrow K[e] \cong V \downarrow K[e']$ as bilinear $\mathfrak{w}_q$-modules.

For nilpotent $e \in \mathfrak{sp}(V)$, it is clear that we can write $V \downarrow K[e] = V_1 \perp \cdots \perp V_t$, where $V_i$ are \emph{orthogonally indecomposable} $K[e]$-modules. (Here orthogonally indecomposable is defined similarly to the group case.) By the next lemma, the index function $\chi_V$ is determined by its restriction to the orthogonally indecomposable summands of $V \downarrow K[e]$.

\begin{lemma}[{\cite[Lemma 5.2]{LiebeckSeitzClass}}]\label{lemma:maxval}
Let $e \in \mathfrak{sp}(V)$ be nilpotent and assume $V \downarrow K[e] = W_1 \perp W_2$ as $K[e]$-modules. Then $\chi_V(m) = \max \{ \chi_{W_1}(m), \chi_{W_2}(m) \}$ for all $m \geq 0$.
\end{lemma}

The orthogonally indecomposable modules were classified by Hesselink. In the case of $\mathfrak{sp}(V)$, there are three types of orthogonally indecomposable modules, defined as follows. (Similar definitions are given in \cite[Section 5.1]{LiebeckSeitzClass}.)

\begin{maar}\label{def:V2l}
For $\ell \geq 1$, we define the module $V(2\ell)$ as follows. Let $n = 2\ell$, and suppose that $V$ has basis $v_1$, $\ldots$, $v_{n}$ with $b(v_i,v_j) = 1$ if $i+j = n+1$ and $0$ otherwise. Define $e: V \rightarrow V$ by \begin{align*} e v_1 &= 0 \\ e v_i &= v_{i-1} \text{ for all } 1 < i \leq n.\end{align*} Then $e \in \mathfrak{sp}(V)$, and we define $V(2\ell)$ as the bilinear $K[e]$-module $V \downarrow K[e]$.
\end{maar}

\begin{maar}\label{def:Wl}
For $\ell \geq 1$, we define the module $W(\ell)$ as follows. Let $n = 2\ell$, and suppose that $V$ has basis $v_1$, $\ldots$, $v_{n}$ with $b(v_i,v_j) = 1$ if $i+j = n+1$ and $0$ otherwise. Define $e: V \rightarrow V$ by \begin{align*} e v_1 &= 0, &  e v_i &= v_{i-1} \text{ for all } 1 < i \leq \ell. \\
e v_{\ell+1} &= 0, &  e v_i &= v_{i-1} \text{ for all } \ell+1 < i \leq n.\end{align*} Then $e \in \mathfrak{sp}(V)$, and we define $W(\ell)$ as the bilinear $K[e]$-module $V \downarrow K[e]$.
\end{maar}

\begin{maar}\label{def:Wkl}
For $\ell \geq 1$ and $0 < k < \ell/2$ we define the module $W_k(\ell)$ as follows. Let $n = 2\ell$, and suppose that $V$ has basis $v_1$, $\ldots$, $v_{n}$ with $b(v_i,v_j) = 1$ if $i+j = n+1$ and $0$ otherwise. Define $e: V \rightarrow V$ by \begin{align*} e v_1 &= 0, & e v_{\ell+1} &= 0 \\ e v_{n-k+1} &= v_{n-k} + v_k & e v_i &= v_{i-1} \text{ for all } i \not\in \{ 1,\ell+1,n-k+1 \}  \end{align*} Then $e \in \mathfrak{sp}(V)$, and we define $W_k(\ell)$ as the bilinear $K[e]$-module $V \downarrow K[e]$.
\end{maar}

The fact that the modules in Definition \ref{def:V2l} -- \ref{def:Wkl} agree with those described by Hesselink in \cite[Proposition 3.5]{Hesselink} is seen as follows. 

	\begin{itemize}
		\item In Definition \ref{def:V2l}, this is clear from the fact that $V \downarrow K[e] = W_{2\ell}$, so $V \downarrow K[e] = V(2\ell)$ as defined by Hesselink. 
		\item In Definition \ref{def:Wl}, we have $V \downarrow K[e] = W_{\ell} \oplus W_{\ell}$. It is easy to see that $b(ev,v) = 0$ for all $v \in V$, so $\chi_V(m) = 0$ for all $m \geq 0$. It follows from \cite[Proposition 3.5, Theorem 3.8]{Hesselink} that $V \downarrow K[e]$ is isomorphic to the module $W(\ell)$ defined by Hesselink.
		\item In Definition \ref{def:Wkl} we have used the representative given in \cite{KorhonenStewartThomas}, which gives $W_k(\ell)$ by \cite[Lemma 3.4]{KorhonenStewartThomas}.
	\end{itemize}

\begin{lause}[{\cite[Proposition 3.5]{Hesselink}}]\label{thm:hesselinkNILindecomp}
Let $e \in \mathfrak{sp}(V)$ be nilpotent such that $V \downarrow K[e]$ is orthogonally indecomposable. Then $V \downarrow K[e]$ is isomorphic to $V(2\ell)$, $W(\ell)$, or $W_k(\ell)$ for some $0 < k < \ell/2$. These modules are characterized by the following properties:

\begin{center}
	\begin{tabular}{l|l|l}
	  \multicolumn{1}{c|}{$V \downarrow K[e]$} & \multicolumn{1}{c|}{Jordan normal form on $V$} & \multicolumn{1}{c}{$\chi_V$} \\ \hline
	  $V(2\ell)$  & $W_{2\ell}$ & $\chi_V(2\ell) = \ell$ \\
		$W(\ell)$   & $W_{\ell} \oplus W_{\ell}$ & $\chi_V(\ell) = 0$ \\
		$W_k(\ell)$ \normalfont{($0 < k < \ell/2$)} & $W_{\ell} \oplus W_{\ell}$ & $\chi_V(\ell) = k$
	\end{tabular}
\end{center}

\end{lause}

By Theorem \ref{thm:hesselinkNILindecomp}, for every nilpotent $e \in \mathfrak{sp}(V)$ we have an orthogonal decomposition $$V \downarrow K[e] = U_1 \perp \cdots \perp U_t,$$ where for all $1 \leq i \leq t$ we have $U_i \cong V(2\ell_i)$, $U_i \cong W(\ell_i)$, or $U_i \cong W_{k_i}(\ell_i)$ ($0 < k_i < \ell_i/2$) for some integer $\ell_i \geq 1$. 

As in the unipotent case, the orthogonally indecomposable summands and their number is not uniquely determined. For example, by Lemma \ref{lemma:maxval} and Theorem \ref{thm:hesselinkmainthm} (i), we have isomorphisms \begin{align*} W(d) \perp V(d') &\cong W_{d'/2}(d) \perp V(d') & \text{ for } d > d' > 0 \text{ even}; \\ W(d) \perp V(d) &\cong V(d) \perp V(d) \perp V(d) & \text{ for } d > 0 \text{ even};\end{align*} of bilinear $K[e]$-modules.

We end this section with two observations about orthogonally indecomposable modules of the form $\sum_{1 \leq i \leq t} W(m_i)$.

\begin{lemma}\label{lemma:nilsingularcharac}
Let $e \in \mathfrak{sp}(V)$ be nilpotent. Then the following statements are equivalent:
	\begin{enumerate}[\normalfont (i)]
		\item There is an orthogonal decomposition $V \downarrow K[e] = \sum_{1 \leq i \leq t} W(m_i)$ for some integers $m_1$, $\ldots$, $m_t$.
		\item There is a totally singular decomposition $V = W \oplus Z$, where $W$ and $Z$ are $K[e]$-submodules of $V$.
		\item $b(ev,v) = 0$ for all $v \in V$.
	\end{enumerate}
\end{lemma}

\begin{proof}(cf. \cite[Lemma 6.12]{Korhonen2020Hesselink}) We prove that (i) $\Rightarrow$ (ii) $\Rightarrow$ (iii) $\Rightarrow$ (i).

(i) $\Rightarrow$ (ii): It is clear from Definition \ref{def:Wl} that each $W(m_i)$ has a decomposition $W(m_i) = W_i \oplus Z_i$, where $W_i$ and $Z_i$ are totally singular $K[e]$-submodules with $W_i \cong W_{m_i} \cong Z_i$. Therefore (i) implies (ii).

(ii) $\Rightarrow$ (iii): We have $b(ew,w) = b(ez,z) = 0$ for all $w \in W$ and $z \in Z$ since $W$ and $Z$ are $e$-invariant and totally singular. Thus $b(e(w+z),w+z) = b(ew,z) + b(ez,w) = 0$ for all $w \in W$ and $z \in Z$, since $e \in \mathfrak{sp}(V)$.

(iii) $\Rightarrow$ (i): Suppose that $b(ev,v) = 0$ for all $v \in V$. Then for the index function of $e$ we have $\chi_V(m) = 0$ for all $m \geq 1$. Therefore every orthogonally indecomposable summand of $V \downarrow K[e]$ must be of the form $W(m)$ for some $m \geq 1$ (Theorem \ref{thm:hesselinkNILindecomp} and Lemma \ref{lemma:maxval}), so (i) holds.\end{proof}

\begin{lemma}\label{lemma:nilsquaredecomp}
Let $e \in \mathfrak{sp}(V)$ be nilpotent. Then $V \downarrow K[e^2] = \sum_{1 \leq i \leq t} W(m_i)$ for some integers $m_1$, $\ldots$, $m_t$. 
\end{lemma}

\begin{proof}
We have $b(e^2v,v) + b(ev,ev) = 0$ for all $v \in V$ since $e \in \mathfrak{sp}(V)$, so $b(e^2v,v) = 0$ for all $v \in V$. Now the claim follows from Lemma \ref{lemma:nilsingularcharac}.\end{proof}

