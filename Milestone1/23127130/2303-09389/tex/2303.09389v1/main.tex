\documentclass[%
 reprint,
 superscriptaddress,
 amsmath,amssymb,
 prl
]{revtex4-2}
\usepackage{graphicx}% Include figure files
%\usepackage{subcaption}
\usepackage{dcolumn}% Align table columns on decimal point
\usepackage{bm}% bold math
%\usepackage{amsmath,latexsym}
\usepackage{csquotes}
\usepackage[english]{babel}
\usepackage{siunitx}
\usepackage{booktabs}
\usepackage{tabularx}
%\usepackage{amssymb}
%\usepackage{caption}

\usepackage{braket}
%\usepackage{subfigure}
%\usepackage{caption}
\usepackage[normalem]{ulem}
\usepackage[caption=false, justification=centerlast]{subfig}
\usepackage{todonotes}

\usepackage[%
colorlinks=true,
urlcolor=blue,
linkcolor=blue,
citecolor=blue
]{hyperref}

\usepackage{tabularx}

\newcommand{\PKU}{Center for Applied Physics and Technology, HEDPS, and SKLNPT, School of Physics, Peking University, Beijing 100871, China}

\newcommand{\FSCN}{Frontiers Science Center for Nano-optoelectronic, Peking University, Beijing 100094, China}

\newcommand{\GILP}{Guangdong Institute of Laser Plasma Accelerator Technology, Guangzhou, China}

\newcommand{\IOQ}{Institute of Optics and Quantum Electronics, Max-Wien-Platz 1, 07743 Jena, Germany}
\newcommand{\HIJ}{Helmholtz-Institute Jena, Fröbelstieg 3, 07743 Jena, Germany}

\begin{document}
	      
%\preprint{APS/123-QED}

\title{Parametric amplification of betatron oscillation by bubble breathing in laser wakefield accelerator}% Force line breaks with \\

\author{Bifeng Lei}
\email{blei@pku.edu.cn}
\affiliation{\PKU}

\author{Daniel Seipt}
\affiliation{\HIJ}
\affiliation{\IOQ}

\author{Bin Liu}
\email{liubin@glapa.cn}
\affiliation{\GILP}

\author{Matt Zepf}
\affiliation{\HIJ}
\affiliation{\IOQ}

\author{Bin Qiao}
\email{bqiao@pku.edu.cn}
\affiliation{\PKU}
\affiliation{\FSCN}

\date{\today}% It is always \today, today,


\begin{abstract}
%%%
Betatron oscillation of trapped electrons in laser-driven long-distance propagating plasma bubble has been investigated with the help of particle-in-cell simulations and theoretical analysis.
%%%
Parametric oscillation of the trapped electron beam is identified as a result of bubble breathing which develops with the steepening and depletion of the laser pulse.
%%
It leads to parametric amplification of the betatron oscillation of the electron beam with the exponential growth of its amplitude. 
%
It results in severe degradation of the qualities of the electron beam and thus is important for long-scale laser wakefield acceleration. 
%
%
\end{abstract}

\maketitle

%\tableofcontents

The development of laser wakefield acceleration (LWFA) has attracted much attention over past decades. 
%%%%
It is promising in delivering low-cost and high-energy electron beams 
which have a far-reaching impact on many areas of nature science, e.g. in strong field quantum electrodynamics~\cite{Poder2018EXP,Cole2018EXP, turner2022strong}, free electron laser~\cite{Wang2021FREE, labat2022seeded,pompili2022free,Galletti2022stable} and particle physics~\cite{Schroeder2010,SCHROEDER2016113,Tanaka2020cur}. 
%%%
%%Most recent efforts are working on 
This is achieved mainly 
%%The achievement is mainly done 
by continuously improving the electron 
%%%
beam qualities of energy~\cite{Leemans2014multi,Gonsalves2019PW}, charge~\cite{Gotzfried2020PHY,Yong2018ultrahigh}, current~\cite{lundh2011few}, emittance~\cite{Brunetti2010LOW} and energy spread~\cite{Ke2021NEAR}.
%
Over the past two decades, great progress has been seen in LWFA, including electron beams with energy close to $8~\si{GeV}$ accelerated in long-scale of 20 centimeter plasma~\cite{Gonsalves2019PW}. 
%
Meter-scale TeV-level electron beams are expected to be realized in the near future (s, e.g., ~\cite{Palastro2020Deph}).
%

In order to realize  more challeging applications with the laser-driven electron beams, 
difficulties are yet to be overcome urgently for the improvement of 
the overall beam qualities and many shots instabilities. 
For example, energy fluctuation and electron beam jitter (EBJ) are the major obstacles for substantial progress to be made. 
%%%%
Besides the experimental efforts, e.g. for better injection~\cite{Bohlen2022sta} or diagnostics~\cite{wan2022direct}, the long-distance acceleration is one of the main research objectives to achieve the high-energy electron beam with excellent qualities at the same time~\cite{Debus2019cir}. 
%


The qualities of an electron beam in LWFA are largely determined by the betatron oscillation of the electrons, which is naturally introduced by the transverse focusing force inside the plasma bubble~\cite{Esarey2009PHY}. 
%%%
The focusing force can be disturbed 
in many cases with, for example, the laser pulse front tilt~\cite{Popp2010all}, high-order modes beating~\cite{Cormier2011con}, non-uniform plasma density~\cite{ma2018ang} or the few-cycle laser pulse with carrier-envelope phase (CEP) effect~\cite{Nerush2009carr,Huijts2021ide}, making the plasma bubble unstable. 
%%
Particularly, the periodic evolution of a plasma bubble has the dominant effects on the electron dynamics~\cite{Nerush2009carr, Chen2016tunable} and thus the beam qualities. 
%
In long-scale LWFA, it is usually ascribed to the oscillation of its centroid trajectory and sheath deformation, 
which intrinsically come from the laser pulse etching and depletion ~\cite{ma2016generation, Rakowski2022tran}. 
% 
While being etched and depleted, the laser front phase evolves and introduces the non-negligible effect in the laser-plasma interaction.
%
It has been shown that bubble wobbling, which describes the centroid oscillation of plasma bubble due to the asymmetric transverse scattering of plasma electrons~\cite{Nerush2009carr}, is the dominant unstable process in short-scale CEP-controlled LWFA and plays an important role in electron injection~\cite{Kim2021sub}, electron beam jitter (EBJ)~\cite{Seidel2022pointing} and betatron radiation~\cite{Chen2021enh}.
%%%%%%
Bubble wobbling can be well controlled by tuning over laser CEP, initial injection,  with additional few-cycle pulse~\cite{Huijts2022wave,Zhang2022car,Kim2021sub}.
%%% 
In long-scale LWFA, with the laser front being seriously depleted, a different unstable process called bubble breathing rises, making the bubble sheath deform and evolve in a breathing-like behaviour~\cite{Kalmykov:2009tn}. 
%%%%%

As far as we know, there still lacks a thorough investigation of the betatron oscillation in long-scale LWFA with the presence of bubble breathing. 
%%%
This becomes urgently needed with the ambition of high energy 
LWFA, especially noting that recent long-scale LWFA experiments in HIJ/Jena~\cite{Seidel2022pointing} and others 
\cite{Leemans2006, Leemans2014aa, Gonsalves2019PW, Gonsalves2015gen}
show clear degeneracy of the beam quality. 
%%% 
In this Letter, with the help of particle-in-cell (PIC) simulations and detailed theoretical analysis, we show that bubble breathing plays a crucial role on betatron oscillation in long-scale LWFA besides bubble wobbling. 
% 
Furthermore, when bubble breathing becomes dominant, parametric amplification of betatron oscillation (PAB) happens in the laser polarization direction, resulting in the exponential growth of the oscillation amplitude.
%%
This results in the significant degradation of the beam qualities.
 
 \begin{figure}[h]
\centering
\centering
	\includegraphics[width=0.49\textwidth]{unstable_bubble}
	\caption{Evolution of plasma bubble at three time instants: $t=$165(a), 500(b), 825(c) with the externally injected electron beam indicated by yellow spots. In each figure, the upper half shows the distribution of electron density $n_p$, and the lower is the density difference between the upper($x>0$) and lower($x<0$) plane calculated as $n_{diff}=n_{p,lower}-n_{p,upper}$. $n_{diff}$ shows the transverse asymmetry of $n_p(t)$. $n_{diff}\approx 0$ indicates a symmetric bubble as shown in (a). The tilted bubble is seen in (b) by the small offset of the upper(green) and lower(purple) boundaries and the deformation is seen by the oscillating boundaries in (c). The black lines show the on-axis laser field at each time.}
	\label{fig:unstable_bubble}
\end{figure}
 
In order to demonstrate the PAB process, 
2D PIC simulations have been carried out with SMILEI~\cite{derouillat_smilei_2018}. The grid resolution is $32\times 16$ per laser wavelength ($\lambda_l = 0.8\mu m$) in longitudinal and transverse directions respectively. 
%
Initially the plasma density is uniform, $n_0=6.28\times 10^{18}\rm{cm^{-3}}$. 
A triangular-like Gaussian laser pulse is configured with the normalized strength $a_0=3$, spotsize $w_0=6$, pulse front duration $\tau_f=2$, and tail $\tau_t=4$ as shown by the black line in Fig.~\ref{fig:unstable_bubble}(a). Be noticed that all quantities are normalized by the plasma parameters, e.g. the time is by plasma frequency $\omega_p=\sqrt{4\pi n_0 e^2/m_e}$ and the length by the plasma wave number $k_p=\omega_p/c$. $m_e$ is mass of electron, $c$ the speed of light in vacuum and $e$ absolute value of electron charge. 
%%%%
The laser temporal profile $f(\zeta)$ is given as $f(\zeta)=\cos (\zeta +\phi_{CEP})h(\zeta)$ with CEP $\phi_{CEP}$ and the temporal pulse envelope profile as $h(\zeta)=e^{-(\zeta-\zeta_0)^2/\tau_0^2}$ for front $\zeta>\zeta_0$, $\tau_0=\tau_f$ and for tail $\zeta\leq\zeta_0$, $\tau_0=\tau_t$, where $\zeta=z-t$ and $\zeta_0$ is the temporal center. 
%
We choose this
temporal asymmetric Gaussian
profile to mimic an etched laser pulse after a long-distance propagation in plasma for computational efficiency.
%
Three evolution stages of bubble and electron beam oscillation have been observed while the laser pulse propagates. 
%
In the early section, 
the laser pulse with a smooth front drives a stable and transversely symmetric bubble as shown in Fig.~\ref{fig:unstable_bubble}(a), where the trapped electrons undergo pure betatron oscillation.
%
With the laser front being etched to be sharp, 
the bubble becomes tilted in Fig.~\ref{fig:unstable_bubble}(b) , where the electrons undergoes the driven harmonic oscillation.
% 
and, in the later time, the bubble sheath becomes unstable when the laser pulse is further depleted in Fig.~\ref{fig:unstable_bubble}(c) where the parametric oscillation of the trapped electrons happens.
%%%%
Essentially, the unstable bubble develops
%
from the laser front-phase-dependent asymmetric scattering of the plasma electrons in the laser polarization direction.
%
It happens when the scattering time of the electrons transversely away from the laser front is short in comparison with the laser field period $T_l$ where the electron dynamics can not be treated by averaging over the laser oscillation. The collective behavior of the plasma wave then exhibits transverse asymmetry and depends on the local laser phase relating to CEP.


\begin{figure}
	\includegraphics[width=0.5\textwidth]{bubble_wobble.pdf}
	\caption{ Time evolution obtained from the PIC simulation in Fig.~\ref{fig:unstable_bubble}.  
	(a) and (b): Transverse force $F_{x}$ and $K(t)=2dF_x/dx$
	on the trapped electron in $\hat{x}$-direction. 
	%
The solid black lines are plotted on axis $x=0$ for (a) and off-axis $x=1.5$ for (b) and the dashed green lines are theoretically calculated from Eqs.~\eqref{eq:xct} and \eqref{eq:Kt} with $R=2\sqrt{a_0}$, $K_B=1.0$, and  $K_E=1.5$. 
	%
	(c): Colormap: evolution of the laser field with its pulse front at $a_0=1$(red), steepness function $\mathcal{S}(t)$ (inset) and position of the externally injected electron beam (purple). The theoretical pulse front (black) is calculated with a pulse front velocity $v_f=1.5/\gamma_p^2$. Purple line indicates the position of electron beam relative to initial laser front.
	%
	(d): Centroid trajectory of the trapped electron beam in $\hat{x}$(black)- and $\hat{y}$(blue)-direction. The dashed lines show the theoretical scaling from \eqref{eq:xFw} (dashed green)  and Eq.~\eqref{eq:x_nPBA} (dashed red).
	%
	The vertical dashed line in the inset marks the position $\omega_{\beta}\simeq 3\omega_w/2$.}
	\label{fig:bubble_wobble}
\end{figure} 


More details of the three stages are revealed in Fig.~\ref{fig:bubble_wobble}. 
%
Initially, in the first stage, the bubble keeps spherical and the electron experiences the linear focusing force, e.g. in $\hat{x}$-direction ${F}_{x}=E_x-B_y=x/2$~\cite{Kostyukov2004phe,Lu2006non}, 
where $E_x$ and $B_y$ are the electric and magnetic field in $\hat{x}$- and $\hat{y}$-directions respectively, normalized by $E_{0}=m_e c \omega_p /e$.
%
Here, betatron oscillation is described by the time-dependent harmonic oscillator with the betatron damping effect due to acceleration~\cite{Mehrling2017}.
%%%
With the development of the interaction with the plasma electrons, the front of the laser pulse steepens. 
The steepening process of the laser pulse front can be characterized by the steepness function which is defined as $\mathcal{S}(t)=\left|\int f(\zeta^{'})^2 g(\zeta^{'}) d \zeta^{'} / \int f(\zeta^{'})^2 d \zeta^{'}\right|$~\cite{Nerush2009carr} with $g(\zeta^{'})=\int_{-\infty}^{\zeta^{'}} f(\zeta^{''})  d\zeta^{''}$. 
%
As shown in the inset of Fig.~\ref{fig:bubble_wobble}(c), $\mathcal{S}(t)$ increases significantly as the laser pulse is etched during $0<t<300$. 
%
This results in the onset of the significant bubble centroid oscillation (Fig.~\ref{fig:bubble_wobble}(a)) with frequency $\omega_w$.  
%
The wobbling frequency can be calculated from the evolution of the pulse front phase $\zeta_f$ as $\omega_w=d \zeta_f /dt \simeq \omega_l (v_{ph}-v_g+v_{etch})/\omega_p\sim 1/\gamma_p$ with $v_{ph}$, $v_g$ and $v_{etch}$ are phase, group and etching velocity of the laser pulse respectively. In the bubble region, PIC simulations agree with $\omega_w\simeq 3/2\gamma_p$ as shown in Fig.~\ref{fig:bubble_wobble}(a). 
%
The bubble wobble provides the additional driven force for betatron oscillation where
the oscillation amplitude increases linearly(Fig.~\ref{fig:bubble_wobble}(d) and inset).
%%%
Later, with the intense asymmetric scattering of electrons from the steepened laser front 
the dynamics of the laser pulse easily become unstable and periodically evolve with the frequency close to that of the bubble wobbling as observed from PICs in Fig.~\ref{fig:bubble_wobble}(c) after $t>300$. 
%%%
It, in turn, deforms the sheath of the plasma bubble~\cite{Nerush2009carr,ma2016generation} with the frequency $\omega_w$ as shown in Fig.~\ref{fig:bubble_wobble}(b). The deformation becomes significant as the laser pulse is depleted after $t>700$.
%
As a result, the parametric oscillation stage develops with the amplitude increasing exponentially. 
%
The bubble deformation experienced by a trapped electron can be described approximately by 
\begin{align}
	K(t) = 2\frac{d F_x}{d x} \simeq K_B(t) + K_E(t) \sin(\omega_w t + \phi_E) \mathrm{,}
	\label{eq:Kt}
\end{align} 
where the phase $\phi_E$ is trivial and $K_B(t)$ and $K_E(t)$ 
vary slowly in comparison with betatron oscillation. They can be treated approximately as constant locally.
%
This 
simple approximation explains the observation in the simulation
very well, as is shown in Fig.~\ref{fig:bubble_wobble}(b). 
%
$K_B(t)$ is the coefficient of the field strength in a stable ion channel and $K_B=1$ in the evacuated bubble~\cite{Kostyukov2004phe}. 
%
The second term on the right hand side (r.h.s) presents the breathing effect.  
%
As a result, the focusing force can be written in a general form as $F_x(t)=K(t)[x-x_c(t)]/2$ where $x_c(t)$ is the centroid trajectory of the bubble.


The dynamics of an electron in each oscillation stage can be understood by analyzing the solution of the equation of motion of the trapped electrons,
\begin{align}
	\frac{d^2 x}{dt^2} + \alpha \frac{dx}{dt}  + K \omega_{\beta}^2 x = K\omega_{\beta}^2x_c(t) \mathrm{,}
	\label{eq:momentum_equation}
\end{align}
where $\omega_{\beta}=1/\sqrt{2\gamma}$ with $\gamma$ Lorentz factor of the electron,
$\alpha=\dot{\gamma}/\gamma$ represents the damping effect due to acceleration
%%%
\footnote{
Noted that Eq.~\eqref{eq:momentum_equation} can be used to describe the centroid trajectory of a trapped electron beam if its beam size and divergence are small in comparison with the plasma wavelength. Therefore, the discussion here also applies to the electron beam's centroid dynamics to some extent. 
}.
%
For a relativistic linearly polarized Gaussian pulse in $\hat{x}$-direction $a_x = a_0 f(\zeta) e^{-(x^2+y^2)/w_0^2}$, the centroid trajectory can be given as~\cite{Nerush2009carr}
\begin{align}
	x_c(t)\simeq a_c \sin(\omega_w t + \phi_w) \mathrm{,}
	\label{eq:xct}
\end{align}
where $\phi_w$ is wobble phase. The amplitude $a_c\sim \theta_{\rm{tilt}} R\simeq a_0 \mathcal{S}(\tau_f) R/ \gamma_p w_0$ presents the deflection of an electron away from the laser axis over the transverse bubble dimension $R$ with amplitude of the tilt angle $\theta_{\rm{tilt}}$. $\gamma_p = \omega_l/\omega_p$ with $\omega_l=2\pi/T_l$ the laser frequency.
%%%


\textit{Harmonic oscillation}. 
%
In the early stage $K_E=0$ and $x_c\simeq 0$ where the laser pulse propagates shortly in plasma with the smooth front, the electron undergoes the harmonic oscillation in transverse directions with amplitude depending on initial injection offset and angle into bubble~\cite{Esarey2002Syn}.
%%%%

\textit{Driven harmonic oscillation}.
%
As the laser pulse front	 is etched to be sharp where the laser pulse is not depleted too much that $K_E < K_B$ as shown in Fig.~\ref{fig:bubble_wobble}(b) between $200<t<700$,  
%
the bubble wobble externally drives betatron oscillation, presented by r.h.s of Eq.~\eqref{eq:momentum_equation}.
%
We can assume the harmonic solution in such form $x(t)=C_1(t)\cos[\psi_{\beta}(t)] + C_2(t) \sin[\psi_{\beta}(t)]$ with the real amplitudes $C_{1,2}(t)$. By inserting it into Eq.~\eqref{eq:momentum_equation} and considering the adiabatic approximation  where $K_E(t)$, $K_B(t)$ and $\gamma(t)$ varies very slowly in comparing with betatron frequency making $\ddot{C}_{1,2}, \alpha \dot{C}_{1,2} \ll \omega_{\beta}$, the oscillation solution is given by 
\begin{align}
	x(t) \simeq A(t) R(t) \left[ x_{\beta}(t) + x_D(t) \right ]
	\label{eq:x_homos}
\end{align} 
where $x_{\beta}(t)=a_{\beta 0} \cos(\psi_{\beta} + \Phi_{\beta 0})$ presents pure betatron oscillation with amplitude $a_{\beta 0}=\sqrt{x_0^2+2\gamma_0\theta_0^2}$ initial injection offset $x_0$ and angle $\theta_0$ of electron into bubble, and initial betatron phase $\Phi_{\beta 0}=\arctan(x_0/2\gamma_0 \theta_0)$. Betatron phase is defined as $\psi_{\beta}(t)=\int_{t_0}^t \sqrt{K}\omega_{\beta}(t')dt'$. $A(t)=e^{-\int_{t_0}^t \alpha(t')/4 dt'} =(\gamma_0/\gamma)^{1/4}$ and $R(t)=(K_0/K)^{1/4}$ with $K_0=K(t=t_0)$ and $t_0$ is the time when the electron is trapped. $x_D(t)=2\omega_{\beta 0} K_0 \int_{t_0}^{t} A(t^{\prime}) R(t^{\prime})^{5}  x_c(t^{\prime}) \sin[\psi_{\beta}(t) - \psi_{\beta}(t^{\prime})]dt^{\prime}$ is the wobble-driven solution where $\omega_{\beta 0}=1/\sqrt{2\gamma_0}$ and $\gamma_0$ is initial Lorentz factor. With the centroid trajectory in Eq.~\eqref{eq:xct}, the amplitude gives
\begin{align}
	|x_D(t)| \simeq \frac{\omega_{\beta 0} K_0 a_c}{4} \mathcal{F}(t;\omega_{\beta},\omega_w) \mathrm{,}
	\label{eq:xFw}
\end{align}
with the parametric function
$\mathcal{F}(t;\omega_{\beta},\omega_w) \simeq \bigg| \int_{t_0}^{t} A(t^{\prime})R(t^{\prime})^{-5} e^{i \int (\omega_{\beta}-\omega_w) dt''}dt^{\prime} \bigg|$. 
%%
$\mathcal{F}(t;\omega_{\beta},\omega_w)$ presents a linear resonance as $\omega_{\beta} \to 3\omega_w/2$ found numerically in Fig.~\ref{fig:parametric}(b). 
%
Here, the pure betatron oscillation is not important since the amplitude $a_{\beta 0}$ is much smaller than the amplified $|X_D|$ as shown in Fig.~\ref{fig:bubble_wobble}(d) where the electron undergoes pure betatron oscillation in $\hat{y}$-direction. 
%
The weak frequency modulation by $K(t)$ is negligible 
since $K_E$ is small 
by comparing the cases for $K_E=0.3$ and $K_E=0$ in Fig.~\ref{fig:parametric}(a) and (b). Therefore, $\mathcal{F}(t;\omega_{\beta},\omega_w)$ totally characterizes the electron dynamics in this section as shown in Fig.~\ref{fig:bubble_wobble}(d) and Fig.\ref{fig:parametric}(a)(upper), 
%{\bf 
and consequently the evolution of beam qualities, e.g. emittance and EBJ, as discussed later.
%}
%
The condition between the frequency $\omega_{\beta}$ and $\omega_w$ as $\omega_{\beta} \simeq 3\omega_w/2$ sets the threshold for the onset of the linear resonance, which implies $\gamma_0<2\gamma_p^2/9$. 
%
The resonant point is given around $t \simeq (16\gamma_p^2/81-\gamma_0) / \sqrt{a_0}$ if dephasing is neglected.
%
This implies the way to mitigate this linear amplification by shifting the resonant point to an early time where the bubble wobbles weakly, for example, with high $\gamma_0$ in low-density plasma, or with the laser pulse of the long-wavelength and smooth front as shown in Fig.~\ref{fig:betatron_radiation}(a).

\begin{figure}
\centering
	\includegraphics[width=0.48\textwidth]{BP_omegaw_KE.pdf}
	\caption{ Numerical results.
	(a): Trajectory $x(t)$ of a trapped electron  calculated from Eq.~\eqref{eq:momentum_equation} for $K_E=0$ (red), $K_E=0.3$ (black) and $K_E=1.5$ (blue) respectively. The dashed blue line indicates the exponential growth of the amplitude with the growth rate $\Gamma(t)$. 
	%
	(b) Evolution of $\mathcal{F}(t;\omega_{\beta}, \omega_w)$. The vertical dashed lines in (a)(upper) and (b) indicate the resonant point when $\omega_{\beta}=3\omega_w/2$.
	%
	 (c) Evolution of $K_{ET}$(black) in Eq.~\eqref{eq:KE_condition}, $|\Delta \omega|/\omega_w$(red) and $|1-\rho|$(dashed red) in Eq.~\eqref{eq:wb_condition}. The vertical dashed green line shows the position after which Eq.~\eqref{eq:KE_condition} and \eqref{eq:wb_condition} break.
	 %
	 (d) Dependence of maximum betatron oscillation amplitude on $K_E$ and $\omega_w$ after propagating $t=600$, calculated from Eq.~\eqref{eq:momentum_equation}. The black line shows the temporal minimum value of the threshold calculated by Eq.~\eqref{eq:KE_condition}. The parameters used is: $a_0=3, w_0=6,x_0=0.1,\theta_0=10 \rm{\mu rad}, \phi_w=0$ and $K_B=1$ for (a)-(c). $\gamma_0=20$ for (a)(upper) and (b). $\gamma_0=200$ for (a)(lower), (c), and (d) to meet the typical condition in LWFAs, so the time scale is different.}
	\label{fig:parametric}
\end{figure}


\textit{Parametric oscillation}.
As the laser pulse is further depleted to be unguided as is shown in Fig.~\ref{fig:bubble_wobble}(c) after $t>700$, the plasma bubble begins to be seriously deformed where $K_E$ grows to be sufficiently large. The bubble breathing begins to modulate the frequency violently. Betatron oscillation in Eq.~\eqref{eq:momentum_equation} is then dominated by the harmonic part presented by its left hand side (l.h.s) due to the parametric resonance at the frequency $\omega_w/2$~\cite{cartmell_introduction_1990}. By assuming the harmonic solution in the form $x(t) = B_1(t) \cos(\omega_w t/2)+ B_2(t) \sin(\omega_w t/2)$ with the  coefficients $B_{1,2}(t)$ and by inserting it into Eq.~\eqref{eq:momentum_equation}, it approximately gives 
\begin{align}
	x(t) \simeq x_a e^{\Gamma(t)} \sin(\omega_w t /2)\mathrm{,}
	\label{eq:x_nPBA}
\end{align} 
where $x_a$ is the initial amplitude and $\Gamma(t)$ can be given as $\Gamma(t) = \int_0^t [\mu(t') - \alpha(t')]dt'$ with $\mu(t) = (K_B \omega_{\beta}^2/\omega_w) \sqrt{K_E^2/4K_B^2 \left( \omega_w^2/4 K_B \omega_{\beta}^2 - 1 \right)^2}$. 
%
The solution presents the exponential growth
%as shown in  
if $\mu(t) - \alpha(t)>0$. This sets the threshold for $K_E$ as 
\begin{equation}
\centering
	K_E>K_{ET} = 2 K_B \sqrt{\frac{\alpha^2 \omega_w^2}{K_B^2 \omega_{\beta}^4} + \left(\frac{\omega_w^2}{4 K_B \omega_{\beta}^2} - 1 \right)^2} \mathrm{.}
	\label{eq:KE_condition}
\end{equation}
Once this condition fails with the decrease of $\omega_{\beta}$ due to acceleration as shown in Fig.~\ref{fig:parametric}(c), the parametric oscillation evolves into  a stable regime. 
%
Eq.~\eqref{eq:KE_condition} also indicates the frequency band for the parametric amplification to occur as 
\begin{align}
	|\Delta \omega| = |\omega_{\beta} - \omega_w| < |1-\rho| \omega_w \mathrm{,}
	\label{eq:wb_condition}
\end{align}
where $\rho=\sqrt{K_B(M-1)/(K_E^2/4 - K_B^2)}/2$ and $M=\sqrt{1 + (K_E^2/4 - K_B^2)(1 + 16 \alpha^2/\omega_w^2)/K_B^2}$. 
%
Similarly, $|\Delta \omega|$ increases due to acceleration by reducing $\omega_{\beta}$ as shown  in Fig.~\eqref{fig:parametric}(c). 
%
Generally, the parametric amplification depends on the parameters, e.g. $K_E$ and $\omega_w$, as shown in Fig.~\ref{fig:parametric}(d) where it shows a tongue-like structure.
%
Since 
$\omega_w$ is related to the ambient plasma density as 
$\omega_w \sim 1/ \gamma_p \sim \sqrt{n_0}$, 
in principle, it is  possible to control 
PAB %should be controlled 
by adjusting laser plasma parameters.
%
However, this is extremely challenging 
in long-scale LWFA 
since $K_E$ increases relative to $K_B$ 
with the laser pulse being seriously depleted to less intense, 
making Eq.~\eqref{eq:KE_condition} satisfied. 
%
On the other side, PAB requires an initial amplitude $x_a$ to grow. It can be  provided by betatron oscillation in earlier stages 
which is initialized by the initial injection offset or angle as seen in Eq.~\eqref{eq:x_homos}.
%
Notably, the linear amplification discussed above may enhance betatron oscillation and then contribute to PAB substantially by providing large $x_a$. 
%
Therefore, it is very necessary to well mitigate linear amplification in early stage which will be beneficial to suppress PAB finally.
 
Generally, PAB limits the performance of LWFA  in various aspects including electron loss, beam emittance and radiation generation, and most importantly, 
EBJ,
%
which measures the shot-to-shot fluctuation of the electron beam pointing angle~\cite{Seidel2022pointing}, e.g. in $\hat{x}$-direction as $\Delta {\theta}_{J}=\sqrt{\langle {\theta_x}^2 \rangle_J - \langle {\theta_x} \rangle_J^2}$
%
where $\langle \dots \rangle_J$ presents the average over shots and the pointing angle can be approximated for ultra-relativistic electron beam as ${\theta_x}\simeq \langle \dot{x} \rangle_b$ with $\dot{x}=dx/dt$ averaging over electrons in the beam.
%
In the case of linear amplification, it is calculated from Eq.~\eqref{eq:x_homos} as $\dot{x} = A(t)R(t) \Lambda(t) [ a_{\beta}  + 2\omega_{\beta 0} \mathcal{F}(t;\omega_{\beta}, \omega_w) ]/K_0^{1/4} \sim \mathcal{F}(t;\omega_{\beta}, \omega_w)$ where $\Lambda(t)=\alpha +\dot{K}/K$. In PAB, $\dot{x}\sim x_a (\mu-\alpha)e^{\Gamma(t)}$.
%
Therefore, the amplification processes dominate EBJ. 
%
A similar process is seen for emittance which is measured as $\epsilon_x=\sqrt{\langle x^2\rangle \langle p_x^2\rangle - \langle x p_x\rangle^2}$ where $p_x=\gamma \dot{x}$ is the transverse momentum.
%

The degradation of the beam quality can be partially understood by the transverse energy pumping due to the betatron amplifications, which are featured by $p_x$ as $\gamma_x=\sqrt{1 + |p_x|^2}$.  
%
In order to demonstrate this pumping effect clearly, 
%%
three PIC simulations were carried out with a temporal Gaussian or two steepened triangular pulses. 
%
As shown in Fig.~\ref{fig:betatron_radiation}(a), with the triangular pulse of a short wavelength ($\lambda_0=0.8\mu m$), two stages of transverse energy pumping are seen clearly when $t<150$ due to linear amplification as $\gamma_x\sim \mathcal{F}(t;\omega_{\beta}, \omega_w)$ and  during $500<t<700$ due to PAB as $\gamma_x\sim e^{\Gamma t}$.
%
With the Gaussian pulse, the linear amplification is negligible since the steepness function $\mathcal{S}(t)$ is very small around the resonant point and the parametric transverse energy pumping becomes significant only after $t>750$ where the pulse has been sufficiently etched.
%
With the triangular pulse of a long wavelength ($\lambda_0=1.2\mu m$), the linear amplification is also weak because the resonant point is shifted to the earlier time that linear amplification can not be excited sufficiently. 
%
This in turn mitigates the degradation of the
 quality of the electron beam.
%
On the other side, highly efficient transverse energy pumping can benefit betatron radiation by increasing betatron oscillation strength. It is maximized from a steepened pulse driver with which the enhancement of the betatron radiation is almost one order in the energy as shown in Fig.~\ref{fig:betatron_radiation}(b). Such a laser pulse may be experimentally prepared by simply passing a Gaussian pulse through a thin (tens of nm) foil before the gas target to tailor the pulse front.

\begin{figure}[h]
	\centering
	\includegraphics[width=0.49\textwidth]{Etransfer_radiation.pdf}
	\caption{(a) Transverse gamma factor of the electron beam, and (b) radiation spectrum for a temporal triangular(black) pulse with front $\tau_f=0.5$ and Gaussian(red) pulse with $\tau_f=4$ laser pulse, respectively. The green line in (a) is for a triangular pulse of wavelength $\lambda_0=1.2\mu m$. Other parameters are the same as $a_0=4$, $w_0=6$, tail duration $\tau_t=4$ and plasma density $n_0=4.6\times 10^{18} \rm{cm^{-3}}$. The dashed vertical line in (a) shows the position where $\omega_{\beta}=3\omega_w/2$ for $\lambda_0=0.8 \mu m$ (black) and for $\lambda_0=1.2\mu m$ (green).}
	\label{fig:betatron_radiation}
\end{figure}
%

In summary, we have investigated betatron oscillation of trapped electron beams 
in long-scale LWFA. 
%%%%
Two different betatron amplification stages have been identified. 
%%%%
When the laser front is etched, the bubble wobbles and the system evolves as a driven oscillator, resulting in linear amplification of the oscilation amplitude of the electron beam. With the further depletion of the laser pulse, bubble breathing dominants, leading to parametric amplification of the betatron oscillation (PAB). 
%%%
Since betatron oscillation plays a fundamental role in the transverse dynamics of the trapped electrons, such an amplification process is pivotal to generating high-quality high-energy electron beams. 
%% and should be suppressed.
It is shown that PAB can be mitigated  by suppressing the 
excitation in the linear-amplification stage, or by using a long-wavelength laser. 
On the other hand, PAB can be ultilized to enhance betatron radiation.
%
By revealing the intrinsic connection between bubble dynamics and betatron amplification, 
we hope this work may inspire effective strategies of how to stabilize the bubble which may lead to the overcome of the beam instability in long-scale LWFAs and thus the success of high-quality high energy laser-driven electron accelerators and potential applications.   

\begin{acknowledgements}
This work is supported by the National Key R\&D Program of China (Grant No. 2022YFA1603200, 2022YFA1603201); the National Natural Science Foundation of China (Grant Nos. 12135001, 11825502); the Strategic Priority Research Program of the Chinese Academy of Sciences (Grant No. XDA25050900); the Science and Technology on Plasma Physics Laboratory (Grant No. 6142A04210110). 
%
B.Q. acknowledges support from the National Natural Science Funds for Distinguished Young Scholars (Grant No.11825502). 
%
Bin Liu acknowledges the support of Guangdong High Level Innovation Research Institute Project, Grant No. 2021B0909050006.
%
The authors gratefully acknowledge that the computing time is provided by the Tianhe-2 supercomputer at the National Supercomputer Center in Guangzhou and the Gauss Centre for Supercomputing e.V. (www.gauss-centre.eu) through the John von Neumann Institute for Computing (NIC) on the GCS Supercomputer JUWELS at Jülich Supercomputing Centre (JSC).
\end{acknowledgements}


\bibliography{main.bib}

\end{document}
%
% ****** End of file main.tex ******
