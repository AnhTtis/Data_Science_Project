\section{Conclusion}\label{S:SectionV}

In this study a dataset consisting of 322 observations from 161 individual industrial drying experiments of bulky filter media products were performed and presented. 
%
 A total of 21 competing MC estimation models have been developed, trained, fitted, tested and compared using this data. 
The models were tested and compared using a five times repeated 10 fold cross validation scheme. 
%
A three layer MLP ANN was found to be the the most successful algorithm for estimating MC in bulky filter media products. The results show that including ICD to the training set of the ANN, hampers the MC estimation performance in the region of interest. This is in contrast to other investigated machine learning approaches, such as ANFIS, PLS, and SVR where either the performance increases or is unchanged when including the ICD in the training set. 
%
For manufacturing purposes, the most interesting region is below 10\% normalised MC as this is where one might consider stopping the drying process. Average model estimation error decreases for all models in this region, however the ANN NIC model performs the best. %
Overall, the results show that the developed ANN MC estimation approach is suitable for industrial usage. 

In general, we conclude that while thin layer drying models have been reported to be well performing for MC estimation in the fields of drying foodstuff and agricultural products, they are unable to encapsulate the underlying variance of the data of the bulky filter media. Present findings furthermore show, that ANNs combined with measurement of drying settings (oven temperature, differential pressure (fan speed) etc.) and only 2 status features, the drying time and product temperature, can be successfully used as a non-destructive MC estimation technique of bulky filter media products. Furthermore, this study establishes a baseline MC estimation performance and presents a dataset that can be used for model development and testing. As such, this study constitutes a significant contribution to both academic researchers and industrial drying designers. Further research into improving the quality of MC estimation by looking at temporal data such as change of input data is recommended. Furthermore, it could be of great interest to be able to predict the evolution of the MC, i.e. the drying curve of a specific filter media in order to be able to estimate a remaining required drying time to use for scheduling and possibly optimization of the driving oven parameters, such as oven temperature and air flow.
