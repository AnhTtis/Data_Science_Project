\section{Experimental setup and data collection}\label{S:SectionIII}

All drying experiments were conducted in a test oven concurrently drying four different filter media, replicating industrial usage. The positions in the oven are weakly coupled and each drying process can be approximated as an independent process. 

\subsection{Drying Procedure and Moisture Content}
The experimental MC was measured using the gravimetric method. 
The drying is split into two phases, namely Drying Phase 1 (DP1) and Drying Phase 2 (DP2). 
DP1 replicates the real-world industrial drying. However, in order to map out the entirety of the drying curve a variation in drying time is induced by extracting the filter media after a predetermined amount of time. 
DP2 lasts 48 hours with an oven temperature of $120 ^{\circ} C$. The purpose of DP2 is to evaporate all MC from the filter media thus enabling the measurement of the solid mass $m_{solid}$ which is used to calculate the experimental MC in the filter media, as seen in (\ref{eq_moisture-content_initial}) and (\ref{eq_moisture-content}).

The mass of the filter media are measured three times during the experiment. The initial (wet) mass, $m_{initial}$, of each filter media is measured before DP1.
$m_{after}$ is measured after DP1, and $m_{solid}$ which is measured after DP2. 

With these measured masses we can now calculate the initial MC as:
\begin{equation} \label{eq_moisture-content_initial}
	MC_{initial} = \frac{m_{initial}-m_{solid}}{m_{solid}}  100 \%,
\end{equation}
and the MC after DP1 as:
\begin{equation} \label{eq_moisture-content}
	MC = \frac{m_{after}-m_{solid}}{m_{solid}}  100 \%.
\end{equation}


\subsection{Dataset}


\begin{figure*}
\centering
\subfloat[]{\includegraphics[height=2.5in]{figures/dataset-mean_blower_blower_dp-histogram_normalised.pdf}\label{fig-dataset-distribution-mean_oven_dp}} %
\subfloat[]{\includegraphics[height=2.5in]{figures/dataset-mean_oven_temperature-histogram_normalised.pdf}\label{fig-dataset-distribution-mean_oven_temperature}} %
\subfloat[]{\includegraphics[height=2.5in]{figures/dataset-initial_mass-histogram_normalised.pdf}\label{fig-dataset-distribution-initialmass}} %
\\
\subfloat[]{\includegraphics[height=2.5in]{figures/dataset-Temperature-vs-MC_normalised.pdf}	\label{fig-dataset-temperature-vs-MC}}
\subfloat[]{\includegraphics[height=2.5in]{figures/dataset-drying_time-vs-MC_normalised.pdf}	\label{fig-dataset-dryingtime-vs-MC}}
\caption{ (a) Distribution of normalized dimensionless mean differential pressure of each filter media during the drying process. (b) Distribution of the normalized dimensionless mean oven temperature of each drying experiment. (c) Distribution of the normalized dimensionless initial mass of filter media. (d). Normalized dimensionless estimated filter media temperature at extraction time as a function of MC. A clear relationship between estimated filter media temperature and MC can be seen. Cluster in upper left corner corresponds to the ICD. (e) MC as a function of normalized dimensionless drying time. Large variance in both drying time and MC can be seen. Cluster in upper left corner corresponds to the ICD.}
\end{figure*}


Automated data collection is used to collect the seven predictor variables that constitute the dataset. The drying time $t_{drying}$, the estimated filter media temperature $\widehat{T}_{filter}$, the oven chamber position $OCP$, the overall mean of the oven input temperature during the drying process of the particular filter media $\widebar{T}_{in}$, the overall mean of the differential pressure across the oven $\widebar{\Delta p}$, the oven temperature at the time of filter extraction $T_{cur}$, and the initial mass of the filter media before drying $m_i$ are collected for each experiment.

Each measurement of the dataset was normalized and scaled such that all values lie in the range of $[0,100]$. The normalized values of a sample $\mathbf{x}$ were calculated using:
\begin{equation}
	\mathbf{z}_k =  \frac{\mathbf{x}_k - min(\mathbf{x}_k^{train})}{max(\mathbf{x}_k^{train})-min(\mathbf{x}_k^{train})} \cdot 100,
\end{equation}
where $\mathbf{z}_k$ is the vector of normalized values of feature $k$, $\mathbf{x}_k$ is the vector of all values of feature $k$, $\mathbf{x}_k^{train}$ is the vector of values of feature $k$ belonging to the training set.

A total of 161 experiments were performed resulting in 322 sets of predictor- and response variable vectors. 161 sets of observations measuring the \textit{initial condition data} (ICD) and 161 sets of predictor- and response variable observations with different drying times. The dataset consists of two classes of datapoints, ICD and \textit{end condition data} (ECD), where the ICD are the sets of observation sampled upon insertion of a filter media into the drying oven, i.e. a drying time of zero minutes. The ICD are information poor, as an equilibrium has not been reached yet and, as an effect, the sensors are sensing the features of the oven and not those of the filter media. The ECD are the sets of observations upon extraction of the filter media from the drying oven, i.e. after the designated drying time for the specific filter media. The ECD are relatively information rich, and regression or estimation can be utilized. The dataset is published in the IEEE DataPort repository and can be found here: https://dx.doi.org/10.21227/hwa2-tp66 \cite{hwa2-tp66-22}.


The features of the dataset can furthermore be classified into two feature types, i.e., status features and oven setting features.



\subsubsection{Oven setting features}
The oven setting features are the features describing the physical environment in which the filter is dried. The oven setting features are the position of the filter media in the oven, the mean oven temperature during the drying time of each specific filter media, the mean oven differential pressure during the specific drying time of the filter media, the current oven temperature, and the initial mass of the filter media before drying begins.

Fig. \ref{fig-dataset-distribution-mean_oven_dp} shows the distribution of the differential pressure over the fan pushing the air into the oven. The differential pressure is correlated with air speed, and thus the mass of air circulating in the oven. As can be seen, one set of 20 drying experiments has been done under other circumstances than the rest of the filter media, and a trained model will need to be able to encompass this deviation in oven setting parameters as well. The outlier data has been included as it will serve to challenge the performance of the produced models.

Fig. \ref{fig-dataset-distribution-mean_oven_temperature} shows the distribution of the mean oven temperature during the drying experiments of each filter media. Here, a binormal distribution can be seen. This is due to the unfortunate deconstruction and reconstruction of the test oven during the multi-month data acquisition period. If the estimation models are able to encompass these different oven setting features, then it only bodes well for the generalizability of the model.

Fig. \ref{fig-dataset-distribution-initialmass} shows the distribution of the initial mass which as can be seen follows a skewed Gaussian distribution.


\subsubsection{Status features}
The status features are the features correlated with the current drying status of the filter i.e., the drying time and the estimated filter media temperature.

Fig. \ref{fig-dataset-temperature-vs-MC} shows the MC as a function of the dimensionless normalised estimated filter media temperature. A clearly dependent relationship between the MC and the estimated filter media temperature can be identified, the lower the temperature the larger the variation in MC as is expected from the behaviour of a typical drying curve. The estimated filter media temperature holds much of the information that the proposed models will be able to utilize in order to make good estimates.

Fig. \ref{fig-dataset-dryingtime-vs-MC} shows a relationship between drying time and MC. There is a large variance along the MC axis, especially for lower drying times. This variance is where the possible gains of utilizing MC estimation can be seen. All low-drying-time or low-MC datapoints represents the possible optimization gains, as early stopping of the drying process can be done if identification of the MC is possible.  


\subsection{Competing Estimation Models}
The proposed ANN-based approach is compared with data-driven models reported as state of the art for different MC estimation applications in the literature. To establish a baseline performance we use semi-empirical thin layer drying models, see Table \ref{tbl:methodology:thin_layer_models}. The thin layer drying models are all fitted using nonlinear least squares in the Matlab curve fitting toolbox \cite{matlabcurvefitting}. 

\begin{table}[]
    \caption{Thin-layer drying models}
    \label{tbl:methodology:thin_layer_models}
    \begin{tabular}{lll}
    Model            & Equation                    & Reference \\ \hline
    Lewis            & $MC =\exp(-kt)$                           &  \cite{lewis1921rate}         \\ 
    Page             & $MC = \exp(-kt^n)$                        &   \cite{page1949factors}        \\ 
    Two term        & $MC = a\exp(-k_1t)+b\cdot\exp(-k_2t)$      &  \cite{madamba1996thin}         \\ 
    Henderson       & $MC = a\exp(-kt)$                         &   \cite{hendersonPabis}        \\ 
    Logarithmic     & $MC = a\exp(-kt)+c$                       &  \cite{yaugciouglu1999drying}         \\ 
    Midilli et al.  & $MC = a\exp(-kt^n)+bt$                    &  \cite{midilli2002new}    \\ \hline
    \end{tabular}
\end{table}


Furthermore, we compare the ANN approach to SVR and RFR as reported by \cite{SaglamC_apple_slices}, and ANFIS as reported by \cite{Amini2021}, and partial least squares (PLS) to act as a baseline for the machine learning models. 
%
All competing models estimate the MC as output. The input for the thin layer drying models is solely drying time. The input for the machine learning models are all the same as that for the ANN. 

All models come in two variations. One trained on the entirety of the data, referred to as With Initial Conditions (WIC), and one trained only on the ECD, referred to as No Initial Conditions (NIC). As postulated earlier, the ICD is relatively information poor, and thus might hamper the estimation performance in the range of interest, the ECD. For practical applications, the quality of the estimates on the ICD can be ignored - as it is a trivial case.


\subsection{Model Performance Validation}
All models are validated using repeated 10 fold cross validation as described by \cite{Burman1989}. Regular 10 fold cross validation was performed by splitting the data into 10 folds, training on all but one fold, and then using the left-out fold for validation. This process was then repeated across all 10 folds, resulting in averages of the estimation error measures as described in (\ref{eq:MSE}), (\ref{eq:MAE}), (\ref{eq:STD}), and (\ref{eq:R2}). 
The data was then shuffled, and the above process was repeated five times. Therefore, all results reported in this section are based on validation data and not training data. Furthermore, all results reported are averages of the five times repeated 10 fold cross validation trials. 

