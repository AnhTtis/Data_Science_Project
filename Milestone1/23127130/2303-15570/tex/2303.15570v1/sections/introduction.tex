\section{Introduction}

Drying is a widely used manufacturing process across many different fields of manufacturing. 
For the manufacturing of filter media, the drying process is the most energy intensive and time consuming process.

The drying of filter media products is a process highly dependent on both the upstream manufacturing steps and the state of the ambient air, due to the hygroscopic nature of the filter media. 
These dependencies introduce variance in the MC of the filter media before the drying process and thus also introduce variance in the required drying times to reach a desired MC threshold for each filter media product. 
Therefore, control of the drying process of filter media products is based on a time threshold ensuring that all filters reach a desired MC threshold. This process comes at the cost of energy and time which is spent over-drying filter media products.

However, with the knowledge of MC of filter media products, it is possible to reduce both drying time and energy expenditure. This is because it is possible to stop the individual drying processes based on the condition of the filter media, instead of a predetermined time threshold. 
Direct measurement of the filter media MC during the drying process is infeasible, introducing the need of alternative methods.

One approach is to model the drying of wet materials. This is a complex, highly nonlinear, dynamic and multivariable thermal process whose underlying mechanisms are not yet fully understood \cite{chasiotisArtificial2020}. It is a highly coupled multivariate problem considering the coupled heat, momentum, and mass transfer which, when modelled, can lead to insights into the underlying process and quality parameters. 

When only estimates of the MC are desired, a variety of soft-computing methods have proven effective for MC estimation during drying of different materials. 
%
A comparison of MC estimation performance using k-nearest neighbour regression, Support Vector Regression (SVR), Random Forest Regresssion (RFR), Artificial Neural Networks (ANN), and Gaussian processes, identified RFR as the most successful algorithm in estimation of drying characteristics and RFR and SVR as the most successful algorithms for MC estimation \cite{SaglamC_apple_slices}. 

MC estimation performance of basil seed mucilage using genetic algorithm-based ANNs and adaptive neuro-fuzzy inference systems (ANFIS) was studied by \cite{Amini2021}. Their results indicate that, while the ANFIS model gave the best total fit of MC, both the ANNs and ANFIS can give good estimations of MC during infra-red drying. 

Drying of marrow slices was investigated in \cite{Ceclu2022}. It was found that amongst the thin layer drying models, the Logarithmic, the Henderson, and Pabis models were the best. However, it was also found that a multi-layer perception (MLP) network using Backpropagation-based training was able to estimate the MC with an insignificant error. 
ANNs were also found successful in MC estimation of edible rose \cite{Qiu2022}, quince slices \cite{chasiotisArtificial2020}, green tea leaves \cite{kalathingal2020artificial}, absinthium leaves \cite{karimi2012optimization}, pistachios \cite{tavakolipour2012neural}, water melon rind pomace \cite{fabani2021producing}, and discarded yellow onions \cite{roman2020convective}. 

A mini-review by \cite{Yang2022} shows that ANNs, in general, are well suited for MC estimation applications utilising microwave drying, and \cite{aghbashlo_application_2015} shows that ANNs are well suited in general for foodstuff drying applications. 
%
ANNs have also shown to be a good tool for other estimation applications, such as estimating the State-of-Charge of batteries for electric vehicles \cite{lipu2019extreme,how2020state}, remaining useful lifetime of batteries \cite{9137406}, breaking pressure \cite{8119882}, solenoid valve remaining useful life-time \cite{9426406} and nitrogen in wheat leaves \cite{7742940}.

The above literature review shows that there exists a plethora of effective MC estimation techniques, which have been widely applied and researched in the field of foodstuff, especially for thin products (thickness magnitude approximately $10^{-4}$ meter to $10^{-2} $ meter). The filter media investigated in this work have a thickness magnitude of approximately $10^{-1}$ meter. It is therefore not a given that the results extend to this category of products. Furthermore, we have been unable to identify any studies of online MC estimation of filter media or similar products.

The objectives of this work are as follows:
\begin{enumerate}
    \item To present and share a dataset of industrial production drying data of bulky filter media drying.
    \item To device a method that can estimate the MC of filter media during the drying process to a degree that is useful for manufacturing.
    \item To compare, quantify and evaluate said method with the state-of-the-art MC estimation models found in the literature.
\end{enumerate}


The rest of this article is organised as follows: Section \ref{S:SectionII} describes the model selection process for determining the proposed ANN architecture for MC estimation. Section \ref{S:SectionIII} describes the experimental setup and data collection, including the dataset and software, and competing estimation models, used in the study. Section \ref{S:SectionIV} contains the results and discussions. Finally section \ref{S:SectionV} concludes the the article.