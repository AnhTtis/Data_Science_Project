\section{Experiment Evaluation}
This section presents the tools used to evaluate the influence of the communication interfaces over the confidence level of the participants. The tests include an explicit and an implicit communication interface, the external HMI and the braking maneuver profile respectively. For this analysis, after each test in Table \ref{tab:testsetup}, they were asked to fill out a questionnaire about their subjective perception of the interaction. In addition, direct measurement variables were registered from the scenario to evaluate changes in their observable behavior.

\begin{figure}
\centerline{\includegraphics[width=\columnwidth]{figures/gentle_braking_maneuver.png}}
\caption{Crossing event example in the gentle braking maneuver.}
\label{fig:varibles_at_crossing_gentle}
\end{figure}

\subsection{Questionnaire}
Throughout the experiment, the participants had short rest periods between tests in which they did not leave the virtual reality when a researcher asked them the following questions about their last interaction with the vehicle:

\begin{itemize}
\item[--] Q1: \emph{What was your level of confidence that the vehicle would stop and yield to you?}
\item[--] Q2: \emph{How did you perceive the braking of the vehicle?}
\item[--] Q3: \emph{Has the visual communication interface improved your confidence to cross?}
\end{itemize}

Answers to these questions are tabulated on a 7-step Likert scale \cite{joshi2015likert} and allow to study the influence of communication interfaces from the subjective point of view of the pedestrian. 

\subsection{Direct Measurements}
In addition to having specific control over traffic conditions, CARLA simulations enable access to all agents and environment variables so we directly obtain the participant's location on the map and their full-body pose. The reconstruction of their trajectory allows us to generate synthetic sequences from multiple points of view based on their real behavior and to extract some valuable parameters such as the crossing decision event, the crossing event or their eye contact with the vehicle. The experiments are recorded and can be replayed to compare results for different sensors or configurations.
The direct measurements used to quantitatively analyze the interaction during the experimentation are the distance to the pedestrian, the vehicle speed and the time-to-collision (TTC) computed as  $TTC=d/v$. Figure \ref{fig:varibles_at_crossing_gentle} shows the evolution of these variables in the gentle braking maneuver, as well as a crossing event example. 


\subsection{Crossing Event}
The crossing event is defined as the pedestrian entering the vehicle lane an exposing themselves to a possible collision, as shown in Figure \ref{fig:vehicle_lane}. It is considered a metric to evaluate the behaviour of the pedestrian: if crosses earlier, s/he feels more confident in the vehicle. The crossing event is used instead of the previous crossing decision event (i.e., the moment when the pedestrian makes the decision to cross the road, which is more difficult to measure) because it can be unequivocally identified  by means of the lane marking. If the subject perceives that the situation is more risky and hesitates to cross, we detect this in a subsequent crossing event. Both braking maneuvers are repeated in every experiment, so the crossing event is needed for directly observable measurements to be meaningful in the study.  

\begin{figure}[t]
  \centering
  \includegraphics[width=0.95\linewidth]{figures/virtual_crossing_event.PNG}
  \caption{The crossing event is defined when the pedestrian enters the vehicle lane and is exposed to a possible collision.}
  \label{fig:vehicle_lane}
\end{figure}





% \section{Some Common Mistakes}\label{SCM}
% \begin{itemize}
% \item The word ``data'' is plural, not singular.
% \item The subscript for the permeability of vacuum $\mu_{0}$, and other common scientific constants, is zero with subscript formatting, not a lowercase letter ``o''.
% \item In American English, commas, semicolons, periods, question and exclamation marks are located within quotation marks only when a complete thought or name is cited, such as a title or full quotation. When quotation marks are used, instead of a bold or italic typeface, to highlight a word or phrase, punctuation should appear outside of the quotation marks. A parenthetical phrase or statement at the end of a sentence is punctuated outside of the closing parenthesis (like this). (A parenthetical sentence is punctuated within the parentheses.)
% \item A graph within a graph is an ``inset'', not an ``insert''. The word alternatively is preferred to the word ``alternately'' (unless you really mean something that alternates).
% \item Do not use the word ``essentially'' to mean ``approximately'' or ``effectively''.
% \item In your paper title, if the words ``that uses'' can accurately replace the word ``using'', capitalize the ``u''; if not, keep using lower-cased.
% \item Be aware of the different meanings of the homophones ``affect'' and ``effect'', ``complement'' and ``compliment'', ``discreet'' and ``discrete'', ``principal'' and ``principle''.
% \item Do not confuse ``imply'' and ``infer''.
% \item The prefix ``non'' is not a word; it should be joined to the word it modifies, usually without a hyphen.
% \item There is no period after the ``et'' in the Latin abbreviation ``et al.''.
% \item The abbreviation ``i.e.'' means ``that is'', and the abbreviation ``e.g.'' means ``for example''.
% \end{itemize}
% An excellent style manual for science writers is \cite{b7}.


% \subsection{Figures and Tables}
% \paragraph{Positioning Figures and Tables} Place figures and tables at the top and 
% bottom of columns. Avoid placing them in the middle of columns. Large 
% figures and tables may span across both columns. Figure captions should be 
% below the figures; table heads should appear above the tables. Insert 
% figures and tables after they are cited in the text. Use the abbreviation 
% ``Fig.~\ref{fig}'', even at the beginning of a sentence.


% \begin{figure}[htbp]
% \centerline{\includegraphics[width=\columnwidth]{figures/visual_angle (2).pdf}}
% \caption{Example of a figure caption.}
% \label{fig}
% \end{figure}

% Figure Labels: Use 8 point Times New Roman for Figure labels. Use words 
% rather than symbols or abbreviations when writing Figure axis labels to 
% avoid confusing the reader. As an example, write the quantity 
% ``Magnetization'', or ``Magnetization, M'', not just ``M''. If including 
% units in the label, present them within parentheses. Do not label axes only 
% with units. In the example, write ``Magnetization (A/m)'' or ``Magnetization 
% \{A[m(1)]\}'', not just ``A/m''. Do not label axes with a ratio of 
% quantities and units. For example, write ``Temperature (K)'', not 
% ``Temperature/K''.