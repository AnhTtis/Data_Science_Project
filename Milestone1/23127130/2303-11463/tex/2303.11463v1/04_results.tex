\section{Results}
This section presents the results obtained both in the questionnaires and in the labeling of the direct measurements. We search for significant differences between the tests from Table \ref{tab:testsetup} to determinate the utility of the communication interfaces involved. In Figure \ref{fig:example_interacion_exterior} we can observe an interaction example with the external HMI activated. The Virtual Reality headset projects the crosswalk onto the participant and allows displacement through the scenario. To compare the responses to the questionnaires and the direct measurements we use the Wilcoxon signed-rank test and the Student t-test respectively. 


\begin{figure*}
\centering
\subfloat[]{\includegraphics[width=0.195\textwidth]{figures/all_0.png}%
\label{fig_ped_1}}
\hfil
\subfloat[]{\includegraphics[width=0.195\textwidth]{figures/all_2.png}%
\label{fig_ped_2}}
\hfil
\subfloat[]{\includegraphics[width=0.195\textwidth]{figures/all_3.png}%
\label{fig_ped_3}}
\hfil
\subfloat[]{\includegraphics[width=0.195\textwidth]{figures/all_4.png}%
\label{fig_ped_4}}
\hfil
\subfloat[]{\includegraphics[width=0.195\textwidth]{figures/all_6.png}%
\label{fig_ped_5}}
\caption{Interaction example between pedestrian and virtual vehicle equipped with external HMI: (a) The pedestrian starts with their back to the crosswalk and is told to turn around when the vehicle approaches. (b) The pedestrian makes eye contact with the vehicle and hesitate to cross. (c) The external HMI switches from red to green. (d) The pedestrian enters the vehicle lane establishing the crossing event. (e) The pedestrian crosses the road.}
\label{fig:example_interacion_exterior}
\end{figure*}



\begin{figure}
\centerline{\includegraphics[width=\columnwidth]{figures/VR_boxplot.png}}
\caption{Box-plots of the distances to the pedestrian at the crossing event.}
\label{fig:boxplot_distance}
\end{figure}

\subsection{Questionnaire Results}
The Wilcoxon Signed Rank test \cite{woolson2007wilcoxon} is a non-parametric statistical hypothesis test used to determinate whether the difference between two related samples taken from the same population is statistically significant. The alternative hypothesis matrix showed in Table \ref{tab:wilcoxon} expresses categorical statements which compare the answers to the questions from Q1 to Q3 obtained in each test from the experiment. A check-mark in a specific cell means the null hypothesis $H_0:\mu_i\leq\mu_j$ is rejected and  the alternative hypothesis $H_1:\mu_i>\mu_j$ is accepted when comparing the answers provided in test $i$ (left column) and in test $j$ (top row). Rejecting $H_0$ and accepting $H_1$ implies there is a significant difference in the answers and that the score in test $i$ is higher than in test $j$.


\begin{table}[htbp]
\renewcommand{\arraystretch}{1.1}
\caption{Wilcoxon Signed Rank test, Q1-Q3, $\alpha$=0.05}
\begin{center}
\begin{tabular}{ccc|p{1cm}p{1cm}p{1cm}p{1cm}}
%\hline
\multicolumn{3}{c|}{\textbf{$H_1:\mu_i>\mu_j$}}& \multicolumn{4}{c}{\textbf{Test number $j$}} \\
%\cline{3-7} 
 \multicolumn{3}{c|}{} & 1 & 2 & 3 & 4\\
\hline
% & & %\multicolumn{4}{|c|}{\textbf{Question Q1}} \\
%\cline{3-7}
\multirow{12}{*}{\rotatebox[origin=c]{90}{\textbf{Test number $i$}}} & \multirow{4}{*}{\rotatebox[origin=c]{90}{\textbf{Q1}}}     & 1   & --   &    &      &\\
                                            &&2   &    & --    &     & \\
                                            &&3   & \checkmark    & \checkmark     & --    & \checkmark\\
                                            &&4   & \checkmark    & \checkmark     &     & --\\

\cline{3-7}
&\multirow{4}{*}{\rotatebox[origin=c]{90}{\textbf{Q2}}} & 1   & --   &     &     &\\
                                                        &&2   & \checkmark    & --    & \checkmark     & \checkmark\\
                                                        &&3   &    &     & --    &\\
                                                        &&4   & \checkmark    &     & \checkmark     & --    \\

\cline{3-7}
&\multirow{4}{*}{\rotatebox[origin=c]{90}{\textbf{Q3}}} & 1   & --   &     &     &\\
                                                        &&2   &    & --    &     &     \\
                                                        &&3   & \checkmark    & \checkmark     & --    &     \\
                                                        &&4   & \checkmark    & \checkmark     &     & --    \\


\end{tabular}
\label{tab:wilcoxon}
\end{center}
\end{table}

Based on the results showed on Table \ref{tab:wilcoxon}, we cannot state that the gentle braking maneuver with the external HMI non-activated contributes to increase the pedestrian's confidence in the vehicle (Q1: test1 vs test2), but we can do state that the gentle braking maneuver with the external HMI activated does contribute to increase  the pedestrian's confidence in the vehicle (Q1: test3 vs test4). The external HMI does contribute to increase the pedestrian's confidence in the vehicle (Q1: test3 vs test1 and test4 vs test2) and pedestrians perceived the aggressive braking maneuver as “more aggressive” or “less conservative” than the gentle braking maneuvers (Q2: test2 vs test1 and test4 vs test3).


\subsection{Direct Measurements Results}
In the direct measurements analysis we use the Student's t-test \cite{student} that determines if there is a significant difference between the means of two samples groups. The alternative hypothesis matrix is represented in Table \ref{tab:student}. A check-mark in a specific cell means the null hypothesis $H_0:\mu_i\leq\mu_j$ is rejected and  the alternative hypothesis $H_1:\mu_i>\mu_j$ is accepted when comparing the direct measurements labeled on test $i$ (left column) and test $j$ (top row). Rejecting $H_0$ and accepting $H_1$ implies distance, speed and/or TTC at the crossing event in test $i$ are significantly higher than in test $j$. Figure \ref{fig:boxplot_distance} shows the box-plots of the distance between the pedestrian and the vehicle in the labeled crossing event in each trial.


\begin{table}[htbp]
\renewcommand{\arraystretch}{1.1}
\caption{Student t-test, $\alpha$=0.05}
\begin{center}
\begin{tabular}{ccc|p{1cm}p{1cm}p{1cm}p{1cm}}
%\hline
\multicolumn{3}{c|}{\textbf{$H_1:\mu_i>\mu_j$}}& \multicolumn{4}{c}{\textbf{Test number $j$}} \\
%\cline{3-7} 
 \multicolumn{3}{c|}{} & 1 & 2 & 3 & 4\\
\hline
% & & %\multicolumn{4}{|c|}{\textbf{Question Q1}} \\
%\cline{3-7}
\multirow{12}{*}{\rotatebox[origin=c]{90}{\textbf{Test number $i$}}} & \multirow{4}{*}{\rotatebox[origin=c]{90}{\textbf{Distance}}}   & 1   & --   & \checkmark    &      &\\
                                                &&2   &    & --    &     &\\
                                                &&3   & \checkmark    & \checkmark     & --    & \checkmark\\
                                                &&4   &    & \checkmark     &     & --\\
\cline{3-7}
&\multirow{4}{*}{\rotatebox[origin=c]{90}{\textbf{Speed}}}  & 1   & --   & \checkmark     &     &\\
                                                            &&2   &    & --    &     &     \\
                                                            &&3   & \checkmark    & \checkmark     & --    & \checkmark     \\
                                                            &&4   &    & \checkmark     &     & --    \\
\cline{3-7}
&\multirow{4}{*}{\rotatebox[origin=c]{90}{\textbf{TTC}}}    & 1   & --   &     &     &\\
                                                            &&2   & \checkmark    & --    & \checkmark     & \checkmark     \\
                                                            &&3   &    &     & --    &     \\
                                                            &&4   &    &     &     & --    \\
\end{tabular}
\label{tab:student}
\end{center}
\end{table}

Based on the results showed on Table \ref{tab:student} we can state that the gentle braking maneuver does contribute to increase the distance at the crossing event (distance: test1 vs test2 and test3 vs test4) and the external HMI does contribute to increase the distance at the crossing event (distance: test3 vs test1 and test4 vs test2). The alternative hypothesis matrix of the vehicle speed confirms the previous statements: the greater the distance to the pedestrian, the greater the vehicle speed due to its constant deceleration. The aggressive braking maneuver with the external HMI non-activated increases the time-to-collision (TTC: test2 vs test1, test2 vs test3 and test2 vs test4). It is inferred that in test number 2 the pedestrian confidence drops dramatically and many participants waited for the vehicle to come to a complete stop. 



\subsection{Results Discussion}
In the responses to Q1, participants express greater confidence whenever the external HMI is activated. It should be noted that the virtual environment does not distort the appreciation of the braking maneuver, since in the responses to Q2 the aggressive maneuver is always described as “more aggressive” than the gentle maneuver. However, it draws our attention that the non-activation of the external HMI in combination with the aggressive maneuver implies that the same braking maneuver is perceived as even more aggressive (Q2: test2 vs test4). We can make the statement that the activation of the external HMI has much more influence on the risk perception of the participant, above the type of maneuver used in the test.

If we look at the distance to the pedestrian and the vehicle speed in Table \ref{tab:student}, we obtain the same information of the crossing event since the braking maneuver follows a constant deceleration. If participants cross earlier, we can infer they feel more confident, because the vehicle is farther away from coming to a complete stop. Despite the fact that in the questionnaire the participants claimed that they mostly felt safer with the external HMI activated, even with the aggressive maneuver (Q1: test4 vs test1), in practice they also crossed the road earlier when the vehicle followed a gentle braking maneuver. In any case, the activation of the external HMI continues to have a very high influence in making the decision to cross sooner. The non-activation of the external HMI in combination with the aggressive braking maneuver rises sharply the time-to-collision (TTC: test2 vs test1, test2 vs test3 and test2 vs test4). We suggest that this is because the participants perceive the situation as high risk and wait for the vehicle to reduce its speed to almost zero.

