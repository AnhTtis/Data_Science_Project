\begin{abstract}
The traditional simulation methods present some limitations, such as the \emph{reality gap} between simulated experiences and real-world performance. In the field of autonomous driving research, we propose the handling of an immersive virtual reality system for pedestrians to include in simulations real behaviors of agents that interact with the simulated environment in real time, to improve the quality of the virtual-world data and reduce the gap. 

In this paper we employ a digital twin to replicate a study on communication interfaces between autonomous vehicles and pedestrians, generating an equivalent virtual scenario to compare the results and establish qualitative and quantitative measurements of the discrepancy. The goal is to evaluate the effectiveness and acceptability of implicit and explicit forms of communication in both scenarios and to verify that the behavior carried out by the pedestrian inside the simulator through a virtual reality interface is directly comparable with their role performed in a real traffic situation.
\end{abstract}

%\begin{IEEEkeywords}
%automated driving, autonomous vehicles, reality gap, virtual reality, interactions, human machine interface (HMI)
%\end{IEEEkeywords}