\begin{abstract}
This paper presents the results of tests of interactions between real humans and simulated vehicles in a virtual scenario. Human activity is inserted into the virtual world via a virtual reality interface for pedestrians. The autonomous vehicle is equipped with a virtual Human-Machine interface (HMI) and drives through the digital twin of a real crosswalk. The HMI was combined with gentle and aggressive braking maneuvers when the pedestrian intended to cross. The results of the interactions were obtained through questionnaires and measurable variables such as the distance to the vehicle when the pedestrian initiated the crossing action. The questionnaires show that pedestrians feel safer whenever HMI is activated and that varying the braking maneuver does not influence their perception of danger as much, while the measurable variables show that both HMI activation and the gentle braking maneuver cause the pedestrian to cross earlier.
\end{abstract}

%\begin{IEEEkeywords}
%automated driving, autonomous vehicles, reality gap, virtual reality, interactions, human machine interface (HMI)
%\end{IEEEkeywords}