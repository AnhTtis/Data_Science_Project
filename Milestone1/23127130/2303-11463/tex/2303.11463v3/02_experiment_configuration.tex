\section{Experiment Description}
The designed experiment aims to produce an interaction  between a simulated autonomous vehicle and a real pedestrian within a virtual environment. Since the traffic scenario is generated by the CARLA autonomous driving simulator \cite{carla2017}, we are not only able to conduct the experiment under controlled conditions, but also to evaluate the result by analysing the participants performance through direct and indirect measurements and questions. This section describes the scene recreated in the simulator, the immersive and maneuverable virtual reality interface between the real subject and the virtual world, and the different experiment settings.

\subsection{Experiment Configuration}
As the objective of this experiment is to carry out a study of communication interfaces for autonomous vehicles under simulated conditions but whose result can be compared with the one obtained in the real setting, we have developed a digital twin of a real crosswalk from a georeferenced map to establish the same dimensions of the road, including the same arrangement of the rest of elements (i.e., traffic signs, other parked vehicles, vegetation) to faithfully reproduce the visibility conditions.

As shown in Figure \ref{fig:schematic1}, the vehicle circulates autonomously on the road when it reaches the crosswalk just as the pedestrian intends to cross perpendicular to the opposite sidewalk. The pedestrian can detect the vehicle a few meters ahead of the crosswalk before starting the crossing action, and lighting and weather conditions are favorable. The participant is instructed to wait with their back to the roadway until the vehicle is 40 meters from the crosswalk and is then instructed to turn around and move towards the roadway. The vehicle speed is 30 km/h and it starts a braking maneuver using a constant deceleration until it comes to a complete stop at the edge of the crosswalk to yield to the pedestrian.

\begin{figure}
\centering
\subfloat[]{\includegraphics[width=0.49\columnwidth]{figures/green-HMI.PNG}%
\label{fig:grail_green}}
\hfil
\subfloat[]{\includegraphics[width=0.49\columnwidth]{figures/red-HMI.PNG}%
\label{fig:grail_red}}
\caption{External HMI activated communicating a green status (a) and a red status (b).}
\label{fig:external_HMI}
\end{figure}

\subsection{Test Variations}
The virtual autonomous vehicle is equipped with an external communication interface, a so-called GRAIL (Green Assistant Interfacing Light) \cite{GRAIL}, which is represented in the simulator by a bar along the
entire front of the car that changes color to communicate its intentions and current status to other agents on the road. As Figures \ref{fig:grail_green} and \ref{fig:grail_red} show, the interface emits a red color to warn the pedestrian that the vehicle has not detected any obstacles in its path and that it does not plan to execute any braking maneuver, while the green color anticipates a stop to avoid a collision. It is also possible that the interface is turned off so the pedestrian does not have any information about the vehicle status. 

Furthermore, we add implicit form of communication by varying the braking profile. To study whether the pedestrian perceives to be in a situation of greater risk depending on the braking profile, we define a gentle braking maneuver, in which the vehicle decelerates at -0.9 m/s\textsuperscript{2}, and a second, more aggressive braking maneuver, when the vehicle decelerates at -1.8 m/s\textsuperscript{2}. In both cases, the vehicle reduces its 30 km/h speed to a complete stop, but the aggressive maneuver simulates less vehicle anticipation of the encounter.

\subsection{Test Batch}
We designed five tests to assess the influence of each communication technique over the pedestrian level of confidence and their perceived level of safety during the experiment. Table \ref{tab:testsetup} shows the variations in the braking maneuver as well as activation of the external communication interface. 
Test number 0 purpose is to prime the participants for the environment and potential risk of a collision. All tests were performed in random order except test number 0, which was always performed first for each participant.

\begin{table}[htbp]
\renewcommand{\arraystretch}{1.1}
\caption{Experimentation Tests settings}
\begin{center}
\begin{tabular}{c|c|c|c|c}
%\hline
\textbf{Test}& \textbf{Braking} & \textbf{External} & \textbf{Stop}\\
%\cline{3-4} 
\textbf{Number} & \textbf{Maneuver} & \textbf{HMI} &  \\
\hline
0   & -             & -     & No \\
1   & Gentle        & -     & Yes\\
2   & Aggressive    & -     & Yes\\
3   & Gentle        & GRAIL & Yes\\
4   & Aggressive    & GRAIL & Yes\\
%\hline
\end{tabular}
\label{tab:testsetup}
\end{center}
\end{table}

\subsection{Virtual Reality Setup}
In order to allow the experiment to be conducted within a virtual environment, we harness the full immersive system for pedestrians described in \cite{CarlaCHIRA} which adds some features to the CARLA simulator such as real-time avatar control, positional sound and the body tracking of the subject interacting with the scene through virtual reality. In this way, we take advantage of all the different options that CARLA offers to simulate specific traffic scenarios while there is a real subject playing the role of a pedestrian and being part of the simulation. We use Oculus Quest 2, created by Meta, as a head mounted device (HMD) and Perception Neuron Studio (PNS) motion capture system for full-body tracking \cite{PNS2022}. Quest 2 is connected to PC via WiFi and projects onto their lenses the CARLA spectator view. At the same time, the captured pose and motion of the subject is integrated into the virtual scenario, so the simulated sensors attached to the autonomous vehicle (i.e., radar, LiDAR, cameras) can be aware of their presence. So that the development of the experiment was not hindered, we reserved a preset area (3 x 8 meters) free of obstacles where the participant could act as a real pedestrian inside the simulation.

\subsection{Participants}
18 volunteers from inside and outside the University area, consisted of 12 male and 6 female who ranged in age from 24 to 62, agreed to participate in the experiment. Most of them had never had any virtual reality experience before and they were informed about the risk of dizziness or disorientation. Fortunately, all the participants felt good during the experiment and there were no incidents.



% \subsection{Units}
% \begin{itemize}
% \item Use either SI (MKS) or CGS as primary units. (SI units are encouraged.) English units may be used as secondary units (in parentheses). An exception would be the use of English units as identifiers in trade, such as ``3.5-inch disk drive''.
% \item Avoid combining SI and CGS units, such as current in amperes and magnetic field in oersteds. This often leads to confusion because equations do not balance dimensionally. If you must use mixed units, clearly state the units for each quantity that you use in an equation.
% \item Do not mix complete spellings and abbreviations of units: ``Wb/m\textsuperscript{2}'' or ``webers per square meter'', not ``webers/m\textsuperscript{2}''. Spell out units when they appear in text: ``. . . a few henries'', not ``. . . a few H''.
% \item Use a zero before decimal points: ``0.25'', not ``.25''. Use ``cm\textsuperscript{3}'', not ``cc''.)
% \end{itemize}

% \subsection{Equations}
% Number equations consecutively. To make your 
% equations more compact, you may use the solidus (~/~), the exp function, or 
% appropriate exponents. Italicize Roman symbols for quantities and variables, 
% but not Greek symbols. Use a long dash rather than a hyphen for a minus 
% sign. Punctuate equations with commas or periods when they are part of a 
% sentence, as in:
% \begin{equation}
% a+b=\gamma\label{eq}
% \end{equation}

% Be sure that the 
% symbols in your equation have been defined before or immediately following 
% the equation. Use ``\eqref{eq}'', not ``Eq.~\eqref{eq}'' or ``equation \eqref{eq}'', except at 
% the beginning of a sentence: ``Equation \eqref{eq} is . . .''

% \subsection{\LaTeX-Specific Advice}

% Please use ``soft'' (e.g., \verb|\eqref{Eq}|) cross references instead
% of ``hard'' references (e.g., \verb|(1)|). That will make it possible
% to combine sections, add equations, or change the order of figures or
% citations without having to go through the file line by line.

% Please don't use the \verb|{eqnarray}| equation environment. Use
% \verb|{align}| or \verb|{IEEEeqnarray}| instead. The \verb|{eqnarray}|
% environment leaves unsightly spaces around relation symbols.

% Please note that the \verb|{subequations}| environment in {\LaTeX}
% will increment the main equation counter even when there are no
% equation numbers displayed. If you forget that, you might write an
% article in which the equation numbers skip from (17) to (20), causing
% the copy editors to wonder if you've discovered a new method of
% counting.

% {\BibTeX} does not work by magic. It doesn't get the bibliographic
% data from thin air but from .bib files. If you use {\BibTeX} to produce a
% bibliography you must send the .bib files. 

% {\LaTeX} can't read your mind. If you assign the same label to a
% subsubsection and a table, you might find that Table I has been cross
% referenced as Table IV-B3. 

% {\LaTeX} does not have precognitive abilities. If you put a
% \verb|\label| command before the command that updates the counter it's
% supposed to be using, the label will pick up the last counter to be
% cross referenced instead. In particular, a \verb|\label| command
% should not go before the caption of a figure or a table.

% Do not use \verb|\nonumber| inside the \verb|{array}| environment. It
% will not stop equation numbers inside \verb|{array}| (there won't be
% any anyway) and it might stop a wanted equation number in the
% surrounding equation.