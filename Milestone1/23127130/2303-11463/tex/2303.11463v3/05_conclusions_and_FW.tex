%\newpage
\section{Conclusions and Future Work}
Both the questionnaire and direct measurements support the virtual external HMI increases the pedestrian confidence and also leads to an earlier crossing event. On the other hand, in the questionnaire participants do not express greater confidence in gentle braking maneuver compared to aggressive braking maneuver without the activation of the external HMI. We suggest that participants state to feel safer only by the activation of the external HMI due to its high visibility in the virtual scenario, although the gentle braking maneuver also entails an earlier crossing event. 

As future work, it is intended to compare results on the acceptance of implicit and explicit communication interfaces in the real setting (see Fig. \ref{fig:schematic1}). The final goal is to verify that the difference in pedestrian responses in the virtual and real scenario are not statistically significant, thus validating the method to include realistic behaviors in the simulations through the proposal of a fully immersive VR system for pedestrians in the CARLA simulator \cite{CarlaCHIRA}.