\documentclass[twocolumn, 10pt, floatfix, superscriptaddress, booktabs]{revtex4}

\usepackage{amsmath}
\usepackage{amssymb}
\usepackage{amsfonts}   
\usepackage{graphicx}   
\usepackage{verbatim}   
\usepackage{color}      
\usepackage{subfigure}
\usepackage{pgfplotstable}
\usepackage{wrapfig}
\usepackage{hhline}

\usepackage{tikz}
\usetikzlibrary{positioning}
\definecolor{myRed}{RGB}{229,25,50}
\definecolor{myBlue}{RGB}{25,178,255}
\definecolor{myGreen}{RGB}{50,255,0}

\newcommand{\unit}[1]{\ensuremath{\, \mathrm{#1}}}
\newcommand*{\vcenteredhbox}[1]{\begingroup
\setbox0=\hbox{#1}\parbox{\wd0}{\box0}\endgroup}

\begin{document}

\title{Correlated Avalanche Burst Invasion Percolation: Multifractal origins of self organized critiality}
\date{\today}

\author{Ronaldo Ortez}
\email{raortez@ucdavis.edu}
\affiliation{Department of Physics, One Shields Ave., University of California, Davis, CA 95616, United States}
\author{John B. Rundle}
\email{rundle@ucdavis.edu}
\affiliation{Department of Physics, One Shields Ave., University of California, Davis, CA 95616, United States}
\affiliation{Department of Geology, One Shields Ave., University of California, Davis, CA 95616, United States}
\affiliation{Santa Fe Institute, Santa Fe, NM 87501, United States}

\begin{abstract}
	We extend our previous model, avalanche-burst invasion percolation (AIP)model by introducing long-range correlations between sites described by fractional Brownian statistics. In our previous models with independent, random site strengths, we reproduced a unique set of power-laws consistent with some of the b-values observed during induced seismicity. We expand upon these models to produce a family of critical exponents which would be characterized by the local long-range correlations inherent to host sediment. Further, in previous correlated invasion percolation studies, fractal behavior was found in only a subset of the range of Hurst exponent, $H$. We find fractal behavior persists for the entire range of Hurst exponent. Additionally, we show how multiple cluster scaling power laws results from changing the generalized Hurst parameter controlling long-range site correlations, and gives rise to a truly multifractal system. This emergent multifractal behavior plays a central role in allowing us to extend our model to better account for variations in the observed Gutenber-Richter b-values of induced seismicity. 
\end{abstract}

\maketitle


\section{Introduction}
One of the most interesting insights from random percolation(RP) is the emergence of long range correlations from the inherently random process of independently occupying sites with probability, $p$ on a lattice. As is characteristic with complex phenomena we observe the emergence of characteristic features, namely, large scale connectivity that dominates behavior on all scales. Much of percolation's value comes from providing an extremely simple framework from which many puzzling features both can arise and can be understood.  

Given the emergent nature of long-range connectivity, researchers naturally became curious of the effect of implicit long-range correlations imbedded in the lattice structure might have on the critical behavior\cite{weinrib1984long,prakash1992structural,schrenk2013percolation,sahimi1996scaling,makse1996method}. This question is of interest not only from a formal perspective, but also, because long-range correlations(LRC) are described by fractal relations. These same fractal properties have been one of the most profound discoveries in the last few decades because of the explosion in number and diversity of fractal systems, and highlights the vast pervasiveness of fractal structures in nature. This was made famous by Mandelbrot\cite{mandelbrot1982fractal}, and grew into its own paradigm of inquiry.

In our previous paper \cite{Ortez2022critical}, we characterized the critical behavior of our avalanche-burst invasion percolation(AIP) model, which produced a critical distribution of bursts, $n_s(T)$, as a function of strength threshold, $T$. These bursts existed on top of the random long range correlations emergent to percolation near the critical point. We observed a unique burst distribution characterized by exponents, $\tau=1.594 \pm.009$, $\sigma=0.41\pm.01$. These exponents are near but distinct from mean-field cluster scaling,$\tau_{MF}=1.5, \sigma=0.5$. This serves to define a distinct universality class of critical behavior, distinct even from RP. 

Only a few studies have been done on LRC on IP \cite{knackstedt2000invasion,Vidales_1996}, and these studies only looked at the static network type scaling($D_f,D_{min},D_b$) properties which do little to provide insight into how the critical properties change. This is of course largely because the critical description of IP has been poorly understood, and had not been placed within the appropriate framework to assign it various critical properties. This was done with our AIP model, and now we are well positioned to address the topic of LRC's impact on AIP's critical behavior. 

In particular our AIP model lends itself to the description of induced seismicty as was shown in \cite{ortez2021universality}. Invasion percolation simulates the infiltration of invading fluid into a defending substrate that is modeled as a lattice of sites with random and isotropic resistance. An invasion path following a principle of least resistance enforced at each time step will naturally select the subset of sites where we can observe long range correlations between the invaded sites. The additional burst mechanism allows us to identify the conditions which yield scale invariant bursts, and thus, allows us to speculate on the conditions that must exist to produce the observed scale invariant seismic distributions. 

This study can be more accurate because studies of porous media find correlations between pore size in various sedimentary substrates. These studies indicate porous media “sites” are not independent and random, but rather, exhibit long-range correlations. In particular, fractional-brownian statistics seem to well describe the porosity logs within many heterogeneous rock formations at large scales\cite{knackstedt1998simulation}. Similar findings for the permeability distribution have been found for oil reservoirs and aquifers \cite{leary2002power}. Therefore, we now investigate how our AIP algorithms change in the presence of implicitly correlated lattice sites rather than a random lattice independent random sites. 

\begin{figure*}[t] 
	\centering
	\includegraphics[width=1.0\textwidth]{corrLattice_2x2}
	\caption{\footnotesize Sampling of lattices with increasing correlation. We show how the lattice sites become increasingly correlated as the generalized Hurst exponent increases from -1 to 0. a) $H=-1.0$ corresponds to the random case. b) $H=-0.67$ corresponds to antipersistent correlations c) $H=-0.33$ corresponds to persistent correlations d) $H=0.0$ corresponds to increasingly large correlations where clustering of similar strengths is clearly observable}
	\label{fig:corrLattEx}
\end{figure*}

\section{Long Range Correlations}
\label{section:LRC}
Because long range correlations must persist on all scales, it naturally becomes a point of interest to understand how the critical properties of AIP change under these conditions. Naively, one might fail to properly appreciate the unique impact of long-range correlations on critical behavior since one might well consider any other kind of change to the site lattice structure and consider its affects. However, as Harris \cite{harris1974effect} found in considering the affects of random defects on the critical temperature of the Ising model, the only defects that can have an affect are those whose correlation length, $\xi_H$ is comparable to the correlation length of the unmodified lattice, $\xi$. Thus, since near the critical point $\xi$ is described by a power-law, only those defects whose statistics similarly produce long-range correlations could have any affect on the critical behavior. Any short-range correlations would fail to meet this criteria. This reiterates the focal feature of critical behavior, where small scale interactions can eventually become renormalized, and only those that persist on all scales can contribute to its behavior.        

Harris provided a powerful framework for anticipating the affect that changes in lattice structure could have on subsequent behavior. Weinrib \cite{weinrib1984long} extended Harris' formulation specifically to the percolation problem. We can largely adopt much of the existing framework, where we recognize that AIP's critical behavior is described by a critical control parameter, burst threshold $T$, rather than a critical occupation probability, $p_c$. This means that fluctuations in occupation probabilities correspond to fluctuations in bursts described by $T$.  

We consider correlations that are sufficiently long-ranged while also convergent for all distances, and whose auto-correlation function is given by,
\begin{equation}
C(r) \sim r^{-a}
\label{eq:LRCfuntion}
\end{equation} 
where $r$ is the distance between sites, and $a$ is less than dimension $d$. Since this is the percolation problem, the auto-correlation function describes the correlations in site occupation, that is, the likelihood that two sites a distance $r$ are occupied. The long-range correlations are therefore an additional mechanism contributing to the site occupation probability other than the usual uniform occupation probability, $p$. 

We can calculate how these kind of site strength correlations affect the fluctuations in the control parameter, $\langle \delta T^2 \rangle$,  according to 
\begin{equation}
\begin{split}
\langle \delta T^2 \rangle &\sim \xi^{-d} \int_0^\xi dr \ C(r)r^{d-1} \\
&= \xi^{-d} \int_0^\xi dr \ r^{-a+d-1} \\
&\sim \xi^{-a}
\end{split}
\end{equation}
where again, we can define a correlation length characterizing the average spatial extent of fluctations given by, $\xi^{-a}$.

If the system is still to have a single uniform critical transition, then it should be the case that these fluctuations produce a correlation length less than that of unmodified transition. That is the fluctuations should be less than critical fluctuations leading to the condition,
\begin{equation}
\begin{split}
\frac{\langle \delta T^2 \rangle}{(T_c-T)^2} &\sim (T_c-T)^{a\nu-2} \\
&\rightarrow 0
\end{split}
\end{equation}
where we made use of the relation, $\xi \sim (T_c-T)^{-\nu}$ to expand the ratio. For the ratio to go to zero near the critical point we require the exponent to be greater than zero. This leads to the condition on $a$ for the largest value of long-range correlations such that it will affect the critical transition while preserving a existence of a uniform transition. This is given by,
\begin{equation}
a\nu - 2 > 0
\label{eq:HarrisCond}
\end{equation} 
Thus, we can expect changes to the critical behavior if $a>2/\nu$. For our AIP model $\nu \approx 1.30$, we should expect that the minimum value requires $a>3/2$.


\section{Fourier Filter Correlation Method} 
It is common to parameterize such long range scale invariant correlations using the Hurst exponent, where the (auto)correlation function, $C(r)$ defined as $C(r)=\left\langle u(r^\prime)u(r+r^\prime)\right\rangle$ has the following behavior:
\begin{equation}
C(r) \propto r^{2\alpha}
\label{eq:hurstExp}
\end{equation}
The Hurst exponent is given by $H=2\alpha$ and allowed to take on values in range [0,1]. Behavior of the correlations are antipersistent for $H<1/2$ and persistent for $H>1/2$. For $H=1/2$, the statistics follow fractional Gaussian noise, being neither persistent nor antipersistent.

There are a number of techniques for simulating fractional Brownian statistics \cite{fisher2012science}. We use the Fast Fourier transform(FFT) filter technique because of its computational efficiency. This technique relies on imprinting the desired correlations in the Fourier wave vector space, $\vec{k}$, and then applying an inverse FFT(IFFT) to create a lattice with correlated sites of form Equation \eqref{eq:hurstExp}. Formally, we will be working with 2 dimensional Fourier transforms, and it is well known that the Fourier transform of the autocorrelation function gives the Fourier power spectral density. That is, the correlation function, $\langle u(\vec{x})u(\vec{x}+r)\rangle$ and the power spectral density $S(\vec{k})$ are related according to:
\begin{equation}
\langle u(\vec{x})u(\vec{x}+r) \rangle=\int_{\mathcal{R}^n}S(\vec{k})e^{-i2\pi  \vec{k} \cdot \vec{x}}d\vec{k}
\label{eq:corrFcnFT}
\end{equation}
We can make use that we are only concerned with the distance between two points. This leads to a suitable definition of a radial wave vector defined as $k_r = \sqrt{1+s^2+t^2}$ and with a switch of coordinates allows us to write it as a one dimensional Fourier Transform. 
\begin{equation}
C(r)=\int S(k_r)e^{-i2\pi k_r r}2\pi k_r dk_r
\end{equation}
To create correlations of the form Equation \eqref{eq:hurstExp}, our power spectral density should be made to follow the following power-law:
\begin{equation}
S(k_r) \propto \frac{1}{k_r^\beta}
\label{eq:pwrSpectrum}
\end{equation}
To relate the exponents between Equation\eqref{eq:hurstExp} and Equation\eqref{eq:pwrSpectrum} we can solve Equation\eqref{eq:corrFcnFT} after substituting Equation\eqref{eq:pwrSpectrum} which gives the following integral to be solved:
\begin{equation}
C(r) =2\pi \int k_r^{-\beta+1}e^{-2\pi i k_rr}dk_r
\label{eq:corrIntegral}
\end{equation}

To solve the above integral we first make use of the following relation:
\begin{equation}
\frac{1}{k^\beta} =\frac{2\pi^{\beta/2}}{\Gamma(\beta/2)}\int_{0}^{\infty} \lambda^{\beta -1}e^{-\pi \lambda^2k^2} d\lambda
\label{eq:pwrLawExp}
\end{equation}
The right side is easily Fourier transformed and upon switching the order of integration, we get:

\begin{equation}
\int_{\mathcal{R}}e^{-\pi \lambda^2|k|^2}e^{-2\pi i kr} dk = \lambda^{-1}e^{-\pi {|r|}^2/\lambda^2}
\label{eq:gaussianFT}
\end{equation}
Then taking the 2D Fourier transform of both sides and plugging into Equation\eqref{eq:pwrLawExp} gives:
\begin{equation}
\begin{split}
\int_{\mathcal{R}} k^{-\beta+1}e^{-i2\pi  k_rr} dk_r &= \frac{2\pi^{\beta-1/2}}{\Gamma(\beta-1/2)} \int_{0}^{\infty}d\lambda {\lambda}^{\beta-2}\left[\lambda e^{-\pi{|r|}^2/\lambda^2}\right] \\
&=\frac{2\pi^{\beta-1/2}}{\Gamma(\beta-1/2)} \int_{0}^{\infty}d\lambda {\lambda}^{(\beta-2)-1}e^{-\pi{|r|}^2/\lambda^2} \\
&=\frac{2\pi^{\beta-1/2}}{\Gamma(\beta-1/2)} \frac{\Gamma((\beta-2)/2)} {2\pi^{1/2-\beta+1/2}} \frac{1}{{|r|}^{1-\beta+1}} \\
&\propto r^{\beta - 2} \\
\label{eq:intSolve}
\end{split}
\end{equation}
Setting the exponents equal between the final line of Equation \eqref{eq:intSolve} and Equation \eqref{eq:hurstExp} gives:
\begin{equation}
2\alpha=\beta-2
\label{eq:genHurst}
\end{equation}
This gives our final relationship between the Hurst exponent and the appropriate Fourier power spectrum filter function exponent.
\begin{equation}
\beta = 2(\alpha+1)
\label{eq:betaHurstRelation}
\end{equation}

Because the Hurst parameterization is typically 1-d (given by \eqref{eq:hurstExp}), but we rely on a 2d fourier transform parameterized in terms of $\beta$, whose value is shifted by 1 in 2-d relative to 1-d, we need to shift the value of the exponent of $\alpha$ by 1 as is show in \eqref{eq:betaHurstRelation}. 
Thus, if $\alpha=-1.0$ we get no long range correlations, and if $\alpha=0$ we get brownian long range correlations, which goes like $k^{-2}$. Since we construct the correlated lattice by applying a fourier filter characterized by $\beta=2(\alpha+1)$, we use $\alpha$  in range $[-1,0]$.

More importantly, for these reasons we adopt a modified Hurst parameterization which shifts its values by $-1$. This comes at some risk since in much of the literature $\alpha=H$, and use the standard range, $[0,1]$. We choose our parameterization in order to make explicit the need for a mapping between 1-d and 2-d Hurst characterizations. We suspect that this oversight may in part explain why previous authors report compact clusters for $H>0.5$. Over the shifted range we preserve the fractal structure of our clusters. We find compact clusters begin forming for $H>0.5$ which corresponds to an unshifted value of $3/2$. Such a value certainly would drive clusters to become compact. For the generalized Hurst exponent, we expect antipersistence for $H<-1/2$ and persistence for $H>1/2$. For $H=-1/2$ we expect fractional Gaussian statistics.  

\section{Static Network Properties}
\label{section:SNP}
In a previous study we characterized some of the essential network properties of our model \cite{ortez2021universality}. This study utilized free edge boundary(FEB) conditions along both axes primarily due to ease of implementation. In a subsequent study we implemented periodic edge boundary(PEB) conditions in order to better establish the universality class of the exponents characterizing the model. We found PEB conditions reliably yielded the infinite lattice limit for the scaling exponents. Finally, a growth algorithm with PEB complements the implementation of site correlations as the Fourier filter technique described before also enforces periodic conditions in assigning site strengths to the lattice. Here, we outline some of the static network properties and how these change as a result of the input long range correlations.

\begin{figure} 
	\includegraphics[width=0.5\textwidth]{H_Df}
	\caption{\footnotesize The fractal dimension, $D_f$ for different $H$. For the random case, $D_f = 1.895 \pm 0.016$ which is similar to the expected value of RP. The values all seem to be consistent with one another and doesn't suggest much change as the correlation changes over the range of the study. For $H=0.0$, $D_f= 1.939 \pm 0.028$, which is inconsistent at the $1-\sigma$ from some of the other values.}
	\label{fig:DfHStudy}
\end{figure}

The first characteristic exponent is the scaling of occupied cluster sites, $M(L)$, with lattice size $L$. This scales with characteristic fractal dimension $D_f$ according to:
\begin{equation}
M(L) = L^{D_f}
\label{eq:fractalDim}
\end{equation}
We can easily extract exponent $D_f$ using the well known box counting technique \cite{turcotte1997fractals} and perform linear fit using linear least squares (LLS) on a log-log plot. 

Figure \ref{fig:DfHStudy} shows the extracted $D_f$ for different $H$. For the random case, $H=-1.0$, we reproduce the fractal dimension consistent with RP, $D_f=1.895 \pm 0.016$. We find that input site correlations do not significantly affect the fractal dimension measure in the range of our study. This highlights the macro nature of this measure which is relatively insensitive to changes.

This is somewhat consistent with \cite{schrenk2013percolation} which looked at RP with the same long-range correlations and found no change to $D_f$ except for $H>-0.3$ and where the $D_f \rightarrow 1.95$ as $H \rightarrow 0$. The authors of \cite{sahimi1996scaling} found similar behavior. Other authors report no detectable change in $D_f$ \cite{prakash1992structural} which considered equivalent $H$ correlation in the range $[-1,0]$. That we observe a change in $D_f$ for $H>-0.9$ illustrates a difference between IP and RP growth mechanisms. 

Perhaps more importantly is that we observe clear evidence that the site correlations change the density of the invaded sites, since site density is determined by, $\rho \sim L^{d-D_f}$. As observed in the Ising and percolation critical transition, changes in the order parameter induce changes in the density. In our previous characterization of critical behavior of our model, we found that because $\rho$ did not change, the notion of an order parameter seemed superfluous. However, now that this is no longer the case, we should expect the changing density to influence critical behavior of the model. 

Previous studies on the trapping variant of long-range correlated IP in 2D found cluster behavior becomes non-fractal(compact) for $H>0.5$ \cite{knackstedt2002nonuniversality}, though in this study they considered $0 \geq H \geq 1$. In another study the authors of \cite{Vidales_1996} considered a non trapping variant similar to ours and found a minima as we did in the range of our study. While $D_f$ for RP seems to remain unchanged at least for $H<-0.3$, for IP $D_f$ decreases to a minima before likely increasing towards 2 as H increases above zero.  

\begin{figure} 
	\includegraphics[width=0.5\textwidth]{SIP_DminHstudy}
	\caption{\footnotesize The scaling of distance between sites for different $H$. For the random case $D_{min}\approxeq 1.22$, this tends to decrease as $H$ tends to $0$. The loopless condition will prevent a cluster from becoming compact and $D_{min}$ from becoming 1.}
	\label{fig:DminStudy}
\end{figure}

Though the effect of correlations on $D_f$ is relatively small, we can better understand the effect of correlations on the resulting clusters by looking at the minimum distance between invaded sites. This distance is characterized by scaling exponent $D_{min}$, and it changes more significantly for different $H$. This follows another power law:
\begin{equation}
M\left(l\right) \sim l^{D_{l}}
\label{eq:chemDim}
\end{equation}

Where $M(l)$ is the number of sites within lattice spacing $l$ and $D_l$ is the chemical dimension\cite{havlin1984topological}.
With backbone studies one must be more careful with how boundary conditions are imposed (periodic etc.).  Thus it is preferable to use $D_{l}$ which is largely independent of such affects. Further, what we are really interested in is characterizing the compactness of a cluster which describes the types of paths connecting sites. 
We can relate the Pythagorean distance $r$ and $l$ as:

\begin{equation}
l \sim r^{D_{min}}
\label{eq:pythDist}
\end{equation}

Therefore if $d$ is the path distance from the origin to the boundary of lattice size $L$, then $L=nl$ and by Eq. \ref{eq:pythDist} we can write:
\begin{equation}
d \sim r^{D_{min}}
\label{eq:pathDmin}
\end{equation}

Where $D_{min}$ is the fractal dimension of the shortest path.

We find that as $H$ increases, $D_{min}$ tends to decrease. This behavior is reflected in Figure \ref{fig:DminStudy}. We understand this behavior as follows: for the random case, we expect to find “holes”(trapped regions in IP cluster with loops) in the cluster which are also scale invariant. Paths and the distance between sites in the cluster will necessarily become circuitous. If site strengths are correlated such that similar strengths group together, and given that IP grows by breaking the weakest sites, the IP algorithm will naturally seek out connected regions of weaker sites. This means that fewer portions of the lattice will need to be sampled as the path between two connected sites becomes more direct since it is the result of correlations to create connected regions of weak site regions. Similarly, there will be larger regions devoid of any cluster growth as strong sites will likewise preferentially occupy these regions. This helps us understand the behavior of $D_f$ which is related to the density exponent according to $D_f - 2$. The smaller $D_f$ corresponds to a less dense cluster occupying the lattice, although locally in regions around the cluster, the cluster becomes more dense. This trend starts to reverse for $H>-0.1$, as the dense local cluster regions make up more of the lattice than the large voids filled with strong sites. Once again, the loopless condition will prevent $D_{min}$ from becoming 1 because some paths will not be allowed in order to maintain the existence of only 1 path joining sites in the cluster. 

This behavior is similarly summarized by looking at the backbone exponent $D_{BB}$ as the authors of \cite{prakash1992structural} did with RP. They found that as $H$ increases $D_{BB}$ approaches $D_f$, meaning that th majority of the cluster exists along the cluster backbone. This qualitatively has the effect of causing the  cluster to become both more dendritic and compact as the Hurst exponent increases. 
\begin{figure*}[t]
	\includegraphics[width=1.0\textwidth]{SIP_corr2x2}
	\caption{\footnotesize Comparison of clusters grown with different correlation exponent, $H$. As $H \rightarrow 0$ the clusters becomes more dendritic and compact.}
	\label{fig:CorrExample}
\end{figure*}



\section{Critical Threshold}
\label{section:critThresh}
One of the most important features of percolation is its relation to critical phenomena \cite{stephen1976percolation}. In the previous section we looked at the static network properties of a percolation cluster, however, criticality is characterized by the structure of fluctuations near the critical point which grow to dominate all scales of the system. This then becomes much more a question of how the cluster grew to occupy its final form.  

In RP the site occupation probability, $p$, serves as the control parameter for RP's second order phase transition. This phase transition is described by the emergence of global connectedness where the many isolated clusters existing in regime $p<p_c$ conglomerate into a single lattice spanning cluster for $p \sim p_c$. To better elicit the connection between RP and IP, its instructive to understand how RP's critical point manifests itself in IP. 

We begin by looking at the distribution of site strengths of the invaded cluster. In IP all lattice sites are randomly assigned values from a uniform distribution in the range [0,1], but when looking at the distribution of the strengths of invaded sites, we find the selection of strengths to be a regular subset of assigned strengths. In particular, in the limit where the number of invaded sites,$N$, becomes infinite, the invaded strength distribution is described by a step function:

\begin{displaymath}
\lim_{N \to \infty} p(r) = \left\{
\begin{array}{lr}
k & 0 \ge r \ge r_{max} \\
0 & r > r_{max}
\end{array}
\right.
\end{displaymath}

where a random strength, $r$, has constant probability $k$, of being invaded up to some strength, $r_{max}$. These are related according to $1/k = r_{max}$, and its been shown that $r_{max}=p_c$ where $p_c$ is RP's critical occupation probability \cite{chayes1985stochastic}.

Without a threshold, a cluster grown by IP(random) will grow indefinitely, and reproduce many of the characteristic exponents of RP's incipient infinite clusters(IIC), which is why IP is believed to reproduce the emergent incipient infinite cluster which defines RP's connected state \cite{jarai2003invasion}. A threshold in IP effectively acts in the same way as the occupation probability for RP. This can be understood heuristically as follows: we assign lattice site strengths from a uniform distribution in the range [0,1] as is usual, but then only invade sites if its strength is below some predetermined threshold. A cluster will terminate its growth once it exhausts all perimeter sites with strength less than the threshold. However, if the threshold is $r_{max} \geq p_c$ then it becomes possible to grow a cluster infinitely. Thus growing a cluster with threshold equal to $p_c$ will grow an independent realization of RP's emergent IIC.

However, to better understand the critical aspects of IP we need to understand the structure of fluctuations, and here is where the threshold seems to control the type of fluctuations which occur and will be discussed in detail in the next section. (This is the idea behind the notion of our burst mechanism presented in \cite{ortez2021universality}, and indeed we find that at the critical threshold the cluster size distribution is scale invariant.) 

Here we aim to understand how the critical threshold, which is $r_{max}$ in the random case, changes under the application of long-range correlations to lattice site strengths.


\begin{figure} 
	\includegraphics[width=0.5\textwidth]{SIP_invDist}
	\caption{\footnotesize The changing distribution of invaded strengths for different correlation Hurst parameter, $H$. For the independent random case($H=-1.0$) we recover an approximate step function reflecting constant probability of invading a particular site up to $r_{max}$ anywhere in the cluster. As spatial correlations increase it becomes increasingly likely to sample weaker sites.  }
	\label{fig:invDist}
\end{figure}

A flat uniform distribution of invaded sites is evidence that regardless of where in the lattice the growth takes place, the likelihood of a particular strength to be invaded remains constant.
If instead we could sample weaker sites with more regularity than stronger ones, we would no longer observe a flat probability, and subsequently, the threshold would change depending on the local ratio of weak/strong bonds. This is precisely the scenario introduced when introducing long range correlations into the assigned strengths. Fig \ref{fig:invDist} shows how the distribution of invaded sites changes as a result of changing correlation exponent, $H$. 

Previous studies with RP on long-range correlated lattices have shown that the $p_c$ changes depending on the Hurst parameter, $H$ \cite{prakash1992structural}. Other authors used $p-p_c \sim L^{-1/\nu}$ relationship to determine $p_c$, but this becomes problematic since $\nu$ changes as a result of long range correlations in a non-trivial way \cite{schrenk2013percolation}, and as is shown in section \ref{section:LRC}.
\begin{figure} 
	\includegraphics[width=0.5\textwidth]{B2B_study}
	\caption{\footnotesize  Here we show how the "bulk to boundary" ratio changes as a function Hurst correlation exponent $H$. For random case($H=-1.0$) we see the ratio approach $p_c=T_c$, but for $H>-0.5$ the ratio fails to asymptote to a particular value.  }
	\label{fig:B2Bstudy}
\end{figure}
For our purposes we would like to generalize our AIP model which requires us to establish a burst threshold which serves to define distinct bursts grown from within a cluster. An important feature arises with the introduction of long-range correlations: this feature being that the local strength environments produces sufficiently different thresholds such that the notion of global lattice threshold breaks down. 
\begin{table*}[t]
	\begin{center}
		\setlength{\tabcolsep}{10pt}
		\begin{tabular}{l c c c} \hline \hline
			& $D_{f}$ & $D_s$ & $1-\eta$  \\
			\hline
			$H=-1.0$ & $1.897 \pm 0.003$ & $1.860 \pm .002$ & $1.804 \pm .009$  \\
			$H = -0.9$ & $1.894 \pm 0.003$ & $1.857 \pm .002$ & $1.79 \pm .01$ \\
			$H = -0.7$ & $1.888 \pm 0.003$ & $1.863 \pm 0.002$ & $1.79 \pm .01$\\
			$H = -0.5$ & $1.880 \pm 0.002$ & $1.867 \pm 0.001$ & $1.794 \pm 0.007$ \\
			$H = -0.3$ & $1.872 \pm 0.002$  & $1.855 \pm 0.002$ & $1.78 \pm 0.01$ \\
			$H = -0.1$ & $1.871 \pm 0.002$  & $1.850 \pm 0.002$ & $1.77 \pm 0.01$ \\
			\hline \hline
		\end{tabular}
		\caption{ Static scaling exponents.}
		\label{table:static_exponents}
	\end{center}
\end{table*}
Input correlations of type in Eq. \ref{eq:betaHurstRelation} will produce produce mean strength fluctuations defined as  are described by described by:
\begin{equation}
<u(r^\prime)u(r^\prime+r)>=\left\langle\delta s^2\right\rangle - \left\langle\delta s\right\rangle^2
\end{equation}

where $\delta s = u_i - x$ and $u_i$ is strength of the $i$th site and x is the random non-correlated component of the strength. We find the mean strength fluctuations are also described by:
\begin{equation}
\left\langle\delta s^2\right\rangle - \left\langle\delta s\right\rangle^2 \sim r^{-2H}
\end{equation}
which we recognize as also describing the second moment of the strength distribution, which will have well-defined mean for $2H>2$ and well-defined variance for $2H>3$. Thus, by construction, the variance of average strengths is poorly defined since the tail events are not exponentially bounded. This results in infinite variance. Moreover, even average values for quantities resulting from averaging over distinct regions will not be well behaved. Therefore, any averaged macroscopic quantity will be poorly behaved. 

An alternative notion for a burst could rely instead on a "bulk to boundary" ratio, $r_{BB}$. The authors in \cite{prakash1992structural} used a similar argument to determine $p_c$ with long-range correlations where they determined $p_c$ by noting which $p_{occ}$ produced a ratio of 1 between the perimeter of filled and unfilled sites. Using a similar strategy authors argued that using a "bulk to boundary" ratio is a generalized way to determine the critical occupation probability \cite{mertens2017percolation}. However, determining the ratio analytically using:
\begin{equation}
\lim_{N \to \infty} \frac{N}{B(N)}=T_c
\end{equation} 
leads to slightly different results since $T_c \rightarrow p_c$ only in the random case. In Leath's  original paper \cite{leath1976cluster} the expression for the probability of finite clusters of size $n$ with $b$ empty perimeter sites assumed sites with independent random probabilities. One must instead empirically determine the ratio leading to a scale invariant distribution of bursts.  

We empirically determined the ratio, $r_{BB}$ for our clusters for different $H$. We found that the behavior of $r_{BB}$ did not universalize in any way to allow us to preserve the notion of a collective critical point. Not only do the values of stable ratios change, but we find that  for $H>-0.5$, $r_{BB}$ fails to asymptote to a fixed value. These results are shown in Fig\ref{fig:B2Bstudy}.

With random AIP, we established the existence of a critical threshold, but with long-range correlations, these relationships no longer hold. In the next section(sect. \ref{section:crit}), we discuss how the phase transition is smoothed such that there is no longer a power-law divergence with the control parameter as $T \rightarrow T_c$. The notion of criticality itself begins to break down, but its worth wondering whether we have the correct notion of the critical control parameter such that we observe universal critical behavior in the presence of long-range correlations.

\begin{figure*}[t] 
	\includegraphics[width=1.0\textwidth]{strength_fluctuations_HAll_zoom}
	\caption{\footnotesize  Mean site strength fluctuations. We show the expected scaling of lattice site strength fluctuations, $\delta \langle h \rangle_L \sim L^{-H}$. }
	\label{fig:strength_fluctuations}
\end{figure*}

\section{Correlated Critical Behavior}
\label{section:crit}
In this section we explore how the Hurst long range correlations affect AIP's growth mechanism, and in particular, we would like to establish how site correlations impact the system's critical behavior. Of course, since criticality is inherently scale free, its is most sensible to extend AIP's behavior subject to PBC. In the previous study\cite{Ortez2022critical}, we characterized AIP's critical fisher type burst distribution, $n_s(\tau, \sigma)$, subject to PBC. This distribution, in conjunction with the correlation length scaling exponent, $\nu_I$, produces a distinct universality class, though, only as a psuedo-critical system due to the absence of a natural order parameter(and perhaps suggests the suitable way of categorizing all SOC systems). AIP's underlying growth mechanism utilizes the emergent long range correlations of RP near the critical occupation probability. This property is derived from the behavior of fluctuations in average site occupation which scale according to $\delta \langle p_{occ}\rangle \sim L^{-1/\nu_{R}}$(where $\nu_R$ is the RP correlation length scaling exponent). This is the emergent structure that allows scale invariant connected burst sequences to form. However, this inherited behavior from RP is altered in AIP because we switch from RP to IP growth dynamics by instead assigning strengths between [0,1] sampled from a uniform random distribution. AIP then follows a simple path of least resistance algorithm to form a connected self-avoiding sequence of the weakest sites. The set of invaded sites and their associated strengths will form a subset of strengths in the range [0, $T_c$](where $T_c=p_c$) with characteristic length $L^{-1/\nu_I}$ ($\nu_I=1.3$ slightly different from RP's $\nu_{R}=4/3$). Bursts grown at $T_c$ reproduce RP's incipient infinite cluster(IIC), and importantly, do so without producing the adjoining distribution of finite clusters. This is the primary reason the two models differ. In AIP grown near $T_c$, we continually sample IIC's subject to an environment of already populated and grown IIC. If we now input additional correlations that yield fluctuations in strength according to $\delta \langle h \rangle \sim L^{-H}$, this alters the mechanism responsible for the emergent structure, and we expect these to affect the effective correlation length and burst size distribution. If these quantities are altered then the overall critical behavior of the system must necessarily change.   

To that end, we begin with a discussion of the correlation length scaling where we hearken back to the fundamental finite size scaling hypothesis, the bedrock of criticality. We begin by addressing the question of how site strength correlations affect the correlation length, $\xi$. 

\begin{figure*}[t]
	\includegraphics[width=1.0\textwidth]{Cr_r_All}
	\caption{\footnotesize  Here we show how the correlation function for different Hurst correlation values, $H$. We fix the threshold to be $T=0.25$ for all $H$, and we observe how the exponential decay constant, $\xi$ varies from $\sim 1$ in the random case to $\sim L$ in the maximally correlated case.  }
	\label{fig:Cr_All}
\end{figure*}

For lattice systems, the notion of correlation length is generally understood by the correlation function (pairwise correlation function) which empirically has been established to behave according to
\begin{equation}
C(r) \sim r^{d-2+\eta}e^{-r/\xi}
\label{eq:pairwiseCorrFunc}
\end{equation} 
Therefore, the correlation length, $\xi$, characterizes when random correlations become exponentially suppressed as a function of distance, $r$. If $\xi \sim L_{sys}$, then the correlations display long range behavior descried with power law $C(r) \sim r^{2-d+\eta}$. 

As before, the correlation length defines the statistical spatial extent of bursts. That is, the likelihood two sites a distance $r$ apart are to belong to the same burst. Despite the effect of increasingly dominant Hurst correlations, burst characteristics still greatly depend on the burst threshold, and thus, we expect the correlation length to similarly depend on the burst threshold.

Though the characterization of $C(r)$ provides a rather simple, intuitive understanding, it is seldom used in the literature since the pairwise correlation function is often very cumbersome to calculate. Its computations scale according $O(N^2)$ with $N$ sites/particle. Given an individual cluster ensemble contains $10^7$ sites, of which we use $10^2-10^3$ ensemble elements to obtain reliable statistics, aquiring the requisite statistics quickly becomes computationally prohibitive. Fortunately, we were able to rely on a highly optimized and parallelized implementation to efficiently compute, $C(r)$ \cite{2020MNRAS} which gave good results.

We find that the affect of long range correlations on burst formation is to change the required threshold that is likely to produce an infinite sized burst(when $R_{gyr} \sim \xi$). The affect of $H$ on correlation function, $C(r)$ is shown in Fig\ref{fig:Cr_All}. In order to isolate the $H$ dependence, we keep $\epsilon_{T}$ fixed for all curves and vary $H$.  In the random case, we observe the usual $\xi(T)$ that becomes exponentially suppressed for burst of order 2, but with correlations we observe the additional $H$ dependence, $\xi(T,H)$ which can greatly extend the correlation length despite keeping $\epsilon_T$ fixed. In fact, we can nearly reproduce the $C(r)$ near the critical value by merely changing $H$.  One can work out that the expected relation is,
\begin{equation}
\xi(T,H) \sim \epsilon_{T}^{1/H}
\label{eq:Corr_len_T_H}
\end{equation}

which agrees with the empirical results of Fig\ref{fig:Cr_All}. In our formulation, the random case corresponds to $H=-1.0$ which yields a correlation length of order $1/\epsilon_{T}$, while for maximally correlated case, $H=0$, yields a strongly diverging correlation length and becomes strongly limited by the system lattice size, $L_{sys}$. 

This effect shows how Hurst site correlations become increasingly dominant as correlations increase ($H\rightarrow0$). We even find that the correlation length can be more dependent on Hurst correlations parameter, $H$, than the burst threshold used to define, $\epsilon$. This leads us to consider different regimes where the burst behavior is better described by Hurst site correlations and those regimes better described by typical critical parameter. As is common practice we define a characteristic lengths, $\xi_H$ and $\xi_I$ to correspond to these respective length scales. 
\begin{figure*}[t] 
	\includegraphics[width=1.0\textwidth]{SIP_periodic_critical_improved}
	\caption{\footnotesize  Here we show how the cluster distribution moments fail to diverge, except for the random case ($H=-1.0$). The transition region becomes increasingly broadened as the Hurst correlation parameter increases. (left) Represents the zeroth moment or naturally corresponds to the number of bursts as a function of threshold, $T$. (right) The behavior of the second moment corresponding to the average burst size as function of threshold, $T$. }
	\label{fig:critMoments}
\end{figure*}
\begin{comment}
mechanism becomes dominant when it is sufficiently large and why the typical $\epsilon_T$ characterization describing the random AIP correlation length affects behavior over an increasingly narrow range.(for example, $\epsilon_{T,H=-1.0}\in [0,1]$ and $\epsilon_{T,H=0} \in [0,0.25]$) Therefore, we can work out the behavior when $\xi_P \ll \xi_H$, and $\xi_P \sim \xi_H$.
\end{comment}

For length scales much smaller than the system lattice size, $l \ll L_{sys}$ we recover the typical description of critical behavior characterized by the scaling of the critical parameter, and in the random AIP we found suitable scaling behavior for $\epsilon_T = (T_c-T)/T_c$. This characterization describes how the correlation length scales according to $\xi_I \sim \epsilon_T^{-\nu_I}$. This scaling is preserved for all length scales subject to the condition, $l<\xi_I$. For larger length scales we observe crossover phenomena driven by the affects of Hurst correlations on scales, $\xi_I< l <\xi_H$. 

Briefly, we can anticipate this crossover behavior by recalling the arguments of the extended Harris condition which led to Eq.\ref{eq:HarrisCond}. These results ought to describe the changes in correlation scaling under the condition that the global lattice correlation length uniformly diverges. As we will show, the divergences become increasingly smeared as Hurst correlations become increasingly dominant. This follows what was shown in the previous section(\ref{section:critThresh}) which shows the critical point itself being spread out over a range. Also, Fig\ref{fig:TcvsH} shows the burst threshold likely to produce scale invariant bursts as a function of $H$. Not only does the required threshold drop, suggesting fewer range of strengths need to be sampled to grow arbitrarily large bursts, but also, the typical characterization relying on $\epsilon_{T}$ breaks down as the critical burst threshold becomes degenerate. 

Despite the Hurst correlations becoming dominant on the largest of scales and near the critical point, we can still see evidence of more traditional critical behavior described by burst threshold parameter, $\epsilon_{T}$, away from the critical point. By looking at the zeroth and first moments(number of bursts and average burst size) of the burst distribution, we can observe qualitative differences associated with changing $H$. One way to quantify the onset of crossover behavior is by recognizing that the width of the critical transition region scale with $\xi_H^{-H}$. That is, the fluctuations become dominated by the Hurst correlations. However, so long as $\epsilon_T > \xi_H^{-H}$ the scaling should be dominated by the usual scaling dependent on $\epsilon_T$. 

We can observe this regime in Fig\ref{fig:critMoments}, where we can see similar behavior of the moments up until the vicinity of the critical point where the behavior radically departs. This is evidence for crossover phenomena where for $l < \xi_I$ we get behavior largely characterized by $\epsilon_{T}$, and a departure from this behavior for $l>\xi_I$. Further, we can notice that the curves of different $H$ do not lie on top of each other, indicating that the scaling exponents ought to be different. In this case, the zeroth and second moments are given by $\alpha, \gamma$ and we do find changes with these scaling exponents as $H$ changes.


The essential bit of evidence for critical behavior is the existence of fisher type burst distributions, and we observe the existence of scale free burst distributions for all ranges of $H$ we considered.
As we showed previously, we were largely able to define the critical behavior by characterizing the burst distribution, namely, $n_s(\tau, \sigma)$. This strongly suggests that a similar description should be appropriate here where we begin with the description of a critical threshold, $T_c$.



As was insinuated by Fig\ref{fig:TcvsH}, there exist distinct critical thresholds which give rise to distinct cluster/burst distributions for different $H$. So long as $\epsilon_{T} > \xi_I^{-1/\nu_H}$, we expect there to exist a scale invariant distribution of bursts up to some scale $\xi_I$. Our task is to understand precisely how the critical behavior changes with $H$ in this regime. To that end, we might expect a precise relationship governing $\tau(H)$, since from Fig\ref{fig:Htau} we can see an increasing magnitude of $\tau$ as $H$ increases. An increase in the magnitude of $\tau$ translates to the general trend that prefers cluster growth by smaller bursts rather than larger ones. However, Fig\ref{fig:critMoments} shows average burst size decreasing only very near the critical point and above. Below the transition value, the tendency is for cluster growth to occur via larger average burst sizes as Hurst correlations increase. This is further evidence of a crossover type of phenomena near the transition regime. Still, we will want to be able to account for this somewhat contradictory behavior in our description. 





Of course one of the nice features of working with the burst distribution is that we can directly calculate the expected behavior of the average burst size scaling. This is done with the usual moment calculation, 
\begin{equation}
M_k =\epsilon_T^{\frac{1+k-\tau}{\sigma}} \int_{0}^{1}dz\ z^{k-\tau}f[z]
\label{eq:momentInt}
\end{equation}
where here $k=2$, $z=(T_c-T)s^\sigma$, and in the upper limit, we observe the relation $s \gg (T_c -T)^{-1/\sigma}$. Thus, $s_\xi(T) = (T_c -T)^{-1/\sigma}$ behaves as the exponential cutoff cluster size for the cluster size distribution. Since, the integrand evaluates to a constant, we once again get the familiar $\gamma$ exponent relationship, 
\begin{equation}
\gamma = \frac{2-\tau}{\sigma}
\label{eq:gamma_H}
\end{equation}
However, we must characterize the burst cutoff size $s_\xi$, where we observe length burst size relation, $s_\xi = \xi^{D_s}$ from Eq.\ref{eq:fractalDim} and leads to the exponent relation,
\begin{equation}
\frac{1}{\sigma} = D_s \nu
\label{eq:sigma_H}
\end{equation}
There is a slight distinction between $D_s\approx 1.865$ and $D_f\approx 1.896$ in the random case, which does not significantly affect the critical behavior, and these differences decrease for increasing Hurst correlations since $D_s \sim D_F$. As was discussed previously and found in section\ref{section:SNP}, these fundamental mass-length scalings change very little for changing Hurst correlations, therefore, this leads to the important behavior of $\nu$ which we introduced in section\ref{section:LRC}.

\begin{figure} 
	\includegraphics[width=0.5\textwidth]{TcvsH}
	\caption{\footnotesize  Here we show how the value of thresholds,$T$, where $\xi(T,H) \sim L_{sys}$ giving rise to scale invariant burst distribution.}
	\label{fig:TcvsH}
\end{figure}
By plugging Eq.\ref{eq:sigma_H} into Eq.\ref{eq:gamma_H} we establish the following relation, 
\begin{equation}
\gamma = (2-\tau)D_s \nu
\label{eq:gamma_nu}
\end{equation}
which will allow us to determine how the average burst size ought to behave as a function of exponents $\tau, \nu$. We expect from the extended Harris criteria that the site strength correlations become relevant when their associated correlation scaling becomes larger than that of the random case, $\nu_H=1/H>-4/3$. While we do see some minor affects for $H=-0.9$ on the correlation length scaling, we generally observe behavior consistent with the extended Harris criteria, which tells us that for $H>-3/4$ the correlation exponent, $\nu_H$ is given by,
\begin{equation}
\nu_H = 1/H
\label{eq:nu_H}
\end{equation}

We confirm this behavior by calculating $\xi_H$ in the standard way \cite{Ortez2022critical,stauffer1994introduction}. The obtained scaling is reported in Table\ref{table:crit_exponents}. 

\begin{figure} 
	\includegraphics[width=0.5\textwidth]{H_tau}
	\caption{\footnotesize  The burst size frequency-magnitude scaling for different Hurst correlations, $H$. We observe a spectrum of $\tau$ exponents [1.59, 1.90] characterizing the burst size distribution as $H$ changes. As Hurst correlations increase $\tau \rightarrow 2$ which indicates a preference for smaller burst sizes.  }
	\label{fig:Htau}
\end{figure}

\begin{table*}[t]
	\begin{center}
		\setlength{\tabcolsep}{10pt}
		\begin{tabular}{l c c c c c } \hline \hline
			& $T_c$ & $\tau$ & $\nu$ &$\gamma_{obs}$ & $\gamma_{th}$ \\
			\hline
			$H=-1.0$ &$0.5926(5)$ & $1.594(2)$ & $1.301(2)$ &$0.971(5)$ &$0.9825$  \\
			$H = -0.9$ & $0.590(1)$ & $1.597(2)$ & $1.359(2)$ &$1.025(3)$ & $1.017$\\
			$H = -0.7$ & $0.580(5)$ & $1.635(2)$ & $1.47(2)$ &$1.041(5)$ & $1.00$\\
			$H = -0.5$ & $0.570(5)$& $1.711(2)$ & $1.95(2)$ & $1.066(3)$  & $1.052$ \\
			$H = -0.3$ &$0.54(1)$ & $1.810(2)$  & $3.1(1)$ & $1.10(2)$ & $1.09$ \\
			$H = -0.1$ & $0.48(9)$  & $1.90(5)$ & $7.6(5)$ & $1.18(1)$ & $1.4$ \\
			\hline \hline
		\end{tabular}
		\caption{ Critical scaling exponents. comparison of scaling exponents for AIP model with different Hurst correlations $H$, where $H=-1.0$ is the random case($H=0$ in the usual formulation)and Hurst correlations increase with increasing $H$. We used a 4096x4096 lattice with PBC to generate statistics. In order to account for any remaining finite size affects, we set the burst size threshold to be $10^6$. We used at least $10^9-10^6$ bursts for all statistics, depending on the proximity to the critical point. The error represented in parenthesis of the final digit is the error in LLS fit. We find that critical relations start breaking down as $H\rightarrow0$, indicating critical processes no longer govern behavior. }
		\label{table:crit_exponents}
	\end{center}
\end{table*}
As we observed previously, it is somewhat unexpected that the average burst size should increase with decreasing, $\tau$. Eq.\ref{eq:gamma_nu} neatly provides an explanation for why the average burst size increases despite the burst distribution preferring smaller clusters. The necessary condition is for $\nu_H$ to increase more drastically with $H$ than the decrease of $\tau$. Since $\nu_H$ largely follows the extended Harris relation, we can see that this condition is satisfied, and the result is increasing magnitude of the $\gamma$ exponent. We verify Eq.\ref{eq:gamma_nu} by numerical average burst size calculations.   

\begin{figure*}[t] 
	\centering
	\includegraphics[width=1\textwidth]{critCorrLen_comp}
	\caption{\footnotesize Correlation length comparison for different Hurst correlations, $H$. Generated from the statistics of over $10^7$ bursts grown with PBC on lattice of size 4096x4096. (left) We plot the correlation $\xi_T$ vs burst threshold, $T$. (right) We plot the burst critical scaling with critical parameter, $\epsilon_T$. For the random case with $H=-1.0$, we get correlation length scaling exponent, $\nu=1.30$, which is nearly similar to the RP value. Also, we can confirm that for $H>-3/4$ we get correlation length scaling exponent given approximately by $\nu_H \sim 1/H$.  }
	\label{fig:corrLen_allH}
\end{figure*}

We summarize the behavior of the system for length scales, $l < \xi_b$, as being primarily critical in nature since we observe a correlation length governed by $\xi_l \sim \epsilon_{T}^{-\nu_H}$. This correlation length scaling forms the basis for a family of interdependent critical exponents and most importantly the existence of a scale invariant burst distributions characterized by exponents, $\tau$, $\sigma(\nu)$. Some exponents change very little($D_f, D_s, 1-\eta$), while others change quite noticeably ($\tau,\nu$). These family of exponents potentially give rise to a whole host of distinct universality classes if these properties are extrapolated to the entirety of a system, but we are careful to note that these relations only hold up to a certain length scale, namely, for $\xi < \xi_H$. In fact, our system more importantly displays crossover phenomena where the critical point gives way to new behavior governed by the Hurst correlations. 

\begin{figure}
	\centering
	\includegraphics[width=0.5\textwidth]{burstDist_wideRange}
	\caption{\footnotesize Burst scaling for correlated case, $H=-0.10$. Each curve represents the burst statistics for different burst thresholds, $T$. Also shown are the linear fits to each curve represented by LLS and by MLE methods. The threshold becomes degenerate as a wide range of threshold lead to similar burst scaling statistics. We see a family power-laws for thresholds in the range $[0.329,0.534]$ which produce scaling exponents $\tau$ in the range $[1.88,1.92]$}
	\label{fig:corrIP_degTau}
\end{figure}

\begin{comment}
qualitative understanding given by a failure of the avalanche burst mechanism which relies on a particular threshold to yield scale invariant burst, rather a range of thresholds can all produce scale invariant bursts. 
\end{comment}

The reason for crossover phenomena stems from the requirement of the extended Harris criteria which relied on the existence of a global, uniform lattice transition. Should this condition exist, we would fail to observe crossover phenomena, but this condition fails precisely because the transition is smeared over a range of thresholds. The affect of Hurst correlations is to produce distinct regions of site strengths which alter the threshold required to grow scale invariant bursts within these pockets. This range of thresholds is given by $\xi_H ^{-1/H}$ and therefore quickly increase as $H\rightarrow 0$. Further, these correlated regions themselves are scale invariant by construction which is what effectively permits a range of threshold to produce the same scale invariant bursts. We can see primary evidence of this for $ \xi > \xi_H$ by looking at the Fisher burst distributions, $n_s(T)$ for different $T$. Fig\ref{fig:corrIP_degTau} gives an example of the degeneracy of burst distributions for different thresholds for $H=-0.1$. We observe a wide range of thresholds produce nearly the same scaling. This suggests that the burst distribution scaling is governed primarily by the correlated scale invariant pockets of site strengths rather than the avalanche burst threshold mechanism. This fundamentally alters the mechanism generating the critical fisher distribution from the avalanche burst type to Hurst correlations, and importantly admits mechanisms that are non critical in origin (at least not governed by $\xi \sim \epsilon_{T}^{-\nu}$) to produce a scale invariant burst distribution. This last point is particularly important since the presence and characterization of the fisher distribution has largely been adequate in motivating and confirming critical behavior. Here, we have an example of such a distributions absent the usual mechanisms that drive a critical transition; one where the $\epsilon$ characterization fails to describe scaling behavior. Still, the correlation length of the system is a power law, but it does not diverge as a function of the proximity to the critical point, and therefore fails to satisfy the requirements of second order phase transition mechanics. 

\begin{comment}
Now discuss how this can be calculated purely from the correlation function and how it dictates that $\tau$ should not change. THis failure of the fluctuation dissipation theorem tells us that the statistics of the critical phenomena is changed dramatically in the strongly Hurst correlated case. 
\end{comment}
For behavior with length scales, $l>\xi_I$ we observe crossover behavior where the dominant length scale becomes the Hurst correlation length scale, $\xi_H$. On this length scale we expect fluctuations to be governed by $\xi_H ^{-1/H}$ and general property scaling of a function to be given by $f(l,\xi)=\xi_H^af(l/\xi_H)$.

We can find an interesting way this crossover behavior manifests itself within the general framework characterizing our system. Returning to the correlation function, if $\xi(T,H) \rightarrow \infty$, then we get power-law scaling of the correlation function given by $C(r) \sim r^{2-\eta}$ since the exponential term goes to 1. Of course, since we simulate finite sized clusters, the condition of an infinite correlation length is when $\xi > L_{sys}$. This insures that there is very little deviation from power-law over the calculable range of $r$. In practice, this is done by choosing a burst threshold suitably large($T>T_c$) so as to observe  power-law behavior for $r$ up to $L_{sys}$. In this limit we find very little difference between the $\eta$ exponent for different $H$ and are very near the random uncorrelated case. 

\begin{comment}
The negligible affect of Hurst correlations on the scaling of $C(r)$ is reminiscent to the minor effect $H$ had on cluster fractal mass scaling, $D_f$. It is likely that the burst substructure of AIP is no longer relevant as the number of bursts reduces to a single cluster for $T>T_c$. In which case, like the $D_f$ scaling it proves to be too coarse a measure to be sensitive to the dendritic affect Hurst correlations have on cluster structure. Still, it is quite surprising that the long range structure of $C(r)$ isn't affected by $H$, and may suggest that the $\eta$ exponent is explicitly a measure of AIP's growth mechanism. It lies above percolation backbone scaling $d_B$ and below mass scaling $D_f$.
\end{comment}
A complementary technique for calculating the average burst size comes from the fundamental, fluctuation dissipation theorem \cite{ma2018modern} which relates the susceptibility to average site strength correlations. In the language of percolation this becomes a relation between the pairwise correlation function mentioned previously, $C(r)$, and the average burst size, $\langle s \rangle$ \cite{coniglio1979cluster, stauffer1994introduction}. The relationship is given by,
\begin{equation}
\langle s \rangle = 1/V \int dV C(r)
\label{eq:fluctDissThm}
\end{equation}
where $V$ is typically taken to be the correlation volume given by, $\xi^d$. Again, we compute $C(r)$ using \cite{2020MNRAS} with good results. With $C(r)$ given by Eq.\ref{eq:pairwiseCorrFunc} we can rewrite Eq.\ref{eq:fluctDissThm} as,
\begin{equation}
\begin{split}
\langle s \rangle &= \xi^{-2} \int dr \ r^{2-\eta}\exp^{-r/\xi} \\
& = \xi^{-2} \xi^{3-\eta} \int_0^{\infty}dz \ z^{3-\eta} \exp^{-z} \\
& \sim \xi^{1-\eta} \\
& \sim \epsilon_{T}^{\nu(1-\eta)}\\
\end{split}
\label{eq:critSuscScaling}
\end{equation}
where $z=r/\xi$ and the integrand yields a constant. The last line also makes use of the usual critical scaling relation $\xi \sim \epsilon_{T}^{-\nu}$ which we know is valid on smaller length scales. This gives another relation,
\begin{equation} 
1-\eta = (2-\tau)D_s
\end{equation}
by plugging into Eq.\ref{eq:gamma_nu}. Since $1-\eta$ and $D_s$ are nearly constant for changing $H$, we would similarly expect $\tau$ to be nearly constant, but this of course is not true as is shown in Fig\ref{fig:Htau}. Solving the above equation for $\tau$ gives a value near $1.59$, which is the burst distribution scaling for the random case and for $\gamma \sim 1$.  

This contradictory result indicates the dominant fluctuations and average cluster size no longer scale in the same way. Knowing that there exist multiple correlation lengths in the system, it is problematic to integrate over all length scales without distinguishing length regimes. If we consider the scaling of larger bursts, then we find an average burst size that scales according $(1-\eta)/H$, where this relation predicts a much more dramatic increase in cluster sizes as $H$ increases. For example, for $H=-0.3$ we expect $\gamma \sim 2.5$, in contrast to the $1.1$ we observe for scales up to $\xi_I$. The failure of the fluctuation dissipation theorem as originally defined provides further evidence that behavior departs from the usual critical fluctuation scaling dominating percolation transitions, which is entirely interrupted by Hurst correlation effects.

\begin{figure}
	\centering
	\includegraphics[width=0.5\textwidth]{burstEpicenterScalingComp}
	\caption{\footnotesize We show the crossover behavior inherent to AIP subject to different, $H$. We find that average burst epicenter density follows correlation length determined by $\xi_I \sim \epsilon_T^{-1/2}$ for small scales and $\xi_H^{-H}$ at the crossover.}
	\label{fig:burstEpicenters}
\end{figure}

Finally, in our previous work \cite{Ortez2022critical}, we found the burst epicenter scaling differed from the general site mass scaling. This result is significant because it suggests the existence of multiple correlation lengths. One correlation length is associated with the likelihood of sites to occupy different burst clusters, and the other correlation length is associated with the likely distance between burst centers. By looking at the scaling of average burst epicenters density, $\langle \rho_b \rangle$ as a function of length scale, we found the expected crossover behavior where for lengths less than $\xi_I$ we get scale invariant $\langle \rho_b \rangle$ and nearly homogenous $\langle \rho_b \rangle$ for length scales greater. The scaling of $\xi_I$ as function of burst threshold also was found to behave with mean-field correlation exponent, $\xi_I \sim \epsilon_{T}^{-1/2}$, suggesting that burst epicenters were distributed as a random walk about the lattice. 

With the addition of Hurst correlations, there is yet another correlation length to factor, and we find scale invariant $\langle \rho_b \rangle$ on smaller length scales, but the crossover length scales for different $H$ follow $\xi_H^{-H}$ dependence. However, the threshold dependence still follows a random walk characterization. Fig\ref{fig:burstEpicenters} shows how the crossover length of $\langle \rho_b \rangle$ changes for different $H$. Namely, we can see that this length follows $\xi_H^{-H}$ as it gets shorter as $H \rightarrow 0$.   

The behavior of $\langle \rho_b \rangle$ provides an instructive way to understand why the generalized burst scaling breaks down. The crossover length above which $\langle \rho_b \rangle$ becomes uniform, indicates that burst with characteristic length above this must uniformly spaced. A burst with a much larger characteristic length would necessarily cause a great void, and thus because burst centers are distributed according a random walk, this will necessarily limit the maximum size a burst can grow. Since this happens at a smaller length scale for larger correlations, we expect to see a preference for smaller bursts which is what we find since $\tau \rightarrow 2$ for $H \rightarrow 0$. 

\section {Discussion}

In our previous study \cite{ortez2023avalanche}, we found that the AIP model was not the pedagogical critical system. Rather, it was built upon the inherent scale invariance of RP and AIP's growth mechanism, which largely preserved scale invariance while altering the generalized critical scaling behavior in non-trivial ways. Absent an order parameter, it is not appropriately described as the transition of the system from one phase to another, rather, it seems to describe the growth mechanism of the meta-stable spinodal state, since it still produced a characteristic critical fisher type distribution. In this study, we include the addition of long-range order input from site-strength Hurst correlations. The result is that the clustering behavior of sites into bursts is changed, and therefore alters the resulting critical fisher distribution. Up to length scale, $l<\xi_I$, we get descriptions of different classes of meta-stable spinodal type growth governed by $H$, but more importantly, we find that the existence of $n_s(\tau, \sigma)$ is preserved even when the underlying burst mechanism was not critical in nature(ie is not governed by critical parameter, $\epsilon_T$). In this regime, the long-range order of the background correlated lattice of site strengths could preserve scale invariant bursts over a wide range of length scales regardless of burst threshold. 

With these additions, we've been able to distance the behavior of correlated AIP from traditional critical mechanisms. This offers some interesting possibilities. First, because the existence of critical systems required the system to be at a rather finely tuned point between small scale order and large scale disorder, namely near the critical point, it is fair to wonder whether traditional critical theory can appropriately describe all forms of emergent long-range order(a potentially rare event, since we would necessarily have to rely on a single point in all of phase space). SOC systems offered the advantage in that they do not require systems to be finely tuned to a particular value of phase space, rather, a small external driving mechanism allowed the system to form meta-stable growth dynamics which results in the system innately growing in the critical regime. In many ways SOC growth dynamics seem to describe the potentially long-lived meta-stable states describing droplet nucleation and spinodal states. Authors have noted the similarity between SOC variants of tradition critical systems. For example, \cite{grassberger1996self} shows how in SOC singularities arise not from order parameters, but instead, from control parameters which have a critical value. Also, \cite{gabrielli2000invasion} argues for the existence of characteristic ratio driving the behavior of SOC systems. 

While our findings support these claims, here, we also show how scale invariant behavior resulting from the competition of correlation mechanisms allows us to consider an even wider range of possibilities. Ever since Fisher showed that much of critical behavior could be characterized from characterization of a critical distribution, $n_s(\tau, \sigma)$ \cite{fisher1967theory}, its been widely assumed that the existence of Fisher type of distribution demonstrated critical behavior. In this study, we find this not to be true. We find a Fisher distribution even when the driving mechanism is not near its critical value. That is, the fisher distribution exists largely independent of its proximity to the critical value (in our case the critical value is the burst threshold). Thus, not only do we escape needing to apply artificial phase transition mechanics, but also the need to argue for a self-organization about some general critical point. Thus, in systems with implicit long-range correlations, the correlation length of the dynamics is strongly determined by the long-range correlations, and done in such a way that preserves dynamic scale invariant properties(ie burst/cluster formation).

The generality of this result applies directly to broad extrema SOC type systems(of which AIP and CAIP belong), and can naturally be applied to stochastic energy minimization systems like interface motion in disordered media leading to domain walls \cite{cieplak1994optimal}, minimum spanning trees describing strongly disordered spin-glass models \cite{jackson2010theory}, abrupt species morphology changes through gradual changes in biologic fitness \cite{sneppen1995evolution}, and optimal neural topologies \cite{bornholdt2000topological}.

However, as much of our work is focused on the features of the IP process, which is best known as a drainage process of fluid infiltration \cite{stark1991invasion,sornette2006critical,knackstedt2021invasion,klein2007structure}, we focus on CAIP's application to fracture mechanics and induced seismicity\cite{norris2014loopless,rundle2020constrained,ortez2021universality,ortez2023avalanche}.

In \cite{chen1991self}, the authors argued for SOC description of tectonic seismicity producing rupture events with $b\sim 0.4$. This follows the work of \cite{vere1976branching, bebbington1990percolation} among others, that the comparison of model event scalings should be made independent of the $3/2$ energy scaling factor implicit in G-R scaling values. Without this factor, tectonic seimicity is described by $b \sim 2/3$, which is closer to many mean-field models $b=1/2$ and our random AIP model,  $b\sim0.6$. However, with correlations, we can obtain $b$-values in the range $[0.6,1.0]$ depending on the $H$. This can in part account for the larger $b$-values associated with induced seismicity, $[0.8,1.3]$. 

However, given the wide range of observed induced seimic event scalings, it is likely necessary to account for the inherently 3-d injection activity that only in some cases can be constrained to be 2-d. \cite{maxwell2011microseismic} shows how even within the same shale, the dimension of the injection activity can greatly differ. A previous study\cite{norris2015anisotropy} attempted to parameterize 1d and 2d growth through an anisotropic preference for growth along one of the axes. In the limit where growth was strongly directed and along 1 axes, the burst scaling changed from $\tau(2D)=1.52 \rightarrow \tau(1D)=1.45$. This represents a change far too small to account for the diversity of $b$-values associated with induced seismicity if only the dimension is allowed to change. An instructive way to understand why increasing the dimension only has a small affect on burst scaling, is to understand that all burst sizes are nearly equally likely to increase in size as a result of a new degree of freedom. Further, above the critical dimension(d=6) we expect all scaling to be mean field making this notion exact. 

This problem is potentially side-stepped if we extend Hurst site correlations to 3d as well. We find the existence of a burst epicenters to exhibit interesting behavior, being primarily distributed according to typical percolation process on small length scales, but experiences crossover behavior for length scales greater than $\xi_b$ where the burst centers are distributed according to a random walk. This means that clusters larger than $\xi_b$ will not follow percolation type scaling, but tend to uniform density. This necessarily limits the probability of larger clusters. 

In 2d we have already found that larger Hurst correlations prefer growth by smaller bursts, and when we allow this preference to be amplified by an additional degree of freedom, its likely that we will observe an even more dramatic change in burst scaling. We know that the random walk nature of bursting behavior is essentially preserved with the introduction of $H$, and permits a higher density of bursts then would be allowed if it were entirely a percolation system. This can account for event densities which are essentially uniform on large scales, and have mass scaling that is given by the usual non-fractal dimension scaling, $d$. Thus, by extending our model along with Hurst correlations into 3d we expect to reproduce a greater range of $b$-values.

Finally, the primary feature inherited by traditional critical(2nd order phase transition) phenomena is a single correlation length diverging with power-law behavior. In many real physical complex system, there may exist multiple correlation lengths dominating behavior within their respective regimes. We have shown that depending on the growth dynamics, long-range order can subsequently be modified through the competition of correlation mechanisms and alter the scale invariant behavior in non-trivial ways. Contrary to criticality requirements, we find the existence of multiple correlation lengths to be consistent with power-law behavior, and is indicative of SOC type processes where the critical nature is fundamentally changed. We still preserve many essential features, namely, the ability of a stochastic process to manifest itself across a wide range of scales, but the interference of competing correlation mechanisms alters the resulting behavior. Understood across all length scales, the correct approach is perhaps a multifractal one where the characteristic distributions behave with moment description, $M_k(L) \sim L^{y(k)}$, with the key additional understanding that for many of the scales of interest scale invariance is essentially preserved.

The multifractal framework \cite{halsey1986fractal} understood through the lens of the heirarchy of correlation lengths would have a set of fractal scalings describing the dominant singular behavior associated with each length scale. Briefly, in the case of a single length scale, which in addition gives rise to hyperscaling, yields moment distributions in terms of correlation lengths according to, 
\begin{equation}
M_k(l, \xi) \sim \xi^{d-kD_f} f(l/\xi) 
\end{equation}
where $k$ represents the k-th moment,$D_f$ the characteristic mass scaling, and $f(l/\xi)\rightarrow 1$ for $l\ll \xi$. The essential behavior is that exponents of successive moments are equally spaced according $kD_f$, since $k$ is an integer. Should we have $\xi_1>\xi_2$ where $\xi_1, \xi_2$ are the respective characteristic length scaling regimes, with scaling exponents, $D_1, D_2$ then we would expect, for lengths $l>\xi_1$  to behave according to $\xi_2^{d-kD_2}\sim (l^{d-kD_1})^{d-kD_2}f(l/\xi_2)$. This suggest that the moment dependence would be quadratic  with $k$ rather than linear, and thus, for its successive moments to depend on $y(k^2)$. The inherently multifractal behavior of our model provides an excellent case study for developing and better understanding current multifractal analysis techniques. 

\section{Acknowledgements}
The research of RAO and JBR has been supported by a grant from the US Department of Energy to the University of California, Davis.

\bibliography{CAIP_bibliography}
\appendix
\section{Correlation Algorithm}
We execute this Fourier filter technique using an FFT on a NxN dimensional array with complex coefficients \cite{turcotte1997fractals}. The algorithm is outlined as follows:
\begin{enumerate}
	\item We generate a NxN array with each value,$h_{nm}$, assigned a random value from a Guassian probability distribution.
	\item We execute a 2D Fast Fourier Transform(FFT) giving an array of complex coefficients, $H_{st}$.
	\item We define radial wave number $k_r$, which is non-zero for $s=t=0$, as follows:
	\begin{equation}
	k_r=\sqrt{1+s^2+t^2}
	\end{equation}
	\item Since $S(k_{st}) \propto {|H_{st}|}^2$ we define a new set of complex coefficients, $H_{st}$, multiplied by the appropriate filter function:
	\begin{equation}
	H_{st}^\prime = H_{st}/k_r^{\beta/2}
	\end{equation}
	\item Apply an inverse FFT(IFFT) on $H_{st}^\prime $ to produce a new NxN array with coefficients, $h_{st}^\prime$ with the desired correlations.
	\item Apply the error function, $erf(h_{st}^\prime)$, to return a uniform correlated distribution with values in range [0,1].
\end{enumerate}

We illustrate an example of the types of correlations produced by our algorithm in Figure \ref{fig:corrLattEx}.

\end{document}