
\vspace{-2.5mm}
\section{Experiments}
\vspace{-3mm}
\subsection{Data}
\vspace{-1mm}
\noindent \textbf{Singing.} We used the OpenSinger dataset \cite{huang2021multi}, which has $50\si{h}$ clean recordings from $76$ speakers. We used 90\% of the audio from the $71$ speakers for the training. We use the remaining 10\% and the other $5$ speakers' recordings as two separate validation sets, \emph{seen} and \emph{unseen} speaker datasets, respectively, to compare the effect of dry source distribution shift on the model performance.

\noindent \textbf{Drum.} We collected the source signals by ourselves; we rendered the \texttt{[kick]}, \texttt{[snare]}, \texttt{[hat]}, \texttt{[tom]}, \texttt{[ride]}, and \texttt{[crash]} tracks separately with $14$ commercial sampling libraries and MIDI files from the Groove MIDI Dataset \cite{groove2019}. 
We performed equalization to the tracks so that each instrument has the same average frequency response across the kit. 
To generate each reference audio, we sampled a random segment from the dry tracks. Then, we generated a graph with the input nodes corresponding to non-zero energy tracks so that there are no dummy subgraphs. This indicates that our system should also perform a drum instrument recognition task. We used the $11$ kits and the MIDI files from the $\texttt{drummer1/session1-3}$ subset for the training. We used $\texttt{drummer1/eval\_session}$ for the two validation sets; we used the same kits for a \emph{seen} kit validation set and the remaining $3$ kits for an \emph{unseen} kit validation set. 
\section{Results}
\label{results}

\begin{figure*}[ht]
    \centering
    \includegraphics[scale=0.15,trim={0 2.5cm 0 5cm},clip]{images/aoi-single_burst}
    \caption{The time average peak Age of Information with burst and \gls{soa} loss values against the dynamic reliability logic for different network topologies.}
    \label{fig:aoi_burst}\vspace{-0.4cm}
\end{figure*}


This paper focuses on both transport layer and application layer metrics to determine the feasibility of dynamic reliability. For this, we have selected the session packet volume, as transmitted, retransmitted, lost and backlogged packets as \glspl{kpi} for the transport layer; while focusing on the \gls{aoi} for the application layer. The \gls{aoi} was chosen as a crucial indicator for the freshness of packets in real-time applications. More specifically, this work adopts the time average peak \gls{aoi} equation \cite{aoi_equation} depicted in Eq. \ref{aoi}, where $\Delta(r_{i+1})$ is the $i$th update at the time it was received at the server, for a session time period of $\tau$.

\begin{equation}
    \label{aoi}
    \gls{aoi}_\tau = \frac{1}{n-1}\sum_{i=1}^{n-1} \Delta(r_{i+1})
\end{equation}

We include a comparison between the vanilla QUIC implementation which does not enjoy the dynamic reliability extension, with a number of dynamic reliability policies. The tests were run a number of times for statistical significance, with the mean value of vanilla implementation used as a baseline for comparison. The topology utilised both random loss and bursty loss to explore the bounds of dynamic reliability. The \gls{soa} loss in the figures correspond to the loss values presented in Table. \ref{tab:path_char}, for ease of comparison between bursty and random loss scenarios.

\subsection{Transport-Layer KPIs}

To analyse the performance gain at the transport layer due to dynamic reliability, the volume of transmitted and backlogged packets is examined. The figures are in the form of boxplots, which take the vanilla implementation as a benchmark, depicted as the red dashed line.

As seen in Fig. \ref{fig:sent_burst}, the loss plays a crucial role in the performance of the reliability policies. The policies under random loss did incredibly well for the networks with a larger capacity, namely \gls{mmwave} and Sub-6~GHz, whereas for burst loss, the lower network capacities had a larger packet reduction. With the increase in burst loss, the behaviour of the set split reliable policies became unpredictable, if a reliable assignment happened to coincide with a burst loss, the number of transmitted packets increases, and vice versa. On the other hand, in smarter policies, such as Loss-Aware, the performance lightly matched the vanilla baseline, as the reliable assignment dominated the session to compensate for a higher burst loss. Not only that but, the burst loss also impacted the variance of the transmitted packets for the policies.

Unsurprisingly, the unreliable focused policy, 80-20 split, outperformed other policies for all topologies in random and bursty loss scenarios, with an approximate reduction of 80\%. That being said, the majority of the policies reduced the transmitted packets on the link by approximately 70\% for random loss, while the reduction started at $\approx 15\%$ and decreased as the loss increased for the burst loss scenario.

The retransmitted and lost packets, not shown due to space limitations, followed the same trend as the transmitted packets for the random loss scenarios. However, for the burst loss scenarios, the larger capacity networks had a lower reduction in the retransmitted and lost packets. This can be seen as a favorable outcome since the lower capacity networks are scarce on resources. It is important to note that the Loss-Aware policy mimicked the vanilla approach as the burst loss increased, signifying the overwhelming appointment of reliable packets in adapting to the harsh burst loss conditions.
 
Alternatively, Fig. \ref{fig:backlog_burst} clearly shows a stark comparison between the policies and loss scenario in the reduction of the backlogged packets. The Loss-Aware policy for random loss scenario reduced the backlogged packets by up to 50\%, beating all other policies by approximately 30\%. Furthermore, it is clear that the unreliability focused policies resulted in the lowest backlog for the session. In comparison, we notice that the burst loss and the backlogged frequency have a positive correlation, where the maximum reduction of the backlogged packets for the policies is at most 20\%. Much like the transmitted packets, the probability of a burst loss occurrence plays a vital role in the number of retransmissions sent and by extension the number of backlogged packets. Thus, we can conclude that the stress placed on the buffer is a result of the reliable packets which is tightly coupled with the congestion on the session. Whereas, unreliable focused policies did not encounter such a phenomenon regardless if it was experiencing a burst loss.


\subsection{Application-Layer KPIs}

The feasibility of dynamic reliability for real-time applications can be determined by the \gls{aoi}, with comparison across different topologies and policies. If we take a strict approach and consider anything below $10$~ms is real-time \cite{real-time}, then all the reliability policies passed that requirement, which is attractive for real-time applications, as shown in Fig. \ref{fig:aoi_burst}. Utilising the median as an estimate of the runs, the policies in the WLAN and Sub-6~GHz topology with random loss floated around $4-5$~ms with negligible difference, while the \gls{aoi} for \gls{mmwave} was $\approx 2-3$~ms. It is clear that the \gls{aoi} and the network capacity have a negative correlation, as the network capacity decreases, the \gls{aoi} increases. The same correlation is extended to the bursty loss scenarios, where \gls{mmwave} dominated the other topologies. That being said, it is crucial to note that the \gls{aoi} for the reliability policies is often slightly better than or equal to the \gls{aoi} of the vanilla implementation, proving that dynamic reliability reduces the congestion of the session at no cost to the \gls{aoi}.


\vspace{-3mm}
\subsection{Metric}
\vspace{-1.5mm}
\noindent \textbf{Prototype Decoding.}
For each decoding step, we evaluate the node and edge error rate, counting each prediction as an error if either token type, node id, or node/edge type is incorrect.
Using the following metrics, we also compare the decoded graph with the ground truth. (i) Invalid graph rate: we consider a graph invalid if it is cyclic, not connected, or missing necessary connections. (ii) Intersection-over-union (IOU) of the node types: while this metric ignores the graph structure, it checks whether necessary processors and input nodes are decoded somewhere in the graph. For the drum graph, we calculate the IOU for each track and mixing subgraph and average the values. Finally, (iii) we render the ground-truth graph and the estimated prototype graph with default parameters and compare the outputs using multi-scale spectral loss (MSS-default) \cite{engel2020ddsp}. 

\noindent \textbf{Parameter Estimation.} Along with the parameter loss, we evaluate the MSS loss rendered on the oracle prototype with estimated parameters (MSS-oracle) and the fully-decoded graph (MSS-full). 

\noindent \textbf{Listening Test.}  We measured subjective scores with MUltiple Stimuli with Hidden Reference and Anchor (MUSHRA) test \cite{mushra, schoeffler2018webmushra}. 
We asked $8$ graduate students to score the similarity between the reference and the rendered audio with the estimated graph. A total of $48$ sets were scored ($24$ sets for each task and $12$ sets for each \emph{seen} and \emph{unseen} speaker/kits).

\begin{figure*}[t]
    \begin{center}
        \includegraphics[width=2\columnwidth]{figures/pdfs/drum-results.pdf}
        \vskip -2pt
        \vspace{-1mm}
        \caption{An inference result from the proposed method \ding{194} on the drum mixing estimation task (left: ground-truth, right: prediction). 
        First, our model correctly predicted the source instrument types: \texttt{[kick]}, \texttt{[snare]}, \texttt{[hat]}, and \texttt{[ride]}. The ground-truth graph panned \texttt{[kick]} and \texttt{[ride]}, which is also reconstructed by the prediction. However, the prediction failed to estimate (i) the correct multi-band processing of \texttt{[reverb]}, (ii) multiple stages of \texttt{[distortion]} processors, and (iii) modulation of linear filters in the mixing subgraph.
        } 
        \label{fig:drum-est}
    \end{center}
    \vspace{-8.5mm}
\end{figure*}

\vspace{-3mm}
\subsection{Evaluation Results}
\vspace{-1.5mm}

\noindent \textbf{Sanity Check.} 
Table 2 reports the evaluation results. Before training the blind estimation models, we first trained a \ding{192} graph autoencoder by introducing another TokenGT as a graph encoder (we also embedded the parameters for the encoder input). Its evaluation results, e.g., $0$ node error rate, confirm that the graph decoder is powerful enough and the dimension of the latent $z$ is sufficiently large to reconstruct the original graph. Furthermore, its MUSHRA score is comparable to the hidden reference's, agreeing with the objective metrics. Next, we evaluated the \ding{193} graph decoder with latent vectors set to $0$. Its node error rate ($0.502$ and $0.602$) is better than what a random guess would achieve. This is because (i) the probability distribution of the processors is nonuniform, (ii) we sorted the LTI subgraphs, and (iii) the ground-truth intermediate prototype is available to the network, which can be exploited for a better guess.


\noindent \textbf{Performance Analysis.} On the \emph{seen} dry speaker/kit sets, the \ding{194} proposed model  reports the node error rate of $0.215$ and $0.335$ for the singing and drum task, respectively, indicating that the perfect reconstruction of the prototype graph is rare.  
Yet, audio rendered from the estimated graph can be perceptually close to the reference, reporting the MUSHRA score of $76.6_{\pm 4.6}$ and $69.8_{\pm 5.4}$. 
On the \emph{unseen} dry speaker/kit sets, the evaluation results are degraded in most metrics, confirming that the graph estimation from unseen sources is challenging. We note that the majority of the errors come from either (i) the wrong order of processors (see Figure \ref{fig:drum-est} and \ref{fig:singing-est-1}) or (ii) some processors which destroy the signal and make preceding processors harder to notice (see Figure \ref{fig:singing-est-2}). 
Finally, to check the difficulty of the blind estimation, we trained the same model but with \ding{195} dry sources also provided as input by concatenating it with the reference across the channel axis. Indeed, the estimation performance improves by a noticeable margin, indicating that extracting the graph-relevant information solely from the reference is a challenging task.

\noindent \textbf{Blind estimation strategy comparison.} We compared our \ding{194} token-by-token approach with the conventional \ding{196} node-by-node decoding method \cite{li2018learning}. While this model uses the same TokenGT backbone, it estimates the next node type first and then performs an edge prediction task using the transformer outputs. It showed similar or slightly worse performance overall compared to our model \ding{194}, and it had the drawback of having a high invalid graph rate. Finally, we tried a \ding{197} single-stage generation method; we decoded the node/edge parameters along with the categorical types. Since the transformer only has access to the decoded intermediate graph, its parameter estimation performance was much worse than the two-stage approach, resulting in higher MSS-full loss and lower MUSHRA score.


