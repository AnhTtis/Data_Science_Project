\section{Most general Specker-Blatter Theorem}
\label{se:SB}
%\section{Counting restricted topologies: MC-finiteness}
In order to prove  Theorem \ref{th:main} for $\TCMSOL$,
we now state the Specker-Blatter Theorem for $\CMSOL$ with hard-wired constants,
\cite{fischer2022extensions}. This generalizes substantially the original  Specker-Blatter Theorem from 1981,
%\ref{th:specker}, 
but is
still formulated for one sorted relational structures for binary and unary relation symbols.
Note that in \cite{fischer2022extensions} it is also shown that Theorem \ref{th:SB}
does not hold for ternary relations. 

Let $\tau$ be a finite set of binary and unary relation symbols.
Let $Sp_{\phi, r}(n)$ be the number of labeld $\tau$-structures on the set $[r+n]$
where the elements of $[r]$ are hard-wired and which satisfy the formula $\phi$ of $\CMSOL$.

\begin{theorem}[Most general Specker-Blatter Theorem]
\label{th:SB}
\begin{enumerate}[(i)]
\item
$Sp_{\phi,r}(n)$ is MC-finite.
\item
For every $m \in \N^+$ the sequence $Sp_{\phi,r}^m(n) = Sp_{\phi,r}(n) \bmod{m}$
is computable in polynomial time. In fact it is in $\FPT$ (Fixed Paramater Tractable) 
with parameters $\phi, r$ and  $m$.
\end{enumerate}
\end{theorem}
For the technical details and the history of this theorem, the logically inclined reader should consult
\cite{fischer2022extensions}. 

The theorem holds for a fixed number of hard-wired constants. It does not hold for a hard-wired relation.
A relation $R \subseteq [n]^s$ is hard-wired if $R$ has a unique interpretation on $[n]^s$.
We can view the natural order $NO \subseteq [n]^2$ as hard-wired.
The number of equivalence relations (set partitions) on $[n]$ is given by the Bell numbers $B(n)$ which are MC-finite.
They satisfy the hypothesis of the Specker-Blatter Theorem, as an equivalence relation is definable in $\FOL$.

\begin{proposition}
The Specker-Blatter Theorem does not hold in the presence a hard-wired linear order on $[n]$.
\end{proposition}
\begin{proof}
Let $A$ and $B$ be two blocks of a partition of $[n]$.
$A$ and $B$ are {\em crossing} if there are elements
$a_1, a_2 \in A$ and $b_1, b_2 \in B$ such that 
$a_1 < b_1 < a_2 < b_2$ or
$b_1 < a_1 < b_2 < a_2$.
The number $B(n)^{nc}$ of 
non-crossing set partitions on $[n]$ is one of the interpretations of the Catalan numbers $C(n)$, 
\cite{roman2015introduction}, hence $C(n) =B(n)^{nc}$.
Non-crossing set partitions are definable in $\FOL$ in presence of the hard-wired natural order on $[n]$.
But the Catalan numbers are not MC-finite.
\end{proof}

%Our next task is now to show how to reduce the theorem for $\TCMSOL$ to Theorem \ref{th:SB} above.
