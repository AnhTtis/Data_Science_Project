\section{Conclusions}
\label{se:conclu}

We have shown that counting finite topologies $T_{\phi, r}(n)$ on a set of $n$ elements 
subject to restrictions with a fixed finited number $r$ of (hard-wired)
constants expressed by $\phi \in \TCMSOL$ is an MC-finite sequence.
The underlying set $X$ of the topology can be equipped with unary and binary relations, 
but counting now also counts the number of their interpretations.
As we have seen in Section \ref{se:SB}
the Specker-Blatter Theorem does not work for hard-wired relations.

No explicit formulas for $T(n)$ and $T_0(n)$ are known, but we have
$$T(n) = \sum_{k=0}^n S(n, k)\cdot T_0(k),$$ 
where $S(n,k)$ is the Stirling number of the second kind. 
%\cite{butler1973enumeration}.

\begin{problem}
Find better descriptions of $T(n)$ and $T_0(n)$.
\end{problem}

The logic $\TCMSOL$ can express many topological properties.  Here are some possibly challenging
test problems.

In analogy to counting various set partitions
on $[n]$ as described in \cite{FFMR},
one can look at topological set partitions of finite topological spaces on $[n]$ 
such that each block is a connected component, an open, or a closed set.
Note that the topology in a topological set partition is not hard-wired.
We count the set partitions of $X$ and the topologies separately. 
Given $[n]$ the Bell numbers $B(n)$ count the partitions of $[n]$ and for each such partition we count the number of topologies
such that the blocks are connected, open or closed.
The Stirling numbers of the second kind $S(n,k)$ count the partitions of $[n]$ into $k$ blocks, and again we can require
that the blocks are connected, open or closed.

Similarlily, we can look at topological spaces on $[r+n]$ such that the hard-wired constants
of $[r]$ are all in different blocks. Like with the Bell and Stirling numbers, 
the number of non-empty blocks may be arbitrary or fixed to $k$ blocks.
If the topology is assumed to be discrete, we get the
resricted Bell numbers $B_r(n)$ and Stirling numbers of the second kind $S_r(n,k)$ from \cite{broder1984r}.
Otherwise, these give different topological versions of $B_r(n)$ and $S_r(n,k)$.

\begin{problem}
What can we say beyond that the number of such topological set partitions
form an MC-finite sequence?
Are some of them C-finite?
\end{problem}





%For every formula $\phi$ of $\TFOL$ and every positive integer $m$, the sequence
%$T_{\phi, r}^m(n)$ is ultimately periodic modulo $m$. In other words $T_{\phi, r}^m(n)$ is MC-finite.

