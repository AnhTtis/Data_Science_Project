\section{Normal sequences}
\label{se:normal}
Let $s(n)$ be an integer sequence, and $b \in \N^+$.
The sequence $s^b(n) = s(n) \bmod{b}$ 
is normal, if, when we chunk it into substrings of length 
$\ell \ge 1$, then each of the $b^\ell$ possible strings of $[b]^\ell$ appear in $s^b(n)$ with equal limiting frequency. 
It is {\em absolutely normal} if it is normal for every $b$.
The sequence $s^b(n) = s(n) \bmod{b}$ can be viewed as a real number $r_b$ written in base $b$.
A classical theorem  from 1922 by E. Borel says that almost all reals are absolutely normal, 
\cite{everest2003recurrence}.
The proposition below shows that MC-finite integer sequences are very rare.

Let $PR_b$ be the set of integer sequences $s^b(n)$ 
with $s^b(n) = s(n) \bmod{b}$ for some integer sequence $s(n)$.
$PR_b$ is the projection of all integer sequences to sequences over $\Z_b$.
We think of $PR_b$ as a set of reals with the usual topology and its Lebesgue measure.
Let $UP_b \subseteq PR_b$ be the set of sequences $s^b(n) \in PR_b$ which are ultimately periodic.

\begin{proposition}
\label{pr:normal}
\begin{enumerate}[(i)]
\item
Almost all reals are absolutely normal. 
\item
$s(n)$ is MC-finite iff for every $b \in \N^+$ the sequence $s^b(n)$ is ultimately periodic.
\item
If $s^b(n)$ is normal for some $b$, then $s(n)$ is not MC-finite. 
\item
$UP_b \subseteq PR_b$ has measure $0$.
\end{enumerate}
\end{proposition}
Proving that a specific sequence is normal is usually very difficult.
%wikipedia  https://en.wikipedia.org/wiki/Normal_number
It has been an elusive goal to prove the normalcy of numbers that are not artificially constructed. 
While $\sqrt{2}$, $\pi$, $ln(2)$ and $e$ are strongly conjectured to be normal, 
it is still not known whether they are normal or not. 
It has not even been proven that all digits actually occur infinitely many times in the 
decimal expansions of those constants. 
%(for example, in the case of π, the popular claim "every string of numbers eventually occurs in π" is not known to be true).[15] 
It has also been conjectured that every irrational algebraic number is absolutely normal, 
and no counterexamples are known in any base. However, no irrational algebraic number has been proven to be normal in any base. 

\ifskip\else
Here is a challenge:
\begin{conjecture}
The binary sequence $\beta(n) = a(n) \bmod 2$ from Theorem \ref{thm:main} is normal with $b=2$. 
\end{conjecture}
\fi %skip
