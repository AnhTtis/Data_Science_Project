\section{Background}
\subsection{C-finite and MC-finite sequences of integers}
\label{se:mcfinite}
A sequence of integers $s(n)$ is {\em C-finite}\footnote{
These are also called constant-recursive sequences
or  linear-recursive sequences in the literature.
}
if there are constants $p, q \in \N$ and $c_i \in \Z, 0 \leq i \leq p-1$ such that for all $n \geq q$ the
linear recurrence relation
$$
s(n+p) = \sum_{i=0}^{p-1} c_i s(n+i)
$$
holds for $s(n)$.
C-finite sequences have limited growth, see e.g. \cite{everest2003recurrence,kauers2011concrete}:
\begin{proposition}
\label{prop:c-finite}
Let $s_n$ be a C-finite sequence of integers. Then there exists $c \in \N^+$ such that for all $n \in \N$,
$a_n \leq 2^{cn}$.
\end{proposition}
Actually, a lot more can be said, see \cite{flajolet2009analytic}, but we do not need it for our purposes.

To prove that a sequence $s(n)$ of integers is not C-finite, we can use Proposition \ref{prop:c-finite}.
To prove that a sequence $s(n)$ of integers is C-finite, there are several methods:
One can try to find an explicit recurrence relation, one can exhibit a rational generating function,
or one can use a method based on model theory as described in 
\cite{fischer2008linear,fischer2011application}.
%The last method will be briefly discussed in
%Section \ref{se:fm} and further explained in Appendix \ref{se:c-finite}. It is referred to as method FM.

A sequence of integers $s(n)$ is {modular C-finite}, abbreviated as {\em MC-finite}, 
if for every $m \in \N$ there are constants $p_m, q_m \in \N^+$ such that
for every  $n \geq q_m$ there is a linear recurrence relation
$$
s(n+p_m) \equiv \sum_{i=0}^{p_m-1} c_{i,m}  s(n+i) \bmod{m}
$$
with constant coefficients $c_{i,m} \in \Z$. 
%and $n \geq n_0$.
Note that the coefficients $c_{i,m}$ and both $p_m$ and $q_m$  generally do depend on $m$.

We denote by $s^m(n)$ the sequence $s(n) \bmod{m}$. 
\begin{proposition}
The sequence $s(n)$ is MC-finite iff $s^m(n)$ is ultimately periodic for every $m$.
\end{proposition}
\begin{proof}
MC-finiteness clearly implies periodicity. The converse is from \cite{reeds1985shift}.
\end{proof}

Clearly, if a sequence $s(n)$ is C-finite it is also MC-finite with $r_m=r$ and $c_{i,m}=c_i$ for all $m$.
The converse is not true, there are uncountably many MC-finite sequences, but only
countably many C-finite sequences with integer coefficients, see Proposition \ref{pr:many} below.
%Here are some typical examples:
\begin{examples}\ 
\label{ex:mc}
\begin{enumerate}[(i)]
\item
The Fibonacci sequence is C-finite.
\item
If $s(n)$ is C-finite it has at most simple exponential growth, by Proposition \ref{prop:c-finite}.
\item
The Bell numbers $B(n)$ are {\em not C-finite}, but are {\em MC-finite}.
\item
Let $f(n)$ be any integer sequence. The sequence $s_1(n)=2\cdot f(n)$ is ultimately periodic modulo $2$,
but not necessarily MC-finite.
\item
Let $g(n)$ be any integer sequence.
%grow arbitrarily fast. 
The sequence  $s_2(n) = n!\cdot g(n)$ is MC-finite.
\label{many-mc}
%We conclude that there are uncountably many monotonously increasing sequences which are MC-finite.
\item
The sequence $s_3(n)= \frac{1}{2} {2n \choose n}$ is not MC-finite: 
$s_3(n)$ is odd iff $n$ is a power of $2$, and otherwise it is even (Lucas, 1878).
A proof may be found in \cite[Exercise 5.61]{graham1989concrete} or in \cite{specker1990application}.
\item
The Catalan numbers $C(n) = \frac{1}{n+1}{2n \choose n}$ are not MC-finite,
since $C(n)$ is odd iff $n$ is a Mersenne number, i.e.,  $n = 2^m-1$ for some $m$,
see \cite[Chapter 13]{koshy2008catalan}.
%For a recent short proof of this, see \cite{KoshySalmassi}, and for an equivalent result from 1973, 
%see \cite[Theorem2]{alter1973binary}.
\ifskip\else
\item
\label{many-nonmc}
Let $p$ be a prime and $f(n)$ be monotone increasing.
The sequence 
$$
s(n) = \begin{cases}
p^{f(n)} & n \neq p^{f(n)} \\
p^{f(n)}+1 & n = p^{f(n)}
\end{cases}
$$
is monotone increasing but  not ultimately periodic modulo $p$, hence not MC-finite.
%We conclude that there are uncountably many monotonously increasing sequences which are not MC-finite.
\fi %skip
\item
\label{many-nonmc}
Let $p$ be a prime and $f(n)$ be monotone increasing.
The sequence $s(n)=p\cdot f(n)+z(n)$, where $z(n)$ is defined to equal $1$ if $n$ is a power of $p$ 
and to equal $0$ for any other $n$,
is monotone increasing but  not ultimately periodic modulo $p$, hence not MC-finite.
%We conclude that there are uncountably many monotonously increasing sequences which are not MC-finite.
\end{enumerate}
\end{examples}

\begin{proposition}
\label{pr:many}
\begin{enumerate}[(i)]
\item
There are uncountably many monotone increasing sequences which are MC-finite, and uncountably many
which are not MC-finite.
\item
Almost all bounded integer sequences are not MC-finite.
\end{enumerate}
\end{proposition}
\begin{proof}
(i) follows from
Examples \ref{ex:mc}
(\ref{many-mc}) and (\ref{many-nonmc}).
(ii) is shown in
Proposition \ref{pr:normal} in  Appendix \ref{se:normal}
\end{proof}

Although we are mostly interested in MC-finite sequences $s(n)$, it would be natural to check in each example
whether the sequence $s(n)$ is also C-finite. In most concrete examples the answer is negative, 
which can be seen by a growth argument. Proposition \ref{pr:C-fin} in Section \ref{se:models} gives a model theoretic
tool for finite topological structures.
However, we will not elaborate this further.  

\subsection{Counting finite topologies}
\label{se:ftop}
Here we follow the presentation from \cite{may2003finite}.
Let $\mathcal{T}=(X, \mathcal{U})$ be a finite topological
space on the finite labeled set $X$ and let $\cU$ be the family of open sets in $X$.
Counting topologies on $X$ is defined as counting the number of distinct families $\cU$ of subsets of $X$.
By Alexandroff's Theorem \ref{th:0}(i)  and \ref{th:1931} 
this is equivalent to counting the number of labeled finite quasi-orders.
Let $T(n)$ and $T_0(n)$ be the number of topologies  and $T_0$-topologies respectively on
a the set $[n] = \{1, \ldots, n \}$.
Recall that a topology on $[n]$ is $T_0$
if for all $a, b \in [n]$, there is some open set
containing one but not both of them. 
No explicit formulas for $T(n)$ and $T_0(n)$ are known. 

The following is known:
\begin{theorem}
\label{th:0}
\ 
\begin{enumerate}[(i)]
\item
$T(n) = Q(n)$, where $Q(n)$ is the number of pre-orders on $[n]$, \cite{may2003finite}.
It is $A000798$ in the Online Encyclopedia of Integer Sequences,
\cite{oeis}.
\item
$T_0(n) = P(n)$, where $P(n)$ is the number of partial orders on $[n]$, \cite{may2003finite}.
It is $A001035$ in the Online Encyclopedia of Integer Sequences.
\item
$Q(n) = \sum_{k=0}^n S(n, k)\cdot P(k)$, where $S(n,k)$ is the Stirling number of the second kind, \cite{butler1973enumeration}.
\item
$B(n) \leq P(n) \leq Q(n)$, where $B(n)$ are the Bell numbers, which count the number of equivalence relations
on $[n]$. Furthermore, see \cite{de1981asymptotic,berend2010improved},
$$
\left(\frac{n}{e \ln n}\right)^n \leq B(n) \leq \left(\frac{n}{e^{1-\epsilon} \ln n}\right)^n,
$$
\item
The logarithm with base $2$ of both $T(n)$ and $T_0(n)$
goes asymptotically to $\frac{n^2}{4}$
as $n$ goes to infinity,
\cite{kleitman1970number}. 
\end{enumerate}
\end{theorem}

\begin{theorem}
\begin{enumerate}[(i)]
\item
$T(n)$ and $T_0(n)$ are not C-finite.
\item
$T(n)$ and $T_0(n)$ are MC-finite.
\end{enumerate}
\end{theorem}
\begin{proof}
(i) follows from Theorem \ref{th:0}(iv).
\\
(ii) follows from Theorem \ref{th:0}(i) and (ii) and the Specker-Blatter Theorem \ref{th:SB}.
\end{proof}
