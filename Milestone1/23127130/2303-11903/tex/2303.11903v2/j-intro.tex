\section{Introduction}
In the last decade finite topologies have received renewed attention
due to their role in image analysis and data science.
There is a vast literature testifying to this.
We just mention two references as typical examples, \cite{kovalevsky1989finite,chazal2021introduction}.
For mathematical applications of finite topologies, see \cite{barmak2011algebraic}.
Finite metric spaces are studied in \cite{bourgain1986type,linial2003finite}.
The model theory of topological spaces was studied in the late 1970ies, see 
\cite{makowskytopological,flum2006topological,makowsky1981topological,bk:BFxv}.

In \cite{broder1984r} A. Broder introduced the restricted r-Stirling numbers and r-Bell numbers.
They have found various applications in enumerative combinatorics, e.g. see \cite{benyi2019restricted}.
Inspired by this we study in this paper the number of finite topologies on a finite set subject
to various restrictions.


Assume you are given the set $[r+n] = \{1, 2, 3, \ldots , r+m\}$ and you want to count the number of
finite topolgies on the set $[r+n]$ such that
\begin{enumerate}[(i)]
\item Each of the elements in $[r]$ are in a different component, and
\item Each connected component has odd size.
\end{enumerate}
Or think of variations thereof,
where the topologies on the elements of $[r]$ satisfy some prescribed topological configuration, such as
all the singletons in $[r]$ being closed sets, or being pairwise separable by an open set.
In the applications from \cite{kovalevsky1989finite,chazal2021introduction} counting finite topologies
with such or similar restrictions might be interesting.

Counting finite topologies is a difficult problem.
Even for the case without restrictions no explicit formula is known.
There are some asymptotic results, but the best results known so far are congruences modulo a fixed integer $m$.
Computing $T(n)$ for $n=1,2,3,4$ can be done by hand.
There are two papers giving values for $T(5)$: In \cite{EHL} it says that $T(5)= 6942$.
In \cite{Shaafat} it says that $T(5)= 7181$.
E. Specker in 1980 got interested in modular counting of finite topologies
in order to prove which one of the these claims must be false.

A sequence $s(n)$ of integers is {\em C-finite}, 
if the sequence satisfies a linear recurrence relation with constant coefficients.
A sequence $s(n)$ of integers is {\em MC-finite, modularily C-finite}, if for every integer $m$ the sequence
$s^m(n) =s(n) \pmod{m}$ is an ultimately periodic sequence of positive integers. 
E. Specker showed in \cite{ar:Specker88,specker2011application} the following theorem:
\begin{theorem}
\begin{enumerate}[(i)]
\item
The number of finite topologies $T(n)$ on $[n]$ is not C-finite, but it is MC-finite.
\item
For every $m \in \N^+$ there is a polynomial time algorithm which computes $T(n)$ modulo $m$.
\item
$T(n) = 2 \mod{5}$, hence $T(5) \neq 7181$.
\end{enumerate}
\end{theorem}
The proof of his theorem uses both logic and advanced combinatorics.

The purpose of this paper is to show similar results for the number of topologies with restrictions as suggested above.
In the presence of the restrictions we have in mind, Specker's method cannot be applied directly, 
but it can be applied using our recent results together with a suitable definition of a logic
from \cite{fischer2022extensions}.

We will use logic to make the framework of the restrictions precise. 
Let $\mathcal{T}=(X, \mathcal{U})$ be a finite topological
space on the finite set $X$ and $\cU$ the family of open sets in $X$.
We associate with $\cT$ a two sorted first order structure $\cT' = (X, \cU, E)$
where $E \subseteq X \times \cU$ and $x E U$ says that $x$ is an element of $U$. 
If $X = [r+n]$ we use constant symbols $a_1, \ldots , a_r$ which have a fixed interpretation:
$a_i$ is interpreted by $i \in [r]$. We say that the constant symbols $a_i$ are {\em hard-wired}.
The topological restrictions are now described by first order formulas $\phi(a_1, \ldots, a_r)$
over the structure $([r+n], \cU, \in, a_1, \ldots, a_r)$.
We denote by $T_{\phi, r}(n)$ the number of topologies on the set $[r+n]$ which satisfy $\phi(a_1, \ldots, a_r)$.
For a positive integer $m$
we denote by $T_{\phi, r}^m(n)$ the  sequence $T_{\phi, r}(n)$ modulo $m$.

\subsection{Main result} \ \\
Our main result is stated here for topological first order logic $\TFOL$:
\begin{theorem}
\label{th:main}
\begin{enumerate}[(i)]
\item
For every formula $\phi$ of $\TFOL$ and every positive integer $m$, the sequence
$T_{\phi, r}^m(n)$ is ultimately periodic modulo $m$. In other words $T_{\phi, r}^m(n)$ is MC-finite.
\item
Given $\phi$ and $m$, the sequence $T_{\phi, r}^m(n)$ is Fixed Parameter Tractable ($\bFPT$) where the parameters depend on 
$\phi, r$ and $m$.
\end{enumerate}
\end{theorem}

The proof uses recent results on extensions of Specker's method due to the authors,
\cite{fischer2022extensions}. It also uses model theoretic methods as described in \cite{bk:BFxv}.
The same method was applied to prove congruences for restricted Bell and Stirling numbers in 
\cite{FFMR}.
%\cite{fischer2023mc}.
%\cite{FischerMaRa2023}.

One of our main contributions lies in identifying the logic $\TCMSOL$, 
a topological version of Monadic Second Order Logic with modular counting.
This allows us to prove 
Theorem \ref{th:main-1} in Section \ref{se:proofs}, which is like Theorem \ref{th:main} but stated for $\TCMSOL$
instead of $\TFOL$.

