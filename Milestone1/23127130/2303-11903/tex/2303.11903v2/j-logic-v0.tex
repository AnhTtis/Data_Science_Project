\section{Topologies as relational structures}
\label{se:models}
\subsection{The logics $\TCMSOL$ and $\CMSOL$}

Let $\mathcal{T}=(X, \mathcal{U})$ be a finite topological
space on the finite set $X$ and let $\cU$ be the family of open sets in $X$.
We associate with a finite topological space $(X, \cU)$ a two sorted relational structure $\cT = (X, \cU, E)$,
where $E \subseteq X \times \cU$ is a binary relation and $E(x,U)$ says that $x$ is an element of $U$. 
We also require that it satisfies the extensionality axiom
$ U=V \leftrightarrow \forall x (E(x, U)  \leftrightarrow E(x,V)) $.
We denote by $TCMSOL$ the monadic second order logic for structures of this form possibly augmented by constant symbols.
We allow quantification over {\em subsets of $X$}, but only quantification over {\em elements of $\cU$}.
Furthermore, we have a modular counting quantifier $C_{m,a}x \phi(x)$ which says that modulo $m$ there are $a$
elements satisfying $\phi(x)$. $\TFOL$ is the logic without second order quantification and without modular
counting. Similarly, $\CMSOL$ is defined as $\TCMSOL$ for one-sorted structures with one binary relation.

For {\em finite structures} of the form $\cT$ the following are $\TCMSOL$-definable.
\begin{enumerate}[(i)]
\item
$\cU$ is a topology for the finite set $A$: 
(i) $\emptyset  \in \cU$, $A \in \cU$.
(ii) $\cU$ is closed under unions.
(iii) $\cU$ is closed under intersections.
\item
$\cU$ is $T_0$: $\forall a,b \in A \exists U \in \cU 
(E(a,U) \wedge \neg E(b,U) 
\vee
(\neg E(a,U) \wedge  E(b,U)) 
$.
\item
$\cU$ is $T_1$: $\forall a \in A (A - {a} \in \cU)$.
\item
$X$ is connected: There are no two non-empty disjoint open sets $U_1, U_2$ with $U_1 \cup U_2 =X$.
\item
The $\TFOL$-formula  $\phi_{U_x}(x,U)$  says that $U$ is the smallest open set containing $x$: 
\\
%$U_x \in \cU \wedge E(x,U_x) \wedge (\forall U \in \cU (E(x,U) \rightarrow U_x \subseteq U)$.
%---
$(U\in\mathcal{U})\wedge (E(x,U))\wedge (\forall V\in\mathcal{U}(E(x,V)\to V\subseteq U))$
%----
\item
A typical formula which is in $\TCMSOL$ would be: There is a set of points of even cardinality
which is not an open set.
\end{enumerate}

\subsection{Hard-wired constant symbols}
Let $\bar{a}=(a_1, \ldots, a_k)$ be $k$ constant symbols.
For each of them there are $n$ possible interpretations in the set $[n]$.
However, we say that $(a_1, \ldots, a_r)$, for $r \leq k$ are
{\em hard-wired} on $[n]$, if $a_i$ is interpreted by $i \in [n]$.
In the presence of constant symbols (hard-wired or not) $a_1, \ldots, a_k$ we can say:
\begin{enumerate}[(i)]
\item
%$\forall U \bigwedge_i^r E(a_i, U)$, i.e., they form a minimal non-empty open set.
%---
$\{a_1, \ldots, a_k\}$ is a minimal non-empty open set:
$$
\exists U\in\mathcal{U}((\bigwedge_{i=1}^r E(a_i,U))
\wedge 
(\forall x(\bigwedge_{i=1}^r (x\neq a_i)\to\neg E(x,U)))
\wedge
(\forall V\in\mathcal{U}((V\subseteq U)\to((V=U)\vee(V=\emptyset)))))
$$
%(maybe it can be "abbreviated" a bit, about half of it just means $U=\{a_1,\ldots,a_r\}$)
%---
\item
There are pairwise disjoint open sets $ U_1, \ldots , U_r$ such that $a_i$ in $U_i$.
\item
The elements denoted by $a_i$ are all in different connected components.
\end{enumerate}
In analogy to Broder's $r$-Stirling numbers, we also count finite topologies restricted by $\TCMSOL$-formulas
with hard-wired constant symbols.

\subsection{Topological model theory}
In his retiring presidential address presented at the annual meeting of the Association
for Symbolic Logic in Dallas, January 1973
Abraham Robinson suggested, among other topics, to develop a model theory for topological structures,
\cite{robinson1973metamathematical}.
This led to several approaches described in
\cite{ziegler1976language,flum2006topological,bk:BFxv,makowsky1981topological,makowskytopological}.
M. Ziegler introduced the logic $L_t$, which is a fragment of $\TFOL$ with the additional property that
it is invariant in the following sense:
Let
$\phi$ of $L_t$ a formula and $\mathcal{T}=(X, \mathcal{U})$ be a (not necessarily finite) 
topological structure $\mathcal{T}=(X, \mathcal{U})$.
Then 
$$\mathcal{T}=(X, \mathcal{U}) \models \phi \text{ iff  } \mathcal{T}=(X, \mathcal{B})$$
for every basis $\mathcal{B}$ of $\mathcal{U}$.
In fact in \cite{ziegler1976language,flum2006topological,bk:BFxv} $X$ can be replaced by arbitrary first order
structures. $L_t$ now shares most model theoretic characterstics of first order logic, like compactness, L\"owenheim-Skolem
theorems, preservation theorems, etc.  However, if we restrict the topological structures for $L_t$ to be finite,
this is not true anymore.
A topological structure $\mathcal{T}=(X, \mathcal{U})$ is an {\em open substructure} of
$\mathcal{T}'=(Y, \mathcal{V})$ if $X \subseteq Y$, $\mathcal{U} = \{A \subset X: A = A' \cap X, A' \in \mathcal{V})$,
and $X \in \mathcal{V})$. We write $\mathcal{T} \subseteq_o \mathcal{T'}$.
A formula $\phi \in L_t$ is {\em preserved under open extensions}
if for every  pair of topological structures $\mathcal{T} \subseteq_o \mathcal{T}'$ we have
$$
\mathcal{T} \models \phi \text{  implies  } \mathcal{T}' \models \phi .
$$
In \cite{flum2006topological} there is also a syntactical characterization of the formulas preserved under open extensions.
However, it fails if restricted to finite structures.
Nevertheless, we can use this to show (here without proof):

\begin{proposition}
\label{pr:C-fin}
Let $\phi \in L_t$ with $r$ constant symbols (hard-wired or not) which  has arbitrarily large finite models and 
is preserved under open extensions.
Then $T_{\phi,r}(n)$ is not C-finite.
\end{proposition}

