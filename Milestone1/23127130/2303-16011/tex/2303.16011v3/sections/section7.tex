\section{Research of System Physical Characteristics}
\label{phys-research}

\subsection{Research of First PT Terms in Vacuum Energy Density}

In this subsection we are going to calculate the first PT terms for vacuum energy density. Formerly we have obtained the general formulas (\ref{G_1[0]-final}) and (\ref{G_2[0]-rot-inv}), and now we are going to substitute different propagators into these formulas: virton propagator \cite{efimov1985problems}, which is purely non-local, and Euclidean Klein--Gordon propagator (with sharp cut-off), which admits a local limit. Then we will consider a local theory as a certain limit of a nonlocal one, being its regularization. However, let us note that the nonlocal theory is of important independent interest.

\subsubsection{Nonlocal Case}

We start from a typical nonlocal QFT propagator, which is the virton propagator:
\begin{equation}
    G(x)=G(0)\,e^{-\mu^{2}x^{2}},
\end{equation}
where $\mu$ is an ultraviolet cut-off parameter and $G(0)>0$ is a some constant. Let us calculate numerically the following integral:
\begin{equation}
    \psi(d,\alpha)=
    \int_{0}^{\infty}dt\,t^{d-1}
    \left\{\left(1-e^{-2t^2}\right)^{\alpha+\frac{1}{2}}{}_{2}F_{1}\left(\frac{\alpha+1}{2},\frac{\alpha+1}{2};
    \frac{1}{2};e^{-2t^2}\right)-1\right\}.
\end{equation}

\begin{figure}
\begin{centering}
\includegraphics[width=8cm,height=6cm]{VZ2}
\par\end{centering}
\caption{Results of numerical calculation of the integral $\psi(d,\alpha)$.}
\label{fig:psi-graph}
\end{figure}
The plots for $\psi(d,\alpha)$ depending on $\alpha$ at different $d$ are presented on the Figure \ref{fig:psi-graph}. Let us note that the integrand above is always positive.

As a result, we arrive at the following expression for the vacuum energy density:
\begin{equation}
    w_{vac}=b_{1}gG\left(0\right)^{\frac{\alpha}{2}}-b_{2}\frac{g^{2}G\left(0\right)^{\alpha}}
    {\mu^{d}}+O(g^{3}),
    \label{vac_virton_2}
\end{equation}
where:
\begin{equation*}
    b_{1}=\frac{2^{\frac{\alpha}{2}}
    \varGamma\left(\frac{\alpha+1}{2}\right)}
    {\pi^{1/2}},\quad 
    b_{2}=\frac{2^{\alpha-1}
    \varGamma\left(\frac{\alpha+1}{2}\right)^2
    \pi^{d/2-1}}{\varGamma
    \left(\frac{d}{2}+1\right)}\,\psi(d,\alpha).
\end{equation*}
In all the following calculations, we will see the qualitatively similar behaviour of $w_{vac}$ at small $g$: positive coefficient in linear term and the negative one in quadratic term. Therefore, all the results obtained in the paper are consistent with each other.

\subsubsection{Local Case}

Now let us consider the local QFT and calculate the first terms of vacuum energy density for the Klein--Gordon propagator. We consider the usual cut-off regularization of Gaussian theory Green function with Heaviside step function, i.e. sharp cut-off:
\begin{equation*}
    G(x)=\int_{\mathbb{R}^{d}}
    \frac{dk}{(2\pi)^{d}}
    \frac{e^{ikx}}{k^{2}+m^{2}}\,
    \theta\left(\mu-\left|k\right|\right),
\end{equation*}
where $\mu$ is again the UV cut-off parameter. Since we want to obtain a local theory as a limiting case, it is appropriate to consider the cases of $d=2,3$. Taking into account, that $\mu\gg m$, for $G(0)$ for Klein--Gordon propagator in $d$ dimensions one have:
\begin{equation}
    G(0)=\int_{\mathbb{R}^{d}}
    \frac{dk}{(2\pi)^{d}}
    \frac{\theta\left(\mu-\left|k\right|\right)}
    {k^{2}+m^{2}}\simeq
    \begin{cases}
    \frac{1}{4\pi}\ln\left(\frac{\mu^{2}}{m^{2}}\right), & d=2;\\
    \frac{\mu}{2\pi^{2}}, & d=3.
    \end{cases}
    \label{G(0)-KG}
\end{equation}
It will be convenient not to substitute $G(0)$ explicitly in the obtained formulas.

To calculate the second order perturbation theory, let us note that the integrand depends on the ratio $\frac{G(x)}{G(0)}$, where $G(0)$ is an extremely large. So one can find the value of the integral in the dominating order in $\mu$ by expanding the integrand in powers of $\frac{G(x)}{G(0)}$. So, in principal order in $\mu$:
\begin{equation*} 
    \mathcal{G}_{2}[0]\simeq 
    2^{\alpha-1}\alpha^{2}
    \varGamma\left(\frac{\alpha+1}{2}\right)^2
    g^{2}VG(0)^{\alpha}
    \int_{\mathbb{R}^{d}}dx\,
    \left(\frac{G(x)}{G(0)}\right)^{2},
\end{equation*} 
where we use the analyticity of ${ }_2F_1$ in zero argument and take the values of derivatives from (\ref{2F1-def}). And we can directly calculate the remaining integral:
\begin{equation}  
    \int_{\mathbb{R}^{d}}dx\,G^{2}(x)=
    \int_{\mathbb{R}^{d}}\frac{dk}{\left(2\pi\right)^{d}}
    \frac{\theta\left(\mu-\left|k\right|\right)}{\left(k^{2}+m^{2}\right)^{2}}=
    \begin{cases}
    \frac{1}{2\pi}\int_{0}^{\infty}dk\,
    \frac{k}{\left(k^{2}+m^{2}\right)^{2}}=
    \frac{1}{4m^{2}\pi}, & d=2;\\
    \frac{1}{2\pi^{2}}\int_{0}^{\infty}dk\,
    \frac{k^{2}}{\left(k^{2}+m^{2}\right)^{2}}=
    \frac{1}{8m\pi}, & d=3.
    \end{cases}
    \label{another_intermediate_expression_4}
\end{equation} 
In the second equality of the expression (\ref{another_intermediate_expression_4}), we removed the UV regulator, since this is possible for the corresponding values of $d$. Hence, we receive:
\begin{equation}
    \mathcal{G}_{2}[0]\simeq
    \begin{cases}
    \frac{2^{\alpha-3}\alpha^{2}}{m^{2}\pi}\varGamma\left(\frac{\alpha+1}{2}\right)^2g^{2}VG(0)^{\alpha-2}, & d=2;\\
    \frac{2^{\alpha-4}\alpha^{2}}{m\pi}\varGamma\left(\frac{\alpha+1}{2}\right)^2g^{2}VG(0)^{\alpha-2}, & d=3.
    \end{cases}
\end{equation} 

Putting it all together, for the first PT terms for the vacuum energy density $w_{vac}$ we obtain the expression, that was already mentioned in the section \ref{summary}:
\begin{equation*}
\begin{split}
    & w_{vac}\,m^{-d}=
    a_{1}\frac{gG(0)^{\alpha/2}}{m^{d}}+a_{2}\left(\frac{gG(0)^{\alpha/2}}{m^{d}}\right)^{2}
    \left(\frac{m^{d-2}}{G(0)}\right)^{2}+O(g^{3}) \\
    & a_{1}=2^{\frac{\alpha+1}{2}}
    \varGamma\left(\frac{\alpha+1}{2}\right),\quad a_{2}=\frac{2^{\alpha-1}}{\pi}
    \varGamma\left(\frac{\alpha+1}{2}\right)^{2}
    \begin{cases}
    \frac{1}{4}, & d=2;\\
    \frac{1}{8}, & d=3.
    \end{cases}   
\end{split}
\end{equation*}
Here we have artificially separated some combination of parameters. In the subsection \ref{sect:local limit} it will turn out that a necessary condition for the non-triviality of the local limit is to keep constant exactly these combinations. However, it should be noted that obtaining local theories, as limiting cases of nonlocal ones, generally requires a deeper research.

\subsection{Research of Second-Degree Legendre Polynomial Approximation Formula}

Let us start from the general formula (\ref{vac-en-N=1}) for second-degree Legendre polynomial approximation (in the thermodynamic limit, in which the vacuum energy density is of greatest interest):
\begin{equation} 
\begin{split}
    &w_{vac,N=1}=
    \frac{15\alpha/4}{(\alpha +1) (\alpha +3)}
    \left\{\varGamma\left(\frac{\alpha+1}{2}\right)
    \frac{2^{\alpha/2+2}}
    {\sqrt{\pi}} \frac{1+2\alpha}{5}\,gG(0)^{\alpha/2}+
    \sum\limits_{n=2}^{\infty}
    \frac{(-1)^{n-1}}{n}\,\right.\\ 
    &\times\left.2^{n\alpha/2}\frac{\varGamma
    \left(\frac{n(\alpha+1)}{2}\right)}
    {\varGamma\left(\frac{3n}{2}\right)}
    \left[gG(0)^{\alpha/2}\right]^{n}
    \left\{\prod\limits_{a=1}^{n-1}
    \int_{\mathbb{R}^d}dy_{a}\,
    \frac{G\left(y_{a}\right)}
    {G(0)}\right\}\frac{G\left(\sum
    \limits_{a=1}^{n-1}y_{a}\right)}{G(0)}\right\}.
    \label{another_intermediate_expression_5}
\end{split} 
\end{equation} 
In the following, we will consider only the terms with $n\geq2$, since the case $n=1$ is already simple enough. Now we are going to substitute different propagators into this formula.
\begin{enumerate}
\item \textbf{Nonlocal Case}. For the virton propagator the integrals in the expression (\ref{another_intermediate_expression_5}) are Gaussian, therefore, can be calculated explicitly:
\begin{equation*} 
    \left\{\prod\limits_{a=1}^{n-1}
    \int_{\mathbb{R}^{d}}dy_{a}\,
    \frac{G\left(y_{a}\right)}
    {G(0)}\right\}\frac{G\left(\sum
    \limits_{a=1}^{n-1}y_{a}\right)}{G(0)}=
    \frac{\left(2\pi\right)^{\frac{n-1}{2}d}}
    {\sqrt{n}\mu^{d(n-1)}}.
\end{equation*} 
Thus we arrive at the following expression:
\begin{equation}
    w_{vac,N=1}(z,\mu)=
    \frac{15\alpha/4}{(\alpha+1)(\alpha+3)}
    \frac{\mu^d}{(2\pi)^{d/2}}
    h\left(\left(2\pi\right)^{d/2}
    2^{\alpha/2}z\right),
    \label{w-wac-N=1-Virton}
\end{equation}
for the entire function:
\begin{equation}
    h(z)=\frac{4\alpha-3}{5}
    \frac{\varGamma\left(\frac{(\alpha+1)}{2}\right)}{\varGamma\left(3/2\right)} z +\sum\limits_{n=1}^{\infty}
    \frac{(-1)^{n-1}}{n^{3/2}}z^{n}
    \frac{\varGamma\left(\frac{n(\alpha+1)}{2}\right)}{\varGamma\left(3n/2\right)},
\end{equation}
and the parameter:
\begin{equation}
    z=gG(0)^{\alpha/2}/\mu^d.
    \label{z_vir}
\end{equation} 
The summation over $n$ indeed starts from one, but it is convenient to separate the linear term into two parts.
\item \textbf{Local Case}. Again we restrict ourselves to the cases $d=2,3$. We have already calculated $G(0)$ for Klein--Gordon propagator. For integrals, that we have for $n\geq2$, the following chain of equalities is true:
\begin{equation*}
    \left\{\prod\limits_{a=1}^{n-1}
    \int_{\mathbb{R}^{d}}dy_{a}\,
    \frac{G\left(y_{a}\right)}{G(0)}\right\} 
    \frac{G\left(\sum\limits_{a=1}^{n-1}y_{a}\right)}
    {G(0)}=\int_{\mathbb{R}^{d}}
    \frac{dk}{\left(2\pi\right)^{d}}
    \frac{1}{\left(k^{2}+m^{2}\right)^{n}}  =\frac{\left(m^{2}\right)^{\frac{d}{2}-n}
    \varGamma\left(n-\frac{d}{2}\right)}
    {\left(4\pi\right)^{\frac{d}{2}}\varGamma(n)}.
\end{equation*} 
We removed the UV regulator again. This formula is valid for $d=2,3$. It is convenient to introduce the following dimensionless parameters:
\begin{equation}
    z=\frac{gG(0)^{\alpha/2}}{m^{d}}=\text{const},
    \quad\xi=\frac{m^{d-2}}{G(0)}=\text{const}.
    \label{z-zeta-def-KG}
\end{equation}
In contrast to the nonlocal case, in the local one we define the parameter $z$ in terms of $m$. In the following subsection we will explore the way to scale parameters of a nonlocal theory to obtain a non-trivial local limit. Looking ahead, we announce that keeping exactly these parameters constant is a necessary condition of the non-trivial local limit existence.

\begin{figure}
\begin{centering}
\includegraphics[width=6cm,height=6cm]{virt_en1}\includegraphics[width=8cm,height=6cm]{KG_VAC_EN}
\par\end{centering}
\caption{Plots of vacuum energy density $w_{vac}$ second-degree approximations (\ref{vac-en-N=1}), applied to virton (\ref{w-wac-N=1-Virton}) and Klein--Gordon (\ref{w-wac-N=1-KG}) cases in dimension $d=3$ and $\alpha=4/3$. The picture is qualitatively the same for $d=2$ and others $\alpha\in(1,2)$.}
\label{fig:N=1-graphs}
\end{figure}

So, in these terms the expression for the vacuum energy density reads:
\begin{equation} 
\begin{split}
    &w_{vac,N=1}m^{-d}=
    \frac{15\alpha/4}{(\alpha+1)(\alpha+3)}
    \left\{\varGamma\left(\frac{\alpha+1}{2}\right)
    \frac{2^{\alpha/2+2}}
    {\sqrt{\pi}} \frac{1+2\alpha}{5}\,z\right.\\ 
    &\left.+\sum\limits_{n=2}^{\infty}
    \frac{(-1)^{n-1}2^{n\alpha/2}}
    {\left(4\pi\right)^{d/2}n!}
    \frac{\varGamma\left(\frac{n(\alpha+1)}{2}\right)
    \varGamma\left(n-\frac{d}{2}\right)}
    {\varGamma\left(\frac{3n}{2}\right)}\,
    \left(z\xi\right)^{n}\right\}.
    \label{w-wac-N=1-KG}
\end{split}
\end{equation}
\end{enumerate}

The numerical plots of (\ref{w-wac-N=1-Virton}) and (\ref{w-wac-N=1-KG}) are presented in the Figure \ref{fig:N=1-graphs}. Their discussion will be given in the subsection \ref{sect:plots discussion}. Let us note, that these formulas are not very friendly for numerical calculations, though, their asymptotics may be of separate interest. In the present paper, however, we restrict ourselves to numerical results.

\subsection{Local Theory as a Limit of Nonlocal One}
\label{sect:local limit}

In this subsection we are going to research briefly whether it is possible to scale nonlocal theory in some way to get a local one. We start from the writing the generic term of the polynomial approximation for $\mathcal{G}_{I}[0]$. Everywhere in this subsection the term ``converging integral'' means the converges of integral when the region of integration extends from $Q$ to $\mathbb{R}^{d}$. And the same for the term``diverging integral''. We also will use the notations from the formulas (\ref{comb-form-n=general}) and (\ref{G_I[0]-short}).

According to these formulas, we have the following $n$-particle term for the $\mathcal{G}_{I}[0]$ (to simplify the analysis, only one term from the combinatorial sum is written out and up to a some numerical coefficient depending on the term):
\begin{equation*}
    \left(gG(0)^{\alpha/2}\right)^{n}
    \left\{\prod_{a=1}^{n}\int_{Q}dx_{a}\right\}
    \prod\limits_{a<b}^{n}
    \frac{G(x_{a}-x_{b})^{l_{ab}}}
    {G(0)^{l_{ab}}}=Vm^{d}
    \left(\frac{gG(0)^{\alpha/2}}{m^{d}}\right)^{n}
    \left(\frac{m^{d-2}}
    {G(0)}\right)^{\sum\limits_{a<b}^{n}l_{ab}}.
\end{equation*}
We have written this expression from the dimension considerations and translational invariance of the considering propagator. Let us assume that all the coordinate integrals except one converge, and in this case $m$ is the only dimensional parameter in the integrals. So for the vacuum energy density we have the following expression:
\begin{equation*}
    w_{vac}m^{-d}=
    \Psi\left(\frac{gG(0)^{\alpha/2}}{m^{d}},
    \frac{m^{d-2}}{G(0)}\right),
\end{equation*}
for some function $\Psi$ (formal power series). Therefore, in order to get a non-trivial local limit one should scale $m$, $g$ and $G(0)$ so that the arguments of $\Psi$ stay constant. However, it is only a \textbf{necessary} condition but not sufficient, since for the degrees of the polynomial approximation higher than two there appear diverging integrals and one have to solve the renormalization problem.

It is well-known that in $d=2$ all the power theories are renormalizable and in $d=3$ only the theories with power which $\leq 5$ are renormalizable. This means that for $d=3$ it is impossible to gather all the ``infinities'' coming from the monomials with degrees higher than $4$ into some new parameters. So, we can't write a better approximation for $d=3$ than the fourth-degree approximation. 

The conclusion is that being applied to local theories in $d>2$ our method has a limited precision of degree $4$ in $d=3$ and degree $2$ in $d\geq 4$. Though, even the second-degree approximation can provide some tool for (rough) quantitative research.

\subsection{Hard-Sphere Gas Approximation}

The formulas for vacuum energy density (\ref{vac-en-HSG}) and (\ref{Traveling_back_3}), obtained using HSG, are convenient and simple for numerical calculations. As a practise in statistical physics and numerical simulations show, HSG approximation gives worthy results, consistent with the experiment. Unfortunately, we have to deal with the power series in system parameters, so it requires accurate work with precision of calculations to distinguish the errors
from real results. We make the truncations of both series in the expression (\ref{vac-en-HSG}), namely $n_{0}$ for index $n$ and $N$ for index $q$, and use a numerical experiment to explore the truncations influence on the results. We assume $\alpha=4/3$, and one can check that there will be no qualitative differences for others $\alpha\in(1,2)$.

\begin{figure}
\begin{centering}
\includegraphics[width=7cm,height=5cm]{hard_sphere_comp_1}\includegraphics[width=7cm,height=5cm]{hard_sphere_comp_2}
\par\end{centering}
\caption{Comparative plots of specific energy $\varepsilon_{vac}$ approximations for $\gamma=0.5$ and different values of truncations $(n_0,N)$. Only the complete graph is taken into account in the calculations.}
\label{fig:HSG-graph}
\end{figure}

Recall that in the HSG approach we have approximated the matrix $G_{n}$ by the Heaviside step function and parameters $\gamma$ and $\delta$ in accordance with the expression (\ref{HSG-def}). Let us introduce new variables:
\begin{equation}
    z=gG(0)^{\alpha/2}v,\quad
    \varepsilon_{vac}=w_{vac}v,
    \label{z_hsg}
\end{equation}
where $\varepsilon_{vac}$ is the vacuum energy, contained in one ``hard-ball'' (specific energy). Further, we are going to plot the dependency $\varepsilon_{vac}(\gamma,z)$, since for HSG approximation $\gamma$ and $z$ are the most natural parameters. 

So we argue that a quite accurate and simple approximation for complete graph contribution for $z\in[0,20]$ and $\gamma\in[0,1]$ is given by the following expression:
\begin{equation} 
\begin{split}
    &\varepsilon_{vac}(\gamma,z)\!\approx\!
    \frac{1}
    {\sqrt{\pi}(\alpha+1)(\alpha+3)
    (\alpha+5)(\alpha+7)}
    \left\{\frac{88179}{12597}\, 
    2^{\frac{\alpha }{2}}
    \left(2\alpha^2+2\alpha+15\right)\right.\\ 
    &\left.\times \ (2\alpha+1) \varGamma\left(\frac{\alpha+1}{2}\right)z- 
    3^2\sqrt{\pi}2^{\alpha-14}\alpha\gamma^2 z^2 
    \left[\alpha^2\left(1144\gamma^4+
    3300\gamma^2-3645\right)-6\alpha\right.\right.\\
    &\left.\left.\times\left(1144\gamma^4+
    1540\gamma^2-5085\right)+8\left(1144\gamma^4+
    660\gamma^2-45\right)\right]
    \varGamma(\alpha+1)+\frac{2^{\frac{3\alpha}{2}-1}}{12597}\,
    \alpha\gamma^3z^3\right.\\
    &\times\left.\left[8\left(4160\gamma^6+
    14040\gamma^5+21528\gamma^4-
    13824\gamma^3-55728\gamma ^2-47061
    \gamma+80244\right)\right.\right.\\
    &\left.\left.-
    6\alpha\left(4160\gamma^6+14040\gamma^5+
    21528\gamma^4-9948\gamma^3-
    47976\gamma^2-41247\gamma+33732\right)\right.\right.\\
    &\left.\left.+\alpha^2\left(4160\gamma ^6+
    14040\gamma^5+21528\gamma^4-
    2196\gamma^3-32472\gamma^2-
    29619\gamma +16290\right)\right]\right\}.
    \label{e_vac_g_z}
\end{split} 
\end{equation} 
This expression looks complicated but algebraically it is a polynomial in $z$ and $\gamma$, so it can be manipulated without significant struggles in any system of computer algebra, e.g. Wolfram Mathematica.

\begin{figure}
\begin{centering}
\includegraphics[width=6cm,height=5cm]{hard-sphere-33}\includegraphics[width=8cm,height=6cm]{hard_sphere_2D}
\par\end{centering}
\caption{Plot of specific energy $\varepsilon_{vac}(\gamma=0.5,z)$ (left) and surface of $\varepsilon_{vac}(\gamma,z)$ (right) for $n_{0}=3$, $N=3$ and $\alpha=4/3$.}
\label{fig:HSG-surface-graph}
\end{figure}

At the end of the subsection, it is curious to plot $2$-dimensional surface of $\varepsilon_{vac}(\gamma,z)$ for $n_{0}=3$ and $N=3$ as well as the plot $\varepsilon_{vac}(\gamma,z)$ for some chosen $\gamma$. The results are presented in the Figure \ref{fig:HSG-surface-graph}. The following subsection is devoted to the discussion of plots.

\subsection{Analysis of Plots and Comparison with PT}
\label{sect:plots discussion}

Throughout this paper we have obtained the plots of vacuum energy density or specific energy for three different cases, namely:
\begin{enumerate}
    \item virton vacuum energy density for the second-degree polynomial approximation (Figure \ref{fig:N=1-graphs});
    \item Klein--Gordon vacuum energy density for the second-degree polynomial approximation (Figure \ref{fig:N=1-graphs});
    \item specific energy in HSG approximation (Figure \ref{fig:HSG-surface-graph}).
\end{enumerate}

During all this subsection for every theory from the three considered (virton, Klein--Gordon and HSG) we will use the notation $z$ for the argument, keeping in mind, that its definition is own for every theory and is given by (\ref{z_vir}), (\ref{z-zeta-def-KG}) and (\ref{z_hsg}), correspondingly. For this reason, it is also incorrect to compare the numerical values of $w_{vac}$ for Klein--Gordon and virton theories, because the parameters $\mu$ and $m$ are fundamentally different. The parameter $\mu$ is UV parameter, while the parameter $m$ is IR one. As for the HSG formulas, they could be compared with virton and Klein--Gordon cases separately, but after the proper change of parameters to $\gamma$ and $v$ according to (\ref{HSG_equations_for_parameters}).

\subsubsection{Analysis of HSG Approximation}

In the case of HSG approximation we have found the inflexion point. It was observed from the third-order in the coupling constant $g$ and the sixth-order polynomial approximation, which we encoded in the plot as $(3,3)$ (three particles and sixth degree). It slightly changed with increasing the approximation order that was checked up to $(4,3)$ (four particles and sixth degree). Thus we expect that this result stays relevant for higher orders of approximation. 

Both the virton and Klein--Gordon Gaussian theory Green functions satisfy the HSG approximation conditions. Though, the error of this approximation demands separate research, we can extract some predictions from this fact. For example, we can guess that the next orders of polynomial approximation addition (without HSG) should result in arising of an inflexion point in virton and Klein--Gordon cases. 

\subsubsection{Analysis of PT First Orders}

In the general propagator case there is no point of inflexion in orders $g$ and $g^2$ (\ref{vac-en-first-orders}) and the function $w_{vac}$ is convex upwards (concave function). The existence of the inflexion point will be determined with the next PT orders.

We have also calculated the order $g^3$ for the fourth-degree approximation in the virton theory, but we won't present it here because of this paper size. However, this contribution is positive for $g>0$, so it produces an inflexion point referring to the virton propagator. Recalling the behaviour of polynomial approximations in the HSG approach, we expect that a fourth and higher PT orders as well as higher degrees of approximations won't affect the existence of such an inflexion point. In the Klein--Gordon case the coordinate integrals are much more complicated, so we postpone its research for future papers.

\subsubsection{Analysis of Virton and Klein--Gordon Second-Degree Approximation Formulas}

At the same time, we do not observe an inflexion point in the case of second-order polynomial approximation for the virton (\ref{w-wac-N=1-Virton}) and Klein--Gorgon (\ref{w-wac-N=1-KG}) vacuum energy densities. Numerically its absence follows from plots, and analytically from the asymptotics of the corresponding series. Namely, we have:
\begin{equation}
    w_{vac}^{\text{KG}}=az-b(z\xi)^{3/2}, 
    \quad
    w_{vac}^{\text{v}} =Az+B\ln^{3/2}z,
    \label{N=1-asymptotics}
\end{equation}
for some positive constants $A,B,a,b$. We don't present the derivation of these asymptotics in this paper since they are used only for reference. 

From the written asymptotics it follows that both functions are convex upwards, which (combined with numerical plots) proves the absence of the inflexion points in the approximations of the second degree. Moreover, their convexity differs from the one predicted with the PT first three orders (at least, for the virton case) and the HSG consideration. Recall that we have deduced from these approaches that the right behaviour at infinity is downward convex (convex function). Thus we come to the conclusion that these formulas are not applicable for strong coupling limit.

Moreover, the vacuum energy density in Klein--Gordon theory behaves non-physically in the large coupling constants region because it becomes negative for sufficiently large $z$. It can be seen both from plots and asymptotics. Moreover, from the asymptotic, it follows that such negative values are reached for any $\xi$ if $z$ is sufficiently large.

Summarising the discussion of the second-degree polynomial approximation formulas, they fail to reach a strong coupling limit. Though, they can describe the intermediate values of the coupling constant (rather than very small only), and therefore they can be the computational alternative for PT first orders. The plots of vacuum energy density for second-degree polynomial approximation should be cut on some value of $g$.