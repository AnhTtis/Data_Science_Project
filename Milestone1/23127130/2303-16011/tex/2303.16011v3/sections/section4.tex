\section{Derivation of PT}
\label{sect-PT-derivation}

\subsection{Mathematical Buildup and Definitions}
\label{Mathematical Buildup}

\subsubsection{Gaussian Measure}

Let us rewrite the expression for the complete Green functions GF $\mathcal{Z}$ as an integral over a Gaussian measure $\gamma_{G}$ with functions $g$ and $j$ in $\mathcal{L}^{2}(\mathbb{R}^{d})$ (vector space of Lebesgue square-integrable functions):
\begin{equation}
    \mathcal{Z}\left[j\right]=\int_{\varPhi}
    \gamma_{G}(d\phi)\,e^{-\int_{\mathbb{R}^{d}} dx\,g(x)
    \left|\phi\left(x\right)\right|^{\alpha}+
    \int_{\mathbb{R}^{d}} dx\, j\left(x\right)
    \phi\left(x\right)}
    \label{Complete_Green_functions_GF_Z_Gaussian_measure},
\end{equation}
where we understand functional integral as integral over HS
$\varPhi=L^{2}(\mathbb{R}^{d})$ --- linear space of equivalence classes of functions from $\mathcal{L}^{2}(\mathbb{R}^{d})$.

By definition, Gaussian measure $\gamma_{G}$ is a measure in HS $\varPhi$ defined by some covariance operator (propagator) $G:\varPhi\rightarrow\varPhi$, which is demanded to be trace-class. Let us introduce $\{e_{n}\}_{n=1}^{\infty}$ --- a family of eigenvectors of $G$ with corresponding eigenvalues $\{\lambda_{n}\}_{n=1}^{\infty}$,
so that $Ge_{n}=\lambda_{n}e_{n}$, $\forall n\in\mathbb{N}$. We also denote as $\varPhi_{N}:=\left\langle e_{i}\right\rangle _{i=1}^{N}$ the linear span of first $N$ eigenvectors, and as $P_{N}:\varPhi\rightarrow\varPhi$ a projection operator on this linear span. Then, by definition, integral of any function $f$ depending only on $P_{N}\phi$ over Gaussian measure $\gamma_{G}$ reads: 
\begin{equation}
    \int_{\varPhi}\gamma_{G}(d\phi)f[P_{N}\phi]=
    \int_{\varPhi_{N}}\prod_{k=1}^{N}
    \frac{d\phi_{k}}{\sqrt{2\pi\lambda_{k}}}\,
    e^{-\frac{1}{2}\sum\limits_{k=1}^{N}
    \frac{\phi_{k}^{2}}{\lambda_{k}}}f[P_{N}\phi].
\end{equation}

Then, under certain conditions, the integral of the generic function $f:\varPhi\rightarrow\mathbb{C}$ can be computed using Dominated Convergence Theorem (DCT for short) and convergence of the sequence $f[P_{N}\phi]\rightarrow f[\phi]$ as $N\rightarrow\infty$ almost everywhere:
\begin{equation}
    \int_{\varPhi}\gamma_{G}(d\phi)f[\phi]=
    \lim_{N\rightarrow\infty}
    \int_{\varPhi_{N}}\prod_{k=1}^{N}
    \frac{d\phi_{k}}{\sqrt{2\pi\lambda_{k}}}\,
    e^{-\frac{1}{2}\sum\limits_{k=1}^{N}
    \frac{\phi_{k}^{2}}{\lambda_{k}}}f[P_{N}\phi].
\end{equation}
All the described results on Gaussian measure can be found with derivation in~\cite{Bogachev,GiuseppePrato}.

\subsubsection{Physical and Mathematical Approaches to Measure Theory in Infinite-Dimensional Spaces of Functions}

It is useful to compare the mathematical view on functional integration with the physical one. In physics, we write the expression (\ref{Complete_Green_functions_GF_Z_General}). The thing is that there is no translation invariant countably-additive Lebesgue measure on infinite-dimensional separable BS or HS. So if one wants to develop a measure theory in separable HS, he has to reject either countable 
additivity or translational invariance. So in mathematics, there are several approaches to integration in functional spaces, which are:
\begin{enumerate}
    \item Gaussian countably-additive measure (the way we use in the present paper)~\cite{Bogachev};
    \item Translational invariant finitely-additive measure~\cite{fin-add-measures-3,fin-add-measures-2,fin-add-measures};
    \item Ito integral (Wiener measure, which corresponds to significant restriction or sigma-algebra to the one generated by cylindrical sets only)~\cite{stohastic-calc}.
\end{enumerate}

In this paper, as has been previously explained we use Gaussian measures language. In the way the physicists understand the functional measure, one can write informally:
\begin{equation}
    \gamma_{G}(d\phi) = 
    \mathcal{D}\left[\phi\right]\,
    e^{-\frac{1}{2}
    \int_{\mathbb{R}^{d}}
    \int_{\mathbb{R}^{d}} dx\,dy\, 
    L(x,y)\phi(x)\phi(y)},\quad  G=L^{-1}.
\end{equation}
It means that informally Gaussian measure determined by some propagator $G$ is a ``physical'' measure multiplied by an exponential of minus quadratic part of the action with ``kinetic operator'' $G=L^{-1}$.

\subsubsection{About Continuity of Functionals}

Let us make one important remark that will further ensure the correctness of all subsequent transformations. Consider the following functional (a function defined on a function):
\begin{equation}
    f[\phi]=e^{-\int_{\mathbb{R}^{d}} dx\,g(x)
    \left|\phi\left(x\right)\right|^{\alpha}+
    \int_{\mathbb{R}^{d}} dx\, j\left(x\right)
    \phi\left(x\right)},
\end{equation}
so it is positive, measurable and $f[\phi]\leq e^{\int j\left(x\right)\phi\left(x\right)dx}\leq e^{\left\Vert j\right\Vert _{2}\left\Vert \phi\right\Vert _{2}}$
which is clearly integrable with a finite value of the integral. Moreover, for proving that the sequence $f[P_{N}\phi]\rightarrow f[\phi]$ as $N\rightarrow\infty$ almost everywhere it is necessary and sufficient to show that $\int_{\mathbb{R}^{d}} dx\,g(x)|\phi\left(x\right)|^{\alpha}$
is a continuous functional on $\varPhi$, since exponent and $\int_{\mathbb{R}^{d}} dx\, j\left(x\right)\phi\left(x\right)$
are continuous. 

Let the coupling constant (non-negative function) $g$ be bounded and Lebesgue integrable. Since $\alpha\in(1,2)$, the following is true (we denote the measure for which the coupling constant is the measure density by the same letter $g$):
\begin{equation}
\begin{split}
    &\int_{\mathbb{R}^{d}} dx\,g(x)|\phi\left(x\right)|^{\alpha}=
    \left\Vert\phi\right\Vert_{g,\alpha}^{\alpha}\ ,\quad
    \int_{\mathbb{R}^{d}} dx\,g(x)=g(\mathbb{R}^{d}),\quad
    \sup\limits_{x\in\mathbb{R}^{d}}g(x)=g_s,\\
    &\left|\left\Vert P_{N}\phi\right\Vert_{g,\alpha}-
    \left\Vert\phi\right\Vert_{g,\alpha}\right|\leq
    \left\Vert P_{N}\phi-\phi\right\Vert_{g,\alpha}\leq 
    g(\mathbb{R}^{d})^{\left(\frac{1}{\alpha}-\frac{1}{2}\right)}
    g_{s}^\frac{1}{2}
    \left\Vert P_{N}\phi-\phi\right\Vert _{2}.
\end{split}
\end{equation}
Thus the functional $\int_{\mathbb{R}^{d}} dx\,g(x)|\phi\left(x\right)|^{\alpha}$ is continuous for all $\phi\in\varPhi$. Here we have used a well-known consequence of H{\"o}lder inequality for the set $\mathbb{R}^{d}$ with finite measure $g(\mathbb{R}^{d})$. We would like to underline that this functional is not continuous when $\alpha>2$ because in this case $\left\Vert \phi\right\Vert _{g,\alpha}$ is not finite if $\left\Vert \phi\right\Vert_{2}$ is finite (which is equivalent to $\phi\in\varPhi$).

As a result, one can apply DCT and find that:
\begin{equation}
    \mathcal{Z}\left[j\right]=
    \lim_{N\rightarrow\infty}
    \int_{\varPhi_{N}}\prod_{k=1}^{N}
    \frac{d\phi_{k}}{\sqrt{2\pi\lambda_{k}}}\,
    e^{-\frac{1}{2}\sum\limits_{k=1}^{N}
    \frac{\phi_{k}^{2}}{\lambda_{k}}-
    \int_{\mathbb{R}^{d}} dx\,g(x)
    |(P_{N}\phi)\left(x\right)|^{\alpha}+
    \int_{\mathbb{R}^{d}} dx\, j\left(x\right)
    (P_{N}\phi)\left(x\right)}.
\end{equation}
That's quite a good outlook that if one considers a functional integral from a Gaussian measure point of view, one can't compute a functional integral with $\alpha>2$ (e.g. $\alpha=4$, which is currently the most interesting in the physical case) by simply substituting $P_{N}\phi$ instead of $\phi$ and taking a limit $N\rightarrow\infty$. At least DCT is not applicable in this case. However, it should be remembered that DCT gives only sufficient conditions. Perhaps some other theorem may be useful in the case $\alpha>2$.

\subsection{Reduction of Functional Integral to the Series from Finite-Dimensional Ones (GCPF Expansion)}
\label{subsect:reduction of func int}

In this section, we are going to derive PT type series, which will converge. Since the power of potential is not a natural number, the traditional method of carrying out an interaction action from integral  in the form of a differential operator is not suitable. So, we follow
the other way and in some sense get all Feynman diagrams summed into one integral, which appears to be exactly the coefficient of $g(x_1)\ldots g(x_n)$ in the obtained series. 

The described approach of constructing PT repeats and develops methods described in~\cite{efimov1970nonlocal,efimov1977nonlocal,efimov1985problems,Ogarkov2019I}. Let us proceed to the computation itself. Since the potential $U(\phi)=\left|\phi\right|^\alpha$ is even function, the following is true:
\begin{equation}
    \int_{\mathbb{R}^{d}} dx\,
    g(x) \left|\phi\left(x\right)\right|^{\alpha}=\frac{1}{2\pi}
    \int_{\mathbb{R}^{d}} dx\, g(x)
    \mathcal{F}\left[\mathcal{F}
    \left[\left|\phi\left(x\right)\right|^{\alpha}\right]\right],
\label{Fourier-main-formula}
\end{equation}
where two points are important:
\begin{enumerate}
    \item we understand Fourier transform $\mathcal{F}$ (internal transform is performed over the variable $\phi$, but not $x$) in the sense of distributions from $\mathcal{S}'(\mathbb{R})$ --- the space of linear continuous functionals on the Schwartz space 
    $\mathcal{S}(\mathbb{R})$;
    \item this distributional Fourier transform is consistent with the  following formula for the integrable function $f(\phi)$:
    \begin{equation}
        \mathcal{F}[f(\phi)](t) := \int_\mathbb{R} 
        d\phi\, f(\phi)\, e^{it\phi}.
        \label{Fourier-transform-formula}
    \end{equation}
\end{enumerate}

Unfortunately, the potential $U(\phi) = \left|\phi\right|^{\alpha}$ have Fourier transform only in generalized sense in $\mathcal{S}'(\mathbb{R})$. And as usual, careful computations with generalized Fourier transform are quite subtle, so we would like to avoid them. For that purpose, we approximate $U(\phi)$ with some smooth and integrable function $U_{\varLambda}(\phi)$ from the Schwartz space $\mathcal{S}(\mathbb{R})$, so that $U_{\varLambda}\left(\phi\right)\rightarrow\left|\phi\right|^{\alpha}$ pointwise, when $\varLambda\rightarrow\infty$. Then both $U_{\varLambda}(\phi)$ and $\mathcal{F}\left[U_{\varLambda}(\phi)\right]$ are usual functions in $\mathcal{S}(\mathbb{R})$ rather than distributions, and we are entitled to calculate their Fourier transform via the usual integral formula (\ref{Fourier-transform-formula}). We will make all the transitions for smooth and integrable function $U_{\varLambda}\left(\phi\right)$. Also, we will assume that $0\leq U_{\varLambda}\left(\phi\right)\leq\left|\phi\right|^{\alpha}$ for all $\phi$ and that all the functions $U_{\varLambda}(\phi)$ are even in $\phi$, which will be useful in calculation of PT series majorant. Namely, we can write:
\begin{equation}
\int_{\mathbb{R}^{d}} dx\,
    g(x)\left|\phi\left(x\right)\right|^{\alpha} = \lim_{\varLambda\rightarrow\infty}
    \int_{\mathbb{R}^{d}} dx\,
    g(x) U_{\varLambda}\left[\phi\left(x\right)\right],
\end{equation}
since $0\leq U_{\varLambda}\left[\phi\right]\leq\left|\phi\right|^{\alpha}$ and the left-hand side integral converges absolutely, so DCT justifies this transition. After that, we are able to write:
\begin{equation}
    \int_{\mathbb{R}^{d}} dx\,
    g(x)\left|\phi\left(x\right)\right|^{\alpha} = \lim_{\varLambda\rightarrow\infty} 
    \frac{1}{2\pi}\int_{\mathbb{R}^{d}} dx\, 
    g(x)\mathcal{F}\left[\mathcal{F}
    \left[U_{\varLambda}\left[\phi
    \left(x\right)\right]\right]\right],
\end{equation}
and this formula is much more convenient for the calculations. In the following calculations, we will extract this limit from functional integral also due to DCT and will consider regularized functional $\mathcal{Z}_{\varLambda}\left[j\right]$. And at the end of our computation, we will calculate the limit $\varLambda\rightarrow\infty$. So we start by considering the ``regularized'' functional:
\begin{equation}
    \mathcal{Z}_{\varLambda}\left[j\right]=
    \int_{\varPhi}\gamma_{G}(d\phi)\,e^{-\int_{\mathbb{R}^{d}} dx\,g(x) U_{\varLambda}\left[\phi\left(x\right)\right]+
    \int_{\mathbb{R}^{d}} dx\, 
    j\left(x\right)\phi\left(x\right)},
    \label{GFZ_Reg_Pot}
\end{equation}
and due to DCT described in previous section $\mathcal{Z}_{\varLambda}\left[j\right]\rightarrow\mathcal{Z}\left[j\right]$, when $\varLambda\rightarrow\infty$.

Further, we write the expression (\ref{GFZ_Reg_Pot}) in terms of the Fourier transform of potential and use that $\mathcal{F}\left[U_{\varLambda}(\phi)\right]$ is in $\mathcal{S}(\mathbb{R})$:
\begin{equation}
\label{F_F_representation}
    \mathcal{Z}_{\varLambda}\left[j\right]=
    \int_{\varPhi} \gamma_{G}(d\phi)
    \exp{\left\{-\!\int_{\mathbb{R}^{d}}dx\,g(x)\!
    \int_{\mathbb{R}}
    \frac{dt}{2\pi}\, e^{it\phi(x)}\mathcal{F}
    \left[U_{\varLambda}\left(\phi\right)\right]
    \left(t\right)+\!\int_{\mathbb{R}^{d}}dx\, j\left(x\right)\phi\left(x\right)\right\}}.
\end{equation}
Now let us expand the external exponent in a Taylor series in powers of the coupling constant $g$:
\begin{equation}
\begin{split}
    \mathcal{Z}_{\varLambda}\left[j\right]&=
    \int_{\varPhi}\gamma_{G}(d\phi)
    \sum_{n=0}^{\infty}\frac{\left(-1\right)^{n}}{n!\left(2\pi\right)^{n}}
    \left\{\prod_{a=1}^{n}\int_{\mathbb{R}^{d}}
    \int_{\mathbb{R}} dx_{a}\,dt_{a}\, g(x_a) \mathcal{F}\left[U_{\varLambda}\left(\phi\right)\right]
    \left(t_{a}\right)\right\}\\
    &\times\exp{\left\{i\sum\limits_{a=1}^{n}t_{a}\phi(x_{a})+
    \int_{\mathbb{R}^{d}}dx\, 
    j\left(x\right)\phi\left(x\right)\right\}}.
\end{split}
\end{equation}
Exchanging summation and integration, we arrive at the following expression:
\begin{equation}
\begin{split}
    \mathcal{Z}_{\varLambda}\left[j\right]&=
    \sum_{n=0}^{\infty}\frac{\left(-1\right)^{n}}{n!\left(2\pi\right)^{n}}\left\{ \prod_{a=1}^{n}\int_{\mathbb{R}^{d}}
    \int_{\mathbb{R}} dx_{a}\,dt_{a}\, g(x_a)\mathcal{F}\left[U_{\varLambda}
    \left(\phi\right)\right]\left(t_{a}\right)\right\}\\ 
    &\times \int_{\varPhi} \gamma_{G}(d\phi)
    \exp{\left\{i\sum\limits_{a=1}^{n}t_{a}\phi(x_{a})+
    \int_{\mathbb{R}^{d}}dx\, 
    j\left(x\right)\phi\left(x\right)\right\}}.
\end{split}
\end{equation}
The correctness of this permutation will be proven independently in the next section.

Finally, we have to integrate over $\phi$. After introducing the ``modified'' source $\tilde{j}$ (the second term in the right-hand side contains the Dirac delta function $\delta$):
\begin{equation*}
    \tilde{j}\left(x\right)=j\left(x\right)+
    i\sum_{a=1}^{n}t_{a}\delta(x-x_{a}),
\end{equation*}
we obtain usual Gaussian integral with linear exponent:
\begin{equation}
    \int_{\varPhi}\gamma_{G}(d\phi)\, 
    e^{\int_{\mathbb{R}^{d}} \tilde{j}\left(x\right)\phi\left(x\right)}
    =e^{\frac{1}{2}\langle\,\tilde{j}\,|G|
    \,\tilde{j}\,\rangle}=
    e^{\frac{1}{2}\int_{\mathbb{R}^{d}}
    \int_{\mathbb{R}^{d}} dx\,dy\, 
    G(x,y)\tilde{j}(x)\tilde{j}(y)}.
    \label{GI_well_known_formula}
\end{equation}
Using the expression (\ref{GI_well_known_formula}), we arrive at the following equality:
\begin{equation}
\begin{split}
\label{F_E}
   \mathcal{Z}_{\varLambda}\left[j\right] &=\mathcal{Z}_{0}\left[j\right]
   \sum_{n=0}^{\infty}\frac{\left(-1\right)^{n}}{n!\left(2\pi\right)^{n/2}}\left\{ \prod_{a=1}^{n}\int_{\mathbb{R}^{d}} 
   \int_{\mathbb{R}} dx_{a}\,dt_{a} \, g(x_a) \mathcal{F}\left[U_{\varLambda}
   \left(\phi\right)\right]\left(t_{a}\right)\right\}\\ 
   &\times\exp\left\{-\frac{1}{2}
   \sum_{a,b=1}^{n}\left(G_{n}\right)_{ab}
   t_{a}t_{b}-i\sum_{a=1}^{n}t_{a}
   \varphi\left(x_{a}\right)\right\}.
\end{split}
\end{equation}

Now we are going to use Parseval--Plancherel identity, but disappointingly the matrix $\left(G_{n}\right)_{ab}=G(x_{a},x_{b})$ (the restriction of a propagator on finite-dimensional space $\mathbb{R}^{2n}$) is only semi-positive, which will be proved in section \ref{G-properties}. However, $\det\left(G_{n}\right)_{ab}=0$ only on null sets (sets of zero measure), which made it possible not to take care of it up to this moment. But in the following, we are going to approximate integrand with polynomials so we would like it to be bounded. So we introduce the $\mathcal{Z}_{\varLambda,\varepsilon}$ instead of $\mathcal{Z}_{\varLambda}$, adding $-\varepsilon\sum\limits_{a=1}^n t_a^2\,$ term into the series $n$th term exponent for all $n$. And then we will calculate the limit $\varepsilon\rightarrow +0$ in a final result, using the obtained majorant and its (in)dependence of $\varepsilon$. Hence, we have:
\begin{equation}
\begin{split}
   \mathcal{Z}_{\varLambda,\varepsilon}\left[j\right] &=\mathcal{Z}_{0}\left[j\right]
   \sum_{n=0}^{\infty}\frac{\left(-1\right)^{n}}{n!\left(2\pi\right)^{n/2}}
   \left\{ \prod_{a=1}^{n}
   \int_{\mathbb{R}^{d}}
   \int_{\mathbb{R}} dx_{a}\,dt_{a}\, g(x_a) \mathcal{F}\left[U_{\varLambda}
   \left(\phi\right)\right]
   \left(t_{a}\right)\right\}\\ 
   &\times\exp\left\{-\frac{1}{2}
   \sum_{a,b=1}^{n}(\left(G_{n}\right)_{ab}+
   \varepsilon\delta_{ab})t_{a}t_{b}
   -i\sum_{a=1}^{n}t_{a}
   \varphi\left(x_{a}\right)\right\}.
\end{split}
\end{equation}
At this point, it is evident that every term in the above series converges to the term with the same number in the series for $\mathcal{Z}_{\varLambda}$ when $\varepsilon\rightarrow+0$.

At present, we can use Parseval--Plancherel identity:
\begin{equation} 
\begin{split} 
    \mathcal{Z}_{\varLambda,\varepsilon}\left[j\right] & =\mathcal{Z}_{0}\left[j\right]
    \sum_{n=0}^{\infty}\frac{\left(-1\right)^{n}}{n!\left(2\pi\right)^{n/2}}\left\{ \prod_{a=1}^{n}\int_{\mathbb{R}^{d}}  
    dx_{a} \, g(x_a)\,e^{-\frac{1}{2}\sum\limits_{a,b=1}^{n}
    \left(R_{n}\right)_{ab}\varphi\left(x_{a}\right)
    \varphi\left(x_{b}\right)}\right.\\
    &\left.\times\int_{\mathbb{R}} 
    d\phi_{a}\,U_{\varLambda}
    \left(\phi_{a}\right)\right\}
    \frac{e^{-\frac{1}{2}\sum\limits_{a,b=1}^{n}
    \left(R_{n}\right)_{ab}\phi_{a}\phi_{b}+
    \sum\limits_{a=1}^{n}\phi_{a}\chi\left(x_{a}\right)}}{\sqrt{\det\left(G_{n}+
    \varepsilon 1_{n}\right)}}.
    \label{GFZ_another_transformation}
\end{split} 
\end{equation} 

In the expression (\ref{GFZ_another_transformation}) the following notations are introduced:
\begin{enumerate}
    \item $\left(R_{n}\right)_{ab}=(\left(G_{n}\right)_{ab}+
   \varepsilon\delta_{ab})^{-1}$ is the inverse of $(\left(G_{n}\right)_{ab}+\varepsilon\delta_{ab})$ in finite-dimensional space $\mathbb{R}^{2n}$ and $\left(1_{n}\right)_{ab}=\delta_{ab}$ is the identity matrix in $\mathbb{R}^{2n}$;
   \item $\chi\left(x_{a}\right)=
   \sum\limits_{b=1}^{n}\left(R_{n}\right)_{ab}
   \varphi\left(x_{b}\right)$ is ``effective'' discrete source, since it was obtained from $j$ firstly by acting the ``continuous'' operator $G$, and then by its ``discrete'' inverse.
\end{enumerate}
Let us note that both $R_{n}$ and $G_{n}$ are symmetric, hence diagonalizable in an orthonormal basis by orthogonal transformation. Moreover, all $G_{n}$ are semi-positive, so $G_{n}+\varepsilon 1_{n}$ is positive and hence invertible. This proves that all written integrals converge absolutely, so we can interchange the order of integration in $t_{a}$ and $x_{a}$ variables.

In the section \ref{majorant-section}, we will show that the obtained series converges uniformly in $\varLambda$ for generic $j$ and also in $\varepsilon$ for $j=0$. The idea of the proof is that we can choose $U_{\varLambda}(\phi)$ such, that $0\leq U_{\varLambda}\left[\phi\right]\leq\left|\phi\right|^{\alpha}$. After that, we can apply Weierstrass M-test to the series for $\mathcal{Z}_{\varLambda,\varepsilon}$ if we prove that the series above with $U_{\varLambda}(\phi)$, replaced by $\left|\phi\right|^{\alpha}$, converges. We will do it later in the paper.

Up to this moment, we finish on the expression for the ``normalized'' GF $\mathcal{Z}_{I,\varepsilon}$, removing the regulator $\varLambda$:
\begin{equation} 
\begin{split}
\label{Z_I_eps_final}
    \mathcal{Z}_{I,\varepsilon}\left[j\right]&= 
    \sum_{n=0}^{\infty}\frac{\left(-1\right)^{n}}{n!\left(2\pi\right)^{n/2}}\left\{ \prod_{a=1}^{n}\int_{\mathbb{R}^{d}}  
    dx_{a} \, g(x_a)\,e^{-\frac{1}{2}\sum\limits_{a,b=1}^{n}
    \left(R_{n}\right)_{ab}\varphi\left(x_{a}\right)
    \varphi\left(x_{b}\right)}\right.\\
    &\left.\times\int_{\mathbb{R}} 
    d\phi_{a}\,\left|\phi_{a}\right|^{\alpha}\right\}
    \frac{e^{-\frac{1}{2}\sum\limits_{a,b=1}^{n}
    \left(R_{n}\right)_{ab}\phi_{a}\phi_{b}+
    \sum\limits_{a=1}^{n}\phi_{a}\chi\left(x_{a}\right)}}{\sqrt{\det\left(G_{n}+
    \varepsilon 1_{n}\right)}},
\end{split} 
\end{equation} 
where we denoted $\mathcal{Z}_{I,\varepsilon}\left[j\right]=
\mathcal{Z}_{\varLambda\rightarrow\infty,\varepsilon}
\left[j\right]/\mathcal{Z}_{0}\left[j\right]$, in accordance with the notations, introduced in the section \ref{phys-mot}. In particular, we can write the value $\mathcal{Z}_{I}\left[0\right]$ for the case of $j=0$, also removing the regulator $\varepsilon$:
\begin{equation}
    \mathcal{Z}_{I}\left[0\right]= 
    \sum_{n=0}^{\infty}
    \frac{\left(-1\right)^{n}}
    {n!\left(2\pi\right)^{n/2}}
    \left\{ \prod_{a=1}^{n}
    \int_{\mathbb{R}^{d}}  
    dx_{a}\, g(x_a)
    \int_{\mathbb{R}} 
    d\phi_{a}\,
    \left|\phi_{a}\right|^{\alpha}\right\}
    \frac{e^{-\frac{1}{2}\sum\limits_{a,b=1}^{n}
    \left(G_{n}\right)^{-1}_{ab}\phi_{a}\phi_{b}}}
    {\sqrt{\det \, G_{n}}}.
    \label{Z_I[0]-final-form}
\end{equation}
Here the matrix $G_n$ is degenerate on null sets, but it doesn't affect the integral because of the existence of majorant not depending on $\varepsilon$.

This expansion of GF in perturbation series looks similar to the expansion of the GCPF of non-ideal gas with potential $G(x_{a},x_{b})$, but with additional inner integrals over $\phi_{a}$. Keeping this in mind, we will refer to $\mathcal{Z}_{I,\varepsilon,n}$ as the $n$-particle canonical partition function. These inner integrals are in fact the key difference between statistical physics and QFT, warranting the complication of the last one.

\subsubsection{Properties of $G_{n}$-Matrices}
\label{G-properties}

It is useful to visualize the typical structure of matrix $G_{n}$. We will do this for the translation invariant case (the general case is done in a similar way), since this is the case that will be considered in all the final results. The desired expression is:
\begin{equation}
G_{n}=\left(\begin{array}{cccc}
G(0) & G(x_{1}-x_{2}) & \ldots & G(x_{1}-x_{n})\\
G(x_{1}-x_{2}) & G(0) & \ldots & \ldots\\
\ldots & \ldots & G(0) & G(x_{n}-x_{n-1})\\
G(x_{1}-x_{n}) & \ldots & G(x_{n}-x_{n-1}) & G(0)
\end{array}\right).
\end{equation}

There are two limiting cases for this matrix:
\begin{enumerate}
\item When all $x_{a}$ are equal, then:
\begin{equation*}
G_{n}=\left(\begin{array}{cccc}
G(0) & G(0) & \ldots & G(0)\\
G(0) & G(0) & \ldots & \ldots\\
\ldots & \ldots & G(0) & G(0)\\
G(0) & \ldots & G(0)) & G(0)
\end{array}\right).
\end{equation*}
\item When all $x_{a}$ are infinitely faraway, then:
\begin{equation*}
G_{n}=\left(\begin{array}{cccc}
G(0) & 0 & \ldots & 0\\
0 & G(0) & \ldots & \ldots\\
\ldots & \ldots & G(0) & 0\\
0 & \ldots & 0 & G(0)
\end{array}\right).    
\end{equation*}
\end{enumerate}

Let us prove the semi-positivity of $G_{n}$. Since the operator $G: \varPhi\rightarrow \varPhi$ is positive, the equality $\langle\,j\,|G|\,j\,\rangle \geq 0$ holds $\forall j\in \varPhi$. And for distributions this inequality also holds from the reasons of continuity. Namely, choosing a smooth sequence of functions approximating some distribution of the form:
\begin{equation*}
j(x) =  \sum_{a=1}^n c_a\, \delta(x-x_a), 
\end{equation*}
we get exactly:
\begin{equation*}
\langle\,\vec{c}\,|G_{n}|\,\vec{c}\,\rangle \geq0, \quad \forall \vec{c} = (c_1,\ldots,c_n)\in\mathbb{R}^{n},
\end{equation*}
which completes the proof of semi-positivity of $G_{n}$ for all values of $x_a$. For this reason, $G_{n}+\varepsilon 1_{n}$ is a positive definite matrix for all $n$ as well as its inverse, for any $\varepsilon>0$. 

The other way to prove this fact (in the translation invariant case) is to use Bochner's theorem, which claims the semi-positivity of $G_{n}$ from the fact that propagator $G(x-y)$ is from Schwartz space and its Fourier transform $\mathcal{F}\left[G\right](k)\geq 0$ is non-negative. Though, this proof needs one more additional assumption that $G(x-y)\in\mathcal{S}(\mathbb{R}^{d})$.

By construction, the matrix $G_{n}$ is symmetric, therefore, diagonalizable. Let us denote its eigenvalues as $0\leq\lambda_{1}^{(n)}\leq\lambda_{2}^{(n)}\leq\ldots\leq\lambda_{n}^{(n)}$. Since $\sum\limits_{a=1}^{n}\lambda_{a}^{(n)}=\text{tr}\, G_{n}=nG(0)$, then the matrix sup-norm $\left\Vert G_{n}+\varepsilon 1_{n}\right\Vert \leq\lambda_{n}^{(n)}+n\varepsilon\leq n(G\left(0\right)+\varepsilon)$. In particular, we found the bound for maximum eigenvalue of $G_n$ (and, simultaneously, the minimal eigenvalue of $G_n^{-1}$, when $G_n$ is invertible):

\begin{equation}
    \label{max-G-eigenvalue-bound}
    \lambda^{(n)}_{max} = \lambda^{(n)}_n \leq nG(0). 
\end{equation}

This bound will be used in the future computation of majorizing series.

\subsection{Majorizing Series (Majorant)}
\label{majorant-section}
\subsubsection{Coordinate-Free Part}

We always can exchange summation and integration when the series terms are positive due to MCT. And we have to check the absolute convergence of the obtained series to use DCT. So, it's sufficient to take the absolute value of every term, exchange sum and integral and prove that this series converges. At this moment, we have to find the upper bound of the series terms:
\begin{equation}
    J_{n}\left(\alpha\right)=
    \frac{1}{\sqrt{\det\left(G_{n}+
    \varepsilon 1_{n}\right)}}
    \int_{\mathbb{R}^{n}}  d\phi_{1}\ldots d\phi_{n}\,\left|\phi_{1}\right|^{\alpha}\ldots
    \left|\phi_{n}\right|^{\alpha}e^{-\frac{1}{2}\sum\limits_{a,b=1}^{n}\left(R_{n}\right)_{ab}
    \phi_{a}\phi_{b}+
    \sum\limits_{a=1}^{n}
    \phi_{a}\chi\left(x_{a}\right)},
\end{equation}
where the notation of $J_{n}\left(\alpha\right)$ is introducing for the convenience. Now we are going to estimate $\left|\phi_{1}\right|^{\alpha}\ldots\left|\phi_{n}\right|^{\alpha}$
with restriction $\left\Vert\phi\right\Vert ^{2}=r^{2}$. Our aim is to bound the integrand on every sphere in $\mathbb{R}^{n}$ centered at the origin, and then use spherical coordinates. We start with the Lagrangian function ($\lambda$ is the Lagrange multiplier):
\begin{equation*}
    f(\phi)=\alpha\sum\limits_{a=1}^{n}\ln\phi_{a}-
    \lambda\left(\left\Vert\phi\right\Vert ^{2}-r^{2}\right),
\end{equation*}
and find its unconditional extremum:
\begin{equation*}
    \partial_{a}f(\phi)=\frac{\alpha}{\phi_{a}}
    -2\lambda\phi_{a}=0,
\end{equation*}
hence:
\begin{equation*}
    \phi_{a}^{2}=\frac{\alpha}{2\lambda}.
\end{equation*}
Taking the sum over $a$ to get $\lambda$ we receive:
\begin{equation*}
    r^{2}=\frac{\alpha n}{2\lambda}.
\end{equation*}
So, finally, the maximum is reached at:
\begin{equation*}
    \phi_{a}=\frac{r}{\sqrt{n}}.
\end{equation*}
So, one can estimate (and these bounds are strict):
\begin{equation}
\label{normals-product-bound}
    0\leq\left|\phi_{1}\right|^{\alpha}\ldots
    \left|\phi_{n}\right|^{\alpha}\leq\frac{r^{n\alpha}}{n^{\frac{n\alpha}{2}}}.
\end{equation}

In addition to the previous bound, we can change variables by the orthogonal transformation and diagonalize the exponent in $J_{n}\left(\alpha\right)$ putting $\phi_{a}=
\sum\limits_{b=1}^{n}\left(R_{n}\right)_{ab}^{-1/2}\zeta_{b}$,
plus estimate $\left\Vert\phi\right\Vert \leq\left\Vert G_{n}+\varepsilon 1_{n}
\right\Vert ^{\frac{1}{2}}\left\Vert \zeta\right\Vert$:
\begin{equation*}
    \frac{J_{n}\left(\alpha\right)}{n!}\leq \frac{1}{n^{\frac{n\alpha}{2}}}\left\Vert G_{n}+\varepsilon 1_{n}\right\Vert ^{\frac{n\alpha}{2}}\frac{1}{n!}\int_{\mathbb{R}^{n}}  d\zeta_{1}\ldots d\zeta_{n}\left\Vert \zeta\right\Vert ^{n\alpha}e^{-\frac{\left\Vert \zeta\right\Vert ^{2}}{2}+\sum\limits_{a,b=1}^{n}\left(R_{n}\right)_{ab}^{-1/2}
    \zeta_{a}\varphi\left(x_{b}\right)},
\end{equation*}
where the factor $\sqrt{\det\left(G_{n}+\varepsilon 1_{n}\right)}$ cancels out with the Jacobian after changing of variables in the integral. Here we also use the fact that $G_{n}$ is almost everywhere invertible. Hereafter we use the notation $\left(A_{n}\right)^{\beta}_{ab}$ for the matrix element of a power of matrix, namely:
\begin{equation}
    \left(A_{n}\right)^{\beta}_{ab} = \left(A_{n}^{\beta}\right)_{ab},     
\end{equation}
for $n\times n$ matrix $A_{n}$ and rational number $\beta$. 

Now we have to transform the part with source:
\begin{equation*}
    \left|\sum_{a,b=1}^{n}\left(R_{n} \right)^{1/2}_{ab}
    \zeta_{a}\varphi\left(x_{b}\right)\right|\leq
    \left\Vert \zeta\right\Vert \cdot\left\Vert \sum_{b=1}^{n}\left(R_{n}\right)^{1/2}_{ab}
    \varphi\left(x_{b}\right)\right\Vert.
\end{equation*}
We denote for the shortness (which will not get confused with the source due to index $0$):
\begin{equation*}
    j_{0}:=\left\Vert \sum_{b=1}^{n}\left(R_{n}\right)^{1/2}_{ab}
    \varphi\left(x_{b}\right)\right\Vert =\sqrt{\sum_{a,b=1}^{n}\left(R_{n}\right)_{ab}
    \varphi\left(x_{a}\right)\varphi\left(x_{b}\right)}\geq0,
\end{equation*}
where the last transition follows from the Euclidean norm definition. We underline that this quantity depends on $x$.

In this paper we will mainly consider the calculation of vacuum energy, which corresponds to the case $j=0$. Hence, we won't think about careful estimations for the case of nonzero source, and write the most rough estimation. Namely:
\begin{equation*}
    j_{0} \leq \sqrt{\left\Vert R_n \right\Vert} 
    \left\Vert \phi(x_a) \right\Vert \leq 
    \frac{n}{\sqrt{\varepsilon}}
    \sup\limits_{x\in\mathbb{R}^d} 
    \left|\phi(x)\right|,   
\end{equation*}
where we used the fact that $G_n\geq 0$, and hence maximal eigenvalue of $R_n$ is no more than $1/\varepsilon$. Here $\left\Vert\phi(x_a)\right\Vert$ is a Euclidean norm of the vector $\left(\phi(x_a)\right)_{a=1}^{n}$, which is bounded with $n \sup\limits_{x\in\mathbb{R}^d} \left|\phi(x)\right|$. Then we have:
\begin{equation}
    \frac{J_{n}\left(\alpha\right)}{n!}\leq
    \frac{1}{n^{\frac{n\alpha}{2}}}
    \left\Vert G_{n}+\varepsilon 1_{n} 
    \right\Vert ^{\frac{n\alpha}{2}}\frac{1}{n!}\int_{\mathbb{R}^{n}} 
    d\zeta_{1}\ldots d\zeta_{n}
    \left\Vert \zeta\right\Vert ^{n\alpha}
    e^{-\frac{\left\Vert \zeta\right\Vert ^{2}}{2}+
    \left\Vert \zeta\right\Vert j_{0}}.
    \label{Upper_bound_J_1}
\end{equation}
Next, we recall the inequality $\left\Vert G_{n}+\varepsilon 1_{n}\right\Vert \leq n(G\left(0\right)+\varepsilon)$. The integral in the right hand side of the expression (\ref{Upper_bound_J_1}) can be easily calculated via series expansion in $j_{0}$:
\begin{equation*}
    \frac{J_{n}\left(\alpha\right)}{n!}\leq
    \left(G\left(0\right)+
    \varepsilon\right)^{\frac{n\alpha}{2}}\frac{1}{n!}\frac{n\pi^{\frac{n}{2}}2^{\frac{1}{2}
    \left(n\alpha+n-2\right)}}{\varGamma\left(\frac{n}{2}+1\right)}\sum_{m=0}^{\infty}\frac{2^{\frac{m}{2}}\varGamma\left(\frac{1}{2}(m+n+n\alpha)\right)}{\varGamma(m+1)}\,j_{0}^{m}.
\end{equation*}

As a first step for bounding the obtained series, we use log-convexity of Gamma function $\varGamma$ for $m\geq1$:
\begin{equation*}
    \varGamma\left(\frac{n(\alpha+1)}{2}+
    \frac{m}{2}\right)\leq
    \varGamma\left(\frac{n(\alpha+1)}{2}+\frac{m+1}{2}\right)\leq\varGamma
    \left(n(\alpha+1)\right)^{1/2}
    \varGamma\left(m+1\right)^{1/2},
\end{equation*}
and rewrite:
\begin{equation*}
    \sum_{m=1}^{\infty}\frac{2^{\frac{m}{2}}
    \varGamma\left(\frac{1}{2}(m+n+n\alpha)\right)}{\varGamma(m+1)}\,j_{0}^{m}\leq
    \varGamma\left(n(\alpha+1)\right)^{1/2}
    \sum_{m=1}^{\infty}\frac{2^{m}}{\varGamma\left(m+1\right)^{1/2}}\,j_{0}^{m}.
\end{equation*}
Moreover, for $m\rightarrow\infty$ the equivalence relation holds:
\begin{equation*}
    \frac{\varGamma(m+1)^{1/2}}{2^{m/2}
    \varGamma\left(\frac{m+1}{2}\right)}\sim
    \frac{\sqrt{12m+1}}{2\sqrt{3}
    \sqrt[4]{2\pi}\sqrt[4]{m}}\rightarrow\infty,
\end{equation*}
and it could be numerically checked, that for $m\geq0$:
\begin{equation*}
    \frac{\varGamma(m+1)^{1/2}}{2^{m/2}
    \varGamma\left(\frac{m+1}{2}\right)}\geq\frac{1}{2},
\end{equation*}
so the following upper bound holds:
\begin{equation*}
    \sum_{m=0}^{\infty}\frac{2^{\frac{m}{2}}
    \varGamma\left(\frac{1}{2}(m+n+n\alpha)\right)}{\varGamma(m+1)}\,j_{0}^{m}\leq2
    \varGamma\left(n(\alpha+1)\right)^{1/2}
    \left(1+\sum_{m=1}^{\infty}\frac{1}
    {\varGamma\left(\frac{m+1}{2}\right)}\,
    j_{0}^{m}\right).
\end{equation*}
The last series can be calculated exactly in terms of the error function $\text{erf}$:
\begin{equation*}
    \sum_{m=1}^{\infty}\frac{1}
    {\varGamma\left(\frac{m+1}{2}\right)}\,j_{0}^{m}=
    e^{j_{0}^{2}}j_{0}(\text{erf}(j_{0})+1)\leq2
    e^{j_{0}^{2}}j_{0},
\end{equation*}
so we finish at:
\begin{equation*}
    \sum_{m=0}^{\infty}\frac{2^{\frac{m}{2}}
    \varGamma\left(\frac{1}{2}(m+n+n\alpha)\right)}{\varGamma(m+1)}\,j_{0}^{m}\leq2
    \varGamma\left(n(\alpha+1)\right)^{1/2}e^{j_{0}^{2}}(1+j_{0})
\end{equation*}

Let us write the result:
\begin{equation}
    \frac{J_{n}\left(\alpha\right)}{n!}\leq G\left(0\right)^{\frac{n\alpha}{2}}\frac{1}{n!}\frac{n\pi^{\frac{n}{2}}2^{\frac{1}{2}\left(n\alpha+n\right)}
    \varGamma\left(n(\alpha+1)\right)^{1/2}}
    {\varGamma\left(\frac{n}{2}+1\right)}
    e^{j_{0}^{2}}(j_{0}+1),
    \label{Upper_bound_J_2}
\end{equation}
Finding the asymptotic of the expression (\ref{Upper_bound_J_2}) right hand side, we get to:
\begin{equation*} 
\begin{split}
    \frac{J_{n}\left(\alpha\right)}{n!} & \leq\frac{1}{n^{3/2}}n^{\frac{n\alpha}{2}-n}G\left(0\right)^{\frac{n\alpha}{2}}\pi^{\frac{n-3}{2}}2^{\frac{1}{2}\left(n\alpha+3n-1\right)}(\alpha+1){}^{n(\alpha+1)-1/2}e^{(2-\alpha)n}e^{j_{0}^{2}}\left(j_{0} + 1\right)\\
    & =C_{n}n^{\frac{n\alpha}{2}-n}G\left(0\right)^{\frac{n\alpha}{2}}e^{j_{0}^{2}}\left(j_{0} + 1\right),
\end{split} 
\end{equation*} 
where $C_{n}>0$ is a dimensionless constant which grows no faster than exponentially.

As a result, we see that our series converges for $\alpha<2$. For $\alpha=2$ it also converges, but it could be proven in a much more simple way. Namely, it is easy to calculate GF $\mathcal{Z}$ for $\alpha=2$ exactly, which will be done further in the paper, and the corresponding expansion will converge. For $\alpha=2$ we will consider only the case $g(x)=g\chi_{Q}\left(x\right)$. As the result, we have some at least asymptotic expansion in powers of $g$. So, since even asymptotic expansions in the predetermined system of functions (power functions of the coupling constant $\{g^{n}\}_{n=0}^{\infty}$ in our case) are unique, then these two expansions must coincide. This proves that for $\alpha=2$ our perturbation series expansion also converges. So we have to deal with the coordinate integrals to finish the proof.

\subsubsection{Coordinate Part}

The next aim is to bound the result of coordinate integration. Namely, remembering the notation $g(\mathbb{R}^{d})=\int_{\mathbb{R}^{d}} dx\,g(x)$, we have:
\begin{equation*} 
\begin{split}
    \left|\mathcal{Z}_{I,\varepsilon,n}
    \left[j\right]\right| & 
    \leq\prod_{a=1}^{n}
    \int_{\mathbb{R}^{d}}dx_{a} \, 
    g (x_a)\,e^{-\frac{1}{2}\sum\limits_{a,b=1}^{n}
    \left(R_{n}\right)_{ab}\varphi\left(x_{a}\right)
    \varphi\left(x_{b}\right)}
    \frac{J_{n}\left(\alpha\right)}{n!}\leq 
    C_{n}n^{\frac{n\alpha}{2}-n} 
    G\left(0\right)^{\frac{n\alpha}{2}}\\
    &\times\prod_{a=1}^{n}
    \int_{\mathbb{R}^{d}}  dx_{a} \, g (x_a)\, e^{j_{0}^{2}/2}\left(j_{0} + 1\right).
\end{split} 
\end{equation*} 
Recalling the upper bound for $j_{0}$, we obtain finally:
\begin{equation}
    \left|\mathcal{Z}_{I,\varepsilon,n} 
    \left[j\right]\right|\leq 
    C_{n}n^{\frac{n\alpha}{2}-n} 
    G\left(0\right)^{\frac{n\alpha}{2}} 
    g(\mathbb{R}^{d})^n 
    \exp\left\{\frac{n^2}{2\varepsilon} 
    \left(\sup\limits_{x\in\mathbb{R}^d} 
    \left|\phi(x)\right|\right)^2\right\} 
    \left(\frac{n}{\sqrt{\varepsilon}}\sup\limits_{x\in\mathbb{R}^d} 
    \left|\phi(x)\right|+1\right),
    \label{final-majorant}
\end{equation}
so the obtained series for the GF $\mathcal{Z}_{I,\varepsilon}$ converges for $j=0$. As for $j\neq 0$, the convergence of series requires additional research and calculation of a more accurate majorant, since the obtained one increases infinitely, when $\varepsilon\rightarrow+0$, therefore, it can be argued that the series converges only for $\varepsilon\neq 0$. Though, as it was mentioned above, we mainly restrict ourselves to the case $j=0$.

\subsection{Exponentiation of Series Using Meyer Cluster Expansion}
\label{exponentiation-using-Meyer}

We have just obtained the converging PT series for the GF $\mathcal{Z}_{I,\varepsilon}$. In terms of asymptotic series, there is often proved ``the exponentiation of connected diagrams'', which means roughly that the GF $\mathcal{Z}_{I,\varepsilon}$ is an exponent of other regular (in the source $j$) functional. We are going to establish the same result in our case. It will be useful since after such exponentiation (for the coupling constant $g(x)=g\chi_{Q}\left(x\right)$) we will obtain only the first power of a volume $V=\text{Vol}\,{Q}$ rather than all natural powers (in general, the first power of $g(\mathbb{R}^{d})$). This fact will simplify significantly the extraction of physically measurable quantities, which will be done in the following sections of the paper.

\subsubsection{Formulation and Applicability}

We start with the following definition. Let $(X,\varSigma,\mu)$ be a measure space, that is a triple: $X$ is a set, for example, $\mathbb{R}^{N}$ with some natural number $N$, $\varSigma$ is a $\sigma$-algebra on $X$ and $\mu$ is a complex-valued measure (which shouldn't be confused with the UV-cutoff parameter $\mu$) on a measurable space $(X,\varSigma)$. Complex-valued measures are also called charges. We denote as $|\mu|$ the variation of $\mu$ (which shouldn't be confused with the total variation --- the value of $|\mu|$ on $X$). In the case, where $\mu(dy)=g(y)\nu(dy)$ and $\nu$ is a non-negative measure, $|\mu|(dy)=|g(y)|\nu(dy)$. Given a complex-valued measurable symmetric function $\zeta$ on $X\times X$, we introduce an abstract GCPF as follows:
\begin{equation}
    \mathcal{Z}\left[\mu,\zeta\right]=
    \sum\limits_{n=0}^{\infty}
    \frac{1}{n!}\int_{X}\mu\left(dy_{1}\right)\ldots
    \int_{X}\mu\left(dy_{n}\right)
    \prod_{1\leq i<j\leq n}
    \left(1+\zeta\left(y_{i},y_{j}\right)\right).
\end{equation}
The term $n=0$ in the sum is defined by $1$. In the case of classical gas with interaction, $n$ is the number of particles.

We denote by $\mathbb{G}_{n}$ the set of all (undirected, no loops) graphs with $n$ vertices, and $\mathbb{G}_{C,n}\subset\mathbb{G}_{n}$ the set of all connected graphs with $n$ vertices. Next, let us introduce the following combinatorial function on finite sequences $\left(x_{1},\ldots,x_{n}\right)$:
\begin{equation}
    \varphi\left(x_{1},\ldots,x_{n}\right)=
    \begin{cases}
    1, & \text{ if }n=1;\\
    \frac{1}{n!}
    \sum\limits_{\Gamma\in\mathbb{G}_{C,n}}
    \prod\limits_{(i,j)\in\Gamma}
    \zeta\left(x_{i},x_{j}\right), & 
    \text{ if }n\geqslant2.
\end{cases}
\end{equation}
The product here is taken over edges of $\Gamma$. A sequence $\left(x_{1},\ldots,x_{n}\right)$ is a cluster if the graph with $n$ vertices and two vertices $i$ and $j$ whenever $\zeta\left(x_{i},x_{j}\right)\neq0$ are connected. The cluster expansion allows to express the logarithm of an abstract partition function as a sum over clusters. Namely, the following theorem holds~\cite{ruelle}. 
\begin{Theorem}[Cluster expansion]
Assume that $|1+\zeta(u,y)|\leq 1$ for all $u,y\in X$, and that there is exist a non-negative function $a$ on $X$ such that for all $u\in X$,
\begin{equation}
    \int_{X}|\zeta(u,y)|e^{a(y)}
    |\mu|(dy)\leq a(u),
\end{equation}
and $\int_{X}e^{a(x)}|\mu|(dx)<\infty$. Then the following is true:
\begin{equation}
    \mathcal{Z}\left[\mu,\zeta\right]=
    \exp\left\{\sum\limits_{n=1}^{\infty}
    \int_{X}\mu\left(dy_{1}\right)\ldots\int_{X} \mu\left(dy_{n}\right)\varphi\left(y_{1},\ldots,
    y_{n}\right)\right\}.
\label{Meyer theorem main statement}
\end{equation}
Combined sum and integrals converge absolutely. Furthermore, we have for all $y_{1}\in X$:
\begin{equation}
    1+\sum\limits_{n=2}^{\infty}
    n\int_{X}|\mu|\left(dy_{2}\right)\ldots
    \int_{X} |\mu|\left(dy_{n}\right)
    \left|\varphi\left(y_{1},\ldots,y_{n}\right)\right|
    \leq e^{a\left(y_{1}\right)}.
\end{equation}
\end{Theorem}
Unfortunately, we cannot apply this theorem to the terms $\mathcal{Z}_{I,\varepsilon,n}$ of the PT series for GF $\mathcal{Z}_{I,\varepsilon}$, since there appears $\det  G_n$, spoiling the form of the terms in the GCPF expansion. So, we apply it to the terms $\mathcal{Z}_{I,\varLambda,\varepsilon,n}$ of the PT series for GF $\mathcal{Z}_{I,\varLambda,\varepsilon}[j]=\mathcal{Z}_{\varLambda,\varepsilon}\left[j\right]/
\mathcal{Z}_{0}\left[j\right]$ in the form before using Parseval--Plancherel identity. Hence, we start from ($G_{\varepsilon}=G\left(0\right)+\varepsilon$ for short):
\begin{equation*} 
\begin{split}
    \mathcal{Z}_{I,\varLambda,\varepsilon}
    \left[j\right]& \!=\!
    \sum_{n=0}^{\infty}\frac{1}{n!}
    \left\{\prod_{a=1}^{n}
    \underset{\int_{\mathbb{R}^{d+1}} 
    \mu(dx_{a}\,dt_{a})}
    {\underbrace{\int_{\mathbb{R}^{d}}  
    dx_{a}\,\left(-1\right)
    g\left(x_{a}\right)\int_{\mathbb{R}}  
    \frac{dt_{a}}{2\pi}\,
    \mathcal{F}\left[U_{\varLambda}
    \left(\phi\right)\right]
    \left(t_{a}\right)
    \exp\left(-it_{a}\varphi\left(x_{a}\right)-
    \frac{1}{2}G_{\varepsilon}    
    t_{a}^{2}\right)}}\right\}\\
    &\times\prod_{a<b}^{n}\exp\left(-G\left(x_{a}-x_{b}\right)t_{a}t_{b}\right).
\end{split} 
\end{equation*} 
Let us note that, for clarity from statistical physics point of view, we consider the case of translation invariant propagator $G(x_{a}-x_{b})$, but all the formulas presented below can be easily transferred to the general case of propagator $G(x_{a},x_{b})$. 

In our case the point $y=(x,t)$, the function $1+\zeta\left(y_{a},y_{b}\right)=e^{-\frac{1}{2}G\left(x_{a}-x_{b}\right)t_{a}t_{b}}$, and the measure $\mu$, appearing in the general theory, has the form: 
\begin{equation}
    \mu(dx_{a}\,dt_{a})=-\frac{dx_{a}\,dt_{a}}{2\pi}\,
    g\left(x_{a}\right)\mathcal{F}\left[U_{\varLambda}
    \left(\phi\right)\right]\left(t_{a}\right)
    \exp\left(it_{a}\varphi\left(x_{a}\right)-\frac{1}{2}
    G_{\varepsilon}t_{a}^{2}\right).
    \label{Def_of_com_val_meas}
\end{equation}
To use the theorem one have to find the function $a$ from its formulation. But in the following we will do all the manipulations directly and see that in our case all the transitions are valid without explicit presenting such a function. 

\subsubsection{Rewriting Series in Terms of Exponent}
\label{subsect-exp-rewriting}

Henceforth, we will keep in mind the measure definition (\ref{Def_of_com_val_meas}). Hereby, we start from the expansion of GF $\mathcal{Z}_{I,\varLambda,\varepsilon}$:
\begin{equation}
    \mathcal{Z}_{I,\varLambda,\varepsilon}
    \left[j\right] =\sum_{n=0}^{\infty}\frac{1}{n!}
    \left\{\prod_{a=1}^{n} \int_{\mathbb{R}^{d+1}} \mu(dx_{a}\,dt_{a})\right\}
    \prod_{a<b}^{n}\exp\left(-G\left(x_{a}-x_{b}\right)t_{a}t_{b}\right).
    \label{GFZ_exp_1}
\end{equation}
Using the definition of the function $\zeta$, we rewrite the expression (\ref{GFZ_exp_1}) as follows:
\begin{equation}
    \mathcal{Z}_{I,\varLambda,\varepsilon}
    \left[j\right]=
    \sum_{n=0}^{\infty}\frac{1}{n!}
    \left\{ \prod_{a=1}^{n}
    \int_{\mathbb{R}^{d+1}}  
    \mu(dx_{a}\,dt_{a})\right\} 
    \sum_{\Gamma\in\mathbb{G}_{n}}
    \prod_{\left(a,b\right)\in\Gamma}
    \zeta\left(x_{a},t_{a},x_{b},t_{b}\right).
    \label{GFZ_exp_2}
\end{equation} 
Further, the sum over all possible graphs $\Gamma$ in the right hand side of the expression (\ref{GFZ_exp_2}) can be represented in the following way:
\begin{equation}
    \sum_{\Gamma\in\mathbb{G}_{n}}
    \prod_{\left(a,b\right)\in\Gamma}
    \zeta\left(x_{a},t_{a},x_{b},t_{b}\right)=
    \sum_{k=1}^{n}\frac{1}{k!}\sum_{V_{1},\ldots,V_{k}}
    \sum_{\Gamma_{1},\ldots,\Gamma_{k}}\prod_{l=1}^{k}
    \prod_{\left(a,b\right)\in\Gamma_{l}}
    \zeta\left(x_{a},t_{a},x_{b},t_{b}\right).
    \label{GFZ_exp_3}
\end{equation}
In the expression (\ref{GFZ_exp_3}) we have decomposed the graph $\Gamma$ into the connected parts (graphs) $\left(\Gamma_{1},\ldots,\Gamma_{k}\right)$. Each $\Gamma_{i}\in\mathbb{G}_{C,n}$ is a connected graph with a set of vertices $V_{i}$ with the cardinality $n_{i}$. All the sets $V_{i}$ form a partition of the set $\{1,\ldots,n\}: V_{1}\cup\ldots\cup V_{k}=\{1,\ldots,n\}$, and $V_{i}\cap V_{j}=\varnothing$ if $i\neq j$. There are $k!$ such sequences for each $\Gamma$, since the order of the sets $\Gamma_{i}$ doesn't matter. The sum over $\Gamma$ can thus be realized by first summing over $k$, then over the partitions $V_{1},\ldots, V_{k}$, and then over connected graphs on the sets $V_{i}$. After substituting this decomposition into the GF $\mathcal{Z}_{I,\varLambda,\varepsilon}$ we can sum over the cardinalities $n_{1},\ldots,n_{k} \geq 1$ with condition $n_{1}+\cdots+n_{k}=n$. And the number of partitions of $n$ elements into $k$ subsets with $n_{1},\ldots,n_{k}$ elements is given by the multinomial coefficient $\frac{n!}{n_{1}!\ldots n_{k}!}$, which therefore should be inserted. As a result, the following chain of equalities for GF $\mathcal{Z}_{I,\varLambda,\varepsilon}$ is valid:
\begin{equation} 
\begin{split} 
    &\mathcal{Z}_{I,\varLambda,\varepsilon}
    \left[j\right]=
    1+\sum_{n=1}^{\infty}
    \frac{1}{n!}\sum_{k=1}^{n}
    \frac{1}{k!}\sum_{\underset{n_{1}+\ldots +n_{k}=n}
    {n_{1},\ldots,n_{k}}}
    \frac{n!}{n_{1}!\ldots n_{k}!}\\
    &\times\prod_{l=1}^{k}
    \left\{\sum_{\Gamma_{l}\in\mathbb{G}_{C,n_{l}}}
    \int_{\mathbb{R}^{d+1}}
    d\mu\left(dx_{1}\,dt_{1}\right)\ldots 
    \int_{\mathbb{R}^{d+1}}
    d\mu\left(dx_{n_{l}}\,dt_{n_{l}}\right)
    \prod_{\left(a,b\right)\in\Gamma_{l}}
    \zeta\left(x_{a},t_{a},x_{b},t_{b}\right)\right\}\\
    & =1+\sum_{k=1}^{\infty}\frac{1}{k!}
    \left\{\sum_{n=1}^{\infty}
    \frac{1}{n!}\sum_{\Gamma\in\mathbb{G}_{C,n}}
    \left\{ \prod_{a=1}^{n}
    \int_{\mathbb{R}^{d+1}}
    \mu(dx_{a}\,dt_{a})\right\} \prod_{\left(a,b\right)\in\Gamma}
    \zeta\left(x_{a},t_{a},
    x_{b},t_{b}\right)\right\}^{k}\\
    &=\exp\left\{\sum_{n=1}^{\infty}
    \frac{1}{n!}\sum_{\Gamma\in\mathbb{G}_{C,n}}
    \left\{\prod_{a=1}^{n}
    \int_{\mathbb{R}^{d+1}}
    \mu(dx_{a}\,dt_{a})\right\}
    \prod_{\left(a,b\right)\in\Gamma}
    \zeta\left(x_{a},t_{a},
    x_{b},t_{b}\right)\right\}.
\end{split} 
\end{equation} 

In accordance with the notations, introduced in the section \ref{phys-mot}, let us denote by $\mathcal{G}_{I,\varLambda,\varepsilon}=\ln\mathcal{Z}_{I,\varLambda,\varepsilon}$ the ``normalized'' GF of connected Green functions with regulations $\varLambda$ and $\varepsilon$. Applying cluster expansion we made above, we have the following formula for it:
\begin{equation}
    \mathcal{G}_{I,\varLambda,\varepsilon}
    \left[j\right]=
    \sum_{n=1}^{\infty}\frac{1}{n!}
    \sum_{\Gamma\in\mathbb{G}_{C,n}}
    \left\{\prod_{a=1}^{n}
    \int_{\mathbb{R}^{d+1}}
    \mu(dx_{a}\,dt_{a})\right\} \prod_{\left(a,b\right)\in\Gamma}
    \zeta\left(x_{a},t_{a},x_{b},t_{b}\right).
\end{equation}
To go further, recall the definition of the adjacency matrix $\nu_{ab}\left(\Gamma\right)$ for a given graph $\Gamma$:
\begin{equation}
    \nu_{ab}\left(\Gamma\right)=
    \begin{cases}
    1, & \left(a,b\right)\in\Gamma;\\
    0, & \left(a,b\right)\notin\Gamma.
    \end{cases}
\end{equation}
We note that $\nu_{aa}=0$, which is equivalent to our already existing requisition for the graph to have no loops. Using this definition, as well as the definition of the function $\zeta$, we can write and transform the following product:
\begin{equation}
\begin{split}
    \prod_{\left(a,b\right)\in\Gamma}\zeta
    \left(x_{a},t_{a},x_{b},t_{b}\right)&=
    \prod_{\left(a,b\right)\in\Gamma}
    \left(\exp\left(-G\left(x_{a}-x_{b}\right)t_{a}t_{b}\right)-1\right)\\
    &=\prod_{\left(a,b\right)\in\Gamma}
    \int_{0}^{1}ds_{ab}\,\partial_{s_{ab}}
    \exp\left(-s_{ab}G\left(x_{a}-x_{b}\right)
    t_{a}t_{b}\right)\\
    &=\left\{\prod_{a<b}^{n}
    \int_{0}^{1}ds_{ab}\,
    \partial_{s_{ab}}^{\nu_{ab}
    \left(\Gamma\right)}\right\}
    \exp\left\{-\sum\limits_{a<b}^{n} 
    s_{ab}\nu_{ab}(\Gamma)G\left(x_{a}-x_{b}\right)
    t_{a}t_{b}\right\}
\end{split}
\end{equation} 
We simply have presented the differences as integrals of derivatives ($s_{ab}$ are the auxiliary variables), using the Fundamental Theorem of Calculus. We have also expanded the sum in the exponent to all graph edges and added $\nu_{ab}(\Gamma)$ multipliers that do not affect the terms for presenting edges and annihilate the contributions of the terms corresponding to the absent ones. Further, we understand the powers of differential operators $\partial_{s_{ab}}$ as follows:
\begin{equation}
    \partial_{s_{ab}}^{\nu_{ab}\left(\Gamma\right)}=
    \begin{cases}
    \partial_{s_{ab}}, & \left(a,b\right)\in\Gamma;\\
    1, & \left(a,b\right)\notin\Gamma.
    \end{cases}
\end{equation}
This means exactly that the edges $(a,b)$, that are absent in the graph $\Gamma$, contribute only by $1$, since for them $\nu_{ab}(\Gamma)=0$. And as for the presenting edges, they give the desired difference after the action of the integral of the derivative. Such rewriting of the products provides an unification of the expressions and will be extremely useful for obtaining compact forms of the results that follow in the paper. Finally, in accordance with section \ref{summary}, we recall the convenient notation: 
\begin{equation}
\label{G-Gamma-def}
    \left(G_{n,\Gamma}\right)_{ab}=
    s_{ab}\nu_{ab}\left(\Gamma\right)
    G\left(x_{a}-x_{b}\right)+
    G_{\varepsilon}\delta_{ab}.
\end{equation}
This quantity depends on $\varepsilon$ and $s_{ab}$, but we won't specify it in the notations for shortness.

Traveling back to the GF $\mathcal{G}_{I,\varLambda,\varepsilon}$ in terms of the PT series $\mathcal{G}_{I,\varLambda,\varepsilon} =\sum\limits_{n=1}^{\infty}
\mathcal{G}_{I,\varLambda,\varepsilon,n}$, we find for the $n$-particle contribution, which is the contribution of all connected graphs with $n$ vertexes, the following expression:
\begin{equation*}
\begin{split}
    \mathcal{G}_{I,\varLambda,\varepsilon,n}
    \left[j\right]& \!=\!
    \frac{1}{n!}\sum_{\Gamma\in\mathbb{G}_{C,n}}
    \left\{\prod_{a<b}^{n}\int_{0}^{1}ds_{ab}\,
    \partial_{s_{ab}}^{\nu_{ab}\left(\Gamma\right)}\right\} 
    \left\{ \prod_{a=1}^{n}\int_{\mathbb{R}^{d+1}}  \mu(dx_{a}\,dt_{a})\right\} 
    \exp\left\{-\sum_{a<b}^{n} \left(G_{n,\Gamma}\right)_{ab}t_{a}t_{b}\right\}.
\end{split}
\end{equation*}
We underline that in the PT series for the connected Green functions GF $\mathcal{G}_{I,\varLambda,\varepsilon}$ the index $n$ starts from one rather than zero, as in the PT series for the complete Green functions GF $\mathcal{Z}_{I,\varLambda,\varepsilon}$ in (\ref{GFZ_exp_1}). It is not a mistake, and it arrives from the very statement (\ref{Meyer theorem main statement}) of cluster expansion theorem.

After substituting the measure $\mu(dx_a \, dt_a)$ definition (\ref{Def_of_com_val_meas}), we arrive at the following expression for the $n$-particle contribution:
\begin{equation} 
\label{HSG-start}
\begin{split}
    &\mathcal{G}_{I,\varLambda,\varepsilon,n}
    \left[j\right]
    =\frac{1}{n!}\sum_{\Gamma\in\mathbb{G}_{C,n}}
    \left\{\prod_{a<b}^{n}
    \int_{0}^{1}ds_{ab}\,
    \partial_{s_{ab}}^{\nu_{ab}
    \left(\Gamma\right)}\right\}
    \left\{\prod_{a=1}^{n}
    \int_{\mathbb{R}^{d}}dx_{a}\,
    \left(-1\right)g\left(x_{a}\right)\right\}\\ 
    &\times\left\{\prod_{a=1}^{n}
    \int_{\mathbb{R}}\frac{dt_{a}}{2\pi}\mathcal{F}\left[U_{\varLambda}
    \left(\phi\right)\right](t_{a})\exp\left(it_{a}\varphi\left(x_{a}\right)-
    \frac{1}{2}G_\varepsilon t_{a}^{2}\right)\right\}  
    \exp\left\{-\sum_{a<b}^{n} \left(G_{n,\Gamma}\right)_{ab}t_{a}t_{b}\right\},
\end{split} 
\end{equation} 
and, after applying Parseval--Plancherel identity to integrals over $t_a$, the expression for the $n$-particle contribution becomes: 
\begin{equation} 
\begin{split} 
    \label{G_I-prefinal}
    &\mathcal{G}_{I,\varLambda,\varepsilon,n}
    \left[j\right]=
    \frac{\left(-1\right)^{n}}{n!(2\pi)^{n/2}}
    \sum_{\Gamma\in\mathbb{G}_{C,n}}
    \left\{ \prod_{a<b}^{n}\int_{0}^{1}ds_{ab}\,
    \partial_{s_{ab}}^{\nu_{ab}\left(\Gamma\right)}\right\} 
    \left\{ \prod_{a=1}^{n} \int_{\mathbb{R}^{d}} dx_{a}\, g\left(x_{a}\right)\right\} \frac{1}{\sqrt{\det\left(G_{n,\Gamma}\right)}} \\
    &\times\left\{\prod_{a=1}^{n}
    \int_{\mathbb{R}} d\phi_{a}\,U_{\varLambda}
    \left(\phi_{a}\right)\right\}
    \exp\left\{-\frac{1}{2}\sum_{a,b=1}^{n}\left(\left(G_{n,\Gamma}\right)^{-1}\right)_{ab}
    \left(\phi_{a}-\varphi(x_{a})\right)
    \left(\phi_{b}-\varphi(x_b)\right)\right\}. 
\end{split} 
\end{equation} 
This formula is not a very convenient one since $s_{ab}$ variables are included in a very complicated way because of $\det\left(G_{n,\Gamma}\right)$ and $G_{n,\Gamma}^{-1}$. Though, this expression is a very important one. Indeed, it explains that it is sufficient to consider the terms $\mathcal{Z}_{I,\varLambda,\varepsilon,n}$ in the initial (unexponentiated) GF $\mathcal{Z}_{I,\varLambda,\varepsilon}$, and then, in order to obtain $\mathcal{G}_{I,\varLambda,\varepsilon,n}$, it is enough to apply the operator $\mathcal{O}$ (everything that the operator depends on is omitted in the notation):
\begin{equation}
\label{difference-operator}
    \mathcal{O}=\sum\limits_{\Gamma\in\mathbb{G}_{C,n}}
    \left\{\prod\limits_{a<b}^{n}\int_{0}^{1}ds_{ab}\,
    \partial_{s_{ab}}^{\nu_{ab}\left(\Gamma\right)}\right\},
\end{equation}
and we will use this notation in the following.

Summarising, we can get the results for the connected Green functions GF $\mathcal{G}_{I,\varLambda,\varepsilon}$ from the ones for $\mathcal{Z}_{I,\varLambda,\varepsilon}$ with the following steps:
\begin{enumerate}
    \item changing $\left(G_{n}\right)_{ab}\rightarrow\left(G_{n,\Gamma}\right)_{ab}$ in the \textbf{quadratic} part of the action;
    \item posterior applying the operator $\mathcal{O}$.
\end{enumerate}
Equivalently, on the language of formulas:
\begin{equation}
    \mathcal{G}_{I,\varLambda,\varepsilon,n}
    [\,j,\left(G_{n}\right)_{ab}]=
    \mathcal{O} 
    \mathcal{Z}_{I,\varLambda,\varepsilon,n}
    [\varphi,\left(G_{n,\Gamma}\right)_{ab}].
\label{spiritual formuas for G and Z connection}
\end{equation}
We additionally declared in the arguments of GFs the matrices $G_n$ and $G_{n,\Gamma}$ we use in the obtained perturbative expansions to avoid confusions. The notation $\mathcal{Z}_{I,\varLambda,\varepsilon,n}[\varphi,\left(G_{n,\Gamma}\right)_{ab}]$ means that this functional must be expressed in terms of $\phi$, which is fixed, and the replacement of $\left(G_{n,\Gamma}\right)_{ab}$ is made only in explicit dependence.

Let us also notice the form of $\mathcal{G}_{I,\varLambda,\varepsilon,n}$ without introducing new variables $s_{ab}$:
\begin{equation}
    \mathcal{G}_{I,\varLambda,\varepsilon,n}
    \left[j\right]=
    \frac{1}{n!}\sum_{\Gamma\in\mathbb{G}_{C,n}} 
    \left\{\prod_{a=1}^{n}\int_{\mathbb{R}^{d+1}}
    \mu(dx_{a}\,dt_{a})\right\}   
    \prod\limits_{a<b}^{n}
    \left(\exp\left(-\nu_{ab}
    \left(\Gamma\right)G\left(x_{a}-x_{b}\right)t_{a}t_{b}\right)-1\right).
    \label{GFZ_exp_4}
\end{equation}
The expression (\ref{GFZ_exp_4}) will be useful for the following HSG approximation.

\subsubsection{Final Form of the Connected Green Functions GF}

For the convenience, in accordance with section \ref{summary}, we recall the notation $\mathcal{Q}_{n,\Gamma}$ for the $n$-particle quantum entangler:
\begin{equation}
    \mathcal{Q}_{n,\Gamma}(x,\phi)=
    \frac{1}{2}\sum\limits_{a,b=1}^{n}
    \left(G_{n,\Gamma}\right)_{ab}^{-1}
    \phi_{a}\phi_{b},
\end{equation}
and also we introduce the notation for the linear functional in vector $\phi_{a}$:
\begin{equation}
    l\left(x,\phi\right)=
    \sum_{a,b=1}^{n}\left(G_{n,\Gamma}\right)_{ab}^{-1}
    \phi_a \varphi(x_{b})=\sum_{a=1}^{n}\phi_a \chi_{a},\quad
    \chi_{a}=\sum_{b=1}^{n}\left(G_{n,\Gamma}\right)_{ab}^{-1}
    \varphi(x_{b}).
\end{equation}
Basically, $\chi_{a}$ are the sources in the coordinate representation, which are acted by the operator $G$, after which we ``pull them back'' by the discrete restriction of $G$ ``modulated'' as in the expression (\ref{G-Gamma-def}) inverse.

Now we are going to substitute the introduced notations in the expression (\ref{G_I-prefinal}) as well as remove regulator $\varLambda$ from it. We are eligible for the last action since the majorant for $\mathcal{Z}_{I,\varLambda,\varepsilon}$ doesn't depend on $\varLambda$ for generic $j$ as well as on $\varepsilon$ for $j=0$. Hence, we have:
\begin{equation} \begin{split} 
    &\mathcal{G}_{I,\varepsilon,n}\left[j\right] =\frac{\left(-1\right)^{n}}{n!(2\pi)^{n/2}}\,
    \mathcal{O}\left\{\left\{ \prod_{a=1}^{n}
    \int_{\mathbb{R}^{d}} dx_{a}\, 
    g\left(x_{a}\right)\right\} \frac{1}{\sqrt{\det\left(G_{n,\Gamma}\right)}}\right.\\
    & \left.\times\exp
    \left(-\mathcal{Q}_{n,\Gamma}(x,\varphi)\right)
    \left\{\prod_{a=1}^{n}
    \int_{\mathbb{R}}d\phi_{a}\,
    \left|\phi_{a}\right|^\alpha \right\}
    \exp\left(-\mathcal{Q}_{n,\Gamma}(x,\phi)
    +l\left(x,\phi\right)\right)\right\},
    \label{G_I_e_n_j}
\end{split} \end{equation} 
for an arbitrary source $j$. Here, exactly as in the (\ref{Z_I_eps_final}), we introduced the notation:
\begin{equation*}
    \mathcal{G}_{I,\varepsilon}\left[j\right]=
    \mathcal{G}_{\varLambda\rightarrow\infty,\varepsilon}
    \left[j\right] - \mathcal{G}_{0}\left[j\right]
\end{equation*} 
And for $j=0$ we can remove all the regulators and write: 
\begin{equation} 
    \mathcal{G}_{I,n}\left[0\right]\!=\!
    \frac{\left(-1\right)^{n}}{n!(2\pi)^{n/2}}\,
    \mathcal{O}
    \left\{ \left\{ \prod_{a=1}^{n}
    \int_{\mathbb{R}^{d}} dx_{a}\, 
    g\left(x_{a}\right)\right\}
    \left\{ \prod_{a=1}^{n}
    \int_{\mathbb{R}} d\phi_{a}\, 
    \left|\phi_{a}\right|^\alpha \right\}
    \frac{\exp\left(-\mathcal{Q}_{n,\Gamma}(x,\phi)\right)}{\sqrt{\det\left(G_{n,\Gamma}\right)}}\right\}.
\end{equation}
These formulas will be used intensively in the section \ref{pt-calc}.

\subsubsection{A Short Way to Obtain Coefficients of Exponentiation}

At the end of the section, consider again the coupling constant $g(x)=g\chi_{Q}\left(x\right)$. If the expression $\mathcal{Z}\left(V,G,g\right)=e^{V\cdot f\left(G,g\right)}$ is proved, then the following equality is true:
\begin{equation}
    \mathcal{Z}\left(V,G,g\right)=
    1+Vf\left(G,g\right)+\frac{1}{2}V^{2}f\left(G,g\right)^{2}+\ldots
\end{equation}
So, the thing is that we can extract $f\left(G,g\right)$ as the coefficient in $V$ if we get some expansion of $\mathcal{Z}\left(V,G,g\right)$ in powers of $V$. This follows from the uniqueness of (asymptotic) expansion in a prescribed system of functions (power functions of the volume $\{V^{k}\}_{k=0}^{\infty}$ in our case).