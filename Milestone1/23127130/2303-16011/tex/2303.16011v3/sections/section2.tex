\section{Summary of Problem and Obtained Results} 
\label{summary}

In this paper, Euclidean quantum theory of scalar field $\phi$ in $\mathbb{R}^{d}$ with action $S\left[\phi\right]$:
\begin{equation}
\begin{split}
\label{action}
    S[\phi] & =S_{0}[\phi]+S_{I}[\phi]=\frac{1}{2}
    \int_{\mathbb{R}^{d}}\int_{\mathbb{R}^{d}} dx\,dy\, L(x,y)\phi(x)\phi(y)\\ & +
    \int_{\mathbb{R}^{d}} dx\, g(x)\left|\phi(x)\right|^{\alpha},\quad \alpha\in(1,2),
\end{split}
\end{equation}
is considered. Here $\phi: \mathbb{R}^{d} \rightarrow \mathbb{R}$ --- real-valued function (scalar field), $g: \mathbb{R}^{d} \rightarrow \mathbb{R}$ --- real-valued function (IR coupling constant), $j: \mathbb{R}^{d} \rightarrow \mathbb{R}$ --- real-valued function (source, appearing  in the expression below), $L=G^{-1}$ --- inverse propagator, defined on the space of scalar fields $\varPhi$. We don't specify this space in the present section, but in the following sections we will assume that $\varPhi$ is a separable HS. \textbf{Convention}: hereafter we use the same letter for an operator and its integral kernel, for example, $L$ and $L(x,y)$, and we also use the same physical term for an operator and its integral kernel, for example, the term ``propagator'' for $G$ and $G(x,y)$. It is always clear from the context what exactly we are talking about. Also in the paper we will assume that $L(x,y)=L(y,x)$, hence $G(x,y)=G(y,x)$. Finally, the propagator $G$ is a trace class operator and in all the final calculations, where possible, we will assume the translational invariance of the Gaussian theory, which implies $G(x,y)=G(x-y)$.

Complete (including disconnected parts) Green functions GF $\mathcal{Z}$ in terms of functional integral over primary (opposite to composite) field $\phi$ in such a theory:
\begin{equation}
\begin{split}
    \mathcal{Z}\left[j\right] & =\int_{\varPhi}
    \mathcal{D}\left[\phi\right]\,\exp\left(-\frac{1}{2}
    \int_{\mathbb{R}^{d}}\int_{\mathbb{R}^{d}} dx\,dy\, L(x,y)\phi(x)\phi(y)\right.\\ & \left. -
    \int_{\mathbb{R}^{d}} dx\,g(x)
    \left|\phi\left(x\right)\right|^{\alpha}+
    \int_{\mathbb{R}^{d}} dx\, j\left(x\right)
    \phi\left(x\right)\right),
    \label{Complete_Green_functions_GF_Z}
\end{split}
\end{equation}
is derived in terms of convergent perturbation series in powers of coupling constant $g$, which is a function of $x$ in the general case:
\begin{equation}
\begin{split}
    \mathcal{S}_{\varepsilon}
    \left[\varphi\right]&=
    \sum_{n=0}^{\infty}
    \frac{\left(-1\right)^{n}}{n!}
    \left\{\prod_{a=1}^{n}
    \int_{\mathbb{R}^{d}} dx_{a}\,g(x_{a})
    \int_{\mathbb{R}}d\phi_{a}
    \left|\phi_{a}\right|^{\alpha}\right\}
    \frac{1}{\sqrt{\left(2\pi\right)^{n}
    \det\left(G_{n} +
    \varepsilon 1_{n} \right)_{ab}}}\\
    &\times\exp\left\{-\frac{1}{2}
    \sum\limits_{a,b=1}^{n}
    \left(G_{n} +
    \varepsilon 1_{n}\right)_{ab}^{-1}
    \left[\phi_{a}-\varphi\left(x_{a}\right)\right]
    \left[\phi_{b}-\varphi\left(x_{b}\right)\right]\right\}.
    \label{Interaction_scattering_matrix_S}
\end{split}
\end{equation}

In the expression (\ref{Complete_Green_functions_GF_Z}) we use the notation ``big differential'' under the integral sign only formally: It is well known that there is no Lebesgue measure on infinite-dimensional separable HS and BS. However, together with the part of the integrand, the Gaussian exponent, this notation (up to a constant) means an integral over the Gaussian measure $\gamma_{G}$ with zero mean and covariance operator (propagator) $G$ in the corresponding space. This is how the expression (\ref{Complete_Green_functions_GF_Z}) should be understood in the framework of a rigorous mathematical theory. Details are provided in the following sections.

In the expression (\ref{Interaction_scattering_matrix_S}) the following notations are introduced:
\begin{enumerate}
\item $\mathcal{S}_{\varepsilon}\left[\varphi\right]= \mathcal{Z}\left[j\right]/\mathcal{Z}_{0}\left[j\right]$ 
is the interaction scattering matrix (hereinafter --- $\mathcal{S}_{\varepsilon}$-matrix) and $\mathcal{Z}_{0}\left[j\right]$ is the Gaussian GF;
\item $G_{n}$ is $n\times n$ matrix with elements $\left(G_{n}\right)_{ab}=G(x_{a},x_{b})$ is the restriction of a propagator on finite-dimensional space $\mathbb{R}^{2n}$;
\item $\varphi(x)=Gj\left(x\right)=
\int_{\mathbb{R}^{d}} dy\,
G\left(x,y\right)j\left(y\right)$ is the classical field ($\mathcal{S}_{\varepsilon}$-matrix argument) corresponding to the source $j$;
\item $\varepsilon>0$ is the parameter for defining integrands on null sets (in final results $\varepsilon\rightarrow+0$) and $1_{n}$ is $n\times n$ identity matrix with elements $\delta_{ab}$ (Kronecker delta).
\end{enumerate}

Further in the paper we have proved the existence of the connected Green functions GF $\mathcal{G}_{I,\varepsilon}\left[\varphi\right]=
\ln{\left(\mathcal{Z}\left[j\right]/
\mathcal{Z}_{0}\left[j\right]\right)}$ (index $I$ means "interaction part"). At the same time, it is convenient to consider this functional depending on $\varphi$, but not on $j$. Let us note that another definition of the connected Green functions GF is also introduced in the literature $\mathcal{G}\left[j\right]=
\ln{\left(\mathcal{Z}\left[j\right]/
\mathcal{Z}_{0}\left[j=0\right]\right)}$~\cite{vasil2004field,kopbarsch}. At the same time, such a functional is considered depending on $j$. \textbf{Convention}: hereafter we choose a normalization of functional integration such that the value $\mathcal{Z}_{0}\left[j=0\right]=1$. This choice corresponds to the normalization of the Gaussian measure $\gamma_{G}$ in $\varPhi$.

Further, PT series for $\mathcal{G}_{I,\varepsilon}\left[\varphi\right]=
\sum\limits_{n=0}^{\infty}\mathcal{G}_{I,\varepsilon,n}\left[\varphi\right]$, and the nth term of the series has the following form:
\begin{equation}
\begin{split}
    \mathcal{G}_{I,\varepsilon,n}\left[\varphi\right]&=
    \frac{\left(-1\right)^{n}}{n!}
    \sum_{\Gamma\in\mathbb{G}_{C,n}}
    \left\{\prod_{a<b}^{n}\int_{0}^{1}
    ds_{ab}\,\partial_{s_{ab}}^{\nu_{ab}
    \left(\Gamma\right)}\right\}
    \left\{\prod_{a=1}^{n}
    \int_{\mathbb{R}^{d}} dx_{a}\,g\left(x_{a}\right)
    \int_{\mathbb{R}} d\phi_{a}
    \left|\phi_{a}\right|^{\alpha}\right\}\\
    &\times\frac{1}{\sqrt{\left(2\pi\right)^{n}
    \det\left(G_{n,\Gamma}\right)_{ab}}}
    \exp\left\{-\frac{1}{2}
    \sum\limits_{a,b=1}^{n}
    \left(G_{n,\Gamma}\right)_{ab}^{-1}
    \left[\phi_{a}-\varphi\left(x_{a}\right)\right]
    \left[\phi_{b}-\varphi\left(x_{b}\right)\right]\right\}.
    \label{Modified_connected_Green_functions_GF_G}
\end{split}
\end{equation}

In the expression (\ref{Modified_connected_Green_functions_GF_G}) the following notations are introduced:
\begin{enumerate}
\item $\mathbb{G}_{C,n}$ is the set of all connected undirected graphs with no loops and multiple edges on $n$ vertices; 
\item $\nu_{ab}\left(\Gamma\right)$ is an adjacency matrix of a graph $\Gamma$ (the differential operator $\partial_{s_{ab}}$ is raised to the power of this quantity);
\item $\left(G_{n,\Gamma}\right)_{ab}=
s_{ab}\nu_{ab}\left(\Gamma\right)\left(G_{n}\right)_{ab} + \left( G(0) + \varepsilon \right) \delta_{ab}$, where $G(0)$ is the $G(x,x)$ in the translation invariant case;
\item $s_{ab}$ are auxiliary variables, using for extracting connected contributions to $\mathcal{G}_{I,\varepsilon,n}\left[\varphi\right]$.
\end{enumerate}

Further, we have calculated the first two orders in $g(x)=g\chi_{Q}\left(x\right)$, where $\chi_{Q}$ is the indicator function of $d$-dimensional cube centered at the origin with $\text{Vol}\,{Q}=V$, for the vacuum energy density $w_{vac}=\mathcal{E}/V$, namely (italic $\varGamma$ is the Gamma function, ${}_{2}F_{1}$ is the Gauss hypergeometric function and $\mathcal{E}=-\mathcal{G}_{I,\varepsilon}\left[0\right]$ is the vacuum energy): 
\begin{equation}
\begin{split}
    &w_{vac}=\frac{gG\left(0\right)^{\frac{\alpha}{2}}
    2^{\frac{1+\alpha}{2}}
    \varGamma\left(\frac{\alpha+1}{2}\right)}
    {\left(2\pi\right)^{1/2}}-
    \frac{g^{2}G(0)^{\alpha}
    2^{\alpha-1}\varGamma
    \left(\frac{\alpha+1}{2}\right)^{2}}{\pi}\\
    &\times\int_{Q}dx\,
    \left\{\left(1-\frac{G(x)^{2}}
    {G(0)^{2}}\right)^{\alpha+\frac{1}{2}}
    {}_{2}F_{1}\left(\frac{\alpha+1}{2},
    \frac{\alpha+1}{2};\frac{1}{2};
    \left(\frac{G\left(x\right)}{G\left(0\right)}\right)^{2}\right)-1\right\}
    +O(g^{3}),
\end{split}
\end{equation}
and applied these formulas for the cases of virton propagator~\cite{efimov1985problems,efimov2004blokhintsev} (exponential form factor, which is a smooth UV-cutoff function in momentum space with UV parameter $\mu$) and Euclidean Klein--Gordon propagator with mass $m$. 

For the Klein--Gordon propagator with mass $m$ and sharp UV-cutoff function in momentum space with UV parameter $\mu$ when the modes are frozen at $|k|>\mu$ and the propagator in momentum representation $G(|k|>\mu)=0$, we have obtained for the vacuum energy density $w_{vac}$ the following expression:
\begin{equation}
\begin{split}
    & w_{vac}\,m^{-d}=
    a_{1}\frac{gG(0)^{\alpha/2}}{m^{d}}+a_{2}\left(\frac{gG(0)^{\alpha/2}}{m^{d}}\right)^{2}
    \left(\frac{m^{d-2}}{G(0)}\right)^{2}+O(g^{3}), \\
    & a_{1}=2^{\frac{\alpha+1}{2}}
    \varGamma\left(\frac{\alpha+1}{2}\right),\quad a_{2}=\frac{2^{\alpha-1}}{\pi}
    \varGamma\left(\frac{\alpha+1}{2}\right)^{2}
    \begin{cases}
    \frac{1}{4}, & d=2;\\
    \frac{1}{8}, & d=3.
    \end{cases}   
\end{split}
\end{equation}

Besides, we have provided the following polynomial formula for $\mathcal{G}_{I,\varepsilon,n}\left[\varphi\right]$, derived by Legendre polynomial $P_{n}$ expansion (with an arbitrary coupling constant $g$):
\begin{equation}
    P_{n}(t)=2^{n}\sum_{k=0}^{n}\binom{n}{k}
    \binom{\frac{n+k-1}{2}}{n}t^{k}, \quad t\in[-1,1],
\end{equation}
\begin{equation}
\begin{split}
    &\mathcal{G}_{I,\varepsilon,n}\left[\varphi\right]=
    \frac{\left(-1\right)^{n}}{n!}
    \sum_{\Gamma\in\mathbb{G}_{C,n}}
    \left\{ \prod_{a<b}^{n}\int_{0}^{1}ds_{ab}\,
    \partial_{s_{ab}}^{\nu_{ab}
    \left(\Gamma\right)}\right\} 
    \left\{ \prod_{a=1}^{n}\int_{\mathbb{R}^{d}} 
    dx_{a}\,g(x_{a})\right\}e^{-\mathcal{Q}_{n,\Gamma}(x,\varphi)}\\
    &\times\sum_{k=0}^{\infty}
    \frac{2^{n\alpha/2}G(0)^{2k+n\alpha/2}}{4^{k}k!}\frac{\left(\frac{n(\alpha+1)}{2}\right)_{2k}}{\left(1/2\right)_{2k}}
    \sum_{\nu_{1}+\ldots+\nu_{n}=2k}
    \binom{2k}{\nu_{1}\ldots\nu_{n}}\chi_{1}^{\nu_{1}}\ldots
    \chi_{n}^{\nu_{n}}\sum_{q=0}^{\infty}
    \sum_{i=0}^{q}\left(2q+\frac{1}{2}\right)\\
    &\times\frac{2^{4q-ni+1}
    \varGamma\left(\frac{n(\alpha+1)}{2}\right)}
    {\varGamma\left(\frac{n}{2}+ni+k\right)}
    \sum_{p=0}^{q}\binom{2q}{2p}
    \binom{q+p-\frac{1}{2}}{2q}\frac{1}{2p+\alpha+1}
    \binom{2q}{2i}\binom{q+i-\frac{1}{2}}{2q}\\
    &\times\sum\limits_{\{l_{ab}\}}2^{-\sum\limits_{a=1}^{n}
    \frac{l_{aa}}{2}}
    \frac{\prod\limits_{a=1}^{n}
    \left(2i+\nu_{a}\right)!}
    {\prod\limits_{a<b}^{n}\left(l_{ab}!\right)
    \prod\limits_{a=1}^{n}\left(\frac{l_{aa}}{2}\right)!}
    \prod_{a<b}\left(\frac{\left(G_{n,\Gamma}\right)_{ab}}
    {G(0)}\right)^{l_{ab}}.
    \label{Modified_connected_Green_functions_GF_G_Legendre}
\end{split}
\end{equation}

In the expression (\ref{Modified_connected_Green_functions_GF_G_Legendre}) the following notations are introduced:
\begin{enumerate}
\item $\mathcal{Q}_{n,\Gamma}(x,\varphi)=\frac{1}{2}\sum\limits_{a,b=1}^{n}
\left(G_{n,\Gamma}\right)_{ab}^{-1}\varphi\left(x_{a}\right)\varphi\left(x_{b}\right)$ is the $n$-particle quantum entangler --- the main difficulty in calculating any GF;
\item $\chi_{a}=\sum\limits_{b=1}^{n}
\left(G_{n,\Gamma}\right)^{-1}_{ab}\varphi\left(x_{b}\right)$ is the auxiliary vector quantity, introduced for the compactness of formulas;
\item $(x)_k := x(x+1)(x+2)\ldots(x+k-1)$, and $(x)_0 := 1$ is the rising factorial;
\item $\binom{2k}{\nu_{1}\ldots\nu_{n}}$ is the multinomial coefficient, in particular, $\binom{q}{p}$ is the binomial coefficient;
\item \textbf{Convention}: summation in the last line of the expression (\ref{Modified_connected_Green_functions_GF_G_Legendre}) is carried out over all $l_{ab}\geq 0$ satisfying the conditions $\sum\limits_{b=1}^{n}l_{ab}=2i+\nu_{a}$ (where $l_{ab}=l_{ba}$) and $l_{aa}$ is even. This expression is non-zero for even $\sum\limits_{a=1}^{n}\nu_{a}$ and zero for odd.
\end{enumerate}

This yields the simple approximation formula for $w_{vac}$ if one chooses the degree of approximation polynomial is equal to $2$ and again $g(x)=g\chi_{Q}\left(x\right)$:
\begin{equation}
\begin{split}
 &w_{vac}\approx\frac{3 (2 \alpha +1) 
    \varGamma \left(\frac{\alpha +1}{2}\right)}
    {\sqrt{\pi } (\alpha +1) (\alpha +3)}
    \left(2gG(0)\right)^{\alpha /2} + 
    \frac{15\alpha}{(\alpha+1)(\alpha+3)} \\ 
    &\times
    \sum\limits_{n=2}^\infty
    \frac{(-1)^{n-1}2^{n\alpha/2}}{4n}\,
    \left[gG(0)^{\alpha/2}\right]^{n}
    \frac{\varGamma\left(\alpha_n\right)}
    {\varGamma\left(3n/2\right)}
    \left\{\prod\limits_{a=1}^{n-1}
    \int_{Q} dy_{a}\,\frac{G(y_{a})}{G(0)}\right\} 
    \frac{G\left(\sum\limits_{k=1}^{n-1}y_{k}\right)}{G(0)}.
\end{split}
\end{equation}

Beyond that, we derived another approximation formula for $w_{vac}$, based on HSG approximation of the matrix $G_{n}$. This approximation is quite often used in statistical physics. Supposing for $G_{n}$ the ansatz:
\begin{equation}
    \left(G_n\right)_{ab}\approx
    \left(\gamma\left(J_n\right)_{ab}+
    \left(1-\gamma\right)\delta_{ab}\right)
    G(0)\theta\left(\delta-
    \left|x_{a}-x_{b}\right|\right),
\label{HSG-def}
\end{equation}
with $n\times n$ matrix $J_{n}$ of ones, Heaviside step function $\theta$ and parameters $\delta$ and $\gamma$, we arrive at the following expression for $w_{vac}$:
\begin{equation}
\begin{split}
    &w_{vac}\!=\!-\sum_{n=1}^{\infty}
    \frac{\left(-g\right)^{n}
    G(0)^{n\alpha/2}}{n!}\,
    2^{n\alpha/2}v^{n-1}
    \sum_{\Gamma\in\mathbb{G}_{C,n}}
    \left\{ \prod_{a<b}^{n}\int_{0}^{1}ds_{ab}\,
    \partial_{s_{ab}}^{\nu_{ab}}\right\}
    \sum_{q=0}^{\infty}\sum_{i=0}^{q}
    \left(2q+\frac{1}{2}\right)\\
    &\times\frac{2^{4q-ni+1}\varGamma\left(\frac{n(\alpha+1)}{2}\right)}{\varGamma\left(\frac{n}{2}+ni\right)}
    \binom{2q}{2i}\binom{q+i-\frac{1}{2}}{2q}
    \sum_{p=0}^{q}\binom{2q}{2p}
    \binom{q+p-\frac{1}{2}}{2q}\frac{1}{2p+\alpha+1}\\
    &\times\sum\limits_{\{l_{ab}\}}2^{-\sum\limits_{a=1}^{n}
    \frac{l_{aa}}{2}}
    \frac{
    \left(2i\right)!^n}
    {\prod\limits_{a<b}\left(l_{ab}!\right)
    \prod\limits_{a=1}^{n}\left(\frac{l_{aa}}{2}\right)!}
    \prod_{a<b} \left(\nu_{ab}s_{ab}\gamma\right)^{l_{ab}},
    \label{Modified_connected_Green_functions_GF_G_Legendre_HSGA}
\end{split}
\end{equation}
where $v=\frac{\pi^{d/2}}{\varGamma(d/2+1)}\delta^{d}$ is the ``volume of one hard-sphere particle'' and parameters $\delta$ and $\gamma$ are determined by the equations:
\begin{equation}
    G(0)\gamma v=\int_{\mathbb{R}^{d}} dx\, G(x),\quad 
    G(0)^2\gamma^2 v =\int_{\mathbb{R}^{d}} dx\, G(x)^{2}.
    \label{HSG_equations_for_parameters}
\end{equation}
Let us make one remark about the equations (\ref{HSG_equations_for_parameters}). The HSG approximation is used in the statistical physics for potentials that decay rapidly at infinity in $x$. This is reflected in the equations (\ref{HSG_equations_for_parameters}) when integrating over $\mathbb{R}^{d}$. This usually also means that the product of a power function and the power of the propagator is again integrable. All the propagators considered in the paper satisfy the conditions of HSG approximation applicability.

The other reasonable equations for parameters $\delta$ and $\gamma$ are also possible, and these are chosen to make formulas of HSG approximation simpler for substituting the particular cases of propagators. All the obtained formulas were checked for the limiting case $\alpha=2$ and applied to the virton propagator (exponential form factor, smooth UV-cutoff function) as well as to Euclidean Klein--Gordon propagator with mass $m$ and sharp UV-cutoff function. Finally, we present approximate plots of vacuum energy density $w_{vac}$ for all values of $g$ for both described approximations and provide corresponding approximate formulas. These formulas are valid for strong coupling $g$ as well as for weak, which is quite a rare case for quantum field models.

Using Weierstrass M-test, we proved the convergence of such a series for $j=0$ and we have obtained the majorizing series explicitly. As for $j \neq 0$, we found the majorant depending on the regulator $\varepsilon$. There is still an open question about finding a more accurate majorant whose dependence on epsilon is negligible (for the case of non-zero source $j$). Therefore, we will remove $\varepsilon$ in the quantities calculated for $j=0$, such as vacuum energy and its density. And in the quantities depending on $j$ we won't remove $\varepsilon$, which will be reflected in their indices, such as $\mathcal{G}_{I,\varepsilon}$, etc.

Performed considerations and research will help in the understanding of interacting quantum fields' general properties. Besides, they can potentially be applied for the strong coupling limit of nonlocal $\phi^{4}$ and $\phi^{6}$ theories, which will be the subject of further research and publication of the authors.