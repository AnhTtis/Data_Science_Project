\section{Physical Background and Motivation}
\label{phys-mot}

\subsection{Definitions and Notations}

The starting point of our research is the complete (including disconnected parts) Green functions GF $\mathcal{Z}$ in terms of functional integral over primary field $\phi$. Traveling back, the explicit expression for $\mathcal{Z}$ is:
\begin{equation}
    \mathcal{Z}\left[j\right]=\int_{\varPhi}
    \mathcal{D}\left[\phi\right]\,e^{-\frac{1}{2}
    \int_{\mathbb{R}^{d}}\int_{\mathbb{R}^{d}} dx\,dy\, L(x,y)\phi(x)\phi(y)-
    \int_{\mathbb{R}^{d}} dx\,g(x)
    \left|\phi\left(x\right)\right|^{\alpha}+
    \int_{\mathbb{R}^{d}} dx\, j\left(x\right)
    \phi\left(x\right)}.
    \label{Complete_Green_functions_GF_Z_traveling_back}
\end{equation}
We will also call this functional the GCPF. Let us note that here and after we follow the normalizing convention $\mathcal{Z}_{0}\left[j=0\right]=1$, which makes this definition of GF coherent with those via Gaussian Measure in the next section. 

The function (distribution) $L(x,y)$ is the integral kernel of the operator $L$, which defines the quadratic part of a nonlocal QFT (Gaussian theory) action. Nonlocal QFT can be considered as a regularization of the Euclidean Klein--Gordon theory, but it is also of independent interest. Let us note that the integral kernel $G(x,y)$ of the inverse operator $G=L^{-1}$ is the UV finite propagator of a nonlocal QFT in any space dimension $d$. $G(x,y)$ is also called the Green function of Gaussian theory. In all the final calculations, where possible, we will assume the translational invariance of the Gaussian theory, which implies $G(x,y)=G(x-y)$, but at the same time, we will consider coordinate dependent coupling constant $g$. We will often choose $g(x)=g\chi_{Q}\left(x\right)$, where $\chi_{Q}$ is the indicator function of $d$-dimensional cube centered at the origin with $\text{Vol}\,{Q}=V$. We will also suppose that $G(x)\geq0$ and $G(x)\leq G(0)$ for all $x$. The functional integration is understood in the sense of Gaussian measure with covariance operator (propagator) $G$. In this paper, we assume that $\hbar=c=1$.

Generating functional $\mathcal{Z}$ is a regular functional, so it could be expanded in a functional Taylor series at $j=0$~\cite{vasil2004field,kopbarsch,Ogarkov2020III}:
\begin{equation}
    \mathcal{Z}\left[j\right]=\sum_{n=0}^{\infty}
    \frac{1}{n!}\mathcal{D}_{1,\ldots,n}^{(n)}\,j_{1}\ldots j_{n}\ \ ,\quad
    \mathcal{D}_{1,\ldots,n}^{(n)}=\frac{\delta^{n}\mathcal{Z}[j]}
    {\delta j_{1} \ldots \delta j_{n}}\bigg|_{j=0},
    \label{Complete_Green_functions_GF_Z_FTS}
\end{equation}
where the repeated index $i$ means integration over some spacetime variable $x_{i}$. This means the following:
\begin{equation}
    \mathcal{D}_{1,\ldots,n}^{(n)}\,j_{1} \ldots j_{n} := 
    \int_{\mathbb{R}^{d}}\ldots\int_{\mathbb{R}^{d}} 
    dx_1 \ldots dx_n \, 
    \mathcal{D}^{(n)}(x_1,\ldots,x_n) \,
    j(x_1) \ldots j(x_n).
\end{equation}

Coefficients $\mathcal{D}_{1,\ldots,n}^{(n)}=\mathcal{D}^{(n)}(x_{1},\ldots,x_{n})$ are called $n$-particle or $n$-point correlation functions of a functional $\mathcal{Z}$ since they represent $n$-point correlators of the theory:
\begin{equation}
\begin{split}
    &\frac{1}{n!}\mathcal{D}^{(n)}(x_{1},\ldots,x_{n})=
    \left\langle\phi(x_{1})\ldots\phi(x_{n})\right\rangle\\
    &=\int_{\varPhi}
    \mathcal{D}\left[\phi\right]\,\phi(x_{1})\ldots\phi(x_{n})\,
    e^{-\frac{1}{2}
    \int_{\mathbb{R}^{d}}\int_{\mathbb{R}^{d}} dx\,dy\, L(x,y)\phi(x)\phi(y)-
    \int_{\mathbb{R}^{d}} dx\,g(x)
    \left|\phi\left(x\right)\right|^{\alpha}}.
\end{split}
\end{equation}

Correlators are the object of interest since they are the quantities which determine the cross-section of scattering of particles and therefore can be directly measured in particle physics as well as in statistical physics. In Gaussian theory one has a usual Gaussian integral, and we will denote the corresponding GF as $\mathcal{Z}_{0}$. Also recall the convention we follow: $\mathcal{Z}_{0}\left[j=0\right]=1$. In this case, for GF $\mathcal{Z}_{0}$ we have a simple expression:
\begin{equation}
    \mathcal{Z}_{0}\left[j\right]=e^{\frac{1}{2}
    \int_{\mathbb{R}^{d}}\int_{\mathbb{R}^{d}} dx\,dy\, 
    G(x,y)j(x)j(y)}.
\end{equation}

It is also known that $\mathcal{Z}$ can be represented as an exponent of the other regular functional $\mathcal{G}$, which is called the connected Green functions GF:
\begin{equation}
    \mathcal{Z}\left[j\right]=e^{\mathcal{G}[j]},
    \label{G[j]-def}
\end{equation}
and from statistical physics point of view $\mathcal{Z}$ is a GCPF, as well as $\mathcal{G}$, is a grand thermodynamic potential (up to a temperature factor).

The functional $\mathcal{G}$ can also be expanded in a functional Taylor series, and its coefficients are called connected $n$-particle or $n$-point Green functions:
\begin{equation}
    \mathcal{G}\left[j\right]=
    \ln\mathcal{Z}[0]+\sum_{n=1}^{\infty}\frac{1}{n!}\mathcal{G}_{1,\ldots,n}^{(n)}\,j_{1}\ldots j_{n}\ \ ,\quad
    \mathcal{G}_{1,\ldots,n}^{(n)}=
    \frac{\delta^{n}\mathcal{G}[j]}
    {\delta j_{1}\ldots\delta j_{n}}\bigg|_{j=0}.
\end{equation}

The terminology goes from Feynman diagrams and will be clarified in the following while considering the Meyer cluster expansion. Of course,
all $\mathcal{D}^{(n)}(x_{1},\ldots,x_{n})$ can be expressed via $\mathcal{G}^{(n)}(x_{1},\ldots,x_{n})$ by expanding the exponent. Throughout this paper, we denote connected $n$-particle Green functions with the mathcal font $\mathcal{G}_{1,\ldots,n}^{(n)}$ rather than the Green function of Gaussian theory $G(x-y)$. 

In statistical physics a finite limit $\mathcal{G}/V$ at $V\rightarrow\infty$ exists. Therefore, we can write $\mathcal{G}$ as follows: 
\begin{equation}
    \mathcal{G}[j]=Vf[j],
\end{equation}
where $f$ is a volume density of $\mathcal{G}$, which is simply the pressure for the case of homogeneous systems. It can be shown that $\mathcal{G}[0]=\ln\mathcal{Z}[0]=-\mathcal{E}$, where $\mathcal{E}$ is a vacuum energy of the considering QFT, and it is useful to consider its volume density 
\begin{equation}
\label{vac-en-density-def}
    w_{vac}:=\frac{\mathcal{E}}{V}.
\end{equation}
The quantity $w_{vac}$ is a very useful quantity which in particular can tell whether the system has phase transition or not. This relation is literal. 

From Lehmann--Symanzik--Zimmermann reduction formula, one can obtain that the expression for the $\mathcal{S}$-matrix $n$-particle correlation functions in momentum representation reads (the repeated variable means integration):
\begin{equation}
    \mathcal{S}_{1^{\prime},\ldots,n^{\prime}}^{(n)}=
    \mathcal{D}_{1,\ldots,n}^{(n)}
    \prod_{k=1}^{n}\left(G_{kk^{\prime}}\right)^{-1},
\end{equation}
where $G_{kk^{\prime}}$ is the two-particle Green function of Gaussian theory in momentum representation. Of course, $ G^{-1}= L$, but it is accepted to denote the corresponding operator as $ G^{-1}$. For deriving the $\mathcal{S}$-matrix, that is exactly the functional which functional Taylor coefficients are $\mathcal{S}_{1^{\prime},\ldots,n^{\prime}}^{(n)}$,
one should simply change variables from sources to so-called ``classical fields'' ($\varphi$ instead of $\phi$), namely:
\begin{equation}
    \varphi(x)=Gj\left(x\right)=   
    \int_{\mathbb{R}^{d}} dy\, G(x,y)j(y),
\end{equation}
so the $\mathcal{S}$-matrix satisfies the following relation (up to a certain point, it is not necessary to introduce the regulator $\varepsilon$, mentioned in the section \ref{summary}):
\begin{equation}
    \mathcal{S}[\varphi]=\frac{1}{\mathcal{Z}[0]}
    \mathcal{Z}[ G^{-1}\varphi],
\end{equation}
as well as ``normalized'' GF $\mathcal{Z}_{I}$:
\begin{equation}
    \mathcal{Z}_{I}[j]=\frac{\mathcal{Z}[j]}
    {\mathcal{Z}_{0}[j]},
    \label{Z_I-def}
\end{equation}
and normalized GF $\mathcal{G}_{I}$ of connected Green functions (this functional can be considered both depending on the source and depending on the classical field, which was also mentioned in the section \ref{summary}):
\begin{equation}
    \mathcal{G}_{I}[j]=\ln\mathcal{Z}_{I}[j]=
    \mathcal{G}[j]-\mathcal{G}_{0}[j],
    \label{G_I-def}
\end{equation}
where $\mathcal{G}_{0}[j]$ is the Gaussian theory connected Green functions GF. Such a functional is equal to the quadratic:
\begin{equation}
    \mathcal{G}_{0}[j]=\frac{1}{2}
    \int_{\mathbb{R}^{d}}\int_{\mathbb{R}^{d}} dx\,dy\, 
    G(x,y)j(x)j(y).
\end{equation}

Next, we are going to describe two approaches to the perturbative calculation of $n$-particle functions $\mathcal{D}^{(n)}$ and $\mathcal{G}^{(n)}$. Let us consider the expansions of GFs $\mathcal{Z}$ and $\mathcal{G}$ in series:
\begin{equation}
    \mathcal{Z}\left[j\right]=
    \sum_{n=0}^{\infty}\mathcal{Z}_{n}[j],\quad
    \mathcal{G}[j]=\sum_{n=0}^{\infty}\mathcal{G}_{n}[j],
\end{equation}
which could be power series in some parameters of the system such as coupling constant $g$. Then they induce the corresponding expansions of $n$-particle functions in powers of the same parameter:
\begin{equation}
    \mathcal{D}_{1,\ldots,k}^{(k)}=
    \sum_{n=0}^{\infty}\mathcal{D}_{1,\ldots,k;\,n}^{(k)}\ \ ,\quad
    \mathcal{G}_{1,\ldots,k}^{(k)}=
    \sum_{n=0}^{\infty}\mathcal{G}_{1,\ldots,k;\,n}^{(k)}\ \ ,
\end{equation}
as well as for the $\mathcal{S}$-matrix. In the next subsection we are going to introduce the conventional method for obtaining of such expansions.

\subsection{Standard PT and Its Inapplicability}

Usually, when considering QFT with GF $\mathcal{Z}$:
\begin{equation}
    \mathcal{Z}\left[j\right]=\int_{\varPhi}
    \mathcal{D}\left[\phi\right]\,e^{-\frac{1}{2}
    \int_{\mathbb{R}^{d}}\int_{\mathbb{R}^{d}} dx\,dy\, L(x,y)\phi(x)\phi(y)-
    \int_{\mathbb{R}^{d}} dx\, V(\phi(x))+
    \int_{\mathbb{R}^{d}} dx\, j\left(x\right)
    \phi\left(x\right)},
    \label{Complete_Green_functions_GF_Z_General}
\end{equation}
with analytic interaction potential $V(\phi)$, one can obtain PT series via transformation:
\begin{equation}
    \mathcal{Z}\left[j\right]=
    \exp\left[-\int_{\mathbb{R}^{d}} dx\, 
    V\left(\frac{\delta}{\delta j(x)}\right)\right]\int_{\varPhi}
    \mathcal{D}\left[\phi\right]\,e^{-\frac{1}{2}
    \int_{\mathbb{R}^{d}}\int_{\mathbb{R}^{d}} dx\,dy\, L(x,y)\phi(x)\phi(y)+
    \int_{\mathbb{R}^{d}} dx\, j\left(x\right)
    \phi\left(x\right)},
\end{equation}
where the operator $\exp\left[-\int_{\mathbb{R}^{d}} dx\, V\left(\frac{\delta}{\delta j(x)}\right)\right]$ is understood in the sense of power series in operator $\frac{\delta}{\delta j}$. Consequently, one has to demand the analyticity of the potential $V(\phi)$ in $\phi$ for this method to work. Let us note that it is this operator that leads to asymptotic series in QFT. Then, after introducing classical field $\varphi$, the final formula reads:
\begin{equation}
\begin{split}
    &\mathcal{Z}\left[j\right]=
    \exp\left\{\frac{1}{2}
    \int_{\mathbb{R}^{d}}
    \int_{\mathbb{R}^{d}} dx\,dy\, G(x,y)j(x)j(y)\right\}\\
    &=\exp\left\{\frac{1}{2}
    \int_{\mathbb{R}^{d}}
    \int_{\mathbb{R}^{d}} dx\,dy\, G(x,y)
    \frac{\delta}{\delta\varphi(x)}
    \frac{\delta}{\delta\varphi(y)}\right\}
    \exp\left\{-\int_{\mathbb{R}^{d}} dx\,
    V\left(\varphi(x)\right)\right\}
    \Bigg|_{\varphi=Gj}.
\end{split}
\end{equation}
The operator $\exp\left[-\int_{\mathbb{R}^{d}} dx\, V\left(\frac{\delta}{\delta j(x)}\right)\right]$ should be properly defined since it contains multiple variational derivatives in the same point, but in the final answer all the regularizations of this expression could be removed. Then, expanding both exponents, one can obtain a series in coupling constant $g$. As already mentioned above, these series are usually asymptotic. As one can see, this way relies significantly on the analyticity of the potential $V(\phi)$. It is invalid for the case of fractional power self-interaction potential.

\subsection{Motivation of Study}

At first glance, it can appear to be nonphysical to consider models with interaction potential of the form $V(\phi)=g(x)\left|\phi\right|^{\alpha}$ (explicit dependence on $x$ through $g$ is not indicated in the potential). Though, there are several reasons to study it.

Firstly, it is a kind of exactly solvable model in QFT, where one can produce converging series in $g$. So it is an object of common interest because it is useful to learn the behavior of different QFT systems
to:
\begin{enumerate}
\item understand, what kind of properties they can own and, consequently, what one can look up in other models;
\item develop new methods and approaches, which can be subsequently extended and applied to other systems to gain new results;
\item since we have the converging series for such a theory, we can hope to construct their analytic continuations to points $\alpha>2$, in particular in the vicinity of $\alpha=4$, which is an object of common interest.
\end{enumerate}

Secondly, there are reasonable expectations, that in theory with the potential $V(\phi)=g(x)\left|\phi\right|^{\alpha}$ it is possible to complete explicit transition in obtained formulas from weak to strong coupling regimes. Moreover, there is the \textbf{conjecture} on the following duality between scalar field theories:
\begin{equation}
    L\leftrightarrow G,\quad
    g(x)\leftrightarrow \frac{1}{g(x)},\quad
    |\phi(x)|^{\alpha}\leftrightarrow
    |\phi(x)|^{\frac{\alpha}{\alpha-1}}.
    \label{Duality_fantasy}
\end{equation}
This duality can be very useful for the research of nonlocal $\phi^{4}$ theory in a strong coupling regime as well as a nonperturbative description of phase transitions. 

This duality can be obtained with the use of the Parseval--Plancherel identity to functional integral with the following substitution of saddle-point asymptotic in the interaction part, which turns out to be valid in renormalizable cases. Though, for careful checking of this duality one should obtain PT series of asymptotic expansions of $\phi^{4}$  theory from the strong coupling regime of $|\phi|^{\frac{4}{3}}$, which demands the general studying of $|\phi|^{\alpha}$ theories for $\alpha\in(1,2)$. Describing, checking and using of this duality is a subject of current studies of the authors, and will be the topic of the nearest publications.

On the one hand, it should be noted that the first of three substitutions in the expression (\ref{Duality_fantasy}) poses a danger: if the original Gaussian measure $\gamma_{G}$ was given by a trace class covariance operator $G$, the dual Gaussian measure $\gamma_{L}$ is determined by an operator $L$ with an infinite trace value. Therefore, the integral over $\gamma_{L}$ is ill-defined. This is consistent with the conventional point of view that $\phi^{4}$ is quantum trivial in $d\geq 4$. 

On the other hand, from the Gaussian measure in separable HS point of view, there are no evident reasons for GF to be in any sense ``trivial'' for any $d$ since $x$ is the index of a vector and the field $\phi$ is a vector in the corresponding HS $\varPhi$. The integration theory in $\varPhi$ is insensitive to the dimension of the index and is constructed uniformly for any $\varPhi$. For the exponential of a polynomial, the integral over a ``good'' Gaussian measure is defined in rigorous mathematical sense~\cite{Bogachev,GiuseppePrato}. For this reason, quantum triviality may turn out to be a consequence of the incorrectness of mathematical transformations in the calculation.

%Besides, there are some works that claim the existence of strong coupling non-trivial theory, for instance, after resummation of all the perturbative contributions with the following analytical continuation (link!!!). If this were true, it would explain why the numerical experiments showing the triviality are not correct (since they use the formulas incorrect in the considering region, e.g. power series for $1(-z)^{-1}$ outside its circle of convergence). 

Be that as it may, the authors would like to find rigorous mathematical proof of the quantum triviality or to get to the conclusion that this triviality is a PT artefact only. And the present paper is the first step to ascertain the truth in this complicated question.