\section{Calculation of PT Series Terms}
\label{pt-calc}

Let us start by noting that the integrals of the form:
\begin{equation}
    \int_{\mathbb{R}^{n}} 
    d\phi_{1}\ldots d\phi_{n}\,
    \left|\phi_{1}\right|^{\alpha}\ldots 
    \left|\phi_{n}\right|^{\alpha}
    e^{-\frac{1}{2}\sum\limits_{a,b=1}^{n}
    R_{ab}^{\left(n\right)}
    \phi_{a}\phi_{b}+
    \sum\limits_{a=1}^{n}
    \phi_{a}\chi\left(x_{a}\right)},
    \label{starting-integral-sect-5}
\end{equation}
are the particular cases of so-called Gelfand hypergeometric functions~\cite{Gelfand}. Unfortunately, they are too generic and their properties are excessively complicated. However, if we are going to get some practical results it is pointless to express anything via them. Besides, these integrals are enough sophisticated to be expressed simply in terms of even generalized Gauss and Lauricella hypergeometric functions. 

\subsection{Off-Diagonal Terms Expansion}

The first and most evident way is to expand the quadratic exponent in its off-diagonal terms, and then compute the obtained integrals directly. However, this approach also fails because of the excessive complexity of the obtained terms. Let us demonstrate it. We start with the expansion for zero source:
\begin{equation}
    \mathcal{Z}_{I}\left[0\right]=
    \sum_{n=0}^{\infty}\frac{\left(-1\right)^{n}}{n!\left(2\pi\right)^{n/2}}
    \left\{\prod_{a=1}^{n}
    \int_{\mathbb{R}^{d}} dx_{a}\, g(x_a) 
    \int_{\mathbb{R}} d\phi_{a}\,
    \left|\phi_{a}\right|^{\alpha}\right\}  
    \times\frac{e^{-\frac{1}{2}
    \sum\limits_{a,b=1}^{n}
    \left(G_n\right)_{ab}^{-1}\phi_{a}\phi_{b}}}
    {\sqrt{\det\left(G_n\right)}}.
\end{equation} 
Here we have already removed the regulators $\varLambda$ and $\varepsilon$ in accordance with (\ref{Z_I[0]-final-form}).

Then, substituting the series:
\begin{equation}
    e^{-\sum\limits_{a<b}^{n}
    \left(G_n\right)_{ab}^{-1}
    \phi_{a}\phi_{b}}=
    \left\{\prod\limits_{a<b}^{n}
    \sum_{l_{ab}=1}^{\infty}\right\}
    (-1)^{\,\sum\limits_{a<b}^{n}l_{ab}}
    \prod\limits_{a<b}^{n}
    \frac{\left(\left(G_n\right)_{ab}^{-1}
    \right)^{l_{ab}}}{l_{ab}!}
    \prod_{a=1}^{n}
    \phi_{a}^{\,\sum\limits_{b|b\neq a}^{n}l_{ab}},
\end{equation}
and calculating integrals over $\phi_{a}$, we arrive at the following expression:
\begin{equation} 
\begin{split} 
    \mathcal{Z}_{I}\left[0\right]&=
    \sum_{n=0}^{\infty}
    \frac{\left(-1\right)^{n}}
    {n!\left(2\pi\right)^{n/2}}
    \left\{\prod\limits_{a<b}^{n}
    \sum_{l_{ab}=1}^{\infty}\right\}
    \left(-\sqrt{\frac{2}{G(0)}}
    \right)^{n+1+
    \sum\limits_{a<b}^{n}l_{ab}}\\
    &\times\left\{\prod_{a=1}^{n}
    \int_{\mathbb{R}^{d}}dx_{a}\, g(x_a)\right\}
    \frac{1}{\sqrt{\det\left(G_n\right)}}
    \prod\limits_{a<b}^{n}
    \frac{\left(\left(G_n\right)_{ab}^{-1}
    \right)^{l_{ab}}}{l_{ab}!}.
\end{split} 
\end{equation} 

There are two main obstacles for applying this formula:
\begin{enumerate}
    \item it is necessary to calculate the inverse of $n\times n$ matrix, depending on $x_a$;
    \item it is necessary to integrate the elements of the inverse matrix over $x_a$.
\end{enumerate}
These are the main reasons we will develop other computational techniques for the practical calculation of some concrete
quantities. 

\subsection{Approximation of Generic Term via Polynomials}
\label{subsect-pol-appr}

In this section we will approximate prefactor $\left|\phi_{1}\ldots\phi_{n}\right|^{\alpha}$
as a function $f(\zeta):=|\zeta|^{\alpha}$ for $\zeta=\phi_{1}\ldots\phi_{n}$ with polynomials. There is a couple of constructive ways, and we compare several of them. We will consider three types of polynomial approximations:
\begin{enumerate}
    \item Legendre polynomial approximation,
    \item Chebyshev polynomial approximation,
    \item Bernstein polynomial approximation.
\end{enumerate}
However, we will present the explicit formulas only for the Legendre polynomials, since they will be the most suitable for solving the problems, formulated in our paper. To show this, we will consider other families of polynomials. However, they will not be required to display our final results, so they will not be explicitly given. The complete guide about all the mentioned families of polynomials can be found in \cite{bernst-pol,ort-pol}.

\subsubsection{Background on Constructive Approximation with Legendre Polynomials}
\label{Legendre-appr-bg}

Legendre polynomials are defined as an orthogonal system with respect to the integral scalar product generated with the Lebesgue measure on the closed interval $[-1,1]$. That is, $P_{n}$ is a polynomial of degree $n$, such that (we denoted the independent variable as $t$, since the domains of $t$ and $\zeta$ do not coincide): 
\begin{equation}
    \int_{-1}^{1}dt\, P_{m}(t) P_{n}(t)=0
    \quad\text{ if }n\neq m.
\end{equation}
The standardization $P_{n}(1)=1$ fixes the normalization of the Legendre polynomials with respect to the $L^{2}$-norm on the closed interval $[-1,1]$. There is an explicit formula for Legendre polynomial of degree $n$:
\begin{equation}
    P_{n}(t)=2^{n}\sum_{k=0}^{n}
    \binom{n}{k}
    \binom{\frac{n+k-1}{2}}{n}t^{k},\quad t\in[-1,1],
    \label{Legendre-def}
\end{equation}
where $\binom{n}{k}$ is the binomial coefficient, and they obey:
\begin{equation}
    \int_{-1}^{1}dt\, P_{m}(t) P_{n}(t)=
    \frac{2}{2n+1}\,\delta_{mn}.
\end{equation}

The feature of the formula (\ref{Legendre-def}) is that due to the second binomial coefficient, all the terms for $k$ with the different residue modulo $2$ are zero. So, for even $n$ polynomial $P_n$ consists of even monomials, and for odd $n$ --- of odd monomials. So, about a half of the terms in the sum in (\ref{Legendre-def}) are zero, and one should bear this in mind during practical calculations. The first few even Legendre polynomials according to this definition are:
\begin{equation}
    P_0(t) = 1,\quad P_2(t) = \frac{1}{2}(3t^2-1),
    \quad P_4(t) = \frac{1}{8}(35t^4-30t^2+3).
\end{equation}

Further, for the absolutely continuous function $f$ with derivative $f'$ of bounded variation on $[-1,1]$, the expansion in Legendre polynomials converges uniformly and absolutely:
\begin{equation}
    \sum_{n=0}^{\infty}c_{n}P_{n}(t)
    \underset{[-1,1]}{\rightrightarrows}f(t),
\end{equation}
due to Jackson's theorem~\cite{ort-pol}. Here, the coefficients $c_{n}$ are determined by the formulas: 
\begin{equation}
    c_{0}=\frac{1}{2}\int_{-1}^{1}dt\, f(t),\quad 
    c_{n>0}=\frac{2n+1}{2}\int_{-1}^{1}dt\, f(t)P_{n}(t).
    \label{initial-coefficients-def}
\end{equation}
There also exists a bound for the residue of the series truncated on $N$th term, but it is not very useful for our case. It is so because it is an estimate for the absolute value of error, so the consequent estimation of GF will be excessively rough, since we have to take the sign into account.

As it will be explained in the next subsection, we will approximate a function $f(t)=|t|^\alpha$, for $\alpha\in(1,2)$ and $t\in [-1,1]$, choosing some ``normalized'' variable $t$. For the considering case the conditions of Jackson's theorem are satisfied for $\alpha\in (1,2)$, so we have the uniform and absolute convergence of Legendre series to $f(t)$. We write for the finite-terms approximation, using that $f$ is even (for the convenience of writing some further formulas, we denote the coefficients of the Legendre polynomial as $u_{i}$), the following expression:
\begin{equation}
    h_{N}(t)=\sum_{q=0}^{N}c_{q}(f)P_{2q}(t), \quad 
    P_{2q}(t)=\sum_{i=0}^{q}u_{i}t^{2i}.
\label{pol-appr-def}
\end{equation}
Let us note that because of the absence of the odd Legendre polynomials in the approximation, we will use the numeration for the approximation coefficients as in (\ref{pol-appr-def}) rather than in (\ref{initial-coefficients-def}). And ultimately, to avoid any misconceptions, we will write for the coefficients with the ``new'' numeration $c_q(f)$ rather than $c_q$. Let us also note that the total degree of polynomial approximation $h_{N}$ is $\deg h_N = 2N$.

The coefficients $c_{q}(f)$ for $f(t)=|t|^\alpha$ have the form:
\begin{equation}
    c_{q>0}(f)=2\left(2q+\frac{1}{2}\right)
    \int_{0}^{1}dt\, t^{\alpha}P_{2q}(t)=
    \left(2q+\frac{1}{2}\right)
    \sum_{p=0}^{q}\binom{2q}{2p}
    \binom{q+p-\frac{1}{2}}{2q}
    \frac{2^{2q+1}}{2p+\alpha+1},
\end{equation}
and this formula is fair also for $q=0$, as a computation shows. So for $h_{N}$ we get:
\begin{equation} 
    h_{N}(t)=
    \sum_{q=0}^{N}
    \left(2q+\frac{1}{2}\right)
    \sum_{p=0}^{q}
    \binom{2q}{2p}\binom{q+p-\frac{1}{2}}{2q}
    \frac{2^{4q+1}}{2p+\alpha+1}\sum_{i=0}^{q}
    \binom{2q}{2i}\binom{q+i-\frac{1}{2}}{2q}t^{2i}.
\end{equation} 

In the following subsections we will use only Legendre polynomials. Therefore we can expand $f$ in a series, putting $N\rightarrow \infty$. However, for the comparison of the Legendre approximation with the Bernstein one (which is only approximation rather than expansion), we will keep $N$ finite until the end of the different polynomial approximations analysis.

\subsubsection{Construction of Approximation for the Terms}

According to theorems on polynomial approximation, e.g. the Stone--Weierstrass theorem, it is possible to approximate the function by polynomials on compacts only. Moreover, if we try to approximate (even pointwise) a function $\left|\zeta\right|^{\alpha}$ for all $\zeta\in\mathbb{R}$ via any kind of polynomials, e.g. Hermite or Laguerre polynomials (multiplied by the corresponding weight functions so that such expansions converge), we obtain a diverging series after term-by-term integration. And the thing is that there is a problem in the radial (in terms of spherical coordinates in $\mathbb{R}^{n}$) direction. As a possible solution, before approximating by polynomials, one should evaluate the integral (\ref{starting-integral-sect-5}) in the radial direction, introducing spherical coordinates in $\mathbb{R}^{n}$ (hyperspherical coordinates, if $n>3$, but for brevity we will use this term for $n>1$). In the remaining integral over hypersphere we can approximate $\left|t\right|^{\alpha}$, where $t$ --- some convenient function on the hypersphere with values in $[-1,1]$, by polynomials in $t$. There will be no such a problem since on a hypersphere $t\in[-1,1]$, so all the approximations and expansions will converge. After such an approximation one can go back to $\mathbb{R}^{n}$ where it is easier to calculate the integrals using the source trick. It is a brief summary of the material in the current subsection.

Let us denote the integral, which we are going to work with, as ($n>1$):
\begin{equation}
    I_{n}=\left\{\prod_{a=1}^{n}
    \int_{\mathbb{R}}d\phi_{a}\,
    \left|\phi_{a}\right|^\alpha\right\}
    \exp\left(-\mathcal{Q}_{n,\Gamma}(x,\phi)
    +l\left(x,\phi\right)\right).
\label{I_n-def}
\end{equation}
The integrand has the symmetry under $\phi\to-\phi$, since $\mathcal{Q}_{n,\Gamma}$ and $\left|\phi_{a}\right|^\alpha$ are even functions, hence (in terms of the hyperbolic cosine):
\begin{equation}
    I_{n}=\left\{\prod_{a=1}^{n}
    \int_{\mathbb{R}}d\phi_{a}\,
    \left|\phi_{a}\right|^\alpha\right\} 
    \exp\left(-\mathcal{Q}_{n,\Gamma}(x,\phi)
    \right)\cosh\left(l\left(x,\phi\right)\right).
\end{equation} 
In this integral we can introduce hyperspherical coordinates and proceed as:
\begin{equation}
    I_{n}=\int_{0}^{\infty}
    dr\int_{S^{n-1}}d\Omega\, 
    r^{n-1+n\alpha}
    \left|\xi_{1}\ldots\xi_{n}\right|^{\alpha}
    \exp\left(-r^{2}\mathcal{Q}_{n,\Gamma}(x,\xi)
    \right)\cosh\left(rl\left(x,\xi\right)\right),
    \label{cosine_to_talk_about}
\end{equation}
where we choose $\phi_{a}=r\xi_{a}$ and $\vec{\xi}\in S^{n-1}$ is the unit normal vector. Integration over $r$ gives:
\begin{equation}
    I_{n}=\frac{1}{2}\int_{S^{n-1}}d\Omega\,
    \left|\xi_{1}\ldots\xi_{n}\right|^{\alpha}
    \frac{\varGamma\left(\frac{n(\alpha+1)}{2}\right)}
    {\mathcal{Q}_{n,\Gamma}(x,\xi)^{\frac{1}{2}(\alpha+1)n}}\,{}_{1}F_{1}\left(\frac{n(\alpha+1)}{2};
    \frac{1}{2};\frac{l\left(x,\xi\right)^{2}}
    {4\mathcal{Q}_{n,\Gamma}(x,\xi)}\right).
    \label{Confluent_Hypergeometric_function_1}
\end{equation} 
In the expression (\ref{Confluent_Hypergeometric_function_1}) we follow the commonly accepted notation of ${ }_{1}F_{1}$ for the confluent hypergeometric function. This function is defined in terms of the following series (and this series converges for any finite value of $z$ and thus defines an entire function of $z$):
\begin{equation}
   {}_{1}F_{1}(a;b;z)=
   \sum_{n=0}^{\infty}\frac{(a)_{n}z^n}{(b)_{n}n!},
\end{equation}
where $(a)_{n}$ is a rising factorial. Then we expand $_{1}F_{1}$ and obtain the following expression:
\begin{equation}
    I_{n}\!=\!\frac{1}{2}\sum_{k=0}^{\infty}
    \frac{1}{4^{k}k!}
    \frac{\left(\frac{n(\alpha+1)}
    {2}\right)_{2k}}{\left(1/2\right)_{2k}}
    \varGamma\left(\frac{n(\alpha+1)}{2}\right)
    \int_{S^{n-1}}\frac{d\Omega}
    {\mathcal{Q}_{n,\Gamma}(x,\xi)^{\frac{n}{2}}}
    \frac{\left|\xi_{1}\ldots\xi_{n}
    \right|^{\alpha}}
    {\mathcal{Q}_{n,\Gamma}(x,\xi)^{\frac{\alpha n}{2}}}\frac{l\left(x,\xi\right)^{2k}}
    {\mathcal{Q}_{n,\Gamma}(x,\xi)^{k}}.
\end{equation}
One can easily notice that this expansion can be obtained by expanding $\cosh$ in the formula (\ref{cosine_to_talk_about}). As in the section \ref{majorant-section} on majorant calculation, we can bound the following relation, using (\ref{max-G-eigenvalue-bound}) and (\ref{normals-product-bound}):
\begin{equation*}
    0\leq\frac{\left|\xi_{1}\ldots\xi_{n}\right|}{\mathcal{Q}_{n,\Gamma}(x,\xi)^{\frac{n}{2}}}\leq
    2^{n/2}G(0)^{n/2},
\end{equation*}
for all $x_{a}$. And now we can specify the function $t$:
\begin{equation}
    t=\frac{1}{2^{n/2}G(0)^{n/2}}
    \frac{\left|\xi_{1}\ldots 
    \xi_{n}\right|}
    {\mathcal{Q}_{n,\Gamma}(x,\xi)^{\frac{n}{2}}}.
\end{equation}

Having defined an appropriate function $t$, consider finite-degree even polynomial approximation (\ref{pol-appr-def}) of the function $f$. Let us note the following: the expression (\ref{pol-appr-def}) describes a general polynomial approximation such that the polynomials $h_N$ are defined for all integer $N>0$ and $\deg h_N = 2N$. and we will often use the same notations, as for Legendre polynomials, but for generality and simplification of formulas, we won't specify the concrete form of (real) coefficients $u_i$ and $c_q$. So, here $\{P_{2q}(t)\}_{q=0}^{\infty}$ is some general set of polynomials suitable for pointwise approximation of the function $f$, in other words, $h_{N}(t)\rightarrow f(t)$ for all $t\in[-1,1]$. We will also suppose for generality, that the coefficients $c_{q}(f)$ can also depend on $N$, which is actual for approximation with Bernstein polynomials.

In the following we will denote as $I_{n,N}$ the integrals obtained from $I_n$ by substitution polynomial approximation $h_{N}(t)$ instead of $f(t)$ in the integrand. Let us note that $I_{n,N}\rightarrow I_n$ due to the DCT, when $N\rightarrow\infty$. This fact follows from the uniform convergence of Legendre series to the considered function (as discussed in subsection \ref{subsect-pol-appr}), which means the uniform boundedness of partial sums. This convergence will also be checked numerically in the subsection \ref{numerical-appr-an}.

Thus we arrive at the following expression:
\begin{equation*} 
\begin{split}
    &I_{n,N}=\frac{1}{2}\sum_{k=0}^{\infty}
    \frac{\left(2G(0)\right)^{n\alpha/2}}
    {4^{k}k!}\frac{\left(\frac{n(\alpha+1)}{2}\right)_{2k}}{\left(1/2\right)_{2k}}
    \varGamma\left(\frac{n(\alpha+1)}{2}\right)
    \sum_{q=0}^{N}c_{q}(f)\\
    &\times\int_{S^{n-1}}\frac{d\Omega}
    {\mathcal{Q}_{n,\Gamma}(x,\xi)^{\frac{n}{2}}}P_{2q}\left(\frac{\left|\xi_{1}\ldots
    \xi_{n}\right|}{2^{n/2}G(0)^{n/2}
    \mathcal{Q}_{n,\Gamma}(x,\xi)^{\frac{n}{2}}}\right)\frac{l\left(x,\xi\right)^{2k}}
    {\mathcal{Q}_{n,\Gamma}(x,\xi)^{k}}.
\end{split} 
\end{equation*} 
Recalling the identity for $n,s>0$:
\begin{equation*}
    \int_{0}^{\infty}dr\,e^{-r^{2}}r^{n+2s-1}=
    \frac{1}{2}\varGamma\left(\frac{n}{2}+s\right),
\end{equation*}
we can return to $\mathbb{R}^{n}$ due to homogeneity:
\begin{equation*} 
\begin{split}
    &I_{n,N}=\sum_{k=0}^{\infty}
    \frac{\left(2G(0)\right)^{n\alpha/2}}{4^{k}k!}\frac{\left(\frac{n(\alpha+1)}{2}\right)_{2k}}{\left(1/2\right)_{2k}}
    \sum_{q=0}^{N}\sum_{i=0}^{q}
    \frac{u_{i}c_{q}(f)\varGamma
    \left(\frac{n(\alpha+1)}{2}\right)}{2^{ni}G(0)^{ni}
    \varGamma\left(\frac{n}{2}+ni+k\right)}\\
    &\times\int_{\mathbb{R}^n}
    d\phi_{1}\ldots d\phi_{n}\,    
    \left(\phi_{1}\ldots\phi_{n}\right)^{2i}
    l\left(x,\phi\right)^{2k}
    e^{-\mathcal{Q}_{n,\Gamma}(x,\phi)}.
\end{split} 
\end{equation*} 
It is worth noting that without the ratio $\varGamma\left(\frac{n(\alpha+1)}{2}\right)/\varGamma\left(\frac{n}{2}+ni+k\right)$ we would receive $\cosh$ after summation instead
of confluent hypergeometric function ${ }_{1}F_{1}$. At the same time, we don't write out the last one, since it is convenient to continue the transformations of the series itself. 

We can rewrite the integrand, using the multinomial formula (all the $\beta_{a}\geq 0$):
\begin{equation*}
    l\left(x,\phi\right)^{2k}=
    \sum_{\beta_{1}+\ldots +\beta_{n}=2k}
    \binom{2k}{\beta_{1}\ldots
    \beta_{n}}\chi_{1}^{\beta_{1}}\ldots\chi_{n}^{\beta_{n}}
    \phi_{1}^{\beta_{1}}\ldots\phi_{n}^{\beta_{n}},
\end{equation*}
and proceed with the integrals of the form (which are no more than the moments of multidimensional Gaussian distribution with covariance matrix $G_{n,\Gamma}$):
\begin{equation}
    \int_{\mathbb{R}^n}d\phi_{1}\ldots d\phi_{n}\,
    \phi_{1}^{m_1}\ldots\phi_{n}^{m_{n}}\, 
    e^{-\mathcal{Q}_{n,\Gamma}(x,\phi)},
\end{equation}
for some integer constants $m_{a}\geq 0$. There is a common method for their calculation consisting in introducing auxiliary variables $\eta_a$, called ``sources'', in similar fashion to path integrals, and further differentiation over them. Namely, we use the identity, following from DCT:
\begin{equation}
    \int_{\mathbb{R}^n}d\phi_{1}\ldots d\phi_{n}\,
    \phi_{1}^{m_1}\ldots\phi_{n}^{m_{n}}\, 
    e^{-\mathcal{Q}_{n,\Gamma}(x,\phi)+
    \sum\limits_{a=1}^{n}\eta_{a}\phi_{a}}= 
    \partial_{1}^{m_1}\ldots 
    \partial_{n}^{m_n}\,
    e^{-\frac{1}{2}\sum\limits_{a,b=1}^{n}
    \left(G_{n,\Gamma}\right)_{ab}\eta_{a}\eta_{b}},
    \label{correlators-differential-formula}
\end{equation}
where the partial derivatives $\partial_{i}:=\frac{\partial}{\partial\eta_{i}}$. If all $\eta_{a}=0$, we get the desired equality.

Keeping the previous expression in mind, we obtain:
\begin{equation} 
\begin{split} 
    &I_{n,N}=\sum_{k=0}^{\infty}
    \frac{\left(2G(0)\right)^{n\alpha/2}}{4^{k}k!}\frac{\left(\frac{n(\alpha+1)}
    {2}\right)_{2k}}{\left(1/2\right)_{2k}}
    \sum_{q=0}^{N}\sum_{i=0}^{q}
    \sum_{\beta_{1}+\ldots +\beta_{n}=2k}\binom{2k}
    {\beta_{1}\ldots\beta_{n}}u_{i}c_{q}(f)\\
    &\times\frac{\varGamma\left(\frac{n(\alpha+1)}{2}\right)\sqrt{(2\pi)^{n}
    \det\left(G_{n,\Gamma}\right)}}{2^{ni}G(0)^{ni}\varGamma\left(\frac{n}{2}+ni+k\right)}\chi_{1}^{\beta_{1}}\ldots \chi_{n}^{\beta_{n}}\partial_{1}^{\beta_{1}+2i}\ldots \partial_{n}^{\beta_{n}+2i}\bigg|_{\eta_{a}=0}
    e^{-\frac{1}{2}\sum\limits_{a,b=1}^{n}
    \left(G_{n,\Gamma}\right)_{ab}\eta_{a}\eta_{b}}.
\end{split} 
\end{equation} 

Finally, for the connected Green functions GF $n$th term approximation $\mathcal{G}_{I,n}[j]_{N}$ we obtain the following expression:
\begin{equation} 
\begin{split}
    &\mathcal{G}_{I,\varepsilon,n}
    \left[j\right]_{N}=
    \frac{\left(-1\right)^{n}}{n!}
    \sum_{\Gamma\in\mathbb{G}_{C,n}}
    \left\{ \prod_{a<b}^{n}\int_{0}^{1}ds_{ab}\,
    \partial_{s_{ab}}^{\nu_{ab}
    \left(\Gamma\right)}\right\} 
    \left\{\prod_{a=1}^{n}
    \int_{\mathbb{R}^{d}}dx_{a}\, 
    g\left(x_{a}\right)\right\}\, 
    e^{-\mathcal{Q}_{n,\Gamma}
    \left(x,\varphi\right)}\\
    &\times\sum_{k=0}^{\infty}
    \frac{\left(2G(0)\right)^{n\alpha/2}}{4^{k}k!}
    \frac{\left(\frac{n(\alpha+1)}{2}\right)_{2k}}{\left(1/2\right)_{2k}}
    \sum_{\beta_{1}+\ldots +\beta_{n}=2k}
    \binom{2k}{\beta_{1}\ldots
    \beta_{n}}\chi_{1}^{\beta_{1}}\ldots \chi_{n}^{\beta_{n}}\sum_{q=0}^{N}
    \sum_{i=0}^{q}u_{i}c_{q}(f)\\
    &\times\frac{\varGamma
    \left(\frac{n(\alpha+1)}{2}\right)}
    {2^{ni}G(0)^{ni}\varGamma
    \left(\frac{n}{2}+ni+k\right)}\,
    \partial_{1}^{\beta_{1}+2i}\ldots 
    \partial_{n}^{\beta_{n}+2i}
    \bigg|_{\eta_{a}=0}
    e^{-\frac{1}{2}\sum\limits_{a,b=1}^{n}
    \left(G_{n,\Gamma}\right)_{ab}\eta_{a}\eta_{b}}.
 \label{G_I[j]-general-polynomial-final-without-comb}
\end{split} 
\end{equation} 
Starting from this formula and below, we found its useful to introduce the notation $\mathcal{G}_{I,\varepsilon,n}\left[j\right]_{N}$. We specify the value $N$ to the right of $j$, so as not to clutter up the number of indices to the left of $j$. By definition, $\mathcal{G}_{I,\varepsilon,n}\left[j\right]_{N}$ is obtained from $\mathcal{G}_{I,\varepsilon,n}\left[j\right]$ by substituting the approximation $I_{n,N}$ instead of $I_n$. And owing to the convergence $I_{n,N}\rightarrow I_n$, when $N\rightarrow\infty$, it is also true that $\mathcal{G}_{I,\varepsilon,n}\left[j\right]_{N}\rightarrow \mathcal{G}_{I,\varepsilon,n}\left[j\right]$. Recall that in the expression (\ref{G_I[j]-general-polynomial-final-without-comb}) $s_{ab}\nu_{ab}$ and $\varepsilon$ enter in three ways:
\begin{enumerate}
\item $\mathcal{Q}_{n,\Gamma}\left(x,\varphi\right)$ in the exponent;
\item $\chi_{a}$ in the prefactor;
\item $\left(G_{n,\Gamma}\right)_{ab}$ in the
exponent.
\end{enumerate}

It is interesting that in fact the formula (\ref{G_I[j]-general-polynomial-final-without-comb}) provides the expansion of the connected Green functions GF for fractional-power interaction theory in terms of contributions of power-interaction theories with some ``weights''. However, it is not the same if one would like to expand $\left|\phi\right|^\alpha$ in the action (\ref{action}) itself. In such an approach there would be no ``damping factor'' of the form $\varGamma\left(\frac{n(\alpha+1)}{2}\right)/\varGamma\left(\frac{n}{2}+ni+k\right)$, and the series would be divergent due to its absence. And the reason for this factor to appear is that before the approximation we reduced the integral from non-compact $\mathbb{R}^n$ to a compact $S^{n-1}$. As a result, we won't lose the convergence on any step and therefore get the convergent series rather than asymptotic. And this achievement was the primary aim of the present paper.

\subsubsection{Combinatorial Expansion}
\label{comb-exp-derivation}

We want to derive analytical formulas for the quantities:
\begin{equation}
    \partial_{1}^{m_{1}}\ldots \partial_{n}^{m_{n}}\bigg|_{\eta_{a}=0}
    e^{-\frac{1}{2}\sum\limits_{a,b=1}^{n}
    \left(G_{n,\Gamma}\right)_{ab}\eta_{a}\eta_{b}} ,
\end{equation}
for any integer $m_{a}\geq 0$, for $a=1,\ldots,n$. The authors are sure that such formulas were derived a long time ago, but we could not find them in the literature in a suitable form. So let us derive them from the basics for our own usage. In fact, the formulas we will obtain are nothing else but the general formulas for the symmetry coefficients (up to some factor) of Feynman diagrams.

As already mentioned about the expression (\ref{correlators-differential-formula}), in this expression it is not hard to recognize the moments of $n$-dimensional Gaussian distribution with the covariance matrix $G_{n,\Gamma}$, which we will denote by brackets of ``averaging'':
\begin{equation}
\begin{split}
    \left\langle \phi_{1}^{m_{1}}\ldots
    \phi_{n}^{m_{n}}\right\rangle_{G_{n,\Gamma}}&: = 
    \left\{\int_{\mathbb{R}^n}
    \frac{d\phi_{1}\ldots d\phi_{n}}
    {(2\pi)^{n/2}\sqrt{\det{G_{n,\Gamma}}}}\,
    \phi_{1}^{m_1}\ldots\phi_{n}^{m_{n}}\, 
    e^{-\mathcal{Q}_{n,\Gamma}(x,\phi)+
    \sum\limits_{a=1}^{n}\eta_{a}\phi_{a}}\right\} \bigg|_{\eta_{a}=0}\\
    &= \partial_{1}^{m_1}\ldots 
    \partial_{n}^{m_n}\bigg|_{\eta_{a}=0}\ 
    e^{-\frac{1}{2}\sum\limits_{a,b=1}^{n}
    \left(G_{n,\Gamma}\right)_{ab}\eta_{a}\eta_{b}}. 
    \label{correlators-def}
\end{split}
\end{equation}
The index of the angle brackets indicates the covariance matrix we use for the computation of this quantity. We will refer to the quantity ($\ref{correlators-def}$) as correlator or multidimensional Gaussian moment with integer powers. We have already introduced another notion of correlator in the section \ref{phys-mot} (referring to the path integral), but it will be clear from the context what notion we mean in every particular case. We will also refer to the variables $\phi_a$ in correlators as fields. 

For now, we will consider covariance matrix $G_n$ rather than $G_{n,\Gamma}$ for the simplification of notations (in this subsection we will also omit the index $n$ for the same reason). Though, we won't suppose any of its special properties except it is symmetric and has equal terms on the diagonal which we will denote as  $G_{aa}=G(0)$ (a generalization without this property is straightforward). Then the result for $G_{n,\Gamma}$ can be obtained with the simple substitution $G_{ab}\mapsto\left(G_{n,\Gamma}\right)_{ab}$.

There is the Isserlis--Wick theorem which claims that the multidimensional Gaussian moment with integer powers can be expressed in terms of the covariance matrix elements as the sum over all possible pairings:
\begin{equation}
    \partial_{i_1}\ldots\partial_{i_m}
    \bigg|_{\eta_{a}=0}
    e^{-\frac{1}{2}\sum\limits_{a,b=1}^{n}
    G_{ab}\eta_{a}\eta_{b}}=
    \sum_{\text{all pairings of}\ \{i_{a}\}} 
    G_{i_{a_{1}} i_{a_{2}}}\ldots 
    G_{i_{a_{m-1}}i_{a_{m}}}\ ,
\label{Wick-theorem}
\end{equation}
for even $m$ and are equal to zero for odd $m$. Let us note that the notations in (\ref{correlators-def}) and (\ref{Wick-theorem}) differ by rearranging of derivations and $m=\sum\limits_{a=1}^{n} m_{a}$ is a total their number.

Now we are going to explain what do we mean under this summation. Informally, we sum over all pairings of the indices set, with account of possible repetitions of $i_{a}$. More formally, consider even $m$ and denote the set of all possible pairings of $\{1,\ldots,m\}$ as $\mathcal{P}_{m}$. For example, for $m=4$ the set of all possible pairings reads as follows:
\begin{equation*}
    \mathcal{P}_{4} = 
    \{ \ 
    \{\{1,2\}, \{3,4\}\},
    \{ \{ 1,3 \}, \ \{ 2,4 \}\},
    \{ \{ 1,4 \}, \ \{ 2,3 \}\}\} 
    \ \}.
\end{equation*}
We will denote the general element of $\mathcal{P}_{m}$ as $\{ \ \{a_1, a_2\}, \ \ldots \,  \{a_{m-1}, a_{m}\} \ \}$. And these indices are exactly ones from the formula (\ref{Wick-theorem}). So one can determine the set of pairings with the array 
$\{a_k\}_{k=1}^{m}$, and we will use this array to determine the indices of $i_{a}$.

And now we are going to calculate the number of all equal terms in the sum over all pairings in the expression (\ref{Wick-theorem}) assuming, that $i_1,\ldots,i_{m_1} = 1$; $i_{m_1+1},\ldots,i_{m_2} = 2$; $\ldots$ and $i_{m_{n-1}+1},\ldots,i_{m_n}=n$, which corresponds to our object of interest (\ref{correlators-def}).

We start from the case of $n=2$, so the index $a$ takes only values $1$ and $2$. Then there are only two different types of terms $G_{ij}$:
\begin{enumerate}
    \item when $i=j$, then $G_{ij}=G(0)$;
    \item when $i\neq j$, then $G_{ij}=G_{12}$.
\end{enumerate}
As a result, every term in (\ref{Wick-theorem}) has the form $G(0)^{q}\left(G_{12}\right)^{p}$ for some integers $p,q\geq 0$. So we have to count the number of pairings $\{a_k\}_{k=1}^{m}$, corresponding to every possible $p$ and $q$. Suppose we consider $\left\langle\phi_1^{m_1}\phi_2^{m_2}\right\rangle_{G}$ for $m_{1}>0$ and $m_{2}>0$. From the conservation of the total points number one can deduce that $m=m_1+m_2=2q+2p$. So, for given $m_{1}$ and $m_{2}$ the number $q$ is uniquely determined with $p$.

We receive the combinatorial factor from the following considerations: there are $m_{1}$ fields with index $1$ and $m_{2}$ fields with index $2$. It is possible to visualize them with the Figure \ref{fig:comb-formila-pic}. We will denote all the enterings of the fields with the points, and for every pairing we will draw the segment ending in the points constituting a pair. We will also refer to these segments as edges, being inspired by the graph theory. In the following we will mean under the notion ``configuration of pairings'' the set of all pairings which give the equal contributions in the sum (\ref{Wick-theorem}). In fact, this is a set of pairings (subset of $\mathcal{P}_m$), consisting of $\{ \ \{a_1, a_2\}, \ \ldots \, \{a_{m-1}, a_{m}\} \ \}$ for some $a_k$ which possess the same number of pairings between different clusters of points and inside every cluster. For $n=2$ it is equivalent that they have the same $p$, but the number of ``degrees of freedom'' $n(n-1)/2$ rises for greater $n$. One can draw an analogy with statistical physics: we identify microstate with the particular set $\{ \ \{a_1, a_2\}, \ \ldots \, \{a_{m-1}, a_{m}\} \ \}$ and macrostate is what we call the configuration of pairs.

Then we have $m_{1}+m_{2}$ points in total, and they all have to be connected by such segments in pairs. And these points can be naturally distributed between two clusters, corresponding to the indices of the fields. For the convenience, we also introduce the notations $l_{a}:=m_{a}-p$ for the number of points which will be paired with the points from the same cluster. From the Isserlis--Wick theorem (\ref{Wick-theorem}) it follows, in particular, that for odd $m_{1}+m_{2}$ the considering correlator is zero, so we will consider only even case. So, the term $G(0)^{q} \left(G_{12}\right)^{p}$ corresponds to $p$ segments with the ends in different clusters, and $q$ segments, which have ends in the same cluster. To draw all the segments which such condition one have to:
\begin{enumerate}
    \item choose $p$ points from every cluster which will give rise to the segments with ends in the different clusters;
    \item choose the way we draw the lines between the chosen $p$ points in every cluster;
    \item choose the way the remaining $m_1-p$ and $m_2-p$ points in first and second clusters correspondingly will be connected with the segments inside their clusters.
\end{enumerate}

The first number is $\binom{m_1}{p}\binom{m_2}{p}$ by combinatorial definition, according to the combinatorial rule of product, where $\binom{n}{k} = \frac{n!}{k!(n-k)!}$ is a number of combinations of $k$ elements from a subset of cardinality $n$. The second one is $p!$, since for the first point (for any fixed enumeration) in the first cluster we have $p$ variants of choosing a pair, for the second one --- $(p-1)$, and so on. And the third number equals to $\frac{l_a!}{2^{l_a/2}\left(l_a/2\right)!}$ for each of two clusters, so in total these two contributions should be multiplied. The explanation for the last formula is that we have to choose firstly one pair, which is possible in $l_a(l_a-1)/2$ ways, then the another one, for which we have $(l_a-2)(l_a-3)/2$ ways, and so on. Finally, we have to divide the product of such ways' numbers by the permutation number of points' pairs, which is $\left(l_a/2\right)!$. We also note that the numbers $l_a$ have to be even for any possible configuration, and the number of pairs inside the cluster is $l_a/2$.

In total for the number $M_{p}$ of pairings with contribution $G(0)^{q}\left(G_{12}\right)^{p}$, using the same combinatorial rule of product, we have the following expression:
\begin{equation}
    M_p = \binom{m_1}{p}\binom{m_2}{p}\cdot p!\cdot 
    \frac{l_1!}{2^{l_1/2}\left(l_1/2\right)!} 
    \frac{l_2!}{2^{l_2/2}\left(l_2/2\right)!}= 
    \frac{m_1! m_2!}
    {2^{l_1/2+l_2/2} p!(l_1/2)!(l_2/2)!}\, .
\end{equation}
We have specially written this formula in such notations to make its generalisation for $n>2$ more clear.  

At this moment, for the two-dimensional Gaussian moment with integer powers $m_{1}$ and $m_{2}$ the expression reads:
\begin{equation}
    \left\langle\phi_{1}^{m_1}
    \phi_{2}^{m_2}\right\rangle _{G}=
    \sum\limits_{p=0}^{\text{min}\{m_{1},m_{2}\}}
    \frac{m_1!m_2!}
    {2^{\frac{m_1+m_2}{2}-p} p!
    \left(\frac{m_1-p}{2}\right)! 
    \left( \frac{m_2-p}{2}\right)!}\,
    \left(G_{12}\right)^{p}
    G\left(0\right)^{\frac{m_1+m_2}{2}-p},
\label{comb-form-n=2}
\end{equation} 
where the summation is carried out over all $p$, such that $l_a=m_a-p$ are even, since only these configurations of pairings are possible. For the considering case, when $m_{1}+m_{2}$ is even, the two cases are realisable: $m_{1}$ and $m_{2}$ are both even or both odd. In the first case the summation condition means that the summation is carried out over only even $p$, and in the second --- over only odd $p$. The sense of these facts becomes clear from Figure \ref{fig:comb-formila-pic}.

\begin{figure}
\begin{centering}
\includegraphics[width=10cm,height=5cm]{comb-form-n=2}
\par\end{centering}
\caption{Illustration of combinatorial formula for the two-dimensional Gaussian moment with integer powers $m_{1}$ and $m_{2}$.}
\label{fig:comb-formila-pic}
\end{figure}

Let us generalize the resulting formulas to the $n$-dimensional case. Acting in the same spirit, one can verify the validity of the following expression:
\begin{equation}
    \left\langle\phi_{1}^{m_1}
    \ldots\phi_{n}^{m_n}\right\rangle _{G}= 
    \sum\limits_{\{l_{ab}\}}
    2^{-\sum\limits_{a=1}^{n}
    \frac{l_{aa}}{2}}
    \frac{\prod\limits_{a=1}^{n}
    \left(m_a\right)!}
    {\prod\limits_{a<b}^{n}\left(l_{ab}!\right)
    \prod\limits_{a=1}^{n}
    \left(\frac{l_{aa}}{2}\right)!}\,
    \prod\limits_{a<b}^{n}
    \left(G_{ab}\right)^{l_{ab}}
    G\left(0\right)^{\frac{1}{2}
    \sum\limits_{a=1}^{n}l_{aa}}.
\label{comb-form-n=general}
\end{equation} 
In accordance with the notations, introduced in the section \ref{summary}, the summation in the expression (\ref{comb-form-n=general}) is carried out over all $l_{ab}\geq 0$, satisfying the conditions $\sum\limits_{b=1}^{n}l_{ab}=m_{a}$ (where $l_{ab}=l_{ba}$) and $l_{aa}$ is even. This expression is non-zero for even $\sum\limits_{a=1}^{n}m_{a}$ and zero for odd. The formula (\ref{comb-form-n=general}) truly generalises (\ref{comb-form-n=2}) and we will use it in our further calculations.

Let us derive the expression (\ref{comb-form-n=general}). It can be done in the same manner as for $n=2$, using the same graphical interpretation. We start from introducing slightly different notations. In general case we have $n$ clusters, consisting of $m_{1},\ldots, m_{n}$ points, correspondingly. At present, the contribution to the sum in (\ref{Wick-theorem}) of a given pairings configuration is prescribed with the numbers $l_{ab}$ (indices $a,b\in\{1,\ldots,n\}$):
\begin{enumerate}
    \item the numbers $l_{ab}$ ($a\neq b$) of edges connecting $a$th and $b$th clusters;
    \item the numbers $l_{aa}$ of edges connecting the vertices inside the $a$th cluster.
\end{enumerate}
We will suppose that in the notation $l_{ab}$ always $a\leq b$, but for the clarity of formulas (summation conditions in (\ref{comb-form-n=general})) we will imply $l_{ab}=l_{ba}$ each time we use both orders of indices, remembering that there are only $n(n-1)/2$ such numbers in fact. Each such a configuration of pairings gives the contribution $\prod\limits_{a<b}^{n}\left(G_{ab}\right)^{l_{ab}}  G\left(0\right)^{\frac{1}{2}\sum\limits_{a=1}^{n}l_{aa}}$. The numbers $l_{ab}$ are visualised in the Figure \ref{fig:comb-formila-n=3-pic}.

Further, it is necessary to count how many pairings configurations are described with the set of numbers $l_{ab}$ for $a<b$ and $l_{aa}$. The logic is similar to the case $n=2$. We have to multiply:
\begin{enumerate}
    \item the number of ways to choose from every $a$th cluster $l_{ab}$ points going to $b$th cluster for all $b$. This is a typical problem of multiple choices from $m_{a}$ objects firstly $l_{a1}$, then $l_{a2}$ and so on until $l_{an}$ objects (including $l_{aa}$ objects). As a result, this number equals to the multinomial coefficient $\binom{m_a}{l_{a1} \ \ldots \ l_{an}}$;
    \item for every $a$th cluster, the number of ways to form pairs inside of it from the remaining $l_{aa}$ points, which is $\frac{l_{aa}!}{2^{l_{aa}/2}(l_{aa}/2)!}$, with the same explanation as for $n=2$;
    \item for every pair of clusters, $a$th and $b$th, the number of ways to draw the edges from $l_{ab}$ chosen points in $a$th cluster to $l_{ab}$ chosen points in $b$th cluster. As a result, this number equals to $l_{ab}!$.
\end{enumerate}

\begin{figure}
\begin{centering}
\includegraphics[width=12cm,height=10cm]{comb-form-n=3}
\par\end{centering}
\caption{Illustration of combinatorial formula for the three-dimensional Gaussian moment with integer powers $m_{1}$, $m_{2}$ and $m_{3}$.}
\label{fig:comb-formila-n=3-pic}
\end{figure}

Multiplying all the described factors due to combinatorial rule of product, for the total number of ways $M_{\{ l_{ab}\}}$ to realise the configuration of pairings with numbers $l_{ab}$ for $a<b$ and $l_{aa}$ we obtain the following result:
\begin{equation}
    M_{\{ l_{ab}\}}=\prod\limits_{a=1}^{n}
    \binom{m_a}{l_{a1}\ldots l_{an}} 
    \prod\limits_{a=1}^{n}
    \frac{l_{aa}!}{2^{l_{aa}/2}(l_{aa}/2)!}
    \prod\limits_{a<b}^{n}(l_{ab})!\,.
\end{equation}
Recalling the formula for multinomial coefficients through the factorials, we arrive at the following expression for 
$M_{\{ l_{ab}\}}$:
\begin{equation}
    M_{\{ l_{ab}\}}=
    2^{-\sum\limits_{a=1}^{n}
    \frac{l_{aa}}{2}}
    \frac{\prod\limits_{a=1}^{n}
    \left(m_a\right)!}
    {\prod\limits_{a<b}^{n}\left(l_{ab}!\right)
    \prod\limits_{a=1}^{n}
    \left(\frac{l_{aa}}{2}\right)!}.
\end{equation}
Now let us write the expression for the $n$-dimensional Gaussian moment in terms of $M_{\{ l_{ab}\}}$:
\begin{equation}
    \left\langle\phi_{1}^{m_1}\ldots
    \phi_{n}^{m_n}\right\rangle _{G}=
    \sum\limits_{\{l_{ab}\}} 
    \prod\limits_{a<b}^{n}
    \left(G_{ab}\right)^{l_{ab}}  
    G\left(0\right)^{\frac{1}{2}
    \sum\limits_{a=1}^{n}l_{aa}} 
    M_{\{ l_{ab}\}}.
    \label{comb-form-n=general_M_implicit}
\end{equation} 
The expression (\ref{comb-form-n=general_M_implicit}) gives us exactly the result (\ref{comb-form-n=general}) after substituting the explicit form of $M_{\{ l_{ab}\}}$. Let us note again, that all $l_{aa}$ have to be divisible by $2$, otherwise such a configuration is impossible. This condition is reflected in the summation condition in (\ref{comb-form-n=general}).

\subsubsection{Numerical Analysis of the Polynomial Approximations}
\label{numerical-appr-an}

In this subsection we are going to use different families of polynomials for the approximation and compare the results. Let us compare pointwise approximations of $|t|^{\alpha}$ by Bernstein, Chebyshev and Legendre polynomials with each other. The results are represented on the Figure \ref{fig:graph-pol-apprs} for $\alpha=4/3$. For other $\alpha\in(1,2)$ there is no significant difference. From the Figure \ref{fig:graph-pol-apprs} one can draw a conclusion that Legendre and Chebyshev polynomials are better for the approximation of integrands than Bernstein ones, since for the same degree $2N$ in the ``majority of points'' they give smaller error. Moreover, errors of Legendre and Chebyshev polynomial approximations oscillate and change sign, rather than Bernstein polynomial approximation, so one should expect that there will be some ``cancellation of errors''  in integrals, which will additionally raise the precision of the approximation.

\begin{figure}
\begin{centering}
\includegraphics[width=6.5cm,height=4cm]{pol_apprs} \includegraphics[width=6.5cm,height=4cm]{pol_errors} 
\par\end{centering}
\caption{Comparative plots of approximations $h_N$ (\ref{pol-appr-def}) by polynomials of the tenth degree ($N=5$)}
\label{fig:graph-pol-apprs}
\end{figure}

Now we are going to compare Bernstein, Chebyshev and Legendre polynomial approximations for integrals. We would like to simplify the general formula (\ref{G_I[j]-general-polynomial-final-without-comb}), since we don't want to waste too many resources for choosing the best variant of the listed three. Namely, we will:
\begin{enumerate}
    \item assume that the source $j=0$ to escape additional parameter, as well as $\varepsilon=0$, since it is possible to remove this regulator for zero source;
    \item replace all the graphs $\Gamma$ in the sum inside the operator $\mathcal{O}$ (\ref{difference-operator}) with the complete graph $K_{n}$, which is the graph without loops, where every two vertices are connected with an edge. This means that $\nu_{ab}\left(K_{n}\right)=\left(J_{n}\right)_{ab}-\delta_{ab}$ (the matrix $J_{n}$ is the matrix of ones) for all $a,b\in\{1,\ldots,n\}$, and the sum over graphs transforms into the contribution of the $K_{n}$, multiplied by the total number of connected graphs on $n$ vertices, which we will denote by $|\mathbb{G}_{C,n}|$ (the explicit value of this number in terms of $n$ is not needed);
    \item assume that $G(x)\equiv G(0) = \text{const}$ for all $x$. From this follows that $G_{n}$ becomes proportional to a matrix of ones, i.e. $G_{n}=G(0)J_{n}$. Besides, $\left(G_{n,K_n}\right)_{aa}=G(0)$ and $\left(G_{n,K_n}\right)_{ab}=s_{ab}G(0)$ for $a \neq b$. 
\end{enumerate}

\begin{figure}
\begin{centering}
\includegraphics[width=7cm,height=4cm]{BPA}\includegraphics[width=7cm,height=4cm]{ChPA}
\par\end{centering}
\begin{centering}
\includegraphics[width=7cm,height=4cm]{LPA}
\par\end{centering}
\caption{Comparative plots of approximations by polynomials of integrals $H_{L,B,Ch}(n,N)$}
\label{fig:graph-pol-apprs-int}
\end{figure}

This gives us the following rough approximation of connected Green functions GF, keeping in mind the definition (\ref{correlators-def}):
\begin{equation} 
\begin{split}
    \mathcal{G}_{I,n}[0]_{N} 
    &\approx\frac{\left(-1\right)^{n}
    \left(2G(0)\right)^{n\alpha/2}}{n!} 
    |\mathbb{G}_{C,n}|
    \left\{ \prod_{a<b}^{n}
    \int_{0}^{1}ds_{ab}\,
    \partial_{s_{ab}}\right\} 
    \left\{\prod_{a=1}^{n}
    \int_{\mathbb{R}^{d}} dx_{a} \, 
    g\left(x_{a}\right)\right\} \\ 
    &\times\sum\limits_{q=0}^{N}
    \sum\limits_{i=0}^{q}
    \frac{u_{i}c_{q}(f)}{2^{ni}G(0)^{ni}} \frac{\varGamma\left(\frac{n(\alpha+1)}{2}\right)}{\varGamma\left(\frac{n}{2}+ni\right)} \left\langle\phi^{2i}_{1}\ldots
    \phi^{2i}_{n}\right\rangle_{G_{n,K_n}}.
\end{split}
\end{equation} 

Now we can calculate coordinate integrals, since all the dependence of $x_a$ is left only in coupling constants $g(x_a)$ as well as the result of the integro-differential operator over the variables $s_{ab}$ action. To that end, we should substitute the combinatorial formula (\ref{comb-form-n=general}). And we arrive at the following expression:
\begin{equation}
    \left\{\prod_{a<b}^{n}\int_{0}^{1}ds_{ab}\,
    \partial_{s_{ab}}\right\} 
    \left\langle\phi^{2i}_{1}\ldots\phi^{2i}_{n}
    \right\rangle_{\frac{G_{n,K_n}}{G(0)}}\!=\!
    \sum\limits_{\{l_{ab}\}}
    \frac{\left(2i\right)!^{n}\, 
    2^{-\sum\limits_{a=1}^{n}
    \frac{l_{aa}}{2}}}
    {\prod\limits_{a<b}^{n}
    \left(l_{ab}!\right)
    \prod\limits_{a=1}^{n}
    \left(\frac{l_{aa}}{2}\right)!}\,
    \left\{ \prod_{a<b}^{n}\int_{0}^{1}ds_{ab}\,
    \partial_{s_{ab}} s_{ab}^{l_{ab}}\right\},
\end{equation}
and the last factor gives nothing else but the additional restriction of the summation condition $l_{ab}>0$ for $a<b$. However, we ignore this condition, despite the fact that this will lead to a worse approximation, since this will also lead to a decrease in the resources expended. Folding combinatorial sum back into correlator for the dimensionless covariance matrix (which is exactly the matrix of ones $J_n$), we get to the approximation:
\begin{equation}
    \mathcal{G}_{I,n}[0]_{N} 
    \!\approx\! \frac{\left(-1\right)^{n}
    \left(2G(0)\right)^{n\alpha/2}}{n!} 
    |\mathbb{G}_{C,n}| 
    g\left(\mathbb{R}^{d}\right)^n \sum\limits_{q=0}^{N}\sum 
    \limits_{i=0}^{q}\frac{u_{i} c_{q}(f)}{2^{ni+1}} \frac{\varGamma\left(\frac{n(\alpha+1)}{2}\right)}{\varGamma\left(\frac{n}{2}+ni \right)} \left\langle\phi^{2i}_{1}\ldots
    \phi^{2i}_{n}\right\rangle_{J_n}.
    \label{G_I[0]-rough-estimation}
\end{equation}
From the formula (\ref{G_I[0]-rough-estimation}) it is clear that for the comparison of different approximations it is useful to consider the following quantities $H\left(n,N\right)$:
\begin{equation}
    H\left(n,N\right)= 
    \frac{(-1)^n}{\left(2G(0)\right)^{n\alpha/2}
    |\mathbb{G}_{C,n}|g\left(\mathbb{R}^{d}\right)^n}\,
    \mathcal{G}_{I,n}[0]_{N},
    \label{H-def}
\end{equation}
where we have removed the common (depending on $n$ only) factors for all polynomial approximations from (\ref{G_I[0]-rough-estimation}), but left the factor $1/n!$ to avoid factorial growth. Recall that the degree of polynomial approximation in all the cases is equal to $2N$. So in fact all the parameters of connected Green functions GF polynomial approximations are roughly encoded in function $H$, explicit formula for which is:
\begin{equation}
    H\left( n,N \right)=\sum\limits_{q=0}^{N}
    \sum \limits_{i=0}^{q}
    \frac{u_{i} c_{q}(f)}{2^{ni}} \frac{\varGamma\left(\frac{n(\alpha+1)}{2}\right)}{\varGamma\left(\frac{n}{2}+ni \right)} \left\langle\phi^{2i}_{1}\ldots
    \phi^{2i}_{n}\right\rangle_{J_n}.
\label{H-formula}
\end{equation}

Further, for different kinds of polynomial approximations there will be different coefficients $u_i$ and $c_q(f)$. In the following, we will mark the family of polynomials we use for the approximation in the index of $H$. Thus, we have: 
\begin{equation*} 
\begin{split}
    &H_{L}(n,N):=\sum\limits_{q=0}^{N}
    \sum\limits_{i=0}^{q}
    \left(2q+\frac{1}{2}\right)
    \sum \limits_{p=0}^{q}
    \binom{2q}{2p}\binom{q+p-\frac{1}{2}}{2q}
    \frac{2^{4q-ni+1}}{2p+\alpha+1}\\
    &\times\binom{2q}{2i}\binom{q+i-\frac{1}{2}}{2q}
    \frac{\varGamma\left(\frac{n(\alpha+1)}{2}\right)}{G(0)^{ni}
    \varGamma\left(\frac{n}{2}+ni\right)n!}\,
    \left\langle\phi^{2i}_{1}\ldots
    \phi^{2i}_{n}\right\rangle_{J_n},
\end{split} 
\end{equation*} 
for Legendre polynomial approximation, and:
\begin{equation*} 
    H_{B}(n,N):=\sum\limits_{q=0}^{N}
    \sum\limits_{i=q}^{N}
    \left(\frac{q}{N}\right)^{\alpha/2}
    \frac{(-1)^{i+q}N!}{q!(i-q)!(N-i)!} \frac{\varGamma\left(\frac{n(\alpha+1)}{2}\right)}
    {2^{ni}G(0)^{ni}
    \varGamma\left(\frac{n}{2}+ni\right)n!}\left\langle\phi^{2i}_{1}\ldots
    \phi^{2i}_{n}\right\rangle_{J_n},
\end{equation*} 
for Bernstein polynomial approximation, as well as:
\begin{equation*} 
\begin{split}
    &H_{Ch}(n,N):=
    \frac{\varGamma\left(\frac{\alpha}{2}+
    \frac{1}{2}\right)}
    {\varGamma\left(\frac{\alpha}{2}+1\right)}
    \frac{\varGamma\left(\frac{n(\alpha+1)}{2}\right)}{\varGamma\left(\frac{n}{2}\right)}+
    2\sum \limits_{q=1}^{N}\sum\limits_{i=0}^{q}
    \frac{(-1)^{i}q^{2}
    \varGamma\left(\frac{n(\alpha+1)}{2}\right)}
    {2^{(n+2)i-4q}G(0)^{ni}n!
    \varGamma\left(\frac{n}{2}+ni\right)}\\ 
    &\times\frac{(2q-i-1)!}{i!(2q-2i)!}
    \left(\sum\limits_{p=0}^{q}
    (-1)^{p}\frac{(2q-p-1)!}{p!(2q-2p)!}2^{-2p}
    \frac{\varGamma\left(q-p+\frac{\alpha}{2}+\frac{1}{2}\right)}{\varGamma\left(q-p+\frac{\alpha}{2}+1\right)}\right)
    \left\langle\phi^{2i}_{1}\ldots
    \phi^{2i}_{n}\right\rangle_{J_n},
\end{split} 
\end{equation*}
for Chebyshev polynomial approximation. The plots of $H_{L,B,Ch}(n,N)$ for $n=2,\ldots,5$ and $N=1,2,4$ and $\alpha=4/3$ are presented in the Figure \ref{fig:graph-pol-apprs-int}. The picture for other values of $\alpha \in (1,2)$ is qualitatively the same. 

From the plots one can see that Legendre and Chebyshev approximations for integrals also converge much better than Bernstein approximation. At the same time, formulas for approximation by Legendre polynomials are simpler than for approximation by Chebyshev polynomials. This, in particular, explains our choice of the Legendre polynomials. Let us also note that using the second-degree approximation gives an error which is about $20\%$ and the error of fourth-degree is about $10\%$. However, it is important to note that this error occurs in the model system considered in this subsection. As the results of the following subsections show, in a more realistic case, the errors can be larger.

\subsubsection{Final Results of Polynomial Approximations for Connected Green Functions GF}
\label{final-forms-pol-appr}

According to all points described above, we will focus on Legendre polynomial approximation. And exactly this kind of approximation we will mean in the following under the term ``polynomial approximation''. We will write these expressions in compact forms using the already introduced notations as well as few new ones. Namely, we will additionally introduce the notations for the arguments of Gamma functions and a new operator $\mathcal{O}_{g}$:
\begin{equation}
    \alpha_n:=\frac{n(\alpha+1)}{2},\quad 
    n_i:=\frac{n}{2}+i,\quad
    \mathcal{O}_{g}=\mathcal{O} 
    \left\{\prod_{a=1}^{n}
    \int_{\mathbb{R}^{d}} dx_{a}\,
    g\left(x_{a}\right)\right\}.
    \label{gamma-func-args-notation}
\end{equation}
Then the expression (\ref{G_I[j]-general-polynomial-final-without-comb}) will take a form:
\begin{equation} 
\begin{split}
    &\mathcal{G}_{I,\varepsilon,n}[j]_{N} =
    \frac{\left(-1\right)^{n}}{n!}\,\mathcal{O}_{g}\,
    e^{-\mathcal{Q}_{n,\Gamma}
    \left(x,\varphi\right)}
    \sum\limits_{k=0}^{\infty}
    \frac{\left(2\right)^{n\alpha/2}G(0)^{2k + n\alpha/2}}{4^{k}k!} \frac{\left(\alpha_n\right)_{2k}}
    {\left(1/2\right)_{2k}}
    \sum\limits_{q=0}^{N}
    \sum\limits_{i=0}^{q}
    \frac{u_{i} c_{q}(f)}{2^{ni+1}}\\
    &\times\sum_{\beta_{1}+\ldots +\beta_{n}=2k}
    \binom{2k}{\beta_{1}\ldots\beta_{n}}
    \chi_{1}^{\beta_{1}}\ldots \chi_{n}^{\beta_{n}}
    \frac{\varGamma\left(\alpha_n\right)}
    {\varGamma\left(n_i + k\right)} \left\langle\phi^{2i+\beta_1}_{1}\ldots
    \phi^{2i+\beta_n}_{n}
    \right\rangle_{\frac{G_{n,\Gamma}}{G(0)}}\, ,
    \label{G_i[j]-short}
\end{split} 
\end{equation}
where we have rewritten correlators, using dimensionless covariance matrix $G_{n,\Gamma}/G(0)$ for the convenience. And for zero source $j$ similarly:
\begin{equation} 
    \mathcal{G}_{I,n}[0]_{N} =
    \frac{\left(-1\right)^{n}
    \left(2G(0)\right)^{n\alpha/2}}{n!}\,
    \mathcal{O}_{g}\sum\limits_{q=0}^{N}
    \sum\limits_{i=0}^{q}
    \frac{u_{i} c_{q}(f)}{2^{ni+1}} \frac{\varGamma\left(\alpha_n\right)}
    {\varGamma\left(n_i\right)} 
    \left\langle\phi^{2i}_{1}\ldots
    \phi^{2i}_{n}\right\rangle_{\frac{G_{n,\Gamma}}{G(0)}}\, .
    \label{G_I[0]-short}
\end{equation} 
In this form the structure of expressions is much more clear, so in the following we will use these formulas for representing connected Green functions GF.

For the reference we also write down the extended form of the expressions (\ref{G_i[j]-short}) and (\ref{G_I[0]-short}). Substituting all the introduced notation, we receive big, big formulas: 
\begin{equation} 
\begin{split}
    &\mathcal{G}_{I,\varepsilon,n}[j]_{N}\!=\!
    \frac{\left(-1\right)^{n}}{n!}\sum \limits_{\Gamma\in\mathbb{G}_{C,n}}
    \left\{ \prod_{a<b}^{n}
    \int_{0}^{1}ds_{ab}\,
    \partial_{s_{ab}}^{\nu_{ab}
    \left(\Gamma\right)}\right\} 
    \left\{ \prod_{a=1}^{n}\int_{\mathbb{R}^{d}} dx_{a}\,g\left(x_{a}\right)\right\}\,
    e^{-\mathcal{Q}_{n,\Gamma}
    \left(x,\varphi\right)}\\
    &\times\sum\limits_{k=0}^{\infty}
    \frac{\left(2G(0)\right)^{n\alpha/2}}{4^{k}k!}\frac{\left(\frac{n(\alpha+1)}{2}\right)_{2k}}{\left(1/2\right)_{2k}}\sum 
    \limits_{\beta_{1}+\ldots +\beta_{n}=2k}
    \binom{2k}{\beta_{1}\ldots\beta_{n}}
    \chi_{1}^{\beta_{1}}\ldots \chi_{n}^{\beta_{n}}\\
    &\times\sum\limits_{q=0}^{N}
    \sum \limits_{i=0}^{q}
    \left(2q+\frac{1}{2}\right)2^{4q-ni+1} 
    G(0)^{2k}
    \frac{\varGamma
    \left(\frac{n(\alpha+1)}{2}\right)}
    {\varGamma\left(\frac{n}{2}+ni+k\right)}\\ 
    &\times\sum\limits_{p=0}^{q}
    \binom{2q}{2p}\binom{q+p-\frac{1}{2}}{2q}
    \frac{1}{2p+\alpha+1}
    \binom{2q}{2i}\binom{q+i-\frac{1}{2}}{2q}\\
    &\times\sum\limits_{\{l_{ab}\}}
    2^{-\sum\limits_{a=1}^{n}\frac{l_{aa}}{2}}
    \frac{\prod\limits_{a=1}^{n}
    \left(2i+\beta_{a}\right)!}
    {\prod\limits_{a<b}^{n}\left(l_{ab}!\right)
    \prod\limits_{a=1}^{n}
    \left(\frac{l_{aa}}{2}\right)!}
    \prod\limits_{a<b}^{n}
    \left(\frac{s_{ab}\nu_{ab}(\Gamma)
    G\left(x_{a}-x_{b}\right)}
    {G(0)}\right)^{l_{ab}}.
    \label{main_polynomial_approximation_formula}
\end{split} 
\end{equation} 
In accordance with the notations, introduced in the section \ref{summary}, the summation in the last line of the expression (\ref{main_polynomial_approximation_formula}) is carried out over all $l_{ab}\geq 0$, satisfying the conditions $\sum\limits_{b=1}^{n}l_{ab}=2i+\beta_{a}$ (where $l_{ab}=l_{ba}$) and $l_{aa}$ is even. The expression (\ref{main_polynomial_approximation_formula}) is the main polynomial approximation formula for the following computations. And for $j=0$ we can also remove the regulator $\varepsilon$ and write:
\begin{equation} 
\begin{split}
    &\mathcal{G}_{I,n}[0]_{N}=\frac{(-1)^{n}
    \left(2G(0)\right)^{n\alpha/2}}{n!} 
    \sum\limits_{\Gamma\in\mathbb{G}_{C,n}}
    \left\{\prod\limits_{a<b}^{n}
    \int_{0}^{1}ds_{ab}\,
    \partial_{s_{ab}}^{\nu_{ab}(\Gamma)}\right\} 
    \left\{\prod\limits_{a=1}^{n}
    \int_{\mathbb{R}^d} dx_{a}\, g(x_a)\right\} \\ 
    &\!\times\!\sum\limits_{q=0}^{N}
    \sum\limits_{i=0}^{q}
    \left(2q+\frac{1}{2}\right) 
    \sum\limits_{p=0}^{q}
    \binom{2q}{2p}\binom{q+p-\frac{1}{2}}{2q}
    \frac{2^{4q-ni+1}}{2p+\alpha+1}
    \frac{\varGamma
    \left(\frac{n(\alpha+1)}{2}\right)}
    {\varGamma\left(\frac{n}{2}+ni\right)}
    \binom{2q}{2i}
    \binom{q+i-\frac{1}{2}}{2q}\\ 
    &\!\times\!\sum\limits_{\{l_{ab}\}}
    2^{-\sum\limits_{a=1}^{n}\frac{l_{aa}}{2}}
    \frac{\prod\limits_{a=1}^{n}
    \left(2i\right)!}
    {\prod\limits_{a<b}^{n}\left(l_{ab}!\right)
    \prod\limits_{a=1}^{n}
    \left(\frac{l_{aa}}{2}\right)!}
    \prod\limits_{a<b}^{n}
    \left(\frac{s_{ab}\nu_{ab}(\Gamma)
    G\left(x_{a}-x_{b}\right)}
    {G(0)}\right)^{l_{ab}}.
\end{split} 
\label{G_I[0]-final-form-full}
\end{equation}
As it follows from the discussion in the section \ref{phys-mot}, vacuum energy $\mathcal{E}=-\mathcal{G}[0]$. So in the following computations of vacuum energy density this formula will be our starting point.

Finally, let us note that due to the convergence of Legendre series we have obtained in fact the expression for $\mathcal{G}_{I,\varepsilon,n}[j]$ in terms of repeated series. In other words, in formulas (\ref{G_i[j]-short}-\ref{G_I[0]-final-form-full}) one can proceed to the limit $N\rightarrow\infty$:
\begin{equation}
    \mathcal{G}_{I,\varepsilon,n}[j] = \lim_{N \rightarrow \infty} \mathcal{G}_{I,\varepsilon,n}[j]_{N}. 
\end{equation}
Moreover, at least for $j=0$ the sums over $n$ and $q$ can be permuted because of the absolute convergence of Legendre series. Hence, at least for $\mathcal{G}_I[0]$ we have obtained the expression in terms of double series which is very convenient for calculations. Therefore:
\begin{equation} 
    \mathcal{G}_I[0] = \sum\limits_{n=1}^\infty \sum \limits_{q=0}^\infty  \sum\limits_{i=0}^{q}
    \frac{\left(-1\right)^{n}
    \left(2G(0)\right)^{n\alpha/2}}{n!}\,
    \frac{u_{i}c_{q}(f)}{2^{ni+1}} \frac{\varGamma\left(\alpha_n\right)}
    {\varGamma\left(n_i\right)}\, 
    \mathcal{O}_{g}\left\langle
    \phi^{2i}_{1}\ldots
    \phi^{2i}_{n}\right\rangle_{\frac{G_{n,\Gamma}}{G(0)}}\, ,
    \label{G_I[0]-series}
\end{equation} 
and the same for $\mathcal{G}_{I,\varepsilon}[j]$. These formulas are the main result of the presented paper. In the next subsections we will rewrite the result of applying the operator $\mathcal{O}_g$ to the correlator to make this formula more similar to usual formulas from Feynman diagram approach.

\subsubsection{Two Types of Graphs in the Formulas for Connected Green Functions GF}
\label{note-about-graphs}

To go further, let us provide an improvement of graphical interpretation of the terms from Isserlis--Wick theorem (\ref{Wick-theorem}), developed in the subsection \ref{comb-exp-derivation}. Our new graphical interpretation will be more convenient for our purposes as long as we deal with all the combinatorics.

We are going to consider the clusters (in terms of subsection \ref{comb-exp-derivation}) as vertices, and to draw the edges (corresponding to pairings) between the clusters themselves (in particular, starting and ending clusters can coincide). Therefore, we are going to describe the particular pairings' configuration in terms of the unoriented graph with possible loops and multiple edges, possibly disconnected. It is illustrated on the Figure \ref{fig:Wick-Meyer-graphs}, from which one can recognise the ordinary Feynman diagrammatics. Informally, we took the graphical interpretation from the subsection \ref{comb-exp-derivation} and ``constricted'' every cluster to a point. And the pairings inside every cluster became loops on the corresponding vertices. 

Considering a correlator $\left\langle\phi_{1}^{m_1}\ldots\phi_{n}^{m_n}\right\rangle$ (where we won't specify the covariance matrix in the notation for the brevity) we will enumerate vertices with the indices of corresponding fields $\phi_{a}$ or with the coordinates $x_{a}$ (which come from the covariance matrix with the same indices). In such notations the degree of a vertex (the number of edges that a given vertex belongs to) with the number $a$ equals to $m_{a}$ and $l_{ab}$ is the number of edges between the vertices $a$ and $b$.  Let us note that the adjacency matrix of the described graph is $\left(l_{ab}\right)_{a,b=1}^n$. In the following we will call these graphs as Isserlis--Wick graphs. In the literature they are also known as Feynman graphs.

\begin{figure}
    \centering
    \includegraphics[width=10cm]{Wick-Meyer.png}
    \caption{Illustration of typical Isserlis--Wick (black) and Meyer (red) graphs. The presented Isserlis--Wick graph refers to the correlator $\left\langle\phi_{1}^{6}\phi_{2}^{5}\phi_{3}^{3}\phi_{4}^{10} \phi_{5}^{3}\phi_{6}^{1}\phi_{7}^{4}\right\rangle$.}
    \label{fig:Wick-Meyer-graphs}
\end{figure}

Therefore, in this paper we have two types of graphs:
\begin{enumerate}
    \item Isserlis--Wick graphs, that are the subject of discussion in this subsection. We will denote them as $\Gamma_{IW}$;
    \item Meyer graphs, which came from the Meyer cluster expansion in the subsection \ref{exponentiation-using-Meyer}, and still remaining in the operator $\mathcal{O}$. We will denote them as $\Gamma_{M}$;
\end{enumerate}
And they are essentially different. While the Isserlis--Wick graphs can have loops, multiple edges or be disconnected, the Meyer graphs, in opposite, have no loops, multiple edges and have to be connected. However, there is a link between these two types of graphs, explaining also why physicists have only one notion of Feynman diagrams, serving simultaneously for all the purposes. Namely, for the Meyer graphs that give a non-zero contribution to the expression (\ref{G_I[0]-final-form-full}) the following inclusion holds:
\begin{equation}
    \Gamma_{M}\subset\Gamma_{IW}\, .
    \label{graphs-inclusion}
\end{equation}
This inclusion is non-trivial for the sets of edges, since the sets of vertices are the same by definition. Let us prove this inclusion. Consider the formula (\ref{G_I[0]-final-form-full}). Next, consider concrete $\Gamma_{M}$. Let it contain an edge between vertices $a$ and $b$. In this case $\nu_{ab}\left(\Gamma_{M}\right)=1$. Then differentiation with respect to $s_{ab}$ is non-trivial if and only if the corresponding value $l_{ab}>0$. But this means that there is such an edge in $\Gamma_{IW}$ defined by the given set $l_{ab}$. Repeating this argument for all the edges in $\Gamma_{M}$, we obtain the required inclusion.

In the same manner, if $\nu_{ab}(\Gamma_{M})=0$ for distinct $a$ and $b$, then for every non-zero contribution has to be $l_{ab}=0$ for the same $a$ and $b$. This means that if there is at least one edge in $\Gamma_{IW}$, then there is an edge in $\Gamma_{M}$.

The following useful statement is also true: the summation over Meyer and Isserlis--Wick graphs reduces to the summation over connected Isserlis--Wick graphs, at least for the zero source. Let us prove this statement. With the direct considering of cases, it can be shown that:
\begin{equation}
    \int_0^1 ds \, 
    \partial^\nu_s (s \nu)^l =
    \nu (1-\delta_{l,0}) + 
    (1-\nu) \delta_{l,0} = 
    \begin{cases}
        \delta_{l,0}\, , \ \ \ \qquad \nu = 0; \\
        1-\delta_{l,0}\, , \quad \nu = 0;
    \end{cases},
    \label{Meyer-Wick correspondense}
\end{equation}
for $\nu\in\{0,1\}$ and $l\geq 0$. From the formula (\ref{Meyer-Wick correspondense}) it follows that we have the following product in the expression (\ref{G_I[0]-final-form-full}):
\begin{equation}
    \left\{\prod\limits_{a<b}^{n}
    \int_{0}^{1}ds_{ab}\,
    \partial_{s_{ab}}^{\nu_{ab}
    \left(\Gamma\right)}\right\} 
    \prod\limits_{a<b}^{n}
    \left(s_{ab}\nu_{ab}(\Gamma)\right)^{l_{ab}} = \prod\limits_{a<b}^{n}\left\{\nu_{ab}(\Gamma) 
    (1-\delta_{l_{ab},0}) + 
    (1-\nu_{ab}(\Gamma))\delta_{l_{ab},0} \right\}.
\end{equation}

However, for this product to be non-zero each of its terms have to be non-zero. And for every set $\{l_{ab}\}$ as follows from (\ref{Meyer-Wick correspondense}) there is only one possible adjacency matrix $\nu_{ab}(\Gamma)$, satisfying this condition, and it has the following form:
\begin{equation} 
    \nu_{ab}(\Gamma) = 
    \begin{cases}
        0, \quad l_{ab}=0\, ;\\
        1, \quad l_{ab}>0\, ;
    \end{cases},
    \label{Meyer graph conditions}
\end{equation}
for all nonequal $a$ and $b$. This expression can also be interpreted as the condition for Meyer graph to have the edge if and only if there exists at least one edge between these vertices in Isserlis--Wick graph.

Therefore, for every $\{l_{ab}\}$ (every Isserlis--Wick graph) in the sum over connected (Meyer) graphs $\sum\limits_{\Gamma\in\mathbb{G}_{C,n}}$ there is no more than one term stays alive, and it is completely determined with the condition (\ref{Meyer graph conditions}). Though, in the sum over Meyer graphs there could be no such a graph, since Meyer graphs are always connected but the condition (\ref{Meyer graph conditions}) can be satisfied for disconnected graphs also. So the connectivity of Isserlis--Wick graph is the necessary and sufficient condition for the term to give the non-zero contribution.

Summarising, the operator $\mathcal{O}$ leaves only connected Isserlis--Wick graphs, and we finish at the expression:
\begin{equation}
\begin{split}
    &\mathcal{G}_{I,n}[0]_{N} =
    \frac{\left(-1\right)^{n}
    \left(2G(0)\right)^{n\alpha/2}}{n!}\,
    \sum\limits_{q=0}^{N}
    \sum\limits_{i=0}^{q}
    \frac{u_{i} c_{q}(f)}{2^{ni+1}} \frac{\varGamma\left(\alpha_n\right)}
    {\varGamma\left(n_i\right)}\\
    &\times\left\{\prod\limits_{a=1}^{n}
    \int_{\mathbb{R}^d} dx_{a}\, g(x_a)\right\}
    \left\langle
    \phi^{2i}_{1}\ldots\phi^{2i}_{n}
    \right\rangle^C_{\frac{G_{n,\Gamma}}{G(0)}}\, ,
    \label{G_I[0]-connected-graphs}
\end{split}
\end{equation} 
where the index ``$C$'' means that the summation in the correlator is carried out only over the connected Isserlis--Wick graphs. Equivalently, in the combinatorial sum (\ref{comb-form-n=general}) only those $\{l_{ab}\}$ should be taken into account, that correspond to the connected graphs. In calculations using computer algebra systems, one can check the connectivity of a Isserlis--Wick graph using built-in functions, recalling that $\left(l_{ab}\right)_{a,b=1}^n$ give the adjacency matrix.

In conclusion, the expression (\ref{G_I[0]-connected-graphs}) is an explicit representation of (connected Green functions) GF for fractional power interaction as a weighted sum of integer-power interactions contributions. However, all our effort were for deriving the explicit formulas for weights and obtaining the expression in terms of the converging series rather than asymptotic, which turned out to be quite a complicated task.

\subsubsection{Simple Approximate Formula for General Term}
\label{subsect:approx form of second degree}

In this subsection we will not use the general results on Meyer and Isserlis--Wick graphs, derived in subsection \ref{note-about-graphs} since in the particular case that we are going to consider in this subsection, it is more convenient to carry out all the calculations in a different, simpler and more intuitive way. However, for the higher degrees of polynomials their use is unavoidable because of the calculations complexity.

As one can see from the model system, considered in the subsection \ref{numerical-appr-an}, approximation with the Legendre polynomial of the second degree gives an error which is about $20\%$, and this is quite an accurate result. Remarkably, in this case one can finish up with a not very complicated formula for the vacuum energy. To obtain such a formula we will start from the expression (\ref{G_I[0]-short}) and assume $g(x)=g\chi_{Q}\left(x\right)$:
\begin{equation*} 
\begin{split} 
    \mathcal{G}_{I,n}[0]_{N} & =\frac{\left(-g\right)^{n}
    \left(2G(0)\right)^{n\alpha/2}}{n!}\,\mathcal{O} 
    \left\{\prod_{a=1}^{n}\int_{Q} dx_{a}\right\}
    \sum\limits_{q=0}^{N}
    \sum\limits_{i=0}^{q}
    \frac{u_{i} c_{q}(f)}{2^{ni+1}} \frac{\varGamma\left(\alpha_n\right)}
    {\varGamma\left(n_i\right)}
    \left\langle\phi^{2i}_{1}\ldots
    \phi^{2i}_{n}\right\rangle_{\frac{G_{n,\Gamma}}{G(0)}}\, .
\end{split}
\end{equation*} 
If one writes down this formula in more details, there will appear the following coefficients:
\begin{equation} 
    A_{n,q,\alpha,i}=
    \left(2q+\frac{1}{2}\right)
    \frac{\varGamma\left(\alpha_n\right)}
    {\varGamma\left(n_i\right)}
    \sum\limits_{p=0}^{q}
    \binom{2q}{2p}\binom{q+p-\frac{1}{2}}{2q}
    \frac{2^{4q-ni+1}}{2p+\alpha+1}
    \binom{2q}{2i}\binom{q+i-\frac{1}{2}}{2q},
    \label{A-coefficients}
\end{equation} 
which we denote for the further use. 

So, we want to calculate $\mathcal{G}_{I,n}[0]_{N}$ analytically for $N=1$, which corresponds to the second-degree polynomial approximation. We consider the terms with $n=1,2$ and $n>2$ separately. For the first two orders the calculation is straightforward:
\begin{equation}
     \mathcal{G}_{I,1}[0]_{1}=
     -\frac{3(2\alpha+1)\varGamma
     \left(\frac{\alpha+1}{2}\right)}
     {\sqrt{\pi}(\alpha+1)(\alpha+3)}\,
     \left(2gG(0)\right)^{\alpha/2}V,
\end{equation}
for $n=1$ and:
\begin{equation}
    \mathcal{G}_{I,2}[0]_{1}=
    \frac{15\alpha\varGamma(\alpha +1)}
    {16(\alpha+1)(\alpha+3)}\,
    \left(2gG(0)\right)^{\alpha}
    V\int_{Q}dy\,
    \left(\frac{G(y)}{G(0)}\right)^2,
\end{equation}
for $n=2$. Further it will turn out that this expression will occasionally satisfy the formula for $n>2$, so we will put this term into the sum over $n$ in the following. One important remark: in the calculation of such coordinate integrals one has to change variables and the integration domain will deform. Though, we are interested mainly in finding results in the thermodynamic limit (finding the leading contribution in $V$, when $V\rightarrow \infty$). And in this limit the integration domain deformation is not important. We will describe the thermodynamic limit in subsection \ref{subsect:first-orders-PT}, and it will be more convenient to discuss such integrals there.

Let us proceed to the case $n>2$. Because of the operator $\mathcal{O}$, the constant terms in all $s_{ab}$ will not contribute due to the differentiation. In more detail: it is so, since the sum in (\ref{difference-operator}) is carried out over the connected graphs, so they have at least one edge for $n>1$. As a result, in $\mathcal{O}$ at least exists one derivative, which will annihilate the constant term. This means, due to (\ref{comb-form-n=general}), we
can consider only $i\neq0$, so there will remain only a single term in a sum $\sum\limits_{q=0}^{N}\sum\limits_{i=0}^{q}$, namely $i=q=N=1$:
\begin{equation}
    \mathcal{G}_{I,n}[0]_{1} =
    \frac{(-g)^{n}\left(2G(0\right)^{n\alpha/2}}
    {n!}\,A_{n,1,\alpha,1}\, 
    \mathcal{O}\left\{\prod\limits_{a=1}^{n}
    \int_{Q}dx_{a}\right\} 
    \left\langle\phi^{2}_{1}\ldots
    \phi^{2}_{n}\right\rangle_{\frac{G_{n,\Gamma}}{G(0)}}.
    \label{N=2-half-way}
 \end{equation} 

At present, for the evaluation of the Isserlis--Wick correlators for the second degrees of fields, it is useful to directly sort all the pairings rather than use the obtained combinatorial formulas. Namely, one can sort all the configurations of pairings for
$\left\langle \phi_{1}^{2}\ldots \phi_{n}^{2}\right\rangle_{\frac{G_{n,\Gamma}}{G(0)}}$ by the number $r$ of vertices, pairing with themselves. Other $n-r$ vertices have to constitute some number of closed chains, i.e. graphs whose vertex degrees are equal to $2$, in terms of recently proposed graphical interpretation. This is the case since the degree of every vertex in any Isserlis--Wick graph is two, and all such graphs are disjoint unions of loops and chains. Then we can write:
\begin{equation}
    \left\langle\phi_{1}^{2}\ldots 
    \phi_{n}^{2}\right\rangle_{\frac{G_{n,\Gamma}}{G(0)}}=
    \sum\limits_{r=0}^{n}G(0)^{r-n} 
    \left\{\sum\limits_{i_{1},\ldots,i_{n-r}}
    \left(G_{n,\Gamma}\right)_{{i_{1}},{i_{2}}}
    \left(G_{n,\Gamma}\right)_{{i_{2}},{i_{3}}}\ldots  
    \left(G_{n,\Gamma}\right)_{{i_{n-r}},{i_{1}}}+
    \ldots\right\}.
    \label{some_intermediate_expression}
\end{equation} 
The summation indices $i_{a}$ in the right-hand side of the expression (\ref{some_intermediate_expression}) take values in $\{1,\ldots, n\}$ and enumerate all different closed chains (one closed chain as a graph can be prescribed with different sequences of indices, what will be discussed in more details further). In the expression (\ref{some_intermediate_expression}) we have written only the terms with the unique chain, and denoted the others as ``$\ldots$''. We won't write them down explicitly, since in the following they will give only zero contribution.

Further, substituting the last expression into the expression (\ref{N=2-half-way}), we arrive at the following result:
\begin{equation}
\begin{split}
    &\mathcal{G}_{I,n}[0]_{1} =
    \frac{(-g)^{n}\left(2G(0)\right)^{n\alpha/2}}{n!}
    \, A_{n,1,\alpha,1}\, 
    \mathcal{O}\left\{\prod\limits_{a=1}^{n}
    \int_{Q} dx_{a}\right\}\\
    &\times\left\{\sum\limits_{r=0}^{n}
    \sum\limits_{i_{1},\ldots,i_{n-r}}
    \frac{s_{i_{1},i_{2}}\nu_{i_{1},i_{2}}
    \left(\Gamma\right)
    G(x_{i_{1}}-x_{i_{2}})}{G(0)}\ldots  
    \frac{s_{i_{n-r},i_{1}}\nu_{i_{n-r},i_{1}}
    \left(\Gamma\right)
    G(x_{i_{n-r}}-x_{i_{1}})}{G(0)}+ 
    \ldots\right\}.
    \label{appr-form-half-way}
\end{split} 
\end{equation} 
Because of the operator $\mathcal{O}$, each term in the sum $\sum \limits_{r=0}^{n}$ survives if and only if $\nu_{ab}\left(\Gamma\right)=1$ for $a$ of $b$ belonging to all the vertices which both lie in the chain. From this follows that the graph $\Gamma$ has to be contained in this chain. But since $\Gamma$ is connected, this condition can't be satisfied for the disconnected Isserlis--Wick graphs. This means, that in the sum in (\ref{appr-form-half-way}) survive only the terms corresponding to the connected graphs referring to pairings. In particular, all the terms denoted as ``$\ldots$'' disappear, since they have at least two non-trivial chains and hence they are disconnected. And only the term with $r=0$ in the remaining sum gives the non-zero contribution in $\sum \limits_{\Gamma\in\mathbb{G}_{C,n}}$. One can note that this term corresponds to a cyclic chain, schematically written as $i_{1}-i_{2}-\ldots-i_{n}-i_{1}$ for some distinct $i_{a}\in\{1, \ldots, n\}$. 

Moreover, due to the invariance of every term under permutations of $x_{a}$ in coordinate integration, we can suppose that $i_{a}=a$. And the number of different chains from $n$ vertices equals to $2^n n!/2n$, since $n!$ is a total number of permutations of $x_a$, and we have to divide it by the number of permutations, assigning the same chain, which is $2n$. The thing is, we can make cyclic permutations of finite set of all $x_{a}$, which don't change a chain (and there are $n$ of such transformations) as well as reflections of finite set of all $x_{a}$ (which should not be confused with the transformation $x_{a}\rightarrow -x_{a}$), for which we have exactly $2$ ways. In total it gives us the factor $2n$. Moreover, we have to multiply the obtained number $n!/(2n)$ by $2^n$ since we have two possible choices of fields in each vertex. Though this derivation is valid only for $n>2$, this formula also gives the right result for $n=2$. The illustration of different enumerations of vertices in a given chain is also presented on the Figure \ref{fig:graphs-N-2}.

\begin{figure}
    \centering
    \includegraphics[width=12cm]{graphs-N-2}
    \caption{Typical possible graph from Isserlis--Wick theorem for the for ten-dimensional Gaussian moment with integer powers, equal to $2$. The vertices are enumerated with the coordinates, corresponding to the field insertions. Additionally, the several possible ways of vertices enumeration are presented (clockwise and counterclockwise), illustrating the genesis of a combinatorial factor.}
    \label{fig:graphs-N-2}
\end{figure}

Calculating the integrals over all $s_{ab}$ except for $s_{1,2},\ldots,s_{n,1}$, we obtain:
\begin{equation*} 
\begin{split}
    &\mathcal{G}_{I,n}[0]_{1}=
    \frac{(-2g)^{n}
    \left(2G(0)\right)^{n\alpha/2}}{2n}
    \,A_{n,1,\alpha,1}
    \left\{\prod\limits_{a=1}^{n}
    \int_{Q} dx_{a}\right\}\\ 
    &\times\left\{
    \int_{0}^{1}ds_{12}\,\partial_{s_{12}}\ldots
    \int_{0}^{1}ds_{n1}\,\partial_{s_{n1}}\right\} 
    \frac{s_{12}G(x_{1}-x_{2})}{G(0)}\ldots
    \frac{s_{n,1}G(x_{n}-x_{1})}{G(0)},
\end{split} 
\end{equation*} 
where we have used that after permutation of $x_{a}$ there will be $n!$ equal terms, so $n!$ in the numerator and denominator cancel each other out. The remaining integrals over $s_{ab}$ are easily calculated:
\begin{equation*}
    \mathcal{G}_{I,n}[0]_{1}=
    \frac{(-2g)^{n}
    \left(2G(0)\right)^{n\alpha/2}}{2n}
    \,A_{n,1,\alpha,1}
    \left\{\prod\limits_{a=1}^{n}
    \int_{Q} dx_{a}\right\}
    \frac{G(x_{1}-x_{2})}{G(0)}\ldots
    \frac{G(x_{n}-x_{1})}{G(0)}.
\end{equation*}
Let us change the variables to simplify this expression:
\begin{equation*}
    \begin{cases}
    x_{a}-x_{a+1}=y_{a}, \ \text{ if } a\in\{1,\ldots,n-1\},\\
    x_{n}=y_{n};
    \end{cases}\Rightarrow
    \vec{y}=\left(\begin{array}{ccccc}
    1 & -1 & 0 & \ldots  & 0\\
    0 & 1 & -1 & 0 & \ldots \\
    \ldots  & \ldots  & \ldots  & \ldots  & \ldots \\
    0 & \ldots  & 0 & 1 & -1\\
    0 & 0 & \ldots  & 0 & 1
    \end{array}\right)\vec{x},
\end{equation*}
and the last argument is the sum of all the previous:
\begin{equation*}
    x_{n}-x_{1}=(x_{n}-x_{n-1})+(x_{n-1}-x_{n-2})
    +\ldots +(x_{2}-x_{1}).
\end{equation*}
The Jacobian of such a transformation is equal to $1$, since it is a linear map with upper-triangular matrix. Here we also assume the limit $V\rightarrow\infty$, so we won't change the integration domain (since it is ``big enough yet''), referring to the independent consideration of thermodynamic limit in the subsection \ref{subsect:first-orders-PT}. So we receive the result for $n>2$:
\begin{equation*}
    \mathcal{G}_{I,n}[0]_{1}=
    \frac{(-2g)^{n}
    \left(2G(0)\right)^{n\alpha/2}}{2n}\,
    VA_{n,1,\alpha,1}
    \left\{\prod\limits_{a=1}^{n-1}
    \int_{Q} dy_{a}\,\frac{G(y_{a})}{G(0)}\right\}
    \frac{G\left(\sum\limits_{k=1}^{n-1}y_{k}\right)}{G(0)}.
\end{equation*}
The coefficients $A_{n,1,\alpha,1}$ can also be simplified:
\begin{equation*}
    A_{n,1,\alpha,1}=2^{-n-1}
    \frac{\varGamma\left(\alpha_n\right)}{\varGamma\left(3n/2\right)}\frac{15\alpha}
    {(\alpha+1)(\alpha+3)},
\end{equation*}
so finally the expression for vacuum energy density (\ref{vac-en-density-def}) reads:
\begin{equation}
\begin{split}
    &w_{vac,N=1}=\frac{3 (2 \alpha +1) 
    \varGamma \left(\frac{\alpha +1}{2}\right)}
    {\sqrt{\pi } (\alpha +1) (\alpha +3)}
    \left(2gG(0)\right)^{\alpha /2} + 
    \frac{15\alpha}{(\alpha+1)(\alpha+3)} \\ 
    &\times
    \sum\limits_{n=2}^\infty
    \frac{(-1)^{n-1}2^{n\alpha/2}}{4n}\,
    \left[gG(0)^{\alpha/2}\right]^{n}\frac{\varGamma\left(\alpha_n\right)}{\varGamma\left(3n/2\right)}
    \left\{\prod\limits_{a=1}^{n-1}
    \int_{Q} dy_{a}\,\frac{G(y_{a})}{G(0)}\right\} 
    \frac{G\left(\sum\limits_{k=1}^{n-1}y_{k}\right)}{G(0)},
\label{vac-en-N=1}
\end{split} 
\end{equation} 
where we assume the thermodynamic limit $V\rightarrow\infty$.

The expression (\ref{vac-en-N=1}) is the simple approximate formula we worked for, and in the sections \ref{phys-calc} and \ref{phys-research} we will apply it for the calculations of vacuum energy density for particular cases of propagators $G$. Unfortunately, the analysis of similar formula even for fourth-degree approximation polynomial, i.e. $N=2$, becomes much more complicated due to significant variety of Isserlis--Wick graphs with the desired properties. Let us make a remark that in this subsection we have got in a different way all the results from \ref{note-about-graphs} for the particular case, namely we have checked that all the non-zero contributions correspond to the connected Isserlis--Wick graphs.

Let us also note that in the limiting case $\alpha=2$:
\begin{equation}
    w_{vac,N=1}=-\sum\limits_{n=1}^\infty 
    \frac{(-2g)^{n}}{2n}
    \left\{\prod\limits_{a=1}^{n-1}
    \int_{Q} dy_{a}\, G(y_{a})\right\}
    G\left(\sum\limits_{k=1}^{n-1}y_{k}\right),
\end{equation}
which coincides with the exact result for $\alpha=2$. We will obtain this result in the subsection \ref{alpha=2-verivication}. This is a simple test of the expression (\ref{vac-en-N=1}), since the function $f(t)=t^{2}$ belongs to the linear span of zero-degree and second-degree Legendre polynomials. 

\subsection{Hard-Sphere Gas Approximation}

The obtained general formulas from the subsection \ref{final-forms-pol-appr} are still difficult for analytical calculations since of the coordinate integrals. So we have to make some assumptions to obtain simpler and hence more useful formulas. Namely, the formulas become significantly simpler if we use for $G_{n}$ the HSG approximation, inspired by statistical physics. In this subsection, we will also consider the coupling constant in the form $g(x)=g\chi_{Q}\left(x\right)$, emulating a system with fixed coupling constant in finite volume $V$. In accordance with the expression (\ref{HSG-def}) in the section \ref{summary}, we keep all the diagonal elements equal to $G(0)$, since they are constant, and ``cut off'' all the off-diagonal elements for $|x_{a}-x_{b}|>\delta$. For the graphical illustration on the Figure \ref{fig:HSG-sence}, we will denote the right hand side of the expression (\ref{HSG-def}) as $G_{n,\text{HSG}}$. Finally, according to the section \ref{summary}, we will denote the ``volume of one hard-sphere particle'' as $v$. Let us note, that all the quantities in HSG approximation will be expressed in terms of $v$ and $\gamma$, so the obtained formulas won't have an explicit dependence on the definition of these parameters.

\begin{figure}
    \centering
    \includegraphics[width=12cm]{HSG-sense}
    \caption{Qualitative illustration of HSG approximation.}
    \label{fig:HSG-sence}
\end{figure}

We start from the formula (\ref{GFZ_exp_4}), which precedes the Parseval--Plancherel identity, in order to avoid the appearance of $G_{n,\Gamma}^{-1}$ and $1/\sqrt{\det G_{n,\Gamma}}$. Substituting the notion of measure (\ref{Def_of_com_val_meas}) and suggesting that $j=0$, we arrive at the following expression:
\begin{equation*} 
\begin{split}
    &\mathcal{G}_{I,\varLambda,\varepsilon,n}[0] 
    =\frac{\left(-g\right)^{n}}{n!} 
    \sum\limits_{\Gamma\in\mathbb{G}_{C,n}} 
    \left\{\prod\limits_{a=1}^{n}
    \int_{Q}dx_{a}\int_{\mathbb{R}}
    \frac{dt_{a}}{2\pi}\,
    \mathcal{F}\left[U_{\varLambda}
    \left(\phi\right)\right]
    \exp\left(-\frac{1}{2}
    G_\varepsilon t_{a}^{2}\right)\right\}\\
    &\times\prod\limits_{a<b}^{n} 
    \left\{\exp\left[-\nu_{ab}
    \left(\Gamma\right)
    G\left(x_{a}-x_{b}\right)
    t_{a}t_{b}\right]-1\right\}.
\end{split} 
\end{equation*} 

Applying for the matrix $G_n$ HSG approximation (\ref{HSG-def}) and introducing auxiliary variables $s_{ab}$, we obtain the following result:
\begin{equation} 
\begin{split} 
    &\mathcal{G}_{I,\varLambda,\varepsilon,n}[0]
    =\frac{\left(-g\right)^{n}}{n!}\,
    \mathcal{O}\left\{\prod\limits_{a=1}^{n}
    \int_{Q}dx_{a}\prod\limits_{a<b}^{n}
    \theta\left(\delta-
    \left|x_{a}-x_{b}\right|\right)\right\}\\
    &\times\left\{\prod\limits_{a=1}^{n}
    \int_{\mathbb{R}}\frac{dt_{a}}{2\pi}\,
    \mathcal{F}\left[U_{\varLambda}
    \left(\phi\right)\right]
    \exp\left(-\frac{1}{2}G_\varepsilon 
    t_{a}^{2}\right)\right\}
    \prod\limits_{a<b}^{n}
    \exp\left[-\gamma G(0)s_{ab}
    \nu_{ab}\left(\Gamma\right)t_{a}t_{b}\right],
    \label{Traveling_back}
\end{split} 
\end{equation} 
where we have used the fact, that:
\begin{equation*}
    \exp\left[-\gamma G(0)\nu_{ab}\left(\Gamma\right)
    \theta\left(x_{a}-x_{b}\right)t_{a}t_{b}\right]-1= 
    \theta\left(x_{a}-x_{b}\right)
    \left\{\exp\left[-\gamma G(0)\nu_{ab}\left(\Gamma\right)
    t_{a}t_{b}\right]-1\right\},
\end{equation*}
so the dependence of the coordinates factorises. Let us note that all the dependence of coordinates remains only in Heaviside functions. 

Now we can calculate the integrals over $x_a$ approximately with a commonly accepted approximation analogical to one in the statistical physics of hard-sphere gas. We can place the first particle in the full volume which gives us $V$ and the following particle should be no further from the first than
$\delta$:
\begin{equation*}
    \left\{\prod\limits_{a=1}^{n}
    \int_{Q} dx_{a}\right\}
    \prod\limits_{a<b}^{n}
    \theta\left(\delta-
    \left|x_{ab}\right|\right)\approx V v^{n-1}.
\end{equation*}
Traveling back to expression (\ref{Traveling_back}), we get:
\begin{equation} 
\begin{split} 
    &\mathcal{G}_{I,\varLambda,\varepsilon,n}[0]=
    \frac{\left(-g\right)^{n}}{n!}\, Vv^{n-1} 
    \mathcal{O}\left\{\prod\limits_{a=1}^{n}
    \int_{\mathbb{R}}\frac{dt_{a}}{2\pi}\,
    \mathcal{F}\left[U_{\varLambda}
    \left(\phi\right)\right]
    \exp\left(-\frac{1}{2}G_\varepsilon  
    t_{a}^{2}\right)\right\}\\ 
    &\times\prod\limits_{a<b}^{n}
    \exp\left(-\gamma G(0)\nu_{ab}
    \left(\Gamma\right)s_{ab} t_{a}t_{b}\right).
    \label{Traveling_back_2}
\end{split} 
\end{equation} 

The expression (\ref{Traveling_back_2}) is much simpler than expression (\ref{Traveling_back}), because the integrals over $x_{a}$ in (\ref{Traveling_back_2}) have already been ``calculated''. Well, it is still a problem to calculate directly the remaining integrals over $t_{a}$, so we use the developed technique of polynomial approximations. Exactly like in the subsections \ref{subsect-exp-rewriting} and \ref{subsect-pol-appr}, we get to the following expression (``taking out of brackets'' all the transformations that lead us to it):
\begin{equation} 
\begin{split}
    &\mathcal{G}_{I,n}[0]=
    \frac{\left(-g\right)^{n}
    \left(2G(0)\right)^{n\alpha/2}}{n!}Vv^{n-1}\, \sum\limits_{q=0}^{\infty}
    \sum\limits_{i=0}^{q}2^{-ni-1}
    \frac{\varGamma\left(\alpha_n \right)}{\varGamma\left(n_i\right)}\, u_i c_q(f)\\
    &\times\mathcal{O}
    \left\{\partial_{1}^{2i}\ldots \partial_{n}^{2i}\bigg|_{\eta_{a}=0}
    e^{-\frac{1}{2}\sum\limits_{a,b=1}^{n}s_{ab}
    \nu_{ab}\left(\Gamma\right)
    \left(\gamma\left(J_n\right)_{ab}+
    \left(1-\gamma\right)\delta_{ab}\right)
    \eta_{a}\eta_{b}}\right\},
    \label{G_I_HSG_for_dis}
\end{split} 
\end{equation} 
where we also used the convergence of the constructed polynomial approximations and wrote the result as a series as well as removed all the regulators.

Finally, substituting the explicit expression for the correlators (\ref{comb-form-n=general}) and using the definition of vacuum energy density $w_{vac}$ (\ref{vac-en-density-def}), we arrive at the result:
\begin{equation}
\begin{split}
    &w_{vac}=-\sum\limits_{n=1}^{\infty} 
    \frac{\left(-g\right)^{n}
    \left(2G(0)\right)^{n\alpha/2}}{n!}\,v^{n-1}
    \sum\limits_{q=0}^{\infty}
    \sum\limits_{i=0}^{q}
    \frac{\left(2i\right)!^n}{2^{ni+1}}
    \frac{\varGamma \left(\alpha_n \right)}{\varGamma\left(n_i\right)}\, u_i c_q(f)\\ 
    &\times\mathcal{O}
    \left\{\sum\limits_{\{l_{ab}\}}
    2^{-\sum\limits_{a=1}^{n}
    \frac{l_{aa}}{2}}
    \frac{\prod\limits_{a<b}^{n}
    \left(s_{ab}\nu_{ab}(\Gamma)\gamma\right)^{l_{ab}}}
    {\prod\limits_{a<b}^{n}\left(l_{ab}!\right)
    \prod\limits_{a=1}^{n}
    \left(\frac{l_{aa}}{2}\right)!}
    \right\}.
    \label{vac-en-HSG}
\end{split}
\end{equation}

One can also rewrite the last combinatorial sum, introducing
$l_{aa}=2q_{a}$:
\begin{equation} 
\begin{split} 
    &\sum\limits_{\{l_{ab}\}}
    2^{-\sum\limits_{a=1}^{n}
    \frac{l_{aa}}{2}}
    \frac{\prod\limits_{a<b}^{n}
    \left(s_{ab}\nu_{ab}(\Gamma)\gamma\right)^{l_{ab}}}
    {\prod\limits_{a<b}^{n}\left(l_{ab}!\right)
    \prod\limits_{a=1}^{n}
    \left(\frac{l_{aa}}{2}\right)!}\\
    =&\gamma^{ni}\left\{\prod\limits_{a=1}^{n}
    \sum\limits_{q_{a}=0}^{i-1}\right\}
    (2\gamma)^{-\sum\limits_{a=1}^{n} q_{a}}
    \frac{1}{\prod\limits_{a=1}^{n}q_{a}!}
    \sum\limits_{\{l_{ab}\}^{'}}
    \frac{1}{\prod\limits_{a<b}^{n}
    \left(l_{ab}!\right)} \prod\limits_{a<b}^{n}
    \left(s_{ab}\nu_{ab}
    \left(\Gamma\right)\right)^{l_{ab}}.
    \label{Traveling_back_3}
\end{split} 
\end{equation} 
\textbf{Convention}: the last summation in the second line of the expression (\ref{Traveling_back_3}) is carried out over all off-diagonal ($a<b$) $l_{ab}>0$ satisfying the conditions $\sum\limits_{b\neq a}^{n}l_{ab}=2i-2q_{a}$ (where still $l_{ab}=l_{ba}$). The expression (\ref{Traveling_back_3}) is useful for decreasing of the computational time for numerical calculations using the formula (\ref{vac-en-HSG}).

Summarizing, we went ahead with some relatively simple approximate expression for vacuum energy density $w_{vac}$ for a generic nonlocal propagator $G$. It can be simply realized using any soft suitable for symbolic computations, such as Wolfram Mathematica. Its main advantage is that unlike in general formulas for GF one doesn't need to compute numerically or analytically coordinate integrals. This fact makes us suppose that HSG approximation is good enough for investigating the qualitative and partly quantitative properties of nonlocal theories.