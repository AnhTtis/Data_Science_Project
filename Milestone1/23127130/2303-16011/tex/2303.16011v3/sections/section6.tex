\section{PT Calculation of System's Physical Characteristics}
\label{phys-calc}

Using the results obtained one can calculate the following characteristics of the quantum scalar field in nonlocal theory:
\begin{enumerate}
\item quantum scalar field vacuum energy density;
\item quantum scalar field Green functions in terms of functional integral over primary field $\phi$, for example, two-particle Green function;
\item quantum scalar field composite operators Green functions, which are also often called ``form factors''.
\end{enumerate}
In this paper, we will focus on the vacuum energy density computation. We will use the first few exactly calculated terms of PT series in coupling constant $g$ as well as approximate expressions for general terms and explore what physical results can be obtained from them. Derivation for the non-zero source general case remains the same but includes more technical details and large calculations. Everywhere in sections \ref{phys-calc} and \ref{phys-research} we will suppose $g(x)=g\chi_{Q}(x)$ in all the obtained formulas. 

We start by considering special cases $n=1,2$ of the general formula (\ref{spiritual formuas for G and Z connection}) (everywhere in this section we refer to graphs only as the Meyer graphs \ref{note-about-graphs}):
\begin{enumerate}
\item for $n=1$ there are no $s_{ab}$ variables, since $a<b$, and the only graph is the single-point, therefore:
\begin{equation}
    \mathcal{G}_{1}[0,G_{1}=G(0)]=
    \mathcal{Z}_{1}[0,G_{1,\Gamma}=G(0)];
    \label{Spiritual_formula_n_1}
\end{equation}
\item for $n=2$ there is one $s_{ab}$ variable, namely $s_{12}$, and the only one connected graph without loops $\Gamma$, for which $\nu_{12}=1$. So the matrix $G_{2,\Gamma}$ has a form:
\begin{equation*}
    G_{2,\Gamma}=
    \left(\begin{array}{cc}
    G(0) & s_{12}G(x_{1}-x_{2})\\
    s_{12}G(x_{1}-x_{2}) & G(0)
    \end{array}\right),
\end{equation*}
and the formula (\ref{spiritual formuas for G and Z connection}) in this case reads as follows:
\begin{equation}
    \mathcal{G}_{2}[0,\left(G_{2}\right)_{ab}]=
    \int_{0}^{1}ds_{12}\,
    \partial_{s_{12}}\mathcal{Z}_{2}[0,\left(G_{2,\Gamma}\right)_{ab}].
    \label{Spiritual_formula_n_2}
\end{equation}
\end{enumerate}
Let us note, that we consider only the zero source case, for which the index $I$ of GFs can be omitted due to the definitions (\ref{Z_I-def}) and (\ref{G_I-def}). 

Now we are going to calculate the first orders of PT from section \ref{sect-PT-derivation} exactly. And it turns out that for $n=1,2$ it is possible to do without using the constructed polynomial approximations or HSG approximation. For $n=1$ the result can be expressed in terms of elementary, and for $n=2$ in terms of Gauss hypergeometric functions.

\subsection{First Orders of Complete Green Functions and Connected Green Functions GFs}
\label{subsect:first-orders-PT}

In this subsection we will provide exact expressions for $\mathcal{Z}_{n}$ for $n=1,2$. Then we get from them the corresponding $\mathcal{G}_{n}$. We start with the expression (\ref{Z_I[0]-final-form}):
\begin{equation} 
    \mathcal{Z}_{n}
    \left[0,\left(G_{n}\right)_{ab}\right]=
    \frac{\left(-g\right)^{n}}
    {n!\left(2\pi\right)^{n/2}}
    \left\{\prod\limits_{a=1}^{n}
    \int_{Q} dx_{a}\, 
    \int_{\mathbb{R}} d\phi_{a}\,
    \left|\phi_{a}\right|^{\alpha}\right\}
    \frac{e^{-\frac{1}{2}\sum\limits_{a=1}^{n}
    \left(G_n\right)_{ab}^{-1}
    \phi_{a}\phi_{b}}}
    {\sqrt{\det\left(G_n\right)}}.
\label{Z_n}
\end{equation} 
The matrix $G_n$ can be degenerate on null sets, though is doesn't affect the convergence of integrals because of the obtained majorant in \ref{majorant-section}.

The first order calculation is straightforward. Applying the scaling transformation of $\phi$ and calculating the integrals over $\phi$ and then over $x$, we arrive at the following result:
\begin{equation}
    \mathcal{G}_{1}\left[0,G(0)\right]=
    \mathcal{Z}_{1}\left[0, G(0)\right]=
    -\frac{gG(0)^{\frac{\alpha}{2}}V2^{\frac{\alpha}{2}}
    \varGamma\left(\frac{\alpha+1}{2}\right)}{\pi^{1/2}}.
\label{G_1[0]-final}
\end{equation} 

The second order calculation demands more mental and technical effort. First, we write the matrix $R_2:=\left(G_{2}\right)^{-1}$, inverse to the matrix $G_2$:
\begin{equation}
    R_{2} (x_{1},x_{2})=
    \frac{1}{G(0)^2 - G(x_{1}-x_{2})^{2}}
    \left(\begin{array}{cc}
    G(0) & -G(x_{1}-x_{2})\\
    -G(x_{1}-x_{2}) & G(0)
    \end{array}\right).
\end{equation}
This matrix is defined almost everywhere in $x_a$ space. So for the complete Green functions GF two-particle contribution we have the multiple integral:
\begin{equation*} 
\begin{split}
    &\mathcal{Z}_{2}[0,\left(G_{2}\right)_{ab}]=
    \frac{g^{2}}{4\pi}
    \int_{Q}\int_{Q}
    \frac{dx_{1}\,dx_{2}}
    {\sqrt{G(0)^{2}-G(x_{1}-x_{2})^{2}}}\\ 
    &\times\int_{\mathbb{R}}
    \int_{\mathbb{R}} d\phi_{1}\,d\phi_{2}\,
    \left|\phi_{1}\right|^{\alpha}
    \left|\phi_{2}\right|^{\alpha}\,
    e^{-\frac{1}{2}R_{11}^{(2)}\phi_{1}^{2}-
    \frac{1}{2}R_{11}^{(2)}\phi_{2}^{2}-
    R_{12}^{(2)}\phi_{1}\phi_{2}}.
\end{split} 
\end{equation*} 
To calculate this integral, one should firstly rescale $\phi_{1}$ and $\phi_{2}$ variables to make the coefficients in $\phi_{1}^2$ and $\phi_{2}^2$ in the exponent equal to one. Then, calculating the obtained integrals alternately, one can express the result in terms of Gauss hypergeometric function ${ }_2F _1$:
\begin{equation} 
\begin{split} 
    &\mathcal{Z}_{2}[0,\left(G_{2}\right)_{ab}]=
    \frac{g^{2}G(0)^{\alpha}2^{\alpha-1}
    \varGamma\left(\frac{\alpha+1}{2}\right)^{2}}{\pi}
    \int_{Q}\int_{Q}dx\,dy\,
    \left(1-\frac{G(x-y)^{2}}
    {G(0)^{2}}\right)^{\alpha+\frac{1}{2}}\\
    & \times{}_{2}F_{1}\left(\frac{\alpha+1}{2},\frac{\alpha+1}{2};\frac{1}{2};\left(\frac{G\left(x-y\right)}{G\left(0\right)}\right)^{2}\right).
    \label{some_intermediate_expression_for_Z}
\end{split} 
\end{equation} 
Here ${ }_2F_1$ is the Gauss hypergeometric function:
\begin{equation}
    { }_2F_1(a,b;c;z)=\sum_{n=0}^{\infty} 
    \frac{(a)_n(b)_n}{(c)_n}\frac{z^n}{n!},
\label{2F1-def}
\end{equation}
for $|z|<1$ and the analytical continuation of this series for $|z|>1$. The parameters $a$, $b$ and $c$ are positive in our paper, $(a)_{n}$, $(b)_{n}$ and $(c)_{n}$ are the rising factorials. The expression (\ref{some_intermediate_expression_for_Z}) can be simplified, but we move on to calculating $\mathcal{G}_{2}$.

For calculating $\mathcal{G}_{2}$, we have to recall (\ref{Spiritual_formula_n_2}). In the case $n=2$ it is useful to rewrite the operator $\mathcal{O}$ as the difference of values for $s_{12} = 1$ and $s_{12}=0$, rather than the integration of the derivative. So the following formula holds:
\begin{equation}
    \mathcal{G}_{2}[0,\left(G_{2}\right)_{ab}]=
    \mathcal{Z}_{2}[0,\left(G_{2,\Gamma}\right)_{ab}] -\mathcal{Z}_{2}[0,G(0) \delta_{ab}].
\end{equation}
In the result, since ${ }_2F_1(a,b;c;0) = 1$ for positive $a$, $b$, $c$, we obtain: 
\begin{equation}
\begin{split} 
    &\mathcal{G}_{2}[0,\left(G_{2}\right)_{ab}]=
    \frac{g^{2}G(0)^{\alpha}2^{\alpha-1}
    \varGamma\left(\frac{\alpha+1}{2}\right)^{2}}{\pi}
    \int_{Q}\int_{Q}dx\,dy\,
    \left\{\left(1-\frac{G(x-y)^{2}}{G(0)^{2}}\right)^{\alpha+\frac{1}{2}}\right.\\ 
    &\times\left.{ }_{2}F_{1}
    \left(\frac{\alpha+1}{2},\frac{\alpha+1}{2};
    \frac{1}{2};\left(\frac{G\left(x-y\right)}{G\left(0\right)}\right)^{2}\right)-1\right\}.
\end{split} 
\label{G_2[0]-two-integrals}
\end{equation}

Let us note that the coordinate integrals in $\mathcal{G}_{1}$ and $\mathcal{G}_{2}$ diverge as $V\rightarrow \infty$ (in the thermodynamic limit). However, the integrals for $\mathcal{Z}_{1}$ and $\mathcal{Z}_{2}$ diverge as $V$ and $V^2$, correspondingly. The thing is, the connected Green functions GF $\mathcal{G}[0]$ is constructed so that all the terms diverging in total (with prefactors also taken into account) faster than $V$ vanish. As a result, we obtain that $\mathcal{G}[0]\sim V$, which makes correct the definition of vacuum energy density. This fact is well known in statistical physics. Thus, in the thermodynamic limit, we obtain the following result:
\begin{equation} 
\begin{split} 
    &\mathcal{G}_{2}[0,\left(G_{2}\right)_{ab}]=
    \frac{g^{2}G(0)^{\alpha}V2^{\alpha-1}
    \varGamma\left(\frac{\alpha+1}{2}\right)^{2}}{\pi}
    \int_{\mathbb{R}^{d}}dx\,
    \left\{\left(1-\frac{G(x)^{2}}
    {G(0)^{2}}\right)^{\alpha+
    \frac{1}{2}}\right.\\ 
    &\times\left.{ }_{2}F_{1}
    \left(\frac{\alpha+1}{2},\frac{\alpha+1}{2};
    \frac{1}{2};\left(\frac{G\left(x\right)}
    {G\left(0\right)}\right)^{2}\right)-1\right\}.
\label{G_2[0]-trans-inv-final}
\end{split} 
\end{equation} 
Informally, this result is expected, since it can be obtained by changing the variables $\Xi=x+y$ and $\xi=x-y$, and integration over $\Xi$ gives the factor $V$. Though, to prove it in a more strict manner we should make some effort. Namely, denoting the integrand in (\ref{G_2[0]-two-integrals}) as $f$, consider the following transformations:
\begin{equation}
    \int_{Q}\int_{Q}dx\,dy\, f(x-y)=
    \int_{Q}dy\,\left\{\int_{\mathbb{R}^{d}}dx\, f(x-y)-
    \int_{\mathbb{R}^{d}\backslash Q}dx\, f(x-y)\right\}.
\end{equation}
In the first integral we shift the variable $x$ in the inner integration, and in the second integral we make the change of variables to $\Xi$ and $\xi$, therefore:
\begin{equation}
    \int_{Q}dy
    \int_{\mathbb{R}^{d} \backslash Q}dx\, f(x-y)= 
    V\int_{\mathbb{R}^{d}\backslash 2Q}d\xi\,
    f(\xi)+\int_{2Q}d\xi\,\xi_1 \ldots \xi_d\, f(\xi),
\end{equation}
where we have introduced the notation $2Q$ for the cube with the same centre, as $Q$, and doubled lengths of the edges. Further, for the applicability of the thermodynamic limit, we require that:
\begin{enumerate}
    \item $\int_{\mathbb{R}^{d}\backslash 2Q}d\xi\,
    f(\xi)\rightarrow 0$, when $V\rightarrow \infty$;
    \item $|\int_{2Q}d\xi\,
    \xi_1 \ldots \xi_d\, f(\xi)|<\infty$, when $V\rightarrow \infty$.
\end{enumerate}
These requirements can be satisfied, for example, if the propagator $G$ satisfies the HSG approximation applicability conditions, which we assume to be satisfied throughout this paper, as already discussed after the equations (\ref{HSG_equations_for_parameters}). As a result, we have proven the expansion for $V\rightarrow \infty$:
\begin{equation}
    \int_{Q}\int_{Q}dx\,dy\, f(x-y)=
    V\left\{\int_{\mathbb{R}^{d}}dx\,f(x)+o(1)\right\}.
\end{equation}
This expression finishes the proof of (\ref{G_2[0]-trans-inv-final}).

Finally, we also introduce the hyperspherical coordinates for rotational invariant theories, which leads us to the following result:
\begin{equation} 
\begin{split} 
    &\mathcal{G}_{2}[0,\left(G_{2}\right)_{ab}] =\frac{g^{2}G(0)^{\alpha}
    V2^{\alpha-1}\varGamma^{2}
    \left(\frac{\alpha+1}{2}\right)}{\pi}
    \frac{d\pi^{d/2}}{\varGamma(d/2+1)}\\
    &\times\int_{0}^{\infty}dr\,r^{d-1}
    \left\{\left(1-\frac{G(r)^{2}}{G(0)^{2}}\right)^{\alpha+\frac{1}{2}} { }_{2}F_{1}\left(\frac{\alpha+1}{2},\frac{\alpha+1}{2};\frac{1}{2};\left(\frac{G\left(r\right)}{G\left(0\right)}\right)^{2}\right)-1\right\}.
\end{split} 
\label{G_2[0]-rot-inv}
\end{equation} 
cause, physically, we usually consider theories with a propagator $G$ that depends only on the distance between particles. This is true, in particular, for Euclidean Klein--Gordon (with sharp cut-off) and virton propagators, which we will describe and use in the section \ref{phys-research}.

\subsection{First Orders of Vacuum Energy Density}

In this short subsection, we list the expressions for the vacuum energy density $w_{vac}$ (\ref{vac-en-density-def}). Collecting the contributions of the first two PT orders, we arrive at the final expression in thermodynamic limit:
\begin{equation}
\begin{split}
    &w_{vac}=\frac{gG\left(0\right)^{\frac{\alpha}{2}}2^{\frac{1+\alpha}{2}}
    \varGamma\left(\frac{\alpha+1}{2}\right)}{\left(2\pi\right)^{1/2}}-
    \frac{g^{2}G(0)^{\alpha}
    2^{\alpha-1}\varGamma
    \left(\frac{\alpha+1}{2}\right)^{2}}{\pi}\\
    &\times\int_{\mathbb{R}^{d}}dx\,
    \left\{\left(1-\frac{G(x)^{2}}
    {G(0)^{2}}\right)^{\alpha+\frac{1}{2}}
    {}_{2}F_{1}\left(\frac{\alpha+1}{2},
    \frac{\alpha+1}{2};\frac{1}{2};
    \left(\frac{G\left(x\right)}{G\left(0\right)}\right)^{2}\right)-1\right\}.
\label{vac-en-first-orders}
\end{split}
\end{equation}
And for rotational invariant theory similarly:
\begin{equation} 
\begin{split}
    &w_{vac}=\frac{gG\left(0\right)^{\frac{\alpha}{2}}2^{\frac{1+\alpha}{2}}
    \varGamma\left(\frac{\alpha+1}{2}\right)}{\left(2\pi\right)^{1/2}}-
    \frac{g^{2}G(0)^{\alpha}
    2^{\alpha-1}\varGamma
    \left(\frac{\alpha+1}{2}\right)^{2}}{\pi}
    \frac{d\pi^{d/2}}{\varGamma(d/2+1)}\\
    &\times\int_{0}^{\infty}dr\, r^{d-1}
    \left\{\left(1-\frac{G(r)^{2}}
    {G(0)^{2}}\right)^{\alpha+\frac{1}{2}} 
    {}_{2}F_{1}\left(\frac{\alpha+1}{2},
    \frac{\alpha+1}{2};\frac{1}{2};
    \left(\frac{G\left(r\right)}{G\left(0\right)}\right)^{2}\right)-1\right\}.
\end{split} 
\end{equation} 
We will consider important particular cases in the following.

\subsection{Verification of PT Formulas for the First Orders for $\alpha=2$}
\label{alpha=2-verivication}

We can calculate independently the GF $\mathcal{Z}$ for $\alpha=2$, which yields simply Gaussian integral. Using the result one can check the formulas for the vacuum energy density, obtained above. In this subsection we restrict ourselves to the case of a zero source and denote all the results of the presented independent calculation specific for $\alpha=2$ with the upper index: something${}^{(\alpha=2)}$.

\subsubsection{Exact Result}

We start from the GF $\mathcal{Z}$ in terms of path integral in physical notations of section \ref{phys-mot}:
\begin{equation*}
    \mathcal{Z}^{(\alpha=2)}\left[0\right]=
    \int_{\varPhi}\mathcal{D}\left[\phi\right]\,
    e^{-\frac{1}{2}\int_{\mathbb{R}^{d}}
    \int_{\mathbb{R}^{d}} dx\,dy\, 
    L(x,y)\phi(x)\phi(y)-
    \int_{\mathbb{R}^{d}} 
    dx\,g\left(x\right)
    \phi\left(x\right)^{2}},
\end{equation*}
with the same requirements to $L$ and $G=L^{-1}$, as in the subsection \ref{Mathematical Buildup}. In terms of the Gaussian measure $\gamma_{G}$, this expression reads as follows:
\begin{equation*}
    \mathcal{Z}^{(\alpha=2)}\left[0\right]=
    \int_{\varPhi}\gamma_{G}(d\phi)\,
    e^{-\int_{\mathbb{R}^{d}} 
    dx\,g\left(x\right)
    \phi\left(x\right)^{2}}.
\end{equation*}
Such an integral is well-known and equals to~\cite{GiuseppePrato} (hereafter $\text{Id}$ is the identity operator):
\begin{equation}
    \mathcal{Z}^{(\alpha=2)}\left[0\right]=
    \frac{1}{\sqrt{\det\left(\text{Id}+ 2Gg\right)}}.
\label{func-det-form}
\end{equation}
We define the product $Gg$ of operators $G$ and $g$ as the result of applying first the new operator $g$, which consists in multiplying by the function $g$ (we denote this function with the same symbol), and then the operator $G$. The operator $Gg$ is trace class as a product of trace class $G$ and bounded $g$. Therefore, it is compact and has a countable set of eigenvalues, and, possibly, zero in its spectrum, and nothing more. We denote these eigenvalues as $\left(Gg\right)_n$ for $n\in\mathbb{N}$. The infinite-dimensional determinant is defined as follows:
\begin{equation}
    \det\left(\text{Id}+2 Gg\right):= 
    \prod\limits_{n=1}^{\infty}
    \left\{1+2\left(Gg\right)_{n}\right\}.
\end{equation}
Let us note that it is exactly the Fredholm determinant of the operator $Gg$. This infinite product converges, since converges the series $\text{tr}\,
\left(Gg\right)=\sum\limits_{n=1}^{\infty}\left(Gg\right)_{n}$. The expression (\ref{func-det-form}) can be obtained with the same methods, as PT in the subsection \ref{subsect:reduction of func int}. One should apply DCT and then calculate the remaining Gaussian integrals, which gives the desired formula. For the GF $\mathcal{G}$ we get therefore: 
\begin{equation}
    \mathcal{G}^{(\alpha=2)}[0]=
    -\frac{1}{2}\text{tr}
    \ln\left(\text{Id}+2Gg\right), 
\end{equation} 
where we have used the formula $\ln\det(1+A)=\text{tr}\ln(1+A)$ for trace class operator $A$. This formula can be proved using the definition of function of an operator through the values of this function on the eigenvalues of $A$, and we won't present its derivation here, referring to textbooks in functional analysis. Expanding the logarithm of an operator, using equivalent for analytical functions and bounded operators definition of function of an operator, we obtain the following series:
\begin{equation}
    \mathcal{G}^{(\alpha=2)}[0]=
    \frac{1}{2}\sum\limits_{n=1}^{\infty}
    \frac{(-2)^{n}}{n}\,\text{tr} \left\{ \left(Gg\right)^{n}\right\}.
\end{equation}

Now we rewrite the powers of operators in terms of integrals, using the notion of operator's $G$ integral kernel $G(x-y)$ and also substitute the coupling constant $g(x) = g \chi_Q (x)$, which cuts off the integration domains:
\begin{equation}
    \mathcal{G}^{(\alpha=2)}[0] =
    \frac{1}{2}\sum\limits_{n=1}^{\infty}
    \frac{(-2g)^{n}}{n}
    \left\{\prod\limits_{a=1}^{n}
    \int_{Q}dx_{a}\right\} 
    G(x_{n}-x_{1})
    \prod\limits_{a=1}^{n-1}
    G(x_{a}-x_{a+1}).
\end{equation} 
Similarly to the subsection \ref{subsect:approx form of second degree}, we change variables and reduce the number of integrals by one, which leads to the appearance of the factor $V$:
\begin{equation}
    \mathcal{G}^{(\alpha=2)}[0]=
    \frac{1}{2}V\sum\limits_{n=1}^{\infty}
    \frac{(-2g)^{n}}{n}\left\{\prod\limits_{a=1}^{n-1}
    \int_{Q}dy_{a}\, G(y_{a})\right\} 
    G\left(\sum\limits_{a=1}^{n-1}y_{a}\right).
\end{equation}
Finally we can write down the expression for the vacuum energy density (\ref{vac-en-density-def}):
\begin{equation}
    w_{vac}^{(\alpha=2)}=
    -\frac{\mathcal{G}[0]}{V}=
    -\frac{1}{2}\sum\limits_{n=1}^{\infty}
    \frac{(-2g)^{n}}{n}
    \left\{\prod\limits_{a=1}^{n-1}
    \int_{Q}dy_{a}\, G(y_{a})\right\} 
    G\left(\sum\limits_{a=1}^{n-1}y_{a}\right).
\end{equation}
From this expression one can see that it coincides with Legendre polynomial approximation for $\alpha=2$, obtained in the subsection \ref{subsect:approx form of second degree}.

\subsubsection{Comparison of PT Formulas for the First Orders for $\alpha=2$ with the Exact Result}

Let us recall the formulas (\ref{G_1[0]-final}) and (\ref{G_2[0]-trans-inv-final}) for $\mathcal{G}_{1}$ and $\mathcal{G}_{2}$ and substitute $\alpha=2$ into them. Taking into account, that:
\begin{equation}
    {}_{2}F_{1}\left(\frac{3}{2},
    \frac{3}{2};\frac{1}{2};z^{2}\right)=
    \frac{2z^{2}+1}{\left(1-z^{2}\right)^{5/2}},
\end{equation}
we receive the following compact expressions:
\begin{equation}
    \mathcal{G}_{1}\left[0,G(0)\right]=
    -gG\left(0\right)V,\quad
    \mathcal{G}_{2}[0,\left(G_{2}\right)_{ab}]=
    {g^{2}V}\int_{\mathbb{R}^{d}}dx\, 
    G\left(x\right)^{2},
\end{equation}
which coincides with directly obtained result for $\alpha=2$. So, our formulas pass this simple test, which witnesses indirectly their rightness.