\section{Introduction}

Why do we need a nonlocal quantum field theory (QFT for short) in physics? Nonlocal QFT is a perturbatively, in each order of the perturbation theory (PT for short) with respect to the small coupling constant $g$, nonrenormalizable theory. The set of such theories is much wider than the set of perturbatively renormalizable ones. In particular, a nonlocal QFT can be formulated in a spacetime of arbitrary dimension $d$. Hence, this research is in some sense ``orthogonal'' to two-dimensional $d=2$ conformal and integrable QFTs. Thus, nonlocal QFT was created precisely in order to consider the spacetime dimension $d$ as an arbitrary parameter~\cite{efimov1970nonlocal,efimov1977nonlocal,efimov1985problems,basuev1973conv,basuev1975convYuk,ivanov1989confinement,ivanov1993quark,efimov2004blokhintsev,moffat1989,moffat1990,ekmw1991,moffat1991,moffat1994,moffat2011gravity,moffat2011higgs,moffat2016,alebastrov1973proof,alebastrov1974proof,efimov1977cmp,efimov1979cmp,Ogarkov2019I,Ogarkov2019II,Modesto2014,Modesto2018,Modesto2015}.

Further, nonlocal QFT makes it possible to describe the high-energy physics of scalar particles (scalar nonlocal QFT) and in quantum electrodynamics (QED for short), as well as low-energy physics in quantum chromodynamics (QCD for short). For example, to describe the low-energy physics of light hadrons in QCD and quark confinement, the so-called Virton-Quark Model and the Quark Confinement Model of Hadrons are generally accepted~\cite{efimov1985problems,ivanov1989confinement,ivanov1993quark,efimov2004blokhintsev}. Nowadays, research in the field of nonlocal quantum theory of gravity and nonlocal QFT in curved spacetime is also gaining popularity~\cite{moffat2011gravity,moffat2016,Modesto2014,Modesto2018,Modesto2015}.

Another reason to research nonlocal QFT in Euclidean spacetime (the metric signature is all pluses) comes from the so-called grand canonical partition function/nonlocal, \textbf{nonpolynomial} Euclidean QFT scattering matrix ($\mathcal{S}$-matrix for short) duality~\cite{efimov1970nonlocal,efimov1977nonlocal,efimov1985problems,basuev1973conv,basuev1975convYuk,efimov1977cmp,efimov1979cmp,Ogarkov2019I,Ogarkov2019II,brydges1999review,rebenko1988review,brydges1980cmp,polyakov1987gauge,polyakov1977quark,samuel1978grand,o1999duality}. This duality is well known, for example, in statistical physics and plasma physics, and its essence is as follows: $\mathcal{S}$-matrix of nonlocal nonpolynomial Euclidean QFT, originally formulated in terms of the functional (path) integral, can be rewritten in terms of the grand canonical partition function (GCPF for short) of some ``gas with interaction''. This duality is used in both directions. On the one hand, this makes it possible to apply the methods of statistical physics in QFT, in particular, to use the cluster expansion when calculating the $\mathcal{S}$-matrix. On the other hand, this makes it possible to apply QFT methods in statistical physics, in particular, to use the saddle-point method to calculate the functional (path) integral equal to the GCPF. Here let us note that this duality is used in the research of sin-Gordon and sinh-Gordon models, more precisely, their nonlocal versions.

Finally, within the framework of nonlocal QFT, it is possible to research the strong coupling limit in interaction constant $g$~\cite{efimov1985problems,efimov1977cmp,efimov1979cmp,Ogarkov2019I,Ogarkov2019II}. The method used to research the strong coupling limit in nonlocal QFT originated in the theory of polarons (in polaronics). This method is based on finding the minorant (lower estimate) and majorant (upper estimate) for the quantity of interest to us, as functions of the coupling constant $g$. The Jensen and H\"{o}lder inequalities are often used to find these estimates. If both estimates tend to each other in the limit $g\rightarrow+\infty$, then each of them coincides with the quantity of interest to us (in this limit). In fact, this is the squeeze theorem. For example, the vacuum energy $\mathcal{E}(g)$ was obtained in the strong coupling limit in~\cite{efimov1985problems,efimov1977cmp,efimov1979cmp}.

Why do we need a nonlocal quantum field theory in mathematics? The rigorous definition of the functional integral in a nonlocal QFT, the proof of the definition correctness (the existence and uniqueness of such an object), as well as the calculation of the latter, is a complicated but interesting mathematical problem~\cite{Mazzucchi2008,Mazzucchi2009,DeWitt-Morette,Montvay,Kleinert,Johnson,Steiner,Mosel,Simon,Popov1976,Shavgulidze2015,vasil2004field,zinn1989field,Bogachev,GiuseppePrato}. Let us note that even the calculation of such an object needs some definition. In what terms is the functional integral considered calculated? As quoted above, there is extensive mathematical literature devoted to the theory of integration in separable Hilbert and Banach spaces (HS and BS for short) with respect to Gaussian measures. At the same time, the ``list of integrals'' given in such literature is hardly sufficient for practical applications: results are usually given in cases where the integrand is a polynomial or again a Gaussian exponent, which corresponds to a Gaussian (free, without interaction) QFT. Our paper aims to narrow this gap a bit by adding another result to the ``list of integrals''.

Another complicated but interesting mathematical problem is the rigorous definition of the Dyson--Schwinger, Schwinger--Tomonaga and functional (nonperturbative, exact) renormalization group flow (FRG flow for short) equations in nonlocal QFT~\cite{Popov1976,vasil2004field,zinn1989field,kopbarsch,wipf2012statistical,rosten2012fundamentals,igarashi2009realization,Ogarkov2020III,Ogarkov2021IV}. These equations are formulated in terms of variational derivatives, therefore, they are differential. Again, as quoted above, there is extensive mathematical literature devoted to differential equations in infinite-dimensional spaces (separable HS and BS). However, as in the case of integrals, the ``list of solutions'' given in such literature is hardly sufficient for practical applications. Obtaining solutions to such equations in terms of functional Taylor series is associated with solving infinite hierarchies (systems) of integro-differential equations for various families of Green functions. Such hierarchies are usually solved approximately. Let us note that FRG flow equations formulated as nonlocal QFT appear in many physical sciences: QFT itself, condensed matter physics, critical behaviour theory, stochastic theory of turbulence, etc.

Finally, in the framework of nonlocal QFT research in mathematics, let us note one possible generalization to the case of locally convex topological vector space (LCTVS for short)~\cite{Smolyanov}. Such a seemingly mathematical generalization can lead to far-reaching consequences in physics. The topology of LCTVS is given by a family (set of arbitrary cardinality) of seminorms, generalizing separable HS and BS. In fact, instead of one scalar product or norm, a family of such arises. Such a family, hypothetically, can be associated with an infinite set of different ultraviolet (UV for short) scales, which makes it possible to deselect one of them. It turns out, a theory in which there is no distinguished UV scale. And, since the selected UV scale is the main problem of nonlocal QFT, such a hypothetical theory may already be a fundamental theory of nature. At the time of writing this paper, such a study seems to be the subject of the future.

There are still a few questions left to answer in the Introduction section. Why a scalar field? On the one hand, from the theory of groups and Lie algebras' point of view, the quantum theory of a scalar field has the smallest number of symmetries. For this reason, in such a theory there are no Ward--Takahashi identities that can simplify the $\mathcal{S}$-matrix calculation. In this sense, the scalar theory is the most complicated. On the other hand, to construct such a theory, it is the measure theory in infinite-dimensional spaces that is needed. In this sense, the scalar theory is the simplest, no additional symmetry relations need to be taken into account.

Why nonpolynomial self-interaction? In nonpolynomial theories that satisfy some general principles, the $\mathcal{S}$-matrix in terms of the GCPF is a convergent series in the interaction constant $g$. Let us note that for polynomial theories, for example, for the theories $\phi^{4}$ and $\phi^{6}$, such a series, being a PT series in the interaction constant $g$, is asymptotic. An asymptotic series arises as a consequence of the incorrectness of the mathematical transformations performed in the process of its derivation. In particular, the corollary of the monotone convergence theorem (MCT for short) for alternating series is violated. And although the latter is only a sufficient condition, but not necessary and sufficient, nevertheless, the emergence of an asymptotic series means that the permutation of summation and functional integration doesn't really take place. In nonpolynomial theories of a certain class, this problem doesn't arise. 

Why fractional power self-interaction $|\phi|^{\alpha}$, where $\alpha\in(1,2)$? The research of such a theory seems natural for a number of reasons. The first reason is that such $\alpha$ is greater than the power in the source term and less than the power in the Gaussian theory. Therefore, in such a theory, it is possible to apply the corollary of the MCT for alternating series. The second, less obvious, reason is that such a theory is implicitly related to all the polynomial theories at once. This will be demonstrated in our paper. And, finally, it can be expected that the method of calculating the $\mathcal{S}$-matrix in such a theory can be applied in the future to a wider class of theories. In this sense, in addition to the research of the theory itself, the result of the paper is a new method.

Finally, why $\mathcal{S}$-matrix and vacuum energy $\mathcal{E}$? The problem of QFT is considered solved if an explicit expression for the $\mathcal{S}$-matrix is obtained. Here again, it is important to note that the very notion of an ``explicit'' expression needs to be defined. It is generally assumed that the corresponding functional integral must be calculated. The $\mathcal{S}$-matrix contains complete information about scattering: the variational derivatives of the $\mathcal{S}$-matrix determine the Green functions due to the interaction of particles. The Green functions make it possible to reconstruct the corresponding scattering cross-sections, which are experimentally observable quantities. For this reason, the research of the $\mathcal{S}$-matrix is the most important. Further, the simplest quantity contained in the $\mathcal{S}$-matrix is the vacuum energy $\mathcal{E}$ of the theory. It is this value that is most easily calculated. Let us note that the calculation of the functional integral leads to results that depend on the UV scale of the theory. In some cases, this dependency may be undesirably ``strong''. In this case, one has to invent some kind of subtraction scheme or change the original formulation of the theory. In this paper, we aim to calculate the functional integral in nonlocal QFT for given values of the theory parameters and to research how the results obtained depend on the corresponding parameters.

This paper has the following structure. In section $2$ we present a summary of the problem and obtained results. This section can be read independently of the rest of the paper. All the necessary definitions and notations are given within the section. A section $3$ is devoted to physical background and motivation. A section $4$ is devoted to:
\begin{enumerate}
    \item explanation of the Gaussian measure concept;
    \item derivation of PT series and proof of its convergence;
    \item exponentiation of the obtained series.
\end{enumerate}
A section $5$ is devoted to the calculation of PT series terms using polynomial approximation. Sections $6$ and $7$ are devoted to the calculation and research of vacuum energy density, namely:
\begin{enumerate}
    \item exact computation of first and second order in coupling constant $g$;
    \item computation with hard-sphere gas (HSG for short) approximation;
    \item expression for the second-degree polynomial approximation.
\end{enumerate}
In the Discussion section, we give a final discussion of all the results obtained in the paper. In the Conclusions section, we highlight further possible areas of research.