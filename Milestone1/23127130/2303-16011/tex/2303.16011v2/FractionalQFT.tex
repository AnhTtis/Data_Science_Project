\documentclass{article}

\usepackage{arxiv}
\usepackage{mathrsfs}
\usepackage{cmap}   
\usepackage[T2C]{fontenc}
\usepackage[utf8]{inputenc}
\usepackage[utf8]{inputenc} % allow utf-8 input
\usepackage[T1]{fontenc}    % use 8-bit T1 fonts
\usepackage{hyperref}       % hyperlinks
\usepackage{url}            % simple URL typesetting
\usepackage{booktabs}       % professional-quality tables
\usepackage{amsfonts}       % blackboard math symbols
\usepackage{nicefrac}       % compact symbols for 1/2, etc.
\usepackage{microtype}      % microtypography
\usepackage{lipsum}		% Can be removed after putting your text content
\usepackage{graphicx}
\usepackage{natbib}
\usepackage{doi}

%% LyX 2.3.6.1 created this file.  For more info, see http://www.lyx.org/.
%% Do not edit unless you really know what you are doing.
\usepackage{amsmath}
\usepackage{amsthm}
\usepackage{amssymb}
\usepackage{stmaryrd}


%%%%%%%%%%%%%%%%%%%%%%%%%%%%%% LyX specific LaTeX commands.
\DeclareRobustCommand{\cyrtext}{%
  \fontencoding{T2A}\selectfont\def\encodingdefault{T2A}}
\DeclareRobustCommand{\textcyr}[1]{\leavevmode{\cyrtext #1}}

\theoremstyle{plain}
\newtheorem*{prop*}{\protect\propositionname}
\newtheorem{theorem}{theorem}

\usepackage{babel}
\providecommand{\propositionname}{Proposition}


\title{Nonlocal Scalar QFT with Fractional Power Self-Interaction}

	%% examples of more authors

   \author{ Daniel V. Skliannyi\\
	Weizmann Institute of Science\\
	\texttt{Daniil.skliannyi@weizmann.ac.il}\\
        \And
        Stanislav L. Ogarkov \\
	Moscow Institute of Physics\\
	  and Technology (MIPT),\\
	\texttt{Ogarkov.sl@phystech.edu} \\
 	\And 
        Nikita A. Ignatyuk \\
        Moscow Institute of Physics\\
        and Technology (MIPT),\\
        Skolkovo University of \\
        Science and Technolgy\\
    	\texttt{Ignatyuk.na@phystech.edu} \\
	%% \AND
	%% Coauthor \\
	%% Affiliation \\
	%% Address \\
	%% \texttt{email} \\
	%% \And
	%% Coauthor \\
	%% Affiliation \\
	%% Address \\
	%% \texttt{email} \\
	%% \And
	%% Coauthor \\
	%% Affiliation \\
	%% Address \\
	%% \texttt{email} \\
}

% Uncomment to remove the date
%\date{}

% Uncomment to override  the `A preprint' in the header
%\renewcommand{\headeright}{Technical Report}
%\renewcommand{\undertitle}{Technical Report}
\renewcommand{\shorttitle}{Scalar QFT with Fractional Interaction}

%%% Add PDF metadata to help others organize their library
%%% Once the PDF is generated, you can check the metadata with
%%% $ pdfinfo template.pdf
\hypersetup{
pdftitle={Scalar QFT with Fractional Power Interaction},
pdfsubject={mphys},
pdfauthor={Daniel V. Skliannyi, Stanislav L. Ogarkov,  Nikita A. Ignatyuk},
pdfkeywords={Quantum field theory, path integration, fractional interaction, converge perturbation series, vacuum energy, statistical physics},}

\begin{document}
\maketitle

\title{} 
\maketitle
\begin{abstract}
The main aim of this paper is to derive a new perturbation theory that has converging series, which is derived from the nonlocal scalar quantum field theory (QFT) with fractional power potential. We construct the perturbation theory for the generating functional (GF) of complete Green functions (including disconnected functions parts) $\mathcal{Z}\left[j\right]$ as well as for the GF of connected Green functions $\mathcal{G}\left[j\right]=\ln \mathcal{Z}\left[j\right]$ in powers of coupling constant $g$. Its has IR-finite terms, which are also UV-finite in local limit after simple rescaling. We have proved that the obtained series, which has the form of a grand canonical partition function, is dominated by a convergent series, in other words, has majorant in the presence and absence of field source, which allows to expand beyond weak coupling limit. Vacuum energy density in second order in $g$ was calculated and researched for different types of Gaussian part $S_{0}[\phi]$ of the action $S[\phi]$. Using the polynomial expansion, the general calculable series for $\mathcal{G}\left[j\right]$ was derived. We provide, compare and research simplifications in cases of second degree polynomial approximation and hard-sphere gas approximations. The developed formalism allows to research physical properties of the considering system across the entire range of coupling constant, and we use it for the study of vacuum energy of the model.
\end{abstract}
\newpage{}

\tableofcontents

\newpage{}

\part{Summary of Problem and Obtained Results}

In this paper the quantum theory of scalar field $\phi$ in $\mathbb{R}^{d}$ with action $S\left[\phi\right]$:
\begin{equation}
S[\phi]=S_{0}[\phi]+S_{I}[\phi]=\frac{1}{2}
\int dx\,dy\,L(x,y)\phi(x)\phi(y)+\int dx\,g(x)\left|\phi(x)\right|^{\alpha},\qquad 1\leq\alpha\leq2,
\nonumber
\end{equation}
is considered. Here $L=G^{-1}$ is the 
inverse propagator. GF $\mathcal{Z}\left[j\right]$ in this theory:
\begin{equation}
\mathcal{Z}\left[j\right]=\int D\left[\phi\right]\,e^{-\frac{1}{2}\int dx\,dy\,L(x,y)\phi(x)\phi(y)-\int dx\,g(x)\left|\phi\left(x\right)\right|^{\alpha}-\int dx\, j\left(x\right)\phi\left(x\right)},
\nonumber
\end{equation}
is derived in terms of convergent perturbation series in powers of coupling constant $g$, which is a function of $x$ in general case:
\begin{align*}
\mathcal{S}\left[\varphi\right]=\sum_{n=0}^{\infty}
\frac{\left(-1\right)^{n}}{n!}
\left\{\prod_{a=1}^{n}\int dx_{a}\,g(x_{a})
\int d\phi_{a}\left|\phi_{a}\right|^{\alpha}\right\}
\frac{1}{\sqrt{\left(2\pi\right)^{n}
\det\left(G_{n}\right)_{ab}}}\\
\times\exp\left\{-\frac{1}{2}
\sum\limits_{a,b=1}^{n}
\left(G_{n}\right)_{ab}^{-1}
\left[\phi_{a}-\varphi\left(x_{a}\right)\right]
\left[\phi_{b}-\varphi\left(x_{b}\right)\right]\right\},
\nonumber
\end{align*}
where:
\begin{enumerate}
\item $\mathcal{S}\left[\varphi\right]=
\mathcal{Z}\left[j\right]/\mathcal{Z}_{0}\left[j\right]$
is the interaction $\mathcal{S}$-matrix and $\mathcal{Z}_{0}\left[j\right]$ is the Gaussian GF; 
\item $G_{n}$ is $n\times n$ matrix with elements $\left(G_{n}\right)_{ab}=G(x_{a}-x_{b})$; 
\item $\varphi(x)=\hat{G}j\left(x\right)=\int dy\,G\left(x-y\right)j\left(y\right)$ is the classical field corresponding to the source $j$.
\end{enumerate}
Using Weierstrass M-test, we proved the convergence of such a series. We have obtained the majorizing series explicitly.

Further in the paper we have proved the existence of the corresponding GF of connected Green functions $\mathcal{G}_{I}\left[\varphi\right]=
\ln{\left(\mathcal{Z}\left[j\right]/
\mathcal{Z}_{0}\left[j\right]\right)}$. Let us note that sometimes another definition of connected Green functions GF is introduced in the literature $\mathcal{G}_{c}\left[\varphi\right]=
\ln{\left(\mathcal{Z}\left[j\right]/
\mathcal{Z}_{0}\left[j=0\right]\right)}$ (\cite{Kopbarsch}). The perturbation theory series for $\mathcal{G}_{I}\left[\varphi\right]=
\sum\mathcal{G}_{I,n}\left[\varphi\right]$ has the following form:
\begin{align*}
\mathcal{G}_{I,n}\left[\varphi\right] & =
\frac{\left(-1\right)^{n}}{n!}
\sum_{\Gamma\in\mathbb{G}_{C,n}}
\left\{\prod_{a<b}^{n}\int_{0}^{1}
ds_{ab}\,\partial_{s_{ab}}^{\nu_{ab}
\left(\Gamma\right)}\right\}
\left\{\prod_{a=1}^{n}
\int dx_{a}\,g\left(x_{a}\right)
\int d\phi_{a}
\left|\phi_{a}\right|^{\alpha}\right\}\\
 & \times\frac{1}{\sqrt{\left(2\pi\right)^{n}
\det\left(G_{n}\right)_{ab}}}
\exp\left\{-\frac{1}{2}
\sum\limits_{a,b=1}^{n}
\left(G_{n,\Gamma}\right)_{ab}^{-1}
\left[\phi_{a}-\varphi\left(x_{a}\right)\right]
\left[\phi_{b}-\varphi\left(x_{b}\right)\right]\right\},
\end{align*}
where:
\begin{enumerate}
\item $\mathbb{G}_{C,n}$ is the set of all connected undirected graphs with no loops and multiple edges on $n$ vertices; 
\item $\nu_{ab}\left(\Gamma\right)$ is an adjacency matrix of a graph $\Gamma$;
\item $\left(G_{n,\Gamma}\right)_{ab}=
s_{ab}\nu_{ab}\left(G_{n}\right)_{ab}$ and $s_{ab}$ are auxiliary variables, using for extracting connected contributions.
\end{enumerate}

Further, we have calculated the first two orders in $g(x)=g\chi_{Q}\left(x\right)$, where $\chi_{Q}$ is the indicator function of $d$-dimensional cube centered at the origin with $\text{Vol}\,{Q}=V$, for the vacuum energy density $\mathcal{W}_{vac}$, namely ($\varGamma$ is the Gamma function): 
\begin{align*}
w_{vac}=\frac{2^{\frac{1+\alpha}{2}}g}{\left(2\pi\right)^{1/2}}\varGamma\left(\frac{1}{2}(\alpha+1)\right)\left(G\left(0\right)\right)^{\frac{\alpha}{2}}-\frac{g^{2}G(0)^{\alpha}V2^{\alpha-1}\varGamma^{2}\left(\frac{1}{2}(\alpha+1)\right)}{\pi}\frac{d\pi^{d/2}}{\varGamma(d/2+1)}\\
\times\int_{0}^{\infty}dr\,r^{d-1}\left[\left(1-\frac{G(r)^{2}}{G(0)^{2}}\right)^{\alpha+\frac{1}{2}}{}_{2}F_{1}\left(\frac{1}{2}(\alpha+1),\frac{1}{2}(\alpha+1);\frac{1}{2};\left(\frac{G\left(r\right)}{G\left(0\right)}\right)^{2}\right)-1\right]+O(g^{3})
\end{align*}
and applied these formulas for the cases of Virton propagator (\cite{Efimov_problems}) and Euclidean Klein-Gordon propagator. For the Klein-Gordon propagator with sharp UV-cutoff $\varLambda$ in momentum space when the modes are frozen at $|k|>\varLambda$ and the propagator in momentum representation $G(|k|>\varLambda)=0$, we have obtained for the vacuum energy density $w_{vac}$ the following expression:
\begin{align*}
&w_{vac}\left(g,\varLambda\right)
m^{-d}=a_{1}\frac{gG(0)^{\alpha/2}}{m^{d}}+a_{2}\left(\frac{gG(0)^{\alpha/2}}{m^{d}}\right)^{2}
\left(\frac{m^{d-2}}{G(0)}\right)^{2}+...\\
&a_{1}=2^{\frac{\alpha+1}{2}}\Gamma\left(\frac{1}{2}(\alpha+1)\right),\qquad a_{2}=\frac{2^{\alpha-1}}{\pi}\Gamma^{2}\left(\frac{1}{2}(\alpha+1)\right)
\begin{cases}
\frac{1}{4}, & d=2,\\
\frac{1}{8}, & d=3.
\end{cases}
\end{align*}
Besides, we have provide the following polynomial formula for $\mathcal{G}_{I,n}\left[\varphi\right]$, derived by Legendre polynomial expansion:
\begin{align*}
\mathcal{G}_{I,n}\left[\varphi\right] & =\frac{\left(-1\right)^{n}g^{n}}{n!}\sum_{\Gamma\in\mathbb{G}_{C,n}}\left\{ \prod_{a<b}^{n}\int_{0}^{1}ds_{ab}\partial_{s_{ab}}^{\nu_{ab}\left(\Gamma\right)}\right\} \left\{ \prod_{a=1}^{n}\int dx_{a}\right\} \exp\left(-\frac{1}{2}\sum_{a,b}\tilde{G}^{-1}_{ab}\bar{\phi}_{a}\bar{\phi}_{b}\right)\\
 & \times\sum_{k=0}^{\infty}\frac{2^{n\alpha/2}G(0)^{n\alpha/2}}{4^{k}k!}\frac{\left(\frac{n(\alpha+1)}{2}\right)_{2k}}{\left(1/2\right)_{2k}}\sum_{\nu_{1}+...+\nu_{n}=2k}\binom{n}{\nu_{1}.....\nu_{n}}\chi_{1}^{\nu_{1}}...\chi_{n}^{\nu_{n}}\sum_{q=0}^{\infty}\sum_{i=0}^{q}\left(2q+\frac{1}{2}\right)2^{4q-ni+1}\\
 & \times\frac{\Gamma\left(\frac{n(\alpha+1)}{2}\right)}{\Gamma\left(\frac{n}{2}+n(i+k)\right)}\left(\sum_{p=0}^{q}\binom{2q}{2p}\binom{q+p-\frac{1}{2}}{2q}\frac{1}{2p+\alpha+1}\right)\binom{2q}{2i}\binom{q+i-\frac{1}{2}}{2q}\\
 & \times\sum_{\begin{array}{c}
\sum_{b}l_{ab}=2k+c_{a},\\
l_{aa}\vdots2
\end{array}}\left(\prod_{a<b}\left[\frac{\tilde{G}_{ab}}{G(0)}\right]^{l_{ab}}\right)\frac{\prod_{a=1}^{n}\left(2k+c_{a}\right)!}{\left(\prod_{a<b}\left(l_{ab}!\right)\right)2^{\sum_{a}\frac{l_{aa}}{2}}\left(\prod_{a=1}^{n}\left(\frac{l_{aa}}{2}\right)!\right)},
\end{align*}
where $\chi$ also depend on $s_{ab}$ and $\nu_{ab}$ via $\tilde{G}_{ab}$. This yields the simple approximate formula for $w_{vac}$ if one choose the degree of approximation polynomial is equal to $2$:
\begin{equation}
w_{vac}\approx\frac{15\alpha}{\alpha^{2}+4\alpha+3}\sum_{n=1}^{\infty}\frac{(-1)^{n-1}}{n}2^{n(\alpha/2-1)-2}g^{n}\frac{\Gamma\left(\frac{n(\alpha+1)}{2}\right)}{\Gamma\left(\frac{n}{2}+n\right)}\left\{ \prod_{a=1}^{n-1}\int dt_{a}G(t_{a})\right\} G\left(\sum_{k=1}^{n-1}t_{k}\right).
\nonumber
\end{equation}

Beyond that, we derived another approximation formula for $w_{vac}$, based on hard-sphere approximation of $G(x)$. Namely, supposing $G(x)\approx\gamma G(0)\theta(\delta-|x|)$, we have:
\begin{align*}
w_{vac}	=\sum_{n=1}^{\infty}\frac{\left(-g\right)^{n}}{n!}G(0)^{n\alpha/2}2^{n\alpha/2}v^{n-1}\sum_{\Gamma\in\mathbb{G}_{C,n}}\left\{ \prod_{a<b}^{n}\int_{0}^{1}ds_{ab}\partial_{s_{ab}}^{\nu_{ab}}\right\} \sum_{q=0}^{\infty}\sum_{i=0}^{q}\left(2q+\frac{1}{2}\right)2^{4q-ni+1}
\\
\times\frac{\Gamma\left(\frac{n(\alpha+1)}{2}\right)}{\Gamma\left(\frac{n}{2}+ni\right)}\binom{2q}{2i}\binom{q+i-\frac{1}{2}}{2q}
	\left(\sum_{p=0}^{q}\binom{2q}{2p}\binom{q+p-\frac{1}{2}}{2q}\frac{1}{2p+\alpha+1}\right)
 \\
 \times\sum_{\begin{array}{c}
\sum_{b}l_{ab}=2k,\\
l_{aa}\vdots2,\ l_{ab}>0
\end{array}}\left(\prod_{a<b}\gamma^{l_{ab}}\right)\frac{\prod_{a=1}^{n}\left(2k+c_{a}\right)!}{\left(\prod_{a<b}\left(l_{ab}!\right)\right)2^{\sum_{a}\frac{l_{aa}}{2}}\left(\prod_{a=1}^{n}\left(\frac{l_{aa}}{2}\right)!\right)},
\end{align*}
where $v=\frac{\pi^{d/2}}{\Gamma(d/2+1)}\delta^{d}$ and $\delta$ and $\gamma$ are determined by the equations:
\begin{equation}
G(0)\gamma v=\int G(x)dx,\qquad \gamma^2 G(0)^2 v =\int G^2(x)dx.
\nonumber
\end{equation}

The other reasonable equations are also possible, and these ones are chosen to make formulas of hard-sphere gas approximation simpler for substituting the particular cases of propagators. All the obtained formulas were checked for $\alpha=2$ and applied to
Virton as a typical nonlocal propagator and Euclidean Klein-Gordon propagator in $d=2,3$ as a typical local. Finally, we plot approximate plots of vacuum energy for all values of $g$ for both described approximations and provide corresponding approximate formulas. These formulas are valid for strong coupling as well as for weak, which is quite a rare case for quantum field models.

Performed considerations and research will help in understanding of interacting quantum fields general properties. Besides, they can potentially be applied for the strong coupling case of $\phi^{4}$-theory, which will be the subject of further research and publications of the authors.

\part{Physical Background and Motivation}

\paragraph{Definitions and Notations}

In this paper we consider Euclidean theory of quantum scalar field with action:
\begin{equation}
S[\phi(x)]=\frac{1}{2}\int d^{d}xd^{d}y\ L(x,y)\phi(x)\phi(y)+g\int dx\left|\phi(x)\right|^{\alpha},\qquad1\leq\alpha\leq2,
\end{equation}
and corresponding generating functional, which we also will call grand canonical partition function:
\begin{equation}
Z\left[j\left(x\right)\right]=\int D\phi e^{-\frac{1}{2}\int d^{d}xd^{d}y\ L(x,y)\phi(x)\phi(y)-g\int\left|\phi\left(x\right)\right|^{\alpha}dx-\int j\left(x\right)\phi\left(x\right)dx}.
\end{equation}
Here $\alpha$ is assumed to be real, and $1\leq\alpha\leq2$. In the first term on the right hand side of the action, the function (distribution)
$L(x,y)$ is the kernel of the operator $\hat{L}$, which defines the quadratic part of the action of a nonlocal QFT (Gaussian theory). Nonlocal QFT can be considered as a regularization of the Euclidean Klein-Gordon theory, but it is also of independent interest. Let us note that the kernel $G(x,y)$ of the inverse operator $\hat{G}=\hat{L}^{-1}$ is the ultraviolet finite propagator of a nonlocal QFT in any space dimension. $G(x,y)$ is also called Green function of Gaussian theory. We will assume the translational invariance of the Gaussian theory, which implies $G(x,y)=G(x-y)$, but at the same time we will consider the theory in a $d$-dimensional box of volume $V$, which implies periodical
boundary conditions. We will also suppose that $G(x)\geq0$ and $G(x)\leq G(0)$ for all $x$. The functional integration in the formula above is understood
in the sense of Gaussian measure with covariation operator $G$ (see Appendix A for details). In this paper we assume that $\hbar=c=1$. For the sake of simplicity we mainly will consider $g(x)=g\chi_{Q}\left(x\right)$, where $\chi_{Q}$ is the indicator function of $d$-dimensional cube centered at the origin with $\text{Vol}\,{Q}=V$, and will write simply $g$ and understand coordinate integration as the integration over a volume $V$, which is assumed to be large enough (thermodynamical limit).

Generating functional $Z[j(x)]$ is a regular functional, so it could be expanded in a functional Taylor series at $j=0$:
\begin{equation}
Z\left[j\left(x\right)\right]=\sum_{n=0}^{\infty}\frac{1}{n!}\mathcal{D}_{1,...,n}^{(n)}j_{1}...j_{n},\qquad\mathcal{D}_{1,...,n}^{(n)}=\frac{\delta^{n}Z[j(x)]}{\delta j_{1}...\delta j_{n}}\bigg|_{j(x)=0},
\end{equation}
where the repeated index $i$ means integration over some space variable $x_{i}$. Coefficients $\mathcal{D}_{1,...,n}^{(n)}=\mathcal{D}^{(n)}(x_{1},...,x_{n})$
are called $n$-point functions of functional $Z$ and they represent $n$-point correlators of the theory described by $\mathcal{Z}\left[j\right]$:
\begin{equation}
\left\langle \phi(x_{1})....\phi(x_{n})\right\rangle =\int D\phi\ \phi(x_{1})....\phi(x_{n})e^{-\frac{1}{2}\int d^{d}xd^{d}y\ L(x,y)\phi(x)\phi(y)-g\int\left|\phi\left(x\right)\right|^{\alpha}dx-\int j\left(x\right)\phi\left(x\right)dx}=\frac{1}{n!}\mathcal{D}^{(n)}(x_{1},...,x_{n})
\end{equation}
Correlators are the object of interest since they are the quantities which determines section of cross scattering of particle states and therefore can be directly measured in particle physics as well as
in statistical physics. In Gaussian theory ($g=0$) one have a usual Gaussian integral, and we will denote the corresponding generating functional as:
\begin{equation}
\mathcal{Z}_{0}\left[j\left(x\right)\right]=e^{\frac{1}{2}\int dxdyj\left(x\right)G(x,y)j\left(y\right)}
\end{equation}

It is also known that $\mathcal{Z}\left[j\right]$ can be represented as an exponent of other regular functional $\mathcal{G}[j]$, which is called generating functional
of connected diagrams:
\begin{equation}
\mathcal{Z}\left[j\right]=e^{\mathcal{G}[j]},
\end{equation}
and from statistical physics point of view $Z$ is a grand
canonical partition function as well as $\mathcal{G}$ is a grand thermodynamic potential, up to a temperature factor which we set to $1$, since it can be restored from dimensional considerations.

The functional $\mathcal{G}[j]$ can also be expanded in a functional Taylor series, and its coefficients are called connected diagrams or connected $n$-particle Green functions:
\begin{equation}
\mathcal{G}\left[j\left(x\right)\right]=\ln Z[0]+\sum_{n=1}^{\infty}\frac{1}{n!}\mathcal{G}_{1,...,n}^{(n)}j_{1}...j_{n},\qquad\mathcal{G}_{1,...,n}^{(n)}=\frac{\delta^{n}\mathcal{G}[j(x)]}{\delta j_{1}...\delta j_{n}}\bigg|_{j(x)=0}
\end{equation}
the terminology goes from Feynman diagrams and will be clarified in the following while consideration of Meyer cluster expansion. Of course,
all $\mathcal{D}^{(n)}(x_{1},...,x_{n})$ can be expressed via $\mathcal{G}^{(n)}(x_{1},...,x_{n})$
by expanding the exponent. Throughout this paper we denote connected $n$-particle Green functions with the mathcal font $\mathcal{G}_{1,...,n}^{(n)}$ rather than Green function of Gaussian theory $G(x-y)$. 

In statistical physics usually there are exists a finite limit of $\frac{\mathcal{G}}{V},$ which allows to write in thermodynamic limit $V\longrightarrow\infty$: 
\begin{equation}
\mathcal{G}[j]=Vf[j],
\end{equation}
where $f$ is a volume density of $\mathcal{G}$, which is simply the pressure for the case of homogeneous systems. It can be shown that $\mathcal{G}[0]=\ln Z[0]=-E_{vac}$, where $E_{vac}$ is a vacuum energy of the considering QFT, and it is useful to consider its volume
density $w_{vac}:=\frac{E_{vac}}{V}$. The quantity $w_{vac}$ is a very useful quantity which in particular can tell whether the system has phase transition or not. By the way, this relation is the literal. 

From Lehmann -- Symanzik -- Zimmermann reduction formula, one can obtain that $\mathcal{S}$-matrix $n$-particle functions in momentum representation reads:
\begin{equation}
\mathcal{S}_{1^{\prime}\ldots n^{\prime}}^{(n)}=\mathcal{D}_{1,...,n}^{(n)}\prod_{k=1}^{n}\left(G_{kk^{\prime}}^{(2)}\right)^{-1},
\end{equation}
where $G_{kk^{\prime}}^{(2)}$ is a Green function of Gaussian theory. Of course, $\hat{G}^{-1}=\hat{L}$, but it is accepted to denote corresponding operator as $\hat{G}^{-1}$. For deriving the $\mathcal{S}$-matrix
generating functional, that is exactly the functional which functional Taylor coefficients are $\mathcal{S}_{1^{\prime}\ldots n^{\prime}}^{(n)}$,
one should simply change variables from sources to so-called ``classical fields'', namely:
\begin{equation}
\bar{\phi}(y)=\int dx\ G(x-y)j(x),
\end{equation}
so the $\mathcal{S}$-matrix generating functional satisfy:
\begin{equation}
\mathcal{S}[\bar{\phi}(x)]=\frac{1}{Z[0]}Z[\hat{G}^{-1}\bar{\phi}]
\end{equation}
as well as ``normalized'' generating functional:
\begin{equation}
Z_{I}[j(x)]=\frac{Z[j(x)]}{\mathcal{Z}_{0}[j(x)]},
\end{equation}
and normalized generating functional of connected diagrams:
\begin{equation}
\mathcal{G}_{I}[j]=+\ln Z_{I}[j(x)]=\mathcal{G}[j]-\mathcal{G}_{0}[j],
\end{equation}
where $\mathcal{G}_{0}[j]$ is generating functional of connected diagrams of Gaussian theory and equals to:
\begin{equation}
\mathcal{G}_{0}[j]=\frac{1}{2}\int dxdyj\left(x\right)G(x,y)j\left(y\right)
\end{equation}

\paragraph{Perturbation Theory Decompositions}

Let us consider a decomposition of GF $Z$ in series:
\begin{equation}
\mathcal{Z}\left[j\right]=\sum_{n=0}^{\infty}Z_{n}[j],\qquad\mathcal{G}[j]=\sum_{n=0}^{\infty}\mathcal{G}_{n}[j]
\end{equation}
which could be power series in some parameter of the system such as coupling constant $g$. Then induce the corresponding decompositions of $n$-point functions in powers of the same parameter:
\begin{equation}
\mathcal{D}_{1...k}^{(k)}=\sum_{n=0}^{\infty}\mathcal{D}_{1,...,k;\ n}^{(k)},\qquad\mathcal{G}_{1...k}^{(k)}=\sum_{n=0}^{\infty}\mathcal{G}_{1,...,k;\!n}^{(k)},
\end{equation}
as well as for the $\mathcal{S}$-matrix generating functional. 

\paragraph{Standard Perturbation Theory and Its Inapplicability}

Usually, when considering the quantum field theory with generating functional:
\begin{equation}
Z\left[j\left(x\right)\right]=\int D\phi e^{-\frac{1}{2}\int d^{d}xd^{d}y\ L(x,y)\phi(x)\phi(y)-\int d^{d}xV(\phi(x))-\int j\left(x\right)\phi\left(x\right)dx},
\end{equation}
with analytic interaction $V(\phi)$, one can obtain Feynman perturbation series via transformation:
\begin{equation}
Z\left[j\left(x\right)\right]=\exp\left[-\int d^{d}x\ V\left(\frac{\delta}{\delta j(x)}\right)\right]\int D\phi e^{-\frac{1}{2}\int d^{d}xd^{d}y\ L(x,y)\phi(x)\phi(y)-\int j\left(x\right)\phi\left(x\right)dx},
\end{equation}
where the operator $\exp\left[-\int d^{d}x\ V\left(\frac{\delta}{\delta j(x)}\right)\right]$
is understood in sense of power series in operator $\frac{\delta}{\delta j(x)}$.
Consequently, one have to demand the analyticity of $V(\phi)$ in $\phi$ for this method to work.

Then, after introducing a new field $\tilde{\phi}$, the final formula has the form:
\begin{equation}
\mathcal{Z}\left[j\right]=\left.\exp\left\{ -\frac{1}{2}\mathcal{G}_{xy}^{(2)}\frac{\delta^{2}}{\delta\tilde{\varphi}_{x}\delta\tilde{\varphi}_{y}}\right\} \mathrm{e}^{\frac{\mathrm{i}}{\hbar}\left\{ \int d^{d}x\ V\left(\tilde{\phi}(x)\right)+j_{x^{\prime}}\tilde{\varphi}_{x^{\prime}}\right\} }\right|_{\tilde{\varphi}=0}.
\end{equation}
Here operator $V\left(\frac{\delta}{\delta j(x)}\right)$ should be properly defined since it contains multiple variational derivatives in the same point, but in final answer all the regularizations of this expression could be removed. Then, expanding both exponents, one can obtain series in coupling constant. These series are usually asymptotic, and is is caused by the carrying out the operator $\exp\left[-\int d^{d}x\ V\left(\frac{\delta}{\delta j(x)}\right)\right]$, 

As one can see, this way rely significantly on the analyticity of $V(\phi)$. It is invalid for the case of fractional power self-interaction potential.

\paragraph{Motivation of Study}

At the first glance it can appear to be nonphysical to consider models with interaction of the form $V(\phi)=g\left|\phi(x)\right|^{\alpha}$.
Though, there are several reasons to study it.

Firstly, it is a kind of exactly solvable model in QFT, where one can produce converging series. So it is an object of common interest because it is useful to learn the behavior of different QFT systems
to:
\begin{enumerate}
\item understand, what kind of properties they can own and, consequently, what one can look up in other models
\item develop new methods and approaches, which can be subsequently extended and applied to other systems to gain new results
\end{enumerate}

Secondly, there are reasonable expectations, that in theory with $V(\phi)=g\left|\phi(x)\right|^{\alpha}$
it is possible to complete explicitly transition in obtained formulas from weak to strong coupling regimes. Moreover, there is likely to be the following duality between the scalar field theories (we denote
for the symmetry $g=\lambda^{\alpha}$):
\begin{equation}
\hat{L}\leftrightarrow\hat{L}^{-1},\qquad\frac{1}{\alpha}\lambda^{\alpha}\left|\phi(x)\right|^{\alpha}\leftrightarrow\frac{1}{\alpha/(\alpha-1)}\left(\pm\frac{i}{\lambda}\right)^{\frac{\alpha}{\alpha-1}}\left|\psi(x)\right|^{\frac{\alpha}{\alpha-1}},\qquad\alpha\leftrightarrow\frac{\alpha}{\alpha-1} \qquad \lambda\leftrightarrow\pm\frac{i}{\lambda},
\end{equation}
which matches strong coupling regime of one theory with weak coupling regime of other theory. For instance, it tells that for $\phi^{4}$:
\begin{equation}
\hat{L}\leftrightarrow\hat{G},\qquad\mu^{4}\left|\phi(x)\right|^{4}\leftrightarrow\left(\frac{i}{\mu}\right)^{\frac{4}{3}}\left|\psi(x)\right|^{\frac{4}{3}},
\end{equation}
which can be extremely useful for the researching of $\phi^{4}$ in strong-coupling regime as well as non-perturbative description of its phase transitions. 

This duality can be obtained with the use of Plancherel's identity to functional integral with the following substitution of saddle-point asymptotic in the interaction part, which turn's out to be valid in
renormalizable cases. Though, for careful checking of this duality one should obtain perturbation series asymptotic expansions of $\lambda^{4}\phi^{4}$ from the strong coupling regime of $\left(\frac{i}{\lambda}\right)^{\frac{4}{3}}\left|\psi\right|^{\frac{4}{3}}$,
which demands the general studying of $\left|\phi\right|^{\alpha}$-theories for $1\leq\alpha\leq2$ (the case $\alpha=2$ is trivial and can be a key for the obtained formulas validation).

Describing, checking and using of this duality is a subject of current studies of the authors, and will be the topic of the nearest publications.

\part{Derivation of Perturbation Theory}

\section{Mathematical Buildup and Definitions}

\subsection{Gaussian Measure}

Let us start from fractional QFT for $1<\alpha<2$, $g>0$ and $j\in\mathcal{L}^{2}(X)$:
\begin{equation}
Z\left[j\left(x\right)\right]=\int_{\mathcal{H}}D\mu_{G}\left(\phi\right)e^{-g\int\left|\phi\left(x\right)\right|^{\alpha}dx-\int j\left(x\right)\phi\left(x\right)dx},
\end{equation}
where we understand functional integral as integral over Hilbert space
$\mathcal{H}=\{\phi\in\mathcal{L}^{2}(X)\ \big|\ \phi(L\vec{e}_{i}+x)=\phi(x),\ i=1,..,d\}$
of periodic functions on $X$ for $X$ - $d$-dimensional cube with length of edge $L$ and volume $V=L^{d}$ with Gaussian measure. 

By definition, a Gaussian measure is a measure on Hilbert space defined by some covariation operator $L:\mathcal{H}\longrightarrow\mathcal{H}$,
which is demanded to be trace-class. Let us introduce 
$\{e_{n}\}_{n=1}^{\infty}$
- a family of eigenvectors of $L$ with corresponding eigenvalues $\{\lambda_{n}\}_{n=1}^{\infty}$,
so that $Ge_{n}=\lambda_{n}e_{n}$ $\forall n$. We also denote as $\mathcal{H}_{N}:=\left\langle e_{i}\right\rangle _{i=1}^{N}$
a subspace generated by first $N$ eigenvectors, and as $P_{N}:\mathcal{H}\longrightarrow\mathcal{H}$ a projection operator on this subspace. Then, by definition, integral with respect to Gaussian measure with operator $L$ of any function $f$ depending only on $P_{N}\phi$ rather than $\phi$ itself is: 

\begin{equation}
\int_{\mathcal{H}}f[\phi]D\mu_{G}\left(\phi\right)=\int_{\mathcal{H}_{N}}f[P_{N}\phi]\left(\prod_{k=1}^{N}\sqrt{\frac{1}{2\pi\lambda_{k}}}d\phi_{k}\right)e^{-\frac{1}{2}\sum_{k=1}^{N}\frac{\phi_{k}^{2}}{\lambda_{k}}},
\end{equation}

Then integral of generic function $f:\mathcal{H}\longrightarrow\mathbb{C}$
can be computed using Lebesgue Dominated Convergence Theorem, if $f[P_{N}\phi]\longrightarrow f[\phi]$:

\begin{equation}
\int_{\mathcal{H}}f[\phi]D\mu_{L}\left(\phi\right)=\lim_{N\longrightarrow\infty}\int_{\mathcal{H}_{N}}f[P_{N}\phi]D\mu_{L}\left(\phi\right)=\lim_{N\longrightarrow\infty}\int_{\mathcal{H}_{N}}f[P_{N}\phi]\left(\prod_{k=1}^{N}\sqrt{\frac{1}{2\pi\lambda_{k}}}d\phi_{k}\right)e^{-\frac{1}{2}\sum_{k=1}^{N}\frac{\phi_{k}^{2}}{\lambda_{k}}},
\end{equation}

All the described theoretical-measure results can be found with derivation, for example, in (\cite{Bogachev_measure}).

\subsection{About Continuity of Functionals}

Let us make one notice which we will not use in further exposition. In our case:
\[
f[\phi]=e^{-g\int\left|\phi\left(x\right)\right|^{\alpha}dx-\int j\left(x\right)\phi\left(x\right)dx},
\]
so it is positive, measurable and $f[\phi]\leq e^{-\int j\left(x\right)\phi\left(x\right)dx}\leq e^{\left\Vert j\right\Vert _{2}\left\Vert \phi\right\Vert _{2}}$
which is clearly integrable with finite value of integral. Moreover, for proving that $f[P_{N}\phi]\longrightarrow f[\phi]$ almost everywhere (a.e.) it is necessary and sufficient to show that $\int\left|\phi\left(x\right)\right|^{\alpha}dx$
is a continuous functional on $\mathcal{H}$, since exponent and $\int j\left(x\right)\phi\left(x\right)dx$
are evidently continuous. But, since $1<\alpha\leq2$:
\[
\int\left|\phi\left(x\right)\right|^{\alpha}dx=\left\Vert \phi\right\Vert _{\alpha}^{\alpha},\qquad\left|\left\Vert P_{N}\phi\right\Vert _{\alpha}-\left\Vert \phi\right\Vert _{\alpha}\right|\leq\left\Vert P_{N}\phi-\phi\right\Vert _{\alpha}\leq V^{\left(\frac{1}{\alpha}-\frac{1}{2}\right)}\left\Vert P_{N}\phi-\phi\right\Vert _{2}
\]
which is also continuous for all $\phi\in\mathcal{H}$. Here we have used a well-known consequence from H{\"o}lder inequality for set with finite measure $V$. We would like to underline that this functional
is not continuous when $\alpha>2$, because in this case $\left\Vert \phi\right\Vert _{\alpha}$
is not finite if $\left\Vert \phi\right\Vert _{2}$ is finite (which is equivalent to $\phi\in\mathcal{H}$).

As a result, one can apply Lebesgue Dominated Convergence Theorem and find that:
\[
Z\left[g,j\left(x\right)\right]=\lim_{N\longrightarrow\infty}\int_{\mathcal{H}_{N}}\prod_{k=1}^{N}\sqrt{\frac{1}{2\pi\lambda_{k}}}e^{-\frac{1}{2}\sum_{k=1}^{N}\frac{\phi_{k}^{2}}{\lambda_{k}}-g\int\left|\phi_{N}\left(x\right)\right|^{\alpha}dx-\int j\left(x\right)\phi_{N}\left(x\right)dx}d\phi_{k}
\]

That's quite a good outlook that if one consider a functional integral from Gaussian measure point of view, one can't compute functional integral with $\alpha>2$ (e.g. $\alpha=4$, which is currently the
most interesting for physics case) by simply substituting $P_{N}\phi$ instead of $\phi$ and taking a limit $N\longrightarrow\infty$. Unlike, the considering case is much more transparent for calculations. Moreover,
further we will see that the derived perturbation series will diverge for $\alpha>2$, which, of course, not a new result.

\section{Reduction of Functional Integral to the Series from Finite-Dimensional Ones (Grand Canonical Partition Function Decomposition)}

In this section we are going to derive some kind of perturbation series for considering series, which will converge. Since power of a potential is not a natural number, the traditional method with carrying out from integral an interaction action in the form of a differential operator is not suitable. So, we follow
the other way and in some sense get all Feynman diagrams summed into one integral, which appears to be exactly the coefficient of $g^{n}$ in the obtained series. The described approach of constructing a perturbation theory repeats and develops methods described in \cite{Efimov1970}, \cite{Chebotarev_2019}. 

 
Let us proceed to the computation itself. Firstly, notice that:
\begin{equation}
\int\left|\phi\left(x\right)\right|^{\alpha}dx=\frac{1}{2\pi}\int\mathcal{F}\left[\mathcal{F}\left[\left|\phi\left(x\right)\right|^{\alpha}\right]\right]dx,
\end{equation}
where we understand Fourier transform in the sense of distributions. Of course, $\left|\phi\left(x\right)\right|^{\alpha}$ is not a regular function for a.e. $x$, so approximate it by some regular function $U_{\Lambda}(x)$ from the Schwartz space, as an example, so that $U_{\Lambda}\left[\phi\left(x\right)\right]\longrightarrow\left|\phi\left(x\right)\right|^{\alpha},$ when
$\Lambda\longrightarrow\infty$. We will make transitions for regular functional $U_{\Lambda}\left[\phi\left(x\right)\right]$ (family of
regular functions depending on $x$ in some kind), and then will apply the arguments of continuity. Moreover, we insert the factor $e^{-\frac{\epsilon}{2}\int\left|\phi\left(x\right)\right|^{2}dx}$ under the integral to avoid the problems with convergence. 

So we start from considering the ``regularized'' functional:
\begin{equation}
Z_{\Lambda,\epsilon}\left[j\left(x\right)\right]=\int D\mu_{G}\left(\phi\right)e^{-\frac{\epsilon}{2}\int\left|\phi\left(x\right)\right|^{2}dx-g\int U_{\Lambda}\left[\phi\left(x\right)\right]dx-\int j\left(x\right)\phi\left(x\right)dx},
\end{equation}
and due to the Lebesgue Dominated Convergence Theorem described in previous section:
\[
Z_{\Lambda,\epsilon}\left[j\left(x\right)\right]\longrightarrow Z\left[j\left(x\right)\right],\qquad\Lambda\longrightarrow\infty,\ \epsilon\longrightarrow0,
\]
and these limits are permutable.

We can substitute Fourier image of potential and use that $\mathcal{F}\left[U_{\Lambda}(x)\right]$
is also regular function:
\begin{align*}
Z_{\Lambda,\epsilon}\left[j\left(x\right)\right] & =\int D\mu_{G}\left(\phi\right)e^{-g\int U_{\Lambda}\left[\phi\left(x\right)\right]dx-\int j\left(x\right)\phi\left(x\right)dx-\frac{\epsilon}{2}\int\left|\phi\left(x\right)\right|^{2}dx}\nonumber \\
 & =\int D\mu_{G}\left(\phi\right)e^{-g\int dx\int e^{it\phi(x)}\mathcal{F}\left[U_{\Lambda}\left[\phi\left(x\right)\right]\right]\left(t\right)\frac{dt}{2\pi}-\int j\left(x\right)\phi\left(x\right)dx-\frac{\epsilon}{2}\int\left|\phi\left(x\right)\right|^{2}dx}
\end{align*}
Taking the Taylor series:
\[
Z_{\Lambda,\epsilon}\left[j\left(x\right)\right]=\int D\mu_{G}\left(\phi\right)\sum_{n=0}^{\infty}\frac{\left(-g\right)^{n}}{n!\left(2\pi\right)^{n}}\left\{ \prod_{a=1}^{n}\int\int dx_{a}dt_{a}\mathcal{F}\left[U_{\Lambda}\left[\phi\right]\right]\left(t_{a}\right)\right\}\]
\[\times e^{i\sum_{a=1}^{n}t_{a}\phi(x_{a})-\int j\left(x\right)\phi\left(x\right)dx-\frac{\epsilon}{2}\int\left|\phi\left(x\right)\right|^{2}dx}
\]
Exchanging summation and integration we receive:
\begin{equation*}
Z_{\Lambda,\epsilon}\left[j\left(x\right)\right]=\sum_{n=0}^{\infty}\frac{\left(-g\right)^{n}}{n!\left(2\pi\right)^{n}}\left\{ \prod_{a=1}^{n}\int\int dx_{a}dt_{a}\mathcal{F}\left[U_{\Lambda}\left[\phi\right]\right]\left(t_{a}\right)\right\}\end{equation*} 
\[\times \int D\mu_{G}\left(\phi\right)e^{i\sum_{a=1}^{n}t_{a}\phi(x_{a})-\int j\left(x\right)\phi\left(x\right)dx-\frac{\epsilon}{2}\int\left|\phi\left(x\right)\right|^{2}dx}\]
The correctness of this permutation will be proven independently in the next section.

Finally, we have to take the integral over $\phi$. Notice that:
\[
\int D\mu_{G}\left(\phi\right)e^{i\sum_{a=1}^{n}t_{a}\phi(x_{a})-\int j\left(x\right)\phi\left(x\right)dx-\frac{\epsilon}{2}\int\left|\phi\left(x\right)\right|^{2}dx}=\int D\mu_{G}\left(\phi\right)e^{-\int\tilde{j}\left(x\right)\phi\left(x\right)dx-\frac{\epsilon}{2}\int\left|\phi\left(x\right)\right|^{2}dx},
\]
for:
\begin{equation*}
\tilde{j}\left(x\right)=j\left(x\right)-i\sum_{a=1}^{n}t_{a}\delta(x-x_{a}),
\end{equation*}
and such integrals are the limit of usual Gaussian integrals:
\[
\int D\mu_{L}\left(\phi\right)e^{-\int\tilde{j}\left(x\right)\phi\left(x\right)dx-\frac{\epsilon}{2}\int\left|\phi\left(x\right)\right|^{2}dx}=\lim_{N\longrightarrow\infty}\int_{\mathcal{H}_{N}}\prod_{k=1}^{N}\sqrt{\frac{1}{2\pi\lambda_{k}}}e^{-\frac{1}{2}\sum_{k=1}^{N}\frac{\phi_{k}^{2}}{\lambda_{k}+\epsilon}-\sum_{k=1}^{N}\tilde{j}_{k}\phi_{k}}d\phi_{k}=
\]
\[
=e^{\frac{1}{2}\sum_{k=1}^{N}\left(\lambda_{k}+\epsilon\right)|j_{k}|^{2}}=e^{\frac{1}{2}\left\langle (G+\epsilon)\tilde{j},\tilde{j}\right\rangle }
\]
where we derived the equality for regular $\tilde{j}$ and then used the continuity for singular distributions. Consequently, if we denote
by $\overline{\phi}(x_{a}):=\int dy\left(G\left(x_{a}-y\right)+\epsilon\right)j\left(y\right)$,
$G_{ab}:=G(x_{a}-x_{b})$ and $\mathcal{Z}_{0}\left[j\left(x\right)\right]:=e^{\frac{1}{2}\int\int dxdyj\left(x\right)G(x,y)j\left(y\right)}$,
we result with:
\begin{equation}
Z_{\Lambda,\epsilon}\left[j\left(x\right)\right]=\mathcal{Z}_{0}\left[j\left(x\right)\right]\sum_{n=0}^{\infty}\frac{\left(-g\right)^{n}}{n!\left(2\pi\right)^{n/2}}\left\{ \prod_{a=1}^{n}\int\int dx_{a}dt_{a}\mathcal{F}\left[U_{\Lambda}\left[\phi\right]\right]\left(t_{a}\right)\right\} 
\end{equation}
\[\times e^{-\frac{1}{2}\left(\sum_{a,b=1}^{n}t_{a}(G_{ab}+\epsilon\delta_{ab})t_{b}\right)-i\sum_{a=1}^{n}t_{a}\overline{\phi}\left(x_{a}\right)}\]
Basically, $\mathcal{Z}_{0}\left[j\left(x\right)\right]$ is no more than generating functional for Gaussian theory, namely $g=0$ case. Here
$\overline{\phi}(x_{a}):=\int dy\left(G\left(x_{a}-y\right)+\epsilon\right)j\left(y\right)$ are no more than classical fields in points $x_{a}$, corresponding
to the sources $j$. 

We can use Plancherel's Identity:
\begin{align}
Z_{\Lambda,\epsilon}\left[j\left(x\right)\right] & =\mathcal{Z}_{0}\left[j\left(x\right)\right]\sum_{n=0}^{\infty}\frac{\left(-g\right)^{n}}{n!\left(2\pi\right)^{n/2}}\left\{ \prod_{a=1}^{n}\int\int dx_{a}d\phi_{a}U_{\Lambda}\left[\phi_{a}\right]\right\} \frac{e^{\frac{1}{2}\sum_{a,b=1}^{n}\left(-i\overline{\phi}(x_{a})+i\phi_{a}\right)R_{ab}^{(n)}\left(-i\overline{\phi}(x_{b})+i\phi_{b}\right)}}{\sqrt{\det\left(G^{(n)}+\epsilon\right)}}\nonumber \\
 & =\mathcal{Z}_{0}\left[j\left(x\right)\right]\sum_{n=0}^{\infty}\frac{\left(-g\right)^{n}}{n!\left(2\pi\right)^{n/2}}\left\{ \prod_{a=1}^{n}\int dx_{a}e^{-\frac{1}{2}\sum_{a,b}j_{G}\left(x_{a}\right)R_{ab}^{(n)}j_{G}\left(x_{b}\right)}\int d\phi_{a}U_{\Lambda}\left[\phi_{a}\right]\right\}
\end{align}
\[\times \frac{e^{-\frac{1}{2}\sum\phi_{a}R_{ab}^{(n)}\phi_{b}+\sum_{a=1}^{n}\phi_{a}\chi\left(x_{a}\right)}}{\sqrt{\det\left(G^{(n)}+\epsilon\right)}}\]
Where $G^{(n)}$ is restriction matrix from $G_{ab}$ for the case of $n$ coordinates, $\chi\left(x_{a}\right)=\sum_{b=1}^{n}R_{ab}^{(n)}\overline{\phi}\left(x_{b}\right)$ and we use new notation $R_{ab}^{(n)}=\left(\left(G^{(n)}+\epsilon\right)^{-1}\right)_{ab}$.
Notice that both $R^{(n)}$ and $G^{n}$ are symmetric, hence diagonalizable in orthonormal frame by orthogonal transformation. Moreover, all $G^{(n)}$ are positive, since $G$ is positive:
\[
\left\langle G^{(n)}j,j\right\rangle \geq0,
\]
and for distributions this inequality also holds. So, $G^{(n)}+\epsilon$ is positive definite matrix for all $n$ as well as its inverse. This proves that all written integrals are converge absolutely, so we can interchange the order of integration in $t_{a}$ and $x_{a}$ variables.

In the next section we will show that the obtained series converges uniformly on $\Lambda$ and $\epsilon$, since we can choose $U_{\Lambda}(x)$ so that for a.e. $x$ there will be
\[
0\leq U_{\Lambda}\left[\phi\left(x\right)\right]\leq\left|\phi\left(x\right)\right|^{\alpha},
\]
for instance, supposing that $U_{\Lambda}\left[\phi\left(x\right)\right]=\left|\phi\left(x\right)\right|^{\alpha}\cdot\eta\left(\frac{\left|\phi\left(x\right)\right|}{\Lambda}\right)$, where $\eta(\phi)$ is a bump-function. So, we can apply Weierstrass M-test to the series for $Z_{\Lambda,\epsilon}$ if we prove that the series above with 
$U_{\Lambda}(x)=\left|\phi\left(x\right)\right|^{\alpha}$
converges. We will do it in the next part independently.

Up to this moment, we finish at:
\begin{align}
Z_{I}\left[j\left(x\right)\right] & =\sum_{n=0}^{\infty}\frac{\left(-g\right)^{n}}{n!\left(2\pi\right)^{n/2}}\left\{ \prod_{a=1}^{n}\int dx_{a}e^{-\frac{1}{2}\sum_{a,b}\overline{\phi}\left(x_{a}\right)\left(G^{(n)}\right)_{ab}^{-1}\overline{\phi}\left(x_{b}\right)}\int d\phi_{a}\left|\phi_{a}\right|^{\alpha}\right\} \nonumber \\
 & \times\frac{e^{-\frac{1}{2}\sum\phi_{a}\left(G^{(n)}\right)_{ab}^{-1}\phi_{b}+\sum_{a=1}^{n}\phi_{a}\chi\left(x_{a}\right)}}{\sqrt{\det\left(G^{(n)}\right)}},
\end{align}
where we denoted $Z_{I}\left[j\left(x\right)\right]=Z\left[j\left(x\right)\right]/\mathcal{Z}_{0}\left[j(x)\right]$,
in accordance with the notations in introductory section.

It is worth noting that for the case of coupling constant depending on coordinates, $g=g(x)$, we can similarly derive that:
\begin{align}
Z_{I}\left[j\left(x\right)\right] & =\sum_{n=0}^{\infty}\frac{\left(-1\right)^{n}}{n!\left(2\pi\right)^{n/2}}\left\{ \prod_{a=1}^{n}\int dx_{a}\ g(x_{a})e^{-\frac{1}{2}\sum_{a,b}\overline{\phi}\left(x_{a}\right)\left(G^{(n)}\right)_{ab}^{-1}\overline{\phi}\left(x_{b}\right)}\int d\phi_{a}\left|\phi_{a}\right|^{\alpha}\right\} \times\nonumber \\
 & \times\frac{e^{-\frac{1}{2}\sum\phi_{a}\left(G^{(n)}\right)_{ab}^{-1}\phi_{b}+\sum_{a=1}^{n}\phi_{a}\chi\left(x_{a}\right)}}{\sqrt{\det\left(G^{(n)}\right)}},
\end{align}
but we will mostly assume $g(x)=g=const$ to avoid over complication of formulas.

At this moment we have obtained the decomposition of $Z_{I}$ into the perturbation series in the powers of coupling constant. We will denote the $n$-th term as $Z_{I,n}$:
\begin{equation}
Z_{I}=\sum_{n=0}^{\infty}Z_{I,n}
\end{equation}
Clearly, this decomposition also yields the same for $Z$:
\[
Z=\sum_{n=0}^{\infty}Z_{n},
\]
where $Z_{n}$ can be directly obtained from $Z_{I,n}$ recalling the definition of $\mathcal{Z}_{0}\left[j\left(x\right)\right]$.

This expansion of GF in perturbation series looks similarly to the expansion of grand canonical partition function of non-ideal gas with potential $G(x_{a}-x_{b})$, but with additional inner integrals over $t_{a}$. Keeping this in mind we will refer to $Z_{n}$ as $n$-particle partition function. This inner integrals are in fact the key difference between statistical physics and quantum field theory, warranting the complication of the last one.

\paragraph{Properties of $G^{(n)}$-Matrices}

Also, it is useful to visualize typical structure of matrix $G$:
\begin{equation}
G^{(n)}=\left(\begin{array}{cccc}
G(0) & G(x_{1}-x_{2}) & \ldots & G(x_{1}-x_{n})\\
G(x_{1}-x_{2}) & G(0) & \ldots & \ldots\\
\ldots & \ldots & G(0) & G(x_{n}-x_{n-1})\\
G(x_{1}-x_{n}) & \ldots & G(x_{n}-x_{n-1}) & G(0)
\end{array}\right)
\end{equation}

Also there are two evident cases for this matrix:
\begin{enumerate}
\item When all $x_{a}$ are equal, then:
\[
G^{(n)}=\left(\begin{array}{cccc}
G(0) & G(0) & \ldots & G(0)\\
G(0) & G(0) & \ldots & \ldots\\
\ldots & \ldots & G(0) & G(0)\\
G(0) & \ldots & G(0)) & G(0)
\end{array}\right)
\]
\item When all $x_{a}$ are infinitely faraway, then:
\[
G^{(n)}=\left(\begin{array}{cccc}
G(0) & 0 & \ldots & 0\\
0 & G(0) & \ldots & \ldots\\
\ldots & \ldots & G(0) & 0\\
0 & \ldots & 0 & G(0)
\end{array}\right)
\]
\end{enumerate}
As we have already noticed, $G^{(n)}\geq0$, so its diagonalizable. Let us denote its eigenvalues as $0\leq\lambda_{1}^{(n)}\leq\lambda_{2}^{(n)}\leq...\leq\lambda_{n}^{(n)}$.
Since $\sum_{a=1}^{n}\lambda_{a}^{(n)}=tr\ G^{(n)}=nG(0)$, then $\left\Vert G^{(n)}+\epsilon\right\Vert \leq\lambda_{n}^{(n)}+n\epsilon\leq n(G\left(0\right)+\epsilon)$.

\section{Majorizing Series (Majorant)}

\subsection{Coordinate-Free Part}

We always can exchange summation and integration when functions are positive due to Levi Monotone Convergence Theorem. And we have to check absolute convergence of the obtained series to use Lebesgue Dominated Convergence Theorem. So, it's sufficient to take absolute value of every term, exchange sum and integral and prove that this series converges. 

\subsubsection{Expression of Majorant via Hypergeometric Functions}

At this moment we have to find top estimation of the terms of our series:
\[
J_{n}\left(\alpha\right)=\frac{1}{\sqrt{\det\left(G^{(n)}+\epsilon\right)}}\int dt_{1}...dt_{n}\left|t_{1}\right|^{\alpha}...\left|t_{n}\right|^{\alpha}e^{-\frac{1}{2}\sum_{a,b=1}^{n}R_{ab}^{\left(n\right)}t_{a}t_{b}+\sum_{a=1}^{n}t_{a}\chi\left(x_{a}\right)}
\]

Estimate $\left|t_{1}\right|^{\alpha}...\left|t_{n}\right|^{\alpha}$
with restriction $\left\Vert t\right\Vert ^{2}=r^{2}$:
\[
f=\alpha\sum_{a}\ln t_{a}-\lambda\left(\left\Vert t\right\Vert ^{2}-r^{2}\right)
\]
\[
\partial_{a}f=\frac{\alpha}{t_{a}}-2\lambda t_{a}=0
\]
 hence:
\[
t_{a}^{2}=\frac{\alpha}{2\lambda}
\]
Take the sum over $a$ to get $\lambda$:
\[
r^{2}=\frac{\alpha n}{2\lambda}
\]
So, finally, the maximum is reached at:
\[
\phi_{a}=\frac{r}{\sqrt{n}}
\]
So, one can estimate (and these bound are strict):
\[
0\leq\left|t_{1}\right|^{\alpha}...\left|t_{n}\right|^{\alpha}\leq\frac{r^{n\alpha}}{n^{\frac{n\alpha}{2}}}
\]
In addition to previous bound we can change variables by the orthogonal transformation and diagonalize the exponent putting $t_{a}=\sum_{b=1}^{n}R_{ab}^{-1/2}\zeta_{b}$
plus estimate $\left\Vert t\right\Vert \leq\left\Vert G^{(n)}+\epsilon\right\Vert ^{\frac{1}{2}}\left\Vert \zeta\right\Vert $:
\[
\frac{J_{n}\left(\alpha\right)}{n!}\leq\frac{\sqrt{\det\left(G^{(n)}+\epsilon\right)}}{\sqrt{\det\left(G^{(n)}+\epsilon\right)}}\frac{1}{n^{\frac{n\alpha}{2}}}\left\Vert G^{(n)}+\epsilon\right\Vert ^{\frac{n\alpha}{2}}\frac{1}{n!}\int d\zeta_{1}...d\zeta_{n}\left\Vert \zeta\right\Vert ^{n\alpha}e^{-\frac{\left\Vert \zeta\right\Vert ^{2}}{2}+\sum_{a,b=1}^{n}\zeta_{a}G_{ab}^{(n)-1/2}\overline{\phi}\left(x_{b}\right)},
\]
Now we have to transform the part with source:
\[
\left|\sum_{a,b=1}^{n}\zeta_{a}G_{ab}^{(n)-1/2}\overline{\phi}\left(x_{b}\right)\right|\leq\left\Vert \zeta\right\Vert \cdot\left\Vert \sum_{b=1}^{n}G_{ab}^{(n)-1/2}\overline{\phi}\left(x_{b}\right)\right\Vert.
\]
We need to prove the following not very difficult statement:
\[
j_{0}\leq\sup_{j}\sqrt{\left\langle (G+\epsilon)^{-1}Gj,Gj\right\rangle }\leq\sup_{j}\sqrt{\left\langle Gj,j\right\rangle }\leq\sqrt{\left\Vert G\right\Vert }\left\Vert j\right\Vert 
\]
we will denote $j_{0}:=\left\Vert \sum_{b=1}^{n}G_{ab}^{(n)-1/2}\overline{\phi}\left(x_{b}\right)\right\Vert =\sqrt{\sum_{a=1}^{n}\left(\sum_{b=1}^{n}G_{ab}^{(n)-1/2}\overline{\phi}\left(x_{b}\right)\right)^{2}}=\sqrt{\sum_{a,b=1}^{n}R_{ab}^{(n)}\overline{\phi}\left(x_{a}\right)\overline{\phi}\left(x_{b}\right)}\geq0$.
Then we have:
\[
\frac{J_{n}\left(\alpha\right)}{n!}\leq\frac{1}{n^{\frac{n\alpha}{2}}}\left\Vert G^{(n)}\right\Vert ^{\frac{n\alpha}{2}}\frac{1}{n!}\int d\zeta_{1}...d\zeta_{n}\left\Vert \zeta\right\Vert ^{n\alpha}e^{-\frac{\left\Vert \zeta\right\Vert ^{2}}{2}+\left\Vert \zeta\right\Vert j}
\]
And recall that $\left\Vert G^{(n)}+\epsilon\right\Vert \leq n(G\left(0\right)+\epsilon).$
The last integral is easily computed via series expansion in $j_{0}$:
\[
\frac{J_{n}\left(\alpha\right)}{n!}\leq\left(G\left(0\right)+\epsilon\right)^{\frac{n\alpha}{2}}\frac{1}{n!}\frac{n\pi^{\frac{n}{2}}2^{\frac{1}{2}\left(n\alpha+n-2\right)}}{\Gamma\left(\frac{n}{2}+1\right)}\sum_{m=0}^{\infty}\frac{2^{\frac{m}{2}}\Gamma\left(\frac{1}{2}(m+n+n\alpha)\right)}{\Gamma(m+1)}j_{0}^{m}.
\]

\subsubsection{Bounds for the Series}

As a first step, we use log-convexity of gamma-function for $m\geq1$:
\[
\Gamma\left(\frac{n(\alpha+1)}{2}+\frac{m}{2}\right)\leq\Gamma\left(\frac{n(\alpha+1)}{2}+\frac{m+1}{2}\right)\leq\Gamma\left(n(\alpha+1)\right)^{1/2}\Gamma\left(m+1\right)^{1/2},
\]
and rewrite:
\[
\sum_{m=0}^{\infty}\frac{2^{\frac{m}{2}}\Gamma\left(\frac{1}{2}(m+n+n\alpha)\right)}{\Gamma(m+1)}j_{0}^{m}\leq\Gamma\left(n(\alpha+1)\right)^{1/2}\sum_{m=1}^{\infty}\frac{2^{m}}{\Gamma\left(m+1\right)^{1/2}}j_{0}^{m}.
\]
Moreover, for $m\longrightarrow\infty$:
\[
\frac{\Gamma(m+1)^{1/2}}{2^{m/2}\Gamma\left(\frac{m+1}{2}\right)}\sim\frac{\sqrt{12m+1}}{2\sqrt{3}\sqrt[4]{2\pi}\sqrt[4]{m}}\longrightarrow\infty,
\]
and it could be numerically checked that for $m\geq0$:
\[
\frac{\Gamma(m+1)^{1/2}}{2^{m/2}\Gamma\left(\frac{m+1}{2}\right)}\geq\frac{1}{2},
\]
so one can bound:
\[
\sum_{m=0}^{\infty}\frac{2^{\frac{m}{2}}\Gamma\left(\frac{1}{2}(m+n+n\alpha)\right)}{\Gamma(m+1)}j_{0}^{m}\leq2\Gamma\left(n(\alpha+1)\right)^{1/2}\left(1+\sum_{m=1}^{\infty}\frac{1}{\Gamma\left(\frac{m+1}{2}\right)}j_{0}^{m}\right).
\]
The last series can be calculated exactly:
\[
\sum_{m=1}^{\infty}\frac{1}{\Gamma\left(\frac{m+1}{2}\right)}j_{0}^{m}=e^{j_{0}^{2}}j_{0}(\text{erf}(j_{0})+1)\leq2e^{j_{0}^{2}}j_{0},
\]
so we finish at:
\[
\sum_{m=0}^{\infty}\frac{2^{\frac{m}{2}}\Gamma\left(\frac{1}{2}(m+n+n\alpha)\right)}{\Gamma(m+1)}j_{0}^{m}\leq2\Gamma\left(n(\alpha+1)\right)^{1/2}e^{j_{0}^{2}}(1+j_{0})
\]
 
\subsubsection{Total Result}

Let us write the results:
\[
|\frac{|g|^{n}J_{n}\left(\alpha\right)}{n!}\leq g^{n}G\left(0\right)^{\frac{n\alpha}{2}}\frac{1}{n!}\frac{n\pi^{\frac{n}{2}}2^{\frac{1}{2}\left(n\alpha+n\right)}\Gamma\left(n(\alpha+1)\right)^{1/2}}{\Gamma\left(\frac{n}{2}+1\right)}e^{j_{0}^{2}}(j_{0}+1),
\]
where $j_{0}=\sqrt{\sum_{a,b=1}^{n}R_{ab}^{(n)}\bar{\phi}\left(x_{a}\right)\bar{\phi}\left(x_{b}\right)}$.
Finding the asymptotic of R.H.S., we get to:
\[
\frac{|g|^{n}J_{n}\left(\alpha\right)}{n!}\leq\frac{|g|^{n}}{n^{3/2}}n^{\frac{n\alpha}{2}-n}G\left(0\right)^{\frac{n\alpha}{2}}\pi^{\frac{n-3}{2}}2^{\frac{1}{2}\left(n\alpha+3n-1\right)}(\alpha+1){}^{n(\alpha+1)-1/2}e^{(2-\alpha)n}e^{j_{0}^{2}}j_{0}=C_{n}n^{\frac{n\alpha}{2}-n}|g|^{n}G\left(0\right)^{\frac{n\alpha}{2}}e^{j_{0}^{2}}j_{0},
\]
where $C_{n}>0$ is a dimensionless constant which grows no faster than exponentially.

So we see that our series likely to converge for $\alpha<2$. So we have to move to coordinate integrals to finish the proof. For $\alpha=2$ it is also converges, but it could be proven in a much more simple
way. Namely, it is easily to calculate $\mathcal{Z}\left[j\right]$ for $\alpha=2$ exactly, which will be done further in the paper, and that expansion will converge. So, since even asymptotic expansions in the predetermined system of functions ($\{g^{n}\}_{n=0}^{\infty}$ in our case) are unique, then
these two expansions must coincide. This proves that for $\alpha=2$ our perturbation series expansion also converges.

\subsection{Coordinate Part}

The next aim is to bound the result of coordinate integration. Namely:
\[
|Z_{n}|\leq|g|^{n}\prod_{a=1}^{n}\int dx_{a}e^{-\frac{1}{2}\sum_{a,b}\overline{\phi}\left(x_{a}\right)R_{ab}^{(n)}\overline{\phi}\left(x_{b}\right)}\frac{J_{n}\left(\alpha\right)}{n!}\leq C_{n}n^{\frac{n\alpha}{2}-n}G\left(0\right)^{\frac{n\alpha}{2}}\prod_{a=1}^{n}\int dx_{a}e^{j_{0}^{2}/2}j_{0}
\]
So, recalling our bound for $j_{0}$, we obtain finally:
\[
|Z_{n}|\leq|g|^{n}\prod_{a=1}^{n}\int dx_{a}e^{-\frac{1}{2}\sum_{a,b}\overline{\phi}\left(x_{a}\right)R_{ab}^{(n)}\bar{\phi}\left(x_{b}\right)}\frac{J_{n}\left(\alpha\right)}{n!}\leq C_{n}n^{\frac{n\alpha}{2}-n}G\left(0\right)^{\frac{n\alpha}{2}}V^{n}\sqrt{\left\Vert G\right\Vert }\left(\left\Vert j\right\Vert +1\right)e^{\frac{1}{2}\left\Vert G\right\Vert \left\Vert j\right\Vert ^{2}},
\]
so the obtained series converges, and even more, converges uniformly in $j$ on every bounded set in $\mathcal{L}^{2}(X)$.

\section{Exponentiation of Series Using Meyer Cluster Expansion}

We have just obtained some converging series for generating functional $Z_{n}$. In terms of asymptotic series there is often proved ``the exponentiation of connected diagrams'', which means roughly that the generating functional $Z$ is an exponent of other regular
(in source) functional. We are going to establish the same result in our case. It will be useful since after such exponentiation we will have only the first power of a volume rather than all natural powers. This fact will simplify significantly extracting the physically
measurable quantities, which will be done in the following chapters of the paper.

\subsection{Formulation and Applicability}

We start with the following theorem: Let $(X,\Sigma,\mu)$ be a measurable space, more precisely, space with measure, where $\mu$ is, in general, a complex-valued measure. We denote as $|\mu|$ the total variation of $\mu$. The total variation $|\mu|$ is essentially the
absolute value of $\mu$. In the case, where $d\mu(y)=g(y)d\nu(y)$ and $\nu$ is a non-negative measure, $d|\mu|(y)=|g(y)|d\nu(y)$. Given a complex measurable symmetric function $\zeta$ on $X\times X$,
we introduce the partition function by:
\begin{equation}
Z=\sum_{n\geqslant0}\frac{1}{n!}\int\mathrm{d}\mu\left(y_{1}\right)\ldots\int\mathrm{d}\mu\left(y_{n}\right)\prod_{1\leqslant i<j\leqslant n}\left(1+\zeta\left(y_{i},y_{j}\right)\right).
\end{equation}
The term $n=0$ of the sum is understood to be $1$. In case of classical gas $n$ represents the number of particles.

We denote by $\mathcal{G}_{n}$ the set of all (undirected, no loops) graphs with $n$ vertices, and $\mathcal{C}_{n}\subset\mathcal{G}_{n}$ the set of connected graphs of $n$ vertices. We introduce the following combinatorial function on finite sequences $\left(x_{1},\ldots,x_{n}\right)$
in $X$ :
\begin{equation}
\varphi\left(x_{1},\ldots,x_{n}\right)=\begin{cases}
1 & \text{ if }n=1\\
\frac{1}{n!}\sum_{\Gamma\in\mathcal{C}_{n}}\prod_{(i,j)\in\Gamma}\zeta\left(x_{i},x_{j}\right) & \text{ if }n\geqslant2
\end{cases}
\end{equation}
The product is over edges of $\Gamma$. A sequence $\left(x_{1},\ldots,x_{n}\right)$ is a cluster if the graph with $n$ vertices and an edge between $i$
and $j$ whenever $\zeta\left(x_{i},x_{j}\right)\neq0$, is connected. The cluster expansion allows to express the logarithm of the partition function as a sum (or an integral) over clusters. Namely, the following
theorem holds (\cite{ruelle1999statistical}).
\begin{theorem}[Cluster expansion]
Assume that $|1+\zeta(u,y)|\leqslant1$ for all
$u,y\in X$, and that there exists a non-negative function $a$ on $X$ such that for all $u\in X$,
\begin{equation}
\int|\zeta(u,y)|\mathrm{e}^{a(y)}\mathrm{d}|\mu|(y)\leqslant a(u)
\end{equation}
 and $\int\mathrm{e}^{a(x)}\mathrm{d}|\mu|(x)<\infty$. Then the following is true:
\begin{equation}
Z=\exp\left\{ \sum_{n\geqslant1}\int\mathrm{d}\mu\left(y_{1}\right)\ldots\int\mathrm{d}\mu\left(y_{n}\right)\varphi\left(y_{1},\ldots,y_{n}\right)\right\} .
\end{equation}
Combined sum and integrals converge absolutely. Furthermore, we have for all $x_{1}\in X$:
\begin{equation}
1+\sum_{n\geqslant2}n\int\mathrm{d}|\mu|\left(y_{2}\right)\ldots\int\mathrm{d}|\mu|\left(y_{n}\right)\left|\varphi\left(y_{1},\ldots,y_{n}\right)\right|\leqslant\mathrm{e}^{a\left(y_{1}\right)}
\end{equation}
\end{theorem}
Unfortunately, we cannot apply this theorem for $Z_{\epsilon}$ itself, since there appears $\det G(x)$, spoiling the form of terms in Grand partition decomposition. So, we apply it for $Z_{\Lambda,\epsilon}$ in the form before using Plancherel's identity. Hence, we start from:
\begin{align*}
Z_{I,\Lambda,\epsilon}\left[g,\bar{\phi}\right] & =\sum_{n=0}^{\infty}\frac{1}{n!}\left\{ \prod_{a=1}^{n}\underset{d\mu\left(g_{a}\right)}{\underbrace{\int dx_{a}\left(-1\right)g\left(x_{a}\right)\int \frac{dt_{a}}{2\pi}\mathcal{F}\left[U_{\Lambda}\left[\phi\right]\right]\left(t_{a}\right)\exp\left(-it_{a}\bar{\phi}\left(x_{a}\right)-\frac{1}{2}G\left(0\right)t_{a}^{2}\right)}}\right\} \\
 & \times\prod_{a<b}^{n}\exp\left(-G\left(x_{a}-x_{b}\right)t_{a}t_{b}\right)
\end{align*}
For generality we consider the case with possibly varying coupling constant. In our case $y=(x,t)$, $1+\zeta\left(y_{a},y_{b}\right)=e^{-\frac{1}{2}G_{ab}t_{a}t_{b}}$,
and: 
\begin{equation}
\mathrm{d}\mu(x_{a},t_{a})=-dx_{a}\frac{dt_{a}}{2\pi}g\left(x_{a}\right)\mathcal{F}\left[U_{\Lambda}\left[\phi\right]\right]\left(t_{a}\right)\exp\left(it_{a}\bar{\phi}\left(x_{a}\right)-\frac{1}{2}G\left(0\right)t_{a}^{2}\right)
\end{equation}
To use the theorem one have to find the function $a(u)$ from its formulation. But in the following we will do all manipulations directly and see that in our case all transitions are valid and without explicit presenting such a function. 

\subsection{Rewriting Series in Terms of Exponent}

Hereby, we start from the expansion:
\begin{align*}
Z_{I,\Lambda}\left[j\right] & =\sum_{n=0}^{\infty}\frac{1}{n!}\left\{ \prod_{a=1}^{n}\underset{d\mu\left(g_{a}\right)}{\underbrace{\int dx_{a}\left(-1\right)g\left(x_{a}\right)\int \frac{dt_{a}}{2\pi}\mathcal{F}\left[U_{\Lambda}\left[\phi\right]\right](t_{a})\exp\left(it_{a}\bar{\phi}\left(x_{a}\right)-\frac{1}{2}G\left(0\right)t_{a}^{2}\right)}}\right\} \nonumber \\
 & \times\prod_{a<b}^{n}\exp\left(-G\left(x_{a}-x_{b}\right)t_{a}t_{b}\right)
\end{align*}
For generality we assume that $g=g(x)$. 

Let us define:
\begin{equation}
1+\exp\left(-G\left(x_{a}-x_{b}\right)t_{a}t_{b}\right)-1=1+\zeta\left(x_{a},t_{a},x_{b},t_{b}\right)
\end{equation}
One can rewrite:
\begin{align}
\prod_{\left(a,b\right)\in\Gamma}\zeta\left(x_{a},t_{a},x_{b},t_{b}\right) & =\prod_{\left(a,b\right)\in\Gamma}\left(\exp\left(-G\left(x_{a}-x_{b}\right)t_{a}t_{b}\right)-1\right)\nonumber \\
 & =\prod_{\left(a,b\right)\in\Gamma}\left(-\int_{0}^{1}ds_{ab}\left[G\left(x_{a}-x_{b}\right)t_{a}t_{b}\right]\exp\left(-s_{ab}G\left(x_{a}-x_{b}\right)t_{a}t_{b}\right)\right)\nonumber \\
 & =\prod_{\left(a,b\right)\in\Gamma}\left(\int_{0}^{1}ds_{ab}\partial_{s_{ab}}\exp\left(-s_{ab}G\left(x_{a}-x_{b}\right)t_{a}t_{b}\right)\right),
\end{align}
where we simply have rewritten differences as integrals of derivatives, using Fundamental Theorem of Calculus. 

We can substitute it to $Z_{I,\Lambda}$:
\begin{align*}
Z_{I,\Lambda}\left[j\right] & =\sum_{n=0}^{\infty}\frac{1}{n!}\left\{ \prod_{a=1}^{n}\int d\mu\left(g_{a}\right)\right\} \prod_{a<b}^{n}\left[1+\zeta\left(x_{a},t_{a},x_{b},t_{b}\right)\right]\nonumber \\
 & =\sum_{n=0}^{\infty}\frac{1}{n!}\left\{ \prod_{a=1}^{n}\int d\mu\left(g_{a}\right)\right\} \sum_{\Gamma\in\mathbb{G}_{n}}\prod_{\left(a,b\right)\in\Gamma}\zeta\left(x_{a},t_{a},x_{b},t_{b}\right)
\end{align*}
where $\mathcal{G}_{n}$ -- set of all connected graph with $\text{n}$ vertexes and $\left(a,b\right)\in\Gamma$ -- pair of vertex in $\mathcal{G}_{n}$.
We can rewrite this expression in the following way:
\begin{equation*}
\sum_{\Gamma\in\mathbb{G}_{n}}\prod_{\left(a,b\right)\in\Gamma}\zeta\left(x_{a},t_{a},x_{b},t_{b}\right)=\sum_{k=1}^{n}\frac{1}{k!}\sum_{\nu_{1},...,\nu_{k}}\sum_{\Gamma_{1},...,\Gamma_{k}}\prod_{l=1}^{k}\prod_{\left(a,b\right)\in\Gamma_{l}}\zeta\left(x_{a},t_{a},x_{b},t_{b}\right).
\end{equation*}
Here $\Gamma_{l}\in\mathcal{C}_{n,l}$ --- a set of all connected undirected graphs without loops and with total number of edges $l$, $\nu_{l}$ -- adjacency matrix. One can substitute it to $S$-matrix:
\begin{align}
Z_{I,\Lambda}\left[j\right] & =1+\sum_{n=1}^{\infty}\frac{1}{n!}\left\{ \prod_{a=1}^{n}\int d\mu\left(g_{a}\right)\right\} \sum_{k=1}^{n}\frac{1}{k!}\sum_{\nu_{1},...,\nu_{k}}\sum_{\Gamma_{1},...,\Gamma_{k}}\prod_{l=1}^{k}\prod_{\left(a,b\right)\in\Gamma_{l}}\zeta\left(x_{a},t_{a},x_{b},t_{b}\right)\nonumber \\
 & =1+\sum_{n=1}^{\infty}\frac{1}{n!}\sum_{k=1}^{n}\frac{1}{k!}\sum_{\underset{n_{1}+...+n_{k}=n}{n_{1},...,n_{k}}}\frac{n!}{n_{1}!...n_{k}!}\prod_{l=1}^{k}\left\{ \sum_{\Gamma_{l}\in\mathbb{G}_{C,n_{l}}}\int d\mu\left(g_{1}\right)...\int d\mu\left(g_{1}\right)\prod_{\left(a,b\right)\in\Gamma_{l}}\zeta\left(x_{a},t_{a},x_{b},t_{b}\right)\right\} =\nonumber \\
 & =1+\sum_{k=1}^{\infty}\frac{1}{k!}\left\{ \sum_{n=1}^{\infty}\frac{1}{n!}\sum_{\Gamma\in\mathbb{G}_{C,n}}\left\{ \prod_{a=1}^{n}\int d\mu\left(g_{a}\right)\right\} \prod_{\left(a,b\right)\in\Gamma}\zeta\left(x_{a},t_{a},x_{b},t_{b}\right)\right\} ^{k}=\nonumber \\
 & =\exp\left(\sum_{n=1}^{\infty}\frac{1}{n!}\sum_{\Gamma\in\mathbb{G}_{C,n}}\left\{ \prod_{a=1}^{n}\int d\mu\left(g_{a}\right)\right\} \prod_{\left(a,b\right)\in\Gamma}\zeta\left(x_{a},t_{a},x_{b},t_{b}\right)\right)=\exp\left(\ln\left(Z_{I,\Lambda}\left[j\right]\right)\right)
\end{align}
Let us denote by $\mathcal{G}_{\Lambda}$ the generating functional of connected diagrams with regulations $\text{\ensuremath{\Lambda}}$ and $\epsilon$.
So we have the following formula for it:
\begin{equation}
\mathcal{G}_{I,\Lambda}=\ln\left(Z_{I,\Lambda}\left[j\right]\right)=\sum_{n=1}^{\infty}\frac{1}{n!}\sum_{\Gamma\in\mathbb{G}_{C,n}}\left\{ \prod_{a=1}^{n}\int d\mu\left(g_{a}\right)\right\} \prod_{\left(a,b\right)\in\Gamma}\zeta\left(x_{a},t_{a},x_{b},t_{b}\right),
\end{equation}
where, recall, $\mathcal{G}=\mathcal{G}_{I}+\frac{1}{2}\int dxdy\ G(x-y)j(x)j(y)$.
We can rewrite this expression in the following way:
\begin{align*}
\mathcal{G}_{I} & =\sum_{n=1}^{\infty}\frac{1}{n!}\sum_{\Gamma\in\mathbb{G}_{C,n}}\left\{ \prod_{a=1}^{n}\int d\mu\left(g_{a}\right)\right\} \prod_{\left(a,b\right)\in\Gamma}\zeta\left(x_{a},t_{a},x_{b},t_{b}\right)\nonumber \\
 & =\sum_{n=1}^{\infty}\mathcal{G}_{I,n}
\end{align*}
where:
\begin{align}
\mathcal{G}_{I,n} & =\frac{1}{n!}\sum_{\Gamma\in\mathbb{G}_{C,n}}\left\{ \prod_{a=1}^{n}\int d\mu\left(g_{a}\right)\right\} \prod_{\left(a,b\right)\in\Gamma}\zeta\left(x_{a},t_{a},x_{b},t_{b}\right)\nonumber \\
 & =\frac{1}{n!}\sum_{\Gamma\in\mathbb{G}_{C,n}}\left\{ \prod_{a=1}^{n}\int d\mu\left(g_{a}\right)\right\} \left\{ \prod_{\left(a,b\right)\in\Gamma}\int_{0}^{1}ds_{ab}\partial_{s_{ab}}\right\} \exp\left(-\sum_{\left(a,b\right)\in\Gamma}s_{ab}G\left(x_{a}-x_{b}\right)t_{a}t_{b}\right)\nonumber \\
 & =\frac{1}{n!}\sum_{\Gamma\in\mathbb{G}_{C,n}}\left\{ \prod_{\left(a,b\right)\in\Gamma}\int_{0}^{1}ds_{ab}\partial_{s_{ab}}\right\} \left\{ \prod_{a=1}^{n}\int d\mu\left(g_{a}\right)\right\} \exp\left(-\sum_{\left(a,b\right)\in\Gamma}s_{ab}G\left(x_{a}-x_{b}\right)t_{a}t_{b}\right)
\end{align}
is the contribution of all connected graph with $n$ vertexes. One can introduce adjacency matrix of a given graph $\Gamma$:
\begin{equation}
\nu_{ab}\left(\Gamma\right)=\begin{cases}
1 & \left(a,b\right)\in\Gamma\\
0 & \left(a,b\right) \notin \Gamma
\end{cases}
\end{equation}
Hence we have:
\begin{equation*}
\mathcal{G}_{I,n}=\frac{1}{n!}\sum_{\Gamma\in\mathbb{G}_{C,n}}\left\{ \prod_{a<b}^{n}\int_{0}^{1}ds_{ab}\partial_{s_{ab}}^{\nu_{ab}\left(\Gamma\right)}\right\} \left\{ \prod_{a=1}^{n}\int d\mu\left(g_{a}\right)\right\} \exp\left(-\sum_{a<b}^{n}\nu_{ab}\left(\Gamma\right)s_{ab}G\left(x_{a}-x_{b}\right)t_{a}t_{b}\right)
\end{equation*}
where we placed $\nu_{ba}\left(\Gamma\right)s_{ba}G\left(x_{b}-x_{a}\right)=\nu_{ab}\left(\Gamma\right)s_{ab}G\left(x_{a}-x_{b}\right)$
for $b>a$ and denoted $\tilde{G}_{ab}=\nu_{ab}\left(\Gamma\right)s_{ab}G\left(x_{a}-x_{b}\right)$.
Finally we have:
\begin{align*}
\mathcal{G}_{I,n} & =\frac{1}{n!}\sum_{\Gamma\in\mathbb{G}_{C,n}}\left\{ \prod_{a<b}^{n}\int_{0}^{1}ds_{ab}\partial_{s_{ab}}^{\nu_{ab}\left(\Gamma\right)}\right\}\\ 
&
\left\{ \prod_{a=1}^{n}\int dx_{a}\left(-1\right)g\left(x_{a}\right)\int \frac{dt_{a}}{2\pi}\mathcal{F}\left[U_{\Lambda}\left[\phi\right]\right](t_{a})\exp\left(it_{a}\bar{\phi}\left(x_{a}\right)-\frac{1}{2}G\left(0\right)t_{a}^{2}\right)\right\} \nonumber \\
 & \times\exp\left(-\sum_{a<b}^{n}\nu_{ab}\left(\Gamma\right)s_{ab}G\left(x_{a}-x_{b}\right)t_{a}t_{b}\right)
\end{align*}
or, after applying Plancherel's theorem: 
\begin{align}
\mathcal{G}_{I,n} & =\frac{1}{n!}\sum_{\Gamma\in\mathbb{G}_{C,n}}\left\{ \prod_{a<b}^{n}\int_{0}^{1}ds_{ab}\partial_{s_{ab}}^{\nu_{ab}\left(\Gamma\right)}\right\} \left\{ \left(-1\right)^{n}\left\{ \prod_{a=1}^{n}\int dx_{a}g\left(x_{a}\right)\right\} \frac{1}{\sqrt{(2\pi)^{n}\det\left(G\right)}}\right.\nonumber \\
 & \left.\times\left[\prod_{a=1}^{n}\int d\phi_{a}U_{\Lambda}\left(\phi_{a}\right)\right]\exp\left(-\frac{1}{2}\sum_{a,b=1}^{n}\left(\tilde{G}_{ab}\right)^{-1}\left(\phi(x_{a})-\bar{\phi}(x_{a})\right)\left(\phi(x_{b})-\bar{\phi}(x)\right)\right)\right\} 
\end{align}
Though this formula is not a very convenient one since $s_{ab}$ variables are included in a very complicated way because of the inverse matrix, this decomposition is a very important one because it explains why it is sufficient to consider the terms in the initial
(unexponentiated) form, and then to get $\mathcal{G}_{\Lambda,n}$
we have only to act on every term by operator $\sum_{\Gamma\in\mathbb{G}_{C,n}}\left\{ \prod_{a<b}^{n}\int_{0}^{1}ds_{ab}\partial_{t_{ab}}^{\nu_{ab}\left(\Gamma\right)}\right\}$.
So one can manipulate with the terms in the unexponentiated grand canonical partition function and then apply all the results to the generating
functional of connected diagrams by simple changing:
\begin{equation}
G_{ab}=G\left(x_{a}-x_{b}\right)\mapsto\tilde{G}_{ab}=\nu_{ab}\left(\Gamma\right)s_{ab}G\left(x_{a}-x_{b}\right),
\end{equation}
in the \textbf{quadratic} part of the action and   posterior action of the operator $\sum_{\Gamma\in\mathbb{G}_{C,n}}\left\{ \prod_{a<b}^{n}\int_{0}^{1}ds_{ab}\partial_{t_{ab}}^{\nu_{ab}\left(\Gamma\right)}\right\}$, remaining the classical fields $\bar{\phi}$ unchanged. We will use
this consideration to avoid agglomerations of notations. 

Generally, one can write, assuming that $\nu_{aa}=0$, which is equivalent to our already existing requisition for graph to have no loops:
\begin{equation}
\mathcal{G}_{I,n}[j(x),G_{ab}]=\sum_{\Gamma\in\mathbb{G}_{C,n}}\left\{ \prod_{a<b}^{n}\int_{0}^{1}ds_{ab}\partial_{s_{ab}}^{\nu_{ab}\left(\Gamma\right)}\right\} Z_{I,n}[\bar{\phi}(x),\nu_{ab}\left(\Gamma\right)s_{ab}G_{ab}],
\end{equation}
since the part of $\mathcal{G}_{I,n}$ in the right hand side $\sum_{\Gamma\in\mathbb{G}_{C,n}}\left\{ \prod_{a<b}^{n}\int_{0}^{1}ds_{ab}\partial_{t_{ab}}^{\nu_{ab}\left(\Gamma\right)}\right\}$ coincides with $Z_{I,n}$ with the replacements described above. Here,
recall: 
\[
Z_{I}=\sum_{n=0}^{\infty}Z_{I,n}\qquad Z=\sum_{n=0}^{\infty}Z_{n},
\]
Moreover, for the $n$-particle functions one can also write from the considerations of homogeneity in $j(x)$:
\begin{equation}
\mathcal{G}_{I,1,...,k;\ n}^{(k)}[G_{ab}]=\sum_{\Gamma\in\mathbb{G}_{C,n}}\left\{ \prod_{a<b}^{n}\int_{0}^{1}ds_{ab}\partial_{s_{ab}}^{\nu_{ab}\left(\Gamma\right)}\right\} \mathcal{D}_{I,1,...,k;\ n}^{(k)}[\nu_{ab}\left(\Gamma\right)s_{ab}G_{ab}],
\end{equation}
where, again, the substitution of $\nu_{ab}\left(\Gamma\right)s_{ab}G_{ab}$
is done in every entering of $G^{(n)}$-matrix in the formula, but not in the $G(x-y)$ in $\bar{\phi}$. 

One can also notice the form of $\mathcal{G}_{I,n}$ without introducing new variables $s_{ab}$, which will be useful for the following hard-sphere gas approximation:
\begin{align}
\mathcal{G}_{I,n} & =\frac{1}{n!}\sum_{\Gamma\in\mathbb{G}_{C,n}}\left\{ \prod_{a=1}^{n}\int dx_{a}\left(-1\right)g\left(x_{a}\right)\int \frac{dt_{a}}{2\pi}\mathcal{F}\left[U_{\Lambda}\left[\phi\right]\right](t_{a})\exp\left(it_{a}\bar{\phi}\left(x_{a}\right)-\frac{1}{2}G\left(0\right)t_{a}^{2}\right)\right\} \nonumber \\
 & \times\prod_{a<b}\left(\exp\left(-\nu_{ab}\left(\Gamma\right)s_{ab}G\left(x_{a}-x_{b}\right)t_{a}t_{b}\right)-1\right),
\end{align}
which can be obtained directly from $(62)$ with substituting $\zeta\left(x_{a},t_{a},x_{b},t_{b}\right)=\exp\left(-G\left(x_{a}-x_{b}\right)t_{a}t_{b}\right)-1$
or from the final formula after calculating the integrals over $s_{ab}$.

\subsection{A Short Way to Obtain Coefficients of Exponentiation}

One can notice that if we proved the decomposition of a form $Z=e^{V\cdot f(G,g,j)}$, then we certainly have:
\begin{equation}
Z=e^{Vf(G,g,j)}=1+Vf(G,g,j)+\frac{1}{2}V^{2}f(G,g,j)^{2}+...
\end{equation}
So, the thing is we can extract $f(G,g,j)$ as the coefficient in $V$ if we get some expansion of $Z$ in powers of $V.$ This follows from the uniqueness of (asymptotic) decomposition in prescribed system of functions, namely $\{V^{k}\}_{k=1}^{\infty}$ in our case. We will use this outlook in the following computations. 

\part{Calculation of Perturbation Terms}

One can easily mark that integrals of the form:
\begin{equation}
\int dt_{1}...dt_{n}\left|t_{1}\right|^{\alpha}...\left|t_{n}\right|^{\alpha}e^{-\frac{1}{2}\sum_{a,b=1}^{n}R_{ab}^{\left(n\right)}t_{a}t_{b}+\sum_{a=1}^{n}t_{a}\chi\left(x_{a}\right)}
\end{equation}
are the particular cases of so-called Gelfand hypergeometric functions (\cite{GELFAND}). Unfortunately, they are too generic and their properties are excessively complicated. So if we are going to get some practical results it is pointless to express anything via them.

Besides, this integrals are enough sophisticated to be expressed simply in terms of even generalized Gauss and Lauricella hypergeometric functions. 

\section{Off-Diagonal Terms Expansion}

The first and most evident way is to try to expand quadratic exponent in its off-diagonal terms, and then compute the obtained integrals directly. However, this approach also fails. Let us demonstrate it. We start from the expansion:
\begin{align}
Z_{I}\left[j\left(x\right)\right] & =\sum_{n=0}^{\infty}\frac{\left(-g\right)^{n}}{n!\left(2\pi\right)^{n/2}}\left\{ \prod_{a=1}^{n}\int dx_{a}e^{-\frac{1}{2}\sum_{a,b}\overline{\phi}\left(x_{a}\right)\left(G^{(n)}\right)_{ab}^{-1}\overline{\phi}\left(x_{b}\right)}\int d\phi_{a}\left|\phi_{a}\right|^{\alpha}\right\} \nonumber \\
 & \times\frac{e^{-\frac{1}{2}\sum\phi_{a}\left(G^{(n)}\right)_{ab}^{-1}\phi_{b}+\sum_{a=1}^{n}\phi_{a}\chi\left(x_{a}\right)}}{\sqrt{\det\left(G^{(n)}\right)}}
\end{align}
and substitute the series:
\begin{equation}
e^{-\sum_{a<b}\left(G^{(n)}\right)_{ab}^{-1}\phi_{a}\phi_{b}}=\left(\prod_{a<b}\sum_{l_{ab}=1}^{\infty}\right)(-1)^{\sum_{a<b}l_{ab}}\left(\prod_{a<b}\frac{\left\{ \left(G^{(n)}\right)_{ab}^{-1}\right\} ^{l_{ab}}}{l_{ab}!}\right)\prod_{a=1}^{n}\phi_{a}^{\sum_{b|b\neq a}l_{ab}},
\end{equation}
then we have, computing the integrals by $\phi_{a}$:
\begin{align*}
Z_{I}\left[j\left(x\right)\right] & =\sum_{n=0}^{\infty}\frac{\left(-g\right)^{n}}{n!\left(2\pi\right)^{n/2}}\left(\prod_{a<b}^{n}\sum_{l_{ab}=1}^{\infty}\right)\left(-\sqrt{\frac{2}{G(0)}}\right){}^{n+1+\sum_{a,b|a<b}l_{ab}}\\
 & \times\left\{ \prod_{a=1}^{n}\int\frac{dx_{a}}{\sqrt{\det\left(G^{(n)}\right)}}\left(\prod_{a<b}\frac{\left\{ \left(G^{(n)}\right)_{ab}^{-1}\right\} ^{l_{ab}}}{l_{ab}!}\right)e^{-\frac{1}{2}\sum_{a,b}\overline{\phi}\left(x_{a}\right)\left(G^{(n)}\right)_{ab}^{-1}\overline{\phi}\left(x_{b}\right)}\right\} \\
 & \times\prod_{a=1}^{n}\left(\sum_{s_{a}=0}^{\infty}\frac{\overline{\phi}\left(x_{a}\right)^{s_{a}}}{s_{a}!}\left((-1)^{\sum_{b|a<b}l_{ab}+s_{a}}+1\right)\Gamma\left(\frac{1}{2}\left(\sum_{b|a<b}l_{ab}+s_{a}+\alpha+1\right)\right)\right)
\end{align*}
Expanding this expression in powers of $\overline{\phi}\left(x_{a}\right)^{s_{a}}$,
one can get formulas for correlators. Though, it is not very efficient way, since it is not very clear how to compute coordinate integrals. That is why we developed several other treatments below.

Finally, it is useful to write explicitly a case of $j=0$, which means the exponent of vacuum energy:
\begin{equation}
Z_{I}=\sum_{n=0}^{\infty}\frac{\left(-g\right)^{n}}{n!\left(2\pi\right)^{n/2}}\left(\prod_{a<b}^{n}\sum_{l_{ab}=1}^{\infty}\right)\left(-\sqrt{\frac{2}{G(0)}}\right){}^{n+1+\sum_{a,b|a<b}l_{ab}}\prod_{a=1}^{n}\int\frac{dx_{a}}{\sqrt{\det\left(G^{(n)}\right)}}\prod_{a<b}\frac{\left\{ \left(G^{(n)}\right)_{ab}^{-1}\right\} ^{l_{ab}}}{l_{ab}!}
\end{equation}
This expression is much more compact, but it is still to complicated for analytical calculations. That is the main reason one head to develop other computational techniques for practical calculating some concrete
quantities. 

\section{Approximation of Generic Term via Polynomials}

In this section we will approximate prefactor $\left|t_{1}\cdot...\cdot t_{n}\right|^{\alpha}$,
as a function $f(t):=|t|^{\alpha}$ for $t=t_{1}\cdot...\cdot t_{n}$
by polynomials. There are a couple of constructive ways, and we compare several of them. 

\subsection{Background on Constructive Approximation by Polynomials}

\subsubsection{Bernstein Polynomials}

There are $n+1$ Bernstein basis polynomials of degree
$n$. They are defined as follows:
\begin{equation}
b_{\nu,n}(x)=\left(\begin{array}{l}
n\\
\nu
\end{array}\right)x^{\nu}(1-x)^{n-\nu},\quad\nu=0,\ldots,n
\end{equation}
These Bernstein basis polynomials form a basis for the
vector space of polynomials of degree at most $n$ with real coefficients. A linear combination of Bernstein basis polynomials: 
\begin{equation}
B_{n}(x)=\sum_{\nu=0}^{n}\beta_{\nu}b_{\nu,n}(x)
\end{equation}
with some coefficients $\beta_{\nu}$ is called Bernstein polynomial. The following theorem holds (\cite{lorentz2013bernstein}):
\begin{theorem}
Let $f$ be a continuous function on the interval $[0,1]$. Consider the Bernstein polynomial:
\begin{equation}
B_{n}(f)(x)=\sum_{\nu=0}^{n}f\left(\frac{\nu}{n}\right)b_{\nu,n}(x),
\end{equation}
Than, uniformly on $[0,1]$:
\begin{equation}
\lim_{n\rightarrow\infty}B_{n}(f)=f
\end{equation}
\end{theorem}
These polynomials are used to obtain a constructive proof for the Weierstrass Approximation Theorem.

\subsubsection{Chebyshev Polynomials}

The Chebyshev polynomials of the first kind $T_{n}$ are defined by:
\begin{equation}
T_{n}(\cos\theta)=\cos(n\theta)
\end{equation}
There is an explicit formula for them:
\begin{equation}
T_{n}(x)=\frac{n}{2}\sum_{k=0}^{\left[\frac{n}{2}\right]}(-1)^{k}\frac{(n-k-1)!}{k!(n-2k)!}(2x)^{n-2k},
\end{equation}
The Chebyshev polynomials of the first kind $T_{n}$ constitute a complete orthogonal system with respect to weight $w(x)=\frac{1}{\sqrt{1-x^{2}}}$
in $\mathcal{L}^{2}([-1;1],w(x)dx)$:
\begin{equation}
\int_{-1}^{1}T_{n}(x)T_{m}(x)\frac{\mathrm{d}x}{\sqrt{1-x^{2}}}=\begin{cases}
0 & \text{ if }n\neq m,\\
\pi & \text{ if }n=m=0,\\
\frac{\pi}{2} & \text{ if }n=m\neq0.
\end{cases}
\end{equation}
However, the important fact is that for bounded periodic
$C^{1}(-1;1)$ function expansion in Chebyshev polynomials:
\begin{equation}
\sum_{n=0}^{\infty}c_{n}T_{n}(x)\rightrightarrows_{[-1;1]}f(x),
\end{equation}
converges uniformly (\cite{szego1939orthogonal}). Here the coefficients $c_{n}$ are determined by the formulas:
\[
c_{0}=\frac{1}{\pi}\int_{-1}^{1}f(x)\frac{\mathrm{d}x}{\sqrt{1-x^{2}}},\qquad c_{n>0}=\frac{2}{\pi}\int_{-1}^{1}f(x)T_{n}(x)\frac{\mathrm{d}x}{\sqrt{1-x^{2}}},
\]
which are no more then usual projection on basis vectors in $\mathcal{L}^{2}([-1;1],w(x)dx)$.

\subsubsection{Legendre Polynomials}

Legendre polynomials are defined as an orthogonal system with respect to the weight function $w(x)=1$ on the interval $[-1,1]$. That is, $P_{n}(x)$ is a polynomial of degree $n$, such that: 
\begin{equation}
\int_{-1}^{1}P_{m}(x)P_{n}(x)dx=0\quad\text{ if }n\neq m
\end{equation}
The standardization $P_{n}(1)=1$ fixes the normalization of the Legendre polynomials with respect to the $\mathcal{L}^{2}$ norm on the interval $[-1;1]$. There is an explicit formula for Legendre polynomials:
\begin{equation}
P_{n}(x)=2^{n}\sum_{k=0}^{n}x^{k}\left(\begin{array}{l}
n\\
k
\end{array}\right)\left(\begin{array}{c}
\frac{n+k-1}{2}\\
n
\end{array}\right),
\end{equation}
and they obey:
\begin{equation}
\int_{-1}^{1}P_{m}(x)P_{n}(x)dx=\frac{2}{2n+1}\delta_{mn}
\end{equation}

Again, for the bounded periodic function from $C^{1}(-1;1)$ the expansion in Legendre polynomials:
\begin{equation}
\sum_{n=0}^{\infty}c_{n}P_{n}(x)\rightrightarrows_{[-1;1]}f(x),
\end{equation}
converges uniformly (\cite{szego1939orthogonal}). Here coefficients $c_{n}$ are determined by the formulas: 
\[
c_{0}=\frac{1}{2}\int_{-1}^{1}f(x)dx,\qquad c_{n>0}=\frac{2n+1}{2}\int_{-1}^{1}f(x)P_{n}(x)\mathrm{d}x,
\]

\subsection{Construction of Approximation for the Terms}

\subsubsection{Preliminary Transformations}

Unfortunately, one can approximate function by Bernstein polynomials on compacts only. Moreover, if we try to approximate (even point-wise) a function $\left|t\right|^{\alpha}$, $t=t_{1}\cdot...\cdot t_{n}$
for all $t>0$ via any kind of polynomials (e.g. Hermite or Laguerre polynomials, which are convenient since our function belongs to all corresponding $\mathcal{L}^{2}$-spaces) we obtain a diverging series after integration. And the thing is there is a problem in radial direction. As a possible solution, before approximation by polynomials one should evaluate the integral in radial direction (over radial variable $r$). In the remaining integral over hypersphere we can approximate $\left|\xi\right|^{\alpha}$, where $\xi$ -- some function of $t_{a}$, by polynomials in $t$. Here will be no such a problem since on a hypersphere $\xi\in[-M;M]$ for some fixed $M$, so all the approximations and expansions will converge much better. And after such an approximation one can go back to $\mathbb{R}^{n}$ where it is easier to calculate the integrals using the source trick. We will underline all enterings of $s_{ab}\nu_{ab}$ in the formulas by explicit writing of $\tilde{G}_{ab}$ rather than $G_{ab}$

So, let us define quadratic form:
\begin{equation}
Q(\phi,\phi)=\frac{1}{2}\sum_{a,b=1}^{n}\tilde{G}_{ab}^{-1}\phi_{a}\phi_{b}
\end{equation}
and linear function of $\phi$:
\begin{equation}
l\left(\phi\right)=\sum_{a,b=1}^{n}\tilde{G}_{ab}^{-1}\phi(x_{a})\bar{\phi}(x_{b})=\sum_{a=1}^{n}\phi(x_{a})\chi_{a},\qquad\chi_{a}=\sum_{b=1}^{n}\tilde{G}_{ab}^{-1}\bar{\phi}(x_{b})
\end{equation}
Basically, $\chi_{a}$ are the sources in the coordinate representation, which are acted by the operator $\hat{G}$, after which we ``pull them back'' by the inverse to the discrete restriction $G^{(n)}$. Here we also suppose $g=g(x)$ for more generality in this subsection.

Hence we have:
\begin{align}
\mathcal{G}_{I,n} & =\frac{1}{n!}\sum_{\Gamma\in\mathbb{G}_{C,n}}\left\{ \prod_{a<b}^{n}\int_{0}^{1}ds_{ab}\partial_{s_{ab}}^{\nu_{ab}\left(\Gamma\right)}\right\} \left\{ \left(-1\right)^{n}\left\{ \prod_{a=1}^{n}\int dx_{a}g\left(x_{a}\right)\right\} \frac{1}{\sqrt{(2\pi)^{n}\det\left(G\right)}}\right.\\
 & \left.\times\exp\left(-Q\left(\bar{\phi},\bar{\phi}\right)\right)\left\{ \prod_{a=1}^{n}\int d\phi_{a}U_{\Lambda}\left(\phi_{a}\right)\right\} \exp\left(-Q\left(\phi,\phi\right)+l\left(\phi\right)\right)\right\} 
\end{align}
Let us consider:
\begin{equation}
I_{n}=\left\{ \prod_{a=1}^{n}\int d\phi_{a}U_{\Lambda}\left(\phi_{a}\right)\right\} \exp\left(-Q\left(\phi,\phi\right)+l\left(\phi\right)\right).
\end{equation}
This integrand has the symmetry under $\phi\to-\phi$ cause $Q$ and
$U$ are even functions, hence:
\begin{align*}
I_{n} & =\left\{ \prod_{a=1}^{n}\int d\phi_{a}U_{\Lambda}\left(\phi_{a}\right)\right\} \exp\left(-Q\left(\phi,\phi\right)+l\left(\phi\right)\right)\\
 & =\left\{ \prod_{a=1}^{n}\int d\phi_{a}U_{\Lambda}\left(\phi_{a}\right)\right\} \exp\left(-Q\left(\phi,\phi\right)-l\left(\phi\right)\right)
\end{align*}
and:
\begin{align*}
I_{n} & =\left\{ \prod_{a=1}^{n}\int d\phi_{a}U_{\Lambda}\left(\phi_{a}\right)\right\} \exp\left(-Q\left(\phi,\phi\right)\right)\frac{\exp\left(l\left(\phi\right)\right)+\exp\left(-l\left(\phi\right)\right)}{2}\\
 & =\left\{ \prod_{a=1}^{n}\int d\phi_{a}U_{\Lambda}\left(\phi_{a}\right)\right\} \exp\left(-Q\left(\phi,\phi\right)\right)\cosh\left(l\left(\phi\right)\right)
\end{align*}
In this integral we can introduce hyperspherical coordinates, take the limit $\Lambda\to\infty$ and substitute $U=\left|\phi\right|^{\alpha}$:
\begin{equation}
I_{n}=\int_{0}^{\infty}dr\int_{S^{n-1}}d\Omega r^{n-1+n\alpha}\left|\nu_{1}...\nu_{n}\right|^{\alpha}\exp\left(-r^{2}Q\left(\nu,\nu\right)\right)\cosh\left(rl\left(\nu\right)\right)
\end{equation}
where we use the fact that we can choose $\phi_{a}=rs_{a}$ where $1_{a}\in S^{n-1}$,
and integration over radius gives:
\begin{align}
I_{n} & =\int_{0}^{\infty}dr\int_{S^{n-1}}d\Omega r^{n-1+n\alpha}\left|\nu_{1}...\nu_{n}\right|^{\alpha}\exp\left(-r^{2}Q\left(\nu,\nu\right)\right)\cosh\left(rl\left(\nu\right)\right)\\
 & =\frac{1}{2}\int_{S^{n-1}}d\Omega\left|\nu_{1}...\nu_{n}\right|^{\alpha}\frac{\Gamma\left(\frac{n(\alpha+1)}{2}\right)}{Q\left(\nu,\nu\right)^{\frac{1}{2}(\alpha+1)n}}{}_{1}F_{1}\left(\frac{n(\alpha+1)}{2};\frac{1}{2};\frac{l\left(\nu\right)^{2}}{4Q\left(\nu,\nu\right)}\right)
\end{align}
Then we expand: $_{1}F_{1}$:
\begin{equation}
I_{n}=\frac{1}{2}\sum_{k=0}^{\infty}\frac{1}{4^{k}k!}\frac{\left(\frac{n(\alpha+1)}{2}\right)_{2k}}{\left(1/2\right)_{2k}}\Gamma\left(\frac{n(\alpha+1)}{2}\right)\int_{S^{n-1}}\frac{d\Omega}{Q\left(\nu,\nu\right)^{\frac{n}{2}}}\frac{\left|\nu_{1}...\nu_{n}\right|^{\alpha}}{Q\left(\nu,\nu\right)^{\frac{\alpha n}{2}}}\frac{l\left(\nu\right)^{2k}}{Q\left(\nu,\nu\right)^{k}},
\end{equation}
One can easily notice that it is equivalent to the expansion of $\cosh$ in formula $(100)$. As in the section with majorant calculation, we can bound the following relation, using the properties of $G^{(n)}$ matrix:
\[
0\leq\frac{\left|\nu_{1}...\nu_{n}\right|}{Q\left(\nu,\nu\right)^{\frac{n}{2}}}\leq2^{n/2}G(0)^{n/2}\frac{1/n^{n/2}}{1/n^{n/2}}=2^{n/2}G(0)^{n/2}
\]
for all $x_{a}$. Then we consider a function:
\begin{equation}
f(t)=|t|^{\alpha},\qquad t=\frac{1}{2^{n/2}G(0)^{n/2}}\frac{\left|\nu_{1}...\nu_{n}\right|}{Q\left(\nu,\nu\right)^{\frac{n}{2}}},
\end{equation}
where we have introduced the normalized argument to approximate it by some kind of polynomials. So we write for finite-terms approximations using that $f$ is even:
\begin{equation}
f_{N}(t)=\sum_{q=0}^{N}c_{q}(f)P_{q}(t),
\end{equation}
where $\{P_{q}(t)\}_{q=0}^{\infty}$ is the described complete set of polynomials and $P_{q}(t)=\sum_{i=0}^{q}u_{i}t^{2i}$. We will suppose that all $P_{q}(t)$ are even. Coefficients $c_{q}(f)$ can also depend on $N$. For the case of Bernstein polynomials substituting it into the integral we obtain:
\[
I_{n}=\frac{1}{2}\sum_{k=0}^{\infty}\frac{\left(2G(0)\right)^{n\alpha/2}}{4^{k}k!}\frac{\left(\frac{n(\alpha+1)}{2}\right)_{2k}}{\left(1/2\right)_{2k}}\Gamma\left(\frac{n(\alpha+1)}{2}\right)\sum_{q=0}^{N}c_{q}(f)\int_{S^{n-1}}\frac{d\Omega}{Q\left(\nu,\nu\right)^{\frac{n}{2}}}P_{q}\left(\frac{\left|\nu_{1}...\nu_{n}\right|}{2^{n/2}G(0)^{n/2}Q\left(\nu,\nu\right)^{\frac{n}{2}}}\right)\frac{l\left(\nu\right)^{2k}}{Q\left(\nu,\nu\right)^{k}}.
\]
Recalling the identity:
\[
\int_{0}^{\infty}e^{-r^{2}}r^{n+2s-1}dr=\frac{1}{2}\Gamma\left(\frac{n}{2}+s\right),
\]
we can return to $\mathbb{R}^{n}$ due to homogeneity:
\[
I_{n}=\sum_{k=0}^{\infty}\frac{\left(2G(0)\right)^{n\alpha/2}}{4^{k}k!}\frac{\left(\frac{n(\alpha+1)}{2}\right)_{2k}}{\left(1/2\right)_{2k}}\sum_{q=0}^{N}\sum_{i=0}^{q}u_{i}c_{q}(f)\frac{\Gamma\left(\frac{n(\alpha+1)}{2}\right)}{2^{ni}G(0)^{ni}\Gamma\left(\frac{n}{2}+n(i+k)\right)}\int \left(\nu_{1}...\nu_{n}\right)^{2i}l\left(\phi\right)^{2k}e^{-Q\left(\phi,\phi\right)}d\phi_{1}...d\phi_{n}.
\]
It is worth noting that without $\frac{\Gamma\left(\frac{n(\alpha+1)}{2}\right)}{\Gamma\left(\frac{n}{2}+n(i+k)\right)}$ we would receive $\cosh$ after summation instead
of confluent hypergeometric function $_{1}F_{1}$. 

We can rewrite, using multinomial formula:
\[
l\left(\nu\right)^{2k}=\sum_{a_{1}...a_{2k}}^{n}\chi_{a_{1}}...\chi_{a_{2k}}s_{a_{1}}...s_{a_{2k}}=
\]
\[
=\sum_{\nu_{1}+...+\nu_{n}=2k}\binom{n}{\nu_{1}.....\nu_{n}}\chi_{1}^{\nu_{1}}...\chi_{n}^{\nu_{n}}s_{1}^{\nu_{1}}...s_{n}^{\nu_{n}}
\]
After that, introducing the new source $\eta$, we obtain:
\begin{align}
I_{n} & =\sqrt{(2\pi)^{n}\det\left(G\right)}\sum_{k=0}^{\infty}\frac{\left(2G(0)\right)^{n\alpha/2}}{4^{k}k!}\frac{\left(\frac{n(\alpha+1)}{2}\right)_{2k}}{\left(1/2\right)_{2k}}\sum_{q=0}^{N}\sum_{i=0}^{q}\sum_{\nu_{1}+...+\nu_{n}=2k}\binom{n}{\nu_{1}.....\nu_{n}}u_{i}c_{q}(f)\\
 & \times\frac{\Gamma\left(\frac{n(\alpha+1)}{2}\right)}{2^{ni}G(0)^{ni}\Gamma\left(\frac{n}{2}+n(i+k)\right)}\chi_{1}^{\nu_{1}}...\chi_{n}^{\nu_{n}}\partial_{1}^{\nu_{1}+2i}...\partial_{n}^{\nu_{n}+2i}\bigg|_{\eta=0}e^{-\frac{1}{2}\sum_{a,b=1}^{n}G_{ab}^{(n)}\eta_{a}\eta_{b}},
\end{align}
where $\partial_{i}:=\frac{\partial}{\partial\eta_{i}}$.

Finally, for terms in generating functional of connected diagrams we get:
\begin{align}
\mathcal{G}_{I,n}[j]_{N} & =\frac{\left(-1\right)^{n}}{n!}\sum_{\Gamma\in\mathbb{G}_{C,n}}\left\{ \prod_{a<b}^{n}\int_{0}^{1}ds_{ab}\partial_{s_{ab}}^{\nu_{ab}\left(\Gamma\right)}\right\} \left\{ \prod_{a=1}^{n}\int dx_{a}g\left(x_{a}\right)\right\} \exp\left(-Q\left(\bar{\phi},\bar{\phi}\right)\right)\\
 & \times\sum_{k=0}^{\infty}\frac{\left(2G(0)\right)^{n\alpha/2}}{4^{k}k!}\frac{\left(\frac{n(\alpha+1)}{2}\right)_{2k}}{\left(1/2\right)_{2k}}\sum_{\nu_{1}+...+\nu_{n}=2k}\binom{n}{\nu_{1}.....\nu_{n}}\chi_{1}^{\nu_{1}}...\chi_{n}^{\nu_{n}}\sum_{q=0}^{N}\sum_{i=0}^{q}u_{i}c_{q}(f)\\
 & \times\frac{\Gamma\left(\frac{n(\alpha+1)}{2}\right)}{2^{ni}G(0)^{ni}\Gamma\left(\frac{n}{2}+n(i+k)\right)}\partial_{1}^{\nu_{1}+2i}...\partial_{n}^{\nu_{n}+2i}\bigg|_{\eta=0}e^{-\frac{1}{2}\sum_{a,b=1}^{n}s_{ab}\nu_{ab}G_{ab}^{(n)}\eta_{a}\eta_{b}}
\end{align}
Recall that here $s_{ab}\nu_{ab}$ enters in:
\begin{enumerate}
\item $Q\left(\bar{\phi},\bar{\phi}\right)$ in the exponent;
\item $\chi_{a}$ in the prefactor;
\item $\sum_{a,b=1}^{n}s_{ab}\nu_{ab}G_{ab}^{(n)}
\eta_{a}\eta_{b}$ in the
exponent.
\end{enumerate}
This will be very useful for the following calculations.

\subsubsection{Combinatorial Expansion}

We want to derive analytical formulas for the quantity $\partial_{1}^{\nu_{1}+2i}...\partial_{n}^{\nu_{n}+2i}\bigg|_{\eta=0}e^{-\frac{1}{2}\sum_{a,b=1}^{n}G_{ab}\eta_{a}\eta_{b}}$.
It is not hard to recognize here usual Gaussian moments $\left\langle \varphi_{1}^{2k+c_{1}}...\varphi_{i}^{2k+c_{i}}...\varphi_{n}^{2k+c_{n}}\right\rangle _{0}$
with respect to Gaussian distribution with matrix $G^{(n)}$. In QFT notations, we use the index ``0'' to underline that it is a correlator in Gaussian theory with matrix $G^{(n)}$, which means there is no any
non-Gaussian part.

It is well known Wick's theorem (Isserlis' theorem) that they are expressed as a following sum:
\begin{equation}
\partial_{1}^{\nu_{1}+2i}...\partial_{n}^{\nu_{n}+2i}\bigg|_{\eta=0}e^{-\frac{1}{2}\sum_{a,b=1}^{n}G_{ab}\eta_{a}\eta_{b}}=\sum_{\text{all pairings}a_{i}a_{i+1}}G_{a_{1}a_{2}}\cdot...\cdot G_{a_{s-1}a_{s}},
\end{equation}
for even $s$ and are equal to zero for odd $s$, where $s=2ni+\sum_{a=1}^{n}\nu_{n}$. So now we are going to calculate the number of all equal terms in this sum. 

Let us calculate Gaussian theory correlation functions. One can calculate it by Wick's theorem (Isserlis' theorem):
\begin{align*}
\left\langle \varphi_{1}^{2k}\varphi_{2}^{2k}\right\rangle _{0} & =\sum_{l=0}^{k}\left(\frac{2k!}{(2k-2l)!2l!}\right)^{2}\left(\frac{(2k-2l)!}{2^{k-l}(k-l)!}\right)^{2}G^{2l}\left(x_{1}-x_{2}\right)G^{2k-2l}\left(0\right)\\
 & =\sum_{l=0}^{k}\left(\frac{2k!}{2^{k-l}(k-l)!2l!}\right)^{2}G^{2l}\left(x_{1}-x_{2}\right)G^{2k-2l}\left(0\right)
\end{align*}
We receive combinatorial factor from the following considerations. We have $2k$ fields at $x_{1}$ and $2k$ fields at $x_{2}$, hence we have to connect all pairs construct of $4k$ fields. We can connect $l$ fields from point $x_{1}$ to $l$ fields from $x_{2}$, but in this case we can't connect remaining set of fields in each $x_{k}$ point. Hence we have to connect $2\cdot2l$ fields between $x_{1}$ and $x_{2}$ and $2*(2k-2l)$ inside. One can visualize it by picture:

\begin{figure}
\begin{centering}
\includegraphics[width=10cm,height=3.5cm]{comb_form}
\par\end{centering}
\caption{Illustration of combinatorial formula for correlators for $n=2$}

\end{figure}

The factor that arise from choose of fields connected between points is:
\[
Be_{l}^{k}=\left(\frac{2k!}{(2k-2l)!2l!}\right)^{2}
\]
where second degree arise due to the fact that we choose $2l$ fields at point $x_{1}$ and $2l$ fields at $x_{2}$. Connection of fields in one point give us:
\[
In_{l}^{k}=\left(\frac{(2k-2l)!}{2^{k-l}(k-l)!}\right)^{2}
\]
where second degree arise in the same way. We can generalize it for $n$ points:
\[
\left\langle \varphi_{1}^{2k}...\varphi_{n}^{2k}\right\rangle _{0}=\sum_{\begin{array}{c}
\sum_{b}l_{ab}=2k,\\
l_{aa}\vdots2
\end{array}}\left(\prod_{a<b}G^{l_{ab}}\left(x_{a}-x_{b}\right)\right)G\left(0\right)^{\frac{1}{2}\sum_{a}l_{aa}}\frac{\left(2k!\right)^{n}}{\left(\prod_{a<b}\left(l_{ab}!\right)\right)2^{\sum_{a}\frac{l_{aa}}{2}}\left(\prod_{a=1}^{n}\left(\frac{l_{aa}}{2}\right)!\right)}
\]
where $l_{ab}$ is a number of lines that connect fields at $x_{a}$ and at $x_{b}$ with each other and $l_{ab}=l_{ba}$. 

We can also solve the summation condition over $l_{aa}$ and rewrite:
\[
\left\langle \varphi_{1}^{2k}...\varphi_{n}^{2k}\right\rangle _{0}=\sum_{\begin{array}{c}
\sum_{b}l_{ab}=2k,
\\
l_{aa}\vdots2,\ l_{ab}\geq0
\end{array}}\left(\prod_{a<b}G^{l_{ab}}\left(x_{a}-x_{b}\right)\right) \]\[ \times G^{nk-\sum_{a<b}l_{ab}}\left(0\right)\frac{\left(2k!\right)^{n}}{\left(\prod_{a<b}\left(l_{ab}!\right)\right)2^{nk-\sum_{a<b}l_{ab}}
\left(\prod_{a=1}^{n}\left(k-\frac{1}{2}\sum_{b|b\neq a}l_{ab}\right)!\right)},
\]
which can be more useful for practical calculations.

For more general case one can obtain similarly:
\begin{equation}
\left\langle \varphi_{1}^{2k+c_{1}}...\varphi_{i}^{2k+c_{i}}...\varphi_{n}^{2k+c_{n}}\right\rangle _{0}=\sum_{\begin{array}{c}
\sum_{b}l_{ab}=2k+c_{a},\\
l_{aa}\vdots2,\ l_{ab}\geq0
\end{array}}\left(\prod_{a<b}G^{l_{ab}}\left(x_{a}-x_{b}\right)\right) \end{equation}\[ \times G\left(0\right)^{\frac{1}{2}\sum_{a}l_{aa}}\frac{\prod_{a=1}^{n}\left(2k+c_{a}\right)!}{\left(\prod_{a<b}\left(l_{ab}!\right)\right)2^{\sum_{a}\frac{l_{aa}}{2}}\left(\prod_{a=1}^{n}\left(\frac{l_{aa}}{2}\right)!\right)},
\]
for even $\sum_{a}c_{a}$, and $0$ for odd.

\subsubsection{Substitution of Particular Kinds of Polynomials}

Now we are going to use different kinds of polynomials and compare the rates of convergence.

\paragraph{Bernstein Polynomials}

In this case we have for approximating polynomial:
\begin{equation}
h_{N}(t)=\sum_{q=0}^{N}\sum_{l=0}^{N-q}\left(\frac{q}{N}\right)^{\alpha/2}\left(\begin{array}{l}
N\\
q
\end{array}\right)\left(\begin{array}{l}
N-q\\
\ \ l
\end{array}\right)(-1)^{l}t^{2(q+l)}\end{equation}\[=\sum_{q=0}^{N}\sum_{p=q}^{N}\left(\frac{q}{N}\right)^{\alpha/2}\frac{N!}{q!(p-q)!(N-p)!}(-1)^{p+q}t^{2p},
\]
where we have rewritten $|t|^{\alpha}=|t^{2}|^{\alpha/2}$ to construct an even polynomial via Bernstein approximation. We want the even approximation to avoid the great amount of zero terms which are likely to tangle the analysis of the expressions. We receive after the substitution:
\begin{equation}
\mathcal{G}_{I,n}[j]_{N} =\frac{\left(-1\right)^{n}}{n!}\sum_{\Gamma\in\mathbb{G}_{C,n}}\left\{ \prod_{a<b}^{n}\int_{0}^{1}ds_{ab}\partial_{s_{ab}}^{\nu_{ab}\left(\Gamma\right)}\right\} \left\{ \prod_{a=1}^{n}\int dx_{a}g\left(x_{a}\right)\right\} \exp\left(-Q\left(\bar{\phi},\bar{\phi}\right)\right)
\end{equation}
\begin{align*}
 & \times\sum_{k=0}^{\infty}\frac{\left(2G(0)\right)^{n\alpha/2}}{4^{k}k!}\frac{\left(\frac{n(\alpha+1)}{2}\right)_{2k}}{\left(1/2\right)_{2k}}\sum_{\nu_{1}+...+\nu_{n}=2k}\binom{n}{\nu_{1}.....\nu_{n}}\chi_{1}^{\nu_{1}}...\chi_{n}^{\nu_{n}}\\
 & \times\sum_{q=0}^{N}\sum_{i=q}^{N}\left(\frac{q}{N}\right)^{\alpha/2}\frac{(-1)^{i+q}N!}{q!(i-q)!(N-i)!}\frac{\Gamma\left(\frac{n(\alpha+1)}{2}\right)}{2^{ni}G(0)^{ni}\Gamma\left(\frac{n}{2}+n(i+k)\right)}\partial_{1}^{\nu_{1}+2i}...\partial_{n}^{\nu_{n}+2i}\bigg|_{\eta=0}e^{-\frac{1}{2}\sum_{a,b=1}^{n}s_{ab}\nu_{ab}G_{ab}^{(n)}\eta_{a}\eta_{b}}
\end{align*}

\paragraph{Chebyshev Polynomials}

Approximating polynomial:
\begin{equation}
h_{N}(t)=\sum_{q=0}^{N}c_{2q}T_{2q}(t),
\end{equation}
since $f$ is even, and the coefficients:
\[
c_{0}(f)=\frac{1}{\pi}\int_{-1}^{1}\frac{|t|^{\alpha}}{\sqrt{1-t^{2}}}dt=\frac{1}{\sqrt{\pi}}\frac{\Gamma\left(\frac{\alpha}{2}+\frac{1}{2}\right)}{\Gamma\left(\frac{\alpha}{2}+1\right)},
\]
and:
\[
c_{2q>0}(f)=\frac{4q}{\pi}\int_{0}^{1}\frac{t^{\alpha}}{\sqrt{1-t^{2}}}T_{2q}(t)dt=\frac{2q}{\sqrt{\pi}}\sum_{p=0}^{q}(-1)^{p}\frac{(2q-p-1)!}{p!(2q-2p)!}2^{2q-2p}\frac{\Gamma\left(q-p+\frac{\alpha}{2}+\frac{1}{2}\right)}{\Gamma\left(q-p+\frac{\alpha}{2}+1\right)}
\]
where we have used the integral:
\[
\int_{0}^{1}\frac{x^{\alpha+2p}}{\sqrt{1-x^{2}}}\,dx=\frac{\sqrt{\pi}\Gamma\left(p+\frac{\alpha}{2}+\frac{1}{2}\right)}{2\Gamma\left(p+\frac{\alpha}{2}+1\right)}
\]
so we have finally:
\begin{align*}
h_{N}(t) & =\frac{1}{\sqrt{\pi}}\frac{\Gamma\left(\frac{\alpha}{2}+\frac{1}{2}\right)}{\Gamma\left(\frac{\alpha}{2}+1\right)}+\frac{2}{\sqrt{\pi}}\sum_{q=1}^{N}q^{2}\left(\sum_{p=0}^{q}(-1)^{p}\frac{(2q-p-1)!}{p!(2q-2p)!}2^{2q-2p}\frac{\sqrt{\pi}\Gamma\left(p+\frac{\alpha}{2}+\frac{1}{2}\right)}{2\Gamma\left(p+\frac{\alpha}{2}+1\right)}\right)\\
 & +\sum_{i=0}^{q}(-1)^{i}\frac{(2q-i-1)!}{i!(2q-2i)!}2^{2q-2i}t^{2q-2i}
\end{align*}
and for the connected diagrams generating functional:
\begin{equation}
\mathcal{G}_{I,n}[j]_{N}  =\frac{\left(-1\right)^{n}}{\sqrt{\pi}n!}\sum_{\Gamma\in\mathbb{G}_{C,n}}\left\{ \prod_{a<b}^{n}\int_{0}^{1}ds_{ab}\partial_{s_{ab}}^{\nu_{ab}\left(\Gamma\right)}\right\} \left\{ \prod_{a=1}^{n}\int dx_{a}g\left(x_{a}\right)\right\} \exp\left(-Q\left(\bar{\phi},\bar{\phi}\right)\right)
\end{equation}
\begin{align*}
 & \times\sum_{k=0}^{\infty}\frac{\left(2G(0)\right)^{n\alpha/2}}{4^{k}k!}\frac{\left(\frac{n(\alpha+1)}{2}\right)_{2k}}{\left(1/2\right)_{2k}}\sum_{\nu_{1}+...+\nu_{n}=2k}\binom{n}{\nu_{1}.....\nu_{n}}\chi_{1}^{\nu_{1}}...\chi_{n}^{\nu_{n}}\left(\frac{\Gamma\left(\frac{\alpha}{2}+\frac{1}{2}\right)}{\Gamma\left(\frac{\alpha}{2}+1\right)}\frac{\Gamma\left(\frac{n(\alpha+1)}{2}\right)}{\Gamma\left(\frac{n}{2}+nk\right)}\right.\\
 & \left.\times\partial_{1}^{\nu_{1}}...\partial_{n}^{\nu_{n}}\bigg|_{\eta=0}e^{-\frac{1}{2}\sum_{a,b=1}^{n}G_{ab}\eta_{a}\eta_{b}}+2\sum_{q=1}^{N}\sum_{i=0}^{q}\frac{(-1)^{i}q^{2}\Gamma\left(\frac{n(\alpha+1)}{2}\right)}{2^{(n+2)i-4q}G(0)^{ni}\Gamma\left(\frac{n}{2}+n(i+k)\right)}\frac{(2q-i-1)!}{i!(2q-2i)!}\right.\\
 & \left.\times\left(\sum_{p=0}^{q}(-1)^{p}\frac{(2q-p-1)!}{p!(2q-2p)!}2^{-2p}\frac{\Gamma\left(q-p+\frac{\alpha}{2}+\frac{1}{2}\right)}{\Gamma\left(q-p+\frac{\alpha}{2}+1\right)}\right)\partial_{1}^{\nu_{1}+2i}...\partial_{n}^{\nu_{n}+2i}\bigg|_{\eta=0}e^{-\frac{1}{2}\sum_{a,b=1}^{n}s_{ab}\nu_{ab}G_{ab}^{(n)}\eta_{a}\eta_{b}}\right)
\end{align*}

\paragraph{Legendre Polynomials}

Similarly, approximating polynomial:
\begin{equation}
h_{N}(t)=\sum_{q=0}^{N}c_{2q}P_{2q}(t),
\end{equation}
due to the fact that $f$. The coefficients $c_{2q}$:
\[
c_{2q>0}(f)=2\left(2q+\frac{1}{2}\right)\int_{0}^{1}t^{\alpha}P_{2q}(t)dt=\left(2q+\frac{1}{2}\right)2^{2q+1}\sum_{p=0}^{q}\binom{2q}{2p}\binom{q+p-\frac{1}{2}}{2q}\frac{1}{2p+\alpha+1},
\]
so for $f_{N}$ we get:
\begin{equation}
h_{N}(t)=\sum_{q=0}^{N}\left(2q+\frac{1}{2}\right)2^{4q+1}\left(\sum_{p=0}^{q}\binom{2q}{2p}\binom{q+p-\frac{1}{2}}{2q}\frac{1}{2p+\alpha+1}\right)\sum_{i=0}^{q}\binom{2q}{2i}\binom{q+i-\frac{1}{2}}{2q}t^{2i},
\end{equation}
and the corresponding approximation:
\begin{equation}
\mathcal{G}_{I,n}[j]_{N} =\frac{\left(-1\right)^{n}}{n!}\sum_{\Gamma\in\mathbb{G}_{C,n}}\left\{ \prod_{a<b}^{n}\int_{0}^{1}ds_{ab}\partial_{s_{ab}}^{\nu_{ab}\left(\Gamma\right)}\right\} \left\{ \prod_{a=1}^{n}\int dx_{a}g\left(x_{a}\right)\right\} \exp\left(-Q\left(\bar{\phi},\bar{\phi}\right)\right)
\end{equation}
\begin{align*}
 & \times\sum_{k=0}^{\infty}\frac{\left(2G(0)\right)^{n\alpha/2}}{4^{k}k!}\frac{\left(\frac{n(\alpha+1)}{2}\right)_{2k}}{\left(1/2\right)_{2k}}\sum_{\nu_{1}+...+\nu_{n}=2k}\binom{n}{\nu_{1}.....\nu_{n}}\chi_{1}^{\nu_{1}}...\chi_{n}^{\nu_{n}}\sum_{q=0}^{N}\sum_{i=0}^{q}\left(2q+\frac{1}{2}\right)2^{4q-ni+1}\\
 & \times\frac{\Gamma\left(\frac{n(\alpha+1)}{2}\right)}{G(0)^{ni}\Gamma\left(\frac{n}{2}+n(i+k)\right)}\left(\sum_{p=0}^{q}\binom{2q}{2p}\binom{q+p-\frac{1}{2}}{2q}\frac{1}{2p+\alpha+1}\right)\binom{2q}{2i}\binom{q+i-\frac{1}{2}}{2q}\\
 & \times\partial_{1}^{\nu_{1}+2i}...\partial_{n}^{\nu_{n}+2i}\bigg|_{\eta=0}e^{-\frac{1}{2}\sum_{a,b=1}^{n}s_{ab}\nu_{ab}G_{ab}^{(n)}\eta_{a}\eta_{b}}
\end{align*}


\subsubsection{Comparison of Different Kind of Approximations}

For a first glance let us compare pointwise approximations of $|t|^{\alpha}$ by Bernstein, Chebyshev and Legendre polynomials with the methods described above. The results are represented in picture $(2)$ for
$\alpha=4/3$. For other $\alpha\in(1;2)$ there is no significant difference. From this one can draw a conclusion that Legendre and Chebyshev polynomials are better for approximation of integrands than a Bernstein ones, since for the same degree in the ``majority of points'' they gives smaller error. Moreover, errors of Legendre and Chebyshev polynomial approximation oscillate and change sign, rather than Bernstein approximation, so one should expect that in integrals there will be some ``cancellation of errors'' which will additionally raise the precision of the approximation. One can also note that even if one approximate by some other polynomials, the results are still
worse than for Legendre and Chebyshev.

\begin{figure}

\begin{centering}
\includegraphics[width=8cm,height=5cm]{pol_apprs} \includegraphics[width=8cm,height=5cm]{pol_errors} 
\par\end{centering}
\begin{centering}
\includegraphics[width=8cm,height=5cm]{BPA}\includegraphics[width=8cm,height=5cm]{ChPA}
\par\end{centering}
\begin{centering}
\includegraphics[width=8cm,height=5cm]{LPA}
\par\end{centering}
\caption{Comparative plots of approximations by polynomials for tenth degree (first two pictures), comparative plots of approximations by polynomials
of integrals $h_{L,B,Ch}(N,n)$}
 
\end{figure}

Since the formulas for Legendre polynomial approximation fetched simpler than for Chebyshev and the approximations are very close, we will use Legendre polynomials for further computations. Though it is interesting to compare Bernstein and Legendre approximations for integrals themselves. So we will compare for different $n$ the errors in obtained expressions, putting for the simplicity $G(x)\equiv1$. This substitution will provide the worst convergence on $N$, since there is no a factor of additional decrease of the influence of higher degree monomials.

Namely, let us compute the following expressions for $\alpha=4/3$, for Legendre: 
\begin{align*}
h_{L}(N,n) & :=\sum_{q=0}^{N}\sum_{i=0}^{q}\left(2q+\frac{1}{2}\right)2^{4q-ni+1}\left(\sum_{p=0}^{q}\binom{2q}{2p}\binom{q+p-\frac{1}{2}}{2q}\frac{1}{2p+\alpha+1}\right)\\
 & \times\binom{2q}{2i}\binom{q+i-\frac{1}{2}}{2q}\frac{\Gamma\left(\frac{n(\alpha+1)}{2}\right)}{2^{ni}G(0)^{ni}\Gamma\left(\frac{n}{2}+ni\right)n!}\partial_{1}^{2i}...\partial_{n}^{2i}\bigg|_{\eta=0}e^{-\frac{1}{2}\sum_{a,b=1}^{n}G_{ab}\eta_{a}\eta_{b}},
\end{align*}
Bernstein:
\[
h_{B}(N,n):=\sum_{q=0}^{N}\sum_{i=q}^{N}\left(\frac{q}{N}\right)^{\alpha/2}\frac{(-1)^{i+q}N!}{q!(i-q)!(N-i)!}\times\frac{\Gamma\left(\frac{n(\alpha+1)}{2}\right)}{2^{ni}G(0)^{ni}\Gamma\left(\frac{n}{2}+ni\right)n!}\partial_{1}^{2i}...\partial_{n}^{2i}\bigg|_{\eta=0}e^{-\frac{1}{2}\sum_{a,b=1}^{n}G_{ab}\eta_{a}\eta_{b}},
\]
as well as Chebyshev approximations:
\[
h_{Ch}(N,n):=\frac{\Gamma\left(\frac{\alpha}{2}+\frac{1}{2}\right)}{\Gamma\left(\frac{\alpha}{2}+1\right)}\frac{\Gamma\left(\frac{n(\alpha+1)}{2}\right)}{\Gamma\left(\frac{n}{2}+nk\right)}+2\sum_{q=1}^{N}\sum_{i=0}^{q}\frac{(-1)^{i}q^{2}\Gamma\left(\frac{n(\alpha+1)}{2}\right)}{2^{(n+2)i-4q}G(0)^{ni}n!\Gamma\left(\frac{n}{2}+n(i+k)\right)}\frac{(2q-i-1)!}{i!(2q-2i)!}\times
\]
\[
\left.\times\left(\sum_{p=0}^{q}(-1)^{p}\frac{(2q-p-1)!}{p!(2q-2p)!}2^{-2p}\frac{\Gamma\left(q-p+\frac{\alpha}{2}+\frac{1}{2}\right)}{\Gamma\left(q-p+\frac{\alpha}{2}+1\right)}\right)\right)
\]
where we have included the factor $1/n!$ from $\mathcal{G}=\sum_{n=1}^{\infty}\frac{1}{n!}\mathcal{G}_{n}$ to avoid factorial growth of factor caused by the neglect of certain factors, since we are going to look at dynamics in $n$. Recall that the degree of polynomial approximation in both cases is equal
to $2N$.

The plots of $h_{L,B,Ch}(N,n)$ for $n=2,...,5$ and $N=1,2,4$ (with degrees $2,4,6$, correspondingly) are presented in the picture $(2)$. From the plots one can see that for integrals Legendre and Chebyshev approximations also converge much better. And similarly we will use Legendre polynomials since the corresponding formulas are simpler. Moreover, we note that using the second degree approximation will give an error about $20\%$ and the error of fourth degree is about
$10\%$.

\subsubsection{Results of Analysis of Different Polynomial Approximations}

According to all points described above, we will focus on Legendre polynomial approximation. And exactly this kind of approximations we will mean in the following under the term ``polynomial approximation''. Finally, substituting the combinatorial formula in the Legendre approximation formula, we receive: 
\begin{align*}
\mathcal{G}_{I,n}[j]_{N} & =\frac{\left(-1\right)^{n}}{n!}\sum_{\Gamma\in\mathbb{G}_{C,n}}\left\{ \prod_{a<b}^{n}\int_{0}^{1}ds_{ab}\partial_{s_{ab}}^{\nu_{ab}\left(\Gamma\right)}\right\} \left\{ \prod_{a=1}^{n}\int dx_{a}g\left(x_{a}\right)\right\} \exp\left(-Q\left(\bar{\phi},\bar{\phi}\right)\right)\\
 & \times\sum_{k=0}^{\infty}\frac{\left(2G(0)\right)^{n\alpha/2}}{4^{k}k!}\frac{\left(\frac{n(\alpha+1)}{2}\right)_{2k}}{\left(1/2\right)_{2k}}\sum_{\nu_{1}+...+\nu_{n}=2k}\binom{n}{\nu_{1}.....\nu_{n}}\chi_{1}^{\nu_{1}}...\chi_{n}^{\nu_{n}}\sum_{q=0}^{N}\sum_{i=0}^{q}\left(2q+\frac{1}{2}\right)2^{4q-ni+1}\\
 & \times G(0)^{\nu_{1}+....+\nu_{n}}\frac{\Gamma\left(\frac{n(\alpha+1)}{2}\right)}{\Gamma\left(\frac{n}{2}+n(i+k)\right)}\left(\sum_{p=0}^{q}\binom{2q}{2p}\binom{q+p-\frac{1}{2}}{2q}\frac{1}{2p+\alpha+1}\right)\binom{2q}{2i}\binom{q+i-\frac{1}{2}}{2q}\\
 & \times\sum_{\begin{array}{c}
\sum_{b}l_{ab}=2i+\nu_{a},\\
l_{aa}\vdots2,\ l_{ab}\geq0
\end{array}}\left(\prod_{a<b}\left[\frac{s_{ab}\nu_{ab}(\Gamma)G\left(x_{a}-x_{b}\right)}{G(0)}\right]^{l_{ab}}\right)\frac{\prod_{a=1}^{n}\left(2i+\nu_{a}\right)!}{\left(\prod_{a<b}\left(l_{ab}!\right)\right)2^{\sum_{a}\frac{l_{aa}}{2}}\left(\prod_{a=1}^{n}\left(\frac{l_{aa}}{2}\right)!\right)}
\end{align*}
This is the main polynomial approximation formula for the following. 

\subsection{Simple Approximate Formula for General Term}

As one can see from the previous sections, approximation with the Legendre polynomial of second degree can give an error around $5\%\ldots10\%$, which is quite accurate result. Remarkably, in this case one can easily take also the coordinate integrals and finish up with not very complicated formula for the vacuum energy. So, let us find such an approximation formula. We will assume $g(x)=g=const$ to obtain some concrete formulas. Moreover, for the sake of simplicity we restrict our attention to the case of $j=0$. We start from $(121)$ for $j=0$:
\[
\mathcal{G}_{I,n}[0]_{N}=\frac{(-1)^{n}\left(G(0)2\right)^{n\alpha/2}g^{n}}{n!}\sum_{\Gamma\in\mathbb{G}_{C,n}}\left\{ \prod_{a<b}^{n}\int_{0}^{1}ds_{ab}\partial_{s_{ab}}^{\nu_{ab}(\Gamma)}\right\} \left\{ \prod_{a=1}^{n}\int dx_{a}\right\} \sum_{q=0}^{N}\sum_{i=0}^{q}\left(2q+\frac{1}{2}\right)\times
\]
\[
\times2^{4q-ni+1}\frac{\Gamma\left(\frac{n(\alpha+1)}{2}\right)}{G(0)^{ni}\Gamma\left(\frac{n}{2}+ni\right)}\left(\sum_{p=0}^{q}\left(\begin{array}{c}
2q\\
2p
\end{array}\right)\left(\begin{array}{c}
q+p-\frac{1}{2}\\
2q
\end{array}\right)\frac{1}{2p+\alpha+1}\right)\left(\begin{array}{c}
2q\\
2i
\end{array}\right)\left(\begin{array}{c}
q+i-\frac{1}{2}\\
2q
\end{array}\right)
\]
\[
\left.\times\partial_{1}^{2i}\ldots\partial_{n}^{2i}\right|_{\eta=0}e^{-\frac{1}{2}\sum_{a,b=1}^{n}s_{ab}\nu_{ab}G_{ab}^{(n)}\eta_{a}\eta_{b}}
\]
Let us denote the coefficients:
\[
\mathscr{A}_{n,q,\alpha,i}=\left(2q+\frac{1}{2}\right)\frac{\Gamma\left(\frac{n(\alpha+1)}{2}\right)}{\Gamma\left(\frac{n}{2}+ni\right)}2^{4q-ni+1}\left(\sum_{p=0}^{q}\left(\begin{array}{c}
2q\\
2p
\end{array}\right)\left(\begin{array}{c}
q+p-\frac{1}{2}\\
2q
\end{array}\right)\frac{1}{2p+\alpha+1}\right)\left(\begin{array}{c}
2q\\
2i
\end{array}\right)\left(\begin{array}{c}
q+i-\frac{1}{2}\\
2q
\end{array}\right)
\]
and substitute the combinatorial formula:
\[
\mathcal{G}_{I,n}[0]_{N}=\frac{(-1)^{n}g^{n}}{n!}2^{n\alpha/2}\sum_{\Gamma\in\mathbb{G}_{C,n}}\left\{ \prod_{a<b}^{n}\int_{0}^{1}ds_{ab}\partial_{s_{ab}}^{\nu_{ab}(\Gamma)}\right\} \left\{ \prod_{a=1}^{n}\int dx_{a}\right\} 
\]
\[
\times\sum_{q=0}^{N}\sum_{i=0}^{q}\mathscr{A}_{n,q,\alpha,i}\sum_{\begin{array}{c}
\sum_{b}l_{ab}=2i,\\
l_{aa}\vdots2,\ l_{ab}\geq0
\end{array}}\left(\prod_{a<b}\left[\frac{s_{ab}\nu_{ab}(\Gamma)G\left(x_{a}-x_{b}\right)}{G(0)}\right]^{l_{ab}}\right)\frac{\left(2i\right)!^{n}}{\left(\prod_{a<b}\left(l_{ab}!\right)\right)2^{\sum_{a}\frac{l_{aa}}{2}}\left(\prod_{a=1}^{n}\left(\frac{l_{aa}}{2}\right)!\right)}.
\]
We want to calculate it analytically for $N=1$ (second degree polynomial approximation) and zero source. Since the operator $\prod_{a<b}^{n}\int_{0}^{1}ds_{ab}\partial_{s_{ab}}^{\nu_{ab}(\Gamma)}$, where $\nu_{ab}(\Gamma)$ is an adjacency matrix of a connected graph, the constant terms in $s_{ab}$ will not contribute, because of there is an edge in a graph such that the corresponding term will become zero after differentiation. This means we
can consider only $i\neq0$:
\[
\mathcal{G}_{I,n}[0]_{1}=\frac{(-1)^{n}g^{n}}{n!}\left(2G(0)\right)^{n\alpha/2}\mathscr{A}_{n,1,\alpha,1}\sum_{\Gamma\in\mathbb{G}_{C,n}}\left\{ \prod_{a<b}^{n}\int_{0}^{1}ds_{ab}\partial_{s_{ab}}^{\nu_{ab}(\Gamma)}\right\} \left\{ \prod_{a=1}^{n}\int dx_{a}\right\} 
\]
\[
\times\sum_{\begin{array}{c}
\sum_{b}l_{ab}=2,\\
l_{aa}\vdots2
\end{array}}\left(\prod_{a<b}\left[\frac{s_{ab}\nu_{ab}(\Gamma)G\left(x_{a}-x_{b}\right)}{G(0)}\right]^{l_{ab}}\right)\frac{2^{n}}{\left(\prod_{a<b}\left(l_{ab}!\right)\right)2^{\sum_{a}\frac{l_{aa}}{2}}\left(\prod_{a=1}^{n}\left(\frac{l_{aa}}{2}\right)!\right)}
\]
At present, for evaluation of Wick's correlators it is useful to directly sort all the pairings rather than use the obtained combinatorial formulas. Namely, one can sort all the configurations in Wick's correlator
$\left\langle \phi_{1}^{2}...\phi_{n}^{2}\right\rangle _{0}$ by the number of $r$ vertices pairing with themselves. Other vertices have to constitute the closed chain of length $n-r$. Then we can write:
\[
\left\langle \phi_{1}^{2}...\phi_{n}^{2}\right\rangle _{0}=\partial_{1}^{2i}\ldots\partial_{n}^{2i}{}_{\eta=0}e^{-\frac{1}{2}\sum_{a,b=1}^{n}G_{ab}^{(n)}\eta_{a}\eta_{b}}=\sum_{r=0}^{n}\frac{C_{n}^{r}G^{r}(0)}{2(n-r)}\sum_{i_{1},...,i_{n-r}}G(x_{i_{1}}-x_{i_{2}})G(x_{i_{2}}-x_{i_{3}})\cdot....\cdot G(x_{i_{n-r}}-x_{i_{1}}),
\]
where for $n=r$ the factor $\frac{1}{n-r}$ is understood as $1$. The combinatorial factor $\frac{C_{n}^{r}}{2(n-r)}$ reads as follows. We have $C_{n}^{r}$ variants for choosing $r$ points from $n$, which will be the points paired with themselves, and we divide it by $\frac{1}{2(n-r)}$, which corresponds to cyclic permutation of points in the chain and the reflection of the chain. Then we receive:
\[
\mathcal{G}_{I,n}[0]_{1}=\frac{(-1)^{n}g^{n}}{n!}\left(2G(0)\right)^{n\alpha/2}\mathscr{A}_{n,1,\alpha,1}\sum_{\Gamma\in\mathbb{G}_{C,n}}\left\{ \prod_{a<b}^{n}\int_{0}^{1}ds_{ab}\partial_{s_{ab}}^{\nu_{ab}(\Gamma)}\right\} \frac{1}{G(0)^{n}}
\]
\[
\times\left\{ \prod_{a=1}^{n}\int dx_{a}\right\} \sum_{r=0}^{n}\frac{C_{n}^{r}}{2(n-r)}G^{r}(0)\sum_{i_{1},...,i_{n-r}}\left(s_{i_{1},i_{2}}\nu_{i_{1},i_{2}}G(x_{i_{1}}-x_{i_{2}})\right)\cdot....\cdot\left(s_{i_{n-r},i_{1}}\nu_{i_{n-r},i_{1}}(\Gamma)G(x_{i_{n-r}}-x_{i_{1}})\right)
\]
Because of the action of operator $\sum_{\Gamma\in\mathbb{G}_{C,n}}\left\{ \prod_{a<b}^{n}\int_{0}^{1}ds_{ab}\partial_{s_{ab}}^{\nu_{ab}(\Gamma)}\right\} $
for every term in a sum $\sum_{r=0}^{n}$ it survives if and only if $\nu_{ab}=0$ for $a$ belonging to all the vertices which lie not in the chain. But since $\Gamma$ is connected, this condition can't be satisfied for $r\neq0$. For $r=0$, the only nonzero contribution in $\sum_{\Gamma\in\mathbb{G}_{C,n}}$ will be from cyclic graph $\Gamma_{0}$ $x_{i_{1}}-x_{i_{2}}-....-x_{i_{n}}-x_{i_{1}}$. Moreover, due to symmetric invariance of every term under permutations of $x_{a}$, we can suppose that $i_{a}=a$.

So we obtain, removing the integrals over all $s_{ab}$ except for $s_{1,2},...,s_{n,1}$:
\[
\mathcal{G}_{I,n}[0]_{1}=\frac{(-1)^{n}}{n!}\left(2G(0)\right)^{n\alpha/2}\frac{n!}{2n}g^{n}\frac{1}{G(0)^{n}}\mathscr{A}_{n,1,\alpha,1}\left\{ \prod_{a=1}^{n}\int dx_{a}\right\} \left\{ \int_{0}^{1}ds_{12}\partial_{s_{12}}\cdot...\cdot\int_{0}^{1}ds_{n1}\partial_{s_{n1}}\right\} 
\]
\[
\times\left(s_{12}G(x_{1}-x_{2})\right)\cdot....\cdot\left(s_{n,1}G(x_{n}-x_{1})\right),
\]
where we have used that after permutation of $x_{a}$ there will be $n!$ equal terms. So, the remaining integrals over $s_{ab}$ are easily calculated:
\[
\mathcal{G}_{I,n}[0]_{1}=\frac{(-1)^{n}}{n!}\left(2G(0)\right)^{n\alpha/2}\frac{n!}{2n}g^{n}\mathscr{A}_{n,1,\alpha,1}\left\{ \prod_{a=1}^{n}\int dx_{a}\right\} \frac{G(x_{1}-x_{2})}{G(0)}\cdot....\cdot\frac{G(x_{n}-x_{1})}{G(0)}
\]
Let us change variables:
\[
\begin{cases}
x_{k}-x_{k+1}=t_{k} & k=\overline{1,n-1}\\
x_{n}=t_{n}
\end{cases}\to\vec{t}=\left(\begin{array}{ccccc}
1 & -1 & 0 & ... & 0\\
0 & 1 & -1 & 0 & ...\\
... & ... & ... & ... & ...\\
0 & ... & 0 & 1 & 1\\
0 & 0 & ... & 0 & 1
\end{array}\right)\vec{x}
\]
and last argument is the sum of all previous:
\[
x_{n}-x_{1}=(x_{n}-x_{n-1})+(x_{n-1}-x_{n-2})+...+(x_{2}-x_{1})
\]
and the Jacobian of this change is equal to $1$, hence:
\[
\mathcal{G}_{I,n}[0]_{N=1}=\left(2G(0)\right)^{n\alpha/2}\frac{(-1)^{n}}{2n}Vg^{n}\mathscr{A}_{n,1,\alpha,1}\left\{ \prod_{a=1}^{n-1}\int dt_{a}\frac{G(t_{a})}{G(0)}\right\} \frac{G\left(\sum_{k=1}^{n-1}t_{k}\right)}{G(0)}.
\]
The coefficients can be simplified:
\[
\mathscr{A}_{n,1,\alpha,1}=2^{-n-1}\frac{\Gamma\left(\frac{n(\alpha+1)}{2}\right)}{\Gamma\left(\frac{n}{2}+n\right)}\frac{15\alpha}{\alpha^{2}+4\alpha+3},
\]
so finally:
\[
\mathcal{G}_{I,n}[0]_{N=1}=\frac{(-1)^{n}}{n}2^{n(\alpha/2-1)-2}Vg^{n}G(0)^{n\alpha/2}\frac{\Gamma\left(\frac{n(\alpha+1)}{2}\right)}{\Gamma\left(\frac{n}{2}+n\right)}\frac{15\alpha}{\alpha^{2}+4\alpha+3}\left\{ \prod_{a=1}^{n-1}\int dt_{a}\frac{G(t_{a})}{G(0)}\right\} \frac{G\left(\sum_{k=1}^{n-1}t_{k}\right)}{G(0)},
\]
and vacuum energy density:
\[
w_{vac,N=1}=\sum_{n=1}^\infty \frac{(-1)^{n-1}}{n}2^{n(\alpha/2-1)-2}\left[gG(0)^{\alpha/2}\right]^{n}\frac{\Gamma\left(\frac{n(\alpha+1)}{2}\right)}{\Gamma\left(\frac{n}{2}+n\right)}\frac{15\alpha}{\alpha^{2}+4\alpha+3}\left\{ \prod_{a=1}^{n-1}\int dt_{a}\frac{G(t_{a})}{G(0)}\right\} \frac{G\left(\sum_{k=1}^{n-1}t_{k}\right)}{G(0)}
\]
For $\alpha=2$, in particular:
\[
w_{vac,N=1}=\sum_{n=1}^\infty \frac{(-1)^{n-1}}{2n}g^{n}\left\{ \prod_{a=1}^{n-1}\int dt_{a}G(t_{a})\right\} G\left(\sum_{k=1}^{n-1}t_{k}\right),
\]
which coincides with the exact answer for $\alpha=2$. This is a simple check of the obtained result since $f(t)=t^{2}$ belongs to the linear span of zero and second Legendre polynomials.

\section{Hard-Sphere Gas Approximation}

One can easily note that the obtained formulas are quite difficult for computations, so we have to do some assumptions to obtain simpler and hence useful formulas. Namely, the formulas become significantly simpler if we use for $G(x-y)$ the ``hard-sphere gas'' approximation, inspired by analogy with non-ideal gas partition function. We are going to approximate:
\begin{equation}
G_{ab}=\left(\gamma+\left(1-\gamma\right)\delta_{ab}\right)G(0)\theta\left(\delta-\left|x_{ab}\right|\right),
\end{equation}
keeping all the diagonal terms are equal to $G(0)$, since they are constant, and approximating $G_{ab}\approx\gamma G(0)\theta\left(\delta-\left|x_{ab}\right|\right)$
for $a\neq b$ and some positive parameters $\gamma$ and $\delta$.
Let $v=\frac{\pi^{d/2}}{\Gamma(d/2+1)}\delta^{d}$ be the volume related to every ``particle''. The procedure of determining $\gamma$ and $\delta$ is ambiguous, so we will require:
\begin{equation}
\gamma G(0)v=\int G(x)dx,\qquad\gamma^{2}G(0)^{2}v=\int G(x)^{2}dx.
\end{equation}

We start from the formula $(76)$:
\begin{align*}
\mathcal{G}_{I,n}[j] & =\frac{\left(-g\right)^{n}}{n!}\sum_{\Gamma\in\mathbb{G}_{C,n}}\left\{ \prod_{a<b}^{n}\int_{0}^{1}ds_{ab}\partial_{s_{ab}}^{\nu_{ab}}\right\} \left\{ \prod_{a=1}^{n}\int_{\mathbb{R}^{d}}dx_{a}\int_{\mathbb{R}}\frac{dt_{a}}{2\pi}\mathcal{F}\left[U_{\Lambda}\left[\phi\right]\right]\exp\left(it_{a}\bar{\phi}\left(x_{a}\right)-\frac{1}{2}G\left(0\right)t_{a}^{2}\right)\right\} \\
 & \times\prod_{a<b}\exp\left(-\nu_{ab}\left(\Gamma\right)s_{ab}G\left(x_{a}-x_{b}\right)t_{a}t_{b}\right),
\end{align*}
for $g(x)=g=const$ and before applying Plancherel's identity to avoid $G_{ab}^{-1}$ and $\frac{1}{\sqrt{\det G}}$. Let us substitute:
\begin{equation}
G_{ab}=\left(\gamma+\left(1-\gamma\right)\delta_{ab}\right)G(0)\theta\left(\delta-\left|x_{ab}\right|\right)
\end{equation}
and obtain:
\begin{align}
\mathcal{G}_{I,n}[j] & =\frac{\left(-g\right)^{n}}{n!}\left(\sum_{\Gamma\in\mathbb{G}_{C,n}}\left\{ \prod_{a<b}^{n}\int_{0}^{1}ds_{ab}\partial_{s_{ab}}^{\nu_{ab}}\right\} \right)\left\{ \prod_{a=1}^{n}\int dx_{a}\right\} \prod_{a<b}\theta\left(\delta-\left|x_{ab}\right|\right)\\
 & \times\left\{ \prod_{a=1}^{n}\int_{\mathbb{R}}\frac{dt_{a}}{2\pi}\mathcal{F}\left[U_{\Lambda}\left[\phi\right]\right]\exp\left(it_{a}\bar{\phi}\left(x_{a}\right)-\frac{1}{2}G\left(0\right)t_{a}^{2}\right)\right\} \prod_{a<b}\exp\left(-\nu_{ab}\left(\Gamma\right)s_{ab}G\left(x_{a}-x_{b}\right)t_{a}t_{b}\right)\nonumber 
\end{align}
We can calculate volume integral approximately with commonly accepted approximation analogical to one in the statistical physics of hard-sphere gas. We can place first particle in full volume what gives us $V$ and the following particle should be no further from the first than
$\delta$:
\[
\left(\prod_{a=1}^{n}\int dx_{a}\right)\left(\prod_{a<b}\theta\left(\delta-\left|x_{ab}\right|\right)\right)\approx Vv^{n-1}
\]
where we denote by $v$ the volume of sphere with radius $\delta$, in $d$-dimension $v=\frac{\pi^{d/2}}{\Gamma\left(\frac{d}{2}+1\right)}\delta^{d}$.
Hence we have:
\begin{align}
\mathcal{G}_{I,n}[j]= & \frac{\left(-g\right)^{n}}{n!}Vv^{n-1}\left(\sum_{\Gamma\in\mathbb{G}_{C,n}}\left\{ \prod_{a<b}^{n}\int_{0}^{1}ds_{ab}\partial_{s_{ab}}^{\nu_{ab}}\right\} \right)\\
 & \left\{ \prod_{a=1}^{n}\int_{\mathbb{R}}\frac{dt_{a}}{2\pi}\mathcal{F}\left[U_{\Lambda}\left[\phi\right]\right]\exp\left(it_{a}\bar{\phi}\left(x_{a}\right)-\frac{1}{2}G\left(0\right)t_{a}^{2}\right)\right\} \prod_{a<b}\exp\left(-\nu_{ab}\left(\Gamma\right)s_{ab}G\left(x_{a}-x_{b}\right)t_{a}t_{b}\right)\nonumber 
\end{align}

This we obtained a much simplified expression for $\mathcal{G}_{I,n}$, with integrals over coordinates removed. Well, it is still a problem to calculate directly the remaining inner integral, so we use the developed technique of polynomial approximations. Exactly like in
the section $6$, we rewrite:
\begin{align*}
\mathcal{G}_{I,n}[j] & =\frac{\left(-g\right)^{n}}{n!}Vv^{n-1}\sum_{\Gamma\in\mathbb{G}_{C,n}}\left\{ \prod_{a<b}^{n}\int_{0}^{1}ds_{ab}\partial_{s_{ab}}^{\nu_{ab}}\right\} \exp\left(-Q\left(\bar{\phi},\bar{\phi}\right)\right)\\
 & \times\sum_{k=0}^{\infty}\frac{2^{n\alpha/2}G(0)^{n\alpha/2}}{4^{k}k!}\frac{\left(\frac{n(\alpha+1)}{2}\right)_{2k}}{\left(1/2\right)_{2k}}\sum_{\nu_{1}+...+\nu_{n}=2k}\binom{n}{\nu_{1}.....\nu_{n}}\chi_{1}^{\nu_{1}}...\chi_{n}^{\nu_{n}}\sum_{q=0}^{\infty}\sum_{i=0}^{q}\left(2q+\frac{1}{2}\right)2^{4q-ni+1}\\
 & \times\frac{\Gamma\left(\frac{n(\alpha+1)}{2}\right)}{\Gamma\left(\frac{n}{2}+n(i+k)\right)}\left(\sum_{p=0}^{q}\binom{2q}{2p}\binom{q+p-\frac{1}{2}}{2q}\frac{1}{2p+\alpha+1}\right)\binom{2q}{2i}\binom{q+i-\frac{1}{2}}{2q}\\
 & \times\partial_{1}^{\nu_{1}+2i}...\partial_{n}^{\nu_{n}+2i}\bigg|_{\eta=0}e^{-\frac{G(0)}{2}\sum_{a,b=1}^{n}s_{ab}\nu_{ab}\left(\gamma+\left(1-\gamma\right)\delta_{ab}\right)\eta_{a}\eta_{b}}
\end{align*}
Finally, inserting the combinatorial formula for correlators we arrive at the following expression:
\begin{align*}
\mathcal{G}_{I,n}[j] & =\frac{\left(-g\right)^{n}}{n!}Vv^{n-1}\sum_{\Gamma\in\mathbb{G}_{C,n}}\left\{ \prod_{a<b}^{n}\int_{0}^{1}ds_{ab}\partial_{s_{ab}}^{\nu_{ab}}\right\} \exp\left(-Q\left(\bar{\phi},\bar{\phi}\right)\right)\\
 & \times\sum_{k=0}^{\infty}\frac{2^{n\alpha/2}G(0)^{n\alpha/2}}{4^{k}k!}\frac{\left(\frac{n(\alpha+1)}{2}\right)_{2k}}{\left(1/2\right)_{2k}}\sum_{\nu_{1}+...+\nu_{n}=2k}\binom{n}{\nu_{1}.....\nu_{n}}\chi_{1}^{\nu_{1}}...\chi_{n}^{\nu_{n}}\sum_{q=0}^{\infty}\sum_{i=0}^{q}\left(2q+\frac{1}{2}\right)2^{4q-ni+1}\\
 & \times\frac{\Gamma\left(\frac{n(\alpha+1)}{2}\right)}{\Gamma\left(\frac{n}{2}+n(i+k)\right)}\left(\sum_{p=0}^{q}\binom{2q}{2p}\binom{q+p-\frac{1}{2}}{2q}\frac{1}{2p+\alpha+1}\right)\binom{2q}{2i}\binom{q+i-\frac{1}{2}}{2q}\\
 & \times G(0)^{\nu_{1}+...+\nu_{n}}\sum_{\begin{array}{c}
\sum_{b}l_{ab}=2i+\nu_{a},\\
l_{aa}\vdots2
\end{array}}\left(\prod_{a<b}s_{ab}^{l_{ab}}\gamma^{l_{ab}}\nu_{ab}^{l_{ab}}\right)\frac{\prod_{a=1}^{n}\left(2i+\nu_{a}\right)!}{\left(\prod_{a<b}\left(l_{ab}!\right)\right)2^{\sum_{a}\frac{l_{aa}}{2}}\left(\prod_{a=1}^{n}\left(\frac{l_{aa}}{2}\right)!\right)}
\end{align*}

In this paper we limit ourselves to considering only zero source case, or vacuum energy in other words. In this case:
\begin{align}
w_{vac,n} & =(-1)^{n-1}\frac{\left(gG(0)^{\alpha/2}\right)^{n}}{n!}2^{n\alpha/2}v^{n-1}\sum_{q=0}^{\infty}\sum_{i=0}^{q}\left(2q+\frac{1}{2}\right)2^{4q-ni+1}\frac{\Gamma\left(\frac{n(\alpha+1)}{2}\right)}{\Gamma\left(\frac{n}{2}+ni\right)}\binom{2q}{2i}\binom{q+i-\frac{1}{2}}{2q}\\
 & \times\sum_{\Gamma\in\mathbb{G}_{C,n}}\left\{ \prod_{a<b}^{n}\int_{0}^{1}ds_{ab}\partial_{s_{ab}}^{\nu_{ab}}\right\} \left(\sum_{p=0}^{q}\binom{2q}{2p}\binom{q+p-\frac{1}{2}}{2q}\frac{1}{2p+\alpha+1}\right)\\
 &\sum_{\begin{array}{c}
\sum_{b}l_{ab}=2i,\\
l_{aa}\vdots2,\ l_{ab}>0
\end{array}}\frac{\left(2i\right)!^{n}\prod_{a<b}\left(\gamma\nu_{ab}s_{ab}\right)^{l_{ab}}}{\left(\prod_{a<b}\left(l_{ab}!\right)\right)2^{\sum_{a}\frac{l_{aa}}{2}}\left(\prod_{a=1}^{n}\left(\frac{l_{aa}}{2}\right)!\right)}\nonumber 
\end{align}
where $w_{vac,n}$ is a contribution to $w_{vac}$ from $\mathcal{G}_{n}[0]$, namely
\[
w_{vac}=\sum_{n=1}^{\infty}w_{vac,n},
\]
so the total formula for vacuum energy density reads:
\begin{align}
w_{vac} & =\sum_{n=1}^{\infty}(-1)^{n-1}\frac{\left(gG(0)^{\alpha/2}\right)^{n}}{n!}2^{n\alpha/2}v^{n-1}\sum_{q=0}^{\infty}\sum_{i=0}^{q}\left(2q+\frac{1}{2}\right)2^{4q-ni+1}\frac{\Gamma\left(\frac{n(\alpha+1)}{2}\right)}{\Gamma\left(\frac{n}{2}+ni\right)}\binom{2q}{2i}\binom{q+i-\frac{1}{2}}{2q}\\
 & \times\sum_{\Gamma\in\mathbb{G}_{C,n}}\left\{ \prod_{a<b}^{n}\int_{0}^{1}ds_{ab}\partial_{s_{ab}}^{\nu_{ab}}\right\} \left(\sum_{p=0}^{q}\binom{2q}{2p}\binom{q+p-\frac{1}{2}}{2q}\frac{1}{2p+\alpha+1}\right)\\ &\sum_{\begin{array}{c}
\sum_{b}l_{ab}=2i,\\
l_{aa}\vdots2,\ l_{ab}>0
\end{array}}\gamma^{\sum_{a<b}l_{ab}}\frac{\left(2i\right)!^{n}\prod_{a<b}\left(\nu_{ab}s_{ab}\right)^{l_{ab}}}{\left(\prod_{a<b}\left(l_{ab}!\right)\right)2^{\sum_{a}\frac{l_{aa}}{2}}\left(\prod_{a=1}^{n}\left(\frac{l_{aa}}{2}\right)!\right)}\nonumber 
\end{align}

One can also rewrite the combinatorial sum, introducing
$l_{aa}=2q_{a}$:
\begin{align}
\sum_{\begin{array}{c}
\sum_{b}l_{ab}=2i,\\
l_{aa}\vdots2,\ l_{ab}>0
\end{array}}\gamma^{\sum_{a<b}l_{ab}}\frac{\left(2i\right)!^{n}}{\left(\prod_{a<b}\left(l_{ab}!\right)\right)2^{\sum_{a}\frac{l_{aa}}{2}}\left(\prod_{a=1}^{n}\left(\frac{l_{aa}}{2}\right)!\right)} & =\nonumber \\
=\left(2i\right)!^{n}\gamma^{ni}\sum_{q_{a}=0}^{i-1}\frac{1}{(2\gamma)^{\sum q_{a}}\prod_{a=1}^{n}q_{a}!}\sum_{\begin{array}{c}
\sum_{b\neq a}l_{ab}=2i-2q_{a},\\
l_{ab}>0
\end{array}}\frac{1}{\prod_{a<b}l_{ab}!} & \prod_{a<b}\left(\nu_{ab}s_{ab}\right)^{l_{ab}}
\end{align}

\part{Calculation and Research of Physical Characteristics of the System}

Using the the result obtained one can calculate some interesting physical characteristics such as:
\begin{enumerate}
\item Vacuum energy;
\item Correlators of first orders, namely, $2$-point function.
\end{enumerate}
We will use first few exactly calculated terms of our series as well as approximate expressions for general terms and research which physical results we can obtain from them. 

We note that we have perturbation series:
\[
\mathcal{Z}\left[j\right]=\sum_{n=0}^{\infty}Z_{n}[j],\qquad Z_{I}[j]=\sum_{n=0}^{\infty}Z_{I,n}[j],\qquad\mathcal{G}[j]=\sum_{n=0}^{\infty}\mathcal{G}_{n}[j],\qquad\mathcal{G}_{I}[j]=\sum_{n=0}^{\infty}\mathcal{G}_{I,n}[j],
\]
and they induce the ones for $k$-particles functions:
\[
\mathcal{D}_{1...k}^{(k)}=\sum_{n=0}^{\infty}\mathcal{D}_{1,...,k;\ n}^{(k)},\qquad\mathcal{D}_{I,1...k}^{(k)}=\sum_{n=0}^{\infty}\mathcal{D}_{I,1,...,k;\ n}^{(k)}\qquad\mathcal{G}_{1...k}^{(k)}=\sum_{n=0}^{\infty}\mathcal{G}_{1,...,k;\!n}^{(k)},\qquad
\mathcal{G}_{I,1...k}^{(k)}=
\sum_{n=0}^{\infty}\mathcal{G}_{I,1,...,k;\!n}^{(k)}.
\]
 And the formerly derived connections:
\begin{equation}
\mathcal{G}_{I,n}[j(x),G_{ab}]=\sum_{\Gamma\in\mathbb{G}_{C,n}}\left\{ \prod_{a<b}^{n}\int_{0}^{1}ds_{ab}\partial_{s_{ab}}^{\nu_{ab}\left(\Gamma\right)}\right\} Z_{I,n}[\bar{\phi}(x),\nu_{ab}\left(\Gamma\right)s_{ab}G_{ab}],
\end{equation}
\begin{equation}
\mathcal{G}_{I,1,...,k;\ n}^{(k)}[G_{ab}]=\sum_{\Gamma\in\mathbb{G}_{C,n}}\left\{ \prod_{a<b}^{n}\int_{0}^{1}ds_{ab}\partial_{s_{ab}}^{\nu_{ab}\left(\Gamma\right)}\right\} \mathcal{D}_{I,1,...,k;\ n}^{(k)}[\nu_{ab}\left(\Gamma\right)s_{ab}G_{ab}],
\end{equation}
In particular, for $n=0,1,2$ and $k=0,2$:
\begin{enumerate}
\item $\mathcal{G}_{I,0}$ doesn't depend on source $g$ and there is no $s_{ab}$ and graphs, so for $n=0$:
\[
\mathcal{G}_{I,0}[G_{ab}]=Z_{I,0}[\nu_{ab}\left(\Gamma\right)s_{ab}G_{ab}];
\]
\item $\mathcal{G}_{I,1}$ yet depends on source $j$, and there is still no $s_{ab}$, $a<b$, so for $n=1$ we also have simply:
\[
\mathcal{G}_{I,1}[G_{ab}]=Z_{I,1}[G_{ab}],\qquad\mathcal{G}_{I,1,...,k;\ 1}^{(k)}[G_{ab}]=\mathcal{D}_{I,1,...,k;\ 1}^{(k)}[G_{ab}];
\]
\item $\mathcal{G}_{I,1}$ also depends on source $j$, and there is one $s_{ab}$-variable, namely $s_{12}$, and one graph with $\nu_{12}=1$,
$\nu_{11,22}=0$:
\[
\mathcal{G}_{I;\ 2}[0,G_{ab}]=\int_{0}^{1}ds_{12}\partial_{s_{12}}Z_{I,2}[0,\nu_{ab}\left(\Gamma\right)s_{12}G_{ab}],
\]
\[
\mathcal{G}_{I,1,2;\ 2}^{(2)}[G_{ab}]=\left\{ \int_{0}^{1}ds_{12}\partial_{s_{12}}\right\} \mathcal{D}_{I,1,2;\ 2}^{(2)}[0,\nu_{ab}\left(\Gamma\right)s_{12}G_{ab}].
\]
\end{enumerate}
We will use this formulas to obtain the connected $n$-particle function later.

Besides, let us note one useful property of the obtained expressions. Namely we have:
\[
Z[j(x)]=\mathcal{Z}_{0}\left[j(x)\right]Z_{I}[j(x)],\qquad\mathcal{Z}_{0}\left[j\left(x\right)\right]=e^{\frac{1}{2}\int\int dxdyj\left(x\right)G(x,y)j\left(y\right)},
\]
hence:
\[
\mathcal{D}^{(2)}(y_{1},y_{2})=\frac{\delta^{2}}{\delta j(y_{1})\delta j(y_{2})}\bigg|_{j(y)=0}Z[j(x)]=G(y_{1}-y_{2})+\frac{\delta^{2}}{\delta j(y_{1})\delta j(y_{2})}\bigg|_{j(y)=0}Z_{I}[j(x)]
\]
The second term is exactly:
\[
\mathcal{D}_{I}^{(2)}(y_{1},y_{2})=\frac{\delta^{2}}{\delta j(y_{1})\delta j(y_{2})}\bigg|_{j(y)=0}Z_{I}[j(x)],
\]
so we have:
\[
\mathcal{D}^{(2)}(y_{1},y_{2})=G(y_{1}-y_{2})+\mathcal{D}_{I}^{(2)}(y_{1},y_{2}),
\]
and in the following subsection we will calculate $\mathcal{D}_{I}^{(2)}(y_{1},y_{2})$.
We will denote the contribution to $\mathcal{D}_{I}^{(2)}(y_{1},y_{2})$ from $Z_{I,n}$ as $\mathcal{D}_{I,n}^{(2)}(y_{1},y_{2})$.

\section{Calculations of Physical Characteristics}

\subsection{Expressions for First Terms via Series in Source $j$}

Now we will provide exact expressions for $Z_{n}$ for $n=0,1,2,3$. Then we get from them the corresponding connected contributions. We start from the expressions:
\begin{equation}
Z_{n}\left[j\left(x\right)\right]=\mathcal{Z}_{0}\left[j\left(x\right)\right]\frac{\left(-1\right)^{n}g^{n}}{n!\left(2\pi\right)^{n/2}}\prod_{a=1}^{n}\int dx_{a}e^{-\frac{1}{2}\sum_{a,b}\overline{\phi}\left(x_{a}\right)\left(G^{(n)}\right)_{ab}^{-1}\overline{\phi}\left(x_{b}\right)}\int d\phi_{a}\left|\phi_{a}\right|^{\alpha}\frac{e^{-\frac{1}{2}\sum\phi_{a}\left(G^{(n)}\right)_{ab}^{-1}\phi_{b}+\sum_{a=1}^{n}\phi_{a}\chi\left(x_{a}\right)}}{\sqrt{\det\left(G^{(n)}\right)}}
\end{equation}
Recall that, after removing of all the regulators:
\begin{equation}
\overline{\phi}(x_{a}):=\int dyG\left(x_{a}-y\right)j\left(y\right),\qquad G_{ab}:=G(x_{a}-x_{b}),\qquad R_{ab}^{(n)}=\left(G^{(n)}\right)_{ab}^{-1}
\end{equation}
\begin{equation}
\chi\left(x_{a}\right)=\sum_{b=1}^{n}R_{ab}^{(n)}\overline{\phi}\left(x_{b}\right),\qquad\mathcal{Z}_{0}\left[j\left(x\right)\right]:=e^{\frac{1}{2}\int\int dxdyj\left(x\right)G(x,y)j\left(y\right)}
\end{equation}
In the following we will call $n$-th term as the $n$-particle case.

\subsubsection{First Orders}

\paragraph{0-Particles}

It gives Gaussian theory:
\[
Z_{I,0}[j(x)]=\mathcal{G}_{I,0}[j(x)]=1
\]

\paragraph{1-Particles}

In this case $G^{(1)}=G(0)$, $R^{(1)}=G(0)^{-1}$ are $1\times1$ matrices.

One have, simply computing all the integrals after rescaling of $\phi$:
\begin{align*}
Z_{I,1}[j(x)] & =\frac{-gG(0)^{\frac{1+\alpha}{2}}}{\left(2\pi\right)^{1/2}}\int dx\frac{e^{-\frac{1}{2}\bar{\phi}^{2}\left(x\right)/G(0)}}{\sqrt{G(0)}}\int_{-\infty}^{+\infty}d\phi\left|\phi\right|^{\alpha}e^{-\frac{1}{2}\phi^{2}+\phi\cdot\chi\left(x\right)G(0)^{1/2}}\\
 & =-\frac{gG(0)^{\frac{\alpha}{2}}}{\left(2\pi\right)^{1/2}}\int dx\ e^{-\frac{1}{2}\bar{\phi}^{2}\left(x\right)/G(0)}\sum_{k=0}^{\infty}\frac{\chi^{k}\left(x\right)\left(1+\left(-1\right)^{k}\right)}{k!}G(0)^{k/2}\int_{0}^{+\infty}d\phi\phi^{k+\alpha}e^{-\frac{1}{2}\phi^{2}}\\
 & =-\frac{2gG(0)^{\frac{\alpha}{2}}}{\left(2\pi\right)^{1/2}}\sum_{k=0}^{\infty}\frac{\Gamma\left(\frac{1}{2}(2k+\alpha+1)\right)}{(2k)!}2^{\frac{2k-1+\alpha}{2}}G(0)^{k}\int dxe^{-\frac{1}{2}\bar{\phi}^{2}\left(x\right)/G(0)}\chi^{2k}\left(x\right)
\end{align*}
Here in second line we take series of exponent with currents, third line -- change of integration limits due to symmetry of the integrand. Thus we have the following final answer:
\begin{equation}
Z_{I,1}[j(x)]=-\frac{2gG(0)^{\frac{\alpha}{2}}}{\left(2\pi\right)^{1/2}}\sum_{k=0}^{\infty}\frac{\Gamma\left(\frac{1}{2}(2k+\alpha+1)\right)}{(2k)!}2^{\frac{2k-1+\alpha}{2}}G(0)^{k}\int dxe^{-\frac{1}{2}\bar{\phi}^{2}\left(x\right)/G(0)}\chi^{2k}\left(x\right)
\end{equation}
From this formula one can directly extract all the variational derivatives of $Z_{1}\left[j\left(x\right)\right]$ with respect to $j(x)$. It is not very useful to simplify this expression hereafter.

Using the formulas for connected contributions for $n=1$ we have:
\begin{equation}
\mathcal{G}_{I,1}[j(x),G_{ab}]=Z_{I,1}[j(x),G_{ab}]=-\frac{2gG(0)^{\frac{\alpha}{2}}}{\left(2\pi\right)^{1/2}}\sum_{k=0}^{\infty}\frac{\Gamma\left(\frac{1}{2}(2k+\alpha+1)\right)}{(2k)!}2^{\frac{2k-1+\alpha}{2}}G(0)^{k}\int dxe^{-\frac{1}{2}\bar{\phi}^{2}\left(x\right)/G(0)}\chi^{2k}\left(x\right)
\end{equation}


\subsubsection{Second Order}

In this case:
\[
G^{(2)}(x_{1},x_{2})=\left(\begin{array}{cc}
G(0) & G(x_{1}-x_{2})\\
G(x_{1}-x_{2}) & G(0)
\end{array}\right),
\]
and:
\[
R^{(2)}(x_{1},x_{2})=\frac{1}{G(0)^{2}-G(x_{1}-x_{2})^{2}}\left(\begin{array}{cc}
G(0) & -G(x_{1}-x_{2})\\
-G(x_{1}-x_{2}) & G(0)
\end{array}\right)
\]
which are $2\times2$ matrices. One have, similarly:
\[
Z_{I,2}[j(x)]=\frac{g^{2}}{4\pi}\int dx_{1}dx_{2}\frac{e^{-\frac{1}{2}\sum_{a,b=1}^{2}\bar{\phi}\left(x_{a}\right)R_{ab}^{(2)}\bar{\phi}\left(x_{b}\right)}}{\sqrt{G(0)^{2}-G(x_{1}-x_{2})^{2}}}\times
\]
\[
\times\int d\phi_{1}d\phi_{2}\left|\phi_{1}\right|^{\alpha}\left|\phi_{2}\right|^{\alpha}e^{-\frac{1}{2}\phi_{1}^{2}R_{11}^{(2)}-\frac{1}{2}\phi_{2}^{2}R_{11}^{(2)}-\phi_{1}R_{12}^{(2)}\phi_{2}+\sum_{a=1}^{2}\phi_{a}\chi\left(x_{a}\right)}
\]
The change of the variables of integral result in:
\begin{align*}
Z_{I,2}[j(x)] & =\frac{g^{2}}{4\pi}\int dx_{1}dx_{2}\frac{e^{-\frac{1}{2}\sum_{a,b=1}^{2}\bar{\phi}\left(x_{a}\right)R_{ab}^{(2)}\bar{\phi}\left(x_{b}\right)}}{\left(G(0)\right)^{1+\alpha}\left(G(0)^{2}-G(x_{1}-x_{2})^{2}\right)^{-\alpha-1/2}}\cdot\\
 & \cdot\int d\phi_{1}d\phi_{2}\left|\phi_{1}\right|^{\alpha}\left|\phi_{2}\right|^{\alpha}e^{-\frac{1}{2}\phi_{1}^{2}-\frac{1}{2}\phi_{2}^{2}-\phi_{1}\frac{R_{12}^{(2)}}{R_{11}^{(2)}}\phi_{2}+\sum_{a=1}^{2}\phi_{a}\frac{\chi\left(x_{a}\right)}{\left(R_{11}^{(2)}\right)^{\frac{1}{2}}}}
\end{align*}
We can take series of exponent:
\[
Z_{I,2}[j(x)] =\frac{g^{2}}{4\pi}\int dxdy\frac{e^{-\frac{1}{2}\sum_{a,b=1}^{2}\bar{\phi}\left(x_{a}\right)R_{ab}^{(2)}\bar{\phi}\left(x_{b}\right)}}{\left(G(0)\right)^{1+\alpha}\left(G(0)^{2}-G(x_{1}-x_{2})^{2}\right)^{-\alpha-1/2}}\\
\]
\[
 \times \sum_{l,k=0}^{\infty}\frac{\chi^{l}\left(x\right)\chi^{k}\left(y\right)}{\left(G(0)\right)^{\frac{l+k}{2}}n!k!}\left(G(0)^{2}-G(x_{1}-x_{2})^{2}\right)^{\frac{l+k}{2}}\int_{-\infty}^{+\infty}\int_{-\infty}^{+\infty}d\phi_{1}d\phi_{2}\phi_{1}^{l}\phi_{2}^{k}\left|\phi_{1}\right|^{\alpha}\left|\phi_{2}\right|^{\alpha}e^{-\frac{1}{2}\phi_{1}^{2}-\frac{1}{2}\phi_{2}^{2}-\phi_{1}\frac{R_{12}^{(2)}}{R_{11}^{(2)}}\phi_{2}}.
\]
Here $x_{1}=x,\ x_{2}=y$. Consider and calculate the next integral:
\[
I_{nk}=\int_{-\infty}^{+\infty}\int_{-\infty}^{+\infty}d\phi_{1}d\phi_{2}\phi_{1}^{l}\phi_{2}^{k}\left|\phi_{1}\right|^{\alpha}\left|\phi_{2}\right|^{\alpha}e^{-\frac{1}{2}\phi_{1}^{2}-\frac{1}{2}\phi_{2}^{2}-\phi_{1}\frac{R_{12}^{(2)}}{R_{11}^{(2)}}\phi_{2}}=
\]
\begin{align*}
 & =\int_{0}^{+\infty}\int_{0}^{+\infty}d\phi_{1}d\phi_{2}\phi_{1}^{l}\phi_{2}^{k}\left|\phi_{1}\right|^{\alpha}\left|\phi_{2}\right|^{\alpha}e^{-\frac{1}{2}\phi_{1}^{2}-\frac{1}{2}\phi_{2}^{2}-\phi_{1}\frac{R_{12}^{(2)}}{R_{11}^{(2)}}\phi_{2}}+\\
 & +\left(-1\right)^{k}\int_{0}^{+\infty}\int_{0}^{+\infty}d\phi_{1}d\phi_{2}\phi_{1}^{l}\phi_{2}^{k}\left|\phi_{1}\right|^{\alpha}\left|\phi_{2}\right|^{\alpha}e^{-\frac{1}{2}\phi_{1}^{2}-\frac{1}{2}\phi_{2}^{2}+\phi_{1}\frac{R_{12}^{(2)}}{R_{11}^{(2)}}\phi_{2}}\\
 & +\left(-1\right)^{l}\int_{0}^{+\infty}\int_{0}^{+\infty}d\phi_{1}d\phi_{2}\phi_{1}^{l}\phi_{2}^{k}\left|\phi_{1}\right|^{\alpha}\left|\phi_{2}\right|^{\alpha}e^{-\frac{1}{2}\phi_{1}^{2}-\frac{1}{2}\phi_{2}^{2}+\phi_{1}\frac{R_{12}^{(2)}}{R_{11}^{(2)}}\phi_{2}}\\
 & +\left(-1\right)^{l+k}\int_{0}^{+\infty}\int_{0}^{+\infty}d\phi_{1}d\phi_{2}\phi_{1}^{l}\phi_{2}^{k}\left|\phi_{1}\right|^{\alpha}\left|\phi_{2}\right|^{\alpha}e^{-\frac{1}{2}\phi_{1}^{2}-\frac{1}{2}\phi_{2}^{2}-\phi_{1}\frac{R_{12}^{(2)}}{R_{11}^{(2)}}\phi_{2}}=
\end{align*}
\[
I_{nk}=\int_{-\infty}^{+\infty}\int_{-\infty}^{+\infty}d\phi_{1}d\phi_{2}\phi_{1}^{l}\phi_{2}^{k}\left|\phi_{1}\right|^{\alpha}\left|\phi_{2}\right|^{\alpha}e^{-\frac{1}{2}\phi_{1}^{2}-\frac{1}{2}\phi_{2}^{2}-\phi_{1}\frac{R_{12}^{(2)}}{R_{11}^{(2)}}\phi_{2}}=
\]
\begin{align*}
 & =\int_{0}^{+\infty}\int_{0}^{+\infty}d\phi_{1}d\phi_{2}\phi_{1}^{l}\phi_{2}^{k}\left|\phi_{1}\right|^{\alpha}\left|\phi_{2}\right|^{\alpha}e^{-\frac{1}{2}\phi_{1}^{2}-\frac{1}{2}\phi_{2}^{2}-\phi_{1}\frac{R_{12}^{(2)}}{R_{11}^{(2)}}\phi_{2}}+\\
 & +\left(-1\right)^{k}\int_{0}^{+\infty}\int_{0}^{+\infty}d\phi_{1}d\phi_{2}\phi_{1}^{l}\phi_{2}^{k}\left|\phi_{1}\right|^{\alpha}\left|\phi_{2}\right|^{\alpha}e^{-\frac{1}{2}\phi_{1}^{2}-\frac{1}{2}\phi_{2}^{2}+\phi_{1}\frac{R_{12}^{(2)}}{R_{11}^{(2)}}\phi_{2}}\\
 & +\left(-1\right)^{l}\int_{0}^{+\infty}\int_{0}^{+\infty}d\phi_{1}d\phi_{2}\phi_{1}^{l}\phi_{2}^{k}\left|\phi_{1}\right|^{\alpha}\left|\phi_{2}\right|^{\alpha}e^{-\frac{1}{2}\phi_{1}^{2}-\frac{1}{2}\phi_{2}^{2}+\phi_{1}\frac{R_{12}^{(2)}}{R_{11}^{(2)}}\phi_{2}}\\
 & +\left(-1\right)^{l+k}\int_{0}^{+\infty}\int_{0}^{+\infty}d\phi_{1}d\phi_{2}\phi_{1}^{l}\phi_{2}^{k}\left|\phi_{1}\right|^{\alpha}\left|\phi_{2}\right|^{\alpha}e^{-\frac{1}{2}\phi_{1}^{2}-\frac{1}{2}\phi_{2}^{2}-\phi_{1}\frac{R_{12}^{(2)}}{R_{11}^{(2)}}\phi_{2}}=
\end{align*}
\begin{align*}
 & =\left(1+\left(-1\right)^{l+k}\right)\int_{0}^{+\infty}\int_{0}^{+\infty}d\phi_{1}d\phi_{2}\phi_{1}^{l+\alpha}\phi_{2}^{k+\alpha}e^{-\frac{1}{2}\phi_{1}^{2}-\frac{1}{2}\phi_{2}^{2}-\phi_{1}\frac{R_{12}^{(2)}}{R_{11}^{(2)}}\phi_{2}}+\\
 & +\left(\left(-1\right)^{k}+\left(-1\right)^{l}\right)\int_{0}^{+\infty}\int_{0}^{+\infty}d\phi_{1}d\phi_{2}\phi_{1}^{l+\alpha}\phi_{2}^{k+\alpha}e^{-\frac{1}{2}\phi_{1}^{2}-\frac{1}{2}\phi_{2}^{2}+\phi_{1}\frac{R_{12}^{(2)}}{R_{11}^{(2)}}\phi_{2}}=
\end{align*}
\[
=\left(1+\left(-1\right)^{l+k}\right)\int_{0}^{+\infty}\int_{0}^{+\infty}d\phi_{1}d\phi_{2}\phi_{1}^{l+\alpha}\phi_{2}^{k+\alpha}e^{-\frac{1}{2}\phi_{1}^{2}-\frac{1}{2}\phi_{2}^{2}}\left(e^{-\phi_{1}\frac{R_{12}^{(2)}}{R_{11}^{(2)}}\phi_{2}}+\left(-1\right)^{k}e^{\phi_{1}\frac{R_{12}^{(2)}}{R_{11}^{(2)}}\phi_{2}}\right)=
\]
\begin{align*}
 & =\left(1+\left(-1\right)^{l+k}\right)\int_{0}^{+\infty}\int_{0}^{+\infty}d\phi_{1}d\phi_{2}\phi_{1}^{l+\alpha}\phi_{2}^{k+\alpha}e^{-\frac{1}{2}\phi_{1}^{2}-\frac{1}{2}\phi_{2}^{2}-\phi_{1}\frac{R_{12}^{(2)}}{R_{11}^{(2)}}\phi_{2}}+\\
 & +\left(\left(-1\right)^{k}+\left(-1\right)^{l}\right)\int_{0}^{+\infty}\int_{0}^{+\infty}d\phi_{1}d\phi_{2}\phi_{1}^{l+\alpha}\phi_{2}^{k+\alpha}e^{-\frac{1}{2}\phi_{1}^{2}-\frac{1}{2}\phi_{2}^{2}+\phi_{1}\frac{R_{12}^{(2)}}{R_{11}^{(2)}}\phi_{2}}=
\end{align*}
\[
=\left(1+\left(-1\right)^{l+k}\right)\int_{0}^{+\infty}\int_{0}^{+\infty}d\phi_{1}d\phi_{2}\phi_{1}^{l+\alpha}\phi_{2}^{k+\alpha}e^{-\frac{1}{2}\phi_{1}^{2}-\frac{1}{2}\phi_{2}^{2}}\left(e^{-\phi_{1}\frac{R_{12}^{(2)}}{R_{11}^{(2)}}\phi_{2}}+\left(-1\right)^{k}e^{\phi_{1}\frac{R_{12}^{(2)}}{R_{11}^{(2)}}\phi_{2}}\right)=
\]
First integral is the hypergeometric function:{\footnotesize{}
\begin{align*}
 & =2^{\frac{1}{2}(\alpha+l-1)}\left(1+\left(-1\right)^{l+k}\right)\left(\left((-1)^{k}+1\right)\Gamma\left(\frac{1}{2}(l+\alpha+1)\right)\int_{0}^{+\infty}d\phi_{2}\phi_{2}^{k+\alpha}e^{-\frac{1}{2}\phi_{2}^{2}}\,_{1}F_{1}\left(\frac{1}{2}(l+\alpha+1);\frac{1}{2};\frac{1}{2}\left(\frac{R_{12}^{(2)}}{R_{11}^{(2)}}\right)^{2}\phi_{2}^{2}\right)+\right.\\
 & \left.+\sqrt{2}\frac{R_{12}^{(2)}}{R_{11}^{(2)}}\left((-1)^{k}-1\right)\Gamma\left(\frac{1}{2}(l+\alpha+2)\right)\int_{0}^{+\infty}d\phi_{2}\phi_{2}^{k+\alpha+1}e^{-\frac{1}{2}\phi_{2}^{2}}\,_{1}F_{1}\left(\frac{1}{2}(l+\alpha+2);\frac{3}{2};\frac{1}{2}\left(\frac{R_{12}^{(2)}}{R_{11}^{(2)}}\right)^{2}\phi_{2}^{2}\right)\right)
\end{align*}
}
One can calculate last integral via known formula below:
\[
\begin{aligned}2^{1-s}\int_{0}^{\infty}e^{-x^{2}/2}x^{2s-1}{}_{p}F_{q}\left(a_{1},\ldots,a_{p},b_{1},\ldots\right. & \left.,b_{q};\frac{a}{2}x^{2}\right)dx=\\
 & =\Gamma(s)\,_{p+1}F_{q}\left(s,a_{1},\ldots,a_{p};b_{1},\ldots,b_{q};a\right)
\end{aligned}
\]
\[
\begin{cases}
s_{1}=\frac{k+\alpha+1}{2}\\
a^{(1)}=\left(\frac{R_{12}^{(2)}}{R_{11}^{(2)}}\right)^{2}
\end{cases}
\]
\[
\begin{cases}
s_{2}=\frac{k+\alpha+2}{2}\\
a^{(2)}=\left(\frac{R_{12}^{(2)}}{R_{11}^{(2)}}\right)^{2}
\end{cases}
\]
Finally:{\scriptsize{}
\begin{align*}
I_{lk} & =2^{\frac{1}{2}(2\alpha+l+k-2)}\left(1+\left(-1\right)^{l+k}\right)\left(\left((-1)^{k}+1\right)\Gamma\left(\frac{1}{2}(l+\alpha+1)\right)\Gamma\left(\frac{1}{2}(k+\alpha+1)\right)\,_{2}F_{1}\left(\frac{1}{2}(k+\alpha+1),\frac{1}{2}(l+\alpha+1);\frac{1}{2};\left(\frac{R_{12}^{(2)}}{R_{11}^{(2)}}\right)^{2}\right)-\right.\\
 & -2\frac{R_{12}^{(2)}}{R_{11}^{(2)}}\left(1+(-1)^{k+1}\right)\Gamma\left(\frac{1}{2}(l+\alpha+2)\right)\Gamma\left(\frac{1}{2}(k+\alpha+2)\right)\,_{2}F_{1}\left(\frac{1}{2}(k+\alpha+2),\frac{1}{2}(l+\alpha+2);\frac{3}{2};\left(\frac{R_{12}^{(2)}}{R_{11}^{(2)}}\right)^{2}\right)
\end{align*}
}
\begin{align*}
I_{lk} & =2^{\frac{1}{2}(2\alpha+l+k-2)}\left(1+\left(-1\right)^{l+k}\right)\sum_{i=0}^{1}\left(1+(-1)^{k+i}\right)\left(-2\frac{R_{12}^{(2)}}{R_{11}^{(2)}}\right)^{i}\Gamma\left(\frac{1}{2}(l+\alpha+1+i)\right)\Gamma\left(\frac{1}{2}(k+\alpha+1+i)\right)\cdot\\
 & \cdot\,_{2}F_{1}\left(\frac{1}{2}(k+\alpha+1+i),\frac{1}{2}(l+\alpha+1+i);\frac{1}{2}+i;\left(\frac{R_{12}^{(2)}}{R_{11}^{(2)}}\right)^{2}\right)
\end{align*}
We can rename this sum as $S_{nk}^{(2)}$. Hence we receive:
\[
I_{nk}=2^{\frac{1}{2}(2\alpha+l+k-2)}\left(1+\left(-1\right)^{l+k}\right)S_{l,k}^{(2)}.
\]
Further:
\begin{align*}
Z_{I,2}[j(x)] & =\frac{g^{2}}{4\pi}\int dxdy\frac{e^{-\frac{1}{2}\bar{\phi}^{2}\left(x\right)R_{11}^{(2)}-\bar{\phi}\left(x\right)\bar{\phi}\left(y\right)R_{12}^{(2)}-\frac{1}{2}\bar{\phi}^{2}\left(y\right)R_{11}^{(2)}}}{\left(G(0:)\right)^{1+\alpha}\left(G(0)^{2}-G(x_{1}-x_{2})^{2}\right)^{-\alpha-1/2}}\cdot\\
 & \cdot\sum_{l,k=0}^{\infty}\frac{\chi^{l}\left(x\right)\chi^{k}\left(y\right)}{G(0)^{\frac{n+k}{2}}n!k!}\left(G(0)^{2}-G(x_{1}-x_{2})^{2}\right)^{\frac{l+k}{2}}2^{\frac{1}{2}(2\alpha+l+k-2)}\left(1+\left(-1\right)^{l+k}\right)S_{l,k}^{(2)}
\end{align*}
Finally we can substitute the expression for $R^{(2)}$:
\begin{align}
Z_{I,2}[j(x)] & =\frac{g^{2}G(0)^{\alpha}}{4\pi}\int dxdy\ \exp\left[-\frac{\frac{1}{2}\bar{\phi}^{2}\left(x\right)G\left(0\right)-\bar{\phi}\left(x\right)\bar{\phi}\left(y\right)G\left(x-y\right)+\frac{1}{2}\bar{\phi}^{2}\left(y\right)G\left(0\right)}{G^{2}\left(0\right)-G^{2}\left(x-y\right)}\right]\cdot\\
 & \cdot\sum_{l,k=0}^{\infty}\frac{\chi^{l}\left(x\right)\chi^{k}\left(y\right)G(0)^{\frac{l+k}{2}}}{l!k!}\left(1-\frac{G\left(x-y\right)^{2}}{G(0)^{2}}\right)^{\frac{2\alpha+l+k+1}{2}}2^{\frac{(2\alpha+l+k-2)}{2}}\left(1+\left(-1\right)^{l+k}\right)S_{l,k}^{(2)}.
\end{align}
Here:
\begin{align}
S_{l,k}^{(2)} & =\sum_{i=0}^{1}\left(1+(-1)^{k+i}\right)\left(-2\frac{G\left(x-y\right)}{G\left(0\right)}\right)^{i}\Gamma\left(\frac{1}{2}(l+\alpha+1+i)\right)\Gamma\left(\frac{1}{2}(k+\alpha+1+i)\right)\cdot\\
 & \cdot\,_{2}F_{1}\left(\frac{1}{2}(k+\alpha+1+i),\frac{1}{2}(l+\alpha+1+i);\frac{1}{2}+i;\left(\frac{G\left(x-y\right)}{G\left(0\right)}\right)^{2}\right)
\end{align}


\subsection{Vacuum Energy Computation at First Orders}

Now we proceed to the case of coefficient of $j(x)^{0}$ in the expansions obtained above. Let us note that in this case $\mathcal{Z}_{0}\left[0\right]=1$.

\subsubsection{Grand Canonical Partition Function}

\paragraph{0-Particles}

It is the trivial case. It gives Gaussian theory:
\[
Z_{0}\left[0\right]=1
\]

\paragraph{1-Particles}

Recall, that we start from:
\begin{equation}
Z_{I,1}[j(x)]=-\frac{2gG(0)^{\frac{\alpha}{2}}}{\left(2\pi\right)^{1/2}}\sum_{k=0}^{\infty}\frac{\Gamma\left(\frac{1}{2}(2k+\alpha+1)\right)}{(2k)!}2^{\frac{2k+1+\alpha}{2}}G(0)^{k}\int dxe^{-\frac{1}{2}\bar{\phi}^{2}\left(x\right)/G(0)}\chi^{2k}\left(x\right)
\end{equation}
Substituting $j=0$:
\begin{equation}
Z_{1}\left[0\right]=-V\frac{2^{\frac{3+\alpha}{2}}gG\left(0\right)^{\frac{\alpha}{2}}}{\left(2\pi\right)^{1/2}}\Gamma\left(\frac{1}{2}(\alpha+1)\right),
\end{equation}
where $V$ is a volume of the system. 

\paragraph{2-Particles}

We have, similarly:
\begin{align}
Z_{2}\left[0\right] & =Z_{2,I}\left[0\right]=\frac{g^{2}2^{\alpha-1}\Gamma^{2}\left(\frac{1}{2}(\alpha+1)\right)}{\pi}\int dxdy\frac{\left(G^{2}\left(0\right)-G^{2}\left(x-y\right)\right)^{\frac{2\alpha+1}{2}}}{\left(G\left(0\right)\right)^{\alpha+1}}\\
 & \times\,_{2}F_{1}\left(\frac{1}{2}(\alpha+1),\frac{1}{2}(\alpha+1);\frac{1}{2};\left(\frac{G\left(x-y\right)}{G\left(0\right)}\right)^{2}\right)=\left/x\to x+y,\ y\to x-y\right/\\
 & =\frac{g^{2}V2^{\alpha-1}\Gamma^{2}\left(\frac{1}{2}(\alpha+1)\right)}{\pi}\int dy\frac{\left(G^{2}\left(0\right)-G^{2}\left(y\right)\right)^{\frac{2\alpha+1}{2}}}{\left(G\left(0\right)\right)^{\alpha+1}}\,_{2}F_{1}\left(\frac{1}{2}(\alpha+1),\frac{1}{2}(\alpha+1);\frac{1}{2};\left(\frac{G\left(y\right)}{G\left(0\right)}\right)^{2}\right)
\end{align}


\subsubsection{Connected Diagrams Generating Functional}

Recall the formula for the connected contributions for $n=1$, since there are no integrals over $s_{ab}$ at all, we have:
\begin{equation}
\mathcal{G}_{1}[0]=Z_{1}[0]=-V\frac{2^{\frac{1+\alpha}{2}}g}{\left(2\pi\right)^{1/2}}\Gamma\left(\frac{1}{2}(\alpha+1)\right)\left(G\left(0\right)\right)^{\frac{\alpha}{2}},
\end{equation}
as well as for $n=2$:
\[
\mathcal{G}_{2}[j(x)]=\left\{ \int_{0}^{1}ds_{12}\partial_{s_{12}}^{\nu_{12}}\right\} Z_{2}[j(x),\nu_{ab}s_{ab}G_{ab}],
\]
since there is the only connected graph without loops and multiple edges on $n=2$ vertices. So we have:
\begin{equation}
\mathcal{G}_{2}[0]=\frac{g^{2}G(0)^{\alpha}V2^{\alpha-1}\Gamma^{2}\left(\frac{1}{2}(\alpha+1)\right)}{\pi}\int dy\left[\left(1-\frac{G(y)^{2}}{G(0)^{2}}\right)^{\alpha+\frac{1}{2}}{}_{2}F_{1}\left(\frac{1}{2}(\alpha+1),\frac{1}{2}(\alpha+1);\frac{1}{2};\left(\frac{G\left(y\right)}{G\left(0\right)}\right)^{2}\right)-1\right],
\end{equation}
where the region of integration in the last integral can be expanded to all $\mathbb{R}^{d}$ because of its convergence. We can also introduce the spherical coordinates for rotational-invariant theories:
\begin{align}
\mathcal{G}_{2}[0] & =\frac{g^{2}G(0)^{\alpha}V2^{\alpha-1}\Gamma^{2}\left(\frac{1}{2}(\alpha+1)\right)}{\pi}\frac{d\pi^{d/2}}{\Gamma(d/2+1)}\\
 & \times\int_{0}^{\infty}r^{d-1}dr\left[\left(1-\frac{G(r)^{2}}{G(0)^{2}}\right)^{\alpha+\frac{1}{2}}{}_{2}F_{1}\left(\frac{1}{2}(\alpha+1),\frac{1}{2}(\alpha+1);\frac{1}{2};\left(\frac{G\left(r\right)}{G\left(0\right)}\right)^{2}\right)-1\right]
\end{align}

And finally, recalling the definition of $E_{vac}$ and $\mathcal{G}$, we have first and second order expressions  for the vacuum energy:
\begin{equation}
E_{vac}=-\mathcal{G}_{1}[0]-\mathcal{G}_{2}[0]-....
\end{equation}
and for vacuum energy density:
\[
w_{vac}=\frac{2^{\frac{1+\alpha}{2}}g}{\left(2\pi\right)^{1/2}}\Gamma\left(\frac{1}{2}(\alpha+1)\right)\left(G\left(0\right)\right)^{\frac{\alpha}{2}}-\frac{g^{2}G(0)^{\alpha}V2^{\alpha-1}\Gamma^{2}\left(\frac{1}{2}(\alpha+1)\right)}{\pi}\frac{d\pi^{d/2}}{\Gamma(d/2+1)}
\]
\[
\times\int_{0}^{\infty}r^{d-1}dr\left[\left(1-\frac{G(r)^{2}}{G(0)^{2}}\right)^{\alpha+\frac{1}{2}}{}_{2}F_{1}\left(\frac{1}{2}(\alpha+1),\frac{1}{2}(\alpha+1);\frac{1}{2};\left(\frac{G\left(r\right)}{G\left(0\right)}\right)^{2}\right)-1\right]
\]
We will consider important particular cases in the following.

\subsubsection{Verification of the Obtained Formulas for $\alpha=2$ }

We can independently calculate generating functional for $\alpha=2$, which yields simply Gaussian integral. Using the result one can check the formulas for vacuum energy and correlators obtained above.

\paragraph{Exact Answer}

We start from path integral:
\[
Z\left[j\left(x\right)\right]=\int D\phi e^{-\frac{1}{2}\int d^{d}xd^{d}y\ L(x,y)\phi(x)\phi(y)-g\int\phi(x)^{2}dx-\int j\left(x\right)\phi\left(x\right)dx},
\]
which is well-known and equals to:
\begin{equation}
Z\left[j\left(x\right)\right]=\sqrt{\frac{\det\hat{G}^{-1}}{\det\left(\hat{G}^{-1}+g\right)}}e^{+\frac{1}{2}\left\langle \left(\hat{G}^{-1}+g\right)^{-1}j,j\right\rangle }.
\end{equation}
For connected diagrams generating functional we get therefore: 
\begin{align}
\mathcal{G}[j] & =\ln \mathcal{Z}\left[j\right]=\frac{1}{2}\ln\left(\det\hat{G}^{-1}\right)-\frac{1}{2}\ln\left(\det\hat{G}^{-1}\left(1+\hat{G}g\right)\right)+\frac{1}{2}\int\hat{G}(1+\hat{G}g)^{-1}j(x)j(y)dxdy\\
 & =-\frac{1}{2}\textrm{Tr}\left(\ln\left(1+\hat{G}g\right)\right)+\frac{1}{2}\sum_{k=0}^{\infty}\left(-g\right)^{k}\int\hat{G}^{k+1}j(x)j(y)dxdy,
\end{align}
where we used the formula $\ln\det A=tr\ \ln A$ and expanded the inverse operator in the series. Expanding the logarithm of an operator we obtain:
\begin{equation}
\mathcal{G}[j]=\frac{1}{2}\sum_{n=1}^{\infty}\frac{(-g)^{n}}{n}\textrm{Tr}\left(\hat{G}^{n}\right)+\frac{1}{2}\sum_{k=0}^{\infty}\left(-g\right)^{k}\int\hat{G}^{k+1}j(x)j(y)dxdy,
\end{equation}
and rewrite the power of operators in terms of integrals:
\begin{align}
\mathcal{G}[j] & =\frac{1}{2}\sum_{n=1}^{\infty}\frac{(-g)^{n}}{n}\int\prod_{a=1}^{n}dx_{a}G(x_{n}-x_{1})\prod_{k=1}^{n-1}G(x_{k}-x_{k+1})\\
 & +\frac{1}{2}\sum_{k=0}^{\infty}\left(-g\right)^{k}\int dy\prod_{a=1}^{n}dx_{a}G(x_{n}-x_{1})\prod_{k=1}^{n-1}G(x_{k}-x_{k+1})j(x_{n})j(y)
\end{align}
Similarly to the section $6.3$, we change variables and reduce the number of integrals by one:
\begin{equation}
\mathcal{G}[0]=\frac{1}{2}V\sum_{n=1}^{\infty}\frac{(-g)^{n}}{n}\left\{ \prod_{a=1}^{n-1}\int dt_{a}G(t_{a})\right\} G\left(\sum_{k=1}^{n-1}t_{k}\right).
\end{equation}
Now one can write density of vacuum energy:
\begin{equation}
w_{vac}=-\frac{\ln Z[0]}{V}=\frac{1}{2}\sum_{n=1}^{\infty}\frac{(-1)^{n+1}g^{n}}{n}\left\{ \prod_{a=1}^{n-1}\int dt_{a}G(t_{a})\right\} G\left(\sum_{k=1}^{n-1}t_{k}\right)
\end{equation}
It coincide with Legendre polynomial approximation for $\alpha=2$. It became exact due to the fact that $t^{2}$ lies in span of zero and first order of Legendre polynomials. Recall, that in section $6.3$ we got the formula:
\begin{equation}
w_{vac,N=1}=\sum_{n=1}^{\infty}\frac{(-1)^{n-1}}{2n}g^{n}\left\{ \prod_{a=1}^{n-1}\int dt_{a}G(t_{a})\right\} G\left(\sum_{k=1}^{n-1}t_{k}\right),
\end{equation}
which coincides with current computation.

\paragraph{Comparison of the Obtained Formulas for the First Terms for $\alpha=2$ with the Exact Ones}

Recall the formulas for $\mathcal{G}_{1,2}[0]$ from section $8.2$:
\begin{equation}
\mathcal{G}_{1}[0]=-V\frac{2^{\frac{1+\alpha}{2}}g}{\left(2\pi\right)^{1/2}}\Gamma\left(\frac{1}{2}(\alpha+1)\right)\left(G\left(0\right)\right)^{\frac{\alpha}{2}},
\end{equation}
\begin{align}
\mathcal{G}_{2}[0] & =\frac{g^{2}G(0)^{\alpha}V2^{\alpha-1}\Gamma^{2}\left(\frac{1}{2}(\alpha+1)\right)}{\pi}\frac{d\pi^{d/2}}{\Gamma(d/2+1)}\\
 & \times\int_{0}^{\infty}r^{d-1}dr\left[\left(1-\frac{G(r)^{2}}{G(0)^{2}}\right)^{\alpha+\frac{1}{2}}{}_{2}F_{1}\left(\frac{1}{2}(\alpha+1),\frac{1}{2}(\alpha+1);\frac{1}{2};\left(\frac{G\left(r\right)}{G\left(0\right)}\right)^{2}\right)-1\right]
\end{align}
and substitute $\alpha=2$ into them. Taking into account that:
\begin{equation}
_{2}F_{1}\left(\frac{1}{2}(\alpha+1),\frac{1}{2}(\alpha+1);\frac{1}{2};z^{2}\right)=\frac{2z^{2}+1}{\left(1-z^{2}\right)^{5/2}}
\end{equation}
we receive, that:
\begin{equation}
\mathcal{G}_{1}[0]=-gG\left(0\right)V,\qquad\mathcal{G}_{2}[0]=\frac{g^{2}V}{2}\int dxG\left(x\right)^{2},
\end{equation}
so for the first two orders of connected diagrams generating functional at zero source one gets:
\begin{equation}
\mathcal{G}[0]=-gG\left(0\right)V+\frac{g^{2}V}{2}\int dxG\left(x\right)^{2}+...
\end{equation}
which also coincides with directly obtained results.

\section{Research of the Physical Characteristics}

\subsection{Research of First Terms in Vacuum Energy}

Now we are going to calculate first terms of vacuum energy. Formerly we have obtained the general formulas:
\[
\mathcal{G}_{1}[0]=-2^{\frac{\alpha+1}{2}}\Gamma\left(\frac{1}{2}(\alpha+1)\right)gV\left(G\left(0\right)\right)^{\frac{\alpha}{2}}
\]
\[
\mathcal{G}_{2}[0]=2^{\alpha}\Gamma^{2}\left(\frac{1}{2}(\alpha+1)\right)g^{2}VG\left(0\right)^{\alpha}\int dy\left(\left(1-\frac{G^{2}\left(y\right)}{G^{2}\left(0\right)}\right)^{\alpha+\frac{1}{2}}{}_{2}F_{1}\left(\frac{1}{2}(\alpha+1),\frac{1}{2}(\alpha+1);\frac{1}{2};\left(\frac{G\left(y\right)}{G\left(0\right)}\right)^{2}\right)-1\right),
\]
and now we are going to substitute here particular kinds of propagators: Virton propagator (\cite{Efimov_problems}) as a typical nonlocal one and Klein-Gordon as a typical local one. Here and after we will consider a local theory as a certain limit of a nonlocal one, being its regularization. However, let us note that the nonlocal theory is of important independent interest.

\subsubsection{Nonlocal Case}

We start from typical nonlocal QFT propagator, which is the Virton propagator:
\begin{equation}
G(x)=G(0)e^{-m^{2}x^{2}},
\end{equation}
where $m$ is a ``mass''. We can substitute it in general formulas:
\begin{equation}
\mathcal{G}_{2}[0]=\frac{2^{\alpha}\Gamma^{2}\left(\frac{1}{2}(\alpha+1)\right)g^{2}VG\left(0\right)^{\alpha}d\pi^{d/2}}{m^{d}\Gamma\left(\frac{d}{2}+1\right)}\int t^{d-1}dt\left(\left(1-e^{-2t}\right)^{\alpha+\frac{1}{2}}{}_{2}F_{1}\left(\frac{1}{2}(\alpha+1),\frac{1}{2}(\alpha+1);\frac{1}{2};e^{-2t}\right)-1\right)
\end{equation}
Let us calculate integral numerically. Denote:
\begin{equation}
\psi(d,\alpha)=\int_{0}^{\infty}t^{d-1}dt\left(\left(1-e^{-2t}\right)^{\alpha+\frac{1}{2}}{}_{2}F_{1}\left(\frac{1}{2}(\alpha+1),\frac{1}{2}(\alpha+1);\frac{1}{2};e^{-2t}\right)-1\right)
\end{equation}

\begin{figure}
\begin{centering}
\includegraphics[width=8cm,height=6cm]{VZ2}
\par\end{centering}
\caption{Results of numerical computation of $\psi(d,\alpha)$}
\end{figure}
The plots of $\psi(d,\alpha)$ from $\alpha$ for different $d$ are represented in picture $3$. 

As a result, we arrive at the expressions:
\begin{equation}
\mathcal{G}_{2}[0]=\frac{2^{\alpha}\Gamma^{2}\left(\frac{1}{2}(\alpha+1)\right)g^{2}VG\left(0\right)^{\alpha}d\pi^{d/2}}{m^{d}\Gamma\left(\frac{d}{2}+1\right)}\psi(d,\alpha),
\end{equation}
and for the density of vacuum energy:
\begin{equation}
w_{vac}=\frac{E_{vac}}{V}=b_{1}\left(gG\left(0\right)^{\frac{\alpha}{2}}\right)-b_{2}\frac{g^{2}VG\left(0\right)^{\alpha}}{m^{d}}+...,
\end{equation}
where:
\[
b_{1}=2^{\frac{\alpha+1}{2}}\Gamma\left(\frac{1}{2}(\alpha+1)\right),\qquad b_{2}=\frac{2^{\alpha}\Gamma^{2}\left(\frac{1}{2}(\alpha+1)\right)\pi^{d/2}}{\Gamma\left(\frac{d}{2}+1\right)}\psi(d,\alpha)
\]
One can see that in contrary to the local case which we consider in next subsection the second term doesn't change the sign for $1\leq\alpha\leq2$. 

\subsubsection{Local Case}

Now let us consider the local QFT and calculate the first terms of vacuum energy for Klein-Gordon propagator. We consider the usual cut-off regularization of Gaussian theory Green function with Heaviside theta-function. Namely:
\[
G(x)=\int\frac{d^{d}k}{(2\pi)^{d}}\frac{e^{ikx}}{k^{2}+m^{2}}\theta\left(\mu-\left|k\right|\right),
\]
where $\mu\gg m$ is the cut-off parameter. We consider Klein-Gordon propagator for the cases of $d=2,3$.

For $G(0)$ for Klein-Gordon propagator in $d$ dimensions one have:
\begin{equation}
G(0)=\int\frac{\theta\left(\mu-\left|k\right|\right)}{k^{2}+m^{2}}\frac{d^{d}k}{\left(2\pi\right)^{3}}=\begin{cases}
\frac{1}{4\pi}\ln\left(\frac{\mu^{2}}{m^{2}}\right), & d=2\\
\frac{\mu}{2\pi^{2}}. & d=3
\end{cases}
\end{equation}
Hence for first order of vacuum energy we have in terms of cut-off:
\[
\mathcal{G}_{1}[0]=\begin{cases}
-2^{\frac{\alpha+1}{2}}\pi^{\frac{\alpha}{2}}gV\Gamma\left(\frac{1}{2}(\alpha+1)\right)\ln^{\frac{\alpha}{2}}\left(\frac{\mu^{2}}{m^{2}}\right) & d=2\\
-2^{\frac{3\alpha+1}{2}}\pi^{\frac{\alpha}{2}}\Gamma\left(\frac{1}{2}(\alpha+1)\right)gV\mu^{\frac{\alpha}{2}} & d=3
\end{cases},
\]
or in terms of $G(0)$, which is the initial form:
\[
\mathcal{G}_{1}[0]=-2^{\frac{\alpha+1}{2}}\Gamma\left(\frac{1}{2}(\alpha+1)\right)gV\left(G\left(0\right)\right)^{\frac{\alpha}{2}}
\]

Regarding the second order: Its integrand depends on the relation $\frac{G(x)}{G(0)}$, where $G(0)$ is an extremely large in cut-off parameter $\mu$. So one can find the value of the integral in the dominating order in $\mu$ by expanding the integrand in powers of $\frac{G(x)}{G(0)}$. So, in principal order in $\mu$:
\begin{align*}
\mathcal{G}_{2}[0] & \simeq2^{\alpha-1}\alpha^{2}\Gamma^{2}\left(\frac{1}{2}(\alpha+1)\right)g^{2}VG(0)^{\alpha}\int dy\left(\frac{G(y)}{G(0)}\right)^{2}
\end{align*}
We can easily calculate the remaining integral:
\begin{align}
\int dy\ G^{2}(y) & =\int dy\int\frac{d^{d}k_{1}d^{d}k_{2}}{\left(2\pi\right)^{2d}}e^{i\left(k_{1}+k_{2},y\right)}\frac{1}{\left(k_{1}^{2}+m^{2}\right)}\frac{1}{\left(k_{2}^{2}+m^{2}\right)}\nonumber \\
 & =\int\frac{d^{d}k_{1}d^{d}k_{2}}{\left(2\pi\right)^{d}}\delta\left(k_{1}+k_{2}\right)\frac{1}{\left(k_{1}^{2}+m^{2}\right)}\frac{1}{\left(k_{2}^{2}+m^{2}\right)}\nonumber \\
 & =\int\frac{d^{d}k}{\left(2\pi\right)^{d}}\frac{1}{\left(k^{2}+m^{2}\right)^{2}}\nonumber \\
 & =\begin{cases}
\frac{1}{2\pi}\int_{0}^{\infty}dk\frac{k}{\left(k^{2}+m^{2}\right)^{2}}=\frac{1}{4m^{2}\pi} & d=2\\
\frac{1}{2\pi^{2}}\int_{0}^{\infty}dk\frac{k^{2}}{\left(k^{2}+m^{2}\right)^{2}}=\frac{1}{8m\pi} & d=3
\end{cases}
\end{align}
Hence, we receive:
\begin{align}
\mathcal{G}_{2}[0] & \simeq\begin{cases}
2^{\alpha-3}\alpha^{2}\frac{1}{m^{2}\pi}\Gamma^{2}\left(\frac{1}{2}(\alpha+1)\right)g^{2}VG(0)^{\alpha-2} & d=2\\
2^{\alpha-4}\alpha^{2}\frac{1}{m\pi}\Gamma^{2}\left(\frac{1}{2}(\alpha+1)\right)g^{2}VG(0)^{\alpha-2} & d=3
\end{cases}
\end{align}

Next, we obtain the first terms of the corresponding decomposition by the coupling constant $g$ for the vacuum energy density $w_{vac}(g)$.
\begin{equation}
w_{vac}(g,m)m^{-d}=a_{1}\frac{gG(0)^{\alpha/2}}{m^{d}}+a_{2}\left(\frac{gG(0)^{\alpha/2}}{m^{d}}\right)^{2}\left(\frac{m^{d-2}}{G(0)}\right)^{2}+...
\end{equation}
\[
a_{1}=2^{\frac{\alpha+1}{2}}\Gamma\left(\frac{1}{2}(\alpha+1)\right),\qquad a_{2}=\frac{2^{\alpha-1}}{\pi}\Gamma^{2}\left(\frac{1}{2}(\alpha+1)\right)\begin{cases}
\frac{1}{4}, & d=2\\
\frac{1}{8}, & d=3
\end{cases}
\]
Here we have used the results of the following subsection on obtaining the local theory as a limit of a nonlocal one, and artificially have separated the parameters which have to be kept constant.

\subsection{Research of Approximation Formula with Legendre Polynomials of Second Degree}

Let us start from:
\begin{align*}
w_{\text{vac},N=1} & =\sum_{n=1}^{\infty}\frac{(-1)^{n-1}}{n}2^{n(\alpha/2-1)-2}\left[gG(0)^{\alpha/2}\right]^{n}\frac{\Gamma\left(\frac{n(\alpha+1)}{2}\right)}{\Gamma\left(\frac{n}{2}+n\right)}\frac{15\alpha}{\alpha^{2}+4\alpha+3}\\
 & \times\left\{ \prod_{a=1}^{n-1}\int dt_{a}\frac{G\left(t_{a}\right)}{G(0)}\right\} \frac{G\left(\sum_{k=1}^{n-1}t_{k}\right)}{G(0)}
\end{align*}
For $n=1$ there are no integration at all and one can write:
\[
w_{\text{vac},N=1}=\frac{15\alpha/4}{\alpha^{2}+4\alpha+3}\left(2^{\alpha/2-1}\frac{\Gamma\left(\frac{\alpha+1}{2}\right)}{\Gamma(3/2)}gG(0)^{\alpha/2}\right.
\]
\[
\left.+\sum_{n=2}^{\infty}\frac{(-1)^{n-1}}{n}2^{n(\alpha/2-1)}\left[gG(0)^{\alpha/2}\right]^{n}\frac{\Gamma\left(\frac{n(\alpha+1)}{2}\right)}{\Gamma\left(\frac{n}{2}+n\right)}\times\left\{ \prod_{a=1}^{n-1}\int dt_{a}\frac{G\left(t_{a}\right)}{G(0)}\right\} \frac{G\left(\sum_{k=1}^{n-1}t_{k}\right)}{G(0)}\right)
\]
In the following we will consider $n\geq2$. 

\paragraph{Nonlocal Case}

For the Virton propagator the integrals in series are Gaussian:
\begin{align*}
\left\{ \prod_{a=1}^{n-1}\int dt_{a}\frac{G\left(t_{a}\right)}{G(0)}\right\} \frac{G\left(\sum_{k=1}^{n-1}t_{k}\right)}{G(0)} & =\left\{ \prod_{a=1}^{n-1}\int dt_{a}\right\} e^{-2m^{2}\sum_{a=1}^{n-1}t_{a}^{2}-2m^{2}\sum_{k<l}^{n-1}t_{k}t_{l}}\\
 & =\frac{\left(2\pi\right)^{\frac{n-1}{2}}d}{\sqrt{n}m^{d(n-1)}}
\end{align*}
Hence:
\[
w_{\text{vac},N=1}=\frac{15\alpha/4}{\alpha^{2}+4\alpha+3}\left(2^{\alpha/2-1}\frac{\Gamma\left(\frac{\alpha+1}{2}\right)}{\Gamma(3/2)}gG(0)^{\alpha/2}\right.
\]
\[
\left.+\sum_{n=2}^{\infty}\frac{(-1)^{n-1}\left(2\pi\right)^{\frac{n-1}{2}d}}{n^{3/2}m^{d(n-1)}}2^{n(\alpha/2-1)}\left[gG(0)^{\alpha/2}\right]^{n}\frac{\Gamma\left(\frac{n(\alpha+1)}{2}\right)}{\Gamma\left(\frac{n}{2}+n\right)}\right)
\]
One can notice that in this case:
\[
w_{\text{vac},N=1}(m,z)=\frac{15\alpha/4}{\alpha^{2}+4\alpha+3}m^d\left[\frac{\Gamma\left(\frac{\alpha+1}{2}\right)}{\Gamma(3/2)}z+\frac{1}{\sqrt{2\pi}}h\left(2\pi z\right)\right]=\frac{15\alpha/4}{\alpha^{2}+4\alpha+3}m\tilde{h}(z),
\]
where $z=gG(0)^{\alpha/2}/m^d$, and $h$ is a regular function in a neighborhood of a zero argument.

%The numerical plot of $\tilde{h}(z)=\left(\frac{15\alpha/4}{\alpha^{2}+4\alpha+3}m\right)^{-1}w_{\text{vac},N=1}$ is presented on the picture ... .

\paragraph{Local Case}

We also will restrict ourselves to cases $d=2,3$. For Klein-Gordon propagator:
\[
G\left(t_{a}\right)=\int\frac{d^{d}k_{a}}{\left(2\pi\right)^{d}}\frac{1}{k_{a}^{2}+m^{2}}e^{i\left(k_{a},t_{a}\right)},
\]
we have already calculated $G(0)$. For integrals we have for $n\geq2$:
\begin{align*}
\left\{ \prod_{a=1}^{n-1}\int dt_{a}G\left(t_{a}\right)\right\} G\left(\sum_{k=1}^{n-1}t_{k}\right) & =\left\{ \prod_{a=1}^{n-1}\int dt_{a}\right\} \left(\prod_{a=1}^{n}\int\frac{d^{d}k_{a}}{\left(2\pi\right)^{d}}\frac{1}{k_{a}^{2}+m^{2}}\right)e^{i\sum_{a}\left(k_{a}+k_{n},t_{a}\right)}\\
 & =\left(\prod_{a=1}^{n}\int\frac{d^{d}k_{a}}{\left(2\pi\right)^{d}}\frac{1}{k_{a}^{2}+m^{2}}\right)\left\{ \prod_{a=1}^{n-1}\int dt_{a}\right\} e^{i\sum_{a}\left(k_{a}+k_{n},t_{a}\right)}\\
 & =\int\frac{d^{d}k_{n}}{\left(2\pi\right)^{d}}\frac{1}{\left(k_{n}^{2}+m^{2}\right)^{n}}\\
 & =\frac{d}{\left(4\pi\right)^{\frac{d}{2}}\Gamma\left(\frac{d}{2}+1\right)}\int_{0}^{\infty}r^{d-1}dr\frac{1}{\left(r^{2}+m^{2}\right)^{n}}\\
 & =\frac{d}{\left(4\pi\right)^{\frac{d}{2}}\Gamma\left(\frac{d}{2}+1\right)}\frac{\left(m^{2}\right)^{\frac{d}{2}-n}\Gamma\left(\frac{d}{2}\right)\Gamma\left(n-\frac{d}{2}\right)}{2\Gamma(n)}\\
 & =\frac{\left(m^{2}\right)^{\frac{d}{2}-n}\Gamma\left(n-\frac{d}{2}\right)}{\left(4\pi\right)^{\frac{d}{2}}\Gamma(n)}.
\end{align*}
This formula is applicable only for $d=2,3$ and $n\geq2$. We arrive at the following expressions:
\[
w_{\text{vac},N=1}=\frac{15\alpha/4}{\alpha^{2}+4\alpha+3}\left(2^{\alpha/2-1}\frac{\Gamma\left(\frac{\alpha+1}{2}\right)}{\Gamma(3/2)}gG(0)^{\alpha/2}\right.
\]
\[
\left.+\sum_{n=2}^{\infty}\frac{(-1)^{n-1}\left(m^{2}\right)^{\frac{d}{2}-n}}{\left(4\pi\right)^{d/2}n!}2^{n(\alpha/2-1)}\left[gG(0)^{\alpha/2-1}\right]^{n}\frac{\Gamma\left(\frac{n(\alpha+1)}{2}\right)\Gamma\left(n-\frac{d}{2}\right)}{\Gamma\left(\frac{n}{2}+n\right)}\right)
\]
In the following subsection we will study the way to scale parameters of a nonlocal theory to obtain a non-trivial local limit. Looking ahead, we announce that one have to keep the following parameters constant:
\[
z = \frac{gG(0)^{\alpha/2}}{m^{d}}=const \qquad\frac{m^{d-2}}{G(0)}=\xi=const
\]
So, in these terms the vacuum energy density takes form:
\[
w_{\text{vac},N=1}m^{-d}=\frac{15\alpha/4}{\alpha^{2}+4\alpha+3}\left(2^{\alpha/2-1}\frac{\Gamma\left(\frac{\alpha+1}{2}\right)}{\Gamma(3/2)}z+\sum_{n=2}^{\infty}\frac{(-1)^{n-1}2^{n(\alpha/2-1)}}{\left(4\pi\right)^{d/2}n!}\left(z\xi\right)^{n}\frac{\Gamma\left(\frac{n(\alpha+1)}{2}\right)\Gamma\left(n-\frac{d}{2}\right)}{\Gamma\left(\frac{n}{2}+n\right)}\right)
\]

\subsection{Local Theory as a Limit of Nonlocal}

Here we are going to explore the form of a generic term of polynomial expansion to understand whether it is possible to scale nonlocal theory in some way to get a local one. 

According to the method described above, we have for $n$-particle term and monomial of degree $2i$ with $\sum_{a}l_{aa}=2j$ for some $j$ the corresponding term in the energy density of the form:
\[
\#\frac{\left(gG(0)^{\alpha/2}\right)^{n}}{V}\int dx_{1}...dx_{n}\prod_{a<b}\frac{G(x_{a}-x_{b})^{l_{ab}}}{G(0)^{l_{ab}}}=\#\frac{\left(gG(0)^{\alpha/2}\right)^{n}}{m^{-d}}\frac{m^{(d-2)(in-q)}}{G(0)^{in-q}}m^{-dn},
\]
where $\#$ is a some numerical coefficient depending on term. We have written this expression from dimension considerations, assuming that all integrals converge, which will be discussed later and in this case $m$ is the only dimensional parameter in the integrals. Let us rewrite the expression as follows:
\[
\#\frac{\left(gG(0)^{\alpha/2}\right)^{n}}{V}\int dx_{1}...dx_{n}\prod_{a<b}\frac{G(x_{a}-x_{b})^{l_{ab}}}{G(0)^{l_{ab}}}=\#m^{d}\left(\frac{gG(0)^{\alpha/2}}{m^{d}}\right)^{n}\cdot\left(\frac{m^{d-2}}{G(0)}\right)^{in-q},
\]
which coincides with already received formula for $i=1$ (there was $q=0$ for every term with non-zero contribution). So for the vacuum energy density we have following expression, taking into the account also the
term $n=1$:
\[
w_{vac}m^{-d}=\psi\left(\frac{gG(0)^{\alpha/2}}{m^{d}},\frac{m^{d-2}}{G(0)}\right),
\]
for some regular function $\psi$. In order to get a non-trivial local limit one should scale $m$, $g$ and $G(0)$ so that:
\[
z = \frac{gG(0)^{\alpha/2}}{m^{d}}=const \qquad\frac{m^{d-2}}{G(0)}=\xi=const
\]
The physical sense of the first parameter is an ``effective coupling energy'' in a characteristic volume $m^d$ of a theory, as well the second one is a ``relative'' $G(0)$.

However, if one wants to increase the degree of approximation, the renormalizability problem of the theory has to be solved. It corresponds exactly to the case of diverging integrals in the formulas above, so the other dimensional parameter appears, which is the cut-off $\mu$.
Recall the common-known formulas for superficial degree of divergence:
\[
D=d+\left[2i\left(\frac{d-2}{2}\right)-d\right]V-\left(\frac{d-2}{2}\right)M,
\]
where $d$ is a space dimension, $2i$ is a power of interaction, $V$ is the number of vertices and $M$ is a number of external lines. 

So one can write that in $d=2$ all the power theories are renormalizable and in $d=3$ only the power theories with $\leq5$ are renormalizable. This means, while considering the local theories with the described method of polynomial
approximation, one can't increase the power of approximation infinitely for general dimension, except for $d=2$. This means that in $d=3$ there is no correct local limit of a monomials with degree higher than $4$, since we have only the even degrees in expansion.

The conclusion is that being applied to local theories in $d>2$ our method has the limited precision of degree $4$ in $d=3$ and degree $2$ in $d\geq4$. Though, even the second degree approximation can provide quite a small error, as the example of hard-sphere gas will show.

\subsection{Resummation of Second Degree Approximate Formulas}

We have obtained the explicit expressions for the second degree approximation. Well, it is not very convenient to use power series around $0$ to compute the value of a function in a point which is not close to $0$. The possible way to overcome this obstacle is the resummation of the series into some other function, and then substitute the relevant argument into it. Unfortunately, the obtained series are not expressed in terms of simple functions, through they are expressed via generalized hypergeometric functions with complicated coefficients, which are not match for numerical calculations at all. Here we provide a simple and useful method of such a resummation.

\begin{figure}
\begin{centering}
\includegraphics[width=8cm,height=5cm]{virt_en1}\includegraphics[width=8cm,height=5cm]{virt_en2}
\par\end{centering}
\caption{Qualitative plots of Virton vacuum energy density $w_{vac}(z)$ in second degree polynomial approximation for two different ranges of $z$, where $z=\frac{gG(0)^{\alpha/2}}{m^d}$ }
\end{figure}

The key idea is to replace the relation of gamma-functions by a single gamma-function, some power factor and exponential factor, rescaling the argument, using Stirling formula. This method is accurate enough since Stirling formula has small error even for unit argument. Moreover, we will use it for indices $n\geq3$ to additionally avoid approximation errors. It is worth noting that the threshold number for small error depends on the value of $\alpha$. 

After such a transformation a form of series terms simplifies significantly, and the series is much more likely to be summed analytically. Namely, we have the approximation of the form:
\[
\sum_{n}\frac{\Gamma\left(\frac{n(\alpha+1)}{2}\right)}{\Gamma\left(\frac{n}{2}+n\right)n^{3/2}}\tilde{z}^{n}\approx\#\sum_{n}\frac{1}{\Gamma\left(n(1-\alpha/2)\right)n^{k}}w^{n},
\]
where $w\sim\tilde{z}$. Here the argument of Gamma-function in the R.H.S. is adjusted from the condition for every term to have the equal growth rate in $n^{n}$. The remaining factors, arising from Stirling formula, go to $n^{k}$ or $w$. We illustrate this approximation on few examples. 

One can make an exact transformation, adding the residual of R.H.S. and L.H.S. of this approximation, so computation of error will reduce to summation of power series with error of Stirling formula inside the coefficients.

\paragraph{Nonlocal Case}

We start from the series with Virton propagator:
\[
T_{v}(u)=\sum_{n=3}^{\infty}\frac{(-1)^{n-1}}{n^{3/2}}\frac{\Gamma\left(\frac{n(\alpha+1)}{2}\right)}{\Gamma\left(\frac{n}{2}+n\right)}u^{n},\qquad u:=(2\pi)^{1/2}2^{\alpha/2-1}\frac{gG(0)^{\alpha/2}}{m^{d}},
\]
where we have started from $n=3$ for better precision. We know that the asymptotic of gamma-function, that is defined by Stirling formula, accurate enough from $n\approx2$, hence we can find a simple expression for the ratio of two gamma-functions:
\[
\frac{\Gamma\left(\frac{n(\alpha+1)}{2}\right)}{\Gamma\left(\frac{n}{2}+n\right)}\approx\frac{2\sqrt{\pi}3^{\frac{1}{2}-\frac{3n}{2}}(2-\alpha)^{n-\frac{\alpha n}{2}}(\alpha+1)^{\frac{1}{2}(\alpha+1)n}}{\sqrt{-\alpha^{2}+\alpha+2}}\frac{1}{\Gamma\left(\left(1-\frac{\alpha}{2}\right)n\right)\sqrt{n}}=a\frac{\beta^{n}}{\Gamma\left(\left(1-\frac{\alpha}{2}\right)n\right)\sqrt{n}},
\]
for 
\[
\beta=3^{-\frac{3}{2}}(2-\alpha)^{1-\frac{\alpha}{2}}(\alpha+1)^{\frac{1}{2}(\alpha+1)},\qquad a=\frac{2\sqrt{\pi}3^{\frac{1}{2}}}{\sqrt{-\alpha^{2}+\alpha+2}}
\]
One can substitute it to the vacuum energy density:
\[
T_{v}(z)\approx a\sum_{n=3}^{\infty}\frac{(-1)^{n}}{n^{2}}\frac{\beta^{n}u^{n}}{\Gamma\left(\left(1-\frac{\alpha}{2}\right)n\right)}
\]

Fortunately, such series can be calculated analytically for particular values of $\alpha$, namely the rational ones with not very great denominators, such as $\alpha=4/3,3/2,5/3,5/4,...$. For $\alpha=4/3$ this series is equal to:
\[
T_{v}(u)\approx a\left[-\frac{3\beta^{5}z^{5}\,_{2}F_{2}\left(1,\frac{5}{3};\frac{8}{3},\frac{8}{3};-u^{3}\beta^{3}\right)}{50\Gamma\left(\frac{2}{3}\right)}+\frac{3\beta^{4}z^{4}\,_{2}F_{2}\left(1,\frac{4}{3};\frac{7}{3},\frac{7}{3};-u^{3}\beta^{3}\right)}{16\Gamma\left(\frac{1}{3}\right)}+\frac{1}{9}\left(\text{Ei}\left(-u^{3}\beta^{3}\right)-3\log(\beta u)-\gamma\right)\right]
\]
So we have obtain the analytical formula for not necessary small values of $z$:
\[
w_{\text{vac},N=1}=m\frac{15\alpha/4}{\alpha^{2}+4\alpha+3}\left(2^{\alpha/2-1}\frac{\Gamma\left(\frac{\alpha+1}{2}\right)}{\Gamma(3/2)}z-\frac{\sqrt{\pi}}{16}2^{\alpha}\Gamma\left(\alpha+1\right)z^{2}\right.
\]
\[
\left.+\frac{a}{\sqrt{2\pi}}\sum_{n=2}^{\infty}\frac{(-1)^{n-1}\left(2\pi\right)^{\frac{n}{2}}}{n^{2}}\frac{2^{n(\alpha/2-1)}\beta^{n}}{\Gamma\left(\left(1-\frac{\alpha}{2}\right)n\right)}z^{n}\right),
\]
which is valid for all ``good'' $\alpha$. Here we have denoted $z=\frac{gG(0)^{\alpha/2}}{m}$. In particular case of $\alpha=4/3$:
\[
w_{\text{vac},N=1}=m\frac{15\alpha/4}{\alpha^{2}+4\alpha+3}\left(2^{\alpha/2-1}\frac{\Gamma\left(\frac{\alpha+1}{2}\right)}{\Gamma(3/2)}z-\frac{\sqrt{\pi}}{16}2^{\alpha}\Gamma\left(\alpha+1\right)z^{2}\right.
\]
\[
\left.+\frac{a}{\sqrt{2\pi}}\left[-\frac{3z^{5}\,_{2}F_{2}\left(1,\frac{5}{3};\frac{8}{3},\frac{8}{3};-u^{3}\right)}{50\Gamma\left(\frac{2}{3}\right)}+\frac{3\beta^{4}z^{4}\,_{2}F_{2}\left(1,\frac{4}{3};\frac{7}{3},\frac{7}{3};-u^{3}\right)}{16\Gamma\left(\frac{1}{3}\right)}+\frac{1}{9}\left(\text{Ei}\left(-u^{3}\right)-3\log(u)-\gamma\right)\right]\right),
\]
for $u=2^{-1/3}\sqrt{2\pi}\beta z$.

The qualitative graph of such expression can be found in picture. One can see that Virton energy has a point of inflection. From this plot one can see that for Virton matter there is quite a wide range of temperatures for which their pressure ranges very slowly up and down around average value, which is the interesting property of Virton
model.

\paragraph{Local Case}

Let us repeat the calculations of the previous subsection for the local case. Namely, we start from the series expansion: 
\[
T_{KG}(u)=\sum_{n=3}^{\infty}\frac{(-1)^{n-1}}{n!}\frac{\Gamma\left(\frac{n(\alpha+1)}{2}\right)\Gamma\left(n-\frac{d}{2}\right)}{\Gamma\left(\frac{n}{2}+n\right)}u^{n}.
\]
Here $u=2^{\alpha/2-1}\left[gG(0)^{\alpha/2}\right]\left[m^{d}G(0)\right]^{-1}=2^{\alpha/2-1}z/\xi$.
We know that asymptotic of gamma-function, that is defined by Stirling formula, accurate enough from $n>2$, hence we can find a simple expression for the ratio of two gamma-functions:
\[
\frac{\Gamma\left(\frac{n(\alpha+1)}{2}\right)\Gamma\left(n-\frac{d}{2}\right)}{n!\Gamma\left(\frac{n}{2}+n\right)}\approx2\sqrt{\pi}3^{\frac{1}{2}-\frac{3n}{2}}(2-\alpha)^{-\frac{\alpha n}{2}+n-\frac{1}{2}}(\alpha+1)^{\frac{1}{2}(\alpha n+n-1)}\frac{1}{n^{\frac{d+3}{2}}\Gamma\left(\left(1-\frac{\alpha}{2}\right)n\right)}
\]
Thus, we have:
\[
T_{KG}(u)\approx a\sum_{n=3}^{\infty}\frac{(-1)^{n-1}}{n^{\frac{d+3}{2}}}\frac{(\beta u)^{n}}{\Gamma\left(\left(1-\frac{\alpha}{2}\right)n\right)},
\]
for $a=2\sqrt{\pi}3^{\frac{1}{2}}(2-\alpha)^{-\frac{1}{2}}(\alpha+1)^{-\frac{1}{2}}$ and $\beta=3^{-\frac{3}{2}}(2-\alpha)^{-\frac{\alpha}{2}+1}(\alpha+1)^{\frac{1}{2}(\alpha+1)}$. For generic $\alpha$ we have:
\[
w_{\text{vac},N=1}m^{-d}=\frac{15\alpha/4}{\alpha^{2}+4\alpha+3}\left(2^{\alpha/2-1}\frac{\Gamma\left(\frac{\alpha+1}{2}\right)}{\Gamma(3/2)}z-\frac{2^{\alpha-4}\sqrt{\pi}}{(4\pi)^{d/2}}\Gamma(\alpha+1)z\xi\right.
\]
\[
\left.+a\sum_{n=3}^{\infty}\frac{(-1)^{n-1}}{n^{\frac{d+3}{2}}}2^{n(\alpha/2-1)}\frac{(\beta z\xi)^{n}}{\Gamma\left(\left(1-\frac{\alpha}{2}\right)n\right)}\right),
\]
for the variables $z=\frac{gG(0)^{\alpha/2}}{m^{d}}$ and $\xi=\frac{m^{d-2}}{G(0)}$, which are kept constant while passing to a local limit.

As in the nonlocal case, this expression can be resummed in quite simple form for particular values of $\alpha$. Namely, for $\alpha=4/3$:
\[
\sum_{n=3}^{\infty}\frac{(-1)^{n-1}}{n^{\frac{d+3}{2}}}\frac{(\beta u)^{n}}{\Gamma\left(\left(1-\frac{\alpha}{2}\right)n\right)}=\frac{1}{27}\beta^{3}u^{3}\,_{3}F_{3}\left(1,1,1;2,2,2;-u^{3}\beta^{3}\right)+
\]
\[
+\frac{3\beta^{5}u^{5}\,_{3}F_{3}\left(1,\frac{5}{3},\frac{5}{3};\frac{8}{3},\frac{8}{3},\frac{8}{3};-u^{3}\beta^{3}\right)}{250\Gamma\left(\frac{2}{3}\right)}-\frac{3\beta^{4}u^{4}\,_{3}F_{3}\left(1,\frac{4}{3},\frac{4}{3};\frac{7}{3},\frac{7}{3},\frac{7}{3};-u^{3}\beta^{3}\right)}{64\Gamma\left(\frac{1}{3}\right)},
\]
so for vacuum energy density we can write:
\[
w_{\text{vac},N=1}m^{-d}=\frac{45}{91}\left(\frac{\Gamma\left(7/6\right)}{2^{1/3}\Gamma(3/2)}z-\frac{\sqrt{\pi}\Gamma(7/3)}{2^{8/3}(4\pi)^{d/2}}z\xi+\frac{a(4/3)}{27}\beta^{3}u^{3}\,_{3}F_{3}\left(1,1,1;2,2,2;-\beta^{3}\frac{z^{3}\xi^{3}}{2}\right)\right.
\]
\[
\left.+a(4/3)\left[\frac{2^{-5/3}3\beta^{5}z^{5}\xi^{5}\,_{3}F_{3}\left(1,\frac{5}{3},\frac{5}{3};\frac{8}{3},\frac{8}{3},\frac{8}{3};-\frac{\beta^{3}}{2}z^{3}\xi^{3}\right)}{250\Gamma\left(\frac{2}{3}\right)}-\frac{2^{-4/3}3\beta^{4}z^{4}\xi^{4}\,_{3}F_{3}\left(1,\frac{4}{3},\frac{4}{3};\frac{7}{3},\frac{7}{3},\frac{7}{3};-\frac{\beta^{3}}{2}z^{3}\xi^{3}\right)}{64\Gamma\left(\frac{1}{3}\right)}\right]\right)
\]

\subsection{Hard-Sphere Gas Approximation}

The general formula for hard-sphere gas approximation is not very useful for analytical calculations since of its cumbersomeness. For this reason, we write explicitly the first few terms. For the sake of simplicity we consider the contribution of the complete graphs solely. This formula is also quite convenient and simple for numerical computations. Unfortunately, we have to deal with power series in system parameters, so it requires accurate work with precision of computations to distinguish errors
from real results. We make the truncation of both series in $w_{vac}$, namely $n_{0}$ and $N$, and using a numerical experiment explore the effect of $n_{0}$ and $N$ on the results. We study the case of $\alpha=4/3$ as well as in the previous subsection, and one can check that there
will no qualitative differences for other $\alpha\in[1;2]$.

\begin{figure}
\begin{centering}
\includegraphics[width=9cm,height=5cm]{hard_sphere_comp_1}\includegraphics[width=8cm,height=5cm]{hard_sphere_comp_2}
\par\end{centering}
\caption{Comparative plots of approximations of $\epsilon_{vac}$ for $\gamma=0.5$ and different values of truncations $(n_0,N)$ for the complete graph only taken into the account}

\end{figure}

Recall that in this approach we approximated the Green function of Gaussian theory by Heaviside-function:
\[
G_{ab}=\left(\gamma+\left(1-\gamma\right)\delta_{ab}\right)G(0)\theta(\delta-|x_{a}-x_{b}|),
\]
for $\gamma$ and $\delta$ chosen in some way. Let us introduce new variables:
\[
z=gG(0)^{\alpha/2}v,\qquad\epsilon_{vac}=
w_{vac}v(z,\gamma),
\]
where $v$ is a volume of $d$-dimensional ball of radius $\delta$ and $\epsilon_{vac}$ is a vacuum energy contained in one ``hard-ball''. Further we can plot the dependency $\epsilon_{vac}(\gamma,z)$, since for this kind of approximation $\gamma$ and $z$ are more natural parameters than the other ones from Gaussian theory Green function. 

We start from learning the influence of the truncation $n_{0}$ and $N$ which we do to avoid infinite series. For the simplification of the analysis we substitute $\gamma=0.5$ as a typical value of this parameter. It is not necessary to consider different $\gamma$ since the set if its possible values is bounded and $|\gamma|\leq1$. The plots for different truncations are presented in Figure $5$ with corresponding labels. As a result one can see that even at $z\approx10$ series converges rapidly enough, even for approximation $n_{0}=3,$ $N=6$, which means approximation up to $3$-particle interactions with polynomial of $12$-th degree. One also can see that lowering of $N$ downs the plot as well as increasing of $n_{0}$ raises the plot. For minimal-efforts and maximal precision approximation one can take $n_{0}=N$, for instance, $(3,3)$-case, which is quite a good fitting of other results of even higher degrees.

So we argue that quite accurate and simple approximation for complete graph contribution for $z\in[0;20]$ and $\gamma\in[0;1]$ will be the following expression for $\alpha=4/3$:
\[
\epsilon_{vac}(\gamma,z)\approx0.00550077\gamma^{9}z^{3}+0.0185651\gamma^{8}z^{3}+0.0284665\gamma^{7}z^{3}-0.0352499\gamma^{6}z^{3}-
\]
\[
-0.0195409\gamma^{6}z^{2}-0.0890649\gamma^{5}z^{3}+0.0112736\gamma^{4}z^{2}+0.298303\gamma^{3}z^{3}-0.325141\gamma^{2}z^{2}+0.848084z
\]
and exactly this expression is the most convenient for use. For general $\alpha$ this expression is too large.

Now it is curious to plot $2$-dimensional surface of $\epsilon_{vac}(\gamma,z)$ for $n_{0}=3$ and $N=3$. The results are presented in in Figure $6$.

\begin{figure}
\begin{centering}
\includegraphics[width=9cm,height=7cm]{hard_sphere_2D}
\par\end{centering}
\caption{Surface of $\epsilon_{vac}(\gamma,z)$ for $n_{0}=3$ and $N=3$}

\end{figure}


\part*{Conclusion}

In this paper the theory with fraction-power interaction $g|\phi|^{\alpha}$ was considered. We provided comprehensive research of its properties, namely:
\begin{enumerate}
\item Recipe of the obtaining perturbation series for generating functional
$\mathcal{Z}\left[j\right]=\sum_{n}Z_{n}[j]$ in powers of coupling constant $g$, proof of its convergence and explicit formulas for connected contributions
$\mathcal{G}[j]=\ln \mathcal{Z}\left[j\right]=\sum_{n}\mathcal{G}_{n}[j]$; 
\item Exact computation of vacuum energy density with accuracy of $O(g^{3})$ and verification for $\alpha=2$ with
considering particular cases of Virton and Euclidean Klein-Gordon propagators;
\item Calculable and converging approximations for $Z_{n}[j]$ and $\mathcal{G}_{n}[j]$ with any precision, obtained with polynomial approximation of function
$f(t)=|t|^{\alpha}$in $[-1;1]$;
\item Simple and resummable approximation formulas of second degree approximation for vacuum energy density $w_{vac}$, connecting weak and strong coupling
regimes, and their research;
\item Hard-sphere gas approximation for $Z_{n}[j]$ and $\mathcal{G}_{n}[j]$ and its research.
\end{enumerate}
There were made multiple checks and the comparing of the obtained results. The provided research of field theories with interaction $g|\phi|^{\alpha}$ may shed light on the strong coupling behavior of $\phi^{4}$-theory \textbf{providing such a definition that the latter is not trivial for} $d\geq4$. Besides, the research carried out in the paper produces new techniques of treating non-polynomial potentials and nonlocal field theories.

\newpage{}

\bibliographystyle{unsrtnat}
\nocite{*}
\bibliography{FractionalQFT}

\section*{}

\end{document}
