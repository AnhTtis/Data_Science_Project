\begin{abstract}
    Federated learning (FL) has been promoted as a popular technique for training machine learning (ML) models over edge/fog networks. 
    Traditional implementations of FL have largely neglected the potential for inter-network cooperation, treating edge/fog devices and other infrastructure participating in ML as separate processing elements. Consequently, FL has been vulnerable to several dimensions of network heterogeneity, such as varying computation capabilities, communication resources, data qualities, and privacy demands.
    % , which can enhance and push edge/fog networks beyond a simple sum of their constituent devices' capabilities. 
    % By neglecting cooperative opportunities, current FL implementations separate edge/fog network devices and infrastructure from each other, and thus suffer from diverse forms of network heterogeneity, such as varying computation capabilities, communication resources, data qualities, and privacy demands. 
    {\color{black}We advocate for cooperative federated learning (CFL), a cooperative edge/fog ML paradigm built on device-to-device (D2D) and device-to-server (D2S) interactions. Through D2D and D2S cooperation, CFL counteracts network heterogeneity in edge/fog networks through enabling a model/data/resource pooling mechanism, which will yield substantial improvements in ML model training quality and network resource consumption.}
    % {\color{black} We advocate for cooperative federated learning (CFL), a cooperative edge/fog paradigm built on device-to-device (D2D) and device-to-server (D2S) interactions.} 
    % {\color{black} Through D2D and D2S cooperation, CFL counteracts network heterogeneity in edge/fog through enabling a model/data/resource pooling mechanism over the network, which yields substantial improvements in ML model training quality and network resource consumption.} 
    %savings in network resources through careful design of the associated cooperation mechanics.
    % Through careful design of cooperation mechanics among network devices and infrastructure, cooperative edge/fog can be further exploited to integrate unlabeled data and heterogeneous device privacy. 
    We propose a set of core methodologies that form the foundation of D2D and D2S cooperation and present preliminary experiments that demonstrate their benefits. We also discuss new FL functionalities enabled by this cooperative framework such as the integration of unlabeled data and heterogeneous device privacy into ML model training. %privacy restriction diversity 
    %showing the promise of a cooperative edge/fog paradigm for network intelligence applications. 
    % For each pillar, we propose a set of core technologies, embedded within a network schematic, to enable ML training  % present a network schematic embedded with a set of proposed technologies designed to enhance resource and ML training utility. 
    % We also demonstrate how such edge/fog cooperation can enable FL to adapt to scenarios involving unlabeled data. 
    Finally, we describe some open research directions at the intersection of cooperative edge/fog and FL. 
    % some enticing implications at the intersection of cooperative edge/fog and FL.  %of joining cooperative edge/fog and FL. 
    % future applications of FL operating on cooperative edge/fog networks. 
    
    % These shortcomings are further exacerbated by multi-structural heterogeneity among network devices, composed of varying computation capabilities, communication resources, data qualities, and privacy demands. 
    % Since federated learning frameworks are innately designed with diverse network heterogeneity in mind, we envision the incorporation of device-to-device and device-to-server cooperation via model/data/resource pooling. This can be especially useful to improve not only resource efficiency and machine learning classification performance but also integration of unlabeled data and eventually generalize FL in diverse network settings and scenarios. 
    
    % The growth of Internet of Things (IoT) devices and networking, in particular edge and fog networks, has been in tandem with the popularity of federated learning (FL). 
    % While FL has proven effective in developing network-wide machine intelligence in distributed networks such as those seen in the edge/fog networking, the key property of edge/fog networking - namely inter-network cooperation - 
    % Federated learning has proven valuable in developing machine intelligence in large-scale heterogeneous networks while preserving a degree of data privacy via device independence. 
    % However, modern edge and fog networks, and indeed the entire IoT paradigm, are such that the whole network is greater than the sum of its parts - by connection to each other, devices and the network are together able to derive greater benefits. 
    % Since federated learning frameworks are innately designed with diverse network heterogeneity in mind, we envision the incorporation of device-to-device and device-to-server cooperation via model/data/resource pooling. This can be especially useful to improve not only resource efficiency and machine learning classification performance but also integration of unlabeled data and eventually generalize FL in diverse network settings and scenarios. 
    % Our preliminary work has developed the initial formulations and specifically demonstrated that XXX.
\end{abstract}