\documentclass[conference]{IEEEtran}
\IEEEoverridecommandlockouts
\usepackage{cite}
\usepackage{amsmath,amssymb,amsfonts}
\usepackage{algorithmic}
\usepackage{graphicx}
\usepackage{textcomp}
\usepackage{xcolor}
\usepackage{array}
\usepackage{color,soul}
\def\BibTeX{{\rm B\kern-.05em{\sc i\kern-.025em b}\kern-.08em
    T\kern-.1667em\lower.7ex\hbox{E}\kern-.125emX}}
\begin{document}

\title{A Novel Neural Network Approach for Predicting the Arrival Time of Buses for Smart On-Demand Public Transit}


\author{\IEEEauthorblockN{1\textsuperscript{st} Narges Rashvand}
\IEEEauthorblockA{\textit{Department of Electrical and Computer Engineering} \\
\textit{The University of North Carolina at Charlotte}\\
Charlotte, NC, USA \\
nrashvan@uncc.edu}
\and
\IEEEauthorblockN{2\textsuperscript{nd} Sanaz Sadat Hosseini}
\IEEEauthorblockA{\textit{Department of Civil and Environmental Engineering} \\
\textit{The University of North Carolina at Charlotte}\\
Charlotte, NC, USA \\
shossei7@uncc.edu}
\and
\IEEEauthorblockN{3\textsuperscript{rd} Mona Azarbayjani}
\IEEEauthorblockA{\textit{Department of Architecture} \\
\textit{The University of North Carolina at Charlotte}\\
Charlotte, NC, USA \\
mazarbay@uncc.edu}
\and
\IEEEauthorblockN{4\textsuperscript{th} Hamed Tabkhi}
\IEEEauthorblockA{\textit{Department of Electrical and Computer Engineering} \\
\textit{The University of North Carolina at Charlotte}\\
Charlotte, NC, USA \\
htabkhiv@uncc.edu}
}

\maketitle

\begin{abstract}
Among the major public transportation systems in cities, bus transit has its problems, including more accuracy and reliability when estimating the bus arrival time for riders. This can lead to delays and decreased ridership, especially in cities where public transportation is heavily relied upon. A common issue is that the arrival times of buses do not match the schedules, resulting in latency for fixed schedules. According to the study in this paper on New York City bus data, there is an average delay of around eight minutes or 491 seconds mismatch between the bus arrivals and the actual scheduled time. This research paper presents a novel AI-based data-driven approach for estimating the arrival times of buses at each transit point (station). Our approach is based on a fully connected neural network and can predict the arrival time collectively across all bus lines in large metropolitan areas. Our neural-net data-driven approach provides a new way to estimate the arrival time of the buses, which can lead to a more efficient and smarter way to bring the bus transit to the general public. Our evaluation of the network bus system with more than 200 bus lines, and 2 million data points, demonstrates less than 40 seconds of estimated error for arrival times. The inference time per each validation set data point is less than 0.006 ms.  
\end{abstract}

\begin{IEEEkeywords}
Bus Arrival Time, Public Transit, Smart City, AI-based Approach, Neural Networks 
\end{IEEEkeywords}


\section{Introduction}
%\hl{you do need more citations to support your arguments}
Over the past fifty years in the US, there has been a decline in the proportion of workers who use public transportation to commute to work. This drop is primarily attributed to the government's decision to separate land-use development planning and transportation, resulting in suburban sprawl, uneven distribution of public services, and a growing reliance on cars in most American cities \cite{b1, b2}. Despite significant efforts to improve public transportation over the past two decades, ridership levels in the United States have not increased substantially and have fallen short of projections. Factors contributing to this trend include urban sprawl, suburbanization, private car ownership, low gas prices, reductions in transit system services, and the rise of ride-hailing services such as Uber and Lyft \cite{b3, b4}. 

%\hl{need to add citations below to support the argument}
According to the American Public Transportation Association (APTA), the COVID-19 pandemic has also significantly reduced transit usage, with a decrease of over 50 percent between 2019 and 2020 (Figure \ref{fig1}). Post-COVID, the ridership of public transportation in the U.S. has increased from 19\% in April 2020 to 72\% in September 2022, the highest level since 2019 \cite{b5, b6}.

\begin{figure*}[ht]
    \centering
    \includegraphics[width=\textwidth]{APTA_Ridership_Revised.pdf}
    \caption{Quarterly Public Transportation Ridership by Mode. In 2020 and 2021, public transportation ridership was less than half its pre-pandemic level. While bus ridership has recovered somewhat, it was much lower in the second quarter of 2022 than in the final pre-pandemic quarter. Bus ridership for commuters grew by 66\% in the second quarter of 2022 \cite{b5}.}
    \label{fig1}
\end{figure*}

%\hl{You need to add more citations to support your arguments in the below paragraph}
Recently smart cities have been an emerging topic, drawing the attention of scholars in various fields \cite{b7,b8}. Among the essentials of a ”Smart City” is transit reliability, which is imperative for commuters seeking to eliminate lengthy commutes and public transit wait times. To better serve people whose daily lives are heavily dependent on public transportation, developed cities worldwide improve their transit scheduling systems. As noted earlier, various factors can impact transit ridership. An essential aspect of public bus transit is service predictability to reduce unnecessary wait time and improve the reliability of planning a bus trip ahead of time \cite{b2}. Unreliable transit services can disrupt a commuter's travel plans and may cause them to switch to alternative modes of transportation, such as driving a car. Operational uncertainty and delays can reduce the confidence of transit users and ultimately lead to a decrease in ridership and greater reliance on alternative modes of transportation. Unreliable services force transit users to spend more time waiting for transit, resulting in longer wait times at transit stops \cite{b9, b10, b11}.

%\hl{Add citations for the below paragraph}
Many cities offer dedicated mobile applications for their bus transit, offering bus schedules and providing passengers and commuters a way to plan their travel ahead of time \cite{b12}. These apps often lack real-time information, frustrating passengers trying to plan. Many commuters rely on other applications (e.g., Google Maps or Wayze \cite{b13}) to plan their commute with the public bus. However, these apps rely on crowd-sourced information from other users and often cannot provide sufficient accuracy. Furthermore, they do not share their internal data with local bust transits to enhance scheduling, and operational efficiency \cite{b13}. 

Public transit has the potential to offer real-time estimated arrival times to passengers, similar to private sectors ride-sharing services such as Uber or Lyft \cite{b14, b15, b16}. Smart data-driven approaches can potentially provide higher predictability and efficiency to the entire public bus transit system at scale to create Uber/Lyft-level reliability and predictability in public bus transit. What is needed is regional data collection, analysis, pattern detection, and prediction to improve the accuracy of arrival time and bus trip planning across the entire bus network. By doing so, transit agencies could potentially increase ridership and promote a more sustainable and affordable alternative to driving. This, in turn, could help create a more environmentally friendly and equitable society \cite{b17, b18}. 

\begin{figure*}[htp]
    \centering
    \includegraphics[width=17.5cm]{Application_Technology_New_Version2.pdf}
    \caption{Integration of the AI-based Prediction Model to a current Mobile Application for Bus System. By integrating our model into a mobile application that displays commuter transit demands and bus location information in real-time, commuters can be informed of both the scheduled arrival time of the bus and the expected arrival time based on our model predictions for bus arrival times.}
    \label{fig2}
\end{figure*}



This paper presents a deep-learning-based prediction model for predicting station bus arrival times. Our model provides a unified, Fully Connected Neural Networks (FCNNs) formulation for predicting the arrival time of the bus based on historical and environmental information. Our model can generalize across many bus lines within the same bus transit networks with much higher scalability and generalization power than classical machine learning approaches. We leverage New York City Bus System Dataset with more than 200 bus lines and 2 million data points for the analysis. 

Our results show that using our AI-based prediction model based on fully connected neural networks (FCNNs), we can achieve an average estimated error of 40 seconds in bus arrival times, which is a significant result compared with the average delay times of the bus arrival times in the dataset. According to our study, FCNNs are more effective and scalable than traditional machine learning approaches, such as SVR, for analyzing transportation problems with large input features. 

The ultimate goal of this work is to properly integrate the developed model into existing public bus transit mobile applications for public bus transit systems. One example is shown in Figure \ref{fig2}, which can be integrated into the current mobile applications for bus transit in the cities. We aim to improve the overall bus transit experience by utilizing this approach and remove significant waiting times for passengers. 

In summary, the contributions of this paper are:

\begin{itemize}
  \item Proposing a unified deep-learning-based Fully Connected Neural Network (FCNN) formulation for predicting the bus arrival times across many bus lines within the same bus transit network.  
  \item Evaluating the accuracy of the proposed model in the New York City Bus System Dataset with more than 200 bus lines and 2 million data points.   
  \item Demonstrating our approach's scalability and generalization benefits compared to classical machine learning models such as Support Vector Regression (SVR).
\end{itemize}


The rest of this paper is structured as follows. The subsequent section is devoted to bus arrival time prediction literature review. The third section contains a detailed description of the dataset used in this study and some exploratory data analysis. Our methodology is illustrated in section four. Finally, our model is validated, and we present our conclusions in section six.


\section {Related Works}
Our objective in this section is to assess the efficiency of examples similar to our research demonstrating how data-driven approaches can be used for bus transit systems, arrival time prediction, and scheduling optimization. Public transportation is a crucial component of a connected and smart community. Therefore, citizens demand real-time information regarding transportation assets' arrival and departure. In many cities worldwide, intelligent transportation systems with demand-responsive services are being implemented to bridge the gap between public transportation and private cars. In some early research, data analytics has been used to optimize public bus schedules and minimize passenger wait times.

Different technologies can be utilized which could generate real-time data for bus arrival time prediction. Among them, Global Positioning Systems (GPS), Automatic Passengers Counter Systems (APCS), and Crowdsensing solutions in which users cooperate with the system through a mobile application are the most popular ones \cite{b19, b20}. 

The problem of bus arrival time prediction problem was studied by considering different models and various essential factors. In a study by N. Gaikwad and S. Varma \cite{b19} the crucial features for bus arrival time prediction and standard evaluation metrics were presented. The main factors affecting bus arrival time are the source, destination, bus location coordinates, traffic density, bus stop in the way followed by several intersections, stop-to-stop distance, workday, and so on. 

In another study by Rafidah Md.\cite{b11}, bus arrival time has been predicted by using the Support Vector Regression (SVR) model. Petaling Jaya City Bus data was used in this study, including a sequence of bus stations, bus station names, the coordinate of the bus stations, timestamps, and the distance covered from the previous station to the next station. They also implemented their prediction model with and without weather data and showed that adding weather parameters for their data set shows a negligible difference in their prediction error.

Also, a study by F. Sun, Y. Pan, J. White, and A. Dubey \cite{b22} introduced a public transportation decision support system for short-term and long-term prediction of arrival bus times. This study used the real-world historical data of two Nashville bus system routes. The approach of this research combined the clustering analysis and Kalman filters with a shared route segment model to produce more accurate arrival time predictions and, based on their results, compared to the basic arrival time prediction model that Nashville MTA was using, their system reduced arrival time prediction errors by 25\% on average when predicting the arrival delay an hour ahead and 47\% when predicting within a 15-minute future time window \cite{b22}.

S. Basak, F. Sun, S. Sengupta, and A. Dubey have conducted a similar study \cite{b23}, using unsupervised clustering mechanisms to optimize transit on-time performance. As a local case study, they analyzed the monthly and seasonal delays of the Nashville metro region and clustered months with similar patterns. In this paper, they presented a stochastic optimization toolchain along with sensitivity analyses for choosing the optimal hyperparameters, and they solved the optimization problem by using a single-objective optimization task as well as a greedy algorithm, a genetic algorithm (GA), and a particle swarm optimization (PSO) algorithm \cite{b23}.

According to the newest research in \cite{b24}, dynamic data-driven application systems (DDDAS) that use real-time sensors and a data-driven decision support system can provide online model learning and multi-time-scale analytics to enhance the intelligence of the system. As part of their study, the authors analyzed an online bus arrival prediction system in Nashville using historical and real-time streaming data, which can be packaged as modular, distributed, and resilient micro-services. The long-term delay analysis service excludes noise from outliers in historical data to identify delay patterns associated with different hours, days, and seasons for specific time points and route segments. City planners can use the feedback data generated by these analytics services to improve bus schedules and increase rider satisfaction \cite{b24}. 

In addition, another study by S. Nannapaneni and A. Dubey \cite{b25} researched rerouting a single bus to better serve spatially and temporally changing travel demands. The aim was to propose a flexible framework for public transit rerouting. The study was demonstrated on Route 7 of the Nashville Metropolitan Transit Authority (MTA). The authors identified several flex stops with high travel demand using clustering since people living far from bus routes tend to choose alternate transit modes, leading to increased traffic congestion. They categorized the bus stops along the static routes into critical and non-critical stops and added slack time to account for travel delays during the existing static scheduling process. As a result, flexible routes resulted in less additional travel time than available slack time. The effectiveness of rerouting was analyzed using the percentage increase in travel demand \cite{b25}.\newline

\section{Dataset} 
New York City bus data (June 2017) is used in this study. 232 bus lines between 01 June and 30 June 2017 were inspected to collect this data, and these records were captured in 10-minute increments from 4468 buses.

\subsection{Dataset Description}\label{AA}
 The dataset is a critical component of every AI-based system. The dataset used for this study was selected due to its rich properties. More than 6 million data generated in a month are included in this dataset. Not only does this data set have a vast number of records, but also it consists of the most relevant parameters to the problem of arrival time prediction. 
 Each record contains the information in the format of 17 fields including Vehiclelocation.Longitude, VehicleLocation.Latitude, DestinationLong, DestinationLat, OriginLong, OriginLat, RecordedAtTime, ArrivalTime, ScheduledArrivalTime, DistanceFromStop, OriginName, DestinationName, PublishedLineName, NextStopPointName, ArrivalProximityText, VehicleRef and DirectionRef,\newline
 
The first 6 fields are the current bus location, destination, and origin coordinates. Other field descriptions are as below: 
\begin{itemize}
    \item RecordedAtTime is the checkpoint time in which the current location of the bus is recorded and used as the bus observation time in this study.
    \item ArrivalTime is the time when the bus arrives at the next stop.
    \item ScheduledArrivalTime is from the published bus timetable, indicating the scheduled time for the bus to arrive at the next stop. 
    \item DistanceFromStop is the distance of the bus from the next stop at the observation time. 
    \item Origin and destination are defined by OriginName and DestinationName.
    \item PublishedLineName represents in which line bus operates. 
    \item NextStopPointName is the name of the next bus stop. 
    \item ArrivalProximityText shows the current status of the bus in terms of a text, including at stop, approaching, and how many miles the bus is away.
    \item VehicleRef is the reference number for every bus whose location is being tracked. 
    \item DirectionRef field indicates inbound or outbound bus direction.
\end{itemize}


\newcolumntype{p}[1]{>{\centering\arraybackslash}m{#1}}
\begin{table}[htbp]
\caption{Features in New York dataset}
\begin{center} 
\begin{tabular}{|p{4cm}|}
\hline
\textbf{Dataset Features}\\
\hline
VehicleLocation.Longitude\\
\hline
VehicleLocation.Latitude\\
\hline
DestinationLong\\
\hline
DestinationLat\\
\hline
OriginLong\\
\hline
OriginLat\\
\hline
RecordedAtTim\\
\hline
ArrivalTime\\
\hline
ScheduledArrivalTime\\
\hline
DistanceFromStop\\
\hline
OriginName \\
\hline
DestinationName\\
\hline
NextStopPointName\\
\hline
PublishedLineName \\
\hline
ArrivalProximityText \\
\hline
VehicleRef \\
\hline
DirectionRef\\
\hline
\end{tabular}
\label{tab_NN}
\end{center}
\end{table}

\subsection{Cleaning the data and Preprocessing}\label{AA}
 Data is first cleaned and preprocessed to get meaningful concepts from this dataset. Then, the most related features are created, which will explain in the next section.\newline
 While around 6 million data instances are available in this dataset, they can not be considered logical observations. Since the goal is to predict the arrival time of the bus to the next station, we are only interested in the data points in which the bus is moving between bus stations. So, data is first filtered based on the "ArrivalProximityText" field, and data samples with at-stop value dropped from the data points, where actual arrival time almost equals the bus observation time. As well as this, data points have been filtered based on the value of the "DirectionRef" field, and data with outbound directions are selected. By doing the previous steps, around 2 million records were available to work with.

\subsection{Analysis of the Data}\label{AA}
Different indicators can measure the quality of service in public transit infrastructures. On-time performance at stops is an essential factor. The unpredictable nature of delay has been selected as the top reason people avoid bus transit systems in many cities \cite{b23}.

\begin{figure}[htp]
    \centering
    \includegraphics[width=\linewidth]{f15.pdf}
    \caption{Average delay among all bus lines. Initial analysis of records in the New York dataset shows that the average delay across all bus lines equals 491 seconds.  }
    \label{de_2}
\end{figure}


So, data is first analyzed regarding mismatching between the scheduled arrival time and the actual arrival time. Mismatch time refers to any difference between the bus's scheduled time and arrival time. When the bus arrives at the bus stop earlier, passengers might miss the bus, and also, for late buses, public transportation infrastructure suffers from the delay. Any of these two arrival time variations impact commuters’ satisfaction significantly \cite{b23}. Our study found that the average delay and mismatch time across all lines of this dataset is around 8 minutes (491 seconds) and 6 minutes, respectively. The average delay for these 232 bus lines has been illustrated in Figure \ref{de_2}.

\section{Methodology}
\subsection{Extracting Features}\label{AA}
The New York data set has 232 lines, and each line has been segmented into the number of bus stops. Assigning each line an integer value would not be a practical approach since an ordered relationship exists between integer values and may lead to poor performance of the model. One-hot encoding applies to categorical variables like bus lines without an ordinal relationship. This encoding helps the bus lines be injected into the model in terms of binary variables. Applying one-hot encoding on bus lines expands the input features and adds 232 more inputs. On the other hand, bus stops have some order, and they are fed into the model through integer encoding. The bus stop input variable can help the model track the traffic conditions and passenger flow varying from one bus stop to another.

As mentioned before, the bus records in this dataset were collected for a month. Because of the wide time variation, time is injected into the model in terms of two categories rather than feeding it directly into the model. This approach avoids injecting a lot of noise into the model. 

First, based on the day of the bus operation, the variable "day type" is added to the input features, which can get two values, "weekend" and "workday". The other time-related variable is the rush hour status. According to the operation time of the bus, this feature assigns to each record of the dataset, determining whether the bus operates during rush hour or not. Rush hour in New York spans from morning 6 AM to 10 AM and 3 PM to 7 PM \cite{b26}.  

In addition to the features that were previously mentioned, there are two distance-related features in the model. The distance input feature, the most important feature among other features, indicates how many meters the bus is far from the next stop. Far status is another distance-related feature added to input features according to the distance value. This input feature is a binary feature and changes depending on whether it is less than a distance of 750 meters or not. This threshold value was obtained through trial and error. 

Trip time ($Tr$) is the output feature which is the time period needed for a bus to reach the next bus station from its current location. It is calculated in seconds by subtracting the bus observation time ($T_{ob}$)  from the actual arrival time ($T_{ar}$). In real problems, the arrival time of the bus can be calculated by knowing the trip time.  

	\begin{equation} \label{eqn}
	Tr = {T_{ar}} - {T_{ob}}
	\end{equation}
Table~\ref{tab_NN_3} summarizes the input and output features that were produced during the feature extraction step.
\newcolumntype{p}[1]{>{\centering\arraybackslash}m{#1}}


\begin{table}[]
\begin{center}
\caption{Input and Output Features }
\begin{tabular}{|l|c|}

\hline
Input Features & \begin{tabular}[c]{@{}c@{}}One-hot Encoded Bus Lines \\ Distance \\ Day Type \\ Rush Hour Status \\ Bus Stops \\ Far Status\end{tabular} \\ \hline
Output Feature & Trip Time                                                                                                                                \\ \hline
\end{tabular}
\label{tab_NN_3}
\end{center}
\end{table}



\subsection{Feature Scaling}\label{AA}
Due to the different range of input features, data needs to be scaled prior to being injected into the model. Some input features, like rush hour status, are in the binary form and represented by 1 and 0, while others like distance, can be hundreds of meters. Without feature scaling, the model can be affected by the different range of features, assigning higher weights to the features with large scale. So, Min-Max scaling is used to transform the value of all input features to the range of 0 and 1.  
\subsection{Train and Validation Sets}\label{AA}
The dataset is divided into a train and validation set. 80\% of the dataset has been categorized as a training set for the training of our model, while 20\% of the dataset has been used as a validation set. The total number of instances is 2.13 million. 1.7 million is used for training our model, while the rest is utilized for validation. It is also worth mentioning that the average of training data samples for each line is 7327. 

\subsection{Model}\label{AA}
Artificial Neural Networks (ANNs) are very common in forecasting bus trip time. Previous studies showed that ANN had the ability to predict nonlinear relationships in complex problems. \cite{b27}. 

In this study, we make use of FCNNs to predict the bus trip time. As discussed in the previous section, due to the large number of bus lines, FCNNs can handle high-dimensional feature spaces by using hidden layers and non-linear activation functions.\hfill\\
To determine the most well-trained architecture for our model, different numbers of hidden layers and neurons in each layer were tested and the best model with more prediction ability has been chosen. 

Our model has seven layers, including an input layer, 5 hidden layers, and an output layer. As it can be seen in Figure \ref{NN Model-2}, there are 237 input features feeding into the model, including 232 input features generated from one-hot encoding on bus lines, distance, day type, rush hour status, bus stops, and far status. In the first hidden layer, the model learns more complex representations of input features by increasing neurons to 320. The number of neurons decreases step by step in the next hidden layers, from 200 in the second hidden layer to 5 neurons in the fifth hidden layer. The output layer consists of one neuron, predicting the bus trip time based on input data transformed from previously hidden layers.

Another important hyperparameter in FCNNs is the choice of activation function that plays a key role by introducing non-linearity to the model. Rectified Linear Unit (ReLU) function is used as an activation function in our model. Since the number of hidden layers in our model is 5, ReLU is a better choice than Sigmoid and Hyperbolic Tangent (Tanh), helping to mitigate the vanishing gradient problem.

\newcolumntype{p}[1]{>{\centering\arraybackslash}m{#1}}
\begin{table}[htp]
\caption{Structure of Fully Connected Neural Network for all Bus Lines }
\begin{center} 
\begin{tabular}{|p{5cm}|p{1.5cm}|}
\hline
\multicolumn{2}{|c|}{\textbf{Parameters}} \\
\hline
Number of inputs & 237\\
\hline
Number of hidden layers & 5\\
\hline
Activation function & ReLU\\
\hline
Number of neurons in the first layer & 320\\
\hline
Number of neurons in the second layer & 200\\
\hline
Number of neurons in the third layer & 100\\
\hline
Number of neurons in the fourth layer & 40\\
\hline
Number of neurons in the fifth layer & 5\\
\hline
Number of outputs & 1\\
\hline
\end{tabular}
\label{tab_NN_1}
\end{center}
\end{table}

\begin{figure}[htp]
    \centering
    \includegraphics[width=9cm]{NN_Model_Revised.pdf}
    \caption{Structure of our model based on the Fully Connected Neural Network. One-hot encoding applies to bus lines and extends it to 232 features. These converted features with other 5 features, including distance, day type, rush hour status, bus stops, and far status feed to the Fully Connected Neural Network. The proposed model consists of 5 hidden layers and ReLU as an activation function. The number of neurons in each hidden layer can also be seen in the figure. H1-320N indicates that the first hidden layer consists of 320 neurons. 
   }
    \label{NN Model-2}
\end{figure}


\section{accuracy Analysis and Model Comparison}
 \subsection{Performance Measurements}\label{AA}
The performance evaluation of arrival time predicted by the model can be done using different measures, including Mean Absolute Percentage Error (MAPE), Mean Square Error (MSE), and Root Mean Square Error (RMSE). \newline
In this work, the accuracy of the model is evaluated via RMSE, showing the difference between the predicted trip time and the actual trip time in seconds. Since RMSE has the same unit as the prediction value, it can provide a better basis for comparison. RMSE can be represented as the following equation where $t_{act}$ is the actual bus trip time and $t_{pred}$ stands for the predicted bus trip time based on the proposed model, and n is the sample size for prediction. Lower RMSE represents better performance in prediction. 

\begin{equation} \label{eqn}
	RMSE=\sqrt \frac {\Sigma^{n}_{i=1}  (t_{act} - t_{pred})^2} {n}
	\end{equation}

 \subsection{Results and Model Performance Discussion}\label{AA} 

 
 \subsubsection{Results for Fully connected NN on all bus lines}\label{AA} \hfill\\
 The training process was implemented on a system with an Intel i7-1185G7 processor with 4 cores and a speed of 3.00 GHz.
 Table~\ref{tab_t} shows the results of applying our model on 232 bus lines with the learning rate of $1e-2$. 

\newcolumntype{p}[1]{>{\centering\arraybackslash}m{#1}}
\begin{table}[htp]
\caption{Results of our model on All Bus Lines}
\begin{center} 
\begin{tabular}{|p{2.5cm}|p{1.5cm}|}
\hline
\multicolumn{2}{|c|}{\textbf{Results}} \\
\hline
Training RMSE & 35.69 s\\
\hline
Validation RMSE & 35.74 s\\
\hline
\end{tabular}
\label{tab_t}
\end{center}
\end{table}

 It can be observed that the average RMSE for all bus lines is 35.74 seconds. In other words, the predicted arrival time of the bus to the next station has an error lower than 36 seconds. \newline
Figure \ref{fig:RMSE_validation} shows the RMSE over each bus line. While the highest prediction error is 119.99 seconds in line number 160, the lowest RMSE belongs to line number 76, with the RMSE equal to 12.42 seconds. Figure \ref {fig:RMSE_validation1} illustrates the delay and RMSE comparison among all bus lines.

\begin{figure}[h]
    \centering
    \includegraphics[width=\linewidth]{f17.pdf}
    \caption{Performance of the model for all bus lines. The average RMSE across all bus lines in the validation set is 35.74 seconds. Among the prediction error values, bus line 160 has the greatest RMSE, with a value of 119.99 seconds. In contrast, the lowest error belongs to bus line 76 with an RMSE equal to 12.42 seconds.}
    \label{fig:RMSE_validation}
\end{figure}


\begin{figure}[b]
    \centering
    \includegraphics[width=7.3cm]{f33_Revised.pdf}
    \caption{Comparison between actual delay and RMSE across validation set samples, containing 426,323 samples. The average delay for all bus lines is 438 seconds. The error of the prediction over samples in the validation set is less than 36 seconds.}
    \label{fig:RMSE_validation1}
\end{figure}
% \includegraphics[width=1\linewidth, trim={20px 200px 20px 200px}, clip]{f1.pdf}
The large RMSE values in comparison to the other ones could be due to the absence of some relevant features in the bus trip time prediction problem. There is a wide range of other factors affecting the bus trip time but not available in this dataset. For instance, weather type can change bus trip time from one stop to another. Passenger demand is another feature that this dataset does not include. By equipping buses with passenger counting systems, passenger demand for each bus stop can be recorded as well. This parameter impacts the bus dwell time, referring to the time a bus spends at a stop without moving. In Figure \ref{fig:box}, we have also shown RMSE distribution. Training time and inference time per each validation set data point are also presented in Table~\ref{tab1}.

\begin{figure}[h]
    \centering
    \includegraphics[width=7.5cm]{final_box.pdf}
    \caption{RMSE distribution in the form of a boxplot. It displays the difference in RMSE values by showing the median, which is 35.74 seconds.}
    \label{fig:box}
\end{figure} 

\newcolumntype{p}[1]{>{\centering\arraybackslash}m{#1}}
\begin{table}[htbp]
\caption{Model Properties on All Bus Lines}
\begin{center} 
\begin{tabular}{|p{3.5cm}|p{2cm}|}
\hline
\multicolumn{2}{|c|}{\textbf{Model Properties}} \\
\hline
Total Training Time & 7171s\\
\hline
Total Inference Time & 2.42s\\
\hline
Inference Time per each Validation sample & 0.00578 ms\\
\hline
Number of Parameters &  164710 \\
\hline
Computational Complexity & 165380 Mac \\
\hline
\end{tabular}
\label{tab1}
\end{center}
\end{table}

The average inference time for each validation data point is 0.00578 ms. This implies if a passenger sends a request to the cloud to get the bus arrival time for their trip, it takes less than 0.006 ms to produce AI-based predictions. It should be noted this inference time indicates the required time only for one access. When thousands of passengers make requests to the cloud for bus arrival time, it will grow significantly. \newline
 
\subsubsection{Scalability Comparison between our model and SVR }\label{AA} \hfill\\
Since there are more than 200 lines in the dataset, a generalized model is needed to predict the arrival time with the lowest possible error for all bus lines.
This section illustrates the scalability comparison of our model and SVR for this dataset. The reason for making this comparison is that SVR is among the other machine learning approaches which are popular for bus arrival time prediction problems and a lot of researchers used SVR with the Radial Basis Function (RBF) kernel for bus arrival time prediction\cite{b21}.  

So, for a different number of bus lines, SVR with RBF kernel was used. The experimental results, as can be observed in Figure~\ref{SVR_NN_2}, indicate that in a small number of lines, SVR and our model prediction patterns are almost the same. RMSE for prediction on 10 lines using FCNN and SVR is 22.84 and 26.67 respectively. when the number of lines rises from 10 to 20, RMSE is 24.90 and 33.98, showing a notable increase in RMSE for the SVR model, and when the number of bus lines surpasses 30, SVR becomes untrainable on this dataset. 
Hence, in terms of scalability, our model has better prediction ability than SVR and this is the reason why FCNN was selected for the whole dataset. 
 
\begin{figure}[h]
    \centering
    \includegraphics[width=\linewidth]{f19.pdf}
    \caption{Scalability comparison of FCNN and SVR. These two models were evaluated for different numbers of bus lines. In the range of 1 to 20 bus lines, the accuracy of SVR prediction decreased significantly, while NN performance remains almost unchanged. When the number of bus lines exceeds 30, SVR can not be trained on this dataset.}
    \label{SVR_NN_2}
\end{figure}

\section{Conclusion}
In this research paper, we developed an AI-based prediction model for an intelligent bus transit system that passengers can access on demand. Our goal is to improve the overall bus transit experience by utilizing this approach and remove significant waiting times for passengers.

In conclusion, the input features in our prediction model included one-hot encoded bus lines, distance, day type, rush hour status, bus stops, and far status. The final results of our model which was based on FCNNs showed an average estimated error of less than 40 seconds on all bus lines in the dataset, which is a significant improvement compared to the average delay time in the dataset. Our AI-based model can be integrated into a smart mobile application, providing real-time information to commuters. 

This paper only focused on some features related to bus trip time due to the limitations of the used dataset. Further studies are needed to include other factors, such as passenger flow, bus dwell time, and weather conditions as well. Besides, future work can be focused on clustering bus lines to the groups having similar delay patterns that can improve the model's prediction ability.

\begin{thebibliography}{00}

\bibitem{b1} “US Public Transit Has Struggled to Retain Riders over the Past Half Century. Reversing This Trend Could Advance Equity and Sustainability. | Urban Institute.” https://www.urban.org/urban-wire/us-public-transit-has-struggled-retain-riders-over-past-half-century-reversing-trend-could-advance-equity-and-sustainability (accessed Feb. 25, 2023). 

\bibitem{b2} S. S. Pulugurtha, R. Mishra, and S. L. Jayanthi, “Does Transit Service Reliability Influence Ridership?”, doi: 10.31979/mti.2022.2118. 

\bibitem{b3} G. D. Erhardt, J. M. Hoque, V. Goyal, S. Berrebi, C. Brakewood, and K. E. Watkins, “Why has public transit ridership declined in the United States?,” Transp Res Part A Policy Pract, vol. 161, pp. 68–87, Jul. 2022, doi: 10.1016/J.TRA.2022.04.006. 

\bibitem{b4} “(PDF) Understanding the Recent Transit Ridership Decline in Major US Cities: Service Cuts or Emerging Modes?”, https://www.researchgate.net/publication/330599129 Understanding the Recent Transit Ridership Decline in Major US Cities Service Cuts or Emerging Modes (accessed Mar. 01, 2023).

\bibitem{b5} N. York and S. Francisco, “Public Transportation Ridership: Implications of Recent Trends for Federal Policy”, Accessed: Feb. 25, 2023. [Online]. Available: https://www.cbo.gov/system/files/2022-03/57636-Transportation.pdf. 

\bibitem{b6} K. Takeaways, “S E P T E M B E R 2 0 2 2 APTA Public Transportation Ridership Update,” 2020. 

\bibitem{b7} Pazho, A.D., Neff, C., Noghre, G.A., Ardabili, B.R., Yao, S., Baharani, M. and Tabkhi, H., 2023. Ancilia: Scalable Intelligent Video Surveillance for the Artificial Intelligence of Things. arXiv preprint arXiv:2301.03561.

\bibitem{b8} Noghre, G.A., Katariya, V., Pazho, A.D., Neff, C. and Tabkhi, H., 2022. Pishgu: Universal Path Prediction Architecture through Graph Isomorphism and Attentive Convolution. arXiv preprint arXiv:2210.08057.

\bibitem{b9} H. Xu and J. Ying, “Bus arrival time prediction with real-time and historic data,” Cluster Comput, vol. 20, no. 4, pp. 3099–3106, Dec. 2017, doi: 10.1007/S10586-017-1006-1/FIGURES/6.

\bibitem{b10} H. Huang, L. Huang, R. Song, F. Jiao, and T. Ai, “Bus Single-Trip Time Prediction Based on Ensemble Learning,” Comput Intell Neurosci, vol. 2022, 2022, doi: 10.1155/2022/6831167.

\bibitem{b11} G. Zhong, T. Yin, L. Li, J. Zhang, H. Zhang, and B. Ran, “Bus Travel Time Prediction Based on Ensemble Learning Methods,” IEEE Intelligent Transportation Systems Magazine, vol. 14, no. 2, pp. 174–189, 2022, doi: 10.1109/MITS.2020.2990175.

\bibitem{b12} J. Fu, L. Wang, M. Pan, Z. Zuo, and Q. Yang, “Bus arrival time prediction and release: System, database and android application design,” Lecture Notes in Computer Science (including subseries Lecture Notes in Artificial Intelligence and Lecture Notes in Bioinformatics), vol. 8631 LNCS, no. PART 2, pp. 404–416, 2014, doi: 10.1007/978-3-319-11194-0-33/COVER.

\bibitem{b13} “Moovit Faces Challenges to Become ‘Waze for Public Transit.’” https://thenextweb.com/news/moovit-waze-public-transit-challange (accessed Mar. 15, 2023).

\bibitem{b14} “NextBus TM Real-time passenger information.”
\bibitem{b15} “A Multi-Agent Reinforcement Learning approach for bus holding control strategies.” http://www.atsinternationaljournal.com/index.php/2015-issues/special-issue-2015-vol2/792-a-multi-agent-reinforcement-learning-approach-for-bus-holding-control-strategies (accessed Jan. 06, 2023).

\bibitem{b16} “Transport, data analytics and AI: Why TfL’s latest initiative is good news - Technology Services Group.” https://www.tsg.com/insights/transport-data-analytics-and-ai-why-tfls-latest-initiative-is-good-news/ (accessed Mar. 12, 2023).

\bibitem{b17} E. I. Diab, M. G. Badami, and A. M. El-Geneidy, “Bus Transit Service Reliability and Improvement Strategies: Integrating the Perspectives of Passengers and Transit Agencies in North America,” http://dx.doi.org/10.1080/01441647.2015.1005034, vol. 35, no. 3, pp. 292–328, May 2015, doi: 10.1080/01441647.2015.1005034.

\bibitem{b18} “Smart public transport – key to solving the urban challenge - Telia Company.” https://www.teliacompany.com/en/news/news-articles/2017/report-smart-public-transportation/ (accessed Mar. 12, 2023).

\bibitem{b19} N. Gaikwad and S. Varma, “Performance Analysis of Bus Arrival Time Prediction Using Machine Learning Based Ensemble Technique,” SSRN Electronic Journal, Mar. 2019, doi: 10.2139/SSRN.3358828. 

\bibitem{b20} T. Yin, G. Zhong, J. Zhang, S. He, and B. Ran, “A prediction model of bus arrival time at stops with multi-routes,” Transportation Research Procedia, vol. 25, pp. 4623–4636, Jan. 2017, doi: 10.1016/J.TRPRO.2017.05.381. 

\bibitem{b21} R. Md Noor et al., “PREDICT ARRIVAL TIME BY USING MACHINE LEARNING ALGORITHM TO PROMOTE UTILIZATION OF URBAN SMART BUS,” 2020, doi: 10.20944/PREPRINTS202002.0197.V1. 

\bibitem{b22} F. Sun, Y. Pan, J. White, and A. Dubey, “Real-Time and Predictive Analytics for Smart Public Transportation Decision Support System,” 2016 IEEE International Conference on Smart Computing, SMARTCOMP 2016, Jun. 2016, doi: 10.1109/SMARTCOMP.2016.7501714. 

\bibitem{b23} S. Basak, F. Sun, S. Sengupta, and A. Dubey, “Data-driven optimization of public transit schedule,” Lecture Notes in Computer Science (including subseries Lecture Notes in Artificial Intelligence and Lecture Notes in Bioinformatics), vol. 11932 LNCS, pp. 265–284, 2019.

\bibitem{b24} F. Sun, A. Dubey, J. White, and A. Gokhale, “Transit-hub: a smart public transportation decision support system with multi-timescale analytical services,” Cluster Comput, vol. 22, no. 1, pp. 2239–2254, Jan. 2019, doi: 10.1007/S10586-018-1708-Z/FIGURES/11. 

\bibitem{b25} S. Nannapaneni and A. Dubey, “Towards Demand-Oriented Flexible Rerouting of Public Transit Under Uncertainty,” 2019, doi: 10.1145/3313237.3313302. 

\bibitem{b26} "NEW YORK CITY TRANSIT KEY PERFORMANCE METRICS", available: https://new.mta.info/document/100341, (accessed Mar. 15, 2023).

\bibitem{b27} C. Bai, Z. R. Peng, Q. C. Lu, and J. Sun, “Dynamic bus travel time prediction models on road with multiple bus routes,” Comput Intell Neurosci, vol. 2015, 2015, doi: 10.1155/2015/432389. 


\end{thebibliography}

\end{document}




