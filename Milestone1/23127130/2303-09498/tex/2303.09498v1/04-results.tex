\section{Results}

We now present quantitative results that show how biasing explanations that accompany recommendations directly affects users' preferences, as expressed by their item selections. 

\subsection{Participants}

A total of 129 participants participated in our study who are paid contractors, and received a standard contracted wage (complying with living wage laws in their country of employment).  All of them are US-based and native English speakers.  In terms of gender, they are 62.7\% women, 35.7\% men, 1.6\% prefer not to say.  Their distribution by age: 5.6\% 18--24, 29.4\% 25--34, 32.5\% 35--44, 19.8\% 45--54, 9.5\% 55--64, 3.2\% over 65.  Their self-reported amount of average time spent per week watching movies: 5.7\% $<$2 hours, 37.9\% 2--5 hours, 29\% 6--10 hours, 16.9\% 11-16 hours, 10.5\% $>$ 16 hours. 
Participants on average took 41 seconds to rate a batch of five movies in Stage 1 (seen/liked), and 43 seconds to choose the movie they would want to watch from the three recommendations in Stage 2.

\subsection{Effect of the Presence of Explanations}
\label{sec:results:effect}

\begin{figure}[t]
    \centering
    \includegraphics[width=0.45\textwidth]{figures/baseline_no_expl.pdf}
    \vspace*{-0.5\baselineskip}
    \caption{Baseline condition, no explanations.}
    \label{fig:baseline_no_expl}
    \Description{Bar plots showing selection frequencies for (A) highest-scoring, (B) mid-scoring, and (C) lowest-scoring buckets.}
\end{figure}

Recall that in each instance, participants are provided with three recommendations where we know which the raters should prefer the most. Our first test validates that the preferences obey the expected order, and whether the presence of neutral explanations affects this order.
%
Specifically, let the three recommendations be $x_a \in A$, $x_b \in B$ and $x_c \in C$, where $A$, $B$, and $C$ are buckets with the highest-scoring, mid-scoring, and lowest-scoring items, respectively (cf. Fig.~\ref{fig:item_sampling}).
Let the relative frequency with which each bucket is chosen (corresponding to users' item selections), referred to as their \emph{selection frequency}, be denoted as $p_A$, $p_B$, and $p_C$ respectively.  We then have an expected ordering of $p_A \succ p_B \succ p_C$. 

When explanations are not shown, we find that across all participants, $p_A = 0.43 \pm 0.07$, $p_B = 0.31 \pm 0.07$ and $p_C = 0.26 \pm 0.06$ with a 95\% confidence interval, computed using the Goodman method~\citep{Goodman:1965:Technometrics}; see Fig.~\ref{fig:baseline_no_expl}. Thus, the expected ordering holds, and the difference between the highest- and lowest-scoring buckets is statistically significant.

When neutral explanations are included, we observe $p^n_A = 0.4 \pm 0.07$, $p^n_B = 0.28 \pm 0.07$, and $p^n_C = 0.32 \pm 0.07$, as shown in Fig.~\ref{fig:baseline_expl}.
While the highest-scoring bucket continues to receive the most selections, the relative ordering between buckets B and C, surprisingly, is now swapped.
Also, the selection frequencies for all three buckets come closer together, resulting in overlapping confidence intervals.
%
This means that despite the careful experiment design, explanations seem to have some uncontrolled effects.  It could be, for example, that mid- and low-scoring items are not that well distinguished by the recommender system and it is a random effect due to noise in the data.
However, the ``well-behaving'' baseline setting without explanations and the relatively large sample size (n=258) suggest otherwise.  It could also be that neutral explanations that highlight both positive and negative aspects invite more ``risky'' selections by users, giving lower-ranked suggestion a try.
In the remainder of our analysis, we take the neutral explanations setting (Fig.~\ref{fig:baseline_expl}) as our baseline.  However, the relative ordering of buckets B and C warrants further investigation.

\begin{figure}[t]
    \centering
    \includegraphics[width=0.45\textwidth]{figures/baseline_expl.pdf}
    \vspace*{-0.5\baselineskip}
    \caption{Baseline condition, neutral explanations.}
    \label{fig:baseline_expl}
    \Description{Bar plots showing selection frequencies for (A) highest-scoring, (B) mid-scoring, and (C) lowest-scoring buckets.}
\end{figure}


\begin{table*}[t]
    \caption{Selection frequency of different buckets (rows) depending on the presence and direction of bias in explanations (columns).
    Grey cell background indicates when the bias happens in the same bucket as the selection.  Green/red arrows show the change in selection frequency with respect to the no bias setting.}
    \label{tab:effect_of_bias}
    \vspace*{-0.5\baselineskip}
    \centering
    \begin{tabular}{c||c||c|c||c|c||c|c}
        \hline
        \multirow{2}{*}{\textbf{Selection}} & \multirow{2}{*}{\textbf{No bias}} & \multicolumn{2}{c||}{\textbf{Bias in A}} & \multicolumn{2}{c||}{\textbf{Bias in B}} & \multicolumn{2}{c}{\textbf{Bias in C}} \\
         & & \textbf{+/++} & \textbf{-/-{-}} & \textbf{+/++} & \textbf{-/-{-}} & \textbf{+/++} & \textbf{-/-{-}} \\
        \hline
        A & 0.40\,$\pm$\,0.07 
            & \cellcolor{gray!25}0.49\,$\pm$\,0.09\greenup & \cellcolor{gray!25}0.37\,$\pm$\,0.09\reddown
            & 0.37\,$\pm$\,0.09\reddown & 0.41\,$\pm$\,0.09\greenup
            & 0.36\,$\pm$\,0.08\reddown & 0.50\,$\pm$\,0.09\greenup \\
        B & 0.28\,$\pm$\,0.07 
            & 0.25\,$\pm$\,0.08\reddown & 0.34\,$\pm$\,0.08\greenup 
            & \cellcolor{gray!25}0.42\,$\pm$\,0.09\greenup & \cellcolor{gray!25}0.30\,$\pm$\,0.08\greenup 
            & 0.27\,$\pm$\,0.08\reddown & 0.29\,$\pm$\,0.08\greenup \\ 
        C & 0.32\,$\pm$\,0.07 
            & 0.26\,$\pm$\,0.08\reddown & 0.29\,$\pm$\,0.08\reddown 
            & 0.21\,$\pm$\,0.08\reddown & 0.28\,$\pm$\,0.08\reddown 
            & \cellcolor{gray!25}0.38\,$\pm$\,0.09\greenup & \cellcolor{gray!25}0.21\,$\pm$\,0.07\reddown \\ 
        \hline
    \end{tabular}
    \vspace*{-0.5\baselineskip}
\end{table*}

\begin{table*}[t]
    \caption{Selection frequency of different buckets (rows) depending on the presence and direction of bias in explanations (columns), for itemized explanations (top block) vs. fluent-NL explanations (bottom block); cell annotations are the same as in Table~\ref{tab:effect_of_bias}.}
    \label{tab:effect_of_bias_expl_type}
    \vspace*{-0.5\baselineskip}
    \centering
    \begin{tabular}{c||c||c|c||c|c||c|c}
        \hline
        \multirow{2}{*}{\textbf{Selection}} & \multirow{2}{*}{\textbf{No bias}} & \multicolumn{2}{c||}{\textbf{Bias in A}} & \multicolumn{2}{c||}{\textbf{Bias in B}} & \multicolumn{2}{c}{\textbf{Bias in C}} \\
         & & \textbf{+/++} & \textbf{-/-{-}} & \textbf{+/++} & \textbf{-/-{-}} & \textbf{+/++} & \textbf{-/-{-}} \\
        \hline
        \multicolumn{8}{l}{\textbf{Itemized explanations}} \\
        \hline
        A & 0.40\,$\pm$\,0.10 
            & \cellcolor{gray!25}0.49\,$\pm$\,0.13\greenup & \cellcolor{gray!25}0.35\,$\pm$\,0.11\reddown
            & 0.39\,$\pm$\,0.12\reddown & 0.47\,$\pm$\,0.13\greenup
            & 0.34\,$\pm$\,0.12\reddown & 0.55\,$\pm$\,0.12\greenup \\
        B & 0.28\,$\pm$\,0.09 
            & 0.26\,$\pm$\,0.11\reddown & 0.34\,$\pm$\,0.11\greenup 
            & \cellcolor{gray!25}0.45\,$\pm$\,0.13\greenup & \cellcolor{gray!25}0.19\,$\pm$\,0.11\reddown 
            & 0.26\,$\pm$\,0.11\reddown & 0.27\,$\pm$\,0.11\reddown \\ 
        C & 0.32\,$\pm$\,0.10 
            & 0.25\,$\pm$\,0.11\reddown & 0.31\,$\pm$\,0.11\reddown 
            & 0.16\,$\pm$\,0.10\reddown & 0.33\,$\pm$\,0.12\greenup 
            & \cellcolor{gray!25}0.40\,$\pm$\,0.12\greenup & \cellcolor{gray!25}0.19\,$\pm$\,0.10\reddown \\ 
        \hline
        \multicolumn{8}{l}{\textbf{Fluent-NL explanations}} \\
        \hline
        A & 0.40\,$\pm$\,0.10 
            & \cellcolor{gray!25}0.49\,$\pm$\,0.12\greenup & \cellcolor{gray!25}0.40\,$\pm$\,0.13\phantom{\reddown}
            & 0.34\,$\pm$\,0.12\reddown & 0.36\,$\pm$\,0.12\reddown
            & 0.38\,$\pm$\,0.12\reddown & 0.45\,$\pm$\,0.12\greenup \\
        B & 0.28\,$\pm$\,0.09 
            & 0.24\,$\pm$\,0.10\reddown & 0.33\,$\pm$\,0.12\greenup 
            & \cellcolor{gray!25}0.39\,$\pm$\,0.13\greenup & \cellcolor{gray!25}0.40\,$\pm$\,0.12\greenup 
            & 0.27\,$\pm$\,0.11\reddown & 0.31\,$\pm$\,0.12\greenup \\ 
        C & 0.32\,$\pm$\,0.10 
            & 0.26\,$\pm$\,0.10\reddown & 0.27\,$\pm$\,0.12\reddown 
            & 0.27\,$\pm$\,0.12\reddown & 0.24\,$\pm$\,0.11\reddown 
            & \cellcolor{gray!25}0.35\,$\pm$\,0.12\greenup & \cellcolor{gray!25}0.23\,$\pm$\,0.11\reddown \\ 
        \hline    
    \end{tabular}
\end{table*}

\subsection{Effect of Explanations Biased towards Positive or Negative}

By adjusting the number of positive and negative aspects, we can bias explanations in a positive or negative direction either weakly (e.g., three positive aspects and one negative aspect) or strongly (e.g., four negative aspects and no positives). 
Table~\ref{tab:effect_of_bias} shows the effect of bias on item selections (rows) depending on the position and direction of bias (columns).
For simplicity, we do not distinguish between the amount of bias (weakly or strongly positive/negative) nor the type of explanation (fluent-NL or itemized), but report on aggregated counts.

Bias in a given bucket has an effect on selections both in the same bucket (highlighted as grey in Table~\ref{tab:effect_of_bias}) and in other buckets.
For example, biasing positively items in C increases selections in C, but also decreases selections in A and~B.
Similarly, negatively biasing items in A drives selections down in A, while moving selections up in B.
This intuitively makes sense, but there are a few exceptions when this expected behavior cannot be observed, e.g., negative bias in A increases selections in B, but not in C.  
Despite these anomalies that remain to be investigated in the future, it is clear that biasing explanations has a large effect on the selections people make.  Two extremes are worth noting: (1) negatively biasing the most relevant recommendation reduces the selection of what is believed to be the best recommendation by 9\%, and (2) positively biasing the least relevant recommendation increases the selection of that item by twice as much, almost 19\%, compared to the no bias baseline.
Notice that the selection frequency of the least relevant suggestion with a positive bias (0.38\,$\pm$\,0.09) reaches that of the most relevant suggestion with a negative bias (0.37\,$\pm$\,0.09).



\subsection{Effect of Explanation Format: Fluent-NL vs. Itemized Explanations}

In our design, explanations were presented in two ways: as a list of attributes, and as fluent text that mentions the same attributes. 
The last question we ask is: To what extent do our findings depend on the particular explanation format?
Table~\ref{tab:effect_of_bias_expl_type} breaks down the previous results by explanation format. Our main findings are as follows.
First, there is no difference in results in the no bias setting.
Second, when explanations are biased, itemized explanations behave more ``as expected,'' i.e., when bias happens in the same bucket as the selection (grey cells), then positive bias always means an increase and negative bias always causes a drop in selection frequency.  This is not the case for fluent-NL explanations. 
Third, we observe that the differences between the positive and negative bias settings within a given bucket tend to be much larger in case of itemized explanations.  This intuitively makes sense, as the positive and negative aspects are made explicit with visual thumbs up/down icons, while fluent text can be more prone to hiding differences.
Despite these differences, results indicate a consistent pattern of change across the two explanation formats: the arrows indicating change point in the same direction in 14 out of the 18 cells.
