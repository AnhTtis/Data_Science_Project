\section{Appendix: Study Design}
\label{app:stage2}

We provide further details on the study design and data preparation.

\subsection{Aspect Extraction}
\label{app:stage2:aspects}

The data collected using crowdsourcing has been automatically cleaned and pre-filtered; this includes fixing capitalization, removing trailing whitespace, filtering too long aspects, and removing aspects that are substrings of other aspects.
However, the data needed further manual cleaning and filtering by the paper authors, as not all extracted aspects fit the template (i.e., human workers did not follow the instructions closely enough), aspects may be too harsh or offensive, or sound too personal to be used as explanations.  As part of the manual cleaning process, some aspects were slightly rewritten and near-duplicates were removed. Of the 9,635 movie-aspect pairs collected originally, 7,414 remained after automatic pre-filtering, and 5,948 after the end of the manual filtering.

It is worth emphasizing that the recommendations are personalized, while the explanations accompanying them are not, i.e., all participants receiving the same recommendation under the same experimental condition will see the same explanation for that item, to ensure that there is no uncontrolled bias.

\begin{table}[t]
    \caption{Sequences determining whether to list positive or negative aspects first in explanations.}
    \label{tab:stage2_sequences}
    \vspace*{-0.5\baselineskip}
    \begin{tabular}{|c|c|c|c|c|c|c|}
        \hline
        \multirow{2}{*}{\textbf{Sequence}} & \multicolumn{6}{c|}{\textbf{Condition}} \\
        \cline{2-7}
        & \textbf{\#3} & \textbf{\#4} & \textbf{\#5} & \textbf{\#6} & \textbf{\#7} & \textbf{\#8} \\
        \hline
        \hline
        1 & N & P & P & N & N & P \\
        \hline
        2 & P & N & N & P & P & N \\
        \hline
        3 & N & P & P & N & N & P \\
        \hline
        4 & P & N & N & P & P & N \\
        \hline
        5 & N & P & P & N & N & P \\
        \hline
        6 & P & N & N & P & P & N \\
        \hline
    \end{tabular}
\end{table}

\begin{figure*}[t]
    \centering
    \includegraphics[width=0.95\textwidth]{figures/stage2_ui_fixed.png}
    \vspace*{-0.5\baselineskip}
    \caption{User interface for item selection (Stage 2).}
    \label{fig:task2}
    \Description{Screenshot consisting of an instruction, three items in a table, and a submit button. The instruction reads "Indicate which movies you have watched and if you have liked them." The three items are shown below each other, as three rows. The headings of the table are: "Item," "Seen," "Rating." The item column shows the movie poster and title, seen has two radio buttons with "Yes" and "No" options, and Rating has three radio buttons with "Disliked," "Neutral," and "Liked" options.}
\end{figure*}

\subsection{Explanations}
\label{app:stage2:explanations}

When both positive and negative aspects are displayed (conditions \#3--\#8), we make those fully balanced for each participant as well as across all participants by cycling through the sequences shown in Table~\ref{tab:stage2_sequences}.
These sequences follow a Latin Square design where the binary value P/N is determined by the least significant bit.

\subsection{User Interfaces}
\label{app:ui}

Figures~\ref{fig:task1} and~\ref{fig:task2} show screenshots of the user interfaces used in Stages 1 and 2, respectively.


\begin{figure}[t]
    \centering
    \includegraphics[width=0.48\textwidth]{figures/stage1_ui.png}
    \vspace*{-0.5\baselineskip}
    \caption{User interface for item consumption and preference elicitation (Stage 1).}
    \label{fig:task1}
    \Description{Screenshot consisting of an instruction, three items in a table, and a submit button. The instruction reads "Select the movie that you would most likely watch among these." The three items are shown next to each other, as three columns. Each has a movie poster, a title with a radio button in front, a synopsis text, and a list of itemized explanations.}
\end{figure}
